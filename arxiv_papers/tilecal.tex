\newcommand{\AtlasCoordFootnote}{
ATLAS uses a right-handed coordinate system with its origin at the nominal interaction point (IP)
in the centre of the detector and the \(z\)-axis along the beam pipe.
The \(x\)-axis points from the IP to the centre of the LHC ring,
and the \(y\)-axis points upwards.
Cylindrical coordinates \((r,\phi)\) are used in the transverse plane, 
\(\phi\) being the azimuthal angle around the \(z\)-axis.
The pseudorapidity is defined in terms of the polar angle \(\theta\) as \(\eta = -\ln \tan(\theta/2)\).
Angular distance is measured in units of \(\Delta R \equiv \sqrt{(\Delta\eta)^{2} + (\Delta\phi)^{2}}\).}

\subsection{Detector overview}

The TileCal is a non-compensating sampling calorimeter that employs steel as absorber material and scintillating tiles constituting the active medium placed perpendicular to the beam axis. The scintillation light produced by the ionising particles crossing the detector is collected from each tile edge by a wavelength-shifting (WLS) optical fibre and guided to a photomultiplier tube, see Figure~\ref{fig:TileCalReadout}. 

The calorimeter covers a pseudorapidity range of $|\eta| < 1.7$ and is divided into three segments along the beam axis: one central long barrel (LB) section that is 5.8~m in length ($|\eta| < 1.0$), and two extended barrel (EB) sections ($0.8 < |\eta| < 1.7$) on either side of the LB that are each 2.6~m long\footnote{\AtlasCoordFootnote}. Full azimuthal coverage around the beam axis is achieved with 64 wedge-shaped modules, each covering $\Delta \phi = 0.1$ radians. 
%TileCal is divided in four partitions: a Long Barrel (LB) and an Extended Barrel (EB) per A/C side of the detector; and each partition comprehends 64 wedged modules along the azimuthal direction, providing a granularity $\Delta\phi=0.1$. 
Moreover, these are radially separated into 3 layers: A, B/BC and D. The readout cell units at each module are defined by the common readout of bundles of WLS fibres through a single PMT, as shows Figure~\ref{fig:TileCalSegmentation}. The great majority of the cells have an independent readout by left/right PMTs for each cell side providing redundancy for the cell energy measurement. Additionally, single scintillator plates are placed in the gap region between the barrels (E1 and E2 cells) and in the crack in front of the ATLAS electromagnetic calorimeter End-Cap (E3 and E4 cells).

The data acquisition system of the TileCal is split into four partitions, the ATLAS A-side ($\eta > 0$) and C-side ($\eta < 0$) for both the LB and EB, yielding four logical partitions: LBA, LBC, EBA, and EBC. In total, the TileCal has 5182 cells and 9852 PMTs. 
PMT model Hamamatsu R7877 is used, which is a special customised 8-stage fine-mesh version of Hamamatsu R5900~\cite{Crouau:1997tka}.  
%The Hamamatsu R7877 PMTs are used, special customised version based on R5900 
%A special customised PMTs Hamamatsu R7877 
% A special customised version of the Hamamatsu model R5900 is used. TileCal PMT version is Hamamatsu R7877.
The front-end electronics~\cite{Anderson:2005ym} receive the electrical signals from the PMTs, which are shaped, amplified with two different gains in a 1:64 proportion, and then digitised at 40~MHz sampling frequency~\cite{Berglund:2008zz}.
The bi-gain system is used in order to achieve a 16-bit dynamic range using 10-bit ADCs.
%The High Gain (HG) and Low Gain (LG) readout ensure a better precision under a wider energy range. 
The digital samples are stored in a pipeline memory. Upon ATLAS Level~1~\cite{TRIG-2019-04} trigger decision, seven signal samples are sent to the detector back-end electronics for the reconstruction of the signal amplitude. Complementarily, the PMT signals are integrated over a long period of time (10--20~ms) with analog integrator electronics to measure the energy deposited during caesium calibration scans and the charge induced by proton--proton ($pp$) collisions.

\begin{figure}[htbp]
\begin{center}
    \includegraphics[width=0.5\textwidth]{figures/TileCal_Module4.pdf}
    \caption{Sketch of a TileCal module, showing the scintillation light readout from the tiles by wavelength-shifting optical fibres and photomultiplier tubes (PMTs).}\label{fig:TileCalReadout}
\end{center}
\end{figure}


\begin{figure}[htbp]
\begin{center}
    \includegraphics[width=1.\textwidth]{figures/TileCal_Schematics.pdf}
    \caption{Scheme of the TileCal cell layout in the plane parallel to the beam axis, on the positive $\eta$ side of the detector. The single scintillators E1 and E2 (gap cells), and E3 and E4 (crack cells) located between the barrel and the end-cap are also displayed.}
\label{fig:TileCalSegmentation}
\end{center}
\end{figure}

\subsection{Signal reconstruction}
\label{sec:signal_energy_reco}

%An effect of a charged particle passing through the plastic scintillating tiles in a calorimeter cell is the production of an analog electrical signal. 
In each TileCal channel, an analog electrical signal is sampled with seven samples at 25~ns spacing synchronised with the LHC master clock. These samples are referred to as $S_i$, where $1\leq i \leq 7$, and are in units of ADC counts. Depending on the amplitude of the pulse, either High or Low Gain is used to maximise the signal to noise ratio while avoiding saturation. To reconstruct the sampled signal produced during physics runs, the Optimal Filtering (OF) method is used in the Tile Calorimeter~\cite{Cleland:2002rya, Fullana:816152}. The method linearly combines the samples $S_i$ to calculate the amplitude $A$, phase $\tau$ with respect to the 40 MHz clock and pedestal $p$ of the pulse:

\begin{equation}
A=\sum_{i=1}^{n=7}a_iS_i,\hspace{3em}
A\tau=\sum_{i=1}^{n=7}b_iS_i,\hspace{3em}
p=\sum_{i=1}^{n=7}c_iS_i
\label{eq:of}
\end{equation}

where $a_i$, $b_i$ and $c_i$ are linear coefficients optimised to minimise the bias on the reconstructed quantities introduced by the electronic noise. The normalised pulse shape function, taken as the average pulse shape from test beam data, is used to determine the coefficients. Separate functions are defined for high and low gain. The pulse shape and coefficients are stored in a dedicated database for calibration constants. 


%The expected time of the pulse peak is calibrated such that for particles originating from collisions at the interaction point the pulse should 
The system clock in each digitiser~\cite{Berglund:2008zz} is tuned so that the signal pulses, originating from collisions at the interaction point, 
peak at the central (fourth) sample, synchronous with the LHC clock. The reconstructed value of $\tau$ represents the small time phase in ns between the expected pulse peak and the time of the actual reconstructed signal peak, arising from fluctuations in particle travel time and uncertainties in the electronics readout.

To reconstruct the signals produced in each TileCal channel by the laser calibration system, the same OF method is used as during the physics runs. 
In this case, the pulse shape function corresponding to the signal produced by laser is used to calculate the linear coefficients $a_i$, $b_i$ and $c_i$ from Equation~\eqref{eq:of}. 

%\label{sec:signal_energy_reco}
%\begin{equation}
%A=\sum_{i=1}^{n=7}a_iS_i,\hspace{3em}
%A\tau=\sum_{i=1}^{n=7}b_iS_i,\hspace{3em}
%p=\sum_{i=1}^{n=7}c_iS_i
%\label{eq:of}
%\end{equation}
%
%The cell energy $E\ [\mathrm{GeV}]$ is reconstructed from the signal amplitude $A\ [\mathrm{ADC}]$ as
%
%%\begin{equation}
%%E\ [\mathrm{GeV}] = A\ [\mathrm{ADC}] \times C_{\mathrm{pC}\to \mathrm{GeV}} \times f_{\mathrm{ADC}\to \mathrm{pC}} \times f_{\mathrm{Cs}} \times f_{\mathrm{Laser}}
%% \label{eq:channelEnergy}
%%\end{equation}
%
%\begin{equation}
%E\ [\mathrm{GeV}] = \frac{A\ [\mathrm{ADC}]}{C_{\mathrm{pC}\to \mathrm{GeV}} \times f_{\mathrm{ADC}\to \mathrm{pC}} \times f_{\mathrm{Cs}} \times f_{\mathrm{Laser}}}
% \label{eq:channelEnergy}
%\end{equation}
%
%The $f_{\mathrm{pC}\to \mathrm{GeV}}$ conversion factor is the absolute electromagnetic (EM) energy scale constant measured in tests with electron beams of known energy conducted between 2001 and 2003~\cite{testBeam}. $f_{\mathrm{ADC}\to \mathrm{pC}}$ is the adjustable charge to ADC counts conversion factor determined by charge injection, and the remaining factors, $f_{\mathrm {Cs}}$ and $f_{\mathrm{Laser}}$, are calibration factors measured regularly with the TileCal calibration systems. These are updated frequently in the TileCal data base and used by the data preparation software to maintain the cell energy response stable over time.

\subsection{Energy reconstruction and calibration}

At each level of the TileCal signal reconstruction, there is a dedicated calibration system to monitor the behaviour of the different detector components. 
Three calibration systems are used to maintain a time-independent electromagnetic (EM) energy scale, and account for variations in the hardware and electronics. A movable caesium radioactive $\gamma$-source calibrates the optical components and the PMTs but not the front-end electronics~\cite{Blanchot:2020lyh}. The laser system monitors the PMTs and front-end electronic components used for collision data. The charge injection system (CIS) calibrates the front-end electronics~\cite{TCAL-2010-01}. Figure~\ref{fig:TileCalCalibrationChain} shows a flow diagram summarising the different calibration systems along with the paths followed by the signals from different sources. These three complementary calibration systems also aid in identifying the source of problematic channels. Moreover, the minimum-bias currents (''Particles`` in Figure~\ref{fig:TileCalCalibrationChain}) are used to validate response changes observed by the caesium calibration system. 

\begin{figure}[htbp]
\centering
\includegraphics[width=1.\textwidth]{figures/TileCalCalibrationChain}
\caption{The signal paths for each of the three calibration systems used by the TileCal. The physics signal is denoted by the thick solid line and the path taken by each of the calibration systems is shown with dashed lines.}
\label{fig:TileCalCalibrationChain}
\end{figure}

In each TileCal channel, the signal amplitude $A$ is reconstructed in units of ADC counts using the OF algorithm defined in Equation~\eqref{eq:of}. The reconstructed energy $E$ in units of GeV is derived from the signal amplitude as follows: 

\begin{equation}
E\ [\mathrm{GeV}]=\frac{A\ [\mathrm{ADC}]}{f_{\mathrm{pC}\to \mathrm{GeV}}\cdot f_{\mathrm{Cs}}\cdot f_{\mathrm{Las}} \cdot f_{\mathrm{ADC}\to \mathrm{pC}}}
\label{eq:channelEnergy}
\end{equation}

where each $f_i$ represents a calibration constant or correction factor. 
The factors can evolve in time because of variations in PMT high voltage, stress induced on the PMTs by high light flux or ageing of scintillators due to radiation damage. The calibration systems are used to monitor the stability of these factors and provide corrections for each channel.

The $f_{\mathrm{pC}\to \mathrm{GeV}}$ conversion factor is the absolute EM energy scale constant measured in test beam campaigns~\cite{testBeam}. $f_{\mathrm{ADC}\to \mathrm{pC}}$ is the charge to ADC counts conversion factor determined regularly by charge injection, and the remaining factors, $f_{\mathrm {Cs}}$ and $f_{\mathrm{Las}}$, are calibration factors measured frequently with the TileCal calibration systems. These are updated frequently in the database according to an \textit{interval of validity} (IOV) and used by the data preparation software to keep the cell energy response stable over time. The IOV has a start and end run identifier, between which the stored conditions are valid and applicable to data, and is also stored in the database. 
%These are updated frequently in the database and used by the data preparation software to maintain the cell energy response stable over time.
