\documentclass[a4paper,11pt]{article}
\usepackage{jinstpub} % for details on the use of the package, please see the JINST-author-manual
\usepackage{lineno}
\usepackage{subfig}
\usepackage{placeins}
%\linenumbers

% Proceedings/Special Issues
% Please note that this macro will be edited in production 
%% \proceeding{N$^{\text{th}}$ Workshop on X\\
%% When\\
%% Where}



\title{\boldmath Laser calibration of the ATLAS Tile Calorimeter during LHC Run~2}

% Collaborations

%% [A] If main author
%% \collaboration{\includegraphics[height=17mm]{collabroation-logo}\\[6pt]
%%  XXX collaboration}

%% or
%% [B] If "on behalf of"
%% \collaboration[c]{on behalf of XXX collaboration}


% Authors
% Please note that in JINST a corresponding author is required alongside with their e-mail addres
% The "\note" macro will give a warning: "Ignoring empty anchor...", you can safely ignore it.

%% [A] simple case: 2 authors, same institution
%% \author[1]{A. Uthor\note{Corresponding author.}}
%% \author{and A. Nother Author}
%% \affiliation{Institution,\\Address, Country}

%% or, e.g.
%% [B] more complex case: 4 authors, 3 institutions, 2 footnotes
%% \author[a,b,1]{F. Irst,\note{Corresponding author.}}
%% \author[c]{S. Econd,}
%% \author[a,2]{T. Hird\note{Also at Some University.}}
%% \author[c,2]{and Fourth}
%% \affiliation[a]{Institution_1,\\Address, Country}
%% \affiliation[b]{Institution_2,\\Address, Country}
%% \affiliation[c]{Institution_3,\\Address, Country}

\author[a]{M. N. Agaras}
\author[b]{A. Ahmad}
\author[c]{D. E. Boumediene}
\author[c]{D. Calvet}
\author[d]{P. Conde Muíño}
\author[e]{A. Cortes Gonzalez}
\author[f]{T. Davidek}
\author[g]{G. Di Gregorio}
\author[a]{L. Fiorini}
\author[b]{P. Klimek}
\author[e]{F. Martins}
\author[b]{M. Mlynarikova}
\author[e]{B. C. Pinheiro Pereira}
\author[e,1]{R. Pedro,\note{Corresponding author.}}
\author[h]{R. Rosten}
\author[c]{C. Santoni}
\author[g]{F. Scuri}
\author[i]{A. Solodkov}
\author[i]{O. Solovyanov}
\author[j]{Y. Smirnov}
\author[e]{F. Veloso}
\author[b]{H. Wilkens}

\affiliation[a]{Institut de Física d'Altes Energies (IFAE), Barcelona Institute of Science and Technology, Barcelona; Spain}
\affiliation[b]{CERN, Geneva; Switzerland}
\affiliation[c]{LPC, Université Clermont Auvergne, CNRS/IN2P3, Clermont-Ferrand, France}
\affiliation[d]{Laborat\'{o}rio de Instrumenta\c{c}\~{a}o e F\'{i}sica Experimental de Part\'{i}culas - LIP, Lisboa; Portugal}
\affiliation[e]{Institut für Physik, Humboldt Universität zu Berlin, Berlin; Germany}
\affiliation[f]{Charles University, Faculty of Mathematics and Physics, Prague; Czech Republic}
\affiliation[g]{INFN Sezione di Pisa; Italy}
\affiliation[h]{Ohio State University, Columbus OH; United States of America}
\affiliation[i]{Institute for High Energy Physics of the National Research Centre Kurchatov Institute, Protvino; Russia}
\affiliation[j]{Northern Illinois University, DeKalb; United States of America}

% E-mail addresses: only for the corresponding author
\emailAdd{rute.pedro@cern.ch}
\emailAdd{pawel.klimek@cern.ch}

\abstract{ This article reports the laser calibration of the hadronic Tile Calorimeter of the ATLAS experiment in the LHC Run~2 data campaign. The upgraded Laser~II calibration system is described. The system was commissioned during the first LHC Long Shutdown, exhibiting a stability better than 0.8\% for the laser light monitoring. The methods employed to derive the detector calibration factors with data from the laser calibration runs are also detailed. These allowed to correct for the response fluctuations of the 9852 photomultiplier tubes of the Tile Calorimeter with a total uncertainty of 0.5\% plus a luminosity-dependent sub-dominant term. Finally, we report the regular monitoring and performance studies using laser events in both standalone runs and during proton collisions. These studies include channel timing and quality inspection, and photomultiplier linearity and response dependence on anode current.}

\keywords{Calorimeter, Detector alignment and calibration methods}

%\arxivnumber{1234.56789} % Only if you have one


\begin{document}
\maketitle
\flushbottom

%-------------------------------------------------------------------------------
\section{Introduction}
\label{sec:intro}
% \leavevmode
% \\
% \\
% \\
% \\
% \\
\section{Introduction}
\label{introduction}

AutoML is the process by which machine learning models are built automatically for a new dataset. Given a dataset, AutoML systems perform a search over valid data transformations and learners, along with hyper-parameter optimization for each learner~\cite{VolcanoML}. Choosing the transformations and learners over which to search is our focus.
A significant number of systems mine from prior runs of pipelines over a set of datasets to choose transformers and learners that are effective with different types of datasets (e.g. \cite{NEURIPS2018_b59a51a3}, \cite{10.14778/3415478.3415542}, \cite{autosklearn}). Thus, they build a database by actually running different pipelines with a diverse set of datasets to estimate the accuracy of potential pipelines. Hence, they can be used to effectively reduce the search space. A new dataset, based on a set of features (meta-features) is then matched to this database to find the most plausible candidates for both learner selection and hyper-parameter tuning. This process of choosing starting points in the search space is called meta-learning for the cold start problem.  

Other meta-learning approaches include mining existing data science code and their associated datasets to learn from human expertise. The AL~\cite{al} system mined existing Kaggle notebooks using dynamic analysis, i.e., actually running the scripts, and showed that such a system has promise.  However, this meta-learning approach does not scale because it is onerous to execute a large number of pipeline scripts on datasets, preprocessing datasets is never trivial, and older scripts cease to run at all as software evolves. It is not surprising that AL therefore performed dynamic analysis on just nine datasets.

Our system, {\sysname}, provides a scalable meta-learning approach to leverage human expertise, using static analysis to mine pipelines from large repositories of scripts. Static analysis has the advantage of scaling to thousands or millions of scripts \cite{graph4code} easily, but lacks the performance data gathered by dynamic analysis. The {\sysname} meta-learning approach guides the learning process by a scalable dataset similarity search, based on dataset embeddings, to find the most similar datasets and the semantics of ML pipelines applied on them.  Many existing systems, such as Auto-Sklearn \cite{autosklearn} and AL \cite{al}, compute a set of meta-features for each dataset. We developed a deep neural network model to generate embeddings at the granularity of a dataset, e.g., a table or CSV file, to capture similarity at the level of an entire dataset rather than relying on a set of meta-features.
 
Because we use static analysis to capture the semantics of the meta-learning process, we have no mechanism to choose the \textbf{best} pipeline from many seen pipelines, unlike the dynamic execution case where one can rely on runtime to choose the best performing pipeline.  Observing that pipelines are basically workflow graphs, we use graph generator neural models to succinctly capture the statically-observed pipelines for a single dataset. In {\sysname}, we formulate learner selection as a graph generation problem to predict optimized pipelines based on pipelines seen in actual notebooks.

%. This formulation enables {\sysname} for effective pruning of the AutoML search space to predict optimized pipelines based on pipelines seen in actual notebooks.}
%We note that increasingly, state-of-the-art performance in AutoML systems is being generated by more complex pipelines such as Directed Acyclic Graphs (DAGs) \cite{piper} rather than the linear pipelines used in earlier systems.  
 
{\sysname} does learner and transformation selection, and hence is a component of an AutoML systems. To evaluate this component, we integrated it into two existing AutoML systems, FLAML \cite{flaml} and Auto-Sklearn \cite{autosklearn}.  
% We evaluate each system with and without {\sysname}.  
We chose FLAML because it does not yet have any meta-learning component for the cold start problem and instead allows user selection of learners and transformers. The authors of FLAML explicitly pointed to the fact that FLAML might benefit from a meta-learning component and pointed to it as a possibility for future work. For FLAML, if mining historical pipelines provides an advantage, we should improve its performance. We also picked Auto-Sklearn as it does have a learner selection component based on meta-features, as described earlier~\cite{autosklearn2}. For Auto-Sklearn, we should at least match performance if our static mining of pipelines can match their extensive database. For context, we also compared {\sysname} with the recent VolcanoML~\cite{VolcanoML}, which provides an efficient decomposition and execution strategy for the AutoML search space. In contrast, {\sysname} prunes the search space using our meta-learning model to perform hyperparameter optimization only for the most promising candidates. 

The contributions of this paper are the following:
\begin{itemize}
    \item Section ~\ref{sec:mining} defines a scalable meta-learning approach based on representation learning of mined ML pipeline semantics and datasets for over 100 datasets and ~11K Python scripts.  
    \newline
    \item Sections~\ref{sec:kgpipGen} formulates AutoML pipeline generation as a graph generation problem. {\sysname} predicts efficiently an optimized ML pipeline for an unseen dataset based on our meta-learning model.  To the best of our knowledge, {\sysname} is the first approach to formulate  AutoML pipeline generation in such a way.
    \newline
    \item Section~\ref{sec:eval} presents a comprehensive evaluation using a large collection of 121 datasets from major AutoML benchmarks and Kaggle. Our experimental results show that {\sysname} outperforms all existing AutoML systems and achieves state-of-the-art results on the majority of these datasets. {\sysname} significantly improves the performance of both FLAML and Auto-Sklearn in classification and regression tasks. We also outperformed AL in 75 out of 77 datasets and VolcanoML in 75  out of 121 datasets, including 44 datasets used only by VolcanoML~\cite{VolcanoML}.  On average, {\sysname} achieves scores that are statistically better than the means of all other systems. 
\end{itemize}


%This approach does not need to apply cleaning or transformation methods to handle different variances among datasets. Moreover, we do not need to deal with complex analysis, such as dynamic code analysis. Thus, our approach proved to be scalable, as discussed in Sections~\ref{sec:mining}.
%-------------------------------------------------------------------------------


%-------------------------------------------------------------------------------
\section{The ATLAS Tile Calorimeter}
\label{sec:tilecal}
\newcommand{\AtlasCoordFootnote}{
ATLAS uses a right-handed coordinate system with its origin at the nominal interaction point (IP)
in the centre of the detector and the \(z\)-axis along the beam pipe.
The \(x\)-axis points from the IP to the centre of the LHC ring,
and the \(y\)-axis points upwards.
Cylindrical coordinates \((r,\phi)\) are used in the transverse plane, 
\(\phi\) being the azimuthal angle around the \(z\)-axis.
The pseudorapidity is defined in terms of the polar angle \(\theta\) as \(\eta = -\ln \tan(\theta/2)\).
Angular distance is measured in units of \(\Delta R \equiv \sqrt{(\Delta\eta)^{2} + (\Delta\phi)^{2}}\).}

\subsection{Detector overview}

The TileCal is a non-compensating sampling calorimeter that employs steel as absorber material and scintillating tiles constituting the active medium placed perpendicular to the beam axis. The scintillation light produced by the ionising particles crossing the detector is collected from each tile edge by a wavelength-shifting (WLS) optical fibre and guided to a photomultiplier tube, see Figure~\ref{fig:TileCalReadout}. 

The calorimeter covers a pseudorapidity range of $|\eta| < 1.7$ and is divided into three segments along the beam axis: one central long barrel (LB) section that is 5.8~m in length ($|\eta| < 1.0$), and two extended barrel (EB) sections ($0.8 < |\eta| < 1.7$) on either side of the LB that are each 2.6~m long\footnote{\AtlasCoordFootnote}. Full azimuthal coverage around the beam axis is achieved with 64 wedge-shaped modules, each covering $\Delta \phi = 0.1$ radians. 
%TileCal is divided in four partitions: a Long Barrel (LB) and an Extended Barrel (EB) per A/C side of the detector; and each partition comprehends 64 wedged modules along the azimuthal direction, providing a granularity $\Delta\phi=0.1$. 
Moreover, these are radially separated into 3 layers: A, B/BC and D. The readout cell units at each module are defined by the common readout of bundles of WLS fibres through a single PMT, as shows Figure~\ref{fig:TileCalSegmentation}. The great majority of the cells have an independent readout by left/right PMTs for each cell side providing redundancy for the cell energy measurement. Additionally, single scintillator plates are placed in the gap region between the barrels (E1 and E2 cells) and in the crack in front of the ATLAS electromagnetic calorimeter End-Cap (E3 and E4 cells).

The data acquisition system of the TileCal is split into four partitions, the ATLAS A-side ($\eta > 0$) and C-side ($\eta < 0$) for both the LB and EB, yielding four logical partitions: LBA, LBC, EBA, and EBC. In total, the TileCal has 5182 cells and 9852 PMTs. 
PMT model Hamamatsu R7877 is used, which is a special customised 8-stage fine-mesh version of Hamamatsu R5900~\cite{Crouau:1997tka}.  
%The Hamamatsu R7877 PMTs are used, special customised version based on R5900 
%A special customised PMTs Hamamatsu R7877 
% A special customised version of the Hamamatsu model R5900 is used. TileCal PMT version is Hamamatsu R7877.
The front-end electronics~\cite{Anderson:2005ym} receive the electrical signals from the PMTs, which are shaped, amplified with two different gains in a 1:64 proportion, and then digitised at 40~MHz sampling frequency~\cite{Berglund:2008zz}.
The bi-gain system is used in order to achieve a 16-bit dynamic range using 10-bit ADCs.
%The High Gain (HG) and Low Gain (LG) readout ensure a better precision under a wider energy range. 
The digital samples are stored in a pipeline memory. Upon ATLAS Level~1~\cite{TRIG-2019-04} trigger decision, seven signal samples are sent to the detector back-end electronics for the reconstruction of the signal amplitude. Complementarily, the PMT signals are integrated over a long period of time (10--20~ms) with analog integrator electronics to measure the energy deposited during caesium calibration scans and the charge induced by proton--proton ($pp$) collisions.

\begin{figure}[htbp]
\begin{center}
    \includegraphics[width=0.5\textwidth]{figures/TileCal_Module4.pdf}
    \caption{Sketch of a TileCal module, showing the scintillation light readout from the tiles by wavelength-shifting optical fibres and photomultiplier tubes (PMTs).}\label{fig:TileCalReadout}
\end{center}
\end{figure}


\begin{figure}[htbp]
\begin{center}
    \includegraphics[width=1.\textwidth]{figures/TileCal_Schematics.pdf}
    \caption{Scheme of the TileCal cell layout in the plane parallel to the beam axis, on the positive $\eta$ side of the detector. The single scintillators E1 and E2 (gap cells), and E3 and E4 (crack cells) located between the barrel and the end-cap are also displayed.}
\label{fig:TileCalSegmentation}
\end{center}
\end{figure}

\subsection{Signal reconstruction}
\label{sec:signal_energy_reco}

%An effect of a charged particle passing through the plastic scintillating tiles in a calorimeter cell is the production of an analog electrical signal. 
In each TileCal channel, an analog electrical signal is sampled with seven samples at 25~ns spacing synchronised with the LHC master clock. These samples are referred to as $S_i$, where $1\leq i \leq 7$, and are in units of ADC counts. Depending on the amplitude of the pulse, either High or Low Gain is used to maximise the signal to noise ratio while avoiding saturation. To reconstruct the sampled signal produced during physics runs, the Optimal Filtering (OF) method is used in the Tile Calorimeter~\cite{Cleland:2002rya, Fullana:816152}. The method linearly combines the samples $S_i$ to calculate the amplitude $A$, phase $\tau$ with respect to the 40 MHz clock and pedestal $p$ of the pulse:

\begin{equation}
A=\sum_{i=1}^{n=7}a_iS_i,\hspace{3em}
A\tau=\sum_{i=1}^{n=7}b_iS_i,\hspace{3em}
p=\sum_{i=1}^{n=7}c_iS_i
\label{eq:of}
\end{equation}

where $a_i$, $b_i$ and $c_i$ are linear coefficients optimised to minimise the bias on the reconstructed quantities introduced by the electronic noise. The normalised pulse shape function, taken as the average pulse shape from test beam data, is used to determine the coefficients. Separate functions are defined for high and low gain. The pulse shape and coefficients are stored in a dedicated database for calibration constants. 


%The expected time of the pulse peak is calibrated such that for particles originating from collisions at the interaction point the pulse should 
The system clock in each digitiser~\cite{Berglund:2008zz} is tuned so that the signal pulses, originating from collisions at the interaction point, 
peak at the central (fourth) sample, synchronous with the LHC clock. The reconstructed value of $\tau$ represents the small time phase in ns between the expected pulse peak and the time of the actual reconstructed signal peak, arising from fluctuations in particle travel time and uncertainties in the electronics readout.

To reconstruct the signals produced in each TileCal channel by the laser calibration system, the same OF method is used as during the physics runs. 
In this case, the pulse shape function corresponding to the signal produced by laser is used to calculate the linear coefficients $a_i$, $b_i$ and $c_i$ from Equation~\eqref{eq:of}. 

%\label{sec:signal_energy_reco}
%\begin{equation}
%A=\sum_{i=1}^{n=7}a_iS_i,\hspace{3em}
%A\tau=\sum_{i=1}^{n=7}b_iS_i,\hspace{3em}
%p=\sum_{i=1}^{n=7}c_iS_i
%\label{eq:of}
%\end{equation}
%
%The cell energy $E\ [\mathrm{GeV}]$ is reconstructed from the signal amplitude $A\ [\mathrm{ADC}]$ as
%
%%\begin{equation}
%%E\ [\mathrm{GeV}] = A\ [\mathrm{ADC}] \times C_{\mathrm{pC}\to \mathrm{GeV}} \times f_{\mathrm{ADC}\to \mathrm{pC}} \times f_{\mathrm{Cs}} \times f_{\mathrm{Laser}}
%% \label{eq:channelEnergy}
%%\end{equation}
%
%\begin{equation}
%E\ [\mathrm{GeV}] = \frac{A\ [\mathrm{ADC}]}{C_{\mathrm{pC}\to \mathrm{GeV}} \times f_{\mathrm{ADC}\to \mathrm{pC}} \times f_{\mathrm{Cs}} \times f_{\mathrm{Laser}}}
% \label{eq:channelEnergy}
%\end{equation}
%
%The $f_{\mathrm{pC}\to \mathrm{GeV}}$ conversion factor is the absolute electromagnetic (EM) energy scale constant measured in tests with electron beams of known energy conducted between 2001 and 2003~\cite{testBeam}. $f_{\mathrm{ADC}\to \mathrm{pC}}$ is the adjustable charge to ADC counts conversion factor determined by charge injection, and the remaining factors, $f_{\mathrm {Cs}}$ and $f_{\mathrm{Laser}}$, are calibration factors measured regularly with the TileCal calibration systems. These are updated frequently in the TileCal data base and used by the data preparation software to maintain the cell energy response stable over time.

\subsection{Energy reconstruction and calibration}

At each level of the TileCal signal reconstruction, there is a dedicated calibration system to monitor the behaviour of the different detector components. 
Three calibration systems are used to maintain a time-independent electromagnetic (EM) energy scale, and account for variations in the hardware and electronics. A movable caesium radioactive $\gamma$-source calibrates the optical components and the PMTs but not the front-end electronics~\cite{Blanchot:2020lyh}. The laser system monitors the PMTs and front-end electronic components used for collision data. The charge injection system (CIS) calibrates the front-end electronics~\cite{TCAL-2010-01}. Figure~\ref{fig:TileCalCalibrationChain} shows a flow diagram summarising the different calibration systems along with the paths followed by the signals from different sources. These three complementary calibration systems also aid in identifying the source of problematic channels. Moreover, the minimum-bias currents (''Particles`` in Figure~\ref{fig:TileCalCalibrationChain}) are used to validate response changes observed by the caesium calibration system. 

\begin{figure}[htbp]
\centering
\includegraphics[width=1.\textwidth]{figures/TileCalCalibrationChain}
\caption{The signal paths for each of the three calibration systems used by the TileCal. The physics signal is denoted by the thick solid line and the path taken by each of the calibration systems is shown with dashed lines.}
\label{fig:TileCalCalibrationChain}
\end{figure}

In each TileCal channel, the signal amplitude $A$ is reconstructed in units of ADC counts using the OF algorithm defined in Equation~\eqref{eq:of}. The reconstructed energy $E$ in units of GeV is derived from the signal amplitude as follows: 

\begin{equation}
E\ [\mathrm{GeV}]=\frac{A\ [\mathrm{ADC}]}{f_{\mathrm{pC}\to \mathrm{GeV}}\cdot f_{\mathrm{Cs}}\cdot f_{\mathrm{Las}} \cdot f_{\mathrm{ADC}\to \mathrm{pC}}}
\label{eq:channelEnergy}
\end{equation}

where each $f_i$ represents a calibration constant or correction factor. 
The factors can evolve in time because of variations in PMT high voltage, stress induced on the PMTs by high light flux or ageing of scintillators due to radiation damage. The calibration systems are used to monitor the stability of these factors and provide corrections for each channel.

The $f_{\mathrm{pC}\to \mathrm{GeV}}$ conversion factor is the absolute EM energy scale constant measured in test beam campaigns~\cite{testBeam}. $f_{\mathrm{ADC}\to \mathrm{pC}}$ is the charge to ADC counts conversion factor determined regularly by charge injection, and the remaining factors, $f_{\mathrm {Cs}}$ and $f_{\mathrm{Las}}$, are calibration factors measured frequently with the TileCal calibration systems. These are updated frequently in the database according to an \textit{interval of validity} (IOV) and used by the data preparation software to keep the cell energy response stable over time. The IOV has a start and end run identifier, between which the stored conditions are valid and applicable to data, and is also stored in the database. 
%These are updated frequently in the database and used by the data preparation software to maintain the cell energy response stable over time.

%-------------------------------------------------------------------------------


%-------------------------------------------------------------------------------
\section{The Laser II calibration system}
\label{sec:laser}
\subsection{Laser calibration system}

The TileCal laser system is installed in the ATLAS main service cavern, USA15, located about 100~m away from the detector. In summary, it consists of a laser source, light guides and beam expanders, an optical filter wheel to adjust the light intensity, and beam splitters to dispatch the light to the Tile Calorimeter PMTs through 400 clear optical fibres, 100 to 120~m long. In addition, the system is equipped with a calibration setup designed to monitor the light in various points of the dispatching chain, and with dedicated control and acquisition electronics boards.

The original Laser~I system used during the LHC Run~1 operation~\cite{bib:laser_run_1} was upgraded during the LHC Long Shutdown~1 to a newer version, referred to as Laser~II, which is used since the beginning of the Run~2 data taking. The main purpose of the laser upgrade was to overcome the shortcomings observed in the previous system while maintaining its precision and stability marks. The essential aspects to improve were the stability of the beam expander distributing light to the 400 clear fibres, the photodiodes grounding, reproducibility of the filter wheel position, the electronics, and the overall estimate of the laser light injected into the PMTs of the calorimeter.


\subsection{Upgraded Laser~II system}

The Laser~II system can be described by six main functional blocks: the optics box, the optical filter patch panel, the photodiode box, the PHOCAL (PHOtodiode CALibration) module, a PMT box and a VME crate featuring the LASCAR (LASer CAlibration Rod) electronics card. These blocks are briefly described below, highlighting the upgrades with respect to the previous Laser~I version.


\begin{figure}[htbp]
\begin{center}
    \includegraphics[width=0.8\textwidth]{figures/optics_box}
    \caption{Scheme of the Laser~II optics box, depicting the internal elements and optical paths.}\label{fig:optics_box}
\end{center}
\end{figure}

\begin{figure}[htbp]
\begin{center}
%    \includegraphics[width=0.45\textwidth]{WP_20141009_16_17_31_Pro}
     \includegraphics[width=0.45\textwidth]{figures/WP_20141009_16_17_31_Pro_enhanced}
    \caption{Picture of the optics box (cover removed) placed on the anti-vibration rails and coupled to the fiber bundle.}\label{pic:optics_box}
\end{center}
\end{figure}

\subsubsection*{Optics box}

The light source of the laser system, installed in the so-called optics box, is a commercial Q-switched diode-pumped solid state laser manufactured by SPECTRA-PHYSICS~\cite{ref:laser}, kept from the predecessor system. The frequency doubler permits the infrared laser to emit 532~nm green light, close to the wavelength of the light coming from the detector WLS fibres, peaked at 480~nm. The time width of the individual pulses generated by the laser is 10~ns. Besides the laser source, the optics box houses also the main optical components acting on the laser beam across the light path, as depicted in Figure~\ref{fig:optics_box}. A picture of the laser box is shown in Figure~\ref{pic:optics_box}.

A beam splitter is located at the output of the laser cavity. It divides the laser primary beam into two parts: a small fraction of the light is sent back to a light mixer and the major part is transmitted through a beam expander and a 45\textdegree{} dielectric mirror to a filter wheel. The light exiting the mixer is collected by five clear optical fibres: two are coupled to the PMT box to the two PMTs responsible for generating the trigger signal for the Laser~II data acquisition (DAQ); and three are connected to the monitoring photodiodes (D0 to D2) located in the photodiode box. 

In the expander, the beam spot is expanded from 700~$\mu$m to 2~mm, reducing the light power density to the forthcoming optical elements. The light reflected by the following mirror passes through a motorised filter wheel hosting eight neutral density filters with varied optical densities, with the filter transmissions ranging between 100\% (no filter) and 0.3\%. The combination of this transmission variation and the range of intensities where the laser operation is stable allows to calibrate the TileCal PMTs in an equivalent cell energy range of 500~MeV to 1~TeV. 

The light transmitted by the selected filter is fed into a light mixer by a beam splitter placed downstream the wheel. Three clear fibres routed to the photodiode box collect the light for monitoring (diodes D3 to D5). A second 45\textdegree{} dielectric mirror reflects the light through a shutter into the final beam expander where the laser light is finally dispatched to the detector by 400 clear fibres. Four fibres route the light output of the expander to the photodiode box to monitor its transmission (photodiodes D6 to D9). 
%The light shutter is typically kept closed during LHC collisions to prevent the detector PMTs to be accidentally illuminated by calibration light in case of failure in the control of the Laser~II system.

The 400 long clear fibres are bundled together and transfer the light coming out of the optics box to the TileCal modules (one fibre for two half modules of the central Long barrel, one fibre for each half module per extended barrel and 16 spare fibres). The association between TileCal PMTs and the clear fibre in the bundle is as follows:
\begin{itemize}
\item Long Barrel: one fibre per full LB module, for even PMT numbers in A side and odd numbers C side. Conversely, another fibre for the same LB module, for odd PMT numbers in A side and even numbers in C side.
\item Extended Barrel: one fibre per module per EB side for the even number PMTs and another one for the odd PMTs.
\end{itemize}
Inside each detector module an optical system composed of a light mixer in air dispatches the light to each PMT with individual clear fibres.

This optics box comprises major upgrades with respect to the previous system. Now, a compact design of the optical layout fully includes all the optical elements in one single box, whereas in the Run~1 system the optical elements were located into two different boxes optically connected with a liquid fiber. The optics box is set in an horizontal position to minimise the dust accretion on optical parts and to ease interventions and is mounted on an anti-vibration system, improving beam stability. The final beam expander is new. It was re-designed to improve uniformity in the distribution of the 2~mm beam spot across the 400 fibres' bundle, which has a circular surface of 30~mm diameter. Finally, the system now permits a better estimate of the laser light injected in the calorimeter through a redundant monitoring of the light transmitted in different points of the optical line with 10 photodiodes.

\subsubsection*{Optical filters}


A patch panel with ten optical filters is used to adjust the intensity of the light read by each of the monitoring photodiodes in the photodiode box. In this set up, each one of the ten optical fibres reading out the light at the various points of the beam path in the optics box (after the laser head, after the filter wheel, and at the output of the beam expander) is coupled to a given optical filter in the patch panel. The optical density of the filters range from 0.5 to 2.5 and are such that for each light point probed there is always at least two filters of equal density, providing a redundant light intensity probe for the monitoring photodiodes.

\subsubsection*{Photodiode box}

The photodiode box is a rack containing a set of ten modules, each composed of a Si PIN photodiode (Hamamatsu S3590-08~\cite{ref:photodiode}) coupled to a pre-amplifier, a control card, and a charge injection card to inject an electrical charge into the ten pre-amplifiers. A set of two fibres is connected to the rear end of the photodiode box, in front of each photodiode. One fibre conveys the laser light for monitoring and is connected from the patch panel with the optical filters. The other one comes from the PHOCAL module, where LED light is injected to assess the stability of the photodiodes. In order to minimise the photodiodes' response dependency on the temperature, the temperature in rack is controlled by the water and fan cooling system. The temperature of each photodiode is monitored and kept constant at approximately 30~$^\circ$C with a long term stability below 1~$^\circ$C.

\subsubsection*{PHOCAL module}

This module implements a redundant internal calibration scheme using a blue LED (Nichia blue NSPB520S $\lambda$=470~nm~\cite{ref:led}) to monitor the ten photodiodes of the Laser~II system. The calibration light is simultaneously transmitted to a reference photodiode (Hamamatsu S2744, active area: 10x20 mm2, spectral response range from 320 to 1100~nm~\cite{ref:bigphoto}) providing the signal for the photodiodes' response normalisation. PHOCAL also contains a radioactive source of $^{241}$Am, releasing mostly $\alpha$ particles of 5.6 MeV with an activity of 3.7~kBq. This source ensures the monitoring of the reference photodiode. This module is an addition with respect to the Laser~I system installed in Run~1, where the existing photodiodes were all monitored with a moveable scheme of the $^{241}$Am source.

\subsubsection*{PMT box}

The PMT box contains two PMTs (Hamamatsu R5900~\cite{ref:pmt}) reading out two optical fibres from the optical box. These provide the trigger signal for the Laser~II acquisition system when the laser is flashing. The PMT box also includes a control module used to drive the shutter and the filter wheel in the optics box.

\subsubsection*{LASCAR electronics}

\begin{figure}
\centering
\includegraphics[height=7cm]{figures/Lascar_photo_black.pdf}\quad
\includegraphics[height=7cm]{figures/LASCAR_Cote.pdf}
\caption{Views of the LASCAR card.}\label{fig:laslascar}
\end{figure}


LASCAR, viewed in Figure~\ref{fig:laslascar}, is the electronics board for the acquisition and control of the Laser~II system. It digitises the analog signals (from the eleven photodiodes, the two PMTs and the charge injection system), contains a chip to retrieve the LHC clock signal, makes the interface to drive the laser and contains the module for charge injection and monitoring of the pre-amplifier and digitisation chain of the photodiodes.

The LASCAR board is housed in a VME crate. It's central brain, a Field Programmable Gate Array (FPGA) Cyclone V manufactured by ALTERA~\cite{ref:altera-cyclone}, controls and provides the interface to the main components:

\begin{itemize}

\item \textbf{Charge ADC (QDC):} 32 channel 14-bit\footnote{One of the bits is used to indicate whether the signal is positive or negative, resulting in an effective 13-bit dynamic range for a maximum integrated charge of 2000~pC.} QDC that performs a 500~ns integration and digitization of the analog input charge signals coming from the eleven photodiodes, the two PMTs and from the charge injection system. Prior to the QDC, the analog signal pass through a charge amplifier circuit with two possible gains ($\times 1$ and $\times 4$).
\item \textbf{LILAS (LInearity LASer) card:} This module is responsible for injecting a known charge into the readout electronics (photodiodes pre-amplifiers and digitisation) to monitor its linearity and stability with time. A digital signal from the FPGA is converted to an electric charge by the LILAS 16-bit DAC. The charge is then injected directly into a QDC channel and distributed to the PHOCAL and Photodiode box through lemo cables.
\item \textbf{Time-to-Digital Converter (TDC)}: A TDC is used to measure the Laser time response as function of its intensity. The device has two channels and a time resolution of 280~ps. LASCAR is equipped with a delay system to insure the adequate laser pulse timing irrespective of laser amplitude.
\item \textbf{Timing, Trigger and Control Receiver (TTCrx)}: The TTCrx is an ASIC chip that receives the LHC signals relative to bunch crossing, event counter reset and trigger.
\item \textbf{HOLA}: The High-speed Optical link for ATLAS was conceived to send data fragments via optical fibre to the Read Out System of ATLAS upon receiving a Level 1 Accept trigger from the ATLAS central DAQ.
\item \textbf{LASER Interface:} This mixed analog and digital board is used to control the laser head. The laser intensity is set through an analog signal (0 to 4~V) and the trigger is set with a TTL signal.
\end{itemize}


\subsection{Operating modes}

The Laser~II system can be operated independently as a stand-alone system or integrated in the ATLAS detector data acquisition framework.

\subsubsection*{Stand-alone operating mode}

The stand-alone operation of the Laser~II allows to verify that the system is responding as expected, to monitor its stability and to perform its internal calibration. In this internal calibration mode, LASCAR controls the Laser~II components, switching the shutter off by default, without sending any laser pulse to TileCal PMTs. The following running modes are possible:

\begin{itemize}
\item \textbf{Pedestal mode:} This mode is used to measure a high number of events when no input signal is injected (from the laser, the LED or the radioactive source).
\item \textbf{Alpha source mode:} This mode is used to measure the response of the reference photodiode in the PHOCAL module to the $\alpha$ particles emitted by the $^{241}$Am source.
\item \textbf{LED mode:} In this mode, the LED signal is transmitted to all the photodiodes, including the reference photodiode, via optical fibres. It allows to probe the stability of the photodiodes used to monitor the laser light.
\item \textbf{Linearity mode:} This mode resources the LILAS card to inject a known electrical charge into the preamplifiers of the photodiodes in order to assess the stability of the electronics. It also allows to vary the injected charge to evaluate the linearity of the readout electronics.
\item \textbf{Laser mode:} In this mode, the laser signals of adjustable intensity are sent in the system. The light can be transmitted to the TileCal PMTs, depending on the status of the shutter located inside the optics box. %This mode was mainly used during the Laser~II commissioning and for debugging purposes.
\end{itemize}

A standard internal calibration run combines all the above running modes, starting with the pedestal mode. Once enough pedestal events have been recorded, LASCAR is switched to the next calibration mode, starting from the alpha mode, then the LED mode and the linearity mode increasing the injected charge from 0 to 60000 DAC counts ($\sim$1.9~pC) by steps of 10000 ($\sim$0.3~pC). Finally, the internal calibration ends with the laser mode.
% David Calvet: The maximum voltage of the DAC is 4.096 V and the injection capacitor is 0.5 pF. Thus, the maximum charge (for DAC=65535) should be close to 2.05 pC.

\subsubsection*{ATLAS DAQ operating mode}

The ATLAS DAQ mode is the main operating mode of Laser~II. Its role is to calibrate the TileCal PMTs with laser light. To do so, the Laser~II DAQ is integrated within the global ATLAS DAQ infrastructure, which handles the readout of the PMT signal induced by the laser and the Laser~II run control. This mode is used in two ways:

\begin{itemize}
\item \textbf{Laser mode:} This is the main mode to perform the dedicated calibration runs, when the TileCal is operated independently of the remaining ATLAS detector. 
%Laser pulses are sent to the calorimeter by request of the ATLAS DAQ through the SHAre Few Trigger (SHAFT) interface which forwards the request to LASCAR. 
Laser pulses are sent to the calorimeter by request of the SHAre Few Trigger board (SHAFT) to LASCAR. 
Amplitudes of the signals produced by the photodiodes and the PMTs of the Laser~II system are sent back to the ATLAS DAQ by LASCAR. At the start of run, the filter wheel position and the laser intensity are configured and the shutter opened.
\item \textbf{Laser-in-gap mode:} Laser pulses are emitted in empty bunch-crossings during standard physics runs of the LHC. The TileCal is synchronised with the other ATLAS sub-detectors. The light is fired only in exclusive periods of the beam orbit, where no collisions can occur. The SHAFT board sends a request to LASCAR at a fixed time with respect to the beginning of the LHC orbit to synchronise the laser pulses with the pre-defined orbits and ensure no overlap between laser and physics events. Upon pulse emission, a laser calibration request is sent to the ATLAS central trigger processor by the SHAFT interface. This arrangement synchronises the pulse emission and the TileCal readout with ATLAS DAQ in physics runs.

\end{itemize}

\subsection{Stability of the laser system}

The installation of Laser~II involved a commissioning phase where the performance and the stability of the system were evaluated in the course of the first three months of Run~2. The main parameters to monitor are the ones obtained with the operation of the laser internal calibration mode. The measurements included the pedestal of the photodiodes, the response of the electronics to a known injected charge, the signal of the monitoring photodiodes in response to the PHOCAL LED pulses, and the response of the PHOCAL photodiode to the $^{241}$Am $\alpha$-source. The results are shown in Figure~\ref{fig:laser_stability}. 


\begin{figure}[htbp]
\begin{center}
\subfloat[Pedestal\label{fig:pedestal}]{\includegraphics[height=0.28\textwidth]{figures/LaserII_Ped_Stability_HG}}
\subfloat[Charge Injection\label{fig:charge_injection}]{\includegraphics[height=0.28\textwidth]{figures/LaserII_CIS_Stability_LG_20k}}\\
\subfloat[LED\label{fig:led}]{\includegraphics[height=0.28\textwidth]{figures/LaserII_Led_Stability_HG}}
\subfloat[$^{241}$Am $\alpha-$source\label{fig:alpha}]{\includegraphics[height=0.28\textwidth]{figures/LaserII_Alpha_Stability}}
    \caption{Relative (a) pedestal and (b) charge injection signal for each photodiode (D0 to D9 and reference photodiode in PHOCAL) as a function of time in the course of the first three months of Run~2. The mean signal values are normalised to the mean signal value of the first measurement. (c) Relative response to the PHOCAL LED for each photodiode (D0 to D9) as a function of time. Signal values are normalised to the PHOCAL photodiode signal and to the mean signal value of the first measurement. (d) Relative response of the PHOCAL photodiode to the $^{241}$Am $\alpha$-source subtracting a constant pedestal or correcting the pedestal for the observed fluctuation in time. The mean signal values are normalised to the mean signal value of the first measurement.}\label{fig:laser_stability}
\end{center}
\end{figure}

Figure~\ref{fig:pedestal} shows the average pedestal value recorded in the Laser II stand-alone acquisition mode in high gain for the monitoring photodiodes (D0 to D9) and for the PHOCAL photodiode. The data is normalised relatively to the first data point. The pedestals of the D0--D9 photodiodes are stable within 0.8\% during the commissioning period, whereas for the PHOCAL diode a maximum fluctuation of 1.8\% is observed. 

The stability of the readout electronics response is obtained by injecting a constant charge of 171~pC, 256~pC or 342~pC in several runs across the considered time period. For each injected charge, the signal is acquired in low gain and in high gain. Figure~\ref{fig:charge_injection} shows the results obtained for a 256~pC injected charge in low gain readout. The data, normalised to the first day of data taking, are shown for the electronics channels corresponding to each photodiode and the pedestals are subtracted. All channels exhibit a consistent up-drift reaching 0.8\% at the most in the end of the data taking period. 

In Figure~\ref{fig:led}, the outcome of the photodiode monitoring with the PHOCAL LED in stand-alone high gain runs of Laser~II is presented. The values are normalised to the first data point and the pedestals are subtracted. The response of the photodiodes to the calibration light is very stable in time. The maximum fluctuations do not overcome 0.4\% and do not exhibit any particular trend with time.

Finally, the PHOCAL response to the $^{241}$Am internal $\alpha$-source is displayed in Figure~\ref{fig:alpha} for the low and high gain signal acquisition mode. The response is normalised to the first data point and the pedestals are subtracted. Figure~\ref{fig:alpha} shows a consistent down-drift for the two gain modes, that reach $-0.8$\% in the end of the period under analysis. Given the larger pedestal variation observed for the PHOCAL photodiode, seen in Figure~\ref{fig:pedestal}, the pedestals are further corrected for this fluctuation. The correction has a substantial effect on the obtained photodiode response since the signal induced by the radioactive source (around 600 and 2500 ADC counts in low and high gain, respectively) is just three to six times larger than the pedestal values (around 100 and 750 ADC counts in low and high gain, respectively). Figure~\ref{fig:alpha} also shows the corrected responses, exhibiting a maximum relative variation of about 0.4\%.

The effects of fluctuations of the light monitoring system are taken into account in the calibration of the TileCal PMTs with a run-by-run correction factor. This will be described in Section~\ref{sec:calibration}.


%Material:
%\begin{itemize}
%\item A. Blanco, D. Calvet et al, Upgrade of the Laser calibration system of the ATLAS Tile Calorimeter, ATL-TILECAL-INT-2016-004, https://cds.cern.ch/record/2217913
%\item P. Gris, LASERII performance plots, ATL-COM-TILECAL-2015-071, https://cds.cern.ch/record/2058183
%\item F. Scuri, Performance of the ATLAS Tile LaserII calibration system, https://inspirehep.net/literature/1498841
%\item P. Gris, F. Scuri et al, Performance of the upgraded Laser calibration system of the ATLAS Tile Calorimeter, https://cds.cern.ch/record/2290200 
%\end{itemize}

%-------------------------------------------------------------------------------


%-------------------------------------------------------------------------------
\section{Calibration of the calorimeter with Laser II}
\label{sec:calibration}
\subsection{MuJoCo Model Calibration}
\label{app:model-calibration}

The MuJoCo XML model of the hand requires many parameters, which are then used as the mean of the randomized distribution of each parameter for the environment. Even though substantial randomization is required to achieve good performance on the physical robot, we have found that it is important for the mean of the randomized distributions to correspond to reasonable physical values. We calibrate these values by recording a trajectory on the robot, then optimizing the default value of the parameters to minimize error between the simulated and real trajectory.


To create the trajectory, we run two policies in sequence against each finger. The first policy measures behavior of the joints near their limits by extending the joints of each finger completely inward and then completely outward until they stop moving. The second policy measures the dynamic response of the finger by moving the joints of each finger inward and then outward in a series of oscillations. The recorded trajectory across all fingers lasts a few minutes.


To optimize the model parameters, these trajectories are then replayed as open-loop action sequences in the simulator. 
The optimization objective is to match joint angles after $1$ second of running actions. Parameters 
are adjusted using iterative coordinate descent until the error is minimized. We exclude modifications to the XML
that does not yield improvement over $0.1\%$.


For each joint, we optimize the damping, equilibrium position, static friction loss, stiffness, margin, and the minimum and maximum of the joint range. For each actuator, we optimize the proportional gain, the force range, and the magnitude of backlash in each direction.  Collectively, this corresponds to 264 parameter values.



%-------------------------------------------------------------------------------


%-------------------------------------------------------------------------------
\section{Monitoring with Laser during physics runs}
\label{sec:laseringap}
%Material:
%\begin{itemize}
%\item T. Davidek et al, Time calibration and monitoring in the Tile Calorimeter, https://cds.cern.ch/record/2715632
%\end{itemize}

\subsection{Time monitoring}

\newcommand{\tlaser}{$t^{\mathrm{laser}}_{\mathrm{chan}}$}

%\section{Time monitoring and correction}
The TileCal does not only provide a measurement of the energy that is
deposited in the calorimeter, but it also measures the time when
particles and jets hit the calorimeter cell. This information is
particularly utilised in the removal of signals which do not originate
from $pp$ collisions along with the time-of-flight
measurements of hypothetical heavy slow particles that would reach the
calorimeter.
% For these measurements, the time synchronisation of all
% calorimeter channels represents an important issue.

%%% Few words about the signal reco
The time calibration is also important for the energy reconstruction
itself. As explained in Section~\ref{sec:signal_energy_reco},
physics collision events are reconstructed with the OF algorithm (see
Eq.~(\ref{eq:of})), whose weights depend on the 
expected phase. If the real signal phase significantly differs from the
expected one, the reconstructed amplitude is underestimated. 
Consequently, the time synchronisation of all calorimeter channels
represents an important issue. While the final time calibration is performed
with $pp$ collision data, laser data are extensively used to check its
stability and to spot eventual problems.

Laser calibration events are shot during empty bunch-crossings of
physics runs with a frequency of about 1~Hz. These events, also
referred to as laser-in-gap events, were originally proposed for the
PMT response monitoring. However, they are also extensively used for the
time calibration stability monitoring.

The monitoring tool creates a 2D histogram for each channel and fills
them with the reconstructed time ({\tlaser}) and luminosity block for each
event. These histograms are stored and automatically examined for
anomalies, which include average {\tlaser} being off zero in at least
few consecutive luminosity blocks, unstable {\tlaser} or fraction of
events off zero by more than 20~ns.

The former feature typically indicates a sudden change of the timing settings of
the corresponding digitiser, so-called timing jump. These timing jumps
are corrected by adjusting the associated time constant in the affected
period, as shown in Figure~\ref{fig:timing_jump}.
%
While the timing jumps were very frequent during
Run~1~\cite{bib:laser_run_1} and a lot of effort was invested into their 
correction, they appeared very rarely during Run~2 due
to improved stability of the electronics. This allowed us to focus on
other problems observed with the monitoring tool.

\begin{figure}[htbp]
\centering
    \subfloat[\label{fig:timing_jump_a}]{\includegraphics[width=0.5\linewidth]{figures/laser_325790_ebc09_ch_30_35_express_1}}
    \subfloat[\label{fig:timing_jump_b}]{\includegraphics[width=0.5\linewidth]{figures/laser_325790_ebc09_ch_30_35_blk_1}}
    \caption{An example of the timing jump of +15~ns in EBC09
    channels 30--35 before (a) and after (b) the time constant
    correction as identified with the laser-based monitoring tool. The
    dashed line indicates the expected mean time value.}
    \label{fig:timing_jump}
\end{figure}

% Few channels suffer from {\tlaser} sometimes off by 1 or 2
% bunch-crossings, i.e.\ $\pm25$ or \SI{\pm 50}{ns}. This feature
% affects all three channels managed by the same Data Management Unit
% (TileDMU)~\cite{bib:digi} as demonstrated in
% Fig.~\ref{fig:time_BC_offset}. The problem is intermittent, with a
% frequency typically at a percent-level; nevertheless, the observed
% bunch-crossing offset and affected events are fully correlated across
% the three channels.
% %
% Such events also occur in physics collision data, whose energy and
% time are reconstructed with the Optimal Filtering (OF)
% algorithm. As its parameters depend on the signal phase, a bad
% channel time results into incorrectly determined energy. Studies have
% shown that a difference of \SI{25}{ns} between the actual and supposed
% time phases degrades the reconstructed energy by 35\%. For this reason,
% a dedicated software tool was developed to detect cases
% affected by the bunch-crossing offset in physics data on-the-fly and prevent them from
% propagation to subsequent object reconstruction.

Few channels suffer from {\tlaser} sometimes off by 1 or 2
bunch-crossings, i.e. $\pm25$ or $\pm50$~ns. This feature
affects all three channels managed by the same Data Management Unit
(TileDMU)~\cite{Berglund:2008zz}. The problem is intermittent, with a 
rate at a percent-level; nevertheless, the observed
bunch-crossing offset and affected events are fully correlated across
the three channels. An example is shown in Fig.~\ref{fig:time_BC_offset_a}.
%
Such events also occur in physics collision data at a very similar
rate as in the laser data. Studies have shown that a difference of
25~ns between the actual and supposed time phases degrades the
reconstructed energy by 35\%. For this reason, a dedicated software
tool was developed to detect cases affected by the bunch-crossing
offset in physics data on-the-fly and prevent them from propagation to
subsequent object reconstruction. Figure~\ref{fig:time_BC_offset_b} compares the reconstructed time in affected channels before and
after this tool is applied. The affected events close to +25~ns
are clearly reduced.  

\begin{figure}[htbp]
\centering
    \subfloat[\label{fig:time_BC_offset_a}]{\includegraphics[width=0.5\linewidth]{figures/laser_339037_eba40_final_500x500_1}}
    \subfloat[\label{fig:time_BC_offset_b}]{\includegraphics[width=0.5\linewidth]{figures/physics_EBA40_ch39_41_pub_500x500}}
    \caption{The reconstructed time of laser events as a function of
    luminosity blocks (a): three channels belonging the the same
    TileDMU are superimposed. The majority of events, centred around zero, are well
    timed-in. The events with the bunch-crossing offset are centred at
    +25~ns and these events are fully correlated across the
    three channels. The reconstructed time in physics events in
    the same three channels before (original) and after (corrected)
    the algorithm mitigating the bunch-crossing offset events
    applied (b): the algorithm significantly reduces events centred around
    +25~ns.}
    \label{fig:time_BC_offset}
\end{figure}

\subsection{Dependence of the PMT response on the anode current}

The assumption of a linear relationship between the PMT signal amplitude and the cell energy deposits requires the PMT response to be independent of current. For most cells, the range of currents is small enough that any non-linearity is negligible. In contrast, highly exposed cells, such as the E~cells, experience a large current range between low and high luminosity runs, and between a caesium calibration run and a physics run. Therefore, those cells provide the necessary data to investigate such effects and are used to study the PMT response as a function of the anode current. Particular attention is paid to the difference between response of the E1 and E2~cells with respect to the E3 and E4~cells. The latter are the most exposed TileCal cells, where the larger particle fluence result in larger PMT currents. PMTs with active HV dividers~\cite{ATLAS-TDR-28} are installed in the readout of these cells to mitigate the current dependence, in principle affording larger stability. How well they do so must be understood when using the cells across a wide current range.

The measurement of the anode current comes from data of the TileCal readout of minimum-bias events that are collected during each run.
These minimum-bias data are read out via slow current integrators which were installed for the readout of low signals from the radioactive caesium source used in the calorimeter calibration. The integrators average the current in each cell over a long time window of 10--20~ms to suppress fluctuations in event-to-event energy deposition, diverting only a small fraction of each PMT's output from the primary signal. Large depositions from hard-scattering are also suppressed on long time scales.

Laser-in-gap data and the current measurements of minimum-bias events were analysed for three runs taken in 2018. The particular set of runs were selected to explore a wide current range while minimising the overlap of currents. The PMT signal amplitudes caused by the laser pulses are first subtracted from the pedestal, primarily present due to electronics noise, which comes mostly from the front-end electronics used to shape the signal for the ADC, as well as from the presence of beam-induced and other non-collision background. The signal amplitude is not normalised to the reference diode, as done to determine the PMT calibration described in Section~\ref{sec:determination_of_the_calibration_constants}, since the small instability associated to this monitoring device is often larger than the effects being studied, and so are the uncertainties associated with its correction. Instead, a cleaner approach to minimise the impact of laser intensity fluctuations is adopted, normalising the measurements of the channels from E~cells of interest to a reference TileCal PMT with negligible current range on the same module. In this study, the left PMT of the D6~cell is used as the reference. The E~cell normalised response $R^{\mathrm{E/D6_L}}$ is defined per module as the ratio between the signal amplitudes of the E~cell PMT ($A^\mathrm{E}$) and D6 left PMT ($A^{\mathrm{D6_L}}$):

\begin{equation}\label{eq:Ecell_nD6L}
R^{\mathrm{E/D6_L}} = \frac{\mathrm{A^E}}{A^{\mathrm{D6_L}}}
\end{equation}

The minimum-bias current decays throughout a physics run as the proton beams decay, so the normalised E cell response changes as well if there is a dependence on the current. To determine this dependence, the actual cell response at any given current is compared to the nominal response at zero current. Therefore, the measurement in any given luminosity block is normalised to the mean measurement in the zero current period, i.e. before collisions begin:

\begin{equation}\label{eq:normalisedEcell_nD6L}
\frac{R^{\mathrm{E/D6_L}}}{R_{\mathrm{current=0}}^{\mathrm{E/D6_L}}}
\end{equation}

The baseline used for normalisation is the average E cell response ratio before stable beam declaration, where the first luminosity block with non-zero luminosity appears in the collision run. For each luminosity block of the chosen run, the average and RMS of the E/D6 cell response ratio to laser-in-gap pulses is calculated and normalised to the equivalent quantity at zero current. This is plotted as a function of the average anode current measured with the integrator readout of physics signals during the same luminosity block as shown in Figure~\ref{fig:fittedLiG} for the combination of the three selected runs.

\begin{figure}[htbp]
\begin{center}
\subfloat[E1]{\includegraphics[height=0.49\textwidth,angle=-90, trim=6cm 0 0 0, clip=true]{figures/gr_combined_avg50_E1_EBA10_D6_L}}
\subfloat[E2]{\includegraphics[height=0.49\textwidth,angle=-90, trim=6cm 0 0 0, clip=true]{figures/gr_combined_avg50_E2_EBA10_D6_L}}\\
\subfloat[E3]{\includegraphics[height=0.49\textwidth,angle=-90, trim=6cm 0 0 0, clip=true]{figures/gr_combined_avg50_E3_EBA11_D6_L}}
\subfloat[E4]{\includegraphics[height=0.49\textwidth,angle=-90, trim=6cm 0 0 0, clip=true]{figures/gr_combined_avg50_E4_EBA10_D6_L}}
\caption{Normalised E/D6 cell response ratio as a function of current for an example channel from each family. The low current fit is in blue, while the high current fit is in red. ``Switch lines'' indicates the transition current between the low and high current fits. ``Fit = 1'' indicates the current at which the normalised E/D6 cell response ratio intercepts 1, with a negative value resulting in a non-physical intercept at 0 current. The fitted ratio can be applied as a function of current to correct for the current-dependence of the PMT response.}
\label{fig:fittedLiG}
\end{center}
\end{figure}

Data are pruned by luminosity block if the laser pedestal is not stable or if the number of measurements in the luminosity block is less than 100. Luminosity blocks with a small number of measurements typically overlap with emittance scans or beam adjustments performed by ATLAS and in which the laser is disabled, so these are discarded. To further smooth the data, the measurements are averaged every $50~\mu\mathrm{A}$. The minimum current is chosen to avoid using data from fluctuations in the zero current measurement. The data are adjusted with a piecewise pair of linear fits:

\begin{itemize}
  \item A low current fit from $0.08~\mathrm{\mu A}$ until the transition current
  \item A high current fit from the transition current using all data up through $10~\mathrm{\mu A}$ 
\end{itemize}

The transition current is chosen by calculating the combined $\chi^2/n_{\mathrm{DoF}}$ for the two linear fits with possible transition currents in steps of $0.05~\mathrm{\mu A}$ up to a maximum possible transition current value of $2.30~\mathrm{\mu A}$. The current yielding the minimum $\chi^2/n_{\mathrm{DoF}}$ is chosen as the endpoint of the first linear fit and the beginning of the second fit, with piecewise continuity enforced. The procedure is also shown in Figure~\ref{fig:fittedLiG}. The transition current between low and high current regimes ranges between 0.8 and 2.1$\mathrm{\mu A}$. The PMT response dependence on current is stronger for E1 and E2 cells than for E3 and E4 cells where the active dividers were installed, especially in the high current regime. These results bring evidence that the active dividers are effectively stabilising the PMT response across a wide range of current operation.

Such a study can be used to determine a correction to calibrate the PMT response over current from the normalised PMT response ratio fitted function. In  Figure~\ref{fig:fittedLiG}, it can be seen that a maximum 2--3\% correction would be necessary at extreme high current for E1 and E2 cells, respectively. This requires a precise current measurement throughout the range of currents that may be experienced by the different cells. This study demonstrates that such correction should be more important for cells with passive dividers experiencing higher currents, including A cells. 
%, a work recommended for the coming LHC runs.



%-------------------------------------------------------------------------------


%-------------------------------------------------------------------------------
\section{Channel monitoring and PMT linearity}
\label{sec:monitoring}
%Material
%\begin{itemize}
%\item N. Agaras et al, PMT monitoring with Laser, https://cds.cern.ch/record/2648374
%\item F. Veloso et al, Survey of the ATLAS Tile Calorimeter linearity using the Laser Calibration System, https://cds.cern.ch/record/2749305
%\item G. Di Grigorio, Robustness studies of the photomultipliers reading out TileCal, the central hadron calorimeter of the ATLAS experiment, https://doi.org/10.1016/j.nima.2018.09.047
%\end{itemize}

\subsection{Automated channel monitoring}

Channels having pathological problems need to be promptly identified during the data taking periods. Therefore, an automated daily monitoring utilising the laser system is setup in order to identify and diagnose the channels' issues. This is achieved by analysing the recent laser calibration runs. After each laser run, several monitoring figures are produced by an automated software in order to control the PMT stability and provide the list of problematic channels with possible source of issues. All TileCal channels, including the masked channels after data quality checks, are analysed and flagged according to the algorithm described below.

The automated laser monitoring algorithm is based on the analysis of the PMT response variation, measured for each channel using the Direct method (explained in Section~\ref{sec:determination_of_the_calibration_constants}), and the comparison with other data (applied HV, global behaviour of group of channels associated to the same cell type). 

The time evolution of the channels' response is daily monitored using LG and HG laser runs taken in 15 preceding days, and three categories of channels are defined:
%The flags and the assigment procedure of the flags are listed below:

\begin{itemize}
  \item \textbf{Normal channels:} Channels that have no deviation or a deviation compatible with the mean deviation of similar cells. These channels can be calibrated safely and do not require a special attention.
  \item \textbf{Suspicious channels:} Channels with a deviation slightly higher than the mean deviation of similar cells or a deviation compatible with the one expected from the variation of the HV supply. These channels can be calibrated safely but some follow up may be needed.
    \item \textbf{Channels to be checked:} Channels with large deviations (>10\%) that cannot be explained by the mean deviation observed in cells of similar type nor by HV changes; or channels having a non linear behaviour during the 15 preceding days (jumps or fast drifts). These channels should not be calibrated unless the origin of the effect is understood. In most of the cases, especially in case of a fast drift, the channels need to be masked.
\end{itemize}

% Online monitoring is producing the overview of the PMT response as given in Figures~\ref{fig:map_2018} and \ref{fig:phimap_2018}. 

While this categorisation assists during channel calibration, the automated monitoring of laser data also identifies channels with pathological behaviour and determines the source of the issues encountered, complementing data quality assessment activities. During the data taking period, laser runs are chosen with approximately 10 days interval. For each chosen date, both HG and LG runs are analysed and channels are classified according to their problems as follows:

\begin{itemize}
  \item \textbf{Bad channels:} Channels with large PMT drift (>10\%) or having a wrong behaviour during the 15 preceding days.
  \item \textbf{HV unstable:} Channels with large PMT drift (>10\%) but compatible with HV variation.
%   if the deviation of the channel is compatible with HV and greater than 10\%
  \item \textbf{No laser data:} Channels with low laser signal amplitude (e.g. caused by a laser fibre problem).
  %if the amplitude of the channel is small 
  \item \textbf{Bad laser data:} Channels with corrupted laser calibration data or having problematic reference.
%  if the data during the run is corrupted or having problematic reference.
\end{itemize}

A channel is reported to be problematic if it manifests any type of issue listed above. Figure~\ref{fig:overall} shows the fraction of problematic channels, observed in 2017 and 2018, as a function of time. The maximum number of such channels did not exceed 4 and 3\%, respectively.
%in any gain as it is shown in Figure~\ref{fig:overall} for 2018 and for 2017.

\begin{figure}[h!]
  \centering
  \subfloat[\label{fig:overall_a}]{\includegraphics[height=0.35\linewidth]{figures/PMTmonitoring2017.pdf}}\quad
  \subfloat[\label{fig:overall_b}]{\includegraphics[height=0.35\linewidth]{figures/PMTmonitoring2018.pdf}}
    \caption{\label{fig:overall} The fraction of the problematic channels identified by the  laser monitoring algorithm in 2017 (a) and 2018 (b).
    %overall number of all the problematic channels in 2017 (a) and 2018 (b).
    } 
\end{figure}
\FloatBarrier
%As can be seen in Figure~\ref{fig:overall}, the number of problematic channels in 2017 is greater than in 2018. 



\subsection{PMT linearity monitoring}

The TileCal PMT calibration is performed using laser light with a constant intensity. This procedure contributes to ensure that the calorimeter measures the same output over time for the same input energy deposition, i.e. that its response is stable. However, to guarantee that the calibration factors are accurate across the entire dynamic range of the PMT response and the output signal is directly proportional to the energy deposit one needs to assess the PMT linearity.

The linearity of the TileCal PMT channels was monitored during the Run~2 operation with laser calibration data acquired between 2016 and 2019. The dataset corresponded to a combination of standard laser calibration low gain runs using different filter wheel positions and laser intensity varied in the range of 12k to 18k in DAC counts. The linearity of a given PMT channel is evaluated by comparing the PMT signal to the signal of the reference photodiode D6 of the Laser~II system. The response of the channels should increase linearly with the light intensity, in the same way that the PMT channels respond to increases in the energy deposited in the calorimeter. 

PMTs lose linearity shortly before the saturation point. In addition, the TileCal ADCs saturate at an upper limit of 1023 counts. The saturation amplitude is given by $A_{\mathrm{max}}\;\mathrm{[pC]} = (1023-p)/f_\mathrm{ADC\to pC}$, where $p$ and $f_\mathrm{ADC\to pC}$ are the pedestal and CIS constant values, respectively. Values above the ADC saturation amplitude are not reliable and any non-linear behaviour above it should not be related exclusively to the PMT. Typically, TileCal readout channels start to loose linearity above $\sim750$~pC and reach saturation at $\sim850$~pC. This behaviour can be observed for the TileCal channels that receive enough light. To avoid this issue, amplitudes above $A_{\mathrm{max}} - \sigma_A$, where $\sigma_A$ is the standard deviation of the amplitudes, are excluded from the analysis.

\begin{figure}[htbp]
\centering
\subfloat[\label{fig:PMT_fit}]{\includegraphics[width=0.45\textwidth]{figures/PMT_06_0_EBA01_05_l}}\hfill
\subfloat[\label{fig:integral}]{\includegraphics[width=0.48\textwidth]{figures/integral}}
\caption{(a) Signal in EBA01 PMT 5 (channel 4) versus signal in photodiode~6 using the dataset taken on 2016-07-10. The red line shows the obtained fit. The inset shows the magnified distribution of the low laser-intensity region. (b) Schematic representation of the area obtained by intercepting the joint data points and the fit function, in grey. The deviation from linearity is defined as the ratio between the grey area and the fit function integral between the first ($x_0$) and last ($x_1$) points.}
\end{figure}

\begin{figure}[htbp]
\centering
\includegraphics[width=0.5\textwidth]{figures/LinearEvolutionPlots20.pdf}
\caption{The percentage of PMTs within 1\,\% and 2\,\% deviation from linearity as a function of time for the weekly calibration runs taken between 2016-07-10 and 2019-01-18 are shown.\label{fig:linearity}}
\end{figure}

The PMT signal amplitude versus the reference diode signal is plotted for a set of runs of varied light output taken in the same day. A linear fit is performed iteratively for each data point and all the points with smaller amplitudes. The fit comprising more points within one standard deviation of the fitted line is chosen as final for further analysis. 
An example can be seen in Figure~\ref{fig:PMT_fit}.

The deviation from linearity, in percentage, is defined as the ratio between the area delimited by data points intercepted by the linear fit, and the integral of the linear fit, as shown in Figure~\ref{fig:integral}. %{\color{red}A channel is labelled as having a knee, if the maximum linear point is below the maximum recorded point or below $1.5\sigma$ from the fitted line. If a knee is identified, then the corresponding data points are not taken into account to compute the deviation from linearity.}

Figure~\ref{fig:linearity} shows the percentage of PMTs within 1~\% and 2~\% deviation from linearity as a function of time for the weekly calibration runs taken between 2016 and 2019. The percentage of channels with deviation from linearity less than 1\% is $(99.66 \pm 0.11)\%$ and less than 2\% is $(99.72 \pm 0.09)\%$ considering this period.


\FloatBarrier
%-------------------------------------------------------------------------------

\FloatBarrier


%-------------------------------------------------------------------------------
\section{Conclusion}
\label{sec:conclusion}
%-------------------------------------------------------------------------------

The Laser~II calibration system of the ATLAS Tile Calorimeter probes individually the 9852 PMTs of the detector, together with the readout electronics of each channel. It is one of the three dedicated systems ensuring the calibration of the full calorimeter response. The Laser~II system has undergone a substantial upgrade during the LHC Long Shutdown 1 that improved its stability for the calibration of the calorimeter in Run~2. The readout electronics was renewed and a new light splitter was installed; the system now includes ten photodiodes to monitor the light across its path in the Optics box.

Laser runs were used to regularly determine the PMT response and calculate the calibration constants $f_\mathrm{Las}$, which were updated weekly for the cell energy reconstruction. The PMT response fluctuations are highly correlated with the LHC operations, with the response decreasing with integrated current and recovering in following technical stops of the collider. The calibrations obtained with the laser and caesium source are consistent within 0.4\%, dominating the systematic uncertainty on the PMT calibration scale. In addition, a sub-dominant uncertainty on the PMT relative inter-calibration was found to have a luminosity dependence. The linearity of the PMTs was studied to ensure that the calibration factors are accurate across the dynamic range of the PMT response.% and 99.66\% of them were found to have a deviation from the linearity smaller than 1\%.

Laser events were also used to evaluate the timing of the readout electronics, monitor the stability of time calibration and detect pathological behaviours in the calorimeter channels in data quality activities, contributing to the high TileCal performance in Run 2.
% Moreover, it is shown that the PMT response determined with laser can be used to study the model of PMT degradation with integrated current, which constitutes an important ingredient to estimate the TileCal performance in future runs.

\section*{Acknowledgments}

The authors would like to acknowledge the entire TileCal community for their contribution with the discussions related to this work, the operations acquiring laser calibration data, the input from the data quality activities, and the careful review of this report.


% Bibliography

%% [A] Recommended: using JHEP.bst file
\bibliographystyle{JHEP}
\bibliography{laserRun2Paper.bib}

%% or
%% [B] Manual formatting (see below)
%% (i) We suggest to always provide author, title and journal data or doi:
%% in short all the informations that clearly identify a document.
%% (ii) please avoid comments such as "For a review'', "For some examples",
%% "and references therein" or move them in the text. In general, please leave only references in the bibliography and move all
%% accessory text in footnotes.
%% (iii) Also, please have only one work for each \bibitem.

\end{document}
