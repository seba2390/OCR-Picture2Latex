\documentclass[]{article}

\usepackage[margin=1.0in]{geometry}
\usepackage[capposition=top]{floatrow}
\usepackage{amsmath}
\usepackage{amsfonts}
\usepackage{graphicx}

%opening
\title{Transportability without positivity: a synthesis of statistical and simulation modeling}
\author{Paul N Zivich\textsuperscript{1,2}, 
	    Jessie K Edwards\textsuperscript{2}, 
	    Eric T Lofgren\textsuperscript{3}, 
	    Stephen R Cole\textsuperscript{2}, \\
	    Bonnie E Shook-Sa\textsuperscript{4}, 
	    Justin Lessler\textsuperscript{2,5,6}}
\date{%
	\small
	\textsuperscript{1}Institute of Global Health and Infectious Diseases, University of North Carolina at Chapel Hill, Chapel Hill, NC\\%
	\textsuperscript{2}Department of Epidemiology, Gillings School of Global Public Health, University of North Carolina at Chapel Hill, Chapel Hill, NC\\%
	\textsuperscript{2}Paul G. Allen School for Global Health, Washington State University, Pullman, WA\\%
	\textsuperscript{4}Department of Biostatistics, Gillings School of Global Public Health, University of North Carolina at Chapel Hill, Chapel Hill, NC\\%
	\textsuperscript{5}Carolina Population Center, University of North Carolina at Chapel Hill, Chapel Hill, NC\\[2ex]%
	\textsuperscript{6}Department of Epidemiology, Johns Hopkins Bloomberg School of Public Health, Baltimore, MD\\[2ex]%
	\today
}

\begin{document}

\maketitle

\begin{abstract}
	When estimating an effect of an action with a randomized or observational study, that study is often not a random sample of the desired target population. Instead, estimates from that study can be transported to the target population. However, transportability methods generally rely on a positivity assumption, such that all relevant covariate patterns in the target population are also observed in the study sample. Strict eligibility criteria, particularly in the context of randomized trials, may lead to violations of this assumption. Two common approaches to address positivity violations are restricting the target population and restricting the relevant covariate set. As neither of these restrictions are ideal, we instead propose a synthesis of statistical and simulation models to address positivity violations. We propose corresponding g-computation and inverse probability weighting estimators. The restriction and synthesis approaches to addressing positivity violations are contrasted with a simulation experiment and an illustrative example in the context of sexually transmitted infection testing. In both cases, the proposed model synthesis approach accurately addressed the original research question when paired with a thoughtfully selected simulation model. Neither of the restriction approaches were able to accurately address the motivating question. As public health decisions must often be made with imperfect target population information, model synthesis is a viable approach given a combination of empirical data and external information based on the best available knowledge.
\end{abstract}

\section*{Introduction}
To aid in public health decision making, epidemiologists produce quantitative estimates of the effect of taking different actions (e.g., exposure, treatments, interventions) for specific populations, commonly referred to as target populations. However, the available data are often not a random sample from the desired target population, a problem shared by both randomized trials and observational studies. Instead, one can attempt to `transport' estimates from a separate study sample to the desired target population \cite{westreich_transportability_2017, dahabreh_generalizing_2019, dahabreh_toward_2020, bareinboim_causal_2016, degtiar_review_2023, keiding_perils_2016, cerda_systems_2019}. Methods for transportability have generally proceeded under the assumption that population membership and potential outcomes are independent conditional on a set of observed covariates (i.e., conditional exchangeability) \cite{westreich_transportability_2017, dahabreh_generalizing_2019, dahabreh_toward_2020, degtiar_review_2023}. This assumption requires that all relevant covariate patterns in the target population are observed in the study sample (i.e., deterministic positivity \cite{zivich_positivity_2022}), so that individuals in the study sample (properly reweighted) can `stand-in' for the target population. However, strict eligibility criteria or practical constraints may lead to violations of this assumption.

Here, we consider transportability when positivity does not hold. To address the positivity violation, we review two common approaches: restricting the target population to regions where the positivity assumption holds, and restricting the adjustment set of covariates to a subset where the positivity assumption holds. Both approaches have problems: restricting the target population no longer addresses the motivating question, and restricting the adjustment set requires that the excluded covariates induce little-to-no bias. To avoid these shortcomings, we propose a third approach that uses a combination or synthesis of statistical and simulation modeling. We apply each of these approaches to an illustrative example on an intervention to improve sexually transmitted infection (STI) testing coverage (i.e., proportion tested).

\section*{Motivating Problem}

Consider the following scenario: a clinical colleague approaches us with the goal of increasing STI testing coverage among their patients. They wish to offer patients either electronically ordered self-sampling STI test kits (e-STI) or standard walk-in STI testing via text message. While the clinic is unable to conduct its own randomized trial comparing these approaches, our colleague has access to data from a randomized trial comparing these strategies at a different clinic. They also have a data set containing key demographic variables of patients from their clinic. 

Let $A_i$ indicate the STI testing information presented via text message to patient $i$, with $A_i=1$ being e-STI and $A_i=0$ being walk-in STI testing. Let $Y_i^a$ indicate the (binary) potential outcome of STI testing uptake under the STI testing option $a\in\{0,1\}$, and $Y_i$ be the observed outcome. Additionally, $V_i$ indicate age (in years) and $W_i$ indicates gender (male, female) of patient $i$. Say these covariates were determined to be important factors related to uptake of testing based on substantive knowledge and differed between the clinic and trial populations. Finally, $R_i$ indicates whether patient $i$ was in the clinic data ($R_i=1$) or the randomized trial data ($R_i=2$). For inference, we proceed under a population model, where data are considered to be a random sample of a much larger population \cite{lehmann_elements_2004}.

Our colleague's question is: what is the difference in the probability of having an STI test by 6 weeks for e-STI versus walk-in STI testing being offered among their clinic patients? This parameter can be written as
\[\psi = E[Y^1 | R=1] - E[Y^0 | R=1]\]
with the observed clinic and trial data consisting of $(V_i, W_i, R_i=1)$ and $(Y_i, A_i, V_i, W_i, R_i=2)$, respectively.

To mimic the described scenario, we use data from the \textit{GetTested} trial ($n_2=2063$) \cite{wilson_internet-accessed_2017}, a randomized trial comparing e-STI versus walk-in STI testing. One of the primary outcomes of the trial was any STI test within 6 weeks of randomization. To focus our illustration, observations were treated as marginally randomized (\textit{GetTested} used stratified randomization with different allocation probabilities, so an actual analysis must incorporate this to be unbiased) and observations with missing outcome data (324, 16\%) were ignored. This data is referred to as the `full data', and consisted of the assigned testing option, outcome, age, and gender. To induce a deterministic positivity violation, we restricted the \textit{GetTested} trial data to only men (the `restricted data', $n_2=1016$, 49\%). Because the \textit{GetTested}  trial included women, we can benchmark the different approaches based on the restricted data against the full data. As a stand-in for the clinical target population, we simulated a data set ($n_1=1000$) where age and gender distributions differed from the \textit{GetTested} trial (Table \ref{tab1}).

\begin{table}[]
	\caption{Descriptive statistics for the clinic and restricted trial data}
	\centering
	\begin{tabular}{lcccc}
		\hline
		& \multicolumn{2}{c}{Trial ($n_2=1016$)\textsuperscript{*}} &  & Clinic ($n_1=1000$) \\
		& e-STI               & Walk-in           &  &                     \\ \cline{2-3} \cline{5-5} 
		Age\textsuperscript{†}                     & 23 {[}20, 26{]}     & 22 {[}22,26{]}    &  & 21 {[}19,24{]}      \\
		Female                   & 0 (0\%)             & 0 (0\%)           &  & 574 (57\%)          \\
		Any STI test at 6 weeks\textsuperscript{‡} & 289 (53\%)          & 99 (21\%)         &  & -                   \\ \hline
	\end{tabular}
	\floatfoot{STI: sexually transmitted infection. \\
	* The \textit{GetTested} trial restricted to only males.\\
	† The first number is the median, with the bracketed numbers being the 25th and 75th percentiles, respectively.\\
	‡ Any STI test completed at 6 weeks post randomization.}
	\label{tab1}
\end{table}

\subsection*{Nonparametric Point Identification}

To proceed, $\psi$ is written in terms of the observable data without placing parametric constraints on the relationships or functional forms between variables, referred to as \textit{nonparametric point identification}. For the randomized trial, we rely on the following assumptions
\[Y_i = Y_i^a \text{ if } a=A_i\]
\[E[Y^a | R=2] = E[Y^a | A=a, R=2] \text{ for } a \in\{0, 1\}\]
\[\Pr(A=a | R=2) > 0 \text{ for } a\in\{0, 1\}\]
which correspond to causal consistency, marginal exchangeability for treatment, and positivity for treatment, respectively \cite{zivich_positivity_2022, hernan_estimating_2006, cole_consistency_2009}. These assumptions are given by design for a marginally randomized trial. To transport the trial results to the clinic population \cite{westreich_transportability_2017}, the following additional assumptions may be considered
\begin{equation}
	\begin{aligned}
		E[Y^a | R=1,V=v,W=w] = E[Y^a | R=2,V=v,W=w] \\
		\text{ for all } v,w \text{ where } \Pr(V=v,W=w | R=1) > 0
	\end{aligned}
	\label{exch1}
\end{equation}
\begin{equation}
	\Pr(R=2 | V=v, W=w) > 0 \text{ for all } v,w \text{ where } \Pr(V=v,W=w | R=1) > 0
	\label{pos1}
\end{equation}
where \ref{exch1} is exchangeability between the trial and target populations and \ref{pos1} is the corresponding positivity assumption. As the target population consists of both men and women but the trial was limited to only men, assumtion \ref{pos1} is violated and $\psi$ is not nonparametrically point identified under these assumptions.

\section*{Potential Solutions}

While one always has the option to state that the question cannot be satisfactorily addressed given the available data, we instead consider three possible solutions to addressing the motivating research question: restricting the target population to only men, restricting the covariate set for exchangeability between the trial and clinic such that we no longer need to account for gender, and synthesizing statistical and simulation models. To aid with intuition, the parameter of interest is re-expressed using the law of total expectation
\begin{equation}
	E[Y^a | R=1] = E[Y^a | R=1,W=0]\Pr(W=0 | R=1) +  E[Y^a | R=1,W=1]\Pr(W=1 | R=1)
	\label{parameter}
\end{equation}
As $\Pr(W=w | R=1)$ is identified given the clinic data and $E[Y^a | R=1, W=0]$ is identified with the trial data, the challenge is how to identify $E[Y^a | R=1, W=1]$.

\subsection*{Solution 1: Restrict the Target Population}

As $E[Y^a | R=1, W=1]$ is not identifiable from the trial, we could revise the parameter of interest to be the average causal effect among \textit{only men} at the clinic, $\psi' = E[Y^1 - Y^0 | R=1,W=0]$. Effectively, the original research question is modified by restricting the target population to regions where positivity is met. Identification of this revised parameter replaces \ref{exch1} and \ref{pos1} with
\begin{equation}
	\begin{aligned}
		E[Y^a | R=1,V=v,W=0] = E[Y^a | R=2,V=v,W=0]  \\
		\text{ for all } v \text{ where } \Pr(V=v | R=1,W=0) > 0
	\end{aligned}
	\label{exch2}
\end{equation}
\begin{equation}
	\Pr(R=2 | V=v, W=0) > 0 \text{ for all } v \text{ where } \Pr(V=v | R=1,W=0) > 0
	\label{pos2}
\end{equation}
Here, \ref{pos2} is not violated, so $\psi'$ is nonparametrically point identified. For consistent estimation, either g-computation or inverse probability weighting (IPW) estimators for transportability can be applied (see Appendix A) \cite{westreich_transportability_2017, dahabreh_extending_2020}.

While $\psi'$ is nonparametrically identified, it is distinct from $\psi$. Specifically, this revised analysis no longer addresses the original research question as only a portion of the original target population is considered. Regardless, decisions regarding STI testing at the clinic must be made for women (i.e., avoidance of a decision is still a decision). The unwillingness to address the original question may lead the clinic to restrict e-STI testing offers to men. If e-STI testing is as or more effective in women than men, our analytical decision would then lead to a potential increase or perpetuation of inequalities in STI testing uptake by gender. Alternatively, if e-STI dissuades women but not men from completing STI tests, then adoption e-STI testing for all clinic patients could similarly result in inequalities. By restricting the target population, this analysis provides limited guidance on the policy the clinic should set.

\subsection*{Solution 2: Restricted Covariates for Exchangeability}

A second option is to modify the conditional exchangeability assumption in \ref{exch1} by replacing the adjustment set $\{V,W\}$ with $\{V\}$. Therefore, \ref{exch1} and \ref{pos1} are replaced by
\begin{equation}
	E[Y^a | R=1,V=v] = E[Y^a | R=2,V=v] \text{ for all } v \text{ where } \Pr(V=v | R=1) > 0
	\label{exch3}
\end{equation}
\begin{equation}
	\Pr(R=2 | V=v) > 0 \text{ for all } v \text{ where } \Pr(V=v | R=1) > 0
	\label{pos3}
\end{equation}
This update to exchangeability implies that the effect of STI testing options is homogenous on the difference scale by gender \cite{webster-clark_directed_2021}. Therefore, $\psi$ for the target population is nonparametrically point identified under these revised assumptions. Again, either g-computation or IPW estimators could be used \cite{westreich_transportability_2017, dahabreh_extending_2020}.

Restricting the covariate set to age seems unjustified, especially if discussion with clinical experts led us to originally include gender in the set of covariates for exchangeability between the trial and clinic. Further, this analysis has men directly `stand-in' for women, which leads to similar concerns discussed above if effectiveness differs by gender.

\subsection*{Solution 3: Model Synthesis}

As a third option, we propose a synthesis of statistical and simulation (e.g., mechanistic, agent-based, microsimulation) models, where the simulation model is denoted by $m(a,W_i=1;\beta)$. Importantly, this simulation model is based on knowledge external from the available data. Often, specific values for the simulation model parameters, $\beta$, may not be known but a range, set, or distribution of plausible values (denoted as $\tilde{\beta}$) are feasible to specify. To maintain the generality of our discussion for the proposed approach, how a simulation model is built is left unspecified, with further discussion on this process provided at the end of this section. For estimation, we propose extensions of g-computation and IPW estimators.

\subsubsection*{G-computation}

First, we propose an outcome modeling approach akin to standard g-computation \cite{snowden_implementation_2011}. To generate predicted values of $Y$ under $a$, a combination of statistical and simulation modeling is used
\[f_a(V_i,W_i;\alpha,\beta) = t\{s(a,V_i,W_i=0;\alpha) + m(a, W_i=1;\beta)\}\]
where $s(a,V_i,W_i=0;\alpha)$ is a statistical model for $E[Y | A_i = a, V_i, W_i =0]$ and $t()$ is a generic transformation (e.g., identity for linear models, inverse logit for logistic models). Here, the simulation model, $m(a, W_i=1;\beta)$, shifts the fitted statistical model to generate predicted values of $E[Y^a | V, W=1]$. The statistical and simulation models can then be used to generate predicted values of $Y$ under $a\in\{0,1\}$ for both $W=0$ and $W=1$. The synthesis g-computation estimator for $\psi$ is 
\[\hat{\psi}_g = \frac{1}{n_1} \sum_{i=1}^{n} \{f_1(V_i,W_i;\hat{\alpha}, \tilde{\beta}) I(R_i = 1)\} - \frac{1}{n_1} \sum_{i=1}^{n} \{f_0(V_i,W_i;\hat{\alpha}, \tilde{\beta}) I(R_i = 1)\}\]
where $n$ is the number of observations in the combined clinic and trial data, and $n_1 = \sum_{i=1}^{n} I(R_i = 1)$. To apply this estimator, first the nuisance parameters for the statistical model, $\alpha$, are estimated using the trial data. Keeping with the motivating example, a logistic regression model could be fit predicting STI test completion among men. Next, a simulation model is developed to update the statistical model values for women. Then $\hat{\psi}_g$ is computed based on $f_a(V_i,W_i;\hat{\alpha}, \tilde{\beta})$. To incorporate the uncertainty in $\hat{\alpha}$ and the distribution of $\tilde{\beta}$, we propose the following semiparametric bootstrap or Monte Carlo procedure. First, randomly draw $\tilde{\beta}^*$ from the specified distribution for $\tilde{\beta}$, and $\hat{\alpha}^*$ from a multivariate normal distribution based on $\hat{\alpha}$ and the estimated covariance matrix for $\hat{\alpha}$. Next, resample with replacement the clinic data. Using $\hat{\alpha}^*$, $\tilde{\beta}^*$, and the resampled clinic data, compute $\hat{\psi}_g$ as described above. This process is then repeated many times (e.g., $10^5$ times). Results from all the repetitions can then be jointly summarized by the set of results (e.g., mean and standard deviation, percentiles) or presented visually (e.g., histogram, violin plot, boxplot).

A difficulty in application of this g-computation estimator is the specified simulation model parameters must be conditional on covariates. For example, consider the following synthesis model
\begin{equation*}
	\begin{aligned}
		f_a(V_i, W_i; \alpha, \beta) = & \text{expit}\{s(a,V_i, W_i=0;\alpha) + m(a,W_i=1;\beta)\} \\
		= & \text{expit}\{\alpha_0 + \alpha_1 a + \alpha_2 V_i + \beta_0 W_i + \beta_1 a W_i\}
	\end{aligned}
\end{equation*}
Here, $\beta$ is conditional on $V_i$. As suggested by the Table 2 Fallacy and shown in application \cite{westreich_table_2013, bandoli_revisiting_2018, williamson_factors_2020, westreich_comment_2021, keller_rates_2018}, the direction of conditional coefficients can be counterintuitive. Additionally, the best available external information for simulation models are often marginal parameters \cite{murray_comparison_2017, murray_challenges_2020}. Therefore, properly specifying $\beta$ for the synthesis g-computation estimator may be challenging. To avoid simulation model parameters being conditional on $V$, we propose an IPW estimator.

\subsubsection*{Inverse Probability Weighting}

The synthesis IPW estimator is
\[\hat{\psi}_w = \frac{1}{n_1} \sum_{i=1}^{n} 
\{h_1(V_i,W_i;\hat{\gamma}, \tilde{\delta}, \hat{\eta}) I(R_i = 1)\} - 
\frac{1}{n_1} \sum_{i=1}^{n} \{h_0(V_i,W_i;\hat{\gamma}, \tilde{\delta}, \hat{\eta}) I(R_i = 1)\}\]
where
\[h_a(V_i,W_i;\gamma, \delta, \eta) = t\{s(a,V_i,W_i=0;\gamma, \eta) + m(a,W_i=1;\delta)\}\]
a key distinction from g-computation is the chosen statistical model, with the IPW estimator consisting of a marginal structural model. For example, the following logistic marginal structural model could be specified
\[t\{s(a,V_i,W_i=0;\gamma, \eta)\} = \text{expit}\{\gamma_0 + \gamma_1 A_i\}\]
where $\eta$ are nuisance parameters for the inverse probability weights and only apply to the statistical model (i.e., not used in the simulation model). Therefore, 
\begin{equation*}
	\begin{aligned}
		h_a(V_i,W_i;\gamma, \delta, \eta) = & \text{expit}\{s(a,V_i,W_i=0;\gamma, \eta) + m(a,W_i=1;\delta)\} \\
		= & \text{expit}\{\gamma_0 + \gamma_1 a + \delta_0 W_i + \delta_1 a W_i\}
	\end{aligned}
\end{equation*}
with $\delta$ not conditional on $V_i$.

To implement this estimator, the overall inverse probability weights are estimated. The weights for the illustrative example are
\[\frac{1}{\Pr(A_i=a | R_i = 2)} \times \frac{\Pr(R_i = 2 | V_i, W_i = 0)}{\Pr(R_i = 2 | V_i, W_i = 0)}\]
where a logistic model can be used to estimate the probability of membership in the clinic data restricted to men, $\Pr(R_i = 1 | V_i, W_i = 0)$ \cite{westreich_transportability_2017}. The parameters of a marginal structural model can then be estimated using weighted maximum likelihood estimation with the clinic data. For point and variance estimation, an analogous Monte Carlo procedure with $\hat{\gamma}$ and $\tilde{\delta}$ can be used.

While the synthesis IPW estimator avoids the need for $\delta$ to be conditional on $V$, $\delta$ are assumed to originate from a population with a similar age distribution to the clinic. If the chosen distribution for $\delta$ was based on a population with a different age distribution, then biased estimates of $\psi$ can result. While differences in the age distribution could be accounted for through a more richly specified simulation model, selection of the simulation model parameters for a richer model may be difficult without equally rick background information.

\subsubsection*{Building a Simulation Model}

A primary challenge for application of the proposed estimators is the development of a simulation model. As the simulation model and corresponding parameters are inestimable given the available data, external information must be relied upon. Simulation models can be developed using a variety of strategies, including building a model based on published tables \cite{krijkamp_microsimulation_2018, caglayan_microsimulation_2018, grummon_health_2019}, or modeling intermediate mechanisms (i.e., a mechanistic model) \cite{lessler_mechanistic_2016, kirkeby_practical_2021}. We provide the following advice, but note more detailed discussions on the design of simulation models exist as part of a rich literature in both epidemiology and disease ecology \cite{krijkamp_microsimulation_2018, roberts_conceptualizing_2012, railsback_agent-based_2011, slayton_modeling_2020}. First, the range or distribution of simulation model parameters should aim to capture both uncertainty in the chosen model and variability of the information sources. Second, a diverse set of external information sources may be used to inform selection of plausible parameter distributions (e.g., trials, observational studies, studies on other treatments with similar mechanisms of action, pharmacokinetic studies, animal models, etc.). Third, involvement of a variety of subject-matter experts, using best practices in eliciting expert views \cite{bojke_good_2021, ohagan_expert_2019}, may better capture the state of the evidence base. For building complex models, seeking out modeling-specialist collaborators is highly advisable.

Building a simulation model may appear to entail stronger assumptions relative to restricting the target population or set of covariates. However, ease of implementation is often confused with the simplicity of assumptions. Both restriction approaches can be framed as special cases of the synthesis approach. Restricting the covariate set is equivalent to a simulation model that always contributes zero to $f_a$ or $h_a$ (i.e., $m(a,W_i=1;\beta) = 0$ with complete certainty). If one (intentionally or mistakenly) uses results from the restricted population to draw inference for the original target population (i.e., assumes $\psi=\psi'$), the same assumptions as the restricted covariate set are implicitly imposed, in addition to assuming the distribution of $V$ does not differ by $W$ (i.e., $V_i,W_i=0)$ stands in for $(V_i,W_i=1)$). When viewed in this manner, approaches based on restricting the population or covariate set are shown to require more restrictive assumptions relative to the synthesis approach. So, even if one is skeptical that such external knowledge could exist, model synthesis constitutes an alternative to addressing the original question that clarifies the underlying assumptions and allows for their exploration.

\section*{Application}

As stated previously, the restricted trial and clinic data are used for the illustrative application (Table \ref{tab1}). We estimated the difference in probability of 6-week STI test uptake between text messages offering e-STI versus walk-in STI testing in the clinic population. Analyses used Python 3.9.5 (Beaverton, OR) with the following packages: \texttt{NumPy} \cite{harris_array_2020}, \texttt{SciPy} \cite{virtanen_scipy_2020}, \texttt{pandas} \cite{mckinney_data_2010}, and \texttt{delicatessen} \cite{zivich_delicatessen_2022}. The corresponding code is provided at github.com/pzivich/publications-code. Appendix B includes a brief simulation study demonstrating the performance of the various estimators.

Age was included in all nuisance models with both a linear and quadratic term, and no interaction terms. Details on the standard g-computation and IPW estimators are provided in Appendix A. As a benchmark, we present results that used the full data. The remainder of analyses used the restricted data. For the restricted target population approach, estimators were applied to the clinic data include both men and women. For point and variance estimation, we used M-estimation \cite{zivich_delicatessen_2022, stefanski_calculus_2002}.

For the synthesis approach, the described g-computation and IPW estimators were applied with 10,000 Monte Carlo iterations. The synthesis models were 
\[f_{A_i}(V_i, W_i; \alpha, \beta) = \text{expit}(\alpha_0 + \alpha_1 A_i + \alpha_2 V_i + \alpha_3 V_i^2 + \beta_0 W_i + \beta_1 A_i W_i)\]
for g-computation and 
\[h_{A_i}(V_i, W_i; \alpha, \beta) = \text{expit}(\gamma_0 + \gamma_1 A_i + \delta_0 W_i + \delta_1 A_i W_i)\]
for the IPW estimator. Several different specifications for $\beta$ and $\delta$ were compared (Table \ref{tab2}). The first assumed a strict null (i.e., no effect with certainty), where results from the restricted covariate set and model synthesis were expected to match (up to Monte Carlo error). The second set proposed a null but uncertain distribution using a trapezoidal distribution ranging from -2 to 2 (corresponding to odds ratios from 0.14 to 7.4) and was uniform between -1 to 1 \cite{fox_method_2005}. This selection corresponded to a case where uncertainty was allowed with linearly decreasing belief for values beyond -1,1 and anything beyond -2,2 is thought to be impossible. The third set corresponded to a case where accurate (valid and precise) outside information was available. As a stand-in, we estimated the coefficients of the synthesis model above using the full \textit{GetTested} data. The corresponding $\beta$ or $\delta$ estimates were then used as inputs for the simulation model. Notice that $\beta$ and $\delta$ differ in Table \ref{tab2} due to the distinction between marginal and conditional covariate effects. This choice of simulation model parameters is expected to provide a similar point estimate to the full data, but variance estimates are expected to differ due to the covariance between simulation model parameters being ignored and additional uncertainty in the estimated statistical model parameters in the restricted data. Finally, the fourth set multiplied the previous by negative one to correspond to inaccurate external information.

\begin{table}[]
	\caption{Simulation model parameter distributions}
	\begin{tabular}{llcc}
		\hline
		&                        & $\beta_0$ or $\delta_0$                     & $\beta_1$ or $\delta_1$                     \\ \cline{3-4} 
		\multicolumn{2}{l}{G-computation} &                                             &                                             \\
		& Strict null            & 0                                           & 0                                           \\
		& Uncertain null         & $\text{Trapezoid}(-2,-1,1,2)$               & $\text{Trapezoid}(-2,-1,1,2)$               \\
		& Accurate\textsuperscript{*}              & $\text{Normal}(\mu=-0.0160, \sigma=0.1761)$ & $\text{Normal}(\mu=-0.6270, \sigma=0.2227)$ \\
		& Inaccurate\textsuperscript{†}             & $\text{Normal}(\mu=0.0160, \sigma=0.1761)$  & $\text{Normal}(\mu=0.6270, \sigma=0.2227)$  \\
		\multicolumn{2}{l}{IPW}           &                                             &                                             \\
		& Strict null            & 0                                           & 0                                           \\
		& Uncertain null         & $\text{Trapezoid}(-2,-1,1,2)$               & $\text{Trapezoid}(-2,-1,1,2)$               \\
		& Accurate\textsuperscript{*}              & $\text{Normal}(\mu=0.1380, \sigma=0.1931)$  & $\text{Normal}(\mu=-0.6914, \sigma=0.2460)$ \\
		& Inaccurate\textsuperscript{†}             & $\text{Normal}(\mu=-0.1380, \sigma=0.1931)$ & $\text{Normal}(\mu=0.6914, \sigma=0.2460)$  \\ \hline
	\end{tabular}
	\floatfoot{IPW: inverse probability weighting. \\
	* The accurate simulation model inputs were based on fitting corresponding statistical models using the full \textit{GetTested} data set.\\
	† The inaccurate simulation model inputs were based on the accurate parameters but flipping the sign for the center of the distribution.}
	\label{tab2}
\end{table}

\subsection*{Results}

Results for g-computation and IPW estimators are presented in Figures \ref{fig1} and \ref{fig2}, respectively. Here, the restricted target population and restricted covariate set approaches similarly overestimated the effectiveness of offering e-STI testing for the clinic population, as offering e-STI testing was more effective among men. The similar results were attributable to men and women having similar age distributions in the clinic population. For the model synthesis, results were dependent on the choice of simulation model parameters, as expected. When the simulation model assumed a strict null, the results were nearly identical to the restriction approaches. When the uncertain null distributions were used, results were less precise. However, e-STI testing remained beneficial. In the setting with accurate external knowledge, the synthesis approach gave a nearly identical point estimate to the full data approach, but with a larger estimated variance. In the setting with inaccurate knowledge, the synthesis approach over-estimated the effectiveness of e-STI testing.

\begin{figure}
	\centering
	\caption {G-computation results for transportation of the average causal effect to the clinic.}
	\includegraphics[width=0.9\linewidth]{fig1_gcomp.png}
	\floatfoot{RD: risk difference, 95\% CI: confidence interval. Dashed vertical line indicates the full data point estimate.}
	\label{fig1}
\end{figure}

\begin{figure}
	\centering
	\caption {Inverse probability weighting estimator results for transportation of the average causal effect to the clinic.}
	\includegraphics[width=0.9\linewidth]{fig2_ipw.png}
	\floatfoot{RD: risk difference, 95\% CI: confidence interval. Dashed vertical line indicates the full data point estimate.}
	\label{fig2}
\end{figure}


\section*{Discussion}

Here, we examined three solutions to dealing with positivity violations in transportability problems: restrict the target population, restrict the covariate set for exchangeability, and a synthesis of statistical and simulation modeling. Model synthesis allows epidemiologists to address questions where positivity is violated (unlike restricting the target population) but also avoids modifying the conditional exchangeability assumption (unlike restricting the covariate set). Even if one is willing to assume null or homogeneous effects across non-positive regions, model synthesis allows one to incorporate uncertainty into this assumption. For estimation with the synthesis approach, we proposed two estimators and demonstrated their application. Unsurprisingly, the synthesis approach was sensitive to the quality of the external information.

Previous epidemiologic research has largely focused on comparing statistical and simulation modeling to each other, or triangulation of statistical and simulation model results \cite{murray_comparison_2017, mooney_g-computation_2021, murray_emulating_2021, ip_reconciling_2013, hernan_invited_2015, buchanan_disseminated_2021, el-sayed_social_2012, lofgren_re_2017, ackley_dynamical_2022, halloran_simulations_2017, keyes_invited_2017, edwards_invited_2017, lofgren_mathematical_2014, arnold_dag_2019, ackley_compartmental_2017, mooney_g-computation_2022}; but less epidemiologic research has explicitly described how these methods can be integrated to improve our understanding \cite{cerda_systems_2019}. Regardless, a variety of approaches can be framed as syntheses of statistical and simulation modeling, with notable examples including quantitative bias analysis, sensitivity analysis, Bayesian methods, and measurement error correction \cite{greenland_bayesian_2009, greenland_interval_2004, robins_sensitivity_2000, shepherd_does_2008}. Outside of epidemiology, research in physics, engineering, and climate science have used hybrids of statistical and simulation modeling (often referred to as `phenomenological' and `mathematical' models, respectively) \cite{rahmstorf_semi-empirical_2007, wright_semi-empirical_nodate, sausen_efficiency_2018, rezaei_hybrid_2020}.

Our discussion focused on point identification, but one could instead opt for partial identification via nonparametric bounds \cite{manski_nonparametric_1990, cole_nonparametric_2019}, which avoids exchangeability and positivity assumptions. Here, we could have calculated the lower and upper bounds for pieces of $\psi$ that violated the positivity assumption by predicting either 0 or 1 for all women, respectively. These can then be combined to get lower and upper bounds for $\psi$. Importantly, the proposed synthesis g-computation and IPW estimators correspond to these bounds when the simulation model always generates predictions of 0 (or 1) for women. Therefore, the synthesis approach can also be viewed as a way to narrow the nonparametric bounds via external information.

Outside information was incorporated through a simulation model but there are alternative frameworks for integrating external information. For example, if another trial on e-STI testing among only women was available, the parameter of interest could have been estimated using a fusion design \cite{cole_illustration_2022}, whereby the two trials and clinic data are combined. In this case, the fusion design is likely to be preferred since it means the parameter could be nonparametrically identified. However, access to individual-level data of both trials may not be possible. Simulation models offer a viable alternative that can be accomplished with only published summary results \cite{lofgren_estimated_2021}. 

Future areas ripe for extending the proposed synthesis framework include addressing other sources of systematic errors, alternative estimators, and more precisely stating the inferential model. Here, we focused on transportability under and assumption of marginal randomization of treatment. However, conditional exchangeability and positivity assumptions are commonly used to address other systematic errors, including confounding, missing data, and measurement error \cite{zivich_positivity_2022}. Returning to the example, observations in the \textit{GetTested} trial with missing outcomes were dropped, but this analysis strategy would not be recommended in practice \cite{ware_missing_2012}. To account for missing outcomes by observed covariates, one could consider multiple imputation or inverse probability of missingness weights \cite{perkins_principled_2018}. If missingness depended on unobserved covariates or the missing values themselves; then sensitivity analyses \cite{robins_sensitivity_2000, greenland_basic_1996}, or extensions of the proposed synthesis could be applied. While some extensions of model synthesis may be trivial, there are important details to consider. For example, conditional exchangeability for treatment and the average causal effect requires as symmetric positivity assumption, not present in the positivity assumption for transportability \cite{zivich_positivity_2022, zivich_use_2022}. Additionally, positivity for some systematic errors may remain uncorrectable (e.g., the average causal effect of pregnancy on blood pressure in a population that includes persons unable to become pregnant). Second, alternative estimators might be of interest. Here, we proposed extensions of g-computation and IPW. Further variations on the statistical models, simulation models, or Monte Carlo procedure are also possible. The use of alternative statistical models, machine learning, and potential for multiply robust estimation are of interest \cite{bang_doubly_2005, zivich_machine_2021}. Further, testable implications for simulation models, similar to the natural course for the g-formula \cite{keil_parametric_2014} or balance with IPW \cite{austin_moving_2015}, are also of interest. Finally, the inferential specifics for model synthesis were left largely unspecified. While we refer to the intervals as `confidence intervals', there is no reason to believe that these intervals will always operate the same as standard frequentist confidence intervals, an issue shared with semi-Bayes and other interval estimation methods \cite{greenland_interval_2004, greenland_bayesian_2006}.

Decisions must be made, with or without data on the entire target population. The logical positivist dream of the data alone guiding all actions is long dead \cite{quine_main_1951, robins_impossibility_1999, fajardo-fontiveros_fundamental_2022}. To make any reasonable progress in learning about causal effects, we depend on relevant knowledge external to the data \cite{robins_data_2001}. Within transportability and causal inference more generally, external information has often been limited to the choice of a statistical model and the relations between variables (i.e., arrows in a causal diagram). Here, we extended this idea to include substantive knowledge outside of the data to address positivity violations through joint use of statistical and simulation modeling.

\section*{Acknowledgments}

This work was supported in part by T32-AI007001 (PNZ), R01-AI157758 (JKE, BES, SRC), R01- GM140564 (JKE, JL), and R35-GM147013 (ETL).
Corresponding code is available at https://github.com/pzivich/publications-code

\small
\bibliography{biblio}{}
\bibliographystyle{ieeetr}

\newpage 

\section*{Appendix}

\subsection*{Appendix A: Estimators}

In the applied example and simulations, M-estimation is used for the restricted population and covariate set estimators. Briefly, an M-estimator, $\hat{\theta}$, is the solution for $\theta$ in the estimating equation $\sum_{i=1}^{n} \varphi(O_i;\theta) = 0$; where $O_i$ is the observed data for $n$ independent and identically distributed units, $\theta$ is a $k$ dimensional vector of parameters, and $\varphi$ is a known $k$-dimensional vector of estimating functions that do not depend on $i$ \cite{zivich_delicatessen_2022, stefanski_calculus_2002}. The variances of the parameter estimates can be estimated using the empirical sandwich estimator
\[V(O_i; \hat{\theta}) = \frac{1}{n} \left[B(O_i; \hat{\theta})^{-1} F(O_i; \hat{\theta}) \{B(O_i; \hat{\theta})^{-1}\}^{T}\right]\]
where the `bread' is
\[B(O_i; \hat{\theta}) = \frac{1}{n} \sum_{i=1}^{n} \{- \varphi(O_i; \hat{\theta})\}\]
with $\varphi'(O_i;\hat{\theta})$ indicating the matrix of partial derivatives, and the `filling' is
\[F(O_i; \hat{\theta}) = \frac{1}{n} \sum_{i=1}^{n} \{\varphi(O_i; \hat{\theta}) \varphi(O_i; \hat{\theta})^T\}\]
Importantly, the sandwich estimator allows for the propagation of uncertainty between parameters that depend on each other. In short, a consistent variance can be easily estimated while avoiding more computationally demanding procedures, like the nonparametric bootstrap.

\subsubsection*{Restrict the Target Population}

For the restricted target population g-computation estimator, the stacked estimating functions were
\[\varphi(O_i; \theta) = 
\begin{bmatrix}
	I(R_i = 2) \left(\left[ Y_i - \text{expit}\left\{g(A_i,V_i)^T \alpha \right\} \right] g(A_i, V_i)\right)\\
	I(R_i = 1, W_i = 0) \left[\text{expit}\left\{g(0,V_i)^T \alpha\right\} - \theta_0\right] \\
	I(R_i = 1, W_i = 0) \left[\text{expit}\left\{g(1,V_i)^T \alpha\right\} - \theta_1\right] \\
	(\theta_1 - \theta_0) - \theta_2
\end{bmatrix}\]
where $\theta = (\alpha, \theta_0, \theta_1, \theta_2)$, $g(A_i,V_i) = (1, A_i, V_i, V_i^2)$ is the design matrix and $\text{expit}(b) = \frac{\exp(b)}{1 + \exp(b)}$. The first estimating function is a logistic regression model for $Y_i$ among the trial data. The second and third estimating functions are the predicted mean under $A_i=0$ and $A_i=1$ among men in the clinic data, respectively. The last estimating function is the risk difference.

The inverse probability weighting (IPW) estimator consisted of the following stacked estimating functions
\[\varphi(O_i; \theta) = 
\begin{bmatrix}
	I(R_i=2) \left\{A_i - \text{expit}(\mu)\right\} \\
	I(W_i = 0) \left(\left[I(R_i = 1) - \text{expit}\left\{l(V_i)^T \sigma\right\}\right] l(V_i) \right) \\
	(Y_i - \theta_0) \frac{I(R_i = 2) \text{expit}\left\{l(V_i)^T \sigma \right\} }{1 - \text{expit}\left\{l(V_i)^T \sigma \right\}} \frac{I(A_i = 0)}{1 - \text{expit}(\mu)} \\
	(Y_i - \theta_1) \frac{I(R_i = 2) \text{expit}\left\{l(V_i)^T \sigma \right\} }{1 - \text{expit}\left\{l(V_i)^T \sigma \right\}} \frac{I(A_i = 1)}{\text{expit}(\mu)} \\
	(\theta_1 - \theta_0) - \theta_2
\end{bmatrix}\]
where $\theta = (\mu, \sigma, \theta_0, \theta_1, \theta_2)$ and $l(V_i) = (1, V_i, V_i^2)$ is the design matrix. The first estimating function is an intercept-only regression model for assigned STI testing in the trial data. The second estimating function is a logistic model for the conditional probability of being in the clinic, fit using only men. The third and fourth estimating functions are the Hajek estimators for $A_i=0$ and $A_i=1$, respectively. The last estimating function is the risk difference.

\subsubsection*{Restricting the Covariate Set}

The g-computation estimator for the restricted covariate set was implemented via the following stacked estimating functions
\[\varphi(O_i; \theta) = 
\begin{bmatrix}
	I(R_i = 2) \left(\left[ Y_i - \text{expit}\left\{g(A_i,V_i)^T \alpha \right\} \right] g(A_i, V_i)\right)\\
	I(R_i = 1) \left[\text{expit}\left\{g(0,V_i)^T \alpha\right\} - \theta_0\right] \\
	I(R_i = 1) \left[\text{expit}\left\{g(1,V_i)^T \alpha\right\} - \theta_1\right] \\
	(\theta_1 - \theta_0) - \theta_2
\end{bmatrix}\]
The key difference from the previous g-computation estimating functions is predictions are generated in the second and third equations for both men and women.

The stacked estimating functions for the IPW estimator 
\[\varphi(O_i; \theta) = 
\begin{bmatrix}
	I(R_i=2) \left\{A_i - \text{expit}(\mu)\right\} \\
	\left[I(R_i = 1) - \text{expit}\left\{l(V_i)^T \sigma\right\}\right] l(V_i) \\
	(Y_i - \theta_0) \frac{I(R_i = 2) \text{expit}\left\{l(V_i)^T \sigma \right\} }{1 - \text{expit}\left\{l(V_i)^T \sigma \right\}} \frac{I(A_i = 0)}{1 - \text{expit}(\mu)} \\
	(Y_i - \theta_1) \frac{I(R_i = 2) \text{expit}\left\{l(V_i)^T \sigma \right\} }{1 - \text{expit}\left\{l(V_i)^T \sigma \right\}} \frac{I(A_i = 1)}{\text{expit}(\mu)} \\
	(\theta_1 - \theta_0) - \theta_2
\end{bmatrix}\]
Again, the key difference from the IPW estimator under the restricted target population was the second estimating function not being restricted to men (i.e., $W_i = 0$).

\subsection*{Appendix B: Simulations}

\subsubsection*{Data Generation}

Covariates for the clinic population ($R_i = 1$) were generated according to the following distributions
\[W_i \sim \text{Bernoulli}(0.667)\]
\[V_i \sim \text{Trapezoid}(\text{min}=18, \text{mode}_1 = 18, \text{mode}_2 = 25, \text{max} = 30)\]
$V_i$ was additional rounded to be an integer. Covariates for the trial population ($R_i=2$) were generated from
\[W_i \sim \text{Bernoulli}(0.0)\]
\[V_i \sim \text{Trapezoid}(\text{min}=18, \text{mode}_1 = 25, \text{mode}_2 = 30, \text{max} = 30)\]
with $V_i$ being rounded again. To summarize, the clinic population included men and women, and consisted of a younger population than the trial. Outcomes were generated from the following model
\[Y_i \sim \text{Bernoulli}(p_{Y_{i}})\]
\[p_{Y_i} = \text{expit}(-3.25 + 1.50 A_i + 0.08 - 0.02 W_i -0.65 A_i W_i)\]
The true value for $\psi$ ($0.216697$) was approximated by simulating the potential outcomes for 10 million observations with the clinic covariate pattern.

\subsubsection*{Methods}

The previously described estimating functions for the restricted g-computation and IPW estimators were applied in the simulations, with the following modifications to the design matrices for the corresponding nuisance models: $g(A_i, V_i) = (1, A_i, V_i)$ and $l(V_i) = (1, V_i, V_i I(V_i > 25))$. For model synthesis, the model for g-computation was
\[f_a(V_i, W_i; \alpha, \beta) = \text{expit}(\alpha_0 + \alpha_1 a + \alpha_2 V_i + \beta_0 W_i + \beta_1 A_i W_i)\]
and the model for the IPW estimator was
\[h_a(V_i, W_i; \gamma, \delta, \eta) = \text{expit}(\gamma_0 + \gamma_1 a + \delta_0 W_i + \delta_1 A_i W_i)\]
Point, lower, and upper confidence intervals were estimated by the median, 2.5\textsuperscript{th}, and 97.5\textsuperscript{th} percentiles, respectively, from 5000 repetitions of the semiparametric bootstrap. Five sets of $\beta$ and $\delta$ specifications for each estimator were compared: strict null, uncertain null, accurate, inaccurate, and accurate with covariance. The strict null assumed all $\beta = \delta = 0$. Results for this specification are expected to match the restricted target population and covariate set. The uncertain null used a trapeziod distribution that is uniform from -1 to 1 and ranges from -2 to 2. For the accurate parameter specification, we used a `secret' trial to reflect variable external knowledge in the simulations. In tandem with the clinic and trial data sets, a secret second trial was conducted in each iteration of the simulation. The secret trial consisted of 2000 observations generated following the covariate pattern in the clinic population with $A_i$ randomly assigned and outcomes generated from the preceding model for $p_{Y_i}$. The secret trial was used to estimate $\hat{\beta}$, $\hat{\delta}$, $\widehat{Var}(\hat{\beta})$, and $\widehat{Var}(\hat{\delta})$; which were then provided to the model synthesis estimators. Importantly, no individual level data from the secret trial was used by the model synthesis estimators (i.e., the synthesis estimators only saw the estimated parameters from the secret trial). By taking this approach, the simulation model parameters were accurate in expectation but vary between iterations. The accurate specification is expected to be unbiased, but confidence interval coverage is expected to be conservative since the covariance between the parameters in $\hat{\beta}$ or $\hat{\delta}$ was ignored. The inaccurate parameter specification used the accurate specification parameters multiplied by negative one. This specification was expected to be biased due to the reversal of the direction of the relationships. The accurate with covariance parameters used the secret trial, but the estimated covariance matrices for $\hat{\beta}$ and $\hat{\delta}$ were used to draw parameters from a multivariate normal distribution (as opposed to independent normal distributions). As the covariance between parameters is not ignored, confidence intervals are expected to have nominal coverage.

Simulations consisted of 2000 iterations with $n_1 = 1000$ for the clinic data and $n_2 = 1000$ for the the trial data. Evaluation metrics were bias, confidence limit difference (CLD), and 95\% confidence interval coverage \cite{morris_using_2019}. As the variance estimates for the model synthesis approach were not assumed to be normally distributed in all cases, we do not report the average standard error or the standard error ratio. Bias was defined as the mean of the difference between the estimate and true value. CLD was defined as the mean of the difference between the upper and lower confidence intervals, and is an indicator of precision (e.g., smaller CLD indicates greater precision). 95\% confidence interval coverage was defined as the proportion of intervals that contained the truth. Simulations were conducted using Python 3.9.5 (Beaverton, OR) with the following packages: \texttt{Numpy} \cite{harris_array_2020}, \texttt{SciPy} \cite{virtanen_scipy_2020}, \texttt{pandas} \cite{mckinney_data_2010}, and \texttt{delicatessen} \cite{zivich_delicatessen_2022}.

\subsubsection*{Results}

The restricted target population and restricted covariate set were biased and had poor confidence interval coverage (Appendix Table \ref{atab1}). Results were approximately equal across these approaches since the distribution of $V_i$ did not differ by $W_i$. As expected, the performance of the model synthesis approach depended on the selection of parameters. For the strict null, results for g-computation and IPW were nearly identical to the restriction approaches. For the uncertain null, there was some bias, but 95\% confidence interval coverage was 100\%. This over-coverage results from the wide confidence interval, as indicated by the CLD. For the accurate specification, bias was near zero and confidence interval coverage was slightly above 95\% for both estimators. G-computation was more precise than IPW as seen in the CLD. For the inaccurate specification, bias was the greatest relative to all other approaches. Finally, the accurate specification with the covariance matrix had near zero bias and 95\% confidence interval coverage, which is in line with the expected frequentist performance of the synthesis model for this mechanistic model specification.

\begin{table}[h]
	\caption{Simulation results}
	\centering
	\begin{tabular}{lllccc}
		\hline
		&       &                               & Bias   & CLD   & 95\% CI Coverage \\ \cline{4-6} 
		\multicolumn{3}{l}{Restrict target population} &        &       &                  \\
		& \multicolumn{2}{l}{G-computation}     & 0.105  & 0.112 & 4\%              \\
		& \multicolumn{2}{l}{IPW}               & 0.106  & 0.159 & 27\%             \\
		\multicolumn{3}{l}{Restrict covariate set}     &        &       &                  \\
		& \multicolumn{2}{l}{G-computation}     & 0.105  & 0.112 & 4\%              \\
		& \multicolumn{2}{l}{IPW}               & 0.106  & 0.158 & 26\%             \\
		\multicolumn{3}{l}{Model synthesis}            &        &       &                  \\
		& \multicolumn{2}{l}{G-computation}     &        &       &                  \\
		&       & Strict Null                   & 0.104  & 0.111 & 4\%              \\
		&       & Uncertain Null                & 0.077  & 0.430 & 100\%            \\
		&       & Accurate                      & -0.003 & 0.168 & 98\%             \\
		&       & Inaccurate                    & 0.204  & 0.163 & 0\%              \\
		&       & Accurate with covariance      & -0.003 & 0.151 & 95\%             \\
		& \multicolumn{2}{l}{IPW}               &        &       &                  \\
		&       & Strict Null                   & 0.105  & 0.157 & 27\%             \\
		&       & Uncertain Null                & 0.077  & 0.448 & 100\%            \\
		&       & Accurate                      & -0.001 & 0.201 & 96\%             \\
		&       & Inaccurate                    & 0.204  & 0.192 & 2\%              \\
		&       & Accurate with covariance      & -0.001 & 0.186 & 94\%             \\ \hline
	\end{tabular}
	\floatfoot{CLD: confidence limit difference, CI: confidence interval, IPW: inverse probability weighting. Results are for 2000 iterations.\\
	Bias was defined as mean of the difference between the estimate and true value, where the true value was based on the potential outcomes of 10 million simulated observations.\\
	CLD was defined as the mean of the difference between the upper and lower CI. 95\% CI coverage was defined as the proportion of intervals that contained the truth.}
	\label{atab1}
\end{table}


\end{document}
