\documentclass[review,12pt]{elsarticle}
\usepackage{amsthm}
\usepackage{lineno,hyperref}
\usepackage{moreverb}
\usepackage{graphicx,subfig,wrapfig}
\usepackage{algorithm}
\usepackage{algpseudocode}
\usepackage{color}
\usepackage{multirow}
\usepackage{amsmath}
\usepackage{amssymb}
\usepackage{txfonts}
\usepackage{afterpage}
\usepackage[margin=0.8in]{geometry}
\usepackage{mathrsfs}
\usepackage{tikz,amsmath}
\usepackage{pstricks}
\usepackage{diagbox}
\usepackage{arydshln}
\usepackage{dsfont}
\usepackage{smartdiagram}
%\usepackage{ulem}
\usepackage[normalem]{ulem}
\usepackage{todonotes}
\usepackage{booktabs}
\usepackage{appendix}
\usepackage{verbatim}
%\usepackage{smartdiagram}

\usetikzlibrary{calc,trees,positioning,arrows,chains,shapes.geometric,%
    decorations.pathreplacing,decorations.pathmorphing,shapes,%
    matrix,shapes.symbols}

\tikzset{
>=stealth',
  punktchain/.style={
    rectangle, 
    rounded corners, 
    % fill=black!10,
    draw=black, very thick,
    text width=10em, 
    minimum height=2em, 
    text centered, 
    on chain},
  line/.style={draw, thick, <-},
  element/.style={
    tape,
    top color=white,
    bottom color=blue!50!black!60!,
    minimum width=6em,
    draw=blue!40!black!90, very thick,
    text width=10em, 
    minimum height=3.5em, 
    text centered, 
    on chain},
  every join/.style={->, thick,shorten >=1pt},
  decoration={brace},
  tuborg/.style={decorate},
  tubnode/.style={midway, right=2pt},
}

\tikzstyle{arrow} = [thick,->,>=stealth]
% adding line numbers
%\usepackage{lineno}
%\linenumbers %Adding line numbers to documents


\newtheorem{thm}{Theorem}
\newtheorem{lem}[thm]{Lemma}
\newdefinition{rmk}{Remark}
\newproof{pf}{Proof}
\newcommand{\bn}{\boldsymbol{n}}
\newcommand{\Q}{\boldsymbol{Q}}
\newcommand{\bD}{\boldsymbol{D}}
\newcommand{\bG}{\boldsymbol{G}}
\newcommand{\bx}{\boldsymbol{x}}
\newcommand{\by}{\boldsymbol{y}}
\newcommand{\bX}{\boldsymbol{X}}
\newcommand{\bY}{\boldsymbol{Y}}
\newcommand{\bZ}{\boldsymbol{Z}}
\newcommand{\bs}{\boldsymbol{s}}
\newcommand{\bt}{\boldsymbol{t}}
\newcommand{\ba}{\boldsymbol{a}}
\newcommand{\bb}{\boldsymbol{b}}
\newcommand{\bc}{\boldsymbol{c}}
\newcommand{\bz}{\boldsymbol{z}}
\newcommand{\bq}{\boldsymbol{q}}
\newcommand{\bW}{\boldsymbol{W}}
\newcommand{\bu}{\boldsymbol{u}}
\newcommand{\bw}{\boldsymbol{w}}
\newcommand{\mx}{\mb{x}}
\newcommand{\pB}{\mathfrak{b}}


\newcommand{\bxi}{\boldsymbol{\xi}}
\newcommand{\btheta}{\boldsymbol{\theta}}
\newcommand{\bsigma}{\boldsymbol{\sigma}}
\newcommand{\bmu}{\boldsymbol{\mu}}

\newcommand{\mbI}{\mathbf{I}}
\newcommand{\mbT}{\mathbf{T}}
\newcommand{\mbbR}{\mathbb{R}}
\newcommand{\mbbE}{\mathbb{E}}
\newcommand{\mbv}{\mathbf{v}}
\newcommand{\mbW}{\mathbf{W}}
\newcommand{\mbLambda}{\mathbf{\Lambda}}
\newcommand{\mbSigma}{\mathbf{\Sigma}}
\newcommand{\mbU}{\mathbf{U}}
\newcommand{\mbQ}{\mathbf{Q}}
\newcommand{\mbP}{\mathbf{P}}
\newcommand{\norm}[2]{\left\| #1 \right\|_{#2}}

\newcommand{\mb}{\mathbf}

%
%\renewcommand{\baselinestretch}{1.0}
%
%\newcommand{\ql}[1]{{\color{magenta}{#1}}}
\newcommand{\ql}[1]{{\color{violet}{#1}}}
\newcommand{\xw}[1]{{\color{blue}{#1}}}
\newcommand{\revs}[1]{{\color{blue}{#1}}}
\renewcommand{\algorithmicrequire}{ \textbf{Input:}}     
\renewcommand{\algorithmicensure}{ \textbf{Output:}}
\renewcommand{\mathbf}{\boldsymbol}
\newcommand{\xs}{\mathbb}

\DeclareMathOperator*{\argmax}{arg\,max}
\DeclareMathOperator*{\argmin}{arg\,min}


\hyphenation{MATLAB}
\newcommand{\comm}[1]{}



% remove priprint footnote
\makeatletter
\def\ps@pprintTitle{%
   \let\@oddhead\@empty
   \let\@evenhead\@empty
   \let\@oddfoot\@empty
   \let\@evenfoot\@oddfoot
}
\makeatother


\modulolinenumbers[5]

\journal{Journal of Computational Physics}
%\journal{}

%%%%%%%%%%%%%%%%%%%%%%%
%% Elsevier bibliography styles
%%%%%%%%%%%%%%%%%%%%%%%
%% To change the style, put a % in front of the second line of the current style and
%% remove the % from the second line of the style you would like to use.
%%%%%%%%%%%%%%%%%%%%%%%

%% Numbered
%\bibliographystyle{model1-num-names}

%% Numbered without titles
%\bibliographystyle{model1a-num-names}

%% Harvard
%\bibliographystyle{model2-names.bst}\biboptions{authoryear}

%% Vancouver numbered
%\usepackage{numcompress}\bibliographystyle{model3-num-names}

%% Vancouver name/year
%\usepackage{numcompress}\bibliographystyle{model4-names}\biboptions{authoryear}

%% APA style
%\bibliographystyle{model5-name\usepackage{tikz,amsmath}s}\biboptions{authoryear}

%% AMA style
%\usepackage{numcompress}\bibliographystyle{model6-num-names}

%% `Elsevier LaTeX' style
\bibliographystyle{elsarticle-num}
%%%%%%%%%%%%%%%%%%%%%%%

\begin{document}

\begin{frontmatter}
		
		%\title{Deep density approximation with dimension reduction for high-dimensional Bayesian inverse problems}
            \title{Dimension-reduced KRnet maps for high-dimensional Bayesian inverse problems}
		%\tnotetext[mytitlenote]{Fully documented templates are available in the elsarticle package on \href{http://www.ctan.org/tex-archive/macros/latex/contrib/elsarticle}{CTAN}.}
		
		%%% Group authors per affiliation:
		%\author{Elsevier\fnref{myfootnote}}
		%\address{Radarweg 29, Amsterdam}
		%\fntext[myfootnote]{Since 1880.}
		
		%% or include affiliations in footnotes:
		\author[mymainaddress]{Yani Feng}
		\ead{fengyn@shanghaitech.edu.cn}
		\author[mysecondaryaddress]{Kejun Tang}
		\ead{tangkejun@icode.pku.edu.cn}
		
		\author[mythirdaddress]{Xiaoliang Wan}
		\ead{xlwan@lsu.edu}
		
		\author[mymainaddress]{Qifeng Liao\corref{mycorrespondingauthor}}
		\cortext[mycorrespondingauthor]{Corresponding author}
		\ead{liaoqf@shanghaitech.edu.cn}
		
		\address[mymainaddress]{School of Information Science and Technology, ShanghaiTech University, Shanghai 201210, China}
		\address[mysecondaryaddress]{Changsha Institute for Computing and Digital Economy, Peking University, Changsha 410205, China}
		\address[mythirdaddress]{Department of Mathematics and Center for Computation 
			and Technology, 
			Louisiana State University, Baton Rouge 70803, USA}
	
		
		\begin{abstract}
We present a dimension-reduced KRnet map approach (DR-KRnet) for high-dimensional Bayesian inverse problems, which is based on an explicit construction of a map that pushes forward the prior measure to the posterior measure in the latent space. 
% We present a new Bayesian inference approach for high-dimensional inverse problems, which is based on an explicit construction of a map that pushes forward the prior measure to the posterior measure in the latent space.  
Our approach consists of two main components: data-driven VAE prior and density approximation of the posterior of the latent variable. 
In reality, it may not be trivial to initialize a prior distribution that is consistent with available prior data; in other words, the complex prior information is often beyond simple hand-crafted priors. 
We employ variational autoencoder (VAE) to approximate the underlying distribution of the prior dataset, which is achieved through a  latent variable and a decoder. Using the decoder provided by the VAE prior, we reformulate the problem in a low-dimensional latent space. In particular, we seek an invertible transport map given by KRnet to approximate the posterior distribution of the latent variable. 
%In order to obtain a more expressive map, the normalizing flows are considered. However, the parameters of interest can be of high dimensionality in many practical problems, which renders the direct application of normalizing flows less effective. 
%Aside from the computational considerations, it is not trivial to initialize a prior distribution that is consistent with available prior datasets, i.e., the complex prior information cannot be easily captured by hand-crafted priors. 
%A data-driven prior called the VAE prior is built based on the datasets to address the challenge of choosing an appropriate prior.
%The VAE prior incorporates the dimension reduction given by VAE, which provides a general prior represented by deep neural networks instead of hand-crafted priors.
%Especially, the VAE-GAN priors can be viewed as a dimension reduction technique and generate more realistic samples %than VAE-priors. 
%Then in the low-dimensional latent space, an invertiable transport map, called KRnet map, is established to approximate the posterior distribution of the latent variable. 
Moreover, an efficient physics-constrained surrogate model without any labeled data is constructed to reduce the computational cost of solving both forward and adjoint problems involved in likelihood computation.
%at each optimization step. 
With numerical experiments, we demonstrate the accuracy and efficiency of DR-KRnet for high-dimensional Bayesian inverse problems.
		\end{abstract}
		
		\begin{keyword}
   dimension reduction; 
  KRnet;
			Bayesian inference;  VAE priors. 
		\end{keyword}
		
	\end{frontmatter}


\section{Introduction}\label{section_intro}
Bayesian inverse problems arise frequently in science and engineering, with applications ranging from subsurface and atmospheric transport to chemical kinetics. The primary task of such problems is to recover spatially varying unknown parameters from noisy and incomplete observations. Quantifying the uncertainty in the unknown parameters \cite{efendiev2006preconditioning,marzouk2007stochastic,wang2016gaussian,huan2013simulation,li2015adaptive,lieberman2010parameter,cui2016dimension} is then essential for predictive modeling and simulation-based decision-making.

The Bayesian statistical approach provides a foundation for inference from data and past knowledge. Indeed, the Bayesian setting casts the inverse solution as a posterior probability distribution over the unknown parameters. Though conceptually straightforward, characterizing the posterior, e.g., sample generation, marginalization, computation of moments, etc., is often computationally challenging especially when the dimensionality of the unknown parameters is large. 
% marginalizing; evaluating moments---faces the computational challenges.
%In most practical problems, the posterior distributions do not admit an analytical form and particularly when complex physical models enter the likelihood function---numerical approaches are needed. 
The most commonly used method for posterior simulation is Markov Chain Monte Carlo (MCMC) \cite{robert1999monte}. MCMC is an exact inference method and easy to implement. However, MCMC suffers from many limitations. An efficient MCMC algorithm depends on the design of effective proposal distributions, which becomes difficult when the target distribution contains strong correlations, particularly in high-dimensional cases.  Moreover, MCMC often requires a large number of iterations, where the forward model needs to be solved at each iteration. If the model is computationally intensive, e.g., a PDE with high-dimensional spatially-varying parameter, MCMC becomes prohibitively expensive.  
While considerable efforts have been
devoted to reducing the computational cost, e.g., \cite{lieberman2010parameter,li2014adaptive,cui2015data,jiang2017multiscale,liao2019adaptive,wang2018adaptive}, many challenges still remain in inverse problems.
Furthermore, the iteration process of MCMC is not associated with a clear convergence criterion to imply when the process has adequately captured the posterior.

As an alternative strategy to MCMC sampling, variational inference (VI) is widely used to approximate posterior distributions in Bayesian inference. Compared to MCMC, VI tends to be faster and easier to scale to large data. The idea of VI is to seek the best approximation of the posterior distribution within a family of 
parameterized density models. In \cite{goh2022solving, xia2023vi, tewari-deeppriors-2022}, coupling deep generative priors with VI to solve Bayesian inverse problems is studied.  
%first posit a family of densities and then to find a member of that family
%which is close to the posterior. Closeness is measured by Kullback–Leibler divergence. 
However, for high-dimensional distributions, it is still very challenging to
obtain an accurate posterior approximation due to the curse of dimensionality. For instance, the commonly used mean-field approach \cite{blei2017variational} assumes mutual independence between dimensions to achieve a tractable density model, which in general results in underestimated second-order moments.
%is usually under-parameterized and thus not sufficiently flexible to capture the full complexity of the true posterior.   
%where
%extra assumptions are often introduced for efficiency. For instance, a mean field approximation  
To remedy the issue, more capable density models are needed, where the mutual-independence assumption is relaxed. One strategy to do this is to seek 
%some researchers are interested in explicitly constructing 
a map that pushes the prior to the posterior, where the conditional dependence structure can be exploited and encoded into the map for more efficiency \cite{spantini2018inference}.  
%which has the more expressive power than classical VI. 
More specifically, the map transforms a random variable $z$, distributed according to the prior, into a random variable $y$, distributed according to the posterior. Such transformations can be viewed as transport maps between probability measures, %. The existence of a transport map is guaranteed, but such a map is seldom unique. 
whose existence is not unique. A certain structure needs to be introduced to determine a map. 
%Obtaining a particular map needs imposing additional particular structures. 
%Several methods are used to construct the map, e.g., parametric approximations of 
Some typical structures include the Knothe–Rosenblatt (K-R) rearrangement \cite{el2012bayesian}, neural ODE \cite{chen2018neural} and the composition of many simple maps used in flow-based deep generative models such as NICE \cite{Dinh_2014}, real NVP \cite{dinh2016density}, KRnet \cite{tang2020deep,adda_2022,wan2022vae}, to name a few.
%, including normalizing flow \cite{rezende2015variational}, real NVP \cite{dinh2016density}, KRnet %\cite{tang2020deep}. 
%The expressiveness of the map is essential for good performance, with under-expressive models leading to both %increased bias and under-estimation of posterior variance.
%In this work, we employ KRnet \cite{tang2020deep,adda_2022}, which adapts the structure of K-R rearrangement into a deep generative model and provides a better modeling capability than real NVP.
%richer, more faithful posterior approximations than real NVP. 

Two challenges need to be addressed for many practical Bayesian inverse problems. 
%when employing the map-based variational Bayes approach. 
%However, the application of Bayesian inference based on the construction of maps to practical inverse problems %leads to two significant challenges.  The first is that 
%First, Bayesian inference requires an explicit formula of prior, however, the prior knowledge may be only %available implicitly in terms of historical data or previously acquired solutions. Most problems use simple %distribution as the prior density. However, the true prior may be much more complex and then using the %handcrafted prior density may cause biased posterior. 
First, prior knowledge is often available in terms of historical data or previously acquired solutions, which should be consistent with the prior distribution of Bayesian inference. Unfortunately, the true prior may be much more complex than any commonly used explicit density models. The second challenge is the curse of dimensionality, which demands a trade-off between density approximation and sample generation for high-dimensional cases. 
%Especially, the parameter of interest suffers the curse of dimensionality in many practical problems, which renders the construction, representation and evaluation of KRnet intractable.
% Aside from the computational considerations, the prior does not have a closed form, but we only have access to the prior data. 
To deal with these two challenges, data-driven priors and dimension reduction can be incorporated.  For example, VAE-priors \cite{goh2022solving, xia2023vi, tewari-deeppriors-2022, xia2022bayesian,zhihang2023domain} and GAN priors \cite{patel2022solution} are proposed to learn the prior distribution from data, where a mapping from the low-dimensional latent space to the high-dimensional parameter space is established and MCMC or VI is subsequently implemented in terms of the latent random variable. To further increase efficiency, a surrogate model can be employed to avoid the expensive forward problem and the adjoint problem  \cite{tromp2005seismic,cao2003adjoint} that are needed for either sampling or variational inference approaches. One popular choice for surrogate modeling is the deep neural network which is able to provide a good approximation of high-dimensional parametric PDEs. For example, physics-informed neural networks (PINN) \cite{raissi2019physics} has attracted broad attention for solving PDEs, which embeds the laws of physics into the loss function. Zhu et al. \cite{zhu2018bayesian,zhu2019physics} propose a dense convolutional encoder-decoder network for PDEs with high-dimensional random inputs.

% In this work, we develop a variational Bayes approach in the latent space for both density approximation and sample generation. 
In this work, we propose a dimension-reduced KRnet map approach (DR-KRnet) for high-dimensional Bayesian inverse problems.
The main idea is to approximate the posterior distribution in terms of the latent variable that is learned from historical data, where VAE is used for dimension reduction and KRnet is used for density approximation. We first use abundant historical data to train a VAE prior, where we use the learned decoder to transfer the inference to the latent variable. We then minimize the Kullback-Leibler divergence between the posterior for the latent variable and the density model induced by KRnet. Using the decoder and the approximate posterior of the latent variable, we can compute the desired statistics efficiently because KRnet defines a transport map that provides exact samples with neglected costs. To further increase the efficiency, we also develop a convolutional encoder-decoder network as the surrogate model \cite{zhu2018bayesian,zhu2019physics}. Compared to sampling-based approaches, the main advantages of our strategy are twofold: First, we take advantages of two capable deep generative models, i.e., VAE and KRnet, to obtain an explicit density model that is sufficiently expressive for a high-dimensional posterior distribution. Second, KRnet, a normalizing flow model, is able to effectively deal with a moderately large number of dimensions and can be more robust than MCMC. 
%by approximating the posterior distribution of the latent random variable instead of sampling it.  
% To address these two challenges, data-driven priors are expressed by well-traineVariational Autoencoder (VAE) with prior data. 
%The main advantages are twofold: KRnet can be more effective and robust than MCMC when the number of dimensions is moderately large, and our approach results in an explicit density model for the posterior distribution of the latent random variable. 
%Combining the dimension reduction from VAE, numerical efficiency from the surrogate model, and the transport map from KRnet, we propose an explicit density approximation of the posterior, where the likelihood and exact samples can be efficiently obtained.  We first learn a VAE prior from the data. Using the obtained decoder, we use the KRnet-induced density model to approximate the posterior distribution of the latent variable. The approximate distribution is then used to compute the desired statistics such that MCMC is avoided. 

This paper is organized as follows. In section \ref{section_problem}, we describe the formulation of the Bayesian inverse problems that will be considered in this work. In section \ref{section_method}, our dimension-reduced KRnet map approach (DR-KRnet) for high-dimensional Bayesian inverse problems is presented, where we provide a scheme for building the neural network structure of VAE priors, introduce the KRnet to construct the map between the prior and posterior and build the physics-constrained surrogate model. In section \ref{section_experiments}, with two numerical experiments we demonstrate that our DR-KRnet can infer high-dimensional parameters efficiently. The paper is concluded in section \ref{section_conclude}.
\section{Bayesian inverse problems}\label{section_problem}
To begin with, details of the forward  model considered in this paper are addressed as follows. Let $\mathcal{S}$ denote a spatial domain that is bounded, connected and with a polygonal boundary $\partial \mathcal{S}$, and $s\in \mathcal{S}$ is  a spatial variable. The physics of the problem considered is governed by a PDE over the spatial domain $\mathcal{S}$: find $u(s,y(s))$ such that
\begin{equation}\label{physical_problem}
	\begin{aligned}
		&\mathcal{L}\Big(s,u(s,y(s));y(s)\Big)=h(s),\quad \forall s\in \mathcal{S},\\
		&\pB\Big(s,u(s,y(s));y(s)\Big)=g(s),\quad \forall s\in \partial\mathcal{S},
	\end{aligned}
\end{equation}
where $\mathcal{L}$ is a partial differential operator and $\pB$ is a boundary operator, both of which can depend on the unknown spatial-varying parameter $y(s)$, $h(s)$ is the source function, and  $g(s)$ specifies boundary conditions.

\subsection{Bayesian framework}
We consider the task of inferring the parameter $y\in \mathbb{R}^n$ from observations $\mathcal{D}_{obs}\in \mathbb{R}^m$ under the assumption that there exists a forward model $\mathcal{F}$ determined by \eqref{physical_problem} that maps the unknown parameter $y$ to the observations $\mathcal{D}_{obs}$:
\begin{align}
    \mathcal{D}_{obs}=\mathcal{F}(y)+\epsilon,
\end{align}
where $\epsilon\in \mathbb{R}^m$ is the measurement noise. Let $\pi_{\epsilon}(\epsilon)$ be the distribution of $\epsilon$, and one can obtain the distribution of $\mathcal{D}_{obs}$ conditioned on $y$:
\begin{align}
    \pi(\mathcal{D}_{obs}|y)=\pi_{\epsilon}(\mathcal{D}_{obs}-\mathcal{F}(y)).
\end{align}

Since often $m\ll n$, inverse problems are in general ill-posed, i.e.,\ one may not be able to uniquely recover the parameter $y$ given the noisy observations $\mathcal{D}_{obs}$. In the Bayesian setting, the parameters to be inferred are treated as random variables. Given the observations $\mathcal{D}_{obs}$, one assigns a prior distribution $\pi(y)$ encoding the prior information on the parameter of interest, and the posterior $\pi(y|\mathcal{D}_{obs})$ can then be calculated via the Bayes' rule:
\begin{equation}\label{yposterior}
\pi(y|\mathcal{D}_{obs})=\frac{\pi(\mathcal{D}_{obs}|y)\pi(y)}{C} \,\,\,
	\propto \,\,\,\underbrace{\pi(\mathcal{D}_{obs}|y)\pi(y)}_{\hat{\pi}(y)},
\end{equation}
where $\pi(\mathcal{D}_{obs}|y)$ is the likelihood function, and the evidence or marginal likelihood $C=\int \pi(\mathcal{D}_{obs}|y)\pi(y) dy$ is a normalization constant. 
% and $C$ is the evidence or marginal likelihood $C=\int \pi(\mathcal{D}_{obs}|y)\pi(y) d_y$, which is often intractable because of computing a high-dimensional integral.

Since the map $\mathcal{F}$ from $y$ to $\mathcal{D}_{obs}$ is typically nonlinear and the evidence is often intractable, especially for high-dimensional problems, the posterior, in general, cannot be obtained in a closed form. Therefore, the Bayesian inference needs to characterize the unnormalized posterior, which is usually achieved by variational inference (VI) or sampling approaches such as MCMC. However, extra assumptions are often introduced in VI, e.g., the family of parameterized density models for approximating the posterior distribution is a diagonal covariance Gaussian distribution, and sampling approaches such as MCMC become less efficient for sufficiently large $n$ and $m$. To improve efficiency, dimension reduction can be introduced such that VI \cite{goh2022solving,xia2023vi} or MCMC \cite{xia2022bayesian,patel2022solution} can be implemented in a low-dimensional latent space, where the dimension reduction is achieved, either explicitly or implicitly, by deep generative models. In this work we replace MCMC or VI with a normalizing flow to develop a dimension-reduced KRnet map approach (DR-KRnet) that is completely based on deep generative modeling.
%instead of a coupling between density approximation (VAE prior) and exact sampling (MCMC) or VI.

\subsection{Inference with a map}
The core idea of our approach is to find a map that pushes forward the prior to the posterior in the latent space. Before taking into account dimension reduction and surrogate modeling, we look at how normalizing flows approximate the posterior of $y$. 
Let $z \in \xs{R}^n$ be a random variable that has a known distribution, e.g., the standard Gaussian. 
%and $y$ be two random variables distributed according to the prior and the posterior respectively. 
We seek an invertible map $f: \mathbb{R}^n\to\mathbb{R}^n$
\begin{align}
	z=f(y),
\end{align}
which depends on the observations $\mathcal{D}_{obs}$, the forward model $\mathcal{F}$, and the distribution of the measurement noise $\epsilon$. Assuming the map $f$ exists, we have the posterior by the change of variables 
\begin{align}
	p_y(y)=p_{z}(f(y)) \left |\det\nabla_{y} f \right|.
\end{align}
In practice, we will learn the map $f(\cdot)$ by minimizing the Kullback-Leibler divergence between $p_y(y)$ and the posterior:
%To find the optimal map $f$, we minimize the following Kullback-Leibler divergence,
\begin{align}
	D_{KL}\left(p_y||\pi\left(y|\mathcal{D}_{obs}\right)\right)&=\int p_{y}\log \frac{p_{y}}{\pi(y|\mathcal{D}_{obs})} dy \nonumber\\
	&=\int p_{y}\log \frac{p_{y}}{\hat{\pi}(y)} dy+\log C \nonumber\\
	&=\int p_{z}\log \frac{p_{y}(f^{-1}(z))}{\hat{\pi}(f^{-1}(z))} dz +\log C\nonumber\\
	&\approx\frac{1}{I}\sum_{i=1}^I\log p_{y}\left(f^{-1}\left(z^{(i)}\right)\right)-\frac{1}{I}\sum_{i=1}^I\log \hat{\pi}\left(f^{-1}\left(z^{(i)}\right)\right)+\log C,\quad z^{(i)} \sim p_{z}.\label{highkl}
\end{align}
%One method for guaranteeing a unique map is to enforce a "triangular" structure for the map.
It is noted that normalizing flows provide an explicit density model and an efficient way to generate exact samples of $y$ through the invertible map $y=f^{-1}(z)$. Normalizing flows can be much more expressive than classical density models, e.g., the Gaussian model subject to a diagonal covariance matrix used in the mean-field approach. Yet the construction, representation, and evaluation of these generative models grow challenging in high-dimensional cases.
%  In particular, with the Knothe-Rosenblatt rearrangement, our newly proposed flow-based generative model, called KRnet map, provides the richer, more faithful posterior approximations than real NVP. However, the computational cost of Bayesian inference with KRnet map has a square growth with the dimensionality. 
%  Especially, the parameter of interest can be of high dimensionality in many practical problems, which renders these models infeasible. 
Moreover, the prior $\pi(y)$ is often provided through historical data $\{y^{(i)}\}_{i=1}^N$, which may be significantly different from the commonly used prior such as the Gaussian distribution and needs to be modeled explicitly. Furthermore, 
%it can be seen from \eqref{highkl} that the prior in $\hat{\pi}(f^{-1}\left(z^{(i)}\right))$ should be a closed form, and 
each sample $z$ requires an evaluation of the computationally expensive forward function $\mathcal{F}$. 
%In practice, the prior information $\pi(y)$ doesn't have a closed form but one often has access to a historical dataset $\{y^{(i)}\}_{i=1}^N$, where $y^{(i)}$ can be regarded as the samples from the prior $\pi(y)$.
To address these problems, we use a dimension-reduced VAE prior to model $\pi(y)$ through  historical data and then, in the low-dimensional latent space, apply KRnet to seek an invertible map $f$ with respect to a surrogate model for the forward problem.
\section{Dimension-reduced KRnet maps}\label{section_method}
In this section, we present a dimension-reduced KRnet map approach (DR-KRnet) in detail, which consists of three parts (choices of prior, likelihood computation, and posterior approximation).
%since we employ the Bayesian framework. 
First, the VAE prior is introduced to capture the features of $\{y^{(i)}\}_{i=1}^N$. Next, the KRnet map is adopted for pushing forward the prior to the posterior in the low-dimensional latent space. In addition, physics-constrained surrogate modeling is used to compute the likelihood function efficiently.
% \subsection{VAE-GAN priors for dimension reduction}\label{vae_gan_section}
\subsection{VAE priors for dimension reduction}\label{vae_gan_section}
As a dimension reduction method, variational autoencoder (VAE) builds the relationship between the latent space and the original high-dimensional parameter. We briefly recall the VAE. Assume that there exists a latent random variable $x\in \mathbb{R}^d$ ($d\ll n$) with a marginal distribution $p_{x,\theta}$, where $\theta$ includes the model parameters. The joint distribution $p_{x,y,\theta}$ of $x$ and $y$ is then described by the conditional distribution $p_{y|x,\theta}$, i.e., $p_{x,y,\theta}=p_{y|x,\theta}p_{x,\theta}$.
% The main task of VAE is to generate $y$ from $p_y$.
 According to Bayes' rule,
\begin{align}
    p_{y,\theta}=\frac{p_{x,y,\theta}}{p_{x|y,\theta}}=\frac{p_{y|x,\theta}p_{x,\theta}}{p_{x|y,\theta}}.
\end{align}
The posterior distribution $p_{x|y,\theta}$ is in general
intractable, and then an approximation model $q_{x|y,\phi}$ is needed, where $\phi$ includes the model parameters. The optimal parameters $\theta$ and $\phi$ are determined by minimizing the KL divergence
\begin{align}
    D_{KL}(q_{x|y,\phi}||p_{x|y,\theta})=D_{KL}(q_{x|y,\phi}||p_{x,\theta})-\mathbb{E}_{q_{x|y,\phi}}[\log p_{y|x,\theta} ]+\log p_{y,\theta}\geq 0.
\end{align}
The minimization of $D_{KL}(q_{x|y,\phi}||p_{x|y,\theta})$ is equivalent to the maximization of
the variational lower bound of $\log p_{y,\theta}$, which is defined as 
\begin{align}\label{eq_vae_loss}
    \mathcal{L}_{\theta,\phi}(y)=\mathbb{E}_{q_{x|y,\phi}}[\log p_{y|x,\theta} ]-D_{KL}(q_{x|y,\phi}||p_{x,\theta}).
\end{align}
% A VAE consists of two networks that encode a data sample $y\in \mathbb{R}^m$
% to a latent representation $x\in \mathbb{R}^n$ and decode the latent representation back to data space,
% respectively:
% \begin{align}
%     x\sim \text{Enc}(y)=q_{x|y,\phi}, \quad, \Tilde{y}\sim\text{Dec}(x)=p_{y|x,\theta},
% \end{align}
% where $\phi$ and $\theta$ denote model parameters for the encoder and the decoder respectively.
% The VAE regularizes the encoder by imposing a prior over the latent distribution $p_{x,\theta}$.  
In the canonical VAE, we specify the PDF models respectively for $p_{y|x,\theta},\,q_{x|y,\phi}$ and $p_{x,\theta}$ as follows:
\begin{equation}
	\begin{aligned}
		p_{y|x,\theta}&=\mathcal{N}\left(\mu_{de,\theta}\left(x\right), \text{diag}\left(\sigma_{de,\theta}^{\odot 2}\left(x\right)\right)\right),\\
		q_{x|y,\phi}&=\mathcal{N}\left(\mu_{en,\phi}\left(y\right), \text{diag}\left(\sigma_{en,\phi}^{\odot 2}\left(y\right)\right)\right), \label{caen}\\
		p_{x,\theta}&=\mathcal{N}(0,\mathbf{I}),
	\end{aligned}
\end{equation}
where ${*}^{\odot2}$ means the component-wise square operation. The tuples $(\mu_{en,\theta}(y), \sigma_{en,\theta}(y))$ and $\left(\mu_{de,\theta}(x), \sigma_{de,\theta}(x)\right)$ are modeled via neural networks, i.e.,
\begin{align}
&\left(\mu_{en,\theta}(y), \sigma_{en,\theta}(y)\right)=\text{NN}_{en}(y;\theta),\label{vae1}\\
&x=\mu_{en,\phi}(y)+\sigma_{en,\phi}(y)\odot \varepsilon,\quad \varepsilon \sim \mathcal{N}(0,\mathbf{I}),\label{vae2}\\
&\left(\mu_{de,\theta}(x), \sigma_{de,\theta}(x)\right)=\text{NN}_{de}(x;\phi),\label{vae3}\\
&\hat{y}=\mu_{de,\theta}(x),
\end{align}
where $\text{NN}_{de}$ and $\text{NN}_{en}$ characterize the encoder and decoder neural networks to describe the relation between a data sample $y\in\mathbb{R}^n$ and a latent representation $x\in\mathbb{R}^d$, and $\hat{y}$ is the reconstruction of  $y$. 
% Figure \ref{vae_gan} illustrates the structure of VAE. 

Given a prior dataset $Y:=\{y^{(i)}\}_{i=1}^N$, the expectation of the variational lower bound  \eqref{eq_vae_loss} can be approximated via 
the Monte Carlo estimation
\begin{align}
    \mathbb{E}_{p_{y,\theta}}\left[\mathcal{L}_{\theta,\phi}(y)\right]&\approx \frac{1}{N}\sum_{i=1}^N\mathcal{L}_{\theta,\phi}\left(y^{(i)}\right)\nonumber\\
    &\approx\underbrace{\frac{1}{N}\sum_{i=1}^N\left[\log p_{y^{(i)}|x^{(i)},\theta}-(\log q_{x^{(i)}|y^{(i)},\phi}-\log p_{x^{(i)},\theta})\right]}_{\hat{\mathcal{L}}_{\theta,\phi}(Y)}, \label{vae_loss_discri}
\end{align}
where $x^{(i)}$ can be generated by substituting $y^{(i)}$ into  \eqref{vae2}.
% which are modeled by neural networks, characterize the encoder and decoder to describe the relation between a data sample $y\in\mathbb{R}^n$ and a latent representation $x\in\mathbb{R}^d$:
%In all, a VAE consists of two networks that encode a data sample $y\in \mathbb{R}^n$
%to a latent representation $x\in \mathbb{R}^d$ and decode the latent representation back to data space,
%respectively:
%\begin{equation}
% \begin{align}
% 	x&=\text{Enc}(y)=\mu_{en,\phi}(y)+\sigma_{en,\phi}(y)\odot \epsilon_1,\label{vae_first}\\
% 	\hat{y}&=\text{Dec}(x)=\mu_{de,\phi}(x)+\sigma_{de,\phi}(x)\odot \epsilon_2,\label{vae_second}
% \end{align}
% %\end{equation}
% where $\epsilon_1\sim \mathcal{N}(0,\mathbf{I})$, $\epsilon_2\sim \mathcal{N}(0,\mathbf{I})$, and $\hat{y}$ is the reconstruction of  $y$.
% The VAE loss is minus the sum of the expected log likelihood (the reconstruction error) and a prior regularization term:
% \begin{align}
%     \mathcal{L}_{VAE}&=-\mathbb{E}_{q_{x|y,\phi}}\Big[\log\frac{p_{y|x,\theta}p_{x,\theta}}{q_{x|y,\phi}}\Big] \nonumber\\
%     &=-\mathbb{E}_{q_{x|y,\phi}}[\log p_{y|x,\theta} ]+D_{KL}(q_{x|y,\phi}||p_{x,\theta})
% \end{align}
% The main drawback of VAE model is that it tends to produce unrealistic, blurry samples, which seriously influences the accuracy of estimating posterior mean in inverse problems. As we all know, an appealing property of GAN is that its discriminator contributes to generating realistic data that resembles the prior datasets $\{y^{(i)}\}_{i=1}^N$. 
% A GAN consists of two neural networks: the generator network $y=\text{Gen}(x)$ while the discriminator network assigns probability $p=\text{Dis}(y)\in [0,1]$ that $y$ is an actual training sample and probability $1-p$ that $y$ is generated by the generator network. Appplying the advantages of VAE and GAN, we propose a more accuracy method---VAE-GAN prior for characterizing the prior datasets, which combines VAE and GAN by replacing the generator of GAN by VAE. The network structure of VAE-GAN priors is shown in Fig.\ \ref{vae_gan}. The parameters of VAE is updated by minimizing the following loss
% \begin{align}\label{gen_loss}
% 	\mathcal{L}_{\text{gen}}=\mathcal{L}_{VAE}+\beta \log (\text{Dis}(\text{Dec}(x))),
% \end{align}
% where $\mathcal{L}_{VAE}=-\mathcal{L}_{\theta,\phi}(y)$, $\beta$ is a tunable parameter and minimizing the second term aims to generate realistic samples. The parameters of discriminator is updated by minimizing the following loss
% \begin{align}
% \mathcal{L}_{\text{GAN}}=\log(\text{Dis}(y))+\log (1-\text{Dis}(\text{Dec}(x))).
% \end{align}
% The training process of VAE-GAN priors is illustrated in Algorithm \ref{alg_vae_gan}.
%where the first is to enforce that discrimininator views the actual training sample as 
We pre-train the VAE priors by Algorithm \ref{alg_vae_gan}, of which the output  is the pre-trained decoder $p_{y|x,\theta^*}$. Here $\theta^*$ consists of the optimal parameters of the decoder.
The inference over $y$ in  \eqref{yposterior} is replaced by infering the latent variable $x$ from the observations, formulated as
\begin{align}
	\pi(x|\mathcal{D}_{obs})\propto &{\pi(\mathcal{D}_{obs}|x)\pi(x)}, \nonumber\\
	&=\left(\int \pi(\mathcal{D}_{obs}|y,x) \pi(y|x) dy\right) \pi(x)\nonumber\\
	&= \underbrace{\left(\int \pi(\mathcal{D}_{obs}|y,x) p_{y|x,\theta^*} dy\right) p_{x,\theta^*}}_{\hat{\pi}(x)}, \label{xposterior}
\end{align}
where $\pi(\mathcal{D}_{obs}|y,x) $ is the likelihood function, $p_{y|x,\theta^*}$ is the pre-trained decoder, and $p_{x,\theta^*}$ is a simple prior distribution of VAE, e.g., the standard Gaussian. % priors.
% \begin{figure}
% 	\centering
% 	\includegraphics[width=0.8\textwidth]{image/351676253784_.pic.jpg}
% 	\caption{The illustration of VAE.}
% 	\label{vae_gan}
% \end{figure}
\begin{algorithm}[H]
	\caption{Training the VAE priors}
	\label{alg_vae_gan}
	\begin{algorithmic}[1]
		\Require The prior dataset $Y:=\{y^{(i)}\}_{i=1}^N$, maximum epoch number $E$, batch size $n_{batch}$, learning rate $\eta$.
           \State Divide $Y$ into $N_b$ mini-batches $\{Y_j\}_{j=1}^{N_b}$ where $N_b=\frac{N}{n_{batch}}$.
           \State Initialize $\theta$ and $\phi$ for the encoder and decoder networks.		
           \For {$i = 1:E$}
		\For {$j=1:N_b$}
		%\State Sample $n_{batch}$ datapoints $Y$ from $\{y^{(i)}\}_{i=1}^N$
            \State Construct the noise set $S_j=\{\varepsilon^k\sim \mathcal{N}(0,\mathbf{I}),k=1,2,\dots,n_{batch}\}$.
            \State Apply $Y_j$ and $S_j$ to compute  \eqref{vae1}--\eqref{vae3}.
	    % \State Obtain $\mu_{en,\phi}$ and $\sigma_{en,\phi}$ through the encoder
		\State Compute  $-\hat{\mathcal{L}}_{\theta,\phi}(Y_j)$ in  \eqref{vae_loss_discri} and its gradients $-\nabla_{\theta}\hat{\mathcal{L}}_{\theta,\phi}(Y_j),\,-\nabla_{\phi}\hat{\mathcal{L}}_{\theta,\phi}(Y_j)$.
		% \State $\mathcal{L}_{\text{GAN}}\leftarrow \log(\text{Dis}(Y))+\log (1-\text{Dis}(\Tilde{Y})).$
		\State Update the parameters $(\theta,\phi)$ using gradient-based optimization algorithm (e.g., Adam optimizer \cite{kingma2014adam} with learning rate $\eta$).
		\EndFor
		\EndFor
         \State Let $\theta^*=\theta$, where $\theta$ includes the parameters of the decoder networks at the last epoch.
		\Ensure The probabilistic decoder $p_{y|x,\theta^*}$.
%		\Ensure probabilistic encoder $q_{x|y,\phi^*}$, probabilistic decoder $p_{y|x,\theta^*}$.
	\end{algorithmic}
\end{algorithm}
% \begin{rmk}[Some thoughts about VAE-GAN]
% {\color{red}The loss of VAE has two terms: for any $y$,
% \[
% \mathcal{L}_{VAE}=D_{KL}(q_{x|y}\|p_x)+\mathbb{E}_{q_{x|y}}[\log p_{y|x}].
% \]
% The second term is a reconstruction error: If we let $p_{y|x}\sim \mathcal{N}(f(x),c\mathbf{I})$, the second term is 
% \[
% \frac{1}{2c}\|y-\text{Dec}(\text{Enc}(y))\|^2,
% \]
% i.e., the reconstruction error is measured in terms of the $L_2$ norm. The first term of $\mathcal{L}_{VAE}$ force the posterior $q_{x|y}$ close to $p_x$, which can be regarded as a regularization term. In equation (15), if we let 
% \[
% \mathcal{L}_{GAN}=\mathbb{E}[\log(\text{Dis}(y))]+\mathbb{E}[\log(1-\text{Dis}(\text{Dec}(\text{Enc}(y))))]
% \]
% it can be regarded as an extra reconstruction error by minimizing the difference between the distributions of $y$ and $\text{Dec}(\text{Enc}(y))$ using the Jensen-Shannon divergence. This is the current choice. 

% Since we only need the decoder, we may emphasize the prior distribution by simply replacing the generator of GAN with the decoder such that
% \[
% \mathcal{L}_{GAN}=\mathbb{E}[\log(\text{Dis}(y))]+\mathbb{E}[\log(1-\text{Dis}(\text{Dec}(x)))],
% \]
% where $x\sim p_x(x)$. This will force the joint distribution $p_{y|x}p_x$ to be more consistent with the data distribution. The loss of VAE plays a role of regularization. 

% Another idea to make the decoder sharper is to modify the loss of VAE as
% \[
% \mathcal{L}_{VAE}=D_{KL}(q_{x|y}\|p_x)+\mathbb{E}_{q_{x|y}}[\log p_{y|x}]+\lambda\|y-\mu_{de}(\text{Enc}(y))\|^2,
% \]
% where the extra penalty term ignores the noise in the decoder. Minimizing the noise in the decoder corresponds to maximizing the mutual information between $y$ and $x$. 

% I am not sure which one works better since the modeling capability of VAE is limited. 
% }
% \end{rmk}

\subsection{KRnet map}
In  \eqref{xposterior}, let $\pi(x|\mathcal{D}_{obs})=C^{-1}\hat{\pi}(x)$, $x\in \mathbb{R}^d$, $d\ll n$. In the low-dimensional latent space of the pre-trained VAE prior, we intend to approximate the posterior $\pi(x|\mathcal{D}_{obs})$ by constructing a map that pushes forward the prior to the posterior. In other words, we seek a transport map $\mathcal{T}$: $z \mapsto x$ such that $\mathcal{T}_{\#} \mu_z = \mu_x$, where $d\mu_z=p_{z,\theta^*}dz$ and $d\mu_x=\pi(x|\mathcal{D}_{obs})dx$ are the probability measures of $z$ and $x$ respectively, and $\mathcal{T}_{\#} \mu_{z}$ is the push-forward of $\mu_{z}$ satisfying $\mu_{x}(B) = \mu_{z}(\mathcal{T}^{-1}(B))$ for every Borel set $B$. The Knothe-Rosenblatt rearrangement tells us that the transport map $\mathcal{T}$ may have a lower-triangular structure 
\begin{equation}
    {z} = \mathcal{T}^{-1}({x}) = \left[ 
    \begin{array}{l}
    \mathcal{T}_1(x_1) \\
    \mathcal{T}_2(x_1, x_2) \\
    \vdots \\
    \mathcal{T}_{d}(x_1, \ldots, x_d)
    \end{array}
    \right].
\end{equation} 
This mapping can be regarded as a limit of sequence of optimal transport maps when the quadratic cost degenerates \cite{carlier2010knothe}. 

%With the Knothe-Rosenblatt rearrangement, our proposed flow-based generative model, called KRnet, provides an expressive model for density  approximation. KRnet provides an invertible mapping $x=f^{-1}(z)$, which can push the prior $z\sim p_{z,\theta^*}$ towards the posterior $x\sim \pi(x|\mathcal{D}_{obs})$. 


 %Let $\mu_Z$ and $\mu_X$ be the probability measures of two random variables $X,Z\in\xs{R}^d$ respectively. A mapping $\mathcal{T}$: $Z \mapsto X$ is called a transport map such that $\mathcal{T}_{\#} \mu_Z = \mu_X$, where $\mathcal{T}_{\#} \mu_{Z}$ is the push-forward of $\mu_{Z}$ such that $\mu_{X}(B) = \mu_{Z}(\mathcal{T}^{-1}(B))$ for every Borel set $B$.

%Noticing that the invertible mapping $f(x)$ also defines a transport map, we then incorporate the triangular structure of the Knothe-Rosenblatt rearrangement into the definition of $f(x)$ which results in KRnet as a generalization of real NVP \cite{dinh2016density}. 
The basic idea of KRnet is to define the structure of a normalizing flow $f({x})$ in terms of the Knothe-Rosenblatt rearrangement which results in KRnet as a generalization of real NVP \cite{dinh2016density}. Let ${x} = \left[{x}^{(1)}, \ldots, {x}^{(K)}\right]^\mathsf{T}$ be a partition of ${x}$, where ${x}^{(i)} =\left[x_{1}^{(i)}, \ldots, x_{m}^{(i)}\right]^\mathsf{T}$ with $1 \leq K \leq d, 1 \leq m \leq d$, and $\sum_{i=1}^K \mathrm{dim}\left({x}^{(i)}\right) = d$. Our KRnet takes an overall form
\begin{equation} \label{eqn_KR}
  {z} = f({x}) = \left[ 
    \begin{array}{l}
    f_1\left({x}^{(1)}\right) \\
    f_2\left({x}^{(1)}, {x}^{(2)}\right) \\
    \vdots \\
    f_{K}\left({x}^{(1)}, \ldots, {x}^{(K)}\right)
    \end{array}
    \right].
\end{equation}
Each $f_{i}$, $i=2,\ldots,K$, is constructed with real NVP by stacking a sequence of simple bijections. KRnet provides a more expressive density model than real NVP for the same model size. More details about KRnet can be found in \cite{tang2020deep,adda_2022}.

%KRnet consists of one  outer loop and $K-1$ inner loops. The outer loop has $K-1$ stages, corresponding to the $K-1$ mappings $f_{i}$ in equation \eqref{eqn_KR} with $i=2,\ldots,K$, and for each stage, an inner loop of $L$ affine coupling layers is defined. More specifically, we have
%\begin{equation}\label{krnet_expr}
%z = f(x) = L_{N} \circ f_{[K-1]}^{\textsf{outer}} \circ \cdots \circ f_{[1]}^{\textsf{outer}} (x),
%\end{equation}
%where $f_{[i]}^{\textsf{outer}}$ is defined as
%\begin{equation}
%f_{[k]}^{\textsf{outer}} = L_S \circ f_{[k, L]}^{\textsf{inner}} \circ \cdots \circ %f_{[k,1]}^{\textsf{inner}} \circ L_R.
%\end{equation}
%Here $f_{[k,i]}^{\textsf{inner}}$ indicates a combination of one affine coupling layer and one scale and bias layer, and $L_N$, $L_S$ and $L_R$ indicate the nonlinear layer, the squeezing layer and the rotation layer, respectively.
%More details about KRnet can be found in \cite{tang2020deep,adda_2022}.

%The basic idea of KRnet is to define the structure of $f({x})$ in terms of the Knothe-Rosenblatt rearrangement. Let $\mu_Z$ and $\mu_X$ be the probability measures of two random variables $X,Z\in\xs{R}^d$ respectively. A mapping $\mathcal{T}$: $Z \mapsto X$ is called a transport map such that $\mathcal{T}_{\#} \mu_Z = \mu_X$, where $\mathcal{T}_{\#} \mu_{Z}$ is the push-forward of $\mu_{Z}$ such that $\mu_{X}(B) = \mu_{Z}(\mathcal{T}^{-1}(B))$ for every Borel set $B$. The Knothe-Rosenblatt rearrangement tells us that the transport map $\mathcal{T}$ may have a lower-triangular structure 
\begin{equation}
    {z} = \mathcal{T}^{-1}({x}) = \left[ 
    \begin{array}{l}
    \mathcal{T}_1(x_1) \\
    \mathcal{T}_2(x_1, x_2) \\
    \vdots \\
    \mathcal{T}_{d}(x_1, \ldots, x_d)
    \end{array}
    \right].
\end{equation}
This mapping can be regarded as a limit of sequence of optimal transport maps when the quadratic cost degenerates \cite{carlier2010knothe}. Noticing that the invertible mapping $f(x)$ also defines a transport map, we then incorporate the triangular structure of the Knothe-Rosenblatt rearrangement into the definition of $f(x)$ which results in KRnet as a generalization of real NVP \cite{dinh2016density}. Let ${x} = [{x}^{(1)}, \ldots, {x}^{(K)}]^\mathsf{T}$ be a partition of ${x}$, where ${x}^{(i)} = [x_{1}^{(i)}, \ldots, x_{m}^{(i)}]^\mathsf{T}$ with $1 \leq K \leq d, 1 \leq m \leq d$, and $\sum_{i=1}^K \mathrm{dim}({x}^{(i)}) = d$. Our KRnet takes an overall form
\begin{equation} \label{eqn_KR}
  {z} = f({x}) = \left[ 
    \begin{array}{l}
    f_1({x}^{(1)}) \\
    f_2({x}^{(1)}, {x}^{(2)}) \\
    \vdots \\
    f_{K}({x}^{(1)}, \ldots, {x}^{(K)})
    \end{array}
    \right].
\end{equation}
Here each $f_{i}$ is an invertible mapping  for $ i = 2, \ldots, K$. Each $f_{i}$ is constructed by stacking a sequence of simple bijections, each of which is a shallow neural network, and thus the overall mapping $f$ is a deep neural network. KRnet consists of one  outer loop and $K-1$ inner loops. The outer loop has $K-1$ stages, corresponding to the $K-1$ mappings $f_{i}$ in equation \eqref{eqn_KR} with $i=2,\ldots,K$, and for each stage, an inner loop of $L$ affine coupling layers is defined. More specifically, we have
\begin{equation}\label{krnet_expr}
z = f(x) = L_{N} \circ f_{[K-1]}^{\textsf{outer}} \circ \cdots \circ f_{[1]}^{\textsf{outer}} (x),
\end{equation}
where $f_{[i]}^{\textsf{outer}}$ is defined as
\begin{equation}
f_{[k]}^{\textsf{outer}} = L_S \circ f_{[k, L]}^{\textsf{inner}} \circ \cdots \circ f_{[k,1]}^{\textsf{inner}} \circ L_R.
\end{equation}
Here $f_{[k,i]}^{\textsf{inner}}$ indicates a combination of one affine coupling layer and one scale and bias layer, and $L_N$, $L_S$ and $L_R$ indicate the nonlinear layer, the squeezing layer and the rotation layer, respectively.
More details about KRnet can be found in \cite{tang2020deep,adda_2022}.


Let $q_{x,\alpha}$ be the PDF model induced by a KRnet with model parameters $\alpha$, and then  \eqref{eqn_KR} is reformulated into
\begin{equation}\label{eqn_KR_alpha}
    {z} = f_{\alpha}({x}).
\end{equation}
To approximate $\pi(x|\mathcal{D}_{obs})$ in  \eqref{xposterior}, we minimize the KL divergence between $q_{x,\alpha}$ and $\pi(x|\mathcal{D}_{obs})$
\begin{align*}
	D_{KL}\left(q_{x,\alpha}||\pi\left(x|\mathcal{D}_{obs}\right)\right)=\int q_{x,\alpha}\log \frac{q_{x,\alpha}}{\pi(x|\mathcal{D}_{obs})} dx
	=\int q_{x,\alpha}\log \frac{q_{x,\alpha}}{\hat{\pi}(x)} dx+\log C,
\end{align*}
which is equivalent to minimize the following functional
\begin{align}
	\int q_{x,\alpha}\log \frac{q_{x,\alpha}}{\hat{\pi}(x)} dx 
	&=\int p_{z,\theta^*}\log \frac{q_{x,\alpha}\left(f_{\alpha}^{-1}(z)\right)}{\hat{\pi}\left(f_{\alpha}^{-1}(z)\right)} dz \nonumber\\
	&\approx\frac{1}{I}\sum_{i=1}^I\log q_{x,\alpha}\left(f_{\alpha}^{-1}\left(z^{(i)}\right)\right)-\frac{1}{I}\sum_{i=1}^I\log \hat{\pi}\left(f_{\alpha}^{-1}\left(z^{(i)}\right)\right),\quad z^{(i)} \sim p_{z,\theta^*}. \label{firstloss}
\end{align}
Let $x^{(i)}=f_{\alpha}^{-1}\left(z^{(i)}\right)$. The second term of the right hand of  \eqref{firstloss} is obtained as
\begin{align*}
	-\frac{1}{I}\sum_{i=1}^I\log \hat{\pi}\left(x^{(i)}\right)
	&=-\frac{1}{I}\sum_{i=1}^I\log \left(\int \pi\left(\mathcal{D}_{obs}|y,x^{(i)}\right) p_{y|x^{(i)},\theta^*} dy\right) -\frac{1}{I}\sum_{i=1}^I\log p_{x^{(i)},\theta^*}\\
	&\leq -\frac{1}{I}\sum_{i=1}^I\int p_{y|x^{(i)},\theta^*} \log \pi\left(\mathcal{D}_{obs}|y,x^{(i)}\right) dy-\frac{1}{I}\sum_{i=1}^I\log p_{x^{(i)},\theta^*}\\
	&\approx -\frac{1}{I}\frac{1}{J}\sum_{i=1}^I\sum_{j=1}^J \log \pi\left(\mathcal{D}_{obs}|y^{(i,j)},x^{(i)}\right)-\frac{1}{I}\sum_{i=1}^I\log p_{x^{(i)},\theta^*},\quad y^{(i,j)}\sim p_{y|x^{(i)},\theta^*}, 
\end{align*}
where the Jensen's inequality is applied and $\pi\left(\mathcal{D}_{obs}|y^{(i,j)},x^{(i)}\right)$ is the likelihood function. Since the first term on the right-hand side corresponds to the expectation of $\log\pi(\mathcal{D}_{obs}|y,x)$ with respect to the joint distribution given by $p_{y|x,\theta^*}p_{x,\theta^*}$, we may simply let $J=1$.  
%For simplify, we can just let $J=1$ by noting that
%\begin{align}
%	&-\frac{1}{I}\sum_{i=1}^I\log \hat{\pi}\left(x^{(i)}\right) \nonumber\\
%	&\leq -\frac{1}{I}\sum_{i=1}^I\log \pi(\mathcal{D}_{obs}|y^{(i)},x^{(i)})-\frac{1}{I}\sum_{i=1}^I\log p_{x^{(i)},\theta^*},\quad y^{(i)}\sim p_{y|x^{(i)},\theta^*}. \label{secondloss}
%\end{align}
%Substituting \eqref{secondloss} into \eqref{firstloss}, we need to minimize the following objective function
We then reach our objective function for minimization
\begin{align}
	\mathcal{L}_{KRnet}&=\frac{1}{I}\sum_{i=1}^I\log q_{x^{(i)},\alpha}-\frac{1}{I}\sum_{i=1}^I \log \pi\left(\mathcal{D}_{obs}|y^{(i)},x^{(i)}\right)-\frac{1}{I}\sum_{i=1}^I\log p_{x^{(i)},\theta^*}, \label{krnet_loss}
 %\\
%		&=\frac{1}{I}\sum_{i=1}^I\Big[\log q_{x^{(i)},\alpha}-\log %\pi(\mathcal{D}_{obs}|y^{(i)},x^{(i)})-\log p_{x^{(i)},\theta^*}\Big], \label{krnet_loss}
\end{align}
where $x^{(i)}=f_{\alpha}^{-1}\left(z^{(i)}\right),\, z^{(i)} \sim p_{z,\theta^*}$ and $y^{(i)}\sim p_{y|x^{(i)},\theta^*} $.

Once KRnet has been trained by minimizing $\mathcal{L}_{KRnet}$, we can estimate the moments of the posterior  $\pi(y|\mathcal{D}_{obs})$ through the pre-trained decoder,
\begin{align}
	\mathbb{E}[y]&=\int y \left(\int p_{y|x,\theta^*}q_{x,\alpha^*}dx \right) dy
	\approx \frac{1}{N_s}\sum_{i=1}^{N_s}\int y p_{y|x^{(i)},\theta^*} dy\nonumber\\
	&\approx \frac{1}{N_s}\sum_{i=1}^{N_s} \mu_{de,\theta^*}\left(x^{(i)}\right),\quad x^{(i)}=f_{\alpha^*}^{-1}\left(z^{(i)}\right),\, z^{(i)} \sim p_{z,\theta^*}, \label{mean_compute}
\end{align}
\begin{align}
	\mathbb{V}[y]&=\mathbb{E}\left[\left(y-\mathbb{E}[y]\right)\left(y-\mathbb{E}[y]\right)^\mathsf{T}\right]\nonumber\\
	&=\int (y-\mathbb{E}[y])(y-\mathbb{E}[y])^\mathsf{T}\left(\int p_{y|x,\theta^*}q_{x,\alpha}dx \right) dy\nonumber\\
	&\approx \frac{1}{N_s}\sum_{i=1}^{N_s}\int (y-\mathbb{E}[y])(y-\mathbb{E}[y])^\mathsf{T} p_{y|x^{(i)},\theta^*} dy\nonumber\\
	&\approx \frac{1}{N_s}\sum_{i=1}^{N_s} \text{diag}\left(\sigma_{de,\theta^*}^{\odot 2}\left(x^{(i)}\right)\right),\quad x^{(i)}=f_{\alpha^*}^{-1}\left(z^{(i)}\right),\, z^{(i)} \sim p_{z,\theta^*},\label{variance_compute}
\end{align}
where $ p_{y|x^{(i)},\theta^*}=\mathcal{N}\left(\mu_{de,\theta^*}\left(x^{(i)}\right), \text{diag}\left(\sigma_{de,\theta^*}^{\odot 2}\left(x^{(i)}\right)\right)\right)$ is the pre-trained decoder given in section \ref{vae_gan_section}, $\alpha^*$ represents the optimal parameters of KRnet, and $N_s$ is the number of posterior samples.
%and ${*}^{\odot2}$ means the component-wise square operation. 
\subsection{Physics-constrained surrogate modeling}
Asides from the pre-trained decoder, we need to pre-train a surrogate model for the forward problem such that we may efficiently minimize $\mathcal{L}_{KRnet}$ given in  \eqref{krnet_loss} by stochastic gradient-based optimization \cite{bottou2018optimization}.
%for efficiently achieving the forward computation and the adjoint method at the optimization step in \eqref{krnet_loss}.
Assume that the governing %the physical system described in \eqref{physical_problem} 
equations are defined on a two-dimensional regular $H\times W$ grid, where $H$ and $W$ denote the number of grid points along the two axes of the spatial domain. We transform the surrogate modeling problem into an image-to-image regression problem through a mapping 
%, with the regression
%function defined as
\begin{align}\label{surroagte_map}
	\hat{\mathcal{F}}:y\in \mathbb{R}^{d_y\times H\times W }\to u\in \mathbb{R}^{d_u\times H\times W }.
\end{align}
Here $d_y$ and $d_u$ are treated as the number of channels in the input and output images, similar to the RGB channels in natural images. More specifically, the surrogate model $u=\hat{\mathcal{F}}_{\Theta}(y)$ with model parameters $\Theta$ is composed of convolutional encoder and decoder networks, i.e.,\ $u=\text{decoder}\circ\text{encoder}(y)$. The surrogate model is trained without labeled data, in other words, the PDE will not be simulated for some chosen $y$. Similar to PINN, it is trained \cite{raissi2019physics} by enforcing the constraints given by  \eqref{physical_problem}, i.e., we minimize the following objective function:
%learning to solve the PDE with given boundary conditions, using the following loss function
\begin{align}\label{surrogate_loss}
	\mathcal{J}\left(\Theta;\{y^{(i)}\}_{i=1}^N\right)=\frac{1}{N}\sum_{i=1}^N\left[ {\left\Arrowvert \mathcal{R}\left(\hat{\mathcal{F}}_{\Theta}\left(y^{(i)}\right),y^{(i)}\right)\right\Arrowvert}_2^2+\beta{\left\Arrowvert\mathcal{B}\left(\hat{\mathcal{F}}_{\Theta}\left(y^{(i)}\right)\right)\right\Arrowvert}_2^2\right],
\end{align}
where 
$\mathcal{R}\left(\hat{\mathcal{F}}_{\Theta}\left(y^{(i)}\right),y^{(i)}\right)=\mathcal{L}\left(\hat{\mathcal{F}}_{\Theta}\left(y^{(i)}\right);y^{(i)}\right)-h$ and $\mathcal{B}\left(\hat{\mathcal{F}}_{\Theta}\left(y^{(i)}\right)\right)=\pB\left(\hat{\mathcal{F}}_{\Theta}\left(y^{(i)}\right);y^{(i)}\right)-g$ measure how well $\hat{\mathcal{F}}_{\Theta}\left(y^{(i)}\right)$ satisfies the partial differential equations and the boundary conditions, respectively, and $\beta>0$ is a penalty parameter. %the weight (Lagrange multiplier) to softly enforce the boundary conditions. 
Both $\mathcal{R}\left(\hat{\mathcal{F}}_{\Theta}\left(y^{(i)}\right),y^{(i)}\right)$ and $\mathcal{B}\left(\hat{\mathcal{F}}_{\Theta}\left(y^{(i)}\right)\right)$ may involve
integration and differentiation with respect to the spatial coordinates, which are approximated with highly efficient discrete operations, e.g.,\ Sobel filters \cite{ma2004invitation,zhu2019physics}. The surrogate trained with the loss function  \eqref{surrogate_loss} is called
physics-constrained surrogate. The %physics constrained model is implemented by 
training process is summarized in Algorithm \ref{alg_surrogate}.

Once we obtain the pre-trained decoder $p_{y|x,\theta^*}$ and the pre-train surrogate model $\hat{\mathcal{F}}_{\Theta^*}(y)$, we can find the transport map from the prior to the posterior in the low-dimensional latent space, 
%and then compute the posterior moments through pre-trained decoder, 
which is implemented in Algorithm \ref{alg_krnet}. The whole process of seeking the dimension-reduced KRnet map (DR-KRnet) is shown in Figure \ref{krnet_flow}.

\begin{algorithm}[H]
	\caption{Training the physics-constrained surrogate model}
	\label{alg_surrogate}
	\begin{algorithmic}[1]
		\Require The prior dataset $Y:=\{y^{(i)}\}_{i=1}^N$, maximum epoch number $E$, batch size $n_{batch}$, and learning rate $\eta$.
            \State Divide $Y$ into $N_b$ mini-batches $\{Y_j\}_{j=1}^{N_b}$ where $N_b=\frac{N}{n_{batch}}$.
            \State Initialize $\Theta$ for the surrogate networks.
		\For {$i = 1:E$}
		\For {$j=1:N_b$}
            \State Compute the objective function $\mathcal{J}(\Theta; Y_j)$ in  \eqref{surrogate_loss} and its gradient $\nabla_{\Theta}\mathcal{J}(\Theta; Y_j)$.
		\State Update the parameters $\Theta$ using gradient-based optimization algorithm (e.g., Adam optimizer \cite{kingma2014adam} with learning rate $\eta$).
		\EndFor
		\EndFor
  \State Let $\Theta^*=\Theta$, where $\Theta$ includes the parameters of the surrogate networks at the last epoch.
		\Ensure The surrogate model $u=\hat{\mathcal{F}}_{\Theta^*}(y)$ with optimal parameters $\Theta^*$.
	\end{algorithmic}
\end{algorithm}
\begin{algorithm}[H]
	\caption{Dimension-reduced KRnet maps (DR-KRnet)}
	\label{alg_krnet}
	\begin{algorithmic}[1]
		\Require Pre-trained decoder $p_{y|x,\theta^*}=\mathcal{N}\left(\mu_{de,\theta^*}\left(x\right), \text{diag}\left(\sigma_{de,\theta^*}^{\odot 2}\left(x\right)\right)\right)$, pre-trained surrogate model $\hat{\mathcal{F}}_{\Theta^*}$, sample size from $\mathcal{N}(0,\mathbf{I})$ $I$, sample size for posterior distribution $N_s$, batch size $n_{batch}$, maximum epoch number $E$, learning rate $\eta$.
  \State Generate the training dataset $Z:=\{z^{(i)}\}_{i=1}^I$ where $z^{(i)}\sim \mathcal{N}(0,\mathbf{I})$.
  \State Divide $Z$ into $N_b$ mini-batches $\{Z_j\}_{j=1}^{N_b}$ where $N_b=\frac{I}{n_{batch}}$.
  \State Initialize $\alpha$ of the KRnet map.
		\For {$i = 1:E$}
		\For {$j=1:N_b$}
		\State Compute $X_j=f_{\alpha}^{-1}(Z_j)$ in  \eqref{eqn_KR_alpha}.
            \State Compute the high-dimensional parameters: $Y_j=\mu_{de,\theta^*}(X_j).$
		\State Compute the surrogate model: $U_j=\hat{\mathcal{F}}_{\Theta^*}(Y_j).$
		\State Compute the loss function $\mathcal{L}_{KRnet}$ in  \eqref{krnet_loss} and its gradient $\nabla_{\alpha}\mathcal{L}_{KRnet}$.
		\State Update the parameters $\alpha$ using gradient-based optimization algorithm (e.g., Adam optimizer \cite{kingma2014adam} with learning rate $\eta$).
		% $\alpha \leftarrow\alpha+\eta\nabla_{\alpha}\mathcal{L}_{KRnet}$
		\EndFor
		\EndFor
  \State Let $\alpha^*=\alpha$, where $\alpha$ includes the parameters of the KRnet map at the last epoch.
		\State Sample $\{z^{(i)}\}_{i=1}^{N_s} $ where $z^{(i)}\sim \mathcal{N}(0,\mathbf{I})$.
		\State $x^{(i)}=f_{\alpha^*}^{-1}\left(z^{(i)}\right)$, for $i=1,2,\dots,N_s$.
		\State Compute the posterior mean $\hat{\mathbb{E}}[y]=\frac{1}{N_s}\sum_{i=1}^{N_s} \mu_{de,\theta^*}\left(x^{(i)}\right)$ in  \eqref{mean_compute}.
             \State Compute the posterior variance $\hat{\mathbb{V}}[y]=\frac{1}{N_s}\sum_{i=1}^{N_s} \text{diag}\left(\sigma_{de,\theta^*}^{\odot 2}\left(x^{(i)}\right)\right)$ in  \eqref{variance_compute}.
		\Ensure The posterior mean $\hat{\mathbb{E}}[y]$ and the posterior variance $\hat{\mathbb{V}}[y]$.
	\end{algorithmic}
\end{algorithm}
\begin{figure}
	\centering
	\includegraphics[width=1.0\textwidth]{image/KRnet_NN.png}
	\caption{The full workflow of seeking the dimension-reduced KRnet map.}
	\label{krnet_flow}
\end{figure}
\section{Numerical experiments}\label{section_experiments}
We consider single-phase, steady-state Darcy flows. Let $\alpha(s)$ denote an unknown permeability field. The pressure field $u(s,y(s))$ is defined by the following diffusion equation
\begin{align}\label{diffusion}
    -\nabla \cdot\left(\alpha(s)\nabla u(s,\alpha(s))\right)=h(s),\quad s\in \mathcal{S},
\end{align}
where the physical domain $\mathcal{S}=(0,1)^2\in\mathbb{R}^2$ is considered. We use homogeneous Dirichlet boundary conditions on the left and right boundaries and homogeneous Neumann boundary conditions on the top
and bottom boundaries, i.e.,
\begin{align*}
    &u(s,\alpha(s))=0,\quad s\in  \{0\} \times [0, 1],\\
     &u(s,\alpha(s))=0,\quad s\in  \{1\} \times [0, 1],\\
    &\alpha(s)\nabla u(s,\alpha(s))\cdot \mathbf{n}=0,\quad s\in \{(0, 1) \times \{0\}\} \cup \{(0, 1) \times \{1\}\},
\end{align*}
where $\mathbf{n}$ is the outward-pointing normal to the Neumann boundary. The source term is specified as $h(s)=3$.
In the following numerical experiments, the computation domain $\mathcal{S}$ is discretized by a uniform 64×64 grid, i.e., $H=64,W=64$ in \eqref{surroagte_map}. The goal in this paper is to infer the log-permeability field $y(s)=\log \alpha(s)$ from noisy and incomplete observations.

%To enforce the non-negative permeability constraint, we consider the log-permeability as the parameter of interest
%in the inverse problem with $y(s,\xi)=\log \alpha(s,\xi)$. Thus the inverse problem for the above model is to infer the unknown log-permeability field from given noisy pressure measurements. 
% In this paper, the exact log-permeability field $x_{exact}$ is also defined on a 64×64 uniform grid. 
% We consider 64 pressure observations that are uniformly located in $[0.0625 + 0.125i, 0.0625 +
% 0.125i],i = 0, 1, 2,\dots, 7$. 
% Noisy observations are formulated by adding a $5\%$ independent additive Gaussian random noise to the simulation pressure field which is obtained by a mixed finite element formulation
% implemented in FEniCS \cite{langtangen2017solving}.

We assume that the log-permeability field $y(s)$ is a Gaussian random field (GRF), i.e.,\,$y(s)\sim \mathcal{GP}\left(m(s),k(s_1,s_2)\right)$, where $m(s)$ and $k(s_1,s_2)$ are the mean and covariance functions, respectively. Let $s_1=[s_{1,1},s_{1,2}]^\mathsf{T}$ and $s_2=[s_{2,1},s_{2,2}]^\mathsf{T}$ denote two arbitrary spatial locations. The covariance function $k(s_1,s_2)$ is taken as
\begin{align}\label{covariance}
    k(s_1,s_2)=\sigma^2\exp\left(-\sqrt{\left(\frac{s_{1,1}-s_{2,1}}{l_1}\right)^2+\left(\frac{s_{1,2}-s_{2,2}}{l_2}\right)^2}\right),
\end{align}
where $\sigma^2$ is the variance, and $l_1,\,l_2$ are the length scales. 
This random field can be approximated by
a truncated Karhunen-Lo{\`e}ve expansion (KLE),
\begin{align}
    y(s)\approx m(s)+\sum_{k=1}^{d_{KL}}\sqrt{\lambda_k}y_k(s)\xi_k,
\end{align}
where $d_{KL}\in\mathbb{N}_+$, $y_k(s)$ and $\lambda_k$ are the eigenfunctions and
eigenvalues of $k(s_1,s_2)$ and $\{\xi_k\}_{k=1}^{d_{KL}}$ are i.i.d. Gaussian random variables of zero mean and unit variance. 
%The prior of $\{\xi_k\}_{k=1}^{d_{KL}}$ are set to be independent standard normal distributions. 
We set $m(s)=1$ and $\sigma^2=0.5$ in the numerical experiments. %As usual, 
We set $d_{KL}$ large enough such that $95\%$ of the total variance of the exponential covariance function are captured. 

We now generate the datasets as historical data for the training of VAE priors. 
 One can assume that the data are from random fields of different length scales. 
 %are not fixed in the prior distribution. In this work, 
 More specifically, we consider two different experimental setups with an increasing difficulty. The length scales are set to be $l_1 = l_2 = 0.2 + 0.01i,i = 0, 1, 2,\dots, 9$ in test problem 1 and 
% and the corresponding $d_{KL}\in [387,737]$. 
 $l_1 = l_2 = 0.1 + 0.01i,i = 0, 1, 2,\dots, 9$ in test problem 2.
% and the corresponding $d_{KL}\in [800,1800]$. 
%The parameter $y$ is uniformly discretized into a 64×64 grid.
We generate 2000 samples for each length scale and combine them to obtain the prior datasets $\{y^{(i)}\}_{i=1}^N$ for training VAE priors, where $N=20000$. Note that the KLE method
is inappropriate for dealing with varying correlation lengths but the VAE priors in section \ref{vae_gan_section} for characterizing the prior do not have such a limitation.  The architectures of the neural networks used in this paper have been summarized in Appendix. All neural network models are trained on a single NVIDIA GeForce GTX 1080Ti GPU card.

%  Given posterior samples $\{y^{(i)}\}_{i=1}^{N_s}$, the posterior
% mean and variance estimates are computed through \eqref{mean_compute}--\eqref{variance_compute}.
To access the accuracy of the estimated posterior mean field, relative errors are defined as
\begin{align}\label{error_compute}
    \epsilon_{relative}:={\left\Arrowvert\mathbb{E}[y]-y_{exact}\right\Arrowvert}_2/{\Arrowvert y_{exact}\Arrowvert}_2,
\end{align}
where $y_{exact}$ is the exact log-permeability field, and $\mathbb{E}[y]$ can be approximated by computing the posterior mean through \eqref{mean_compute}.

\subsection{Test problem 1}
Given the prior dataset $\{y^{(i)}\}_{i=1}^N$ with $y^{(i)}\in\mathbb{R}^{ 64\times 64}$, the latent variable is set to $x\in \mathbb{R}^{36}$ and then we train the corresponding VAE prior. 
%then train the VAE priors with Algorithm \ref{alg_vae_gan}. For the training hyperparameters 
In Algorithm \ref{alg_vae_gan}, we assign the batch size $n_{batch} = 100$ and the maximum epoch number $E=200$, and 
%. For the optimization in Algorithm \ref{alg_vae_gan}, 
employ the Adam optimizer with a learning rate $\eta=0.0001$. 
The architecture of the VAE prior is given in \ref{vae_nn}. 
%The VAE prior is trained 
%The neural networks are trained 
%with $E=200$ epochs. 
%The training procedure takes about 5 minutes. 
To generate samples that are consistent with the prior dataset,  one can sample a latent variable $x$ from Gaussian distribution $\mathcal{N}(0,\mathbf{I})$, and then generate the samples of $y$ by the learned decoder $p_{y|x,\theta^*}$ , i.e., $y=\mu_{de,\theta^*}(x)$. Some samples generated by the VAE prior are shown in Figure \ref{samplesVAE}.
% \begin{figure}
%     \centering
%     \includegraphics[width=0.8\textwidth]{Bayesian_inference_with_DGM/image/vae_36dim_gaussian_samples.png}
%     \caption{ 64×64 resolution random samples using the decoder $p_{y|x,\theta^*}$ of VAE prior, where $x\in \mathbb{R}^{36}$ is sampled from $\mathcal{N} (0, \mathbf{I})$, test problem 1.}
%     \label{samplesVAE}
% \end{figure}
\begin{figure}
	\centering
	\subfloat[][Prior sample 1]{\includegraphics[width=.28
 \textwidth]{image/Sample_prior1.png}}\quad
	\subfloat[][Prior sample 2]{\includegraphics[width=.28\textwidth]{image/Sample_prior2.png}}\quad
	\subfloat[][Prior sample 3]{\includegraphics[width=.28\textwidth]{image/Sample_prior3.png}}
	\\
	\subfloat[][Prior sample 4]{\includegraphics[width=.28\textwidth]{image/Sample_prior4.png}}\quad
	\subfloat[][Prior sample 5]{\includegraphics[width=.28\textwidth]{image/Sample_prior5.png}}
   \quad \subfloat[][Prior sample 6]{\includegraphics[width=.28\textwidth]{image/Sample_prior6.png}}
	%\caption{Isometry and variance quality for synthetic data.}
	\caption{64×64 resolution random samples generated from $p_{y|x,\theta^*}p_x$ for test problem 1, where $x\in \mathbb{R}^{36}$, $p_{y|x,\theta^*}$ is the decoder of the VAE prior and  $p_x=\mathcal{N} (0, \mathbf{I})$.}
    \label{samplesVAE}
\end{figure}
%It is easy to see that the VAE prior has successfully captured the prior datasets $\{y^{(i)}\}_{i=1}^N$.
%because the patterns and features of the new prior samples are similar to that of the prior datasets.
% \begin{figure}
%     \centering
%     \includegraphics[width=0.8\textwidth]{image/prior_data_gaussian_lowwdim11_12.png}
%     \caption{The randomly sampled realizations in the training dataset $\{y^{(i)}\}_{i=1}^N$, test problem 1 .}
%     \label{priordata_lowdim}
% \end{figure}
% \begin{figure}
%     \centering
%     \includegraphics[width=0.8\textwidth]{image/vae_gan_16dim_gaussian_lowwdim11_12_samples.png}
%     \caption{ 64×64 resolution random samples using the decoder $p_{y|x,\theta^*}$, where $x\in \mathbb{R}^{16}$ is sampled from $\mathcal{N} (0, I)$, test problem 1.}
%     \label{samplesVAEGAN_lowdim}
% \end{figure}

% As mentioned above, we only obtain the datasets $\{y^{(i)}\}_{i=1}^N$ from historical sample
% information. Variational Auto-Encoders (VAEs) are powerful models for learning low-dimensional representations of datasets.
%\subsubsection{Surrogate results}
%The most computational cost in the Bayesian inference involves the forward computation and at each optimization step the adjoint method. In order to accelerate the inference, we propose learning the neural networks surrogate
%to replace the adjoint method, which is not only automatic differentiation but also extremely fast with the deep learning frameworks like Tensorflow 2.0. 
% The 2D unit square domain $\mathcal{S}=(0,1)^2$ is discretized into a
% uniform 64x64 grid, i,e., $H=W=64$.
With the dataset $\{y^{(i)}\}_{i=1}^N$, we conduct Algorithm \ref{alg_surrogate} to train the surrogate model.  
%only the input data, i.e.,
%the log-permeability prior 
The loss function for \eqref{diffusion} and the architecture of the surrogate model are given in \ref{surrogate_nn}. In Algorithm \ref{alg_vae_gan}, the batch size and 
the maximum epoch number
are set to $n_{batch}=100$ and  $E=100$ respectively, and the Adam optimizer is employed
%For the optimization , 
 with a learning rate $\eta=0.001$. 
%The networks are trained for $E=100$ epochs.
Figure \ref{surrogate_plot} shows the performance of the trained surrogate model by comparing the prediction of the surrogate model with the simulation given by the finite element method implemented in FEniCS \cite{langtangen2017solving}. 
The difference between the surrogate pressure $\hat{u}$ and the simulation pressure $u$ is defined by $\hat{u}-u$ (see Figure \ref{surrogate_plot}(d)). The relative errors ($\Arrowvert \hat{u}-u \Arrowvert_2/\Arrowvert u \Arrowvert_2$) is 0.05694.
%Using the given true log-permeability as the input of the learned surrogate, we predict the corresponding surrogate pressure field, as shown in the third plot of Figure \ref{surrogate_plot}.  The simulation pressure field illustrated in the second plot of Figure \ref{surrogate_plot} is obtained by a mixed finite element formulation
%implemented in FEniCS \cite{langtangen2017solving}.
% It is seen that the maximum absolute error between the finite element simulation and the surrogate prediction is only about 0.01 ({\color{red}Can you normalize it using the $L_2$ norm of the finite element simulation}). 
%It indicates that the learned surrogate can make a good prediction for the forward computation.
% \begin{figure}
%     \centering
%     \includegraphics[width=1.0\textwidth]{image/surrogate_gaussian_plot.png}
%     \caption{Illustration of surrogate results, test problem 1. The four figures from left to right are the exact log-permeability to be estimated, the corresponding pressure computed by the simulator, the corresponding pressure computed by the surrogate model, and their difference.}
%     \label{surrogate_plot}
% \end{figure}
\begin{figure}
	\centering
	\subfloat[][The exact log-permeability]{\includegraphics[width=.28
 \textwidth]{image/Exact.png}}\quad
	\subfloat[][Simulation pressure]{\includegraphics[width=.28\textwidth]{image/Simulation.png}}\\
	\subfloat[][Surrogate pressure]{\includegraphics[width=.28\textwidth]{image/Surrogate.png}}\quad
	\subfloat[][Difference between simulation pressure and surrogate pressure]{\includegraphics[width=.28\textwidth]{image/Difference.png}}
	\caption{Illustration of surrogate results for test problem 1.}
    \label{surrogate_plot}
\end{figure}
% \begin{figure}
%     \centering
%     \includegraphics[width=1.0\textwidth]{image/lowdim_surrogate_plot_1900.png}
%     \caption{Illustration of surrogate results, test problem 1. The four figures from left to right are the true log-permeability to be estimated, the corresponding
% pressure computed by the simulator, the corresponding pressure computed by the surrogate model, and their difference.}
%     \label{lowdim_surrogate_plot}
% \end{figure}
%\subsubsection{Bayesian inverse results}

% Our strategy consists of two main components: data-driven VAE prior and KRnet, which is referred as DR-KRnet. 
Our DR-KRnet is compared with VAEprior-MCMC    
%the performance of the proposed method is compared with VAEprior-MCMC 
which uses MCMC to sample the posterior of the latent variable given by the same VAE prior as in DR-KRnet. We consider 64 pressure observations from locations $[0.0625 + 0.125i, 0.0625 +
0.125i],i = 0, 1, 2,\dots, 7$. 
Noisy observations are formulated by adding $5\%$ independent additive Gaussian noise to the simulated pressure field (see Figure\ \ref{surrogate_plot}(b)).

Using the pre-trained decoder and the surrogate, we seek a KRnet with Algorithm \ref{alg_krnet} to approximate the posterior in the latent space.  For KRnet, we partition the components of $x\in\mathbb{R}^{36}$ to 6 equal groups and deactivate one group after 8 affine coupling layers, where the bijection given by each coupling layer is based on the outputs of a neural network with two fully connected hidden layers of 48 neurons (More detailed about the structure of KRnet can be found in \cite{tang2020deep,adda_2022}). In Algorithm \ref{alg_krnet}, the batch size and 
the maximum epoch number
are set to $n_{batch}=100$ and  $E=5$ respectively, and the Adam optimizer is applied
 with a learning rate $\eta=0.01$. 
 % The Adam optimizer is employed with a learning
% rate $\eta=0.01$. 
The sample size from standard Gaussian distribution is $I=5000$. The sample size for posterior distribution is $N_s=2000$. 
%The loss function of KRnet converges at the third epoch. 
For VAEprior-MCMC, we consider the preconditioned Crank Nicolson MCMC (pCN-MCMC) method \cite{beskos2008mcmc,cotter2013mcmc} 
%in \ref{pcn_alg} $N_{pcn}=10000$ $N_s=2000$
and then run 10000 iterations to ensure its convergence. 
% The step size of the random movement from the current state to a new position is controlled by the free parameter $\gamma$ (see \ref{pcn_alg}). The free parameter is set to $\gamma=0.08$. 
For all implementations of the MCMC algorithm, the last 2000 states are retained and regarded as the posterior samples.  The corresponding acceptance rate (numbers of accepted samples
divided by the total sample size) is $30.64\%$. 

Figure \ref{inverse_vae} and Figure \ref{inverse_vae_mcmc} provide the inversion results given by DR-KRnet and VAEprior-MCMC respectively. It is seen that the two strategies yield consistent mean and variance and posterior samples. %The detailed results of the two methods regarding 
More details about accuracy and efficiency are presented in Table~\ref{test1}, where the relative errors are computed through \eqref{error_compute}. Time consumption for DR-KRnet (see Algorithm \ref{alg_krnet}) and VAEprior-MCMC is the computational cost of approximating the posterior of the latent variable by KRnet and MCMC respectively.
%(corresponding to Algorithm \ref{alg_krnet} and Algorithm \ref{pcn_compute}). 
DR-KRnet yields a smaller relative error than VAEprior-MCMC with a computational cost reduced by half.
%The time consumption for DR-KRnet is about half of that for VAEprior-MCMC, and DR-KRnet can give smaller relative errors.

% he peak-signal-to-noise (PSNR) and the structural similarity index measure (SSIM) \cite{hore2010image} are two well-known image quality metrics. The PSNR and SSIM are shown in Table \ref{t1}. Compared with inversion results through VAE priors, the posterior means through VAE-GAN priors are of better image quality in terms of the PSNR and SSIM. 
% \begin{figure}
%     \centering
%      \includegraphics[width=1.0\textwidth]{Bayesian_inference_with_DGM/image/agaussian_inverse_36dim_xia_vae.png}
%     \caption{The exact log-permeability field with a uniform 64 × 64 grid, the mean and variance of the estimated posterior distribution through DR-KRnet in the first row, test problem 1. Three posterior samples from the posterior in the second row.}
%     \label{inverse_vae}
% \end{figure}
\begin{figure}
	\centering
	\subfloat[][The exact log-permeability field]{\includegraphics[width=.28
 \textwidth]{image/KRnet_exact_lowdim.png}}\quad
	\subfloat[][Posterior mean]{\includegraphics[width=.28\textwidth]{image/KRnet_mean_lowdim.png}}\quad
	\subfloat[][Posterior variance]{\includegraphics[width=.28\textwidth]{image/KRnet_variance_lowdim.png}}
	\\
	\subfloat[][Posterior sample 1]{\includegraphics[width=.28\textwidth]{image/KRnet_p1_lowdim.png}}\quad
	\subfloat[][Posterior sample 2]{\includegraphics[width=.28\textwidth]{image/KRnet_p2_lowdim.png}}
   \quad \subfloat[][Posterior sample 3]{\includegraphics[width=.28\textwidth]{image/KRnet_p3_lowdim.png}}
	%\caption{Isometry and variance quality for synthetic data.}
	\caption{The inversion results of DR-KRnet for test problem 1. }
    \label{inverse_vae}
\end{figure}
\begin{figure}
	\centering
	\subfloat[][The exact log-permeability field]{\includegraphics[width=.28
 \textwidth]{image/MCMC_exact.png}}\quad
	\subfloat[][Posterior mean]{\includegraphics[width=.28\textwidth]{image/MCMC_mean.png}}\quad
	\subfloat[][Posterior variance]{\includegraphics[width=.28\textwidth]{image/MCMC_variance.png}}
	\\
	\subfloat[][Posterior sample 1]{\includegraphics[width=.28\textwidth]{image/MCMC_p1.png}}\quad
	\subfloat[][Posterior sample 2]{\includegraphics[width=.28\textwidth]{image/MCMC_p2.png}}
   \quad \subfloat[][Posterior sample 3]{\includegraphics[width=.28\textwidth]{image/MCMC_p3.png}}
	%\caption{Isometry and variance quality for synthetic data.}
	\caption{The inversion results of VAEprior-MCMC for test problem 1. }
    \label{inverse_vae_mcmc}
\end{figure}

% \begin{figure}
%     \centering
%     % \includegraphics[width=1.0\textwidth]{image/show_darcy_flow_cnn_xia_8_48_1000_10_8_72_36dim_88.png}
%      \includegraphics[width=1.0\textwidth]{Bayesian_inference_with_DGM/image/a_inverse36dim_xia_baseline_0.02.png}
%     \caption{The exact log-permeability field with a uniform 64 × 64 grid, the mean and variance of the estimated posterior distribution through VAEprior-MCMC in the first row, test problem 1. Three posterior samples from the posterior in the second row.}
%     \label{inverse_vae_mcmc}
% \end{figure}

\begin{table}
		\caption{Comparisons of DR-KRnet and VAEprior-MCMC, test problem 1.}
		\label{test1}
		\centering
		\begin{tabular}{lll}
			\toprule
			%\cmidrule(r){1-2}
			 Model&  $\epsilon_{relative}$ &Time consumption\\
			\midrule
			VAEprior-MCMC &0.4014& 5.2224 minutes\\
			DR-KRnet & 0.3914&1.9608 minutes\\
			\bottomrule
		\end{tabular}
\end{table}
% \begin{figure}
%     \centering
%     \includegraphics[width=1.0\textwidth]{image/lowdim_vae_1900.png}
%     \caption{The exact log-permeability field with a uniform 64 × 64 grid, the mean and standard deviation of the estimated posterior distribution through VAE in the first row. Three posterior samples from the posterior in the second row.}
%     \label{inverse_vae_lowdim}
% \end{figure}
% \begin{figure}
%     \centering
%     \includegraphics[width=1.0\textwidth]{image/lowdim_gan_1900.png}
%     \caption{The exact log-permeability field with a uniform 64 × 64 grid, the mean and standard deviation of the estimated posterior distribution through VAE-GAN in the first row. Three posterior samples from the posterior in the second row.}
%     \label{inverse_vae_gan_lowdim}
% \end{figure}

\subsection{Test problem 2}
%\subsubsection{Generative results}
In this case, we consider a larger $d_{KL}$ subject to smaller correlation lengths when generating the prior dataset. 
%is more difficult than test problem 1. 
Using $N=20000$ images as the prior dataset, we train the VAE priors with Algorithm \ref{alg_vae_gan}, where the architecture of the neural
networks is described in \ref{vae_nn}. Here the hyperparameters are the same as those of test problem 1 %. The only difference is the dimensionality 
except that the dimension of the latent variable is increased to 64. 
%$x$, where $x\in \mathbb{R}^{64}$. Since the problem becomes more difficult, we choose a higher dimension to relieve the information compression. 
% Fig.\ \ref{samplesVAEGAN} shows 6 random samples generated by the learned VAE-GAN priors. 
Figure \ref{samplesVAE_highdim} includes 6 realizations given by the decoder of the trained VAE prior. 
%with a 64×64 resolution that are output of the trained decoder network in VAE priors. 
%We can see that the trained VAE prior provides enough prior information and especially, has a great generative ability. 
% \begin{figure}
%     \centering
%     \includegraphics[width=0.8\textwidth]{Bayesian_inference_with_DGM/image/samples_highdim_gaussian_vae_64dim.png}
%     \caption{ 64×64 resolution random samples using the decoder $p_{y|x,\theta^*}$ of VAE prior, where $x\in \mathbb{R}^{64}$ is sampled from $\mathcal{N}(0, \mathbf{I})$, test problem 2.}
%     \label{samplesVAE_highdim}
% \end{figure}
\begin{figure}
	\centering
	\subfloat[][Prior sample 1]{\includegraphics[width=.28
 \textwidth]{image/Sample_prior1_highdim.png}}\quad
	\subfloat[][Prior sample 2]{\includegraphics[width=.28\textwidth]{image/Sample_prior2_highdim.png}}\quad
	\subfloat[][Prior sample 3]{\includegraphics[width=.28\textwidth]{image/Sample_prior3_highdim.png}}
	\\
	\subfloat[][Prior sample 4]{\includegraphics[width=.28\textwidth]{image/Sample_prior4_highdim..png}}\quad
	\subfloat[][Prior sample 5]{\includegraphics[width=.28\textwidth]{image/Sample_prior5_highdim.png}}
   \quad \subfloat[][Prior sample 6]{\includegraphics[width=.28\textwidth]{image/Sample_prior6_highdim.png}}
	
	\caption{64×64 resolution random samples generated from $p_{y|x,\theta^*}p_x$ for test problem 2, where $x\in \mathbb{R}^{64}$, $p_{y|x,\theta^*}$ is the decoder of the VAE prior and  $p_x=\mathcal{N} (0, \mathbf{I})$.}
    \label{samplesVAE_highdim}
\end{figure}
% We apply the datasets to pre-train the VAE for dimension reduction, where the latent variables $x\in \mathbb{R}^{36}$. For the pre-trained decoder $p_{y|x,\theta^*}$, we can sample the latent variables
% $x$ from $\mathcal{N} (0, I)$, and use these latent variables as input to the model $p_{y|x,\theta^*}$. Fig.\  \ref{samplesVAE} shows 6 realizations
% with 64×64 resolution that are output of this decoder network. Fig.\  \ref{reconVAE} illustrates the reconstruction quality of VAE. 
% \begin{figure}
%     \centering
%     \includegraphics[width=0.8\textwidth]{image/prior_data_gaussian.png}
%     \caption{The randomly sampled realizations in the training dataset $\{y^{(i)}\}_{i=1}^N$, test problem 2.}
%     \label{priordata}
% \end{figure}

% \begin{figure}
%     \centering
%     \includegraphics[width=0.8\textwidth]{image/vae_gan_36dim_gaussian_samples.png}
%     \caption{ 64×64 resolution random samples using the decoder $p_{y|x,\theta^*}$ of VAE-GAN, where $x\in \mathbb{R}^{36}$ is sampled from $\mathcal{N} (0, I)$, test problem 2.}
%     \label{samplesVAEGAN}
% \end{figure}


%%\subsubsection{Surrogate results}
The setups and hyperparameters of the surrogate model are the same as those of test problem 1. The performance of the surrogate model is illustrated in Figure \ref{highdim_surrogate_plot}, where the simulated pressure field given by the finite element method and the predicted pressure field given by the surrogate model are shown in Figure \ref{highdim_surrogate_plot}(b)--(c) respectively for the log-permeability shown in Figure \ref{highdim_surrogate_plot}(a). 
%The given true log-permeability shown in the first plot of Figure \ref{highdim_surrogate_plot} is regarded as the input of the learned surrogate and then the surrogate pressure field is achieved, as shown in the third images of Figure \ref{highdim_surrogate_plot}. 
The difference between the surrogate pressure $\hat{u}$ and the simulation pressure $u$ is defined by $\hat{u}-u$ shown in Figure \ref{highdim_surrogate_plot}(d).
% The maximum absolute error between the finite element simulation and the surrogate prediction is only about 0.03 {\color{red} Can you normalized it?}. %It implies that the learned surrogate can efficiently approximate the forward model with little loss in accuracy.
The relative errors ($\Arrowvert \hat{u}-u \Arrowvert_2/\Arrowvert u \Arrowvert_2$) is 0.08260.
Compared to test problem 1, the prediction of the surrogate model captures the solution sufficiently well with a slight loss in accuracy.    
% \begin{figure}
%     \centering
%     \includegraphics[width=1.0\textwidth]{image/surrogate_gaussian_plot_highdim.png}
%     \caption{Illustration of surrogate results, test problem 2. The four figures from left to right are the exact log-permeability to be estimated, the corresponding
% pressure computed by the simulator, the corresponding pressure computed by the surrogate model, and their difference.}
%     \label{highdim_surrogate_plot}
% \end{figure}
\begin{figure}
	\centering
	\subfloat[][The exact log-permeability]{\includegraphics[width=.28
 \textwidth]{image/Exact_highdim.png}}\quad
	\subfloat[][Simulation pressure]{\includegraphics[width=.28\textwidth]{image/Simulation_highdim.png}}\\
	\subfloat[][Surrogate pressure]{\includegraphics[width=.28\textwidth]{image/Surrogate_highdim.png}}\quad
	\subfloat[][Difference between simulation pressure and surrogate pressure]{\includegraphics[width=.28\textwidth]{image/Difference_highdim.png}}
	\caption{Illustration of surrogate results for test problem 2.}
    \label{highdim_surrogate_plot}
\end{figure}
%\subsubsection{Bayesian inversion results}
% \begin{figure}
%     \centering
%     \includegraphics[width=0.8\textwidth]{image/vae_36dim_cnn_8_48_xia_nchw.png}
%     \caption{Sample 1-3 are the reconstruction results of True 1-3 through VAE, respectively.}
%     \label{reconVAE}
% \end{figure}
% \begin{figure}
%     \centering
%     \includegraphics[width=0.8\textwidth]{image/vae_gan_vae_reconstru.png}
%     \label{reconVAE_GAN}
%     \caption{Sample 1-3 are the reconstruction results of True 1-3 through VAE-GAN, respectively.}
% \end{figure}

%We apply the DR-KRnet for high-dimensional Bayesian inference and provide a comparison with VAEprior-MCMC. 
Since test problem 2 is more challenging than test problem 1, we consider 225 pressure observations that are uniformly located in $[0.0625 + 0.125i, 0.0625 + 0.0625i],\,i = 0, 1, 2,\dots, 14$. The observations are generated from the simulated pressure field by adding $1\%$ independent additive Gaussian noise. 
%A$1\%$ independent additive Gaussian random noise is added to the simulation pressure field which is obtained by a mixed finite element formulation
%implemented in FEniCS \cite{langtangen2017solving}.

%Based on the pretrained decoder and surrogate model, we use KRnet to approximate $\pi(x|\mathcal{D}_{obs})$ directly with Algorithm \ref{alg_krnet}. 
For DR-KRnet, the architecture of the KRnet is the same as that in test problem 1 except that the components of $x\in\mathbb{R}^{64}$ are divided into 8 even groups. %The loss function of KRnet converges at the third epoch. 
For VAEprior-MCMC, %, we use preconditioned Crank-Nicholson (pCN) algorithm in \ref{pcn_alg} to sample the posterior of the latent variables. 
 we run 10000 iterations and then set the last 2000 states as posterior samples,
 %$\gamma=0.02$, 
 %$N_s=2000$ and $N_{pcn}=10000$, 
 and the acceptance rate is $24.07\%$. The inversion results for the two methods are shown in Figures \ref{inverse_vae_highdim}--\ref{inverse_vae_mcmc_highdim}. It is seen that for this case the inversion result of DR-KRnet is consistent with the exact log-permeability but VAEprior-MCMC fails to approximate the posterior of the latent variables.  
 %and hence DR-KRnet is more reliable than VAEprior-MCMC, especially in high-dimensional cases. 
 In Table \ref{test22}, more scenarios are considered in terms of the dimension of the latent variable $d$. It is seen that 
% different dimensions of the latent variable are considered and then 
DR-KRnet outperforms VAEprior-MCMC in terms of both accuracy and computational cost.
\begin{figure}
	\centering
	\subfloat[][The exact log-permeability field]{\includegraphics[width=.28
 \textwidth]{image/KRnet_exact.png}}\quad
	\subfloat[][Posterior mean]{\includegraphics[width=.28\textwidth]{image/KRnet_mean.png}}\quad
	\subfloat[][Posterior variance]{\includegraphics[width=.3\textwidth]{image/KRnet_variance.png}}
	\\
	\subfloat[][Posterior sample 1]{\includegraphics[width=.28\textwidth]{image/KRnet_p1.png}}\quad
	\subfloat[][Posterior sample 2]{\includegraphics[width=.28\textwidth]{image/KRnet_p2.png}}
   \quad \subfloat[][Posterior sample 3]{\includegraphics[width=.28\textwidth]{image/KRnet_p3.png}}
	%\caption{Isometry and variance quality for synthetic data.}
	\caption{The inversion results of DR-KRnet for test problem 2. }
    \label{inverse_vae_highdim}
\end{figure}
% \begin{figure}
%     \centering
%      \includegraphics[width=1.0\textwidth]{Bayesian_inference_with_DGM/image/ahighdim_gaussian_inverse_64dim_xia_vae.png}
%     \caption{The exact log-permeability field with a uniform 64 × 64 grid, the mean and variance of the estimated posterior distribution through DR-KRnet in the first row. Three posterior samples from the posterior in the second row.}
%     \label{inverse_vae_highdim}
% \end{figure}

% Compared with inversion results through VAE priors, the posterior means through VAE-GAN priors perform a better image quality in terms of the PSNR and SSIM.
\begin{figure}
	\centering
	\subfloat[][The exact log-permeability field]{\includegraphics[width=.28
 \textwidth]{image/MCMC_exact_highdim.png}}\quad
	\subfloat[][Posterior mean]{\includegraphics[width=.28\textwidth]{image/MCMC_mean_highdim.png}}\quad
	\subfloat[][Posterior variance]{\includegraphics[width=.3\textwidth]{image/MCMC_variance_highdim.png}}
	\\
	\subfloat[][Posterior sample 1]{\includegraphics[width=.28\textwidth]{image/MCMC_p1_highdim.png}}\quad
	\subfloat[][Posterior sample 2]{\includegraphics[width=.28\textwidth]{image/MCMC_p2_highdim.png}}
   \quad \subfloat[][Posterior sample 3]{\includegraphics[width=.28\textwidth]{image/MCMC_p3_highdim.png}}
	%\caption{Isometry and variance quality for synthetic data.}
	\caption{The inversion results of VAEprior-MCMC for test problem 2. }
    \label{inverse_vae_mcmc_highdim}
\end{figure}
% \begin{figure}
%     \centering
%     % \includegraphics[width=1.0\textwidth]{image/darcy_flow_cnn_xia_8_48_1000_10_8_48_36dim_gan.png}
%      \includegraphics[width=1.0\textwidth]{Bayesian_inference_with_DGM/image/ahighdim_gaussian_inverse_64dim_xia_vae.png}
%     \caption{The exact log-permeability field with a uniform 64 × 64 grid, the mean and variance of the estimated posterior distribution through VAEprior-MCMC in the first row. Three posterior samples from the posterior in the second row.}
%     \label{inverse_vae_mcmc_highdim}
% \end{figure}
% \begin{table}
% 		\caption{Comparisons of DR-KRnet and VAEprior-MCMC, test problem 2.}
% 		\label{test2}
% 		\centering
% 		\begin{tabular}{lll}
% 			\toprule
% 			%\cmidrule(r){1-2}
% 			 Model& Time consumption (hours) &Relative errors\\
% 			\midrule
% 			VAEprior-MCMC & 0.1303&2.2340\\
% 			DR-KRnet &  0.06675&0.4171 \\
% 			\bottomrule
% 		\end{tabular}
% \end{table}
\begin{table}
		\caption{Comparisons of DR-KRnet and VAEprior-MCMC for test problem 2.}
		\label{test22}
		\centering
		\begin{tabular}{ccccc}
			\toprule
			%\cmidrule(r){1-2}
			 Model&$d$ & $\epsilon_{relative}$ &Time consumption&Acceptance rate\\
			\midrule
			DR-KRnet &36 &0.3854&1.9842 minutes&-\\
			DR-KRnet & 64& 0.4206&2.6844 minutes &- \\
                \midrule
                VAEprior-MCMC &36 &2.9391&4.9152 minutes&$25.64\%$ \\
			VAEprior-MCMC &64& 2.0616&5.4462 minutes&$24.07\%$\\
                VAEprior-MCMC &128 &2.0760&5.1006 minutes&$27.97\%$\\
			VAEprior-MCMC &256& 2.6581&4.824 minutes& $30.5\%$ \\ 
                VAEprior-MCMC &512& 2.2229&5.388 minutes&$31.8\%$\\
                
   \bottomrule
		\end{tabular}
\end{table}
% \begin{table}
% 		\caption{PSNR and SSIM of the resulting posterior mean through VAE and VAE-GAN, test problem 2.}
% 		\label{t1}
% 		\centering
% 		\begin{tabular}{lll}
% 			\toprule
% 			%\cmidrule(r){1-2}
% 			 Model& VAE &VAE-GAN\\
% 			\midrule
% 			PSNR & 10.637803&11.463784\\
% 			SSIM &  0.7434861& 0.75867176  \\
% 			\bottomrule
% 		\end{tabular}
% \end{table}
%right color map
% \begin{table}
% 		\caption{PSNR and SSIM of the resulting posterior mean through VAE and VAE-GAN, test problem 2.}
% 		\label{t1}
% 		\centering
% 		\begin{tabular}{lll}
% 			\toprule
% 			%\cmidrule(r){1-2}
% 			 Model& VAE &VAE-GAN\\
% 			\midrule
% 			PSNR & 0.78&0.74\\
% 			SSIM &  12.39& 10.93  \\
% 			\bottomrule
% 		\end{tabular}
% \end{table}
% \subsection{Test problem 2: $d_{KL}=(800,1800)$}
% \subsubsection{Generative results}
% \begin{figure}
%     \centering
%     \includegraphics[width=0.8\textwidth]{image/prior_data_gaussian_highdim.png}
%     \caption{The randomly sampled realizations in the training dataset $\{y^{(i)}\}_{i=1}^N$, test problem 1 .}
%     \label{priordata_highdim}
% \end{figure}
% \begin{figure}
%     \centering
%     \includegraphics[width=0.8\textwidth]{image/samples_highdim_gaussian_vaegan_64dim.png}
%     \caption{ 64×64 resolution random samples using the decoder $p_{y|x,\theta^*}$, where $x\in \mathbb{R}^{16}$ is sampled from $\mathcal{N} (0, I)$, test problem 1.}
%     \label{samplesVAEGAN_highdim}
% \end{figure}


%\subsubsection{Bayesian inverse results}

% \begin{figure}
%     \centering
%     % \includegraphics[width=1.0\textwidth]{image/darcy_flow_cnn_xia_8_48_1000_10_8_48_36dim_gan.png}
%      \includegraphics[width=1.0\textwidth]{image/highdim_gaussian_inverse_64dim_xia_39012_64obs_gan.png}
%     \caption{The exact log-permeability field with a uniform 64 × 64 grid, the mean and standard deviation of the estimated posterior distribution through VAE-GAN in the first row. Three posterior samples from the posterior in the second row.}
%     \label{inverse_vae_gan_highdim}
% \end{figure}
\section{Conclusions}\label{section_conclude}
We have presented a dimension-reduced KRnet map approach (DR-KRnet) for high-dimensional Bayesian inverse problems, which applies the KRnet to construct an invertible transport map from the prior to the posterior in the low-dimensional latent space of a VAE prior. The key idea of our approach is to employ a deep generative model, called KRnet, to approximate the posterior distribution in the latent space, which allows this approach to incorporate the dimension reduction technique into the Bayesian framework. In this way, the proposed approach can be suitable for practical problems when we only have access to high-dimensional prior data. With the aid of KRnet, our approach can provide an effective and efficient algorithm for both probability approximation and sample generation of posterior distributions. Numerical experiments illustrate that DR-KRnet can solve high-dimensional Bayesian inverse problems. Overall, inference with KRnet maps conducts with greater reliability and efficiency than MCMC, particularly in high-dimensional Bayesian inverse problems. Several promising avenues exist for future work. First, VAE is easy to train and we can couple DR-KRnet with information theory to design new data-driven priors. Second, we can apply our approach to more challenging problems such as petroleum reservoir simulation.


\bigskip
\textbf{Acknowledgments:}
The authors thank Yingzhi Xia for helpful suggestions and
discussions.

\bigskip
\textbf{Funding:}
Y. Feng and Q. Liao are supported by the National Natural Science Foundation of China (No. 12071291), 
the Science and Technology Commission of Shanghai Municipality (No. 20JC1414300), and the Natural Science Foundation of Shanghai (No. 20ZR1436200). K. Tang is supported by the China Postdoctoral Science Foundation under grant 2022M711730,
and X. Wan’s work was supported by the National Science Foundation under grant DMS-1913163.

\begin{appendix}
\section{The neural network architecture of VAE priors}\label{vae_nn}
In section \ref{vae_gan_section}, convolutional neural networks (CNN) are applied to construct the encoder and decoder of VAE priors. Table \ref{vae_ar} presents the neural network architectures of VAE priors, where the dimension of latent variables in test problem 1-2 is $d=36$ and $d=64$, respectively.
\begin{table}[h]
    \centering
    \small
    \caption{The neural network architecture of VAE priors for test problem 1--2.}
     \label{vae_ar}
    \begin{tabular}{|c|c|}
    \hline
       Encoder  & Decoder \\
    \hline
       Input: $y$  & Input: $x$ \\
    \hline
    BatchNormalization&Dense($8*8*48$, activation=`relu')\\
    \hline
    Conv2D(16,2,2,activation=`relu')&Reshape((48, 8, 8))\\
    \hline
    BatchNormalization&BatchNormalization\\
    \hline
    Conv2D(16,3,1,padding=`same',activation=`relu')&Conv2DTranspose(64,3,2,padding=`same',activation=`relu')\\
    \hline
    BatchNormalization&BatchNormalization\\
    \hline
    Conv2D(32,2,2,activation=`relu')&Conv2DTranspose(64,3,1,padding=`same',activation=`relu')\\
    \hline
    BatchNormalization&BatchNormalization\\
    \hline
    Conv2D(32,3,1,padding=`same',activation=`relu')&Conv2DTranspose(32,3,2,padding=`same',activation=`relu')\\
    \hline
    BatchNormalization&BatchNormalization\\
    \hline
    Conv2D(64,2,2,activation=`relu')&Conv2DTranspose(32,3,1,padding=`same',activation=`relu')\\
     \hline
    BatchNormalization&BatchNormalization\\
    \hline
Conv2D(64,3,1,padding=`same',activation=`relu')&Conv2DTranspose(16,3,2,padding=`same',activation=`relu')\\
    \hline 
    Flatten&BatchNormalization\\
    \hline 
Dense($2d$)&Conv2DTranspose(16,3,1,padding=`same',activation=`relu')\\
    \hline
    \multirow{3}{*}{
    Output: $\left(\mu_{en},\log\left(\sigma_{en}^2\right)\right)$}&BatchNormalization\\
    %\hline 
    &Conv2DTranspose(2,3,1,padding=`same')\\
    %\hline 
    &Output: $\left(\mu_{de},\log\left(\sigma_{de}^2\right)\right)$\\
    \hline
    \end{tabular}
   
\end{table}
\section{The neural network architecture of physics-constrained surrogate model}\label{surrogate_nn}
In this paper, we apply convolutional neural networks (CNN) for physics-constrained surrogate model. The neural network architectures of the surrogate are listed in Table \ref{surrogate_archi_table}. For Darcy flows, the equation loss and boundary loss of the loss function \eqref{surrogate_loss} are defined as:
\begin{align}
    {\left\Arrowvert \mathcal{R}\left(\hat{\mathcal{F}}_{\theta}\left(y^{(i)}\right),y^{(i)}\right)\right\Arrowvert}_2^2={\left\Arrowvert \nabla \cdot\tau\left(y^{(i)}\right)-h\right\Arrowvert}_2^2 +  {\left\Arrowvert\tau\left(y^{(i)}\right)+\exp\left(y^{(i)}\right)\nabla u\left(y^{(i)}\right)\right\Arrowvert}_2^2,\\
    {\left\Arrowvert\mathcal{B}\left(\hat{\mathcal{F}}_{\theta}\left(y^{(i)}\right)\right)\right\Arrowvert}_2^2 =  {\left\Arrowvert u\left(y^{(i)}\right)\right\Arrowvert}_2^2 +  {\left\Arrowvert \exp\left(y^{(i)}\right)\nabla u\left(y^{(i)}\right)\cdot \mathbf{n}\right\Arrowvert}_2^2.    
\end{align}
In addition, the weight $\beta$ in  \eqref{surrogate_loss} set to 100 for test problem 1--2.
\begin{table}[H]
    \centering
    \caption{The neural network architecture of PDE surrogate for test problem 1--2.}
     \label{surrogate_archi_table}
    \begin{tabular}{|c|c|}
    \hline
        Networks& Feature maps  \\
    \hline 
    Input: $y$ &$(1,64,64)$\\
    \hline
    Conv2D&(48,2,2,activation=`relu')\\
    \hline 
    Conv2D&(144,3,1,padding=`same',activation=`relu')\\
    \hline 
    Conv2D&(72,2,2,activation=`relu')\\
    \hline 
    Conv2D&(200,3,1,padding=`same',activation=`relu')\\
    \hline 
    UpSampling2D&2\\
    \hline 
    Conv2D&(100,3,1,padding=`same',activation=`relu')\\
    \hline 
    Conv2D&(196,3,1,padding=`same',activation=`relu')\\
    \hline 
    UpSampling2D&2\\
    \hline 
    Conv2D&(3,3,1,padding=`same',activation=`relu')\\
    \hline
    Output: $(u(y),\tau_1,\tau_2)$ &$(3,64,64)$   \\ \hline
    \end{tabular}
\end{table}

% \section{MCMC algotithm}\label{pcn_alg}
% In this paper, we apply the following pCN-MCMC algorithm to sample the posterior of the latent variables, which is the baseline for our DR-KRnet. More details can be found in Algorithm \ref{pcn_compute}.
% \begin{algorithm}[h]
% 	\caption{The pCN-MCMC algorithm with VAE priors}
%  \label{pcn_compute}
% 	\begin{algorithmic}[1]
% 		\Require The likelihood $\pi(\mathcal{D}_{obs}|x)$, pre-trained decoder $p_{y|x,\theta^*}=\mathcal{N}\left(\mu_{de,\theta^*}\left(x\right), \text{diag}\left(\sigma_{de,\theta^*}^{\odot 2}\left(x\right)\right)\right)$, pre-trained surrogate model $\hat{\mathcal{F}}_{\Theta^*}$, chain length $N_{pcn}$, the number of samples from the posterior $N_s$, the hyperparameter $\gamma$.
%             \State Draw a sample $x^{(1)}$ from  $\mathcal{N}(0,\mathbf{I})$.
% 		\For {$i = 1:N_{pcn}$}
% 		\State Given an appropriate step size $\gamma$, propose 
%   $$
%   x^*=\sqrt{1-\gamma^2}x^{(i)}+\gamma \zeta, \quad, \zeta \sim \mathcal{N}(0,\mathbf{I}).
%   $$
% 		\State Compute the acceptance ratio
%   $$
%   \alpha=\text{min}\left(1,\frac{\pi(\mathcal{D}_{obs}|x^*)}{\pi(\mathcal{D}_{obs}|x^{(i)})}\right),
%   $$
%   where the likelihood $\pi(\mathcal{D}_{obs}|\cdot)\approx \pi(\mathcal{D}_{obs}|\mu_{de,\theta^*}(\cdot),\cdot)$ is computed through the decoder and the forward model.
% 		\State Draw $\rho\sim U[0,1]$.
%          \If {$\rho\leq \alpha$}
% 		\State Let $x^{(i+1)}=x^*$.
% 		\Else
% 		\State Let $x^{(i+1)}=x^{(i)}$.
% 		\EndIf
% 		\EndFor
%             \State Posterior samples $y^{(i-N_{pcn}+N_s)}=\mu_{de,\theta^*}\left(x^{(i)}\right)$, for $i=N_{pcn}-N_s+1,\dots,N_{pcn}$ .
% 		\Ensure posterior samples $\{y^{(i)}\}_{i=1}^{N_s}$.
% 	\end{algorithmic}
%\end{algorithm}
\end{appendix}

%\section*{References}

\bibliography{feng}

\end{document}