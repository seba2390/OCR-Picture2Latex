\subsection{Laser calibration system}

The TileCal laser system is installed in the ATLAS main service cavern, USA15, located about 100~m away from the detector. In summary, it consists of a laser source, light guides and beam expanders, an optical filter wheel to adjust the light intensity, and beam splitters to dispatch the light to the Tile Calorimeter PMTs through 400 clear optical fibres, 100 to 120~m long. In addition, the system is equipped with a calibration setup designed to monitor the light in various points of the dispatching chain, and with dedicated control and acquisition electronics boards.

The original Laser~I system used during the LHC Run~1 operation~\cite{bib:laser_run_1} was upgraded during the LHC Long Shutdown~1 to a newer version, referred to as Laser~II, which is used since the beginning of the Run~2 data taking. The main purpose of the laser upgrade was to overcome the shortcomings observed in the previous system while maintaining its precision and stability marks. The essential aspects to improve were the stability of the beam expander distributing light to the 400 clear fibres, the photodiodes grounding, reproducibility of the filter wheel position, the electronics, and the overall estimate of the laser light injected into the PMTs of the calorimeter.


\subsection{Upgraded Laser~II system}

The Laser~II system can be described by six main functional blocks: the optics box, the optical filter patch panel, the photodiode box, the PHOCAL (PHOtodiode CALibration) module, a PMT box and a VME crate featuring the LASCAR (LASer CAlibration Rod) electronics card. These blocks are briefly described below, highlighting the upgrades with respect to the previous Laser~I version.


\begin{figure}[htbp]
\begin{center}
    \includegraphics[width=0.8\textwidth]{figures/optics_box}
    \caption{Scheme of the Laser~II optics box, depicting the internal elements and optical paths.}\label{fig:optics_box}
\end{center}
\end{figure}

\begin{figure}[htbp]
\begin{center}
%    \includegraphics[width=0.45\textwidth]{WP_20141009_16_17_31_Pro}
     \includegraphics[width=0.45\textwidth]{figures/WP_20141009_16_17_31_Pro_enhanced}
    \caption{Picture of the optics box (cover removed) placed on the anti-vibration rails and coupled to the fiber bundle.}\label{pic:optics_box}
\end{center}
\end{figure}

\subsubsection*{Optics box}

The light source of the laser system, installed in the so-called optics box, is a commercial Q-switched diode-pumped solid state laser manufactured by SPECTRA-PHYSICS~\cite{ref:laser}, kept from the predecessor system. The frequency doubler permits the infrared laser to emit 532~nm green light, close to the wavelength of the light coming from the detector WLS fibres, peaked at 480~nm. The time width of the individual pulses generated by the laser is 10~ns. Besides the laser source, the optics box houses also the main optical components acting on the laser beam across the light path, as depicted in Figure~\ref{fig:optics_box}. A picture of the laser box is shown in Figure~\ref{pic:optics_box}.

A beam splitter is located at the output of the laser cavity. It divides the laser primary beam into two parts: a small fraction of the light is sent back to a light mixer and the major part is transmitted through a beam expander and a 45\textdegree{} dielectric mirror to a filter wheel. The light exiting the mixer is collected by five clear optical fibres: two are coupled to the PMT box to the two PMTs responsible for generating the trigger signal for the Laser~II data acquisition (DAQ); and three are connected to the monitoring photodiodes (D0 to D2) located in the photodiode box. 

In the expander, the beam spot is expanded from 700~$\mu$m to 2~mm, reducing the light power density to the forthcoming optical elements. The light reflected by the following mirror passes through a motorised filter wheel hosting eight neutral density filters with varied optical densities, with the filter transmissions ranging between 100\% (no filter) and 0.3\%. The combination of this transmission variation and the range of intensities where the laser operation is stable allows to calibrate the TileCal PMTs in an equivalent cell energy range of 500~MeV to 1~TeV. 

The light transmitted by the selected filter is fed into a light mixer by a beam splitter placed downstream the wheel. Three clear fibres routed to the photodiode box collect the light for monitoring (diodes D3 to D5). A second 45\textdegree{} dielectric mirror reflects the light through a shutter into the final beam expander where the laser light is finally dispatched to the detector by 400 clear fibres. Four fibres route the light output of the expander to the photodiode box to monitor its transmission (photodiodes D6 to D9). 
%The light shutter is typically kept closed during LHC collisions to prevent the detector PMTs to be accidentally illuminated by calibration light in case of failure in the control of the Laser~II system.

The 400 long clear fibres are bundled together and transfer the light coming out of the optics box to the TileCal modules (one fibre for two half modules of the central Long barrel, one fibre for each half module per extended barrel and 16 spare fibres). The association between TileCal PMTs and the clear fibre in the bundle is as follows:
\begin{itemize}
\item Long Barrel: one fibre per full LB module, for even PMT numbers in A side and odd numbers C side. Conversely, another fibre for the same LB module, for odd PMT numbers in A side and even numbers in C side.
\item Extended Barrel: one fibre per module per EB side for the even number PMTs and another one for the odd PMTs.
\end{itemize}
Inside each detector module an optical system composed of a light mixer in air dispatches the light to each PMT with individual clear fibres.

This optics box comprises major upgrades with respect to the previous system. Now, a compact design of the optical layout fully includes all the optical elements in one single box, whereas in the Run~1 system the optical elements were located into two different boxes optically connected with a liquid fiber. The optics box is set in an horizontal position to minimise the dust accretion on optical parts and to ease interventions and is mounted on an anti-vibration system, improving beam stability. The final beam expander is new. It was re-designed to improve uniformity in the distribution of the 2~mm beam spot across the 400 fibres' bundle, which has a circular surface of 30~mm diameter. Finally, the system now permits a better estimate of the laser light injected in the calorimeter through a redundant monitoring of the light transmitted in different points of the optical line with 10 photodiodes.

\subsubsection*{Optical filters}


A patch panel with ten optical filters is used to adjust the intensity of the light read by each of the monitoring photodiodes in the photodiode box. In this set up, each one of the ten optical fibres reading out the light at the various points of the beam path in the optics box (after the laser head, after the filter wheel, and at the output of the beam expander) is coupled to a given optical filter in the patch panel. The optical density of the filters range from 0.5 to 2.5 and are such that for each light point probed there is always at least two filters of equal density, providing a redundant light intensity probe for the monitoring photodiodes.

\subsubsection*{Photodiode box}

The photodiode box is a rack containing a set of ten modules, each composed of a Si PIN photodiode (Hamamatsu S3590-08~\cite{ref:photodiode}) coupled to a pre-amplifier, a control card, and a charge injection card to inject an electrical charge into the ten pre-amplifiers. A set of two fibres is connected to the rear end of the photodiode box, in front of each photodiode. One fibre conveys the laser light for monitoring and is connected from the patch panel with the optical filters. The other one comes from the PHOCAL module, where LED light is injected to assess the stability of the photodiodes. In order to minimise the photodiodes' response dependency on the temperature, the temperature in rack is controlled by the water and fan cooling system. The temperature of each photodiode is monitored and kept constant at approximately 30~$^\circ$C with a long term stability below 1~$^\circ$C.

\subsubsection*{PHOCAL module}

This module implements a redundant internal calibration scheme using a blue LED (Nichia blue NSPB520S $\lambda$=470~nm~\cite{ref:led}) to monitor the ten photodiodes of the Laser~II system. The calibration light is simultaneously transmitted to a reference photodiode (Hamamatsu S2744, active area: 10x20 mm2, spectral response range from 320 to 1100~nm~\cite{ref:bigphoto}) providing the signal for the photodiodes' response normalisation. PHOCAL also contains a radioactive source of $^{241}$Am, releasing mostly $\alpha$ particles of 5.6 MeV with an activity of 3.7~kBq. This source ensures the monitoring of the reference photodiode. This module is an addition with respect to the Laser~I system installed in Run~1, where the existing photodiodes were all monitored with a moveable scheme of the $^{241}$Am source.

\subsubsection*{PMT box}

The PMT box contains two PMTs (Hamamatsu R5900~\cite{ref:pmt}) reading out two optical fibres from the optical box. These provide the trigger signal for the Laser~II acquisition system when the laser is flashing. The PMT box also includes a control module used to drive the shutter and the filter wheel in the optics box.

\subsubsection*{LASCAR electronics}

\begin{figure}
\centering
\includegraphics[height=7cm]{figures/Lascar_photo_black.pdf}\quad
\includegraphics[height=7cm]{figures/LASCAR_Cote.pdf}
\caption{Views of the LASCAR card.}\label{fig:laslascar}
\end{figure}


LASCAR, viewed in Figure~\ref{fig:laslascar}, is the electronics board for the acquisition and control of the Laser~II system. It digitises the analog signals (from the eleven photodiodes, the two PMTs and the charge injection system), contains a chip to retrieve the LHC clock signal, makes the interface to drive the laser and contains the module for charge injection and monitoring of the pre-amplifier and digitisation chain of the photodiodes.

The LASCAR board is housed in a VME crate. It's central brain, a Field Programmable Gate Array (FPGA) Cyclone V manufactured by ALTERA~\cite{ref:altera-cyclone}, controls and provides the interface to the main components:

\begin{itemize}

\item \textbf{Charge ADC (QDC):} 32 channel 14-bit\footnote{One of the bits is used to indicate whether the signal is positive or negative, resulting in an effective 13-bit dynamic range for a maximum integrated charge of 2000~pC.} QDC that performs a 500~ns integration and digitization of the analog input charge signals coming from the eleven photodiodes, the two PMTs and from the charge injection system. Prior to the QDC, the analog signal pass through a charge amplifier circuit with two possible gains ($\times 1$ and $\times 4$).
\item \textbf{LILAS (LInearity LASer) card:} This module is responsible for injecting a known charge into the readout electronics (photodiodes pre-amplifiers and digitisation) to monitor its linearity and stability with time. A digital signal from the FPGA is converted to an electric charge by the LILAS 16-bit DAC. The charge is then injected directly into a QDC channel and distributed to the PHOCAL and Photodiode box through lemo cables.
\item \textbf{Time-to-Digital Converter (TDC)}: A TDC is used to measure the Laser time response as function of its intensity. The device has two channels and a time resolution of 280~ps. LASCAR is equipped with a delay system to insure the adequate laser pulse timing irrespective of laser amplitude.
\item \textbf{Timing, Trigger and Control Receiver (TTCrx)}: The TTCrx is an ASIC chip that receives the LHC signals relative to bunch crossing, event counter reset and trigger.
\item \textbf{HOLA}: The High-speed Optical link for ATLAS was conceived to send data fragments via optical fibre to the Read Out System of ATLAS upon receiving a Level 1 Accept trigger from the ATLAS central DAQ.
\item \textbf{LASER Interface:} This mixed analog and digital board is used to control the laser head. The laser intensity is set through an analog signal (0 to 4~V) and the trigger is set with a TTL signal.
\end{itemize}


\subsection{Operating modes}

The Laser~II system can be operated independently as a stand-alone system or integrated in the ATLAS detector data acquisition framework.

\subsubsection*{Stand-alone operating mode}

The stand-alone operation of the Laser~II allows to verify that the system is responding as expected, to monitor its stability and to perform its internal calibration. In this internal calibration mode, LASCAR controls the Laser~II components, switching the shutter off by default, without sending any laser pulse to TileCal PMTs. The following running modes are possible:

\begin{itemize}
\item \textbf{Pedestal mode:} This mode is used to measure a high number of events when no input signal is injected (from the laser, the LED or the radioactive source).
\item \textbf{Alpha source mode:} This mode is used to measure the response of the reference photodiode in the PHOCAL module to the $\alpha$ particles emitted by the $^{241}$Am source.
\item \textbf{LED mode:} In this mode, the LED signal is transmitted to all the photodiodes, including the reference photodiode, via optical fibres. It allows to probe the stability of the photodiodes used to monitor the laser light.
\item \textbf{Linearity mode:} This mode resources the LILAS card to inject a known electrical charge into the preamplifiers of the photodiodes in order to assess the stability of the electronics. It also allows to vary the injected charge to evaluate the linearity of the readout electronics.
\item \textbf{Laser mode:} In this mode, the laser signals of adjustable intensity are sent in the system. The light can be transmitted to the TileCal PMTs, depending on the status of the shutter located inside the optics box. %This mode was mainly used during the Laser~II commissioning and for debugging purposes.
\end{itemize}

A standard internal calibration run combines all the above running modes, starting with the pedestal mode. Once enough pedestal events have been recorded, LASCAR is switched to the next calibration mode, starting from the alpha mode, then the LED mode and the linearity mode increasing the injected charge from 0 to 60000 DAC counts ($\sim$1.9~pC) by steps of 10000 ($\sim$0.3~pC). Finally, the internal calibration ends with the laser mode.
% David Calvet: The maximum voltage of the DAC is 4.096 V and the injection capacitor is 0.5 pF. Thus, the maximum charge (for DAC=65535) should be close to 2.05 pC.

\subsubsection*{ATLAS DAQ operating mode}

The ATLAS DAQ mode is the main operating mode of Laser~II. Its role is to calibrate the TileCal PMTs with laser light. To do so, the Laser~II DAQ is integrated within the global ATLAS DAQ infrastructure, which handles the readout of the PMT signal induced by the laser and the Laser~II run control. This mode is used in two ways:

\begin{itemize}
\item \textbf{Laser mode:} This is the main mode to perform the dedicated calibration runs, when the TileCal is operated independently of the remaining ATLAS detector. 
%Laser pulses are sent to the calorimeter by request of the ATLAS DAQ through the SHAre Few Trigger (SHAFT) interface which forwards the request to LASCAR. 
Laser pulses are sent to the calorimeter by request of the SHAre Few Trigger board (SHAFT) to LASCAR. 
Amplitudes of the signals produced by the photodiodes and the PMTs of the Laser~II system are sent back to the ATLAS DAQ by LASCAR. At the start of run, the filter wheel position and the laser intensity are configured and the shutter opened.
\item \textbf{Laser-in-gap mode:} Laser pulses are emitted in empty bunch-crossings during standard physics runs of the LHC. The TileCal is synchronised with the other ATLAS sub-detectors. The light is fired only in exclusive periods of the beam orbit, where no collisions can occur. The SHAFT board sends a request to LASCAR at a fixed time with respect to the beginning of the LHC orbit to synchronise the laser pulses with the pre-defined orbits and ensure no overlap between laser and physics events. Upon pulse emission, a laser calibration request is sent to the ATLAS central trigger processor by the SHAFT interface. This arrangement synchronises the pulse emission and the TileCal readout with ATLAS DAQ in physics runs.

\end{itemize}

\subsection{Stability of the laser system}

The installation of Laser~II involved a commissioning phase where the performance and the stability of the system were evaluated in the course of the first three months of Run~2. The main parameters to monitor are the ones obtained with the operation of the laser internal calibration mode. The measurements included the pedestal of the photodiodes, the response of the electronics to a known injected charge, the signal of the monitoring photodiodes in response to the PHOCAL LED pulses, and the response of the PHOCAL photodiode to the $^{241}$Am $\alpha$-source. The results are shown in Figure~\ref{fig:laser_stability}. 


\begin{figure}[htbp]
\begin{center}
\subfloat[Pedestal\label{fig:pedestal}]{\includegraphics[height=0.28\textwidth]{figures/LaserII_Ped_Stability_HG}}
\subfloat[Charge Injection\label{fig:charge_injection}]{\includegraphics[height=0.28\textwidth]{figures/LaserII_CIS_Stability_LG_20k}}\\
\subfloat[LED\label{fig:led}]{\includegraphics[height=0.28\textwidth]{figures/LaserII_Led_Stability_HG}}
\subfloat[$^{241}$Am $\alpha-$source\label{fig:alpha}]{\includegraphics[height=0.28\textwidth]{figures/LaserII_Alpha_Stability}}
    \caption{Relative (a) pedestal and (b) charge injection signal for each photodiode (D0 to D9 and reference photodiode in PHOCAL) as a function of time in the course of the first three months of Run~2. The mean signal values are normalised to the mean signal value of the first measurement. (c) Relative response to the PHOCAL LED for each photodiode (D0 to D9) as a function of time. Signal values are normalised to the PHOCAL photodiode signal and to the mean signal value of the first measurement. (d) Relative response of the PHOCAL photodiode to the $^{241}$Am $\alpha$-source subtracting a constant pedestal or correcting the pedestal for the observed fluctuation in time. The mean signal values are normalised to the mean signal value of the first measurement.}\label{fig:laser_stability}
\end{center}
\end{figure}

Figure~\ref{fig:pedestal} shows the average pedestal value recorded in the Laser II stand-alone acquisition mode in high gain for the monitoring photodiodes (D0 to D9) and for the PHOCAL photodiode. The data is normalised relatively to the first data point. The pedestals of the D0--D9 photodiodes are stable within 0.8\% during the commissioning period, whereas for the PHOCAL diode a maximum fluctuation of 1.8\% is observed. 

The stability of the readout electronics response is obtained by injecting a constant charge of 171~pC, 256~pC or 342~pC in several runs across the considered time period. For each injected charge, the signal is acquired in low gain and in high gain. Figure~\ref{fig:charge_injection} shows the results obtained for a 256~pC injected charge in low gain readout. The data, normalised to the first day of data taking, are shown for the electronics channels corresponding to each photodiode and the pedestals are subtracted. All channels exhibit a consistent up-drift reaching 0.8\% at the most in the end of the data taking period. 

In Figure~\ref{fig:led}, the outcome of the photodiode monitoring with the PHOCAL LED in stand-alone high gain runs of Laser~II is presented. The values are normalised to the first data point and the pedestals are subtracted. The response of the photodiodes to the calibration light is very stable in time. The maximum fluctuations do not overcome 0.4\% and do not exhibit any particular trend with time.

Finally, the PHOCAL response to the $^{241}$Am internal $\alpha$-source is displayed in Figure~\ref{fig:alpha} for the low and high gain signal acquisition mode. The response is normalised to the first data point and the pedestals are subtracted. Figure~\ref{fig:alpha} shows a consistent down-drift for the two gain modes, that reach $-0.8$\% in the end of the period under analysis. Given the larger pedestal variation observed for the PHOCAL photodiode, seen in Figure~\ref{fig:pedestal}, the pedestals are further corrected for this fluctuation. The correction has a substantial effect on the obtained photodiode response since the signal induced by the radioactive source (around 600 and 2500 ADC counts in low and high gain, respectively) is just three to six times larger than the pedestal values (around 100 and 750 ADC counts in low and high gain, respectively). Figure~\ref{fig:alpha} also shows the corrected responses, exhibiting a maximum relative variation of about 0.4\%.

The effects of fluctuations of the light monitoring system are taken into account in the calibration of the TileCal PMTs with a run-by-run correction factor. This will be described in Section~\ref{sec:calibration}.


%Material:
%\begin{itemize}
%\item A. Blanco, D. Calvet et al, Upgrade of the Laser calibration system of the ATLAS Tile Calorimeter, ATL-TILECAL-INT-2016-004, https://cds.cern.ch/record/2217913
%\item P. Gris, LASERII performance plots, ATL-COM-TILECAL-2015-071, https://cds.cern.ch/record/2058183
%\item F. Scuri, Performance of the ATLAS Tile LaserII calibration system, https://inspirehep.net/literature/1498841
%\item P. Gris, F. Scuri et al, Performance of the upgraded Laser calibration system of the ATLAS Tile Calorimeter, https://cds.cern.ch/record/2290200 
%\end{itemize}
