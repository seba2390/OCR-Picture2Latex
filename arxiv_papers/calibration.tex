%Material
%\begin{itemize}
%\item P. Klimek et al, ATLAS Tile Calorimeter calibration with the Laser system during LHC Run-2, https://cds.cern.ch/record/2721936
%\end{itemize}

\subsection{Calibration procedure}

%\begin{itemize}
%\item Runs we take
%\item relative calibration
%\item constants update during data taking
%\item IOV concept
%\item reprocessing
%\end{itemize}

%..... The laser system is employed to perform the PMT response calibration relative to the previous global calibration of the TileCal detector with the caesium scan. Thus, to determine the laser calibration constants, a laser run taken close to the global calibration day is used to set the reference signals for each PMT. ......

As can be seen in Equation~\ref{eq:channelEnergy}, the reconstruction of the energy in TileCal depends on several constants, some of them being updated regularly. The main calibration of the TileCal energy scale is obtained using the caesium system~\cite{Blanchot:2020lyh}. However, since a caesium scan needs a pause in the $pp$ collisions of at least six hours, this calibration cannot be performed very often. Therefore, regular relative calibrations are accomplished between two caesium scans using the laser system. Moreover, during the LHC technical stop at the beginning of data taking period in 2016, few liquid traces coming from the caesium hydraulic system were found in the detector cavern. Since then until the end of Run~2, caesium scans were restricted to be taken only during the end of year technical stops, due to risk of the leak. In absence of the caesium calibration, the laser became the main calibration system, calibrating the PMTs and readout electronics. In order to address the fast drift of PMT response caused by the large instantaneous luminosity, the laser calibration constants were updated every 1--2 weeks, since July 2016. These constants were used in so-called prompt data processing, performed during the data taking period. 

Each year, the data recorded by the ATLAS detector is reprocessed. Data reprocessing consists of the update of the physics dataset (proton--proton and heavy ion collision runs) with updated conditions and calibration constants. Moreover, a reprocessing of the full Run~2 dataset was performed during LHC Long Shutdown 2 at the end of Run~2. This step is necessary to apply new reconstruction and calibration algorithms as well as the corrections that were impossible to be done or missed during prompt data processing. The IOVs are readjusted and chosen to coincide with the data taking periods. For laser calibration, they occur every 1--2 weeks in order to smoothly follow the evolution of PMTs response during the data taking period. 

The method to compute the laser constant $f_{\mathrm{Las}}$ introduced in Equation~\ref{eq:channelEnergy} is based on the analysis of specific laser calibration runs, taken daily during the data taking period, for which both the laser system photodiodes and the TileCal PMTs are read out. The laser calibration employs two types of successive laser runs:
\begin{itemize}
  \item Low Gain run (labeled as LG) consists of $\sim$10,000 pulses with a constant amplitude and the filter attenuation factor equal to 3,
  \item High Gain run (labeled as HG) consists of $\sim$20,000 pulses with a constant amplitude and the filter attenuation factor equal to 330.
\end{itemize}

The laser system is employed to perform the PMT response calibration relative to the previous global calibration of the TileCal detector with the caesium scan. Thus, to determine the laser calibration constants, a laser run taken close to the caesium scan is used to set the reference signals for each PMT. 
%The laser calibration is a relative calibration with respect to a laser reference run taken right after each caesium scan. 
By definition, if the response of a channel to a given laser intensity is stable (the response of the PMT and of the associated readout electronics are stable), the laser constant $f_{\mathrm{Las}}$ is 1. The references were set close to the start of each year's $pp$ collision runs. 
%The following references were used during LHC Run~2, for both prompt processing and reprocessing of data:
%\begin{itemize}
%	\item 2015 - LG: 271880, HG: 271882 (2015-07-17), IOV: 273795 (2015-07-27)
%	\item 2016 - LG: 294144, HG: 294145 (2016-04-01)
%	\item 2017 - LG: 317215, HG: 317216 (2017-03-06)
%	\item 2018 - 
%	\begin{itemize}
%		\item proton-proton runs: LG: 344221, HG: 344222 (2018-02-15)
%		\item heavy ion runs: LG: 364531, HG: 364533 (2018-10-27)
%	\end{itemize}
%\end{itemize}
%The references for the PMT gain $G{_i^{\mathrm{ref}}}$ from Equation~\ref{eq:gain_ratio}, used in the Combined method, significantly fluctuate run by run. The fluctuations are caused by the laser light instability and statistical uncertainty. Therefore, they are averaged over the laser runs taken within $\pm 10$ days with respect to the reference runs. The laser runs used for averaging were taken in the period before the physics collision started. 
The laser references and laser constants are stored in the conditions database. 
%in the following folders:
%\begin{itemize}
%	\item Laser references: \texttt{/TILE/OFL02/CALIB/CES}
%	\item Laser constants: \texttt{/TILE/OFL02/CALIB/LAS/LIN}
%\end{itemize}

The laser calibration procedure evolved during Run~2. Due to increasing instantaneous luminosity and response variation observed in all PMTs, the methods to derive laser constants were adapted. The applied methods are described in detail in Section~\ref{sec:determination_of_the_calibration_constants}. 
%In this section, the description of the procedure to produce the laser calibration constants $f_{\mathrm{Las}}$ using these methods for prompt processing and reprocessing of data 2015--2018 is presented. 


\subsection{Determination of the calibration constants}
\label{sec:determination_of_the_calibration_constants}

The laser runs are constituted by a set of laser pulses with corresponding signal readout from the individual PMTs, from which the pedestal is subtracted. 
%which is subtracted from pedestal at a pulse-by-pulse basis. 
For each pulse, the normalised response of a PMT channel, the ratio $R_{i,p}$, is defined as:

\begin{equation}
    R_{i,p} = \frac{A_{i,p}}{A_{\mathrm{D6},{p}} }
    \label{eq:Rip}
\end{equation}

where $p$ denotes the pulse, $A_i$ is the reconstructed signal amplitude of the PMT readout channel $i$ and $A_{\mathrm{D6},{p}}$ is the signal amplitude measured by the photodiode 6 (D6) in the laser box. The D6 measures the laser light after the beam expander and probes the beam close to the TileCal PMTs in the best dynamic range among available photodiodes D6--D9. The average of the ratio $R_{i,p}$ over all pulses of the laser run, denoted as $R_i\equiv \langle R_{i,p}\rangle$, is analysed for each PMT.

The laser calibration factors employed to reconstruct the cell energy, in Equation~\ref{eq:channelEnergy}, are simply the relative response of the channel:

\begin{equation}
    f_{\mathrm{Las}}^i = \frac{R_i}{R_i^{\mathrm{ref}}}
    \label{eq:fLaser}
\end{equation}

where $R_i^{\mathrm{ref}}$ is the normalised response of the PMT channel $i$ during the laser reference run. For monitoring purposes, these factors are usualy presented in percentage as a relative response variation:

\begin{equation}
    (f_{\mathrm{Las}}^i - 1)\times 100\;[\%]
\label{eq:PMTdriftCorrected}
\end{equation}

The measurement of $f_{\mathrm{Las}}^i$ may be influenced by instabilities with origin at the laser system itself, both at a global level, i.e. affecting equally all the detector PMTs, or at the fibre level, i.e. affecting the set of PMTs associated with each clear fibre. To take these effects into account, global and fibre corrections are determined, such that the corrected laser constant reads as:

\begin{equation}
    f_{\mathrm{Las}}^i \to f_{\mathrm{Las}}^i \times \frac{1}{\alpha_{\mathrm{G}} \times \alpha_{\mathrm{f}(i)}}
    \label{eq:fLaserOpticsCorrections}
\end{equation}

\begin{itemize}                                                                               
\item The global correction $\alpha_{\mathrm{G}}$ is associated with a coherent drift of all channels. The effect can be related to an instability of the reference diode, from the variation of light received by the TileCal PMTs or common ageing of the long fibres.
                                                                                       
\item The fibre correction $\alpha_{\mathrm{f}(i)}$, computed per fibre $\mathrm{f}(i)$, is associated with a time variation of the light transmission from fibre to fibre.
\end{itemize}                                                                                

During Run~2, two methods were used to evaluate these optics corrections: the so-called Direct and Combined methods. In the Direct method, the global and fibre corrections are simply determined from the average response variations of a set of stable PMTs reading outermost and least irradiated cells in the D layer used as references, and the sub-set of D-layer PMTs associated with the fibre, respectively. 
%PMTs reading outermost cells were used, since those cells were least irradiated and expected to be stable. 
This method was used to calibrate and monitor the detector response during 2015--2017 data taking but revealed to be inadequate for calibration when the response of the reference PMTs started to fluctuate due to larger integrated currents in the middle of the 2017 run. Then the Combined method was developed and employed in the 2018 TileCal calibration and also for the reprocessing of 2017 data. Instead of relying on the stability of a set of reference PMTs, the gain of the PMT is explicitly evaluated to determine the optics corrections by the Combined method.


\subsubsection*{Direct method}

In the Direct method, the global correction is evaluated from the relative response of all PMTs reading cells in the D-layer:

\begin{equation}
  \alpha_{\mathrm{G}} = \bigg\langle \frac{R_i}{R_i^{\mathrm{ref}}} \bigg\rangle^{\mathrm{D-cells}}
  \label{eq:globalCorrectionDM}
\end{equation}

The fibre corrections are evaluated using information from PMTs of the D layer for the fibres associated to the LB, and from PMTs reading the D, B13, B14 and B15 cells for the EB\footnote{These cells are less exposed to particle fluence, so their readout PMTs experience smaller integrated currents and a more stable response.}, corrected from global effects. This quantity is evaluated for each long clear fibre $\mathrm{f}(i)$ as 

\begin{equation}
  \alpha_{\mathrm{f}(i)} = \frac{1}{\alpha_{\mathrm{G}}} \bigg\langle \frac{R_i}{R_i^{\mathrm{ref}}} \bigg\rangle^{\mathrm{D,B-cells}}_{\mathrm{f}(i)}
  \label{eq:fibreCorrectionDM}
\end{equation}

In Equations~\ref{eq:globalCorrectionDM} and~\ref{eq:fibreCorrectionDM}, $\langle\;\rangle$ represents a geometric weighted average, where the weight is proportional to the number of laser pulses in the run, and the average RMS of the PMT signals.

Saturated channels, channels with bad status in the TileCal condition database, and channels for which the absolute difference between the applied and requested HV is above 
%HV$_{\mathrm{set}}$ and the actual PMT HV is above 
10~V ($\Delta \mathrm{HV}>$10~V) are excluded from the computation of the optics corrections. Moreover, an iterative procedure rejects outlier channels, more than $3\sigma$ apart from the $R_i/R_i^{\mathrm{ref}}$ distribution average.


\subsubsection*{Combined method}

In the Combined method, the actual PMT gain is measured based on the statistical nature of photoelectron production and multiplication inside the PMT. It assumes that the noise is negligible with respect to the laser-induced PMT signals and that the laser light is coherent. Under these conditions, the two main contributions to the PMT signal fluctuations to the laser scans are the poissonian fluctuations in the photoelectron emission spectrum and multiplication, and the variation of the intensity of the light source~\cite{Bures:74}. The PMT gain $G$ can be written as:

\begin{equation}
	G = \frac{1}{f \cdot e}\cdot \left( \frac{\mathrm{Var}[q]}{\langle q\rangle} - k \cdot \langle q\rangle \right)
	\label{eq:gain_statistical_method}
\end{equation}

where $e=1.6\times 10^{-19}$~C is the electron charge constant, $f$ stands for the excess noise factor extracted from the known gain of the individual PMT dynodes~\cite{Arisaka:2000id}.  For the eight dynode TileCal PMTs, $f=1.3$ at the nominal gain of $G=10^5$, $\langle q \rangle$ is the average value of PMT anode charge associated to each laser pulse, and $\mathrm{Var}[q]$ is the variance of the anode charge distribution. 
%$\langle q \rangle$ is the average anode charge and $\mathrm{Var}[q]$ the variance. 
The coherence factor $k = \frac{\mathrm{Var}[I]}{\langle I\rangle ^2}$ depends on the characteristics of the light source itself but not on the light intensity. The factor $k$ ranges from 0, for an ideal fully coherent light source, to 1, for a totally incoherent light source, and is determined with a set of PMTs measuring the same light source. For any PMT pair $i$ and $j$, $k$ is given by the average PMT measured charge $q_i$ and $q_j$ respectively, and the covariance $\mathrm{Cov}[q_i, q_j]$ of the charge measurements, as:

\begin{equation}
	k = \frac{\mathrm{Cov}[q_i, q_j]}{\langle q_i\rangle \langle q_j\rangle}
	\label{eq:PMT_excess_noise_factor}
\end{equation}

In order to decrease the dependence of the gain measurement on the $k$ factor determination, to which the sensitivity is more limited, the PMT gain is analysed in high gain laser calibration runs taken with filter wheel in position~8 with 31.6\% transmission (optical density of 2.5). For these runs the light intensity is lower leading also to a lower average PMT anode charge $\langle q \rangle$, thus the $k$ term in Equation~\ref{eq:gain_statistical_method} has a smaller effect on the gain measurement.

Moreover, since the PMT gain determination presents significant fluctuations, the average over a set of runs within $\pm$10 days around the laser reference run is taken to set the PMT reference gain, $G_i^{\mathrm{ref}}$. The PMT gain $G_i$ is used as an independent measure of the PMT signal and as the basis to evaluate the optics corrections by the Combined method. The global correction is determined from the average ratio between the PMT relative response and the PMT relative gain using PMTs reading the D-layer and the BC1, BC2, B13, B14, B15 cells:


\begin{equation}
    \alpha_{\mathrm{G}} = \bigg\langle \frac{R_i}{R_i^{\mathrm{ref}}} \Big/ \frac{G_i}{G_i^{\mathrm{ref}}} \bigg\rangle^{\mathrm{D,B-cells}}
    \label{eq:global_combined}
\end{equation}

The fibre corrections are determined in approximately the same way as the global correction except that the average runs over all the channels connected to a common long fibre $\mathrm{f}(i)$, with the global correction taken into account to avoid double correcting:

\begin{equation}
    \alpha_{\mathrm{f}(i)} = \frac{1}{\alpha_{\mathrm{G}}} \bigg\langle \frac{R_i}{R_i^{\mathrm{ref}}} \Big/ \frac{G_i}{G_i^{\mathrm{ref}}} \bigg\rangle_{\mathrm{f}(i)}
    \label{eq:fibre_combined}
\end{equation}

As for the Direct method, the PMTs having bad status or with $\Delta \mathrm{HV}>$10~V or saturated channels are discarded from analysis.


\subsection{Evolution of the optics corrections}

\begin{figure}[htbp]
\centering
\subfloat[\label{fig:optics_correction_2018_a}]{\includegraphics[width=0.5\linewidth]{figures/global_history}}
\subfloat[\label{fig:optics_correction_2018_b}]{\includegraphics[width=0.5\linewidth]{figures/LB10C_fiber_history}}
\caption{Evolution of the (a) global correction and (b) LB10C fibre correction associated with even/odd numbered PMTs in LBA10/LBC10 over time in 2018. The corrections are determined using laser high gain runs with the Combined method and are calculated as a weighted geometric mean. The corresponding errors are included in the data points.}
\label{fig:optics_correction_2018}
\end{figure}


Figure~\ref{fig:optics_correction_2018} shows the time evolution in 2018 of the global correction and the LB10C fibre correction (associated with the even/odd numbered PMTs in LBA10/LBC10), both shown in percentage and determined with the Combined method using laser runs taken in high gain. The global correction in 2018 is stable in time within 1\% and the correction is about the same order. During Run 2, the magnitude of this correction did not exceed 2.5\%. The fibre correction shown is generally representative of the 384 clear fibres in total. For all the years, the magnitude of the corrections did not exceed 1\% and was also found to be constant throughout the time.

The global correction dominates the scale of the PMT calibration. Its precision should match the global scale uncertainty on the PMT calibration assessed from laser and caesium comparisons presented in Section~\ref{sec:CsLas}, and thus be better than 0.4\%. The accuracy on the global correction was further assessed using two symmetric sets of PMTs, one composed of PMTs reading the TileCal A side and another with PMTs installed in the C side to derive independent corrections. The corrections obtained for the A and C sides matched well below the sub-percent level for all years in Run~2, attesting the robustness of the Combined method at disentangling the effects of fluctuations in the monitored light intensity common to all PMTs.


\subsection{Comparison with caesium calibration}
\label{sec:CsLas}

The response variation of PMTs measured with the laser system should match the full detector response variation obtained with the caesium system within short periods of time, where fluctuations from the scintillators and WLS fibers can be safely neglected. Thus, the comparison between the laser and caesium measurements constitute a procedure to validate the laser algorithm itself, employed to validate the Combined method.

During 2015 and 2016, three periods of low integrated luminosity were available within consecutive caesium scans. Figure~\ref{fig:CsLas2015Nov} shows the response variation between July 17 and November 3, 2015, obtained with the caesium system as a function of the response variation obtained with the laser. The results are displayed at channel level and separating channels per layer A, B/BC and D. The great majority of channels have the same response variation for laser and for caesium.

\begin{figure}[htbp]
\centering
\subfloat[\label{fig:CsLas2015Nov_a}]{\includegraphics[width=0.51\linewidth]{figures/y2015_iovII_CsLas_COLZ}}
\subfloat[\label{fig:CsLas2015Nov_b}]{\includegraphics[width=0.49\linewidth]{figures/y2015_iovII_CsLas_SCAT}}
\caption{Response variation (in \%) measured by caesium (y-axis) and by laser employing the Combined method (x-axis) between July 17 and November 3, 2015 for (a) all TileCal channels and (b) the channels in the A-, B/BC- and D-layers. Special channels not calibrated by the caesium system, such as the E-cells, are not included.
%  , i.e. all C10-, E-cells, and MBTS are not included (right). 
% The points visible on the left plot but missing on the right one correspond to the regular C10-cells.
%  (left) for each TileCal channel and (right) for each layer.
}
\label{fig:CsLas2015Nov}
\end{figure}

The corresponding distribution of the ratio between the caesium constants ($f_\mathrm{Cs}$) and the laser calibration constants ($f_\mathrm{Las}$), calculated to address the response variation of the PMTs during the same period of time, 
is shown in Figure~\ref{fig:CsLas2015Nov_1d} 
% for the TileCal channels 
separated by layer and Long/Extended barrel. 
%The calibration constants are specifically set to correct for the response evolution occurred exclusively during the period of time being evaluated. 
Each distribution is fitted with a Gaussian function to measure its average and standard deviation. The differences observed between the caesium and the laser systems are more evident in the extended barrel and on the A layer. These regions of the calorimeter are less shielded and thus the effects of radiation damage to scintillator and WLS fibre are faster. The average difference is well bellow 0.1\% and the standard deviation is 0.6\%.

For the three periods analysed, the maximum average difference observed was 0.4\%. This value is taken as the uncertainty on the scale of the PMT calibration with laser.

\begin{figure}[htbp]
\centering
\subfloat[\label{fig:CsLas2015Nov_1d_a}]{\includegraphics[width=0.5\linewidth]{figures/y2015_iovII_CsLas_Ratio_LB}}
\subfloat[\label{fig:CsLas2015Nov_1d_b}]{\includegraphics[width=0.5\linewidth]{figures/y2015_iovII_CsLas_Ratio_EB}}
\caption{Ratio between the caesium calibration constants ($f_\mathrm{Cs}$) and the laser calibration constants calculated with Combined method ($f_\mathrm{Las}$) for channels in Layer A, B/BC and D in the (a) Long Barrel and (b) Extended Barrel. Special channels not calibrated by the caesium system, such as the E-cells, are not included.}
\label{fig:CsLas2015Nov_1d}
\end{figure}

\subsection{Uncertainties on the PMT calibration}

Besides the systematic uncertainty on the PMT calibration scale, the uncertainty on the PMT relative inter-calibration, mostly sourced at the fibre correction procedure and at the channel-level readout, is evaluated. To do so, an indirect comparison between the responses to caesium source and laser, measured with left and right PMTs reading the same cell, is performed evaluating the following observable: 

\begin{equation}
\Delta f^{\mathrm{L-R}}_{\mathrm{Cs/Las}}=\left( \frac{f^{\mathrm{L}}_{\mathrm{Cs}}}{f^{\mathrm{L}}_{\mathrm{Las}}}- 
\frac{f^{\mathrm{R}}_{\mathrm{Cs}}}{f^{\mathrm{R}}_{\mathrm{Las}}} \right)
\label{eq:sys_Cs-Laser}
\end{equation}

where $f^{\mathrm{L(R)}}_{\mathrm{Las}}$ and $f^{\mathrm{L(R)}}_{\mathrm{Cs}}$ are the calibration constants corresponding to the cell relative response to laser and caesium source measured by the left (right) channel. With this quantity, the scintillator effects common to both left/right readouts are cancelled out. Assuming that the WLS fibre response from the left and right sides of the cell has a similar behaviour, the width of the distribution of $\Delta f^{\mathrm{L-R}}_{\mathrm{Cs/Las}}$ is driven by the uncertainties of the laser measurement and caesium measurements. The inter-calibration systematic uncertainty on the laser calibration was then determined by disentangling the contributions from the caesium uncertainty and constraining with measurements of $f^{\mathrm{L}}_{\mathrm{Cs}}-f^{\mathrm{R}}_{\mathrm{Cs}}$ and $f^{\mathrm{L}}_{\mathrm{Las}}-f^{\mathrm{R}}_{\mathrm{Las}}$. The results obtained with 2018 data are shown in Figure~\ref{fig:sigma_Las}. A dependence of the systematic uncertainty on the integrated luminosity, more pronounced for the extended barrel, is observed. The effect is due to a correlation between the integrated PMT charge and the response down-drift, with consequent increase in the spread of the response for a given PMT sample.

\begin{figure}[htbp]
\centering
\includegraphics[width=0.5\textwidth]{figures/Uncertainties_PMT_response}
\caption{Uncertainties on the PMT inter-calibration with the Laser~II system using the Combined method as a function of the integrated luminosity for the Long Barrel and for the Extended Barrel. The results are obtained using laser and caesium calibration data collected in 2018. The two points at 63.3~fb$^{-1}$ result from two successive caesium scans without LHC beam. The uncertainty is parametrised as a function of the luminosity by fitting the data points with a linear function. A global scale systematic resulting from direct comparison between laser and caesium data was found to be 0.4\%. This value should be summed in quadrature to obtain the total laser uncertainty. In 2018, three caesium scans were performed in LB (red points) and four in EB (blue points).}
\label{fig:sigma_Las}
\end{figure}

The total uncertainty on the laser calibration of a PMT, corresponding to the quadratic sum of the 0.4\% scale systematic and the luminosity-dependent inter-calibration uncertainty, in the Long Barrel ($\sigma_{\mathrm{Las,tot}}^{\mathrm{LB}}$) and in the Extended Barrel ($\sigma_{\mathrm{Las,tot}}^{\mathrm{EB}}$) yields:

\begin{equation}
\begin{aligned}
\sigma_{\mathrm{Las,tot}}^{LB} [\%] =0.4\oplus(0.3+0.0016\times L)\; [\%] \\
\sigma_{\mathrm{Las,tot}}^{EB} [\%] =0.4\oplus(0.3+0.0032\times L)\; [\%]
\end{aligned}
\end{equation}

\subsection{Overview of the PMT response}
The laser system is used to measure the evolution of the PMT response as a function of time. 
%In this Section, the average response variation of the PMTs as a function of time during the LHC Run~2 is presented. 
The Combined method, discussed in Section~\ref{sec:determination_of_the_calibration_constants}, is utilised to calculate the response variation with respect to a set of reference runs. In particular, the Equation~\ref{eq:PMTdriftCorrected} is used to obtain the response variation for each PMT. 
Channels marked with bad data quality status, unstable high voltage or flagged as problematic by any calibration system are discarded. For each cell type, the average response is obtained by a Gaussian fit to the distribution of PMT response variation. The $\chi ^2$ fit method is applied. The Gaussian approximation is used in order to obtain the average variation that is not affected by outliers. 

A sample of the mean response variation in the PMTs for each cell type averaged over $\phi$, measured with the laser system during the entire $pp$ collisions data-taking period in 2018, is shown in Figure~\ref{fig:map_2018}. The most affected cells are those located at the inner radius and in the gap and crack region with down-drift up to 4.5\% and 6\%, respectively. Those cells are the most irradiated and their readout PMTs experience the largest anode current. 

\begin{figure}[htbp]
\centering
    \includegraphics[width=1.0\textwidth]{figures/tile_laser_map364147}
    \caption{The mean response variation in the PMTs for each cell type, averaged over $\phi$, observed during the entire $pp$ collisions data-taking period in 2018 (between laser calibration runs taken on 18 April 2018 and 22 October 2018) calculated using the Combined method. For each cell type, the response variation is defined as the mean of a Gaussian fit to the response variations in the channels associated with given cell type. A total of 64 modules in $\phi$ were used for each cell type, with the exclusion of known pathological channels.
    %The average PMT response variation (in \%) per TileCal cell type as a function of $|\eta|$ and radius, observed during the entire high $\langle \mu \rangle$ proton-proton collisions data taking period in 2018 (between April 18 and October 22, 2018) calculated using the Combined method.
    }\label{fig:map_2018}
\end{figure}

Figure~\ref{fig:phimap_2018} shows the average response variation of the channels per layer and along the azimuthal angle $\phi$ for the same period in 2018. Each $\phi$ bin corresponds to one LB/EB module averaged over the A and C sides. Channels with low signal amplitude, bad data quality status or unstable high voltage are discarded in the average response calculation. It can be seen that PMTs reading the cells in layers closest to the beam axis, composed of A cells, are the most affected. Next layers, formed of the BC and D cells are significantly less affected. We observe larger uniformity across the modules in $\phi$ in layers with a larger number of channels (eg. 40 channels in the LB A layer, see Figure~\ref{fig:map_2018}), where the effect of discarding one bad quality channel has less impact. On the other hand, layers for which the spread in the response of the channels is larger (eg. EB layer A against LB layer A, see Figure~\ref{fig:map_2018}) are more affected in the $\phi$ uniformity with bad channel removal.
%These Figures provide useful insight about the uniformity of the PMT variation over $\phi$.

\begin{figure}[htbp]
\centering
    \subfloat[\label{fig:phimap_2018_a}]{\includegraphics[width=0.495\linewidth]{figures/364147_tile_laser_lb_map_phi_LasPaper}}\hfill
    \subfloat[\label{fig:phimap_2018_b}]{\includegraphics[width=0.495\linewidth]{figures/364147_tile_laser_eb_map_phi_LasPaper}}
    \caption{The mean response variation in the PMTs for each cell type, averaged over $\eta$, observed during the entire $pp$ collisions data-taking period in 2018 (between laser calibration runs taken on 18 April 2018 and 22 October 2018) in LB (a) and EB (b), calculated using the Combined method. For each cell type, the response variation is defined as the mean of a Gaussian fit to the response variations in the channels associated with given cell type. Known pathological channels were excluded.
   % The average PMT response variation (in \%) as a function of polar angle $\phi$ and layers for long barrel (left) and extended barrel (right), observed during the entire high $\langle \mu \rangle$ proton-proton collisions data taking period in 2017 (between April 18 and October 22, 2018). The plot is made using the Combined method.
    }\label{fig:phimap_2018}
\end{figure}

Figure~\ref{fig:LaserDrift_run2_a} shows the time evolution of the mean response variation in the PMTs for each layer observed during the entire Run~2. The PMT response variation strongly depends on the delivered luminosity by the LHC. Therefore, the delivered luminosity is also shown for comparison. The observed PMTs response variation is the result of three competing factors: i) the constant up-drift observed when PMTs are in rest; ii) the down-drift during high instantaneous luminosity period when PMTs are under stress; iii) the fast partial recovery after stress observed during technical stops. These effects result in 6\% accumulated mean response variation in the PMTs for the cells located at the inner layer at the end of Run~2. For the B/BC and D layers, the average PMT response degradation during $pp$ collisions was almost totally recovered in technical stops, resulting in $-1.5$\% accumulated PMT response variation at the end of Run~2 for the layer~B/BC and even in +0.5\% balance for the layer~D. 
%PMTs reading the layer~A exhibit larger response gradients and reach the end of the Run~2 with an average $-5.5$\% response drift. 

Figure~\ref{fig:LaserDrift_run2_b} shows the Gaussian width distribution as a function of time observed for each layer during the entire Run~2. The Gaussian width for all layers increases with time during high instantaneous luminosity period when PMTs are under stress. It is caused by the different behaviour of different PMTs over time which are at different $|\eta|$ positions. During technical stops, when PMTs are at rest, some inversion of this effect is observed resulting from the recovery of the most affected PMTs to the average response in a given layer or cell type.

\begin{figure}[htbp]
\centering
    \subfloat[\label{fig:LaserDrift_run2_a}]{\includegraphics[width=0.5\linewidth]{figures/Averagelayers_Run2}}
    \subfloat[\label{fig:LaserDrift_run2_b}]{\includegraphics[width=0.5\linewidth]{figures/layers_gausianWidth_Run2}}
    \caption{The mean response variation in the PMTs (a) and Gaussian width (b) for each layer, as a function of time, observed during the entire Run~2 (between stand-alone laser calibration runs taken on 17 July 2015 and 22 October 2018). For each layer, the response variation is defined as the mean of a Gaussian fit to the variations in the channels associated with given layer. Known pathological channels are excluded. The laser calibration runs were not taken during the ATLAS end-of-year technical stops. Moreover, the laser system was not operational due to technical problems in the period September 10--27, 2016. Thus, no laser data can be seen in the plots for these time intervals. The LHC delivered luminosity is shown for comparison in grey. The vertical dashed lines show the start of $pp$ collisions in respective years.
%    (a) Average response variation (in \%) and (b) Gaussian width per TileCal layer as a function of time in entire Run~2 calculated using the Combined method. For each layer, the average response is obtained by a Gaussian fit to the distribution of PMT response variation with respect to an average reference prior to the start of collisions (including all laser runs in the time period of $\pm 10$ days of July 17, 2015). The vertical dashed lines show the start of proton-proton collisions in respective years. The laser system was not operational due to technical problems in the period September 10--27, 2016. Thus, no Laser data can be seen in the time evolution plots for this time interval. 
    }\label{fig:LaserDrift_run2}
\end{figure}

\FloatBarrier
