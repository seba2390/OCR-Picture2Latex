The ATLAS Tile Calorimeter (TileCal)~\cite{TCAL-2010-01} is the central hadronic calorimeter of the ATLAS experiment~\cite{PERF-2007-01} at CERN's Large Hadron Collider (LHC). The TileCal is a scintillator-based calorimeter employing photomultiplier tubes (PMTs) to measure the scintillation light. The TileCal is crucial to identify and measure the energy and direction of hadronic jets, provides information for the online trigger system and participates in the reconstruction of the missing transverse momentum associated to weakly-interacting particles. Thus, the TileCal plays a central role in the reconstruction of collision events for subsequent physics analyses.

The stability and resolution of the calorimeter response are parameters with direct impact on the precision of the reconstruction of jets and missing energy by the ATLAS experiment. To control these parameters, the TileCal is equipped with dedicated systems that allow to monitor the different components of the detector and calibrate its energy measurements. These procedures were conducted during the LHC Run~1 and Run~2 data taking campaign, contributing to a good operation and performance of the TileCal~\cite{TCAL-2017-01}. An important aspect was to correct for the response variation of the PMTs, achieved with a laser system. This article describes the laser calibration of the calorimeter in the LHC Run~2 data taking campaign. A previous report about the laser calibration in Run~1 can be found in Ref.~\cite{bib:laser_run_1}.

The calorimeter is briefly described in Section~\ref{sec:tilecal} and the Laser~II system operating during Run~2 is detailed in Section~\ref{sec:laser}. A major upgrade of the system employed in Run~1 was performed in 2014 and this is reported here. The description of the calibration procedure is presented in Section~\ref{sec:calibration}. Section~\ref{sec:laseringap} describes the monitoring of the calorimeter timing and PMT dependence on anode current with laser pulses fired during physics runs. Channel quality monitoring and studies of PMT linearity based on laser calibration data are reported in Section~\ref{sec:monitoring}. Finally, conclusions are drawn in Section~\ref{sec:conclusion}.



