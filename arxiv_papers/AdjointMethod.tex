\documentclass{article}
\usepackage[utf8]{inputenc}
\usepackage{amsmath, amssymb, amsfonts, xcolor}

\newcommand{\bu}{\mathbf{u}}
\newcommand{\bp}{\mathbf{p}}
\newcommand{\bn}{\mathbf{n}}
\newcommand{\bv}{\mathbf{v}}

\title{Variational forms for inverse problems}

\begin{document}


\section{Inverse problems}
Inverse problems are well established and mathematically understood in the context of control theory. Typically, the setup is similar to ours. There is some system whose solution $u(x)$ is fixed by a PDE $\mathcal{F}$ with parameters $q$:
\begin{align*}
    \mathcal{F}(u, q) = 0
\end{align*}
In addition, we assume there are some measurements of the solution field $u^m$, and we want to use these to find the experimental parameters $q$. In spirit, we want to find 
\begin{align*}
    q^* &= \arg\min_q\mathcal{J}(u(q))
    \\
    &= \arg\min_q ||u(q; x) - u^m(x)||^2
\end{align*}
Unfortunately, it's hard to perform the optimization over $q$.
\section{Lagrange Formulation}
In keeping with the notation of the model, we consider $u(x)$ to be our state, $Y(x)$ our control parameter, and $v(x)$ and $p(x)$ will initially be Lagrange multipliers. The data measurements we denote by $\hat u$.
\begin{align*}
    \mathcal{L}(u(x), Y(x), v(x), p(x)) = \mathcal{J}(u,\hat u) - \int_{\Omega} v(x)\mathcal{F}(u, q) - \int_{\partial\Omega}v(x)\mathcal{G}(u, q)
\end{align*}
We will ultimately want to take a (Frechet/functional) derivative of the Lagrangian w.r.t $u(x)$, so we will push derivatives via integration by parts:
\begin{align*}
    \int_\Omega A_{i} \partial_j B_{ij} 
    &=\int_\Omega \partial_j (A_{i} B_{ij} ) -\int_\Omega  (\partial_j A_{i}) B_{ij} 
    \\
    &= -\int_\Omega  (\partial_j A_{i}) B_{ij} +
    \int_{\partial\Omega} A_{i} B_{ij} n_j dS
\end{align*}
where in the second step we use the divergence theorem.
We now calculate the elastic Lagrangian in its full glory. Starting from 
\begin{align*}
    \mathcal{L} &= \mathcal{J}(u,\hat u) - \int_{\Omega} v_i(x) \left(\mathcal{D}u - Yu \right)_i - \int_{\partial\Omega}v_i(x)\left(\sigma_{ij}n_j + \sigma n_i\right)
    \\
    &= \mathcal{J}(u,\hat u) {\color{blue}- \int_{\Omega} v_i \partial_j \sigma_{ij}^{u}} + \int_{\Omega}Yu_iv_i - \mathcal{L}_{\partial\Omega}
\end{align*}
We focus on the blue term:
We first calculate I:
\begin{align*}
    - \int_{\Omega} v_i \partial_j \sigma_{ij}^{u} &= + \int_{\Omega} (\partial_j v_i)  \sigma_{ij}^{u} - \int_{\partial\Omega} v_i \sigma_{ij}^{u}n_j
    \\
    &= \int_{\Omega} (\partial_j v_i) \left(\alpha \delta_{ij}\partial_k u_k + \frac{\beta}{2}(\partial_j u_i + \partial_i u_j)\right) - \int_{\partial\Omega} v_i \sigma_{ij}^{u}n_j
    \\
    \\
    &= -\alpha\int_{\Omega} u_k \partial_k(\partial_i v_i) + \alpha\int_{\partial\Omega}u_k \partial_i v_i n_k
    \\
    &\quad\quad-\frac{\beta}{2}\int_{\Omega} u_i\partial_j(\partial_j v_i) +\frac{\beta}{2}\int_{\partial\Omega}u_i\partial_j v_i n_j
    \\
    &\quad\quad-\frac{\beta}{2}\int_{\Omega} u_j\partial_i(\partial_j v_i) +\frac{\beta}{2}\int_{\partial\Omega}u_j\partial_j v_i n_i
    \\
    &\quad\quad- \int_{\partial\Omega} v_i \sigma_{ij}^{u}n_j
\end{align*}
Under some index relabeling, in particular swapping $i$ and $j$ in the third line to get $v_j$, we can cast it in the form
\begin{align*}
    &= -\int_{\Omega} u_i \left(\alpha\partial_j(\delta_{ij}\partial_i v_i) +
    \frac{\beta}{2}\partial_j(\partial_j v_i + \partial_i v_j)\right)
    \\
    &\quad\quad +\int_{\partial\Omega}u_i n_j\left(\alpha\delta_{ij} \partial_k v_k + \frac{\beta}{2}(\partial_j v_i + \partial_i v_j)\right)
    \\
    &\quad\quad- \int_{\partial\Omega} v_i \sigma_{ij}^{u}n_j
    \\
    &= -\int_{\Omega} u_i \partial_j\sigma_{ij}^v + \int_{\partial\Omega} \left(u_i \sigma_{ij}^v - v_i \sigma_{ij}^{u}\right)n_j
\end{align*}
Where we noticed that the derivatives of $v$ are precisely the stress tensor, for displacements $v_i$.
To summarize, we can write our Lagrangian now as 
\begin{align*}
    \mathcal{L} &= \mathcal{J}(u,\hat u) - \int_{\Omega} u_i \partial_j\sigma_{ij}^v - Yu_iv_i +
    \int_{\partial\Omega} \left(u_i \sigma_{ij}^v - v_i \sigma_{ij}^{u}\right)n_j - v_i (\sigma_{ij}^u n_j + \sigma n_i)
    \\
    &=
     \mathcal{J}(u,\hat u) - \int_{\Omega} u_i (\mathcal{F}v)_i  +
    \int_{\partial\Omega} \left(u_i \sigma_{ij}^v - v_i \sigma_{ij}^{u}\right)n_j 
\end{align*}
The adjoint equation is given by $\delta\mathcal{L}/\delta u_i = 0$, from which we get the conditions, assuming $\mathcal{J}=\frac{1}{2}\int (u_i - \hat{u}_i)^2$. \textbf{We further assume that $v_i=0$ on the boundary.}
\begin{align*}
    \begin{cases}
     \mathcal{F}v &= (u_i - \hat{u}_i) \quad\text{on $\Omega$}
     \\
     \sigma_{ij}^v n_j &= \overset{?}{-}(u_i - \hat{u}_i) \quad\text{on $\partial\Omega$}
    \end{cases}
\end{align*}
Assuming the PDE and adjoint conditions are fulfilled (which correspond to $\delta\mathcal{L}/\delta v_i = 0$ and $\delta\mathcal{L}/\delta u_i = 0$, respectively), we have 
\begin{align*}
     \frac{\delta\mathcal{J}}{\delta Y} = \frac{\delta\mathcal{L}}{\delta Y} = v_iu_i
\end{align*}
In the algorithms laid out in Tr{\"o}ltzsch, typically one solves for $u_n(x), v_n(x)$ given a parameter $q_n(x)$ ($n$ is an index for the step/iteration), and then updates $q_{n+1}(x)=q_{n}-\eta v_n(x)$. \cite{Troeltzsch2010}[section Lemma 2.30; 2.12.2; 3.7.1].















\iffalse
\section{Lagrange Formulation}
In keeping with the notation of the model, we consider $u(x)$ to be our state, $Y(x)$ our control parameter, and $v(x)$ and $p(x)$ will initially be Lagrange multipliers. The data measurements we denote by $\hat u$.
\begin{align*}
    \mathcal{L}(u(x), Y(x), v(x), p(x)) = \mathcal{J}(u,\hat u) - \int_{\Omega} v(x)\mathcal{F}(u, q) - \int_{\partial\Omega}p(x)\mathcal{G}(u, q)
\end{align*}
We will ultimately want to take a (Frechet/functional) derivative of the Lagrangian w.r.t $u(x)$, so we will push derivatives on to the lagrange multipliers using Green's identities:
\begin{align*}
    G_1: &\quad\quad \int_\Omega v \nabla^2 u = \int_\Omega u \nabla^2 v + \int_{\partial\Omega} \left( v(\nabla u) - u(\nabla v)\right)\cdot \bn ds \
    \\
    G_2: &\quad\quad \int_\Omega v \left( \nabla\cdot \bu\right) = -\int_\Omega  \left(\bu\cdot\nabla\right)v + \int_{\partial\Omega} v(\bu\cdot\bn) ds
\end{align*}
$G_1$ is for scalar fields $v$, $u$, i.e. single components of the vectors $\bv$, $\bu$, and $G_2$ is for one scalar field $v$ and one vector field $\bu$.
We now calculate the elastic Lagrangian in its full glory. Starting from 
\begin{align*}
    \mathcal{L} &= \mathcal{J}(u,\hat u) - \underbrace{\int_{\Omega} v_i(x) \left(\mathcal{D}u - Yu \right)_i}_{\equiv \text{I}} - \underbrace{\int_{\partial\Omega}p_i(x)\left(\sigma_{ij}n_j + \sigma n_i\right)}_{\equiv \text{II}}
\end{align*}
We first calculate I:
\begin{align*}
    \text{I} &= \int_{\Omega} v_i(x) \left(\mathcal{D}u - Yu \right)_i
    \\
    &=\int_{\Omega} v_i(x) \left(\partial_j\sigma_{ij} - Yu_i \right)
    \\
    &=\int_{\Omega} v_i(x) \left(\partial_j\left(\alpha\delta_{ij}u_{kk}+\frac{\beta}{2}(\partial_j u_i + \partial_i u_j)\right) - Yu_i \right)
    \\
    &=\int_{\Omega}\alpha \bv\cdot\nabla (\nabla\cdot\bu)+\frac{\beta}{2}\left(\bv\cdot(\nabla^2 \bu) +(\bv\cdot\nabla)(\nabla\cdot\bu)\right) - Y\bv\cdot\bu 
    \\
    &=\int_{\Omega}\left(\alpha+\frac{\beta}{2}\right) \underbrace{\bv\cdot\nabla (\nabla\cdot\bu)}_{i}+\frac{\beta}{2}(\underbrace{\bv\cdot(\nabla^2 \bu)}_{ii} - Y\bv\cdot\bu 
\end{align*}
We can use $G_2$ (in reverse) to handle $(i)$, and $G_1$ to handle $(ii)$.
\begin{align*}
    (i) &=\int_{\Omega} \bv\cdot\nabla (\nabla\cdot\bu)
    \equiv\int_{\Omega} (\bv\cdot\nabla)\phi
    \\
    &\overset{G_2}{=} -\int_\Omega \phi (\nabla\cdot \bv)+\int_{\partial\Omega} \phi (\bv\cdot\bn)
    \\
    &= -\int_\Omega (\nabla\cdot\bu)(\nabla\cdot \bv)+\int_{\partial\Omega} (\nabla\cdot\bu)(\bv\cdot\bn)
    \intertext{On the first term we can again use $G_2$, now with the scalar field $\nabla\cdot\bv$}
    &= +\int_\Omega (\bu\cdot\nabla)(\nabla\cdot \bv) - \int_{\partial\Omega}(\nabla\cdot \bv)(\bu\cdot\bn)+\int_{\partial\Omega} (\nabla\cdot\bu)(\bv\cdot\bn)
    \intertext{Next, we consider $ii$ component-wise}
    (ii) &= \int_\Omega v_i\cdot(\nabla^2 u_i)
    \\
    &= \int_\Omega u_i\cdot(\nabla^2 v_i) + \int_{\partial\Omega} \left( v_i(\nabla u_i) - u_i(\nabla v_i)\right)\cdot \bn 
\end{align*}
Putting it together, we see that I becomes
\begin{align*}
    \text{I} &=\int_{\Omega}\left(\alpha+\frac{\beta}{2}\right) \bv\cdot\nabla (\nabla\cdot\bu)+\frac{\beta}{2}(\bv\cdot(\nabla^2 \bu) - Y\bv\cdot\bu 
    \\
    &\overset{(i)}{=}\int_{\Omega}\left(\alpha+\frac{\beta}{2}\right) (\bu\cdot\nabla)(\nabla\cdot \bv) 
    +\frac{\beta}{2}(\bv\cdot(\nabla^2 \bu) - Y\bv\cdot\bu  
    \\
    &\quad +\int_{\partial\Omega}\left(\alpha+\frac{\beta}{2}\right)(\nabla\cdot\bu)(\bv\cdot\bn) - (\nabla\cdot \bv)(\bu\cdot\bn)
    \\
    &\overset{(ii)}{=}\int_{\Omega}\left(\alpha+\frac{\beta}{2}\right) (\bu\cdot\nabla)(\nabla\cdot \bv) 
    +\frac{\beta}{2}(\bu\cdot(\nabla^2 \bv) - Y\bv\cdot\bu  
    \\
    &\quad +\int_{\partial\Omega}\left(\alpha+\frac{\beta}{2}\right)\left((\nabla\cdot\bu)(\bv\cdot\bn) - (\nabla\cdot \bv)(\bu\cdot\bn)\right)
    + \frac{\beta}{2}\left( v_i(\nabla u_i) - u_i(\nabla v_i)\right)\cdot \bn 
\end{align*}
Where we keep the indices in the last term because else it will be confusing what is being dotted into what.
By inspecting the first term, we see that we can actually write it as 
\begin{align*}
    \text{I} &=
    \underbrace{\int_{\Omega}\bu\cdot(\mathcal{D}\bv - Y\bv)}_{\langle \bu, \mathcal{F}\bv\rangle} +\int_{\partial\Omega}...
\end{align*}
Now we turn to II:
\begin{align*}
    \text{II}&=\int_{\partial\Omega}p_i(x)\left(\sigma_{ij}n_j + \sigma n_i\right)
    \\
    &=\int_{\partial\Omega} p_i\left(\left(\alpha\delta_{ij}u_{kk}+\frac{\beta}{2}(\partial_j u_i + \partial_i u_j)\right)n_j + \sigma n_i\right)
    \\
    &=\int_{\partial\Omega} \alpha (\nabla\cdot\bu)(\bp\cdot \bn)+\frac{\beta}{2}(p_i \nabla u_i)\cdot\bn + n_j(\bp\cdot\nabla) u_j) + \sigma \bp\cdot\bn
\end{align*}
We combine all boundary terms to write
\begin{align*}
    \mathcal{L}&=\mathcal{J}-\langle \mathcal{D}v, u\rangle + \langle Yv, u\rangle - \mathcal{L}_{\partial\Omega}
    \intertext{with}
    \mathcal{L}_{\partial\Omega} &= \int_{\partial\Omega}\left(\alpha+\frac{\beta}{2}\right)\left((\nabla\cdot\bu)(\bv\cdot\bn) - (\nabla\cdot \bv)(\bu\cdot\bn)\right)
    + \frac{\beta}{2}\left( v_i(\nabla u_i) - u_i(\nabla v_i)\right)\cdot \bn 
    \\
    &\quad -\alpha (\nabla\cdot\bu)(\bp\cdot \bn)-\frac{\beta}{2}(p_i \nabla u_i)\cdot\bn - n_j(\bp\cdot\nabla) u_j) - \sigma \bp\cdot\bn
    \\
    &= \int_{\partial\Omega} \alpha(\nabla\cdot\bu)[(\bv\cdot\bn)-(\bp\cdot\bn)]
    \\
    &\quad+\frac{\beta}{2}\left\{(\nabla\cdot\bu)(\bv\cdot\bn) - (\nabla\cdot \bv)(\bu\cdot\bn)
    + \left( v_i(\nabla u_i) - u_i(\nabla v_i)\right)\cdot \bn 
    -(p_i \nabla u_i)\cdot\bn - n_j(\bp\cdot\nabla) u_j)\right\}
    \\
    &\quad- \sigma \bp\cdot\bn
\end{align*}
That's pretty nasty. We proceed in a \textit{very} handwavy way. First, we assume now that $\bp$ and $\bv$ actually coincide:
\begin{align*}
    \mathcal{L}_{\partial\Omega} &= \int_{\partial\Omega}
    \\
    &\quad+\frac{\beta}{2}\left\{(\nabla\cdot\bu)(\bv\cdot\bn) - (\nabla\cdot \bv)(\bu\cdot\bn)
    + \left( v_i(\nabla u_i) - u_i(\nabla v_i)\right)\cdot \bn 
    -(v_i \nabla u_i)\cdot\bn - n_j(\bv\cdot\nabla) u_j)\right\}
    \\
    &\quad- \sigma \bv\cdot\bn 
    \\
    &= \int_{\partial\Omega}
    \\
    &\quad+\frac{\beta}{2}\left\{(\nabla\cdot\bu)(\bv\cdot\bn) - (\nabla\cdot \bv)(\bu\cdot\bn)
    + \left( - u_i(\nabla v_i)\right)\cdot \bn  - n_j(\bv\cdot\nabla) u_j)\right\}
    \\
    &\quad- \sigma \bv\cdot\bn 
\end{align*}
Second, we do the same thing that Seidl 2019 [eqn. 25] and Troeltzsch [right below eqn. 2.3], which is restrict the space for the function $\bv$ to those which vanish on the boundary. That kills most terms, leaving
\begin{align*}
    \delta \mathcal{L}_{\partial\Omega} h 
    &= \int_{\partial\Omega}\frac{\beta}{2}\left\{- (\nabla\cdot \bv)(\bu\cdot\bn)
    + - u_i(\nabla v_i)\cdot \bn \right\}h
\end{align*}
We can now actually do the derivative 
\begin{align*}
    \frac{\delta \mathcal{L}_{\partial\Omega}}{\delta u_k(x)} h
    &= \int_{\partial\Omega}\frac{\beta}{2}\left\{- (\nabla\cdot \bv)n_k
    - \underbrace{\nabla v_k \cdot \bn}_{\partial_{\bn}v_k} \right\}h
\end{align*}
\newpage
we can use the derivative \cite{Troeltzsch2010}[section Lemma 2.30; 2.12.2; 3.7.1]
\begin{align*}
    \frac{\partial f}{\partial q(x)} = -v(x)
\end{align*}
In the algorithms laid out in Tr{\"o}ltzsch, typically one solves for $u_n(x), v_n(x)$ given a parameter $q_n(x)$ ($n$ is an index for the step/iteration), and then updates $q_{n+1}(x)=q_{n}-\eta v_n(x)$.

\section{Weak Formulation for Fenics}
We want a weak solution given the strong equations
\begin{align*}
    \partial_j \sigma_{ij} &= Yu_i \quad\quad\text{in bulk $\Omega$}
    \\
    \sigma_{ij}n_j&=\sigma_a n_i\quad\quad\text{on boundary $\partial\Omega$}
\end{align*}
For the first, multiply by a test function $v_i$:
\begin{align*}
    \mathcal{L}_1&=\int_{\Omega} v_i\partial_j\sigma_{ij} - Yv_i u_i 
    \\
    &=\int_{\Omega} -\partial_j v_i + \sigma_{ij} - Yv_i u_i 
\end{align*}
\fi


\end{document}
