\subsection{Proofs for Lemma \ref{factor_ti}}\label{factor1}
First of all, we present some useful lemmas that will be used in the proof of Lemma \ref{factor_ti}. Given two high-dimensional Gaussians, we can bound their  total variation distance as follows.
\begin{lemma}[Theorem 1.2 and Proposition 2.1 in \cite{devroye2018total}]\label{tv_norm}
Suppose that $d>1$. Let $\mu_1\neq\mu_2\in\R^d$. Then, we have
\$
\frac{1}{200}\leq \frac{\TV\big(\mN(\mu_1,I_d),\mN(\mu_2,I_d)\big)}{\min\{1,\|\mu_1-\mu_2\|_2\}}\leq 1.
\$
\end{lemma}

\begin{lemma}[Theorem 1.1 in \cite{devroye2018total}]\label{tv_norm_2}
Suppose that $d>1$. Let $\mu\in\R^d$ and $\Sigma_1\neq\Sigma_2$ be positive definite $d\times d$ matrices. Then, we have
\$
\frac{1}{100}\leq \frac{\TV\big(\mN(\mu,\Sigma_1),\mN(\mu,\Sigma_2)\big)}{\min\{1,\|\Sigma^{-1/2}_1\Sigma_2\Sigma^{-1/2}_1-I_d\|_{\rF}\}}\leq \frac{3}{2}.
\$
\end{lemma}

Recall that we define $\mathcal{B}:=\{B\in\R^{d\times r}\,|\,\|B\|_2\leq D\}$. Let $B\in\mathcal{B}$ and $B^* $ be the ground truth parameter. We denote by $\sigma^* _{\max}$ and $\sigma^* _{\min}$ the largest and smallest singular value of $B^* $, respectively. Moreover, we denote the singular value decomposition of $B$ and $B^* $ by $B=U\Sigma V$ and $B^* =U^* \Sigma^*  V^* $, respectively. Here $\Sigma, \Sigma^* \in\R^{r\times r}$ are diagonal matrices and $U,U^* \in\R^{d\times r}$, $V,V^* \in\R^{r\times d}$ are matrices with orthogonal columns. Let 
\%\label{092401}
M:=BB^T=U\Lambda U^T,\quad M^* :=B^* B^{* T}=U^* \Lambda^*  U^{* T}, 
\%
where $\Lambda:=\Sigma\Sigma^T$ and $\Lambda^{*}:=\Sigma^* \Sigma^{* T}$. We define
\%\label{092402}
O:=\argmin_{O\in\mathcal{O}^{r\times r}}\|UO-U^* \|_{\rF}.
\%
Then, we have the following lemmas.
\begin{lemma}\label{D-K}
For $M,M^* $ defined in \eqref{092401} and $O$ defined in \eqref{092402}, there exists some absolute constants $c>1$ such that
\$
\|UO-U^* \|_{\rF}\leq \frac{c}{(\sigma^* _{\min})^2}\|M-M^* \|_{\rF}.
\$
Here $\sigma^* _{\min}$ is the smallest singular value of the true parameter $B^* $.
\end{lemma}
\begin{proof}
An application of Davis-Kahan Theorem \citep{davis1970rotation}.
\end{proof}
\begin{lemma}\label{yuxin}
For $M,M^* $ defined in \eqref{092401} and $O$ defined in \eqref{092402}, there exists some absolute constants $c$ such that
\$
\|\Lambda^{1/2}O-O\Lambda^{* 1/2}\|_{\rF}\leq \frac{4c(\sigma^* _{\max})^2}{(\sigma^* _{\min})^3}\|M-M^* \|_{\rF}.
\$
Here $\sigma^* _{\min}$ is the smallest singular value of the true parameter $B^* $.
\end{lemma}
\begin{proof}[Proof of Lemma \ref{yuxin}]
Our proof is inspired by \cite{ma2018implicit}. By Lemma 2.1 in \cite{schmitt1992perturbation}, we have
\%\label{092403}
\|\Lambda^{1/2}O-O\Lambda^{* 1/2}\|_{\rF}\leq \frac{1}{\sqrt{\sigma_{\min}(M^* )}}\|O^{T}\Lambda O-\Lambda^* \|_{\rF}=\frac{1}{\sigma^* _{\min}}\|O^{T}\Lambda O-\Lambda^* \|_{\rF}.
\%
Note that $\Lambda=U^TMU$ and $\Lambda^* =U^{* T}M^* U^* $. Thus, we have
\%\label{092404}
\|O^{T}\Lambda O-\Lambda^* \|_{\rF}&=\|O^TU^TMUO-U^{* T}M^* U^* \|_{\rF}\notag\\
&\leq \|O^TU^TMUO-O^TU^TM^* UO\|_{\rF}+\|O^TU^TM^* UO-U^{* T}M^* UO\|_{\rF}\notag\\
&\quad+\|U^{* T}M^* UO-U^{* T}M^* U^* \|_{\rF}\notag\\
&\leq \|M-M^* \|_{\rF}+2\|M^* \|_2\|UO-U^* \|_{\rF}\notag\\
&\leq \|M-M^* \|_{\rF}+2c\bigg(\frac{\sigma^* _{\max}}{\sigma^* _{\min}}\bigg)^2\|M-M^* \|_{\rF}\notag\\
&\leq 4c\bigg(\frac{\sigma^* _{\max}}{\sigma^* _{\min}}\bigg)^2\|M-M^* \|_{\rF},
\%
where the third inequality follows from Lemma \ref{D-K}. Combing \eqref{092403} and \eqref{092404}, we have
\$
\|\Lambda^{1/2}O-O\Lambda^{* 1/2}\|_{\rF}\leq \frac{4c(\sigma^* _{\max})^2}{(\sigma^* _{\min})^3}\|M-M^* \|_{\rF}.
\$


\end{proof}












Now we are ready to prove Lemma \ref{factor_ti}.

\begin{proof}[Proof of Lemma \ref{factor_ti}]
Let $\mathcal{O}^{r\times r}:=\{O\in\R^{r\times r}\,|\, OO^T=O^TO=I_r\}$. First of all, we show that for any $(B,\beta,O)\in\mathcal{B}\times\mathcal{C}\times\mathcal{O}$, it holds that $\P_{B,\beta}(x,y)=\P_{BO,O^T\beta}(x,y)$. This can be easily seen by the following observation,
\$
\P_{BO,O^T\beta}\sim\mN\bigg(0,
  \begin{bmatrix}
    BO(BO)^{T} & BOO^{T}\beta\\
    \beta^TOO^TB^T &  (O^{T}\beta)^{T}O^{T}\beta
  \end{bmatrix}
\bigg)=\mN\bigg(0,
  \begin{bmatrix}
    BB^T & B\beta\\
    \beta^TB^T &  \beta^{T}\beta
  \end{bmatrix}
  \bigg)\sim\P_{B,\beta}.
\$


By Lemma \ref{D-K}, it holds for some constant $c>1$ that
\%\label{092302}
\|UO-U^* \|_{\rF}\leq \frac{c}{(\sigma^* _{\min})^2}\|BB^T-B^*  B^{* T}\|_{\rF}.
\%
By Lemma \ref{yuxin}, it holds for some constant $c>1$ that
\%\label{092301}
\|\Sigma O-O\Sigma^* \|_{\rF}\leq \frac{4c(\sigma^* _{\max})^2}{(\sigma^* _{\min})^3}\|BB^T-B^*  B^{* T}\|_{\rF}.
\%
Let $\hat O:=V^{-1}OV^* \in\mathcal{O}^{r\times r}$. By \eqref{092302} and \eqref{092301}, we have
\%\label{092303}
\|B\hat O-B^* \|_{\rF}&=\|U\Sigma OV^* -U^* \Sigma^* V^* \|_{\rF}\notag\\
&\leq \|U\Sigma O-U^* \Sigma^* \|_{\rF}\notag\\
&\leq \|U\Sigma O-UO\Sigma^* \|_{\rF}+\|UO\Sigma^* -U^* \Sigma^* \|_{\rF}\notag\\
&\leq \|\Sigma O-O\Sigma^* \|_{\rF}+\|UO-U^* \|_{\rF}\|\Sigma^* \|_2\notag\\
&\leq c\cdot \bigg(\frac{4(\sigma^* _{\max})^2}{(\sigma^* _{\min})^3}+\frac{\sigma^* _{\max}}{(\sigma^* _{\min})^2}\bigg)\cdot\|BB^T-B^*  B^{* T}\|_{\rF}\notag\\
&\leq \frac{5c(\sigma^* _{\max})^2}{(\sigma^* _{\min})^3}\cdot\|BB^T-B^*  B^{* T}\|_{\rF}.
\%
Note that
\%\label{092304}
\TV\big(\P_{B\hat O}(x,z),\P_{B^* }(x,z)\big)&=\int|p_{B\hat O}(x\,|\,z)-p_{B^* }(x\,|\,z)|p(z)\,dxdz\notag\\
&=\int\TV\big(\mN(B\hat O z,I_d),\mN(B^*  z,I_d)\big)p(z)\,dz\notag\\
&\leq \int \min\{1, \|B\hat O z-B^* z\|_2\}p(z)\,dz\notag\\
&\leq \min\big\{1, \E[\|B\hat O z-B^* z\|_2]\big\},
\%
where the first inequality follows from Lemma \ref{tv_norm}. We can show that
\%\label{092305}
\E[\|B\hat O z-B^* z\|_2]&\leq \Big(\E\big[\|B\hat O z-B^* z\|^2_2\big]\Big)^{1/2}\notag\\
&=\Big(\E\big[z^T(B\hat O-B^* )^{T}(B\hat O-B^* )z\big]\Big)^{1/2}\notag\\
&=\Big(\E\big[{\rm Tr}\big((B\hat O-B^* )^{T}(B\hat O-B^* )zz^T\big)\big]\Big)^{1/2}\notag\\
&=\Big({\rm Tr}\big((B\hat O-B^* )^{T}(B\hat O-B^* )\big)\Big)^{1/2}\notag\\
&=\|B\hat O-B^* \|_{\rF}.
\%
By \eqref{092303}, \eqref{092304} and \eqref{092305}, it holds that
\%\label{092309}
&\TV\big(\P_{B\hat O}(x,z),\P_{B^* }(x,z)\big)\notag\\
&\quad\leq \min\big\{1,\|B\hat O-B^* \|_{\rF}\big\}\notag\\
&\quad\leq \min\bigg\{1, \frac{5c(\sigma^* _{\max})^2}{(\sigma^* _{\min})^3}\cdot\|BB^T-B^*  B^{* T}\|_{\rF}\bigg\}\notag\\
&\quad\leq \frac{5c(\sigma^* _{\max})^2}{(\sigma^* _{\min})^3}\cdot \big((\sigma^* _{\max})^2+1\big)\cdot\min\bigg\{1,\frac{\|BB^T-B^* B^{* T}\|_{\rF}}{(\sigma^* _{\max})^2+1}\bigg\},
\%
where the last inequality follows from $c>1$ and
\$
\frac{(\sigma^* _{\max})^2+1}{\sigma^* _{\min}}\geq\frac{2\sigma^* _{\max}}{\sigma^* _{\min}}>1.
\$
By Lemma \ref{tv_norm_2}, we have 
\%\label{092306}
&\TV(p_{B}(x),p_{B^* }(x))\notag\\
&\quad\geq \frac{1}{100}\min\big\{1,\|(B^* B^{* T}+I_d)^{-1/2}(BB^T-B^* B^{* T})(B^* B^{* T}+I_d)^{-1/2}\|_{\rF}\big\}.
\%
Note that
\%\label{092307}
&\|(B^* B^{* T}+I_d)^{-1/2}(BB^T-B^* B^{* T})(B^* B^{* T}+I_d)^{-1/2}\|_{\rF}\notag\\
&\quad\geq \frac{\|BB^T-B^* B^{* T}\|_{\rF}}{\|B^* B^{* T}+I_d\|_2}\geq\frac{\|BB^T-B^* B^{* T}\|_{\rF}}{(\sigma^* _{\max})^2+1}.
\%
Thus, by \eqref{092306} and \eqref{092307}, it holds that
\%\label{092308}
\TV(p_{B}(x),p_{B^* }(x))\geq \frac{1}{100}\min\bigg\{1,\frac{\|BB^T-B^* B^{* T}\|_{\rF}}{(\sigma^* _{\max})^2+1}\bigg\}
\%

Finally, by \eqref{092309} and \eqref{092308}, we have
\$
\TV\big(\P_{B\hat O}(x,z),\P_{B^* }(x,z)\big)&\leq \frac{500c(\sigma^* _{\max})^2\big((\sigma^* _{\max})^2+1\big)}{(\sigma^* _{\min})^3}\cdot\TV(p_{B}(x),p_{B^* }(x))\notag\\
&\leq\frac{500c(\sigma^* _{\max}+1)^4}{(\sigma^* _{\min})^3}\cdot\TV(p_{B}(x),p_{B^* }(x)).
\$

\end{proof}