\documentclass[reprint,amsmath,amssymb,aps,prl,superscriptaddress]{revtex4-1}

\usepackage{graphicx}
\usepackage{dcolumn}
\usepackage{bm}
\usepackage{color}
\linespread{1.0}
\usepackage{braket}  

\begin{document}

\title{Quantum Phase Diagram of PT-Symmetry or Broken in a Non-Hermitian Photonic Structure}

\author{Xinchen Zhang}
\thanks{These two authors contribute equally.}
\author{Yun Ma}
\thanks{These two authors contribute equally.}
\affiliation{State Key Laboratory for Mesoscopic Physics, Department of Physics, Peking University, Beijing 100871, China}
\author{Qi Liu}
\affiliation{State Key Laboratory for Mesoscopic Physics, Department of Physics, Peking University, Beijing 100871, China}
\affiliation{Frontiers Science Center for Nano-optoelectronics $\&$  Collaborative Innovation Center of Quantum Matter $\&$ Beijing Academy of Quantum Information Sciences, Peking University, Beijing 100871, China}
\author{Yali Jia}
\affiliation{State Key Laboratory for Mesoscopic Physics, Department of Physics, Peking University, Beijing 100871, China}
\author{Qi Zhang}
\affiliation{State Key Laboratory for Mesoscopic Physics, Department of Physics, Peking University, Beijing 100871, China}
\affiliation{Institute of Navigation and Control Technology, China North Industries Group Corporation, Beijing 100089, China}
\author{Zhanqiang Bai}
\affiliation{School of Mathematical Sciences, Soochow University, Suzhou, 215006, China}
\author{Junxiang Zhang}
\affiliation{Zhejiang Province Key Laboratory of Quantum Technology and Device, Department of Physics, Zhejiang University, Hangzhou 310027, China}
\author{Qihuang Gong}
\author{Ying Gu}
\email{ygu@pku.edu.cn}
\affiliation{State Key Laboratory for Mesoscopic Physics, Department of Physics, Peking University, Beijing 100871, China}
\affiliation{Frontiers Science Center for Nano-optoelectronics $\&$  Collaborative Innovation Center of Quantum Matter $\&$ Beijing Academy of Quantum Information Sciences, Peking University, Beijing 100871, China}
\affiliation{Collaborative Innovation Center of Extreme Optics, Shanxi University, Taiyuan, Shanxi 030006, China}
\affiliation{Peking University Yangtze Delta Institute of Optoelectronics, Nantong 226010, China}



\date{\today}
\begin{abstract}
Classically, PT symmetry or broken in photonic structures is well studied, where only average effect of gain and loss on each optical mode is considered. 
However, in quantum, the role of gain or loss in a non-hermitian  system is totally different, the specific quantum optical effect induced by which has never been studied. 
Here, we analytically obtained the PT-symmetry and PT-broken regime bounded by two exceptional lines in a bi-photonic structure with both gain and loss simultaneously existing.
For the consideration of reality, the steady state condition under the weak gain is identified.
We defined the exchange operator  to represent the photon exchange between two modes and further to characterize the transition from PT symmetry to broken.
Also, in the PT broken bi-waveguide system, multi-photon state can be on-demand engineered through the quantum interference.
Quantum PT-Phase diagram with steady state regime is the basis to study the quantum state fabrication, quantum interferences, and logic operations in non-hermitian quantum systems.

\end{abstract}

\maketitle
\section{I. Introduction}

In the nature, the exchange of energy and particles between the physical system and the external environment is universal. 
Open physical system is usually described by non-Hermitian Hamiltonian \cite{2020.AdvPhys}. 
Especially, the non-Hermitian system with PT-symmetry can give the eigenvalues of complete real numbers like Hermitian system \cite{1998.PhysRevLett,2017.NatPhys,2017.NatPhoto}.
 However, the non-Hermitian  system may also have the eigenvalues of conjugate complex numbers, that is, the breaking of PT-symmetry. 
  PT-symmetry or broken depends on the relationship of system parameters and they are separated by exceptional point (EP) \cite{2019.science}. 
 Among many physical branches, optical realization of PT-symmetry is relatively easy because we can obtain complex potential function by modulating the refractive index of the optical material \cite{2007.OpticsLett}. 
 Optical waveguides \cite{2009.PhysRevLett,2010.NatPhys} and whispering gallery microcavities \cite{2014.NatPhoto,2014.NatPhys} with gain and loss are good candidates to realize PT-symmetry or broken directly through the two-mode coupling. Recently, PT-symmetry has also been observed on optical lattice \cite{2012.Nature} and metasurface \cite{2014.PhysRevLett.(Manifestation}. 
 Nearly all kinds of PT-symmetric photonic structures are non-reciprocal \cite{2011.Science} or unidirectional visible \cite{2011.PhysRevLett} , which can be used to optical isolation devices \cite{2014.NatPhoto}. 
 In addition, PT symmetric optics has many applications in enhanced sensing \cite{2014.PRL}, laser \cite{2014.science.(Loss-induced,2014.science.(Single} and chiral optics \cite{2016.PNAS}.

The above statements are the properties and applications of PT-symmetry or broken under classical optics. 
For a single mode, we only need to consider the average effect of gain and loss. For example, in the paper \cite{2010.NatPhys}, both waveguides have the same loss. At the same time, one of the waveguides is loaded with a gain effect that is twice the loss in value, so on average, it becomes a balanced gain-loss structure. However, if we consider the quantum light field, that is, quantum jump \cite{2019.PRA.(QuantumExceptionalPoints} plays an important role, the gain and loss will no longer be the temporal inverse processes \cite{2018.EPL}. Because gain will inevitably bring quantum noise, while loss will not reduce quantum noise \cite{1982.PRD}. Non-Hermitian quantum photonics is a discipline to expansively define PT-symmetry \cite{2020.SciPost} and to explore the quantum properties of light in this background. It can be applied to integrated quantum optical circuits and quantum information processing \cite{2009.NatPhoto}. At present, the main method to study non-Hermitian quantum photonics is to solve the Lindblad master equation \cite{1976.CMP} for the quantum state transmission in a non-Hermitian two-waveguide coupling structure. The Lindblad master equation controls the density matrix of quantum state with the Liouville operator. The eigenvalue degeneracy point of the Liouville operator is the EP of this system \cite{2019.PRA.(QuantumExceptionalPoints}.
 In this way, people have found the spontaneous generation of photons \cite{2012.PRA,2017.ApplSci}, the connection between PT-symmetry and quantum interference effect \cite{2019.NP,2018.OL,2022.OE}, the effect of gain or loss on quantum entanglement \cite{2010.OE,2021.QuanSciTech}, and so on. 
 
For a non-Hermitian quantum photonics, the case that only gain or loss is in a single optical mode  has been widely studied. Another case, closer to reality, with both gain and loss simultaneously existing, has not been fully  taken into account \cite{2019.PRA.(Scully-Lamb}. After considering this more general situation thoroughly, what are the conditions and some new findings for PT-symmetry or broken?

In the following, we first analytically obtain the quantum phase diagram of PT-symmetry or broken in bi-photonic structure with both gain and loss simultaneously existing. 
The phase diagram shows the parameter conditions for PT symmetry or broken with the steady-state solution regime [Fig. 1].
 In order to verify the EP line that separating PT-symmetry or broken, we numerically calculated the eigenvalues of the Liouville operator, and found that the degeneracy point of the eigenvalues is consistent with our analytical derivation. 
  Then, we defined a new physical quantity exchange factor, which can represent the energy exchange between two optical modes, and can also be used to characterize PT symmetry or broken. 
  
It is worth noting that if we take the quantum PT-symmetric waveguide system as a whole, it is equivalent to a non-Hermitian beam splitter. Beam splitters are very important devices in classical and quantum optics, and they play an important role in quantum interference experiments. However, the absorption of light by materials is always unavoidable, so the non-Hermitian beam splitter came into being \cite{1998.PRA}. 
Non-Hermitian beam splitters also have unique properties and applications, such as quantum coherent perfect absorption \cite{2016.PRL}, anti-bunching of bosons \cite{2017.Science}, preparation of squeezed states \cite{2019.SPIE}, and fabrication of multi-bit quantum gates \cite{2021.JOSAB}. 
So at last, we tried to transport multi-photon states in the non-Hermitian two-waveguide coupling structure. We found that some on-demand quantum states can be prepared in the PT broken regime. Our PT phase diagram and related results are the basis for the study of non-Hermitian quantum photonics and have great application potential in quantum interference, quantum state engineering, quantum beam splitting and so on.

%%%%\section{II.  Model setup for single waveguide}
\section{II. Quantum PT phase diagram with steady state regime}

%%%Fig1.PDF
\begin{figure*}[htb]
  \centering
  % Requires \usepackage{graphicx}
  \includegraphics[width=\textwidth]{Fig1.pdf}\\
  \caption{\label{fig:Fig1}(a) Bi-photonic structures composed of two optical modes $\hat{a}_{1}$, $\hat{a}_{2}$ with the coupling coefficient $\mu$, each with loss rate $\gamma_{i}$ and gain rate $\beta_{i}$, $i=1, 2$.  (b) The phase diagram from PT-symmetry to broken. Two exceptional lines are shown as red lines, satisfying $|(\gamma_{1}-\beta_{1})-(\gamma_{2}-\beta_{2})|/\mu=2$. The part between two red lines is PT symmetry regime while another outside them is PT broken. The yellow part is the regime where the steady state exists, satisfying  both $\gamma_{1}-\beta_{1}+\gamma_{2}-\beta_{2}>0$ and $(\gamma_{1}-\beta_{1})(\gamma_{2}-\beta_{2})+\mu^{2}>0$. Yellow star: $(\gamma_1-\beta_1)/\mu=1, (\gamma_2-\beta_2)/\mu=0$; Grey star: $(\gamma_1-\beta_1)/\mu=2, (\gamma_2-\beta_2)/\mu=0$; Red star: $(\gamma_1-\beta_1)/\mu=3, (\gamma_2-\beta_2)/\mu=0$. (c) The real part of numerically calculated eigenvalues of Liouvillian $\mathcal{L}(\mu)$. The degeneracy occurs at $\mu=1 $. Here, $\gamma_1=3.1, \beta_1=0.1, \gamma_2=1.1, \beta_2=0.1.$}
\label{fig1}
\end{figure*}

%first paragraph

Consider a bi-photonic structure with loss and gain simultaneously existing [Fig. 1(a)], 
whose Hamiltonian is 
\begin{equation}
\hat{H}=\hbar \omega_1 \hat{a}_{1}^{\dagger} \hat{a}_{1}+\hbar \omega_2 \hat{a}_{2}^{\dagger} \hat{a}_2+\hbar \mu (\hat{a}_{1}^{\dagger} \hat{a}_{2}+\hat{a}_{2}^{\dagger}\hat{a}_{1})
\end{equation}
where $\hat{a}_{i}$ and $\hat{a}_{i}^{\dagger}$ are the boson annihilation and creation operator, respectively, and $\mu$ is the coupling strength between two cavities. For simplicity, we let $\omega_1=\omega_2=\omega$ .
With the steady state condition and weak incident light, the gain saturation effect can be neglected  \cite{1974.LaserPhysics}. 
Then the non-hermitian system is governed by Lindblad master equation \cite{1976.CMP},
\begin{equation}
\begin{aligned}
\frac{d \hat{\rho}}{d t}= 
-\frac{i}{\hbar}[\hat{H}, \hat{\rho}]
&+\sum_{i=1,2}\gamma_{i}(2 \hat{a_{i}} \hat{\rho} \hat{a}_{i}^{\dagger}
-\hat{\rho} \hat{a}_{i}^{\dagger} \hat{a}_{i}-\hat{a}_{i}^{\dagger} \hat{a}_{i} \hat{\rho})\\
&+\sum_{i=1,2}\beta_{i}(2 \hat{a}_{i}^{\dagger} \hat{\rho} \hat{a}_{i}
-\hat{\rho} \hat{a}_{i} \hat{a}_{i}^{\dagger}
-\hat{a}_{i} \hat{a}_{i}^{\dagger} \hat{\rho})
\end{aligned}
\end{equation}
where $\gamma_{i}$ ($\beta_{i}$) is the loss (gain) coefficient of the $i$th structure. From the Eq. (2) one can see that gain and loss in quantum regime are totally different processes, which can not be combined into one term.
%second paragraph
To construct the quantum PT phase diagram, we derive the evolution equations of  $\left\langle \hat{a_1}\right\rangle$ and $\left\langle \hat{a_2}\right\rangle$ with $t$  based on above Eq. (2) ,
\begin{equation}
\begin{gathered}
\frac{d}{d t}\left(\begin{array}{l}
\left\langle \hat{a}_{1}\right\rangle \\
\left\langle \hat{a}_{2}\right\rangle
\end{array}\right)=M\left(\begin{array}{c}
\left\langle \hat{a}_{1}\right\rangle \\
\left\langle \hat{a}_{2}\right\rangle
\end{array}\right)
\end{gathered}
\end{equation}
where
\begin{equation}
\begin{gathered}
M=\left(\begin{array}{cc}
-\gamma_{1}+\beta_{1} & -i \mu \\
-i \mu & -\gamma_{2}+\beta_{2}
\end{array}\right).
\end{gathered}
\end{equation}
The eigenvalues of the matrix $M$ are
\begin{equation}
\begin{aligned}
\omega_{\pm}=&\frac{1}{2}i(-\gamma_{1}+\beta_{1}-\gamma_{2}+\beta_{2})\\
&\mp\frac{1}{2}i\sqrt{[(-\gamma_{1}+\beta_{1})-(-\gamma_{2}+\beta_{2})]^{2}-4\mu^{2}}.
\end{aligned}
\end{equation}
The degeneracy of eigenvalues $\omega_{\pm}$ which satisfies $|(-\gamma_{1}+\beta_{1})-(\gamma_{2}+\beta_{2})|=2\mu$, are called exceptional lines (EPLs), shown as red lines in PT-phase diagram in Fig. 1(b). 
The evolution of  $\left\langle \hat{a_1}\right\rangle$ and $\left\langle \hat{a_2}\right\rangle$ represents the properties of our system. The EPLs calculated from the matrix $M$ are the boundaries between PT symmetry and PT broken of our system. 
The area between two red lines in which $|(-\gamma_{1}+\beta_{1})-(\gamma_{2}+\beta_{2})|<2\mu$ is PT symmetric while the areas outside two lines in which $|(-\gamma_{1}+\beta_{1})-(\gamma_{2}+\beta_{2})|>2\mu$ are PT broken. 
PT phase diagram in quantum system are quite different from the classical system, in that each single point on quantum PT phase diagram contains infinite cases.
%third

Next one can see that, the EPLs that we have obtained by this new analytical method are consistent with that numerically solved by the master equation. 
The master equation Eq. (2) could be written as $\frac{d \hat{\rho}}{d t}=\mathcal{L}\hat{\rho}$. With the super operator Liouvillian $\mathcal{L}$, the quantum evolution properties of the system are described entirely. 
Given a set of complete quantum state basis vectors, or a superior limit of photon number, the Liouvillian $\mathcal{L}$ can be expressed by a high dimension matrix $L$. 
We can also find EPs according to the splitting of eigenvalues of the matrix $L$ \cite{2019.PRA.(QuantumExceptionalPoints}. 
Fig. 1(c) shows the eigenvalues $\lambda_i$ as functions of coupling coefficient $\mu$. An obvious splitting point locates at $\mu=1$, which is identical to our theoretical prediction.

%forth
Furthermore, the evolution of the mean photon number $\langle\hat{n_1}\rangle=\langle\hat{a}_1^\dagger\hat{a_1}\rangle$, $\langle\hat{n_2}\rangle=\langle\hat{a}_2^\dagger\hat{a_2}\rangle$ of two modes, and an exchange factor $\langle\hat{\eta}\rangle=\langle i(\hat{a}_{2}^\dagger\hat{a}_{1}-\hat{a}_{1}^\dagger\hat{a}_{2})\rangle$ can be written as,
\begin{equation}
\begin{aligned}
\frac{d}{d t} \left\langle \hat{n}_{1} \right\rangle &= 2\left(\beta_{1}-\gamma_{1}\right) \left\langle \hat{n}_{1} \right\rangle +\mu \left\langle \hat{\eta} \right\rangle+2\beta_{1}\\
\frac{d}{d t} \left\langle \hat{n}_{2} \right\rangle &= 2\left(\beta_{2}-\gamma_{2}\right) \left\langle \hat{n}_{2} \right\rangle -\mu \left\langle \hat{\eta} \right\rangle+2\beta_{2}\\
\frac{d}{d t}\left\langle \hat{\eta} \right\rangle &=2\mu \left\langle \hat{n}_{2} \right\rangle-2\mu \left\langle \hat{n}_{1} \right\rangle+ \left(\beta_{1}+\beta_{2}-\gamma_{1}-\gamma_{2}\right) \left\langle \hat{\eta} \right\rangle
\end{aligned}
\end{equation}
whose solutions satisfy the steady state conditions that both $\gamma_{1}+\gamma_{2}-\beta_{1}-\beta_{2}>0$ and $(\gamma_{1}-\beta_{1})(\gamma_{2}-\beta_{2})+\mu^{2}>0$,  shown as the yellow regime of phase diagram in Fig. 1(b). 
Under the steady state conditions, the final steady state value of mean photon number of two modes as well as $\langle\hat{\eta}\rangle$ can be written as 
\begin{equation}
\begin{aligned}
\left\langle \hat{n}_{1} \right\rangle_{s s}&=\frac{\Delta_{1}-\beta_{1}\left(\Delta_{2}+2\beta_{2}\gamma_{2}-\gamma_{2}^{2}\right)}{\Delta_{3}}\\
\left\langle \hat{n}_{2} \right\rangle_{s s}&=\frac{\Delta_{1}-\beta_{2}\left(\Delta_{2}+2\beta_{1}\gamma_{1}-\gamma_{1}^{2}\right)}{\Delta_{3}}\\
\left\langle \hat{\eta} \right\rangle_{s s}&=\frac{2\mu\left(\beta_{2}\gamma_{1}-\beta_{1}\gamma_{2}\right)}{\Delta_{3}}
\end{aligned}
\end{equation}
with $\Delta_1=(\beta_1+\beta_2)(\beta_1\beta_2+\mu^2)$, 
$\Delta_2=\beta_1\gamma_2+\beta_2\gamma_1-\gamma_1\gamma_2$,
and $\Delta_3=(\gamma_1+\gamma_2-\beta_1-\beta_2)[(\gamma_1-\beta_1)(\gamma_2-\beta_2)+\mu^2]$. 
Here, the parameters in the steady state region should be satisfied with the weak gain condition.
\begin{figure*}[htb]
  \centering
  % Requires \usepackage{graphicx}
  \includegraphics[width=\textwidth]{Fig2.pdf}\\
  \caption{Different transmission behaviors in PT symmetry regime and PT broken regime characterized by the exchange factor $\eta$ or quadrature amplitudes $X_{1,2}, Y_{1,2}$. (a) The evolution of $\eta$ at different loss rate $\gamma_1$. Here, $ \beta_1=0.1, \gamma_2=0.1, \beta_2=0.1, \mu=1$. The initial state is $|0,1\rangle$. (b) The exchange of $X_1$ and $Y_2$ in PT symmetry regime, where  $\gamma_1=1.1$ (The yellow star in Fig. 1(b)). Inset: The exchange of $X_2$ and $Y_1$ in PT symmetry regime. (c) The exchange of $X_1$ and $Y_2$ in PT broken regime, where  $\gamma_1=3.1$ (The red star in Fig. 1(b)). Inset: The exchange of $X_2$ and $Y_1$ in PT broken regime. Other parameters in (b)(c) are the same as that in (a). The initial state in (b)(c) is coherent state $|\alpha=1+i,0\rangle$.}
\label{fig2}
\end{figure*}
From Eq. (7), one can see that, for one steady state point, there are infinite sets of parameters $\gamma_1$, $\beta_1$, $\gamma_2$, and $\beta_2$ corresponding to infinite steady state values. 
But owing to the decoherence effects of loss and gain, the steady state will become a thermal optical field without any quantum feature. 
Our following discussions are limited within the steady state regime.

\section{III. Exchange Factor and Quadrature Amplitudes}

To characterize the PT-symmetry or broken, we rewrite the exchange operator $\hat{\eta}$ as
\begin{equation}
\hat{\eta}=2(\hat{X}_2\hat{Y}_1-\hat{X}_1\hat{Y}_2)
\end{equation}
with  $\hat{X}_{1,2}=(\hat{a}_{1,2}+\hat{a}^\dagger_{1,2})/2, \hat{Y}_{1,2}=i(\hat{a}_{1,2}-\hat{a}^\dagger_{1,2})/2$. 
%In this section, we will pay attention to the physical meaning and properties of this operator.
$\hat{\eta}$ is an Hermitian operator so its expectation value $\langle\hat{\eta}\rangle$ is a real number, so we can call it exchange factor.
According to the Eq. (6), $\langle\hat{\eta}\rangle$ can characterize the exchange between two modes, that is, $\langle\hat{\eta}\rangle>0$  indicates the energy (photon) is flowing from mode 2 to mode 1 and vise versa.
However, as an important  physical quantity in PT-symmetry, exchange factor has not been defined or studied yet in the previous literature.
For convenience, here we directly use letters without hats such as $\eta, X_1, Y_2$ to represent the expectation values of the corresponding operators, i.e., $\eta=\langle\hat{\eta}\rangle, X_1=\langle\hat{X_1}\rangle, Y_2=\langle\hat{Y_2}\rangle$

Fig. 2(a) gives the evolution of $\eta$ with varying the loss rate $\gamma_1$.
Here, $ \beta_1=0.1, \gamma_2=0.1, \beta_2=0.1, \mu=1$.
$\eta$  experiences the phase of PT symmetry with $\gamma_1=1.1$, through the EP with $\gamma_1=2.1$, to PT broken with $\gamma_1=3.1$, corresponding to the yellow star, gray star and red star in Fig. 1(b), respectively.
It is seen that, when PT symmetry is unbroken, $\eta$  oscillates with varying $t$. In contrast, when PT symmetry is broken, $\eta$ monotonically decreases after a rise and then comes to steady state.
Therefore, exchange factor $\eta$ can fully characterize the properties of PT symmetry or broken.
Moreover, according to the Eq. (7), considering a specific situation, if $\gamma_2=\beta_2$, then the steady state value of exchange factor will be
\begin{equation}
\eta_{ss}=\frac{2\gamma_2}{\mu}.
\end{equation}
It will only depends on $\gamma_2$ and $\mu$, but have nothing to do with $\gamma_1$ and $\beta_1$.
For example, in Fig. 2(a), every curve finally approaches to a fixed value 0.2, no matter how large the loss of mode 1 is. 
Especially, the blue curve corresponding to an EP point is the fastest one to come to steady state,  which is a counter-intuitive phenomenon.

%Now let us introduce several quadrature amplitude operators,
%They are common Hermitian operators to describe the boson modes, which can also characterize PT symmetry or broken.
Correspondingly, we explore the exchanging processing from PT-symmetry to broken in the form of the expectation values of $\hat{X}_{1,2}$ and $\hat{Y}_{1,2}$.
From Eq. (8),  the exchange occurs between $X_1$ and $Y_2$, or between $X_2$ and $Y_1$.
For the PT-symmetry regime, the quadrature amplitudes $X_{1,2}, Y_{1,2}$ oscillate continuously in Fig. 2(b).
While for PT-broken, they decay exponential when $t>2$ as shown in Fig. 2(c).
Inputting the Fock state $\ket{n}$, the initial values of $X_{1,2}, Y_{1,2}$ are all 0.
According to the Eq. (3), at any time $X_{1,2}, Y_{1,2}$ will equal to 0. So only using the exchange between quadrature amplitudes,  one can not distinguish  the PT-symmetry or broken if only inputting Fock states.
%Only when we input coherent states, the quadrature amplitudes can characterize PT phase like Fig. 2(b)(c). (Sup. Mat. 5)
While, whatever for Fock states or coherent states, one can clearly distinguish them through the exchange factor $\eta$.

\begin{figure*}[htb]
  \centering
  % Requires \usepackage{graphicx}
  \includegraphics[width=\textwidth]{Fig3.pdf}\\
 \caption{Quantum state engineering based on PT symmetry or PT broken. Schematics of coupled waveguides with input Fock states (a) $|3\rangle|6\rangle$ and (b) $|6\rangle|3\rangle$ input. In the case of (a), photon number distribution of quantum state at $z=0.3cm$  for (c) PT-symmetry and (d) PT broken.  
 In the case of (b), photon number distribution of quantum state at $z=0.5cm$  for (e) PT-symmetry and (f) PT broken.  
 Here, the parameters for (c) and (e) are the same as Fig. 2(b), corresponding the yellow star in Fig. 1(b). The parameters for (d) and (f) are the same as Fig. 2(c), corresponding to the red star in Fig. 1(b).}
\label{fig3}
\end{figure*}

\section{IV. Quantum State Engineering}

The above theory is applicable to two-mode coupling in various photonic structures. In this section, let's take the coupled waveguides as an example to study the quantum state transmission or engineering.
To start with, we need to change the evolution time $t$ to the propagation distance $z$ in the waveguides.
% and the unit of related parameters like loss rates will become $cm^{-1}$ rather than GHz.
Shown as in Fig. 3(a), the coupled waveguides  with gain and loss can be looked as a non-Hermitian beam splitter. 
If we input quantum states from two left ports, transformed quantum states will be output from two right ports, with a part of photons absorbed or generated by material.
In the following, we will mainly  study the quantum state transmission  in the full-parameter space of our quantum PT phase diagram.

First, we consider the case that there is no more than two photons input.
For PT symmetry and PT broken, the transmission behavior will be different. This reflects the concept of "dominant waveguide". 
If PT is broken, photons will mainly concentrate on one of the waveguides, so that the output state basically contains only two states, $\ket{01}$ and $\ket{02}$. This waveguide is called "dominant waveguide". From $\eta$ part of the Eq. (6) and Eq. (7), we can also find that which one is the "dominate waveguide" depends on the ratio of gain rate and loss rate $\beta_i/\gamma_i$ . 
However, PT symmetry leads to a mixture of multiple results, and the probability of $\ket{01}$, $\ket{02}$, $\ket{10}$ and $\ket{20}$ will all exceed 10\%. The concept of "dominant waveguide" will be absent. 

Then consider the case that each port inputs one photon, that is, the HOM experiment. We find that there is an obvious depression in the probability of $\ket{11}$ states under any parameter, and the depression probability can reach 0 when only with loss, which reflects the quantum interference phenomenon between waveguides. This special case can be analytically solved by Lie algebra method \cite{2019.NP}. 
Moreover, in the PT broken regime, we can directly (without post-selection) obtain the mixed states of Fock state $\ket{1}$ and $\ket{2}$ from two Fock states $\ket{1}$ using this passive structure.
In the literature \cite{2012.PRA}, vacuum state, single photon state and two photons NOON state are inputted in the gain-loss coupled waveguides. Then photons will be spontaneously generated with non-classical characters. 

%But the authors supposed that losses can be compensated by gain mechanisms. This statement is inaccurate so we compare the case that gain and loss both exist with that only gain exists, and we find the former case will soon collapse to steady state. 

Next, let's consider Fock state $\ket{M,N}$ with more photons, with the total number of photons $M+N<10$.
We study the output quantum states in the case of PT symmetry or broken.
It should be emphasized that the output quantum states here are not the steady states obtained by transmitting a long enough distance, but are states directly extracted after propagating to a certain distance. 
This distance is usually slightly smaller than the photon exchange distance $z_0=\pi/4\mu$.
On this distance, the quantum state has been well transformed by the structure and has not lost too much quantum feature.
As shown in Fig. 3, in the PT symmetry regime, the photon number distribution is dispersed. It is due to photons are exchanging constantly between two waveguides. 
While in the PT broken regime, the photon number distribution is concentrative, that is because most photons are gathered in "dominate waveguide?, which is very useful for preparing quantum states.

With the same parameters, and the same total photon number 9, there is some difference between input $\ket{36}$ and $\ket{63}$. 
The number of photons in the "dominant waveguide" plays a leading role in the output quantum state, while the number of photons in the "inferior waveguide" is only a perturbation.
In Fig. 3(d), the highest probability state is $\ket{07}$, while in Fig. 3(f), it is $\ket{04}$.
So after a photon exchange distance $z_0$, no matter how many photons there are at the beginning in waveguide 1, waveguide 2 can only receive one photon. And all the photons in waveguide 1 disappear. We call it "multiple-single photon exchange".
%We verify many other cases, and we also get the above conclusion. (Sup. Mat.)
Therefore, according to the "multiple-single photon exchange", if we want a quantum state that mainly contains $\ket{0,N+1}$, then we can choose $\ket{M,N}$ to input.
%The final value of $M$ could be further modulated, and by adjusting the loss and gain parameters, more optimized results may be obtained.
This method of quantum states engineering has potential to apply to multi-photon state generation, quantum gate and so on.

\section{V. Summary}

In summary, we have analytically obtained the quantum PT phase diagram with the steady state regime in non-Hermitian  photonic structures. 
We have defined an exchange operator to characterize the PT-symmetry phase and PT-broken phase.
Based on this phase diagram, we have engineered the multi-photon quantum state in the coupled waveguides structure.
The present work has constructed the basic theory of quantum PT symmetry bi-photonic structure as well as its application to quantum state engineering.
The established theory can be extended to study many related quantum behaviors, such as gain saturation effect, quantum entanglement, and continuous variable states, and may have potential applications in quantum interference, on-chip quantum information processing and scalable quantum networks. 


\acknowledgments
\textit{Acknowledgments.} This work is supported by the National Natural Science Foundation of China under Grants Nos. 11974032,  11525414, and 11734001, and by the Key R$\&$D Program of Guangdong Province under Grant No. 2018B030329001.

% \bibliography{total}

%apsrev4-2.bst 2019-01-14 (MD) hand-edited version of apsrev4-1.bst
%Control: key (0)
%Control: author (8) initials jnrlst
%Control: editor formatted (1) identically to author
%Control: production of article title (0) allowed
%Control: page (0) single
%Control: year (1) truncated
%Control: production of eprint (0) enabled
\begin{thebibliography}{38}%
\makeatletter
\providecommand \@ifxundefined [1]{%
 \@ifx{#1\undefined}
}%
\providecommand \@ifnum [1]{%
 \ifnum #1\expandafter \@firstoftwo
 \else \expandafter \@secondoftwo
 \fi
}%
\providecommand \@ifx [1]{%
 \ifx #1\expandafter \@firstoftwo
 \else \expandafter \@secondoftwo
 \fi
}%
\providecommand \natexlab [1]{#1}%
\providecommand \enquote  [1]{``#1''}%
\providecommand \bibnamefont  [1]{#1}%
\providecommand \bibfnamefont [1]{#1}%
\providecommand \citenamefont [1]{#1}%
\providecommand \href@noop [0]{\@secondoftwo}%
\providecommand \href [0]{\begingroup \@sanitize@url \@href}%
\providecommand \@href[1]{\@@startlink{#1}\@@href}%
\providecommand \@@href[1]{\endgroup#1\@@endlink}%
\providecommand \@sanitize@url [0]{\catcode `\\12\catcode `\$12\catcode
  `\&12\catcode `\#12\catcode `\^12\catcode `\_12\catcode `\%12\relax}%
\providecommand \@@startlink[1]{}%
\providecommand \@@endlink[0]{}%
\providecommand \url  [0]{\begingroup\@sanitize@url \@url }%
\providecommand \@url [1]{\endgroup\@href {#1}{\urlprefix }}%
\providecommand \urlprefix  [0]{URL }%
\providecommand \Eprint [0]{\href }%
\providecommand \doibase [0]{https://doi.org/}%
\providecommand \selectlanguage [0]{\@gobble}%
\providecommand \bibinfo  [0]{\@secondoftwo}%
\providecommand \bibfield  [0]{\@secondoftwo}%
\providecommand \translation [1]{[#1]}%
\providecommand \BibitemOpen [0]{}%
\providecommand \bibitemStop [0]{}%
\providecommand \bibitemNoStop [0]{.\EOS\space}%
\providecommand \EOS [0]{\spacefactor3000\relax}%
\providecommand \BibitemShut  [1]{\csname bibitem#1\endcsname}%
\let\auto@bib@innerbib\@empty
%</preamble>
\bibitem [{\citenamefont {Ashida}\ \emph {et~al.}(2020)\citenamefont {Ashida},
  \citenamefont {Gong},\ and\ \citenamefont {Ueda}}]{2020.AdvPhys}%
  \BibitemOpen
  \bibfield  {author} {\bibinfo {author} {\bibfnamefont {Y.}~\bibnamefont
  {Ashida}}, \bibinfo {author} {\bibfnamefont {Z.}~\bibnamefont {Gong}},\ and\
  \bibinfo {author} {\bibfnamefont {M.}~\bibnamefont {Ueda}},\ }\bibfield
  {title} {\bibinfo {title} {Non-hermitian physics},\ }\href@noop {} {\bibfield
   {journal} {\bibinfo  {journal} {Advances in Physics}\ }\textbf {\bibinfo
  {volume} {69}},\ \bibinfo {pages} {249} (\bibinfo {year} {2020})}\BibitemShut
  {NoStop}%
\bibitem [{\citenamefont {Bender}\ and\ \citenamefont
  {Boettcher}(1998)}]{1998.PhysRevLett}%
  \BibitemOpen
  \bibfield  {author} {\bibinfo {author} {\bibfnamefont {C.~M.}\ \bibnamefont
  {Bender}}\ and\ \bibinfo {author} {\bibfnamefont {S.}~\bibnamefont
  {Boettcher}},\ }\bibfield  {title} {\bibinfo {title} {Real spectra in
  non-hermitian hamiltonians having $\mathcal{PT}$ symmetry},\ }\href@noop {}
  {\bibfield  {journal} {\bibinfo  {journal} {Phys. Rev. Lett.}\ }\textbf
  {\bibinfo {volume} {80}},\ \bibinfo {pages} {5243} (\bibinfo {year}
  {1998})}\BibitemShut {NoStop}%
\bibitem [{\citenamefont {El-Ganainy}\ \emph {et~al.}(2018)\citenamefont
  {El-Ganainy}, \citenamefont {Makris}, \citenamefont {Khajavikhan} \emph
  {et~al.}}]{2017.NatPhys}%
  \BibitemOpen
  \bibfield  {author} {\bibinfo {author} {\bibfnamefont {R.}~\bibnamefont
  {El-Ganainy}}, \bibinfo {author} {\bibfnamefont {K.~G.}\ \bibnamefont
  {Makris}}, \bibinfo {author} {\bibfnamefont {M.}~\bibnamefont {Khajavikhan}},
  \emph {et~al.},\ }\bibfield  {title} {\bibinfo {title} {Non-hermitian physics
  and $\mathcal{PT}$ symmetry},\ }\href@noop {} {\bibfield  {journal} {\bibinfo
   {journal} {Nature Physics}\ }\textbf {\bibinfo {volume} {14}},\ \bibinfo
  {pages} {11} (\bibinfo {year} {2018})}\BibitemShut {NoStop}%
\bibitem [{\citenamefont {Feng}\ \emph {et~al.}(2017)\citenamefont {Feng},
  \citenamefont {El-Ganainy},\ and\ \citenamefont {Ge}}]{2017.NatPhoto}%
  \BibitemOpen
  \bibfield  {author} {\bibinfo {author} {\bibfnamefont {L.}~\bibnamefont
  {Feng}}, \bibinfo {author} {\bibfnamefont {R.}~\bibnamefont {El-Ganainy}},\
  and\ \bibinfo {author} {\bibfnamefont {L.}~\bibnamefont {Ge}},\ }\bibfield
  {title} {\bibinfo {title} {Non-hermitian photonics based on parity–time
  symmetry},\ }\href@noop {} {\bibfield  {journal} {\bibinfo  {journal} {Nature
  Photonics}\ }\textbf {\bibinfo {volume} {11}},\ \bibinfo {pages} {752}
  (\bibinfo {year} {2017})}\BibitemShut {NoStop}%
\bibitem [{\citenamefont {Miri}\ and\ \citenamefont
  {Al{\`u}}(2019)}]{2019.science}%
  \BibitemOpen
  \bibfield  {author} {\bibinfo {author} {\bibfnamefont {M.-A.}\ \bibnamefont
  {Miri}}\ and\ \bibinfo {author} {\bibfnamefont {A.}~\bibnamefont {Al{\`u}}},\
  }\bibfield  {title} {\bibinfo {title} {Exceptional points in optics and
  photonics},\ }\href@noop {} {\bibfield  {journal} {\bibinfo  {journal}
  {Science}\ }\textbf {\bibinfo {volume} {363}},\ \bibinfo {pages} {eaar7709}
  (\bibinfo {year} {2019})}\BibitemShut {NoStop}%
\bibitem [{\citenamefont {El-Ganainy}\ \emph {et~al.}(2007)\citenamefont
  {El-Ganainy}, \citenamefont {Makris}, \citenamefont {Christodoulides},\ and\
  \citenamefont {Musslimani}}]{2007.OpticsLett}%
  \BibitemOpen
  \bibfield  {author} {\bibinfo {author} {\bibfnamefont {R.}~\bibnamefont
  {El-Ganainy}}, \bibinfo {author} {\bibfnamefont {K.~G.}\ \bibnamefont
  {Makris}}, \bibinfo {author} {\bibfnamefont {D.~N.}\ \bibnamefont
  {Christodoulides}},\ and\ \bibinfo {author} {\bibfnamefont {Z.~H.}\
  \bibnamefont {Musslimani}},\ }\bibfield  {title} {\bibinfo {title} {Theory of
  coupled optical $\mathcal{PT}$-symmetric structures},\ }\href@noop {}
  {\bibfield  {journal} {\bibinfo  {journal} {Opt. Lett.}\ }\textbf {\bibinfo
  {volume} {32}},\ \bibinfo {pages} {2632} (\bibinfo {year}
  {2007})}\BibitemShut {NoStop}%
\bibitem [{\citenamefont {Guo}\ \emph {et~al.}(2009)\citenamefont {Guo},
  \citenamefont {Salamo}, \citenamefont {Duchesne} \emph
  {et~al.}}]{2009.PhysRevLett}%
  \BibitemOpen
  \bibfield  {author} {\bibinfo {author} {\bibfnamefont {A.}~\bibnamefont
  {Guo}}, \bibinfo {author} {\bibfnamefont {G.~J.}\ \bibnamefont {Salamo}},
  \bibinfo {author} {\bibfnamefont {D.}~\bibnamefont {Duchesne}}, \emph
  {et~al.},\ }\bibfield  {title} {\bibinfo {title} {Observation of
  $\mathcal{P}\mathcal{T}$-symmetry breaking in complex optical potentials},\
  }\href@noop {} {\bibfield  {journal} {\bibinfo  {journal} {Phys. Rev. Lett.}\
  }\textbf {\bibinfo {volume} {103}},\ \bibinfo {pages} {093902} (\bibinfo
  {year} {2009})}\BibitemShut {NoStop}%
\bibitem [{\citenamefont {R{\"u}ter}\ \emph {et~al.}(2010)\citenamefont
  {R{\"u}ter}, \citenamefont {Makris}, \citenamefont {El-Ganainy} \emph
  {et~al.}}]{2010.NatPhys}%
  \BibitemOpen
  \bibfield  {author} {\bibinfo {author} {\bibfnamefont {C.~E.}\ \bibnamefont
  {R{\"u}ter}}, \bibinfo {author} {\bibfnamefont {K.~G.}\ \bibnamefont
  {Makris}}, \bibinfo {author} {\bibfnamefont {R.}~\bibnamefont {El-Ganainy}},
  \emph {et~al.},\ }\bibfield  {title} {\bibinfo {title} {Observation of
  parity–time symmetry in optics},\ }\href@noop {} {\bibfield  {journal}
  {\bibinfo  {journal} {Nature Physics}\ }\textbf {\bibinfo {volume} {6}},\
  \bibinfo {pages} {192} (\bibinfo {year} {2010})}\BibitemShut {NoStop}%
\bibitem [{\citenamefont {Chang}\ \emph {et~al.}(2014)\citenamefont {Chang},
  \citenamefont {Jiang}, \citenamefont {Hua} \emph {et~al.}}]{2014.NatPhoto}%
  \BibitemOpen
  \bibfield  {author} {\bibinfo {author} {\bibfnamefont {L.}~\bibnamefont
  {Chang}}, \bibinfo {author} {\bibfnamefont {X.}~\bibnamefont {Jiang}},
  \bibinfo {author} {\bibfnamefont {S.}~\bibnamefont {Hua}}, \emph {et~al.},\
  }\bibfield  {title} {\bibinfo {title} {Parity–time symmetry and variable
  optical isolation in active–passive-coupled microresonators},\ }\href@noop
  {} {\bibfield  {journal} {\bibinfo  {journal} {Nature Photonics}\ }\textbf
  {\bibinfo {volume} {8}},\ \bibinfo {pages} {524} (\bibinfo {year}
  {2014})}\BibitemShut {NoStop}%
\bibitem [{\citenamefont {Peng}\ \emph
  {et~al.}(2014{\natexlab{a}})\citenamefont {Peng}, \citenamefont
  {{\"O}zdemir}, \citenamefont {Lei} \emph {et~al.}}]{2014.NatPhys}%
  \BibitemOpen
  \bibfield  {author} {\bibinfo {author} {\bibfnamefont {B.}~\bibnamefont
  {Peng}}, \bibinfo {author} {\bibfnamefont {{\c{S}}.~K.}\ \bibnamefont
  {{\"O}zdemir}}, \bibinfo {author} {\bibfnamefont {F.}~\bibnamefont {Lei}},
  \emph {et~al.},\ }\bibfield  {title} {\bibinfo {title} {Parity-time-symmetric
  whispering-gallery microcavities},\ }\href@noop {} {\bibfield  {journal}
  {\bibinfo  {journal} {Nature Physics}\ }\textbf {\bibinfo {volume} {10}},\
  \bibinfo {pages} {394} (\bibinfo {year} {2014}{\natexlab{a}})}\BibitemShut
  {NoStop}%
\bibitem [{\citenamefont {Regensburger}\ \emph {et~al.}(2012)\citenamefont
  {Regensburger}, \citenamefont {Bersch}, \citenamefont {Miri} \emph
  {et~al.}}]{2012.Nature}%
  \BibitemOpen
  \bibfield  {author} {\bibinfo {author} {\bibfnamefont {A.}~\bibnamefont
  {Regensburger}}, \bibinfo {author} {\bibfnamefont {C.}~\bibnamefont
  {Bersch}}, \bibinfo {author} {\bibfnamefont {M.-A.}\ \bibnamefont {Miri}},
  \emph {et~al.},\ }\bibfield  {title} {\bibinfo {title} {Parity-time synthetic
  photonic lattice},\ }\href@noop {} {\bibfield  {journal} {\bibinfo  {journal}
  {Nature}\ }\textbf {\bibinfo {volume} {488}},\ \bibinfo {pages} {167}
  (\bibinfo {year} {2012})}\BibitemShut {NoStop}%
\bibitem [{\citenamefont {Lawrence}\ \emph {et~al.}(2014)\citenamefont
  {Lawrence}, \citenamefont {Xu}, \citenamefont {Zhang} \emph
  {et~al.}}]{2014.PhysRevLett.(Manifestation}%
  \BibitemOpen
  \bibfield  {author} {\bibinfo {author} {\bibfnamefont {M.}~\bibnamefont
  {Lawrence}}, \bibinfo {author} {\bibfnamefont {N.}~\bibnamefont {Xu}},
  \bibinfo {author} {\bibfnamefont {X.}~\bibnamefont {Zhang}}, \emph {et~al.},\
  }\bibfield  {title} {\bibinfo {title} {Manifestation of $\mathcal{PT}$
  symmetry breaking in polarization space with terahertz metasurfaces},\
  }\href@noop {} {\bibfield  {journal} {\bibinfo  {journal} {Phys. Rev. Lett.}\
  }\textbf {\bibinfo {volume} {113}},\ \bibinfo {pages} {093901} (\bibinfo
  {year} {2014})}\BibitemShut {NoStop}%
\bibitem [{\citenamefont {Feng}\ \emph {et~al.}(2011)\citenamefont {Feng},
  \citenamefont {Ayache}, \citenamefont {Huang} \emph {et~al.}}]{2011.Science}%
  \BibitemOpen
  \bibfield  {author} {\bibinfo {author} {\bibfnamefont {L.}~\bibnamefont
  {Feng}}, \bibinfo {author} {\bibfnamefont {M.}~\bibnamefont {Ayache}},
  \bibinfo {author} {\bibfnamefont {J.}~\bibnamefont {Huang}}, \emph {et~al.},\
  }\bibfield  {title} {\bibinfo {title} {Nonreciprocal light propagation in a
  silicon photonic circuit},\ }\href@noop {} {\bibfield  {journal} {\bibinfo
  {journal} {Science}\ }\textbf {\bibinfo {volume} {333}},\ \bibinfo {pages}
  {729} (\bibinfo {year} {2011})}\BibitemShut {NoStop}%
\bibitem [{\citenamefont {Lin}\ \emph {et~al.}(2011)\citenamefont {Lin},
  \citenamefont {Ramezani}, \citenamefont {Eichelkraut} \emph
  {et~al.}}]{2011.PhysRevLett}%
  \BibitemOpen
  \bibfield  {author} {\bibinfo {author} {\bibfnamefont {Z.}~\bibnamefont
  {Lin}}, \bibinfo {author} {\bibfnamefont {H.}~\bibnamefont {Ramezani}},
  \bibinfo {author} {\bibfnamefont {T.}~\bibnamefont {Eichelkraut}}, \emph
  {et~al.},\ }\bibfield  {title} {\bibinfo {title} {Unidirectional invisibility
  induced by $\mathcal{P}\mathcal{T}$-symmetric periodic structures},\
  }\href@noop {} {\bibfield  {journal} {\bibinfo  {journal} {Phys. Rev. Lett.}\
  }\textbf {\bibinfo {volume} {106}},\ \bibinfo {pages} {213901} (\bibinfo
  {year} {2011})}\BibitemShut {NoStop}%
\bibitem [{\citenamefont {Wiersig}(2014)}]{2014.PRL}%
  \BibitemOpen
  \bibfield  {author} {\bibinfo {author} {\bibfnamefont {J.}~\bibnamefont
  {Wiersig}},\ }\bibfield  {title} {\bibinfo {title} {Enhancing the sensitivity
  of frequency and energy splitting detection by using exceptional points:
  Application to microcavity sensors for single-particle detection},\
  }\href@noop {} {\bibfield  {journal} {\bibinfo  {journal} {Phys. Rev. Lett.}\
  }\textbf {\bibinfo {volume} {112}},\ \bibinfo {pages} {203901} (\bibinfo
  {year} {2014})}\BibitemShut {NoStop}%
\bibitem [{\citenamefont {Peng}\ \emph
  {et~al.}(2014{\natexlab{b}})\citenamefont {Peng}, \citenamefont
  {{\"O}zdemir}, \citenamefont {Rotter} \emph
  {et~al.}}]{2014.science.(Loss-induced}%
  \BibitemOpen
  \bibfield  {author} {\bibinfo {author} {\bibfnamefont {B.}~\bibnamefont
  {Peng}}, \bibinfo {author} {\bibfnamefont {{\c{S}}.~K.}\ \bibnamefont
  {{\"O}zdemir}}, \bibinfo {author} {\bibfnamefont {S.}~\bibnamefont {Rotter}},
  \emph {et~al.},\ }\bibfield  {title} {\bibinfo {title} {Loss-induced
  suppression and revival of lasing},\ }\href@noop {} {\bibfield  {journal}
  {\bibinfo  {journal} {Science}\ }\textbf {\bibinfo {volume} {346}},\ \bibinfo
  {pages} {328} (\bibinfo {year} {2014}{\natexlab{b}})}\BibitemShut {NoStop}%
\bibitem [{\citenamefont {Feng}\ \emph {et~al.}(2014)\citenamefont {Feng},
  \citenamefont {Wong}, \citenamefont {Ma}, \citenamefont {Wang},\ and\
  \citenamefont {Zhang}}]{2014.science.(Single}%
  \BibitemOpen
  \bibfield  {author} {\bibinfo {author} {\bibfnamefont {L.}~\bibnamefont
  {Feng}}, \bibinfo {author} {\bibfnamefont {Z.~J.}\ \bibnamefont {Wong}},
  \bibinfo {author} {\bibfnamefont {R.-M.}\ \bibnamefont {Ma}}, \bibinfo
  {author} {\bibfnamefont {Y.}~\bibnamefont {Wang}},\ and\ \bibinfo {author}
  {\bibfnamefont {X.}~\bibnamefont {Zhang}},\ }\bibfield  {title} {\bibinfo
  {title} {Single-mode laser by parity-time symmetry breaking},\ }\href@noop {}
  {\bibfield  {journal} {\bibinfo  {journal} {Science}\ }\textbf {\bibinfo
  {volume} {346}},\ \bibinfo {pages} {972} (\bibinfo {year}
  {2014})}\BibitemShut {NoStop}%
\bibitem [{\citenamefont {Peng}\ \emph {et~al.}(2016)\citenamefont {Peng},
  \citenamefont {{\"O}zdemir}, \citenamefont {Liertzer} \emph
  {et~al.}}]{2016.PNAS}%
  \BibitemOpen
  \bibfield  {author} {\bibinfo {author} {\bibfnamefont {B.}~\bibnamefont
  {Peng}}, \bibinfo {author} {\bibfnamefont {{\c{S}}.~K.}\ \bibnamefont
  {{\"O}zdemir}}, \bibinfo {author} {\bibfnamefont {M.}~\bibnamefont
  {Liertzer}}, \emph {et~al.},\ }\bibfield  {title} {\bibinfo {title} {Chiral
  modes and directional lasing at exceptional points},\ }\href@noop {}
  {\bibfield  {journal} {\bibinfo  {journal} {Proc Natl Acad Sci U S A.}\
  }\textbf {\bibinfo {volume} {113(25)}},\ \bibinfo {pages} {6845} (\bibinfo
  {year} {2016})}\BibitemShut {NoStop}%
\bibitem [{\citenamefont {Minganti}\ \emph {et~al.}(2019)\citenamefont
  {Minganti}, \citenamefont {Miranowicz}, \citenamefont {Chhajlany},\ and\
  \citenamefont {Nori}}]{2019.PRA.(QuantumExceptionalPoints}%
  \BibitemOpen
  \bibfield  {author} {\bibinfo {author} {\bibfnamefont {F.}~\bibnamefont
  {Minganti}}, \bibinfo {author} {\bibfnamefont {A.}~\bibnamefont
  {Miranowicz}}, \bibinfo {author} {\bibfnamefont {R.~W.}\ \bibnamefont
  {Chhajlany}},\ and\ \bibinfo {author} {\bibfnamefont {F.}~\bibnamefont
  {Nori}},\ }\bibfield  {title} {\bibinfo {title} {Quantum exceptional points
  of non-hermitian hamiltonians and liouvillians: The effects of quantum
  jumps},\ }\href@noop {} {\bibfield  {journal} {\bibinfo  {journal} {Phys.
  Rev. A}\ }\textbf {\bibinfo {volume} {100}},\ \bibinfo {pages} {062131}
  (\bibinfo {year} {2019})}\BibitemShut {NoStop}%
\bibitem [{\citenamefont {Scheel}\ and\ \citenamefont
  {Szameit}(2018)}]{2018.EPL}%
  \BibitemOpen
  \bibfield  {author} {\bibinfo {author} {\bibfnamefont {S.}~\bibnamefont
  {Scheel}}\ and\ \bibinfo {author} {\bibfnamefont {A.}~\bibnamefont
  {Szameit}},\ }\bibfield  {title} {\bibinfo {title} {$\mathcal{PT}$-symmetric
  photonic quantum systems with gain and loss do not exist},\ }\href@noop {}
  {\bibfield  {journal} {\bibinfo  {journal} {EPL}\ }\textbf {\bibinfo {volume}
  {122}},\ \bibinfo {pages} {34001} (\bibinfo {year} {2018})}\BibitemShut
  {NoStop}%
\bibitem [{\citenamefont {Caves}(1982)}]{1982.PRD}%
  \BibitemOpen
  \bibfield  {author} {\bibinfo {author} {\bibfnamefont {C.~M.}\ \bibnamefont
  {Caves}},\ }\bibfield  {title} {\bibinfo {title} {Quantum limits on noise in
  linear amplifiers},\ }\href@noop {} {\bibfield  {journal} {\bibinfo
  {journal} {Phys. Rev. D}\ }\textbf {\bibinfo {volume} {26}},\ \bibinfo
  {pages} {1817} (\bibinfo {year} {1982})}\BibitemShut {NoStop}%
\bibitem [{\citenamefont {Huber}\ \emph {et~al.}(2020)\citenamefont {Huber},
  \citenamefont {Kirton}, \citenamefont {Rotter},\ and\ \citenamefont
  {Rabl}}]{2020.SciPost}%
  \BibitemOpen
  \bibfield  {author} {\bibinfo {author} {\bibfnamefont {J.}~\bibnamefont
  {Huber}}, \bibinfo {author} {\bibfnamefont {P.}~\bibnamefont {Kirton}},
  \bibinfo {author} {\bibfnamefont {S.}~\bibnamefont {Rotter}},\ and\ \bibinfo
  {author} {\bibfnamefont {P.}~\bibnamefont {Rabl}},\ }\bibfield  {title}
  {\bibinfo {title} {{Emergence of $\mathcal{PT}$-symmetry breaking in open
  quantum systems}},\ }\href@noop {} {\bibfield  {journal} {\bibinfo  {journal}
  {SciPost Phys.}\ }\textbf {\bibinfo {volume} {9}},\ \bibinfo {pages} {052}
  (\bibinfo {year} {2020})}\BibitemShut {NoStop}%
\bibitem [{\citenamefont {O'brien}\ \emph {et~al.}(2009)\citenamefont
  {O'brien}, \citenamefont {Furusawa},\ and\ \citenamefont
  {Vu{\v{c}}kovi{\'c}}}]{2009.NatPhoto}%
  \BibitemOpen
  \bibfield  {author} {\bibinfo {author} {\bibfnamefont {J.~L.}\ \bibnamefont
  {O'brien}}, \bibinfo {author} {\bibfnamefont {A.}~\bibnamefont {Furusawa}},\
  and\ \bibinfo {author} {\bibfnamefont {J.}~\bibnamefont
  {Vu{\v{c}}kovi{\'c}}},\ }\bibfield  {title} {\bibinfo {title} {Photonic
  quantum technologies},\ }\href@noop {} {\bibfield  {journal} {\bibinfo
  {journal} {Nature Photonics}\ }\textbf {\bibinfo {volume} {3}},\ \bibinfo
  {pages} {687} (\bibinfo {year} {2009})}\BibitemShut {NoStop}%
\bibitem [{\citenamefont {Lindblad}(1976)}]{1976.CMP}%
  \BibitemOpen
  \bibfield  {author} {\bibinfo {author} {\bibfnamefont {G.}~\bibnamefont
  {Lindblad}},\ }\bibfield  {title} {\bibinfo {title} {On the generators of
  quantum dynamical semigroups},\ }\href@noop {} {\bibfield  {journal}
  {\bibinfo  {journal} {Communications in Mathematical Physics}\ }\textbf
  {\bibinfo {volume} {48}},\ \bibinfo {pages} {119} (\bibinfo {year}
  {1976})}\BibitemShut {NoStop}%
\bibitem [{\citenamefont {Agarwal}\ and\ \citenamefont {Qu}(2012)}]{2012.PRA}%
  \BibitemOpen
  \bibfield  {author} {\bibinfo {author} {\bibfnamefont {G.~S.}\ \bibnamefont
  {Agarwal}}\ and\ \bibinfo {author} {\bibfnamefont {K.}~\bibnamefont {Qu}},\
  }\bibfield  {title} {\bibinfo {title} {Spontaneous generation of photons in
  transmission of quantum fields in $\mathcal{PT}$-symmetric optical systems},\
  }\href@noop {} {\bibfield  {journal} {\bibinfo  {journal} {Phys. Rev. A}\
  }\textbf {\bibinfo {volume} {85}},\ \bibinfo {pages} {031802} (\bibinfo
  {year} {2012})}\BibitemShut {NoStop}%
\bibitem [{\citenamefont {Huerta~Morales}\ and\ \citenamefont
  {Rodríguez-Lara}(2017)}]{2017.ApplSci}%
  \BibitemOpen
  \bibfield  {author} {\bibinfo {author} {\bibfnamefont {J.~D.}\ \bibnamefont
  {Huerta~Morales}}\ and\ \bibinfo {author} {\bibfnamefont {B.~M.}\
  \bibnamefont {Rodríguez-Lara}},\ }\bibfield  {title} {\bibinfo {title}
  {Photon propagation through linearly active dimers},\ }\href@noop {}
  {\bibfield  {journal} {\bibinfo  {journal} {Applied Sciences}\ }\textbf
  {\bibinfo {volume} {7}} (\bibinfo {year} {2017})}\BibitemShut {NoStop}%
\bibitem [{\citenamefont {Klauck}\ \emph {et~al.}(2019)\citenamefont {Klauck},
  \citenamefont {Teuber}, \citenamefont {Ornigotti} \emph {et~al.}}]{2019.NP}%
  \BibitemOpen
  \bibfield  {author} {\bibinfo {author} {\bibfnamefont {F.}~\bibnamefont
  {Klauck}}, \bibinfo {author} {\bibfnamefont {L.}~\bibnamefont {Teuber}},
  \bibinfo {author} {\bibfnamefont {M.}~\bibnamefont {Ornigotti}}, \emph
  {et~al.},\ }\bibfield  {title} {\bibinfo {title} {Observation of
  $\mathcal{PT}$-symmetric quantum interference},\ }\href@noop {} {\bibfield
  {journal} {\bibinfo  {journal} {Nature Photonics}\ }\textbf {\bibinfo
  {volume} {13}},\ \bibinfo {pages} {883} (\bibinfo {year} {2019})}\BibitemShut
  {NoStop}%
\bibitem [{\citenamefont {Longhi}(2018)}]{2018.OL}%
  \BibitemOpen
  \bibfield  {author} {\bibinfo {author} {\bibfnamefont {S.}~\bibnamefont
  {Longhi}},\ }\bibfield  {title} {\bibinfo {title} {Quantum interference and
  exceptional points},\ }\href@noop {} {\bibfield  {journal} {\bibinfo
  {journal} {Opt. Lett.}\ }\textbf {\bibinfo {volume} {43}},\ \bibinfo {pages}
  {5371} (\bibinfo {year} {2018})}\BibitemShut {NoStop}%
\bibitem [{\citenamefont {Zhou}(2022)}]{2022.OE}%
  \BibitemOpen
  \bibfield  {author} {\bibinfo {author} {\bibfnamefont {J.}~\bibnamefont
  {Zhou}},\ }\bibfield  {title} {\bibinfo {title} {Characterization of
  $\mathcal{PT}$-symmetric quantum interference based on the coupled mode
  theory},\ }\href@noop {} {\bibfield  {journal} {\bibinfo  {journal} {Opt.
  Express}\ }\textbf {\bibinfo {volume} {30}},\ \bibinfo {pages} {23600}
  (\bibinfo {year} {2022})}\BibitemShut {NoStop}%
\bibitem [{\citenamefont {Rai}\ \emph {et~al.}(2010)\citenamefont {Rai},
  \citenamefont {Das},\ and\ \citenamefont {Agarwal}}]{2010.OE}%
  \BibitemOpen
  \bibfield  {author} {\bibinfo {author} {\bibfnamefont {A.}~\bibnamefont
  {Rai}}, \bibinfo {author} {\bibfnamefont {S.}~\bibnamefont {Das}},\ and\
  \bibinfo {author} {\bibfnamefont {G.~S.}\ \bibnamefont {Agarwal}},\
  }\bibfield  {title} {\bibinfo {title} {Quantum entanglement in coupled lossy
  waveguides},\ }\href@noop {} {\bibfield  {journal} {\bibinfo  {journal} {Opt.
  Express}\ }\textbf {\bibinfo {volume} {18}},\ \bibinfo {pages} {6241}
  (\bibinfo {year} {2010})}\BibitemShut {NoStop}%
\bibitem [{\citenamefont {Roccati}\ \emph {et~al.}(2021)\citenamefont
  {Roccati}, \citenamefont {Lorenzo}, \citenamefont {Palma} \emph
  {et~al.}}]{2021.QuanSciTech}%
  \BibitemOpen
  \bibfield  {author} {\bibinfo {author} {\bibfnamefont {F.}~\bibnamefont
  {Roccati}}, \bibinfo {author} {\bibfnamefont {S.}~\bibnamefont {Lorenzo}},
  \bibinfo {author} {\bibfnamefont {G.~M.}\ \bibnamefont {Palma}}, \emph
  {et~al.},\ }\bibfield  {title} {\bibinfo {title} {Quantum correlations in
  $\mathcal{PT}$-symmetric systems},\ }\href@noop {} {\bibfield  {journal}
  {\bibinfo  {journal} {Quantum Science and Technology}\ }\textbf {\bibinfo
  {volume} {6}},\ \bibinfo {pages} {025005} (\bibinfo {year}
  {2021})}\BibitemShut {NoStop}%
\bibitem [{\citenamefont {Arkhipov}\ \emph {et~al.}(2019)\citenamefont
  {Arkhipov}, \citenamefont {Miranowicz}, \citenamefont {Di~Stefano} \emph
  {et~al.}}]{2019.PRA.(Scully-Lamb}%
  \BibitemOpen
  \bibfield  {author} {\bibinfo {author} {\bibfnamefont {I.~I.}\ \bibnamefont
  {Arkhipov}}, \bibinfo {author} {\bibfnamefont {A.}~\bibnamefont
  {Miranowicz}}, \bibinfo {author} {\bibfnamefont {O.}~\bibnamefont
  {Di~Stefano}}, \emph {et~al.},\ }\bibfield  {title} {\bibinfo {title}
  {Scully-lamb quantum laser model for parity-time-symmetric whispering-gallery
  microcavities: Gain saturation effects and nonreciprocity},\ }\href@noop {}
  {\bibfield  {journal} {\bibinfo  {journal} {Phys. Rev. A}\ }\textbf {\bibinfo
  {volume} {99}},\ \bibinfo {pages} {053806} (\bibinfo {year}
  {2019})}\BibitemShut {NoStop}%
\bibitem [{\citenamefont {Barnett}\ \emph {et~al.}(1998)\citenamefont
  {Barnett}, \citenamefont {Jeffers}, \citenamefont {Gatti},\ and\
  \citenamefont {Loudon}}]{1998.PRA}%
  \BibitemOpen
  \bibfield  {author} {\bibinfo {author} {\bibfnamefont {S.~M.}\ \bibnamefont
  {Barnett}}, \bibinfo {author} {\bibfnamefont {J.}~\bibnamefont {Jeffers}},
  \bibinfo {author} {\bibfnamefont {A.}~\bibnamefont {Gatti}},\ and\ \bibinfo
  {author} {\bibfnamefont {R.}~\bibnamefont {Loudon}},\ }\bibfield  {title}
  {\bibinfo {title} {Quantum optics of lossy beam splitters},\ }\href@noop {}
  {\bibfield  {journal} {\bibinfo  {journal} {Phys. Rev. A}\ }\textbf {\bibinfo
  {volume} {57}},\ \bibinfo {pages} {2134} (\bibinfo {year}
  {1998})}\BibitemShut {NoStop}%
\bibitem [{\citenamefont {Roger}\ \emph {et~al.}(2016)\citenamefont {Roger},
  \citenamefont {Restuccia}, \citenamefont {Lyons} \emph {et~al.}}]{2016.PRL}%
  \BibitemOpen
  \bibfield  {author} {\bibinfo {author} {\bibfnamefont {T.}~\bibnamefont
  {Roger}}, \bibinfo {author} {\bibfnamefont {S.}~\bibnamefont {Restuccia}},
  \bibinfo {author} {\bibfnamefont {A.}~\bibnamefont {Lyons}}, \emph {et~al.},\
  }\bibfield  {title} {\bibinfo {title} {Coherent absorption of n00n states},\
  }\href@noop {} {\bibfield  {journal} {\bibinfo  {journal} {Phys. Rev. Lett.}\
  }\textbf {\bibinfo {volume} {117}},\ \bibinfo {pages} {023601} (\bibinfo
  {year} {2016})}\BibitemShut {NoStop}%
\bibitem [{\citenamefont {Vest}\ \emph {et~al.}(2017)\citenamefont {Vest},
  \citenamefont {Dheur}, \citenamefont {Devaux} \emph {et~al.}}]{2017.Science}%
  \BibitemOpen
  \bibfield  {author} {\bibinfo {author} {\bibfnamefont {B.}~\bibnamefont
  {Vest}}, \bibinfo {author} {\bibfnamefont {M.-C.}\ \bibnamefont {Dheur}},
  \bibinfo {author} {\bibfnamefont {{\'E}.}~\bibnamefont {Devaux}}, \emph
  {et~al.},\ }\bibfield  {title} {\bibinfo {title} {Anti-coalescence of bosons
  on a lossy beam splitter},\ }\href@noop {} {\bibfield  {journal} {\bibinfo
  {journal} {Science}\ }\textbf {\bibinfo {volume} {356}},\ \bibinfo {pages}
  {1373} (\bibinfo {year} {2017})}\BibitemShut {NoStop}%
\bibitem [{\citenamefont {Wubs}\ and\ \citenamefont
  {Hardal}(2019)}]{2019.SPIE}%
  \BibitemOpen
  \bibfield  {author} {\bibinfo {author} {\bibfnamefont {M.}~\bibnamefont
  {Wubs}}\ and\ \bibinfo {author} {\bibfnamefont {A.~{\"U}.~C.}\ \bibnamefont
  {Hardal}},\ }\bibfield  {title} {\bibinfo {title} {{Preparing pure states
  with lossy beam splitters using quantum coherent absorption of squeezed
  light}},\ }in\ \href@noop {} {\emph {\bibinfo {booktitle} {Quantum
  Nanophotonic Materials, Devices, and Systems 2019}}},\ Vol.\ \bibinfo
  {volume} {11091}\ (\bibinfo  {publisher} {SPIE},\ \bibinfo {year} {2019})\
  p.\ \bibinfo {pages} {110910Z}\BibitemShut {NoStop}%
\bibitem [{\citenamefont {Davis}\ and\ \citenamefont
  {G\"{u}ney}(2021)}]{2021.JOSAB}%
  \BibitemOpen
  \bibfield  {author} {\bibinfo {author} {\bibfnamefont {J.~E.}\ \bibnamefont
  {Davis}}\ and\ \bibinfo {author} {\bibfnamefont {D.~O.}\ \bibnamefont
  {G\"{u}ney}},\ }\bibfield  {title} {\bibinfo {title} {Effect of loss on
  linear optical quantum logic gates},\ }\href@noop {} {\bibfield  {journal}
  {\bibinfo  {journal} {J. Opt. Soc. Am. B}\ }\textbf {\bibinfo {volume}
  {38}},\ \bibinfo {pages} {C153} (\bibinfo {year} {2021})}\BibitemShut
  {NoStop}%
\bibitem [{\citenamefont {III}\ \emph {et~al.}(1974)\citenamefont {III},
  \citenamefont {Scully},\ and\ \citenamefont {W.~Lamb}}]{1974.LaserPhysics}%
  \BibitemOpen
  \bibfield  {author} {\bibinfo {author} {\bibfnamefont {M.~S.}\ \bibnamefont
  {III}}, \bibinfo {author} {\bibfnamefont {M.~O.}\ \bibnamefont {Scully}},\
  and\ \bibinfo {author} {\bibfnamefont {J.}~\bibnamefont {W.~Lamb}},\
  }\href@noop {} {\emph {\bibinfo {title} {Laser Physics}}}\ (\bibinfo
  {publisher} {Addison-Wesley Publishing Company},\ \bibinfo {year}
  {1974})\BibitemShut {NoStop}%
\end{thebibliography}%

\end{document}