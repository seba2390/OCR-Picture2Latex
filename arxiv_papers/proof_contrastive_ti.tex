\subsection{Proofs for Lemma \ref{contrastive_ti}}\label{proof_contrastive_ti}
Recall that in the setting of contrastive learning, we assume that $x$ and $x'$ are sampled independently from the same distribution $\P(x)$. And we assume the label $t$ that captures the similarity between $x$ and $x'$ satisfies
\$
&\P(t=1\,|\,x,x')=\frac{1}{1+e^{-f_{\theta^* }(x)^Tf_{\theta^* }(x')}},\notag\\
&\P(t=-1\,|\,x,x')=\frac{1}{1+e^{f_{\theta^* }(x)^Tf_{\theta^* }(x')}}.
\$
Lemma \ref{contrastive_ti} directly follows from the following lemma.
\begin{lemma}\label{contrastive_ti_weak}
There exists $O\in\R^{r\times r}$, $O^TO=OO^T=I_{r}$ such that
\$
\TV\big(\P_{Of_{\theta}}(x,z),\P_{f_{\theta^* }}(x,z)\big)\leq c\cdot\sqrt{\frac{1}{\sigma_{\min}(\E[f_{\theta^* }(x)f_{\theta^* }(x)^{T}])}}\cdot H\big(\P_{f_{\theta}}(x,x',t),\P_{f_{\theta^* }}(x,x',t)\big).
\$
Here $c$ is some absolute constants.
\end{lemma}


We first prove the following lemma, which is the core of the proof of Lemma \ref{contrastive_ti_weak}.
\begin{lemma}\label{ti_1}
Suppose that $\E[f_{\theta}(x)f_{\theta^* }(x)^{T}]=\E[f_{\theta^* }(x)f_{\theta}(x)^{T}]$ are positive semi-definite matrices. Then we have
\$
\E\big[\big(f_{\theta}(x)^{T}f_{\theta}(x')-f_{\theta^* }(x)^{T}f_{\theta^* }(x')\big)^2\big]\geq (2\sqrt{2}-2)\sigma_{\min}\big(\E[f_{\theta^* }(x)f_{\theta^* }(x)^{T}]\big)\cdot\E[\|f_{\theta^* }(x)-f_{\theta}(x)\|^2_2].
\$
\end{lemma}
\begin{proof}[Proof of Lemma \ref{ti_1}]
For notation simplicity, we denote $\Delta(x):=f_{\theta^* }(x)-f_{\theta}(x)$. It then holds that
\%\label{101401}
&\E\big[\big(f_{\theta}(x)^{T}f_{\theta}(x')-f_{\theta^* }(x)^{T}f_{\theta^* }(x')\big)^2\big]\notag\\
&=\E\big[\big(f_{\theta^* }(x)^T\Delta(x')+\Delta(x)^Tf_{\theta^* }(x')-\Delta(x)^T\Delta(x')\big)^2\big]\notag\\
&=\E\big[\big(\Delta(x)^T\Delta(x')\big)^2-2\sqrt{2}\Delta(x)^T\Delta(x')f_{\theta^* }(x')^T\Delta(x)+2f_{\theta^* }(x)^T\Delta(x')f_{\theta^* }(x')^T\Delta(x)\big]\notag\\
&\quad +(4-2\sqrt{2})\E[f_{\theta}(x')^T\Delta(x)\Delta (x)^Tf_{\theta^* }(x')]+(2\sqrt{2}-2)\E[f_{\theta^* }(x')^T\Delta(x)\Delta (x)^Tf_{\theta^* }(x')].
\%




For the first term of \eqref{101401}, we have
\%
&\E\big[\big(\Delta(x)^T\Delta(x')\big)^2-2\sqrt{2}\Delta(x)^T\Delta(x')f_{\theta^* }(x')^T\Delta(x)+2f_{\theta^* }(x)^T\Delta(x')f_{\theta^* }(x')^T\Delta(x)\big]\notag\\
&=\Tr\Big(\E[\Delta(x')\Delta(x')^T\Delta(x)\Delta(x)^T-2\sqrt{2}\Delta(x')f_{\theta^* }(x')^T\Delta(x)\Delta(x)^T+2\Delta(x')f_{\theta^* }(x')^T\Delta(x)f_{\theta^* }(x)^T]\Big)\notag\\
&=\Tr\Big(\big(\E[\Delta(x)\Delta(x)^T]\big)^2-2\sqrt{2}\E[\Delta(x)f_{\theta^* }(x)^T]\cdot\E[\Delta(x)\Delta(x)^T]+2\big(\E[\Delta(x)f_{\theta^* }(x)^T]\big)^2\Big)\notag\\
&=\Tr\Big(\big(\E[\Delta(x)\Delta(x)^T]-\sqrt{2}\E[\Delta(x)f_{\theta^* }(x)^T]\big)^2\Big),
\%
where the second equation follows from our assumption that $x,x'$ are i.i.d. Note that $\E[f_{\theta}(x)f_{\theta^* }(x)^{T}]=\E[f_{\theta^* }(x)f_{\theta}(x)^{T}]$. Thus, we obtain
\%
\Big(\E[\Delta(x)\Delta(x)^T]-\sqrt{2}\E[\Delta(x)f_{\theta^* }(x)^T]\Big)^T&=\E[\Delta(x)\Delta(x)^T]-\sqrt{2}\E[f_{\theta^* }(x)\Delta(x)^T]\notag\\
&=\E[\Delta(x)\Delta(x)^T]-\sqrt{2}\E[\Delta(x)f_{\theta^* }(x)^T],
\%
which implies that $\E[\Delta(x)\Delta(x)^T]-\sqrt{2}\E[\Delta(x)f_{\theta^* }(x)^T]$ is symmetric. It then holds that
\%\label{101402}
&\E\big[\big(\Delta(x)^T\Delta(x')\big)^2-2\sqrt{2}\Delta(x)^T\Delta(x')f_{\theta^* }(x')\Delta(x)+2f_{\theta^* }(x)^T\Delta(x')f_{\theta^* }(x')^T\Delta(x)\big]\notag\\
&=\Tr\Big(\big(\E[\Delta(x)\Delta(x)^T]-\sqrt{2}\E[\Delta(x)f_{\theta^* }(x)^T]\big)^2\Big)\geq 0.
\%


For the second term of \eqref{101401}, we have
\%\label{101403}
\E[f_{\theta}(x')^T\Delta(x)\Delta (x)^Tf_{\theta^* }(x')]=\Tr\Big(\E[f_{\theta^* }(x')f_{\theta}(x')^{T}]\cdot\E[\Delta(x)\Delta(x)^T]\Big)\geq 0,
\%
where the inequality follows from the fact $\E[f_{\theta^* }(x')f_{\theta}(x')^{T}]\succcurlyeq 0$ and $\E[\Delta(x)\Delta(x)^T]\succcurlyeq 0$.



For the third term of \eqref{101401}, we have
\%\label{101404}
\E[f_{\theta^* }(x')^T\Delta(x)\Delta (x)^Tf_{\theta^* }(x')]&=\Tr\Big(\E[f_{\theta^* }(x)f_{\theta^* }(x)^T]\cdot\E[\Delta(x)\Delta (x)^T]\Big)\notag\\
&\geq \sigma_{\min}\big(\E[f_{\theta^* }(x)f_{\theta^* }(x)^T]\big)\Tr\Big(\E[\Delta(x)\Delta (x)^T]\Big)\notag\\
&=\sigma_{\min}\big(\E[f_{\theta^* }(x)f_{\theta^* }(x)^T]\big)\E[\|\Delta(x)\|^2_2].
\%
Combining \eqref{101401}, \eqref{101402}, \eqref{101403} and \eqref{101404}, we have
\%
\E\big[\big(f_{\theta}(x)^{T}f_{\theta}(x')-f_{\theta^* }(x)^{T}f_{\theta^* }(x')\big)^2\big]\geq (2\sqrt{2}-2)\sigma_{\min}\big(\E[f_{\theta^* }(x)f_{\theta^* }(x)^T]\big)\E[\|\Delta(x)\|^2_2]
\%
\end{proof}

With Lemma \ref{ti_1}, we prove Lemma \ref{contrastive_ti_weak} in the following.

\begin{proof}[Proof of Lemma \ref{contrastive_ti_weak}]
We consider the singular value decomposition (SVD) of $\E[f_{\theta}(x)f_{\theta^* }(x)^T]=U_1\Sigma_1 V^T_1$ and $\E[f_{\theta^* }(x)f_{\theta}(x)^{T}]=(\E[f_{\theta}(x)f_{\theta^* }(x)^T])^T=V_1\Sigma_1 U^T_1$. We define $O:=V_1 U^T_1\in\R^{r\times r}$, which satisfies $O^TO=OO^T=I_r$. It then holds that
\%
\E[Of_{\theta}(x)f_{\theta^* }(x)^T]=\E\big[f_{\theta^* }(x)\big(Of_{\theta}(x)\big)^{T}\big]=V_1\Sigma_1 V^T_1,
\%
which are positive semi-definite matrices. By Lemma \ref{ti_1}, we have
\%\label{101501}
&\E\big[\big(f_{\theta}(x)^{T}f_{\theta}(x')-f_{\theta^* }(x)^{T}f_{\theta^* }(x')\big)^2\big]\notag\\
&\quad\geq (2\sqrt{2}-2)\sigma_{\min}\big(\E[f_{\theta^* }(x)f_{\theta^* }(x)^{T}]\big)\cdot\E[\|f_{\theta^* }(x)-Of_{\theta}(x)\|^2_2].
\%
For Hellinger distance, we have
\%\label{101502}
&2H^2\big(\P_{f_{\theta}}(x,x',t),\P_{f_{\theta^* }}(x,x',t)\big)\notag\\
&=\int \Big(\sqrt{p_{f_{\theta}}(x,x',t)}-\sqrt{p_{f_{\theta^* }}(x,x',t)}\Big)^2\,dtdxdx'\notag\\
&=\int \Big(\sqrt{p_{f_{\theta}}(t=1\,|\,x,x')}-\sqrt{p_{f_{\theta^* }}(t=1\,|\,x,x')}\Big)^2p(x,x')\,dxdx'\notag\\
&\quad+\int \Big(\sqrt{p_{f_{\theta}}(t=0\,|\,x,x')}-\sqrt{p_{f_{\theta^* }}(t=0\,|\,x,x')}\Big)^2p(x,x')\,dxdx'
\%
For the first term of \eqref{101502}, we have
\%
&\int \Big(\sqrt{p_{f_{\theta}}(t=1\,|\,x,x')}-\sqrt{p_{f_{\theta^* }}(t=1\,|\,x,x')}\Big)^2p(x,x')\,dxdx'\notag\\
&\quad=\int \Big(\sqrt{h\big(f_{\theta}(x)^{T}f_{\theta}(x')\big)}-\sqrt{h\big(f_{\theta^* }(x)^{T}f_{\theta^* }(x')}\Big)^2p(x,x')\,dxdx',
\%
where 
\%
h(a):=\frac{1}{1+e^{-a}}.
\%
By Cauchy-Schwartz inequality, we have $|f_{\theta}(x)^{T}f_{\theta}(x')|\leq \|f_{\theta}(x)\|_2\|f_{\theta}(x')\|_2\leq 1$. Note that for any $a,b\in[-1,1]$, we have
\%
&\Big(\sqrt{h(a)}-\sqrt{h(b)}\Big)^2\notag\\
&=\frac{\big(h(a)-h(b)\big)^2}{\Big(\sqrt{h(a)}+\sqrt{h(b)}\Big)^2}\geq \frac{1}{4}\big(h(a)-h(b)\big)^2=\frac{1}{4}h'(\xi)^2(a-b)^2\geq \frac{1}{2+e+e^{-1}}(a-b)^2.
\%
Thus, it holds that
\%\label{101503}
&\int \Big(\sqrt{p_{f_{\theta}}(t=1\,|\,x,x')}-\sqrt{p_{f_{\theta^* }}(t=1\,|\,x,x')}\Big)^2p(x,x')\,dxdx'\notag\\
&\geq \frac{1}{2+e+e^{-1}}\int\big(f_{\theta}(x)^{T}f_{\theta}(x')-f_{\theta^* }(x)^{T}f_{\theta^* }(x')\big)^2p(x,x')\,dxdx'\notag\\
&=\frac{1}{2+e+e^{-1}}\E\big[\big(f_{\theta}(x)^{T}f_{\theta}(x')-f_{\theta^* }(x)^{T}f_{\theta^* }(x')\big)^2\big].
\%

Similarly, For the second term of \eqref{101502}, we have 
\%\label{101504}
&\int \Big(\sqrt{p_{f_{\theta}}(t=0\,|\,x,x')}-\sqrt{p_{f_{\theta^* }}(t=0\,|\,x,x')}\Big)^2p(x,x')\,dxdx'\notag\\
&\geq\frac{1}{2+e+e^{-1}}\E\big[\big(f_{\theta}(x)^{T}f_{\theta}(x')-f_{\theta^* }(x)^{T}f_{\theta^* }(x')\big)^2\big]
\%
Combining \eqref{101502}, \eqref{101503} and \eqref{101504}, we have
\%\label{101505}
H^2\big(\P_{f_{\theta}}(x,x',t),\P_{f_{\theta^* }}(x,x',t)\big)\geq \frac{1}{2+e+e^{-1}}\E\big[\big(f_{\theta}(x)^{T}f_{\theta}(x')-f_{\theta^* }(x)^{T}f_{\theta^* }(x')\big)^2\big].
\%

We choose $O\in\R^{r\times r}$ that satisfies \eqref{101501}. For the TV distance, we have
\%
\TV\big(\P_{Of_{\theta}}(x,z),\P_{f_{\theta^* }}(x,z)\big)=\frac{1}{2}\int |p_{Of_{\theta}}(z\,|\,x)-p_{f_{\theta^* }}(z\,|\,x)|p(x)\,dx.
\%
Note that $z\,|\,x\sim\mN(f_{\theta}(x),I_r)$. By Lemma \ref{tv_norm}, we have 
\%\label{101506}
\TV\big(\P_{Of_{\theta}}(x,z),\P_{f_{\theta^* }}(x,z)\big)&=\frac{1}{2}\int |p_{Of_{\theta}}(z\,|\,x)-p_{f_{\theta^* }}(z\,|\,x)|p(x)\,dx\notag\\
&\leq \frac{1}{2}\int\min\{1,\|Of_{\theta}(x)-f_{\theta^* }(x)\|_2\}p(x)\,dx\notag\\
&\leq \frac{1}{2}\min\bigg\{1, \int\|Of_{\theta}(x)-f_{\theta^* }(x)\|_2 p(x)\,dx\bigg\}\notag\\
&= \frac{1}{2}\min\big\{1,\E[\|Of_{\theta}(x)-f_{\theta^* }(x)\|_2]\big\}.
\%
Combining \eqref{101501}, \eqref{101505} and \eqref{101506}, we show that
\%
&\TV\big(\P_{Of_{\theta}}(x,z),\P_{f_{\theta^* }}(x,z)\big)\notag\\
&\leq \frac{1}{2}\E[\|Of_{\theta}(x)-f_{\theta^* }(x)\|_2]\notag\\
&\leq \frac{1}{2}\sqrt{\E[\|Of_{\theta}(x)-f_{\theta^* }(x)\|^2_2]}\notag\\
&\leq \frac{1}{2}\sqrt{\frac{1}{(2\sqrt{2}-2)\sigma_{\min}\big(\E[f_{\theta^* }(x)f_{\theta^* }(x)^{T}]\big)}\E\big[\big(f_{\theta}(x)^{T}f_{\theta}(x')-f_{\theta^* }(x)^{T}f_{\theta^* }(x')\big)^2\big]}\notag\\
&\leq\frac{1}{2}\sqrt{\frac{2+e+e^{-1}}{(2\sqrt{2}-2)\sigma_{\min}\big(\E[f_{\theta^* }(x)f_{\theta^* }(x)^{T}]\big)}} H\big(\P_{f_{\theta}}(x,x',t),\P_{f_{\theta^* }}(x,x',t)\big).
\%
Thus, we prove Lemma \ref{contrastive_ti_weak}.
\end{proof}

Lemma \ref{contrastive_ti_weak} directly implies Lemma \ref{contrastive_ti}.
\begin{proof}[Proof of Lemma \ref{contrastive_ti}]
For any $\theta\in\Theta$, we choose $O\in\R^{r\times r}$ that satisfies Lemma \ref{contrastive_ti_weak}. It then holds that
\$
\TV\big(\P_{f_{\theta},O^{T}\beta^* }(x,y),\P_{f_{\theta^* },\beta^* }(x,y)\big)&=\TV\big(\P_{Of_{\theta},\beta^* }(x,y),\P_{f_{\theta^* },\beta^* }(x,y)\big)\\
&\leq\TV\big(\P_{Of_{\theta}}(x,z),\P_{f_{\theta^* }}(x,z)\big)\notag\\
&\leq c\cdot\sqrt{\frac{1}{\sigma_{\min}(\E[f_{\theta^* }(x)f_{\theta^* }(x)^{T}])}} \cdot H\big(\P_{f_{\theta}}(x,x',t),\P_{f_{\theta^* }}(x,x',t)\big).
\$
Thus, we prove that the model is $\kappa^{-1}$-weakly-informative, where 
\%
\kappa=c\cdot\sqrt{\frac{1}{\sigma_{\min}(\E[f_{\theta^* }(x)f_{\theta^* }(x)^{T}])}}.
\%
Here $c$ is some absolute constants.

\end{proof}