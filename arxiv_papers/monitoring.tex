%Material
%\begin{itemize}
%\item N. Agaras et al, PMT monitoring with Laser, https://cds.cern.ch/record/2648374
%\item F. Veloso et al, Survey of the ATLAS Tile Calorimeter linearity using the Laser Calibration System, https://cds.cern.ch/record/2749305
%\item G. Di Grigorio, Robustness studies of the photomultipliers reading out TileCal, the central hadron calorimeter of the ATLAS experiment, https://doi.org/10.1016/j.nima.2018.09.047
%\end{itemize}

\subsection{Automated channel monitoring}

Channels having pathological problems need to be promptly identified during the data taking periods. Therefore, an automated daily monitoring utilising the laser system is setup in order to identify and diagnose the channels' issues. This is achieved by analysing the recent laser calibration runs. After each laser run, several monitoring figures are produced by an automated software in order to control the PMT stability and provide the list of problematic channels with possible source of issues. All TileCal channels, including the masked channels after data quality checks, are analysed and flagged according to the algorithm described below.

The automated laser monitoring algorithm is based on the analysis of the PMT response variation, measured for each channel using the Direct method (explained in Section~\ref{sec:determination_of_the_calibration_constants}), and the comparison with other data (applied HV, global behaviour of group of channels associated to the same cell type). 

The time evolution of the channels' response is daily monitored using LG and HG laser runs taken in 15 preceding days, and three categories of channels are defined:
%The flags and the assigment procedure of the flags are listed below:

\begin{itemize}
  \item \textbf{Normal channels:} Channels that have no deviation or a deviation compatible with the mean deviation of similar cells. These channels can be calibrated safely and do not require a special attention.
  \item \textbf{Suspicious channels:} Channels with a deviation slightly higher than the mean deviation of similar cells or a deviation compatible with the one expected from the variation of the HV supply. These channels can be calibrated safely but some follow up may be needed.
    \item \textbf{Channels to be checked:} Channels with large deviations (>10\%) that cannot be explained by the mean deviation observed in cells of similar type nor by HV changes; or channels having a non linear behaviour during the 15 preceding days (jumps or fast drifts). These channels should not be calibrated unless the origin of the effect is understood. In most of the cases, especially in case of a fast drift, the channels need to be masked.
\end{itemize}

% Online monitoring is producing the overview of the PMT response as given in Figures~\ref{fig:map_2018} and \ref{fig:phimap_2018}. 

While this categorisation assists during channel calibration, the automated monitoring of laser data also identifies channels with pathological behaviour and determines the source of the issues encountered, complementing data quality assessment activities. During the data taking period, laser runs are chosen with approximately 10 days interval. For each chosen date, both HG and LG runs are analysed and channels are classified according to their problems as follows:

\begin{itemize}
  \item \textbf{Bad channels:} Channels with large PMT drift (>10\%) or having a wrong behaviour during the 15 preceding days.
  \item \textbf{HV unstable:} Channels with large PMT drift (>10\%) but compatible with HV variation.
%   if the deviation of the channel is compatible with HV and greater than 10\%
  \item \textbf{No laser data:} Channels with low laser signal amplitude (e.g. caused by a laser fibre problem).
  %if the amplitude of the channel is small 
  \item \textbf{Bad laser data:} Channels with corrupted laser calibration data or having problematic reference.
%  if the data during the run is corrupted or having problematic reference.
\end{itemize}

A channel is reported to be problematic if it manifests any type of issue listed above. Figure~\ref{fig:overall} shows the fraction of problematic channels, observed in 2017 and 2018, as a function of time. The maximum number of such channels did not exceed 4 and 3\%, respectively.
%in any gain as it is shown in Figure~\ref{fig:overall} for 2018 and for 2017.

\begin{figure}[h!]
  \centering
  \subfloat[\label{fig:overall_a}]{\includegraphics[height=0.35\linewidth]{figures/PMTmonitoring2017.pdf}}\quad
  \subfloat[\label{fig:overall_b}]{\includegraphics[height=0.35\linewidth]{figures/PMTmonitoring2018.pdf}}
    \caption{\label{fig:overall} The fraction of the problematic channels identified by the  laser monitoring algorithm in 2017 (a) and 2018 (b).
    %overall number of all the problematic channels in 2017 (a) and 2018 (b).
    } 
\end{figure}
\FloatBarrier
%As can be seen in Figure~\ref{fig:overall}, the number of problematic channels in 2017 is greater than in 2018. 



\subsection{PMT linearity monitoring}

The TileCal PMT calibration is performed using laser light with a constant intensity. This procedure contributes to ensure that the calorimeter measures the same output over time for the same input energy deposition, i.e. that its response is stable. However, to guarantee that the calibration factors are accurate across the entire dynamic range of the PMT response and the output signal is directly proportional to the energy deposit one needs to assess the PMT linearity.

The linearity of the TileCal PMT channels was monitored during the Run~2 operation with laser calibration data acquired between 2016 and 2019. The dataset corresponded to a combination of standard laser calibration low gain runs using different filter wheel positions and laser intensity varied in the range of 12k to 18k in DAC counts. The linearity of a given PMT channel is evaluated by comparing the PMT signal to the signal of the reference photodiode D6 of the Laser~II system. The response of the channels should increase linearly with the light intensity, in the same way that the PMT channels respond to increases in the energy deposited in the calorimeter. 

PMTs lose linearity shortly before the saturation point. In addition, the TileCal ADCs saturate at an upper limit of 1023 counts. The saturation amplitude is given by $A_{\mathrm{max}}\;\mathrm{[pC]} = (1023-p)/f_\mathrm{ADC\to pC}$, where $p$ and $f_\mathrm{ADC\to pC}$ are the pedestal and CIS constant values, respectively. Values above the ADC saturation amplitude are not reliable and any non-linear behaviour above it should not be related exclusively to the PMT. Typically, TileCal readout channels start to loose linearity above $\sim750$~pC and reach saturation at $\sim850$~pC. This behaviour can be observed for the TileCal channels that receive enough light. To avoid this issue, amplitudes above $A_{\mathrm{max}} - \sigma_A$, where $\sigma_A$ is the standard deviation of the amplitudes, are excluded from the analysis.

\begin{figure}[htbp]
\centering
\subfloat[\label{fig:PMT_fit}]{\includegraphics[width=0.45\textwidth]{figures/PMT_06_0_EBA01_05_l}}\hfill
\subfloat[\label{fig:integral}]{\includegraphics[width=0.48\textwidth]{figures/integral}}
\caption{(a) Signal in EBA01 PMT 5 (channel 4) versus signal in photodiode~6 using the dataset taken on 2016-07-10. The red line shows the obtained fit. The inset shows the magnified distribution of the low laser-intensity region. (b) Schematic representation of the area obtained by intercepting the joint data points and the fit function, in grey. The deviation from linearity is defined as the ratio between the grey area and the fit function integral between the first ($x_0$) and last ($x_1$) points.}
\end{figure}

\begin{figure}[htbp]
\centering
\includegraphics[width=0.5\textwidth]{figures/LinearEvolutionPlots20.pdf}
\caption{The percentage of PMTs within 1\,\% and 2\,\% deviation from linearity as a function of time for the weekly calibration runs taken between 2016-07-10 and 2019-01-18 are shown.\label{fig:linearity}}
\end{figure}

The PMT signal amplitude versus the reference diode signal is plotted for a set of runs of varied light output taken in the same day. A linear fit is performed iteratively for each data point and all the points with smaller amplitudes. The fit comprising more points within one standard deviation of the fitted line is chosen as final for further analysis. 
An example can be seen in Figure~\ref{fig:PMT_fit}.

The deviation from linearity, in percentage, is defined as the ratio between the area delimited by data points intercepted by the linear fit, and the integral of the linear fit, as shown in Figure~\ref{fig:integral}. %{\color{red}A channel is labelled as having a knee, if the maximum linear point is below the maximum recorded point or below $1.5\sigma$ from the fitted line. If a knee is identified, then the corresponding data points are not taken into account to compute the deviation from linearity.}

Figure~\ref{fig:linearity} shows the percentage of PMTs within 1~\% and 2~\% deviation from linearity as a function of time for the weekly calibration runs taken between 2016 and 2019. The percentage of channels with deviation from linearity less than 1\% is $(99.66 \pm 0.11)\%$ and less than 2\% is $(99.72 \pm 0.09)\%$ considering this period.


\FloatBarrier