%Material:
%\begin{itemize}
%\item T. Davidek et al, Time calibration and monitoring in the Tile Calorimeter, https://cds.cern.ch/record/2715632
%\end{itemize}

\subsection{Time monitoring}

\newcommand{\tlaser}{$t^{\mathrm{laser}}_{\mathrm{chan}}$}

%\section{Time monitoring and correction}
The TileCal does not only provide a measurement of the energy that is
deposited in the calorimeter, but it also measures the time when
particles and jets hit the calorimeter cell. This information is
particularly utilised in the removal of signals which do not originate
from $pp$ collisions along with the time-of-flight
measurements of hypothetical heavy slow particles that would reach the
calorimeter.
% For these measurements, the time synchronisation of all
% calorimeter channels represents an important issue.

%%% Few words about the signal reco
The time calibration is also important for the energy reconstruction
itself. As explained in Section~\ref{sec:signal_energy_reco},
physics collision events are reconstructed with the OF algorithm (see
Eq.~(\ref{eq:of})), whose weights depend on the 
expected phase. If the real signal phase significantly differs from the
expected one, the reconstructed amplitude is underestimated. 
Consequently, the time synchronisation of all calorimeter channels
represents an important issue. While the final time calibration is performed
with $pp$ collision data, laser data are extensively used to check its
stability and to spot eventual problems.

Laser calibration events are shot during empty bunch-crossings of
physics runs with a frequency of about 1~Hz. These events, also
referred to as laser-in-gap events, were originally proposed for the
PMT response monitoring. However, they are also extensively used for the
time calibration stability monitoring.

The monitoring tool creates a 2D histogram for each channel and fills
them with the reconstructed time ({\tlaser}) and luminosity block for each
event. These histograms are stored and automatically examined for
anomalies, which include average {\tlaser} being off zero in at least
few consecutive luminosity blocks, unstable {\tlaser} or fraction of
events off zero by more than 20~ns.

The former feature typically indicates a sudden change of the timing settings of
the corresponding digitiser, so-called timing jump. These timing jumps
are corrected by adjusting the associated time constant in the affected
period, as shown in Figure~\ref{fig:timing_jump}.
%
While the timing jumps were very frequent during
Run~1~\cite{bib:laser_run_1} and a lot of effort was invested into their 
correction, they appeared very rarely during Run~2 due
to improved stability of the electronics. This allowed us to focus on
other problems observed with the monitoring tool.

\begin{figure}[htbp]
\centering
    \subfloat[\label{fig:timing_jump_a}]{\includegraphics[width=0.5\linewidth]{figures/laser_325790_ebc09_ch_30_35_express_1}}
    \subfloat[\label{fig:timing_jump_b}]{\includegraphics[width=0.5\linewidth]{figures/laser_325790_ebc09_ch_30_35_blk_1}}
    \caption{An example of the timing jump of +15~ns in EBC09
    channels 30--35 before (a) and after (b) the time constant
    correction as identified with the laser-based monitoring tool. The
    dashed line indicates the expected mean time value.}
    \label{fig:timing_jump}
\end{figure}

% Few channels suffer from {\tlaser} sometimes off by 1 or 2
% bunch-crossings, i.e.\ $\pm25$ or \SI{\pm 50}{ns}. This feature
% affects all three channels managed by the same Data Management Unit
% (TileDMU)~\cite{bib:digi} as demonstrated in
% Fig.~\ref{fig:time_BC_offset}. The problem is intermittent, with a
% frequency typically at a percent-level; nevertheless, the observed
% bunch-crossing offset and affected events are fully correlated across
% the three channels.
% %
% Such events also occur in physics collision data, whose energy and
% time are reconstructed with the Optimal Filtering (OF)
% algorithm. As its parameters depend on the signal phase, a bad
% channel time results into incorrectly determined energy. Studies have
% shown that a difference of \SI{25}{ns} between the actual and supposed
% time phases degrades the reconstructed energy by 35\%. For this reason,
% a dedicated software tool was developed to detect cases
% affected by the bunch-crossing offset in physics data on-the-fly and prevent them from
% propagation to subsequent object reconstruction.

Few channels suffer from {\tlaser} sometimes off by 1 or 2
bunch-crossings, i.e. $\pm25$ or $\pm50$~ns. This feature
affects all three channels managed by the same Data Management Unit
(TileDMU)~\cite{Berglund:2008zz}. The problem is intermittent, with a 
rate at a percent-level; nevertheless, the observed
bunch-crossing offset and affected events are fully correlated across
the three channels. An example is shown in Fig.~\ref{fig:time_BC_offset_a}.
%
Such events also occur in physics collision data at a very similar
rate as in the laser data. Studies have shown that a difference of
25~ns between the actual and supposed time phases degrades the
reconstructed energy by 35\%. For this reason, a dedicated software
tool was developed to detect cases affected by the bunch-crossing
offset in physics data on-the-fly and prevent them from propagation to
subsequent object reconstruction. Figure~\ref{fig:time_BC_offset_b} compares the reconstructed time in affected channels before and
after this tool is applied. The affected events close to +25~ns
are clearly reduced.  

\begin{figure}[htbp]
\centering
    \subfloat[\label{fig:time_BC_offset_a}]{\includegraphics[width=0.5\linewidth]{figures/laser_339037_eba40_final_500x500_1}}
    \subfloat[\label{fig:time_BC_offset_b}]{\includegraphics[width=0.5\linewidth]{figures/physics_EBA40_ch39_41_pub_500x500}}
    \caption{The reconstructed time of laser events as a function of
    luminosity blocks (a): three channels belonging the the same
    TileDMU are superimposed. The majority of events, centred around zero, are well
    timed-in. The events with the bunch-crossing offset are centred at
    +25~ns and these events are fully correlated across the
    three channels. The reconstructed time in physics events in
    the same three channels before (original) and after (corrected)
    the algorithm mitigating the bunch-crossing offset events
    applied (b): the algorithm significantly reduces events centred around
    +25~ns.}
    \label{fig:time_BC_offset}
\end{figure}

\subsection{Dependence of the PMT response on the anode current}

The assumption of a linear relationship between the PMT signal amplitude and the cell energy deposits requires the PMT response to be independent of current. For most cells, the range of currents is small enough that any non-linearity is negligible. In contrast, highly exposed cells, such as the E~cells, experience a large current range between low and high luminosity runs, and between a caesium calibration run and a physics run. Therefore, those cells provide the necessary data to investigate such effects and are used to study the PMT response as a function of the anode current. Particular attention is paid to the difference between response of the E1 and E2~cells with respect to the E3 and E4~cells. The latter are the most exposed TileCal cells, where the larger particle fluence result in larger PMT currents. PMTs with active HV dividers~\cite{ATLAS-TDR-28} are installed in the readout of these cells to mitigate the current dependence, in principle affording larger stability. How well they do so must be understood when using the cells across a wide current range.

The measurement of the anode current comes from data of the TileCal readout of minimum-bias events that are collected during each run.
These minimum-bias data are read out via slow current integrators which were installed for the readout of low signals from the radioactive caesium source used in the calorimeter calibration. The integrators average the current in each cell over a long time window of 10--20~ms to suppress fluctuations in event-to-event energy deposition, diverting only a small fraction of each PMT's output from the primary signal. Large depositions from hard-scattering are also suppressed on long time scales.

Laser-in-gap data and the current measurements of minimum-bias events were analysed for three runs taken in 2018. The particular set of runs were selected to explore a wide current range while minimising the overlap of currents. The PMT signal amplitudes caused by the laser pulses are first subtracted from the pedestal, primarily present due to electronics noise, which comes mostly from the front-end electronics used to shape the signal for the ADC, as well as from the presence of beam-induced and other non-collision background. The signal amplitude is not normalised to the reference diode, as done to determine the PMT calibration described in Section~\ref{sec:determination_of_the_calibration_constants}, since the small instability associated to this monitoring device is often larger than the effects being studied, and so are the uncertainties associated with its correction. Instead, a cleaner approach to minimise the impact of laser intensity fluctuations is adopted, normalising the measurements of the channels from E~cells of interest to a reference TileCal PMT with negligible current range on the same module. In this study, the left PMT of the D6~cell is used as the reference. The E~cell normalised response $R^{\mathrm{E/D6_L}}$ is defined per module as the ratio between the signal amplitudes of the E~cell PMT ($A^\mathrm{E}$) and D6 left PMT ($A^{\mathrm{D6_L}}$):

\begin{equation}\label{eq:Ecell_nD6L}
R^{\mathrm{E/D6_L}} = \frac{\mathrm{A^E}}{A^{\mathrm{D6_L}}}
\end{equation}

The minimum-bias current decays throughout a physics run as the proton beams decay, so the normalised E cell response changes as well if there is a dependence on the current. To determine this dependence, the actual cell response at any given current is compared to the nominal response at zero current. Therefore, the measurement in any given luminosity block is normalised to the mean measurement in the zero current period, i.e. before collisions begin:

\begin{equation}\label{eq:normalisedEcell_nD6L}
\frac{R^{\mathrm{E/D6_L}}}{R_{\mathrm{current=0}}^{\mathrm{E/D6_L}}}
\end{equation}

The baseline used for normalisation is the average E cell response ratio before stable beam declaration, where the first luminosity block with non-zero luminosity appears in the collision run. For each luminosity block of the chosen run, the average and RMS of the E/D6 cell response ratio to laser-in-gap pulses is calculated and normalised to the equivalent quantity at zero current. This is plotted as a function of the average anode current measured with the integrator readout of physics signals during the same luminosity block as shown in Figure~\ref{fig:fittedLiG} for the combination of the three selected runs.

\begin{figure}[htbp]
\begin{center}
\subfloat[E1]{\includegraphics[height=0.49\textwidth,angle=-90, trim=6cm 0 0 0, clip=true]{figures/gr_combined_avg50_E1_EBA10_D6_L}}
\subfloat[E2]{\includegraphics[height=0.49\textwidth,angle=-90, trim=6cm 0 0 0, clip=true]{figures/gr_combined_avg50_E2_EBA10_D6_L}}\\
\subfloat[E3]{\includegraphics[height=0.49\textwidth,angle=-90, trim=6cm 0 0 0, clip=true]{figures/gr_combined_avg50_E3_EBA11_D6_L}}
\subfloat[E4]{\includegraphics[height=0.49\textwidth,angle=-90, trim=6cm 0 0 0, clip=true]{figures/gr_combined_avg50_E4_EBA10_D6_L}}
\caption{Normalised E/D6 cell response ratio as a function of current for an example channel from each family. The low current fit is in blue, while the high current fit is in red. ``Switch lines'' indicates the transition current between the low and high current fits. ``Fit = 1'' indicates the current at which the normalised E/D6 cell response ratio intercepts 1, with a negative value resulting in a non-physical intercept at 0 current. The fitted ratio can be applied as a function of current to correct for the current-dependence of the PMT response.}
\label{fig:fittedLiG}
\end{center}
\end{figure}

Data are pruned by luminosity block if the laser pedestal is not stable or if the number of measurements in the luminosity block is less than 100. Luminosity blocks with a small number of measurements typically overlap with emittance scans or beam adjustments performed by ATLAS and in which the laser is disabled, so these are discarded. To further smooth the data, the measurements are averaged every $50~\mu\mathrm{A}$. The minimum current is chosen to avoid using data from fluctuations in the zero current measurement. The data are adjusted with a piecewise pair of linear fits:

\begin{itemize}
  \item A low current fit from $0.08~\mathrm{\mu A}$ until the transition current
  \item A high current fit from the transition current using all data up through $10~\mathrm{\mu A}$ 
\end{itemize}

The transition current is chosen by calculating the combined $\chi^2/n_{\mathrm{DoF}}$ for the two linear fits with possible transition currents in steps of $0.05~\mathrm{\mu A}$ up to a maximum possible transition current value of $2.30~\mathrm{\mu A}$. The current yielding the minimum $\chi^2/n_{\mathrm{DoF}}$ is chosen as the endpoint of the first linear fit and the beginning of the second fit, with piecewise continuity enforced. The procedure is also shown in Figure~\ref{fig:fittedLiG}. The transition current between low and high current regimes ranges between 0.8 and 2.1$\mathrm{\mu A}$. The PMT response dependence on current is stronger for E1 and E2 cells than for E3 and E4 cells where the active dividers were installed, especially in the high current regime. These results bring evidence that the active dividers are effectively stabilising the PMT response across a wide range of current operation.

Such a study can be used to determine a correction to calibrate the PMT response over current from the normalised PMT response ratio fitted function. In  Figure~\ref{fig:fittedLiG}, it can be seen that a maximum 2--3\% correction would be necessary at extreme high current for E1 and E2 cells, respectively. This requires a precise current measurement throughout the range of currents that may be experienced by the different cells. This study demonstrates that such correction should be more important for cells with passive dividers experiencing higher currents, including A cells. 
%, a work recommended for the coming LHC runs.


