
\subsection{Bracketing Number}\label{factor2}
% \subsection{Proofs for Lemma \ref{factor_bn}}\label{factor2}
By an application of $\epsilon$-discretization technique, we upper bound the bracketing number of $\mP(\mathcal{B})$ as follows.

\begin{lemma}\label{factor_bn}
Let $\mP_{\mathcal{X}} (\mathcal{B}):=\{\mN (0,BB^T+ I_d)\,|\, B\in\mathcal{B}\}$, where $\mathcal{B}=\{B\in\R^{d\times r}\,|\,\|B\|_2\leq D\}$ for some $D>0$. Then the entropy can be bounded as follows,
\$
\log N_{\b}(\mP_{\mathcal{X}}(\mathcal{B}),1/m)\leq 4dr\log\big(24mdr(D^2+1)\big).
\$
\end{lemma}

% By \eqref{092601}, we have $x\sim\mN(0,BB^T+I_d)$. Under Assumption \ref{factor_bound}, it holds that $1\leq\|BB^T+I_d\|_2\leq D^2+1$. Based on this observation, we can bound the bracketing number as follows.



% \begin{lemma}\label{factor_bn}
% Let $\mP (\mathcal{B}):=\{\mN (0,BB^T+ I_d)\,|\, B\in\mathcal{B}\}$. Under Assumption \ref{factor_bound}, the entropy can be bounded as follows, 
% \$
% \log N_{\b}(\mP(\mathcal{B}),1/m)\leq 4dr\log\big(24mdr(D^2+1)\big).
% \$
% \end{lemma}


\begin{proof}[Proof of Lemma \ref{factor_bn}]
We consider a set of Gaussian distribution
\$
\mP_{\mathcal{X}}(\mathcal{B}):=\bigg\{p_{\Sigma}(x)=\frac{1}{\sqrt{(2\pi)^d|\Sigma|}}e^{-\frac{1}{2}x^{T}\Sigma^{-1}x}\,\bigg|\, \Sigma=BB^{T}+I_d, B\in\mathcal{B}\bigg\},
\$
where $\mathcal{B}=\{B\in\R^{d\times r}\,|\,\|B\|_2\leq D\}$. Note that 
\%
\lambda_{\max}(\Sigma^{-1})=\big(\lambda_{\min}(\Sigma)\big)^{-1}=1,~\lambda_{\min}(\Sigma^{-1})=\big(\lambda_{\max}(\Sigma)\big)^{-1}\geq \frac{1}{D^2+1}.
\%
Here we denote by $\lambda_{\max}(\Sigma^{-1})$ and $\lambda_{\min}(\Sigma^{-1})$ the largest eigenvalue and the smallest eigenvalue of $\Sigma^{-1}$, respectively. Our goal is to find a $1/m$-bracket $\mN_{\b}(\mP_{\mathcal{X}}(\mathcal{B}),1/m)$ of $\mP_{\mathcal{X}}(\mathcal{B})$. In other words, for any $p_{\Sigma}(x)\in\mP_{\mathcal{X}}(\mathcal{B})$, we need to define $\bar p_{\Sigma}(x)\in\mN_{\b}(\mP_{\mathcal{X}}(\mathcal{B}),1/m)$ such that
\begin{itemize}
    \item $\bar p_{\Sigma}(x)\geq p_{\Sigma}(x),~\forall x\in\R^d$
    \item $\int |\bar p_{\Sigma}(x)-p_{\Sigma}(x)|\,dx\leq 1/m$.
\end{itemize}
Note that rank$(BB^{T})=r<d$ and $ \Sigma=BB^{T}+I_d$. Thus, the eigendecomposition of $\Sigma^{-1}$ has the following form
\%
\Sigma^{-1}=V 
  \begin{bmatrix}
    \lambda_1 & & & & &\\
    & \ddots & & & &\\
    & & \lambda_{r} & & &\\
    & & & 1 & &\\
    & & & & \ddots &\\
    & & & & & 1
  \end{bmatrix}
  V^{T}
  =U
  \begin{bmatrix}
    \lambda_1-1 & &\\
    & \ddots &\\
    & & \lambda_r-1
  \end{bmatrix}
  U^{T}+I_d,
\%
where $VV^T=V^{T}V=I_d$ and $U\in\R^{d\times r}$ is the first $r$ columns of $V$. For notation simplicity, we denote
\$
\Lambda:=  
  \begin{bmatrix}
    \lambda_1-1 & &\\
    & \ddots &\\
    & & \lambda_r-1
  \end{bmatrix}.
\$
Thus, we have $\Sigma^{-1}=U\Lambda U^{T}+I_d$. For some fixed $0<\epsilon\leq (D^2+1)^{-1}/2$ (which we will choose later), if $\lambda_i\in[k\epsilon, (k+1)\epsilon)$ for some $k\in\mathbb{Z}$, we define $\bar \lambda_i:=(k-1)\epsilon$. Note that $\lambda_i\geq \lambda_{\min}(\Sigma^{-1})\geq (D^2+1)^{-1}$. Thus, it holds that $k\geq 2$ and $\bar \lambda_i=(k-1)\epsilon\geq\epsilon>0$. Moreover, we have $\epsilon\leq\lambda_i-\bar \lambda_i\leq 2\epsilon$. We define
\$
\bar{\Lambda}:=  
  \begin{bmatrix}
    \bar{\lambda}_1-1 & &\\
    & \ddots &\\
    & & \bar{\lambda_r}-1
  \end{bmatrix}.
\$
For the matrix $U=(u_{i,j})\in\R^{d\times r}$, if $u_{i,j}\in[\frac{k\epsilon}{3\sqrt{dr}},\frac{(k+1)\epsilon}{3\sqrt{dr}})$ for some $k\in\mathbb{Z}$, we define $\bar{u}_{i,j}:=\frac{k\epsilon}{3\sqrt{dr}}$ and $\bar{U}:=(\bar{u}_{i,j})\in\R^{d\times r}$. It then holds that
\%
\|U-\bar{U}\|_2\leq \|U-\bar{U}\|_{F}=\sqrt{\sum_{i,j}|u_{i,j}-\bar{u}_{i,j}|^2}\leq \sqrt{dr}\cdot\frac{\epsilon}{3\sqrt{dr}}=\frac{\epsilon}{3}.
\%
We define
\%
\overline{\Sigma^{-1}}:=\bar{U}\bar{\Lambda}\bar{U}^{T}+I_d.
\%
Note that $(D^2+1)^{-1}\leq \lambda_i\leq 1$ and $|u_{i,j}|\leq 1$. Thus, we totally have 
\%\label{092209}
\bigg(\frac{1-(D^2+1)^{-1}}{\epsilon}\bigg)^{r}\cdot\bigg(\frac{6\sqrt{dr}}{\epsilon}\bigg)^{dr}=\bigg(\frac{D^2}{(D^2+1)\epsilon}\bigg)^r\cdot\bigg(\frac{6\sqrt{dr}}{\epsilon}\bigg)^{dr}
\%
many $\bar{\Sigma}^{-1}$. Note that for any $\|x\|_2=1$, we have
\$
x^{T}(\Sigma^{-1}-\overline{\Sigma^{-1}})x&=x^{T}(U^T\Lambda U-\bar{U}\bar{\Lambda}\bar{U}^{T})x\\
&=x^{T}U^{T}(\Lambda-\bar{\Lambda})Ux+x^T(U-\bar{U})^T\bar{\Lambda}(U+\bar{U})x\\
&\geq \lambda_{\min}(\Lambda-\bar{\Lambda})-\|(U-\bar{U})^T\bar{\Lambda}(U+\bar{U})\|_2\\
&\geq \lambda_{\min}(\Lambda-\bar{\Lambda})-\|U-\bar{U}\|_2\cdot\|\bar{\Lambda}(U+\bar{U})\|_2\\
&\geq \epsilon-3\bigg(2\epsilon+\frac{D^2}{D^2+1}\bigg)\|U-\bar{U}\|_2\\
&\geq \epsilon-3\bigg(2\epsilon+\frac{D^2}{D^2+1}\bigg)\cdot\frac{\epsilon}{3}\geq0,
\$
where the third inequality follows from
\$
\|\bar{\Lambda}(U+\bar{U})\|_2\leq \|\bar{\Lambda}\|_2\|U+\bar{U}\|_2\leq \bigg(2\epsilon+1-\frac{1}{D^2+1}\bigg)\cdot\bigg(2+\frac{\epsilon}{3}\bigg)\leq 3\bigg(2\epsilon+\frac{D^2}{D^2+1}\bigg).
\$
and the last inequality follows from our assumption $\epsilon\leq (D^2+1)^{-1}/2$. Thus, for any $x\in\R^{d}$, it holds that
\%\label{092208}
x^{T}(\Sigma^{-1}-\overline{\Sigma^{-1}})x\geq 0.
\%

We consider $\bar{p}_{\Sigma}(x)$ of the following form
\$
\bar{p}_{\Sigma}(x)=c\sqrt{\frac{|\overline{\Sigma^{-1}}|}{(2\pi)^d}}e^{-\frac{1}{2}x^{T}\overline{\Sigma^{-1}}x}.
\$
By \eqref{092208}, we have:  $\bar{p}_{\Sigma}(x)\geq {p}_{\Sigma}(x)$ holds for any $x\in\R^d$ if and only if
\$
c\geq\sqrt{\frac{|\Sigma^{-1}|}{|\overline{\Sigma^{-1}}|}}=\sqrt{\frac{\lambda_1\ldots\lambda_r}{\bar\lambda_1\ldots\bar\lambda_r}}.
\$
Note that
\$
\frac{\lambda_i}{\bar\lambda_i}\leq \frac{(k+1)\epsilon}{(k-1)\epsilon}=1+\frac{2}{k-1}\leq 1+\frac{4}{k}\leq 1+4(D^2+1)\epsilon,
\$
where the second inequality follows from $k\geq2$ and the last inequality follows from $k\epsilon\geq (D^2+\sigma^2)^{-1}$. We then obtain that
\$
\sqrt{\frac{\lambda_1\ldots\lambda_r}{\bar\lambda_1\ldots\bar\lambda_r}}\leq \big(1+4(D^2+1)\epsilon\big)^{r/2}.
\$
Let $c=(1+4(D^2+1)\epsilon)^{r/2}$. It then holds that
\$
c\geq \sqrt{\frac{\lambda_1\ldots\lambda_r}{\bar\lambda_1\ldots\bar\lambda_r}},
\$
which implies $\bar{p}_{\Sigma}(x)\geq {p}_{\Sigma}(x)$ holds for any $x\in\R^d$.
Note that
\$
\int |\bar p_{\Sigma}(x)-p_{\Sigma}(x)|\,dx=c-1= (1+4(D^2+1)\epsilon)^{r/2}-1\leq4(D^2+1)\epsilon r,
\$
where the last inequality follow from $(1+x)^{r/2}\leq 1+rx$ for $x\leq r^{-1}$. Let 
\%\label{092210}
\epsilon=\frac{1}{4(D^2+1)m r}.
\%
We have
\$
\int |\bar p_{\Sigma}(x)-p_{\Sigma}(x)|\,dx\leq4(D^2+1)\epsilon r=\frac{1}{m}.
\$
By \eqref{092209} and \eqref{092210}, we show that
\$
N_{\b}(\mP_{\mathcal{X}}(\mathcal{B}),1/m)\leq(4rmD^2)^r\cdot\big(24rm(D^2+1)\sqrt{dr}\big)^{dr},
\$
which implies 
\$
\log N_{\b}(\mP_{\mathcal{X}}(\mathcal{B}),1/m)\leq 4dr\log\big(24mdr(D^2+1)\big).
\$
\end{proof}