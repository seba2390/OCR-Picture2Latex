We now describe how to sample from $q(\cdot \mid y_{0:T})$ using similar methods as in \citet{Sarkka2021temporal,yaghoobi2021parallel,Yaghoobi2022sqrt}. Given that both the target of the log-likelihood (as a sum of $T$ independent terms) and the marginal log-likelihood of the LGSSM approximation~\citep{Sarkka2021temporal} can be computed in $\bigO(\log(T))$ on parallel hardware, this is the only missing piece for implementing a parallel-in-time version of our auxiliary Kalman samplers. While several different formulations~\citep[see, e.g.,][]{Doucet:2010,Fruhwirth1994data} may be employed to do so, we here focus on the forward filtering backward sampling (FFBS)~\citep{Fruhwirth1994data} approach.

We can compute the filtering distributions $q(x_t \mid y_{0:t}) = \mathcal{N}(x_t; m_t, P_t)$ for \eqref{eq:general-LGSSM} in parallel using the methods of \citet{Sarkka2021temporal}. Furthermore, we know~\citep[by adding a bias term to][Proposition 1]{Fruhwirth1994data} that
\begin{equation}
    \label{eq:backward}
    \begin{split}
        q(x_T \mid y_{0:T}) &= \mathcal{N}(x_T; m_T, P_T) \\
        q(x_t \mid x_{t+1}, y_{0:t}) &= \mathcal{N}\left(x_t; m_t + G_{t} x_{t+1} - F_t m_t - b_t, \Sigma_t\right), \quad t < T,
    \end{split}
\end{equation}
where $G_t = P_t F_t^{\top}\left(F_t P_t F_t^{\top} + Q_t\right)^{-1}$ and $\Sigma_t = P_t - G_t (F_t P_t F_t^{\top} + Q_t) G_t^{\top}$ for all $t < T$.

We can furthermore rearrange the terms to express $\hat{X}_t \sim q(x_t \mid \hat{X}_{t+1}, y_{0:t})$ recursively as
\begin{equation}
    \label{eq:recursive-sampling}
    \hat{X}_t = G_{t} \hat{X}_{t+1} + \nu_t,
\end{equation}
where all the $\nu_t$'s are independent, and $\nu_t \sim \mathcal{N}(m_t - G_t (F_t m_t + b_t), \Sigma_t)$ for all $t < T$. We also let $G_T = 0$, so that we can then define $\nu_T \sim \mathcal{N}(m_T, P_T)$. Because the means and covariances of the $\nu_t$'s only depend on the LGSSM coefficients and the filtering means and covariances at time $t$, they can be sampled fully in parallel. To sample from $q(x_{0:T} \mid y_{0:T})$ we then need to apply \eqref{eq:recursive-sampling} to the pre-sampled sequence $U_t \sim \nu_t$, $t=0, \ldots, T$. However, the recursive dependency in \eqref{eq:recursive-sampling} is not directly parallelisable, and we instead need to rephrase it in terms of an associative operator, which will allow us to use prefix-sum primitives~\citep{blelloch1989scans}. Thankfully, this is readily done by considering the $\circ$ operator defined as follows
\begin{equation}
    \label{eq:sampling-op}
    \begin{split}
    (G_{ij}, U_{ij})
        &= (G_i, U_i) \circ (G_j, U_j), \quad \text{where} \\
        G_{ij} &= G_i G_j, \quad U_{ij} = G_i U_j + U_i.
    \end{split}
\end{equation}
\begin{proposition}
    \label{prop:prefix-sum-sampling}
    The backward prefix-sum of operator $\circ$ applied to the sequence $(G_t, U_t)$, $t=0, \ldots, T$, recovers the pathwise smoothing distribution $q(x_{0:T} \mid y_{0:T})$, that is, if $(\tilde{G}_t, \tilde{U}_t) = (G_t, U_t) \circ \ldots \circ (G_T, U_T)$, then $(\tilde{U}_0, \ldots, \tilde{U}_T)$ is distributed according to $q(x_{0:T} \mid y_{0:T})$.
\end{proposition}
\begin{proof}
    The operator $\circ$ defined in \eqref{eq:sampling-op} is clearly associative. We prove that its result corresponds to sampling from the pathwise smoothing distribution by reversed induction: suppose that $(\tilde{U}_t, \ldots, \tilde{U}_T)$ is distributed according to $q(x_{t:T} \mid y_{0:T})$, then $\tilde{U}_{t-1} = G_{t-1} \tilde{U}_t + U_{t-1}$, which is distributed according to $q(x_{t-1} \mid \tilde{U}_{t}, y_{0:t-1})$ as discussed before, so that $(\tilde{U}_{t-1}, \ldots, \tilde{U}_T)$ is distributed according to $q(x_{t-1:T} \mid y_{0:T})$. The initial case follows from the definition of $U_T$.
\end{proof}

To summarise, in order to perform prefix-sum sampling of LGSSMs, it suffices to use the parallel-in-time Kalman filtering method of \citet{Sarkka2021temporal} to compute the filtering means and covariances $m_t$, $P_t$, $t=0, \ldots, T$, then form all the elements $G_t$ and sample $U_t$ fully in parallel, and finally, apply the prefix-sum primitive~\citep{blelloch1989scans} to $(G_t, U_t)_{t=0}^T$ with the associative operator $\circ$.