\documentclass{article}
\usepackage[utf8]{inputenc}
\usepackage{amsmath, amssymb, amsfonts, xcolor}

\title{Underpredictions}

\begin{document}


\section{Elasticity}
The stress tensor is given by
\begin{align*}
    \sigma_{ij} =\underbrace{\frac{E}{1+\nu}}_{\alpha}\left[\underbrace{\frac{\nu}{1-2\nu}}_{\beta}u_{kk}\delta_{ij} + u_{ij}\right]
\end{align*}
The elastic equations are
\begin{align*}
    h\partial_j\sigma_{ij} &= Yu_i 
    \\
    h\sigma_{ij}n_j &= -h\sigma^a n_i 
\end{align*}
The first equation can be rescaled like 
\begin{align*}
    h\alpha\left(\beta \partial_ju_{kk}\delta_{ij} + \partial_j u_{ij} \right) &= Yu_i 
    \\
    \left(\beta \partial_ju_{kk}\delta_{ij} + \partial_j u_{ij} \right) &= \frac{Y}{h\alpha}u_i 
    \\
    {\color{blue}\beta} \partial_ju_{kk}\delta_{ij} + \partial_j u_{ij} &= {\color{blue}\tilde{Y}}u_i 
\end{align*}
which has two free parameters.
The final equation can be rescaled to 
\begin{align*}
    h\alpha\left(\beta u_{kk}\delta_{ij} + u_{ij} \right) &= -h\sigma^a n_i
    \\
    \beta u_{kk}\delta_{ij} + u_{ij} &= -\frac{\sigma^a}{\alpha} n_i
    \\
    {\color{blue}\beta}  u_{kk}\delta_{ij} + u_{ij} &= -{\color{blue}\tilde{\sigma}^a}  n_i
\end{align*}
so that in the end we have three parameters, $\beta, \tilde{Y}, \tilde{\sigma}^a$.

$\beta$ is dimensionless, and since $\nu\in(-1, 0.5)$, can take values in $(-1/3, \infty)$. 

$\tilde Y$ has units $1/L^2$.

$\tilde \sigma_a$ has units $1/L$.

\section{Conversions}
For $\tilde Y$
\begin{align*}
    \tilde Y\frac{1}{\tilde{\text{px}}^2} &= \tilde Y\frac{1}{(4\text{px})^2} 
    \\
    &= \tilde Y\frac{1}{(4\cdot 0.17\mu\text{m})^2} 
    \\
    &= 2.16\cdot \tilde Y\frac{1}{\mu\text{m}^2} 
    \\
    \Rightarrow Y \,\mu\text{m}^{-2} &=  2.16\cdot\tilde{Y}\, \tilde{\text{px}}^{-2} 
\end{align*}
where we used that 1) we downsample by 4 pixels, and 2) that $1$ px = 0.17 $\mu$m.

Though dimensionless, we might want to convert $\beta$ to a Poisson ratio
\begin{align*}
    \beta &= \frac{\nu}{1-2\nu}
    \\
    \beta(1-2\nu) &= \nu
    \\
    \beta &= \nu(1 + 2\beta)
    \\
    \nu &= \frac{\beta}{1 + 2\beta}
\end{align*}
for our observed values of $\beta\in(0.2, 0.3)$ this gives $\nu\in(0.14, 0.19)$.

It's impossible to recover $E$, the Young's modulus, because we only measure $\tilde{Y} \propto Y/E$ and $\tilde{\sigma}_a \propto \sigma_a/E$, and any rescaling of $E$ will lead to no difference in the predictions from theory.
\end{document}
