\section*{Appendix}
\begin{figure}[ht]
  \centering

  \begin{subfigure}[b]{0.2\textwidth}
  \includegraphics[scale=1.2]{figures/drawing_convnet.pdf}
  \caption{\ConvNet}
  \label{fig:drawing_convnet}
  \end{subfigure}
  ~
  \begin{subfigure}[b]{0.2\textwidth}
  \includegraphics[scale=1.2]{figures/drawing_small_convnet.pdf}
  \caption{\SmallConvNet}
  \label{fig:drawing_small_convnet}
  \end{subfigure}

  \caption{The \ConvNet and \SmallConvNet architectures. White boxes denote convolutional layers, black thick lines stand for fully connected layers. Arrows show information flow, dashed lines indicate a max-pooling operation.}
\end{figure}

\begin{figure*}[ht]
  \centering
  \includegraphics[scale=1.0]{figures/drawing_unet.pdf}
  \caption{A schematic drawing of the \AUNet architecture. White boxes denote convolutional layers, their width corresponds to the number of convolutional kernels, their height corresponds to the size of the resulting feature maps. Arrows show information flow, dashed lines indicate either a max-pooling operation, if the line goes from a higher to a lower box, or an upscaling operation if the line goes from a lower to a higher box. Grey boxes next to white boxes denote a concatenation of feature maps. We made two adaptations to the original UNet architecture. The first is the use of upscaling operations instead of deconvolutions, and the second adaptation is the last layer having convolutions with a large kernel width in the frequency direction. \label{fig:drawing_unet}}
\end{figure*}

\begin{table}[htp]
\centering
\scalebox{0.6}{%
\begin{tabular}{llr}
	Layer                & Output          & No. of          \\ 
	Type                 & Dimensions      & Params          \\ 
\hline
	Input                & 1x5x229         &                 \\ 
	Conv (Id)            & 32x5x229@3x3    & 288             \\ 
	BatchNorm            & 32x5x229        & 128             \\ 
	Relu                 & 32x5x229        &                 \\ 
	Conv (Id)            & 32x3x227@3x3    & 9216            \\ 
	BatchNorm            & 32x3x227        & 128             \\ 
	Relu                 & 32x3x227        &                 \\ 
	MaxPool              & 32x3x113@1x2    &                 \\ 
	Dropout, p=0.25      & 32x3x113        &                 \\ 
	Conv (Id)            & 64x1x111@3x3    & 18432           \\ 
	BatchNorm            & 64x1x111        & 256             \\ 
	Relu                 & 64x1x111        &                 \\ 
	MaxPool              & 64x1x55@1x2     &                 \\ 
	Dropout, p=0.25      & 64x1x55         &                 \\ 
	Dense (Id)           & 512             & 1802240         \\ 
	BatchNorm            & 512             & 2048            \\ 
	Relu                 & 512             &                 \\ 
	Dropout, p=0.5       & 512             &                 \\ 
	Dense (Sigmoid)      & 88              & 45144           \\ 
\hline
	                     &                 & $\sum$ 1877880 
\end{tabular}%
}
\caption{The \ConvNet Architecture}
\label{table:convnet}
\end{table}
\begin{table}[htp]
\centering
\scalebox{0.6}{%
\begin{tabular}{llr}
	Layer                & Output          & No. of          \\ 
	Type                 & Dimensions      & Params          \\ 
\hline
	Input                & 1x5x229         &                 \\ 
	Conv (Id)            & 8x5x229@3x3     & 72              \\ 
	BatchNorm            & 8x5x229         & 32              \\ 
	Relu                 & 8x5x229         &                 \\ 
	Conv (Id)            & 8x3x227@3x3     & 576             \\ 
	BatchNorm            & 8x3x227         & 32              \\ 
	Relu                 & 8x3x227         &                 \\ 
	MaxPool              & 8x3x113@1x2     &                 \\ 
	Dropout, p=0.25      & 8x3x113         &                 \\ 
	Conv (Id)            & 8x1x111@3x3     & 576             \\ 
	BatchNorm            & 8x1x111         & 32              \\ 
	Relu                 & 8x1x111         &                 \\ 
	MaxPool              & 8x1x55@1x2      &                 \\ 
	Dropout, p=0.25      & 8x1x55          &                 \\ 
	Conv (Id)            & 8x1x53@1x3      & 192             \\ 
	BatchNorm            & 8x1x53          & 32              \\ 
	Relu                 & 8x1x53          &                 \\ 
	MaxPool              & 8x1x26@1x2      &                 \\ 
	Dropout, p=0.25      & 8x1x26          &                 \\ 
	Dense (Id)           & 16              & 3328            \\ 
	BatchNorm            & 16              & 64              \\ 
	Relu                 & 16              &                 \\ 
	Dropout, p=0.5       & 16              &                 \\ 
	Dense (Sigmoid)      & 23              & 391             \\ 
\hline
	                     &                 & $\sum$ 5327    
\end{tabular}%
}
\caption{The \SmallConvNet Architecture}
\label{table:small_convnet}
\end{table}
\begin{table}[htp]
\centering
\scalebox{0.6}{%
\begin{tabular}{llr}
	Layer                & Output          & No. of          \\ 
	Type                 & Dimensions      & Params          \\ 
\hline
	Input                & 1x256x256       &                 \\ 
	Conv (Id)            & 32x256x256@3x3  & 288             \\ 
	BatchNorm            & 32x256x256      & 128             \\ 
	Elu                  & 32x256x256      &                 \\ 
	Conv (Id)            & 32x256x256@3x3  & 9216            \\ 
	BatchNorm            & 32x256x256      & 128             \\ 
	Elu                  & 32x256x256      &                 \\ 
	MaxPool              & 32x128x128@2x2  &                 \\ 
	Conv (Id)            & 32x128x128@3x3  & 9216            \\ 
	BatchNorm            & 32x128x128      & 128             \\ 
	Elu                  & 32x128x128      &                 \\ 
	Conv (Id)            & 32x128x128@3x3  & 9216            \\ 
	BatchNorm            & 32x128x128      & 128             \\ 
	Elu                  & 32x128x128      &                 \\ 
	MaxPool              & 32x64x64@2x2    &                 \\ 
	Conv (Id)            & 64x64x64@3x3    & 18432           \\ 
	BatchNorm            & 64x64x64        & 256             \\ 
	Elu                  & 64x64x64        &                 \\ 
	Conv (Id)            & 64x64x64@3x3    & 36864           \\ 
	BatchNorm            & 64x64x64        & 256             \\ 
	Elu                  & 64x64x64        &                 \\ 
	MaxPool              & 64x32x32@2x2    &                 \\ 
	Conv (Id)            & 64x32x32@3x3    & 36864           \\ 
	BatchNorm            & 64x32x32        & 256             \\ 
	Elu                  & 64x32x32        &                 \\ 
	Conv (Id)            & 64x32x32@3x3    & 36864           \\ 
	BatchNorm            & 64x32x32        & 256             \\ 
	Elu                  & 64x32x32        &                 \\ 
	MaxPool              & 64x16x16@2x2    &                 \\ 
	Conv (Id)            & 128x16x16@3x3   & 73728           \\ 
	BatchNorm            & 128x16x16       & 512             \\ 
	Elu                  & 128x16x16       &                 \\ 
	Conv (Id)            & 128x16x16@3x3   & 147456          \\ 
	BatchNorm            & 128x16x16       & 512             \\ 
	Elu                  & 128x16x16       &                 \\
\end{tabular}%
\begin{tabular}{llr}
 
	Upscale              & 128x32x32       &                 \\ 
	Concat               & 192x32x32       &                 \\ 
	Conv (Id)            & 128x32x32@3x3   & 221184          \\ 
	BatchNorm            & 128x32x32       & 512             \\ 
	Elu                  & 128x32x32       &                 \\ 
	Conv (Id)            & 128x32x32@3x3   & 147456          \\ 
	BatchNorm            & 128x32x32       & 512             \\ 
	Elu                  & 128x32x32       &                 \\ 
	Upscale              & 128x64x64       &                 \\ 
	Concat               & 192x64x64       &                 \\ 
	Conv (Id)            & 64x64x64@3x3    & 110592          \\ 
	BatchNorm            & 64x64x64        & 256             \\ 
	Elu                  & 64x64x64        &                 \\ 
	Conv (Id)            & 64x64x64@3x3    & 36864           \\ 
	BatchNorm            & 64x64x64        & 256             \\ 
	Elu                  & 64x64x64        &                 \\ 
	Upscale              & 64x128x128      &                 \\ 
	Concat               & 96x128x128      &                 \\ 
	Conv (Id)            & 32x128x128@3x3  & 27648           \\ 
	BatchNorm            & 32x128x128      & 128             \\ 
	Elu                  & 32x128x128      &                 \\ 
	Conv (Id)            & 32x128x128@3x3  & 9216            \\ 
	BatchNorm            & 32x128x128      & 128             \\ 
	Elu                  & 32x128x128      &                 \\ 
	Upscale              & 32x256x256      &                 \\ 
	Concat               & 64x256x256      &                 \\ 
	Conv (Id)            & 32x256x256@3x3  & 18432           \\ 
	BatchNorm            & 32x256x256      & 128             \\ 
	Elu                  & 32x256x256      &                 \\ 
	Conv (Id)            & 32x256x128@3x3  & 9216            \\ 
	BatchNorm            & 32x256x128      & 128             \\ 
	Elu                  & 32x256x128      &                 \\ 
	Conv (Sigmoid)       & 1x256x88@1x41   & 1313            \\ 
\hline
	                     &                 & $\sum$ 964673  
\end{tabular}%
}
\caption{The \AUNet Architecture}
\label{table:aunet}
\end{table}

