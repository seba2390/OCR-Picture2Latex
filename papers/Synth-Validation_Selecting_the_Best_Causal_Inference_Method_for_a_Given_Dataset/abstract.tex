\begin{abstract}

Many decisions in healthcare, business, and other policy domains are made without the support of rigorous evidence due to the cost and complexity of performing randomized experiments. Using observational data to answer causal questions is risky: subjects who receive different treatments also differ in other ways that affect outcomes. Many causal inference methods have been developed to mitigate these biases. However, there is no way to know which method might produce the best estimate of a treatment effect in a given study. In analogy to cross-validation, which estimates the prediction error of predictive models applied to a given dataset, we propose \emph{synth-validation}, a procedure that estimates the estimation error of causal inference methods applied to a given dataset. In synth-validation, we use the observed data to estimate generative distributions with known treatment effects. We apply each causal inference method to datasets sampled from these distributions and compare the effect estimates with the known effects to estimate error. Using simulations, we show that using synth-validation to select a causal inference method for each study lowers the expected estimation error relative to consistently using any single method.

\end{abstract} 