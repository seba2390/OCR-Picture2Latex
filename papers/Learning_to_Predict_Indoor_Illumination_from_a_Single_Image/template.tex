\documentclass[acmtog]{acmart}

\usepackage{booktabs} % For formal tables


\makeatletter
\def\runningfoot{\def\@runningfoot{}}
\def\firstfoot{\def\@firstfoot{}}
\makeatother
\settopmatter{printacmref=false} % Removes citation information below abstract
\renewcommand\footnotetextcopyrightpermission[1]{} % removes footnote with conference information in first column
\pagestyle{plain} % removes running headers

% TOG prefers author-name bib system with square brackets
\citestyle{acmauthoryear}
\setcitestyle{square}

\usepackage[ruled]{algorithm2e} % For algorithms
\renewcommand{\algorithmcfname}{ALGORITHM}
\SetAlFnt{\small}
\SetAlCapFnt{\small}
\SetAlCapNameFnt{\small}
\SetAlCapHSkip{0pt}
\IncMargin{-\parindent}

% Metadata Information

\setcopyright{none}
%\acmJournal{TOG}
%\acmYear{2017}\acmVolume{36}\acmNumber{6}\acmArticle{176}\acmMonth{11}
%\acmDOI{10.1145/3130800.3130891}



\makeatletter
\newcommand\footnoteref[1]{\protected@xdef\@thefnmark{\ref{#1}}\@footnotemark}
\makeatother

\usepackage{steinmetz}
\usepackage{physics}
\usepackage{graphicx}
\usepackage{amssymb}
\usepackage{gensymb}
%\usepackage[usenames,dvipsnames]{color}
\usepackage{mathtools}
\usepackage{multirow}
%\usepackage{algorithm}
%\usepackage{algorithmic}
\usepackage[english]{babel} % English language/hyphenation

\newcommand{\eg}{{\em e.g.,~}}
\newcommand{\ie}{{\em i.e.,~}}
\newcommand{\etc}{{\em etc.}}
\newcommand{\etal}{{\em et~al.~}}
\newcommand{\superscript}[1]{\ensuremath{^{\textrm{#1}}}}
\newcommand{\subscript}[1]{\ensuremath{_{\textrm{#1}}}}

\newcommand{\fn}[1]{Figure~\ref{fig:#1}}
\newcommand{\en}[1]{Equation~\ref{eqn:#1}}
\newcommand{\sn}[1]{Section~\ref{sec:#1}}
\newcommand{\tn}[1]{Table~\ref{tab:#1}}
\newcommand{\cn}[1]{Section~\ref{chap:#1}}
\newcommand{\an}[1]{Appendix~\ref{apnd:#1}}
\newcommand{\syn}[1]{{\em #1}}
\newcommand{\iu}{{i\mkern1mu}}

\DeclareMathOperator*{\argmin}{arg\,min}

% \newcommand{\ersin}[1]{\textcolor{red}{[\textsc{ERSIN}: \emph{#1}]}}
\newcommand{\JF}[1]{\textcolor{red}{[\textsc{JF}: \emph{#1}]}}
\newcommand{\MAG}[1]{\textcolor{green}{[\textsc{MAG}: \emph{#1}]}}
\newcommand{\KS}[1]{\textcolor{blue}{[\textsc{KS}: \emph{#1}]}}
% \newcommand{\emiliano}[1]{\textcolor{blue}{[\textsc{Emiliano}: \emph{#1}]}}
% \newcommand{\CG}[1]{\textcolor{red}{[\textsc{CG}: \emph{#1}]}}


% Document starts
\begin{document}

% Title portion
\title{Learning to Predict Indoor Illumination from a Single Image}

\author{Marc-Andr\'e Gardner}
\affiliation{Universit\'e Laval}
\email{marc-andre.gardner.1@ulaval.ca}
\author{Kalyan Sunkavalli}
\author{Ersin Yumer}
\author{Xiaohui Shen}
\author{Emiliano Gambaretto}
\affiliation{Adobe Research}
\author{Christian Gagn\'e}
\author{Jean-Fran\c cois Lalonde}
\affiliation{Universit\'e Laval}

\renewcommand\shortauthors{}

\begin{abstract}
We propose an automatic method to infer high dynamic range illumination from a single, limited field-of-view, low dynamic range photograph of an indoor scene.
%Inferring scene illumination from a single photograph is a challenging problem; the pixel intensities observed in a photograph are a complex function of scene geometry, reflectance properties, and illumination, all of which are unknown in our case. 
In contrast to previous work that relies on specialized image capture, user input, and/or simple scene models, we train an end-to-end deep neural network that directly regresses a limited field-of-view photo to HDR illumination, without strong assumptions on scene geometry, material properties, or lighting. We show that this can be accomplished in a three step process: 1) we train a robust lighting classifier to automatically annotate the location of light sources in a large dataset of LDR environment maps, 2) we use these annotations to train a deep neural network that predicts the location of lights in a scene from a single limited field-of-view photo, and 3) we fine-tune this network using a small dataset of HDR environment maps to predict light intensities. This allows us to automatically recover high-quality HDR illumination estimates that significantly outperform previous state-of-the-art methods. Consequently, using our illumination estimates for applications like 3D object insertion, produces photo-realistic results that we validate via a perceptual user study.

%\MAG{We capitalize on the high number of low-dynamic range panoramas available to train a network and then fine tune it using a limited HDR dataset.} 

 
\end{abstract}

%
% The code below should be generated by the tool at
% http://dl.acm.org/ccs.cfm
% Please copy and paste the code instead of the example below. 

%
% End generated code
%
%
% The code below should be generated by the tool at
% http://dl.acm.org/ccs.cfm
% Please copy and paste the code instead of the example below. 
%
\begin{CCSXML}
<ccs2012>
<concept>
<concept_id>10010147.10010178.10010224.10010225.10010227</concept_id>
<concept_desc>Computing methodologies~Scene understanding</concept_desc>
<concept_significance>500</concept_significance>
</concept>
<concept>
<concept_id>10010147.10010371.10010382</concept_id>
<concept_desc>Computing methodologies~Image manipulation</concept_desc>
<concept_significance>500</concept_significance>
</concept>
<concept>
<concept_id>10010147.10010371.10010382.10010236</concept_id>
<concept_desc>Computing methodologies~Computational photography</concept_desc>
<concept_significance>300</concept_significance>
</concept>
</ccs2012>
\end{CCSXML}

\ccsdesc[500]{Computing methodologies~Scene understanding}
\ccsdesc[500]{Computing methodologies~Image manipulation}
\ccsdesc[300]{Computing methodologies~Computational photography}
%
% End generated code
%
\keywords{indoor illumination, deep learning}

%\thanks{} 

\begin{teaserfigure}
\centering
\setlength{\fboxsep}{0pt}%
\setlength{\fboxrule}{1pt}%
  \includegraphics[width=0.24\textwidth]{images/stock-composites/35-crop.jpg}
	\begin{picture}(0,0)
		\put(-126,63){\fcolorbox{white}{white}{\includegraphics[height=1cm]{images/stock-composites/35-jpeg_envmap.jpg}}}
	\end{picture}
  \includegraphics[width=0.24\textwidth]{images/stock-composites/35-plants.jpg} 
	\hspace{0.2cm}
  \includegraphics[width=0.24\textwidth]{images/stock-composites/10-crop.jpg} 
	\begin{picture}(0,0)
		\put(-126,63){\fcolorbox{white}{white}{\includegraphics[height=1cm]{images/stock-composites/10-jpeg_envmap.jpg}}}
	\end{picture}
  \includegraphics[width=0.24\textwidth]{images/stock-composites/10-bicycle.jpg}
\caption{Given a single LDR image of an indoor scene, our method automatically predicts HDR lighting (insets, tone-mapped for visualization). Our method learns a direct mapping from image appearance to scene lighting from large amounts of real image data; it does not require any additional scene information, and can even recover light sources that are not visible in the photograph, as shown in these examples. Using our lighting estimates, virtual objects can be realistically relit and composited into photographs.}
\label{f:teaser}
\end{teaserfigure}
 
\maketitle


\vspace{15em}
% !TEX root = ../arxiv.tex

Unsupervised domain adaptation (UDA) is a variant of semi-supervised learning \cite{blum1998combining}, where the available unlabelled data comes from a different distribution than the annotated dataset \cite{Ben-DavidBCP06}.
A case in point is to exploit synthetic data, where annotation is more accessible compared to the costly labelling of real-world images \cite{RichterVRK16,RosSMVL16}.
Along with some success in addressing UDA for semantic segmentation \cite{TsaiHSS0C18,VuJBCP19,0001S20,ZouYKW18}, the developed methods are growing increasingly sophisticated and often combine style transfer networks, adversarial training or network ensembles \cite{KimB20a,LiYV19,TsaiSSC19,Yang_2020_ECCV}.
This increase in model complexity impedes reproducibility, potentially slowing further progress.

In this work, we propose a UDA framework reaching state-of-the-art segmentation accuracy (measured by the Intersection-over-Union, IoU) without incurring substantial training efforts.
Toward this goal, we adopt a simple semi-supervised approach, \emph{self-training} \cite{ChenWB11,lee2013pseudo,ZouYKW18}, used in recent works only in conjunction with adversarial training or network ensembles \cite{ChoiKK19,KimB20a,Mei_2020_ECCV,Wang_2020_ECCV,0001S20,Zheng_2020_IJCV,ZhengY20}.
By contrast, we use self-training \emph{standalone}.
Compared to previous self-training methods \cite{ChenLCCCZAS20,Li_2020_ECCV,subhani2020learning,ZouYKW18,ZouYLKW19}, our approach also sidesteps the inconvenience of multiple training rounds, as they often require expert intervention between consecutive rounds.
We train our model using co-evolving pseudo labels end-to-end without such need.

\begin{figure}[t]%
    \centering
    \def\svgwidth{\linewidth}
    \input{figures/preview/bars.pdf_tex}
    \caption{\textbf{Results preview.} Unlike much recent work that combines multiple training paradigms, such as adversarial training and style transfer, our approach retains the modest single-round training complexity of self-training, yet improves the state of the art for adapting semantic segmentation by a significant margin.}
    \label{fig:preview}
\end{figure}

Our method leverages the ubiquitous \emph{data augmentation} techniques from fully supervised learning \cite{deeplabv3plus2018,ZhaoSQWJ17}: photometric jitter, flipping and multi-scale cropping.
We enforce \emph{consistency} of the semantic maps produced by the model across these image perturbations.
The following assumption formalises the key premise:

\myparagraph{Assumption 1.}
Let $f: \mathcal{I} \rightarrow \mathcal{M}$ represent a pixelwise mapping from images $\mathcal{I}$ to semantic output $\mathcal{M}$.
Denote $\rho_{\bm{\epsilon}}: \mathcal{I} \rightarrow \mathcal{I}$ a photometric image transform and, similarly, $\tau_{\bm{\epsilon}'}: \mathcal{I} \rightarrow \mathcal{I}$ a spatial similarity transformation, where $\bm{\epsilon},\bm{\epsilon}'\sim p(\cdot)$ are control variables following some pre-defined density (\eg, $p \equiv \mathcal{N}(0, 1)$).
Then, for any image $I \in \mathcal{I}$, $f$ is \emph{invariant} under $\rho_{\bm{\epsilon}}$ and \emph{equivariant} under $\tau_{\bm{\epsilon}'}$, \ie~$f(\rho_{\bm{\epsilon}}(I)) = f(I)$ and $f(\tau_{\bm{\epsilon}'}(I)) = \tau_{\bm{\epsilon}'}(f(I))$.

\smallskip
\noindent Next, we introduce a training framework using a \emph{momentum network} -- a slowly advancing copy of the original model.
The momentum network provides stable, yet recent targets for model updates, as opposed to the fixed supervision in model distillation \cite{Chen0G18,Zheng_2020_IJCV,ZhengY20}.
We also re-visit the problem of long-tail recognition in the context of generating pseudo labels for self-supervision.
In particular, we maintain an \emph{exponentially moving class prior} used to discount the confidence thresholds for those classes with few samples and increase their relative contribution to the training loss.
Our framework is simple to train, adds moderate computational overhead compared to a fully supervised setup, yet sets a new state of the art on established benchmarks (\cf \cref{fig:preview}).

\section{Background and Motivation}

\subsection{IBM Streams}

IBM Streams is a general-purpose, distributed stream processing system. It
allows users to develop, deploy and manage long-running streaming applications
which require high-throughput and low-latency online processing.

The IBM Streams platform grew out of the research work on the Stream Processing
Core~\cite{spc-2006}.  While the platform has changed significantly since then,
that work established the general architecture that Streams still follows today:
job, resource and graph topology management in centralized services; processing
elements (PEs) which contain user code, distributed across all hosts,
communicating over typed input and output ports; brokers publish-subscribe
communication between jobs; and host controllers on each host which
launch PEs on behalf of the platform.

The modern Streams platform approaches general-purpose cluster management, as
shown in Figure~\ref{fig:streams_v4_v6}. The responsibilities of the platform
services include all job and PE life cycle management; domain name resolution
between the PEs; all metrics collection and reporting; host and resource
management; authentication and authorization; and all log collection. The
platform relies on ZooKeeper~\cite{zookeeper} for consistent, durable metadata
storage which it uses for fault tolerance.

Developers write Streams applications in SPL~\cite{spl-2017} which is a
programming language that presents streams, operators and tuples as
abstractions. Operators continuously consume and produce tuples over streams.
SPL allows programmers to write custom logic in their operators, and to invoke
operators from existing toolkits. Compiled SPL applications become archives that
contain: shared libraries for the operators; graph topology metadata which tells
both the platform and the SPL runtime how to connect those operators; and
external dependencies. At runtime, PEs contain one or more operators. Operators
inside of the same PE communicate through function calls or queues. Operators
that run in different PEs communicate over TCP connections that the PEs
establish at startup. PEs learn what operators they contain, and how to connect
to operators in other PEs, at startup from the graph topology metadata provided
by the platform.

We use ``legacy Streams'' to refer to the IBM Streams version 4 family. The
version 5 family is for Kubernetes, but is not cloud native. It uses the
lift-and-shift approach and creates a platform-within-a-platform: it deploys a
containerized version of the legacy Streams platform within Kubernetes.

\subsection{Kubernetes}

Borg~\cite{borg-2015} is a cluster management platform used internally at Google
to schedule, maintain and monitor the applications their internal infrastructure
and external applications depend on. Kubernetes~\cite{kube} is the open-source
successor to Borg that is an industry standard cloud orchestration platform.

From a user's perspective, Kubernetes abstracts running a distributed
application on a cluster of machines. Users package their applications into
containers and deploy those containers to Kubernetes, which runs those
containers in \emph{pods}. Kubernetes handles all life cycle management of pods,
including scheduling, restarting and migration in case of failures.

Internally, Kubernetes tracks all entities as \emph{objects}~\cite{kubeobjects}.
All objects have a name and a specification that describes its desired state.
Kubernetes stores objects in etcd~\cite{etcd}, making them persistent,
highly-available and reliably accessible across the cluster. Objects are exposed
to users through \emph{resources}. All resources can have
\emph{controllers}~\cite{kubecontrollers}, which react to changes in resources.
For example, when a user changes the number of replicas in a
\code{ReplicaSet}, it is the \code{ReplicaSet} controller which makes sure the
desired number of pods are running. Users can extend Kubernetes through
\emph{custom resource definitions} (CRDs)~\cite{kubecrd}. CRDs can contain
arbitrary content, and controllers for a CRD can take any kind of action.

Architecturally, a Kubernetes cluster consists of nodes. Each node runs a
\emph{kubelet} which receives pod creation requests and makes sure that the
requisite containers are running on that node. Nodes also run a
\emph{kube-proxy} which maintains the network rules for that node on behalf of
the pods. The \emph{kube-api-server} is the central point of contact: it
receives API requests, stores objects in etcd, asks the scheduler to schedule
pods, and talks to the kubelets and kube-proxies on each node. Finally,
\emph{namespaces} logically partition the cluster. Objects which should not know
about each other live in separate namespaces, which allows them to share the
same physical infrastructure without interference.

\subsection{Motivation}
\label{sec:motivation}

Systems like Kubernetes are commonly called ``container orchestration''
platforms. We find that characterization reductive to the point of being
misleading; no one would describe operating systems as ``binary executable
orchestration.'' We adopt the idea from Verma et al.~\cite{borg-2015} that
systems like Kubernetes are ``the kernel of a distributed system.'' Through CRDs
and their controllers, Kubernetes provides state-as-a-service in a distributed
system. Architectures like the one we propose are the result of taking that view 
seriously.

The Streams legacy platform has obvious parallels to the Kubernetes
architecture, and that is not a coincidence: they solve similar problems.
Both are designed to abstract running arbitrary user-code across a distributed
system.  We suspect that Streams is not unique, and that there are many
non-trivial platforms which have to provide similar levels of cluster
management.  The benefits to being cloud native and offloading the platform
to an existing cloud management system are: 
\begin{itemize}
    \item Significantly less platform code.
    \item Better scheduling and resource management, as all services on the cluster are 
        scheduled by one platform.
    \item Easier service integration.
    \item Standardized management, logging and metrics.
\end{itemize}
The rest of this paper presents the design of replacing the legacy Streams 
platform with Kubernetes itself.


\begin{figure}
\centering

\def\picScale{0.08}    % define variable for scaling all pictures evenly
\def\colWidth{0.5\linewidth}

\begin{tikzpicture}
\matrix [row sep=0.25cm, column sep=0cm, style={align=center}] (my matrix) at (0,0) %(2,1)
{
\node[style={anchor=center}] (FREEhand) {\includegraphics[width=0.85\linewidth]{figures/FREEhand.pdf}}; %\fill[blue] (0,0) circle (2pt);
\\
\node[style={anchor=center}] (rigid_v_soft) {\includegraphics[width=0.75\linewidth]{figures/FREE_vs_rigid-v8.pdf}}; %\fill[blue] (0,0) circle (2pt);
\\
};
\node[above] (FREEhand) at ($ (FREEhand.south west)  !0.05! (FREEhand.south east) + (0, 0.1)$) {(a)};
\node[below] (a) at ($ (rigid_v_soft.south west) !0.20! (rigid_v_soft.south east) $) {(b)};
\node[below] (b) at ($ (rigid_v_soft.south west) !0.75! (rigid_v_soft.south east) $) {(c)};
\end{tikzpicture}


% \begin{tikzpicture} %[every node/.style={draw=black}]
% % \draw[help lines] (0,0) grid (4,2);
% \matrix [row sep=0cm, column sep=0cm, style={align=center}] (my matrix) at (0,0) %(2,1)
% {
% \node[style={anchor=center}] {\includegraphics[width=\colWidth]{figures/photos/labFREEs3.jpg}}; %\fill[blue] (0,0) circle (2pt)
% &
% \node[style={anchor=center}] {\includegraphics[width=\colWidth, height=160pt]{figures/stewartRender.png}}; %\fill[blue] (0,0) circle (2pt);
% \\
% };

% %\node[style={anchor=center}] at (0,-5) (FREEstate) {\includegraphics[width=0.7\linewidth]{figures/FREEstate_noLabels2.pdf}};

% \end{tikzpicture}

\caption{\revcomment{2.3}{(a) A fiber-reinforced elastomerc enclosure (FREE) is a soft fluid-driven actuator composed of an elastomer tube with fibers wound around it to impose specific deformations under an increase in volume, such as extension and torsion. (b) A linear actuator and motor combined in \emph{series} has the ability to generate 2 dimensional forces at the end effector (shown in red), but is constrained to motions only in the directions of these forces. (b) Three FREEs combined in \emph{parallel} can generate the same 2 dimensional forces at the end effector (shown in red), without imposing kinematic constraints that prohibit motion in other directions (shown in blue).}}

% \caption{A fiber-reinforced elastomeric enclosure (FREE) (top) is a soft fluid-driven actuator composed of an elastomer tube with fibers wound around it to impose deformation in specific directions upon pressurization, such as extension and torsion. \revcomment{2.3}{In this paper we explore the potential of combining multiple FREEs in parallel to generate fully controllable multi-dimensional spacial forces}, such as in a parallel arrangement around a flexible spine element (bottom-left), or a Stewart Platform arrangement (bottom-right).}

\label{fig:overview}
\end{figure}


\section{LDR panorama light source detection}
\label{sec:lightdetection}

In order to use LDR panoramas for training our CNN to detect light sources, we must first detect areas in the panoramas which correspond to bright light sources. To do so, we propose a novel light source detector, and show that it significantly outperforms the approach of Karsch et al.~\shortcite{karsch-tog-14}. 
%An overview of the light detection pipeline is shown in Fig.~\ref{f:hogflowchart}. 

\subsection{Light classification} 

After converting to grayscale, the panorama $P$ is rotated by $90^\circ$ about the pitch angle to yield $P_\text{rot}$, so that the zenith is aligned with the horizon line. This rotation is needed to account for the large distortions caused by the equirectangular projection, which severely stretches regions around the poles. Features are then computed over $P$ and $P_\text{rot}$ separately on square patches at five different scales\footnote{We use $30\times30$ squares at the lowest scale, multiplying their size by 1.5 at each scale.}. In particular, we use HOG~\cite{dalal-cvpr-05}, the mean patch elevation, as well as its mean, standard deviation, and 99th percentile intensity values. These features are used to train two logistic regression classifiers for small (e.g. spotlights and lamps) and large (e.g. windows, reflections) light sources. We found that training classifiers for these two types of classes separately yielded better performance, as these types of light sources significantly differ from one another. 

The resulting logistic regression classifiers are then applied in a sliding-window fashion over $P$ and $P_\text{rot}$ to yield a score at each pixel. 
%(fig.~\ref{f:hogflowchart}). 
Scores from both classifiers are added, then merged on a per-pixel manner according to their elevation angles $S_\text{merged} = S\cos(\theta) + S_\text{rot}^*\sin(\theta)$, where $S$ indicates the regression scores, $\theta$ the pixel elevation, and $S_\text{rot}^*$ is $S_\text{rot}$ rotated back to the original orientation. The resulting scores are then thresholded to obtain a binary mask, refined with a dense CRF~\cite{krahenbuhl-nips-12}, and adjusted with opening and closing morphological operations. The optimal threshold is obtained by maximizing the intersection-over-union (IoU) score between the resulting binary mask and the ground truth labels on the training set. 

\begin{figure}
    \centering
    \includegraphics[width=0.64\columnwidth]{images/globalPrCurves.pdf}
    \caption[]{Precision-recall curves for the light detector on the test set for our detectors and the one of Karsch et al.~\shortcite{karsch-tog-14}. In blue, the curve for the spotlights and lamps detection, and in green, the curve for the windows and light reflections. In red, the result for \cite{karsch-tog-14}. In cyan, the curve for a baseline detector relying solely on the intensity of a pixel. Note that because of the inherent uncertainty of the importance of a light (including reflections) relative to the others (even for a human annotator), a perfect match between human and algorithm predictions is highly unlikely.}
    \label{fig:prcurves}
\end{figure}

\begin{figure}[t]
\centering
\setlength{\tabcolsep}{1pt}
\begin{tabular}{cc}
\includegraphics[width=0.48\linewidth]{images/samplesPreclassifier/pano1.jpg} & 
\includegraphics[width=0.48\linewidth]{images/samplesPreclassifier/pano1_mask.jpg} \\
\includegraphics[width=0.48\linewidth]{images/samplesPreclassifier/pano2.jpg} & 
\includegraphics[width=0.48\linewidth]{images/samplesPreclassifier/pano2_mask.jpg} \\
\includegraphics[width=0.48\linewidth]{images/samplesPreclassifier/pano3.jpg} & 
\includegraphics[width=0.48\linewidth]{images/samplesPreclassifier/pano3_mask.jpg} \\
\includegraphics[width=0.48\linewidth]{images/samplesPreclassifier/pano4.jpg} & 
\includegraphics[width=0.48\linewidth]{images/samplesPreclassifier/pano4_mask.jpg} \\
\includegraphics[width=0.48\linewidth]{images/samplesPreclassifier/pano5.jpg} & 
\includegraphics[width=0.48\linewidth]{images/samplesPreclassifier/pano5_mask.jpg} \\
\end{tabular}
\caption{Light detection results on SUN360 panoramas. (left) the input LDR panoramas; (right) light detection results, shown in cyan and overlaid on the original panorama for reference. The detector is able to handle a wide range of lighting arrangements, including large light patches and spotlights.}
\label{f:lightdetection-results}
\end{figure}

\subsection{Training details and evaluation} 

To train the classifiers, we manually annotate a set of 400 panoramas from the SUN360 database. Four types of light sources are labelled: spotlights, lamps, windows, and (bounce) reflections. We use 80\% of the panoramas for training, and 20\% for testing. The classifier is first trained using labeled lights as positive samples and random negative samples. Subsequently, hard negative mining~\cite{felzenszwalb-pami-10} is used over the entire training set. We discard the bottom 15\% of the panoramas because this region often contain watermarks and light sources are seldom located below the camera. 

Fig.~\ref{fig:prcurves} reports a comparison of precision-recall curves for our two detectors, a baseline method which directly maps the intensity of a pixel to its probability of belonging to a light source, and the approach of Karsch et al.~\shortcite{karsch-tog-14}. As expected, the baseline performs poorly on LDR data like SUN360. The detector from \cite{karsch-tog-14} offers better performance, but our performs significantly better at any level of recall. Fig.~\ref{f:lightdetection-results} shows light detection results on example panoramas from the SUN360 dataset. 





%\begin{table}
%\centering
%\begin{tabular}{ccccc}
%\toprule
%\multirow{2}{*}{Method} & \multicolumn{2}{c}{All lights} & %\multicolumn{2}{c}{Spotlights only} \\
% & mean & median & mean & median \\
%\midrule
%\cite{karsch-tog-14} 	& 0.268 & 0.222 & 0.121 & 0.045 \\
%Ours 					& 0.326 & 0.298 & 0.305 & 0.299 \\
%\bottomrule
%\end{tabular}
%\caption[]{Comparison between our light detector and that of Karsch et al.~\shortcite{karsch-tog-14} on a test set of 80 hand-labeled LDR panoramas from the SUN360 database. We report the mean and median intersection-over-union (IoU) score for all light sources, and for spotlights only. Our approach significantly outperforms that of \cite{karsch-tog-14}, especially for detecting bright (but small) spotlights.}
%\label{t:lightdetection-iou}
%\end{table}

% From Marc-André:
% OUR DETECTOR :
% MEAN / MED IoU :  0.326159233623 0.298747417355
% MIN / MAX IoU :  0.0145741504337 0.79900990099
% MEAN / MED spots intersect :  0.305134993465 0.284664612467
% #############################
% KARSCH'S DETECTOR
% MEAN / MED IoU :  0.268475126288 0.22152084079
% MIN / MAX IoU :  0.0 0.854946427713
% MEAN / MED spots intersect :  0.12067937909 0.0453275380317



% \begin{figure}
% \centering
% \includegraphics[width=0.75\linewidth]{images/prCurves_pass2.png}
% \caption{Precision-recall curves for the light detector on the test set. In red, the curve for the spotlights and lamps detection, and in blue, the curve for the windows and light reflections. Note that because of the inherent uncertainty of the importance of a light (including reflections) relative to the others (even for a human annotator), a perfect match between human and algorithm predictions is highly unlikely. 
% \JF{Remove gray background, make axes black, make vectorized version}}
% \label{f:4prcurves}
% \end{figure}


%% NOTES FROM MARC-ANDRE
% General outline :
% \begin{enumerate}
%     \item We need to find a way to find the light sources in an LDR image. We cannot just use a pixel value because of the quick saturation (so a white pixel might not be contributing at all to the scene illumination).
%     \item Since we have the panoramas, most of the time we actually see these lights (except in the cases of occlusions). So we can train a light detector.
%     \item We use a well-known approach with a combination of HOG and logistic regression (by all means, a logistic regression is similar to a Linear SVM). We could have use more complicated approaches, but we show here that even this simpler approach works well and, in any way, we just need the light labeled to \textit{train} our network. Also, we append some information to the HOG descriptors vectors, like the elevation of the center of the patch, the mean color, etc.) Finally, we train two different classifiers, of which we combine the output, one for the 
%     spotlights and lamps, and another for windows and reflections, since their descriptors might be quite different.


%     \item To train the classifier, we manually annotated 400 panoramas. Each source have been labeled as one of these categories: spotlight, lamp, window, and reflection. Of these 400 panoramas, we retain 20\% (80 panoramas) as test set. The classifier is first trained using labeled lights as positive samples and random negative samples. Subsequently, we do hard negative mining over all the training set and train again the classifier using these hard negative samples. When evaluating the classifier, we drop the bottom 15\% of the panoramas because of the watermarks and also since the probability of a light being almost under the camera is quite low.


%     \item The first issue is that while HOG descriptors deal reasonably well with rotation and minor scaling, they do not work well with general deformation. Unluckily, a latlong projection distorts a lot the panorama, especially near the zenith, where it is not unreasonable to find a light... To alleviate this issue, we actually extract HOG features from two environment maps. The first one is the one passed as input (gray level), and the second one is rotated such as the zenith becomes part of the horizon. The ceiling lights are then less distorted and the same classifier may be used to find them.

%     \item As usual with HOGs, we use a sliding window at different scales. However, the lights are not perforce rectangular yet we would like to be able to roughly segment them. Each time a detection is fired, we add a constant value to its corresponding zone. This value is decreasing along with the scale. In the end, we obtain an heatmap of the most probable spots for light sources.

%     \item We combine the heatmap produced with the rotated and non-rotated panoramas, weighting them with a simple cosine rule, based on the fact that at the horizon, each detector has the maximum precision; on the contrary, at the zenith, the detector accuracy is quite poor.
%     \item Once we have merged the heatmap, we threshold it so to get a binary mask.

%     \item We apply a CRF on it to add some locality to the detections. Finally, we use standard morphological operators to remove outliers and inliers.
% \end{enumerate}
\section{Panorama recentering warp}
\label{sec:warping}

\begin{figure}[!t]
\centering
\includegraphics[width=0.99\linewidth]{images/warping/warping.pdf}
\caption{The importance of light locality for indoor scenes. Left, a photo for which we want to estimate the lighting conditions. The photo was cropped from the ``original'' panorama (top row, middle). Treating this panorama as the light source for the photo is wrong; its center of projection is in front of the scene in the photo, and relighting a virtual bunny (top row, right) makes it appear to be backlit. The correct HDR panorama, captured with a light probe at the position of the cropped photo, is shown in the middle row, and captures the location of the lights on top of the scene. We introduce a warping operator that can be estimated with no scene information, and distorts the original panorama to approximate the location of the light sources on the top (bottom row). Relighting an object with the warped panorama yields results that are much closer to the ground truth.}
\label{f:warping-problem}
\end{figure}

Detecting the light sources in LDR panoramas is not sufficient for training the CNN to learn lighting from a single photo. The fundamental problem is that the panorama does not represent the lighting conditions in the \emph{cropped scene}, since the panorama center of projection can be arbitrarily far away from the location of the scene points in the cropped photo. Fig.~\ref{f:warping-problem} illustrates this issue. The photo shown on the left was cropped from the ``original'' panorama in the middle column. Treating this original panorama as a light source is incorrect, and results in a backlit bunny. We captured the \emph{actual} lighting conditions by placing a light probe at the scene (middle column of fig.~\ref{f:warping-problem}). Notice how the lighting conditions at the scene differ from those in the original panorama. To allow the use of the SUN360 database (from which we can crop photos but do not have access to the scenes to capture ground truth lighting) for training, we present a novel method that warps the original panorama to approximate the lighting in the cropped photo (bottom row). 
%While our warping function is an approximation, the lighting conditions and relit object obtained with the warped panorama are much closer to the ground truth than the original version, and does not require physical access to the scene (bottom row of fig.~\ref{f:warping-problem}). We now detail this warping function and provide several qualitative examples that demonstrate the need for such an approach. 

\subsection{Warping operator}

The aim of the warping operator is to generate the panorama that would be captured by a virtual camera placed at a point in the cropped photo. This is a challenging problem that is made especially harder by the fact that we do not know the scene geometry, and we make two assumptions to make this task feasible. First, we assume that the scene lies on a sphere, i.e., all scene points are equidistant from the original center of projection. Second, we assume that an image warping suffices to model the effect of moving the camera, i.e., occlusions are not an important factor. These assumptions may not hold for all scene points; however, note that our goal is to model light sources, which are typically located at scene extremities (ceiling, walls, etc.) and are better approximated by these assumptions.

Let us assume that the panorama is placed on the unit sphere, i.e. $x^2 + y^2 + z^2 = 1$, with the camera that captured this panorama at the origin of this sphere. The outgoing rays emanating from a virtual camera placed at $(x_0,y_0,z_0)$, can be parameterized as:
%
\begin{equation}
x(t) = v_x t + x_0  \quad
y(t) = v_y t + y_0  \quad
z(t) = v_z t + z_0  \,.
\label{eq:warp2}
\end{equation}
%
Intersecting these rays with the panorama sphere yields:
%
\begin{equation}
    (v_x t + x_0)^2 + (v_y t + y_0)^2 + (v_z t + z_0)^2 = 1 \,.
    \label{eq:warp3}
\end{equation}
% 

As illustrated in fig.~\ref{f:warp-basics}, we want to model the effect of using a virtual camera whose nadir is at $\beta$. The angle $\beta$ corresponds to the point in the panorama where the photo is extracted, and we will discuss how this is computed shortly. For the case of translating along the $z$-axis, this results in a new camera center, $\{x_0, y_0, z_0\}$ = $\{0, 0, \sin \beta\}$. Warping in arbitrary directions can trivially be achieved by rotating the environment map before and after the warp. Substituting this in eq.~\ref{eq:warp3}, results in the following second degree equation:
%
\begin{equation}
(v_x^2 + v_y^2 + v_z^2)t^2 + 2v_z t \sin\beta  + \sin^2\beta - 1 = 0 \,.
\label{eq:warp4}
\end{equation}
%
Solving (\ref{eq:warp4}) for $t$ (keeping only positive solutions, as negative roots represent the intersection on the other side of the sphere), maps the coordinates from the original environment map to the ones in the warped camera coordinate system. 
%An illustrated example of the effect of the warp operator on an equirectangular panorama is shown in fig.~\ref{f:warp-explanations}. 

\begin{figure}[!t]
    \centering
    \includegraphics[width=0.375\linewidth]{images/diag_explanations_warp.eps}
    \caption{Overview of the warping problem, illustrated in 2D for simplicity. The circle represents a slice of the spherical panorama along the $y$--$z$ plane, with the center of projection (illustrated by a camera) at its center. The aim of the warp operator is create a virtual center of projection with a nadir at an angular distance of $\beta$ with respect to the original nadir. The angle $\beta$ corresponds to the point in the panorama where the photo is extracted.}
    \label{f:warp-basics}
\end{figure}

\begin{figure}[!t]
\centering
\footnotesize
\setlength{\tabcolsep}{1pt}
\begin{tabular}{cc}
\includegraphics[width=0.493\linewidth]{images/warping/findbeta_input_withx.png} &
\includegraphics[width=0.493\linewidth]{images/warping/findbeta_normals.png} \\
(a) Input image & (b) Normals \\
\includegraphics[width=0.493\linewidth]{images/warping/findbeta_output_withx.png} &
\includegraphics[width=0.493\linewidth]{images/warping/findbeta_panowarp.png} \\
(c) Original panorama & (d) Warped panorama \\
\end{tabular}
\caption{$\beta$ selection procedure. From a given crop picture (a), we extract the normals using the method of  Bansal et al.~\shortcite{bansal2016marr} (b). We pick the insertion point by looking at the lowest pixel with a horizontal surface (green X in (a)) and backproject it on to the panorama (c). This gives us the point where we would like the nadir to be, from which $\beta$ can be trivially recovered. We then warp the panorama using this $\beta$ (d).}
\label{f:warp-beta-pick}
\end{figure}


The value of $\beta$ in eq.~(\ref{eq:warp4}) represents the point in the panorama where the photo is extracted. We expect that users will want to insert objects on to flat horizontal surfaces in the photo, and we reflect this in the choice of $\beta$ as follows (see fig.~\ref{f:warp-beta-pick}): we first use the approach of Bansal et al.~\shortcite{bansal2016marr} to detect surface normals in the cropped image, and find flat surfaces by thresholding based on the angular distance between surface normal and the up vector. We back-project the $y$-coordinate of the lowest point of the largest flat area (i.e., the lowest point on the flattest horizontal surface) on to the panorama to obtain $\beta$. In cases where no horizontal surfaces are found (e.g., a flat vertical wall), no warp is applied as the panorama is assumed to be sufficiently close to scene. Note that we always assume the insertion point to be x-centered ---that is, we do not ask the network to estimate the light at far-left or far-right of the image.

% \begin{figure}
% \centering
% \footnotesize
% \setlength{\tabcolsep}{1pt}
% \begin{tabular}{cc}
% \includegraphics[height=2.2cm]{images/tilingpattern_0.png} &
% \includegraphics[height=2.2cm]{images/tilingpattern_sphere_0.png} \\
% \includegraphics[height=2.2cm]{images/tilingpattern_30.png} &
% \includegraphics[height=2.2cm]{images/tilingpattern_sphere_30.png} \\
% \includegraphics[height=2.2cm]{images/tilingpattern_60.png} &
% \includegraphics[height=2.2cm]{images/tilingpattern_sphere_60.png} \\
% (a) Equirectangular panorama & (b) Projected on a sphere
% \end{tabular}
% \caption{Effect of the warp operator on panoramas. Top row: the original environment map. Middle row: warp with $\beta=30^\circ$. Bottom row: warp with $\beta=60^\circ$.}
% \label{f:warp-explanations}
% \end{figure}

\begin{figure}[!t]
\centering
\footnotesize
\setlength{\tabcolsep}{1pt}
\begin{tabular}{ccc}
\includegraphics[width=0.325\linewidth]{images/warping/images/good/noWarp/pano0767-others-135.jpg} &
\includegraphics[width=0.325\linewidth]{images/warping/images/good/withWarp/pano0767-others-135.jpg} & 
\includegraphics[width=0.325\linewidth]{images/warping/images/good/envyDepth/composeHenrique767.jpg} \\
\includegraphics[width=0.325\linewidth]{images/warping/images/good/noWarp/pano0460-others-270.jpg} &
\includegraphics[width=0.325\linewidth]{images/warping/images/good/withWarp/pano0460-others-270.jpg} &
\includegraphics[width=0.325\linewidth]{images/warping/images/good/envyDepth/composeHenrique460.jpg} \\
\includegraphics[width=0.325\linewidth]{images/warping/images/good/noWarp/pano0618-others-00.jpg} &
\includegraphics[width=0.325\linewidth]{images/warping/images/good/withWarp/pano0618-others-00.jpg} & 
\includegraphics[width=0.325\linewidth]{images/warping/images/good/envyDepth/composeHenrique618.jpg} \\
\includegraphics[width=0.325\linewidth]{images/warping/images/good/noWarp/pano0634-others-135.jpg} &
\includegraphics[width=0.325\linewidth]{images/warping/images/good/withWarp/pano0634-others-135.jpg} &
\includegraphics[width=0.325\linewidth]{images/warping/images/good/envyDepth/composeHenrique634.jpg} \\
(a) Original panorama & (b) Our warp & (c) \cite{banterle-cgf-13}
\end{tabular}
\caption{Comparison of objects relit with (a) the original panoramas, (b) our warped panoramas, and (c) panoramas warped using EnvyDepth~\cite{banterle-cgf-13}. The objects relit by our panoramas closely approximate those obtained with EnvyDepth, without the lengthy manual annotation required.}
\label{f:warp-results}
\end{figure}

\subsection{Impact on lighting estimation}

Fig.~\ref{f:warping-problem} compares our warped panorama with a ground truth panorama captured in-place for one scene. We also compare our spherical warp with a geometry-based warp obtained with EnvyDepth~\cite{banterle-cgf-13}, a system that extracts spatially-varying lighting from environment maps by projecting them onto proxy geometry estimated from manual annotations. Comparative relighting results using the original, spherical warp, and geometry-based warp panoramas are presented in fig.~\ref{f:warp-results}. While our operator makes several simplifying scene assumptions, these results illustrate that relighting with our approach provides a close approximation to more expensive techniques, while being completely automatic and without requiring access to the scene. In contrast, the manual labeling process required for the geometric warp takes around 10 minutes per panorama. 

The main limitation of our warping operator is that it fails to appropriately model occlusions. Since we treat the panorama as a projection on a sphere, lights that illuminate a scene point, but are not visible from the original camera are not handled by this approach. However, these situations are rare, and as we show in our results, our network filters them out as outliers, and learns a robust scene appearance to illumination mapping.


%objects protruding from the main surfaces (e.g. a table, columns, etc.) appear unrealistically distorted in the warped versions. However, since our goal is to obtain a more accurate representation for lighting, the distortions have minimal impact on the learning.

%\emiliano{note: This seems to be also relevant as trick for rendering scenes lit by a single IBL. You can assume the IBL refers to the origin of the scene and give it a radius. Then use this trick to adjust the shading away from the center. Also the sampling function is trivial to implement in a shader. This seems to be a relevant and basic CG problem, I am surprised nobody else tackled it. }


% Of course, this operator does not solve everything. In particular:
% \begin{itemize}
%     \item It greatly reduces the resolution in the front zone (and increases it behind the camera). SUN360 having 9000x4500 panoramas, this is not a huge issue in our case.
%     \item This assumes horizontal surfaces everywhere, which is clearly not the case for indoor scenes. We thus use the normal information (using a CNN to estimate them) to ensure that pathological cases do not happen.
%     \item This does not solve the issue in case of severe occlusions, but nothing can (the information we want is simply not there). All in all, as our results in fig.~\ref{f:warp-results} clearly show, we still get a consistently better lighting estimation than by using the IBL at the camera position.
% \end{itemize}


\section{Learning from LDR panoramas}
\label{sec:learning}

Now that we have the tools to extract accurate lighting information from LDR panoramas, we detail our approach for learning the relationship between a single photo and its lighting conditions. 


\begin{table}
\caption[]{The proposed CNN architecture. After a series of 7 convolutional layers (conv), some with residual connections (res), a fully-connected layer (FC) segues to two heads. The heads aim at reconstructing the light mask $\mathbf{y}_\text{mask}$ (left) and the RGB panorama $\mathbf{y}_\text{RGB}$ (right) through a series of deconvolutional layers (deconv). The ELU activation function~\cite{clevert-iclr-16} and batch normalization are used on all layers except the outputs, which are sigmoids for light mask and tangent hyperbolic for panorama prediction. The stride at each layer is indicated between parentheses. The ``res'' identifiers indicate residual layers~\cite{he-cvpr-16}. }
\centering
\begin{tabular}{c}
\toprule
\textbf{Layer (stride)} \\
\midrule
Input \\
\midrule
conv9-64 (2) \\
conv4-96 (2) \\
res3-96  (1) \\
res4-128 (2) \\
res4-192 (2) \\
res4-256 (2) \\
\midrule
FC-1024 \\
\end{tabular}
\\
\begin{tabular}{cc}
\midrule
FC-8192 & FC-6144 \\
deconv4-256 (2) & deconv4-192 (2) \\
deconv4-128 (2) & deconv4-128 (2) \\
deconv4-96 (2) & deconv4-64 (2) \\
deconv4-64 (2) & deconv4-32 (2) \\
deconv4-32 (2) & deconv4-24 (2) \\
conv5-1 (1) & conv5-3 (1) \\
Sigmoid & Tanh \\*[-.5em]
\noindent\rule{3.2cm}{.8pt} &
\noindent\rule{3.2cm}{.8pt} \\
Output: light mask $\mathbf{y}_\text{mask}$ &
Output: RGB panorama $\mathbf{y}_\text{RGB}$ \\
\end{tabular}
\label{t:learning-architecture}
\end{table}

\subsection{Training data preparation}
\label{sec:ldr-data-prep}

For each SUN360 indoor panorama, we compute the light mask to represent ground truth during the learning process (sec.~\ref{sec:lightdetection}). We then take 8 crops from each panorama at random elevation between $-30^\circ$ and $30^\circ$ and make a projection of them as rectilinear photos. Using our recentering warp (sec.~\ref{sec:warping}), we generate a corresponding warped panorama (and light mask) for each rectilinear photo. We also rotate the warped panorama and corresponding light mask so that the crop region always sits at center azimuth (fig.~\ref{f:overview}).  At the end of this process, we have 96,000 input-output pairs, where the input is a photo, and the output is a pair of a warped panorama and its corresponding light mask.

\subsection{Network architecture} 

As shown in table~\ref{t:learning-architecture}, we use a convolutional neural network that takes the photo as input, produces a low-dimensional encoding of the input through a series of convolutions downstream and splits into two upstream expansions, with two distinct tasks: (1) intensity estimation / binary light mask prediction, and (2) RGB panorama prediction. The encoder is split into two standard convolution layers, followed by four residual layers~\cite{he-cvpr-16}. The two individual decoders are exclusively composed of deconvolution layers. The input photo is of size $256\times192$, whereas the panorama and light mask outputs are of size $256\times128$. Each time a stride of 2 is encountered with a convolution (deconvolution) layer, the resolution of its output feature map is divided (multiplied) by two. The output light mask $\mathbf{x}_\text{mask}$ represents the probability of light for each pixel in the environment map. The RGB panorama $\mathbf{x}_\text{mask}$ serves as a high level colored texture in the final environment map. Please see sec.~\ref{sec:experiments} for several examples of estimated light masks and RGB panoramas.


\begin{figure}
\centering
\footnotesize
\setlength{\tabcolsep}{1pt}
\begin{tabular}{ccc}
\includegraphics[width=0.32\linewidth]{images/cosfilter/base.png} &
\includegraphics[width=0.32\linewidth]{images/cosfilter/cos_1.png} &
\includegraphics[width=0.32\linewidth]{images/cosfilter/cos_5.png} \\
(a) Original & (b) $\alpha e=1$ & (c) $\alpha e=5$ \\
\includegraphics[width=0.32\linewidth]{images/cosfilter/cos_10.png} &
\includegraphics[width=0.32\linewidth]{images/cosfilter/cos_20.png} &
\includegraphics[width=0.32\linewidth]{images/cosfilter/cos_80.png} \\
(d) $\alpha e=10$ & (e) $\alpha e=20$ & (f) $\alpha e=80$ \\
\end{tabular}
\caption{Effect of the cosine blurring from eq.~(\ref{e:filter}) on the light mask at various blurring levels. Note how this simple, differentiable scheme allows a smooth progression towards higher frequency content over time, but without the ringing artifacts of spherical harmonics.}
\label{f:learning-filter}
\end{figure}

\subsection{Loss function} 

For the RGB panorama prediction task, we use an L2 distance on the pixel output:
\begin{equation}
    \mathcal{L}_\text{L2}(\mathbf{y}, \mathbf{t}) = \frac{1}{N}\sum_{i=1}^{N} \mathbf{s}_i (\mathbf{y}_i - \mathbf{t}_i)^2 \,,
\label{e:rgbloss}
\end{equation}
where $N=\mathtt{width}\times\mathtt{height}\times 3$ is the total number of elements in the image, $\mathbf{y}$ is the network prediction, $\mathbf{t}$ the ground truth panorama and $\mathbf{s}_i$ the solid angle for pixel $i$.

Designing the loss function for the light mask $\mathbf{y}_\text{mask}$ is not as straightforward. Take, for example, the standard L2 or binary cross entropy losses computed on the light mask directly. If a small bright spotlight is estimated to be located slightly off its ground truth location, a huge penalty will incur. Since pinpointing the exact location of all the light sources from a single photo is not necessary, we instead blur the target light mask with a filter and compute the L2 loss on the blurred version. The filter starts with a coarse, low-frequency representation of the target light mask and progressively sharpens it over training time. To this end, we design a filter based on the cosine distance, followed by an L2 loss for the light mask: 
\begin{equation}
    \mathcal{L}_\text{cos}(\mathbf{y}, \mathbf{t}, e) = \frac{1}{N}\sum_{i=1}^{N} (\mathcal{F}(\mathbf{y}, i, e) - \mathcal{F}(\mathbf{t}, i, e))^2 \,,
    \label{e:maskloss}
\end{equation}
where $e$ is a real value corresponding to the current epoch (formally, $e=\textrm{\#epochs}+\textrm{\#current-mini-batch}/\textrm{\#total-mini-batches}$.). The filter $\mathcal{F}$ is defined as:
\begin{equation}
    \mathcal{F}(\mathbf{p}, i, e) = \frac{1}{K_i} \sum_{\omega \in \Omega_i} \mathbf{p}(\omega) s(\omega) \left( \omega \cdot n_i \right)^{\alpha e},
    \label{e:filter}
\end{equation}
where $\Omega_i$ is the hemisphere centered at pixel $i$ on the panorama $\mathbf{p}$,  $n_i$ the unit normal at pixel $i$, and $K_i$ the sum of solid angles on $\Omega_i$. $\omega$ is a unit vector in a specific direction on $\Omega_i$ and $s(\omega)$ the solid angle for the pixel in the direction $\omega$. As seen before, we define $e\in\mathbb{R}$ as the real valued number of training samples collectively seen, normalized by the total number of training samples. 

\begin{figure}
\centering
\footnotesize
\setlength{\tabcolsep}{1pt}
\begin{tabular}{cc}
\includegraphics[width=0.493\linewidth]{images/lossLDR.pdf} &
\includegraphics[width=0.493\linewidth]{images/lossHDR.pdf} \\
(a) LDR network & (b) HDR network
\end{tabular}
\caption{Evolution of training and test loss over the number of epochs for the (a) LDR and (b) HDR training.}
\label{f:learning-curves}
\end{figure}

Since eq.~\ref{e:filter} is differentiable, back-propagation can be used to efficiently train our CNN. Fig.~\ref{f:learning-filter} shows a visual example of the effect of the cosine distance filter on a binary light mask. Note how the target light mask becomes progressively sharper over time. 

\begin{figure*}[!t]
\centering
\footnotesize
\setlength{\tabcolsep}{1pt}
\begin{tabular}{cccc}
\includegraphics[height=2.5cm]{{images/bunnyRenders/pano_ajwmyarsmgyaxz-others-90-1.25792-0.96443_render}.jpg} & 
\includegraphics[height=2.5cm]{{images/bunnyRenders/pano_ajwmyarsmgyaxz-others-90-1.25792-0.96443_mask}.jpg} & 
\hspace{.5em}
%\includegraphics[height=2.5cm]{{images/bunnyRenders/pano_abumqtqhptujdn-kitchen-135-1.06244-0.98464_render}.jpg} & 
%\includegraphics[height=2.5cm]{{images/bunnyRenders/pano_abumqtqhptujdn-kitchen-135-1.06244-0.98464_mask}.jpg} \\
\includegraphics[height=2.5cm]{{images/bunnyRenders/pano_agrayivbwqkxds-others-270-1.63912-0.96887_render}.jpg} & 
\includegraphics[height=2.5cm]{{images/bunnyRenders/pano_agrayivbwqkxds-others-270-1.63912-0.96887_mask}.jpg} \\
%
\includegraphics[height=2.5cm]{{images/bunnyRenders/pano_adghmppfkzisfi-others-135-1.50414-0.96319_render}.jpg} & 
\includegraphics[height=2.5cm]{{images/bunnyRenders/pano_adghmppfkzisfi-others-135-1.50414-0.96319_mask}.jpg} & 
\hspace{.5em}
\includegraphics[height=2.5cm]{{images/bunnyRenders/pano_ajxsprezaqhacq-restaurant-315-1.87811-1.08249_render}.jpg} & 
\includegraphics[height=2.5cm]{{images/bunnyRenders/pano_ajxsprezaqhacq-restaurant-315-1.87811-1.08249_mask}.jpg} \\
%\includegraphics[height=2.5cm]{{images/bunnyRenders/pano_aedlvoixsqeuog-others-315-1.05417-1.13734_render}.jpg} & 
%\includegraphics[height=2.5cm]{{images/bunnyRenders/pano_aedlvoixsqeuog-others-315-1.05417-1.13734_mask}.jpg} \\
%
\includegraphics[height=2.5cm]{{images/bunnyRenders/pano_akpwzsghfylcvp-museum-270-1.56694-0.96451_render}.jpg} & 
\includegraphics[height=2.5cm]{{images/bunnyRenders/pano_akpwzsghfylcvp-museum-270-1.56694-0.96451_mask}.jpg} & 
\hspace{.5em}
%\includegraphics[height=2.5cm]{{images/bunnyRenders/pano_akelvanxoywmql-workshop-180-1.44527-1.07197_render}.jpg} & 
%\includegraphics[height=2.5cm]{{images/bunnyRenders/pano_akelvanxoywmql-workshop-180-1.44527-1.07197_mask}.jpg} \\
\includegraphics[height=2.5cm]{{images/bunnyRenders/pano_ajzjecrfajjfdl-church-45-1.93004-1.12643_render}.jpg} & 
\includegraphics[height=2.5cm]{{images/bunnyRenders/pano_ajzjecrfajjfdl-church-45-1.93004-1.12643_mask}.jpg} \\
%
\includegraphics[height=2.5cm]{{images/bunnyRenders/pano_ahffjewqynvufc-others-180-1.62782-1.18096_render}.jpg} & 
\includegraphics[height=2.5cm]{{images/bunnyRenders/pano_ahffjewqynvufc-others-180-1.62782-1.18096_mask}.jpg} & 
\hspace{.5em}
\includegraphics[height=2.5cm]{{images/bunnyRenders/pano_alauchiodctyya-others-45-1.81179-1.15641_render}.jpg} & 
\includegraphics[height=2.5cm]{{images/bunnyRenders/pano_alauchiodctyya-others-45-1.81179-1.15641_mask}.jpg} \\
%
(a) Relit with estimate & (b) Predicted light probability & 
\hspace{.5em}
(c) Relit with estimate & (d) Predicted light probability
\end{tabular}
\caption{Evaluation of the LDR network at predicting light source positions. For each example, we show a virtual bunny model inserted in a background image and relit with the LDR network estimate for that image ((a) and (c)), and the predicted lighting probabilities overlaid on the panorama ((b) and (d)). As can been seen, our method generalizes to a wide range of indoor scenes and illumination conditions. Many more examples are available in the supplementary material.}
\label{f:relighting-bunnies}
\end{figure*}

The global loss function is then defined as:
%
\begin{equation}
    \mathcal{L}(\mathbf{y}, \mathbf{t}, e) = w_1 \mathcal{L}_\text{L2}(\mathbf{y}_\text{RGB}, \mathbf{t}_\text{RGB})  
    + w_2 \mathcal{L}_\text{cos}(\mathbf{y}_\text{mask}, \mathbf{t}_\text{mask}, e) \,.
\label{e:globloss}
\end{equation}
%
In our experiments, we use $w_1=100$, $w_2=1$, and $\alpha=3$.

Our filtering scheme also has a rendering-based interpretation. It is well known that surface reflection for Lambertian objects can be modeled as low-pass filtering~\cite{ramamoorthi-sig-01}, while specular objects preserve more high frequencies of the illumination. In this sense, our loss function can be thought of evaluating the inferred illumination in terms of the resulting \emph{appearance} of spheres with increasingly glossy surface reflectance. In this vein, we experimented with directly representing the illumination with spherical harmonics (gradually increasing the number of coefficients to represent higher frequencies of illumination), but found that the network had a tendency to overfit to the ringing artifacts caused by high frequencies in the binary light mask.

\subsection{Training details} 
\label{sec:training-details}

We use $85\%$ of the panoramas as training data, and $15\%$ as test data. Note that we generate the train-test split such that no crop of the test panoramas exist in the training set. Hence, all tests are performed for scenes and lighting conditions that have not been seen by the network before. We use the ADAM optimizer~\cite{kingma2014adam} with a minibatch size of $64$, learning rate of $0.005$, and momentum parameters of $\beta_1=0.9$, $\beta_2=0.999$. Fig.~\ref{f:learning-curves}-(a) shows the loss (from eq.~(\ref{e:globloss})) curves on the training and test set during training. Training takes roughly 40 hours on an Nvidia Titan X Pascal GPU. At test time, lighting inference (both mask and RGB) from a photo takes approximately 10ms. The batch size was selected so it fills the 12GB memory of the GPU.



\section{Learning high dynamic range illumination}
\label{sec:hdr}

Up to this point we have trained a network that can predict the \emph{position} of the light sources quite accurately (see sec.~\ref{sec:experiments}), but, since it was trained on LDR data, it does not know about the \emph{intensities} of the light sources. In this section, we further train the network on a novel dataset of high dynamic range panoramas which enables it to jointly reason about light source direction and intensity. 

\subsection{A new dataset of HDR indoor panoramas}

We have captured a novel dataset of 2,100 high-resolution ($7768\times3884$), high dynamic range indoor panoramas. To do so, a Canon 5D Mark III camera with a Sigma 8mm fisheye lens was mounted on a tripod equipped with a robotic panoramic tripod head, and programmed to shoot 7 bracketed exposures at $60^\circ$ increments. The photos were shot in RAW mode, and automatically stitched into a 22 f-stop HDR $360^\circ$ panorama using the PTGui Pro commercial software. The dynamic range is sufficient to correctly expose all pixels in the scenes, including the light sources. Panoramas were captured in a wide variety of indoor environments, such as schools, houses, apartments, museums, laboratories, factories, sports facilities, etc. A visual overview of panoramas in our novel HDR dataset is shown in the supplementary material. The size and variety of this dataset is significantly larger than other similar datasets in the literature (which consist of tens of panoramas), making it extremely useful for training and testing a wide range of problems from scene inference, high dynamic range image processing, and rendering\footnote{This dataset is publicly available at \url{http://www.jflalonde.ca/projects/deepIndoorLight}.}. 

\subsection{Adapting the network to HDR data}

Since the light sources are not saturated in the HDR data, the network can be adjusted to directly learn the light source \emph{intensities} $\mathbf{y}_\text{int}$ instead of the binary light mask $\mathbf{y}_\text{mask}$. To do so, the network undergoes the following four simple changes. First, the weights of the last layer of the light mask predictor (``conv5-1'' in table~\ref{t:learning-architecture}) are initialized to random values. Second, training is performed to update only the weights of the decoders---that is, up to the FC-1024 layer in table~\ref{t:learning-architecture}. This is done to avoid overfitting on the encoder. Third, the target intensity $\mathbf{t}_\text{int}$ is defined as the log of the HDR intensity ($\log_{10}$ is used). Low intensities (below the median of the training dataset) are clamped to 0, since we only care about the light sources: in the unusual case where no pixels would be over this threshold, the ambient term given by the RGB recovery should be enough to light the scene. Finally, the loss is modified to:
%
\begin{align}
    \mathcal{L}_\text{HDR}(\mathbf{y}, \mathbf{t}, e) &= w_1 \mathcal{L}_\text{L2}(\mathbf{y}_\text{RGB}, \mathbf{t}_\text{RGB}) \nonumber \\ 
    &+ w_2 \mathcal{L}_\text{cos}(\mathbf{y}_\text{int}, \mathbf{t}_\text{int}, e)
    + w_3 \mathcal{L}_\text{L2}(\mathbf{y}_\text{int}, \mathbf{t}_\text{int}, e)  \,,
\label{e:hdrloss}
\end{align}
%
where $\mathcal{L}_\text{L2}$ and $\mathcal{L}_\text{cos}$ were defined in eq.~(\ref{e:rgbloss}) and (\ref{e:maskloss}) respectively, and $e$ is continued from training on the LDR data (so the HDR intensities are not overblurred). The L2 term on the intensity was added to reduce deconvolution artifacts. Here, $w_1 = 10$, $w_2 = 1$, and $w_3 = 0.1$. Training is otherwise performed with the same parameters as in sec.~\ref{sec:training-details}, and, just as with the LDR data, 85\% of the HDR data was used for training and 15\% for testing. Similar to the LDR data (sec.~\ref{sec:ldr-data-prep}), 8 crops were extracted from each panorama in the HDR dataset, yielding 14,000 input-output pairs. These are tone-mapped to ensure that the input to the network are LDR images. Finally, the panoramas are also warped using the same procedure as their LDR counterparts. Fig.~\ref{f:learning-curves}-(b) shows the loss (from eq.~(\ref{e:hdrloss})) curves on the training and test set during training. 



%!TEX ROOT = ../../centralized_vs_distributed.tex

\section{{\titlecap{the centralized-distributed trade-off}}}\label{sec:numerical-results}

\revision{In the previous sections we formulated the optimal control problem for a given controller architecture
(\ie the number of links) parametrized by $ n $
and showed how to compute minimum-variance objective function and the corresponding constraints.
In this section, we present our main result:
%\red{for a ring topology with multiple options for the parameter $ n $},
we solve the optimal control problem for each $ n $ and compare the best achievable closed-loop performance with different control architectures.\footnote{
\revision{Recall that small (large) values of $ n $ mean sparse (dense) architectures.}}
For delays that increase linearly with $n$,
\ie $ f(n) \propto n $, 
we demonstrate that distributed controllers with} {few communication links outperform controllers with larger number of communication links.}

\textcolor{subsectioncolor}{Figure~\ref{fig:cont-time-single-int-opt-var}} shows the steady-state variances
obtained with single-integrator dynamics~\eqref{eq:cont-time-single-int-variance-minimization}
%where we compare the standard multi-parameter design 
%with a simplified version \tcb{that utilizes spatially-constant feedback gains
and the quadratic approximation~\eqref{eq:quadratic-approximation} for \revision{ring topology}
with $ N = 50 $ nodes. % and $ n\in\{1,\dots,10\} $.
%with $ N = 50 $, $ f(n) = n $ and $ \tau_{\textit{min}} = 0.1 $.
%\autoref{fig:cont-time-single-int-err} shows the relative error, defined as
%\begin{equation}\label{eq:relative-error}
%	e \doteq \dfrac{\optvarx-\optvar}{\optvar}
%\end{equation}
%where $ \optvar $ and $ \optvarx $ denote the the optimal and sub-optimal scalar variances, respectively.
%The performance gap is small
%and becomes negligible for large $ n $.
{The best performance is achieved for a sparse architecture with  $ n = 2 $ 
in which each agent communicates with the two closest pairs of neighboring nodes. 
This should be compared and contrasted to nearest-neighbor and all-to-all 
communication topologies which induce higher closed-loop variances. 
Thus, 
the advantage of introducing additional communication links diminishes 
beyond}
{a certain threshold because of communication delays.}

%For a linear increase in the delay,
\textcolor{subsectioncolor}{Figure~\ref{fig:cont-time-double-int-opt-var}} shows that the use of approximation~\eqref{eq:cont-time-double-int-min-var-simplified} with $ \tilde{\gvel}^* = 70 $
identifies nearest-neighbor information exchange as the {near-optimal} architecture for a double-integrator model
with ring topology. 
This can be explained by noting that the variance of the process noise $ n(t) $
in the reduced model~\eqref{eq:x-dynamics-1st-order-approximation}
is proportional to $ \nicefrac{1}{\gvel} $ and thereby to $ \taun $,
according to~\eqref{eq:substitutions-4-normalization},
making the variance scale with the delay.

%\mjmargin{i feel that we need to comment about different results that we obtained for CT and DT double-intergrator dynamics (monotonic deterioration of performance for the former and oscillations for the latter)}
\revision{\textcolor{subsectioncolor}{Figures~\ref{fig:disc-time-single-int-opt-var}--\ref{fig:disc-time-double-int-opt-var}}
show the results obtained by solving the optimal control problem for discrete-time dynamics.
%which exhibit similar trade-offs.
The oscillations about the minimum in~\autoref{fig:disc-time-double-int-opt-var}
are compatible with the investigated \tradeoff~\eqref{eq:trade-off}:
in general, 
the sum of two monotone functions does not have a unique local minimum.
Details about discrete-time systems are deferred to~\autoref{sec:disc-time}.
Interestingly,
double integrators with continuous- (\autoref{fig:cont-time-double-int-opt-var}) ad discrete-time (\autoref{fig:disc-time-double-int-opt-var}) dynamics
exhibits very different trade-off curves,
whereby performance monotonically deteriorates for the former and oscillates for the latter.
While a clear interpretation is difficult because there is no explicit expression of the variance as a function of $ n $,
one possible explanation might be the first-order approximation used to compute gains in the continuous-time case.
%which reinforce our thesis exposed in~\autoref{sec:contribution}.

%\begin{figure}
%	\centering
%	\includegraphics[width=.6\linewidth]{cont-time-double-int-opt-var-n}
%	\caption{Steady-state scalar variance for continuous-time double integrators with $ \taun = 0.1n $.
%		Here, the \tradeoff is optimized by nearest-neighbor interaction.
%	}
%	\label{fig:cont-time-double-int-opt-var-lin}
%\end{figure}
}

\begin{figure}
	\centering
	\begin{minipage}[l]{.5\linewidth}
		\centering
		\includegraphics[width=\linewidth]{random-graph}
	\end{minipage}%
	\begin{minipage}[r]{.5\linewidth}
		\centering
		\includegraphics[width=\linewidth]{disc-time-single-int-random-graph-opt-var}
	\end{minipage}
	\caption{Network topology and its optimal {closed-loop} variance.}
	\label{fig:general-graph}
\end{figure}

Finally,
\autoref{fig:general-graph} shows the optimization results for a random graph topology with discrete-time single integrator agents. % with a linear increase in the delay, $ \taun = n $.
Here, $ n $ denotes the number of communication hops in the ``original" network, shown in~\autoref{fig:general-graph}:
as $ n $ increases, each agent can first communicate with its nearest neighbors,
then with its neighbors' neighbors, and so on. For a control architecture that utilizes different feedback gains for each communication link
	(\ie we only require $ K = K^\top $) we demonstrate that, in this case, two communication hops provide optimal closed-loop performance. % of the system.}

Additional computational experiments performed with different rates $ f(\cdot) $ show that the optimal number of links increases for slower rates: 
for example, 
the optimal number of links is larger for $ f(n) = \sqrt{n} $ than for $ f(n) = n $. 
\revision{These results are not reported because of space limitations.}
\mySection{Related Works and Discussion}{}
\label{chap3:sec:discussion}

In this section we briefly discuss the similarities and differences of the model presented in this chapter, comparing it with some related work presented earlier (Chapter \ref{chap1:artifact-centric-bpm}). We will mention a few related studies and discuss directly; a more formal comparative study using qualitative and quantitative metrics should be the subject of future work.

Hull et al. \citeyearpar{hull2009facilitating} provide an interoperation framework in which, data are hosted on central infrastructures named \textit{artifact-centric hubs}. As in the work presented in this chapter, they propose mechanisms (including user views) for controlling access to these data. Compared to choreography-like approach as the one presented in this chapter, their settings has the advantage of providing a conceptual rendezvous point to exchange status information. The same purpose can be replicated in this chapter's approach by introducing a new type of agent called "\textit{monitor}", which will serve as a rendezvous point; the behaviour of the agents will therefore have to be slightly adapted to take into account the monitor and to preserve as much as possible the autonomy of agents.

Lohmann and Wolf \citeyearpar{lohmann2010artifact} abandon the concept of having a single artifact hub \cite{hull2009facilitating} and they introduce the idea of having several agents which operate on artifacts. Some of those artifacts are mobile; thus, the authors provide a systematic approach for modelling artifact location and its impact on the accessibility of actions using a Petri net. Even though we also manipulate mobile artifacts, we do not model artifact location; rather, our agents are equipped with capabilities that allow them to manipulate the artifacts appropriately (taking into account their location). Moreover, our approach considers that artifacts can not be remotely accessed, this increases the autonomy of agents.

The process design approach presented in this chapter, has some conceptual similarities with the concept of \textit{proclets} proposed by Wil M. P. van der Aalst et al. \citeyearpar{van2001proclets, van2009workflow}: they both split the process when designing it. In the model presented in this chapter, the process is split into execution scenarios and its specification consists in the diagramming of each of them. Proclets \cite{van2001proclets, van2009workflow} uses the concept of \textit{proclet-class} to model different levels of granularity and cardinality of processes. Additionally, proclets act like agents and are autonomous enough to decide how to interact with each other.

The model presented in this chapter uses an attributed grammar as its mathematical foundation. This is also the case of the AWGAG model by Badouel et al. \citeyearpar{badouel14, badouel2015active}. However, their model puts stress on modelling process data and users as first class citizens and it is designed for Adaptive Case Management.

To summarise, the proposed approach in this chapter allows the modelling and decentralized execution of administrative processes using autonomous agents. In it, process management is very simply done in two steps. The designer only needs to focus on modelling the artifacts in the form of task trees and the rest is easily deduced. Moreover, we propose a simple but powerful mechanism for securing data based on the notion of accreditation; this mechanism is perfectly composed with that of artifacts. The main strengths of our model are therefore : 
\begin{itemize}
	\item The simplicity of its syntax (process specification language), which moreover (well helped by the accreditation model), is suitable for administrative processes;
	\item The simplicity of its execution model; the latter is very close to the blockchain's execution model \cite{hull2017blockchain, mendling2018blockchains}. On condition of a formal study, the latter could possess the same qualities (fault tolerance, distributivity, security, peer autonomy, etc.) that emanate from the blockchain;
	\item Its formal character, which makes it verifiable using appropriate mathematical tools;
	\item The conformity of its execution model with the agent paradigm and service technology.
\end{itemize}
In view of all these benefits, we can say that the objectives set for this thesis have indeed been achieved. However, the proposed model is perfectible. For example, it can be modified to permit agents to respond incrementally to incoming requests as soon as any prefix of the extension of a bud is produced. This makes it possible to avoid the situation observed on figure \ref{chap3:fig:execution-figure-4} where the associated editor is informed of the evolution of the subtree resulting from $C$ only when this one is closed. All the criticisms we can make of the proposed model in particular, and of this thesis in general, have been introduced in the general conclusion (page \pageref{chap5:general-conclusion}) of this manuscript.





\section*{Acknowledgements}

The authors would like to thank Yannick Hold-Geoffroy for his help in setting up the renders and the user study. We would also like to thank Henrique Weber for his help with EnvyDepth, and Jean-Michel Fortin for his work on HDR data capture.

Parts of this work were done while Marc-Andr\'e Gardner was an intern at Adobe Research. This work was partially supported by the REPARTI Strategic Network and the FRQNT New Researcher Grant 2016NC189939. We gratefully acknowledge the support of Nvidia with the donation of the GPUs used for this research, funding from Adobe to cover the cost of HDR dataset acquisition, as well as a generous gift from Adobe to J.-F. Lalonde. 

\bibliographystyle{ACM-Reference-Format}
\bibliography{template}
\end{document}