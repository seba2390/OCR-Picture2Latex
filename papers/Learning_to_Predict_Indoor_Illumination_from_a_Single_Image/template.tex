\documentclass[acmtog]{acmart}

\usepackage{booktabs} % For formal tables


\makeatletter
\def\runningfoot{\def\@runningfoot{}}
\def\firstfoot{\def\@firstfoot{}}
\makeatother
\settopmatter{printacmref=false} % Removes citation information below abstract
\renewcommand\footnotetextcopyrightpermission[1]{} % removes footnote with conference information in first column
\pagestyle{plain} % removes running headers

% TOG prefers author-name bib system with square brackets
\citestyle{acmauthoryear}
\setcitestyle{square}

\usepackage[ruled]{algorithm2e} % For algorithms
\renewcommand{\algorithmcfname}{ALGORITHM}
\SetAlFnt{\small}
\SetAlCapFnt{\small}
\SetAlCapNameFnt{\small}
\SetAlCapHSkip{0pt}
\IncMargin{-\parindent}

% Metadata Information

\setcopyright{none}
%\acmJournal{TOG}
%\acmYear{2017}\acmVolume{36}\acmNumber{6}\acmArticle{176}\acmMonth{11}
%\acmDOI{10.1145/3130800.3130891}



\makeatletter
\newcommand\footnoteref[1]{\protected@xdef\@thefnmark{\ref{#1}}\@footnotemark}
\makeatother

\usepackage{steinmetz}
\usepackage{physics}
\usepackage{graphicx}
\usepackage{amssymb}
\usepackage{gensymb}
%\usepackage[usenames,dvipsnames]{color}
\usepackage{mathtools}
\usepackage{multirow}
%\usepackage{algorithm}
%\usepackage{algorithmic}
\usepackage[english]{babel} % English language/hyphenation

\newcommand{\eg}{{\em e.g.,~}}
\newcommand{\ie}{{\em i.e.,~}}
\newcommand{\etc}{{\em etc.}}
\newcommand{\etal}{{\em et~al.~}}
\newcommand{\superscript}[1]{\ensuremath{^{\textrm{#1}}}}
\newcommand{\subscript}[1]{\ensuremath{_{\textrm{#1}}}}

\newcommand{\fn}[1]{Figure~\ref{fig:#1}}
\newcommand{\en}[1]{Equation~\ref{eqn:#1}}
\newcommand{\sn}[1]{Section~\ref{sec:#1}}
\newcommand{\tn}[1]{Table~\ref{tab:#1}}
\newcommand{\cn}[1]{Section~\ref{chap:#1}}
\newcommand{\an}[1]{Appendix~\ref{apnd:#1}}
\newcommand{\syn}[1]{{\em #1}}
\newcommand{\iu}{{i\mkern1mu}}

\DeclareMathOperator*{\argmin}{arg\,min}

% \newcommand{\ersin}[1]{\textcolor{red}{[\textsc{ERSIN}: \emph{#1}]}}
\newcommand{\JF}[1]{\textcolor{red}{[\textsc{JF}: \emph{#1}]}}
\newcommand{\MAG}[1]{\textcolor{green}{[\textsc{MAG}: \emph{#1}]}}
\newcommand{\KS}[1]{\textcolor{blue}{[\textsc{KS}: \emph{#1}]}}
% \newcommand{\emiliano}[1]{\textcolor{blue}{[\textsc{Emiliano}: \emph{#1}]}}
% \newcommand{\CG}[1]{\textcolor{red}{[\textsc{CG}: \emph{#1}]}}


% Document starts
\begin{document}

% Title portion
\title{Learning to Predict Indoor Illumination from a Single Image}

\author{Marc-Andr\'e Gardner}
\affiliation{Universit\'e Laval}
\email{marc-andre.gardner.1@ulaval.ca}
\author{Kalyan Sunkavalli}
\author{Ersin Yumer}
\author{Xiaohui Shen}
\author{Emiliano Gambaretto}
\affiliation{Adobe Research}
\author{Christian Gagn\'e}
\author{Jean-Fran\c cois Lalonde}
\affiliation{Universit\'e Laval}

\renewcommand\shortauthors{}

\begin{abstract}
We propose an automatic method to infer high dynamic range illumination from a single, limited field-of-view, low dynamic range photograph of an indoor scene.
%Inferring scene illumination from a single photograph is a challenging problem; the pixel intensities observed in a photograph are a complex function of scene geometry, reflectance properties, and illumination, all of which are unknown in our case. 
In contrast to previous work that relies on specialized image capture, user input, and/or simple scene models, we train an end-to-end deep neural network that directly regresses a limited field-of-view photo to HDR illumination, without strong assumptions on scene geometry, material properties, or lighting. We show that this can be accomplished in a three step process: 1) we train a robust lighting classifier to automatically annotate the location of light sources in a large dataset of LDR environment maps, 2) we use these annotations to train a deep neural network that predicts the location of lights in a scene from a single limited field-of-view photo, and 3) we fine-tune this network using a small dataset of HDR environment maps to predict light intensities. This allows us to automatically recover high-quality HDR illumination estimates that significantly outperform previous state-of-the-art methods. Consequently, using our illumination estimates for applications like 3D object insertion, produces photo-realistic results that we validate via a perceptual user study.

%\MAG{We capitalize on the high number of low-dynamic range panoramas available to train a network and then fine tune it using a limited HDR dataset.} 

 
\end{abstract}

%
% The code below should be generated by the tool at
% http://dl.acm.org/ccs.cfm
% Please copy and paste the code instead of the example below. 

%
% End generated code
%
%
% The code below should be generated by the tool at
% http://dl.acm.org/ccs.cfm
% Please copy and paste the code instead of the example below. 
%
\begin{CCSXML}
<ccs2012>
<concept>
<concept_id>10010147.10010178.10010224.10010225.10010227</concept_id>
<concept_desc>Computing methodologies~Scene understanding</concept_desc>
<concept_significance>500</concept_significance>
</concept>
<concept>
<concept_id>10010147.10010371.10010382</concept_id>
<concept_desc>Computing methodologies~Image manipulation</concept_desc>
<concept_significance>500</concept_significance>
</concept>
<concept>
<concept_id>10010147.10010371.10010382.10010236</concept_id>
<concept_desc>Computing methodologies~Computational photography</concept_desc>
<concept_significance>300</concept_significance>
</concept>
</ccs2012>
\end{CCSXML}

\ccsdesc[500]{Computing methodologies~Scene understanding}
\ccsdesc[500]{Computing methodologies~Image manipulation}
\ccsdesc[300]{Computing methodologies~Computational photography}
%
% End generated code
%
\keywords{indoor illumination, deep learning}

%\thanks{} 

\begin{teaserfigure}
\centering
\setlength{\fboxsep}{0pt}%
\setlength{\fboxrule}{1pt}%
  \includegraphics[width=0.24\textwidth]{images/stock-composites/35-crop.jpg}
	\begin{picture}(0,0)
		\put(-126,63){\fcolorbox{white}{white}{\includegraphics[height=1cm]{images/stock-composites/35-jpeg_envmap.jpg}}}
	\end{picture}
  \includegraphics[width=0.24\textwidth]{images/stock-composites/35-plants.jpg} 
	\hspace{0.2cm}
  \includegraphics[width=0.24\textwidth]{images/stock-composites/10-crop.jpg} 
	\begin{picture}(0,0)
		\put(-126,63){\fcolorbox{white}{white}{\includegraphics[height=1cm]{images/stock-composites/10-jpeg_envmap.jpg}}}
	\end{picture}
  \includegraphics[width=0.24\textwidth]{images/stock-composites/10-bicycle.jpg}
\caption{Given a single LDR image of an indoor scene, our method automatically predicts HDR lighting (insets, tone-mapped for visualization). Our method learns a direct mapping from image appearance to scene lighting from large amounts of real image data; it does not require any additional scene information, and can even recover light sources that are not visible in the photograph, as shown in these examples. Using our lighting estimates, virtual objects can be realistically relit and composited into photographs.}
\label{f:teaser}
\end{teaserfigure}
 
\maketitle


\vspace{15em}
Reinforcement learning has achieved great success in areas such as Game-playing \citep{silver2018general,vinyals2019grandmaster}, robotics \cite{kober2013reinforcement}, large language models \citep{ouyang2022training}, etc.
However, due to safety concerns or physical limitations, in some real-world reinforcement learning problems, we must consider additional constraints that may influence the optimal policy and the learning process \citep{garcia2015comprehensive}.
% For example, a robotic arm must not take actions that may cause harm to itself or the environments.
A standard framework to handle such cases is the constrained Markov Decision Process (CMDP) \citep{altman1999constrained}.
Within the CMDP framework, the agent has to maximize
the expected cumulative reward while
obeying a finite number of constraints, which are usually in the form of expected cumulative cost criteria.

However, we are sometimes concerned with the problem with a continuum of constraints.
For example,
the constraints we meet might be time-evolving or subject to uncertain parameters, which
cannot be formulated as an ordinary CMDP
(see Examples \ref{Example_Time_Evolving} and  \ref{Example_Uncertain}).
In this paper we would study a generalized CMDP  
to address the above problem.  Because the constraints are not only infinite-number but also lie
in a continuous set,
the generalization is not trivial. Fortunately, we find that we can borrow the idea behind semi-infinite programming (SIP) \citep{remez1934determination, hettich1993semi} to deal with the semi-infinite constraints.
Accordingly, we propose \emph{semi-infinitely constrained Markov decision processes} (SICMDPs)
as a novel complement to the ordinary CMDP framework.
%More specifically,  an SICMDP model %, we consider 
%contains a continuum of constraints whereas an ordinary CMDP contains a finite number of constraints. 

%This generalization is natural but not trivial. However, we can brows the idea  
%The idea is quite natural and can be backtracked
%to the practice of extending linear programming to linear semi-infinite programming (LSIP) %\cite{remez1934determination, GobernaLSIO1998}.
%In addition, 
%As a complementary approach to the ordinary CMDP framework, 
%SICMDP can be used to model these problems  which cannot be described by a finite number of constraints
%that are not covered by .
%For example,
%the restrictions we consider can be time-evolving or subject to uncertain parameters
%, thus
%cannot be described by a finite number of constraints but a continuum of constraints 
%(see Examples \ref{Example_Time_Evolving} and  \ref{Example_Uncertain}).

We also present two reinforcement learning algorithms to solve SICMDPs called SI-CRL and SI-CPO, respectively.
SI-CRL is a model-based reinforcement learning algorithm designed for tabular cases, and SI-CPO is a policy optimization algorithm for non-tabular cases.
% and analyze its performance both theoretically and empirically.
The main challenge is that we need to deal with a continuum of constraints, thus reinforcement learning algorithms for ordinary CMDPs do not work anymore.
In SI-CRL, we tackle this difficulty by first transforming the reinforcement learning problem to an equivalent LSIP problem, which can then be solved using methods in the LSIP literature like the dual exchange methods \citep{Hu1990,reemtsen1998numerical}.
In SI-CPO, we resort to the idea of cooperative stochastic approximation developed in \cite{lan2020algorithms, wei2020comirror}.
As far as we know, we are the first to introduce tools from semi-infinitely programming (SIP) into the reinforcement learning community for solving constrained reinforcement learning problems.

% To the best of our knowledge, we are the first to apply tools from semi-infinitely programming (SIP) to solve reinforcement learning problems.
Furthermore, we give theoretical analysis for both SI-CRL and SI-CPO.
We decompose the error of SI-CRL into two parts: the statistical error from approximating the true SICMDP with an offline dataset and the optimization error due to the fact that the solution of the LSIP problem obtained by the dual exchange method is inexact.
On the optimization side, we show that the iteration complexity of SI-CRL is $O\left(\left\{\mathrm{diam}(Y)L\sqrt{|\gS|^2|\gA|m}/\left[(1-\gamma)\epsilon\right]\right\}^m\right)$.
On the statistical side, we show that the sample complexity of SI-CRL is $\widetilde O\left(\frac{|S|^2|A|^2}{\epsilon^2(1-\gamma)^3}\right)$ if the offline dataset is generated by a generative model, and $\widetilde O\left(\frac{|S||A|}{\nu_{\min} \epsilon^2(1-\gamma)^3}\right)$ if the dataset is generated by a probability measure $\nu$ as considered in \cite{chen2019information}.
Here $\widetilde O$ means that all logarithm terms are discarded.
For SI-CPO, things become a little more complicated because other than the statistical error and the optimization error, we also need to consider the function approximation error, which comes from imperfect policy parametrizations.
It is shown if the function approximation error can be controlled to $O(\epsilon)$ order, the iteration complexity of SI-CPO is $\widetilde{O}\left(\frac{1}{\epsilon^2(1-\gamma)^6}\right)$ and the sample complexity of SI-CPO is $\widetilde{O}(\frac{1}{\epsilon^4(1-\gamma)^{10}})$.
Here our iteration complexity bound is equivalent to a typical $\widetilde O(1/\sqrt{T})$ global convergence rate.

We perform a set of numerical experiments to illustrate the SICMDP model and validate our proposed algorithms.
Specifically, we examine two numerical examples, namely the discharge of sewage and ship route planning.
Through the discharge of sewage example, we show the advantage of the SICMDP framework over the CMDP baseline obtained by naive discretization in modeling realistic sequential decision-making problems.
Moreover, we demonstrate the effectiveness of the SI-CRL and SI-CPO algorithms in such tabular environments. 
In the ship route planning example, we illustrate the benefits of the SICMDP framework and the ability of the SI-CPO algorithm to address complex continuous control tasks involving continuous state spaces with modern deep reinforcement learning techniques.

% In summary, our contributions are listed as follows.
% First, we present the SICMDP model, which can be viewed as a generalization of the ordinary CMDP model.
% Second, we propose an algorithm to perform reinforcement learning for SICMDPs, which is called SI-CRL, and we believe that we are the first to apply tools from SIP
% to solve reinforcement learning problems.
% Third, we give a theoretical analysis of SI-CRL and identify both its sample complexity and iteration complexity.
% In addition, we perform numerical experiments to illustrate the SICMDP model and validate the SI-CRL algorithm.
% \{This paragraph can be removed!!! \}






% Panoptic segmentation

% 3D segmentation

% Multi-object tracking

% Online 3D panoptic:

% PanopticFusion: (IROS 2019)
% https://arxiv.org/pdf/1903.01177.pdf
%
% - most similar to ours
% - PSPNet + M-RCNN + 2D fusion
% - volumetric mapping, 
% - greedy matching with IoU -> optimal only with 0.5 threshold
% - voxel & class weighting
% - CRF regularisation
%
% - good:
%
% - bad:
%  - CRF post-processing step
%  - greedy data-association
%    - can't be tuned for lower overlap ratios -> has to have high framerate, large changes in viewpoint could break this
%    - IoU: sensitive to 2D labels projecting over object borders (CRF and voxel weighting seem to alleviate this)

% Voxblox++: (Robotics & automation letters 2019)
% https://arxiv.org/pdf/1903.00268.pdf
% https://github.com/ethz-asl/voxblox-plusplus
%
% - M-RCNN + geometric segmentation + fusion 
% - data association of geometric segments with 3D overlap (no. points inside volume), fixed threshold for min number of points
% - instance label is assigned to a segment based on highest overlap
% - only one detected segment per reference label, as in PanopticFusion and Ours
% - TSDF Integration 
%
% good: 
% - because of geometric segmentation objects with no associated semantic class can also be segmented
% bad:
% - two different object segment types -> confusing, overly complicated ?
% - quite inaccurate (fixed below)

% Reconstructing Interactive 3D Scenes by Panoptic Mapping and CAD Model Alignments (ICRA 2021)
% https://arxiv.org/pdf/2103.16095.pdf
% https://github.com/hmz-15/Interactive-Scene-Reconstruction
%
% - based heavily on Voxblox++, much more accurate
% - Scene-graph ("contact graph") for mapping object relations
% - Search & replace voxels with CAD models, with geometrical and physical constraints
% - Object 6D pose
% - Format for robot interaction
%
% - Segmentation: bilateral fusion of geomatric and semantic segments -> reduce segmentation noise compared to Voxblox++
% - Fusion: triplet count improves consistency over Voxblox++ pairwise count strategy (take semantic label into account in addition to instance and geometry)
% - Fusion: instance labels are also combined if there is enough overlap with common geometric label for long enough time
%   - this means multiple detections can match the same reference unlike ours, voxblox++ and PanopticFusion ?
%

% Panoptic-MOPE: (ROBOTICS AND AUTOMATION LETTERS 2020)
% https://ieeexplore.ieee.org/stamp/stamp.jsp?tp=&arnumber=8977356
% https://github.com/hoangcuongbk80/Object-RPE/tree/panoptic-mope
%
% - novel RGB-D semantic segmentation model + M-RCNN
% - camera tracking based on "addaptively weighted optimization of geometric, appearance, and semantic cues"
% - surfel map: 
%   - how does it scale ? authors satate they tested on room-sized environments, but could be applied in larger scale as well ...
%     - could maybe be applied as VO in a SLAM algorithm ...
%   - demo only on a small pallet + surroundings, might not be applicable in large-scale SLAM

% US VS THEM:
%
% - based heavily on PanopticFusion, with modifications:
%   - instead of greedy data-association (which seems to be the case in others as well), we solve LAP (JPDA?)
%     - overlap threshold can be tuned, which renders the algorithm more flexible
%     - could be extended to dynamic tracking ?
%   - multiple options for association likelihood
%   - outlier rejection (either clustering or probabilistic)
%   - test different options for decreasing processing time
%   - no post-processing
%
% - model-agnostic:
%   - completely separated from segmentation
%   - does not care how point clouds are obtained -> applicable for LIDAR segmentation (e.g. EfficientLPS) as well
%
% - also agnostic to localisation method
%   - could, however, be utilised to find landmark locations / poses

% More compact version of this paragraph to introduction to save space?
%Panoptic segmentation -- proposed in \cite{panoptic_segmentation} -- aims to solve the unified task of semantic- and instance segmentation. Semantic classes are separated to \textit{stuff} -- amorphous, unquantifiable regions like sky, road or floor -- and \textit{things} -- quantifiable objects. The distinction between the two can vary depending on the application, but a semantic class can only belong to one or another. The article also proposes a unified panoptic evaluation metric, coined \textbf{Panoptic Quality} (PQ). Many 2D approaches to panoptic segmentation -- \textit{e.g.} \cite{panopticfpn,seamless,panoptic_deeplab,efficientps} -- have since been proposed. Deep neural networks for performing semantic- or instance segmentation directly on the 3D reconstruction -- \textit{e.g.} on \cite{scannet,s3dis,paris_lille_3d} -- have also been proposed, but since they require the reconstructed 3D scene, they are mostly offline approaches and therefore out of scope for this work. Some recent works also apply panoptic segmentation to point clouds -- \textit{e.g.} methods in the SemanticKITTI panoptic segmentation competition \cite{semantic_kitti} -- mostly aimed at segmenting LiDAR output. They are suitable for online processing, but similar to RGB-D images require a method for tracking object instances persistent in both time and space. In fact, our proposed method, as well as some others mentioned in this work, could use segmented LiDAR point clouds as an input similarly to RGB-D images.

PanopticFusion \cite{panopticfusion} is the first work to propose online integration of panoptic image segmentations to a 3D reconstruction. They integrate point clouds generated from segmented images to a TSDF voxel volume \cite{tsdf,voxblox} by greedily matching detected segments with the reconstruction and regulating each voxel's corresponding instance with a weighting function. Semantic labels are inferred in a bayesian manner based on confidence scores provided by the segmentation model. They also apply a Conditional Random Field (CRF) to regularise the reconstruction, improving results significantly. Voxblox++ \cite{voxblox++} -- introduced later the same year -- is a similar approach that also integrates segmented RGB-D images into a TSDF volume. It leverages geometric segmentation of depth images to improve instance segmentation accuracy. Both geometric and semantic segments are used to compute a pair-wise weight, which is used to greedily match them with segments in the reconstruction. Because of the geometric segmentation, the method allows segmentation of objects with no known semantic class in addition to objects recognised by the instance segmentation model. 

Recently, \cite{interactive_3d_scenes} built upon the idea of Voxblox++. They apply Voxblox++ for 3D instance integration, with two small but effective modifications: the pair-wise weight is replaced by a triplet weight that also takes semantic labels into account in the fusion, and -- in addition to geometric segments -- instance segments are fused if they overlap by a significant amount. The article introduces a method for searching and aligning CAD models to reconstructed objects based on geometry and semantic class, as well as geometrical and physical rules. With the CAD models, a contact graph and interactive virtual scene are reconstructed to allow a robot to simulate its interaction with the environment. SceneGraphFusion \cite{scenegraphfusion} is another approach that forms a scene graph online from a stream of RGB-D images, but unlike the above-mentioned approach, it generates the graph with a deep neural network, after which the panoptic labels for geometrically segmented portions of the 3D reconstruction are produced a side product.

Panoptic-MOPE \cite{panoptic_mope} is another recent approach, which integrates sequences of RGB-D images into a surfel reconstruction. Unlike other mentioned approaches -- which assume the camera pose either known or estimated elsewhere -- it also tracks camera movements based on geometric-, appearance- and semantic cues. The method also applies a novel RGB-D panoptic segmentation model. Although it is only tested on room-sized environments, the authors claim it could be scaled to larger environments as well.
\section{System Overview} \label{sec:overview}

In this section, we give an overview of our keyword search system. Our approach can be summarized as a two-phase framework, as illustrated in Figure \ref{fig:overview}. 

\begin{figure} [h]
	\centering
	\scalebox{1.0}[1.0]
	{
		\resizebox{\linewidth}{!}
		{
			\includegraphics[scale=0.5]{visio_pics/approach_overview_simp.pdf}
		}
	}
	\vspace{-0.3in}
	\caption{An Overview of Our Approach.}
	\label{fig:overview}
	\vspace{-0.2in}
\end{figure}

\subsection{Phase-I: Segmentation and Annotation} \label{sec:phase1}
The first phase is to segment the keyword token sequence $RQ = \{k_1, k_2, ..., k_m\}$ into several \emph{terms} and each term is annotated with one of the three characters $\{$entity, class, relation$\}$. The converted query is called \emph{annotated query}. Formally, we denote an \textit{annotated query} as $AQ= \{t_1:c_1, t_2:c_2, ..., t_l:c_l\}$, where each $t_i$ is a term and  $c_i \in \{$entity, class, relation$\}$. Note that the first phase (i.e., segmentation and annotation) is not the focus of this paper, as it has been studied extensively \cite{hua2015short,han2011generative, li2011faerie, nakashole2012patty,cai2013large}. We briefly describe the implementation of the first phase as follows. 

For each continuous subsequence $s$ in $RQ$, we check whether it could be matched to an entity, a class, or a relation of the RDF dataset, by employing the existing techniques of entity linking \cite{han2011generative,li2011faerie, ratinov2011local} and relation paraphrasing \cite{nakashole2012patty,cai2013large}. If $s$ is matched, we regard $s$ as a \emph{candidate term} $t_i$, and annotate it with the corresponding character (entity, class, or relation).
We may find that two candidate terms $t_i$ and $t_j$ \emph{overlap} with each other. We say $t_i$ overlaps with $t_j$ if and only if they have at least one common token. Obviously, if two terms overlap, they cannot occur at the same segmentation result. For example, ``university'' and ``university locate USA'' cannot occur in the same segmentation result. We build a \emph{candidate term graph} to describe the mutually exclusive relations: (1) each candidate term $t_i$ is represented a vertex; (2) there is an edge between $t_i$ and $t_j$ if and only of there is \emph{no} overlapping tokens between $t_i$ and $t_j$. Thus each maximal clique in the candidate term graph stands for a possible segmentation result. To obtain top-$N$ best $AQ$, we employ the maximal clique algorithm \cite{bron1973algorithm}, and adopt the pairwise metrics in \cite{hua2015short} to rank the segmentation result.
In out example, we get the top-2 $AQ$ as shown in Figure \ref{fig:overview}.

In the first phase, we have converted keyword token sequence $RQ$ into top-$N$ $AQ$ by some off the shelf techniques. Furthermore, these terms in $AQ$ have been matched to some elementary query graph building blocks (i.e., entity/class vertices and predicate edges). Specifically, if a term $t_i$ is annotated with ``entity'' or ``class'', it will be matched to candidate entity/class vertices in RDF graph.
%We do not distinguish entity vertex or class vertex until generating SPARQL, since we have the same operation on both of them.
If a term $t_i$ is annotated with ``relation'', it will be matched to a set of candidate predicates.

\vspace{-0.05in}
\begin{example}\label{example:aq}
	Given a keyword token sequence $RQ = \{$scientist, graduate, from, university, locate, USA$\}$, we obtain the annotated query $AQ=\{$``scientist'':class, ``graduate from'':relation, ``university'':class, ``locate'':relation, ``USA'':entity $\}$, where ``scientist'' is matched to $\{$dbo:Scientist$\}$, ``university'' is matched to $\{$dbo:University$\}$, ``USA'' is matched to two possible entities $\{$res:USA$\_$Today, res:United$\_$\\States$\}$ due to the ambiguity. Also, the relations ``graduate from'' and ``locate'' also match to two candidate predicates $\{$dbo:almaMater, dbo:education$\}$ and $\{$dbo:country, dbo:location$\}$
\end{example}

\vspace{-0.1in}
\subsection{Phase-II: Query Graph Assembly}
In the second phase, we concentrate on how to assemble a query graph $Q$ based on these elementary building blocks. Formally, the query graph assembly problem is defined as follows:

\vspace{-0.05in}
\begin{definition} \textbf{(Query Graph Assembly Problem)}\label{def:querygraphassembling}
	Given $n$ terms $t^{v}_i$ ($i=1,...,n$) annotated with ``entity'' or ``class'', and $m$ terms $t_j^{e}$ ($j=1,...,m$) annotated with ``relation'', each term $t^{v}_i$ is matched to a set $V_i$ of candidate entity/class vertices and each $t^{e}_j$ is matched to a set $E_j$ of candidate predicate edges. Let $\Upsilon=\{V_1, ..., V_n\}$ and $\Gamma=\{E_1, ..., E_m\}$. A valid assembly query graph is denoted as $Q (V_Q,E_Q)$, which satisfies the following constraints:
	\begin{enumerate}
		\item $|V_Q|=n$, and $\forall V_i \in \Upsilon, V_Q \cap V_i \not= \phi$; 
		\emph{/*each entity or class vertex set $V_i$ has exactly one vertex in  $Q$*/}
		\item $|E_Q|=m$, and $\forall E_j \in \Gamma, E_Q \cap E_j \not= \phi$. 
		\emph{/*each predicate edge set $E_j$ has exactly one edge in $Q$*/}
	\end{enumerate}
	Each edge $e(\langle v_1, v_2\rangle,p) \in E_Q$ connects a pair of vertices $\langle v_1, v_2\rangle \in V_Q$ by a predicate $p$.
	The assembly cost of $Q$ is
	\begin{equation} \label{equ:cost}
	cost(Q)=\sum_{e(\langle v_1, v_2\rangle,p) \in E_Q}{w(\langle v_1, v_2\rangle,p)}
	\end{equation}
	where  $w(\langle v_1, v_2\rangle,p)$ denotes the triple assembly cost. 
	
	
	The \emph{query graph assembly} (QGA for short) problem is to construct a valid graph $Q$ with the minimum assembly cost. 
\end{definition} 

%\vspace{-0.06in}

There are two aspects that should be explained for QGA.

\textbf{\emph{1. Constraints:}}
The two constraints in Definition \ref{def:querygraphassembling} mean that each term $t^{v}_i$ ($1\leq i \leq n$) only corresponds to a single entity/class vertex in $Q$. For example, although ``USA'' may match two candidate entities dbo:USA$\_$Today and dbo:United$\_$States, in the final query graph $Q$, ``USA'' only matches a single entity (dbo:United$\_$States). It is analogue for the relation term $t_j^{e}$ ($1\leq j \leq m$). 

\textbf{\emph{2. Disengaged Edges:}}
A \emph{predicate edge} $e(\langle \cdot,\cdot \rangle,p)$ (in $E_j$) does not have two fixed endpoints but its edge label is fixed to predicate $p$. Thus, a predicate edge can be also called a \emph{disengaged} edge. The triple assembly cost $w(\langle v_1, v_2\rangle,p)$ measures the goodness of assembling $\langle v_1, v_2\rangle$ and $p$ into an edge in $Q$. Then the goal of the QGA problem is to determine the endpoints of $e(\langle \cdot,\cdot \rangle,p)$ to minimize the overall $cost(Q)$.

After finding the optimal $Q$ with minimum $cost(Q)$, we can translate it to SPARQL statements naturally, as illustrated in Figure \ref{fig:overview}.

\subsection{Graph Embedding Cost Model}
Note that the triple assembly cost $w(\langle v_1, v_2\rangle,p)$ can be any positive cost function, which does not affect the hardness of QGA. In other words, the QGA problem is a general computing framework to interpret the input keywords as SPARQL, which does not depend on any specific triple assembly cost function.

\begin{figure} [b]
	\centering
	\scalebox{0.55} [0.50]
	{
		\resizebox{\linewidth}{!}
		{
			\includegraphics[scale=1.0]{visio_pics/transe_visualizing.pdf}
		}
	}
	\caption{Visualizing the Intuition of Graph Embedding.}
	\label{fig:transe_visualizing}
	\vspace{-0.2in}
\end{figure}

The only thing affected by the selection of triple assembly cost function is the system's accuracy. A good cost function can guide to assemble correct query graph $Q$ that implies users' query intention. The process of assembling $\langle v_1, v_2\rangle$ and $p$ into a triple is analogue to ``link prediction'' problem in the RDF knowledge graph \cite{miller2009nonparametric}. Given two entity/class vertices $v_1$ and $v_2$, the link prediction is to ``predict'' the predicate $p$ between $v_1$ and $v_2$, and $w(\langle v_1,v_2\rangle, p)$ is a \emph{measure} of the prediction. Recent research show that the graph embedding technique is superior to other traditional approaches, such as \cite{miller2009nonparametric,nickel2011three,jenatton2012latent}. In the graph embedding model, all subjects (s), objects (o) and predicates (p) are encoded as multi-dimensional vectors $\overrightarrow{s}$, $\overrightarrow{p}$ and $\overrightarrow{o}$ such that $\overrightarrow s  + \overrightarrow p  \approx \overrightarrow o $ if $\langle s,p,o \rangle \in G$ (i.e., $\langle s,p,o\rangle$ is a triple in RDF graph); while $\overrightarrow s  + \overrightarrow p$ should be far away from $\overrightarrow o$ otherwise. Figure \ref{fig:transe_visualizing} visualizes the intuition. From the intuition, the structural information among entities, classes and relations in RDF graph is embedded into vectors. Therefore, we define the triple assembly cost based on graph embedding vectors as follows.


\begin{definition}\textbf{ (Triple Assembly Cost) }. \label{def:tripleassemblycost}
	Given two entity/class vertices $v_1$ and $v_2$ and a predicate edge $p$, the \emph{cost} of assembly triple $(v_1,p,v_2)$ is denoted as follows:
	\begin{equation}\label{equ:tripleassembly}
	w(\langle v_1, v_2\rangle, p) = MIN(| \overrightarrow{v_1} + \overrightarrow p  - \overrightarrow {v_2 } |, | \overrightarrow{v_2} + \overrightarrow p  - \overrightarrow {v_1 } |)
	\end{equation}
	where $\overrightarrow{v_1}$, $\overrightarrow {v_2}$ and $\overrightarrow {p}$ are the encoded multi-dimensional vectors  of $v_1$, $v_2$ and $p$, respectively. 
	%where $|x|_+$ denotes the positive part of $x$. 
\end{definition}  

\begin{figure} [h]
	\centering
	\scalebox{1.0}
	{
		\resizebox{\linewidth}{!}
		{
			\includegraphics[scale=1.0]{visio_pics/query_graph_elements_example2.pdf}
		}
	}
%	\caption{Candidate Entity/Class Vertices and Predicate Edges.}
	\caption{Elementary Query Graph Building Blocks.}
	\label{fig:graph_elements_exp2}
	\vspace{-0.3in}
\end{figure}

\begin{figure} [h]
	\newcommand{\mywidth}{0.23\textwidth}
	\centering
	\begin{subfigure}[t]{\mywidth}
		\centering
		\resizebox{\linewidth}{!}
		{
			\includegraphics{visio_pics/query_assembly_graph_q1.pdf}
			
		}
		\caption{$cost(Q_1)=1.76$}
		\label{fig:assembly_query_graph_q1}
	\end{subfigure}
	\begin{subfigure}[t]{\mywidth}
		\centering
		\resizebox{1.0\linewidth}{!}
		{
			\includegraphics{visio_pics/query_assembly_graph_q2.pdf}
		}
		\caption[font=\small]{$cost(Q_2)=2.46$}
		%        \vspace{0.1in}
		\label{fig:assembly_query_graph_q2}
	\end{subfigure}
	\caption{Possible Assembly Query Graphs.}
	%    \vspace{-0.1in}
	\label{fig:assembly_query_graph}
	\vspace{-0.1in}
\end{figure}

\begin{example} In our example, there are three entity/class terms ``scientist'', ``university'', ``USA'' and two relation terms ``graduate from''  and ``locate''. Their corresponding entity/class vertices and predicate edges are shown in Figure \ref{fig:graph_elements_exp2}. There are two different assembly query graph $Q_1$ and $Q_2$ in Figure \ref{fig:assembly_query_graph}, among which $cost(Q_1)<cost(Q_2)$. Thus, the QGA problem result is $Q_1$ (Figure \ref{fig:assembly_query_graph_q1}).
%It means that we interpret the keywords as a query graph $Q$ and evaluate $Q$ using SPARQL query engine to return answers to users. 
\end{example}

\section{LDR panorama light source detection}
\label{sec:lightdetection}

In order to use LDR panoramas for training our CNN to detect light sources, we must first detect areas in the panoramas which correspond to bright light sources. To do so, we propose a novel light source detector, and show that it significantly outperforms the approach of Karsch et al.~\shortcite{karsch-tog-14}. 
%An overview of the light detection pipeline is shown in Fig.~\ref{f:hogflowchart}. 

\subsection{Light classification} 

After converting to grayscale, the panorama $P$ is rotated by $90^\circ$ about the pitch angle to yield $P_\text{rot}$, so that the zenith is aligned with the horizon line. This rotation is needed to account for the large distortions caused by the equirectangular projection, which severely stretches regions around the poles. Features are then computed over $P$ and $P_\text{rot}$ separately on square patches at five different scales\footnote{We use $30\times30$ squares at the lowest scale, multiplying their size by 1.5 at each scale.}. In particular, we use HOG~\cite{dalal-cvpr-05}, the mean patch elevation, as well as its mean, standard deviation, and 99th percentile intensity values. These features are used to train two logistic regression classifiers for small (e.g. spotlights and lamps) and large (e.g. windows, reflections) light sources. We found that training classifiers for these two types of classes separately yielded better performance, as these types of light sources significantly differ from one another. 

The resulting logistic regression classifiers are then applied in a sliding-window fashion over $P$ and $P_\text{rot}$ to yield a score at each pixel. 
%(fig.~\ref{f:hogflowchart}). 
Scores from both classifiers are added, then merged on a per-pixel manner according to their elevation angles $S_\text{merged} = S\cos(\theta) + S_\text{rot}^*\sin(\theta)$, where $S$ indicates the regression scores, $\theta$ the pixel elevation, and $S_\text{rot}^*$ is $S_\text{rot}$ rotated back to the original orientation. The resulting scores are then thresholded to obtain a binary mask, refined with a dense CRF~\cite{krahenbuhl-nips-12}, and adjusted with opening and closing morphological operations. The optimal threshold is obtained by maximizing the intersection-over-union (IoU) score between the resulting binary mask and the ground truth labels on the training set. 

\begin{figure}
    \centering
    \includegraphics[width=0.64\columnwidth]{images/globalPrCurves.pdf}
    \caption[]{Precision-recall curves for the light detector on the test set for our detectors and the one of Karsch et al.~\shortcite{karsch-tog-14}. In blue, the curve for the spotlights and lamps detection, and in green, the curve for the windows and light reflections. In red, the result for \cite{karsch-tog-14}. In cyan, the curve for a baseline detector relying solely on the intensity of a pixel. Note that because of the inherent uncertainty of the importance of a light (including reflections) relative to the others (even for a human annotator), a perfect match between human and algorithm predictions is highly unlikely.}
    \label{fig:prcurves}
\end{figure}

\begin{figure}[t]
\centering
\setlength{\tabcolsep}{1pt}
\begin{tabular}{cc}
\includegraphics[width=0.48\linewidth]{images/samplesPreclassifier/pano1.jpg} & 
\includegraphics[width=0.48\linewidth]{images/samplesPreclassifier/pano1_mask.jpg} \\
\includegraphics[width=0.48\linewidth]{images/samplesPreclassifier/pano2.jpg} & 
\includegraphics[width=0.48\linewidth]{images/samplesPreclassifier/pano2_mask.jpg} \\
\includegraphics[width=0.48\linewidth]{images/samplesPreclassifier/pano3.jpg} & 
\includegraphics[width=0.48\linewidth]{images/samplesPreclassifier/pano3_mask.jpg} \\
\includegraphics[width=0.48\linewidth]{images/samplesPreclassifier/pano4.jpg} & 
\includegraphics[width=0.48\linewidth]{images/samplesPreclassifier/pano4_mask.jpg} \\
\includegraphics[width=0.48\linewidth]{images/samplesPreclassifier/pano5.jpg} & 
\includegraphics[width=0.48\linewidth]{images/samplesPreclassifier/pano5_mask.jpg} \\
\end{tabular}
\caption{Light detection results on SUN360 panoramas. (left) the input LDR panoramas; (right) light detection results, shown in cyan and overlaid on the original panorama for reference. The detector is able to handle a wide range of lighting arrangements, including large light patches and spotlights.}
\label{f:lightdetection-results}
\end{figure}

\subsection{Training details and evaluation} 

To train the classifiers, we manually annotate a set of 400 panoramas from the SUN360 database. Four types of light sources are labelled: spotlights, lamps, windows, and (bounce) reflections. We use 80\% of the panoramas for training, and 20\% for testing. The classifier is first trained using labeled lights as positive samples and random negative samples. Subsequently, hard negative mining~\cite{felzenszwalb-pami-10} is used over the entire training set. We discard the bottom 15\% of the panoramas because this region often contain watermarks and light sources are seldom located below the camera. 

Fig.~\ref{fig:prcurves} reports a comparison of precision-recall curves for our two detectors, a baseline method which directly maps the intensity of a pixel to its probability of belonging to a light source, and the approach of Karsch et al.~\shortcite{karsch-tog-14}. As expected, the baseline performs poorly on LDR data like SUN360. The detector from \cite{karsch-tog-14} offers better performance, but our performs significantly better at any level of recall. Fig.~\ref{f:lightdetection-results} shows light detection results on example panoramas from the SUN360 dataset. 





%\begin{table}
%\centering
%\begin{tabular}{ccccc}
%\toprule
%\multirow{2}{*}{Method} & \multicolumn{2}{c}{All lights} & %\multicolumn{2}{c}{Spotlights only} \\
% & mean & median & mean & median \\
%\midrule
%\cite{karsch-tog-14} 	& 0.268 & 0.222 & 0.121 & 0.045 \\
%Ours 					& 0.326 & 0.298 & 0.305 & 0.299 \\
%\bottomrule
%\end{tabular}
%\caption[]{Comparison between our light detector and that of Karsch et al.~\shortcite{karsch-tog-14} on a test set of 80 hand-labeled LDR panoramas from the SUN360 database. We report the mean and median intersection-over-union (IoU) score for all light sources, and for spotlights only. Our approach significantly outperforms that of \cite{karsch-tog-14}, especially for detecting bright (but small) spotlights.}
%\label{t:lightdetection-iou}
%\end{table}

% From Marc-André:
% OUR DETECTOR :
% MEAN / MED IoU :  0.326159233623 0.298747417355
% MIN / MAX IoU :  0.0145741504337 0.79900990099
% MEAN / MED spots intersect :  0.305134993465 0.284664612467
% #############################
% KARSCH'S DETECTOR
% MEAN / MED IoU :  0.268475126288 0.22152084079
% MIN / MAX IoU :  0.0 0.854946427713
% MEAN / MED spots intersect :  0.12067937909 0.0453275380317



% \begin{figure}
% \centering
% \includegraphics[width=0.75\linewidth]{images/prCurves_pass2.png}
% \caption{Precision-recall curves for the light detector on the test set. In red, the curve for the spotlights and lamps detection, and in blue, the curve for the windows and light reflections. Note that because of the inherent uncertainty of the importance of a light (including reflections) relative to the others (even for a human annotator), a perfect match between human and algorithm predictions is highly unlikely. 
% \JF{Remove gray background, make axes black, make vectorized version}}
% \label{f:4prcurves}
% \end{figure}


%% NOTES FROM MARC-ANDRE
% General outline :
% \begin{enumerate}
%     \item We need to find a way to find the light sources in an LDR image. We cannot just use a pixel value because of the quick saturation (so a white pixel might not be contributing at all to the scene illumination).
%     \item Since we have the panoramas, most of the time we actually see these lights (except in the cases of occlusions). So we can train a light detector.
%     \item We use a well-known approach with a combination of HOG and logistic regression (by all means, a logistic regression is similar to a Linear SVM). We could have use more complicated approaches, but we show here that even this simpler approach works well and, in any way, we just need the light labeled to \textit{train} our network. Also, we append some information to the HOG descriptors vectors, like the elevation of the center of the patch, the mean color, etc.) Finally, we train two different classifiers, of which we combine the output, one for the 
%     spotlights and lamps, and another for windows and reflections, since their descriptors might be quite different.


%     \item To train the classifier, we manually annotated 400 panoramas. Each source have been labeled as one of these categories: spotlight, lamp, window, and reflection. Of these 400 panoramas, we retain 20\% (80 panoramas) as test set. The classifier is first trained using labeled lights as positive samples and random negative samples. Subsequently, we do hard negative mining over all the training set and train again the classifier using these hard negative samples. When evaluating the classifier, we drop the bottom 15\% of the panoramas because of the watermarks and also since the probability of a light being almost under the camera is quite low.


%     \item The first issue is that while HOG descriptors deal reasonably well with rotation and minor scaling, they do not work well with general deformation. Unluckily, a latlong projection distorts a lot the panorama, especially near the zenith, where it is not unreasonable to find a light... To alleviate this issue, we actually extract HOG features from two environment maps. The first one is the one passed as input (gray level), and the second one is rotated such as the zenith becomes part of the horizon. The ceiling lights are then less distorted and the same classifier may be used to find them.

%     \item As usual with HOGs, we use a sliding window at different scales. However, the lights are not perforce rectangular yet we would like to be able to roughly segment them. Each time a detection is fired, we add a constant value to its corresponding zone. This value is decreasing along with the scale. In the end, we obtain an heatmap of the most probable spots for light sources.

%     \item We combine the heatmap produced with the rotated and non-rotated panoramas, weighting them with a simple cosine rule, based on the fact that at the horizon, each detector has the maximum precision; on the contrary, at the zenith, the detector accuracy is quite poor.
%     \item Once we have merged the heatmap, we threshold it so to get a binary mask.

%     \item We apply a CRF on it to add some locality to the detections. Finally, we use standard morphological operators to remove outliers and inliers.
% \end{enumerate}
\section{Panorama recentering warp}
\label{sec:warping}

\begin{figure}[!t]
\centering
\includegraphics[width=0.99\linewidth]{images/warping/warping.pdf}
\caption{The importance of light locality for indoor scenes. Left, a photo for which we want to estimate the lighting conditions. The photo was cropped from the ``original'' panorama (top row, middle). Treating this panorama as the light source for the photo is wrong; its center of projection is in front of the scene in the photo, and relighting a virtual bunny (top row, right) makes it appear to be backlit. The correct HDR panorama, captured with a light probe at the position of the cropped photo, is shown in the middle row, and captures the location of the lights on top of the scene. We introduce a warping operator that can be estimated with no scene information, and distorts the original panorama to approximate the location of the light sources on the top (bottom row). Relighting an object with the warped panorama yields results that are much closer to the ground truth.}
\label{f:warping-problem}
\end{figure}

Detecting the light sources in LDR panoramas is not sufficient for training the CNN to learn lighting from a single photo. The fundamental problem is that the panorama does not represent the lighting conditions in the \emph{cropped scene}, since the panorama center of projection can be arbitrarily far away from the location of the scene points in the cropped photo. Fig.~\ref{f:warping-problem} illustrates this issue. The photo shown on the left was cropped from the ``original'' panorama in the middle column. Treating this original panorama as a light source is incorrect, and results in a backlit bunny. We captured the \emph{actual} lighting conditions by placing a light probe at the scene (middle column of fig.~\ref{f:warping-problem}). Notice how the lighting conditions at the scene differ from those in the original panorama. To allow the use of the SUN360 database (from which we can crop photos but do not have access to the scenes to capture ground truth lighting) for training, we present a novel method that warps the original panorama to approximate the lighting in the cropped photo (bottom row). 
%While our warping function is an approximation, the lighting conditions and relit object obtained with the warped panorama are much closer to the ground truth than the original version, and does not require physical access to the scene (bottom row of fig.~\ref{f:warping-problem}). We now detail this warping function and provide several qualitative examples that demonstrate the need for such an approach. 

\subsection{Warping operator}

The aim of the warping operator is to generate the panorama that would be captured by a virtual camera placed at a point in the cropped photo. This is a challenging problem that is made especially harder by the fact that we do not know the scene geometry, and we make two assumptions to make this task feasible. First, we assume that the scene lies on a sphere, i.e., all scene points are equidistant from the original center of projection. Second, we assume that an image warping suffices to model the effect of moving the camera, i.e., occlusions are not an important factor. These assumptions may not hold for all scene points; however, note that our goal is to model light sources, which are typically located at scene extremities (ceiling, walls, etc.) and are better approximated by these assumptions.

Let us assume that the panorama is placed on the unit sphere, i.e. $x^2 + y^2 + z^2 = 1$, with the camera that captured this panorama at the origin of this sphere. The outgoing rays emanating from a virtual camera placed at $(x_0,y_0,z_0)$, can be parameterized as:
%
\begin{equation}
x(t) = v_x t + x_0  \quad
y(t) = v_y t + y_0  \quad
z(t) = v_z t + z_0  \,.
\label{eq:warp2}
\end{equation}
%
Intersecting these rays with the panorama sphere yields:
%
\begin{equation}
    (v_x t + x_0)^2 + (v_y t + y_0)^2 + (v_z t + z_0)^2 = 1 \,.
    \label{eq:warp3}
\end{equation}
% 

As illustrated in fig.~\ref{f:warp-basics}, we want to model the effect of using a virtual camera whose nadir is at $\beta$. The angle $\beta$ corresponds to the point in the panorama where the photo is extracted, and we will discuss how this is computed shortly. For the case of translating along the $z$-axis, this results in a new camera center, $\{x_0, y_0, z_0\}$ = $\{0, 0, \sin \beta\}$. Warping in arbitrary directions can trivially be achieved by rotating the environment map before and after the warp. Substituting this in eq.~\ref{eq:warp3}, results in the following second degree equation:
%
\begin{equation}
(v_x^2 + v_y^2 + v_z^2)t^2 + 2v_z t \sin\beta  + \sin^2\beta - 1 = 0 \,.
\label{eq:warp4}
\end{equation}
%
Solving (\ref{eq:warp4}) for $t$ (keeping only positive solutions, as negative roots represent the intersection on the other side of the sphere), maps the coordinates from the original environment map to the ones in the warped camera coordinate system. 
%An illustrated example of the effect of the warp operator on an equirectangular panorama is shown in fig.~\ref{f:warp-explanations}. 

\begin{figure}[!t]
    \centering
    \includegraphics[width=0.375\linewidth]{images/diag_explanations_warp.eps}
    \caption{Overview of the warping problem, illustrated in 2D for simplicity. The circle represents a slice of the spherical panorama along the $y$--$z$ plane, with the center of projection (illustrated by a camera) at its center. The aim of the warp operator is create a virtual center of projection with a nadir at an angular distance of $\beta$ with respect to the original nadir. The angle $\beta$ corresponds to the point in the panorama where the photo is extracted.}
    \label{f:warp-basics}
\end{figure}

\begin{figure}[!t]
\centering
\footnotesize
\setlength{\tabcolsep}{1pt}
\begin{tabular}{cc}
\includegraphics[width=0.493\linewidth]{images/warping/findbeta_input_withx.png} &
\includegraphics[width=0.493\linewidth]{images/warping/findbeta_normals.png} \\
(a) Input image & (b) Normals \\
\includegraphics[width=0.493\linewidth]{images/warping/findbeta_output_withx.png} &
\includegraphics[width=0.493\linewidth]{images/warping/findbeta_panowarp.png} \\
(c) Original panorama & (d) Warped panorama \\
\end{tabular}
\caption{$\beta$ selection procedure. From a given crop picture (a), we extract the normals using the method of  Bansal et al.~\shortcite{bansal2016marr} (b). We pick the insertion point by looking at the lowest pixel with a horizontal surface (green X in (a)) and backproject it on to the panorama (c). This gives us the point where we would like the nadir to be, from which $\beta$ can be trivially recovered. We then warp the panorama using this $\beta$ (d).}
\label{f:warp-beta-pick}
\end{figure}


The value of $\beta$ in eq.~(\ref{eq:warp4}) represents the point in the panorama where the photo is extracted. We expect that users will want to insert objects on to flat horizontal surfaces in the photo, and we reflect this in the choice of $\beta$ as follows (see fig.~\ref{f:warp-beta-pick}): we first use the approach of Bansal et al.~\shortcite{bansal2016marr} to detect surface normals in the cropped image, and find flat surfaces by thresholding based on the angular distance between surface normal and the up vector. We back-project the $y$-coordinate of the lowest point of the largest flat area (i.e., the lowest point on the flattest horizontal surface) on to the panorama to obtain $\beta$. In cases where no horizontal surfaces are found (e.g., a flat vertical wall), no warp is applied as the panorama is assumed to be sufficiently close to scene. Note that we always assume the insertion point to be x-centered ---that is, we do not ask the network to estimate the light at far-left or far-right of the image.

% \begin{figure}
% \centering
% \footnotesize
% \setlength{\tabcolsep}{1pt}
% \begin{tabular}{cc}
% \includegraphics[height=2.2cm]{images/tilingpattern_0.png} &
% \includegraphics[height=2.2cm]{images/tilingpattern_sphere_0.png} \\
% \includegraphics[height=2.2cm]{images/tilingpattern_30.png} &
% \includegraphics[height=2.2cm]{images/tilingpattern_sphere_30.png} \\
% \includegraphics[height=2.2cm]{images/tilingpattern_60.png} &
% \includegraphics[height=2.2cm]{images/tilingpattern_sphere_60.png} \\
% (a) Equirectangular panorama & (b) Projected on a sphere
% \end{tabular}
% \caption{Effect of the warp operator on panoramas. Top row: the original environment map. Middle row: warp with $\beta=30^\circ$. Bottom row: warp with $\beta=60^\circ$.}
% \label{f:warp-explanations}
% \end{figure}

\begin{figure}[!t]
\centering
\footnotesize
\setlength{\tabcolsep}{1pt}
\begin{tabular}{ccc}
\includegraphics[width=0.325\linewidth]{images/warping/images/good/noWarp/pano0767-others-135.jpg} &
\includegraphics[width=0.325\linewidth]{images/warping/images/good/withWarp/pano0767-others-135.jpg} & 
\includegraphics[width=0.325\linewidth]{images/warping/images/good/envyDepth/composeHenrique767.jpg} \\
\includegraphics[width=0.325\linewidth]{images/warping/images/good/noWarp/pano0460-others-270.jpg} &
\includegraphics[width=0.325\linewidth]{images/warping/images/good/withWarp/pano0460-others-270.jpg} &
\includegraphics[width=0.325\linewidth]{images/warping/images/good/envyDepth/composeHenrique460.jpg} \\
\includegraphics[width=0.325\linewidth]{images/warping/images/good/noWarp/pano0618-others-00.jpg} &
\includegraphics[width=0.325\linewidth]{images/warping/images/good/withWarp/pano0618-others-00.jpg} & 
\includegraphics[width=0.325\linewidth]{images/warping/images/good/envyDepth/composeHenrique618.jpg} \\
\includegraphics[width=0.325\linewidth]{images/warping/images/good/noWarp/pano0634-others-135.jpg} &
\includegraphics[width=0.325\linewidth]{images/warping/images/good/withWarp/pano0634-others-135.jpg} &
\includegraphics[width=0.325\linewidth]{images/warping/images/good/envyDepth/composeHenrique634.jpg} \\
(a) Original panorama & (b) Our warp & (c) \cite{banterle-cgf-13}
\end{tabular}
\caption{Comparison of objects relit with (a) the original panoramas, (b) our warped panoramas, and (c) panoramas warped using EnvyDepth~\cite{banterle-cgf-13}. The objects relit by our panoramas closely approximate those obtained with EnvyDepth, without the lengthy manual annotation required.}
\label{f:warp-results}
\end{figure}

\subsection{Impact on lighting estimation}

Fig.~\ref{f:warping-problem} compares our warped panorama with a ground truth panorama captured in-place for one scene. We also compare our spherical warp with a geometry-based warp obtained with EnvyDepth~\cite{banterle-cgf-13}, a system that extracts spatially-varying lighting from environment maps by projecting them onto proxy geometry estimated from manual annotations. Comparative relighting results using the original, spherical warp, and geometry-based warp panoramas are presented in fig.~\ref{f:warp-results}. While our operator makes several simplifying scene assumptions, these results illustrate that relighting with our approach provides a close approximation to more expensive techniques, while being completely automatic and without requiring access to the scene. In contrast, the manual labeling process required for the geometric warp takes around 10 minutes per panorama. 

The main limitation of our warping operator is that it fails to appropriately model occlusions. Since we treat the panorama as a projection on a sphere, lights that illuminate a scene point, but are not visible from the original camera are not handled by this approach. However, these situations are rare, and as we show in our results, our network filters them out as outliers, and learns a robust scene appearance to illumination mapping.


%objects protruding from the main surfaces (e.g. a table, columns, etc.) appear unrealistically distorted in the warped versions. However, since our goal is to obtain a more accurate representation for lighting, the distortions have minimal impact on the learning.

%\emiliano{note: This seems to be also relevant as trick for rendering scenes lit by a single IBL. You can assume the IBL refers to the origin of the scene and give it a radius. Then use this trick to adjust the shading away from the center. Also the sampling function is trivial to implement in a shader. This seems to be a relevant and basic CG problem, I am surprised nobody else tackled it. }


% Of course, this operator does not solve everything. In particular:
% \begin{itemize}
%     \item It greatly reduces the resolution in the front zone (and increases it behind the camera). SUN360 having 9000x4500 panoramas, this is not a huge issue in our case.
%     \item This assumes horizontal surfaces everywhere, which is clearly not the case for indoor scenes. We thus use the normal information (using a CNN to estimate them) to ensure that pathological cases do not happen.
%     \item This does not solve the issue in case of severe occlusions, but nothing can (the information we want is simply not there). All in all, as our results in fig.~\ref{f:warp-results} clearly show, we still get a consistently better lighting estimation than by using the IBL at the camera position.
% \end{itemize}

\section{Learning object shape models} \label{sec:learning-shapes}


\begin{figure}[!t]
	\centering
	\includegraphics[width=\textwidth]{paper_figures/real_data_shape_acquisition.pdf}
	\caption{%
		%Learning shape models. \textcolor{red}{update caption}
		Learning a voxel-based shape models $p(\mathbf{O}^{(m)})$ for a novel object from a set of 5 depth images.
		Our shape priors capture uncertainty about voxel occupancy due to self-occlusion (right).
	}
	\label{fig:shapes}
\end{figure}

3DP3 does not require hard-coded shape models. Instead, it uses
probabilistic inference to learn non-parametric models of 3D object
shape $p(\mathbf{O}^{(m)})$ that account for uncertainty due to
self-occlusion. We focus on the restricted setting of learning from
scenes containing a single isolated object ($N=1)$ of known type
($c_1$). Our approach works best for views that lead to minimal
uncertainty about the exterior shape of the object; more general,
flexible treatments of shape learning and shape uncertainty are beyond
the scope of this paper.

First, we group the depth images by the object type ($c_1$), so that
we have $M$ independent learning problems.  Let $\mathbf{I}_{1:T} :=
(\mathbf{I}_1, \ldots, \mathbf{I}_T)$ denote the depth observations
for one object type, with object shape denoted $\mathbf{O}$.  The
learning algorithm uses Bayesian inference in another generative model
$p'$.  The posterior $p'(\mathbf{O} | \mathbf{I}_{1:T})$ produced by
this algorithm becomes the prior $p(\mathbf{O})$ used in
Section~\ref{sec:objects}.

We start with a uninformed prior distribution 
$p'(\mathbf{O}) := \prod_{i=1}^h \prod_{j=1}^w \prod_{\ell=1}^l p_{\mathrm{occ}}^{O_{ij\ell}} (1 - p_{\mathrm{occ}})^{(1 - O_{ij\ell})}$
on the 3D shape of an object type,
for a per-voxel occupancy probability $p_{\mathrm{occ}}$ (in our experiments, 0.5).
We learn about the object's shape by observing a sequence of depth images $\mathbf{I}_{1:T}$ that contain views of the object, which is assumed to be static relative to other contents of the scene, which we call the `map' $\mathbf{M}$.
(In our experiments the map contains the novel object, a floor, a ceiling, and four walls of a rectangular room).
We posit the following joint distribution over object shape ($\mathbf{O}$) and the observed depth images,
conditioned on the map ($\mathbf{M}$) and the poses of the camera relative to the map over time ($\mathbf{x}_1, \ldots, \mathbf{x}_T \in SE(3)$): $p'(\mathbf{O}, \mathbf{I}_{1:T} | \mathbf{M}, \mathbf{x}_{1:T})
:=
p'(\mathbf{O})
\prod_{t=1}^T p'(\mathbf{I}_t | \mathbf{O}, \mathbf{M}, \mathbf{x}_t)$.


The likelihood $p'$ is a 
depth image likelihood on a latent 3D voxel occupancy grid (see supplement for details).
For this model, we can compute $p'(\mathbf{O} | \mathbf{M}, \mathbf{x}_{1:T}, \mathbf{I}_{1:T})
= \prod_{ij\ell} p'(O_{ij\ell} | \mathbf{M}, \mathbf{x}_{1:T}, \mathbf{I}_{1:T})
$ exactly using ray marching to decide if a voxel cell is occupied, unoccupied, or unobserved (due to being occluded by another occupied cell), and the resulting distribution on $\mathbf{O}$
can be compactly represented as an array of probabilities ($\in [0, 1]^{h \times w \times l}$).
However, in real-world scenarios the map $\mathbf{M}$ and the camera poses $\mathbf{x}_{1:T}$ are not known with certainty.
To handle this, our algorithm takes as input uncertain beliefs
about $\mathbf{M}$ and $\mathbf{x}_{1:T}$
($q_{\mathrm{SLAM}}(\mathbf{M}, \mathbf{x}_{1:T}) \approx p'(\mathbf{M}, \mathbf{x}_{1:T} | \mathbf{I}_{1:T})$)
that are produced by
a separate probabilistic SLAM (simultaneous localization and mapping) module,
and take the form of a weighted collection of $K$ particles $(\mathbf{M}^{(k)}, \mathbf{x}_{1:T}^{(k)})$: $q_{\mathrm{SLAM}}(\mathbf{M}, \mathbf{x}_{1:T}) = \sum_{k=1}^K w_k \delta_{\mathbf{M}^{(k)}}(\mathbf{M}) \delta_{\mathbf{x}_{1:T}^{(k)}}(\mathbf{x}_{1:T})$.
Various approaches to probabilistic SLAM can be used; we implemented it
using sequential Monte Carlo (SMC) in Gen (more detail in supplement).
From the beliefs $q_{\mathrm{SLAM}}(\mathbf{M}, \mathbf{x}_{1:T})$ produced by SLAM, we approximate the object shape posterior via:
\[
\hat{p}'(\mathbf{O} | \mathbf{I}_{1:T})
:= \iint p'(\mathbf{O} | \mathbf{M}, \mathbf{x}_{1:T}, \mathbf{I}_{1:T}) q_{\mathrm{SLAM}}(\mathbf{M}, \mathbf{x}_{1:T}) d\mathbf{M} d\mathbf{x}_{1:T}
= \sum_{k=1}^K w_k p'(\mathbf{O} | \mathbf{M}^{(k)}, \mathbf{x}_{1:T}^{(k)}, \mathbf{I}_{1:T})
\]
Note that while $p'(\mathbf{O} | \mathbf{M}^{(k)}, \mathbf{x}_{1:T}^{(k)}, \mathbf{I}_{1:T})$ for each $k$
can be compactly represented,
the mixture distribution $\hat{p}'(\mathbf{O} | \mathbf{I}_{1:T})$ lacks the conditional independencies that make this possible.
%However, this object shape posterior does not have an analytic form and thus would require us to store the full particle approximation (from SLAM) and all depth images for each object we ever observe.
To produce a more compact representation of beliefs about the object's shape,
we fit a variational approximation $q_{\varphi}(\mathbf{O})$
that assumes independence among voxels 
($q_{\varphi}(\mathbf{O}) := \prod_{i\in[h]} \prod_{j\in[w]} \prod_{\ell\in[l]} \varphi_{ij\ell}^{O_{ij\ell}} \cdot (1 - \varphi_{ij\ell})^{(1-O_{ij\ell})}$) to $\hat{p}'(\mathbf{O} | \mathbf{I}_{1:T})$
using $\varphi^* := \argmin_{\varphi} \mathrm{KL}(\hat{p}'(\mathbf{O} | \mathbf{I}_{1:T}) || q_{\varphi}(\mathbf{O}))$
(see supplement for details).
This choice of variational family is sufficient for representing uncertainty about the occupancy of voxels in the \emph{interior} of an object shape.
Note that our shape-learning experiments did not result in significant uncertainty about the \emph{exterior} shape of objects%
\footnote{%
The lack of significant exterior shape uncertainty in shape-learning experiments allowed us to
implement an optimization:
Instead of the relative poses of an object's contact planes depending on $\mathbf{O}$ as described in Section~\ref{sec:model},
we assign each object type a set of six contact planes derived from the faces of the smallest axis-aligned bounding cuboid that completely contains all occupied voxels in one sample $\mathbf{O}$ from the learned prior $p(\mathbf{O}) := q_{\bm{\varphi}}(\mathbf{O})$.
},
and in the presence of such uncertainty, a less severe variational approximation may be needed for robust inference of scene graphs from depth images.
Fig. \ref{fig:shapes} shows input depth images ($\mathbf{I}_{1:T}$) and resulting shape prior learned from $T=5$ observations.
After learning these shape distributions $q_{\bm{\varphi}}(\mathbf{O}) \approx \hat{p}'(\mathbf{O} | \mathbf{I}_{1:T})$ for each distinct object type, we use them as the shape priors $p(\mathbf{O}_i)$ within the generative model of Section~\ref{sec:model}. The supplement includes the results of a quantitative evaluation of the accuracy of shape learning.


\section{Learning high dynamic range illumination}
\label{sec:hdr}

Up to this point we have trained a network that can predict the \emph{position} of the light sources quite accurately (see sec.~\ref{sec:experiments}), but, since it was trained on LDR data, it does not know about the \emph{intensities} of the light sources. In this section, we further train the network on a novel dataset of high dynamic range panoramas which enables it to jointly reason about light source direction and intensity. 

\subsection{A new dataset of HDR indoor panoramas}

We have captured a novel dataset of 2,100 high-resolution ($7768\times3884$), high dynamic range indoor panoramas. To do so, a Canon 5D Mark III camera with a Sigma 8mm fisheye lens was mounted on a tripod equipped with a robotic panoramic tripod head, and programmed to shoot 7 bracketed exposures at $60^\circ$ increments. The photos were shot in RAW mode, and automatically stitched into a 22 f-stop HDR $360^\circ$ panorama using the PTGui Pro commercial software. The dynamic range is sufficient to correctly expose all pixels in the scenes, including the light sources. Panoramas were captured in a wide variety of indoor environments, such as schools, houses, apartments, museums, laboratories, factories, sports facilities, etc. A visual overview of panoramas in our novel HDR dataset is shown in the supplementary material. The size and variety of this dataset is significantly larger than other similar datasets in the literature (which consist of tens of panoramas), making it extremely useful for training and testing a wide range of problems from scene inference, high dynamic range image processing, and rendering\footnote{This dataset is publicly available at \url{http://www.jflalonde.ca/projects/deepIndoorLight}.}. 

\subsection{Adapting the network to HDR data}

Since the light sources are not saturated in the HDR data, the network can be adjusted to directly learn the light source \emph{intensities} $\mathbf{y}_\text{int}$ instead of the binary light mask $\mathbf{y}_\text{mask}$. To do so, the network undergoes the following four simple changes. First, the weights of the last layer of the light mask predictor (``conv5-1'' in table~\ref{t:learning-architecture}) are initialized to random values. Second, training is performed to update only the weights of the decoders---that is, up to the FC-1024 layer in table~\ref{t:learning-architecture}. This is done to avoid overfitting on the encoder. Third, the target intensity $\mathbf{t}_\text{int}$ is defined as the log of the HDR intensity ($\log_{10}$ is used). Low intensities (below the median of the training dataset) are clamped to 0, since we only care about the light sources: in the unusual case where no pixels would be over this threshold, the ambient term given by the RGB recovery should be enough to light the scene. Finally, the loss is modified to:
%
\begin{align}
    \mathcal{L}_\text{HDR}(\mathbf{y}, \mathbf{t}, e) &= w_1 \mathcal{L}_\text{L2}(\mathbf{y}_\text{RGB}, \mathbf{t}_\text{RGB}) \nonumber \\ 
    &+ w_2 \mathcal{L}_\text{cos}(\mathbf{y}_\text{int}, \mathbf{t}_\text{int}, e)
    + w_3 \mathcal{L}_\text{L2}(\mathbf{y}_\text{int}, \mathbf{t}_\text{int}, e)  \,,
\label{e:hdrloss}
\end{align}
%
where $\mathcal{L}_\text{L2}$ and $\mathcal{L}_\text{cos}$ were defined in eq.~(\ref{e:rgbloss}) and (\ref{e:maskloss}) respectively, and $e$ is continued from training on the LDR data (so the HDR intensities are not overblurred). The L2 term on the intensity was added to reduce deconvolution artifacts. Here, $w_1 = 10$, $w_2 = 1$, and $w_3 = 0.1$. Training is otherwise performed with the same parameters as in sec.~\ref{sec:training-details}, and, just as with the LDR data, 85\% of the HDR data was used for training and 15\% for testing. Similar to the LDR data (sec.~\ref{sec:ldr-data-prep}), 8 crops were extracted from each panorama in the HDR dataset, yielding 14,000 input-output pairs. These are tone-mapped to ensure that the input to the network are LDR images. Finally, the panoramas are also warped using the same procedure as their LDR counterparts. Fig.~\ref{f:learning-curves}-(b) shows the loss (from eq.~(\ref{e:hdrloss})) curves on the training and test set during training. 



\begin{table}[t!]
\centering
\caption{Voice conversion \& F0 manipulation results. MOS results are reported with 95\% confidence interval. VDE, and FFE are reported for F0 manipulation while PER, WER, EER, and MOS are reported for voice conversion. Notice, for VDE, and FFE higher is the better since F0 was flattened.}
\label{tab:conv}

\resizebox{1\columnwidth}{!}{
\begin{tabular}{c@{~} | c@{~} | c@{~}c@{~} | c@{~} | c@{~} ||  c@{~}c@{~} }
\toprule
\multirow{2}{*}{Dataset} & \multirow{2}{*}{Method} & \multicolumn{4}{c||}{Voice Conversion} & \multicolumn{2}{c}{F0 Manipulation} \\
\cmidrule{3-8}
& & PER~$\downarrow$ & WER~$\downarrow$ & EER~$\downarrow$ & MOS~$\uparrow$ & VDE~$\uparrow$ & FFE~$\uparrow$ \\
\midrule
VCTK & GT  & 17.16 & 4.32 & 3.25 & 4.11$\pm$0.29 & -- & -- \\
\midrule 
\multirow{3}{*}{LJ}
% & ASR-TTS   & 50.74  & --     & 66.08 & 32.96 & 1.46 \\
& CPC       & 22.22 	& 16.11 		& 0.46 		& 3.57$\pm$0.15 		& \bf 46.68 & \bf 48.71\\
& HuBERT    & \bf 19.09 & \bf 12.23 & \bf 0.31  & \bf 3.71$\pm$0.24 & 39.20 		& 48.42\\
& VQ-VAE    & 40.88 	& 36.96 		& 9.65 		& 2.90$\pm$0.17 		& 10.54 	& 12.08 \\
\midrule 
\multirow{3}{*}{VCTK} 
% & ASR-TTS   & 68.88  & --    & 41.77 & 13.55 & 6.48 \\
& CPC       &  23.58 		& 15.98 		& \bf 4.83  &  3.42 $\pm$ 0.24 		& \bf 25.29 & \bf 26.97 \\
& HuBERT    &  \bf 20.85 	& \bf 12.72 & 6.01  		& \bf  3.58 $\pm$ 0.28 	& 23.46 	& 26.67 \\
& VQ-VAE    & 36.88  		& 29.44 		& 11.56 		& 3.08 $\pm$ 0.34 		& 7.03  	& 7.80  \\
\bottomrule
\end{tabular}}
\vspace{-0.4cm}
\end{table}

\vspace{-0.1cm}
\section{Results}
\vspace{-0.1cm}
Our results cover
% We report results for 
three different settings: (i) speech reconstruction experiments; (ii) speaker conversion and F0 manipulation; (iii) bitrate analysis with subjective tests for speech codec evaluation. We employ two datasets: LJ~\cite{ljspeech17} single speaker dataset and VCTK~\cite{vctk} multi-speaker dataset. All datasets were resampled to a 16kHz sample rate.

% \paragraph*{Implementation Details.}
% \smallskip
\noindent{\bf Implementation Details\quad} 
\label{sec:impl}
We follow the same setup as in~\cite{lakhotia2021generative}. For CPC, we used the model from~\cite{Riviere2020}, which was trained on a ``clean'' 6k hour sub-sample of the LibriLight dataset~\cite{Kahn2020,Riviere2020}. We extract a downsampled representation from an intermediate layer with a 256-dimensional embedding and a hop size of 160 audio samples. For HuBERT we used a \textsc{Base} 12 transformer-layer model trained for two iterations~\cite{hsu2020hubert} on 960 hours of LibriSpeech corpus~\cite{Panayotov2015}. 
% This model encodes every 320 raw audio samples into a 768-dimensional vector. 
This model downsamples the raw audio $\times320$ into a sequence of 768-dimensional vectors. Similarly to~\cite{lakhotia2021generative}, activations were extracted from the sixth layer.

%CPC: We use a dictionary of 100 units, leading to a bitrate of 700bps.
%HuBERT: A dictionary of 100 units is used, leading to a bitrate of 350bps. 
%VQVE: The VQ-VAE discrete code operates at a bitrate of 800bps.
% For both CPC and HuBERT, the k-means algorithm is applied to convert continuous frames to discrete codes, using the LibriSpeech clean-100h~\cite{Panayotov2015} dataset. 
For CPC and HuBERT, the k-means algorithm is trained on LibriSpeech clean-100h~\cite{Panayotov2015} dataset to convert continuous frames to discrete codes. We quantize both learned representations with $K=100$ centroids. Leading to a bitrate of 700bps for CPC and 350bps for HuBERT.

% VQ-VAE
Similarly to CPC models, we trained the VQ-VAE content encoder model on the ``clean'' 6K hours subset from the LibriLight dataset. We use an encoder operating on the raw signal to extract discrete units, similar to~\cite{jukebox}. In addition, ``random restarts'' were performed when the mean usage of a codebook vector fell below a predetermined threshold. Finally, we used HiFiGAN (architecture and objective) as the decoder instead of a simple convolutional decoder, as it improved the overall audio quality. This model encodes the raw audio into a sequence of discrete tokens from 256 possible tokens~\cite{garbacea2019low} with a hop size of 160 raw audio samples. The VQ-VAE discrete code operates at a bitrate of 800bps. We additionally experimented with 100 discrete units for VQ-VAE, however results were the best for 256. This finding is consistent with~\cite{garbacea2019low}.

% verification model
The speaker verification network uses the architecture proposed in~\cite{heigold2016end}. It was trained on the VoxCeleb2~\cite{voxceleb2} dataset, achieving a 7.4\% Equal Error Rate (EER) for speaker verification on the test split of the VoxCeleb1~\cite{Nagrani17} dataset.

% pitch
Only a single F0 representation is considered across all evaluated models, trained on the VCTK dataset.
% The F0 is extracted from the raw audio using YAAPT~\cite{yaapt} algorithm, using a window size of 20ms and a 5ms hop. 
The F0 is extracted from the raw audio using a window size of 20ms and a 5ms hop. 
As a result, the F0 sequence is sampled at 200Hz. 
% We apply the quantization described at Sec.~\ref{sec:method}, using a pitch codebook of $K'=20$ tokens and an encoder that downsamples the pitch by $\times16$. 
The quantization described at Sec.~\ref{sec:method}, is applied using an F0 codebook of $K'=20$ tokens and an encoder that downsamples the signal by $\times16$. Hence, the discrete F0 representation is sampled at 12.5Hz, leading to a bitrate of 65bps. The final bitrate of the evaluated codecs is the sum of the pitch code bitrate with the content code bitrate.

% \paragraph*{Evaluation Metrics}
% \smallskip
\noindent{\bf Evaluation Metrics\quad} 
We consider both subjective and objective evaluation metrics. For subjective tests, we report the Mean Opinion Scores (MOS). In which human evaluators rate the naturalness of audio samples on a scale of 1--5. Each experiment, included 50 randomly selected samples rated by 30 raters. For objective evaluation, we consider: (i) Equal Error Rate~(EER) as an automatic speaker verification metric obtained using a pre-trained speaker verification network. We report EER between test utterances and enrolled speakers; (ii) Voicing Decision Error (VDE)~\cite{nakatani2008method}, which measures the portion of frames with voicing decision error; (iii) F0 Frame Error (FFE)~\cite{chu2009reducing}, measures the percentage of frames that contain a deviation of more than 20\% in pitch value or have a voicing decision error; (iv) Word Error Rate (WER) and Phoneme Error Rate (PER), proxy metrics to the intelligibility of the generated audio. We used a pre-trained ASR network~\cite{baevski2020wav2vec} on both reconstructed and converted samples to calculate both metrics. %To generate target phonemes, the g2p-en~\cite{g2pE2019} Grapheme2Phoneme module was used.

% \vspace{-0.1cm}
% \smallskip
\noindent{\bf Reconstruction \& Conversion}
% \vspace{-0.1cm}
We start by reporting the reconstruction performance. Results are summarized in Table~\ref{tab:recon}. When considering the intelligibility of the reconstructed signal HuBERT reaches the lowest PER and WER scores across all models, where both CPC and HuBERT are superior to VQ-VAE. However, when considering F0 reconstruction VQ-VAE outperforms both HuBERT and CPC by a significant margin. This results are somewhat intuitive, bearing in mind VQ-VAE objective is to fully reconstruct the input signal. In terms of subjective evaluation, all models reach similar MOS scores, with one exception of CPC on LJ. 

%Notice, since the same F0 units are used for each method, this result implies the VQ-VAE units contain some information about the F0 of the signal, enabling better reconstruction. Regarding speaker information, the CPC gets the lowest EER. 

To better evaluate the disentanglement properties of each method with respect to speaker identity and F0, we conducted an additional set of experiments aiming at speaker conversion and F0 manipulation. For voice conversion, we converted each test utterance into five random target speakers. Next, we employed a speaker verification network, which extracts \emph{d-vector} representation to evaluate speaker-converted utterances' similarity to real speaker utterances (low error-rate indicates good conversion), providing measurement to the speaker identity's disentanglement from the evaluated coding method. The error-rate is reported between converted test utterances and enrolled speakers. For the LJ speech single speaker dataset, we converted samples from the VCTK dataset to the single speaker and enrolled all VCTK speakers together with the single speaker. Results are summarized in Table~\ref{tab:conv} (left). Unlike resynthesis results, on voice conversion CPC and HuBERT outperform VQ-VAE on both LJ and VCTK datasets, indicating VQ-VAE contains more information about the speaker in the encoded units, hence producing more artifacts. Notice, this also affects WER, PER, and the overall subjective quality (MOS). 

Next, to evaluate the presence of F0 in the discrete units, we flattened the F0 units before synthesizing the signal and calculated VDE and FFE with respect to the original F0 values. F0 flattening was done by setting the speakers' mean F0 value across all voiced frames. In this experiment, we expected units that contain F0 information to be better at F0 reconstruction over disentangled units. Results are summarized in Table~\ref{tab:conv} (right). Notice VQ-VAE can still reconstruct the F0 almost at the same level as when using the original F0 as conditioning (5.2 vs 7.03, and 5.59 vs 7.8), in contrast to CPC and HuBERT.

\begin{figure}[t!]
\centering
\includegraphics[width=0.65\columnwidth, trim={50 20 70 20}]{figures/codec_2.pdf}
% \caption{MUSHRA subjective listening test results as a function of bitrate per second for various methods. Purple dots denote the baseline methods, and green dots the proposed SSL based method.} 
\caption{MUSHRA subjective quality results as a function of bitrate per second. Purple dots denote the baseline methods, and green dots the proposed SSL based method.} 
\label{fig:codec}
\vspace{-0.5cm}
\end{figure}

% \vspace{-0.1cm}
% \smallskip
\noindent{\bf Speech Codec}
Our final experiment evaluates the obtained speech units as a low bitrate speech codec. 
% Therefore, we evaluate how the performance varies as a function of the number of discrete units. Changing the number of units is equivalent to varying the bitrate of the encoded signal. 
We use a subjective MUSHRA-type listening test~\cite{series2014method} to measure the perceived quality of the proposed speech codec with regard to its bitrate constraints. In MUSHRA evaluations, listeners are presented with a labeled uncompressed signal for reference, a set of test samples to rate, a copy of the uncompressed reference, and a low-quality anchor. Listeners are asked to rate each test utterance and the copy of the uncompressed reference with respect to the labeled reference in a scale of 1-100.

The experiment is performed on the VCTK dataset~\cite{vctk}. For evaluation, we used 20 utterances from 5 speakers. The set of speakers in the test data is disjoint with those in the training data. For this experiment, HuBERT models with 50, 100, and 200 units were trained as described in Sec.~\ref{sec:impl}. For comparison, we included other speech codecs in our evaluation: Opus~\cite{valin2012definition} wideband at 9 kbps VBR, Codec2~\cite{rowe2011codec} at 2.4 kbps and LPCNet~\cite{valin2019real} operating at 1.6 kbps. The LPCNet model was trained from scratch on the VCTK dataset following the experimental setup in~\cite{valin2019real}. The VQ-VAE model employs the HiFiGAN decoder trained on the LibriLight dataset to match the amount of data reported in~\cite{garbacea2019low}. We compressed the anchor sample with Speex~\cite{valin2016speex} at 4 kbps as a low anchor. Fig.~\ref{fig:codec} depicts the results. HuBERT with 50 units reaches the best MUSHRA score while its bitrate is only 365bps, which is significantly lower than the baseline methods.
In this paper, 2D and 3D CNN models were used to generate pelvic sCTs from T1-weighted MR images. Our sCT generation methods were fully automated, requiring no deformable registration or manual segmentation of bone tissues. As shown in Figure~\ref{fig3}, the 2D and 3D CNN models generated high quality sCTs. MAE curves shown in Figure~\ref{fig4} indicated that both models could precisely estimate soft-tissue HU values but had difficulty in reproducing air and high-density bone tissues. 

The MAEs within the body contour across all patients were 40.5 $\pm$ 5.4 HU and 37.6 $\pm$ 5.1 HU for the 2D and 3D models, respectively. The time required for generating a pelvic sCT using our CNN models was about 5.5 s. Our MAE results are comparable to previous studies. Kim $et \ al.$\cite{RN41} presented a voxel-based weighted summation method that produced an MAE of 74.3 $\pm$ 3.9 HU. However, manual contouring of bone tissues required for this method can be tedious and time-consuming. An MAE of 40.5 $\pm$ 8.2 HU was achieved by Dowling $et \ al.$\cite{RN11} using an average MRI-CT atlas from 38 patients. Andreasen $et \ al.$\cite{RN42} reported an MAE of 54 $\pm$ 8 HU using an atlas-based method with pattern recognition, and its prediction time was about 20.8 min. Another random forest model proposed by Andreasen $et \ al.$\cite{RN43} generated sCTs with an MAE of 58 $pm$ 9 HU. A hybrid method suggested by Siversson $et \ al.$ \cite{RN45} obtained an MAE of 36.5 $\pm$ 4.1 HU when ignoring errors introduced by gas cavities. This hybrid method was implemented in the cloud-based commercial software MriPlanner (Spectronic Medical AB, Helsingborg, Sweden), which required 50 to 80 min to generate a sCT.\cite{RN45} The patch-based 3D context-aware generative adversarial network presented by Nie $et \ al.$\cite{RN26} achieved an MAE of 39.0 $\pm$ 4.6 HU. 

Our CNN models reproduced low-density bone as shown in Figure ~\ref{fig4}. The bone-region DSCs were 0.81 $\pm$ 0.04 and 0.82 $\pm$ 0.04 from the 2D and 3D models, respectively. These results are comparable to reported DSC results of 0.79 $\pm$ 0.12\cite{RN10} and 0.91$\pm$0.03{\cite{RN11}}, where the authors compared bone contours manually drawn on the sCT and CT.

It was feasible to train the proposed 3D model with 16 image volumes from scratch. Results of the Wilcoxon signed-rank tests shown in Table~\ref{tab1} demonstrated a statistically significant improvement in overall MAE, bone DSC, and bone precision of the 3D model compared to the 2D model. However, as shown in Figure~\ref{fig4}, the 2D model seemed to perform better in estimating the high-density bone HU values. It should be noted that smaller overall MAEs do not guarantee improved sCT dose calculation and patient positioning performance. While the models performed well, we will continue to acquire more patient data to potentially improve model accuracy and further test model differences.

As this was a retrospective study, the MR image voxel sizes were not matched, resulting in different voxel intensities between images. This may have affected the sCT generation accuracy although we applied intensity normalization. A potential study could examine how voxel size variations affects sCT estimation. 

The proposed 3D model can be implemented on a 12 GB GPU to process volumetric images with dimensions of 256 $\times$ 256 $\times$ 30. More GPU memory would be required to process higher resolution 3D images. Considering the limited access to multi-GPU systems, a 3D architecture with fewer convolutional layers could be considered to deal with higher resolutions. However, the performance could be affected by the reduced parameters and smaller receptive fields of the less complex model. Another approach would be to extract 30-slice sub-volumes from CT and MR images for training the 3D model. The sCT could then be generated by averaging 30-slice sCT sub-volumes produced by the model. 

A number of techniques could be investigated for improving model performance.  Nie $et \ al.$\cite{RN26} showed that introducing an additional adversarial discriminator improved overall sCT quality. The same approach could be adapted in our proposed 2D and 3D CNN models.  Non-rigid deformation\cite{RN44} could also be applied to both CT and MR images in the process of the on-the-fly data augmentation to produce more training pairs. Multiple MR images acquired with different sequences could be fed into models to provide more information for distinguishing different tissues. Multi-GPU systems with more memory would enable the exploration of larger batch sizes for training CNN models, which could reduce variances in gradient estimation and accelerate the training. 



\section*{Acknowledgements}

The authors would like to thank Yannick Hold-Geoffroy for his help in setting up the renders and the user study. We would also like to thank Henrique Weber for his help with EnvyDepth, and Jean-Michel Fortin for his work on HDR data capture.

Parts of this work were done while Marc-Andr\'e Gardner was an intern at Adobe Research. This work was partially supported by the REPARTI Strategic Network and the FRQNT New Researcher Grant 2016NC189939. We gratefully acknowledge the support of Nvidia with the donation of the GPUs used for this research, funding from Adobe to cover the cost of HDR dataset acquisition, as well as a generous gift from Adobe to J.-F. Lalonde. 

\bibliographystyle{ACM-Reference-Format}
\bibliography{template}
\end{document}