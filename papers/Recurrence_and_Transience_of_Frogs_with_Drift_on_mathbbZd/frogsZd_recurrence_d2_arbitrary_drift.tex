\subsection*{Recurrence for $d = 2$ and arbitrary drift}

In this section we prove Theorem~\ref{thm_d=2_arbitrary_drift_i}. Throughout the section let $\alpha < 1$ be fixed.
We couple the frog model with independent site percolation on $\Z^2$. Let $K$ be an integer that will be chosen later. We tessellate $\Z^2$ with segments $(Q_x)_{x \in \Z^2}$ of size $2K+1$. For every $x = (x_1, x_2) \in \Z^2$ we define
\begin{align*}
 q_x &= q_x(K) = \bigl( x_1,  (2K+1)x_2\bigr), \\
 Q_x &= Q_x(K) = \{y \in \Z^2 \colon y_1 = x_1, \lvert y_2-(2K+1)x_2 \rvert \leq K\}.
\end{align*}
We call the site $x \in \Z^2$ open if there are frog paths from $q_x$ to $q_{x+e}$ in $Q_x$ for all $e \in \mathcal{E}_2$. As before, we denote the probability of a site to be open by $p(K,w)$. Note that this probability does not depend on $x$.

\begin{lemma}\label{lemma_d_2_arbitrary_drift_percolation_parameter_bound}
 For $\alpha <1$, in the frog model $\fm(2,\pi_{w,\alpha})$ we have
 \begin{equation*}
  \lim_{K \to \infty} \liminf_{w \to 0} p(K,w) =1.
 \end{equation*}
\end{lemma}

\begin{proof}
We claim that there is a constant $c=c(\alpha)>0$ such that for any $K \in \N_0$ and $x \in Q_0$
\begin{equation}\label{proof_lemma_d_2_arbitrary_drift_percolation_parameter_bound_1}
 \liminf_{w \to 0} \P\Bigr(\bigcap_{ e \in \mathcal{E}_2} \{x \to q_{e}\} \Bigl) \geq c.
\end{equation}
We can estimate the probability in \eqref{proof_lemma_d_2_arbitrary_drift_percolation_parameter_bound_1} by
\begin{equation*}
\P\Bigr(\bigcap_{e \in \mathcal{E}_2} \{x \to q_{e}\} \Bigl) \geq \P\bigl(x \to q_{-e_2} \bigr) \P\bigl(q_{-e_{2}} \to q_{-e_1} \bigr) \P\bigl(q_{-e_{1}} \to q_{e_2} \bigr) \P\bigl(q_{e_{2}} \to q_{e_1} \bigr).
\end{equation*}
The probability of moving in $\pm e_2$-direction for $\lceil w^{-1} \rceil$ steps is $(1-w)^{\lceil w^{-1} \rceil}$. Conditioning on moving in this way, we just deal with a simple random walk on $\Z$. There exists a constant $c_1>0$ such that this random walk hits $-K$ within $\lceil w^{-1} \rceil$ steps with probability at least $c_1$ for all $w$ close to $0$.
Therefore,
\begin{equation} \label{proof_lemma_d_2_arbitrary_drift_percolation_parameter_bound_2}
\P\bigl(x \to q_{-e_2} \bigr) \geq c_1 (1-w)^{\lceil w^{-1} \rceil} 
                              \geq \frac{c_1}{4}.
\end{equation}
The probability of moving exactly once in $-e_1$-direction and otherwise in $\pm e_2$-direction within $\lceil w^{-1} \rceil+1$ steps is
\begin{equation*}
\bigl(\lceil w^{-1} \rceil +1\bigr) \frac{(1-\alpha)w}{2} (1-w)^{\lceil w^{-1} \rceil} \geq \frac{1-\alpha}{8}
\end{equation*}
for $w$ close to $0$. Therefore, analogously to \eqref{proof_lemma_d_2_arbitrary_drift_percolation_parameter_bound_2} there exists a constant $c_2>0$ such that
\begin{equation*}
\P\bigl(q_{-e_2} \to q_{-e_1} \bigr) \geq \frac{c_2(1-\alpha)}{8}
\end{equation*}
for $w$ sufficiently close to $0$. The two remaining probabilities $\P\bigl(q_{-e_{1}} \to q_{e_2} \bigr)$ and $\P\bigl(q_{e_{2}} \to q_{e_1} \bigr)$ can be estimated analogously, which implies \eqref{proof_lemma_d_2_arbitrary_drift_percolation_parameter_bound_1}.

If frog $0$ activates all frogs in $Q_0$ and any of these $2K$ frogs manages to visit the centres of all neighbouring segments, then $0$ is open. By independence of the trajectories of the individual particles in $Q_0$ this implies
\begin{equation}\label{proof_lemma_d_2_arbitrary_drift_percolation_parameter_bound_3}
 p(K,w) \geq \P\Bigl( \bigcap_{x \in Q_0} \{0 \to x\} \Bigr) \biggl(1- \Bigl(1-\P\Bigl(\bigcap_{1\leq i \leq 4} \{x \to q_{e_i}\}\Bigr)\Bigr)^{2K} \biggr).
\end{equation}
As in the proof of Lemma~\ref{lemma_recurrence_small_drift} one can show that for $w \to 0$ the first factor in \eqref{proof_lemma_d_2_arbitrary_drift_percolation_parameter_bound_3} converges to $1$. Therefore, taking limits in \eqref{proof_lemma_d_2_arbitrary_drift_percolation_parameter_bound_3} and using \eqref{proof_lemma_d_2_arbitrary_drift_percolation_parameter_bound_1} yields the claim.
\end{proof}

\begin{proof}[Proof of Theorem~\ref{thm_d=2_arbitrary_drift_i}]
By Lemma~\ref{lemma_d_2_arbitrary_drift_percolation_parameter_bound} we can choose $K$ big and $w$ small enough such that $p(K,w) > p_c(2)$, where $p_c(2)$ is the critical parameter for independent site percolation on $\Z^2$. As in the proof of Theorem~\ref{thm_d=2_arbitrary_weight_i} and Theorem~\ref{thm_d>2_arbitrary_weight} the coupling with supercritical percolation now yields recurrence of the frog model. As we rescaled the lattice $\Z^2$ slightly different this time, the box $B_n$ defined in the proof of Theorem~\ref{thm_d=2_arbitrary_weight_i} and Theorem~\ref{thm_d>2_arbitrary_weight} now corresponds to the box
\begin{equation*}
\fb_n = \{y \in \Z^2 \colon y_1 =-n,\, \lvert y_2\rvert \leq (2K+1)\sqrt{n} +K\}.
\end{equation*}
Since only asymptotics in $n$ matter for the proof, it otherwise works unchanged.
\end{proof}


