%%%%%%%%%%%%%%%%%%%%%%%%%%%%%%%%%%%%%%%%%%%%%%%%%%%%%%%%%%%%%%%%%%%%%%%%%5
% Transience for $d=2$ and arbitrary weight
%%%%%%%%%%%%%%%%%%%%%%%%%%%%%%%%%%%%%%%%%%%%%%%%%%%%%%%%%%%%%%%%%%%%%%%%%%

\subsection*{Transience for $d=2$ and arbitrary weight}

\begin{proof}[Proof of Theorem~\ref{thm_d=2_arbitrary_weight_ii}]
Let $w >0$ be fixed throughout the proof. As in the proof of Theorem~\ref{thm_d=2_arbitrary_drift_ii} and Theorem~\ref{thm_d>2_arbitrary_drift_ii} we dominate the frog model by a branching random walk. This time we use a one-dimensional branching random walk on $\Z$. For the construction of the process, let $\xi$ be the number of activated frogs in an independent one-dimensional frog model $\fm^*(1,\pi_{\text{sym}}, 1-w)$ with two active frogs at $0$ initially. At time $n=0$, the branching random walk starts with one particle in the origin. At every time $n \in \N$, the process repeats the following two steps. First, every particle produces offspring independently of all other particles with the number of offspring being distributed as $\xi$. Then, each particle jumps to the right with probability $\frac{1+\alpha}{2}$ and to the left with probability $\frac{1-\alpha}{2}$. 

As an intermediate step to understand the relation between the frog model and this branching random walk on $\Z$, we first couple the frog model with a branching random walk on $\Z^2$ with initially one particle at $0$. Partition the lattice $\Z^2$ into hyperplanes $(H_n)_{n \in \Z}$ as defined in \eqref{definition_hyperplane}. Let the frog model $\fm(2,\pi_{w,\alpha})$ with initially two active frogs at $0 \in H_0$ evolve and stop every frog when it first enters $H_1$ or $H_{-1}$. Every frog leaves its hyperplane in every step with probability $w$. Thus, the number of stopped frogs is distributed according to $\xi$. A stopped frog is in $H_1$ with probability $\frac{1+\alpha}{2}$ and in $H_{-1}$ with probability $\frac{1-\alpha}{2}$. The stopped particles form the offspring of the particle at $0$ in the branching random walk. We repeat this procedure to generate the offspring of an arbitrary particle in the branching random walk. Introduce an ordering of all particles in the branching random walk and let the particles branch one after another. Before generating the offspring of the $i$-th particle, refill every vertex which is no longer occupied by a sleeping frog with an extra independent sleeping frog. Unstop frog $i$ and let it continue its work as usual, ignoring all other stopped frogs. Note that there is a sleeping frog at the starting vertex of frog $i$ that is immediately activated. This explains our definition of $\xi$. %From the point of view of frog $i$ it sees an environment that equals the one seen by frog $0$ in distribution. 
Again stop every frog once it enters one of the neighbouring hyperplanes. These newly stopped frogs form the offspring of the $i$-th particle. This procedure creates a branching random walk with independent identically distributed offspring. Every vertex visited in the frog model is obviously also visited by the branching random walk. 

Now, project all particles in the intermediate two-dimensional branching random walk onto the first coordinate. This creates a branching random walk on $\Z$ distributed as the one described above. The construction shows that transience of this one-dimensional branching random walk implies transience of the frog model.

To prove that the one-dimensional branching random walk is transient for $\alpha$ close to $1$, we proceed as in the proof of Theorem~\ref{thm_d=2_arbitrary_drift_ii} and Theorem~\ref{thm_d>2_arbitrary_drift_ii}. The proof only differs in the calculation of the parameter $\mu$ defined by  
\begin{equation*}
 \mu = \E \Bigl[ \sum_{i \in D_1} \e^{-\theta  X_1^i} \Bigr] 
\end{equation*}
for $\theta >0$ with $D_1$ denoting the set of descendants in the first generation of the branching random walk and $X_1^i$ the $e_1$-coordinate of the location of particle $i \in D_1$. Here, we immediately get
\begin{equation*}
 \mu = \frac12  \bigl((1-\alpha) \e^{\theta} + (1+\alpha) \e^{-\theta}\bigr) \E[\xi].
\end{equation*}
Lemma~\ref{lemma_1d_fm} implies $\E[\xi] < \infty$. Thus, we can choose $\theta = \log\bigl(2\E[\xi]\bigr)$. Then $\lim_{\alpha \to 1} \mu = \frac12$ and by continuity $\mu < 1$ for $\alpha$ close to $1$, as required.
\end{proof}



