The frog model is a model of interacting random walks or, more generally, Markov chains on a graph $G=(V,E)$ in discrete time $\N_0$. It can be described as follows: There is one distinguished 
vertex $x_0\in V$, called the origin, and at time $0$ there is exactly one active particle (awake frog) at $x_0$. At every other vertex $x$, there is a (possibly zero) number $\eta_x$ of sleeping frogs. 
The frog at $x_0$ now starts walking randomly on the graph and each time it visits a site with sleeping frogs, they immediately become active and start performing random walks and waking up sleeping frogs themselves, independently of each other and of all other frogs. The transition mechanism of the individual frogs is the same for all frogs. The frog model is called recurrent if the probability that the origin $x_0$
is visited infinitely often equals $1$, otherwise the model is called transient. The frog model with $V=\Z^d$, $E$ the set of nearest-neighbour edges on $\Z^d$, $x_0:=0$, $\eta_x=1$ for each $x\in\Z^d\setminus\{0\}$ and the underlying random walk being simple random walk (SRW) on $\Z^d$ was studied by Telcs and Wormald \cite{TW99} who, however, called it egg model. The name frog model was only later suggested by Durrett. 
In \cite{TW99}, it is in particular shown that the frog model is recurrent for each dimension $d$. See also \cite{P01}.
Note that the frog model on $\Z^d$ with SRW is trivially recurrent for $d=1,2$, due to P\'{o}lya's theorem. Thus, in \cite{GS09} Gantert and Schmidt considered the frog model on $\Z$ with the underlying random walk having a drift to the right. They considered both fixed and i.i.d.~random initial configurations $(\eta_x)_{x\in\Z\setminus\{0\}}$ of sleeping frogs and derived a criterion separating transience from recurrence. In the case of an i.i.d.~initial configuration of sleeping frogs they also proved a zero-one law, which says that the probability of infinitely many returns to $0$ equals $1$ if $\E[\log^+(\eta)]=\infty$, and equals $0$ otherwise. Remarkably, this result only depends on the distribution of $\eta$ and does, in particular, not depend on the value of the drift. 
The recurrence part of the latter result was generalised to any dimension~$d$ by D\"obler and Pfeifroth in \cite{DP14}. They proved that the frog model on $\Z^d$ with underlying (irreducible) random walk which has an arbitrary drift to the right is recurrent provided that $\E[\log^+(\eta)^{\frac{d+1}{2}}]=\infty$. Another sufficient recurrence condition involving the tail behaviour of $\eta$ is derived in \cite{KZ17}. 
Kosygina and Zerner proved in \cite{KZ17} a zero-one law under the general condition that the frog trajectories are given by a transitive Markov chain. 
Recurrence and transience for the frog model on the $d$-ary tree have recently been investigated in \cite{HJJ14} and \cite{HJJ16} by Hoffman, Johnson and Junge. 
Other publications on the frog model include \cite{AMP02}, \cite{FMS04}, \cite{GNR17}, \cite{HW16}, \cite{JJ16} and \cite{JJ16sto} and \cite{R17} and references therein (the list is not exhaustive).

In the present article we study recurrence and transience of the frog model on $\Z^d$ for $d \geq 2$ when the underlying transition mechanism is not  symmetric. We assume that at each vertex in $\Z^d\setminus\{0\}$ there is exactly one sleeping frog at time $0$. 
Given this assumption, and using the zero-one law proved in \cite{KZ17}, one can now classify the transition laws of the particles in a recurrent and a transient class. Our proofs show that both regimes exist. In order to give more quantitative statements, we focus on a model in which the particles perform nearest neighbour random walks which are balanced in all but one direction. More precisely,
set $\mathcal{E}_d=\{\pm e_j \colon 1\leq j\leq d\}$ where $e_j$ denotes the $j$-th standard basis vector in $\R^d$, $j=1,\dotsc,d$. 
The particles move according to the following transition probabilities, which depend on two parameters $w \in [0,1]$ and $\alpha \in [0,1]$:
\begin{equation}\label{transition_function}
 \pi_{w,\alpha}(e) =
 \begin{cases}
  \frac{w(1+\alpha)}{2} & \text{for $e=e_1$} \\
  \frac{w(1-\alpha)}{2} & \text{for $e=-e_1$} \\
  \frac{1-w}{2(d-1)}    & \text{for $e \in\{\pm e_2,\dotsc, \pm e_d\}$} 
 \end{cases}
\end{equation}
The parameter $w$ is the weight of the drift axis $e_1$, i.e.~the random walk chooses to go in direction $\pm e_1$ with probability $w$. The parameter $\alpha$ describes the strength of the drift, i.e.~if the random walk has chosen to move in drift direction, it takes a step in direction $e_1$ with probability $\frac{1+\alpha}{2}$ and in direction $-e_1$ with probability $\frac{1-\alpha}{2}$. All other directions are balanced and equally probable.
Sometimes we need to consider the corresponding one-dimensional model where we have to demand $w=1$, i.e.~the transition probabilities are defined by $\pi_{\alpha}(e_1)=1-\pi_{\alpha}(-e_1)=\frac{1 + \alpha}{2}$. 
We denote the frog model on $\Z^d$ with parameters $w$ and $\alpha$ by $\fm(d,\pi_{w,\alpha})$.

First, let us discuss the extreme cases. For $w=1$ the frog model is one-dimensional and thus transient for any $\alpha \in (0,1]$ and recurrent for $\alpha=0$ by \cite{GS09}.
For $\alpha =1$ one easily checks that it is transient for any $w \in (0,1]$.
If $w=0$, then $\fm(d,\pi_{0, \alpha})$ is equivalent to the symmetric frog model in $d-1$ dimensions and hence recurrent.
If $\alpha =0$, we are back in the balanced case and the model is recurrent. This follows from Theorem~\ref{thm_d=2_arbitrary_weight_i} and Theorem~\ref{thm_d>2_arbitrary_weight} below.

In dimension $d=2$ the frog model is recurrent whenever $\alpha$ or $w$ are sufficiently small, i.e.~if the underlying transition mechanism is almost balanced. It is transient for $\alpha$ or $w$ close to $1$.

\begin{thm}\label{thm_d=2_arbitrary_weight}
 Let $d =2$ and $w \in (0,1)$. 
 \begin{enumerate}
  \item\label{thm_d=2_arbitrary_weight_i} There exists $\alpha_r = \alpha_r(w) > 0$ such that the frog model $\fm(d,\pi_{w,\alpha})$ is recurrent for all $0 \leq \alpha \leq \alpha_r$.
  \item\label{thm_d=2_arbitrary_weight_ii} There exists $\alpha_t = \alpha_t(w) < 1$ such that the frog model $\fm(d,\pi_{w,\alpha})$ is transient for all $\alpha_t \leq \alpha \leq 1$.
 \end{enumerate}
\end{thm}

\begin{thm}\label{thm_d=2_arbitrary_drift}
 Let $d=2$ and $\alpha \in (0,1)$.
 \begin{enumerate}
  \item\label{thm_d=2_arbitrary_drift_i} There exists $w_r = w_r(\alpha) > 0$ such that the frog model $\fm(d,\pi_{w,\alpha})$ is recurrent for all $0 \leq w \leq w_r$.
  \item\label{thm_d=2_arbitrary_drift_ii} There exists $w_t = w_t(\alpha) < 1$ such that the frog model $\fm(d,\pi_{w,\alpha})$ is transient for all $w_t \leq w \leq 1$.
 \end{enumerate}
\end{thm}

In dimension $d \geq 3$ the frog model is also recurrent if the transition probabilities are almost balanced. Further, for any fixed drift parameter $\alpha \in (0,1]$ it is transient if the weight $w$ is close to $1$. However, in contrast to $d=2$, for fixed $w \in [0,1)$ there is not always a transient regime. This follows from Theorem~\ref{thm_d>2_arbitrary_drift_i} below.

\begin{thm}\label{thm_d>2_arbitrary_weight}
 Let $d \geq 3$ and $w \in (0,1)$. 
 There exists $\alpha_r = \alpha_r(d,w) > 0$ such that the frog model $\fm(d,\pi_{w,\alpha})$ is recurrent for all $0 \leq \alpha \leq \alpha_r$.
\end{thm}


\begin{thm}\label{thm_d>2_arbitrary_drift}
 Let $d\geq 3$ and $\alpha \in (0,1)$.
 \begin{enumerate}
  \item\label{thm_d>2_arbitrary_drift_i} There exists $w_r > 0$, independent of $d$ and $\alpha$, such that the frog model $\fm(d,\pi_{w,\alpha})$ is recurrent for all $0 \leq w \leq w_r$.
  \item\label{thm_d>2_arbitrary_drift_ii} There exists $w_t = w_t(\alpha) < 1$, independent of $d$, such that the frog model $\fm(d,\pi_{w,\alpha})$ is transient for all $w_t \leq w \leq 1$.
 \end{enumerate}
\end{thm}

The results are graphically summarised in Figure~\ref{phase_diagram}. Note that the above theorems only make statements about the existence of recurrent, respectively transient regimes. We do not describe their shapes, as might be suggested by the curves depicted in Figure~\ref{phase_diagram}. For a discussion about their shape we refer the reader to Conjecture~\ref{con_critical_curve} at the end of this paper.

\begin{figure}[h]
\centering
\begin{tikzpicture}[baseline=0pt]
\begin{axis}[
    title = {$d = 2$},
    area legend,
    axis y line =box, 
    axis x line =box, 
    xtick={1},
    xticklabels={$1$},
    ytick={1},
    yticklabels={$1$},
    xlabel=$\alpha$,
    ylabel=$w$,
    every axis y label/.style={at={(ticklabel cs:0.5)},rotate=0}, 
    every axis x label/.style={at={(ticklabel cs:0.5)},rotate=0}, 
    extra x ticks={0},
    extra x tick labels={$0$},
    extra x tick style={xticklabel style={anchor=north east}},
    xmin=0,
    xmax=1,
    ymin=0,
    ymax=1,
    smooth,
    axis on top,
    x=4cm,
    y=4cm
    ]
\addplot[mark=none, pattern=my crosshatch dots, draw=black]  coordinates {(0,1) (0.1,0.4) (0.4, 0.1) (1,0) } \closedcycle; \label{recurrent} 
\addplot[mark=none, pattern=north east lines, draw=black]  coordinates {(1,0) (0.9, 0.6) (0.6,0.9) (0,1)  (1,1)} \closedcycle; \label{transient} 
\end{axis}                    
\end{tikzpicture}%
%
\hskip 40pt
%
\begin{tikzpicture}[baseline=0pt]
 \begin{axis}[
    title = {$d \geq 3$},
    axis y line =box, 
    axis x line =box, 
    xtick={1},
    xticklabels={$1$},
    ytick={1},
    yticklabels={$1$},
    xlabel=$\alpha$,
    ylabel=$w$,
    every axis y label/.style={at={(ticklabel cs:0.5)},rotate=0},
    every axis x label/.style={at={(ticklabel cs:0.5)},rotate=0},
    extra x ticks={0},
    extra x tick labels={$0$},
    extra x tick style={xticklabel style={anchor=north east}},
    xmin=0,
    xmin=0,
    xmax=1,
    ymin=0,
    ymax=1,
    smooth,
    axis on top,
    x=4cm,
    y=4cm
    ]
\addplot[mark=none, pattern=my crosshatch dots, draw=black]  coordinates {(0,1) (0.1,0.5) (0.4, 0.25) (1,0.2) } \closedcycle; 
\addplot[mark=none, pattern=north east lines, draw=black]  coordinates {(1,0.8) (0,1)  (1,1)} \closedcycle;
\end{axis}
\end{tikzpicture}%
%%
\caption{Phase diagram for the frog model $\fm(d,\pi_{w,\alpha})$: the recurrent regime is marked by \ref{recurrent}, the transient one by \ref{transient}.}
\label{phase_diagram}
\end{figure}

These results show that, in contrast to $d=1$, recurrence and transience do depend on the drift in every dimension $d \geq 2$. This disproves the last conjecture in \cite{GS09} that some condition on the moments of $\eta$ would separate transience from recurrence as in the one-dimensional case.

The paper is organised as follows. In Section~\ref{preliminaries} we introduce notation used throughout the article, and collect some basic facts and results about random walks, percolation and the frog model, which are needed in the proofs. The proofs of the main results are presented in Section~\ref{proofs}. Further questions and conjectures are discussed in Section~\ref{open_problems}.









