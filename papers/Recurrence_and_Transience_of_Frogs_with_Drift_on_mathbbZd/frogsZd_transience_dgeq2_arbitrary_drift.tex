%%%%%%%%%%%%%%%%%%%%%%%%%%%%%%%%%%%%%%%%%%%%%%%%%%%%%%%%%%%%%%%%%%%%%%%%%5
% Transience for $d\geq 2$ and arbitrary drift
%%%%%%%%%%%%%%%%%%%%%%%%%%%%%%%%%%%%%%%%%%%%%%%%%%%%%%%%%%%%%%%%%%%%%%%%%%

\subsection*{Transience for $d\geq 2$ and arbitrary drift}

\begin{proof}[Proof of Theorem~\ref{thm_d=2_arbitrary_drift_ii} and Theorem~\ref{thm_d>2_arbitrary_drift_ii}]
Let the parameters $\alpha>0$ and $d\geq 2$ be fixed throughout the proof. 
For $x \in \Z^d$ we define 
\begin{equation}
 L_x = \{y \in \Z^d \colon y_i = x_i \text{ for all $2 \leq i \leq d$}\}.
\end{equation}
$L_x$ consists of all vertices which agree in all coordinates with $x$ except the $e_1$-coordinate. 
The key observation used in the proof is that all particles mainly move along these lines if the weight $w$ is large.

We dominate the frog model by a branching random walk on $\Z^d$. At time $n=0$ the branching random walk starts with one particle at the origin. At every step in time every particle produces offspring as follows: For every particle located at $x \in \Z^d$ consider an independent copy of the frog model. At any vertex $z \in \Z^d \setminus L_x$ the particle produces $|\{y \in L_x \colon x \fp{L_x} y, y \to z\}|$ many children. Notice that this number might be $0$ or infinite. The particle does not produce any offspring at a vertex in $L_x$. 
Further, note that the particles reproduce independently of each other as we use independent copies of the frog model to generate the offspring.

One can couple this branching random walk with the original frog model.
To explain the coupling, let us briefly describe how to go from the original frog model to the branching random walk. Recall that the frog model is entirely determined by a set of trajectories $(S_n^x)_{n \in \N_0, x \in \Z^d}$ of random walks. We use this set of trajectories to produce the particles in the first generation of the branching random walk, i.e.~the children of the particle initially at $0$, as explained above. Now, assume that the first $n$ generations of the branching random walk have been created. Enumerate the particles in the $n$-th generation. When generating the offspring of the $i$-th particle in this generation, delete all trajectories of the frog model used for generating the offspring of a particle~$j$ with $j < i$ or a particle in an earlier generation, and replace them by independent trajectories. Otherwise, use the original trajectories.

One can check that the branching random walk dominates the frog model in the following sense: 
For every frog in $\Z^d \setminus L_0$ that is activated and visits $0$ there is a particle at $0$ in the branching random walk. Thus, the number of visits to the origin by particles in the branching random walk is at least as big as the number of visits to $0$ by frogs in the frog model, not counting those visits to $0$ made by frogs initially in $L_0$. 
Note that, if the frog model was recurrent, then almost surely there would be infinitely many frogs in $\Z^d \setminus L_0$ activated that return to $0$. In particular, also in the branching random walk infinitely many particles would return to $0$. Therefore, to prove transience of the frog model it suffices to show that in the branching random walk only finitely many particles return to $0$ almost surely.

Let $D_n$ denote the set of descendants in the $n$-th generation of the branching random walk. Further, for $i \in D_n$ let $X_n^i$ be the $e_1$-coordinate of the location of particle $i$. Define for $\theta >0$ and $n \in \N_0$
\begin{equation}
 \mu = \E \Bigl[ \sum_{i \in D_1} \e^{-\theta  X_1^i} \Bigr] \qquad \text{and}  \qquad  M_n = \frac{1}{\mu^n} \sum_{i \in D_n} \e^{-\theta X_n^i}.
\end{equation}
We claim that $\mu <1$ for $w$ close to $1$ and $\theta$ small, which, in particular, implies that $(M_n)_{n \in \N_0}$ is well-defined. We show this claim in the end of the proof. We next show that $(M_n)_{n \in \N_0}$ is a martingale with respect to the filtration $(\mathcal{F}_n)_{n \in \N_0}$ with $\mathcal{F}_n=\sigma \bigl(D_1, \ldots, D_{n}, (X^i_1)_{i \in D_1}, \ldots, (X^i_{n})_{i \in D_{n}} \bigr)$.

Obviously, $M_n$ is $\mathcal{F}_n$-measurable. For a particle $i \in D_n$ denote its descendants in generation $n+1$ by $D_{n+1}^i$. Since particles branch independently, we get
\begin{align*}
\E[ M_{n+1} | \mathcal{F}_n ] &= \E \Bigl[\frac{1}{\mu^{n+1}} \sum_{i \in D_{n+1}} \e^{-\theta X_{n+1}^i} \bigm\vert \mathcal{F}_n \Bigr] \\
                              &= \frac{1}{\mu^n} \sum_{i \in D_{n}} \e^{-\theta X_{n}^i} \cdot \frac{1}{\mu} \E \Bigr[ \sum_{j \in D_{n+1}^i} \e^{-\theta \left( X_{n+1}^j - X_{n}^i \right)} \Bigl].
\end{align*}
Note that the expectation on the right hand side is independent of $i$ and $n$ and therefore, by the definition of $\mu$, we conclude
\begin{align*}
\E[ M_{n+1} | \mathcal{F}_n ] = M_n.
\end{align*}
This calculation also yields $\E[\lvert M_n\rvert]= \E[M_n]=\E[M_0]=1$, and therefore $M_n \in \mathcal{L}^1$. This in particular implies that $M_n$ is finite almost surely for every $n \in \N_0$. Thus, $X_n^i=0$ can only occur for finitely many $i \in D_n$ almost surely for every $n \in \N_0$, i.e.~in every generation only finitely many particles can be at $0$.
By the martingale convergence theorem, there exists an almost surely finite random variable $M_\infty$, such that $\lim_{n \to \infty} M_n = M_\infty$ almost surely.
Combining this with $\mu <1$, we get $\lim_{n \to \infty}\sum_{i \in D_{n}} \e^{-\theta X_{n}^i} = 0$ almost surely. Hence, $X_n^i=0$ for some $i \in D_n$ occurs only for finitely many times $n$. Overall, this shows that the branching random walk is transient.

It remains to show $\mu < 1$. Note that the particles in $D_1$ are at vertices in the set $\{y \in \Z^d\setminus L_0 \colon 0 \fp{L_0} y\}$. Therefore, for the calculation of $\mu$ we first need to consider all sites in $L_0$ that are reached from $0$ by frog paths in $L_0$. The idea is to control the number of frogs activated on the negative $e_1$-axis using Lemma~\ref{lemma_1d_fm} and estimating the number of frogs activated on the positive $e_1$-axis by assuming the worst case scenario that all of them will be activated. Then, for every $k \in \Z$ we have to estimate the number of vertices with $e_1$-coordinate $k$ visited by each of these active frogs on the $e_1$-axis. Due to the definition of $\mu$, the sites visited by frogs on the positive $e_1$-axis do not contribute much to $\mu$. 
Recall that $H_k$ denotes the hyperplane that consists of all vertices with $e_1$-coordinate equal to $k\in \Z$, see \eqref{definition_hyperplane}. For $k,i \in \Z$ define 
\begin{equation*}
N_{k,i} = \lvert\{x \in H_k \setminus L_0 \colon (i,0, \ldots, 0) \to x\}\rvert. 
\end{equation*}
As $N_{k,i}$ equals $N_{k-i,0}$ in distribution for all $i,k \in \Z$, we get 
\begin{align} \label{proof_transience_arbirtrary_drift_1}
  \mu &= \E \Bigl[ \sum_{i \in D_1} \e^{-\theta  X_1^i} \Bigr] \nonumber \\
      &= \sum_{i=-\infty}^{\infty} \sum_{k=-\infty}^{\infty} \P\bigl(0 \fp{L_0} (i,0, \ldots, 0)\bigr) \E[N_{k,i}] \e^{-\theta k} \nonumber\\
      &= \sum_{k=-\infty}^{\infty}  \E[N_{k,0}] \e^{-\theta k} \sum_{i=-\infty}^{\infty} \e^{-\theta i} \P\bigl(0 \fp{L_0} (i,0, \ldots, 0)\bigr).
\end{align}
Note that $\P\bigl(0 \fp{L_0} (i,0, \ldots, 0)\bigr)$ is smaller or equal than the probability of the event $\{0 \fp{\Z} i\}$ in the frog model $\fm(1,1,\alpha)$. Hence, by Lemma~\ref{lemma_1d_fm}, there is a constant $c_1 >0$ such that $\P\bigl(0 \fp{L_0} (i,0, \ldots, 0)\bigr) \leq \e^{c_1i}$ for all $i \leq 0$. Thus, \eqref{proof_transience_arbirtrary_drift_1} implies that for $\theta<c_1$ there is a constant $c_2=c_2(\theta)< \infty$ such that
\begin{equation}\label{proof_transience_arbirtrary_drift_2}
 \mu \leq c_2 \sum_{k=-\infty}^{\infty}  \E[N_{k,0}] \e^{-\theta k}.
\end{equation}
Next, we estimate $\E[N_{k,0}]$, the expected number of vertices in $H_k \setminus L_0$ visited by a single particle starting at $0$. Recall that the trajectory of frog $0$ is denoted by $(S_n^0)_{n\in \N_0}$. We define $T_k = \min\{n \in \N_0 \colon S_n^0 \in H_k\}$, the entrance time of the hyperplane $H_k$, and $T_k' = \max\{n \in \N_0 \colon S_n^0 \in H_k\}$, the last time frog $0$ is in the hyperplane $H_k$. Obviously, $N_{k,0}=0$ on the event $\{T_k = \infty\}$. Hence, assume we are on $\{T_k < \infty\}$. The particle can only visit a vertex in $H_k \setminus L_0$ at time $T_k$ if the random walk took at least one step in non-$e_1$-direction up to time $T_k$. This happens with probability $\E[1-w^{T_k}]$. Furthermore, the number of vertices visited in $H_k$ after time $T_k$ can be estimated by the number of steps in non-$e_1$-direction taken between times $T_k$ and $T_k'$. This number is binomially distributed and, thus, its expectation equals $(1-w)\E[T_k'-T_k]$. Overall, this implies
\begin{align*}
 \E[N_{k,0}] \leq  \P(T_k < \infty) \bigr(\E\bigl[1- w^{T_k} \mid T_k < \infty \bigr] + (1-w) \E\bigl[T_k' - T_k \mid T_k <\infty\bigr] \bigl).
\end{align*}
For $k < 0$ the probability $\P(T_k < \infty)$ decays exponentially in $k$ by Lemma~\ref{lemma_hitting_probability_hyperplane}. Therefore, we can choose $\theta$ small such that $\P(T_k < \infty) \e^{-\theta k} \leq  \e^{-\theta \lvert k \rvert}$ for all $k \in \Z$. Thus, \eqref{proof_transience_arbirtrary_drift_2} implies
\begin{equation}\label{proof_transience_arbirtrary_drift_3}
 \mu \leq  c_2 \sum_{k=-\infty}^{\infty}  \e^{-\theta \lvert k \rvert}\Bigr( \E\bigl[1- w^{T_k} \mid T_k < \infty \bigr] + (1-w) \E\bigl[T_k' - T_k \mid T_k <\infty\bigr] \Bigl).
\end{equation}
Note that the sum in \eqref{proof_transience_arbirtrary_drift_3} is finite as $\E\bigl[T_k' - T_k \mid T_k <\infty\bigr]$ is independent of $k$.
By monotone convergence $\lim_{w \to 1} \mu =0$ and the right hand side of \eqref{proof_transience_arbirtrary_drift_3} is continuous in $w$. Therefore, we can choose $w$ close to $1$ such that $\mu < 1$, as claimed.
\end{proof}

