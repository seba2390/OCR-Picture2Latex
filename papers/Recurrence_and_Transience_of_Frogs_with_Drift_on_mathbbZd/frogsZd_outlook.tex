We believe that there is a monotone curve separating the transient from the recurrent regime in the phase diagram shown in Figure~\ref{phase_diagram}.

\begin{con}\label{con_critical_curve}
For every dimension $d$ there exists a decreasing function $f_d \colon [0,1] \to [0,1]$ such that the frog model $\fm(d,\pi_{w,\alpha})$ is recurrent for all $w,\alpha \in [0,1]$ such that $w<f_d(\alpha)$ and transient for all $w,\alpha \in [0,1]$ such that $w>f_d(\alpha)$.
\end{con}

Intuitively, the frog model approximates a binary branching random walk for $d \to \infty$ from below, as each frog activates a new frog in every step if there are 'infinitely' many directions to choose from. This leads to the following conjecture.

\begin{con}\label{con_high_d}
 The sequence of functions $(f_d)_{d \in \N}$ is increasing in $d$.
\end{con}

In the proof of Theorem~\ref{thm_d>2_arbitrary_drift_i} we use Lemma~\ref{lemma_recurrence_high_d_percolation_parameter_bound} to show that in the frog model with death a frog cluster is dense with positive probability if the survival probability is larger than $\frac34$ and $d$ is large. 
Indeed, we believe that every infinite frog cluster is dense. Hence, $\fm(d,\pi_{w,\alpha})$ would be recurrent for all $\alpha<1$ if $\fm^*(d-1,\pi_{\text{sym}},1-w)$ has a positive survival probability. Further, we believe that the critical survival probability is decreasing in $d$. See also the discussion in \cite[Chapter~1.2]{AMP02pt}. This would imply that $f_d(1^{-})$ is increasing in $d$.

The comparison with a binary branching random walk raises another question.  Let 
\begin{equation*}
 g \colon [0,1] \to [0,1],\ g(\alpha) = \min\bigl\{1, (2(1-\sqrt{1-\alpha^2}))^{-1}\bigr\}.
\end{equation*}
A binary branching random walk on $\Z^d$ with transition probabilities as in \eqref{transition_function} is recurrent iff $w < g(\alpha)$, see \cite[Section~4]{GM06}. 

\begin{question}
Does the sequence of functions $(f_d)_{d \in \N}$ converge pointwise to $g$ as $d \to \infty$?
\end{question}

