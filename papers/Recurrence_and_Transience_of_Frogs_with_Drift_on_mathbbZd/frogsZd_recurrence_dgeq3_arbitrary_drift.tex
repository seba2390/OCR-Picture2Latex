%%%%%%%%%%%%%%%%%%%%%%%%%%%%%%%%%%%%%%%%%%%%%%%%
% Recurrence for arbitrary drift and $d \geq 3$
%%%%%%%%%%%%%%%%%%%%%%%%%%%%%%%%%%%%%%%%%%%%%%%%

\subsection*{Recurrence for arbitrary drift and $d \geq 3$}

The proof of Theorem \ref{thm_d>2_arbitrary_drift_i} again relies on the idea of comparing the frog model with percolation. But instead of looking at the whole space $\Z^d$ as in the previous proofs, we consider a sequence of $(d-1)$-dimensional hyperplanes $(H_{-n})_{n \in \N_0}$ with $H_{-n}$ as defined in \eqref{definition_hyperplane}. We compare the frogs in each hyperplane with supercritical percolation, ignoring the frogs once they have left their hyperplane and all the frogs from other hyperplanes. Within a hyperplane we now deal with a frog model without drift, but allow the frogs to die in each step with probability $w$ by leaving their hyperplane, i.e.~we are interested in $\fm^*(d-1,\pi_{\text{sym}},1-w)$. Hence, the argument does not depend on the value of the drift parameter $\alpha<1$. 

We start with one active particle in the hyperplane $H_0$. With positive probability this particle initiates an infinite frog cluster in $H_0 $ if $w$ and therefore the probability to leave the hyperplane is sufficiently small. Every frog eventually leaves $H_0$ and has for every $n \in \N$ a positive chance of activating a frog in the hyperplane $H_{-n}$, which might start an infinite cluster there. This is the only time where we need $\alpha <1$ in the proof of Theorem~\ref{thm_d>2_arbitrary_drift_i}. Using the denseness of such clusters we can then proceed as before.

We split the proof of Theorem~\ref{thm_d>2_arbitrary_drift_i} into two parts:

\begin{prop}\label{prop_d>2_arbitrary_drift_large_d}
 There is $d_0 \in \N$ and $w_r > 0$, independent of $d$ and $\alpha$, such that the frog model $\fm(d,\pi_{w,\alpha})$ is recurrent for all $0 \leq w \leq w_r$, $0 \leq \alpha < 1$ and $d \geq d_0$.
\end{prop}

\begin{prop}\label{prop_d>2_arbitrary_drift_small_d}
 For every $d\geq 3$ there is $w_r = w_r(d) > 0$, independent of $\alpha$, such that the frog model $\fm(d,\pi_{w,\alpha})$ is recurrent for all $0 \leq w \leq w_r$ and all $0 \leq \alpha < 1$.
\end{prop}

We first prove Proposition~\ref{prop_d>2_arbitrary_drift_large_d}.
As indicated above we need to study the frog model with death and no drift in $\Z^{d-1}$. To increase the readability of the paper let us first work in dimension $d$ instead of $d-1$ and with a general survival parameter $s$, i.e.~we investigate $\fm^*(d, \pi_{\text{sym}}, s)$ for $d \geq 2$. 

We tessellate $\Z^{d}$ with cubes $(Q'_x)_{x \in \Z^{d}}$ of size $3^{d}$. More precisely, for $x \in \Z^{d}$ we define
\begin{align*}
 Q'_x &= \{y \in \Z^{d} \colon \lVert y-3x \rVert_{\infty} \leq 1\}.
\intertext{Further, for technical reasons, for $a \in (\frac23, 1)$ we define}
 W_x &= \{y \in Q'_x \colon \lVert y-3x \rVert_1 \leq ad\},
\end{align*}
where $\lVert z \rVert_1 = \sum_{i=1}^{2d} \lvert z_i \rvert$ is the graph distance from $z \in \Z^d$ to $0$.  Informally, $W_x$ is the set of all vertices in $Q'_x$ which are ``sufficiently close'' to the centre of the cube.
Consider the box $Q'_x$ for some $x \in \Z^{d}$ and let $o \in W_x$. If there are frog paths in $Q_x'$ from $o$ to vertices close to the centres of all neighbouring boxes, i.e.~if the event 
\begin{equation*}
\bigcap_{e \in \mathcal{E}_d} \bigcup_{y \in W_{x + e}} \{o \fp{Q'_x} y\}
\end{equation*}
occurs, we call the vertex $o$ good. Note that this event only depends on the trajectories of all the frogs originating in the cube $Q'_x$ and the choice of $o$. If $o$ is good and is activated, then also the neighbouring cubes are visited. We show that the probability of a vertex being good is bounded from below uniformly in $d$ and this bound does not depend on the choice of $o$.

\begin{lemma}\label{lemma_recurrence_high_d_percolation_parameter_bound}
Consider the frog model $\fm^*(d,\pi_{\text{sym}},s)$. There are constants $\beta > 0$ and $d_0 \in \N$ such that for all $d \geq d_0$, $s > \frac34$, $\frac23 < a < 2- \frac{1}{s}$, $x \in \Z^d$ and $o \in W_x$
\begin{equation*}
\P(\text{$o$ is good}) > \beta.
\end{equation*}
\end{lemma}

To show this we first need to prove that many frogs in the cube are activated. In the proof of Theorem \ref{thm_d=2_arbitrary_weight_i} and Theorem \ref{thm_d>2_arbitrary_weight} this is done by means of Lemma~\ref{lemma_recurrence_cube_size} using the shape theorem. Here, we use a lemma that is analogous to Lemma~2.5 in \cite{AMP02pt}.

\begin{lemma}\label{lemma_recurrence_high_d_K_d}
 Consider the frog model $\fm^*(d,\pi_{\text{sym}},s)$. There exist constants $\gamma >0$, $\mu > 1$ and $d_0 \in \N$ such that for all $d \geq d_0$, $s > \frac34$, $\frac23 < a < 2- \frac{1}{s}$ and $o \in W_0$ we have
 \begin{equation*}
  \P\Bigl( \bigl\lvert \bigl\{y \in W_0 \colon o \fp{Q'_0} y \bigr\}\bigr\rvert \geq \mu^{\sqrt{d}}  \Bigr) \geq \gamma.
 \end{equation*}
\end{lemma}



\begin{proof}[Proof of Lemma \ref{lemma_recurrence_high_d_K_d}]
The proof consists of two parts. In the first part we show that with positive probability there are exponentially many vertices in $Q'_0$ reached from $o$ by frog paths in $Q'_0$, and in the second part we prove that many of these vertices are indeed in $W_0$. For the first part we closely follow the proof of Lemma~2.5 in \cite{AMP02pt} and rewrite the details for the convenience of the reader.

We examine the frog model with initially one active frog at $o$ and one sleeping frog at every other vertex in $Q'_0$ for $\sqrt{d}$ steps in time. Consider the sets $\mathcal{S}_0=\{o\}$ and $\mathcal{S}_k = \{x \in Q'_0 \colon \lVert x-o \rVert_1=k, \lVert x-o \rVert_{\infty}=1\}$ for $k \geq 1$ and let $\xi_k$ denote the set of active frogs which are in $\mathcal{S}_k$ at time $k$. We will show that, conditioned on an event to be defined later, the process $(\xi_k)_{k \in \N_0}$ dominates a process $(\tilde{\xi_k})_{k \in \N_0}$, which again itself dominates a supercritical branching process. The process $(\tilde{\xi_k})_{k \in \N_0}$ is defined as follows. Initially, there is one particle at $o$. Assume that the process has been constructed up to time $k \in \N_0$. In the next step each particle in $\tilde{\xi}_k$ survives with probability $s$. If it survives, it chooses one of the neighbouring vertices uniformly at random. If that vertex belongs to $\mathcal{S}_{k+1}$ and no other particle in $\tilde{\xi}_k$ intends to jump to this vertex, the particle moves there, activates the sleeping particle, and both particles enter $\tilde{\xi}_{k+1}$. Otherwise, the particle is deleted. In particular, if two or more particles attempt to jump to the same vertex, all of them will be deleted. Obviously, $\tilde{\xi}_k \subseteq \xi_k$ for all $k \in \N_0$. %Thus, we only need to prove the claim for $\tilde{\xi}_k$.

First, we show that for $d$ large it is unlikely that two particles in $\tilde{\xi}_k$ attempt to jump to the same vertex. To make this argument precise we need to introduce some notation. For $x \in \mathcal{S}_k$ and $y \in \mathcal{S}_{k+1}$ with $\lVert x-y \rVert_1=1$ define 
\begin{align*}
\mathcal{D}_x &= \{z \in \mathcal{S}_{k+1} \colon \lVert x-z\rVert_1 = 1\},\\
\mathcal{A}_y &= \{z \in \mathcal{S}_k \colon \lVert z-y \rVert_1 =1 \},\\
\mathcal{E}_x &= \{z \in \mathcal{S}_k \colon \mathcal{D}_x \cap \mathcal{D}_z \neq \emptyset \}.
\end{align*}
$\mathcal{D}_x$ denotes the set of possible descendants of $x$, $\mathcal{A}_y$ the set of ancestors of $y$ and $\mathcal{E}_x$ the set of enemies of $x$. Note that $\mathcal{E}_x = \bigcup_{y \in \mathcal{D}_x} (\mathcal{A}_y \setminus \{x\})$ is a disjoint union.
Let $n_x=\sum_{i=1}^d \1_{\{o_i=0,\, x_i\neq0\}}$. Then one can check that
\begin{align}\label{proof_recurrence_high_d_K_d_0}
 \lvert \mathcal{D}_x\rvert &= 2(d-\lVert o \rVert_1-n_x) + \lVert o \rVert_1 - (k-n_x) = 2d -\lVert o \rVert_1 - k - n_x,\\
 \lvert \mathcal{A}_y\rvert &= k+1. \nonumber
\end{align}
For $x \in \mathcal{S}_k$ let $\chi(x)$ denote the number of particles of $\tilde{\xi}_k$ in $x$. Note that $\chi(x) \in \{0,2\}$ for any $x \in \mathcal{S}_k$ with $k \in \N$.

Let $\zeta_{xy}^k$ denote the indicator function of the event that there is $z \in \mathcal{E}_x$ with $\chi(z)\geq 1$ such that one of the particles at $z$ intends to jump to $y$ at time $k+1$. If $\zeta_{xy}^k=1$, then a particle on $x$ cannot move to $y$ at time $k+1$.

Further, we introduce the event $U_x= \{\chi(z)=2 \text{ for all } z \in \mathcal{E}_x\}$. This event describes the worst case for $x$, when it is most likely that particles at $x$ will not be able to jump.
For $k \leq \sqrt{d}$ we have
\begin{equation*}
 \P(\zeta_{xy}^k=1) \leq \P(\zeta_{xy}^k=1 \mid U_x) \leq \sum_{z \in \mathcal{A}_y \setminus \{x\}} \frac{2s}{2d} = \frac{ks}{d} \leq \frac{1}{\sqrt{d}}.
\end{equation*}
Given $\sigma > 0$ we choose $d$ large such that $\P(\zeta_{xy}^k=1) < \sigma$ for all $k \leq \sqrt{d}$.
Now, we consider the set of all descendants $y$ of $x$ such that there is a particle at some vertex $z \in \mathcal{E}_x$ that tries to jump to $y$ at time $k+1$. This set contains $\sum_{y \in \mathcal{D}_x} \zeta_{xy}^k$ elements. Let $\zeta_x^k$ denote the indicator function of the event $\bigl\{\sum_{y \in \mathcal{D}_x} \zeta_{xy}^k > 2\sigma d\bigr\}$. If $\zeta_{x}^k=1$, then more than $2\sigma d$ of the $2d$ neighbours of $x$ are blocked to a particle at $x$.

The random variables $\{\zeta_{xy}^k \colon y \in \mathcal{D}_x\}$ are independent with respect to $\P(\cdot \mid U_x)$ as $\mathcal{E}_x = \bigcup_{y \in \mathcal{D}_x} (\mathcal{A}_y \setminus \{x\})$ is a disjoint union. Using $2d-ad-2k \leq \lvert\mathcal{D}_x\rvert \leq 2d$ and a standard large deviation estimate we get for $k \leq \sqrt{d}$
\begin{align*}
 \P(\zeta_x^k=1) &\leq \P\biggl(\sum_{y \in \mathcal{D}_x} \zeta_{xy}^k > 2\sigma d \Bigm\vert U_x\biggr) \\
                 &\leq \P\biggl(\frac{1}{\lvert \mathcal{D}_x\rvert} \sum_{y \in \mathcal{D}_x} \zeta_{xy}^k > \sigma \Bigm\vert U_x \biggr)\\
                 &\leq \e^{-c_1\lvert\mathcal{D}_x\rvert} \\
                 &\leq \e^{-c_2 d}
\end{align*}
with constants $c_1, c_2 > 0$. Next, let us consider the bad event
\begin{equation*}
 B = \bigcup_{k=1}^{\sqrt{d}} \bigcup_{x \in \tilde{\xi}_k} \{\zeta_x^k=1\}.
\end{equation*}
Then with $\lvert\tilde{\xi}_k\rvert \leq 2^k \leq 2^{\sqrt{d}}$ we get
\begin{equation*}
 \P(B) \leq \sqrt{d} \cdot 2^{\sqrt{d}} \cdot \e^{-c_2 d}.
\end{equation*}
In particular $\P(B)$ can be made arbitrarily small for $d$ large.
Conditioned on $B^c$, in each step for every particle there are at least 
\begin{equation*}
\lvert\mathcal{D}_x\rvert-2 \sigma d -1 \geq (2-a-2\sigma)d - 3 \sqrt{d} 
\end{equation*}
available vertices in $\mathcal{S}_{k+1}$, i.e.~vertices a particle at $x$ can jump to in the next step. Thus, conditioned on $B^c$, the process $\tilde{\xi}_k$ dominates a branching process with mean offspring at least
\begin{equation*}
 \frac{\bigl((2-a-2\sigma)d - 3 \sqrt{d}\bigr) \cdot 2 \cdot s}{2d}.
\end{equation*}
For $\sigma$ small and $d$ large the mean offspring is bigger than $1$ as we assumed $a < 2-\frac{1}{s}$. Since a supercritical branching process grows exponentially with positive probability, there are constants $c_3 >1$, $q \in (0,1)$ that do not depend on $d$ such that
\begin{equation}\label{proof_recurrence_high_d_K_d}
\P\bigl( \lvert\tilde{\xi}_{\sqrt{d}}\rvert \geq c_3^{\sqrt{d}}\bigr) \geq q.
\end{equation}
For the second part of the proof condition on the event $\bigr\{\lvert\tilde{\xi}_{\sqrt{d}}\rvert \geq c_3^{\sqrt{d}}\bigl\}$ and choose $0 < \varepsilon <a-\frac23$. If $\lVert o \rVert_1 \leq (a-\varepsilon)d$, all particles of $\tilde{\xi}_{\sqrt{d}}$ are in $W_0$ for $d$ large. This immediately implies the claim of the lemma. Otherwise, let $n=\lvert\tilde{\xi}_{\sqrt{d}}\rvert$, enumerate the particles in $\tilde{\xi}_{\sqrt{d}}$ and let $\tilde{S}^i$, $1 \leq i \leq n$, denote the position of the $i$-th particle. Further, we define for $1 \leq i \leq n$
\begin{equation*}
 X_i =
 \begin{cases}
  1 & \text{if $\lVert \tilde{S}^i \rVert_1 \leq \lVert o \rVert_1 $}, \\
  0 & \text{otherwise.}
 \end{cases}
\end{equation*}
It suffices to show that $\P(X_1=1)>0$. Then Lemma~\ref{lemma_sum_rv} applied to the random variables $X_1, \ldots, X_n$ implies that with positive probability a positive proportion of the particles in $\tilde{\xi}_{\sqrt{d}}$ indeed have $L_1$-norm smaller than $o$, and are thus in $W_0$. Together with \eqref{proof_recurrence_high_d_K_d} this finishes the proof.

For the proof of the claim let $\tilde{S}^1_k$ denote the position of the ancestor of $\tilde{S}^1$ in $\mathcal{S}_k$, where $0 \leq k \leq \sqrt{d}$. Note that $\tilde{S}^1_0 = o$ and $\tilde{S}^1_{\sqrt{d}} = \tilde{S}^1$.

We are interested in the process $(\lVert \tilde{S}_k^1 \rVert_1)_{1 \leq k \leq \sqrt{d}}$. By the construction of the process $(\tilde{\xi}_k)_{k \in \N_0}$ it either increases or decreases by $1$ in every step. The positions $\tilde{S}_k^1$ and $\tilde{S}_{k+1}^1$ differ in exactly one coordinate. If this coordinate is changed from $0$ to $\pm 1$, then $\lVert \tilde{S}_{k+1}^1\rVert_1$ = $\lVert \tilde{S}_k^1 \rVert_1 +1$. If it is changed from $\pm 1$ to $0$, then we have $\lVert \tilde{S}_{k+1}^1\rVert_1$ = $\lVert \tilde{S}_k^1 \rVert_1 -1$. There are at least $(a-\varepsilon)d-\sqrt{d}$ many $\pm 1$-coordinates in $\tilde{S}_k^1$ that can be changed to $0$. As we also know that $\tilde{S}_{k+1}^1 \in \mathcal{D}_{\tilde{S}_k^1}$, we have for all $k \leq \sqrt{d}$ by \eqref{proof_recurrence_high_d_K_d_0} and the choice of $\varepsilon$
\begin{equation*}
 \P\bigl(\lVert \tilde{S}_{k+1}^1\rVert_1 = \lVert \tilde{S}_k^1 \rVert_1 -1\bigr) 
	\geq \frac{(a-\varepsilon)d-\sqrt{d}}{\lvert\mathcal{D}_{\tilde{S}_k^1}\rvert} 
	\geq \frac{(a-\varepsilon)d - \sqrt{d}}{2d - (a-\varepsilon)d} 
	> \frac12
\end{equation*}
for $d$ large. Hence, $\lVert \tilde{S}_k^1 \rVert_1$ dominates a random walk with drift on $\Z$ started in $\lVert o \rVert_1$. Therefore, 
\begin{equation*}
\P(X_1 = 1) = \P\bigl(\lVert \tilde{S}_{\sqrt{d}}^1 \rVert_1 \leq \lVert o \rVert_1\bigr) \geq \frac12,
\end{equation*}
which finishes the proof.
\end{proof}


\begin{proof}[Proof of Lemma~\ref{lemma_recurrence_high_d_percolation_parameter_bound}]
By Lemma~\ref{lemma_recurrence_high_d_K_d}, with probability at least $\gamma$ there are frog paths in $Q'_x$ from $o$ to at least $\mu^{\sqrt{d}}$ vertices in $W_x$ for $d$ large. We divide the frogs on these vertices into $2d$ groups of size at least $\mu^{\sqrt{d}}/2d$ and assign each group the task of visiting one of the neighbouring boxes $W_{x+e}$, $e \in \mathcal{E}_d$. Notice that this job is done if at least one of the frogs in the group visits at least one vertex in the neighbouring box. If all groups succeed, $o$ is good. Any frog in any group is just three steps away from its respective neighbouring box $W_{x+e}$, $e \in \mathcal{E}_d$, and thus has probability at least $(\frac{s}{2d})^3$ of achieving its group's goal. Hence,
\begin{equation*}
\P(\text{$o$ is good}) \geq  \Bigl(1- \Bigl(1-\Bigl(\frac{s}{2d}\Bigr)^3\Bigr)^{\mu^{\sqrt d}/{2d}} \Bigr)^{2d} \gamma 
                       \geq \frac{\gamma}{2}
\end{equation*}
for $d$ large.  
\end{proof}

In the other recurrence proofs we couple the frog model with percolation by calling a cube open if its centre is good. Here, the choice of a ``starting'' vertex, like the centre, is not independent of the other cubes. Therefore, we cannot directly couple the frog model with independent percolation. However, the following lemma allows us to compare the distributions of a frog cluster and a percolation cluster.

\begin{lemma}\label{lemma_recurrence_high_d_fc=c}
Consider the frog model $\fm^*(d,\pi_{\text{sym}},s)$. Let $\beta >0$ and assume that $\P(\text{$o$ is good}) > \beta$ for all $o \in W_x$, $x \in \Z^d$. Further, consider independent site percolation on $\Z^d$ with parameter $\beta$. Then for all sets $A \subseteq \Z^d$, $v \in \Z^d$ and for all $k \geq 0$
\begin{equation*}
 \P(\lvert A \cap C_v\rvert \geq k) \leq \P\Bigl(\Bigl\lvert \bigcup_{x \in A}Q'_x\cap \fc_{3v}^*\Bigr\rvert \geq k\Bigr).
\end{equation*}
\end{lemma}

\begin{proof}
For technical reasons we introduce a family of independent Bernoulli random variables $(X_o)_{o \in \Z^d}$ which are also independent of the choice of all the trajectories of the frogs and satisfy $\P(X_o=1) = \P(\text{$o$ is good})^{-1}\beta$. Their job will be justified soon. Further, we fix an ordering of all vertices in $\Z^d$.

Now we are ready to describe a process that explores a subset of the frog cluster $\fc_{3v}^*$. Its distribution can be related to the cluster $C_v$ in independent site percolation with parameter $\beta$. The process is a random sequence $(R_t, D_t, U_t)_{t\in \N_0}$ of tripartitions of $\Z^d$. As the letters indicate, $R_t$ will contain all sites reached by time $t$, $D_t$ all those declared dead by time $t$, and $U_t$ the unexplored sites. We construct the process in such a way that for all $t \in \N_0$, $x \in R_t$ and $e \in \mathcal{E}_d$ there is $y \in W_{x +e}$ such that there is a frog path from $3v$ to $y$ in $\bigcup_{x \in R_t}Q'_x$. We start with $R_0 = D_0 = \emptyset$ and $U_0 = \Z^d$. If $3v$ is good and $X_{3v}=1$, set $U_1 = \Z^d \setminus \{v\}$, $R_1=\{v\}$, and $D_1=\emptyset$. Otherwise, stop the algorithm. If the process is stopped at time $t$, let $U_s = U_{t-1}$, $R_s = R_{t-1}$ and $D_s = D_{t-1}$ for all $s \geq t$. Assume we have constructed the process up to time $t$. Consider the set of all sites in $U_t$ that have a neighbour in $R_t$. If it is empty, stop the process. Otherwise, pick the site $x$ in this set with the smallest number in our ordering. By the choice of $x$ there is $y \in W_x$ such that there is a frog path from $3v$ to $y$ in $\bigcup_{z \in R_t} Q'_z$. Choose any vertex $y$ with this property. If $y$ is good and $X_y = 1$, set 
\begin{equation*}
R_{t+1} = R_t \cup \{x\},\ D_{t+1} = D_t, \ U_{t+1}=U_t \setminus \{x\}. 
\end{equation*}
Otherwise, update the sets as follows:
\begin{equation*}
R_{t+1} = R_t,\ D_{t+1} = D_t \cup \{x\}, \ U_{t+1}=U_t \setminus \{x\}
\end{equation*}
In every step $t$ the algorithm picks an unexplored site $x$ and declares it to be reached or dead, i.e.~added to the set $R_{t}$ or $D_t$. The probability that $x$ is added to $R_t$ equals $\beta$. This event is (stochastically) independent of everything that happened before time $t$ in the algorithm. Note that every unexplored neighbour of a reached site will eventually be explored due to the fixed ordering of all sites.

In the same way we can explore independent site percolation on $\Z^d$ with parameter $\beta$. Construct a sequence $(R_t', D_t', U_t')_{t\in \N_0}$ of tripartitions of $\Z^d$ as above, but whenever the algorithm evaluates whether a site $x$ is declared reached or dead we toss a coin independently of everything else. Note that $\bigcup_{t \in \N_0} R_t' = C_v$, where $C_v$ is the cluster containing $v$. This exploration process is well known for percolation, see e.g.~\cite[Proof of Theorem 4, Chapter 1]{BR06}.

By construction, $\bigcup_{t\in \N_0} R_t$ equals the percolation cluster $C_v$ in distribution. The claim follows since for every $x \in \bigcup_{t\in \N_0} R_t$ there is a $y \in W_x$ such that there is a frog path from $3v$ to $y$, i.e.~$y \in \fc_{3v}^*$. 
\end{proof}

Now we can show Proposition~\ref{prop_d>2_arbitrary_drift_large_d}. Note that we are again working with the frog model $\fm(d,\pi_{w,\alpha})$ (without death).

\begin{proof}[Proof of Proposition~\ref{prop_d>2_arbitrary_drift_large_d}]
Throughout this proof we assume that $d$ is so large that Lemma~\ref{lemma_recurrence_high_d_percolation_parameter_bound} is applicable for $d-1$ and $p_c(d-1) < \beta$, where $\beta$ is the constant introduced in the statement of Lemma~\ref{lemma_recurrence_high_d_percolation_parameter_bound}. This is possible because of Lemma~\ref{lemma_pc_high_d}. These assumptions in particular imply that we can use Lemma~\ref{lemma_recurrence_high_d_fc=c} and that the percolation introduced there is supercritical.

Consider the sequence of hyperplanes $(H_{-n})_{n \in \N_0}$ defined in \eqref{definition_hyperplane} and let $A$ denote the event that there is at least one frog $v_n$ activated in every hyperplane $H_{-n}$. For technical reasons we want $v_n$ of the form $v_n = (-n, 3w_n)$ for some $w_n \in \Z^{d-1}$. We first show that $A$ occurs with positive probability. To see this consider the first hyperplane $H_0$ and couple the frogs in this hyperplane with $\fm^*(d-1, \pi_{\text{sym}}, 1-w)$ in the following way: Whenever a frog takes a step in $\pm e_1$-direction, i.e.~leaves its hyperplane, it dies instead. By \cite[Theorem 1.8]{AMP02pt} (or Lemma~\ref{lemma_recurrence_high_d_fc=c}) this process survives with positive probability if $w$ is sufficiently small (independent of the dimension $d$). This means that infinitely many frogs are activated in $H_0$. Obviously, this implies the claim. 

From now on we condition on the event $A$. Note that $\fc_{v_n} \subseteq \fc_0$ for $n\in\N$.
Analogously to the proofs in the last sections we introduce boxes 
\begin{equation*}
\fb_n' = \{-n\} \times [-(3\sqrt{n}+1), 3\sqrt{n}+1]^{d-1} 
\end{equation*}
for $n \in \N$. 
We claim that analogously to Lemma~\ref{percolation_density} there are constants $a, b>0$ and $N \in \N$ such that for $n \geq N$
\begin{equation} \label{proof_thm_high_d_1}
 \P\bigl(\lvert\fb_n' \cap \fc_0\rvert \geq a n^{(d-1)/2}\bigr) \geq b.
\end{equation}
To prove this claim let $a,b>0$ and $N \in \N$ be the constants provided by Lemma \ref{percolation_density} for percolation with parameter $\beta$. For $n \geq N$ couple the frog model with $\fm^*(d-1, \pi_{\text{sym}}, 1-w)$ in the hyperplane $H_n$ as above. Let $B_n' = [-\sqrt{n}, \sqrt{n}]^{d-1}$ and note that $B_n'$ corresponds to $\fb_n'$ restricted to $H_n$ after rescaling. Then by Lemma~\ref{lemma_recurrence_high_d_fc=c} and Lemma~\ref{percolation_density}
\begin{align*}
 \P \bigl(\lvert\fb_n' \cap \fc_{v_n}\rvert \geq a n^{(d-1)/2 } | A \bigr)
    &\geq \P \bigl(\lvert\fb_n' \cap (\{-n\} \times \fc_{3w_n}^*)\rvert \geq a n^{(d-1)/2) } | A \bigr) \\
    &\geq \P\bigl(\lvert B_n' \cap C_{w_n}\rvert \geq a n^{(d-1)/2)}  | A \bigr) \\
    &\geq b.
\end{align*}
Here, $C_{w_n}$ is the open cluster containing $w_n$ in a percolation model with parameter $\beta$ in $\Z^{d-1}$, independently of the frogs.
As $\fc_{v_n} \subseteq \fc_0$, this implies inequality~\eqref{proof_thm_high_d_1}.

By Lemma~\ref{lemma_hitting_probability_RW_drift} and \eqref{proof_thm_high_d_1}, the probability that there is at least one activated frog in $\fb_n'$ that reaches $0$ is at least
\begin{equation*}
 \Bigl(1-(1-c'n^{-(d-1)/2})^{an^{(d-1)/2}}\Bigr)b \geq \bigl(1 - \e^{-ac'}\bigr)b,
\end{equation*}
where $c'>0$ is a constant. Altogether we get by Lemma~\ref{lemma_sum_rv}
\begin{align*}
 \P(\text{$0$ visited infinitely often}) &=    \lim_{n \to \infty} \P(\text{$0$ is visited $\varepsilon n$ many times }) \\
                                         &\geq \lim_{n \to \infty} \P\biggl( \sum_{i=1}^n \1_{\{\exists x \in \fb_n' \cap \fc_{0} \colon x \to 0 \}} \geq \varepsilon n \biggr)\\
                                         &\geq \Bigl(\bigl(1 - \e^{-ac'}\bigr)b - \varepsilon \Bigr) > 0
\end{align*}
for $\varepsilon$ sufficiently small. The claim now follows from Theorem~\ref{lemma_zero_one_law}.
\end{proof}



%%%%%%%%%%%%%%%%%%%%%%%%%%%%%%%%%%%%%%%%%%%%%%%%%%%%%%%%%%%%%%%%%%%%%%%%%%%%%%%%%%%%%%%%%%%%%%%%%%%%%%%%%%%%
%%%%%%%%%%%%%%%%%%%%%%%%%%%%%%%%%%%%%%%%%%%%%%%%%%%%%%%%%%%%%%%%%%%%%%%%%%%%%%%%%%%%%%%%%%%%%%%%%%%%%%%%%%%%
%%%%%%%%%%%%%%%%%%%%%%%%%%%%%%%%%%%%%%%%%%%%%%%%%%%%%%%%%%%%%%%%%%%%%%%%%%%%%%%%%%%%%%%%%%%%%%%%%%%%%%%%%%%%

To prove Proposition~\ref{prop_d>2_arbitrary_drift_small_d} we again first study the frog model with death $\fm^*(d, \pi_{\text{sym}},s)$ in the hyperplanes and couple it with percolation. This time we use cubes of size $(2K+1)^{d}$ for some $K \in \N_0$. By choosing $K$ large we increase the number of frogs in the cubes. In the proof of the previous proposition this was done by increasing the dimension $d$. For $x \in \Z^d$ and $K \in \N_0$ we define
\begin{align*}
 q_x &= q_x(K) = (2K+1)x, \\
 Q_x &= Q_x(K) = \{y \in \Z^{d} \colon \lVert y-q_x \rVert_{\infty} \leq K\}.
\end{align*}

Note that this definition coincides with \eqref{def_box}.
In analogy to Lemma~\ref{lemma_recurrence_high_d_fc=c} the frog cluster dominates a percolation cluster.

\begin{lemma}\label{lemma_frog_model_with_death_percolation_arbitrary_d}
For $d \geq 2$ consider the frog model $\fm^*(d,\pi_{\text{sym}},s)$ and supercritical site percolation on $\Z^d$. There are constants $s_r(d) < 1$ and $K \in \N_0$ such that for any $s \geq s_r(d)$, $A \subseteq \Z^d$, $v \in \Z^d$ and for all $k \geq 0$
\begin{equation*}
 \P(\lvert A \cap C_v \rvert \geq k) \leq \P\Bigl(\Bigl\lvert \bigcup_{x \in A}Q_x\cap \fc_{q_v}^*\Bigr\rvert \geq k\Bigr).
\end{equation*}
\end{lemma}

\begin{proof}
We couple the frog model with percolation as follows: A site $x \in \Z^{d}$ is called open if for every $e \in \mathcal{E}_{d}$ there exists a frog path from $q_x$ to $q_{x + e}$ in $Q_x$. Note that $x \in C_v$ now implies $q_x \in \fc_{q_v}^*$ for any $v \in \Z^d$. We denote the probability of a site $x$ to be open by $p(K,s)$. By Lemma~\ref{lemma_recurrence_cube_size} $p(K,1)$ is close to $1$ for $K$ large. As in the proof of Lemma~\ref{lemma_recurrence_small_drift} one can show that $\lim_{s \to 1} p(K,s)= p(K,1)$. Thus, we can choose $K \in \N$ and $s_r >0$ such that $p(K,s) > p_c(d)$ for all $s > s_r$, i.e. the percolation is supercritical. 
\end{proof}

\begin{proof}[Proof of Proposition~\ref{prop_d>2_arbitrary_drift_small_d}]
 Using Lemma~\ref{lemma_frog_model_with_death_percolation_arbitrary_d} instead of Lemma~\ref{lemma_recurrence_high_d_fc=c} and boxes $Q_x$ instead of $Q_x'$, the proof is analogous to the proof of Proposition~\ref{prop_d>2_arbitrary_drift_large_d}.
\end{proof}


\begin{proof}[Proof of Theorem \ref{thm_d>2_arbitrary_drift_i}]
 Theorem \ref{thm_d>2_arbitrary_drift_i} follows from Proposition~\ref{prop_d>2_arbitrary_drift_large_d} and Proposition~\ref{prop_d>2_arbitrary_drift_small_d}.
\end{proof}

