%%%%%%%%%%%%%%%%%%%%%%%%%%%%%%%%%%%%%%%%%%%%%%%%
% Recurrence for $d \geq 2$ and arbitrary weight
%%%%%%%%%%%%%%%%%%%%%%%%%%%%%%%%%%%%%%%%%%%%%%%%

\subsection*{Recurrence for $d \geq 2$ and arbitrary weight}
In this section we prove Theorem \ref{thm_d=2_arbitrary_weight_i} and Theorem \ref{thm_d>2_arbitrary_weight}. Throughout this section assume that $w <1$ is fixed. 
To illustrate the basic idea of the proof we first sketch it for $d=2$. We call a site $x$ in $\Z^2$ open if the trajectory $(S_n^x)_{n \in \N_0}$ of frog $x$ includes the four neighbouring vertices $x \pm e_1, x \pm e_2$ of $x$, i.e.~if $x \to x \pm e_1$ and $x \to x \pm e_2$. Note that for this definition it does not matter whether frog $x$ is ever activated or not. All sites are open independently of each other due to the independence of the trajectories of the frogs. Furthermore, the probability of a site to be open is the same for all sites. Consider the percolation cluster $C_0$ that consists of all sites that can be reached from $0$ by open paths, i.e.~paths containing only open sites. Note that all frogs in $C_0$ are activated as frog $0$ is active in the beginning. In this sense the frog model dominates the percolation.
As we are in $d=2$, the probability of a site $x$ being open equals $1$ for $\alpha=0$ and by continuity is close to $1$ if $\alpha$ is close to $0$. Thus, if $\alpha$ is close enough to $0$ the percolation is supercritical. 
Hence, with positive probability the cluster $C_0$ containing the origin is infinite. By Lemma~\ref{percolation_density} this infinite cluster contains many sites close to the negative $e_1$-axis. This shows that many frogs that are initially close to this axis get activated. Each of them travels in the direction of the $e_1$-axis and has a decent chance of visiting $0$ on its way. Hence, this will happen infinitely many times. This argument shows that the origin is visited by infinitely many frogs with positive probability. Using the zero-one law stated in Theorem~\ref{lemma_zero_one_law} yields the claim.

In higher dimensions the probability of a frog to visit all its neighbours is not close to $1$ however small the drift may be. We can still make the reasoning work by using a renormalization type argument. 
To make this argument precise let $K$ be a non-negative integer that will be chosen later. We tessellate $\Z^d$ for $d \geq 2$ with cubes $(Q_x)_{x \in \Z^d}$ of size $(2K+1)^d$. For every $x \in \Z^d$ we define
\begin{align}\label{def_box}
\begin{split}
 q_x &= q_x(K) = (2K+1)x,\\
 Q_x &= Q_x(K) = \{y \in \Z^d \colon \lVert y-q_x \rVert_{\infty} \leq K\}, 
\end{split}
\end{align}
where $\lVert x \rVert_{\infty} = \max_{1 \leq i \leq d}{\lvert x_i \rvert}$ is the supremum norm. 
We call a site $x \in \Z^d$ open if for every $e \in \mathcal{E}_d$ there exists a frog path from $q_x$ to $q_{x + e}$ in $Q_x$. Otherwise, $x$ is said to be closed. 
The probability of a site $x$ to be open does not depend on $x$, but only on the drift parameter $\alpha$ and the cube size $K$. We denote it by $p(K, \alpha)$. This defines an independent site percolation on $\Z^d$, which, as mentioned before, is dominated by the frog model in the following sense: For any $x \in C_0$ the frog at $q_x$ will be activated in the frog model, i.e.~$q_x \in \fc_0$ with $\fc_0$ as defined in \eqref{def_frog_cluster}. 

In the next two lemmas we show that the probability $p(K,\alpha)$ of a site to be open is close to $1$ if the drift parameter $\alpha$ is small and the cube size $K$ is large. We first show this claim for the symmetric case $\alpha=0$.

\begin{lemma} \label{lemma_recurrence_cube_size}
For every $w<1$ in the frog model $\fm(d, \pi_{w,0})$ we have
 \begin{equation*}
  \lim_{K \to \infty} p(K, 0) =1.
 \end{equation*}
\end{lemma}

\begin{proof}
For $d=2$ we obviously have $p(K, 0) = 1$ for all $K \in \N_0$ as balanced nearest random walk on $\Z^2$ is recurrent. Therefore, we can assume $d \geq 3$. The proof of the lemma relies on the shape theorem (Theorem~\ref{lemma_shape_theorem}) for the frog model. This theorem assumes equal weights on all directions. As in our model the $e_1$-direction has a different weight, we need a workaround. We couple our model with a modified frog model on $\Z^{d-1}$ in which the frogs in every step stay where they are with probability $w$ and move according to a simple random walk otherwise. A direct coupling shows that, up to any fixed time, in the modified frog model on $\Z^{d-1}$ there are at most as many frogs activated as in the frog model $\fm(d,\pi_{w,0})$. Note that Theorem~\ref{lemma_shape_theorem} holds true for the modified frog model on $\Z^{d-1}$, see Remark~\ref{remark_shape}. Let $\xi_K$, respectively $\xi_K^{\text{mod}}$, be the set of all sites visited by active frogs by time~$K$ in the frog model $\fm(d,\pi_{w,0})$, respectively the modified frog model on $\Z^{d-1}$. Further, let $\overline{\xi_K^{\text{mod}}} := \{x + (-\frac12, \frac12]^{d-1} \colon x \in \xi_K^{\text{mod}}\}$. By Theorem~\ref{lemma_shape_theorem} there exists a non-trivial convex symmetric set $\mathcal{A}=\mathcal{A}(d) \subseteq \R^{d-1}$ and an almost surely finite random variable~$\mathcal{K}$ such that 
 \begin{equation*}
   \mathcal A  \subseteq \frac{\overline{\xi_K^{\text{mod}}}}{K}
 \end{equation*}
for all $K \geq \mathcal{K}$. This implies that there is a constant $c_1 = c_1(d) > 0$ such that $\lvert \xi_K^{\text{mod}}\rvert \geq c_1 K^{d-1}$ for all $K \geq \mathcal{K}$. By the coupling the same statement holds true for $\xi_K$.
As $\xi_K \subseteq Q_0(K)$ and any vertex in $\xi_K$ can be reached by a frog path from $0$ in $Q_0$, this implies
\begin{equation*}
 \Bigl\lvert\Bigl\{y \in Q_0\colon 0 \fp{Q_0} y\Bigr\} \Bigr\rvert \geq \lvert \xi_K\rvert \geq c_1 K^{d-1}
\end{equation*}
for all $K \geq \mathcal{K}$. Thus we have at least $c_1 K^{d-1}$ vertices in the box $Q_0$ that can be reached by frog paths from $0$. Each frog in $Q_0$ has a chance to reach the centre $q_{e}$ of a neighbouring box. More precisely, by Lemma~\ref{lemma_hitting_probability_SRW} there is a constant $c_2 =c_2(d)>0$ such that
\begin{equation} \label{proof_lemma_recurrence_cube_size_0}
 \P \bigl( y \to q_{e} \bigr) \geq \frac{c_2}{K^{d-2}}
\end{equation} 
for any vertex $y \in Q_0$ and $e \in \mathcal{E}_d$.
Hence, for any $e \in \mathcal{E}_d$
\begin{align} \label{proof_lemma_recurrence_cube_size_1}
 \P\bigl( (0 \fp{Q_0} q_{e})^c  \mid K \geq \mathcal{K} \bigr) 
	&= \P \Bigl( \bigl\{y \not\to q_{e} \text{ for all } y \in Q_0 \text{ with } 0 \fp{Q_0} y\bigr\}  \bigm\vert K \geq \mathcal{K} \Bigr) \nonumber\\
        &\leq \Bigl(1-\frac{c_2}{K^{d-2}}\Bigr)^{c_1K^{d-1}} \nonumber\\
        & \leq \e^{-c_1c_2K},
\end{align}
where we used for the first inequality the fact that a frog moves independently of all frogs in $Q_0$ once it will never return to $Q_0$ and the uniformity of the bound in \eqref{proof_lemma_recurrence_cube_size_0}. Therefore,
\begin{align} \label{proof_lemma_recurrence_cube_size_2}
 p(K,0) &\geq  \P\Bigl(\bigcap_{e \in \mathcal{E}_d} \{0 \fp{Q_0} q_{e} \} \Bigm\vert K \geq \mathcal{K} \Bigl) \P_{0}(K \geq \mathcal{K}) \nonumber\\
        &\geq  \biggl[ 1- 2d \operatorname{e}^{-c_1 c_2 K} \biggr] \P(K \geq \mathcal{K}).
\end{align}
Since $\mathcal K$ is almost surely finite, we have $\lim_{K \to \infty}\P_{0}(K \geq \mathcal{K}) =1$. Thus, the right hand side of \eqref{proof_lemma_recurrence_cube_size_2} tends to $1$ in the limit $K\to \infty$.
\end{proof}



\begin{lemma}\label{lemma_recurrence_small_drift}
For fixed $w <1$, in the frog model $\fm(d,\pi_{w,\alpha})$ we have for all $K \in \N_0$
 \begin{equation*}
  \liminf_{\alpha \to 0} p(K, \alpha) \geq p(K,0).
 \end{equation*}
\end{lemma}

\begin{proof}
Let $L(a,b,c,K)$ be the number of possible realizations such that all $q_{x \pm e}$, $e \in \mathcal{E}_d$, are visited by frogs in $Q_0$ for the first time after in total (of all frogs) exactly $a$ steps in $e_1$-direction, $b$ steps in $-e_1$-direction and $c$ steps in all other directions. Note that $L(a,b,c,K)$ is independent of $\alpha$. We have
\begin{equation*}
p(K, \alpha) =   \sum_{a,b,c=1}^\infty L(a,b,c,K) \biggl(\frac{w(1+\alpha)}{2}\biggr)^a \biggl(\frac{w(1-\alpha)}{2}\biggr)^b \biggl(\frac{1-w}{2(d-1)}\biggr)^c.
% \\
%            &\xrightarrow[\alpha \to 0]{} \sum_{a,b,c=1}^\infty L(a,b,c,K) \biggl(\frac{w}{2}\biggr)^{a+b} \biggl(\frac{1-w}{2(d-1)}\biggr)^c = p(K, 0). \qedhere
\end{equation*} 
The claim now follows from Fatou's Lemma.
\end{proof}



\begin{proof}[Proof of Theorem \ref{thm_d=2_arbitrary_weight_i} and Theorem \ref{thm_d>2_arbitrary_weight}]
By Lemma \ref{lemma_recurrence_cube_size} and Lemma \ref{lemma_recurrence_small_drift} we can assume that $K$ is big enough and $\alpha >0$ small enough such that $p(K, \alpha)> p_c(d)$, i.e.~the percolation with parameter $p(K, \alpha)$ on $\Z^d$ constructed at the beginning of this section is supercritical. 

Consider boxes $B_n = \{-n\} \times [-\sqrt{n},\sqrt{n}]^{d-1}$ for $n \in \N$. By Lemma~\ref{percolation_density} there are constants $a,b > 0$ and $N \in \N$ such that for all $n \geq N$
\begin{equation*}
 \P(\lvert B_n \cap C_0\rvert \geq a n^{(d-1)/2})>b.
\end{equation*}
After rescaling, the boxes $B_n$ correspond to the boxes
\begin{equation*}
\fb_n = \{y \in \Z^d \colon \lvert y_1 + (2K+1)n \rvert \leq K,\, \lvert y_i \rvert \leq (2K+1)\sqrt{n} +K,\, 2 \leq i \leq d\}.
\end{equation*}
Recall that $\fc_0$ consists of all vertices reachable by frog paths from $0$ as defined in \eqref{def_frog_cluster}, and note that $x \in B_n \cap C_0$ implies $q_x \in  \fb_n \cap \fc_0$. 
This shows
\begin{equation}\label{proof_thm_recreg_1}
 \P(\lvert \fb_n \cap \fc_0 \lvert \geq a n^{(d-1)/2})>b
\end{equation}
for $n$ large enough. Analogously to \eqref{proof_lemma_recurrence_cube_size_1}, by Lemma~\ref{lemma_hitting_probability_RW_drift} and \eqref{proof_thm_recreg_1} the probability that at least one frog in $\fb_n$ is activated and reaches $0$ is at least
\begin{equation*}
 \Bigl(1-(1-cn^{-(d-1)/2})^{an^{(d-1)/2}}\Bigr)b \geq \bigl(1 - \e^{-ac}\bigr)b,
\end{equation*}
where $c=c(K,d,w)>0$ is a constant. Altogether we get by Lemma~\ref{lemma_sum_rv}
\begin{align*}
 \P(\text{$0$ visited infinitely often}) &=    \lim_{n \to \infty} \P(\text{$0$ is visited $\varepsilon n$ many times }) \\
                                         &\geq \liminf_{n \to \infty} \P\biggl( \sum_{i=1}^n \1_{\{\exists x \in \fb_i \cap \fc_{0} \colon x \to 0 \}} \geq \varepsilon n \biggr) \\
                                         &\geq \bigl(1 - \e^{-ac}\bigr)b - \varepsilon > 0
\end{align*}
for $\varepsilon$ sufficiently small. The claim now follows from Theorem~\ref{lemma_zero_one_law}. 
\end{proof}
















