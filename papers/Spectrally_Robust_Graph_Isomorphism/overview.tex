In this section, we briefly describe our main ideas for the two results claimed above.

\paragraph{\textbf {Hardness for \GD.} } We prove that \GD is NP-hard by reducing from a known NP-hard problem, {\bf Hamiltonian Cycle in the cubic subgrid graphs}. Papadimitriou and Vazirani~\cite{PapadimitriouV84} showed that finding hamiltonian cycle in a subgrid graph is NP-hard even if the degrees of all the vertices is bounded by $3$. We set $H$ to be the cycle of length $n$ and show that if $G$ is a cubic subgrid graph, then there exists $\pi$ satisfying $L_{\pi(H)} \preceq L_G$ iff $G$ contains a hamiltonian cycle. The proof can also be modified by deleting one edge from $G$ and $H$ so that two new graphs are cubic subgrid and a path. This shows NP-hardness of \GD problem even if $H$ is a path and $G$ is a cubic graph. 

If $G$ contains a hamiltonian cycle, then there is a permutation $\pi$ such that $\pi(H)$ is a subgraph of $G$. Hence, $L_{\pi(H)} \preceq L_G$. To prove the converse, assume that $G$ does not contain a hamiltonian cycle. We prove that for any permutation $\pi$, there exists $x$ such that $x^T L_{\pi(H)} x > x^T L_G x$. We start by assuming that $\pi$ is an identity permutation; otherwise, we can consider the permuted graph $G$. If $G'$ and $H'$ are the graphs obtained by deleting shared edges between $G$ and $H$, then it is sufficient to find a vector $x$ such that $x^T L{G'} x > x^T L_{H'} x$ to show that $x^T L_G x > x^T L_H x$. If in the resulting graphs $G',H'$, there exists a vertex $v$ with $deg_{G'}(v) =1$ and $deg_{H'}(v) \geq 1$, then the resistance between the $u$ and $v$ such that $(u,v) \in E_H'$ is lower in $G'$ than in $H'$. Hence, there exists $x$ such that $x^T L_{H'} x > x^T L_{G'} x$. And if there exists a vertex $v$ with $deg_{G'}(v) = 1, deg_{H'}(v) = 0$, then we can simply ignore the edge incident to $v$. Next, we show that if there exists a set $S$ such that number of edges going out of $S$ in $H'$ is more than in $G'$, then the cut vector $x$ shows $x^T L_{H'} x > x^T L_{G'} x$. Along with few additional cases, these steps helps us construct a vector $x$ such that $x^T L_{G'} x < x^T L_{H'} x$ which in turn shows that $x^T L_G x < x^T L_H x$. Hence, $H$ is not spectrally dominated by the graph $G$ if $G$ does not contain a hamiltonian cycle. 

\paragraph{\textbf {Algorithm Overview for SGI.} } 
%Our algorithm for spectral
%isomorphism on trees is theoretically interesting both because of the
%techniques we developed in order to design the algorithm in theorem
%\ref{thm:main} as well as its implications. 
%We next describe our main technical contributions to SGI.
We will now describe our main technical contributions and provide an
overview of our algorithm for SGI.
%
By substituting appropriate vectors $x$ in the definition of condition
number~\eqref{eqn:conditionnumber}, we can easily see that 
%It follows immediately from the definition of condition number
%that 
if two graphs $G$ and $H$ are $\alpha$-spectrally isomorphic under
some permutation $\pi$ so that
$\frac{1}{\alpha} \leq \frac{x^T L_{G} x}{x^T L_{\pi(H)} x} \leq
\alpha$, then $\pi$ also preserves the cuts and effective resistances
between graphs $G$ and $H$ as well. Namely, if $\delta_G(S)$ and
$\delta_{H}(\pi(S))$ are two corresponding cuts in $G$ and $H$, and
$R_G(u,v)$, $R_{H}(\pi(u),\pi(v))$ are the effective resistances
between two corresponding pairs of vertices in $G$ and $H$, then:
%
\begin{equation} \label{eqn:cuts} \frac{1}{\alpha} \leq
  \frac{\delta_G(S)}{\delta_{H}(\pi(S))} \leq \alpha
\end{equation}
%
and 
%
\begin{equation}\label{eqn:effectiveresistances}
  \frac{1}{\alpha} \leq \frac{R_G(u,v)}{R_{H}(\pi(u),\pi(v))} \leq \alpha.
\end{equation}
%
%We will henceforth refer to the minimum $\alpha$ and $\beta$ that
%satisfy \cref{eqn:cuts} and \cref{eqn:effectiveresistances}
%respectively, as the cut distortion and the distance distortion.
 
Our first contribution in this paper is to show that in the case where
$G$ and $H$ are trees, these conditions are also sufficient for $G$
and $H$ to be spectrally isomorphic. Namely, if
%
\begin{equation*}
\frac{1}{\alpha} \leq \frac{\delta_G(S)}{\delta_{H}(\pi(S))} \leq \alpha
\end{equation*}
%
and 
%
\begin{equation*}
\frac{1}{\beta} \leq \frac{R_G(u,v)}{R_{H}(\pi(u),\pi(v))} \leq \beta,
\end{equation*}
%
then
%
\begin{equation*}
  \frac{1}{\alpha\cdot \beta} \leq \frac{x^T L_{G} x}{x^T L_{\pi(H)}
    x} 
  \leq \alpha \cdot \beta.
\end{equation*}
%
One useful property of the trees is that, the effective resistance is
equal to the distance. This means we can replace
\eqref{eqn:effectiveresistances} with the following, arguably more
useful condition:
%
\begin{equation}\label{eqn:distances} \frac{1}{\beta} \leq
  \frac{d_G(u,v)}{d_{H}(\pi(u),\pi(v))} \leq \beta.
\end{equation}
%
Here $d_G, d_H$ are the induced graph distances on graphs $G$ and $H$,
respectively.
%
\begin{remark} The above implication \footnote{Namely that if there is
    a permutation $\pi$ such that both the cuts and effective
    resistances are similar between $G$ and $\pi(H)$ then $G$ and $H$
    are also $\alpha$-spectrally isomorphic, for some $\alpha$ that
    depends quadratically on how close are the cuts and effective
    resistances between the two graphs.} does not hold for general
  graphs. In \Cref{sec:A}, we exhibit two graphs $G$ and $H$ for which
  cuts are within a constant factor and effective resistances
  are within a factor of $O(\log n)$ of each other, yet the
  condition number is at least $\sqrt{n}$.
\end{remark}
%
Our task of finding a spectral isomorphism now gets reduced to finding
a permutation such that both the cuts and distances are similar
between the two graphs. In order to search for the optimal permutation
among $n!$ many possible permutations, we use a rather complicated
dynamic programming argument. We first fix some arbitrary node in $G$
to be the root and guess the corresponding node in $H$. Next we order
the nodes of $G$ with respect to their height. Starting with the
bottom layer of leaves, we maintain a set of ``good'' partial
permutations, which are maps of the corresponding subtree into the
nodes of $H$ while preserving cuts and distances. Unfortunately,
naively keeping track of all such partial permutations and trying to
aggregate them as we move further up in the tree will not work due to
the simple fact that number of permutations might grow
exponentially. Furthermore, the image of current subtree of $G$ we are
looking at, in $H$ need not be a subtree itself -- it might very well
map to an independent set in $H$!

In order to resolve this issue, we augment our partial mappings to
keep track of the mapping of additional nodes {\bf from $H$ to $G$},
which we call ``critical sets''. To put it simply, the critical set of
a subset correspond to the set of (both) endpoints of edges crossing
that subset. However this idea faces some immediate challenges:
%
\begin{enumerate}[(a)]
\item\label{item:diff0} Permutations found for different subtrees may
  not match along their critical sets.
\item\label{item:diff1} The number of choices for critical sets might be
  too large.
\end{enumerate}
%
A simple approach for resolving~\eqref{item:diff1} would be to show
that the size of critical sets is small, which is common in usual
dynamic programming situations. Unfortunately in our case we cannot
bound the size of the critical sets of vertices. In order to resolve
the issues above, we show that, even though we cannot bound the size
of the critical sets of vertices, we can still bound the number of
choices for the critical sets by $O(n^{k^2})$ where $k$ is the
condition number. This lets us get
around the difficulty from~\eqref{item:diff1}. Additionally, we prove
that for any subtree, there exists a set $S$ of polynomially many
permutations such that, for any valid permutation $\pi$ of the
subtree, one of the permutations in $S$ agrees with $\pi$ on the
critical sets of vertices; thus resolving
challenge~\eqref{item:diff0}. Hence, for each subtree we only need to
store polynomially many valid permutations, which allows us to run our
algorithm in polynomial time.



%%% Local Variables:
%%% mode: latex
%%% TeX-master: "Spectral Isomorphism"
%%% End:
