This section outlines a proof for Theorem~\ref{thm:main}. We first introduce necessary notation. 



\begin{definition}
	The support $\sigma(G,H)$ of $G$ by $H$ is the smallest number 
	$\gamma$ such that $\gamma H$ dominates $G$. 
\end{definition}

\begin{definition}
	The condition $\kappa(G,H)$ of a pair of graphs $G$ and $H$
	is the smallest number $\kappa$ such that $G$ and $H$ are $\kappa$-similar. We have $\kappa(G,H) = \sigma(G,H) \sigma(H,G)$.	
\end{definition}

We denote by $d_G(u,v)$ the distance between $u$ and $v$ in $G$ using
the shortest path metric. For $S\subseteq V$, we denote by
$\delta_G(S)$ the set of of edges crossing the cut $(S,V-S)$
in $G$. 
 

We now briefly review well-known facts~\cite{GuatteryM00};
a more detailed version of this paragraph along with proofs can be
found in the appendix of the full version of the paper. 
Given two trees $G$ and $H$ there is an obvious way to embed
the edges of $G$ into $H$: each edge $(u,v)$ is routed over
the unique path between vertices $(u,v)$ in $H$. The \emph{dilation}
of the embedding is defined by: 
$
     {\mathbf d} = \max_{(u,v)\in E_G} d_H(u,v).
$
The congestion $c_e$ of an edge $e\in E_H$ is the number
of $G$-edges that are routed over $e$. The \emph{congestion} ${\mathbf c}$
of the embedding is then defined as $\max_{e\in E_H} c_e$. 
An upper bound of $\kappa$ on the condition number
implies the same upper bound on both $c$ and $d$. On the other hand, the product $\cn \dl$
is an upper bound on $\sigma(G,H)$, which is at most a quadratic
over-estimation of $\sigma(G,H)$. 


Our algorithm finds a mapping  that controls both the dilation and the congestion of the embeddings from $G$ to $H$ and vice versa, thus obtaining a quadratic approximation to the condition number as a corollary.

\smallskip
\noindent \textbf{Remark:} To simplify the presentation and the proof, we assume uniform upper bounds on the congestion and the dilation of both embeddings ($G$ to $H$ and $H$ to $G$), rather than handling them separately. This formally proves a $\kappa^3$-approximation to $\kappa$-similarity. 
We omit a $\kappa$-approximation algorithm for the final version of the paper. 

Formally, our result can be stated as follows:

\begin{theorem}\label{thm:cuts_distances_algo} Suppose $G$ and $H$ are two trees 
  for which there exists a bijective mapping
  $\pi:V(G) \rightarrow V(H)$ satisfying the following properties:
  \begin{itemize}
  \item For all $(u,v) \in E(G)$, $d_H(\pi(u),\pi(v)) \leq \ell$
  \item For all $(u,v) \in E(H)$, $d_G(\pi^{-1}(u),\pi^{-1}(v)) \leq \ell$
  \item For $S \subset V(G)$ such that $|\delta_G(S)| = 1$, $|\delta_H(\pi(S))| \leq k$.
  \item For $S \subset V(H)$ such that $|\delta_H(S)| = 1$, $|\delta_G(\pi^{-1}(S))| \leq k$.
  \end{itemize}
  Then, there exists an algorithm to find such a mapping in time
  $n^{O(k^2d)}$ where $d$ is the maximum degree of a vertex in $G$ or
  $H$.
\end{theorem}

Our main result, theorem \ref{thm:main}, follows immediately as a corollary from the fact that
$\max\{k,l\} \leq \sigma(H,G) \leq kl$ (which is proved in the full version of the paper) and 
using the fact that $\max \{\sigma(G,H), \sigma(H,G) \} \leq \kappa(G,H) \leq \sigma(G,H) \sigma(H,G)$.



%immediately as a corollary
%by %invoking theorem~\ref{thm:cuts-edges-trees}, and 
%
\begin{corollary}
  Given two tree graphs $G$ and $H$ with condition number $\kappa$ and maximum  degree $d$, there exists an algorithm running in time  $n^{O(\kappa^2d)}$ which finds a mapping certifying that condition
  number is at most $\kappa^4$.
\end{corollary}
%
The algorithm uses dynamic programming; it proceeds by recursively finding mappings for different subtrees and merging them. The challenge is to find partial mappings of subtrees which also map their boundaries in such a way that enables different mappings to be appropriately merged. Notice that it is not enough to consider just the boundary vertices of the subtrees and their images. Instead, we need to additionally consider the boundary edges of those vertex sets, which correspond to cuts induced on the graph. 

%Hence, in addition to ``guessing'' the mapping of boundary vertices we also need to ``guess'' the mapping of the cuts induced by the boundary edges of the partially mapped sets. 
\medskip

\noindent \textbf{\large Definitions and Lemmas.} To proceed with the proof, we introduce some definitions. We fix $k$ and $\ell$ to be defined as in Theorem~\ref{thm:cuts_distances_algo}. Also, we fix an arbitrary ordering $L$ on the edges of $H$. Without loss of generality it is convenient to root the trees such that we always map the two roots to each other. Let $r_G$ be the root of tree $G$ and $r_H$ be the root of tree $H$.

Suppose that $u$ is a vertex in $G$ and $T^G_u$ is the subtree rooted at $u$ in $G$. If $T_u^G$ is mapped to the set $T$ in $H$, then its boundary includes the vertex $u$, the edge
incident to $u$, the boundary vertices of $T$, and the cuts induced by
edges going out of $T$. Hence, in addition to considering the mapping
of boundary vertices, we need to consider the mapping of sets $T'$
such that $\delta_H(T') = \{e\}$ where $e \in \delta_H(T)$. This
notion is formalized in the following two definitions. 

\begin{definition}\label{def:gamma}
  Let $\Gamma$ be the set of tuples
  $(u,T,v,u_1,\dots,u_x,S_1,\dots,S_x)$ satisfying the following
  properties:
  \begin{itemize}
  \item
    $u, u_1, \dots, u_x \in V(G), v \in V(H), T \subset V(H),
    S_1, \dots, S_x \subset V(G)$;
  \item $r_G \not \in S_1,\dots,S_x, r_H \not \in T$;
  \item $u,u_1,\dots,u_x \neq r_G, v \neq r_H$;
  \item $|\delta_H(T)| = x \leq k$ and
    $\forall j \in [1,x], |\delta_G(S_j)| \leq k $.
  \end{itemize}
\end{definition}

For $\alpha \in \Gamma$, we use the indicator variable $z_{\alpha}$ to
denote
% we say that $z_{\alpha} = 1$
if there is a mapping $\pi$ which realizes $\alpha$ and preserves the
distances and cuts for edges in $T_u^G$ and $T$. A permutation $\pi$ realizing $\alpha$ is formally defined below. Intuitively, this
mapping maps the subtree rooted at $u$ in $G$ to the set $T$ in $H$,
vertex $u$ to vertex $v$. It also maps $u_1,\dots,u_x$ to the vertex
boundary of the set $T$, and maps sets $S_1,\dots,S_x$ to the cuts
induced by the boundary edges of $T$. The formal definition of
$z_\alpha$ is as follows:
%
\begin{definition}
  %
  \label{def:z_alpha}
  %
  For $\alpha=(u,T,v,u_1,\dots,u_x,S_1,\dots,S_x)\in \Gamma$, let
  $\delta_H(T) = \{e_1,\dots,e_x\}$ be such that for $i<j$, $e_i$ is
  ordered before $e_j$ in ordering $L$. Let $v_j = e_j \cap T$,
  $T_u^G$ be the vertex set in the subtree rooted at $u$ in $G$ and
  for $e \in E(G)$, let $T_e^G$ be the vertex set in the subtree under
  edge $e$. Formally speaking $T_e^G \subset V(G)$ such that
  $\delta_G(T_e^G)= \{e\}$ and $r_G \not \in T_e^G$ ($T_e^H$ is similarly
  defined). We define $z_\alpha= 1$ if there exists a mapping
  $\pi:V(G) \rightarrow V(H)$ such that:
  \begin{enumerate}
  \item \label{item:def-z-alpha-1}
    $\pi(T_u^G) = T, \pi(u) = v, \forall j \in [1,x], \pi(u_j) = v_j,
    \pi(S_j) = T_{e_j}^H$, $\pi(r_G) = r_H$.
  \item \label{item:def-z-alpha-2}
    $\forall (u,v) \in E[G[T_u^G]], d_H(\pi(u),\pi(v)) \leq \ell$
  \item \label{item:def-z-alpha-3}
    $\forall (u,v) \in E[H[T]], d_G(\pi^{-1}(u),\pi^{-1}(v)) \leq
    \ell$
  \item \label{item:def-z-alpha-4}
    $\forall e \in E[G[T_u^G]]$, $|\delta_H(\pi(T_e^G)| \leq k$.
  \item \label{item:def-z-alpha-5}
    $\forall e \in E[H[T]]$, $|\delta_G(\pi^{-1}(T_e^H)| \leq k$.
  \end{enumerate}
  %
  We refer to such a mapping $\pi$ as a certificate of $z_\alpha =
  1$. Moreover, we define $z_{\alpha,\pi} = 1$ if $\pi$ is a
  certificate of $z_{\alpha} = 1$ and $0$ otherwise.
\end{definition}

\begin{claim}\label{prop:z_alpha_pi}
  There exists a $poly(n)$ time algorithm which given
  $\alpha \in \Gamma, \pi:V(G) \rightarrow V(H)$, outputs the value of
  $z_{\alpha,\pi}$.
\end{claim}

Our goal is to design an algorithm which computes $z_{\alpha}$ for
every $\alpha \in \Gamma$. However, for our algorithm to run in
polynomial time, we need $\Gamma$ to not be exponentially large.

\begin{lemma}  $|\Gamma| \leq n^{O(k^2)}$.
\end{lemma}
\begin{proof}
  Let $\alpha=(u,T,v,u_1,\dots,u_x,S_1,\dots,S_x)$. We prove the lemma
  by bounding the number of choices for each parameter.
  %
  \begin{itemize}
  \item The number of choices of $u$ is upper bounded by $n$.
  \item Since $|\delta_H(T)|=x$, the number of choices $\delta_H(T)$
    is upper bounded by ${m \choose x}$ where $m$ is the number of
    edges. By substituting $m = n-1$, we get that the number of
    different $\delta_H(T)$, i.e. the number of different $T$'s, is
    upper bounded by ${n-1 \choose x}$.
  \item The number of different $v$ and $u_j$ is at most $n$ for each
    $j \in [1,x]$.
  \item Similarly to the argument for $T$, the number of different
    $S_j$ with $|\delta_H(S_j)| \leq k$ is at most
    $\sum_{t=1}^k {n-1 \choose t}$.
  \end{itemize}
  %
  For $x \in [1,k]$, the number of different tuples $\alpha$ in
  $\Gamma$ with $|\delta_H(T)|=x$ is at most:
  \begin{equation*}
    n\cdot {n-1 \choose x} n \cdot n^x \cdot \left[\sum_{t=1}^k {n-1
      \choose t}\right]^x 
    %
    = n^{O(k\cdot x)}.
  \end{equation*}
  Since $x \le k$, this gives us an upper bound of $n^{O(k^2)}$ on
  $|\Gamma|$.
\end{proof}


Suppose $\pi$ is the optimal mapping from $G$ to $H$ which yields a
mapping with cut distortion $k$ and distance distortion $\ell$ and
also certifies $z_{\alpha} = 1$ for some $\alpha$. Our recursive
algorithm does not necessarily obtain the same certificate as $\pi$
for $z_{\alpha} = 1$. So, before we show how to compute $z_{\alpha}$,
we examine certain properties of $z_{\alpha}$. In particular, we start
by proving that if both $\pi$ and $\gamma$ certify $z_{\alpha} = 1$ so
that $z_{\alpha,\pi}=z_{\alpha,\gamma} = 1$, then they not only match
on the boundary vertices but also on the cuts induced by boundary
edges.

\begin{lemma}\label{lem:z_alpha_properties}
  For $\alpha = (u,T,v,u_1,\dots,u_x,S_1,\dots,S_x)$, let $\pi$ and
  $\gamma$ be two mappings such that
  $z_{\alpha,\pi}=z_{\alpha,\gamma} = 1$. Then:
  
  \begin{enumerate}

  \item \label{item:z_alpha_1} $\pi(u) = \gamma(u)$.
  
  \item \label{item:z_alpha_2} $\pi(T_u^G) = \gamma(T_u^G)$.
  
  \item \label{item:z_alpha_3}
    For every boundary vertex $w$ of $T$(in $T$ with an incident
    edge in $\delta_H(T)$), $\pi^{-1}(w) =
    \gamma^{-1}(w)$. Equivalently, $\pi(u_j) = \gamma(u_j)$ for
    $j \in [1,x]$.
  \item \label{item:z_alpha_4} For every edge $e \in \delta_H(T)$,
    $\pi^{-1}(T_e^H) = \gamma^{-1}(T_e^H)$.
    
  \item \label{item:z_alpha_5} For every connected component $C$ in
    $H \setminus \delta_H(T)$, $\pi^{-1}(C) = \gamma^{-1}(C)$.
    
  \end{enumerate}  
    
\end{lemma}
%
\begin{proof}
  Items \ref{item:z_alpha_1}-\ref{item:z_alpha_4}.
  follow directly from the definition of $z_{\alpha,\pi}$. Consider a
  connected component $C$ in $H \setminus \delta_H(T)$. Let
  $\delta_H(C) = \{e_{i_1},\dots,e_{i_t}\}$. Without loss of
  generality, assume that $e_{i_1}$ is the edge closest to the root
  $r_H$. Then:
  %
  \begin{equation*}
    \gamma^{-1}(C) = \gamma^{-1}(T_{e_{i_1}}) \setminus \cup_{j=2}^t \gamma^{-1}(T_{e_{i_j}}).
  \end{equation*}
  %
  Item \ref{item:z_alpha_3} implies that
  $\gamma^{-1}(T_{e_{i_j}}) = \pi^{-1}(T_{e_{i_j}})$ for $j \in [1,t]$
  thus proving $\pi^{-1}(C) = \gamma^{-1}(C)$.
\end{proof}

The next lemma is somewhat like a converse of the previous lemma. It shows
that if we have a mapping $\pi$ such that $z_{\alpha,\pi} = 1$ and
another mapping $\gamma$ such that $\gamma$ matches with $\pi$ on the
subtree and the boundary vertices and edges, then
$z_{\alpha,\gamma} = 1$ as well.
%
\begin{lemma}\label{lem:changing_permutation} Let
  $\alpha = (u,T,v,u_1,\dots,u_x,S_1,\dots,S_x)$ and
  $\pi:V(G) \rightarrow V(H)$ be such that $z_{\alpha,\pi} = 1$. Let
  $\gamma:V(G) \rightarrow V(H)$ be such that
  \begin{enumerate}
  \item $\gamma(w) = \pi(w)$ for $w \in T_u^G$
  \item $\gamma(u_j) = \pi(u_j)$ for $j \in [1,x]$
  \item $\gamma(S_j) = \pi(S_j)$ for $j \in [1,x]$ ($\pi$ and $\gamma$
    may not be identical on every element of $S_j$)
  \end{enumerate}
  Then, $z_{\alpha,\gamma} = 1$.
\end{lemma}
%
\begin{proof}
  Follows immediately from
  % $z_{\alpha,\gamma}=1$ can be verified easily by checking
  % each condition of
  definition~\ref{def:z_alpha}.
\end{proof}
%
Next we show how to change the optimal mapping such that it agrees
with the mapping found by our algorithm on the subtree and is still
optimal. Following lemma formalizes this statement:
%
\begin{lemma}\label{lem:combining_two_permutations}
  Let $\pi$ be a mapping such that
  $z_{\alpha,\pi}=z_{\alpha_1,\pi} = 1$ where
  \begin{center}
  $\alpha = (a,T,b,u_1,\dots,u_x,S_1,\dots,S_x) \in \Gamma$ and   $\alpha_1 =
  (a^1,T^1,b^1,u^1_1,\dots,u_{x_1}^1,S_1^1,\dots,S_{x_1}^1) \in
  \Gamma$
  \end{center}
 such that $a_1$ is a child of $a$ in $G$. Suppose,
  $\gamma_1$ is another mapping such that $z_{\alpha_1,\gamma_1} = 1$.
  Let $\zeta$ be defined as follows: $\zeta(u) = \gamma_1(u)$ for
  $u \in T_{a^1}^G$ and $\zeta(u) = \pi(u)$ otherwise. Then,
  $z_{\alpha,\zeta} = 1$.
\end{lemma}
%
\begin{proof}
  Items \ref{item:def-z-alpha-1}-\ref{item:def-z-alpha-4}
  % First four conditions
  of lemma \ref{def:z_alpha} can be easily verified
  for $z_{\alpha,\zeta}$. Here, we prove that the following property
  holds:
  %\begin{equation*}
    $\forall e \in E[H[T]], |\delta_G(\zeta^{-1}(T_e^H))| \leq k.$
  %\end{equation*}
  %
  
  There are three possible cases, that we consider separately. \\
    \noindent \textbf{1.} $e = (b,b_1)$, $e \in \delta_H(T^1)$.  \\
      By definition,    $\zeta^{-1}(T_e^H) = \gamma_1^{-1}(T_e^H)$. Since,
    $z_{\alpha_1,\gamma_1} = 1$, we get
    $|\zeta^{-1}(T_e^H)| = |\gamma_1^{-1}(T_e^H)| \leq k$.
    
    \smallskip
    \noindent \textbf{2.} $e \in E[H[T\setminus T^1]]$. \\ By definition,
    $\zeta^{-1}(T_e^H) = \pi^{-1}(T_e^H)$. Since, $z_{\alpha,\pi} = 1$,,
    we get $|\zeta^{-1}(T_e^H)| = |\pi^{-1}(T_e^H)| \leq k$.
    
    \smallskip
    \noindent \textbf{3.} $e \in E[H[T^1]]$. \\ By definition,
    $\zeta^{-1}(T_e^H) = \gamma_1^{-1}(T_e^H)$ and since,
    $z_{\alpha_1,\gamma_1} = 1$, we get $|\zeta^{-1}(T_e^H)| \leq k$.
\end{proof}
%
The above lemmas show that even if we find mappings for subtrees which
are different from the optimal mappings, they can still be merged with
the optimal mappings. Hence, we may just find any of the mappings for
each $\alpha$ and then recursively combine mappings. The following
lemma states the result and shows how to make it constructive:
%
\begin{lemma}\label{lem:combining_permutation}
  Let $a \in V(G)$ be a vertex in $G$ with children $a^1, \dots,
  a^t$. Let 
  \begin{center}
  	$\alpha = (a,T,b,u_1,\dots,u_x,S_1,\dots,S_x)$ and 
  	  $\alpha_1 = (a^1,T^1,b^1,u_1^1,\dots,u_{x_1}^1,
  	S_1^1,\dots,S_{x_1}^1),\dots,\alpha_t =
  	(a^t,T^t,b^t,u_1^t,\dots,u_{x_t}^t, S_1^t,\dots,S_{x_t}^t)\in
  	\Gamma$.
  \end{center}
  Let $\pi:V(G) \rightarrow V(H)$ be a mapping such that
  $z_{\alpha,\pi}=z_{\alpha_1,\pi}=\dots=z_{\alpha_j,\pi} = 1$ for all
  $j \in [1,t]$.  If for each $j \in [1,t]$ there exists a mapping
   $\gamma_j : V(G) \rightarrow V(H)$ with $z_{\alpha_j,\gamma_j} =1$,
  then there exists $\pi':V(G) \rightarrow V(H)$ such that
  $z_{\alpha,\pi'} = 1$ and $\pi'(w) = \gamma_j(w)$ for
  $w \in T_{a_j}^G$ where $j \in [1,t]$. Moreover, given
  $\{\gamma_j \mid j \in [1,t]\}$, such $\pi'$ can be found in time
  $poly(n)$.
\end{lemma}
\begin{proof}
  Let $\zeta:V(G) \rightarrow V(H)$ such that $\zeta(w) = \gamma_j(w)$
  for $w \in T_{a_j}^G$ where $j \in [1,t]$ and $\zeta(w) = \pi(w)$
  otherwise. By lemma~\ref{lem:combining_two_permutations},
  $z_{\alpha,\zeta} = 1$.

  \noindent {\bf Construction of $\pi'$:} Let $\pi'(w) = \gamma_j(w) $
  for $w \in T_{a_j}^G, j \in [1,t]$ and $\pi'(a) =b$. For
  $w \not\in T_a^G$, define $\pi'$ such that $\pi'(S_j) = T_j$ for
  $j \in [1,x]$.   Setting $\pi = \zeta, \gamma = \pi'$ in lemma~\ref{lem:changing_permutation}, we get $z_{\alpha,\pi'} = 1$.  Easy
  to see that $\pi'$ is constructed in polynomial time.
\end{proof}
%
Lemma \ref{lem:combining_permutation} suggests that we can recursively
compute $z_{\alpha}$. Namely, we can show the following:
%
\begin{lemma}\label{lem:computing_z_alpha}
  There exists an algorithm with running time $poly(n,|\Gamma|^d)$
  which calculates $z_{\alpha}$ for each $\alpha \in
  \Gamma$. Additionally if $z_{\alpha} = 1$, it also computes
  $\pi_{\alpha}$ such that $z_{\alpha,\pi_{\alpha}} = 1$.
\end{lemma}
%
\begin{proof}
  Consider $\alpha = (a,T,b,u_1,\dots,u_x,S_1,\dots,S_x)\in \Gamma$
  with $z_{\alpha} = 1$ and $\pi:V(G) \rightarrow V(H)$ be the mapping
  such that $z_{\alpha,\pi}=1$. Let the children of $a$ be
  $a^1,\dots,a^t$.
  \begin{claim}
    For
    $j \in [1,t],\exists \alpha_j =
    (a^j,T^j,b^j,u_j^1,\dots,u_{x_j}^1, S_1^1,\dots,S_{x_j}^1)\in
    \Gamma$ such that $z_{\alpha_j,\pi} = 1$.
  \end{claim}
  %
  To construct a mapping $\pi'$ such that $z_{\alpha,\pi'} = 1$, we
  guess $\alpha_1,\dots,\alpha_t$ and use lemma
  \ref{lem:combining_permutation} to construct such a mapping. It
  requires mapping $\gamma_j$ such that $z_{\alpha_j,\gamma_j} = 1$
  which can be assumed to be constructed recursively. The number of
  choices of $\alpha_1,\dots,\alpha_t$ is upper bounded by
  $|\Gamma|^t$ which is upper bounded by $|\Gamma|^d$, as the degree
  of any vertex is at most $d$. For any such choice, algorithm in lemma~\ref{lem:combining_permutation} runs in time $poly(n)$. Hence,
  computing $z_{\alpha}$ takes $|\Gamma|^d poly(n)$ time for each
  $\alpha$ and $|\Gamma|^{d+1} poly(n) = poly(n,|\Gamma|^d)$ time for
  all $\alpha \in \Gamma$.

  If $z_{\alpha} = 0$, then for any of the mappings $\pi'$ considered
  above has $z_{\alpha,\pi'} = 0$. This can be checked in $poly(n)$
  time for each $\pi'$ (proposition~\ref{prop:z_alpha_pi}).
\end{proof}

\begin{proof}(of theorem~\ref{thm:cuts_distances_algo}) Let $\pi$ be a mapping
  $\pi: V(G) \rightarrow V(H)$ such that:
  %

  (a) For all $(u,v) \in E(G)$, $d_H(\pi(u),\pi(v)) \leq \ell$
  
  (b) For all $(u,v) \in E(H)$,
    $d_G(\pi^{-1}(u),\pi^{-1}(v)) \leq \ell$
  
  (c) For $S \subset V(G)$ s.t. $|\delta_G(S)| = 1$,
    $|\delta_H(\pi(S))| \leq k$.
  
  (d) For $S \subset V(H)$ s.t. $|\delta_H(S)| = 1$,
    $|\delta_G(\pi^{-1}(S))| \leq k$.

  %
  First, we start by guessing the roots of $G$ and $H$ and define
  $\Gamma$. Then, using  lemma \ref{lem:computing_z_alpha}, we can calculate
  $z_{\alpha}$ for $\alpha \in \Gamma$. It does not give us a mapping
  $\pi'$ satisfying the conditions above since for
  $\alpha = (u,T,v,u_1,\dots,u_x,S_1,\dots,S_x) \in \Gamma$, we have
  $u \neq r_G$. However, a proof almost identical to that of
  lemma~\ref{lem:computing_z_alpha} works here as well. Assume $r_G$ has
  children $a_1,\dots,a_t$. Then there exists
  $\alpha_j = (a^j,T^j,b^j,u_j^1,\dots,u_{x_j}^1,
  S_1^1,\dots,S_{x_j}^1)\in \Gamma$ such that $z_{\alpha_j,\pi} =
  1$. Then, similarly to the proof of lemma~\ref{lem:computing_z_alpha} we
  can guess $\alpha_j,j \in [1,t]$ and compute $\pi'$ in time
  $poly(n)\cdot |\Gamma|^d$, which satisfies the conditions described
  above.
\end{proof}


% \subsection{Proof Overview}
% Algorithm is a dynamic programming algorithm where we find the
% partial mapping from a subtree rooted at a vertex in
% $G$ to some set $T$ in
% $H$. Unlike the case of isomorphic mapping, spectrally close graphs
% do not have the property that $T$ is a subtree in
% $H$. By the cut property, we can prove that boundary of
% $T$ is small and hence, number of choices of
% $T$ is bounded. To perform induction, we however need more
% information from the partial mapping. In addition to mapping of
% $T$, we also need the information about mapping of boundary vertices
% of $T$.
%
% Let $u$ be a node with children $u_1$ and $u_2$ in
% $G$. Given a partial mapping of subtree rooted at $u_1$ and
% $u_2$, to get the mapping for the tree rooted at
% $u$, this can be seen as adding a new vertex and two new edges. And,
% we can simply check condition~\ref{},~\ref{} in polynomial
% time. However, the mapping from $G$ to
% $H$ must also satisfy condition~\ref{}, for which partial mappings
% must contain information about inverse mapping of the boundary
% vertices.
%
%\vcomment{Not good. Need to re-write}

%Let $G$ and $H$ be two trees such that there is a mapping
% $\pi: V(G) \rightarrow V(H)$ which preserves both distances and
% cuts. Idea of the algorithm is very simple. We try and extend the
% dynamic programming algorithm for tree isomorphism problem. Consider
% a vertex $u$ in $G$ with children $a_1,\dots,a_t$. Suppose for each
% $i$, we have found partial mapping from subtree rooted at each
% $a_i$($T_{a_i}^G$) such that for edges in these subtrees,
% property~\ref{} and ~\ref{} is true. Next, algorithm tries to merge
% the mapping of $T_{a_i}^G$ along with the mapping of vertex
% $u$. Unlike the case of exact tree isomorphism, this algorithms runs
% into certain issues:
%\begin{itemize}
%\item Mapping of $T_{a_i}^G$ might not be a subtree of $H$. It may even be disconnected in $H$.
%\item If $G$ and $H$ are isomorphic, then any mapping which for each
%  edgein $G$, preserve the distance and the cuts also preserve cuts
%  and distances in the other direction. However, this is not true
%  about for approximate mappings.
%\end{itemize}
%
%First issue is resolved by observing that since,
% $(T_{a_i}^G, V(G) \setminus T_{a_i}^G)$ is a single edge cut and
% hence, $(\pi(T_{a_i}^G), V(H)\setminus \pi(T_{a_i}^G))$ must be at
% most $k$ which in turn gives us that number of choices of
% $\pi(T_{a_i}^G)$ is at most ${n-1 \choose k}$. To resolve the second
% issue, we find partial mappings which preserves cuts and distances
% in both direction. Moreover, to be able to combine the partial
% mappings of $T_{a_1}^G,\dots,T_{a_t}^G$, we also guess(some
% different word) the mapping of boundary of $T_{a_i}^G$ and
% $\pi_i(T_{a_i}^G)$. Boundary include both the edges cut and the
% vertices incident to these edges. For vertices on the boundary, we
% guess the mapping in the other graph and for edges in the boundary,
% we guess the mapping of the subtree under that edge. It turns out
% that this much information is sufficient to inductively combine the
% partial mappings of $T_{a_i}^G$.

%
%\subsection{Proof of Theorem~\ref{thm:cuts_distances_algo}}
%
%
%Our algorithm basically computes $z_\alpha$ for $\alpha \in \Gamma$
% recursively. Additionally, if $z_\alpha =1$ algorithm would compute
% a mapping $\pi: V(G) \rightarrow V(H)$ which certifies
% $z_\alpha = 1$. To make sense of algorithm, we first prove two
% lemmas. First lemma argues that size of set $\Gamma$ is bounded and
% second says that if recursively, we find mappings certifying
% $z_\alpha = 1$ which are different than optimal mapping, then we can
% still merge these mappings in a consistent way.
%
%
%
% This lemma trivially suggests a dynamic programming algorithm to
% compute $z_\alpha$ for each $\alpha$. Before, we prove the lemma, we
% prove another lemma based on the definition of $z_\alpha$.
%
%\vcomment{ Describe in English about the properties this lemma proved}
%%\begin{proof}
%%  Let $\delta_H(T) = \{e_1,\dots,e_i\}$ and $T_e^H = T_j$. By
%%  definition, $\pi_1^{-1}(T_j) = \pi_2^{-1}(T_j) = S_j$. Next,
%%  consider a connected component $C$ in $H \setminus
%%  \delta_H(T)$. Let $\delta_H(C) = \{e_{i_1},\dots,e_{i_t}\}$. WLOG,
%%  we assume that $e_{i_1}$ is the top edge in the tree $H$. Then,
%%\begin{equation*}
%%\pi_1^{-1}(C) = \pi_1^{-1}(T_{e_{i_1}}) \setminus \cup_{j=2}^t \pi_1^{-1}(T_{e_{i_j}})
%%\end{equation*}
%%By definition, $e_{i_j} \in \delta_H(T)$. Hence, by part 1,
%% $\pi_1^{-1}(T_{e_{i_j}})$ depends only on $\alpha$ which implies
%% that $\pi_1^{-1}(C)$ depends only on $\alpha$. Similarly,
%% $\pi_2^{-1}(C)$ also depends only on $C$. This implies
%% $\pi_1^{-1}(C) = \pi_2^{-1}(C)$.
%%
%%For boundary vertex $w$ of $T$, $\pi_1^{-1}(w) = \pi_2^{-1}(w)$ is true by definition of $z_{\alpha}$.
%\end{proof}

%\vcomment{Following proof would be split into two proofs for two of the lemmas described above}
%
%\begin{proof}(Lemma~\ref{lem:combining_permutation}) Before proving
%  the lemma, we prove that if we define a mapping such that it
%  respects $\gamma_j$ on $T_{a_j}^G$ and $\pi$ on the rest of the
%  vertices, then it certifies $z_{\alpha} =1$. This would certify the
%  existence of $\gamma$ promised in
%  lemma~\ref{lem:combining_permutation}. However, since, we do not
%  know $\pi$, we need additional arguments to make $\zeta$
%  constructive.
%  \begin{lemma}\label{lem:combining_partial_with_optimal}
%    Let $\zeta:V(G) \rightarrow V(H)$ be such that if for some
%    $j \in [1,t], w \in T^G_{a_j}$, $\zeta(w) = \gamma_j(w)$ and
%    $\zeta(w) = \pi(w)$ otherwise. Then, $\zeta$ certifies
%    $z_{\alpha} = 1$.
%\end{lemma}
%\begin{proof} We prove that $\zeta$ certifies that $z_{\alpha} = 1$
%  by proving each of the properties mentioned in
%  definition~\ref{def:z_alpha} one by one.
%
%  \noindent {\bf $\zeta(u) = v$:} For $j \in [1,t]$,
%  $u \not \in T_{a_j}^G$. Hence, $\zeta(u) = \pi(u)=v$ by definition.
%
%  \noindent {\bf $\zeta(T_u^G) = T$:} Since, $\gamma_j$ and $\pi$
%  both certify $z_{\alpha_j} = 1$, we have
%  $\zeta_j(T_{a_j}^G) = \gamma_j(T_{a_j}^G) = \pi(T_{a_j}^G) =
%  T$. Also, $\zeta(u) = \pi(u)$ by definition. By observing that
%  $T_u^G = \{u\} \cup_{j=1}^t T_{a_j}^G$, we get that
%  $\zeta(T_u^G) = \pi(T_u^G) = T$.
%
%  \noindent {\bf $\zeta(u_j) = v_j$:} Any boundary vertex of $T$ is
%  either a boundary vertex of $T^j$ for some $j \in [1,t]$ or the
%  boundary vertex is $u$. By lemma~\ref{lem:z_alpha_properties},
%  $\gamma_j$ and $\pi$ agree on mapping of boundary vertices of $T^j$
%  and hence, $\zeta$ and $\pi$ agree on mapping of boundary vertex of
%  $T^j$. As observed earlier, $\zeta$ and $\pi$ agree on $u$ as
%  well. Hence, $\zeta$ and $\pi$ agree on boundary vertices of $T$.
%
%
%  \noindent {\bf
%  $\forall (x,y) \in E[G[T_u^G]], d_H(\zeta(x),\zeta(y)) \leq k$:} If
%  $x,y \in T_{a_j}^G$ for some $j$, then $\zeta(x) = \gamma_j(x)$ and
%  $\zeta(y) = \gamma_j(y)$. Since, $\zeta$ certifies
%  $z_{\alpha_j} = 1$, we have that
%  $d_H(\gamma_j(x),\gamma_j(y)) \leq k$ and hence, the claimed
%  property is true. Else, lets assume that $x = u, y = a_j$ for some
%  $j$. By definition, $\zeta(a_j) = \gamma_j(a_j) = \pi(a_j)$. Also,
%  by definition $\zeta(u) = \pi(u)$. Hence,
%  $\zeta(x) = \pi(x), \zeta(y) = \pi(y)$. Since, $\pi$ certifies
%  $z_{\alpha} = 1$, we get that $d_H(\zeta(x),\zeta(y)) \leq k$.
%
%
%  \noindent {\bf
%  $\forall (x,y) \in E[H[T]], d_G(\pi^{-1}(u),\pi^{-1}(v)) \leq k$:}
%  If $x,y \in T^j$ for some $j$, then the argument is similar to the
%  above argument. Else, $x = v$ and $y$ is a boundary vertex in
%  $T^j$. By lemma~\ref{lem:z_alpha_properties}
%  $\gamma_j^{-1}(y) = \pi^{-1}(y)$. Hence,
%  $\zeta^{-1}(y) = \pi^{-1}(y)$. Also, by definition,
%  $\zeta^{-1}(v) = \pi^{-1}(v)$. Since, $\pi$ certifies
%  $z_{\alpha} = 1$, we have that
%  $d_H(\pi^{-1}(x),\pi^{-1}(y)) = d_H(\zeta^{-1}(x),\zeta^{-1}(y))
%  \leq k$.
%
%  \noindent {\bf $\forall S' \subset T_u^G$ s.t.
%  $|\delta_G(S')| = 1, \delta_H(\zeta(S')) \leq k$:} Either
%  $S' \subset T_{a_j}^G$ or $S' = T_{a_j}^G$ for some $j \in
%  [1,t]$. By definition, $\zeta(S') = \gamma_j(S')$ and since,
%  $\gamma_j$ certifies $z_{\alpha_j} = 1$, we have that
%  $|\delta_H(\gamma_j(S'))| \leq k$. Hence,
%  $|\delta_H(\zeta(S'))| \leq k$.
%
%
%  \noindent {\bf $\forall e \in E[H[T]], T' \subset V(H)$
%  s.t. $\delta_H(T') = \{e\}$ and $r_H \not \in T'$,
%  $|\delta_G(\pi^{-1}(T')| \leq k$:}
%
% \end{proof}
%
%% \noindent {\bf Running time Analysis:} Lets focus on the inductive
%% step. Assuming that we are given $\pi_i$ for each child $a_i$,
%% algorithm guess $\alpha_i$ for the child and combines them. Since,
%% $\alpha_i \in \Gamma$, \# of touples $(\alpha_1,\dots,\alpha_t)$ is
%% at most $|\Gamma|^t$. Since, $|\Gamma|\leq n^{O(k^2)}$ and
%% $t \leq d$, algorithm runs in time $n^{O(k^2d)}$.
%\end{proof}
%%% Local Variables:
%%% mode: latex
%%% TeX-master: "Spectral Isomorphism"
%%% End:
