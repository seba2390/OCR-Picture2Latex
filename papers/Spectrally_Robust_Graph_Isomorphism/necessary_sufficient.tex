We first describe the combinatorial implications of a bounded condition number. 

\begin{lemma} [cuts] \label{lem:cut-nec}
  Let $G$ and be two graphs such that for all $x$,   
  $\beta \qf(H,x) \leq \qf(G,x) \leq \gamma \qf(H,x)$.
  Then, for any $S\subseteq V$, we have
  $\beta \delta_H(S) \leq \delta_G(S) \leq \gamma \delta_H (S)$.
\end{lemma}
%
\begin{proof}
  Let $x_S$ be the indicator vectors with $x_u=1$ for $u\in S$
  and $x_u=0$ for $u \not \in S$. We have $\qf(G,x_S) = \delta_G(S)$ and similarly for $H$. 
  The proof follows by  substituting  $x_S$ into $x$ in the given inequalities. 
\end{proof}
%


%
\begin{lemma} [distances] \label{lem:dist-nec}%[\cite{doyle-and-snelll}]
  % 
  Let $G$ and $H$ be two trees such that for all $x$,   
  $\beta \qf(H,x) \leq \qf(G,x) \leq \gamma \qf(H,x)$
  Then
  $\beta \le \frac{d_H(u,v)}{d_G(u,v)} \le \gamma$ for any pair of
  nodes $u\neq v \in V$.
  % 
\end{lemma}
%
\begin{remark}
  %
  If we replace the shortest path metric in lemma~\ref{lem:dist-nec} with
  that of the resistance distance (the resistance between two
  equivalent points on an electrical network corresponding to $G$);
  then it holds for any graph~\cite{doylesnell00}.
  %
\end{remark}
%
\begin{proof}
 Assuming that for all $x$ we have $x^T L_G x \leq \gamma x^T L_H x$, 
 it can be shown that we also have  $x^T L_H^{\dagger} x \leq \gamma x^T L_G^{\dagger} x$~\cite{support03boman}, where $L_G^{\dagger}$ is the pseudo-inverse of $L$.  
 It then suffices to prove that $x^T L_G^{\dagger} x = d_G(u, v)$ when $G$ is a
 tree.  Given then a pair $u, v$,   consider the vector $x=[x_a]_{a\in V}$ where $$x_a =
  \begin{cases}
    +1 & \mbox{if $u = a$,} \\
    -1 & \mbox{if $v = a$,} \\
    0 & \mbox{otherwise.}
  \end{cases}$$
  %
  Consider the path $\pi$ from $u$ to $v$ in $G$. For any node $a$, we
  will abuse the notation and use $\pi(a)$ to denote the closest
  ancestor of $a$ along the path $\pi$, i.e., the first node in $\pi$
  one encounters along the walk from $a$ to (say) $u$. Now consider
  the following vector $y=[y_a]_{a\in V}$ where $y_a$ is equal to the
  distance of $\pi(a)$ to $u$.  By construction and the graph being a
  tree, $y^T L y$ counts the number of edges along the path $\pi$.
  Since the vector $x$ we chose is orthogonal to all $1$'s vector,
  whose span is equal to the kernel of $L$, it suffices to prove that
  $x = L y$.
  For any $a\in V$, we have:
  \[
    (L y)_a 
    = \sum_{b \in N(a)} (y_a - y_b)
    =  \sum_{b\in N(a) \cap \pi} (y_a - y_b).
  \]
  There are four cases, three of which are trivial:
  %
  \begin{itemize}
  \item If $a=u$, then this sum is $-1$. 
  \item If $a=v$, then this sum is $+1$. 
  \item If $a \notin \pi$, the sum becomes $0$.
  \item Finally if $a \in \pi\setminus\{u,v\}$, with its successor and
    predecessor along $\pi$ being $s$ and $t$, respectively; then:
    \[
      (L y)_a 
      = (y_a - y_s) + (y_a - y_t)
      = 1 - 1 = 0. 
    \]    
  \end{itemize}
  Therefore $L y = x = e_u - e_v$ as expected.
\end{proof}
%
%\subsection{Sufficient Conditions}
%
It turns out that cuts and distances are also sufficient to
get an upper bound on the support numbers and thus on the condition number~\cite{GuatteryM00}.
% 
\begin{lemma}
  \label{thm:dila-cong-suff}
  Given two graphs $G$ and $H$  if there exist a flow $f$
  in $H$ such that the following  conditions are true:
  \begin{itemize}
  \item For each edge $(u,v) \in E_G$, $f$ routes one unit of flow
    from $u$ to $v$ in $H$ over paths of length at most $\alpha$.
%  \item For each edge $(u,v) \in E_H$, $f_1$ routes one unit of flow
%    from $u$ to $v$ in $G$ over paths of length at most $\alpha$.
  \item Flow $f$ has congestion at most $\beta$ in $H$.
  \end{itemize}
  Then for all $x$, $\qf(G,x) \le \alpha\beta \qf(H,x)$, i.e. $\sigma(G,H) \leq \alpha\beta$.
\end{lemma}
%
\begin{proof}
  First we will prove that if such $f_2$ exists, then
  $L \preceq \alpha\beta M$. For any vector $x\in \mathbb{R}^n$:
  \begin{eqnarray*}
     \qf(G,x)
     & = &
        \sum_{uv\in E(G)}
        (x_u - x_v)^2 
        = \sum_{uv\in E(G)}
        \left[\sum_{ab \in f(uv)} (x_a - x_b)\right]^2 \\
     & \le &
          \sum_{uv\in E(G)}
          |f(uv)| 
          \sum_{ab \in f(uv)} (x_a - x_b)^2 \\
    & \le & \alpha \sum_{ab\in E(H)}
          \left|
          \left\{
          uv \in E(G)
          \big| 
          ab \in f(uv)
          \right\}
          \right|
          (x_a - x_b)^2 \\
        &  \le & \alpha \beta \sum_{ab\in E(H)}
          (x_a - x_b)^2 \\
    & =& \alpha \beta \qf(H,x).
  \end{eqnarray*}
  Here $f(uv)$ denotes the edges along the path
  assigned to the demand pair $u$ and $v$.  	
\end{proof}
%


\begin{theorem}
  \label{thm:cuts-edges-trees}
  Given trees $G$ and $H$ with Laplacian matrices $L$ and
  $M$, respectively on the same set of nodes, let $k$ and $\ell$ be the minimum values for
  which:
  \begin{enumerate}
  \item (Stretch) For any edge $(u,v)$ of $G$, 
   %(resp. $(s,t)$ of    $H$), 
   $d_H(u,v) \le \ell$ %(resp. $d_G(s,t) \le \ell$).
  \item  (Cut) For any edge $(u,v)$ of $H$, 
  %(resp. $(s,t)$ of $H$),
    $\Big|\delta_G\big[T_u^G(v)\big]\Big| \le k$
   % (resp. $\Big|\delta_G\big[T_s^H(t)\big]\Big| \le k$).
  \end{enumerate} 
  %
  Then $\max\{k,\ell\} \le \sigma(G, H) \le k \ell$.
  %
  %
  Here, for two nodes $u$ and $v$ of a tree $G$, $T_u^G(v)$ denotes
  the subtree of $G$ at $v$ when $u$ is identified as the root and
  $\delta_H\big[T_u^G(v)\big]$ denotes the corresponding set of edges
  in $H$ that cross the cut $(A,A')$, where $A$ contains the set of
  nodes of $T_u^G(v)$.
\end{theorem}
%
\begin{proof}
  The lower bound, $\max\{k,\ell\} \le \sigma (G, H)$, follows easily
  from lemmas~\ref{lem:cut-nec} and~\ref{lem:dist-nec}.
  In order to prove the upper bound, we will consider the natural
  multicommodity flow $f$ with demand graph $G$ and capacity
  graph $H$.  For each edge
  $(u,v)$ of $G$, $f$ has a unit flow along the unique path between
  $u$ and $v$ in $H$.   By the stretch condition, $f$ routes flows through
  paths of length at most $\ell$. Now we will bound the congestion.
  Consider any edge $e=(u,v)$ of $H$. Let $A$ be the connected
  component of $H$ containing $u$ after removing $e$.  Observe that
  this is the same as subtree of $H$ at $u$ when $v$ is identified as
  the root, $A = T_v^H(u)$. If $(s,t)$ is an edge of $G$ which sends
  flow across $e$, then $s$ and $t$ should lie in different connected
  components of $G$ after the removal of $e$. If we assume, without
  loss of generality, that $s \in A$; then $t \in \overline{A}$. So
  the congestion of $e$ is equal to the number of edges of $G$
  crossing $A$, $|\delta_H(A)| = |\delta_G[T_u^H(v)]| \le k$. 
   Thus we can invoke \ref{thm:dila-cong-suff} and
  obtain the desired upper bound, $\sigma (G, H) \le k \ell$.
\end{proof}



%%% Local Variables:
%%% mode: latex
%%% TeX-master: "Spectral Isomorphism"
%%% End:
