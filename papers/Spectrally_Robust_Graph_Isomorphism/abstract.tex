\noindent We initiate the study of spectral generalizations of the graph isomorphism problem. 

\smallskip
  \noindent \textbf{(a)} The \emph{Spectral Graph Dominance (SGD)} problem: \\
  On input of two graphs $G$ and $H$ does there exist a permutation
  $\pi$ such that $G\preceq \pi(H)$?
	
\smallskip  	
  \noindent \textbf{(b)} The \emph{Spectrally Robust Graph Isomorphism ($\kappa$-SRGI)} problem: \\
  On input of two graphs $G$ and $H$, find the smallest number $\kappa$ over all permutations $\pi$ such that $  \pi(H) \preceq G\preceq \kappa c \pi(H)$ for some $c$.  SRGI is a natural formulation of the network alignment problem that has various applications, most notably in computational biology. 
  
 
\smallskip  
  
\noindent  $G\preceq c H$ means that for all vectors $x$ we have $x^T L_G x \leq c x^T L_H x$, where $L_G$ is the Laplacian $G$.
  
\smallskip  

We prove NP-hardness for SGD. 
We also present a $\kappa^3$-approximation algorithm for SRGI for the case when both $G$ and $H$ are bounded-degree trees. The algorithm runs in polynomial time when $\kappa$ is a constant. 



  %
  %Moreover, our algorithm finds an optimal permutation that minimizes
  %the distortion (ratio of distances between any pair of nodes) and
  %preserves cuts (ratio of the numbers of edges crossing any subset).
  %
  %We investigate
  %NP-hardness of some closely related problem and consequently show evidence that SGI
  %might be NP-hard even in the case where $G$ or $H$ is an (unbounded
  %degree) tree.
