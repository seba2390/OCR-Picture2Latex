
%\medskip
%\noindent \textbf {Related Work.} 
The Robust Graph Isomorphism problem (RGI) asks for a permutation
that minimizes the number of mismatched edges. 
O'Donnell~{\it et al.}~\cite{odonnell} gave a constant
factor hardness for RGI . The Minimum Distortion problem (MD) views graphs as distance metrics, using
the shortest path metric. The goal is to find a mapping between
the two metrics so as to minimize the maximum
distortion. The connection
between SRGI and MD stems from the observation that if two tree graphs
$G$ and $H$ are $\kappa$-similar up to a permutation
$\pi$, then the distortion between the
induced graph distances of $G$ and $\pi(H)$ is at most $\kappa$. 
For the MD problem, Kenyon {\it et al.}~\cite{KRS} gave an
algorithm which finds a solution with distortion at most $\alpha$
(provided that it exists) in time
$\mathrm{poly} (n) \exp(d^{O(\alpha^3)})$, for a tree of degree
at most $g$ and an arbitrary weighted graph. They also prove that this problem is NP-hard to
approximate within a constant factor.

The term `spectral alignment' has been used before in~\cite{FeiziQMMKJ16} in the context of spectral relaxation of the graph matching function. The algorithm in~\cite{Patrok12} is more spectral `in spirit' because it
uses directly the spectral of the normalized Laplacians of several subgraphs to construct complicated `graph signatures' that are then compared for similarity. There is no underlying objective function that drives 
the computation of these signatures, but we imagine that the proposed algorithm or some variant 
of it, may be a reasonably good practical candidate for SRGI.
The work by Tsourakakis~\cite{Tsourakakis14} proposes an algorithm that searches
for the optimal permutation via a sequence of transpositions; however
the running time of the algorithm does not have any non-trivial sub-exponential
upper bound. 






\begin{comment}
To the best of our knowledge, the
spectral graph isomorphism and graph dominance problems has not been addressed before. Here
we briefly survey previous work on related problems.

Graph isomorphism is a well studied
problem~\cite{luks,GI1,GI2,GI3,GI4} and it was recently shown in a
breakthrough paper of Laszlo Babai that the exact problem can be
solved in quasi-polynomial time.

Two weaker notions of graph isomorphism were considered before. In the
first one, the goal is to minimize the number of mismatched edges. For
this version, O'Donnell~{\it et al.}~\cite{odonnell} gave a constant
factor hardness. In the second version, the graph is treated as a
distance metric (using the shortest path distance) and one tries to
find a mapping between two metrics so as to minimize the maximum
distortion. For this case, Kenyon {\it et al.}~\cite{KRS} gave an
algorithm which finds a solution with distortion at most $\alpha$
(provided that it exists) in time
$\mathrm{poly} (n) \exp(d^{O(\alpha^3)})$, where $d$ is the maximum
degree of the graphs. They also prove that this problem is NP-hard to
approximate within a factor of $\frac{4}{3}$.

The use of condition number, i.e. the ratio of Rayleigh quotients of
associated Laplacian matrices, to measure the similarity of graphs can
be traced back to preconditioning which is prevalent in linear system
solvers and scientific computing~\cite{GuatteryM00}. In a nutshell,
the goal of preconditioners is to approximate one graph with another
much simpler graph such that the condition number is small. To the
best of our knowledge, algorithmically testing whether two graphs are
spectrally close to each other have not been considered before in the
literature.
%\alinote{Yiannis said they thought about this problem with David
%Tolliver. How to handle that here?}
\end{comment}


%%% Local Variables:
%%% mode: latex
%%% TeX-master: "Spectral Isomorphism"
%%% End:
