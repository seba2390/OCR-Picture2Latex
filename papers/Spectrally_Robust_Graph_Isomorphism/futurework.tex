%
%\alinote{I thought it is better to present all questions in paragraphs
 % instead of separate environments. What do you think?}
%

%
From an algebraic standpoint, the problems we considered in this work have natural generalizations to pairs of positive definite matrices $(A,B)$, and the corresponding eigenvalue problem $Ax = \lambda P^T B Px$. SGD generalizes to minimizing the maximum eigenvalue, and SGRI generalizes to finding the permutation $P$ that minimizes the condition number $\kappa(A,B)$. 
But the problem appears to be much harder in some sense: one can construct `pathological' examples of $A$ and $B$ with just two distinct eigenspaces that are nearly identical, but different enough to cause unbounded condition numbers due to the eigenvalue gap. This makes implausible the existence of non-trivial subexponential time algorithms for the general case. 

On the other hand, besides their potential for applications, Laplacians seem to offer an interesting mathematical ground with a wealth of open problems. In this paper we presented the first algorithmic result, for unweighted trees. The algorithm is admittedly complicated, but it can at least be viewed as an indication of algorithmic potential, as we are not aware of any fact that would preclude a $\kappa^2$-approximation for general graphs. To make such algorithmic progress, we would likely have to give up on the combinatorial interpretations of the condition number, and use deeper spectral properties of Laplacians. 

