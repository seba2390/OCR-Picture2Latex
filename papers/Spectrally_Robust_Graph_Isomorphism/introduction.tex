Network alignment, a problem loosely defined as the comparison of graphs under permutations, has a very long history of applications in disparate fields~\cite{Emmert-Streib:2016}. Notably, alignment of protein and other biological networks are among the most recent and popular applications~\cite{Patrok12,FeiziQMMKJ16}. There are several heuristic algorithms for the problem; naturally some of them are based on generalizations of the graph isomorphism problem, mostly including variants of the robust graph isomorphism problem which asks for a permutation that minimizes the number of `mismatched' edges~\cite{odonnell}.

Robust graph isomorphism may not be always an appropriate problem for applications where one wants to certify the `functional' equivalence of two graphs. Consider for example the case when $G$ and $H$ are two random constant-degree expanders. While they can be arguably functionally equivalent (e.g. as information dispersers), all permutations will incur a large number of edge mismatches, deeming the two graphs very unsimilar.
Functional equivalence is of course an application-dependent notion. In the case of protein networks, it is understood that proteins act as electron carriers~\cite{hanukoglu1996}. Thus it is reasonable to model them as electrical resistive networks that are algebraically captured by graph Laplacian matrices~\cite{doylesnell00}. Going back to the graph isomorphism problem, we note the simple fact that the Laplacian matrices of two isomorphic graphs share the same eigenvalues, with the corresponding eigenspaces being identical up to the isomorphism. We can them aim for a \textbf{spectrally robust} version of graph isomorphism (SRGI) which allows for similar eigenvalues and approximately aligned eigenspaces, up to a permutation. 

In lieu of using directly the eigenvalues and eigenspaces to define SRGI, we will rely on the much cleaner notion of spectral graph similarity, which underlies spectral sparsification of graphs, a notion that has been proven extremely fruitful in algorithm design~\cite{Batson2013, Koutis:2012}. 
More concretely, let us introduce the precise notion of similarity
we will be using. 

\begin{definition} [dominance] \label{def:dominance}
	We say that graph $G$ dominates graph $H$ ($G\preceq H$), when 
	for all vectors $x$, we have $x^T L_G x \leq x^T L_H x$, where $L_G$ is the standard Laplacian matrix for $G$.  
\end{definition}	

\begin{definition} [$\kappa$-similarity] \label{def:similarity}
	We say that graphs $G$ and $H$ are $\kappa$-similar, when
	when there exist numbers $\beta$ and $\gamma$, such that $\kappa=\gamma/\beta$ and 
	$
	        \beta H \preceq G \leq \gamma H.\footnote{Graphs are weighted and $cG$ is graph $G$ with its edge weights multiplied by $c$.}
	$
\end{definition}

We are now ready to introduce our main problem. 

\vspace{2ex}
{\SRGI} ($\kappa$-SRGI): Given two graphs $G,H$, does there exist a permutation $\pi$ on $V(G)$ such that $G$ and $\pi(H)$ are $\kappa$-similar?
\vspace{2ex}

It can been shown that this definition does imply approximately equal eigenvalues and aligned eigenspaces~\cite{Koutis-thesis}, thus testing for $\kappa$-similarity under permutations is indeed a spectrally robust version of graph isomorphism. Going back to our example with the two random expanders, it is well-understood that $G$ and $\pi(H)$ will be $\kappa$-similar for a constant $\kappa$ and for all permutations $\pi$, 
which is what we intuitively expect. 


We view spectrally robust graph isomorphism as an interesting theoretical problem due to its close relationship with other fundamental algorithmic questions. In particular, it can be easily seen that $\kappa$-SRGI is equivalent to the graph isomorphism problem when $\kappa=1$. As we will discuss in more detail, SRGI can also be viewed as a natural generalization of the minimum distortion problem~\cite{KRS}. 
Up to our knowledge, the spectral-similarity approach to network alignment has been mentioned earlier only in~\cite{Tsourakakis14}. 
In view of the vast number of works on GI (\cite{GI1,GI2, GI3,
	GI4,GI5,luks,babai} to mention a few) as well as the works on the
robust graph isomorphism problem \cite{odonnell} and the minimum
distortion problem \cite{KRS}, we find it surprising that SRGI
has not received a wider attention.

The goal of this work is to prove some initial results on SRGI and stimulate further research. Towards that end, we provide the first
algorithm for this problem, for the case when both graphs are trees. 

\begin{theorem} \label{thm:main} 
	%
	Given two $\kappa$-similar trees $G$ and $H$ of maximum degree
	$d$, there exists an algorithm running in time $O(n^{O(k^2d)})$ which finds a mapping certifying that the they are at most $\kappa^4$-similar. 
\end{theorem}

The algorithm for trees is already highly involved, which gives grounds for speculating that the problem is NP-hard. We give evidence that this may be indeed true by turning our attention to the one-sided version of the problem. 

\vspace{2ex}
{\GD} (SGD): Given two graphs $G,H$, does there exist a permutation $\pi$ such that $G$ dominates $\pi(H)$? 
\vspace{2ex}

Given two graphs $G$ and $H$ that have the same eigenvalues, it is not hard to prove
that if $G$ and $H$ are not isomorphic, then $G$ cannot dominate $H$ (and vice-versa).
Combining this with the fact that isomorphic graphs have the same eigenvalues, 
we infer that SGD is at least graph isomorphism-hard. The second contribution of this work is the following theorem.

\begin{theorem}\label{thm:np-hardness}
	The \GD problem is NP-hard.
\end{theorem}

Theorem~\ref{thm:np-hardness} is proved in Section~\ref{sec:gd}. We can actually prove a slightly stronger theorem that restricts one of the input graphs to be a tree. 


\subsection{Related Work}
The Robust Graph Isomorphism problem (RGI) asks for a permutation
that minimizes the number of mismatched edges. 
O'Donnell~{\it et al.}~\cite{odonnell} gave a constant
factor hardness for RGI . The Minimum Distortion problem (MD) views graphs as distance metrics, using
the shortest path metric. The goal is to find a mapping between
the two metrics so as to minimize the maximum
distortion. The connection
between SRGI and MD stems from the observation that if two tree graphs
$G$ and $H$ are $\kappa$-similar up to a permutation
$\pi$, then the distortion between the
induced graph distances of $G$ and $\pi(H)$ is at most $\kappa$. 
For the MD problem, Kenyon {\it et al.}~\cite{KRS} gave an
algorithm which finds a solution with distortion at most $\alpha$
(provided that it exists) in time
$\mathrm{poly} (n) \exp(d^{O(\alpha^3)})$, for a tree of degree
at most $g$ and an arbitrary weighted graph. They also prove that this problem is NP-hard to
approximate within a constant factor.

The term `spectral alignment' has been used before in~\cite{FeiziQMMKJ16} in the context of spectral relaxation of the graph matching function. The algorithm in~\cite{Patrok12} is more spectral `in spirit' because it
uses directly the spectral of the normalized Laplacians of several subgraphs to construct complicated `graph signatures' that are then compared for similarity. There is no underlying objective function that drives 
the computation of these signatures, but we imagine that the proposed algorithm or some variant 
of it, may be a reasonably good practical candidate for SRGI.
The work by Tsourakakis~\cite{Tsourakakis14} proposes an algorithm that searches
for the optimal permutation via a sequence of transpositions; however
the running time of the algorithm does not have any non-trivial sub-exponential
upper bound. 



\begin{comment}

 
 


Given two symmetric positive semi-definite $n$ by $n$ matrices $A$ and $B$ with
the same kernel, their relative condition number $\kappa(A,B)$ is
defined to be the ratio of the largest to the smallest eigenvalue of
$AB^{\dagger}$, where $\dagger$ denotes pseudo-inverse (see
e.g. \cite{spielman} for defn). It follows from the definition that $\kappa(A,B)$ is the minimum positive number such that for all $x\in \mathcal{R}^n$:
%
\begin{equation}\label{eqn:conditionnumber}
  \frac{1}{\kappa(A,B)} \leq \frac{x^T A x}{x^T B x} \leq \kappa(A,B).
\end{equation}
%
When $A$ and $B$ are the Laplacian matrices of graphs $G$ and $H$
respectively, we say that $G$ and $H$ are $\alpha$-spectrally close if
the condition number, $\kappa(A,B)$ is $\alpha$. We say that $B$ $\alpha$-\emph{dominates} $A$ if for all $x\in \mathcal{R}^n $ we have: $$\frac{x^T A x}{x^T B x}\leq 1/\alpha$$.
%

%
In this paper we initiate the study of spectral versions of the notorious graph isomorphism problem. We introduce the \emph{spectral graph
  isomorphism (SGI)}: Given two graphs $G,H$ on $n$ nodes, find a
permutation $\pi$ such that, for all $x\in \mathcal{R}^n$:
%
\begin{equation*}
\frac{1}{\alpha} \leq \frac{x^T L_{G} x}{x^T L_{\pi(H)} x} \leq \alpha
\end{equation*}
%
If such permutation exists, we say that $G$ and $H$ are
$\alpha$-spectrally isomorphic.

We also consider a one-sided version of SGI, namely the \emph{graph dominance (GD)} problem: given two graphs $G$ and $H$ as input, find a permutation
  $\pi$ such that, for all $x\in \mathcal{R}^n$: $x^TL_{\pi(H)}x \leq x^T L_{G}x$ or, equivalently, $L_{\pi(H)}\preceq L_{G}$ (where $\preceq$ denotes PSD order). 



We find the spectral graph isomorphism and graph dominance problems theoretically
interesting due to their close relationship to other fundamental
algorithmic questions. In particular, SGI is a natural approximation
version of the well-known graph isomorphism problem
\cite{GI1,GI2,luks,babai}, which is easily seen to be equivalent to SGI when $\alpha =1$. It can also be seen as the spectral analog of robust
graph isomorphism~\cite{odonnell}: On an input of two graphs which are
almost isomorphic in the sense that they only differ in a small amount
of edges, find an``almost-isomorphism''. Moreover, for the special
case of SGI,
% independently interesting which we study in this paper
where both of the input graphs $G$ and $H$ are restricted to be trees,
the spectral graph isomorphism problem can be viewed as a natural
generalization of the minimum distortion (MD) problem, which was
introduced in~\cite{KRS}: On an input of two $n$-point metric spaces,
find a bijection between them with minimum distortion. This connection
between SGI and MD stems from the observation that if two tree graphs
$G$ and $H$ are $\alpha$-spectrally isomorphic, then the distortion between the
induced graph distances of $G$ and $\pi(H)$ is at most $\alpha$.  In
view of the vast number of works on GI (\cite{GI1,GI2, GI3,
  GI4,GI5,luks,babai} to mention a few) as well as the works on the
robust graph isomorphism problem \cite{odonnell} and the minimum
distortion problem \cite{KRS}, we find it surprising that our problem
has not been studied before. 


Moreover, the GD problem can also be seen as a one-sided version of graph isomorphism: two graphs $G$ and $H$ are isomorphic if and only if  $L_G\preceq L_{\pi(H)}$ and $ L_{\pi(H)}\preceq L_G$ for some $\pi$. GD asks for dominance from one-side.


Part of the appeal of spectral graph isomorphism, in contrast to graph
isomorphism, is that it captures the fact that even if two graphs $G$
and $H$ look very different combinatorially, the graphs might be close
under some appropriate metric. If that is the case, then $H$ can be
used as a proxy for $G$ in computations without introducing too much
error. This makes our problem appealing to a variety of practical
applications. We believe that SGI might have large impact in computer
vision, graphics, and machine learning, and in particular to shape
matching and object recognition. There have been a series of works on
the use of graph spectra for shape matching and retrieval in computer
vision \cite{SMD05,SSGD03} and geometry processing \cite{JZ07,
  RWP06}. Eigenvalues of graphs are closely related to almost all
major graph invariants and serve as compact global shape descriptors
and establish a correspondence for computing the similarity distance
between two shapes. Thus if a matrix models the structures of a shape,
either in terms of topology or geometry, then we would expect its set
of eigenvalues (spectrum) to provide an adequate characterization of
the shape. So we believe that our problem of spectral graph
isomorphism is a natural fit for these approaches. Let's look at the
examples of identifying a hand-written number (or any given shape) by
comparing it with a set of stored prototype shapes or matching of
facial features. We propose the following approach which uses our
spectral graph isomorphism problem: The input shape as well as the
prototype shapes will first be modeled by a graph in some canonical
fashion e.g. \cite{levy10,BMP02} \footnote{Typical feature-based
  approaches to these applications first use an edge detector to
  extract a shape’s silhouette or contours, and then represent the
  shape by a sample of points on the detected curve(s)
  \cite{BMP02}. More concretely, the input to these applications is
  modeled as some appropriate mesh. A matrix M which represents a
  discrete linear operator based on the structure of the input mesh is
  constructed, typically as a discretization of some continuous
  operator. This matrix can be seen as incorporating pairwise
  relations between mesh elements and it typically corresponds to the
  graph adjacency or Laplacian operator of a relevant
  graph.}. Subsequently, one will look for a spectral isomorphism
between the given graph and graphs that model the prototype
shapes. Since the spectrum of graphs provides an adequate
characterization of the shape, we would match the input shape to a
prototype shape to which the input shape is spectrally closest to. We
believe that spectral graph isomorphism can also be used in more
sophisticated techniques presented in \cite{SMD05,SSGD03,JZ07, RWP06}.



Our goal in this paper is twofold; to prove some initial results about
the spectral graph isomorphism,  and the graph dominance problem, as well as to stimulate further work
on the area. Towards that end, we make the following formal definitions: 
%




\vspace{2ex}
{\GD}: Given two graphs $G,H$, does there exist a permutation $\pi$ on $V(G)$ such that $L_H \preceq L_{\pi(G)}$?
\vspace{2ex}



\vspace{2ex}
%
%\noindent 
{\CN}: Given two graphs $G(V,E_G)$ and $H(V,E_H)$, find a permutation
$\pi$ of $V$ so as to minimize $\kappa(L_G,L_{\pi(H)})$. 
%
\vspace{2ex}


%
Our first result is a hardness result for the graph dominance problem:


\begin{theorem}\label{thm:np-hardness}
The \GD problem is NP-hard, even if both the input graphs are bounded degree and one the input graphs is a tree.
\end{theorem}

Moreover, our proof shows that there exists a constant $\delta>0$ such that it is also NP-hard to distinguish between the case when there exists $\pi$ such that $L_{\pi(H)} \preceq L_G$ or there is no $\pi$ such that $L_{\pi(H)} \preceq(1+\delta) L_G$. Hence, the optimization version of \GD is NP-hard.

We note that, interestingly, the ``two sided'' version of \GD, which is equivalent to graph isomorphism, is no longer NP-hard \cite{babai}. 
Motivated by the hardness result of \GD, even when one of the graphs is a bounded degree tree, we next consider the spectral graph isomorphism problem in the case where the two input graphs are bounded degree trees.

Our second result is an algorithm which approximates
\CN within a quadratic factor when both of the input graphs are trees
of bounded degree. 
%
\begin{theorem} \label{thm:main} 
  %
  Given two trees $G$ and $H$ with relative condition number (after
  permutation of the vertices of one graph) $k$ and maximum degree
  $d$, there exists an algorithm running in time $O(n^{O(k^2d)})$
  which finds a mapping certifying that the condition number is at
  most $k^2$.
\end{theorem}
%
This special case of SGI when the input graphs are trees is
independently of both theoretical and practical interest. On the
theoretical side, we observe that the results in \cite{KRS} imply that
the following problem, which is closely related to SGI, is NP-hard
even if one of the input graphs is a tree: 
%

\vspace{1ex}
%
{\CD}: (Informal Statement) Given two graphs $G(V,E_G)$ and
$H(V,E_H)$, find a permutation $\pi$ of $V$ which approximates both
the cuts and the distances between $G$ and $\pi(H)$. For a formal
definition of \CD, see \ref{sec:prelim}.
%
\vspace{1ex}
%

%
Interestingly, \CD is the problem we solve in order to design our
spectral graph isomorphism algorithm stated in \ref{thm:main}. This
reduction suggests that SGI might be computationally harder than graph
isomorphism, since it is known that there exists a quasi-polynomial
time algorithm for GI (\cite{babai}). Moreover, even though there is a
simple, polynomial time algorithm for graph isomorphism when the input
graphs are trees~\cite{tree-iso}; our hardness result suggests that
spectral graph isomorphism is likely to be hard even in the case where
at least one of the the input graphs is a tree, similarly to the case for \GD.
%

%
On the practical side, in several instances of the image recognition
and shape matching problems mentioned above, the resulting graph which
models the input shape is a tree graph. For instance, medial-axis
based structures such as the shock graph \cite{skeleton1} and
skeletons of certain objects \cite{skeleton2} are all tree graphs. Our
algorithm for spectral graph isomorphism on trees is a good
algorithmic fit for those applications.

\end{comment}



%%% Local Variables:
%%% mode: latex
%%% TeX-master: "Spectral Isomorphism"
%%% End:
