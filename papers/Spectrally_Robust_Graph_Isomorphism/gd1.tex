\noindent \textbf{Preliminaries}. Given a weighted graph $G=(V,E,w)$ we denote by $E_G$ its edges. The Laplacian $L_G$ of $G$ is the matrix defined by $L(i,j)=-w_{ij}$ and $L(i,i) = \sum_{i\neq j} w_{ij}$. %Because there is a 1-1 correspondence between graphs and Laplacians we will overload notation and use $G$ interchangeably. 
The quadratic form $\qf(G,x)$  of $G$ is the function defined as:
\begin{equation} \label{eq:quadratic}
\qf(G,x) = x^T L_G x = \sum_{i,j} w_{ij}(x_i-x_j)^2.
\end{equation}
Let $G^{\infty}$ be the infinite graph with vertex set equal to all points on the plane with integer coordinates. There is an edge between two points of $G^{\infty}$ if they have Euclidean distance one. A \textit{cubic subgrid} is a finite subgraph of $G^{\infty}$ such that all of its nodes have degree at most $3$. 

%We will prove Theorem~\ref{thm:np-hardness} via a reduction from the following problem . 

The main ingredient of the proof is the following theorem. 

\begin{theorem}\label{thm:hamiltonian_spectral}
Let $G$ be a cubic subgrid and $C$ be the cycle graph, both on $n$ vertices. There exists
a permutation $\pi$ such that $\pi(C) \preceq G$ if and only if $G$ contains a Hamiltonian cycle. 
\end{theorem}
\begin{proof}  If $G$ contains a Hamiltonian cycle $\pi(C)$, then equation~\ref{eq:quadratic} directly implies that $\pi(C)\preceq G$.  To prove the converse  assume that $G$ does not contain a Hamiltonian cycle and let $H$ be a permutation of $C$ such that $|E_G\cap E_H|$ is maximized. We prove a number of claims and lemmas. 
	

\begin{claim}\label{claim:deleting_shared_edges}
Let $G', H'$ be the graphs obtained by deleting the common edges between $G$ and $H$ respectively. Then, $\qf(G,x)< \qf(H,x)$ if and only if $\qf(G',x) < \qf(H',x)$.
\end{claim}

\begin{proof}
Let $F$ be the graph induced by the edges shared by $G$ and $H$. By equation~\ref{eq:quadratic} we have
$\qf(G,x) = \qf(G',x)+ \qf(F,x)$ and $ \qf(H,x) = \qf(H',x)+ \qf(F,x)$. The claim follows.
\end{proof}

\begin{claim}\label{claim:deleting_degree_1_edge}
Let $v$ be a vertex with $deg_{G'}(v)=1$, $deg_{H'}(v)=0$ and let $G''$ be the graph obtained from $G'$ after deleting the edge incident to $v$, and set $H''=H'$. Then, there exists a vector $x$ s.t. $\qf(H',x)>\qf(G',x)$ iff there exists a vector $y$ s.t. $\qf(H'',y)>\qf(G'',y)$.
\end{claim}
\begin{proof}
Let $x$ be a vector such that $\qf(H',x)>\qf(G',x)$. Since $G''$ is a subgraph of $G'$, we have $\qf(G'',x) \leq \qf(G',x) < \qf(H',x) = \qf(H'',x) $, and we can take $y =x$. For the converse, assume that there is a vector $y$ such that $\qf(H'',y)>\qf(G'',y)$. Let $(v,w)$ be the edge incident to $v$ in $G'$. We define a vector $x$ as follows: $x_{u} = y_u$ for all $u \neq v$ and $x_v = y_w$. Since, $deg_{H''}(v) = 0$, we have 
$\qf(H'',y) = \qf(H'',x)$. On the other hand,  $G'$ and $G''$ agree on all the edges except $(v,w)$. Hence, 
$\qf(G',x) = \qf(G'',x) + (x_v - x_w)^2 = \qf(G'',x) $. The two vectors $x$ and $y$ agree on all the entries except at $v$, and the degree of $v$ in $G''$ is zero. Hence, $\qf(G'',x) = \qf(G'',y)$. Combining all the inequalities, we get: \\
$ \qf(G',x)= \qf(G'',x) = \qf(G'',x) < \qf(H'',y) = \qf(H'',x) = \qf(H',x)$.
\end{proof}


\begin{claim}\label{claim:degree_G_H}
Let $G'$ and $H'$ be the graphs obtained by deleting the shared edges between $G$ and $H$ as in
Claim~\ref{claim:deleting_shared_edges}.
Let $\tilde{G}$ and $\tilde{H}$ be the graphs obtained starting from $G'$ and $H'$ and repeatedly
applying the edge deletion operation of Claim~\ref{claim:deleting_degree_1_edge}. Then, for any vertex $u$, $deg_{\tilde{G}}(u) \leq deg_{\tilde{H}}(u) + 1$.  
\end{claim}
\begin{proof}
Since $G$ is a cubic subgrid graph and $H$ is a cycle, $ deg_G(u) \leq 3$, $deg_H(u) = 2$, for all vertices $u$. Deleting edges shared between $G$ and $H$ decreases the degree of any given vertex by the same amount in $G$ and $H$. Moreover, at any given step, we only delete edges from $G'$. Hence, $deg_{\tilde{G}}(u) \leq deg_{\tilde{H}}(u) + 1$. 
\end{proof}

\begin{claim}\label{claim:degree_1_G_H} 
Let $G'$ and $H'$ be the graphs obtained by deleting the shared edges between $G$ and $H$ as in
Claim~\ref{claim:deleting_shared_edges}. If there exists a vertex $v$ such that $deg_{G'}(v) = 1, deg_{H'}(v) \geq 1$. Then, there exists a vector $x$ such that $\qf(H',x) > \qf(G',x)$.
\end{claim}
\begin{proof} Let the edge incident to $v$ in $G'$ be $(v,w)$ and an edge incident to $v$ in $H'$ be $(v,u)$. Since, $H'$ and $G'$ do not share any edge, we have $u \neq w$. Let $x \in \cR^n$ be a vector defined as follows: $x_v = 0, x_w = \frac{1}{2}$ and  $x_t = 1$ otherwise. We have $
\qf(H',x) > (x_v-x_u)^2 = 1,
$
and
\begin{equation*}
\qf(G',x) = (x_v - x_w)^2 + \sum_{(w,a) \in E_G', a \neq v} (x_w-x_a)^2
\end{equation*}
Vertex $w$ has at most two neighbors other than $v$  in $G$, since $deg_{G'}(v) \leq 3$ and for any such neighbor $a$, we have $(x_a-x_w)^2 = (\frac{1}{2}-1)^2 = 1/4$. Hence $\qf(G',x) \leq 3/4 <1 \leq \qf(H',x)$.
\end{proof}

Let $G'$ and $H'$ be the graphs obtained by deleting the shared edges between $G$ and $H$ as in
Claim~\ref{claim:deleting_shared_edges}. Claims~\ref{claim:degree_G_H} and~\ref{claim:degree_1_G_H}
allow us to assume without loss of generality that there is no degree one vertex in $G'$ and for all vertices $u$, $deg_{G'}(u) \leq deg_{H'}(u) + 1$. For 
convenience, we will refer to the edges of $G'$ as \textbf{black edges} and edges of $H'$ as \textbf{blue edges}.

\begin{lemma}\label{lem:w2_degree3}
If there exist five vertices $u,v,w_1,w_2,w_3$ such that 
\begin{itemize}
\item $(u,w_1),(w_1,w_2),(w_2,w_3)$ are black edges and $(v,w_1), (v,w_2)$ are \textbf{not} black edges.
\item $(u,v)$ is a blue edge and $(u,w_2)$ is \textbf{not} a blue edge.
\end{itemize}
Then, there exists a vector $x$ such that $\qf(H',x)> \qf(G',x).$
\end{lemma}
\begin{proof} Let $x$ be the vector with $x_u=0$, $x_v$=2, $x_{w_1}=\frac{1}{3}$ and $x_{w_2}=\frac{2}{3}$, and $x_t=1$ otherwise. 
We have
\begin{eqnarray*}
\qf(H',x)  & \geq & (x_u - x_v)^2 + \sum_{\substack{(u,a) \in E_{H'}\\ a \neq v}} (x_u - x_a)^2 + \sum_{\substack{(v,b)\in E_{H'} \\ b \neq u}} (x_v - x_b)^2\\
& = & 4 + (deg_{H'}(u) - 1)\cdot(0-1)^2  +   (deg_{H'}(v) - 1) \cdot (2-1)^2 \\ 
& = & deg_{H'}(u) + deg_{H'}(v) + 2 \geq deg_{G'}(u) + deg_{G'}(v) \mbox{  \qquad      (Claim~\ref{claim:degree_G_H})}
\end{eqnarray*}
and
\begin{eqnarray*}
\qf(G',x)  & = &  (x_u - x_{w_1})^2 + (x_{w_1} - x_{w_2})^2 + (x_{w_2} - x_{w_3})^2+ \sum_{\substack{(u,a) \in E_{G'}\\ a \neq w_1}} (x_u - x_a)^2 \\
          & &   + \sum_{\substack{(w_1,c) \in E_{G'}  \\ c \neq w_2,u}} (x_{w_1} - x_c)^2 + \sum_{(v,b) \in E_{G'}} (x_v - x_b)^2 + \sum_{\substack{(w_2,d) \in E_{G'}\\ d \neq w_1,w_3}}(x_{w_2} - x_d)^2.
\end{eqnarray*}

We observe that \textbf{(i)}~The first three terms are equal to $\frac{1}{9}$. \textbf{(ii)}~There is at most one edge $(w_1,c)$ for $c\neq w_2,u$. Also, since $w_1$ is not incident to $v$, we have $x_c=1$. Thus the fifth term is at most $\frac{4}{9}$. \textbf{(iii)}  There is at most one edge $(w_2,d)$ for $d\neq w_1,w_3$. In addition, $G'$ is a subgrid, so there is no cycle of length 3 and $w_2$ is not incident to $u$. Also, $w_2$ is not incident to $v$, by assumption. So, it must be that $x_d=1$ and the last term is at most equal to $\frac{1}{9}$. \textbf{(iv)}~Since $G'$ and $H'$ do not share an edge, $u$ is not connected to $v$. By assumption $u$ is also not incident to $w_2$. So, it must be that $x_a=1$ and the fourth term is equal to $deg_{G'}(u)-1$. \textbf{(v)}~Vertex $v$ is not connected to $u, w_1, w_2$. Thus it must be $x_b=1$ and the sixth term is equal to $deg_{G'}(v)$. 
Collecting the terms gives $\qf(G
,x)\leq deg_{G'}(u)+ deg_{G'}(v) -\frac{1}{9}$ and the Lemma follows. 
\end{proof}
\begin{lemma}\label{lem:w2_degree2}
If there exist four different vertices $u,v,w_1,w_2$ such that 
\begin{itemize}
\item $w_1$ has only two black adjacent edges $(u,w_1)$ and $(w_1,w_2)$ 
\item $(u,v)$ is a blue edge.
\end{itemize}
Then, there exists a vector $x$ such that $\qf(H',x)> \qf(G',x).$
\end{lemma}
\begin{proof}
Let $x$ be a vector with $ x_u = 0, x_v = 2, x_{w_1} = \frac{1}{2}$ and $x_t=1$ otherwise. 
We have
\begin{eqnarray*}
\qf(H',x)  & \geq & (x_u - x_v)^2 + \sum_{\substack{(u,a) \in E_{H'}\\ a \neq v}} (x_u - x_a)^2 + 
\sum_{\substack{(v,b)\in E_{H'}\\ b \neq u}} (x_v - x_b)^2\\
& = & 4 + (deg_{H'}(u) - 1) \cdot (0-1)^2 + (deg_{H'}(v) - 1) \cdot (2-1)^2   \mbox{\qquad (no shared edges)}\\
& = & deg_{H'}(u) + deg_{H'}(v) + 2 \geq deg_{G'}(u) + deg_{G'}(v) \mbox{\qquad (Claim~\ref{claim:degree_G_H})}
\end{eqnarray*}
and
\begin{eqnarray*}
\qf(G',x) & = &  (x_u - x_{w_1})^2 + (x_{w_1} - x_{w_2})^2 + \sum_{\substack{(u,a) \in E_{G'}\\ a \neq w_1}} (x_u - x_a)^2 + \sum_{(v,b) \in E_{G'}} (x_v - x_b)^2 
\end{eqnarray*}

Since $G'$ and $H'$ do not share an edge, $(x_u - x_a)^2$ and $(x_v-x_b)^2$ terms are $(0-1)^2$ and $(2-1)^2$ respectively. We have
\begin{eqnarray*}
\qf(G',x) &=& \frac{1}{4} + \frac{1}{4} + (deg_{G'}(u) - 1) \cdot 1 + deg_{G'}(v) \cdot 1 = deg_{G'}(u) + deg_{G'}(v) -\frac{1}{2}.
\end{eqnarray*}
The Lemma follows. 
\end{proof}

\begin{lemma}\label{lem:vertex_degree_3}
If there exists a degree three vertex in $G'$, then there exists a vector $x$ such that $\qf(H',x)> \qf(G',x).$ 
\end{lemma}
\begin{proof}
Since $deg_{G'}(u) = 3, deg_{H'}(u) \geq 2$ by claim~\ref{claim:degree_G_H}. Consider the underlying grid of which $G$ is a subgraph. Pick the edge $(u,v) \in E_{H^{'}}$ which is either not axis-parallel or is axis-parallel and $v$ is at distance at least $2$ in the grid. Since $u$ has degree $3$ in $G'$, there exists a neighbor $w_1$ of $u$ in $G'$ such that any path from $w_1$ to $v$ in $G'$ has length at least $3$. 

If $w_1$ has degree $2$ in $G'$, then we set $w_2$ to be the neighbor of $w_1$ other than $u$. It is then straightforward to check that $u,v,w_1,w_2$ satisfy the condition of Lemma~\ref{lem:w2_degree2}. Hence, there exists $x$ such that $\qf(H',x)> \qf(G',x).$

If $w_1$ is not of degree $2$ in $G'$, it must have degree $3$ since there are no degree $1$ vertices in $G'$. Let $w_2$ be the neighbor of $w_1$ other than $u$ such that $(u,w_2) \not \in E_H'$. Such a neighbor exists since $u$ has at most one neighbor in $H'$ other than $v$ and $v$ is not incident to $w_1$ due to the fact that any path of length from $w_1$ to $v$ has length at least $3$. Let $w_3$ be the neighbor of $w_2$ other than $w_1$. Such a neighbor must exist since there is no vertex of degree $1$ in $G'$. Now, we prove that these vertices satisfy the condition of Lemma~\ref{lem:w2_degree3}. By construction $(u,w_1),(w_1,w_2),(w_2,w_3)$ are black eges. Any path from $w_1$ to $v$ in $G'$ has length at least $3$ which implies that $(w_1,v),(w_2,v)$ are not black edges. Also, by construction, $(u,v)$ is a blue edge and $(u,w_2)$ is not a blue edge. Hence, by lemma~\ref{lem:w2_degree3} there exists $x$ such that $\qf(H',x)> \qf(G',x).$ 
\end{proof}

\begin{lemma}\label{lem:cut_vector}
If there exists a set $S$ of vertices such that no edges leave $S$ in $G'$, but at least one edge leaves $S$ in $H'$ then then there exists a vector $x \in \cR^n$ such that $\qf(H',x)> \qf(G',x).$
\end{lemma}
\begin{proof}
Let $x$ be defined as follows: $x_u = 1$ for $u \in S$ and $x_u = 0$ for $u \not \in S$. The $\qf(G',x)$ is equal to the number of edges leaving $S$ in $G'$, and similarly for $H'$. The lemma follows. 
\end{proof}

\begin{lemma}\label{lem:cycle_length_5}
If there exists a cycle of length more than 4 in $G'$ such that all vertices of cycle have degree $2$ in $G'$, then there exists $x \in \cR^n$ such that $\qf(H',x)> \qf(G',x).$
\end{lemma}
\begin{proof}
Let $C$ be the set of vertices in the cycle. For any vertex $v$ in $C$, $deg_{G'}(v) = 2$. By claim~\ref{claim:degree_G_H}, $deg_{H'}(v) \geq 1$. If there is no blue edge connecting two vertices of $C$ in $G'$, then there are at least $|C|$ edges going out of $C$ in $H'$ and no edge going out of $C$ in $G'$. Then, by lemma~\ref{lem:cut_vector}, there exists $x$ such that $\qf(H',x)> \qf(G',x)$. 

In the complementary case, suppose there is an edge $(a,b) \in E_{H'}$ such that $a,b \in C$. Let the two paths from $a$ to $b$ on $C$ be $P_1 = a,w_1,\dots,w_{k_1},b$ and $P_2 = a,v_1,\dots,v_{k_2},b$. Since $G'$ and $H'$ do not share any edge, $\min(k_1,k_2) \geq 1$. And since the cycle has length at least $5$, $\max(k_1,k_2) \geq 2$. Let a vector $x$ be defined as follows: $x_{a} = 0, x_b = 1, x_{w_i} = \frac{i}{k_1+1}, x_{v_i} = \frac{i}{k_2+1}$ and for $u \not \in C$, set $x_u = 0$. We have

\begin{eqnarray*}
\qf(G',x) & = & \sum_{(u,v) \in P_1} (x_u - x_v)^2 + \sum_{(u,v) \in P_2} (x_u-x_v)^2 \\
& = & (k_1+1)\cdot \frac{1}{(k_1+1)^2} + (k_2+1)\cdot \frac{1}{(k_2+1)^2} \leq  \frac{5}{6}.
%& = & \frac{1}{k_1+1} + \frac{1}{k_2+1} \leq \frac{1}{2} + \frac{1}{3} = \frac{5}{6}.
\end{eqnarray*}
The last inequality holds because $\min(k_1,k_2) \geq 1, \max(k_1,k_2) \geq 2$. \\ On the other hand
$\qf(H',x) \geq (x_a-x_b)^2 =1 $ and the lemma follows. 
\end{proof}


%We prove that for such an instance, we can re-route the edges of the cycle such that all the blue edges match the black edges. This would imply that $G$ contains a Hamiltonian cycle.

\begin{lemma}\label{lem:cycles_length_4}
If $G'$ contains a set of disjoint cycles of length $4$ and $H'$ edges only have endpoints on the same cycle, then there is a cycle $\tilde{H}$ such that $|E_{\tilde{H}} \cap E_G| = |E_H \cap E_G|+2$.
\end{lemma}
\begin{proof}
 Consider a cycle $C$ of length $4$ in $G'$ such that there is no blue edge between a vertex in the cycle and a vertex not in the cycle. Since $deg_{G'}(v) \geq 2$ for all vertices in the cycle, $deg_{H'}(v) \geq 1$ by~\cref{claim:degree_G_H}. Since $G'$ and $H'$ do not share any edge and the cycle has length $4$, we must have $deg_{H'}(v) = 1$ for all $v \in C$. And for vertices not in the length $4$ cycles, $deg_{H'}(v) = 0$. Let the edges of $H$ be $F_1 \cup F_2$ where $F_1$ are the edges shared between $G$ and $H$ and $F_2$ are the diagonal edges in the disjoint cycles of length $4$. Let $C$ be one such cycle in $G'$ with vertices $v_1,v_2,v_3,v_4$ in this order and $(v_1,v_3) \in F_2, (v_2,v_4) \in F_2$. Let $H_1 = (V, F_1 \cup F_2 \setminus \{(v_1,v_3), (v_2,v_4)\}\cup \{(v_1,v_2),(v_3,v_4)\}), H_2 = (V,F_1 \cup F_2 \setminus \{(v_1,v_3), (v_2,v_4)\}\cup \{(v_1,v_4),(v_2,v_3)\})$. Then, one of the $H_1$ or $H_2$ is a cycle of length $n$. We let $\tilde{H}$ be that cycle. 
\end{proof}

\noindent {\bf Finishing the proof:}  Recall that we have assumed that $G$ does not contain a Hamiltonian cycle and let $H$ is a permutation of $C$ such that $|E_G\cap E_H|$ is maximized. To show that $G$ does \textbf{not} dominate $H$, we need to construct a vector $x$ such that $\qf(H,x)>\qf(G,x)$. 

Starting with $G$ and $H$, we form two graphs $G'$ and $H'$ as follows: (i) delete from $G$ and $H$ all common edges, (ii) iteratively and greedily delete all vertices such that $deg_{G}(u) = 1, deg_{H}(u) = 0$. Then Claims~\ref{claim:deleting_shared_edges} and \ref{claim:deleting_degree_1_edge} show that it suffices to find a vector $x$ such that $\qf(H',x)>\qf(G',x)$. 

If there is still a vertex with $deg_{G'}(u) = 1$, then $deg_{H'}(u)$ must be at least $1$ and hence, by Claim~\ref{claim:degree_1_G_H}, there exists a vector $x$ such that $\qf(H',x) > \qf(G',x)$. 
Also, if there is a vertex $u$ with degree $3$ in $G'$, then by lemma~\ref{lem:vertex_degree_3} there exists $x$ such that $\qf(H',x)>\qf(G',x)$.

If there are no degree $1$ or degree $3$ vertices $G'$, then $G'$ must be a collection of isolated vertices and cycles. If there is a vertex $v$ such that $deg_{G'}(v) = 0, deg_{H'}(v) \geq 1$, then by setting $S = \{v\}$, lemma~\ref{lem:cut_vector} implies that there exists a vector $x$ such that $\qf(H',x)>\qf(G',x)$. 


So, if none of the above cases occurs, then $G'$ is a collection of disjoint cycles and $H'$ edges are only incident to vertices of the cycles. If there is a cycle of length at least $5$, then by lemma~\ref{lem:cycle_length_5} there exists $x$ such that $\qf(H',x)>\qf(G',x)$. Otherwise, if there is at least one blue edge with end points on two different cycles of length $4$, then by setting $S$ to be the vertex set of the cycle of length $4$ Lemma~\ref{lem:cut_vector} implies that there exists $x$ such that $\qf(H',x)>\qf(G',x)$. 

So, either $G'$ and $H'$ are empty or $G'$ consists of a collection of disjoint cycles of length $4$ such that blue edges have end points in the same cycle. In the first case, $G$ trivially contains a Hamiltonian cycle since $H'$ is empty. This is a contradiction to the assumption that $G$ does not contain a Hamiltonian cycle. 
In the second case $G$ Lemma~\ref{lem:cycles_length_4} contradicts our assumption about the maximality
of $|E_G\cap E_H|$. 
\end{proof}


\begin{proof}(Theorem~\ref{thm:np-hardness})
The problem of detecting if a cubic subgrid contains a Hamiltonian cycle is NP-complete~\cite{PapadimitriouV84}.
Hence Theorem~\ref{thm:hamiltonian_spectral}  is a direct reduction, and the theorem follows. 
\end{proof}




\medskip