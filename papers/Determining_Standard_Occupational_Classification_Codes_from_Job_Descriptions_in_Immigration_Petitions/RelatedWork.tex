\section{Related Work}
\label{sec:RelatedWork}

Machine learning methods have been applied in the past to various problems in the legal domain \cite{Surden2014,10.1007/978-3-030-19823-7_31,Faggella2020}, such as outcome forecasting \cite{10.2307/4099370,10.2307/3688543,Katz2016,Aletras2016PredictingJD,Medvedeva2019UsingML}, document discovery \cite{DBLP:conf/icail/YangGFY17,Cormack2014EvaluationOM}, document categorization \cite{Lemley2007,DBLP:conf/bigdataconf/WeiQYZ18,Silva2018DocumentTC,DBLP:conf/fedcsis/UndaviaMO18,DBLP:conf/cikm/LuCAK11,DBLP:conf/propor/FurquimL12,Kumar2012}
and legal drafting \cite{DBLP:conf/afips/SprowlBCEK84,Betts2017,Miller}.

Application of machine learning methods to immigration law is a much newer area of research. The problem of predicting outcomes of refugee claims has been considered in \cite{DBLP:conf/icail/DunnSSC17,DBLP:conf/icail/ChenE17}. In contrast, our paper focuses on work visa applications as opposed to refugee claims, and seeks to programmatically select SOC codes as opposed to predicting case outcomes.

In \cite{DBLP:conf/icdm/MukherjeeODJAWH20}, two problems related to work visa applications are considered, namely, categorization of supporting documents of visa petitions, and drafting responses in reaction to Requests For Evidence (RFE). Our work is different from \cite{DBLP:conf/icdm/MukherjeeODJAWH20} in that we focus on identifying SOC codes programmatically in an effort to proactively reduce the chances of RFE issuance.

Interestingly, application of natural language processing to determine SOC code has been studied in the epidemiological context. Specifically, the SOCcer (Standardized Occupation Coding for Computer-assisted Epidemiologic Research) model \cite{pmid27102331} predicts SOC code based on industry, job title, and job tasks. Our work is different from SOCcer in the following ways. First, while SOCcer is trained and evaluated using health-related datasets, we focus on data related to work visa petitions. Second, while SOCcer uses an ensemble of classifiers, three of which are based on job title, one on industry, and one on task, we seek to predict SOC code using description (i.e., tasks and responsibilities) alone. This is due to our observation that in work visa related data, job titles do not map to SOC codes in a consistent way and that the number of distinct SOC codes associated with an industry such as the software industry is huge. Third, unlike SOCcer, we benchmark a broad variety of models and compare them in terms of accuracy and training time. Finally, our benchmarking includes two different text vectorization approaches, namely sparse vectorization (using TF-IDF $n$-grams  \cite{DBLP:conf/ecml/Joachims98}) and dense vectorization (using doc2vec \cite{DBLP:conf/icml/LeM14}, a neural network).     

The next section describes our approach in detail.









