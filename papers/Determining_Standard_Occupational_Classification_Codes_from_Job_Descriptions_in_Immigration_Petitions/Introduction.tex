\section{Introduction}
\label{sec:Introduction}

The process of obtaining U.S.~work visas has become increasingly difficult in recent years. Data from the U.S.~Citizenship and Immigration Services (USCIS) \cite{uscis-rfe-stats} show that between Fiscal Year (FY) 2015 and FY 2019, the approval rate of H-1B visa petitions has declined substantially. In response to some (though not all) visa petitions, the USCIS requests further information by issuing a Request For Evidence (RFE) \cite{RFE}. Per \cite{uscis-rfe-stats}, the percentage of petitions that resulted in RFEs has increased dramatically between FY 2015 and FY 2019, reaching 40.2\% in 2019. Further, the percentage of petitions with RFE that were approved has decreased significantly during this time frame, reaching a low of 62.4\% in FY 2018. In FY 2020, approval rates showed some improvement but were still lower than those in FY 2015; similarly, RFE issuance rate somewhat decreased but was still higher than that in FY 2015 \cite{bal-rfe-stats}. We note that issuance of RFE, petitioner's response, and subsequent review of response by the USCIS, add significant delays even for petitions that are ultimately approved. Per the USCIS, the most common reason for RFE issuance is the petitioner (employer) not being able to establish that the position is a \textit{specialty occupation} \cite{RFE}  (please refer to \cite{H1B} and \cite{SpecialtyOccupation} for explanations of this term). Therefore, accurate characterization of job positions is critical to the successful and timely completion of the visa petitions.

The U.S.~Bureau of Labor Statistics (BLS) has created the Standard Occupational Classification (SOC) \cite{SOC-1} to categorize jobs into 867 occupational categories. Each category is denoted by an SOC code. The mapping of SOC codes to categories is given in \cite{SOC-2}. To minimize the chances of a visa petition getting delayed (due to RFE) or denied, it is important that the petitioner specify the SOC code that best describes the duties associated with the position. Typically, this requires careful reading of the job description and SOC code definitions to find the best match. Although SOC codes are organized hierarchically to facilitate search, the process is tedious, and results in enormous repetitive workload for immigration law firms.

In this paper, we focus on the problem of algorithmically determining SOC codes based on job descriptions. Applying techniques from natural language processing (NLP), we build a variety of predictive models that accept free form textual descriptions of job duties as input, and yield SOC code as output. Using real world data, we empirically evaluate these models with respect to quality of prediction and training time, and identify models that are best suited for this task.

The rest of this paper is organized as follows. Section \ref{sec:RelatedWork} reviews related work. In Section \ref{sec:Methodology}, we describe our approach. Section \ref{sec:Evaluation} presents our experimental evaluation which are interpreted in Section \ref{sec:Discussion}. Finally, Section\ref{sec:Conclusion} summarizes the paper and discusses future directions.


