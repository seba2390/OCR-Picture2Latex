\documentclass[twocolumn,secnumarabic,amssymb, nobibnotes, aps, prd, nofootinbib]{revtex4-2}
%\usepackage{acrofont}%NOTE: Comment out this line for the release version!
\newcommand{\revtex}{REV\TeX\ }
\newcommand{\classoption}[1]{\texttt{#1}}
\newcommand{\macro}[1]{\texttt{\textbackslash#1}}
\newcommand{\m}[1]{\macro{#1}}
\newcommand{\env}[1]{\texttt{#1}}
\setlength{\textheight}{9.5in}
\usepackage[T1]{fontenc}
\usepackage[pdftex]{graphicx,color}
\usepackage{hyperref}
\usepackage{subfigure}
\usepackage{enumerate}
\usepackage[all]{hypcap} %ref to the top of the figure

\hypersetup{pdfstartview={XYZ null null 0.99}, bookmarksnumbered=true, bookmarksopen=true, colorlinks=true, linkcolor=black, citecolor=black, urlcolor=black, filecolor=black, anchorcolor=black, runcolor=black}

\begin{document}

\title{Promoting the transition to quantum thinking: development of a secondary school course for addressing knowledge revision, organization, and epistemological challenges}


\author{Giacomo Zuccarini}%
\email{giacomo.zuccarini@unipv.it}
\affiliation{Department of Physics, University of Pavia, Via Bassi 6, Pavia 27100, Italy}
\author{Marisa Michelini}%
\affiliation{Department of Mathematics, Computer Science and Physics, University of Udine, via delle scienze 206,
33100 Udine, Italy}
\date{\today}


\begin{abstract}
We describe the development of a course of quantum mechanics for secondary school designed to address the challenges related to the revision of classical knowledge, to the building of a well-organized knowledge structure on the discipline, and to the development of a plausible and reliable picture of the quantum world. The course is based on a systemic approach to conceptual change, which relies on its analysis in the transition from classical to quantum mechanics, and coordinates cognitive and epistemic aspects. We show how our approach drives the derivation of design principles, how these principles guide the development of the instructional sequence and of its strategies, how their implementation requires the blending of different research perspectives and learning systems. The first challenge is addressed through a path of revision of classical concepts and constructs which leverages prior knowledge according to the dynamics of each notion in theory change. The second by adopting a framework that promotes the construction of a unifying picture of quantum measurement across contexts. The third by designing the course around a modelling process that engages students in epistemic practices of the theoretical physicist, such as generating and/or running thought experiments, and mathematical modelling in a purely theoretical setting. All is aimed to help students accept the quantum description of the world as a plausible product of their own inquiry. This process is assisted by the discussion of the facets of the foundational debate that are triggered by each of the suggested interpretive choices, with the goal to promote an awareness of its cultural significance, of the limits the chosen stance, of the open issues. Data on the cycles of refinement illustrate how a set of activities have been made effective in addressing the challenges at a local level, and the process by which the learning outcomes informed their development.


\end{abstract}

\maketitle

\section{Introduction} \label{Sec:Intro}
Research on the teaching and the learning of quantum mechanics (QM) holds a special position in physics education and science education at large, since it is at the crossroads of general research threads and key topics in the field.

First of all, both students learning introductory science topics and students learning QM face a substantial challenge in achieving an effective knowledge revision. In theory change from classical mechanics (CM) to QM, basic classical terms, such as `measurement' and `state', undergo a shift in meaning. Students struggle to interpret the properties of their quantum counterparts, as reported by research conducted at different educational levels. Investigations on upper division students elicited several issues with the new features of ideal quantum measurement \cite{Zhu2012}; at a sophomore level, the interpretation of its probabilistic character, and as a result, of quantum uncertainty has been recognized as a major challenge to students \cite{Ayene2011};  in the context of photon polarization, research revealed difficulties to interpret the concept of quantum state, identified by secondary school students as a physical quantity \cite{Pospiech2021}. The impossibility to visualize quantum systems and the nonintuitive nature of the new versions of the concepts represent an educational bottleneck that can be overcome with the support of mathematical sense-making. However, also familiar constructs such as `vector' and `vector superposition' change both in properties and representational role \cite{Pospiech2021}. Not surprisingly, students struggle to develop a consistent physical interpretation of the quantum version of these constructs: even at the beginning of graduate instruction, they have difficulties to identify the referent of vector superposition in QM, as they tend to associate it with mixed states, which can be described classically as lack of knowledge about the state of the system \cite{Passante2015}.

Another challenge faced both by introductory science students and physics majors enrolled in a QM course is the difficulty to overcome knowledge fragmentation, i.e., the organization of knowledge in small, disconnected pieces of contextual nature, which can be productively applied locally but which lack global coherence. This concerns, in the two cases, student interpretation of natural phenomena \cite{Vosniadou2008, diSessa2014}, and their knowledge of the quantum model \cite{Johnston1998}. Research conducted at the end of upper-division QM courses and at the beginning of graduate instruction suggests that student reasoning is strongly context-dependent \cite{Marshman2015}, and therefore that the development of a globally consistent knowledge structure may be only halfway even after prolonged periods of instruction. So far as we know, no investigation of this issue is available on secondary school students and non-physics-or-engineering majors who received traditional instruction on QM. However, the more limited scope of teaching/learning sequences (TLSs) designed for such student populations enhances the risk of promoting the construction of disconnected models valid only in the context of an individual phenomenon or experiment \cite{Malgieri2017}.

A challenge specifically related to the learning of QM is due to the controversial character of its scientific epistemology: the nature of the systems described by the mathematical formalism, the completeness or not of the information we can get on them, and the explanation of observations in the lab depend on the chosen interpretive stance. The traditional presentation of the theory comes with a seemingly counterintuitive picture of the world, which requires students to revise or renounce very basic tenets about nature such as the well-defined position of physical objects \cite[e.g.,][]{Griffiths2018}. Research indicates that an elementary understanding of the core concepts of the theory does not necessarily imply the acceptance of QM as a personally convincing description of physical reality \cite{Ravaioli2017}. In order to be accepted, the quantum model needs to be perceived by students as plausible and reliable.

Overall, a major goal of physics education research (PER) on QM is helping students overcome the manyfold  challenges involved in learning QM. For this purpose, the PER community has produced in recent years a number of instructional materials and TLSs, drawing on different educational approaches. For instance, C. Singh and her coworkers built on their own research on common difficulties to design and revise interactive tutorials (QuILTs), with the aim to promote the construction of schemas consistent with QM principles (e.g., \cite{Marshman2017}). Favoring the development of an integrated model has been a basic aim of Malgieri \textit{et al.}, who implemented Feynman's sum-over-paths approach by using GeoGebra simulations, so as to allow secondary school students to analyze different experimental setups with the same conceptual tools \cite{Malgieri2017}.  The TLS of Wittmann and Morgan for nonscience majors places special emphasis on personal epistemology (not to be confounded with scientific epistemology) as a means to help students work with nonintuitive content and to strengthen their understanding of scientific modelling \cite{Wittmann2020}. Baily and Finkelstein made of the controversial nature of the interpretation of QM and of the discussion of students' beliefs about it a topic onto itself, by designing a modern physics course for engineering majors aimed to help them develop more consistent views of quantum phenomena, more sophisticated views of uncertainty, and greater interest in QM \cite{Baily2015}.

Given the growing consensus in PER and science education at large to shift the focus from difficulties to student resources, i.e., pieces of prior knowledge that can be productively used in the learning process \cite{Coppola2013, Goodhew2019}, researchers are starting to ask how to put conceptual, symbolic and epistemological resources of students in the service of learning QM \cite{Dreyfus2017, Dini2017}. However, as regards instructional materials on QM, there is a need to identify the possible links between specific sets of available knowledge elements or structures and possible educational strategies, and to empirically test their effectiveness.

In order to find indications on how to address the aforementioned challenges and needs in the learning of QM, we chose to work in the framework of conceptual change. Since the release of \textit{The Structure of Scientific Revolutions} by T. Kuhn \cite{Kuhn1962}, the theory change from CM to QM has been seen as an exemplary case of conceptual change in the history of science \cite{Thagard1992}. Educational research shows that this is a central element also behind the challenges students face in learning QM \cite{Tsaparlis2009, Singh2015, Lewerissa2017}. With the rise of educational models of conceptual change \cite{Posner1982, Potvin2020}, a number of researchers started to consider the problem of teaching QM as the design of strategies to effectively promote a conceptual change in individual learners \cite{Thagard1992, Kalkanis2003, Tsaparlis2013, Malgieri2017}.

In general, a conceptual change approach to science teaching can be described as ``seeking to foster understanding and the adoption of scientific ideas as new systems of interpretation'' \cite{Amin2014} of natural phenomena.
In the context of introductory science topics, challenges related to knowledge revision \cite{Carey1999, Chi2013, Vosniadou2008}, fragmentation \cite{Vosniadou2008, diSessa2014}, epistemology \cite{Amin2014}, as well as the use of resources \cite{diSessa2014} have been widely investigated in this framework. Indeed, educational models of conceptual change have been primarily developed to account for the transition from na\"{\i}ve to scientific knowledge, a process associated with the modification of conceptual structures formed in the context of lay culture. The transition from CM to QM, instead, requires changes in knowledge structures about a scientific theory, and developed as a result of instruction. While long-established models of conceptual change can represent a valuable aid also in addressing this transition, there is a need to develop approaches to the teaching of QM that take into account the differences involved in learning a successive theory.

In addition, studies on conceptual change are undergoing a systemic turn, with a growing number of researchers coming to understand this process in terms of multiple interacting elements, which include, e.g., cognitive aspects, epistemic cognition, social interaction \cite{Amin2014}. Teaching-learning sequences on QM have prevalently focused on one or the other of these factors. Educational proposals based on a systemic approach, coordinating multiple factors at different levels of analysis, are currently lacking.

In this paper, we describe the development of a QM course for secondary school based on a systemic approach, which relies on an analysis of conceptual change in the learning of QM as a successive theory, and integrates cognitive and epistemic aspects including the examination and use of epistemic practices of theoretical physicists, and a careful approach to interpretive themes. Our goal is to address all the aforementioned challenges, taking advantage of available knowledge elements and intuition in the process. The analysis of conceptual change is aimed to provide an in-depth characterization of the first two challenges, suggesting how to leverage prior knowledge for achieving an effective revision, and how to identify conceptual tools suitable to make predictions on quantum processes across contexts. Student exploration of authentic practices in a theory-building activity aims to propose strategies for promoting the development of a plausible and reliable picture of the quantum world, and is assisted in this process by the development of an awareness of the foundational debate.

\section{Theoretical framework: the identification of the design principles} \label{Sec:2}

\subsection{Transition to quantum mechanics as a successive theory: cognitive aspects}
Our analysis was one source of inspiration for the development of a model of the transition from the understanding of a theory to the understanding of its successor presented by Zuccarini and Malgieri \cite{Zuccarini2022}. The model includes an exploration of the impact of theory change on various forms of challenges (cognitive, epistemic, affective) and the identification of strategies for promoting the understanding of the new content.

As regards the cognitive aspects, in the design of our course we focused only on the transition from CM to QM, and specifically on two cognitive signatures of the knowledge of a scientific theory, which ideally represent a significant component of the initial state of the learner. They are the understanding of, and the ability to use for descriptive, explanatory and problem-solving purposes
\begin{enumerate}[1.]
\item different public representations of relevant concepts: linguistic, mathematical, visual, etc. \cite{Arabatzis2020};
\item the exemplars of the theory: tasks and resolution strategies encountered in lectures, exercises, laboratory assignments, textbooks, etc. \cite[p. 134]{Hoyningen1993}.
\end{enumerate}
Theory change is always accompanied by change in exemplars and in relevant concepts at different representational levels (new formation, evolution, disappearance \cite{Arabatzis2020}). Therefore, we need to consider not only ontological change in concepts, but also change in constructs used by the scientific community to represent these concepts, as well as the change in tasks and in resolution strategies. These features mark important differences with conceptual change processes at introductory level, since na\"{\i}ve science is neither socially shared nor mathematized.

Research shows that trajectories of concepts and constructs from CM to QM often give rise to learning challenges. In general, conceptual dynamics such as new formation may involve, e.g., coalescence or differentiation of familiar entities in nonintuitive terms. Evolution may determine difficulties to identify which aspects of a familiar entity can be productively used in the new theory and which not, to develop a consistent understanding of the new aspects, and to clearly discriminate between the old and the new version. Disappearance may deprive students of important resources in organizing scientific knowledge. Change in exemplars - that may be strongly context-dependent - is reasonably related to knowledge fragmentation. However, it is clear that each factor of change may have an influence on both challenges, and therefore that overcoming these challenges requires a coordination of knowledge revision and knowledge organization strategies.

The analysis was developed from 2014 onwards, in parallel with the course presented in this article. Design experiments described in Section \ref{Sec:5} show how the principles of design were implemented during the cycles of refinement. After the end of the experiments, the framework underwent further development in many aspects, among which
the adoption and revision of dynamic frames: a tool to visualize theory change in concepts and constructs \cite{Andersen2006}. The custom syntax of the frames is ideal to compactly illustrate which aspects of a notion can lead to productive reasoning in which theoretical context. Therefore, we present the tool in the next pages, explaining how it is related to the previous work on the course.

According to this description of the impact of theory change on cognitive challenges, the basis for addressing knowledge revision and its organization in the transition from CM and QM are respectively the educational analysis of change in concepts/constructs, and of change in exemplars.
We present them in two separate subsections.

\subsubsection{Change in concepts and constructs: the challenges and the strategies} \label{Sec:2.1.1}
The design of the course was informed by the description of educationally relevant changes in individual notions from CM to QM, which, according to tools used by researchers on conceptual change, were displayed in comparison tables (see, e.g., Vosniadou, 2008, table 1.1 \cite{Vosniadou2008}). Zuccarini and Malgieri \cite{Zuccarini2022} converted these tables into dynamic frames, an instrument used by philosophers of science to visualize aspects of the categorical structure of a concept in a scientific theory, and therefore to analyze its dynamics in theory change \cite{Andersen2006}. The traditional format of a single frame was subsequently adapted to the direct description of change, not only in concepts (ontological change) but also in constructs (representational change). For clarity, in this article we represent change in scientific notions by means of dynamic frames.

An example is provided by Fig. \ref{FIG:1}.a and \ref{FIG:1}.b.
\begin{figure*}[!htbp]
    \centering
\begin{tabular}{|r|l|} \hline
    \includegraphics[width=.48\textwidth]{FIGURE1a} &
    \includegraphics[trim=0 0 0 -5, width=.52\textwidth]{FIGURE1b} \\ \hline % trim option: in order not to cover the horizontal line (hline)
\end{tabular}
    \caption{Visualizing categorical change: (a) the concept of \emph{system quantity}; (b) the \emph{vector} construct.}
    \label{FIG:1}
\end{figure*}
The first one displays the visualization of change in the concept of \textit{system quantity}. This expression refers to physical quantities describing properties of systems and includes both dynamical variables, that in QM become observables, and parameters such as mass, that in non-relativistic QM behaves as a classical quantity. Fig. \ref{FIG:1}.b describes change in the \textit{vector} construct, that in CM is primarily used to represent physical quantities, while in QM typically refers to the state of a system.

In the frame representation, the categorical structure of each entity is visualized as a hierarchy of nodes that starts from the \textit{superordinate concept/construct} (on the left in the figures) and is organized into sets of \textit{values} (conceptual constituents, on the right), each set corresponding to a different \textit{attribute} that specifies the relation between the set and the superordinate concept. In our case, the superordinate concept is either a basic term of both CM and QM (\ref{FIG:1}.a) or a construct evolving in theory change (\ref{FIG:1}.b). A value is white if it pertains to an instance of the classical version of the superordinate notion, black if it pertains to an instance of the quantum one, gray to both theories.

From an educational perspective, this visual representation of conceptual dynamics from CM to QM is potentially productive in two ways. First, while other modern theories present a clear demarcation line between their phenomena and classical ones (a low $v/c$ ratio in special relativity), in QM the so-called ``classical limit'' is a deep and controversial issue \cite{Klein2012}. It appears that students need to bridge, at a conceptual and a formal level, the world of the new theory to that of the old one, in order to facilitate the transition between the two perspectives. A visualization of continuity and change in concepts and constructs allows us to offer them this kind of support, not in terms of limiting processes, but of categorical structure. Second, while we had already identified different patterns of change which informed strategies for the revision of classical knowledge, the frame format helps to pinpoint and describe these patterns in a compact way. For instance:
\begin{itemize}
\item \textit{categorical generalization}: each value of an attribute either pertains to both theories or only to the quantum one (Fig. \ref{FIG:1}.a, but also \textit{measurement});
\item \textit{value disjunction}: each value of an attribute either pertains only to the classical theory or only to the quantum one (Fig. \ref{FIG:1}.b, but also \textit{superposition}).
\end{itemize}
In the first case, students can productively use their classical intuition in the discussion of those properties that are common to the old and the new version of the notion before addressing the situations corresponding only to the new version. In the second one, the properties of the old version are different from the corresponding ones of the new version, and prior intuition can be used as a contrast after making the new properties available to students.

All this leads us to our first design principle:\\

\centerline{\fbox{\begin{minipage}{\columnwidth}
  \centering{PRINCIPLE OF KNOWLEDGE REVISION}\\
  the analysis of continuity and change in concepts and constructs will be used for developing
  \begin{itemize}
  \item trajectory-dependent strategies for a smooth transition to their quantum versions
  \item end-of-unit tables containing interpretive tasks on selected aspects of their trajectory  $\Rightarrow$\\
  promoting the discrimination between the classical and the quantum version of a notion by identifying the correct context of application of each aspect
  \end{itemize}
  as a result, this approach to knowledge revision provides an opportunity to address student's need of comparability with CM
\end{minipage}}}


\subsubsection{Change in exemplars: the challenges and the strategies} \label{Sec:2.1.2}
The analysis of challenges related to knowledge fragmentation in QM has played a fundamental role in the development of the course. A difficulty was represented by the search for quantum exemplars at secondary school level. As a matter of fact, quantum formalism is among the less common curriculum content in traditional TLSs for secondary school students, as well as real lab assignments and simulated experiments \cite{Stadermann2019}. In upper-division courses, instead, students are exposed to the basic mathematical machinery of the non-relativistic QM and to plenty of exercises in lectures, recitations, homework and exams. As a result, analyzing the nature of these tasks and corresponding resolution strategies became the key for contrasting classical and quantum exemplars.

This work fed into a recent publication on the structure of quantum knowledge for instruction \cite{Zuccarini2020}. According to it, textbooks and educational research mainly focus on the following tasks and related subtasks: finding information (1) on the results of the measurement of an observable on a state, (2) on the time evolution of the state, and (3) on the time evolution of the probability distribution of an observable on a state. The strategies for accomplishing them are radically different from those used in solving CM problems, and vary depending on the type of system and other conditions.

Discussing the issues related to the resolution of quantum tasks requires the adoption of a theoretical perspective suitable to understand how scientific concepts function in determining a particular class of information about the physical world. One perspective specifically designed for this purpose is the coordination class theory \cite{diSessa1998, diSessa2016}, which models “expert” concepts as systems partitioned into functional units. In this perspective, the aforementioned tasks become three different coordination classes. A support in describing their structure is provided by the concept maps presented in \cite{Zuccarini2020}, which display general pathways of qualitative and quantitative solutions related to each task, that can be employed in every context. For instance, getting information on the measurement of an observable on a state is represented in three maps, respectively for a state expressed as a superposition of other states, as an eigenstate of a given observable, or as an eigenstate of a complete set of compatible observables. See Fig. \ref{FIG:2} for the second map.
\begin{figure*}[!htpb]
    \centering
       \fbox{\includegraphics[width=14cm]{FIGURE2}}
    \caption{Measurement on an eigenstate of an observable.}
    \label{FIG:2}
\end{figure*}

In the coordination class framework, these maps can be interpreted as a visualization of the quantum coordination classes. By analyzing Fig. \ref{FIG:2} through the lens of coordination class terminology \cite{diSessa2016}, we infer that the \emph{extraction}, i.e. the initial information, is the knowledge of the state, of the observable we want to measure, and in some cases also of the Hamiltonian. The \emph{inferential net} is composed of the relevant knowledge elements (entities, prediction tools, procedures, etc.) and of the net of connections between them. The \emph{readout strategy} is a path from the extraction to the result, whose direction of travel is indicated by arrowheads on the lines connecting the elements. A \emph{concept projection} is the part of the coordination class that is active in performing a particular task (\emph{readout}). In Fig. \ref{FIG:2}, for instance, it could be the smaller map describing the measurement of the momentum on an energy eigenstate of a harmonic oscillator, with the removal of the box on compatibility, of the kernel (which is empty, since the two observables involved do not admit a simultaneous eigenstate), and as a result, of the boxes at the centre of the figure on the change of basis and on the analysis of the superposition.

Coordination class theory hypothesizes two particular and characteristic challenges: \emph{span} (having adequate conceptual resources to operate the concept across a wide range of contexts) and \emph{alignment} (being able to determine the same concept-characteristic information across diverse circumstances) \cite{Levrini2008}. From the analysis of Fig. \ref{FIG:2}, it is immediate to identify at least two reasons behind the difficulty to build a global knowledge structure in QM. First, the context specific elements of quantum coordination classes are in turn complex objects, such as the concept of eigenstate of an observable, or the structure of the Hamiltonian of a system (its set of eigenstates and corresponding eigenvalues). Second, the subtasks related to using prediction tools and procedures are also complex, unfamiliar, and highly variable from context to context: the determination of the commutator of two observables, the resolution of the eigenvalue problem for energy, the change of basis, etc.

While these maps represent a general guide to the structure of quantum tasks, they are unsuitable for instruction at secondary school level. If we aim to provide school students with valuable support to make predictions on quantum processes across contexts, we need to considerably simplify the picture. The choices we made are the following: set aside time evolution to focus only on measurement; give priority to qualitative predictions; set aside operators, commutators, and eigenvalue equations. After this work of reduction, we are left with the acquisition, the loss\footnote{In the measurement of $Q$, a system possessing a definite value of $O$ loses it if $O$,$Q$ incompatible and the system is not in a simultaneous eigenstate of both. Otherwise, it retains the value.}, and the retention of definite values of observables in measurement depending on the relations between them, and with the nature of this process (stochastic or determinate). The result can be interpreted as a reduction in the elements of Fig. \ref{FIG:2}.

As a first brick to discuss the relations between observables (compatibility and incompatibility), we rely on binary relations between their values that arise in measurement. In our course, a value of a \emph{system quantity} - classical or quantum - that can be said to be either possessed by a physical system (when the probability to measure it is $1$) or not, is denoted as physical ``property'' and relations existing between values are denominated as ``relations between properties''. The language and the existence criterion for a property are borrowed from the Geneva-Brussels approach \cite[see, e.g.,][]{Debianchi2011}. The concept of property we use is a strongly restricted version of the original one, which includes not only values but also the union of disjoint intervals of values. Unless indicated otherwise, we will describe ideal measurements of discrete and continuous quantities only in terms of single values.

The relations between properties of interest to us are defined as follows: two different properties, $P_a$ and $P_b$, belonging respectively to the \emph{system quantities} $O$ and $Q$, not necessarily distinct from each other, are
\begin{itemize}
    \item \emph{mutually unacquirable}: if any system possessing one of them retains it and can never acquire the other in the measurement of the corresponding quantity. No system can ever possess $P_a$ and $P_b$ at the same time (mutual exclusivity);
    \item \emph{incompatible}: if any system possessing one of them loses it and may stochastically acquire the other in the measurement of the corresponding quantity. No system can ever possess $P_a$ and $P_b$ at the same time (mutual exclusivity);
    \item \emph{compatible}: if any system possessing only one of them retains it and may stochastically acquire the other in the measurement of the corresponding quantity. If the system possesses $P_a$ and $P_b$ at the same time, it retains them in the measurement of any of the corresponding quantities.
\end{itemize}

\emph{Mutually unacquirable} properties are, in the first place, different properties of the same quantity, but also properties of different quantities that are mutually unacquirable due to physical constraints. An example of the latter situation is the following: if the azimuthal quantum number of a system is $l=1$, it is not possible for this system either to possess $m=4$ or to acquire it in the measurement of $L_z$, and viceversa. An arbitrary value of position is always \emph{incompatible} with any value of its conjugate momentum. A property of spin is always \emph{compatible} with properties of spatial observables (position, momentum, kinetic energy, orbital angular momentum, etc.). As with the relations between observables, relations between properties are invariant across contexts except for those between energy properties and properties of other observables. For the latter, the validity of the relations is restricted to the type of system described by the Hamiltonian at hand.

This framework can be used as a conceptual basis for an educational reconstruction of quantum measurement across contexts and of the transition from the classical to the quantum measurement task. Since mutual unacquirability and incompatibility naturally arise in the exploration of spin or photon polarization measurements, these relations can be introduced in one of these contexts at an elementary level. Given the possibility to address measurement in a simple quantitative form (Malus's law for photon polarization and its equivalent for spin), they can be justified to students as empirical regularities that are specific to quantum systems. Moving on to the relations between \emph{system quantities} is almost immediate: two quantities are compatible if every property of each one is compatible with at least one property of the other, otherwise they are incompatible. Except in a limited number of cases (when two quantities are incompatible, but admit simultaneous eigenstates), relations between quantities can be qualitatively assessed in a similar way:\\
the \textit{system quantities} $O$ and $Q$ are
\begin{itemize}
        \item \emph{incompatible}: if any system possessing a property of one of them loses it in the measurement of the other quantity and stochastically acquires one property of the latter. No system can ever possess properties of $O$ and $Q$ at the same time;
        \item \emph{compatible}: if any system possessing only a property of one of them retains it in the measurement of the other quantity and stochastically acquires one property of the latter. If the system possesses properties of $O$ and $Q$ at the same time, it retains them in the measurement of these quantities.
\end{itemize}

Other contexts can be added to the picture either by using relations between properties or between observables. In both cases, the measurement process can be initially discussed at a qualitative level. If we opt for framing the issue in terms of relations between observables, by knowing which relations exist between the observables that initially have a definite value and the measured observable, we determine which of them are definite after the measurement and which not. In addition, the possession in advance or the lack of a property of the measured observable allows us to assess the nature of the process: respectively determinate or stochastic. This task can be accomplished also in the presence of degeneracy.

If the course includes the discussion of the vector representation of the state, quantitative predictions on discrete state systems are also possible. Measurements on systems such as harmonic oscillators, potential wells, bound states of hydrogen-like atoms can be studied by exploiting the conversion of the relations into algebraic constraints. In particular, with the introduction of the state vector, mutual unacquirability of properties becomes orthogonality of the corresponding states, and paves the way for examining the superposition of a finite number of vectors. A task within the reach of school mathematics.

As regards the transition from classical to quantum measurement, the relations represent a further instrument for addressing student's need of comparability, since mutual unacquirability and compatibility can be expressed also
in classical terms. In the classical regime, all \emph{system quantities} are compatible with one another, and every point particle always possesses one property of each quantity. Thus, the emergence of incompatibility can be identified as an explanation of theory change with relation to measurement and the description of systems at a point in time.

As a result, we are able to formulate the second principle:\\

\centerline{\fbox{\begin{minipage}{.9\columnwidth}
  \centering{PRINCIPLE\\ OF KNOWLEDGE ORGANIZATION}\\
  the framework of the relations between properties and then between observables will be developed together with students in the simple context of two-state systems, and will be used to
    \begin{itemize}
  \item promote the construction of a unifying picture of quantum measurement and the ability to manage it in problem-solving, allowing students to explore this process in other scientifically significant contexts
  \item  promote a smooth transition to a quantum perspective and help address student's need of comparability between CM an QM, since it constitutes a transtheoretical framework
  \end{itemize}
  \end{minipage}}}

%\vspace{5mm}

\subsection{Transition to quantum mechanics: personal and scientific epistemology} \label{Sec:2.2}

\subsubsection{Personal epistemology: theoretical modelling cycles} \label{Sec:2.2.1}
Personal epistemology may be introduced as an individual's answers to questions such as ``how do you know?'' and ``why do you believe?'' \cite{Wittmann2020}. Recent reviews on conceptual change and epistemic cognition report that there is a convincing body of research establishing a connection between more sophisticated epistemologies and deeper conceptual understanding in a particular domain \cite{Amin2014, Elby2016}. QM represents an ideal context for exploiting this synergy: a focus on epistemology may promote the learning of counterintuitive quantum content; on the other hand, a course of QM may be an opportunity for studying the practices of scientific modelling. Wittmann and Morgan, for instance, structured large part of their course around activities in which students work to build new concepts and create new knowledge, using lecture time to discuss and debate ideas in a peer-instruction format \cite{Wittmann2020}.

In order to put the aforementioned synergy in the service of learning QM, we too chose to focus on knowledge-building activities. However, given the wide range of possible activities of this kind, we endeavoured to identify the most appropriate ones for the content at hand. According to Sandoval \textit{et al.}, the conceptual, procedural, and epistemic expertise of a discipline is bound up in its specific practices \cite{Sandoval2016}. But what practices characterize the construction of QM as a knowledge domain? A peek at the history of physics in the early 20th century suggests that theory-building is at the core of these practices. We concluded that involving students in theoretical modelling activities could be a promising strategy for helping them accept the quantum description of the world as a plausible and reliable product of their own inquiry, developing theoretical reasoning skills in the process. However, educational research on the epistemic practices that characterize the work of theoretical physicists is currently lacking. In Fig. \ref{FIG:3}, we propose a list of historically significant practices of theoretical nature used by physicists for building new scientific knowledge.

The specification of these practices served as a starting point for the design of strategies to engage students in theoretical modelling cycles, which is performed by drawing on perspectives concerning the use of mathematics in physics \cite{Uhden2012, Redish2015}, the conduction of thought experiments \cite{Gilbert2000, Stephens2012}, on different forms of inquiry- and modelling-based approaches, e.g., the ISLE learning system \cite{Etkina2015}.

\begin{figure*}[!htpb]
\centerline{\fbox{\begin{minipage}{.9\textwidth}
\centering{EPISTEMIC PRACTICES OF THEORETICAL NATURE IN THE HISTORY OF PHYSICS}\\
FUNDAMENTAL: generating, extending and revising interpretive models that act as comprehensive systems of explanation with the aim to develop a unified picture.
\begin{enumerate}
 \item building new knowledge on a topic by means of thought experiments (e.g., Galileo's free fall experiment, Maxwell's demon, Einstein's elevator)
 \item interpreting already known laws within the framework of new models (e.g., Clausius, Maxwell and Boltzmann's interpretation of thermodynamic quantities and laws within the framework of atomistic models)
 \item deepening the theoretical investigation of a phenomenology by adopting multiple perspectives (e.g., Euler's two specifications of the flow field in fluid mechanics)% see Calero, Genesis of fluid mechanics, p. 427, note 28.
 \item identifying mathematical constructs suitable to describe features of physical objects and processes (e.g., Newton's adoption of constructs of infinitesimal calculus for describing gravity and mechanics at large)
 \item analyzing mathematical constructs already representing features of physical objects or processes to deduce results that have not been unveiled yet (e.g., Lagrange's laws of fluid dynamics: an application of one of Euler's specifications of the flow field to special cases with new mathematical methods)% see Calero, pp. 450-451.
 \item starting from results found in one context and extending or adapting them to other contexts (e.g., Maxwell's hydrodynamic model of the magnetic lines of force)
  \end{enumerate}
  \end{minipage}}}
  \caption{Practices of theoretical nature that have been historically used by physicists for building new knowledge.}
    \label{FIG:3}
\end{figure*}

The third principle underlying the design of our course is the following:\\

\centerline{\fbox{\begin{minipage}{.9\columnwidth}
  \centering{EPISTEMIC PRINCIPLE }\\
  design the course around a modelling process that includes theoretical practices used by physicists
  in the historical development of the discipline, with the goal to help students
    \begin{itemize}
  \item accept the quantum description of the world as a plausible and reliable product of their own inquiry, thus promoting a smooth transition to a quantum perspective
   \item  build theoretical reasoning skills
    \end{itemize}
\end{minipage}}}

%\vspace{5mm} %5mm vertical space


\subsubsection{Scientific epistemology: approach to interpretation} \label{Sec:2.2.2}
Research on students transitioning from classical to quantum thinking shows that when interpretive themes are deemphasized, interest in QM decreases, while learners still develop a variety of (sometimes scientifically undesirable) views about the interpretation of quantum phenomena \cite{Baily2015}. For this reason, we built our course around a \emph{clearly specified} form of standard approach \cite{Bub1997}, schematically set apart from other schools of thought by means of rules of correspondence between the structure of the theory and its physical referents in the world:
\begin{enumerate}
    \item a pure state provides complete information on the behavior of an individual quantum system (ruling out statistical interpretations);
    \item an observable of a system has a determinate value if and only if the quantum state of the system is an eigenstate of the operator representing the observable (ruling out modal interpretations);
    \item the quantum description of processes includes two different types of state evolution: in the absence of measurement, the unitary evolution governed by the Schr\"{o}dinger equation; in measurement, the evolution prescribed by the projection postulate (ruling out other no-collapse interpretations).
\end{enumerate}
As mentioned at the end of each statement, all have been questioned by part of the scientific community, with the third being the most unsatisfactory one for a variety of reasons \cite{Bub1997}, starting from the measurement problem \cite{Schlosshauer2007}.

An additional interpretive choice concerns the wave-particle duality. Baily and Finkelstein adopt a ``matter-wave perspective'' \cite{Baily2015}, that allows students to interpret without paradoxes how a system can ``know'' whether two paths are open or only one of them in a ``which-way'' experiment. However, if the system propagates as a wave, students may ask what kind of medium supports or, equivalently, is perturbed by this wave. For this reason, in the construction of a full quantum model of a system, we adopt a field ontology, a perspective put forward in education also in recent years \cite[e.g.,][]{Hobson2013}.\\

\centerline{\fbox{\begin{minipage}{.9\columnwidth}
  \centering{EPISTEMOLOGICAL PRINCIPLE }\\
  design the course around a clearly specified form of (standard) interpretation in order to
    \begin{itemize}
  \item build a coherent educational proposal
  \item identify which facets of the foundational debate are triggered by each interpretive choice, and how to discuss them  according to the educational level of the students, helping them develop an awareness of the cultural significance of the debate, of the limits the chosen stance, of the open issues
    \end{itemize}
\end{minipage}}}


\section{Implementing the principles: development of the path and of the activities}

In this section, we illustrate how the principles of design drove the development of the course and of individual activities within it. The content and the modelling process were informed by the interplay of the \textit{Principle of Knowledge Revision}, the \textit{Principle of Knowledge Organization} and the \textit{Epistemic Principle}.  The construction of the learning path is described in Section \ref{Sec:3.1}. In the subsequent sections, each principle is addressed separately, showing how it guided the design of individual activities aimed to implement it. The first and the second principles are discussed respectively in Section \ref{Sec:3.2} and \ref{Sec:3.3}. Converting the specific practices listed in Fig. \ref{FIG:3} into authentic inquiry activities has been a particularly complex task that required the examination of different research perspectives and approaches (Section \ref{Sec:3.4.1}). In Sections \ref{Sec:3.4.2} - \ref{Sec:3.4.5} we show how they were blended in the design of various types of activities. The impact of the \textit{Epistemological Principle} on the design was conditioned by the chosen contexts and topics, and is addressed in Section \ref{Sec:3.5}.

\subsection{Structuring the learning path} \label{Sec:3.1}
The main source of inspiration and materials for this course has been an educational path for the introduction of QM in the context of polarization developed and evaluated by the PER group of the University of Udine \cite[e.g.,][]{Ghirardi1996, Michelini2004, Michelini2019}. The Udine's path is focused on the superposition principle and its consequences, starting from the distinction between properties and states. It makes use of hands-on activities with cheap experimental tools (polarizing filters, birefringent crystals), quantitative measurements with light intensity sensors, and of JQM \cite{Michelini2002}, an open-ended environment for computer-simulated experiments on photon polarization. However, the two curricula are different with respect to their design principles, various strategies, physical situations included and sequence. The sequence of activities of our course will be presented in Section \ref{Sec:4}, and displayed in full in Fig. \ref{FIG:9}. Here, we describe the the construction of the learning path, showing how the interplay of the first three principles defined its shape.

As a matter of fact, starting with polarization is compatible with the implementation of each of the principles. Since the phenomenon can be experienced by means of classical light beams and explained both in classical and quantum terms, it easily lends itself to a gradual building of a quantum model of the physical situation (\textit{Epistemic Principle}) and to the revision of classical concepts and mathematical constructs (\textit{Principle of Knowledge Revision}). In addition, two relations between properties (mutual unacquirability and incompatibility) naturally arise in photon polarization measurements. Along with compatibility, they represent the conceptual tools needed for extending the examination of measurement to distant physical situations (in our case, the hydrogen-like atom), promoting the construction of a unifying picture across contexts (\textit{Principle of Knowledge Organization}).

The introductory phases of our modelling cycle are described according to the template of the \emph{Model of Modelling} \cite{Gilbert2002, Gilbert2016}, an artifactual view that ascribes particular importance to the process of the creation and expression of the model. In these stages, the artifact is denoted as `proto-model', since it will be complete only after its expression by means of external modes of representation:
\begin{enumerate}[1.]
    \item \textit{Creation of the proto-model}:
        \begin{enumerate}[(a)]
        \item experiences for supporting its creation: (1) exploration of the  phenomenology of the linear polarization of light (interaction of macroscopic beams with polarizing filters/birefringent crystals); (2) empirical determination of its quantitative laws (Malus's law for beams polarized at $\theta$ incident on a filter with axis at $\phi$: $I_{out}=I_{in}\cos^2{(\theta-\phi)}$, reduction to half for unpolarized ones: $I_{out}=I_{in}/2$); (3) presentation of fundamental experiments on the detection \cite{Grangier1986, Grangier2005} and polarization of single photons after passing a filter;
        \item sources: the heuristic criterion according to which the hypotheses on the behavior of individual photons must be compatible (1) with the experimental evidence on the detection and polarization of a photon, and (2) with the classical phenomenology and laws for macroscopic light beams.
        \end{enumerate}
    \item \textit{Expression of the proto-model}: a fundamental mode of representation used in this course is the iconic language of JQM for the depiction of idealized physical situations and experiments involving the polarization of single photons. The representation includes photons - visualized by means of their polarization property (Fig. \ref{FIG:4}) - and devices such as single photon sources, polarizing filters, calcite crystals, screens and counters (Fig. \ref{FIG:5}). This language will represent an essential support for the implementation of theoretical epistemic practices such as thought experiments and the interpretation of classical laws of polarization in terms of photons. Mathematical modes of representation accompany these activities (e.g., Malus's law) and support the implementation of mathematical modelling practices (e.g., hypothesizing a mathematical representation of the quantum state and interpreting the meaning of its properties);
\begin{figure}[!ht]
    \centering  % Requires \usepackage{graphicx}
       \fbox{\includegraphics[width=\linewidth]{FIGURE4}}
    \caption{Iconic representation of the photon polarization \protect\cite{Michelini2002}. Students are informed that the segments are not to be intended as real physical representations of single photons, but as a support for theoretical reasoning about photon polarization and related physical situations.}
    \label{FIG:4}
\end{figure}
\begin{figure}[!ht]
    \centering  % Requires \usepackage{graphicx}
       \fbox{\includegraphics[width=\linewidth]{FIGURE5}}
    \caption{Iconic representation of a single photon source with a predetermined polarization property (vertical, in this case), one vertically polarized photon, a polarizing filter with an arbitrary axis (here at $45^{\circ}$), a birefringent crystal with a $0^{\circ}$ and a $90^{\circ}$ channel, a screen placed on the extraordinary one, two photon counters.}
    \label{FIG:5}
\end{figure}
\end{enumerate}

After its creation and expression, the full-fledged model is ready to be developed and revised through a process conducted by means of theoretical epistemic activities (\textit{Epistemic Principle}), where students need to reinterpret, at a single-photon level, macro-phenomena and macro-laws which have already been explored by means of cheap experimental tools. It starts as the model of an object (the photon) for what concerns its detection and polarization. It soon grows to become a model of the interaction between photons and devices composed of filters/crystals followed by counters. The interaction with crystals and detectors is interpreted as an instance of the quantum measurement process, leading students to identify the relations between the initial property and those that correspond to possible outcomes of measurement. By means of this interpretive lens, the discussion can go beyond the scope of polarization: the relations are applied at a global level (incompatibility: position or velocity measurement on a system) and in the context of the hydrogen-like atom (compatibility: measurements of $E$, $L$, $L_z$, $S_z$). The relations between properties are then upgraded in terms of relations between observables. Next, the model can be embedded into the algebraic language of the polarization state vectors. Another inroad into the context of the hydrogen-like atom is made to introduce and discuss its state vector in terms of quantum numbers and calculate transition probabilities by means of vector superposition. The model is thus ready to undergo a major revision, incorporating also the propagation of photons - wave-like interference included - and their entanglement, therefore leading to the construction of a far-reaching model of radiation (the photon) and matter (the hydrogen-like atom).

A fundamental choice is addressing the quantum state, its vector and then quantum superposition only after the discussion of the concepts of measurement and observable. There are various reasons behind this choice. First, this sequence allows us to focus on the revision of a notion at a time (\textit{Principle of Knowledge Revision}). This would not be possible if we started directly with the superposition principle, that in QM is inextricably linked to all the aforementioned notions. The possibility to postpone the introduction of the state and superposition is granted by the \textit{Principle of Knowledge Organization}, which provides instruments for discussing quantum measurement and observables without resorting to the concept of state. Second, implementing the \textit{Epistemic Principle} involves structuring math modelling activities, e.g., related to the introduction of the state vector, that may cause a high cognitive load. In the context of polarization, building the mathematical representation of the state requires a consistent understanding of the single-photon interpretation of the Malus's law as probabilistic law of transition between different polarization properties: $p(\theta\mapsto\phi)=cos^2{(\theta-\phi)}$. Since this topic has been widely discussed in the unit on measurement, the state of polarization can be simply presented as a change of perspective on the same phenomena, without adding new physical content. This allows students to focus exclusively on the revision of the concept of state and on math modelling activities, thus reducing the cognitive load. One example is expressing the law of transition in terms of relations between state (ket) vectors\footnote{In the context of linear polarization, there is no need of complex numbers. Therefore, we do not introduce bra vectors and express the Born rule by using the square of a dot product.}: $(|\theta\rangle\cdot|\phi\rangle)^2=cos^2{(\theta-\phi)}$. Third, our course includes not only the context of photon polarization, but also of the hydrogen-like atom. An immediate examination of the concept and mathematical representation of the quantum state of the latter would be too challenging to our student population. Instead, the knowledge of measurement processes on this type of system together with the discussion of the polarization state vector represent a natural basis on which to build the state of a hydrogen-like atom and the corresponding (ket) vector in terms of quantum numbers\footnote{We restrict the mathematical discussion to superposition states with real coefficients: no need of bra and square moduli.}: $|n, l, m, s \rangle$.

The inclusion of the hydrogen-like atom offers various educational opportunities. In the discussion of the state, it allows us to break the one-to-one correspondence between properties and states that characterizes linear polarization (identifying the $|n, l, m, s \rangle$ state of a hydrogen-like atom requires the specification of four properties), as well as the identity of the angle between polarization properties and corresponding state vectors  (directions in the state space of the hydrogen-like atom are clearly unrelated to directions in the physical space). In the case of superposition, linear combinations of $|n, l, m, s \rangle$ vectors make it possible to generalize the discussion of measurement and observables to situations in which no known quantity is initially defined, and to address the normalization of the state vector after a measurement, that is trivial when the components of a superposition are limited to two terms, as in the context of polarization. Moreover, it allows us to introduce the concept of product state, since $|n, l, m, s \rangle$ is the composition of two states, $|n, l, m\rangle$ and $|s \rangle$, and the first expression is the contracted form of $|n, l, m\rangle|s\rangle$. In the course, we leave out any reference to the mathematical construct known as tensor product, but explain that the last expression is a way to denote a state (the global state of the atom) that depends on two component states (its spatial state and its spin state). Last, the context naturally lends itself to an interdisciplinary approach in collaboration with the chemistry teacher on topics such as orbitals and the atomic structure.

A solid understanding of the concept of state, of its vector, and of quantum superposition represent a strong basis for building a consistent interpretation of quantum interference and entanglement at a conceptual and mathematical level. Hence, the course ends with the discussion of propagation (``which-path'' experiments) and entanglement (first, of spatial and polarization modes of a photon, then of the polarization of different photons, and finally in the measurement problem).

To sum up, we identified the following path of learning and concept revision from CM to QM as potentially productive: linear polarization $\rightarrow$ measurement  $\rightarrow$ system quantity $\rightarrow$ state $\rightarrow$ vector $\rightarrow$ superposition $\rightarrow$ interference $\rightarrow$ general model (of a system) $\rightarrow$ correlation between internal components of the state (in QM, they can be entangled).

\subsection{Structuring activities according to the principle of knowledge revision} \label{Sec:3.2}

This section is devoted to the design of activities to support students in the revision of classical concept and constructs according to the \emph{Principle of Knowledge Revision}. We describe two cases: the first concerning the ontological shift of a concept (measurement), the second the representational shift of a construct (vector superposition). Here we examine the path for the introduction of quantum measurement, and the end-of-unit table on superposition. Such tables are scheduled at the end of a unit and are designed to promote the discrimination between the classical and the quantum version of a notion with a birds's eye view on the revision process.


\subsubsection{Measurement} \label{Sec:3.2.1}
In the transition to a quantum picture, the trajectory of the concept of \emph{ideal measurement} (see Fig. \ref{FIG:12})
\begin{figure}[!htpb]
       \fbox{\includegraphics[width=\columnwidth]{FIGURE12}}
    \caption{Ideal measurement: concept trajectory from CM to QM \cite{Zuccarini2022}.}
    \label{FIG:12}
\end{figure}
and, as a consequence, its revision, are of crucial importance. In the context of polarization there is a specific challenge to take into account. While the linear polarization of macroscopic light beams can have any orientation in the plane of polarization and is identified by measuring its angle, the linear polarization of a photon can also have any orientation, but its measurement gives one of two angles that may differ from the initial one. Research found that students have difficulties in interpreting the quantum case as a two-state system \cite{Singh2015} and to frame the physical situation as a measurement: in our first design experiments, some of the students wondered how it was possible to describe it as such, because to them it was just ``a weird interaction altering the property''.

As can be deduced from Fig. \ref{FIG:12}, the trajectory of the concept of measurement in theory change is an instance of \emph{categorical generalization} (see Section \ref{Sec:2.1.1}). This type of dynamics suggests to start from the special case pertaining to both CM and QM, which is described by the gray boxes in the figure, i.e., the classical-like and determinate case. In general, it happens when the system possesses in advance a property of the measured quantity (in  the context of polarization: when the initial property of the photon coincides with one of the two possible outcomes of measurement). This situation is familiar to students, since it can be interpreted as an ideal classical measurement. Then, we move on to discuss its new feature, active and stochastic (when the photon does not possess in advance a property of the measured polarization quantity), as a form of generalization of the first case. Finally, it is possible to focus on the conditional nature of measurement and the different physical situations corresponding to each case.

In the cycles of refinement of our course, we considered two possible devices for measuring photon polarization: 1) polarizing filter + detector; 2) birefringent crystal + detectors placed on the output channels. A prerequisite to discuss quantum measurement by means of these devices is framing their interaction with a photon in terms of information obtained on its polarization property as a result of the process. Some authors who introduce quantum measurement in the context of polarization start the discussion with the analysis of vector superposition \cite{Heyde2020, Michelini2019}, as a valuable guide for this purpose : e.g., a photon prepared in $|45^{\circ}\rangle=\frac{1}{\sqrt{2}}|0^{\circ}\rangle+\frac{1}{\sqrt{2}}|90^{\circ}\rangle$ is absorbed by a filter with axis at $0^{\circ}$ as a result of a transition to $|90^{\circ}\rangle$, or equivalently, of the acquisition of the property at $90^{\circ}$. Since in our learning path, we explicitly renounce to start from the state and superposition in order to revise a notion at a time, the discussion of polarization measurements must be preceded by an activity designed to support students in interpreting the absorption of a photon, either by a polarizing filter or by detectors placed after a crystal, as the result of a transition in polarization property or (in the classical-like case) of the retention of the initial property. Since at this stage, polarization observables have not been introduced yet, we suggest students to use the expression ``outcome-property'', which denotes the properties associated with the outcomes of the polarization measurement at hand.

A possible outline of the revision of measurement, expressed as a sequence of goals, could be the following: supporting students in recognizing that
\begin{itemize}
    \item in the context of photon polarization, every possible outcome of the interaction of a photon with the device can be interpreted as the consequence of the retention/aquisition of a polarization property;
    \item the classical-like, passive and determinate interaction with the device can be interpreted as a measurement (familiar case, in which the system already possesses an outcome-property);
    \item the purely quantum, active and stochastic interaction with the device can be interpreted as a measurement (categorical generalization, corresponding to the new situation in which the system does not possess an outcome-property);
    \item the nature of measurement is conditional, passive and determinate when the system already possesses an outcome-property, active and stochastic when it does not.
\end{itemize}

\subsubsection{Superposition} \label{Sec:3.2.2}
Quantum superposition is a fundamental topic in the learning path. Hence, the development of the course has been heavily influenced by the need to promote an effective revision of the representational properties of vector superposition. The design of these activities is based on a careful examination of the trajectory of this construct in theory change, which is displayed in Fig. \ref{FIG:17}.
\begin{figure}[!htpb]
       \fbox{\includegraphics[width=\columnwidth]{FIGURE17}}
    \caption{Vector superposition: concept trajectory from CM to QM \cite{Zuccarini2022}.}
    \label{FIG:17}
\end{figure}
The attributes included in the figure identify the basic representational features of vector superposition and the changes it undergoes in the transition to the new paradigm.

Interference and entanglement are not addressed during the initial discussion of superposition, but much later together with propagation. Therefore, the revision process related to these aspects (in the figure: \textit{Ability to produce interference} and \textit{Factorizability into component vectors}) was postponed to the final unit of the course.

In this section, we examine the remaining features, \textit{Procedure and goal}, \textit{Number of physical entities involved}, and \textit{Constraint on the component vectors}, where the pattern of \emph{value disjunction} (see Section \ref{Sec:2.1.1}) occurs in two out of three cases. This patterns suggests a different strategy to address the revision process: first building an understanding of the new features of the construct, and then contrasting them with those of its classical counterparts, in order to identify which of the familiar features lead to unproductive reasoning in a quantum context. Hence, prior intuition is used as a contrast at the end of the instructional sequence on the topic.

Since we are dealing with the revision of the representational features of a mathematical construct, which become evident in the transition from a physical situation to its mathematical representation (mathematization) and viceversa (interpretation), we need to focus on these processes in a quantum context. Mathematization has been already performed during the introduction of the state vector. Here we use interpretive tasks.

Procedure and the goal (decomposition of the state vector in a given basis to obtain info on the measurement of the corresponding observable) are introduced in worksheet items on polarization measurements, 1) of the standard observable (horizontal and vertical properties), 2) of different polarization observables, where we go in depth on the issues of decomposition and basis change.

The last task can be used also to draw the attention on the physical referent of quantum superposition: one entity, the state vector, that can be decomposed in different bases depending on the information we need to derive. In later teaching experiments, additional support has been provided by means of embodied cognition, as described in Zuccarini and Malgieri \cite{Zuccarini2022}. In this activity, the perceptual experience of passive rotations (simulating the passage of the same state from superposition in the basis of one observable to that in the basis of another) was put in the service of promoting the awareness that quantum superposition involves only one physical entity.

The new constraints on the number of component vectors (maximum number equal to the dimensions of the state space) and on their directions (orthogonal to one another) are dealt with by means of interpretive tasks on the meaning of quantum superposition in measurement. In the context of polarization, we have two mutually unaquirable results, and therefore only superpositions of two orthogonal vectors. For helping students make this connection, we administer tasks on the examination of various mathematical expressions: superpositions of three polarization state vectors and of two non-orthogonal vectors, asking to determine whether the formulas have physical meaning.

The addition of the hydrogen-like atom provided a further context with more general and significant features. By discussing measurement on a superposition of its eigenstates in terms of quantum numbers (see Section \ref{Sec:3.1}), it is possible, e.g., to generalize the physical meaning of the constraint on the number of component vectors.

The end of the instructional sequence is represented by the end-of-unit table on superposition. Here we guide students to compare the features of familiar forms of vector superposition (of forces and waves) represented in Fig. \ref{FIG:17} with quantum superposition, using prior intuition as a contrast. As a matter of fact, the representational role of this mathematical process in QM is very different from that of the most commonly used forms of superposition in CM. In particular, a consistent interpretation of the referent (the decomposition of only one vector) is a prerequisite for addressing the quantum notion of interference, which is totally internal to the individual system. From this derives the seemingly contradictory statement of Dirac: ``Each photon then interferes only with itself'' \cite[p. 9]{Dirac1967}. In such respect, the structure of the frame of Fig. \ref{FIG:17} represents a useful template that can be converted into a comparison table, where students are required to identify whether a given statement (e.g.: a goal of superposition is to find the resultant) pertains to one or more forms of superposition (forces, waves, quantum).


\subsection{Structuring activities according to the principle of knowledge organization} \label{Sec:3.3}
Implementing the principle entails the construction of activities supporting students in identifying the relations between properties and using them to manage the measurement process in multiple contexts. Here we describe the activation of the principle: first, the design of activities for the introduction of the relations in the context of polarization; then, use of the relations for qualitatively assessing measurement at a global level (position and velocity) and in the context of the hydrogen-like atom ($E$, $L$, $L_z$, $S_z$, $x$).

\subsubsection{Introducing the relations between properties} \label{Sec:3.3.1}
As we said in Section \ref{Sec:2.1.2}, the relations of mutual unaquirability and incompatibility naturally arise in polarization measurements. After the revision of the concept of measurement both at a qualitative and quantitative level (according to the probabilistic interpretation of Malus's law), and after the introduction of polarization observables, which are identified by their two outcome-properties ($\phi$, $\phi+90^{\circ}$), students have all they need to determine whether there exists a measurement in which a photon loses its initial property $\theta$ and may stochastically acquire a property $\phi$ (incompatibility) or not (mutual unacquirability, in which case $\phi=\theta+90^{\circ}$).

Therefore, instead of lecturing on the definition of the relations, we can support students in acquiring the appropriate perspective for their introduction, by asking them to assess different statements on the retention, loss, and acquisition of properties in measurement. The template on which to build the task is provided by a table containing a concise definition of the possible relations between two properties, $P_a$ and $P_b$ (Fig. \ref{FIG:22}).
\begin{figure}[!htpb]
       \includegraphics[width=\columnwidth]{FIGURE22}
    \caption{Definition of the relations between properties: summary table.}
    \label{FIG:22}
\end{figure}
The statements are four because, to complete the picture, we consider a further relation: identity, embodying the fact that if the initial property ($P_a$) is also a possible outcome of measurement, the system retains this property. The statement may seem trivial at first, but corresponds to an important feature of quantum systems: when the system has a property of an observable $O$ at an instant (either as a result of preparation or acquired in a previous measurement), sufficiently rapid measurements of $O$ will provide the same property with certainty.

The core of the task is the following: given a photon prepared with an arbitrary polarization property $P_a$, we ask, for each of the four statements corresponding to the definition of different relations, whether the event occurs in measurement and, if it does, under which conditions. The item can be totally abstract or set in the usual context of calcite crystals and counters, as a support for student reasoning about measurement. In any case, this task corresponds to an increase in the level of abstraction of the activities, since the properties at hand are totally arbitrary. We assume that the work done on transition probabilities and on the revision of measurement represents an adequate basis for the analysis of the statements.

After the activity, each statement can be identified by the instructor as the definition of a relation between properties.

Finally, we observe that this interpretive activity corresponds to a practice of the theoretical physicists: ``deepening the theoretical investigation of a phenomenology by adopting multiple perspectives'', and thus concurs to implement also the \emph{Epistemic Principle}. As a matter of fact, we are dealing with a change in perspective on the same phenomenon: from the revision of the concept of measurement to the analysis of what relation is established by measurement between two given properties. In accordance with the basic goals of the practice, such change of perspective significantly extends the scope of the student journey through the quantum realm.

\subsubsection{Extending the use of the relations between properties to other contexts} \label{Sec:3.3.2}
Since the only transition law students know at this stage is the probabilistic interpretation of Malus's law for polarization, and quantum superposition is not yet available, the discussion of measurement of different observables in other contexts has to start at a qualitative level. Here, the relations between properties can act as an organizing principles for managing the process.

Students may address quantum exemplars by relying on the revision of measurement and of the definition of the relations between observables. They include: 1) deducing what relation there can be between properties of the same observable and of different ones based on their behaviour in measurement; 2) deriving which observables will have a definite value after a measurement and the nature of the process (stochastic or determinate) based the knowledge of the relations between the properties of the measured observable and of other ones.

The cases we consider are
\begin{itemize}
    \item position and velocity measurements at a global level: for discussing mutual unaquirability of the properties of the same quantity, and the measurable consequences of incompatibility between the properties of different observables;
    \item measurements of $E$, $L$, $L_z$, $S_z$, $x$ in the context of the hydrogen-like atom: the first four observables, for discussing compatibility between the properties of different observables and its measurable consequences; position, for discussing what happens if its properties are incompatible with those of some observables ($E$, $L$, $L_z$), and compatible with others ($S_z$).
\end{itemize}

As before, also this activity is an epistemic practice of the theoretical physicist: ``starting from results found in one context and extending or adapting them to other contexts.'' In order to support students in running this practice, the tasks can be presented in terms of a \emph{structured inquiry} (see Section \ref{Sec:3.4.1}). This means that the instructor defines the problem (extending the use of the relations) and the procedure (here: a sequence of inferential and interpretive questions), while students generate an explanation based on the theoretical knowledge at their disposal.

\subsection{Structuring activities according to the epistemic principle}  \label{Sec:3.4}

\subsubsection{Approaches and research perspectives for running theoretical epistemic practices} \label{Sec:3.4.1}

\emph{Inquiry-based learning and Modelling-based teaching}. The first expression denotes a set of active-learning approaches devised to support students in the design and conduction of investigations that mirror the processes
used in science for building its body of knowledge. In this work, we describe the forms of inquiry on which we drew to develop epistemic activities by referring to Llewellyn \cite{Llewellyn2012}. They differ in the amount of information and direct instruction provided to students during the activity:
\begin{itemize}
    \item Demonstrated inquiry: the instructor poses the research questions or generates the hypotheses to test, plans the inquiry procedure and communicates the results. A demonstrated inquiry differs from conventional demonstrations in the way the instructor poses questions to the students during the presentation, soliciting input in the design of the experiment;
    \item Structured inquiry: questions/hypotheses and procedure are still provided by the instructor; however, students generate an explanation supported by the evidence they have, possibly drawing implications;
    \item Guided inquiry: in this case, the instructor provides the questions/hypotheses, but the student is responsible for designing and conducting the investigation, as well as for drawing conclusions.
    \item Self-directed inquiry: the instructor provides the initial orientation, but students raise their own questions/hypotheses, design their own procedures, and organize and analyze their own results.
\end{itemize}
In all cases, the instructor offers content and process support during all phases of the inquiry, providing extensive scaffolding. Since in our course, the goal of inquiry is not to ‘test variables’ but to develop and refine a scientific explanation in the form of a theoretical model, it will not be conducted by means of controlled experiments, but of hypothetico-deductive reasoning (a mental process in which students progress from a hypothesis to specific conclusions by means of logical inferences), thought experiments, etc. The requirements for running a specific inquiry cycle (experiences, sources) will be identified - when needed - by using the template of the model of modelling, which has been presented in Section \ref{Sec:3.1}.

\emph{Mathematical modelling strategies}. In order to provide insight on the ways in which mathematics can be put in the service of physical modelling, we drew on theoretical studies on the role and the language of mathematics in physics. Uhden \textit{et al.} identified two fundamental aspects to consider \cite{Uhden2012}: the deeply tangled unity of mathematical and physical models, and the distinction between technical and structural role of mathematics in physics. The latter is the role of math in structuring physical concepts and situations that is made explicit in the processes of mathematization and interpretation. Another perspective is provided by Redish and Kuo, who analyze the language of mathematics in physics by means of cognitive linguistics in a resources framework \cite{Redish2015}, and suggest to initially focus on physical intuition and embodied experience rather than equations and principles. A common theme of both studies is the line of development of the discourse: from concrete to abstract (from physical issues to mathematics) followed by a new interpretive activity, aimed at clarifying further physical implications of the newly introduced structure (Fig. \ref{FIG:6}).
\begin{figure}[!htpb]
       \fbox{\includegraphics[width=\columnwidth]{FIGURE6}}
    \caption{Contrasting a traditional chain of activation and one highlighting the structural role of mathematics in physics.}
    \label{FIG:6}
\end{figure}
Variations on this theme are used in the design of inquiry activities, highlighting the structural role of math in the modelling of physical concepts and situations. See, for instance, the design of activities on the revision of superposition that has been outlined in Section \ref{Sec:3.2.2}. Since the construction of an idealized representation of the physical situations discussed in our course has been performed in the creation and initial expression of the proto-model, we can skip this step in the structural chain of activation described in Fig. \ref{FIG:6}.

\emph{Operationalizing thought experiments}. Thought experiments may play a significant role in the presentation of modern physics, opening ``a unique window to the strange and unknown world of super-large and super-small scales''\cite{Galili2007}, where real experiments are excluded from regular classroom activity. This practice is employed for a variety of purposes \cite{Brown1991}: facilitating a conclusion drawn from available experiences and sources, rejecting a given explanation (Galileo's leaning tower of Pisa experiment), illustrating some of the counter-intuitive or unsatisfying aspects of a theory (Maxwell's demon, Schr\"{o}dinger's cat), finding new constraints that help guide positive modifications of a theory (Einstein's elevator).

A definition of thought experiment that is potentially productive in education has been proposed by Stephens and Clement \cite{Stephens2012}, who emphasize the process rather than the product: performing an untested thought experiment [..] ``is the act of considering an untested, concrete system (the `experiment' or case) and attempting to predict aspects of its behavior. Those aspects of behavior must be new and untested in the sense that the subject has not observed them before nor been informed about them.'' This emphasis on the relationship between the agent and the process allows us to widen the scope of thought experiments in educational practice: students making a prediction for an unfamiliar analogy, running a model for the first time, or applying a model to an unfamiliar transfer problem, are performing an untested thought experiment.

As regards creating and running a thought experiment, Gilbert and Reiner \cite{Gilbert2000} propose an analytical schema composed of six steps:
\begin{enumerate}
    \item posing a question or a hypothesis;
    \item creating an imaginary world, consisting of entities (objects, or mental creations which can be treated as objects) relating to each other in a regulated manner;
    \item designing the thought experiment;
    \item performing the thought experiment mentally;
    \item producing an outcome of the thought experiment with the use of the laws of logic;
    \item drawing a conclusion.
\end{enumerate}
While thought experiments may play a variety of roles, we focus on their use as a form of testing procedure. We qualify these procedures as \emph{humble thought experiments}, because they are not meant to achieve the purposes of historically significant thought experiments (e.g., Einstein's elevator); yet, their structure corresponds to that described by Gilbert and Reiner \cite{Gilbert2000} for a thought experiment, and their conduction may be within the reach of secondary school students. In order to design and analyze tasks in which students are actively engaged in running a thought experiment, we refer to the ISLE learning framework \cite{Etkina2015}, in which the phases of testing experiments are clearly identified and associated with scientific abilities. These abilities are described in rubrics introduced in Etkina et al. \cite{Etkina2006}, which are to be used as self-assessment instruments. For scientific abilities related to testing experiments, see Etkina \cite[Appendix B]{Etkina2015}. While the rubrics have been developed as a support in the design and running of experiments in the lab, they may be easily adapted to activities in which the instructor
\begin{itemize}
    \item provides the issue to explore, encouraging students to generate different hypotheses and to test them by running - step by step - a thought experiment specifically designed by the instructor (Section \ref{Sec:3.4.3});
    \item provides a hypothesis and asks students to design a thought experiment to test it, to run the thought experiment, and to draw appropriate conclusions on the initial hypothesis (Section \ref{Sec:3.4.4}).
\end{itemize}

%\vspace{5mm}

\subsubsection{Thought experiment: Description of an unpolarized beam in terms of photons} \label{Sec:3.4.3}

This activity marks the start of student work in modelling and of the use of worksheets in the course. At this point of the course, students have explored the polarization of macroscopic light beams by means of cheap experimental materials and discussed evidence on the detection of the photon and on its polarization after the passage through a filter.

The task is extending the model, describing unpolarized beams in terms of photons. Since this is the beginning of the theoretical modelling cycle, its purposes are manyfold: a) inviting students to engage in theoretical modelling tasks; b) discussing the basic heuristic principle that drives the modelling process: our hypotheses on the behavior of individual photons must be compatible with the classical quantitative laws for macroscopic beams; c) minimizing the axiomatic basis of the course; d) putting prior intuition in the service of learning QM.

The last purpose leads us to the genesis of the activity. In the first teaching experiments, we investigated spontaneous models of photon polarization (see Section \ref{Sec:5.4.2} for data), identifying a strong tendency to interpret unpolarized light as made of
\begin{itemize}
    \item photons polarized in different directions (sketching differently oriented segments: correct model);
    \item unpolarized photons (empty balls);
    \item photons polarized at all angles (stars).
\end{itemize}

As a result, we saw the possibility of settling the matter by means of a modelling activity in which all three hypotheses are elicited in a whole-class discussion, and the unproductive ones are ruled out by asking a prediction on the number of photons transmitted by a filter (possible reduction according to the first hypothesis, no reduction for the other ones) and comparing it with the classical law for unpolarized light (reduction to half). This kind of activity can be described as a testing experiment of theoretical nature or, in short, as a thought experiment.

In order to identify the correct hypothesis and discard the others, students need the following experiences: 1) Classical law for unpolarized light; 2) Classical law for polarized light: the intensity of a beam of polarized light passing through a filter with a different axis is reduced; 3)  polarization is a property of the single photon; 4) photons of a polarized beam are all polarized in the same direction;  5) light intensity grows with the number of photons emitted at a given instant. The source is the compatibility of the hypothesis on the single photons with the classical laws, that apply to beams of many photons. All is available to students except 5), which is intuitive and shortly discussed by the instructor, and the source, that is made explicit in the last task asked by the activity.

The relevant scientific abilities are based on a subset of those described by Etkina. They are tested for each hypothesis and can be used in the analysis of student answers as target performance of the main phases:
\begin{enumerate}[(a)]
    \item able to make an assumption on the action of the experimental device (filter with vertical axis) on the systems, based on the hypothesis;
    \item able to make a reasonable prediction (on the number of transmitted photons) based on the assumption;
    \item able to decide whether the prediction is compatible with the outcome prescribed by the macroscopic laws (reduction to half).
\end{enumerate}

The proposed task is a form of inquiry with elements of the \emph{structured inquiry} and of the \emph{self-directed inquiry} \cite{Llewellyn2012}, in which
\begin{itemize}
    \item 1\textsuperscript{st} phase: the instructor provides the issue to explore: how to describe an unpolarized beam of four photons
    \item 2\textsuperscript{nd} phase: students generate different hypotheses in a whole class discussion\ \\
        (the three spontaneous hypotheses elicited by research: photons polarized in different directions; photons polarized in all directions at the same time; unpolarized photons)
    \item 3\textsuperscript{rd} phase: students are asked to make an assumption on the action of a polarizing filter with vertical axis on the photon beam based on each hypothesis\ \\
        (respectively: reducing the number of photons and changing the polarization of the transmitted ones; eliminating all polarization properties that differ from $90^{\circ}$; adding a polarization property at $90^{\circ}$)
    \item 4\textsuperscript{th} phase: for each hypothesis, students are asked to make a prediction on the transmission process\ \\ (according to the first hypothesis, a possible reduction in the number of photons; according to the others, transmission with certainty)
    \item 5\textsuperscript{th} phase: students draw an appropriate conclusion on each hypothesis by comparing the corresponding prediction with the prescription of the macroscopic law, i.e., the reduction to half.\ \\
        (since for stars and empty circles no photon is absorbed, the only hypothesis left is the first one)
\end{itemize}


\subsubsection{Interpreting already known laws within the framework of new models: Malus's law} \label{Sec:3.4.2}

This activity is scheduled immediately after the thought experiment. The experiences and sources needed for the re-interpretation of Malus's law in terms of photons as a stochastic law of transition are all available. A support for this task is represented by the thought experiment itself, which draws student attention to the absorption of individual photons in the interaction with a filter.

Therefore, in order to help students develop a consistent interpretation of Malus's law, we found it sufficient to design a \emph{structured inquiry} \cite{Llewellyn2012} which, in its final form, is organized as follows:
\begin{itemize}
    \item students are engaged in an analysis of possible interactions between a photon prepared with polarization at $0^{\circ}$ and a filter, changing the angle $\phi$ of its axis in order to highlight the dichotomy between the determinate case ($\phi=0^{\circ}$ or $\phi=90^{\circ}$) and the uncertain one ($\phi=45^{\circ}$). In each situation, students are asked to predict whether a photon is transmitted or not, knowing that by definition it is not possible to transmit half photon. After that, the instructor shows the possible results by means of JQM screenshots.
    \item in the light of Malus's law (and implicitly of the heuristic principle), students are asked to make a prediction on the perspective of a photon meeting a filter (denoted as ``Horoscope of the photon''), taking into account the polarization property of the photon and the axis of the filter.
    \item students are asked to generalize their prediction to an arbitrary angle of polarization of the photon and axis of the filter.
\end{itemize}


\subsubsection{Thought experiment: Superposition as statistical mixture of component states?} \label{Sec:3.4.4}
While the thought experiment described in Section \ref{Sec:3.4.3} plays a platonic role \cite{Brown1991}, both destructive and constructive (ruling out some hypotheses and identifying the one that is compatible with available evidence), this thought experiment plays only a destructive role: excluding the possibility that quantum superposition can be interpreted as a statistical mixture.

The activity has two purposes: 1) helping students distinguish between superposition states and mixed states; 2) launching an epistemological discussion on quantum uncertainty and the completeness of QM (the Einstein-Bohr debate). The first one corresponds to addressing a persistent issue in the learning of the theory (see Section \ref{Sec:Intro}). For the second one, we need to draw students' attention to the consequences of interpreting a superposition as a mixture of the component states: it entails that a measurement on a single system is deterministic, that the observable to be measured is definite, and that the use of probability is due to lack of knowledge about the state of the system. By analyzing and rejecting the claim, we pave the way for introducing one of the main problems with the standard interpretation of QM: how is it possible that identical systems interacting with the same measurement device in the same conditions may give different and unpredictable results?

For discussing the topic, we propose to students a \emph{guided inquiry} \cite{Llewellyn2012}. The hypothesis is provided by the instructor, pretending it has been advanced by students of the previous years: ``the expression $|30^{\circ}\rangle = \sqrt{3}/2|0^{\circ}\rangle + 1/2|90^{\circ}\rangle$ means that the state $|30^{\circ}\rangle$ is composed of a random mixture of photons in the states $|0^{\circ}\rangle$ and $|90^{\circ}\rangle$, in a proportion corresponding to the square of the coefficients.'' In order to make the most of it for our second purpose, we guide students to unfold the physical consequences of the hypothesis with a series of questions. Then, we ask them to design a thought experiment to test the hypothesis, to run the thought experiment, and to draw the conclusions on the hypothesis.

The prerequisites for running the activity are the following experiences: 1) awareness of the probabilistic meaning of superposition in measurement, which is the first topic addressed in the introduction to superposition, 2) awareness that if we measure the observable to which the initial property of the system belongs ($30^{\circ}$, in this case), we find this property with certainty, which has been addressed in the revision of measurement. Source of the mixture hypothesis is the alternative interpretation of the referent of superposition: not an individual entity that is decomposed, but a composition of different entities. The strength of this hypothesis is that its goal is the same as that of the consistent interpretation (finding information on measurement), and that both associate component states and the square of the coefficients respectively with the possible results and probabilities.

The thought experiment is elementary: all we need is to direct the beam to a filter with axis at $30^{\circ}$. The prediction associated with the mixture hypothesis is that some of the photons will be absorbed (on average: 3/8). However, we know that at a macroscopic level, all the light polarized at $30^{\circ}$ will be transmitted by the filter. The hypothesis is false.

The scientific abilities that are associated with the conduction of the thought experiment are again a subset of those described by Etkina, and are expressed as in her article \cite[Appendix B]{Etkina2015}:
\begin{enumerate}[a.]
\item able to design a reliable experiment that tests the hypothesis;
\item able to make a reasonable prediction based on the hypothesis;
\item able to decide whether the prediction and the outcome disagree.
\end{enumerate}


\subsubsection{Identifying and interpreting mathematical constructs for describing physical situations and deriving new results: State vector and Entangled superposition} \label{Sec:3.4.5}

\emph{Identification and interpretation of the mathematical representation of the state}. The refinement of the first mathematical modelling activity has been briefly described in a previous work \cite{Pospiech2021}. The main issue concerned the discrimination between quantum states and measurable properties, which was challenging in the context of linear polarization. As a matter of fact, the correspondence between polarization properties, that are represented by directions in the plane of polarization, and quantum states, that are represented by abstract vectors with the same angular relations as the properties (e.g. $0^{\circ}$ $\rightarrow$ $|0^{\circ}\rangle$, $90^{\circ}$ $\rightarrow$ $|90^{\circ}\rangle$), suggests the identification of the state vector with the property of the system or, in general, with a physical quantity lying in the lab space. We started with a mathematization task: as the quantum state describes the behavior of the system in measurement, we asked which mathematical objects are suitable to describe the transition rule from one state to another by means of simple operations (sums/products) between the entities representing each state. In this way, we fostered an algebraic form of reasoning instead of a misleading visual analogy. Among algebraic entities, students chose vectors, since the transition rule depends on directions. The solution of the issue was refined by adding an interpretive question on the physical dimensions of the state vector and by asking about its nature only after introducing the state vector of the hydrogen-like atom, which is unequivocal in this respect.


\emph{Identification and interpretation of an entangled superposition of modes}. This task is addressed after the work on propagation, where students are led to conclude that the position of a photon between a direct and reversed calcite crystal (see Fig. \ref{FIG:8})
\begin{figure}
       \fbox{\includegraphics[width=\columnwidth]{FIGURE8}}
    \caption{Iconic representation of a ``which-way" experiment.}
    \label{FIG:8}
\end{figure}
is indefinite, to identify a new form of interference, and to build a full quantum model of a system for measurement and propagation. They are also expected to know the concept of component states ($|n, l, m \rangle$ and $|s\rangle$) and of product state, which coincides with the global state of the atom ($|n, l, m \rangle |s\rangle$).

In the context of the course, the ideal physical situation for a smooth and compact discussion of entanglement is the usual case of a photon incident on a calcite crystal followed by two detectors, which has been used from the start to introduce quantum measurement. The only difference is that here we do not focus on preparation or measurement (see Fig. \ref{FIG:7}),
\begin{figure} \centering
\begin{tabular}{|r|l|} \hline
    \includegraphics[width=.5\columnwidth]{FIGURE7a} &
    \includegraphics[width=.5\columnwidth]{FIGURE7b} \\ \hline
\end{tabular}
    \caption{(a) Preparation; (b) Measurement.}
    \label{FIG:7}
\end{figure}
but on the properties of the particle beyond the crystal and just before the absorption. It is worth noting the richness of this simple context: if we choose to discuss position, we address the wave-particle duality, if we discuss the connection between position and polarization, we are led to the entanglement of modes.

The activity we propose to students is a \emph{structured inquiry} \cite{Llewellyn2012}, in which we guide them to build the experiences and the sources for activating a mathematization and interpretation task. For this purpose, we need first to specify the relevant experiences - concerning the description of photon polarization after the crystal and of its measurement - and the sources - concerning the basic ingredients of the formal representation:
\begin{itemize}
\item Experience 1: the knowledge of the fact that polarization is indefinite after the crystal;
\item Experience 2: the knowledge of the fact that by measuring position you also measure polarization and vice versa;
\item Source 1: the mathematical representation of the spatial state of the photon in an elementary form;
\item Source 2: the product of spatial and polarization states;
\item Source 3: the physical interpretation of the component vectors and of the coefficients of a superposition state.
\end{itemize}
The two experiences are not yet available to students. Source 3 is already known from the study of superposition, source 1 and 2 are known as regards the hydrogen-like atom and need to be applied to the photon. The modes of representation are available to students from the discussion of measurement (iconic language of JQM) and of the state of the hydrogen-like atom (ket representation of product vectors).

The sequence of the activities is dictated by the structural chain of activation described in Section \ref{Sec:3.4.1}:
\begin{enumerate}
    \item exploration of the physical situation, highlighting those aspects that are relevant to the issue at hand (experience 1 and 2);
    \item introduction of the mathematical ingredients needed to derive the new construct (sources 1 and 2);
    \item mathematization: task for supporting the identification of the construct (source 3);
    \item interpretation: task for analyzing the new construct (rediscovering and deepening the content of the qualitative experiences).
\end{enumerate}

Experience 1: just before starting the part on entanglement, we use the definition of the possession of a property (a system possesses a property if and only if the probability to measure it is 1) to guide students to determine that, after the crystal, not only the position of a photon prepared in the superposition state $|\psi\rangle=1/\sqrt{2}|0^{\circ}\rangle+1/\sqrt{2}|90^{\circ}\rangle$  is indefinite, but also its polarization. As a matter of fact, there is 1/2 probability that the photon is collected by the detector in the $0^{\circ}$ channel or by that in the $90^{\circ}$ channel.

Experience 2: we direct student attention to a fact known from the revision of measurement, but that had not been emphasized until now: a measurement of position after the crystal is a polarization measurement and vice versa. In particular, each position property is correlated to a specific polarization property and vice versa.
%ELIMINATE FIG 46

Sources 1 and 2: after the second experience, position has clearly come into play. Therefore, we propose students to analyze the global state of the photon, using as a reference the description of the hydrogen-like atom. For this purpose, we introduce the spatial state of the photon in terms of three position (eigen)states:
\begin{itemize}
    \item localized immediately after the source: $|x\rangle$
    \item localized at the entrance of the detector on the ordinary channel at $0^{\circ}$: $|x_1\rangle$
    \item localized at the entrance of the detector on the extraordinary channel at $90^{\circ}$: $|x_2\rangle$
\end{itemize}
With these tools available, we ask students about two special cases: the global state of the photon at the time of its absorption by a detector, if it is prepared in $|x\rangle|0^{\circ}\rangle$, which is $|x_1\rangle|0^{\circ}\rangle$, and if it is prepared in $|x\rangle|90^{\circ}\rangle$, leading to $|x_2\rangle|90^{\circ}\rangle$.

The mathematizing question concerns how to write the global state of a photon prepared in the arbitrary state of polarization $|x\rangle(a|0^{\circ}\rangle + b|90^{\circ}\rangle)$ at the time of its absorption by a detector. Based on the results of the previous questions, a candidate can be $a|x_1\rangle|0^{\circ}\rangle + b|x_2\rangle|90^{\circ}\rangle$. Interpretive questions follow, asking about the properties of this state, which coincide with the results of Experience 1 and 2.

We conclude the section by observing that these activities allow us to immediately apply the conceptual and mathematical discussion of the entanglement of modes to a new physical situation: the purely quantum entanglement of different systems. A new mathematical modelling activity can be implemented by describing the physical situation of two photons emitted by parametric down-conversion. Information provided to students is the following: the possible results of polarization measurements on one of the photons, the effect of this measurement on the other photon, and the transition probability. Based on these elements, students can pass from an expression like $a|x_1\rangle|0^{\circ}\rangle \pm b|x_2\rangle|90^{\circ}\rangle$ to a structurally identical formula such as $a|0^{\circ}_1\rangle|90^{\circ}_2\rangle \pm b|90^{\circ}_1\rangle|0^{\circ}_2\rangle$.

\subsection{Implementing the epistemological principle} \label{Sec:3.5}

\subsubsection{Impact of the interpretive choices on the coherence of the design}  \label{Sec:3.5.1}
Here we describe how the interpretive choices are used to strengthen the internal coherence of the course. In what follows, first we recall the individual choice, then we explain how it affects the design.

\begin{itemize}
    \item a pure state provides complete information on the behavior of an individual quantum system (ruling out statistical interpretations);
\end{itemize}

In the course, we adopt a single system ontology. Therefore, we always refer to individual systems, favoring a probabilistic language over a statistical one. Ensembles of systems, identically prepared or not, are treated on a probabilistic basis, making use of the law of large numbers when appropriate. The implementation of this language choice played a productive role in the running of epistemic practices such as the interpretation of Malus's law in terms of photons, where a gradual transition from a discussion of quantum objects in terms oscillating between ensembles and individual systems, to a language carefully centered on the latter, was accompanied by an increasing rate of success (see Section \ref{Sec:5.4.1}).

\begin{itemize}
    \item an observable of a system has a determinate value if and only if the quantum state of the system is an eigenstate of the operator representing the observable (ruling out modal interpretations);
\end{itemize}

Since we do not use operators in the course, we do not introduce the terms ``eigenstate'' and ``eigenvalue''. However, the definition of the possession of a property by a system stands for the eigenstate-eigenvalue link: a system possesses a property if and only if the probability to measure it is $1$. The language of properties also helps suggest students a coherent interpretation of quantum superposition. As a matter of fact, while the superposition of linear polarization states is usually interpreted as a ``neither, nor'' situation (the system is in neither of the component states, and has neither of the corresponding properties), a superposition of two position eigenstates is sometimes interpreted as the system being ``in both places.'' However, the link between possessing a property and measuring it with certainty allows us to reconcile this case with the general frame: the system has neither of the component position properties. Its position is indefinite. For a productive use of this language in the development of an activity, see Section \ref{Sec:3.4.5}: after the passage of a photon through a calcite crystal, it is possible to prove that both position and polarization of the system are indefinite by using the same criterion.

\begin{itemize}
    \item the quantum description of processes includes two different types of state evolution: in the absence of measurement, the unitary evolution governed by the Schr\"{o}dinger equation; in measurement, the evolution prescribed by the projection postulate (ruling out no-collapse interpretations);
\end{itemize}

In the course, we always promote a clear distinction between measurement and propagation. While dealing with transitions in
measurement, we make use of iconic representations showing an initial situation, e.g., in which a photon has just been emitted by a
single-photon source (Fig. \ref{FIG:7}.a), and a final one, e.g., in which the photon has been absorbed and counted by a detector (Fig. \ref{FIG:7}.b). In these situations, we always direct student attention to the preparation and the measurement process.
The only exception occurs near the end of the course, when we discuss the ``which-way'' experiment by means of a photon beam directed to a device composed of a sequence of two calcite crystals, one reversed with respect to the other, followed by a filter and a detector (Fig. \ref{FIG:8}). This shift in focus is essential both in the discussion of the wave-particle duality and in that of entanglement (see Section \ref{Sec:3.4.5}).\\

\begin{itemize}
    \item in the construction of a full quantum model for propagation and measurement, we adopt a field ontology;
\end{itemize}

While we feel that this perspective can be perceived as plausible by students, who can make a connection with already familiar classical fields (especially in the case of a photon), ascribing a quantum field ontology to physical systems is a controversial operation \cite[e.g.,][pp. 133-135]{Passon2019, Norsen2017}. For this reason, we adhere to a cautious approach, suggesting students to model the system as a ``field of actual and potential properties.'' This expression means that the field describes the system in terms of properties it possesses (e.g., one might be a property of a spin component) and of ``potential properties that can possibly be actualized [...] through measurement processes'' \cite{Debianchi2013}. For more information on the concept of potential property and its transition to actuality, see also C. J. Isham \cite{Isham1995}. Based on the examination of the ``which-way" experiment, students are led to identify two further elements of revision in the concept of field: differently from a classical field, a quantum one displays a punctual interaction with detectors (we can identify the detector with which the interaction takes place), but this interaction affects the entire field at the same instant, i.e., in a non-local way.\\

\subsubsection{Structuring the discussion of epistemological themes} \label{Sec:3.5.2}

The three rules of correspondence naturally lend themselves to a discussion, respectively, of the completeness of the theoretical description, of indefiniteness and uncertainty, and of the measurement problem. Based on the examination of the ``which-way'' experiment and entanglement, it is also possible to add to the picture a discussion of the problem of locality.

Format, content and placement of the activities on the foundational debate need not only be instrumental to the implementation of the \textit{Epistemological Principle}, but also compatible with the educational level of the student population at hand, the structure of the learning path, and the duration of the course (12 hours).
The chosen format consists in a short introductory lecture given by the instructor, followed by a whole class discussion of the topic, which can be supported by pre-class reading assignments. This structure and sequence is in accordance with the method used to conduct worksheet-based activities (see Section \ref{Sec:4.3}), which represent the common thread of the course, and is adapted to the complexity of foundational topics by extending the time for the whole class debate, that needs to be at center stage, and by providing multiple forms of support. The texts are preferably selected among those works of leading scientists whose understanding does not require sophisticated mathematical or physical knowledge \cite[e.g., excerpts from][]{Schilpp1998, Bell2010}. Since the first three units of the course concern preparation, measurement, and their formalism, while propagation, wave-particle duality, and entanglement are addressed in the last unit, it is natural to discuss first the debates on indefiniteness, uncertainty, and completeness.

The first occasion to introduce the problem of indefiniteness and uncertainty may be the extension of the relations between properties to the case of position and velocity, in Unit 1, where students deal with the loss of the property of one observable in the measurement of the other (a limiting case of the uncertainty principle). Another occasion is offered by an activity of the third unit, which is designed to promote the distinction between a superposition state such as $|\psi\rangle=a|0^{\circ}\rangle+b|90^{\circ}\rangle$ and a mixture of a fraction of $a^2$ photons prepared in $|0^{\circ}\rangle$ and $b^2$ in $|90^{\circ}\rangle$, and to launch the discussion of related epistemological issues (see Section \ref{Sec:5.4.3} for a description of the goals and of the development of this activity). Completeness may be examined in Unit 2, during the discussion of the quantum state, or together with the other issues in the third unit.

In the initial versions of the course, we discussed the Heisenberg's microscope thought-experiment and Bohr's criticism of it in the first unit, in order to contrast a disturbance interpretation of the principle - where system properties are well-defined but it is not possible to measure them simultaneously with an arbitrary precision - with an interpretation in which they are not well-defined \cite{Tanona2004}. For the revision of this activity, see Section \ref{Sec:5.5}. In the third unit, instead, we discussed the debate between Einstein and Bohr on the completeness of quantum mechanics (e.g., hidden variables) and - again - the uncertainty principle, leaving out the part on the EPR paradox, which will be taken up when dealing with entanglement \cite{Schilpp1998}. The discussion of these issues was concluded in the fourth unit, where we proposed students, as a plausible interpretation of the wave-particle duality, an ontology based on the ``field of actual and potential properties.''

After the conceptual and mathematical examination of the polarization entanglement of two photons produced by parametric down-conversion, students are presented with the problem of non-locality, which is discussed only at a qualitative level. This discussion allows us to emphasize the importance of the foundational debate in the development of scientific knowledge. In the words of Alain Aspect, ``there was a lesson to be drawn: questioning the `orthodox' views, including the famous `Copenhagen interpretation', might lead to an improved understanding of the quantum mechanics formalism, even though that formalism remained impeccably accurate'' \cite[][p. xix]{Bell2010}. As regards technological development, it is possible to illustrate to students that a deeper understanding of entanglement is at the root of a second quantum revolution that is now unfolding \cite[e.g.,][]{House2018}, and that John Bell has been its prophet \cite{Bell2010}.

By having students work on the modelling and interpretation of the mathematical description of entanglement, we gain a further opportunity: ending the course with the discussion of the measurement problem. We illustrate the Schr\"{o}dinger's cat thought experiment and more in general the measurement problem,
indicating three lines of solution proposed by members of the scientific community: 1) accept the standard interpretation and modify the dynamics of the theory; 2) accept the dynamics and modify the standard interpretation; 3) accept both the standard interpretation and the dynamics, and try to show that their conflict can be ignored for all practical purposes \cite{Bub1998}.
As an instance of the first, we mention the Ghirardi-Rimini-Weber's theory \cite{Norsen2017}. The second is illustrated by hinting at the Everett's ``many
worlds'' interpretation \cite{Norsen2017}. The third is represented by the decoherence research program \cite{Schlosshauer2007}. We explain that decoherence provides an answer to the nonobservability of interference effects on macroscopic scales. However, outside the scope of decoherence remains the explanation of why only a particular outcome is realized in each measurement \cite{Schlosshauer2007}: one of the most significant open issues in modern physics, that affects also our proposed interpretation of the wave-particle duality.


\section{The course} \label{Sec:4}

\subsection{Structure of the course and types of activities} \label{Sec:4.1}
The course is designed for an optimal duration of 12 hours, even if some design experiments lasted only ten. The time devoted to each topic is organized as follows: four hours for Unit 1, two for Unit 2, two for Unit 3, four for Unit 4. Lessons are divided into two-hour blocks, that represent a compromise between the time required to engage secondary school students in a series of inquiry- and modelling-based activities they are not accustomed to, and the need to limit the cognitive load associated with the discussion of non-intuitive and novel content.

The structure of the path in terms of units, individual activities and their typology, is displayed in Fig. \ref{FIG:9}.
\begin{figure*}[!htpb]
       \fbox{\includegraphics[width=\textwidth]{FIGURE9}}
    \caption{The structure of the sequence as a composition of building blocks: units and individual activities. Two-colour boxes with a white half represent lectures aimed to implement the design principle associated to the other color. The other two-colour boxes represent active-learning strategies that play more than one role.}
    \label{FIG:9}
\end{figure*}
The figure is intended as a guide to design and instruction. As regards the former, it allows us to identify and monitor which principle is to be implemented and at what step, as well as the interplay between principles. As regards instruction, it gives guidance on the crucial aspects to focus on at each step, which could be knowledge revision, knowledge organization, epistemic cognition, etc. By examining the figure, it is possible to see that, while each unit builds on the previous ones, individual units can be described as self-contained. In accordance with the implementation of the first and the second design principles, each unit is concluded with a bird's-eye view across contexts on the revision, due to theory change, of the basic concepts and constructs addressed in it. However, except for Unit 1, all the others can be introduced by means of a driving question or a need emerging from previous units, that suggests students the importance to acquire further knowledge \cite{Wittmann2020}. For Unit 2 on the quantum state, the driving question is an issue implicitly raised in Unit 1: how to prepare/identify identical quantum systems if some of the observables are necessarily indefinite (activity explicitly displayed in figure). Unit 3 on superposition is associated with the need to quantitatively determine the possible results of measurements on hydrogen-like atoms, which have been qualitatively explored in Units 1 and 2. For Unit 4, the question is how to describe propagation with the mathematical tools introduced in Unit 3 for describing measurement.

The typology of each activity has been displayed in Fig. \ref{FIG:9} by means of a color code. By looking at the color distribution, it is evident that a large majority of the activities are linked to the implementation of the four principles. The following is a synthetic description of each type of activity:
\begin{itemize}
    \item \textit{Knowledge revision activity}: relying on the representation of the conceptual trajectory of a classical notion \cite{Zuccarini2022} with the aim to promote a consistent interpretation of its quantum counterpart (often structured in terms of interpretive tasks);
    \item \textit{Knowledge organization activity}: relying on the relations between properties/observables in order to build a coherent body of knowledge by using the same conceptual tools for the analysis of different physical situations;
    \item \textit{Epistemic practice}: inquiry- and modelling-based activity that mirrors the processes used in theoretical physics for building new scientific knowledge. We remind the reader that, by running this kind of activities, students build knowledge that is \emph{new for the learner}. In order to promote an awareness of the nature of each practice and of its significance in the development of the discipline, the activity is followed (less frequently: preceded) by what we call ``a historical snapshot.'' It consists of a two-minute lecture on the practice and on a historically significant example of how it has been used by theoretical physicists in the building of classical physics knowledge (Fig. \ref{FIG:3} includes the summary of a historical snapshot on each practice).
    \item \textit{NoS and HoS debate}:  discussion of issues concerning the scientific epistemology of QM and the historical development of the discipline. The activity involves a ten-minute lecture with the aid of the slide presentation followed by a whole class discussion. Except for ``the troubled history of light quanta'' (Fig. \ref{FIG:9}, activity 1.3), which is instrumental to introduce the discrete nature of electromagnatic radiation, and to highlight the tangled and non-linear relation between experiment and theory in scientific development \cite{Kragh1992}, the other activities of this kind have been already described in Section \ref{Sec:3.5.2};
    \item \textit{Empirical exploration}: of the polarization of macroscopic light beams by using cheap experimental materials such as polarizing filters and calcite crystals. During the exploration, their action on the beams is visualized on the wall by means of a overhead projector (see Fig. \ref{FIG:10} and Fig. \ref{FIG:11}). The activity is conducted as a form of \emph{demonstrated inquiry} \cite{Llewellyn2012}: the instructor poses questions to the students, soliciting input in the design of the exploration, encouraging them to form hypotheses, to make predictions, and to explain the results. Three empirical explorations are scheduled at different points of the course: right at the start of the learning path, to introduce the phenomenology of the interaction of light with polarizing filters (Fig. \ref{FIG:9}, activity 1.1); after the probabilistic interpretation of the Malus's law, to present the phenomenology of birefringence, thus providing the experience needed for the modelling of quantum measurement at a microscopic scale (Fig. \ref{FIG:9}, activity 1.8); at the beginning of Unit 4, to present a simple form of ``which-way'' experiment, paving the way to the discussion of propagation and entanglement (Fig. \ref{FIG:9}, activity 4.1).
\end{itemize}
\begin{figure}[!htpb]
       \fbox{\includegraphics[width=\columnwidth]{FIGURE10}}
    \caption{The active nature of polarizing filters: by inserting a third filter between two filters with perpendicular axes, we observe an increase in transmitted intensity.}
    \label{FIG:10}
\end{figure}
\begin{figure} \centering
\begin{tabular}{|r|l|} \hline
    \includegraphics[width=.42\columnwidth]{FIGURE11a} &
    \includegraphics[width=.58\columnwidth]{FIGURE11b} \\ \hline
\end{tabular}
    \caption{(a) The phenomenon of birefringence; (b) The outgoing light beams are polarized, as shown by adding a filter on the crystal.}
    \label{FIG:11}
\end{figure}

\subsection{Instruments}
The instruments we use in the course are the following: 1) worksheets, 2) cheap experimental tools, 3) the JQM environment for simulated experiments, 4) a specific use of language, 5) a slide presentation, 6) homework: reinforcement exercises, reading assignments, and slides used in the previous lessons.

Worksheets are designed to emphasize written explanations of student reasoning. In this course, they represent the common thread underpinning the development of learning from beginning to end, and the main instrument for collecting data on student learning. For each unit, we designed a worksheet of two-three pages of tasks. Each worksheet is divided into blocks with a general goal that is split into conceptual micro-steps addressed in different questions \cite{McDermott2013}. With the exception of lectures and of former worksheet questions that have been converted into oral ones, the sequence of activities displayed in Fig. \ref{FIG:9} mirrors the structure of the worksheets. All worksheets but the last one end with a block containing one or more concept revision tables (Fig. \ref{FIG:9}, activities 1.18, 2.10, and 3.9).

As we have seen in the last section, the exploration of the phenomenology of light polarization at a macro-level is performed
thanks to kits including an overhead projector, passive filters, polarizing filters (Fig. \ref{FIG:10}), calcite crystals and tracing paper with a black dot in order to examine the phenomenon of birefringence (Fig. \ref{FIG:11}).

We already introduced the JQM environment for simulated experiments in Section \ref{Sec:3.1}. With one notable exception, in the current version of the course, the adoption of this instrument is limited to its visual code, which is used both in the slide presentation and in the tasks proposed to students (see, e.g., Fig. \ref{FIG:15}). This code is instrumental to building a highly idealized environment designed to help students focus on essential theoretical aspects. The exception concerns the ISLE-like ``which-path'' activity, in which the simulation plays the role of a testing experiment  (Fig. \ref{FIG:9}, activity 4.2).

Language in the slide presentation and in questions has been structured according to the following guidelines: first, the adoption of the language of ``properties'' and of their relations in order to provide a unified framework for describing measurement, state, and superposition at a point in time; second, the use of colloquial language and student sketches in whole class discussions (as in Fig. \ref{FIG:39}) and, when possible, in questions (e.g., describing activity 1.7 in terms of a ``horoscope of the photon'', as illustrated in Section \ref{Sec:5.4.1}).

Every aspect of the lessons (lectures, worksheet questions, correct answers, discussion of the results of empirical explorations) is supported by slides on the multimedia board or on a projector. At the end of each lesson, the slide presentation used in the classroom is made available to students in the form of a pdf file.

As regards homework, we already discussed reading assignments in Section \ref{Sec:3.5.2}. Homework exercises contain further interpretive questions (e.g., on the physical meaning of the sign of a superposition) and questions for deepening the development of specific aspects of the model (e.g., mathematically deriving the reduction to half of the intensity of an unpolarized beam of photons passing a filter).

The combined use of worksheets, slides, and of an instructor diary reporting student comments and reactions, offered us the possibility to monitor their learning paths during design experiments, identifying unsolved difficulties in the specific question or slide in which they were elicited. As a result, these instruments helped the researchers in their investigation of student ideas and in the refinement of the course.


\subsection{Methods: conduction of worksheet-based and oral activities} \label{Sec:4.3}
During the whole course, students are actively engaged in a modelling process. Therefore their role and that of the instructor are defined in accordance with inquiry-based learning approaches. Worksheet activities represent the bulk of the lesson time, and are conducted in the following way:
\begin{itemize}
    \item \textit{Phases of the activity}: 1) preparation: the instructor displays the slide containing the worksheet items, reads them, and specifies the number of minutes allotted for the writing assignment (depending on its difficulty); 2) writing assignment: each student on her/his individual worksheet, discussion with the deskmate is allowed and encouraged; 3) whole class discussion; 4) answer of the instructor: displayed on the slide and, if necessary, discussed.
    \item \textit{Role of the instructor during the writing assignment}: walking through the class, listening, observing, checking the progress of each student, answering clarification requests and posing stimulus questions (when realizing that some students are stuck) to help them overcome difficulties and to support their reasoning.
    \item \textit{Role of the instructor during the whole class discussion}: facilitating the discussion, e.g., asking a student to share her/his answer, inviting those who have given different answers to express their point of view in the attempt to convince their peers, asking further clarifications if the explanation is not fully clear to the other students, and going on in this process until a consensus has been reached.
\end{itemize}

Oral questions are displayed on a slide and directly addressed in a whole class discussion, after which the answer of the instructor is shown.

\section{Cycles of refinement} \label{Sec:5}

\subsection{Design-Based Research: data collection and analysis}

The course has been refined in cycles of testing and revision conducted in the framework of Design-Based Research (DBR). This framework is a collection of approaches devised for ``engineering'' teaching and learning sequences, and systematically studying them within the context defined by practices, activities and materials - in short, by the means - that are designed to support that learning \cite{Bakker2015}. DBR consists of cycles composed of three phases: preparation, design experiment, retrospective analysis. The results of a retrospective analysis feed a new design phase. When patterns stabilize after a few cycles, the instructional sequence at hand can become part of an emerging instruction theory.

The course has been experimented in classroom contexts of various nature.

The first one is the Summer School of Excellence on Modern Physics, held every year at the University of Udine, Italy. It consists of a one-week full immersion program in modern physics topics. The course was held in the years 2014-2018. Participant students ranged from a minimum of 29 in 2014 to a maximum of 41 in 2015. They were selected among a large number of applicants from a wide range of Italian regions. All of them had just completed the penultimate year of secondary school.

The second context consists of regular classrooms from Italian secondary schools. The course was held in Liceo Statale Corradini, in the city of Thiene, in November 2018 and in Liceo Scientifico Statale Alessi, in the city of Perugia, in February 2019. In the Italian system, Liceo is a type of school attended by students who intend to continue their studies in university. The design experiment involved three classes of the final year from Liceo Corradini, for a total of 61 students, and two classes of the same year from Liceo Alessi, for a total of 39 students.

The third context concerned self selected students from Liceo Scientifico Galilei, in the city of Trieste, at the end of March 2019. The course was offered as an optional study program, and was attended by 18 students.

In this work, we do not test the effectiveness of the course, but the refinement of individual activities. For this purpose, the differences between the three kinds of student population did not represent an issue. Future directions include the analysis of a pre-post-test administered in regular classrooms. Here we report on cycles of refinement concerning a set of activities chosen to illustrate the implementation of each of the four principles of design. For each cycle, we describe the preparation phase, the worksheet items used to implement the design, and the retrospective analysis of design experiments. Except for a limited number of recently added activities, cycles were iterated until patterns stabilized.

Data sources consist of written answers to worksheet questions, occasionally enriched by notes reported in the instructor diary during design experiments. Data were analyzed for correctness and for student lines of reasoning, since both informed the revision of the activities. The second type of analysis was conducted according to qualitative research methods \cite{Erickson2012}: the identification of crucial conceptual content and the examination of literature on learning difficulties in QM guided the building of a-priori categories. Then, based on conceptual elements introduced by student answers, the categories were revised. This process led to the identification of clusters and coherence elements in student reasoning.

Since the sample changed from experiments to experiment, in order to improve readability and to enable comparison, the rates of answers as regards both correctness and student reasoning are reported by means of percentages.

\subsection{Knowledge revision activities}  \label{Sec:5.2}

\subsubsection{Measurement} \label{Sec:5.2.1}

\emph{Goal}. Revision of the concept of measurement, intended as consistent interpretation of the interaction of a quantum system with the measurement device, and recognition of the conditional nature of the process. In the context of polarization, this revision presupposes the interpretation of the absorption of a photon by a filter or counter following a crystal in terms of a transition in property (in the classical-like case: of its retention).

\emph{1\textsuperscript{st} Version - Description of the activity}. In the initial version of the course (Summer School of Excellence, 2014, 28 students), we decided to discuss measurement in the context of the interaction of a photon with a filter followed by a counter, since the activity was scheduled right after an extensive work with polarizing filters, both at a macroscopic level and in terms of photons (see Section \ref{Sec:5.4.1}). The worksheet block designed to support the conceptual development of students is summarized in Fig. \ref{FIG:13}.
\begin{figure*}[!htpb]
       \includegraphics[width=\textwidth]{FIGURE13}
    \caption{Worksheet block on quantum measurement: 2014 version. The correct answers are in green.}
    \label{FIG:13}
\end{figure*}
First, in order to focus on the relation between the absorption by a filter with a given axis and the property of the photon, we asked them to find what property corresponds to this outcome in the determinate, classical-like case (item \textbf{C1}). Then, we examined the general case, providing students with an interpretation of the transmission of a photon by a filter in terms of retention/acquisition of a polarization property, and asked them to interpret the absorption in the same conceptual frame (\textbf{C2}). Finally, we gave a definition of quantum measurement as the determination of this property, and asked about its conditional nature (\textbf{C3}).

\emph{1\textsuperscript{st} Version - Results}. While 75\% of the students consistently answered \textbf{C1}, in \textbf{C2} only 18\% interpreted absorption in terms of acquisition of a property in the direction perpendicular to the axis of the filter or of its retention. Most of them simply turned the sentence used for transmission into the negative form: ``the photon had not - or had not acquired - a property in the transmission axis $\Rightarrow$ it is not transmitted.'' As to \textbf{C3}, its results were again affected by the insufficient conceptual construction achieved in the previous step. Only 2 students gave a consistent answer: ``It depends: [it does] not [acquire a new property] if the photon's property is $=$ or $\bot$ to the permitted direction, otherwise it acquires one of these two properties.'' Three answers were incomplete: students correctly stated that measurement is passive if the photon's initial property coincides with the axis of the filter or is orthogonal to it, but they did not discuss how the property may change if the angles are different. Most students focused only on transmission.

\emph{2\textsuperscript{nd} Version - Description of the activity}. A year later (Summer School of Excellence, 2015, 41 students), we revised the design, referring to measurement from the start in terms of outcome-properties, and
- most important - adding an iconic support: two diagrams depicting the possible transitions of the initial property in case of transmission and of absorption (see Fig. \ref{FIG:14}). In this way, we intended to suggest students to interpret the transition associated with absorption in the stochastic case as a generalization of what it is known to happen in the determinate case.
\begin{figure} \centering
\begin{tabular}{|r|l|} \hline
    \includegraphics[width=.5\columnwidth]{FIGURE14a} &
    \includegraphics[width=.5\columnwidth]{FIGURE14b} \\ \hline
\end{tabular}
    \caption{Figures added to the worksheet block in 2015 Summer School of Excellence, Udine. The diagrams depict a polarizing filter with vertical axis and the possible transitions in the polarization property. Left: two photons prepared respectively with horizontal and vertical polarization: determinate outcome. Right: one photon prepared with polarization at $45^{\circ}$: stochastic outcome.}
    \label{FIG:14}
\end{figure}

\emph{2\textsuperscript{nd} Version - Results}. This support was not effective. In the probabilistic case, only 17\% interpreted absorption in terms of acquisition of a property in the direction perpendicular to the axis. As regards the conditionality of the active nature of measurement, only 5\% of the students interpreted it consistently: ``for uncertain interactions, measurement determines the property acquired by the system'', while 17\% said that measurement can be active or not, but without specifying when: ``the initial property may change in some cases.'' Even worse, some students wondered how the whole situation could be described as a measurement, and not as ``just a weird interaction altering the property''.

\emph{3\textsuperscript{rd} Version - Description of the activity}. Given the need to provide students with a context where the results of the interaction with the device can always and clearly be described in terms of the acquisition/retention of outcome-properties, in 2016 (Summer School of Excellence, 27 students) we resolved to use a measurement device composed of a birefringent crystal and two counters. As preparatory activities, we designed an empirical exploration of the physical situation both at the macroscopic scale, performed with real instruments (Fig. \ref{FIG:9}, activity 1.8), and at the single photon level by means of a predict-observe-explain sequence \cite{White1992} (Fig. \ref{FIG:9}, activity 1.9). The latter was conducted with the aid of JQM screenshots on single photons prepared with properties at $0^{\circ}$, $90^{\circ}$, $45^{\circ}$ that go through a measurement device composed of a calcite crystal with $0^{\circ}$ and $90^{\circ}$ channels and a detector on each one (see Fig. \ref{FIG:7}).

After that, students were administered a revised worksheet block on measurement (see Fig. \ref{FIG:15}).
\begin{figure*}[!htpb]
       \includegraphics[width=\textwidth]{FIGURE15}
    \caption{Worksheet block on quantum measurement: 2016 version.}
    \label{FIG:15}
\end{figure*}
Item \textbf{C1} is an elementary form of thought experiment designed to promote a consistent interpretation of the absorption of the photon by the counters in terms of a transition in polarization property. Item \textbf{C2} is not represented in the figure, sice it is not related to the issue at hand, and will be discussed in Section \ref{Sec:5.4.1}. \textbf{C3} is the most important item, promoting a consistent framing of the classical-like measurement, in which random mixtures of photons with $0^{\circ}$ or $90^{\circ}$ polarization are sent to a birefringent crystal with channels corresponding to the same properties. In \textbf{C4}, after giving a definition of quantum measurement as a generalization to the new case in which the system does not possess an outcome-property, we ask about the conditional nature of measurement.

\emph{3\textsuperscript{rd} Version - Results}. As regards \textbf{C1}, 85\% of the students identified the outcome-properties of a photon prepared at $45^{\circ}$ (probabilistic case) as $0^{\circ}$  and $90^{\circ}$. Most students designed consistent experiments to prove their statement, using one filter on each channel, one with axis at $0^{\circ}$ and the other at $90^{\circ}$, either corresponding to the polarization associated with the channel (all photons pass the filters), or the opposite case (all absorbed by the filters). The experiment with absorbed photons is better designed than the other, as we get a transition in photon polarization directly on the filters, while in the other case the two filters are transparent to the photon and the transition takes place in the counters. Still, both lines of reasoning were productive. All students but one (26) interpreted the situation described in item \textbf{C3} (determinate interaction) as a classical measurement, half of them explicitly adding we get to know the initial property according to the channel/detector in which the photon is counted. As regards the revision of the concept of measurement (\textbf{C4}), 78\% gave consistent answers, recognizing the conditional nature of quantum measurement and its constraints, e.g. ``if the property does not coincide with the outcome-properties, the system collapses into one of them. If it coincides, measurement does not change it.'' Even more important, no student showed a reluctance to interpret the interaction as a measurement both in the determinate and the stochastic case, probably due to the productive framing promoted by item \textbf{C3}.

Given the high rate of success (see the progression in Fig. \ref{FIG:16}), in the following design experiments the items have been converted into oral questions, since we intended to focus on the refinement of later parts of the course.
\begin{figure}[!htpb]
       \includegraphics[width=\columnwidth]{FIGURE16}
    \caption{Worksheet results in 2014-2016. In this graph and in the next ones, S.R.C. means student's response correctness.}
    \label{FIG:16}
\end{figure}

\subsubsection{Superposition} \label{Sec:5.2.2}
\emph{Goal}.  Consistently contrasting the features of the familiar forms of vector superposition (of forces and waves) with quantum superposition.

\emph{1\textsuperscript{st} Version - Description of the activity} A end-of-unit table on superposition was first administered to students in the Summer School of Excellence 2017 (32 students). At that stage of development of the course, no discussion of the superposition of states of the hydrogen-like atom had been designed yet. Therefore, the correct answers included in figure have been structured according to the learning goals of activities 3.1 and 3.4 on the superposition of polarization states. The tasks are represented in Fig. \ref{FIG:18}, and correspond to activity 3.9.
\begin{figure*}[!htpb]
       \includegraphics[width=\textwidth]{FIGURE18}
    \caption{End-of-unit table on superposition filled with correct answers: 2017 version.}
    \label{FIG:18}
\end{figure*}
Since students generally find it difficult to discriminate between system properties and state vectors \cite{Pospiech2021}, the first row of the table was devoted to this issue. Thus, we seized the occasion to reinforce the understanding of a fundamental difference between classical vectors (primarily used to represent physical quantities and lying in the Euclidean space) and the state vector (defined in an abstract Hilbert space), as displayed in Fig. \ref{FIG:1}.b. The second row concerns the constraints, the third the goal, the fourth the procedure, in which we highlight the need to decompose one entity (the resultant).


\emph{1\textsuperscript{st} Version - Results}. All students identified the forces depicted in figure as physical quantities, and 88\% did the same for (the amplitude of) waves. In the quantum box, 84\% of them answered ``no'', sometimes adding consistent explanations: ``state vectors are unit vectors with no measurement units'' (15\%), ``they are dimensionless'' (15\%), ``they express probability'' (9\%). A small minority gave inconsistent interpretations of the concept of state, seen either as ``a percentage'', ``a set of values'', a ``dimensionless number.''
Also the identification of constraints - or their absence -, discussed in the second statement, did not pose a serious challenge. All students but two agreed that superposition of forces and of waves has a physical meaning independently of the number of component vectors and of the angles between them. In the quantum case, a large majority of the students consistently identified at least one constraint (72\%). Of them, 18\% gave a complete answer (e.g.: ``No, it is necessary that the vectors are 2 and are orthogonal, because they are the only mutually unacquirable conditions''), 36\% focused only on orthogonality, the rest stated there must be no more than two vectors, the rest mentioned the mutual unacquirability of the corresponding properties. Among the students who did not identify any constraint, most recognized the importance of the angle and/or of the number of vectors in quantum superposition.

The interpretation of the last two statements on the goal and the referent of superposition proved the most difficult for students. As to the goal, while in the classical contexts all students agreed that ``The goal of superposition is to determine the resultant'', in the quantum one, only 43\% consistently explained why this is not the case. The remaining students either did not answer (9\%) or agreed with the statement also in the quantum case. This shows that the classical framing of superposition tends to be transferred to the quantum case, even after specific activities designed to promote a consistent interpretation of the goal of the procedure.
The fourth statement involved the procedure and the referent at the same time: ``for obtaining physical information, the \emph{only} procedure  is decomposing the resultant into orthogonal components.'' This task elicited substantial issues even in the classical contexts. We thought that, by emphasizing the term ``only'' and by giving this task after assessing whether the goal of each form of superposition is adding vectors to ``determine the resultant'', students would come to the conclusion that this statement does not fit the superposition of forces and waves. However, 45\% of the students put at least a cross also in a classical box. Our best guess on the reasons behind this choice is that students interpreted the statement as proposing a role \emph{also} for decomposition, and not an exclusive one.
The assessment of the quantum case showed that the activities included in Unit 3 had not been sufficient to promote a solid understanding of state superposition as orthogonal decomposition. Only 43\% of the students agreed with the statement, providing a consistent explanation, e.g., ``yes, its goal is to calculate the probability of alternative results.'' Another 22\% also answered yes, but with inconsistent or irrelevant explanations.

\emph{2\textsuperscript{nd} Version - Description of the activity}. These results prompted us to revise both the previous activities of Unit 3 and the table. In order to strengthen the support for identifying the referent of quantum superposition as one physical entity - the state of the system - we added an activity relying on embodied cognition (see Zuccarini and Malgieri \cite{Zuccarini2022}). In that year, we also added to the picture the superposition of eigenstates of the hydrogen-like atom in terms of quantum numbers, thus providing a more general context. Finally, we revised the wording of the statements and their order (Fig. \ref{FIG:19}).
\begin{figure}[!htpb]
       \includegraphics[width=\columnwidth]{FIGURE19}
    \caption{Table statements: from 2017 to 2018. In red, the old wording that has been modified. In green, the new one}
    \label{FIG:19}
\end{figure}
We moved the fourth statement to the first row and changed it radically in order to address only the referent of superposition (one physical entity in the quantum case), leaving the discussion of the goal to the third statement. The awareness of the subtle conceptual issues related to the dual role of superposition in classical contexts (determining the resultant and decomposing it) brought to a slight change in the third statement (from \emph{the} to \emph{a} ``goal of superposition is to determine the resultant'').

\emph{2\textsuperscript{nd} Version - Results}. The new design was experimented in the Summer School of Excellence 2018 (30 students). The issues on the classical boxes were successfully solved: 87\% recognized the vectors as physical quantities, and all students but one put a cross in the other boxes.
As regards quantum superposition, a comparison of the consistent results obtained in 2017 and 2018 is displayed in Fig. \ref{FIG:20}.
\begin{figure}[!htpb]
       \includegraphics[width=\columnwidth]{FIGURE20}
    \caption{Boxes on quantum superposition: correct answers with consistent explanations in 2017 and 2018}
    \label{FIG:20}
\end{figure}
A strong improvement is evident in the answers in the two boxes on the referent (first on the right) and the goal (third on the right) of quantum superposition. In both boxes, 87\% of the students answered correctly, providing consistent explanations. In the one on the referent, 53\% did not limit themselves to state that quantum superposition concerns one entity, but interpreted the process as a decomposition of this entity. As to the goal, beside recognizing that determining the resultant is not among the goals of this form of superposition, most students identified its objectives as ``calculating [transition] probabilities'' (37\%) or ``decomposing the state vector'' (30\%).
With relation to the other two statements, the results of the two design experiments were comparable, with a slight decrease in performance in 2018. Students assessed the second statement with similar reasoning in both years. As regards the statement on constraints, we need to take into account the greater complexity introduced by the superposition of states of the hydrogen-like atom (where, for bound states, we can have a countably infinite number of components, all orthogonal to one another). Most students discussed the statement by referring to the context of polarization, e.g. ``In polarization, for instance, we can have up to two components $\rightarrow$ two values'' (47\%). However, some students gave global explanations by connecting the number of vectors to the spectrum of the measured observable: ``No, it depends on the number of values that a property can assume'' (13\%), which is true in the absence of degeneracy.


\subsection{Knowledge organization activities} \label{Sec:5.3}

\subsubsection{Introducing the relations between properties} \label{Sec:5.3.1}
\emph{Goal}. Supporting students in acquiring the appropriate perspective for the introduction of the relations between properties: analyzing measurement in terms of loss, retention and acquisition of properties.

\emph{1\textsuperscript{st} Version - Description of the activity}. Identified in Fig. \ref{FIG:9} as activity number 1.11 , it was administered during the Summer School of Excellence 2018 (30 students). Students are asked to assess four statements concerning the retention, loss, and acquisition of polarization properties in measurement, specifying whether the event described in each statements occurs and, if it does, under which conditions (see Fig. \ref{FIG:21}).
\begin{figure*}[!htpb]
       \includegraphics[width=\textwidth]{FIGURE21}
    \caption{Identifying the possible relations between properties: Summer School of Excellence, Udine, 2018 version.}
    \label{FIG:21}
\end{figure*}
Since both stochastic and determinate measurements are governed by Malus's law, this version relied exclusively on its interpretation in terms of photons, leaving out any details on the measurement device.


\emph{1\textsuperscript{st} Version - Results}. Data are displayed in Fig. \ref{FIG:23}.
\begin{figure}[!htpb]
       \includegraphics[width=\columnwidth]{FIGURE23}
    \caption{Results of the task number 1.11: Summer School of Excellence, Udine, 2018 version. In this graph and in the next ones, P.S. means percentage of students.}
    \label{FIG:23}
\end{figure}
An answer is considered \emph{consistent} if its content matches that of the correct one reported in Fig. \ref{FIG:21}, both in terms of outcome-properties and (if needed) of angular relations between $P_a$ and $P_b$. \emph{Partial} means that the student has identified one of the conditions for the occurrence of the event described in the statement but not all of them, or that has added unneeded conditions. For instance, in statement 3 a student may recognize that $P_a$ is not an outcome-property, but neglect the fact that $P_b$ must be one (e.g., ``The system always loses $P_a$, unless it is an outcome-property''), while in statement 1 may add an angular relation between $P_a$ and $P_b$, when all you need is that $P_a$ is an outcome-property (e.g., ``I must use a filter parallel to $P_a$ and $\perp P_b$'').

By looking at the histogram, we see that most students consistently discussed statement 1 and 4, half of them correctly identified the conditions for statement 2, while statement 3 on incompatibility was largely unsuccessful. From the content of inconsistent answers, we see that often students interpreted the task differently from what we intended: e.g., focusing on the mathematical description of Malus's law (e.g., ``Yes, if $p(P_a \rightarrow P_b) = \cos^2(P_a-P_b)$'') instead of specifying conditions on the outcome properties. This issue with the framing of the task is mirrored in the small number of partially correct answers. In addition, the lack of details on the measurement device, which was meant to guide the students towards an abstract and general perspective, did not discourage them to use concrete tools to support their reasoning. The relative majority of students referred to polarizing filters (35\%), others referred to crystals (11\%), while another 33\% reasoned abstractly on outcome properties and angles, and the remaining 21\% answered without giving explanations (e.g., in statement 4: ``Never'', ``No'', ``Impossible''). Probably, the reference to Malus's law in the item text, which had been previously discussed in the context of the photon-filter interaction, activated the use of this resource. In the case of statements 1, 2, and 3 (statement 4 was not an issue), the reference to filters was mostly unproductive: only 44\% of those who mentioned filters provided a consistent answer, versus 60\% for crystals and 64\% for abstract answers. Last, by reasoning on filters, an old issue reappeared: the difficulty to interpret the absorption of the photon as a consequence of a transition in property (see Section \ref{Sec:5.2.1}). For instance, in response to statement 1: ``The only outcome-property is $P_a$''; to statement 2: ``Only $P_a$ is an outcome property.''

\emph{2\textsuperscript{nd} Version - Description of the activity}. Consequently, we removed any reference to Malus's law, using calcite crystals as a context, and limiting the arbitrariness of the role of $P_b$, by making it one of the two outcome-properties of the measurement. The statements assessed in the task were left unchanged, but the introduction was radically modified (see Fig. \ref{FIG:24}). The activity was experimented in Liceo Statale Corradini, Thiene, in november 2018. It was administered in two of the three classrooms involved in the course, for a total of 40 students. In the third classroom, the task was discussed orally for lack of time. The correct answers are exactly the same as in Fig. \ref{FIG:21}, except for statement 3, where there is no need to identify $P_b$ as an outcome-property (detail specified in the introduction).
\begin{figure}[!htpb]
       \includegraphics[width=\columnwidth]{FIGURE24}
    \caption{Identifying the possible relations between properties, introduction of the worksheet block: Liceo Statale Corradini 2018 version.}
    \label{FIG:24}
\end{figure}

\emph{2\textsuperscript{nd} Version - Results}. Data are displayed in Fig. \ref{FIG:25}. The rate of consistent answers is almost identical as in the Summer school, except for statement 3, in which it increases from 30\% to 53\%. This represents a definite achievement, if we consider that Summer School students are selected among a large number of applicants from all over Italy, while the design experiment at Liceo Statale Corradini involved regular classrooms.
\begin{figure}[!htpb]
       \includegraphics[width=\columnwidth]{FIGURE25}
    \caption{Results of the task number 1.11: Liceo Statale Corradini, Thiene, 2018 version.}
    \label{FIG:25}
\end{figure}
In addition, students from the Liceo Statale Corradini generally interpreted the task as intended by the researchers, which is mirrored in the much higher rate of partially correct answers, and practically all students adopted an abstract perspective. This shows that the context and the visual representation of polarization measurement by means of a crystal and two counters favored the use of a global approach to the task. Reasons of failure are the difficulty to identify all the conditions for the occurrence of an event, both as regards the identification of the outcome properties and of the possible angle between $P_a$ and $P_b$. Difficulties of both kinds are noticeable in this answer to statement 4: ```Yes, $P_a$ is an outcome-property, $P_a \perp P_b$.'' The improvement is even more evident if we compare the sum of consistent and partial answers (see Fig. \ref{FIG:26}).
\begin{figure}[!htpb]
       \includegraphics[width=\columnwidth]{FIGURE26}
    \caption{Comparison of the results of the task number 1.11.}
    \label{FIG:26}
\end{figure}

\subsubsection{Extending the use of the relations between properties to other contexts} \label{Sec:5.3.2}
\emph{Goal}. Qualitatively managing measurement at a global level (position and velocity) and in the context of the hydrogen-like atom.

\emph{General description of the activities}. The tasks are organized into two separate worksheet blocks, one on position and velocity, the other on the hydrogen-like atom. Here, we discuss the results obtained in the Summer School of Excellence 2018 (30 students) and in the three classrooms of Liceo Statale Corradini (61 students). The worksheet blocks used in the two design experiments are identical, except for a question added to the second block in the design experiment at Liceo Statale Corradini.

\emph{Description of the activity on position and velocity}. The worksheet block is displayed in Fig. \ref{FIG:27}.
\begin{figure*}[!htpb]
       \includegraphics[width=\textwidth]{FIGURE27}
    \caption{Using the relations between properties to discuss ideal measurement at a global level (position and velocity): worksheet block administered in the Summer School of Excellence, Udine, 2018 of Excellence and Liceo Statale Corradini, Thiene, 2018.}
    \label{FIG:27}
\end{figure*}
Basically, students are required to deduce that all the properties of the same quantities are mutually
unacquirable, and to qualitatively determine the results of the measurement of an observable ($x$) on a system which has a property of an incompatible observable ($v$) in terms of change in properties and type of process, either determinate or stochastic. For the last task, the definition of the relations alone does not offer complete knowledge on the result of measurement, since it only mentions the \emph{possible} acquisition of a given property of $x$. In order to conclude that a single property of the measured quantity is always found, students must activate resources on the revision of measurement: this is the case both in CM and QM.

\emph{Results of the activity on position and velocity}. The comparison of the rate of consistent answers given by students in the two design experiments is presented in Fig. \ref{FIG:28}.
\begin{figure}[!htpb]
       \includegraphics[width=\columnwidth]{FIGURE28}
    \caption{Comparison of the results of the task number 1.13 in Fig. \ref{FIG:9}.}
    \label{FIG:28}
\end{figure}
As we can see, while the student populations are very different from each other, results were very similar, with the notable case of the most difficult question (the fourth one, on changes in incompatible properties), in which regular classrooms performed slightly better than summer school students. A possible explanation of this result is the revision of activity 1.11 discussed in the previous section, which saw a high rate of success exactly on the behavior of incompatible properties (see Fig. \ref{FIG:26}, statement 3). While students performed very well in this activity, assigning a physical meaning to their own answers to the third and the fourth question was a totally different matter. Students belonging to regular classrooms were puzzled by a phenomenon they had not encountered in the context of polarization: the absence of any property of a physical quantity. Some of those attending the summer school (20\%) suggested explanations in terms of a ``perturbation'' introduced by the measurement device, which represents a common semiclassical model \cite{Ayene2011}, that will be addressed in the course by discussing Heisenberg's microscope thought-experiment and Bohr's criticism of it (see Section \ref{Sec:5.5}).

\emph{Description of the activity on the hydrogen-like atom}. The worksheet block is displayed in Fig. \ref{FIG:29}.
\begin{figure*}[!htpb]
       \includegraphics[width=\textwidth]{FIGURE29}
    \caption{Using the relations between properties to discuss ideal measurement in the context of the hydrogen-like atom: the worksheet blocks administered in the Summer School of Excellence, Udine, 2018, and Liceo Statale Corradini, Thiene, 2018, are identical, except for item F1.2, which was added in the Liceo version.}
    \label{FIG:29}
\end{figure*}
In the activities on the hydrogen-like atom, students are asked to perform the following tasks: \textbf{F1.1} recognizing that, if a system can have properties of different observables at the same time, these properties need to be compatible; \textbf{F1.2}  qualitatively determining the results of the measurement of an observable ($L$) on a system which has no property of that observable, but possesses a property of a compatible observable ($E$); \textbf{F2.1} in the case of multiple properties possessed by the system at the same time ($E$, $L$, $L_z$, $S_z$), some of which are compatible and some incompatible with the observable to measure ($x$), determining whether compatibility or incompatibility prevails (actually the latter: the system cannot have also a position property); \textbf{F2.2} qualitatively determining the results of the measurement of $x$ on the bound state at hand.

Question \textbf{F1.2} was added in the version for Liceo Statale Corradini. The reasons behind this addition are the following. The situation described in the item can and will be discussed also in Unit 3 in the form of a superposition of bound states of the hydrogen-like atom. Therefore, we wanted to tighten the coherence of the course, proposing the same issue at both qualitative and quantitative level. Moreover, we intended to investigate whether students were able to answer a question which is analogous to the last question of the block on position and velocity, but in the compatible case.

\emph{Results of the activity on the hydrogen-like atom}. Data on both design experiments are shown in terms of rate of consistent answers in Fig. \ref{FIG:30}.
\begin{figure}[!htpb]
       \includegraphics[width=\columnwidth]{FIGURE30}
    \caption{Comparison of the results of the task number 1.16 in Fig. \ref{FIG:9}.}
    \label{FIG:30}
\end{figure}
While in both experiments, more than half of the students consistently answered all of the questions, their results were generally worse than that obtained in the first block, with summer school students outperforming regular classrooms. As to the latter, we still get a very high percentage of consistent answers to \textbf{F1.1} (85\%) and \textbf{F2.1} (81\%), while 60\% consistently answered \textbf{F2.2}. Question \textbf{F2.1} turned out to be the most difficult for them, with 53\% of consistent answers. Many students did not connect the situation described in the item to the relation of compatibility (28\% of inconsistent answers, including ``mutual unacquirability'', ``incompatibility'', and an innovative ``direct relation''), even if most of them had correctly answered question \textbf{F1.1}. Others said that if the properties are compatible, nothing changes in measurement (18\%). In question \textbf{F2.2}, instead, student explanations were quite similar to those reported for the fourth item on position and velocity. In general, the physical situations presented in question \textbf{F1.2} and \textbf{F2.2} were totally new to students, who had never encountered phenomena related to compatible observables in previous parts of the course. Here, the carefully selected and motivated students of the Summer School of Excellence have been quicker to respond to the challenge posed by the new context.

\emph{General comments on the results of both activities}. A positive remark concerns the ease with which students replaced the concept of ``outcome-property'' with the more suitable expression ``property of an observable.'' The former had been introduced in the peculiar context of the polarization of the photon, where it is not immediate to interpret interactions of photons with devices as measurements on the photon (as we have seen in Section \ref{Sec:5.2.1}), and to describe the two possible outcomes as values of an observable of polarization. However, also thanks to the wording of the definitions of the relations between properties and of the items of the two blocks, all students referred to properties of observables, and none mentioned outcome-properties anymore. In addition, since relations between observables will be defined by referring to properties that are acquired in measurement (see Section \ref{Sec:2.1.2}), the fact that students spontaneously interpreted the result of measurement as the acquisition of one property of the measured observable (and not the \emph{possible} acquisition of a generic property $P_b$, as in the definition of the relations) represented a progress towards the upgrade to the relations between observables. Last, in the second block some students used the expression ``indefinite'' with relation to observables (e.g., in answering question \textbf{F2.2}: ``$E$, $L$, $L_z$ become indefinite while $S_z$ is retained''), which means that this aspect of the revision of the concept of \emph{system quantity}  has been internalized by them. After these activities, the work in Unit 1 is almost completed: all we have to do is  applying the relations between properties to ideal classical measurements, which resulted trivial for students (virtually all identified mutual unacquirability and compatibility, while incompatibility is out of the picture), defining the relations between observables, and administering the summary table on the revision of the concepts of measurement and \emph{system quantity}.

\subsection{Epistemic practices} \label{Sec:5.4}

In the previous sections on the DBR cycles, we already presented activities corresponding to theoretical practices for the construction of scientific knowledge: the elementary thought experiment in Section \ref{Sec:5.2.1}, the change in perspective described in Section \ref{Sec:5.3.1} and the extension of results found in one context to other contexts in \ref{Sec:5.3.2}. These activities are instrumental to the implementation of more than one principle of design at a time. Here we present activities that are exclusively designed to implement the \emph{Epistemic Principle}, with a focus on thought experiments and mathematical modelling.

\subsubsection{Thought experiment: Description of an unpolarized beam in terms of photons} \label{Sec:5.4.2}

\emph{Goals}. Identifying a consistent description of unpolarized beams in terms of photons by running a thought experiment specifically designed by the instructor.

\emph{Preliminary activity 1 - Description}. In the Summer School of Excellence 2015 (41 students), we investigated spontaneous models of photon polarization after exploring the interaction of light beams with polarizing filters and presenting some empirical evidence on the detection of light in discrete energy quanta. Students were asked to draw a sketch of a vertically polarized beam in terms of photons, and then of an unpolarized beam.

\emph{Preliminary activity 1 - Results}. About half of the students interpreted vertically polarized light as composed of photons uniformly polarized in the vertical direction. They represented them either as vertical segments or double arrows, with written answers reporting the content of their sketch: e.g., ``photons polarized in the same direction are used for polarized light.'' This is coherent with the exploration performed on macroscopic light beams, in which students observed that, by rotating a filter of $180^{\circ}$, the intensity of the transmitted light does not change, and therefore that its polarization property can be identified with a line in a plane perpendicular to the direction of propagation. These sketches also show that the representation of the photons used in JQM can be perceived as intuitive by students. The answers of the other half of the students proved that ascribing a polarization property to the individual photon is not the only natural solution: many of them interpreted polarization as a group property. For instance, vertically polarized photons were represented as balls or dots arranged in vertical rows.

As regards unpolarized light, those students who interpreted polarization as a property of the individual photon drew three different kinds of sketches (see Fig. \ref{FIG:39}): segments or double arrows oriented in different directions (79\%), some of them explicitly adding ``randomly oriented''; stars, that represented photons polarized in all directions (11\%); empty circles or dots, that represented unpolarized photons.
\begin{figure*}[!htpb]
       \fbox{\includegraphics[width=\textwidth]{FIGURE39}}
    \caption{Spontaneous models of unpolarized light in terms of photons: Summer School of Excellence, Udine, 2015}
    \label{FIG:39}
\end{figure*}

\emph{Preliminary activity 2 - Description}.  In the Summer School of Excellence 2017 (32 students), we investigated how they interpret unpolarized light in terms of photons after learning that polarization is a property of the individual photon, and after looking in JQM at the visual representation of a beam of photons transmitted by a filter with vertical axis. We asked them to draw a sketch of an unpolarized beam by using a photon model of light.

\emph{Preliminary activity 2 - Results}. Almost all students (88\%) interpreted a beam of unpolarized light as made of photons polarized in different directions, drawing segments oriented at various angles. Almost half of them (46\%) explicitly added that the angles are ``randomly distributed.'' As in 2015, alternative interpretations included only unpolarized photons, represented by empty balls (7\%), and photons polarized at all angles, represented by stars (2\%).

\emph{1\textsuperscript{st} Version - Description of the activity}. Fig. \ref{FIG:40} displays the worksheet block administered in Liceo Scientifico Statale Alessi, 2019 (39 students attending the lesson).
\begin{figure}[!htpb]
    \includegraphics[width=\columnwidth]{FIGURE40}
    \caption{Worksheet block: unpolarized light in terms of photons, Liceo Alessi, Perugia, 2019}
    \label{FIG:40}
\end{figure}
The block corresponds to the activity described in Section \ref{Sec:3.4.3}.

\emph{1\textsuperscript{st} Version - Results}. The analysis of the answers is structured according to the scientific abilities involved in the task (see \ref{Sec:3.4.3}).
\begin{figure}[!htpb]
    \includegraphics[width=\columnwidth]{FIGURE41}
    \caption{Consistent application of the abilities, Liceo Alessi, Perugia, 2019. In this graph and in the next ones, P.S.C.A.A. means percentage of students consistently applying an ability.}
    \label{FIG:41}
\end{figure}
The application of ability (a) is considered successful depending on the hypothesis under scrutiny: for stars and empty balls, if the answer is compatible with the hypothesis  (respectively: elimination of all polarization properties that differ from $90^{\circ}$; addition of a polarization property at $90^{\circ}$); for segments oriented in different directions, which are supposed to have one polarization property, if the answer is compatible with Malus's law for polarized beams. The application of the ability (b) is successful if it is coherent with the role ascribed to the filter (regardless of the consistency of the assumption). The application of the ability (c) is successful when the answer is coherent with those given before, and uses as empirical term of comparison either the reduction to half or its qualitative version (reduction of the number of photons).

The rate of consistent application of the abilities according to each hypothesis in Liceo Scientifico Statale Alessi is also displayed in Fig. \ref{FIG:41}. Since only two alternatives are displayed in the worksheet block, students had to choose which hypotheses they wished to discuss. All of them opted for photons polarized in different directions (\textit{hypothesis 1}) and photons polarized at all angles (\textit{hypothesis 2}), that were proposed by them during the class discussion. The possibility that photons are not polarized was added by the instructor at the end of the discussion, but no student considered it. As expected, assessing \textit{hypothesis 1} was much harder for students than assessing \textit{hypothesis 2}. The discussion of quantum uncertainty and the stochastic interpretation of Malus's law were scheduled only after the task. As a consequence, deciding whether the number of transmitted photons according to \textit{hypothesis 1} could be half the number of the incident ones was not an easy task. Only 21\% of the students consistently applied ability (a) with relation to the first hypothesis. A qualitative approach resulted more productive than a quantitative one: 75\% of consistent answers focused on the fact that polarizing the light means/implies lowering the intensity or that the filter selects only some of the photons. Inconsistent answers revealed that the main issue with Malus's law is the idea that ``the filter selects only photons oriented as its axis'' (33\%). Such a condition is way too restrictive (infinitesimal), but this pattern might explain why a further 15\% of the students wrote that no photon is transmitted by the filter: e.g., ``it blocks the photons'', ``it does not let any photon pass through.'' As to ability (b), students were generally able to formulate a prediction that was consistent with the hypotheses and with their assumption on the action of the filter. However, when it came to assess the validity \textit{hypothesis 1}, further difficulties arose: 28\% of the students left the item blank (only 12\% in \textit{hypothesis 2}, and another 28\% wrote incoherent or irrelevant answers, sometimes trying to use Malus's law for polarized beams instead of the reduction to half. This shows that while 41\% of the students where able to draw coherent conclusions on \textit{hypothesis 1} - often based on wrong premise -, the most significant difficulty concerned the use of Malus's law. In general, only two students consistently answered the whole task. Another aspect is worth to be mentioned: despite the good results obtained in the assessment of \textit{hypothesis 2}, an issue arose in relation to it, i.e., the idea that ``by removing all components [i.e., all of the properties] but the vertical one, the intensity of the four transmitted photons is reduced by half'' (15\% of the students).

\emph{2\textsuperscript{nd} Version - Description of the activity}. Short after the design experiment in Perugia, we held the course at Liceo Scientifico Galilei, Trieste. In view of the new design experiment, we revised the previous part of the course on the introduction of light quanta, adding that in all considered cases the intensity of a light beam is not dependent on the polarization of its photons. Since the task was considered a preliminary activity designed to show students how to run a theoretical testing experiment, we revised its structure by clearly articulating the phases of such a procedure. In addition, we weakened the conditions of acceptance of a hypothesis, replacing the mathematical expression ``satisfy the experimental results'' with a more qualitative one (``are compatible with empirical evidence'').  The worksheet block is displayed in Fig. \ref{FIG:42}.
\begin{figure}[!htpb]
    \includegraphics[width=\columnwidth]{FIGURE42}
    \caption{Worksheet block: unpolarized light in terms of photons, Liceo Galilei, Trieste, 2019}
    \label{FIG:42}
\end{figure}

\emph{2\textsuperscript{nd} Version - Results}. Also in this case, all students opted for discussing photons polarized in different directions (\textit{hypothesis 1}) and photons polarized at all angles (\textit{hypothesis 2}). However, the self-selected students of Liceo Galilei (18 attending the lesson) achieved much better results than regular classrooms. First, 66\% of them were able to apply ability (a) with relation to the first hypothesis (21\% in Perugia). Surprisingly, while we had not mentioned uncertainty and probability before, half of them spontaneously adopted a probabilistic approach to Malus's law: ``if they are not vertical, there is a certain probability'', ``photons may be stochastically transmitted or not.'' Others, while not mentioning probability, still displayed a global approach to the application of the law in terms of photons: ``photons pass or not based on the angular difference.'' The only issue with this part of the task, was the same as in Perugia: the idea that photons are transmitted exclusively if their polarization is identical to the axis of the filter. In general, Almost 30\% of the students consistently completed the task. An additional 11\% identified the portion of transmitted photons according to \textit{hypothesis 1} as half of the emitted ones, wrote the same value for the prediction based on macroscopic laws, but left the section on the conclusion blank. The issue with \textit{hypothesis 2} appeared to be completely solved: all students consistently assessed the hypothesis.

\emph{3\textsuperscript{rd} Version - Design}. After this experiment, we revised the task, leaving out quantitative elements in favor of qualitative ones: we ask if, based on each hypothesis, there can be or not a reduction in the number of photons as a result of the transmission process. Since we wanted students to assess all three possible hypotheses, we added another space for hypothesis 3): unpolarized photons.

\subsubsection{Interpreting already known laws within the framework of new models: Malus's law} \label{Sec:5.4.1}
\emph{Goal}. Developing an interpretation of Malus's law in terms of photons as a stochastic law of transition from the initial polarization property to that coinciding with the axis of a filter. In the quantum context, it expresses the probability that the photon is transmitted by the filter.

\emph{1\textsuperscript{st} Version - Description of the activity}. The worksheet block used in the Summer School of Excellence 2014 (28 students) is displayed in Fig. \ref{FIG:31}.
\begin{figure*}[!htpb]
       \includegraphics[width=\textwidth]{FIGURE31}
    \caption{Worksheet block on the interpretation of Malus's law: 2014 version.}
    \label{FIG:31}
\end{figure*}
Our initial intention was to guide students to actively develop a probabilistic perspective by means of a \emph{structured inquiry} strategy \cite{Llewellyn2012} in which we engage students in a predict-observe-explain sequence \cite{White1992} with simulated experiments in the JQM environment (Fig. \ref{FIG:9}, activity 1.6), followed by an interpretive question on its results (Fig. \ref{FIG:9}, activity 1.7).

\emph{1\textsuperscript{st} Version - Results}. All students but one answered the prediction task (\textbf{B1.1}) by saying they expected half of the photons to reach the detector: 37\% of them explicitly used the formula for macroscopic light beams (e.g., ``5/10, because $I=I_0\cos^2{45^{\circ}}$''). The others based their prediction on the knowledge of the angle (``because the angle is $45^{\circ}$''). This shows they spontaneously applied their knowledge of Malus's law in the context of single photon experiments before running the simulation. However, most students regarded their prediction as deterministic, and only after having performed the experiment in JQM they realized they were dealing with small numbers, and therefore that they could detect a fraction of photons which differed from half. The crucial item was the task on the interpretation of Malus's law ``for individual photons'' (\textbf{B7}). Its results are displayed in Fig. \ref{FIG:32}.
\begin{figure}[!htpb]
       \includegraphics[width=\columnwidth]{FIGURE32}
    \caption{Results of item \textbf{B7}: 2014 version.}
    \label{FIG:32}
\end{figure}
A large majority of students (83\%) ascribed to the law a statistical nature (e.g., ``Its meaning is statistical: the number of photons is reduced to half''). Even if the item text asked about its meaning for individual photons, two students explicitly stated: ``it does not describe the behavior of one photon'', ``it is not valid for the individual photon.'' In general, student reasoning was strictly related to the specific example discussed in the worksheet (photons prepared at $45^{\circ}$ incident on a filter with horizontal transmission axis): no student wrote the general formula of Malus's law in terms of photons. The wording of the interpretive question has not been effective 1) in directing student attention to the single photon and 2) in helping them develop a general perspective on the law, leading instead to forms of reasoning centered on the specific situation discussed in the previous items.

\emph{2\textsuperscript{nd} Version - Description of the activity}. The worksheet block used in the Summer School of Excellence 2015 (41 students) is presented in Fig. \ref{FIG:33}.
\begin{figure}[!htpb]
       \includegraphics[width=\columnwidth]{FIGURE33}
    \caption{Worksheet block on the interpretation of Malus's law: 2015 version.}
    \label{FIG:33}
\end{figure}
In 2015, the number of tasks was reduced to focus on the interpretive question (\textbf{A1.3}). This item was renamed with an expression taken from lay culture: ``horoscope of the photon'' , and was clearly formulated as a prediction on a single photon. A further specification was added to the item for highlighting the global nature of the request: asking to take into account the polarization property of the photon and the axis of the filter.


\emph{2\textsuperscript{nd} Version - Results}. Data on \textbf{A1.3} the are displayed in Fig. \ref{FIG:34}.
\begin{figure}[!htpb]
       \includegraphics[width=\columnwidth]{FIGURE34}
    \caption{Results of item A1.3, the ``horoscope of the photon'': 2015 version.}
    \label{FIG:34}
\end{figure}
The majority of the students (61\%) discussed the item in terms of probability: 52\% of them in the case of a photon polarized at $45^{\circ}$ passing through a filter with axis at $0^{\circ}$, 48\% writing the general formula (e.g., ``The probability is $cos^2\alpha$, which is the angle between the polarization property and the axis of the filter''). A significant minority of the students (17\%) kept focusing on beams. The remaining answers were irrelevant.

\emph{{Same Version - New needs due changes in subsequent activities}}. In the Summer School of Excellence 2016 (27 students), there was no change in the activity. However, as we saw in Section \ref{Sec:5.2.1}, the subsequent worksheet blocks underwent a major revision: replacing filters with calcite crystals to introduce quantum measurement (Fig. \ref{FIG:9}, activities 1.8-1.10).  This corresponded not only to a strong improvement in the understanding of quantum measurement, but had a significant impact also on the activity concerning Malus's law, eliciting an issue that had not come to light before: the difficulty to transfer the calculation of the transition probability from the interaction of a photon with a filter to that with a crystal$+$counters. Evidence on this issue was provided by answers to item \textbf{C2}, (see Fig. \ref{FIG:35} for the text of the item).
\begin{figure}[!htpb]
    \includegraphics[width=\columnwidth]{FIGURE35}
    \caption{Transfer of Malus's law to the context of birefringent crystals.}
    \label{FIG:35}
\end{figure}
Only 37\% of the students wrote a consistent answer. The others either assumed that the photon was initially polarized at $45^{\circ}$ (15\%), or wrote in both cases $cos^2 \theta$ (another 15\%), or gave highly inconsistent answers (33\%): e.g., ``the probabilities are respectively 0\% and 100\%'', or ``79 and 21.''

\emph{Goal - Refinement}. a) Helping students develop a single photon interpretation of the law of Malus; b) supporting them in transferring the calculation of the transition probability from the context of filters to that of crystals.

\emph{3\textsuperscript{rd} Version - Description of the activity}. The following year (2017) saw the last change in the design of the worksheet block, displayed in Fig. \ref{FIG:36}.
\begin{figure*}[!htpb]
       \includegraphics[width=\textwidth]{FIGURE36}
    \caption{Worksheet block on the interpretation of Malus's law, including its transfer to the context of calcite crystals: 2017 version.}
    \label{FIG:36}
\end{figure*}
The strategy used to achieve both goals was encouraging students to focus only on a single photon and on the unifying features of the interactions between the photon and both devices (filter$+$counter and crystal$+$counters). These are: 1) the fact that most interactions have uncertain outcome, but in two special cases the outcome is determinate; 2) the existence of a transition between the initial angle of polarization and the angle determined by the interaction.

As a consequence, the introductory items (\textbf{B1}-\textbf{B2}) examined the interaction of a single photon with a filter, discussing from the start the fundamental dichotomy between ``interactions with a determinate outcome'' and ``interactions with an uncertain outcome.'' After the Horoscope of the photon (\textbf{B3}), we added a specific item (\textbf{B4}) designed to generalize its results to the case of an arbitrary initial angle of preparation ($\theta$) and resulting angle after the interaction ($\phi$), focusing exclusively on the angles and not on the device. In item \textbf{C2}, on the calculation of transition probabilities in the context of calcite crystals, beams were replaced by a single photon. Most importantly, we eliminated the use of simulated experiments in JQM, converting this activity a into a purely theoretical form of inquiry.

\emph{3\textsuperscript{rd} Version - Results}. Data of 2017 on the Horoscope of the photon (\textbf{B3}) are displayed in a comparison with results of the equivalent item in the design experiments of 2014 and 2015.
Specifically, we compare two different dimensions of student reasoning on Malus's law: 1) the alternative between a probabilistic and a statistical interpretation (Fig. \ref{FIG:37}); 2) local reasoning vs. global reasoning, i.e., focusing on the transition probability for an angle of $45^{\circ}$ between the initial property and the outcome-property or on the general formula (Fig. \ref{FIG:38}).
\begin{figure}[!htpb]
    \includegraphics[width=\columnwidth]{FIGURE37}
    \caption{Comparison table: statistical vs. probabilistic interpretations.}
    \label{FIG:37}
\end{figure}
\begin{figure}[!htpb]
    \includegraphics[width=\columnwidth]{FIGURE38}
    \caption{Comparison table: local vs. global reasoning.}
    \label{FIG:38}
\end{figure}
Since we promote a probabilistic interpretation and a global form of reasoning, we see that in both respects there has been a steady improvement over the years. After administering the Horoscope, generalizing the formula of Malus's law to arbitrary angles (\textbf{B4}) was a trivial task: all but one student answered correctly. The only student giving a wrong answer put a plus instead of a minus: $cos^2(\theta+\phi)$. This means that, if we consider the two items as components of a single task on the interpretation of Malus's law, virtually all students developed a consistent understanding of the topic. By using this sequence of items and, in particular, by discussing transition in terms of the angle between the initial property and the outcome-property, question \textbf{C2} on the transition probability in the context of crystals became straightforward. All students correctly answered the item, even if there was a long teaching/learning session in between (activity 1.8 and most of 1.9).


\subsubsection{Thought experiment: Superposition as statistical mixture of component states?} \label{Sec:5.4.3}

\emph{Goals}. Distinguishing between superposition states and mixed states. Developing an awareness of the implications of interpreting a superposition as a mixture of the component states.

\emph{1\textsuperscript{th} Version - Description of the activity}. The worksheet block used in February 2019 in Liceo Scientifico Statale Alessi (33 students attending the lesson) is displayed in Fig. \ref{FIG:43}.
\begin{figure*}[!htpb]
       \includegraphics[width=\textwidth]{FIGURE43}
    \caption{Worksheet block on the interpretation of quantum superposition: Liceo Alessi, 2019.}
    \label{FIG:43}
\end{figure*}
We briefly examine questions needed to highlight the implications of the hypothesis: if the superposition is a mixture, the case is isomorphic to that already presented in the worksheet block on the revision of measurement (see Section \ref{Sec:5.2.1}, 2016 version). Therefore, we expected that students could interpret it as a classical measurement, easily answering question \textbf{A1.1.1}. The answer to the following item, \textbf{A1.1.2}, is a logical consequence of the former. \textbf{A1.1.3} requires students to shift focus from the single photon to the beam and, as the first question, has been discussed orally in the introduction to quantum measurement.

\emph{1\textsuperscript{th} Version - Results}. A large majority of students answered \textbf{A1.1.1}, \textbf{A1.1.2}, and \textbf{A1.1.3} consistently (79\%). Inconsistent answers to \textbf{A1.1.1} were all due to the same issue elicited in sections \ref{Sec:5.2.1} and \ref{Sec:5.4.1}: students focused on the beam instead of the single photon, thus coming to the conclusion that measurement is probabilistic and the observable is indefinite. Also 35\% students who consistently answered \textbf{A1.1.1} and \textbf{A1.1.2} wavered when it came to decide between a beam and a single photon. At first, they focused on the beam, only to change their mind later: from ``the nature of the interaction is stochastic'' to ``it is certain'', from ``indefinite'' to ``definite.'' These students clearly understood the consequences of interpreting superposition as a mixture, since in \textbf{A1.1.3} they stated that the uncertainty was due to lack of knowledge on the state of each photon, but  inconsistently answered \textbf{A1.1.1} because of the wording of the item, or simply of the fact that the alternative between single object and beam has proven to be a tricky issue.

Data on the thought experiment are displayed in Fig. \ref{FIG:44}.
\begin{figure}[!htpb]
    \includegraphics[width=\columnwidth]{FIGURE44}
    \caption{Results of the thought experiment: Liceo Alessi, Perugia, 2019}
    \label{FIG:44}
\end{figure}
The analysis of the answers is structured according to the scientific abilities involved in the task (see \ref{Sec:3.4.4}). Only 36\% of the students consistently ran the thought experiment. Among these students, most used a filter with axis at $30^{\circ}$ (82\%), the others a crystal with channels at $30^{\circ}$ and $120^{\circ}$. Most tested, as expected, the transition probability, which is 1, contrarily to the hypothesis (82\%). Others, interestingly, a logical consequence of the hypothesis: the fact that only photons polarized at $0^{\circ}$ and $90^{\circ}$ should exist. The idea was rejected by using arbitrarily oriented filters: e.g., ``direct the photon beam prepared in the state $|30^{\circ}\rangle$ to a filter with axis at an angle $\theta$ that is different from 0° and 90°. Transmitted photons acquire a polarization property at $\theta$, and the hypothesis is not satisfied.'' Some students gave partially consistent answers, correctly applying only one or two of the abilities needed. This shows that running a self-generated thought experiment, as simple as it can be, requires the coordination of different abilities, and that the previous thought experiments students ran during the course were not necessarily sufficient to develop an awareness of how to design an effective one. Worse, many student did not even try to write anything and left the answer blank (30\%). A noteworthy aspect concerns 12\% of the students, who either used a filter at $0^{\circ}$ or a crystal at $0^{\circ}$ and $90^{\circ}$, thus coming to the conclusion that the hypothesis was confirmed: the testing experiment and the prediction were identical to the hypothesis itself. A need emerged to give more content support in the item text, explicitly stating that we intend to test whether the hypothesis is also valid for the measurement of different polarization observables.

\emph{2\textsuperscript{nd} Version - Description of the activity}. The new worksheet block, administered in Liceo Scientifico Galilei (16 self-selected students attending the lesson), is displayed in Fig. \ref{FIG:45}.
\begin{figure*}[!htpb]
       \includegraphics[width=\textwidth]{FIGURE45}
    \caption{Worksheet block on the interpretation of quantum superposition: Liceo Galilei, Trieste, 2019}
    \label{FIG:45}
\end{figure*}
The formulation of the mixture hypothesis was revised, adding its logical consequence (``photons are polarized only at $0^{\circ}$ or $90^{\circ}$''), which has been productive for some students. The new item \textbf{A1.1} provides a visual support for this statement. In \textbf{A1.1.1} (now \textbf{A1.2.1}), we added a verbal prompt for focusing on the single photon, asking students to think back to the initial task on measurement performed on mixtures of photons already prepared at 0° and 90°. Finally, we specified more clearly the task involved in the design and running of the thought experiment.

\emph{2\textsuperscript{nd} Version - Results}. All the students consistently answered the first three questions, and 75\% of them successfully ran the thought experiment, most using a filter. The only student who designed an experiment with a crystal wrote a very clear and complete answer: ``I rotate the crystal by $30^{\circ}$, so that the channels are at $30^{\circ}$, $120^{\circ}$. In this case, the property of a beam polarized at $30^{\circ}$ is one of the outcome-properties $\Rightarrow$ certain result. Instead, not all photons of the random mixture are transmitted, the process is stochastic. False.'' Of the 4 students who did not answer consistently, two designed reliable experiments that tested the hypothesis, but did not make a correct prediction; one proposed an experiment for measuring the $0^{\circ}$,$90^{\circ}$ observable, thus confirming the hypothesis. The last one left the answer blank.

\subsubsection{Identifying and interpreting mathematical constructs for describing physical situations and deriving new results: Entangled superposition} \label{Sec:5.4.4}

\emph{Goals}. Identifying and interpreting the mathematical expression of an entangled superposition of modes.

\emph{1\textsuperscript{st} Version - Description of the activity}. The worksheet block with the mathematization and interpretation tasks is displayed in Fig. \ref{FIG:47}. The worksheet was administered for the first time in Liceo Scientifico Galilei, Trieste, 2019 (17 students attending the lesson).
\begin{figure*}[!htpb]
       \includegraphics[width=\textwidth]{FIGURE47}
    \caption{Worksheet block on the derivation and interpretation of entangled superposition: Liceo Galilei, Trieste, 2019}
    \label{FIG:47}
\end{figure*}

\emph{1\textsuperscript{st} Version - Results}. 76\% of the students consistently answered the mathematization task (\textbf{C5}), proposing a superposition state that is compatible with the situation at hand: $a|x_1\rangle|0^{\circ}\rangle+b|x_2\rangle|90^{\circ}\rangle$. Another student wrote a similar expression, but using the square of the coefficients: $a^2|x_1\rangle|0^{\circ}\rangle+b^2|x_2\rangle|90^{\circ}\rangle$. More than half of them added consistent explanations, either focusing on the interpretation of superposition or on the change in state from the initial situation to the final one. For the first line of reasoning, ``the state of the photon is a superposition of the state corresponding to $0^{\circ}$, that is $|x_1\rangle|0^{\circ}\rangle$ and the state corresponding to $90^{\circ}$, that is $|x_2\rangle|90^{\circ}\rangle$. The probability to find the photon in that states are $a^2$ and $b^2$'', for the second one, ``we do not have $|x\rangle|\theta\rangle$ anymore because the photon is after the crystal, and since it is only probabilistic, both outcomes must be included [in the superposition].'' The others wrote a consistent formula without adding any explanation. Inconsistent answers included one student who wrote a separable state with equal coefficients: $(1/\sqrt{2}|0^{\circ}\rangle+1/\sqrt{2}|90^{\circ}\rangle)(1/\sqrt{2}|x_1\rangle+1/\sqrt{2}|x_2\rangle)$.
Another student wrote the initial expression of the global state, just replacing the arbitrary coefficients with the square of the usual ones: $|x\rangle (1/2|0^{\circ}\rangle+1/2|90^{\circ}\rangle)$. The third student inappropriately transferred the knowledge acquired in the context of the hydrogen-like atom, writing a very inconsistent expression in terms of quantum numbers: $a^2b^2|n_1+n_2, l_1+l_2, m_1+m_2, s_1+s_2\rangle$. As we will see, negative transfer from this context represented a serious issue in the last question of the worksheet (\textbf{C6.2}). All students used the plus sign in the superposition, which mirrors the sign of the initial superposition. However, in QM it is not possible to reconstruct a state by means of a measurement, as we do not get information on the phases \cite{Michelini2014}. Given that we focused on the identification of the superposition of entangled states, this issue was left out from the discussion.

Moving on to examine the interpretive task (\textbf{C6.1}) on the possession of properties of position and polarization before measurement, also in this case a large majority of students gave consistent answers (71\%). These students displayed different forms of productive reasoning: some by using the definition of the possession of a property (``no, since I do not have a probability equal to 1 to measure them''), or by linking superposition to uncertainty (``no, given the fact that it does not possess any property for sure, it is represented by a superposition'') or focusing on the change in state from the initial situation to the final one (``at the beginning it possesses them, but at the end it does not''). Some students also added that position and polarization properties are compatible and correlated. As we talked about compatibility between position and spin properties only in Unit 1 (see Fig. \ref{FIG:9} and Section \ref{Sec:5.3.2} for data), this is a case of productive transfer from the context of the hydrogen-like atom. The remaining students either said that the system possesses the involved properties, or focused instead on the relation between the two observables: ``not compatible before measurement.''

The last item (\textbf{C6.2}) asked how system properties change if we measure one of the two observables at hand. This question was by far the least successful of the three:  only 47\% of the students gave consistent answers. In general, these answers were very concise: ``by measuring one observable, we acquire both properties.'' One students highlighted the correlation between the two observables: ``I acquire a property and the corresponding property of the other observable.'' This is notable, since we had not emphasized this aspect of entanglement before. Another one productively referred to the discussion of Experience 2, quoting a sentence we used in the slide presentation: ``if I measure position, I also measure polarization and vice versa.'' Except for one student, who left the answer blank, the others gave inconsistent answers, 75\% of them by inappropriately transferring their knowledge on the hydrogen-like atom: ``if I measure $x$, I certainly lose $E$, $L$, $L_z$, and retain spin, if I possess it.'' This is perfectly in line with the answer to item \textbf{F2.2} (see Fig. \ref{FIG:29}) on the measurement of position on an atom in a bound state. After all, the wording of this question was very similar to that of \textbf{F2.2}.  Others claimed that nothing changes in measurement, ``since the properties of position and spin are compatible'', another statement that was used only with relation to the atom.

\emph{2\textsuperscript{nd} Version - Design}. In general, while the mathematization task was very successful, the interpretive ones require some revision. As a matter of fact, the issue concerning negative transfer from the context of the hydrogen-like atom heavily affected the results of \textbf{C6.2}. The main suggestions come from productive lines of reasoning used by students. \textbf{C6.1}: add to the item text a reference to the definition of the possession of a property by a system. \textbf{C6.1}: add a reference to Experience 2, which concerns only the photon and not the atom.

\subsection{Epistemological debates} \label{Sec:5.5}
Since epistemological debates are addressed in a whole class discussion without the use of worksheets or other written assignments, their refinement could be based only on the instructor's diary.

Here we limit ourselves to report on the revision of the first debate on the problem of indefiniteness and uncertainty. As described in Section \ref{Sec:3.5.2}, in the initial versions of the course, we discussed the Heisenberg's microscope thought-experiment and Bohr's criticism of it in Unit 1, after the application of the relations between properties to the case of position and velocity. Students were generally at ease with an interpretation of uncertainty as caused by measurement disturbance, that they could reconcile with their classical intuition on point-like particles. However, when this view was questioned, raising the possibility that uncertainty is an intrinsic property of quantum systems, some students clearly showed their discomfort. In Thiene, one of the best performing students explicitly complained that, if that was the case, we should conclude that QM is an absurd theory and makes no sense. Up until then, she had taken active part to all the worksheet activities and to the whole class discussions that ensued. After that, and for the rest of the lesson, the level of her engagement significantly declined.

In the retrospective analysis, we considered the possibility to add more content support to this activity, including an anticipation of the discussion on the wave-particle duality. However, this would have subverted the structure of the course which, in accordance with the gradual construction of content in spin-first approaches \cite{Zuccarini2020} and recent textbooks written in collaboration with physics education researchers \cite{McIntyre2012}, scheduled the discussion of propagation only after a careful examination of the system at a point in time and of its behavior in measurement.

For this reason, we opted for providing students with an operative idea of indefinite quantity, that could give empirical meaning to this situation and be immediately connected with the now familiar context of polarization: ``a quantity of a system is called indefinite when the ideal measurement of this quantity on a large ensemble of identical systems gives different results according to a probabilistic distribution'' (see Fig. \ref{FIG:9}, activity 1.14). Of course, this begged the question of how to establish whether two systems are identical according to QM. Therefore, we told students that this would have been the driving question of the next unit since, in order to give a reasonable answer, we would have needed the concept and the formal representation of the quantum state.

The discussion of Heisenberg's microscope was moved to Unit 4, activity 4.5, where, based on the adoption of a field ontology, it was possible to contemplate the idea that quantum uncertainty is due to a measurement disturbance, and to reject it without regret.

With this revision, the discussion of the issue at hand did not cause any visible discomfort.


\section{Conclusions}

Learning quantum mechanics involves overcoming manyfold challenges, among which knowledge revision, fragmentation, and the construction of a personally plausible and reliable picture of the quantum model. Extensive investigations of cognitive and epistemological challenges, as well of the use of student resources and intuition in the discussion of introductory science topics have been conducted in the framework of conceptual change \cite{Vosniadou2008, diSessa2014}, leading to the development of effective approaches to science teaching \cite{Amin2014}. Also in the case of QM, various researchers have considered the problem of teaching the subject as the design of strategies to effectively promote a conceptual change in individual learners \cite[e.g.,][]{Thagard1992, Kalkanis2003, Tsaparlis2013, Malgieri2017}. In general, the progress from a classical to a quantum perspective requires changes in knowledge structures about a scientific theory, which have been developed as a result of instruction. Since educational models of conceptual change have been introduced to account for a different kind of change (the transition from na\"{\i}ve to scientific knowledge), there is a need to revise the framework, examining the differences involved in learning a successive theory. In addition, while studies on conceptual change are undergoing a systemic turn \cite{Amin2014}, there is a lack of educational proposals on QM that coordinate multiple interacting aspects at different levels of analysis. Last, in accordance with the shift in focus from difficulties to resources \cite{Goodhew2019}, there is a need to identify the links between student intuition and strategies for learning QM.

In this article, we describe the development of a course for secondary school that is based on an examination of conceptual change in the transition from CM to QM, and integrates cognitive and epistemic aspects including the adoption of epistemic practices of theoretical physicists, and a structured approach to interpretive themes. The aim is to help fill the gaps in analysis and design, addressing all the challenges mentioned at the beginning of this section, and taking advantage of available knowledge elements and intuition in the task.

This systemic approach led to the derivation of the following design principles: \emph{Principle of Knowledge Revision}, \emph{Principle of Knowledge Organization}, \emph{Epistemic Principle}, and \emph{Epistemological Principle}.

The first one relies on the analysis of continuity and change in basic concepts and constructs to promote the understanding of their quantum counterparts and the ability to discriminate between aspects of the old and the new notions, identifying their correct context of application. The instruments used in this process suggest strategies to leverage prior knowledge according to specific patterns of change in the trajectory of each notion.

The second principle concerns the development of conceptual tools denoted as \emph{relations between properties}, that act as organizing elements in the construction of a unifying picture of quantum measurement across contexts, and help address student's need of comparability with CM by offering interpretive keys for the shift from the classical to the quantum task.

The third one proposes to design the course around a modelling process that includes epistemic practices of the theoretical physicist, with the goal to help students accept the quantum description of the world as a plausible and reliable product of their own inquiry.

The last principle proposes to work in the context of a clearly specified form of interpretation, so as to identify and discuss the facets of the foundational debate that are triggered by each choice, with the aim to help students develop an awareness of the cultural significance of the debate, of the limits the chosen stance, of the open issues.

The structure of the content and of the modelling process has been informed by the interplay of the first three principles, and developed in the template of the model of modelling \cite{Gilbert2016}, a perspective devised to analyze the process of modelling in science education. The result is a model that starts from the description of a property of an object (photon polarization), and is extended and revised through a process conducted by means of theoretical epistemic practices (\emph{Epistemic Principle}), gradually incorporating quantum measurement, state, superposition, propagation and entanglement (\emph{Principle of Knowledge Revision}). Thanks to the \emph{Principle of Knowledge Organization}, each step allows students to advance in parallel in the development of an elementary model of the hydrogen-like atom.

To illustrate in full the implementation of the principles, we show how each of them guided us in the design of a set of individual activities. A special attention is devoted to the conversion of epistemic practices of the theoretical physicist into active learning strategies: inquiry-based learning \cite{Llewellyn2012}, the model of modelling, and different research perspectives on the role of mathematics in physics \cite{Uhden2012, Redish2015} converge to structure the chain of activation used for mathematical modelling in a purely theoretical context. Then, these frameworks are blended together with aspects of the ISLE learning system, such as the conduction of testing experiments and the rubrics of scientific abilities \cite{Etkina2006, Etkina2015}, to convert thought experiments into theoretical inquiry activities. Last, we show how the \emph{Epistemological Principle} guided us to strengthen the coherence of the course and to design the discussion of epistemological themes.

The course is presented in Section \ref{Sec:4}, which includes an outline of its structure, of the types of activities that are designed to implement the principles, of the instruments and methods. A bird's eye view of the sequence and the types of activities is provided in Fig. \ref{FIG:9}.

The second part of the article describes the cycles of refinement of the same set of activities previously discussed. In this work, we do not test the global effectiveness of the course, but show how a wide range of different inquiry and modelling activities has been made effective in addressing the challenges at a local level, thanks to the cycles of preparation, experiment, and analysis of the learning outcomes conducted in the framework of design-based research \cite{Bakker2015}. This concerns in particular the refinement of theoretical epistemic activities, that have become innovative and effective forms of inquiry for engaging students in the development of theoretical skills (e.g., generating and/or running thought experiments).

Future directions include the analysis of a pre-post-test administered in regular classrooms, in order to evaluate
the effectiveness of the course as a whole.

\begin{thebibliography}{99}
\bibitem{Zhu2012} G. Zhu and C. Singh, Improving students' understanding of quantum measurement. I. Investigation of difficulties, Phys. Rev. ST Phys. Educ. Res., \textbf{8}, 010117 (2012).
\bibitem{Ayene2011} M. Ayene, J. Kriek, and D. Baylie, Wave-particle duality and uncertainty principle: Phenomenographic categories of description of tertiary physics students' depictions, Phys. Rev. ST Phys. Educ. Res., \textbf{7}, 020113 (2011).
\bibitem{Pospiech2021} G. Pospiech, A. Merzel, G. Zuccarini, E. Weissman, G. Katz, I. Galili, L. Santi, and M. Michelini, The role of mathematics in teaching quantum physics at high school, in \textit{Teaching-Learning Contemporary Physics: From Research to Practice}, edited by B. Jarosievitz and C. S\"{u}k\"{o}sd (Springer Nature Switzerland AG, Cham, Switzerland, 2021), pp. 47-70.
\bibitem{Passante2015} G. Passante, J. Emigh, and P.S. Shaffer, Student ability to distinguish between superposition states and mixed states in quantum mechanics. Phys. Rev. ST Phys. Educ. Res., \textbf{11}, 020135 (2015).
\bibitem{Vosniadou2008} S. Vosniadou, The framework theory approach to the problem of conceptual change, in  \textit{International handbook of research on conceptual change, 1st edition}, edited by S. Vosniadou (Routledge, New York and London, 2008), pp. 3-34.
\bibitem{diSessa2014} A.A. diSessa, A history of conceptual change research: Threads and fault lines, in \textit{The Cambridge handbook of the learning sciences, Second edition}, edited by R. K. Sawyer (Cambridge University Press, 2014), pp. 265-281.
\bibitem{Johnston1998} I.D. Johnston, K. Crawford, and P.R. Fletcher, Student difficulties in learning quantum mechanics, Int. J. Sci. Educ., \textbf{20}, 427-446 (1998).
\bibitem{Marshman2015} E. Marshman and C. Singh, Framework for understanding the patterns of student difficulties in quantum mechanics, Phys. Rev. ST Phys. Educ. Res. \textbf{11}, 020119 (2015).
\bibitem{Malgieri2017} M. Malgieri, P. Onorato, and A. de Ambrosis, Test on the effectiveness of the sum over paths approach in favoring the construction of an integrated knowledge of quantum physics in high school, Phys. Rev. Phys. Educ. Res. \textbf{13}, 010101 (2017).
\bibitem{Griffiths2018} D.J. Griffiths, and D.F. Schroeter, \textit{Introduction to quantum mechanics}, Third Edition (Cambridge University Press, Cambridge, United Kingdon, 2018).
\bibitem{Ravaioli2017} G. Ravaioli and O. Levrini, Accepting quantum physics: Analysis of secondary school students' cognitive needs, in \textit{Electronic Proceedings of the ESERA 2017 Conference. Research, Practice and Collaboration in Science Education, Part 2}, edited by O. Finlayson, E. McLoughlin, S. Erduran, and P. Childs (2017). \url{https://www.dropbox.com/s/t1ri3ql7ufpihun/Part_2_eBook.pdf?dl=0}
\bibitem{Marshman2017} E. Marshman and C. Singh, Investigating and improving student understanding of quantum mechanics in the context of single photon interference, Phys. Rev. Phys. Educ. Res. \textbf{13}, 010117 (2017).
\bibitem{Wittmann2020} M.C. Wittmann and J.T. Morgan, Foregrounding epistemology and everyday intuitions in a quantum physics course for nonscience majors, Phys. Rev. Phys. Educ. Res. \textbf{16}, 020159 (2020).
\bibitem{Baily2015} C. Baily and N.D. Finkelstein, Teaching quantum interpretations: Revisiting the goals and practices of introductory quantum physics course, Phys. Rev. ST Phys. Educ. Res. \textbf{11}, 020124 (2015).
\bibitem{Coppola2013} P. Coppola and J. Krajcik, Discipline-centered post-secondary science education research: Understanding university level science learning, J. Res. Sci. Teach., \textbf{50}, 627-638 (2013).
\bibitem{Goodhew2019} L.M. Goodhew, A.D. Robertson, P.R.L. Heron, and R.E. Scherr, Student conceptual resources for understanding mechanical wave propagation, Phys. Rev. Phys. Educ. Res. \textbf{15}, 020127 (2019).
\bibitem{Dreyfus2017} B.W. Dreyfus, A. Elby, A. Gupta, and E.R. Sohr, Mathematical sense-making in quantum mechanics: An initial peek, Phys. Rev. Phys. Educ. Res. \textbf{13}, 020141 (2017).
\bibitem{Dini2017} V. Dini and D. Hammer, Case study of a successful learner's epistemological framings of quantum mechanics, Phys. Rev. Phys. Educ. Res. \textbf{13}, 010124 (2017).
\bibitem{Kuhn1962} T. Kuhn, \textit{The Structure of Scientific Revolutions} (University of Chicago Press, Chicago, 1962).
\bibitem{Thagard1992} P. Thagard, \textit{Conceptual revolutions}, (Princeton University Press, Princeton, New Jersey, 1992).
\bibitem{Tsaparlis2009} G. Tsaparlis and G. Papaphotis, High school Students' Conceptual Difficulties and Attempts at Conceptual Change: The case of basic quantum chemical concepts, Int. J. Sci. Educ. \textbf{31}, 895-930 (2009).
\bibitem{Singh2015} C. Singh and E. Marshman, Review of student difficulties in upper-level quantum mechanics, Phys. Rev. ST Phys. Educ. Res., \textbf{11}, 020117 (2015).
\bibitem{Lewerissa2017} K. Krijtenburg-Lewerissa, H.J. Pol, A. Brinkman, and W.R. Van Joolingen, Insights into teaching quantum mechanics in secondary and lower undergraduate education, Phys. Rev. Phys. Educ. Res., \textbf{13}, 010109 (2017).
\bibitem{Posner1982} G.J. Posner, K.A. Strike, P.W. Hewson, and W.A. Gertzog, Accomodation of a scientific conception: Toward a theory of conceptual change, Sci. Educ., \textbf{66}(2), 211-227 (1982).
\bibitem{Potvin2020} P. Potvin, L. Nenciovici, G. Malenfant-Robichaud, F. Thibault, O. Sy, M.-A. Mahhou, A. Bernard, G. Allaire-Duquette, J. M. Blanchette Sarrasin, L. M. Brault Foisy, et al. Models of conceptual change in science learning: Establishing an exhaustive inventory based on support given by articles published in major journals. Stud. Sci. Educ., 56(2), 157–211 (2020).
\bibitem{Tsaparlis2013} G. Tsaparlis, Learning and teaching the basic quantum chemical concepts, in \textit{Concepts of matter in science education}, edited by G. Tsaparlis and H. Sevian (Springer, Dordrecht, 2013), pp. 437-460.
\bibitem{Kalkanis2003} G. Kalkanis, P. Hadzidaki, and D. Stavrou, An instructional model for a radical conceptual change towards quantum mechanics concepts, Sci. Educ. \textbf{87}, 257-280 (2003).
\bibitem{Amin2014} T.G. Amin, C. Smith, and M. Wiser, Student conceptions and conceptual change: Three overlapping phases of research, in \textit{Handbook of research on science education, Volume II}, edited by N. G. Lederman and S. K. Abell (Routledge, New York, 2014), pp. 71-95.
\bibitem{Carey1999} S. Carey, Sources of Conceptual Change, in \textit{Conceptual development: Piaget's legacy}, edited by E. K. Scholnick, K. Nelson, S. A. Gelman, P. H. Miller (Lawrence Erlbaum Associates Publishers, 1999), pp. 293-327.
\bibitem{Chi2013} M. Chi, Two Kinds and Four Sub-Types of Misconceived Knowledge, Ways to Change It, and the Learning Outcomes, in \textit{International Handbook of Research on Conceptual Change, 2nd edition}, edited by S. Vosniadou (Routledge, New York and London, 2013), pp. 49-70.
\bibitem{Zuccarini2022} G. Zuccarini and M. Malgieri, Modeling and Representing Conceptual Change in the Learning of Successive Theories, Sci. Educ. (Dordr.), (2022). \href{https://doi.org/10.1007/s11191-022-00397-1}{DOI:10.1007/s11191-022-00397-1}
\bibitem{Arabatzis2020} T. Arabatzis, What are scientific concepts?, in \textit{What is scientific knowledge? An introduction to Contemporary Philosophy of Science}, edited by K. McCain and K. Kampourakis (Routledge, New York and London, 2020), pp. 85-99.
\bibitem{Hoyningen1993} P. Hoyningen-Huene, \textit{Reconstructing scientific revolutions: Thomas S. Kuhn's philosophy of science} (University of Chicago Press, Chicago, 1993).
\bibitem{Andersen2006} H. Andersen, P. Barker, and X. Chen, \textit{The cognitive structure of scientific revolutions} (Cambridge University Press, Cambridge, 2006).
\bibitem{Klein2012} U. Klein, What is the limit $\hslash \to 0$ of quantum theory?, Am. J. Phys., \textbf{80}, 1009 (2012).
\bibitem{Stadermann2019} H. K. E. Stadermann, E. van den Berg, and M. J. Goedhart, Analysis of secondary school quantum physics curricula of 15 different countries: Different perspectives on a challenging topic, Phys. Rev. ST Phys. Educ. Res., \textbf{15}, 010130 (2019).
\bibitem{Zuccarini2020} G. Zuccarini, Analyzing the structure of basic quantum knowledge for instruction, Am. J. Phys., \textbf{88}, 385 (2020).
\bibitem{diSessa1998} A. A. diSessa and B. Sherin, What changes in conceptual change?, Int. J. Sci. Educ., \textbf{20}, 1155 (1998).
\bibitem{diSessa2016} A. A. diSessa, B. Sherin, and M. Levin, Knowledge analysis: An introduction, in {Knowledge and interaction: A synthetic agenda for the learning sciences}, edited by A. A. diSessa, M. Levin, and N. J. S. Brown  (Routledge, New York, 2016), pp. 30-61.
\bibitem{Levrini2008} O. Levrini and A. A. diSessa, How students learn from multiple contexts and definitions: Proper time as a coordination class, Phys. Rev. ST Phys. Educ. Res., \textbf{4}, 010107 (2008).
\bibitem{Debianchi2011} M. Sassoli De Bianchi, Ephemeral properties and the illusion of microscopic particles, Found. Sci.,  \textbf{16}, 393 (2011).
\bibitem{Elby2016} A. Elby, C. Macrander, and D. Hammer, Epistemic cognition in science, in  \textit{Handbook of Epistemic cognition}, edited by J.A. Greene, W.A. Sandoval, and I. Br{\aa}ten (Routledge New York, NY, 2016), pp. 113-127.
\bibitem{Sandoval2016} W.A. Sandoval, J.A. Greene, and I. Br{\aa}ten, Understanding and promoting thinking about knowledge: Origins, issues, and future directions of research on epistemic cognition, Rev. Res. Educ., \textbf{40}, 457-496 (2016).
\bibitem{Uhden2012} O. Uhden, R. Karam, M. Pietrocola, and G. Pospiech, Modelling Mathematical Reasoning in Physics Education, Sci. Educ. (Dordr.), \textbf{21}, 485-506 (2012).
\bibitem{Redish2015} E. F. Redish and E. Kuo, Language of Physics, Language of Math: Disciplinary Culture and Dynamic Epistemology, Sci. Educ. (Dordr.), \textbf{24}, 561-590 (2015).
\bibitem{Stephens2012} A. L. Stephens and J. J. Clement, The Role of Thought Experiments in Science and Science Learning, In \textit{Second international handbook of science education}, edited by B. J. Fraser, K. T. Campbell, and J. McRobbie (Springer, Dordrecht, 2015), pp. 157-175.
\bibitem{Gilbert2000} J. K. Gilbert and M. Reiner, Thought Experiments in Science Education: Potential and Current Realisation, Int. J. Sci. Educ., \textbf{22}, 265 (2000).
\bibitem{Etkina2015} E. Etkina, Millikan Award lecture: Students of Physics - Listeners, Observers, or Collaborative Participants in Physics Scientific Practices?, Am. J. Phys., \textbf{83}, 669 (2015).
\bibitem{Bub1997} J. Bub, \textit{Interpreting the quantum world} (Cambridge University Press, Cambridge, UK, 1997).
\bibitem{Schlosshauer2007} M. Schlosshauer, \textit{Decoherence and the quantum-to-classical-transition} (Springer, 2007).
\bibitem{Hobson2013} A. Hobson, There are no particles, there are only fields, Am. J. Phys., \textbf{81}, 211 (2013).
\bibitem{Ghirardi1996} G. Ghirardi, R. Grassi, and M. Michelini, A Fundamental Concept in Quantum Theory, in \textit{Thinking Physics for Teaching}, edited by C. Bernardini, C. Tarsitani, and M. Vicentini (Springer, New York, 1996) pp. 329-334.
\bibitem{Michelini2004} M. Michelini, R. Ragazzon, L. Santi, and A. Stefanel, Discussion of a Didactic Proposal on Quantum Mechanics with Secondary School Students, Il Nuovo Cimento C, \textbf{27}, 555 (2004).
\bibitem{Michelini2019} M. Michelini and A. Stefanel, A path to build basic Quantum Mechanics ideas in the context of light polarization and learning outcomes of secondary students, J. Phys.: Conf. Ser., 1929 012052 (2021).
\bibitem{Michelini2002} M. Michelini, L. Santi, A. Stefanel, and G. Meneghin, A resource environment to introduce quantum physics in secondary school, in \textit{Proceedings of MPTL-7 International Workshop} (2002), Retrieved from: \url{www.ud.infn.it/URDF/ffc/quanto/articoli/art_11.pdf}
\bibitem{Gilbert2002} R. Justi and J. K. Gilbert, Modelling, teachers' views on the nature of modelling, and implications for the education of modellers, Int. J. Sci. Educ., \textbf{24}(4), 369-387 (2002).
\bibitem{Gilbert2016} J. K. Gilbert and R. Justi, \textit{Modelling-Based Teaching in Science Education} (Springer, Switzerland, 2016).
\bibitem{Grangier1986} P. Grangier, G. Roger, and A. Aspect, Experimental evidence for a photon anticorrelation effect on a beam splitter: a new light on single-photon interferences, EPL (Europhysics Letters), \textbf{1}, 173 (1986).
\bibitem{Grangier2005} P. Grangier, Experiments with single photons, Seminaire Poincar\'{e} (2005), retrieved from: \url{http://www.bourbaphy.fr/grangier}
\bibitem{Heyde2020} K. Heyde and J. L. Wood, \textit{Quantum Mechanics for Nuclear Structure, Volume 1} (IOP Publishing, Bristol, UK, 2020).
\bibitem{Dirac1967} P. A. M. Dirac, \textit{The principles of quantum mechanics, 4th ed. (revised)} (Clarendon Press, UK, 1967)
\bibitem{Llewellyn2012} D. Llewellyn, \textit{Teaching high school science through inquiry and argumentation, 2nd edition} (Corwin Press, Rochester, NY, 2012).
\bibitem{Galili2007} I. Galili, Thought Experiments: Determining Their Meaning, Sci. Educ. (Dordr.), \textbf{18}, 1-23 (2007).
\bibitem{Brown1991} J. Brown, \textit{The Laboratory of the Mind: Thought Experiments in the Natural Sciences} (Routledge, London, UK, 1991).
\bibitem{Etkina2006} E. Etkina, A. Van Heuvelen, S. White-Brahmia, D. T. Brookes, M. Gentile, S. Murthy, D. Rosengrant, and A. Warren, Scientific abilities and their assessment, PRST-PER, \textbf{2}, 020103 (2006).
\bibitem{Passon2019} O. Passon, T. Z\"{u}gge and J. Grebe-Ellis, Pitfalls in the teaching of elementary particle physics, Phys. Educ., \textbf{54}, 015014 (2019).
\bibitem{Norsen2017} T. Norsen, \textit{Foundations of Quantum Mechanics: An Exploration of the Physical Meaning of Quantum Theory} (Springer International Publishing, Cham, Switzerland, 2017).
\bibitem{Debianchi2013} M. Sassoli De Bianchi, Quantum fields are not fields; comment on ``There are no particles, there are only fields,'' by Art Hobson [Am. J. Phys. \textbf{81}, 211 (2013)], Am. J. Phys., \textbf{81}, 707 (2013).
\bibitem{Isham1995} C. J. Isham, \textit{Lectures on quantum theory: Mathematical and structural foundations} (Imperial College Press, London, UK, 1995).
\bibitem{Schilpp1998} P. A. Schilpp, \textit{Albert Einstein, Philosopher-Scientist: The Library of Living Philosophers Volume VII, 3rd edition} (Open Court Publishing Co, US, 1998).
\bibitem{Bell2010} J. S. Bell and A. Aspect, \textit{Speakable and Unspeakable in Quantum Mechanics, 2nd edition, with a new introduction by Alain Aspect} (Cambridge University Press, UK, 2010).
\bibitem{Tanona2004} S. Tanona, Uncertainty in Bohr's response to the Heisenberg microscope, Stud. Hist. Philos. Sci. \textbf{35}, 483-507 (2004).
\bibitem{House2018} House of Representatives 6227-National Quantum Initiative Act (2018) 115th Congress, 164:132 STAT. 5092. \url{www.congress.gov/bill/115th-congress/house-bill/6227}.
\bibitem{Bub1998} J. Bub, Quantum measurement problem,  in \textit{Routledge Encyclopedia of Philosophy}, edited by E. Craig (Taylor and Francis, 1998), URL: \url{https://www.rep.routledge.com/articles/thematic/quantum-measurement-problem/v-1}
\bibitem{Kragh1992} H. Kragh, A Sense of History: History of Science and the Teaching of Introductory Quantum Theory, Sci. Educ. (Dordr.) \textbf{1}, 349-363 (1992).
\bibitem{McDermott2013} L. C. McDermott, Improving the teaching of science through discipline-based education research: An example from physics, EJMSE, \textbf{1}(1), 1-12 (2013).
\bibitem{Bakker2015} A. Bakker and D. van Eerde (2015), An introduction to design-based research with an example from statistics education, in \textit{Approaches to qualitative research in mathematics education}, edited by A. Bikner-Ahsbahs, C. Knipping, and N. Presmeg (Springer Netherlands, 2015), pp. 429-466.
\bibitem{Erickson2012} F. Erickson, Qualitative research methods for science education, in \textit{Second international
handbook of science education}, edited by  B.J. Fraser, K. Tobin, and C.J. McRobbie (Springer, Dordrecht, 2012), pp. 1451-1469.
\bibitem{White1992} R. White and R. Gunstone, \textit{Probing understanding} (The Falmer Press, London, UK, 1992).
\bibitem{Michelini2014} M. Michelini and G. Zuccarini, University students’ reasoning on physical information encoded in quantum state at a point in time, in \textit{Proceedings of PERC 2014}, edited by P. V. Engelhardt, A. Churukian, and D. L. Jones. (Minneapolis, USA, 2014), pp. 187-190.
\bibitem{McIntyre2012} D. H. McIntyre, C. A. Manogue, C. A., and J. Tate, \textit{Quantum Mechanics: A Paradigms Approach} (Pearson Addison-Wesley, San Francisco, CA, 2012).


\end{thebibliography}

\end{document}





