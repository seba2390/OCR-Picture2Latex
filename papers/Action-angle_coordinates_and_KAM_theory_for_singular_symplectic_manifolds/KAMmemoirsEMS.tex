%-----------------------------------------------------------------------
% Beginning of chapter.tex
%-----------------------------------------------------------------------
%
%  This is a sample file for use with AMS-LaTeX.  It provides an example
%  of how to set up a file for a book to be typeset with AMS-LaTeX.
%
%  This is the driver file.  Separate chapters should be included at
%  the end of this file.
%  ***** DO NOT USE THIS FILE AS A STARTER FOR YOUR BOOK. *****
%  Follow the guidelines in the file chapter.template.
%
%%%%%%%%%%%%%%%%%%%%%%%%%%%%%%%%%%%%%%%%%%%%%%%%%%%%%%%%%%%%%%%%%%%%%%%%

\documentclass{amsbook}

\usepackage{amsmath,amsthm,verbatim,amssymb,amsfonts,amscd, graphicx,lmodern,color,braket,tikz,mathtools,enumitem,float}
\usepackage{afterpage}
\usepackage[toc,page]{appendix}
\usepackage{graphicx}  %demo is to compile without .png [demo]
\usepackage{tikz-cd}
\usepackage{pgfplots}
\usepackage{subfigure}
\usepackage{longtable}
\usetikzlibrary{arrows}
\usetikzlibrary{calc}
\usetikzlibrary{matrix}

\includeonly{preface,introduction,chap1,chap2,chap3,biblio,bibliarnau,index}

\newtheorem{theorem}{Theorem}[chapter]
\newtheorem*{theorem*}{Theorem}
\newtheorem{lemma}[theorem]{Lemma}

\theoremstyle{definition}
\newtheorem{definition}[theorem]{Definition}
\newtheorem{example}[theorem]{Example}
\newtheorem{xca}[theorem]{Exercise}


\newtheorem{proposition}[theorem]{Proposition}
\newtheorem{corollary}[theorem]{Corollary}
\newtheorem{examples}[theorem]{Examples}
\newtheorem*{examples*}{Examples}

\newtheorem*{theoremA}{Theorem A}
\newtheorem*{theoremB}{Theorem B}
\newtheorem*{theoremC}{Theorem C}
\newtheorem*{theoremD}{Theorem D}
\newtheorem*{theoremE}{Theorem E}

\theoremstyle{remark}
\newtheorem{remark}[theorem]{Remark}

\def\Rn{\ensuremath\mathbb{R}^n}
\def\R{\ensuremath\mathbb{R}}

\def\M{\mathcal{M}}
\def\T{\mathbb{T}}
\def\E{\mathcal{E}}
\def\h{\mathfrak{h}}

\def\area{\text{area}}
\def\id{\text{Id}}

\numberwithin{section}{chapter}
\numberwithin{equation}{chapter}

%    Absolute value notation
\newcommand{\abs}[1]{\lvert#1\rvert}

%    Blank box placeholder for figures (to avoid requiring any
%    particular graphics capabilities for printing this document).
\newcommand{\blankbox}[2]{%
  \parbox{\columnwidth}{\centering
%    Set fboxsep to 0 so that the actual size of the box will match the
%    given measurements more closely.
    \setlength{\fboxsep}{0pt}%
    \fbox{\raisebox{0pt}[#2]{\hspace{#1}}}%
  }%
}


\usepackage{graphicx}



% ARROWS
\usepackage{tikz-cd}
\usepackage{circuitikz}
%\usetikzlibrary{positioning}
\usetikzlibrary{shapes.geometric, arrows}
\newcommand{\midarrowr}{\tikz \draw[-triangle 90] (0,0) -- +(.1,0);}
\newcommand{\midarrowu}{\tikz \draw[-triangle 90] (0,0) -- +(0,.1);}
\newcommand{\midarrowd}{\tikz \draw[-triangle 90] (0,0.1) -- (0,0);}
\newcommand{\midarrowl}{\tikz \draw[-triangle 90] (0.1,0) -- (0,0);}





\newcommand\pgfmathsinandcos[3]{%
  \pgfmathsetmacro#1{sin(#3)}%
  \pgfmathsetmacro#2{cos(#3)}%
}
\newcommand\LatitudePlane[4][current plane]{%
  \pgfmathsinandcos\sinEl\cosEl{#2} % elevation
  \pgfmathsinandcos\sint\cost{#3} % latitude
  \pgfmathsetmacro\yshift{\cosEl*\sint}
  \tikzset{#1/.estyle={cm={\cost,0,0,\cost*\sinEl,(#4,\yshift)}}} %
}
\newcommand\DrawLatitudeCircle[3][1]{
  \LatitudePlane{\angEl}{#2}{#3}
  \tikzset{current plane/.prefix style={scale=#1}}
  \pgfmathsetmacro\sinVis{sin(#2)/cos(#2)*sin(\angEl)/cos(\angEl)}
  % angle of "visibility"
  \pgfmathsetmacro\angVis{asin(min(1,max(\sinVis,-1)))}
  \draw[current plane, thick, color = red] (\angVis:1) arc (\angVis:-\angVis-180:1);
  \draw[current plane,dashed, thick, color = red] (180-\angVis:1) arc (180-\angVis:\angVis:1);
}

\newcommand\DrawGenus[7]{% (x, y), size, size*1.5, rotation, color, thickness
  \pgfmathsetmacro{\xstart}{#1 - (0.985*#4)}
  \pgfmathsetmacro{\ystart}{#2 + (0.2*#3)}
  %\draw (#1, #2) node[] {axis};
	\draw[color = #6, rotate around={#5:(#1,#2)}, #7] (\xstart, \ystart) arc (190:350:#4  and #3);
	\draw[color = #6, rotate around={#5:(#1,#2)}, #7] (\xstart, \ystart) arc (190:210:#4  and #3) arc (150:30:#4  and #3) arc (330:350:#4  and #3);
}

\newcommand\DrawFilledGenus[8]{% (x, y), size, size*1.5, rotation, color, thickness
  \pgfmathsetmacro{\xstart}{#1 - (0.985*#4)}
  \pgfmathsetmacro{\ystart}{#2 + (0.2*#3)}
  %\draw (#1, #2) node[] {axis};
	\draw[color = #6, rotate around={#5:(#1,#2)}, #7] (\xstart, \ystart) arc (190:350:#4  and #3);
	\draw[color = #6, rotate around={#5:(#1,#2)}, #7] (\xstart, \ystart) arc (190:210:#4  and #3) arc (150:30:#4  and #3) arc (330:350:#4  and #3);
	\draw[color = #6, rotate around={#5:(#1,#2)}, #7, fill = #8] (\xstart, \ystart) arc (190:210:#4  and #3) arc (150:30:#4  and #3) arc (-30:-150:#4  and #3);
}

\newcommand\DrawDonut[7]{% (x, y), size, size*1.5, rotation
  \pgfmathsetmacro{\fctr}{.08}
  \pgfmathsetmacro{\newwidth}{0.5*#4}
  \pgfmathsetmacro{\newheight}{0.5*#3}
  \draw[color = #6, rotate around={#5:(#1,#2)}, #7] (#1, #2) ellipse (#4  and #3);
  \DrawGenus{#1}{#2}{\newheight}{\newwidth}{#5}{#6}{#7}
}

\newcommand\DrawFilledDonut[8]{% (x, y), size, size*1.5, rotation
  \pgfmathsetmacro{\fctr}{.08}
  \pgfmathsetmacro{\newwidth}{0.5*#4}
  \pgfmathsetmacro{\newheight}{0.5*#3}
  \draw[color = #6, rotate around={#5:(#1,#2)}, #7, fill = #6] (#1, #2) ellipse (#4  and #3);
  \DrawFilledGenus{#1}{#2}{\newheight}{\newwidth}{#5}{#6}{#7}{#8}
  \DrawGenus{#1}{#2}{\newheight}{\newwidth}{#5}{#6}{#7}
}

\newcommand\DrawFilledDonutops[8]{% (x, y), size, size*1.5, rotation
  \pgfmathsetmacro{\fctr}{.08}
  \pgfmathsetmacro{\newwidth}{0.5*#4}
  \pgfmathsetmacro{\newheight}{0.5*#3}
  \draw[color = #6, rotate around={#5:(#1,#2)}, #7, fill = #6, opacity = .6] (#1, #2) ellipse (#4  and #3);
  \DrawFilledGenus{#1}{#2}{\newheight}{\newwidth}{#5}{#6}{#7}{#8}
}

\newcommand\DrawFilledDonutopt[8]{% (x, y), size, size*1.5, rotation
  \pgfmathsetmacro{\fctr}{.08}
  \pgfmathsetmacro{\newwidth}{0.5*#4}
  \pgfmathsetmacro{\newheight}{0.5*#3}
  \draw[color = #6, rotate around={#5:(#1,#2)}, #7, fill = #6, opacity = .2] (#1, #2) ellipse (#4  and #3);
  \DrawFilledGenus{#1}{#2}{\newheight}{\newwidth}{#5}{#6}{#7}{#8}
}

\newcommand\DrawTwoDonut[7]{% (x, y), size, size*1.5, rotation, color, thick
  \pgfmathsetmacro{\fctr}{.08}
  \pgfmathsetmacro{\newwidth}{0.25*#4}
  \pgfmathsetmacro{\newheight}{0.25*#3}
  \draw[color = #6, rotate around={#5:(#1,#2)}, #7] (#1, #2) ellipse (#4  and #3);
  \pgfmathsetmacro{\xcoordstart}{#1 - (#4*.4)*cos(#5) - (0.05*#3)*sin(#5)}
  \pgfmathsetmacro{\ycoordstart}{#2 - (#4*.4)*sin(#5) - (0.05*#3)*cos(#5)}
  \DrawGenus{\xcoordstart}{\ycoordstart}{\newheight}{\newwidth}{#5}{#6}{#7}
  \pgfmathsetmacro{\xcoordstart}{#1 + (#4*.4)*cos(#5) - (0.05*#3)*sin(#5)}
  \pgfmathsetmacro{\ycoordstart}{#2 + (#4*.4)*sin(#5) - (0.05*#3)*cos(#5)}
  \DrawGenus{\xcoordstart}{\ycoordstart}{\newheight}{\newwidth}{#5}{#6}{#7}
}

\newcommand\DrawFilledTwoDonutops[8]{% (x, y), size, size*1.5, rotation, color, thick
  \pgfmathsetmacro{\fctr}{.08}
  \pgfmathsetmacro{\newwidth}{0.25*#4}
  \pgfmathsetmacro{\newheight}{0.25*#3}
  \draw[color = #6, rotate around={#5:(#1,#2)}, #7, fill = #6, opacity = .4] (#1, #2) ellipse (#4  and #3);
  \draw[color = #6, rotate around={#5:(#1,#2)}, #7] (#1, #2) ellipse (#4  and #3);
  \pgfmathsetmacro{\xcoordstart}{#1 - (#4*.4)*cos(#5) - (0.05*#3)*sin(#5)}
  \pgfmathsetmacro{\ycoordstart}{#2 - (#4*.4)*sin(#5) - (0.05*#3)*cos(#5)}
  \DrawFilledGenus{\xcoordstart}{\ycoordstart}{\newheight}{\newwidth}{#5}{#6}{#7}{#8}
  \pgfmathsetmacro{\xcoordstart}{#1 + (#4*.4)*cos(#5) - (0.05*#3)*sin(#5)}
  \pgfmathsetmacro{\ycoordstart}{#2 + (#4*.4)*sin(#5) - (0.05*#3)*cos(#5)}
  \DrawFilledGenus{\xcoordstart}{\ycoordstart}{\newheight}{\newwidth}{#5}{#6}{#7}{#8}
}


\tikzstyle{mytheorembox} = [draw=vdgreen, fill=blue!20, very thick, rectangle, rounded corners, inner sep=10pt, inner ysep=15pt]
\tikzstyle{mytheoremfancytitle} =[fill=vdgreen, text=white]

%


\definecolor{vdblue}{rgb}{0,0,.3}
\definecolor{dblue}{rgb}{0,0,.7}
\definecolor{lblue}{rgb}{.3,.3,1}
\definecolor{vlblue}{rgb}{.7,.7,1}
\definecolor{vvlblue}{rgb}{.9,.9,1}


\definecolor{vdred}{rgb}{.3,0,0}
\definecolor{dred}{rgb}{.7,0,0}
\definecolor{lred}{rgb}{1,.3,.3}
\definecolor{vlred}{rgb}{1,.7,.7}

\definecolor{vdgreen}{rgb}{0,.2,0}
\definecolor{dgreen}{rgb}{0,.4,0}
\definecolor{lgreen}{rgb}{.3,1,.3}
\definecolor{vlgreen}{rgb}{.7,1,.7}

\definecolor{lyellow}{rgb}{1,1,.3}


\definecolor{gray1}{rgb}{0.22,0.22,0.22}
\definecolor{gray2}{rgb}{0.28,0.28,0.28}
\definecolor{gray3}{rgb}{0.36,0.36,0.36}
\definecolor{gray4}{rgb}{0.44,0.44,0.44}
\definecolor{gray5}{rgb}{0.52,0.52,0.52}
\definecolor{gray6}{rgb}{0.6,0.6,0.6}
\definecolor{gray7}{rgb}{0.68,0.68,0.68}
\definecolor{gray8}{rgb}{0.76,0.76,0.76}

\definecolor{color1}{rgb}{1,0,0}
\definecolor{color2}{rgb}{0.98,0,0.816}
\definecolor{color3}{rgb}{0.717,0,1}
\definecolor{color4}{rgb}{0,0,1}
\definecolor{color5}{rgb}{0,1,1}
\definecolor{color6}{rgb}{0,1,0}
\definecolor{color8}{rgb}{1,1,0}
\definecolor{color7}{rgb}{1,0.651,0}



\begin{document}
\frontmatter
\title{Action-angle coordinates and KAM theory for singular symplectic manifolds}

%    Information for first author
\author{Eva Miranda}
%    Address of record for the research reported here
\address{Laboratory of Geometry and Dynamical Systems, Department of Mathematics \& IMTech, Universitat Polit\`ecnica de Catalunya, Barcelona and CRM, Centre de Recerca Matemàtica, Bellaterra}
%    Current address
%\curraddr{UPC-Edifici P, Avinguda del Doctor Maranón, 44-50, 08028, Barcelona, Spain}
\email{evamiranda@upc.edu}
%    \thanks will become a 1st page footnote.

\thanks{ Both authors are supported by the project PID2019-103849GB-I00 of the Spanish State Agency  AEI /10.13039/501100011033 and by the AGAUR Gencat project 2021 SGR 00603. Eva Miranda is supported by the Catalan Institution for Research and Advanced Studies via an ICREA Academia Prize 2021 and by the Alexander Von Humboldt foundation via a Friedrich Wilhelm Bessel Research Award.  Eva Miranda is also supported by the Spanish State
Research Agency, through the Severo Ochoa and Mar\'{\i}a de Maeztu Program for Centers and Units
of Excellence in R\&D (project CEX2020-001084-M). Eva Miranda also acknowledges partial support from the grant
“Computational, dynamical and geometrical complexity in fluid dynamics”, Ayudas Fundación BBVA a Proyectos de Investigación Científica 2021.  }

%    Information for second author
\author{Arnau Planas}
\address{Department of Mathematics, Universitat Polit\`ecnica de Catalunya, Barcelona}
\email{arnau.planas.bahi@gmail.com}

\date{May, 2023}
\subjclass[2020]{ 53D05, 53D20, 70H08, 37J35, 37J40 ;\\ 37J39, 58D19 }
\keywords{\texttt{amsbook}, AMS-\LaTeX}

\maketitle
\dedicatory{Dedicated to the memory of Amelia Galcerán Sorribes.  }
\setcounter{page}{4}
\tableofcontents

% --------------------------------------------------------------------------- %
% Preface (abstract, classification, author details, etc.)
% --------------------------------------------------------------------------- %

% --------------------------------------------------------------------------- %
% Title, Author, and Page Heads
% --------------------------------------------------------------------------- %
\title{A Domain-Specific Language and Editor for Parallel Particle Methods }

\author{SVEN KAROL$^1$, TOBIAS NETT$^1$, JERONIMO CASTRILLON$^1$ and IVO F.
SBALZARINI$^{1,2}$ 
\affil{$^1$: Technische Universit\"at Dresden, Faculty of Computer Science, Dresden, Germany\\
$^2$: Center for Systems Biology Dresden, Max Planck Institute of Molecular Cell Biology 
and Genetics, Dresden, Germany}}


% --- End of Author Metadata ---

\markboth{S. Karol, T. Nett, J. Castrillon and I. F. Sbalzarini}{A Domain-Specific Language and Editor for Parallel Particle Methods}

%
% At a minimum you need to supply the author names, year and a title.
% IMPORTANT:
% Full first names whenever they are known, surname last, followed by a period.
% In the case of two authors, 'and' is placed between them.
% In the case of three or more authors, the serial comma is used, that is, all author names
% except the last one but including the penultimate author's name are followed by a comma,
% and then 'and' is placed before the final author's name.
% If only first and middle initials are known, then each initial
% is followed by a period and they are separated by a space.
% The remaining information (journal title, volume, article number, date, etc.) is 'auto-generated'.
\acmformat{Sven Karol, Tobias Nett, Jeronimo Castrillon and Ivo F.~Sbalzarini. 2017. A Domain-specific Language and Editor for Parallel Particle Methods}

% --------------------------------------------------------------------------- %
% Abstract
% --------------------------------------------------------------------------- %

\begin{abstract}
% JCM: abstract on 200 chars
Domain-specific languages (DSLs) are of increasing importance in scientific high-performance computing to reduce development costs, raise the level of abstraction and, thus, ease scientific programming. However, designing DSLs is not easy, as it requires knowledge of the application domain and experience in language engineering and compilers. Consequently, many DSLs follow a weak approach using macros or text generators, which lack many of the features that make a DSL comfortable for programmers. Some of these features---e.g., syntax highlighting, type inference, error reporting---are easily provided by language workbenches, which combine language engineering techniques and tools in a common ecosystem. In this paper, we present the Parallel Particle-Mesh Environment (PPME), a DSL and development environment for numerical simulations based on particle methods and hybrid particle-mesh methods. PPME uses the Meta Programming System (MPS), a projectional language workbench. PPME is the successor of the Parallel Particle-Mesh Language, a Fortran-based DSL that uses conventional implementation strategies. We analyze and compare both languages and demonstrate how the programmer's experience is improved using static analyses and projectional editing\revii{, i.e., code-structure editing, constrained by syntax, as opposed to free-text editing}. We present an explicit domain model for particle abstractions and the first formal type system for particle methods.
%%  Domain-specific languages (DSLs) are of increasing importance in scientific high-performance computing to reduce development costs, raise the level of  
%%  abstraction and, thus, ease scientific programming.
%%  However, designing and implementing DSLs is not an easy task, as it requires knowledge of the application domain 
%%  as well as skills and experience in language engineering and compiler construction. Consequently, many  
%%  DSLs follow a weak approach using macros or text generators, which lack many of 
%%  the features that make a DSL a comfortable tool for scientific programmers. Some of these features---e.g., syntax highlighting, instant
%%  analysis, type inference, error reporting, and code completion---are easily provided by language workbenches, which group and combine language
%%  engineering techniques and tools in a common ecosystem.
%%  % 
%%  In this paper, we present the Parallel Particle-Mesh Environment (PPME), a DSL and development environment for numerical simulations based on   
%%  particle methods and hybrid particle-mesh methods. PPME has been created using the meta programming system (MPS), a projectional language workbench developed by JetBrains.
%%  PPME is the successor of the Parallel Particle-Mesh Language (PPML), a Fortran-based DSL that used conventional implementation strategies, such as macro
%%  expansion. We analyze and compare both languages and demonstrate how the programmer's experience can be improved using static
%%  analyses and projectional editing. Furthermore, we present an explicit domain model for particle abstractions and the first formal type system for particle
%%   methods.  
 %
%  We present the Parallel Particle-Mesh Environment (PPME) is a DSL and projectional editor for numerical simulations based on the particle method.
%  PPME implements a generative approach: it generates parallel Fortran code that links with the parallel particle-mesh library (PPM), which is also implemented in Fortran.
%  PPM provides efficient implementations of the particle and mesh abstractions, discrete numerics, as well as an abstraction layer on the underlying HPC hardware.
%
%  The presented DSL supports built-in abstractions such as particles, properties, fields, loops, as well as systems of partial differential equations and differential operators such as the Laplacian.
%  Equations can be written using conventional mathematical notations.
%
%  - analysis features such as, for instance type analysis and dead-code analysis. 
%  - potential optimizations to improve the efficiency of generated code
%  - enhancements to improve user experience
%
%  - extensible, modular
%  - design and implementation of a type system
%  - type system extension with physical units/dimensions
%  - tool integration for static analysis and optimization
%  - evaluation
\end{abstract}

% --------------------------------------------------------------------------- %
% ACM Classification
% --------------------------------------------------------------------------- %
%
% ACM publications are classified according to the ACM Computing Classification Scheme (CCS). CCS codes are used both in the typeset version of the publications and in the metadata in the various databases. Therefore you need to provide both TEX commands and XML metadata with the paper.
%
% The code below should be generated by the tool at
% http://dl.acm.org/ccs.cfm
% Please copy and paste the code instead of the example below. 
%
\begin{CCSXML}
 <ccs2012>
  <concept>
   <concept_id>10011007.10011006.10011066.10011070</concept_id>
   <concept_desc>Software and its engineering~Application specific development environments</concept_desc>
   <concept_significance>500</concept_significance>
  </concept>
<concept>
<concept_id>10010147.10010341.10010349.10010355</concept_id>
<concept_desc>Computing methodologies~Agent / discrete models</concept_desc>
<concept_significance>300</concept_significance>
</concept>
<concept>
<concept_id>10010147.10010341.10010366.10010368</concept_id>
<concept_desc>Computing methodologies~Simulation languages</concept_desc>
<concept_significance>300</concept_significance>
</concept>
<concept>
<concept_id>10002950.10003705.10003707</concept_id>
<concept_desc>Mathematics of computing~Solvers</concept_desc>
<concept_significance>100</concept_significance>
</concept>
  </ccs2012>
\end{CCSXML}

\ccsdesc[500]{Software and its engineering~Application specific development environments}
\ccsdesc[300]{Computing methodologies~Agent / discrete models}
\ccsdesc[300]{Computing methodologies~Simulation languages}
\ccsdesc[100]{Mathematics of computing~Solvers}

%
% The command \terms is obsolete; we no longer use “General Terms” line.
%
%\terms{Design, Languages}

% --------------------------------------------------------------------------- %
% Keywords
% --------------------------------------------------------------------------- %
%
% The “Additional Keywords and Phrases” item on the title page is provided by the \keywords declaration, listed alphabetically.
% There is no prescribed list of “additional keywords;” use any that you want.
\keywords{language workbenches, mathematical software, MPS, particle methods, scientific computing}

\begin{bottomstuff}
\textit{Preprint}. This work is partly supported by the German Research Foundation (DFG) within the Cluster of Excellence “Center for Advancing Electronics Dresden” (EXC 1056). 
\end{bottomstuff}
% --------------------------------------------------------------------------- %
% Generate title
% --------------------------------------------------------------------------- %
\maketitle


\mainmatter
\section{Introduction}  \label{sec:introduction}

\newcommand\inexpIntro[3]{#1?(#2,#3).}
\newcommand\rinexpIntro[3]{*#1?(#2,#3).}
\newcommand\outexpIntro[3]{#1!(#2,#3).}
\newcommand\outatomIntro[3]{#1!(#2,#3)}

We propose a fully automated method for proving termination of \(\pi\)-calculus processes.
Although there have been a lot of studies on termination analysis for the \(\pi\)-calculus
and related calculi~\cite{Deng06IC,Demangeon07,SangiorgiTermination,KobayashiHybrid,Yoshida04IC,DBLP:journals/jlp/DemangeonHS10,Venet98SAS}, most of them have been rather theoretical,
and there have been surprisingly little efforts in developing  fully automated termination
verification methods and tools based on them. To our knowledge,
Kobayashi's \typical{}~\cite{TyPiCal,KobayashiHybrid} is the only exception that
can prove termination of \(\pi\)-calculus processes (extended with natural numbers)
fully automatically, but its termination analysis is quite limited (see Section~\ref{sec:relatedwork}).

Our method is based on a reduction to termination analysis for sequential programs:
we translate a \(\pi\)-calculus process \(P\) to a sequential program \(S_P\), so that
if \(S_P\) is terminating, so is \(P\). The reduction allows us to use
powerful, mature methods and tools
for termination analysis of sequential programs~\cite{heizmann2016ultimate,freqterm,DBLP:conf/lics/PodelskiR04,Kuwahara2014Termination,DBLP:journals/cacm/CookPR11}.

The idea of the translation is to convert a chain of communications on replicated input
channels to a chain of recursive function calls of the target sequential program.
Let us consider the following Fibonacci process:
\begin{align*}
    & \rinexpIntro{\fib}{n}{r}
        \ifexp{n<2}{ \soutatom{r}{1} \\ &\quad}
                   { \nuexp{s_1} \nuexp{s_2} (\outatomIntro{\fib}{n-1}{s_1} \PAR \outatomIntro{\fib}{n-2}{s_2} \PAR \sinexp{s_1}{x}\sinexp{s_2}{y}\soutatom{r}{x+y}) \\}
    & \PAR \outatomIntro{\fib}{m}{r}
\end{align*}
Here, the process
$\rinexpIntro{\fib}{n}{r} \ldots$ is a function server that computes the \(n\)-th Fibonacci number
in parallel and returns the result to \(r\),
and $\outatom{\fib}{m}{r}$ sends a request for computing the \(m\)-th Fibonacci number;
those who are not familiar with the syntax of the \(\pi\)-calculus may wish to consult
Section~\ref{sec:targetlanguage} first.
To prove that the process above is terminating for any integer \(m\),
it suffices to show that there is no infinite chain of communications on $\fib$:
\[
    \fib(m,r) \to \fib(m_1,r_1) \to \fib(m_2,r_2) \to \cdots.
\]
We convert the process above to the following program:\footnote{The actual translation
  given later is a little more complex.}
\begin{verbatim}
 let rec fib(n) = if n<2 then () else (fib(n-1) [] fib(n-2)) in
 fib(m)
\end{verbatim}
Here, \texttt{[]} represents the non-deterministic choice.
Note that, although the calculation of Fibonacci numbers is not preserved,
for each chain of communications on \texttt{fib}, there is a corresponding
sequence of recursive calls:
\[
\mathtt{fib}(m) \to \mathtt{fib}(m_1) \to \mathtt{fib}(m_2) \to \cdots.
\]
Thus, the termination of the sequential program above implies the termination of
the original process.
As shown in the example above, (i) each communication on a replicated input channel
is converted to a function call, (ii) each communication on a non-replicated input
channel is just removed (or, in the actual translation, replaced by a call of
a trivial function defined by \(f(\seq{x})=(\,)\)), and (iii) parallel composition
is replaced by a non-deterministic choice.
We formalize the translation outlined above and prove its correctness.

The basic translation sketched above sometimes loses too much information.
For example, consider the following process:
\begin{align*}
    & \rinexpIntro{\pre}{n}{r} \soutatom{r}{n-1} \\
    & \PAR \rinexpIntro{f}{n}{r} \ifexp{n<0}{ \soutatom{r}{1} }
                                       { \nuexp{s} (\outatomIntro{\pre}{n}{s} \PAR \sinexp{s}{x}\outatomIntro{f}{x}{r}) } \\
    & \PAR \outatomIntro{f}{m}{r}
\end{align*}
The translation sketched above would yield:
\begin{verbatim}
  let pred(n) = n-1 in
  let rec f(n) = if n<0 then () else (pred(n) [] f(*)) in
  f(m)
\end{verbatim}
Here, \texttt{*} represents a non-deterministic integer: since we have removed
the input $\sinatom{s}{x}$, we do not have information about the value of \( x \).
As a result, the sequential program above is non-terminating, although the original
process is terminating.
To remedy this problem, we also refine the basic translation above by using a refinement
type system for the \(\pi\)-calculus. Using the refinement type system,
we can infer that the value of \(x\) in the original process is less than \(n\),
so that we can refine the definition of \texttt{f} to:
\begin{verbatim}
 let rec f(n) = ... else (pred(n) [] let x=* in assume(x<n);f(x))
\end{verbatim}
The target program is now terminating, from which
we can deduce that the original process is also terminating.
We have implemented an automated tool based on the refined translation above.

The contributions of this paper are summarized as follows.
\begin{itemize}
\item The formalization of the basic translation from the \(\pi\)-calculus
  (extended with integers) to sequential programs, and a proof of its correctness.
\item The formalization of a refined translation based on a refinement type system.
\item An implementation of the refined translation, including automated refinement type
  inference based on CHC solving, and experiments to evaluate the effectiveness of
  our method.
\end{itemize}

The rest of this paper is structured as follows.
Section~\ref{sec:targetlanguage} introduces the source and target languages
of our translation.
Section~\ref{sec:approach} 
formalizes the basic translation, and proves its correctness.
Section~\ref{sec:refinement} refines the basic translation by using a refinement type system.
Section~\ref{sec:implementation} reports an implementation and experiments.
Section~\ref{sec:relatedwork} discusses related work,
and Section~\ref{sec:conclusion} concludes the paper.


%
%\include{chap1}
%-----------------------------------------------------------------------
% Beginning of chap1.tex
%-----------------------------------------------------------------------
%
%  AMS-LaTeX sample file for a chapter of a monograph, to be used with
%  an AMS monograph document class.  This is a data file input by
%  chapter.tex.
%
%  Use this file as a model for a chapter; DO NOT START BY removing its
%  contents and filling in your own text.
%
%%%%%%%%%%%%%%%%%%%%%%%%%%%%%%%%%%%%%%%%%%%%%%%%%%%%%%%%%%%%%%%%%%%%%%%%

\part[Action-angle coordinates and cotangent models]{Action-angle coordinates and cotangent models for $b^m$-integrable systems}


In this part,  we consider the semilocal classification for any $b^m$-Poisson manifold in a neighbourhood of an invariant compact submanifold.
The compact submanifolds under consideration are the compact invariant leaves of the distribution $\mathcal D$ generated by the Hamiltonian vector fields $X_{f_i}$ of an integrable system. An integrable system is given by a set of $n$ functions on a $2n$-dimensional symplectic manifold  which we can order in a map $F=(f_1, \dots, f_n)$. Historically, integrable systems were introduced to actually \emph{integrate} Hamiltonian systems $X_H$ using the \emph{first-integrals $f_i$} and, classically, we identify $H=f_1$. It turns out that in the symplectic context the compact regular orbits of the distribution $\mathcal D$ coincide with the fibers $F^{-1}(F(p))$ for any point $p$ on these orbits/fibers.
The fact that the orbit coincides with the connected fiber is part of the magic of symplectic duality.

The same picture is reproduced for singular symplectic manifolds of $b^m$-type or $b^m$-Poisson manifolds as we will see in this chapter.


The study of action-angle coordinates has interest from this geometrical point of view of the classification of geometric structures in a neighbourhood of a compact submanifold of a $b^m$-Poisson manifold. It also has interest from a dynamical point of view as these compact submanifolds now coincide with invariant subsets of the Hamiltonian system under consideration.

From a geometric point of view, the existence of action-angle coordinates determines a \emph{unique} geometrical model for the $b^m$-Poisson (or $b^m$-symplectic) structure in a neighbourhood of the invariant set. From a dynamical point of view, the existence of action-angle coordinates provides a normal form theorem that can be used to study stability and perturbation problems of the Hamiltonian systems (as we will see in the last chapter of this monograph).

An important ingredient that makes our action-angle coordinate theorem \emph{brand-new } from the symplectic perspective is that the system under consideration is more general than Hamiltonian, it is $b^m$-Hamiltonian as the first-integrals of the system can be $b^m$-functions which are not necessarily smooth functions. Dynamically, this means that we are adding to the set of Hamiltonian invariant vector fields, the \emph{modular vector field} of the integrable system.

In contrast to the standard action-angle coordinates for symplectic manifolds, our action-angle theorem comes with $m$ additional invariants associated with the modular vector field which can be interpreted in cohomological terms as the projection of the $b^m$-cohomology class determined by the modular vector field on the first cohomology group of the critical hypersurface under the Mazzeo-Melrose correspondence.

The strategy of the proof of the action-angle coordinate systems is the search of a toric action (so this takes us back to the motivation of the use of \emph{symmetries} in this monograph). In contrast to the symplectic case, it is not enough that this action is Hamiltonian as then a direction of the Liouville torus would be missing. We need the toric action to be $b^m$-Hamiltonian.
The structure of this proof looks like the one in \cite{KMS16} but encounters serious technical difficulties as in order to check that the \emph{natural} action to be considered is
$b^m$-Hamiltonian we need to go deeper inspired by \cite{Scott16}  in the relation between the geometry of the modular vector field and the coefficients of the Taylor series $c_i$ of one of the first-integrals. This allows us to understand new connections between the geometry and analysis of $b^m$-Poisson structures not explored before.

Once we prove the existence of this $b^m$-Hamiltonian action the proof looks very close to the one in \cite{KMS16}.

In the second chapter of this part
we re-state the action-angle theorem as a cotangent lift theorem with the following mantra:


 \emph{Every integrable system on a $b^m$-Poisson manifold looks like a $b^m$-cotangent lift in a neighborhood of a Liouville torus.}




\chapter{An action-angle theorem for $b^m$-symplectic manifolds}

\section{Basic definitions}
\subsection{On $b^m$-functions}
 The definition of the analogue of $b$-functions in the $b^m$-setting is somewhat delicate. The set of $^{b^m}\mathcal{C}^\infty(M)$ needs to be such that for all the functions $f\in ^{b^m}\mathcal{C}^\infty(M)$, its differential $df$ is a $b$-form, where $d$ is the $b^m$-exterior differential.
Recall that a form in $^{b^m}\Omega^k(M)$ can be locally written as
$$\alpha\wedge\frac{dx}{x^m} + \beta$$
where $\alpha \in \Omega^{k-1}(M)$ and $\beta \in \Omega^{k}(M)$. Recall also that
$$d\left(\alpha\wedge\frac{dx}{x^m} + \beta\right) = d\alpha\wedge\frac{dx}{x^m} + d\beta.$$

We need $df$ to be a well-defined $b^m$-form of degree $1$. Let $f = g\frac{1}{x^{k-1}}$, then $df = dg \frac{1}{x^{k-1}} - g\frac{k-1}{x^k}dx$. This from can only be a $b^m$-form if and only if $g$ only depends on $x$. If $f = g\log(x)$, then $dg\log(x) + g\frac{1}{x}dx$, which imposes $dg=0$ and hence $g$ to be constant.

With all this in mind, we make the following definition.
\begin{definition}
The set of $b^m$-functions is defined recursively according to the formula $$~^{b^m} \mathcal{C}^\infty(M)= x^{-(m-1)}\mathcal{C}^\infty(x) + ~^{b^{m-1}} \mathcal{C}^\infty(M)$$

\noindent  {with $\mathcal{C}^\infty(x)$ the set of smooth functions in the defining function $x$} and $$^{b}\mathcal{C}^\infty(M)=\{ g \log\vert x\vert+ h, g \in \mathbb{R}, h\in \mathcal{C}^\infty(M)\}.$$


\end{definition}


\begin{remark}
A $^{b^m} \mathcal{C}^\infty (M)$-function can be written as
$$f = a_0 \log x + a_1\frac{1}{x} + \ldots + a_{m-1}\frac{1}{x^{m-1}} + h$$
where $a_i, h \in \mathcal{C}^\infty(M)$.
\end{remark}

%\begin{remark}
%The Hamiltonian vector field associated to a $b^m$-function $H$ is a smooth vector field. Because locally we can take the expressions:
%$$\Pi = x_1^m\frac{\partial}{\partial x_1} \wedge \frac{\partial}{\partial y_1} + \sum_{i = 2}^{m} \frac{\partial}{\partial x_i}\wedge\frac{\partial}{\partial y_i} \text{ and }H = c_0 \log{x_1} + \sum_{i = 1}^{m-1}c_i \frac{1}{x_1^i} + f.$$
%Then if we compute $X_H = \Pi(dH,\cdot) = c_0 y^{m-1}\frac{\partial}{\partial y_1} + \sum_{i = 1}^{m-1} c_i i y^{m-i-1} \frac{\partial}{\partial y_1} + \Pi(f,\cdot)$, we obtain a smooth vector field.
%\end{remark}

\begin{remark}
From this chapter on we are only considering $b^m$-manifolds $(M,x,Z)$ with $x$ defined up to order $m$. I.e. we can think of $x$ as a jet of a function that coincides up to order $m$ to some defining function.
This is the original viewpoint of Scott in \cite{Scott16} which we adopt from now on. The difference with respect to the other chapters is that we do not fix an specific function.
%The reader may as well adjust her/his reading to assume the function has been fixed in case this is more convenient.
\end{remark}


\begin{definition}
We say that two $b^m$-integrable systems $F_1, F_2$ are equivalent if there exists $\varphi$, a $b^m$-symplectomorphism, i.e. a diffeomorphism preserving both $\omega$ and the critical set $Z$ (``up to order $m$''\textcolor{black}{\footnote{I.e. it preserves the jet $x$}}), such that  $\varphi \circ F_1 = F_2$.
\end{definition}

\begin{remark}
The Hamiltonian vector field associated to a $b^m$-function $f$ is a smooth vector field. Let us compute it  locally using the $b^m$-Darboux theorem:
$$\Pi = x_1^m\frac{\partial}{\partial x_1} \wedge \frac{\partial}{\partial y_1} + \sum_{i = 2}^{m} \frac{\partial}{\partial x_i}\wedge\frac{\partial}{\partial y_i} \text{ and } f = a_0 \log{x_1} + \sum_{i = 1}^{m-1}a_i \frac{1}{x_1^i} + h.$$
Then if we compute

$$\begin{array}{rcl}
\displaystyle df & = & \displaystyle \overbrace{a_0}^{c_1}\frac{1}{x_1} + \sum_{i=1}^{m-1} \overbrace{(a_i' - (i-1)a_{i-1})}^{c_i}\frac{1}{x_1^i} dx_1 \\
& & \quad -\overbrace{(m-1)a_{m-1}}^{c_m}\frac{1}{x_1^m}dx_1 + dh\\
\\
 & = & \displaystyle \sum_{i = 1}^{m}\frac{c_i}{x_1^i} dx_1 + dh.
\end{array}$$
Then,

\begin{equation}\label{eq:bmhamiltonianvf}
X_f = \Pi(df,\cdot) = \sum_{i=1}^{m} c_i x_1^{m-i}\frac{\partial}{\partial y_1} + \Pi(dh,\cdot),
\end{equation}

 we obtain a smooth vector field.



\end{remark}


%
%
%Let $(M^{2n},Z,x)$ be a $b^m$-manifold. The set of $b^m$-functions $^{b^m} \mathcal{C}^\infty (M)$ is defined by:
%$$^{b^m} \mathcal{C}^\infty (M) = x^{-(m-1)}\mathcal{I}\oplus ^b\mathcal{C}^\infty(M)$$
%where $^b\mathcal{C}^\infty(M) = \{c\log(x) + h \quad h \in \mathcal{C}^\infty(M), c \in \mathbb{R}\}$ and $\mathcal{I}$ are the functions that locally around $Z$ only depend on $x$}

\section{On $b^m$-integrable systems}

In this section we present the definition of a $b^m$-integrable system as well as some observations about these objects.

\begin{definition}
Let $(M^{2n},Z,x)$ be a $b^m$-manifold, and let $\Pi$ be a $b^m$-Poisson structure on it.
$F = (f_1, \ldots, f_n)$\footnote{$f_i$ are $b^m$-functions.} is a $b^m$-integrable system\footnote{ In this monograph we only consider integrable systems of maximal rank $n$.} if:

\begin{enumerate}%[i)]
\item $df_1,\ldots,df_n$ are independent on a dense subset of $M$ \textcolor{black}{and in all the points} of $Z$ where independent means that the form $df_1\wedge\ldots\wedge df_n$ is non-zero as a section of $\Lambda^n(^{b^m} T^{*}(M))$,
\item  the functions $f_1,\ldots, f_n$ Poisson commute pairwise.
\end{enumerate}
\end{definition}


\begin{definition}
 The points of $M$ where $df_1,\ldots, df_n$ are independent  are called \textbf{regular} points.
\end{definition}


The next remarks will lead us to a normal form for the first function $f_1$.
\begin{remark}\label{independence}
Note that $df_1,\ldots, df_n$ are independent on a point if and only if $X_{f_1},\ldots,X_{f_n}$ are independent at that point. This is because the map
$$^{b^m}TM \rightarrow ^{b^m}T^*M:u\mapsto \omega_p(u,\cdot)$$
is an isomorphism.
\end{remark}

\begin{remark}
The condition of $d f_1, \ldots, d f_n$ being independent must be understood as $d f_1 \wedge \ldots \wedge df_n$ being a non-zero section of $\bigwedge^n (\enskip ^{b^m}T^*M)$.

\end{remark}

\begin{remark}\label{rk:hamiltonian_vf}
By remark \ref{independence}  the vector fields $X_{f_1},\ldots, X_{f_n}$ have to be independent. This implies that one of the $f_1,\ldots, f_n$ has to be a singular (non-smooth) $b^m$-function with a singularity of maximal degree. If we write $f_i = c_{0,i}\log(x_1) + \sum_{j=1}^{m-1} \frac{c_{j,i}}{x_1^j} + \tilde{f}_1$
$$X_{f_i} = \sum_{j = 1}^{m} x_1^{m-j} \hat{c}_{j,i} \frac{\partial}{\partial y_1} +  X_{\tilde{f}_i}$$
where $\hat c_{j,i}(x) = \frac{d(c_{j,i})}{dx}-(j-1)c_{j-1,i}$. If there is no $b^m$-function with a singularity of maximum degree all the terms in the $\partial/\partial y_1$ direction become 0 at $Z$. And hence $X_{f_1},\ldots, X_{f_n}$ cannot have maximal rank at $Z$.
% begaus hamiltonian VF can only have up to rank n
\end{remark}

%\begin{remark}
%If one of the $f_i$ has a singularity of maximal degree, no other function can have a singularity, because $d f_1 \wedge \ldots \wedge d f_n$ would not be a $b^m$-form and in particular would not be a non-zero section of $\bigwedge^n \enskip ^{b^m}TM$.
%\end{remark}


%\begin{remark}
%From now on we will assume that given a $b^m$-integrable system $(f_1, \ldots, f_n)$, $f_1,\dots,f_{n-1}$ are smooth functions and $f_n = c_{0}\log(x_1) + \sum_{j=1}^{m-1} \frac{c_{j}}{x_1^j}$. Observe that $f_n$ we are taking the smooth part of $f_n$ equal to 0, $\tilde f_n = 0$. This is not necessarily true. But we impose this as a necessary condition to a $b^m$-integrable system

%\end{remark}

\textcolor{black}{
\begin{lemma}
Let $F = (f_1,\ldots,f_n)$ a $b^m$-integrable system. If $f_1$ has a singularity of maximal degree, there exists an equivalent integrable system $F' = (f_1',\ldots,f_n')$ where $f_1'$ has a singularity of maximal degree and \textbf{no other} $f_i'$ has singularity of any degree.
\end{lemma}
\begin{proof}
Let $f_i = \underbrace{c_{0,i}\log(x_1) + \sum_{j=1}^{m-1}\frac{c_{j,1}}{x_1^j}}_{\zeta_i(x_1)} + \tilde f_i = \zeta_i(x_1) + \tilde f _i$.
By remark \ref{rk:hamiltonian_vf}\footnote{Here have used the $b^m$-Darboux theorem to do the computations.},
$$X_{f_i} = \underbrace{\sum_{i=1}^{m} x_1^{m-j} \hat c_{j,i}}_{g_i(x_1)}\frac{\partial}{\partial y_1} + X_{\tilde f_i} = g_i(x_1)\frac{\partial}{\partial y_1} + X_{\tilde f_i}.$$
Note that $g_i(x_1) = g_i(0) = \hat c_{m,i}$ at $Z$. Let us look at the distribution given by the Hamiltonian vector fields $X_{f_i} = g_i(x_1)\frac{\partial}{\partial y_1} + X_{\tilde f_i}$. This distribution is the same that the one given by:
\begin{equation}\label{eq:distribution}
\{X_{f_1}, X_{f_2} - \frac{g_2(x_1)}{g_1(x_1)} X_{f_1},\ldots, X_{f_n} - \frac{g_n(x_1)}{g_1(x_1)} X_{f_1}\}.
\end{equation}
Observe that for $i > 1$, $X_{f_i} - \frac{g_i(x_1)}{g_1(x_1)} X_{f_1} = X_{\tilde f_i} + \frac{g_2(x_1)}{g_1(x_1)} X_{\tilde f_1}$. Also $g_1(x_1)$ is different from $0$ close to $Z$ because at $Z$ $g_1(x_1) = \hat c_{m,1}$.
Since the distribution given by these vector fields is the same, an integrable system that has Hamiltonian vector fields \ref{eq:distribution} would be equivalent to $F$. From the expression \ref{eq:distribution} it is clear that the new vector fields commute. And it is also true that this new vector fields are Hamiltonian. Let us take $F'$ the set of functions that have as Hamiltonian vector fields \ref{eq:distribution}.
\end{proof}
}

From now on we will assume the integrable system to have only one singular function and this function to be $f_1$.

\begin{remark}
Because we asked  $X_{f_1},\ldots,X_{f_n}$ to be linearly independent at all the points of $Z$ and using the previous remarks  $c_m := c_{m,1} \neq 0$ at all the points of $Z$.
\end{remark}


Furthermore, we can assume $f_1$  to have a smooth part equal to zero as subtracting the smooth part of $f_1$ to all the functions gives an equivalent system.
Also, we can assume that $c_m$ is 1 because dividing all the functions of the $b^m$-integrable system by $c_m$ also gives us an equivalent system.


\textbf{As a summary, we can assume $f_1 = a_0 \log(x) + a_1 1/x + \ldots + a_{m-2}1/x^{m-2} + 1/x^{m-1}$ and $f_2, \ldots, f_n$ to be smooth, $a_0\in \mathbb{R}$ and $a_1, \ldots, a_{m-2} \in \mathcal{C}^\infty(x)$.}

Also we are going to state lemma 3.2 in \cite{GMP17}, because we are going to use it later in this section. The result states that if we have a toric action on a $b^m$-symplectic manifold (which we will prove in a neighbourhood of a Liouville torus), then we can assume the coefficients $a_2, \ldots, a_{m-2}$ to be constants. More precisely

\begin{lemma}\label{lemma:ctt_coefs}
There exists a neighborhood of the critical set $U = Z\times (-\varepsilon, \varepsilon)$ where the moment map $\mu: M \rightarrow \mathfrak{t}^*$ is given by
$$\mu = a_1 \log|x| + \sum_{i=2}^{m}a_i \frac{x^{-(i-1)}}{i-1} + \mu_0$$
with $a_i \in \mathfrak{t}^0_L$ and $\mu_0$ is the moment map for the $T_L$-action on the symplectic leaves of the foliation.
\end{lemma}


\section{Examples of $b^m$-integrable systems}

The following example illustrates why it is necessary to use the definition of $b^m$-function as considered above. There are natural examples of changes of coordinates in standard integrable systems on symplectic manifolds that yield  $b^m$-symplectic manifolds but do not give well-defined $b^m$-integrable systems.

\begin{example}
Consider a time change in the two body problem, to obtain a $b^2$-integrable system. In the classical approach to solve the $2$-body problem  the following two conserved quantities are obtained:
$$\begin{array}{rcl}
f_1 & = & \frac{\mu y^2}{2} + \frac{l^2}{2\mu r^2} - \frac{k}{r},\\
f_2 & = & l,
\end{array}
$$
with symplectic form $\omega = dr\wedge dy + dl \wedge d\alpha$, where $r$ is the distance between the two masses and $l$ is the angular momentum.
We also know that $l$ is constant along the trajectories. Because $l$ is a constant of the movement, we can do a symplectic reduction on its level sets. The form on the symplectic reduction becomes $dr \wedge dy$. To simplify the notation, we will use $x$ instead of $r$.
Then $\omega = dx\wedge dy$. With hamiltonian function given by $f = \frac{\mu}{2}y^2 + \frac{l}{2\mu}\frac{1}{x^2} - k \frac{1}{x}$. Hence, the equations are:
$$
\begin{array}{rcl}
 \dot{x} & = & \frac{\partial f}{\partial y},\\
 \dot{y} & = & -\frac{\partial f}{\partial x}.
\end{array}
$$
Doing a time change $\tau = x^3 t$ then $\frac{d x}{d\tau} = \frac{1}{x^3} \frac{dx}{dt}$. With this time coordinate, the equations become:
$$
\begin{array}{rcl}
 \dot{x} & = & \frac{1}{x^3}\frac{\partial f}{\partial y},\\
 \dot{y} & = & -\frac{1}{x^3}\frac{\partial f}{\partial x}.
\end{array}
$$
These equations can be viewed as the motion equations given by a $b^3$-symplectic form $\omega = \frac{1}{x^3} dx\wedge dy$.

Let us check that this is actually a $b^m$-integrable system.
\begin{itemize}%[a)]
\item All the functions Poisson commute is immediate because we only have one.
\item $df = \mu y dy + (\frac{k}{x^2} - \frac{l}{\mu}\frac{1}{x^3})dx$ is a $b^3$-form because the term with $dx$ does not depend on $y$.
\item All the functions are independent, this is true because $df$ does not vanish as a $b^3$-form.
\end{itemize}
\end{example}

\begin{example}
In the paper \cite{Marle} the author builds an action of $SL(2,\mathbb{R})$ over $(P,\omega_P)$ where
$P = \{\xi\in \mathbb{C} | i(\bar{\xi} - \xi)>0\}$ is the complex semi-plane, with moment map $J_P(\xi) = \frac{R}{\xi_{im}}((|\xi|^2 + 1),2\xi_r,\pm (|\xi|^2 + 1))$, where the $\pm$ sign depends on the choice of the hemisphere projected by the stereographic projection. From now on we will take the sign $+$. Also the symplectic form $\omega_P$ has the following expression:

$$\omega_P = \pm \frac{R}{\xi_{im}^2} d\xi_r\wedge d\xi_{im}$$

In order to simplify the notation we identify $P$ with the real half-plane $P = \{x,y\in\mathbb{R}^2|y>0\}$. With this identification, the moment map becomes $J_p(x,y)=\frac{R}{y}(x^2 + y^2 + 1, 2x, x^2 + y^2 + 1)$. Obviously, this moment map does not give an integrable system. The symplectic form writes as:

$$\omega_P = \frac{R}{y^2} dy\wedge dx.$$

This form can be viewed as a $b^2$-form if we extend $P$ including the line $\{y=0\}$ as its singular set.
Let us consider only one of the components of $J_P$ as $b^m$-function and let us see if it gives a $b^m$-integrable system. First we will try with $f_1 = \frac{R}{y}(x^2 + y^2 + 1) $ and then $f_2 = \frac{R}{y}(2x)$.
\begin{enumerate}%[i)]
\item $f_1 = \frac{R}{y}(x^2 + y^2 + 1)$
We have to check three things to see if this gives a $b^2$-integrable system.
\begin{enumerate}%[a)]
\item All the functions Poisson commute is immediate because we only have one.
\item All the functions are $b^m$-functions. This point does not hold because $d f_1= \frac{R}{y^2}(2xydx + (y^2-x^2 - 1)dy)$ and the first component makes no sense as a section of $\Lambda^1 (^{b^2}T^*M)$.
\item All the functions are independent. In this case, we need to check that $df_1$ does not vanish, but since it is not a $b^m$-form it makes no sense  to be a non-zero section of $\Lambda^1 (^{b^2}T^*M)$.
\end{enumerate}
\item $f_2 = \frac{R}{y}(2x)$
\begin{enumerate}

%[a)]
\item Same as before.
\item All the functions are $b^m$-functions. This point does not hold because $d f_2= \frac{2R}{y}dx - \frac{2Rx}{y^2}dy$ and the first component makes no sense as a section of $\Lambda^1 (^{b^2}T^*M)$.
\item Same as before.
\end{enumerate}
\end{enumerate}

\end{example}



\begin{example} Toric actions give natural examples of integrable systems where the component functions are given by the moment map. In the case of surfaces:
    $S^1$-actions on surfaces give natural examples of $b^m$-integrable systems. Only torus and spheres admit circle actions. 
    
    In the picture below two integrable systems on the $2$-sphere depending on the degree $m$. On the right the image of the moment map that defines the integrable system. The action is by rotations along the central axis.

    Namely consider the sphere $S^2$ as a  $b^m$-symplectic manifold having as critical set the equator:
        
        \[(S^2, Z = \{h = 0\}, \omega=\frac{d h}{h^m}\wedge d\theta),\] with $h\in\left[-1,1\right]$ and $\theta\in\left[0,2\pi\right)$.

For $m=1$: The computation $\iota_{\frac{\partial}{\partial \theta}}\omega=- \frac{d h}{h}=-d( \log |h|),$ tells us that the function $\mu(h,\theta)= \log |h|$ is the moment map and defines a $b$-integrable system.

For higher values of $m$: $\iota_{\frac{\partial}{\partial \theta}}\omega=- \frac{d h}{h^m}=-d(-\frac{1}{(m-1)h^{m-1}}),$ and the  moment map is $\mu(h,\theta)= -\frac{1}{(m-1)h^{m-1}}$ which defines a $b^m$-integrable system.
    
 \begin{figure}[h!]
            \begin{tikzpicture}[scale=0.8]
                \pgfmathsetmacro{\rlinex}{6}
                \pgfmathsetmacro{\baseptd}{8}
                \pgfmathsetmacro{\rlineybottom}{2.75}
                \pgfmathsetmacro{\rlineymid}{4.25}
                \pgfmathsetmacro{\rlineytop}{5.75}
                \pgfmathsetmacro{\vertstretch}{1.9}	
                \pgfmathsetmacro{\yshift}{4.25}	

                \def\R{1.6}
                \pgfmathsetmacro{\circlex}{1.4}
                \draw[dashed, very thick, color = magenta] (\circlex + \R, \rlineymid) arc (0:180:{\R} and {\R * .2});
                \draw[very thick, fill = pink, opacity = .5] (\circlex, \rlineymid) circle (\R);
                \draw[very thick] (\circlex, \rlineymid) circle (\R);
                \draw[very thick, color = red] (\circlex + \R, \rlineymid) arc (0:-180:{\R} and {\R * .2});
                
                \draw (\circlex + 2, \rlineymid) edge node[above] {$\mu, m = 1$} (\rlinex - 0.5, \rlineymid);
                \draw[->] (\circlex + 2, \rlineymid) -- (\rlinex - 0.5, \rlineymid);

                \draw (\rlinex, \rlineybottom) -- (\rlinex, \rlineytop) node[right] {};

                \draw[black, fill = black] (\rlinex, \rlineymid) circle(.3mm);

                \draw[dblue, fill = dblue] (\rlinex - 0.2, \rlineymid) circle (1pt);
                \draw[dblue, fill = dblue] (\rlinex + 0.2, \rlineymid) circle (1pt);

                \draw[line width = 2pt, join = round, dblue, <-] (\rlinex - 0.2, \rlineybottom - 0.15) -- (\rlinex - 0.2, \rlineymid );
                \draw[line width = 2pt, join = round, dblue, <-] (\rlinex + 0.2, \rlineybottom - 0.15) -- (\rlinex + 0.2, \rlineymid);
                \draw [very thick, ->] (\circlex, \rlineymid + 1.3*\R) ++(\R * -.5, 0) arc (180:320: {\R * .5} and {\R * .1});
                \draw [very thick] (\circlex, \rlineymid + 1.3*\R) ++(\R * -.5, 0) arc (180:0: {\R * .5} and {\R * .1});

                \coordinate (shift) at (8,0);
                \begin{scope}[shift=(shift)]
                \pgfmathsetmacro{\rlinex}{6}
                \pgfmathsetmacro{\baseptd}{8}
                \pgfmathsetmacro{\rlineybottom}{2.75}
                \pgfmathsetmacro{\rlineymid}{4.25}
                \pgfmathsetmacro{\rlineytop}{5.75}
                \pgfmathsetmacro{\vertstretch}{1.9}	
                \pgfmathsetmacro{\yshift}{4.25}	

                \def\R{1.6}
                \pgfmathsetmacro{\circlex}{1.4}
                \draw[dashed, very thick, color = magenta] (\circlex + \R, \rlineymid) arc (0:180:{\R} and {\R * .2});
                \draw[very thick, fill = pink, opacity = .5] (\circlex, \rlineymid) circle (\R);
                \draw[very thick] (\circlex, \rlineymid) circle (\R);
                \draw[very thick, color = red] (\circlex + \R, \rlineymid) arc (0:-180:{\R} and {\R * .2});
                
                \draw (\circlex + 2, \rlineymid) edge node[above] {$\mu, m = 2$} (\rlinex - 0.5, \rlineymid);
                \draw[->] (\circlex + 2, \rlineymid) -- (\rlinex - 0.5, \rlineymid);

                \draw (\rlinex, \rlineybottom) -- (\rlinex, \rlineytop) node[right] {};

                \draw[black, fill = black] (\rlinex, \rlineymid) circle(.3mm);

                \draw[dblue, fill = dblue] (\rlinex - 0.2, \rlineymid) circle (1pt);
                \draw[dblue, fill = dblue] (\rlinex + 0.2, \rlineymid) circle (1pt);

                \draw[line width = 2pt, join = round, dblue, <-] (\rlinex - 0.2, \rlineybottom - 0.15) -- (\rlinex - 0.2, \rlineymid );
                \draw[line width = 2pt, join = round, dblue, ->] (\rlinex - 0.2, \rlineymid) -- (\rlinex - 0.2, \rlineytop + 0.15);
                
                \draw[line width = 2pt, join = round, dblue, <-] (\rlinex + 0.2, \rlineybottom - 0.15) -- (\rlinex + 0.2, \rlineymid );
                \draw[line width = 2pt, join = round, dblue, ->] (\rlinex + 0.2, \rlineymid) -- (\rlinex + 0.2, \rlineytop + 0.15);
                
                \draw [very thick, ->] (\circlex, \rlineymid + 1.3*\R) ++(\R * -.5, 0) arc (180:320: {\R * .5} and {\R * .1});
                \draw [very thick] (\circlex, \rlineymid + 1.3*\R) ++(\R * -.5, 0) arc (180:0: {\R * .5} and {\R * .1});
                \end{scope}
            \end{tikzpicture}
            \caption{Integrable systems associated to the moment map of an $S^1$-action by rotations  on a $b^m$-symplectic $2$-sphere $S^2$.}
            \label{fig:S2}
        \end{figure}
    \end{example}

\begin{example}
    Consider now as $b^2$-symplectic manifold the $2$-torus
\[
(\mathbb{T}^2, Z = \{\theta_1 \in \{0, \pi\}\}, \omega=  \frac{d\theta_1}{\sin^2\theta_1}\wedge d\theta_2)
\]
with standard coordinates: $\theta_1, \theta_2 \in \left[0, 2\pi \right)$.  Observe that the critical  hypersurface $Z$ in this example is not connected. It is the union of two disjoint circles. Consider  the circle action of rotation on the $\theta_2$-coordinate with fundamental vector field $\frac{\partial}{\partial\theta_2}$. As the following computation holds,
$$\iota_{\frac{\partial}{\partial\theta_2}}\omega = - \frac{d \theta_1}{\sin^2 \theta_1} = d\left(\frac{\cos\theta_1}{\sin\theta_1}\right).$$
The fundamental vector field of the $S^1$-action defines $^{b^2}C^{\infty}$-integrable system given by the function $-\frac{\cos\theta_1}{\sin\theta_1}$.


\end{example}


\begin{figure}[ht]
\begin{center}
    

\begin{tikzpicture}[scale=0.8]

\pgfmathsetmacro{\rlinex}{6}
\pgfmathsetmacro{\rlineybottom}{2.75}
\pgfmathsetmacro{\rlineymid}{4.25}
\pgfmathsetmacro{\rlineytop}{5.75}

\def\R{1.6}
\pgfmathsetmacro{\donutx}{1.5}
\pgfmathsetmacro{\sizer}{1.8}	


%top of rotation arrow
\draw [very thick] (\donutx - 0.3*\sizer, \rlineymid) arc (180:0: {\sizer * .4} and {\sizer * .1});

%draw the donut on the left
\draw [red, very thick, dashed] (\donutx - 0.5, \rlineymid - .55*\sizer) arc (90:270:{.16*\sizer} and {0.32*\sizer});
\draw [red, very thick, dashed] (\donutx - 0.5, \rlineymid + .55*\sizer) arc (270:90:{.16*\sizer} and {0.32*\sizer});

\DrawFilledDonutops{\donutx - 0.5}{\rlineymid}{.6*\sizer}{1.2*\sizer}{-90}{yellow!30}{very thick}{white}
\DrawDonut{\donutx - 0.5}{\rlineymid}{.6*\sizer}{1.2*\sizer}{-90}{black}{very thick}

\draw [red, very thick] (\donutx - 0.5, \rlineymid - .55*\sizer) arc (90:-90:{.16*\sizer} and {0.32*\sizer});
\draw [red, very thick] (\donutx - 0.5, \rlineymid + .55*\sizer) arc (-90:90:{.16*\sizer} and {0.32*\sizer});

\draw [very thick] (\donutx - .3*\sizer, \rlineymid) arc (180:132: {\sizer * .4} and {\sizer * .1});
\draw [very thick, ->] (\donutx - .3*\sizer, \rlineymid) arc (180:325: {\sizer * .4} and {\sizer * .1});



%draw the arrow
\draw     (\donutx + 1, \rlineymid) edge node[above] {$\mu$} (\rlinex - 0.5, \rlineymid);
\draw[->] (\donutx + 1, \rlineymid) -- (\rlinex - 0.5, \rlineymid);

\draw (\rlinex, \rlineybottom) -- (\rlinex, \rlineytop) node[right] {};

\draw[black, fill = black] (\rlinex, \rlineymid) circle(.3mm);

\draw[line width = 2pt, join = round, dblue, <->] (\rlinex - 0.2, \rlineybottom - 0.15) -- (\rlinex - 0.2, \rlineytop + 0.15);
\draw[line width = 2pt, join = round, dblue, <->] (\rlinex + 0.2, \rlineybottom - 0.15) -- (\rlinex + 0.2, \rlineytop + 0.15);
 
\end{tikzpicture}

\end{center}
% \end{center}
 \caption{Integrable system given by an $S^1$-action on a $b^2$-torus $\mathbb{T}^2$ and its associated moment map.}
 \label{fig:torus}
 \end{figure}
\begin{example} \label{ex:bmtorus}
The former example can be made general to produce examples of $b^m$-integrable systems on a $b^m$-symplectic manifold for any integer $m$
\[
(\mathbb{T}^2, Z = \{\theta_1 \in \{0, \pi\}\}, \omega=  \frac{d\theta_1}{\sin^m\theta_1}\wedge d\theta_2).
\]
Then 
$$\iota_{\frac{\partial}{\partial\theta_2}}\omega = - \frac{d \theta_1}{\sin^m \theta_1} = d\left(\frac{|\cos\theta_1|}{\cos\theta_1} \frac{_2F_1 \left ( \frac{1}{2}, \frac{1 - m}{2}; \frac{3 - m}{2}; \sin^2(\theta_1) \right )}{(1-m) \sin^{m - 1} \theta_1}\right),$$
with $_2F_1$  the hypergeometric function. 

Thus, the associated $S^1$-action has as  $^{b^m}C^{\infty}$-Hamiltonian the function 
$$- \frac{|\cos\theta_1|}{\cos\theta_1} \frac{_2F_1 \left ( \frac{1}{2}, \frac{1 - m}{2}; \frac{3 - m}{2}; \sin^2(\theta_1) \right )}{(1-m) \sin^{m - 1} \theta_1}$$ \noindent which defines a $b^m$-integrable system.
\end{example}
Now we give a couple  of examples of $b^m$-integrable systems.


\begin{example}
This example uses the product of $b^m$-integrable systems on a $b^m$-symplectic manifold with an integrable system on a symplectic manifold. Given $(M_1^{2n_1}, Z,x,\omega_1)$ a $b^m$-symplectic manifold with $f_1,\ldots,f_{n_1}$ a $b^m$-integrable system and $(M_2^{2n_2},\omega_2)$ a symplectic manifold with $g_1,\ldots,g_{n_2}$ an integrable system. Then $(M_1\times M_2, Z\times M_2, x, \omega_1 + \omega_2)$ is a $b^m$-symplectic manifold and $(f_1,\ldots,f_{n_1},g_1,\ldots,g_{n_2})$ is a $b^m$-integrable system on the higher dimensional manifold.

In particular by combining the former examples of $b^m$-integrable systems on surfaces and arbitrary integrable systems on symplectic manifolds we obtain examples of $b^m$-integrable systems in any dimension.
\end{example}


\begin{example}\textbf{(From integrable systems on cosymplectic manifolds to $b^m$-integrable systems:)}

Using the extension theorem (Theorem 50) of \cite{GMP14} we can extend any integrable system $(f_2,\dots, f_n)$ to an integrable system in a neighbourhood of a cosymplectic manifold $(Z, \alpha, \omega) $ by just adding a $b^m$-function $f_1$ to the integrable system so that the new integrable system is $(f_1, f_2,\dots, f_n)$  and considering the associated $b^m$-symplectic form:

\begin{equation}\label{eq:normalform}\tilde{\omega}=p^*\alpha\wedge\frac{dt}{t^m}+p^*\omega. \end{equation}

(t is the defining function of $Z$).

\end{example}

\section{Looking for a toric action}

In this section we pursue the proof of action-angle coordinates for $b^m$-integrable systems by recovering a torus group action. This action is associated to the Hamiltonian vector fields associated to $X_{f_i}$.

This is the same strategy used for $b$-integrable systems in \cite{KMS16}- One of the main difficulties is  to prove that the coefficients $a_1,\ldots, a_n$ can be considered  as constant functions. This makes it more difficult to prove the existence of a $\mathbb{T}^n$-action in the general $b^m$-case than in the $b$-case, but once we have it we can use the results in \cite{GMW17} to assume that the coefficients $a_1,\ldots, a_n$ are constant functions.



%\begin{remark}
%$df_1,\ldots,df_n$ are independent if and only if $X_{f_1},\ldots, X_{f_n}$ are independent. This holds because the map $^{b^m}TM\rightarrow ^{b^m} T^{*}M:u\mapsto \omega(u,\cdot)$ is bijective.
%\end{remark}

%\begin{remark}
%Because $X_{f_1},\ldots,X_{f_n}$ have to be linearly independent, one of the functions has a singularity of maximal degree. Looking at the expression \ref{eq:bmhamiltonianvf} it is easy to see that without the term of maximal degree all the terms in the component $\partial/\partial y_i$ are $0$ at $Z$.
%\end{remark}


In this section we provide some preliminary material that will be needed later:
\begin{proposition}\label{prop:mod_period}
Let $(M,Z,x,\omega)$ be a $b^m$-symplectic manifold such that $Z$ is connected with modular period $k$.
Let $\pi:Z\rightarrow S^1 \simeq \mathbb{R}/k\mathbb{Z}$ be the projection to the base of the corresponding mapping torus.
Let $\gamma: S^1 = \mathbb{R}/k\mathbb{Z} \rightarrow Z$ be any loop such that $\pi\circ\gamma$ is positively oriented and has constant velocity 1. Then the following are equal:
\begin{enumerate}%[i)]
\item The modular period of $Z$,

\item $\int_\gamma \iota_{\mathbb{L}} \omega$,

\item The value $a_{m-1}$ for any $^{b^m}\mathcal{C}^\infty(M)$-function
$$f = a_0 \log(x) + \sum_{j= 1}^{m-1} a_j\frac{1}{x^j} + h$$
such that the hamiltonian vector field $X_f$ has 1-periodic orbits homotopic in $Z$ to some $\gamma$.
\end{enumerate}
\end{proposition}

\begin{proof}

Let us first prove that (1)=(2) and then that (2)=(3).
\begin{itemize}
\item[(1)=(2)] Let us denote by $\mathcal{V}_{mod}$ the modular vector field. Recall from \cite{GMW17} that $\iota_{\mathbb{L}}(\mathcal{V}_{mod})$ is the constant function 1. Let $s:[0,k]\rightarrow Z$ be the trajectory of the modular vector field. Because the modular period is $k$, $s(0)$ and $s(k)$ are in the same leaf $\mathcal{L}$. Let $\hat s :[0,k+1]\rightarrow Z$ a smooth extension of $s$ such that $s|_{[k,k+1]}$ is a path in $\mathcal{L}$ joining $\hat s (k) = s (k)$ to $\hat s (k+1) = s (0)$. This way $\hat s$ becomes a loop. Then,

$$k=\int_0^k 1 dt = \int_S \iota_{\mathbb{L}} \omega = \int_{\hat s } \iota_{\mathbb{L}}\omega=\int_\gamma\iota_{\mathbb{L}}\omega$$

\item[(2)=(3)] Let $r:[0,1] \mapsto Z$ be the trajectory of $X_f$ the hamiltonian vector field of $f$. Recall that $X_f$ satisfies
$$\iota_{X_f}\omega = \sum_{j=1}^m c_j \frac{dx}{x^i} + dh.$$
Let $x^m\frac{\partial}{\partial x}$ be a generator of the linear normal bundle $\mathbb{L}$.
We know that $X_f$ is 1-periodic and its trajectory is homotopic to $\gamma$. Hence,
$$
\begin{array}{rcl}
 k = \int_r \iota_{\mathbb{L}}\omega & = & \displaystyle \int_0^1 \iota_{x^m\frac{\partial}{\partial x}} \omega(X_f|_{r(t)})dt\\
 \\
 & = & \displaystyle \int_0^1 -(\sum_{j=1}^{m}c_i\frac{dx}{x^i} + dh)\cdot(x^m\frac{\partial}{\partial x})|_{r(t)}dt\\
 \\
 & = & -c_m = -a_{m-1}\\

\end{array}
$$
\end{itemize}

\end{proof}

We will also need a Darboux-Carathéodory theorem for $b^m$-symplectic manifolds:
\begin{theorem}[Darboux-Carath\'{e}odory ($b^m$-version)]\label{bmdarbouxcaratheodory}
Let
$$(M^{2n},x,Z, \omega)$$
 be a $b^m$-symplectic manifold and $m$ be a point on $Z$. Let $f_1,\ldots,f_n$ be a $b^m$-integrable system. Then there exist \textcolor{black}{$b^m$}-functions $(q_1,\ldots,q_n)$ around $m$ such that
$$\omega = \sum_{i=1}^n df_i\wedge dq_i$$
and the vector fields $\{X_{f_i},X_{q_j}\}_{i,j}$ commute.
If $f_1$ is not smooth (recall that $f_1 = a_0\log(x) + \sum_{j=1}^{m-1}a_j\frac{1}{x^i}$ with $a_n \neq 0$ on $Z$ and $a_0 \in \mathbb{R}$) the $q_i$ can be chosen to be smooth functions, and $(x,f_2,\ldots,f_n,q_1,\ldots,q_n)$ is a system of local coordinates.
\end{theorem}

\begin{proof}
The first part of this proof is exactly as in \cite{KMS16}.
Assume now $\displaystyle f_1 = a_0\log(x) + \sum_{j=1}^{m-1}a_j\frac{1}{x^i}$.
We modify the induction requiring also that $\mu_i$ (in addition to be in $K_i$) is also in $T^* M \subseteq ^bT^* M$.
We can also ask this extra condition while asking $\mu_i(X_{f_i})= 1$, we only have to check that $X_{f_i}$ does not vanish in $TM$. This is clear because $X_{f_i}$ does not vanish at $^b TM$ and
$$0 = \{f_n,f_i\} = \left(\sum_{i=1}^m \tilde{a}_i\frac{dx}{x^i}\right)(X_{f_i}) = \left(\frac{dx}{x^m} \sum_{i=1}^{m}a_i x^i\right)(X_{f_i}).$$

All the terms in the last expression vanish except for the one of degree $m$.

Then $dx/x^m$ is in the kernel of $X_{f_i}$, hence $X_{f_i}$ does not vanish on $TM$ and the $q_i$ can be chosen to be smooth.

$\{X_x,X_{f_2},\ldots,X_{f_{n}},X_{q_1},\ldots X_{q_n}\}$ commute because $\{X_{f_i},X_{q_i}\}_{i,j}$ commute. Then
$$dx\wedge df_2\ldots\wedge df_n\wedge dq_1 \wedge \ldots \wedge d q_n$$
is a non-zero section of $\bigwedge^n(^b TM)$. And hence
$$(x,f_2,\ldots, f_{n-1},q_1,\ldots,q_n)$$
 are local coordinates.

\end{proof}

Before proceeding with the proof of the action-angle coordinates, we need to prove that in a neighbourhood of a Liouville torus the fibration is semilocally trivial:

\begin{lemma}[Topological Lemma]\label{lemma:topological}
Let $m \in Z$ be a regular point of a $b^m$-integrable system $(M,x,Z,\omega,F)$. Assume that the integral manifold $\mathcal{F}_m$ through $m$ is compact. Then there exists a neighborhood $U$ of $\mathcal{F}_m$ and a diffeomorphism
$$\phi:U \simeq \mathbb{T}^n\times B^n$$
which takes the foliation $\mathcal{F}$ to the trivial foliation $\{\mathbb{T}^n\times\{b\}\}_{b\in B^n}$.
\end{lemma}

\begin{proof}
We follow the steps of \cite{LMV11}. In this case, the only extra step that must be checked is that the foliation given by the $b^m$-hamiltonian vector fields of $F = (f_1,f_2, \ldots, f_n)$ is the same as the one given by the level sets of $\tilde F := (x, f_2,\ldots, f_n)$. In our case $f_1 = a_0\log(x) + \sum_{u=1}^{m-1} a_i\frac{1}{x^i}$, where $a_0 \in \mathbb{R}$, $a_i \in \mathcal{C}^\infty(x)$, $a_{m-1} = 1$. Hence the foliations are the same.
Then as in \cite{LMV11}, we take an arbitrary Riemannian metric on $M$ and this defines a canonical projection $\psi:U \rightarrow \mathcal{F}_m$. Let us define $\phi := \psi\times \tilde F$. We obtain the commutative diagram (Figure \ref{fig:diagram_topological}).

\begin{figure}[H]
\centering
\begin{tikzcd}
U \arrow[rr, dashed, "\phi"] \arrow[rrd, "\tilde F"]& & \mathbb{T}^n\times B^n \arrow[d,"p"]\\
 & & B^n
\end{tikzcd}
\caption{Commutative diagram of the construction of the isomorphism of $b^m$-integrable systems.}
\label{fig:diagram_topological}
\end{figure}

which provides the necessary equivalence of $b^m$-integrable systems.
\end{proof}

%\begin{lemma}[$b^m$-Poincaré]\label{lemma:poincare}
%Let $\psi$ be a $b^m$-form such that $i^*\psi = 0$ where $i$ is the immersion of the
%\end{lemma}
%\begin{proof}
%The proof of this lemma works exactly as in the standard Poincaré lemma, because the homotopy formula works the same way for $b^m$-forms.
%\end{proof}

\section{Action-angle coordinates on $b^m$-symplectic \protect\\ manifolds}


In a neighbourhood of one of our Liouville tori all we can assume about the form of our $b^m$-symplectic structure is  that is given by the Laurent series defined in \cite{Scott16}.

That is to say, we can assume that in a tubular neighborhood $U$ of $Z$
$$\omega = \sum_{j=1}^{m-1}\frac{dx}{x^i}\wedge\pi^*(\alpha_i) + \beta,$$
where $\pi:U\rightarrow Z$ is the projection of the tubular neighborhood onto $Z$, $\alpha_i$ are closed smooth de Rham forms on $Z$ and $\beta$ a de Rham form on $M$ of degree $2$.

In \cite{BKM,anastasiaeva} normal forms are given for group actions in a neighbourhood of the orbit. Below we provide a normal for the integrable system in a neighbourhood of an orbit of the torus action associated to the integrable system. This theorem is finer than the $b^m$-symplectic slice theorem provided in \cite{anastasiaeva} as it also gives information about the first integrals. 

\begin{figure}
\centering
\begin{tikzpicture}
%%%%%%%%%regprogramming

\pgfmathsetmacro{\sizer}{1}
\pgfmathsetmacro{\basept}{5}	

\pgfmathsetmacro{\xone}{0}
\pgfmathsetmacro{\xtwo}{1.5}
\pgfmathsetmacro{\xthree}{3}
\pgfmathsetmacro{\xfour}{4.5}
\pgfmathsetmacro{\xfive}{6}
\pgfmathsetmacro{\xsix}{7.5}
\pgfmathsetmacro{\xseven}{9}

\pgfmathsetmacro{\ymid}{4.25}
\pgfmathsetmacro{\ytop}{5.65}
\pgfmathsetmacro{\ybottom}{2.85}

\DrawFilledDonutops{\xone}{\ymid}{.55*\sizer}{1.2*\sizer}{-90}{vlblue}{very thick}{white}
\DrawFilledDonutops{\xtwo}{\ymid}{.55*\sizer}{1.2*\sizer}{-90}{vlblue}{very thick}{white}
\DrawFilledDonutops{\xthree}{\ymid}{.55*\sizer}{1.2*\sizer}{-90}{vlblue}{very thick}{white}
\DrawFilledDonutops{\xfour}{\ymid}{.55*\sizer}{1.2*\sizer}{-90}{magenta}{very thick}{white}
\DrawFilledDonutops{\xfive}{\ymid}{.55*\sizer}{1.2*\sizer}{-90}{vlblue}{very thick}{white}
\DrawFilledDonutops{\xsix}{\ymid}{.55*\sizer}{1.2*\sizer}{-90}{vlblue}{very thick}{white}
\DrawFilledDonutops{\xseven}{\ymid}{.55*\sizer}{1.2*\sizer}{-90}{vlblue}{very thick}{white}

\DrawDonut{\xone}{\ymid}{.55*\sizer}{1.2*\sizer}{-90}{dblue}{very thick}
\DrawDonut{\xtwo}{\ymid}{.55*\sizer}{1.2*\sizer}{-90}{dblue}{very thick}
\DrawDonut{\xthree}{\ymid}{.55*\sizer}{1.2*\sizer}{-90}{dblue}{very thick}
\DrawDonut{\xfour}{\ymid}{.55*\sizer}{1.2*\sizer}{-90}{dred}{very thick}
\DrawDonut{\xfive}{\ymid}{.55*\sizer}{1.2*\sizer}{-90}{dblue}{very thick}
\DrawDonut{\xsix}{\ymid}{.55*\sizer}{1.2*\sizer}{-90}{dblue}{very thick}
\DrawDonut{\xseven}{\ymid}{.55*\sizer}{1.2*\sizer}{-90}{dblue}{very thick}

\draw[very thick,magenta,vdblue, ->](\xone, \ybottom - 0.3) -- +(0, -1);
\draw[very thick, vdblue,  ->] (\xtwo, \ybottom - 0.3) -- +(0, -1);
\draw[very thick, vdblue, ->] (\xthree, \ybottom - 0.3) -- +(0, -1);
\draw[very thick, red, ->] (\xfour, \ybottom - 0.3) -- +(0, -1);
\draw[very thick,vdblue, ->] (\xfive, \ybottom - 0.3) -- +(0, -1);
\draw[very thick,vdblue, ->] (\xsix, \ybottom - 0.3) -- +(0, -1);
\draw[very thick,vdblue, ->] (\xseven, \ybottom - 0.3) -- +(0, -1);
\draw[ultra thick, purple](\xone - 1, \ybottom - 1.5) -- (\xseven + 1, \ybottom - 1.5);
\end{tikzpicture}
\caption{Fibration by Liouville tori: The middle fiber of the point $p \in Z$ in magenta, the neighbouring Liouville tori in blue.}
\label{fig:tori}
\end{figure}

One of the non-trivial steps of the proof is to associate a toric action to the integrable system. The connection to normal forms of group actions will become even more evident when we discuss the associated cotangent models.



\begin{theoremA}[Action-angle coordinates for $b^m$-symplectic manifolds]
Let $(M,x,\omega, F)$ be a $b^m$-integrable system, where $F = (f_1 = a_0 \log(x) + \sum_{j=1}^{m-1} a_j\frac{1}{x^j}, \ldots,f_n)$ with $a_j$  for $j>1$ functions in $x$. Let $m\in Z$ be a regular point and let us assume that the integral manifold of the distribution generated by the $X_{f_i}$ through $m$ is compact. Let $\mathcal{F}_m$ be the Liouville torus through $m$.
Then, there exists a neighborhood $U$ of $\mathcal{F}_m$ and coordinates $(\theta_1,\ldots,\theta_n,\sigma_1,\ldots,\sigma_n):\mathcal{U}\rightarrow\mathbb{T}^n\times B^n$ such that:

\begin{enumerate}
\item We can find an equivalent integrable system $F = (f_1 = a_0'\log(x) + \sum_{j=1}^{m-1} a_j'\frac{1}{x^j}, \ldots, f_n)$ such that the coefficients $a_0',\ldots, a_{m-1}' $ of $f_1$ are constants $\in \mathbb{R}$,
\item $$\omega|_\mathcal{U} = \left(\sum_{j=1}^m c_j'\frac{c}{\sigma_1^j}d\sigma_1\wedge d\theta_1\right) + \sum_{i=2}^{n} d \sigma_i\wedge d\theta_i$$ where $c$ is the modular period and $c_j' = -(j-1)a_{j-1}'$, also
\item the coordinates $\sigma_1,\ldots,\sigma_n$ depend only on $f_1,\ldots f_n$.
\end{enumerate}

%Also, \textcolor{black}{$\langle c\cdot c_j', X \rangle = \alpha_j(X)$ for $X \in \mathfrak{t}$}.
\end{theoremA}

\begin{proof}
The idea of this proof is to construct an equivalent $b^m$-integrable system whose fundamental vector fields define a $\mathbb{T}^n$-action on a neighborhood of $\mathbb{T}^n\times\{0\}$.
It is clear that all the vector fields $X_{f_1},\ldots,X_{f_n}$ define a torus action on each Liouville tori $\mathbb{T}^n\times\{b\}$ where $b\in B^n$, but this does not guarantee that their flow defines a toric action on all $\mathbb{T}^n\times B^n$.
The proof is structured in three steps. The first one is the uniformization of the periods, i.e. we define an $\mathbb{R}^n$-action on a neighborhood of $\mathbb{T}^n\times\{0\}$ such that the lattice defined by its kernel at every point is constant. This allows to induce an actual action of a torus (as the periods are constant) of rank n: A $\mathbb T^n$ action by taking quotients. The second step consists in checking that this action is actually $b^m$-Hamiltonian. And in the final step we apply theorem \ref{bmdarbouxcaratheodory} to obtain the expression of $\omega$.
\begin{enumerate}

\item Uniformization of periods.

Let $\Phi_{X_F}^s$ be defined as the joint flow by the Hamiltonian vector fields of the action:
\begin{equation}\label{eq:action1}
\begin{array}{rcl}
 \Phi: \mathbb{R}^n\times(\mathbb{T}^n\times B^n)& \rightarrow & (\mathbb{T}^n\times B^n)\\
 ((s_1,\ldots,s_n),(x,b)) & \mapsto & \Phi_{X_{f_1}}^{s_1}\circ\cdots\circ\Phi_{X_{f_n}}^{s_n}((x,b))\\

\end{array}
\end{equation}

this defines an $\mathbb{R}^n$-action on $\mathbb{T}^n\times B^n$.
For each $b\in B^n$ at a single orbit $\mathbb{T}^n\times\{b\}$ the kernel of this action is a discrete subgroup of $\mathbb{R}^n$. We will denote the lattice given by this kernel $\Lambda_b$. Because the orbit is compact, the rank of $\Lambda_b$ is maximal i.e. $n$. This lattice is known as the period lattice of $\mathbb{T}^n\times\{b\}$ as we know by standard arguments in group theory that the lattice has to be of maximal rank so as to have a torus as a quotient.
In general we can not assume that $\Lambda_b$ does not depend on $b$. The process of uniformization of the periods modifies the action \ref{eq:action1} in such a way that $\Lambda_b = \mathbb{Z}^n$ for all $b$.
Let us consider the following Hamiltonian vector field $\sum_{i=1}^n k_iX_{f_i}$. The $b^m$-function that generates this Hamiltonian vector field is:
$$k_1\left(a_0\log(x) + \sum_{j=1}^{m-1}a_j\frac{1}{x^j}\right) + \sum_{i=2}^n k_i f_i$$
where recall that $a_{m-1}$ is constant equal 1. Observe that the coefficient multiplying $1/x^{m-1}$ is $k_1$. By proposition \ref{prop:mod_period} $k_1 = c$ the modular period. In this case $c = [\alpha_m]$.

Hence, for $b\in B^{n-1}\times\{0\}$ the lattice $\Lambda_b$ is contained in $\mathbb{R}^{n-1}\times c \mathbb{Z} \subseteq \mathbb{R}^n$.
Pick $(\lambda_1,\ldots,\lambda_n): B^n\rightarrow \mathbb{R}^n$ such that:
\begin{itemize}%[a)]
\item $(\lambda_1(b),\ldots,\lambda_n(b))$ is a basis of $\Lambda_b$ for all $b\in B^n$,
\item \textcolor{black}{$\lambda_i^n$ vanishes along $B^{n-1}\times\{0\}$ at order $m$ for $i<n$ and $\lambda_i$ is equal to $c$ along $B^{n-1}\times\{0\}$.}
\end{itemize}
In the previous points, $\lambda_i^j$ denotes the $j$-th component of $\lambda_i$. The first condition can be satisfied by using the implicit function theorem. That is because $\Phi(\lambda,m) = m$ is regular with respect to the $s$ coordinates. The second condition is automatically true because $\Lambda_b \subseteq \mathbb{R}^{n-1}\times c\mathbb{Z}$. We define the uniformed flow as:
\begin{equation}\label{eq:action2}
\begin{array}{rcl}
\tilde \Phi: \mathbb{R}^n\times(\mathbb{T}^n\times B^n)& \rightarrow & (\mathbb{T}^n\times B^n)\\
((s_1,\ldots,s_n),(x,b))& \mapsto & \Phi(\sum_{i=1}^n s_i \lambda_i,(x,b))\\

\end{array}
\end{equation}
\item The $\mathbb{T}^n$-action is $b^m$-Hamiltonian.
The objective of this step is to find $b^m$-functions $\sigma_1,\ldots,\sigma_n$ such that $X_{\sigma_i}$ are the fundamental vector fields of the $\mathbb{T}^n$-action $Y_i = \sum_{j=1}^n \lambda_i^j X_{f_j}$.

By using the Cartan formula for a $b^m$-symplectic form, we obtain:

$$
\begin{array}{rcl}
\mathcal{L}_{Y_i}\mathcal{L}_{Y_i}\omega & = & \mathcal{L}_{Y_i}(d(\iota_{Y_i}\omega) + \iota_{Y_i}d\omega)\\
 & = &  \mathcal{L}_{Y_i}(d(-\sum_{j=1}^{n} \lambda_i^j df_i))\\
 & = &  -\mathcal{L}_{Y_i}(\sum_{j=1}^n d\lambda_i^j\wedge df_j) = 0\\
\end{array}
$$

Note that $\lambda_i^j$ are constant on the level sets of $F$ as $\Phi(\lambda, m) = m$  and the level sets of $F$ are invariant by $\Phi$.

Recall that if $Y$ is a complete periodic vector field and $P$ is a bivector such that $\mathcal{L}_Y\mathcal{L}_Y P = 0$, then $\mathcal{L}_Y P = 0$.
So, the vector fields $Y_i$ are Poisson vector fields.
To show that each $\iota_{Y_i}\omega$ has a $^{b^m}\mathcal{C}^\infty$ primitive we will see that $[\iota_{Y_i}\omega] = 0$ in the $b^m$-cohomology.

One one hand, if $i >1$, $\iota_{Y_i}\omega$ vanishes at $Z$. This holds because $Y_i$ has not any component $\partial/\partial Y$.

Recall Proposition 6 from \cite{GMP14}:


\begin{proposition} If $\omega \in ^b \Omega(M)$ with $ \omega|_Z=0$, then $\omega\in \Omega(M)$.
\end{proposition}
In  a similar way for $b^m$-forms we have,


\begin{proposition} If $\omega \in ^{b^m} \Omega(M)$ with $\omega|_Z$ vanishing up to order $m$, then $\omega \in \Omega(M)$.
\end{proposition}


Thus as $\iota_{Y_i}\omega$ vanishes at $Z$, the $b^m$-forms  $\iota_{Y_i}\omega$ are indeed  smooth.
Thus  we can now apply the standard Poincaré lemma and as these forms are closed they are locally exact. This proves that all the vector fields $Y_i$ with $i>1$ are indeed Hamiltonian.

On the other hand, the fact that $\iota_{Y_1} \omega = c df_1$ is obvious.



Then, because we have a toric action that is Hamiltonian, we can use lemma 3.2 in \cite{GMP17}, and  we get an equivalent system such that $a_i$ are all constant and moreover $\langle a_i', X\rangle = \alpha_i(X^\omega)$. Note that by dividing by $a_{m-1}'$, we can still assume $a_{m-1}'=1$ to be consistent with our notation, but  we then have to multiply $f_1\cdot c$ in the next step.


\item Apply Darboux-Carathéodory theorem.

The construction above gives us some candidates $\sigma_1 = c f_1,\sigma_2,\ldots,\sigma_n$ for the action coordinates.

We now apply the Darboux-Carathéodory theorem and express the form in terms of $x$:
$$\omega = \left(\sum_{j= 1}^{m} c \frac{c_j}{x^j} dx\wedge d q_1\right)+ \sum_{i=2}^{n}d\sigma_i\wedge dq_i.$$

Since the vector fields $X_{\sigma_i} = \frac{\partial}{\partial q_i}$ are fundamental fields of the $\mathbb{T}^n$-action the flow \ref{eq:action2} gives a linear action on the $q_i$ coordinates.

Observe that the coordinate system is only defined in $\mathcal{U}$. It may not be valid at points outside $\mathcal{U}$ that may be in the orbit of points in $\mathcal{U}$. Let us see that the charts can be extended to these points.

Define $\mathcal{U}'$ the union of all tori that intersect $\mathcal{U}$.
We will see that the coordinates are valid at $\mathcal{U}'$.

Let $\{p_i,\theta_j\}$ be the extension of $\{\sigma_i,q_j\}$. It is clear that $\{p_i,\theta_j\} = \delta_{ij}$ by its construction in the Darboux-Carathéodory theorem.

To see that $\{\theta_i,\theta_j\} = 0$ we take the flows by $X_{p_k}$ and extend the expression to the whole $\mathcal{U}'$:

$$X_{p_k}(\{\theta_i, \theta_j\}) = \{\{\theta_i, \theta_j\},p_k\} = \{\theta_i,\delta_{ij}\} - \{\theta_j, \delta_{jk}\} = 0.$$

The fact that $\omega$ is preserved is obvious because $X_{p_k}$ are hamiltonian vector fields and thus they preserve the $b^m$-symplectic forms. Moreover, $t,\theta_1,p_2,\theta_2,\ldots,p_n,\theta_n$ are independent on $\mathcal{U}'$ and hence are a coordinate system in a neighbourhood of the torus.

\end{enumerate}
\end{proof}

\begin{remark}
In the proof  we have seen that there exists an equivalent integrable system where the coefficients of the singular function are indeed constant. From now on, when considering a $b^m$-integrable system we are going to make this assumption.
\end{remark}

\begin{remark}
    By means of the desingularization transformation we may obtain an action-angle coordinate theorem for folded manifolds as we do in Part 3 for the KAM theorem for folded symplectic manifolds. This folded action-angle theorem is a particular case of the one obtained in \cite{EvaRobert}.
\end{remark}

\chapter[Action-angle coordinates and cotangent lifts]{Reformulating the action-angle coordinate via cotangent lifts}

The action-angle theorem for symplectic manifolds (also known as action-angle coordinate theorem) can be reformulated in terms of a cotangent lift.


Recall that given a Lie group action on any manifold its cotangent lifted action is automatically Hamiltonian. By considering the action of a torus on itself by translations this action can be lifted to its cotangent bundle and give a semilocal normal form theorem as the Arnold-Liouville-Mineur theorem for symplectic manifolds. If we now replace this cotangent lift to the cotangent bundle to a lift to the $b^m$-cotangent bundle we obtain the semilocal normal form of the main theorem of this chapter.


Let start recalling the symplectic and $b$-symplectic case following \cite{KM17}.


\section{Cotangent lifts and Arnold-Liouville-Mineur in Symplectic Geometry}

Let $G$ be a Lie group and let $M$ be any smooth manifold. Given a group action $\rho:G\times M\longrightarrow M$, we define its cotangent lift as the action on $T^\ast M$ given by $\hat{\rho_g}:=\rho^\ast_{g^-1}$ where $g\in G$. We then have a commuting diagram





\begin{figure}[h]
\centering
\begin{tikzcd}
T^\ast M  \arrow[rr, "\hat{\rho_g}"] \arrow[d,"\hat{\pi}"] & & T^\ast M \arrow[d,"\pi"]\\
M  \arrow[rr, "\rho_g"]  & & M
\end{tikzcd}
\caption{Commutiative diagram of the construction of the isomorphism of $b^m$-integrable systems.}
\label{fig:diagram_cotangent_lift}
\end{figure}

where $\pi$ is the canonical projection from $T^\ast M$ to $M$.

 The cotangent bundle $T^*M$ is a symplectic manifold endowed
with the exact symplectic form given by the differential of the Liouville one-form $\omega=-d\lambda$. The Lioville one-form can be defined intrinsically:
\begin{equation}\label{liouvilleform}
 \langle \lambda_p, v\rangle:= \langle p, (\pi_p)_\ast (v)\rangle
\end{equation}
with  $v\in T(T^*M), p\in T^*M$.

A standard argument (see for instance \cite{GS90}) shows that the cotangent lift $\hat{\rho}$  is Hamiltonian with moment map $\mu:T^*M \to \mathfrak{g}^*$ given by
\begin{equation*}\label{eqn:lift}
\langle\mu(p),X \rangle := \langle \lambda_p ,X^\#|_{p} \rangle =\langle p,X^\#|_{\pi(p)}\rangle,
\end{equation*}
where  $p\in T^*M$, $X$ is an element of the Lie algebra $\mathfrak{g}$ and we use the same symbol $X^\#$ to denote the fundamental vector field of $X$ generated by the action on $T^\ast M$ or $M$.
This  construction is known  as the {\bf cotangent lift}.

 In the special case where the manifold $M$ is a torus $\T^n$ and the group is $\T^n$ acting by translations, we obtain the following explicit structure: Let $\theta_1,\ldots,\theta_n$  be the standard ($S^1$-valued) coordinates on $\T^n$ and let
\begin{equation}\label{co}
\underbrace{\theta_1,\ldots,\theta_n}_{=:\theta}, \underbrace{t_1, \ldots, t_n}_{=:t}
\end{equation}
be the corresponding chart on $T^\ast \T^n$, i.e. we associate to the coordinates \eqref{co} the cotangent vector $\sum_i t_i d \theta_i \in T^\ast_\theta \T^n$.
The Liouville one-form is given in these coordinates by
$$ \lambda = \sum_{i=1}^n t_i d \theta_i $$
and its negative differential is the standard symplectic form on $T^\ast \T^n$:
\begin{equation}\label{eq:omegacan}
\omega_{can} = \sum_{i=1}^n d \theta_i \wedge d t_i .
\end{equation}
%
Denoting by $\tau_\beta$ the translation by $\beta \in \T^n$ on $\T^n$, its lift to $T^\ast \T^n$ is given by
$$ \hat \tau_\beta: (\theta, t) \mapsto (\theta + \beta, t).$$
The moment map $\mu_{can}: T^\ast \T^n \to \mathfrak{t^\ast} $ of the lifted action with respect to the canonical symplectic form is
\begin{equation}\label{eq:mucan}
\mu_{can}(\theta,t) = \sum_i t_i d\theta_i,
\end{equation}
where the $\theta_i$ on the right hand side are understood as elements of $ \mathfrak{t^\ast}$ in the obvious way. Even simpler, if we identify $ \mathfrak{t^\ast}$ with $\R^n$ by choosing the standard basis $\frac{\partial}{\partial \theta_i}$ of  $\mathfrak{t}$ then the moment map is just the projection onto the second component of $T^\ast \T^n \cong \T^n \times \R^n$.  Note that the components of $\mu$ naturally define an integrable system on $T^\ast \T^n$.


 We can rephrase the Arnold-Liouville-Mineur theorem in terms of the symplectic cotangent model:

\begin{theorem} Let $F=(f_1,\ldots,f_n)$ be an integrable system on the symplectic manifold $(M,\omega)$. Then semilocally around a regular Liouville torus the system is equivalent to the cotangent model $(T^\ast \T^n)_{can}$ restricted to a neighbourhood of the zero section $(T^\ast \T^n)_0$ of $T^\ast \T^n$.
\end{theorem}


\section{The case of $b^m$-symplectic manifolds}



Let us start by introducing the twisted $b^m$-cotangent model for torus actions. This model has additional invariants: the modular vector field of the connected component of the critical set and the modular weights of the associated toric action.
Consider $T^\ast \T^n$ be endowed with the standard coordinates $(\theta, t)$, $\theta \in \T^n$, $t \in \R^n$ and consider again the action on $T^\ast \T^n$ induced by lifting translations of the torus $\T^n$. We will now  view this action as a $b^m$-Hamiltonian action with respect to a suitable $b^m$-symplectic form. In analogy to the classical Liouville one-form we define the following  non-smooth one-form away from the hypersurface $Z=\{t_1 = 0\}$~:

$$ \left(c c_1 \log|t_1| + \sum_{i=2}^{m}c c_i \frac{t_1^{-(i-1)}}{-(i-1)}\right) d \theta_1 + \sum_{i=2}^n t_i d\theta_i.$$

When differentiating this form we obtain a $b^m$-symplectic form on $T^\ast \T^n$ which we call (after a sign change) the {\bf twisted $b^m$-symplectic form  }on $T^\ast \T^n$ with invariants $(c c_1, \dots, c c_m)$:
\begin{equation}\label{eq:twistedform}
 \omega_{tw, c}:=\left(\sum_{j=1}^m c_j\frac{c}{t_1^j}d t_1\wedge d\theta_1\right) + \sum_{i=2}^{n} d t_i\wedge d\theta_i,
\end{equation}
where $c$ is the modular period.
The moment map of the lifted action  is then given by
\begin{equation}\label{eq:bmucan}\mu_{tw, q_0, \dots, q_{m-1})}:=( q_0 \log|t_1| + \sum_{i=2}^{m}q_i t_1^{-(i-1)} ,t_2, \ldots, t_n),
\end{equation}
where we are identifying $\mathfrak{t^\ast}$ with $\R^n$ and  $c_j = -(j-1)q_{j-1}$.

We call this lift together with the $b^m$-symplectic form \ref{eq:twistedform} the {\bf twisted $b^m$-cotangent lift} with modular period $c$ and invariants $(c_1, \dots, c_m)$. Note that the components of the moment map define a $b^m$-integrable system on $(T^\ast \T^n, \omega_{tw, (c c_1, \dots, c c_m)})$.

The model of twisted $b^m$-cotangent lift allows us to express the action-angle coordinate theorem for $b^m$-integrable systems in the following way:

\begin{theorem} Let $F=(f_1,\ldots,f_n)$ be a $b^m$-integrable system on the $b^m$-symplectic manifold $(M,\omega)$.  Then semilocally around a regular Liouville torus $\mathbb T$, which lies inside the critical hypersurface $Z$ of $M$, the system is equivalent to the cotangent model $(T^\ast \T^n)_{tw, (c c_1, \dots, c c_m)} $ restricted to a neighbourhood of $(T^\ast \T^n)_0$. Here $c$ is the modular period of the connected component of $Z$ containing  $\mathbb T$ and the constants $(c_1, \dots, c_m)$ are the invariants associated to the integrable system and its associated toric action.
\end{theorem}


\endinput

%-----------------------------------------------------------------------
% End of chap1.tex
%-----------------------------------------------------------------------

%-----------------------------------------------------------------------
% Beginning of chap1.tex
%-----------------------------------------------------------------------
%
%  AMS-LaTeX sample file for a chapter of a monograph, to be used with
%  an AMS monograph document class.  This is a data file input by
%  chapter.tex.
%
%  Use this file as a model for a chapter; DO NOT START BY removing its
%  contents and filling in your own text.
% 
%%%%%%%%%%%%%%%%%%%%%%%%%%%%%%%%%%%%%%%%%%%%%%%%%%%%%%%%%%%%%%%%%%%%%%%%

\part{A KAM theorem for $b^m$-symplectic manifolds} 

The KAM theorem explains how integrable systems behave under small perturbations. More precisely, it studies how an integrable system in action-angle coordinates responds to a small perturbation on its Hamiltonian. The trajectories of an integrable system in action-angle coordinates can be seen as linear trajectories over a torus. The KAM theorem finds a way to transform these original trajectories to other linear trajectories over some transformed torus. The KAM theorem states that most of these tori, and the linear solutions of the system on these tori, survive if the perturbation is small enough.


In this part, we give a new KAM theorem for $b^m$-symplectic manifolds with detailed proof. This is contained in the first chapter of this part. 
Moreover, we devote three more chapters to applications:

\begin{enumerate}
\item \textbf{Desingularization of $b^m$-integrable systems.} We present a way to use the desingularization of $b^m$-symplectic manifolds presented in \cite{GMW17} to construct standard smooth integrable systems from $b^m$-integrable systems. This desingularized integrable system is uniquely defined.
\item \textbf{Desingularization of the KAM theorem on $b^m$-symplectic manifolds.} In this section we use the desingularization of $b^m$-integrable systems in conjunction with the KAM theorem for $b^m$-symplectic manifolds to deduce the original KAM theorem as well as a completely new KAM theorem for folded symplectic forms.
\item \textbf{Potential applications to Celestial mechanics.} We overview a list of motivating examples from Celestial mechanics where regularization transformations give rise to $b^m$-symplectic forms. We discuss some potential applications of perturbation theory in this set-up.
 \end{enumerate}
\chapter{A new KAM theorem}


The objective of this chapter is to give a construction of KAM theory in the setting of $b^m$-symplectic manifolds and with $b^m$-integrable systems. The core of the chapter is the construction of the proper statement and the proof of the equivalent of the KAM theorem on $b^m$-symplectic manifolds.


This chapter is  divided different sections:

\begin{enumerate}
\item \textbf{On the structure of the proof.} On this section we are going to present the main ideas that are going to appear in the proper statement and proof of the main theorem. The idea of the theorem is to build a sequence of $b^m$-symplectomorphisms such that its limit transforms the hamiltonian to only depend on the action coordinates.
\item \textbf{Technical results and definitions.} On this section we present some technical results and definitions that are key for the proof of the main theorem.
\item \textbf{KAM theorem on $b^m$-symplectic manifolds.} On this section we present the statement and the proof of the main result of this chapter. The proof is structured in 6 parts. In the first part we define the parameters that are going to be used to define the sequence of $b^m$-symplectomorphisms. In the second part we build precisely this sequence of $b^m$-symplectomorphisms. In the third part we see that the sequence of frequency maps of the transformed Hamiltonian functions at every step converges. In the fourth part we see that the sequence of $b^m$-symplectomorphisms converges. In the fifth part we obtain results on the stability of the trajectories under the original perturbation. In the sixth part, we find bounds to explain how close the invariant tori are from the unperturbed.
Finally, we obtain a bound for the measure of the set of invariant tori.

\end{enumerate}

\section{On the structure of the proof}

The first thing we do is to reduce our study to the case the perturbation is not a $b^m$-function but an analytic one. This is because any purely singular perturbation only affects the component in the direction of the modular vector field and can be easily controlled.

The idea of the proof is really similar to the classical KAM case.
We want to build a diffeomorphism such that its transformed hamiltonian only depends on the action coordinates. But it is not possible to build this diffeomorphism in one step. What we do, as it is done in the classical case, it is to build a sequence of diffeomorphisms such that the part of the hamiltonian depending on the angular variables decreases at every step. The idea is to remove the first $K$ terms of its Fourier expression at every step while making $K$ rapidly increase. This is done by assuming the diffeomorphism comes as the flow at time 1 generated by a Hamiltonian function. In this way one can use the Lie Series in conjunction with the Fourier series to find the expression for the hamiltonian function that generates our diffeomorphism. The final diffeomorphism will be the composition of all the diffeomorphisms obtained at each step. One of the main difficulties of the proof, as in the classical case, is to prove that these diffeomorphisms converge and to prove some bounds of its norm.

We also note that for our $b^m$-symplectic setting, the diffeomorphisms we consider leave the defining function of the critical set invariant up to order $m$, this will have an important role later. Also observe in particular that the critical set can not be transformed by any perturbation given by a $b^m$-function.\\
Next we give some technical definitions and results. We define the norms we are going to use to do all the estimates. We set the notation for the proof and the statement of the theorem. We define the notion of non-resonance for a neighborhood of the critical set of the $b^m$-symplectic manifold. We study the set of all possible non-resonant vectors. And we state the inductive lemma, which gives us estimates and constructions for every step of our sequence of diffeomorphisms.

After all this discussion we are in conditions to properly state the $b^m$-version of the KAM theorem. One important difference to the classical KAM theorem is that we have to guarantee that at $Z$ the set of non-resonant vectors does not become the whole set of frequencies. This condition can be understood as the perturbation being smaller than some constant multiplied by the inverse of the modular period.

The proof of the theorem is done in six different steps by following the structure on \cite{D}. Since we are going to use the inductive lemma at every step, first we define the parameters and sets to which  we are going apply such lemma. Then we check that we can actually apply the lemma and obtain some extra estimates for the results of the lemma. After this we see that the sequence of frequency vectors converges. We do the same with the sequence of canonical transformations. Then we get some bounds for the size of the components of the final diffeomorphism. Next we characterize the tori that survive by the perturbation. Finally we give some estimates for the measure of the set of these tori.

Note that our version of the $b^m$-KAM theorem improves the one in \cite{KMS16} in several ways. Firstly it is applicable to $b^m$-symplectic structures not only for $b$-symplectic. Also we give several estimates that are not obtained in \cite{KMS16}, this estimates have sense in a neighborhood of the critical set $Z$, while \cite{KMS16} only studied the behavior at $Z$. Finally the type of perturbation we consider is far more general, since we do not have any condition of the form of the perturbation but only on its size.


\subsection{Reducing the problem to an analytical perturbation.}

In the standard KAM, we assume to have an analytic Hamiltonian $h(I)$ depending only on the action coordinates and we add to it a small analytical perturbation $R(\phi,I)$. This perturbed system receives the name of \emph{nearly integrable system}. And then find a new coordinate system such that $h(I) + R(\phi,I) = \tilde h(\tilde I)$ where most of the quasi-periodic orbits are preserved and can be mapped to the unperturbed quasi-periodic orbits by means of the coordinate change.

In our setting we may assume $h(I)$ to not be analytical and be a $b^m$-function. Also the  perturbation $R(\phi, I)$ may as well be considered a $b^m$-function. In the following lines we justify without loss of generality that actually we can assume the perturbation to be analytical.

Let us state this more precisely. Let $(M,x,Z,\omega,F)$ be a $b^m$-manifold with a $b^m$ integrable system $F$ on it. Consider action angle coordinates on a neighborhood of $Z$.
Then we can assume the expressions:

$$\omega = \left(\sum_{j=1}^{m}\frac{c_j}{I_1^j}\right) d I_1 \wedge d\phi_1 + \sum_{i=2}^n dI_i\wedge d\phi_i, \text{ and }$$
$$F = (q_0' \log I_1 + \sum_{i=1}^{m-1} q_i'\frac{1}{I_1^i} + h(I), f_2, \ldots, f_n)$$

where $h,f_2,\ldots,f_n$ are analytical.

Let the Hamiltonian function of our system be the first component of the moment map $\hat h' = q'_0\log I_1 + \sum_{i=1}^{m-1} q_i'\frac{1}{I_1^i} + h = \zeta' + h$, where $\zeta' := q_0' \log I_1 + \sum_{i=1}^{m-1} q_i' \frac{1}{I_1^i}$. Note that $d\zeta' = \sum_{i=1}^m \hat q_i'\frac{1}{I_1'}$, where $\hat q_i ' = -(i-1)q_{i-1}'$. Note that by the result of the previous chapter $c_j/\hat q_j ' = \mathcal{K}$ the modular period. In particular $c_m/\hat q_m ' = \mathcal{K}$.

The hamiltonian system given by $\hat h'$ can be easily solved by $\phi = \phi_0 + u' t, I = I_0$ where $u'$ is going to be defined in the following sections.
Consider now a perturbation of this system: $\hat H' = \hat h'(I) = \hat R(I,\phi)$, where $\hat R$ is a $b^m$-function $\hat R(I,\phi) = R_{\zeta}(I_1) + R(I,\phi)$ where $R_\zeta(I_1) = (r_0 \log I_1 + \sum_{i=1}^{m-1} r_i \frac{1}{I_1^i})$ is the singular part. Then we can consider the perturbations $R_\zeta(I_1)$ and $R(I,\phi)$ separately. This way, we may consider $R_\zeta(I)$ as part of $\hat h'(I)$. Then we have a new hamiltonian
$$\hat h(I) = (q_0' + r_0) \log I_1 + \sum_{i=1}^{m-1}(q_i' + r_i)\frac{1}{I_1^i} + h = q_0 \log I_1 + \sum_{i=1}^{m-1} q_i \frac{1}{I_1^i} + h.$$

Now, instead of the identity $\mathcal{K} \hat q_j' = c_j$ we will have $\mathcal{K} (\hat q_j - \hat r_j) = c_j$, which implies $\mathcal{K}\left(1 - \frac{\hat r_j}{\hat q_j ' + \hat r_j}\right) = \frac{c_j}{\hat q_j}$. In particular

$$\mathcal{K}\left(1 - \frac{\hat r_m}{\hat q_m ' + \hat r_m}\right) = \frac{c_m}{\hat q_m}$$

Let us define $\mathcal{K}' = \mathcal{K}\left(1 - \frac{\hat r_m}{\hat q_m ' + \hat r_m}\right)$.
So from now on we assume $\hat h = q_0 \log I_1 + \sum_{i=1}^{m-1} q_i \frac{1}{I_1^i} + h$, that the perturbation $R(\phi,I)$ is analytical, and we have the condition $\frac{c_m}{\hat q_m} = \mathcal{K}'$. Observe that this system with only the singular perturbation is still easy to solve in the same way that the system previous to this perturbation was.

\subsection{Looking for a $b^m$-symplectomorphism}

Assume we have a Hamiltonian function $H = \hat{h}(I) + R(\phi,I)$ in action-angle coordinates. Where $\hat{h}(I)$ is the singular component of the $b^m$-integrable system, i.e.
\begin{equation}\label{eq:bm-hamiltonian}
\hat{h}(I) = h(I) + q_0 \log(I_1) + \sum_{i = 1}^{m-1}q_i\frac{1}{I_1^i},
\end{equation}
where $h(I)$ is analytical\footnote{If another component of the moment map is chosen to be the hamiltonian of the system, the result still holds: the computations can be replicated assuming $\hat{h}(I) = h(I)$.}.
Assume also that the $b^m$-symplectic form $\omega$\footnote{In classical KAM, $\omega$ is used to denote the frequency vector $\frac{\partial h}{\partial I}$. We need $\omega$ to denote the $b^m$-symplectic form so we are going to use $u$ to denote the frequency vector.} in these coordinates is expressed as:
\begin{equation}\label{eq:bm-symplectic}
\omega = \left(\sum_{j = 1}^m\frac{c_{j}}{I_1^j}\right)dI_1\wedge d \phi_1 + \sum_{i=2}^{n}dI_i\wedge d\phi_i.
\end{equation}

And finally, the expression for the frequency vector is:

$$\hat{u} = \frac{\partial \hat{h}}{\partial I} = \frac{\partial(h(I) + q_0 \log(I_1) + \sum_{i = 1}^{m-1}q_i\frac{1}{I_1^i})}{\partial I}$$ $$= \left(u_1 + \sum_{i = 1}^{m}\frac{\hat{q}_i}{I_1^i}, u_2, \ldots, u_n\right),
$$

where $\hat{q}_1 = q_0$ and $\hat{q}_{i-1} = -iq_i$ if $i \neq 0$.

The objective is to follow the steps of the usual KAM construction (the steps followed are highly inspired in \cite{D}) replacing the standard symplectic form for $\omega$ and taking as hamiltonian the $b^m$-function $\hat{h}$.

\begin{remark}
The objective of the construction is to find a diffeomorphism (actually a $b^m$-symplectomorphism) $\psi$ such that $H\circ\psi=h(\tilde{I})$. This is done inductively, by taking $H\circ\psi=H \circ \phi_1\circ\ldots\circ \phi_q \circ\ldots$, while trying to make $R(\phi,I)$ smaller at every step.
\end{remark}

\textbf{Let us focus in one single step}

Recall the classical formula:

\begin{lemma}\label{lemma:lie_taylor} See \cite{D}.
$$f\circ\phi_t=\sum_{j=0}^{\infty}\frac{t^j}{j!}L_W^jf,\quad L_W^jf=\{L_W^{j-1}f,W\}$$
Where $W$ is the Hamiltonian that generates the flow $\phi_t$, and $\{\cdot,\cdot\}$ is the corresponding Poisson bracket.
\end{lemma}

We will denote $r_k(H,W,t)=\sum_{j=k}^\infty\frac{t^j}{j!}L_W^jH$.

\begin{equation}\label{eq:new_hamiltonian_expression}
\begin{array}{rcl}
\displaystyle{H\circ\phi = \left.H\circ\phi\right|_{t=1}}& = & \displaystyle{\sum_{j=0}^{\infty}\frac{t^j}{j!}\left.L_W^j\underbrace{H}_{\hat{h}+R}\right|_{t=1}}\\
\\
&=& \displaystyle{ \hat{h}+R\{\hat{h}+R,W\}+r_2(H,W,1)}\\
\\
&=& \displaystyle{ \hat{h}+R+\{\hat{h},W\}+\{R,W\}+r_2(\hat{h},W,1)}\\
\\
& & \quad \displaystyle{+r_2(R,W,1)}\\
\\
&=& \displaystyle{ \hat{h}+\underbrace{R+\{\hat{h},W\}}_{\begin{subarray}{c}\text{We want to cancel} \\ \text{this term as} \\ \text{fast as we can}\end{subarray}}+r_2(\hat{h},W,1)+r_2(R,W,1)}
\end{array}
\end{equation}

We want $\{\hat{h},W\}+R_{\leq k} =0$, equivalently $\{W,\hat{h}\}= R_{\leq k} $, where $R_{\leq k} $ means the Fourier expression of $R$ up to order $K$:
$$R_{\leq k} = \sum_{\begin{subarray}{c} k\in\mathbb{R}^n \\ |k|_1 \leq K\end{subarray}} R_k(I)e^{ik\cdot\phi}$$

Let us impose the condition $\{W,\hat{h}\}=R_{\leq K}$. Let us write the expression of the Poisson bracket associated to the $b^m$-symplectic form.

$$
\begin{array}{rcl}
\{W,\hat{h}\} & = & \displaystyle \left(\frac{1}{\sum_{j = 1}^m \frac{c_j}{I_1^j}}\right)\left(\frac{\partial W}{\partial \phi_1}\frac{\partial \hat{h}}{\partial I_1} - \frac{\partial W}{\partial I_1}\frac{\partial \hat{h}}{\partial \phi_1}\right)\\
& & + \displaystyle{\sum_{i = 2}^n \left(\frac{\partial W}{\partial \phi_i}\frac{\partial \hat{h}}{\partial I_i} - \frac{\partial W}{\partial I_i}\frac{\partial \hat{h}}{\partial \phi_i}\right)}
\end{array}
$$

%Using the two expanded expressions:

%$$
%\begin{array}{rcl}
% \displaystyle{\sum_{i = 2}^n \left(\frac{\partial W}{\partial \phi_i}\frac{\partial h}{\partial I_i} - \frac{\partial W}{\partial I_i}\frac{\partial h}{\partial \phi_i}\right)} + I_1^m\left(\frac{\partial W}{\partial \phi_i}\frac{\partial h}{\partial I_i} - \frac{\partial W}{\partial I_i}\frac{\partial h}{\partial \phi_i}\right) & = & \displaystyle{\sum_{\begin{subarray}{c} k\in\mathbb{R}^n \\ |k|_1 \leq K\end{subarray}} R_k(I)e^{ik\cdot\phi}}
%\end{array}
%$$

Because $\hat{h}$ depends only on $I$, $\frac{\partial \hat{h}}{\partial \phi_i} = 0$ for all $i$. Moreover, the singular part of the $b^m$-function only depends on $I_1$ and hence its derivatives with respect to the other variables are also 0. Using that $\frac{\partial \hat{h}}{\partial I} = u + \sum_{i = 1}^{m}\frac{\hat{q}_i}{I_1^i}$ the previous expression can be  simplified:

$$
\begin{array}{rcl}
\{W,\hat{h}\} & = & \displaystyle \left(\frac{u_1 + \sum_{i = 1}^{m}\frac{\hat{q}_i}{I_1^i}}{\sum_{j = 1}^m \frac{c_j}{I_1^j}}\right)\frac{\partial W}{\partial \phi_1} + \displaystyle\sum_{i = 2}^n \frac{\partial W}{\partial \phi_i}u_i
\end{array}
$$

To expand the expression further we develop $W$ in its Fourier expression: $W=\sum_{\begin{subarray}{c} k\in\mathbb{R}^n \\ |k|_1 \leq K\end{subarray}}W_k(I)e^{ik\phi}$. The Fourier expansion is added up to order $K$, because it is only necessary for the expressions to agree up to order $K$. With this notations the condition becomes:

\begin{longtable}{rcl}
$\{W,\hat{h}\}_{\leq K}$ & $=$ & $\displaystyle
 \left(\frac{u_1 + \sum_{i = 1}^{m}\frac{\hat{q}_i}{I_1^i}}{\sum_{j = 1}^m \frac{c_j}{I_1^j}}\right)\frac{\partial }{\partial \phi_1}\left(\sum_{\begin{subarray}{c} k\in\mathbb{R}^n \\ |k|_1 \leq K\end{subarray}}W_k(I)e^{ik\phi}\right)$\\
 \\
& & $\qquad \qquad \qquad  + \displaystyle\sum_{j = 2}^n u_j \frac{\partial }{\partial \phi_j}\left(\sum_{\begin{subarray}{c} k\in\mathbb{R}^n \\ |k|_1 \leq K\end{subarray}}W_k(I)e^{ik\phi}\right)$\\
\\
& $=$ & $\displaystyle\left(\frac{u_1 + \sum_{i = 1}^{m}\frac{\hat{q}_i}{I_1^i}}{\sum_{j = 1}^m \frac{c_j}{I_1^j}}\right)\left(\sum_{\begin{subarray}{c} k\in\mathbb{R}^n \\ |k|_1 \leq K\end{subarray}}W_k(I)e^{ik\phi}ik_1\right)$\\
\\
& &  $\qquad \qquad \qquad + \displaystyle\sum_{j = 2}^n u_j \left(\sum_{\begin{subarray}{c} k\in\mathbb{R}^n \\ |k|_1 \leq K\end{subarray}}W_k(I)e^{ik\phi}ik_j\right)$
\\
\\
& = & $\displaystyle \sum_{\begin{subarray}{c} k\in\mathbb{R}^n \\ |k|_1 \leq K\end{subarray}} W_k(I)e^{ik\phi}\cdot\left( i k_1 \left(\frac{u_1 + \sum_{i = 1}^{m}\frac{\hat{q}_i}{I_1^i}}{\sum_{j = 1}^m \frac{c_j}{I_1^j}}\right)
+ \sum_{j=2}^nik_ju_j\right)$\\
\\
$ = R_{\leq K}$\\
\end{longtable}
Then, it is possible to make the two sides of the equation equal by imposing the condition term by term:

\begin{equation}\label{eq:solve_coefs}
\begin{array}{rcl}
W_k(I) & = & \displaystyle R_k(I)\frac{1}{i\left(k_1 \left(\frac{u_1 + \sum_{i = 1}^{m}\frac{\hat{q}_i}{I_1^i}}{\sum_{j = 1}^m \frac{c_j}{I_1^j}}\right)  + \sum_{j = 2}^n k_ju_j\right)}
\\
\\
& = & \displaystyle R_k(I)\frac{1}{i\left(k_1 \left(\frac{u_1 + \sum_{i = 1}^{m}\frac{\hat{q}_i}{I_1^i}}{\sum_{j = 1}^m \frac{c_j}{I_1^j}}\right) + \bar{k}\bar{u}\right)},
\end{array}
\end{equation}

where we adopted the notation $\sum_{j = 2}^n k_ju_j = \bar{k}\bar{u}$.

\begin{remark}
Observe that the expression \ref{eq:solve_coefs} has no sense when $k = \vec{0}$ and hence $\{W,h\}_0=R_0$\footnote{The zero term of the Fourier series can be seen as the angular average of the function} can not be solved. Let $W_0(I)=0$, then $\{h,W\}_{\leq K} = R_{\leq K} -R_0$.
\end{remark}

Plugging the results above into the equation  \ref{eq:new_hamiltonian_expression}, one obtains:

$$
H\circ\phi = \hat{h}  + R_0 + R_{\geq K} + r_2(\hat{h},W,1) + r_1(R,W,1)
$$

With this construction the diffeomorphism $\phi$ is found. But this is  only the first of many steps. If $q$ denotes the number of the iteration of this procedure, in general, we obtain:

\begin{equation}
\begin{array}{rcl}
H^{(q)}  =  H^{(q-1)}\circ\phi^{(q)} & = &  \hat{h}^{(q-1)}  + R_0^{(q-1)} + R_{\geq K}^{(q-1)} \\
& & + r_2(h^{(q-1)},W^{(q)},1) + r_1(R^{(q-1)},W^{(q)},1),
\end{array}
\end{equation}

and at every step:

\begin{equation}\label{eq:iterative_h_and_R}
\begin{cases}
\hat{h}^{(q)}  =  \hat{h}^{(q-1)} + R_0^{(q-1)}\\
R^{(q)}  =  R_{>K}^{(q-1)} + r_2(\hat{h}^{(q-1)}, W^{(q)},1) + r_1(R^{(q-1)}, W^{(q)},1)\\
\end{cases}
\end{equation}

\subsection{On the change of the defining function under \\ $b^m$-symplectomorphisms}

Note that since we are considering $b^m$-manifolds it only makes sense to consider $I_1$ up to order $m$, see \cite{Scott16}. When talking about defining functions we are interested in $[I_1]$, its jet up to order $m$.
By definition $b^m$-maps preserve $I_1$ up to order $m$ and $b^m$-vector fields $X$ are such that $\mathcal{L}_X(I_1) = g\cdot I_1^m$ for $g\in\mathcal{C}^\infty(M)$.

\begin{lemma}
Let $\phi_t$ be the integral flow of $X$ a $b^m$-vector field, then $\phi_t$ is a $b^m$-map.
\end{lemma}
\begin{proof}
We want
$$I_1\circ \phi_t = I_1 + I_1^m\cdot g$$ for some $g\in \mathcal{C}^\infty(M)$.
We will use \ref{lemma:lie_taylor}.
$$I_1 \circ \phi_t = \sum_{j=0}^\infty \frac{t^j}{j!}L_X^j I_1 = I_1 + \mathcal{L}_X(I_1) + \sum_{j=2}^\infty\frac{t^j}{j!}L_X^j I_1$$
$$= I_1 +  I_1^m + \sum_{j=2}^\infty\frac{t^j}{j!}L_X^j I_1.$$
On the other hand, let us prove by induction $L_X^k I_1 = g^{(k)} I_1^m$. The first case is obvious, assume the case $k$ holds and let us prove the case $k+1$.

$$
\begin{array}{rcl}
L_X^{k+1} I_1 & = & \{L_X^k I_1, X\} \\
& = & \{g^{(k)} I_1^m, X\} \\
& = & (L_X g^{(q)}) I_1^m + g^{(k)}\cdot m I_1^{m-1}L_X I_1 \\
& = & (L_X g^{(k)} + g^{(k)}\cdot m \cdot I_1^{m-1}\cdot g) I_1^m \\
& = & g^{(k+1)} I_1^m \\
\end{array}
$$

where $g^{(k+1)} = L_X g^{(k)} + g^{(k)}\cdot m \cdot I_1^{m-1}\cdot g$.

\end{proof}

\begin{lemma}
The Hamiltonian vector flow of some smooth hamiltonian function $h$ is a $b^m$-vector field.
\end{lemma}
\begin{proof}
At each point of $Z$ the following identity holds $\mathcal{L}_{X_h} I_1 = I_1^m\frac{\partial f}{\partial \phi_1}$. The result can be extended at a neighborhood of $Z$.
\end{proof}

Observe that combining the two previous results we get that the hamiltonian flow of a function preserves $I_1$ up to order $m$.


%A $b^m$-symplectomorphism must preserve the defining function of the critical set $Z$ by definition, but we are going to the check that actually the $b^m$-symplectomorphisms defined above send $I_1$ to $I_1$ which is more demanding than just leaving $Z$ invariant. Which is actually true because we are considering the analytic version.


%Let $h$ be any $b^m$-function $h = q_0 \log I_1 + \sum_{i=1}^{m-1} q_i \frac{1}{I_1^i} + f$, $q_0 \in \mathbb{R}$, $q_i \in \mathcal{C}^\infty(I_1)$ if $i \geq 2$.
%In this more general case $\hat q_i = -(i-1)q_{i-1} + \partial(q_i)/\partial I_1$.
%Then,
%$$dh = \sum_{i=1}^m \frac{\hat q_i}{I_1^i} dI_1 + \sum_{i=1}^m \frac{\partial f}{\partial I_i} dI_i + \sum_{i= 1}^m \frac{\partial f}{\partial \phi_i} d\phi_i.$$
%Assume also,
%$$\Pi = \frac{1}{\sum_{i=1}^m\frac{c_i}{I_1^i}}\frac{\partial}{\partial I_1}\wedge\frac{\partial}{\partial \phi_1} + \sum_{j=2}^n \frac{\partial}{\partial I_j}\wedge\frac{\partial}{\partial \phi_j}.$$
%With this equation we can compute
%$$X_h = \Pi(dh,\cdot) = \frac{\sum_{i=1}^m \frac{\hat q_i}{I_1^i}}{\sum_{i=1}^m \frac{c_i}{I_1^i}}\frac{\partial}{\partial \phi_1} + \frac{\frac{\partial f}{\partial I_i}}{\sum_{i=1}^m \frac{c_i}{I_1^i}}\frac{\partial}{\partial \phi_1} - \frac{\frac{\partial f}{\partial \phi_1}}{\sum_{i=1}^m \frac{c_i}{I_1^i}}\frac{\partial}{\partial I_1} + \sum_{j=2}^{n}\frac{\partial f}{\partial I_j}\frac{\partial }{\partial \phi_j} + \sum_{j=2}^{n}\frac{\partial f}{\partial \phi_j}\frac{\partial}{\partial I_j}$$


%We can use formula \ref{lemma:lie_taylor} to see if the function $I_1$ is preserved by the diffeomorphism generated by $X_h$. We have to check that $L_h^q I_1 = 0$ at $Z$ for all $q \geq 1$.

%$$L_h^1 I_1 = \{I_1,h\} = -X_h\cdot I_1 = \frac{\frac{\partial f}{\partial \phi_1}}{\sum_{i=1}^m\frac{c_i}{I_1^i}}.$$

%Which vanishes at $Z$. Moreover,

%$$
%\begin{array}{rcl}
%L_h^2 I_1 = \{L_h^1, h\} = -X_h\cdot L_h^1 I_1 &=& -\frac{\sum_{i=1}^m\frac{\hat q_1}{I_1^i}}{(\sum_{i=1}^m\frac{c_i}{I_1^i})^2}\frac{\partial^2 f}{\partial \phi_1^2} - \frac{\frac{\partial f}{\partial I_1}\frac{\partial^2 f}{\partial \phi^2}}{\sum_{i=1}^m \frac{c_i}{I_1^i})^2}\\
% & & \quad +  \frac{\frac{\partial f}{\partial \phi_1}}{\sum_{i=1}^m \frac{c_i}{I_1^i}}\frac{\frac{\partial^2 f}{\partial \phi_1 \partial I_1}(\sum_{i=1}^m \frac{c_i}{I_1^i}) + \frac{\partial f}{\partial \phi_1}(\sum_{i=1}^{m} (-i\frac{c_i}{I_1^{i+1}}))}{(\sum_{i=1}^m \frac{c_i}{I_1^i})^2}\\
% & & \quad + \sum_{j=2}^n \frac{\partial f}{\partial I_j} \frac{\frac{\partial^2 f}{\partial\phi_1 \phi_j}}{\sum_{i=1}^m \frac{c_i}{I_1^i}} + \sum_{j=2}^n \frac{\partial f}{\partial \phi_j} \frac{\frac{\partial^2 f}{\partial\phi_1 I_j}}{\sum_{i=1}^m \frac{c_i}{I_1^i}}.\\
%\end{array}
%$$

%This way it is easy to prove by induction that every time we iterate $L_h^q I_1$ the maximum degree of the denominators increase faster than the maximum degree of the numerators. Thus this shows what we wanted.

%\begin{remark}
%Since the diffeomorphisms leave $I_1$ invariant they will also leave $\zeta$ invariant, hence the singular part of the hamiltonian will not be modified by the diffeomorphisms outlined above.
%\end{remark}


\section{Technical results}


As the non-singular part of our functions we will be considering analytic functions on $\mathbb{T}\times G$, $G \subset \mathbb{R}^n$. The easiest way to work with these functions is to consider them as holomorphic functions on some complex neighborhood. Let us define formally this neighborhood.

$$\mathcal{W}_{\rho_1}(\mathbb{T}^n) := \{ \phi : \Re\phi \in \mathbb{T}^n, |\Im \phi|_{\infty} \leq \rho_1\},$$
$$\mathcal{V}_{\rho_2}(G) := \{I \in \mathbb{C}^n : |I - I'|\leq \rho_2 \text{ for some } I' \in G\},$$
$$\mathcal{D}_\rho (G) := \mathcal{W}_{\rho_1}(\mathbb{T}^n) \times \mathcal{V}_{\rho_2}(G),$$

where $|\cdot|_\infty$ denotes the maximum norm and $|\cdot|_2$ denotes de Euclidean norm.
Now it is necessary to clarify the norms that are going to be used on these sets.

\begin{definition}
Let $f$ be an action function (only depending on the $I$-coordinates), and $F$ an action vector field.
$$
\begin{array}{rl}
|f|_{G,\eta} := \sup_{I\in\mathcal{V}_\eta(G)} |f(I)|,  & |f|_G := |f|_{G,0}\\
|F|_{G,\eta,p} := \sup_{I\in\mathcal{V}_\eta(G)} |F(I)|_p,  & |F|_{G,\eta} := |F|_{G,\eta,2}\\

\end{array}
$$
Now, assume $f(I,\phi)$ to be an action-angle function written using its Fourier expansion as $\sum_{k\in\mathbb{Z}^n} f_k(I)e^{ik\cdot\phi}$, and $F$ to be an action-angle vector field.
$$
\begin{array}{rcl}
|f|_{G,\rho} := \sup_{(\phi,I)\in\mathcal{D}_\rho(G)} |f(I)|, & \|f\|_{G,\rho}:=\sum_{k\in\mathbb{Z}^n} |f_k|_{G,\rho_2} e^{|k|_1\rho_1}\\
|F|_{G,\rho,p} := \sum_{k\in\mathbb{Z}^n} |F_k|_{G,\rho_2,p} e^{|k|_1\rho_1}, & \|F\|_{G,\rho} = \|F\|_{G,\rho,2}\\
\end{array}
$$
\end{definition}

\begin{lemma}[Cauchy Inequality]
$$\left\|\frac{\partial f}{\partial \phi}\right\|_{G,(\rho_1-\delta_1, \rho_2),1}\leq \frac{1}{e\delta_1}\left\|f\right\|_{G,\rho}$$
$$\left\|\frac{\partial f}{\partial I}\right\|_{G,(\rho_1, \rho_2-\delta_2),\infty}\leq \frac{1}{\delta_2}\|f\|_{G,\rho}$$
\end{lemma}

\begin{definition}
If $Df=(\frac{\partial f}{\partial \phi}, \frac{\partial f}{\partial I})$,
$$\|Df\|_{G,\rho,c}:= \max\left(\|\frac{\partial f}{\partial \phi}\|_{G,\rho,1},c\|\frac{\partial f}{\partial I}\|_{G,\rho,\infty}\right)$$
\end{definition}

\begin{definition}To simplify our notation, let us define:
$$\mathcal{A}(I_1) = \frac{\sum_{j = 1}^{m}\frac{\hat{q}_j}{I_1^j}}{\sum_{j=1}^m \frac{c_j}{I_1^j}} \quad \text{ and } \quad \mathcal{B}(I_1) = \frac{1}{\sum_{j=1}^m \frac{c_j}{I_1^j}}.$$
\end{definition}

\begin{remark}
With this notation, equation \ref{eq:solve_coefs} can be written as:
$$W_k(I) = \frac{R_k(I)}{i(k_1 \mathcal{B}(I_1) u_1 + \bar{k}\bar{u} + k_1 \mathcal{A}(I_1))}$$
\end{remark}



Observe that $\mathcal{A}(I_1)$ and $\mathcal{B}(I_1)$ are analytic (holomorphic on the complex extended domain) where the denominator does not vanish. We can assume that this does not happen by shrinking the domain $G$ in the direction of $I_1$. Observe, in particular, that when $I_1 \rightarrow 0$, $\mathcal{A}(I_1) \rightarrow \hat q_m/c_m = 1/\mathcal{K}'$ the inverse of the modular period and $\mathcal{B}(I_1) \rightarrow 0$. In this way,  the norms of $\mathcal{A}(I_1)$ and $\mathcal{B}(I_1)$ are bounded and well defined. We will denote these norms by $K_{\mathcal{A}}$ and $K_{\mathcal{B}}$ respectively. Also, since $\mathcal{A}(I_1)$ and $\mathcal{B}(I_1)$ are analytic, their derivatives will also be bounded, and we will denote the norms of these derivatives by $K_{\mathcal{A}'}$ and $K_{\mathcal{B}'}$.


To further simplify  the notation in the following computations we introduce the definition:

\begin{definition}

$$
\bar{\mathcal{A}} = \left(\begin{array}{c}
\mathcal{A}\\
0 \\
\end{array}\right)
\quad \text{ and } \quad
\bar{\mathcal{B}} = \left(\begin{array}{cc}
\mathcal{B} & 0\\
0 & \text{Id}_{n-1,n-1}\\
\end{array}\right)
$$

\end{definition}

\begin{remark}
With this notation, equation \ref{eq:solve_coefs} can be written as:
\begin{equation}\label{eq:solve_coefs_simplified}
W_k(I) = \frac{R_k(I)}{i(k \bar{\mathcal{B}}(I_1) u + k \bar{\mathcal{A}}(I_1))}
\end{equation}


\end{remark}

\begin{definition}
Having fixed $\omega$, a $b^m$-symplectic form (as in equation \ref{eq:bm-symplectic}) and $\hat{h}$ a $b^m$-function (as in equation \ref{eq:bm-hamiltonian}) as a hamiltonian. Given an integer $K$ and $\alpha > 0$, $F \subset \mathbb{R}^n$ (or $\mathbb{C}^n$) the space of frequencies is said to be $\alpha, K$-non-resonant with respect to $(c_1,\ldots,c_m)$ and $(\hat{q}_1,\ldots, \hat{q}_m)$ if
$$|k \bar{\mathcal{B}}(I_1) u + k \bar{\mathcal{A}}(I_1)|\geq \alpha, \forall k \in \mathbb{Z}\setminus \{0\}, |k|_1\leq K, \forall u \in F.$$
We are going to use the notation $\alpha, K, c,\hat{q}$-non-resonant.
\end{definition}



\begin{remark}
The non-resonance condition is established on $u = \partial h/\partial I$, not on $\hat{u} = \partial \hat h/\partial I$, because our non-resonance condition already takes into account the singularities. In this way we can use the analytic character of $u$.
\end{remark}

\begin{remark}
If $\left|\frac{\partial u}{\partial I}\right|_{G,\rho_2}$ is bounded by $M'$,  then $\left|\frac{\partial }{\partial I}\left( \bar{\mathcal{B}}u + \bar{\mathcal{A}}\right)\right|_{G,\rho_2}$ is also bounded:

\begin{equation}\label{eq:M_def}
\begin{array}{rcl}
\left|\frac{\partial }{\partial I}\left( \bar{\mathcal{B}}u + \bar{\mathcal{A}}\right)\right|_{G,\rho_2} &
 \leq &
 \left|\frac{\partial \bar{\mathcal{B}}}{\partial I} u
  + \bar{\mathcal{B}} \frac{\partial u}{\partial I}
   + \frac{\partial \bar{\mathcal{A}}}{\partial I}\right|_{G,\rho_2}\\
\\
& \leq & K_{\mathcal{B}'} |u|_{G,\rho_2} + K_{\mathcal{B}}M' + K_{\mathcal{A}} =: M.\\
\\
\end{array}
\end{equation}

\end{remark}

\begin{remark}
When we consider the standard KAM theorem, the frequency vector $u$ is relevant because the solution to the Hamilton equations of the unperturbed problem has the form:

$$I = I_0, \quad \phi = \phi_0 + ut.$$

Let us see what plays the role of $u$ in our $b^m$-KAM theorem. Let us find the coordinate expression of the solution to $\iota_{X_{\hat{h}}} \omega=  d \hat{h}$, where $\omega$ is a $b^m$-symplectic form in action-angle coordinates.

$$X_{\hat{h}} = \dot{I}_1\frac{\partial}{\partial I_1} + \ldots + \dot{I}_n\frac{\partial}{\partial I_n},$$

where $\dot{I}_1,\ldots, \dot{I}_n$ are the functions we want to find.

$$d\hat{h} = \left(\sum_{j=1}^m \hat{q}_i \frac{1}{I_1^j}\right) dI_1 + dh,$$

and hence,

$$X_{\hat{h}} = \Pi(d \hat{h},\cdot ) = \frac{\sum_{i=1}^m \frac{\hat q_i}{I_1^i}}{\sum_{i=1}^m \frac{c_j}{I_1^j}}\frac{\partial}{\partial \phi_i} + X_h.$$

Hence $\phi = \phi_0 + (\underbrace{\bar{\mathcal{B}} u + \bar{\mathcal{A}}}_{u'})t$. So the frequency vector that we are going to be concerned about is going to be $u'$ instead of $\hat{u} = \frac{\partial}{\partial I} \hat{h}$.
\end{remark}
\begin{lemma}
If $u$ is one-to-one from $\mathcal{G}$ to its image then $u' = \bar{\mathcal{B}} u + \bar{\mathcal{A}}$ is also one-to-one from $\mathcal{G}'$ to its image in a neighborhood of $Z$, while at $Z$ it is the projection of $u$ such that the first coordinate is sent to $\frac{\hat q_m}{c_m} = 1/\mathcal{K}'$ the inverse of the modular period, were  $\mathcal{G}'\subseteq \mathcal{G}$.
\end{lemma}
\begin{proof}
Because $$u' = \left(\frac{1}{\sum_{j=1}^m \frac{c_j}{I_1^j}} u_1 + \frac{\sum_{j=1}^m \frac{\hat{q}_j}{I_1^j}}{\sum_{j=1}^m \frac{c_j}{I_1}},u_2,\ldots,u_n\right),$$
and  $\mathcal{B}$ is invertible outside $I_1 = 0$, shrinking $\mathcal{G}$ if necessary in the first dimension the map is one-to-one. But at the critical set $\{I_1 = 0\}$, $u'$ is a projection of $u$ where the first component is sent to the constant value $\frac{\hat q_m}{c_m} = \frac{1}{\mathcal{K}'}$.
\end{proof}
\begin{lemma}
If $u(G)$ is $\alpha,K, c,\hat{q}$-non-resonant, then $u(\mathcal{V}_{\rho_2}(G))$ is $\frac{\alpha}{2},K, c,\hat{q}$-non-resonant, assuming that $\rho_2\leq \frac{\alpha}{2MK}$ and $\left|\frac{\partial u}{\partial I}\right|_{G,\rho_2} \leq M'$
\end{lemma}
\begin{proof}
Fix $k\in\mathbb{Z}\setminus\{0\}$, we want to bound $|k \bar{\mathcal{B}}(I_1) v + k \bar{\mathcal{A}}(I_1)|$ where $v\in u(\mathcal{V}_{\rho_2}(G))$ as a function on $v$.
Given $v \in u(\mathcal{V}_{\rho_2}(G))$ we  ask whether there is any bound for the distance to some $v'\in u(G)$.
$$v \in u(\mathcal{V}_{\rho_2}(G)) \Rightarrow v = u(x), x \in \mathcal{V}_{\rho_2}(G)\Rightarrow\exists y \in G \text{ such that } |x-y| \leq \rho_2.$$
Take $v'=u(y)$.


$$|v-v'|\leq|x-y|\left|\frac{\partial u}{\partial I}\right|_{G,\rho_2} \leq \rho_2 M' \leq \rho_2  M/K_\mathcal{B} \leq \frac{\alpha}{2MK} M/K_\mathcal{B} = \frac{\alpha}{2K K_\mathcal{B}}.$$

Where we used equation \ref{eq:M_def} in the third inequality.

$$
\begin{array}{rcl}
\displaystyle |k_1 \mathcal{B}(I_1) v_1 + \bar{k}\bar{v} + k_1 \mathcal{A}(I_1)| & \geq & \displaystyle \underbrace{|k_1 \mathcal{B}(I_1) v'_1 + \bar{k}\bar{v'} + k_1 \mathcal{A}(I_1)|}_{\geq \alpha} \\
& & -|k_1 \mathcal{B}(I_1) (v_1-v'_1) + \bar{k}(\bar{v} - \bar{v}')|\\
\\
& \geq & \displaystyle \alpha - K_\mathcal{B}\underbrace{|k\cdot(v-v')|}_{\leq K \alpha /(2K K_\mathcal{B})} \\
\\
 & \geq & \alpha -\alpha/2 = \alpha/2
\end{array}
$$

\end{proof}

\begin{proposition}\label{prop:14}
Let $\hat{h}(I)$ be a $b^m$-function as in equation \ref{eq:bm-hamiltonian}. Assume $h(I)$ and $R(\phi, I)$ be real analytic on $\mathcal{D}_\rho(G)$, $u(G) = \frac{\partial h}{\partial I}(G)$ is $\alpha, K, c, \hat{q}$-non-resonant.
Assume also that
$|\frac{\partial}{\partial I} u|_{G,\rho_2} \leq M'$
and $\rho_2 \leq \frac{\alpha}{2MK}$. Let $c > 0$ given.
Then $R_0(\phi, I)$, $W_{\leq K}(\phi, I)$ given by the previous construction are both real analytic on $\mathcal{D}_\rho(G)$ and the following bounds hold
\begin{enumerate}
\item\label{prop:14:item:1} $||D R_0||_{G,\rho,c} \leq ||D R||_{G,\rho,c}$
\item\label{prop:14:item:2} $||D (R - R_0)||_{G,\rho,c} \leq ||D R_0||_{G,\rho,c}$
\item\label{prop:14:item:3} $||D W||_{G,\rho,c} \leq \frac{2A}{\alpha}||D R_0||_{G,\rho,c}$
\end{enumerate}
Where $A = 1 + \frac{2Mc}{\alpha}$
\end{proposition}

\begin{proof}
Inequalities \ref{prop:14:item:1} and \ref{prop:14:item:2} are obvious because of the Fourier expression.
Let us prove inequality \ref{prop:14:item:3}. Let us expand $R(\phi,I)$ and $W(\phi, I)$ in their Fourier expression:
$$R = \sum_{\begin{subarray}{c} k\in\mathbb{R}^n\end{subarray}} R_k(I)e^{ik\cdot\phi}, \quad W = \sum_{\begin{subarray}{c} k\in\mathbb{R}^n \end{subarray}} W_k(I)e^{ik\cdot\phi}.$$
We will bound this expression finding  term-by-term bounds.

$$\frac{\partial R}{\partial \phi} = \sum_{\begin{subarray}{c} k\in\mathbb{R}^n\end{subarray}} R_k(I)e^{ik\cdot\phi}ik.$$

Hence, if we denote  $[ \frac{\partial R}{\partial \phi} ]_k $ the $k$-th term of the Fourier expansion of $\frac{\partial R}{\partial \phi}$, we have:

$$\left[ \frac{\partial R}{\partial \phi} \right]_k = R_k ik.$$

Let us compute the derivative of $W_k$ with respect to the $I$ variables:

\begin{longtable}{rcl}
$\displaystyle \frac{\partial W_k}{\partial I}$ & $=$ & $\displaystyle\frac{\partial}{\partial I}\left(\frac{R_k}{i(k \bar{\mathcal{B}}(I_1) u  + k \bar{\mathcal{A}}(I_1))}\right)$\\
\\
& $=$ & $\displaystyle \frac{\partial R_k/\partial I }{i(k \bar{\mathcal{B}}(I_1) u  + k \bar{\mathcal{A}}(I_1)))} - \frac{R_k i \frac{\partial}{\partial I }(k \bar{\mathcal{B}}(I_1) u  + k \bar{\mathcal{A}}(I_1)))}{[i(k \bar{\mathcal{B}}(I_1) u  + k \bar{\mathcal{A}}(I_1)))]^2}$\\
\\
& $=$ & $\displaystyle \frac{\partial R_k/\partial I }{i(k \bar{\mathcal{B}}(I_1) u  + k \bar{\mathcal{A}}(I_1)))} + \frac{R_k i k \frac{\partial}{\partial I }(\bar{\mathcal{B}}(I_1) u  + \bar{\mathcal{A}}(I_1)))}{[(k \bar{\mathcal{B}}(I_1) u  + k \bar{\mathcal{A}}(I_1)))]^2}$\\
\\
& $=$ & $\displaystyle \frac{\partial R_k/\partial I }{i(k \bar{\mathcal{B}}(I_1) u  + k \bar{\mathcal{A}}(I_1)))} + \frac{[\frac{\partial R_k}{\partial \phi}]_k \frac{\partial}{\partial I }(\bar{\mathcal{B}}(I_1) u  + \bar{\mathcal{A}}(I_1)))}{[(k \bar{\mathcal{B}}(I_1) u  + k \bar{\mathcal{A}}(I_1)))]^2}.$\\
\end{longtable}


Then, we take norms ($|\cdot|_{G,\rho_2,\infty}$) on each side of the equation.

$$
\begin{array}{rcl}
\displaystyle \left|\frac{\partial W_k}{\partial I}\right|_{G,\rho_2, \infty} & \leq & \displaystyle \frac{2}{\alpha}\left|\frac{\partial R_k}{\partial I}\right|_{G,\rho_2, \infty} +
\displaystyle \frac{4M}{\alpha^2}\left|\left[\frac{\partial R_k}{\partial \phi}\right]_k\right|_{G,\rho_2, \infty}\\
\\
& \leq & \displaystyle \frac{2}{\alpha}\left|\frac{\partial R_k}{\partial I}\right|_{G,\rho_2, \infty} +
\displaystyle \frac{4M}{\alpha^2}\left|\left[\frac{\partial R_k}{\partial \phi}\right]_k\right|_{G,\rho_2, 1}.
\end{array}
$$

Taking the supremum at the whole domain:

$$
\begin{array}{rcl}
\displaystyle \left\|\frac{\partial W_k}{\partial I}\right\|_{G,\rho_2, \infty} & \leq & \displaystyle \frac{2}{\alpha}\left\|\frac{\partial R_k}{\partial I}\right\|_{G,\rho_2, \infty} +
\displaystyle \frac{4M}{\alpha^2}\left\|\left[\frac{\partial R_k}{\partial \phi}\right]_k\right\|_{G,\rho_2, 1}.
\end{array}
$$

Moreover,

$$
\begin{array}{rcl}
\displaystyle \frac{\partial W(I)}{\partial \phi} & = & \displaystyle \frac{\partial}{\partial \phi}\left(\sum_{\begin{subarray}{c} k\in\mathbb{R}^n \end{subarray}} W_k(I)e^{ik\cdot\phi}\right)\\
\\
 & = & \displaystyle \frac{\partial}{\partial \phi}\left(\sum_{\begin{subarray}{c} k\in\mathbb{R}^n \end{subarray}} ik W_k(I)e^{ik\cdot\phi}\right).\\
\end{array}
$$

Hence, the $k$-th term of the Fourier series of $\frac{\partial W}{\partial \phi}$ is

$$\left[\frac{\partial W}{\partial \phi}\right]_k = W_k ik = \frac{R_k}{i(k \bar{\mathcal{B}}(I_1) u  + k \bar{\mathcal{A}}(I_1)))}ik $$
$$= \frac{1}{i(k \bar{\mathcal{B}}(I_1) u  + k \bar{\mathcal{A}}(I_1)))}\left[\frac{\partial R}{\partial \phi}\right]_k.$$

Taking norms ($\|\cdot\|_{G,\rho,1}$) at each side:

$$ \left\|\frac{\partial W}{\partial \phi}\right\|_{G,\rho,1} \leq \frac{2}{\alpha}\left\|\frac{\partial W}{\partial \phi}\right\|_{G,\rho,1}.$$

Then,


\begin{longtable}{rcl}
$\displaystyle \left\|D W\right\|_{G,\rho, c}$ & $=$ & $\displaystyle \max\left(\left\|\frac{\partial W}{\partial \phi}\right\|_{G,\rho,1}, c\left\|\frac{\partial W}{\partial I}\right\|_{G,\rho,\infty}\right)$\\
\\
& $\leq$ & $\displaystyle \max\left(\frac{2}{\alpha}\left\|\frac{\partial R}{\partial \phi}\right\|_{G,\rho,1}, c  \frac{2}{\alpha}\left\|\frac{\partial R}{\partial I}\right\|_{G,\rho_2, \infty} +
c\frac{4M}{\alpha^2}\left\|\frac{\partial R}{\partial \phi}\right\|_{G,\rho_2, 1} \right)$\\
\\
& $\leq$ & $\displaystyle \max\left(\frac{2}{\alpha}\left\|\frac{\partial R}{\partial \phi}\right\|_{G,\rho,1}, \frac{2}{\alpha}\left\|DR\right\|_{G,\rho_2, c} +
c \frac{4M}{\alpha^2}\left\| DR\right\|_{G,\rho_2, c} \right)$\\
\\
& $=$ & $\displaystyle \max\left(\frac{2}{\alpha}\left\|\frac{\partial R}{\partial \phi}\right\|_{G,\rho,1}, \frac{2}{\alpha}\left(1 + \frac{2M}{\alpha}c\right)\left\| DR\right\|_{G,\rho_2, c} \right)$\\
\\
& $\leq$ & $\displaystyle \frac{2}{\alpha}\left(1 + \frac{2M}{\alpha}c\right)\left\| DR\right\|_{G,\rho_2, c}$\\
\\
& $\leq$ & $\displaystyle \frac{2}{\alpha}A\left\| DR\right\|_{G,\rho_2, c}$,\\
\end{longtable}


where $A$ is as desired.


\end{proof}

Recall the Cauchy inequalities, see \cite{JP93}:

\begin{equation}\label{eq:cauchy_ineq}
\begin{array}{llr}
\displaystyle \left\|\frac{\partial f}{\partial \phi}\right\|_{G,(\rho_1, \rho_2),1} & \leq & \displaystyle \frac{1}{e \delta_1}\|f\|_{G,\rho} \\
\displaystyle \left\|\frac{\partial f}{\partial I}\right\|_{G,(\rho_1, \rho_2-\delta_2),\infty}& \leq & \displaystyle \frac{1}{\delta_2}\|f\|_{G,\rho}
\end{array}
\end{equation}

\begin{lemma}\label{lemma:1.1}
Let $f,g$ be analytic functions on $\mathcal{D}_\rho(G)$, where
$0<\delta = (\delta_1, \delta_2) < \rho = (\rho_1, \rho_2)$ and $c > 0$.
Define $\hat\delta_c := \min(c\delta_1, \delta_2)$. The following inequalities hold:
\begin{enumerate}
\item $\|Df\|_{G,\rho- \delta,c} \leq \frac{c}{\hat\delta_c}\|f\|_{G,\rho}$
\item $\|\{f,g\}\|_{G,\rho}\leq \frac{2}{c}\|Df\|_{G,\rho, c}\cdot \|D g\|_{G, \rho,c}$
\item $\|D(f_{>K})\|_{G, (\rho-\delta_1, \rho_2),c} \leq e^{-K\delta_1}\|Df\|_{G,\rho, c}$
\end{enumerate}
\end{lemma}

\begin{proof}
Let us prove each point separately.
\begin{enumerate}
\item %$$\|Df\|_{G,\rho-\delta, c} = \max \left\{\left\|\frac{\partial f}{\partial \phi}\right\|_{G,\rho-\delta, 1}, c\left\|\frac{\partial f}{\partial I}\right\|_{G,\rho-\delta,\infty}\right\}$$

Using the Cauchy inequalities one obtains the following:
$$\left\|\frac{\partial f}{\partial \phi}\right\|_{G,\rho-\delta,1} = \left\|\frac{\partial f}{\partial \phi}\right\|_{G,(\rho_1-\delta_1, \rho_2- \delta_2), 1} $$
$$\leq \left\|\frac{\partial f}{\partial \phi}\right\|_{G,(\rho_1-\delta_1, \rho_2), 1} \leq \frac{1}{e\delta_1} \|f\|_{G,\rho},$$
$$\left\|\frac{\partial f}{\partial I}\right\|_{G,\rho-\delta,\infty} = \left\|\frac{\partial f}{\partial I}\right\|_{G,(\rho_1-\delta_1, \rho_2- \delta_2), \infty} $$
$$\leq \left\|\frac{\partial f}{\partial I}\right\|_{G,(\rho_1, \rho_2-\delta_2), \infty} \leq \frac{1}{\delta_1} \|f\|_{G,\rho}.$$
Putting the two inequalities inside the definition of the norm:
$$
\begin{array}{rcl}
\|D f\|_{G,\rho-\delta, c} &=& \displaystyle \max\left\{\left\|\frac{\partial f}{\partial \phi}\right\|_{G,\rho-\delta, 1}, c\left\|\frac{\partial f}{\partial I}\right\|_{G, \rho-\delta, \infty}\right\} \\
\\
& \leq & \displaystyle \max \left\{\frac{1}{e \delta_1}\|f\|_{G,\rho}, \frac{c}{\delta_2}\|f\|_{G, \rho}\right\}\\
\\
& \leq & \displaystyle \max \left\{\frac{1}{e\delta_1}\frac{c}{c}, \frac{c}{\delta_2}\right\}\|f\|_{G,\rho}\\
\\
& \leq & \displaystyle \max \left\{\frac{c}{e \hat\delta_c},\frac{c}{\hat\delta_c}\right\}\|f\|_{G,\rho},\\
\end{array}
$$

where the last inequality holds because $\hat\delta_c = \min(c\delta_1,\delta_2)$.

\item
Let us find the expression of $\{f,g\}$ for a $b^m$-symplectic structure.
$\{f,g\} = \omega(X_f, X_g)$ where $X_f$ and $X_g$ are such that $\iota_{X_f}\omega=df$ and $\iota_{X_g}\omega=dg$. Let restrict the computations only to $f$.
$$df = \sum_{i=1}^n\frac{\partial f}{\partial \phi_1}d\phi_1, \quad X_f = \sum_{i= 1}^n a_i \frac{\partial}{\partial \phi_i} + \sum_{i=1}^n b_i \frac{\partial}{\partial \phi_i}.$$

Where $a_i$ and $b_i$ are coefficients to be determined by imposing the following condition:

$$\iota_{X_f}\omega = \left(\sum_{j=1}^m\frac{c_j}{ I_1^j}\right)(a_1 dI_1 - b_1 d\phi_1) + \sum_{i = 2}^n (a_i dI_i - b_i d\phi_i) = df.$$

Then, solving for the coefficients the following expressions are obtained:

$$a_1  = \frac{1}{\left(\sum_{j=1}^m\frac{c_j}{ I_1^j}\right)}\frac{\partial f}{\partial \phi_1} \quad \text{and} \quad a_i = \frac{\partial f}{\partial \phi_i} \text{ for } i \neq 1,$$
$$b_1  = -\frac{1}{\left(\sum_{j=1}^m\frac{c_j}{ I_1^j}\right)}\frac{\partial f}{\partial \phi_1} \quad \text{and} \quad b_i = -\frac{\partial f}{\partial \phi_i} \text{ for } i \neq 1.$$

Hence, the expression for the hamiltonian vector fields becomes:

$$
X_f = \frac{1}{\left(\sum_{j=1}^m\frac{c_j}{ I_1^j}\right)}\left(\frac{\partial f}{\partial \phi_1}\frac{\partial }{\partial \phi_1}- \frac{\partial f}{\partial I_1}\frac{\partial }{\partial I_1}\right) + \sum_{i = 1}^n\left(\frac{\partial f}{\partial \phi_i}\frac{\partial }{\partial \phi_i}- \frac{\partial f}{\partial I_i}\frac{\partial }{\partial I_i}\right),$$
$$
X_g =  \frac{1}{\left(\sum_{j=1}^m\frac{c_j}{ I_1^j}\right)}\left(\frac{\partial g}{\partial \phi_1}\frac{\partial }{\partial \phi_1}- \frac{\partial g}{\partial I_1}\frac{\partial }{\partial I_1}\right) + \sum_{i = 1}^n\left(\frac{\partial g}{\partial \phi_i}\frac{\partial }{\partial \phi_i}- \frac{\partial g}{\partial I_i}\frac{\partial }{\partial I_i}\right).
$$

Then the Poisson bracket applied to the two functions:

$$
\begin{array}{rcl}
\{f,g\}=\omega(X_f,X_g) & = & \displaystyle \frac{1}{\left(\sum_{j=1}^m\frac{c_j}{ I_1^j}\right)}\left(\frac{\partial f}{\partial I_1}\frac{\partial g}{\partial \phi_1} - \frac{\partial f}{\partial \phi_1}\frac{\partial g}{\partial I_1}\right)
\\
& & \displaystyle \quad + \sum_{i = 2}^n \left(\frac{\partial f}{\partial I_i}\frac{\partial g}{\partial \phi_i} - \frac{\partial f}{\partial \phi_i}\frac{\partial g}{\partial I_i}\right).
\end{array}
$$

And hence the norm of the Poisson bracket becomes:

$$
\begin{array}{rcl}
\|\{f,g\}\|_{G, \rho} & = & \displaystyle \left\|\frac{1}{\left(\sum_{j=1}^m\frac{c_j}{ I_1^j}\right)}\left(\frac{\partial f}{\partial I_1}\frac{\partial g}{\partial \phi_1} - \frac{\partial f}{\partial \phi_1}\frac{\partial g}{\partial I_1}\right)\right. \\
& & \quad \displaystyle+ \left.\sum_{i = 2}^n \left(\frac{\partial f}{\partial I_i}\frac{\partial g}{\partial \phi_i} - \frac{\partial f}{\partial \phi_i}\frac{\partial g}{\partial I_i}\right) \right\|_{G, \rho} \\
& \leq & \displaystyle \left\|\sum_{i = 1}^n \left(\frac{\partial f}{\partial I_i}\frac{\partial g}{\partial \phi_i} - \frac{\partial f}{\partial \phi_i}\frac{\partial g}{\partial I_i}\right) \right\|_{G, \rho} \\
\end{array}
$$

Where we assumed $\left|\sum_{j=1}^m\frac{c_j}{ I_1^j}\right| \geq 1$. This assumption makes sense, because we are interested in the behaviour close the critical set $Z$. Close enough to the critical set this expression holds. Then,

\begin{longtable}{rcl}
$\|\{f,g\}\|_{G, \rho}$ & $\leq$ & $\displaystyle \sum_{i = 1}^n \left\|\frac{\partial f}{\partial I_i}\right\|_{G, \rho}\left\|\frac{\partial g}{\partial \phi_i}\right\|_{G, \rho} +\sum_{i = 1}^n \left\|\frac{\partial f}{\partial \phi_i}\right\|_{G, \rho}\left\|\frac{\partial g}{\partial I_i}\right\|_{G, \rho}$\\
\\
& $\leq$ & $\displaystyle \left|\frac{\partial f}{\partial I}\right|_{G,\rho,\infty} \left|\frac{\partial g}{\partial I}\right|_{G,\rho,1} + \left|\frac{\partial f}{\partial I}\right|_{G,\rho,1} \left|\frac{\partial g}{\partial I}\right|_{G,\rho,\infty}$\\
\\
& $\leq$ & $\displaystyle \frac{1}{c}|Df\|_{G,\rho,c}\|Dg\|_{G,\rho,c} + \frac{1}{c}|Df\|_{G,\rho,c}\|Dg\|_{G,\rho,c}$\\
\\
& $\leq$ & $\displaystyle \frac{2}{c}\|Df\|_{G,\rho,c}\|Dg\|_{G,\rho,c}.$
\end{longtable}


\item Lastly,

$$
\left\|D(f_{>K})\right\|_{G,(\rho_1 - \delta_1,\rho_2),1} $$
$$= \max \left\{ \left\|\frac{\partial f_{>K}}{\partial \phi}\right\|_{G, (\rho_1-\delta_1, \rho_1), 1}, c\left\|\frac{\partial f_{>K}}{\partial I}\right\|_{G, (\rho_1-\delta_1, \rho_1), \infty}\right\}.
$$

We will proceed by bounding each term separately. On one hand:

\begin{longtable}{rcl}
$\displaystyle\left\|\frac{\partial f}{\partial \phi}\right\|_{G,(\rho_1,\rho_2),1}$ & $=$ & $\displaystyle \left\|\sum_{k\in\mathbb{Z}^n} ikf_k(I)e^{ik\phi}\right\|_{G,(\rho_1,\rho_2),1}$\\
\\
& $\geq$ & $\displaystyle \sum_{k\in\mathbb{Z}^n} k\left\|f_k(I)\right\|_{G,\rho_2,1}e^{|k|_1\rho_1}$\\
\\
& $\geq$ & $\displaystyle \sum_{\substack{k\in\mathbb{Z}^n \\ |k|_1 > K}} k\left\|f_k(I)\right\|_{G,\rho_2,1}e^{|k|_1(\rho_1 + \delta_1 - \delta_1)}$\\
\\
& $\geq$ & $\displaystyle e^{K \delta_1}\sum_{\substack{k\in\mathbb{Z}^n \\ |k|_1 > K}} k\left\|f_k(I)\right\|_{G,\rho_2,1}e^{|k|_1(\rho_1 - \delta_1)}$\\
\\
& $=$ & $\displaystyle e^{K \delta_1}\left\|\frac{\partial f_{>K}}{\partial \phi}\right\|_{G,(\rho_1-\delta_1,\rho_2),1}.$\\
\end{longtable}


On the other hand:

\begin{longtable}{rcl}
$\displaystyle \left\|\frac{\partial f}{\partial I}\right\|_{G,(\rho_1,\rho_2),\infty}$& $=$ & $\displaystyle \left\|\sum_{k \in \mathbb{Z}^n} \frac{\partial f_k(I)}{\partial I} e^{ik\phi}\right\|_{G,(\rho_1,\rho_2),\infty}$\\
\\
& $\geq$ & $\displaystyle \sum_{k \in \mathbb{Z}^n} \left\|\frac{\partial f_k(I)}{\partial I}\right\|_{G,\rho_2,\infty} e^{|k|_1\rho_1}$ \\
\\
& $\geq$ & $\displaystyle \sum_{\substack{k\in\mathbb{Z}^n \\ |k|_1 > K}} \left\|\frac{\partial f_k(I)}{\partial I}\right\|_{G,\rho_2,\infty} e^{|k|_1(\rho_1 + \delta_1 - \delta_1)}$ \\
\\
& $\geq$ & $\displaystyle e^{K\delta_1}\sum_{\substack{k\in\mathbb{Z}^n \\ |k|_1 > K}} \left\|\frac{\partial f_k(I)}{\partial I}\right\|_{G,\rho_2,\infty} e^{|k|_1(\rho_1 - \delta_1)}$\\
\\
& $\geq$ & $\displaystyle e^{K\delta_1}\left\|\frac{\partial f_{>K}}{\partial I}\right\|_{G,(\rho_1-\delta_1,\rho_2),\infty}.$\\
\end{longtable}


Hence $\|D(f_{>k})\|_{G,(\rho_1-\delta_1,\rho_2),c} \leq e^{-K\delta_1}\|Df\|_{G,\rho,c}$.

\end{enumerate}
\end{proof}

Now we define a norm that indicates how close a map $\Phi$ is to the identity.

\begin{definition}
Let $x = (\phi,I) \in \mathbb{C}^{2n}$, then
$$|x|_c := \max(|\phi|_1, c|I|_\infty)$$
\end{definition}

\begin{definition}
For a map $\Upsilon :\mathcal{D}_\rho(G)\rightarrow \mathbb{C}^{2n}$ its norm and the norm of its derivative its defined as:
$$ |\Upsilon|_{G,\rho,c}:=\sup_{x\in\mathcal{D}_\rho(G)}|\Upsilon(x)|_c,$$
$$ |D\Upsilon|_{G,\rho,c}:=\sup_{x\in\mathcal{D}_\rho(G)}|D\Upsilon(x)|_c,$$

where $\displaystyle |D\Upsilon(x)|_c = \sup_{\substack{y\in\mathbb{R}^{2n} \\ |y|_c = 1}}|D\Upsilon(x)\cdot y|_c$
\end{definition}

\begin{lemma}
If $\Upsilon$ is analytic on $\mathcal{D}_\rho(G)$, then $|D\Upsilon|_{G,\rho-\delta,C} \leq \frac{|\Upsilon|_{G,\rho,c}}{\hat\delta_c}$
\end{lemma}


\begin{proof}
Observe that if we consider $\|.\|$ any norm on $\mathbb{C}^n$ and a matrix $A$ of size $n\times n$, and $\|A\|$ defines the induced norm of matrices  i.e.
$$\|A\| = \sup_{\substack{y\in\mathbb{C}^{2n} \\ \|y\| = 1}}\|A\cdot y\|$$
then one has that $\|(\|a_1\|',\ldots,\|a_n\|')\| \leq \|A\|$ where $a_j$ denotes the $j$-th row of $A$. Also note that $\|\cdot\|'$ can be a any norm consider the infinity norm. This can be easily proven in the following way:
$$\|A\cdot y\| = \left\|\left(\begin{array}{c}a_1\cdot y \\ \vdots \\ a_n\cdot y\end{array}\right)\right\| \leq \left\|\left(\begin{array}{c}
\|a_1\|'\|y\|' \\ \vdots \\ \|a_n\|'\|y\|'
\end{array}\right)\right\|$$
Where $\forall y \in \mathbb{C}^n \text{ such that } \|y\| = 1$.
Let $a_j$ be the rows of $D\Upsilon(x)$,
$$ a_j = \left(\frac{\partial \Upsilon_j}{\partial \phi},\frac{\partial \Upsilon j}{\partial I}\right),$$
and be $\|a_j\|'$ its norm.
With this property in mind we proceed as follows:
\begin{longtable}{rcl}
$|D\Upsilon|_{G,\rho-\delta,c}$ & $=$ & $\displaystyle\sup_{x\in \mathcal{D}_{\rho-\delta}(G)} |D\Upsilon(x)|_c$ \\
\\
& $\leq$ & $\displaystyle\sup_{x\in \mathcal{D}_{\rho-\delta}(G)} |(|a_1|_\infty,\ldots,|a_n|_\infty)|_c$\\
%\\
%& $\leq$ & $ \left|\left(\max\left(\sup_{x\in\mathcal{D}_{\rho-\delta}}\left|\frac{\partial \Upsilon_1}{\partial \phi}\right|_1, c\sup_{x\in\mathcal{D}_{\rho-\delta}}\left|\frac{\partial \Upsilon_1}{\partial I}\right|_\infty\right),\ldots\right.\right.$\\
%\\
%& & $\qquad\left.\left.\ldots,\max\left(\sup_{x\in\mathcal{D}_{\rho-\delta}}\left|\frac{\partial \Upsilon_{2n}}{\partial \phi}\right|_1, c\sup_{x\in\mathcal{D}_{\rho-\delta}}\left|\frac{\partial \Upsilon_{2n}}{\partial I}\right|_\infty\right)\right)\right|_c$\\
%\\
%& $=$ & $ \left|\left(\max\left(\left\|\frac{\partial \Upsilon_1}{\partial \phi}\right\|_{G,\rho-\delta,1}, c\left\|\frac{\partial \Upsilon_1}{\partial I}\right\|_{G,\rho-\delta,\infty}\right),\ldots\right.\right.$\\
%\\
%& & $\qquad\left.\left.\ldots,\max\left(\left\|\frac{\partial \Upsilon_{2n}}{\partial \phi}\right\|_{G,\rho-\delta,1}, c\left\|\frac{\partial \Upsilon_{2n}}{\partial I}\right\|_{G,\rho-\delta,\infty}\right)\right)\right|_c$\\

\\
& $\leq$ & $ \left|\left(\sup_{x\in \mathcal{D}_{\rho-\delta}}\left\|D \Upsilon_1\right\|_{\infty},\ldots,\sup_{x\in \mathcal{D}_{\rho-\delta}} \left\|D \Upsilon_{2n}\right\|_{\infty}\right)\right|_c$\\
\\
& $=$ & $ \left|\left(\left\|D \Upsilon_1\right\|_{G,\rho-\delta,\infty},\ldots,\left\|D \Upsilon_{2n}\right\|_{G,\rho-\delta,\infty}\right)\right|_c$\\
\\
& $\leq$ & $ \left|\left(\frac{1}{\delta_1}\left\| \Upsilon_1\right\|_{G,\rho},\ldots,\frac{1}{\delta_1}\left\| \Upsilon_{2n}\right\|_{G,\rho}\right)\right|_c$\\
%\\
%& $\leq$ & $ \left|\left(\max\left(\frac{c}{e\hat\delta_c}\left\| \Upsilon_1\right\|_{G,\rho}, \frac{c}{\hat\delta_c}\left\| \Upsilon_1\right\|_{G,\rho}\right),\ldots\right.\right.$\\
%\\
%& & $\qquad\left.\left.\ldots,\max\left(\frac{c}{e\hat\delta_c}\left\| \Upsilon_{2n}\right\|_{G,\rho}, \frac{c}{\hat\delta_c}\left\| \Upsilon_{2n}\right\|_{G,\rho}\right)\right)\right|_c$\\
%\\
%& $\leq$ & $\left|\frac{c}{\hat\delta_c}\left\|\Upsilon_1\right\|_{G,\rho},\ldots,\frac{c}{\hat\delta_c}\left\|\Upsilon_{2n}\right\|_{G,\rho}\right|_c$\\
\\
& $\leq$ & $\frac{1}{\hat\delta_c}\left|\left\|\Upsilon_1\right\|_{G,\rho},\ldots,\left\|\Upsilon_{2n}\right\|_{G,\rho}\right|_c$\\
\\
& $=$ & $\frac{1}{\hat\delta_c}\sup_{x\in\mathcal{D}_\rho(G)}\left|\Upsilon_1,\ldots,\Upsilon_{2n}\right|_c = \frac{1}{\hat\delta_c}\sup_{x\in\mathcal{D}_\rho(G)}\left|\Upsilon\right|_c$\\
\\
& $=$ & $ \frac{1}{\hat\delta_c}\left|\Upsilon\right|_{G,\rho,c}$\\

\end{longtable}


\end{proof}

\begin{lemma}\label{lemma:1.2} Let $W$ be an analytic function on $\mathcal{D}_\rho(G)$, $\rho > 0$ and let $\Phi_t$ be its Hamiltonian flow at time $t$ ($t>0$). Let $\delta=(\delta_1,\delta_2)>0$ and $c>0$ given. Assume that $\|DW\|_{G,\rho,c}\leq \hat\delta_c$. Then, $\Phi_t$ maps $\mathcal{D}_{\rho-t\delta}(G)$ into $\mathcal{D}_{\rho}(G)$ and one has:
\begin{enumerate}
\item $|\Phi_t-\id|_{G,\rho-t\delta,c}\leq t \|DW\|_{G,\rho,c}$,
\item $\Phi(\mathcal{D}_{\rho}(G)) \supset \mathcal{D}_{\rho-t\delta}(G)$ for $\rho'\leq \rho-t\delta$,
\item Assuming that $\|DW\|_{G,\rho,c} < \hat\delta_c/2e$, for any given function $f$ analytic on $\mathcal{D}_\rho(G)$, and for any integer $m\geq 0$, the following bound holds:
$$
\begin{array}{lcl}
\|r_m(f,W,t)\|_{G,\rho-t\delta} \\
 \leq  \displaystyle \sum_{l= 0}^\infty \left[\frac{1}{\binom{l+m}{m}}\cdot \left(\frac{2e\|DW\|_{G,\rho,c}}{\hat\delta_c}\right)^l\right]\frac{t^m}{m!}\|L_W^m f\|_{G,\rho}\\
\\
 =  \displaystyle\gamma_m\left(\frac{2e\|DW\|_{G,\rho,c}}{\hat\delta_c}\right)\cdot t^m \|L_W^m f\|_{G,\rho},\\
\end{array}
$$

where for $0\leq x \leq 1$ we define

$$\gamma_m (x) \coloneqq \sum_{l=0}^\infty \frac{l!}{(l+m)!} x^l$$

\end{enumerate}
\end{lemma}
\begin{proof}

During the proof we are going to denote $\Phi_s(\phi_0,I_0)$ by $(\phi(s),I(s))$.

%\begin{figure}
%\begin{center}
%\begin{tikzpicture}[line cap=round,line join=round,>=triangle 45,x=1.0cm,y=1.0cm,scale=0.4]
%\clip(0,-4.5) rectangle (8,4);
%\definecolor{qqqqff}{rgb}{0.,0.,1.}
%\definecolor{ccwwff}{rgb}{0.0,0.1,1.0}
%\definecolor{qqzzff}{rgb}{0.0,0.6,1.0}
%\definecolor{xdxdff}{rgb}{0.0,0.8,1.0}
%\draw [rotate around={47.7:(3.65,-0.2)},line width=1.2pt,color=xdxdff] (3.65,-0.2) ellipse (4.04cm and 2.85cm);
%\draw [rotate around={47.686:(3.65,-0.2)},line width=1.2pt,color=qqzzff] (3.65,-0.2) ellipse (3.5cm and 2.47cm);
%\begin{small}
%\draw [color=ccwwff] (3,-2.1)-- ++(-5pt,-5pt) -- ++(10pt,10pt) ++(-10pt,0) -- ++(10pt,-10pt);
%\draw[color=ccwwff] (3.26,-1.27) node {$(\phi_0,I_0)$};
%\draw[color=xdxdff] (5.74,3.9) node {$\mathcal{D}_{\rho}(G)$};
%\draw[color=qqzzff] (4.6,1.4) node {$\mathcal{D}_{\rho-t\delta}(G)$};
%\end{small}
%\end{tikzpicture}
%\caption{Diagram of the sets of the proof and the starting point of the Hamiltonian flow of $W$.}\label{fig:1}
%\end{center}
%\end{figure}


Let us find the coordinate expression of the hamiltonian flow for the expression \ref{eq:bm-symplectic} of a $b^m$-symplectic form. Recall that the equation for the hamiltonian flow is $\frac{d}{ds}\phi_i(s)=\{\phi_i,W\}$ and $\frac{d}{ds}I_i(s)=\{I_i,W\}$.
$$\{\phi_i,W\} = \frac{1}{\left(\sum_{j=1}^m \frac{c_j}{I_1^j}\right)}\left(\frac{\partial \phi_i}{\partial I_1}\cdot\frac{\partial W}{\partial \phi_1} - \frac{\partial \phi_i}{\partial \phi_1}\cdot\frac{\partial W}{\partial I_1}\right)$$ 
$$+ \sum_{j=2}^n\left(\frac{\partial \phi_i}{\partial I_j}\cdot\frac{\partial W}{\partial \phi_j} - \frac{\partial \phi_i}{\partial \phi_j}\cdot\frac{\partial W}{\partial I_j}\right).$$
Hence,
$$
\displaystyle \frac{d}{ds}\phi_i(s) = -\frac{1}{\left(\sum_{j=1}^m \frac{c_j}{I_1^j}\right)}\frac{ \partial W}{\partial I_1} \text{ if } i = 1 \text{ and }
\displaystyle \frac{d}{ds}\phi_i(s) = -\frac{ \partial W}{\partial I_i} \text{ if } i \neq 1.
$$
On the other side,
$$\{I_i,W\} = \frac{1}{\left(\sum_{j=1}^m \frac{c_j}{I_1^j}\right)}\left(\frac{\partial I_i}{\partial I_1}\cdot\frac{\partial W}{\partial \phi_1} - \frac{\partial I_i}{\partial \phi_1}\cdot\frac{\partial W}{\partial I_1}\right) $$
$$+ \sum_{j=2}^n\left(\frac{\partial I_i}{\partial I_j}\cdot\frac{\partial W}{\partial \phi_j} - \frac{\partial I_i}{\partial \phi_j}\cdot\frac{\partial W}{\partial I_j}\right).$$
Hence,
$$
\frac{d}{ds}I_i(s) = \frac{1}{\left(\sum_{j=1}^m \frac{c_j}{I_1^j}\right)}\frac{ \partial W}{\partial \phi_1} \text{ if } i = 1 \text{ and } \frac{d}{ds}I_i(s) = \frac{ \partial W}{\partial \phi_i} \text{ if } i \neq 1.
$$

\begin{enumerate}
\item

Assume now that $0<s_0 \leq t$. Then,

\begin{longtable}{rcl}
$|\phi(s_0)-\phi_0|_\infty $ & $\leq $ & $s_0\sup_{0<s\leq s_0}|\phi'(s)|_\infty $ \\
& $=$ & $s_0\sup_{0<s\leq s_0}\left(\max(|\phi_1'(s)|,\ldots,|\phi_n'(s)|)\right) $ \\
\\
& $=$ & $\displaystyle s_0\sup_{0<s\leq s_0}\left(\max\left(\left|\frac{1}{\left(\sum_{j=1}^m\frac{c_j}{I_1^j}\right)}\frac{\partial W}{\partial I_1}\right|,\left|\frac{\partial W}{\partial I_2}\right|,\ldots,\left|\frac{\partial W}{\partial I_n}\right|\right)\right)$ \\
\\
& $\leq$ & $s_0\sup_{0<s\leq s_0}\left|\frac{\partial W}{\partial I}\right|_\infty  \leq s_0\left\|\frac{\partial W}{\partial I}\right\|_{G,\rho,\infty}$ \\


\end{longtable}
Where we have used again that on the domain $\mathcal{D}_\rho(G)$ the inequality $\left|\sum_{j=1}^m\frac{c_j}{I_1^j}\right| \geq 1$ holds.
%The last inequality is true if because $s_0\leq t$ then for $0<s\leq s_0$ the flow stays in $\mathcal{D}_\rho(G)$.
Similarly, $|I(s_0) - I_0| \leq s_0\|\frac{\partial W}{\partial \phi}\|_{G,\rho,1}$, and hence $|\Phi_t - \id|_{G,\rho-t\delta,c} \leq t\|D W\|_{G,\rho,c}$.

Because
\begin{equation}\label{eq:bound_flow}
\begin{array}{lr}
|\phi(s)-\phi_0|_\infty \leq t \|\frac{\partial W}{\partial I}\|_{G,\rho,\infty} \leq t \frac{\hat\delta_c}{c}\leq t\delta_1\frac{c}{c} = t\delta_1 & \forall 0 < s \leq s_0\\
|I(s)-I_0|_1 \leq t \|\frac{\partial W}{\partial \phi}\|_{G,\rho,1} \leq t \hat\delta_c\leq t\delta_2 & \forall 0 < s \leq s_0\\
\end{array}
\end{equation}


hence, $(\phi(s),I(s)) \in \mathcal{D}_{\rho-t\delta + t\delta}(G) = \mathcal{D}_\rho(G)$ for all $0 < s \leq s_0$.
\item

Repeat the same argument as in \ref{eq:bound_flow} with $\phi(-s)$. If $(\phi_0,I_0)\in \mathcal{D}_{\rho'-t\delta}$, then $(\phi(-s),I(-s)) \in \mathcal{D}_{\rho'-t\delta+t\delta}(G) = \mathcal{D}_{\rho'}$.
Hence, $$\mathcal{D}_{\rho'}(G) \supset \Phi^{-1}(\mathcal{D}_{\rho'-t\delta}(G)),$$ then $\Phi(\mathcal{D}_{\rho'}(G)) \supset \mathcal{D}_{\rho'-t\delta}(G)$.

\item
Consider $f$ an analytical function.
By the previous construction $f\circ\Phi_t$ is defined in $\mathcal{D}_{\rho -t\delta}(G)$.
Because $W$ is analytic we also have that $f\circ\Phi_t$ is analytic and we can expand its Lie series.
Let $m \in \mathbb{Z}, l \geq m+1, j = m+1,\ldots,l$ then
$$
\begin{array}{rcl}
\|L_W^jf\|_{G,\rho-(j-m)t\eta} & \leq & \frac{2}{c}\|D(L_W^{j-1} f)\|_{G,\rho-(j-m)t\eta,c}\|DW\|_{G,\rho,c}\\
 & \leq & \frac{2}{t\hat\eta_c}\|L_W^{j-1} f\|_{G,\rho-(j-1-m)t\eta}\|D W\|_{G,\rho,c},\\
\end{array}
$$

where we used lemma \ref{lemma:1.1} and defined $\eta = \frac{\delta}{(l-m)}$ and $\hat\eta_c = \min(c\eta_1,\eta_2)$.

Then,

$$\begin{array}{rcl}
\|L_W^l f\|_{G,\rho-t\delta} & \leq & \left(\frac{2\|DW\|_{G,\rho,c}}{t\hat\eta_c}\right)^{l-m}\\
&\leq& e^{l-m}\cdot(l-m)!\left(\frac{2\|DW\|_{G,\rho,c}}{\hat\delta_c}\right)^{l-m}\|L_W^m f\|_{G,\rho},
\end{array}$$

where we used that $\hat\eta_c = \frac{\hat\delta_c}{l-m}$ and $(l-m)^{(l-m)}\leq e^{l-m}\cdot(l-m)!$
And hence, the bound for $\|r_m(f,W,t)\|_{G,\rho-t\delta}$ is

$$\sum_{l=m}^\infty \frac{t^l}{l!}\|L_W^l f\|_{G,\rho-t\delta} \leq \left[\sum_{l= m}^\infty \frac{(l-m)!}{l!}\left(\frac{2e \|DW\|_{G,\rho,c}}{\hat\delta_c}\right)^{l-m}\right]\cdot t^m \|L_W^m f\|_{G,\rho}$$

and this series converges if $\|DW\|_{G,\rho,c} \leq \frac{\hat\delta_c}{2e}$.

\end{enumerate}
\end{proof}

\begin{theorem}\label{lemma:iterative}[Iterative Lemma] $H(\phi,I) = \hat h(I) + R(\phi,I)$ where $\hat h (I)$ is as in equation \ref{eq:bm-hamiltonian} defined on $\mathcal{D}_\rho(G)$. Let $\hat u = \frac{\partial \hat h }{\partial I}$ and $u = \frac{\partial h }{\partial I}$, and assume $u$ is $\alpha,K,c,\hat q$-non-resonant. Assume that $\left|\frac{\partial}{\partial I } u\right|_{G,\rho_2} \leq M'$. Let $\delta < \rho$ and $c > 0$, $A = 1 + \frac{2Mc}{\alpha}$. Assume that $\rho_2 \leq \frac{\alpha}{2MK}$, $\|DR\|_{G,\rho,c} \leq \frac{\alpha \hat \delta_c}{74A}$.
Then, there exists a real analytic map $\Phi:\mathcal{D}_{\rho-\frac{\delta}{2}}(G) \rightarrow \mathcal{D}_\rho(G)$, such that $H\circ \Phi = \hat h + \tilde{R}$,with:

\begin{enumerate}
\item $\|D\tilde{R}\|_{G,\rho-\delta,c} \leq e^{-K\delta_1}\|DR\|_{G,\rho,c} + \frac{14A}{\alpha\hat\delta_c} \|DR\|^2_{G,\rho,c}$,
\item $|\Phi-\id|_{G,\rho-\frac{\delta}{2},c} \leq \frac{2A}{\alpha}\|DR\|_{G,\rho,c}$,
\item $\Phi(\mathcal{D}_{\rho'}(G)) \supset \mathcal{D}_{\rho'-\frac{\delta}{2}}(G)$ for $\rho'\leq \rho-\frac{\delta}{2}$
\end{enumerate}

\end{theorem}

\begin{proof}
Recall that $\left|\frac{\partial}{\partial I } u\right|_{G,\rho_2} \leq M'$  implies $\left|\frac{\partial}{\partial I}( \bar{\mathcal{B}}u + \bar{\mathcal{A}})\right|_{G,\rho_2}\leq M$ by equation \ref{eq:M_def}.
By equation \ref{eq:iterative_h_and_R}

$$R^{(q)} = R^{(q-1)}_{>K} + r_2(\hat h^{(q-1)}, W^{(q)},1) + r_1(R^{(q-1)},W^{(q)},1).$$

To simplify the notation we are going to omit the index of the iteration:

\begin{equation}\label{eq:R_tilde}
\tilde R = R_{>K} + r_2(\hat h^, W,1) + r_1(R,W,1).
\end{equation}

Where $W$ is defined in terms of its Fourier expressions by equation \ref{eq:solve_coefs_simplified}:

$$
W_k(I) = \frac{R_k(I)}{i(k \bar{\mathcal{B}}(I_1) u + k \bar{\mathcal{A}}(I_1))}
$$

By proposition \ref{prop:14}: $\|DW\|_{G,\rho,c}\leq \frac{2A}{\alpha}\|DR\|_{G,\rho,c} \leq \frac{2A}{\alpha}\frac{\alpha\hat\delta_c}{74A} = \frac{\hat\delta_c}{37}$.
And $\Phi$ is defined as in lemma \ref{lemma:1.1}: $\Phi:\mathcal{D}_{\rho-\frac{\delta}{2}}(G) \rightarrow \mathcal{D}_{\rho}(G)$.
\begin{enumerate}
\item

Differentiating equation \ref{eq:R_tilde} we obtain:
$$D \tilde R = D R_{>K} + D r_2(\hat h, W,1) + D r_1(R,W,1).$$
Taking norms at every side of the expression:
\begin{longtable}{rcl}
$\|D\tilde R\|_{G,\rho-\delta,c}$ & $=$ & $\|D R_{>K} + D r_2(\hat h, W,1) + D r_1(R,W,1)\|_{G,\rho-\delta,c}$\\
& $\leq$ & $\|D R_{>K}\|_{G,\rho-\delta,c} + \|D r_2(\hat h, W,1)\|_{G,\rho-\delta,c}$ \\
& & $ + \|D r_1(R,W,1)\|_{G,\rho-\delta,c}$\\
& $\leq$ & $e^{-K\delta_1}\|DR\|_{G,\rho,c} $\\
& & $+ \frac{2c}{\hat\delta_c}\left(\|r_2(\hat h, W,1)\|_{G,\rho-\frac{\delta}{2},c} + \|r_1(R,W,1)\|_{G,\rho-\frac{\delta}{2},c}\right)$
\end{longtable}


Let us further develop the two last terms of the previous expression, by using lemma \ref{lemma:1.2}:

$$
\begin{array}{rcl}
 \|r_2(\hat h, W,1)\|_{G,\rho-\frac{\delta}{2},c} & \leq & \gamma_2\left(\frac{2e\|DW\|_{G,\rho,c}}{\hat\delta_c/2}\right)\|L_W^2 h\|_{G,\rho}\\
 & \leq & \gamma_2\left(\frac{4e\|DW\|_{G,\rho,c}}{\hat\delta_c}\right)\|\{\{h,W\},W\}\|_{G,\rho},\\
\end{array}
$$


$$
\begin{array}{rcl}
 \|r_1(\hat h, W,1)\|_{G,\rho-\frac{\delta}{2},c} & \leq & \gamma_1\left(\frac{2e\|DW\|_{G,\rho,c}}{\hat\delta_c/2}\right)\|L_W^1 R\|_{G,\rho}\\
 & \leq & \gamma_1\left(\frac{4e\|DW\|_{G,\rho,c}}{\hat\delta_c}\right)\|\{R,W\}\|_{G,\rho}.\\
\end{array}
$$

Then, using the second statement of lemma \ref{lemma:1.1} and that $\{W,h\} = R_{\leq K}$:
$$\|\{R,W\}\|_{G,\rho}\leq\frac{2}{c}\|DR\|_{G,\rho,c}\|DW\|_{G,\rho,c}, \text{ and }$$
$$\begin{array}{rcl}
|\{\{h,W\},W\}\|_{G,\rho} & = & \displaystyle\|\{R_{\leq K},W\}\|_{G,\rho} \\
\\
&\leq& \displaystyle\frac{2}{c}\|DR_{\leq K}\|_{G,\rho,c}\|DW\|_{G,\rho,c} \\
\\
&\leq& \displaystyle\frac{2}{c}\|DR\|_{G,\rho,c}\|DW\|_{G,\rho,c}.
\end{array}$$

Moreover, it is easy to see that $\gamma_1(x) = \frac{-\log(1-x)}{x}$ and $\gamma_2(x) = \frac{x + (1-x)\log(1-x)}{x^2}$. Observe that these functions are monotonously increasing in $x$. Recall that $\|DW\|_{G,\rho,c} \leq \frac{2A}{\alpha}\|DR\|_{G,\rho,c}$. Then,

\begin{longtable}{rcl}
$\|r_1(\hat h, W,1)\|_{G,\rho-\frac{\delta}{2},c} $\\
$ \qquad +  \|r_2(\hat h, W,1)\|_{G,\rho-\frac{\delta}{2},c}$ & $\leq$ & $\gamma_1\left(\frac{4e\|DW\|_{G,\rho,c}}{\hat\delta_c}\right)\|\{R,W\}\|_{G,\rho}$\\
& & $+ \gamma_2\left(\frac{4e\|DW\|_{G,\rho,c}}{\hat\delta_c}\right)\|\{\{h,W\},W\}\|_{G,\rho}$\\
& $\leq$ & $\gamma_1\left(\frac{4e\|DW\|_{G,\rho,c}}{\hat\delta_c}\right)\frac{2}{c}\|DR\|_{G,\rho,c}\|DW\|_{G,\rho,c}$\\
& & $+ \gamma_2\left(\frac{4e\|DW\|_{G,\rho,c}}{\hat\delta_c}\right)\frac{2}{c}\|DR\|_{G,\rho,c}\|DW\|_{G,\rho,c}$\\
& $\leq$ & $\gamma_1\left(\frac{4e\|DW\|_{G,\rho,c}}{\hat\delta_c}\right)\frac{2}{c}\frac{2A}{\alpha}\|DR\|^2_{G,\rho,c}$\\
& & $+ \gamma_2\left(\frac{4e\|DW\|_{G,\rho,c}}{\hat\delta_c}\right)\frac{2}{c}\frac{2A}{\alpha}\|DR\|^2_{G,\rho,c}$\\
& $\leq$ & $\frac{2}{c}[\gamma_1(\frac{4e}{37}) + \gamma_2(\frac{4e}{37})]\frac{2A}{\alpha}\|DR\|^2_{G,\rho,c}$\\
& $=$ & $\frac{4A}{\alpha c}[\gamma_1(\frac{4e}{37}) + \gamma_2(\frac{4e}{37})]\|DR\|_{G,\rho,c}^2$.\\
\end{longtable}

Moreover $\gamma_1(\frac{4e}{37}) + \gamma_2(\frac{4e}{37}) \approx 1.741\ldots < \frac{7}{4}$.

Then,

$$\begin{array}{rcl}
\|D\tilde R\|_{G,\rho-\delta,c} & \leq & e^{-K\delta_1}\|DR\|_{G,\rho,c} + \frac{2c}{\hat\delta_c}\frac{4A}{\alpha c}\frac{7}{4}\|\|_{G,\rho,c}^2\\
 & \leq & e^{-K\delta_1}\|DR\|_{G,\rho,c} + \frac{14A}{\hat\delta_c\alpha}\|DR\|^2_{G,\rho,c},\\
\end{array}
$$

as we wanted to prove.

\item Direct from lemma \ref{lemma:1.2}:
$$|\Phi - \id|_{G,\rho.\frac{\delta}{2},c}\leq\|DW\|_{G,\rho,c}\leq\frac{2A}{\alpha}\|DR\|_{G,\rho,c}$$
\item Also direct from lemma \ref{lemma:1.2}:
$$\Phi(\mathcal{D}_\rho(G)) \supset \mathcal{D}_{\rho'-\frac{\delta}{2}}(G), \text{ for } \rho' \leq \rho - \delta/2$$
\end{enumerate}
\end{proof}


\begin{definition} $\Delta_{c,\hat q}(k,\alpha) = \{J \in \mathbb{R}^n \text{ such that } |k \bar{\mathcal{B}}(I_1) J + k \bar{\mathcal{A}}(I_1)| < \alpha\}$
\end{definition}

\begin{lemma}\label{lemma:mesure_resonances}With the previous definitions we have the following bounds.

Outside of $Z$:
$$\text{meas}\left(F \cap \Delta_{c,\hat q}(k,\alpha)\right) \leq (\textnormal{diam} F)^{n-1}\frac{2\alpha}{|k|_{2,\omega}}.$$
At $Z$:
$$\text{meas}\left(F \cap \Delta_{c,\hat q}(k,\alpha)\right)
\left\{
\begin{array}{lcl}
=0 & & \text{if } \alpha \leq \frac{|k_1|}{\mathcal{K}'}\\
\leq (\text{diam} F)^n & & \text{if } \alpha > \frac{|k_1|}{\mathcal{K}'} \\
\end{array}
\right.
$$
\end{lemma}

\begin{proof}
It is important to understand the geometry of the set $\Delta_{c,\hat q}(k,\alpha)$. Recall that $k \bar{\mathcal{B}}(I_1) J = k_1 \mathcal{B}(I_1) J_1 + \bar{k}\bar{J}$, hence this part of the expression can be interpreted as the scalar product of the vector $J$ with the vector $(k_1 \mathcal{B}(I_1),k_2,\ldots,k_n)$. Then the set $\{J \in \mathbb{R}^n \text{ such that } |k \bar{\mathcal{B}}(I_1) J| < \alpha\}$ is the space between two hyperplanes orthogonal to $(k_1 \mathcal{B}(I_1),k_2,\ldots,k_n)$. Adding the term $k \bar{\mathcal{A}}(I_1)$ only applies a transition to the previous set. Let us find what is the separation between the hyperplanes.
Assume $J$ is parallel to $(k_1 \mathcal{B}(I_1),k_2,\ldots,k_n)$ with lengths $a$:
$$J = a\frac{(k_1 \mathcal{B}(I_1),k_2,\ldots,k_n)}{|k|_{2,\omega}},$$
where $|k|_{2,\omega} = \sqrt{B(I_1)^2k_1^2 + k_2^2 + \ldots k_n^2}$.
Then,
$$
\begin{array}{rcl}
 J\cdot (B(I_1),k_1,\ldots,k_n) & = & c(B(I_1)k_1^2 + k_2^2 + \ldots k_n^2)\frac{1}{|k|_{2,\omega}}\\
 & = & a|k|_{2,\omega} \leq \alpha \Leftrightarrow a \leq \frac{\alpha}{|k|_{2,\omega}}.
\end{array}
$$
And finally,
$$\text{meas}\left(F \cap \Delta_{c,\hat q}(k,\alpha)\right) \leq (\textnormal{diam} F)^{n-1}\frac{2\alpha}{|k|_{2,\omega}}.$$
\begin{figure}[h!]
\begin{center}
%\definecolor{magenta}{rgb}{0.,0.2,0.6}
\begin{tikzpicture}[line cap=round,line join=round,>=triangle 45,x=1.0cm,y=1.0cm, scale = 0.6]
\clip(-6.3,-4.3) rectangle (8,6.3);
\fill[line width=1.pt,color=cyan,fill=cyan,fill opacity=0.8] (-6.305315071273844,2.143755827017319) -- (-5.80474032812592,2.817028856551277) -- (1.9374647098374522,-2.939257417027438) -- (1.4368899666895276,-3.612530446561396) -- cycle;
\fill[line width=1.pt,color=magenta,fill=magenta,fill opacity=0.8] (-4.838555792187489,4.116547057388468) -- (-4.318304103434274,4.8162855787615415) -- (3.4239009345290983,-0.9400006948171737) -- (2.9036492457758833,-1.639739216190248) -- cycle;
\draw [line width=1.pt,domain=-6.7128271268301205:6.518034851701254] plot(\x,{(--5.7156-4.*\x)/5.38});
\draw [line width=1.pt,domain=-6.7128271268301205:6.518034851701254] plot(\x,{(--12.088--5.38*\x)/4.});
\draw [->,line width=1.pt] (-0.9382935360133854,1.7599951940619964) -- (0.4,3.56);
\draw [line width=1.pt,color=cyan] (-6.305315071273844,2.143755827017319)-- (-5.80474032812592,2.817028856551277);
\draw [line width=1.pt,color=cyan] (-5.80474032812592,2.817028856551277)-- (1.9374647098374522,-2.939257417027438);
\draw [line width=1.pt,color=cyan] (1.9374647098374522,-2.939257417027438)-- (1.4368899666895276,-3.612530446561396);
\draw [line width=1.pt,color=cyan] (1.4368899666895276,-3.612530446561396)-- (-6.305315071273844,2.143755827017319);
\draw [line width=1.pt,color=magenta] (-4.838555792187489,4.116547057388468)-- (-4.318304103434274,4.8162855787615415);
\draw [line width=1.pt,color=magenta] (-4.318304103434274,4.8162855787615415)-- (3.4239009345290983,-0.9400006948171737);
\draw [line width=1.pt,color=magenta] (3.4239009345290983,-0.9400006948171737)-- (2.9036492457758833,-1.639739216190248);
\draw [line width=1.pt,color=magenta] (2.9036492457758833,-1.639739216190248)-- (-4.838555792187489,4.116547057388468);
\draw [line width=1.pt] (-6.055027699699883,2.4803923417842997)-- (1.687177338263491,-3.275893931794417);
\draw [line width=1.pt] (-4.578429947810884,4.466416318075007)-- (3.163775090152491,-1.2898699555037108);
\draw [->,line width=.5pt] (-1.3994012200506503,-0.9810399850924533) -- (0.07719653183835051,1.004983991198252);
\draw [->,line width=.5pt] (0.07719653183835062,1.0049839911982523) -- (-1.3994012200506503,-0.9810399850924532);
\begin{small}
\draw[color=black] (3,2.7) node {$(\mathcal{B}(I_1)k_1,k_2,\ldots,k_n)$};
\draw[color=cyan] (4.5,-4) node {$|k \bar{\mathcal{B}}(I_1) J + k \bar{\mathcal{A}}(I_1)| < \alpha$};
\draw[color=magenta] (5.3,-1.4) node {$|k \bar{\mathcal{B}}(I_1) J| < \alpha$};
\draw[color=black] (0.4,-0.8) node {$-k\bar{\mathcal{A}}(I_1)$};
\end{small}
\end{tikzpicture}
\end{center}
\caption{Graphical representation of the set $\Delta_{c,\hat q}(\alpha)$}

\end{figure}
The previous formula can not be applied if when we are at $Z$ and $k = (k_1,0,\ldots,0)$. At $Z$, 
$$\Delta_{c,\hat q}(K,\alpha) = \{J \in \mathbb{R}^n \text{ such that } |\bar K \bar J + k_1\frac{\hat q_m}{c_m}| < \alpha\}.$$
And if $k=(k_1,0,\ldots,0)$ then 
$$\Delta_{c,\hat q}(K,\alpha) = \{J \in \mathbb{R}^n \text{ such that } |k_1 \frac{\hat q_m}{c_m}| < \alpha\}.$$
Then

$$
\Delta_{c,\hat q}(k,\alpha) =
\left\{
\begin{array}{rcl}
\mathbb{R}^n & & \text{ if } |k_1| < \alpha\frac{c_m}{\hat q_m} = \alpha \mathcal{K}',\\
\{\emptyset\} & & \text{ if } |k_1| \geq \alpha\frac{c_m}{\hat q_m} = \alpha \mathcal{K}'.
\end{array}
\right.
$$

Using this last identity, the statement we wanted to prove is immediate.

\end{proof}

\begin{definition}
$G-b := \{I\in G \text{ such that } \mathcal{U}_b(I)\subset G\}$, where $\mathcal{U}_b(I)$ is the ball of radius $b$ centered at $I$.
\end{definition}
\begin{definition}
$F$ is a $D$-set if $\textnormal{meas}[(F-b_1)\setminus (F-b_2)] \leq D(b_2 - b_1)$.
\end{definition}

\begin{lemma}\label{lemma:measures_nonresonant}
Let $F \subset \mathbb{R}^n$ be a $D$-set for $d \geq 0$, $\tau > 0$, $\beta \geq 0$ and $k \geq 0$ an integer. Consider the set
$$F(d,\beta,K) := (F-d)\setminus \bigcup_{\substack{k\in\mathbb{Z}^n\setminus\{0\} \\ |k|_1 \leq K}} \Delta_{c,\hat q}\left(k,\frac{\beta}{|k|_1^\tau}\right).$$
Then, outside of $Z$:
\begin{enumerate}
\item\label{eq:meas_nonr_1} If $d'\geq d$,  $\beta'\geq \beta$, $k' \geq k$, then
$$
\textnormal{meas}[F(d,\beta,k)\setminus F(d',\beta',k')] \leq
 $$
$$ D(d' - d) + 2(\textnormal{diam}F)^{n-1}
\left(\sum_{\substack{k\in\mathbb{Z}^n\setminus\{0\}
\\ |k|_1 \leq K}}\frac{\beta' - \beta}{|k|_1^\tau|k|_{2,\omega}}
 + \sum_{\substack{k\in\mathbb{Z}^n\setminus\{0\} \\ 0 < |k|_1 \leq K}}
\frac{\beta'}{|k|_1^\tau|k|_{2,\omega}}\right)
$$
\item\label{eq:meas_nonr_2} For every $b \geq 0$
$$\textnormal{meas}[F(d,\beta,K)\setminus(F(d,\beta,K)-b)]\leq (D + 2^{n+1}(\dim F)^{n-1}K^n)b$$
\end{enumerate}
And inside of $Z$, if we assume $\beta \leq \frac{1}{\mathcal{K}'}$, the equation \ref{eq:meas_nonr_1} holds adding only the terms $\bar k \neq 0$ and \ref{eq:meas_nonr_2} holds without any change.
\end{lemma}



\begin{proof}
Recall that
$$\Delta_{c,\hat q}\left(k,\frac{\beta}{|k|_1^\tau}\right) = \left\{J\in\mathbb{R}^n \text{ such that } \left|k\bar{\mathcal{B}}(I_1)J + k\bar{\mathcal{A}}(I_1)\right| < \frac{\beta}{|k|_1^\tau}\right\}.$$

First we will prove the results outside of $Z$ and then
\begin{enumerate}
\item Let us expand the expression of $\textnormal{meas}[F(d,\beta,k)\setminus F(d',\beta',k')]$:
$$\left[(F-d)\setminus \bigcup_{\substack{k\in\mathbb{Z}^n\setminus\{0\} \\ |k|_1 \leq K}} \Delta_{c,\hat q}\left(k,\frac{\beta}{|k|_1^\tau}\right)\right]\setminus\left[(F-d)\setminus \bigcup_{\substack{k\in\mathbb{Z}^n\setminus\{0\} \\ |k|_1 \leq K}} \Delta_{c,\hat q}\left(k,\frac{\beta}{|k|_1^\tau}\right)\right].$$
Now we use the following property on the previous expression:
$$\begin{array}{rcl}
(A\setminus B)\setminus(C\setminus D) &  = &  [(A\setminus B)\setminus C]\cup [(A\setminus B)\cap D]\\
 & \subset & (A\setminus C)\cup[(A\setminus B)\cap D] \\
 & = & (A\setminus C)\cup(A\cap(D\setminus B)),
\end{array}
$$
where the last equality holds true because $D \supset B$. Using this property we have that $\textnormal{meas}[F(d,\beta,k)\setminus F(d',\beta',k')]$  is included in

$$
\begin{array}{rcl}

[(F-d)\setminus(F-d')]&\cup&\displaystyle\left[(F-d)\cap\left[\left(\bigcup_{\substack{k\in\mathbb{Z}^n\setminus\{0\} \\ |k|_1 \leq K'}} \Delta_{c,\hat q}\left(k,\frac{\beta'}{|k|_1}\right)\right)\right.\right.\\
\\
& & \displaystyle \qquad \qquad \qquad
\setminus\left.\left.\left(\bigcup_{\substack{k\in\mathbb{Z}^n\setminus\{0\} \\ |k|_1 \leq K}} \Delta_{c,\hat q}\left(k,\frac{\beta}{|k|_1}\right) \right)\right]\right].

\end{array}
$$

And this expression is equivalent to:

$$
\begin{array}{rcl}
[(F-d)\setminus (F-d')] & \cup &  \displaystyle \bigcup_{\substack{k\in\mathbb{Z}^n\setminus\{0\} \\ |k|_1 \leq K}} \left((F-d)\cap\left(\Delta_{c,\hat q}\left(k,\frac{\beta'}{|k|_1^\tau}\right)
\right.\right.
\\
& & \displaystyle \qquad \qquad \qquad \qquad \qquad \left.\left.
\setminus\Delta_{c,\hat q}\left(k,\frac{\beta}{|k|_1^\tau}\right)\right)\right)\\
\\
 & \cup & \displaystyle \bigcup_{\substack{k\in\mathbb{Z}^n\setminus\{0\} \\ K < |k|_1 \leq K'}} \left((F-d)\cap\Delta_{c,\hat q}\left(k,\frac{\beta'}{|k|_1^\tau}\right)\right).
\end{array}
$$

Now, using lemma \ref{lemma:mesure_resonances} we obtain:

$$
\textnormal{meas}(F(d,\beta,K)\setminus F(d',\beta',K')) \leq
$$
$$
\leq D(d'-d) + (\textnormal{diam}F)^{n-1}\left(\sum_{\substack{k\in\mathbb{Z}^n\setminus\{0\} \\ |k|_1 \leq K}}\frac{2(\beta'-\beta)}{|k|_1^\tau|k|_{2,\omega}}+ \sum_{\substack{k\in\mathbb{Z}^n\setminus\{0\} \\ K < |k|_1 \leq K'}}\frac{2\beta'}{|k|_1^\tau|k|_{2,\omega}}\right)
$$

\item
Observe that:


\begin{longtable}{rcl}
 $F(d,\beta,K) - b$ & $=$& $\displaystyle\left[(F-d) \setminus \bigcup_{\substack{k\in\mathbb{Z}^n\setminus\{0\} \\ |k|_1 \leq K}} \Delta_{c,\hat q} \left(k,\frac{\beta}{|k|_1^\tau}\right)\right] - b$\\
 & $\supset$ & $\displaystyle (F - (d+b))\setminus \bigcup_{\substack{k\in\mathbb{Z}^n\setminus\{0\} \\ |k|_1 \leq K}} \Delta_{c,\hat q} \left(k,\frac{\beta}{|k|_1^\tau} + b|k|_{2,\omega}\right).$ \\
\end{longtable}

Then,

\begin{longtable}{rcl}
& & $\textnormal{meas}[(F(d,\beta,K))\setminus(F(d,\beta,K)-b)]$ \\
& $\leq$ & $ \textnormal{meas}
\left[\left((F-d)\setminus\bigcup_{\substack{k\in\mathbb{Z}^n\setminus\{0\} \\ |k|_1 \leq K}}\Delta_{c,\hat q} \left(k,\frac{\beta}{|k|_1^\tau}\right)\right)\setminus
\right.$\\
& & \qquad $ \left.
\left((F-(d+b))\setminus\bigcup_{\substack{k\in\mathbb{Z}^n\setminus\{0\} \\ |k|_1 \leq K}}\Delta_{c,\hat q} \left(k,\frac{\beta}{|k|_1^\tau}\right)\right)\right]$ \\
& $\leq$ & $ \textnormal{meas}\left[(F-d)\setminus(F-(d+b))\cup \right.
$\\
& & \qquad $ \left.
\bigcup_{\substack{k\in\mathbb{Z}^n\setminus\{0\} \\ |k|_1 \leq K}}\left((F-d)\cap\left(\Delta_{c,\hat q} \left(k,\frac{\beta}{|k|_1^\tau}+ b|k|_{2,\omega}\right)\right)\right)\setminus\left(\Delta_{c,\hat q} \left(k,\frac{\beta}{|k|_1^\tau}\right)\right)\right]$  \\
& $\leq$ & $ Db + \sum_{\substack{k\in\mathbb{Z}^n\setminus\{0\} \\ |k|_1 \leq K}} (\textnormal{diam} F)^{n-1}\frac{2b|k|_{2,\omega}}{|k|_{2,\omega}}$  \\
& $\leq$ & $ Db + 2^n K^n(\textnormal{diam} F)^{n-1}\cdot 2 = Db + 2^{n+1}K^n(\textnormal{diam} F)^{n-1}$,  \\
\end{longtable}


where in the last inequality we used that the number of vectors $k$ such that $|k|_1\leq K$ is less or equal than $2^n K^n$.

\end{enumerate}

The previous identities worked outside of $Z$. Let us understand the set $F(d,\beta,K)$ when we are ate $Z$.
\begin{longtable}{rcl}
$F(d,\beta,K)$ & $:=$ & $(F-d)\setminus \bigcup_{\substack{k\in\mathbb{Z}^n\setminus\{0\} \\ |k|_1 \leq K}} \Delta_{c\hat q}(k,\frac{\beta}{|k|_1^\tau})$ \\
 & $=$ & $(F-d)\setminus \left[\left(\bigcup_{\substack{k\in\mathbb{Z}^n\setminus\{0\} \\ |k|_1 \leq K \\ \bar k \neq 0}} \Delta_{c\hat q}(k,\frac{\beta}{|k|_1^\tau})\right) \right. $\\
 & & $\quad \cup \left. \left(\bigcup_{\substack{k\in\mathbb{Z}^n\setminus\{0\} \\ |k|_1 \leq K \\ \bar k = 0}} \Delta_{c\hat q}(k,\frac{\beta}{|k|_1^\tau})\right)\right]$ \\
 & $=$ & $(F-d)\setminus \left[\left(\bigcup_{\substack{k\in\mathbb{Z}^n\setminus\{0\} \\ |k|_1 \leq K \\ \bar k \neq 0}} \Delta_{c\hat q}(k,\frac{\beta}{|k|_1^\tau})\right)  \right. $\\
 & & $\quad \cup \left. \left(\bigcup_{\substack{k_1\in\mathbb{Z}\setminus\{0\} \\ |k|_1 \leq \frac{\beta}{|k_1|^\tau}\mathcal{K}'} } \mathbb{R}^n \right)\right]$.\\
\end{longtable}
Note that if for some $k_1 \in \mathbb{Z}\setminus \{0\}$, $|k|_1 \geq \frac{\beta}{|k|_1^\tau}\mathcal{K}'$, we take out all the possible frequencies.
Then seems natural to ask $|k|_1 \geq \frac{\beta}{|k|_1^\tau}\mathcal{K}'$ for all $k_1 \in \mathbb{Z}\setminus\{0\}$, which holds if and only if $|k_1|^{1+\tau} \geq \beta K'$ for all $k_1 \in \mathbb{Z}\setminus\{0\}$ or simply $\beta\leq \frac{1}{\mathcal{K}'}$ which we assumed. Then
$$F(d,\beta,K) := (F-d)\setminus \bigcup_{\substack{k\in\mathbb{Z}^n\setminus\{0\} \\ |k|_1 \leq K \\ \bar k \neq 0}} \Delta_{c\hat q}(k,\frac{\beta}{|k|_1^\tau}).$$
Hence we can replicate the proof of \ref{eq:meas_nonr_1} only with the terms $\bar k \neq 0$. And the bound of \ref{eq:meas_nonr_2} can be slightly improved by using that the number of vectors $k\in \mathbb{Z}^n\setminus\{0\}$ such that $|k|_1 \leq K$ and $|\bar k| \neq 0$ is bounded by $2^n K^n - K$, but since it is not a big improve, for the sake of simplicity we assume the bound \ref{eq:meas_nonr_2} at $Z$.


\end{proof}


\begin{lemma}\label{lemma:2.3}
Let $G \subset \mathbb{R}^n$ be compact. $u,\tilde u: G \rightarrow \mathcal{R}^n$ maps of class $\mathcal{C}^2$. $|\tilde u - u|\leq \varepsilon$. Assume that $u$ is one-to-one on $G$, let $F = u(G)$. Consider the following bounds:
$$\left|\frac{\partial u}{\partial I}\right|_G \leq M, \left|\frac{\partial u}{\partial I}(I)\cdot v\right| \geq \mu|v| \quad \forall v\in\mathbb{R}^n, \forall I \in G,$$
$$\left|\frac{\partial \tilde u }{\partial I}\right|_G\leq\tilde M, \left|\frac{\partial \tilde u}{\partial I^2}\right|_G\leq\tilde M_2, \left|\frac{\partial \tilde u}{\partial I}(I) v\right|\geq\tilde\mu|v| \quad \forall v \in \mathbb{R}^n, \forall I \in G,$$
$\tilde \mu < \mu$ and $\tilde M < M$. Assume $\varepsilon \leq \tilde mu^2/(4\tilde M_2)$.
Then, given a subset $\tilde F \subset F - \frac{4M\varepsilon}{\tilde \mu}$ and writing $\tilde G = (\tilde u)^{-1}(\tilde F)$, the map $\tilde u$ is one-to-one from $\tilde G$ to $\tilde F$ and
$$ \tilde G \subset G - \frac{2\epsilon}{\tilde \mu}, \quad u(\tilde G) \supset \tilde F - \varepsilon.$$
Moreover,
$$|(\tilde u)^{-1} - u^{-1}|_{\tilde F} \leq \frac{\varepsilon}{\mu}$$
\end{lemma}

\begin{proof}
The statement is not any different than the classical one, so we are not going to prove it in here. A proof can be found in \cite{D}.
\end{proof}


\begin{lemma}[Inductive lemma]\label{lemma:inductive}
Let $G \subset \mathbb{R}^n$ be a compact.
$$H(\phi,I)=\hat h(I) + R(\phi,I)$$
 where $\hat h$ is defined as in \ref{eq:bm-hamiltonian} in the domain $\mathcal{D}_\rho(G)$,and $R(\phi,I)$ analytic on the same domain.
Let $\hat u = \frac{\partial \hat h}{\partial I}$ and $u = \frac{\partial h}{\partial I}$. Assume that $|\frac{\partial}{\partial I} u|_{G,\rho_2} \leq M'$ and $|u|_G \leq L$. Also, assume that $u$ is non-degenerate:
$$\left|\frac{\partial u}{\partial I}v\right| \geq \mu|v| \quad \forall I \in \mathcal{G}.$$
Let $\tilde M > M$, $\tilde L > L$ and $\tilde \mu < \mu$.
Assume $u$ is one-to-one on $G$ and denote $F = u(G)$. Assume $\tau > 0$, $0 < \beta \leq 1$ and $K$ given. Assume also that
$$F\cap \Delta_{c,\hat q}\left(K,\frac{\beta}{|k|_1^\tau}\right) = \emptyset, \quad \forall k \in \mathbb{Z}^n,|k|_1 \leq K, k\neq0.$$

Denote $\epsilon:=\|DR\|_{G,\rho,c}$, $\eta:=|R_0|_{G,\rho_2}$ and $\xi:=\left|\frac{\partial R_0}{\partial I}\right|_{G,\rho_2}$.
\begin{enumerate}
\item\label{eq:inductive_lemma_1} $\rho_2 \leq \frac{\beta}{2MK^{\tau+1}}$
\item\label{eq:inductive_lemma_2} $\epsilon \leq \min\left(\frac{\beta \hat \delta_c}{74 A K^\tau},\frac{\tilde\mu^2(\rho_2-\delta_2)}{4\tilde M}\right)$
\item\label{eq:inductive_lemma_3} \textcolor{black}{$\xi \leq \min\left((\tilde M - M)\delta_2/\mathcal{R}, (\mu - \tilde \mu)\rho_2\right)$}
%$\xi \leq \min\left((\tilde M - M)\delta_2/\mathcal{R}, \tilde L - L, (\mu - \tilde \mu)\rho_2\right)$
\end{enumerate}
Then there exists a real canonical transformation $$\Phi:\mathcal{D}_{\rho-\frac{\delta}{2}}(G)\rightarrow \mathcal{D}_\rho(G)$$
 and a decomposition $H\circ\Phi = \tilde{\hat h}(I) + \tilde R (\phi,I)$. Writing $\tilde u = \frac{\partial}{\partial I} \tilde h$ one has.
\begin{enumerate}
\item $|\tilde u - u|_{G,\rho_2}=\xi, \quad |\tilde h - h|_{G,\rho_2} = \eta,$
\item $\tilde \epsilon := \|D\tilde R \|_{G,\rho-\delta,c}\leq e^{-K\delta_1}\epsilon + \frac{14AK^\tau}{\beta\hat\delta_c}\epsilon^2,$
\item $\tilde \eta := |\tilde R_0|_{G,\rho_2-\frac{\delta_2}{2}}\leq\frac{7AK^\tau}{c\beta}\epsilon^2,$
\item $|\Phi -\id|_{G,\rho-\frac{\delta}{2},c}\leq\frac{2AK^\tau}{\beta}\epsilon,$
\item $\left|\frac{\partial}{\partial I} \tilde u\right|_{G,\rho_2} \leq \tilde M'$, $|\tilde u|_G\leq \tilde L,$
\item $|\frac{\partial \tilde u}{\partial I}v| \geq \tilde \mu |v| \quad \forall I \in \mathcal{G},$
\item Given a subset $\tilde F \subset F -\frac{4M\epsilon}{\tilde \mu}$, $\tilde G (\tilde u)^{-1}(\tilde F)$ the map $\tilde u$ is one-to-one from $\tilde G$ to $\tilde F$, $\tilde G \subset G -\frac{2\epsilon}{\tilde \mu}$, $u(\tilde G) \supset \tilde F -\epsilon$. Moreover $|\tilde u^{-1} - u^{-1}|_{\tilde F} \leq \epsilon/\mu$.
\end{enumerate}

\end{lemma}


\begin{proof}
%Recall that $A^\omega$ is defined as $A^\omega = \left(\begin{array}{rl}
%\mathcal{A}(I_1) + \mathcal{B}(I_1) & 0  \\
% 0 &  \text{Id}_{n-1,n-1}\\
%\end{array}\right).$
The set $u(I)$ is $\beta/K^\tau, K$-non-resonant with respect to $\omega$. This implies that
\begin{equation}\label{eq:nonres}
|k_1\mathcal{B}(I_1)u_1 + \bar{k}\bar{u} + \mathcal{A}(I_1)u_1| \geq \beta/K^\tau. \geq \frac{\beta}{|k|_1^\tau} \geq \frac{\beta}{K^\tau}.
\end{equation}
Then $\rho_2\leq \frac{\beta/K^\tau}{2MK} = \frac{\beta}{2MK^{\tau+1}}$, $\|DR\|_{G,\rho,c} \leq \frac{\beta/K^\tau \hat{\delta}_c}{74A} = \frac{\beta \hat{\delta}_c}{74AK^\tau}$.
We apply the iterative lemma (Theorem \ref{lemma:iterative}) to obtain $\Phi:\mathcal{D}_{\rho-\frac{\delta}{2}}(G) \rightarrow \mathcal{D}_\rho(G)$, such that $H\circ \Phi = \tilde{h} + \tilde{R}$ where $\tilde{h} = h + R_0$.

We have taken out the points that are not $\beta/K^\tau,K$-non-resonant with respect to $\omega$.
Because of conditions \ref{eq:inductive_lemma_1} and \ref{eq:inductive_lemma_2} we can apply the Iterative lemma.
Now let us prove each of the points in the statement.
\begin{enumerate}
\item We know by definition that $\tilde{u} = \frac{\partial \tilde{h}}{\partial I} = \frac{\partial (h + R_0)}{\partial I} = \frac{\partial h}{\partial I} + \frac{R_0}{\partial I}$, hence:
$$|\tilde u - u|_{G,\rho_2} = |\frac{\partial h}{\partial I} +\frac{\partial R_0}{\partial I} - \frac{\partial h}{\partial I}|_{G,\rho_2} = |\frac{\partial R_0}{\partial I}|_{G,\rho_2} = \xi$$
$$\tilde h = h + R_0 \Rightarrow |\tilde h - h|_{G,\rho_2} = |h + R_0 - h|_{G,\rho_2} = |R_0|_{G,\rho_2} = \eta$$
\item By the iterative lemma:
$$ \begin{array}{rcl}
\|D \tilde R\|_{G,\rho-\delta,c} & \leq & e^{-K \delta_1}\|DR\|_{G,\rho,c} + \frac{14A}{\alpha \hat \delta_c}\|DR\|_{G,\rho,c}\\
 & \leq & e^{-K\delta_1}\varepsilon + \frac{14A}{\alpha \hat \delta_c}\varepsilon^2\\
 & = & e^{-K\delta_1}\varepsilon + \frac{14A K^\tau}{\beta \hat \delta_c}\varepsilon^2,
\end{array}
$$
where we have used that $\alpha = \frac{\beta}{K^\tau}$.
\item At this point we use an inequality used in the proof of the iterative Lemma (theorem \ref{lemma:iterative}).
$$\begin{array}{rcl}
|\tilde R_0|_{G,\rho_2 - \delta_2/2} & \leq & |r_2(h,W,1) + r_1(R,W,1)|_{G,\rho_2 - \delta_2/2}\\
 & \leq & \frac{7A}{\alpha c} \|DR\|^2_{G,\rho,c} = \frac{7AK^\tau}{\beta}\varepsilon^2.
\end{array}$$
\item Also using the the iterative Lemma:
$$|\Phi-\text{id}|_{G,\rho-\delta/2,c} \leq \frac{2A}{\alpha} \|DR\|_{G,\rho,c} = \frac{2AK^\tau}{\beta}\|DR\|_{G,\rho,c}.$$
\item Recall that $|\frac{\partial}{\partial I} A^\omega \tilde u|_{G,\rho_2 - \delta_2}\leq \tilde M$, $|\tilde u|_G \leq \tilde L$, $\tilde h = h + R_0$, $|\frac{\partial}{\partial I} A^\omega u|_{G,\rho_2}\leq M$, $|u|_G \leq L$.
Note that $\mathcal{A}(I_1) \leq m \cdot \max_j(q_j)/\min_j(c_j)$ and $\mathcal{B}(I_1) \leq 1/ \min_j(c_j)$. Hence $\mathcal{A}(I_1) + \mathcal{B}(I_1) \leq \max_j(q_j)/\min_j(c_j) + 1/ \min_j(c_j) := \mathcal{R}$, and we have that $|A^\omega| \leq \mathcal{R}$.
$$
\begin{array}{rcl}
 |\frac{\partial}{\partial I} A^\omega \tilde u|_{G,\rho_2-\delta_2}& = &
 |\frac{\partial}{\partial I} A^\omega \tilde u + \frac{\partial}{\partial I} A^\omega u - \frac{\partial}{\partial I} A^\omega  u|_{G,\rho_2-\delta_2} \\
 & \leq & |\frac{\partial}{\partial I} A^\omega (\tilde u - u)|_{G,\rho_2-\delta_2} + |\frac{\partial}{\partial I} A^\omega  u|_{G,\rho_2-\delta_2}\\
 & \leq & |\frac{\partial}{\partial I} A^\omega R_0|_{G,\rho_2-\delta_2} + M \\
 & \leq & \frac{|A^\omega|_{G,\rho_2}|R_0|_{G,\rho}}{\delta_2} + M\\
 & \leq & \frac{|A^\omega|_{G,\rho_2} \cdot \xi}{\delta_2} + M \\
 & \leq & \frac{\mathcal{R}\xi}{\delta_2} + M\\
 & \leq & \displaystyle \frac{(\frac{(\tilde M - M)\delta_2}{\mathcal{R}})\mathcal{R}}{\delta_2} + M \leq \tilde M - M + M = \tilde M,
\end{array}
$$
where $\xi \leq (\tilde M - M)\delta_2/\mathcal{R}$.
\item We know $|\frac{\partial u}{\partial I}(I) v| \geq \mu |v|$ for all $I \in \mathcal{G}$, then $|\frac{\partial u}{\partial I}(I) v|_G \geq \mu|v|$.
We want to find $|\frac{\partial \tilde u}{\partial I} (I) v|_G \geq \mu' |v|$ if $\mu' < \mu$.
$$
\begin{array}{rcl}
 |\frac{\partial \tilde u}{\partial I} v|_G &  = & |(\frac{\partial \tilde u}{\partial I} + \frac{\partial u}{\partial I} - \frac{\partial u}{\partial I})v|_G \\
 & = & |(\frac{\partial^2 R_0}{\partial I^2} + \frac{\partial u}{\partial I})v|_G\\
  & \geq & -|\frac{\partial^2 R_0}{\partial I^2} v|_G + |\frac{\partial u}{\partial I} v|_G\\
  & \geq & \mu|v| - |\frac{\partial^2 R_0}{\partial I^2}|_G|v|\\
  & \geq & \mu|v| - |\frac{\partial R_0}{\partial I}|_G\frac{1}{\delta_2}|v|\\
    & \geq & \mu|v| - \frac{\xi}{\rho_2}|v| = (\mu - \xi/\rho_2)|v| \geq \mu'|v|,\\
\end{array}
$$
where we have used that $|\frac{\partial^2 R_0}{\partial I^2}|_G \leq |\frac{\partial R_0}{\partial I}|\frac{1}{\rho_2}$, and also that $\mu' < \mu -\xi/\rho_2$, hence $\xi \leq (\mu - \mu')\rho_2$.
\item To apply lemma \ref{lemma:2.3} we only need to check that $\varepsilon \leq \frac{\tilde \mu ^2}{ \tilde M_2}$.
$\tilde M_2$ can be chosen such that $|\frac{\partial^2 u}{\partial I^2}|_G \leq \tilde M_2$. Note that $|\frac{\partial^2 u}{\partial I^2}|_G \leq |\frac{\partial^2 u}{\partial I^2}|_{G,\rho_2-\delta_2}$.
$$
\begin{array}{rcl}
|\frac{\partial u}{\partial I}|_{G,\rho_2-\delta_2} \leq \tilde M & \Rightarrow & |\frac{\partial^2 u}{\partial I^2}|_{G, \rho_2-\delta_2}(\rho_2-\delta_2) \leq |\frac{\partial u}{\partial I}|_{G,\rho_2-\delta_2} \leq \tilde M\\
 & \Rightarrow & |\frac{\partial^2 u}{\partial I ^2}|_{G,\rho_2-\delta_2} \leq \frac{\tilde M}{\rho_2-\delta_2} = \tilde M_2\\
 & \Rightarrow & |\frac{\partial^2 u}{\partial I^2}|_G \leq \tilde M_2
\end{array}
$$
Then $\varepsilon \leq \frac{\tilde M}{4 \tilde M_2}$ if and only if $\varepsilon \leq \mu^2/(4\frac{\tilde M}{(\rho_2 - \delta_2)})$ if and only if $\varepsilon \leq \frac{\mu^2(\rho_2-\delta_2)}{4\tilde M}$ which it is assumed in the statement.
\end{enumerate}

\end{proof}

\section{ A KAM theorem on $b^m$-symplectic manifolds}


\begin{theoremB}[ A $b^m$-KAM theorem]\label{th:bm_kam}
Let $\mathcal{G} \subset \mathbb{R}^n$, $n\geq 2$ be a compact set.
Let $H(\phi, I) = \hat h (I) + f(\phi,I)$, where $\hat h$ is a $b^m$-function $\hat h (I) = h(I) + q_0 \log(I_1) + \sum_{i=1}^{m-1} \frac{q_i}{I_1^i}$ defined on $\mathcal{D}_\rho(G)$, with $h(I)$ and $f(\phi,I)$ analytic.
Let $\hat u = \frac{\partial \hat h}{\partial I}$ and $u = \frac{\partial h}{\partial I}$.
Assume $|\frac{\partial u}{\partial I}|_{G,\rho_2} \leq M$, $|u|_{\mathcal{G}} \leq L$.
Assume that $u$ is $\mu$ non-degenerate ($|\frac{\partial u}{\partial I}v|\geq \mu|v|$ for some $\mu \in \mathbb{R}^+$ and $I \in \mathcal{G}$. Take $a = 16M$.
Assume that $u$ is one-to-one on $\mathcal{G}$ and its range $F = u(\mathcal{G})$ is a $D$-set.
Let $\tau>n-1,\gamma>0$ and $0 < \nu < 1$. Let
\begin{enumerate}
\item \begin{equation}\label{eq:kam1}
\varepsilon:=\|f\|_{\mathcal{G}, \rho} \leq \frac{\nu^2 \mu^2 \hat \rho^{2\tau+2}}{2^{4\tau+32}L^6M^3} \gamma^2,
\end{equation}
\item \begin{equation}\label{eq:kam2}
\gamma \leq \min(\frac{8LM\rho_2}{\nu \hat \rho^{\tau+1}}, \frac{L}{\mathcal{K}'})
\end{equation}
\item \begin{equation}\label{eq:kam3}
\mu \leq \min(2^{\tau+5}L^2 M,2^7\rho_1 L^4 K^{\tau+1},\beta\nu^{\tau+1}2^{2\tau+1}\rho_1^\tau),
\end{equation}
\end{enumerate}
where $\hat \rho := \min \left(\frac{\nu\rho_1}{12(\tau+2)},1\right)$.
Define the set $\hat G = \hat G_\gamma := \{I \in  \mathcal{G}-\frac{2\gamma}{\mu} | u(I) \text{ is } \tau,\gamma,c,\hat q- Dioph.\}$.
Then, there exists a real continuous map $\mathcal{T}: \mathcal{W}_{\frac{\rho_1}{4}}(\mathbb{T}^n)\times \hat G \rightarrow \mathcal{D}_\rho(\mathcal{G})$ analytic with respect the angular variables such that
\begin{enumerate}
\item\label{kam:point1} For all $I \in \hat G$ the set $\mathcal{T}(\mathbb{T}^n\times \{I\})$ is an invariant torus of $H$, its frequency vector is equal to $u(I)$.
\item\label{kam:point2} Writing $\mathcal{T}(\phi,I)=(\phi + \mathcal{T}_\phi(\phi,I), I + \mathcal{T}_I(\phi,I))$ with estimates

\textcolor{black}{
$$|\mathcal{T}_\phi(\phi,I)| \leq \frac{2^{2\tau + 15} M L^2}{\nu^2 \hat \rho^{2\tau+1}}\frac{\varepsilon}{\gamma^2}$$
$$|\mathcal{T}_I(\phi,I))| \leq \frac{2^{10+\tau} L (1+M)}{\nu \hat \rho^{\tau+1}}\frac{\varepsilon}{\gamma}$$
}


\item\label{kam:point3} $\text{meas} [(\mathbb{T}^n\times \mathcal{G})\setminus\mathcal{T}(\mathbb{T}^n\times \hat G)] \leq C \gamma$ where $C$ is
\textcolor{black}{a really complicated constant depending on $n$,  $\mu$,  $D$,  $\text{diam} F$,  $M$, $\tau$, $\rho_1$, $\rho_2$, $K$ and $L$.}

\end{enumerate}

\end{theoremB}

\begin{proof}
This proof, as the one in \cite{D} is going to be structured in six sections. First we define the parameters used in each iteration while building the diffeomorphism. After that, we prove that we can apply the inductive lemma \ref{lemma:inductive} and we exhibit some bound that hold using the results of the inductive lemma. Next, we find that the sequence of frequency vectors and the sequence of diffeomorphisms that we built actually converges. Then we find estimates of the components of the canonical transformation that we have built. Then we find a way to identify the invariant tori and finally we give a bound for the measure of the set of invariant tori.
\begin{enumerate}
\item Choice of parameters

We are going to make iterative use of proposition \ref{lemma:inductive}. So we need to properly define all the parameters in the statement for every iteration.
Let:
$$
\left\{
\begin{array}{rcl}
M_q & = & (2 - \frac{1}{2^q})M, \\
L_q & = & (2 - \frac{1}{2^q})L, \\
\mu_q & = & (1 + \frac{1}{2^q})\frac{\mu}{2}.
\end{array}
\right.
$$

Note that $M_q, L_q$ monotonically increase from $M$ to $2M$ and $L$ to $2L$ when $q\rightarrow\infty$. On the other hand $\mu_q$ monotonically decreases from $\mu$ to $\mu/2$.
Also, let:

$$
\left\{
\begin{array}{rcl}
K_0 & = & 0, \\
K_q & = & K\cdot q^{q-1}, q \geq 1,
\end{array}
\right.
$$

where  $K$ is the minimum natural number greater or equal than $1/\hat \rho$ and \textcolor{black}{greater or equal than $(\frac{\nu\beta}{\mu 2^{2\tau+12}})^{1/\tau}$}. Moreover, $\beta:= \gamma/L \leq 1$, and

$$
\left\{
\begin{array}{rcl}
 \rho^{(q)}& = & (\rho_1^{(q)}, \rho_2^{(q)}), \\
 \rho_1^{(q)}& = & (1+\frac{1}{2^{\nu q}})\frac{\rho_1}{4}, \\
 \rho_2^{(q)}& = & \frac{\nu\beta}{32 M K_{q+1}^{\tau+1}}.
\end{array}
\right.
$$

Notice that $\rho_1^{(q)}$ decreases monotonically from $\rho_1/2$ to $\rho_1/4$. Also, $\rho_2^{(q)}$ decreases to 0. We also denote:

$$
\left\{
\begin{array}{rcl}
 \delta_1^{(q)} & = & \rho_1^{(q-1)} - \rho_1^{(q)}, \\
 \delta_2^{(q)} &= &\rho_2^{(q-1)}-\rho_2^{(q)}, \\
 c_q & = & \frac{\delta_2^{(q)}}{\delta_1^{(q)}}.
\end{array}
\right.
$$

Note that
$$
\begin{array}{rcl}
\delta_1^{(q)} & = & \left(1+\frac{1}{2^{\nu(q-1}}\right)\frac{\rho_1}{4}-\left(1 + \frac{1}{2^{\nu q}}\right)\frac{\rho_1}{4} \\
& = & \left(\frac{1}{2^{\nu(q-1)}} - \frac{1}{2^{\nu q}}\right)\frac{\rho_1}{4} \\
& = & \frac{1 - 1/2^\nu}{2^{\nu(q-1)}}\frac{\rho_1}{4}.\\
\end{array}
$$

Also, since $0 < \nu < 1$ then $\nu/2 \leq 1-1/2^\nu \leq \nu$. Plugging this in the previous equation we obtain:

\begin{equation}\label{eq:delta1ineq}
\frac{\nu \rho_1}{2^{\nu(q-1)}8} \leq \delta_1^{(q)} \leq \frac{\nu \rho_1}{2^{\nu(q-1)}4}.
\end{equation}

Also,

$$
\begin{array}{rcl}
\delta_2^{(q)} & = & \frac{\nu \beta}{32 M K_q^{\tau+1}} - \frac{\nu \beta}{32 M K_{q+1}^{\tau+1}}\\
& = & \frac{\nu \beta}{32M (K2^{q-1})^{\tau+1}} - \frac{\nu \beta}{32 M (K 2^q)^{\tau+1}}\\
& = & \frac{\nu \beta}{32M(K2^{q-1})^{\tau + 1}}\left(1-\frac{1}{2^{\tau+1}}\right) .\\
\end{array}
$$

Also, since $\tau > 0$ then $1/2 \leq (1 - 1/2^{\tau+1}) \leq 1$. Using this in the previous equation:

\begin{equation}\label{eq:delta2ineq}
\frac{\nu \beta}{64 M K_q^{\tau+1}}\leq \delta_2^{(q)} \leq \frac{\nu \beta}{32 M K_q^{\tau+1}}.
\end{equation}

Using equations \ref{eq:delta1ineq} and \ref{eq:delta2ineq} we find bounds for $c_q$

$$
\left\{
\begin{array}{rcl}
c_q & \leq & \frac{\left(\frac{\nu \beta}{32 M K_q^{\tau+1}}\right)}{\left(\frac{\nu \rho_1}{2^{\nu(q-1)}}\right)} = \frac{\beta 2^{\nu(q-1)}}{4 M K_q^{\tau+1}\rho_1},\\
c_q & \geq & \frac{\left(\frac{\nu \beta}{64 M K_q^{\tau+1}}\right)}{\left(\frac{\nu \rho_1}{2^{\nu(q-1)} 4}\right)} = \frac{\beta 2^{\nu(q-1)}}{16 M K_q^{\tau+1}\rho_1}.\\
\end{array}
\right.
$$

Then, we also define

$$
\left\{
\begin{array}{rcl}
\beta_q & = & (1-\frac{1}{2^{\nu q}})\beta,\\
\beta'_q & = & \frac{\beta_q + \beta_{q+1}}{2}.\\
\end{array}
\right.
$$

Observe that both $\beta_q$ and $\beta'_q$ tend to $\beta$. Also observe that $\beta'_q \geq \frac{\nu}{4}\beta$, because:

$$
\begin{array}{rcl}
\beta'_q & = & \frac{\beta_q + \beta_{q+1}}{2} \\
& = & \frac{\left(1 - \frac{1}{2^{\nu q}}\right) + \left(1 - \frac{1}{2^{\nu(q+1)}}\right)}{2}\beta \\
& = & \left(1 - \left(\frac{1 + \frac{1}{2^\nu}}{2^{\nu q}}\right)\frac{1}{2}\right)\beta \\
& \geq & \left(1 - (1 - 1/2^\nu)\frac{1}{2}\right)\beta \geq  \frac{\nu}{4}\beta.
\end{array}
$$

As $K$ is the minimal natural number such that $K \geq 1/\hat\rho$ then $K \leq 2/\hat \rho$. Hence $\hat\rho \leq \frac{2}{K}$. Also $$\frac{1}{\hat\rho^{\tau+1}} \geq \left(\frac{K}{2}\right)^{\tau+1}.$$

Recall that $\hat \rho = \min(\frac{\nu \rho_1}{12(\tau+2)},1)$ and, in particular, $\hat \rho \leq \nu\rho_1$ and $\hat \rho \leq 1$.

By definition $\gamma \leq \frac{8LM\rho_2}{\nu \hat \rho^{\tau+1}}$. And because $\beta = \gamma/L$:
$$\beta L \leq \frac{8L M \rho_2}{\nu \hat \rho^{\tau+1}} \leq \frac{8L M \rho_2 K^{\tau+1}}{\nu}.$$

Because we assumed $\varepsilon \leq \frac{\nu^2 \mu^2 \hat\rho^{2\tau+2}}{2^{4\tau + 32} L ^6 M^3} \gamma^2$ then, using that $\gamma = L\beta$ and $\hat \rho \leq 2/K$:

\begin{equation}\label{eq:kam_epsilon_1}
\varepsilon \leq \frac{\nu^2 \mu^2 \left(\frac{2}{K}\right)^{2\tau+2}}{2^{4\tau + 32} L ^6 M^3} \leq \frac{\nu^2 \mu^2 \beta^2}{2^{4\tau + 30} L ^4 M^3 K^{2\tau+2}}.
\end{equation}

Also using again the assumption that $\varepsilon \leq \frac{\nu^2 \mu^2 \hat \rho^{2\tau+2}}{2^{4\tau+32} L^6 M^3} \gamma^2$ we want to prove that

\begin{equation}\label{eq:kam_epsilon_2}
\varepsilon \leq \frac{\nu^3 \rho_1 \beta^2}{2^{2\tau+22} M K^{2\tau +1}}.
\end{equation}

 It is enough to check that:

$$\frac{\nu^2 \mu^2 \hat \rho^{2\tau+2} L^2 \beta^2}{2^{4\tau + 32} L^6 M^3} \leq \frac{\nu^3 \rho_1 \beta^2}{2^{2\tau + 22} M K^{2\tau+1}}$$

where we used $\gamma = L\beta$. Now observing that $\hat \rho \leq \nu \rho_1$ it suffices to see

$$\frac{\nu^2 \mu^2 \nu^{2\tau + 2} \rho_1^{2\tau +2} L^2 \beta^2}{2^{4\tau + 32} L^6 M^3}\leq\frac{\nu^3 \rho_1 \beta^2}{2^{2\tau+22} M K^{2\tau+1}},$$

which simplifies to

$$\frac{\mu^2 \rho_1^{2\tau+2}}{2^{4\tau+10}L^4M^2} \leq \frac{1}{K^{2\tau+1}}.$$

Using that $K \geq 1/(\nu\rho_1)$ is enough to check that

$$\frac{\mu^2 \rho_1^{2\tau+1}\nu^{2\tau+2}}{2^{2\tau+12} L^4 M^2} \leq (\nu \rho_1)^{2\tau+1},$$

which holds if and only if $\mu \leq 2^{\tau+5}L^2 M$ as we assumed.

\item Induction

Let us take $G_0 = \mathcal{G}$.
Now the goal is to construct a decreasing sequence of compact sets $G_q \subset \mathcal{G}$ and a sequence of real analytic canonical transformations

$$\Phi^{(q)}:\mathcal{D}_{\rho^{(q)}}(G_q) \rightarrow \mathcal{D}_{\rho^{(q-1)}}(G_{q-1}), \quad q \geq 1.$$

Denoting $\Psi^{(q)}= \Phi^{1}\circ \cdots \circ \Phi^{(q)}$ the transformed Hamiltonian functions will be noted by $H^{(q)} = H\circ \Psi^{(q)} = \hat h^{(q)}(I) + R^{(q)}(\phi,I)$. Moreover, $u^{(q)} = \frac{\partial h^{(q)}}{\partial I}$ and $\hat u^{(q)} = \frac{\partial \hat h^{(q)}}{\partial I}$.

We are going to show that the following bounds hold for all $q \geq 0$:

\begin{enumerate}
\item\label{eq:induction1} $\varepsilon_q := \|DR^{(q)}\|_{G_q,\rho^{(q)},c_{q+1}} \leq \frac{8\varepsilon}{\nu \rho_1 2^{(2\tau+2)q}},$
\item\label{eq:induction2} $\eta_q:=|R_0^{(q)}|_{G_q,\rho_2^{(q)}} \leq \textcolor{black}{\frac{\varepsilon}{2^{(2\tau+3)q}}}$ and $\xi_q:=|\frac{\partial R_0^{(q)}}{\partial I}|_{G_q,\rho_2^{(q)}} \leq \textcolor{black}{ \frac{4 M K^{\tau+1 \varepsilon}}{\nu \beta 2^{(\tau+2)q}}},$
\item\label{eq:induction3} $|\frac{\partial^2 h^{(q)}}{\partial I^2}|_{G_q,\rho_2^{(q)}}\leq M_q, \quad |u^{(q)}| \leq L_q \quad \forall I\in G_q,$
\item\label{eq:induction4} $u^{(q)}$ is $\mu_q$-non-degenerate on $G_q$,
\item\label{eq:induction5} $u^{(q)}$ is one-to-one on $G_q$, and $u^{(q)}(G_q)=F_q$ where we define:
$$F_q := (F - \beta_q)\setminus \bigcup_{\substack{k\in\mathbb{Z}^n\setminus\{0\} \\ |k|_1 \leq K}} \Delta_{c_q,\hat q}(K,\frac{\beta_q}{|k|_1^\tau})$$
\end{enumerate}

To prove this we proceed by induction. For $q = 0$:

$$
\left\{
\begin{array}{l}
G_0 = \mathcal{G}, \\
h^{(0)} = h,\hat h^{(0)} = \hat h,\\
R^{(0)} = f.
\end{array}
\right.
$$

Using the definitions from the previous point:

$$
\left\{
\begin{array}{rcl}
\rho_1^{(0)} &=& (1+1)\frac{\rho_1}{4} = \rho_1/2,\\
\rho_2^{(0)} &=& \frac{\nu \beta}{32 M K^{\tau+1}} \leq \frac{\rho_2}{2},
\end{array}
\right.
$$

where in the last inequality we have used that $\beta \leq \frac{8M\rho_2 K^{\tau+1}}{\nu}$ and hence $\rho_2 \geq \frac{\beta\nu}{8 M K^{\tau+1}}$.

Then,
\textcolor{black}{
$$\varepsilon_0 = \|Df\|_{G,\rho(0),c_1} = \|Df\|_{G,\rho(1)+\delta(1)}.$$
Now, let us use that $|D\Upsilon|_{G,\rho-\delta,c}\leq \frac{2|\Upsilon|_{G,\rho,c}}{\hat \delta_c}$ while having in mind that $\hat \delta_{c_1}^{(1)} = \min(c_1\delta_1^{(1)},\delta_2^{(1)})$.
Then,
$$\|D f\|_{G,\rho(1),c_1} \leq \frac{c_1 |f|_{G,\rho(0)}}{\hat \delta_{c_1}} \leq \frac{|f|_{G,\rho(0)}}{\delta_1^{(1)}} \leq \frac{|f|_{G,\rho(0)} 8}{\nu \rho_1} = \frac{8\varepsilon}{\nu \rho_1},$$
where we have used $\delta_1^{(1)}\geq \frac{\nu \rho_1}{8\cdot 2^{\nu(1-1)}} = \frac{\nu\rho_1}{8}$.
}
This proves the first step of the induction  for \ref{eq:induction1}).

Let us prove now the base case for \ref{eq:induction2}).
On one side $\eta_0 = |R_0^{(0)}|_{G_0,\rho_2(0)} \leq \frac{\varepsilon}{2^{(2\tau+3)0}} = \varepsilon$, which holds because $|R_0^{(0)}|_{G_0,\rho_2^{(0)}} \leq |R^{(0)}|_{\mathcal{G},\rho^{(0)}} = |f|_{\mathcal{G},\rho(0)} = \epsilon$.
On the other hand $\xi_0 = |\frac{\partial R_0^{(0)}}{\partial I}|_{G_0,\rho_2^{(0)}} \leq |\frac{\partial R_0^{(0)}}{\partial I}|_{G_0,\rho_2 -\rho_2/2} \leq \frac{1}{\rho_2/2}\|R_0\|_{G,\rho} \leq \frac{2\varepsilon}{\rho_2} \leq \frac{\varepsilon}{\rho_2^{(0)}}$, where we used that $\rho_2(0) \leq \rho_2/2 = \rho_2 - \rho_2/2$.

The base case of \ref{eq:induction3}) is immediate because $|\frac{\partial^2 h^{(0)}}{\partial I^2}|_{G_0,\rho_2^{(0)}} \leq |\frac{\partial^2 h}{\partial I^2}|_{\mathcal{G},\rho_2} = M = M_0$ and also $|u^{(0)}|_{G_0} = |u|_{\mathcal{G}} \leq L = L_0$.

The base case of \ref{eq:induction4} holds because $u^{(0)} = u$ is $\mu$ non-degenerate in $\mathcal{G} = G_0$.

The base case of \ref{eq:induction5} holds because $u^{(0)} = u$ is one-to-one in $G_0 = \mathcal{G}$ by hypothesis. $u^{(0)}(G_0) = F_0$ where $F_0 = (F - \beta_0)\setminus \{\emptyset\} = F$ because $K_0 = 0$ and $\beta_0 = 0$.

For $q \geq 1$, we assume the statements hold for $q-1$ and we prove it for $q$. Let us apply proposition \ref{lemma:inductive} (Inductive Lemma) to $H^{(q-1)} = h^{q-1} + R^{q-1}$ with $K_q$ instead of $K$.

We have to be careful with the condition $F\cap \Delta_{c,\hat q}(k,\frac{\beta}{|k|_1^\tau}) = \emptyset$ $\forall k \in \mathbb{Z}^n, |k|_1 \leq K_q, k\neq 0$ and with the definition
$$F_{q-1}:=(F - \beta_{q-1}) \setminus \bigcup_{\substack{k\in\mathbb{Z}^n\setminus\{0\} \\ |k|_1 \leq K}} \Delta_{c,\hat q}(k,\frac{\beta_{q-1}}{|k|_1^\tau}),$$
 because the resonances have to be removed up to order $K_q$, not $K_{q-1}$.
Let us define
$$F_{q-1}':= (F-\beta_{q-1}) \setminus \bigcup_{\substack{k\in\mathbb{Z}^n\setminus\{0\} \\ |k|_1 \leq K}} \Delta_{c,\hat q}(k,\frac{\beta_{q-1}'}{|k|_1^\tau})$$

where we simply replaced $\beta_{q-1}$ for $\beta_{q-1}'$ because $\Delta_{c,\hat q}(k,\frac{\beta_{q-1}}{|k|_1^\tau})$ makes no sense when $q=1,\beta_{q-1}=0$.

Accordingly let us define $G_{q-1}' := (u^{(q-1)})^{-1}(F_{q-1}')$. The conditions in proposition \ref{lemma:inductive} are going to be satisfied with $F_{q-1}'$, $\beta_{q-1}'$, $K_q$, $M_{q-1}$, $L_{q-1}$, $\mu_{q-1}$,$\rho^{(q-1)}$, $\delta^{(q)}$, $c_q$, $M_q$, $L_q$, $\mu_q$ replacing $F, \beta, K, M,L, \mu, \rho,\delta, c, \tilde M, \tilde L, \tilde \mu$. And also $a = 16M \geq 8M_q$.

We are now going to check that \ref{eq:inductive_lemma_1}, \ref{eq:inductive_lemma_2} and \ref{eq:inductive_lemma_3} are satisfied so we can apply proposition \ref{lemma:inductive}.
 \begin{itemize}

 \item[--] \ref{eq:inductive_lemma_1} We want to see that $\rho_2^{(q-1)}\leq \frac{\beta_{q-1}'}{2M_qK_q^{\tau+1}}$.
 By definition $\rho_2^{q-1} = \frac{\nu \beta}{32 M K_q^{\tau+1}} \leq \frac{4 \beta_{q-1}'}{32 M K_q^{\tau+1}}\leq \frac{\beta_{q-1}'}{8M_q K_q^{\tau+1}} \leq \frac{\beta_{q-1}}{2 M_q K_q^{\tau+1}}$, where we used that $M_q \geq M$.
 \item[--] \ref{eq:inductive_lemma_2} We want to see that $\varepsilon_{q-1} \leq \min \left(\frac{\beta_{q-1}\hat\rho_c^{(q)}}{74 A_q K_{q-1}^\tau}, \frac{\mu^\tau_q(\rho_2^{(q-1)}-\delta_c^{(q-1)})}{4 M_q}\right)$, where $A_q := 1 + \frac{2M_{q-1} c_q K_q^\tau}{\beta_{q-1}'}$.


 Notice that:


$$
\begin{array}{rcl}
A_q & := & 1 + \frac{2 M_{q-1} c_q K_q^\tau}{\beta_{q-1}'} \\
 & \leq & 1 + \frac{8 M_{q-1} c_q K_q^\tau}{\nu \beta}\\
 & \leq & 1 + \frac{8 M_{q-1}\beta 2^{\nu(q-1)}K_q^\tau}{4M K_q^{\tau+1}\rho_1 \nu \beta}\\
 & = & 1 + \frac{2 M_{q-1} 2 ^{\nu(q-1)}}{M K_q \rho_1 \nu} \\
 & = & 1 + \frac{2 M_{q-1} 2^{\nu(q-1)}}{M K 2^{q-1} \rho_1 \nu} \\
 & \leq & 1 + \frac{4 M 2^{q-1}}{M K 2^{q-1} \rho_1 \nu} \\
 & = & 1 + \frac{4}{K \rho_1 \nu} \leq 1 + 4 = 5\\
\end{array}
$$

 First, we  check that $\varepsilon_{q-1} \leq \frac{\beta_{q-1}\hat\rho_c^{(q)}}{74 A_q K_{q-1}^\tau}$.

 By induction hypothesis we know that $\varepsilon_{q-1} \leq \frac{8\varepsilon}{\nu \rho_1 2^{(2\tau+2)(q-1)}}$. Hence it is enough to see
 $$\frac{8\varepsilon}{\nu \rho_1 2^{(2\tau+2)(q-1)}} \leq \frac{\beta_{q-1}' \delta_2^{(q)}}{75\cdot 5 K_q^\tau}.$$

 Notice that
 $$
 \begin{array}{rcl}
 \frac{\beta_{q-1}' \delta_2^{(q)}}{379 K_q^\tau} & \geq & \frac{\nu \beta}{4}\frac{\nu \beta}{64 M K_q^{\tau+1}}\frac{1}{37 K_q^\tau} \\
 & = & \frac{\nu^2 \beta^2}{4\cdot 64 \cdot 370}\frac{1}{M K_q^{2\tau+1}}\\
 & = & \frac{\nu^2\beta^2}{4\cdot 64 \cdot 370 \cdot M 2^{(q-1)(2\tau+1)} K^{2\tau+1}}.
 \end{array}
$$

And this holds if the following is true:

$$\frac{8 \varepsilon}{\nu \rho_1} \leq \frac{\nu^2 \beta^2}{4\cdot 64 \cdot 379 \cdot K^{2\tau + 1} M}\Leftrightarrow \varepsilon \leq \frac{\nu^3 \beta^2 \rho_1}{2^{12} 135 M K^{2\tau+1}}.$$

This is true because in the previous section we have seen that $\varepsilon \leq \frac{\nu^3 \rho_1 \beta^2}{2^{2\tau+30}M K^{2\tau+1}}$


 Let us now prove that $\varepsilon \leq \frac{\mu_q^2 (\rho_2^{(q-1)} - \delta_2^{(q-1)})}{2 M_q}$. First of all observe that $(\rho_2^{(q-1)} - \delta_2^{(q)}) = \rho_2^{(q)}$.
 So, what we want to prove is equivalent to proving $\varepsilon_{q-1}\leq \frac{\mu_q^2 \rho_2^{(q)}}{2 M_q}$.

 On the other hand, we know that $\varepsilon_{q-1}\leq \frac{8\varepsilon}{\nu \rho_1 2^{(2\tau + 2)(q-1)}}$. And observe also that $\frac{\mu_q^2 \rho_2^{(q)}}{2M_q} \geq \frac{(\mu/2)^2\frac{\nu\beta}{32 M K^{\tau+1}}}{2M} = \frac{\mu^2 \nu \beta}{2^8 M^2 K^{\tau+1}}$.

 If are able to check that $\frac{8\varepsilon}{\nu \rho_1 2^{(2\tau+2)(q-1)}} \leq \frac{\mu\nu\beta}{2^8 M^2 K^{\tau+1}}$ we would be fine. The previous equation holds if and only if the following holds,
 $$\varepsilon \leq \frac{\mu \nu^2 \beta \rho_1 2^{(2\tau+2)(q-1)}}{2^{11}M^2 K^{\tau+1}}.$$

If we knew beforehand that $\varepsilon \leq \frac{\mu \nu^2 \beta \rho_1 2^{-(2\tau+2)}}{2^{11}M^2 K^{\tau+1}} = \frac{\mu \nu^2 \beta \rho_1}{2^{2\tau+13}M^2 K^{\tau+1}}$ we would be done.

But we also know that

$$\epsilon \leq \frac{\nu^2 \mu^2 \beta^2}{2^{2\tau+30}L^4 M^3 K^{2\tau+2}}.$$

Then it is enough to check that

$$\frac{\nu^2 \mu^2 \beta^2}{2^{2\tau+30}L^4 M^3 K^{2\tau+2}}\leq \frac{\mu \nu^2 \beta \rho_1}{2^{2\tau+13} M^2 K^{\tau+1}}.$$

And this holds because $\mu \leq 2^7 \rho_1 L^4 K^{\tau+1}$

\item[--] \ref{eq:inductive_lemma_3} Lastly we want to see that
$$\xi_{q-1} \leq \min((M_q-M_{q-1})\frac{\delta_2^{(q)}}{\mathcal{R}},(\mu_{q-1}-\mu_{q})\rho_2^{(q-1)}).$$
%$$\xi_{q-1} \leq \min((M_q-M_{q-1})\frac{\delta_2^{(q)}}{\mathcal{R}},L_q-L_{q-1},(\mu_{q-1}-\mu_{q})\rho_2^{(q-1)}).$$

Observe that $\mathcal{R}$ does not depend on $q$ because at each iteration $\hat{h}^{(q)}$ singular part is not modified. $\hat h^{(q)}= \hat h^{(q)} + R_0^{(q)}$ and $R_0$ is analytic depending only on the action coordinates.
By induction hypothesis, we know that
$$\xi_{q-1} = |\frac{\partial R_0^{(q)}}{\partial I}|_{G_q,\rho_2^{(q)}} \leq \frac{4MK^{\tau+1}\varepsilon}{\nu \beta 2^{(\tau+2)q}}.$$
We are going to check the \textcolor{black}{two} different inequalities separately
\begin{enumerate}
\item $\xi_{q-1} \leq (M_q-M_{q-1})\frac{\delta_2^{(q)}}{\mathcal{R}}$.
Note that $M_q =(2-\frac{1}{2^q})M$, then $M_q-M_{q-1} = \frac{M}{2^q}$.
$$\delta_2^{(q)} \geq \frac{\nu \beta}{64 M (K 2^{q-1})^{\tau+1}} \geq \frac{\nu\beta}{64 M (K 2^q)^{\tau+1}}$$
$$= \frac{\nu \beta}{64 M K^{\tau+1}}\frac{1}{2^{q\tau+q}}.$$
We deduce
$$(M_q-M_{q-1})\delta_2^{(q)} \geq \frac{\nu \beta}{64 K^{\tau+1}}\frac{1}{2^{\tau q + 2 q}}.$$
Hence we only need to check that
$$\frac{4MK^{\tau+1}\varepsilon}{\nu \beta 2^{(\tau+2)q}}\leq\frac{\nu \beta}{64 K^{\tau+1}}\frac{1}{2^{\tau q + 2 q}}.$$
The previous condition holds if and only if
$$\frac{4 M K^{\tau+1}\varepsilon}{\nu\beta} \leq \frac{\nu \beta}{2^6 K^{\tau+1}} \Leftrightarrow \varepsilon \leq \frac{\nu^2 \beta^2}{2 K^{2\tau+2}M}.$$

On the other hand, let us use again that $\varepsilon \leq \frac{\nu^2 \mu ^2 \beta^2}{2^{2\tau+30}L^4 M^3 K^{2\tau+2}}$. If we apply the condition $\mu \leq 2^{\tau+6}L^2 M$ in the last expression we obtain:

$$\varepsilon \leq \frac{\nu^2 \beta^2 2^{2\tau+12}L^4M^2}{2^{2\tau+30}L^4 M^3 K^{2\tau+2}} = \frac{\nu^2\beta^2}{2^8 K^{2\tau+2} M}.$$

%\item $\xi_{q-1} \leq L_q-L_{q-1}$.
%Observe that
%$$L_q = (2-\frac{1}{2^q})L, \quad L_q - L_{q-1} = \frac{1}{2^q}.$$
%Hence it is enough to check that
%$$\frac{2^2 M K^{\tau+1} \varepsilon}{\nu \beta 2^{(\tau+2)q}} \leq \frac{L}{2^q}.$$
%But observe also that

%$$\frac{2^2 M K^{\tau+1} \varepsilon}{\nu \beta 2^{(\tau+2)q}} \leq \frac{2^2 M K^{\tau+1} \varepsilon}{\nu \beta 2^q}.$$

%So, checking the following is enough

%$$\varepsilon \leq \frac{\nu \beta L}{4 M K^{\tau+1}}.$$

%But we know that

%\begin{longtable}{rcl}
%$\varepsilon$ & $\leq$ & $\frac{\nu^3 \rho_1 \beta^2}{2^{2\tau+20 M K^{2\tau +1}}}$\\
%& $\leq$ & $\frac{\nu^3 2 \beta^2}{2^{2\tau+20} M K^{2\tau+1} \nu K}$\\
%& $=$& $\frac{\nu^2 \beta^2}{2^{2\tau+19} M K^{2\tau+2}}$\\
%& $\leq$ & $\frac{\nu \beta}{2^{2\tau+19}}\frac{\nu \beta}{4M K^{2\tau+2}}$\\
%& $\leq$ & $\frac{\nu \beta}{4M K^{2\tau+2}} \leq \frac{\nu \beta}{4M K^{2\tau+2}} \textcolor{black}{L}$\\
%\end{longtable}

%where we have used that $\rho \leq \frac{2}{\nu K}$, $\nu < 1$ and $\beta \leq 1$.

\item $\xi_{q-1} \leq (\mu_{q-1}-\mu_{q})\rho_2^{(q-1)}$.

Observe that

$$\mu_q = (1 + \frac{1}{2^q})\frac{\mu}{2},$$
$$(\mu_{q-1} - \mu_q) = ((1 + \frac{1}{2^{q-1}})-(1 + \frac{1}{2^{q}}))\frac{\mu}{2} = (\frac{1}{2^{q-1}} - \frac{1}{2^{q}})\frac{\mu}{2} $$
$$= \left(\frac{2-1}{2^q}\right)\frac{\mu}{2} = \frac{1}{2^q}\frac{\mu}{2} = \frac{\mu}{2^{q+1}}$$
Also,
$$\rho_2^{(q-1)} = \frac{\nu \beta}{32 M K_q^{\tau+1}} = \frac{\nu\beta}{32 M (K 2^{q-1})^{\tau+1}} $$
$$\geq \frac{\nu \beta}{32 M K^{\tau+1} 2^{q(\tau+1)}}.$$

Then,
$$(\mu_{q-1}-\mu_q)\rho_2^{(q-1)} \geq \frac{\mu}{2^{q+1}}\frac{\nu\beta}{32 M K^{\tau+1} 2^{q(\tau+1)}}.$$

Then we only have to check that

$$
\begin{array}{rcl}
\frac{4 M K^{\tau+1}\varepsilon}{\nu \beta 2^{\tau q + 2q -2}} & \leq & \frac{\mu}{2^{q+1}}\frac{\nu \beta}{32 M K^{\tau+1} 2^{q(\tau+1)}}\\
& = & \frac{\mu}{2^{\tau q + 2q +1}}\frac{\nu \beta}{32 M K^{\tau+1} }.\\
\end{array}
$$

Which holds if and only if

$$\frac{M K^{\tau+1} \varepsilon}{\nu \beta 2^{-2}} \leq \frac{\mu}{2}\frac{\nu \beta}{32 M K^{\tau+1}}.$$

Then,

$$\varepsilon \leq \frac{\mu 2^{-2}}{2}\frac{\nu^2 \beta^2}{32 M^2 K^{2\tau+2}} = \frac{\mu \nu^2 \beta^2}{2^8 M^2 K^{2\tau+2}}.$$


But we know

\begin{longtable}{rcl}
$\varepsilon $ & $\leq$ & $\frac{\nu^2 \mu^2 \beta^2}{2^{\tau+30} L^4 M^3 K^{2\tau+2}}$ \\
& $\leq$ & $\frac{\nu^2 \mu 2^{\tau+5}L^4M\beta^2}{2^{\tau+30}L^4 M^3 K^{2\tau+2}}$ \\
& $=$ & $\frac{\nu^2 \mu \beta^2}{2^25 M^2 K^{2\tau+2}}$ \\
%& $=$ & $\frac{\mu \nu^2 \beta^2}{2^6 M^2 K^{2\tau+2}} \textcolor{black}{\frac{2}{L^2}}$ \\
& $\leq$ & $\frac{\mu \nu^2 \beta^2}{2^8 M^2 K^{2\tau+2}}$ \\
\end{longtable}

as we wanted. In the second inequality we used that $\mu \leq 2^{\tau+5}L^4 M$.%$\mu \leq 2^{\tau+6}L^2 M$

\end{enumerate}

So, finally, we can apply the inductive lemma \ref{lemma:inductive} with the parameters mentioned previously in this section. Hence we obtain a canonical transformation $\Phi^{(q)}$ and a transformed hamiltonian $H^{(q)} = h^{(q)} + R^{(q)}$.
The new domains $G_q \subset G_{q-1}'$ are going to be specified in the following lines. So now we are going to prove \ref{eq:induction1},\ref{eq:induction2},\ref{eq:induction3},\ref{eq:induction4},\ref{eq:induction5}.

\begin{itemize}
\item \ref{eq:induction1}. We want to see $\varepsilon_q := \|DR^{(q)}\|_{G_q, \rho^{(q)}, c_{q+1}} \leq \frac{8 \varepsilon}{\nu \rho_1 2^(2\tau+2)q}$.

By the second result of proposition \ref{lemma:inductive} we have:

\begin{equation}\label{eq:1qtobound}
\varepsilon_q \leq e^{-K_q \delta_1^{(q)}}\varepsilon_{q-1} + \frac{14 A_q K_q^\tau}{\beta_{q-1}'\delta_2^{(q)}} \varepsilon_{q-1}^2.
\end{equation}


Now we are going to bound each term of the right hand of the expression at a time.

Recall that $\delta_1^{(q)} \geq \frac{\nu \rho_1}{8 2^{\nu(q-1)}}$.

$$
\begin{array}{rcl}
  K_q \delta_1^{(q)} &  \geq &  K 2^{q-1} \frac{\nu \rho_1}{8} 2^{-\nu(q-1)} \\
  &  =  &  \frac{\nu \rho_1}{8}K 2^{(1-\nu)(q-1)} \\
  &  \geq &  \frac{12(\tau+2)}{8} 2^{(1-\nu)(q-1)} \\
  &  =  &  (3/2\tau + 3)2(1-\nu)(q-1) \\
  &  \geq &  (2\tau+3)\frac{3}{4} \geq (2\tau+3)\ln 2, \\
\end{array}
$$

where we used that $K\hat\rho\geq 1$ and hence $K \geq \frac{12(\tau+2)}{\nu\rho_1}$.
So we conclude that $e^{-K_q \delta_1^{(q)}} \leq \frac{1}{2^{2\tau+3}}$, and we have bounded the first term of \ref{eq:1qtobound}. Let us bind the second one.

On one hand, we have that

$$\frac{14 A_q K_q^\tau}{\beta_{q-1}'} \leq \frac{14\cdot5 K_q^\tau}{\frac{\nu \beta}{4}} \leq \frac{2^9 K_q^2}{\nu \beta} $$%\textcolor{black}{\leq \frac{2^8 K_q^2}{\nu \beta}},

where we have used that $\beta_q'\geq \frac{\nu\beta}{4}$ and $ A_q \leq 5$.

Now we are going to apply that $\varepsilon_{q-1} \leq \frac{8 \varepsilon}{\nu \rho_1 2^{(2\tau+2)(q-1)}}$, $\delta_2^{(q)}\geq \frac{\nu \beta}{64 M K_q^{\tau+1}}$ and $\epsilon \leq \frac{\nu^3 \rho_1 \beta^2}{2^{2\tau+22} M K^{2\tau+1}}$ to obtain

\begin{longtable}{rcl}
 $\ \frac{14 A_q K_q^\tau}{\beta_{q-1}' \delta_2^{(q)}} \varepsilon_{q-1}$ & $ =$ & $ \frac{14 A_q K_q^\tau}{\beta_{q-1}'}\frac{1}{\delta_2^{(q)}}\varepsilon_{q-1}$ \\
 & $ \leq$ & $ \frac{2^9 K_q^\tau}{\nu\beta} \frac{64 M K_q^{\tau+1}}{\nu \beta}\frac{8\varepsilon}{\nu \rho_1 2^{(2\tau+2)(q-1)}}$ \\
  & $ \leq$ & $ \frac{2^{18} M K_q^{2\tau+1}}{\nu^3 \beta^2 \rho_1 2^{(2\tau+2)(q-1)}}\frac{\nu^3 \rho_1 \beta^2}{2^{2\tau+22}M K^{2\tau + 1}}$ \\
  & $ \leq$ & $ 2^{18} 2^{(q-1)(2\tau+1)-(2\tau+2)(q-1)-(2\tau+22)}$ \\
  & $ =$ & $ 2^{(1-q)}2^{-2\tau-4} = \frac{1}{2^{2\tau+3}2^{q-1}}.$ \\
\end{longtable}
This gives us the bound of the second term of \ref{eq:1qtobound}. Now we put both bounds together:

$$\varepsilon_q \leq \frac{1}{2^{2\tau+3}}\varepsilon_{q-1} + \frac{1}{2^{2\tau+3}}\frac{1}{2^{q-1}}\varepsilon_{q-1} \leq \frac{1}{2^{2\tau+2}}\varepsilon_{q-1}.$$

That implies $\varepsilon_q \leq \frac{\epsilon}{2^{(2\tau+2)(q-1)}}$
as we wanted. \textcolor{black}{Because we can assume $\nu\rho_1 \leq 1$}.

\item \ref{eq:induction2}

Let us write $\sigma_2^{(q)} = \rho_2^{(q-1)} - \delta_2^{(q)}/2 = \rho_2^{(q)} + \delta_2^{(q)}/2 \geq \rho_2^{(q)}$, then $\eta_q = |R_0^{(q)}|_{G_q, \rho_2^{(q)}} \leq |R_0^{(q)}|_{G_q,\sigma_2^{(q)}}$.

By the inductive lemma \ref{lemma:inductive}:

\begin{longtable}{rcl}
 $\eta_q$ & $\leq$ & $\frac{7 A_q K_q^\tau}{c_q \beta_{q-1}'} \varepsilon_{q-1}^2 $\\
 & $\leq$ & $\frac{7 A_q K_q^\tau}{\beta_{q-1}'}\varepsilon_{q-1}^2 \frac{\delta_1^{(q)}}{\delta_2^{(q)}}$\\
 & $=$ & $\frac{14 A_q K_q^\tau}{\beta_{q-1}' \delta_2^{(q)}}\varepsilon_{q-1}^2\frac{\delta_1^{(q)}}{2}$\\
 & $\leq$ & $\frac{1}{2^{2\tau+3}2^{q-1}}\varepsilon_{q-1}\frac{\delta_1^{q)}}{2}$\\
 & $\leq$ & $\frac{1}{2}\frac{\delta_1^{(q)}}{2^{2\tau+3}2^{q-1}}\frac{\varepsilon}{\nu\rho_1 2^{(2\tau+2)(q-1)}}$ \\
 & $\leq$ & $\frac{1}{2}\frac{\nu \rho_1}{4\cdot 2^{\nu(q-1)}}\frac{1}{2^{2\tau+3}2^{q-1}}\frac{8\varepsilon}{\nu\rho_1 2^{(2\tau+2)(q-1)}}$ \\
 & $\leq$ & $\frac{\varepsilon}{2^{(2\tau + 3)q}}$. \\
\end{longtable}

For the second part we only need to apply  Cauchy inequalities:

$$\xi_q \leq \frac{2}{\delta_2^{(q)}} |R_0^{q}|_{G_q,\rho_2^{(q)}} \leq \frac{2}{\delta_2^{(q)}}\frac{\varepsilon}{2^{(2\tau+3)q}}.$$

\item \ref{eq:induction3} and \ref{eq:induction4} are direct from lemma \ref{lemma:inductive}.
\item \ref{eq:induction5} We need to consider again the results from lemma \ref{lemma:inductive} with $F_q$ as $F$.
We have to check the condition $F_q \subset  F_{q-1}' - \textcolor{black}{\frac{4M_{q-1}\varepsilon_{q-1}}{\mu_q}}$.
Let us define $d_q:=\frac{\beta_q - \beta_{q-1}}{2 K_q^{\tau+1}}$.
Using that $F_{q-1}' := (F-\beta_{q-1}) \setminus \bigcup_{\substack{k\in\mathbb{Z}^n\setminus\{0\} \\ |k|_1 \leq K}} \Delta_{c,\hat q}(k.\frac{\beta_{q-1}'}{|k|_1^\tau})$ we have
$$F_{q-1}' - d_q \supset (F-(\beta_{q-1}+d_q))\setminus \bigcup_{\substack{k\in\mathbb{Z}^n\setminus\{0\} \\ |k|_1 \leq K}}\Delta_{c,\hat q}(k,\frac{\beta_{q-1}'}{|k|_1^\tau} + |k|d_q).$$

Moreover,

$$
\left\{
\begin{array}{rcl}
\beta_{q-1} + d_q & \leq & \beta_q, \text{ and}\\
\frac{\beta_{q-1}'}{|k|_1^\tau} + |k|d_q & = & \frac{\beta_{q-1}' + |k|_1^\tau|k|\frac{\beta_q - \beta_{q-1}}{2 K_q^{\tau+1}}}{|k|_1^\tau} \\
& \leq & \frac{\beta_{q-1}' + K_q^{\tau+1}\frac{\beta_q - \beta_{q-1}}{2 K_q^{\tau+1}}}{|k|_1^\tau} = \frac{\beta_{q-1}' + \frac{\beta_q}{2} - \frac{\beta_{q-1}}{2}}{|k|_1^\tau} = \frac{\beta_q}{|k|_1^\tau}.
\end{array}
\right.
$$

Now if we see that $\frac{4 M_{q-1} \varepsilon_{q-1}}{\mu_q} \leq d_q$ we will have the inclusion we want.
Observe that $\frac{4 M_{q-1}}{\mu_q} \leq \frac{4\cdot 2M}{\mu/2} = \frac{16M}{\mu}$. So, it is enough to check that $\frac{16M}{\mu} \varepsilon_{q-1} \leq d_q$.

\begin{longtable}{rcl}
$\varepsilon_{q-1}$ & $\leq$ & $\frac{8\varepsilon}{\nu \rho_1 2^{(2\tau+2)q}}$ \\
& $\leq$ & $\frac{8 \nu^3 \rho_1 \beta^2}{\nu \rho_1 2^{(2\tau+2)q}2^{(2\tau+22)}M K^{2\tau+1}}$\\
& $\leq$ & $\frac{8 \nu^2 \beta^2}{2^{(2\tau+2)q + 2\tau + 20}M K^{2\tau+1}}$ \\
& $\leq$ & $\frac{8 \nu^2 \beta\frac{2(\beta_q-\beta_{q-1})}{\nu}}{2^{(\tau+1)q + (\tau+1)q + 2\tau + 20} M K^{2\tau+1}}$\\
& $=$ & $\frac{8\nu\beta 2(\beta_q - \beta_{q-1})}{2^{(\tau+1) + (\tau+1)q + 2\tau + 20}M K_{q}^{\tau+1}K^\tau}$\\
& $=$ & $\frac{8\nu\beta 2}{2^{(\tau+1) + (\tau+1)q + 2\tau + 19}M K^\tau}\frac{(\beta_q-\beta_{q-1})}{2 K_q^{\tau+1}}$\\
& $=$ & $\frac{\nu \beta}{2^{(\tau+1) + (\tau+1)q + 2\tau + 15} M K^\tau} d_q$\\
& $\leq$ & $\frac{\nu\beta}{2^{3\tau+16}M K^\tau}dq.$
\end{longtable}

Hence, it is enough to prove the following:

$$\frac{16 M}{\mu}\frac{\nu \beta}{2^{3\tau+16} M K^\tau}d_q \leq d_q.$$

Wich holds if and only if

$$\frac{16 M}{\mu}\frac{\nu\beta}{2^{\tau+16} M K^\tau} \leq 1 \Leftrightarrow K^\tau \geq \frac{\nu\beta}{\mu 2^{\tau+12}},$$

which we assumed when choosing $K$.


\end{itemize}

 \end{itemize}

\item Convergence of diffeomorphisms

Now we are going to prove the convergence of the successive maps

$u^{(q)}: G_q \rightarrow F_q$

i.e. we want to see that exist proper sets $G^*, F^*$ and an analytical map $u^*$ such that $u^{(q)}:G_q \rightarrow F_q$ converge to $u^*: G^* \rightarrow F^*$.

Let us use lemma \ref{lemma:inductive} as before.

For $q \geq 1$ we obtain

$$|u^{(q)} - u^{(q-1)}|_{G_q} \leq \xi_q \quad \text{ and } \quad |(u^{(q)})^{-1} -(u^{(q-1)})^{-1}|_{F_q}\leq \frac{\varepsilon_q}{\mu_q}.$$


Now, because the following two inequalities hold

$$
\left\{
\begin{array}{rcl}
\xi_q & \leq & \frac{4M K^{\tau+1}\varepsilon}{\nu \beta 2^{(\tau+2)q}}\\
\frac{\varepsilon_q}{\mu_q} & \leq & \frac{8\varepsilon}{\nu \rho_1 2^{2\tau+2}q} \frac{1}{(1+\frac{1}{2^q})\frac{\mu}{2}} = \frac{8\varepsilon 2^{q-1}}{\nu \beta 2^{(2\tau+2)q}(2^q+1)\mu}
\end{array}
\right.
$$

the sequences $u^{q}$ and $(u^{(q)})^{-1}$ converge to maps $u^*$ and $\Upsilon$ respectively.
These maps are defined on the following sets:

$$
\begin{array}{rcl}
G^* & := & \bigcap_{q\geq 0} G_q,\\
F^* & := & \bigcap_{q\geq 0} F_q = (F - \beta)\setminus \bigcup_{\substack{k\in\mathbb{Z}^n\setminus\{0\} \\ |k|_1 \leq K}}\Delta_{c,\hat q}(k,\frac{\beta}{|k|_1^\tau}).\\

\end{array}
$$

The second equality holds because $F^*$ is a compact for being the intersection of compact sets. We can now deduce that


\begin{longtable}{rcl}
$|u^* - u^{(q)}|_{G^*}$ & $\leq$ & $\sum_{s\geq q} |u^{(q)} - u^{(q-1)}|_{G^*}$ \\
& $\leq$ & $\sum_{s\geq q} |u^{(q)} - u^{(q-1)}|_{G}$ \\
& $\leq$ & $\sum_{s\geq q} \xi_q$. \\

\end{longtable}

with the same argument we see that $|\Upsilon - (u^{(q)})^{-1}|_{F^*} \leq \ldots \leq \sum_{s \geq q} \frac{\varepsilon_q}{\mu_q}$.

The next steps are going to be to prove that $G_q \subset G_{q-1} - \frac{2\varepsilon_{q-1}}{\mu_{q-1}}$ and $F_q \subset F_{q-1} - \frac{4 M_{q-1}\varepsilon_{q-1}}{\mu_{q-1}}$. If we check it and take the limit we would have:
$$G^* \subset G_q - \sum_{s\geq q} \frac{2\varepsilon_q}{\mu_q} \quad \text{ and } \quad F^* \subset F_q - \sum_{s\geq q} \frac{4 M_q \varepsilon_q}{\mu_q}.$$

Let us first check $F_q \subset F_{q-1} - \frac{4 M_{q-1} \varepsilon_{q-1}}{\mu_{q-1}}$. Let us define $x := \frac{4M_{q-1}}{\mu_{q-1}}$.

$$
\begin{array}{rcl}
 F_{q-1} -x& \supset & (F-(\beta_{q-1}+x))\setminus\bigcup_{\substack{k\in\mathbb{Z}^n\setminus\{0\} \\ |k|_1 \leq K}} \Delta_{c,\hat q}(k,\frac{\beta_{q-1}}{|k|_q^\tau} + |k|x) \\
 & \supset & (F-(\beta_{q-1}+x))\setminus\bigcup_{\substack{k\in\mathbb{Z}^n\setminus\{0\} \\ |k|_1 \leq K}} \Delta_{c,\hat q}(k,\frac{\beta_{q-1} + K_q^{\tau+1}x}{|k|_q^\tau}). \\
\end{array}
$$

To have the inclusion we want, we have to check that:

\begin{enumerate}
\item $\beta_{q-1}+x \leq \beta_q.$
\item $\frac{\beta_{q-1} + K_q^{\tau+1}x}{|k|_1^\tau} \leq \frac{\beta_q}{|k|_1^\tau} \Leftrightarrow \beta_{q-1} + K_q^{\tau+1}x.$
\end{enumerate}

Since the second one implies the first we will only check the second one.

\begin{longtable}{rcl}
$\beta_{q-1} + K_q^{\tau+1}x$ & $=$ & $\beta_{q-1} + K_q^{\tau+1}\frac{4M_{q-1}\varepsilon_{q-1}}{\mu_{q-1}}$\\
& $\leq$ & $\beta_{q-1} + K_q^{\tau+1}\frac{16M\varepsilon_{q-1}}{\mu}$\\
& $\leq$ & $\beta_{q-1} + K_q^{\tau+1}d_q$\\
& $=$ & $\beta_{q-1} + K_q^{\tau+1}\frac{\beta_q-\beta_{q-1}}{2K_q^{\tau+1}}$\\
& $=$ & $\beta_{q-1} - \beta_{q-1}/2 + \beta_{q}/2$\\
& $=$ & $\frac{\beta_{q-1} + \beta_{q}}{2}$\\
& $=$ & $\beta_{q}$\\
\end{longtable}

Where we have used that $16M\varepsilon_q/\mu\leq d_q$ and that $\beta_q$ is monotonically increasing with $q$.

The inclusion $G_q \subset G_{q-1} - \frac{2\varepsilon_{q-1}}{\mu_{q-1}}$ is given as a result of the lemma \ref{lemma:inductive}.

So we proved what we wanted. We are now going to see that $u^*$ is one-to-one on $G^*$ and that $u^*(G^*) = F^*$.

Tale $I\in G^*$, we have that $u^{(q)}(I)\in F_q$ for every $q$. Hence $u^*(I) \in F^*$, and we deduce that $u^*(G^*) \subset F^*$. With the same argument, we see $\Upsilon(F^*) \subset G^*$.
Let us prove that $\Upsilon(u^*(I))=I$.

$$
\begin{array}{rcl}
 |\Upsilon(u^*(I)) - I|& \leq & |\Upsilon(u^*(I)) - (u^{(q)})^{-1}(u^* (I)) \\
 & & \quad + (u^{(q)})^{-1}(u^*(I)) - (u^{(q)})^{-1}(u^{(q)}(I))|\\
 & \leq & |\Upsilon(u^*(I)) - (u^{(q)})^{-1}(u^* (I))| \\
 & & \quad + |(u^{(q)})^{-1}(u^*(I)) - (u^{(q)})^{-1}(u^{(q)}(I))|\\
 & \leq & |\Upsilon - (u^{(q)})^{-1}|_{F^*} + \frac{1}{\mu_q}|u^* -u^{(q)}|_{G^*}.\\
\end{array}
$$

Where to bound the second term we used the mean value theorem, i.e. $|u^{(q)}(x) - u^{(q)}(y)|_{G_q} \leq |\frac{\partial}{\partial I} u^{(q)}|_{G_q}|x-y|$, and the fact that because of the $\mu_q$-non-degeneracy, $|\frac{\partial u^{(q)}}{\partial I}| \geq \mu_q|v|$, $\forall v \in \mathbb{R}^n$ and $\forall I' \in G_q$.
Note that we can use the mean value theorem because $u^*(I) - u^{(q)}(I)$ belongs to $F_q$ because $\frac{4 M_q \varepsilon_q}{\mu_q} \geq \xi_q$. Let us prove this inequality. If we want to see $\frac{4 M_q \varepsilon_q}{\mu_q} \geq \xi_q$, it is enough to see $\frac{4 M \varepsilon_q}{\mu} \geq \xi_q$.

\begin{longtable}{rcl}
$\xi_q$ & $\leq$ & $\frac{2}{\delta_2^{(q)}} |R_0^{(q)}|_{G_q,\sigma_2^{(q)}}$\\
& $\leq$ & $\frac{2}{\delta_2^{(q)}} \frac{\delta_1^{(q)}\varepsilon_{q-1}}{2}\frac{1}{2^{2\tau+3}2^{q-1}}$\\
& $=$ & $\frac{1}{c_q}\frac{1}{2^{2\tau+2}2^{q-1}}\varepsilon_{q-1} \leq \frac{4M}{\mu}\varepsilon_{q-1}$\\
\end{longtable}

The last inequality holds true if and only if


\begin{longtable}{rcl}
$\mu$ & $\leq$ & $\frac{\beta 2^{\nu(q-1)}2^{2\tau+3}2^{q-1}}{K_q^{\tau+1}\rho_1 4}$ \\
& $=$ & $\frac{\beta 2^{\nu(q-1)}2^{2\tau+3}2^{q-1}}{K^{\tau+1} 2^{(\tau+1)(q-1)}\rho_1 4}$ \\
& $\leq$ & $\frac{\beta 2^{2\tau+3}}{K^{\tau+1}\rho_1 4}$ \\
& $=$ & $\frac{\beta 2^{2\tau+1}}{K^{\tau+1}\rho_1}$ \\
& $\leq$ & $\frac{\beta 2^{2\tau+1}}{(\frac{1}{\nu \rho_1})^{\tau+1}\rho_1} = \beta \nu^{\tau+1}2^{2\tau+1}\rho_1^tau$ \\
\end{longtable}

as we assumed in the statement of the theorem.
Since the bound obtained tends to 0, we have $\Upsilon(u^*(I)) = I$ and hence $u^*$ is one-to-one. Analogously we obtain $u^*(\Upsilon(J)) = J \quad \forall J \in F^*$. Finally, $u^*$ is one-to-one and $u^*(G^* ) = F^*$. Note also that from the inductive lemma we obtain $|h^{(q)}-h^{(q-1)}|_{G_q,\rho_2^{(q-1)}}\leq \eta_{q-1}$.
Also, observe the following bound that we are going to use in the next sections.

$$|u^* - u^{(q)}|_{G^*} \leq \sum_{s\geq q} \frac{4 M K^{\tau+1}\varepsilon}{\nu\beta 2^{(\tau+2)s}}.$$

\item Convergence of the canonical transformations

Let $\sigma^{(q)} = \rho^{(q-1)} - \delta_2^{(q)}/2$. Observe that this definition implies that $\sigma^{(q)} - \rho^{(q)} = \delta_2^{(q)}$ and $\sigma^{(q)} - \delta_2^{(q)} =  \rho^{(q)}$.
Observe that applying the inductive lemma \ref{lemma:inductive}:

\begin{longtable}{rcl}\label{eq:conv_can_trans}
$|\Phi^{(q)} - \text{id}|_{G_q,\sigma^{(q)}, c_q}$ & $\leq $ & $\frac{2A_{q-1} K_q^{\tau}}{\beta_{q-1}'}\varepsilon_{q-1}$ \\
& $\leq$ & $\frac{2\cdot 5 \cdot 4}{\nu\beta}\frac{8\varepsilon}{\nu\rho_1 2^{(2\tau+2)(q-1)}}$ \\
& $\leq$ & $\frac{2^9 K^\tau \varepsilon}{\nu^2 \rho_1 \beta 2^{(\tau+2)(q-1)}}$ \\
& $\leq$ & $\frac{2^9 K^\tau \nu^3 \rho_1 \beta^2}{\nu^2 \rho_1 \beta 2^{(\tau+2)(q-1)}2^{2\tau+22}M K^{2\tau+1}}$ \\
& $\leq$ & $\frac{2^9 \nu \beta}{2^{(\tau+2)(q-1)}2^{2\tau+20} M K^{\tau+1}}$ \\
& $=$ & $\frac{\nu\beta}{2^6 M (K 2^{q-1})^{\tau+1}}\frac{2^9}{2^{(q-1)}2^{2\tau+14}}$ \\
& $\leq$ & $\delta_2^{(q)}\frac{1}{2^{(q-1)} 2^{2\tau+5}}$ \\
& $\leq$ & $\frac{\delta_2^{(q)}}{2^{(q-1)} 32},$ \\
\end{longtable}

where we have used that $\delta_2^{(q)}\geq \frac{\nu \beta}{84 M K_qP {\tau+1}}$, $\varepsilon \leq \frac{\nu^3 \rho_1 \beta^2}{2^{2\tau+20}M K^{2\tau+1}}$, $\beta \leq \frac{8M K^{\tau+1}\rho_2}{\nu}$ and $\beta_{q-1}' \geq \frac{\nu \beta}{4}$.


Now, recall that $\hat\delta_c = \min(c\delta_1, \delta_2)$, then $\hat{\delta}_{c_q} = \min(c_q \delta_1^{(q)}, \delta_2^{(q)}) = \min(\delta_2^{(q)},\delta_2^{(q)}) = \delta_2^{(q)}$.

Now using that $|D \Upsilon|_{G,\rho-\delta,c} \leq \frac{|\Upsilon|_{G,\rho,c}}{\hat\delta_c}$, we can obtain:

\begin{longtable}{rcl}
$|D\Phi^{(q)} - \text{Id}|_{G_q,\rho^{(q)}, c_q}$ & $=$ & $|D(\Phi^{(q)}) - \text{id}|_{G_q, \rho^{(q)}, c_q}$ \\
 & $\leq$ & $|D(\Phi^{(q)}) - \text{id}|_{G_q, \sigma^{(q)} - \delta_2^{(q)}, c_q}$ \\
 & $\leq$ & $\frac{|\Phi^{(q)}-\text{id}|_{G_q,\sigma^{(q)},c_q}}{\hat\delta_{c_q}}$ \\
 & $\leq$ & $\frac{|\Phi^{(q)}-\text{id}|_{G_q,\sigma^{(q)},c_q}}{\delta^{(q)}_{2}}$ \\
 & $\leq$ & $\frac{2 |\Phi^{(q)} - \text{id}|_{G_q,\sigma^{(q)},c_q}}{\delta_2^{(q)}}$ \\
 & $\leq$ & $\frac{2}{\delta_2^{(q)}}\frac{\delta_2^{(q)}}{2^{(q-1)}\cdot 32} \leq \frac{1}{2^{q-1}16} \leq \frac{1}{2^{(q-1)}4}$ \\
\end{longtable}

Let $x,y$ be such that the segment joining them is contained in $\mathcal{D}_{\rho^{(q)}}(G_q)$. Using the mean value theorem one can deduce the following bound:

$$|\Phi^{q}(x) - \Phi^{q}(y)|_{c_q} \leq |D \Phi^{(q)}|_{G_q,\rho^{(q)}, c_q}\cdot|x-y|_{c_q}.$$

By \ref{eq:conv_can_trans}, in particular $|\Phi^{(q)}(x)-x|_{c_q} \leq \delta_2^{q}$ and $|\Phi^{(q)}(y)-y|_{c_q} \leq \delta_2^{q}$. Then the segment that join $\Phi^{(q)}(x)$ and $\Phi^{(q)}(y)$ is contained in $\mathcal{D}_{\rho^{(q-1)}}(G_{q-1}) = \mathcal{D}_{\rho^{(q)}+\delta^{(q)}}$, because $G_q \subset G_{q-1} - \frac{2 \varepsilon_{q-1}}{\mu_{q-1}}$ and because $\rho^{(q)}-\rho^{(q-1)} \leq \delta_2^{(q)}$ because $\rho^{(q)}-\rho^{(q-1)} = \delta_2^{(q)}$.

Therefore we can apply the mean value theorem once again:

$$
\begin{array}{lcl}
|\Phi^{(q-1)}(\Phi^{(q)}(x)) - \Phi^{(q-1)}(\Phi^{(q)}(y))|_{c_{q-1}} \\
  \leq  |D\Phi^{(q-1)}|_{G_{q-1},\rho^{q-1},c_{q-1}}|\Phi^{(q)}(x) - \Phi^{(q)}(y)|_{c_{q-1}}\\
  \leq  2^{\tau+1-\nu}|D\Phi^{(q-1)}|_{G_{q-1},\rho^{q-1},c_{q-1}}|\Phi^{(q)}(x) - \Phi^{(q)}(y)|_{c_{q}},\\
\end{array}
$$

where we have used that $c_{q-1}/c_q = \frac{\delta_2^{(q-1)}/\delta_1^{(q-1)}}{\delta_2^{(q)}/\delta_1^{(q)}} = \frac{\delta_2^{(q-1)}}{\delta_2^{(q)}}\frac{\delta_1^{(q)}}{\delta_1^{(q-1)}} = 2^{\tau+1}\frac{1}{2^\nu} = 2^{\tau+1-\nu}$.

Using the previous bounds and iterating by $q$, we obtain the following:

\begin{longtable}{lcl}
$|\Psi^{(q)}(x) - \Psi^{(q)}(y)|_{c_1}$ \\
 $\leq$  $2^{(\tau+1-\nu)(q-1)}|D\Phi^{(1)}|_{G_1,\rho^{(1)},c_1}\cdot\ldots\cdot|D\Phi^{(q)}|_{G_q,\rho^{(q)},c_q}|x-y|_{c_q}$ \\
 $\leq$  $2^{(\tau+1-\nu)(q-1)}(1+\frac{1}{4})(1+\frac{1}{4\cdot 2})\cdot\ldots\cdot(1 + \frac{1}{4\cdot 2^{q-1}})|x-y|_{c_q}$\\
 $\leq$  $2^{(\tau+1-\nu)(q-1)}e^{1/2}|x-y|_{c_q} \leq 2^{(\tau+1-\nu)(q-1)}\cdot 2|x-y|_{c_q}.$\\
\end{longtable}

Which holds for $q\geq 1$ and for every $x,y$ such that the segment joining them is contained in $\mathcal{D}_{\rho^{(q)}}(G_q)$. Now, given $q\geq 2$ and $x\in\mathcal{D}_{\rho^{(q)}}(G_q)$ let $y = \Phi^{(q)}(x)$:


\begin{longtable}{rcl}\label{eq:conv_can_trans2}
$|\Psi^{(q)}(x) - \Psi^{(q-1)}(x)|_{c_1}$ & $=$ & $|\Psi^{(q-1)}(\Phi^{(q)}(x)) - \Psi^{(q-1)}(x)|_{c_1}$ \\
& $\leq$ & $2^{(\tau+1+\nu)(q-2)}2|\Phi^{(q)}(x) - x|_{c_{q-1}}$\\
& $\leq$ & $2^{(\tau+1+\nu)(q-1)}2|\Phi^{(q)}(x) - x|_{c_{q}}$\\
& $\leq$ & $2^{(\tau+1+\nu)(q-1)}2\delta_2^{(q)}$\\
& $\leq$ & $2^{(\tau+1+\nu)(q-1)}2\frac{2^8 K^\tau \varepsilon}{\nu^2\rho_1\beta 2^{(\tau+2)(q-1)}}$\\
& $=$ & $\frac{2^9 K^\tau \varepsilon}{\nu^2 \rho_1 \beta 2^{(1+\nu)(q-1)}}.$\\
\end{longtable}

Which holds even for $q=1$ by setting $\Psi^{(0)} = \text{id}$ by \ref{eq:conv_can_trans}. Hence \ref{eq:conv_can_trans2} implies that $\Psi^{(q)}$ converges to a map

$$\Psi^*: D_{(\rho_1/4, 0)}(G^*) = \mathcal{W}_{\frac{\rho_1}{4}}(\mathbb{T}^n)\times G^* \rightarrow \mathcal{D}_\rho(G).$$

And we deduce for every $q\geq 0$ that

$$|\Psi^* -\Psi^{(q)}|_{G^*,(\frac{\rho_1}{4},0),c_1} \leq \frac{2^10 K^\tau \varepsilon}{\nu^2 \rho_1 \beta 2^{(1+\nu)q}}.$$

Moreover by taking the limit to the equation

$$H\circ \Psi^{(q)} = h^{(q)} + R^{(q)}$$

we see that $H\circ \Psi^* = h^*(I)$ on $\mathcal{D}_{(\frac{\rho_1}{4},0)}(G^*)$.

\item Stability estimates

Next we see that for $q\rightarrow \infty$, the motions associated to the transformed hamiltonian $\hat H^{(q)} = \hat h^{(q)} + R^{(q)}$ and the quasi-periodic motions of $\hat h^{(q)}$ get closer and closer.

Let us denote
$$
\left\{
\begin{array}{rcl}
x^{(q)}(t) = ( \phi^{(q)}(t), I ^{(q)}(t)) & \quad & \text{the trajectory of }  H^{(q)},\\
\hat x^{(q)}(t) = (\hat \phi^{(q)}(t),\hat I ^{(q)}(t)) & \quad & \text{the trajectory of } \hat H^{(q)}\\
\end{array}
\right.
$$

corresponding to a given initial condition $x^{(q)}(0) = x_0^* = (\phi_0^*, I_0^* ) \in \mathbb{T}^n\times G_q$. Let

$$
\left\{
\begin{array}{rcl}
\tilde x^{(q)}(t) & := & (\tilde \phi^{(q)}(t), I_0^* ) = (\phi_0^*  + u^{(q)}(I_0^*))t, I_0^*,\\
\hat{ \tilde x}^{(q)}(t) & := &  (\hat{\tilde \phi}^{(q)}(t), I_0^* ) = (\phi_0^*  + u'^{(q)}(I_0^*))t, I_0^*\\
\end{array}
\right.
$$

the corresponding trajectories of the integrable parts of $h^{(q)}$ and $\tilde h^{(q)}$ respectively. Recall that $\hat h^{(q)}(I) = h^{(q)}(I) + \zeta^{(q)}(I_1) = h^{(q)}(I) + q_0 \log(I_1) + \sum_{i=1}^{m-1} q_i\frac{1}{I_1^i}$ and $u'^{(q)} = \bar{\mathcal{B}}u^{(q)}+ \bar{\mathcal{A}}(I_1)$. It is clear that $\tilde x^{(q)}(t)$ and $\hat{\tilde x}^{(q)}(t)$ are defined for all $t\in\mathbb{R}$.

Let us denote:


$ T_q = \inf\{t>0: | I^{(q)}(t) - I_0^* | > \delta_2^{(q+1)} \text{ or } | \phi^{(q)}(t) - {\tilde \phi}^{(q)}(t)|_\infty > \delta_1^{(q+1)}\}.$
$\hat T_q = \inf\{t>0: |\hat I^{(q)}(t) - I_0^* | > \delta_2^{(q+1)} \text{ or } |\hat \phi^{(q)}(t) - \hat{\tilde \phi}^{(q)}(t)|_\infty > \delta_1^{(q+1)}\}.$

Observe that $x^{(q)}(t)$ and $\hat x^{(q)}(t)$ are defined and belong do $\mathcal{D}_{\rho^{(q)}}(G_q)$, for $0\leq t\leq T_q$ and $0\leq t \leq \hat T_q$ respectively, because $\delta^{(q)} \leq \rho^{(q)}$. Also recall the Hamiltonian equations. Let us first state the motion equations for our Hamiltonian function $\hat H^{(q)}$:

$$ \iota_{X_{\hat H^{(q)}}} \omega = d\hat H^{(q)} , \quad \text{ or } \quad X_{\hat H^{(q)}}  = \Pi(d \hat H^{(q)}, \cdot).$$

Let us write

$$X_{\hat H^{(q)}} = \dot{\hat{I}}_1^{(q)}\frac{\partial}{\partial I_1} + \ldots \dot{\hat{I}}_n^{(q)}\frac{\partial}{\partial I_n} + \dot{\hat{\phi}}_1^{(q)}\frac{\partial}{\partial \phi_1} + \ldots + \dot{\hat{\phi}}_n^{(q)}\frac{\partial}{\partial \phi_n}.$$

Moreover

$$
\begin{array}{rcl}
d \hat H^{(q)} & = & d \hat h^{(q)} + dR^{(q)} \\
 & = & d\zeta^{(q)} + dh^{(q)} + dR^{(q)} \\
 & = & \sum_{i=1}^n \frac{\partial \zeta^{(q)}}{\partial I_i} +
\underbrace{\sum_{i=1}^n \frac{\partial \zeta^{(q)}}{\partial \phi_i}}_{=0} +
\sum_{i=1}^n \frac{\partial h^{(q)}}{\partial I_i} \\
&& \quad +
\underbrace{\sum_{i=1}^n \frac{\partial h^{(q)}}{\partial \phi_i}}_{=0} +
\sum_{i=1}^n \frac{\partial R^{(q)}}{\partial I_i} +
\sum_{i=1}^n \frac{\partial R^{(q)}}{\partial \phi_i}.
\end{array}
$$

Recall

$$\omega = \left(\sum_{j=1}^m\frac{c_j}{I_1^j}\right)dI_1\wedge d\phi_1 + \sum_{i=2}^{n} dI_i \wedge d\phi_i, $$
$$ \Pi = \frac{1}{\left(\sum_{j=1}^{m}\frac{c_j}{I_1^J}\right)}\frac{\partial}{\partial I_1} \wedge \frac{\partial}{\partial \phi_1} + \sum_{i=2}^{n} \frac{\partial}{\partial I_i}\wedge\frac{\partial}{\partial \phi_i}.$$

Then:

$$
\left\{
\begin{array}{rcl}
 \dot{\hat{I}}_j^{(q)}  & = &  -\frac{\partial R^{(q)}}{\partial \phi_j}(\hat x^{(q)}(t)), \quad \text{ if }  j \neq 1 \text{ and }\\
 \dot{\hat{I}}_1^{(q)} & = & -\frac{1}{\left(\sum_{i=1}^{m}\frac{c_i}{I_1^i}\right)}\frac{\partial R^{(q)}}{\partial \phi_j}(\hat x^{(q)}(t)) = -\mathcal{B}(I_1) \frac{\partial R^{(q)}}{\partial \phi_1}(\hat x^{(q)}(t)).\\
\end{array}
\right.
$$

Observe that

\begin{equation}\label{eq:stab_est_1}
|\dot{\hat I}_1^{(q)}(t)| \leq \left|\frac{\partial R^{(q)}}{\partial \phi_1}(\hat x^{(q)}(t))\right|.
\end{equation}

Moreover,


$$
\left\{
\begin{array}{rcl}
 \dot{\hat \phi}_j^{(q)} & = & \hat{u}_j^{(q)}(\hat I^{(q)}(t)) + \frac{\partial R^{(q)}}{\partial I_j}(\hat x^{(q)}(t)) \\ 
 &=& u_j^{(q)}(\hat I^{(q)}) + \frac{\partial R^{(q)}}{\partial I_j}(\hat x^{(q)}(t)) \quad \text{ if }  j \neq 1 \\
 \dot{\hat{\phi}}_1^{(q)} & = & \underbrace{(\mathcal{B}(I_1)u_1^{(q)} + \mathcal{A}(I_1))}_{u_1^{(q)}}(\hat I^{(q)}(t)) + \mathcal{B}(I_1)\frac{\partial R^{(q)}}{\partial I_1}(\hat x^{(q)}(t)), \\
\end{array}
\right.
$$

where we have used that $\hat u_j^{(q)} = u_j'^{(q)}$ if $j \neq 1$.
Using \ref{eq:stab_est_1} we obtain

$$|\dot{\hat I}^{(q)}(t)| \leq \left\|\frac{\partial R^{(q)}}{\partial \phi}\right\|_{G_q,\rho^{(q)}} \leq \varepsilon_q.$$

Hence,

\begin{longtable}{rcl}
 $|\dot{\hat \phi}^{(q)} - u'^{(q)}(I_0^*)|_\infty$ & $=$ & $|u'^{(q)}(\hat I^{(q)}(t)) + \bar{\mathcal{B}}\frac{\partial R^{(q)}}{\partial I_1}(\hat x^{(q)}(t)) - u'^{(q)}(I_0^*)|_\infty$\\
 & $\leq $ & $|u'^{(q)}(\hat I ^{(q)})(\hat I^{(q)}(t)) - u'^{(q)}(I_0^* )|_\infty + |\frac{\partial R^{(q)}}{\partial I}(\hat x^{(q)}(t))|_\infty$\\
 & $\leq$ & $M_q'|\hat I^{(q)}(t) - I_0^*| + \|\frac{\partial R^{(q)}}{\partial I}\|_{F_q, \rho^{(q)},\infty}$\\
 & $\leq$ & $M_q|\hat I^{(q)}(t) - I_0^*| + \frac{\varepsilon_q}{c_{q+1}}$\\
 & $\leq$ & $2M\delta_2^{(q+1)} + \frac{\varepsilon_q}{c_{q+1}} \leq 3M\delta_2^{(q+1)}.$\\
\end{longtable}
Where in the last bound we used that

\begin{equation}\label{eq:sta_est_2}
\frac{\varepsilon_q}{c_{q+1}} \leq M \delta_2^{(q+1)},
\end{equation}

that holds because:

\begin{longtable}{rcl}
$\frac{\varepsilon_q}{c_{q+1}}$ & $\leq$ & $\frac{16 M K_{q+1}^{\tau+1} \rho_1}{\beta 2^{\nu q}}\frac{8 \varepsilon}{\nu \rho_1 2^{(2\tau+2)q}}$ \\
 & $\leq$ & $\frac{16 M K_{q+1}^{\tau+1}\rho_1}{\beta e^{\nu q}}\frac{8}{\nu\rho_1 2^{(2\tau+2)q}}\frac{\nu^2 \mu^2 \beta^2}{2^{\tau+30}L^4 M^2 K^{2\tau+2}}$ \\
 & $\leq$ & $\frac{2^7 K_{q+1}^{\tau+1}\nu\mu^2\beta}{2^{(2\tau+2)q + \nu q + \tau + 30} L^4 M^2 K^{2\tau+2}}$\\
 & $\leq$ & $\frac{\nu\beta\mu^2}{K_{q+1}^{\tau+1} 2^{\nu q + \tau + 23} L^4 M^2}$\\
 & $=$ & $\frac{\mu^2}{2^{\nu q + \tau + 17} L^4 M}\frac{\nu\beta}{2^6 M K_{q+1}^{\tau+1}}$\\
 & $\leq$ & $\frac{\mu^2}{2^{\tau+17+\nu q} L^4 M}\delta_2^{(q+1)}$\\
 & $\leq$ & $\frac{2^{2\tau+12}L^4 M^2}{2^{\tau+17+\nu q} L^4 M}\delta_2^{(q+1)}$\\
 & $\leq$ & $2^{\tau-5-\nu q} \delta_2^{(q+1)}$\\
 & $\leq$ & $\frac{2^\tau}{2^{5+\nu q}} M \delta_2^{(q+1)}$ \\
 & $\leq$ & $M \delta_2^{(q+1)}\quad $ if $q$ is large enough.\\
\end{longtable}

Thus, since one of the inequalities defining $\hat T_q$ has to be an equality for $t = T_q$ we obtain,

$$
\begin{array}{rcl}
\delta_2^{(q+1)} & = & |\hat I^{(q)}(T_q) - I_0^*| \leq T_q \varepsilon_q, \quad \text{ or}\\
\delta_1^{(q+1)} & = & |\hat \phi^{(q)}(T_q) - \hat{\tilde \phi}^{(q)}(T_1)|_\infty \leq T_q 3M\delta_2^{(q+1)}.\\
\end{array}
$$

Hence, $\hat T_q \geq \min(\frac{\delta_2^{(q+1)}}{\varepsilon_q},\frac{\delta_1^{(q+1)}}{3 M \delta_2^{(q+1)}}) \geq \frac{1}{3 M c_{q+1}}$, where we used again \ref{eq:sta_est_2}.

Let us denote $T_q' := \frac{1}{3 M c_{q+1}}$, then $\hat T_q \geq T_q'$. This implies

$$|\hat x^{(q)}(t) - \hat{\tilde x}^{(q)}(t)|_{c_{q+1}} \leq \delta_2^{(q+1)} \quad \text{ for }  |t| \leq T_q'.$$

Since $\hat H^{(q)} = \hat H \circ \Psi^{(q)}$ and $\Psi^{(q)}$ is canonical it turns out that $\Psi^{(q)}(\hat x^{(q)}(t))$ is a trajectory of $\hat H$ defined for $t \leq T_q'$. It is important to observe that for $q$ big enough this trajectory remains near the torus $\Psi^{(q)}(\mathbb{T}^n\times \{I_0^*\})$. Moreover $T_q'$ tends to infinity when $q \rightarrow \infty$.

\item Invariant tori

Assume now that $x_0^* \in \mathbb{T}^n\times G^*$ and let us write


$$
\left\{
\begin{array}{rcl}
x^* (t) &  = &  (\phi^*_0 + u^*(I_0^*)t, I_0^*) \\
\hat x^* (t) & = & (\phi^*_0 + u'^*(I_0^*)t, I_0^*)
\end{array}
\right.
\quad \text{ for } t\in\mathbb{R}.
$$

Note that

$$
\begin{array}{rcl}
|\hat{\tilde{x}}^{(q)}(t) - \hat{x}^*(t)|_{c_{q+1}} & \leq & c_{q+1}|u'^{(q)}(I_0^*) - u'^* (I_0^*)|_\infty |t|\\
 & \leq & c_{q+1}|u'^{(q)} - u'^* |_{G^*, \infty} |t|.\\
\end{array}
$$

And observe that if $|t| \leq \frac{\delta_1^{(q+1)}}{|u'^{(q)} - u'^*|_{G^* ,\infty}} =: T_q''$ then,

$$
\begin{array}{rcl}
|\hat{\tilde{x}}^{(q)}(t) - \hat{x}^*(t)|_{c_{q+1}} & \leq &  c_{q+1}|u'^{(q)} - u'^* |_{G^*, \infty} \frac{\delta_1^{(q+1)}}{|u'^{(q)} - u'^*|_{G^* ,\infty}}\\
 & \leq& \frac{\delta_2^{q+1}}{\delta_1^{q+1}}\delta_1^{q+1} = \delta_2^{q+1}.\\
\end{array}
$$

Observe that 
$$|u'^*  - u'^{(q)}|_{G^*} = |\bar{\mathcal{B}} u^* +\bar{\mathcal{A}} - \bar{\mathcal{B}} u^{(q)} - \bar{\mathcal{A}}|_{G^*}$$
$$ = |\mathcal{B}(u^* - u^{(q)})|_{G^*} \leq |u^* -u^{(q)}|_{G^*},$$
 close enough to $Z$.

Hence the bound obtained for $|u^* -u^{(q)}|_{G^*}$ also holds for $|u'^* -u'^{(q)}|_{G^*}$.

$$|u'^* -u'^{(q)}|_{G^*} \leq \sum_{s\geq q} \frac{4 M K^{\tau+1}\varepsilon}{\nu \beta 2^{(\tau+2)s}} \leq \frac{8 M K^{\tau+1}\varepsilon}{\nu\beta 2^{(\tau+2)q}}.$$

Using this bound, we see that $T_q''$ tends to infinity because

$$T_q'' \geq \left(\frac{\nu \rho_1}{8\cdot 2^{\nu q}}\right)\left(\frac{\nu \beta 2^{(\tau+2)q}}{8 M K^{\tau+1}\varepsilon}\right) = \frac{\nu^2 \beta \rho_1}{64 M K^{\tau+1}\varepsilon} 2^{(\tau+2-\nu)q}.$$

Then

$$ |\hat x^{(q)}(t) - \hat x^* (t)|_{c_{q+1}} \leq |\hat x^{(q)}(t) - \hat{\tilde x}^{(q)}(t)|_{c_{q+1}} + |\hat{\tilde x}^{(q)}(t) - \hat x^* (t)|_{c_{q+1}} \leq 2 \delta_2^{(q+1)}.$$

when $t \leq T_q''':= \min(T_q', T_q'')$.

Next, we see that the trajectory $\Psi^{(q)}(x^{(q)}(t))$ is very close to $\Psi^*(x^*(t))$ for large values of $q$. This is true because when $|t| \leq T_q'''$.

\begin{longtable}{lcl}\label{eq:inv_tor_1}
$|\Psi^{(q)}(\hat x^{(q)} - \Psi^* (\hat x^* (t)))|_{c_1}$\\
 $\leq$  $|\Psi^{(q)}(\hat x^{(q)}(t)) - \Psi^{(q)}(\hat x^* (t))|_{c_1} + |\Psi^{(q)}(\hat x^*(t)) - \Psi^* (\hat x^* (t))|_{c_1}$\\
 $\leq$  $2^{(\tau+1-\nu)(q-1)}\cdot 2|\hat x^{(q)}(t) - \hat x^* (t)|_{c_q} + |\Psi^{(q)} - \Psi^*|_{G^* , (\rho_1/4,0),c_1}$\\
 $\leq$  $2^{(\tau+1-\nu)(q-1)}\cdot 4 \delta_2^{(q+1)} + |\Psi^{(q)} - \Psi^*|_{G^*,(\rho_1/4,0),c_1}$\\
 $\leq$  $2^{(\tau+1-\nu)(q-1)}\cdot 4 \delta_2^{(q+1)} + \frac{2^{10} K^\tau \varepsilon}{\nu^2\rho_1\beta 2^{(1+\nu)q}}$\\
 $\leq$  $\frac{c_1}{c_{q+1}}\frac{4\delta_2^{(q+1)}}{2^{(\tau+1-\nu)}} + \frac{2^{10} K^\tau \varepsilon}{\nu^2\rho_1\beta 2^{(1+\nu)q}}$\\
 $\leq$  $\frac{c_1 4}{2^{(\tau+1-\nu)}}\frac{\delta_1^{(q+1)}}{\delta_2^{(q+1)}}\delta_2^{(q+1)} + \frac{2^{10} K^\tau \varepsilon}{\nu^2\rho_1\beta 2^{(1+\nu)q}}$\\
 $\leq$  $\frac{c_1 4}{2^{(\tau+1-\nu)}}\delta_1^{(q+1)} + \frac{2^{10} K^\tau \varepsilon}{\nu^2\rho_1\beta 2^{(1+\nu)q}}$\\
\end{longtable}

where we used that $c_{q-1}/c_q = 2^{\tau+1-\nu}$ then $c_1/c_{q+1} = 2^{(\tau+1-\nu)q}$.

The bound \ref{eq:inv_tor_1} tends to zero. So we deduce, for every fixed $t$, $\Psi^{(q)}(\hat x^{(q)}(t))$ exits or $q$ large enough and its limit is $\Phi^* (\hat x^* (t))$.
This fact and the continuity of the flow of $\hat H$ imply that $\Psi^* (\hat x^* (t))$ is also a trajectory of $\hat H$, which is defined for all $t\in\mathbb{R}$.

This holds for every initial condition $x_0^*=(\phi_0^* , I_0^* ) \in \mathbb{T}^n \times G^* $ for this reason $\Psi^* (\mathbb{T}^n\times\{I_0^*\})$ is an invariant torus of $\hat H$, with frequency vector $u'^*(I_0^* )$. Observe that the energy on the torus is $\hat H(\Psi^* (\phi_0^* , I_0^* )) = h^* (I_0^* )$.

The preserved invariant tori are completely determined by the transformed actions $I_0^* \in G^* $. We are now going to characterize the preserved tori by the original action coordinates.

First, let us see that $u(\hat G) \subset F^* $. Recall that:

$$\Delta_{c,\hat q}(k,\alpha) = \{J \in \mathbb{R} \text{ such that } |k\bar{\mathcal{B}}u(I) + k\bar{\mathcal{A}}| < \alpha \},$$
$$\hat G = \{I \in \mathcal{G} - \frac{2\gamma}{\mu} \text{ such that } |k\bar{\mathcal{B}}u(I) + k\bar{\mathcal{A}}|< \frac{\beta}{|k|_1^\tau} \}.$$

With this definition is obvious that if $I\in \hat G$ then $u(I)$ is $\frac{\beta}{|k|_1^\tau},K,c,\hat q$-non-resonant. Hence $u(I) \notin \Delta_{c,\hat q}(k,\frac{\beta}{|k|^\tau_1})$ for all $k \neq 0$. Then $u(\hat G) \subset F^* $.

We want to find a correspondence between the invariant tori of $\hat h$ and the invariant tori of the perturbed system $\hat H = \hat h + R$, or in the new coordinates $\hat h^*$.

\begin{figure}
\centering
\begin{tikzcd}
\mathcal{D}_\rho(G) = \mathcal{W}_{\rho_1}(\mathbb{T}^n\times\mathcal{V}_{\rho_1}(G)) & & \mathcal{W}_{\rho_1/4}(\mathbb{T}^n)\times G^* \arrow[ll, "\Psi^* "] \arrow[dddd,  shift left=1.5ex, "u^*"]\\\
\mathcal{W}_{\rho_1/4}(\mathbb{T}^n)\times  \arrow[u, hook, "i"] G & & \\
\mathcal{W}_{\rho_1/4}(\mathbb{T}^n)\times \arrow[u, hook, "i"] \arrow[d, "\pi"] \hat G & & \\
\hat G \arrow[d, "u|_{\hat G}"]
 \arrow[uuu, controls={+(5,-2) and +(4.5,0)},end anchor={[yshift=0ex,xshift=-7ex]south east},"\mathcal{T}"]
& & \\
u(\hat G) \subset F \arrow[rr,hook,"i"] & & F^* \arrow[uuuu,"(u^*)^{-1}"]
\end{tikzcd}
\caption{Diagram of the different maps and sets used in the proof.}
\label{fig:diagram_kam}
\end{figure}


Recall 
$$u' = \bar{\mathcal{B}} u + \bar{\mathcal{A}},$$ 
$$u'^* = \bar{\mathcal{B}} u^* + \bar{\mathcal{A}} = (\frac{1}{\sum_{i=1}^m\frac{c_i}{I_1^i}} u_1^* +\frac{\sum_{i=1}^m\frac{\hat q_i}{I_1^i}}{\sum_{i=1}^m\frac{c_i}{I_1^i}}, u_2^*, \ldots, u_n^*).$$
Observe \textcolor{black}{$u'^*(0,I_2,\ldots,I_n) = \frac{\hat q_m}{c_m} = \frac{1}{\mathcal{K}'}$} the inverse of the modular period, hence $u'^*$ and $u'$ are not one-to-one at $Z$ because they project the first component of $u^*$ and $u$ to \textcolor{black}{$\frac{1}{\mathcal{K}'}$}.

Let us define $I_0^* = (u^* )^{-1}(u(I_0))$, recall that $u$ and $u^* $ are indeed one-to-one even though $u'$ and $u'^*$ are not, so $I_0^*$ is properly defined.

With this definition $u^*(I_0^*) = u(I_0)$ and this implies $u'^*(I_0^*) = u'(I_0)$.
Now, let us define $\mathcal{T}(\phi_0,I_0) = \Psi^*(\phi_0, I_0^*)$.

We obtain \ref{eq:kam1} because the set $\mathcal{T}(\mathbb{T}^n\times\{I_0\})$ is an invariant torus of the hamiltonian flow of $\hat H$ with frequency vector $u'^*(I_0^*)$ because $\mathbb{T}^n\times\{I_0^*\}$ is an invariant torus for the hamiltonian flow of $\hat h^*$. And we have seen that  $u'^*(I_0^*) = u'(I_0)$.
In a nutshell, the original frequencies (of the unperturbed system) $u(I_0)$ for $I_0 \in \hat G$ are in $F^*$ and hence can be seen as frequencies of the unperturbed system in the new coordinates $u^* (I_0^*)$. Hence we can conclude that for this $I_0 \in \hat G$ its new (perturbed) solution is also linear in a torus $(\phi_0 + u'^* t, I_0^*) \in \Psi^*(\mathbb{T}^n\times \{I_0^*\}) = \mathcal{T}(\mathbb{T}^n\times\{I_0\})$. And the new frequency vector $u'^* $ is such that $u'^* = u'$.

Let us now prove \ref{eq:kam2}. Let us write, for $(\phi_0,I_0^* ) \in \mathcal{W}_{\frac{\rho_1}{4}}(\mathbb{T}^n) \times G^*$.

$$\Psi^*(\phi_0,I_0^* ) = (\phi_0 + \Psi_\phi^*(\phi_0, I_0^*), I_0^* + \Psi_I^* (\phi_0,I_0^*)).$$

And for $(\phi_0,I_0) \in \mathcal{W}_{\frac{\rho_1}{4}(\mathbb{T}^n)\times \hat G}$.

$$\mathcal{T}(\phi_0,I_0) = (\phi_0 + \mathcal{T}_\phi(\phi_0,I_0), I_0 + \mathcal{T}_I(\phi_0,I_0)).$$

Then, for $(\phi_0,I_0) \in \mathcal{W}_{\frac{\rho_1}{4}(\mathbb{T}^n)\times \hat G}$:

$$ \mathcal{T}_\phi(\phi_0,I_0) = \Psi^*_\phi(\phi_0, I_0^*), \quad \text{ and} \quad \mathcal{T}_I(\phi_0,I_0) = \Psi^*_I(\phi_0,I_0^*) + I_0 - I_0^*.$$

Let us bound the norms of these terms:

$$
\begin{array}{rcl}
 |\Psi_\phi^* (\phi_0,I_0^* )|_\infty & \leq & \frac{1}{c_1}|\Psi^* -\text{id}|_{G^*,(\frac{\rho_1}{4},0),c_1}\\
 & \leq & \frac{16 M K^{\tau+1} \rho_1}{\beta}\frac{2^{10} K^\tau \varepsilon}{\nu^2 \rho_1 \beta}\\
 & \leq & \frac{2^{14} M K^{2\tau+1} \varepsilon}{\nu^2 \beta^2},\\
\end{array}
$$
where we used that $c_1 \geq \frac{\beta}{16 M K^{\tau+1}\rho_1}$. Then,

$$
\begin{array}{rcl}
 \Psi_I^*(\phi_0,I_o^*) & \leq & |\Phi^* - \text{id}|_{G^*,(\frac{\rho_1}{4},0,c_1)}\\
 & \leq & \frac{2^{10} k^\tau \varepsilon}{\nu^2 \rho_1 \beta}. \\
\end{array}
$$

Now it only remains the term $I_0^* -I_0$:

$$|I_0^* - I_0| \leq |(u^*)^{(-1)} - (u)^{(-1)}|_{F^*} \leq \sum_{s\geq 0} \xi_s $$
$$\leq \sum_{s\geq 0} \frac{4 M K^{\tau+1}\varepsilon}{\nu \beta 2^{(\tau+2)s}} \leq \frac{8M K^{\tau+1}\varepsilon}{\nu \beta 2^{(\tau+2)}}.$$

Let us put everything together and use $\hat \rho \leq \nu \rho_1$, $K \leq 2/\hat \rho$ and $\beta = \gamma/L$.


\begin{longtable}{rcl}
 $|\Psi_\phi^* (\phi_0, I_0^* )|_\infty$ & $\leq$ & $\frac{2^{14} M (\frac{2}{\hat \rho})^{2\tau+1}\varepsilon}{\nu^2 (\frac{\gamma}{L})^2}$\\
 &$\leq$& $\frac{2^{2\tau + 15} M L^2}{\nu^2 \hat \rho^{2\tau+1}}\frac{\varepsilon}{\gamma^2}$\\
\end{longtable}

\begin{longtable}{rcl}
 $|\Psi^*(\phi_0,I_0^*)| + |I_0^* - I_0| $ & $\leq$ & $\frac{2^{10}(\frac{2}{\hat \rho})^\tau\varepsilon}{\nu \hat \rho (\frac{\gamma}{L})} + \frac{8M (\frac{2}{\hat \rho})^{\tau+1} \varepsilon}{\nu (\frac{\gamma}{L}) 2^{(\tau+2)}}$\\
 & $=$ & $\frac{2^{10+\tau} L\varepsilon}{\nu \hat \rho^{\tau+1} \gamma} + \frac{8 M 2^{\tau+1} L \varepsilon}{\nu \hat \rho^{\tau+1}\gamma 2^{(\tau+2)}}$\\
 & $\leq$& $\frac{2^{20+\tau} L\varepsilon + M 2^{\tau+4}L\varepsilon}{\nu \hat \rho^{\tau+1}\gamma} \leq \frac{2^{10+\tau} L (1+M)}{\nu \hat \rho^{\tau+1}}\frac{\varepsilon}{\gamma}$\\
\end{longtable}

\item Estimate of the measure

Finally, we carry out the estimate of part \ref{kam:point3}. Let us write $$\hat G^* = (u^*)^{-1}(u(\hat G)).$$
The invariant tori fill the set 
$$\mathcal{T}(\mathbb{T}^n\times \hat G) = \Psi^* (\mathbb{T}^n\times \hat G^*)$$
 i.e. all the tori inside $\mathcal{T}(\mathbb{T}^n\times \hat G)$ are invariant although there are more of them.
Because $\Psi^{(q)}$ are hamiltonian transformations, in particular, they preserve the volume:

$$\text{meas}[\Psi^{(q)}(\mathbb{T}^n\times \hat G^* )] = \text{meas}(\mathbb{T}^n\times \hat G^*) = (2\pi)^{n}\text{meas}(\hat G^*).$$

Now, let us consider the measure of the limit:

$$\text{meas}[\Psi^*(\mathbb{T}^n\times \hat G^*)].$$

To do this we use the superior limit of sets:

$$\bigcap_{n=q}^{\infty} \bigcup_{j=q}^{\infty} (\Psi^{(j)}(\mathbb{T}^n \times \hat G^*)).$$

Because $\Psi^{(j)}(\mathbb{T}^n\times \hat G^*)$ are compact and we have the bound 
$$|\Psi^* - \Psi^{(q)}|_{G^* , (\frac{\rho_1}{4},0), c_1} \leq \frac{2^{10} K^\tau \varepsilon}{\nu^2 \rho_1 \beta 2^{(1+\nu)q}},$$
 $\bigcup_{j=q}^\infty(\Psi^{(j)}(\mathbb{T}^n\times \hat G^*))$ is also compact. All the measures are well-defined and we can say that

$$\text{meas}[\Psi^*(\mathbb{T}^n\times \hat G^* )] \geq (2\pi)^n\text{meas}(\hat G^*).$$

Then, to bound the measure of the complement of the invariant set it is enough to bound the measure of $\mathcal{G}\setminus \hat G^*$.

But first, we are going to define some auxiliary sets. Let $\tilde \beta = \frac{2 \gamma M}{\mu}$, $\tilde\beta_q = (1-\frac{1}{2^{\nu q}})\tilde \beta$. Note that \textcolor{black}{$\tilde \beta \geq \beta$ if and only if $\mu \leq 2ML$ and we assumed $\mu \leq 2^{\tau+6}L^2 M$.}

Then, for $q \geq 0$ we define

$$\tilde F_q = (F - \tilde\beta_q) \setminus \bigcup_{\substack{k\in\mathbb{Z}^n\setminus\{0\} \\ |k|_1 \leq K}} \Delta_{c,\hat q}(k,\frac{\tilde \beta_q}{|k|_1^\tau}), \quad \tilde G_q = (u^{(q)})^{-1}(\tilde F_q)$$

and


$$\tilde F^* = \bigcap_{q\geq 0} \tilde F_q = (F - \tilde\beta) \setminus \bigcup_{\substack{k\in\mathbb{Z}^n\setminus\{0\} \\ |k|_1 \leq K}} \Delta_{c,\hat q}(k,\frac{\tilde \beta}{|k|_1^\tau}), \quad \tilde G^* = \bigcap_{q\geq 0} \tilde G_q.$$

In order to prove the bounds, we need to prove previously the inclusions $\tilde G^* \subset \hat G^*$ and $\tilde G_0 \subset \mathcal{G}$.

\begin{enumerate}
\item $\mathcal{G} \supset \tilde G_0 = (u^{(0)})^{-1}(\tilde F_0) = (u)^{-1}(F-\tilde\beta)$, but we know $u(\mathcal{G}) = F$.

\item $\tilde G^* \subset \hat G^*$. Take $I\in \tilde G^*$, then $I\in \tilde G_q \forall q\geq 0$. Hence $\exists J \in \tilde \tilde F_q \forall q$ such that $u^{(q)}(I)$. Then $\exists J \in \tilde F^*$ such that $u^*(J) = I$.

If we check that $J \in u(\tilde G)$ we obtain $(u^*)^{-1}(J) = I \in \hat G^*$ and we will be done.
We want $\tilde F^* \subset u(\hat G)$. Because we are taking out all the resonances in $\tilde F^*$ it is enough to see $(F - \tilde \beta) \subset u(\mathcal{G} - \frac{2\gamma}{\mu})$. We only need to use that $|\frac{\partial u}{\partial I}|_{\mathcal{G},\rho_2} \leq M$. Then $F - \tilde\beta \subset u(\mathcal{G}-\frac{2\gamma}{\mu})$. This holds if and only if $\frac{\tilde \beta}{M} \leq \frac{2\gamma}{\mu}$ which is true because $\hat \beta \leq \frac{2\gamma M}{\mu}$.

\end{enumerate}

Then, we proceed as follows

\begin{longtable}{rcl}

$\text{meas}(\mathcal{G}\setminus \hat G^*)$ & $\leq$ & $\text{meas}(\mathcal{G}\setminus \tilde G^*)$\\

& $\leq$ & $\text{meas}(\tilde G_0\setminus \tilde G^*)$\\
& $\leq$ & $\sum_{q=1}^{\infty}\text{meas}(\tilde G_{q-1} \setminus  \tilde G_q).$\\

\end{longtable}

For $q \geq 1$ we obtain the following estimate:

$$\text{meas}(\tilde G_{q-1}\setminus \tilde G_q) \leq \frac{1}{|\det(\frac{\partial u^{(q-1)}}{\partial I}(I))|} \text{meas}(\overbrace{\tilde F_{q-1}}^{u^{(q-1)}(\tilde G_{q-1})}\setminus(\overbrace{\tilde F_q - \varepsilon_{q-1})}^{u^{(q-1)}(\tilde G_q)}).$$

Where we have used lemma \ref{lemma:inductive}. Also $\det(\frac{\partial u^{(q-1)}}{\partial I}(I)) \geq \mu_{q-1}^n$ because of the $\mu_{q-1}$-non-degeneracy condition all the eigenvalues have to be greater than $\mu_{q-1}$.

$$
\begin{array}{rcl}
\text{meas}(\tilde G_{q-1} \setminus \tilde G_q) & \leq & \frac{1}{\mu_{q-1}^n}\text{meas}(\tilde F_{q-1} \setminus(\tilde F_q - \varepsilon_{q-1}))\\
 & \leq& \frac{2^n}{\mu^n} \text{meas}(\tilde F_{q-1} \setminus(\tilde F_q - \varepsilon_{q-1})).\\
\end{array}
$$

Now, we are going to apply lemma \ref{lemma:measures_nonresonant} with 
$$\tilde F_{q-1} = F(\tilde \beta_{q-1}, \tilde \beta_{q-1}, K_q-1)$$
and $\tilde F_q = F(\tilde \beta_q, \tilde \beta_q, K_q)$.

%Recall that $|k|_{2,\omega} = \sqrt{\mathcal{B}(I_1)k_1^2 + k_2^2 + \ldots k_n^2}$ and observe that at $Z$, $|k|_{\tau,\omega}$ vanishes for $k$ parallel to $(1,0,\ldots,0)$,

Applying the lemma:

$$
\begin{array}{lcl}
\displaystyle \text{meas}(\tilde F_{q-1} \setminus \tilde F_q)  \leq  \displaystyle D(\tilde \beta_q - \tilde \beta_{q-1})\\
 \displaystyle + 2(\text{diam} F)^{n-1}\left(\sum_{\substack{k\in\mathbb{Z}^n\setminus\{0\} \\ |k|_1 \leq K_{q-1}}} \frac{\tilde \beta_q - \tilde \beta_{q-1}}{|k|_1^\tau |k|_{2,\omega}} + \sum_{\substack{k\in\mathbb{Z}^n\setminus\{0\} \\ K_{q-1} \leq |k|_1 \leq K_{q}}} \frac{\tilde \beta_q}{|k|_1^\tau |k|_{2,\omega}} \right)\\
\end{array}
$$

and

$$
\text{meas}(\tilde F_q \setminus (\tilde F_q - \varepsilon_q)) \leq (D + 2^{n+1}(\text{diam} F)^{n-1} K^n)\varepsilon_q.
$$

Putting everything together (and using that $\tilde \beta_0 = 0$), we get

\begin{equation}\label{eq:final_bound}
\begin{array}{rcl}
 \text{meas}(\mathcal{G} \setminus \hat G^*) & \leq & \displaystyle \frac{2^n}{\mu^n}\left( D \tilde \beta + 2(\text{diam}F)^{n-1}\sum_{\substack{k\in\mathbb{Z}^n\setminus\{0\} \\ }}\frac{\tilde \beta}{|k|_1^\tau|k|_{2,\omega}}\right.\\
 & & \quad \displaystyle \left. + D\sum_{q=1}^\infty \varepsilon_{q-1} + 2^{n+1}(\text{diam} F)^{n-1} \sum_{q=1}^{\infty} K_q^n \varepsilon_{q-1} \right).\\
\end{array}
\end{equation}

We now only have to check that the series  in the previous expression converge. Let us check that they converge at $Z$ first and then outside of $Z$. Recall that at $Z$ we take the vectors $\bar k \neq 0$.

\begin{longtable}{rcl}
$\sum_{\substack{k\in\mathbb{Z}^n\setminus\{0\} \\ \textcolor{black}{\bar k \neq 0}}} \frac{1}{|k|_1^\tau |k|_{2,\omega}}$ & $\leq$ & $\sum_{\substack{k\in\mathbb{Z}^n\setminus\{0\} \\ \textcolor{black}{\bar k \neq 0}}} \frac{1}{|k|_1^\tau|\bar k|}$\\
 & $\leq$ & $\sum_{\substack{\bar k\in\mathbb{Z}^{n-1}\setminus\{0\} \\ \textcolor{black}{\bar k \neq 0}}}\sum_{k_n \in \mathbb{Z}} \frac{\sqrt{n}}{(|\bar k|_1 + |k_n|)^\tau|\bar k|_1}$\\
  & $\leq$ & $\sqrt{n} 2^{n-1} \sum_{j=1}^{\infty}\sum_{k_n \in \mathbb{Z}} \frac{j^{n-3}}{j + |k_n|)^\tau}$\\
\end{longtable}
where we used that the number of vectors $\bar k \in \mathbb{Z}^{n-1}$ with $|\bar k|_1 = j \geq 1$ can be bounded by $2^{n-1} j^{n-2}$. This series can be bounded by comparing it to an integral:

$$\sum_{k_n \in \mathbb{Z}} \frac{1}{(j + |k_n|)^\tau} \leq \frac{1}{j^\tau} + 2 \int_0^\infty \frac{dx}{(j+x)^\tau} $$
$$= \frac{1}{j^\tau} + \frac{2}{(\tau+1) j^{\tau-1}} \leq \frac{\tau+1}{\tau-1}\frac{1}{j^{\tau+1}}.$$

Where we used that $\tau > 1$ because $n\geq 2$. Then

$$\sum_{\substack{k\in\mathbb{Z}^{n}\setminus\{0\} \\ \textcolor{black}{\bar k \neq 0}}} \frac{1}{|k|_1^\tau|k|_{2,\omega}}\leq \frac{\sqrt{n} 2^{n-1}(\tau+1)}{\tau-1} \sum_{j=1}^\infty \frac{1}{j^{\tau-n+2}}$$

which converges by the condition $\tau > n-1$.

Now let us check that it converges outside of $Z$.

$$
\begin{array}{rcl}
\sum_{k\in \mathbb{Z}^n\setminus \{0\}}\frac{1}{|k|_1^\tau |k|_{2,\omega}} & = & \sum_{\substack{k \in \mathbb{Z}^n \setminus \{0\} \\ \bar k \neq 0}} \frac{1}{|k|_1^\tau |k|_{2,\omega}} + \sum_{\substack{k \in \mathbb{Z}^n \setminus \{0\} \\ \bar k = 0}} \frac{1}{|k|_1^\tau |k|_{2,\omega}}\\
& = & \sum_{\substack{k \in \mathbb{Z}^n \setminus \{0\} \\ \bar k \neq 0}} \frac{1}{|k|_1^\tau |\bar k|} + \sum_{k_1 \in \mathbb{Z}} \frac{1}{|k|_1^\tau |k_1^2\mathcal{B}(I_1)^2|}\\
\end{array}
$$


We have seen before that the first term converges.
The second term:

$$\sum_{k_1 \in \mathbb{Z}}\frac{1}{|k_1|^\tau|k_1^\tau \mathcal{B}(I_1)^2|} = \frac{1}{\mathcal{B}(I_1)^2}\sum_{k_1 \in \mathbb{Z}}\frac{1}{k_1^{\tau+2}},$$

which converges $\forall I_1 \neq 0$, i.e. outside of $Z$.

Now we go back to the expression \ref{eq:final_bound}.
The other terms of that expression can be bounded simultaneously inside and outside $Z$.
Now we only have to check that the third series converges, because if the third converges so does the second.
We only have to check that $\sum_{q=1}^\infty K_q^n \varepsilon_{q-1}$ converges. We will use that $\varepsilon \leq \frac{8 \varepsilon}{\nu \rho_1 2^{(2\tau+2)(q-1)}}$.

\begin{longtable}{rcl}
$\sum_{q=1}^\infty K_q^n \varepsilon_{q-1}$ & $=$ & $K^n \sum_{q=1}^\infty 2^{n(q-1)} \varepsilon_{q-1}$ \\
 & $=$ & $K^n \sum_{q=1}^\infty \frac{8\varepsilon 2^{n(q-1)}}{\nu \rho_1 2^{(2\tau+2)(q-1)}}$ \\
  & $=$ & $K^n \frac{8\varepsilon}{\nu \rho_1} \sum_{q=1}^\infty \frac{1}{2^{(2\tau+2-n)(q-1)}}.$ \\
\end{longtable}
Which converges if and only if $2\tau+2-n \geq 1$. And we are done because $2\tau \geq n-1$ since $\tau  \geq n-1$ by hypothesis.

Putting everything together:


$$
\begin{array}{llc}
\text{meas}(\mathcal{G}\setminus \hat G^*) \\
 \leq  \displaystyle \frac{2^n}{\mu^n}\left( D^2\frac{2\gamma M}{\mu} + 2(\text{diam}F)^{n-1}\frac{2\gamma M}{\mu}\frac{\sqrt{n} 2^{n-1} (\tau+1)}{\tau-1} \sum_{j=1}^\infty \frac{1}{j^{\tau - n + 2}}\right.\\
 \displaystyle  D\frac{8\varepsilon}{\nu \rho_1}\sum_{q=1}^{\infty}\frac{1}{2^{}(2\tau+2)(q-1)} \\
\displaystyle \left. + 2^{n+1}(\text{diam} F)^{n-1} K^n \frac{8\varepsilon}{\nu \rho_1} \sum_{q=1}^\infty \frac{1}{2^{(2\tau+2-n)(q-1)}}\right)\\
\end{array}
$$

Now using that

$$
\varepsilon \leq \frac{\nu^2 \mu^2 \beta^2}{2^{\tau+30} L^4 M^3 K^{2\tau+1}} \leq \frac{2^{\tau-18}\cdot 8 M K^{\tau+1} \rho_2}{L M K^{2\tau+2}}\gamma \leq \frac{2^{\tau-15}\rho_2}{L K^{\tau+1}}\gamma
$$

We can write $\text{meas}(\mathcal{G}\setminus \hat G^*) \leq C' \gamma$ where $C'$ depends only on $n$,  $\mu$,  $D$,  $\text{diam} F$,  $M$,  $\tau$,  $\rho_1$,  $\rho_2$,  $L$,  $K$ and if we efine $C = (2\pi)^n C'$. Hence,

$$\text{meas}[(\mathbb{T}^n \times \mathcal{G}) \setminus \mathcal{T}(\mathbb{T}^n \hat G)] \leq C \gamma.$$



\end{enumerate}
\end{proof}


\chapter[Desingularization of $b^m$-integrable systems]{Desingularization of $b^m$-integrable systems}

In this chapter, we follow \cite{GMW17}, for the definition of the desingularization of the $b^m$-symplectic form.

\begin{definition}The \textbf{$f_\epsilon$-desingularization} $\omega_\epsilon$ form of $\omega= \frac{dx}{x^{m}}\wedge \left(\sum_{i=0}^{m-1}x^i\alpha_{m-i}\right) + \beta$ is:
$$\omega_\epsilon = df_\epsilon \wedge \left(\sum_{i=0}^{m-1}x^i\alpha_{m-i}\right) + \beta.
$$
Where in the even case, $f_\epsilon(x)$ is defined as $\epsilon^{-(2k -1)}f(x/\epsilon)$.
And $f \in \mathcal{C}^\infty(\mathbb{R})$ is an odd smooth function satisfying $f'(x) > 0$ for all $x \in \left[-1,1\right] $ and satisfying outside that
\begin{equation}
f(x) = \begin{cases}
\frac{-1}{(2k-1)x^{2k-1}}-2& \text{for} \quad x < -1,\\
\frac{-1}{(2k-1)x^{2k-1}}+2& \text{for} \quad x > 1.\\
\end{cases}
\end{equation}
And in the odd case, $f_\epsilon(x) = \epsilon^{-(2k)}f(x/\epsilon)$. And $f \in \mathcal{C}^\infty(\mathbb{R})$ is an even smooth positive function which satisfies: $f'(x) < 0$ if $x < 0$, $f(x) = -x^2 + 2$ for $x \in [-1,1]$, and
\begin{equation}
f(x) = \begin{cases}
\frac{-1}{(2k+2)x^{2k+2}}-2& \text{if } k > 0, x \in \mathbb{R}\setminus[-2,2]\\
\log(|x|)& \text{if } k = 0, x \in \mathbb{R}\setminus[-2,2].\\
\end{cases}
\end{equation}
\end{definition}

\begin{remark}
With the previous definition, we obtain smooth symplectic (in the even case) or smooth folded symplectic (in the odd case) forms that agree outside an $\epsilon$-neighbourhood with the original $b^m$-forms. Moreover, there is a convergence result in terms of $m$. See \cite{GMP17} for the details.
\end{remark}

To simplify notation, we introduce $F_{\epsilon}^{m-i}(x) = (\frac{d}{dx}f_\epsilon(x))x^i$, and hence $F_{\epsilon}^{i}(x) = (\frac{d}{dx}f_\epsilon(x))x^{m-i}$. With this notation the desingularization $\omega_\epsilon$ is written:

$$\omega_\epsilon = \sum_{i = 0}^{m-1} F_{\epsilon}^{m-i}(x) dx\wedge \alpha_{m-i} + \beta. $$


\begin{definition}
The desingularization for $(M,\omega,\mu)$ is the triple $(M,\omega_\varepsilon,\mu_\epsilon)$ where $\omega_\varepsilon$ is defined as above and $\mu_\varepsilon$ is:

$$
\mu    \mapsto \mu_\epsilon = \left(f_{1\epsilon} = \sum_{i = 1}^{m} \hat{c}_i G_{\epsilon}^i(x), f_2(\tilde I,\tilde \phi), \ldots, f_n(\tilde I,\tilde \phi)\right),
$$

where 
$$\mu = \left(f_1 = c_0 \log(x) + \sum_{i = 1}^{m-1} c_i \frac{1}{x^i},f_2(I,\phi) \ldots, f_n(I,\phi)\right)$$ 
$$G_\epsilon^i(x) = \int_0^x F_\epsilon^i(\tau)d\tau,$$
and $\hat{c}_1 = c_0$ and $\hat{c}_{i-1} = -ic_i$ if $i \neq 0$. Also

$$
\left\{
\begin{array}{lrcl}
\tilde I = (\tilde I_1, I_2,\ldots, I_n), &\quad \tilde I_1 & = & \int_0^{I_1}\left(\frac{\sum_{i=1}^m \mathcal{K} \hat c_i F_\varepsilon^i(\tau)}{\sum_{i=1}^m \frac{\mathcal{K} \hat c_j}{\tau^j}}\right) d\tau \\

\tilde \phi_1 = (\tilde \phi_1, \phi_2,\ldots, \phi_n), & \quad \tilde \phi_1 & = & \left(\frac{\sum_{i=1}^m \mathcal{K} \hat c_i F_\varepsilon^i(I_1)}{\sum_{i=1}^m \frac{\mathcal{K} \hat c_j}{I_1^j}}\right) \phi_1
\end{array}
\right.
$$

\end{definition}

\begin{remark}
Observe that with the last definition, when $\epsilon$ tends to $0$, $\mu_\epsilon$ tends to $\mu$.
\end{remark}

\begin{theoremC}
The desingularization transforms a $b^m$-integrable system into an integrable system for $m$ even on a symplectic manifold. For $m$ odd the desingularization transforms it into a folded integrable system. The integrable systems are such that:
$$X_{f_j}^\omega = X_{f_{j\epsilon}}^{\omega_\epsilon}.$$
\end{theoremC}

\begin{proof}
Let us first check the singular part, i.e. let us check that that $X_{f_1}^\omega = X_{f_{1\epsilon}}^{\omega_\epsilon}$. Let us compute the two equations that define each one of the vector fields. We have to impose $-df_1 = \iota_{X_{f_1}^\omega}\omega$ and $-df_{1\epsilon} = \iota_{X_{f_{1\epsilon}}^{\omega_\epsilon}}\omega_\epsilon$.
But observe first that we can rewrite $\omega = \sum_{i=1}^m \frac{1}{x^i}dx\wedge\alpha_i + \beta$ and $\omega_\epsilon = \sum_{i=1}^m F_\epsilon^idx\wedge\alpha_i + \beta$. The conditions translate as:


$$-\sum_{i = 1}^{m}\hat{c}_{i} \frac{1}{x^i}dx = \iota_{X_{f_1}^\omega}\left(\sum_{i = 1}^{m} \frac{1}{x^i} dx\wedge \alpha_i + \beta\right),$$
$$-\sum_{i = 0}^{m-1}\hat{c}_i F_\epsilon^i(x)dx = \iota_{X_{f_{1\epsilon}}^{\omega_\epsilon}} \left(\sum_{i = 0}^{m-1} F_{\epsilon}^i(x) dx\wedge \alpha_i + \beta\right).$$

Since the toric action leaves the form $\omega$ invariant, in particular, the singular set is invariant, and then $X_{f_{1\epsilon}}^{\omega_\epsilon}$ and $X_{f_{1}}^{\omega}$ are in the kernel of $dx$. Moreover, since $\beta$ is a symplectic form in each leaf of the foliation and $X_{f_{1\epsilon}}^{\omega_\epsilon}$ and $X_{f_{1}}^{\omega}$ are transversal to this foliation, they are also in the kernel of $\beta$.

$$-\sum_{i = 0}^{m-1}\hat{c}_i \frac{1}{x^i}dx = \sum_{i = 0}^{m-1} \frac{1}{x^i} dx\wedge \alpha_i(X_{f_1}^\omega),$$
$$-\sum_{i = 0}^{m-1}\hat{c}_i F_\epsilon^i(x)dx = \sum_{i = 0}^{m-1} F_{\epsilon}^i(x) dx\wedge \alpha_i(X_{f_{1\epsilon}}^{\omega_\epsilon}).$$

Then, the conditions over $X_{f_1}^\omega$ and $X_{f_{1\epsilon}}^{\omega_\epsilon}$ are respectively:

$$-\hat{c}_i = \alpha_i(X_{f_1}^\omega),$$
$$-\hat{c}_i = \alpha_i(X_{f_{1\epsilon}}^{\omega_\epsilon}).$$

Then, the two vector fields have to be the same.

Let us now see $X_{f_j}^\omega = X_{f_{j\epsilon}}^{\omega_\epsilon}$ for $j > 1$.
Assume now we have the $b^m$-symplectic form in action-angle coordinates $\omega = \sum_{i=1}^m \frac{\mathcal{K}\hat c_i}{I_1^i}dI_1\wedge d\phi_1 + \sum_{i=1}^n dI_i\wedge d\phi_i$.

The differential of the functions are

$$
\begin{array}{rcl}
d f_i^\varepsilon & = & \frac{\partial f_i^\varepsilon}{\partial I_1} dI_1 + \frac{\partial f_i^\varepsilon}{\partial \phi_1} d\phi_1 + \sum_{j= 2}^n\left(\frac{\partial f_i^\varepsilon}{\partial I_j} dI_j + \frac{\partial f_i^\varepsilon}{\partial \phi_j} d\phi_j\right)\\
 & = & \frac{\partial f_i}{\partial I_1}\left(\frac{\sum_{i=1}^m \mathcal{K} \hat c_i F_\varepsilon^i(\tau)}{\sum_{i=1}^m \frac{\mathcal{K} \hat c_j}{\tau^j}}\right) dI_1 + \frac{\partial f_i}{\partial \phi_1}\left(\frac{\sum_{i=1}^m \mathcal{K} \hat c_i F_\varepsilon^i(\tau)}{\sum_{i=1}^m \frac{\mathcal{K} \hat c_j}{\tau^j}}\right) d\phi_1 \\
 & & \quad + \sum_{j= 2}^n\left(\frac{\partial f_i^\varepsilon}{\partial I_j} dI_j + \frac{\partial f_i^\varepsilon}{\partial \phi_j} d\phi_j\right).\\
\end{array}
$$

On the other hand, the desingularized form is:

$$\omega^\varepsilon = \sum_{j=1}^m \mathcal{K} \hat c_i F_\varepsilon^j (I_1) d I_1 \wedge d\phi_1 + \sum_{j=2}^m dI_j \wedge d\phi_j.$$

Hence, one can see that the expression for both $X_{f_j}^\omega$ and $X_{f_{j\epsilon}}^{\omega_\epsilon}$ is

$$X_{f_j}^\omega = X_{f_{j\epsilon}}^{\omega_\epsilon} = \frac{\frac{\partial f_i}{\partial I_1}}{\sum_{i=1}^m\frac{\mathcal{K}\hat c_i}{I_1^i}}\frac{\partial}{\partial \phi_1} - \frac{\frac{\partial f_i}{\partial \phi_1}}{\sum_{i=1}^m\frac{\mathcal{K}\hat c_i}{I_1^i}}\frac{\partial}{\partial I_1} +\sum_{j= 2}^n\left(\frac{\partial f_i^\varepsilon}{\partial I_j} dI_j + \frac{\partial f_i^\varepsilon}{\partial \phi_j} d\phi_j\right)$$

\end{proof}

\begin{remark}
The previous lemma tells us that the dynamics of the desingularized system are identical to the dynamics of the original $b^m$-integrable system in the $b^m$-symplectic manifold.
\end{remark}

Hence the desingularized $b^m$-form goes to a folded symplectic form in the case $m=2k+1$ and to symplectic for $m=2k$. And the $b^m$-integrable system goes to a folded integrable system (see \cite{EvaRobert}) in the case $m=2k+1$ and to a standard integrable system for $n=2k$.

%\begin{remark} The previous lemma works for any $\epsilon$ small enough: we are not taking the limit. In this way, we have constructed a set of smooth hamiltonian vector fields in a $b^m$-symplectic manifold with a $b^m$-integrable system. Moreover these vector fields coincide with the hamiltonian vector fields associated to the desingularized symplectic manifold and the desingularized integrable system. Then, we can see this our $b^m$-integrable system as an integrable system in the standard sense. Hence, we can apply to it results for standard integrable systems, for instance the KAM theorem.
%\end{remark}



\chapter[$b^m$-KAM Desingularization]{Desingularization of the KAM theorem on $b^m$-symplectic manifolds}

The idea of this section is to recover some version of the classical KAM theorem by ``desingularizing the $b^m$-KAM theorem", as well as a new version of a KAM theorem that works for folded symplectic forms. Observe that no KAM theorem is known for folded symplectic forms. The best that is known is a KAM theorem for presymplectic structures that was done in \cite{presymplectic}. Desingularizing the KAM means applying the $b^m$-KAM in the $b^m$-manifold and then translating the result to the desingularized setting.

To be able to obtain proper desingularized theorems we need to identify which integrable systems can be obtained as a desingularization of a $b^m$-integrable system. To simplify computations we are going to use a particular case of $b^m$-integrable systems, where $f_1=\frac{1}{I_1^{m-1}}$. We call these systems \emph{simple}. Observe that by taking a particular case of $b^m$-integrable systems we will not get all the systems that can be obtained by desingularizing a $b^m$-integrable system, but some of them.

\begin{enumerate}
\item \textbf{Even case} $m = 2k$.

$F=(f_1 = \frac{1}{I_1^{2k-1}},f_2 ,\ldots, f_n)$, $\omega = \frac{1}{I_1^m} d I_1 \wedge d\phi_1 + \sum_{j=1}^n dI_j \wedge d\phi_j$.
Observe that close to $Z$ in the even case we can assume $f(I_1) = c I_1$ for some $c > (2 -\frac{1}{2^{2k-1}})$. Then $f_\varepsilon(I_1) = \frac{1}{\varepsilon^{(2k-1)}}\frac{c I_1}{\varepsilon}= c' I$, hence $\omega_\varepsilon = c' dI_1\wedge d\phi_1 + \sum_{j=1}^n dI_j \wedge d\phi_j$. Also $F_\varepsilon^m(I_1) = c'$, $G_\varepsilon^m(I_1) = c' I_1$.
Then,

$$
\left\{
\begin{array}{rcl}
\tilde I_1 & = & \int_0^{I_1}\frac{c'}{1/\tau^m}d \tau = \int_0^{I_1} c' \tau^m d\tau = c'\frac{I_1^{m+1}}{m+1},\\
\tilde \phi_1 & = & \frac{c'}{1/I_1^m}\phi_1 = c' I_1^m \phi_1
\end{array}
\right.
$$

\begin{equation}\label{eq:desingularized_even}
F^\varepsilon = ((m-1)c_{m-1}c' I_1, f_2(\tilde I, \tilde \phi), \ldots f_n(\tilde I, \tilde \phi)).
\end{equation}

Hence, the systems in this form can be viewed as a desingularization of a $b^m$-integrable system.

\begin{theoremD}[Desingularized KAM for symplectic manifolds]
Consider a neighborhood of a Liouville torus of an integrable system $F_\varepsilon$ as in \ref{eq:desingularized_even} of a symplectic manifold $(M, \omega_\varepsilon)$ semilocally endowed with coordinates $(I,\phi)$, where $\phi$ are the angular coordinates of the torus, with $\omega_\varepsilon = c' dI_1 \wedge d\phi_i + \sum_{j= 1}^n dI_j\wedge d\phi_j$. Let $H=(m-1)c_{m-1}c' I_1 + h(\tilde I) + R(\tilde I,\tilde \phi)$ be a nearly integrable system where
$$
\left\{
\begin{array}{rcl}
\tilde I_1 & = & c'\frac{I_1^{m+1}}{m+1},\\
\tilde \phi_1 & = & c' I_1^m \phi_1 ,
\end{array}
\right.
$$
and
$$
\left\{
\begin{array}{rcl}
\tilde I & = & (\tilde I_1, I_2, \ldots, I_n),\\
\tilde \phi & = & (\tilde \phi_1, \phi_2, \ldots, \phi_n).
\end{array}
\right.
$$
Then the results for the $b^m$-KAM theorem \ref{th:bm_kam} applied to $H_{\text{sing}} = \frac{1}{I_1^{2k-1}} + h(I) + R(I,\phi)$ hold for this desingularized system.
\end{theoremD}

\begin{remark} This theorem is not as general as the standard KAM, but we also know extra information about the dynamics. For instance, the perturbation of trajectories in tori inside of $Z$ will be trajectories lying inside of $Z$. In this sense, the theorem is new because it leaves invariant an hypersurface of the manifold.
\end{remark}

\item \textbf{Odd case} $m = 2k+1$.

$F = (f_1 = \frac{1}{I_1^{2k}}, f_2,\ldots,f_n)$ and $\omega = \frac{1}{I_1^{2k+1}} dI_1 \wedge d\phi_1 + \sum_{j=1}^n dI_j\wedge d \phi_j$.
Before continuing we need the following notions defined in \cite{EvaRobert}.

\begin{definition}
A function $f:M\rightarrow \mathbb{R}$ in a folded symplectic manifold $(M,\omega)$ is folded if $df|_Z(v)=0$ for all $v\in V=\textup{ker}\omega|_Z$.
\end{definition}

\begin{definition}
An integrable system in a folded symplectic manifold $(M,\omega)$ with critical surface $Z$ is a set of functions $F =(f_1,\ldots,f_n)$ such that they define Hamiltonian vector fields which are independent ($df_1\wedge \ldots \wedge d f_n \neq 0$ in the folded cotangent bundle) on a dense subset of $Z$ and $M$, and commute with respect to $\omega$.
\end{definition}

Note that we need to prove that the desingularized functions in this case are folded.

Observe that close to $Z$ in the odd case we can assume $f(I_1) = -I_1^2 + 2$. Then $f_\varepsilon(I_1) = \varepsilon^{-(2k)} f(\frac{I_1}{\varepsilon}) = \frac{1}{\varepsilon^{2k}}(-(\frac{I_1}{\varepsilon})^2 + 2) = c I_1^2 + \frac{2}{\varepsilon^{2k}}$. Then

$$\omega_\varepsilon = 2cI_1 dI_1 \wedge d\phi_1 + \sum_{j=1}^n dI_j \wedge d\phi_j.$$

Also $F_\varepsilon^m(I_1) = 2c I_1$, $G_\varepsilon^m(I_1) = cI_1^2$. Then,

$$
\left\{
\begin{array}{rcl}
\tilde I_1 & = & \int_0^{I_1}\frac{2 c \tau}{1/\tau^m} d\tau = 2c \frac{I_1^{(m+2)}}{(m+2)},\\
\tilde \phi_1 & = & 2c I_1^{m+1}\phi_1
\end{array}
\right.
$$

Then the desingularized moment map becomes

\begin{equation}\label{eq:desingularized_odd}
F^\varepsilon = ((m-1)c_{m-1}c I_1^2, f_2(\tilde I, \tilde \phi), \ldots f_n(\tilde I, \tilde \phi)).
\end{equation}

It is a simple computation to check that these functions are actually folded and hence they form a folded integrable system. Note that the systems of the form \ref{eq:desingularized_odd} can be viewed as a desingularization of a $b^m$-integrable system. Then, as we proceeded in the even case:

\begin{theoremE}[Desingularized KAM for folded symplectic manifolds]
Consider a neighborhood of a Liouville torus of an integrable system $F_\varepsilon$ as in \ref{eq:desingularized_odd} of a folded symplectic manifold $(M, \omega_\varepsilon)$ semilocally endowed with coordinates $(I,\phi)$, where $\phi$ are the angular coordinates of the Torus, with $\omega_\varepsilon = 2cI_1 dI_1 \wedge d\phi_1 + \sum_{j=2}^m dI_j \wedge d\phi_j$.
Let $H = (m-1)c_{m-1} cI_1^2 + h(\tilde I) + R(\tilde I, \tilde \phi)$ a nearly integrable system with
$$
\left\{
\begin{array}{rcl}
\tilde I_1 & = &  2c\frac{I_1^{m+2}}{m+2},\\
\tilde \phi_1 & = &  2c I_1^{m+1} \phi_1 ,
\end{array}
\right.
$$
and
$$
\left\{
\begin{array}{rcl}
\tilde I & = & (\tilde I_1, I_2, \ldots, I_n),\\
\tilde \phi & = & (\tilde \phi_1, \phi_2, \ldots, \phi_n).
\end{array}
\right.
$$
Then the results for the $b^m$-KAM theorem \ref{th:bm_kam} applied to $H_{\text{sing}} = \frac{1}{I_1^{2k}} + h(I) + R(I,\phi)$ hold for this desingularized system.
\end{theoremE}

\begin{remark}
The last two theorems can be improved if we consider $b^m$-integrable systems not necessarily \emph{simple}.
\end{remark}



\end{enumerate}


\chapter{ Applications to Celestial mechanics}\label{ch:binvitation}

The theory presented in this monograph establishes a formalized approach to perturbation theory in singular situations that manifest in real-world physical systems. We offer numerous illustrations from Celestial mechanics to support this theory, and conclude by outlining the potential applications of our KAM theory in detecting periodic trajectories.



Within this chapter, we showcase a variety of examples from Celestial Mechanics that involve the occurrence of singular symplectic forms. A few of these instances are elaborated upon in \cite{binvitation}. The majority of these singularities arise as a result of implementing "regularization" methods. We encourage readers to consult the book \cite{knauf} for a pedagogical approach to the study of regularization. 

This compilation of examples is of particular significance for this booklet, as the theoretical outcomes we attain, such as action-angle coordinates or KAM, can be readily employed in the range of problems outlined below.




Structures that are symplectic almost everywhere can emerge from coordinate transformations  that do not preserve the canonical symplectic structure. 

For instance: For the Kepler problem given a configuration space $\mathbb{R}^2$ and phase space $T^*\mathbb{R}^2$,  the traditional (canonical) Levi-Civita transformation described as follows: identify $\mathbb{R}^2\cong\mathbb{C}$ so that $T^*\mathbb{R}^2\cong T^*\mathbb{C}\cong \mathbb{C}^2$ and treat $(q,p)$ as complex variables $(q_1+iq_2:=u,p_1+ip_2:=v)$. Take the following change of coordinates $(q,p) = (u^2/2, v/\bar{u})$, where $\bar{u}$ denotes the complex conjugation of~$u$. The resulting coordinate change can easily be seen to preserve the canonical symplectic form. 
Nevertheless, this canonical transformation can increase the complexity of the Hamiltonian equations, making it more challenging to analyze the dynamical aspects of the system. Hence, it becomes intriguing to explore alternative coordinate transformations that do not preserve the symplectic form. Some of them induce new singular forms where our geometrical and dynamical techniques can be applied.

Other examples are discussed in \cite{DKM17}.

\section{The Kepler Problem}

In suitable coordinates in $T^*\left(\mathbb{R}^2\setminus\{0\}\right)$, the Kepler problem has Hamiltonian
\begin{equation}
 H(q,p)=\frac{\|p\|^2}{2}-\frac{1}{\|q\|}.
\end{equation}
With the canonical Levi-Civita transformation $(q,p) = (u^2/2, v/\bar{u})$, this expression becomes
\begin{equation}
 H(u,v)=\frac{\|v\|^2}{2\|\bar{u}\|^2}-\frac{1}{\|u\|^2}.
\end{equation}

 

To avoid the complications arising from preserving the canonical symplectic form, we suggest an alternative approach. By retaining the momentum as is, we can consider a transformation of the form $(q,p) = (u^2/2, p)$, which may lead to a simpler Hamiltonian. However, this transformation is not a symplectomorphism, and as a result, the symplectic form on $T^*\mathbb{R}^2$ pulls-back under the transformation to a two-form  which is symplectic almost everywhere but degenerates on a hypersurface of $T^*\mathbb{R}^2$


Namely, the Liouville one-form $p_1 dq_1+p_2 dq_2=\Re(p d\bar{q})$ pulls back to
\begin{eqnarray*}
    \theta=\Re\left(p d\left(\frac{\bar{u}^2}{2}\right)\right)&=&\Re\left(p \bar{u}d\bar{u}\right)\\
    &=&p_1(u_1du_1-u_2du_2)+p_2(u_2du_1+u_1du_2)
\end{eqnarray*}
and the associated $2$-form  $-d\theta$ yields a form that is almost everywhere symplectic 
\begin{eqnarray*}
   \omega=u_1du_1\wedge dp_1 - u_2 du_1 \wedge dp_2 + u_2 du_2 \wedge dp_1 + u_1 du_2\wedge dp_2.
\end{eqnarray*}

To examine the characteristics of this form, we take its wedge product with itself, yielding:
\begin{eqnarray*}
   \omega\wedge\omega=(u_1^2-u_2^2) du_1 \wedge dp_1 \wedge du_2\wedge dp_2
\end{eqnarray*}
which is degenerate along the hypersurface given by $u_1=\pm u_2$.

We now consider the restriction of the form to the critical set. It does not have maximal rank so it is not a folded symplectic structure.
This form is  degenerately folded and the folding hypersurface is not regular and is described by the equations $u_1=\pm u_2$.

\section{The Problem of Two Fixed Centers}

We now regularize the problem of two fixed centers. 

The problem of two fixed centers is associated to the motion of a satellite moving in a gravitational potential generated by two fixed massive bodies. Additionally, we assume that the satellite's motion is limited to a plane in $\mathbb{R}^3$ that includes the two massive bodies.

The Hamiltonian function in suitable coordinates reads:
\begin{equation}
    H=\frac{p^2}{2m}-\frac{\mu}{r_1}-\frac{1-\mu}{r_2}
\end{equation}
where $\mu$ is the mass ratio of the two bodies (i.e. $\mu=\frac{m_1}{m_1+m_2})$.

Euler was the first to demonstrate the integrability of this problem, using \emph{elliptic} coordinates in which the coordinate lines are confocal ellipses and hyperbolas.

Explicitly, consider a coordinate system in which the two centers are placed at $(\pm 1, 0)$, in which the (Cartesian) coordinates are given by $(q_1, q_2)$. Then the elliptic coordinates of the system are given by
\begin{align}
q_1&=\sinh\lambda\cos\nu\\
q_2&=\cosh\lambda\sin\nu
\end{align}
for $(\lambda, \nu)\in \mathbb{R}\times S^1$. Thus lines of $\lambda=c$ and $\nu=c$ are given by confocal hyperbola and ellipses in the plane, respectively. Similar to the Levi-Civita transformation this results in a double-branched covering with branch points at the centers of attraction.

Pulling back the canonical symplectic structure $\omega=dq \wedge dp$ we find
\begin{equation}
\omega= \cosh\lambda \cos\nu (d\lambda \wedge dp_1+d\nu \wedge dp_2) -\sinh\lambda\sin\nu(d\nu \wedge dp_1 + d\lambda \wedge d p_2)
\end{equation}
which is degenerate along the hypersurface $(\lambda,\nu)$ satisfying $\cosh\lambda\cos\nu=\sinh\lambda\sin\lambda$.

\section{Double Collision and McGehee coordinates}\label{Sec:Doublecollision}

In this section, we describe another example of $b$-symplectic structure appearing quite naturally in physical dynamical systems. From this example, it would seem natural that a collection of different examples for $b^m$-symplectic models or even $b^m$-folded models would follow. But one finds a major problem while pursuing these examples.  By examining why this example cannot be extended to construct $b^m$-symplectic models or $b^m$-folded models for any $m$, we can identify a general pattern.

First, let us introduce the McGehee coordinate change for the problem of double collision.

The system of two particles moving under the influence of the generalized potential $U(x) = -|x|^{-\alpha}$, $\alpha > 0$, where $|x|$ is the distance between the two particles, is studied by McGehee in \cite{McGehee}. We fix the center of mass at the origin and hence can simplify the problem to the one of a single particle moving in a central force field.


The equation of motion can be written as,
\begin{equation}
\ddot{x} = -\nabla U(x) = -\alpha |x|^{-\alpha-2}x
\end{equation}
where the dot represents the derivative with respect to time. In the Hamiltonian formalism, this equation becomes
\begin{equation}
\begin{array}{rcl}
\dot{x} & = &  y, \\
\dot{y} & = & -\alpha |x|^{-\alpha-2}x.
\end{array}
\end{equation}
To study the behavior of this system, the following change of coordinates is suggested in \cite{McGehee}:
\begin{equation}\label{eq:mcgeheechange}
\begin{array}{rcl}
x & = & r^\gamma e^{i\theta}, \\
y & = & r^{-\beta\gamma}(v + iw)e^{i\theta}
\end{array}
\end{equation}
where the parameters $\beta$ and $\gamma$ are related with $\alpha$ as follows:
\begin{equation}\label{eq:relations}
\begin{array}{rcl}
\beta & = & \alpha/2, \\
\gamma & = & 1/(1 + \beta).
\end{array}
\end{equation}
Identifying the plane $\mathbb{R}^2$ with the complex plane $\mathbb{C}$, we can write the symplectic form of this problem as $\omega = \Re (dx\wedge  d\overline{y})$.


\begin{remark}
To check that a form $\omega$ is actually a $b^m$-symplectic form, it is not enough to check that the multi-vector field dual to $\omega\wedge\omega$ is a section of $\bigwedge^{2n}(^{b^m}TM)$  which is transverse to the zero section. One has to check additionally that the Poisson structure dual to $\omega$ itself is a proper section of $\bigwedge^{2}(^{b^m}TM)$.
\end{remark}


\begin{proposition} Under the coordinate change (\ref{eq:mcgeheechange}), the symplectic form $\omega$ is sent to a $b$-symplectic structure for $\alpha = 2$.
\end{proposition}

\begin{proof}
The proof of this proposition is a straightforward computation.
Observe that the change is not a smooth change, so we are not working with standard De Rham forms. But, at the end of the computation, it will become clear that the form is a $b$-symplectic form, and hence the computations are legitimate.
If one does the change of variables, we obtain:
\begin{equation}
\begin{array}{rcl}
 \bar{y} & = & r^{\beta\gamma}(v - iw)e^{-i\theta}. \\
 dx & = & \gamma r^{\gamma-1} e^{i\theta}dr + r^{\gamma}e^{i\theta}id\theta. \\
 d\bar{y} & = & r^{-\beta\gamma-1}(-\beta\gamma)(v - iw)e^{-i\theta}dr + r^{-\beta\gamma}e^{-i\theta}dv \\
 & & \quad + r^{\beta\gamma}(v - iw)e^{-i\theta}(-i)d\theta. \\
\end{array}
\end{equation}
By wedging the previous two forms, we obtain:
\begin{equation}
\begin{array}{rcl}
dx \wedge d\bar{y} & = & dr\wedge dv (\gamma r^{\gamma-1-\beta\gamma})\\
& + & dr\wedge dw (\gamma r^{\gamma-1-\beta\gamma})\\
& + & dr\wedge d\theta (\gamma r^{\gamma-1-\beta\gamma}(-iv-w))\\
& + & d\theta \wedge dr (ir^{\gamma-1-\beta\gamma}(-\beta\gamma)(v-iw))\\
& + & d\theta\wedge dv(ir^{\gamma-\beta\gamma})\\
& + & d\theta\wedge dw(ir^{\gamma-\beta\gamma}(-i)).\\
\end{array}
\end{equation}
Now we can take the real part of this form and use that $\gamma - 1 - \beta\gamma = -\alpha\gamma$.
 In the new coordinates, the form reads.
\begin{equation}\label{eq:omegachanged}
\begin{array}{rcl}
\omega = \Re(dx \wedge d\bar{y}) & = & \gamma r^{-\beta\gamma + \gamma - 1} dr\wedge dv - \gamma(1 - \beta)r^{-\beta \gamma + \gamma -1}wdr\wedge d\theta \\
&-& r^{-\beta\gamma + \gamma} dw \wedge d\theta.
\end{array}
\end{equation}
Moreover, we can use that $\gamma(1+\beta) = 1$ to simplify the previous expression further to:
\begin{equation}
\omega = (dr\wedge dv + dr\wedge dw)\gamma r^{-\alpha\gamma} + dr\wedge d\theta(wr^{-\alpha\gamma}) + d\theta\wedge dw (r^{-\alpha\gamma + 1}).
\end{equation}
In order to classify this structure,  we wedge it with itself and look at the structure of the form in the singular set.
Wedging this form, we obtain
\begin{equation}
\begin{array}{rcl}
\omega\wedge \omega  & = &  -\gamma r^{-2\beta\gamma + 2\gamma - 1}dr\wedge dv \wedge d\theta\wedge dw\\
& = & \displaystyle -\gamma r^{\frac{2 -3\alpha}{2 + \alpha}}dr\wedge dv \wedge d\theta\wedge dw.
\end{array}
\end{equation}
where we use (\ref{eq:relations}). Let us set $f(\alpha) = \frac{2 -3\alpha}{2 + \alpha}$. This function does not take values lower than $-3$ or higher than $1$. 
When $\alpha = 2$ this gives us a $b$-symplectic structure:
$$\omega\wedge\omega = -\gamma r dr\wedge dv \wedge d\theta \wedge dw.$$
The section of $ \bigwedge^4(^bTM)$ given by the dual structure of $\omega\wedge\omega$ is clearly transverse to the zero section.

On the other hand if $\alpha = 2$, then $\beta = 1$ and hence:

$$\omega = \gamma r^{-1} dr\wedge \omega \wedge dv,$$

and its dual Poisson structure is clearly also a proper section of $ \bigwedge^2(^bTM)$.

\end{proof}

\begin{remark}
One may ask if for other values of $\alpha$ it is possible to obtain other $b^m$-symplectic structures for different $m$. For example for $\alpha = 6$,  as $\omega\wedge\omega = -\gamma r^{-2}dr\wedge dv \wedge d\theta \wedge dw$, so it seems likely to obtain a $b^2$-symplectic form. But from the expression of $\omega$ it becomes clear that it is not a proper section of $\bigwedge^2(^{b^2}T^*M)$
\end{remark}



\section[Applications]{The restricted three-body problem}\label{Sec:Escapesingularity}

In this last section of the monograph, we catch up with the circular planar restricted three-body problem.

The restricted elliptic $3$-body problem is a simplified version of the $3$-body problem. It describes the trajectory of a body with negligible mass moving in the gravitational field of two massive bodies called primaries, orbiting in elliptic Keplerian motion. The restricted planar version assumes that all motion occurs in a plane.

The associated Hamiltonian of the particle can be written as:
\begin{equation}
H(q,p)=\frac{\|p\|^2}{2}+\frac{1-\mu}{\|q-q_1\|}+\frac{\mu}{\|q-q_2\|}=T+U
\end{equation}
wit $\mu$  the reduced mass of the system.

\begin{figure}
	
	\tikzset{>=latex}
	\centering
	\definecolor{xdxdff}{rgb}{0.49019607843137253,0.49019607843137253,1.}
	\definecolor{qqqqff}{rgb}{0.,0.,1.}
	\begin{tikzpicture}[line cap=round,line join=round,x=0.8cm,y=0.8cm]
	\clip(-6,-2.8) rectangle (7,4.5);
	\draw [->] (5.96,-0) -- (-0.3,3.92);
	\draw [->] (-3.78,0) -- (-0.3,3.92);
	\draw (5.96,-0)-- (-3.78,0);
	\draw [->] (0,0) -- (-0.3,3.92);
	\draw[color=black] (-3.7,-1.1) node {$m_1 = 1 - \mu$};
	\draw[color=black] (5.9,-1.1) node {$m_2 = \mu$};
	\draw [fill=black] (-0.3,3.92) circle (2pt);
	\draw [fill=black] (-3.78,0) circle (1.5pt);
	\draw [fill=black] (5.96,-0) circle (1.5pt);
	\draw[color=black] (-0.4,4.29) node {$q$};
	\draw[color=black] (2.3,3.2) node {$r_2 = q - q_2$};
	\draw[color=black] (-2.54,3.2) node {$r_1 = q - q_1 $};
	\draw [fill=black] (0,0) circle (2pt);
	\draw[color=black] (-0.3,-0.3) node {$\text{Center of mass}$};
	\draw[color=black] (0.2,1.8) node {$r$};
	\draw [fill=cyan,draw=none,fill opacity=1] (-3.78,0) circle (0.5cm);
	\draw [fill=magenta,draw=none,fill opacity=1] (5.96,0) circle (0.3cm);
	\draw[color=black] (-3.78,0) node {$q_1$};
	\draw[color=black] (5.96,0) node {$q_2$};
	\end{tikzpicture}
	\caption{Scheme of the three-body problem.}
\end{figure}


As it was observed in \cite{kiesenhofermirandascott}, it is possible to associate a singular structure to this problem. Consider the symplectic form on $\mathrm{T}^{\ast} \mathbb{R}^2$ in polar coordinates,
After making a change to polar coordinates $(q_1,q_2)=(r\cos\alpha,r\sin\alpha)$ and the corresponding canonical change of momenta we find the Hamiltonian function
\begin{equation}
H(r,\alpha,P_r,P_\alpha)=\frac{P_r^2}{2}+\frac{P_\alpha ^2}{2r^2}+U(r\cos\alpha,r\sin\alpha)
\end{equation}
where $P_r,P_\alpha$ are the associated canonical momenta and with potential energy:
$U(r\cos\alpha,r\sin\alpha)$

The McGehee change of coordinates is used to examine the behavior of orbits near infinity, see also \cite{delshams2015global}:
\begin{equation}\label{eqn:McGehee}
r=\frac{2}{x^2}.
\end{equation}
The corresponding change for the canonical momenta is easily seen to be
\begin{equation}
P_r=-\frac{x^3}{4}P_x.
\end{equation}
The Hamiltonian is transformed to
\begin{equation}
H(r,\alpha,P_r,P_\alpha)=\frac{x^6P_x^2}{32}+\frac{x^4P_\alpha^2}{8}+U(x,\alpha).
\end{equation}
By  transforming the position coordinate~(\ref{eqn:McGehee}) without modifying the momentum associated to $r$, we are left with a simpler Hamiltonian, however, the pull-back of the symplectic form  is no longer symplectic, but exhibits a singularity of order $3$ and it is called $b^3$-symplectic:
\begin{equation}
\omega= \frac{4}{x^3} dx \wedge dP_r + d\alpha\wedge d P_\alpha.
\end{equation}

Adding the line at infinity provides a description of the dynamics within the critical set $Z=\{x=0\}$.
From the change of coordinates implemented, we might think that the dynamics within  $Z$ may have  no physical meaning, but its interplay with the dynamics close to $Z$ gives information about the behaviour of escape orbits sometimes identified as \emph{singular periodic orbits} (see \cite{MO20} and \cite{cedricdanieleva}).




Given an autonomous Hamiltonian system of a symplectic manifold  of dimension $2n$, the level sets of the Hamiltonian function are often endowed with a contact structure
( a contact structure is given by a one form $\alpha$ satisfying a condition of type $\alpha\wedge (d\alpha)^{n-1}\neq 0$).

In \cite{MO18, MO20}  applications of the $b$-apparatus are discussed in this context. In particular, the notion of $b^m$-contact structures is introduced by translating the condition above for $b^m$-forms. The classical notions in the contact realm such as  Reeb vector fields can also be introduced in this set-up.

By considering the McGehee change as we did in the contact context,  the following theorem is proved in \cite{MO20}:

\begin{theorem}\label{thm:bcontact3bp}
After the McGehee change, the Liouville vector field $Y=p\frac{\partial}{\partial p}$ is a $b^3$-vector field that is everywhere transverse to the level sets of the Hamiltonian $\Sigma_c$ for $c>0$ and the level-sets $(\Sigma_c,\iota_Y \omega)$ for $c>0$ are $b^3$-contact manifolds. Topologically, the critical set of this contact manifold is a cylinder (which can be interpreted as a subset of the line at infinity) and the Reeb vector field admits infinitely many non-trivial periodic orbits on the critical set.
\end{theorem}



The  KAM theorem in this monograph can be applied to find new periodic orbits of the restricted three-body problem close to infinity by perturbing the periodic orbits described above (see also \cite{MO20}).  This perturbation technique is an old method in perturbation theory, possibly originating from Poincaré himself (known as Poincaré's continuation method, as mentioned in \cite{meyeroffin}).  This paves the way for further research that will be pursued elsewhere.

\section{Escape orbits in Celestial mechanics and Fluid dynamics}
Reeb vector fields and their dynamics are closely related to Beltrami vector fields, which provide stationary solutions of the Euler equations. Indeed, Etnyre and Ghrist \cite{EG} revealed the existence of a "mirror" that reflects Reeb vector fields as Beltrami vector fields. Thus, the applications of KAM theory to find periodic orbits can be exported to understand periodic orbits of stationary fluid flows.  Also, this correspondence yields the possibility to translate concepts in celestial mechanics into  fluid dynamics. Among these concepts is the notion of escape orbits.
The Reeb-Beltrami correspondence was extended to the $b$-setting in \cite{danielevarobert}.

In \cite{MO20} we introduced the notion of singular periodic orbit of a $b$-Reeb vector field $R_{\alpha}$:
\begin{definition}
		Let $(M,\alpha)$ be $b$-contact manifold manifold with critical hypersurface $Z$. Denote by $R_{\alpha}$ its $b$-Reeb vector field.  A \emph{singular periodic orbit} $\gamma$ is an orbit such that $\lim_{t \to  \pm \infty} \gamma(t) =p_{\pm} \in Z$ where $R_{\alpha}(p_{\pm})=0$.
	\end{definition}
	

	
	\begin{figure}[hbt!]\label{fig:singularorbit}
\begin{center}
\begin{tikzpicture}[scale=2.2]
 \draw[color=blue](1,0) arc (0:180:1 and 1);
% \draw (-1,0) -- (1,0);
 \draw[dashed][color=red] (-2,0) --  (2,0);
 \fill (1,0) circle[radius=0.5pt];
 \fill (-1,0) circle[radius=0.5pt];
 %\draw (-0.75,0.2) ..controls +(0,0.5) and +(0,0.5).. node {\midarrow} (0.75,0.2); % arc semicircle up
 %\draw (-0.75,0.2) ..controls +(0,0.3) and +(-0.5,0).. (0,0.7); % arc1 upleft
% \draw (0,0.7) ..controls +(0.5,0) and +(0,0.3).. (0.75,0.2); % arc1 upright
% \draw (-0.75,0.2) ..controls +(0,-0.2) and +(0,-0.2).. (0.75,0.2); %arc1 down
%  \draw (-0.6,0.3) ..controls +(0,0.05) and +(-0.5,0).. (0,0.6); % arc2 upleft
% \draw (0,0.6) ..controls +(0.5,0) and +(0,0.05).. (0.6,0.3); % arc2 upright
% \draw (-0.6,0.3) ..controls +(0,-0.2) and +(0,-0.2).. (0.6,0.3); %arc2 down
%\flecha[shift={(0,0)},black,scale=2,rotate=];
%\fill [shift={(0.05,0)},scale=0.05,rotate=0]   (0,0) -- (-1,-0.7) -- (-1,0.7) -- cycle;
\fill [shift={(-0.05,1)},scale=0.05,rotate=180]   (0,0) -- (-1,-0.7) -- (-1,0.7) -- cycle; %arrow up
%\fill [shift={(-0.05,0.7)},scale=0.05,rotate=180]   (0,0) -- (-1,-0.7) -- (-1,0.7) -- cycle; %arrow up (2nd)
%\fill [shift={(0.05,0.15)},scale=0.05,rotate=0]   (0,0) -- (-1,-0.7) -- (-1,0.7) -- cycle;
%\fill [shift={(0.05,0.24)},scale=0.05,rotate=0]   (0,0) -- (-1,-0.7) -- (-1,0.7) -- cycle;
%  \draw[shift={(0,0.15)},scale=0.6] (-0.6,0.3) ..controls +(0,0.05) and +(-0.5,0).. (0,0.6); % arc2 upleft
% \draw[shift={(0,0.15)},scale=0.6] (0,0.6) ..controls +(0.5,0) and +(0,0.05).. (0.6,0.3); % arc2 upright
% \draw[shift={(0,0.15)},scale=0.6] (-0.6,0.3) ..controls +(0,-0.2) and +(0,-0.2).. (0.6,0.3); %arc2 down
\end{tikzpicture}

\caption{A singular periodic orbit}

\end{center}
\end{figure}

The singular Weinstein conjecture was conjecture in \cite{MO20}. 
In \cite{cedricdanieleva} the existence of  singular periodic orbits  and  generalizations
such as oscillatory motions is investigated for singular structures. Singular periodic orbits are a particular case of escape orbits.




Escape orbits for $b$-Beltrami or $b$-Reeb vector fields, are orbits whose $\alpha$- or $\omega$-limit set lies on the critical set associated to the $b$-structure. The $b$-Reeb Beltrami correspondence together with results of Uhlenbeck \cite{uhlenbeck} on Laplacian eigenfunctions yields that for the majority of asymptotically exact $b$-metrics, $b$-Beltrami vector fields have escape orbits.

 The following theorem proved in \cite{cedricdanielevajosep} gives a lower bound on the number of escape orbits for generic classes of $b$-Beltrami or $b$-Reeb vector fields. The lower bound depends on the number of connected components of the critical set of $Z$ but can often be infinite.

\begin{theorem}{\cite{cedricdanielevajosep}}
    Let $(M,Z)$ be a $3$-dimensional $b$-manifold. Then for a generic asymptotically exact $b$-metric, any $b$-Beltrami vector field has either at least $2N$ or infinitely many escape orbits, where $N$ is the number of connected components of $Z$.
\end{theorem}

In view of the $b$-Reeb-Beltrami correspondence, this implies that generic $b$-Reeb vector fields within a special class of $b$-contact forms also have at least $2N$ or infinitely many escape orbits.

Poincaré continuation method and our KAM results thus can be used to localize the singular counterparts or periodic orbits, either escape orbits or singular periodic orbits in these new scenarios described in \cite{eva}, \cite{MO20},\cite{cedricdanieleva}, and \cite{cedricdanielevajosep}.
\endinput

%-----------------------------------------------------------------------
% End of chap1.tex
%-----------------------------------------------------------------------

%\include{chap4}
\backmatter


\bibliographystyle{amsalpha}
% \typeout{}
\bibliography{bibliarnau}


%@book {Alexandrovgeom,
    AUTHOR = {Alexander, Stephanie and Kapovitch, Vitali and Petrunin,
              Anton},
     TITLE = {An invitation to {A}lexandrov geometry},
    SERIES = {SpringerBriefs in Mathematics},
      NOTE = {CAT(0) spaces},
 PUBLISHER = {Springer, Cham},
      YEAR = {2019},
     PAGES = {xii+88},
      ISBN = {978-3-030-05311-6; 978-3-030-05312-3},
   MRCLASS = {53C23 (51F99 53C20 53C45 53C70)},
  MRNUMBER = {3930625},
MRREVIEWER = {Loreno Heer},
       DOI = {10.1007/978-3-030-05312-3},
       URL = {https://doi.org/10.1007/978-3-030-05312-3},
}

@article {surveymetricentropy,
    AUTHOR = {Gin\'{e}, Evarist},
     TITLE = {Empirical processes and applications: an overview},
      NOTE = {With a discussion by Jon A. Wellner and a rejoinder by the
              author},
   JOURNAL = {Bernoulli},
  FJOURNAL = {Bernoulli. Official Journal of the Bernoulli Society for
              Mathematical Statistics and Probability},
    VOLUME = {2},
      YEAR = {1996},
    NUMBER = {1},
     PAGES = {1--38},
      ISSN = {1350-7265},
   MRCLASS = {60F17 (60F15 62G30)},
  MRNUMBER = {1394050},
MRREVIEWER = {Miguel A. Arcones},
       DOI = {10.2307/3318565},
       URL = {https://doi.org/10.2307/3318565},
}

@article{afsari2011riemannian,
  title={Riemannian $L^p$-center of mass: existence, uniqueness, and convexity},
  author={Afsari, Bijan},
  journal={Proceedings of the American Mathematical Society},
  volume={139},
  number={2},
  pages={655--673},
  year={2011}
}


@article{afsari2013convergence,
  title={On the convergence of gradient descent for finding the Riemannian center of mass},
  author={Afsari, Bijan and Tron, Roberto and Vidal, Ren{\'e}},
  journal={SIAM Journal on Control and Optimization},
  volume={51},
  number={3},
  pages={2230--2260},
  year={2013},
  publisher={SIAM}
}

@article{agueh2011barycenters,
  title={Barycenters in the Wasserstein space},
  author={Agueh, Martial and Carlier, Guillaume},
  journal={SIAM Journal on Mathematical Analysis},
  volume={43},
  number={2},
  pages={904--924},
  year={2011},
  publisher={SIAM}
}


@article{altschuler2021averaging,
  title={Averaging on the Bures-Wasserstein manifold: dimension-free convergence of gradient descent},
  author={Altschuler, Jason and Chewi, Sinho and Gerber, Patrik R and Stromme, Austin},
  journal={Advances in Neural Information Processing Systems},
  volume={34},
  pages={22132--22145},
  year={2021}
}


@article{altschuler2021wasserstein,
  title={Wasserstein barycenters can be computed in polynomial time in fixed dimension.},
  author={Altschuler, Jason M and Boix-Adsera, Enric},
  journal={J. Mach. Learn. Res.},
  volume={22},
  pages={44--1},
  year={2021}
}

@article{altschuler2022wasserstein,
  title={Wasserstein barycenters are NP-hard to compute},
  author={Altschuler, Jason M and Boix-Adsera, Enric},
  journal={SIAM Journal on Mathematics of Data Science},
  volume={4},
  number={1},
  pages={179--203},
  year={2022},
  publisher={SIAM}
}

@incollection{bacak2014convex,
  title={Convex analysis and optimization in Hadamard spaces},
  author={Bac{\'a}k, Miroslav},
  booktitle={Convex Analysis and Optimization in Hadamard Spaces},
  year={2014},
  publisher={de Gruyter}
}

@article{bhatia2006riemannian,
  title={Riemannian geometry and matrix geometric means},
  author={Bhatia, Rajendra and Holbrook, John},
  journal={Linear algebra and its applications},
  volume={413},
  number={2-3},
  pages={594--618},
  year={2006},
  publisher={Elsevier}
}

@incollection{bhatia2009positive,
  title={Positive definite matrices},
  author={Bhatia, Rajendra},
  booktitle={Positive Definite Matrices},
  year={2009},
  publisher={Princeton university press}
}

@article{bhatia2019bures,
  title={On the Bures--Wasserstein distance between positive definite matrices},
  author={Bhatia, Rajendra and Jain, Tanvi and Lim, Yongdo},
  journal={Expositiones Mathematicae},
  volume={37},
  number={2},
  pages={165--191},
  year={2019},
  publisher={Elsevier}
}

@article{bhattacharya2003large,
  title={Large sample theory of intrinsic and extrinsic sample means on manifolds},
  author={Bhattacharya, Rabi and Patrangenaru, Vic},
  journal={Annals of statistics},
  volume={31},
  number={1},
  pages={1--29},
  year={2003},
  publisher={Institute of Mathematical Statistics}
}

@article{bhattacharya2005large,
  title={Large sample theory of intrinsic and extrinsic sample means on manifolds: {I}{I}},
  author={Bhattacharya, Rabi and Patrangenaru, Vic},
  journal={Annals of statistics},
  pages={1225--1259},
  year={2005},
  publisher={JSTOR}
}


@article{bhattacharya2017omnibus,
  title={Omnibus CLTs for {F}r{\'e}chet means and nonparametric inference on non-Euclidean spaces},
  author={Bhattacharya, Rabi and Lin, Lizhen},
  journal={Proceedings of the American Mathematical Society},
  volume={145},
  number={1},
  pages={413--428},
  year={2017}
}


@article{billera2001geometry,
  title={Geometry of the space of phylogenetic trees},
  author={Billera, Louis J and Holmes, Susan P and Vogtmann, Karen},
  journal={Advances in Applied Mathematics},
  volume={27},
  number={4},
  pages={733--767},
  year={2001},
  publisher={Elsevier}
}



@inproceedings{boria2020Frechet,
  title={{F}r{\'e}chet Mean Computation in Graph Space through Projected Block Gradient Descent},
  author={Boria, Nicolas and Negrevergne, Benjamin and Yger, Florian},
  booktitle={ESANN 2020},
  year={2020}
}



@book{bridson2013metric,
  title={Metric spaces of non-positive curvature},
  author={Bridson, Martin R and Haefliger, Andr{\'e}},
  volume={319},
  year={2013},
  publisher={Springer Science \& Business Media}
}

@book{burago2022course,
  title={A course in metric geometry},
  author={Burago, Dmitri and Burago, Yuri and Ivanov, Sergei},
  volume={33},
  year={2022},
  publisher={American Mathematical Society}
}

@book{chavel2006riemannian,
  title={Riemannian geometry: a modern introduction},
  author={Chavel, Isaac},
  volume={98},
  year={2006},
  publisher={Cambridge university press}
}

@inproceedings{chewi2020gradient,
  title={Gradient descent algorithms for {B}ures-{W}asserstein barycenters},
  author={Chewi, Sinho and Maunu, Tyler and Rigollet, Philippe and Stromme, Austin J},
  booktitle={Conference on Learning Theory},
  pages={1276--1304},
  year={2020},
  organization={PMLR}
}


@inproceedings{claici2018stochastic,
  title={Stochastic wasserstein barycenters},
  author={Claici, Sebastian and Chien, Edward and Solomon, Justin},
  booktitle={International Conference on Machine Learning},
  pages={999--1008},
  year={2018},
  organization={PMLR}
}


@inproceedings{cuturi2014fast,
  title={Fast computation of Wasserstein barycenters},
  author={Cuturi, Marco and Doucet, Arnaud},
  booktitle={International conference on machine learning},
  pages={685--693},
  year={2014},
  organization={PMLR}
}


@article{eltzner2019stability,
  title={Stability of the Cut Locus and a Central Limit Theorem for {F}réchet Means of {R}iemannian Manifolds},
  author={Eltzner, Benjamin and Galaz-Garcia, Fernando and Huckemann, Septhan F and Tuschmann, Wilderich},
  journal={arXiv preprint arXiv:1909.00410},
  year={2019}
}

@article{eltzner2019smeary,
  title={{A smeary central limit theorem for manifolds with application to high-dimensional spheres}},
  author={Eltzner, Benjamin and Huckemann, Stephan F},
  journal={The Annals of Statistics},
  volume={47},
  number={6},
  pages={3360--3381},
  year={2019},
  publisher={Institute of Mathematical Statistics}
}



@article {Frechet48,
    AUTHOR = {Fr\'{e}chet, Maurice},
     TITLE = {Les \'{e}l\'{e}ments al\'{e}atoires de nature quelconque dans un espace
              distanci\'{e}},
   JOURNAL = {Ann. Inst. H. Poincar\'{e}},
  FJOURNAL = {Annales de l'Institut Henri Poincar\'{e}},
    VOLUME = {10},
      YEAR = {1948},
     PAGES = {215--310},
      ISSN = {0365-320X},
   MRCLASS = {60.0X},
  MRNUMBER = {27464},
MRREVIEWER = {J. L. Doob},
       URL = {http://www.numdam.org/item?id=AIHP_1948__10_4_215_0},
}


@article{hotz2013sticky,
  title={Sticky central limit theorems on open books},
  author={Hotz, Thomas and Skwerer, Sean and Huckemann, Stephan and Le, Huiling and Marron, J Stephen and Mattingly, Jonathan C and Miller, Ezra and Nolen, James and Owen, Megan and Patrangenaru, Vic},
  journal={Annals of Applied Probability},
  year={2013}
}

@article{karcher1977Riemannian,
  title={{R}iemannian center of mass and mollifier smoothing},
  author={Karcher, Hermann},
  journal={Communications on pure and applied mathematics},
  volume={30},
  number={5},
  pages={509--541},
  year={1977},
  publisher={Wiley Online Library}
}


@book{kendall2009shape,
  title={Shape and shape theory},
  author={Kendall, David George and Barden, Dennis and Carne, Thomas K and Le, Huiling},
  volume={500},
  year={2009},
  publisher={John Wiley \& Sons}
}


@article{kendall2011limit,
  title={Limit theorems for empirical {F}r{\'e}chet means of independent and non-identically distributed manifold-valued random variables},
  author={Kendall, Wilfrid S and Le, Huiling},
  journal={Brazilian Journal of Probability and Statistics},
  volume={25},
  number={3},
  pages={323--352},
  year={2011},
  publisher={Brazilian Statistical Association}
}

@inproceedings{kroshnin2019complexity,
  title={On the complexity of approximating Wasserstein barycenters},
  author={Kroshnin, Alexey and Tupitsa, Nazarii and Dvinskikh, Darina and Dvurechensky, Pavel and Gasnikov, Alexander and Uribe, Cesar},
  booktitle={International conference on machine learning},
  pages={3530--3540},
  year={2019},
  organization={PMLR}
}

@article{kroshnin2021statistical,
  title={Statistical inference for Bures--Wasserstein barycenters},
  author={Kroshnin, Alexey and Spokoiny, Vladimir and Suvorikova, Alexandra},
  journal={The Annals of Applied Probability},
  volume={31},
  number={3},
  pages={1264--1298},
  year={2021},
  publisher={Institute of Mathematical Statistics}
}

@article{kubo1980means,
  title={Means of positive linear operators},
  author={Kubo, Fumio and Ando, Tsuyoshi},
  journal={Mathematische Annalen},
  volume={246},
  number={3},
  pages={205--224},
  year={1980},
  publisher={Springer}
}


@book {Ledouxconcentration,
    AUTHOR = {Ledoux, Michel},
     TITLE = {The concentration of measure phenomenon},
    SERIES = {Mathematical Surveys and Monographs},
    VOLUME = {89},
 PUBLISHER = {American Mathematical Society, Providence, RI},
      YEAR = {2001},
     PAGES = {x+181},
      ISBN = {0-8218-2864-9},
   MRCLASS = {28C15 (28A35 46B09 60E15 82B44)},
  MRNUMBER = {1849347},
MRREVIEWER = {Werner Linde},
       DOI = {10.1090/surv/089},
       URL = {https://doi.org/10.1090/surv/089},
}


@article{le2017existence,
  title={Existence and consistency of Wasserstein barycenters},
  author={Le Gouic, Thibaut and Loubes, Jean-Michel},
  journal={Probability Theory and Related Fields},
  volume={168},
  number={3},
  pages={901--917},
  year={2017},
  publisher={Springer}
}

@article{LimPalfia14,
author = {Lim, Yongdo and Pálfia, Miklós},
year = {2014},
month = {05},
pages = {},
title = {Weighted deterministic walks for the least squares mean on Hadamard spaces},
volume = {46},
journal = {Bulletin of the London Mathematical Society},
doi = {10.1112/blms/bdu008}
}


@inproceedings{massart2019curvature,
  title={Curvature of the manifold of fixed-rank positive-semidefinite matrices endowed with the Bures--Wasserstein metric},
  author={Massart, Estelle and Hendrickx, Julien M and Absil, P-A},
  booktitle={Geometric Science of Information: 4th International Conference, GSI 2019, Toulouse, France, August 27--29, 2019, Proceedings},
  pages={739--748},
  year={2019},
  organization={Springer}
}

@article {ohtapalfia,
    AUTHOR = {Ohta, Shin-ichi and P\'{a}lfia, Mikl\'{o}s},
     TITLE = {Discrete-time gradient flows and law of large numbers in
              {A}lexandrov spaces},
   JOURNAL = {Calc. Var. Partial Differential Equations},
  FJOURNAL = {Calculus of Variations and Partial Differential Equations},
    VOLUME = {54},
      YEAR = {2015},
    NUMBER = {2},
     PAGES = {1591--1610},
      ISSN = {0944-2669},
   MRCLASS = {53C23 (49L20 49M37 58C05)},
  MRNUMBER = {3396425},
MRREVIEWER = {Takumi Yokota},
       DOI = {10.1007/s00526-015-0837-y},
       URL = {https://doi.org/10.1007/s00526-015-0837-y},
}

@article {Funano10,
    AUTHOR = {Funano, Kei},
     TITLE = {Rate of convergence of stochastic processes with values in
              {$\R$}-trees and {H}adamard manifolds},
   JOURNAL = {Osaka J. Math.},
  FJOURNAL = {Osaka Journal of Mathematics},
    VOLUME = {47},
      YEAR = {2010},
    NUMBER = {4},
     PAGES = {911--920},
      ISSN = {0030-6126},
   MRCLASS = {60F05 (58J65 60B05)},
  MRNUMBER = {2791571},
MRREVIEWER = {Longmin Wang},
       URL = {http://projecteuclid.org/euclid.ojm/1292854310},
}

@book{simon2019loewner,
  title={Loewner's Theorem on Monotone Matrix Functions},
  author={Simon, Barry},
  year={2019},
  publisher={Springer}
}

@Article{Troyanov,
author = {M. Troyanov},
title = {Concentration et inégalité de {P}oincaré},
journal = {Séminaire Borel 2001 (IIIème Cycle Romand de Mathématiques).  Berne},
year = {2001},
OPTorganization = {Séminaire Borel 2001 (IIIème Cycle Romand de Mathématiques).  Berne},
OPTpublisher = {•},
OPTnote = {•},
OPTannote = {•}
}



@book{surveyineg,
    AUTHOR = {An\'{e}, C\'{e}cile and Blach\`ere, S\'{e}bastien and Chafa\"{\i}, Djalil and
              Foug\`eres, Pierre and Gentil, Ivan and Malrieu, Florent and
              Roberto, Cyril and Scheffer, Gr\'{e}gory},
     TITLE = {Sur les in\'{e}galit\'{e}s de {S}obolev logarithmiques},
    SERIES = {Panoramas et Synth\`eses [Panoramas and Syntheses]},
    VOLUME = {10},
      NOTE = {With a preface by Dominique Bakry and Michel Ledoux},
 PUBLISHER = {Soci\'{e}t\'{e} Math\'{e}matique de France, Paris},
      YEAR = {2000},
     PAGES = {xvi+217},
      ISBN = {2-85629-105-8},
   MRCLASS = {46N20 (26D15 46E99 58J60 60J10 60J60)},
  MRNUMBER = {1845806},
MRREVIEWER = {Emmanuel Russ},
}


@incollection {sturm03,
    AUTHOR = {Sturm, Karl-Theodor},
     TITLE = {Probability measures on metric spaces of nonpositive
              curvature},
 BOOKTITLE = {Heat kernels and analysis on manifolds, graphs, and metric
              spaces ({P}aris, 2002)},
    SERIES = {Contemp. Math.},
    VOLUME = {338},
     PAGES = {357--390},
 PUBLISHER = {Amer. Math. Soc., Providence, RI},
      YEAR = {2003},
   MRCLASS = {60B05 (28C15 28C99 53C21)},
  MRNUMBER = {2039961},
MRREVIEWER = {Vladimir I. Bogachev},
       DOI = {10.1090/conm/338/06080},
       URL = {https://doi.org/10.1090/conm/338/06080},
}



@article {fastconv,
    AUTHOR = {Le Gouic, T. and Paris, Q. and Rigollet, P. and Stromme, A.J.},
     TITLE = {Fast convergence of empirical barycenters in {A}lexandrov spaces and the {W}asserstein space},
   JOURNAL = {J. Eur. Math. Soc. },
      YEAR = {2022},
       DOI = {10.4171/jems/1234},
       URL = {https://doi.org/10.4171/jems/1234},
}

@article {convrate,
    AUTHOR = {Ahidar-Coutrix, A. and Le Gouic, T. and Paris, Q.},
     TITLE = {Convergence rates for empirical barycenters in metric spaces:
              curvature, convexity and extendable geodesics},
   JOURNAL = {Probab. Theory Related Fields},
  FJOURNAL = {Probability Theory and Related Fields},
    VOLUME = {177},
      YEAR = {2020},
    NUMBER = {1-2},
     PAGES = {323--368},
      ISSN = {0178-8051},
   MRCLASS = {60B05 (51F99 51K10 60B10 62G05)},
  MRNUMBER = {4095017},
       DOI = {10.1007/s00440-019-00950-0},
       URL = {https://doi.org/10.1007/s00440-019-00950-0},
}

@article {Ohtaconvexity,
    AUTHOR = {Ohta, Shin-ichi},
     TITLE = {Convexities of metric spaces},
   JOURNAL = {Geom. Dedicata},
  FJOURNAL = {Geometriae Dedicata},
    VOLUME = {125},
      YEAR = {2007},
     PAGES = {225--250},
      ISSN = {0046-5755},
   MRCLASS = {53C21 (46E35 58E20)},
  MRNUMBER = {2322550},
MRREVIEWER = {Jeremy T. Tyson},
       DOI = {10.1007/s10711-007-9159-3},
       URL = {https://doi.org/10.1007/s10711-007-9159-3},
}

@incollection{Eltzner2021,
    AUTHOR = {Tran, Do and Eltzner, Benjamin and Huckemann, Stephan},
     TITLE = {Smeariness {B}egets finite sample smeariness},
 BOOKTITLE = {Geometric science of information},
    SERIES = {Lecture Notes in Comput. Sci.},
    VOLUME = {12829},
     PAGES = {29--36},
 PUBLISHER = {Springer, Cham},
      YEAR = {[2021]},
   MRCLASS = {60E05 (62R30)},
  MRNUMBER = {4424306},
       DOI = {10.1007/978-3-030-80209-7\_4},
       URL = {https://doi.org/10.1007/978-3-030-80209-7_4},
}


@book{vershynin2018high,
  title={High-dimensional probability: An introduction with applications in data science},
  author={Vershynin, Roman},
  volume={47},
  year={2018},
  publisher={Cambridge university press}
}


@article {Yokota16,
    AUTHOR = {Yokota, Takumi},
     TITLE = {Convex functions and barycenter on {CAT}(1)-spaces of small
              radii},
   JOURNAL = {J. Math. Soc. Japan},
  FJOURNAL = {Journal of the Mathematical Society of Japan},
    VOLUME = {68},
      YEAR = {2016},
    NUMBER = {3},
     PAGES = {1297--1323},
      ISSN = {0025-5645},
   MRCLASS = {53C23},
  MRNUMBER = {3523548},
MRREVIEWER = {Bo\.{z}ena Pi\polhk atek},
       DOI = {10.2969/jmsj/06831297},
       URL = {https://doi.org/10.2969/jmsj/06831297},
}

@article {Yokota18,
    AUTHOR = {Yokota, Takumi},
     TITLE = {Convex functions and {$p$}-barycenter on {${\rm
              CAT}(1)$}-spaces of small radii},
   JOURNAL = {Tsukuba J. Math.},
  FJOURNAL = {Tsukuba Journal of Mathematics},
    VOLUME = {41},
      YEAR = {2017},
    NUMBER = {1},
     PAGES = {43--80},
      ISSN = {0387-4982},
   MRCLASS = {53C23},
  MRNUMBER = {3705774},
MRREVIEWER = {Bo\.{z}ena Pi\polhk atek},
       DOI = {10.21099/tkbjm/1506353559},
       URL = {https://doi.org/10.21099/tkbjm/1506353559},
}

@inproceedings{zhang2016first,
  title={First-order methods for geodesically convex optimization},
  author={Zhang, Hongyi and Sra, Suvrit},
  booktitle={Conference on Learning Theory},
  pages={1617--1638},
  year={2016},
  organization={PMLR}
}

@article{zhang2018towards,
  title={Towards Riemannian accelerated gradient methods},
  author={Zhang, Hongyi and Sra, Suvrit},
  journal={arXiv preprint arXiv:1806.02812},
  year={2018}
}

@inproceedings{ziezold1977expected,
  title={On expected figures and a strong law of large numbers for random elements in quasi-metric spaces},
  author={Ziezold, Herbert},
  booktitle={Transactions of the Seventh Prague Conference on Information Theory, Statistical Decision Functions, Random Processes and of the 1974 European Meeting of Statisticians},
  pages={591--602},
  year={1977},
  organization={Springer}
}
%%!TEX root = ../main.tex

\section{Indexing and Super Key Generation}\label{sec:index}
We propose an index structure that retains the space efficiency of the single-attribute inverted index as introduced in Section~\ref{sec:preliminaries}.
% Table~\ref{tab:variables} summarizes the variables used in this paper.

% % \begin{table}[]
%     \scriptsize
%     \centering
%     \caption{\answerRthree{List of variables used in the paper.}}
%     \label{tab:variables}
% \begin{tabular}{l|l}
% \toprule
% \textit{\textbf{Variable}} &
% \textbf{Description} \\ \toprule

% % $PL$ & Posting List that represents the location of a value in a table in corpus  \\
% % $m$ & The dimensionality of the join key  \\
% $\jmath$ & Joinability score  \\
% % $D$ & The input dataset\\
% % $Q$ & The set of columns in $D$ that form the composite key  \\
% % $T$ & Collection of tables that joinable tables are discovered from \\
% % $FP$ & False Positive  \\
% % $FN$ & False Negative  \\
% % $P$ & Number of all possible permutations\\
% % $c$ & The number of columns in a table\\
% % $v_i$ & Value i in the inverted index that maps to a list PL\\
% % $k$ & The number of joinable tables the system discovers \\
% $S_{ij}$ & The superkey of the $j^{th}$ PL of $v_i$\\
% $a$ & Bit array that contains the generated hash\\
% $L_{unique}$ & The number of unique values in the corpus\\
% $L_a$ & Hash size\\
% $\alpha$ & The optimum number of 1 bits in a\\
% $\beta$ & The number of bits in each character segment\\
% $L$ & The number of bits dedicated to the length segment\\
% $l_v$& Length of value v\\
% $\lambda$&Average location of the character to be hashed in the value\\
% $x$& Relative position of the character to be hashed in the value\\
% % $d$& Input dataset in the running example\\
% % $T_1$&Joinable table in the running example\\
% $\jmath_k$&Joinability score of the least joinable table among top-k discovered tables\\
% % $t$&Current candidate table to be evaluated\\
% $L_t$& The number of fetched PLs of table t\\
% $r_{checked}$&The number of evaluated rows from t\\
% $r_{matched}$&The number of row from t that are joinable to the input dataset\\
% % $h(k)$&hash result of key value k\\
% $q\textquotesingle$ & Query column with minimum cardinality\\
% \toprule
% \end{tabular}
% \end{table}





\begin{table}[]
    \scriptsize
    \centering
    \caption{\answerRthree{List of variables.}}
    \label{tab:variables}
\begin{tabular}{l|l}
\toprule
\textit{\textbf{Variable}} &
\textbf{Description} \\ \toprule
$\jmath$ & Joinability score  \\
$S_{ij}$ & The superkey of the $j^{th}$ PL of $v_i$\\
$a$ & Bit array that contains the generated hash\\
$C_{unique}$ & The number of unique values in the corpus\\
$|a|$ & Hash size\\
$\alpha$ & The optimum number of 1 bits in a\\
$\beta$ & The number of bits in each character segment\\
$|a_l|$ & The number of bits dedicated to the length segment\\
$l_v$& Length of value v, i.e., the number of characters in v\\
$\lambda$&Average location of the character to be hashed in the value\\
$x$& Relative position of the character to be hashed in the value\\
% $d$& Input dataset in the running example\\
% $T_1$&Joinable table in the running example\\
$\jmath_k$&Joinability score of the least joinable table among top-k discovered tables\\
% $t$&Current candidate table to be evaluated\\
$L_t$& The number of fetched PLs of table t\\
$r_{checked}$&The number of evaluated rows from t\\
$r_{matched}$&The number of row from t that are joinable to the input dataset\\
% $h(k)$&hash result of key value k\\
$q\textquotesingle$ & Query column with minimum cardinality\\
\toprule
\end{tabular}
\end{table}


\subsection{Desiderata}
\label{sec:desiderata}
As generating a multi-attribute inverted index requires an exponential number of index entries per table,
it is important to extend the traditional single-attribute inverted index to be applicable for n-ary joins.
Ideally, we need an additional index element per table row that exposes the existence of join attribute value combinations.
We, thus, add an additional element to the index structure named \textit{super key}. This changes the inverted index defined in Equation~\ref{eq_conventional_inverted_index} to $v_i \mapsto \{(T_{i1},\ C_{i1},\ R_{i1},\ S_{i1}), (T_{i2},\ C_{i2},\ R_{i2},\ S_{i2}),\ ... \}$, where $S_{ij}$ is a fixed-size bit array, i.e.,~\textit{Super Key}.

As one cannot anticipate which attribute combinations are relevant for n-ary joins, we need a hash value that retains them all.
The idea is to have a fixed size super key that aggregates hash values for each row value so that we can verify the existence of a composite key without checking all values of the row.
\system aggregates the hash results of individual values into a fixed-sized bit vector using the logical bit-wise \texttt{OR} operation. Thus, the super key masks the hash values of each single row value, which ensures that no value is missed, when probing with the super key with the same hash function.



Consider our previous example once again. Rows $2$, $5$, and $6$ are candidate rows to be joined with the first key in the input dataset d based on the first value ``Muhammad''. Now, to drop the $5^{th}$ and $6^{th}$ row from the candidate rows, the super key for these rows should convey that the values ``Lee'' and ``US'' do not simultaneously exist in these candidate rows.
Yet, the drawback of the aggregated hash value within a fixed hash table is that the super key might mask the hash values of non-existent values as well.

Therefore, the goal is to design a hash function where the aggregation of cell values from different columns results in different super keys.
A general approach to this problem is to use a bloom filter.
However, off-the-shelf bloom filters have two drawbacks.
First, they use hash functions that assume a uniform distribution.
Therefore, any arbitrary pair of cell values can result in overlapping bits in the final bit array, which increases the chance of FPs.
Second, they are agnostic to the distinguishing properties within columns, i.e., character distributions and positions.

%\vspace{0.1cm}
%\noindent{\bf Desired Hash Function.} To overcome the aforementioned shortcomings of bloom filters, we need a hash function whose parameters are independent of the query table and the individual tables of the corpus. The {\em desired hash function} should encode values in a way that syntactically similar values in different columns and syntactically different values of the same column fall into different bit positions of the super key. As the hash space and ultimately the number of bits are limited, we can only hash a finite number of value features, which have to be the most differentiating properties of a row value.

\subsection{\hash}\label{subsec:synhash}
We propose \hash, a hash function that encodes syntactic features into distinguishable hashes.
As the super key is an OR-aggregation of these hash results, it may mask non-existent values and pass FPs. 
Thus, our goal is to disperse 1-bits in a way that we decrease the likelihood of two different values from different columns turning the same set of bits to 1. Ideally we want as few 1-bits as possible in the super key to reduce the probability that it covers the super key of random value combinations.
\hash leverages three syntactic properties of the cell values to meet this goal:
the {\em least frequent characters}, {\em their location}, and the {\em value length}.

\subsection{Feature Generation and Encoding} \label{subsec:hash_generation_process}
We now turn our attention to how we extract the aforementioned features to apply \hash on a row value.
We first discuss the number of bits to generate the hash and its relationship to the table corpus.
Then, we explain our segmentation process that allocates different parts of the hash table for different features of a value.
Then, we elaborate on the process of mapping the character features to hash bits.
Finally, we explain how to use the hash space to relocate the generated bits per cell value to prevent partial matches.
We use the example in Figure~\ref{fig:hash_array_example} to elaborate each hash generation step.

\subsubsection{Required number of bits}
The super key encodes each value into a fixed-size bit array $a$. 
On the one hand, as we want the hash values of different individual strings to differ, we want to use all possible bits to encode as many values as possible, i.e.,~$2^{|a|}$, to reduce the number of collisions.
On the other hand, the super key should contain as few `$1$s' as possible to avoid masking FPs.
As a result, underpinning \hash results should be constructed in a way that there is an upper bound of $1$ bits for each hash.
With this goal in mind, Equation~\ref{eq:number_of_ones} calculates $\alpha$, i.e., the optimum number of `$1$s' that is required to generate unique hash values, where $C_{unique}$ is the number of unique values in the corpus and $|a|$ is the hash size.
\begin{equation}\label{eq:number_of_ones}
\small
    \argmin_{\alpha} {|a| \choose \alpha} > C_{unique}.
\end{equation}
The binomial term calculates the number of possible bit combinations of size $\alpha$ over $|a|$ bits.
The minimum value for $\alpha$
corresponds to the number of `$1$' bits needed per \hash result.
For a $128$-bit hash space and $700M$ unique values as existing in DWTC webtables, $\alpha$ is equal to $6$. 
In fact, out of the $\alpha$ bits, one bit is always reserved to encode the value size and $\alpha-1$ bits for the actual value.
Assume that the hash size in our illustrating example is $128$ bits and $\alpha = 4$ ($3$ for the characters and $1$ for the value length).

\subsubsection{Encoding the characters}\label{subsubsec:encoding_chars}
As we use $\alpha-1$ bits to encode each value, we want different values to use different bit segments of $a$.
Thus the characters should be maximally different across words. We can obtain this property based on the character frequency.
\textbf{Lemma.} Least frequent characters lead to fewer collisions.
% \vspace{-.3cm}
\begin{proof}[Proof.]
Given a random word $w_1$ consisting of letters ${l_1,..l_n}$ each with a probability of occurrence $P(l_i)$, we sample $K<n$
letters $S= {s_1,..,s_k}$, where $K=\alpha - 1$. 
Given a random word $w_2$ from the same alphabet, we want to reduce the probability to sample the same set of characters $K$.
The probability to obtain a word with the same $K$ characters is $P(s_1)\cdot P(s_2)\cdot\dots\cdot P(s_k)$. 
This product is minimised whenever a factor is replaced with a smaller probability. Thus, when picking the $K$ least frequent character of the alphabet we obtain $\forall \hat{s}_i \in \{w_1 - S\}, \forall s_i\in S: P(s_i) < P(\hat{s}_i)$. 
Replacing any $s_i$ with $\hat{s}_i$ results in $P(s_1)\cdot P(s_2)\cdot\dots\cdot P(s_k) < P(s_1)\cdot P(\hat{s}_i)\cdot\dots\cdot P(s_k)$, which leads to a higher probability for a collision.
\end{proof}


To further distinguish the frequencies across domains, we pick the $\alpha-1$ least frequent characters inside a word as the differentiator. For single-word cell values with flat frequency distributions we draw based on lexicographical order of the characters.

\noindent\textit{Segmentation.}
We segment the hash space into smaller blocks, depending on the length of the hash array to encode the features of a cell value.
This segmentation specifies how many bits each feature needs. 
Depending on the number of possible characters, \hash splits the hash array into as many smaller fixed-size segments, one for each character. 
In our case, we consider all 37 alphanumeric characters including space, which results in 37 segments of size $\beta$.
We create an additional segment to encode the length of the string value.
Sticking with 37 as the number of characters, we can calculate $\beta$ as follows:
\begin{equation}
\small
    \argmax_{\beta} {(37 \cdot \beta < |a|)}
\end{equation}
$|a|$ is the length of the hash array: we have $\beta=3$ with a hash size of \textit{128} bits.
This segmentation dedicates the largest possible sub-array to encode character features, because the character features, including the position of the characters, are more discriminative than the length of the value.
The rest of the hash array can be allocated to the length segment: $|a_l| = |a| - (37 \cdot \beta)$.
For a hash size of $128$ bits, the length segment would comprise of $17$ bits ($128 - 37 \cdot 3$). 
$99.99\%$ of the English words have fewer than $17$ characters~\cite{mayzner1965tables}. Therefore, the explained segmentation can cover almost all possible words in English language. In our real-world data lakes, Dresden webtable and German open data, over $83\%$ of the cell values have at most $17$ characters. For larger hash sizes, i.e., $512$, $|a_l|=31$  and covers the length of more than $98\%$ of the cell values in our corpora.

In Figure~\ref{fig:hash_array_example}, we want to obtain the hash results for values ``muhammad'' ($C_1$), ``lee'' ($C_2$), and ``us'' ($C_3$) using \hash and then aggregate them into the super key of the row. The red characters in the table cells represent the selected least-frequent characters.

\subsubsection{Encoding the character locations}
Generally, one bit per character would be enough to encode its occurrence in a value.% i.e., ``0'' for absence and ``1'' for the existence of the character.
However, if more bits are available ($\beta>1$), we can use them to encode the relative character position inside the original string to further distinguish the hash results of different values. For this purpose, we divide the string in $\beta$ equal areas from left to right. 
We encode a character by checking in which area the character appears and then we set only the corresponding bit among the $\beta$ character bits from left to right.
More formally, if $l_v$ is the length of the value, i.e., the number of characters in the value, and $\lambda$ is the location of the character we would like to encode (we take average of the locations in case of character repetition), then, $x$, i.e., the bit index in the character segment that represents the relative location of the character is calculated as: $x = \ceil*{\frac{\lambda\cdot \beta}{l_{v}}}$

% We illustrate this with an example:
% For $\beta=1$, we can only store the absence/existence of a character, therefore, the hash values would be equal for the two words ``loop'' and ``pool''. This is because both values are comprised of the same characters thus, the same character segments would be set and the length is equal. However with $\beta=3$, we assign $3$ bits per character in the hash array and consider three consecutive areas of each string. Now, character hash segments will look differently. In the hash of the value ``loop'', the left most bit of segment ``L'' turns to 1, i.e.,~$<100>$, because character ``L'' is located in the most left part of the string.
% For ``pool'', this bit assignment of character ``L'' would be $<001>$ because ``l'' appears in the last third area of ``pool''.
% The opposite applies to the encoding of ``P''.
% The encoding for ``O'' as the average of the two middle positions turns the middle bit of the ``O'' segment for both hash results.

Returning to our illustrating example, we calculate the average location in the original value for each of the least frequent characters.
For instance, the average location of the characters ``u'' and ``d'' in cell $C_1$ are $1$ and $8$ respectively.
The first hash array represents the \hash result of $C_1$.
With 3 bits for each character ($\beta = 3$):
the characters with the average location of less than $3$ ($\frac{8}{3}=3$) turn the first bit (100);
between and including $3$ and $5$ turn the second bit (010); and
above $5$ are encoded with the third bit of their corresponding segments (001).
For example, characters ``u'' and ``d'' in $C1$ are located in the first and the third bits of their segments respectively.
We use blue color to trace the character ``u'' in the hash array.

\subsubsection{Encoding the length}\label{subsub_length}
With the segmentation step of \hash, a fixed-size segment is dedicated to the length of the value $l_v$.
Storing the actual binary representation of the value length can be problematic as it leads to an unknown number of $1$ bits in the segment. This can also mask the length values of the columns with shorter values. For instance, the encoding of $l_v=7$,i.e., $<111>$, can mask the encoding of a value with $l_v=2$, i.e., $<010>$.
To address this problem, we reserve one bit per possible $l_v\mod |a_l|$. This way, we maintain the number of used bits $i$ and different length values do not mask each other because each length turns a distinguished bit to set.
Encoding the length feature of the values introduces another benefit to the system:
Positioning the length segment as the left-most segment allows the system to use short circuit optimization, which skips unnecessary bit operations.
If there is no value in the candidate row with the same length as a query value, \system does not need to check the character segments.
%\ma{do we need the next example now? because we have the illustrating example now here?}
Consider our running example. The cell values ``Boxer'' and ``Birder'' in rows $5$ and $6$ respectively. Both of these values start with the encoded character ``B''. Therefore, ``B'' and its position cannot differentiate these two cell values. However, the values have different lengths, which makes their hashes distinguishable.

% Going back to our example, as $|C1|=8$, it will be encoded as $8\ mod\ 17 = 8$ and turn the eighth bit of the length segment.

\subsubsection{Bit rotation}\label{subsubsec:bitrotation}
As a final measure to differentiate hash values without using more $1$ bits than $i$, we reduce the likelihood of so-called random matches through rotation of the character segments.
A random match occurs when a key value that is partially masked by the individual hash functions of individual row values is masked by the aggregated super key.
For example, we could have a query key that contains rare characters that occur in two different columns. Or if one value shares the length of the key and the other only shares the characters of the key value that we are searching for.
To prevent these kinds of FPs, \hash rotates the character hash segments of a value $v$ by its length $l_v$ to left.
The most left bits that fall off will be moved to the beginning of the bit vector.
For example, a 3-bit rotation of '01100101' equals '00101011'. Note that the rotation only applies to the character-related segments.
%This rotation ensures that the hash of the same value is always rotated into the same bit structure.
As a result, a random match becomes less likely because the length of the random key value must match the length of the column that overlaps in terms of the least frequent characters.
%With bit rotation, \hash connects the value length and the character features without additional $1$ bits.
% Continuing our example in \ref{subsub_length}, we exploit the length of the values to re-locate the character bits so that the character ``B'' in ``Boxer'' is mapped to a completely different bit than the one for the same ``B'' in ``Birder''.

\begin{figure}
    \center{\includegraphics[scale=0.14]
          {figures/aggregation_example_2.pdf}}
            %   \vspace{-.6cm}
    \caption{Example of \hash and the results aggregation.}
    \label{fig:hash_array_example}
\end{figure}


\textbf{Lemma.} Rotation reduces inter-column bit collisions. 
% \vspace{-.3cm}
\begin{proof}[Proof.]
Given that two cell values $c_1$ and $c_2$ are masked by a given super key $sk$, there are two general cases that lead to $sk$ masking false positives. In both cases the length bits must overlap. Either two values from the same domain or from different domains have the same length respectively. In all collision cases, the $K$ characters of each cell value $c_1$ and $c_2$ are masked, where $K = \alpha - 1$. 
Now assume that each of the $K$ characters is rotated based on the length of $c_1$ and $c_2$ respectively. Now, for a join candidate $\hat{c}_1$ and $\hat{c}_2$, the collision occurs only if there is an exact match of characters and length between any pair of $c_i$ and $\hat{c}_j$. In all other cases there is a 1-bit in a position that is not masked by $sk$.
We now prove that the opposite leads to a contradiction.
There is a collision under rotation of two cell values with different length where there is no exact match between any pair of $c_i$ and $\hat{c}_j$. It follows that $\forall c_i \exists \hat{c}_j: |c_i|=|\hat{c}_j|$ with an injective mapping. 
Thus, there is a pair of $c_i$ and $\hat{c}_j$ with the same length, where one of the encoded $K$ characters is different. Therefore $K-1$ character bits will be rotated the same way and are covered. Every other $c_i$ covers at most $K$ characters. Thus, there must be an exact match between $c_i$ and $\hat{c}_j$.
\end{proof}

The previous lemma shows that rotation reduces the number of collisions. This is also shown in our micro-benchmarks where we analyze its impact on filtering.
Continuing our illustrating example, the character segment bits all rotate by $8$ positions to the left.
As can be seen in the rotated array, the segment for ``u'' moved from the 108th to the 100th most left bit.
The final aggregated results for C2 and C3 are shown in separate hash array excerpts.
Note that because of the rotation in the hash results, the character ``u'' appears in different locations for $C_1$ and $C_2$ respectively.

\subsection{Index Updates}
There are three possible types of edits on table corpora that lead to updates in the index-level: {\em insert}, {\em update}, and {\em delete}.
Inserting a new table to the corpus requires the following updates.
Other than generating the PL items for the cell values in the newly added table, a super key is generated for each row. 
Inserting a new row to an existing table also follows the same procedure. 
Adding a new column to an existing table requires applying \hash on each individual column value and replacing the corresponding super key by the result of a bit-wise \texttt{OR} operation with the new \hash result.
Updating the value of a cell requires a complete re-hash of the corresponding super key.
Deleting a table/row only requires deleting the PL items for the table/row.
Yet, deleting a column from a table might change the super key entries, triggering a rehashing of all rows.
Although some of the aforementioned updates require regeneration of the super key, the system can handle the changes locally to the affected table.
%Generally, the most frequent index updates in a data lake would be \textit{table inserts}, which do not affect any of the already generated index values.
\end{document}

%-----------------------------------------------------------------------
% End of chapter.tex
%-----------------------------------------------------------------------
