%-----------------------------------------------------------------------------
% Beginning of preface.tex
%-----------------------------------------------------------------------------
%
% AMS-LaTeX 1.2 sample file for a monograph, based on amsbook.cls.
% This is a data file input by chapter.tex.
%%%%%%%%%%%%%%%%%%%%%%%%%%%%%%%%%%%%%%%%%%%%%%%%%%%%%%%%%%%%%%%%%%%%%%%%

\chapter*{Preface}

{\begin{flushright}{\begin{minipage}{5cm}{\emph{I confess I envy the planets — they've got their own orbits and nothing stands on their way.} 
\\
Intermezzo, Mykhailo Kotsiubynsky.}\end{minipage}
}\end{flushright}}



This monograph explores classification and perturbation problems for integrable systems on a class of Poisson manifolds called $b^m$-Poisson manifolds. This is the first class of Poisson manifolds for which perturbation theory is established outside the symplectic category. Even if the class of $b^m$-Poisson manifolds is not ample enough to represent the wild set of Poisson manifolds, this investigation can be seen as a first step for the study of perturbation theory for general Poisson manifolds. Whenever the Poisson manifold can be described as an $E$-symplectic manifold (\cite{enest} and \cite{evageoff}) the theory developed in this monograph yields more than a mild generalization in Poisson Geometry and, this \emph{toy example},  sets the path to consider KAM theory in the general realm of Poisson manifolds. Reduction theorems and $b^m$-symplectic manifolds have been recently explored in \cite{anastasiaeva}. This monograph contributes to the theory opening the investigation of perturbation theory on these manifolds thus completing other facets in the study of their dynamics as the recent work on the Arnold conjecture \cite{cedricjoaquimeva}.

Symplectic geometry has been the common language of physics as the position-momentum tandem can be modelled over a cotangent bundle. Cotangent bundles are naturally endowed with a symplectic form which is a non-degenerate closed $2$-form. The symplectic form of the cotangent bundle is given by the differential of the Liouville one-form.

$b^m$-Poisson manifolds are manifolds that are symplectic away from a hypersurface along which they satisfy some transversality properties. They often model problems on symplectic manifolds with boundary such as the study of their deformation quantization and celestial mechanics.  As on the complementary of the critical set the manifolds are symplectic, extending the investigation of Hamiltonian dynamics to this realm is key to understanding the Hamiltonian Dynamics on the compactification of symplectic manifolds. Several regularization transformations used in celestial mechanics (as McGehee or Moser regularization) provide examples of such compactifications.

One of the interesting properties of $b^m$-Poisson manifolds is that their investigation can be achieved considering the language of $b^m$-forms. That is to say, we can work with forms that are symplectic away from the critical set and admit a smooth extension as a \emph{form over a Lie algebroid} generalizing De Rham forms as form over the standard Lie algebroid of the tangent bundle of the manifold. To consider $b^m$-forms the standard tangent bundle is replaced by  the $b^m$-tangent bundle. This allows us to mimic symplectic geometry by replacing the cotangent bundle with the dual of the $b^m$-tangent bundle. However, Poisson geometry leaves its footprint and new invariants which can be identified as the modular class of the Poisson structure arise already at the semilocal level.

Contrary to the initial expectations,  several of the results for $b^m$-symplectic manifolds do not resemble the $b$-case so far. Considering these more general singularities yields a better understanding of the general Poisson case and the different levels of complexity. As an illustration of this phenomena: in the study of quantization of those systems an interesting pattern makes the quantization radically different in the even and odd case \cite{GMWbquant, GMWbmquant} and the resulting model is finite-dimensional in the $b$-case. Understanding how the different degrees $m$ are related is a hard task: The desingularization technique introduced by Guillemin-Miranda-Weitsman in \cite{GMW17} turned out to have important applications in the investigation of complexity properties of toric $b^m$-symplectic manifolds \cite{GMWbmconvexity} and to the study of the Arnold conjecture in this set-up \cite{cedricjoaquimeva}. In this monograph, we explore a new facet of these manifolds: that of perturbation theory.

%This monograph starts by recalling the equivariant classification of $b^m$-Poisson structures investigating, in particular, the analog of Moser's classification theorem for symplectic surfaces and their equivariant analogues.
%To provide some basic examples and fix ideas we include several results of classification of $b^m$-surfaces. The classification invariants in the case of surfaces are encoded in a cohomology called $b^m$-cohomology which has been deeply studied by \cite{Scott16}. Mazzeo-Melrose type formula for $b^m$-cohomology decomposes it in two pieces which can  be read off the De Rham cohomology of both the ambient manifold $M$ and the critical hypersurface.  As an outcome of this identification, the Poisson classification of these manifolds is given by the De Rham cohomology of the manifold and the hypersurface.

%This classification is extended to the equivariant setting if we assume that the singular forms are preserved by the group action of a compact Lie group.
%These techniques can be extended to the classification of $b^m$-Nambu structures which we do not consider in this monograph.


In the second part of the monograph, we consider integrable systems on these manifolds. Under mild conditions, integrable systems on these singular manifold have an associated "generalized" Hamiltonian action of tori in a neighbourhood of a Liouville torus.  We use this generalized Hamiltonian group action to prove the existence of action-angle coordinates in a neighborhood of a Liouville torus.  The action-angle coordinate theorem that we prove furnishes a semi-local normal form within the vicinity of a Liouville torus for the $b^m$-symplectic structure.  which depends on the modular weight of the connected component of the critical set in which the Liouville torus is lying as well as the modular weights of the associated toric action. This action-angle theorem allows us to identify a neighborhood of the Liouville torus with the $b^m$-cotangent lift of the action of a torus acting by translations on itself. This interpretation of the action-angle theorem as cotangent lift allows us to pinpoint the modular weight as their only semilocal invariant. In doing so, we compare this action-angle coordinate theorem with the classical action-angle coordinate theorems for symplectic manifolds and an action-angle theorem for folded symplectic manifolds (\cite{EvaRobert}).


In part 3 of the monograph we study perturbation theory in this new set-up and examine some potential applications to physical systems. In particular, we prove a KAM theorem for $b^m$-Poisson manifolds which clearly refines and improves the one obtained for $b$-Poisson manifolds in \cite{KMS16}. As an outcome of this result together with the extension of the desingularization techniques of Guillemin-Miranda-Weitsman to the realm of integrable systems, we obtain a KAM theorem for folded symplectic manifolds where KAM theory has never been considered before. In the way, we also obtain a brand-new KAM theorem for symplectic manifolds where the perturbation keeps track of a distinguished hypersurface. In celestial mechanics, this distinguished hypersurface can be the line at infinity or the collision set. Escape orbits in celestial mechanics can often be compactified as singular orbits associated with these singular structures.


\aufm{Barcelona, May 2023, Eva Miranda and Arnau Planas}

%-----------------------------------------------------------------------------
% End of preface.tex
%-----------------------------------------------------------------------------
