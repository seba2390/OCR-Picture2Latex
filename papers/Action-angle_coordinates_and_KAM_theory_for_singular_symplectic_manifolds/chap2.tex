%-----------------------------------------------------------------------
% Beginning of chap1.tex
%-----------------------------------------------------------------------
%
%  AMS-LaTeX sample file for a chapter of a monograph, to be used with
%  an AMS monograph document class.  This is a data file input by
%  chapter.tex.
%
%  Use this file as a model for a chapter; DO NOT START BY removing its
%  contents and filling in your own text.
%
%%%%%%%%%%%%%%%%%%%%%%%%%%%%%%%%%%%%%%%%%%%%%%%%%%%%%%%%%%%%%%%%%%%%%%%%

\part[Action-angle coordinates and cotangent models]{Action-angle coordinates and cotangent models for $b^m$-integrable systems}


In this part,  we consider the semilocal classification for any $b^m$-Poisson manifold in a neighbourhood of an invariant compact submanifold.
The compact submanifolds under consideration are the compact invariant leaves of the distribution $\mathcal D$ generated by the Hamiltonian vector fields $X_{f_i}$ of an integrable system. An integrable system is given by a set of $n$ functions on a $2n$-dimensional symplectic manifold  which we can order in a map $F=(f_1, \dots, f_n)$. Historically, integrable systems were introduced to actually \emph{integrate} Hamiltonian systems $X_H$ using the \emph{first-integrals $f_i$} and, classically, we identify $H=f_1$. It turns out that in the symplectic context the compact regular orbits of the distribution $\mathcal D$ coincide with the fibers $F^{-1}(F(p))$ for any point $p$ on these orbits/fibers.
The fact that the orbit coincides with the connected fiber is part of the magic of symplectic duality.

The same picture is reproduced for singular symplectic manifolds of $b^m$-type or $b^m$-Poisson manifolds as we will see in this chapter.


The study of action-angle coordinates has interest from this geometrical point of view of the classification of geometric structures in a neighbourhood of a compact submanifold of a $b^m$-Poisson manifold. It also has interest from a dynamical point of view as these compact submanifolds now coincide with invariant subsets of the Hamiltonian system under consideration.

From a geometric point of view, the existence of action-angle coordinates determines a \emph{unique} geometrical model for the $b^m$-Poisson (or $b^m$-symplectic) structure in a neighbourhood of the invariant set. From a dynamical point of view, the existence of action-angle coordinates provides a normal form theorem that can be used to study stability and perturbation problems of the Hamiltonian systems (as we will see in the last chapter of this monograph).

An important ingredient that makes our action-angle coordinate theorem \emph{brand-new } from the symplectic perspective is that the system under consideration is more general than Hamiltonian, it is $b^m$-Hamiltonian as the first-integrals of the system can be $b^m$-functions which are not necessarily smooth functions. Dynamically, this means that we are adding to the set of Hamiltonian invariant vector fields, the \emph{modular vector field} of the integrable system.

In contrast to the standard action-angle coordinates for symplectic manifolds, our action-angle theorem comes with $m$ additional invariants associated with the modular vector field which can be interpreted in cohomological terms as the projection of the $b^m$-cohomology class determined by the modular vector field on the first cohomology group of the critical hypersurface under the Mazzeo-Melrose correspondence.

The strategy of the proof of the action-angle coordinate systems is the search of a toric action (so this takes us back to the motivation of the use of \emph{symmetries} in this monograph). In contrast to the symplectic case, it is not enough that this action is Hamiltonian as then a direction of the Liouville torus would be missing. We need the toric action to be $b^m$-Hamiltonian.
The structure of this proof looks like the one in \cite{KMS16} but encounters serious technical difficulties as in order to check that the \emph{natural} action to be considered is
$b^m$-Hamiltonian we need to go deeper inspired by \cite{Scott16}  in the relation between the geometry of the modular vector field and the coefficients of the Taylor series $c_i$ of one of the first-integrals. This allows us to understand new connections between the geometry and analysis of $b^m$-Poisson structures not explored before.

Once we prove the existence of this $b^m$-Hamiltonian action the proof looks very close to the one in \cite{KMS16}.

In the second chapter of this part
we re-state the action-angle theorem as a cotangent lift theorem with the following mantra:


 \emph{Every integrable system on a $b^m$-Poisson manifold looks like a $b^m$-cotangent lift in a neighborhood of a Liouville torus.}




\chapter{An action-angle theorem for $b^m$-symplectic manifolds}

\section{Basic definitions}
\subsection{On $b^m$-functions}
 The definition of the analogue of $b$-functions in the $b^m$-setting is somewhat delicate. The set of $^{b^m}\mathcal{C}^\infty(M)$ needs to be such that for all the functions $f\in ^{b^m}\mathcal{C}^\infty(M)$, its differential $df$ is a $b$-form, where $d$ is the $b^m$-exterior differential.
Recall that a form in $^{b^m}\Omega^k(M)$ can be locally written as
$$\alpha\wedge\frac{dx}{x^m} + \beta$$
where $\alpha \in \Omega^{k-1}(M)$ and $\beta \in \Omega^{k}(M)$. Recall also that
$$d\left(\alpha\wedge\frac{dx}{x^m} + \beta\right) = d\alpha\wedge\frac{dx}{x^m} + d\beta.$$

We need $df$ to be a well-defined $b^m$-form of degree $1$. Let $f = g\frac{1}{x^{k-1}}$, then $df = dg \frac{1}{x^{k-1}} - g\frac{k-1}{x^k}dx$. This from can only be a $b^m$-form if and only if $g$ only depends on $x$. If $f = g\log(x)$, then $dg\log(x) + g\frac{1}{x}dx$, which imposes $dg=0$ and hence $g$ to be constant.

With all this in mind, we make the following definition.
\begin{definition}
The set of $b^m$-functions is defined recursively according to the formula $$~^{b^m} \mathcal{C}^\infty(M)= x^{-(m-1)}\mathcal{C}^\infty(x) + ~^{b^{m-1}} \mathcal{C}^\infty(M)$$

\noindent  {with $\mathcal{C}^\infty(x)$ the set of smooth functions in the defining function $x$} and $$^{b}\mathcal{C}^\infty(M)=\{ g \log\vert x\vert+ h, g \in \mathbb{R}, h\in \mathcal{C}^\infty(M)\}.$$


\end{definition}


\begin{remark}
A $^{b^m} \mathcal{C}^\infty (M)$-function can be written as
$$f = a_0 \log x + a_1\frac{1}{x} + \ldots + a_{m-1}\frac{1}{x^{m-1}} + h$$
where $a_i, h \in \mathcal{C}^\infty(M)$.
\end{remark}

%\begin{remark}
%The Hamiltonian vector field associated to a $b^m$-function $H$ is a smooth vector field. Because locally we can take the expressions:
%$$\Pi = x_1^m\frac{\partial}{\partial x_1} \wedge \frac{\partial}{\partial y_1} + \sum_{i = 2}^{m} \frac{\partial}{\partial x_i}\wedge\frac{\partial}{\partial y_i} \text{ and }H = c_0 \log{x_1} + \sum_{i = 1}^{m-1}c_i \frac{1}{x_1^i} + f.$$
%Then if we compute $X_H = \Pi(dH,\cdot) = c_0 y^{m-1}\frac{\partial}{\partial y_1} + \sum_{i = 1}^{m-1} c_i i y^{m-i-1} \frac{\partial}{\partial y_1} + \Pi(f,\cdot)$, we obtain a smooth vector field.
%\end{remark}

\begin{remark}
From this chapter on we are only considering $b^m$-manifolds $(M,x,Z)$ with $x$ defined up to order $m$. I.e. we can think of $x$ as a jet of a function that coincides up to order $m$ to some defining function.
This is the original viewpoint of Scott in \cite{Scott16} which we adopt from now on. The difference with respect to the other chapters is that we do not fix an specific function.
%The reader may as well adjust her/his reading to assume the function has been fixed in case this is more convenient.
\end{remark}


\begin{definition}
We say that two $b^m$-integrable systems $F_1, F_2$ are equivalent if there exists $\varphi$, a $b^m$-symplectomorphism, i.e. a diffeomorphism preserving both $\omega$ and the critical set $Z$ (``up to order $m$''\textcolor{black}{\footnote{I.e. it preserves the jet $x$}}), such that  $\varphi \circ F_1 = F_2$.
\end{definition}

\begin{remark}
The Hamiltonian vector field associated to a $b^m$-function $f$ is a smooth vector field. Let us compute it  locally using the $b^m$-Darboux theorem:
$$\Pi = x_1^m\frac{\partial}{\partial x_1} \wedge \frac{\partial}{\partial y_1} + \sum_{i = 2}^{m} \frac{\partial}{\partial x_i}\wedge\frac{\partial}{\partial y_i} \text{ and } f = a_0 \log{x_1} + \sum_{i = 1}^{m-1}a_i \frac{1}{x_1^i} + h.$$
Then if we compute

$$\begin{array}{rcl}
\displaystyle df & = & \displaystyle \overbrace{a_0}^{c_1}\frac{1}{x_1} + \sum_{i=1}^{m-1} \overbrace{(a_i' - (i-1)a_{i-1})}^{c_i}\frac{1}{x_1^i} dx_1 \\
& & \quad -\overbrace{(m-1)a_{m-1}}^{c_m}\frac{1}{x_1^m}dx_1 + dh\\
\\
 & = & \displaystyle \sum_{i = 1}^{m}\frac{c_i}{x_1^i} dx_1 + dh.
\end{array}$$
Then,

\begin{equation}\label{eq:bmhamiltonianvf}
X_f = \Pi(df,\cdot) = \sum_{i=1}^{m} c_i x_1^{m-i}\frac{\partial}{\partial y_1} + \Pi(dh,\cdot),
\end{equation}

 we obtain a smooth vector field.



\end{remark}


%
%
%Let $(M^{2n},Z,x)$ be a $b^m$-manifold. The set of $b^m$-functions $^{b^m} \mathcal{C}^\infty (M)$ is defined by:
%$$^{b^m} \mathcal{C}^\infty (M) = x^{-(m-1)}\mathcal{I}\oplus ^b\mathcal{C}^\infty(M)$$
%where $^b\mathcal{C}^\infty(M) = \{c\log(x) + h \quad h \in \mathcal{C}^\infty(M), c \in \mathbb{R}\}$ and $\mathcal{I}$ are the functions that locally around $Z$ only depend on $x$}

\section{On $b^m$-integrable systems}

In this section we present the definition of a $b^m$-integrable system as well as some observations about these objects.

\begin{definition}
Let $(M^{2n},Z,x)$ be a $b^m$-manifold, and let $\Pi$ be a $b^m$-Poisson structure on it.
$F = (f_1, \ldots, f_n)$\footnote{$f_i$ are $b^m$-functions.} is a $b^m$-integrable system\footnote{ In this monograph we only consider integrable systems of maximal rank $n$.} if:

\begin{enumerate}%[i)]
\item $df_1,\ldots,df_n$ are independent on a dense subset of $M$ \textcolor{black}{and in all the points} of $Z$ where independent means that the form $df_1\wedge\ldots\wedge df_n$ is non-zero as a section of $\Lambda^n(^{b^m} T^{*}(M))$,
\item  the functions $f_1,\ldots, f_n$ Poisson commute pairwise.
\end{enumerate}
\end{definition}


\begin{definition}
 The points of $M$ where $df_1,\ldots, df_n$ are independent  are called \textbf{regular} points.
\end{definition}


The next remarks will lead us to a normal form for the first function $f_1$.
\begin{remark}\label{independence}
Note that $df_1,\ldots, df_n$ are independent on a point if and only if $X_{f_1},\ldots,X_{f_n}$ are independent at that point. This is because the map
$$^{b^m}TM \rightarrow ^{b^m}T^*M:u\mapsto \omega_p(u,\cdot)$$
is an isomorphism.
\end{remark}

\begin{remark}
The condition of $d f_1, \ldots, d f_n$ being independent must be understood as $d f_1 \wedge \ldots \wedge df_n$ being a non-zero section of $\bigwedge^n (\enskip ^{b^m}T^*M)$.

\end{remark}

\begin{remark}\label{rk:hamiltonian_vf}
By remark \ref{independence}  the vector fields $X_{f_1},\ldots, X_{f_n}$ have to be independent. This implies that one of the $f_1,\ldots, f_n$ has to be a singular (non-smooth) $b^m$-function with a singularity of maximal degree. If we write $f_i = c_{0,i}\log(x_1) + \sum_{j=1}^{m-1} \frac{c_{j,i}}{x_1^j} + \tilde{f}_1$
$$X_{f_i} = \sum_{j = 1}^{m} x_1^{m-j} \hat{c}_{j,i} \frac{\partial}{\partial y_1} +  X_{\tilde{f}_i}$$
where $\hat c_{j,i}(x) = \frac{d(c_{j,i})}{dx}-(j-1)c_{j-1,i}$. If there is no $b^m$-function with a singularity of maximum degree all the terms in the $\partial/\partial y_1$ direction become 0 at $Z$. And hence $X_{f_1},\ldots, X_{f_n}$ cannot have maximal rank at $Z$.
% begaus hamiltonian VF can only have up to rank n
\end{remark}

%\begin{remark}
%If one of the $f_i$ has a singularity of maximal degree, no other function can have a singularity, because $d f_1 \wedge \ldots \wedge d f_n$ would not be a $b^m$-form and in particular would not be a non-zero section of $\bigwedge^n \enskip ^{b^m}TM$.
%\end{remark}


%\begin{remark}
%From now on we will assume that given a $b^m$-integrable system $(f_1, \ldots, f_n)$, $f_1,\dots,f_{n-1}$ are smooth functions and $f_n = c_{0}\log(x_1) + \sum_{j=1}^{m-1} \frac{c_{j}}{x_1^j}$. Observe that $f_n$ we are taking the smooth part of $f_n$ equal to 0, $\tilde f_n = 0$. This is not necessarily true. But we impose this as a necessary condition to a $b^m$-integrable system

%\end{remark}

\textcolor{black}{
\begin{lemma}
Let $F = (f_1,\ldots,f_n)$ a $b^m$-integrable system. If $f_1$ has a singularity of maximal degree, there exists an equivalent integrable system $F' = (f_1',\ldots,f_n')$ where $f_1'$ has a singularity of maximal degree and \textbf{no other} $f_i'$ has singularity of any degree.
\end{lemma}
\begin{proof}
Let $f_i = \underbrace{c_{0,i}\log(x_1) + \sum_{j=1}^{m-1}\frac{c_{j,1}}{x_1^j}}_{\zeta_i(x_1)} + \tilde f_i = \zeta_i(x_1) + \tilde f _i$.
By remark \ref{rk:hamiltonian_vf}\footnote{Here have used the $b^m$-Darboux theorem to do the computations.},
$$X_{f_i} = \underbrace{\sum_{i=1}^{m} x_1^{m-j} \hat c_{j,i}}_{g_i(x_1)}\frac{\partial}{\partial y_1} + X_{\tilde f_i} = g_i(x_1)\frac{\partial}{\partial y_1} + X_{\tilde f_i}.$$
Note that $g_i(x_1) = g_i(0) = \hat c_{m,i}$ at $Z$. Let us look at the distribution given by the Hamiltonian vector fields $X_{f_i} = g_i(x_1)\frac{\partial}{\partial y_1} + X_{\tilde f_i}$. This distribution is the same that the one given by:
\begin{equation}\label{eq:distribution}
\{X_{f_1}, X_{f_2} - \frac{g_2(x_1)}{g_1(x_1)} X_{f_1},\ldots, X_{f_n} - \frac{g_n(x_1)}{g_1(x_1)} X_{f_1}\}.
\end{equation}
Observe that for $i > 1$, $X_{f_i} - \frac{g_i(x_1)}{g_1(x_1)} X_{f_1} = X_{\tilde f_i} + \frac{g_2(x_1)}{g_1(x_1)} X_{\tilde f_1}$. Also $g_1(x_1)$ is different from $0$ close to $Z$ because at $Z$ $g_1(x_1) = \hat c_{m,1}$.
Since the distribution given by these vector fields is the same, an integrable system that has Hamiltonian vector fields \ref{eq:distribution} would be equivalent to $F$. From the expression \ref{eq:distribution} it is clear that the new vector fields commute. And it is also true that this new vector fields are Hamiltonian. Let us take $F'$ the set of functions that have as Hamiltonian vector fields \ref{eq:distribution}.
\end{proof}
}

From now on we will assume the integrable system to have only one singular function and this function to be $f_1$.

\begin{remark}
Because we asked  $X_{f_1},\ldots,X_{f_n}$ to be linearly independent at all the points of $Z$ and using the previous remarks  $c_m := c_{m,1} \neq 0$ at all the points of $Z$.
\end{remark}


Furthermore, we can assume $f_1$  to have a smooth part equal to zero as subtracting the smooth part of $f_1$ to all the functions gives an equivalent system.
Also, we can assume that $c_m$ is 1 because dividing all the functions of the $b^m$-integrable system by $c_m$ also gives us an equivalent system.


\textbf{As a summary, we can assume $f_1 = a_0 \log(x) + a_1 1/x + \ldots + a_{m-2}1/x^{m-2} + 1/x^{m-1}$ and $f_2, \ldots, f_n$ to be smooth, $a_0\in \mathbb{R}$ and $a_1, \ldots, a_{m-2} \in \mathcal{C}^\infty(x)$.}

Also we are going to state lemma 3.2 in \cite{GMP17}, because we are going to use it later in this section. The result states that if we have a toric action on a $b^m$-symplectic manifold (which we will prove in a neighbourhood of a Liouville torus), then we can assume the coefficients $a_2, \ldots, a_{m-2}$ to be constants. More precisely

\begin{lemma}\label{lemma:ctt_coefs}
There exists a neighborhood of the critical set $U = Z\times (-\varepsilon, \varepsilon)$ where the moment map $\mu: M \rightarrow \mathfrak{t}^*$ is given by
$$\mu = a_1 \log|x| + \sum_{i=2}^{m}a_i \frac{x^{-(i-1)}}{i-1} + \mu_0$$
with $a_i \in \mathfrak{t}^0_L$ and $\mu_0$ is the moment map for the $T_L$-action on the symplectic leaves of the foliation.
\end{lemma}


\section{Examples of $b^m$-integrable systems}

The following example illustrates why it is necessary to use the definition of $b^m$-function as considered above. There are natural examples of changes of coordinates in standard integrable systems on symplectic manifolds that yield  $b^m$-symplectic manifolds but do not give well-defined $b^m$-integrable systems.

\begin{example}
Consider a time change in the two body problem, to obtain a $b^2$-integrable system. In the classical approach to solve the $2$-body problem  the following two conserved quantities are obtained:
$$\begin{array}{rcl}
f_1 & = & \frac{\mu y^2}{2} + \frac{l^2}{2\mu r^2} - \frac{k}{r},\\
f_2 & = & l,
\end{array}
$$
with symplectic form $\omega = dr\wedge dy + dl \wedge d\alpha$, where $r$ is the distance between the two masses and $l$ is the angular momentum.
We also know that $l$ is constant along the trajectories. Because $l$ is a constant of the movement, we can do a symplectic reduction on its level sets. The form on the symplectic reduction becomes $dr \wedge dy$. To simplify the notation, we will use $x$ instead of $r$.
Then $\omega = dx\wedge dy$. With hamiltonian function given by $f = \frac{\mu}{2}y^2 + \frac{l}{2\mu}\frac{1}{x^2} - k \frac{1}{x}$. Hence, the equations are:
$$
\begin{array}{rcl}
 \dot{x} & = & \frac{\partial f}{\partial y},\\
 \dot{y} & = & -\frac{\partial f}{\partial x}.
\end{array}
$$
Doing a time change $\tau = x^3 t$ then $\frac{d x}{d\tau} = \frac{1}{x^3} \frac{dx}{dt}$. With this time coordinate, the equations become:
$$
\begin{array}{rcl}
 \dot{x} & = & \frac{1}{x^3}\frac{\partial f}{\partial y},\\
 \dot{y} & = & -\frac{1}{x^3}\frac{\partial f}{\partial x}.
\end{array}
$$
These equations can be viewed as the motion equations given by a $b^3$-symplectic form $\omega = \frac{1}{x^3} dx\wedge dy$.

Let us check that this is actually a $b^m$-integrable system.
\begin{itemize}%[a)]
\item All the functions Poisson commute is immediate because we only have one.
\item $df = \mu y dy + (\frac{k}{x^2} - \frac{l}{\mu}\frac{1}{x^3})dx$ is a $b^3$-form because the term with $dx$ does not depend on $y$.
\item All the functions are independent, this is true because $df$ does not vanish as a $b^3$-form.
\end{itemize}
\end{example}

\begin{example}
In the paper \cite{Marle} the author builds an action of $SL(2,\mathbb{R})$ over $(P,\omega_P)$ where
$P = \{\xi\in \mathbb{C} | i(\bar{\xi} - \xi)>0\}$ is the complex semi-plane, with moment map $J_P(\xi) = \frac{R}{\xi_{im}}((|\xi|^2 + 1),2\xi_r,\pm (|\xi|^2 + 1))$, where the $\pm$ sign depends on the choice of the hemisphere projected by the stereographic projection. From now on we will take the sign $+$. Also the symplectic form $\omega_P$ has the following expression:

$$\omega_P = \pm \frac{R}{\xi_{im}^2} d\xi_r\wedge d\xi_{im}$$

In order to simplify the notation we identify $P$ with the real half-plane $P = \{x,y\in\mathbb{R}^2|y>0\}$. With this identification, the moment map becomes $J_p(x,y)=\frac{R}{y}(x^2 + y^2 + 1, 2x, x^2 + y^2 + 1)$. Obviously, this moment map does not give an integrable system. The symplectic form writes as:

$$\omega_P = \frac{R}{y^2} dy\wedge dx.$$

This form can be viewed as a $b^2$-form if we extend $P$ including the line $\{y=0\}$ as its singular set.
Let us consider only one of the components of $J_P$ as $b^m$-function and let us see if it gives a $b^m$-integrable system. First we will try with $f_1 = \frac{R}{y}(x^2 + y^2 + 1) $ and then $f_2 = \frac{R}{y}(2x)$.
\begin{enumerate}%[i)]
\item $f_1 = \frac{R}{y}(x^2 + y^2 + 1)$
We have to check three things to see if this gives a $b^2$-integrable system.
\begin{enumerate}%[a)]
\item All the functions Poisson commute is immediate because we only have one.
\item All the functions are $b^m$-functions. This point does not hold because $d f_1= \frac{R}{y^2}(2xydx + (y^2-x^2 - 1)dy)$ and the first component makes no sense as a section of $\Lambda^1 (^{b^2}T^*M)$.
\item All the functions are independent. In this case, we need to check that $df_1$ does not vanish, but since it is not a $b^m$-form it makes no sense  to be a non-zero section of $\Lambda^1 (^{b^2}T^*M)$.
\end{enumerate}
\item $f_2 = \frac{R}{y}(2x)$
\begin{enumerate}

%[a)]
\item Same as before.
\item All the functions are $b^m$-functions. This point does not hold because $d f_2= \frac{2R}{y}dx - \frac{2Rx}{y^2}dy$ and the first component makes no sense as a section of $\Lambda^1 (^{b^2}T^*M)$.
\item Same as before.
\end{enumerate}
\end{enumerate}

\end{example}



\begin{example} Toric actions give natural examples of integrable systems where the component functions are given by the moment map. In the case of surfaces:
    $S^1$-actions on surfaces give natural examples of $b^m$-integrable systems. Only torus and spheres admit circle actions. 
    
    In the picture below two integrable systems on the $2$-sphere depending on the degree $m$. On the right the image of the moment map that defines the integrable system. The action is by rotations along the central axis.

    Namely consider the sphere $S^2$ as a  $b^m$-symplectic manifold having as critical set the equator:
        
        \[(S^2, Z = \{h = 0\}, \omega=\frac{d h}{h^m}\wedge d\theta),\] with $h\in\left[-1,1\right]$ and $\theta\in\left[0,2\pi\right)$.

For $m=1$: The computation $\iota_{\frac{\partial}{\partial \theta}}\omega=- \frac{d h}{h}=-d( \log |h|),$ tells us that the function $\mu(h,\theta)= \log |h|$ is the moment map and defines a $b$-integrable system.

For higher values of $m$: $\iota_{\frac{\partial}{\partial \theta}}\omega=- \frac{d h}{h^m}=-d(-\frac{1}{(m-1)h^{m-1}}),$ and the  moment map is $\mu(h,\theta)= -\frac{1}{(m-1)h^{m-1}}$ which defines a $b^m$-integrable system.
    
 \begin{figure}[h!]
            \begin{tikzpicture}[scale=0.8]
                \pgfmathsetmacro{\rlinex}{6}
                \pgfmathsetmacro{\baseptd}{8}
                \pgfmathsetmacro{\rlineybottom}{2.75}
                \pgfmathsetmacro{\rlineymid}{4.25}
                \pgfmathsetmacro{\rlineytop}{5.75}
                \pgfmathsetmacro{\vertstretch}{1.9}	
                \pgfmathsetmacro{\yshift}{4.25}	

                \def\R{1.6}
                \pgfmathsetmacro{\circlex}{1.4}
                \draw[dashed, very thick, color = magenta] (\circlex + \R, \rlineymid) arc (0:180:{\R} and {\R * .2});
                \draw[very thick, fill = pink, opacity = .5] (\circlex, \rlineymid) circle (\R);
                \draw[very thick] (\circlex, \rlineymid) circle (\R);
                \draw[very thick, color = red] (\circlex + \R, \rlineymid) arc (0:-180:{\R} and {\R * .2});
                
                \draw (\circlex + 2, \rlineymid) edge node[above] {$\mu, m = 1$} (\rlinex - 0.5, \rlineymid);
                \draw[->] (\circlex + 2, \rlineymid) -- (\rlinex - 0.5, \rlineymid);

                \draw (\rlinex, \rlineybottom) -- (\rlinex, \rlineytop) node[right] {};

                \draw[black, fill = black] (\rlinex, \rlineymid) circle(.3mm);

                \draw[dblue, fill = dblue] (\rlinex - 0.2, \rlineymid) circle (1pt);
                \draw[dblue, fill = dblue] (\rlinex + 0.2, \rlineymid) circle (1pt);

                \draw[line width = 2pt, join = round, dblue, <-] (\rlinex - 0.2, \rlineybottom - 0.15) -- (\rlinex - 0.2, \rlineymid );
                \draw[line width = 2pt, join = round, dblue, <-] (\rlinex + 0.2, \rlineybottom - 0.15) -- (\rlinex + 0.2, \rlineymid);
                \draw [very thick, ->] (\circlex, \rlineymid + 1.3*\R) ++(\R * -.5, 0) arc (180:320: {\R * .5} and {\R * .1});
                \draw [very thick] (\circlex, \rlineymid + 1.3*\R) ++(\R * -.5, 0) arc (180:0: {\R * .5} and {\R * .1});

                \coordinate (shift) at (8,0);
                \begin{scope}[shift=(shift)]
                \pgfmathsetmacro{\rlinex}{6}
                \pgfmathsetmacro{\baseptd}{8}
                \pgfmathsetmacro{\rlineybottom}{2.75}
                \pgfmathsetmacro{\rlineymid}{4.25}
                \pgfmathsetmacro{\rlineytop}{5.75}
                \pgfmathsetmacro{\vertstretch}{1.9}	
                \pgfmathsetmacro{\yshift}{4.25}	

                \def\R{1.6}
                \pgfmathsetmacro{\circlex}{1.4}
                \draw[dashed, very thick, color = magenta] (\circlex + \R, \rlineymid) arc (0:180:{\R} and {\R * .2});
                \draw[very thick, fill = pink, opacity = .5] (\circlex, \rlineymid) circle (\R);
                \draw[very thick] (\circlex, \rlineymid) circle (\R);
                \draw[very thick, color = red] (\circlex + \R, \rlineymid) arc (0:-180:{\R} and {\R * .2});
                
                \draw (\circlex + 2, \rlineymid) edge node[above] {$\mu, m = 2$} (\rlinex - 0.5, \rlineymid);
                \draw[->] (\circlex + 2, \rlineymid) -- (\rlinex - 0.5, \rlineymid);

                \draw (\rlinex, \rlineybottom) -- (\rlinex, \rlineytop) node[right] {};

                \draw[black, fill = black] (\rlinex, \rlineymid) circle(.3mm);

                \draw[dblue, fill = dblue] (\rlinex - 0.2, \rlineymid) circle (1pt);
                \draw[dblue, fill = dblue] (\rlinex + 0.2, \rlineymid) circle (1pt);

                \draw[line width = 2pt, join = round, dblue, <-] (\rlinex - 0.2, \rlineybottom - 0.15) -- (\rlinex - 0.2, \rlineymid );
                \draw[line width = 2pt, join = round, dblue, ->] (\rlinex - 0.2, \rlineymid) -- (\rlinex - 0.2, \rlineytop + 0.15);
                
                \draw[line width = 2pt, join = round, dblue, <-] (\rlinex + 0.2, \rlineybottom - 0.15) -- (\rlinex + 0.2, \rlineymid );
                \draw[line width = 2pt, join = round, dblue, ->] (\rlinex + 0.2, \rlineymid) -- (\rlinex + 0.2, \rlineytop + 0.15);
                
                \draw [very thick, ->] (\circlex, \rlineymid + 1.3*\R) ++(\R * -.5, 0) arc (180:320: {\R * .5} and {\R * .1});
                \draw [very thick] (\circlex, \rlineymid + 1.3*\R) ++(\R * -.5, 0) arc (180:0: {\R * .5} and {\R * .1});
                \end{scope}
            \end{tikzpicture}
            \caption{Integrable systems associated to the moment map of an $S^1$-action by rotations  on a $b^m$-symplectic $2$-sphere $S^2$.}
            \label{fig:S2}
        \end{figure}
    \end{example}

\begin{example}
    Consider now as $b^2$-symplectic manifold the $2$-torus
\[
(\mathbb{T}^2, Z = \{\theta_1 \in \{0, \pi\}\}, \omega=  \frac{d\theta_1}{\sin^2\theta_1}\wedge d\theta_2)
\]
with standard coordinates: $\theta_1, \theta_2 \in \left[0, 2\pi \right)$.  Observe that the critical  hypersurface $Z$ in this example is not connected. It is the union of two disjoint circles. Consider  the circle action of rotation on the $\theta_2$-coordinate with fundamental vector field $\frac{\partial}{\partial\theta_2}$. As the following computation holds,
$$\iota_{\frac{\partial}{\partial\theta_2}}\omega = - \frac{d \theta_1}{\sin^2 \theta_1} = d\left(\frac{\cos\theta_1}{\sin\theta_1}\right).$$
The fundamental vector field of the $S^1$-action defines $^{b^2}C^{\infty}$-integrable system given by the function $-\frac{\cos\theta_1}{\sin\theta_1}$.


\end{example}


\begin{figure}[ht]
\begin{center}
    

\begin{tikzpicture}[scale=0.8]

\pgfmathsetmacro{\rlinex}{6}
\pgfmathsetmacro{\rlineybottom}{2.75}
\pgfmathsetmacro{\rlineymid}{4.25}
\pgfmathsetmacro{\rlineytop}{5.75}

\def\R{1.6}
\pgfmathsetmacro{\donutx}{1.5}
\pgfmathsetmacro{\sizer}{1.8}	


%top of rotation arrow
\draw [very thick] (\donutx - 0.3*\sizer, \rlineymid) arc (180:0: {\sizer * .4} and {\sizer * .1});

%draw the donut on the left
\draw [red, very thick, dashed] (\donutx - 0.5, \rlineymid - .55*\sizer) arc (90:270:{.16*\sizer} and {0.32*\sizer});
\draw [red, very thick, dashed] (\donutx - 0.5, \rlineymid + .55*\sizer) arc (270:90:{.16*\sizer} and {0.32*\sizer});

\DrawFilledDonutops{\donutx - 0.5}{\rlineymid}{.6*\sizer}{1.2*\sizer}{-90}{yellow!30}{very thick}{white}
\DrawDonut{\donutx - 0.5}{\rlineymid}{.6*\sizer}{1.2*\sizer}{-90}{black}{very thick}

\draw [red, very thick] (\donutx - 0.5, \rlineymid - .55*\sizer) arc (90:-90:{.16*\sizer} and {0.32*\sizer});
\draw [red, very thick] (\donutx - 0.5, \rlineymid + .55*\sizer) arc (-90:90:{.16*\sizer} and {0.32*\sizer});

\draw [very thick] (\donutx - .3*\sizer, \rlineymid) arc (180:132: {\sizer * .4} and {\sizer * .1});
\draw [very thick, ->] (\donutx - .3*\sizer, \rlineymid) arc (180:325: {\sizer * .4} and {\sizer * .1});



%draw the arrow
\draw     (\donutx + 1, \rlineymid) edge node[above] {$\mu$} (\rlinex - 0.5, \rlineymid);
\draw[->] (\donutx + 1, \rlineymid) -- (\rlinex - 0.5, \rlineymid);

\draw (\rlinex, \rlineybottom) -- (\rlinex, \rlineytop) node[right] {};

\draw[black, fill = black] (\rlinex, \rlineymid) circle(.3mm);

\draw[line width = 2pt, join = round, dblue, <->] (\rlinex - 0.2, \rlineybottom - 0.15) -- (\rlinex - 0.2, \rlineytop + 0.15);
\draw[line width = 2pt, join = round, dblue, <->] (\rlinex + 0.2, \rlineybottom - 0.15) -- (\rlinex + 0.2, \rlineytop + 0.15);
 
\end{tikzpicture}

\end{center}
% \end{center}
 \caption{Integrable system given by an $S^1$-action on a $b^2$-torus $\mathbb{T}^2$ and its associated moment map.}
 \label{fig:torus}
 \end{figure}
\begin{example} \label{ex:bmtorus}
The former example can be made general to produce examples of $b^m$-integrable systems on a $b^m$-symplectic manifold for any integer $m$
\[
(\mathbb{T}^2, Z = \{\theta_1 \in \{0, \pi\}\}, \omega=  \frac{d\theta_1}{\sin^m\theta_1}\wedge d\theta_2).
\]
Then 
$$\iota_{\frac{\partial}{\partial\theta_2}}\omega = - \frac{d \theta_1}{\sin^m \theta_1} = d\left(\frac{|\cos\theta_1|}{\cos\theta_1} \frac{_2F_1 \left ( \frac{1}{2}, \frac{1 - m}{2}; \frac{3 - m}{2}; \sin^2(\theta_1) \right )}{(1-m) \sin^{m - 1} \theta_1}\right),$$
with $_2F_1$  the hypergeometric function. 

Thus, the associated $S^1$-action has as  $^{b^m}C^{\infty}$-Hamiltonian the function 
$$- \frac{|\cos\theta_1|}{\cos\theta_1} \frac{_2F_1 \left ( \frac{1}{2}, \frac{1 - m}{2}; \frac{3 - m}{2}; \sin^2(\theta_1) \right )}{(1-m) \sin^{m - 1} \theta_1}$$ \noindent which defines a $b^m$-integrable system.
\end{example}
Now we give a couple  of examples of $b^m$-integrable systems.


\begin{example}
This example uses the product of $b^m$-integrable systems on a $b^m$-symplectic manifold with an integrable system on a symplectic manifold. Given $(M_1^{2n_1}, Z,x,\omega_1)$ a $b^m$-symplectic manifold with $f_1,\ldots,f_{n_1}$ a $b^m$-integrable system and $(M_2^{2n_2},\omega_2)$ a symplectic manifold with $g_1,\ldots,g_{n_2}$ an integrable system. Then $(M_1\times M_2, Z\times M_2, x, \omega_1 + \omega_2)$ is a $b^m$-symplectic manifold and $(f_1,\ldots,f_{n_1},g_1,\ldots,g_{n_2})$ is a $b^m$-integrable system on the higher dimensional manifold.

In particular by combining the former examples of $b^m$-integrable systems on surfaces and arbitrary integrable systems on symplectic manifolds we obtain examples of $b^m$-integrable systems in any dimension.
\end{example}


\begin{example}\textbf{(From integrable systems on cosymplectic manifolds to $b^m$-integrable systems:)}

Using the extension theorem (Theorem 50) of \cite{GMP14} we can extend any integrable system $(f_2,\dots, f_n)$ to an integrable system in a neighbourhood of a cosymplectic manifold $(Z, \alpha, \omega) $ by just adding a $b^m$-function $f_1$ to the integrable system so that the new integrable system is $(f_1, f_2,\dots, f_n)$  and considering the associated $b^m$-symplectic form:

\begin{equation}\label{eq:normalform}\tilde{\omega}=p^*\alpha\wedge\frac{dt}{t^m}+p^*\omega. \end{equation}

(t is the defining function of $Z$).

\end{example}

\section{Looking for a toric action}

In this section we pursue the proof of action-angle coordinates for $b^m$-integrable systems by recovering a torus group action. This action is associated to the Hamiltonian vector fields associated to $X_{f_i}$.

This is the same strategy used for $b$-integrable systems in \cite{KMS16}- One of the main difficulties is  to prove that the coefficients $a_1,\ldots, a_n$ can be considered  as constant functions. This makes it more difficult to prove the existence of a $\mathbb{T}^n$-action in the general $b^m$-case than in the $b$-case, but once we have it we can use the results in \cite{GMW17} to assume that the coefficients $a_1,\ldots, a_n$ are constant functions.



%\begin{remark}
%$df_1,\ldots,df_n$ are independent if and only if $X_{f_1},\ldots, X_{f_n}$ are independent. This holds because the map $^{b^m}TM\rightarrow ^{b^m} T^{*}M:u\mapsto \omega(u,\cdot)$ is bijective.
%\end{remark}

%\begin{remark}
%Because $X_{f_1},\ldots,X_{f_n}$ have to be linearly independent, one of the functions has a singularity of maximal degree. Looking at the expression \ref{eq:bmhamiltonianvf} it is easy to see that without the term of maximal degree all the terms in the component $\partial/\partial y_i$ are $0$ at $Z$.
%\end{remark}


In this section we provide some preliminary material that will be needed later:
\begin{proposition}\label{prop:mod_period}
Let $(M,Z,x,\omega)$ be a $b^m$-symplectic manifold such that $Z$ is connected with modular period $k$.
Let $\pi:Z\rightarrow S^1 \simeq \mathbb{R}/k\mathbb{Z}$ be the projection to the base of the corresponding mapping torus.
Let $\gamma: S^1 = \mathbb{R}/k\mathbb{Z} \rightarrow Z$ be any loop such that $\pi\circ\gamma$ is positively oriented and has constant velocity 1. Then the following are equal:
\begin{enumerate}%[i)]
\item The modular period of $Z$,

\item $\int_\gamma \iota_{\mathbb{L}} \omega$,

\item The value $a_{m-1}$ for any $^{b^m}\mathcal{C}^\infty(M)$-function
$$f = a_0 \log(x) + \sum_{j= 1}^{m-1} a_j\frac{1}{x^j} + h$$
such that the hamiltonian vector field $X_f$ has 1-periodic orbits homotopic in $Z$ to some $\gamma$.
\end{enumerate}
\end{proposition}

\begin{proof}

Let us first prove that (1)=(2) and then that (2)=(3).
\begin{itemize}
\item[(1)=(2)] Let us denote by $\mathcal{V}_{mod}$ the modular vector field. Recall from \cite{GMW17} that $\iota_{\mathbb{L}}(\mathcal{V}_{mod})$ is the constant function 1. Let $s:[0,k]\rightarrow Z$ be the trajectory of the modular vector field. Because the modular period is $k$, $s(0)$ and $s(k)$ are in the same leaf $\mathcal{L}$. Let $\hat s :[0,k+1]\rightarrow Z$ a smooth extension of $s$ such that $s|_{[k,k+1]}$ is a path in $\mathcal{L}$ joining $\hat s (k) = s (k)$ to $\hat s (k+1) = s (0)$. This way $\hat s$ becomes a loop. Then,

$$k=\int_0^k 1 dt = \int_S \iota_{\mathbb{L}} \omega = \int_{\hat s } \iota_{\mathbb{L}}\omega=\int_\gamma\iota_{\mathbb{L}}\omega$$

\item[(2)=(3)] Let $r:[0,1] \mapsto Z$ be the trajectory of $X_f$ the hamiltonian vector field of $f$. Recall that $X_f$ satisfies
$$\iota_{X_f}\omega = \sum_{j=1}^m c_j \frac{dx}{x^i} + dh.$$
Let $x^m\frac{\partial}{\partial x}$ be a generator of the linear normal bundle $\mathbb{L}$.
We know that $X_f$ is 1-periodic and its trajectory is homotopic to $\gamma$. Hence,
$$
\begin{array}{rcl}
 k = \int_r \iota_{\mathbb{L}}\omega & = & \displaystyle \int_0^1 \iota_{x^m\frac{\partial}{\partial x}} \omega(X_f|_{r(t)})dt\\
 \\
 & = & \displaystyle \int_0^1 -(\sum_{j=1}^{m}c_i\frac{dx}{x^i} + dh)\cdot(x^m\frac{\partial}{\partial x})|_{r(t)}dt\\
 \\
 & = & -c_m = -a_{m-1}\\

\end{array}
$$
\end{itemize}

\end{proof}

We will also need a Darboux-Carathéodory theorem for $b^m$-symplectic manifolds:
\begin{theorem}[Darboux-Carath\'{e}odory ($b^m$-version)]\label{bmdarbouxcaratheodory}
Let
$$(M^{2n},x,Z, \omega)$$
 be a $b^m$-symplectic manifold and $m$ be a point on $Z$. Let $f_1,\ldots,f_n$ be a $b^m$-integrable system. Then there exist \textcolor{black}{$b^m$}-functions $(q_1,\ldots,q_n)$ around $m$ such that
$$\omega = \sum_{i=1}^n df_i\wedge dq_i$$
and the vector fields $\{X_{f_i},X_{q_j}\}_{i,j}$ commute.
If $f_1$ is not smooth (recall that $f_1 = a_0\log(x) + \sum_{j=1}^{m-1}a_j\frac{1}{x^i}$ with $a_n \neq 0$ on $Z$ and $a_0 \in \mathbb{R}$) the $q_i$ can be chosen to be smooth functions, and $(x,f_2,\ldots,f_n,q_1,\ldots,q_n)$ is a system of local coordinates.
\end{theorem}

\begin{proof}
The first part of this proof is exactly as in \cite{KMS16}.
Assume now $\displaystyle f_1 = a_0\log(x) + \sum_{j=1}^{m-1}a_j\frac{1}{x^i}$.
We modify the induction requiring also that $\mu_i$ (in addition to be in $K_i$) is also in $T^* M \subseteq ^bT^* M$.
We can also ask this extra condition while asking $\mu_i(X_{f_i})= 1$, we only have to check that $X_{f_i}$ does not vanish in $TM$. This is clear because $X_{f_i}$ does not vanish at $^b TM$ and
$$0 = \{f_n,f_i\} = \left(\sum_{i=1}^m \tilde{a}_i\frac{dx}{x^i}\right)(X_{f_i}) = \left(\frac{dx}{x^m} \sum_{i=1}^{m}a_i x^i\right)(X_{f_i}).$$

All the terms in the last expression vanish except for the one of degree $m$.

Then $dx/x^m$ is in the kernel of $X_{f_i}$, hence $X_{f_i}$ does not vanish on $TM$ and the $q_i$ can be chosen to be smooth.

$\{X_x,X_{f_2},\ldots,X_{f_{n}},X_{q_1},\ldots X_{q_n}\}$ commute because $\{X_{f_i},X_{q_i}\}_{i,j}$ commute. Then
$$dx\wedge df_2\ldots\wedge df_n\wedge dq_1 \wedge \ldots \wedge d q_n$$
is a non-zero section of $\bigwedge^n(^b TM)$. And hence
$$(x,f_2,\ldots, f_{n-1},q_1,\ldots,q_n)$$
 are local coordinates.

\end{proof}

Before proceeding with the proof of the action-angle coordinates, we need to prove that in a neighbourhood of a Liouville torus the fibration is semilocally trivial:

\begin{lemma}[Topological Lemma]\label{lemma:topological}
Let $m \in Z$ be a regular point of a $b^m$-integrable system $(M,x,Z,\omega,F)$. Assume that the integral manifold $\mathcal{F}_m$ through $m$ is compact. Then there exists a neighborhood $U$ of $\mathcal{F}_m$ and a diffeomorphism
$$\phi:U \simeq \mathbb{T}^n\times B^n$$
which takes the foliation $\mathcal{F}$ to the trivial foliation $\{\mathbb{T}^n\times\{b\}\}_{b\in B^n}$.
\end{lemma}

\begin{proof}
We follow the steps of \cite{LMV11}. In this case, the only extra step that must be checked is that the foliation given by the $b^m$-hamiltonian vector fields of $F = (f_1,f_2, \ldots, f_n)$ is the same as the one given by the level sets of $\tilde F := (x, f_2,\ldots, f_n)$. In our case $f_1 = a_0\log(x) + \sum_{u=1}^{m-1} a_i\frac{1}{x^i}$, where $a_0 \in \mathbb{R}$, $a_i \in \mathcal{C}^\infty(x)$, $a_{m-1} = 1$. Hence the foliations are the same.
Then as in \cite{LMV11}, we take an arbitrary Riemannian metric on $M$ and this defines a canonical projection $\psi:U \rightarrow \mathcal{F}_m$. Let us define $\phi := \psi\times \tilde F$. We obtain the commutative diagram (Figure \ref{fig:diagram_topological}).

\begin{figure}[H]
\centering
\begin{tikzcd}
U \arrow[rr, dashed, "\phi"] \arrow[rrd, "\tilde F"]& & \mathbb{T}^n\times B^n \arrow[d,"p"]\\
 & & B^n
\end{tikzcd}
\caption{Commutative diagram of the construction of the isomorphism of $b^m$-integrable systems.}
\label{fig:diagram_topological}
\end{figure}

which provides the necessary equivalence of $b^m$-integrable systems.
\end{proof}

%\begin{lemma}[$b^m$-Poincaré]\label{lemma:poincare}
%Let $\psi$ be a $b^m$-form such that $i^*\psi = 0$ where $i$ is the immersion of the
%\end{lemma}
%\begin{proof}
%The proof of this lemma works exactly as in the standard Poincaré lemma, because the homotopy formula works the same way for $b^m$-forms.
%\end{proof}

\section{Action-angle coordinates on $b^m$-symplectic \protect\\ manifolds}


In a neighbourhood of one of our Liouville tori all we can assume about the form of our $b^m$-symplectic structure is  that is given by the Laurent series defined in \cite{Scott16}.

That is to say, we can assume that in a tubular neighborhood $U$ of $Z$
$$\omega = \sum_{j=1}^{m-1}\frac{dx}{x^i}\wedge\pi^*(\alpha_i) + \beta,$$
where $\pi:U\rightarrow Z$ is the projection of the tubular neighborhood onto $Z$, $\alpha_i$ are closed smooth de Rham forms on $Z$ and $\beta$ a de Rham form on $M$ of degree $2$.

In \cite{BKM,anastasiaeva} normal forms are given for group actions in a neighbourhood of the orbit. Below we provide a normal for the integrable system in a neighbourhood of an orbit of the torus action associated to the integrable system. This theorem is finer than the $b^m$-symplectic slice theorem provided in \cite{anastasiaeva} as it also gives information about the first integrals. 

\begin{figure}
\centering
\begin{tikzpicture}
%%%%%%%%%regprogramming

\pgfmathsetmacro{\sizer}{1}
\pgfmathsetmacro{\basept}{5}	

\pgfmathsetmacro{\xone}{0}
\pgfmathsetmacro{\xtwo}{1.5}
\pgfmathsetmacro{\xthree}{3}
\pgfmathsetmacro{\xfour}{4.5}
\pgfmathsetmacro{\xfive}{6}
\pgfmathsetmacro{\xsix}{7.5}
\pgfmathsetmacro{\xseven}{9}

\pgfmathsetmacro{\ymid}{4.25}
\pgfmathsetmacro{\ytop}{5.65}
\pgfmathsetmacro{\ybottom}{2.85}

\DrawFilledDonutops{\xone}{\ymid}{.55*\sizer}{1.2*\sizer}{-90}{vlblue}{very thick}{white}
\DrawFilledDonutops{\xtwo}{\ymid}{.55*\sizer}{1.2*\sizer}{-90}{vlblue}{very thick}{white}
\DrawFilledDonutops{\xthree}{\ymid}{.55*\sizer}{1.2*\sizer}{-90}{vlblue}{very thick}{white}
\DrawFilledDonutops{\xfour}{\ymid}{.55*\sizer}{1.2*\sizer}{-90}{magenta}{very thick}{white}
\DrawFilledDonutops{\xfive}{\ymid}{.55*\sizer}{1.2*\sizer}{-90}{vlblue}{very thick}{white}
\DrawFilledDonutops{\xsix}{\ymid}{.55*\sizer}{1.2*\sizer}{-90}{vlblue}{very thick}{white}
\DrawFilledDonutops{\xseven}{\ymid}{.55*\sizer}{1.2*\sizer}{-90}{vlblue}{very thick}{white}

\DrawDonut{\xone}{\ymid}{.55*\sizer}{1.2*\sizer}{-90}{dblue}{very thick}
\DrawDonut{\xtwo}{\ymid}{.55*\sizer}{1.2*\sizer}{-90}{dblue}{very thick}
\DrawDonut{\xthree}{\ymid}{.55*\sizer}{1.2*\sizer}{-90}{dblue}{very thick}
\DrawDonut{\xfour}{\ymid}{.55*\sizer}{1.2*\sizer}{-90}{dred}{very thick}
\DrawDonut{\xfive}{\ymid}{.55*\sizer}{1.2*\sizer}{-90}{dblue}{very thick}
\DrawDonut{\xsix}{\ymid}{.55*\sizer}{1.2*\sizer}{-90}{dblue}{very thick}
\DrawDonut{\xseven}{\ymid}{.55*\sizer}{1.2*\sizer}{-90}{dblue}{very thick}

\draw[very thick,magenta,vdblue, ->](\xone, \ybottom - 0.3) -- +(0, -1);
\draw[very thick, vdblue,  ->] (\xtwo, \ybottom - 0.3) -- +(0, -1);
\draw[very thick, vdblue, ->] (\xthree, \ybottom - 0.3) -- +(0, -1);
\draw[very thick, red, ->] (\xfour, \ybottom - 0.3) -- +(0, -1);
\draw[very thick,vdblue, ->] (\xfive, \ybottom - 0.3) -- +(0, -1);
\draw[very thick,vdblue, ->] (\xsix, \ybottom - 0.3) -- +(0, -1);
\draw[very thick,vdblue, ->] (\xseven, \ybottom - 0.3) -- +(0, -1);
\draw[ultra thick, purple](\xone - 1, \ybottom - 1.5) -- (\xseven + 1, \ybottom - 1.5);
\end{tikzpicture}
\caption{Fibration by Liouville tori: The middle fiber of the point $p \in Z$ in magenta, the neighbouring Liouville tori in blue.}
\label{fig:tori}
\end{figure}

One of the non-trivial steps of the proof is to associate a toric action to the integrable system. The connection to normal forms of group actions will become even more evident when we discuss the associated cotangent models.



\begin{theoremA}[Action-angle coordinates for $b^m$-symplectic manifolds]
Let $(M,x,\omega, F)$ be a $b^m$-integrable system, where $F = (f_1 = a_0 \log(x) + \sum_{j=1}^{m-1} a_j\frac{1}{x^j}, \ldots,f_n)$ with $a_j$  for $j>1$ functions in $x$. Let $m\in Z$ be a regular point and let us assume that the integral manifold of the distribution generated by the $X_{f_i}$ through $m$ is compact. Let $\mathcal{F}_m$ be the Liouville torus through $m$.
Then, there exists a neighborhood $U$ of $\mathcal{F}_m$ and coordinates $(\theta_1,\ldots,\theta_n,\sigma_1,\ldots,\sigma_n):\mathcal{U}\rightarrow\mathbb{T}^n\times B^n$ such that:

\begin{enumerate}
\item We can find an equivalent integrable system $F = (f_1 = a_0'\log(x) + \sum_{j=1}^{m-1} a_j'\frac{1}{x^j}, \ldots, f_n)$ such that the coefficients $a_0',\ldots, a_{m-1}' $ of $f_1$ are constants $\in \mathbb{R}$,
\item $$\omega|_\mathcal{U} = \left(\sum_{j=1}^m c_j'\frac{c}{\sigma_1^j}d\sigma_1\wedge d\theta_1\right) + \sum_{i=2}^{n} d \sigma_i\wedge d\theta_i$$ where $c$ is the modular period and $c_j' = -(j-1)a_{j-1}'$, also
\item the coordinates $\sigma_1,\ldots,\sigma_n$ depend only on $f_1,\ldots f_n$.
\end{enumerate}

%Also, \textcolor{black}{$\langle c\cdot c_j', X \rangle = \alpha_j(X)$ for $X \in \mathfrak{t}$}.
\end{theoremA}

\begin{proof}
The idea of this proof is to construct an equivalent $b^m$-integrable system whose fundamental vector fields define a $\mathbb{T}^n$-action on a neighborhood of $\mathbb{T}^n\times\{0\}$.
It is clear that all the vector fields $X_{f_1},\ldots,X_{f_n}$ define a torus action on each Liouville tori $\mathbb{T}^n\times\{b\}$ where $b\in B^n$, but this does not guarantee that their flow defines a toric action on all $\mathbb{T}^n\times B^n$.
The proof is structured in three steps. The first one is the uniformization of the periods, i.e. we define an $\mathbb{R}^n$-action on a neighborhood of $\mathbb{T}^n\times\{0\}$ such that the lattice defined by its kernel at every point is constant. This allows to induce an actual action of a torus (as the periods are constant) of rank n: A $\mathbb T^n$ action by taking quotients. The second step consists in checking that this action is actually $b^m$-Hamiltonian. And in the final step we apply theorem \ref{bmdarbouxcaratheodory} to obtain the expression of $\omega$.
\begin{enumerate}

\item Uniformization of periods.

Let $\Phi_{X_F}^s$ be defined as the joint flow by the Hamiltonian vector fields of the action:
\begin{equation}\label{eq:action1}
\begin{array}{rcl}
 \Phi: \mathbb{R}^n\times(\mathbb{T}^n\times B^n)& \rightarrow & (\mathbb{T}^n\times B^n)\\
 ((s_1,\ldots,s_n),(x,b)) & \mapsto & \Phi_{X_{f_1}}^{s_1}\circ\cdots\circ\Phi_{X_{f_n}}^{s_n}((x,b))\\

\end{array}
\end{equation}

this defines an $\mathbb{R}^n$-action on $\mathbb{T}^n\times B^n$.
For each $b\in B^n$ at a single orbit $\mathbb{T}^n\times\{b\}$ the kernel of this action is a discrete subgroup of $\mathbb{R}^n$. We will denote the lattice given by this kernel $\Lambda_b$. Because the orbit is compact, the rank of $\Lambda_b$ is maximal i.e. $n$. This lattice is known as the period lattice of $\mathbb{T}^n\times\{b\}$ as we know by standard arguments in group theory that the lattice has to be of maximal rank so as to have a torus as a quotient.
In general we can not assume that $\Lambda_b$ does not depend on $b$. The process of uniformization of the periods modifies the action \ref{eq:action1} in such a way that $\Lambda_b = \mathbb{Z}^n$ for all $b$.
Let us consider the following Hamiltonian vector field $\sum_{i=1}^n k_iX_{f_i}$. The $b^m$-function that generates this Hamiltonian vector field is:
$$k_1\left(a_0\log(x) + \sum_{j=1}^{m-1}a_j\frac{1}{x^j}\right) + \sum_{i=2}^n k_i f_i$$
where recall that $a_{m-1}$ is constant equal 1. Observe that the coefficient multiplying $1/x^{m-1}$ is $k_1$. By proposition \ref{prop:mod_period} $k_1 = c$ the modular period. In this case $c = [\alpha_m]$.

Hence, for $b\in B^{n-1}\times\{0\}$ the lattice $\Lambda_b$ is contained in $\mathbb{R}^{n-1}\times c \mathbb{Z} \subseteq \mathbb{R}^n$.
Pick $(\lambda_1,\ldots,\lambda_n): B^n\rightarrow \mathbb{R}^n$ such that:
\begin{itemize}%[a)]
\item $(\lambda_1(b),\ldots,\lambda_n(b))$ is a basis of $\Lambda_b$ for all $b\in B^n$,
\item \textcolor{black}{$\lambda_i^n$ vanishes along $B^{n-1}\times\{0\}$ at order $m$ for $i<n$ and $\lambda_i$ is equal to $c$ along $B^{n-1}\times\{0\}$.}
\end{itemize}
In the previous points, $\lambda_i^j$ denotes the $j$-th component of $\lambda_i$. The first condition can be satisfied by using the implicit function theorem. That is because $\Phi(\lambda,m) = m$ is regular with respect to the $s$ coordinates. The second condition is automatically true because $\Lambda_b \subseteq \mathbb{R}^{n-1}\times c\mathbb{Z}$. We define the uniformed flow as:
\begin{equation}\label{eq:action2}
\begin{array}{rcl}
\tilde \Phi: \mathbb{R}^n\times(\mathbb{T}^n\times B^n)& \rightarrow & (\mathbb{T}^n\times B^n)\\
((s_1,\ldots,s_n),(x,b))& \mapsto & \Phi(\sum_{i=1}^n s_i \lambda_i,(x,b))\\

\end{array}
\end{equation}
\item The $\mathbb{T}^n$-action is $b^m$-Hamiltonian.
The objective of this step is to find $b^m$-functions $\sigma_1,\ldots,\sigma_n$ such that $X_{\sigma_i}$ are the fundamental vector fields of the $\mathbb{T}^n$-action $Y_i = \sum_{j=1}^n \lambda_i^j X_{f_j}$.

By using the Cartan formula for a $b^m$-symplectic form, we obtain:

$$
\begin{array}{rcl}
\mathcal{L}_{Y_i}\mathcal{L}_{Y_i}\omega & = & \mathcal{L}_{Y_i}(d(\iota_{Y_i}\omega) + \iota_{Y_i}d\omega)\\
 & = &  \mathcal{L}_{Y_i}(d(-\sum_{j=1}^{n} \lambda_i^j df_i))\\
 & = &  -\mathcal{L}_{Y_i}(\sum_{j=1}^n d\lambda_i^j\wedge df_j) = 0\\
\end{array}
$$

Note that $\lambda_i^j$ are constant on the level sets of $F$ as $\Phi(\lambda, m) = m$  and the level sets of $F$ are invariant by $\Phi$.

Recall that if $Y$ is a complete periodic vector field and $P$ is a bivector such that $\mathcal{L}_Y\mathcal{L}_Y P = 0$, then $\mathcal{L}_Y P = 0$.
So, the vector fields $Y_i$ are Poisson vector fields.
To show that each $\iota_{Y_i}\omega$ has a $^{b^m}\mathcal{C}^\infty$ primitive we will see that $[\iota_{Y_i}\omega] = 0$ in the $b^m$-cohomology.

One one hand, if $i >1$, $\iota_{Y_i}\omega$ vanishes at $Z$. This holds because $Y_i$ has not any component $\partial/\partial Y$.

Recall Proposition 6 from \cite{GMP14}:


\begin{proposition} If $\omega \in ^b \Omega(M)$ with $ \omega|_Z=0$, then $\omega\in \Omega(M)$.
\end{proposition}
In  a similar way for $b^m$-forms we have,


\begin{proposition} If $\omega \in ^{b^m} \Omega(M)$ with $\omega|_Z$ vanishing up to order $m$, then $\omega \in \Omega(M)$.
\end{proposition}


Thus as $\iota_{Y_i}\omega$ vanishes at $Z$, the $b^m$-forms  $\iota_{Y_i}\omega$ are indeed  smooth.
Thus  we can now apply the standard Poincaré lemma and as these forms are closed they are locally exact. This proves that all the vector fields $Y_i$ with $i>1$ are indeed Hamiltonian.

On the other hand, the fact that $\iota_{Y_1} \omega = c df_1$ is obvious.



Then, because we have a toric action that is Hamiltonian, we can use lemma 3.2 in \cite{GMP17}, and  we get an equivalent system such that $a_i$ are all constant and moreover $\langle a_i', X\rangle = \alpha_i(X^\omega)$. Note that by dividing by $a_{m-1}'$, we can still assume $a_{m-1}'=1$ to be consistent with our notation, but  we then have to multiply $f_1\cdot c$ in the next step.


\item Apply Darboux-Carathéodory theorem.

The construction above gives us some candidates $\sigma_1 = c f_1,\sigma_2,\ldots,\sigma_n$ for the action coordinates.

We now apply the Darboux-Carathéodory theorem and express the form in terms of $x$:
$$\omega = \left(\sum_{j= 1}^{m} c \frac{c_j}{x^j} dx\wedge d q_1\right)+ \sum_{i=2}^{n}d\sigma_i\wedge dq_i.$$

Since the vector fields $X_{\sigma_i} = \frac{\partial}{\partial q_i}$ are fundamental fields of the $\mathbb{T}^n$-action the flow \ref{eq:action2} gives a linear action on the $q_i$ coordinates.

Observe that the coordinate system is only defined in $\mathcal{U}$. It may not be valid at points outside $\mathcal{U}$ that may be in the orbit of points in $\mathcal{U}$. Let us see that the charts can be extended to these points.

Define $\mathcal{U}'$ the union of all tori that intersect $\mathcal{U}$.
We will see that the coordinates are valid at $\mathcal{U}'$.

Let $\{p_i,\theta_j\}$ be the extension of $\{\sigma_i,q_j\}$. It is clear that $\{p_i,\theta_j\} = \delta_{ij}$ by its construction in the Darboux-Carathéodory theorem.

To see that $\{\theta_i,\theta_j\} = 0$ we take the flows by $X_{p_k}$ and extend the expression to the whole $\mathcal{U}'$:

$$X_{p_k}(\{\theta_i, \theta_j\}) = \{\{\theta_i, \theta_j\},p_k\} = \{\theta_i,\delta_{ij}\} - \{\theta_j, \delta_{jk}\} = 0.$$

The fact that $\omega$ is preserved is obvious because $X_{p_k}$ are hamiltonian vector fields and thus they preserve the $b^m$-symplectic forms. Moreover, $t,\theta_1,p_2,\theta_2,\ldots,p_n,\theta_n$ are independent on $\mathcal{U}'$ and hence are a coordinate system in a neighbourhood of the torus.

\end{enumerate}
\end{proof}

\begin{remark}
In the proof  we have seen that there exists an equivalent integrable system where the coefficients of the singular function are indeed constant. From now on, when considering a $b^m$-integrable system we are going to make this assumption.
\end{remark}

\begin{remark}
    By means of the desingularization transformation we may obtain an action-angle coordinate theorem for folded manifolds as we do in Part 3 for the KAM theorem for folded symplectic manifolds. This folded action-angle theorem is a particular case of the one obtained in \cite{EvaRobert}.
\end{remark}

\chapter[Action-angle coordinates and cotangent lifts]{Reformulating the action-angle coordinate via cotangent lifts}

The action-angle theorem for symplectic manifolds (also known as action-angle coordinate theorem) can be reformulated in terms of a cotangent lift.


Recall that given a Lie group action on any manifold its cotangent lifted action is automatically Hamiltonian. By considering the action of a torus on itself by translations this action can be lifted to its cotangent bundle and give a semilocal normal form theorem as the Arnold-Liouville-Mineur theorem for symplectic manifolds. If we now replace this cotangent lift to the cotangent bundle to a lift to the $b^m$-cotangent bundle we obtain the semilocal normal form of the main theorem of this chapter.


Let start recalling the symplectic and $b$-symplectic case following \cite{KM17}.


\section{Cotangent lifts and Arnold-Liouville-Mineur in Symplectic Geometry}

Let $G$ be a Lie group and let $M$ be any smooth manifold. Given a group action $\rho:G\times M\longrightarrow M$, we define its cotangent lift as the action on $T^\ast M$ given by $\hat{\rho_g}:=\rho^\ast_{g^-1}$ where $g\in G$. We then have a commuting diagram





\begin{figure}[h]
\centering
\begin{tikzcd}
T^\ast M  \arrow[rr, "\hat{\rho_g}"] \arrow[d,"\hat{\pi}"] & & T^\ast M \arrow[d,"\pi"]\\
M  \arrow[rr, "\rho_g"]  & & M
\end{tikzcd}
\caption{Commutiative diagram of the construction of the isomorphism of $b^m$-integrable systems.}
\label{fig:diagram_cotangent_lift}
\end{figure}

where $\pi$ is the canonical projection from $T^\ast M$ to $M$.

 The cotangent bundle $T^*M$ is a symplectic manifold endowed
with the exact symplectic form given by the differential of the Liouville one-form $\omega=-d\lambda$. The Lioville one-form can be defined intrinsically:
\begin{equation}\label{liouvilleform}
 \langle \lambda_p, v\rangle:= \langle p, (\pi_p)_\ast (v)\rangle
\end{equation}
with  $v\in T(T^*M), p\in T^*M$.

A standard argument (see for instance \cite{GS90}) shows that the cotangent lift $\hat{\rho}$  is Hamiltonian with moment map $\mu:T^*M \to \mathfrak{g}^*$ given by
\begin{equation*}\label{eqn:lift}
\langle\mu(p),X \rangle := \langle \lambda_p ,X^\#|_{p} \rangle =\langle p,X^\#|_{\pi(p)}\rangle,
\end{equation*}
where  $p\in T^*M$, $X$ is an element of the Lie algebra $\mathfrak{g}$ and we use the same symbol $X^\#$ to denote the fundamental vector field of $X$ generated by the action on $T^\ast M$ or $M$.
This  construction is known  as the {\bf cotangent lift}.

 In the special case where the manifold $M$ is a torus $\T^n$ and the group is $\T^n$ acting by translations, we obtain the following explicit structure: Let $\theta_1,\ldots,\theta_n$  be the standard ($S^1$-valued) coordinates on $\T^n$ and let
\begin{equation}\label{co}
\underbrace{\theta_1,\ldots,\theta_n}_{=:\theta}, \underbrace{t_1, \ldots, t_n}_{=:t}
\end{equation}
be the corresponding chart on $T^\ast \T^n$, i.e. we associate to the coordinates \eqref{co} the cotangent vector $\sum_i t_i d \theta_i \in T^\ast_\theta \T^n$.
The Liouville one-form is given in these coordinates by
$$ \lambda = \sum_{i=1}^n t_i d \theta_i $$
and its negative differential is the standard symplectic form on $T^\ast \T^n$:
\begin{equation}\label{eq:omegacan}
\omega_{can} = \sum_{i=1}^n d \theta_i \wedge d t_i .
\end{equation}
%
Denoting by $\tau_\beta$ the translation by $\beta \in \T^n$ on $\T^n$, its lift to $T^\ast \T^n$ is given by
$$ \hat \tau_\beta: (\theta, t) \mapsto (\theta + \beta, t).$$
The moment map $\mu_{can}: T^\ast \T^n \to \mathfrak{t^\ast} $ of the lifted action with respect to the canonical symplectic form is
\begin{equation}\label{eq:mucan}
\mu_{can}(\theta,t) = \sum_i t_i d\theta_i,
\end{equation}
where the $\theta_i$ on the right hand side are understood as elements of $ \mathfrak{t^\ast}$ in the obvious way. Even simpler, if we identify $ \mathfrak{t^\ast}$ with $\R^n$ by choosing the standard basis $\frac{\partial}{\partial \theta_i}$ of  $\mathfrak{t}$ then the moment map is just the projection onto the second component of $T^\ast \T^n \cong \T^n \times \R^n$.  Note that the components of $\mu$ naturally define an integrable system on $T^\ast \T^n$.


 We can rephrase the Arnold-Liouville-Mineur theorem in terms of the symplectic cotangent model:

\begin{theorem} Let $F=(f_1,\ldots,f_n)$ be an integrable system on the symplectic manifold $(M,\omega)$. Then semilocally around a regular Liouville torus the system is equivalent to the cotangent model $(T^\ast \T^n)_{can}$ restricted to a neighbourhood of the zero section $(T^\ast \T^n)_0$ of $T^\ast \T^n$.
\end{theorem}


\section{The case of $b^m$-symplectic manifolds}



Let us start by introducing the twisted $b^m$-cotangent model for torus actions. This model has additional invariants: the modular vector field of the connected component of the critical set and the modular weights of the associated toric action.
Consider $T^\ast \T^n$ be endowed with the standard coordinates $(\theta, t)$, $\theta \in \T^n$, $t \in \R^n$ and consider again the action on $T^\ast \T^n$ induced by lifting translations of the torus $\T^n$. We will now  view this action as a $b^m$-Hamiltonian action with respect to a suitable $b^m$-symplectic form. In analogy to the classical Liouville one-form we define the following  non-smooth one-form away from the hypersurface $Z=\{t_1 = 0\}$~:

$$ \left(c c_1 \log|t_1| + \sum_{i=2}^{m}c c_i \frac{t_1^{-(i-1)}}{-(i-1)}\right) d \theta_1 + \sum_{i=2}^n t_i d\theta_i.$$

When differentiating this form we obtain a $b^m$-symplectic form on $T^\ast \T^n$ which we call (after a sign change) the {\bf twisted $b^m$-symplectic form  }on $T^\ast \T^n$ with invariants $(c c_1, \dots, c c_m)$:
\begin{equation}\label{eq:twistedform}
 \omega_{tw, c}:=\left(\sum_{j=1}^m c_j\frac{c}{t_1^j}d t_1\wedge d\theta_1\right) + \sum_{i=2}^{n} d t_i\wedge d\theta_i,
\end{equation}
where $c$ is the modular period.
The moment map of the lifted action  is then given by
\begin{equation}\label{eq:bmucan}\mu_{tw, q_0, \dots, q_{m-1})}:=( q_0 \log|t_1| + \sum_{i=2}^{m}q_i t_1^{-(i-1)} ,t_2, \ldots, t_n),
\end{equation}
where we are identifying $\mathfrak{t^\ast}$ with $\R^n$ and  $c_j = -(j-1)q_{j-1}$.

We call this lift together with the $b^m$-symplectic form \ref{eq:twistedform} the {\bf twisted $b^m$-cotangent lift} with modular period $c$ and invariants $(c_1, \dots, c_m)$. Note that the components of the moment map define a $b^m$-integrable system on $(T^\ast \T^n, \omega_{tw, (c c_1, \dots, c c_m)})$.

The model of twisted $b^m$-cotangent lift allows us to express the action-angle coordinate theorem for $b^m$-integrable systems in the following way:

\begin{theorem} Let $F=(f_1,\ldots,f_n)$ be a $b^m$-integrable system on the $b^m$-symplectic manifold $(M,\omega)$.  Then semilocally around a regular Liouville torus $\mathbb T$, which lies inside the critical hypersurface $Z$ of $M$, the system is equivalent to the cotangent model $(T^\ast \T^n)_{tw, (c c_1, \dots, c c_m)} $ restricted to a neighbourhood of $(T^\ast \T^n)_0$. Here $c$ is the modular period of the connected component of $Z$ containing  $\mathbb T$ and the constants $(c_1, \dots, c_m)$ are the invariants associated to the integrable system and its associated toric action.
\end{theorem}


\endinput

%-----------------------------------------------------------------------
% End of chap1.tex
%-----------------------------------------------------------------------
