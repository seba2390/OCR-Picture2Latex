\part{Introduction and preliminaries}
\chapter{Introduction}
\label{ch:introduction}

Both symplectic and Poisson geometry emerge from the study of classical mechanics. Both are broad fields widely studied and with powerful results. But as Poisson structures are far more general than the symplectic ones, most outstanding results in symplectic geometry do not translate well to Poisson manifolds. $b^m$-Poisson structures, also known as $b^m$-symplectic structures, bridge this gap by extending symplectic structures in a controlled manner. This allows fundamental results in symplectic geometry to carry over to $b^m$-symplectic geometry. However, adapting theories such as deformation or Moser theory requires additional work, as discussed in \cite{GMPS} and other sources.

%Here is where $b^m$-Poisson structures come to play. $b^m$-Poisson structures (or $b^m$-symplectic structures) lie somewhere between these two worlds. They extend symplectic structures but in a really controlled way. This is why fundamental results in symplectic geometry still work in $b^m$-symplectic geometry. However, an adaptation of these theories like deformation or Moser theory requires some work (see \cite{GMPS} and others).

The study of $b^m$-Poisson geometry sparked from the study of symplectic manifold with boundary (see \cite{Melrose93} and \cite{NT96}). In the last years, the interest in this field increased after the classification result for $b$-Poisson structures obtained in \cite{Radko02}. Later on, \cite{GMP14} translated these structures to the language of forms and started applying symplectic tools to study them. A lot of papers in the following years studied different aspects of these structures: \cite{GMP10}, \cite{GMP14}, \cite{GMP15}, \cite{GMW17}, \cite{MO2} and  \cite{GUAL14} are some examples.

Inspired by the study of manifolds with boundary, we work on a pair of manifolds $(M,Z)$ where $Z$ is a hypersurface and call this pair $b$-manifold

In this context, \cite{Scott16} generalized the $b$-symplectic forms by allowing higher degrees of degeneracy of the Poisson structures. The $b^m$-symplectic structures inherit most of the properties of $b$-symplectic structures. This booklet focuses on  different aspects of the investigation of $b^m$-symplectic structures covering mainly integrable systems and KAM theory. First, we present some preliminary notions necessary to address the problem of perturbation.  We present an action-angle theorem for $b^m$-Poisson structures and state and prove the KAM theory equivalent in manifolds with $b^m$-symplectic structures. 


\section{Structure and results of this monograph}

\subsection{Part $1$: Introduction and Preliminaries}

In the preliminaries, we give the basic notions that lead to the questions we are addressing in this booklet. In the first part, we introduce the concept of $b$-Poisson manifolds or $b$-symplectic manifolds, a class of Poisson manifold which is symplectic outside a critical hypersurface. It study comes motivated by the investigation of manifolds with boundary. Next, we talk about a generalization of these structures, that allows a higher degree of degeneracy of the structure: the $b^m$-symplectic structures. These structures are the main focus of our investigations. A key concept that will play an important role in this book is the study of the desingularization of these singular structures. Finally, we give a short introduction to KAM theory, a theory that will be generalized in the setting of $b^m$-manifolds in the last chapter.



Motivation comes from several examples of singular symplectic structures appearing naturally in classical problems of celestial mechanics which are discussed in the last chapter of the monograph. We also describe the difficulties of finding these examples and the subtleties of dealing with these singular structures in the exploration of conservative systems.









\subsection{Part $2$: Action-angle coordinates and cotangent models for $b^m$-integrable systems}

In this Chapter we define the concept of $b^m$-functions and $b^m$-integrable systems. We present several examples of $b^m$-integrable systems that come from classical mechanics. After all this, we present a version of the action-angle theorem for $b^m$-symplectic manifolds.

\begin{theoremA}
Let $(M,x,\omega,F)$ be a $b^m$-integrable system, where $F = (f_1 = a_0 \log(x) + \sum_{j=1}^{m-1} a_j\frac{1}{x^j}, f_2,\ldots,f_n)$. Let $m\in Z$ be a regular point, and such that the integral manifold through $m$ is compact. Let $\mathcal{F}_m$ be the Liouville torus through $m$.
Then, there exists a neighborhood $U$ of $\mathcal{F}_m$ and coordinates $(\theta_1,\ldots,\theta_n,\sigma_1,\ldots,\sigma_n):\mathcal{U}\rightarrow\mathbb{T}^n\times B^n$ such that:

\begin{enumerate}
\item We can find an equivalent integrable system $F = (f_1 = a_0'\log(x) + \sum_{j=1}^{m-1} a_j'\frac{1}{x^j})$ such that $a_0',\ldots, a_{m-1}' \in \mathbb{R}$,
\item $$\omega|_\mathcal{U} = \left(\sum_{j=1}^m c_j'\frac{c}{\sigma_1^j}d\sigma_1\wedge d\theta_n\right) + \sum_{i=2}^{n} d \sigma_i\wedge d\theta_i$$ where $c$ is the modular period and $c_j' = -(j-1)a_{j-1}'$, also
\item the coordinates $\sigma_1,\ldots,\sigma_n$ depend only on $f_n,\ldots f_n$.
\end{enumerate}

\end{theoremA}

\subsection{Part $3$: KAM theory on $b^m$-symplectic manifolds and applications to Celestial Mechanics}
In this chapter we provide several  KAM theorems for (singular) symplectic manifolds including $b^m$-symplectic manifolds.

 We begin by considering perturbation theory for $b^m$-symplectic manifolds. Then we give an outline of how to construct the $b^m$-symplectomorphism that will be the main character of the proof of the KAM theorem for $b^m$-symplectic manifolds. After this, we show some technical results that are needed for the proof. These technical results even if quite similar to the standard KAM equivalents, have some subtleties that need to be addressed. We end the chapter with the proof of the $b^m$-KAM theorem and several applications to establish KAM theorems in other singular situations (folded symplectic manifolds) and on symplectic manifolds with prescribed invariant hypersurfaces.

The first KAM theorem is the following:

\textcolor{black}{
\begin{theoremB}
Let $\mathcal{G} \subset \mathbb{R}^n$, $n\geq 2$ be a compact set.
Let $H(\phi, I) = \hat h (I) + f(\phi,I)$, where $\hat h$ is a $b^m$-function $\hat h (I) = h(I) + q_0 \log(I_1) + \sum_{i=1}^{m-1} \frac{q_i}{I_1^i}$ defined on $\mathcal{D}_\rho(G)$, with $h(I)$ and $f(\phi,I)$ analytic.
Let $\hat u = \frac{\partial \hat h}{\partial I}$ and $u = \frac{\partial h}{\partial I}$.
Assume $|\frac{\partial u}{\partial I}|_{G,\rho_2} \leq M$, $|u|_\xi \leq L$.
Assume that $u$ is $\mu$ non-degenerate ($|\frac{\partial u}{\partial I}|\geq \mu|v|$ for some $\mu \in \mathbb{R}^+$ and $I \in \mathcal{G}$. Take $a = 16M$.
Assume that $u$ is one-to-one on $\mathcal{G}$ and its range $F = u(\mathcal{G})$ is a $D$-set.
Let $\tau>n-1,\gamma>0$ and $0 < \nu < 1$. Let
\begin{enumerate}
\item \begin{equation}\label{eq:int_kam1}
\varepsilon:=\|f\|_{\mathcal{G}, \rho} \leq \frac{\nu^2 \mu^2 \hat \rho^{2\tau+2}}{2^{4\tau+32}L^6M^3} \gamma^2,
\end{equation}
\item \begin{equation}\label{eq:int_kam2}
\gamma \leq \min(\frac{8LM\rho_2}{\nu \hat \rho^{\tau+1}}, \frac{L}{\mathcal{K}'})
\end{equation}
\item \begin{equation}\label{eq:int_kam3}
\mu \leq \min(2^{\tau+5}L^2 M,2^7\rho_1 L^4 K^{\tau+1},\beta\nu^{\tau+1}2^{2\tau+1}\rho_1^\tau),
\end{equation}
\end{enumerate}
where $\hat \rho := \min \left(\frac{\nu\rho_1}{12(\tau+2)},1\right)$.
Define the set $\hat G = \hat G_\gamma := \{I \in  \mathcal{G}-\frac{2\gamma}{\mu} | u(I) \text{ is } \tau,\gamma,c,\hat q- Dioph.\}$.
Then, there exists a real continuous map $\mathcal{T}: \mathcal{W}_{\frac{\rho_1}{4}}(\mathbb{T}^n)\times \hat G \rightarrow \mathcal{D}_\rho(\mathcal{G})$ analytic with respect the angular variables such that
\begin{enumerate}
\item\label{kam:point1} For all $I \in \hat G$ the set $\mathcal{T}(\mathbb{T}^n\times \{I\})$ is an invariant torus of $H$, its frequency vector is equal to $u(I)$.
\item\label{kam:point2} Writing $\mathcal{T}(\phi,I)=(\phi + \mathcal{T}_\phi(\phi,I), I + \mathcal{T}_I(\phi,I))$ with estimates
$$|\mathcal{T}_\phi(\phi,I)| \leq \frac{2^{2\tau + 15} M L^2}{\nu^2 \hat \rho^{2\tau+1}}\frac{\varepsilon}{\gamma^2}$$
$$|\mathcal{T}_I(\phi,I))| \leq \frac{2^{10+\tau} L (1+M)}{\nu \hat \rho^{\tau+1}}\frac{\varepsilon}{\gamma}$$
\item\label{kam:point3} $\text{meas} [(\mathbb{T}^n\times \mathcal{G})\setminus\mathcal{T}(\mathbb{T}^n\times \hat G)] \leq C \gamma$ where $C$ is
a really complicated constant depending on $n$,  $\mu$,  $D$,  $\text{diam} F$,  $M$, $\tau$, $\rho_1$, $\rho_2$, $K$ and $L$.
\end{enumerate}
\end{theoremB}
}


Also, we obtain a way to associate a standard symplectic integrable system or a folded integrable system to a $b^m$-integrable system, depending on the parity of $m$. This is done in such a way that the dynamics of the desingularized system are the same as the dynamics of the original one. So it defines a \emph{honest} desingularization of the integrable system.

\begin{theoremC}
The desingularization transforms a $b^m$-integrable system into an integrable system  on a symplectic manifold for even $m$. For $m$ odd, the desingularization associates to it  a folded integrable system. The integrable systems satisfy:
$$X_{f_j}^\omega = X_{f_{j\epsilon}}^{\omega_\epsilon}.$$
\end{theoremC}

By employing this desingularization technique in conjunction with the previous $b^m$-KAM theorem, we are able to derive two novel KAM theorems. The first of these theorems is a KAM theorem for standard symplectic manifolds, where the perturbation has a particular expression. This result is more restrictive than the standard KAM theorem but allows us to guarantee that the perturbations leave a given hypersurface invariant. This means that the tori belonging to that hypersurface remain on the hypersurface after the perturbation. There are various scenarios and contexts where this can be advantageous, such as in Celestial Mechanics problems where it is desirable to monitor a specific hypersurface, such as the line at infinity. The higher-order singularities allow us to consider perturbations that are tangent to the hypersurface up to a certain order.

\begin{theoremD}
Consider a neighborhood of a Liouville torus of an integrable system $F_\varepsilon$ as in \ref{eq:desingularized_even} of a symplectic manifold $(M, \omega_\varepsilon)$ semilocally endowed with coordinates $(I,\phi)$, where $\phi$ are the angular coordinates of the torus, with $\omega_\varepsilon = c' dI_1 \wedge d\phi_i + \sum_{j= 1}^n dI_j\wedge d\phi_j$. Let $H=(m-1)c_{m-1}c' I_1 + h(\tilde I) + R(\tilde I,\tilde \phi)$ be a nearly integrable system where
$$
\left\{
\begin{array}{rcl}
\tilde I_1 & = & c'\frac{I_1^{m+1}}{m+1},\\
\tilde \phi_1 & = & c' I_1^m \phi_1 ,
\end{array}
\right.
$$
and
$$
\left\{
\begin{array}{rcl}
\tilde I & = & (\tilde I_1, I_2, \ldots, I_n),\\
\tilde \phi & = & (\tilde \phi_1, \phi_2, \ldots, \phi_n).
\end{array}
\right.
$$
Then the results for the $b^m$-KAM theorem \ref{th:bm_kam} applied to $H_{\text{sing}} = \frac{1}{I_1^{2k-1}} + h(I) + R(I,\phi)$ hold also for this desingularized system.
\end{theoremD}

The second one is a KAM theorem for folded-symplectic manifolds, where KAM theory has not been considered to date.

\begin{theoremE}
Consider a neighborhood of a Liouville torus of an integrable system $F_\varepsilon$ as in \ref{eq:desingularized_odd} of a folded symplectic manifold $(M, \omega_\varepsilon)$ semilocally endowed with coordinates $(I,\phi)$, where $\phi$ are the angular coordinates of the Torus, with $\omega_\varepsilon = 2cI_1 dI_1 \wedge d\phi_1 + \sum_{j=2}^m dI_j \wedge d\phi_j$.
Let $H = (m-1)c_{m-1} cI_1^2 + h(\tilde I) + R(\tilde I, \tilde \phi)$ a nearly integrable system with
$$
\left\{
\begin{array}{rcl}
\tilde I_1 & = &  2c\frac{I_1^{m+2}}{m+2},\\
\tilde \phi_1 & = &  2c I_1^{m+1} \phi_1 ,
\end{array}
\right.
$$
and
$$
\left\{
\begin{array}{rcl}
\tilde I & = & (\tilde I_1, I_2, \ldots, I_n),\\
\tilde \phi & = & (\tilde \phi_1, \phi_2, \ldots, \phi_n).
\end{array}
\right.
$$
Then the results for the $b^m$-KAM theorem \ref{th:bm_kam} applied to $H_{\text{sing}} = \frac{1}{I_1^{2k}} + h(I) + R(I,\phi)$ also hold for this desingularized system.
\end{theoremE}

Last but not least, we illustrate the connection between $b^m$-symplectic structures and classical mechanics by providing several examples. Several potential applications to celestial mechanics and fluid dynamics are discussed.


\chapter{A primer on singular symplectic manifolds}
\label{ch:preliminaries}

In this first chapter of the booklet we introduce basic notions
 on singular symplectic structures, as well as some concepts on standard KAM theory. Those are the two main pillars of this monograph.

{Let $M$ be a smooth manifold, a \textbf{Poisson structure} on $M$ is a bilinear map
$\{\cdot, \cdot\}:C^\infty(M)\times C^\infty(M) \rightarrow C^\infty(M)$
which is skew-symmetric and satisfies both the Jacobi identity and the Leibniz rule. It is possible to express $\{f,g\}$ in terms of a bivector field via the following equality $\{f,g\}=\Pi(df\wedge dg)$ with $\Pi$ a section of $\Lambda^2(TM)$. $\Pi$ is the associated \textbf{Poisson bivector}. We will use indistinctively the {terminology} of Poisson structure when referring to the bracket or the Poisson bivector. }

A {\emph{$b$-Poisson {bivector field}}} on a manifold $M^{2n}$ is a Poisson bivector such that the map
\begin{equation}\label{eq:transverse}
F: M \rightarrow \bigwedge^{2n} TM: p \mapsto (\Pi(p))^n
\end{equation}
is transverse to the zero section. Then, a pair $(M,\Pi)$ is called a \textbf{$b$-Poisson manifold} and the vanishing set $Z$ of $F$ is called the \textbf{critical hypersurface}. {Observe that $Z$ is an embedded hypersurface}.



This class of Poisson structures was studied by Radko \cite{Radko02} in dimension two and considered in numerous papers in the last years: \cite{GMP10}, \cite{GMP14}, \cite{GMP15}, \cite{GMW17}, \cite{MO2} and  \cite{GUAL14}  among others.

\section{$b$-Poisson manifolds}

Next, we recall classification theorem of $b$-Poisson surfaces as presented by Olga Radko and  the cohomological re-statement and proof given by Guillemin, Miranda and Pires in \cite{GMP14}.

{ In what follows, $(M,\Pi)$  will be a closed smooth surface with a $b$-Poisson structure on it, and  $Z$  its critical hypersurface.}

{Let $h$ be the distance function to $Z$ as in \cite{MO2}}\footnote{{Notice the difference with \cite{Radko02} where $h$ is assumed to be a global defining function.}}.

\begin{definition}\label{lvol} The {\textbf{Liouville volume of $(M, \Pi)$}}  is the following limit: $ V(\Pi )\coloneqq \lim _{\epsilon \to 0}\int _{|h|>\epsilon }\omega^n $\footnote{{For surfaces $n = 1$.}}.
\end{definition}

{
The previous limit exists and it is independent of the choice of the defining function $h$ of $Z$ (see \cite{Radko02} for the proof).}

\begin{definition}
{For any $(M,{\Pi})$ oriented Poisson manifold, let $\Omega$} {be} a volume form on it, and let $u_f$ denote the Hamiltonian vector field of a smooth function $f:M\rightarrow\mathbb{R}$. The \textbf{modular vector field} {$X^{\Omega}$} is the derivation defined as follows:
	$$f\mapsto \frac{\mathcal{L}_{u_f}\Omega}{\Omega}.$$
\end{definition}	
	
	\begin{definition} Given $\gamma$ a connected component of {  the critical set $Z(\Pi )$ of a closed $b$-Poisson manifold $(M,\Pi)$},   the \textbf{modular period} of $\Pi$ around $\gamma$ is defined as:
	
	
	$$T_{\gamma}(\Pi )\coloneqq \textrm{period of }\, {X^{\Omega}}|_{\gamma }. $$
	
\end{definition}

{
\begin{remark}
The modular vector field $X^\Omega$ of the $b$-Poisson manifold $(M,Z)$ does not depend at $Z$ on the choice of $\Omega$ because for different choices for volume form the difference of modular vector fields  is a Hamiltonian vector field. Observe that this Hamiltonian vector field vanishes on the critical set as $\Pi$ vanishes there too.
\end{remark}
}

\begin{definition}
{
Let $\mathcal{M}_n(M)=\mathcal{C}_n(M)/\sim$ where $\mathcal{C}_n(M)$ is the space of disjoint oriented curves and $\sim$ identifies two sets of curves if there is an orientation-preserving diffeomorphism mapping the first one to the  second one and preserving the orientations of the curves.
}
\end{definition}


The following theorem classifies $b$-symplectic structures on surfaces using these invariants:
\begin{theorem}[\textbf{Radko} \cite{Radko02}]\label{Radko}
{Consider two $b$-Poisson structures $\Pi$, $\Pi'$ on {a closed} {orientable} surface $M$. Denote its critical {hypersurfaces} by $Z$ and $Z'$. These two $b$-Poisson structures are globally equivalent (there exists a global {orientation preserving} diffeomorphism sending $\Pi$ to $\Pi'$) if and only if the following coincide}:
	\begin{itemize}
	    \item the equivalence classes of $[Z]$ and $[Z']\in\mathcal{M}_n(M)$,
	    \item their modular periods around the connected components of $Z$ and $Z'$,
	    \item their  Liouville volume.
	\end{itemize}
\end{theorem}



An appropriate formalism to deal with these structures was introduced in \cite{GMP10}.

\begin{definition}
A \textbf{$b$-manifold}\footnote{The `$b$' of $b$-manifolds stands for `boundary', as initially considered by Melrose {(Chapter 2 of \cite{Melrose93})} for the study of pseudo-differential operators on manifolds with boundary.} is a pair $(M,Z)$ of a manifold and an {embedded} hypersurface.
\end{definition}

 In this way, the concept of $b$-manifold previously introduced by Melrose is generalized to consider additional geometric structures on the manifold.

\begin{definition}
A \textbf{$b$-vector field} on a $b$-manifold $(M,Z)$ is a vector field tangent to the hypersurface $Z$ at every point $p\in Z$.
\end{definition}

{
\begin{definition}
A \textbf{$b$-map} from $(M,Z)$ to $(M',Z')$ is a {smooth} map $\phi:M \rightarrow M'$ such that $\phi^{-1}(Z') = Z$ and $\phi$ is transverse to $Z'$.
\end{definition}
}

Observe that if $x$ is a local defining function for $Z$ {and $(x, x_1, \ldots, x_{n-1})$ are local coordinates in a neighborhood of $p \in Z$} then {the $C^\infty(M)$-module of $b$-vector fields has the following local basis}
\begin{equation}\label{eq:generatebvectors}
\{x \frac{\partial}{\partial{x}}, \frac{\partial}{\partial{x_1}},\ldots, \frac{\partial}{\partial{x_{n-1}}}\}.
\end{equation}

In contrast to \cite{GMP10}, in this monograph we are not requiring the existence of a global defining function for $Z$ and orientability of $M$. However,  we require the existence of a defining  function in a neighborhood of each {point} of $Z$. {By relaxing this condition, the normal bundle of $Z$  need not  be trivial.}


{Given $(M,Z)$ a $b$-manifold, \cite{GMP10} shows that there exists a vector bundle,
denoted by $^b TM$  whose smooth sections are $b$-vector fields. This bundle
is called the \textbf{$b$-tangent bundle} of $(M,Z)$.}

The \textbf{$b$-cotangent bundle} $^b T^*M$ is defined using duality. A \textbf{$b$-form} is a  section of the $b$-cotangent bundle. {Around a point $p\in Z$ the $C^\infty(M)$-module of {these} sections has the following local basis:}
\begin{equation}\label{eq:generatebforms}
\{\frac{1}{x} dx, d x_1,\ldots, d x_{n-1}\}.
\end{equation}
In the same way we define a \textbf{$b$-form of degree $k$} to be a section of the bundle $\bigwedge^k(^b T^*M)$, the set of these forms is denoted $^b\Omega^k(M)$. Denoting by $f$ the distance function\footnote{Originally in \cite{GMP10} $f$ stands for a global function, but for non-orientable manifolds we may  use the distance function instead.} to the {critical hypersurface} $Z$, we may write the following decomposition as in \cite{GMP10} {for any $\omega \in ^b\Omega^k(M)$} :

\begin{equation}\label{eq:decomposition}
\omega=\alpha\wedge\frac{df}{f}+\beta, \text{ with } \alpha\in\Omega^{k-1}(M) \text{ and } \beta\in\Omega^k(M).
\end{equation}

This decomposition allows to extend the differential of the {de} Rham complex $d$ to $^b\Omega(M)$ by setting $d\omega=d\alpha\wedge\frac{df}{f}+d\beta.$ 

Degree $0$ functions are called $b$-functions and and near $Z$ can be written as
$$ c \log |x| + g, $$
where $c\in \R, g\in C^\infty,$ and $x$ is a local defining function.




\begin{figure}
\centering

\begin{tikzpicture}

\pgfmathsetmacro{\ellbasex}{7}
\pgfmathsetmacro{\ellbasey}{3.5}

\pgfmathsetmacro{\majoraxis}{2}
\pgfmathsetmacro{\minoraxis}{.5}

\pgfmathsetmacro{\rlineymid}{4.25}
\def\R{1.6}
\pgfmathsetmacro{\circlex}{1.5}


\draw[dashed, very thick, color = red] (\circlex + \R, \rlineymid) arc (0:180:{\R} and {\R * .2});
\draw[very thick, fill = magenta, opacity = .6] (\circlex, \rlineymid) circle (\R);
\draw[very thick] (\circlex, \rlineymid) circle (\R);
\draw[rotate around={-45:(\circlex, \rlineymid - .3)},dblue, fill = dblue] (\circlex, \rlineymid - .3) ellipse (.4 and .2);
\draw[very thick, color = red] (\circlex + \R, \rlineymid) arc (0:-180:{\R} and {\R * .2});

\node (circaround) at (\circlex, \rlineymid - .3) [circle,draw=none,thick, minimum size=1cm] {};
												
\draw[ultra thick, ->] (circaround)  to [bend right = 25] (\ellbasex - 1.2*\majoraxis, \ellbasey - .3);


\draw[very thick, fill = dblue] (\ellbasex - \majoraxis, \ellbasey) arc (-180:180:{\majoraxis} and {\minoraxis});
\node (baseone) at (\ellbasex - \majoraxis * .27, \ellbasey - \minoraxis * 1.25) {};
\node (basetwo) at (\ellbasex + \majoraxis * .27, \ellbasey + \minoraxis * 1.25) {};
\draw[very thick, red] (baseone) -- (basetwo);

\draw[draw = none, fill = orange] (\ellbasex - \majoraxis, \ellbasey + 1) arc (180:230:{\majoraxis} and {\minoraxis}) arc (-90:-47.4:.75) node (cylh) {} -- ++(0.95, 0.875) arc(-47.4:-90:.75) arc(100:180:{\majoraxis} and {\minoraxis}) -- cycle;
\draw[draw = none, fill = purple] (cylh)  arc (-47.4:0:.75) -- ++(0.95, 0.875) arc(0:-47.4:.75) -- cycle;

\draw[very thick] (\ellbasex - \majoraxis, \ellbasey + 1) arc (180:230:{\majoraxis} and {\minoraxis}) arc (-90:0:.75);
\draw[very thick] (\ellbasex - \majoraxis, \ellbasey + 1) arc (180:100:{\majoraxis} and {\minoraxis}) arc (-90:0:.75);

\draw[draw = none, fill = green] (\ellbasex + \majoraxis, \ellbasey + 1.1) arc (0:50:{\majoraxis} and {\minoraxis}) arc (270:180:.75) -- ++(-0.95, -0.875) arc(180:270:.75) arc (-80:0:{\majoraxis} and {\minoraxis}) -- cycle;

\draw[very thick] (\ellbasex +\majoraxis, \ellbasey + 1.1) arc (0:50:{\majoraxis} and {\minoraxis}) arc (270:180:.75);
\draw[very thick] (\ellbasex +\majoraxis, \ellbasey + 1.1) arc (0:-80:{\majoraxis} and {\minoraxis}) arc (270:180:.75);

\end{tikzpicture}
\caption{Artistic representation of a $b$-function on a $b$-manifold  near the critical hypersurface.} \label{fig:L1}
\end{figure}





The associated cohomology is called \textbf{$b$-cohomology} and it is denoted by \textbf{$^b H^*(M)$}.

\begin{definition}
A \textbf{$b$-symplectic} form on a $b$-manifold $(M^{2n},Z)$ is defined as a non-degenerate closed $b$-form of degree $2$ (i.e., $\omega_p$ is of maximal rank as an element of $\Lambda^2(\,^b T_p^* M)$ for all $p\in M$).
\end{definition}

The notion {of} $b$-symplectic forms is dual to {the} notion of $b$-Poisson structures. The advantage of using forms rather than bivector fields is that symplectic tools can be `easily' exported.

Radko's classification theorem \cite{Radko02} can be translated  into this language. {This translation was already formulated in} \cite{GMP10}:

 \begin{theorem}[\textbf{Radko's theorem in $b$-cohomological language, \cite{GMP14}}] Let $S$ be a {closed} orientable surface and  let  $\omega_0$ and $\omega_1$ be two $b$-symplectic forms on {$(S,Z)$}  defining the same $b$-cohomology class (i.e.,$[\omega_0]= [\omega_1]$).  Then there exists a diffeomorphism {$\phi:S\rightarrow S$} such that $\phi^*\omega_1 = \omega_0$.
\end{theorem}

\section{On $b^m$-Symplectic manifolds}
\subsection{Basic definitions}
By relaxing the transversality condition allowing higher order singularities (\cite{ARNOLD85} and \cite{ARNOLD75}) we may consider other symplectic structures with singularities as done by
Scott \cite{Scott16} with $b^m$-symplectic structures.
{
 Let $m$ be a positive integer a \textbf{$b^m$-manifold} is a \textbf{$b$-manifold} $(M,Z)$ together with
 a $b^m$-tangent bundle attached to it. The $b^m$-tangent bundle is (by Serre-Swan theorem \cite{SWAN}) a vector bundle, $^{b^m} TM $ whose sections are given by, $$\Gamma(^{b^m}TM)=\{v \in \Gamma(TM): v(x) \quad \text{is tangent to order } m \text{ to } Z \},$$
where $x$ is a defining function for the critical set $Z$  in a neighborhood of each connected component of $Z$ and can be defined as
$x:M\setminus Z \rightarrow (0,\infty), x \in C^\infty(M)$ such that:
\begin{itemize}
\item $x(p) = d(p)$ a distance function from $p$ to $Z$ for $p: d(p) \leq 1/2$
\item $x(p) = 1$ on $M\setminus\{p\in M \text{ such that }d(p) < 1\}$.\footnote{{Then a $b^m$-manifold will be a triple $(M, Z, x)$, but for the sake of simplicity we refer to it as a pair $(M,Z)$ and we tacitly assume that the function $x$ is fixed.}}
\end{itemize}
(This definition of $x$ allows us to extend the construction in  \cite{Scott16}  to the non-orientable case as in \cite{MO2}.)
 We may define the notion of a $b^m$-map as a map in this category (see \cite{Scott16}).



 
The sections of this bundle are referred to as \textbf{$b^m$-vector fields} and their flows define $b^m$-maps.
In local coordinates, the sections of the $b^m$-tangent bundle  are generated by:
\begin{equation}\label{eq:generatebmvectors}
\{x^m \frac{\partial}{\partial{x}}, \frac{\partial}{\partial{x_1}},\ldots, \frac{\partial}{\partial{x_{n-1}}}\}.
\end{equation}
\noindent
}



Proceeding \emph{mutatis mutandis} as in the $b$-case one defines  the  $b^m$-cotangent bundle ($^{b^m} T^*M $), the $b^m$-{de} Rham complex and the $b^m$-symplectic structures.




A {\bf Laurent Series} of a closed $b^m$-form $\omega$ is a decomposition of $\omega$ in a tubular neighborhood $U$ of $Z$
of the form
\begin{equation}\label{eqn:laurent0}
\omega = \frac{dx}{x^m} \wedge (\sum_{i = 0}^{m-1}\pi^*(\alpha_{i})x^i) + \beta
\end{equation}

\noindent with $\pi: U \to Z$ the projection of the tubular neighborhood onto $Z$, $\alpha_{i}$ a closed smooth {de} Rham form on $Z$ and $\beta$  a {de} Rham form on $M$.

In \cite{Scott16} it is proved that in  a  neighborhood of $Z$, every   closed $b^m$-form  $\omega$ can be written in a  Laurent form of type (\ref{eqn:laurent0}) having fixed a (semi)local defining function.

 $b^m$-Cohomology is related to de Rham cohomology via the following theorem:
\begin{theorem}[\textbf{$b^m$-Mazzeo-Melrose}, \cite{Scott16}]\label{thm:Mazzeo-Melrose}
{Let $(M,Z)$ be a $b^m$-manifold, then:}
\begin{equation}\label{eqn:Mazzeo-Melrose}
 ^{b^m}H^p(M) \cong H^p(M)\oplus(H^{p-1}(Z))^m.
\end{equation}
\end{theorem}

{The isomorphism constructed in the proof of the theorem above is non-canonical (see \cite{Scott16}).}

The Moser path method can be generalized to $b^m$-symplectic structures (see \cite{evageoff} for the generalization from surfaces in \cite{Scott16} to general manifolds):

{\begin{theorem}[\textbf{Moser path method}]\label{mpm} Let $\omega_t$ be a path of $b^m$-symplectic forms defining the same $b^m$-cohomology class  $[\omega_t]$ on $(M^{2n}, Z)$ with $M^{2n}$ closed and orientable  then there exist  a $b^m$-symplectomorphism $\varphi: (M^{2n}, Z)\longrightarrow (M^{2n}, Z)$ such that $\varphi^*(\omega_1) = \omega_0$. \end{theorem}}



An outstanding consequence of Moser path method is a global classification of {closed} orientable $b^m$-symplectic surfaces \`{a} la Radko in terms of $b^m$-cohomology classes.

\begin{theorem}[\textbf{Classification of {closed} orientable $b^m$-{surfaces}, \cite{Scott16}}]\label{thm:scott}
Let $\omega_0$ and $\omega_1$ be two $b^m$-symplectic forms on a {closed orientable} connected $b^m$-surface $(S,Z)$.
Then, the following conditions are equivalent:
\begin{itemize}

  \item their $b^m$-cohomology classes coincide $[\omega_0] = [\omega_1]$,
  \item the surfaces are globally $b^m$-symplectomorphic,
  \item the Liouville volumes of $\omega_0$ and $\omega_1$ and the numbers $$\int_{\gamma} \alpha_{i}$$ for all connected
  components $\gamma \subseteq Z$ and all $1 \leq i \leq m$ coincide (where
  $\alpha_{i}$ are the one-forms appearing in the Laurent decomposition
  of the two $b^m$-forms {of degree 2}, $\omega_0$ and $\omega_1$).
\end{itemize}
\end{theorem}

\begin{definition}
The numbers $[\alpha_i] = \int_\gamma \alpha_i$ are called \emph{modular weights} for the connected components $\gamma \subset Z$.
\end{definition}

{A relative version of Moser's path method is proved in \cite{GMW17}. As a corollary we obtain the following local description of a $b^m$-symplectic manifold:}

\begin{theorem}[\textbf{$b^m$-Darboux theorem, {\cite{GMW17}}}]\label{theorem:Darbouxbn}
Let $\omega$ be a $b^m$-symplectic form on $(M,Z)$ and $p\in Z$. Then we can find a coordinate chart $(U,x_1,y_1,\ldots,x_n,y_n)$ centered at
$p$ such that on $U$ the hypersurface $Z$ is locally defined by $x_1=0$ and
$$\omega=\frac{d x_1}{x_1^m}\wedge {d y_1}+\sum_{i=2}^n d x_i\wedge d y_i.$$
\end{theorem}



{
\begin{remark}
 For the sake of simplicity sometimes we will omit any explicit reference to the critical set  $Z$ and we will talk directly about $b^m$-symplectic structures on manifolds $M$ implicitly assuming that $Z$ is the vanishing locus of  $\Pi^n$  where $\Pi$ is the Poisson vector field dual to the $b^m$-symplectic form.
\end{remark}
}


Next, we present two lemmas that allow us to talk about $b^m$-symplectic structures and $b^m$-Poisson as two different presentations of the same geometrical structure on a $b$-manifold. The lemma below shows that they are dual to each other and, thus, in one-to-one correspondence.


\begin{lemma}
Let $\omega$ be a $b^m$-symplectic and $\Pi$ its dual vector field, then $\Pi$ is a $b^m$-Poisson structure.
\end{lemma}

\begin{proof}
The quickest way to do this is to take the inverse, which is a bivector field, and observe that it is a Poisson structure (because $d\omega=0$ implies $[\Pi, \Pi]=0$). To see that it is $b^m$-Poisson
it is enough to check it locally for any point along the critical set. Take a point $p$ on the critical set $Z$ and apply the $b^m$-Darboux theorem  to get $\omega= dx_1/x_1^m\wedge dy_1+ \sum_{i>1} dx_i\wedge dy_i$
This means that in the new coordinate system $$\Pi=x_1^m\frac{\partial}{\partial x_1}\wedge \frac{\partial}{\partial y_1}+\sum_{i>1} \frac{\partial}{\partial x_i}\wedge \frac{\partial}{\partial y_i}$$
and thus $\Pi$ is a $b^m$-Poisson structure.

\end{proof}


Conversely,
\begin{lemma}
Let $\Pi$ be $b^m$-Poisson and $\omega$ its dual vector field, then $\omega$ is a $b^m$-symplectic structure.
\end{lemma}


\begin{proof}
 If $\Pi$ transverse à la Thom on $Z$ with singularity of order m then because of Weinstein's splitting theorem we can locally write

$$\Pi=x_1^m\frac{\partial}{\partial x_1}\wedge \frac{\partial}{\partial y_1}+\sum_{i>1} \frac{\partial}{\partial x_i}\wedge \frac{\partial}{\partial y_i}$$

now its inverse is $\omega= dx_1/x_1^m\wedge dy_1+ \sum_{i>1} dx_i\wedge dy_i$  which is a $b^m$-symplectic form.
\end{proof}

Hence we have a correspondence from $b^m$-symplectic structures to $b^m$-Poisson structures.



\section{{Desingularizing $b^m$-Poisson manifolds}}\label{sec:deblogging}

{In \cite{GMW17} Guillemin, Miranda and Weitsman presented a desingularization procedure for $b^m$-symplectic manifolds proving that we may associate a family of folded symplectic or symplectic forms to a given $b^m$-symplectic structure depending on the parity of $m$. Namely,}

\begin{theorem}[\textbf{Guillemin-Miranda-Weitsman}, \cite{GMW17}]\label{thm:deblogging}
 {Let $\omega$ be a $b^m$-symplectic structure  on a {closed orientable manifold} $M$  and let $Z$ be its critical hypersurface.
\begin{itemize}
\item If $m=2k$, there exists  a family of symplectic forms ${\omega_{\epsilon}}$ which coincide with  the $b^{m}$-symplectic form
    $\omega$ outside an $\epsilon$-neighborhood of $Z$ and for which  the family of bivector fields $(\omega_{\epsilon})^{-1}$ converges in
    the $C^{2k-1}$-topology to the Poisson structure $\omega^{-1}$ as $\epsilon\to 0$ .
\item If $m=2k+1$, there exists  a family of folded symplectic forms ${\omega_{\epsilon}}$ which coincide with  the $b^{m}$-symplectic form
    $\omega$ outside an $\epsilon$-neighborhood of $Z$.
\end{itemize}}

\end{theorem}


{As a consequence of Theorem \ref{thm:deblogging}, any closed orientable manifold that
supports a $b^{2k}$-symplectic structure necessarily supports a symplectic structure.}


{ In \cite{GMW17} explicit formulae are given for even and odd cases. Let us refer here to the even-dimensional case as these formulae will be used later on.}



Let us  briefly recall how the desingularization is defined and  the main result in \cite{GMW17}. Recall that we can express the $b^{2k}$-form as:

\begin{equation}\label{eq:laurent2}
\omega = \frac{dx}{x^{2k}}\wedge \left(\sum_{i=0}^{2k-1}x^i\alpha_i\right) + \beta.
\end{equation}


This expression holds on a $\epsilon$-tubular neighborhood of a given connected component of $Z$. This expression comes directly from equation \ref{eqn:laurent0}, to see a proof of this result we refer to \cite{Scott16}.

\begin{definition}\label{dfn:deblogging} {Let $(S,Z,x)$, be a $b^{2k}$-manifold, where $S$ is a closed orientable manifold and let $\omega$ be a $b^{2k}$-symplectic form. Consider  the decomposition given by the expression (\ref{eq:laurent2}) on an $\epsilon$-tubular neighborhood $U_\epsilon$ of a connected component of $Z$.}


	Let $f \in \mathcal{C}^\infty(\mathbb{R})$ be an odd smooth function satisfying $f'(x) > 0$ for all $x \in \left[-1,1\right] $ and satisfying outside that
\begin{equation}
f(x) = \begin{cases}
\frac{-1}{(2k-1)x^{2k-1}}-2& \text{for} \quad x < -1,\\
\frac{-1}{(2k-1)x^{2k-1}}+2& \text{for} \quad x > 1.\\
\end{cases}
\end{equation}
	Let $f_\epsilon(x)$ be defined as $\epsilon^{-(2k -1)}f(x/\epsilon)$.



The \textbf{$f_\epsilon$-desingularization} $\omega_\epsilon$ is {a form that is defined on $U_\epsilon$ by the following expression:}
	$$\omega_\epsilon = df_\epsilon \wedge \left(\sum_{i=0}^{2k-1}x^i\alpha_i\right) + \beta.
	$$
%\begin{figure}[h!]
%	\centering
%	\includegraphics[scale=.7]{Deblogging.eps}
%	\caption{Smooth and odd extension of $f$ inside the interval {$[-1,1]$ such that $f' > 0$}.}
%	\label{fig:deblogging}
%\end{figure}
\end{definition}

{This desingularization procedure is also known as \textbf{deblogging} in the literature.}

{
\begin{remark}
Though there are infinitely many choices for $f$, we will assume that we choose one, and assume it fixed through the rest of the discussion. It would be interesting to discuss the existence of an  isotopy of forms under a change of function $f$.
\end{remark}
}

{
\begin{remark}
Because $\omega_\epsilon$ can be trivially extended  to the whole $S$ in such a way that it agrees with $\omega$ (see  \cite{GMW17}) outside a neighborhood of $Z$, we can  talk about the $f_\epsilon$-desingularization of $\omega$  as a form on $S$.
\end{remark}
}

\chapter{A crash course on KAM theory}

The last part of this monograph is entirely dedicated to prove a KAM theorem for $b^m$-symplectic structures and to find applications. So the aim of this section is to give a quick overview of the traditional KAM theorem. The setting of the KAM theorem is a symplectic manifold with action-angle coordinates and an integrable system in it. The theorem says that under small perturbations of the Hamiltonian "most" of the Liouville tori survive.

Consider $\mathbb{T}^n\times G \subset \mathbb{T}^n \times \mathbb{R}^n$ with action-angle coordinates in it $(\phi_1,\ldots,\phi_n,I_1,\ldots,I_n)$ and the standard symplectic form $\omega$ in it. And assume the Hamiltonian function of the system is given by $h(I)$ a function only depending on the action coordinates. Then the Hamilton equations of the system are given by

$$\iota_{X_h} \omega = dh$$

where $X_h$ is the vector field generating the trajectories. Because $h$ does not depend on $\phi$ the angular variables the system is really easy to solve, and the equations are given by

$$x(t) = (\phi(t), I(t)) = (\phi_0 + ut, I_0),$$

where $u = \partial h/\partial I$ is called the frequency vector. These motions for a fixed initial condition are inside a Liouville torus, and are called quasi-periodic.

The KAM theorem studies what happens to such systems when a small perturbation is applied to the Hamiltonian function, i.e. we consider the evolution of the system given by the Hamiltonian $h(I) + R(I,\phi)$, where we think of the term $R(I,\phi)$ as the small perturbation in the system. With this in mind, the Hamilton equations can be written as

$$
\begin{array}{l}
\dot{\phi} = u(I) + \frac{\partial}{\partial I}R(I,\phi),
\dot{I} = -\frac{\partial}{\partial \phi}R(I,\phi),
\end{array}
$$


Another important concept to have in mind is the concept of \emph{rational dependency}. A frequency $u$ is rationally dependent if $\langle u , k \rangle= 0$ for some $k \in \mathbb{Z}^n$, if there exists no $k$ satisfying the condition then the vector $u$ is called rationally independent. There is a stronger concept of being rationally independent and that is the concept of being Diophantine. A vector $u$ is $\gamma$,$\tau$-diophantine if $\langle u, k\rangle \geq \frac{\gamma}{|k|_1^\tau}$ for all $k\in \mathbb{Z}^n\setminus\{0\}$. $\gamma > 0$ and $\tau > n-1$.

The KAM theorem states that the Liouville tori with frequency vector satisfying the diophantine condition survive under the small perturbation $R(I,\phi)$. There are conditions relating the size of the perturbation with $\gamma$ and $\tau$. Also, the set of tori satisfying the Diophantine condition has measure $1 - C\gamma$ for some constant $C$.

Now we give a proper statement of the theorem as was given in \cite{D}.

\begin{theorem}[Isoenergetic KAM theorem]\label{theorem:standard_KAM}
Let $\mathcal{G} \subset \mathbb{R}^n$, $n > 2$, a compact, and let $H(\phi,I) = h(I) + f(\phi,I)$ real analytic on $\mathcal{D}_\rho(\mathcal{G})$. Let $\omega = \partial h/\partial I$, and assume the bounds:

$$\left|\frac{\partial^2 h}{\partial I^2}\right|_{\mathcal{G}, \rho_2} \leq M, \quad |\omega|_{\mathcal{G}} \leq L \quad \text{and} \quad |\omega_n(I)| \geq l \forall I \in \mathcal{G}.$$

Assume also that $\omega$ is $\mu$-isoenergetically non-degenerate on $\mathcal{G}$. For $a = 16M/l^2$, assume that the map $\Omega = \Omega_{\omega,h,a}$ is one-to-one on $\mathcal{G}$, and that its range $F = \Omega(\mathcal{G})$ is a $D$-set. Let $\tau > n-1$, $\gamma > 0$ and $0 < \nu < 1$ given, and assume:

$$\varepsilon := \|f\|_{\mathcal{G},\rho} \leq \frac{\nu^2 l^6 \mu^2 \hat \rho^{2\tau+2}}{2^{4\tau+32}L^6 M^3}\cdot \gamma^2, \quad \gamma \leq \min\left(\frac{8L M\rho_2}{\nu l \hat \rho^{\tau+1}}, l\right),$$

where we write $\rho := \min \left(\frac{\nu \rho_1}{12(\tau+2)},1\right)$. Define the set

$$\hat G = \hat G_\gamma := \left\{I \in \mathcal{G} - \frac{2\gamma}{\mu} : \omega(I) \text{is} \tau,\gamma-\text{Diophantine} \right\}.$$

Then, there exists a real continuous map $\mathcal{T}:\mathcal{W}_{\frac{\rho_1}{4}}(\mathbb{T}^n)\times \hat G \rightarrow \mathcal{D}_\rho(\mathcal{G})$, analytic with respect to the angular variables, such that:

\begin{enumerate}
\item For every $I \in \hat G$, the set $\mathcal{T}(\mathbb{T}^n\times\{I\})$ is an invariant torus of $H$, its frequency vector is colinear to $\omega(I)$ and its energy is $h(I)$.
\item Writing
$$\mathcal{T}(\phi,I) = (\phi + \mathcal{T}_\phi(\phi,I), I + \mathcal{T}_I(\phi,I)),$$
one has the estimates
$$|\mathcal{T}_\phi|_{\hat G, (\frac{\rho_1}{4},0), \infty} \leq \frac{2^{2\tau+15} L^2 M}{\nu^2 l^2 \hat \rho^{2\tau+1}}\frac{\varepsilon}{\gamma^2}, \quad |\mathcal{T}_I|_{\hat G, (\frac{\rho_1}{4},0)} \leq \frac{2^{\tau+16} L^3 M}{\nu l^3 \mu \hat \rho^{\tau+1}}\frac{\varepsilon}{\gamma}$$
\item $\text{meas}[(\mathbb{T}^n\times \mathcal{G}) \setminus \mathcal{T}(\mathbb{T}^n\times \hat G)] \leq C\gamma$, where $C$ is a very complicated constant depending on $n$, $\tau$, $\text{diam} F$, $D$, $\hat{\rho}$, $M$,  $L$,  $l$,  $\mu$.

\end{enumerate}

\end{theorem}

\begin{remark}
This version of the KAM theorem is the isoenergetic one, this version ensures that the energy of the Liouville Tori identified by the diffeomorphism after the perturbation remains the same as before the perturbation. Our version of the $b^m$-KAM is not isoenergetic for the sake of simplifying the computations.
\end{remark}

Also, we should outline that the KAM theorem has already been explored in singular symplectic manifolds before. In \cite{KMS16} the authors proved a KAM theorem for $b$-symplectic manifolds, for a particular kind of perturbations.

\begin{theorem}[KAM Theorem for $b$-Poisson manifolds]

Let $\T^n \times B_r^n$ be endowed with standard coordinates $(\varphi,y)$ and the $b$-symplectic structure. Consider a $b$-function
$$H = k \log|y_1| + h(y)$$
 on this manifold, where $h$ is analytic. Let $y_0$ be a point in $B_r^n$ with first component equal to zero, so that the corresponding level set $\T^n \times \{y_0\}$ lies inside the critical hypersurface $Z$.

Assume that the frequency map
$$\tilde \omega: B^n_r \to \R^{n-1}, \quad \tilde \omega( y):= \frac{\partial h}{\partial \tilde y}(y)$$
 has a Diophantine value $\tilde \omega := \tilde \omega(y_0)$ at $y_0 \in B^n$ and that it is non-degenerate at $y_0$ in the sense that the Jacobian $ \frac{\partial \tilde \omega}{\partial \tilde y} (y_0) $ is regular.

Then the torus $\T^n \times \{ {y_0}\}$  persists under sufficiently small perturbations of $H$ which have the form mentioned above, i.e. they are given by $\epsilon P$, where $\epsilon \in \R$ and $P \in ^b \! C^\infty(\T^n \times B_r^n)$ has the form
\begin{align*}
 P(\varphi, y) &= k' \log |y_1| + f(\varphi,y) \\
 f(\varphi, y) &= f_1(\tilde \varphi, y ) + y_1 f_2(\varphi, y) + f_3(\varphi_1,y_1).
\end{align*}
More precisely, if $|\epsilon|$ is sufficiently small, then the  perturbed system
  $$ H_\epsilon =H + \epsilon P$$
admits an invariant torus $\mathcal{T}$.% close to $\T^n \times \{ {y_0}\}$.

Moreover, there exists a diffeomorphism $\T^n \to \mathcal{T}$ close\footnote{By saying that the diffeomorphism is ``$\epsilon$-close to the identity'' we mean that, for given $H, P$ and $r$, there is a constant $C$ such that $\|\psi - \id\| < C \epsilon.$} to the identity taking the flow $\gamma^t$  of the perturbed system on $\mathcal{T}$ to the linear flow  on $\T^n$ with frequency vector
$$ \left(\frac{k+\epsilon k'}{c}, \tilde \omega \right).$$
\end{theorem}


