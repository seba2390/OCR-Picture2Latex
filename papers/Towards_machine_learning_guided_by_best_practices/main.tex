\documentclass[10pt,conference]{IEEEtran}
\IEEEoverridecommandlockouts
% The preceding line is only needed to identify funding in the first footnote. If that is unneeded, please comment it out.
\usepackage{cite}
\usepackage{amsmath,amssymb,amsfonts}
\usepackage{algorithmic}
\usepackage{graphicx}
\usepackage{textcomp}
\usepackage{xcolor}
\def\BibTeX{{\rm B\kern-.05em{\sc i\kern-.025em b}\kern-.08em
    T\kern-.1667em\lower.7ex\hbox{E}\kern-.125emX}}

% OLD PREAMBLE:

% \usepackage{jsen}
% \usepackage{cite}
% \usepackage{amsmath,amssymb,amsfonts, bbm, mathtools}
% \usepackage{algorithm,algorithmic}
% \usepackage{graphicx}
% \usepackage{textcomp}
% \usepackage{wrapfig}
% \usepackage{xfrac}
% \usepackage{stackengine}
% \usepackage{subfigure}
% \def\delequal{\mathrel{\ensurestackMath{\stackon[1pt]{=}{\scriptstyle\Delta}}}}



% \usepackage{color, soul}
% \newcommand{\hlt}[1]{\hl{#1}}
% \newcommand{\red}[1]{\textcolor{red}{#1}}

% \def\BibTeX{{\rm B\kern-.05em{\sc i\kern-.025em b}\kern-.08em
%     T\kern-.1667em\lower.7ex\hbox{E}\kern-.125emX}}
% \markboth{\journalname, VOL. XX, NO. XX, XXXX 2017}
% {Author \MakeLowercase{\textit{et al.}}: Preparation of Papers for IEEE TRANSACTIONS and JOURNALS (February 2017)}
% \definecolor{abstractbg}{rgb}{0.89804,0.94510,0.83137}
% \setlength{\fboxrule}{0pt}
% \setlength{\fboxsep}{0pt}

% NEW PREAMBLE:


\usepackage{amsmath,amsfonts,amssymb,bbm, amsthm, xfrac}
\usepackage{algorithmic}
\usepackage{algorithm}
\usepackage{array, multirow}
% \usepackage[caption=false,font=normalsize,labelfont=sf,textfont=sf]{subfig}
\usepackage{caption, subcaption}
\usepackage{textcomp}
\usepackage{stfloats}
\usepackage{url}
\usepackage{verbatim}
\usepackage{graphicx}
\usepackage{cite}
\usepackage{caption}
\usepackage{subcaption}
\hyphenation{}

\theoremstyle{plain}
\newtheorem{theorem}{Theorem}

\usepackage{color, soul}
\newcommand{\hlt}[1]{\hl{#1}}
\newcommand{\red}[1]{\textcolor{red}{#1}}

\begin{document}

\title{Towards machine learning guided by best practices}

\author{\IEEEauthorblockN{Anamaria Mojica-Hanke}
\IEEEauthorblockA{\textit{University of Passau} \\
Passau, Germany}
\IEEEauthorblockA{\textit{Universidad de los Andes} \\
	Bogota, Colombia\\
	ai.mojica10@uniandes.edu.co}

}

\maketitle

\begin{abstract}


Nowadays, machine learning (ML) is being used in software systems with multiple application fields, from medicine to software engineering (SE). On the one hand, the popularity of ML in the industry can be seen in the statistics showing its growth and adoption.  On the other hand, its popularity can also be seen in research, particularly  in SE, where multiple studies related to the use of Machine Learning in Software Engineering have been published in conferences and journals.  At the same time, researchers and practitioners have shown that machine learning has some particular challenges and pitfalls. In particular, research has shown that ML-enabled systems  have a different development process than traditional software, which also describes  some of the challenges of ML applications.  In order to mitigate some of the identified challenges and pitfalls, white and gray literature has proposed a set of recommendations based on their own experiences  and focused on their domain (\eg biomechanics), but for the best of our knowledge, there is no guideline focused on the SE community. %This thesis aims to reduce this gap by answering research questions that help to understand the practices used and discussed by practitioners and researchers in the SE community. As a result we will present a set of best practices with a focus on SE. 
This thesis aims to reduce the gap of not having clear guidelines in the SE community by using  possible sources of practices such as question-and-answer communities and also previous research studies. As a result, we will  present a set of practices with an SE perspective, for researchers and practitioners, including a tool for searching them. 

\end{abstract}

\begin{IEEEkeywords}
Machine learning, good practices, software engineering
\end{IEEEkeywords}

\section{Motivations for Empirical Study}
\label{sec:motivations}
The key question that we try to answer is when and why we should use standard
iteration space tiling over cache oblivious tiling.  The two approaches
perform similar partitioning of the iteration space, but the schedules given
to the partitions are different.  Theoretically, cache oblivious code seems to
have advantages over iteration space tiling.  However, many factors complicate
the actual performance, which made our initial experiments difficult to
interpret.  In this section, we describe the obstacles between the theory and
practice we have identified.

We use Single-Level Tiling (SLT) for iteration space tiling, and Cache
Oblivious Tiling (COT) for cache oblivious techniques in this
paper, which are further described in Section~\ref{sec:background}.

\paragraph{Recursion Overhead} This is a well-known overhead of
COT~\cite{yotov2007experimental}.  The recursion introduces overheads, such as
function call overhead, and increased register pressure.  Furthemore, the
functions force inter-procedural analysis/optimization, known to be more
difficult for compilers well.  Thus, the leaf tiles must be ``sufficiently
large'' to avoid excessive overhead due to the recursion.

 \paragraph{Recursive Split Constraints the Tile Sizes} In typical cache
 oblivious algorithms, the problem is recursively split into halves in each
 dimension. This is in fact a rather coarse-grained exploration of the
 hierarchical partitioning of the iteration space. For instance, if the
 current problem size is $B^3$, then the next sub-problem would be
 $(\frac{B}{2})^3$.  If the best problem size for utilizing a level of cache
 is $(B-x)^3$ where $x\ll \frac{B}{2}$ then the subproblems due to
 divide-and-conquer will not match the best.  This is another factor that
 necessitates fine tuning of leaf tile sizes even for COT, since the utilization
 rate of L1 cache has strong impact on performance.  

%\paragraph{COT Leads to Imbalanced Tiles} Current COT tools recursively split
%the problem into halves in each dimension.  If the original bounds are not
%powers of two, every power-of-two leaf will be paired with a non-power-of-two
%leaf.  Since leaf tile sizes are often carefully tuned, thismeans that half
%the leaves will be suboptimal.  Our code generator incorporates a simple
%optimization that ensures that such suboptimal leaf nodes only occur at the
%boundaries of the iteration space.

\paragraph{COT has more Conflict Misses} The divide-and-conquer execution
order may negatively affect cache interference, especially with high
dimensional data.  This happens when the memory is allocated such that the
accesses are contiguous along some direction in the iteration space (typically
along innermost canonical axis).  With lexicographic order of execution, this
contiguity is largely preserved in the tiled execution.  However,
divide-and-conquer executes neighboring tiles in all dimensions, and many of
those tiles access some distant location in memory.  In contrast to accessing
contiguous regions of memory, accessing various segments of the memory
increases the chances of conflicts.

\paragraph{Hardware Prefetching}  Modern architectures are equipped with
hardware prefetchers that can bring data to the L1 cache. When
having sufficient locality at L2 or LLC makes the program compute-bound, then
the latency to L2/LLC can be hidden by the prefetcher. For such programs, it is
unnecessary to tile for the fastest cache, and larger tiles targeting slower
caches improve performance by maximizing prefetcher
effectiveness~\cite{mehta2016turbotiling}. When the primary objective is speed,
the leaf tiles for COT should also be large, which negates the benefit of
divide-and-conquer, as the leafs are already targeting slower caches.
Prefetching have little impact on parallel executions, since prefetching is
bandwidth limited. When multiple cores try to prefetch at the same time,
the bandwidth limit is quickly reached, and the latency hiding effect is
lost. Furthermore, smaller tile sizes are better for parallel execution for
load balancing  reasons.


These factors limit the effectiveness of COT in various ways and are also
closely tied to the characteristics of the computation. Our empirical study
illustrate the impact of these factors on polyhedral computations.

% Local Variables: ***
% TeX-master: "TACO2017.tex" ***
% fill-column: 78 ***
% End: ***

The industry standard for pose edition is to create rigs, a collection of pieces of software designed to manipulate a character's skeleton. The rig describes the skeleton's bones, how they relate to each other, are constrained in their possible motion and are deformed. These rules are loosely specified and creating a good rig requires a detailed understanding of physics and anatomy, as well as technical and artistic skills. Rigging is thus a time consuming task even for experienced animators, and even more so in large scale productions which often require a different in-depth rig for each character in the cast.
Previous work has helped alleviate this difficulty by providing efficient tools to speed up/and or ease the rigging process, relying on inverse kinematics or data-driven methods.
\subsection{Character pose design}
\subsubsection{Inverse Kinematics (IK)}
IK solvers are a family of methods commonly used in robotics, engineering and computer graphics, in which the parameterization of a kinematic chain is determined from the position of its end effector.
They are a staple tool in pose design software, ensuring the respect of elementary constraints during pose edition. Their de-facto role is to guarantee the length of the limbs, and in some cases to enforce the orientation angle range of a joint.
Many IK solutions have been studied over the years \cite{aristidou_inverse_2018}; usually revolving around approximated linearizations or heuristics. 

Numerical methods require a set of iterations to achieve a satisfactory solution formulated by a cost function to be minimized.
IK solutions can generally be divided into three sub-categories: Jacobian \cite{Siciliano_Handbook_Robot_2007}, Newtonians \cite{cohen_ik_1996} and Heuristics. Most software implement heuristic methods such as Cyclic Coordinate Descent (CCD) \cite{wang_ccd_1991} or 
Forward-Backward Reaching IK (FABRIK) \cite{aristidou_fabrik:_2011} due to their simplicity and extensibility. 

The main drawback of 
these solvers is that they manipulate kinematic chains without taking into account many morphological aspects that make a pose more or less plausible. They offer a first level of help to users but are not sufficient to guarantee a realistic pose. Many joints constraints are dependent on each other and require subjective, human-made approximations.

\subsubsection{Data-driven pose edition}
Data-driven methods offer promising opportunities to solve these approximations. Using real-life data can help in modelling the complex inter-dependencies of skeletons and providing users with smarter edition tools.
While it is still an early field of research, some solutions have been studied. Wu \etal \cite{wu_posing_2009} propose a method for natural character posing from a large motion database. It employs adaptive KD-clustering to select a representative frame from a database and sparse approximations to accelerate training and posing. 
Huang \etal in \cite{Huang_IK_MGDM_2017} present a method based on the formulation of multi-variate Gaussian distribution models (MGDMs), which learn the joint constraints of a kinematic skeleton from motion capture data. 

Some work has also been dedicated to finding new editing interfaces. \modify{}{Instead of the usual setup manipulating joints directly, Guay \etal \cite{guay_line_2013} articulate a framework based on the conceptual "line of action" which describes the overall pose dynamics. They provide a mathematical definition of the line of action, and a interface in which the software modifies the pose to follow a user-provided line. In the same line of though} Garcia \etal \cite{garcia_sketching_2019} propose \modify{a method transforming doodle of trajectories (position and orientation over time) }{a virtual reality-based interface where the user's hands motion (position and orientation over time) are transformed} into sequences of actions and then into detailed character animations using a dataset of parametrized motion clips automatically fitted to the trajectory. 

% ==> DL et Latent Space. 
\subsection{Neural modelling of human motion}
Neural networks have received a great amount of attention over the last decade and shown impressive result in modelling complex data. Human motion has not been spared and deep learning methods have proven their capability of generating realistic motion in a number of difficult cases. 

The literature in neural-based animation include example in user-controlled character navigation \cite{Holden2017} and interactions with the environment \cite{starke_neural_2019}. 
Holden \etal \cite{Holden2020} also show that neural networks can be used to replace parts of existing data-driven methods, improving their scalability potential.
More recently, some work has also focused on improving smaller parts of the animation pipeline rather than replacing it completely. Berson et al. \cite{berson_intuitive_2020} leverage neural networks to provide an interactive system to edit facial animation. 

% Wrap up
Data-driven IK and pose editing can relieve animators from time-consuming, back-and-forth pose adjustments by applying constraints extracted from real-world data. Recently, neural-network-based approaches have demonstrated their ability to model the intricacies of human motion while scaling to large amount of data and retaining a fast inference time. In this paper we seek to take advantage of these properties to create an efficient posing tool, intuitively usable even by a inexperienced user.
\section{Research Questions }
\label{sec:contr}

The following research questions (RQs) aim to reduce the gap towards having a clear source of ML practices oriented to SE. For that, the RQs are going to be crowdsourced data-driven, which means that the sources of information that will support them are not from a single source of information, but from multiple sources which are not centralized and with different origins. %Each of the research questions will be evaluate via interviews with practitioners or something like that.


\vspace{3pt} 
%\begin{quote}\textbf{RQ1.}  \emph{What is the perspective of the ML practitioners in relation to best practices, and to which stages of ML are the practices related? }\end{quote}
\begin{quote}\textbf{RQ1.}  \emph{What is the perspective of ML practitioners on best practices, and in which ML stages are they located?}\end{quote}% \MARIO{perspective of ML practitioners related to best practices.... what do you mean? "perspectiva de los practioners relacionados con las buenas prácticas"? sounds weird}

Answering this research question will help the practitioners' community to understand which stages of the process of developing  ML systems have best practices associated with them, and also could help to avoid pitfalls that are being executed by  omitting, with or without knowing, technical requirements or knowledge of the system.  We will answer this question by studying Stack Exchange posts in which practitioners ask questions about different topics, including ML, as mentioned in previous literature.  
And in order to minimize false positives, we will conduct filtering on the relevant posts based on topic (ML) and quality. After filtering the data, a process of analysis should be carried out in order to extract the possible practices. Subsequently, the practices  should be validated by ML experts in order to filter out the practices that may be outdated or are not considered good practices. %\textit{That will allow us to present a validated set of practices including a taxonomy, from the practitioners perspective.}

\begin{quote}\textbf{RQ2.} \emph{What is the perspective  and adoption  of ML practices by researchers and their studies? } \end{quote} %\MARIO{Perspective of ML practices ? Same comment than before}

The identified practices in this research question will give an indication of what practices are being used and reported by the SE research community.  This will help the SE research community to (i) identify possible points to strengthen the research and focus when describing their studies and protocols, (ii) identify possible good practices with SE examples,  which could facilitate the use of good practices and avoid making mistakes. We will answer this question by sampling ML-related papers from SE conferences and identifying ML practices, then they will be categorized in the different ML pipeline stages and SE applications (\ie defect modeling).  Complementary to this, we will conduct a survey and interviews with ML research experts in order to identify their opinion on the use of ML practices and the consequences of omitting them. %\textit{As a result, the study will enable to present a set of practices that were extracted from SE research  articles}


\begin{quote}\textbf{RQ3} \emph{What are the practices identified and adopted by practitioners and researchers? }\end{quote}

Answering this \textit{RQ}  will give both practitioners and researchers a better perspective on the used and identified ML practices. This will help the SE community,  in general,  to be aware of possible practices that are being used and/or omitted. For this \textit{RQ}, we will compile a handbook of practices from the  perspective of  SE researchers and practitioners.  As this \textit{RQ} complements the previous two \textit{RQs}, we will consider the results obtained in \textit{RQ1} and \textit{RQ2} while comparing and complementing the identified practices from both perspectives. In addition, we will enrich the practices with complementary information, such as use cases and previous research, to provide context and examples of their use. Also, we will provide the nature of the perspective (\ie researchers, practitioners, or both). 

\begin{quote}\textbf{RQ4} \emph{To what extend do the identified practices affect previous research?  }\end{quote}

Understanding how the use (or lack of use) of the practices  affects the result of research studies will give the community an idea of the impact and importance that this could cause.  Understanding the impact could generate more awareness of the use and report of the ML practices followed during the study.  For this, we will use a sample of  SE studies that use ML, that can be replicable, which allows us to obtain the same/similar results and then apply or omit ML practices and evaluate how that affects the results that were reported by the studies.  Kindly note that the study will be executed in a way in which, when reporting the results, they will not directly point to specific studies. This study will be a reflective exercise rather than a finger-pointing one, as previously done by the study conducted by Arp \etal~\cite{ArpQuiPen_22} when identifying dos and dont's in ML in computer security.

\iffalse
four papers
1) Paper that was submitted and is being reviewed (describe what we have done)
2) Paper from the research perspective ( if at some point can be done)  - Paper with David and Dennys (describe what we have done)

DESCRIBE ALSO CAIN 

3) Handbook of practices
4) What if we apply them to previous research?

\fi
\section{Current status}  
\label{sec:contr}

This section describes the results achieved so far for each research question in the context of the related work.


Regarding our first research question, in a paper currently under review~\cite{mojicapp}, we used as data source 14 StackExchange websites, including StackOverflow. We decided to use this family of Q\&A because of its popularity in the SE community, which can be seen in multiple studies that have used StackOverflow as a data source to analyze different topics in SE, \eg~\cite{zhang2021study, mondal2023automatic, chatterjee2020finding}. In addition, to its popularity in the SE community, StackOverflow has also been used to study ML-related topics such as expertise and challenges~\cite{Alshangiti_2019}, problems and challenges for ML libraries~\cite{Islam_2019}, and popular deep learning topics~\cite{Han_2020}.


%Regarding our first research question, and taking into account the popularity of communities of questions and answers (Q\&A) like StackOverflow, that has been used in previous studies to analyze different aspects in SE, \eg~\cite{zhang2021study, mondal2023automatic, chatterjee2020finding} and It has also been used to study ML-related topics such as expertise and challenges~\cite{Alshangiti_2019}, problems and challenges for ML libraries~\cite{Islam_2019}, and popular deep learning topics~\cite{Han_2020}.  In a paper currently under review~\cite{mojicapp}, we used not only the StackOverflow website as the main data source, but also 13 other StackExchange websites that were considered as potential data sources where ML practitioners ask and answer questions. 

As a result of selecting Q\&A Stack Exchange websites, filtering posts from them to extract the possible ML practices, and analyzing them, we obtained 157 ML best practices and their taxonomy.  The practices were obtained by executing an open-code procedure in which tags for the different practices were identified and assigned together with the identification of the ML pipeline stage(s) associated with each possible practice. For  stage identification, we used a predefined ML pipeline built by Amershi \etal \cite{amershi2019software}. As a result of the open-coding process, a list of 187 practices was identified, but only 157 were considered  best practices by ML experts after a validation process. 

Another outcome of the open-coding process is a four-level taxonomy. The first level of the taxonomy consists of the 10 ML-pipeline stages proposed by Amershi \etal \cite{amershi2019software}. The second level consists of categories  that encompass multiple tasks for each ML pipeline stage, \eg the learning category in the model training stage. The third level of the taxonomy is composed of an action/task that can be performed in each ML stage. The fourth level is the practice itself.

Upon further analysis of the practices, we identified which ML-pipeline stages had the highest number of practices and which ones had the lowest number of practices.  On the one side, the ML pipeline stages with the highest number of identified practices were model training and data cleaning, which could indicate an interest of practitioners in those two stages. The interest could be related to (i) model training being the core of the ML pipeline, as it enables the use of a model; (ii) data cleaning is a stage where  data scientists spend most of their time~\cite{anacondainc_2022}.   


On the other side, model deployment, model monitoring, and data labeling are the ones with the lowest number of identified practices. Regarding the low number of practices in model deployment and monitoring, this could be due to  these stages are more  related to the “operations” staff~\cite{LewisGrace2021WAIN} (\ie staff in charge of deploying, operating, and monitoring  ML-enabled systems).  Regarding the low number of practices identified for the data labeling stage, it could be related to the intrinsic nature of this stage.  By this, we mean that this stage is inherently  not mandatory in the ML pipeline, as it is only needed when  ground truth is required, \eg supervised and semi-supervised learning. In addition, sometimes, the data used to train models has already been labeled, which could lead to efforts being focused on other ML phases. 

Another  aspect that we noticed when analyzing the identified practices in the Q\&A websites is that they did not cover some specific topics.  Ethics is an example of a topic that was not discussed/covered by the identified practices.  This could indicate that there is a need to explore other sources of information to find ML-best practices, such as  technical blogs like the one presented by IBM~\cite{ibm_ethics}, which presents  an ethical framework for ML. 

When analyzing the validation of the ML experts, we noticed some  aspects to highlight. Firstly,  the majority of the practices considered good were validated by all the experts, which means that there was unanimous agreement. However, 30 practices were rejected by the experts, as only half or less of the experts considered them valid best practices. After inspecting the practices that the ML experts rejected, we noticed that, in most cases, the opinion was divided. This means that most of the time, half of the experts considered that the practices were not good ones, but the other half considered them good ones. This could mean that those practices, with divided opinions, were not well-known, or without a use case scenario, were not clear. In addition, some practices that were considered contradictory, \ie practices that indicate opposite actions, were agreed as good practices by the experts, which could also be an indicator of the need for more context to present an actionable practice. 

Concerning \textit{RQ2}, we are working on a study in which research-track  articles from top SE conferences are being analyzed in order to understand the reported and used ML practices. In this study, we are also taking as a reference the ML pipeline proposed by Amershi \etal \cite{amershi2019software}, for categorizing the identified practices. In preliminary results, we have found that the least mentioned stages, \ie stages in which few practices were identified, are related to model requirements, model deployment, and model monitoring. The last two were expected as it is not common to describe/execute those two stages in a research study. While the first mentioned stage, model requirements, could be considered the basis of a research study that uses ML, as it could define how the models should be built, and not defining it properly could cause disastrous consequences. 


Regarding  \textit{RQ3}, as part of the process of presenting a handbook of practices with both perspectives, practitioners, and researchers, we are currently designing an approach/tool, that will not only be able to be referenced but that will be useful in a practical way. With that, we mean that the practices will be associated and enriched with context, examples, and possible identified limitations.  In addition, this tool should present the aforementioned information in a friendly way, which will allow the users of the tool to find relevant information without going through an entire book, blog, or research article. For that, we are identifying ways  that  practices could be presented in a more interactive way,  like the appendix presented by Serban \etal \cite{serban2020adoption} for the SE practices for ML, the practices presented by Google in their  ``People + AI guided book'' \cite{goole_pair},  the ``Deep Learning Tuning Book''\cite{google_play_book} focused on the process of  model hyperparameter tuning addressed to engineers and researchers. We also take as a reference other white and gray literature aforementioned in the related work that, for each practice, present additional information, such as use cases, \eg~\cite{wujek2016best, ArpQuiPen_22}.



\iffalse


\fi
\section{Timeline}
\label{sec:time}

I am currently in my third year of my four-year Ph.D. program. In my third year, I plan to continue working on  my research, focusing mainly on \textit{RQ2} and \textit{RQ3}, while also finishing answering \textit{RQ1}.   In my last year, we will focus on \textit{RQ4} in order to complete it in my fourth year.

\section{Conclusion}
\label{sec:concl}

As a conclusion of this thesis, a synthesis of all four research questions will be provided, including a proposed tool to retrieve the practices.  This will help to reduce the gap of not having a clear handbook of ML practices applied to SE, since the set of validated practices will be oriented to the SE community with practitioners' and researchers' perspectives, which will be complemented with context and SE use cases.





\balance
\bibliographystyle{IEEEtran}
\bibliography{local}

\end{document}
