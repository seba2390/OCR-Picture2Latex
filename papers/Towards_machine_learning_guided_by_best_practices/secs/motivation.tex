
\section{Motivation - Problem definition}
\label{sec:motiv}

Machine learning (ML) has multiple fields of applications,  such as finance, medicine, education,  and software engineering. Indeed, in each  of these fields, ML has different kinds of applications; for example, fraud detection~\cite{awoyemi2017credit} and trading~\cite{sebastiao2021forecasting};  cancer detection~\cite{saba2020recent} and outbreak prediction~\cite{ardabili2020covid};  program translation~\cite{roziere2020unsupervised}; and code transformation~\cite{tufano2019learning}. Showing the wide adoption of ML for different tasks and disciplines and its capability to affect  multiple domains. 

The aforementioned  impact could also be seen in the industry, as it has not only grown in its usage demand but also its popularity and benefits.  For instance, a \textit{McKinsey} report by Chui \etal states that 50\% of the respondents to their study answered that their companies had adopted artificial intelligence  in at least one business function~\cite{chui_hall_mayhew_singla_2022};  this is also supported by the most recent \textit{NewVantage Partner 2022 Data and AI Executive survey (NPAIS)}, that  shows that 91\% of organizations are investing in AI activities~\cite{bean_2022}. Complementary,  on the same survey, NPAIS, it is reported that 92.1\% of organizations are realizing measurable benefits~\cite{bean_2022}, which could also be seen in the survey realized by  \textit{McKinsey} in which it is reported that AI has helped in increasing the revenue of companies in sales and product development~\cite{chui_hall_mayhew_singla_2022}. 


On the not-that-bright side,  recent studies have shown that ML-enabled systems (\ie systems that have at least one ML component)  have  challenges~\cite{Alshangiti_2019}, pitfalls~\cite{bone2015applying, biderman2020pitfalls, MichaelLones2021}, problems~\cite{sculley2015hidden}, or mismatches~\cite{LewisGrace2021WAIN} as any software development process and system. Some studies also have shown that  ML challenges and problems are also  publicly discussed in communities  such as Stack Overflow~\cite{Alshangiti_2019, Islam_2019, hamidi2021towards}. Moreover, some studies indicate that ML systems have particular problems and challenges~\cite{LewisGrace2021WAIN, sculley2015hidden}, and could be related to inadequate documentation and communication across the different actors involved in the ML development~\cite{LewisGrace2021WAIN}; or technical debt~\cite{sculley2015hidden}.   In addition,  the development of ML systems involves different phases from the traditional software ones~\cite{amershi2019software}, and in each of those phases, different challenges could be presented~\cite{LewisGrace2021WAIN, Alshangiti_2019, hamidi2021towards}.


Concerning ML challenges from an industry perspective, some surveys show common fears and challenges faced by companies.  Including problems  when collecting data\cite{creg_baker_2019}, data quality~\cite{creg_baker_2019}, versioning, and reproducibility of the models~\cite{thormundsson_2022}.  There are also some risks associated with artificial intelligence, such as equity and fairness or personal/individual privacy, that companies are working to mitigate~\cite{chui_hall_mayhew_singla_2022}.

In order to avoid, mitigate  or deal with the  ML challenges, pitfalls, and problems, some studies have proposed a series of recommended guidelines  and best practices based on their own experience and focused on their respective discipline, \eg~\cite{biderman2020pitfalls, MichaelLones2021, halilaj2018machine}. Additionally, there is a plethora of publications (\eg books, blogs) on the field of ML; for example, grey literature, such as the Google  article by Zinkevich~\cite{zinkevich_2021}, is  publicly available and could be considered as a first step with general practices derived from anecdotal evidence. However,  to the best of our knowledge, there are no handbooks listing best practices for using ML focused on SE practitioners or researchers, \ie software engineers and researchers. As ML is becoming more and more involved in SE development projects, bad practices should be avoided to prevent inadequate model planning, implementation, tuning, testing, deployment, and monitoring of ML implementations, \eg \cite{ LewisGrace2021WAIN, amershi2019software}.  The interest in ML has also been displayed in the SE community as more workshops, and conferences related to ML are being colocated within SE conferences, \eg~\cite{furlinger_2023, cain_2022, icse_2022, A2C2_2022, MaLTeSQuE_2022, EASEAI_2022}.  We aim to  reduce the gap of not having the aforementioned handbook by (i) studying what the best practices discussed by practitioners are; (ii) analyzing what  the practices used by SE researchers when executing studies that involve ML  models are, and (iii) building a taxonomy and a handbook of ML practices that could be used by the SE community that uses ML in order to guide their ML development and research.


% In order to deal with, avoid or mitigate  the  ML challenges, pitfalls, and problems, some studies have proposed a series of recommended guidelines  and best practices based on their own experience and focused on their respective discipline, \eg~\cite{biderman2020pitfalls, MichaelLones2021, halilaj2018machine}. Additionally, note that there is a plethora of publications (\eg books, blogs) on the field of ML; for example, grey literature such as the Google  article by Zinkevich~\cite{zinkevich_2021} is  publicly available and could be considered as a first step with general practices derived from anecdotal evidence. However,  to the best of our knowledge, there are no handbooks listing best practices for using ML focused on SE practitioners or researchers, \ie software engineers and researchers. As ML is becoming more and more involved in SE development projects, bad practices should be avoided to prevent inadequate model planning, implementation, tuning, testing, deployment, and monitoring of ML implementations.  We aim to  reduce this gap by (i) studying the best practices discussed by practitioners are; (ii) analyzing what  the practices used by SE researchers when executing studies that involve ML  models are, and (iii) building a taxonomy and a handbook of ML practices that could be used by the SE community that uses ML in order to guide their ML development and research. 
%\ANA[idea]{Ana, please ensure to write why it is needed in SE}



