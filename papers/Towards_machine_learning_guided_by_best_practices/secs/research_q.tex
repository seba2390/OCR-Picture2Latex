\section{Research Questions }
\label{sec:contr}

The following research questions (RQs) aim to reduce the gap towards having a clear source of ML practices oriented to SE. For that, the RQs are going to be crowdsourced data-driven, which means that the sources of information that will support them are not from a single source of information, but from multiple sources which are not centralized and with different origins. %Each of the research questions will be evaluate via interviews with practitioners or something like that.


\vspace{3pt} 
%\begin{quote}\textbf{RQ1.}  \emph{What is the perspective of the ML practitioners in relation to best practices, and to which stages of ML are the practices related? }\end{quote}
\begin{quote}\textbf{RQ1.}  \emph{What is the perspective of ML practitioners on best practices, and in which ML stages are they located?}\end{quote}% \MARIO{perspective of ML practitioners related to best practices.... what do you mean? "perspectiva de los practioners relacionados con las buenas prácticas"? sounds weird}

Answering this research question will help the practitioners' community to understand which stages of the process of developing  ML systems have best practices associated with them, and also could help to avoid pitfalls that are being executed by  omitting, with or without knowing, technical requirements or knowledge of the system.  We will answer this question by studying Stack Exchange posts in which practitioners ask questions about different topics, including ML, as mentioned in previous literature.  
And in order to minimize false positives, we will conduct filtering on the relevant posts based on topic (ML) and quality. After filtering the data, a process of analysis should be carried out in order to extract the possible practices. Subsequently, the practices  should be validated by ML experts in order to filter out the practices that may be outdated or are not considered good practices. %\textit{That will allow us to present a validated set of practices including a taxonomy, from the practitioners perspective.}

\begin{quote}\textbf{RQ2.} \emph{What is the perspective  and adoption  of ML practices by researchers and their studies? } \end{quote} %\MARIO{Perspective of ML practices ? Same comment than before}

The identified practices in this research question will give an indication of what practices are being used and reported by the SE research community.  This will help the SE research community to (i) identify possible points to strengthen the research and focus when describing their studies and protocols, (ii) identify possible good practices with SE examples,  which could facilitate the use of good practices and avoid making mistakes. We will answer this question by sampling ML-related papers from SE conferences and identifying ML practices, then they will be categorized in the different ML pipeline stages and SE applications (\ie defect modeling).  Complementary to this, we will conduct a survey and interviews with ML research experts in order to identify their opinion on the use of ML practices and the consequences of omitting them. %\textit{As a result, the study will enable to present a set of practices that were extracted from SE research  articles}


\begin{quote}\textbf{RQ3} \emph{What are the practices identified and adopted by practitioners and researchers? }\end{quote}

Answering this \textit{RQ}  will give both practitioners and researchers a better perspective on the used and identified ML practices. This will help the SE community,  in general,  to be aware of possible practices that are being used and/or omitted. For this \textit{RQ}, we will compile a handbook of practices from the  perspective of  SE researchers and practitioners.  As this \textit{RQ} complements the previous two \textit{RQs}, we will consider the results obtained in \textit{RQ1} and \textit{RQ2} while comparing and complementing the identified practices from both perspectives. In addition, we will enrich the practices with complementary information, such as use cases and previous research, to provide context and examples of their use. Also, we will provide the nature of the perspective (\ie researchers, practitioners, or both). 

\begin{quote}\textbf{RQ4} \emph{To what extend do the identified practices affect previous research?  }\end{quote}

Understanding how the use (or lack of use) of the practices  affects the result of research studies will give the community an idea of the impact and importance that this could cause.  Understanding the impact could generate more awareness of the use and report of the ML practices followed during the study.  For this, we will use a sample of  SE studies that use ML, that can be replicable, which allows us to obtain the same/similar results and then apply or omit ML practices and evaluate how that affects the results that were reported by the studies.  Kindly note that the study will be executed in a way in which, when reporting the results, they will not directly point to specific studies. This study will be a reflective exercise rather than a finger-pointing one, as previously done by the study conducted by Arp \etal~\cite{ArpQuiPen_22} when identifying dos and dont's in ML in computer security.

\iffalse
four papers
1) Paper that was submitted and is being reviewed (describe what we have done)
2) Paper from the research perspective ( if at some point can be done)  - Paper with David and Dennys (describe what we have done)

DESCRIBE ALSO CAIN 

3) Handbook of practices
4) What if we apply them to previous research?

\fi