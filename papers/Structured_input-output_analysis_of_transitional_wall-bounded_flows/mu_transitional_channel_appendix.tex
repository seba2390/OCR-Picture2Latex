

\iftoggle{thesis}{\chapter{Proof of theorems \ref{lemma:mu_componentwise_inf}-\ref{thm:scaling_Re_Pr}} }{\section{Proof of theorems \ref{lemma:mu_componentwise_inf}-\ref{thm:scaling_Re_Pr}} }



\iftoggle{thesis}{\section{Proof of theorem \ref{lemma:mu_componentwise_inf}}}{\subsection{Proof of theorem \ref{lemma:mu_componentwise_inf}}}
\label{appendix:mu_componentwise_inf}

\begingroup
\allowdisplaybreaks
\begin{myproof}
We define the following sets of uncertainty:
\begin{subequations}
\label{eq:uncertain_set_appendix}
\begin{align}
\mathbfsbilow{\widehat{U}}_{\Upxi,ux}:=\left\{\text{diag}\left(-\mathbfsbilow{\widehat{u}}_{\xi}^{\text{T}},\mathsfbi{0},\mathsfbi{0}\right):-\mathbfsbilow{\widehat{u}}^{\text{T}}_{\xi}\in \mathbb{C}^{N_y\times 3N_y}\right\},\label{eq:uncertain_set_u}\\
\mathbfsbilow{\widehat{U}}_{\Upxi,vy}:=\left\{\text{diag}\left(\mathsfbi{0},-\mathbfsbilow{\widehat{u}}_{\xi}^{\text{T}},\mathsfbi{0}\right):-\mathbfsbilow{\widehat{u}}^{\text{T}}_{\xi}\in \mathbb{C}^{N_y\times 3N_y}\right\},\label{eq:uncertain_set_v}\\
\mathbfsbilow{\widehat{U}}_{\Upxi,wz}:=\left\{\text{diag}\left(\mathsfbi{0},\mathsfbi{0},-\mathbfsbilow{\widehat{u}}_{\xi}^{\text{T}}\right):-\mathbfsbilow{\widehat{u}}^{\text{T}}_{\xi}\in \mathbb{C}^{N_y\times 3N_y}\right\}.\label{eq:uncertain_set_w}
\end{align}
\end{subequations}
Here, $\mathsfbi{0}\in \mathbb{C}^{N_y\times 3N_y}$ is a zero matrix with the size $N_y\times 3N_y$. 
Then, using the definition of the structured singular value in definition \ref{def:mu}, we have:
\begin{subequations}
\label{eq:mu_componentwise}
\begin{align}
    &\mu_{\mathbfsbilow{\widehat{U}}_{\Upxi,ux}}\left[\mathbfsbilow{H}_{\nabla}(k_x,k_z,\omega)\right]\nonumber\\
    =&\frac{1}{\text{min}\{\bar{\sigma}[\mathbfsbilow{\widehat{u}}_{\Upxi,ux}]\,:\,\mathbfsbilow{\widehat{u}}_{\Upxi,ux}\in \mathbfsbilow{\widehat{U}}_{\Upxi,ux},\,\text{det}[\mathsfbi{I}-\mathbfsbilow{H}_{\nabla}(k_x,k_z,\omega)\mathbfsbilow{\widehat{u}}_{\Upxi,ux}]=0\}}\label{eq:mu_componentwise_1}\\
    =&\frac{1}{\text{min}\{\bar{\sigma}[-\mathbfsbilow{\widehat{u}}^{\text{T}}_{\xi}]\,:\,-\mathbfsbilow{\widehat{u}}^{\text{T}}_{\xi}\in \mathbb{C}^{N_y\times 3N_y},\,\text{det}[\mathsfbi{I}_{3N_y}-\mathbfsbilow{H}_{\nabla ux}(k_x,k_z,\omega)(-\mathbfsbilow{\widehat{u}}^{\text{T}}_{\xi})]=0\}}\label{eq:mu_componentwise_2}\\
    =&\bar{\sigma}[\mathbfsbilow{H}_{\nabla ux}(k_x,k_z,\omega)].\label{eq:mu_componentwise_3}
\end{align}
\end{subequations}
Here, the equality \eqref{eq:mu_componentwise_1} is obtained by substituting the uncertainty set in \eqref{eq:uncertain_set_u} into definition \ref{def:mu}. The equality \eqref{eq:mu_componentwise_2} is obtained by performing block diagonal partition of terms inside of $\text{det}[\cdot]$ and employing zeros in the uncertainty set in equation \eqref{eq:uncertain_set_u}. Here, $\mathbfsbilow{H}_{\nabla ux}$ is the discretization of $\mathcal{H}_{\nabla ux}$ and $\mathsfbi{I}_{3N_y}\in \mathbb{C}^{3N_y\times 3N_y} $ in \eqref{eq:mu_componentwise_2} is an identity matrix with matching size $(3N_y\times 3N_y)$, whereas  $\mathsfbi{I} \in \mathbb{C}^{9N_y\times 9N_y}$ in \eqref{eq:mu_componentwise_1}. The equality \eqref{eq:mu_componentwise_3} is using the definition of unstructured singular value; see e.g., \citep[equation (11.1)]{zhou1996robust}. 

Similarly, we have:
\begin{subequations}
\label{eq:mu_componentwise_vwrho}
\begin{align}
    \mu_{\mathbfsbilow{\widehat{U}}_{\Upxi,vy}}\left[\mathbfsbilow{H}_{\nabla}(k_x,k_z,\omega)\right]=\bar{\sigma}[\mathbfsbilow{H}_{\nabla vy}(k_x,k_z,\omega)],\\
    \mu_{\mathbfsbilow{\widehat{U}}_{\Upxi,wz}}\left[\mathbfsbilow{H}_{\nabla}(k_x,k_z,\omega)\right]=\bar{\sigma}[\mathbfsbilow{H}_{\nabla wz}(k_x,k_z,\omega)].
\end{align}
\end{subequations}
Using the fact that $\mathbfsbilow{\widehat{U}}_{\Upxi}\supseteq \mathbfsbilow{\widehat{U}}_{\Upxi,ij}$ with $ij=ux,vy,wz$ and equalities in \eqref{eq:mu_componentwise}-\eqref{eq:mu_componentwise_vwrho}, we have:
\begin{align}
     \mu_{\mathbfsbilow{\widehat{U}}_{\Upxi}}\left[\mathbfsbilow{H}_{\nabla}(k_x,k_z,\omega)\right]\geq\mu_{\mathbfsbilow{\widehat{U}}_{\Upxi,ij}}\left[\mathbfsbilow{H}_{\nabla}(k_x,k_z,\omega)\right]=\bar{\sigma}[\mathbfsbilow{H}_{\nabla ij}(k_x,k_z,\omega)].
     \label{eq:mu_componentwise_inequality}
\end{align}
Applying the supreme operation $\underset{\omega \in \mathbb{R}}{\text{sup}}[\cdot]$ on \eqref{eq:mu_componentwise_inequality} and using definitions of $\|\cdot\|_{\mu}$ and $\|\cdot \|_{\infty}$ we have:
\begin{subequations}
\begin{align}
    \|\mathcal{H}_{\nabla }\|_{\mu}\geq \|\mathcal{H}_{\nabla ux}\|_{\infty},\;\; \|\mathcal{H}_{\nabla }\|_{\mu}\geq \|\mathcal{H}_{\nabla vy}\|_{\infty},\;\;
    \|\mathcal{H}_{\nabla }\|_{\mu}\geq \|\mathcal{H}_{\nabla wz}\|_{\infty}.\tag{\theequation a-c}
\end{align}
\end{subequations}
This directly results in inequality \eqref{eq:mu_larger_than_all_diagonal_component} of theorem \ref{lemma:mu_componentwise_inf}. 
\end{myproof}
\endgroup




\iftoggle{thesis}{\section{Proof of theorem \ref{thm:scaling_Re_Pr}}}{\subsection{Proof of theorem \ref{thm:scaling_Re_Pr}}}

\label{appendix:proof_scaling_Re_Pr}

\begin{myproof}
Under the assumptions of streamwise constant $k_x=0$ for plane Couette flow or plane Poiseuille flow in theorem \ref{thm:scaling_Re_Pr}, the operator $\widehat{\mathcal{A}}$, $\widehat{\mathcal{B}}$, and $\widehat{\mathcal{C}}$ can be simplified and decomposed as:
\begin{subequations}
\label{eq:operator_ABC_appendix_proof}
\begin{align}
    \widehat{\mathcal{A}}(k_x,k_z)=&\begin{bmatrix}
   \frac{\widehat{{\nabla}}^{-2}\widehat{{\nabla}}^4}{Re} & 0 \\
    -\text{i}k_zU' & \frac{\widehat{{\nabla}}^2}{Re} 
    \end{bmatrix},\\
    \mathcal{\widehat{B}}(k_x,k_z)=&
    \begin{bmatrix}
    0 & -k_z^2\widehat{{\nabla}}^{-2} & -\text{i}k_z\widehat{{\nabla}}^{-2} \partial_y \\
    \text{i}k_z & 0 & 0 
    \end{bmatrix}=:\begin{bmatrix}
    0 & \widehat{\mathcal{B}}_{y,1} & \widehat{\mathcal{B}}_{z,1} \\
    \mathcal{B}_{x,2} & 0 & 0
    \end{bmatrix},\\
    \mathcal{\widehat{C}}(k_x,k_z)=&\begin{bmatrix}
    0 & -\text{i}/k_z  \\
    \mathcal{I} & 0 \\
    \text{i} \partial_y/k_z & 0
    \end{bmatrix}=:\begin{bmatrix}
    0 &  \mathcal{\widehat{C}}_{u,2}  \\
     \mathcal{\widehat{C}}_{v,1} & 0 \\
     \mathcal{\widehat{C}}_{w,1} & 0 
    \end{bmatrix}.
\end{align}
\end{subequations}
Here, we employ a matrix inverse formula for the lower triangle block matrix:
\begin{align}
        \begin{bmatrix}
    L_{11} & 0 \\
    L_{21} & L_{22}
    \end{bmatrix}^{-1}=\begin{bmatrix}
    L_{11}^{-1} & 0\\
    -L_{22}^{-1}L_{21}L_{11}^{-1} & L_{22}^{-1} 
    \end{bmatrix}
\end{align}
to compute $\left(\text{i}\omega\mathcal{I}_{2\times 2}-\widehat{\mathcal{A}}\right)^{-1}$. Then, we employ a change of variable $\Omega=\omega Re$ similar to \citep{jovanovic2004modeling,Jovanovic2005,jovanovic2020bypass} to obtain $\mathcal{H}_{\nabla ij}$ with $i=u,v,w$, and $j=x,y,z$ as:
\begingroup
\allowdisplaybreaks
\begin{subequations}
\begin{align}
    \mathcal{H}_{\nabla ux}=&\widehat{\boldsymbol{\nabla}}\widehat{\mathcal{C}}_{u,2}Re\left(\text{i}\Omega \mathcal{I}-\widehat{{\nabla}}^2\right)^{-1}\widehat{\mathcal{B}}_{x,2},\\
    \mathcal{H}_{\nabla uy}=&\widehat{\boldsymbol{\nabla}}\widehat{\mathcal{C}}_{u,2}Re\left(\text{i}\Omega \mathcal{I}-\widehat{{\nabla}}^2\right)^{-1}(-\text{i}k_zU')Re\left(\text{i}\Omega \mathcal{I}-\widehat{{\nabla}}^{-2}\widehat{{\nabla}}^{4}\right)^{-1} \widehat{\mathcal{B}}_{y,1},\\
    \mathcal{H}_{\nabla uz}=&\widehat{\boldsymbol{\nabla}}\widehat{\mathcal{C}}_{u,2}Re\left(\text{i}\Omega \mathcal{I}-\widehat{{\nabla}}^2\right)^{-1}(-\text{i}k_zU')Re\left(\text{i}\Omega \mathcal{I}-\widehat{{\nabla}}^{-2}\widehat{{\nabla}}^{4}\right)^{-1} \widehat{\mathcal{B}}_{z,1},\\
    \mathcal{H}_{\nabla vx}=&0,\\
    \mathcal{H}_{\nabla vy}=&\widehat{\boldsymbol{\nabla}}\widehat{\mathcal{C}}_{v,1}Re\left(\text{i}\Omega \mathcal{I}-\widehat{{\nabla}}^{-2}\widehat{{\nabla}}^{4}\right)^{-1} \widehat{\mathcal{B}}_{y,1},\\
    \mathcal{H}_{\nabla vz}=&\widehat{\boldsymbol{\nabla}}\widehat{\mathcal{C}}_{v,1}Re\left(\text{i}\Omega \mathcal{I}-\widehat{{\nabla}}^{-2}\widehat{{\nabla}}^{4}\right)^{-1} \widehat{\mathcal{B}}_{z,1},\\ \mathcal{H}_{\nabla wx}=&0,\\
    \mathcal{H}_{\nabla wy}=&\widehat{\boldsymbol{\nabla}}\widehat{\mathcal{C}}_{w,1}Re\left(\text{i}\Omega \mathcal{I}-\widehat{{\nabla}}^{-2}\widehat{{\nabla}}^{4}\right)^{-1} \widehat{\mathcal{B}}_{y,1},\\
    \mathcal{H}_{\nabla wz}=&\widehat{\boldsymbol{\nabla}}\widehat{\mathcal{C}}_{w,1}Re\left(\text{i}\Omega \mathcal{I}-\widehat{{\nabla}}^{-2}\widehat{{\nabla}}^{4}\right)^{-1} \widehat{\mathcal{B}}_{z,1}. 
\end{align}
\end{subequations}
\endgroup
Taking the operation that $\|\cdot\|_{\infty}=\underset{\omega\in\mathbb{R}}{\text{sup}}\bar{\sigma}[\cdot]=\underset{\Omega\in\mathbb{R}}{\text{sup}}\bar{\sigma}[\cdot]$, we have the scaling relation in theorem \ref{thm:scaling_Re_Pr}. 
\end{myproof}