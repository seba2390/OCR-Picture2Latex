
\section{Hadronic molecules in lattice QCD}
\label{sec:lattice}

Lattice QCD is, in principle, the tool to calculate the spectrum of QCD from
first principles. There has been a remarkable progress in the last years in this
field, see
e.g.~\cite{Durr:2008zz,Baron:2010bv,Edwards:2011jj,Liu:2012ze,Liu:2016kbb}.
Still, the extraction of the properties of resonances and, in particular, of
hadronic molecules, from finite volume {calculations}  poses severe
challenges.
When QCD is put on an Euclidean space-time lattice {with a finite
space-time volume, asymptotic states cannot be defined and right-hand cuts are replaced by
poles, thus preventing a direct calculation of scattering and resonance
properties.}
This obstacle was overcome by L\"uscher a long time ago. He derived a relation
between the energy shift in the finite volume and the scattering phase shift in
the continuum \cite{Luscher:1990ux,Luscher:1986pf}, see also
Refs.~\cite{Wiese:1988qy,DeGrand:1990ip}.
This approach has become  known and used as L\"uscher's method.
More precisely, in order to determine the mass and width from the measured
spectrum, one first extracts the scattering phase shift by using the L\"uscher
equation. In the next step, using some parameterization for the $K$-matrix
(e.g., the effective range expansion), a continuation into the complex energy
plane is performed. As noted in Sec.~\ref{sec:Sproperties}, resonances
correspond to poles of the scattering $T$-matrix on the second Riemann sheet,
and the real and imaginary parts of the pole position define the mass and the
width of a resonance. A nice example is given by the $\rho$-meson, that has been
considered using L\"uscher's method, e.g., in
Refs.~\cite{Feng:2010es,Lang:2011mn,Aoki:2011yj,Dudek:2012xn}.
In these papers it has already been shown that even for such realistic
calculations of a well isolated resonance,
the inclusion of {hadron-hadron type interpolating operators} is mandatory,
it is simply not sufficient to represent the decaying resonance by properly
chosen quark bilinears {(for mesons)}.
For the discussion of hadronic molecules (or most other hadron resonances), this
method needs to be extended in various directions, such as considering higher
partial waves and spins
\cite{Bernard:2008ax,Luu:2011ep,Konig:2011nz,Konig:2011ti,Briceno:2012yi,
Gockeler:2012yj}, moving frames
\cite{Rummukainen:1995vs,Bour:2011ef,Davoudi:2011md,Fu:2011xz,Leskovec:2012gb,
Gockeler:2012yj}, multi-channel scattering
\cite{Liu:2005kr,Lage:2009zv,Bernard:2010fp,Doring:2011ip,Li:2012bi,Guo:2012hv},
including the use of unitarized chiral perturbation theory (and related methods)
\cite{Doring:2011vk,MartinezTorres:2011pr,Doring:2011nd,Albaladejo:2012jr,
Wu:2014vma,Hu:2016shf}, and three-particle final states
\cite{Kreuzer:2010ti,Polejaeva:2012ut,Briceno:2012rv,Hansen:2014eka,
Meissner:2014dea,Hansen:2015zga,Hansen:2015zta,Hansen:2016ync,Hansen:2016fzj,Guo:2017ism}.


Here, we will not attempt to review the lattice QCD approach to the hadron
spectrum in any detail but just focus on the bits and pieces that are relevant
for the investigation of possible hadronic molecules.
In Sec.~\ref{sec:reso} we summarize the L\"uscher method and its extension to
the multi-channel space, followed by a discussion of the compositeness criterion
in a finite volume, see Sec.~\ref{sec:compo}.
In Sec.~\ref{sec:qmdep}, we discuss how the quark mass dependence of certain
observables can be used to differentiate hadronic molecules from more compact
multi-quark states and in Sec.~\ref{sec:latres}, we briefly summarize pertinent
lattice QCD calculations for the possible molecular states containing charm
quarks. A short final subsection is devoted to certain states made of light
quarks only.


\subsection{Resonances in a finite volume}
\label{sec:reso}

The essence of the L\"uscher approach can be understood in a simple
nonrelativistic model for the scattering of identical, spinless particles of
mass $m$ in 1+1 dimensions.
In the CM frame, the relative momentum is quantized according to $p =(2\pi/L)n$,
with $L$ the spatial lattice extension and $n$ an integer. In case of no
interactions between these particles, the energy of the two-particle system is
simply given by $E=2m+p^2/m$, which means that the free energy level-$n$ scales
as $n^2/L^2$ with the volume and thus levels with different $n$ do not
intersect. In the presence of interactions, this behaviour is modified. Let us
assume that this interaction leads to a narrow resonance at $\sqrt{s_R} = E_R -
i\Gamma_R/2$, that is $\Gamma_R \ll E_R$. In the infinite volume limit, this
interaction leads to a phase shift $\delta (p)$  in the asymptotic wave
function. Furthermore, in the presence of a resonance, the phase shift will
change by $\pi$ (known as Levinson's theorem~\cite{Levinson:1949zz}).
In a finite volume, this behavior translates into the boundary condition
\begin{equation}
p L + 2\delta(p) = 2\pi \, m~,~~~m\in {\mathbb Z}~ .
\end{equation}
This condition provides the link between the volume dependence of the
energy levels in the interacting system and the continuum phase shift.
If one follows an energy level inwards from the asymptotic region to
smaller lattice sizes, in the vicinity of a resonance, this boundary 
condition causes a visible distortion, the so-called {\em avoided level
crossing}, {\sl c.f.} Fig.~\ref{fig:avoided}. The plateau, where the energy of
the two-particle system is almost volume-independent, corresponds to the real
part of the pole $E_R$. The imaginary part of the pole is given by the slope
according to $\left.d \delta(p)/dE\right|_{E_R} = 2/\Gamma_R$. Clearly, this method can only
work when certain conditions are fulfilled. First, the method as described 
here is restricted to the elastic two-particle case. Second, one has to make
sure that the interaction range of the particles is much smaller than the
size of the box to make the notion of asymptotic states possible. Third,
to suppress polarization effects that arise from the interactions of the 
lightest particles in the theory with each other around the torus, one
has to choose $L$ such that $1/m \ll L$.

%-----------------------------------------------------------
\begin{figure}[t!]
\begin{center}
 \includegraphics[width=0.4\textwidth]{./figures/avoided_E.pdf}
\caption{Energy levels of an interacting two-particle system. In
case of a resonance in this system, the energy levels exhibit the
avoided level crossing (plateau) that allows to read off the 
resonance energy $E_R$ directly.} 
\label{fig:avoided}
\end{center}
\end{figure}
%-----------------------------------------------------------

We now consider the extension of the L\"uscher method to the multi-channel
case, as most hadronic molecules are located close to a two-particle threshold
or between two close-by thresholds. To achieve this extension, an 
appropriate tool is a particular version of an NREFT,
because up to the energies where multi-particle inelastic states become
significant, such a framework is completely equivalent to the relativistic
field theory, provided the couplings in the nonrelativistic framework
are determined from matching to the relativistic $S$-matrix elements, for
details and further references, see \cite{Bernard:2008ax,Colangelo:2006va,Gasser:2011ju}. For the one-channel case, it was already shown in 
Ref.~\cite{Beane:2003yx}
that using such an NREFT, one obtains at a very simple and transparent 
proof of L\"uscher's formula.

To keep the presentation simple, we first consider a two-channel LSE in NREFT in
the infinite volume.
Let us consider antikaon-nucleon scattering in the region of the $\Lambda(1405)$
resonance, $\bar KN\to\bar KN, \Sigma\pi$.
The channel number 1 refers to $\bar KN$ and 2 to $\Sigma\pi$ with total isospin
$I = 0$.
The resonance $\Lambda (1405)$ is located between two thresholds, on the second
Riemann sheet, close to the real axis.
The two  thresholds are given by  $s_t = (m_N + M_K)^2$ and
$s'_t=(m_\Sigma+M_\pi)^2$. We work in the isospin limit and neglect the fact
that there are really two poles --- see Refs.~\cite{Oller:2000fj,Jido:2003cb}
and Sec.~\ref{sec:1405th}.\footnote{Note that in this  two-channel formulation
one only has one pole corresponding to one $\Lambda (1405)$. To deal with the
two-pole scenario requires the inclusion of more channels and explicit isospin
breaking.}
For energies above the $\bar KN$  threshold, $s > (m_N + M_K)^2$, the
coupled-channel LSE for the $T$-matrix elements $T_{ij}(s)$ in
dimensionally regularized NREFT reads (we only consider $S$-waves here)
\begin{eqnarray}
\label{eq:LSinfini} 
T_{11}  &=& H_{11} + H_{11} \, iq_1 T_{11} + H_{12} \,
iq_2 T_{21}~,\nonumber\\
T_{21}  &=& H_{21} + H_{21} \, iq_1 T_{11} + H_{22} \, iq_2 T_{21} \,, 
\end{eqnarray}
with
$q_1 = \lambda^{1/2} (s,m_N^2,M_K^2)/(2\sqrt{s})$, $q_2 = \lambda^{1/2}
(s,m_\Sigma^2,M_\pi^2)/(2\sqrt{s})$ and $\lambda(x,y,z)=x^2+y^2+z^2-2xy-2yz-2zx$
is the K\"all\'en function.
Furthermore, the $H_{ij}(s)$ denote the driving potential in the corresponding
channel, i.e. the matrix element of the interaction Hamiltonian between the free
two-particle states.
Continuation of the CM momentum $q_1$ below threshold $(m_\Sigma + M_\pi)^2 <s<
(m_N + M_K)^2$ is obtained via \beq iq_1 \to -\kappa_1 =
-\frac{(-\lambda(s,M_K^2,m_N^2))^{1/2}}{2\sqrt{s}} \ .
\eeq The resonance corresponds to a pole on the second Riemann sheet in the
complex $s$-plane. Its position is given by the solution of the secular 
equation \beq\label{eq:secular} \Delta (s) = 1 + \kappa_1^R \, H_{11} -
\kappa_2^R \, H_{22} - \kappa_1^R \kappa_2^R \, \left( H_{11} H_{22} -
H_{12}^2\right) \eeq with $\kappa_1^R = -(-\lambda
(s_R,m_N^2,M_K^2))^{1/2}/(2\sqrt{s_R})$ and $\kappa_2^R = (-\lambda
(s_R,m_\Sigma^2,M_\pi^2)^{1/2})/(2\sqrt{s_R})$.
The energy and width of the resonance are then  given by $\sqrt{s_R} = E_R - i
\Gamma_R/2$.

Consider next the same problem in a finite volume. Only discrete
values of the three-momentum are allowed, given by $\bm{k} = 
(2\pi/L)\bm{n}$, with $\bm{n}$ a triplet of integer numbers. 
Thus, we replace the three-momentum integration in the loops by a discrete
sum (see  Ref.~\cite{Bernard:2008ax} for more details).
The rotational symmetry is broken to a cubic symmetry, so mixing of 
different partial waves occurs. Here,  however, we only consider $S$-waves, 
neglecting the small mixing to higher partial waves. 
If necessary, the mixing can be easily included at a later 
stage, see e.g.~\cite{Bernard:2008ax,Doring:2012eu}.
The finite-volume version of the LSE
Eq.~(\ref{eq:LSinfini}) then takes the form
\beqa
T_{11}  &=& H_{11} - \frac{2  Z_{00}(1;k_1^2)}{\sqrt{\pi}L}\,  H_{11}
T_{11} - \frac{2 Z_{00}(1;k_2^2)}{\sqrt{\pi}L}\,  H_{12} T_{21}~,\nonumber\\
T_{21}  &=& H_{21} - \frac{2 Z_{00}(1;k_1^2)}{\sqrt{\pi}L}\,  H_{21}
T_{11} - \frac{2  Z_{00}(1;k_2^2)}{\sqrt{\pi}L}\, H_{22} T_{21}~,\nonumber\\
\eeqa
with 
\beqa
k_{1/2}^2 &=& \left(\frac{L}{2\pi}\right)^2 \, 
\frac{\lambda(s,M_{K/\pi}^2,m_{N/\Sigma}^2)}{4s}~,\nonumber\\
Z_{00} (1;k^2) &=& \frac{1}{\sqrt{4\pi}} \,\lim_{r\to 1}\sum_{\bm{n} \in 
{\mathbb R}^3}
\frac{1}{({\bm{n}\,}^2 - k^2)^r}~.
\eeqa
Here, we have neglected the terms that vanish exponentially at large 
$L$.
The secular equation that determines the spectrum can be brought into the
form
\beqa\label{eq:pseudophase}
&&\qquad  1 - \frac{2}{\sqrt{\pi} L} \,  Z_{00}(1;k_2^2)\, F(s,L) = 0~, 
\nonumber\\
&& F(s,L) = \left[ H_{22} -   \frac{2 Z_{00}(1;k_1^2)}{\sqrt{\pi}L}\,   
(H_{11}H_{22} - H_{12}^2)\right] \nonumber\\ 
&& \qquad \qquad \times \left[1 - \frac{2 Z_{00}(1;k_1^2)}{\sqrt{\pi}L}\,  
H_{11}\right]^{-1}.
\eeqa
This can be rewritten as 
\beqa\label{eq:1channel}
\delta(s,L) &=& -\phi(k_2) + n\,\pi~, \quad n = 0,1,2, \ldots  \nonumber\\
\phi(k_2) &=& -\arctan \frac{\pi^{3/2} \, k_2}{ Z_{00} (1;k_2^2)}~,
\eeqa
with
\beq
\tan\delta(s,L) = q_2(s) \,  F(s, L)~.  
\eeq
$\delta (s,L)$ is called the {\em pseudophase}.
The dependence of the pseudophase on $s$ and $L$
is very different from that of the usual scattering 
phase.
Namely, the elastic phase extracted from the lattice data by using
L\"uscher's formula is independent of the volume modulo terms that 
exponentially vanish at a large $L$. Further, the energies where the
phase passes through $\pi/2$ lie close to the real resonance locations.
In contrast with this, the pseudophase contains the function
$Z_{00}(1;k_1^2)$, which does not vanish exponentially at
a large $L$ and a positive $k_1^2$.
Moreover, 
% it contains the tower of ``resonances'' 
the tangent of the pseudophase contains a tower of poles
at the energies given by the roots of the equation
$1-({2}/{\pi L})\,Z_{00}(1;k_1^2)\,H_{11}=0$. On the other hand,
in the infinite volume
this equation reduces to $1+\kappa_1^R\,H_{11}=0$, {\sl c.f.} with
Eq.~(\ref{eq:secular}), which has only one root below threshold very close to
the position of the $\Lambda(1405)$. Other roots in a finite volume 
% are not related to the dynamics of the system
% in the infinite volume and 
stem from oscillations of $Z_{00}(1,k_1^2)$ 
between $-\infty$ and $+\infty$ when the variable $k_1^2$ varies 
along the positive semi-axis. {Their locations depend on $H_{11}$ and thus
contain information of the infinite-volume interaction}.
This is an effect of discrete energy levels in the ``shielded'' channel, {which is the channel with the lower
threshold in the coupled-channel system}.
The pseudophase depends on the three real functions $H_{11},H_{12},H_{22}$. Based on 
synthetic data it was shown in 
Ref.~\cite{Lage:2009zv} that a measurement of the lowest two eigenvalues
at energies between 1.4 and 1.5~GeV allows one to reconstruct the
pseudophase and to extract in principle the pole position. It was further
pointed out in that work that two-particle
thresholds also lead to an avoided level crossing, so the extraction
of the resonance properties from the corresponding plateaus in the
energy dependence of certain levels is no longer possible.
In the case of real data, taking into account the
uncertainties of each measurements, one has to measure more levels
on a finer energy grid. To obtain a sufficient amount of data in a
given volume, twisting and asymmetric boxes can also be helpful.
First such {calculations} have become available
recently and will be discussed below.

An alternative formulation, that allows the use of  fully relativistic
two-particle propagators and can easily be matched to the representation
of a given scattering amplitude based on unitarized chiral perturbation
theory (UCHPT) was worked out in~\cite{Doring:2011vk}. 
The method is based on the observation that
in coupled-channel UCHPT, certain
resonances are dynamically generated, e.g. the light scalar mesons in the
coupled $\pi\pi/\bar KK$ system. The basic idea is to
consider this approach  in a finite volume to
produce the volume-dependent discrete energy spectrum.
Reversing the argument, one is then able to fit the parameters of the
chiral potential to the measured energy spectrum on the
lattice and, in the next step, determine the resonance locations
by solving the scattering equations in the infinite
volume. By construction, this method fulfills the constraints from
chiral symmetry such as the appearance of Adler zeros at certain
unphysical points.
%
For recent developments using a relativistic framework, we refer
to~\cite{Briceno:2015csa,Briceno:2015dca,Briceno:2015tza}.

\subsection{Quark mass dependence}
\label{sec:qmdep}


To reduce numerical noise as well as to speed up algorithms, 
lattice {calculations}
are often performed at unphysical values of the light quark masses. 
While this at first sight may appear as a disadvantage, it is indeed
a virtue as it enables a new handle on investigating the structure
of certain states. However, in the case of multiple coupled channels,
one also has to be aware that thresholds and poles can move very
strongly as a function of the quark masses. This intricate interplay
between $S$-wave thresholds and resonances needs to be accounted for 
when one tries to extract the resonance properties.

To address the first issue, we specifically consider the charm-strange
mesons $D_{s0}^*(2317)$ and $D_{s1}(2460)$. As shown in \cite{Cleven:2010aw},
in the molecular picture describing these as $DK$ and $D^*K$ bound states,
a particular pion and kaon mass dependence arises. Consider first the
dependence on the light quark masses, that can be mapped onto the
pion mass dependence utilizing the Gell-Mann--Oakes--Renner 
relation~\cite{GellMann:1968rz},
$M_{\pi^\pm}^2 = B(m_u+m_d)$, that naturally arises in QCD as the leading 
term in the chiral expansion of the Goldstone boson mass. Here, $B$ is
related to the vacuum expectation value of the quark condensate. In fact, this
relation is fulfilled to better than 94\% in QCD~\cite{Colangelo:2001sp}.
As shown in Ref.~\cite{Cleven:2010aw}, the pion mass dependence of such
a molecular state is much more pronounced than for a simple $c\bar s$ state.
Even more telling and unique is, however, the kaon mass dependence.
For that, consider the  $M_K$ dependence of the mass of a bound state
of a kaon and some other hadron. The mass of such a kaonic bound state
is given by
\begin{equation}
M = M_K+M_h-E_B,
\end{equation}
where $M_h$ is the mass of the other hadron, and $E_B$ denotes the binding
energy. Although both $M_h$ and $E_B$ have some kaon mass dependence, it is 
expected to be a lot weaker
than that of the kaon itself. Thus, the important implication of this simple 
formula is that the leading
kaon mass dependence of a kaon-hadron bound state is {\em linear, and the 
slope is unity}. The only
exception to this argument is if the other hadron
is also a kaon or anti-kaon. In this case, the
leading kaon mass dependence is still linear but with the 
slope being changed to two.
Hence, as for the $DK$ and $D^*K$ bound states, one expects that
their masses are linear in the kaon mass, and the slope is 
approximately one. This expectation is borne out by
the explicit calculations performed in~\cite{Cleven:2010aw}.
Early lattice QCD attempts to investigate this peculiar kaon mass dependence
have led to inconclusive results~\cite{mcneile}. Other papers that discuss 
methods 
to analyze the structure of states based on their quark mass dependence
or the behavior at large number of colors are 
e.g.~\cite{Hanhart:2008mx,Pelaez:2010fj,Bernard:2010fp,Albaladejo:2012te,
Guo:2011pa,Nebreda:2011cp,Pelaez:2006nj,Guo:2015dha}.

The second issue we want to address briefly is the intricate interplay
of $S$-wave thresholds and resonance pole positions with varying quark masses,
as detailed in Ref.~\cite{Doring:2013glu}. In that paper, pion-nucleon
scattering in the $J^P = 1/2^-$ sector in the finite volume and at
varying quark masses based on UCHPT was studied. In the infinite volume,
both the $N(1535)$ and the  $N(1650)$ are dynamically generated
from the coupled channel dynamics of the isospin $I=1/2$ and strangeness
$S=0$ $\pi N, \eta N, K\Lambda$  and $K\Sigma$ system. Having fixed the
corresponding LECs in the infinite volume, one can straightforwardly
calculate the spectrum in the finite volume provided one knows the
octet Goldstone boson masses, the masses of the ground-state octet 
baryons and the meson decay constants. Such sets of data at different
quark masses are given by ETMC and QCDSF. ETMC provides masses
and decay constants for $M_\pi=269$~MeV and the kaon mass 
close to its physical value~\cite{Alexandrou:2009qu,Ottnad:2012fv}.
Quite differently, the QCDSF Collaboration~\cite{Bietenholz:2011qq}
obtains the baryon and meson masses  from an alternative approach to 
tune the quark masses. Most importantly, while the lattice size and 
spacing are comparable to those of the ETMC, the strange quark mass 
differs significantly from the physical value. The latter results in a
different ordering of the masses of the ground-state octet mesons and, 
consequently, in a different ordering of meson-baryon thresholds.
For the ETMC parameters,  all thresholds are moved to higher energies. 
The cusp at the $\eta N$ threshold has become more pronounced, but
no clear resonance shapes are visible. Indeed, going to the complex
plane, one finds that the poles are hidden {in the Riemann sheets which
are not directly connected to the physical one by crossing the cut at the
energies corresponding to the real parts of the poles}.
Using the QCDSF parameters, the situation is very different. In contrast to the ETMC case, a 
clear resonance signal is visible below the $K\Lambda$ threshold, 
that is the first inelastic channel in this parameter setup. Indeed, one
finds a pole  on the corresponding Riemann sheet. Unlike in the 
ETMC case, it is not hidden behind a threshold. Between
the $K\Lambda$ and the $K\Sigma$ threshold, there is only a hidden pole. 
The $K\Sigma$ and $\eta N$ thresholds are almost
degenerate, and on sheets corresponding to these higher-lying 
thresholds one only finds hidden poles. For more details, the reader
is referred to Ref.~\cite{Doring:2013glu}. In that paper, strategies
to overcome such type of difficulties are also discussed.

{It is worthwhile to mention that the composition of a hadron in general
may vary when changing the quark masses. However, as long as the quark masses are
not very different from the physical values, the quark mass dependence is rather
suggestive towards revealing the internal structure as different structures
should result in different quark mass dependence. }

\subsection{Measuring compositeness on lattice}
\label{sec:compo}

As discussed in Sec.~\ref{sec:weinberg}, the Weinberg compositeness criterion
offers a possibility to disentangle compact bound states from loosely bound
hadronic molecules. {By measuring the low-energy scattering observables in
lattice using the L\"uscher formalism discussed before, one can extract the
compositeness by using Eqs.~\eqref{eq:arwein}. For related work, see, e.g.,
Refs.~\cite{Suganuma:2007uv,MartinezTorres:2011pr,Ozaki:2012ce,
Albaladejo:2013aka}. It is pointed out in~\cite{Agadjanov:2014ana} that the use
of partially twisted boundary conditions is cheaper than studying the volume dependence in lattice
for measuring the compositeness.} 
% The finite volume formulation of this approach
% was given in Ref.~\cite{Agadjanov:2014ana}. 
The basic object in that work is the
scattering amplitude in the finite volume, which can be obtained from the
corresponding loop function $\tilde G_L^{\bm{\theta}}(s) = G(s)+\Delta
G_L^{\bm{\theta}}(s)$~\cite{Doring:2011vk}, where $\Delta G_L^{\bm{\theta}}$ can
be related to the modified L\"uscher function $Z_{00}^{\bm{\theta}}$ via
\begin{equation}
  \label{deltaG-luscher1}
  \Delta G_L^{\bm{\theta}}(s)=\frac1{8\pi\sqrt{s}}\left(
    ik - \frac{2}{\sqrt{\pi}L} Z_{00}^{\bm{\theta}}(1,\hat k^2)
  \right)+\cdots,
\end{equation}
where $\hat k = kL/(2\pi)$ and ellipsis denote terms that are exponentially
suppressed with the lattice size $L$~\cite{Doring:2011vk}. Here, in case 
of twisted boundary conditions, the momenta also depend on the twist angle
$\bm{\theta}$ according to 
$\bm{q}_n=(2\pi/L)\bm{n}+(\bm{\theta}/L),~0\leq\theta_i<2\pi$. 
In case of a bound state with mass $M$ in the infinite volume, the 
scattering amplitude should have a pole at $s=M^2$, with the corresponding 
binding momentum $k_B\equiv i\kappa$, $\kappa>0$, given by the equation
\begin{equation}
  \label{inf-vol-pole-eq}
  \psi(k_B^2)+\kappa = -8\pi M\Big[V^{-1}(M^2) - G(M^2)\Big] = 0\,,
\end{equation}
with $\psi(k^2)$ the analytic continuation of $k\cot\delta(k)$ for
arbitrary complex values of $k^2$. From this, it is straightforward
to evaluate the pole position shift,
\begin{eqnarray}
  \label{mass-shift}
\kappa_L - \kappa &=& \frac1{1-2\kappa\psi'(k_B^2)} \left[ -8\pi M_L \Delta
G_L^{\bm{\theta}} (M_L^2)\right. \nonumber\\
&& \qquad\qquad\qquad ~~~ + \left. \psi'(k_B^2)(\kappa_L-\kappa)^2 \right],
\end{eqnarray}
where the prime denotes differentiation with respect to $k^2$. 
This equation gives the bound state pole position $\kappa_L$ (and thus
the finite volume mass $M_L$) as a function of the infinite-volume 
parameters $g^2$ and $\kappa$. Having determined these parameters
from the bound state levels $\kappa_L$, one is then able to determine
the wave function renormalization constant $Z$ in the infinite volume.
In Ref.~\cite{Agadjanov:2014ana}, this procedure is scrutinized using
synthetic lattice data, for a simple toy model and a molecular model
for the charm scalar meson  $D_{s0}^*(2317)$. An important finding of this
work is that the extraction of $Z$ is facilitated by using twisted
boundary conditions, measuring the  dependence of the spectrum on the 
twist angle. Also, the limitations of this approach are discussed in 
detail. It remains to be seen how useful this method is for real 
lattice data. For related papers, also making use of twisted boundary
conditions to explore the nature of states, see e.g.
Refs.~\cite{Ozaki:2012ce,Briceno:2013hya,Korber:2015rce}.
A different approach to quantify compositeness in a finite volume has
recently been given in Ref.~\cite{Tsuchida:2017gpb}. Using this 
method, the $\bar KN$ component of the $\Lambda(1405)$
was found to be 58\%,  and the $\Sigma\pi$ and other components
also contribute to its structure. This is interpreted as a reflection
of the two-pole scenario of the  $\Lambda(1405)$.

\subsection{Lattice QCD results on the charm-strange mesons
and \texorpdfstring{$\bm{XYZ}$~}~states}
\label{sec:latres}

There have been quite a few studies of the charm-strange mesons and
some of the $XYZ$ states in lattice QCD. However, there are very few 
conclusive results at present, so we expect that this section will be 
outdated most quickly.

Let us consider first the charm-strange mesons. A pioneering lattice study of
the low-energy interaction between a light pseudoscalar meson and a charmed
pseudoscalar meson was presented in Ref.~\cite{Liu:2012zya}.
The scattering lengths of the five channels $D\bar K(I=0)$, $D\bar K(I=1)$, $D_s
K$, $D\pi(I=3/2)$ and $D_s\pi$ were calculated based on four ensembles with pion
masses of 301, 364, 511 and 617~MeV. These channels are free of contributions
from disconnected diagrams.
SU(3) UCHPT as developed in Ref.~\cite{Guo:2009ct} was used to perform the
chiral extrapolation. The LECs of the chiral Lagrangian were determined from a
fit to the lattice results. With the same set of LECs and the masses of the
involved mesons set to their physical values,  predictions for the other
channels including $DK(I=0)$, $DK(I=1)$, $D\pi(I=1/2)$ and $D_s\bar K$ were
made. In particular, it was found that the attractive interaction in the
$DK(I=0)$ channel is strong enough so that a pole is generated in the unitarized
scattering amplitude.  Within $1\sigma$ uncertainties of the parameters, the
pole is at $2315^{+18}_{-28}$~MeV, and it is always below the $DK$ threshold.
From calculating the wave function normalization constant, it was found that
this pole is mainly an $S$-wave $DK$ bound state (the pertinent scattering
length being close to $-1$~fm as predicted in~\cite{Guo:2009ct} for such a
molecular state using Eq.~\eqref{eq:arwein}).
Further, a much sharper prediction of the isospin breaking  decay width of the
$D_{s0}^*(2317)\to D_s\pi$ could  be given
\begin{equation}
\label{Eq:DecayWidth}
 \Gamma(D_{s0}^*(2317)\to D_s \pi) = (133\pm22)~{\rm keV}~,
\end{equation}
to be contrasted with the molecular prediction without lattice
data, $\Gamma(D_{s0}^*(2317)\to D_s \pi)=(180\pm110)$~keV~\cite{Guo:2008gp},
and typical quark model predictions for a $c\bar s$ charm scalar meson of the 
order of 10~keV, see 
e.g. Refs.~\cite{Godfrey:2003kg,Faessler:2007gv}. For a similar study 
using a covariant UCHPT instead
of the heavy-baryon formalism, see Ref.~\cite{Altenbuchinger:2013vwa}.

A systematic study of the charm scalar and axial mesons at lighter
pion masses ($M_\pi = 156$ and $266$~MeV) was performed in 
Refs.~\cite{Mohler:2012na,Mohler:2013rwa,Lang:2014yfa}. These data
were later reanalyzed with the help of finite-volume 
UCHPT~\cite{Torres:2014vna}. Most notably, the $DK$ scattering with 
$J^P=0^+$ was investigated in \cite{Mohler:2013rwa}, using $DK$ as well
as $c\bar s$ interpolating fields. Clear evidence of a bound state below
the $DK$ threshold was found and the corresponding scattering length
was $a_0 = -1.33(20)\,$fm, consistent with the molecular scenario.
The analysis of Ref.~\cite{Torres:2014vna} found a 70\% $DK$ ($D^*K$) 
component in the $D_{s0}^*(2317)\,(D_{s1}(2460))$ state.

The most systematic study in the coupled-channel $D\pi, D\eta$ and $D_s\bar K$
system with isospin $1/2$ and $3/2$ was reported in Ref.~\cite{Moir:2016srx}.
Using a large basis of quark-antiquark and meson-meson basis states, the
finite volume energy spectrum could be calculated to high precision, allowing
for the extraction of the scattering amplitudes in the $S$-, $P$- and $D$-waves.
With the help of the coupled-channel L\"uscher formalism and various
parameterizations of the $T$-matrix, three poles were found in the complex 
plane: a 
$J^P=0^+$ near-threshold bound state, $M_S=(2275.9\pm 0.9)\,$MeV,
with a large coupling to $D\pi$,
a deeply bound $J^P=1^-$ state,   $M_P=(2009\pm 2)\,$MeV,
and evidence for a $J^P=2^+$ narrow 
resonance coupled predominantly to$D\pi$,  $M_D=(2527\pm 3)\,$MeV.
An interesting observation was made in Ref.~\cite{Albaladejo:2016lbb}.
Using UCHPT, it was shown that there are in fact two 
($I=1/2, J^P=0^+$) poles  in the region of the $D_0^*(2400)$ in 
the coupled-channel $D\pi, D\eta, 
D_s\bar K$ scattering amplitudes. {They couple differently to the involved
channels and thus should be understood as two states.} Having all the parameters
fixed from earlier studies in Ref.~\cite{Liu:2012zya}, the energy levels for the coupled-channel system in a 
finite volume were predicted. These agree remarkably well with the
lattice QCD results in~\cite{Moir:2016srx}. The intricate interplay of 
close-by thresholds and resonance poles already pointed out 
in~\cite{Doring:2013glu} is also found, and it is stressed that more
high-statistics data are needed to better determine the higher mass pole.

We now turn to the $XYZ$ states. Consider first the $X(3872)$. There
have been a number of studies using diquark-diquark or tetraquark
interpolating fields over the years, but none of these has been 
conclusive, see e.g.~\cite{Chiu:2006hd,Yang:2012mya}. 
Evidence for a bound state with $J^{PC}=1^{++}$
$(11\pm7)$~MeV below the $D\bar D^*$ threshold was reported in 
Ref.~\cite{Prelovsek:2013cra}. 
This establishes a candidate for the  $X(3872)$ in addition to the
near-by scattering states  $D\bar D^*$ and $J/\psi \rho$. This computation
was performed at $M_\pi =266\,$MeV but in a small volume $L\simeq 2\,$fm. 
This finding was validated using the Highly Improved Staggered Quark 
action~\cite{Lee:2014uta}. Finally, a refined study allowing also
for the mixing of tetraquark interpolators with $\bar c c$ components
was presented in~\cite{Padmanath:2015era}.
A candidate for the $X(3872)$ with $I = 0$ is observed very 
close to the  experimental state only if both $\bar cc$ and $D\bar D^*$ 
interpolators are included. However, the candidate is 
not found if diquark-antidiquark and $D\bar D^*$ are used in the 
absence of $\bar c c$. Note that in 
Refs.~\cite{Jansen:2013cba,Garzon:2013uwa,Jansen:2015lha,Baru:2015tfa}
strategies for extracting the properties of the $X(3872)$ from 
finite-volume data (at unphysical quark masses) have been worked out.

Consider next the $Z_c(3900)$. Various lattice calculations have
been performed, which, however, did not lead to  conclusive results,
see e.g. 
Refs.~\cite{Prelovsek:2013xba,Prelovsek:2014swa,Chen:2014afa,
Ikeda:2016zwx}. 
For example, in the most recent work \cite{Ikeda:2016zwx} it
was argued that this state is most probably a threshold cusp. Also,
a systematic analysis of most of these data using a finite volume version of 
the framework in Ref.~\cite{Albaladejo:2015lob} did not allow
for a definite conclusion
on the nature of the $Z_c(3900)$~\cite{Albaladejo:2016jsg}.

The Chinese Lattice QCD Collaboration has also studied $D^* \bar D_1$ 
\cite{Meng:2009qt,Chen:2016lkl} and $D^*\bar D$ scattering 
\cite{Chen:2015jwa}
with the aim of investigating the structure
of the $Z_c(4430)$ and $Z_c(4025)$, respectively.  These studies were
mostly exploratory and no definite statements could be drawn.

\subsection{Lattice QCD results on hadrons built from light quarks}
\label{sec:latres2}

Here we summarize briefly some very recent results on hadrons made 
entirely of light $u,d,s$ quarks, more precisely, the scalar mesons 
$f_0(500)$  and $a_0(980)$ as well as $\Lambda(1405)$.

The  first determination of the energy dependence of the isoscalar $\pi\pi$
elastic scattering phase shift and the extraction of the $f_0(500)$ based on
dynamical QCD using the methods described above was given by the Hadron Spectrum
Collaboration in Ref.~ \cite{Briceno:2016mjc}.
From the volume dependence of the spectrum the $S$-wave phase shift up to the
$K\bar K$ threshold could be extracted. The calculations were performed at pion
masses of 236 and 391~MeV. The resulting amplitudes are described in terms of a
scalar meson which evolves from a bound state below the $\pi\pi$ threshold at
the heavier quark mass to a broad resonance at the lighter quark mass.
{This is } in line with the prediction of Ref.~\cite{Hanhart:2008mx} based
on UCHPT to one loop.
 Earlier, the same collaboration had analyzed the coupled channel
$\pi\eta, K\bar K, \pi\eta'$ system with isospin $I=1$ and extracted properties
of the $a_0(980)$ meson~\cite{Dudek:2016cru}. {The model-independent
lattice data on energy levels were reanalyzed using UCHPT in
Refs.~\cite{Guo:2016zep,Doring:2016bdr}.} In particular, Ref.~\cite{Guo:2016zep}
pointed out some ambiguities in the $I=1$ solution.


There have been quite a few studies of the $\Lambda(1405)$ as a simple
three-quark baryon state by various lattice collaborations. In view of the
intricacies of the coupled channel $K^- p$ scattering discussed earlier,
we will not further consider these as coupled-channel effects must be 
considered.
An exception is the analysis of Ref.~\cite{Hall:2014uca}  based on 
the PCAS-CS ensembles~\cite{Aoki:2008sm} 
with  three-quark sources allowing for scalar and vector diquark configurations
that leads to the vanishing of the strange magnetic form factor of the 
$\Lambda(1405)$
at the physical pion mass. It is argued that this can only happen if the
$\Lambda(1405)$ is mostly an antikaon-nucleon molecule. This is further 
validated
by applying a finite-volume Hamiltonian approach to the measured energy 
levels~\cite{Wu:2014vma}. This lattice QCD result appears to be at odds with the
accepted two-pole scenario. However, as pointed out in the UCHPT analysis of 
Ref.~\cite{Molina:2015uqp}, these results exhibit some shortcomings. It is
argued in that work, that what is really discussed in \cite{Hall:2014uca} is 
the heavier of the two poles. In particular the complete absence of the 
$\pi \Sigma$ threshold in these data is discussed, as this threshold  would couple
stronger to the lighter pole. This effect is presumably due to the neglect of 
the baryon-meson interpolating fields in Ref.~\cite{Hall:2014uca}. The required
operators are also specified in~\cite{Molina:2015uqp}. It will be interesting to see
lattice QCD {calculations} including all the relevant channels and required
interpolating fields. {We also point out that
better methods to calculate the matrix elements of unstable states has been
given in~\cite{Bernard:2012bi,Briceno:2015tza}.}
