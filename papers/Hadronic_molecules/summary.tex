
\section{Summary and outlook}
\label{sec:sum}


In this review, we have discussed the experimental indications for and
the theoretical approaches to hadronic molecules, which are a particular
manifestation of non-conventional states in the spectrum of QCD. The 
observation that these multi-hadron bound states appear close to or in 
between two-particle thresholds allows one to write down nonrelativistic
effective field theories. This gives a systematic access to the production,
decay processes and other reactions involving hadronic molecules. In the last
decade or so, through precise measurements of the spectrum of QCD invloving 
charm and bottom quarks, more and more potential hadronic molecules have
been observed. We have shown how explicit calculations of various decay modes
can be used to test this scenario. This is the only way to eventually
disentangle hadronic molecules from other multi-quark states like, e.g., 
tetraquarks. More detailed and accurate measurements are therefore called
for, complemented by first-principle lattice QCD calculations with
parameters close to the physical point and accounting for the 
involved coupled-channel dynamics related. More than 60 years after
Weinberg's groundbreaking work on the question whether the deuteron is an
elementary particle, we are now in the position to identify many more of such
loosely bound states in the spectrum of QCD and to obtain a deeper 
understanding of the mechanism underlying the appearance and binding 
of hadronic molecules.
