
\section{Introduction}
\label{sec:1}


With the discovery of the deuterium in 1931 and the neutron in 1932, the first
bound state of two hadrons, {\sl i.e.}, the deuteron composed of one proton and
one neutron, became known.  The deuteron is very shallowly bound, by a mere MeV
per nucleon, i.e. it is located just below the neutron-proton continuum
threshold. Furthermore, it has a sizeable spatial extension. These two features
can be used for defining a hadronic molecule. A more precise definition will be
given in the course of this review.

Then the first meson, the pion, as the carrier particle of the nuclear force
proposed in 1935 by  Yukawa was discovered in 1947, followed by the discovery of
a second meson, the kaon, in the same year.
Since then, many different hadrons have been observed. Naturally hadronic
molecules other than the deuteron have been expected.  The first identified
meson-baryon molecule, {\sl i.e.}, the $\Lambda(1405)$ resonance composed of one
kaon and one nucleon, was predicted by Dalitz and Tuan in
1959~\cite{Dalitz:1959dn} and observed in the hydrogen bubble chamber at
Berkeley in 1961~\cite{Alston:1961zzd} several years before the  quark model was
proposed.
With the quark model developed in the early 1960s, it became clear that  hadrons
are not elementary particles, but composed of quarks and antiquarks. In the
classical quark model, a baryon is composed of three quarks and a meson is
composed of one quark and one antiquark.
In this picture, the $\Lambda(1405)$ resonance would be an excited state of a
three-quark ($uds$) system with one quark in an orbital $P$-wave excitation. Ten
years later, the theory of the strong interactions, Quantum Chromodynamics
(QCD), was proposed to describe the interactions between quarks as well as
gluons. The gluons  are the force carriers of the theory that also exhibit
self-interactions due to the non-abelian nature of the underlying gauge group,
SU(3)$_C$, where $C$ denotes the color degree of freedom.
In QCD the basic constituents of the hadrons are both quarks and gluons.
Therefore, the structure of hadrons is more complicated than the classical quark
model allows. There may be glueballs (which contain only valence gluons),
hybrids (which contain valence quarks as well as gluons) and multiquark states
(such as tetraquarks or pentaquarks). Note, however, that in principle the quark
model also allows for certain types of multiquark states~\cite{GellMann:1964nj}.

While the classical quark model is very successful in explaining properties of
the spatial ground states of the flavor SU(3) vector meson nonet, baryon octet
and decuplet, it fails badly even for the lowest spatial excited states in both
meson and baryon sectors.

In the meson sector, the lowest spatial excited SU(3) nonet is supposed to be
the lowest scalar nonet which includes the $f_0(500)$, the $\kappa(800)$, the
$a_0(980)$ and the $f_0(980)$. In the classical constituent quark model, these
scalars should be $q\bar q~(L=1)$ states, where $L$ denotes the orbital angular
momentum, with the $f_0(500)$ as an  $(u\bar u+d\bar d)/\sqrt{2}$ state, the
$a_0^0(980)$ as an $(u\bar u-d\bar d)/\sqrt{2}$ state and the $f_0(980)$ as
mainly an $s\bar s$ state. This picture, however, fails to explain why the mass
of the $a_0(980)$ is degenerate with the $f_0(980)$ instead of being close to
the$f_0(500)$,
 as it is the case of the $\rho$ and the $\omega$ in the vector nonet. Instead,
this kind of mass pattern can be easily understood in the tetraquark
picture~\cite{Jaffe:1976ig} or in a scenario where these states are dynamically
generated from the meson-meson
interaction~\cite{Weinstein:1982gc,Janssen:1994wn,Oller:2000ma}, with the
$f_0(980)$ and the $a_0(980)$ coupling strongly to the $\bar KK$ channel with
isospin 0 and 1, respectively.

In the baryon sector, a similar phenomenon seems also to be
happening~\cite{Zou:2007mk}.
In the classical quark model, the lowest spatial excited baryon is expected to
be a ($uud$) $N^*$ state with one quark in an orbital angular momentum $L=1$
state to have spin-parity $1/2^-$.
However, experimentally, the lowest negative parity $N^*$ resonance is found to
be the $N^*(1535)$, which is heavier than two other spatial excited baryons:
the $\Lambda^*(1405)$ and the $N^*(1440)$. This is the long-standing mass
reversal problem for the lowest spatial excited baryons. Furthermore, it is also
difficult to understand the strange decay properties of the $N^*(1535)$, which
seems to couple strongly to the final states with strangeness~\cite{Liu:2005pm},
as well as the strange decay pattern of another member of the $1/2^-$-nonet, the
$\Lambda^*(1670)$, which has a coupling to $\Lambda\eta$ much larger than to
$NK$ and $\Sigma\pi$ according to its branching ratios listed in the tables in
the Review of Particle Physics by the Particle Data Group
(PDG)~\cite{Olive:2016xmw}.
All these difficulties can be easily understood by assuming large five-quark
components in them~\cite{Zou:2007mk,Liu:2005pm,Helminen:2000jb} or considering
them to be dynamically generated  meson-baryon
states~\cite{Oller:2000ma,Kaiser:1995eg,Oset:1997it,Oller:2000fj,Inoue:2001ip,
GarciaRecio:2003ks,Hyodo:2002pk,Magas:2005vu,Huang:2007zza,Bruns:2010sv}.

No matter which configurations are realized in multiquark states, such as
colored diquark correlations or colorless hadronic clusters, the mass and decay
patterns for the lowest meson and baryon nonets strongly suggest that one must
go beyond the classical, so-called quenched, quark model. The unquenched picture
has been further supported by more examples of higher excited states in the
light quark sector, such as the $f_1(1420)$ as a $K^*\bar K$
molecule~\cite{Tornqvist:1993ng}, and by many newly observed states with heavy
quarks in the first decade of the new century, such as the $D^*_{s0}(2317)$ as a
$DK$ molecule or tetraquark state, $X(3872)$ as $D^*\bar D$ molecule or
tetraquark state~\cite{Chen:2016qju}. In fact, the possible existence of
hadronic molecules composed of two charmed mesons was already proposed 40 years
ago by Voloshin and Okun~\cite{Voloshin:1976ap} and supported by T{\"o}rnqvist
later within a one-pion exchange model~\cite{Tornqvist:1993ng}.

However, although many hadron resonances were proposed to be dynamically
generated states from various hadron-hadron interactions or multiquark states,
most of them cannot be clearly distinguished from classical quark model states
due to tunable ingredients and possible large mixing of various configurations
in these models.
A nice example is the already mentioned $\Lambda(1405)$.
Until 2010, {\sl i.e.}, 40 years after it  was predicted and observed as the
$\bar KN$ molecule, the PDG~\cite{Nakamura:2010zzi} still claimed that ``the
clean $\Lambda_c$ spectrum has in fact been taken to settle the decades-long
discussion about the nature of the $\Lambda(1405)$ --- true 3-quark state or
mere $\bar KN$ threshold effect? --- unambiguously in favor of the first
interpretation."  Only after many delicate analyses of various relevant
processes, the PDG~\cite{Olive:2016xmw} now acknowledges the two-pole structure
of the $\Lambda(1405)$ \cite{Oller:2000fj} and thus a dynamical generation is
most probable.

One way to unambiguously identify a multiquark state (including hadronic
molecular configurations) is the observation of resonances decaying into a heavy
quarkonium plus a meson with nonzero isospin made of light quarks or plus a
baryon made of light quarks. Since 2008, several such states have been claimed,
six $Z_c$ states, two $Z_b$ states and two $P_c$ states --- details on the
experimental situation are given in the next section. Among these newly claimed
states, the two $P_c$ states are quite close to the predicted hadronic molecular
states~\cite{Wu:2010jy,Wang:2011rga,Yang:2011wz,Xiao:2013yca}. However, many of
those states are challenged by some proposed kinematic explanations, such as threshold cusp
effects~\cite{Bugg:2011jr,Swanson:2014tra}, triangle singularity
effects~\cite{Chen:2013coa,Wang:2013cya,Guo:2015umn}, etc.
Some of these claims were challenged in Ref.~\cite{Guo:2014iya} where strong
support is presented that at least some of the signals indeed refer to
$S$-matrix poles.

Further experimental as well as theoretical studies are necessary to settle the
question which of the claimed states indeed exist.
Nevertheless the observation of at least some of these new states opens a new
window for the study of multiquark dynamics. Together with many other newly
observed states in the heavy quarkonium sector, they led to a renaissance of
hadron spectroscopy.
Among various explanations of the internal structure of these excitations,
hadronic molecules, being analogues of the deuteron, play a unique role since
for those states predictions can be made with controlled uncertainty, especially
for the states with one of or both hadrons containing heavy quark(s).
In fact most of these observed exotic candidates are indeed closely related to
open flavor $S$-wave thresholds. To study these hadronic molecules, both
nonrelativistic effective field theories and pertinent lattice QCD calculation
are the suitable frameworks. Especially,  Weinberg's famous compositeness
criterion~\cite{Weinberg:1962hj,Weinberg:1963zza} (and extensions thereof),
which pinned down the nature of the deuteron as a proton-neutron bound state, is
applicable here. The pole location in the corresponding hadron-hadron scattering
$S$-matrix could also shed light on the nature of the resonances as extended
hadronic molecules or compact states.

The revival of hadron spectroscopy is also reflected in a number of review
articles.
A few years ago, Klempt and his collaborators have given two broad reviews on
exotic mesons~\cite{Klempt:2007cp} and baryons~\cite{Klempt:2009pi}.
Other more recent pertinent reviews
include~\cite{Brambilla:2010cs,Olsen:2014qna,Oset:2016lyh,Chen:2016qju,
Chen:2016spr,Esposito:2016noz, Lebed:2016hpi,Hosaka:2016pey,Dong:2017gaw,Olsen:2017bmm}. Among
various theoretical models for these new hadrons, we
mainly cite those focusing on hadronic molecules and refer the interesting
readers to the above mentioned comprehensive reviews for more references on
other models.

This paper is organized as follows: In Sec.~\ref{sec:2}, we discuss the
experimental evidences for states that could possibly be hadronic molecules. In
Sec.~\ref{sec:3}, after a short review of the basic $S$-matrix properties, we
give a general definition of hadronic molecules and discuss related aspects.
Then, in Sec.~\ref{sec:4}, nonrelativistic effective field theories tailored to
investigate hadronic molecules are formulated, followed by a brief discussion of
hadronic molecules in lattice QCD in Sec.~\ref{sec:lattice}. Sec.~\ref{sec:6} is
devoted to the discussion of phenomenological manifestations of hadronic
molecules, with a particular emphasis on clarifying certain statements from the
literature that have been used to dismiss certain states as possible hadronic
molecules. We end with a short summary and outlook in Sec.~\ref{sec:sum}.  We
mention that this field is very active, and thus only references that appeared
before April 2017 are included.




