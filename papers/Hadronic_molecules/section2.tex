
\section{Candidates of hadronic molecules --- experimental evidences}
\label{sec:2}

\begin{table*}[tbh]
 \caption{Mesons that contain at most one heavy quark that cannot be easily
 accommodated in the $q\bar q$ quark model.
  Their quantum numbers $I^G(J^{PC})$, masses, widths,
 the nearby $S$-wave thresholds, $m_{\text{threshold}}$, where we add in 
 brackets $M-m_{\text{threshold}}$, and the observed decay modes are listed in 
 order. The data without references are taken from the 2016 edition of the 
 Review of Particle Physics~\cite{Olive:2016xmw}.  
 }
 \begin{ruledtabular}
  \begin{tabular}{l c c c c c}
State & $I^G(J^{PC})$ & $M \ [\mathrm{MeV}]$ & $\Gamma[\mathrm{MeV}]$ & $S$-wave
threshold(s) [MeV] & Decay mode(s) [branching ratio(s)]\tabularnewline
\hline
$f_{0}(500)$~\cite{Pelaez:2015qba}~\footnote{The mass and width
are derived from the pole position quoted in Ref.~\cite{Pelaez:2015qba} via
$\sqrt{s_p}=M-i\Gamma/2$. } & $0^{+}(0^{++})$ & $449^{+22}_{-16}$ & $550\pm 24$
& $\pi\pi(173^{+22}_{-16})$ & $\pi\pi$ [dominant]\tabularnewline &  &  &  &  &
$\gamma\gamma$\tabularnewline
\hline
$\kappa(800)$ & $\frac{1}{2}(0^{+})$ & $682\pm29$ & $547\pm24$ & $K\pi(48\pm 29)$ & $\pi K$\tabularnewline
\hline
$f_{0}(980)$ & $0^{+}(0^{++})$ & $990\pm20$ & $10\sim100$ & $K^+K^-(3\pm20)$ &
$\pi\pi$ [dominant]\tabularnewline &  &  &  & $K^0\bar{K}^0(-5\pm20)$ &
$K\bar{K}$\tabularnewline &  &  &  &  & $\gamma\gamma$\tabularnewline
\hline
$a_{0}(980)$ & $1^{-}(0^{++})$ & $980\pm20$ & $50\sim100$ & $K\bar{K}(-11\pm20)$
& $\eta\pi$ [dominant]\tabularnewline &  &  &  &  & $K\bar{K}$\tabularnewline
 &  &  &  &  & $\gamma\gamma$\tabularnewline
\hline
$f_{1}(1420)$ & $0^{+}(1^{++})$ & $1426.4\pm0.9$ & $54.9\pm2.6$ & $K\bar{K}^{*}(39.1\pm0.9)$ & $K\bar{K}^{*}$(dominant)\tabularnewline
 &  &  &  &  & $\eta\pi\pi$ [possibly seen]\tabularnewline
 &  &  &  &  & $\phi\gamma$\tabularnewline
\hline
$a_{1}(1420)$ & $1^{-}(1^{++})$ & $1414_{-13}^{+15}$ & $153_{-23}^{+8}$ & $K\bar{K}^{*}(27^{+15}_{-13}$) &
$f_{0}(980)\pi$ [seen]\tabularnewline
\hline
$X(1835)$ & $?^{?}(0^{-+})$ & $1835.8_{-3.2}^{+4.0}$ & $112\pm40$ & $p\bar{p}(-40.7^{+4.0}_{-3.2})$ & $p\bar{p}$\tabularnewline
 &  &  &  &  & $\eta^{\prime}\pi\pi$\tabularnewline
 &  &  &  &  & $K_{S}^{0}K_{S}^{0}\eta$\tabularnewline
\hline
$D_{s0}^{*}(2317)^{+}$ & $0(0^{+})$ & $2317.7\pm0.6$ & $<3.8$ & $DK(-45.1\pm 
0.6)$ & $ D_{s}^{+}\pi^{0}$\tabularnewline
\hline
$D_{s1}(2460)^{+}$ & $0(1^{+})$ & $2459.5\pm0.6$ & $<3.5$ & $D^{*}K(-44.7\pm 
0.6)$ & $D_{s}^{*+}\pi^{0}\,[(48\pm 11)\%]$\tabularnewline
 &  &  &  &  & $D_{s}^{+}\gamma\, [(18\pm 4)\%]$\tabularnewline
 &  &  &  &  & $D_{s}^{+}\pi^{+}\pi^{-}\, [(4\pm 1) \%]$\tabularnewline
 &  &  &  &  & $D_{s0}^*(2317)^{+}\gamma\, [(4^{+5}_{-2})\%]$\tabularnewline
 \hline
$D_{s1}^{*}(2860)^{+}$ & $0(1^{-})$ & $2859 \pm 27$ & $159\pm 80 $ & 
$D_1(2420)K(-59 \pm 27)$ & $ DK$\tabularnewline
 &  &  &  &  & $D^*K$ \tabularnewline
\end{tabular}
\end{ruledtabular}
\label{tab:1}
\end{table*}


 \begin{table*}
 \caption{Same as Table~\ref{tab:1} but in the 
 charmonium and bottomonium sectors. A blank in the fifth column means that 
 there is no relevant nearby $S$-wave threshold. 
}
\begin{ruledtabular}
 \begin{tabular}{l  c c c c c }
State & $I^G(J^{PC})$ & $M\,[\mathrm{MeV}]$ & $\Gamma\,[\mathrm{MeV}]$ &
$S$-wave threshold(s) [$\mathrm{MeV}$] & Observed mode(s) (branching
ratios)\tabularnewline
\hline
$X(3872)$ & $0^{+}(1^{++})$ & $3871.69\pm0.17$ & $<1.2$ & $D^{*+}D^{-}+c.c.(-8.15\pm0.20)$ & $B\to K[\bar{D}^{*0}D^{0}](>24\%)$\tabularnewline
 &  &  &  & $D^{*0}\bar{D}^{0}+c.c.(0.00\pm0.18)$ & $B\to
 K[D^{0}\bar{D}^{0}\pi^{0}](>32\%)$\tabularnewline &  &  &  &  & $B\to K[J/\psi\pi^{+}\pi^{-}](>2.6\%)$\tabularnewline
 &  &  &  &  & $B\to K[J/\psi\pi^{+}\pi^{-}\pi^{0}]$\tabularnewline
 &  &  &  &  & $p\bar{p}\to[J/\psi\pi^{+}\pi^{-}]...$\tabularnewline
 &  &  &  &  & $pp\to[J/\psi\pi^{+}\pi^{-}]...$\tabularnewline
 &  &  &  &  & $B\to K[J/\psi\omega](>1.9\%)$\tabularnewline
 &  &  &  &  & $B\to[J/\psi\gamma](>6\times10^{-3})$\tabularnewline
 &  &  &  &  & $B\to[\psi(2S)\gamma](>3.0\%)$\tabularnewline
\hline
$X(3940)$ & $?^{?}(?^{??})$ & $3942.0\pm9$ & $37_{-17}^{+27}$ & $D^{*}\bar{D}^{*}(-75.1\pm 9)$ & $e^{+}e^{-}\to
J/\psi[D\bar{D}^{*}]$\tabularnewline
\hline
$X(4160)$ & $?^{?}(?^{??})$ & $4156_{-25}^{+29}$ & $139_{-60}^{+110}$ & $D^{*}\bar{D}^{*}(139_{-25}^{+29})$ & $e^{+}e^{-}\to
J/\psi[D^{*}\bar{D}^{*}]$\tabularnewline
\hline
$Z_{c}(3900)$ & $1^{+}(1^{+-})$ & $3886.6\pm2.4$ & $28.1\pm2.6$ & $D^{*}\bar{D}(10.8\pm2.4)$ &
$e^{+}e^{-}\to\pi[D\bar{D}^{*}+c.c.]$\tabularnewline
 &  &  &  &  & $e^{+}e^{-}\to\pi[J/\psi\pi]$\tabularnewline
\hline
$Z_{c}(4020)$ & $1(?^{?})$ & $4024.1\pm1.9$ & $13\pm5$ & $D^{*}\bar{D}^{*}(7.0\pm 2.4)$ & $e^{+}e^{-}\to\pi[D^{*}\bar{D}^{*}]$\tabularnewline
 &  &  &  &  & $e^{+}e^{-}\to\pi[h_{c}\pi]$\tabularnewline
 &  &  &  &  & $e^{+}e^{-}\to\pi[\psi'\pi]$\tabularnewline
\hline
$Y(4260)$ & $?^{?}(1^{--})$ & $4251\pm9$ & $120\pm12$ & 
$D_{1}\bar{D}+c.c.(-38.2\pm 9.1)$ & $e^{+}e^{-}\to 
J/\psi\pi\pi$\tabularnewline
 &  &  &  & $\chi_{c0}\omega(53.6\pm 9.0)$ & $e^{+}e^{-}\to
 \pi D\bar D^* +c.c.$\tabularnewline 
  &  &  &  &  & $e^{+}e^{-}\to \chi_{c0}\omega$\tabularnewline
 &  &  &  &  & $e^{+}e^{-}\to
 X(3872)\gamma$\tabularnewline
\hline
$Y(4360)$ & $?^{?}(1^{--})$ & $4346\pm6$ & $102\pm10$ & $D_{1}\bar{D}^{*}+c.c.(-85\pm6)$ &
$e^{+}e^{-}\to\psi(2S)\pi^{+}\pi^{-}$\tabularnewline
\hline
$Y(4660)$ & $?^{?}(1^{--})$ & $4643\pm9$ & $72\pm11$ & $\psi(2S)f_{0}(980)(-33\pm21)$ &
$e^{+}e^{-}\to\psi(2S)\pi^{+}\pi^{-}$\tabularnewline
 &  &  &  & $\Lambda_{c}^{+}\Lambda_{c}^{-}(70\pm 6)$ & \tabularnewline
\hline
$Z_c(4430)^{+}$ & $?(1^{+})$ & $4478_{-18}^{+15}$ & $181\pm31$ & 
$\psi(2S)\rho(17^{+15}_{-18})$ & $B\to
K[\psi(2S)\pi^{+}]$\tabularnewline
 &  &  &  &  & $B\to K[J/\psi\pi^{+}]$\tabularnewline
\hline
$Z_{c}(4200)^{+}$ & $?(1^{+})$ & $4196_{-32}^{+35}$ & $370_{-32}^{+100}$ &  & $\bar{B}^{0}\to K^{-}[J/\psi\pi^{+}]$\tabularnewline
\hline
$Z_{c}(4050)^{+}$ & $?(?^{?})$ & $4051_{-40}^{+24}$ & $82_{-28}^{+50}$ & $D^*\bar{D}^*(34^{+24}_{-40})$ & $\bar{B}^{0}\to K^{-}[\chi_{c1}\pi^{+}]$\tabularnewline
\hline
$Z_{c}(4250)^{+}$ & $?(?^{?})$ & $4248_{-50}^{+190}$ & $177_{-70}^{+320}$ & $\chi_{c1}\rho(-37^{+24}_{-50})$ & $\bar{B}^{0}\to K^{-}[\chi_{c1}\pi^{+}]$\tabularnewline
\hline
$X(4140)$\cite{Aaij:2016nsc,Aaij:2016iza} & $0^{+}(1^{++})$ & $4146.5\pm4.5_{-2.8}^{+4.6}$ & $83\pm21_{-14}^{+21}$ &
$D_{s}\bar D_{s}^{*}(-66.1_{-3.2}^{+4.9})$ & $B^{+}\to K^{+}[J/\psi\phi]$\tabularnewline
\hline
$X(4274)$\cite{Aaij:2016nsc,Aaij:2016iza} & $0^{+}(1^{++})$ & $4273.3\pm8.3_{-3.6}^{+17.2}$ & $56\pm11_{-11}^{+8}$ &
$D_{s}^{*}\bar D_{s}^{*}(-49.1_{-9.1}^{+19.1})$ & $B^{+}\to K^{+}[J/\psi\phi]$\tabularnewline
\hline
$X(4500)$\cite{Aaij:2016nsc,Aaij:2016iza} & $0^{+}(0^{++})$ & $4506\pm11_{-15}^{+12}$ & $92\pm21_{-20}^{+21}$ & $D_{s0}^*(2317)\bar D_{s0}^*(2317)(-129^{+16}_{-19})$ & $B^{+}\to
K^{+}[J/\psi\phi]$\tabularnewline
\hline
$X(4700)$\cite{Aaij:2016nsc,Aaij:2016iza} & $0^{+}(0^{++})$ & $4704\pm10_{-24}^{+14}$ & $120\pm31_{-33}^{+42}$ & $D_{s0}^*(2317)\bar D_{s0}^*(2317)(69^{+17}_{-26})$  & $B^{+}\to
K^{+}[J/\psi\phi]$\tabularnewline
\hline
$Z_{b}(10610)$ & $1^{+}(1^{+})$ & $10607.2\pm2.0$ & $18.4\pm2.4$ & $B\bar{B}^{*}+c.c.(4.0\pm3.2)$ &
$\Upsilon(10860)\to\pi[B\bar{B}^{*}+c.c.]$\tabularnewline
 &  &  &  &  & $\Upsilon(10860)\to\pi[\Upsilon(1S)\pi]$\tabularnewline
 &  &  &  &  & $\Upsilon(10860)\to\pi[\Upsilon(2S)\pi]$\tabularnewline
 &  &  &  &  & $\Upsilon(10860)\to\pi[\Upsilon(3S)\pi]$\tabularnewline
 &  &  &  &  & $\Upsilon(10860)\to\pi[h_{b}(1P)\pi]$\tabularnewline
 &  &  &  &  & $\Upsilon(10860)\to\pi[h_{b}(2P)\pi]$\tabularnewline
\hline
$Z_{b}(10650)$ & $1^{+}(1^{+})$ & $10652.2\pm1.5$ & $11.5\pm2.2$ & $B^{*}\bar{B}^{*}(2.9 \pm1.5)$ &
$\Upsilon(10860)\to\pi[B^{*}\bar{B}^{*}]$\tabularnewline
 &  &  &  &  & $\Upsilon(10860)\to\pi[\Upsilon(1S)\pi]$\tabularnewline
 &  &  &  &  & $\Upsilon(10860)\to\pi[\Upsilon(2S)\pi]$\tabularnewline
 &  &  &  &  & $\Upsilon(10860)\to\pi[\Upsilon(3S)\pi]$\tabularnewline
 &  &  &  &  & $\Upsilon(10860)\to\pi[h_{b}(1P)\pi]$\tabularnewline
 &  &  &  &  & $\Upsilon(10860)\to\pi[h_{b}(2P)\pi]$\tabularnewline
\end{tabular}
\end{ruledtabular}
\label{tab:2}
\end{table*}

In this section we briefly review what is known experimentally about some of the
most promising candidates for exotic states.
Already the fact that those are all located close to some two-hadron continuum
channels indicates that the two-hadron continuum is of relevance for their
existence. {We will show  that many of those states are located near
$S$-wave thresholds in both light and heavy hadron spectroscopy, which is
not only a natural property of hadronic molecules, which are QCD bound states
of two hadrons (a more
proper definition will be given in Sec.~\ref{sec:wein}), but also a
prerequisite for their identification as will be discussed in Sec.~\ref{sec:3}.} In the course of this
review, we will present other arguments why many of these states should be
considered as hadronic molecules and what additional experimental inputs are
needed to further confirm this assignment.



In Tables~\ref{tab:1} and ~\ref{tab:2} we present the current status for exotic
candidates in the meson sector. Exotic candidates in the baryon sector are
listed later in Tab.~\ref{tab:baryon}. Besides the standard properties we also
quote for each state the nearest relevant $S$-wave threshold  as well as its
distance to that threshold.
Note that only thresholds of narrow states are quoted since these are the only
ones of relevance here~\cite{Guo:2011dd}. Otherwise, the bound system would also
be broad~\cite{Filin:2010se}. In addition, as a result of the centrifugal
barrier one expects that if hadronic molecules exist, they should first of all
appear in the $S$-wave which is why in this review we do not consider $P$- or
higher partial waves although there is no principle reason for the non-existence
of molecular states in the $P$-wave.


\subsection{Light mesons}
\label{sec:2-lightmesons}


\subsubsection{Scalars below 1 GeV }


The lowest $S$-wave two-particle thresholds in the hadron sector are those for
 two pseudoscalar mesons, $\pi\pi$, $\pi K$, $\eta\pi$, and $K\bar{K}$.
Those channels carry scalar quantum numbers. The pion pair is either in an
isoscalar or an isotensor state, and the isovector state is necessarily in a
$P$-wave. It turns out that there is neither a resonant structure in the
isotensor $\pi\pi$ nor in the isospin $3/2$ $\pi K$ $S$-wave, however, there are
resonances observed experimentally in all other channels. According to the
conventional quark model, a scalar meson of $q\bar{q}$ with $J^P=0^+$ carries
one unit of orbital angular momentum. Thus, the mass range of the  lowest
scalars is expected to be higher than the lowest pseudoscalars or vectors of
which the orbital angular momentum is zero. However, the lightest scalars have
masses below those of the lightest vectors. Moreover, the mass ordering of the
lightest scalars apparently violates the pattern of other $q\bar{q}$ nonets:
Instead of having the isovectors to be the lowest states, the isovector
$a_0(980)$ states are almost degenerate with one of the isoscalar states,
$f_0(980)$, and those are the heaviest states in the nonet. The other isoscalar,
$f_0(500)$, also known as $\sigma$, has the lightest mass of the multiplet and
an extremely large width. The strange scalar $K_0^*(800)$, also known as
$\kappa$, has a  large width as well. All these indicate some nontrivial
substructure beyond a simple $q\bar{q}$ description.

The mass ordering of these lightest scalars is seen as a strong evidence for the
tetraquark scenario proposed by Jaffe in the
1970s~\cite{Jaffe:1976ig,Jaffe:1976ih}. Meanwhile, they can also be described
as dynamically generated states through meson-meson
scatterings~\cite{Pennington:1973xv,Au:1986vs,Morgan:1993td,Pelaez:2015qba}.
For a theoretical understanding of the $f_0(500)$ pole it is crucial to
recognize that as a consequence of the chiral symmetry of QCD the scalar isoscalar $\pi
\pi$ interaction is proportional to $(2s-M_\pi^2)/F_\pi^2$ at the leading order
(LO) in the chiral expansion. Here, $M_\pi (F_\pi)$ denotes the pion mass (decay
constant).
As a result, the LO scattering amplitude has already hit the
unitarity bound for moderate energies necessitating some type of unitarization,
which at the same time generates  a resonance-like
structure~\cite{Meissner:1990kz}.
This observation, deeply nested in the symmetries of QCD, has indicated the
significance of the $\pi\pi$ interaction for the light scalar mesons. The
history and the modern developments regarding the $f_0(500)$ was recently very
nicely reviewed in~\cite{Pelaez:2015qba}.
Similar to the isoscalar scalar $f_0(500)$ generated from the $\pi\pi$
scattering, the whole light scalar nonet appears naturally from properly unitarized chiral
amplitudes for pseudoscalar-pseudoscalar
scatterings~\cite{Oller:1997ng,Oller:1998hw,GomezNicola:2001as}. Similar
conclusions also follow from more phenomenological
studies~\cite{Weinstein:1990gu,Janssen:1994wn}.
One of the most interesting observations about $a_0(980)$ and $f_0(980)$ is that
their masses are almost exactly located at the $K\bar{K}$ threshold. The
closeness of the $K\bar{K}$ threshold to $a_0(980)$ and $f_0(980)$ and their
strong $S$-wave couplings makes both states good candidates for $\bar KK$
molecular states~\cite{Weinstein:1990gu,Baru:2003qq}.



\subsubsection{Axial vectors $f_1(1420)$, $a_1(1420)$ and implications of the 
triangle singularity}
\label{sec:TS}

The $S$-wave pseudoscalar meson pair scatterings can be extended to $S$-wave
pseudoscalar-vector scatterings and vector-vector scatterings where again
dynamically generated states can be investigated. The $S$-wave
pseudoscalar-vector scatterings can access the quantum numbers $J^P=1^+$, while
the vector-vector scatterings give $J^P=0^+$, $1^+$ and $2^+$.
This suggests that some of the states with those quantum numbers can be affected
by the $S$-wave open thresholds if their masses are close enough to the
thresholds. Or, it might be possible that such scatterings can dynamically
generate states as discussed in the
literature~\cite{Lutz:2003fm,Roca:2005nm,Geng:2008gx}.
Note that not all states found in these studies survive once a more
sophisticated and realistic treatment as outlined in Ref.~\cite{Gulmez:2016scm}
is utilized.

In addition, the quark model also predicts regular $q\bar q$ states in the same
mass range such that it appears difficult to identify the most prominent
structure of the states.

Let us focus on the lowest $1^{++}$ mesons. Despite that these states could be
dynamically generated from the resummed chiral
interactions~\cite{Lutz:2003fm,Roca:2005nm}, there are various experimental
findings consistent with a usual $q\bar q$ nature of the members of the lightest
axial nonet, $f_1(1420)$, $f_1(1285)$, $a_1(1260)$, and
$K_{1A}(1^3P_1)$~\cite{Olive:2016xmw}.
 However, two recent experimental observations expose novel features
in their decay mechanisms which illustrate the relevance of their couplings to
the  two-meson continua. The BESIII Collaboration observed an anomalously large
isospin symmetry breaking in $\eta(1405)/\eta(1475)\to
3\pi$~\cite{BESIII:2012aa}, which could be accounted for by the so-called
triangle singularity (TS) mechanism as studied in Ref.~\cite{Wu:2011yx,Aceti:2012dj}.
This special threshold phenomenon {arises in triangle (three-point loop)
diagrams with special kinematics which} will be detailed in Sect.~\ref{sec:4-pc}.
Physically, it emerges when all the involved vertices in the triangle diagram
can be interpreted as classical processes. For it to happen, one necessary
condition is that all intermediate states in the triangle diagram,
$\bar{K}K^*(K)+c.c.$ for the example at hand, should be able to reach their
on-shell condition simultaneously. As a consequence, the $f_1(1420)$,
which is close to the $\bar K K^*$ threshold and couples to $\bar K K^*$ in an
$S$-wave as well, should also have large isospin violations in $f_1(1420)\to
3\pi$.
This contribution has not been included in the BESIII analysis~\cite{BESIII:2012aa}.  However, a detailed partial wave analysis
suggests the presence of the $f_1(1420)$ contribution via the TS
mechanism~\cite{Wu:2012pg}. Moreover, the TS mechanism predicts structures in
different $C$-parity and isospin (or $G$ parity) channels via the
$\bar{K}K^*(K)+c.c.$ triangle diagrams. The $f_1(1420)$ was speculated long time
ago to be a $\bar K^* K$ molecule from a dynamical study of the $K\bar K\pi$
three-body system~\cite{Longacre:1990uc}.

Apart from the $I=0$, $J^{PC}=1^{++}$ state $f_1(1420)$, one would expect that
the TS will cause enhancements in $I=1$ channels with $C=\pm$. It provides a
natural explanation for the newly observed
 $a_1(1420)$ by the COMPASS Collaboration~\cite{Adolph:2015pws} in $\pi^-p\to
\pi^-\pi^-\pi^+ p$ and $\pi^-\pi^0\pi^0 p$~\cite{Liu:2015taa,Ketzer:2015tqa}. It
should be noted that in
Refs.~\cite{Aceti:2016yeb,Cheng:2016hxi,Debastiani:2016xgg}  the $a_1(1420)$
enhancement is proposed to be caused by the $a_1(1260)$ together with the TS
mechanism and similarly $f_1(1420)$ is produced by $f_1(1285)$.
However, as shown by the convincing experimental data from MARK-III, BESII,
BESIII, and the detailed partial wave analysis of Ref.~\cite{Wu:2012pg}, the
$f_1(1420)$ matches the behavior of a genuine state in the $K\bar{K}\pi$ channel
that is distorted  in other channels by an interference with
 the TS.
This appears to be a more consistent picture to explain the existing data and
underlying mechanisms~\cite{Zhao:2017wey}. These issues are discussed further in
Sec.~\ref{sec:6}.


\subsection{{Open heavy-flavor mesons}} 



Since 2003, quite a few {open heavy-flavor} hadrons have been observed
experimentally. Some of them are consistent with the excited states predicted in
the potential quark model, while the others are not (for a recent review,
see~\cite{Chen:2016spr}). Particular interest has been paid to the
positive-parity charm-strange mesons $D_{s0}^*(2317)$ and $D_{s1}(2460)$ observed
in 2003 by the BaBar~\cite{Aubert:2003fg} and CLEO~\cite{Besson:2003cp}
Collaborations.
The masses of $D_{s0}^*(2317)$ and $D_{s1}(2460)$ are below the $DK$ and $D^*K$
thresholds, respectively, by about the same amount, only 45~MeV (see
Table~\ref{tab:1} and references therein), which makes them natural candidates
for hadronic molecules~\cite{Barnes:2003dj,vanBeveren:2003kd,Szczepaniak:2003vy,
Kolomeitsev:2003ac,Hofmann:2003je,
Guo:2006fu,Guo:2006rp,Gamermann:2006nm,Faessler:2007gv,Flynn:2007ki,
Cleven:2010aw,Wu:2011yb,Cleven:2014oka, Albaladejo:2016hae}, while also other
explanations such as $P$-wave $c\bar s$ states and tetraquarks exist in the
literature.
We will come back to the properties of these states occasionally in this review.
Here, we collect the features supporting the $DK/D^*K$ molecular hypothesis:
\begin{itemize}
  
\item Their masses are about $160~\mev$ and  $70~\mev$, respectively, below the
predicted $0^+$ and $1^+$ charm-strange mesons by the Godfrey--Isgur quark
model~\cite{Godfrey:1985xj,DiPierro:2001dwf}, making them not easy to be
accommodated by the conventional $c\bar{s}$ states.

\item {The mass difference between these two states is equal to the energy
difference between the corresponding $D^{(*)}K$  thresholds. 
This appears to be
 a natural
consequence in the hadronic molecular scenario, since the involved interactions
is approximately heavy quark spin symmetric~\cite{Guo:2009id}.}

\item {The small width of both $D_{s0}^*(2317)$ and $D_{s1}(2460)$
can only be understood if the are isoscalar states~\footnote{Negative result was reported in a search for
the isospin partner of the $D_{s0}^*(2317)$~\cite{Choi:2015lpc}}, for then,
since both of them are below the $DK/D^*K$  thresholds, the only possible
hadronic decay modes are the isovector channels $D_s^+\pi^0$ and
$D_s^{*+}\pi^0$, respectively.} The molecular nature together with the proximity
to the $DK/D^*K$ thresholds leads to a prediction for the width of the states
above 100~keV while other approaches give a width about
10~keV~\cite{Colangelo:2003vg,Godfrey:2003kg}.
These issues are discussed in detail in Sections~\ref{sec:latres}
and~\ref{sec:isospinviol}.

\item  Their radiative decays, {\sl i.e.}, $D_{s0}^*(2317)\to D_s \gamma$ and 
$D_{s1}(2460)\to D^{(*)}_s \gamma$, and production in $B$ decays proceed via
short-range interactions~\cite{Lutz:2007sk,Chen:2013upa,Cleven:2014oka}.
They are therefore insensitive to the molecular component of the states.

\item As will be discussed in Secs.~\ref{sec:wein} and \ref{sec:latres}, the
$DK$ scattering length extracted from LQCD {calculations}~\cite{Liu:2012zya} is
compatible with the result extracted in the molecular scenario for
$D_{s0}^*(2317)$ based on Weinberg's compositeness theorem.

\end{itemize}

The $D_{sJ}(2860)$ observed by the BaBar Collaboration~\cite{Aubert:2006mh} 
presents another example of an interesting charm-strange meson. It decays into 
both $DK$ and $D^*K$ with similar branching fractions~\cite{Olive:2016xmw}.  
One notices that the difference between the $D_{sJ}(2860)$ mass and the 
$D_1(2420)K$ threshold is similar to that between the $D_{s0}^*(2317)$ and 
$DK$. Assuming the $D_{s0}^*(2317)$ to be a $DK$ hadronic molecule, an $S$-wave 
$D_1(2420)K$ bound state with quantum numbers $J^P=1^-$ was predicted to have a 
mass $(2870\pm9)$~MeV, consistent with that of the $D_{sJ}(2860)$, 
in~\cite{Guo:2011dd}, where the ratio of its partial widths into the $DK$ and 
$D^*K$ also gets naturally explained. As a result of heavy quark spin symmetry, 
a $D_2(2460)K$ hadronic molecule with $J^P=2^-$ and a mass of around 2.91~GeV 
was predicted in~\cite{Guo:2011dd}.  A later analysis by the LHCb Collaboration 
suggests that this structure corresponds to two states: $D_{s1}^*(2860)$ with 
$J^P=1^-$ and $D_{s3}^*(2860)$ with $J^P=3^-$~\cite{Aaij:2014xza}. Regular 
$c\bar s$ interpretations for these two states have been nicely summarized 
in~\cite{Chen:2016spr}.

The most recently reported observation of an exotic singly-heavy meson candidate
is a narrow structure in the $B_s^0\pi^\pm$ invariant mass distribution, dubbed
as $X(5568)$, by the D0 Collaboration~\cite{D0:2016mwd}.
Were it a hadronic state, it would be an {isovector} meson containing four
different flavors of valence quarks $(\bar b s\bar u d)$. However, the peak is
located at only about 50~MeV above the $B_s\pi$ threshold. The existence of a
tetraquark, whether or not being a hadronic molecule, at such a low mass is
questioned from the quark model point of view in~\cite{Burns:2016gvy}, and, more
generally, from chiral symmetry and heavy quark flavor symmetry
in~\cite{Guo:2016nhb}. Both the LHCb~\cite{Aaij:2016iev} and
CMS~\cite{CMS:2016fvl} Collaborations quickly reported negative results on the
existence of $X(5568)$ in their data sets.
An alternative explanation for the $X(5568)$ observation is necessary. One
possibility is provided in Ref.~\cite{Yang:2016sws}. Because of these
controversial issues with the $X(5568)$, we will not discuss this structure any
further.


\subsection{Heavy quarkonium-like states: {$XYZ$}}

The possibility of hadronic molecules in the charmonium mass region was 
suggested in~\cite{Voloshin:1976ap,DeRujula:1976zlg} only a couple of years  
after the ``November Revolution'' due to the discovery of the $J/\psi$. Such 
an idea became popular after the discovery of the famous $\X$ by Belle in
2003~\cite{Choi:2003ue}.

Since then, numerous other exotic candidates have been found in the heavy
quarkonium sector as listed in Table~\ref{tab:2}. In fact, it is mainly due to 
the observation of these structures that the study of hadron spectroscopy 
experienced a renaissance.
The naming scheme currently used in the literature for these $XYZ$ states 
assigns isoscalar $J^{PC}=1^{--}$ states as $Y$ and
the isoscalar (isovector) states with other quantum numbers are named as $X 
(Z)$. Note that the charged heavy quarkonium-like states $\Zc^\pm$, 
$\Zcp^\pm$, $\Zb^\pm$, $\Zbp^\pm$ and $Z_c
(4430)^\pm$ are already established as being exotic, since they should contain 
at least two quarks and two anti-quarks with the hidden pair of $c\bar c$ or 
$b\bar b$ providing the dominant parts of their masses.


%-----------------------------------------------------------
\begin{figure}[tbh]
\begin{center}
 \includegraphics[width=\linewidth]{./figures/massspectrum.pdf}
\caption{$S$-wave open charm thresholds and candidates for exotic states in 
charmonium sector. Red solid (blue dashed)
 horizontal lines indicate the thresholds for nonstrange (strange) meson pairs.
 Two additional thresholds involving a charmonium $\chi_{c0}\omega$ and
 $\psi^\prime f_0(980)$ are also shown in the figure as green dotted lines.
The data are taken from Ref.~\cite{Olive:2016xmw}. The exotic candidates
are listed as black dots and green triangles with the latter marking the states 
to be discussed here. Here $D_{s0}^{}$,
$D_{s1}^{}$, $D_{s1}'$ and $D_{s2}^{}$ mean $D_{s0}^*(2317)$, $D_{s1}^{}(2460)$, 
$D_{s1}^{}(2536)$ and $D_{s2}^{}(2573)$, respectively.}
\label{fig:charmoniumlike}
\end{center}
\end{figure}
%----------------------------------------------------------
In the heavy quarkonium mass region, there are quite a few $S$-wave thresholds 
opened by narrow heavy-meson pairs. In the charmonium mass region, the
lowest-lying thresholds are  $D\bar{D}$, $D\bar{D}^*$ and 
$D^*\bar{D}^*$. They are particularly interesting for understanding the 
$X$ and $Z$ states which can couple to them in an $S$-wave. The relevant 
quantum numbers are thus $J^{PC}=1^{+-}$ and $(0,1,2)^{++}$ (for more details, 
see Section~\ref{sec:4-interactions}). 
The $S$-wave thresholds for the $XYZ$ exotic candidates are also shown in 
Table~\ref{tab:2}. In addition,
the exotic candidates in the charmonium sector and the $S$-wave open-charm 
thresholds are shown in Fig.~\ref{fig:charmoniumlike}. Here, the thresholds 
involving particles with a large width, $\gtrsim100$~MeV, have been neglected.

Since only $S$-wave hadronic molecules with small binding energies are 
well-defined (Sec.~\ref{sec:wein}), in the following, we will focus on 
those candidates, {\sl i.e.}, $\X$, $\Zc$, $\Zcp$, $Y(4260)$ in the charmonium 
sector and $\Zb$, $\Zbp$ in the bottomonium sector. All of them have extremely
close-by $S$-wave thresholds except for the $Y(4260)$, as will be discussed 
below.  For the experimental status and phenomenological models of other exotic
candidates, we refer to several recent  
reviews~\cite{Swanson:2006st,Eichten:2007qx,Brambilla:2014jmp,
Esposito:2014rxa,Lebed:2016hpi, Richard:2016eis,Chen:2016qju,Esposito:2016noz}
and references therein.

\subsubsection{$\X$}


In 2003, the Belle Collaboration reported a narrow structure $\X$ in the 
$J/\psi\pi^+\pi^-$ invariant mass distribution in $B^\pm\to K^\pm  
J/\psi\pi^+\pi^-$~\cite{Choi:2003ue} process. It was confirmed shortly after by 
BaBar~\cite{Aubert:2004ns,Aubert:2008gu} in $e^+e^-$ collisions, and by 
CDF~\cite{Acosta:2003zx,Abulencia:2005zc,Abulencia:2006ma,Aaltonen:2009vj} and 
D0~\cite{Abazov:2004kp} in $p\bar{p}$  collisions. Very recently 
LHCb also confirmed its production in $pp$  
collisions~\cite{Aaij:2011sn,Aaij:2013zoa,Aaij:2014ala,Aaij:2015eva} and pinned 
down its quantum numbers to $J^{PC}=1^{++}$, 
which are consistent with the observations of its radiative 
decays~\cite{Abe:2005ix,Aubert:2006aj,Bhardwaj:2011dj} and 
multipion transitions~\cite{Abulencia:2005zc,Abe:2005ix,delAmoSanchez:2010jr}.
The negative result of searching for its
charged partner in $B$ decays~\cite{Aubert:2004zr} indicates that the $\X$ is an 
isosinglet state. 


The most salient feature of the $X(3872)$ is that its mass coincides exactly 
with the $D^0\bar{D}^{*0}$ threshold~\cite{Olive:2016xmw}\footnote{Here we use
the updated ``OUR AVERAGE'' values in PDG2016 for the masses:
$M_{D^0}=(1864.84\pm0.05)$~MeV, $M_{D^{*0}}=(2006.85\pm0.05)$~MeV, and 
$M_{X}=(3871.69\pm0.17)$~MeV from the $J/\psi \pi^+\pi^-$ and $J/\psi\omega$
modes~\cite{Olive:2016xmw}. } 
\begin{equation}
M_{D^0}+M_{D^{*0}}-M_{\X}=(0.00\pm0.18)~\mev\,,
\end{equation}
which indicates the important role of the $D^0\bar{D}^{*0}$ in 
the $X(3872)$ dynamics. That this should be the case can be seen most clearly 
from the large branching fraction~\cite{Gokhroo:2006bt,Adachi:2008sua} (see 
Table~\ref{tab:2})
\begin{eqnarray}
  \mathcal{B}(\X\to \bar D^{0} D^0\pi^0) > 32\% \, ,
\end{eqnarray}
although the $\X$ mass is so close to the $D^0\bar D^{*0}$ and $\bar D^0
D^0\pi^0$ thresholds.
These experimental facts lead naturally to the interpretation 
of the $\X$ as a $D\bar D^{*}$ hadronic
molecule~\cite{Tornqvist:2003na},\footnote{
See also,
e.g.,~\cite{Tornqvist:2004qy,Swanson:2003tb,Close:2003sg,Pakvasa:2003ea,
Wong:2003xk,Voloshin:2003nt,Swanson:2004pp,AlFiky:2005jd,Braaten:2007dw, 
Fleming:2007rp,Liu:2008tn,Dong:2009yp,Ding:2009vj,Zhang:2009vs,Wang:2009aw, 
Lee:2009hy,Gamermann:2009uq,Mehen:2011ds,Nieves:2011vw, Lee:2011rka,
Nieves:2012tt,Li:2012ss, Li:2012cs,Sun:2012zzd, 
Sun:2012sy,Guo:2013sya,Hidalgo-Duque:2013pva,Wang:2013kva,Yamaguchi:2013ty, 
Guo:2014hqa,He:2014nya, Zhao:2014gqa,Karliner:2015ina,
Baru:2015tfa,Jansen:2015lha,Baru:2015nea,Molnar:2016dbo,
Yang:2017prf}.}
which had been predicted by T\"ornqvist with the correct mass a 
decade earlier~\cite{Tornqvist:1993ng}.
As will be discussed in Section~\ref{sec:6}, precise measurements of the 
partial widths of the processes $X(3872)\to D^0\bar D^0\pi^0$ and $X(3872)\to 
D^0\bar D^0\gamma$ are particularly important in understanding the 
long-distance structure of the $X(3872)$.
%
In the $D^0\bar{D}^{*0}$ hadronic molecular scenario, one gets a tremendously 
large 
$D^0\bar{D}^{*0}$ scattering length of $\geq10$~fm, {\sl c.f.} 
Eq.~\eqref{eq:arwein}. 
However, a precision measurement of its mass is necessary
to really distinguish a molecular $\X$ from, e.g., a tetraquark 
scenario~\cite{Maiani:2004vq,Esposito:2014rxa}. This will be discussed further 
in Sec.~\ref{sec:poletrajectories} and in Sec.~\ref{sec:lineshapes}.


Other  observables are also measured which could be sensitive to the internal
structure of the $X(3872)$. The ratio of branching fractions 
\[ R^I\equiv
\frac{\br{\X\to J/\psi \pi^+\pi^-\pi^0}}{\br{\X \to J/\psi\pi^+\pi^-}}
\] 
was
measured to be $1.0\pm 0.4\pm 0.3$ by Belle~\cite{Abe:2005ix} and $0.8\pm 0.3$ by
BaBar~\cite{delAmoSanchez:2010jr}. The value about unity means a significant
isospin breaking because the three and two pions are  from the isoscalar
$\omega$~\cite{Abe:2005ix,delAmoSanchez:2010jr} and from the isovector
$\rho$~\cite{Abulencia:2005zc}, respectively.
Notice that there is a strong  phase space suppression on the isospin conserved
three-pion transition through the $J/\psi\omega$ channel. The fact that the
molecular scenario of $\X$ provides a natural explanation for the value of $R^I$
will be discussed in Sec.~\ref{sec:isospinviol}.


The experimental information available about the radiative decays of the 
$X(3872)$ is~\cite{Aaij:2014ala} 
\begin{equation}
\frac{\br{\X\to \psi^\prime\gamma}}{\br{\X \to J/\psi\gamma}} = 2.46\pm 
0.64\pm 0.29 \ . 
\label{eq:sec2X3872R}
\end{equation}
A value larger than 1 for this ratio was argued to favor the 
$\chi_{c1}(2^3P_1)$ 
interpretation~\cite{Swanson:2004pp} 
over the $D^0\bar{D}^{*0}$ hadronic molecular picture. This, however, is not 
the case~\cite{Mehen:2011ds,Guo:2014taa} as will be demonstrated in 
Sec.~\ref{sec:6}. 


The production  rates of $\X$ in $B^0$ and $B^-$ decays was measured by 
BaBar~\cite{Aubert:2005zh}, {\sl i.e.}, 
\begin{eqnarray}
&& \frac{\mathcal{B}(B^0\to \X K^0\to J/\psi \pi^+\pi^-
K^0)}{\mathcal{B} (B^-\to\X K^-\to J/\psi \pi^+\pi^- K^-)}  \nonumber\\
&=& 0.50\pm 0.30\pm
0.05 \ .
\label{eq:sec2X3872RB}
\end{eqnarray}
We show in Sec.~\ref{sec:6} that this value is also consistent with a   
molecular nature of the $X(3872)$.

One expects mirror images of charmonium-like states to be present in the
bottomonium sector. The $Z_c$ and $Z_b$ states to be discussed in the next
subsection suggest that such phenomena do exist. The analogue of the $\X$ in the
bottom sector, $X_b$, has not yet been identified.
A search for the $X_b$ was carried out by the CMS Collaboration, but no signal
was observed in  the $\Upsilon \pi^+\pi^-$ channel~\cite{Chatrchyan:2013mea}.
However, as pointed out in Ref.~\cite{Guo:2013sya} before the experimental
results and stressed again in Refs.~\cite{Guo:2014sca,Karliner:2014lta}
afterwards, the $X_b\to \Upsilon \pi^+\pi^-$ decay requires an isospin breaking
which should be strongly suppressed due to the extremely small mass differences
between the charged and neutral bottomed mesons and the large difference between
the $B\bar B^*$ threshold and the $\Upsilon(1S)\omega$, $\Upsilon(1S)\rho$
thresholds.
In contrast, other channels such as $X_b\to \Upsilon\pi^+\pi^-\pi^0$, $X_b\to
\chi_{bJ}\pi^+\pi^-$~\cite{Guo:2013sya,Guo:2014sca,Karliner:2014lta} and $X_b\to
\gamma\Upsilon(nS)$~\cite{Li:2014uia} should be a lot more promising for an
$X_b$ search.



\subsubsection{$\Zb$, $\Zbp$ and $\Zc$, $\Zcp$}
\label{sec:zc}


From an analysis of the $\Upsilon(10860)\to \pi^+\pi^-(b\bar b)$ processes in
2011 the Belle Collaboration reported the discovery of two charged states
decaying into $\Upsilon(nS)\pi$  with $n=1,2,3$ and $h_b(mP)\pi$ with
$m=1,2$~\cite{Belle:2011aa}. Their line shapes in a few channels are shown in
Fig.~\ref{fig:Zb5Sfull}. A later analysis at the same experiment allowed
for  an amplitude analysis where the quantum numbers $I^G(J^{P})=1^+(1^{+})$
were strongly favored~\cite{Garmash:2014dhx}.\footnote{The existence of an
isovector $b\bar b q\bar q$ state with exactly these quantum numbers was
speculated long time ago for explaining the puzzling
$\Upsilon(3S)\to\Upsilon(1S)\pi\pi$
transition~\cite{Voloshin:1982ij,Anisovich:1995zu}. The $Z_b$ effects in dipion
transitions among $\Upsilon$ states were reanalyzed using the dispersion
technique recently in~\cite{Chen:2015jgl,Chen:2016mjn}. } This, together with
the fact that the $\Zb$ and $\Zbp$ have masses very close  to the $B\bar B^*$
and $B^*\bar B^*$ thresholds, respectively, makes both excellent candidates for
hadronic molecules~\cite{Bondar:2011ev}\footnote{See also,
e.g.,~\cite{Zhang:2011jja,
Yang:2011rp,Danilkin:2011sh,Sun:2011uh,Cleven:2011gp,Ohkoda:2011vj, Li:2012as, 
Dong:2012hc,Wang:2013daa,Wang:2014gwa, Dias:2014pva,Karliner:2015ina}.}  This
statement finds further support in the observation that both states also decay by far most probably into $B\bar B^*$ and $B^*\bar B^*$,
respectively~\cite{Garmash:2015rfd} (see Tab.~\ref{tab:ZbBr}).\footnote{The branching
fractions were measured by assuming that these channels saturate the decay modes and using the Breit--Wigner (BW) parameterization for the $Z_b$ structures~\cite{Garmash:2015rfd}.
However, there could be non-negligible modes such as the $\eta_b\rho$, and the
branching fractions measured in this way for near-threshold states should not be
used to calculate partial widths by simply multiplying with the BW width.
This point is discussed in detail in~\cite{Chen:2015jgl} for the $Z_b$ case.}
The neutral partner is so far observed only for the lighter
state~\cite{Krokovny:2013mgx}. Very recently, the Belle Collaboration reported
the invariant mass distributions  of $h_b(1P)\pi$ and $h_b(2P)\pi$ channels at
$\Upsilon(11020)$ energy region~\cite{Abdesselam:2015zza}, see
Fig.~\ref{fig:Zb6S}, clearly showing a resonant enhancement in the $Z_b$ mass
region. However, due to the limited statistics it is impossible to judge whether
there are two peaks or just one.



\begin{figure*}[tb]
\begin{center}
 \includegraphics[width=0.24\textwidth]{./figures/fig_BBst_fitC}\hfill
 \includegraphics[width=0.24\textwidth]{./figures/fig_BstBst_fitC}\hfill
 \includegraphics[width=0.24\textwidth]{./figures/fig_hb1_fitC}\hfill
 \includegraphics[width=0.24\textwidth]{./figures/fig_hb2_fitC}\hfill
\end{center}
\caption{Measured line shapes of the two $Z_b$ states in the
$B\bar{B}^*$, $B^*\bar{B}^*$ and $h_b(1P, 2P)\pi$
channels~\cite{Garmash:2015rfd} and a fit using the parameterization of 
Refs.~\cite{Hanhart:2015cua,Guo:2016bjq}. } \label{fig:Zb5Sfull}
\end{figure*}



%-----------------------------------------------------------
\begin{figure}[tb]
\begin{center}
 \includegraphics[height=5cm]{./figures/Zb6S-1}\hfill
 \includegraphics[height=5cm]{./figures/Zb6S-2}
\caption{The missing mass spectra for $h_b(1P)\pi^+\pi^-$ and 
$h_b(2P)\pi^+\pi^-$ channels  in the 
$\Upsilon(11020)$ region. The solid and dashed histograms are the fits with the
$Z_b$ signal fixed from the $\Upsilon(10860)$ analysis and with only a phase 
space distribution, respectively.
Taken from Ref.~\cite{Abdesselam:2015zza}.
} \label{fig:Zb6S}
\end{center}
\end{figure}
%----------------------------------------------------------


%----------------------------------------------------------
\begin{table}
\caption{The reported branching fractions of the known decay modes of $\Zb^+$
and $\Zbp^+$~\cite{Garmash:2015rfd} with the
statistical and systematical uncertainties in order.}
\begin{ruledtabular}
\begin{tabular}{l c c}
channel & $\mathcal{B}$ of $\Zb$ ($\%$) & $\mathcal{B}$ of 
$\Zbp$$(\%)$\tabularnewline
\hline
$\Upsilon(1S)\pi^{+}$ & $0.54^{+0.16+0.11}_{-0.13-0.08}$ & 
$0.17^{+0.07+0.03}_{-0.06-0.02}$\tabularnewline
$\Upsilon(2S)\pi^{+}$ & $3.62^{+0.76+0.79}_{-0.59-0.53}$ & 
$1.39^{+0.48+0.34}_{-0.38-0.23}$\tabularnewline
$\Upsilon(3S)\pi^{+}$ & $2.15^{+0.55+0.60}_{-0.42-0.43}$ & 
$1.63^{+0.53+0.39}_{-0.42-0.28}$\tabularnewline
$h_{b}(1P)\pi^{+}$ & $3.45^{+0.87+0.86}_{-0.71-0.63}$ & 
$8.41^{+2.43+1.49}_{-2.12-1.06}$\tabularnewline
$h_{b}(2P)\pi^{+}$ & $4.67^{+1.24+1.18}_{-1.00-0.89}$ & 
$14.7^{+3.2+2.8}_{-2.8-2.3}$\tabularnewline
$B^{+}\bar{B}^{*0}+\bar{B}^{0}B^{*+}$ & $85.6^{+1.5+1.5}_{-2.0-2.1}$ & 
--\tabularnewline
$B^{*+}\bar{B}^{*0}$ & -- & $73.7^{+3.4+2.7}_{-4.4-3.5}$\tabularnewline
\end{tabular}
\end{ruledtabular}
\label{tab:ZbBr}
\end{table}
%----------------------------------------------------------

Employing sums of BW functions for the resonance signals the experimental
analyses gave masses for both $Z_b$ states slightly above the corresponding open
flavor thresholds together with narrow widths. It seems in conflict with the
hadronic molecular picture, and was claimed to be consistent with the tetraquark
approach~\cite{Esposito:2016itg}. It is therefore important to note that a
recent analysis based on a formalism consistent with unitarity and analyticity
leads for both states to below-threshold pole
positions~\cite{Hanhart:2015cua,Guo:2016bjq}.\footnote{Notice that this,
however, does not exclude the possibility of above-threshold poles. In the used
parameterization, the contact terms are taken to be constants.
The possibility of getting an above-threshold pole is available once
energy-dependence is allowed in the contact terms. Nevertheless, the analyses at
least show that the below-threshold-pole scenario is consistent with the current
data.}

A few years after the discovery of $\Zb$ and $\Zbp$ in the Belle experiment, the
BESIII and Belle Collaborations almost simultaneously claimed the observation of
a charged state in the charmonium mass range,
$\Zc$~\cite{Ablikim:2013mio,Liu:2013dau}. It was shortly after confirmed by a
reanalysis of CLEO-c data~\cite{Xiao:2013iha}, and its neutral partner was also
reported in Refs.~\cite{Xiao:2013iha,Ablikim:2015tbp}. Soon after these
observations, the BESIII Collaboration reported the discovery of another charged
state $\Zcp$~\cite{Ablikim:2013wzq}, and its neutral partner was reported in
Ref.~\cite{Ablikim:2014dxl}.
These charmonium-like states show in many respects similar features as the
heavier bottomonium-like states discussed in the previous paragraphs, although
there are also some differences. On the one hand, while the $\Zc$ is seen in the
$J/\psi \pi$ channel and $\Zcp$ is seen in $h_c\pi$, there is no clear signal of
$\Zcp$ in $J/\psi\pi$ and $\Zc$ in $h_c\pi$, although in the latter case there
might be some indications of $\Zc\to h_c\pi$. This pattern might reflect a
strong mass dependence of the production mechanism~\cite{Wang:2013hga}.
On the other hand, in analogy to $\Zb$ and $\Zbp$, $\Zc$ and $\Zcp$ have masses
very close to the the $D\bar D^*$ and $D^*\bar D^*$ thresholds, respectively,
and they couple most prominently to these open-flavor channels regardless
of the significant phase space
suppression~\cite{Ablikim:2013xfr,Ablikim:2013emm,Ablikim:2015gda,
Ablikim:2015vvn}. The two $Z_c$ states are also widely regarded as hadronic
molecules~\cite{Wang:2013cya,Guo:2013sya,Voloshin:2013dpa,Cui:2013yva,
Wilbring:2013cha,Li:2013xia,
Zhang:2013aoa,Dong:2013iqa,Ke:2013gia,He:2013nwa,
Karliner:2015ina,Chen:2015ata,Gong:2016hlt}.

Analogous to the $Z_b$ case, the experimental analyses of the two $Z_c$ states
based on sums of BW distributions result in masses above the continuum thresholds as
well. However, this does not allow the correct extraction of the pole locations.
In order to obtain reliable pole locations  an analysis in the spirit of
Refs.~\cite{Hanhart:2015cua,Guo:2016bjq} is necessary for these charmonium-like
states. Such an analysis was done for the $\Zc$ in
Ref.~\cite{Albaladejo:2015lob}. By fitting to the available BESIII data in the
$Y(4260)\to J/\psi\pi^+\pi^-$~\cite{Ablikim:2013mio} and the $Y(4260)\to
J/\psi\pi^+\pi^-$~\cite{Ablikim:2015swa} modes, it is found that the current
data are consistent with either an above-threshold resonance pole or a
below-threshold virtual state pole. A comparison of the resonance pole obtained
therein with various determinations in experimental papers is shown in
Fig.~\ref{fig:Zcpole}.
% -----------------------------------------------------------
\begin{figure}[tb]
\begin{center}
  \includegraphics[width=0.586\linewidth]{./figures/Zcpoles.pdf}
 \caption{Poles determined in Ref.~\cite{Albaladejo:2015lob} (0.5~GeV and
 1.0~GeV refer to the cutoff values used therein) in comparison with the mass
 and width values for the $Z_c(3900)$ reported in Refs.~\cite{Ablikim:2013mio, 
 Ablikim:2013xfr, Ablikim:2015swa, Liu:2013dau, Xiao:2013iha}. Taken from
 Ref.~\cite{Albaladejo:2015lob}.
 }
 \label{fig:Zcpole}
\end{center}
\end{figure}
%----------------------------------------------------------






\subsubsection{$Y(4260)$ and other vector states}\label{sec:Y(4260)}

At present, the vector channel with $J^{PC}=1^{--}$, in both the bottomonium and
the charmonium sector, is the best investigated one experimentally, since it can
be accessed directly in $e^+e^-$ annihilations.
Note that a pair of ground state open-flavor mesons, such as $D\bar{D}$,
$D\bar{D}^*+c.c.$, $D^*\bar{D}^*$, $D_s\bar{D}_s$, etc., carry positive parity
in the $S$ wave, and thus cannot be directly accessed in $e^+e^-$ 
annihilations. 
Accordingly, if the $S$-wave hadronic molecules exist in the vector channel,
they should be formed (predominantly) by constituents different from them. In 
particular,
it suggests that the $S$-wave molecular states in the vector channel should be
heavier than those thresholds opened by a pair of ground state $D^{(*)}$ mesons.


%-----------------------------------------------------------
\begin{figure}[tb]
\begin{center}
  \includegraphics[width=0.45\textwidth]{figures/Y4260BESIII2016.pdf}
\caption{ The cross section of $e^+e^-\to \pi^+\pi^-J/\psi$ for center-of-mass 
energies from $3.77~\gev$ to $4.6~\gev$~\cite{Ablikim:2016qzw}.
It shows a clear shoulder around the $D_1\bar D$ threshold (marked by the
vertical gray band) as predicted in Ref.~\cite{Cleven:2013mka}. The red solid curve is from the 
analysis of BESIII~\cite{Ablikim:2016qzw}. A comparison of these data with the
BESIII scan data can be found in~\cite{Gao:2017sqa}. }
\label{fig:Y4260-BESIII}
\end{center}
\end{figure}
%----------------------------------------------------------



As the first $Y$ state, the $Y(4260)$
was observed by  the BaBar Collaboration in the $J/\psi\pi^+\pi^-$ channel
in the initial state radiation (ISR) process
$e^+e^-\to\gamma_\text{ISR} J/\psi\pi^+\pi^-$~\cite{Aubert:2005rm}.
The fitted mass and width are $\left(4259\pm 8^{+2}_{-6}\right)\mev$ and 
$50\ldots 90~\mev$,
respectively. It was confirmed by
CLEO-c~\cite{He:2006kg}, Belle~\cite{Yuan:2007sj} and an additional analysis of
BaBar~\cite{Lees:2012cn}, with, however, mass values varying in different 
analyses. We notice that a recent combined analysis of the BESIII data in four 
different channels $e^+e^-\to \omega\chi_{c0}$~\cite{Ablikim:2015uix}, 
$\pi^+\pi^-h_c$~\cite{BESIII:2016adj}, 
$\pi^+\pi^- J/\psi$~\cite{Ablikim:2016qzw}, 
and $D^0D^{*-}\pi^++c.c.$~\cite{yuantalk}
gives a mass of $(4219.6\pm3.3\pm5.1)$~MeV and a width of 
$(56.0\pm3.6\pm6.9)$~MeV~\cite{Gao:2017sqa}.

The $Y(4260)$ was early recognized as a good candidate for an exotic state since
there are no quark-model states predicted around its mass.
Moreover, it does not show a strong coupling to $D\bar D$ as generally expected
for vector $c\bar c$ states, and it does not show up as a pronounced enhancement
in the inclusive cross sections for $e^+e^-\to$ hadrons (or the famous $R$ value
plot).
It is still believed to be a prime candidate, e.g., for a hybrid
state~\cite{Close:2005iz} (for a recent discussion see
Ref.~\cite{Kalashnikova:2016bta}) or a hadrocharmonium
state~\cite{Dubynskiy:2008mq,Li:2013ssa}.
However, it is also suggested to be a $D_1(2420)\bar D$ molecular
state~\cite{Ding:2008gr,Wang:2013cya,Cleven:2013mka,Li:2013bca,Li:2013yla,
Wu:2013onz} (the hadrocharmonium picture and the molecular picture are
contrasted in Ref.~\cite{Wang:2013kra}).
This picture is further supported by the fact that  the recent high-statistics
data from BESIII~\cite{Ablikim:2016qzw}  (see Fig.~\ref{fig:Y4260-BESIII}) shows
an enhancement at the $D_1(2420)\bar D$ threshold in the $J/\psi\pi\pi$
channel.\footnote{In this  context it is interesting to note that also the
hybrid picture predicts a large coupling of $Y(4260)$ to $D_1(2420)\bar
D$~\cite{Barnes:1995hc,Close:2005iz,Kou:2005gt}, which could be
interpreted as the necessity of considering $D_1\bar D$ as a component.} The
observations of $Z_c(3900)\pi$ (Sec.~\ref{sec:zc}) and
$X(3872)\gamma$~\cite{Ablikim:2013dyn} in the mass region of the $Y(4260)$
provide further support for a sizable $D_1(2420)\bar D$ component in its wave
function as will be discussed in Sec.~\ref{sec:6}.
The suppression of an $S$-wave production, in the heavy quark limit, of the
$1^{--}$ $D_1(2420)\bar D$ pair in $e^+e^-$
collisions~\cite{Eichten:1978tg,Eichten:1979ms,Li:2013yka} could be the reason
for the dip around the $Y(4260)$ mass in the inclusive cross section of
$e^+e^-\to$ hadrons~\cite{Wang:2013kra}.
In addition,  the data from Belle in $e^+e^-\to
\bar{D}D^*\pi$~\cite{Pakhlova:2009jv} and from BESIII on $e^+e^-\to
h_c\pi\pi$~\cite{BESIII:2016adj}, $\chi_{c0}\omega$~\cite{Ablikim:2014qwy}  are
highly nontrivial (Fig.~\ref{fig:Y4260LineShape}) and are claimed to be
consistent with the molecular picture~\cite{Cleven:2013mka,Cleven:2016qbn}.
A combined analysis of the BESIII data in different channels is presented in
Ref.~\cite{Gao:2017sqa}.

The absence of a signal for $Y(4260)$ in $J/\psi
K\bar{K}$~\cite{He:2006kg,Yuan:2007bt,Shen:2014gdm} questions the tetraquark
picture of $Y(4260)$ with a diquark-antidiquark $[cs][\bar c\bar s]$
configuration~\cite{Esposito:2014rxa}.
In addition, the ground state in the tetraquark picture~\cite{Esposito:2014rxa},
$Y(4008)$, is not confirmed by the recent high-statistics data from
BESIII~\cite{Ablikim:2016qzw}.
Meanwhile, the cross sections for $e^+e^-\to
\psi'\pi\pi$~\cite{Aubert:2007zz,Wang:2007ea}, $\eta'
J/\psi$~\cite{Ablikim:2016ymr} and $\eta J/\psi$ and $\pi^0
J/\psi$~\cite{Ablikim:2015xhk} do not show any structure around the $Y(4260)$
energy region. It remains to be seen if these findings allow for further conclusions on the nature of 
 the $Y(4260)$.

It is interesting to observe that some properties of the $Y(4260)$, like its
proximity and strong coupling to the $D_1\bar D$ threshold, are mirror imaged by
the $\Upsilon(11020)$ in the bottomonium sector~\cite{Bondar:2016pox}. Belle II
appears to be an ideal instrument to investigate this connection in more detail
in the future~\cite{Bondar:2016hva}.


%-----------------------------------------------------------
\begin{figure*}[tb]
\parbox{5.5cm}{
  \includegraphics[width=0.31\textwidth]{figures/Y4260a.pdf}}\hfill
\parbox{5.5cm}{\vspace{-0.1cm}
   \includegraphics[width=0.33\textwidth]{figures/Y4260b.pdf}}
 \hfill
\parbox{5.5cm}{ 
     \includegraphics[width=0.29\textwidth]{figures/Y-chic0omega}}
\caption{
The first plot shows the cross sections of the $e^+e^-\to h_c \pi\pi$ (red solid
circles) from BESIII~\cite{Ablikim:2013wzq} and the $e^+e^-\to J/\psi\pi\pi$
(blue hollow circles) from Belle~\cite{Liu:2013dau} (note that the recent BESIII
data for $e^+e^-\to J/\psi\pi\pi$ have much smaller errors as shown in
Fig.~\ref{fig:Y4260-BESIII}).
The first plot is taken from~\cite{Chang-Zheng:2014haa}. The second one is the
line shape for the $D\bar{D}^*\pi$ channel within the $D_1\bar D$ molecular
picture~\cite{Cleven:2013mka} compared to the Belle data~\cite{Pakhlova:2009jv}. 
The predicted line shape is similar to the solid 
line of the right panel of Fig.~\ref{fig:lineshapes2} with unstable constituent 
in Sec.~\ref{sec:3} (note 
that an updated analysis can be found in~\cite{Qin:2016spb} and the new data 
from BESIII can be found in~\cite{Gao:2017sqa}). The last plot is the line 
shape of $e^+e^-\to\chi_{c0}\omega$ measured by BESIII~\cite{Ablikim:2014qwy}
and the bands are theoretical calculations in the $D_1\bar D$ molecular 
picture~\cite{Cleven:2016qbn}.
}
\label{fig:Y4260LineShape}
\end{figure*}
%---------------------------------------------------------


Searching for new decay modes of the $Y(4260)$, BaBar scanned the line shapes of
$e^+e^-\to \psi(2S)\pi^+\pi^-$ and found a new structure, named $Y(4360)$ with a
mass of $(4324\pm 24)~\mev$ and a width of $(172\pm
33)~\mev$~\cite{Aubert:2007zz}. In the same year, Belle \cite{Wang:2007ea}
analyzed the same process and found two resonant structures: a lower one
consistent with $Y(4360)$ and a higher one, named $Y(4660)$, with a mass of
$(4664\pm 11\pm 5)~\mev$ and a width of $(48\pm 15\pm 3)~\mev$. A combined fit
~\cite{Liu:2008hja} to the cross sections of the process $e^+e^-\to
\psi(2S)\pi^+\pi^-$ from both BaBar and Belle gives the parameters for the two
resonances $M_{Y(4360)}=\left(4355^{+9}_{-10}\pm 9\right)\mev$,
$\Gamma_{Y(4360)}=\left(103^{+17}_{-15}\pm 11\right)\mev$ and
$M_{Y(4660)}=\left(4661^{+9}_{-8}\pm6\right)\mev$,
$\Gamma_{Y(4660)}=\left(42^{+17}_{-12}\pm6\right)\mev$ for $Y(4360)$ and
$Y(4660)$, respectively. The fit at the same time provides an upper limit for
$\br{Y(4260)\to \psi(2S)\pi^+\pi^-}\Gamma_{e^+e^-}$ as $4.3~\ev$.
Those measurements were updated in Ref.~ \cite{Wang:2014hta}.
Later on the Belle Collaboration found a structure in the
$\Lambda_c^+\Lambda_c^-$ channel with a peak position $30~\mev$ lower than that
of $Y(4660)$~\cite{Pakhlova:2008vn}, which might either point at an additional
state, called $Y(4630)$, or, be an additional  decay channel of the
$Y(4660)$~\cite{Cotugno:2009ys,Guo:2010tk}. The latter is the view taken in the
2016 Review of Particle Physics~\cite{Olive:2016xmw}.
Of particular interest to this review is the observation of
Ref.~\cite{Guo:2010tk} that within the $\psi^\prime f_0(980)$ hadronic molecular
picture~\cite{Guo:2008zg,Wang:2009hi} the line shape of $Y(4630)$ in the
$\Lambda_c^+\Lambda_c^-$ channel could be understood as the signal of $Y(4660)$
with the $\Lambda_c^+\Lambda_c^-$ final state interaction.
As a byproduct, Ref.~\cite{Guo:2009id} predicts the properties of its spin
partner, an $\eta_c^\prime f_0(980)$ hadronic molecule, at around 4.61~GeV.


As stressed in Refs.~\cite{Wang:2013hga,Bondar:2016pox} the 
production of $Z_c(3900)$ and $Z_c(4020)$ in the mass region of 
$Y(4260)$ and $Y(4360)$
as well as that of $\Zb$ and $\Zbp$ in the mass region of $\Upsilon(10860)$ and 
$\Upsilon(11020)$, respectively, are sensitive to the TS mechanism. A peculiar 
feature of such a mechanism is that whether peaks appear in certain invariant 
mass distributions depends strongly on the kinematics. The recent observation 
of a peak in the $\psi'\pi$ invariant mass distribution by the BESIII 
Collaboration~\cite{Ablikim:2017oaf} shows exactly this behavior.
The correlations between the initial $S$-wave thresholds and the final $S$-wave
thresholds could be a key for understanding the rich phenomena observed in this 
energy region~\cite{Liu:2015taa}. 


\begin{table*}
\caption{Same as Table~\ref{tab:1} but in the  baryon
sector.}
\begin{ruledtabular}
 \begin{tabular}{l c c c c c}
state & $I(J^{P})$ & $M[\mathrm{MeV}]$ & $\Gamma[\mathrm{MeV}]$ & $S$-wave 
threshold(s) [$\mathrm{MeV}$] & Observed mode(s) (branching
ratios)\tabularnewline
\hline
$\Lambda(1405)$ & $0(\frac{1}{2}^{-})$ & $1405.1_{-1.0}^{+1.3}$ & $50.5\pm2.0$ & 
$N\bar{K}(-29.4_{-1.0}^{+1.3})$ &
$\Sigma\pi(100\%)$\tabularnewline
&&&&$\Sigma\pi(76.2^{+1.3}_{-1.0})$&\tabularnewline
\hline
$\Lambda(1520)$& $0(\frac 32^-)$ & $1519.5\pm 1.0$  & $15.6\pm 1.0$  
&$\Sigma(1385)^-\pi^+(-7.3\pm 1.1)$   & $N\bar{K} (45\pm 1)\%$
\tabularnewline
&&&&$\Sigma(1385)^+\pi^-(-2.9 \pm 1.1)$& $\Sigma\pi(42\pm 1)\%$\tabularnewline
&&&&$\Sigma(1385)^0\pi^0(0.8 \pm 1.4)$& $\Lambda\pi\pi(10\pm 
1)\%$\tabularnewline
\hline
$\Lambda(1670)$
% ~\cite{Zhang:2013sva}~\footnote{The mass and width are derived 
% from the pole position of Ref.~\cite{Zhang:2013sva}}
& $0(\frac 12^-)$ &
$\approx1670$ & $\approx35$  &$\Lambda\eta\,(4)$   & $N\bar{K} (20\sim 30)\%$
\tabularnewline
&&&&& $\Sigma\pi(25\sim 55)\%$\tabularnewline
&&&&& $\Lambda\eta(10\sim 25)\%$\tabularnewline
\hline
$\Lambda_c(2595)$ & $0(\frac{1}{2}^{-})$ & $2592.25\pm 0.28$ & $2.6\pm0.6$ & 
$\Sigma_c(2455)^{++}\pi^- (-1.04\pm 0.31)$ &
$\Sigma_c(2455)^{++}\pi^-(24\pm 7)\%$\tabularnewline
 &  &  & & $\Sigma_c(2455)^0\pi^+ (-0.82\pm 0.31)$&
$\Sigma_c(2455)^{0}\pi^+(24\pm 7)\%$\tabularnewline
 &  &  & & $\Sigma_c(2455)^+\pi^0 (4.62 \pm 0.49)$&
$\Lambda_c^{+}\pi^+\pi^-~\text{3-body}~(18\pm 10)\%$\tabularnewline
\hline
$\Lambda_c(2625)$ & $0(\frac{3}{2}^{-})$ & $2628.11\pm 0.19$ & $<0.97$ & 
$\Sigma_c(2455)\pi (36.53\pm 0.24)$ &
$\Lambda_c^+\pi^+\pi^-(67\%)$\tabularnewline
 &  &  & & &
$\Sigma_c(2455)^{++}\pi^-(<5\%)$\tabularnewline
 &  &  & & &
$\Sigma_c(2455)^0\pi^+(<5\%)$\tabularnewline
\hline
$\Lambda_c(2880)$ & $0(\frac{5}{2}^{+})$ & $2881.53\pm 0.35$ & $5.8 \pm 1.1$ & 
$ND^*(-65.91\pm 0.35)$&
$\Lambda_c^+\pi^+\pi^-$\tabularnewline
 & &  & & &
$\Sigma_c(2455)^{0,++}\pi^\pm $\tabularnewline
 & &  & & &
$\Sigma_c(2520)^{0,++}\pi^\pm $\tabularnewline
 & &  & & &
$pD^0 $\tabularnewline
\hline
$\Lambda_c(2940)$ & $0(\frac 32^-)$ &
$2939.3_{-1.5}^{+1.4}$ & $17^{+8}_{-6}$ & 
$ND^*(-8.1_{-1.5}^{+1.4})$ &
$\Sigma_c(2455)^{0,++}\pi^\pm$\tabularnewline
 & \cite{Aaij:2017vbw} &  &  &  &
$pD^0$\tabularnewline
\hline
$\Sigma_c(2800)$ & $1(?^?)$ & $2800_{-4}^{+5}$ & $70^{+23}_{-15}$ & $ND(-6 \pm 
5)$ &
$\Lambda_c^+\pi$\tabularnewline
\hline
$\Xi_c(2970)$ & $\frac{1}{2}(?^?)$ & $2969.4 \pm 1.7$ & $19.0\pm 3.9$ & 
$\Sigma_c(2455)K (20.2 \pm 1.7)$ &
$\Lambda_c^+\bar{K}\pi$\tabularnewline
 &  & & &  &
$\Sigma_c(2455)\bar{K}$\tabularnewline
 &  & & &  &
$\Xi_c2\pi$\tabularnewline
 &  & & &  &
$\Xi_c(2645)\pi$\tabularnewline
\hline
$\Xi_c(3055)$ & $?(?^?)$ & $3055.1 \pm 1.7$ & $11\pm 4$ & $\Sigma_c(2520)K (41.1 
\pm 1.7)$ &
\tabularnewline
 &  & & & $\Xi_c(2970)\pi (-52.3 \pm 2.4)$ & \tabularnewline
 \hline
 $\Xi_c(3080)$ & $\frac{1}{2}(?^?)$ & $3078.4 \pm 0.7$ & $5.0\pm 1.3$ & 
$\Sigma_c(2520)K (64.4 \pm 0.7)$ & $\Lambda_c^+\bar{K}\pi$
\tabularnewline
 &  & & & $\Xi_c(2970)\pi (-29.0 \pm 1.8)$ & 
$\Sigma_c(2455)\bar{K}$\tabularnewline
  &  & & &  & $\Sigma_c(2520)\bar{K}$ \tabularnewline
\hline
$P_c(4380)$ & $\frac 12(\frac 32^?/\frac 52^?)$& 
$4380\pm8\pm29$ & $205\pm18\pm86$ &  $\Sigma_c(2520)\bar{D}$
($-6 \pm 30$)  & $J/\psi p$\tabularnewline
 \cite{Aaij:2015tga} & & &  &  $\Sigma_c(2455)\bar{D}^*$ $(-82 \pm 30)$ 
 &\tabularnewline
\hline
$P_c(4450)$ &$\frac 12(\frac 32^?/\frac 52^?)$ 
&$4449.8\pm1.7\pm2.5$  &  $39\pm5\pm19$&  $\chi_{c1}p$ $(0.9 \pm 3.0)$  &
$J/\psi p$\tabularnewline
\cite{Aaij:2015tga} & & &  &  $\Lambda_c(2595)\bar{D}$ $(-9.9 \pm 3.0)$  &
\tabularnewline & & &  &   $\Sigma_c(2520)\bar{D}^*$ ($-77.2 \pm 3.0$) & \tabularnewline
& & &  &   $\Sigma_c(2520)\bar{D}$ ($64.2 \pm 3.0$) & \tabularnewline
& & &  &   $\Sigma_c(2455)\bar{D}^*$ ($-12.3 \pm 3.0$) & \tabularnewline
\end{tabular}
\end{ruledtabular}
\label{tab:baryon}
\end{table*}

\subsection{Baryon candidates for hadronic molecules}

We now switch to the experimental evidences for hadronic molecules in the baryon
sector. In analogy to  the meson sector we will focus on states which are
located close to $S$-wave thresholds of narrow meson-baryon
pairs~\footnote{Note that also $P_{11}(1440)$ was proposed to have a prominent
$f_0(500)N$ substructure~\cite{Krehl:1999km}. However, the large width of the
$f_0(500)$ prohibits a model-independent study of this claim.}. In the light
baryon spectrum the $\Lambda(1405)$ has been broadly discussed as a $\bar{K}N$
molecular state. A few charm baryons discovered in recent years are close to
$S$-wave thresholds, and has been suggested to be hadronic molecules in the
literature. The recently observed very interesting $P_c(4450)$ and $P_c(4380)$
have also been proposed to be hadronic molecules with hidden charm.


\subsubsection{Candidates in the light baryon sector}
\label{sec:lam1405exp}

The $\Lambda(1405)$ was discovered  in the $\pi\Sigma$ subsystems of  $Kp\to
\Sigma\pi\pi\pi$~\cite{Alston:1961zzd} (see also
Ref.~\cite{Kim:1965zzd,Hemingway:1984pz}).
Further experimental information about this state comes from old scattering
data~\cite{Humphrey:1962zz,Sakitt:1965kh,Watson:1963zz,Ciborowski:1982et}
complemented by the recent  $\bar{K}N$ threshold amplitude extracted from data
on kaonic hydrogen~\cite{Bazzi:2011zj,Bazzi:2012eq} as well as the older
so-called threshold ratios~\cite{Tovee:1971ga,Nowak:1978au}.
There are further data on $\Sigma\pi$ distributions from $pp\to \Sigma^\pm
\pi^\mp K^+p$~\cite{Zychor:2007gf,Agakishiev:2012xk}, the photoproduction
$\gamma p\to K^+\Sigma\pi$~\cite{Moriya:2014kpv} and additional reactions.
It appears also feasible that high energy experiments  like BaBar, Belle,
BESIII, CDF, D0 and LHCb investigate the $\Lambda(1405)$, e.g. via the decays
of heavy hadrons such as $\Lambda_b\to J/\psi
\Lambda(1405)$~\cite{Roca:2015tea}.
Note that a signal of $\Lambda(1405)$ was clearly visible  in an analysis of
$\Lambda_b\to J/\psi Kp$ performed by the LHCb
Collaboration~\cite{Aaij:2015tga}.

The $\Lambda(1405)$ has strangeness $S=-1$ with $I(J^P)=0(1/2^-)$ and a mass
about 30~MeV (see Table~\ref{tab:baryon}) below the $\bar{K}N$ threshold.
Note that a direct experimental determination of the spin-parity quantum numbers
was only given recently by the CLAS Collaboration \cite{Moriya:2014kpv}.
Since its mass is smaller than that of the nucleon counterpart
$N^*(1535)\,1/2^-$ and the mass difference from its spin-splitting partner state
$\Lambda(1520)$ $I(J^P)=0(3/2^-)$ is larger than that between $N^*(1535)\,1/2^-$
and $N^*(1520)\,3/2^-$, it can hardly be accepted by the conventional
three-quark picture of the constituent quark model~\cite{Hyodo:2011ur}.
It is fair to say that the $\Lambda(1405)$ {was most probably} the
first exotic hadron observed~\cite{Alston:1961zzd}. The theoretical aspects of the
$\Lambda(1405)$ will be discussed in Sec.~\ref{sec:1405th}.



\subsubsection{Candidates in the charm baryon sector}

The two light quarks in a charm baryon can be either in the symmetric sextet or
antisymmetric anti-triplet representation of SU(3). Since the color wave
function is totally antisymmetric,
 the spin-flavor-space wave functions must be symmetric. Hence the light-quark
system in the $S$-wave flavor sextet (anti-triplet) has spin $1$ ($0$).
After combining with a heavy quark, the sextet and anti-triplet give the
$B_6(1/2^+)$, $B_6^*(3/2^+)$ and $B_{\bar{3}}(1/2^+)$ baryon multiplets,
respectively, as~\cite{Yan:1992gz}
\begin{eqnarray*}
B_{6}=\left(\begin{array}{ccc}
\Sigma_{c}(2455)^{++} & \frac{1}{\sqrt{2}}\Sigma_{c}(2455)^{+} & 
\frac{1}{\sqrt{2}}\Xi_{c}^{\prime+}\\
\frac{1}{\sqrt{2}}\Sigma_{c}(2455)^{+} & \Sigma_{c}(2455)^{0} & 
\frac{1}{\sqrt{2}}\Xi_{c}^{\prime0}\\
\frac{1}{\sqrt{2}}\Xi_{c}^{\prime+} & \frac{1}{\sqrt{2}}\Xi_{c}^{\prime0} & 
\Omega_{c}^{0}
\end{array}\right),
\end{eqnarray*}
 \begin{eqnarray*}
B_{6}^{*}=\left(\begin{array}{ccc}
\Sigma_{c}(2520)^{++} & \frac{1}{\sqrt{2}}\Sigma_{c}(2520)^{+} & 
\frac{1}{\sqrt{2}}\Xi_{c}(2645)^{+}\\
\frac{1}{\sqrt{2}}\Sigma_{c}(2520)^{+} & \Sigma_{c}(2520)^{0} & 
\frac{1}{\sqrt{2}}\Xi_{c}(2645)^{0}\\
\frac{1}{\sqrt{2}}\Xi_{c}(2645)^{+} & \frac{1}{\sqrt{2}}\Xi_{c}(2645)^{0} & 
\Omega_{c}(2770)^{0}
\end{array}\right),
\end{eqnarray*}
\begin{eqnarray*}
B_{\bar{3}}=\left(\begin{array}{ccc}
0 & \Lambda_{c}^{+} & \Xi_{c}^{+}\\
-\Lambda_{c}^{+} & 0 & \Xi_{c}^{0}\\
-\Xi_{c}^{+} & -\Xi_{c}^{0} & 0
\end{array}\right).
\end{eqnarray*}
All the ground state charm baryons within these three multiplets have been well
established in experiments~\cite{Olive:2016xmw}.
Among the other charm baryons, $\Lambda_c(2765)$, $\Xi_c(2815)$ and
$\Xi_c(3123)$ are not well-established from the experimental
analysis~\cite{Olive:2016xmw}.
The $P$-wave $1/2^-$ and $3/2^-$ antitriplet states are identified as
$[\Lambda_c(2595)^+, \Xi_c(2790)^+,
\Xi_c(2790)^0]$~\cite{Cheng:2006dk,Cheng:2015naa} and $[\Lambda_c(2625)^+,
\Xi_c(2815)^+, \Xi_c(2815)^0]$, respectively~\cite{Cheng:2015naa}. Among the
remaining charm baryons,  the $\Lambda_c(2880)^+$ has a definite spin of
$5/2$~\cite{Olive:2016xmw,Aaij:2017vbw}
and the quantum numbers of the $\Lambda_c(2940)^+$ were measured to be
$J^P=3/2^-$~\cite{Aaij:2017vbw}.
Besides these two charmed baryons, LHCb also measured another charm baryon
$\Lambda_c(2860)^+$ which is consistent with the predication of the orbital
$D$-wave $\Lambda_c^+$ excitation~\cite{Chen:2016phw,Chen:2016iyi,Chen:2017aqm}.
The only available information of other measured charm baryons are their masses
and some of their decay modes.
For recent reviews on the heavy baryons, we recommend~\cite{Klempt:2009pi,
Chen:2016spr}.

Although the $\Lambda_c(2595)^+$ may be accommodated as a regular three-quark
baryon in quark
models~\cite{Copley:1979wj,Pirjol:1997nh,Tawfiq:1998nk,Zhu:2000py,
Blechman:2003mq,Migura:2006ep,Zhong:2007gp}, one cannot neglect one striking
feature of it~\cite{Hyodo:2013iga} which could provide other potential
interpretations:
It lies between the $\Sigma_c(2455)^+\pi^0$ and $\Sigma_c(2455)^0\pi^+$,
$\Sigma_c(2455)^{++}\pi^-$ thresholds as shown in Table~\ref{tab:baryon}.
Thus, the $\Lambda_c(2595)$ is proposed as a dynamically generated state of the
nearby $\Sigma_c(2455)\pi$ coupling with other possible higher
channels~\cite{Lutz:2003jw,Hofmann:2005sw,Mizutani:2006vq,GarciaRecio:2008dp,
JimenezTejero:2009vq,Haidenbauer:2010ch,Romanets:2012hm,Lu:2014ina,
Liang:2014kra,Garcia-Recio:2015jsa,Lu:2016gev,Long:2016oog}, such as $ND$,
$ND^*$ and so on. The strong coupling between the $\Lambda_c(2595)$ and  the
$\Sigma_c(2455)\pi$ channels even leads to a prediction of the existence of a
three-body $\Sigma_c\pi\pi$ resonance in Refs.~\cite{Long:2016oog,Long:2017bgz}.
The analysis in~\cite{Guo:2016wpy}, however, indicates that the compositeness of
$\Sigma_c(2455)^+\pi^0$ is smaller than $10\%$ leaving $\Lambda_c(2595)$
dominated by either other heavier hadronic channels (such as $ND$ and $ND^*$) or
compact quark-gluon structures. Some other charm baryons, such as
$\Lambda_c(2880)$~\cite{Lutz:2003jw},
$\Lambda_c(2940)$~\cite{He:2006is,He:2010zq,Ortega:2012cx,Zhang:2012jk,
Zhao:2016zhf} and
$\Sigma_c(2800)$~\cite{JimenezTejero:2009vq,JimenezTejero:2011fc,Zhang:2012jk},
have also been considered as dynamically generated states from meson-baryon
interactions.  In particular, the $\Lambda_c(2940)^+$ is very close to the
$ND^*$ threshold --- it even overlaps with the threshold if using the recent
LHCb measurement~\cite{Aaij:2017vbw},  and it can 
couple to $ND^*$ in an $S$-wave.
Both are favorable features for treating it as an $ND^*$ hadronic 
molecule~\cite{Ortega:2012cx,Zhao:2016zhf}.

\subsubsection{Pentaquark-like structures with hidden-charm}

Recently, LHCb reported two pentaquark-like structures $P_c(4380)^+$ and
$P_c(4450)^+$ in the $J/\psi p$ invariant mass distribution of $\Lambda_b\to
J/\psi p K^-$~\cite{Aaij:2015tga}. Their masses (widths) are $(4380\pm 8 \pm
29)$~MeV ($(205\pm 18\pm 86)$~MeV) and $(4449.8\pm 1.7\pm 2.5)$~MeV ($(39\pm
5\pm 19)$~MeV), respectively. In this analysis the $\Lambda^*$ states that
appear in the crossed channel were parametrized via BW functions. The LHCb
analysis reported preference of the spin-parity combinations $(3/2^-, 5/2^+)$,
$(3/2^+, 5/2^-)$ or $(5/2^+, 3/2^-)$  for these two states, respectively. The
branching ratio for $\Lambda_b\to J/\psi p K^-$ was also
measured~\cite{Aaij:2015fea}.


The data for the Cabibbo suppressed process $\Lambda_b\to J/\psi p \pi^-$ are
consistent with the existence of these two $P_c$ structures~\cite{Aaij:2016ymb}.
The same experiment also published a measurement of $\Lambda_b^0\to \psi(2S) p
K^-$, but no signals for the $P_c$ states were observed due to either the low
statistics or their absence in the $\psi(2S)p$ channel~\cite{Aaij:2016wxd}.


The production mechanism and the decay pattern imply a five-quark content of
these two states with three light quarks and a hidden heavy $c\bar{c}$ component
if they are hadronic states. In fact, pentaquark-like states with hidden charm
have been predicted in the right mass region as dynamically generated in
meson-baryon interactions a few years before the LHCb
discovery~\cite{Wu:2010jy,Wu:2010vk}.
There are several thresholds in the mass region of the two $P_c$ structures,
namely $\chi_{c1} p$, $\Sigma_c(2520)\bar{D}$, $\Sigma_c(2455)\bar{D}^*$,
$\Lambda_c(2595)\bar{D}$, and $\Sigma_c(2520)\bar{D}^*$ (see
Table~\ref{tab:baryon}), though not all of them couple in $S$-waves to the
reported preferred quantum numbers, suggesting different interpretations of the
two $P_c$ states, such as $\Sigma_c(2455)\bar{D}^*$, $\Sigma_c(2520)\bar{D}$ or
$\Sigma_c(2520)\bar{D}^*$ hadronic
molecules~\cite{Chen:2015loa,Chen:2015moa,Roca:2015dva,Chen:2016otp,
He:2015cea,Karliner:2015ina,Ortega:2016syt,Shimizu:2016rrd}. It has been suggested that their
decay patterns could be used to distinguish among various hadronic molecular
options~\cite{Wang:2015qlf,Lu:2016nnt,Shen:2016tzq,Lin:2017mtz}.
There are also other dynamical studies with different channel
bases~\cite{Xiao:2016ogq,Yamaguchi:2016ote,Azizi:2016dhy,Geng:2017hxc}.

The extreme closeness of the $P_c(4450)$ to the $\chi_{c1} p$ threshold and to a
TS from a $\Lambda^*(1890)\chi_{c1}p$ triangle diagram was pointed out in
Ref.~\cite{Guo:2015umn}. In Ref.~\cite{Bayar:2016ftu}, it is stressed  that the
$\chi_{c1}p$ needs to be in an $S$-wave so as to produce a narrow observable
peak in the $J/\psi p$ invariant mass distribution, and correspondingly $J^P$
needs to be $1/2^+$ or $3/2^+$. This TS and other possible relevant TSs
are also discussed in~\cite{Liu:2015fea,Guo:2016bkl}. It is worthwhile to
emphasize that the existence of TSs in the $P_c$ region does not exclude a 
possible existence of pentaquark states whether or not they are hadronic
molecules. In Ref.~\cite{Meissner:2015mza}, the possibility that the $P_c(4450)$
could be a $\chi_{c1}p$ molecule was investigated.

In order to confirm the existence of the two $P_c$ states and distinguish them
from pure kinematic singularities, {\sl c.f.} discussions in
Sec.~\ref{sec:6-ts}, one can either search for them in other processes, such as
$\Lambda_b\to\chi_{c1}p K^-$~\cite{Guo:2015umn},
photoproduction~\cite{Karliner:2015voa,Kubarovsky:2015aaa,
Wang:2015jsa,Kubarovsky:2016whd,Gryniuk:2016mpk,Blin:2016dlf,Huang:2016tcr},
heavy ion collisions~\cite{Wang:2016vxa}, pion-nucleon
reactions~\cite{Kim:2016cxr,Liu:2016dli,Lin:2017mtz}, or search for their
strange~\cite{Feijoo:2015kts,Feijoo:2015cca,Ramos:2016odk,Chen:2016heh,
Lu:2016roh}, neutral~\cite{Lebed:2015tna,Lu:2015fva} and
bottomonium~\cite{Xiao:2015fia} partners. The $\Lambda_b\to\chi_{c1}p K^-$ decay
process has been observed at the LHCb experiment~\cite{Aaij:2017awb}.


