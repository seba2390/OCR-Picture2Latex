
\section{Identifying hadronic molecules }
\label{sec:3}

Hadronic molecules are analogues of  light nuclei, most notably the deuteron. 
They can be treated to a good approximation as composite systems made of two 
or more hadrons which are bound together via the strong interactions.
In this section the general notion of a molecular state is introduced. As will
be demonstrated for near-threshold bound states this picture can be put into a
formal definition that even allows one to relate observables directly to the
probability to find the molecular component in the bound state wave function.
However, it appears necessary to work with a more general notion of hadronic
molecules as also resonances can be of molecular nature in the sense formulated
above. Before we proceed it appears necessary to review some general properties
of the $S$-matrix. In this subsection also the terminology of a bound state, a
virtual state and a resonance are discussed for those notions will be heavily
used throughout this review.


\subsection{Properties of the {\boldmath$S$}-matrix} 
\label{sec:Sproperties}

%-----------------------------------------------------------------------------------------------
\begin{figure*}
 \centering
   \includegraphics*[width=\linewidth]{./figures/sheets}
   \caption{Sketch of the imaginary part of a typical single--channel amplitude in 
the complex $s$-plane. The solid dots indicate allowed positions for resonance poles,
the cross for a bound state.
The solid line is the physical axis (shifted by $i\epsilon$ into the physical sheet).
The two sheets are connected smoothly along their discontinuities.
\label{sheets}}
\end{figure*}
%-----------------------------------------------------------------------------------------------

The unitary operator that connects asymptotic $in$ and $out$ states is called
the $S$-matrix.
It is an analytic function in the Mandelstam plane
 up to its branch points and poles.
 The $S$-matrix is the quantity that encodes all physics about a certain
 scattering or production reaction. In general it is
assumed that the $S$-matrix is analytic up to:
\begin{itemize}
 
\item {\it Branch points}, which occur at each threshold. On the one hand, there
are the so called right-hand cuts starting from the branch points at the
thresholds for an $s$-channel kinematically allowed process (e.g. at the $\bar
KK$ threshold in the $\pi \pi$ scattering amplitude). On the other hand, when
reactions in the crossed channel become possible one gets the left-hand cuts,
which are usually located in the unphysical region for the reaction studied but
may still influence significantly, e.g., the energy dependence of a reaction
cross section. Branch points can also be located inside the complex plane of 
the unphysical Riemann sheets: This is possible when the reaction goes via an 
intermediate state formed by one or more unstable states. It is clear that these
threshold branch points/cuts are determined kinematically and happen at the loop
level of Feynman diagrams.

In general, a loop Feynman diagram with more-than two intermediate particles has
more complicated kinematical singularities. They are called Landau
singularities~\cite{Landau:1959fi}, see,
e.g.,~\cite{Eden:1966,Chang:1983,Gribov:2009}.
For instance, in triangle diagrams the branch points of two intermediate pairs
can be very close to the physical region simultaneously, and such a situation
gives rise to the so-called {\it triangle singularity} already introduced
in Sec.~\ref{sec:2}. We come back to those in detail in
Sec.~\ref{sec:4-3ploop}.\footnote{The Landau singularity can even be a pole if
the one-loop Feynman diagram has at least five intermediate
particles~\cite{Gribov:2009}. However, this case is irrelevant for us and will
not be considered.}

\item {\it Poles}, which appear due to the interactions inherited in the
dynamics of the underlying theory. Depending on the locations, poles can be
further classified as follows:

-- {\it Bound states}, which appear as poles on the physical sheet. By causality
they are only allowed to occur on the real $s$-axis below the lowest threshold.
The deuteron in the isospin-0 and spin-1 proton-neutron system, which can be
regarded as the first established hadronic molecule, is a nice example.

-- {\it Virtual states}, which appear on the real $s$-axis, however, on the
unphysical Riemann sheet. A well-known example in nuclear physics is
the pole in the isospin-1 and spin-0 nucleon-nucleon scattering. It is within 1~MeV from the
threshold and drives the scattering length to a large value of about 24~fm.

-- {\it Resonances}, which appear as poles on an unphysical Riemann sheet close
to the physical one.  There is no restriction for the location of poles on the
unphysical sheets. Yet, Hermitian analyticity requires that, if there
is a pole at some complex value of $s$, there must be another pole at its
complex conjugate value, $s^*$.
Normally, the pole with a negative imaginary part is closer to the physical axis
and thus influences the observables in the vicinity of the resonance region more
strongly. However, at the threshold both poles are always equally important.
This is illustrated in Fig.~\ref{sheets}.

\end{itemize}


For a discussion of the analytic structure of the $S$-matrix with focus on
scattering experiments we refer to Ref.~\cite{Doring:2009yv} and references
therein.
Any of these singularities leads to some structure in the observables.
In a partial-wave decomposed amplitude additional singularities not related to
resonance physics may emerge as a result of the partial-wave projection. For a
discussion see, e.g., Ref.~\cite{hoehler}.

From the above classification, it is clear that the branch points are
kinematical so that they depend completely on the masses of the involved
particles in a certain physical process, while the poles are of dynamical origin
so that they should appear in many processes as long as they are allowed by
quantum numbers. 

We will call a structure observed experimentally a state if and only if the
origin of this structure is a pole in the $S$-matrix due to dynamics.
On the one hand this definition is quite general as it allows us to also call
the above mentioned pole in the isovector nucleon-nucleon scattering a state.
From the point of view of QCD, this definition appears to be quite natural since
it takes only a marginal change in the strength of the two-hadron potential
(e.g. via a small change in quark masses) to switch from a shallow bound state
to a near-threshold virtual state, and both leave a striking imprint in
observables (we come back to this in Sec.~\ref{sec:lineshapes}). 
{For example, various lattice QCD groups have observed the di-neutron to become a bound
state at quark masses heavier than the physical
value~\cite{Beane:2012vq,Yamazaki:2012hi,Yamazaki:2015asa,Berkowitz:2015eaa}.}
On the other hand, if a structure in the data finds its origin purely in a
kinematical singularity without a nearby pole, it would not be called a state.
There is currently a heated discussion going on in the literature whether some
of the $XYZ$ states are just threshold cusps or triangle
singularities~\cite{Bugg:2004rk,Chen:2011pv,Chen:2013coa,Swanson:2014tra,
Swanson:2015bsa,Pilloni:2016obd,Gong:2016jzb}.
It should be stressed, however, that pronounced near-threshold signals in the
continuum channel related to that threshold must find their origin in a nearby
pole~\cite{Guo:2014iya}.

 {In the physical world, }basically all candidates for hadronic molecules,
 {except for nuclei,} can decay strongly and thus can not be bound
 states in the rigorous sense of the word, since the lowest threshold is defined
 by the production threshold of the decay products.
 However, it still appears justified to e.g. call the $f_0(980)$ a $K\bar K$
 bound state, {or a quasi-bound state in a more rigorous sense,}
  if the corresponding pole is located on the physical sheet for
 the two-kaon system, or a virtual state if it is on the unphysical sheet for
 the two-kaon system, although the lowest threshold is the two-pion threshold.
 

\subsection{Definition of hadronic molecules}
\label{sec:wein}

 In order to proceed it is necessary to first of all define the notion of a
 molecular state. Naively one
 might be tempted to argue that if data can be described by a model 
where all interactions between continuum states come from 
  $s$-channel pole terms the resulting states have to be interpreted as
 ``elementary'' states. However, as we will discuss below, this is in
 general not correct.
 Analogously, a model that contains only non-pole interactions can still at the
 end lead to a pole structure of the $S$-matrix that needs to be interpreted as
 non-molecular. The origin of  the failure of intuition in these circumstances
 is the fact that a hadronic description of hadron dynamics can only be
 understood in the sense of an effective field theory with limited range of
 applicability. In particular the very short-ranged parts of the wave function
 as well as the interaction potential are model-dependent and can not be
 controlled within the hadronic prescription. 

 However, at least for near-threshold bound states (the term ``near'' will be
 quantified in the next subsection) there is a unique property of the wave
 function of a molecular state as long as it is formed by a (nearly) stable
 particle pair in an $S$-wave: The very fact that this particle pair can almost
 go on-shell leaves an imprint with observable consequences in the analytic
 structure of the corresponding amplitude, a feature absent to all other
 possible substructures. In fact, as a consequence of this feature, hadronic
 molecules can be very extended. To see this observe that a bound state wave
 function at large distances scales as $\exp(-\gamma r)/r$, where $r$ is the
 distance between the constituents and $\gamma$ denotes the typical momentum
 scale defined via
 \begin{equation}
 \gamma = \sqrt{2\mu E_B} \ ,
 \label{eq:gamdef}
 \end{equation}
 where $\mu=m_1m_2/(m_1+m_2)$ denotes the reduced mass of the two-hadron system
 and 
 \begin{equation}
  E_B = m_1+m_2-M
  \label{eq:Ebdef}
 \end{equation}
 the binding energy of the state with mass $M$ (note that we
 chose $E_B$ positive so that the bound state is located at $E=-E_B$ {with
 $E$ the energy relative to the threshold}).
 Thus, the size $R$ of a molecular state is given by $ R\sim 1/\gamma$.
  Accordingly, if $X(3872)$ with a binding energy of less than 200~keV with
  respect to the
 $D^0\bar D^{*\, 0}$ threshold were a molecule, it would be at least as large as
 10~fm.
 For a review of properties of systems with large scattering length we refer to
 Ref.~\cite{Braaten:2004rn}.
 
 

All these issues will be discussed in detail in the sections below. The
arguments will start in Sec.~\ref{sec:weinberg} from the classic definition
introduced by Weinberg long ago to model-independently capture  the nature of
the deuteron as a proton-neutron bound state. A detailed discussion of the
derivation will allow us to explain at the same time the limitations of this
definition.
Then in Sec.~\ref{sec:polecounting} it is demonstrated that the Weinberg
criterion is actually identical to the pole counting arguments by Morgan.
In Sec.~\ref{sec:poletrajectories} the generalization of the arguments to
resonances is prepared by a detailed discussion of pole trajectories that emerge
when some strength parameter that controls the location of the the $S$-matrix
poles varies. The compositeness criteria for resonances are very briefly
discussed in Sec.~\ref{sec:resonances}.

 \subsubsection{The Weinberg compositeness criterion}
 \label{sec:weinberg}

We start from the following ansatz for the physical wave function of a bound
state~\cite{Weinberg:1965zz} :
 \begin{equation}
|\Psi \rangle = \left(\lambda|\psi_0\rangle\atop
\chi (\bm{k}) |h_1h_2 \rangle \right),
\end{equation}
 where $|\psi_0\rangle$ denotes the compact component of the state and  $|h_1h_2
 \rangle$ its two-hadron component.
  Here compact denotes an object whose size is controlled by the
  confinement radius $R_{\rm conf.}< 1$ fm. Thus this component is assumed to be more
  compact than $R\sim 1/\gamma$, which denotes the characteristic size of a shallow bound
  state~\footnote{Actually, we may define the notion ``shallow'' by the request
  the $R>R_{\rm conf.}$, which translates into $E_B<1/(2\mu R_{\rm conf.}^2)$.}
  In addition, $\chi (\bm{k})$ is the wave
 function of the two-hadron part, where $\bm{k}$ denotes the relative 
 momentum of the two particles. 
 In this parameterization, by definition, $\lambda$ quantifies the contribution 
 of the compact component of the wave function to the physical wave function of 
 the state. Accordingly $\lambda^2$ denotes the probability to find the
 compact component of the wave function in the physical state, which
 corresponds to the wave function renormalization constant $Z$ in quantum field
 theory. Thus, the goal is to relate $\lambda$ to observables.

 In order to proceed one needs to define the interaction Hamiltonian. As shown 
 by Weinberg in Ref.~\cite{Weinberg:1963zza} under very general conditions one 
 may write
 \begin{equation}
 \hat {\mathcal H} |\Psi\rangle = E |\Psi\rangle,~~
\hat {\mathcal H} = \left(\begin{array}{cc}
\hat{H}_c&\hat{V}\\
\hat{V}&\hat{H}_{hh}^0
\end{array}
\right) .
 \end{equation}
 This expression exploits the observation that it is possible by a proper field 
 redefinition to remove all hadron-hadron interactions from the theory and to 
 cast them into $\psi_0$~\cite{Weinberg:1962hj,Weinberg:1963zza}. Then the 
 two-hadron Hamiltonian is given simply by the kinetic term $\hat{H}_{hh}^0=k^2/
 (2\mu)$, where $\mu = m_1m_2/(m_1+m_2)$
 denotes the reduced mass of the two-hadron system and $m_i$ the mass of hadron
 $h_i$.
 Introducing the transition form factor,
 \begin{equation}
 \langle \psi_0|\hat {V}|h_1 h_2 \rangle = f(\bm k),
 \end{equation}
 one finds the wave function in the momentum space as
 \begin{equation}
 \chi(\bm k)=\lambda \, \frac{f(\bm k)}{E-k^2/(2\mu)} \ .
 \end{equation}
 The wave function of a physical bound state needs to be normalized to have a
 probabilistic interpretation.
 We thus get \begin{eqnarray}\nonumber
1&=&\langle \Psi|\Psi\rangle = \lambda^2 \langle \psi_0|\psi_0\rangle {+}\!\!\! 
\int \!\! \frac{d^3k}{(2\pi)^3} |\chi(\bm k)|^2 \langle h_1h_2|h_1h_2\rangle \\
&=&  \lambda^2 \left\{1
+\int \frac{d^3k}{(2\pi)^3} \frac{f^2(\bm k)}{\left[E_B+k^2/(2\mu)\right]^2}
\right\}  .
\label{eq:norm}
 \end{eqnarray}
As mentioned above, $\lambda^2$ is in fact the wave function renormalization
constant $Z$, since the integral in the last line of Eq.~(\ref{eq:norm}) is 
nothing but the energy derivative of the self-energy. Because of the positivity 
of the integral, $\lambda^2$ is bound in the range between 0 and 1, and thus 
allows for a physical probabilistic interpretation for a bound state.

At this point a comment is necessary: in many textbooks on quantum field theory
it is written that the wave function renormalization constant $Z$ is scheme
dependent and is to be used to absorb the ultraviolet (UV) divergence of the
vertex corrections. Clearly this is correct. However, the scheme dependence and
UV divergence are only for the terms analytic in $E$.
What we find here is the LO piece of $Z$ in an energy expansion
around the threshold,\footnote{More discussion on this point can be found in
Sec.~\ref{sec:4-interactions}.} and as this
piece is proportional to $\sqrt{E}$ it can not be part of the Lagrangian.
Thus, the Weinberg criterion as outlined is based explicitly on the presence of
the two-particle cut which is responsible for the appearance of the square
root, whose presence is a distinct feature of the two-hadron component.

 The integral in Eq.~\eqref{eq:norm} converges if $f(\bm k)$ is a constant. The
 denominator contains solely model independent parameters, while the momentum
 dependence of the numerator is controlled by the relevant momentum range of the
 vertex function that may be estimated by $\beta$, the inverse range of forces.
 Thus, if $\beta \gg \gamma$,  the integral can be evaluated model-independently
 for the case of $S$-wave coupling which implies the constant $g_0=f(0)$ as the
 LO piece of $f(\bm k)$.\footnote{Note that in some works a model for
 the form factor $f(\bm k)$ is employed~\cite{Faessler:2007gv}.} Then one finds
  \begin{equation}
 1 = \lambda^2 \left[1 +\frac{\mu^2g_0^2}{2\pi\sqrt{2\mu E_B}}+ {\cal
 O}\left(\frac{\gamma}{\beta}\right)\right] .
 \end{equation}
From this
 we find the desired relation, namely
 \begin{equation}
 g_0^2 = \frac{2\pi\gamma}{\mu^2}  \left(\frac{1}{\lambda^2}-1\right)  ,
 \label{eq:gunrenorm}
 \end{equation}
which provides a relation between $\lambda^2$, the probability of finding the 
compact component of the wave function inside the physical wave function, and 
$g_0$, the bare coupling coupling constant of the physical state to the 
continuum, or $\lambda g_0$, the physical coupling constant.

The quantity $g_0$ appears also in the physical propagator of the bound state
since the self energy is given by 
\begin{eqnarray}\nonumber
\Sigma(E) &=& - \int \frac{d^3 k}{(2\pi)^3}
\frac{f^2(\bm k)}{E-k^2/(2\mu)+i\epsilon}
\\
&=& \Sigma(-E_B) + i\,g_0^2\frac{\mu}{2\pi}\sqrt{2\mu E+i\epsilon}  +
{\mathcal O}\left(\frac{\gamma}{\beta}\right)  .~~
\end{eqnarray}
We may therefore write for the $T$-matrix of the two continuum particles whose
threshold is close to the location of the bound state,
\begin{equation}
T_\text{NR}(E) = \frac{g_0^2}{E-E_0+\Sigma(E)} + \mbox{non-pole terms} \, ,
\end{equation}
where the subscript ``NR'' is a reminder of the nonrelativistic normalization
used in the above equation.
As long as the pole is close to the threshold, the amplitude near threshold 
should be dominated by the pole term (the non-pole terms are again controlled 
quantitatively by the range of forces). Using $ E_B = -E_0 + \Sigma(-E_B) - 
g_0^2\mu\gamma/(2\pi) $, which absorbs the (divergent) leading contribution of 
the real part into the bare pole energy and at the same
time takes care of the fact that the analytic continuation of the momentum term
also contributes at the pole,\footnote{When $E$ takes real values,
 the square root on the first sheet is defined by $\sqrt{2\mu E+i\epsilon} =
 +i\sqrt{-2\mu E}\, \theta(-E) + \sqrt{2\mu E}\,\theta(E) $.
 } we get
\begin{equation}
T_\text{NR}(E) = \frac{g_0^2}{E+E_B+g_0^2 \mu/(2\pi) (i k+\gamma)} \ ,
\label{eq:poleTmatrix}
\end{equation}
where we have introduced the two-hadron relative momentum $k=\sqrt{2\mu E}$.
Note that Eq.~\eqref{eq:poleTmatrix} is nothing but the one channel version of
the well-known Flatt\'e parametrization~\cite{Flatte:1976xu}.
  Thus, a measurement of near-threshold data allows one in principle to measure
 the composition of the bound state wave function, in line with the effective
 field theory analysis to be discussed later in Sec.~\ref{sec:6}, although in
 practice a reliable extraction of the coupling might be hindered by a scale invariance of the
 Flatt\'e parametrization that appears for large couplings~\cite{Baru:2004xg}.
 The phenomenological implications especially of Eq.~\eqref{eq:gunrenorm} on
 Eq.~\eqref{eq:poleTmatrix} and generalizations thereof will be discussed in
 Sec.~\ref{sec:lineshapes}.

 To make the last statement explicit we may match Eq.~(\ref{eq:poleTmatrix})
 onto the effective range expansion
 \begin{equation}
 T_\text{NR}(E)=- \frac{2\pi}{\mu} \frac{1}{1/a + (r/2)k^2-ik} \ ,
 \label{eq:ere}
 \end{equation}
 and find
 \begin{eqnarray} \nonumber
 a &=& - 2\, \frac{1-\lambda^2}{2-\lambda^2}\left(\frac1{\gamma}\right)+
 {\mathcal O}\left(\frac1\beta\right) , \\
  r&=& -\frac{\lambda^2}{1-\lambda^2} \left(\frac1{\gamma}\right)+ {\mathcal
  O}\left(\frac1\beta\right) .
  \label{eq:arwein}
 \end{eqnarray}
 Thus, for a pure molecule ($\lambda^2=0$) one finds that the scattering
 length gets maximal, $a=-1/\gamma$, and in addition $r={\mathcal O}(1/\beta)$,
 where the latter term is typically positive, while for a compact state 
 ($\lambda^2=1$) one gets $a=-{\mathcal O}(1/\beta)$ (in the presence of a bound
 state the scattering length is necessarily negative within the sign convention 
 chosen here) and $r\to -\infty$.
 These striking differences have severe implications on the line shapes of
 near-threshold states as will be discussed in detail in 
 Sec.~\ref{sec:lineshapes}.
 
 It is illustrative to apply the Weinberg criterion to the deuteron, 
 basically repeating the analysis presented already in Ref.~\cite{Weinberg:1965zz}.
 The scattering length and effective range
 extracted from proton-neutron scattering data in the deuteron channel
 are~\cite{Klarsfeld:1984es}
 \begin{equation}
a = -5.419(7) \ \mbox{fm \ and \ } r=1.764(8) \ \mbox{fm} \ ,
\label{arexp}
\end{equation}
where the sign of the scattering length was adapted to the convention employed here.
Furthermore, the deuteron binding energy reads~\cite{VanDerLeun:1982bhg}~\footnote{The
reference quotes $E_B=2.224575(9)$ MeV, however, for the analysis here such a high accuracy is
not necessary.}
  \begin{equation}
E_B= 2.22 \ \mbox{MeV} \ \Longrightarrow \ \gamma=45.7 \ \mbox{MeV} = 0.23 \ \mbox{fm}^{-1} \ .
\end{equation}
On the other hand, in case of the deuteron the range of forces is
provided by the pion mass --- accordingly the range corrections
that appear in Eqs.~(\ref{eq:arwein}) may in this case be estimated via
\begin{equation}
\frac1{\beta}\sim \frac1{M_\pi}\simeq 1.4 \ \mbox{fm} \ .
\end{equation}
Thus the effective range is of the order of the range corrections (and positive!)
 as required by the compositness criterion for a molecular state.
Using $\lambda^2=0$ in the expression for the scattering length we find
\begin{equation}
a_{\rm mol.} = -(4.3\pm 1.4) \ \mbox{fm} \ 
\end{equation}
also consistent with Eq.~(\ref{arexp}). Based on these considerations Weinberg
concluded that the deuteron is indeed composite.
 
 
 

 As mentioned previously, a location of a molecular state very near a threshold
 is quite natural, while a near-threshold compact state is
 difficult to accomplish~\cite{Jaffe:2007id,Hanhart:2014ssa}. This can now be 
 nicely illustrated on the basis of Eqs.~(\ref{eq:ere}) and (\ref{eq:arwein}). 
 By construction for $k=i\gamma$ the $T$ matrix develops a pole which may be
 read off from Eq.~(\ref{eq:ere})
 \begin{equation}
 \gamma = -\frac{1}{a} + \frac{\gamma^2 r}{2} \, .
 \label{eq:polecond}
 \end{equation}
 For a (nearly) molecular state $a\simeq -1/\gamma$ and $r\sim {\mathcal
 O}(1/\beta)$. Thus, for this case Eq.~(\ref{eq:polecond}) is largely saturated 
 by the scattering length term, and the range term provides only a small
 correction. However, for a predominantly genuine state we have $-1/a\simeq 
 \beta \gg \gamma$ and $r\to -\infty$. Thus in this case a subtle fine tuning 
 between the range term and the scattering length term appears necessary for the
 pole to be located very near threshold.


 While the low-energy scattering of the hadrons that form the bound state is
 controlled by scattering length and effective range, production reactions are
 sensitive to the residue of the bound state pole, which serves as the effective
 coupling constant,  to be called $g_{\rm eff}$, squared of the bound state to
 the continuum.  It is simply given by the bare coupling constant $g_0^2$
 introduced above multiplied by the wave function renormalization constant $Z$,
 which is $\lambda^2$  as explained above,
\begin{equation}
  g_\text{NR}^2\equiv{Z} g_0^2   = \frac{2\pi\gamma}{\mu^2} (1-\lambda^2) \, .
  \label{eq:residue_bs}
\end{equation}
 After switching to a relativistic normalization by multiplying
 with  $\left(\sqrt{2m_1}\sqrt{2m_2}\sqrt{2M}\right)^2$, and
 dropping terms of order $(E_B/M)$, we thus get
 \begin{equation}
 \frac{g_{\rm eff}^2}{4\pi}= 4 M^2\left(\frac{\gamma}{\mu}\right) 
 \left(1-\lambda^2\right)  .
 \label{eq:residue}
 \end{equation}
What is interesting about this equation is that it is bounded from above: The
effective coupling constant of a bound system to the continuum gets maximal for 
a pure two-hadron bound state. Since $1-\lambda^2$ is the probability of finding
the two-hadron composite state component in the physical wave function, it is
sometimes called ``compositeness''. Using Eq.~\eqref{eq:arwein}, the effective
coupling can be expressed in terms of the scattering length
\begin{equation}
  \frac{g_{\rm eff}^2}{4\pi}= \frac{4 M^2}{\mu}\frac{{-a\gamma}}{a+2/\gamma}\,,
  \label{eq:ga}
\end{equation}
which reduces to $-4M^2/(\mu a)$ in the limit of $\lambda^2=0$, reflecting the 
universality of an $S$-wave system with a large scattering 
length~\cite{Braaten:2004rn}.

Before closing this section some comments are necessary.
\begin{itemize}
\item The approach allows for model-independent statements only for $S$-waves,
since otherwise in the last integral of Eq.~(\ref{eq:norm}) there appears in the
numerator of the integrand an additional factor $k^{2L}$ from the centrifugal 
barrier. Accordingly, the integral can no longer be evaluated
model-independently without introducing additional parameters (regulator) to 
cope with the UV divergence.

\item For the same reason the continuum channel needs to be a two-body channel,
since otherwise the momentum dependence of the phase space calls for an 
additional suppression of the integrand. 

\item The binding momentum must be small compared to the inverse range of
forces, since otherwise the range corrections get larger than the terms that
contain the structure information.

\item For the applicability  of the formalism as outlined and an unambiguous
probabilistic interpretation, the state studied must be a bound state, since
otherwise the normalization condition of Eq.~(\ref{eq:norm}) is not applicable
which is at the very heart of the derivation.
However, nowadays there exist generalizations of the Weinberg approach also to
resonances which will be discussed in Sec.~\ref{sec:resonances}.

\item The constituents that form the bound state must be
narrow, since otherwise the bound system would also be
broad~\cite{Filin:2010se,Guo:2011dd}.
\end{itemize}

For long it seemed that the conditions are satisfied only by the deuteron and
Weinberg therefore closed his paper with the phrase~\cite{Weinberg:1965zz}:
``One begins to suspect that Nature is doing her best to keep us from learning
whether the `elementary' particles deserve that title.'' However, as outlined in
the introduction, there are now various near-threshold states confirmed
experimentally that appear to be consistent with those criteria, like $X(3872)$,
$D_{s0}^*(2317)$ and less rigorously $f_0(980)$ and others.

For illustration we would like to compare what is known about the effect of the
$D_{s0}^*(2317)$ on $DK$ scattering to the Weinberg criterion. Clearly, $DK$
scattering can not be measured directly in experiment, however, it can be
studied in lattice QCD using the so-called L\"uscher
method~\cite{Luscher:1990ux}.
A first study using this method for the $DK$ system is presented in
Ref.~\cite{Mohler:2013rwa}.
The scattering length and effective range extracted in this work for the lowest
pion mass ($M_\pi=156$ MeV) are $-(1.33\pm 0.20)$~fm and $(0.27 \pm 0.17)$~fm,
respectively. This number is to be compared to the Weinberg prediction for a
purely molecular state of $a=-(1\pm 0.3)$~fm and $r\sim 0.3$~fm, where the
inverse $\rho$-mass was assumed for the range of forces and we used that for
molecular states the effective range is positive and of the order of the range
of forces.
Scattering lengths of the same size were also extracted from a study of the
scattering of the light pseudoscalars off $D$-mesons using unitarized chiral
perturbation theory~\cite{Liu:2012zya}. Thus from both chiral dynamics on the
hadronic level as well as lattice QCD there are strong indications that
$D_{s0}^*(2317)$ indeed is a $DK$ molecule. The lattice aspects will be further
discussed in Sec.~\ref{sec:lattice}. Clearly, a direct experimental confirmation
of the molecular assignment for $D_{s0}^*(2317)$ is very desirable. A possible
observable could be the hadronic width of $D_{s0}^*(2317)$ as discussed in
Sec.~\ref{sec:isospinviol}.


So far we only focused on bound states. However, also very near-threshold poles
on the second sheet not accompanied by a first sheet pole, so-called virtual
states, leave a striking imprint in observables, {\sl c.f.}
Sec.~\ref{sec:Sproperties}.
A $T$-matrix that has its pole on the second, instead of on the first, sheet
reads, in distinction to Eq.~(\ref{eq:poleTmatrix}),
\begin{equation}
T_\text{NR}(E) = \frac{g_0^2}{E+E_v+ g_0^2 \mu/(2\pi) (i k-\gamma)} \, ,
\label{eq:poleTmatrix2}
\end{equation}
where now the virtual pole is located on the second sheet at $E=-E_v$, with
$E_v>0$ and we still use $\gamma$ to denote $\sqrt{2\mu E_v}$.
Here we use that on the second sheet below threshold the momentum is 
$-i|\sqrt{2\mu E}|$. 


\subsubsection{The pole counting approach}
\label{sec:polecounting}

One of the classic approaches put forward to distinguish molecular states from
genuine ones is the so-called pole counting approach~\cite{Morgan:1992ge},
which may be summarized as: A bound state that is dominated by its compact
component (in the language of the previous section this implies $\lambda^2$
close to 1) manifests itself in two near--threshold singularities (one on the
first sheet, one on the second) while a predominantly molecular bound state
gives rise only to a single near-threshold pole on the first sheet.

To see that these criteria actually map perfectly on the Weinberg criterion it
is sufficient to observe that the poles of Eq.~(\ref{eq:ere}) are given by
\begin{equation}
k_{1/2} = \frac{i}{r} \pm \sqrt{-\frac{1}{r^2}-\frac{2}{ar}} \ .
\end{equation}
{Based on the sign convention employed in this work, cf. Eq.~(\ref{eq:ere}), in the presence of a bound state the scattering length is negative. In
addition, keeping only the leading terms for both the scattering length $a$ and the effective range $r$ as shown in 
Eqs.~(\ref{eq:arwein}), one obtains 
\begin{equation}
k_1 = i\gamma \ , \quad k_2=-i \gamma\left(\frac{2-\lambda^2}{\lambda^2}\right) 
\ .
\end{equation}
Thus
It is easy to see that $k_1$ and $k_2$ are positive and negative 
imaginary numbers, respectively, which implies that the former is a pole on the first Riemann
sheet (a bound state pole), while the latter is located on the second sheet.
When $\lambda$ approaches 0, which implies that the molecular component of the
state becomes increasingly important, the second pole disappears towards negative imaginary infinity,
which leaves $k_1$ as the only relevant pole.
} In particular one gets from this for the asymmetry of the pole locations
\begin{equation}
\frac{|k_1|-|k_2|}{|k_1|+|k_2|} = \lambda^2-1 \ .
\end{equation}
Thus the asymmetry of the pole locations is a direct measure of the amount of 
molecular admixture in the bound state wave function (as defined within the 
Weinberg approach) in line with the findings of Ref.~\cite{Morgan:1992ge}.
The close relation between the two approaches was first observed in 
Ref.~\cite{Baru:2003qq}.



\subsubsection{Remarks about pole trajectories}
\label{sec:poletrajectories}

QCD is characterized by a small number of parameters, namely the quark masses,
the number of colors ($N_c$) and $\Lambda_{\rm QCD}$ (the running coupling constant).
Accordingly those parameters determine completely the hadron spectrum. With
advanced theoretical tools it became possible recently to investigate the
movement of the QCD poles as QCD parameters are varied. There exist studies for
varying quark masses as well as varying numbers of colors, $N_c$,---both of them
allowing for deeper insights into the structure of the investigated states.

Studies that vary the number of colors are available mostly for light quark
systems. For a recent review, we refer to Ref.~\cite{Pelaez:2015qba}. In order
to connect the $N_c$ dependence of a given state in the spectrum to QCD in
Ref.~\cite{Pelaez:2015qba} a unitarized version of chiral perturbation theory,
the so-called inverse amplitude method, is employed, where the fact is exploited
that the leading $N_c$ behavior of the low-energy constants (LECs) is known.
Thus, once the unitarized amplitudes are fitted, e.g., to phase shifts it is
possible to investigate the impact of a varying $N_c$ by proper rescaling of
these LECs.
This kind of study was pioneered by the work reported in
Ref.~\cite{Pelaez:2003dy}, where it was demonstrated that the $N_c$ scaling of
the vector mesons $\rho$ and $K^*$ is in line with expectations for $\bar qq$
states, however,  that of $f_0(500)$  and $K_0^*(800)$ is completely at odds
with them. While the $N_c$ studies allow one to distinguish the quark content of
different states, they do not allow one to disentangle hadronic molecules from
other four-quark structures. In addition, both mentioned resonances are very
broad and as such do not allow one to straightforwardly quantify their molecular
component following the approach of the previous section.

Existing studies where quark masses are varied allow for a more direct contact
to the discussion of the previous section.
Understanding the quark mass dependence of hadrons with different composition is
not only interesting on its own sake, it is also important since lattice QCD
studies can be performed at arbitrary quark masses (and in fact are often
performed at enlarged quark masses {for practical reasons}).
Thus, as soon as we can relate certain pole trajectories to the structure of
the hadron it becomes feasible to ``measure'' the nature of the state using
lattice QCD.
More direct methods to use the lattice to determine the nature of certain
hadrons will be discussed in Sec.~\ref{sec:lattice}.

Let us consider pole trajectories of resonances as some generic strength
parameter is varied.
Here we follow the presentation in Ref.~\cite{Hanhart:2014ssa}. In this work it
is shown that in the presence of a pole the one-channel $S$-matrix can be
written as (for simplicity assuming the masses of the continuum particles to be
equal) \begin{eqnarray} \nonumber S &=&
\frac{(k-k_p-i\xi)(k+k_p-i\xi)}{(k-k_p+i\xi)(k+k_p+i\xi)} \\
&=&  \nonumber \frac{k^2-(k_p^2+\xi^2)-2ik\xi}{k^2-(k_p^2+\xi^2)+2ik\xi} \\
&=&\frac{s-s_0-4i(s-4m^2)^{1/2}\xi}{s-s_0+4i(s-4m^2)^{1/2}\xi} \ ,
\label{eq:Spole}
\end{eqnarray}
where $\xi\geq0$. {The unimodular form of Eq.~\eqref{eq:Spole} is because
of unitarity of the $S$-matrix.} The above parameterization accounts for the
facts that if the $S$-matrix has a pole at some complex momentum on the second
sheet, $k_p-i\xi$, it also has to have a pole $-k_p-i\xi$, which is the
realization of the Schwarz reflection principle in momentum space, and that any
pole on the second sheet is accompanied by a zero on the first.
As shown by the last equality, the momentum space expression can be
straightforwardly mapped onto the $s$-plane, where $s_0=4(k_p^2 + \xi^2 + m^2)$
was introduced. In the $s$-plane the Schwarz reflection principle calls for
poles at complex conjugate points.

%------------------------------------------------------------------------------- 
\begin{figure} 
 \centering
   \includegraphics*[width=\linewidth]{./figures/trajectories}
   \caption{Typical pole trajectories for $S$-waves  (red, solid lines) and for 
higher partial waves (blue, dashed lines)
   in the second sheet of the complex $s$-plane. The thick line denotes the 
branch cut.}\label{fig:poletraj}
\end{figure}
%-------------------------------------------------------------------------------
To investigate the general behavior of the pole trajectories it is sufficient 
to vary the parameter $k_p^2$ from some finite positive
value to some finite negative value. Typical trajectories are shown in 
Fig.~\ref{fig:poletraj}.  The trajectories for $S$-waves
are depicted by the solid lines and for higher partial waves by the dashed 
lines. As long as $k_p^2$ is positive ($k_p$ is real),
Eq.~(\ref{eq:Spole}) develops two complex conjugate poles for all partial waves. 
When $k_p^2$ gets decreased, the poles approach each
other and eventually, for $k_p^2=0$, meet on the real axis. One of the poles 
switches to the first sheet at the
point were $k_p^2+\xi^2=0$, which is the threshold and at least for $S$-waves 
requires a negative value of $k_p^2$ ($k_p$ is imaginary).


The first nontrivial observation that can be read off Eq.~\eqref{eq:Spole} and
Fig.~\ref{fig:poletraj} straightforwardly is that $S$-waves and higher partial
waves behave very differently: The reason for this is the centrifugal barrier
that forces one to introduce a momentum dependence into $\xi$ according to
\begin{equation}
\xi(k) = \tilde \xi k^{2L} \ .
\end{equation}
This has a striking impact on the pole trajectories: For any $L>0$ the 
$\xi$-term is zero at $k=0$ and therefore
the point where the two pole trajectories  meet ($k_p^2=0$) coincides with 
$k=0$ which denotes the threshold. This is different for $S$-waves:
in this case the poles can meet somewhere below the threshold. Then, when 
$k_p^2$ is decreased further to negative 
values, both poles move away from the meeting point, such that one approaches 
the threshold while the other one goes away
from the threshold. This behavior can easily be interpreted via the pole 
counting approach: The further
away from the threshold the point is located where the two trajectories meet 
(this point is determined by the value of 
$\xi$ at the point where $k_p^2=0$), the more asymmetric are the two poles once 
one of them has switched to the first sheet, and thus the  molecular component of
 the state is more pronounced.

To make more explicit the above connection between the trajectories and the
molecular nature of the states we may study the scattering length and the
effective range that emerge from Eq.~(\ref{eq:Spole}):
\begin{equation}
a=-\frac{2\xi}{\xi^2+k_p^2} \ ; \quad r=-\frac{1}{\xi} \ .
\end{equation}
If we now use that the binding momentum $\gamma = \kappa_p-\xi$, where we 
introduced $\kappa_p=i k_p$, we can read off from the above equations and
Eq.~\eqref{eq:arwein}
\begin{equation}
\lambda^2 = 1-\frac{\xi}{\kappa_p} \ .
\end{equation}
To see the implications of this expression we may parameterize the relevant
quantities via
\begin{equation}
\gamma = \epsilon \delta \ ; \ \ \xi = \delta \ ; \ \ \kappa_p=(1+\epsilon)\delta \ \longrightarrow \lambda^2=\frac{\epsilon}{1+\epsilon} \ ,
\end{equation}
where $\delta>0$ and $\epsilon>0$.
Here it was already used that for a bound state to exist with a finite
binding energy $\kappa_p$ must exceed $\xi$. A vanishing binding momentum
($\gamma\to 0$) can be achieved either by $\epsilon\to 0$ for finite $\delta$, 
which immediately implies that $\lambda^2\to 0$, so that the state is purely
molecular, or $\delta\to 0$ for finite $\epsilon$. This case allows for a 
compact admixture of the very near-threshold pole, however, at the price of an 
extreme fine tuning of $\kappa_p$ and $\xi$ as both then need to go to zero 
simultaneously. This is the same kind of fine tuning already observed for 
non-molecular near-threshold states in Sec.~\ref{sec:wein} from a different
perspective.

As an example in Ref.~\cite{Hanhart:2014ssa} it was demonstrated explicitly that
the pole trajectories of the $f_0(500)$ meson as well as the $\rho$
meson that emerge when the quark masses are varied can be easily parameterized
in terms of the parameters $\xi$ and $\kappa_p$ introduced above.
In particular it was shown that while $\xi$ changes only mildly in the parameter
range studied $k_p^2$ changes a lot and in particular $\xi$ is sizable at the
point where $k_p^2$ is zero in the $f_0(500)$ channel. Accordingly the authors
of Ref.~\cite{Hanhart:2014ssa} conclude that, at least for unphysically large
quark masses, the $f_0(500)$ meson behaves like a hadronic molecule.

What should be clear from the considerations above is that states born off
hadron-hadron dynamics with poles above the relevant threshold are necessarily
broad; after all their coupling to this continuum channel is maximal.
This should also be clear from the pole trajectories illustrated in
Fig.~\ref{fig:poletraj}. An example for such a scenario is the very broad
$f_0(500)$ most probably generated by nonperturbative $\pi\pi$ interactions.
Such a property is in contrast to the tetraquark picture advocated in
Ref.~\cite{Esposito:2016itg}, where the authors argue that tetraquarks that are
visible in experiment must be narrow and slightly above threshold. It is
therefore of utmost importance that the pole locations of exotic candidates are
determined with high precision.


\subsubsection{Generalizations to resonances}
\label{sec:resonances}


The first work where the Weinberg approach was generalized to resonances is
Ref.~\cite{Baru:2003qq}, where the spectral density was employed to supplement
the parameter $\lambda^2$ introduced above for bound states.
The subject was later elaborated in various
papers~\cite{Hyodo:2011qc,Aceti:2012dd,Hyodo:2013iga,Hyodo:2013nka,
Sekihara:2014kya,Sekihara:2016xnq,
Guo:2015daa,Xiao:2016dsx,Xiao:2016wbs,Kang:2016ezb,Xiao:2016mon}; however, what
is common to all of them is that a quantitative, probabilistic extraction of the level of compositeness is not possible rigorously as soon as one moves to
resonances. The reason is that states that belong to poles on the second sheet
are not normalizable and as such one looses the condition of Eq.~(\ref{eq:norm})
that is crucial for the probabilistic interpretation.

However, it still appears reasonable to take over the key finding of
Sec.~\ref{sec:wein}, namely that the coupling of a state is larger for a larger
molecular component, also to resonance states.
As we will see in the following paragraphs, in certain situations this leads to
quite striking observable consequences. When being translated to an effective
field theory this observation implies that for molecular states loop
contributions appear always already at LO. The resulting power
counting will be detailed in Sec.~\ref{sec:4}.

When the state of interest is located below the production threshold of the
two hadrons that possibly form the molecular state, one can still distinguish
between quasi-bound states and virtual states, depending on whether the leading
pole is located on the first or the second sheet with respect to the mentioned
two-hadron system. The phenomenological implications of this will be discussed
in some detail in Sec.~\ref{sec:lineshapes}.



\subsection{Characteristic line shapes of hadronic molecules}
\label{sec:lineshapes}


%-----------------------------------------------------------------------------------------------
\begin{figure} 
 \centering
   \includegraphics*[width=\linewidth]{./figures/lineshapes}
   \caption{Sketch of typical near-threshold line shapes that emerge for compact
   (left panel) and molecular states (right panel). The dashed perpendicular line
   indicates the location of the threshold. The $x$--axis shows $M=m_1+m_2+E$.}\label{fig:lineshapes}
\end{figure}
%-----------------------------------------------------------------------------------------------

Besides the deuteron all other (candidates for) hadronic molecules are unstable.
Then the scattering $T$-matrix needs to be modified compared to the form
discussed in Sec.~\ref{sec:wein}. In particular Eq.~(\ref{eq:poleTmatrix}) now
reads
\begin{equation}
T_{\rm in.}(E) = \frac{g^2/2}{E-E_r+(g^2/2)(i k+\gamma)+i\Gamma_0/2} \ ,
\label{eq:poleTmatrix_inel}
\end{equation}
where $E=k^2/(2\mu)$ and $\Gamma_0$ accounts for inelasticities not related
to the channel whose threshold is nearby. Those channels will be called
inelastic channels below.
We also changed the parameter that controls the pole location from $E_B$ to
$-E_r$ since it now refers to a resonance instead of a bound state.
Following the logic of the previous sections in the near-threshold regime the
dominant energy/momentum dependence for molecular states comes from the term
proportional to $g^2$ which is very large in this case, {\sl c.f.}
Eq.~(\ref{eq:gunrenorm}). On the contrary, for compact states the $k^2$ term
controls the momentum dependence. As a result of this in the former case the
line shape of the state that appears in any of the inelastic channels is very
asymmetric while in the latter it is symmetric. The two scenarios are sketched
in Fig.~\ref{fig:lineshapes}.
In addition, the line shape for the molecular state shows a clearly visible
nonanalyticity at the two-particle threshold which would be much weaker in the
other case.\footnote{It is completely absent only if the coupling between the
state with the two hadrons vanishes. However, in this case the $T$-matrix given
in Eq.~\eqref{eq:poleTmatrix_inel} vanishes as well.} Its presence follows 
directly from Eq.~(\ref{eq:poleTmatrix_inel}), since
\begin{equation}
\frac{\partial T_{\rm in.}(E)}{\partial E} \propto -\frac{1+(ig^2/2)(\partial k
/\partial E)}{(E-E_r+g^2/2(i k+\gamma)+i\Gamma_{0}/2)^2} \ ,
\end{equation}
with $\partial k/\partial E=\sqrt{\mu/(2E)}$. It is this derivative that is
not continuous when the energy crosses zero, the location of the threshold.
One might expect from this discussion that the coupling $g^2$ can be read off
from the line shape directly allowing for a direct interpretation of the
structure of the underlying state. However, a scale invariance of the expression
for the line shape appears as soon as the $g^2$-term dominates, and it
hinders a quantitative study in practice~\cite{Baru:2004xg}. In addition, even
for large values of  $g^2$ the nonanalyticity might well not show up clearly in 
the line shape since its visibility depends in addition on a subtle interplay 
of $E_r$, $\Gamma_0$ and $\gamma$.



It was already mentioned at the end of Sec.~\ref{sec:polecounting} that also
near-threshold virtual poles leave a striking impact on observables. In the
presence of inelastic channels a virtual state always leads to a peak of the
line shape exactly at the threshold while a near-threshold bound state still has
strength even below the threshold. For very near-threshold states like the
$X(3872)$ these two scenarios might be difficult to disentangle.



%-----------------------------------------------------------------------------------------------
\begin{figure} %
 \centering
   \includegraphics*[width=\linewidth]{./figures/lineshapes_final}
   \caption{Line shapes that emerge for a bound state (left panel)
   and for a virtual state (right panel) once one of the constituents is 
   unstable.
   The dotted, solid and dashed line show the results for $\Gamma=0, 0.1$ and 
   $1$ MeV, respectively.
   The other parameters of the calculation are given in
Eq.~(\ref{eq:para}).}\label{fig:lineshapes2}
\end{figure}
%-----------------------------------------------------------------------------------------------

As first stressed in Ref.~\cite{Braaten:2007dw} the line shapes in the elastic
channel might well be very interesting, if at least one of the constituents of
the system studied is unstable so that some strength of the amplitude leaks into
the region below the threshold~\footnote{At the same time the nonanalyticity
discussed above gets smeared out.}. To be concrete: If a state is located near
the threshold of particles  $A$ and $B$ and $A$ decays to $a$ and $b$, then  the
spectra in the  $abB$ channel need to be studied both below and above the
nominal $AB$ threshold.
To implement the necessary changes, in the formulae above for narrow
constituents one may simply replace the momentum $k$ in 
Eq.~(\ref{eq:poleTmatrix_inel}) by~\cite{Braaten:2007dw} 
\begin{eqnarray}\nonumber k_{\rm eff} &=&
\sqrt{\mu}\sqrt{\sqrt{E^2{+}\Gamma^2/4}{+}E} \\
& & \qquad + i\sqrt{\mu}\sqrt{\sqrt{E^2{+}\Gamma^2/4}{-}E}\ ,
\label{eq:unstableconst1}
\end{eqnarray}
where $\Gamma$ denotes the width of the unstable constituent, and $\mu$ is the
reduced mass of the two-hadron system evaluated using the mass of the unstable 
state. In addition, the subtraction term $\gamma$ needs to be replaced by
\begin{eqnarray}
\gamma_{\rm eff} = \pm\sqrt{\mu}\sqrt{\sqrt{E_r^2{+}\Gamma^2/4}{-}E_r} \ ,
\label{eq:unstableconst2}
\end{eqnarray}
where the upper sign leads to a (quasi-)bound state while the lower one to a
virtual state.
Clearly, for $\Gamma\to 0$ these expressions map nicely on those used for $k$
and $\gamma$ above for $E_r<0$.
Eqs.~(\ref{eq:unstableconst1}) and (\ref{eq:unstableconst2}) hold as long as the line shape for the unstable
constituent is well described by a BW distribution, namely for
$\Gamma/2\ll m_A-m_a-m_b$.
As soon as the energy dependence of $\Gamma$ starts to matter, more
sophisticated expressions need to be used, {\sl c.f.}
Ref.~\cite{Hanhart:2010wh}.
For simplicity we here use the expressions given above.
 The resulting line shapes in the elastic channel are shown for various values
of $\Gamma$ in Fig.~\ref{fig:lineshapes2},
where for illustration we used the parameters
 \begin{equation}
 E_r = -0.5 \ \mbox{MeV}, ~~ \Gamma_0= 1.5 \ \mbox{MeV},~~ g^2=0.1,~~ \mu =
 0.5\, .
 \label{eq:para} 
 \end{equation}
 The left panel shows the results for the (quasi-)bound state ($\gamma_{\rm
 eff}>0$), and the right one is for the virtual state ($\gamma_{\rm eff}<0$).
 Clearly, above the nominal two-hadron channel ($E=0$) the  spectra in the left
 and the right panels look very much alike, however, for $\Gamma>0$ drastic
 differences appear between the two cases for negative values of $E$.

 Following the Weinberg criterion, for the bound state case it is the relative
 height of the peak for $E<0$ and the bump for $E>0$, which are difficult to
 distinguish for the largest value of $\Gamma$, that is a measure of the
 molecular admixture of the studied state. Therefore a high resolution
 measurement of the line shape of $X(3872)$ would be very valuable to deduce its
 nature~\cite{Hanhart:2007yq,Braaten:2007dw,Hanhart:2010wh,Meng:2014ota,
 Kang:2016jxw}.

 In this context it is interesting to note that a line shape very similar to the
one shown in the left panel of Fig.~\ref{fig:lineshapes2}
 was also predicted for $Y(4260)\to D^*\pi \bar D$ under the assumption that
 $Y(4260)$ is a $D_1\bar D$ molecular state (note that $D^*\pi$ is the most
 prominent decay channel of $D_1(2420)$)~\cite{Wang:2013kra}, see the middle 
panel of Fig.~\ref{fig:Y4260LineShape}. A similar line shape also shows up in
the calculation in Ref.~\cite{Debastiani:2016xgg} for the $f_1(1285)$
strongly coupled to $K^*\bar K$.


Everything said so far had the implicit assumption that there are at most two
near-threshold poles on the two relevant sheets. The possible line shapes change
dramatically as soon as additional poles are located in the near-threshold
regime as is discussed in detail in
Refs.~\cite{Baru:2010ww,Artoisenet:2010va,Hanhart:2011jz}.

\subsection{Heavy quark spin symmetry}
\label{sec:3_HQSS}

 In the limit of infinitely heavy quarks, the spin of heavy quarks decouples
 from the system and is conserved individually. As a result, the total angular
 momentum of the light degrees of freedom becomes a good quantum number as well.
 This gives rise to the so-called heavy quark spin symmetry
 (HQSS)~\cite{Isgur:1989vq}.
 In the real world quarks  are not infinitely heavy, however, heavy quark
 effective field theory allows one to systematically include corrections that
 emerge from finite quark masses
in a systematic expansion in $\Lambda_{\rm QCD}/M_Q$, where $M_Q$ denotes the
heavy quark mass. For an extensive review we refer to
Ref.~\cite{Neubert:1993mb}. HQSS is the origin for the near degeneracy of, e.g.,
$D^*$ and $D$ as well as $B^*$ and $B$. Similarly, it also predicts
straightforwardly multiplets of hadronic molecules made of a heavy hadron or
heavy quarkonium and light hadrons~\cite{Guo:2009id,Yamaguchi:2014era}, and of
hadroquarkonium as well~\cite{Cleven:2015era}.
In addition, HQSS allows one to predict ratios of different transitions
involving heavy hadrons in the same spin multiplet, and in particular for your
interest transitions of hadronic molecules.
Examples for those predictions can be found in
Refs.~\cite{Fleming:2008yn,Fleming:2011xa} where the decays of the $\X$ into the
final states $\chi_{cJ}\pi$ and $\chi_{cJ}\pi\pi$ are discussed in the XEFT
framework {which will be discussed in Sec.~\ref{sec:4-pc}}. The ratios
among various decays of the $Z_b(10610)$ and $Z_b(10650)$ into $h_b(mP)\pi$ and
$\chi_{bJ}(mP)\gamma$ (from the neutral $Z_b$ states) were computed in both
XEFT~\cite{Mehen:2011yh} and \nreft~\cite{Cleven:2013sq} frameworks to be
discussed in the next section.
They are consistent with the result solely based on the
HQSS~\cite{Ohkoda:2012rj}.
For other predictions based on HQSS on the radiative and strong decays of
hadronic molecules in the heavy-quarkonium sector, we refer
to~\cite{Ma:2014ofa,Ma:2014zva}.

In special cases HQSS also allows one to make predictions for bound systems of
two or more heavy
mesons~\cite{Voloshin:2011qa,Mehen:2011yh,Nieves:2012tt,Guo:2013sya,Liu:2013rxa,
Baru:2016iwj} since certain potentials get linked to each
other. This will be discussed in details in Sec.~\ref{sec:4-interactions}.

