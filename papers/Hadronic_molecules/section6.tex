
\section{Phenomenological manifestations of hadronic molecules}
\label{sec:6}


A large number of  theoretical studies on the recently discovered exotic
candidates focus on the computation of masses (for recent reviews see
Refs.~\cite{Brambilla:2014jmp,Chen:2016qju,Lebed:2016hpi,Esposito:2016noz}).
However, from the discussions in Secs.~\ref{sec:3} and \ref{sec:4}, it is
clear that the internal structure and especially the molecular nature of a
physical state manifests itself predominantly in some properly identified
dynamical production and decay processes. For near-threshold hadronic molecules,
the pertinent observables are provided by a set of low-energy quantities: the
scattering length and effective range for the constituent-hadron system or,
equivalently, the effective coupling of the hadronic molecular candidate to its
constituents, since these quantities are heavily intertwined as demonstrated in
Sec.~\ref{sec:3}. As discussed in detail there, the probability to find the
two-hadron component in the physical state, $(1-\lambda^2)$, can be extracted
directly from these quantities.
However, due to the presence of various energy scales driven by different
physics aspects, not all production and decay processes are sensitive to the
effective coupling as will be discussed with various examples mainly on the
$XYZ$ states in this section.
In addition, the implications of heavy quark spin and flavor symmetries on the
spectrum of hadronic molecules as well as the interplay between hadronic
molecules and nearby triangle singularities are presented.


\subsection{Long-distance production and decay mechanisms}
\label{sec:6-long}

As in Sec.~\ref{sec:3}, we denote the wave function of a hadronic molecule
candidate as $\Psi$. In order to allow for a quantitatively controlled analysis,
the state  must be located close to the relevant two-body threshold of $h_1$ and
$h_2$.
Then the long-distance momentum scale is given by $\gamma=\sqrt{2\mu E_B}$, {\sl
c.f.} Eq.~(\ref{eq:gamdef}), with $\gamma\ll\beta$, where $\beta$ is the inverse
of the range of forces. We define two classes of production and decay processes,
namely
\begin{itemize}
  \item long-distance processes, in which the momenta of all particles in the
CM frame of $h_1h_2$ are of $\order{\gamma}$;
  \item short-distance processes, which involve particles with a momentum
$\gtrsim \beta$ in the CM frame of $h_1h_2$.
\end{itemize}
It is shown in this section that only the former are sensitive to the molecular
component of the state investigated. The complications in the latter will be
discussed in the next section.

\subsubsection{Decays into the constituents and transitions between molecular states}


The long-distance processes involving hadronic molecules can be computed using
the EFT machinery introduced in Sec.~\ref{sec:4}. When only
 the degrees of freedom with momenta of $\order{\gamma}$ are kept, the EFT is
XEFT or, more generally, \nreftii.
When all particle energies are much smaller than $\beta^2/(2\mu)$ in the
$h_1h_2$ CM frame, the amplitudes involving pure molecular states for the
pertinent processes  are at LO determined by the scattering length
universality~\cite{Braaten:2004rn}. Decay amplitudes are then proportional to
the effective coupling $g_{\rm eff}$ defined in Eq.~\eqref{eq:ga} which can also
be expressed in terms of the scattering length.
Clearly such an approach cannot be simply applied to predominantly compact
states since then  $g_{\rm eff}$  becomes small and short-distance mechanisms
become more important than hadronic loops.


For instance, as soon as  the $\X$  is treated as a $D\bar D^*$ molecule, the
most important long-distance processes are its decays into $D^0\bar D^{*0}\to
D^0[\bar D^0\pi^0/\bar D^0\gamma]$, discussed in Sec.~\ref{sec:4-nreft2xeft}.
The decay rates and the momentum distributions of the final states serve as a
good probe for the structure of the $\X$~\cite{Voloshin:2003nt}.\footnote{These
decays were also discussed in
Refs.~\cite{Swanson:2003tb,Voloshin:2005rt,Meng:2007cx, Fleming:2007rp,
Liang:2009sp, Stapleton:2009ey, Baru:2011rs,Guo:2014hqa, Polosa:2015tra}.}
Higher order corrections can be calculated using \nreft~detailed in Sec.~\ref{sec:nreft1}.
For the $\Y$, the most important process for the detection of its $D_1\bar D$
component would be the decay into $D_1\bar D\to [D^*\pi/D^*\gamma]\bar
D$~\cite{Cleven:2013mka,Qin:2016spb}.

It may happen that two of the particles in the final states in the above
mentioned three-body decays form another hadronic molecule in the final state.
The transition of a shallow bound state into a light
particle and another shallow bound state receives two enhancements: large
coupling constants for the vertices involving the molecular states, and the
$1/v\simeq 2/(v_1+v_2)$ factor as shown in Sec.~\ref{sec:nreft1}, where $v_1$ and $v_2$ denote the relative velocities of the heavy mesons before
and after the emission of the light particle (see Fig.~\ref{fig:triangle} and
Eq.~\eqref{eq:Ipc}).

The possibility of a near-threshold pole in the $D\bar D$ final state
interaction and its possible influence on the $\X\to D^0\bar D^0\pi^0$
transition was studied in Ref.~\cite{Guo:2014hqa}, {\sl c.f.}
Fig.~\ref{fig:XDDpi_FSI}. Note that experimental information on this
distribution does not exist yet.\footnote{A few calculations based on
phenomenological models suggested the possible existence of a $D\bar D$ bound
state~\cite{Wong:2003xk,Zhang:2006ix,FernandezCarames:2009zz,Zhang:2009vs,
Liu:2010xh,Li:2012mqa}.} However, the interplay of hadronic molecules in the
final and the initial state might well have been observed already as detailed in
the remaining parts of this section.

The $D_1(2420)\bar D$ threshold in the $J^{PC}=1^{--}$ channel is the closest
$S$-wave open-charm threshold that the $\Y$ can couple to. It is at the same
time the lowest $S$-wave open-charm threshold with vector quantum numbers, which
provides a natural explanation why the first (established) exotic vector state
is significantly heavier than the $X(3872)$.
Assuming that $\Y$ is a  $D_1\bar D$ molecule and $\X$ and $Z_c(3900)$ are
$D^*\bar D$ molecules with $J^{PC}=1^{++}$ and $1^{+-}$, respectively, the
decays of $\Y$ into  $Z_c\pi$ and $\X\gamma$ will occur through the mechanisms
shown in Fig.~\ref{fig:ytoxz} (a) and (b), where the type of the light particle
accompanying the hadronic molecular state is controlled by the positive and
negative $C$ parity of the $\pi^0$ and the photon, respectively.
% -----------------------------------------
  \begin{figure}[tb]
    \begin{center}
     \includegraphics[width=\linewidth]{./figures/ytoxz}\\
     \caption{Schematic diagrams for the decays of the $\Y$ to $\Z\pi$ and
to $\X\gamma$ assuming that $D_1 \bar D-c.c.$ dominates the dynamics. The
diagrams from the charge conjugated channel are not shown.
     \label{fig:ytoxz}}
    \end{center}
  \end{figure}
%-----------------------------------------
Since $v\lesssim 0.1$ for both transitions, $1/v$ indeed provides a large
factor, shown as the solid lines in Fig.~\ref{fig:pc}. Therefore,  a copious
production of $\Z$ from  $\Y\to \Z \pi$ transitions as observed at both BESIII
and Belle~\cite{Ablikim:2013mio,Liu:2013dau} appears naturally within this
dynamical picture~\cite{Wang:2013cya}.
In addition, $if$ the above explanation is indeed correct, also the $\X$ must be necessarily
produced with a large rate for the production
 in $\Y$ radiative decays~\cite{Guo:2013nza}.
 Indeed, assuming that the $\Y$ and
$\X$ are pure bound states of $D_1\bar D$ and $D^0\bar D^{*0}$,\footnote{It has
been emphasized in
Sec.~\ref{sec:4-nreft2xeft} that the charged charm mesons need to be taken
into account for the $\X$ in the framework of \nreft~since they are below the
hard scale and should be treated explicitly. The reason for neglecting them here
is that the rate for the $D_1^0\to D^{*0}\gamma$ is at least one
order of magnitude larger than that for the $D_1^+\to D^{*+}\gamma$ from
nonrelativistic quark model
calculations~\cite{Fayyazuddin:1994qu,Godfrey:2005ww,Close:2005se}.}
respectively,
one can get the coupling strengths using the relation in
Eq.~\eqref{eq:residue_bs} with $\lambda^2=Z=0$
\begin{eqnarray}
  |g_{\text{NR},X}^{}| &=& (0.20\pm0.20\pm0.03)~\text{GeV}^{-1/2} \,,
  \nonumber\\ |g_{\text{NR},Y}^{}| &=&
(1.26\pm0.09\pm0.66)~\text{GeV}^{-1/2} \,,
\end{eqnarray}
where we have taken the $\X$ binding energy to be $(90\pm90)$~keV, the first
errors are from the uncertainties of the binding energies, and
the second ones are from the approximation of Eq.~\eqref{eq:residue_bs}
due to neglecting terms suppressed by $\gamma/\beta$. Here, the inverse of the
range of forces is conservatively estimated by $\beta\sim \sqrt{2\, \mu\,
\Delta_\text{th}}$, with $ \Delta_\text{th}$ the difference
between the threshold of the components and the next close one, which is
$M_{D^{*+}} + M_{ D^+ } - M_{D^{*0}} - M_{ D^0 }$ for the $\X$ and $M_{D_1} +
M_{ D^* } - M_{D_1} - M_{ D }$ for the $\Y$, respectively.
The partial width for the $\Y\to\X\gamma$ in \nreft~can
reach a few tens of keV depending on the exact value of the $\X$ binding
energy~\cite{Guo:2013nza}.
The predicted copious production of the $\X$ in the $\Y$ radiative decays
was later confirmed by BESIII~\cite{Ablikim:2013dyn}.


It is worthwhile to mention that if the CM energy of the $e^+e^-$ collisions is
very close to the $D_1(2420)\bar D$ threshold at around 4.29~GeV,\footnote{The
production of an $S$-wave pair of $D_1(2420)$ and $\bar D$, which are
$j_\ell^P=3/2^+$ and $j_\ell=1/2^-$ states, respectively, breaks
HQSS~\cite{Li:2013yka}. This point was in fact already noticed in the classical
papers of the Cornell model, see Table~VI in Ref.~\cite{Eichten:1978tg} and
Table~VIII in Ref.~\cite{Eichten:1979ms}. However, in the energy region about
4.2~GeV HQSS breaking can be sizable~\cite{Wang:2013kra}.} the production of the
$\Z$ and $\X$ via the mechanism under consideration gets even more enhanced
since the kinematical condition for the TS discussed in Sec.~\ref{sec:4-3ploop}
is then (nearly) satisfied.
However, the resulting enhancement is balanced by the fact that the energy is
away from the peak of the $\Y$ spectral function.

It should be mentioned that some experimental observations consistent with
hadronic molecules are also claimed to be consistent with other models. For
instance, treating both the $\X$ and $\Y$ as tetraquarks also leads to a sizable
width for the radiative decay $\Y\to\X\gamma$ ~\cite{Chen:2015dig}.
However, large transitions to the constituents of the hadronic molecules appear
to be unique signatures for the molecular states.
In addition, their importance leaves visible imprints in the line shapes of
states like  the$Y(4260)$~\cite{Cleven:2013mka,Qin:2016spb}.
 	



\subsubsection{More on line shapes}
\label{sec:morelineshapes}

The line shape of a hadronic molecule near its constituent threshold
reflects a long-distance phenomenon and can be used as a criterion for
establishing their nature.
The energy dependence of a hadronic molecule production line shape
generally does not appear to be trivial.

The data available at present for the line shapes of $X(3872)$ appear to be
insufficient for a unique conclusion about the pole locations of the state.
For example, a simultaneous fit of the line shape of the $X(3872)$ in the
$J/\psi\pi\pi$ and the $D^{*0}\bar{D}^0$ invariant mass distributions employing
a generalized  Flatt\'e parametrization~\cite{Hanhart:2007yq} revealed that
$X(3872)$ is a virtual state.
However, as soon as an explicit quarkonium pole is included in the analysis, the
authors of Refs.~\cite{Zhang:2009bv,Kalashnikova:2009gt,Meng:2014ota} find that
$X(3872)$ is the $2^3P_1$ charmonium with a large coupling to the
$D^0\bar{D}^{*0}$ channel --- in the light of the discussion of
Sec.~\ref{sec:wein} one needs to conclude that also in this case the $X(3872)$
has a sizeable molecular admixture. It should also be stressed that in
Ref.~\cite{Hanhart:2007yq} the width of the $D^*$ was omitted, which might
distort the line shapes ({\sl c.f.} Sec.~\ref{sec:lineshapes}) as was pointed
out in Refs.~\cite{Stapleton:2009ey,Braaten:2007dw}. According to these
analyses, once this effect is included, the fit favors a bound state solution.
Another study based on an improved Flatt\'e formula~\cite{Artoisenet:2010va}
notices that the current data can accept $X(3872)$ as both a  $D^0\bar{D}^{*0}$
hadronic molecule or the fine-tuned $2^3P_1$ charmonium coupled with  the
$D^0\bar{D}^{*0}$ channel. Also the more recent analysis of
Ref.~\cite{Kang:2016jxw} finds solutions with either a bound state pole or
virtual states.
Thus, to further investigate the nature of $X(3872)$ a high resolution scan of
its line shapes, especially within a few $\mev$ of the $D^0\bar{D}^{*0}$
threshold, is necessary.

A study of the line shapes of the two $Z_b$ states in $h_b(1P,2P)\pi$ channels
is presented in Ref.~\cite{Cleven:2011gp} based on the \nreft~framework
discussed in Sec.~\ref{sec:4}. By fitting to the $h_b(1P,2P)\pi$ invariant mass
distribution, it is found that the current data are consistent with the two
$Z_b$ states being $B\bar{B}^*$ and $B^*\bar{B}^*$ bound states, respectively.
Their line shapes in the elastic channels are also studied in the
XEFT/\nreftii~(see Sec.~\ref{sec:4-XEFT}) framework with the HQSS breaking
operator included explicitly in Ref.~\cite{Mehen:2013mva} where, however, no
pole locations were extracted. The line shapes of the two $Z_b$ states were
investigated in both elastic and inelastic channels in
Refs.~\cite{Hanhart:2015cua,Guo:2016bjq} based on separable
interactions, see Fig.~\ref{fig:Zb5Sfull}. An
explicit calculation revealed that the line shapes get distorted very little
when a non-separable interaction is included, such as the one-pion-exchange
potential. Similar studies of the $Z_c(3900)$ line shape in both $J/\psi\pi$ and
$D\bar D^*$ can be found in
Refs.~\cite{Albaladejo:2015lob,Zhou:2015jta,Pilloni:2016obd}. Based on these
results one needs to state that  also the data currently available for the
$Z_c(3900)$ are insufficient to pin down the pole locations.


The situation that it is not easy to extract the pole locations is partly
because that the observed peaks for the $\X$, $\Zc$ and $Z_b$ states are too
close to the  thresholds. Being tens of MeV below the $D_1\bar D$ threshold,
which, however, also means a larger uncertainty, the $\Y$ situation is
different. One sees clearly a nontrivial structure around the $D_1\bar D$
threshold in the $\Y$ line shape in the $J/\psi \pi\pi$ invariant mass
distribution, {\sl c.f.} Fig.~\ref{fig:Y4260-BESIII}. It indicates that the
coupling to the $D_1\bar D$ plays an essential role in understanding the $\Y$ in
line with the analysis in Sec.~\ref{sec:lineshapes}.
Correspondingly the molecular picture predicts a highly nontrivial energy
dependence for $e^+e^-\to D_1\bar D\to [D^*\pi]D$~\cite{Cleven:2013mka},
{\sl c.f.} the middle panel in Fig.~\ref{fig:Y4260LineShape}, that awaits
experimental confirmation at, e.g., BESIII.


\subsubsection{Enhanced isospin violations in molecular transitions}
\label{sec:isospinviol}

As argued above in Sec.~\ref{sec:wein} for molecular states the coupling of the
pole to the continuum
channel that forms the
state is large. As a consequence of this loop effects get very important that
can lead to very much
enhanced isospin violations if the molecular state is close to the relevant
threshold.
To be concrete we start with a detailed discussion of the implications of this
observation
for the $f_0$--$a_0$ mixing, first discussed in Ref.~\cite{Achasov:1979xc}.
\begin{figure}[t]
\begin{center}
\epsfig{file=./figures/leadingCSB.pdf, height=2.cm}
\caption{\it Graphical illustration of the leading contribution to the $f_0-a_0$
mixing
    matrix element.}
\label{lcsb}
\end{center}
\end{figure}
If both the isoscalar $f_0(980)$ as well as the isovector $a_0(980)$ were $\bar
KK$ molecular
states, the leading mixing effect of the two scalar mesons would the difference
of the loop functions
of charged and neutral kaons as
depicted in Fig.~\ref{lcsb} (in isospin conserving transitions the sum enters).
In the near-threshold regime we may approximate the loops by their leading
energy dependence provided by the respective unitarity cuts:
\begin{eqnarray}\nonumber
\langle f_0 |T| a_0\rangle &=& ig_{f_0K\bar K}g_{a_0K\bar
    K}\sqrt{s}\left( p_{K^0}-p_{K^+} \right) \\ & & \qquad  \qquad  \qquad +
{\cal
    O}\left({p_{K^0}^2-p_{K^+}^2}\right) \, ,
\label{llam}
\end{eqnarray}
where $p_K$ denotes the modulus of the relative momentum of the kaon pair.
Obviously, this leading contribution model-independently provides a measure of
the product of the effective couplings of $a_0$ and $f_0$ to the kaons and
therefore, as discussed in Sec.~\ref{sec:weinberg},   to the molecular admixture
of both states.
In the isospin limit  the loops cancel exactly. However, as soon as the masses
of the two-hadron states are different due to isospin violations
($M_{K^0}-M_{K^+}=4$ MeV), there appears an offset in the thresholds and the
mentioned cancellation is no longer complete.
This results in a transition between different isospins with all its strength
located  in between the two thresholds.
In Refs.~\cite{Wu:2007jh,Hanhart:2007bd} it was predicted that this very
peculiar effect should show up prominently in the transition $J/\psi \to \phi
\pi^0\eta$, if both $f_0(980)$ and $a_0(980)$ are $K\bar K$ molecular states,
since only then the coupling of the states to the $K\bar K$ is sufficiently
strong for the effect to be observable. A few years later the predicted very
narrow signal was measured at BESIII~\cite{Ablikim:2010aa} providing strong
evidence for a prominent molecular admixture in these light scalar mesons.

Another prominent example where a molecular nature of a near-threshold state
leads to a natural explanation of a huge isospin violation is the equal decay
rate of the $X(3872)$ to the isoscalar $\pi\pi\pi J/\psi$ and the isovector
$\pi\pi J/\psi$ channels, where in both cases the few-pion systems carry vector
quantum numbers and thus may be viewed as coming from the decay of an $\omega$
and a $\rho^0$ meson, respectively. The argument goes as follows:
The mass of the $X(3872)$ is located 7~MeV below the nominal $\omega J/\psi$
threshold, but 5~MeV above the nominal $\rho^0 J/\psi$ threshold. In addition,
the width of the $\omega$ is only 8~MeV such that the decay of the $X(3872)$
into the $\rho^0 J/\psi$ channel is strongly favored kinematically. Therefore,
the $X(3872)$ cannot have a significant isovector component in its wave function
since otherwise it would significantly more often decay into the $\rho^0 J/\psi$
than into the $\omega J/\psi$ channel.
However, a calculation for an isoscalar $X(3872)$ that predominantly decays via
$D^*\bar D$ loops naturally gives the experimental branching ratios for the
pions-$J/\psi$ channels simply because the close proximity of the mass of the
$X$ to the neutral $D^*\bar D$-threshold automatically produces an enhanced
isospin violation in the necessary strength, if the $X(3872)$ is a $D^*\bar D$
molecule, since only then the coupling to continuum is strong
enough~\cite{Gamermann:2009fv}.
An isoscalar nature of the $X(3872)$ is also required from a study of other
possible decay channels~\cite{Mehen:2015efa}, and its effective couplings to the
charged and neutral channels are basically the
same~\cite{HidalgoDuque:2012pq,Guo:2014hqa}.

As the last example in this context we would like to mention the hadronic width
of the isoscalar  $D_{s0}^*(2317)$.
This state is located below the $DK$ threshold and as such can decay strongly
only via isospin violation  into the isovector $D_s\pi $ channel.
The two most prominent decay mechanisms of $D_{s0}^*(2317)$ are an isospin
conserving transition into $D_s\eta $, followed by the isospin violating $\pi
\eta$ mixing amplitude, and the isospin violating difference between a $D^0K^+$
and a $D^+K^0$ loop (subleading contributions to this transition were studied in
Ref.~\cite{Guo:2008gp}).
The former mechanism should be present regardless the nature of the state, which
typically leads to widths of the order of 10~keV~\cite{Colangelo:2003vg}.
The latter mechanism, however, is large, if indeed the $D_{s0}^*(2317)$ were a
$DK$ molecule.
In fact typical calculations for molecular states give a width of the order of
100~keV~\cite{Faessler:2007gv,Lutz:2007sk,Liu:2012zya} --- for more details we
refer to Sec.~\ref{sec:latres}.
Thus, if this admittedly small width could be measured, e.g., at
$\overline{\text{P}}$ANDA, its value would provide direct experimental access to
the nature of $D_{s0}^*(2317)$.


\subsubsection{Enhanced production of hadronic molecules and conventional
hadrons due to triangle singularities}
\label{sec:6-ts}


From the analysis in Sec.~\ref{sec:4-3ploop}, we see that the TS on the
physical boundary is always located close to the threshold
of the intermediate particles. Furthermore, its effect is most pronounced
if  the two intermediate particles are in an $S$-wave, since otherwise the
centrifugal barrier suppresses small momenta. Hadronic
molecules are located naturally near-thresholds as well, and in all cases
considered in this review couple in
$S$-wave to its constituents. Therefore, in the course of this review there are
two important aspects of TS
that need to be discussed:
On the one hand, a TS may lead to a pronounced structure in experimental
observables that could be mistaken as a state; on the other hand, a TS can
enhance the production of a hadronic molecule in a given reaction. Note that
also the production of the conventional
hadrons can by strongly enhanced by a TS within small  energy regions.
 This is accompanied by a significant distortion of the line shapes since
 the location of the TS depends on the invariant masses of the external states.
 For example, the signal in the $\eta\pi\pi$ channel interpreted as $\eta(1405)$
 and the signal in the $K\bar{K}\pi$ interpreted as $\eta(1475)$ could
 find their origin in a single pole accompanied by a
TS~\cite{Wu:2011yx,Wu:2012pg}.
 We note that although
$P$-wave couplings are present in the triangle loop for the $\eta(1405/1475)$
decays,
the perfect satisfaction of the TS condition, Eq.~\eqref{eq:trianglesing},
causes detectable effects in these decays.


Since the  $K\bar K^*$ system can contribute to both $I=0$ and $I=1$ channels
with $J^{PC}=1^{++}$ and $1^{+-}$,
 one expects that
the TS may cause enhancements also in these channels.
In the following we list those possible enhancements and their quantum
numbers which can be searched for in experiment:
\begin{eqnarray}\label{KKstar-TS}
f_1(1420), &\ 0^+, \ 1^{++}:& \ \frac{1}{\sqrt{2}}(K^*\bar{K}-K\bar{K}^*)
\nonumber\\
&& \to K\bar{K}\pi, \ \eta\pi\pi, [3\pi];\nonumber\\
a_1(1420), &\ 1^-, \ 1^{++}:& \ \frac{1}{\sqrt{2}}(K^*\bar{K}-K\bar{K}^*)
\nonumber\\
&&  \to K\bar{K}\pi, \ 3\pi, \ [\eta\pi\pi];\nonumber\\
\tilde{h}_1(1420), &\ 0^-, \ 1^{+-}:& \
\frac{1}{\sqrt{2}}(K^*\bar{K}+K\bar{K}^*) \nonumber\\
&&  \to \rho\pi, \ \omega\eta, \ (\phi\eta), [\omega\pi], \ [\rho\eta], \
[\phi\pi];\nonumber\\
\tilde{b}_1(1420), &\ 1^+, \ 1^{+-}:& \
\frac{1}{\sqrt{2}}(K^*\bar{K}+K\bar{K}^*) \nonumber\\
&&  \to \phi\pi, \ \omega\pi, \rho\eta, \ [\rho\pi], \ [\omega\eta],
\end{eqnarray}
where $\tilde{h}_1(1420)$ and $\tilde{b}_1(1420)$ refer to the TSs whether or
not there exists resonances around.
Note that the $f_1(1420)$ needs to be taken into
account for the angular distribution in the $J/\psi\to\gamma 3\pi$
process~\cite{BESIII:2012aa}, and a detailed
partial wave analysis suggests the presence of the $f_1(1420)$ resonance
together with a TS mechanism~\cite{Wu:2012pg},
while it is argued in Ref.~\cite{Debastiani:2016xgg} that the $f_1(1420)$ is
the manifestation of the $f_1(1285)$ at higher energies due to the TS.
The $a_1(1420)$ has been reported by the
COMPASS Collaboration in $\pi^-p\to \pi^-\pi^-\pi^+ p$ and $\pi^-\pi^0\pi^0
p$~\cite{Adolph:2015pws} and can be well explained by the TS
mechanism~\cite{Liu:2015taa,Ketzer:2015tqa}. The decay channels in the square
brackets are $G$-parity violating and those in the round brackets are limited by
the phase space. One notices that there are states observed in the relevant mass
regions, namely, $a_1(1260)$, $f_1(1285)$, $h_1(1170)$, and
$b_1(1235)$~\cite{Olive:2016xmw}. Although most of these states have masses
outside
of the TS favored mass region, {\sl i.e.} $1.385\sim 1.442$ GeV~\cite{Liu:2015taa},
the $h_1(1380)$ is located at the edge of the TS kinematics and some detectable
effects could be expected~\cite{Guo:2013nza,Ablikim:2013dyn}. Notice that when
there is a TS in action,  the peak position for a resonance could be shifted
towards its location.
Some structures around thresholds of a pair of other light hadrons were also suggested to be due to a TS, for instance, the $f_2(1810)$ around the $K^*\bar
K^*$ threshold~\cite{Xie:2016lvs} and the $\phi(2170)$ around the $N\bar \Delta$
threshold~\cite{Lorenz:2015pba}.

In the heavy quarkonium sector,  the most famous example for an enhanced
production rate via the TS is the observation of the $Z_c(3900)$ at the mass
region of $Y(4260)$~\cite{Wang:2013cya,Wang:2013hga,Liu:2013vfa,
Szczepaniak:2015eza,Gong:2016jzb, Pilloni:2016obd}.
In addition, the sensitivity of the TS to the kinematics of the reaction might
well be the reason why the $Z_c(4020)$ is not seen in the same
decay~\cite{Wang:2013cya,Wang:2013hga}.
As discussed in Sec.~\ref{sec:6-long}, the same mechanism also enhances the
transition $Y(4260)\to \gamma X(3872)$~\cite{Guo:2013nza} and suggests that the
rate for $e^+e^-\to\gamma X_2$, which can be used to search for the spin-2
partner of the $\X$, $X_2$ (see Sec.~\ref{sec:4-interactions}), gets most
enhanced if the $e^+e^-$ collision energy is between 4.4 and
4.5~GeV~\cite{Guo:2014ura}.\footnote{This suggestion is based on the assumption
that the $X_2$ mass is very close to the $D^*\bar D^*$ threshold. If its mass is
tens of MeV below the threshold as suggested in~\cite{Baru:2016iwj}, then the TS
would be further away from the physical boundary and the production would get
less enhanced. } A candidate for the analogue of $Y(4260)$ in the bottomonium
sector is $\Upsilon(11020)$, since it is located close to the $B_1\bar B$
threshold~\cite{Wang:2013hga,Bondar:2016pox}. Here, the TS could
affect both the $Z_b(10610)$ and $Z_b(10650)$~\cite{Wang:2013hga}, although at
the mass of $\Upsilon(11020)$ the production of the lower $Z_b$ state is more
favored since the corresponding TS is closer~\cite{Bondar:2016pox}.
Based on the current statistics in Belle~\cite{Bondar:2016pox}, it is difficult
to judge whether there is one peak or two peaks present in Fig.~\ref{fig:Zb6S}.
In Refs.~\cite{Pakhlov:2014qva,Uglov:2016nql} the structure identified as the
charged exotic $Z_c(4430)$ observed in the $\pi \psi'$ final states by
Belle~\cite{Mizuk:2009da,Chilikin:2013tch} and LHCb~\cite{Aaij:2015zxa} was
claimed to be not connected to a pole but to owe its existence to the presence
of a TS.

Suggestions to search for resonance-like structures due to the TS in the heavy
meson and heavy quarkonium mass regions can be found
in~\cite{Liu:2014spa,Liu:2015cah,Liu:2015taa,Liu:2017vsf}. In particular, the
very recent BESIII results on the $\pi^\pm\psi'$ invariant mass distributions
for the $e^+e^-\to \pi^+\pi^-\psi'$ process at different CM energies seem in
line with the predictions made in~\cite{Liu:2014spa}.
Some of the strongly favored triangle loops are listed in
Table~\ref{tab:trianglesingularity}.


\begin{table*}
\caption{The triangle loops $[M_1M_2M_3]$ corresponding to
Fig.~\ref{fig:triangle}
which have shown large impact on the production of hadronic molecules
and conventional hadrons in experiment are listed in the first column.
The second column is the measured process with the final states in the bracket.
The checkmarks in the last column indicate that the triangle singularity of the
corresponding process locates in the physical region, {\sl i.e.} satisfying
Eq.\eqref{eq:trianglesing}. Although  the singularity of the process without
check mark is not located in the physical region, since it is not far away, it
can still enhance the corresponding production rate significantly.
 }
\begin{ruledtabular}
\begin{tabular}{ l c c }
$[M_{1}M_{2}M_{3}]$ & $A\to BC(\to\text{final states})$ &
Eq.\eqref{eq:trianglesing}\tabularnewline
\hline
$[K^{*}KK]$ & $\eta(1405/1475)\to
a_{0}(980)\pi(\to3\pi)$~\cite{Wu:2011yx,Wu:2012pg} & $\checkmark$\tabularnewline
 & $a_{1}(1420)\to
f_{0}(980)\pi(\to\eta\pi\pi)$~\cite{Ketzer:2015tqa,Liu:2015taa} &
$\checkmark$\tabularnewline
\hline
$[D_{1}DD^{*}]$ & $Y(4260)\to X(3872)\gamma$ ~\cite{Guo:2013nza} &
\tabularnewline
 & $Y(4260)\to
Z_{c}(3900)\pi$~\cite{Wang:2013cya,Wang:2013hga,Szczepaniak:2015eza,
Gong:2016jzb} & \tabularnewline
\hline
$[\Lambda(1890)\chi_{c1}p]$ & $\Lambda_{b}\to
P_{c}(4450)K$~\cite{Guo:2015umn,Liu:2015fea,Bayar:2016ftu} &
$\checkmark$\tabularnewline
\hline
$[D_{s3}(2860)\Lambda_{c}(2595)D]$ & $\Lambda_{b}\to
P_{c}(4450)K$~\cite{Liu:2015fea} & \tabularnewline
\end{tabular}
\end{ruledtabular}
\label{tab:trianglesingularity}
\end{table*}


The $P_c(4450)$  structure observed by LHCb~\cite{Aaij:2015tga} in
$\Lambda_b$ decays, no matter what its nature is, also contains a TS
contribution, as long as it strongly couples in an
$S$-wave~\cite{Bayar:2016ftu} to either
$\chi_{c1}p$~\cite{Guo:2015umn,Liu:2015fea,Guo:2016bkl,
Meissner:2015mza}  or $\Lambda_c(2595)\bar D$~\cite{Liu:2015fea}.
Recent discussions on the role of TSs in the
light baryonic sector can be found in Refs.~\cite{Wang:2016dtb,Roca:2017bvy,
Debastiani:2017dlz,Samart:2017scf}.


It is worthwhile to emphasize that once the kinematics for a process (nearly) satisfies
the TS condition given in Eq.~\eqref{eq:trianglesing},  the
enhancement in reaction rates due to TS contribution is always there, and can
produce a narrow peak once the relevant coupling is in an
$S$-wave~\cite{Bayar:2016ftu}.
The only question is whether it is strong enough to produce the observed or an
observable structure. Complications due to the interference between
the TS and a tree-level contribution is discussed
in~\cite{Schmid:1967ojm,Goebel:1982yb} in the single-channel case, and
in~\cite{Anisovich:1995ab,Szczepaniak:2015hya} for coupled channels.
The key to distinguishing whether a structure is
solely due to a TS or originates from a genuine resonance is the sensitivity of
the TS on kinematics: If in reactions with different kinematics the same structure is
observed, it most likely reflects the existence of a pole (resonance).


\subsection{Short-distance production and decay mechanisms}
\label{sec:6-short}

In Sec.~\ref{sec:6-long} we discussed both decay and production mechanisms
sensitive to the long-range parts of the wave function of a state and therefore
sensitive to its molecular nature. Here we demonstrate that there are
also decays and production reactions that do not allow one to extract the
molecular component of a given state. We start with an example of the former
to then switch to the latter.

It was claimed long ago~\cite{Swanson:2004pp} that the ratio \[
\frac{\mathcal{B}(\X\to \gamma\psi') }{\mathcal{B}(\X\to \gamma J/\psi) }, \]
with the measured value given in Eq.~\eqref{eq:sec2X3872R}, is very sensitive to
the molecular component of the $\X$ wave function. In particular using vector
meson dominance and  a quark model, in Ref.~\cite{Swanson:2004pp} it was
predicted to be about $4\times10^{-3}$ if
the $\X$ is a hadronic molecule with a dominant $D^0\bar D^{*0}$ component plus
a small admixture of the $\rho J/\psi$ and $\omega J/\psi$. However, as
demonstrated in Ref.~\cite{Guo:2014taa}, when radiative decays of $\X$ are
calculated using \nreft~(see Sec.~\ref{sec:nreft1}) field theoretic consistency
calls for the inclusion of a counter term at LO. In other words, the transitions
are controlled by short-distance instead of long-distance dynamics and therefore
do not allow one to extract any information on the molecular component of the
$\X$ wave function.



There also have been many claims that production rates of multiquark states
in high-energy collisions are sensitive to a molecular admixture of those
states.
The production of the $\X$ at hadron colliders was debated
in~\cite{Bignamini:2009sk,Artoisenet:2009wk,Bignamini:2009fn,Artoisenet:2010uu,
Butenschoen:2013pxa,Esposito:2013ada,Meng:2013gga}, and that
of the spin and flavor partners of the $\X$ was discussed
in~\cite{Bignamini:2009fn,Guo:2013ufa,Guo:2014sca}. For the production of the
$\X$ in $B$ decays, we refer
to~\cite{Braaten:2004ai,Braaten:2004fk,Meng:2005er,Fan:2011aa,Meng:2013gga}.
The production of the $\X$ in heavy ion collisions was discussed in
\cite{Cho:2013rpa,Torres:2014fxa} and by the ExHIC Collaboration including other
hadronic molecular candidates~\cite{Cho:2010db,Cho:2011ew,Cho:2015exb}.
In Ref.~\cite{Larionov:2015nea} it was proposed that the hadronic
molecular component of the $\X$ could be extracted from its production in
antiproton-nucleus collisions. Prompt productions of
diquark-antidiquark tetraquarks at the LHC were studied
in~\cite{Guerrieri:2014gfa}.  Here, we discuss to what extent
 high-energy reactions can be used to
disentangle the structure of a near-threshold state.

The underlying physics for the short-distance production and decay processes of
a shallow hadronic molecule  is characterized by vastly different scales.
This allows for a derivation of
factorization formulae for the corresponding
amplitudes~\cite{Braaten:2004ai,Braaten:2004fk,Braaten:2005jj,Braaten:2006sy}.

%-------------------------------------------------------------------
\begin{figure}[t]
    \centering
    \includegraphics[width=\linewidth]{./figures/production_hm}
    \caption{Production of a pair of hadrons (a+b) and the hadronic molecule
formed by them (c) from a source $\Gamma$. Here the shaded area, the solid
lines and the double line denote the source, the constituent hadrons and the
hadronic molecule, respectively.
\label{fig:production}}
\end{figure}
%-------------------------------------------------------------------

We illustrate the production of a $h_1h_2$ pair and a near-threshold state
with wave function
$\Psi$ in Fig.~\ref{fig:production}, where $\Gamma$ denotes a short-distance
source. The pair of $h_1$ and $h_2$ can be produced directly at short
distances shown as diagram (a), and through rescattering shown as diagram (b).
If the rescattering strength were weak, the production would be well approximated by only the
$\Gamma$ term. There would also be no drastic energy dependence in the
near-threshold region, and the strength of the $S$-wave cusp exactly at the
threshold would not be strong enough to produce a narrow
peak~\cite{Guo:2014iya}.
However, if the rescattering is strongly attractive, the amplitude $T$
possesses a pole, which we assume to be located close to the threshold with a small binding
energy
$E_B$. Then one gets for  the production amplitude of a given decay channel $j$ of
the state of interest:
\begin{eqnarray}
    \mathcal{M}_j(\bm k; E) &=&  \Gamma_j^\Lambda(\bm k;E)  \\
    &+&    \sum_i \int_\Lambda\! \frac{d^3q}{(2\pi)^3}
\Gamma_i^\Lambda(\bm{q};E)\, G_i(\bm q;E)\,   T_{ij}(\bm q, \bm k; E)
    \nonumber
    \label{eq:Aproduction_pair}
\end{eqnarray}
where $G_i$ ($\Gamma_i^\Lambda$) is the propagator (short-distance production
amplitude) for the $i$-th intermediate channel and $\Lambda$ denotes the cutoff of
the integral. In principle all dynamical degrees of freedom below the energy
scale $\Lambda$ should be accounted for. The short-distance contribution
$\Gamma_i^\Lambda$ also serves to absorb the $\Lambda$ dependence, and as a
result $\mathcal{M}$ is $\Lambda$ independent.

Let us consider the kinematic situation that the CM momentum of
$h_1$ and $h_2$ is very small, $\sim \gamma$, and $\Lambda$ is much larger than
$\gamma^2/(2\mu)$ but still small enough to prevent other channels from being
dynamical. In this case, the intermediate state for diagram (b) is $h_1 h_2$.
The LO term of the momentum expansion of $\Gamma^\Lambda$ is simply a constant.
The nonrelativistic two-body propagator is $G(\bm q;E) = \left[ E - \bm
q^2/(2\mu) + i\epsilon \right]^{-1}$, and the $T$-matrix is given by
Eq.~\eqref{eq:T1c}. Thus, one obtains
\begin{equation}
  \mathcal{M}(\bm k;E)
  = \Gamma^\Lambda \left[ 1+ \frac{{\Lambda}/{\sqrt{2\pi}}
  -\sqrt{-2\mu E} + \order{\Lambda^{-1}}}{ \gamma -\sqrt{-2\mu E} } \right].
  \label{eq:Mfactorization}
\end{equation}
If $\Gamma^\Lambda\propto \Lambda^{-1}$, the LO $\Lambda$ dependence will be
absorbed~\cite{Braaten:2004ai}, and the factorization
formula~\cite{Braaten:2004ai,Braaten:2004fk,Braaten:2005jj} for the production
of the low-momentum $h_1h_2$ pair follows:
\begin{equation}
  \mathcal{M}(\bm k;E)
  = \frac{\Gamma\,\mu}{(2\pi)^{3/2}} T_\text{NR}^{}(E) + \order{\Lambda^{-1}}\,,
  \label{eq:factorization}
\end{equation}
where $\Gamma\equiv\Gamma^\Lambda\Lambda$ is the short-distance part, and the
long-distance part
$T_\text{NR}^{}(E)=(2\pi/\mu)\left(\gamma -\sqrt{-2\mu E} \right)^{-1} $ is
provided by the scattering $T$-matrix. From the derivation
in~\cite{Braaten:2006sy}, it becomes
clear that the
short-distance part is the Wilson coefficient of the operator production
expansion in the EFT.
A similar factorization formula was derived in~\cite{Guo:2014ppa} with the help
of chiral symmetry for high-energy productions of kaonic bound states predicted
in~\cite{Guo:2011dd}.

The amplitude for the production of the
near-threshold state  is obtained from Eq.~\eqref{eq:factorization} by
replacing $T_\text{NR}^{}(E)$ by the square root of its residue
$g_\text{NR}^{2}$
given in Eq.~\eqref{eq:gNR} and multiplying the factor $1/\sqrt{2\mu}$ to
account for the difference in normalization factors
\begin{equation}
  \mathcal{M}_\Psi
  = \frac{\Gamma\,\sqrt{\mu}}{4(\pi)^{3/2}} g_\text{NR} +
\order{\Lambda^{-1}}\,.
  \label{eq:factorizationX}
\end{equation}
Hence, the production rate, $\propto
g_\text{NR}^2\propto \sqrt{E_B}$, {\sl c.f.}  Eq.~\eqref{eq:residue}, seems
suppressed for very loosely bound states,
which is consistent with the common
intuition~\cite{Braaten:2004fk,Artoisenet:2009wk}.
In particular, one expects a suppression of
a loosely bound state in high-energy reactions.
The factorization explained above is the foundation for the proposal to extract
the short-distance production mechanism of $\X$ in $B_c$ semi-leptonic and
hadronic decays~\cite{Wang:2015rcz}.



Note that it is a straightforward consequence of Eq.~(\ref{eq:factorizationX})
that in ratios of short-distance production rates for two
hadronic molecules related to each other through some symmetry the long-distance
part containing the information of the structure of the states cancels, and the remaining part solely reflects the difference in the
short-distance dynamics and phase space. It could be misleading if such ratios
are taken as evidence in support of or against dominantly composite nature of
given states.



All the derivations of the factorization formula are based on the LO \nreftii~so
far, which allows self-consistently only the possibility that $\Psi$ is a
composite system of $h_1$ and $ h_2$, see the discussion below
Eq.~\eqref{eq:gNR}. If we go to higher orders, momentum-dependent terms need to
be kept in the potential as well as the short-distance production amplitude.
The probability of $\Psi$ to be a $h_1h_2$ composite system would be
$1-\lambda^2<1$, and thus one would also need to introduce a contact
production term for $\Psi$.  These new terms parameterize the
short-distance physics though a detailed dynamics which depends on the more
fundamental theory that cannot be specified within the EFT. Such contributions
could
be interpreted as a short-distance core of the physical wave function $\Psi$.
 The factorization and the related renormalization at higher orders for
the near-threshold production of $h_1 h_2$ and $\Psi$ in short-distance
processes remain to be worked out.


The long-distance contribution in Eqs.~\eqref{eq:factorization} and
\eqref{eq:factorizationX} is calculable in
NREFT, and the short-distance contribution is subject to the more fundamental
theories which are QCD and/or electroweak theory.  For inclusive high-energy
hadron collisions, one is not able to calculate the short-distance contribution
model-independently (for estimates using Monte Carlo event generators,
see~\cite{Bignamini:2009sk,Artoisenet:2009wk,Bignamini:2009fn,Artoisenet:2010uu,
Esposito:2013ada,Guo:2013ufa,Guo:2014ppa,Guo:2014sca}), and thus
only the order of magnitude of the production
cross sections can be estimated. For
the production of hadronic molecules in heavy meson decays such as $B$ decays,
again one normally is only able to get an order-of-magnitude estimate at the
best due to the nonperturbative nature of QCD which dominates the hadronic
effects in the short-distance part.
The production rate of the $\X$ in $B$ decays was estimated in
Ref.~\cite{Braaten:2004ai,Braaten:2004fk}. In particular,
Ref.~\cite{Braaten:2004ai} predicted
that the branching fraction of  $B^0\to \X K^0$ should be much smaller than
that of  $B^+\to\X K^+$ assuming that the $\X$ is a $D^0\bar D^{*0}$
hadronic molecule. The prediction seems to be in contradiction with the later
measurements~\cite{Olive:2016xmw} summarized in Sec.~\ref{sec:2} giving a
ratio of around 0.5, {\sl c.f.} Eq.~(\ref{eq:sec2X3872RB}).
However, as already mentioned before the $\X$ is to a very good approximation an isoscalar state.
Its couplings to the neutral $D^0\bar D^{*0}$ and the charged $D^+D^{*-}$
channels is almost the same even if the isospin breaking is taken into
account~\cite{Gamermann:2009fv,Guo:2014hqa}.
Therefore Eq.~(\ref{eq:factorizationX}) needs to be generalized to coupled channels.
In particular  the production of $\Psi$ gets modified to
\begin{equation}
  \mathcal{M}_\Psi
  = \frac{1}{4(\pi)^{3/2}} \sum_i \Gamma_i\,\sqrt{\mu_i} g_{\text{NR},i} +
\order{\Lambda^{-1}}\,,
  \label{eq:factorizationX_multichannel}
\end{equation}
where the summation runs over all possible intermediate channels below the
cutoff $\Lambda$. Notably,
$g_{\text{NR},i}$ denotes the coupling of the $\Psi$ state to the $i$-th
channel.  As a result, the $\X$
production rates in neutral and charged $B$ decays should be similar. This may
also be understood as that the short-distance parts in~\cite{Braaten:2004ai}
should include the charged channel and were not properly estimated.



We now turn to the discussion of production rates of shallow bound states
like $\X$ at hadron colliders. It was claimed  that
the cross section for the inclusive $X(3872)$ production at high $p_T$ at the
Tevatron in $\bar pp$ collisions is too large to be consistent with the
interpretation of $X(3872)$ as a $D^0\bar D^{*0}$
molecule~\cite{Bignamini:2009sk}. The reasoning was based on the following
estimate for an upper
bound of the cross section:
\begin{eqnarray}
\sigma(\bar pp\to X) &\sim& \left| \int d^3{\bm k}\,\langle X|D^0\bar D^{*0}
({\bm k})\rangle\langle D^0\bar D^{*0}({\bm k})|\bar pp\rangle\right|^2
\nonumber \\
&\simeq& \left| \int_{\cal R} d^3{\bm k}\,\langle X|D^0\bar D^{*0}({\bm
k}) \rangle\langle D^0\bar D^{*0}({\bm k})|\bar pp\rangle\right|^2 \nonumber
\\
&\leq& \int_{\cal R} d^3 {\bm k} \left|\Psi({\bm k})\right|^2
\int_{\cal R} d^3 {\bm k}\left|\langle D^0\bar D^{*0}({\bm k})|\bar
pp\rangle\right|^2 \nonumber \\
&\leq& \int_{\cal R} d^3 {\bm k}\left|\langle D^0\bar D^{*0}({\bm
k})|\bar pp\rangle\right|^2 \nonumber \\ &\sim&
\sigma(\bar pp\to X)^{\rm max} \ ,
\label{eq:keyargument}
\end{eqnarray}
%
where $\cal R$ means that the momentum integration only receives support up to
some characteristic scale
$\cal R$. The upper bound quoted above depends drastically on the value of
$\cal R$. A value of ${\cal R}=35$~MeV$\simeq\gamma$, the binding momentum of
the $\X$, was chosen in~\cite{Bignamini:2009sk}. The so-estimated upper
bound of 0.071~nb is orders of magnitude smaller than the Tevatron result
of 37 to 115~nb.

However, for the derivation in Eq.~\eqref{eq:keyargument} to be valid  $\cal
R$ must be large enough that the wave function of the bound state
gets largely probed for otherwise the symbol between the first and the second
integral needs to be changed
from $\simeq$ to $\gg$ and the whole line of reasoning gets spoiled. But this requirement calls for
 values of $\cal R$ much larger than the binding momentum.
 To demonstrate this claim we switch to the deuteron wave function.
 Fig.~\ref{fig:wfaveraged} shows
%-------------------------------------------------------------------
\begin{figure}[t]
    \centering
    \includegraphics[width=\linewidth]{./figures/normplots}
    \caption{Averaged deuteron wave functions for various cutoff values:
$\Lambda = 0.8, 1.5, 4$ GeV shown as black, red, blue curves, respectively. The
solid (dashed) lines show the result for the wave functions with (without)
one-pion exchange. Taken from~\cite{Albaladejo:2017}.
\label{fig:wfaveraged}}
\end{figure}
%-------------------------------------------------------------------
the averaged deuteron wave function calculated from
$\bar \Psi_\Lambda({\cal R}) = \int_{\cal R} d^3 \bm{k} \,
\Psi_\Lambda(\bm{k})$,
where the subindex $\Lambda$ indicates that a regulator needs to be specified to
get a well defined wave function (for more details,
see~\cite{Nogga:2005fv}).
The right panel in the figure is a zoom in linear scale to the relevant $\cal R$
range. From the figure, it is clear that $\bar \Psi_\Lambda({\cal R})$ is far
from being saturated for ${\cal R}\simeq$45~MeV which is the deuteron binding
momentum. One needs to take ${\cal R}\sim 2M_\pi\sim$300~MeV, the order of the
inverse range of forces as pointed out
in~\cite{Artoisenet:2009wk,Artoisenet:2010uu} based on rescattering arguments,
to get a reasonable estimate so that the second line in
Eq.~\eqref{eq:keyargument} can be a good approximation of the first line. With
such a large support $\cal R$, the upper bound becomes
consistent with the Tevatron
measurement~\cite{Bignamini:2009sk,Artoisenet:2009wk,Artoisenet:2010uu}.
In addition, as discussed in case of the $B$ decays, the charged channels need
to be considered as well.


%-----------------------------------------------------------
\begin{table}[t]
  \caption{Integrated cross sections (in units of nb) reported
in~\cite{Albaladejo:2017} for the inclusive $pp/\bar
p\to X(3872)$ processes in comparison with the
CDF~\cite{Bauer:2004bc} and CMS~\cite{Chatrchyan:2013cld} data converted into
cross sections~\cite{Guo:2014sca}. The ranges of the results cover those
obtained using both Pythia and Herwig. Here, we have
converted the experimental data into cross sections~\cite{Guo:2014sca}. }
\begin{ruledtabular}
\begin{tabular}{ l  c  c  c   }
  $\sigma(pp/\bar p\to X)$   & $\Lambda=0.1$~GeV   & $\Lambda=[0.5,1]$~GeV  &
Experiment \\
 \hline
 Tevatron   & 0.05-0.07         & 5-29   &  37-115   \\%
 LHC 7      & 0.04-0.12         & 4-55   &  13-39    \\%
\end{tabular}
\label{tab:crossec}
\end{ruledtabular}
\end{table}
%-----------------------------------------------------------
In fact, with the factorization derived above, it was suggested that one can
estimate the cross section by combining the use of Monte Carlo event
generators, such as Pythia~\cite{Sjostrand:2007gs} and
Herwig~\cite{Bahr:2008pv},
and EFT to get the short-distance and long-distance contributions,
respectively~\cite{Artoisenet:2009wk,Artoisenet:2010uu,Guo:2014sca}. In
Table~\ref{tab:crossec}, we show the estimates obtained
in~\cite{Albaladejo:2017}. Indeed, if a small cutoff is used,
$\Lambda=0.1$~GeV,  and only the neutral charmed
mesons are considered by using~\cite{Guo:2014ppa,Guo:2014sca}
\begin{eqnarray}
  \sigma(pp/\bar p\to X)
  &\approx& C\, 2\pi^2 \left|\int \frac{d^3{\bm
  k}}{(2\pi)^3} \Psi_X({\bm k})\right|^2 \, \nonumber\\
  &=& C\, 2\pi^2 \left| g_{\text{NR},X}^{} \Sigma_\text{NR}^{}(-E_B)\right|^2,
\end{eqnarray}
the obtained cross sections are orders of magnitude smaller than the data, in
line with the observation in~\cite{Bignamini:2009sk}.
Here $C=d\sigma[D^0\bar D^{*0}]/dk/k^2$, playing the role of
$|\Gamma^\Lambda|^2$ in Eq.~\eqref{eq:Mfactorization}, is a constant determined
from fitting to the
differential cross section of the direct production of the charmed-meson pair
from Pythia and Herwig, $\Psi_X(\bm{k})$ is the momentum-space wave function of
the $\X$, $g_{\text{NR},X}^{}$ is the coupling of the $\X$ to $D^0\bar D^{*0}$,
$\Sigma_\text{NR}(-E_B)$ is the loop function in Eq.~\eqref{eq:SigmaGaussian}
but keeping the full $\Lambda$ dependence from the Gaussian regulator. However,
when a larger cutoff in the range of, e.g., $[0.5,1.0]$~GeV is used, the cross
sections become consistent with both the CDF and CMS measurements. One
important point is that the charged $D^+D^{*-}+c.c.$ channel needs to be taken
into account for this case as discussed above. This is because the binding
momentum for the charged
channel $\gamma_\pm\simeq126$~MeV is well below the cutoff so that the charged
charmed mesons should also play a dynamical role.





Finally, in Ref.~\cite{Esposito:2015fsa}, the cross sections for the production
of light (hyper-)nuclei at small $p_T$ at ALICE~\cite{Adam:2015vda} were
extrapolated to large $p_T$, and it was found that they are much smaller than
the $\X$ production at large $p_T$ at CMS~\cite{Chatrchyan:2013cld}.
Since light (hyper-)nuclei are loosely bound states of baryons, the authors
concluded that loosely bound states are hardly produced at high $p_T$, and
therefore disfavored the hadronic molecular interpretation of the $\X$.
However, although the long-distance contributions for these productions can be
managed in the EFT or universality framework, the short-distance contribution
for the $\X$ is completely different from that for light nuclei. This leads to a
different energy dependence of the cross sections for the light nuclei and the
$\X$, and makes such a direct comparison questionable.
One essential difference is: At short distances, the $\X$ can be produced
through $c\bar c$ or $u\bar u (d\bar d)$, which hadronizes into a pair of
charmed mesons at larger distances, while the minimal quark number in the light
nuclei is always $3N$, with $N$ the number of baryons, giving rise to a
suppression.
Therefore, it is natural that the $\X$ production cross section at very high
$p_T$ is orders of magnitude larger than that of light nuclei. This point also
leads to the critique~\cite{Guo:2016fqg},\footnote{For a response, see
\cite{Brodsky:2016uln}.} which was appreciated in~\cite{Voloshin:2016phx},
against the use of constituent counting rules in hard exclusive processes as a way to identifying mulqituark states with a
hidden-flavor $q\bar q$
pair~\cite{Kawamura:2013iia,Kawamura:2013wfa,Brodsky:2015wza,Chang:2015ioc}.

To summarize, the production and decay processes with a large energy release
involve both long and short distance scales. Only the long-distance part is
sensitive to the low-energy quantities, thus to the hadronic molecular
structure, and can be dealt with in the EFT framework. On the contrary, high
energy production rates depend crucially on what happens at short distances,
which is often unknown though in principle could be extracted from other
reactions depending on the same short-distance physics.
However, despite that it is hard to calculate the integrated production rates,
the differential invariant mass distributions around the near-threshold states
provide a direct access to their line shapes and precise, high-resolution data
on those are urgently called for. This has been discussed in general in
Sec.~\ref{sec:lineshapes} and also in Sec.~\ref{sec:morelineshapes}.



\subsection{Implications of heavy quark spin and flavor symmetries}


It turns out that HQSS, and especially its breaking, is also an important
diagnostic tool when it comes to understanding the structure of certain
states~\cite{Cleven:2015era}. The most straightforward example to illustrate
this point is the spin doublet made of $D_{s0}^*(2317)$ and $D_{s1}(2460)$. On
the one hand, both are not only significantly lighter than the prediction of the
quark model as reported in, e.g., Ref.~\cite{DiPierro:2001dwf}, also their spin
splitting differs. On the other hand, the mass difference of the two states
agrees exactly to the mass difference between the $D$ and the $D^*$, which would
be a natural result if $D_{s0}(2317)$ and $D_{s1}^*(2460)$ were $DK$ and $D^*K$
molecular states, respectively, since at LO in the heavy quark expansion the
$DK$ interaction agrees to the $D^*K$ interaction~\cite{Kolomeitsev:2003ac}.
In complete analogy it was argued in Ref.~\cite{Guo:2009id} that, if indeed the
$J^{PC}=1^{--}$ state $Y(4660)$ were a bound system of $\psi'$ and $f_0(980)$ as
conjectured in Ref.~\cite{Guo:2008zg}, there should exist a $J^{PC}=0^{-+}$
state which is $\eta_c' f_0(980)$ bound system. The mass difference of the
latter state to the $\eta_c' f_0(980)$ should agree to that of $Y(4660)$ to the
$\psi' f_0(980)$ threshold,\footnote{This prediction receives support from a
calculation using QCD sum rules~\cite{Wang:2009cw}.} and it was even possible to
estimate the decay width of the state as well as a most suitable discovery channel.
For more discussion on the implications of HQSS for the spectrum of exotic
states we refer to Ref.~\cite{Cleven:2015era}, where also predictions from other
approaches are contrasted to those of the molecular picture.



%------------------------------------------------------------------------
\begin{table*}[tb]
\caption{\label{tab:partners} Possible spin and flavor partners of heavy-flavor
hadronic molecules. For the experimentally established states, the masses and
decay modes are taken from~\cite{Olive:2016xmw}. The predicted partners are
denoted by question marks.
The predictions from~\cite{Guo:2013sya} are those computed with a 0.5~GeV
cutoff, and the result from~\cite{Baru:2016iwj} is taken from the results with
the cutoff limited between 0.8 and 1.0~GeV.
}
\begin{ruledtabular}
\begin{tabular}{l c c c c }
        $J^{P(C)}$ & State & Main component & Mass (MeV) & (Expected) main decay
        mode(s) \\\hline
       $0^+$ & $D_{s0}^{*}(2317)$ & $DK$ & $2317.7\pm0.6$  & $D_s^+\pi^0$ \\
       $1^+$ & $D_{s1}(2460)$ & $D^*K$ & $2459.5\pm0.6$  &
$D_s^{*+}\pi^0,D_s^{(*)+}\gamma$ \\
       $0^+$ & $B_{s0}^{*}(?)$ & $B\bar K$ & $5730\pm16$  &
       $B_s^{*0}\gamma, B_s^0\pi^0$
       \\
       $1^+$ & $B_{s1}(?)$ & $B^*\bar K$ & $5776\pm16$  &
       $B_s^{(*)0}\gamma, B_s^{*0}\pi^0$
       \\\hline
       $1^-$ & $D_{s1}^*(2860)$ & $D_1(2420)K$ & $2859\pm27$ & $DK$, $D^*K$ \\
       $2^-$ & $D_{s2}^*(?)$ & $D_2(2460)K$ &
$2910\pm9$~\cite{Guo:2011dd} &
       $D^*K,D_s^*\eta$ \\
       $1^-$ & $B_{s1}^* (?)$ & $B_1(5720)\bar K$ &
$6151\pm33$~\cite{Guo:2011dd} & $B^{(*)} \bar K$, $B^{(*)}_s\eta $ \\
       $2^-$ & $B_{s2}^*(?)$ & $B_2(5747)\bar K$ &
$6169\pm33$~\cite{Guo:2011dd} &
       $B^{*} \bar K$, $B^{*}_s\eta$ \\
       \hline
       $1^{++}$ & $X(3872)$ & $D\bar D^*$ & $3871.69\pm0.17$ & $D^0\bar D^0\pi^0$,
       $J/\psi\pi\pi$, $J/\psi\pi\pi\pi$
       \\
       $2^{++}$ & $X_2 (?)$ & $D^*\bar D^*$ & $4012^{+4}_{-5}$~\cite{Guo:2013sya} & $D\bar
       D^{(*)}, J/\psi\omega$
       \\
        &      & &  $3980\pm20$~\cite{Baru:2016iwj}
         &
         \\
       $1^{++}$ & $X_b(?)$ & $B\bar B^*$ &  $10580^{+9}_{-8}$~\cite{Guo:2013sya} &
       $\Upsilon(nS)\omega,\chi_{bJ}\pi\pi$
       \\
       $2^{++}$ & $X_{b2}(?)$ & $B^*\bar B^*$ & $10626^{+8}_{-9}$~\cite{Guo:2013sya} &
       $B\bar B^{(*)},\Upsilon(nS)\omega,\chi_{bJ}\pi\pi$ \\
        $2^{+}$ & $X_{bc}(?)$ & $D^*B^*$& $7322^{+6}_{-7}$~\cite{Guo:2013sya} &
$DB,DB^*,D^*B$ \\
   \end{tabular}
\end{ruledtabular}
\end{table*}
%------------------------------------------------------------------------

More predictions can be made for heavy-flavor hadronic molecules by using heavy
quark flavor symmetry. The LO predictions are rather straightforward if there is
only a single heavy quark in the system. For instance, one would expect the
$D_{s0}^*(2317)$ as a $DK$ bound state to have a bottom partner, a $\bar B K$
bound state, with almost the same binding energy. This prediction together with
the one for the bottom partner of the $D_{s1}(2460)$ are given in the fourth and
fifth rows in Table~\ref{tab:partners}, where the error of 16~MeV accounts for
the use of heavy quark flavor symmetry and is estimated as $2\left( M_D+M_K-
M_{D_{s0}^*(2317) } \right)\times \Lambda_\text{QCD}(m_c^{-1}-m_b^{-1}) $. Such
simple predictions are in a remarkable agreement with the lattice results of the
lowest-lying $0^+$ and $1^+$ bottom-strange mesons:
$(5711\pm13\pm19)$~MeV for the $B_{s0}^*$ and $(5750\pm17\pm19)$~MeV for the
$B_{s1}$~\cite{Lang:2015hza}. This agreement may be regarded as a further
support of the hadronic molecular nature of the $D_{s0}^*(2317)$ and
$D_{s1}(2460)$ states.
For more complicated predictions of these two states using various versions of
UCHPT, we refer to
Refs.~\cite{Kolomeitsev:2003ac,Guo:2006fu,Guo:2006rp,Cleven:2010aw,
Altenbuchinger:2013vwa,Torres-Rincon:2014ffa,Cleven:2014oka}.

A more detailed discussion of spin symmetry partners of hadronic molecules
formed by a pair of $S$-wave heavy mesons can be found in
Sec.~\ref{sec:4-interactions}. Some of the spectroscopic consequences of HQSS
and heavy quark flavor symmetry for hadronic molecules are listed in
Table~\ref{tab:partners}.


\subsection{Baryon candidates for hadronic molecules}
\label{sec:1405th}

For a discussion of and references for charmed baryons and the $P_c$
pentaquark structures as
possible hadronic molecules we refer to Sec.~\ref{sec:2}.
The closing part of this section will be used to describe the most
recent developments about the $\Lambda(1405)$ which basically
settled the debate on the nature of this famous state. What is described
below is yet another example how the interplay of high quality data
and systematic theoretical investigations allows one to identify
the nature of certain states.


As already stressed in Sec.~\ref{sec:lam1405exp}, there are good reasons to
classify the $\Lambda(1405)$ as  an exotic particle, since it does not at all
fit into the pattern of the otherwise in this mass range quite successful quark
models.

%-----------------------------------------------------------
\begin{figure}[t!]
\begin{center}
\includegraphics[width=0.45\textwidth]{figures/poles_after_cut.pdf}
\caption{Positions of the two poles of the $\Lambda(1405)$ in the
complex plane. Filled cirlces:~\cite{Borasoy:2006sr},
filled squares:~\cite{Ikeda:2012au}, triangles:~~\cite{Guo:2012vv},
boxes:~\cite{Mai:2014xna}, spades:~\cite{Roca:2013av}. The labels and
colors are described in the text. Figure courtesy of Maxim Mai.
}
\label{fig:twopole}
\end{center}
\end{figure}

Using information on $\bar KN$ scattering the existence of the $\Lambda(1405)$
was predicted by Dalitz and Tuan~\cite{Dalitz:1959dn,Dalitz:1960du}  before its
observation.
Already this study highlights the importance of the $\bar K N$ dynamics for the
$\Lambda(1405)$. In most modern investigations it appears as a dynamically
generated state through coupled-channel effects among all the ten isospin
channels ($K^-p$, $\bar{K}^0n$, $\pi^0\Lambda$, $\pi^0\Sigma^0$,
$\pi^+\Sigma^-$, $\pi^-\Sigma^+$, $\eta\Lambda$, $\eta\Sigma^0$, $K^+\Xi^-$,
$K^0\Xi^0$) or some of them, in other words, a hadronic molecule. This
coupled-channel problem has been studied by solving the LSE or Bethe--Salpeter
equation with interaction kernels derived from chiral perturbation theory within
a given accuracy. This procedure was first proposed in Ref.~\cite{Kaiser:1995eg}
and further refined in Ref.~\cite{Oller:2000fj} and various follow-ups.
The major finding of Ref.~\cite{Oller:2000fj} was the fact that there are indeed
two poles, one stronger coupled to the $\bar KN$ channel and the other to
$\Sigma\pi$, which should thus be understood as two distinct states.
Both poles are located at the second Riemann sheet, and have shadow poles in
the third one~\cite{Oller:2000fj}.
This two-pole scenario can be understood by considering the SU(3) limit and its
subsequent breaking, see~\cite{Jido:2003cb}. Presently, various groups have
performed calculations to the NLO
accuracy~\cite{Ikeda:2012au,Guo:2012vv,Mai:2012dt}, see  also the recent
comparison of all these works in Ref.~\cite{Cieply:2016jby}
 and the mini-review by the PDG~\cite{Olive:2016xmw}. Including
the photoproduction data on $\gamma p\to K^+ \Sigma\pi$ from
CLAS~\cite{Moriya:2013eb}, one finds that the heavier of the poles is fairly
well pinned down, while the lighter one still shows some sizable spread in its
mass and width~\cite{Mai:2014xna}.
All this is captured nicely in Fig.~\ref{fig:twopole}. Fitting only the
scattering and threshold ratio data with the NLO kernel, one finds two poles,
but with very limited precision~\cite{Borasoy:2006sr}.
Adding the precise kaonic hydrogen data, the situation changes markedly, as
shown by the different solutions found by various groups,
see~\cite{Ikeda:2012au,Mai:2012dt,Guo:2012vv}.
Still, as first pointed out in Ref.~\cite{Mai:2014xna}, even with these data
there is a multitude of solutions with almost equal $\chi^2$, as in depicted by
the boxes labeled $1,\ldots,8$ in Fig.~\ref{fig:twopole}.
However, the photoproduction data severely constrain this space of solutions.
From the eight solutions only two survive, the blue (solution~2) and the green
(solution~4) boxes in the figure from Ref.~\cite{Mai:2014xna} as well as the
modified LO solution depicted by the spades from Ref.~\cite{Roca:2013av}. This
again is a nice example that only through an interplay of various reactions one
is able to pin down the precise structure of hadronic molecules (or other
hadronic resonances). Clearly, more data on $\pi\Sigma$ mass distributions would
be needed to further sharpen these conclusions, see e.g.~\cite{Ohnishi:2015iaq}.





