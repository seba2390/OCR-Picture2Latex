%%This is a very basic article template.
%%There is just one section and two subsections.
%\documentclass[12pt]{article}

\documentclass[aps,rmp,showpacs,twocolumn,nofootinbib,superscriptaddress,
reprint,amsmath,amssymb]{revtex4-1}

% change the citation style to the usual numbers,square
% \setcitestyle{numbers,square}
%\usepackage{url}
\usepackage{epsfig}
\usepackage{psfrag}
\usepackage{times}
\usepackage{amsfonts}
\usepackage{array}
\usepackage{stmaryrd,latexsym,bm,bbm}
\usepackage{color}
\usepackage{epstopdf}
\usepackage[%dvipdfm,
colorlinks=true,
linkcolor=black,
breaklinks=true,
urlcolor=blue,
citecolor=green]{hyperref}
\usepackage[utf8]{inputenc}

\renewcommand{\arraystretch}{1.2}

\definecolor{darkblue}{rgb}{0.,0.,0.7}
\definecolor{light-blue}{rgb}{0.8,0.85,1}
\definecolor{green}{rgb}{0,0.6,0}

\renewcommand{\arraystretch}{1.25}

\newcommand{\red}[1]{{\color{red}#1}}
\newcommand{\blue}[1]{{\color{blue}#1}}

\newcolumntype{L}{>{$}l<{$}} 
\newcolumntype{C}{>{$}c<{$}} 
\newcolumntype{R}{>{$}r<{$}} 


\newcommand{\br}[1]{\mathcal{B}\left(#1\right)}
\newcommand{\order}[1]{\mathcal{O}\!\left(#1\right)\!\,}

\newcommand{\be}{\begin{equation}}
\newcommand{\ee}{\end{equation}}
\newcommand{\beq}{\begin{equation}}
\newcommand{\eeq}{\end{equation}}
\newcommand{\ba}{\begin{eqnarray}}
\newcommand{\ea}{\end{eqnarray}}
\newcommand{\beqa}{\begin{eqnarray}}
\newcommand{\eeqa}{\end{eqnarray}}
\newcommand{\non}{\nonumber}
\newcommand{\sla}{\not\!}
\newcommand{\lang}{\left\langle}
\newcommand{\rang}{\right\rangle}
\newcommand{\al}{&\!\!\!}
\newcommand{\note}[1]{\textsl{\underline{Note added on #1}:}}

\newcommand{\Lag}{\mathcal{L}}
\newcommand{\Ham}{\mathcal{H}}
\newcommand{\tr}[1]{\left\langle #1 \right\rangle}
\newcommand{\Amp}{\mathcal{A}}
\newcommand{\Tr}{\textrm{Tr}}
\newcommand{\re}{\textrm{Re}}
\newcommand{\partiallr}{\overleftrightarrow{\partial}}
\newcommand{\Y}{Y(4260)}
\newcommand{\Z}{Z_c(3900)}
\newcommand{\nreft}{NREFT$_\text{I}$}
\newcommand{\nreftii}{NREFT$_\text{II}$}
\newcommand{\bd}{ \bm{  D  } }
\newcommand{\bdbar}{ \bm{  {\bar D } } }
\newcommand{\aoo}{a_{0}(980)}
\newcommand{\fo}{f_{0}(980)}
\newcommand{\Dso}{D_{s0}^*(2317)}
\newcommand{\Dsl}{D_{s1}(2460)}
\newcommand{\X}{X(3872)}
\newcommand{\Zc}{Z_c(3900)}
\newcommand{\Zcp}{Z_c(4020)}
\newcommand{\Zb}{Z_b(10610)}
\newcommand{\Zbp}{Z_b(10650)}
\newcommand{\mev}{\mathrm{MeV}}
\newcommand{\gev}{\mathrm{GeV}}
\newcommand{\kev}{\mathrm{keV}}
\newcommand{\tev}{\mathrm{TeV}}
\newcommand{\ev}{\mathrm{eV}}
\newcommand{\pb}{\mathrm{pb}}


\newcommand{\itp}{\affiliation{CAS Key Laboratory of Theoretical Physics,
            Institute of Theoretical Physics, Chinese Academy of Sciences,
            Beijing 100190, China}}

\newcommand{\bonn}{\affiliation{Helmholtz-Institut f\"ur Strahlen- und
             Kernphysik and Bethe Center for Theoretical Physics,
             Universit\"at Bonn,  D-53115 Bonn, Germany}}

\newcommand{\fzj}{\affiliation{Institute for
           Advanced Simulation, Institut f\"ur Kernphysik and
           J\"ulich Center for Hadron Physics, Forschungszentrum J\"ulich,
           D-52425 J\"ulich, Germany}}

\newcommand{\ihep}{\affiliation{Institute of High Energy Physics,
           Chinese Academy of Sciences, Beijing 100049, China}}

\newcommand{\ucas}{\affiliation{School of Physical Sciences,
            University of Chinese Academy of Sciences,
            Beijing 100049, China}}

\newcommand{\TPCSF}{\affiliation{Theoretical Physics Center for Science Facilities,
         Chinese Academy of Sciences, Beijing 100049, China}}

%\topmargin -1.5cm
\textheight 23cm

\date{\today}

\begin{document}

\title{{Hadronic molecules} }

\itp
\fzj
\bonn
\ihep
\ucas

\author{Feng-Kun~Guo} \email{fkguo@itp.ac.cn}\itp\ucas

\author{Christoph~Hanhart} \email{c.hanhart@fz-juelich.de}\fzj

\author{Ulf-G.~Mei\ss{}ner} \email{meissner@hiskp.uni-bonn.de}\bonn\fzj

\author{Qian~Wang} \email{wangqian@hiskp.uni-bonn.de}\bonn

\author{Qiang~Zhao} \email{zhaoq@ihep.ac.cn}\ihep\ucas\TPCSF

\author{Bing-Song~Zou} \email{zoubs@itp.ac.cn}\itp\ucas


\begin{abstract}


A large number of experimental discoveries especially in the heavy quarkonium
sector that did not at all fit to the expectations of the until then very
successful quark model led to a renaissance of hadron spectroscopy.
Among various explanations of the internal structure of these excitations,
hadronic molecules, being analogues of light nuclei, play a unique role since
for those predictions can be made with controlled uncertainty. We review
experimental evidences of various candidates of hadronic molecules, and methods
of identifying such structures. Nonrelativistic effective field theories are the
suitable framework for studying hadronic molecules, and are discussed in both
the continuum and finite volumes. Also pertinent lattice QCD results are
presented. Further, we discuss the production mechanisms and decays of hadronic
molecules, and comment on the reliability of certain assertions often made in
the literature.

\end{abstract}

\maketitle
\newpage

\tableofcontents

\newpage

\section{Introduction}\label{int}

Parallelism and concurrency \cite{CM} are the core concepts within computer science. There are mainly two camps in capturing concurrency: the interleaving concurrency and the true concurrency.

The representative of interleaving concurrency is bisimulation/weak bisimulation equivalences. CCS (A Calculus of Communicating Systems) \cite{CCS} \cite{CC} is a calculus based on bisimulation semantics model. CCS has good semantic properties based on the interleaving bisimulation. These properties include monoid laws, static laws, new expansion law for strongly interleaving bisimulation, $\tau$ laws for weakly interleaving bisimulation, and full congruences for strongly and weakly interleaving bisimulations, and also unique solution for recursion.

The other camp of concurrency is true concurrency. The researches on true concurrency are still active. Firstly, there are several truly concurrent bisimulations, the representatives are: pomset bisimulation, step bisimulation, history-preserving (hp-) bisimulation, and especially hereditary history-preserving (hhp-) bisimulation \cite{HHP1} \cite{HHP2}. These truly concurrent bisimulations are studied in different structures \cite{ES1} \cite{ES2} \cite{CM}: Petri nets, event structures, domains, and also a uniform form called TSI (Transition System with Independence) \cite{SFL}. There are also several logics based on different truly concurrent bisimulation equivalences, for example, SFL (Separation Fixpoint Logic) and TFL (Trace Fixpoint Logic) \cite{SFL} are extensions on true concurrency of mu-calculi \cite{MUC} on bisimulation equivalence, and also a logic with reverse modalities \cite {RL1} \cite{RL2} based on the so-called reverse bisimulations with a reverse flavor. Recently, a uniform logic for true concurrency \cite{LTC1} \cite{LTC2} was represented, which used a logical framework to unify several truly concurrent bisimulations, including pomset bisimulation, step bisimulation, hp-bisimulation and hhp-bisimulation.

There are simple comparisons between HM logic and bisimulation, as the uniform logic \cite{LTC1} \cite{LTC2} and truly concurrent bisimulations; the algebraic laws \cite{ALNC}, ACP \cite{ACP} and bisimulation, as the algebraic laws APTC \cite{ATC} and truly concurrent bisimulations; CCS and bisimulation, as truly concurrent bisimulations  and \emph{what}, which is still missing.

In this paper, we design a calculus for true concurrency (CTC) following the way paved by CCS for bisimulation equivalence. This paper is organized as follows. In section \ref{bac}, we introduce some preliminaries, including a brief introduction to CCS, and also preliminaries on true concurrency. We introduce the syntax and operational semantics of CTC in section \ref{sos}, its properties for strongly truly concurrent bisimulations in section \ref{stcb}, its properties for weakly truly concurrent bisimulations in section \ref{wtcb}. In section \ref{app}, we show the applications of CTC by an example called alternating-bit protocol. Finally, in section \ref{con}, we conclude this paper. 

%\documentclass[preprint,12pt]{elsarticle}
%\if0
\usepackage{amssymb}
\usepackage{mathtools}
%\usepackage[dvipdfmx]{graphicx}
\usepackage{cite}
\usepackage{graphicx}
\usepackage{bm}
\usepackage{here}
\usepackage[subrefformat=parens]{subcaption}
\fi
%\usepackage{amssymb}
\usepackage{amsmath}
\usepackage[dvipdfmx]{}
\usepackage[dvipdfmx]{color}
%\usepackage{cite}
%\usepackage{upgreek}
\usepackage{url}
%\usepackage[dvipdfmx]{hyperref}
%\usepackage{pxjahyper}
%\usepackage {hyperref}
\usepackage{graphicx}
\usepackage{bm}
\usepackage{here}
\usepackage{caption}
\usepackage[subrefformat=parens]{subcaption}
\captionsetup{compatibility=false}

%% The amsthm package provides extended theorem environments
%% \usepackage{amsthm}

%% The lineno packages adds line numbers. Start line numbering with
%% \begin{linenumbers}, end it with \end{linenumbers}. Or switch it on
%% for the whole article with \linenumbers after \end{frontmatter}.
%% \usepackage{lineno}

%% natbib.sty is loaded by default. However, natbib options can be
%% provided with \biboptions{...} command. Following options are
%% valid:

%%   round  -  round parentheses are used (default)
%%   square -  square brackets are used   [option]
%%   curly  -  curly braces are used      {option}
%%   angle  -  angle brackets are used    <option>
%%   semicolon  -  multiple citations separated by semi-colon
%%   colon  - same as semicolon, an earlier confusion
%%   comma  -  separated by comma
%%   numbers-  selects numerical citations
%%   super  -  numerical citations as superscripts
%%   sort   -  sorts multiple citations according to order in ref. list
%%   sort&compress   -  like sort, but also compresses numerical citations
%%   compress - compresses without sorting
%%
%% \biboptions{comma,round}

% \biboptions{}

%% This list environment is used for the references in the
%% Program Summary
%%
\newcounter{bla}
\newenvironment{refnummer}{%
\list{[\arabic{bla}]}%
{\usecounter{bla}%
 \setlength{\itemindent}{0pt}%
 \setlength{\topsep}{0pt}%
 \setlength{\itemsep}{0pt}%
 \setlength{\labelsep}{2pt}%
 \setlength{\listparindent}{0pt}%
 \settowidth{\labelwidth}{[9]}%
 \setlength{\leftmargin}{\labelwidth}%
 \addtolength{\leftmargin}{\labelsep}%
 \setlength{\rightmargin}{0pt}}}
 {\endlist}
\begin{document}

\section{O-SUKI-N 3D code algorithm description}
\par

\subsection{O-SUKI-N 3D code structure}
     The O-SUKI-N 3D code system consists of three parts: The Lagrangian fluid code \cite{Schulz}, the data conversion code from the Lagrangian code to the Euler code, and Euler code. The fluid model is the three-temperature model in Ref. \cite{Tahir}. The Lagrangian fluid code, the data conversion code and the Euler code are described below in detail. 
     
     In the Lagrangian fluid code the spatial meshes move together with the fluid motion \cite{Schulz}. However, the Lagrange meshes can not follow the fluid large deformation. On the other hand, the Euler meshes are fixed to the space, and the fluid moves through the meshes. Therefore, just before the void closure time, that is, the stagnation phase, the Lagrangian code is used to simulate the DT fuel implosion. After the void closure time, the Euler code is employed to simulate the DT fuel further compression, ignition and burning. Between the Lagrangian code and the Euler code the data should be converted by the data conversion code. 

	All the simulation process is performed in its integrated way by using the script of "CodeO-SUKI-N-fusion-start.sh". The processes executed by this shell script are as follows: \\
1. Make the stack size infinite.\\
2. Remove all output data file and make the new output files.\\
3. Change the permission of shell scripts to executable. \\
4. Compile the main function of the Lagrangian code and execute it.\\
5. If any problems do not appear during the calculation of the Lagrangian code, compile the main function of the data conversion code and execute it.\\
6. If there is no problem during the data conversion, compile the main function of the Euler code and execute it.\\
     

\subsection{Steps in Lagrangian code}\par
     The Lagrangian code has the following steps: 

\begin{enumerate}
\item Initialize the variables and calculation of total input energy. \par
\item Calculation of time step size.\par
\item Calculation of coordinates.\par
\item Solve equation of motion. \par
\item Solve density by equation of continuity.\par
\item Calculation of artificial viscosity.\par
\item Transfer the data to the OK3. \par
\item Calculation of energy deposition distribution in code OK3. For details of the OK3, see the refs.\cite{ogoyski1,ogoyski2,ogoyski3}. \par
\item Solve energy equations\par
\item Calculation of heat conduction\par
\item Calculation of temperature relaxation among three temperatures.\par
\item Solve equation of state\par
\item Save the results.\par
\item End the Lagrangian calculation right before the void closure.\par
\item Transfer the data to converting code. \par
\end {enumerate}


\subsection{Data Conversion code from Lagrangian fluid code to Euler fluid code}

\begin {enumerate}
\item Read variables saved in Lagrangian code.\par
\item Generate the Eulerian mesh.\par
\item Calculate the interpolation of the physical quantity to them on the Eulerian mesh.\par
\item Write the converted data to the Eulerian code.\par
\end {enumerate}


\subsection{Steps in Eulerian code}

\begin {enumerate}
\item Read the mesh number from the converted data and define the each matrices.\par
\item Initialize the variables.\par
\item Calculation of time step size.\par
\item Solve equation of motion. \par
\item Track the material boundaries of DT, Al and Pb.\par
\item Linearly interpolate the boundary lines and transcribe them on the Eulerian code. \par
\item Discriminate the materials by using the transferred boundary line. \par
\item Solve density by equation of continuity.\par
\item Calculate artificial viscosity.\par
\item Solve energy equations\par
\item Calculation of fusion reaction.\par
\item Calculation of heat conduction\par
\item Calculation of temperature relaxation among three temperatures.\par
\item Solve equation of state.\par
\item Save the results.\par
\item End.
\end{enumerate}

%\include{end}

\section{Syntax and Operational Semantics}\label{sos}

We assume an infinite set $\mathcal{N}$ of (action or event) names, and use $a,b,c,\cdots$ to range over $\mathcal{N}$. We denote by $\overline{\mathcal{N}}$ the set of co-names and let $\overline{a},\overline{b},\overline{c},\cdots$ range over $\overline{\mathcal{N}}$. Then we set $\mathcal{L}=\mathcal{N}\cup\overline{\mathcal{N}}$ as the set of labels, and use $l,\overline{l}$ to range over $\mathcal{L}$. We extend complementation to $\mathcal{L}$ such that $\overline{\overline{a}}=a$. Let $\tau$ denote the silent step (internal action or event) and define $Act=\mathcal{L}\cup\{\tau\}$ to be the set of actions, $\alpha,\beta$ range over $Act$. And $K,L$ are used to stand for subsets of $\mathcal{L}$ and $\overline{L}$ is used for the set of complements of labels in $L$. A relabelling function $f$ is a function from $\mathcal{L}$ to $\mathcal{L}$ such that $f(\overline{l})=\overline{f(l)}$. By defining $f(\tau)=\tau$, we extend $f$ to $Act$.

Further, we introduce a set $\mathcal{X}$ of process variables, and a set $\mathcal{K}$ of process constants, and let $X,Y,\cdots$ range over $\mathcal{X}$, and $A,B,\cdots$ range over $\mathcal{K}$, $\widetilde{X}$ is a tuple of distinct process variables, and also $E,F,\cdots$ range over the recursive expressions. We write $\mathcal{P}$ for the set of processes. Sometimes, we use $I,J$ to stand for an indexing set, and we write $E_i:i\in I$ for a family of expressions indexed by $I$. $Id_D$ is the identity function or relation over set $D$.

For each process constant schema $A$, a defining equation of the form

$$A\overset{\text{def}}{=}P$$

is assumed, where $P$ is a process.

\subsection{Syntax}

We use the Prefix $.$ to model the causality relation $\leq$ in true concurrency, the Summation $+$ to model the conflict relation $\sharp$ in true concurrency, and the Composition $\parallel$ to explicitly model concurrent relation in true concurrency. And we follow the conventions of process algebra.

\begin{definition}[Syntax]\label{syntax}
Truly concurrent processes are defined inductively by the following formation rules:

\begin{enumerate}
  \item $A\in\mathcal{P}$;
  \item $\textbf{nil}\in\mathcal{P}$;
  \item if $P\in\mathcal{P}$, then the Prefix $\alpha.P\in\mathcal{P}$, for $\alpha\in Act$;
  \item if $P,Q\in\mathcal{P}$, then the Summation $P+Q\in\mathcal{P}$;
  \item if $P,Q\in\mathcal{P}$, then the Composition $P\parallel Q\in\mathcal{P}$;
  \item if $P\in\mathcal{P}$, then the Prefix $(\alpha_1\parallel\cdots\parallel\alpha_n).P\in\mathcal{P}\quad(n\in I)$, for $\alpha_,\cdots,\alpha_n\in Act$;
  \item if $P\in\mathcal{P}$, then the Restriction $P\setminus L\in\mathcal{P}$ with $L\in\mathcal{L}$;
  \item if $P\in\mathcal{P}$, then the Relabelling $P[f]\in\mathcal{P}$.
\end{enumerate}

The standard BNF grammar of syntax of CTC can be summarized as follows:

$$P::=A\quad|\quad\textbf{nil}\quad|\quad\alpha.P\quad|\quad P+P\quad |\quad P\parallel P\quad |\quad (\alpha_1\parallel\cdots\parallel\alpha_n).P \quad|\quad P\setminus L\quad |\quad P[f].$$
\end{definition}

\subsection{Operational Semantics}

The operational semantics is defined by LTSs (labelled transition systems), and it is detailed by the following definition.

\begin{definition}[Semantics]\label{semantics}
The operational semantics of CTC corresponding to the syntax in Definition \ref{syntax} is defined by a series of transition rules, named $\textbf{Act}$, $\textbf{Sum}$, $\textbf{Com}$, $\textbf{Res}$, $\textbf{Rel}$ and $\textbf{Con}$ indicate that the rules are associated respectively with Prefix, Summation, Composition, Restriction, Relabelling and Constants in Definition \ref{syntax}. They are shown in Table \ref{TRForCTC}.

\begin{center}
    \begin{table}
        \[\textbf{Act}_1\quad \frac{}{\alpha.P\xrightarrow{\alpha}P}\]

        \[\textbf{Sum}_1\quad \frac{P\xrightarrow{\alpha}P'}{P+Q\xrightarrow{\alpha}P'}\]

        \[\textbf{Com}_1\quad \frac{P\xrightarrow{\alpha}P'\quad Q\nrightarrow}{P\parallel Q\xrightarrow{\alpha}P'\parallel Q}\]

        \[\textbf{Com}_2\quad \frac{Q\xrightarrow{\alpha}Q'\quad P\nrightarrow}{P\parallel Q\xrightarrow{\alpha}P\parallel Q'}\]

        \[\textbf{Com}_3\quad \frac{P\xrightarrow{\alpha}P'\quad Q\xrightarrow{\beta}Q'}{P\parallel Q\xrightarrow{\{\alpha,\beta\}}P'\parallel Q'}\quad (\beta\neq\overline{\alpha})\]

        \[\textbf{Com}_4\quad \frac{P\xrightarrow{l}P'\quad Q\xrightarrow{\overline{l}}Q'}{P\parallel Q\xrightarrow{\tau}P'\parallel Q'}\]

        \[\textbf{Act}_2\quad \frac{}{(\alpha_1\parallel\cdots\parallel\alpha_n).P\xrightarrow{\{\alpha_1,\cdots,\alpha_n\}}P}\quad (\alpha_i\neq\overline{\alpha_j}\quad i,j\in\{1,\cdots,n\})\]

        \[\textbf{Sum}_2\quad \frac{P\xrightarrow{\{\alpha_1,\cdots,\alpha_n\}}P'}{P+Q\xrightarrow{\{\alpha_1,\cdots,\alpha_n\}}P'}\]

        \[\textbf{Res}_1\quad \frac{P\xrightarrow{\alpha}P'}{P\setminus L\xrightarrow{\alpha}P'\setminus L}\quad (\alpha,\overline{\alpha}\notin L)\]

        \[\textbf{Res}_2\quad \frac{P\xrightarrow{\{\alpha_1,\cdots,\alpha_n\}}P'}{P\setminus L\xrightarrow{\{\alpha_1,\cdots,\alpha_n\}}P'\setminus L}\quad (\alpha_1,\overline{\alpha_1},\cdots,\alpha_n,\overline{\alpha_n}\notin L)\]

        \[\textbf{Rel}_1\quad \frac{P\xrightarrow{\alpha}P'}{P[f]\xrightarrow{f(\alpha)}P'[f]}\]

        \[\textbf{Rel}_2\quad \frac{P\xrightarrow{\{\alpha_1,\cdots,\alpha_n\}}P'}{P[f]\xrightarrow{\{f(\alpha_1),\cdots,f(\alpha_n)\}}P'[f]}\]

        \[\textbf{Con}_1\quad\frac{P\xrightarrow{\alpha}P'}{A\xrightarrow{\alpha}P'}\quad (A\overset{\text{def}}{=}P)\]

        \[\textbf{Con}_2\quad\frac{P\xrightarrow{\{\alpha_1,\cdots,\alpha_n\}}P'}{A\xrightarrow{\{\alpha_1,\cdots,\alpha_n\}}P'}\quad (A\overset{\text{def}}{=}P)\]

        \caption{Transition rules of CTC}
        \label{TRForCTC}
    \end{table}
\end{center}
\end{definition}

\subsection{Properties of Transitions}

\begin{definition}[Sorts]\label{sorts}
Given the sorts $\mathcal{L}(A)$ and $\mathcal{L}(X)$ of constants and variables, we define $\mathcal{L}(P)$ inductively as follows.

\begin{enumerate}
  \item $\mathcal{L}(l.P)=\{l\}\cup\mathcal{L}(P)$;
  \item $\mathcal{L}((l_1\parallel \cdots\parallel l_n).P)=\{l_1,\cdots,l_n\}\cup\mathcal{L}(P)$;
  \item $\mathcal{L}(\tau.P)=\mathcal{L}(P)$;
  \item $\mathcal{L}(P+Q)=\mathcal{L}(P)\cup\mathcal{L}(Q)$;
  \item $\mathcal{L}(P\parallel Q)=\mathcal{L}(P)\cup\mathcal{L}(Q)$;
  \item $\mathcal{L}(P\setminus L)=\mathcal{L}(P)-(L\cup\overline{L})$;
  \item $\mathcal{L}(P[f])=\{f(l):l\in\mathcal{L}(P)\}$;
  \item for $A\overset{\text{def}}{=}P$, $\mathcal{L}(P)\subseteq\mathcal{L}(A)$.
\end{enumerate}
\end{definition}

Now, we present some properties of the transition rules defined in Table \ref{TRForCTC}.

\begin{proposition}
If $P\xrightarrow{\alpha}P'$, then
\begin{enumerate}
  \item $\alpha\in\mathcal{L}(P)\cup\{\tau\}$;
  \item $\mathcal{L}(P')\subseteq\mathcal{L}(P)$.
\end{enumerate}

If $P\xrightarrow{\{\alpha_1,\cdots,\alpha_n\}}P'$, then
\begin{enumerate}
  \item $\alpha_1,\cdots,\alpha_n\in\mathcal{L}(P)\cup\{\tau\}$;
  \item $\mathcal{L}(P')\subseteq\mathcal{L}(P)$.
\end{enumerate}
\end{proposition}

\begin{proof}
By induction on the inference of $P\xrightarrow{\alpha}P'$ and $P\xrightarrow{\{\alpha_1,\cdots,\alpha_n\}}P'$, there are fourteen cases corresponding to the transition rules named $\textbf{Act}_{1,2}$, $\textbf{Sum}_{1,2}$, $\textbf{Com}_{1,2,3,4}$, $\textbf{Res}_{1,2}$, $\textbf{Rel}_{1,2}$ and $\textbf{Con}_{1,2}$ in Table \ref{TRForCTC}, we just prove the one case $\textbf{Act}_1$ and $\textbf{Act}_2$, and omit the others.

Case $\textbf{Act}_1$: by $\textbf{Act}_1$, with $P\equiv\alpha.P'$. Then by Definition \ref{sorts}, we have (1) $\mathcal{L}(P)=\{\alpha\}\cup\mathcal{L}(P')$ if $\alpha\neq\tau$; (2) $\mathcal{L}(P)=\mathcal{L}(P')$ if $\alpha=\tau$. So, $\alpha\in\mathcal{L}(P)\cup\{\tau\}$, and $\mathcal{L}(P')\subseteq\mathcal{L}(P)$, as desired.

Case $\textbf{Act}_2$: by $\textbf{Act}_2$, with $P\equiv(\alpha_1\parallel\cdots\parallel\alpha_n).P'$. Then by Definition \ref{sorts}, we have (1) $\mathcal{L}(P)=\{\alpha_1,\cdots,\alpha_n\}\cup\mathcal{L}(P')$ if $\alpha_i\neq\tau$ for $i\leq n$; (2) $\mathcal{L}(P)=\mathcal{L}(P')$ if $\alpha_1,\cdots,\alpha_n=\tau$. So, $\alpha_1,\cdots,\alpha_n\in\mathcal{L}(P)\cup\{\tau\}$, and $\mathcal{L}(P')\subseteq\mathcal{L}(P)$, as desired.
\end{proof} 


\section{Nonrelativistic effective field theories}
\label{sec:4}

All the candidates for hadron resonances, and in particular the candidates of
hadronic molecules, which are the focus of this review, were discovered via
their strong decays into other hadrons. Therefore, to understand these
structures requires  also a study of their decays. Because of the
nonperturbative nature of QCD at hadronic energy scales, a first-principle
calculation of the spectrum of hadronic resonances at the level of quarks and
gluons can only be done using lattice QCD.
Although there has been tremendous progress in lattice QCD, a reliable
calculation of the full hadronic resonance spectrum for physical quark masses is
still out of reach. In addition, even if such calculations were available, the
interpretation of the emerging spectra still requires
 additional theoretical analyses.

Only in the special case discussed in Sec.~\ref{sec:3}, {\sl i.e.} for shallow bound
states coupling in an $S$-wave to a nearby continuum channel comprised of two
stable or at least narrow hadrons, one finds a direct and physical
interpretation for the leading and nonanalytic contribution of the wave function
renormalization constant $Z$ as the (normalizable) probability to find the
continuum contribution in the physical state.
 Because of the closeness of the
threshold to the mass of the physical composite state, such systems are ideal
objects to apply the concept of effective field theories (EFTs), which makes use
of the separation of scales and which per definition include  a
cutoff~\cite{Lepage:1989hf}. Of particular relevance here are the
nonrelativistic EFTs~(NREFTs). Note that the
general principles underlying any EFT are formulated in Weinberg's
paper on phenomenological Lagrangians~\cite{Weinberg:1978kz}.

As mentioned in Sec.~\ref{sec:3}, hadronic molecules are located close to some
strongly coupled thresholds.
We denote the low-energy (low-momentum) scale characterizing such a system,
given by the binding energy (binding momentum) defined in Eq.~(\ref{eq:Ebdef})
(Eq.~(\ref{eq:gamdef})), generically by $Q$.
All other hadronic scales that we may collectively label as $\Lambda$ are thus
regarded as hard. This enables one to construct a perturbation theory in
$Q/\Lambda$, which for near-threshold states should be a small number.
As will become clear below, it depends on the system which scale is appropriate
for $\Lambda$. For example when investigating the $f_0(980)$ as a candidate for a
$\bar KK$ molecular state, the inverse range of forces, the natural candidate
for $\Lambda$, is given by the mass of the allowed lightest exchange meson, the
rho meson. A phenomenologically adequate value for the binding energy is 10~MeV.
It corresponds to a binding momentum of 70~MeV, and thus $Q/\Lambda \sim
1/10$ is a good expansion parameter.\footnote{The subtle interplay of scales in
molecular transitions is  is discussed in detail in Ref.~\cite{Hanhart:2007wa}
on the example of decays of the $f_0(980)$.} Furthermore, the closeness to
threshold also means that the constituent hadrons can be treated
nonrelativistically.

As discussed in the preceding section, the most interesting information about
the structure of a near-threshold state is contained in its coupling strength to
the threshold channel, which measures the probability for finding the two-body
bound state component in the physical state. This is consistent with the
intuition that a state is the more composite the larger its coupling to the
continuum. As shown in Eq.~\eqref{eq:residue}, for a bound state the coupling
reaches its maximal value, if the physical state is purely an $S$-wave bound
state, $\lambda^2=0$. Hence, it is important to extract the value of the
coupling constant for understanding the nature of near-threshold structures.
In addition, a large coupling implies the prominence of hadronic loops not only
in the formation of the state but also in transitions and decays.
In this section, we will discuss the NREFT formalism which is a natural
framework for studying the transitions involving hadronic molecules with a small
energy release. It can also be used to compute the universal long-distance part
of the production/decay processes of hadronic molecules, which will be discussed
in Sec.~\ref{sec:6}.

The analytic structure of the three-point scalar loop integral (including the
TS) will be discussed in Sec.~\ref{sec:4-3ploop}. The power counting rules
for the NREFT treating all intermediate particles on the same footing will be
detailed in Sec.~\ref{sec:nreft1}. We denote such a theory  as \nreft. When
one of the intermediate particles is much more off-shell than the others, it can
be integrated out from \nreft~and one gets another effective field theory, here
called \nreftii, which  was originally introduced as XEFT to study the
properties of the $X(3872)$. The XEFT and its relation to \nreft~will be
discussed in Secs.~\ref{sec:4-XEFT} and \ref{sec:4-nreft2xeft}. 
Sec.~\ref{sec:4-interactions} is devoted to a brief description of the
formation of hadronic molecules.

\begin{figure}[t!]
  \begin{center}
   \vglue2mm
   \includegraphics[width=0.65\linewidth]{./figures/trianglediag}
   \caption{ A triangle diagram illustrating the long-distance contribution to
the transition between two heavy particles $A$ and $B$ with the emission of a
light particle $C$. The two vertical dashed lines denote the two relevant cuts.
   \label{fig:triangle}}
  \end{center}
\end{figure}

The formation of hadronic molecules may be viewed as a result of
nonperturbative hadron-hadron interactions. It is therefore natural to ask if
there is also an impact of hadron loops on the properties of more regular
excited hadrons.  Indeed, for certain transitions the effective field theory
\nreft~mentioned above  predicts prominent loop effects.
As examples, we briefly discuss single-pion/eta transitions and hindered M1
transitions between heavy quarkonia in Sec.~\ref{sec:4-NREFT_ccbar}. It will
become clear that whether the hadron-loop effects are important for
properties of an excited hadron is process-dependent. In particular, the
location of an excited hadron close to a threshold is a necessary but not
a sufficient condition.



\subsection{Power counting schemes}
\label{sec:4-pc}


As demonstrated in Sec.~\ref{sec:3}, the decisive feature of molecular states as
compared to more compact structures is the prominence of a two-hadron cut.
In some decays the cuts induced by intermediate particles might also matter.
To illustrate this point, we start this section by a discussion of the analytic
structure of three-point loop functions. This will shed light on the NREFT
power counting as well.


\subsubsection{Analytic structure of the three-point loop integral}
\label{sec:4-3ploop}


If a hadronic molecule has at least one unstable constituent, it can decay
directly through the decays of that unstable particle when phase space allows.
It can also decay into another heavy particle with a mass of the same order by
emitting light particles such as pions or photons from its constituents. The
mechanism for a transition accompanied by the emission of a single light
particle is depicted in Fig.~\ref{fig:triangle}.
In the figure the two vertical dashed lines show the relevant branch cuts:
They correspond to the time slices at which the intermediate particles can go
onto their mass shells.

We denote the intermediate particles as 
$M_{1,2,3}$ with masses $m_{1,2,3}$, and the external particles as $A,B,C$ 
with masses $m_{A,B,C}$, as shown in Fig.~\ref{fig:triangle}.  If 
all intermediate particles are nonrelativistic we can formulate a power 
counting based on the velocities of the intermediate particles. 
%
Let us start from the scalar triangle loop integral
\begin{widetext}
\begin{eqnarray}
    I(q) &=& i\int\!\frac{d^4l}{(2\pi)^4}
\frac{1}{\left(l^2-m_1^2+i\epsilon\right) \left[(P-l)^2-m_2^2+i\epsilon\right]
    \left[(l-q)^2-m_3^2+i\epsilon\right]}
    \nonumber \\
    &\simeq& \frac{i}{N_m} \int\!\frac{dl^0 d^3\bm l}{(2\pi)^4}
\frac1{\left[l^0{-}T_1(|\bm l|)+i\epsilon\right] \left[P^0{-}l^0{-}
T_2(|\bm{l}|)+i\epsilon\right] \left[l^0{-}E_C{-}T_3(|\bm
l{-}\bm{q}|)+i\epsilon\right] },
    \label{eq:scalarI}
\end{eqnarray}
where $\epsilon=0^+$, $N_m=8m_1m_2m_3$, $T_i(p)=p^2/2m_i$ denotes the kinetic 
energy for a 
heavy meson
with mass $m_i$, and $E_C$ the energy of the particle $C$ in the rest frame 
of the initial particle $A$. The second line is obtained by treating all 
the intermediate states 
nonrelativistically in the rest frame of the initial particle. Performing the 
contour integration over $l^0$, one gets a
convergent integral over the three-momentum. Defining
$\mu_{ij}=m_im_j/(m_i+m_j)$, $b_{12} = m_1+m_2-m_A$ and 
$b_{23}=m_2+m_3+E_C-m_A$, one has
\begin{equation}
  I(q) \simeq \frac{4\mu_{12}\mu_{23} }{N_m} \int\! \frac{d^3\bm 
l}{(2\pi)^3}\left[
  \left(\bm{l}^{\,2} + c_1 -i\epsilon\right) \left(\bm{l}^{\,2} + c_2 - 
\frac{2\mu_{23}}{m_3} \bm{l}\cdot\bm{q} - i\epsilon
  \right) \right]^{-1},
  \label{eq:loopinter}
\end{equation}
where $c_1= 2\mu_{12}b_{12}$, and
$c_2=2\mu_{23}b_{23}+\left(\mu_{23}/m_3\right){ q}^2$ with $q\equiv|\bm q|$.
The two terms in the denominator of the integrand
 contain a unitary cut each, as indicated by the
vertical dashed lines in Fig.~\ref{fig:triangle}. The other
two-body cut crossing the lines of $M_1$ and $M_3$ corresponds to the case that
the particle $M_3$ is propagating back in time (we assume implicitly that
it is $M_1$ and not $M_2$ that can decay to $M_3$ and $C$ in near on-shell 
kinematics). This is a relativistic effect which is
neglected here. The intermediate particles $M_1$ 
and $M_2$ are on shell when $\bm l^{\,2} + c_1 = 0$; $M_2$ and $M_3$ (as well 
as $C$) are on shell for $\bm{l}^{\,2} + c_2 - 
{2\mu_{23}} \bm{l}\cdot\bm{q}/m_3 = 0$. Accordingly,
 $\sqrt{|c_1|}$ and $\sqrt{|c_2|}$ define  two
different momentum scales where the corresponding intermediate states go on
shell.
Their values depend on all of the masses involved and may be very different 
from each other. For the nonrelativistic approximation to hold both must be 
small compared
to $m_i (i=1,2,3)$.

The integral of Eq.~(\ref{eq:loopinter}) can be presented in closed 
form~\cite{Guo:2010ak,Mehen:2015efa}
\begin{eqnarray}
    I(q) &=& {\cal N} \frac{1}{\sqrt{a}} \left[
\arctan\left(\frac{c_2-c_1}{2\sqrt{a(c_1-i\epsilon)}}\right) -
\arctan\left(\frac{c_2-c_1-2a}{2\sqrt{a(c_2-a-i\epsilon)}}\right) \right],
    \label{eq:Iexp} \\
    &=& {\cal N} \frac{1}{\sqrt{a}} \left[
\arcsin\left(\frac{c_2-c_1}{\sqrt{(c_2-c_1)^2+4ac-i\epsilon}}\right)
    - 
\arcsin\left(\frac{c_2-c_1-2a}{\sqrt{(c_2-c_1)^2+4ac-i\epsilon}}\right)
\right].
    \label{eq:Iexp2}
\end{eqnarray}
\end{widetext}
where ${\cal N}=\mu_{12}\mu_{23}/(2\pi m_1m_2m_3)$, and
$a = \left(\mu_{23}/m_3\right)^2 { q}^2$.
Especially Eq.~(\ref{eq:Iexp2}) highlights the presence of a special singularity
at
\begin{equation}
  (c_2-c_1)^2+4ac_1 = 0 \, .
  \label{eq:nrtrising}
\end{equation}
When rewriting the inverse trigonometric functions in terms of 
logarithms, one finds that this is a logarithmic divergence.
The solution of this equation gives the leading Landau 
singularity~\cite{Landau:1959fi} (for early
and recent reviews, see~\cite{Eden:1966,Chang:1983,Aitchison:2015jxa})
for a triangle diagram, also called triangle singularity, evaluated 
in nonrelativistic kinematics~\cite{Guo:2014qra}. The singularity location is 
slightly 
shifted from that found by solving the relativistic Landau equation. A 
comparison for a specific example can be found in the appendix of 
Ref.~\cite{Guo:2014qra}.  

Being nonlinear in all of the involved masses, Eq.~\eqref{eq:nrtrising} as well
as the Landau equation allow for different solutions. However, a direct
evaluation of the loop integral reveals that only in a very restricted
kinematics, one of the solutions produces an observable effect, namely
 when this solution is located on the physical
boundary, {\sl i.e.}, the upper edge of the branch cut in the first Riemann
sheet or alternatively the lower edge of the branch cut in the
second~\cite{Schmid:1967ojm}, see Fig.~\ref{sheets}. In this case, the TS can
produce a narrow peak in the invariant mass distribution, which may even mimic a
resonance. This effect
 was already indicated in Sec.~\ref{sec:TS} and will be further  illustrated  in
 Sec.~\ref{sec:6-ts}.
We therefore discuss now under which circumstances the singularity appears on
the physical boundary. This case is contained in the Coleman--Norton
theorem~\cite{Coleman:1965xm} (for triangle diagrams see
Ref.~\cite{Bronzan:1964zz}). The physical picture becomes most transparent using
the simple triangle singularity equation derived in Ref.~\cite{Bayar:2016ftu}:
\begin{equation}
  q_{\rm on+} = q_{a-}  \, ,
  \label{eq:trianglesing}
\end{equation}
where $q_{\rm on+}$
is the center-of-mass (CM) momentum of particles $M_1$ and $M_2$ when they are
on shell, and $q_{a-}$ is the momentum of particle $M_2$ in the rest frame
of A when $M_2$ and $M_3$ are on shell (being on shell is necessary but not
sufficient to define $q_{a-}$ as will be discussed immediately).
One finds
\begin{eqnarray}
  q_{{\rm on}+} &=& \frac1{2 m_A} \sqrt{\lambda(m_A^2,m_1^2,m_2^2)}\,,
  \nonumber\\
  q_{a-} &=& \gamma \left( \beta \, E_2^* - p_2^* \right) , 
\label{eq:qon}
\end{eqnarray}
where
\begin{equation}
  E_2^* = \frac{m_{B}^2+m_2^2-m_3^2}{2 m_{B}},\quad
  p_2^* = \frac{\sqrt{\lambda(m_{B}^2,m_2^2,m_3^2)}}{2 m_{B}},
\end{equation}
are the energy and the magnitude of the three-momentum of particle $M_2$ in the 
rest frame of particle B, {\sl i.e.} the CM frame of the ($M_2$, $M_3$) 
system, respectively, $\beta=q/E_B$ is the magnitude of the velocity of 
particle B in the rest frame of A, and $\gamma= 1/{\sqrt{1-\beta^2}} =
{E_{B}}/{m_{B}}$ is the Lorentz boost factor. 
Eq.~\eqref{eq:trianglesing} is the condition for the 
amplitude $I(q)$ to have a TS on the physical 
boundary. Note that if particle $M_1$ can go on shell
simultaneously with $M_3$ and $C$, it must be 
unstable. Consequently, its width 
moves the logarithmic divergence into the complex 
plane and the physical amplitude becomes finite.

Let us consider the kinematical region where the momentum of particle $M_2$ 
is positive so that Eq.~\eqref{eq:trianglesing} can be satisfied: $p_2 =q_{a-} 
= \gamma (\beta\,E_2^*-p_2^*) >0$. Then  $p_3=\gamma (\beta\,E_3^*+p_2^*)$ 
(where $E_3^*$ is the
energy of particle $M_3$ in the rest frame of particle B), the momentum of 
particle $M_3$ in the rest frame
of the initial particle, is positive as well.
This means that particles $M_2$ and $M_3$ move in the same direction in that 
frame. The corresponding velocities are given by
\begin{eqnarray}
  \beta_2 = \beta\, \frac{E_2^*-p_2^*/\beta}{E_2^*-\beta\, p_2^*}\,,\quad
  \beta_3 = \beta\, \frac{E_3^*+p_2^*/\beta}{E_3^*+\beta\, p_2^*}\,,
\end{eqnarray}
respectively. It is easy to see that $p_2>0$ leads to 
\begin{equation}
  \beta_3>\beta>\beta_2\,,
\end{equation}
which means that particle $M_3$ moves faster than $M_2$ and in the same 
direction in
the rest frame of the initial particle A.
%
This, together with the requirement that all intermediate particles are on their
mass shells, gives
the condition for having a TS on the physical boundary.
This is in fact the Coleman--Norton theorem~\cite{Coleman:1965xm} applied to the
triangle diagram:
the singularity is on the physical boundary if and only if the diagram can be 
interpreted as a classical process in spacetime. For other discussions about
TSs using the Mandelstam variables, we refer to two recent
works~\cite{Szczepaniak:2015eza,Liu:2015taa} and references therein.

To finish this section, we point out again that the TS mechanism has been around
for more than half a century, but only in recent years has become a viable tool
in hadron physics phenomenology due to the data discussed in this review. In
fact, many of the calculations outlined in which the TS plays a dominant role
can be and often are done without  recourse to an EFT. Still, in a broader view
it can nicely be embedded in the framework outlined here. In any case, whenever
the TS can play a role, it has to be included.

\subsubsection{\nreft}
\label{sec:nreft1}


A key component for any EFT is the power counting in terms of some dimensionless
small quantity, which allows for a systematic expansion and an estimate for the
uncertainty of the calculation caused by the truncation of the series at some
finite order. The natural small quantity in nonrelativistic systems is the
velocity $v$ (measured in units of the speed of light) which is much smaller
than one by assumption.

As mentioned in Sec.~\ref{sec:4-3ploop},  triangle diagrams with all
three intermediate particles being nonrelativistic in fact have two momentum
scales given by $\sqrt{|c_1|}$ and $\sqrt{|c_2|}$. Accordingly, one may define 
$v_1 = \sqrt{|c_1|}/(2\mu_{12})$ and $v_2=\sqrt{|c_2-a|}/(2\mu_{23})$ for the 
velocities of the intermediate mesons. 

From the previous analysis, three-point loop diagrams have two kinds of
singularities: two-body threshold cusps and TSs.
The two-body threshold singularities are encoded in the two velocities defined
above. When the TS, with its location implicitly defined via
Eq.~\eqref{eq:nrtrising}, is not in the considered kinematic region, the loop
function of Eq.~\eqref{eq:Iexp2} can be expanded in a power series as
\begin{eqnarray}
 I(q) &=& \frac{\mathcal{N}}{\sqrt{a} } \Bigg[ \left(\frac{\pi}{2} - 
\frac{2\sqrt{ac_1}}{c_2-c_1} \right) - \left(\frac{\pi}{2} - 
\frac{2\sqrt{ac_2}}{c_2-c_1} \right)  \nonumber\\ 
&& + \order{ \frac{\left(4 a c_1\right)^{3/2}}{(c_2-c_1)^3} } \Bigg] 
\nonumber\\ 
&=& {\cal N} \frac{2}{\sqrt{c_2}+\sqrt{c_1}} + \ldots\,.
  \label{eq:simplifiedI}
\end{eqnarray}
When the masses of all three intermediate particles are similar, 
$m_i\sim m$, the LO term in the above equation may be written as 
\begin{eqnarray}
    I(q) \sim \frac{{\cal N}}{m} \frac{2}{v_1+v_2} \, .
  \label{eq:Ipc}
\end{eqnarray}
Thus the arithmetic mean of the two velocities characterizes the size of the 
triangle loop. It is therefore the relevant parameter to estimate the leading 
loop contribution 
for the transition of a heavy state into a light state and another heavy state.

The power counting in nonrelativistic velocities for a given loop diagram can be
obtained by applying the following rules: The three-momentum of the intermediate
nonrelativistic particles counts as $\order{v}$, the nonrelativistic energy
counts as $\order{v^2}$, and each nonrelativistic propagator is of
$\order{v^{-2}}$. Thus, $I(q)$ scales as $\order{ v^5/(v^2)^3 }=\order{v^{-1}}$.
Comparing with Eq.~\eqref{eq:Ipc}, one sees that the velocity in the power
counting should be understood as the average of $v_1$ and
$v_2$~\cite{Guo:2012tg}.
In addition to the parts discussed the amplitude for a given process can have
factors of the external momentum $q$. To be general we do not count the external
momentum $q$ in powers of $v$, but keep it explicitly.
This defines the power counting as detailed in
Refs.~\cite{Guo:2009wr,Guo:2010zk,Guo:2010ak}. We denote this theory as \nreft.

% -----------------------------------------
\begin{figure}[tb]
  \begin{center}
   \includegraphics[width=0.48\linewidth]{./figures/pc_YZpi}\hfill
   \includegraphics[width=0.48\linewidth]{./figures/pc_YXga}
   \caption{
   Comparison of the power counting rule for the scalar three-point loop 
integral, $1/v$, with the numerical result evaluated using
Eq.~\eqref{eq:Iexp}. The numerical result is normalized to $1/v$ at 
$m_A=4.22$~GeV. The involved masses are given in Eq.~\eqref{eq:masses}, and 
the mass for the final heavy particle takes the value of 3.886~GeV for (a) and 
3.872~GeV for (b).
   \label{fig:pc}}
  \end{center}
\end{figure}
%-----------------------------------------
In order to demonstrate how the power counting rules work, we compare in 
Fig.~\ref{fig:pc} the values of $1/v$ with $v=(v_1+v_2)/2$ and an explicit 
calculation of the loop function as given in Eq.~\eqref{eq:Iexp}. The curves 
are  normalized at $m_A=4.22$~GeV. The values used for the calculation are:
\begin{equation}
  m_1 = 2.420~\text{GeV},~~ m_2= 1.867~\text{GeV}, ~~
 m_3 = 2.009~\text{GeV}. 
\label{eq:masses}
\end{equation}
For the external light particle we take $m_C=0.140$~GeV for 
Fig.~\ref{fig:pc}~(a) and $m_C=0$ for Fig.~\ref{fig:pc}~(b). 
In addition, we take two values for $m_B$, 3.886~GeV and $3.872$~GeV, and the 
results are shown as (a) and (b), respectively. 
Then (a) and (b) correspond to the loop integrals in the amplitudes for the 
$\Y\to\Z\pi$ and $\Y\to\X\gamma$, respectively, which will be discussed later 
in Sec.~\ref{sec:6-long}.
Using 
Eq.~\eqref{eq:trianglesing} or \eqref{eq:nrtrising}, we find that for 
$m_B=3.886$~GeV, there is a TS at $m_A=4.288$~GeV, which is 
the reason for the sharp peak in the dashed line in Fig.~\ref{fig:pc}~(a).
Note that in
the plots the widths of the intermediate mesons were neglected. For 
$m_B=3.872$~GeV, which is smaller than $m_2+m_3$, and $m_C=0$, the TS
moves to the complex plane at $m_A=(4.301-i\,0.018)$~GeV with a clearly
visible effect on the line shape, see the dashed line in 
Fig.~\ref{fig:pc}~(b). 
One sees from the figure that the simple power counting rule of
Eq.~(\ref{eq:Ipc}) agrees remarkably well with the explicit
calculation except for energies very close to the TS.

% -----------------------------------------
\begin{figure}[tb]
  \begin{center}
   \includegraphics[width=0.6\linewidth]{./figures/selfenergy}\\
   \caption{A one-loop two-point self-energy diagram.
   \label{fig:selfenergy}}
  \end{center}
\end{figure}
%-----------------------------------------

In addition to counting the loop integral as discussed above, one also needs to
take into account the vertices in order to obtain a proper estimate for a given
loop amplitude.
To illustrate the method let us start from the simplest two-point self-energy
diagram shown in Fig.~\ref{fig:selfenergy}. We assume that the mass of the state
is close to the threshold of the internal particles that can therefore be
treated nonrelativistically. If the coupling is in an $S$-wave, then the loop
scales as $\order{v^5/(v^2)^2}=\order{v}$.\footnote{Here we only focus on the
velocity scaling and neglect the geometric factor of $1/(4\pi)$.} If the
coupling is in a $P$-wave, each vertex contributes an additional factor of $v$
and the loop scales as $\order{v^3}$. Of course, the real part of the loop
integral is divergent, and the resulting correction to the mass is
scale-dependent.
However, since the scale dependence can be formally absorbed into the bare mass
of the state this discussion is not of relevance here. Thus, we find that the
effect of the two-hadron continuum on the self-energy of heavy quarkonia is
parametrically suppressed, if the state is close to the threshold which implies
a small value of $v$, and that this suppression increases for increasing orbital
angular momentum of the two-hadron state.

Next we consider the one loop diagram for the decay process $\text{A}\to 
\text{B\,C}$, with A and B heavy and C  light, as depicted in 
Fig.~\ref{fig:triangle}. To be concrete, 
we assume that C couples to the intermediate states $M_1$ and $M_3$ in a
$P$-wave (such as the pion couples to the ground state heavy mesons).
This coupling structure leads to  a 
factor of  $q$ (in the rest frame of A). The generalization to 
other situations is easy. The power counting rules for a few typical cases are 
then
as follows:
\begin{itemize}
  \item[(1)] Both A and B couple to the intermediate states in an $S$-wave.
  As a result the final state particles $B$ and $C$ must be in a $P$-wave. 
Therefore the expression of  Eq.~\eqref{eq:Ipc} needs to be multiplied by $q$. 
Still, the $1/v$ enhancement factor quantifies the relative importance of the 
triangle diagram for the transition: the closer both A and B to the 
corresponding thresholds, the more important the intermediate states. On top of 
this may come an additional enhancement
driven, e.g., by large couplings characteristic for molecular states as derived 
in Sec.~\ref{sec:3}.

  \item[(2)] Either A or B couples to the  intermediate states in an 
$S$-wave with the other one in a $P$-wave. In this case, because there is 
only one 
possible linearly independent external momentum for two-body decays, the 
internal momentum at the $P$-wave vertex must be turned into an external 
momentum. The amplitude scales as $\order{q^2/(m^2v)}$. Since the decay 
should be in an $S$-wave in this case, we have introduced a factor $1/m^2$ to 
balance the dimension of the $q^2$ factor~\cite{Guo:2010ak} as in this case
the loop contribution needs to be compared to a constant tree-level 
contribution.
 
  \item[(3)] Both A and B couple to the intermediate states in a $P$-wave. Each 
$P$-wave vertex contributes a factor of the internal momentum. In the power 
counting of \nreft, the external momentum is kept explicitly. As a result, 
there are two possibilities for the scaling of the $P$-wave vertices: Each 
$P$-wave vertex scales either as $m\,v$ or as the external momentum $q$. More 
insights can be obtained if we take a closer look at the relevant tensor loop 
integral:
\begin{equation}
  I^{ij}(\bm q\,) =  i\int\!\frac{d^4l}{(2\pi)^4} l^il^j\times 
\left[\text{integrand of }I(q)\right].
\end{equation}
In the rest frame of the initial particle, it can be decomposed into an 
$S$-wave part and a $D$-wave part as
\begin{equation}
  I^{ij}(\bm q\,) = P_S^{ij} I_S(q) + P_D^{ij} I_D(q),
\end{equation}
where 
\begin{equation}
  P_S^{ij}=\frac{\delta^{ij}}{\sqrt{3} }\,,\quad  
  P_D^{ij}= \frac1{\sqrt{6}} \left( 3\frac{q^iq^j}{q^2} -\delta^{ij} \right), 
\end{equation}
are the $S$- and $D$-wave projectors, respectively, which satisfy 
$P_S^{ij}P_S^{ij} = 1$, $P_D^{ij}P_D^{ij} = 1$, and $P_S^{ij}P_D^{ij} = 0$. 
Then in the $S$-wave part $I_S(q)$, the internal momentum scales as 
$\order{v}$, and $I_S(q)\sim\order{v}$. In the $D$-wave part $I_D(q)$, the 
internal momentum turns external, and one gets $I_D(q)\sim\order{q^2/(m^2 
v)}$, which would have the same scaling as $I_S(q)$ if 
$q/m\sim v$.\footnote{Noticing that $P_S^{ij} l^il^j=l^2/\sqrt{3}$ and 
$P_D^{ij} 
l^il^j=2l^2 P_2(\cos\theta)/\sqrt{6}$ with $P_2(\cos\theta)$ the second 
Legendre polynomial, it can be shown that $I_S(q)$ is UV divergent while 
$I_D(q)$ is UV convergent~\cite{Albaladejo:2015dsa,Shen:2016tzq}. The power 
counting of the $D$-wave part was not discussed in Ref.~\cite{Guo:2010ak}.} For 
the 
decay amplitude, the factor of $q$ from the vertex coupling C to intermediate 
states needs to be taken into account additionally.
 
\end{itemize}


This nonrelativistic power counting scheme was proposed in
Ref.~\cite{Guo:2009wr} to study the coupled-channel effects of charm meson
loops in charmonium transitions, and studied in detail later~\cite{Guo:2010ak}.
Applications to transitions between two heavy quarkonium states can be found in
Refs.~\cite{Guo:2010zk,Guo:2010ca,Guo:2011dv,Mehen:2011tp,Guo:2012tg,
Guo:2014qra}, and to transitions involving one or two $XYZ$ states in
Refs.~\cite{Cleven:2011gp,Cleven:2013sq,Guo:2013nza,
Esposito:2014hsa,Mehen:2015efa,Huo:2015uka,Wu:2016dws,Abreu:2016xlr,
Chen:2016mjn}. In particular in Ref.~\cite{Cleven:2013sq} the implications of
items (1) and (2) are demonstrated.
It is shown that, while the transitions of the $Z_b$ states to $\Upsilon(nS)\pi$
potentially suffer from large higher order corrections, the transitions to 
$h_b(mP)\pi$ and $\chi_{bJ}(mP)\pi$ should be dominated by the triangle
topology. The near-threshold cross section for $e^+e^-\to D\bar D$ was studied
in~\cite{Chen:2012qq} using NREFT as well.


It is clear that the power counting can only be applied to processes where the
intermediate hadrons are nonrelativistic and especially close to their mass
shells. Otherwise the loop diagrams receive contributions from large momenta and
cannot be treated in a simple EFT including only the hadronic degrees of freedom
of A, B, C, $M_1$, $M_2$ and $M_3$.


\subsubsection{\nreftii~and XEFT}
\label{sec:4-XEFT}

Because the $\X$ is arguably the most important and interesting candidate for a
hadronic molecule, here we will discuss in some detail one NREFT
designed specifically for studying the properties of the $\X$. It is called XEFT
and was proposed in Ref.~\cite{Fleming:2007rp} following the
Kaplan--Savage--Wise approach to describe the nucleon-nucleon
system~\cite{Kaplan:1998tg,Kaplan:1998we}. It can be regarded as a special
realization of \nreftii. Similar effective theories can be constructed for other
possible hadronic molecules which are located very close to thresholds. For
instance, in the framework of a similar theory, the $Z_b(10610)$ and
$Z_b(10650)$ were studied in Refs.~\cite{Mehen:2011yh,Mehen:2013mva} and the
$Z_c(3900)$ in Ref.~\cite{Wilbring:2013cha}.

The XEFT assumes the $\X$ to be a hadronic molecule of $D^0\bar D^{*0}+c.c.$.
The tiny binding energy~\cite{Olive:2016xmw}
\begin{equation}
  B_{X}=M_{D^0} +M_{D^{*0}}-M_X=(0.00\pm0.18)~\text{MeV},
  \label{eq:Xbe}
\end{equation}  
implies that the long-distance part of the $\X$ wave function is universal
and is insensitive to the binding mechanism which takes place at a much shorter
distance.
%
The long-distance degrees of freedom are $D^0$, $D^{*0}$, $\bar
D^0$, $\bar D^{*0}$ and $\pi^0$. All of them are treated nonrelativistically.
%
For processes dominated by the long-distance scales such as the decays $\X\to
D^0\bar D^0\pi^0$ and $\X\to D^0\bar D^0\gamma$ which can occur via the decay of
the vector charm meson directly, the XEFT at LO can reproduce the results
from the effective range theory which makes use of the universal two-body wave
function of the $\X$ at asymptotically long 
distances~\cite{Voloshin:2003nt,Voloshin:2005rt}
\begin{equation}
  \psi_X(r) \propto \frac{e^{-\gamma_0 r}}{r},
\end{equation}
where the $\X$ is assumed to be below the $D^0\bar D^{*0}$ threshold, and
$\gamma_0=\sqrt{2\mu_0 B_{X}}\leq20$~MeV with $\mu_0$ the reduced mass of 
$D^0$
and $\bar D^{*0}$.
Yet, it has the merit of being improvable order by order by including local
operators and pion exchanges although unknown short-distance coefficients will 
be involved. For processes involving shorter-distance scales such as the decays 
of the $\X$ into a charmonium and light particles, the XEFT can still be used by
parameterizing the short-distance physics in terms of local operators employing
factorization theorems and the operator product 
expansion~\cite{Braaten:2005jj,Braaten:2006sy}. The XEFT can also be used even
if the $\X$ is a virtual state with a nonnormalizable wave
function~\cite{Hanhart:2007yq} or a resonance above threshold.

The power counting and the NLO corrections to the decay $\X\to D^0\bar D^{0}
\pi^0$ were studied in Ref.~\cite{Fleming:2007rp}. The XEFT was also used to
study the decays of the $\X$ to the $\chi_{cJ}$ with one and two
pions~\cite{Fleming:2008yn,Fleming:2011xa}, the radiative transitions
$\X\to\psi(2S)\gamma$, $\psi(4040)\to\X\gamma$~\cite{Mehen:2011ds} and
$\psi(4160)\to \X\gamma$~\cite{Margaryan:2013tta}, the scattering of an
ultrasoft pion~\cite{Braaten:2010mg} or $D$ and $D^*$~\cite{Canham:2009zq} off
the $\X$, and the quark mass dependence and finite volume corrections of the
$\X$ binding energy~\cite{Jansen:2013cba,Jansen:2015lha}.
The relation between the XEFT and the formalism of \nreft~was clarified in
Ref.~\cite{Mehen:2015efa}.
As an extension of the XEFT in Ref.~\cite{Alhakami:2015uea} a modified power
counting was suggested to take into account also an expansion in the ratio
between the pion mass and the charm meson masses. The need for such an
expansion is removed, however, as soon as Galilean invariance is imposed on the
interactions~\cite{Braaten:2015tga}.


In the following, we use the decay $\X\to D^0\bar D^0\pi^0$ as an example to
illustrate the power counting of the XEFT. The binding momentum
$\gamma_0\leq20$~MeV sets the long-distance momentum scale in this theory.
The typical momenta for the $D^0$ and $D^{*0}$ are of the order $p_D\sim
p_{D^*}\sim\gamma_0$. The pion kinetic energy is less than 7~MeV, and thus the
momentum for either an internal or external pion is also counted as
$p_\pi\sim\gamma_0$. Furthermore, the pion exchange introduces another small
scale $\mu=\sqrt{\Delta^2-M_{\pi^0}^2}\simeq44$~MeV,  with
$\Delta=M_{D^{*0}}-M_{D^0}$. Denoting all the small momentum scales by $Q$, we
have
\begin{equation}
  \{p_D, p_{D^*}, p_\pi, \mu, \gamma_0\} =\order{Q}.
  \label{eq:XEFTpc}
\end{equation}
Thus, the measure for one-loop integral is of $\order{Q^5}$, and each 
nonrelativistic propagator is of $\order{Q^{-2}}$.
All Feynman diagrams can then be assigned a power of $Q$. 


The XEFT keeps as the degrees of freedom only those modes with a very
low-momentum $\sim\gamma_0$.
The binding momentum for the $D^+D^{*-}+c.c.$ channel at the $\X$ mass is
$\gamma_c\simeq126$~MeV. It is treated as a hard scale, and the charged charm
mesons are integrated out from the XEFT.


Denoting the field annihilating the 
$D^0, \bar D^0$, $D^{*0}$ and $\bar D^{*0}$ by $D,\bar D$, $\bd$ and $\bdbar$, 
respectively, and taking the phase 
convention that $\X$ is $\left( D^0\bar D^{*0} +\bar D D^{*0} 
\right)/\sqrt{2}$, the relevant Lagrangian for the calculation up to the NLO is 
written as~\cite{Fleming:2007rp}
\begin{widetext}
\begin{eqnarray}
 {\cal L}_{\rm XEFT} 
  &=& 
 \sum_{\bm{\phi}=\bd,\bdbar } \bm{\phi}^{\dagger} \bigg(i\partial_0 + 
\frac{\bm{\nabla}^2}{2 M_{D^{*0}}
		    }\bigg)\bm{\phi} 
 +  \sum_{{\phi}=D,\bar D } D^\dagger \bigg(i\partial_0 + 
\frac{\bm{\nabla}^2}{2 M_{D^0} } \bigg) D 
 + \pi^\dagger \bigg(i\partial_0 + \frac{\bm{\nabla}^2}{2 M_{\pi^0}}
   + \delta\bigg) \pi
%%%
\nonumber \\
&&+ \left[  
\frac{g}{2 F_\pi} \frac{1}{\sqrt{ 2M_{\pi^0} } }
 \left( D \bd^\dagger \cdot \bm{\nabla}\pi  
   + \bar D^\dagger \bdbar \cdot \bm{\nabla}\pi^\dagger \right) + {\rm
 H.c.} \right]
\nonumber \\
 && 
- \,  
\frac{C_0}{2} \, \left(\bdbar D + \bd \bar D \right)^\dagger 
\cdot \left(\bdbar D + \bd \bar D \right) 
+  \left[   \frac{C_2 }{16} \, 
\left(\bdbar D + \bd \bar D \right)^\dagger 
\cdot \left(\bdbar (\overleftrightarrow \nabla)^2 D 
        + \bd (\overleftrightarrow \nabla)^2 \bar D \right) + {\rm H.c.} \right]
\nonumber \\
%%%
&&+ \left[  \frac{B_1 }{\sqrt{2}}\frac{1}{\sqrt{2 M_{\pi^0}}} \left(\bdbar D + 
\bd \bar D \right)^\dagger \cdot D \bar{D} \bm{\nabla} \pi + {\rm 
H.c.}\right] \nonumber\\
%% new terms %%
&& + \frac{C_\pi}{2 M_{\pi^0}} \left( D^\dagger \pi^\dagger D \pi + 
\bar D^\dagger 
\pi^\dagger \bar D \pi \right) + C_{0D} D\,^\dagger\bar{D}^\dagger D\bar D \,,
\label{eq:XEFTlag}
\end{eqnarray}
\end{widetext}
where $\delta=\Delta-M_{\pi^0}\simeq7$~MeV and
$F_\pi=92.2$~MeV is the pion decay constant. The first line 
contains the kinetic terms for the pseudoscalar and vector charm mesons as 
well as for the nonrelativistic pion, the second is for the axial coupling of 
the pion to charm mesons with $g\simeq0.6$ determined from the $D^{*}$ width, 
the third line contains the LO and NLO contact interaction terms, and the 
fourth line contains the terms for a  short-distance emission of a pion. The contact terms in 
the last line were not considered in Ref.~\cite{Fleming:2007rp}, but,
as is argued below, also contribute to $\X\to D^0\bar D^0\pi^0$ at NLO. In 
particular, the $C_{0D}$ term may have a significant impact on the line shapes
as will become clear in the  discussion below.

%-----------------------------------------
\begin{figure}[tb]
  \begin{center}
   \includegraphics[width=\linewidth]{./figures/XDDpi}\\
   \caption{LO (a) and NLO (b,\ldots, g) diagrams for the calculation of the 
$\X\to D^0\bar D^0\pi^0$ decay width. The circled cross denotes an insertion of 
the $\X$, the thin and thick solid lines represent the pseudoscalar and vector 
charm mesons, respectively, and the dashed lines denote the pions.
   \label{fig:XDDpi}}
  \end{center}
\end{figure}
%-----------------------------------------

The Feynman diagrams relevant for the calculation of the $\X\to D^0\bar
D^0\pi^0$ decay width up to NLO are shown in Fig.~\ref{fig:XDDpi}. Diagram (a)
contributes at LO, (b,c) and (e,f) are the NLO diagrams calculated in
Ref.~\cite{Fleming:2007rp}, and (d,g) are two new diagrams from the new terms in
the last line of the Lagrangian in Eq.~\eqref{eq:XEFTlag}. Here we only discuss
the power counting for each diagram and the contributions missing in the
original work~\cite{Fleming:2007rp}, and refer  to Ref.~\cite{Fleming:2007rp}
for details of the calculation. One essential point of the XEFT is that the
pion-exchange is treated perturbatively based on the observation that the
two-pion exchange contribution is suppressed relative to the one-pion exchange
by
\begin{equation}
  \frac{g^2\mu_0\mu }{8\pi F_\pi^2} \simeq \frac1{20} \cdots \frac1{10} .
\end{equation}
Then the $\X$ is generated through a resummation of the $D\bar D^*$ contact 
terms (the 
charge conjugated $\bar D D^*$ channel is always implied). The 
pole of the $\X$ is at $E=-B_{X}$, and thus at LO 
\begin{equation}
  1+C_0\Sigma_0(-B_{X})=0 \ ,
  \label{eq:Xpole}
\end{equation}
where 
\begin{eqnarray}
  \Sigma_0(E) &=& - \left( \frac{\Lambda_{\rm PDS}}{2\pi} \right)^{4-D}\! 
\int\! \frac{d^{D-1} l}{(2\pi)^{D-1} } \frac1{E-l^2/(2\mu_0)+i\epsilon} 
\nonumber\\
  &=& \frac{\mu_0}{2\pi} \left(\Lambda_{\rm PDS} -\sqrt{-2\mu_0 E 
-i\epsilon} \right)
\label{eq:Sigma}
\end{eqnarray}
is the two-point one-loop integral containing nonrelativistic $D^0$ and $\bar 
D^{*0}$ propagators in the power divergence subtraction (PDS) 
scheme~\cite{Kaplan:1998tg,Kaplan:1998we}, where $E$ is the energy defined 
relative to the threshold and $\Lambda_{\rm PDS}$ is the PDS scale. 
For Eq.~\eqref{eq:Xpole} to be renormalization 
group invariant, $C_0$ needs to absorb the scale dependence of the loop 
integral:
\begin{equation}
  C_0(\Lambda_{\rm PDS}) = \frac{2\pi}{\mu_0\left(\gamma_0 - \Lambda_{\rm PDS} 
\right) }.
\label{eq:C0PDS}
\end{equation}
Keeping only momentum modes of order $Q$, the power counting for the loop 
integral is  $\Sigma_0(E)=\order{Q^5/(Q^2)^2}=\order{Q}$. One sees that 
the scale-independent part of $C_0$, $\bar C_0 = [\Lambda_{\rm PDS} + 
1/C_0( \Lambda_{\rm PDS}) ]^{-1} = 2\pi/(\mu_0\gamma_0)$, indeed scales as 
$Q^{-1}$.

Now we consider the power counting of the decay amplitudes from the 
diagrams 
in Fig.~\ref{fig:XDDpi}. The decay rate can be obtained from these amplitudes 
taking into account properly the wave function renormalization 
$Z$~\cite{Fleming:2007rp} which accounts for the insertion of the $\X$ 
interpolating field shown as circled crosses in Fig.~\ref{fig:XDDpi}. Notice 
that for the calculation of the decay rate up to the NLO, one needs $Z$ up to 
NLO (LO) for the LO (NLO) amplitude. The amplitude from diagram (a) scales as 
$\order{Q/Q^2}=\order{Q^{-1}}$ since there is one nonrelativistic propagator 
and one $P$-wave vertex which gives a factor of $p_\pi\sim Q$. Both one-loop 
diagrams (b) and (c) have four nonrelativistic propagators and three $P$-wave 
vertices, and thus scale as $\order{Q^0}$, one order higher 
than the LO diagram (a). The coefficients $C_2$ and $B_1$ scale as 
$Q^{-2}$~\cite{Fleming:2007rp}. Noticing that there are two derivatives in the 
$C_2$ term and one derivative in the $B_1$ term in the Lagrangian, the 
amplitudes from diagrams (e) and (f) should be counted as $\order{Q^0}$ as well.

Let us discuss diagrams (d) and (g) which were missing in the original 
calculation in Ref.~\cite{Fleming:2007rp}. The $C_\pi$ contact term can be 
matched to the chiral Lagrangian for the interaction between heavy and light 
mesons~\cite{Burdman:1992gh,Wise:1992hn,Yan:1992gz,Guo:2008gp}. At LO of the 
chiral expansion the interaction between pions and pseudoscalar heavy 
mesons receives contributions from the Born term from the exchange of $D^*$, 
which constitutes a subdiagram to (b) and (c), and the Weinberg--Tomozawa term. 
It turns out that the amplitude for $D^0\pi^0\to D^0\pi^0$ vanishes at LO. At 
NLO of the chiral expansion, there are several operators, see 
Refs.~\cite{Guo:2008gp,Guo:2009ct}.  In particular, it is easy to see that the 
$h_0$ and $h_1$ terms therein are proportional to the light quark mass or 
equivalently to $M_\pi^2$. The Feynman rule for the $D^0\pi^0\to D^0\pi^0$ vertex 
from these two terms (using relativistic normalization for all the fields) is
\begin{equation}
  i\,\Amp_{h_0,h_1} = i \frac{2}{3} \left(6h_0+h_1\right) 
\frac{M_\pi^2}{F_\pi^2} .
\end{equation}
The value of $h_1$ is fixed to be 0.42 from the mass splitting between the 
$D_s$ and $D$ mesons, and the $1/N_c$ suppressed parameter $h_0\simeq0.01$ from 
fitting to the lattice data for the pion mass dependence of charm meson 
masses~\cite{Liu:2012zya}. One sees $\Amp_{h_0,h_1}\simeq 0.65$. Hence, by 
matching to the chiral Lagrangian, $C_\pi$ should scale as $Q^0$, which leads 
to the scaling of $\order{Q^0}$ for diagram (d). 

Diagram (g) involves a short-distance contact interaction between $D^0$ and
$\bar D^0$. If the vertex $C_{0D}$ scales as $Q^0$, then diagram (g)
$=\order{Q^0}$. However, the situation could be more complicated. From the HQSS
analysis of the $\X$ in Sec.~\ref{sec:3_HQSS}, the $\X$ as a $D\bar D^*$
hadronic molecule should have three spin partners in the strict heavy quark
limit. One of them has quantum numbers $J^{PC}=0^{++}$ and couples to $D\bar D$
and $D^*\bar D^*$. Therefore, there is the possibility that the $D\bar D$
interaction needs to be resummed to generate a near-threshold pole. In this
case, $C_{0D}$ needs to be promoted to be $\order{Q^{-1}}$, analogous to $C_0$.
Then diagram (g) appears at $\order{Q^{-1} }$ making it a LO contribution.
Clearly this can cause a large correction to the $\X\to D^0\bar D^0\pi^0$  decay
rate. This effect can be seen clearly in Fig.~\ref{fig:XDDpi_FSI}, which is the
result obtained in Ref.~\cite{Guo:2014hqa} using \nreft~in combination with the
framework to be discussed in Sec.~\ref{sec:4-interactions}. The unknown
parameter $C_{0a}$ in the figure parameterizes the isoscalar part of $D^\dag\bar
D^\dag D\bar D$ contact interaction, see Eq.~\eqref{C0++} below, playing a role
similar to $C_{0D}$ introduced in Eq.~\eqref{eq:XEFTlag}.

%-----------------------------------------
\begin{figure}[tb]
  \begin{center}
   \includegraphics[width=\linewidth]{./figures/width_ddpi_05GeV}\\
   \caption{Decay width of the $\X\to D^0\bar D^0\pi^0$ calculated in 
Ref.~\cite{Guo:2014hqa} taking into account the $D\bar D$ final state 
interaction in the framework of Lippmann--Schwinger equation regularized by a 
Gaussian form factor. Here the cutoff in the Gaussian regulator is taken to be 
$\Lambda=0.5$~GeV, and $C_{0a}$ is the unknown isoscalar part of the $D\bar D$ 
contact term. The gray and blue bands correspond to the uncertainty bands 
without and with the $D\bar D$ final state interaction, respectively. The 
vertical line denotes the $D^0\bar D^0$ threshold. Adapted from 
Ref.~\cite{Guo:2014hqa}. 
   \label{fig:XDDpi_FSI}}
  \end{center}
\end{figure}
%-----------------------------------------

Since $\X\to D^0\bar D^0\pi^0$ is an important process sensitive to 
the long-distance structure of the $\X$, it would be interesting to revisit it 
considering the missing diagrams in XEFT. In particular, it was found that the 
nonanalytic corrections from the pion-exchange diagrams (b) and (c) of Fig,~\ref{fig:XDDpi} only 
contribute to $\sim1\%$ of the decay rate~\cite{Fleming:2007rp}. Whether this 
remains true after considering diagram (d) remains to be seen.

It should be stressed that the role of nonperturbative pions on the $\X$ 
properties is studied in various
papers~\cite{Baru:2011rs,Baru:2013rta,Baru:2016iwj} which in many cases 
confirm the results of XEFT. However, also in these studies diagrams of the 
types shown in diagrams (d) and (g) of Fig.~\ref{fig:XDDpi} were not included.

\subsubsection{From \nreft~to XEFT}
\label{sec:4-nreft2xeft}

From the discussions above, we see that all momentum scales much larger than 
$\gamma_0\leq20$~MeV have been integrated out from the XEFT. This is different 
from \nreft, where all nonrelativistic modes are kept as effective degrees 
of freedom including those with a momentum of the order of a few hundreds of 
MeV. \nreft~when applied to the $\X$ can be regarded as the high-energy theory 
for the XEFT. The short-distance operators in XEFT at the scale of a few 
hundreds of MeV can be matched to \nreft. This is
discussed in some detail in Ref.~\cite{Mehen:2015efa} in the context of 
calculations of the reactions $\X\to\chi_{cJ}\pi^0$.

To show the relation between \nreft~and XEFT explicitly let us consider the case 
$c_2\gg c_1$.
The quantities $c_2$ and $c_1$  introduced in Eq.~\eqref{eq:loopinter} define
 the locations of the two-body cuts of the triangle diagram. In the low-momentum region $l\sim \sqrt{c_1}$, 
 the second factor in the integrand of Eq.~\eqref{eq:loopinter} can 
be expanded in powers of $l^2/c_2$ and one gets
\begin{eqnarray}
  I(q) &=& \frac{4\mu_{12}\mu_{23}}{N_m c_2} \int^\Lambda \frac{d^3
l}{(2\pi)^3}\frac1{{l}^{\,2} + c_1 -i\epsilon} \left[1+ \order{\frac{c_1}{c_2}} 
\right] \nonumber\\
&\simeq& \frac{\mu_{12}}{2\pi N_m  \left[b_{23} + q^2/(2 m_3) \right] }  \left(\Lambda_{\rm PDS} -\sqrt{c_1 {-}i\epsilon} \right)\!.
~~~~
\end{eqnarray}
The resulting momentum integral in the first line is divergent
 and  needs to be regularized. The 
natural UV cutoff of the new effective theory
is set by $\Lambda<\sqrt{c_2}$. We denote such a theory as 
\nreftii. It reduces to the XEFT when applied to the $\X$. In order to compare 
with the XEFT, in the second line of the above equation we evaluate the 
integral in the PDS scheme which is equivalent to the sharp cutoff 
regularization by letting $\Lambda_{\rm PDS}= 2 \Lambda/\pi$ and dropping the 
terms of $\order{1/\Lambda}$. For a detailed comparison of 
dimensional versus cutoff regularization we refer to
Ref.~\cite{Phillips:1998uy}.

For $m_{1}=M_{D^{*0}}$, $m_2=M_{D^0}$, $E_C=E_\pi$, and $m_A=M_X$, the second 
line of the above equation reduces to 
\begin{equation}
  -\frac1{N_m \left(E_\pi + \Delta_H \right)} \frac1{ 
C_{0}(\Lambda_{\rm PDS} ) },
\end{equation}
where $\Delta_H=M_{D^0}+m_3-M_X$, and the term $q^2/(2m_3)$ has been neglected. 
Terms of the above form appear in the XEFT amplitudes for transitions between 
the $\X$ and a charmonium with the emission of a light 
particle~\cite{Fleming:2008yn,Fleming:2011xa,Mehen:2011ds,Margaryan:2013tta}.

The different power countings of XEFT and \nreft~has various implications 
that we now illustrate by two examples:

   Since \nreft~keeps all nonrelativistic modes explicitly, the charged
  $D\bar D^*$ channel which has a momentum of $\gamma_c\simeq126$~MeV needs to
  be kept as soft degrees of freedom. On the contrary, the XEFT only keeps the
  ultrasoft neutral charm mesons dynamically and the charged ones are integrated
  out.
  It was pointed out in Ref.~\cite{Mehen:2015efa} that it is crucial to take
  into account the charged charm mesons for the calculation of the
  $\X\to\chi_{cJ}\pi^0$ decay rate in \nreft~because their contribution
cancels to a large extent the one
  from the neutral charm mesons as usual in isospin violating transitions
({\sl c.f.}
the discussion in Sec.~\ref{sec:isospinviol}).\footnote{The role of the
charged
 charm mesons for certain decays of the $\X$ was already stressed in
 \cite{Gamermann:2009fv}.}
  The situation for decays into an isoscalar pion pair, $\X\to\chi_{cJ}\pi\pi$,
  is different. We expect that the charged and neutral channels are still of
  similar order, but add up constructively.
  
  Furthermore, in the XEFT calculation for  $\X\to\chi_{cJ}\pi^0$, there appears a new,
  reaction specific
  short-distance operator, labeled by $C_{\chi,0}$ in
  Fig.~\ref{fig:Xchic1pi}~(c). 
  To estimate its size it is matched onto two contributions in heavy meson
  chiral perturbation theory in Refs.~\cite{Fleming:2008yn,Fleming:2011xa}. Those
  are given by the exchange of a charm meson, which is proportional to the
  $\chi_{cJ}H\bar H$ coupling constant $g_1$, and a contact term accompanied
  by a low energy constant $c_1$, shown as diagram (a)
  and (b) in Fig.~\ref{fig:Xchic1pi}, respectively. The final result in XEFT
  then depends on the unknown ratio $g_1/c_1$. In \nreft, however, the two
  contributions appear at different orders, since the amplitude from diagram (b) is
  suppressed by $v^2$ compared with that from diagram (a).
  
%-----------------------------------------
  \begin{figure}[tb]
    \begin{center}
     \includegraphics[width=\linewidth]{./figures/Xchic1pi}\\
     \caption{Diagrams for calculating the decay rate for the process
     $\X\to\chi_{c1}\pi^0$.
     The circled cross denotes an insertion of the $\X$, the thin and thick 
     solid lines represent the pseudoscalar and vector charm mesons, 
     respectively, the dashed lines present the pions, and the double lines
     correspond to the $\chi_{c1}$.
     \label{fig:Xchic1pi}}
    \end{center}
  \end{figure}
%-----------------------------------------



\subsection{Formation of hadronic molecules}
\label{sec:4-interactions}

While so far the focus was on transitions of molecular candidates, we now 
turn to their formation through two-hadron scattering. 
For illustration we focus in this chapter on the scattering of open-flavor heavy
mesons off their antiparticles in a framework of  
NREFT similar to the EFT for nucleon-nucleon 
interactions~\cite{Epelbaum:2008ga}. 
The example of the formation of $\Lambda(1405)$ from
similar dynamics is discussed in Sec.~\ref{sec:1405th}.
In this section we mainly discuss the method used in 
Refs.~\cite{Nieves:2011vw,Nieves:2012tt,Valderrama:2012jv,
HidalgoDuque:2012pq,Guo:2013sya,Guo:2013xga}. It is based on the 
Lippmann--Schwinger equation (LSE) regularized using a Gaussian vertex form 
factor. The coupled-channel LSE 
reads 
\begin{eqnarray}
      T_{ij}(E;\bm k',\bm k) &=& V_{ij}(\bm k', \bm k) \\
    &+&  \sum_n \int\! \frac{d^3l}{(2\pi)^3} 
\frac{ V_{in}(\bm{k}',\bm{l})\, T_{nj}(E;\bm l, \bm k)  }{  
E-l^2/(2\mu_{n}) - \Delta_{n1} + i\epsilon  }, \nonumber
 \label{eq:lse}
\end{eqnarray}
where $\mu_{n}$ is the reduced mass in the $n$-th channel, $E$ is the 
energy defined relative to threshold of the first channel, and $\Delta_{n1}$ is 
the difference between the $n^{\rm th}$ threshold and the first one. When the 
potential takes a separable form $V_{ij}(\bm k', 
\bm k)=\xi_{i}{(\bm k')}V_{ij}\varphi_{j}{(\bm k)}$, where the $V_{ij}$ are 
constants, the equation can be  
simplified greatly. In addition, for very near-threshold states one should 
expect a momentum 
expansion for the potential to converge fast and a dominance of $S$-waves. 
Both the separability as well as the absence of higher partial waves will be 
spoiled as soon as the one-pion exchange is included on the potential level;
this case will be discussed briefly later in this section.

With a UV regulator such as of the Gaussian form, see, 
e.g., Ref.~\cite{Epelbaum:2008ga}, 
\begin{equation}
  V_{ij}(\bm k',\bm k) =  e^{-\bm k^{\prime2}/\Lambda^2} V_{ij} e^{-\bm 
k^{2}/\Lambda^2} \ ,
\label{eq:potdef}
\end{equation}
the LSE can be solved straightforwardly.  
If the $T$-matrix has a near-threshold bound state pole, the effective coupling 
of this composite state to the constituents can be obtained by calculating the 
residue of the $T$-matrix element at the pole. For simplicity, we consider a 
single-channel problem with the LO contact term: $V(\bm k',\bm k) 
= C_0 e^{-\bm k^{\prime2}/\Lambda^2} e^{-\bm 
k^{2}/\Lambda^2}$. The nonrelativistic $T$-matrix element for the scattering of 
the two hadrons is then given by 
\begin{equation}
T_\text{NR}^{}(E) = \left[ C_0^{-1} + \Sigma_\text{NR}^{}(E) \right]^{-1}, 
\label{eq:T1c}
\end{equation}
where 
\begin{equation}
\Sigma_\text{NR}^{}(E)= \frac{\mu}{2\pi} \left[\frac{\Lambda}{\sqrt{2\pi}} 
- \sqrt{-2\mu E-i\epsilon} \right] +\order{ {\Lambda}^{-1} }
\label{eq:SigmaGaussian}
\end{equation}
is the nonrelativistic two-point scalar loop function 
defined in Eq.~\eqref{eq:Sigma} but evaluated with a Gaussian regulator. After 
renormalization by absorbing the cutoff dependence into $C_0$, we obtain
\begin{equation}
  T_\text{NR}^{}(E)=\frac{2\pi/\mu}{\gamma -\sqrt{-2\mu E-i\epsilon} } +\order{ 
{\Lambda}^{-1} } .
  \label{eq:TNR}
\end{equation}
The binding momentum $\gamma$ was defined in Eq.~(\ref{eq:gamdef}).
The 
effective coupling is obtained by taking the residue at the pole $E=-E_B$:
\begin{eqnarray}
  g_\text{NR}^2 &=& \lim_{E\to -E_B} (E+E_B) T_\text{NR}^{}(E) = \left[  
\Sigma_\text{NR}'(-E_B) \right]^{-1} \nonumber\\
  &=& \frac{2\pi\gamma}{\mu^2} .
  \label{eq:gNR}
\end{eqnarray}
It does not depend on $C_0$, and is scale independent up to terms suppressed by 
$1/\Lambda$.
Multiplying $g_\text{NR}^2$ by the factor $(8 m_1 m_2 M)$ to get the 
relativistic  
normalization, we recover the expression for $g_\text{eff}$ derived in 
Eq.~\eqref{eq:residue} for  
 $\lambda^2=0$. Thus we find that a potential of the
kind given in Eq.~(\ref{eq:potdef}) generates hadronic molecules.
Deviations of this result behavior can be induced, e.g., by momentum
dependent interactions (or terms of order $\gamma/\Lambda$). This
observation formed the basis for the generalization of the Weinberg 
compositeness criterion presented in 
Refs.~\cite{Aceti:2012dd,Hyodo:2011qc,Hyodo:2013nka,Sekihara:2014kya}.

To proceed we first need to say a few words about the scattering of heavy
mesons.
For infinitely heavy quarks the spin of the heavy quark decouples, and
accordingly in a reaction not only the total angular momentum is conserved but
also the spin of the heavy quark and thus the total angular momentum of the
light quark system as well. Therefore, a heavy-light quark system can be labeled
by the total angular momentum of the light quark system $j_\ell$. Accordingly
the ground state mesons $D$ and $D^*$  ($\bar B$ and $\bar B^*$) form a doublet
with $j^P_{\ell}=1/2^-$, where we on purpose deviate from the standard notation
$s_\ell^P$ to remind the reader that the light quark part can well be a lot more
complicated than just a single quark.
Candidates of the next doublets of excited states are $D^*_0(2400)$\footnote{The
$D\pi$ $S$-wave resonant structure is probably more complicated than a single
broad resonance, as demonstrated by a two-pole structure in
Ref.~\cite{Albaladejo:2016lbb}.} and $D_1(2430)$ (the corresponding $B$-mesons
are still to be found), characterized by $j^P_{\ell}=1/2^+$ and a width of about
300~MeV, and $D_1(2420)$ and $D_2^*(2460)$ ($\bar B_1(5721)$ and $\bar
B^*_2(5747)$) with $j^P_{\ell}=3/2^+$ and a width of about 30~MeV. Since the
states with $j^P_{\ell}=1/2^+$ are too broad to support hadronic
molecules~\cite{Filin:2010se,Guo:2011dd}, in what follows we  focus on the
scattering of the ground state mesons off their anti-particles as well as on
that of the $j^P_{\ell}=3/2^+$ mesons off the ground state ones with one of them
containing a heavy quark and the other a heavy anti-quark.

We start with the former system. To be concrete, we take the charm mesons. In
the particle basis, there are six $S$-wave meson pairs with given 
$J^{PC}$~\cite{Nieves:2012tt}:
\begin{eqnarray}
&&0^{++}:\quad\left\{D\bar{D}({^1S_0}),D^*\bar{D}^*({^1S_0})\right\},\nonumber\\
&&1^{+-}:\quad\left\{D\bar{D}^*({^3S_1},-),D^*\bar{D}^*({^3S_1})\right\},
\nonumber\\[-3mm]
\label{eq:basis}  \\[-3mm]
&&1^{++}:\quad\left\{D\bar{D}^*({^3S_1},+)\right\},\nonumber\\
&&2^{++}:\quad\left\{D^*\bar{D}^*({^5S_2})\right\},\nonumber
\end{eqnarray}
where the individual partial waves are labelled as $^{2S+1}L_J$, with $S$, $L$, 
and $J$ denoting the total spin, the angular momentum, and the total momentum 
of the two-meson system, respectively. We define the $C$-parity eigenstates as
\begin{equation}
D\bar{D}^*(\pm)=\frac{1}{\sqrt{2}}\left(D\bar{D}^*\pm D^*\bar{D}\right), \label{eq:DDstar}
\end{equation}
which comply with the convention\footnote{
Notice that a different convention for the $C$-parity operator was used in
Ref.~\cite{Nieves:2012tt}. As a consequence, the off-diagonal transitions
of $V_{\rm LO}^{(0{++})}$ in Ref.~\cite{Nieves:2012tt} have a different sign as 
compared to Eq.~(\ref{C0++}), see also
Sec.~VI~A in Ref.~\cite{Guo:2016bjq} for further
 details of our convention.}
 for the $C$-parity transformation $\hat{C}{\cal M}=\bar{\cal M}$.
%
Because of HQSS, the interaction at LO is independent of the heavy
quark spin, and thus can be described by the matrix elements  $\langle j_{1\, 
\ell}',
j_{2\, \ell}',{j_\ell}| \hat\Ham_I |  j_{1\, \ell}, j_{2\, \ell},{j_\ell} 
\rangle$ where
the light quark systems get coupled to a total light-quark angular momentum of 
the two-meson system, $j_\ell$.
 Thus, for the systems under
consideration, we have two independent terms for each isospin ($I=0$ or $1$):
  $\langle 1/2,1/2,0 | \hat\Ham_I | 1/2,1/2,0 \rangle$ and $
\langle 1/2,1/2,1 | \hat\Ham_I | 1/2,1/2,1 \rangle $.
This simple observation leads to the conclusion that in the strict heavy quark
limit the six pairs in Eq.~\eqref{eq:basis} are grouped into two
multiplets with $j_\ell=0$ and 1, respectively.
In the heavy quark limit, it is convenient to
use a basis of states characterized via ${j_\ell^{PC}}\otimes {s_{c\bar c}^{PC} 
}$, where $s_{c\bar
c}$ refers to the total spin of the $c$ and $\bar c$ pair.
For the case of $S$-wave interactions only, both ${j_\ell^{PC}}$ and ${s_{c\bar
c}^{PC}}$ can only be in $0^{-+}$ or $1^{--}$. Therefore, the spin multiplet 
with
${j_\ell=0}$ contains two states with quantum numbers:
\begin{equation}
  {0_\ell^{-+}}\otimes {0_{c\bar c}^{-+} } = 0^{++}, \qquad
    {0_\ell^{-+}}\otimes {1_{c\bar c}^{--} } = 1^{+-} ,
    \label{eq:sl0}
\end{equation}
and the spin multiplet for ${j_\ell=1}$ has the following four states:
\begin{equation}
  {1_\ell^{--}}\otimes {0_{c\bar c}^{-+} } = 1^{+-}, \qquad
    {1_\ell^{--}}\otimes {1_{c\bar c}^{--} } = 0^{++} \oplus
    {1^{++}} \oplus 2^{++} .
    \label{eq:sl1}
\end{equation}
It becomes clear that if the ${1^{++}}$ state $X(3872)$ is a $D\bar D^*$
molecule, then it is in the multiplet with $j_\ell=1$~\cite{Voloshin:2004mh}, 
and has three spin partners with $J^{PC}=0^{++}$, $2^{++}$ and
$1^{+-}$ in the strict heavy quark limit as pointed out in
Refs.~\cite{Hidalgo-Duque:2013pva,Baru:2016iwj}. Based on an
analogous reasoning it 
was suggested already earlier that  $Z_b(10610)$ and $Z_b(10650)$ might have four more 
isovector partners $W_{b0}^{(\prime)}$, $W^{}_{b1}$ and 
$W^{}_{b2}$~\cite{Bondar:2011ev,Voloshin:2011qa,Mehen:2011yh}. A detailed and
quantitative analysis of these $W_{bJ}$ states can be found in
Ref.~\cite{Baru:2017gwo}.

It is worthwhile to notice that the two $1^{+-}$ states are in different
multiplets with $j_\ell=0$ and $1$, respectively, and thus cannot be related to
each other via HQSS. However, the isovector $Z_b(10610)$ and $Z_b(10650)$ are
located with similar distances to the $B\bar B^*$ and $B^*\bar B^*$ thresholds,
respectively. Such an approximate degeneracy suggests that the isovector
interactions in the $j_\ell=0$ and $j_\ell=1$ sectors are approximately the
same, and the off-diagonal transition strength in the isovector channel between
the two meson pairs with $J^{PC}=1^{+-}$ in Eq.~\eqref{eq:basis} approximately
vanishes. A fit to the Belle data of the $Z_b$ line shapes with HQSS constraints
implemented also leads to nearly vanishing channel coupling~\cite{Guo:2016bjq}.
This points towards an additional ``light quark spin symmetry'' as proposed by
Voloshin very recently~\cite{Voloshin:2016cgm}. While a deeper understanding for
such a phenomenon is still missing, it seems to be realized in the charm sector as
well for the charged $Z_c(3900)$~\cite{Ablikim:2013mio,Liu:2013dau} and
$Z_c(4020)$~\cite{Ablikim:2013wzq} observed by the BESIII and Belle
collaborations.
Note that in Ref.~\cite{Valderrama:2012jv} 
it is argued that channel couplings are suppressed while in Ref.~\cite{Baru:2016iwj}
they were claimed to be important to keep a well defined spin symmetry limit.
We come back to this controversy briefly later in this section.

When the physical nondegenerate masses for the heavy mesons are used, one needs 
to switch to the basis in terms of physical states in Eq.~\eqref{eq:basis}.
 In this basis and for a given set of quantum numbers $\{JPC\}$, the LO EFT
 potentials $V^{(JPC)}_{\rm LO}$, which respect HQSS,
 read~\cite{AlFiky:2005jd,Nieves:2012tt,Valderrama:2012jv}
\begin{eqnarray}\label{C0++}
&&V_{\rm LO}^{(0{++})}=
\begin{pmatrix}
C_{0a} & -\sqrt{3}C_{0b} \\
-\sqrt{3}C_{0b} & C_{0a}-2C_{0b}
\end{pmatrix},
\label{Vct0++}\\
&&V_{\rm LO}^{(1{+-})}=
\begin{pmatrix}
C_{0a}-C_{0b} & 2C_{0b} \\
2C_{0b} & C_{0a}-C_{0b}
\end{pmatrix},
\label{Vct1+-}\\
&&V_{\rm LO}^{(1{++})}=C_{0a}+C_{0b} \label{eq:contact2-a},\label{Vct1++} \\
&&V_{\rm LO}^{(2{++})}=C_{0a}+C_{0b} \label{eq:contact2-b},\label{Vct2++}
\end{eqnarray}
where $C_{0a}$ and $C_{0b}$ are two independent low-energy constants.
Thus, since in the spin symmetry limit $D$ and $D^*$ are degenerate, implying
that $D\bar D^*$ and $D^*\bar D^*$ loops are equal, the above equality of the
potentials in the $1^{++}$ and $2^{++}$ channels immediately predicts equal
binding energies for the two states in this limit.
 
 
Once HQSS violation is introduced into the system by the use of the
physical masses, the
two-multiplet pattern gets changed, however, the close connection between the
$1^{++}$ and $2^{++}$ states persists.
An inclusion of the one-pion exchange necessitates an extension of the basis, 
since now also $D$-waves need to be included. In fact, HQSS is preserved
only if all allowed $D$-waves are kept in the system, even if the
masses of the open flavor states are still kept degenerate~\cite{Baru:2016iwj}.
The probably most striking effect of
the $D$-waves, once the $D^*$-$D$ mass difference is included, is
that now transitions of the $2^{++}$ $D^*\bar D^*$ $S$-wave state to the
$D\bar D$ and $D\bar D^*$ $D$-wave become possible. It allows for a width of
this state of up to several tens of MeV~\cite{Albaladejo:2015dsa,Baru:2016iwj}, 
which might be accompanied by a sizeable shift in mass.
In addition, spin symmetry relations might get modified via the
coupling of the molecular states with regular charmonia as discussed recently
in Ref.~\cite{Cincioglu:2016fkm}.
 
For near-threshold states it is natural to assume that the contact terms are
independent of the heavy quark mass --- phenomenologically they can be viewed as
parameterizing the exchange of light meson resonances. Then one can also predict
the heavy quark flavor partners of the $\X$. The heavy quark spin and flavor
partners of the $\X$ predicted in Ref.~\cite{Guo:2013sya} with $\Lambda=0.5$~GeV
are listed in Table~\ref{tab:predictions}.
 
%------------------------------------------------------------------------
\begin{table}[tb]
\caption{\label{tab:predictions} Predictions of the partners of the $\X$ for 
$\Lambda=0.5$~GeV in Ref.~\cite{Guo:2013sya}.
}
\begin{ruledtabular}
\begin{tabular}{l c c c }
        $J^{PC}$ & States & Thresholds (MeV) & Masses (MeV)
       \\\hline
       $1^{++}$ & $\frac1{\sqrt{2}}(D\bar D^*+D^*\bar D)$ &
       3875.87 & 3871.68 (input)  \\
                       $2^{++}$ & $D^*\bar D^*$ &
       4017.3  & $4012^{+4}_{-5}$  \\
       $1^{++}$ & $\frac1{\sqrt{2}}(B\bar B^*+B^*\bar B)$ &
       10604.4 & $10580^{+9}_{-8}$  \\
                       $2^{++}$ & $B^*\bar B^*$ &
       10650.2 & $10626^{+8}_{-9}$  \\
                       $2^{+}$ & $D^*B^*$ &
       7333.7 & $7322^{+6}_{-7}$  \\ 
   \end{tabular}
\end{ruledtabular}
\end{table}
%------------------------------------------------------------------------
The $Z_b(10610)$ can be related to the $Z_b(10650)$ when the off-diagonal 
interaction is neglected as discussed above.  Their hidden-charm partners
are found to be virtual states in this formalism~\cite{Guo:2013sya}, which may 
correspond to the $Z_c(3900)$ and $Z_c(4020)$. In fact, it is shown in 
Ref.~\cite{Albaladejo:2015lob} that the BES\-III data for the $Z_c(3900)$ in
both the $J/\psi\pi$~\cite{Ablikim:2013mio} and  $D\bar 
D^*$~\cite{Ablikim:2015swa} 
modes can be well fitted with either a resonance above the $D\bar D^*$
threshold or a virtual state below.


The number of the LO contact terms is larger for the interaction between a pair 
of $j_\ell=1/2$ and $j_\ell=3/2$ heavy and anti-heavy mesons.
For each isospin, 0 or 1, in the heavy quark limit, there are four independent 
interactions
denoted as $\langle
j_{1\, \ell},j_{2\, \ell},j_\ell|\hat\Ham_I |j_{1\, \ell}',j_{2\, \ell}',j_\ell
\rangle$, where now $j_\ell$ can take values 1 or 2
\begin{eqnarray}
  F_{Ij_\ell}^d &\equiv& \left\langle \frac12,\frac32,j_\ell \left|\hat\Ham_I 
\right|\frac12,\frac32,j_\ell \right\rangle , \nonumber\\
  F_{Ij_\ell}^c &\equiv& \left\langle \frac12,\frac32,j_\ell \left|\hat\Ham_I 
  \right|\frac32,\frac12,j_\ell \right\rangle .
  \label{eq:VHT}
\end{eqnarray}
The relevant combinations of these constants for a given heavy meson pair can 
be worked out by changing the basis by means of a unitary transformation
(see, e.g.,~\cite{Ohkoda:2012rj,Xiao:2013yca}):
\begin{eqnarray}
  &&| s_{1\,c},j_{1\, \ell},j_1; s_{2\,c},j_{2\, \ell},j_2;J\rangle \nonumber\\
  &=& \sum_{s_{c\bar c}, j_\ell }  
 \sqrt{ (2j_1+1)(2j_2+1) (2s_{c\bar c}+1)(2j_\ell+1) } \nonumber\\
 && \times 
 \begin{Bmatrix}
   s_{1\,c} & s_{2\,c} & s_{c\bar c} \\
   j_{1\, \ell} & j_{2\, \ell} & j_\ell \\
   j_1 & j_2 & J
 \end{Bmatrix}
 |s_{1\,c},s_{2\,c},s_{c\bar c};j_{1\, \ell},j_{2\, \ell},j_{\ell}; J\rangle ,
 ~~~
\end{eqnarray}
where $j_1$ and $j_2$ are the spins of the two heavy mesons, $J$ is the 
total angular momentum of the whole system, and
$s_{1\,c}$ and $s_{2\,c}$ are the spins of the heavy quark. 

Consider two mesons $A$ and $B$; each of them is not a $C$ parity eigenstate, but their linear combination can form $C$ parity eigenstates. With the phase convention specified below \eqref{eq:DDstar},
the $C=\pm$ eigenstates of a flavor-neutral system consisting of a pair of mesons are given by 
\begin{equation}
    |C=\pm \rangle=\frac{1}{\sqrt{2}}\left[A B \pm (-1)^{J-J_{A}-J_{B}} \bar B \bar A\right],
\end{equation}
where $J_A$ and $J_B$ are the spins of the mesons $A$ and $B$, and $J$ is the total spin of the two-body system.

Noticing that the
total spin of the heavy quark and anti-quark $s_{c\bar c}$ is conserved in the
heavy quark limit, and combining the meson pairs into eigenstates of 
$C$-parity, one can obtain the contact terms for the $S$-wave interaction 
between a pair of $j_\ell=\frac12$ and $\frac32$ heavy and anti-heavy mesons. 
The diagonal ones are listed in Table~\ref{tab:HTcontact}.
%-----------------------
\begin{table}
  \caption{ The diagonal contact terms for the $S$-wave interaction between a 
pair of $j_\ell^P=1/2^-$ and $3/2^+$ heavy and anti-heavy mesons.
\label{tab:HTcontact}
}
\begin{ruledtabular}
  \centering\begin{tabular}{L C C}
    J^{PC} & \text{Meson pairs} & \text{Contact terms}\\\hline
    1^{{--}} & \frac{1}{\sqrt{2}} \left(D\bar{D}_1-D_1\bar{D}\right) & \frac{1}{8} \left(-F_{I1}^c-5
F_{I2}^c+3 F_{I1}^d+5 F_{I2}^d\right) \\
& \frac{1}{\sqrt{2}}\left(D^*\bar{D}_1+D_1\bar{D}^*\right) & \frac{1}{16} \left(7 F_{I1}^c-5
F_{I2}^c+11 F_{I1}^d+5 F_{I2}^d\right) \\
& \frac{1}{\sqrt{2}}\left(D^*\bar{D}_2-D_2\bar{D}^*\right) & \frac{1}{16} \left(-5
F_{I1}^c-F_{I2}^c+15 F_{I1}^d+F_{I2}^d\right) \\[2mm]

0^{--} & \frac{1}{\sqrt{2}}\left(D^*\bar{D}_1-D_1\bar{D}^*\right) & F_{I1}^c+F_{I1}^d \\[2mm]

2^{--} & \frac{1}{\sqrt{2}}\left(D\bar{D}_2-D_2\bar{D}\right) & \frac{1}{8} \left(3
F_{I1}^c-F_{I2}^c+3 F_{I1}^d+5 F_{I2}^d\right) \\
& \frac{1}{\sqrt{2}}\left(D^*\bar{D}_1-D_1\bar{D}^*\right) & \frac{1}{16} \left(F_{I1}^c-3
F_{I2}^c+F_{I1}^d+15 F_{I2}^d\right) \\
& \frac{1}{\sqrt{2}}\left(D^*\bar{D}_2{+}D_2\bar{D}^*\right) & \frac{1}{16} \left(9 F_{I1}^c+5 F_{I2}^c+9
F_{I1}^d+7 F_{I2}^d\right)
\\[2mm]

3^{--} & \frac{1}{\sqrt{2}}\left(D^*\bar{D}_2-D_2\bar{D}^*\right) & F_{I2}^d-F_{I2}^c \\[2mm]

0^{-+} & \frac{1}{\sqrt{2}}\left(D^*\bar{D}_1+D_1\bar{D}^*\right) & F_{I1}^d-F_{I1}^c \\ [2mm]

1^{-+} & \frac{1}{\sqrt{2}} \left(D\bar{D}_1+D_1\bar{D}\right) & \frac{1}{8} \left[5
\left(F_{I2}^c+F_{I2}^d\right)+F_{I1}^c+3 F_{I1}^d\right] \\
& \frac{1}{\sqrt{2}}\left(D^*\bar{D}_1-D_1\bar{D}^*\right) & \frac{1}{16} \left[5
\left(F_{I2}^c+F_{I2}^d\right)-7 F_{I1}^c+11 F_{I1}^d\right] \\
& \frac{1}{\sqrt{2}}\left(D^*\bar{D}_2+D_2\bar{D}^*\right) & \frac{1}{16} \left(5 F_{I1}^c+F_{I2}^c+15
F_{I1}^d+F_{I2}^d\right) \\ [2mm]

2^{-+} & \frac{1}{\sqrt{2}}\left(D\bar{D}_2+D_2\bar{D}\right) & \frac{1}{8} \left(-3 F_{I1}^c+F_{I2}^c+3
F_{I1}^d+5 F_{I2}^d\right) \\
& \frac{1}{\sqrt{2}}\left(D^*\bar{D}_1+D_1\bar{D}^*\right) & \frac{1}{16} \left[3 \left(F_{I2}^c+5
F_{I2}^d\right)-F_{I1}^c+F_{I1}^d\right] \\
& \frac{1}{\sqrt{2}}\left(D^*\bar{D}_2{-}D_2\bar{D}^*\right) & \frac{1}{16} \left(-9 F_{I1}^c-5
F_{I2}^c+9 F_{I1}^d+7 F_{I2}^d\right) \\ [2mm]
3^{-+} & \frac{1}{\sqrt{2}}\left(D^*\bar{D}_2+D_2\bar{D}^*\right) & F_{I2}^c+F_{I2}^d \\
\end{tabular}
\end{ruledtabular}
\end{table}
%-----------------------
One sees that the linear combinations are different for all channels, and 
it is not as easy as in case of  the $\X$ to predict spin partners for the $\Y$
based on the assumption that it is predominantly a $D_1\bar D$ state. The
possibility of $S$-wave hadronic molecules with exotic quantum numbers $1^{-+}$
was discussed in~\cite{Wang:2014wga}. 
Here we also give the off-diagonal contact terms.
There are three channels in each of $1^{--}$, $2^{--}$, $1^{-+}$ and $2^{-+}$ sectors, 
we label them in each sector listed in Table~\ref{tab:HTcontact} from top to bottom as 1, 2, and 3. Then the off-diagonal contact terms for the $1^{--}$ sector are 
\begin{equation}
    \begin{aligned}
    V_{12} &= -\frac{1}{8 \sqrt{2}}\left[5\left(F_{I 2}^{c}+F_{I 1}^{d}-F_{I 2}^{d}\right)+F_{I 1}^{c}\right], \\
    V_{13}&= \frac{1}{8} \sqrt{\frac{5}{2}}\left(-3 F_{I 1}^{c}+F_{I 2}^{c}+F_{I 1}^{d}-F_{I 2}^{d}\right), \\
    V_{23}&= \frac{\sqrt{5}}{16} \left(5 F_{I 1}^{c}+F_{I 2}^{c}+F_{I 1}^{d}-F_{I 2}^{d}\right) . 
    \end{aligned}
\end{equation}
The off-diagonal contact terms for the $2^{--}$ sector are 
\begin{equation}
    \begin{aligned}
        V_{12} &= -\frac{1}{8} \sqrt{\frac{3}{2}}\left(F_{I 1}^{c}+5 F_{I 2}^{c}+F_{I 1}^{d}-F_{I 2}^{d}\right), \\
        V_{13} &= \frac{1}{8} \sqrt{\frac{3}{2}}\left(-3 F_{I 1}^{c}+ F_{I 2}^{c}-3 F_{I 1}^{d}+3 F_{I 2}^{d}\right),\\
        V_{23}& =\frac{3}{16}\left(F_{I 1}^{c}-3 F_{I 2}^{c}+F_{I 1}^{d}-F_{I 2}^{d}\right).
    \end{aligned}
\end{equation}
The off-diagonal contact terms for the $1^{-+}$ sector are 
\begin{equation}
\begin{aligned}
    V_{12}&= \frac{1}{8 \sqrt{2}}\left[5\left(F_{I 2}^{c}-F_{I 1}^{d}+F_{I 2}^{d}\right)+F_{I 1}^{c} \right], \\
    V_{13}&= \frac{1}{8} \sqrt{\frac{5}{2}}\left(3 F_{I 1}^{c}-F_{I 2}^{c}+F_{I 1}^{d}-F_{I 2}^{d}\right), \\
    V_{23}&= -\frac{1}{16} \sqrt{5}\left(5 F_{I 1}^{c}+F_{I 2}^{c}-F_{I 1}^{d}+F_{I 2}^{d}\right).
\end{aligned}
\end{equation}
The off-diagonal contact terms for the $2^{-+}$ sector are 
\begin{equation}
    \begin{aligned}
        V_{12} &= \frac{1}{8} \sqrt{\frac{3}{2}}\left(F_{I 1}^{c}+5 F_{I 2}^{c}-F_{I 1}^{d}+F_{I 2}^{d}\right), \\
        V_{13} &= \frac{1}{8} \sqrt{\frac{3}{2}}\left(3 F_{I 1}^{c}- F_{I 2}^{c}-3 F_{I 1}^{d}+3 F_{I 2}^{d}\right),\\
        V_{23} &= -\frac{3}{16} \left(F_{I 1}^{c}-3 F_{I 2}^{c}-F_{I 1}^{d}+F_{I 2}^{d}\right).
    \end{aligned}
\end{equation}

However, one non-trivial prediction for the spectrum of molecular states in the
heavy quarkonium spectrum becomes apparent immediately from the discussion
above: Since the most bound states appear in $S$-waves  the lightest negative
parity vector state can be formed only from $j_\ell^P=1/2^-$ and $3/2^+$
heavy and anti-heavy mesons.
Therefore the mass difference of $X(3872)$ as bound state of two ground state
$j_\ell^P=\frac12^-$ mesons ($D$ and $D^*$) and the lightest exotic vector state
$Y(4260)$ (388 MeV) should be of the order of the mass difference of the
lightest $3/2^+$ state and the $D^*$ (410 MeV). Clearly this prediction is
nicely realized in nature. Note that from this reasoning it also follows that
$if$ the $Y(4008)$ indeed were to exist it could not be a hadronic molecule. In
this context it is interesting to note that the most resent data from BESIII on
$e^+e^-\to J/\psi\pi\pi$~\cite{Ablikim:2016qzw} does not seem to show evidence
for the $Y(4008)$, {\sl c.f.} Fig.~\ref{fig:Y4260-BESIII}.



\subsection{Impact of hadron loops on regular quarkonia}
\label{sec:4-NREFT_ccbar}

In the previous sections we argued that meson loops play a prominent role in
both the formation and the decays of hadronic molecules. One may wonder if they
also have an impact on the properties of regular charmonia.
In this section we demonstrate that certain processes for regular hadrons,
largely well described by the quark model, can also be influenced by significant
meson loop effects, since reaction rates can receive an enhancement due to the
nearly on-shell intermediate heavy mesons.
The origin of this mechanism is that for most heavy quarkonium transitions
$M_{Q\bar Q}-2 M_{Q\bar q}\ll M_{Q\bar q}$, where $M_{Q\bar Q}$ and $M_{Q\bar
q}$ are the masses of the heavy quarkonium and an open-flavor heavy meson,
respectively. As a result, the intermediate heavy mesons are nonrelativistic
with a small velocity
\begin{equation}
    v\sim \sqrt{|{M_{Q\bar Q}}-2m_{Q\bar q}|/m_{Q\bar q}}\ll 1\,,
\end{equation}
and the meson loops in the transitions can be investigated 
using \nreft.
We will highlight this effect on two examples in what follows.\footnote{The
effects of meson loops in heavy quarkonium spectrum are investigated in,
e.g., Refs.~\cite{Eichten:1978tg,Eichten:1979ms,Ono:1983rd,Kalashnikova:2005ui,
Eichten:2005ga,Pennington:2007xr,Barnes:2007xu, Li:2009ad,Danilkin:2009hr,
Ortega:2010qq,Danilkin:2010cc,Liu:2011yp,Bali:2011rd,Zhou:2013ada,
Ferretti:2013faa,Ferretti:2013vua,Ferretti:2014xqa, Hammer:2016prh,Du:2016qcr,
Lu:2016mbb, Lu:2017hma,Zhou:2017dwj}, and in heavy quarkonium transitions in
Refs.~\cite{Ono:1985jt, Lipkin:1988tg,Moxhay:1988ri,Zhou:1990ik,
Li:2007xr,Meng:2007tk,Meng:2008bq,
Liu:2009dr,Zhang:2009kr,Zhang:2010zv,Wang:2011yh, Li:2011ssa,Wang:2012mf,
Guo:2012tj,Li:2013zcr,Cao:2016xqo}.}

We start with the hindered M1 transitions between two $P$-wave heavy quarkonia
with different radial excitations, such as the $h_c(2P)\to \gamma
\chi_{cJ}(1P)$. It was proposed in Ref.~\cite{Guo:2011dv,Guo:2016yxl} that such
transitions are very sensitive to meson-loop effects, and the pertinent partial
widths provide a way to extract the coupling constants between the $P$-wave
heavy quarkonia and heavy open flavor mesons.
% -------------------------------------------------------------------
\begin{figure}[t]
    \centering \includegraphics[width=\linewidth]{./figures/hinderedM1loops}
    \caption{Feynman diagrams for the coupled-channel effects for the hindered
    M1 transitions between heavy quarkonia. The one-loop contributions are given
by (a) and (b). (c) and (d) are two typical two-loop diagrams.
The double, solid, wavy and dashed lines represent heavy quarkonia, heavy
mesons, photons, and pion, respectively. Adapted from Ref.~\cite{Guo:2011dv}.
\label{fig:M1loops}}
\end{figure}
% -------------------------------------------------------------------
In quark models the amplitude for such a transition is proportional to the
overlap of the wave functions of the initial and final heavy quarkonia, which is
tiny and quite model-dependent due to the different radial excitations --- this
is why they are called ``hindered''. This suppression is avoided in the
coupled-channel mechanism of heavy-meson loops.
In this mechanism, the initial and final $P$-wave heavy quarkonia couple to the
ground state pseudoscalar and vector heavy mesons in an $S$-wave. A few diagrams
contributing to this mechanism are shown in Fig.~\ref{fig:M1loops}. In (a), the
photon is emitted via its magnetic coupling to intermediate heavy mesons. In
(b), since the $S$-wave vertices do not have any derivative at LO, the photon
couples in a gauge invariant way to one of the vertices in the two-point loop
diagram. (c) and (d) are two typical two-loop diagrams. From the power counting
rules discussed in Sec.~\ref{sec:nreft1}, Fig.~\ref{fig:M1loops}~(a) provides
the leading contribution, while (b) is of higher order in the velocity counting
because there is one less nonrelativistic propagator. Their amplitudes scale as
\begin{equation}
  \Amp_{(a)}  \sim \frac{E_\gamma}{m_Q v}, \qquad \Amp_{(b)}  \sim 
\frac{E_\gamma v}{m_Q}\,,
\label{eq:M1Aab}
\end{equation}
respectively,
where $E_\gamma$ is the photon energy, and the dependence on the 
coupling constants is dropped. The $1/m_Q$ suppression comes from the fact that 
the polarization of a heavy (anti-)quark needs to be flipped in the M1 
transitions. For the two-loop diagram in (c), the amplitude scales as
\begin{equation}
   \Amp_{\rm (c)} \sim \frac{(v^5)^2}{(v^2)^5}
\frac{g^2}{(4\pi)^2F_\pi^2}  \frac{E_\gamma}{m_Q} M_H^2 = \frac{E_\gamma}{m_Q}
\left(\frac{g M_H}{\Lambda_\chi}\right)^2,
\label{eq:M1Ac}
\end{equation}
where the factor $1/(4\pi)^2$ appears because there is one more loop and the
hadronic scale $\Lambda_\chi=4\pi F_\pi\sim 1$ GeV was introduced as the hard
scale for the chiral expansion. The factor of $M_H^2$ was introduced to match
dimensions of the above equations to those of Eqs.~\eqref{eq:M1Aab}. Diagram (d)
has the same scaling as (c). Since the axial coupling constant $g\simeq0.6$ for
the charm case as determined from the width of $D^*\to D\pi$, and about 0.5 for
bottom~\cite{Flynn:2015xna}, one has $g M_D/\Lambda_\chi\lesssim1$ and $g
M_B/\Lambda_\chi\simeq2$. The value for $v$ defined as $(v_1+v_2)/2$ is about
0.4 for the transitions from the 2P to 1P charmonium states~\cite{Guo:2011dv},
and ranges from 0.3 to 0.2 for the transitions between $1P,2P$ and $3P$
bottomonia~\cite{Guo:2016yxl}. Hence, the two-loop diagrams are suppressed in
the charm sector, while they are of the same order as (a) for the bottom sector.
Therefore, one can make predictions for the charmonium transitions by
calculating the loops corresponding to (a). The results depend on a product of
two unknown coupling constants of the $1P$ and $2P$ charmonia to the charm
mesons. Taking model values for them, the decay width of the
$\chi_{c2}(2P)\to\gamma h_c(1P)$ is of $\mathcal{O}(100~\text{keV})$, two orders
of magnitude larger than the quark model prediction,
1.3~keV~\cite{Barnes:2005pb}.
Although quantitative predictions cannot be made for the bottomonium
transitions, it is expected that once such transitions would be observed they
must be due to coupled-channel effects as the partial widths were predicted to
be in the range from sub-eV to eV level in a quark model calculation that does
not include meson-loop effects~\cite{Godfrey:2015dia}. It is suggested in
\cite{Guo:2011dv,Guo:2016yxl} that the coupled-channel effects can be checked by
comparing results from both fully dynamical and quenched lattice QCD which has
and has no coupled-channel effects, respectively. Recent developments in lattice
QCD calculations of radiative decays~\cite{Dudek:2006ej,Dudek:2009kk,
Shultz:2015pfa,Briceno:2016kkp,Agadjanov:2014kha,Feng:2014gba,Leskovec:2016lrm,
Meyer:2011um,Owen:2015fra} should be helpful in illuminating this issue.



There are other heavy quarkonium transitions driven mainly by the
coupled-channel effects. A detailed study on the transitions between two
charmonia ($S$- and $P$-wave) with the emission of a pion or eta can be found in
Ref.~\cite{Guo:2010ak}. It is found that whether the coupled-channel effects
play a sizable role depends on the process. This is a result of the power
counting analysis; see the itemized discussion in Sec.~\ref{sec:nreft1}.
In particular, the  single-pion/eta transitions between two $S$-wave and
$P$-wave charmonia receive important contribution from charm-meson loops.
Therefore, the long-standing suggestion that the  $\psi'\to J/\psi\eta/\pi^0$
transitions can be used to extract the light quark mass
ratio~\cite{Ioffe:1979rv} needs to be reexamined. In fact, if we assume that
the triangle charm meson-loop diagrams saturate the transitions, the resulting
prediction of $\mathcal{B}(\psi'\to J/\psi\pi^0)/\mathcal{B}(\psi'\to
J/\psi\eta)$ is consistent with the experimental data. These transitions were
revisited considering both the loop and tree diagrams in
Ref.~\cite{Mehen:2011tp}.
Again based on the same power counting rules it was argued that the transitions
$\Upsilon(4S)\to h_b\pi^0/\eta$ have only a small pollution from the
bottom-meson loops, and are dominated by short-distance contribution
proportional to the light quark mass difference~\cite{Guo:2010ca}. They could be
used for the extraction of light quark mass ratio. Furthermore, the prediction,
made before the discovery of the $h_b(1P)$, on the branching fraction of the
order of $10^{-3}$ for the decay $\Upsilon(4S)\to h_b\eta$ was verified by the
Belle measurement, $(2.18\pm0.11\pm0.18)\times10^{-3}$~\cite{Tamponi:2015xzb}.

Parameter-free predictions can be made for ratios of partial widths of decays
dominated by the coupled-channel effects of heavy mesons in the same spin
multiplet, since all the coupling constants will get canceled in the ratios.
Such predictions on the hindered M1 transitions between $P$-wave heavy quarkonia
can be found in Refs.~\cite{Guo:2011dv,Guo:2016yxl}.

In Ref.~\cite{Guo:2012tg}, it is pointed out that coupled-channel effects can
even introduce sizable and nonanalytic pion mass dependence in heavy quarkonium
systems which couple to open-flavor heavy meson pairs in an $S$-wave.

To summarize this subsection, we stress that whether meson-loop effects are
important for the properties of quarkonia or not does not only depend on the
proximity to the relevant threshold, it is  also depends on the particular
transition studied.





In this section, we evaluate our method on two large-scale multimodal video benchmarks. The results show that our method outperforms representative baseline methods and achieves the state-of-the-art performance on both benchmarks. 



\subsection{Datasets and Setups}\label{sec:dataset_setups}
We evaluate our method on two large-scale multimodal video benchmarks: NTU RGB+D~\cite{ntu_rgbd} (classification) and PKU-MMD~\cite{pku_mmd} (detection). These datasets are selected for the following reasons. (1) They are (one of the) largest RGB-D video benchmarks in each category. (2) The privileged information transfer is reasonable because the domains of the two datasets are similar. (3) They contain abundant modalities, which are required for graph distillation. 

We use NTU RGB+D as our dataset in the source domain, and PKU-MMD in the target domain. In our experiments, unless stated otherwise, we apply graph distillation whenever applicable. Specifically, the visual encoders of all modalities are jointly trained on NTU RGB+D by graph distillation. On PKU-MMD, after initializing the visual encoder with the pre-trained weights obtained from NTU RGB+D, we also learn all available modalities by graph distillation on the target domain. By default, only a single modality is used at test time.

\noindent\textbf{NTU RGB+D~\cite{ntu_rgbd}.} 
It contains 56,880 videos from 60 action classes. Each video has exactly one action class and comes with four modalities: RGB, depth, 3D joints, and infrared. The training and testing sets have 40,320 and 16,560 videos, respectively. All results are reported with cross-subject evaluation.

\noindent\textbf{PKU-MMD~\cite{pku_mmd}.} 
It contains 1,076 long videos from 51 action classes. Each video contains approximately 20 action instances of various lengths and consists of four modalities: RGB, depth, 3D joints, and infrared. All results are evaluated based on the Average Precision (mAP) at different temporal Intersection over Union (tIoU) thresholds between the predicted and the ground truth intervals.

\noindent\textbf{Modalities.} We use a total of six modalities in our experiments: RGB, depth (D), optical flow (F), and three skeleton features (S) named Joint-Joint Distances (JJD), Joint-Joint Vector (JJV), and Joint-Line Distances (JLD)~\cite{ding2017investigation,10-stream}, respectively. The RGB and depth videos are provided in the datasets. The optical flow is calculated on the RGB videos using the dual TV-L1 method~\cite{zach2007duality}. The three spatial skeleton features are extracted from 3D joints using the method in \cite{ding2017investigation} and \cite{10-stream}. Note that we select a subset of the ten skeleton features in~\cite{ding2017investigation,10-stream} to ensure the simplicity and reproducibility of our method, and our approach can potentially perform better with the complete set of features.

\noindent\textbf{Baselines.}
In addition to comparing with the state-of-the-art, we implement three representative baselines that could be used to leverage multimodal privileged information: \textit{multi-task learning}~\cite{caruana1998multitask}, \textit{knowledge distillation}~\cite{distillation_hinton}, and \textit{cross-modal distillation}~\cite{distillation_gupta}. For the multi-task model, we predict the raw pixels of the other modalities from the representation of a single modality, and use the $L_2$ distance as the multi-task loss. For the distillation methods, the imitation loss is calculated as the high-temperature cross-entropy loss on the soft logits~\cite{distillation_hinton}, and $L_2$ loss on both representations and soft logits in cross-modal distillation~\cite{distillation_gupta}. These distillation methods originally only support two modalities, and therefore we average the pairwise losses to get the final loss.



\begin{table}[t]
\centering
\scriptsize
\caption{Comparison with state-of-the-art on NTU RGB+D. Our models are trained on all modalities and tested on the single modality specified in the table. The available modalities are RGB, depth (D), optical flow (F), and skeleton (S).}
\label{ntu_state_of_the_art}
\begin{tabular}{lc@{\hskip 0.1in}c@{\hskip 0.8in}l@{\hskip 0.4in}c@{\hskip 0.1in}c}
\toprule
Method & Test Modality & mAP & Method & Test Modality & mAP  \\
\midrule
Shahroudy~\cite{shahroudy2017deep} & RGB+D & 0.749 & Ours & RGB & \textbf{0.895} \\
Liu~\cite{liu2017viewpoint} & RGB+D & 0.775 & Ours  & D & 0.875 \\
Liu~\cite{skeleton_visualization} & S & 0.800 & Ours  & F & 0.857 \\
Ding~\cite{ding2017investigation} & S & 0.823 & Ours  & S & 0.837 \\
Li~\cite{10-stream} & S & 0.829 &&& \\
\bottomrule
\end{tabular}
\end{table}

\begin{table}[t]
\centering
\scriptsize
\caption{Comparison of action detection methods on PKU-MMD with state-of-the-art models. Our models are trained with graph distillation using all privileged modalities
and tested on the modalities specified in the table. ``Transfer'' refers to pre-training on NTU RGB+D on action classification. The available modalities are RGB, depth (D), optical flow (F), and skeleton (S).}
\label{pku_state_of_the_art}
\begin{tabular}{l@{\hskip 0.1in}c@{\hskip 0.1in}c@{\hskip 0.1in}c@{\hskip 0.1in}c}
\toprule
\multicolumn{2}{c}{} & \multicolumn{3}{c}{mAP @ tIoU thresholds ($\theta$)} \\
\cmidrule(r){3-5}
Method & Test Modality & 0.1 & 0.3 & 0.5 \\ 
\midrule
Deep RGB (DR) \cite{pku_mmd} & RGB & 0.507 & 0.323 & 0.147 \\
Qin and Shelton \cite{pku_result_qin} & RGB & 0.650 & 0.510 & 0.294 \\
Deep Optical Flow (DOF) \cite{pku_mmd} & F & 0.626 & 0.402 & 0.168 \\
Raw Skeleton (RS) \cite{pku_mmd} & S & 0.479 & 0.325 & 0.130 \\
Convolution Skeleton (CS) \cite{pku_mmd} & S & 0.493 & 0.318 & 0.121 \\
Wang and Wang \cite{pku_result_wang_workshop} & S & 0.842 & - & 0.743 \\
RS+DR+DOF \cite{pku_mmd} & RGB+F+S & 0.647 & 0.476 & 0.199 \\
CS+DR+DOF \cite{pku_mmd} & RGB+F+S & 0.649 & 0.471 & 0.199 \\
\midrule
Ours (w/o $|$ w/ transfer) & RGB & 0.824 $|$ 0.880 & 0.813 $|$ 0.868 & 0.743 $|$ 0.801 \\
Ours (w/o $|$ w/ transfer) & D   & 0.823 $|$ 0.872 & 0.817 $|$ 0.860 & 0.752 $|$ 0.792 \\
Ours (w/o $|$ w/ transfer) & F   & 0.790 $|$ 0.826 & 0.783 $|$ 0.814 & 0.708 $|$ 0.747 \\
Ours (w/o $|$ w/ transfer) & S   & 0.836 $|$ 0.857 & 0.823 $|$ 0.846 & 0.764 $|$ 0.784 \\
Ours (w/ transfer) & RGB+D+F+S & \bf{0.903} & \bf{0.895} & \bf{0.833} \\
\bottomrule
\end{tabular}
\end{table}



\noindent\textbf{Implementation Details.} 
For action classification, we train the visual encoder from scratch for 200 epochs using SGD with momentum with learning rate $10^{-2}$ and decay to $10^{-1}$ at epoch 125 and 175. $\lambda_1$ and $\lambda_2$ are set to $10,5$ respectively in Eq.~\eqref{eq:message_ab}. At test time we sample 5 clips for inference. For action detection, the visual and sequence encoder are trained for 400 epochs. The visual encoder is trained using SGD with momentum with learning rate $10^{-3}$, and the sequence encoder is trained with the Adam optimizer~\cite{kingma2015adam} with learning rate $10^{-3}$. The activity threshold $\gamma$ is set to $0.4$. For both tasks, we down-sample the frame rates of the datasets by a factor of 3. The clip length and detection window $T_c$ and $T_w$ are both set to 10. For the graph distillation, $\alpha$ is set to 10 in Eq.~\eqref{eq:graph_learning_softmax}. The output dimensions of the visual and sequence encoder are both set to 512. Since it is nontrivial to jointly train on multiple modalities from scratch, we employ curriculum learning~\cite{bengio2009curriculum} to train the distillation graph. To do so, we first fix the distillation graph as an identity matrix (uniform graph) in the first 200 epochs. In the second stage, we compute the constant vector $\mathbf{c}$ in Eq.~\eqref{eq:message_graph_final} according to the cross-validation results, and then learn the graph in an end-to-end manner.



\subsection{Comparison with State-of-the-Art}\label{sec:exp_soa}



\begin{figure}[t]
\begin{center}
\includegraphics[width=\linewidth]{predictions}
\end{center}
\caption{\textbf{A comparison of the prediction results on PKU-MMD.} (a) Both models make correct predictions. (b) The model without distillation in the source makes errors. Our model learns motion and skeleton information from the privileged modalities in the source domain, which helps the prediction for classes such as ``hand waving'' and ``falling''. (c) Both models make reasonable errors.}
\label{fig:detection}
\end{figure}





\noindent\textbf{Action Classification.} Table~\ref{ntu_state_of_the_art} shows the comparison of action classification with state-of-the-art models on NTU RGB+D dataset. Our graph distillation models are trained and tested on the same dataset in the source domain. NTU RGB+D is a very challenging dataset and has been recently studied in numerous studies~\cite{10-stream,liu2017viewpoint,skeleton_visualization,luo2017unsupervised,shahroudy2017deep}. Nevertheless, as we see, our model achieves the state-of-the-art results on NTU RGB+D. It yields a 4.5\% improvement, over the previous best result, using the depth video and a remarkable 6.6\% using the RGB video. After inspecting the results, we found the improvement mainly attributes to the learned graph capturing complementary information across multiple modalities. Fig.~\ref{fig:graph} shows example distillation graphs learned on NTU RGB+D. The results show that our method, without transfer learning, is effective for action classification in the source domain.


\noindent\textbf{Action Detection.} Table~\ref{pku_state_of_the_art} compares our method on PKU-MMD with previous work. Our model outperforms existing methods across all modalities. The results substantiate that our method can effectively leverage the privileged knowledge from multiple modalities. Fig.~\ref{fig:detection} illustrates detection results on the depth modality with and without the proposed distillation.


\subsection{Ablation Studies on Limited Training Data}\label{sec:ablation}
Section~\ref{sec:exp_soa} has shown that our method achieves the state-of-the-art results on two public benchmarks. However, in practice, the training data are often limited in size. To systematically evaluate our method on limited training data, as proposed in the introduction, we construct mini-NTU RGB+D and mini-PKU-MMD by randomly sub-sampling 5\% of the training data from their full datasets and use them for training. For evaluation, we test the model on the full test set.



\begin{table}[t]
\centering
\scriptsize
\caption{The comparison with (a) baseline methods using Privileged Information (PIs) on mini-NTU RGB+D, (b) distillation graphs on mini-NTU RGB+D and mini-PKU-MMD. Empty graph trains each modality independently. Uniform graph uses a uniform weight in distillation. Prior graph is built according to the cross-validation accuracy of each modality. Learned graph is learned by our method. ``D'' refers to the depth modality.}
\subtable[\label{ntu_baselines}Baseline methods using PIs.]
{
  \renewcommand{\arraystretch}{1.1}
  \begin{tabular}{lcc}
  \toprule
  Method & mAP / RGB \\
  \midrule
  Empty graph & 0.464 \\
  Multi-task \cite{caruana1998multitask}  & 0.456 \\
  Cross-distillation \cite{distillation_gupta}  & 0.503 \\
  Knowledge distillation \cite{distillation_hinton}  & 0.524 \\
  Learned graph & \bf{0.619} \\
  \bottomrule
  \end{tabular}
}
\subtable[\label{different_graphs}Different distillation graphs.]{
  \begin{tabular}{l@{\hskip 0.1in}c@{\hskip 0.2in}c}
  \toprule
  \multicolumn{1}{c}{} & mini-NTU & mini-PKU \\
  \cmidrule(r){2-3}
  Graph & {\tiny mAP / RGB} & {\tiny mAP @ 0.5 / D} \\
  \midrule
  Empty graph & 0.464 & 0.501 \\
  Uniform graph & 0.537 & 0.513 \\
  Prior graph & 0.571 & 0.515 \\
  Learned graph & \bf{0.619} & \bf{0.559}\\
  \bottomrule
  \end{tabular}
}
\end{table}

\begin{table}[t]
\centering
\scriptsize
\caption{The mAP comparison on mini-PKU-MMD at different tIoU threshold $\theta$. The depth modality is chosen for testing. ``src'', ``trg'', and ``PI'' stand for source, target, and privileged information, respectively.}
\label{pku_mmd_baselines}
\begin{tabular}{c@{\hskip 0.2in}l@{\hskip 0.4in}c@{\hskip 0.4in}c@{\hskip 0.4in}c}
\toprule
\multicolumn{2}{c}{} & \multicolumn{3}{c}{mAP @ tIoU thresholds ($\theta$)} \\
\cmidrule(r){3-5}
 & Method & $0.1$ & $0.3$ & $0.5$ \\
\midrule
1&trg only & 0.248 & 0.235 & 0.200 \\
2&src + trg & 0.583 & 0.567 & 0.501 \\
3&src w/ PIs + trg & 0.625 & 0.610 & 0.533 \\
4&src + trg w/ PIs & 0.626 & 0.615 & 0.559 \\
5&src w/ PIs + trg w/ PIs & 0.642 & 0.629 & 0.562 \\
\midrule
6&src w/ PIs + trg & 0.625 & 0.610 & 0.533  \\
7&src w/ PIs + trg w/ 1 PI & 0.632 & 0.615 & 0.549 \\
8&src w/ PIs + trg w/ 2 PIs & 0.636 & 0.624 & 0.557 \\
9&src w/ PIs + trg w/ all PIs & 0.642 & 0.629 & 0.562 \\
\bottomrule
\end{tabular}
\end{table}



\noindent\textbf{Comparison with Baseline Methods.} Table~\ref{ntu_baselines} shows the comparison with the baseline models that uses privileged information (see Section~\ref{sec:dataset_setups}). The fact that our method outperforms the representative baseline methods validates the efficacy of the graph distillation method.

\noindent\textbf{Efficacy of Distillation Graph.} Table \ref{different_graphs} compares the performance of predefined and learned distillation graphs. The proposed learned graph is compared with an empty graph (no distillation), a uniform graph of equal weights, and a prior graph computed using the cross-validation accuracy of each modality. Results show that the learned graph structure with modality-specific prior and example-specific information obtains the best results on both datasets.



\begin{figure}[t]
% \mpage{0.48}{\small{(a) Falling}}\hfill
% \mpage{0.48}{\small{(b) Brushing teeth}}\hfill
\begin{center}
\includegraphics[width=0.8\linewidth]{graph}
\end{center}
\caption{\textbf{The visualization of graph distillation on NTU RGB+D.} The numbers indicate the ranks of the distillation weights, with 1 being the largest and 5 being the smallest. (a) Class ``falling'': Our graph assigns more weight to optical flow because optical flow captures the motion information. (b) Class ``brushing teeth'': In this case, motion is negligible, and our graph assigns the smallest weight to it. Instead, it assigns the largest weight to skeleton data.}
\label{fig:graph}
\end{figure}



\noindent\textbf{Efficacy of Privileged Information.} Table~\ref{pku_mmd_baselines} compares our distillation and transfer under different training settings. The input at test time is a single depth modality. By comparing row 2 and 3 in Table~\ref{pku_mmd_baselines}, we see that when transferring the visual encoder to the target domain, the one pre-trained with privileged information in the source domain performs better than its counterpart. As discussed in Section~\ref{sec:collective}, graph distillation can also be applied to the target domain. By comparing row 3 and 5 (or row 2 and 4) of Table~\ref{pku_mmd_baselines}, we see that performance gain is achieved by applying the graph distillation in the target domain. The results show that our graph distillation can capture useful information from multiple modalities in both the source and target domain.

\noindent\textbf{Efficacy of Having More Modalities.} The last three rows of Table \ref{pku_mmd_baselines} show that performance gain is achieved by increasing the number of modalities used as the privileged information. Note that the test modality is depth, the first privileged modality is RGB, and the second privileged modality is the skeleton feature JJD. The results also suggest that these modalities provide each other complementary information during the graph distillation.



\subsection{Graph Distillation on UCF-101}

In this section, we consider graph edge distillation, a special case of graph distillation on UCF-101~\cite{soomro2012ucf101} in which only two modalities (RGB and optical flow) are available. Table~\ref{ucf101} shows the action classification results on UCF-101 using the two-stream architecture proposed in~\cite{two_stream_simonyan}. The optical flow modality performs significantly better than RGB when training from scratch. This is consistent with previous findings that dense optical flow is able to achieve very good performance in spite of limited training data \cite{two_stream_simonyan}. To testify our method, we train a model on the RGB modality from scratch with distillation. Our distilled model performs much better than the model directly trained from scratch. Note that our distilled model outperforms the fine-tuning model that uses pretrained weights on ImageNet.

\begin{table}[ht]
\scriptsize
\centering
\caption{Action classification results on UCF101. For graph distillation model, we distill knowledge from the optical flow stream to the RGB stream.}
\label{ucf101}
\begin{tabular}{l@{\hskip 0.1in}c@{\hskip 0.2in}c@{\hskip 0.1in}c}
\toprule
Method & Test Modality & mAP & Diff. \\
\midrule
From scratch   & Flow & 0.803 & - \\
From scratch   & RGB & 0.484 & +0.000 \\
ImageNet pretrained & RGB & 0.728 & +0.244 \\
Graph distillation & RGB & \textbf{0.757} & \textbf{+0.273} \\
\bottomrule
\end{tabular}
\end{table}


%\documentclass[preprint,12pt]{elsarticle}
%\if0
\usepackage{amssymb}
\usepackage{mathtools}
%\usepackage[dvipdfmx]{graphicx}
\usepackage{cite}
\usepackage{graphicx}
\usepackage{bm}
\usepackage{here}
\usepackage[subrefformat=parens]{subcaption}
\fi
%\usepackage{amssymb}
\usepackage{amsmath}
\usepackage[dvipdfmx]{}
\usepackage[dvipdfmx]{color}
%\usepackage{cite}
%\usepackage{upgreek}
\usepackage{url}
%\usepackage[dvipdfmx]{hyperref}
%\usepackage{pxjahyper}
%\usepackage {hyperref}
\usepackage{graphicx}
\usepackage{bm}
\usepackage{here}
\usepackage{caption}
\usepackage[subrefformat=parens]{subcaption}
\captionsetup{compatibility=false}

%% The amsthm package provides extended theorem environments
%% \usepackage{amsthm}

%% The lineno packages adds line numbers. Start line numbering with
%% \begin{linenumbers}, end it with \end{linenumbers}. Or switch it on
%% for the whole article with \linenumbers after \end{frontmatter}.
%% \usepackage{lineno}

%% natbib.sty is loaded by default. However, natbib options can be
%% provided with \biboptions{...} command. Following options are
%% valid:

%%   round  -  round parentheses are used (default)
%%   square -  square brackets are used   [option]
%%   curly  -  curly braces are used      {option}
%%   angle  -  angle brackets are used    <option>
%%   semicolon  -  multiple citations separated by semi-colon
%%   colon  - same as semicolon, an earlier confusion
%%   comma  -  separated by comma
%%   numbers-  selects numerical citations
%%   super  -  numerical citations as superscripts
%%   sort   -  sorts multiple citations according to order in ref. list
%%   sort&compress   -  like sort, but also compresses numerical citations
%%   compress - compresses without sorting
%%
%% \biboptions{comma,round}

% \biboptions{}

%% This list environment is used for the references in the
%% Program Summary
%%
\newcounter{bla}
\newenvironment{refnummer}{%
\list{[\arabic{bla}]}%
{\usecounter{bla}%
 \setlength{\itemindent}{0pt}%
 \setlength{\topsep}{0pt}%
 \setlength{\itemsep}{0pt}%
 \setlength{\labelsep}{2pt}%
 \setlength{\listparindent}{0pt}%
 \settowidth{\labelwidth}{[9]}%
 \setlength{\leftmargin}{\labelwidth}%
 \addtolength{\leftmargin}{\labelsep}%
 \setlength{\rightmargin}{0pt}}}
 {\endlist}
\begin{document}

\section{Testing the program O-SUKI-N 3D}
The several tests are shown below to present the target fuel implosion dynamics. In the example cases, the HIBs and the target fuel have the following common parameters, which are the same values employed in Ref. \cite{CPC-O-SUKI}: the beam radius at the entrance of a reactor chamber $R_{en}$ = 35 mm, the beam particle density distribution is in the Gaussian profile and all projectile Pb ions have 8 GeV. The target is a multilayered pellet, in which the pellet outer radius is 4 mm, a Pb layer thickness is 0.029 mm, the Al thickness is 0.460 mm, and the DT thickness is 0.083 mm; the Pb, Al and DT layers have the radial mesh numbers of 4, 46 and 30 in these example cases, respectively, and the total mesh number in the theta direction is 90. The input beam pulse is shown in Fig. 12 in Ref. \cite{CPC-O-SUKI}. The beam radius is 3.8mm on the target surface. However, $R_b$ = 3.8mm changes at $\tau_{wb}$ to 3.7mm for the wobbling beam irradiation. Here $\tau_{wb}$ is the rotational period of the beam axis. The rotational frequency is 424MHz ($rotaionnumber$ = 11). 



%% INPUT PULSE  
%\begin{figure}[H]
%		\centering
%		\includegraphics[width=10cm]{images/pulse.eps}
%		\caption{An example for the input beam pulse.}\label{pulse}
%\end{figure}



First the 3D Langrange code was run without the OK3 illumination code. This is the case for $OK\_Switch=10$, and we added the artificial non-uniformity $Y_3^2$ (the spherical harmonics) with the amplitude of $30.0\%$. In Fig. \ref{NoOK3_23_Ti} the ion temperature distribution is shown at $t$=35ns, and in Fig. \ref{NoOK3_23_rho} the mass density distribution is presented at $t$=35ns. The target shape is largely distorted due to the non-uniformity of the HIBs deposition energy distribution.  


%% LAGRANGE CASE WITHOUT OK3
\begin{figure}[H]
		\centering
		\includegraphics[width=10cm]{images/NoOK3_Non23_30_35ns_Ti.eps}
		\caption{Ion temperature in the 3D Lagrange code without OK3 code at $t$=35ns. The non-uniformity distiribution is $Y_3^2$ with the amplitude of $30\%$.}\label{NoOK3_23_Ti}
\end{figure}
\begin{figure}[H]
		\centering
		\includegraphics[width=10cm]{images/NoOK3_Non23_30_35ns_rho.eps}
		\caption{Mass density in the 3D Lagrange code without OK3 code at $t$=35ns. The non-uniformity distriution is $Y_3^2$ with the amplitude of $30\%$.}\label{NoOK3_23_rho}
\end{figure}


We also performed run-through simulation tests. In the example cases, the OK3 code was coupled with the run-through simulations. The implosion data were obtained by the Lagrange code, and the data just before the void closure time were transferred to the Euler code through the data Conversion code. Two cases are computed for the target fuel implosion dynamics with the spiral wobbling or without the oscillating HIBs. These examples are the run-through simulations with the OK3 illumination code ($OK\_Switch = 1$). The input beam pulse, employed in the run-through tests, is shown in Fig. \ref {Beam}. This beam input energy is 5.4MJ. We show the $r-t$ diagram for the case without the HIBs wobbling in Fig. \ref{rt}. The Lagrange-code test results stored in the output directory are visualized in Figs. \ref {fusion_Ti} for the target ion temperature ($T_i$) distributions at $t$ = 0.0, 15.0, 20.0 and 22.5 ns for the case with the HIBs wobbling behavior.  The RMS non-uniformity results are shown in Figs. \ref{fusion_RMS} (a) for DT layer's Ion temperature($T_i$), (b) for DT layer's Mass density($\rho$), (c) for Al layer's Ion temperature($T_i$) and (d) for Al layer's Mass density($\rho$). 
%
When the HIBs have the wobbling motion during the implosion with the wobbling frequency of 424MHz, the radius acceleration distributions are shown in Figs. \ref{Vr_tp} (a) in the $\theta$ direction and (b) in the $\phi$ direction at $t=6.25t_w=11.2ns$ (solid lines) and at $t=6.75t_w=12.2ns$ (dotted lines). Here $t_w$ shows the one rotation time. Figures \ref{Vr_tp} present that the non-uniformity phase of the implosion acceleration is controlled externally by the HIBs wobbling behavior \cite{CPC-O-SUKI, RSato2}.  
%

\begin{figure}[H]
		\centering
		\includegraphics[width=7.5cm]{images/Beam.eps}
		\caption{Input beam pulse shape used in the example run-through tests.}\label{Beam}
\end{figure}
\begin{figure}[H]
		\centering
		\includegraphics[width=8cm]{images/YesWob_SLC.eps}
		\caption{The $r-t$ diagram for the implosion with the HIBs wobbling illumination. The black line area shows the Pb layer, the blue line area Al and the red line area is DT.}\label{rt}
\end{figure}
\begin{figure}[H]
		\centering
		\includegraphics[width=6.5cm]{images/YesWob_Ti_0ns.eps}
		\includegraphics[width=6.5cm]{images/YesWob_Ti_15ns.eps}\\
		\includegraphics[width=6.5cm]{images/YesWob_Ti_20ns.eps}
		\includegraphics[width=6.5cm]{images/YesWob_Ti_225ns.eps}\\
		\caption{Ion temperature distributions in the example run-through test with the HIBs wobbling illumination at (a) $t$=0.0ns, (b) 15.0ns, (c) 20.0ns and (d) 22.5ns.}\label{fusion_Ti}
\end{figure}
\begin{figure}[H]
		\centering
		\includegraphics[width=6.5cm]{images/FusionRMS_DTTi.eps}
		\includegraphics[width=6.5cm]{images/FusionRMS_DTrho.eps}\\
		\includegraphics[width=6.5cm]{images/FusionRMS_AlTi.eps}
		\includegraphics[width=6.5cm]{images/FusionRMS_Alrho.eps}\\
		\caption{RMS non-uniformity histories of (a) the DT ion temperature, (b) the DT mass density, (c) the Al ion temperature and (d) the Al mass density for the cases with the wobbling HIBs (solid lines) and without the wobbling HIBs (dotted lines).}\label{fusion_RMS}
\end{figure}
%
\begin{figure}[H]
		\centering
		\includegraphics[width=6.5cm]{images/theta-Vr.eps}
		\includegraphics[width=6.5cm]{images/phi-Vr.eps}\\
		\caption{Radial acceleration distributions in (a) $\theta$ and (b) $\phi$. The solid lines show the acceleration ditributions at $t=6.25t_w=11.3ns$, and the dotted lines at $t=6.75t_w=12.2ns$.}\label{Vr_tp}
\end{figure}
%

After the Lagrange code computation, the implosion data are converted and transferred to the Euler code. Figures \ref{Ti_EuWobblIgnited} show the ion temperature distributions by the Euler code. Figures \ref{Ti_EuWobblIgnited} show that the DT fuel is ignited and the gain obtained is about 17.5 in this example case. For a realistic HIF reactor design, the implosion parameters should be further optimized to obtain a sufficient gain, which should be larger than 30$\sim$40 in HIF \cite{CPC-O-SUKI, Kawata1, Kawata2, RSato2}. 

\begin{figure}[H]
		\centering
		\includegraphics[width=13cm]{images/EuWobblIgnited.eps}
		\caption{Ion temperature distributions (a) at $t=$24.88ns, (b) at 28.44ns and at 29.21ns.}\label{Ti_EuWobblIgnited}
\end{figure}

\if0
In Fig. \ref{NoOK3_03_Ti}, a non-uniform energy deposition of the HIBs illumination is introduced based on the spherical harmonics $Y_3^0$ with the amplitude of $3.0\%$ in the 3D Lagrange code. The implosion data was obtained by the Lagrange code, and the data just before the void closure time were transferred to the Euler code through the data Conversion code.  Figure \ref{Ti_Eu_Y03} shows the ion temperature distributions  by the Euler code at (a) at $t$=36.36ns, (b) 36.57ns, (c) 41.32ns and (d) 42.41ns. In this example case the DT fuel is not yet ignited due to the insufficient ion temperature. 

\begin{figure}[H]
		\centering
		\includegraphics[width=8.5cm]{images/NoOK3_Non03_03_35ns_Ti.eps}
		\caption{Ion temperature in the 3D Lagrange code without OK3 code at $t$=35ns. The non-uniformity distriution is $Y_3^0$ with the amplitude of $3\%$.}\label{NoOK3_03_Ti}
\end{figure}


%% TIME VS ION TEMPERATURE Euler Y03
\begin{figure}[H]
		\centering
		\includegraphics[width=6.5cm]{images/ion_Eu_Y03_36_36ns.eps}
		\includegraphics[width=6.5cm]{images/ion_Eu_Y03_36_57ns.eps} \\
		\includegraphics[width=6.5cm]{images/ion_Eu_Y03_41_32ns.eps}
		\includegraphics[width=6.5cm]{images/ion_Eu_Y03_42_41ns.eps} \\
		\caption{Ion temperature distributions under a non-uniform energy deposition based on the spherical harmonics $Y_0^3$ by the Euler code,  (a) at $t$=36.36ns, (b) 36.57ns, (c) 41.32ns and (d) 42.41ns.}\label{Ti_Eu_Y03}
\end{figure}
\fi


In order to check the accuracy of the 3D Euler code, we also performed the Euler code tests, using the initial conditions of the 2D Euler code. The initial conditions in the Euler code are the output of the Lagrangian code.  To this end, the 2D Euler initial conditions were converted into 3D. Therefore, the physical values are uniform in the $\phi$ direction. The Lagrangian test 2D results for the target ion temperature ($T_i$) and the mass density ($\rho$) distribution at $t$ = 29 ns are shown in Figs. 14 and 15 in Ref. \cite{CPC-O-SUKI} for the cases with and without the wobbling HIBs.  The 2D Eulerian test results for the fusion energy gain is shown in Fig. 16 in Ref. \cite{CPC-O-SUKI}.  In Fig. \ref{Ti_Eu_3d} we show the ion temperature distributions by the 3D Euler code. The wobbling HIBs are not used in this simulation. In this case the fuel is ignited at $t \sim $30.1ns. The histories of the fusion gain $G$ of the 2D code and the 3D code are shown in Fig. \ref{FusionGain_Eu}. The fusion gain was 52.5 by the 2D code and 57.6 by the 3D code. In addition, we also did another test for the wobbling HIBs (see Figs. 15 and 16 in Ref. \cite{CPC-O-SUKI}), and the fusion gain was 76.1 in 2D \cite{CPC-O-SUKI} and 67.4 in 3D. The results would confirm that the 3D Euler code reproduces the 2D results successfully for the ignition time and the fusion gain obtained. 


%% TIME VS ION TEMPERATURE Euler
\begin{figure}[H]
		\centering
		\includegraphics[width=6.5cm]{images/ion_Eu_30_42ns.eps}
		\includegraphics[width=6.5cm]{images/ion_Eu_30_53ns.eps} \\
		\includegraphics[width=6.5cm]{images/ion_Eu_32_35ns.eps}
		\includegraphics[width=6.5cm]{images/ion_Eu_32_58ns.eps} \\
		\caption{Ion temperature distributions by the 3D Euler code without the HIBs wobbling at (a) $t$=30.42ns, (b) 30.53ns, (c) 32.35ns and (d) 32.58ns}\label{Ti_Eu_3d}
\end{figure}


%% ENERGY GAIN Euler
\begin{figure}[H]
		\centering
		\includegraphics[width=11cm]{images/FusionGain_Eu.eps}
		\caption{Fusion energy gain curves for the cases with 3D code (a solid line) and with 2D code (a dotted line).}\label{FusionGain_Eu}
\end{figure}

We also simulated the double-cone ignition scheme\cite{Double-cone} using a 3D Euler code. The double-cone ignition scheme was proposed by Prof. Jie Zhang \cite{Double-cone}, and the two compressed DT clouds are created by the gold cones. The two DT spherical clouds collide each other like the impact fusion \cite{Winterberg}. In this example case, the compressed DT maximum density of the DT fuel is set to be $1.0\times 10^5$[kg/m$^3$] with the Gaussian spatial distribution. The DT ignition will be attained by an additional heating, which is not taken into consideration in this example. The ion, electron and radiation temperatures are 10[eV] initially in the Euler code. The radius of the fuel is 92[$\mu$m] and the mass was $0.1$[mg]. We set the colliding speed $w$ of the two DT fuel clouds to $3.0\times10^5$ [m/s]. The ion temperature distributions are shown in Fig. \ref{Double_cone_Ti}.


%% DOUBLE-CONE
\begin{figure}[H]
		\centering
		\includegraphics[width=6.5cm]{images/double_cone_0ns.eps}
		\includegraphics[width=6.5cm]{images/double_cone_15_06ns.eps} \\
		\includegraphics[width=6.5cm]{images/double_cone_29_80ns.eps}
		\includegraphics[width=6.5cm]{images/double_cone_46_78ns.eps} \\
		\caption{Ion temperature distributions for the Double-cone ignition scheme \cite{Double-cone} at (a) $t$=0.0ns, (b) 15.06ns, (c) 29.80ns and (d) 46.78ns.}\label{Double_cone_Ti}
\end{figure}

	
%\include{end}



\section{Summary}

We performed a series of galactic disk $N$-body simulations
to investigate the formation and dynamical evolution of spiral arm 
and bar structures in stellar disks which are embedded in live 
dark matter halos.
We adopted a range of initial conditions where the models have similar halo 
rotation curves, but different masses for the disk and bulge components, 
scale lengths, initial $Q$ values, and halo spin parameters.
The results indicate that the bar formation epoch increases exponentially 
as a function of the disk mass fraction with respect to the total mass at the 
reference radius (2.2 times the disk scale length), $f_{\rm d}$.
This relation is a consequence of swing amplification~\citep{1981seng.proc..111T},
which describes the amplification rate of the spiral arm when it transitions from 
leading arm to trailing arm because of the disk's differential rotation.
Swing amplification depends on the properties that characterize the disk, 
Toomre's $Q$, $X$, and $\Gamma$. The growth rate reaches its maximum
for $1<X<2$,  although the position of the peak slightly depends on $Q$ as well as on
$\Gamma$. We computed $X$ for 
$m=2$ ($X_2$), which corresponds to a bar or two-armed spiral, 
for each of our models and found that this value is related to the bar's
formation epoch.

The bar amplitude grows most efficiently when $1<X_2<2$. For models 
with $1<X_2<2$ the bar develops immediately after the start 
of the simulation. As $X_2$ increases beyond $X_2=2$, the growth rate
decreases exponentially. We find that the bar formation epoch increases
exponentially as $X_2$ increases beyond $X_2=2$, in other words
$f_{\rm d}$ decreases. The bar formation epoch exceeds a Hubble time
for $f_{\rm d}\lesssim 0.35$.

Apart from $X$, the growth rate is also influenced by $Q$ where
a larger $Q$ results in a slower growth. This indicates that the bar formation
occurs later for larger values of $Q$. 
Our simulations confirmed this and showed that for the bar ($m=2$) the growth rate
is predicted by swing amplification and becomes visible when it grows beyond a certain amplitude.

Toomre's swing amplification theory further predicts that
the number of spiral arms is related to the mass of the disk, with
massive disks having fewer spiral arms. In addition, larger $\Gamma$
predicts a smaller number of spiral arms.
We confirmed these relations in our simulations. 
The shear rate ($\Gamma$) also affects the pitch angle of spiral
  arms. We further confirmed that our result is consistent with previous
studies.

We found that the disk-to-total mass fraction ($f_{\rm d}$)
and the shear rate ($\Gamma$) are the most important parameters that determine the
morphology of disk galaxies. 
When juxtaposing our models with the Hubble sequence,
the fundamental subdivisions of (barred-)spiral galaxies with 
massive bulges and tightly wound-up spiral arms from S(B)a to S(B)c is 
also be observed as a sequence in our simulations. Where the models 
with either massive bulges or massive disks have more tightly
wound spiral arms. This is because having both a massive disk and bulge results in 
a larger $\Gamma$, i.e., more tightly wound spiral arms. 


Once the
bar is formed it starts to heat the outer parts of the disk.
From this point onwards, 
the self-gravitating spiral arms disappear.
This may be in part caused by the 
lack of gas in our simulations. 
After the bar grows, we no longer discern  
spiral arms in the outer regions of the disk. This could imply
that gas cooling and star formation are required in order to 
maintain spiral structures in barred spiral galaxies for over 
a Hubble time~\citep{1981ApJ...247...77S,1984ApJ...282...61S}.


Our simulations further indicate that non-barred grand-design spirals are
transient structures which immediately evolve into barred
galaxies. Swing amplification teaches us that a massive disk is
required to form two-armed spiral galaxies. This condition, at the
same time, satisfies the short formation time of the bar structure.
Non-barred grand-design spiral galaxies therefore must evolve into barred
galaxies.  We consider that isolated non-barred grand-design spiral galaxies 
are in the process of developing a bar.





\section*{Acknowledgments}

We are grateful to all our collaborators for sharing their insights into the
topics discussed here.
This work is supported in part by DFG and NSFC through funds provided to the
Sino-German CRC 110 ``Symmetries and the Emergence of Structure in QCD" (NSFC
Grant No.~11621131001, DFG Grant No.~TRR110), by NSFC (Grant Nos.11425525,
11521505 and 11647601), by the Thousand Talents Plan for Young Professionals, by
the CAS Key Research Program of Frontier Sciences (Grant No.~QYZDB-SSW-SYS013),
by the CAS President's International Fellowship Initiative (PIFI) (Grant
No.~2017VMA0025), and by the National Key Basic Research Program of China under
Contract No. 2015CB856700.


\bibliography{hmreview-fk}



\end{document}

