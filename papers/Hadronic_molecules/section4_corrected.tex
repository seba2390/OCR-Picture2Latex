
\section{Nonrelativistic effective field theories}
\label{sec:4}

All the candidates for hadron resonances, and in particular the candidates of
hadronic molecules, which are the focus of this review, were discovered via
their strong decays into other hadrons. Therefore, to understand these
structures requires  also a study of their decays. Because of the
nonperturbative nature of QCD at hadronic energy scales, a first-principle
calculation of the spectrum of hadronic resonances at the level of quarks and
gluons can only be done using lattice QCD.
Although there has been tremendous progress in lattice QCD, a reliable
calculation of the full hadronic resonance spectrum for physical quark masses is
still out of reach. In addition, even if such calculations were available, the
interpretation of the emerging spectra still requires
 additional theoretical analyses.

Only in the special case discussed in Sec.~\ref{sec:3}, {\sl i.e.} for shallow bound
states coupling in an $S$-wave to a nearby continuum channel comprised of two
stable or at least narrow hadrons, one finds a direct and physical
interpretation for the leading and nonanalytic contribution of the wave function
renormalization constant $Z$ as the (normalizable) probability to find the
continuum contribution in the physical state.
 Because of the closeness of the
threshold to the mass of the physical composite state, such systems are ideal
objects to apply the concept of effective field theories (EFTs), which makes use
of the separation of scales and which per definition include  a
cutoff~\cite{Lepage:1989hf}. Of particular relevance here are the
nonrelativistic EFTs~(NREFTs). Note that the
general principles underlying any EFT are formulated in Weinberg's
paper on phenomenological Lagrangians~\cite{Weinberg:1978kz}.

As mentioned in Sec.~\ref{sec:3}, hadronic molecules are located close to some
strongly coupled thresholds.
We denote the low-energy (low-momentum) scale characterizing such a system,
given by the binding energy (binding momentum) defined in Eq.~(\ref{eq:Ebdef})
(Eq.~(\ref{eq:gamdef})), generically by $Q$.
All other hadronic scales that we may collectively label as $\Lambda$ are thus
regarded as hard. This enables one to construct a perturbation theory in
$Q/\Lambda$, which for near-threshold states should be a small number.
As will become clear below, it depends on the system which scale is appropriate
for $\Lambda$. For example when investigating the $f_0(980)$ as a candidate for a
$\bar KK$ molecular state, the inverse range of forces, the natural candidate
for $\Lambda$, is given by the mass of the allowed lightest exchange meson, the
rho meson. A phenomenologically adequate value for the binding energy is 10~MeV.
It corresponds to a binding momentum of 70~MeV, and thus $Q/\Lambda \sim
1/10$ is a good expansion parameter.\footnote{The subtle interplay of scales in
molecular transitions is  is discussed in detail in Ref.~\cite{Hanhart:2007wa}
on the example of decays of the $f_0(980)$.} Furthermore, the closeness to
threshold also means that the constituent hadrons can be treated
nonrelativistically.

As discussed in the preceding section, the most interesting information about
the structure of a near-threshold state is contained in its coupling strength to
the threshold channel, which measures the probability for finding the two-body
bound state component in the physical state. This is consistent with the
intuition that a state is the more composite the larger its coupling to the
continuum. As shown in Eq.~\eqref{eq:residue}, for a bound state the coupling
reaches its maximal value, if the physical state is purely an $S$-wave bound
state, $\lambda^2=0$. Hence, it is important to extract the value of the
coupling constant for understanding the nature of near-threshold structures.
In addition, a large coupling implies the prominence of hadronic loops not only
in the formation of the state but also in transitions and decays.
In this section, we will discuss the NREFT formalism which is a natural
framework for studying the transitions involving hadronic molecules with a small
energy release. It can also be used to compute the universal long-distance part
of the production/decay processes of hadronic molecules, which will be discussed
in Sec.~\ref{sec:6}.

The analytic structure of the three-point scalar loop integral (including the
TS) will be discussed in Sec.~\ref{sec:4-3ploop}. The power counting rules
for the NREFT treating all intermediate particles on the same footing will be
detailed in Sec.~\ref{sec:nreft1}. We denote such a theory  as \nreft. When
one of the intermediate particles is much more off-shell than the others, it can
be integrated out from \nreft~and one gets another effective field theory, here
called \nreftii, which  was originally introduced as XEFT to study the
properties of the $X(3872)$. The XEFT and its relation to \nreft~will be
discussed in Secs.~\ref{sec:4-XEFT} and \ref{sec:4-nreft2xeft}. 
Sec.~\ref{sec:4-interactions} is devoted to a brief description of the
formation of hadronic molecules.

\begin{figure}[t!]
  \begin{center}
   \vglue2mm
   \includegraphics[width=0.65\linewidth]{./figures/trianglediag}
   \caption{ A triangle diagram illustrating the long-distance contribution to
the transition between two heavy particles $A$ and $B$ with the emission of a
light particle $C$. The two vertical dashed lines denote the two relevant cuts.
   \label{fig:triangle}}
  \end{center}
\end{figure}

The formation of hadronic molecules may be viewed as a result of
nonperturbative hadron-hadron interactions. It is therefore natural to ask if
there is also an impact of hadron loops on the properties of more regular
excited hadrons.  Indeed, for certain transitions the effective field theory
\nreft~mentioned above  predicts prominent loop effects.
As examples, we briefly discuss single-pion/eta transitions and hindered M1
transitions between heavy quarkonia in Sec.~\ref{sec:4-NREFT_ccbar}. It will
become clear that whether the hadron-loop effects are important for
properties of an excited hadron is process-dependent. In particular, the
location of an excited hadron close to a threshold is a necessary but not
a sufficient condition.



\subsection{Power counting schemes}
\label{sec:4-pc}


As demonstrated in Sec.~\ref{sec:3}, the decisive feature of molecular states as
compared to more compact structures is the prominence of a two-hadron cut.
In some decays the cuts induced by intermediate particles might also matter.
To illustrate this point, we start this section by a discussion of the analytic
structure of three-point loop functions. This will shed light on the NREFT
power counting as well.


\subsubsection{Analytic structure of the three-point loop integral}
\label{sec:4-3ploop}


If a hadronic molecule has at least one unstable constituent, it can decay
directly through the decays of that unstable particle when phase space allows.
It can also decay into another heavy particle with a mass of the same order by
emitting light particles such as pions or photons from its constituents. The
mechanism for a transition accompanied by the emission of a single light
particle is depicted in Fig.~\ref{fig:triangle}.
In the figure the two vertical dashed lines show the relevant branch cuts:
They correspond to the time slices at which the intermediate particles can go
onto their mass shells.

We denote the intermediate particles as 
$M_{1,2,3}$ with masses $m_{1,2,3}$, and the external particles as $A,B,C$ 
with masses $m_{A,B,C}$, as shown in Fig.~\ref{fig:triangle}.  If 
all intermediate particles are nonrelativistic we can formulate a power 
counting based on the velocities of the intermediate particles. 
%
Let us start from the scalar triangle loop integral
\begin{widetext}
\begin{eqnarray}
    I(q) &=& i\int\!\frac{d^4l}{(2\pi)^4}
\frac{1}{\left(l^2-m_1^2+i\epsilon\right) \left[(P-l)^2-m_2^2+i\epsilon\right]
    \left[(l-q)^2-m_3^2+i\epsilon\right]}
    \nonumber \\
    &\simeq& \frac{i}{N_m} \int\!\frac{dl^0 d^3\bm l}{(2\pi)^4}
\frac1{\left[l^0{-}T_1(|\bm l|)+i\epsilon\right] \left[P^0{-}l^0{-}
T_2(|\bm{l}|)+i\epsilon\right] \left[l^0{-}E_C{-}T_3(|\bm
l{-}\bm{q}|)+i\epsilon\right] },
    \label{eq:scalarI}
\end{eqnarray}
where $\epsilon=0^+$, $N_m=8m_1m_2m_3$, $T_i(p)=p^2/2m_i$ denotes the kinetic 
energy for a 
heavy meson
with mass $m_i$, and $E_C$ the energy of the particle $C$ in the rest frame 
of the initial particle $A$. The second line is obtained by treating all 
the intermediate states 
nonrelativistically in the rest frame of the initial particle. Performing the 
contour integration over $l^0$, one gets a
convergent integral over the three-momentum. Defining
$\mu_{ij}=m_im_j/(m_i+m_j)$, $b_{12} = m_1+m_2-m_A$ and 
$b_{23}=m_2+m_3+E_C-m_A$, one has
\begin{equation}
  I(q) \simeq \frac{4\mu_{12}\mu_{23} }{N_m} \int\! \frac{d^3\bm 
l}{(2\pi)^3}\left[
  \left(\bm{l}^{\,2} + c_1 -i\epsilon\right) \left(\bm{l}^{\,2} + c_2 - 
\frac{2\mu_{23}}{m_3} \bm{l}\cdot\bm{q} - i\epsilon
  \right) \right]^{-1},
  \label{eq:loopinter}
\end{equation}
where $c_1= 2\mu_{12}b_{12}$, and
$c_2=2\mu_{23}b_{23}+\left(\mu_{23}/m_3\right){ q}^2$ with $q\equiv|\bm q|$.
The two terms in the denominator of the integrand
 contain a unitary cut each, as indicated by the
vertical dashed lines in Fig.~\ref{fig:triangle}. The other
two-body cut crossing the lines of $M_1$ and $M_3$ corresponds to the case that
the particle $M_3$ is propagating back in time (we assume implicitly that
it is $M_1$ and not $M_2$ that can decay to $M_3$ and $C$ in near on-shell 
kinematics). This is a relativistic effect which is
neglected here. The intermediate particles $M_1$ 
and $M_2$ are on shell when $\bm l^{\,2} + c_1 = 0$; $M_2$ and $M_3$ (as well 
as $C$) are on shell for $\bm{l}^{\,2} + c_2 - 
{2\mu_{23}} \bm{l}\cdot\bm{q}/m_3 = 0$. Accordingly,
 $\sqrt{|c_1|}$ and $\sqrt{|c_2|}$ define  two
different momentum scales where the corresponding intermediate states go on
shell.
Their values depend on all of the masses involved and may be very different 
from each other. For the nonrelativistic approximation to hold both must be 
small compared
to $m_i (i=1,2,3)$.

The integral of Eq.~(\ref{eq:loopinter}) can be presented in closed 
form~\cite{Guo:2010ak,Mehen:2015efa}
\begin{eqnarray}
    I(q) &=& {\cal N} \frac{1}{\sqrt{a}} \left[
\arctan\left(\frac{c_2-c_1}{2\sqrt{a(c_1-i\epsilon)}}\right) -
\arctan\left(\frac{c_2-c_1-2a}{2\sqrt{a(c_2-a-i\epsilon)}}\right) \right],
    \label{eq:Iexp} \\
    &=& {\cal N} \frac{1}{\sqrt{a}} \left[
\arcsin\left(\frac{c_2-c_1}{\sqrt{(c_2-c_1)^2+4ac-i\epsilon}}\right)
    - 
\arcsin\left(\frac{c_2-c_1-2a}{\sqrt{(c_2-c_1)^2+4ac-i\epsilon}}\right)
\right].
    \label{eq:Iexp2}
\end{eqnarray}
\end{widetext}
where ${\cal N}=\mu_{12}\mu_{23}/(2\pi m_1m_2m_3)$, and
$a = \left(\mu_{23}/m_3\right)^2 { q}^2$.
Especially Eq.~(\ref{eq:Iexp2}) highlights the presence of a special singularity
at
\begin{equation}
  (c_2-c_1)^2+4ac_1 = 0 \, .
  \label{eq:nrtrising}
\end{equation}
When rewriting the inverse trigonometric functions in terms of 
logarithms, one finds that this is a logarithmic divergence.
The solution of this equation gives the leading Landau 
singularity~\cite{Landau:1959fi} (for early
and recent reviews, see~\cite{Eden:1966,Chang:1983,Aitchison:2015jxa})
for a triangle diagram, also called triangle singularity, evaluated 
in nonrelativistic kinematics~\cite{Guo:2014qra}. The singularity location is 
slightly 
shifted from that found by solving the relativistic Landau equation. A 
comparison for a specific example can be found in the appendix of 
Ref.~\cite{Guo:2014qra}.  

Being nonlinear in all of the involved masses, Eq.~\eqref{eq:nrtrising} as well
as the Landau equation allow for different solutions. However, a direct
evaluation of the loop integral reveals that only in a very restricted
kinematics, one of the solutions produces an observable effect, namely
 when this solution is located on the physical
boundary, {\sl i.e.}, the upper edge of the branch cut in the first Riemann
sheet or alternatively the lower edge of the branch cut in the
second~\cite{Schmid:1967ojm}, see Fig.~\ref{sheets}. In this case, the TS can
produce a narrow peak in the invariant mass distribution, which may even mimic a
resonance. This effect
 was already indicated in Sec.~\ref{sec:TS} and will be further  illustrated  in
 Sec.~\ref{sec:6-ts}.
We therefore discuss now under which circumstances the singularity appears on
the physical boundary. This case is contained in the Coleman--Norton
theorem~\cite{Coleman:1965xm} (for triangle diagrams see
Ref.~\cite{Bronzan:1964zz}). The physical picture becomes most transparent using
the simple triangle singularity equation derived in Ref.~\cite{Bayar:2016ftu}:
\begin{equation}
  q_{\rm on+} = q_{a-}  \, ,
  \label{eq:trianglesing}
\end{equation}
where $q_{\rm on+}$
is the center-of-mass (CM) momentum of particles $M_1$ and $M_2$ when they are
on shell, and $q_{a-}$ is the momentum of particle $M_2$ in the rest frame
of A when $M_2$ and $M_3$ are on shell (being on shell is necessary but not
sufficient to define $q_{a-}$ as will be discussed immediately).
One finds
\begin{eqnarray}
  q_{{\rm on}+} &=& \frac1{2 m_A} \sqrt{\lambda(m_A^2,m_1^2,m_2^2)}\,,
  \nonumber\\
  q_{a-} &=& \gamma \left( \beta \, E_2^* - p_2^* \right) , 
\label{eq:qon}
\end{eqnarray}
where
\begin{equation}
  E_2^* = \frac{m_{B}^2+m_2^2-m_3^2}{2 m_{B}},\quad
  p_2^* = \frac{\sqrt{\lambda(m_{B}^2,m_2^2,m_3^2)}}{2 m_{B}},
\end{equation}
are the energy and the magnitude of the three-momentum of particle $M_2$ in the 
rest frame of particle B, {\sl i.e.} the CM frame of the ($M_2$, $M_3$) 
system, respectively, $\beta=q/E_B$ is the magnitude of the velocity of 
particle B in the rest frame of A, and $\gamma= 1/{\sqrt{1-\beta^2}} =
{E_{B}}/{m_{B}}$ is the Lorentz boost factor. 
Eq.~\eqref{eq:trianglesing} is the condition for the 
amplitude $I(q)$ to have a TS on the physical 
boundary. Note that if particle $M_1$ can go on shell
simultaneously with $M_3$ and $C$, it must be 
unstable. Consequently, its width 
moves the logarithmic divergence into the complex 
plane and the physical amplitude becomes finite.

Let us consider the kinematical region where the momentum of particle $M_2$ 
is positive so that Eq.~\eqref{eq:trianglesing} can be satisfied: $p_2 =q_{a-} 
= \gamma (\beta\,E_2^*-p_2^*) >0$. Then  $p_3=\gamma (\beta\,E_3^*+p_2^*)$ 
(where $E_3^*$ is the
energy of particle $M_3$ in the rest frame of particle B), the momentum of 
particle $M_3$ in the rest frame
of the initial particle, is positive as well.
This means that particles $M_2$ and $M_3$ move in the same direction in that 
frame. The corresponding velocities are given by
\begin{eqnarray}
  \beta_2 = \beta\, \frac{E_2^*-p_2^*/\beta}{E_2^*-\beta\, p_2^*}\,,\quad
  \beta_3 = \beta\, \frac{E_3^*+p_2^*/\beta}{E_3^*+\beta\, p_2^*}\,,
\end{eqnarray}
respectively. It is easy to see that $p_2>0$ leads to 
\begin{equation}
  \beta_3>\beta>\beta_2\,,
\end{equation}
which means that particle $M_3$ moves faster than $M_2$ and in the same 
direction in
the rest frame of the initial particle A.
%
This, together with the requirement that all intermediate particles are on their
mass shells, gives
the condition for having a TS on the physical boundary.
This is in fact the Coleman--Norton theorem~\cite{Coleman:1965xm} applied to the
triangle diagram:
the singularity is on the physical boundary if and only if the diagram can be 
interpreted as a classical process in spacetime. For other discussions about
TSs using the Mandelstam variables, we refer to two recent
works~\cite{Szczepaniak:2015eza,Liu:2015taa} and references therein.

To finish this section, we point out again that the TS mechanism has been around
for more than half a century, but only in recent years has become a viable tool
in hadron physics phenomenology due to the data discussed in this review. In
fact, many of the calculations outlined in which the TS plays a dominant role
can be and often are done without  recourse to an EFT. Still, in a broader view
it can nicely be embedded in the framework outlined here. In any case, whenever
the TS can play a role, it has to be included.

\subsubsection{\nreft}
\label{sec:nreft1}


A key component for any EFT is the power counting in terms of some dimensionless
small quantity, which allows for a systematic expansion and an estimate for the
uncertainty of the calculation caused by the truncation of the series at some
finite order. The natural small quantity in nonrelativistic systems is the
velocity $v$ (measured in units of the speed of light) which is much smaller
than one by assumption.

As mentioned in Sec.~\ref{sec:4-3ploop},  triangle diagrams with all
three intermediate particles being nonrelativistic in fact have two momentum
scales given by $\sqrt{|c_1|}$ and $\sqrt{|c_2|}$. Accordingly, one may define 
$v_1 = \sqrt{|c_1|}/(2\mu_{12})$ and $v_2=\sqrt{|c_2-a|}/(2\mu_{23})$ for the 
velocities of the intermediate mesons. 

From the previous analysis, three-point loop diagrams have two kinds of
singularities: two-body threshold cusps and TSs.
The two-body threshold singularities are encoded in the two velocities defined
above. When the TS, with its location implicitly defined via
Eq.~\eqref{eq:nrtrising}, is not in the considered kinematic region, the loop
function of Eq.~\eqref{eq:Iexp2} can be expanded in a power series as
\begin{eqnarray}
 I(q) &=& \frac{\mathcal{N}}{\sqrt{a} } \Bigg[ \left(\frac{\pi}{2} - 
\frac{2\sqrt{ac_1}}{c_2-c_1} \right) - \left(\frac{\pi}{2} - 
\frac{2\sqrt{ac_2}}{c_2-c_1} \right)  \nonumber\\ 
&& + \order{ \frac{\left(4 a c_1\right)^{3/2}}{(c_2-c_1)^3} } \Bigg] 
\nonumber\\ 
&=& {\cal N} \frac{2}{\sqrt{c_2}+\sqrt{c_1}} + \ldots\,.
  \label{eq:simplifiedI}
\end{eqnarray}
When the masses of all three intermediate particles are similar, 
$m_i\sim m$, the LO term in the above equation may be written as 
\begin{eqnarray}
    I(q) \sim \frac{{\cal N}}{m} \frac{2}{v_1+v_2} \, .
  \label{eq:Ipc}
\end{eqnarray}
Thus the arithmetic mean of the two velocities characterizes the size of the 
triangle loop. It is therefore the relevant parameter to estimate the leading 
loop contribution 
for the transition of a heavy state into a light state and another heavy state.

The power counting in nonrelativistic velocities for a given loop diagram can be
obtained by applying the following rules: The three-momentum of the intermediate
nonrelativistic particles counts as $\order{v}$, the nonrelativistic energy
counts as $\order{v^2}$, and each nonrelativistic propagator is of
$\order{v^{-2}}$. Thus, $I(q)$ scales as $\order{ v^5/(v^2)^3 }=\order{v^{-1}}$.
Comparing with Eq.~\eqref{eq:Ipc}, one sees that the velocity in the power
counting should be understood as the average of $v_1$ and
$v_2$~\cite{Guo:2012tg}.
In addition to the parts discussed the amplitude for a given process can have
factors of the external momentum $q$. To be general we do not count the external
momentum $q$ in powers of $v$, but keep it explicitly.
This defines the power counting as detailed in
Refs.~\cite{Guo:2009wr,Guo:2010zk,Guo:2010ak}. We denote this theory as \nreft.

% -----------------------------------------
\begin{figure}[tb]
  \begin{center}
   \includegraphics[width=0.48\linewidth]{./figures/pc_YZpi}\hfill
   \includegraphics[width=0.48\linewidth]{./figures/pc_YXga}
   \caption{
   Comparison of the power counting rule for the scalar three-point loop 
integral, $1/v$, with the numerical result evaluated using
Eq.~\eqref{eq:Iexp}. The numerical result is normalized to $1/v$ at 
$m_A=4.22$~GeV. The involved masses are given in Eq.~\eqref{eq:masses}, and 
the mass for the final heavy particle takes the value of 3.886~GeV for (a) and 
3.872~GeV for (b).
   \label{fig:pc}}
  \end{center}
\end{figure}
%-----------------------------------------
In order to demonstrate how the power counting rules work, we compare in 
Fig.~\ref{fig:pc} the values of $1/v$ with $v=(v_1+v_2)/2$ and an explicit 
calculation of the loop function as given in Eq.~\eqref{eq:Iexp}. The curves 
are  normalized at $m_A=4.22$~GeV. The values used for the calculation are:
\begin{equation}
  m_1 = 2.420~\text{GeV},~~ m_2= 1.867~\text{GeV}, ~~
 m_3 = 2.009~\text{GeV}. 
\label{eq:masses}
\end{equation}
For the external light particle we take $m_C=0.140$~GeV for 
Fig.~\ref{fig:pc}~(a) and $m_C=0$ for Fig.~\ref{fig:pc}~(b). 
In addition, we take two values for $m_B$, 3.886~GeV and $3.872$~GeV, and the 
results are shown as (a) and (b), respectively. 
Then (a) and (b) correspond to the loop integrals in the amplitudes for the 
$\Y\to\Z\pi$ and $\Y\to\X\gamma$, respectively, which will be discussed later 
in Sec.~\ref{sec:6-long}.
Using 
Eq.~\eqref{eq:trianglesing} or \eqref{eq:nrtrising}, we find that for 
$m_B=3.886$~GeV, there is a TS at $m_A=4.288$~GeV, which is 
the reason for the sharp peak in the dashed line in Fig.~\ref{fig:pc}~(a).
Note that in
the plots the widths of the intermediate mesons were neglected. For 
$m_B=3.872$~GeV, which is smaller than $m_2+m_3$, and $m_C=0$, the TS
moves to the complex plane at $m_A=(4.301-i\,0.018)$~GeV with a clearly
visible effect on the line shape, see the dashed line in 
Fig.~\ref{fig:pc}~(b). 
One sees from the figure that the simple power counting rule of
Eq.~(\ref{eq:Ipc}) agrees remarkably well with the explicit
calculation except for energies very close to the TS.

% -----------------------------------------
\begin{figure}[tb]
  \begin{center}
   \includegraphics[width=0.6\linewidth]{./figures/selfenergy}\\
   \caption{A one-loop two-point self-energy diagram.
   \label{fig:selfenergy}}
  \end{center}
\end{figure}
%-----------------------------------------

In addition to counting the loop integral as discussed above, one also needs to
take into account the vertices in order to obtain a proper estimate for a given
loop amplitude.
To illustrate the method let us start from the simplest two-point self-energy
diagram shown in Fig.~\ref{fig:selfenergy}. We assume that the mass of the state
is close to the threshold of the internal particles that can therefore be
treated nonrelativistically. If the coupling is in an $S$-wave, then the loop
scales as $\order{v^5/(v^2)^2}=\order{v}$.\footnote{Here we only focus on the
velocity scaling and neglect the geometric factor of $1/(4\pi)$.} If the
coupling is in a $P$-wave, each vertex contributes an additional factor of $v$
and the loop scales as $\order{v^3}$. Of course, the real part of the loop
integral is divergent, and the resulting correction to the mass is
scale-dependent.
However, since the scale dependence can be formally absorbed into the bare mass
of the state this discussion is not of relevance here. Thus, we find that the
effect of the two-hadron continuum on the self-energy of heavy quarkonia is
parametrically suppressed, if the state is close to the threshold which implies
a small value of $v$, and that this suppression increases for increasing orbital
angular momentum of the two-hadron state.

Next we consider the one loop diagram for the decay process $\text{A}\to 
\text{B\,C}$, with A and B heavy and C  light, as depicted in 
Fig.~\ref{fig:triangle}. To be concrete, 
we assume that C couples to the intermediate states $M_1$ and $M_3$ in a
$P$-wave (such as the pion couples to the ground state heavy mesons).
This coupling structure leads to  a 
factor of  $q$ (in the rest frame of A). The generalization to 
other situations is easy. The power counting rules for a few typical cases are 
then
as follows:
\begin{itemize}
  \item[(1)] Both A and B couple to the intermediate states in an $S$-wave.
  As a result the final state particles $B$ and $C$ must be in a $P$-wave. 
Therefore the expression of  Eq.~\eqref{eq:Ipc} needs to be multiplied by $q$. 
Still, the $1/v$ enhancement factor quantifies the relative importance of the 
triangle diagram for the transition: the closer both A and B to the 
corresponding thresholds, the more important the intermediate states. On top of 
this may come an additional enhancement
driven, e.g., by large couplings characteristic for molecular states as derived 
in Sec.~\ref{sec:3}.

  \item[(2)] Either A or B couples to the  intermediate states in an 
$S$-wave with the other one in a $P$-wave. In this case, because there is 
only one 
possible linearly independent external momentum for two-body decays, the 
internal momentum at the $P$-wave vertex must be turned into an external 
momentum. The amplitude scales as $\order{q^2/(m^2v)}$. Since the decay 
should be in an $S$-wave in this case, we have introduced a factor $1/m^2$ to 
balance the dimension of the $q^2$ factor~\cite{Guo:2010ak} as in this case
the loop contribution needs to be compared to a constant tree-level 
contribution.
 
  \item[(3)] Both A and B couple to the intermediate states in a $P$-wave. Each 
$P$-wave vertex contributes a factor of the internal momentum. In the power 
counting of \nreft, the external momentum is kept explicitly. As a result, 
there are two possibilities for the scaling of the $P$-wave vertices: Each 
$P$-wave vertex scales either as $m\,v$ or as the external momentum $q$. More 
insights can be obtained if we take a closer look at the relevant tensor loop 
integral:
\begin{equation}
  I^{ij}(\bm q\,) =  i\int\!\frac{d^4l}{(2\pi)^4} l^il^j\times 
\left[\text{integrand of }I(q)\right].
\end{equation}
In the rest frame of the initial particle, it can be decomposed into an 
$S$-wave part and a $D$-wave part as
\begin{equation}
  I^{ij}(\bm q\,) = P_S^{ij} I_S(q) + P_D^{ij} I_D(q),
\end{equation}
where 
\begin{equation}
  P_S^{ij}=\frac{\delta^{ij}}{\sqrt{3} }\,,\quad  
  P_D^{ij}= \frac1{\sqrt{6}} \left( 3\frac{q^iq^j}{q^2} -\delta^{ij} \right), 
\end{equation}
are the $S$- and $D$-wave projectors, respectively, which satisfy 
$P_S^{ij}P_S^{ij} = 1$, $P_D^{ij}P_D^{ij} = 1$, and $P_S^{ij}P_D^{ij} = 0$. 
Then in the $S$-wave part $I_S(q)$, the internal momentum scales as 
$\order{v}$, and $I_S(q)\sim\order{v}$. In the $D$-wave part $I_D(q)$, the 
internal momentum turns external, and one gets $I_D(q)\sim\order{q^2/(m^2 
v)}$, which would have the same scaling as $I_S(q)$ if 
$q/m\sim v$.\footnote{Noticing that $P_S^{ij} l^il^j=l^2/\sqrt{3}$ and 
$P_D^{ij} 
l^il^j=2l^2 P_2(\cos\theta)/\sqrt{6}$ with $P_2(\cos\theta)$ the second 
Legendre polynomial, it can be shown that $I_S(q)$ is UV divergent while 
$I_D(q)$ is UV convergent~\cite{Albaladejo:2015dsa,Shen:2016tzq}. The power 
counting of the $D$-wave part was not discussed in Ref.~\cite{Guo:2010ak}.} For 
the 
decay amplitude, the factor of $q$ from the vertex coupling C to intermediate 
states needs to be taken into account additionally.
 
\end{itemize}


This nonrelativistic power counting scheme was proposed in
Ref.~\cite{Guo:2009wr} to study the coupled-channel effects of charm meson
loops in charmonium transitions, and studied in detail later~\cite{Guo:2010ak}.
Applications to transitions between two heavy quarkonium states can be found in
Refs.~\cite{Guo:2010zk,Guo:2010ca,Guo:2011dv,Mehen:2011tp,Guo:2012tg,
Guo:2014qra}, and to transitions involving one or two $XYZ$ states in
Refs.~\cite{Cleven:2011gp,Cleven:2013sq,Guo:2013nza,
Esposito:2014hsa,Mehen:2015efa,Huo:2015uka,Wu:2016dws,Abreu:2016xlr,
Chen:2016mjn}. In particular in Ref.~\cite{Cleven:2013sq} the implications of
items (1) and (2) are demonstrated.
It is shown that, while the transitions of the $Z_b$ states to $\Upsilon(nS)\pi$
potentially suffer from large higher order corrections, the transitions to 
$h_b(mP)\pi$ and $\chi_{bJ}(mP)\pi$ should be dominated by the triangle
topology. The near-threshold cross section for $e^+e^-\to D\bar D$ was studied
in~\cite{Chen:2012qq} using NREFT as well.


It is clear that the power counting can only be applied to processes where the
intermediate hadrons are nonrelativistic and especially close to their mass
shells. Otherwise the loop diagrams receive contributions from large momenta and
cannot be treated in a simple EFT including only the hadronic degrees of freedom
of A, B, C, $M_1$, $M_2$ and $M_3$.


\subsubsection{\nreftii~and XEFT}
\label{sec:4-XEFT}

Because the $\X$ is arguably the most important and interesting candidate for a
hadronic molecule, here we will discuss in some detail one NREFT
designed specifically for studying the properties of the $\X$. It is called XEFT
and was proposed in Ref.~\cite{Fleming:2007rp} following the
Kaplan--Savage--Wise approach to describe the nucleon-nucleon
system~\cite{Kaplan:1998tg,Kaplan:1998we}. It can be regarded as a special
realization of \nreftii. Similar effective theories can be constructed for other
possible hadronic molecules which are located very close to thresholds. For
instance, in the framework of a similar theory, the $Z_b(10610)$ and
$Z_b(10650)$ were studied in Refs.~\cite{Mehen:2011yh,Mehen:2013mva} and the
$Z_c(3900)$ in Ref.~\cite{Wilbring:2013cha}.

The XEFT assumes the $\X$ to be a hadronic molecule of $D^0\bar D^{*0}+c.c.$.
The tiny binding energy~\cite{Olive:2016xmw}
\begin{equation}
  B_{X}=M_{D^0} +M_{D^{*0}}-M_X=(0.00\pm0.18)~\text{MeV},
  \label{eq:Xbe}
\end{equation}  
implies that the long-distance part of the $\X$ wave function is universal
and is insensitive to the binding mechanism which takes place at a much shorter
distance.
%
The long-distance degrees of freedom are $D^0$, $D^{*0}$, $\bar
D^0$, $\bar D^{*0}$ and $\pi^0$. All of them are treated nonrelativistically.
%
For processes dominated by the long-distance scales such as the decays $\X\to
D^0\bar D^0\pi^0$ and $\X\to D^0\bar D^0\gamma$ which can occur via the decay of
the vector charm meson directly, the XEFT at LO can reproduce the results
from the effective range theory which makes use of the universal two-body wave
function of the $\X$ at asymptotically long 
distances~\cite{Voloshin:2003nt,Voloshin:2005rt}
\begin{equation}
  \psi_X(r) \propto \frac{e^{-\gamma_0 r}}{r},
\end{equation}
where the $\X$ is assumed to be below the $D^0\bar D^{*0}$ threshold, and
$\gamma_0=\sqrt{2\mu_0 B_{X}}\leq20$~MeV with $\mu_0$ the reduced mass of 
$D^0$
and $\bar D^{*0}$.
Yet, it has the merit of being improvable order by order by including local
operators and pion exchanges although unknown short-distance coefficients will 
be involved. For processes involving shorter-distance scales such as the decays 
of the $\X$ into a charmonium and light particles, the XEFT can still be used by
parameterizing the short-distance physics in terms of local operators employing
factorization theorems and the operator product 
expansion~\cite{Braaten:2005jj,Braaten:2006sy}. The XEFT can also be used even
if the $\X$ is a virtual state with a nonnormalizable wave
function~\cite{Hanhart:2007yq} or a resonance above threshold.

The power counting and the NLO corrections to the decay $\X\to D^0\bar D^{0}
\pi^0$ were studied in Ref.~\cite{Fleming:2007rp}. The XEFT was also used to
study the decays of the $\X$ to the $\chi_{cJ}$ with one and two
pions~\cite{Fleming:2008yn,Fleming:2011xa}, the radiative transitions
$\X\to\psi(2S)\gamma$, $\psi(4040)\to\X\gamma$~\cite{Mehen:2011ds} and
$\psi(4160)\to \X\gamma$~\cite{Margaryan:2013tta}, the scattering of an
ultrasoft pion~\cite{Braaten:2010mg} or $D$ and $D^*$~\cite{Canham:2009zq} off
the $\X$, and the quark mass dependence and finite volume corrections of the
$\X$ binding energy~\cite{Jansen:2013cba,Jansen:2015lha}.
The relation between the XEFT and the formalism of \nreft~was clarified in
Ref.~\cite{Mehen:2015efa}.
As an extension of the XEFT in Ref.~\cite{Alhakami:2015uea} a modified power
counting was suggested to take into account also an expansion in the ratio
between the pion mass and the charm meson masses. The need for such an
expansion is removed, however, as soon as Galilean invariance is imposed on the
interactions~\cite{Braaten:2015tga}.


In the following, we use the decay $\X\to D^0\bar D^0\pi^0$ as an example to
illustrate the power counting of the XEFT. The binding momentum
$\gamma_0\leq20$~MeV sets the long-distance momentum scale in this theory.
The typical momenta for the $D^0$ and $D^{*0}$ are of the order $p_D\sim
p_{D^*}\sim\gamma_0$. The pion kinetic energy is less than 7~MeV, and thus the
momentum for either an internal or external pion is also counted as
$p_\pi\sim\gamma_0$. Furthermore, the pion exchange introduces another small
scale $\mu=\sqrt{\Delta^2-M_{\pi^0}^2}\simeq44$~MeV,  with
$\Delta=M_{D^{*0}}-M_{D^0}$. Denoting all the small momentum scales by $Q$, we
have
\begin{equation}
  \{p_D, p_{D^*}, p_\pi, \mu, \gamma_0\} =\order{Q}.
  \label{eq:XEFTpc}
\end{equation}
Thus, the measure for one-loop integral is of $\order{Q^5}$, and each 
nonrelativistic propagator is of $\order{Q^{-2}}$.
All Feynman diagrams can then be assigned a power of $Q$. 


The XEFT keeps as the degrees of freedom only those modes with a very
low-momentum $\sim\gamma_0$.
The binding momentum for the $D^+D^{*-}+c.c.$ channel at the $\X$ mass is
$\gamma_c\simeq126$~MeV. It is treated as a hard scale, and the charged charm
mesons are integrated out from the XEFT.


Denoting the field annihilating the 
$D^0, \bar D^0$, $D^{*0}$ and $\bar D^{*0}$ by $D,\bar D$, $\bd$ and $\bdbar$, 
respectively, and taking the phase 
convention that $\X$ is $\left( D^0\bar D^{*0} +\bar D D^{*0} 
\right)/\sqrt{2}$, the relevant Lagrangian for the calculation up to the NLO is 
written as~\cite{Fleming:2007rp}
\begin{widetext}
\begin{eqnarray}
 {\cal L}_{\rm XEFT} 
  &=& 
 \sum_{\bm{\phi}=\bd,\bdbar } \bm{\phi}^{\dagger} \bigg(i\partial_0 + 
\frac{\bm{\nabla}^2}{2 M_{D^{*0}}
		    }\bigg)\bm{\phi} 
 +  \sum_{{\phi}=D,\bar D } D^\dagger \bigg(i\partial_0 + 
\frac{\bm{\nabla}^2}{2 M_{D^0} } \bigg) D 
 + \pi^\dagger \bigg(i\partial_0 + \frac{\bm{\nabla}^2}{2 M_{\pi^0}}
   + \delta\bigg) \pi
%%%
\nonumber \\
&&+ \left[  
\frac{g}{2 F_\pi} \frac{1}{\sqrt{ 2M_{\pi^0} } }
 \left( D \bd^\dagger \cdot \bm{\nabla}\pi  
   + \bar D^\dagger \bdbar \cdot \bm{\nabla}\pi^\dagger \right) + {\rm
 H.c.} \right]
\nonumber \\
 && 
- \,  
\frac{C_0}{2} \, \left(\bdbar D + \bd \bar D \right)^\dagger 
\cdot \left(\bdbar D + \bd \bar D \right) 
+  \left[   \frac{C_2 }{16} \, 
\left(\bdbar D + \bd \bar D \right)^\dagger 
\cdot \left(\bdbar (\overleftrightarrow \nabla)^2 D 
        + \bd (\overleftrightarrow \nabla)^2 \bar D \right) + {\rm H.c.} \right]
\nonumber \\
%%%
&&+ \left[  \frac{B_1 }{\sqrt{2}}\frac{1}{\sqrt{2 M_{\pi^0}}} \left(\bdbar D + 
\bd \bar D \right)^\dagger \cdot D \bar{D} \bm{\nabla} \pi + {\rm 
H.c.}\right] \nonumber\\
%% new terms %%
&& + \frac{C_\pi}{2 M_{\pi^0}} \left( D^\dagger \pi^\dagger D \pi + 
\bar D^\dagger 
\pi^\dagger \bar D \pi \right) + C_{0D} D\,^\dagger\bar{D}^\dagger D\bar D \,,
\label{eq:XEFTlag}
\end{eqnarray}
\end{widetext}
where $\delta=\Delta-M_{\pi^0}\simeq7$~MeV and
$F_\pi=92.2$~MeV is the pion decay constant. The first line 
contains the kinetic terms for the pseudoscalar and vector charm mesons as 
well as for the nonrelativistic pion, the second is for the axial coupling of 
the pion to charm mesons with $g\simeq0.6$ determined from the $D^{*}$ width, 
the third line contains the LO and NLO contact interaction terms, and the 
fourth line contains the terms for a  short-distance emission of a pion. The contact terms in 
the last line were not considered in Ref.~\cite{Fleming:2007rp}, but,
as is argued below, also contribute to $\X\to D^0\bar D^0\pi^0$ at NLO. In 
particular, the $C_{0D}$ term may have a significant impact on the line shapes
as will become clear in the  discussion below.

%-----------------------------------------
\begin{figure}[tb]
  \begin{center}
   \includegraphics[width=\linewidth]{./figures/XDDpi}\\
   \caption{LO (a) and NLO (b,\ldots, g) diagrams for the calculation of the 
$\X\to D^0\bar D^0\pi^0$ decay width. The circled cross denotes an insertion of 
the $\X$, the thin and thick solid lines represent the pseudoscalar and vector 
charm mesons, respectively, and the dashed lines denote the pions.
   \label{fig:XDDpi}}
  \end{center}
\end{figure}
%-----------------------------------------

The Feynman diagrams relevant for the calculation of the $\X\to D^0\bar
D^0\pi^0$ decay width up to NLO are shown in Fig.~\ref{fig:XDDpi}. Diagram (a)
contributes at LO, (b,c) and (e,f) are the NLO diagrams calculated in
Ref.~\cite{Fleming:2007rp}, and (d,g) are two new diagrams from the new terms in
the last line of the Lagrangian in Eq.~\eqref{eq:XEFTlag}. Here we only discuss
the power counting for each diagram and the contributions missing in the
original work~\cite{Fleming:2007rp}, and refer  to Ref.~\cite{Fleming:2007rp}
for details of the calculation. One essential point of the XEFT is that the
pion-exchange is treated perturbatively based on the observation that the
two-pion exchange contribution is suppressed relative to the one-pion exchange
by
\begin{equation}
  \frac{g^2\mu_0\mu }{8\pi F_\pi^2} \simeq \frac1{20} \cdots \frac1{10} .
\end{equation}
Then the $\X$ is generated through a resummation of the $D\bar D^*$ contact 
terms (the 
charge conjugated $\bar D D^*$ channel is always implied). The 
pole of the $\X$ is at $E=-B_{X}$, and thus at LO 
\begin{equation}
  1+C_0\Sigma_0(-B_{X})=0 \ ,
  \label{eq:Xpole}
\end{equation}
where 
\begin{eqnarray}
  \Sigma_0(E) &=& - \left( \frac{\Lambda_{\rm PDS}}{2\pi} \right)^{4-D}\! 
\int\! \frac{d^{D-1} l}{(2\pi)^{D-1} } \frac1{E-l^2/(2\mu_0)+i\epsilon} 
\nonumber\\
  &=& \frac{\mu_0}{2\pi} \left(\Lambda_{\rm PDS} -\sqrt{-2\mu_0 E 
-i\epsilon} \right)
\label{eq:Sigma}
\end{eqnarray}
is the two-point one-loop integral containing nonrelativistic $D^0$ and $\bar 
D^{*0}$ propagators in the power divergence subtraction (PDS) 
scheme~\cite{Kaplan:1998tg,Kaplan:1998we}, where $E$ is the energy defined 
relative to the threshold and $\Lambda_{\rm PDS}$ is the PDS scale. 
For Eq.~\eqref{eq:Xpole} to be renormalization 
group invariant, $C_0$ needs to absorb the scale dependence of the loop 
integral:
\begin{equation}
  C_0(\Lambda_{\rm PDS}) = \frac{2\pi}{\mu_0\left(\gamma_0 - \Lambda_{\rm PDS} 
\right) }.
\label{eq:C0PDS}
\end{equation}
Keeping only momentum modes of order $Q$, the power counting for the loop 
integral is  $\Sigma_0(E)=\order{Q^5/(Q^2)^2}=\order{Q}$. One sees that 
the scale-independent part of $C_0$, $\bar C_0 = [\Lambda_{\rm PDS} + 
1/C_0( \Lambda_{\rm PDS}) ]^{-1} = 2\pi/(\mu_0\gamma_0)$, indeed scales as 
$Q^{-1}$.

Now we consider the power counting of the decay amplitudes from the 
diagrams 
in Fig.~\ref{fig:XDDpi}. The decay rate can be obtained from these amplitudes 
taking into account properly the wave function renormalization 
$Z$~\cite{Fleming:2007rp} which accounts for the insertion of the $\X$ 
interpolating field shown as circled crosses in Fig.~\ref{fig:XDDpi}. Notice 
that for the calculation of the decay rate up to the NLO, one needs $Z$ up to 
NLO (LO) for the LO (NLO) amplitude. The amplitude from diagram (a) scales as 
$\order{Q/Q^2}=\order{Q^{-1}}$ since there is one nonrelativistic propagator 
and one $P$-wave vertex which gives a factor of $p_\pi\sim Q$. Both one-loop 
diagrams (b) and (c) have four nonrelativistic propagators and three $P$-wave 
vertices, and thus scale as $\order{Q^0}$, one order higher 
than the LO diagram (a). The coefficients $C_2$ and $B_1$ scale as 
$Q^{-2}$~\cite{Fleming:2007rp}. Noticing that there are two derivatives in the 
$C_2$ term and one derivative in the $B_1$ term in the Lagrangian, the 
amplitudes from diagrams (e) and (f) should be counted as $\order{Q^0}$ as well.

Let us discuss diagrams (d) and (g) which were missing in the original 
calculation in Ref.~\cite{Fleming:2007rp}. The $C_\pi$ contact term can be 
matched to the chiral Lagrangian for the interaction between heavy and light 
mesons~\cite{Burdman:1992gh,Wise:1992hn,Yan:1992gz,Guo:2008gp}. At LO of the 
chiral expansion the interaction between pions and pseudoscalar heavy 
mesons receives contributions from the Born term from the exchange of $D^*$, 
which constitutes a subdiagram to (b) and (c), and the Weinberg--Tomozawa term. 
It turns out that the amplitude for $D^0\pi^0\to D^0\pi^0$ vanishes at LO. At 
NLO of the chiral expansion, there are several operators, see 
Refs.~\cite{Guo:2008gp,Guo:2009ct}.  In particular, it is easy to see that the 
$h_0$ and $h_1$ terms therein are proportional to the light quark mass or 
equivalently to $M_\pi^2$. The Feynman rule for the $D^0\pi^0\to D^0\pi^0$ vertex 
from these two terms (using relativistic normalization for all the fields) is
\begin{equation}
  i\,\Amp_{h_0,h_1} = i \frac{2}{3} \left(6h_0+h_1\right) 
\frac{M_\pi^2}{F_\pi^2} .
\end{equation}
The value of $h_1$ is fixed to be 0.42 from the mass splitting between the 
$D_s$ and $D$ mesons, and the $1/N_c$ suppressed parameter $h_0\simeq0.01$ from 
fitting to the lattice data for the pion mass dependence of charm meson 
masses~\cite{Liu:2012zya}. One sees $\Amp_{h_0,h_1}\simeq 0.65$. Hence, by 
matching to the chiral Lagrangian, $C_\pi$ should scale as $Q^0$, which leads 
to the scaling of $\order{Q^0}$ for diagram (d). 

Diagram (g) involves a short-distance contact interaction between $D^0$ and
$\bar D^0$. If the vertex $C_{0D}$ scales as $Q^0$, then diagram (g)
$=\order{Q^0}$. However, the situation could be more complicated. From the HQSS
analysis of the $\X$ in Sec.~\ref{sec:3_HQSS}, the $\X$ as a $D\bar D^*$
hadronic molecule should have three spin partners in the strict heavy quark
limit. One of them has quantum numbers $J^{PC}=0^{++}$ and couples to $D\bar D$
and $D^*\bar D^*$. Therefore, there is the possibility that the $D\bar D$
interaction needs to be resummed to generate a near-threshold pole. In this
case, $C_{0D}$ needs to be promoted to be $\order{Q^{-1}}$, analogous to $C_0$.
Then diagram (g) appears at $\order{Q^{-1} }$ making it a LO contribution.
Clearly this can cause a large correction to the $\X\to D^0\bar D^0\pi^0$  decay
rate. This effect can be seen clearly in Fig.~\ref{fig:XDDpi_FSI}, which is the
result obtained in Ref.~\cite{Guo:2014hqa} using \nreft~in combination with the
framework to be discussed in Sec.~\ref{sec:4-interactions}. The unknown
parameter $C_{0a}$ in the figure parameterizes the isoscalar part of $D^\dag\bar
D^\dag D\bar D$ contact interaction, see Eq.~\eqref{C0++} below, playing a role
similar to $C_{0D}$ introduced in Eq.~\eqref{eq:XEFTlag}.

%-----------------------------------------
\begin{figure}[tb]
  \begin{center}
   \includegraphics[width=\linewidth]{./figures/width_ddpi_05GeV}\\
   \caption{Decay width of the $\X\to D^0\bar D^0\pi^0$ calculated in 
Ref.~\cite{Guo:2014hqa} taking into account the $D\bar D$ final state 
interaction in the framework of Lippmann--Schwinger equation regularized by a 
Gaussian form factor. Here the cutoff in the Gaussian regulator is taken to be 
$\Lambda=0.5$~GeV, and $C_{0a}$ is the unknown isoscalar part of the $D\bar D$ 
contact term. The gray and blue bands correspond to the uncertainty bands 
without and with the $D\bar D$ final state interaction, respectively. The 
vertical line denotes the $D^0\bar D^0$ threshold. Adapted from 
Ref.~\cite{Guo:2014hqa}. 
   \label{fig:XDDpi_FSI}}
  \end{center}
\end{figure}
%-----------------------------------------

Since $\X\to D^0\bar D^0\pi^0$ is an important process sensitive to 
the long-distance structure of the $\X$, it would be interesting to revisit it 
considering the missing diagrams in XEFT. In particular, it was found that the 
nonanalytic corrections from the pion-exchange diagrams (b) and (c) of Fig,~\ref{fig:XDDpi} only 
contribute to $\sim1\%$ of the decay rate~\cite{Fleming:2007rp}. Whether this 
remains true after considering diagram (d) remains to be seen.

It should be stressed that the role of nonperturbative pions on the $\X$ 
properties is studied in various
papers~\cite{Baru:2011rs,Baru:2013rta,Baru:2016iwj} which in many cases 
confirm the results of XEFT. However, also in these studies diagrams of the 
types shown in diagrams (d) and (g) of Fig.~\ref{fig:XDDpi} were not included.

\subsubsection{From \nreft~to XEFT}
\label{sec:4-nreft2xeft}

From the discussions above, we see that all momentum scales much larger than 
$\gamma_0\leq20$~MeV have been integrated out from the XEFT. This is different 
from \nreft, where all nonrelativistic modes are kept as effective degrees 
of freedom including those with a momentum of the order of a few hundreds of 
MeV. \nreft~when applied to the $\X$ can be regarded as the high-energy theory 
for the XEFT. The short-distance operators in XEFT at the scale of a few 
hundreds of MeV can be matched to \nreft. This is
discussed in some detail in Ref.~\cite{Mehen:2015efa} in the context of 
calculations of the reactions $\X\to\chi_{cJ}\pi^0$.

To show the relation between \nreft~and XEFT explicitly let us consider the case 
$c_2\gg c_1$.
The quantities $c_2$ and $c_1$  introduced in Eq.~\eqref{eq:loopinter} define
 the locations of the two-body cuts of the triangle diagram. In the low-momentum region $l\sim \sqrt{c_1}$, 
 the second factor in the integrand of Eq.~\eqref{eq:loopinter} can 
be expanded in powers of $l^2/c_2$ and one gets
\begin{eqnarray}
  I(q) &=& \frac{4\mu_{12}\mu_{23}}{N_m c_2} \int^\Lambda \frac{d^3
l}{(2\pi)^3}\frac1{{l}^{\,2} + c_1 -i\epsilon} \left[1+ \order{\frac{c_1}{c_2}} 
\right] \nonumber\\
&\simeq& \frac{\mu_{12}}{2\pi N_m  \left[b_{23} + q^2/(2 m_3) \right] }  \left(\Lambda_{\rm PDS} -\sqrt{c_1 {-}i\epsilon} \right)\!.
~~~~
\end{eqnarray}
The resulting momentum integral in the first line is divergent
 and  needs to be regularized. The 
natural UV cutoff of the new effective theory
is set by $\Lambda<\sqrt{c_2}$. We denote such a theory as 
\nreftii. It reduces to the XEFT when applied to the $\X$. In order to compare 
with the XEFT, in the second line of the above equation we evaluate the 
integral in the PDS scheme which is equivalent to the sharp cutoff 
regularization by letting $\Lambda_{\rm PDS}= 2 \Lambda/\pi$ and dropping the 
terms of $\order{1/\Lambda}$. For a detailed comparison of 
dimensional versus cutoff regularization we refer to
Ref.~\cite{Phillips:1998uy}.

For $m_{1}=M_{D^{*0}}$, $m_2=M_{D^0}$, $E_C=E_\pi$, and $m_A=M_X$, the second 
line of the above equation reduces to 
\begin{equation}
  -\frac1{N_m \left(E_\pi + \Delta_H \right)} \frac1{ 
C_{0}(\Lambda_{\rm PDS} ) },
\end{equation}
where $\Delta_H=M_{D^0}+m_3-M_X$, and the term $q^2/(2m_3)$ has been neglected. 
Terms of the above form appear in the XEFT amplitudes for transitions between 
the $\X$ and a charmonium with the emission of a light 
particle~\cite{Fleming:2008yn,Fleming:2011xa,Mehen:2011ds,Margaryan:2013tta}.

The different power countings of XEFT and \nreft~has various implications 
that we now illustrate by two examples:

   Since \nreft~keeps all nonrelativistic modes explicitly, the charged
  $D\bar D^*$ channel which has a momentum of $\gamma_c\simeq126$~MeV needs to
  be kept as soft degrees of freedom. On the contrary, the XEFT only keeps the
  ultrasoft neutral charm mesons dynamically and the charged ones are integrated
  out.
  It was pointed out in Ref.~\cite{Mehen:2015efa} that it is crucial to take
  into account the charged charm mesons for the calculation of the
  $\X\to\chi_{cJ}\pi^0$ decay rate in \nreft~because their contribution
cancels to a large extent the one
  from the neutral charm mesons as usual in isospin violating transitions
({\sl c.f.}
the discussion in Sec.~\ref{sec:isospinviol}).\footnote{The role of the
charged
 charm mesons for certain decays of the $\X$ was already stressed in
 \cite{Gamermann:2009fv}.}
  The situation for decays into an isoscalar pion pair, $\X\to\chi_{cJ}\pi\pi$,
  is different. We expect that the charged and neutral channels are still of
  similar order, but add up constructively.
  
  Furthermore, in the XEFT calculation for  $\X\to\chi_{cJ}\pi^0$, there appears a new,
  reaction specific
  short-distance operator, labeled by $C_{\chi,0}$ in
  Fig.~\ref{fig:Xchic1pi}~(c). 
  To estimate its size it is matched onto two contributions in heavy meson
  chiral perturbation theory in Refs.~\cite{Fleming:2008yn,Fleming:2011xa}. Those
  are given by the exchange of a charm meson, which is proportional to the
  $\chi_{cJ}H\bar H$ coupling constant $g_1$, and a contact term accompanied
  by a low energy constant $c_1$, shown as diagram (a)
  and (b) in Fig.~\ref{fig:Xchic1pi}, respectively. The final result in XEFT
  then depends on the unknown ratio $g_1/c_1$. In \nreft, however, the two
  contributions appear at different orders, since the amplitude from diagram (b) is
  suppressed by $v^2$ compared with that from diagram (a).
  
%-----------------------------------------
  \begin{figure}[tb]
    \begin{center}
     \includegraphics[width=\linewidth]{./figures/Xchic1pi}\\
     \caption{Diagrams for calculating the decay rate for the process
     $\X\to\chi_{c1}\pi^0$.
     The circled cross denotes an insertion of the $\X$, the thin and thick 
     solid lines represent the pseudoscalar and vector charm mesons, 
     respectively, the dashed lines present the pions, and the double lines
     correspond to the $\chi_{c1}$.
     \label{fig:Xchic1pi}}
    \end{center}
  \end{figure}
%-----------------------------------------



\subsection{Formation of hadronic molecules}
\label{sec:4-interactions}

While so far the focus was on transitions of molecular candidates, we now 
turn to their formation through two-hadron scattering. 
For illustration we focus in this chapter on the scattering of open-flavor heavy
mesons off their antiparticles in a framework of  
NREFT similar to the EFT for nucleon-nucleon 
interactions~\cite{Epelbaum:2008ga}. 
The example of the formation of $\Lambda(1405)$ from
similar dynamics is discussed in Sec.~\ref{sec:1405th}.
In this section we mainly discuss the method used in 
Refs.~\cite{Nieves:2011vw,Nieves:2012tt,Valderrama:2012jv,
HidalgoDuque:2012pq,Guo:2013sya,Guo:2013xga}. It is based on the 
Lippmann--Schwinger equation (LSE) regularized using a Gaussian vertex form 
factor. The coupled-channel LSE 
reads 
\begin{eqnarray}
      T_{ij}(E;\bm k',\bm k) &=& V_{ij}(\bm k', \bm k) \\
    &+&  \sum_n \int\! \frac{d^3l}{(2\pi)^3} 
\frac{ V_{in}(\bm{k}',\bm{l})\, T_{nj}(E;\bm l, \bm k)  }{  
E-l^2/(2\mu_{n}) - \Delta_{n1} + i\epsilon  }, \nonumber
 \label{eq:lse}
\end{eqnarray}
where $\mu_{n}$ is the reduced mass in the $n$-th channel, $E$ is the 
energy defined relative to threshold of the first channel, and $\Delta_{n1}$ is 
the difference between the $n^{\rm th}$ threshold and the first one. When the 
potential takes a separable form $V_{ij}(\bm k', 
\bm k)=\xi_{i}{(\bm k')}V_{ij}\varphi_{j}{(\bm k)}$, where the $V_{ij}$ are 
constants, the equation can be  
simplified greatly. In addition, for very near-threshold states one should 
expect a momentum 
expansion for the potential to converge fast and a dominance of $S$-waves. 
Both the separability as well as the absence of higher partial waves will be 
spoiled as soon as the one-pion exchange is included on the potential level;
this case will be discussed briefly later in this section.

With a UV regulator such as of the Gaussian form, see, 
e.g., Ref.~\cite{Epelbaum:2008ga}, 
\begin{equation}
  V_{ij}(\bm k',\bm k) =  e^{-\bm k^{\prime2}/\Lambda^2} V_{ij} e^{-\bm 
k^{2}/\Lambda^2} \ ,
\label{eq:potdef}
\end{equation}
the LSE can be solved straightforwardly.  
If the $T$-matrix has a near-threshold bound state pole, the effective coupling 
of this composite state to the constituents can be obtained by calculating the 
residue of the $T$-matrix element at the pole. For simplicity, we consider a 
single-channel problem with the LO contact term: $V(\bm k',\bm k) 
= C_0 e^{-\bm k^{\prime2}/\Lambda^2} e^{-\bm 
k^{2}/\Lambda^2}$. The nonrelativistic $T$-matrix element for the scattering of 
the two hadrons is then given by 
\begin{equation}
T_\text{NR}^{}(E) = \left[ C_0^{-1} + \Sigma_\text{NR}^{}(E) \right]^{-1}, 
\label{eq:T1c}
\end{equation}
where 
\begin{equation}
\Sigma_\text{NR}^{}(E)= \frac{\mu}{2\pi} \left[\frac{\Lambda}{\sqrt{2\pi}} 
- \sqrt{-2\mu E-i\epsilon} \right] +\order{ {\Lambda}^{-1} }
\label{eq:SigmaGaussian}
\end{equation}
is the nonrelativistic two-point scalar loop function 
defined in Eq.~\eqref{eq:Sigma} but evaluated with a Gaussian regulator. After 
renormalization by absorbing the cutoff dependence into $C_0$, we obtain
\begin{equation}
  T_\text{NR}^{}(E)=\frac{2\pi/\mu}{\gamma -\sqrt{-2\mu E-i\epsilon} } +\order{ 
{\Lambda}^{-1} } .
  \label{eq:TNR}
\end{equation}
The binding momentum $\gamma$ was defined in Eq.~(\ref{eq:gamdef}).
The 
effective coupling is obtained by taking the residue at the pole $E=-E_B$:
\begin{eqnarray}
  g_\text{NR}^2 &=& \lim_{E\to -E_B} (E+E_B) T_\text{NR}^{}(E) = \left[  
\Sigma_\text{NR}'(-E_B) \right]^{-1} \nonumber\\
  &=& \frac{2\pi\gamma}{\mu^2} .
  \label{eq:gNR}
\end{eqnarray}
It does not depend on $C_0$, and is scale independent up to terms suppressed by 
$1/\Lambda$.
Multiplying $g_\text{NR}^2$ by the factor $(8 m_1 m_2 M)$ to get the 
relativistic  
normalization, we recover the expression for $g_\text{eff}$ derived in 
Eq.~\eqref{eq:residue} for  
 $\lambda^2=0$. Thus we find that a potential of the
kind given in Eq.~(\ref{eq:potdef}) generates hadronic molecules.
Deviations of this result behavior can be induced, e.g., by momentum
dependent interactions (or terms of order $\gamma/\Lambda$). This
observation formed the basis for the generalization of the Weinberg 
compositeness criterion presented in 
Refs.~\cite{Aceti:2012dd,Hyodo:2011qc,Hyodo:2013nka,Sekihara:2014kya}.

To proceed we first need to say a few words about the scattering of heavy
mesons.
For infinitely heavy quarks the spin of the heavy quark decouples, and
accordingly in a reaction not only the total angular momentum is conserved but
also the spin of the heavy quark and thus the total angular momentum of the
light quark system as well. Therefore, a heavy-light quark system can be labeled
by the total angular momentum of the light quark system $j_\ell$. Accordingly
the ground state mesons $D$ and $D^*$  ($\bar B$ and $\bar B^*$) form a doublet
with $j^P_{\ell}=1/2^-$, where we on purpose deviate from the standard notation
$s_\ell^P$ to remind the reader that the light quark part can well be a lot more
complicated than just a single quark.
Candidates of the next doublets of excited states are $D^*_0(2400)$\footnote{The
$D\pi$ $S$-wave resonant structure is probably more complicated than a single
broad resonance, as demonstrated by a two-pole structure in
Ref.~\cite{Albaladejo:2016lbb}.} and $D_1(2430)$ (the corresponding $B$-mesons
are still to be found), characterized by $j^P_{\ell}=1/2^+$ and a width of about
300~MeV, and $D_1(2420)$ and $D_2^*(2460)$ ($\bar B_1(5721)$ and $\bar
B^*_2(5747)$) with $j^P_{\ell}=3/2^+$ and a width of about 30~MeV. Since the
states with $j^P_{\ell}=1/2^+$ are too broad to support hadronic
molecules~\cite{Filin:2010se,Guo:2011dd}, in what follows we  focus on the
scattering of the ground state mesons off their anti-particles as well as on
that of the $j^P_{\ell}=3/2^+$ mesons off the ground state ones with one of them
containing a heavy quark and the other a heavy anti-quark.

We start with the former system. To be concrete, we take the charm mesons. In
the particle basis, there are six $S$-wave meson pairs with given 
$J^{PC}$~\cite{Nieves:2012tt}:
\begin{eqnarray}
&&0^{++}:\quad\left\{D\bar{D}({^1S_0}),D^*\bar{D}^*({^1S_0})\right\},\nonumber\\
&&1^{+-}:\quad\left\{D\bar{D}^*({^3S_1},-),D^*\bar{D}^*({^3S_1})\right\},
\nonumber\\[-3mm]
\label{eq:basis}  \\[-3mm]
&&1^{++}:\quad\left\{D\bar{D}^*({^3S_1},+)\right\},\nonumber\\
&&2^{++}:\quad\left\{D^*\bar{D}^*({^5S_2})\right\},\nonumber
\end{eqnarray}
where the individual partial waves are labelled as $^{2S+1}L_J$, with $S$, $L$, 
and $J$ denoting the total spin, the angular momentum, and the total momentum 
of the two-meson system, respectively. We define the $C$-parity eigenstates as
\begin{equation}
D\bar{D}^*(\pm)=\frac{1}{\sqrt{2}}\left(D\bar{D}^*\pm D^*\bar{D}\right), \label{eq:DDstar}
\end{equation}
which comply with the convention\footnote{
Notice that a different convention for the $C$-parity operator was used in
Ref.~\cite{Nieves:2012tt}. As a consequence, the off-diagonal transitions
of $V_{\rm LO}^{(0{++})}$ in Ref.~\cite{Nieves:2012tt} have a different sign as 
compared to Eq.~(\ref{C0++}), see also
Sec.~VI~A in Ref.~\cite{Guo:2016bjq} for further
 details of our convention.}
 for the $C$-parity transformation $\hat{C}{\cal M}=\bar{\cal M}$.
%
Because of HQSS, the interaction at LO is independent of the heavy
quark spin, and thus can be described by the matrix elements  $\langle j_{1\, 
\ell}',
j_{2\, \ell}',{j_\ell}| \hat\Ham_I |  j_{1\, \ell}, j_{2\, \ell},{j_\ell} 
\rangle$ where
the light quark systems get coupled to a total light-quark angular momentum of 
the two-meson system, $j_\ell$.
 Thus, for the systems under
consideration, we have two independent terms for each isospin ($I=0$ or $1$):
  $\langle 1/2,1/2,0 | \hat\Ham_I | 1/2,1/2,0 \rangle$ and $
\langle 1/2,1/2,1 | \hat\Ham_I | 1/2,1/2,1 \rangle $.
This simple observation leads to the conclusion that in the strict heavy quark
limit the six pairs in Eq.~\eqref{eq:basis} are grouped into two
multiplets with $j_\ell=0$ and 1, respectively.
In the heavy quark limit, it is convenient to
use a basis of states characterized via ${j_\ell^{PC}}\otimes {s_{c\bar c}^{PC} 
}$, where $s_{c\bar
c}$ refers to the total spin of the $c$ and $\bar c$ pair.
For the case of $S$-wave interactions only, both ${j_\ell^{PC}}$ and ${s_{c\bar
c}^{PC}}$ can only be in $0^{-+}$ or $1^{--}$. Therefore, the spin multiplet 
with
${j_\ell=0}$ contains two states with quantum numbers:
\begin{equation}
  {0_\ell^{-+}}\otimes {0_{c\bar c}^{-+} } = 0^{++}, \qquad
    {0_\ell^{-+}}\otimes {1_{c\bar c}^{--} } = 1^{+-} ,
    \label{eq:sl0}
\end{equation}
and the spin multiplet for ${j_\ell=1}$ has the following four states:
\begin{equation}
  {1_\ell^{--}}\otimes {0_{c\bar c}^{-+} } = 1^{+-}, \qquad
    {1_\ell^{--}}\otimes {1_{c\bar c}^{--} } = 0^{++} \oplus
    {1^{++}} \oplus 2^{++} .
    \label{eq:sl1}
\end{equation}
It becomes clear that if the ${1^{++}}$ state $X(3872)$ is a $D\bar D^*$
molecule, then it is in the multiplet with $j_\ell=1$~\cite{Voloshin:2004mh}, 
and has three spin partners with $J^{PC}=0^{++}$, $2^{++}$ and
$1^{+-}$ in the strict heavy quark limit as pointed out in
Refs.~\cite{Hidalgo-Duque:2013pva,Baru:2016iwj}. Based on an
analogous reasoning it 
was suggested already earlier that  $Z_b(10610)$ and $Z_b(10650)$ might have four more 
isovector partners $W_{b0}^{(\prime)}$, $W^{}_{b1}$ and 
$W^{}_{b2}$~\cite{Bondar:2011ev,Voloshin:2011qa,Mehen:2011yh}. A detailed and
quantitative analysis of these $W_{bJ}$ states can be found in
Ref.~\cite{Baru:2017gwo}.

It is worthwhile to notice that the two $1^{+-}$ states are in different
multiplets with $j_\ell=0$ and $1$, respectively, and thus cannot be related to
each other via HQSS. However, the isovector $Z_b(10610)$ and $Z_b(10650)$ are
located with similar distances to the $B\bar B^*$ and $B^*\bar B^*$ thresholds,
respectively. Such an approximate degeneracy suggests that the isovector
interactions in the $j_\ell=0$ and $j_\ell=1$ sectors are approximately the
same, and the off-diagonal transition strength in the isovector channel between
the two meson pairs with $J^{PC}=1^{+-}$ in Eq.~\eqref{eq:basis} approximately
vanishes. A fit to the Belle data of the $Z_b$ line shapes with HQSS constraints
implemented also leads to nearly vanishing channel coupling~\cite{Guo:2016bjq}.
This points towards an additional ``light quark spin symmetry'' as proposed by
Voloshin very recently~\cite{Voloshin:2016cgm}. While a deeper understanding for
such a phenomenon is still missing, it seems to be realized in the charm sector as
well for the charged $Z_c(3900)$~\cite{Ablikim:2013mio,Liu:2013dau} and
$Z_c(4020)$~\cite{Ablikim:2013wzq} observed by the BESIII and Belle
collaborations.
Note that in Ref.~\cite{Valderrama:2012jv} 
it is argued that channel couplings are suppressed while in Ref.~\cite{Baru:2016iwj}
they were claimed to be important to keep a well defined spin symmetry limit.
We come back to this controversy briefly later in this section.

When the physical nondegenerate masses for the heavy mesons are used, one needs 
to switch to the basis in terms of physical states in Eq.~\eqref{eq:basis}.
 In this basis and for a given set of quantum numbers $\{JPC\}$, the LO EFT
 potentials $V^{(JPC)}_{\rm LO}$, which respect HQSS,
 read~\cite{AlFiky:2005jd,Nieves:2012tt,Valderrama:2012jv}
\begin{eqnarray}\label{C0++}
&&V_{\rm LO}^{(0{++})}=
\begin{pmatrix}
C_{0a} & -\sqrt{3}C_{0b} \\
-\sqrt{3}C_{0b} & C_{0a}-2C_{0b}
\end{pmatrix},
\label{Vct0++}\\
&&V_{\rm LO}^{(1{+-})}=
\begin{pmatrix}
C_{0a}-C_{0b} & 2C_{0b} \\
2C_{0b} & C_{0a}-C_{0b}
\end{pmatrix},
\label{Vct1+-}\\
&&V_{\rm LO}^{(1{++})}=C_{0a}+C_{0b} \label{eq:contact2-a},\label{Vct1++} \\
&&V_{\rm LO}^{(2{++})}=C_{0a}+C_{0b} \label{eq:contact2-b},\label{Vct2++}
\end{eqnarray}
where $C_{0a}$ and $C_{0b}$ are two independent low-energy constants.
Thus, since in the spin symmetry limit $D$ and $D^*$ are degenerate, implying
that $D\bar D^*$ and $D^*\bar D^*$ loops are equal, the above equality of the
potentials in the $1^{++}$ and $2^{++}$ channels immediately predicts equal
binding energies for the two states in this limit.
 
 
Once HQSS violation is introduced into the system by the use of the
physical masses, the
two-multiplet pattern gets changed, however, the close connection between the
$1^{++}$ and $2^{++}$ states persists.
An inclusion of the one-pion exchange necessitates an extension of the basis, 
since now also $D$-waves need to be included. In fact, HQSS is preserved
only if all allowed $D$-waves are kept in the system, even if the
masses of the open flavor states are still kept degenerate~\cite{Baru:2016iwj}.
The probably most striking effect of
the $D$-waves, once the $D^*$-$D$ mass difference is included, is
that now transitions of the $2^{++}$ $D^*\bar D^*$ $S$-wave state to the
$D\bar D$ and $D\bar D^*$ $D$-wave become possible. It allows for a width of
this state of up to several tens of MeV~\cite{Albaladejo:2015dsa,Baru:2016iwj}, 
which might be accompanied by a sizeable shift in mass.
In addition, spin symmetry relations might get modified via the
coupling of the molecular states with regular charmonia as discussed recently
in Ref.~\cite{Cincioglu:2016fkm}.
 
For near-threshold states it is natural to assume that the contact terms are
independent of the heavy quark mass --- phenomenologically they can be viewed as
parameterizing the exchange of light meson resonances. Then one can also predict
the heavy quark flavor partners of the $\X$. The heavy quark spin and flavor
partners of the $\X$ predicted in Ref.~\cite{Guo:2013sya} with $\Lambda=0.5$~GeV
are listed in Table~\ref{tab:predictions}.
 
%------------------------------------------------------------------------
\begin{table}[tb]
\caption{\label{tab:predictions} Predictions of the partners of the $\X$ for 
$\Lambda=0.5$~GeV in Ref.~\cite{Guo:2013sya}.
}
\begin{ruledtabular}
\begin{tabular}{l c c c }
        $J^{PC}$ & States & Thresholds (MeV) & Masses (MeV)
       \\\hline
       $1^{++}$ & $\frac1{\sqrt{2}}(D\bar D^*+D^*\bar D)$ &
       3875.87 & 3871.68 (input)  \\
                       $2^{++}$ & $D^*\bar D^*$ &
       4017.3  & $4012^{+4}_{-5}$  \\
       $1^{++}$ & $\frac1{\sqrt{2}}(B\bar B^*+B^*\bar B)$ &
       10604.4 & $10580^{+9}_{-8}$  \\
                       $2^{++}$ & $B^*\bar B^*$ &
       10650.2 & $10626^{+8}_{-9}$  \\
                       $2^{+}$ & $D^*B^*$ &
       7333.7 & $7322^{+6}_{-7}$  \\ 
   \end{tabular}
\end{ruledtabular}
\end{table}
%------------------------------------------------------------------------
The $Z_b(10610)$ can be related to the $Z_b(10650)$ when the off-diagonal 
interaction is neglected as discussed above.  Their hidden-charm partners
are found to be virtual states in this formalism~\cite{Guo:2013sya}, which may 
correspond to the $Z_c(3900)$ and $Z_c(4020)$. In fact, it is shown in 
Ref.~\cite{Albaladejo:2015lob} that the BES\-III data for the $Z_c(3900)$ in
both the $J/\psi\pi$~\cite{Ablikim:2013mio} and  $D\bar 
D^*$~\cite{Ablikim:2015swa} 
modes can be well fitted with either a resonance above the $D\bar D^*$
threshold or a virtual state below.


The number of the LO contact terms is larger for the interaction between a pair 
of $j_\ell=1/2$ and $j_\ell=3/2$ heavy and anti-heavy mesons.
For each isospin, 0 or 1, in the heavy quark limit, there are four independent 
interactions
denoted as $\langle
j_{1\, \ell},j_{2\, \ell},j_\ell|\hat\Ham_I |j_{1\, \ell}',j_{2\, \ell}',j_\ell
\rangle$, where now $j_\ell$ can take values 1 or 2
\begin{eqnarray}
  F_{Ij_\ell}^d &\equiv& \left\langle \frac12,\frac32,j_\ell \left|\hat\Ham_I 
\right|\frac12,\frac32,j_\ell \right\rangle , \nonumber\\
  F_{Ij_\ell}^c &\equiv& \left\langle \frac12,\frac32,j_\ell \left|\hat\Ham_I 
  \right|\frac32,\frac12,j_\ell \right\rangle .
  \label{eq:VHT}
\end{eqnarray}
The relevant combinations of these constants for a given heavy meson pair can 
be worked out by changing the basis by means of a unitary transformation
(see, e.g.,~\cite{Ohkoda:2012rj,Xiao:2013yca}):
\begin{eqnarray}
  &&| s_{1\,c},j_{1\, \ell},j_1; s_{2\,c},j_{2\, \ell},j_2;J\rangle \nonumber\\
  &=& \sum_{s_{c\bar c}, j_\ell }  
 \sqrt{ (2j_1+1)(2j_2+1) (2s_{c\bar c}+1)(2j_\ell+1) } \nonumber\\
 && \times 
 \begin{Bmatrix}
   s_{1\,c} & s_{2\,c} & s_{c\bar c} \\
   j_{1\, \ell} & j_{2\, \ell} & j_\ell \\
   j_1 & j_2 & J
 \end{Bmatrix}
 |s_{1\,c},s_{2\,c},s_{c\bar c};j_{1\, \ell},j_{2\, \ell},j_{\ell}; J\rangle ,
 ~~~
\end{eqnarray}
where $j_1$ and $j_2$ are the spins of the two heavy mesons, $J$ is the 
total angular momentum of the whole system, and
$s_{1\,c}$ and $s_{2\,c}$ are the spins of the heavy quark. 

Consider two mesons $A$ and $B$; each of them is not a $C$ parity eigenstate, but their linear combination can form $C$ parity eigenstates. With the phase convention specified below \eqref{eq:DDstar},
the $C=\pm$ eigenstates of a flavor-neutral system consisting of a pair of mesons are given by 
\begin{equation}
    |C=\pm \rangle=\frac{1}{\sqrt{2}}\left[A B \pm (-1)^{J-J_{A}-J_{B}} \bar B \bar A\right],
\end{equation}
where $J_A$ and $J_B$ are the spins of the mesons $A$ and $B$, and $J$ is the total spin of the two-body system.

Noticing that the
total spin of the heavy quark and anti-quark $s_{c\bar c}$ is conserved in the
heavy quark limit, and combining the meson pairs into eigenstates of 
$C$-parity, one can obtain the contact terms for the $S$-wave interaction 
between a pair of $j_\ell=\frac12$ and $\frac32$ heavy and anti-heavy mesons. 
The diagonal ones are listed in Table~\ref{tab:HTcontact}.
%-----------------------
\begin{table}
  \caption{ The diagonal contact terms for the $S$-wave interaction between a 
pair of $j_\ell^P=1/2^-$ and $3/2^+$ heavy and anti-heavy mesons.
\label{tab:HTcontact}
}
\begin{ruledtabular}
  \centering\begin{tabular}{L C C}
    J^{PC} & \text{Meson pairs} & \text{Contact terms}\\\hline
    1^{{--}} & \frac{1}{\sqrt{2}} \left(D\bar{D}_1-D_1\bar{D}\right) & \frac{1}{8} \left(-F_{I1}^c-5
F_{I2}^c+3 F_{I1}^d+5 F_{I2}^d\right) \\
& \frac{1}{\sqrt{2}}\left(D^*\bar{D}_1+D_1\bar{D}^*\right) & \frac{1}{16} \left(7 F_{I1}^c-5
F_{I2}^c+11 F_{I1}^d+5 F_{I2}^d\right) \\
& \frac{1}{\sqrt{2}}\left(D^*\bar{D}_2-D_2\bar{D}^*\right) & \frac{1}{16} \left(-5
F_{I1}^c-F_{I2}^c+15 F_{I1}^d+F_{I2}^d\right) \\[2mm]

0^{--} & \frac{1}{\sqrt{2}}\left(D^*\bar{D}_1-D_1\bar{D}^*\right) & F_{I1}^c+F_{I1}^d \\[2mm]

2^{--} & \frac{1}{\sqrt{2}}\left(D\bar{D}_2-D_2\bar{D}\right) & \frac{1}{8} \left(3
F_{I1}^c-F_{I2}^c+3 F_{I1}^d+5 F_{I2}^d\right) \\
& \frac{1}{\sqrt{2}}\left(D^*\bar{D}_1-D_1\bar{D}^*\right) & \frac{1}{16} \left(F_{I1}^c-3
F_{I2}^c+F_{I1}^d+15 F_{I2}^d\right) \\
& \frac{1}{\sqrt{2}}\left(D^*\bar{D}_2{+}D_2\bar{D}^*\right) & \frac{1}{16} \left(9 F_{I1}^c+5 F_{I2}^c+9
F_{I1}^d+7 F_{I2}^d\right)
\\[2mm]

3^{--} & \frac{1}{\sqrt{2}}\left(D^*\bar{D}_2-D_2\bar{D}^*\right) & F_{I2}^d-F_{I2}^c \\[2mm]

0^{-+} & \frac{1}{\sqrt{2}}\left(D^*\bar{D}_1+D_1\bar{D}^*\right) & F_{I1}^d-F_{I1}^c \\ [2mm]

1^{-+} & \frac{1}{\sqrt{2}} \left(D\bar{D}_1+D_1\bar{D}\right) & \frac{1}{8} \left[5
\left(F_{I2}^c+F_{I2}^d\right)+F_{I1}^c+3 F_{I1}^d\right] \\
& \frac{1}{\sqrt{2}}\left(D^*\bar{D}_1-D_1\bar{D}^*\right) & \frac{1}{16} \left[5
\left(F_{I2}^c+F_{I2}^d\right)-7 F_{I1}^c+11 F_{I1}^d\right] \\
& \frac{1}{\sqrt{2}}\left(D^*\bar{D}_2+D_2\bar{D}^*\right) & \frac{1}{16} \left(5 F_{I1}^c+F_{I2}^c+15
F_{I1}^d+F_{I2}^d\right) \\ [2mm]

2^{-+} & \frac{1}{\sqrt{2}}\left(D\bar{D}_2+D_2\bar{D}\right) & \frac{1}{8} \left(-3 F_{I1}^c+F_{I2}^c+3
F_{I1}^d+5 F_{I2}^d\right) \\
& \frac{1}{\sqrt{2}}\left(D^*\bar{D}_1+D_1\bar{D}^*\right) & \frac{1}{16} \left[3 \left(F_{I2}^c+5
F_{I2}^d\right)-F_{I1}^c+F_{I1}^d\right] \\
& \frac{1}{\sqrt{2}}\left(D^*\bar{D}_2{-}D_2\bar{D}^*\right) & \frac{1}{16} \left(-9 F_{I1}^c-5
F_{I2}^c+9 F_{I1}^d+7 F_{I2}^d\right) \\ [2mm]
3^{-+} & \frac{1}{\sqrt{2}}\left(D^*\bar{D}_2+D_2\bar{D}^*\right) & F_{I2}^c+F_{I2}^d \\
\end{tabular}
\end{ruledtabular}
\end{table}
%-----------------------
One sees that the linear combinations are different for all channels, and 
it is not as easy as in case of  the $\X$ to predict spin partners for the $\Y$
based on the assumption that it is predominantly a $D_1\bar D$ state. The
possibility of $S$-wave hadronic molecules with exotic quantum numbers $1^{-+}$
was discussed in~\cite{Wang:2014wga}. 
Here we also give the off-diagonal contact terms.
There are three channels in each of $1^{--}$, $2^{--}$, $1^{-+}$ and $2^{-+}$ sectors, 
we label them in each sector listed in Table~\ref{tab:HTcontact} from top to bottom as 1, 2, and 3. Then the off-diagonal contact terms for the $1^{--}$ sector are 
\begin{equation}
    \begin{aligned}
    V_{12} &= -\frac{1}{8 \sqrt{2}}\left[5\left(F_{I 2}^{c}+F_{I 1}^{d}-F_{I 2}^{d}\right)+F_{I 1}^{c}\right], \\
    V_{13}&= \frac{1}{8} \sqrt{\frac{5}{2}}\left(-3 F_{I 1}^{c}+F_{I 2}^{c}+F_{I 1}^{d}-F_{I 2}^{d}\right), \\
    V_{23}&= \frac{\sqrt{5}}{16} \left(5 F_{I 1}^{c}+F_{I 2}^{c}+F_{I 1}^{d}-F_{I 2}^{d}\right) . 
    \end{aligned}
\end{equation}
The off-diagonal contact terms for the $2^{--}$ sector are 
\begin{equation}
    \begin{aligned}
        V_{12} &= -\frac{1}{8} \sqrt{\frac{3}{2}}\left(F_{I 1}^{c}+5 F_{I 2}^{c}+F_{I 1}^{d}-F_{I 2}^{d}\right), \\
        V_{13} &= \frac{1}{8} \sqrt{\frac{3}{2}}\left(-3 F_{I 1}^{c}+ F_{I 2}^{c}-3 F_{I 1}^{d}+3 F_{I 2}^{d}\right),\\
        V_{23}& =\frac{3}{16}\left(F_{I 1}^{c}-3 F_{I 2}^{c}+F_{I 1}^{d}-F_{I 2}^{d}\right).
    \end{aligned}
\end{equation}
The off-diagonal contact terms for the $1^{-+}$ sector are 
\begin{equation}
\begin{aligned}
    V_{12}&= \frac{1}{8 \sqrt{2}}\left[5\left(F_{I 2}^{c}-F_{I 1}^{d}+F_{I 2}^{d}\right)+F_{I 1}^{c} \right], \\
    V_{13}&= \frac{1}{8} \sqrt{\frac{5}{2}}\left(3 F_{I 1}^{c}-F_{I 2}^{c}+F_{I 1}^{d}-F_{I 2}^{d}\right), \\
    V_{23}&= -\frac{1}{16} \sqrt{5}\left(5 F_{I 1}^{c}+F_{I 2}^{c}-F_{I 1}^{d}+F_{I 2}^{d}\right).
\end{aligned}
\end{equation}
The off-diagonal contact terms for the $2^{-+}$ sector are 
\begin{equation}
    \begin{aligned}
        V_{12} &= \frac{1}{8} \sqrt{\frac{3}{2}}\left(F_{I 1}^{c}+5 F_{I 2}^{c}-F_{I 1}^{d}+F_{I 2}^{d}\right), \\
        V_{13} &= \frac{1}{8} \sqrt{\frac{3}{2}}\left(3 F_{I 1}^{c}- F_{I 2}^{c}-3 F_{I 1}^{d}+3 F_{I 2}^{d}\right),\\
        V_{23} &= -\frac{3}{16} \left(F_{I 1}^{c}-3 F_{I 2}^{c}-F_{I 1}^{d}+F_{I 2}^{d}\right).
    \end{aligned}
\end{equation}

However, one non-trivial prediction for the spectrum of molecular states in the
heavy quarkonium spectrum becomes apparent immediately from the discussion
above: Since the most bound states appear in $S$-waves  the lightest negative
parity vector state can be formed only from $j_\ell^P=1/2^-$ and $3/2^+$
heavy and anti-heavy mesons.
Therefore the mass difference of $X(3872)$ as bound state of two ground state
$j_\ell^P=\frac12^-$ mesons ($D$ and $D^*$) and the lightest exotic vector state
$Y(4260)$ (388 MeV) should be of the order of the mass difference of the
lightest $3/2^+$ state and the $D^*$ (410 MeV). Clearly this prediction is
nicely realized in nature. Note that from this reasoning it also follows that
$if$ the $Y(4008)$ indeed were to exist it could not be a hadronic molecule. In
this context it is interesting to note that the most resent data from BESIII on
$e^+e^-\to J/\psi\pi\pi$~\cite{Ablikim:2016qzw} does not seem to show evidence
for the $Y(4008)$, {\sl c.f.} Fig.~\ref{fig:Y4260-BESIII}.



\subsection{Impact of hadron loops on regular quarkonia}
\label{sec:4-NREFT_ccbar}

In the previous sections we argued that meson loops play a prominent role in
both the formation and the decays of hadronic molecules. One may wonder if they
also have an impact on the properties of regular charmonia.
In this section we demonstrate that certain processes for regular hadrons,
largely well described by the quark model, can also be influenced by significant
meson loop effects, since reaction rates can receive an enhancement due to the
nearly on-shell intermediate heavy mesons.
The origin of this mechanism is that for most heavy quarkonium transitions
$M_{Q\bar Q}-2 M_{Q\bar q}\ll M_{Q\bar q}$, where $M_{Q\bar Q}$ and $M_{Q\bar
q}$ are the masses of the heavy quarkonium and an open-flavor heavy meson,
respectively. As a result, the intermediate heavy mesons are nonrelativistic
with a small velocity
\begin{equation}
    v\sim \sqrt{|{M_{Q\bar Q}}-2m_{Q\bar q}|/m_{Q\bar q}}\ll 1\,,
\end{equation}
and the meson loops in the transitions can be investigated 
using \nreft.
We will highlight this effect on two examples in what follows.\footnote{The
effects of meson loops in heavy quarkonium spectrum are investigated in,
e.g., Refs.~\cite{Eichten:1978tg,Eichten:1979ms,Ono:1983rd,Kalashnikova:2005ui,
Eichten:2005ga,Pennington:2007xr,Barnes:2007xu, Li:2009ad,Danilkin:2009hr,
Ortega:2010qq,Danilkin:2010cc,Liu:2011yp,Bali:2011rd,Zhou:2013ada,
Ferretti:2013faa,Ferretti:2013vua,Ferretti:2014xqa, Hammer:2016prh,Du:2016qcr,
Lu:2016mbb, Lu:2017hma,Zhou:2017dwj}, and in heavy quarkonium transitions in
Refs.~\cite{Ono:1985jt, Lipkin:1988tg,Moxhay:1988ri,Zhou:1990ik,
Li:2007xr,Meng:2007tk,Meng:2008bq,
Liu:2009dr,Zhang:2009kr,Zhang:2010zv,Wang:2011yh, Li:2011ssa,Wang:2012mf,
Guo:2012tj,Li:2013zcr,Cao:2016xqo}.}

We start with the hindered M1 transitions between two $P$-wave heavy quarkonia
with different radial excitations, such as the $h_c(2P)\to \gamma
\chi_{cJ}(1P)$. It was proposed in Ref.~\cite{Guo:2011dv,Guo:2016yxl} that such
transitions are very sensitive to meson-loop effects, and the pertinent partial
widths provide a way to extract the coupling constants between the $P$-wave
heavy quarkonia and heavy open flavor mesons.
% -------------------------------------------------------------------
\begin{figure}[t]
    \centering \includegraphics[width=\linewidth]{./figures/hinderedM1loops}
    \caption{Feynman diagrams for the coupled-channel effects for the hindered
    M1 transitions between heavy quarkonia. The one-loop contributions are given
by (a) and (b). (c) and (d) are two typical two-loop diagrams.
The double, solid, wavy and dashed lines represent heavy quarkonia, heavy
mesons, photons, and pion, respectively. Adapted from Ref.~\cite{Guo:2011dv}.
\label{fig:M1loops}}
\end{figure}
% -------------------------------------------------------------------
In quark models the amplitude for such a transition is proportional to the
overlap of the wave functions of the initial and final heavy quarkonia, which is
tiny and quite model-dependent due to the different radial excitations --- this
is why they are called ``hindered''. This suppression is avoided in the
coupled-channel mechanism of heavy-meson loops.
In this mechanism, the initial and final $P$-wave heavy quarkonia couple to the
ground state pseudoscalar and vector heavy mesons in an $S$-wave. A few diagrams
contributing to this mechanism are shown in Fig.~\ref{fig:M1loops}. In (a), the
photon is emitted via its magnetic coupling to intermediate heavy mesons. In
(b), since the $S$-wave vertices do not have any derivative at LO, the photon
couples in a gauge invariant way to one of the vertices in the two-point loop
diagram. (c) and (d) are two typical two-loop diagrams. From the power counting
rules discussed in Sec.~\ref{sec:nreft1}, Fig.~\ref{fig:M1loops}~(a) provides
the leading contribution, while (b) is of higher order in the velocity counting
because there is one less nonrelativistic propagator. Their amplitudes scale as
\begin{equation}
  \Amp_{(a)}  \sim \frac{E_\gamma}{m_Q v}, \qquad \Amp_{(b)}  \sim 
\frac{E_\gamma v}{m_Q}\,,
\label{eq:M1Aab}
\end{equation}
respectively,
where $E_\gamma$ is the photon energy, and the dependence on the 
coupling constants is dropped. The $1/m_Q$ suppression comes from the fact that 
the polarization of a heavy (anti-)quark needs to be flipped in the M1 
transitions. For the two-loop diagram in (c), the amplitude scales as
\begin{equation}
   \Amp_{\rm (c)} \sim \frac{(v^5)^2}{(v^2)^5}
\frac{g^2}{(4\pi)^2F_\pi^2}  \frac{E_\gamma}{m_Q} M_H^2 = \frac{E_\gamma}{m_Q}
\left(\frac{g M_H}{\Lambda_\chi}\right)^2,
\label{eq:M1Ac}
\end{equation}
where the factor $1/(4\pi)^2$ appears because there is one more loop and the
hadronic scale $\Lambda_\chi=4\pi F_\pi\sim 1$ GeV was introduced as the hard
scale for the chiral expansion. The factor of $M_H^2$ was introduced to match
dimensions of the above equations to those of Eqs.~\eqref{eq:M1Aab}. Diagram (d)
has the same scaling as (c). Since the axial coupling constant $g\simeq0.6$ for
the charm case as determined from the width of $D^*\to D\pi$, and about 0.5 for
bottom~\cite{Flynn:2015xna}, one has $g M_D/\Lambda_\chi\lesssim1$ and $g
M_B/\Lambda_\chi\simeq2$. The value for $v$ defined as $(v_1+v_2)/2$ is about
0.4 for the transitions from the 2P to 1P charmonium states~\cite{Guo:2011dv},
and ranges from 0.3 to 0.2 for the transitions between $1P,2P$ and $3P$
bottomonia~\cite{Guo:2016yxl}. Hence, the two-loop diagrams are suppressed in
the charm sector, while they are of the same order as (a) for the bottom sector.
Therefore, one can make predictions for the charmonium transitions by
calculating the loops corresponding to (a). The results depend on a product of
two unknown coupling constants of the $1P$ and $2P$ charmonia to the charm
mesons. Taking model values for them, the decay width of the
$\chi_{c2}(2P)\to\gamma h_c(1P)$ is of $\mathcal{O}(100~\text{keV})$, two orders
of magnitude larger than the quark model prediction,
1.3~keV~\cite{Barnes:2005pb}.
Although quantitative predictions cannot be made for the bottomonium
transitions, it is expected that once such transitions would be observed they
must be due to coupled-channel effects as the partial widths were predicted to
be in the range from sub-eV to eV level in a quark model calculation that does
not include meson-loop effects~\cite{Godfrey:2015dia}. It is suggested in
\cite{Guo:2011dv,Guo:2016yxl} that the coupled-channel effects can be checked by
comparing results from both fully dynamical and quenched lattice QCD which has
and has no coupled-channel effects, respectively. Recent developments in lattice
QCD calculations of radiative decays~\cite{Dudek:2006ej,Dudek:2009kk,
Shultz:2015pfa,Briceno:2016kkp,Agadjanov:2014kha,Feng:2014gba,Leskovec:2016lrm,
Meyer:2011um,Owen:2015fra} should be helpful in illuminating this issue.



There are other heavy quarkonium transitions driven mainly by the
coupled-channel effects. A detailed study on the transitions between two
charmonia ($S$- and $P$-wave) with the emission of a pion or eta can be found in
Ref.~\cite{Guo:2010ak}. It is found that whether the coupled-channel effects
play a sizable role depends on the process. This is a result of the power
counting analysis; see the itemized discussion in Sec.~\ref{sec:nreft1}.
In particular, the  single-pion/eta transitions between two $S$-wave and
$P$-wave charmonia receive important contribution from charm-meson loops.
Therefore, the long-standing suggestion that the  $\psi'\to J/\psi\eta/\pi^0$
transitions can be used to extract the light quark mass
ratio~\cite{Ioffe:1979rv} needs to be reexamined. In fact, if we assume that
the triangle charm meson-loop diagrams saturate the transitions, the resulting
prediction of $\mathcal{B}(\psi'\to J/\psi\pi^0)/\mathcal{B}(\psi'\to
J/\psi\eta)$ is consistent with the experimental data. These transitions were
revisited considering both the loop and tree diagrams in
Ref.~\cite{Mehen:2011tp}.
Again based on the same power counting rules it was argued that the transitions
$\Upsilon(4S)\to h_b\pi^0/\eta$ have only a small pollution from the
bottom-meson loops, and are dominated by short-distance contribution
proportional to the light quark mass difference~\cite{Guo:2010ca}. They could be
used for the extraction of light quark mass ratio. Furthermore, the prediction,
made before the discovery of the $h_b(1P)$, on the branching fraction of the
order of $10^{-3}$ for the decay $\Upsilon(4S)\to h_b\eta$ was verified by the
Belle measurement, $(2.18\pm0.11\pm0.18)\times10^{-3}$~\cite{Tamponi:2015xzb}.

Parameter-free predictions can be made for ratios of partial widths of decays
dominated by the coupled-channel effects of heavy mesons in the same spin
multiplet, since all the coupling constants will get canceled in the ratios.
Such predictions on the hindered M1 transitions between $P$-wave heavy quarkonia
can be found in Refs.~\cite{Guo:2011dv,Guo:2016yxl}.

In Ref.~\cite{Guo:2012tg}, it is pointed out that coupled-channel effects can
even introduce sizable and nonanalytic pion mass dependence in heavy quarkonium
systems which couple to open-flavor heavy meson pairs in an $S$-wave.

To summarize this subsection, we stress that whether meson-loop effects are
important for the properties of quarkonia or not does not only depend on the
proximity to the relevant threshold, it is  also depends on the particular
transition studied.



