%!TEX root = mainArchitGNN.tex

%%%%%%%%%%%%%%%%%%%%%%%%%%%%%%%
%%% SECTION : Conclusions   %%%
%%%%%%%%%%%%%%%%%%%%%%%%%%%%%%%

In this paper we proposed two architectures for extending convolutional neural networks to process graph signals. The selection graph neural network replaces the convolution operation with graph filtering by means of linear shift invariant graph filters. Pooling is reinterpreted as a neighborhood summarizing function that gathers the relevant regional information at a subset of nodes, followed by a downsampling. By keeping track of the location of these subsets of nodes in the original graph, convolutional layers can be further computed at deeper layers through the use of zero padding. In this way, the selection GNN respects the original topology that describes the data, while reducing the computational complexity at each layer. Furthermore, the resultant features at each layer can be appropriately analyzed in terms of the original graph (frequency analysis, local filtering).

The aggregation GNN collects, at a single node, diffused versions of the original signal. The resulting signal simultaneously possesses a regular temporal structure and includes all relevant information of the topology of the graph. Since the signal collected at this single node has a temporal structure, a regular CNN can be applied to it. In large scale networks, however, gathering all the information of the graph signal at a single node might be infeasible. In order to overcome this, we proposed a multinode variation of the aggregation GNN in which we use a subset of nodes to subsequently create meaningful features of increasing neighborhoods.

We have tested the proposed architectures in a source localization problem on both synthetic and real datasets, as well as for authorship attribution and the classification of articles of the \texttt{20NEWS} dataset. We considered three different ways of choosing nodes in each architecture, based on three existing sampling techniques (namely, by degree, and by leverage scores computed from experimentally designed sampling and spectral proxies). We compared the results with an existing graph coarsening GNN that employs multiscale hierarchical clustering for the pooling stage. We observe that the multinode aggregation GNN exhibits the best performance.

All in all, the proposed GNN architectures exploit the advances in graph signal processing to present novel constructions of deep learning that are able to handle network data represented as signals supported on graphs.
