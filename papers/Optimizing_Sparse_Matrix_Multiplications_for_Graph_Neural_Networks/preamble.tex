\usepackage{graphicx}
\usepackage{listings}
\usepackage{xspace}
\usepackage{algorithm}
% \usepackage{algorithmic}
\usepackage{multirow}
\usepackage{color}
\usepackage{colortbl}
\usepackage{tabularx}
\usepackage{multirow}
\usepackage{subfigure}
\usepackage{eqparbox}
%\usepackage{natbib}
\usepackage{booktabs}
\usepackage[english]{babel}
\usepackage{xcolor}
\usepackage{svg}
\usepackage{listings}
\usepackage{algpseudocode}
\usepackage{amsmath}
%\usepackage{natbib}

\newcolumntype{b}{X}
\newcolumntype{s}{>{\hsize=.45\hsize}X}

\renewcommand{\algorithmicrequire}{\textbf{Input:}}  % Use Input in the format of Algorithm
\renewcommand{\algorithmicensure}{\textbf{Output:}} % Use Output in the format of Algorithm

% For code writing setting - package: listings
\lstset{
    %columns=fixed,       
    numbers=left,  
    stepnumber=2,
    % visualize the code line number
    frame=none, %tb,                                     % do not show the background frame
    keywordstyle=\color[RGB]{40,40,255},                 % keywords color
    numberstyle=\footnotesize\color{darkgray},           % row style
    commentstyle=\it\color[RGB]{0,96,96},                % comment style
    stringstyle=\rmfamily\slshape\color[RGB]{128,0,0},   % string style
    showstringspaces=false,                              % Do not show the space within the string
    language=python,                                        % language setting: C, C++, bash, Fortran, Python, TeX, make
    basicstyle=\scriptsize, %\scriptsize%\tiny%\footnotesize  % basic fontsize in it
    breaklines=true
}

\newcommand\FIXME[1]{\textcolor{red}{FIX:}\textcolor{red}{#1}}
\newcommand\FIXED[1]{\textcolor{blue}{#1}}

\newcommand\cparagraph[1]{\vspace{1mm}\noindent\textbf{#1}\xspace}
\newcommand{\SystemName}{\textsc{Comfort}\xspace}

\definecolor{Gray}{gray}{0.95}
\definecolor{highlight}{rgb}{1,1,0.04}


\let\OLDthebibliography\thebibliography
\renewcommand\thebibliography[1]{
  \OLDthebibliography{#1}
  \setlength{\parskip}{0pt}
  \setlength{\itemsep}{0pt}
}