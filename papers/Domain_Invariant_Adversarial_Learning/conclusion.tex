In this paper, we investigated the hypothesis that domain invariant representation can be beneficial for robust learning. With this idea in mind, we proposed a new adversarial learning method, called \textit{Domain Invariant Adversarial Learning} (DIAL) that incorporates domain adversarial neural network into the adversarial training process.
The proposed method, DIAL, is theoretically motivated by the domain adaptation generalization bounds.
% where the natural examples act as our source domain, and the corresponding adversarial examples act as our target domain. 
DIAL is generic and can be combined with any network architecture and any adversarial training technique in a wide range of tasks. Additionally, since the domain classifier does not require the class labels, we argue that additional unlabeled data can be leveraged in future work. Our evaluation process included strong adversaries, unforeseen adversaries, unforeseen corruptions, transfer learning tasks, and ablation studies. Using the extensive empirical analysis, we demonstrate the significant and consistent improvement obtained by DIAL in both robustness and natural accuracy. 
%compared to other defence methods on benchmark datasets.
%under various attack settings.
