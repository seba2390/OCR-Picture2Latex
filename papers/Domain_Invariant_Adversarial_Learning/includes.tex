
% %%%%%%%% ICML 2022 EXAMPLE LATEX SUBMISSION FILE %%%%%%%%%%%%%%%%%

% \documentclass[nohyperref]{article}

% % Recommended, but optional, packages for figures and better typesetting:
\usepackage{microtype}
\usepackage{graphicx}
\usepackage{subfigure}
\usepackage{booktabs} % for professional tables

% % hyperref makes hyperlinks in the resulting PDF.
% % If your build breaks (sometimes temporarily if a hyperlink spans a page)
% % please comment out the following usepackage line and replace
% % \usepackage{icml2022} with \usepackage[nohyperref]{icml2022} above.
\usepackage{hyperref}


% % Attempt to make hyperref and algorithmic work together better:
\newcommand{\theHalgorithm}{\arabic{algorithm}}

\RequirePackage{algorithm}
\RequirePackage{algorithmic}

% % For theorems and such
% \usepackage{amsmath}
% \usepackage{amssymb}
% \usepackage{mathtools}
% \usepackage{amsthm}

% % if you use cleveref..
% \usepackage[capitalize,noabbrev]{cleveref}

% %%%%%%%%%%%%%%%%%%%%%%%%%%%%%%%%
% % THEOREMS
% %%%%%%%%%%%%%%%%%%%%%%%%%%%%%%%%
% \theoremstyle{plain}
% \newtheorem{theorem}{Theorem}[section]
% \newtheorem{proposition}[theorem]{Proposition}
% \newtheorem{lemma}[theorem]{Lemma}
% \newtheorem{corollary}[theorem]{Corollary}
% \theoremstyle{definition}
% \newtheorem{definition}[theorem]{Definition}
% \newtheorem{assumption}[theorem]{Assumption}
% \theoremstyle{remark}
% \newtheorem{remark}[theorem]{Remark}

% % Todonotes is useful during development; simply uncomment the next line
% %    and comment out the line below the next line to turn off comments
% %\usepackage[disable,textsize=tiny]{todonotes}
% \usepackage[textsize=tiny]{todonotes}

% %%%%%%%%%%% end ICML %%%%%%%%%%%%%%%%%%


\usepackage{multirow}
\usepackage[utf8]{inputenc} % allow utf-8 input
\usepackage[T1]{fontenc}    % use 8-bit T1 fonts
\usepackage{hyperref}       % hyperlinks
\usepackage{url}            % simple URL typesetting
\usepackage{booktabs}       % professional-quality tables
\usepackage{amsfonts}       % blackboard math symbols
\usepackage{nicefrac}       % compact symbols for 1/2, etc.
\usepackage{xcolor}         % colors

%%%%%%%%%%%%%% Custom %%%%%%%%%%%%%%%%%%%%
% \usepackage{subcaption}
\usepackage{graphicx}
% \bibliographystyle{plain}
\usepackage{amsmath}
\usepackage{float}

% \newcommand{\E}{\mathbb{E}}
% \renewcommand{\P}{\mathbb{P}}
% \newcommand{\R}{\mathbb{R}}
\newcommand{\calD}{\mathcal{D}}
\newcommand{\calX}{\mathcal{X}}
\newcommand{\calZ}{\mathcal{Z}}
\newcommand{\calY}{\mathcal{Y}}
\newcommand{\calS}{\mathcal{S}}
\newcommand{\calH}{\mathcal{H}}

% \newcommand{\sign}{\operatorname{sign}}

\newcommand{\paren}[1]{\left( #1 \right)}
\newcommand{\sqprn}[1]{\left[ #1 \right]}
\newcommand{\norm}[1]{\Vert #1 \Vert}


\newcommand{\calL}{\mathcal{L}}

% \usepackage{algpseudocode}
% \usepackage[linesnumbered, ruled, vlined]{algorithm2e}
\usepackage{setspace}


\newcommand{\DIAL}{\operatorname{DIAL}}
\newcommand{\nat}{\operatorname{nat}}
\newcommand{\rob}{\operatorname{robust}}
\newcommand{\robloss}{\operatorname{rob}}
\newcommand{\adv}{\operatorname{adv}}
\newcommand{\boundary}{\operatorname{boundary}}
\newcommand{\kl}{\operatorname{KL}}
\newcommand{\ce}{\operatorname{CE}}
\newcommand{\bce}{\operatorname{BCE}}
\newcommand{\awp}{\operatorname{AWP}}
\newcommand{\TRADES}{\operatorname{TRADES}}
\newcommand{\disc}{\operatorname{disc}}


\newcommand{\ak}[1]{{\bf[AK: #1]}}
\newcommand{\ml}[1]{{\bf[ML: #1]}}
\newcommand{\hide}[1]{}


%%%%%%%%%%%%%%%%%%%%%%%%%%%%%%%%%%%%%%%%%%


\newcommand\blfootnote[1]{%
  \begingroup
  \renewcommand\thefootnote{}\footnote{#1}%
  \addtocounter{footnote}{-1}%
  \endgroup
}
