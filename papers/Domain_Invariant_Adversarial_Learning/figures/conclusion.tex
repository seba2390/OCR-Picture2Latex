In this paper, we investigated the hypothesis that domain invariant representation can be beneficial for robust learning. With this idea in mind, we proposed a new adversarial learning method, called \textit{Domain Invariant Adversarial Learning} (DIAL) that incorporates Domain Adversarial Neural Network into the adversarial training process.
% where the natural examples act as our source domain, and the corresponding adversarial examples act as our target domain. 
The proposed method is generic and can be combined with any network architecture in a wide range of tasks. Our evaluation process included strong adversaries 
%(e.g., AA)
, unforeseen adversaries 
%(e.g., $\ell_{1}$, $\ell_{2}$)
, unforeseen corruptions, transfer learning tasks, and ablation studies. Using the extensive empirical analysis, we demonstrate the significant and consistent improvement obtained by DIAL in both robustness and natural accuracy compared to other defence methods on benchmark datasets.
%under various attack settings.
