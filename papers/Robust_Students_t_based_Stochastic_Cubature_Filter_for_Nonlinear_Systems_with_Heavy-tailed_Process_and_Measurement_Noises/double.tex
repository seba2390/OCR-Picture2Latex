%% bare_jrnl.tex
%% V1.4
%% 2012/12/27
%% by Michael Shell
%% see http://www.michaelshell.org/
%% for current contact information.
%%
%% This is a skeleton file demonstrating the use of IEEEtran.cls
%% (requires IEEEtran.cls version 1.8 or later) with an IEEE journal paper.
%%
%% Support sites:
%% http://www.michaelshell.org/tex/ieeetran/
%% http://www.ctan.org/tex-archive/macros/latex/contrib/IEEEtran/
%% and
%% http://www.ieee.org/



% *** Authors should verify (and, if needed, correct) their LaTeX system  ***
% *** with the testflow diagnostic prior to trusting their LaTeX platform ***
% *** with production work. IEEE's font choices can trigger bugs that do  ***
% *** not appear when using other class files.                            ***
% The testflow support page is at:
% http://www.michaelshell.org/tex/testflow/


%%*************************************************************************
%% Legal Notice:
%% This code is offered as-is without any warranty either expressed or
%% implied; without even the implied warranty of MERCHANTABILITY or
%% FITNESS FOR A PARTICULAR PURPOSE!
%% User assumes all risk.
%% In no event shall IEEE or any contributor to this code be liable for
%% any damages or losses, including, but not limited to, incidental,
%% consequential, or any other damages, resulting from the use or misuse
%% of any information contained here.
%%
%% All comments are the opinions of their respective authors and are not
%% necessarily endorsed by the IEEE.
%%
%% This work is distributed under the LaTeX Project Public License (LPPL)
%% ( http://www.latex-project.org/ ) version 1.3, and may be freely used,
%% distributed and modified. A copy of the LPPL, version 1.3, is included
%% in the base LaTeX documentation of all distributions of LaTeX released
%% 2003/12/01 or later.
%% Retain all contribution notices and credits.
%% ** Modified files should be clearly indicated as such, including  **
%% ** renaming them and changing author support contact information. **
%%
%% File list of work: IEEEtran.cls, IEEEtran_HOWTO.pdf, bare_adv.tex,
%%                    bare_conf.tex, bare_jrnl.tex, bare_jrnl_compsoc.tex,
%%                    bare_jrnl_transmag.tex
%%*************************************************************************

% Note that the a4paper option is mainly intended so that authors in
% countries using A4 can easily print to A4 and see how their papers will
% look in print - the typesetting of the document will not typically be
% affected with changes in paper size (but the bottom and side margins will).
% Use the testflow package mentioned above to verify correct handling of
% both paper sizes by the user's LaTeX system.
%
% Also note that the "draftcls" or "draftclsnofoot", not "draft", option
% should be used if it is desired that the figures are to be displayed in
% draft mode.
%
\documentclass[journal]{IEEEtran}
\usepackage{subfigure}
\usepackage{amsmath,epsfig,latexsym}
\usepackage{multicol}
\usepackage{bm}
\usepackage{subfigure}
\usepackage{amssymb}
\usepackage{amsfonts}
\usepackage{graphicx}
\usepackage{url}
\usepackage{ccaption}
\usepackage{booktabs} % �����߱�����������������
\usepackage{color}
\usepackage{multirow}
\usepackage{amsthm}

\setlength\arraycolsep{0.5pt}


\newtheorem{remark}{Remark}
\newtheorem{assumption}{Assumption}
\newtheorem{definition}{Definition}
\newtheorem{lemma}{Lemma}
\newtheorem{theorem}{Theorem}
\newtheorem{proposition}{Proposition}

%
% If IEEEtran.cls has not been installed into the LaTeX system files,
% manually specify the path to it like:
% \documentclass[journal]{../sty/IEEEtran}





% Some very useful LaTeX packages include:
% (uncomment the ones you want to load)


% *** MISC UTILITY PACKAGES ***
%
%\usepackage{ifpdf}
% Heiko Oberdiek's ifpdf.sty is very useful if you need conditional
% compilation based on whether the output is pdf or dvi.
% usage:
% \ifpdf
%   % pdf code
% \else
%   % dvi code
% \fi
% The latest version of ifpdf.sty can be obtained from:
% http://www.ctan.org/tex-archive/macros/latex/contrib/oberdiek/
% Also, note that IEEEtran.cls V1.7 and later provides a builtin
% \ifCLASSINFOpdf conditional that works the same way.
% When switching from latex to pdflatex and vice-versa, the compiler may
% have to be run twice to clear warning/error messages.






% *** CITATION PACKAGES ***
%
%\usepackage{cite}
% cite.sty was written by Donald Arseneau
% V1.6 and later of IEEEtran pre-defines the format of the cite.sty package
% \cite{} output to follow that of IEEE. Loading the cite package will
% result in citation numbers being automatically sorted and properly
% "compressed/ranged". e.g., [1], [9], [2], [7], [5], [6] without using
% cite.sty will become [1], [2], [5]--[7], [9] using cite.sty. cite.sty's
% \cite will automatically add leading space, if needed. Use cite.sty's
% noadjust option (cite.sty V3.8 and later) if you want to turn this off
% such as if a citation ever needs to be enclosed in parenthesis.
% cite.sty is already installed on most LaTeX systems. Be sure and use
% version 4.0 (2003-05-27) and later if using hyperref.sty. cite.sty does
% not currently provide for hyperlinked citations.
% The latest version can be obtained at:
% http://www.ctan.org/tex-archive/macros/latex/contrib/cite/
% The documentation is contained in the cite.sty file itself.






% *** GRAPHICS RELATED PACKAGES ***
%
\ifCLASSINFOpdf
  % \usepackage[pdftex]{graphicx}
  % declare the path(s) where your graphic files are
  % \graphicspath{{../pdf/}{../jpeg/}}
  % and their extensions so you won't have to specify these with
  % every instance of \includegraphics
  % \DeclareGraphicsExtensions{.pdf,.jpeg,.png}
\else
  % or other class option (dvipsone, dvipdf, if not using dvips). graphicx
  % will default to the driver specified in the system graphics.cfg if no
  % driver is specified.
  % \usepackage[dvips]{graphicx}
  % declare the path(s) where your graphic files are
  % \graphicspath{{../eps/}}
  % and their extensions so you won't have to specify these with
  % every instance of \includegraphics
  % \DeclareGraphicsExtensions{.eps}
\fi
% graphicx was written by David Carlisle and Sebastian Rahtz. It is
% required if you want graphics, photos, etc. graphicx.sty is already
% installed on most LaTeX systems. The latest version and documentation
% can be obtained at:
% http://www.ctan.org/tex-archive/macros/latex/required/graphics/
% Another good source of documentation is "Using Imported Graphics in
% LaTeX2e" by Keith Reckdahl which can be found at:
% http://www.ctan.org/tex-archive/info/epslatex/
%
% latex, and pdflatex in dvi mode, support graphics in encapsulated
% postscript (.eps) format. pdflatex in pdf mode supports graphics
% in .pdf, .jpeg, .png and .mps (metapost) formats. Users should ensure
% that all non-photo figures use a vector format (.eps, .pdf, .mps) and
% not a bitmapped formats (.jpeg, .png). IEEE frowns on bitmapped formats
% which can result in "jaggedy"/blurry rendering of lines and letters as
% well as large increases in file sizes.
%
% You can find documentation about the pdfTeX application at:
% http://www.tug.org/applications/pdftex





% *** MATH PACKAGES ***
%
%\usepackage[cmex10]{amsmath}
% A popular package from the American Mathematical Society that provides
% many useful and powerful commands for dealing with mathematics. If using
% it, be sure to load this package with the cmex10 option to ensure that
% only type 1 fonts will utilized at all point sizes. Without this option,
% it is possible that some math symbols, particularly those within
% footnotes, will be rendered in bitmap form which will result in a
% document that can not be IEEE Xplore compliant!
%
% Also, note that the amsmath package sets \interdisplaylinepenalty to 10000
% thus preventing page breaks from occurring within multiline equations. Use:
%\interdisplaylinepenalty=2500
% after loading amsmath to restore such page breaks as IEEEtran.cls normally
% does. amsmath.sty is already installed on most LaTeX systems. The latest
% version and documentation can be obtained at:
% http://www.ctan.org/tex-archive/macros/latex/required/amslatex/math/





% *** SPECIALIZED LIST PACKAGES ***
%
%\usepackage{algorithmic}
% algorithmic.sty was written by Peter Williams and Rogerio Brito.
% This package provides an algorithmic environment fo describing algorithms.
% You can use the algorithmic environment in-text or within a figure
% environment to provide for a floating algorithm. Do NOT use the algorithm
% floating environment provided by algorithm.sty (by the same authors) or
% algorithm2e.sty (by Christophe Fiorio) as IEEE does not use dedicated
% algorithm float types and packages that provide these will not provide
% correct IEEE style captions. The latest version and documentation of
% algorithmic.sty can be obtained at:
% http://www.ctan.org/tex-archive/macros/latex/contrib/algorithms/
% There is also a support site at:
% http://algorithms.berlios.de/index.html
% Also of interest may be the (relatively newer and more customizable)
% algorithmicx.sty package by Szasz Janos:
% http://www.ctan.org/tex-archive/macros/latex/contrib/algorithmicx/




% *** ALIGNMENT PACKAGES ***
%
%\usepackage{array}
% Frank Mittelbach's and David Carlisle's array.sty patches and improves
% the standard LaTeX2e array and tabular environments to provide better
% appearance and additional user controls. As the default LaTeX2e table
% generation code is lacking to the point of almost being broken with
% respect to the quality of the end results, all users are strongly
% advised to use an enhanced (at the very least that provided by array.sty)
% set of table tools. array.sty is already installed on most systems. The
% latest version and documentation can be obtained at:
% http://www.ctan.org/tex-archive/macros/latex/required/tools/


% IEEEtran contains the IEEEeqnarray family of commands that can be used to
% generate multiline equations as well as matrices, tables, etc., of high
% quality.




% *** SUBFIGURE PACKAGES ***
%\ifCLASSOPTIONcompsoc
%  \usepackage[caption=false,font=normalsize,labelfont=sf,textfont=sf]{subfig}
%\else
%  \usepackage[caption=false,font=footnotesize]{subfig}
%\fi
% subfig.sty, written by Steven Douglas Cochran, is the modern replacement
% for subfigure.sty, the latter of which is no longer maintained and is
% incompatible with some LaTeX packages including fixltx2e. However,
% subfig.sty requires and automatically loads Axel Sommerfeldt's caption.sty
% which will override IEEEtran.cls' handling of captions and this will result
% in non-IEEE style figure/table captions. To prevent this problem, be sure
% and invoke subfig.sty's "caption=false" package option (available since
% subfig.sty version 1.3, 2005/06/28) as this is will preserve IEEEtran.cls
% handling of captions.
% Note that the Computer Society format requires a larger sans serif font
% than the serif footnote size font used in traditional IEEE formatting
% and thus the need to invoke different subfig.sty package options depending
% on whether compsoc mode has been enabled.
%
% The latest version and documentation of subfig.sty can be obtained at:
% http://www.ctan.org/tex-archive/macros/latex/contrib/subfig/




% *** FLOAT PACKAGES ***
%
%\usepackage{fixltx2e}
% fixltx2e, the successor to the earlier fix2col.sty, was written by
% Frank Mittelbach and David Carlisle. This package corrects a few problems
% in the LaTeX2e kernel, the most notable of which is that in current
% LaTeX2e releases, the ordering of single and double column floats is not
% guaranteed to be preserved. Thus, an unpatched LaTeX2e can allow a
% single column figure to be placed prior to an earlier double column
% figure. The latest version and documentation can be found at:
% http://www.ctan.org/tex-archive/macros/latex/base/


%\usepackage{stfloats}
% stfloats.sty was written by Sigitas Tolusis. This package gives LaTeX2e
% the ability to do double column floats at the bottom of the page as well
% as the top. (e.g., "\begin{figure*}[!b]" is not normally possible in
% LaTeX2e). It also provides a command:
%\fnbelowfloat
% to enable the placement of footnotes below bottom floats (the standard
% LaTeX2e kernel puts them above bottom floats). This is an invasive package
% which rewrites many portions of the LaTeX2e float routines. It may not work
% with other packages that modify the LaTeX2e float routines. The latest
% version and documentation can be obtained at:
% http://www.ctan.org/tex-archive/macros/latex/contrib/sttools/
% Do not use the stfloats baselinefloat ability as IEEE does not allow
% \baselineskip to stretch. Authors submitting work to the IEEE should note
% that IEEE rarely uses double column equations and that authors should try
% to avoid such use. Do not be tempted to use the cuted.sty or midfloat.sty
% packages (also by Sigitas Tolusis) as IEEE does not format its papers in
% such ways.
% Do not attempt to use stfloats with fixltx2e as they are incompatible.
% Instead, use Morten Hogholm'a dblfloatfix which combines the features
% of both fixltx2e and stfloats:
%
% \usepackage{dblfloatfix}
% The latest version can be found at:
% http://www.ctan.org/tex-archive/macros/latex/contrib/dblfloatfix/




%\ifCLASSOPTIONcaptionsoff
%  \usepackage[nomarkers]{endfloat}
% \let\MYoriglatexcaption\caption
% \renewcommand{\caption}[2][\relax]{\MYoriglatexcaption[#2]{#2}}
%\fi
% endfloat.sty was written by James Darrell McCauley, Jeff Goldberg and
% Axel Sommerfeldt. This package may be useful when used in conjunction with
% IEEEtran.cls'  captionsoff option. Some IEEE journals/societies require that
% submissions have lists of figures/tables at the end of the paper and that
% figures/tables without any captions are placed on a page by themselves at
% the end of the document. If needed, the draftcls IEEEtran class option or
% \CLASSINPUTbaselinestretch interface can be used to increase the line
% spacing as well. Be sure and use the nomarkers option of endfloat to
% prevent endfloat from "marking" where the figures would have been placed
% in the text. The two hack lines of code above are a slight modification of
% that suggested by in the endfloat docs (section 8.4.1) to ensure that
% the full captions always appear in the list of figures/tables - even if
% the user used the short optional argument of \caption[]{}.
% IEEE papers do not typically make use of \caption[]'s optional argument,
% so this should not be an issue. A similar trick can be used to disable
% captions of packages such as subfig.sty that lack options to turn off
% the subcaptions:
% For subfig.sty:
% \let\MYorigsubfloat\subfloat
% \renewcommand{\subfloat}[2][\relax]{\MYorigsubfloat[]{#2}}
% However, the above trick will not work if both optional arguments of
% the \subfloat command are used. Furthermore, there needs to be a
% description of each subfigure *somewhere* and endfloat does not add
% subfigure captions to its list of figures. Thus, the best approach is to
% avoid the use of subfigure captions (many IEEE journals avoid them anyway)
% and instead reference/explain all the subfigures within the main caption.
% The latest version of endfloat.sty and its documentation can obtained at:
% http://www.ctan.org/tex-archive/macros/latex/contrib/endfloat/
%
% The IEEEtran \ifCLASSOPTIONcaptionsoff conditional can also be used
% later in the document, say, to conditionally put the References on a
% page by themselves.




% *** PDF, URL AND HYPERLINK PACKAGES ***
%
%\usepackage{url}
% url.sty was written by Donald Arseneau. It provides better support for
% handling and breaking URLs. url.sty is already installed on most LaTeX
% systems. The latest version and documentation can be obtained at:
% http://www.ctan.org/tex-archive/macros/latex/contrib/url/
% Basically, \url{my_url_here}.




% *** Do not adjust lengths that control margins, column widths, etc. ***
% *** Do not use packages that alter fonts (such as pslatex).         ***
% There should be no need to do such things with IEEEtran.cls V1.6 and later.
% (Unless specifically asked to do so by the journal or conference you plan
% to submit to, of course. )


% correct bad hyphenation here
\hyphenation{op-tical net-works semi-conduc-tor}


\begin{document}
%
% paper title
% can use linebreaks \\ within to get better formatting as desired
% Do not put math or special symbols in the title.
\title{Robust Student's t based Stochastic Cubature Filter for Nonlinear Systems with Heavy-tailed Process and Measurement Noises}

%
%
% author names and IEEE memberships
% note positions of commas and nonbreaking spaces ( ~ ) LaTeX will not break
% a structure at a ~ so this keeps an author's name from being broken across
% two lines.
% use \thanks{} to gain access to the first footnote area
% a separate \thanks must be used for each paragraph as LaTeX2e's \thanks
% was not built to handle multiple paragraphs
%

\author{Yulong~Huang,
        Yonggang~Zhang,~\IEEEmembership{Senior Member,~IEEE,}


\thanks{This work was supported by the National Natural Science Foundation of China under Grant Nos. 61371173 and 61633008 and the Natural Science Foundation of Heilongjiang Province Grant No. F2016008. Corresponding author is Y. G. Zhang.}

\thanks{Y. L. Huang and Y. G. Zhang are with the Department of Automation, Harbin Engineering University, Harbin 150001, China (e-mail: heuedu@163.com; zhangyg@hrbeu.edu.cn).}}


% note the % following the last \IEEEmembership and also \thanks -
% these prevent an unwanted space from occurring between the last author name
% and the end of the author line. i.e., if you had this:
%
% \author{....lastname \thanks{...} \thanks{...} }
%                     ^------------^------------^----Do not want these spaces!
%
% a space would be appended to the last name and could cause every name on that
% line to be shifted left slightly. This is one of those "LaTeX things". For
% instance, "\textbf{A} \textbf{B}" will typeset as "A B" not "AB". To get
% "AB" then you have to do: "\textbf{A}\textbf{B}"
% \thanks is no different in this regard, so shield the last } of each \thanks
% that ends a line with a % and do not let a space in before the next \thanks.
% Spaces after \IEEEmembership other than the last one are OK (and needed) as
% you are supposed to have spaces between the names. For what it is worth,
% this is a minor point as most people would not even notice if the said evil
% space somehow managed to creep in.


% The paper headers
% The only time the second header will appear is for the odd numbered pages
% after the title page when using the twoside option.
%
% *** Note that you probably will NOT want to include the author's ***
% *** name in the headers of peer review papers.                   ***
% You can use \ifCLASSOPTIONpeerreview for conditional compilation here if
% you desire.




% If you want to put a publisher's ID mark on the page you can do it like
% this:
%\IEEEpubid{0000--0000/00\$00.00~\copyright~2012 IEEE}
% Remember, if you use this you must call \IEEEpubidadjcol in the second
% column for its text to clear the IEEEpubid mark.



% use for special paper notices
%\IEEEspecialpapernotice{(Invited Paper)}




% make the title area
\maketitle

% As a general rule, do not put math, special symbols or citations
% in the abstract or keywords.
\begin{abstract}
In this paper, a new robust Student's t based stochastic cubature filter (RSTSCF) is proposed for nonlinear state-space model with heavy-tailed process and measurement noises. The heart of the RSTSCF is a stochastic Student's t spherical radial cubature rule (SSTSRCR), which is derived based on the third-degree unbiased spherical rule and the proposed third-degree unbiased radial rule. The existing stochastic integration rule is a special case of the proposed SSTSRCR when the degrees of freedom parameter tends to infinity. The proposed filter is applied to a manoeuvring bearings-only tracking example, where an agile target is tracked and the bearing is observed in clutter. Simulation results show that the proposed RSTSCF can achieve higher estimation accuracy than the existing Gaussian approximate filter, Gaussian sum filter, Huber-based nonlinear Kalman filter, maximum correntropy criterion based Kalman filter, and robust Student's t based nonlinear filters, and is computationally much more efficient than the existing particle filter.
\end{abstract}

% Note that keywords are not normally used for peerreview papers.
\begin{IEEEkeywords}
Nonlinear filter, heavy-tailed noise, Student's t distribution, Student's t weighted integral, outlier, nonlinear system
\end{IEEEkeywords}



% For peer review papers, you can put extra information on the cover
% page as needed:
% \ifCLASSOPTIONpeerreview
% \begin{center} \bfseries EDICS Category: 3-BBND \end{center}
% \fi
%
% For peerreview papers, this IEEEtran command inserts a page break and
% creates the second title. It will be ignored for other modes.
\IEEEpeerreviewmaketitle


\section{Introduction}
\IEEEPARstart{N}{onlinear} filtering has been playing an important role in many applications, such as target tracking, detection, signal processing, communication and navigation. Under the Bayesian estimation framework, the nonlinear filtering problem is addressed by calculating the posterior probability density function (PDF) recursively based on the nonlinear state-space model. Unfortunately, there is not a closed form solution for posterior PDF for nonlinear state-space model since a closed PDF for nonlinear mapping doesn't exist \cite{Anderson79}. As a result, there is not an optimal solution for nonlinear filtering problem, and an approximate approach is necessary to obtain a suboptimal solution. In general, the posterior PDF is approximated as Gaussian by assuming the jointly predicted PDF of the state and measurement vectors is Gaussian, and the resultant Gaussian approximate (GA) filter can provide tradeoffs between estimation accuracy and computational complexity \cite{CGAF 2015,Arasaratnam 2009}. Up to present, many variants of the GA filter have been proposed using different Gaussian weighted integral rules \cite{Arasaratnam 2009}--\cite{ECKF 2015}. However, in some engineering applications, such as tracking an agile target that is observed in clutter, the heavy-tailed process noise may be induced by severe manoeuvering and the heavy-tailed measurement noise may be induced by measurement outliers from unreliable sensors \cite{student's t filter 2013,Robust GA smoother 2016,TAES 2017 RSTKF}. The performance of the GA filters may degrade for such engineering applications with heavy-tailed noises since they all model the process and measurement noises as Gaussian distributions so that they are sensitive to heavy-tailed non-Gaussian noises \cite{Robust GA smoother 2016}.

Particle filter (PF) is a common method to address non-Gaussian noises, in which the posterior PDF is approximated as a set of random samples with associated weights based on sequential Monte Carlo sampling technique \cite{Arulampalam 2002}. The PF can model the process and measurement noises as arbitrary distributions, such as the Student's t distributions for heavy-tailed non-Gaussian noises \cite{STPF 1993,STPF 2006}. However, the PF suffers from substantial computational complexities in high-dimensional problems because the number of particles increases exponentially with the dimensionality of the state \cite{visual tracking 2008}. Gaussian sum filter (GSF) is an alternative method to handle heavy-tailed non-Gaussian noises, where the heavy-tailed process and measurement noises are modelled as a finite sum of Gaussian distributions, and the posterior distribution is then approximated as a weighted sum of Gaussian distributions by running a bank of GA filters \cite{GSF 1972}--\cite{GSF 2010}. However, for the GSF, it is very difficult to model the heavy-tailed process and measurement noises accurately using finite Gaussian distributions since the heavy-tailed non-Gaussian noises are induced by the unknown manoeuvering or outliers, which may degrade the estimation performance of the GSF.

To solve the filtering problem of nonlinear state-space model with heavy-tailed non-Gaussian noises, the Huber-based nonlinear Kalman filter (HNKF) has been proposed by minimising a Huber cost function that is a combined $l_{1}$ and $l_{2}$ norm \cite{Nonlinear Huber estimator 2010}. A larger number of variants of the HNKF have been derived based on a linearized or statistical linearized method, such as the Huber-based extended Kalman filter \cite{Huber estimator 1995}, the Huber-based divided difference filter \cite{Huber estimator 2007}, the Huber-based unscented Kalman filter \cite{Nonlinear Huber estimator 2012}, the nonlinear regression Huber Kalman filter \cite{Nonlinear regression HKF 2015} and the adaptively robust unscented Kalman filter (ARUKF) \cite{ARUKF 2014}. However, the influence function of the HNKFs don't redescend, which may deteriorate the estimation performance of the HNKFs \cite{TAES 2017 RSTKF}. The maximum correntropy criterion based Kalman filter (MCCKF) has been proposed by maximising the correntropy of the predicted error and residual \cite{MCCKF 2012b}--\cite{MCCKF 2016}. However, there is a lack of theoretical basis to develop the estimation error covariance matrix of the MCCKF, which may degrade the estimation accuracy \cite{TAES 2017 RSTKF}.

A reasonable approach to improve the estimation performance is utilizing a Student's t distribution to model the heavy-tailed non-Gaussian noise. The Student's t distribution is a generalized Gaussian distribution but has heavier tails than the Gaussian distribution, which makes it more suitable for modelling the heavy-tailed non-Gaussian noise. A general framework of the robust Student's t based nonlinear filter (RSTNF) has been proposed, in which the jointly predicted PDF of the state and measurement vectors is assumed to be Student's t, and the posterior PDF is then approximated as Student's t \cite{TAES 2016}. The heart of the RSTNF is how to calculate the Student's t weighted integral, and the estimation accuracy of the associated RSTNF is determined by the employed numerical integral technique. Many variants of the RSTNF have been derived based on different numerical integral methods, such as the robust Student's t based extended filter (RSTEF) using the first-order linearization \cite{student's t filter 2013}, the robust Student's t based unscented filter (RSTUF) using the unscented transform (UT) \cite{TAES 2016,Fusion 2016 Sarkka}, and the robust Student's t based cubature filter (RSTCF) using the third-degree Student's t spherical radial cubature rule (STSRCR) \cite{Fusion 2016}. However, the existing Student's t integral rules can only capture the third-degree or fifth-degree information of the Taylor series expansion for nonlinear approximation, which may result in limited estimation accuracy. Although the Monte Carlo approach can be used to calculate the Student's t weighted integral, it has low accuracy and slow convergence when the integrand is not approximately constant and the number of random samples is finite \cite{Genz 1998}. Therefore, there is a great demand to develop more accurate numerical integral approach for the Student's t weighted integral to further improve the estimation accuracy of the existing RSTNFs.

In this paper, the Student's t weighted integral is decomposed into the spherical integral and the radial integral based on the spherical-radial transformation. A new stochastic STSRCR (SSTSRCR) is derived based on the third-degree unbiased spherical rule (USR) and the proposed third-degree unbiased radial rule (URR), from which a new robust Student's t based stochastic cubature filter (RSTSCF) is obtained. The existing stochastic integration rule (SIR) \cite{Dunik13} is a special case of the proposed SSTSRCR when the degrees of freedom (dof) parameter tends to infinity. The proposed SSTSRCR can achieve better approximation to the Student's t weighted integral as compared with existing Student's t integral rules. As a result, the proposed RSTSCF has higher estimation accuracy than the existing RSTNFs. The proposed filter and existing filters are applied to a manoeuvring bearings-only tracking example, where an agile target is tracked and the bearing is observed in clutter. Simulation results show that the proposed RSTSCF can achieve higher estimation accuracy than the existing GA filter, GSF, HNKF, MCCKF and RSTNFs, and is computationally much more efficient than the existing PF.

The remainder of this paper is organized as follows. In Section II, a general frame of the RSTNF is reviewed. In Section III, a new SSTSRCR is derived based on the proposed third-degree URR, from which a new RSTSCF is obtained. Also, the relationship between the proposed SSTSRCR and the existing SIR is revealed in Section III. In Section IV, the proposed filter is applied to a manoeuvring bearings-only tracking example and simulation results are given. Concluding remarks are drawn in Section V.



\section{Problem Statement}
Consider the following discrete-time nonlinear stochastic system as represented by the state-space model \cite{TAES 2016}
\begin{equation}
\mathbf{x}_{k}=\mathbf{f}_{k-1}(\mathbf{x}_{k-1})+\mathbf{w}_{k-1} \qquad \textrm{(process equation)}
\end{equation}
%%%%%%%%%%%%%%%%%%%%%%%
\begin{equation}
\mathbf{z}_{k}=\mathbf{h}_{k}(\mathbf{x}_{k})+\mathbf{v}_{k} \qquad\;\; \textrm{(measurement equation)},
\end{equation}
where ${k}$ is the discrete time index, $\mathbf{x}_k\in\mathbb{R}^{n}$ is the state vector, $\mathbf{z}_k\in\mathbb{R}^{m}$ is the measurement vector, and $\mathbf{f}_{k-1}(\cdot)$ and $\mathbf{h}_{k}(\cdot)$ are known process and measurement functions respectively. $\mathbf{w}_k\in\mathbb{R}^{n}$ and $\mathbf{v}_k\in\mathbb{R}^{m}$ are heavy-tailed process and measurement noise vectors respectively, which are induced by process and measurement outliers, and their distributions are modelled as Student's t distributions, i.e.,
\begin{equation}
p(\mathbf{w}_{k})=\mathrm{St}(\mathbf{w}_{k};\mathbf{0},\mathbf{Q}_{k},\nu_{1})
\end{equation}
%%%%%%%%%%%%%%%%%%%%%%%
\begin{equation}
p(\mathbf{v}_{k})=\mathrm{St}(\mathbf{v}_{k};\mathbf{0},\mathbf{R}_{k},\nu_{2}),
\end{equation}
where $\mathrm{St}(\cdot;\mathbf{\mu},\mathbf{\Sigma},v)$ denotes the Student's t PDF with mean vector $\mathbf{\mu}$, scale matrix $\mathbf{\Sigma}$, and dof parameter $v$, $\mathbf{Q}_{k}$ and $\nu_{1}$ are the scale matrix and dof parameter of process noise respectively, and $\mathbf{R}_{k}$ and $\nu_{2}$ are the scale matrix and dof parameter of measurement noise respectively. The initial state vector $\mathbf{x}_{0}$ is also assumed to have a Student's t distribution with mean vector $\mathbf{\hat{x}}_{0|0}$, scale matrix $\mathbf{P}_{0|0}$, and dof parameter $\nu_{3}$, and $\mathbf{x}_{0}$, $\mathbf{w}_{k}$ and $\mathbf{v}_{k}$ are assumed to be mutually uncorrelated.

To achieve the filtering estimation, a general framework of RSTNF is derived for the nonlinear system formulated in equations (1)-(4), where the jointly predicted PDF of the state and measurement vectors is assumed as Student's t, then the posterior PDF of the state vector can be approximated as Student's t \cite{TAES 2016}. The time update and measurement update of the recursive RSTNF are given as follows: \\
\textbf{Time update}
\begin{eqnarray}
&&\mathbf{\hat{x}}_{k|k-1}=\int_{\mathbb{R}^{n}} \mathbf{f}_{k-1}(\mathbf{x}_{k-1})\mathrm{St}(\mathbf{x}_{k-1};\mathbf{\hat{x}}_{k-1|k-1},\mathbf{P}_{k-1|k-1},\nu_{3}) \nonumber\\
&&d\mathbf{x}_{k-1}
\end{eqnarray}
%%%%%%%%%%%%%%%%%%%%%%%%%%%%%%%%%%%%%%
\begin{eqnarray}
&&\mathbf{P}_{k|k-1}=\frac{\nu_{3}-2}{\nu_{3}}\int_{\mathbb{R}^{n}} \mathbf{f}_{k-1}(\mathbf{x}_{k-1})\mathbf{f}_{k-1}^{T}(\mathbf{x}_{k-1})\mathrm{St}(\mathbf{x}_{k-1}; \nonumber\\
&&\mathbf{\hat{x}}_{k-1|k-1},\mathbf{P}_{k-1|k-1},\nu_{3})d\mathbf{x}_{k-1}-\frac{\nu_{3}-2}{\nu_{3}}\mathbf{\hat{x}}_{k|k-1}\mathbf{\hat{x}}_{k|k-1}^{T} \nonumber\\
&&+\frac{\nu_{1}(\nu_{3}-2)}{(\nu_{1}-2)\nu_{3}}\mathbf{Q}_{k-1},
\end{eqnarray}
where $(\cdot)^{T}$ denotes the transpose operation, $\mathbf{\hat{x}}_{k|k-1}$ and $\mathbf{P}_{k|k-1}$ are respectively the mean vector and scale matrix of the one-step predicted PDF $p(\mathbf{x}_{k}|\mathbf{Z}_{k-1})$, $\mathbf{Z}_{k-1}=\{\mathbf{z}_{j}\}_{j=1}^{k-1}$ is the set of $k-1$ measurement vectors, and $\nu_{3}$ denotes the dof parameter of the filtering PDF. \\
\textbf{Measurement update}
%%%%%%%%%%%%%%%%%%%%%%%%%%%%%%%%%%%%%%
\begin{equation}
\Delta_{k}=\sqrt{(\mathbf{z}_{k}-\mathbf{\hat{z}}_{k|k-1})^{T}(\mathbf{P}_{k|k-1}^{zz})^{-1}(\mathbf{z}_{k}-\mathbf{\hat{z}}_{k|k-1})}
\end{equation}
%%%%%%%%%%%%%%%%%%%%%%%%%%%%%%%%%%%%%%
\begin{equation}
\mathbf{K}_{k}=\mathbf{P}_{k|k-1}^{xz}(\mathbf{P}_{k|k-1}^{zz})^{-1}
\end{equation}
%%%%%%%%%%%%%%%%%%%%%%%%%%%%%%%%%%%%%%
\begin{equation}
\mathbf{\hat{x}}_{k|k}=\mathbf{\hat{x}}_{k|k-1}+\mathbf{K}_{k}(\mathbf{z}_{k}-\mathbf{\hat{z}}_{k|k-1})
\end{equation}
%%%%%%%%%%%%%%%%%%%%%%%%%%%%%%%%%%%%%%
\begin{equation}
\mathbf{P}_{k|k}=\frac{(\nu_{3}-2)(\nu_{3}+\Delta_{k}^{2})}{\nu_{3}(\nu_{3}+m-2)}(\mathbf{P}_{k|k-1}-\mathbf{K}_{k}\mathbf{P}_{k|k-1}^{zz}\mathbf{K}_{k}^{T}),
\end{equation}
where $(\cdot)^{\mathrm{-1}}$ denotes the inverse operation, $\mathbf{\hat{x}}_{k|k}$ and $\mathbf{P}_{k|k}$ are respectively the mean vector and scale matrix of the filtering PDF $p(\mathbf{x}_{k}|\mathbf{Z}_{k})$, $\mathbf{\hat{z}}_{k|k-1}$ and $\mathbf{P}_{k|k-1}^{zz}$ are respectively the mean vector and scale matrix of the likelihood PDF $p(\mathbf{z}_{k}|$ $\mathbf{Z}_{k-1})$, and $\mathbf{P}_{k|k-1}^{xz}$ is the cross scale matrix of state and measurement vectors, which are given by
\begin{eqnarray}
\mathbf{\hat{z}}_{k|k-1}&=&\int_{\mathbb{R}^{n}} \mathbf{h}_{k}(\mathbf{x}_{k})\mathrm{St}(\mathbf{x}_{k};\mathbf{\hat{x}}_{k|k-1},\mathbf{P}_{k|k-1},\nu_{3})d\mathbf{x}_{k}
\end{eqnarray}
%%%%%%%%%%%%%%%%%%%%%%%%%%%%%%%%%%%%%%
\begin{eqnarray}
&&\mathbf{P}_{k|k-1}^{zz}=\frac{\nu_{3}-2}{\nu_{3}}\int_{\mathbb{R}^{n}} \mathbf{h}_{k}(\mathbf{x}_{k})\mathbf{h}_{k}^{T}(\mathbf{x}_{k})\mathrm{St}(\mathbf{x}_{k};\mathbf{\hat{x}}_{k|k-1},\mathbf{P}_{k|k-1},
 \nonumber\\
&&\nu_{3})d\mathbf{x}_{k}-\frac{\nu_{3}-2}{\nu_{3}}\mathbf{\hat{z}}_{k|k-1}\mathbf{\hat{z}}_{k|k-1}^{T}+\frac{\nu_{2}(\nu_{3}-2)}{(\nu_{2}-2)\nu_{3}}\mathbf{R}_{k}
\end{eqnarray}
%%%%%%%%%%%%%%%%%%%%%%%%%%%%%%%%%%%%%%
\begin{eqnarray}
&&\mathbf{P}_{k|k-1}^{xz}=\frac{\nu_{3}-2}{\nu_{3}}\int_{\mathbb{R}^{n}} \mathbf{x}_{k}\mathbf{h}_{k}^{T}(\mathbf{x}_{k})\mathrm{St}(\mathbf{x}_{k};\mathbf{\hat{x}}_{k|k-1},\mathbf{P}_{k|k-1},\nu_{3}) \nonumber\\
&&d\mathbf{x}_{k}-\frac{\nu_{3}-2}{\nu_{3}}\mathbf{\hat{x}}_{k|k-1}\mathbf{\hat{z}}_{k|k-1}^{T}.
\end{eqnarray}

The recursive RSTNF is composed of the analytical computations in equations (7)-(10) and the Student's t weighted integrals in equations (5)-(6) and (11)-(13). The key problem in the design of the RSTNF is calculating the nonlinear Student's t weighted integrals formulated in equations (5)-(6) and (11)-(13), whose integrands are all of the form \emph{nonlinear function}$\times$\emph{Student's t PDF}. Therefore, the numerical integral technique is required to implement the RSTNF, which determines the estimation accuracy of associated RSTNF. Next, to further improve the estimation accuracy of existing RSTNFs, a new SSTSRCR will be proposed, based on which a new RSTSCF can be obtained.



\section{Main Results}
\subsection{Spherical-radial transformation}
The Student's t weighted integrals involved in the RSTNF can be written as the general form as follows
\begin{equation}
I[\mathbf{g}]=\int_{\mathbb{R}^{n}}\mathbf{g}(\mathbf{x})\mathrm{St}(\mathbf{x};\mathbf{\mu},\mathbf{\Sigma},\nu)d\mathbf{x},
\end{equation}
where the Student's t PDF is given by
\begin{eqnarray}
&&\mathrm{St}(\mathbf{x};\mathbf{\mu},\mathbf{\Sigma},\nu)=\frac{\Gamma(\frac{\nu+n}{2})}{\Gamma(\frac{\nu}{2})}\frac{1}{\sqrt{|\nu\pi\mathbf{\Sigma}|}}\times \nonumber\\
&&\left[1+\frac{1}{\nu}(\mathbf{x}-\mathbf{\mu})^{\mathrm{T}}\mathbf{\Sigma}^{-1}(\mathbf{x}-\mathbf{\mu})\right]^{-\frac{\nu+n}{2}},
\end{eqnarray}
where $\Gamma(\cdot)$ and $|\cdot|$ denote the Gamma function and determinant operation respectively. To derive the SSTSRCR, the Student's t weighted integral in equation (14) requires to be transformed into a spherical-radial integral form.

A change of variable is utilized as follows
\begin{equation}
\mathbf{x}=\mathbf{\mu}+\sqrt{\nu\mathbf{\Sigma}}\mathbf{y},
\end{equation}
where $\sqrt{\mathbf{\Sigma}}$ is the square-root of scale matrix $\mathbf{\Sigma}$ satisfying $\mathbf{\Sigma}=\sqrt{\mathbf{\Sigma}}\sqrt{\mathbf{\Sigma}}^{T}$.

Substituting equation (16) in equations (14)-(15) and using the identity $|\sqrt{\nu\mathbf{\Sigma}}|=\sqrt{|\nu\mathbf{\Sigma}|}$ yields
\begin{equation}
I[\mathbf{g}]=\int_{\mathbb{R}^{n}}\mathbf{l}(\mathbf{y})(1+\mathbf{y}^{\mathrm{T}}\mathbf{y})^{-\frac{\nu+n}{2}}d\mathbf{y},
\end{equation}
where $\mathbf{l}(\mathbf{y})$ is given by
\begin{equation}
\mathbf{l}(\mathbf{y})=\frac{\Gamma(\frac{\nu+n}{2})}{\Gamma(\frac{\nu}{2})\pi^{\frac{n}{2}}}\mathbf{g}(\mathbf{\mu}+\sqrt{\nu\mathbf{\Sigma}}\mathbf{y}).
\end{equation}

Define $\mathbf{y}=r\mathbf{s}$ with $\mathbf{s}^{T}\mathbf{s}=1$, then equation (17) can be rewritten as \cite{Stroud 1971}
\begin{eqnarray}
I[\mathbf{g}]&=&\int_{0}^{+\infty}\int_{U_{n}}\mathbf{l}(r\mathbf{s})r^{n-1}[1+(r\mathbf{s})^{T}(r\mathbf{s})]^{-\frac{\nu+n}{2}}d\sigma(s)dr \nonumber\\
&=&\int_{0}^{+\infty}\int_{U_{n}}\mathbf{l}(r\mathbf{s})r^{n-1}(1+r^2)^{-\frac{\nu+n}{2}}d\sigma(s)dr,
\end{eqnarray}
where $\mathbf{s}=[\mathbf{s}_{1},\mathbf{s}_{2},\cdots,\mathbf{s}_{n}]^{T}$, $U_{n}=\{\mathbf{s}\in\mathbb{R}^{n}:s_{1}^{2}+s_{2}^{2}+\cdots+s_{n}^{2}=1\}$, and $\sigma(\mathbf{s})$ is the spherical surface measure or an area element on $U_{n}$.

According to equation (19), the Student's t weighted integral in equation (14) can be decomposed into the radial integral
\begin{equation}
I[\mathbf{g}]=\int_{0}^{+\infty}\mathbf{S}(r)r^{n-1}(1+r^2)^{-\frac{\nu+n}{2}}dr,
\end{equation}
and the spherical integral
\begin{equation}
\mathbf{S}(r)=\int_{U_{n}}\mathbf{l}(r\mathbf{s})d\sigma(s).
\end{equation}

Next, a new third-degree SSTSRCR will be derived, in which the spherical and the radial integrals are respectively calculated by the third-degree USR (Section III. B below) and the third-degree URR (Section III. C below). Before deriving the third-degree SSTSRCR, the unbiased integral rule is firstly defined as follows.
\begin{definition}
The integral rule $\int\mathbf{g}(\mathbf{x})p(\mathbf{x})d\mathbf{x}\approx\sum\limits_{l=1}^{N}w_{l}\mathbf{g}(\mathbf{x}_{l})$ is unbiased if and only if \cite{Genz 1998}
\begin{equation}
\int\mathbf{g}(\mathbf{x})p(\mathbf{x})d\mathbf{x}=E\left[\sum\limits_{l=1}^{N}w_{l}\mathbf{g}(\mathbf{x}_{l})\right],
\end{equation}
where $\mathbf{x}_{l}$ and $w_{l}$ are respectively cubature points and corresponding weights, and $E[\cdot]$ denotes the expectation operation.
\end{definition}


\subsection{Unbiased spherical rule}
The Stewart's method is employed to construct the third-degree USR. If $\mathbf{Q}$ is a random orthogonal matrix drawn with a Haar distribution from the set of all matrices in the orthogonal group, the third-degree USR can be constructed as \cite{Genz 1998,Stewart 1980}
\begin{equation}
\mathbf{S}_{u}^{3}(r)\approx\frac{A_{n}}{2n}\sum\limits_{i=1}^{n}\left[\mathbf{l}(-r\mathbf{Q}\mathbf{e}_{i})+\mathbf{l}(r\mathbf{Q}\mathbf{e}_{i})\right],
\end{equation}
where $A_{n}=\frac{2\pi^{\frac{n}{2}}}{\Gamma(\frac{n}{2})}$ is the surface area of the unit sphere, and $\mathbf{e}_{i}$ denotes the $i$-th column of an $n\times{n}$ unit matrix. To produce a random orthogonal matrix $\mathbf{Q}$, a $n\times n$ matrix $\mathbf{U}$ of standard norm variables is first generated, then the required random orthogonal matrix $\mathbf{Q}$ is obtained based on the $QR$ factorization, i.e., $\mathbf{U}=\mathbf{Q}\mathbf{R}$ \cite{Stewart 1980}.

Next, a new third-degree URR will be proposed for the radial integral in equation (20).


\subsection{Unbiased radial rule}
Generally, the monomials $S(r)=1$, $S(r)=r$, $S(r)=r^{2}$, and $S(r)=r^{3}$ need to be matched to derive the third-degree URR. However, only monomials $S(r)=1$ and $S(r)=r^{2}$ need to be matched for the third-degree URR since the USR and the resultant STSRCR are fully symmetry. Thus, two points $\{r_{1}, \omega_{r,1}\}$ and $\{r_{2}, \omega_{r,2}\}$ are sufficient to design the third-degree URR, where one point is used to match monomials $S(r)=1$ and $S(r)=r^{2}$ and the other point is employed to retain unbiasedness. That is to say, the third-degree URR can be written as
\begin{equation}
\int_{0}^{+\infty}\mathbf{S}(r)r^{n-1}(1+r^2)^{-\frac{\nu+n}{2}}dr\approx\omega_{r,1}\mathbf{S}(r_{1})+\omega_{r,2}\mathbf{S}(r_{2}),
\end{equation}
where $\{r_{1}, \omega_{r,1}\}$ and $\{r_{2}, \omega_{r,2}\}$ satisfy the following equations
\begin{equation}
\omega_{r,1}r_{1}^{0}+\omega_{r,2}r_{2}^{0}=\int_{0}^{+\infty}r^{0}r^{n-1}(1+r^2)^{-\frac{\nu+n}{2}}dr
\end{equation}
%%%%%%%%%%%%%%%%%%%%%%%%%%%%%%%%%%%%%%%%%
\begin{equation}
\omega_{r,1}r_{1}^{2}+\omega_{r,2}r_{2}^{2}=\int_{0}^{+\infty}r^{2}r^{n-1}(1+r^2)^{-\frac{\nu+n}{2}}dr
\end{equation}
%%%%%%%%%%%%%%%%%%%%%%%%%%%%%%%%%%%%%%%%%
\begin{equation}
\int_{0}^{+\infty}\mathbf{S}(r)r^{n-1}(1+r^2)^{-\frac{\nu+n}{2}}dr=E\left[\omega_{r,1}\mathbf{S}(r_{1})+\omega_{r,2}\mathbf{S}(r_{2})\right].
\end{equation}

Since there are three equations and four variables in equations (25)-(27), there is one free variable. In order to get the STSRCR with the minimum number of points, $r_{1}$ is chosen as the free variable and set to zero.

\begin{theorem}
If $r_{1}=0$ and the PDF of random variable $r_{2}$ is
\begin{equation}
p(r_{2})=2r_{2}^{n+1}(1+r_{2}^{2})^{-\frac{\nu+n}{2}}/\mathrm{B}(\frac{n+2}{2},\frac{\nu-2}{2}),
\end{equation}
where $\mathrm{B}(\cdot,\cdot)$ denotes the beta function, then the third-degree URR is given by
{\small\begin{equation}
I[\mathbf{g}]\approx\frac{1}{2}\mathrm{B}(\frac{n}{2},\frac{\nu}{2})
\left\{\left[1-\frac{n}{(\nu-2)r_{2}^{2}}\right]\mathbf{S}(0)+\frac{n}{(\nu-2)r_{2}^{2}}\mathbf{S}(r_{2})\right\}.
\end{equation}}
\end{theorem}

\begin{proof}
Firstly, a general integral $\int_{0}^{+\infty}r^{l}r^{n-1}(1+r^2)^{-\frac{\nu+n}{2}}dr$ is calculated to obtain the right-hand parts in equations (25)-(26). Making a change of variable via $t=r^{2}$ results in
\begin{equation}
\int_{0}^{+\infty}r^{l}r^{n-1}(1+r^2)^{-\frac{\nu+n}{2}}dr=\frac{1}{2}\mathrm{B}(\frac{n+l}{2},\frac{\nu-l}{2}),
\end{equation}
where $\mathrm{B}(\cdot,\cdot)$ denotes the beta function.

Substituting equation (30) in equations (25)-(26), we have
\begin{equation}
\omega_{r,1}+\omega_{r,2}=\frac{1}{2}\mathrm{B}(\frac{n}{2},\frac{\nu}{2})
\end{equation}
%%%%%%%%%%%%%%%%%%%%%%%%%%%%%%%%%%%%%%%%%
\begin{equation}
\omega_{r,1}r_{1}^{2}+\omega_{r,2}r_{2}^{2}=\frac{1}{2}\mathrm{B}(\frac{n+2}{2},\frac{\nu-2}{2}).
\end{equation}

Utilizing the identities $\Gamma(a+1)=a\Gamma(a)$ and $\mathrm{B}(a,b)=\frac{\Gamma(a)\Gamma(b)}{\Gamma(a+b)}$ in equation (32) yields
\begin{equation}
\omega_{r,1}r_{1}^{2}+\omega_{r,2}r_{2}^{2}=\frac{n}{2(\nu-2)}\mathrm{B}(\frac{n}{2},\frac{\nu}{2}).
\end{equation}

Employing $r_{1}=0$ in equation (33) yields
\begin{equation}
\omega_{r,2}=\frac{n}{2(\nu-2)r_{2}^{2}}\mathrm{B}(\frac{n}{2},\frac{\nu}{2}).
\end{equation}

Substituting equation (34) in equation (31) results in
\begin{equation}
\omega_{r,1}=\frac{1}{2}\mathrm{B}(\frac{n}{2},\frac{\nu}{2})\left[1-\frac{n}{(\nu-2)r_{2}^{2}}\right].
\end{equation}

Utilizing $r_{1}=0$ and equations (34)-(35), the expectation of the third-degree radial rule with respect to $p(r_{2})$ is written as
{\small\begin{eqnarray}
&&E\left[\omega_{r,1}\mathbf{S}(r_{1})+\omega_{r,2}\mathbf{S}(r_{2})\right]=\frac{1}{2}\mathrm{B}(\frac{n}{2},\frac{\nu}{2})E\left[1-\frac{n}{(\nu-2)r_{2}^{2}}\right]\times \nonumber\\
&&\mathbf{S}(0)+\frac{1}{2}\mathrm{B}(\frac{n}{2},\frac{\nu}{2})E\left[\frac{n}{(\nu-2)r_{2}^{2}}\mathbf{S}(r_{2})\right].
\end{eqnarray}}

Using equations (28) and (30), we have
\begin{eqnarray}
&&E\left[1-\frac{n}{(\nu-2)r_{2}^{2}}\right]=\int_{0}^{+\infty}\frac{2r_{2}^{n+1}(1+r_{2}^{2})^{-\frac{\nu+n}{2}}}{\mathrm{B}(\frac{n+2}{2},\frac{\nu-2}{2})}dr_{2}- \nonumber\\
&&\frac{n}{(\nu-2)}\int_{0}^{+\infty}\frac{2r_{2}^{n-1}(1+r_{2}^{2})^{-\frac{\nu+n}{2}}}{\mathrm{B}(\frac{n+2}{2},\frac{\nu-2}{2})}dr_{2}=0
\end{eqnarray}
%%%%%%%%%%%%%%%%%%%%%%%%%%%%%%%%%%%%
{\small\begin{eqnarray}
&&E\left[\frac{n}{(\nu-2)r_{2}^{2}}\mathbf{S}(r_{2})\right]=\frac{n}{(\nu-2)}\int_{0}^{+\infty}\frac{2r_{2}^{n-1}(1+r_{2}^{2})^{-\frac{\nu+n}{2}}}{\mathrm{B}(\frac{n+2}{2},\frac{\nu-2}{2})}\times  \nonumber\\
&&\mathbf{S}(r_{2})dr_{2}=\frac{2}{\mathrm{B}(\frac{n}{2},\frac{\nu}{2})}\int_{0}^{+\infty}\mathbf{S}(r)r^{n-1}(1+r^2)^{-\frac{\nu+n}{2}}dr.
\end{eqnarray}}

Substituting equations (37)-(38) in equation (36) yields
\begin{equation}
E\left[\omega_{r,1}\mathbf{S}(r_{1})+\omega_{r,2}\mathbf{S}(r_{2})\right]=\int_{0}^{+\infty}\mathbf{S}(r)r^{n-1}(1+r^2)^{-\frac{\nu+n}{2}}dr.
\end{equation}

With $r_{1}=0$, equations (34)-(35) and (39), the third-degree URR can be formulated as equation (29).
\end{proof}
It is very difficult to directly generate random samples from $p(r_{2})$ since $p(r_{2})$ is not a special PDF. To solve this problem, Theorem 2 is presented as follows.
\begin{theorem}
If random variable $\tau=\frac{r_{2}^{2}}{1+r_{2}^{2}}$, then random variable $\tau$ obeys the Beta distribution, i.e.,
\begin{equation}
p(\tau)=\mathrm{Beta}(\tau;\frac{n+2}{2},\frac{\nu-2}{2})=\frac{\tau^{\frac{n+2}{2}-1}(1-\tau)^{\frac{\nu-2}{2}-1}}{\mathrm{B}(\frac{n+2}{2},\frac{\nu-2}{2})},
\end{equation}
where $\mathrm{Beta}(\cdot;\alpha,\beta)$ denotes the beta PDF with parameters $\alpha$ and $\beta$.
\end{theorem}

\begin{proof}
Since $\tau=\frac{r_{2}^{2}}{1+r_{2}^{2}}$ and $r_{2}\in[0,+\infty)$, random variable $\tau\in[0,1)$. According to $\tau=\frac{r_{2}^{2}}{1+r_{2}^{2}}$, $r_{2}$ is formulated as
\begin{equation}
r_{2}=c(\tau)=\sqrt{\frac{\tau}{1-\tau}} \qquad \tau\in[0,1).
\end{equation}

Employing the transformation theorem and equation (41), the PDF of random variable $\tau$ is given by
\begin{equation}
p(\tau)=p_{r_{2}}(c(\tau))c'(\tau),
\end{equation}
where $p_{r_{2}}(\cdot)$ denotes the PDF of $r_{2}$ and $c'(\tau)$ denotes the derivative of $c(\tau)$ with respect to $\tau$ given by
\begin{equation}
c'(\tau)=0.5\tau^{-\frac{1}{2}}(1-\tau)^{-\frac{3}{2}}.
\end{equation}

Substituting equations (28), (41) and (43) in equation (42) obtains
\begin{equation}
p(\tau)=\tau^{\frac{n+2}{2}-1}(1-\tau)^{\frac{\nu-2}{2}-1}/\mathrm{B}(\frac{n+2}{2},\frac{\nu-2}{2}),
\end{equation}
which proves the theorem.
\end{proof}


\subsection{Stochastic STSRCR}
A theorem is first presented to derive the unbiased STSRCR.
\begin{theorem}
If the spherical and radial rules are unbiased, then the resultant STSRCR is also unbiased.
\end{theorem}

\begin{proof}
If the spherical and radial rules are given by
\begin{equation}
\mathbf{S}(r)\approx\sum\limits_{i=1}^{N_{s}}w_{s,i}\mathbf{l}(r\mathbf{s}_{i}); \qquad
I[\mathbf{g}]\approx\sum\limits_{j=1}^{N_{r}}w_{r,j}\mathbf{S}(r_{j}),
\end{equation}
then the STSRCR can be formulated as
\begin{equation}
I[\mathbf{g}]\approx\sum\limits_{j=1}^{N_{r}}\sum\limits_{i=1}^{N_{s}}w_{r,j}w_{s,i}\mathbf{l}(r_{j}\mathbf{s}_{i}),
\end{equation}
where $\mathbf{s}_{i}$ and $w_{s,i}$ are respectively cubature points and weights of the spherical rule, and $r_{j}$ and $w_{r,j}$ are respectively quadrature points and weights of the radial rule.

Since the spherical and radial rules are unbiased, we obtain
\begin{eqnarray}
\mathbf{S}(r)=E\left[\sum\limits_{i=1}^{N_{s}}w_{s,i}\mathbf{l}(r\mathbf{s}_{i})\right]; \; I[\mathbf{g}]=E\left[\sum\limits_{j=1}^{N_{r}}w_{r,j}\mathbf{S}(r_{j})\right].
\end{eqnarray}

Using equation (47) yields
\begin{equation}
I[\mathbf{g}]=E\left\{\sum\limits_{j=1}^{N_{r}}w_{r,j}E\left[\sum\limits_{i=1}^{N_{s}}w_{s,i}\mathbf{l}(r_{j}\mathbf{s}_{i})\right]\right\}.
\end{equation}

Since the set $\{\mathbf{s}_{i},\,w_{s,i}\}_{i=1}^{N_s}$ is independent of the set $\{\mathbf{r}_{j},\,w_{r,j}\}_{j=1}^{N_r}$, we have
\begin{equation}
I[\mathbf{g}]=E\left[\sum\limits_{j=1}^{N_{r}}\sum\limits_{i=1}^{N_{s}}w_{r,j}w_{s,i}\mathbf{l}(r_{j}\mathbf{s}_{i})\right],
\end{equation}
which proves the theorem.
\end{proof}

Using Theorems 1-3 obtains
\begin{eqnarray}
&&I[\mathbf{g}]=E\left\{\left[1-\frac{n}{(\nu-2)r_{2}^{2}}\right]\mathbf{g}(\mathbf{\mu})+\frac{1}{2(\nu-2)r_{2}^{2}}\times\right. \nonumber\\
&&\left.\sum\limits_{i=1}^{n}\left[\mathbf{g}(\mathbf{\mu}-r_{2}\sqrt{\nu\mathbf{\Sigma}}\mathbf{Q}\mathbf{e}_{i})+\mathbf{g}(\mathbf{\mu}+r_{2}\sqrt{\nu\mathbf{\Sigma}}\mathbf{Q}\mathbf{e}_{i})\right]\right\}.
\end{eqnarray}

By employing the Monte Carlo approach, the right-hand parts of equation (50) can be approximated as
\begin{eqnarray}
&&I_{s}^{3}[\mathbf{g}]=\frac{1}{N}\sum\limits_{l=1}^{N}\left\{\left[1-\frac{n}{(\nu-2)r_{2,l}^{2}}\right]\mathbf{g}(\mathbf{\mu})+\frac{1}{2(\nu-2)r_{2,l}^{2}}\times\right. \nonumber\\
&&\left.\sum\limits_{i=1}^{n}\left[\mathbf{g}(\mathbf{\mu}-r_{2,l}\sqrt{\nu\mathbf{\Sigma}}\mathbf{Q}_{l}\mathbf{e}_{i})+\mathbf{g}(\mathbf{\mu}+r_{2,l}\sqrt{\nu\mathbf{\Sigma}}\mathbf{Q}_{l}\mathbf{e}_{i})\right]\right\},
\end{eqnarray}
where $N$ denotes the number of random samples, and $\mathbf{Q}_{l}$ is a random orthogonal matrix, and $r_{2,l}$ is drawn randomly from $p(r_{2})$. The form $I_{s}^{3}[\mathbf{g}]$ denotes the proposed third-degree SSTSRCR, and the implementation pseudocode of the proposed SSTSRCR is shown in Table I.
%%%%%%%%%%%%%%%%%%%%%%%%%%%%%%%%%%%%%%
%%%%%%%%%%%%%%%%%%%%%%%%%%%%%%%%%%%%%%
\begin{table}[!t]
\renewcommand{\arraystretch}{2.0}
\caption{The implementation pseudocode of the proposed SSTSRCR.}
\centering
\begin{tabular}[width=0.8in]{l}
\\
\hline
{\bfseries Inputs}: $\mathbf{\mu}$, $\mathbf{\Sigma}$, $\nu$, $\mathbf{g}(\cdot)$, $n$, $N$. \\
1. Initialization: $I_{s}^{3}[\mathbf{g}]=0$. \\
{\bfseries for} $l=1:N$ \\
2. Generate a $n\times n$ matrix $\mathbf{U}_{l}$ of standard norm variables. \\
3. Obtain the required random orthogonal matrix $\mathbf{Q}_{l}$ using the \\
$QR$ factorization: $\mathbf{U}_{l}=\mathbf{Q}_{l}\mathbf{R}_{l}$. \\
4. Draw the random variable $\tau_{l}$ from a Beta distribution: \\
$\tau_{l}\sim\mathrm{Beta}(\frac{n+2}{2},\frac{\nu-2}{2})$. \\
5. Calculate the random quadrature point $r_{2,l}$: \\
$r_{2,l}=\sqrt{\frac{\tau_{l}}{1-\tau_{l}}}$. \\
6. Update $I_{s}^{3}[\mathbf{g}]$ at current iteration: \\
$I_{s}^{3}[\mathbf{g}]=I_{s}^{3}[\mathbf{g}]+\frac{1}{N}\left\{\left[1-\frac{n}{(\nu-2)r_{2,l}^{2}}\right]\mathbf{g}(\mathbf{\mu})+\frac{1}{2(\nu-2)r_{2,l}^{2}}\times\right.$ \\
$\left.\sum\limits_{i=1}^{n}\left[\mathbf{g}(\mathbf{\mu}-r_{2,l}\sqrt{\nu\mathbf{\Sigma}}\mathbf{Q}_{l}\mathbf{e}_{i})+\mathbf{g}(\mathbf{\mu}+r_{2,l}\sqrt{\nu\mathbf{\Sigma}}\mathbf{Q}_{l}\mathbf{e}_{i})\right]\right\}$. \\
{\bfseries end for} \\
{\bfseries Outputs}: $I[\mathbf{g}]\approx I_{s}^{3}[\mathbf{g}]$. \\
\hline
\end{tabular}
\end{table}
%%%%%%%%%%%%%%%%%%%%%%%%%%%%%%%%%%%%%%
%%%%%%%%%%%%%%%%%%%%%%%%%%%%%%%%%%%%%%
%%%%%%%%%%%%%%%%%%%%%%%%%%%%%%%%%%%%%%
%%%%%%%%%%%%%%%%%%%%%%%%%%%%%%%%%%%%%%
\begin{table}[!t]
\renewcommand{\arraystretch}{2.0}
\caption{The implementation pseudocode for one time step of the proposed RSTSCF.}
\centering
\begin{tabular}[width=0.8in]{l}
\\
\hline
{\bfseries Inputs}: $\mathbf{\hat{x}}_{k-1|k-1}$, $\mathbf{P}_{k-1|k-1}$, $\mathbf{z}_{k}$, $\mathbf{Q}_{k-1}$, $\mathbf{R}_{k}$, $\nu_{1}$, $\nu_{2}$, $\nu_{3}$, $\mathbf{f}_{k-1}(\cdot)$, \\
$\mathbf{h}_{k}(\cdot)$, $n$, $N$. \\
{\bfseries Time update:} \\
1. $\mathbf{\hat{x}}_{k|k-1}=\mathrm{SSTSRCR}(\mathbf{\hat{x}}_{k-1|k-1},\mathbf{P}_{k-1|k-1},\nu_{3},\mathbf{f}_{k-1}(\cdot),n,N)$. \\
2. $\mathbf{P}_{k|k-1}=\frac{\nu_{3}-2}{\nu_{3}}\mathrm{SSTSRCR}(\mathbf{\hat{x}}_{k-1|k-1},\mathbf{P}_{k-1|k-1},\nu_{3},$ \\
$\mathbf{f}_{k-1}(\cdot)\mathbf{f}_{k-1}^{T}(\cdot),n,N)-\frac{\nu_{3}-2}{\nu_{3}}\mathbf{\hat{x}}_{k|k-1}\mathbf{\hat{x}}_{k|k-1}^{T}+\frac{\nu_{1}(\nu_{3}-2)}{(\nu_{1}-2)\nu_{3}}\mathbf{Q}_{k-1}$. \\
{\bfseries Measurement update:} \\
3. $\mathbf{\hat{z}}_{k|k-1}=\mathrm{SSTSRCR}(\mathbf{\hat{x}}_{k|k-1},\mathbf{P}_{k|k-1},\nu_{3},\mathbf{h}_{k}(\cdot),n,N)$. \\
4. $\mathbf{P}_{k|k-1}^{zz}=\frac{\nu_{3}-2}{\nu_{3}}\mathrm{SSTSRCR}(\mathbf{\hat{x}}_{k|k-1},\mathbf{P}_{k|k-1},\nu_{3},\mathbf{h}_{k}(\cdot)\mathbf{h}_{k}^{T}(\cdot),$ \\
$n,N)-\frac{\nu_{3}-2}{\nu_{3}}\mathbf{\hat{z}}_{k|k-1}\mathbf{\hat{z}}_{k|k-1}^{T}+\frac{\nu_{2}(\nu_{3}-2)}{(\nu_{2}-2)\nu_{3}}\mathbf{R}_{k}$. \\
5. $\mathbf{P}_{k|k-1}^{xz}=\frac{\nu_{3}-2}{\nu_{3}}\mathrm{SSTSRCR}(\mathbf{\hat{x}}_{k|k-1},\mathbf{P}_{k|k-1},\nu_{3},\mathbf{x}_{k}\mathbf{h}_{k}(\cdot),n,N)$ \\
$-\frac{\nu_{3}-2}{\nu_{3}}\mathbf{\hat{x}}_{k|k-1}\mathbf{\hat{z}}_{k|k-1}^{T}$. \\
6. $\Delta_{k}=\sqrt{(\mathbf{z}_{k}-\mathbf{\hat{z}}_{k|k-1})^{T}(\mathbf{P}_{k|k-1}^{zz})^{-1}(\mathbf{z}_{k}-\mathbf{\hat{z}}_{k|k-1})}$. \\
7. $\mathbf{K}_{k}=\mathbf{P}_{k|k-1}^{xz}(\mathbf{P}_{k|k-1}^{zz})^{-1}$. \\
8. $\mathbf{\hat{x}}_{k|k}=\mathbf{\hat{x}}_{k|k-1}+\mathbf{K}_{k}(\mathbf{z}_{k}-\mathbf{\hat{z}}_{k|k-1})$. \\
9. $\mathbf{P}_{k|k}=\frac{(\nu_{3}-2)(\nu_{3}+\Delta_{k}^{2})}{\nu_{3}(\nu_{3}+m-2)}(\mathbf{P}_{k|k-1}-\mathbf{K}_{k}\mathbf{P}_{k|k-1}^{zz}\mathbf{K}_{k}^{T})$. \\
{\bfseries Outputs}: $\mathbf{\hat{x}}_{k|k}$, $\mathbf{P}_{k|k}$. \\
\hline
\end{tabular}
\end{table}
%%%%%%%%%%%%%%%%%%%%%%%%%%%%%%%%%%%%%%
%%%%%%%%%%%%%%%%%%%%%%%%%%%%%%%%%%%%%%

According to the Monte Carlo approach, $I_{s}^{3}[\mathbf{g}]$ converges to $I[\mathbf{g}]$ when $N$ tends to infinity, i.e.,
\begin{equation}
\lim\limits_{N\rightarrow+\infty}I_{s}^{3}[\mathbf{g}]=I[\mathbf{g}].
\end{equation}

Thus, the proposed SSTSRCR provides asymptotically exact integral evaluations when $N$ tends to infinity. A new RSTSCF can be obtained by employing the proposed SSTSRCR to calculate the Student's t weighted integrals involved in the RSTNF, and the implementation pseudocode for one time step of the proposed RSTSCF is shown in Table II, where $\mathrm{SSTSRCR}(\cdot)$ denotes the proposed SSTSRCR algorithm. The proposed SSTSRCR can achieve better approximation to the Student's t weighted integral as compared with existing Student's t integral rules. As a result, the proposed RSTSCF has higher estimation accuracy than the existing RSTNFs.
\begin{remark}
The Monte Carlo approach can be also used to calculate the Student's t weighted integral, and it provides asymptotically exact integral evaluations when the number of random samples tends to infinity. However, it has low accuracy and slow convergence when the integrand is not approximately constant and the number of random samples is finite \cite{Genz 1998}. Fortunately, the proposed SSTSRCR is at least exact up to third-degree polynomials for any number of random samples, and it can capture more and more higher-degree moment information as the number of random samples increases.
\end{remark}


\subsection{Relationship between the proposed SSTSRCR and the existing SIR \cite{Dunik13}}
\begin{theorem}
The proposed SSTSRCR will degrade to the existing SIR when the dof parameter $\nu\rightarrow+\infty$, i.e.
\begin{eqnarray}
&&\lim_{\nu\rightarrow+\infty}I_{s}^{3}[\mathbf{g}]=\frac{1}{N}\sum\limits_{l=1}^{N}\left\{\left[1-\frac{n}{\rho_{l}^{2}}\right]\mathbf{g}(\mathbf{\mu})+\frac{1}{2\rho_{l}^{2}}\times\right. \nonumber\\
&&\left.\sum\limits_{i=1}^{n}\left[\mathbf{g}(\mathbf{\mu}-\rho_{l}\sqrt{\mathbf{\Sigma}}\mathbf{Q}_{l}\mathbf{e}_{i})+\mathbf{g}(\mathbf{\mu}+\rho_{l}\sqrt{\mathbf{\Sigma}}\mathbf{Q}_{l}\mathbf{e}_{i})\right]\right\},
\end{eqnarray}
where the right-hand side of the equation (53) is the SIR for the Gaussian weighted integral, and $\rho_{l}$ is drawn randomly from $p(\rho_{l})$ that is given by
\begin{equation}
p(\rho_{l})\propto\rho_{l}^{n+1}e^{-\frac{\rho_{l}^{2}}{2}}.
\end{equation}
\end{theorem}

\begin{proof}
Make a change of variable as follows
\begin{equation}
r_{2,l}=c(\rho_{l})=\frac{\rho_{l}}{\sqrt{\nu-2}}, \quad \mathrm{s.t.} \quad \nu\rightarrow+\infty.
\end{equation}

Substituting equation (55) in equation (51) results in
\begin{eqnarray}
&&I_{s}^{3}[\mathbf{g}]=\frac{1}{N}\sum\limits_{l=1}^{N}\left\{\left[1-\frac{n}{\rho_{l}^{2}}\right]\mathbf{g}(\mathbf{\mu})+\frac{1}{2\rho_{l}^{2}}\sum\limits_{i=1}^{n}\right. \nonumber\\
&&\left.\left[\mathbf{g}(\mathbf{\mu}-\rho_{l}\sqrt{\frac{\nu}{\nu-2}\mathbf{\Sigma}}\mathbf{Q}_{l}\mathbf{e}_{i})+\mathbf{g}(\mathbf{\mu}+\rho_{l}\sqrt{\frac{\nu}{\nu-2}\mathbf{\Sigma}}\mathbf{Q}_{l}\mathbf{e}_{i})\right]\right\}. \nonumber\\
\end{eqnarray}

Taking the limit of equation (56) when the dof parameter $\nu\rightarrow+\infty$, we can obtain equation (53).

Using the transformation theorem and equation (55), the PDF of random variable $\rho_{l}$ is given by
\begin{equation}
p(\rho_{l})=p_{r_{2}}(c(\rho_{l}))c'(\rho_{l}),
\end{equation}
where $p_{r_{2}}(\cdot)$ denotes the PDF of $r_{2}$ , and $c'(\rho_{l})$ denotes the derivative of $c(\rho_{l})$ with respect to $\rho_{l}$ given by
\begin{equation}
c'(\rho_{l})=\frac{1}{\sqrt{\nu-2}}.
\end{equation}

Substituting equations (28), (55) and (58) in equation (57), we obtain
\begin{eqnarray}
&&p(\rho_{l})=2\rho_{l}^{n+1}\lim_{\nu\rightarrow+\infty}\frac{1}{(\nu-2)^{\frac{n+2}{2}}\mathrm{B}(\frac{n+2}{2},\frac{\nu-2}{2})}\times \nonumber\\
&&\lim_{\nu\rightarrow+\infty}\left(1+\frac{\rho_{l}^{2}}{\nu-2}\right)^{-\frac{\nu+n}{2}}.
\end{eqnarray}

Utilizing the identity $\mathrm{B}(a,b)=\frac{\Gamma(a)\Gamma(b)}{\Gamma(a+b)}$, the first limit in equation (59) can be formulated as
\begin{eqnarray}
&&\lim_{\nu\rightarrow+\infty}\frac{1}{(\nu-2)^{\frac{n+2}{2}}\mathrm{B}(\frac{n+2}{2},\frac{\nu-2}{2})}=\frac{2^{-\frac{n+2}{2}}}{\Gamma(\frac{n+2}{2})}\times \nonumber\\
&&\lim_{\nu\rightarrow+\infty}\frac{\Gamma(\frac{\nu-2}{2}+\frac{n+2}{2})}{\Gamma(\frac{\nu-2}{2})(\frac{\nu-2}{2})^{\frac{n+2}{2}}}.
\end{eqnarray}

Using the property of Gamma function $\lim\limits_{t\rightarrow+\infty}\frac{\Gamma(t+\alpha)}{\Gamma(t)t^{\alpha}}=1$ in equation (60) gives
\begin{equation}
\lim_{\nu\rightarrow+\infty}\frac{1}{(\nu-2)^{\frac{n+2}{2}}\mathrm{B}(\frac{n+2}{2},\frac{\nu-2}{2})}=\frac{2^{-\frac{n+2}{2}}}{\Gamma(\frac{n+2}{2})}.
\end{equation}

The second limit in equation (59) can be reformulated as
\begin{eqnarray}
\lim_{\nu\rightarrow+\infty}\left(1+\frac{\rho_{l}^{2}}{\nu-2}\right)^{-\frac{\nu+n}{2}}&=&\lim\limits_{\nu\rightarrow+\infty}s(\nu)^{d(\nu)} \nonumber\\
&=&\lim_{\nu\rightarrow+\infty}s(\nu)^{\left[\lim\limits_{\nu\rightarrow+\infty}d(\nu)\right]},
\end{eqnarray}
where the functions $s(\nu)$ and $d(\nu)$ are given by
\begin{equation}
s(\nu)=\left(1+\frac{\rho_{l}^{2}}{\nu-2}\right)^{\frac{\nu-2}{\rho_{l}^{2}}}
\end{equation}
%%%%%%%%%%%%%%%%%%%%%%%%%%%%%%%%%%%%%
\begin{equation}
d(\nu)=-\frac{\rho_{l}^{2}(\nu+n)}{2(\nu-2)}.
\end{equation}

Using the identity $\lim\limits_{t\rightarrow+\infty}\left(1+\frac{1}{t}\right)^{t}=e$ and equations (63)-(64), equation (62) can be rewritten as
\begin{equation}
\lim_{\nu\rightarrow+\infty}\left(1+\frac{\rho_{l}^{2}}{\nu-2}\right)^{-\frac{\nu+n}{2}}=e^{-\frac{\rho_{l}^{2}}{2}}.
\end{equation}

Substituting equations (61) and (65) in (59), we can obtain (54), which proves the theorem.
\end{proof}

Considering that the Student's t PDF turns into the Gaussian PDF as the dof parameter $\nu\rightarrow+\infty$, we obtain
\begin{eqnarray}
\lim_{\nu\rightarrow+\infty}I[\mathbf{g}]&=&\int_{\mathbb{R}^{n}}\mathbf{g}(\mathbf{x})\lim_{\nu\rightarrow+\infty}\mathrm{St}(\mathbf{x};\mathbf{\mu},\mathbf{\Sigma},\nu)d\mathbf{x} \nonumber\\
&=&\int_{\mathbb{R}^{n}}\mathbf{g}(\mathbf{x})\mathrm{N}(\mathbf{x};\mathbf{\mu},\mathbf{\Sigma})d\mathbf{x}.
\end{eqnarray}

According to the Theorem 4 and equation (66), we can conclude that the proposed SSTSRCR with $\nu\rightarrow+\infty$ can be utilized to calculate the Gaussian weighted integral. Thus, the proposed SSTSRCR is a generalized SIR, which can calculate not only the Gaussian weighted integral but also the Student's t weighted integral.



\section{Simulation study}
In this simulation, the superior performance of the proposed RSTSCF as compared with existing filters is shown in the problem of manoeuvring bearing-only tracking observed in clutter. The target moves according to the continuous white noise acceleration motion model \cite{Dunik13}
\begin{equation}
\mathbf{x}_{k}=\mathbf{F}\mathbf{x}_{k-1}+\mathbf{G}\mathbf{w}_{k-1},
\end{equation}
where $\mathbf{x}_{k}=[x_{k}\;y_{k}\;\dot{x}_{k}\;\dot{y}_{k}]$, and $x_{k}$, $y_{k}$, $\dot{x}_{k}$ and $\dot{y}_{k}$ denote the cartesian coordinates and corresponding velocities respectively; $\mathbf{F}$ and $\mathbf{G}$ denote respectively the state transition matrix and noise matrix given by
\begin{equation}
\mathbf{F}=\left[\begin{array}{cc}
\mathbf{I}_{2} &\quad \Delta{t}\mathbf{I}_{2}\\
\mathbf{0} &\quad \mathbf{I}_{2}
\end{array}\right]\qquad
\mathbf{G}=\left[\begin{array}{cc}
\mathbf{\Gamma} &\quad \mathbf{0}_{2\times1}\\
\mathbf{0}_{2\times1} &\quad \mathbf{\Gamma}
\end{array}\right],
\end{equation}
where $\Delta{t}=1\mathrm{min}$ is the sampling interval, and $\mathbf{I}_{2}$ is the two dimensional identity matrix, and $\mathbf{0}_{2\times1}$ is the two dimensional zero vector, and $\mathbf{\Gamma}=[0.5\Delta{t}^{2} \;\; \Delta{t}]^{T}$.

The target is observed by an angle sensor installed in a manoeuvring platform. If the platform is located at $(x_{k}^{p},y_{k}^{p})$ at time $k$, then the measurement model is given by
\begin{equation}
z_{k}=\mathrm{tan}^{-1}(\frac{y_{k}-y_{k}^{p}}{x_{k}-x_{k}^{p}})+v_{k},
\end{equation}
where $z_{k}$ is the angle between the target and the platform at time $k$. Outlier corrupted process and measurement noises are generated according to \cite{student's t filter 2013,TAES 2017 RSTKF,TAES 2016}
\begin{equation}
\mathbf{w}_{k}\sim
\left\{\begin{array}{ll}
N(\mathbf{0},\mathbf{\Sigma}_{w}) \quad & \mathrm{w.p.} \:\: 0.95 \\
N(\mathbf{0},100\mathbf{\Sigma}_{w}) \quad & \mathrm{w.p.} \:\: 0.05
\end{array}\right.
\end{equation}
%%%%%%%%%%%%%%%%%%%%%%%
\begin{equation}
{v}_{k}\sim
\left\{\begin{array}{ll}
N({0},\Sigma_{v}) \quad & \mathrm{w.p.} \:\: 0.95 \\
N({0},50\Sigma_{v}) \quad & \mathrm{w.p.} \:\: 0.05
\end{array}\right.,
\end{equation}
where $\mathrm{w.p.}$ denotes ``with probability'', the nominal process noise covariance matrix is $\mathbf{\Sigma}_{w}=10^{-6}\mathbf{I}_{2}\mathrm{km}^{2}/\mathrm{min}^{2}$, and the nominal measurement noise variance is $\Sigma_{v}=(0.02\mathrm{rad})^{2}$. Process and measurement noises, which are generated according to equations (70)-(71), have heavier tails.

In our simulation scenario, the initial positions of the target and the platform are respectively $(3\mathrm{km},3\mathrm{km})$ and $(0\mathrm{km},0\mathrm{km})$. The target moves at a constant speed of 180 knots (1 knot is $1.852\mathrm{km}/\mathrm{h}$) with a course of $-135.4^{\circ}$. The platform moves at a constant speed of 50 knots with a initial course of $-80^{\circ}$, and the course reaches $146^{\circ}$ at time $k=15\mathrm{min}$ by executing a manoeuvre \cite{Dunik13}. The initial estimation error covariance matrix is $\bm{P}_{0|0}=\mathrm{diag}[16\mathrm{km}^{2}\,16\mathrm{km}^{2},4\mathrm{km}^{2}/\mathrm{min}^{2},$ $4\mathrm{km}^{2}/\mathrm{min}^{2}]$, and the initial state estimate $\bm{\hat{x}}_{0|0}$ is chosen randomly from $\mathrm{N}(\bm{x}_{0},\bm{P}_{0|0})$, where $\bm{x}_{0}$ denotes the initial true state.

In this simulation, the stochastic integration filter (SIF) \cite{Dunik13}, the ARUKF with free parameter $\kappa=0$ \cite{ARUKF 2014}, the MCCKF with kernel size $\sigma=5$ \cite{MCCKF 2016}, the RSTEF \cite{student's t filter 2013}, the 3rd-degree RSTUF with free parameter $\kappa=3-n$ \cite{TAES 2016}, the 3rd-degree RSTCF \cite{Fusion 2016}, the fifth-degree RSTUF \cite{Fusion 2016 Sarkka}, the robust Student's t based Monte Carlo filter (RSTMCF), the Gaussian sum-cubature Kalman filter (GSCKF) \cite{GSF 2007}, the PF \cite{Arulampalam 2002}, and the proposed RSTSCF are tested. In the RSTMCF, the Student's t weighted integral is calculated using the conventional Monte Carlo approach with $10000$ random samples. In the GSCKF, the process and measurement noises are modelled as $p(\mathbf{w}_{k})=\sum\limits_{i=1}^{5}\alpha_{i}\mathrm{N}\left(\mathbf{w}_{k};\mathbf{0}, \lambda_{i}\mathbf{\Sigma}_{w}\right)$ and $p(v_{k})=\sum\limits_{i=1}^{5}\alpha_{i}\mathrm{N}\left(v_{k};0, \lambda_{i}\Sigma_{v}\right)$, where the weights $\alpha_{1}=0.8$ and $\alpha_{2}=\alpha_{3}=\alpha_{4}=\alpha_{5}=0.05$, and the scale parameters $\lambda_{1}=1$, $\lambda_{2}=50$, $\lambda_{3}=100$, $\lambda_{4}=500$ and $\lambda_{5}=1000$. Moreover, to prevent the computational complexity of the GSCKF increasing exponentially as the time, the posterior distribution is approximated as a weighted sum of five Gaussian terms with the highest weights. In the PF, the process and measurement noises are modelled as Student's t distributions, and the number of particle is chosen as $10000$. In the existing RSTEF, 3rd-degree RSTUF, 3rd-degree RSTCF, fifth-degree RSTUF, RSTMCF, PF and the proposed RSTSCF, the dof parameters are all chosen as $\nu_{1}=\nu_{2}=\nu_{3}=5$ and the scale matrices are all set as $\mathbf{Q}_{k}=\mathbf{\Sigma}_{w}$ and ${R}_{k}={\Sigma}_{v}$. In the SIF and the proposed RSTSCF, the number of random samples is selected as $N=100$. The proposed filter and existing filters are coded with MATLAB and the simulations are run on a computer with Intel Core i7-3770 CPU at 3.40 GHz.

To compare the performances of the proposed filter and existing filters, the RMSEs and the averaged RMSEs (ARMSEs) of the position and velocity are chosen as performance metric. The RMSE and ARMSE in position are respectively defined as
%%%%%%%%%%%%%%%%%%%%%%%%%%%%%%%%%%%
\begin{equation}
\mathrm{RMSE}_{\mathrm{pos}}(\emph{k})=\sqrt{\frac{1}{M}\sum\limits_{\emph{s}=1}^{M}\left(\left(x_{k}^{\emph{s}}-\hat{x}_{k}^{\emph{s}}\right)^{2}+\left(y_{\emph{k}}^{\emph{s}}-\hat{y}_{\emph{k}}^{\emph{s}}\right)\right)^{2}}
\end{equation}
%%%%%%%%%%%%%%%%%%%%%%%%%%%%%%%%%%%
\begin{equation}
\mathrm{ARMSE}_{\mathrm{pos}}=\sqrt{\frac{1}{MT}\sum\limits_{k=1}^{T}\sum\limits_{\emph{s}=1}^{M}\left(\left(x_{\emph{k}}^{\emph{s}}-\hat{x}_{\emph{k}}^{\emph{s}}\right)^{2}+\left(y_{\emph{k}}^{\emph{s}}-\hat{y}_{\emph{k}}^{\emph{s}}\right)\right)^{2}},
\end{equation}
%%%%%%%%%%%%%%%%%%%%%%%%%%%%%%%%%%%
where $M=1000$ denotes the number of Monte Carlo runs, and $T=100\mathrm{min}$ denotes the simulation time, and $(x_{k}^{s},y_{k}^{s})$ and $(\hat{x}_{k}^{s},\hat{y}_{k}^{s})$ respectively denote the true and estimated positions at the $s$-th Monte Carlo run. Similar to the RMSE and ARMSE in position, we can also formulate the RMSE and ARMSE in velocity.
%%%%%%%%%%%%%%%%%%%%%%%%%%%%%%%%%%%%%%%%%%%%%%%%
\begin{table}[!t]
\renewcommand{\arraystretch}{2.0}
\caption{ARMSEs and implementation times of the proposed filter and existing filters.}
\centering
\begin{tabular}{cccc}
\hline
Filters              & $\mathrm{ARMSE_{pos}}$ (km) & $\mathrm{ARMSE_{vel}}$ (km/min)   &  Time (ms)  \\
\hline
SIF                  &     22.85                   &    0.83                           &   45.46     \\
ARUKF                &     $2.57\times10^{25}$     &    $2.41\times10^{24}$            &  $0.41$     \\
MCCKF                &     26.86                   &    0.88                           &  0.13        \\
RSTEF                &     $3.51\times10^{6}$      &    $6.51\times10^{4}$             &  $0.09$      \\
3rd RSTUF            &     88.86                   &    2.43                           &  $0.53$      \\
3rd RSTCF            &     29.90                   &    1.12                           &  $0.50$       \\
5th RSTUF            &     16.29                   &    0.65                           &  $1.0$        \\
RSTMCF               &     53.65                   &    1.65                           &  193.7        \\
GSCKF                &     19.23                   &    0.58                           &  54.6         \\
PF                   &     7.89                    &    0.30                           &  773.0        \\
RSTSCF               &     12.08                   &    0.56                           &  48.2         \\
\hline
\end{tabular}
\end{table}
%%%%%%%%%%%%%%%%%%%%%%%%%%%%%%%%%%%%%%%%%%%%%%%%
%%%%%%%%%%%%%%%%%%%%%%%%%%%%%%%%%%%%%%%%%%%%%%%%
\begin{figure} [!t]
\centering
\includegraphics[height=2.5in,width=3.25in]{01.eps}
\caption{RMSEs of the position from the proposed filter and existing filters}
\end{figure}
%%%%%%%%%%%%%%%%%%%%%%%%%%%%%
\begin{figure} [!t]
\centering
\includegraphics[height=2.5in,width=3.25in]{02.eps}
\caption{RMSEs of the velocity from the proposed filter and existing filters}
\end{figure}
%%%%%%%%%%%%%%%%%%%%%%%%%%%%%%%%%%%%%%%%%%%%%%%%

The RMSEs and ARMSEs of position and velocity from the proposed filter and existing filters are respectively shown in Fig. 1--Fig. 2 and Table III. The implementation times of the proposed filter and existing filters in single step run are given in Table III. Note that the existing ARUKF and RSTEF diverge in the simulation, as shown in Fig. 1--Fig. 2 and Table III.

It is seen from Fig. 1--Fig. 2 and Table III that the RMSEs and ARMSEs of the proposed RSTSCF are smaller than the existing SIF, ARUKF, MCCKF, RSTEF, 3rd-degree RSTUF, 3rd-degree RSTCF, 5th-degree RSTUF, RSTMCF, GSCKF but larger than the existing PF. Furthermore, it can be also seen from Table III that the implementation time of the proposed RSTSCF are greater than the existing SIF, ARUKF, MCCKF, RSTEF, 3rd-degree RSTUF, 3rd-degree RSTCF, 5th-degree RSTUF but significantly smaller than the existing RSTMCF, GSCKF and PF. Therefore, the proposed RSTSCF has better estimation accuracy than the existing SIF, ARUKF, MCCKF, RSTNFs and GSCKF, and is computationally much more efficient than the existing PF.



\section{Conclusion}
In this paper, a new SSTSRCR was derived based on the third-degree USR and the proposed third-degree URR, from which a new RSTSCF was obtained. The existing SIR is a special case of the proposed SSTSRCR when the dof parameter tends to infinity. Simulation results for a manoeuvring bearings-only tracking example illustrated that the proposed RSTSCF can achieve higher estimation accuracy than the existing GA filter, GSF, HNKF, MCCKF and RSTNFs, and is computationally much more efficient than the existing PF.



\begin{thebibliography}{10}

\bibitem{Anderson79}
B. D. O. Anderson and J. B. Moore, \emph{Optimal filtering, Englewood Cliffs}, NJ: Prentice Hall, 1979.


\bibitem{CGAF 2015}
Y. L. Huang, Y. G. Zhang, X. X. Wang, and L. Zhao, ``Gaussian filter for nonlinear systems with correlated noises at the same epoch,'' \emph{Automatica}, vol. 60, no. 10, pp. 122--126, oct. 2015.


\bibitem{Arasaratnam 2009}
I. Arasaratnam and S. Haykin, ``Cubature Kalman filter,'' \emph{IEEE Trans. Autom. Control}, vol. 54, no. 6, pp. 1254$-$1269, June 2009.


\bibitem{ICKF 2015}
Y. G. Zhang, Y. L. Huang, N. Li, and L. Zhao. ``Interpolatory cubature Kalman filters,'' \emph{IET Control Theory \& Appl.}, vol. 9, no. 11, pp. 1731--1739, Sep. 2015.


\bibitem{Ito 2000}
K.~Ito and K.~Xiong, ``Gaussian filters for nonlinear filtering problems,'' \emph{IEEE Trans. Autom. Control}, vol. 45, no. 5, pp. 910$-$927, May 2000.


\bibitem{UKF 2004}
S.~J. Julier and J.~K. Uhlman, ``Unscented filtering and nonlinear estimation,'' \emph{Proc. IEEE}, vol. 92, no. 3, pp. 401$-$422, Mar. 2004.


\bibitem{High-degree CKF 2013}
B. Jia, M. Xin, and Y. Cheng, ``High-degree cubature Kalman filter,'' \emph{Automatica}, vol. 49, no. 2, pp. 510$-$518, Feb. 2013.


\bibitem{Dunik13}
J. Dun\'{\i}k, O. Straka, and M. \v{S}imandl, ``Stochastic integration filter,'' \emph{IEEE Trans. Automat. Control}, vol. 58, no. 6, pp. 1561--1566, June 2013.


\bibitem{ECKF 2015}
Y. G. Zhang, Y. L. Huang, N. Li, and L. Zhao. ``Embedded cubature Kalman filter with adaptive setting of free parameter,'' \emph{Signal Process.}, vol. 114, no. 9, pp. 112--116, Sep. 2015.


\bibitem{student's t filter 2013}
M. Roth, E. \"{O}zkan, and F. Gustafsson, ``A Student's t filter for heavy-tailed process and measurement noise,'' in \emph{2013 IEEE International Conference on Acoustics, Speech and Signal Processing (ICASSP)}, May 2013, pp. 5770--5774.


\bibitem{Robust GA smoother 2016}
Y. L. Huang, Y. G. Zhang, N. Li, and J. Chambers, ``A robust Gaussian approximate fixed-interval smoother for nonlinear systems with heavy-tailed process and measurement noises,'' \emph{IEEE Signal Proc. Let.}, vol. 23, no. 4, pp. 468--472, Apr. 2016.


\bibitem{TAES 2017 RSTKF}
Y. L. Huang, Y. G. Zhang, Z. M. Wu, N. Li, and J. Chambers, ``A novel robust Student's t based Kalman filter,'' to appear in \emph{IEEE Trans. Aero. Elec. Sys.}, 2017.


\bibitem{Arulampalam 2002}
S. Arulampalam, S. Maskell, N. Gordon, and T. Clapp, ``A tutorial on particle filters for on-line non-linear/non-Gaussian Bayesian tracking,'' \emph{IEEE Trans. Signal Process.}, vol. 50, no. 2, pp. 174--189, Feb. 2002.


\bibitem{STPF 1993}
N. Gordon and A. Smith, ``Approximate non-Gaussian Bayesian estimation and modal consistency,'' \emph{J. R. Stat. Soc. B.}, vol. 55, no. 4, pp. 913--918, Apr. 1993.


\bibitem{STPF 2006}
S. J. Li, H. Wang, and T. Y. Chai, ``A t-distribution based particle filter for target tracking,'' in \emph{American Control Conference}, June 2006, pp. 2191--2196.


\bibitem{visual tracking 2008}
J. Loxam and T. Drummond, ``Student mixture filter for robust, realtime visual tracking,'' \emph{in Proceedings of 10th European Conference on Computer Vision: Part III}, 2008.


\bibitem{GSF 1972}
D. L. Alspach and H. Sorenson, ``Non-linear Bayesian estimation using Gaussian sum approximation,'' \emph{IEEE Trans. Autom. Control}, vol. 17, no. 4, pp. 439--448, Apr. 1972.


\bibitem{GSF 2007}
I. Arasaratnam, S. Haykin, and R. Elliott, ``Discrete-time nonlinear filtering algorithms using Gauss-Hermite quadrature,'' \emph{Proc. IEEE}, vol. 95, no. 5, pp. 953$-$977, May 2007.


\bibitem{GSF 2010}
I. Bilik and J. Tabrikian, ``MMSE-based filtering in presence of non-Gaussian system and measurement noise,'' \emph{IEEE Trans. Aero. Elec. Sys.}, vol. 46, no. 3, pp. 1153--1170, July 2010.


\bibitem{Nonlinear Huber estimator 2010}
X. G. Wang, N. G. Cui, and J. F. Guo, ``Huber-based unscented fltering and its application to vision-based relative navigation,'' \emph{IET Radar, Sonar, and Navigation}, vol. 4, no. 1, pp. 134--141, Jan. 2010.


\bibitem{Huber estimator 1995}
F. El-Hawary and Y. Jing, ``Robust regression-based EKF for tracking underwater targets,�� \emph{IEEE J. Oceanic Eng.}, vol. 20, no. 1, pp. 31--41, Jan. 1995.


\bibitem{Huber estimator 2007}
C. D. Karlgaard, and H. Schaub, ``Huber-based divided difference filtering,'' \emph{J. Guid., Control, Dyn.}, vol. 30, no. 3, pp. 885--891, Apr. 2007.


\bibitem{Nonlinear Huber estimator 2012}
L. B. Chang, B. Q. Hu, G. B. Chang, and A. Li, ``Multiple outliers suppression derivative-free filter based on unscented transformation,'' \emph{J. Guid., Control, Dyn.}, vol. 35, no. 6, pp. 1902--1906, June 2012.


\bibitem{Nonlinear regression HKF 2015}
C. D. Karlgaard, Nonlinear regression Huber�CKalman filtering and fixed-interval smoothing, \emph{J. Guid., Control, Dyn.}, vol. 38, no. 2, pp. 322--330, Feb. 2015.


\bibitem{ARUKF 2014}
Y. D. Wang, S. M. Sun, and L. Li, ``Adaptively robust unscented Kalman filter for tracking a maneuvering vehicle,'' \emph{J. Guid., Control, Dyn.}, vol. 37, no. 5, pp. 1696--1701, May 2014.


\bibitem{MCCKF 2012b}
B. Chen and J. C. Principe, ``Maximum correntropy estimation is a smoothed MAP estimation,'' \emph{IEEE Signal Proc. Let.}, vol. 19, no. 8, pp. 491--494, Aug. 2012.


\bibitem{MCCKF 2015a}
B. Chen, J. Wang, H. Zhao, N. Zheng, and J. C. Principe, ``Convergence of a fixed-point algorithm under maximum correntropy criterion,'' \emph{IEEE Signal Proc. Let.}, vol. 22, no. 10, pp. 1723--1727, Oct. 2015.


\bibitem{MCCKF 2015b}
Y. D. Wang, W. Zheng, S. M. Sun, and L. Li, ``Robust information filter based on maximum correntropy criterion,'' \emph{J. Guid., Control, Dyn.}, vol. 39, no. 5, pp. 1126--1131, May 2016.


\bibitem{MCCKF 2016}
R. Izanloo, S. A. Fakoorian, H. S. Yazdi, and D. Simon, ``Kalman filtering based on the maximum correntropy criterion in the presence of non-Gaussian noise,'' in Proceedings of 50th Annual Conference on Information Science and Systems, 2016.


\bibitem{TAES 2016}
Y. L. Huang, Y. G. Zhang, N. Li, and J. Chambers, ``Robust Student's t based nonlinear filter and smoother,'' \emph{IEEE Trans. Aero. Elec. Sys.}, vol. 52, no. 5, pp. 2586--2596, Oct. 2016.


\bibitem{Fusion 2016 Sarkka}
F. Tronarp, R. Hostettler, and S. S\"{a}rkk\"{a}, ``Sigma-point filtering for nonlinear systems with non-additive heavy-tailed noise,'' in \emph{19th International Conference on Information Fusion (FUSION)}, July 2016, pp. 1859-1866.


\bibitem{Fusion 2016}
Y. L. Huang, Y. G. Zhang, N. Li, S. M. Naqvi, and J. Chambers, ``A robust Student's t based cubature filter,'' in \emph{19th International Conference on Information Fusion (FUSION)}, July 2016, pp. 9--16.


\bibitem{Genz 1998}
A. Genz and J. Monahan, ``Stochastic integration rules for infinite regions,'' \emph{SIAM J. Sci. Comput.}, vol. 19, no. 2, pp. 426--439, Feb. 1998.


\bibitem{Stroud 1971}
A. H. Stroud, \emph{Approximate calculation of multiple integrals}, Englewood Cliffs, NJ: Prentice Hall, 1971.


\bibitem{Stewart 1980}
G. W. Stewart, ``The efficient generation of random orthogonal matrices with an application to condition estimation,'' \emph{SIAM J. Numer. Anal.}, vol. 17, no. 3, pp. 403--409, Mar. 1980.


\end{thebibliography}



%\begin{IEEEbiography}[{\includegraphics[width=1in,height=1.25in,clip,keepaspectratio]{Yulong}}]{Yulong Huang}
%received the B.S. degree from the Department of Automation, Harbin Engineering University, Harbin, China, in 2012, and is currently working towards a Ph.D degree in control science and engineering. Since Nov. 2016, he has been a visiting graduate researcher at the electrical engineering department of Columbia University, New York, USA. His current research interests include signal processing, information fusion and their applications in navigation technology, such as inertial navigation and integrated navigation.
%\end{IEEEbiography}



%\begin{IEEEbiography}[{\includegraphics[width=1in,height=1.25in,clip,keepaspectratio]{Yonggang}}]{Yonggang Zhang}
%received the B.S. and M.S. degrees from the Department of Automation, Harbin Engineering University, Harbin, China, in 2002 and 2004, respectively. He received his Ph.D. degree in Electronic Engineering from Cardiff University, UK in 2007 and worked as a Post-Doctoral Fellow at Loughborough University, UK from 2007 to 2008 in the area of adaptive signal processing. Currently, he is a Professor of navigation, guidance, and control in Harbin Engineering University (HEU) in China. His current research interests include signal processing, information fusion and their applications in navigation technology, such as fiber optical gyroscope, inertial navigation and integrated navigation.
%\end{IEEEbiography}





% if have a single appendix:
%\appendix[Proof of the Zonklar Equations]
% or
%\appendix  % for no appendix heading
% do not use \section anymore after \appendix, only \section*
% is possibly needed

% use appendices with more than one appendix
% then use \section to start each appendix
% you must declare a \section before using any
% \subsection or using \label (\appendices by itself
% starts a section numbered zero.)
%





% trigger a \newpage just before the given reference
% number - used to balance the columns on the last page
% adjust value as needed - may need to be readjusted if
% the document is modified later
%\IEEEtriggeratref{8}
% The "triggered" command can be changed if desired:
%\IEEEtriggercmd{\enlargethispage{-5in}}

% references section

% can use a bibliography generated by BibTeX as a .bbl file
% BibTeX documentation can be easily obtained at:
% http://www.ctan.org/tex-archive/biblio/bibtex/contrib/doc/
% The IEEEtran BibTeX style support page is at:
% http://www.michaelshell.org/tex/ieeetran/bibtex/
%\bibliographystyle{IEEEtran}
% argument is your BibTeX string definitions and bibliography database(s)
%\bibliography{IEEEabrv,../bib/paper}
%
% <OR> manually copy in the resultant .bbl file
% set second argument of \begin to the number of references
% (used to reserve space for the reference number labels box)






% biography section
%
% If you have an EPS/PDF photo (graphicx package needed) extra braces are
% needed around the contents of the optional argument to biography to prevent
% the LaTeX parser from getting confused when it sees the complicated
% \includegraphics command within an optional argument. (You could create
% your own custom macro containing the \includegraphics command to make things
% simpler here.)
%\begin{IEEEbiography}[{\includegraphics[width=1in,height=1.25in,clip,keepaspectratio]{mshell}}]{Michael Shell}
% or if you just want to reserve a space for a photo:


\end{document}


