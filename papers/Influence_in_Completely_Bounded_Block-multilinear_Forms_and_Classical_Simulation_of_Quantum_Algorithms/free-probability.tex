\section{Free Probability Primer}
\label{sec:app}

There are many excellent books on free probability theory. In particular, we refer to the book \cite{NS06} for more details than the brief introduction given here.

\subsection{Preliminaries}
\label{sec:free}



\paragraph{$C^*$-algebras.} Let $\CA$ be a unital $C^*$-algebra. For our purposes, we can think of this as an algebra of bounded operators on a complex Hilbert space which is self-adjoint ($a \in \CA$
 implies $a^* \in \CA$), closed in the operator norm $\|\cdot\|$, and contains the identity ($\bone \in \CA$). A faithful trace $\phi$ on $\CA$ is a continuous linear functional $\phi: \CA \to \C$ that is  unital ($\phi(\bone)=1$), positive $\phi(aa^*) \ge 0$, and $\phi(aa^*) = 0$ iff $a=0$. 
 
 The pair $(\CA, \phi)$ where $\CA$ is a unital $C^*$-algebra and $\phi$ is a faithful trace on $\CA$ is called a $C^*$-\emph{probability space}. Elements of $\CA$ are called non-commutative random variables. An example of a $C^*$-probability space is the class $(M_n(\C), \tr_n)$, which is the class of $n \times n$ complex matrices with the normalized trace  functional defined as $\tr_n(M) = \frac{1}{n} \sum_{i=1}^n {M_{ii}}$. General $C^*$-probability spaces allow us to extend these definitions to infinite-dimensional operators, which are needed to define a non-commutative analog of independence called \emph{free independence}. Faithfulness of the trace $\phi$ then ensures that $\|a\|= \lim_{m \to \infty} \phi((aa^*)^m)^{1/2m}$ (see \cite[Proposition 3.17]{NS06}). In particular, this allows one to compute the norm $\|\cdot\|$ by using the trace method and taking higher powers of the trace functional $\phi$, as we will see below.


\paragraph{Free Independence.} Let $(\CA, \phi)$ be a $C^*$-probability space and let $\{\CA_i\}_{i=1}^n$ be unital $*$-subalgebras of  $\CA$. They are said to be \emph{free} (or \emph{freely independent}) if for all $k \in [n]$, for all indices $i_1, \ldots, i_k \in [n]$, and for all $a_1 \in \CA_{i_1}, \ldots, a_k \in \CA_{i_k}$ satisfying $\phi(a_1)=\ldots =\phi(a_k)=0$, the joint \emph{free moment},
\[ \phi(a_1 \cdots a_k) = 0\]
whenever $j_1 \neq j_2, j_2\neq j_3, \ldots, j_{k-1} \neq j_k$, that is, the free moments vanish when all the neighboring elements in the sequence $a_1, \ldots, a_k$ come from  subalgebras with distinct indices, for example, $\phi(a_1a_2a^*_1a^*_2a_3a_2)=0$.

{Non-commutative random variables $a_1, \ldots, a_n \in (\CA, \phi)$ are said to be free if the subalgebras $\{\CA_i\}_{i=1}^n$ are free, where $\CA_i$ is the unital $*$-subalgebra  generated by $a_i$ (the linear span of all monomials $a^{\eps_1}_ia^{\eps_2}_i\cdots a^{\eps_r}_i$ where $\eps_1, \ldots, \eps_r \in \{1,*\}$ and $r \in \N \cup \{0\}$). Note that the corresponding unital $C^*$-subalgebras obtained by taking the norm closure of each $\CA_i$ are also freely independent in this case (see \cite[Exercise 5.23]{NS06}).}



We remark that the set of free non-commutative random variables is an empty set if the underlying $C^*$-probability space is finite (for instance $(M_n(\C), \tr_n)$), so to find non-trivial examples one needs to work with infinite-dimensional $C^*$-probability spaces. 

\paragraph{Free Haar Unitaries and Free Groups.} Let $(\CA, \phi)$ be a $C^*$-probability space. An element $u \in \CA$ is a \emph{Haar unitary} if it is a unitary, i.e.\ $uu^*=u^*u= \bone$, and if $\phi(u^k) = 0$ for all non-zero integers $k$. A family $S = \{u_1, \ldots, u_n\} \in \CA$ in a $C^*$-probability space $(\CA, \phi)$ is called a \emph{free Haar unitary family} if each $u \in S$ is a Haar unitary and if $u_1, \ldots, u_n$ are free. For notational convenience, let us define $S^* = \{u^*_1, \ldots, u^*_n\}$ to be the set of corresponding adjoints.

One can give a very precise condition when the trace $\phi$ evaluated on a non-commutative monomial in the $u_i$'s vanishes in terms of the free group. The \emph{free group} $F_n$ with generating set $S$ is an infinite discrete group constructed as follows: a word is defined to be product of elements of $S \cup S^*$ with $\bot$ denoting the empty word that contains no symbols. A word is called reduced if it does not contain a sub-word of the form $g g^{*}$ or $g^{*} g$ for $g \in S$. Given a word that is not reduced, the process of repeatedly removing such sub-words until it becomes reduced is called reduction. The free group $F_n$ consists of all reduced words that can be built from the symbols in $S \cup S^*$ with the group operation being a product of words followed by reduction. The identity is the empty word $\bot$. 

For a $d$-tuple $\bi  = (i_1, \ldots, i_d) \in [m]^d$, let $u_{\bi}$ denote the non-commutative monomial $u_{i_1}\cdots u_{i_d}$ and write $u^*_{\bi} = (u_{\bi})^* = u^*_{i_d} \cdots u^*_{i_1}$. Let $\bi_1,\ldots, \bi_t, \bj_1, \ldots, \bj_t$ each be a $d$-tuple in $[m]^d$ and consider the degree-$2td$ non-commutative monomial $w = u^{}_{\bi_1}u^*_{\bj_1} u^{}_{\bi_2}u^*_{\bj_2}\cdots u^{}_{\bi_t}u^*_{\bj_t}$. Note that a degree-$2td$ monomial $w$ corresponds to an ordered $2td$-tuple of variables. To illustrate, if $t=1, m=3$ and $\bi_1=(1,2,3)$ and $\bj_1 = (2,2,1)$, then $w = u_{1}u_2u_3(u_2u_2u_1)^* = u_{1}u_2u_3u_1^*u^*_2u^*_2$ and corresponds to the ordered tuple $(u_1, u_2, u_3, u^*_1, u^*_2, u_2^*)$. We can also interpret $w$ as a word in the free group by applying the reduction rules. Then the next proposition follows from the definitions of free independence and Haar unitaries.

\begin{proposition} \label{prop:word}
    $\phi(w) = 1$ iff $w$ reduces to identity in the free group $F_n$, and $\phi(w) = 0$ otherwise.
\end{proposition}

 For a monomial $w$ that reduces to identity in the free group, the procedure for reducing a monomial $w$ as above first removes some adjacent pair $u_k$ (at index $i$) and $u^*_k$ (at index $j)$, then removes another adjacent pair $u_l$ and $u^*_l$ in the resulting word and so on and so forth until we reach the empty word. In particular, this reduction procedure produces a pairing of the set $[2td]$ where the index $i$ and $j$ are paired up iff the  variables at indices $i$ and $j$ in the monomial $w$ are $u_k$ and $u^*_k$ (for some $k$). Moreover, this pairing is what is called a \emph{non-crossing} pairing defined below (see \figref{fig:non-crossing}). Note that a monomial could be reduced to identity in different ways, so there could be many such non-crossing pairings for a given monomial $w$.



%\vspace{5mm}
\begin{figure}[h!]
    \centering
   \includegraphics[width=0.6\textwidth]{noncrossing.pdf}
   \caption{\footnotesize A non-crossing $*$-pairing resulting from the reduction of a word to identity in the free group}
    \label{fig:non-crossing}
\end{figure}

\paragraph{Non-crossing Pairings.}  For any even integer $n$, let $\CP_2(n)$ denote the set of all pairings of $n$, that is, the set of all partitions of $[n]$ where each block is of size two. Let $\NC_2(n) \subseteq \CP_2(n)$ denote the set of all  pairings of $[n]$ that are non-crossing,\emph{ i.e.} pairings  which do not contain blocks $\{i_1,i_3\}, \{i_2, i_4\}$ such that $i_1 < i_2 < i_3 < i_4$. 

For integers $d,m$, we divide the set $[2dm]$ into $2m$ consecutive blocks of $d$ elements each and color consecutive blocks alternatively with red and blue. Formally, for $i \in [2m]$, the elements $\{(i-1)d+1,\ldots, id\}$ are colored red if $i$ is odd and blue if $i$ is even. We define $\NC_2^*(d,m) \subseteq \NC_2(2dm)$ to be the set of those non-crossing pairings of $[2dm]$ which only pair up elements of different colors. We call any pairing in $\NC_2^*(d,m)$ a $*$-pairing.

We shall need the following combinatorial fact about the number of $*$-pairings (see \cite[Corollary 3.2]{KS05}).

\begin{lemma}\label{lem:catalan-fuss}
    For all $d, m$, the number of $*$-pairings  $|\NC_2^*(d,m)|$ equals the Fuss-Catalan number
    \[ 
    C_{d,m} = \frac{1}{m}\binom{m(d+1)}{m-1} = O\left(\frac{(d+1)^{m(d+1)}}{\left(d+\frac1m\right)^{md+1}}\right).
    \] 
\end{lemma}




\subsection{Proofs of \lref{thm:trace} and \thmref{thm:op-norm}}



\begin{proof}[Proof of \lref{thm:trace}]
    Writing $u^*_\bi = (u_\bi)^*$ for a tuple $\bi$ and using linearity of $\phi$, we have that 
    \[ 
    \phi[p(u_1, \ldots, u_t) (p(u_1, \ldots, u_t))^*] =  \sum_{|\bi|,|\bj| \le d} c_{\bi}c_{\bj} \phi(u_{\bi}u^*_{\bj}).
    \]
    From \pref{prop:word}, the term $\phi(u_{\bi}u^*_{\bj})$ is 1 iff $u_{\bi}u^*_{\bj}$ reduces to identity in the free group $F_t$ with generators $u_1, \ldots, u_t$. For the right-hand side above, this only happens when $\bi=\bj$ and thus these are the only non-zero terms. Thus, 
    \[ \phi[p(u_1, \ldots, u_t) (p(u_1, \ldots, u_t))^*] =  \sum_{|\bi| \le d} |c_{\bi}|^2. \qedhere\]
\end{proof}

Below we present the argument of Kemp and Speicher \cite{KS05}. Our exposition follows their proof closely but we adapt it to our context.
\begin{proof}[Proof of \thmref{thm:kemp-speicher}]
     We have that $\|p\| = \lim_{m \to \infty} \left(\phi((pp^*)^m)\right)^{1/(2m)}$ by the faithfulness of the trace $\phi$. Writing $u^*_\bj = (u_\bj)^*$ for a tuple $\bj$, we can compute 
    \begin{align*}
         \phi((pp^*)^m) = \sum_{\substack{|\bi_1|=\ldots=|\bi_m|=d\\|\bj_1|=\ldots=|\bj_m|=d}} c_{\bi_1}\cdots c_{\bi_m}c_{\bj_1}\cdots c_{\bj_m} \phi(u^{}_{\bi_1}u^*_{\bj_1} \cdots u^{}_{\bi_m}u^*_{\bj_m}).
    \end{align*}
    
    Since $u_1, \ldots, u_t$ are free Haar unitaries, \pref{prop:word} implies that $\phi(u^{}_{\bi_1}u^*_{\bj_1} \cdots u^{}_{\bi_m}u^*_{\bj_m})$ is 1 iff the word $u^{}_{\bi_1}u^*_{\bj_1}\cdots u^{}_{\bi_m}u^*_{\bj_m}$ reduces to identity in the free group $F_t$, and is 0 otherwise. Moreover, if the word corresponding to the index $(\bi_1,\bj_1, \ldots, \bi_m,\bj_m)$ reduces to identity, then there exists a $*$-pairing $\pi \in \NC_2^*(d,m)$ which matches only variables with the same indices. We call any such $*$-pairing $\pi$ consistent with the $2dm$-tuple $(\bi_1,\bj_1, \ldots, \bi_m,\bj_m)$ and denote this by the indicator function $\ind[\pi, \bi_1,\bj_1, \ldots, \bi_m,\bj_m]$.
    
    The above implies that we may bound 
    \[ \phi(u^{}_{\bi_1}u^*_{\bj_1} \cdots u^{}_{\bi_m}u^*_{\bj_m})  \le \sum_{\pi \in \NC^*_2(d,m)} \ind[\pi, \bi_1,\bj_1, \ldots, \bi_m,\bj_m],\]
    where the inequality occurs because there could be multiple $*$-pairings consistent with a tuple. We thus have that 
    \begin{align*}
         \phi((pp^*)^m) &\le \sum_{\substack{|\bi_1|=\ldots=|\bi_m|=d\\|\bj_1|=\ldots=|\bj_m|=d}} c_{\bi_1}\cdots c_{\bi_m}c_{\bj_1}\cdots c_{\bj_m} \sum_{\pi \in \NC^*_2(d,m)} \ind[\pi, \bi_1,\bj_1, \cdots, \bi_m,\bj_m]\\
         &= \sum_{\pi \in \NC^*_2(d,m)}  \sum_{\substack{|\bi_1|=\ldots=|\bi_m|=d\\|\bj_1|=\ldots=|\bj_m|=d}} c_{\bi_1}\cdots c_{\bi_m}c_{\bj_1}\cdots c_{\bj_m} \ind[\pi, \bi_1,\bj_1, \cdots, \bi_m,\bj_m].
    \end{align*}
    
    If a term corresponding to a fixed $*$-pairing $\pi$ is non-zero, then the list of indices $(\bi_1, \ldots, \bi_m)$ is the same as $(\bj_1, \ldots, \bj_m)$ up to the exact ordering. Let us relabel $(\bi_1, \ldots, \bi_m) = (a_1, \ldots, a_{dm})$ and $(\bj_1, \ldots, \bj_m) = (b_1, \ldots, b_{dm})$ and let $c_{a_1, \ldots, a_{dm}} = c_{\bi_1}\ldots c_{\bi_m}$ and $c_{b_1, \ldots, b_{dm}} = c_{\bj_1}\cdots c_{\bj_m}$. Since $\pi$ gives a non-crossing bijection between the two lists $(a_1, \ldots, a_{dm})$ and $(b_1,\ldots, b_{dm})$, it holds that $c_{b_1, \ldots, b_{dm}} = c_{\pi(a_1),\ldots, \pi(a_{dm})}$. Thus, the above sum is
    \begin{align*}
        \ \phi((pp^*)^m) &\le \sum_{\pi \in \NC^*_2(d,m)}  \sum_{a_1, \ldots, a_{dm}} c_{a_1, \ldots, a_{dm}} c_{\pi(a_1),\ldots, \pi(a_{dm})}\\
        \ &\le \sum_{\pi \in \NC^*_2(d,m)} \left(\sum_{a_1, \ldots, a_{dm}} |c_{a_1, \ldots, a_{dm}}|^2\right)^{1/2}
        \left(\sum_{a_1,\ldots, a_{dm}} |c_{\pi(a_1),\ldots, \pi(a_{dm})}|^2\right)^{1/2},
    \end{align*}    
     where the inequality follows from Cauchy-Schwarz. The two internal summations are exactly the same since the summation is over all $dm$ tuples of indices and $\pi$ is a bijection. Switching back to the old indexing scheme, the internal summation then equals 
     \[  \sum_{a_1, \ldots, a_{dm}} |c_{a_1, \ldots, a_{dm}}|^2 = \sum_{\substack{|\bi_1|=\ldots=|\bi_m|=d}} |c_{\bi_1}\cdots c_{\bi_m}|^2 = \left(\sum_{|\bi|=d} |c_{\bi}|^2\right)^m.\] 
    Overall, we have
     \begin{align*}   
        \  \phi((pp^*)^m) &\le |\NC^*_2(d,m)|\left(\sum_{|\bi|=d} |c_{\bi}|^2\right)^m.
    \end{align*}
    Using \lref{lem:catalan-fuss} to bound the number of $*$-pairings,
    \[ |\NC^*_2(d,m)| = C_{d,m} = \frac{1}{m}\binom{m(d+1)}{m-1} = O\left(\frac{(d+1)^{m(d+1)}}{\left(d+\frac1m\right)^{md+1}}\right).\]
    Thus, taking the $m$-th root in the limit $m \to \infty$ yields
    \[ \|p\|^2 = \lim_{m \to \infty}  \phi((pp^*)^m)^{1/m} = \frac{(d+1)^{d+1}}{d^d}\left(\sum_{|\bi|=d} |c_{\bi}|^2\right) \le e(d+1) \left(\sum_{|\bi|=d} |c_{\bi}|^2\right).\]
    This completes the proof of the theorem.
\end{proof}

