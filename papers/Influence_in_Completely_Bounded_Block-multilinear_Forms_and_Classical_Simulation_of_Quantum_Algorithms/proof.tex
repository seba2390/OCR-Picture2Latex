\section{Influence in Completely Bounded Block-multilinear Forms}
\label{sec:proof}
\newcommand{\blocks}{\mathrm{blocks}}
\newcommand{\lt}{\mathrm{left}}
\newcommand{\rt}{\mathrm{right}}



In this section we prove the non-commutative root-influence inequality (\thmref{thm:bh-intro}),  the special case of the Aaronson-Ambainis conjecture given in \thmref{thm:aa}, and also briefly mention how the simulation result in \corref{cor:sim} follows from \thmref{thm:aa} and the results in \cite{AA14}. We first need some preliminaries from free probability theory. 



\subsection{Low-degree Polynomials of Haar Random Unitaries}

As discussed in the proof overview, we require bounds on the operator norm (as well as normalized trace) of low-degree polynomials of random unitaries and these follow from known results in free probability theory. Here we explain these connections and also prove some auxillary lemmas needed for the proof of \thmref{thm:bh-intro} and \thmref{thm:aa}. 



Let $z_{\ui}$ denote the non-commutative monomial $z_{i_1} z_{i_2} \cdots z_{i_d}$ for a $d$-tuple $\ui  = (i_1, \ldots, i_d) \in [t]^d$ and let $p(z_1, \ldots, z_t)$ be a non-commutative polynomial in the variables $z_1, \ldots, z_t$. We are interested in computing the operator norm $\|\cdot\|_{\op}$ and the normalized trace  $\tr_N$ of the polynomial $p(z_1, \ldots, z_t)$ (or its higher moments) when substituting $N \times N$ Haar random unitaries for the variables $z_i$.

As explained previously, the theory of free probability gives us tools that allow us to compute  the above in the limit $N \to \infty$. In particular, Voiculescu \cite{V98} showed that the  (normalized) trace of polynomials in Haar random unitaries and their conjugates converges to the trace of the same polynomial evaluated on certain infinite-dimensional operators called \emph{Haar unitaries} that satisfy a non-commutative notion of independence called \emph{free independence}. This was strengthened by Collins and Male \cite{CM11} who showed that such convergence also holds for the operator norm. A short primer on free probability is given in \appref{sec:free}, but for now one can think of $\CA$ as a self-adjoint algebra of bounded linear operators on a Hilbert space and $\phi$ as a trace functional for such operators in the statement given below.


\begin{theorem}[\cite{V98, CM11}] \label{thm:voiculescu}
    Let $p(z_1, \ldots, z_{2t})$ be a non-commutative polynomial in $\BR\langle z_1, \ldots, z_{2t}\rangle$. If $U_1, \ldots, U_t$ are $N \times N$ Haar random unitaries, then almost surely,
    \begin{align*}
     \ \tr_N[p(U_1, \ldots, U_t, U^*_1, \ldots, U_t^*)] &~\xrightarrow[N \to \infty]{}~ \phi[p(u_1, \ldots, u_t, u^*_1, \ldots, u^*_t)],\\
    \  \|p(U_1, \ldots, U_t, U^*_1, \ldots, U_t^*)\|_{\op} &~\xrightarrow[N \to \infty]{}~ \| p(u_1, \ldots, u_t, u^*_1, \ldots, u^*_t)\|,
    \end{align*}
    where $u_1, \ldots, u_t$ are free Haar unitaries in a $C^*$-probability space $(\CA, \phi)$ and $\|\cdot\|$ is the norm for the underlying $C^*$-algebra.
\end{theorem}




Using the above result it suffices to consider free Haar unitaries in a $C^*$-probability space to compute the operator norm and trace of polynomials of random unitaries. For a non-commutative polynomial $p(z_1, \ldots, z_t) = \sum_{|\ui|\le d} c_{\ui}z_{\ui}$, denoting by $\|p\|_2 =  \left(\sum_{|\ui| \le d} |c_{\ui}|^2\right)^{1/2}$, one can show the following easily using techniques from free probability. 

\begin{lemma} \label{thm:trace}
    Let $p(z_1, \ldots, z_t) = \sum_{|\ui|\le d} c_{\ui}z_{\ui} $ be a non-commutative degree-$d$ polynomial in $\R\langle z_1, \ldots, z_t\rangle$ and $u_1, \ldots, u_t$ be free Haar unitaries in a $C^*$-probability space $(\CA, \phi)$. Then, 
     \[ \phi[p(u_1, \ldots, u_t) (p(u_1, \ldots, u_t))^*] =  \|p\|_2^2.\]
\end{lemma}

The above implies that $\tr_N[p(U_1, \ldots, U_t) (p(U_1, \ldots, U_t))^*]$ converges to $\|p\|_2^2$ almost surely as $N \to \infty$. We shall defer the proof of \lref{thm:trace} to \appref{sec:app}, but to aid our intuition we note here that since the  $U_i$'s are independent $N \times N$ Haar random unitaries, the expected value

\[ \BE\left[\tr_N[p(U_1, \ldots, U_t) (p(U_1, \ldots, U_t))^*\right] = \|p\|_2^2,\] 
{and from concentration of measure, it is natural to expect that it converges to the above value}. 


Similarly, to compute the operator norm of $p(U_1, \ldots, U_t)$ for Haar random unitaries one can instead study the norm of the polynomial evaluated on free Haar unitaries. Such bounds are easier to prove using the trace method since free independence imposes strong restrictions on the non-commutative moments. For instance, if $U_1$ and $U_2$ are independent $N \times N$ Haar random matrices, then $\BE[\tr_N(U_1U_2U^*_1U_2^*)]$ is non-zero (albeit quite small), while the corresponding trace evaluated on free Haar unitaries $u_1$ and $u_2$ is zero, that is $\phi(u_1u_2u^*_1u_2^*) = 0$. Thus, computing the trace $\phi[p(u_1,\ldots, u_t, u^*_1, \ldots, u_t^*)]$ reduces to handling the combinatorics of the patterns of $u_i$'s and $u_i^*$'s. 

In particular, we will rely on the following result that follows from the work of Kemp and Speicher \cite{KS05}  who consider the operator norm of homogeneous polynomials evaluated on free $R$-diagonal operators, a class that includes free Haar unitaries as well. We also remark that a bound where the right-hand side below is worse by a multiplicative $O(d^{1/2})$ factor also follows from the work of Haagerup\footnote{We note that Haagerup considered the more general case of polynomials in both $u_i$'s and $u^*_i$'s.}\cite{H78} who proved it in another context, predating even the introduction of free probability theory. 


\begin{theorem}[\cite{KS05}]
\label{thm:kemp-speicher}
    Let $p(z_1, \ldots, z_t) = \sum_{|\ui| = d} c_{\ui}z_{\ui} $ be a homogeneous non-commutative degree-$d$ polynomial in $\R\langle z_1, \ldots, z_t\rangle$ and $u_1, \ldots, u_t$ be free Haar unitaries in a $C^*$-probability space. Then, 
    \[ 
    \|p(u_1, \ldots, u_t)\| \le \sqrt{e(d+1)} \cdot \|p\|_2,
    \]
    where the left-hand side denotes the norm in the underlying $C^*$-algebra. 
\end{theorem}

For completeness, we  introduce the necessary free probability background and some combinatorial details in \appref{sec:app}, and we present the fairly short proof of \thmref{thm:kemp-speicher} (from \cite{KS05}) there in a self-contained way. We shall need to extend the above bound to non-homogeneous polynomials. Let $p(z_1, \ldots, z_t) = \sum_{|\ui| \le d} c_{\ui}z_{\ui}$ and  let $p_k(z_1, \ldots, z_t) = \sum_{|\ui| = k} c_{\ui}z_{\ui}$ denote the degree-$k$ homogeneous part of $p$. Writing $p_k = p_k(u_1, \ldots, u_t)$ for $0 \le k  \le d$ and $p = p(u_1, \ldots, u_t)$, it follows from the triangle inequality,  \thmref{thm:kemp-speicher}, and Cauchy-Schwarz, that
    \begin{align*}
        \ \|p\| &\le \sum_{k=0}^d \|p_k\| 
        \le 
        \sum_{k=0}^d\sqrt{e(k+1)}\|p_k\|_2
        \le
       \sqrt{e}\left(\sum_{k=0}^d (k+1)\right)^{1/2} \left(\sum_{k=0}^d  \|p_k\|^2_2\right)^{1/2} \leq \sqrt{e}(d+1)  \cdot\|p\|_2.
    \end{align*}
Thus, we essentially get the same bound as in the homogeneous case, at the expense of an additional $O(d^{1/2})$ factor.



Collecting all the above we have the following as a direct consequence:

\begin{theorem} \label{thm:op-norm}
    Let $p(z_1, \ldots, z_t) = \sum_{|\ui|\le d} c_{\ui}z_{\ui} $ be a non-commutative degree-$d$ polynomial in $\R\langle z_1, \ldots, z_t\rangle$ and $U_1, \ldots, U_t$ be independent $N \times N$ Haar random unitaries. Then, as $N \to \infty$, the following holds almost surely, 
    \[ \tr_N[p(U_1, \ldots, U_t) (p(U_1, \ldots, U_t))^*] =  \|p\|_2^2,\]
    and
    \[ \|p(U_1, \ldots, U_t)\|_{\op} \le \sqrt{e}(d+1)  \cdot \|p\|_2,\]
    Moreover, the factor $(d+1)$ in the operator norm bound can be improved to $\sqrt{d+1}$ if the polynomial is homogeneous.
\end{theorem}

Based on the above theorem, we prove the following key lemma which captures the polar decomposition strategy mentioned in the earlier proof overview (\secref{sec:bh}). This will serve as the key ingredient in the proof of \thmref{thm:aa} and \thmref{thm:bh-intro}. 

\begin{lemma}\label{lem:polar}
    Let $p$ be a non-commutative degree-$d$ polynomial in $\R\langle y_1, \ldots, y_m, z_1, \ldots, z_t\rangle$ given by
    \[ p(y_1, \ldots, y_m, z_1, \ldots, z_t) = \sum_{i=1}^m y_i q_i(z_1, \ldots, z_t) + q_0(z_1, \ldots, z_t).\]
    Then, for every $\delta > 0$, there exist an integer $N$ and $N \times N$ unitaries $V_1,\ldots, V_m, W_1, \ldots, W_t$ such that 
    \[ \|p(V_1, \ldots, V_m, W_1, \ldots, W_t)\|_{\op} \ge \frac{1}{\sqrt{e}(d+1)} \sum_{i=1}^m \|q_i\|_2 - \delta.\]
    Moreover, the factor in front can be improved to $(e(d+1))^{-1/2}$ if $p$ is homogeneous. 
\end{lemma}

\begin{proof}[Proof of \lref{lem:polar}]
     For an arbitrary integer $N$, let us pick independent $N \times N$ Haar random unitaries $W_1, \ldots, W_t$ which we substitute for the variables $z_1,\ldots,z_t$, respectively, and let $M_i = q_i(W_1, \ldots, W_t)$ be the corresponding random matrices. Then, for any tuple of matrices $V_1, \ldots, V_m$ that we could substitute for the variables $y_1, \ldots, y_m$, we have that 
    \[ 
    p(V_1, \ldots, V_m, W_1, \ldots, W_t) = \sum_{i=1}^m V_i M_i + M_0.
    \] 
     \thmref{thm:op-norm} and union bound imply that as $N \to \infty$, with probability $1$ all the following events simultaneously hold: 
    \begin{itemize}
        \item $\|M_i\|_{\op} \le \sqrt{e}(d+1) \cdot \|q_i\|_2$ for each $i$,
        \item $\tr_N(M^*_iM_i) = \|q_i\|_2^2$ for each $i$, where $\tr_N(M)$ is the normalized trace.
    \end{itemize}
   To show that the operator norm must be large, let us fix a sufficiently large $N$ and a choice of $N\times N$ unitaries $W_1, \ldots, W_t$ such that $M_i$ satisfies $\|M_i\|_{\op} \le \sqrt{e}(d+1) \cdot \|q_i\|_2 + \epsilon$ and $\tr_N(M^*_iM_i) \ge \|q_i\|_2^2 - \epsilon$ for each $0\le i\le m$, where $\epsilon$ can be made arbitrarily small by increasing $N$. For $0 \leq i \leq m$, let $M_i = U_i P_i$ be the left polar decomposition of $M_i$, where $U_i$ is a unitary matrix and $P_i$ is a positive semidefinite matrix.
   
   We select the tuple of unitary matrices $V_1, \ldots, V_m$ that we substitute for the variables $y_1, \ldots, y_m$ to be $V_i = U_0U^*_i$ for $i \in [m]$. With this we have that $\|p(V_1, \ldots, V_m, W_1, \ldots, W_t)\|_{\op}$ is at least
    \begin{align*}
         \Big\|M_0 + \sum_{i=1}^m V_iM_i\Big\|_{\op} & = \Big\|U_0 P_0 + \sum_{i=1}^m U_0 U_i^* U_iP_i \Big\|_{\op} \\
        \ & =  \Big\|U_0 P_0 + \sum_{i=1}^m U_0 P_i\Big\|_{\op}  = \Big\| P_0 + \sum_{i=1}^m  P_i\Big\|_{\op}\ge \tr_N\Big(P_0 + \sum_{i=1}^m P_i\Big) \ge \tr_N\Big(\sum_{i=1}^m P_i\Big),
    \end{align*}
    where the last equality follows since the operator norm is unitarily invariant and the last two inequalities follow from the positive semidefiniteness of the $P_i$'s.

    For every positive semidefinite matrix $P$, we have that $\tr_N(P) \ge {\tr_N(P^2)}/{\|P\|_{\op}}$. 
  
    Hence,
     \[ \|p(V_1, \ldots, V_m, W_1, \ldots, W_t)\|_{\op} \ge \sum_{i=1}^m \frac{\tr_N(P_i^2)}{\|P_i\|_{\op}}.\]
     By our choice of $M_i$, we have that $\tr_N(P_i^2) = \tr_N(M_i^* M_i) \ge \|q_i\|_2^2 - \eps$ and $\|P_i\|_{\op} = \|M_i\|_{\op} \le \sqrt{e}(d+1)\|q_i\|_2 + \eps$. Since $\eps$ can be made arbitrarily small by increasing $N$, it follows that 
      \[ \|p(V_1, \ldots, V_m, W_1, \ldots, W_t)\|_{\op} \ge \frac1{\sqrt{e}(d+1)} \sum_{i=1}^m \|q_i\|_2 - \delta ,\]
     for large enough $N$. The improved bound for the homogeneous case follows directly by plugging the bound of \thmref{thm:op-norm} into the above proof.
\end{proof}





\subsection{Non-commutative root-influence inequality}
\label{sec:bh-proof}


For clarity in the proofs below, we remind our  convention that all tuples or blocks are denoted with boldface fonts (e.g. $\BU_1$ or $\BA$), while a single element is denoted without boldface (e.g. $U_1(i)$ or $A_i$ or $A$). Before proceeding with the proof, we restate the statement for convenience.

\bh*





\begin{proof}[Proof of \thmref{thm:bh-intro}] 
Since $f$ is homogeneous, we can write
   \begin{align*}
    f(\x_1,\ldots, \x_d) &= \sum_{i_1, \ldots, i_d \in [n]} \hf_{i_1, \ldots, i_d} ~x_1(i_1)x_2(i_2)\cdots x_d({i_d}) \\
    \ & = \sum_{i=1}^n  x_1(i) \underbrace{\left(\sum_{i_2,\ldots, i_d \in [n]} \hf_{i_1, \ldots, i_d} ~x_2(i_2)\cdots x_d({i_d})\right)}_{\textstyle := f_i(\x_2,\ldots, \x_d)}.
\end{align*}
 In this case, it follows from \eqref{eqn:inf-tensor} that for each $i \in [n]$, we have 
 \begin{equation}\label{eqn:var}
     \ \Var[f_i] = \|f_i\|^2_2 = \inf_{1,i}(f) \text{ and }  \Var[f] = \sum_{i=1}^n \inf_{1,i}(f).
 \end{equation}

  Let us denote the corresponding non-commutative block-multilinear polynomials by $f(\BU_1, \ldots, \BU_d)$ and $f_i(\BU_2, \ldots,\BU_d)$ where $\BU_b = (U_b(1), \ldots, U_b(n))$ denotes the $b^\text{th}$ block of non-commutative variables. To show a lower bound on $\cbnorm{f}$ it suffices to exhibit a collection of square matrices $\{U_b(i)\}_{b\in [d], i \in [n]}$ with operator norm at most~1, such that $\|f(\BU_1, \ldots, \BU_d)\|_{\op}$ is large. 
  
%  

Applying \lref{lem:polar} for the homogeneous case (with $p = f$, $q_i=f_i$ for $i \in [n]$, and $q_0=0)$, it follows that for every $\delta > 0$ there exists an integer $N$ and a choice of tuples of $N \times N$ unitaries $\BU_1, \ldots, \BU_d$ such that  
      \[ \cbnorm{f} \ge \|f(\BU_1, \ldots, \BU_d)\|_{\op} \ge \frac1{\sqrt{e(d+1)}} \sum_{i\in [n]} \|f_i\|_2  -\delta \stackrel{\eqref{eqn:var}}{\ge}  \frac{1}{\sqrt{e(d+1)}} \left(\sum_{i=1}^n \sqrt{\Inf_{1,i}(f)} \right) -\delta.\]
Taking $\delta \to 0$, we get the statement of the lemma. The proof for the inequality when $b=d$ is the last block follows similarly by using the right polar decomposition.
\end{proof}

\subsection{Aaronson-Ambainis Conjecture for non-homogeneous forms}

In this section, we prove \thmref{thm:aa}, which requires handling non-homogeneous forms. The proof will be similar to the proof of \thmref{thm:bh-intro} but we will need to be careful about certain details. 

\begin{proof}[Proof of \thmref{thm:aa}]
Any block-multilinear polynomial $f(x_1, \ldots, x_d)$ can be written as 
\begin{align*}
    f(\x_1,\ldots, \x_d) &= \BE f + \sum_{b\in [d]} f_b(\x_b, \x_{b+1}, \ldots, \x_d),
\end{align*}
where $f_b$ consists of all monomials of $f$ that start with a variable in the $b^\text{th}$ block $\x_b$. Note that $f_b$ depends only on the variables in blocks $\x_b, \x_{b+1},\ldots, \x_d$. Moreover, it follows from \eqref{eqn:inf-tensor} that 
 \begin{equation}\label{eqn:var-general}
     \ \Var[f] = \sum_{b \in [d]} \|f_b\|_2^2 = \sum_{b \in [d]} \Var[f_b],
 \end{equation}
so there exists a block $\beta \in [d]$ such that $\Var[f_{\beta}] \ge \frac{1}{d}\Var[f]$. 

Since $f_{\beta}$ contributes a lot to the variance, it is natural to try to find an influential variable in the block $\x_{\beta}$. Towards this end,  we pull out the variables $x_{\beta}(i)$ and write
\begin{align*}
    f_{\beta}(\x_{\beta},\ldots, \x_d) &= \sum_{i\in [n]} x_{\beta}(i) f_{\beta,i}(\x_{\beta+1}, \ldots, \x_d),
\end{align*}
for block-multilinear polynomials $f_{\beta,i}(\x_{\beta+1}, \ldots, \x_d)$. Note that some of the $f_{\beta,i}$'s could be identically zero, so let us define $S$ to be the set of those $i$ such that $f_{\beta,i}$ is non-zero. We note that
\begin{align} \label{eqn:part-inf}
  \|f_{\beta,i}\|_2^2  =  \Inf_{\beta,i}(f_{\beta}) \le \Inf_{\beta,i}(f)  
\end{align}
which implies that
\begin{align}\label{eqn:var-main}
    \frac{1}{d} \Var[f] \le \Var[f_{\beta}] = \sum_{i \in S}\|f_{\beta,i}\|_2^2 = \sum_{i \in S} \Inf_{\beta,i}(f_{\beta}).
\end{align}
\begin{sloppypar}
Denote the corresponding non-commutative block-multilinear polynomials by $f(\BU_1, \ldots, \BU_d)$,  $f_b(\BU_{b}, \ldots,\BU_d)$, and $f_{\beta}(\BU_{\beta+1}, \ldots,\BU_d)$ where $\BU_b = (U_b(1), \ldots, U_b(n))$ denotes the $b^\text{th}$ block of non-commutative variables. To show a lower bound on $\cbnorm{f}$ it suffices to exhibit a collection of square matrices $\{U_b(i)\}_{b\in [d], i \in [n]}$ with operator norm at most~1 such that $\|f(\BU_1, \ldots, \BU_d)\|_{\op}$ is large.
\end{sloppypar}
  
 We set the matrices in blocks $\BU_1, \ldots, \BU_{\beta-1}$ to be zero (that is, the all-zero matrix $\BZ$). Note that with this choice all polynomials $f_b(\U_b, \ldots, \U_d)$ where $b < \beta$ vanish and the non-commutative polynomial becomes 
 \[ f(\BZ, \ldots, \BZ, \BU_{\beta}, \BU_{\beta+1}, \ldots, \BU_d) = \sum_{i\in S} U_{\beta}(i) f_{\beta,i}(\BU_{\beta+1}, \ldots, \BU_d) + \sum_{b=\beta+1}^d f_b(\BU_b, \BU_{b+1}, \ldots, \BU_d) + \Ef,\]
  which is a non-commutative polynomial of the form considered in \lref{lem:polar} (with $m = |S|$, $q_i = f_{\beta,i}$ and $q_0 = \sum_{b=\beta+1}^d f_b + \Ef$). Thus, by \lref{lem:polar} for every small $\delta>0$ there exists an integer $N$ and a choice of $N \times N$ matrices for the blocks $\BU_{\beta},\ldots, \BU_d$ such that 
        \begin{align*}
             \ \cbnorm{f} & \ge \|f(\BZ, \ldots, \BZ, \BU_{\beta}, \BU_{\beta+1}, \ldots, \BU_d)\|_{\op} & \\
             \  & \ge \frac1{\sqrt{e}(d+1)} \sum_{i\in S} \|f_{\beta,i}\|_2 -\delta  \stackrel{\eqref{eqn:part-inf}}{=}  \frac{1}{\sqrt{e}(d+1)} \left(\sum_{i \in S} \sqrt{\Inf_{\beta,i}(f_{\beta})} \right) -\delta & \\
             \ &\stackrel{\eqref{eqn:var-main}}{\ge}  \frac{1}{\sqrt{e}(d+1)} \left( \frac{\sum_{i \in S} \Inf_{\beta,i}(f_{\beta})}{\sqrt{\maxinf(f)}} \right) -\delta  \stackrel{\eqref{eqn:part-inf}}{\ge}  \frac{1}{\sqrt{e}(d+1)^{2}} \left( \frac{\Var[f]}{ \sqrt{\maxinf(f)}} \right) -\delta
        \end{align*}
        Taking $\delta \to 0$ and using the assumption that $\|f\|_{\cb} \le 1$, we obtain the statement of the theorem:
     \[
     1\geq \cbnorm{f} \ge \frac{1}{\sqrt{e}(d+1)^{2}} \cdot \frac{\Var[f]}{\sqrt{\maxinf(f)}} \implies \maxinf(f) \ge  \frac{(\Var[f])^2}{e(d+1)^4}. \qedhere
     \]
\end{proof}
 

     
     

\subsection{Approximating completely bounded forms with decision trees}



In this section, we briefly mention how to obtain \corref{cor:sim}.
Aaronson and Ambainis \cite[Theorem 3.3]{AA14} showed that querying the most influential variable reduces the variance of the function~$f$, and if that influence is lower bounded by a polynomial in $\Var[f]/d$, then after $\poly(d)$ queries (the exact quantitative dependence can be read off from their proof), the variance of the function becomes small enough so that it can be approximated almost-everywhere by its expectation.  Since the family of degree-$d$ block-multilinear forms with completely bounded norm at most one is closed under restrictions, one can apply \thmref{thm:aa} repeatedly. This gives us \corref{cor:sim}.