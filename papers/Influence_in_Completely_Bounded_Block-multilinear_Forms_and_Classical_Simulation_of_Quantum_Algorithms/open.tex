\section{Discussion and Open Problems} 
\label{sec:open}

To prove \conjref{conj:folklore} in full generality, one would need to consider arbitrary quantum query algorithms: such an algorithm operating on an input $z \in \pmone^m$ always makes queries to the same oracle $O_z$ (with a control qubit possibly). One can always convert any such algorithm to the type given in \figref{fig:quantum} by replacing the oracle $O_z$ used at each step $b$ with a new oracle $O_{\x_b}$ where $\x_b \in \pmone^{m+1}$. The execution of the original algorithm can then be recovered by substituting $\x_b = (z,1)$ for every $b \in [d]$.
As such one can always obtain a completely bounded block-multilinear form associated with any quantum query algorithm. Conversely, the work \cite{ABP18} shows that the existence of a degree-$2d$ homogeneous block-multilinear form $F : \pmone^{(m+1) \times 2d}$   with completely bounded norm at most one also implies the existence of a $d$-query quantum algorithm whose bias is given by $F((z,1), \ldots, (z,1))$ on every input $z \in \pmone^m$. Thus, completely bounded homogenous block-multilinear forms fully characterize quantum query algorithms in this sense. 



In many works in quantum query complexity that concern worst-case complexity, understanding completely bounded or bounded block-multilinear polynomials is sufficient to prove lower bounds as well as give worst-case classical simulation results (i.e.\ for all inputs), see for instance \cite{AA:forrelation, BGGS21}. However, a transformation that converts a general quantum query algorithm to the type shown in \figref{fig:quantum} is not conducive to the almost-everywhere results considered in this paper, as the size of the input domain increases exponentially and the number of relevant inputs (i.e.\ where each $\x_b$ is set to the same $(z,1)$) becomes an exponentially small fraction of the new domain. 

It thus remains an intriguing open problem to see if the characterization of \cite{ABP18} can be used to make further progress on \conjref{conj:folklore}. One can also hope to make progress on \conjref{conj:folklore} without relying on the connection via influences ---  recently, Aaronson, Ingram and Kretschmer \cite{AIK21} managed to directly prove \conjref{conj:folklore} for the special case where the quantum algorithm queries a sparse oracle, without first proving a special case of \conjref{conj:aa-inf}.



Another interesting direction is to show that the Aaronson-Ambainis conjecture holds for bounded block-multilinear polynomials, that is, polynomials whose sup-norm on the Boolean hypercube is at most one. While this by itself does not suffice for the application to quantum algorithms as explained above, it might pave the way towards \conjref{conj:aa-inf} in full generality. Lastly, the free-probability toolbox has already found several applications in quantum information theory (see e.g.~\cite{Yin:freeprob,CollinsNechita}), and we hope this work will stimulate more applications elsewhere as well.


\paragraph{Acknowledgments.} We thank Scott Aaronson, Srinivasan Arunachalam, Jop Bri\"et and Ryan O'Donnell for helpful comments.