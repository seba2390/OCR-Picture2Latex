%\documentclass[aps,
%prl,
%floats,
%balancelastpage,preprintnumbers,floatfix,nofootinbib,superscriptaddress]{revtex4-2}

%\usepackage{graphicx,amsmath,amssymb,amsfonts, soul}
%\usepackage{hyperref}
%\documentclass[%
%reprint,
%amsmath,amssymb,
%aps,
%superscriptaddress,
%preprintnumbers,
%onecolumn
%]{revtex4-2}
%
%\usepackage{graphicx}% Include figure files
%\usepackage{dcolumn}% Align table columns on decimal point
%\usepackage{bm}% bold math
%\usepackage{hyperref}% add hypertext capabilities
%
%\usepackage{color}
%\usepackage{cancel}
%\usepackage[normalem]{ulem}
%\usepackage{AAS_macro}
%
%\newcommand*\df{\mathop{}\!\mathrm{d}}
%\newcommand{\n}{\nonumber}
%\newcommand*\vx{\mathbf{x}}
%\DeclareMathOperator{\sgn}{sgn}
%
%\makeatletter
%\newcommand*{\rom}[1]{\expandafter\@slowromancap\romannumeral #1@}
%\makeatother
%
%\usepackage{multirow}
%
%\setlength{\tabcolsep}{16pt}
%\renewcommand{\arraystretch}{1.2}
%
%\begin{document}
%
%\title{Supplemental Material: Stochastic Fluctuations of Low-Energy Cosmic Rays\\ and the Interpretation of Voyager Data}
%\maketitle

\section*{The cosmic-ray transport equation}
We have adopted the cosmic-ray (CR) transport equation in terms of kinetic energy $E$ for the study of stochasticity. However, it might be more customary to formulate the CR transport equation in terms of momentum $p$. For definiteness, we now lay out the procedure for the transformation. The equation in terms of momentum is \cite{schlickeiser2002}:  
\begin{equation}
\frac{\partial f}{\partial t}+\frac{\partial}{\partial z}\left(u f\right) -D\nabla^2 f + \frac{1}{p^2} \frac{\partial }{\partial p}\left(\dot{p}p^2f\right)=\Tilde{q}(r, z, p, t) \, . \label{eq:transport-p}
\end{equation}
where $f(r,z,p,t)$ is the phase space density of CRs, that is the number of particles per unit volume in configuration and momentum space, $u=u(z)$ is the advection velocity with only component perpendicular to the Galactic disk, $D=D(p)$ is the isotropic and homogeneous diffusion coefficient, and $\dot{p}$ is the momentum loss rate. The phase space density $f(r,z,p,t)$ is related to the cosmic-ray density $\psi(r,z,E,t)$, the number of particles per unit volume and energy, as $\psi(r,z,E,t)=4\pi p^2 f(r,z,p,t)/v$ and, similarly, we have $q(r,z,E,t)=4\pi p^2 \tilde{q}(r,z,p,t)/v$ where $v$ is the particle's speed. It is then clear that we could now transform Eq. \ref{eq:transport-p} into an equation for $\psi(r,z,E,t)$ by performing the change of variable from $p$ to $E$. We note also that $\dot{E}=\dot{p}v$ and, in fact, the standard literature mostly quote the formulae for the energy loss rate (even when Eq. \ref{eq:transport-p} is adopted for the study of CRs \cite{strong1998,schlickeiser2002}).

\section*{Energy loss rate}

Cosmic-ray protons lose energy mostly due to ionization and proton-proton interaction with the gas in the Galactic disk. The combined energy loss rate for these two processes could be written as \cite{schlickeiser2002,krakau2015}: 
\begin{eqnarray}
&&\dot{E}\simeq \textrm{H}(|z|-h) 1.82\times 10^{-7}\left(\frac{n_{\textrm{H}}}{1\textrm{ cm}^{-3}}\right)\n\\
&&\qquad\times\left[(1+0.185\ln\beta)\frac{2\beta^2}{10^{-6}+2\beta^3}+2.115\left(\frac{E}{1\textrm{ GeV}}\right)^{1.28}\left(\frac{E}{1\textrm{ GeV}}+200\right)^{-0.2}\right]\textrm{ eV/s},
\end{eqnarray}
where $H(|z|-h)$ is the Heaviside function which indicates that these energy loss mechanisms are only effective within the height $h$ of the disk, $n_{\mathrm{H}}$ is the density of hydrogen atoms in the disk, $\beta$ is the ratio between the particle's speed and the speed of light, and $E$ is the kinetic energy of the particle.  

Below a few GeV, the main mechanisms for energy loss of CR electrons are ionization interaction and bremsstrahlung radiation in the Galactic disk. At higher energy, these particles lose energy more effectively not only in the disk but also in the CR halo due to synchrotron radiation and inverse Compton scattering. The energy loss rate could then be parametrized as \cite{schlickeiser2002,mertsch2011,evoli2017}:  
\begin{eqnarray}
&&\dot{E}\simeq 10^{-7}\left(\frac{E}{1\textrm{ GeV}}\right)^2+\textrm{H}(|z|-h) 1.02\times 10^{-8}\left(\frac{n_\mathrm{H}}{1\textrm{ cm}^{-3}}\right)\n\\
&&\qquad\qquad\qquad\qquad\qquad\qquad\times\left\{18.495+2.7\ln\gamma+0.051\gamma\left[1+0.137\left(\ln\gamma+0.36\right)\vphantom{^{\frac{^\frac{}{}}{}}}\right]\vphantom{^{\dfrac{}{}}}\right\}\textrm{ eV/s},
\end{eqnarray}
where $\gamma$ is the Lorentz factor.

\section*{Parameters for the stochastic model}
In Tab. \ref{tab:parameters}, we briefly summarise all the parameters adopted in order for the stochastic uncertainty bands to encompass the data from both Voyager 1 and AMS-02. Most of the parameters including the diffusion coefficient and the injection spectra are constrained from the fits at high energies \cite{evoli2019}. 

In fact, the two parameters that only the low-energy spectra are sensitive to are the number density of hydrogen atoms and the advection speed perpendicular to the disk. In fact, the value of the advection speed has also been given in the fits at high energy but it might vary slightly around 10 km/s depending on the model and the species of CRs considered \cite{evoli2019,mertsch2020}. We note also that $n_{\mathrm{H}}$ is not completely free as the surface density of the disk for our Galactic neighborhood is externally constrained to be around 2 g/cm$^{2}$ which is quite consistent with the value adopted for our fits \cite{ferriere2001}. 

\begin{table}[h!]
\centering
% \captionsetup{justification=centering}
\caption{Externally constrained and fitted parameters for the stochastic model for both CR protons and electrons in the case for the diffusion coefficient scaling with Lorentz factor as presented in the main text.}


	\label{tab:parameters}
	\begin{tabular}{|c|c|c|r|} % four columns, alignment for each
		\hline
		\hline
		\multirow{2}{*}{\shortstack{Fitted parameters\\ for low-energy CRs}} & $n_{\mathrm{H}}$ & Gas density in the disk & 0.9 cm$^{-3}$\\
		\cline{2-4}
		& $u_0$ & Advection speed & 16 km/s\\
		\hline
		\multirow{10}{*}{\shortstack{Constrained parameters\\ from high-energy CRs}} & $R_d$ & $\qquad$ Radius of the Galactic disk $\qquad$ & 15 kpc \\
		\cline{2-4}
		& $H$ & Height of the CR halo & 4 kpc\\
		\cline{2-4}
		& $2h$ & Height of the gas disk for energy loss & 300 pc\\
		\cline{2-4}
		& $2h_s$ & Height of the disk of sources & 80 pc\\
		\cline{2-4}
		& $D(E=10\textrm{ GeV})$ & Diffusion coefficient at 10 GeV & $5\times 10^{28}$ cm$^2$/s\\
		\cline{2-4}
		& $\delta$ & Index of the diffusion coefficient & 0.63\\
		\cline{2-4}
		& $\mathcal{R}_s$ & Source rate & 0.03 yr$^{-1}$\\
		\cline{2-4}
		& $\xi_{CR}^{(p)}$ & Proton acceleration efficiency & 8.7\%\\
		\cline{2-4}
		& $\xi_{CR}^{(e)}$ & Electron acceleration efficiency & 0.55\%\\
		\cline{2-4}
		& $\alpha$ & Index of the injection spectra & 4.23\\   
		\hline
		\hline
	\end{tabular}
\end{table}

We note that the parameters in Tab. \ref{tab:parameters} have been obtained for the diffusion coefficient of the form as presented in the main text $D(E)\sim\beta \gamma^{\delta}$ which is expected when the magneto-static approximation is relaxed meaning the Alfv\'en speed is no longer negligible in comparison to the particle's speed in the resonance condition of wave-particle interaction (see e.g. \cite{schlickeiser1999,schlickeiser2010} for more technical details). In a broader sense, it is probably fair to admit that there remain significant uncertainties since there is currently no direct observations of the mean-free path in the interstellar medium. In order to bracket this uncertainty, we have also repeated our computation with a diffusion coefficient that has a power law dependence on rigidity, $D(E)\sim \beta R^\delta$ where $R$ is the particle's rigidity. We have found that this would not qualitatively change our results since the break in $D(E)$ below roughly 1 GeV does not significantly affect the spectrum of CR protons at low energies as the transport in this regime is dominated by energy loss. For CR electrons, the rigidity or Lorentz factor dependent diffusion coefficients are roughly the same down to 1 MeV. We present also in Fig.~\ref{fg:Drig} the fits for the case of a rigidity-dependent diffusion coefficient with slightly different values for the advection speed $u_0$ and the number density of hydrogen atoms $n_\mathrm{H}$ (see Tab.~\ref{tab:parameters2} for the complete list of parameter values in this case).% = 18 km/s and nH = 0.7 cm−3 in Fig. 1 of this document

\begin{figure}[h]
\centerline{
\includegraphics[width=3.2in, height=2.7in]{fg_jE_p_show_SNR_rh=1000_vA=18_h=151_nH=70_zad=10_Drig_alpha=423_ep=87_zu=39_in.png}
\includegraphics[width=3.2in, height=2.7in]{fg_jE_e_show_SNR_rh=1000_vA=18_h=151_nH=70_zad=10_Drig_alpha=423_ep=5_zu=39_in.png}}
\caption{Stochastic fluctuations of GCR protons (left panel) and electrons (right panel) in comparison with data from Voyager~1~\cite{cummings2016} (blue) and AMS-02~\cite{AMS2014,AMS2015} (green) for the case of rigidity dependent diffusion coefficient. The dotted and solid black curves are respectively the expectation values and the median of the intensities. The shaded regions are the 95\% and 68\% uncertainty ranges.}
\label{fg:Drig}
\end{figure}

\begin{table}[h!]
\centering
% \captionsetup{justification=centering}
\caption{Externally constrained and fitted parameters for the stochastic model for both CR protons and electrons  in the case for the diffusion coefficient scaling with rigidity.}


	\label{tab:parameters2}
	\begin{tabular}{|c|c|r|} % four columns, alignment for each
		\hline
		\hline
		\multirow{2}{*}{\shortstack{Fitted parameters\\ for low-energy CRs}} & $n_{\mathrm{H}}$ & 0.7 cm$^{-3}$\\
		\cline{2-3}
		& $u_0$ & 18 km/s\\
		\hline
		\multirow{10}{*}{\shortstack{Constrained parameters\\ from high-energy CRs}} & $R_d$ & $\qquad$ 15 kpc \\
		\cline{2-3}
		& $H$ & 4 kpc\\
		\cline{2-3}
		& $2h$ & 300 pc\\
		\cline{2-3}
		& $2h_s$ & 80 pc\\
		\cline{2-3}
		& $D(E=10\textrm{ GeV})$ & $5\times 10^{28}$ cm$^2$/s\\
		\cline{2-3}
		& $\delta$ & 0.63\\
		\cline{2-3}
		& $\mathcal{R}_s$ & 0.03 yr$^{-1}$\\
		\cline{2-3}
		& $\xi_{CR}^{(p)}$ & 8.7\%\\
		\cline{2-3}
		& $\xi_{CR}^{(e)}$ & 0.55\%\\
		\cline{2-3}
		& $\alpha$ & 4.23\\   
		\hline
		\hline
	\end{tabular}
\end{table}

\bibliographystyle{apsrev4-2}
\bibliography{mybib}

%\end{document}