\section{Introduction}

For the algorithmic analysis and verification of system models, computing the bisimulation quotient is a natural preprocessing step: it can make the system much smaller while preserving most properties of interest.
This applies equally to probabilistic systems: probabilistic model checkers, e.g., Storm~\cite{Storm20}, speed up the verification process by ``lumping'' together states that are equivalent with respect to probabilistic bisimulation.
While this is a safe approach, it may not be effective when the probabilities in the system are not known precisely.
\begin{figure}[ht]
	\centering
	\tikzstyle{BoxStyle} = [draw, circle, fill=black, scale=0.4,minimum width = 1pt, minimum height = 1pt]
	
	\begin{tikzpicture}[scale=.6,>=latex',shorten >=1pt,node distance=3cm,on grid,auto]
	
	%\node[label]  at (2,2.7) {the MC~$\C$};
	
	\node[state] (s) at (-1,0) {$s_1$};
	\node[state, fill=green] (s2) at (4,0) {$s_2$};
	
	\node[state] (t) at (9,0) {$t_1$};
	\node[state, fill=green] (t2) at (14,0) {$t_2$};
	
%	\node[state] (u) at (19,0) {$u_1$};
%	\node[state, fill=green] (u2) at (24,0) {$u_2$};
	
	\path[->] (s) edge [out=30,in=150] node [midway, above] {$\frac{1}{2}$} (s2);
	\path[->] (s2) edge [out=210,in=-30] node [midway, below] {$\frac{1}{2}$} (s);
	
	\path[->] (t) edge [out=30,in=150] node [midway, above] {$\frac{1}{2}-\epsilon$} (t2);
	\path[->] (t2) edge  [out=210,in=-30]  node [midway, below] {$\frac{1}{2}-\epsilon$} (t);
	
%	\path[->] (u) edge [out=30,in=150] node [midway, above] {$\frac{1}{2}-\frac{\epsilon}{2}$} (u2);
%	\path[->] (u2) edge  [out=210,in=-30]  node [midway, below] {$\frac{1}{2}-\frac{\epsilon}{2}$} (u);
	
	\path[->] (s) edge [loop above]node [midway, above] {$\frac{1}{2}$} (s);
	\path[->] (s2) edge [loop above] node [midway, above] {$\frac{1}{2}$} (s2);
	
	\path[->] (t) edge [loop above] node [midway, above] {$\frac{1}{2}+\epsilon$} (t);
	\path[->] (t2) edge [loop above] node [midway, above] {$\frac{1}{2}+\epsilon$} (t2);
	
%	\path[->] (u) edge [loop above] node [midway, above] {$\frac{1}{2}+\frac{\epsilon}{2}$} (u);
%	\path[->] (u2) edge [loop above] node [midway, above] {$\frac{1}{2}+\frac{\epsilon}{2}$} (u2);
	
	\end{tikzpicture}
\caption{Two intuitively similar labelled Markov chains.}
\label{fig:intro3}
\end{figure}
For example, in the labelled Markov chain shown in \cref{fig:intro3} the states $s_1, t_1$ are intuitively ``similar'', but they are not probabilistically bisimilar even though they carry the same label (here indicated with the colour white) and they both lead, with similar probabilities, to states $s_2, t_2$, which are again intuitively ``similar'' but not probabilistically bisimilar.

For analysis and verification of a probabilistic system, tackling state space explosion is key.
The more symmetry a system has (e.g., due to variables that do not influence the observable behaviour of the model), the greater are the benefits of computing a \emph{quotient} system with respect to probabilistic bisimulation.
However, when the transition probabilities in the Markov chain are perturbed or not known exactly, bisimulation minimisation may fail to capture the symmetries in the system and thus fail to achieve the objective of making the probabilistic system amenable to algorithmic analysis.
The motivation of this paper is to counter this problem.

A principled approach towards this goal may consider notions of \emph{distance} between probabilistic systems or between states in a probabilistic system.
A \emph{probabilistic bisimilarity pseudometric}, based on the Kantorovich metric, was introduced in \cite{DGJP1999,DesharnaisGJP04}, which assigns to each pair of states $s,t$ a number in the interval $[0,1]$ measuring a distance between $s,t$: distance~$0$ means probabilistic bisimilar, and distance~$1$ means, in a sense, maximally non-bisimilar.
This pseudometric can be computed in polynomial time \cite{ChenBW12}, and, quoting \cite{ChenBW12}, has ``been studied in the context of systems biology, games, planning and security, among others''.
The corresponding distances can be intuitively large though: the pseudometric yields a distance less than~$1$ only if the two states can reach, with the same label sequence, two states that are \emph{exactly} bisimilar; see \cite[Section~4]{TangVanBreugel2018}.
As a consequence, for any $\epsilon \gr 0$, the states $s_1, t_1$ in \cref{fig:intro3} have distance~$1$ in the probabilistic bisimilarity pseudometric of \cite{DesharnaisGJP04} (in the undiscounted version).
From a slightly different point of view, a small perturbation of the transition probabilities in the model can change the distance between two states from $0$ to~$1$.

Another pseudometric, \emph{$\epsilon$-bisimulation~$\mathord{\sim_{\epsilon}}$}, was defined in \cite{DesharnaisLavoletteTracol2008}, which avoids this issue.
It has natural characterisations in terms of games and can be computed in polynomial time using network-flow based algorithms \cite{DesharnaisLavoletteTracol2008}.
The runtime of the algorithm from \cite{DesharnaisLavoletteTracol2008} is $O(|S|^7)$, where $S$ is the set of states, thus not practical for large systems.
A more fundamental problem lies in the fact that $\epsilon$-bisimulation is not an equivalence: $s \sim_\epsilon t \sim_\epsilon u$ implies $s \sim_{2 \epsilon} u$ (by the triangle inequality) but not necessarily $s \sim_{\epsilon} u$.
Therefore, efficient minimisation algorithms via quotienting (such as partition refinement for exact probabilistic bisimilarity) are not available for $\epsilon$-bisimulation.

In this paper we develop algorithms that, given a labelled Markov chain~$\M$ with possibly imprecise transition probabilities and a \emph{slightly perturbed} version, say $\Hyp$, of~$\M$, compute a compressed version, $\Q$, of~$\Hyp$.
By slightly perturbed we mean that for each state the successor distributions in $\Hyp$ and~$\M$ have small (say, less than~$\epsilon$) $L_1$-distance.
We hope that $\Q$ is not much bigger than the exact quotient of~$\M$, and we design polynomial-time algorithms that fulfill this hope in practice, but we do not insist on computing the smallest possible~$\Q$.
Indeed, we show that, given an LMC, $\epsilon \gr 0$ and a positive integer $k$, it is {\sf NP}-complete to decide whether there exists a perturbation of at most~$\epsilon$ such that the (exact) bisimulation quotient of the perturbed system is of size $k$.
See, e.g., \cite{BacciBLM18} for an exact but non-polynomial approach, where the target number~$n$ of states is fixed, and a Markov chain with at most~$n$ states is sought that has minimal distance (with respect to the probabilistic bisimilarity pseudometric of \cite{DesharnaisGJP04}) to the given model.

It is not hard to prove (\cref{proposition:approximate-quotient-implies-approximate-bisimulation}) that if an LMC can be made exactly bisimilar to another LMC by a perturbation of at most~$\epsilon$, then these two LMCs are also $\frac{\epsilon}{2}$-bisimilar in the sense of \cite{DesharnaisLavoletteTracol2008}.
If, in turn, two states are $\epsilon$-bisimilar, one can show (see \cite[Theorem~4]{BianA17}) that any linear-time property that depends only on the first $k$ labels has similar probabilities in the two states, where similar means the difference in probabilities is at most $1 - (1-\epsilon)^k$. %However, as we will see later, the algorithms based on $\epsilon$-bisimulation may not produce satisfying minimised systems.
Combining these two results, we obtain a continuity property that says if two LMCs can be made bisimilar by a small perturbation, then any $k$-bounded linear-time property has similar probabilities in the two LMCs.
In other words, our approximate minimisation approximately preserves the probability of bounded linear-time properties.

We apply our approximate minimisation algorithms in a setting of active learning.
Here we assume we do not have access to the transition probabilities of the model; rather, for each state we only \emph{sample} the successor distribution.
Sampling gives us an approximation of the real Markov chain, and our approximate minimisation algorithms apply naturally.
This allows us to lump states that are exactly bisimilar in the real model, but only approximately bisimilar in the sampled model.
We give examples where in this way we \emph{recover} the structure (not the precise transition probabilities) of the quotient of the exact model, knowing only the sampled model.

The rest of the paper is organised as follows. In \cref{section:preliminaries} we introduce $\epsilon$-quotient, a new notion of approximate bisimulation quotient. In \cref{section:approximate-quotient-properties} given an LMC, $\epsilon$ and $k \gr 0$, we show it is ${\sf NP}$-complete to decide whether there exists an $\epsilon$-quotient of size $k$. In \cref{section:minimisation-algorithms} we present our approximate minimisation algorithms.
We put them in a context of active learning in \cref{section:active-LMC-learning}.
In \cref{section:experiments} we evaluate these algorithms on slightly perturbed versions of a number of LMCs taken from the probabilistic model checker PRISM \cite{KNP11}. We conclude in \cref{section:conclusion}. 