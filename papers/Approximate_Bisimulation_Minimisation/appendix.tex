\section{Proofs of \cref{section:approximate-quotient-properties}}\label{appendix: global-bisimulation-properties}

%definitions
We denote by $\SubDist(S)$ the set of probability subdistributions on~$S$. The ceiling of a real number $r$, denoted by $\lceil r\rceil$, is the least integer $n$ such that $n \ge r$.
%%%%%%%%%%%%%%%%%%%%%%%% NP-COMPLETE %%%%%%%%%%%%%%%%%%%%%%%%%%%%%%

\theoremMinimumApproximateQuotientNPComplete*

\begin{proof}
	We first show that this problem is in {\sf NP}. If there is an $\errorParam$-quotient of $\Hyp =<S, L,\tauHyp,\ell>$ of size $k$ which we denote by $\Q$, by definition there is an LMC $\Hyp' = <S, L, \tau', \ell>$ such that $\Hyp'$ and $\Q$ are probabilistic bisimilar. It follows that the state space of $\Hyp'$ can be partitioned into $k$ sets and each set is a probabilistic bisimulation induced equivalence class. We summarise the idea into the following nondeterministic algorithm:  first guess a partition of the state space $S$ into $k$ sets, $E_1, \cdots, E_k$, and then check a) each subset $E_i$ is a probabilistic bisimulation induced equivalence class of $\Hyp'$; b) $\|\tau'(s)-\tauHyp(s)\|_1 \le \errorParam$ for all $s \in S$. This amounts to a feasibility test of the linear program:
	
	
	\begin{align*}
	\exists \tau': S\times S \to [0, 1] \text{ such that } %&\textstyle \tau'(u)(v) \ge 0 \text{ for all $u,v\in S$ and}\\
	&\textstyle\sum_{v \in S} \tau'(u)(v) = 1 \text{ for all $u\in S$ and}\\
	&\|\tauHyp(u) -\tau'(u)\|_1 \le \errorParam \text{ for all $u \in S$ and}\\
	&\tau'(u)(E_j) = \tau'(v)(E_j) \text{ for all $u, v \in E_i$ and all $E_j$},
	\end{align*}
	and hence can be decided in polynomial time.  
	
	The Subset Sum problem is polynomial-time many-one reduction to this problem, hence it is {\sf NP}-hard. 
	
	Given an instance of Subset Sum $<P, N>$ where $P =\{p_1, \cdots, p_n\}$ and $N \in \nat$, we construct an LMC; see \cref{fig:reductionfromSubsset}. Let $T = \sum_{p_i \in P} p_i$. Let $\epsilon = \errorParam = \frac{1}{2T}$ and $k = 5$. In the LMC, state $s$ transitions to state $s_i$ with probability $p_i / T$ for all $1 \le  i \le n$. Each state $s_i$ transitions to $s_a$ and $s_b$ with equal probabilities. State $t$ transitions to $t_1$ and $t_2$ with probability $N / T$ and $1 - N / T$, respectively. State $t_1$ (resp. $t_2$) transitions to $t_a$ (resp. $t_b$) and $t_b$ (resp. $t_a$) with probability $\frac{1}{2} - \epsilon$ and $\frac{1}{2} + \epsilon$, respectively. All the remaining states transition to the successor state with probability one. States $s_b$ and $t_b$ have label $b$ and all other states have label $a$.  
	
	Next, we show that
	$$<S, N> \in {\mbox{Subset Sum}} \iff 
	\text{there is an } \epsilon\text{-quotient of }\Hyp\text{ of size is }k.$$
	
	($\implies$)
	Let $Q \subseteq P$ be the set such that $\sum_{p_i \in Q} p_i = N$. Let us define $\tau': S \to \Dist(S)$ as %such that if $s_i \in P $ then $\tau'(s_i)(s_a) = \frac{(1-\epsilon)}{2}$ and $\tau'(s_i)(s_a) = \frac{(1+\epsilon)}{2}$ otherwise. 
	
	$
	\tau'(u)(v) = \left \{
	\begin{array}{ll}
	\frac{(1-\epsilon)}{2} & \mbox{if $u \in Q, v = s_a$  or $u = t_1, v = t_a$}\\
	\frac{(1+\epsilon)}{2} & \mbox{if $u \in Q, v = s_b$ or $u = t_1, v = t_b$}\\
	\frac{(1+\epsilon)}{2} & \mbox{if $u \not\in Q, v = s_a$ or $u = t_2, v = t_a$}\\
	\frac{(1-\epsilon)}{2} & \mbox{if $u \not\in Q, v = s_b$ or $u = t_2, v = t_b$}\\
	\tau(u)(v) & \mbox{otherwise}\\
	\end{array}
	\right .
	$.
	
	%Consider the following partition of states:
	%$$E_1 = \{s, t\}, E_2 = P \cup \{t_1\}, E_3 = (S\setminus P) \cup \{t_2\}, E_4 = \{s_a, t_a\} \text{ and } E_5 = \{s_b, t_b\}. $$
	We have $\|\tau'(u) - \tauHyp(u)\|_1 \le \epsilon$ for all $u \in S$.	Let $S_Q = \{s_i \in S \suchthat p_i \in Q\}$ and $S_{\overline{Q}} = \{ s_i \in S \suchthat p_i \not\in Q \}$. It is easy to verify that in the LMC $\Hyp' = <S,L,\tau',\ell>$, we have $s_i \sim_{\Hyp'} t_1$ for state $s_i \in S_Q$ and $s_i \sim_{\Hyp'} t_2$ for states $s_i \in S_{\overline{Q}}$. Since $\tau'(s)(S_Q) = \sum_{s_i \in S_Q} \tau'(s)(s_i) = \frac{N}{T} = \tau'(t)(t_1)$ and $\tau'(s)(S_{\overline{Q}}) = \sum_{s_i \in S_{\overline{Q}}} \tau'(s)(s_i) = 1- \frac{N}{T} = \tau'(t)(t_2)$, we have $s \sim_{\Hyp'} t$. There are five probabilistic bisimulation classes of $\Hyp'$: $\{s, t\}$, $S_Q \cup \{t_1\}$, $S_{\overline{Q}} \cup \{t_2\}$, $\{s_a, t_a\}$ and $\{s_b, t_b\}$. It follows that the exact quotient of $\Hyp'$ has five states and it is an $\epsilon$-quotient of $\Hyp$.
	
	($\impliedby$)
	Assume there is an LMC $\Q$ with five states and it is an $\epsilon$-quotient of $\Hyp$. By definition, 	there is a probabilistic transition function $\tau'$ such that $\|\tau'(s)-\tauHyp(s)\|_1 \le \epsilon$ for all $s \in S$ and the LMC $\Hyp' = <S, L, \tau', \ell>$ and $\Q$ are probabilistic bisimilar.
	
	Since any two of the five states $t$, $t_1$, $t_2$, $t_a$ and $t_b$ are not probabilistic bisimilar in $\Hyp'$, each of them will be in a different probabilistic bisimulation class of $\Hyp'$. Since $\|\tau(t_1) - \tau(t_2)\|_1 = 4\epsilon$, we have $t_1 \not\sim_{\Hyp'} t_2$. It is not hard to see that the state $s$ of $\Hyp'$ belongs to the probabilistic bisimulation class that contains $t$. Let $S_1$ be the set of states $s_i$'s such that $s_i \sim_{\Hyp'} t_1$ and $S_2$ be the set of states $s_i$'s such that $s_i \sim_{\Hyp'} t_2$. Let $Q = \{p_i \in P \suchthat s_i \in S_1 \}$. Then, $\tau'(s)(S_1) = \tau'(t)(t_1)$.
	
	Since $\|\tau'(s) - \tauHyp(s)\|_1 \le \epsilon$, we have \begin{equation}
	\label{eqn:np-eq1}
	|\tau'(s)(S_1) -\tauHyp(s)(S_1)| \ls \epsilon.
	\end{equation}
	
	It is not possible to have equality in \eqref{eqn:np-eq1}. Towards a contradiction, assume that we could have $|\tau'(s)(S_1) -\tauHyp(s)(S_1)| = \epsilon$. Without loss of generality, assume $\tau'(s)(S_1) -\tauHyp(s)(S_1) = \epsilon$. We have that 
	\begin{eqnarray*}
		&& \epsilon\\		
		&=& \tau'(s)(S_1) -\tauHyp(s)(S_1) \\
		&=& -1 + 1+ \tau'(s)(S_1) -\tauHyp(s)(S_1) \\
		&=& (\tau'(s)(S_1) -1) + (1-\tauHyp(s)(S_1)) \\
		&=& -\tau'(s)(S_2) + \tauHyp(s)(S_2) 
	\end{eqnarray*}
	Thus,  $|\tauHyp(s)(S_2)  -\tau'(s)(S_2)| = \epsilon$.  It contradicts $\|\tau'(s) -\tauHyp(s)\|_1 \le \epsilon$ as $\|\tau'(s) -\tauHyp(s)\|_1 \ge |\tau'(s) (S_1)-\tauHyp(s)(S_1)| + |\tau'(s) (S_2)-\tauHyp(s)(S_2)| = 2\epsilon$. 
	
	Similarly, since $\|\tau'(t) - \tauHyp(t)\|_1 \le \epsilon$, we have 
	\begin{equation}
	\label{eqn:np-eq2}
	|\tau'(t)(t_1) - \tauHyp(t)(t_1)| \ls \epsilon.
	\end{equation}
	
	
	Then,
	\begin{eqnarray*}
		&&0\\
		&=&	 \tau'(s)(S_1) - \tau'(t)(t_1) \\
		&=&(\tau'(s)(S_1)- \tauHyp(s)(S_1)) +(\tauHyp(t)(t_1) - \tau'(t)(t_1)) +\tauHyp(s)(S_1) - \tauHyp(t)(t_1) \\
		&\le&| \tau'(s)(S_1)-  \tauHyp(s)(S_1)| +|\tauHyp(t)(t_1) - \tau'(t)(t_1)| + \tauHyp(s)(S_1) - \tauHyp(t)(t_1) \\
		%&&\commenteq{triangle inequality}\\
		&\ls& \epsilon + \epsilon + \tauHyp(s)(S_1) - \tauHyp(t)(t_1) \commenteq{\eqref{eqn:np-eq1} and \eqref{eqn:np-eq2}}\\
		&=&2\epsilon + \tauHyp(s)(S_1) - \tauHyp(t)(t_1) 
	\end{eqnarray*}
	
	and 
	\begin{eqnarray*}
		&&0\\
		&=&	\tau'(s)(S_1) - \tau'(t)(t_1) \\
		&=&(\tau'(s)(S_1)-  \tauHyp(s)(S_1)) +(\tauHyp(t)(t_1) - \tau'(t)(t_1)) + \tauHyp(s)(S_1) - \tauHyp(t)(t_1) \\
		&\ge&-|\tau'(s)(S_1)- \tauHyp(s)(S_1)| -|\tauHyp(t)(t_1) - \tau'(t)(t_1)| +  \tauHyp(s)(S_1) - \tauHyp(t)(t_1) \\
		%&&\commenteq{triangle inequality}\\
		&\gr& -\epsilon - \epsilon + \tauHyp(s)(S_1) - \tauHyp(t)(t_1) \commenteq{\eqref{eqn:np-eq1} and \eqref{eqn:np-eq2}}\\
		&=&-2\epsilon +  \tauHyp(s)(S_1) - \tauHyp(t)(t_1) .
	\end{eqnarray*}
	
	From the above, we have 
	\begin{eqnarray*}
		&&-2\epsilon \ls\tauHyp(s)(S_1) - \tauHyp(t)(t_1) \ls 2\epsilon\\
		&\iff& -\frac{1}{T} \ls \tauHyp(s)(S_1) - \tauHyp(t)(t_1) \ls \frac{1}{T} \commenteq{$\epsilon = \frac{1}{2T}$}\\
		&\iff& -\frac{1}{T} \ls \sum_{p_i \in Q} \frac{p_i}{T} - \frac{N}{T} \ls \frac{1}{T}\\
		&\iff& N-1\ls \sum_{p_i \in Q} p_i \ls N+1
	\end{eqnarray*}
	
	Since $\sum_{p_i \in Q} p_i$ is an integer,  it follows that $\sum_{p_i \in Q} p_i = N$.
\end{proof}

%%%%%%%%%%%%%%%%%%%%%%%% APPROXIMATE BISIMULATION %%%%%%%%%%%%%%%%%%%%%%%%%


\propositionApproximateGlobalRelationSubset*

\begin{proof}
	Let $\Q$ be an $\errorParam$-quotient of $\Hyp$. By definition, there is a probability transition function $\tau'$ with $\|\tauHyp(s) - \tau'(s)\|_1 \le \errorParam$ for all $s\in S$ such that $\Q$ is the exact quotient of the LMC $\Hyp' = <S, L, \tau', \ell>$. Let $s$ be an arbitrary state from $\Hyp$. Let $s'$ denote the corresponding state from $\Hyp'$ and $s^{Q} = [s]^{\errorParam}$ the corresponding state from $\Q$.  To prove $s \sim_{\frac{\errorParam}{2}} s^{Q}$, it suffices to prove $s \sim_{\frac{\errorParam}{2}} s'$ since $s' \sim s^{Q}$ and approximate bisimulation satisfies the additivity property~\cite{DesharnaisLavoletteTracol2008}.  
	
	Next, we prove that the following relation $\mathcal{R} = \{(s, s) \suchthat \text{ the first } s \text{ is from } \Hyp \text{ and the second } s \text{ is the corresponding state from } \Hyp'\}$ is an $\frac{\errorParam}{2}$-bisimulation. It is obvious that both states $s$ have the same label and it remains to show that $(\tauHyp(s), \tau'(s)) \in \lifting{\mathcal{R}}_{\frac{\errorParam}{2}}$. To do that, we construct an $\omega \in \Omega(\tauHyp(s), \tau'(s))$ such that $\sum_{(u, v) \in \mathcal{R}} \omega(u, v) \ge 1 -\frac{\errorParam}{2}$. If $\|\tauHyp(s) - \tau'(s)\|_1 = 0$, we can define $\omega(u, u) = \tauHyp(s)(u) = \tau'(s)(u)$ and thus, $\sum_{(u, u) \in \mathcal{R}} \omega(u, u) = \sum_{u \in S}\tauHyp(s)(u) = 1 \ge 1 -\frac{\errorParam}{2}$. For the remainder of this proof, we assume $\|\tauHyp(s) - \tau'(s)\|_1 \gr 0$.
	
	Define $\omega'$ such that $\omega'(u, u) = \min\{\tauHyp(s)(u), \tau'(s)(u)\}$ for all $u \in S$. Let $\alpha \in \SubDist(S)$ such that $\alpha(u) = \max\{\tauHyp(s)(u)-\tau'(s)(u) ,0\}$ where $u \in S$. We have $\|\alpha\|_1 = \sum_{u \in S} \alpha(u) = \sum_{u \in S}  \max\{\tauHyp(s)(u)-\tau'(s)(u), 0\} =\frac{1}{2} \|\tauHyp(s)-\tau'(s)\|_1$. Let $\beta \in \SubDist(S)$ such that $\beta(u) = \max\{\tau'(s)(u)-\tauHyp(s)(u) ,0\}$. Similarly, we have $\|\beta\|_1 = \frac{1}{2} \|\tauHyp(s)-\tau'(s)\|_1$.
	
	Since $\|\alpha\|_1 = \|\beta\|_1 = \frac{1}{2} \|\tauHyp(s)-\tau'(s)\|_1 \gr 0$, there exists a coupling $\gamma \in \Omega(\alpha/\|\alpha\|_1, \beta/ \|\beta\|_1)$. Next, we show that $\omega'+\|\alpha\|_1\gamma \in \Omega(\tauHyp(s), \tau'(s))$. We check the conditions on the left and right marginals, respectively:
	\begin{align*}
	& \phantom{=} \textstyle\sum_{v \in S} (\omega'+\|\alpha\|_1\gamma)(u, v) \\
	&= \textstyle\sum_{v \in S} \omega'(u, v)+\sum_{v \in S} \|\alpha\|_1\gamma(u, v)\\
	&= \textstyle\min\{\tauHyp(s)(u), \tau'(s)(u)\} + \|\alpha\|_1\frac{\alpha(u)}{\|\alpha\|_1} \commenteq{$\sum_{v \in S} \gamma(u, v) = \alpha(u) / \|\alpha\|_1$}\\
	&= \textstyle\min\{\tauHyp(s)(u), \tau'(s)(u)\}  +  \max\{\tauHyp(s)(u)-\tau'(s)(u), 0\} \\
	&= \tauHyp(s)(u) \text{ and }\\
	&\phantom{=} \textstyle\sum_{u\in S} (\omega'+\|\alpha\|_1\gamma)(u, v) \\
	&=\textstyle\sum_{u\in S} (\omega'+\|\beta\|_1\gamma)(u, v) \commenteq{$\|\alpha\|_1 = \|\beta\|_1$}\\
	&= \textstyle\sum_{u \in S} \omega'(u, v)+\sum_{u \in S} \|\beta\|_1\gamma(u, v) \\
	&= \textstyle\min\{\tauHyp(s)(v), \tau'(s)(v)\}  +  \max\{\tau'(s)(v)-\tauHyp(s)(v), 0\} \\
	&= \tau'(s)(v).
	\end{align*}
	
	Finally, we check that the constructed coupling $\omega = \omega'+\|\alpha\|_1\gamma$ satisfies that $\textstyle\sum_{(u, v) \in \mathcal{R}} \omega(u, v) \ge 1 - \frac{\errorParam}{2}$: 
	\begin{align*}
	\textstyle\sum_{(u, v)\in \mathcal{R}}\omega (u, v) &=\textstyle\sum_{(u, u) \in \mathcal{R}} (\omega'+\|\alpha\|_1\gamma)(u, u) \\
	&\ge \textstyle\sum_{(u, u) \in \mathcal{R}} \omega'(u, u) \\
	&= \textstyle\sum_{(u, u) \in \mathcal{R}} \min\{\tauHyp(s)(u), \tau'(s)(u)\} \\	
	&=\textstyle\sum_{u \in S} \min\{\tauHyp(s)(u), \tau'(s)(u)\}\\
	&=\textstyle\sum_{u \in S} \big( \tauHyp(s)(u) - \max\{\tauHyp(s)(u) - \tau'(s)(u), 0\} \big)\\
	&=\textstyle\sum_{u \in S} \tauHyp(s)(u) - \sum_{u \in S} \max\{\tauHyp(s)(u) - \tau'(s)(u), 0\}\\	
	&=1 - \textstyle\sum_{u \in S} \max\{\tauHyp(s)(u) - \tau'(s)(u), 0\}\\	
	&=1 - \frac{1}{2} \|\tauHyp(s)-\tau'(s)\|_1\\		
	&\ge 1 - \frac{\errorParam}{2}\qedhere
	\end{align*}
\end{proof}

%%%%%%%%%%%%%%%%%%%%%%%%%%%%%%%%%%%%%%%%%%%%%%%%%%%%%%%%%%%%%%%%%%
%%%%%%%%%%%%%%%%%%%%%%%%%TRANSITIVE CLOSURE%%%%%%%%%%%%%%%%%%%%%%%%%

\begin{example}\label{example:s-t-not-global-2epsilon-related}
Consider the LMC in \cref{fig:example-not-subseteq-R-2-epsilon}. We have that $t_1 \sim_{\epsilon}  s  \sim_{\epsilon}  t \sim_{\epsilon}  s_1$. It is not hard to see that $t_1 \sim_{\epsilon}  s$ and $t \sim_{\epsilon}  s_1$. Let $\R$ be a reflexive and symmetric relation such that $\{(s, t), (s_1, t), (s, t_1)\} \subseteq \R$ and $(s_1, t_1) \not\in \R$. To show $s \sim_{\epsilon}  t$, it suffices to show that $(\tau(s), \tau(t)) \in \lifting{\R}_{\epsilon}$; more precisely, a coupling $\omega \in \Omega(\tau(s), \tau(t))$ such that $\sum_{ (u,v) \in \R} \omega(u, v) \ge 1 - \epsilon$. Let $\omega(s, t) = \omega(s_1, t) = \omega(s, t_1) = \frac{1}{4}$, $\omega(x,x) = \frac{1}{4} - \epsilon$, $\omega(x, t) = \epsilon$ and $\omega(u,v) = 0$ for all other $u, v \in S$. It is easy to check that $\omega \in \Omega(\tau(s), \tau(t))$ and it satisfies that  $\sum_{ (u,v) \in \mathcal{R}} \omega(u, v) \ge 1 - \epsilon$. Next, we show the resulting LMC in \cref{fig:exampleMerge2} which is obtained by merging the states that are related by the transitive closure of $\sim_{\epsilon}$. This LMC is a $3\epsilon$-quotient of the LMC in \cref{fig:example-not-subseteq-R-2-epsilon}.

\begin{figure}[h]
	\begin{tikzpicture}
	\tikzstyle{BoxStyle} = [draw, circle, fill=black, scale=0.05,minimum width = 0.001pt, minimum height = 0.001pt]
	\node[state] (s) at (1,4) {$s$};
	\node[state] (s1) at (0,2.5) {$s_{1}$};
	\node[state, fill=green] (x) at (2,2.5) {$x$};
	\path[->] (s) edge [loop left] node  [midway,left] {$\frac{1}{2}$} (s);
	\path[->] (s) edge node  [midway,left,xshift=-0.1cm,yshift=0.1cm]  {$\frac{1}{4}$} (s1);
	\path[->] (s) edge node  [midway,right,xshift=0.1cm,yshift=0.1cm] {$\frac{1}{4}$} (x);
	\path[->] (x) edge [out=-70, in=-110, looseness=4] node [near start, right] {$1$} (x);
	
	\node[state] (t) at (4.35,4) {$t$};
	\node[state] (t1) at (3.35,2.5) {$t_{1}$};
	\node[state, fill=green] (y) at (5.35,2.5) {$x$};
	\path[->] (t) edge [loop left] node  [midway,left] {$\frac{1}{2}+\epsilon$} (t);
	\path[->] (t) edge node  [midway,left,xshift=-0.1cm,yshift=0.1cm] {$\frac{1}{4}$} (t1);
	\path[->] (t) edge node  [midway,right,xshift=0.1cm,yshift=0.1cm] {$\frac{1}{4}-\epsilon$} (y);
	\path[->] (y) edge[out=-70, in=-110, looseness=4] node [near start, right] {$1$} (y);
	
	\node[state] (s1) at (11.05,4) {$s_1$};
	\node[state] (t1) at (10.05,2.5) {$t_{1}$};
	\node[state, fill=green] (x) at (12.05,2.5) {$x$};
	\path[->] (s1) edge [loop left] node  [midway,left] {$\frac{1}{2}+2\epsilon$} (s1);
	\path[->] (s1) edge node  [midway,left,xshift=-0.1cm,yshift=0.1cm] {$\frac{1}{4}$} (t1);
	\path[->] (s1) edge node  [midway,right,xshift=0.1cm,yshift=0.1cm] {$\frac{1}{4}-2\epsilon$} (x);
	\path[->] (x) edge  [out=-70, in=-110, looseness=4] node [near start, right]{$1$} (x);
	
	\node[state] (t1) at (7.7,4) {$t_1$};
	\node[state] (ts1) at (6.7,2.5) {$s_{1}$};
	\node[state, fill=green] (y) at (8.7,2.5) {$x$};
	\path[->] (t1) edge [loop left] node  [midway,left] {$\frac{1}{2}-\epsilon$} (t1);
	\path[->] (t1) edge node  [midway,left,xshift=-0.1cm,yshift=0.1cm] {$\frac{1}{4}$} (ts1);
	\path[->] (t1) edge node  [midway,right,xshift=0.1cm,yshift=0.1cm] {$\frac{1}{4}+\epsilon$} (y);
	\path[->] (y) edge  [out=-70, in=-110, looseness=4] node [near start, right] {$1$} (y);
	\end{tikzpicture}
	
	\caption{An LMC in which we have $t_1 \sim_{\epsilon} s \sim_{\epsilon} t \sim_{\epsilon} s_1$ }
	\label{fig:example-not-subseteq-R-2-epsilon}
\end{figure}

\begin{figure}[h]
	\centering
	\begin{tikzpicture}
	\tikzstyle{BoxStyle} = [draw, circle, fill=black, scale=0.05,minimum width = 0.001pt, minimum height = 0.001pt]
	
	\node[state] (t) at (12,4) {$u$};
	\node[state, fill=green] (y) at (12,2.5) {$x$};
	\path[->] (t) edge [loop left] node  [midway,left] {$\frac{3}{4}+\frac{\epsilon}{2}$} (t);
	\path[->] (t) edge node  [midway,right,xshift=0.1cm] {$\frac{1}{4}-\frac{\epsilon}{2}$} (y);
	\path[->] (y) edge [loop right] node [midway, right] {$1$} (y);
	%\node at (12, 1.5) {(c) $2\epsilon$-quotient};	
	%	\node[state] (ss1) at (7,4) {$s_1$};
	%	\node[state, fill=green] (z) at (7,2.5) {$x$};
	%	\path[->] (ss1) edge [loop left] node  [midway,left] {$\frac{3}{4}+2\epsilon$} (ss1);
	%	\path[->] (ss1) edge node  [midway,right,xshift=0.1cm] {$\frac{1}{4}-2\epsilon$} (z);
	%	\path[->] (z) edge [loop right] node [midway, right] {$1$} (z);
	\end{tikzpicture}
	\caption{An $\epsilon'$-quotient ($\epsilon'$ is at least $3\epsilon$) obtained by merging the states that are related by the transitive closure of $\sim_{\epsilon}$: $s$, $s_1$, $t$ and $t_1$.}
	\label{fig:exampleMerge2} 
\end{figure}

\end{example}

%\theoremGlobalBisimulationStronger*


\begin{restatable}{theorem}{theoremGlobalBisimulationStronger}\label{theorem:global-bisimulation-is-stronger}
	Let $n \in \integer^{+}$ and $\epsilon \in (0,  \frac{1}{(\lceil\frac{n}{2}\rceil +1)2^{\lceil\frac{n}{2}\rceil +1}}]$. There is an LMC $\M(n)$ such that merging the states that are related by the transitive closure of $\sim_{\epsilon}$ results in an $\epsilon'$-quotient where $\epsilon'$ is at least $(n+1)\epsilon$. 
\end{restatable}
\begin{proof}
	Let  $n \in \integer^{+}$ and $\epsilon \in (0, \frac{1}{(\lceil\frac{n}{2}\rceil +1)2^{\lceil\frac{n}{2}\rceil +1}}]$. For LMCs $\M(2n-1)$ and $\M(2n)$, we have $\epsilon \in (0, \frac{1}{(n+1)2^{n+1}}]$.
	
	In \cref{fig:example-not-subseteq-R-n-epsilon-odd}, we have $s \sim_{\epsilon} t$. There are $2n+2$ states in total: $s$, $t$, $s_i$ where $i\in\{1,\ldots,n\}$, $t_i$ where $i\in \{1,\ldots,(n-1)\}$ and $x$. All states have the same label but state $x$. State $s$ transitions to $s_i$ with probability $\frac{1}{2^{i+1}}$ where $i \in \{1,\ldots,n\}$. It transitions back to itself with probability $\frac{1}{2}$ and to state $x$ with probability $\frac{1}{2^{n+1}}$. State $t$ transitions to $t_i$ with probability $\frac{1}{2^{i+1}}$ where $i \in \{1,\ldots,(n-2)\}$ and to $t_{n-1}$ with probability $\frac{3}{2^{n+1}}$. It transitions back to itself with probability $\frac{1}{2}+\epsilon$ and to state $x$ with probability $\frac{1}{2^{n+1}}-\epsilon$.  If $n$ is an even number, we have $s_{n-1} \sim_{\epsilon}  t_{n-2}  \cdots s_2 \sim_{\epsilon}  t_1 \sim_{\epsilon}  s \sim_{\epsilon} t \sim_{\epsilon}  s_1 \sim_{\epsilon}  t_2 \cdots  t_{n-1} \sim_{\epsilon} s_{n}$. If $n$ is an odd number, we have $s_{n} \sim_{\epsilon}  t_{n-1}  \cdots s_2 \sim_{\epsilon}  t_1 \sim_{\epsilon}  s \sim_{\epsilon} t \sim_{\epsilon}  s_1 \sim_{\epsilon}  t_2 \cdots  t_{n-2} \sim_{\epsilon} s_{n-1}$. The  LMC obtained by merging the states that are related by the transitive closure of $\sim_{\epsilon}$ is shown in \cref{fig:exampleMerge2n}(a). This LMC is a $\epsilon'$-quotient of the LMC in \cref{fig:example-not-subseteq-R-n-epsilon-odd} where $\epsilon'$ is at least $2n\epsilon$.
	
	%$t_{n}  \sim_{\epsilon} s_{n-1}     \cdots s_2 \sim_{\epsilon}  t_1 \sim_{\epsilon}  s \sim_{\epsilon} t \sim_{\epsilon}  s_1 \sim_{\epsilon}  t_2 \cdots  t_{n-1} \sim_{\epsilon} s_{n}$ 	
	
	
	In \cref{fig:example-not-subseteq-R-n-epsilon-even}, we have $s \sim_{\epsilon} t$. There are $2n+3$ states in total: $s$, $t$, $s_i$ where $i\in\{1,\ldots,n\}$, $t_i$ where $i\in \{1,\ldots,n\}$ and $x$. All states have the same label but state $x$. State $s$ transitions to $s_i$ with probability $\frac{1}{2^{i+1}}$ where $i \in \{1,\ldots,n\}$. It transitions back to itself with probability $\frac{1}{2}$ and to state $x$ with probability $\frac{1}{2^{n+1}}$. State $t$ transitions to $t_i$ with probability $\frac{1}{2^{i+1}}$ where $i \in \{1,\ldots,n\}$. It transitions back to itself with probability $\frac{1}{2}+\epsilon$ and to state $x$ with probability $\frac{1}{2^{n+1}}-\epsilon$.  If $n$ is an even number, we have $s_{n} \sim_{\epsilon}  t_{n-1}  \cdots s_2 \sim_{\epsilon}  t_1 \sim_{\epsilon}  s \sim_{\epsilon} t \sim_{\epsilon}  s_1 \sim_{\epsilon}  t_2 \cdots  s_{n-1} \sim_{\epsilon} t_{n}$. If $n$ is an odd number, we have $t_{n} \sim_{\epsilon}  s_{n-1}  \cdots s_2 \sim_{\epsilon}  t_1 \sim_{\epsilon}  s \sim_{\epsilon} t \sim_{\epsilon}  s_1 \sim_{\epsilon}  t_2 \cdots  t_{n-1} \sim_{\epsilon} s_{n}$. The  LMC obtained by merging the states that are related by the transitive closure of $\sim_{\epsilon}$ is shown in \cref{fig:exampleMerge2n}(b). This LMC is a $\epsilon'$-quotient of the LMC in \cref{fig:example-not-subseteq-R-n-epsilon-even} where $\epsilon'$ is at least $(2n+1)\epsilon$.
	
\end{proof}

\begin{figure}[t]
		\tikzstyle{BoxStyle} = [draw, circle, fill=black, scale=0.05,minimum width = 0.001pt, minimum height = 0.001pt]
	\begin{subfigure}[h]{\textwidth}
		\resizebox{\textwidth}{!}{
			\begin{tikzpicture}              
			\node[state] (us) at (3,4) {$s$};
			\node[BoxStyle] (st) at (3,3.2){};
			\node[state] (u1) at (0.5,1.8) {$s_{1}$};
			\node at (2,1.8) {$\cdots$};	
			\node[state] (u2) at (3,1.8) {$s_{i}$};
			\node at (4,1.8) {$\cdots$};	
			\node[state] (u3) at (5.5,1.8) {$s_{n}$};	
			\node[state, fill=green] (un) at (5.5,3.2) {$x$};
			\path[-] (us) edge node  [near end,left] {} (st);
			\path[->] (st) edge [out=180,in=180,looseness=3] node  [midway,left] {$\frac{1}{2}$} (us);
			\path[->] (st) edge node  [midway,left, xshift=-0.2cm] {$\frac{1}{4}$} (u1);
			\path[->] (st) edge node  [midway,left,yshift=-0.1cm] {$\frac{1}{2^{i+1}}$} (u2);
			\path[->] (st) edge node  [midway,right, xshift=0.2cm] {$\frac{1}{2^{n+1}}$} (u3);
			\path[->] (st) edge node  [midway,above] {$\frac{1}{2^{n+1}}$} (un);
			\path[->] (un) edge [loop above] node [midway, above] {$1$} (un);
			
			\node[state] (ut) at (10,4) {$t$};
			\node[BoxStyle] (tt) at (10,3.2){};
			\node[state] (t2) at (12.5,1.8) {\small $ t_{n-1}$};
			\node at (11,1.8) {$\cdots$};
			\node[state] (ti) at (10,1.8) {$t_{i}$};
			\node at (9,1.8) {$\cdots$};
			\node[state] (t1) at (7.5,1.8) {$t_{1}$};
			\node[state, fill=green] (tn) at (12.5,3.2) {$x$};
			\path[-] (ut) edge node  [near end,left] {} (tt);
			\path[->] (tt) edge [out=180,in=180,looseness=3] node  [midway,left] {$\frac{1}{2} +\epsilon$} (ut);
			\path[->] (tt) edge node  [midway,left, xshift=-0.3cm] {$\frac{1}{4}$} (t1);
			\path[->] (tt) edge node  [midway,left, yshift=-0.1cm] {$\frac{1}{2^{i+1}}$} (ti);	
			\path[->] (tt) edge node  [midway,right, xshift=0.2cm] {$\frac{3}{2^{n+1}}$} (t2);
			\path[->] (tt) edge node  [midway,above] {$\frac{1}{2^{n+1}}-\epsilon$} (tn);
			\path[->] (tn) edge [loop above] node [midway, above] {$1$} (tn);
			\end{tikzpicture}}
	\end{subfigure}
	\begin{subfigure}[h]{\textwidth}
		\resizebox{\textwidth}{!}{
			\begin{tikzpicture}
			\node[state] (us) at (3,4) {$t_j$};
			\node[BoxStyle] (st) at (3,3.2){};
			\node[state] (u1) at (0.5,1.8) {$s_{1}$};
			\node at (2,1.8) {$\cdots$};	
			\node[state] (u2) at (3,1.8) {$s_{i}$};
			\node at (4,1.8) {$\cdots$};	
			\node[state] (u3) at (5.5,1.8) {$s_{n}$};	
			\node[state, fill=green] (un) at (5.5,3.2) {$x$};
			\path[-] (us) edge node  [near end,left] {} (st);
			\path[->] (st) edge [out=180,in=180,looseness=3] node  [midway,left] {$\frac{1}{2}-j\epsilon$} (us);
			\path[->] (st) edge node  [midway,left, xshift=-0.2cm] {$\frac{1}{4}$} (u1);
			\path[->] (st) edge node  [midway,left,yshift=-0.1cm] {$\frac{1}{2^{i+1}}$} (u2);
			\path[->] (st) edge node  [midway,right, xshift=0.2cm] {$\frac{1}{2^{n+1}}$} (u3);
			\path[->] (st) edge node  [midway,above] {\small$\frac{1}{2^{n+1}}+j\epsilon$} (un);
			\path[->] (un) edge [loop above] node [midway, above] {$1$} (un);
			
			\node[state] (ut) at (10,4) {$s_j$};
			\node[BoxStyle] (tt) at (10,3.2){};
			\node[state] (t2) at (12.5,1.8) {\small $ t_{n-1}$};
			\node at (11,1.8) {$\cdots$};
			\node[state] (ti) at (10,1.8) {$t_{i}$};
			\node at (9,1.8) {$\cdots$};
			\node[state] (t1) at (7.5,1.8) {$t_{1}$};
			\node[state, fill=green] (tn) at (12.5,3.2) {$x$};
			\path[-] (ut) edge node  [near end,left] {} (tt);
			\path[->] (tt) edge [out=180,in=180,looseness=3] node  [midway,above,xshift=-1cm] {\small $\frac{1}{2}+(j+1)\epsilon$} (ut);
			\path[->] (tt) edge node  [midway,left, xshift=-0.2cm] {$\frac{1}{4}$} (t1);
			\path[->] (tt) edge node  [midway,left, yshift=-0.1cm] {$\frac{1}{2^{i+1}}$} (ti);	
			\path[->] (tt) edge node  [midway,right, xshift=0.2cm] {$\frac{3}{2^{n+1}}$} (t2);
			\path[->] (tt) edge node  [midway,above] {\tiny $\frac{1}{2^{n+1}}-(j+1)\epsilon$} (tn);
			\path[->] (tn) edge [loop above] node [midway, above] {$1$} (tn);
			\end{tikzpicture}} \caption{$s_j$ for odd $j \in \{1,\ldots,n\}$ and $t_j$ for odd $j \in \{1,\ldots,(n-1)\}$}\label{fig:odd-n-odd-j}
	\end{subfigure}
	\begin{subfigure}[h]{\textwidth}
		\resizebox{\textwidth}{!}{
			\begin{tikzpicture}
			\node[state] (us) at (3,4) {$s_j$};
			\node[BoxStyle] (st) at (3,3.2){};
			\node[state] (u1) at (0.5,1.8) {$s_{1}$};
			\node at (2,1.8) {$\cdots$};	
			\node[state] (u2) at (3,1.8) {$s_{i}$};
			\node at (4,1.8) {$\cdots$};	
			\node[state] (u3) at (5.5,1.8) {$s_{n}$};	
			\node[state, fill=green] (un) at (5.5,3.2) {$x$};
			\path[-] (us) edge node  [near end,left] {} (st);
			\path[->] (st) edge [out=180,in=180,looseness=3] node  [midway,left] {$\frac{1}{2}-j\epsilon$} (us);
			\path[->] (st) edge node  [midway,left, xshift=-0.2cm] {$\frac{1}{4}$} (u1);
			\path[->] (st) edge node  [midway,left,yshift=-0.1cm] {$\frac{1}{2^{i+1}}$} (u2);
			\path[->] (st) edge node  [midway,right, xshift=0.2cm] {$\frac{1}{2^{n+1}}$} (u3);
			\path[->] (st) edge node  [midway,above] {\small$\frac{1}{2^{n+1}}+j\epsilon$} (un);
			\path[->] (un) edge [loop above] node [midway, above] {$1$} (un);
			
			\node[state] (ut) at (10,4) {$t_j$};
			\node[BoxStyle] (tt) at (10,3.2){};
			\node[state] (t2) at (12.5,1.8) {\small $ t_{n-1}$};
			\node at (11,1.8) {$\cdots$};
			\node[state] (ti) at (10,1.8) {$t_{i}$};
			\node at (9,1.8) {$\cdots$};
			\node[state] (t1) at (7.5,1.8) {$t_{1}$};
			\node[state, fill=green] (tn) at (12.5,3.2) {$x$};
			\path[-] (ut) edge node  [near end,left] {} (tt);
			\path[->] (tt) edge [out=180,in=180,looseness=3] node  [midway,above,xshift=-1cm] {\small $\frac{1}{2}+(j+1)\epsilon$} (ut);
			\path[->] (tt) edge node  [midway,left, xshift=-0.2cm] {$\frac{1}{4}$} (t1);
			\path[->] (tt) edge node  [midway,left, yshift=-0.1cm] {$\frac{1}{2^{i+1}}$} (ti);	
			\path[->] (tt) edge node  [midway,right, xshift=0.2cm] {$\frac{3}{2^{n+1}}$} (t2);
			\path[->] (tt) edge node  [midway,above] {\tiny $\frac{1}{2^{n+1}}-(j+1)\epsilon$} (tn);
			\path[->] (tn) edge [loop above] node [midway, above] {$1$} (tn);
			\end{tikzpicture}} \caption{$s_j$ for even $j \in \{1,\ldots,n\}$ and $t_j$ for even $j \in \{1,\ldots,(n-1)\}$}\label{fig:odd-n-even-j}
	\end{subfigure}
	\caption{The LMC $\M(2n-1)$ in which $s \sim_{\epsilon} t$. }\label{fig:example-not-subseteq-R-n-epsilon-odd}
\end{figure}

\begin{figure}[t]
	\tikzstyle{BoxStyle} = [draw, circle, fill=black, scale=0.05,minimum width = 0.001pt, minimum height = 0.001pt]
	\begin{subfigure}[h]{\textwidth}
		\resizebox{\textwidth}{!}{
			\begin{tikzpicture}              
			\node[state] (us) at (3,4) {$s$};
			\node[BoxStyle] (st) at (3,3.2){};
			\node[state] (u1) at (0.5,1.8) {$s_{1}$};
			\node at (2,1.8) {$\cdots$};	
			\node[state] (u2) at (3,1.8) {$s_{i}$};
			\node at (4,1.8) {$\cdots$};	
			\node[state] (u3) at (5.5,1.8) {$s_{n}$};	
			\node[state, fill=green] (un) at (5.5,3.2) {$x$};
			\path[-] (us) edge node  [near end,left] {} (st);
			\path[->] (st) edge [out=180,in=180,looseness=3] node  [midway,left] {$\frac{1}{2}$} (us);
			\path[->] (st) edge node  [midway,left, xshift=-0.2cm] {$\frac{1}{4}$} (u1);
			\path[->] (st) edge node  [midway,left,yshift=-0.1cm] {$\frac{1}{2^{i+1}}$} (u2);
			\path[->] (st) edge node  [midway,right, xshift=0.2cm] {$\frac{1}{2^{n+1}}$} (u3);
			\path[->] (st) edge node  [midway,above] {$\frac{1}{2^{n+1}}$} (un);
			\path[->] (un) edge [loop above] node [midway, above] {$1$} (un);
			
			\node[state] (ut) at (10,4) {$t$};
			\node[BoxStyle] (tt) at (10,3.2){};
			\node[state] (t2) at (12.5,1.8) {$ t_{n}$};
			\node at (11,1.8) {$\cdots$};
			\node[state] (ti) at (10,1.8) {$t_{i}$};
			\node at (9,1.8) {$\cdots$};
			\node[state] (t1) at (7.5,1.8) {$t_{1}$};
			\node[state, fill=green] (tn) at (12.5,3.2) {$x$};
			\path[-] (ut) edge node  [near end,left] {} (tt);
			\path[->] (tt) edge [out=180,in=180,looseness=3] node  [midway,left] {$\frac{1}{2}+\epsilon$} (ut);
			\path[->] (tt) edge node  [midway,left, xshift=-0.2cm] {$\frac{1}{4}$} (t1);
			\path[->] (tt) edge node  [midway,left, yshift=-0.1cm] {$\frac{1}{2^{i+1}}$} (ti);	
			\path[->] (tt) edge node  [midway,right, xshift=0.2cm] {$\frac{1}{2^{n+1}}$} (t2);
			\path[->] (tt) edge node  [midway,above] {$\frac{1}{2^{n+1}}-\epsilon$} (tn);
			\path[->] (tn) edge [loop above] node [midway, above] {$1$} (tn);
			\end{tikzpicture}}
	\end{subfigure}
	\begin{subfigure}[h]{\textwidth}
		\resizebox{\textwidth}{!}{
			\begin{tikzpicture}
			\node[state] (us) at (3,4) {$t_j$};
			\node[BoxStyle] (st) at (3,3.2){};
			\node[state] (u1) at (0.5,1.8) {$s_{1}$};
			\node at (2,1.8) {$\cdots$};	
			\node[state] (u2) at (3,1.8) {$s_{i}$};
			\node at (4,1.8) {$\cdots$};	
			\node[state] (u3) at (5.5,1.8) {$s_{n}$};	
			\node[state, fill=green] (un) at (5.5,3.2) {$x$};
			\path[-] (us) edge node  [near end,left] {} (st);
			\path[->] (st) edge [out=180,in=180,looseness=3] node  [midway,left] {$\frac{1}{2}-j\epsilon$} (us);
			\path[->] (st) edge node  [midway,left, xshift=-0.2cm] {$\frac{1}{4}$} (u1);
			\path[->] (st) edge node  [midway,left,yshift=-0.1cm] {$\frac{1}{2^{i+1}}$} (u2);
			\path[->] (st) edge node  [midway,right, xshift=0.2cm] {$\frac{1}{2^{n+1}}$} (u3);
			\path[->] (st) edge node  [midway,above] {\small$\frac{1}{2^{n+1}}+j\epsilon$} (un);
			\path[->] (un) edge [loop above] node [midway, above] {$1$} (un);
			
			\node[state] (ut) at (10,4) {$s_j$};
			\node[BoxStyle] (tt) at (10,3.2){};
			\node[state] (t2) at (12.5,1.8) {$ t_{n}$};
			\node at (11,1.8) {$\cdots$};
			\node[state] (ti) at (10,1.8) {$t_{i}$};
			\node at (9,1.8) {$\cdots$};
			\node[state] (t1) at (7.5,1.8) {$t_{1}$};
			\node[state, fill=green] (tn) at (12.5,3.2) {$x$};
			\path[-] (ut) edge node  [near end,left] {} (tt);
			\path[->] (tt) edge [out=180,in=180,looseness=3] node  [midway,above,xshift=-1cm] {\small $\frac{1}{2}+(j+1)\epsilon$} (ut);
			\path[->] (tt) edge node  [midway,left, xshift=-0.2cm] {$\frac{1}{4}$} (t1);
			\path[->] (tt) edge node  [midway,left, yshift=-0.1cm] {$\frac{1}{2^{i+1}}$} (ti);	
			\path[->] (tt) edge node  [midway,right, xshift=0.2cm] {$\frac{1}{2^{n+1}}$} (t2);
			\path[->] (tt) edge node  [midway,above] {\tiny $\frac{1}{2^{n+1}}-(j+1)\epsilon$} (tn);
			\path[->] (tn) edge [loop above] node [midway, above] {$1$} (tn);
			\end{tikzpicture}} \caption{$s_j$ and $t_j$ for odd $j \in \{1,\ldots,n\}$}\label{fig:even-n-odd-j}
	\end{subfigure}
	\begin{subfigure}[h]{\textwidth}
		\resizebox{\textwidth}{!}{
			\begin{tikzpicture}
			\node[state] (us) at (3,4) {$s_j$};
			\node[BoxStyle] (st) at (3,3.2){};
			\node[state] (u1) at (0.5,1.8) {$s_{1}$};
			\node at (2,1.8) {$\cdots$};	
			\node[state] (u2) at (3,1.8) {$s_{i}$};
			\node at (4,1.8) {$\cdots$};	
			\node[state] (u3) at (5.5,1.8) {$s_{n}$};	
			\node[state, fill=green] (un) at (5.5,3.2) {$x$};
			\path[-] (us) edge node  [near end,left] {} (st);
			\path[->] (st) edge [out=180,in=180,looseness=3] node  [midway,left] {$\frac{1}{2}-j\epsilon$} (us);
			\path[->] (st) edge node  [midway,left, xshift=-0.2cm] {$\frac{1}{4}$} (u1);
			\path[->] (st) edge node  [midway,left,yshift=-0.1cm] {$\frac{1}{2^{i+1}}$} (u2);
			\path[->] (st) edge node  [midway,right, xshift=0.2cm] {$\frac{1}{2^{n+1}}$} (u3);
			\path[->] (st) edge node  [midway,above] {\small$\frac{1}{2^{n+1}}+j\epsilon$} (un);
			\path[->] (un) edge [loop above] node [midway, above] {$1$} (un);
			
			\node[state] (ut) at (10,4) {$t_j$};
			\node[BoxStyle] (tt) at (10,3.2){};
			\node[state] (t2) at (12.5,1.8) {$ t_{n}$};
			\node at (11,1.8) {$\cdots$};
			\node[state] (ti) at (10,1.8) {$t_{i}$};
			\node at (9,1.8) {$\cdots$};
			\node[state] (t1) at (7.5,1.8) {$t_{1}$};
			\node[state, fill=green] (tn) at (12.5,3.2) {$x$};
			\path[-] (ut) edge node  [near end,left] {} (tt);
			\path[->] (tt) edge [out=180,in=180,looseness=3] node  [midway,above,xshift=-1cm] {\small $\frac{1}{2} +(j+1)\epsilon$} (ut);
			\path[->] (tt) edge node  [midway,left, xshift=-0.2cm] {$\frac{1}{4}$} (t1);
			\path[->] (tt) edge node  [midway,left, yshift=-0.1cm] {$\frac{1}{2^{i+1}}$} (ti);	
			\path[->] (tt) edge node  [midway,right, xshift=0.2cm] {$\frac{1}{2^{n+1}}$} (t2);
			\path[->] (tt) edge node  [midway,above] {\tiny $\frac{1}{2^{n+1}}-(j+1)\epsilon$} (tn);
			\path[->] (tn) edge [loop above] node [midway, above] {$1$} (tn);
			\end{tikzpicture}} \caption{$s_j$ and $t_j$ for even $j \in \{1,\ldots,n\}$}\label{fig:even-n-even-j}
	\end{subfigure}
	\caption{The LMC $\M(2n)$ in which $s \sim_{\epsilon} t$ but not $s R_{2n\epsilon} t$ }\label{fig:example-not-subseteq-R-n-epsilon-even}
\end{figure}

\begin{figure}[H]
	\centering
	\begin{tikzpicture}
	\tikzstyle{BoxStyle} = [draw, circle, fill=black, scale=0.05,minimum width = 0.001pt, minimum height = 0.001pt]
	
	\node[state] (t) at (12,4) {$u$};
	\node[state, fill=green] (y) at (12,2.5) {$x$};
	\path[->] (t) edge [loop left] node  [midway,left] {$1 - \frac{1}{2^{n+1}}$} (t);
	\path[->] (t) edge node  [midway,right,xshift=0.1cm] {$\frac{1}{2^{n+1}}$} (y);
	\path[->] (y) edge [loop right] node [midway, right] {$1$} (y);
	\node at (12, 1.5) {(a) $\epsilon'$-quotient ($\epsilon'$ is at least $2n\epsilon$)};	
	
	\node[state] (t) at (18,4) {$u$};
	\node[state, fill=green] (y) at (18,2.5) {$x$};
	\path[->] (t) edge [loop left] node  [midway,left] {$1 - \frac{1}{2^{n+1}} + \frac{\epsilon}{2}$} (t);
	\path[->] (t) edge node  [midway,right,xshift=0.1cm] {$\frac{1}{2^{n+1}} - \frac{\epsilon}{2}$} (y);
	\path[->] (y) edge [loop right] node [midway, right] {$1$} (y);
	\node at (18, 1.5) {(b) $\epsilon'$-quotient  ($\epsilon'$ is at least $(2n+1)\epsilon$)};	
	\end{tikzpicture}
	\caption{(a) An $\epsilon'$-quotient with $\epsilon'$ at least $2n\epsilon$ obtained by merging the states that are related by the transitive closure of $\sim_{\epsilon}$ comprising $s$, $t$, $s_j$ where $j \in \{1, \ldots, n\}$ and $t_j$ where $j \in \{1, \ldots, n-1\}$; (b) An $\epsilon'$-quotient with $\epsilon'$ at least $(2n+1)\epsilon$ obtained by merging the states that are related by the transitive closure of $\sim_{\epsilon}$ comprising $s$, $t$, $s_j$ and $t_j$ for all $j \in \{1, \ldots, n\}$.}
	\label{fig:exampleMerge2n} 
\end{figure}
%%%%%%%%%%%%%%%%%%%%%%%%%%%%%%%%%%%%%%%%%%%%%%%%%%%%%%%%%%%%%%%

%TODO add some words to intro the lemma
 
\begin{lemma}\label{lemma:adjust-probability-distribution-exists}
	Let $X$ be a partition of $S$. Let $\mu \in\Dist(S)$ and $\gamma \in \Dist(X)$. One can compute in polynomial time a probability distribution $\nu \in \Dist(S)$ such that $(\nu(E))_{E \in X} = \gamma$ and $\|\mu - \nu\|_1 = \|(\mu(E))_{E \in X} - (\nu(E))_{E \in X}\|_1$.
\end{lemma}
\begin{proof}
	\cref{alg:adjust-transition-probability}: We can define such a probability distribution $\nu$ by initializing it to $\mu$. For each $E \in X$, if $\mu(E) \le \gamma(E)$, we choose a state $x \in E$ and increase the value of $\nu(x)$ by $\mu(E) - \gamma(E)$, as shown on line~5 of \cref{alg:adjust-transition-probability}. Otherwise $\mu(E) \gr \gamma(E)$, we run Algorithm~\ref{alg:adjust-transition-probability2} to decrease $\nu(x)$ at one or more states $x \in E$ such that $\nu(E) = \gamma(E)$.
\end{proof}
\noindent
\begin{minipage}[h]{0.53\textwidth}
	\vspace{0pt}  
	\setlength{\algomargin}{0.001in}
	\begin{algorithm}[H]
		\DontPrintSemicolon
		\KwIn{$\mu \in \Dist(S)$, a partition $X$ of $S$ and $\gamma \in \Dist(X)$}
		\KwOut{$\nu \in \Dist(S)$ such that $(\nu(E))_{E \in X} = \gamma$ and $\|\mu - \nu\|_1 = \|(\mu(E))_{E \in X} - (\nu(E))_{E \in X}\|_1$}
		$\nu := \mu$ \;
		\ForEach{$E \in X$}{
			$m: = \mu(E) - \gamma(E)$\\
			\eIf{$m \le 0$}{
				pick $x \in E$; $\nu(x) := \nu(x) + m$\;
			}{
				decreaseProb($\nu$, $E$, $m$)
				%			\ForEach{$x \in E$}{
				%				\eIf{$\nu(x) \ge m$}{
				%					$\nu(x) := \nu(x) - m$\;
				%					$m := 0$ \;	
				%				}{
				%					$\nu(x) := 0$\;
				%					$m := m - \nu(x)$ \;	
				%				}
				%				\If{ $m = 0$}{
				%					break\;
				%				}
				%			}
			}
		}
		\caption{Adjust Probability Distribution}
		\label{alg:adjust-transition-probability}
	\end{algorithm}
\end{minipage}
~
\begin{minipage}[h]{0.45\textwidth}
	\setlength{\algomargin}{0.01in}
	\begin{algorithm}[H] 
		\LinesNotNumbered
		\DontPrintSemicolon
		\KwIn{$\nu \in \Dist(S)$, $E \subseteq S$ and $m$}
		\KwOut{$\nu(E)$ decreased by $m$}
		\ForEach{$x \in E$}{
			\eIf{$\nu(x) \ge m$}{
				$\nu(x) := \nu(x) - m$\;
				$m := 0$ \;	
			}{
				$m := m - \nu(x)$ \;	
				$\nu(x) := 0$\;
			}
			\If{ $m = 0$}{
				break\;
			}
		}
		\caption{decreaseProb}
		\label{alg:adjust-transition-probability2}
	\end{algorithm}
\end{minipage}
%%%%%%%%%%%%%%%%%%%%%%%% ADDITIVITY %%%%%%%%%%%%%%%%%%%%%%%%%%%%%%

\lemmaAdditivityProperty*
%\lemmaTriangleInequality*

\begin{proof}
	%The existence of the two surjective functions $f$ and $g$ suggests that $|S_1| \ge |S_2| \ge |S_3|$. 
	Let $\M_1 = <S_1, L, \tau_1, \ell_1>$, $\M_2 = <S_2, L, \tau_2, \ell_2>$ and $\M_3 = <S_3, L, \tau_3, \ell_3>$ be three LMCs. In addition, let $\M_2$ be an $\epsilon_1$-quotient of $\M_1$ and $\M_3$ be an $\epsilon_2$-quotient of $\M_2$. 
	
	By definition, both $\M_2$ and $\M_3$ are quotients, that is, no two states in $\M_2$ or $\M_3$ are probabilistic bisimilar. Let $s_1$ be an arbitrary state from $\M_1$ and $s_2 = [s_1]^{\epsilon_1}_{\M_2}$ be the corresponding state from $\M_2$. The labels of $s_1$ and $s_2$ must coincide. In addition, there exists a transition function $\tau_1': S_1 \times S_1 \to [0, 1]$ such that $\|\tau_1(x_1)- \tau_1'(x_1)\|_1 \le \epsilon_1$ and $x_1 \sim [x_1]^{\epsilon_1}_{\M_2}$ for all $x_1 \in S_1$ in the LMC combining $\M_1' = <S_1, L, \tau_1', \ell_1>$ and $\M_2$. 
	
	Let $s_3 = [s_2]^{\epsilon_2}_{\M_3}$ from $\M_3$ be the state that corresponds to $s_2$. Similarly, the labels of $s_2$ and $s_3$ must be the same. Also, there exists a transition function $\tau_2': S_2 \times S_2 \to [0, 1]$ such that $\|\tau_2(x_2) -\tau_2'(x_2)\|_1\le \epsilon_2$ and $x_2 \sim [x_2]^{\epsilon_2}_{\M_3}$ for all $x_2 \in S_2$ in the LMC combining $\M_2' = <S_2, L, \tau_2', \ell_2>$ and $\M_3$.
	
	To prove $\M_3$ is an $(\epsilon_1 + \epsilon_2)$-quotient of $\M_1$, we show the existence of a transition function $\tau_1'': S_1 \to \Dist(S_1)$ such that $\|\tau_1(x_1) -\tau_1''(x_1)\|_1\le \epsilon_1+ \epsilon_2$ and $x_1 \sim [x_2]^{\epsilon_2}_{\M_3}$ for all $x_1 \in S_1$ in the LMC combining $\M_1'' = <S_1, L, \tau_1'', \ell_1>$ and $\M_3$ where $x_2 = [x_1]^{\epsilon_1}_{\M_2}$.
	
	%Next, we show how to construct the transition function $\tau_1'': S_1 \to \Dist(S_1)$. Let the range of $f$ be $S_2'$ and the range of $g$ be $S_3'$. We have $S_2' \subseteq S_2$ and $S_3' \subseteq S_3$. 
	
	Let us define a function $f: S_2 \to 2^{S_1}$ such that $f(x_2) = \{x_1 \suchthat x_2 = [x_1]^{\epsilon_1}_{\M_2}\}$. Similarly, define a function $g: S_3 \to 2^{S_2}$ such that $g(x_3) = \{x_2 \suchthat x_3 = [x_2]^{\epsilon_2}_{\M_3}\}$. The function $f$ induces a partition $X_1$ over $S_1$: $X_1 = \{ f(x_2) \suchthat x_2 \in S_2\}$. 
	
	Let $x_1 \in S_1$ and $x_2 = [x_1]^{\epsilon_1}_{\M_2}$. We have
	\begin{align}
	&\tau_1'(x_1)(f(y_2))= \tau_2(x_2)(y_2) \text{ for all $y_2 \in S_2$, that is,  }(\tau_1'(x_1)(E))_{E \in X_1} = \tau_2(x_2) \label{eqn:property-of-X1}.
	\end{align} 
	
	Similarly, 	let $x_2 \in S_2$ and $x_3 = [x_2]^{\epsilon_2}_{\M_3}$. We have
	\begin{align}
	&\tau_2'(x_2)(g(y_3))= \tau_3(x_3)(y_3) \text{ for all $y_3 \in S_3$}\label{eqn:property-of-g}.
	\end{align} 
	
	By \cref{lemma:adjust-probability-distribution-exists}, given $X_1$, $\tau_1'(x_1)$ and $\tau_2'(x_2)$, one can compute in polynomial time a probability distribution $\nu_{x_1}$ such that 
	\begin{align}
	&\nu_{x_1}(f(y_2))= \tau_2'(x_2)(y_2) \text{ for all } y_2 \in S_2 \text{, that is,  }  \label{eqn:property1-of-nu-s1}\\
	&(\nu_{x_1}(E))_{E \in X_1} = \tau_2'(x_2) \text{ and }  \nonumber\\
	&\|\nu_{x_1} -  \tau_1'(x_1)\|_1 = \| (\nu_{x_1}(E))_{E \in X_1} - (\tau_1'(x_1)(E))_{E \in X_1}\|_1 \label{eqn:property2-of-nu-s1}.
	\end{align}
	
	For each $x_1 \in S_1$, we compute such $\nu_{x_1}$. Define $\tau_1''$ as $\tau_1''(x_1) = \nu_{x_1}$ for all $x_1 \in S_1$. We first show that $\|\tau_1''(x_1)- \tau_1(x_1)\|_1 \le \epsilon_1+\epsilon_2$ for all $x_1 \in S_1$. Let $x_1 \in S_1$ and $x_2 = [x_1]^{\epsilon_1}_{\M_2}$. We have 
	\begin{eqnarray*} 
		%&&    2d_{\tv}(\tau_1''(x_1), \tau_1(x_1)) \\
		&&    \|\tau_1''(x_1) - \tau_1(x_1)\|_1 \\
		&=& \|\nu_{x_1} - \tau_1(x_1)\|_1 \\
		&\le& \|\nu_{x_1} - \tau_1'(x_1)\|_1 +   \|\tau_1'(x_1) - \tau_1(x_1)\|_1 \commenteq{triangle inequality}\\
		& \le & \|\nu_{x_1} - \tau_1'(x_1)\|_1 +   \epsilon_1 \commenteq{by assumption}\\
		& = & \sum_{E \in X_1} |\nu_{x_1}(E) - \tau_1'(x_1)(E)| +   \epsilon_1\commenteq{\cref{eqn:property2-of-nu-s1}}\\
		& = & \sum_{y_2 \in S_2} |\nu_{x_1}(f(y_2)) - \tau_1'(x_1)(f(y_2))| +   \epsilon_1\commenteq{definition of $X_1$}\\ 	 	
		& = &  \sum_{y_2 \in S_2} |\tau_2'(x_2)(y_2) - \tau_2(x_2)(y_2)| +   \epsilon_1\commenteq{\cref{eqn:property1-of-nu-s1} and \cref{eqn:property-of-X1}}\\ 	 	
		& \le &  \epsilon_2 +   \epsilon_1\commenteq{by assumption}\\ 	 	 			 			
	\end{eqnarray*}
	
	Let $\M_1'' = <S_1, L, \tau_1'', \ell_1>$. It remains to show that $x_1 \sim [x_2]^{\epsilon_2}_{\M_3}$ for all $x_1 \in S_1$ in the LMC combining $\M_1'' = <S_1, L, \tau_1'', \ell_1>$ and $\M_3$ where $x_2 = [x_1]^{\epsilon_1}_{\M_2}$. 
	
	%Let the set $S_3'' \subseteq S_3$ be the range of $h$. 
	
	Define a function $h: S_3 \to 2^{S_1}$ such that $h(x_3) = \{x_1 \suchthat x_2 \in g(x_3) \land x_1 \in f(x_2) \}$. A partition $X_1'$ over $S_1$ is induced by $h$: $X_1' = \{ h(x_3) \suchthat x_3 \in S_3 \}$. It suffices to show that $x_3 \sim x_1$ for any $x_3 \in S_3$ and any $x_1 \in h(x_3)$ in the LMC combining $\M_1''$ and $\M_3$. Let $x_1 \in S_1$,  $x_2 = [x_1]^{\epsilon_1}_{\M_2}$ and $x_3 = [x_2]^{\epsilon_2}_{\M_3}$. We have $x_1 \in h(x_3)$. Let $y_3 \in S_3$. Then,
	\begin{eqnarray*} 
		&&   \tau_1''(x_1)(h(y_3))  \\
		&=& \nu_{x_1}(h(y_3))  \commenteq{definition of $\tau_1''$}\\
		&=& \sum_{y_1 \in h(y_3)} \nu_{x_1}(y_1)  \\
		&=& \sum_{y_2 \in g(y_3)} \sum_{y_1 \in  f(y_2)} \nu_{x_1}(y_1) \commenteq{definition of $h$}\\
		&=& \sum_{y_2 \in g(y_3)} \nu_{x_1}(f(y_2)) \\
		&=& \sum_{y_2 \in g(y_3)} \tau_2'(x_2)(y_2)\commenteq{\cref{eqn:property1-of-nu-s1}}\\ 
		&=& \tau_2'(x_2)(g(y_3))\\ 
		&=& \tau_3(x_3)(y_3)\commenteq{\cref{eqn:property-of-g}} 				
	\end{eqnarray*}
	
	Since $\tau_1''(x_1)(h(y_3)) = \tau_3(x_3)(y_3) $ for all $y_3 \in S_3$, we have $x_1 \sim x_3$ in the LMC combining $\M_1''$ and $\M_3$. By definition of approximate quotient, we have that $\M_3$ is an $(\epsilon_1 + \epsilon_2)$-quotient of $\M_1$.
\end{proof}

\newpage
\section{Proofs of \cref{subsection:local-bisimilarity-distances}}\label{appendix:local-bisimilarity-distance}

\begin{lemma}\label{lemma:local-bisimilarity-distances-witness-transition-funciton}
Let us define a transition function $\tauHyp' \in {\rm T}$ as 
\[
	\tauHyp'(x) = \left \{
	\begin{array}{ll}
	\tauHyp(x)& \mbox{if $x \not\in \{s, t\}$}\\
	\nu_x & \mbox{otherwise}
	\end{array}
	\right .
\]
where $\nu_s$ (resp. $\nu_t$) is computed by running \cref{alg:adjust-transition-probability} with $X$, $\tauHyp(s)$ (resp. $\tauHyp(t)$) and $\frac{(\tauHyp(s)(E))_{E \in X} + (\tauHyp(t)(E))_{E \in X}}{2}$. We have $d_{\local}^{\Hyp}(s, t) = \max\{\|\tauHyp'(s) - \tauHyp(s)\|_1, \|\tauHyp'(t) - \tauHyp(t)\|_1 \} $. 
\end{lemma}
\begin{proof}
	Let $\gamma = \frac{(\tauHyp(s)(E))_{E \in X} + (\tauHyp(t)(E))_{E \in X}}{2}$. The following function $f: \Dist(X) \to \mathbb{R}$ attains its minimum at $\gamma$: $f(\phi) = \max\{\|\phi - (\tauHyp(s)(E))_{E \in X}\|_1, \|\phi - (\tauHyp(t)(E))_{E \in X}\|_1\} \text{ where } \phi \in \Dist(X)$.

By running \cref{alg:adjust-transition-probability} with $X$, $\tauHyp(s)$ (resp. $\tauHyp(t)$) and $\gamma$, we compute a probability distribution $\nu_s$ (resp. $\nu_t$). By \cref{lemma:adjust-probability-distribution-exists}, we have that $(\nu_s(E))_{E \in X} = (\nu_t(E))_{E \in X}  = \gamma$ and $\|\nu_s - \tauHyp(s)\|_1 = \|\gamma - (\tauHyp(s)(E))_{E \in  X}\|_1 = \|\gamma - (\tauHyp(t)(E))_{E \in  X}\|_1 = \|\nu_t - \tauHyp(t)\|_1$. Define the probability transition function $\tauHyp' \in {\rm T}$ as 	
\[
\tauHyp'(x) = \left \{
\begin{array}{ll}
\tauHyp(x)& \mbox{if $x \not\in \{s, t\}$}\\
\nu_x & \mbox{otherwise}
\end{array}
\right .
\]

Let $\Hyp' = <S, L, \tauHyp', \ell>$. We have $s\sim_{\Hyp'} t$ since $ (\tauHyp'(s)(E))_{E \in X} =(\nu_s(E))_{E \in X} = \gamma = (\nu_t(E))_{E \in X} = (\tauHyp'(t)(E))_{E \in  X}$.  It remains to show that $\tauHyp'= \arg\min_{\tau \in {\rm T} } \max\{ \|\tauHyp'(s) - \tauHyp(s)\|_1 , \|\tauHyp'(t) - \tauHyp(t)\|_1\}$. Let $\tau \in {\rm T}$. Then,
\begin{eqnarray*} 
	&&\max\big\{\|\tau(s) -\tauHyp(s)\|_1, \|\tau(t) -\tauHyp(t)\|_1\big\} \\%\commenteq{$\tau'' = \adjust(\tau', s, t, u)$} \\
	&=& \max\big\{ \sum_{v \in S} |\tau(s)(v) -\tauHyp(s)(v)|, \sum_{v \in S} |\tau(t)(v)-\tauHyp(t)(v)|\big\}\\
	&=&\max\big\{\sum_{E \in X } \sum_{v \in E} |\tau(s)(v) -\tauHyp(s)(v)|, \\
	&& \phantom{\max;;}\sum_{E \in X } \sum_{v \in E} |\tau(t)(v)-\tauHyp(t)(v)|\big\}\\
	&\ge& \max\big\{\|(\tau(s)(E))_{E \in X} -(\tauHyp(s)(E))_{E \in X}\|_1 ,  \\
	&& \phantom{\max;;} \|(\tau(t)(E))_{E \in X} -(\tauHyp(t)(E))_{E \in X}\|_1\big\}\commenteq{triangle inequality}.\\
	&\ge& \max\big\{\|\gamma -(\tauHyp(s)(E))_{E \in X}\|_1 ,  \|\gamma-(\tauHyp(t)(E))_{E \in X}\|_1\big\}\\
	&& \phantom{\max;;} \commenteq{$(\tau(s)(E))_{E \in X}  = (\tau(t)(E))_{E \in X} $ since $s \mathord{\equiv_{X}} t$; $f$ attains minimum at $\gamma$}\\
	&=& \max\big\{\|(\tauHyp'(s)(E))_{E \in X} -(\tauHyp(s)(E))_{E \in X}\|_1, \\
	&& \phantom{\max;;} \|(\tauHyp'(t)(E))_{E \in X} -(\tauHyp(t)(E))_{E \in X}\|_1\big\}\\
	&=& \max\big\{\|\tauHyp'(s) -\tauHyp(s)\|_1, \|\tauHyp'(t) -\tauHyp(t)\|_1\big\} 
\end{eqnarray*}	
\end{proof}


\propositionAdjustTransitionFunction*

\begin{proof}
Let $\gamma = \frac{(\tauHyp(s)(E))_{E \in X} + (\tauHyp(t)(E))_{E \in X}}{2}$. By running \cref{alg:adjust-transition-probability} with $X$, $\tauHyp(s)$ (resp. $\tauHyp(t)$) and $\gamma$, we compute a probability distribution $\nu_s$ (resp. $\nu_t$). By \cref{lemma:adjust-probability-distribution-exists}, we have that $(\nu_s(E))_{E \in X} = (\nu_t(E))_{E \in X}  = \gamma$ and $\|\nu_s - \tauHyp(s)\|_1 = \|\gamma - (\tauHyp(s)(E))_{E \in  X}\|_1 = \|\gamma - (\tauHyp(t)(E))_{E \in  X}\|_1 = \|\nu_t - \tauHyp(t)\|_1$. Define the probability transition function $\tauHyp' \in {\rm T}$ as 	
\[
\tauHyp'(x) = \left \{
\begin{array}{ll}
\tauHyp(x)& \mbox{if $x \not\in \{s, t\}$}\\
\nu_x & \mbox{otherwise.}
\end{array}
\right .
\]

It follows from \cref{lemma:local-bisimilarity-distances-witness-transition-funciton} that $d_{\local}^{\Hyp}(s, t) = \max\{\|\tauHyp'(s) - \tauHyp(s)\|_1, \|\tauHyp'(t) - \tauHyp(t)\|_1 \} $. Then, $d_{\local}^{\Hyp}(s, t) = \max\{\|\gamma - (\tauHyp(s)(E))_{E \in X}\|_1, \|\gamma - (\tauHyp(t)(E))_{E \in X}\|_1\} = \frac{1}{2} \|(\tauHyp(s)(E))_{E \in X} - (\tauHyp(t)(E))_{E \in X}\|_1$.
\end{proof}

%\propositionAdjustTransitionFunctionThreeStates*
% \begin{proof}
%	Let $s, t, u$ with the same label and $\gamma = (\tau'(u)(E))_{E \in  X_u}$. By running \cref{alg:adjust-transition-probability} with $\tau'(s)$ (resp. $\tau'(t)$), $X_u$ and $\gamma$, we compute a probability distribution $\nu_s$ (resp. $\nu_t$).  We have that $|\nu_s - \tau'(s)| = |\gamma - (\tau'(s)(E))_{E \in  X_u}|$ and $ |\nu_t - \tau'(t)| = |\gamma - (\tau'(t)(E))_{E \in  X_u}|$. Define the probability transition function $\tau_2$ as
%	\[
%	\tau_2(x) = \left \{
%	\begin{array}{ll}
%	\tau'(x)& \mbox{if $x \not\in \{s, t\}$}\\
%	\nu_x & \mbox{otherwise}
%	\end{array}
%	\right .
%	\]
%	
%	Let $\M_2 = <S, L, \tau_2, \ell>$. We have $X_u$ is a probabilistic bisimulation in $\M_2$ and $s \sim_{\M_2} t \sim_{\M_2} u$. It follows that $\tau_2 \in {\rm T}_{s, t}^{X_u}$. It remains to show that $\tau_2= \arg\min_{\tau \in {\rm T}_{s, t}^{X_u}} \max\{ |\tau(s) - \tau'(s)| , |\tau(t) - \tau'(t)|\}$. Let $\tau \in {\rm T}_{s, t}^{X_u}$. Then,
%	\begin{eqnarray*} 
%		&&\max\big\{|\tau(s) -\tau'(s)|, |\tau(t) -\tau'(t)|\big\} \\
%		&=& \max\big\{ \sum_{v \in S} |\tau(s)(v) -\tau'(s)(v)|, \sum_{v \in S} |\tau(t)(v)-\tau'(t)(v)|\big\}\\
%		&=&\max\big\{\sum_{E \in X_u } \sum_{v \in E} |\tau(s)(v) -\tau' (s)(v)|, \\
%		&& \phantom{\max;;}\sum_{E \in X_u } \sum_{v \in E} |\tau(t)(v)-\tau'(t)(v)|\big\}\\
%		&\ge& \max\big\{|(\tau(s)(E))_{E \in X_u} -(\tau'(s)(E))_{E \in X_u}| ,  \\
%		&& \phantom{\max;;} |(\tau(t)(E))_{E \in X_u} -(\tau'(t)(E))_{E \in X_u}|\big\}\commenteq{triangle inequality}.\\
%		&=& \max\big\{|\gamma -(\tau'(s)(E))_{E \in X_u}| ,  |\gamma-(\tau'(t)(E))_{E \in X_u}|\big\}\\
%		&& \phantom{\max;;} \commenteq{$s \sim t \sim u$ in $<S, L, \tau, \ell>$}\\
%		&=& \max\big\{|(\tau_2(s)(E))_{E \in X_u} -(\tau'(s)(E))_{E \in X_u}|, \\
%		&& \phantom{\max;;} |(\tau_2(t)(E))_{E \in X_u} -(\tau'(t)(E))_{E \in X_u}|\big\}\\
%		&=&\max\big\{|\tau_2(s) -\tau'(s)|, |\tau_2(t) -\tau'(t)|\big\} 
%	\end{eqnarray*}
%\end{proof}



%TODO redo this lemma
The following lemma holds for the LMCs $Q_i$ in \cref{alg:local-distance-merge-algorithm}.
\begin{restatable}{lemma}{lemmaSmallGlobalDistanceLoopInvariant}\label{lemma:small-global-distance-loop}
	For all $i \in \nat$, we have $Q_{i+1}$ is an $\epsilon_2$-quotient of $Q_{i}$.
	%$d_{\glob}^{\Hyp^{i} \oplus \Hyp^{i+1}} (x_i, x_{i+1}) \le \epsilon_2$ where $x_i$ is a state from $\Hyp^{i}$ and $x_{i+1}$ is the corresponding state in $\Hyp^{i+1}$.
\end{restatable}
%\lemmaSmallGlobalDistanceLoopInvariant*
\begin{proof}
	%Let $x_i$ be a state from $\Q_{i}$ and $x_{i+1}$ be the corresponding state from $\Q_{i+1}$. 
	Let $i \in \nat$. Assume the algorithm steps into the $i$'th iteration and selects state $s$ and $t$ from $\Q_i$ which have the least local bisimilarity distance. 
	
	%Let $x$ be the state in $\M^{i+1}$ such that $x_{i+1}= [x]_{\sim_{\M^{i+1}}}$. 
	Let $\gamma = \frac{(\tau^{\Q_i}(s)(E))_{E \in X_i} + (\tau^{\Q_i}(t)(E))_{E \in X_i}}{2}$. By running \cref{alg:adjust-transition-probability} with $X$, $\tau^{\Q_i}(s)$ (resp. $\tau^{\Q_i}(t)$) and $\gamma$, we compute a probability distribution $\nu_s$ (resp. $\nu_t$). By \cref{lemma:adjust-probability-distribution-exists}, we have that $(\nu_s(E))_{E \in X_i} = (\nu_t(E))_{E \in X_i}  = \gamma$ and $\|\nu_s - \tau^{\Q_i}(s)\|_1 = \|\gamma - (\tau^{\Q_i}(s)(E))_{E \in  X_i}\|_1 = \|\gamma - (\tau^{\Q_i}(t)(E))_{E \in  X_i}\|_1 = \|\nu_t - \tau^{\Q_i}(t)\|_1$. Define the probability transition function $\tau' \in {\rm T}$ as 	
	\[
	\tau'(x) = \left \{
	\begin{array}{ll}
	\tau^{\Q_i}(x)& \mbox{if $x \not\in \{s, t\}$}\\
	\nu_x & \mbox{otherwise.}
	\end{array}
	\right .
	\]
	
	It follows from \cref{lemma:local-bisimilarity-distances-witness-transition-funciton} that $d_{\local}^{\Q_{i}} (s, t) = \max\{\|\tau^{\Q_{i}}(s) - \tau'(s) \|_1, \|\tau^{\Q_{i}}(t) - \tau'(t)\|_1\}$. By the definition of $\tau'$, for all $u$ from $\Q_{i}$ we have \begin{equation}\label{eqn:bound-L1-distance-for-all-states}
	\|\tau^{\Q_{i}}(u) - \tau'(u)\|_1 \le \epsilon_2
	\end{equation} 
	since $\|\tau^{\Q_{i}}(u) - \tau'(u)\|_1 \le \max\{\|\tau^{\Q_{i}}(s) - \tau'(s) \|_1, \|\tau^{\Q_{i}}(t) - \tau'(t)\|_1\} = d_{\local}^{\Q_{i}} (s, t) \le \epsilon_2$.

	It is not hard to prove that the LMC $\Hyp' = <S^{\Q_i}, L, \tau', \ell^{\Q_{i}}>$ is probabilistic bisimilar to the LMC $\M_{i+1}'$. Then, since $\Q_{i+1}$ is probabilistic bisimilar to $\M_{i+1}'$, we have $\Q_{i+1}$ is probabilistic bisimilar to $\Hyp'$. Thus, by \eqref{eqn:bound-L1-distance-for-all-states} and that $Q_{i+1}$ is a quotient, we have $Q_{i+1}$ is an $\epsilon_2$-quotient of $Q_i$.

\end{proof}

%\theoremBoundingGlobalDistance*
%TODO redo the proof of this theorem
%\begin{proof}
%	Let $i \in \nat$. Let $x$ be a state from $\M$ and $x'$ be the corresponding state from $\Hyp^{i}$. We have
%	
%	\begin{eqnarray*}
%		&&d_{\glob}^{\M \oplus \Hyp^{i}} (x, x') \\
%		&\le& d_{\glob}^{\M \oplus \Hyp}(x, x_{h}) + d_{\glob}^{\Hyp \oplus \Hyp^{0}}(x_{h}, x_0) + d_{\glob}^{\Hyp^{0}\oplus \Hyp^{i}}(x_0, x') \\
%		&&\commenteq{$d_{\glob}$ is a pseudometric}\\
%		&& \commenteq{$x_{h}$ in $\Hyp$ and $x_0$ in $\Hyp^{0}$ correspond to $x$ in $\M$, respectively}\\
%		&\le& \epsilon+ d_{\glob}^{\Hyp \oplus \Hyp^{0}}(x_{h}, x_0) + d_{\glob}^{\Hyp^{0}\oplus \Hyp^{i}}(x_0, x') \\%\commenteq{}
%		%&& \commenteq{with probability $\ge (1-\delta)^{|S|}$}\\
%		&=& \epsilon+ 0 + d_{\glob}^{\Hyp^{0}\oplus \Hyp^{i}}(x_0, x') \commenteq{$x_h \sim x_0$ in the LMC $\Hyp \oplus \Hyp^{0}$}\\
%		&\le& \epsilon+i \epsilon_2 \commenteq{Lemma~\ref{lemma:small-global-distance-loop}}\qedhere
%	\end{eqnarray*}
%\end{proof}

\section{Proofs of \cref{subsection:approximate-partition-refinement}}\label{appendix: approximate-partition-refinement}

%Let $x \in S$ and $E_x \in  X$ be the set such that $x \in E_x$. Let $\gamma_{E_x} = \sum_{u \in  E_x} \frac{(\tauHyp(u)(E))_{E \in X}}{|E_x|}$ be a probability distribution over $X$. For all $x \in S$, we run \cref{alg:adjust-transition-probability} with $X, \tauHyp(x)$ and $\gamma_{E_x}$ to obtain $\nu_x$, a probability distribution over $S$.
%
%\begin{restatable}{proposition}{propositionTransitionFunctionSmallDifference}\label{proposition:transition-function-small-difference}
%Let the transition function $\tauHyp': S \to \Dist(S)$ be $\tauHyp'(x) = \nu_x$ for all $x \in S$. We have $|\tauHyp'(x) - \tauHyp(x)| \le \epsilon_2$ for all $x \in S$.  Furthermore, in the LMC $\Hyp' = <S, L, \tauHyp', \ell>$, we have $s \sim_{\Hyp'} t$ for all $s, t$ from the same set of $X$.  
%\end{restatable} 
%
%\begin{proof}
%		Let $x \in S$ and $E_x \in  X$ be the set such that $x \in E_x$. Let $\gamma_{E_x} = \sum_{u \in  E_x} \frac{(\tauHyp(u)(E))_{E \in X}}{|E_x|}$ be a probability distribution over $X$. By \cref{lemma:adjust-probability-distribution-exists}, we have $(\nu_x(E))_{E \in X} = \gamma_{E_x}$ and $|\nu_x - \tauHyp(x)| = \sum_{E \in X} |\nu_x(E) - \tauHyp(x)(E)|$. Define the transition function $\tauHyp'$ as $\tauHyp'(x) = \nu_x$ for all $x \in S$. Furthermore, we have  
%	\begin{eqnarray*}
%		&& \sum_{E \in X} |\nu_x(E) - \tauHyp(x)(E)| \\
%		& = & \sum_{E \in X} |\gamma_{E_x}(E) - \tauHyp(x)(E)|\\
%		& = & \sum_{E \in X} |\big(\sum_{u \in  E_x} \frac{\tauHyp(u)(E)}{|E_x|}\big) - \tauHyp(x)(E)|\\
%		& = & \sum_{E \in X}|\sum_{u \in  E_x} \frac{\tauHyp(u)(E) - \tauHyp(x)(E)}{|E_x|} | \\
%		& \le & 	\sum_{E \in X}| \sum_{u \in  E_x} \frac{|\tauHyp(u)(E) - \tauHyp(x)(E)|}{|E_x|} | \commenteq{triangle inequality}\\
%		& = & 	\sum_{u \in  E_x}  \sum_{E \in X} \frac{|\tauHyp(u)(E) - \tauHyp(x)(E)|}{|E_x|} \\
%		& = & 	\sum_{u \in  E_x}  \frac{\sum_{E \in X} |\tauHyp(u)(E) - \tauHyp(x)(E)|}{|E_x|} \\
%		&\le& \sum_{u \in  E_x} \frac{\epsilon_2}{|E_x|}  \commenteq{$\forall u \in E_x: \sum_{E \in X}|\tauHyp(u)(E) - \tauHyp(x)(E)| \le \epsilon_2 $}\\
%		& =& \epsilon_2
%	\end{eqnarray*}
%	
%	%$|\tau''(u) - \tau'(u)| \le \epsilon'$. For each $E' \in X$, let $\tau''(u)(E') =  \mu_{u}(E')$.
%	
%	%	\begin{eqnarray*}
%	%	&& |\tau''(u) - \tau'(u)| \\
%	%	&=&  \sum_{v \in S} |\tau''(u)(v) - \tau'(u)(v)| \\
%	%	&=&  \sum_{E' \in X}\sum_{v \in E'} |\tau''(u)(v) - \tau'(u)(v)| \\
%	%	&\ge&  \sum_{E' \in X}|\tau''(u)(E') - \tau'(u)(E')| \commenteq{triangle inequality}\\	
%	%	&=& \sum_{E' \in X}|\mu_{u}(E') - \tau'(u)(E')| \\
%	%	&=& \sum_{E' \in X}|\sum_{x \in  E_u} \frac{\tau'(x)(E')}{|E_u|}- \tau'(u)(E')| \\	
%	%	&=& \sum_{E' \in X}|\sum_{x \in  E_u} \frac{\tau'(x)(E') - \tau'(u)(E')}{|E_u|} | \\	
%	%	\end{eqnarray*}
%	
%	%Define the transition function $\tau'$ such that $\tau'(x) = \nu_x$ for all $x \in S$. 
%	Let $E \in X$ and $s, t$ be two arbitrary states from $E$. We have $\ell(s) = \ell(t)$ and $\tauHyp'(s)(E') = \gamma_{E}(E') = \gamma_{E}(E') = \tauHyp'(t)(E')$ for any $E' \in X$, that is, $s \sim t$ in the LMC $\Hyp' = <S, L, \tauHyp', \ell>$.
%	%$|(\tau''(s)(E))_{E \in X} - (\tau'(t)(E))_{E \in X}| \le \epsilon'$.
%\end{proof} 

%TODO redo this lemma
The following lemma holds for the LMCs $Q_i$ in \cref{alg:approximate-partition-refinement-merge-algorithm}.
\begin{lemma}\label{lemma:small-global-distance-loop-approximate-partition-refinement}
	For all $i \in \nat$, we have $Q_{i+1}$ is an $\epsilon_2$-quotient of $Q_{i}$.
	%For all $i \in \nat$, $d_{\glob}^{\Hyp^{i} \oplus \Hyp^{i+1}} (x_i, x_{i+1}) \le \epsilon_2$ where $x_i$ is a state from $\Hyp^{i}$ and $x_{i+1}$ is the corresponding state in $\Hyp^{i+1}$.
\end{lemma}

\begin{proof}
	Let $i \in \nat$. Let $x \in S^{\Q_i}$ and $E_x \in  X_i$ be the set such that $x \in E_x$. Let $\gamma_{E_x} = \sum_{u \in  E_x} \frac{(\tau^{\Q_i}(u)(E))_{E \in X}}{|E_x|}$ be a probability distribution over $X_i$. We run \cref{alg:adjust-transition-probability} with $X, \tau^{\Q_i}(x), \gamma_{E_x}$ to obtain $\nu_x$, a probability distribution over $S^{\Q_i}$. By \cref{lemma:adjust-probability-distribution-exists}, we have $(\nu_x(E))_{E \in X_i} = \gamma_{E_x}$ and $\|\nu_x - \tau^{\Q_i}(x)\|_1 = \sum_{E \in X_i} |\nu_x(E) - \tau^{\Q_i}(x)(E)|$. Define the transition function $\tau'$ as $\tau'(x) = \nu_x$ for all $x \in S^{\Q_i}$.
	
	We have  
	\begin{equation}\label{eqn:bound-transition-function-2}
	\|\tau'(x) - \tau^{\Q_i}(x)\|_1 \le \epsilon_2
	\end{equation}
	since 
	\begin{eqnarray*}
		&& |\tau'(x) - \tau^{\Q_i}(x)| \\
		&=& \sum_{E \in X_i}  |\nu_x(E) - \tau^{\Q_i}(x)(E)| \\
%		& = &\sum_{E \in X_i} |\tau'(x)(E) - \tau^{\Q_i}(x)(E)| \\
%		& = & \sum_{E \in X_i} |\nu_x(E) - \tau^{\Q_i}(x)(E)| \\
		& = & \sum_{E \in X_i} |\gamma_{E_x}(E) - \tau^{\Q_i}(x)(E)|\\
		& = & \sum_{E \in X_i} |\big(\sum_{u \in  E_x} \frac{\tau^{\Q_i}(u)(E)}{|E_x|}\big) - \tau^{\Q_i}(x)(E)|\\
		& = & \sum_{E \in X_i}|\sum_{u \in  E_x} \frac{\tau^{\Q_i}(u)(E) - \tau^{\Q_i}(x)(E)}{|E_x|} | \\
		& \le & 	\sum_{E \in X_i}| \sum_{u \in  E_x} \frac{|\tau^{\Q_i}(u)(E) - \tau^{\Q_i}(x)(E)|}{|E_x|} | \commenteq{triangle inequality}\\
		& = & 	\sum_{u \in  E_x}  \sum_{E \in X_i} \frac{|\tau^{\Q_i}(u)(E) - \tau^{\Q_i}(x)(E)|}{|E_x|} \\
		& = & 	\sum_{u \in  E_x}  \frac{\sum_{E \in X_i} |\tau^{\Q_i}(u)(E) - \tau^{\Q_i}(x)(E)|}{|E_x|} \\
		&\le& \sum_{u \in  E_x} \frac{\epsilon_2}{|E_x|}  \commenteq{$\forall u \in E_x: \sum_{E \in X_i}|\tau^{\Q_i}(u)(E) - \tau^{\Q_i}(x)(E)| \le \epsilon_2 $}\\
		& =& \epsilon_2.
	\end{eqnarray*}

	Let us consider the LMC $\Q_i' = <S^{Q_i}, L, \tau', \ell^{\Q_{i}}>$. Since for all $s, t$ in the same set $E$ of $X_i$ we have $\ell^{\Q_{i}}(s) = \ell^{\Q_{i}}(t)$ and $\tau'(s) = \tau'(t)$, states in the same set of $E$ of $X_i$ are probabilistic bisimilar in the LMC $\Q_i'$. It is not hard to prove that $\Q_i'$ is probabilistic bisimilar to $\M_{i+1}$. Since $\Q_{i+1}$ is probabilistic bisimilar to $\M_{i+1}$, $\Q_{i+1}$ is probabilistic bisimilar to $\Q_{i}'$. By \eqref{eqn:bound-transition-function-2} and that $\Q_{i+1}$ is a quotient LMC, we have $\Q_{i+1}$ is an $\epsilon_2$-quotient of $\Q_i$.
\end{proof}

%TODO redo this theorem
\theoremBoundGlobalDistanceApproximatePartitionRefinement*
\begin{proof}
	Let $i \in \nat$. The first part of the statement follows from \cref{lemma:small-global-distance-loop} for $\Q_i$ in \cref{alg:local-distance-merge-algorithm}. The first part of the statement follows from \cref{lemma:small-global-distance-loop-approximate-partition-refinement} for $\Q_i$ in \cref{alg:approximate-partition-refinement-merge-algorithm}. %Let $x$ be a state from $\Hyp^{0} $ and $x_i$ be the corresponding state from $\Hyp^{i}$. 
	
	We prove the second part of the statement by induction. The base case $i = 0$ follows from the fact that $\Q_0 = \Hyp /_{\sim_{\Hyp}}$. For the induction step, we assume $\Q_{i}$ is an $i\epsilon_2$-quotient of $\Hyp$. Then, from the first part that $\Q_{i+1}$ is an $\epsilon_2$-quotient of $\Q_{i}$ and the additivity lemma, we have $\Q_{i+1}$ is an $(i+1)\epsilon_2$-quotient of $\Hyp$.
\end{proof}

\corollaryBoundQuotientError*
\begin{proof}
	It is not hard to see that $\Q_0$ is an $\epsilon$-quotient of $\M$. It then follows from \cref{theorem:bounding-global-distance} and \cref{lemma:additivity-property} that $\Q_i$ is an $(\epsilon+i\epsilon_2)$-quotient of $\M$.
\end{proof}

%\section{Approximation LMC by Sampling} \label{appendix:sampling}
%Assume that we want to learn the transition probabilities of an LMC $\M$, that is, the state space, the labelling and the transitions are known. We also assume the system under learning (SUL) could answer the query $\nxt$ which takes a state $s$ as input and returns a successor state of $s$ according to the transition probability distribution $\tau(s)$.
%
%%Sampling
%We can approximate the transition probabilities via sampling. Given a state $s$ of the LMC. We denote by $x_s$ the number of successor states of $s$ and $n_s$ the number of times we query the SUL on $\nxt(s)$. Let $N_{s,t}$ be the frequency counts of the query result $t$, that is, the number of times a successor state $t$ appears as the result returned by the queries. We approximate the transition probability distribution $\tau(s)$ by $\tau'(s)$ where $\tau'(s)(t) = \frac{N_{s, t}}{n_s}$ for all successor states $t$ of $s$. (Such an estimator is called an empirical estimator in the literature.)
%
%Intuitively, the more queries we ask the SUL, the more accurate the approximate probability distribution $\tau'(s)$ would be. In fact, the following theorem holds \cite[Section~6.4]{BazilleGenestJegourelSun2020}, \cite{Chen2015}.
%
%\begin{theorem}\label{theorem:sampling-size} Let $\epsilon \gr 0$ be an error parameter and $\delta \gr 0$ be an error bound. We have $$\Pr(|\tau(s) - \tau'(s)| \gr \epsilon) \ls \delta$$ for $n_s \ge \frac{1}{2 \epsilon^{2}}\ln(\frac{2x_s}{\delta})$.
%\end{theorem}
%
%Equivalently, we have $$\Pr(|\tau(s) - \tau'(s)| \le \epsilon) \ge (1-\delta)$$ for $n_s \ge \frac{1}{2 \epsilon^{2}}\ln(\frac{2x_s}{\delta})$.
%
%For every state $s \in S$, we query the SUL on $\nxt(s)$ for $n_s \ge \frac{1}{2 \epsilon^{2}}\ln(\frac{2x_s}{\delta})$ times. We then approximate the transition function $\tau$ by $\tau'$ and construct a hypothesis LMC $\Hyp = <S, L, \tau', \ell>$. Since the queries $\nxt(s)$ and $\nxt(t)$ for all $s,t \in S$ and $s \not= t$ are mutually independent, by Theorem~\ref{theorem:sampling-size} and the definition of the global bisimilarity distance, we have that $\Pr(d_{\glob}^{\M \oplus \Hyp}(s, s') \le \epsilon) \ge (1-\delta)^{|S|} $ where $s$ is in the state space of the SUL $\M$ and $s'$ is the corresponding state in $\Hyp$.
%
%%Assume we have constructed a hypothesis LMC  $\Hyp = <S, L, \tau', \ell>$. 
%
%After a hypothesis LMC  $\Hyp$ has been constructed, suppose we want to minimise $\Hyp$ such that with high probability the difference between the new LMC after minimisation and the SUL is small. We use the global bisimilarity distance to measure the difference between the new LMC and the SUL. It suffices to compute and merge states with small global bisimilarity distances. As it is $\sf NP$-complete to compute the global bisimilarity distances by \cref{theorem: global-epsilon-bisimulation-NP-complete}, it is not straightforward. Instead, we propose two polynomial-time algorithms to minimise the hypothesis LMC. 

%\section{Proofs of \cref{section:active-LMC-learning}}
%%TODO redo this corollary
%\begin{restatable}{corollary}{corollaryBoundGlobalDistance}\label{corollary:bounding-global-distance}
%	For all $i \in \nat$, we have $\Pr(d_{\glob}^{\M \oplus \Hyp^{i}} (x, x_i) \le \epsilon + i \epsilon_2) \ge (1-\delta)^{|S|}$ where $x$ is a state from the SUL $\M$ and $x_i$ is the corresponding state in $\Hyp^{i}$.
%\end{restatable}
%%\corollaryBoundGlobalDistance*
%\begin{proof}
%	Let $i \in \nat$. Let $x$ be a state from $\M$ and $x_i$ be the corresponding state from $\Hyp^{i}$. Let  $x_h$ of $\Hyp$ and $x_0$ of $\Hyp^{0}$ correspond to $x$ of $\M$, respectively. 
%	
%	Since $\Hyp\sim \Hyp^{0}$, we have $d_{\glob}^{\Hyp \oplus \Hyp^{0} }(x_h, x_{0}) =0$. Since the LMC $\Hyp^{0}$ is a quotient, by \cref{lemma:additivity-property}, we have $ d_{\glob}^{\M \oplus \Hyp }(x, x_{h})  = d_{\glob}^{\M \oplus \Hyp^{0} }(x, x_{0}) + d_{\glob}^{  \Hyp^{0} \oplus \Hyp}(x_0, x_{h})   = d_{\glob}^{\M \oplus \Hyp^{0} }(x, x_{0}) $. Then, by $\Pr(d_{\glob}^{\M \oplus \Hyp} (x, x_h) \le \epsilon ) \ge (1-\delta)^{|S|}$, we have $\Pr(d_{\glob}^{\M \oplus \Hyp^{0}} (x, x_0) \le \epsilon ) \ge (1-\delta)^{|S|}$.
%	
%	Finally, since the LMC $\Hyp^{0}$ is a quotient, it follows from \cref{theorem:bounding-global-distance} and \cref{lemma:additivity-property} that $\Pr(d_{\glob}^{\M \oplus \Hyp^{i}} (x, x_i) \le \epsilon + i \epsilon_2) \ge (1-\delta)^{|S|}$.
%\end{proof}

%\section{Example in \cref{section:experiments}}\label{section:appendix-example}



\section{More Experimental Results}\label{appendix:more-results}

Let us fix the error bound $\delta = 0.01$. In the tables, local and apr stand for the minimisation algorithms using local bisimilarity distance and approximate partition refinement, respectively. The rows are highlighted in yellow when the structure of the quotient of the original model is successfully recovered. While the rows are highlighted in red when $\epsilon_2$ is too big, that is, \modify{the minimisation algorithms aggressively merge some states in the perturbed LMCs, resulting in quotients of which the size are smaller than that of the quotients of the unperturbed LMCs}. %(and even the original)

%herman small


%\begin{table}[h]
\begin{figure}[h!]
\begin{floatrow}
	%\begin{tabularx}{\textwidth}{@{}X@{}X@{}}
	\capbtabbox[.49\textwidth]{
		\noindent\begin{tabular}{|c|c|c|c|}
			\hline %\\$\epsilon = 0.001$
			\multicolumn{4}{|c|}{Herman3}\\
			\hline
			\multirow{1}{*}{\shortstack[l]{$\epsilon = 0.001$}}& 
			\# states&	\# trans&	\# iter\\
			\hline 						
			$\M$ \& $\Hyp$	&		 		8	  &			28				&\\
			$\M/_{\sim_{\M}}$	&		 		2	  &			3				&\\		
			$\Hyp/_{\sim_{\Hyp}}$      & 		3  &	       7             & \\
			\hline
			\multicolumn{4}{|c|}{Perturbed LMC \#1 - \#5}\\ 
			\hline
			\multicolumn{4}{|c|}{$\epsilon_2 \in\{ 0.00001,  0.0001\}$}\\
			\hline
			local      & 3 & 7 & 0  \\
			apr		  & 3 & 7 & 0  \\
			\hline
			\multicolumn{4}{|c|}{$\epsilon_2 \in \{0.001, 0.01, 0.1\}$}\\
			\hline
			\rowcolor{yellow}
			local     & 2 & 3 & 1  \\
			\rowcolor{yellow}	
			apr		  & 2 & 3 & 1  \\
			\hline	%padding
			\rowcolor{white}
			\multicolumn{4}{c}{}\\
			\multicolumn{4}{c}{}\\
			\multicolumn{4}{c}{}\\
			\multicolumn{4}{c}{}\\
			\multicolumn{4}{c}{}\\
		\end{tabular}
	}{}
	%	~
	\capbtabbox[.49\textwidth]{
		\begin{tabular}{|c|c|c|c|}
		\hline %\\$\epsilon = 0.001$
		\multicolumn{4}{|c|}{Herman7}\\
		\hline
		\multirow{1}{*}{\shortstack[l]{$\epsilon = 0.001$}}& 
		\# states&	\# trans&	\# iter\\
		\hline 						
		$\M$ \& $\Hyp$	&		 		128	  &			2188				&\\
		$\M/_{\sim_{\M}}$	&		 		9	  &			49				&\\		
		$\Hyp/_{\sim_{\Hyp}}$      & 		115  &	       1925            & \\
		\hline
		\multicolumn{4}{|c|}{Perturbed LMC \#2}\\ 
		\hline
		\multicolumn{4}{|c|}{$\epsilon_2 \in\{ 0.00001,  0.0001\}$}\\
		\hline
		local \& apr      & 115 & 1925 & 0  \\
		%apr		  & 115 & 1925 & 0  \\
		\hline
		\multicolumn{4}{|c|}{$\epsilon_2 = 0.001$}\\
		\hline
		local      & 114 & 1809 & 1  \\
		apr		  & 115 & 1925 & 0  \\
		\hline
		\multicolumn{4}{|c|}{$\epsilon_2 = 0.01$}\\
		\hline
		local      & 114 & 1809 & 1  \\
		\rowcolor{yellow}
		apr		  & 9 & 49 & 1  \\
		\rowcolor{white}
		\hline		
		\multicolumn{4}{|c|}{$\epsilon_2 = 0.1$}\\
		\hline
		%\rowcolor{red!40}
		local      & 114 & 1809 & 1  \\
		\rowcolor{red!40}
		apr		  & 10 & 60 & 1  \\
		\hline
	\end{tabular}	
	%\end{tabularx}
	}{}
	%\caption{} \label{table:herman}%\belowcaptionskip
%\end{table}
\end{floatrow}
\end{figure}
The next tables show the results of running the two minimisation algorithms on the LMCs that model Herman's self-stabilisation algorithm with $3$ processes and $7$ processes, respectively. All the perturbed LMCs are obtained by sampling with $\epsilon = 0.001$. For the model with $7$ processes, only the minimisation algorithm using approximate partition refinement with $\epsilon_2 = 0.01$ can recover the structure of the quotient of the original model. For this model,  though the minimisation algorithm with $\epsilon_2 = 0.1$ does not perfectly recover the structure of the quotient of the original LMC, the final minimised LMC is quite close. 

%herman13
%herman big

\begin{figure}[h!]	
	\begin{floatrow}
%\begin{table}[h!]
	%\begin{tabularx}{\textwidth}{@{}X@{}X@{}}
		\capbtabbox[.49\textwidth]{
			\noindent\begin{tabular}{|c|c|c|c|}
				\hline %\\$\epsilon = 0.001$
				\multicolumn{4}{|c|}{Herman13}\\
				\hline
				\multirow{1}{*}{\shortstack[l]{$\epsilon = 0.001$}}& 
				\# states&	\# trans&	\# iter\\
				\hline 						
				$\M$ \& $\Hyp$	&		 8192	& 1594324			&\\
				$\M/_{\sim_{\M}}$	&		 		190	& 12857			&\\		
				$\Hyp/_{\sim_{\Hyp}}$      & 		8167 &	1585929            & \\
				\hline
				\multicolumn{4}{|c|}{Perturbed LMC \#2}\\ 
				\hline
				\multicolumn{4}{|c|}{$\epsilon_2 \in\{ 0.00001, 0.0001\}$}\\
				\hline
				apr      & 8167	&1585929&	0 \\
				\hline
				\multicolumn{4}{|c|}{$\epsilon_2 = 0.001$}\\
				\hline
				apr		  &8166	&1577761&	1 \\
				\hline		
				\multicolumn{4}{|c|}{$\epsilon_2 = 0.01$}\\
				\hline
				apr		  &192	&13250	&1 \\
				\hline							
				\multicolumn{4}{|c|}{$\epsilon_2 = 0.1$}\\
				\hline
				\rowcolor{red!40}	
				apr		   & 7608 &	1439629 &	1  \\
				\hline%padding
							\rowcolor{white}	
				\multicolumn{4}{c}{}\\
				\multicolumn{4}{c}{}\\
			\end{tabular}
		}{}
		\capbtabbox[.49\textwidth]{
			\begin{tabular}{|c|c|c|c|}
				\hline
				\multicolumn{4}{|c|}{Herman15}\\
				\hline
				\multirow{1}{*}{\shortstack[l]{$\epsilon = 0.0001$}}& 
				\# states&	\# trans&	\# iter\\
				\hline 
				$\M$ \& $\Hyp$ &	32768 &	14348908    & \\
				$\M/_{\sim_{\M}}$	&		 		612	& 104721	  		&\\								
				$\Hyp/_{\sim_{\Hyp}}$     & 	32739 &	14323591    & \\
				\hline
				\multicolumn{4}{|c|}{Perturbed LMC \#3}\\ 
				\hline
				\multicolumn{4}{|c|}{$\epsilon_2 = 0.00001$}\\
				\hline
				apr		&  32739 &	14323591 &	0\\
				\hline
				\multicolumn{4}{|c|}{$\epsilon_2 = 0.0001$}\\
				\hline
				apr		&  32738 &	14290851&	1\\
				\hline			
				\multicolumn{4}{|c|}{$\epsilon_2 =  0.001$}\\
				\hline
				apr		 &  3489 &	1918364	& 1 \\
				\hline		
				\multicolumn{4}{|c|}{$\epsilon_2 =  0.01$}\\
				\hline
				\rowcolor{yellow}
				apr		 &  612	& 104721 &	1\\
				\rowcolor{white}
				\hline					
				\multicolumn{4}{|c|}{$\epsilon_2 = 0.1$}\\
				\hline
				\rowcolor{red!40}
				apr		 &  25893	& 11090774	& 1\\
				\hline
			\end{tabular}	
		}{}
	\end{floatrow}
%\caption{Results for Herman's self-stabilisation algorithm with $13$ and $15$ processes.}
%\label{table:herman13-15}
\end{figure}
%\end{tabularx}

%\end{table}

The table above (left) shows the results of running the minimisation algorithm using approximate partition refinement on the LMC that models Herman's self-stabilisation algorithm with $13$ processes. The perturbed LMCs are obtained by perturbing the probabilities with $\epsilon = 0.001$. For this model, the minimisation algorithm could not recover the structure of the quotient of the original model, but the minimised LMC obtained with $\epsilon_2 = 0.01$ is quite close. The table above (right) shows the results of running the minimisation algorithms using approximate partition refinement on the LMC that models Herman's self-stabilisation algorithm with $15$ processes. The perturbed LMCs are obtained by perturbing the probabilities with $\epsilon = 0.0001$. For this model, the minimisation algorithm successfully recovers the quotient of the original model. However, when $\epsilon_2 = 0.1$, the value is too big that the minimisation algorithm\modify{aggressively merges states in the perturbed model and results in a quotient of which the size is even smaller than that of the quotient of the unperturbed LMC.}



%leader
For the LMCs that model the synchronous leader election protocol by Itai and Rodeh, the exact partition refinement can always recover the structure of the quotient of the original model. Let $\epsilon \in \{0.00001 ,0.0001, 0.001, 0.01, 0.1\}$. The tables on the top of the page show the results of running the two minimisation algorithms on the LMCs that model the synchronous leader election protocol by Itai and Rodeh with $N =5, K=5$ and $N =6, K=4$ on the left and right, respectively. All the perturbed LMCs are obtained by perturbing the probabilities. 

\begin{figure}[t]	
	\begin{floatrow}
%\begin{table}[t]
%\begin{tabularx}{\textwidth}{@{}X@{}X@{}}
	\capbtabbox[.49\textwidth]{
	\noindent\begin{tabular}{|c|c|c|c|}
	\hline
	Leader5-5&	\# states&	\# trans&	\# iter\\
	\hline \hline						
	$\M$ \& $\Hyp$		& 	12709   &			15833	    &\\
	$\M /_{\sim_{\M}}$  &			12  &	       13             & \\
	\rowcolor{yellow}
	$\Hyp /_{\sim_{\Hyp}}$  &			12  &	       13             & \\
	\rowcolor{white}
	\hline
	\multicolumn{4}{|c|}{Perturbed LMC \#1-\#5} \\
	\hline
	\multicolumn{4}{|c|}{$\epsilon_2 \in \{0.00001, 0.0001, 0.001, 0.01, 0.1\}$}\\
	\hline
	\rowcolor{yellow}
	 apr \& local	 & 12 & 13 & 0  \\
	\hline
	\end{tabular}
	}{}
\capbtabbox[.49\textwidth]{
	\begin{tabular}{|c|c|c|c|}
		\hline
		Leader6-4&	\# states&	\# trans&	\# iter\\
		\hline \hline						
		$\M$ \& $\Hyp$		& 	20884   &			24979	    &\\
		$\M /_{\sim_{\M}}$  &			14  &	       15             & \\
		\rowcolor{yellow}
		$\Hyp /_{\sim_{\Hyp}}$  &			14  &	       15             & \\
		\rowcolor{white}
		\hline
		\multicolumn{4}{|c|}{Perturbed LMC \#1-\#5} \\
		\hline
		\multicolumn{4}{|c|}{$\epsilon_2 \in \{0.00001, 0.0001, 0.001, 0.01, 0.1\}$}\\
		\hline
		\rowcolor{yellow}
		apr	\& local & 14 & 15 & 0  \\
		\hline
	\end{tabular}
	}{}
\end{floatrow}
%\caption{Results for synchronous leader election protocol by Itai and Rodeh.}
%\label{table:sync-leader-election}
\end{figure}
%\end{tabularx}

%\end{table}


%brp
\begin{figure}[h!]
	\begin{floatrow}
		%\begin{tabularx}{\textwidth}{@{}X@{}X@{}}
		\capbtabbox[.49\textwidth]{
			\noindent\begin{tabular}{|c|c|c|c|}
				\hline
				\multicolumn{4}{|c|}{BRP16-3}\\
				\hline
				\multirow{1}{*}{\shortstack[l]{$\epsilon = 0.01$}}& 
				\# states&	\# trans&	\# iter\\
				\hline 						
				$\M$ \& $\Hyp$	&		 		886	  &			1155				&\\
				$\M/_{\sim_{\M}}$	&		 		440	  &			616				&\\		
				$\Hyp/_{\sim_{\Hyp}}$      & 		671  &	       940             & \\
				\hline
				\multicolumn{4}{|c|}{Perturbed LMC \#3}\\ 
				\hline
				\multicolumn{4}{|c|}{$\epsilon_2 = 0.00001$}\\
				\hline
				apr      & 668 & 936 & 1  \\
				\hline
				\multicolumn{4}{|c|}{$\epsilon_2 = 0.0001$}\\
				\hline
				apr		  & 651 & 912 & 1  \\
				\hline
				\multicolumn{4}{|c|}{$\epsilon_2 = 0.001$}\\
				\hline
				apr		  & 574 & 806 & 2  \\
				\hline	
				\multicolumn{4}{|c|}{$\epsilon_2 = 0.01$}\\
				\hline
				apr		  & 472 & 663 & 2  \\
				\hline						
				\multicolumn{4}{|c|}{$\epsilon_2 = 0.1$}\\
				\hline
				\rowcolor{red!40}
				apr		   & 100 & 195 & 1   \\
				\hline
			\end{tabular}
		}{}
		\capbtabbox[.49\textwidth]{
			\begin{tabular}{|c|c|c|c|}
				\hline
				%\multirow{1}{*}{\shortstack[l]{BRP64-4%\\$\epsilon = 0.001$	}} & 
				\multicolumn{4}{|c|}{BRP64-4}\\
				\hline
				\multirow{1}{*}{\shortstack[l]{$\epsilon = 0.001$}}& 					
				\# states&	\# trans&	\# iter\\
				\hline 
				$\M$ \& $\Hyp$ &	4359		 &			5763    & \\
				$\M/_{\sim_{\M}}$	&		 		2185	&  3081	  		&\\								
				$\Hyp/_{\sim_{\Hyp}}$     & 		3453	& 4857      & \\
				\hline
				\multicolumn{4}{|c|}{Perturbed LMC \#4}\\ 
				\hline
				\multicolumn{4}{|c|}{$\epsilon_2 = 0.00001$}\\
				\hline
				apr		&  3377 &	4754 &	1\\
				\hline
				\multicolumn{4}{|c|}{$\epsilon_2 = 0.0001$}\\
				\hline
				apr		&  3034 &	4279&	2\\
				\hline			
				\multicolumn{4}{|c|}{$\epsilon_2 =  0.001$}\\
				\hline
				apr		 &  2350 &	3318&	4 \\
				\hline
				\multicolumn{4}{|c|}{$\epsilon_2 =  0.01$}\\
				\hline
				\rowcolor{yellow}
				apr		 &  2185 &	3081 &	1 \\
				\rowcolor{white}
				\hline			
				\multicolumn{4}{|c|}{$\epsilon_2 = 0.1$}\\
				\hline
				\rowcolor{red!40}
				apr		 &  388	&771&	1\\
				\hline
			\end{tabular}	
		}{}
	\end{floatrow}
\end{figure}

The table above (left) shows the results of running the minimisation algorithm using approximate partition refinement on the LMC that models the bounded retransmission protocol with $N =16$ and $\mathit{MAX} = 3$. The perturbed LMCs are obtained by perturbing the probabilities with $\epsilon = 0.01$. The minimisation algorithm fails to recover the structure of the quotient of the original model in this case. This is due to the fact that the only $\epsilon_2$ in this experiment that is greater than $\epsilon$ is $0.1$. This value, however, is too big and the minimisation algorithm aggressively merges states in the perturbed model.  The table above (right) shows the results of running the minimisation algorithm using approximate partition refinement on the LMC that models the bounded retransmission protocol (with $N =64$ and $\mathit{MAX} = 4$). The perturbed LMCs are obtained by perturbing the probabilities with $\epsilon = 0.001$. The minimisation algorithms successfully recover the structure of the quotient of the original model when $\epsilon_2 = 0.01$. When $\epsilon_2 = 0.1$, it is again too big and the minimisation algorithm aggressively merges states in the perturbed model.


%crowds

The table above (left) shows the results of running the minimisation algorithms using approximate partition refinement on the LMC that models the Crowds protocol \cite{ReiterR98} with $\mathit{TotalRuns}=4$ and $\mathit{CrowdSize}=5$. The perturbed LMCs are obtained by perturbing the probabilities with $\epsilon = 0.001$. The table above (left) shows the results of running the minimisation algorithms using approximate partition refinement on the LMC that models the Crowds protocol \cite{ReiterR98} with $\mathit{TotalRuns}=6$ and $\mathit{CrowdSize}=5$. The perturbed LMCs are obtained by perturbing the probabilities with $\epsilon = 0.01$.  For both models, the minimisation algorithms successfully recover the structure of the quotient of the original model when $\epsilon_2 \gr \epsilon$. 

\begin{figure}[t!]
	\begin{floatrow}
		%\begin{tabularx}{\textwidth}{@{}X@{}X@{}}
		\capbtabbox[.49\textwidth]{
		\noindent\begin{tabular}{|c|c|c|c|}
			\hline %\\$\epsilon = 0.0001$
			%\multirow{1}{*}{\shortstack[l]{Crowds4-5}}& 
			\multicolumn{4}{|c|}{Crowds4-5}\\
			\hline
			\multirow{1}{*}{\shortstack[l]{$\epsilon = 0.0001$}}& 			
			\# states&	\# trans&	\# iter\\
			\hline 						
			$\M$ \& $\Hyp$	&		 	3515 &	6035				&\\
			$\M/_{\sim_{\M}}$	&		 		34	  &			42				&\\		
			$\Hyp/_{\sim_{\Hyp}}$      & 		1679 &	4199            & \\
			\hline
			\multicolumn{4}{|c|}{Perturbed LMC \#5}\\ 
			\hline
			\multicolumn{4}{|c|}{$\epsilon_2 = 0.00001$}\\
			\hline
			apr      & 1656	& 4087 &	1  \\
			\hline
			\multicolumn{4}{|c|}{$\epsilon_2 = 0.0001$}\\
			\hline
			apr		  &735	&1320&	4 \\
			\hline					
			\multicolumn{4}{|c|}{$\epsilon_2 = \{0.001, 0.01, 0.1\}$}\\
			\hline
			\rowcolor{yellow}
			apr		   & 34	&42	&1   \\
			\rowcolor{white}
			\hline%padding
			\multicolumn{4}{c}{}\\
			\multicolumn{4}{c}{}\\
		\end{tabular}
	}{}
	\capbtabbox[.49\textwidth]{
		\begin{tabular}{|c|c|c|c|}
			\hline
			%\multirow{1}{*}{\shortstack[l]{Crowds6-5%\\$\epsilon = 0.01$}} & 
			\multicolumn{4}{|c|}{Crowds6-5}\\
			\hline
			\multirow{1}{*}{\shortstack[l]{$\epsilon = 0.001$}}& 			
			\# states&	\# trans&	\# iter\\
			\hline 
			$\M$ \& $\Hyp$ &	18817	& 32677    & \\
			$\M/_{\sim_{\M}}$	&		 		50	&  62	  		&\\								
			$\Hyp/_{\sim_{\Hyp}}$     & 	9237 &	23097      & \\
			\hline
			\multicolumn{4}{|c|}{Perturbed LMC \#2}\\ 
			\hline
			\multicolumn{4}{|c|}{$\epsilon_2 = 0.00001$}\\
			\hline
			apr		&  9235	& 23087 &	1\\
			\hline
			\multicolumn{4}{|c|}{$\epsilon_2 = 0.0001$}\\
			\hline
			apr		&  9086 &	22358	& 2\\
			\hline			
			\multicolumn{4}{|c|}{$\epsilon_2 =  0.001$}\\
			\hline
			apr		 &  5249 &	9746 &	5 \\
			\hline			
			\multicolumn{4}{|c|}{$\epsilon_2 \in \{0.01,  0.1\}$}\\
			\hline
			\rowcolor{yellow}
			apr		 &  50	&62&	1\\
			\hline
		\end{tabular}	
	}{}
\end{floatrow}
\end{figure}

%egl
\begin{figure}[h!]
	\begin{floatrow}
		\capbtabbox[.49\textwidth]{
			\noindent\begin{tabular}{|c|c|c|c|}
				\hline %\\$\epsilon = 0.0001$
				%\multirow{1}{*}{\shortstack[l]{EGL5-2}}& 
				\multicolumn{4}{|c|}{EGL5-2}\\
				\hline
				\multirow{1}{*}{\shortstack[l]{$\epsilon = 0.0001$}}& 				
				\# states&	\# trans&	\# iter\\
				\hline 						
				$\M$ \& $\Hyp$	&		 	33790 &	34813				&\\
				$\M/_{\sim_{\M}}$	&		 		472	 & 507			&\\		
				$\Hyp/_{\sim_{\Hyp}}$      & 		551	& 681            & \\
				\hline
				\multicolumn{4}{|c|}{Perturbed LMC \#3}\\ 
				\hline
				\multicolumn{4}{|c|}{$\epsilon_2 = 0.00001$}\\
				\hline
				apr      & 535	& 647 & 	1  \\
				\hline
				\multicolumn{4}{|c|}{$\epsilon_2 = 0.0001$}\\
				\hline
				apr		  &485 &	536&	2 \\
				\hline					
				\multicolumn{4}{|c|}{$\epsilon_2 = \{0.001, 0.01, 0.1\}$}\\
				\hline
				\rowcolor{yellow}
				apr		   & 472 &	507	& 1   \\
				\rowcolor{white}
				\hline%padding
				\multicolumn{4}{c}{}\\
				\multicolumn{4}{c}{}\\
			\end{tabular}
		}{}
		\capbtabbox[.49\textwidth]{
			\begin{tabular}{|c|c|c|c|}
				\hline
				%\multirow{1}{*}{\shortstack[l]{EGL5-4%\\$\epsilon = 0.01$}} & 
				\multicolumn{4}{|c|}{EGL5-4}\\
				\hline
				\multirow{1}{*}{\shortstack[l]{$\epsilon = 0.001$}}& 						
				\# states&	\# trans&	\# iter\\
				\hline 
				$\M$ \& $\Hyp$ &	74750 &	75773    & \\
				$\M/_{\sim_{\M}}$	&		 		992	& 1027	  		&\\								
				$\Hyp/_{\sim_{\Hyp}}$     & 	1071 &	1201     & \\
				\hline
				\multicolumn{4}{|c|}{Perturbed LMC \#4}\\ 
				\hline
				\multicolumn{4}{|c|}{$\epsilon_2 = 0.00001$}\\
				\hline
				apr		&  1063	& 1184 &	1\\
				\hline
				\multicolumn{4}{|c|}{$\epsilon_2 = 0.0001$}\\
				\hline
				apr		&  1053	& 1162 &	1\\
				\hline			
				\multicolumn{4}{|c|}{$\epsilon_2 =  0.001$}\\
				\hline
				\rowcolor{yellow}
				apr		 &  992	& 1027	& 2 \\
				\rowcolor{white}
				\hline			
				\multicolumn{4}{|c|}{$\epsilon_2 \in \{0.01,  0.1\}$}\\
				\hline
				\rowcolor{yellow}
				apr		 &  992	& 1027 &	1\\
				\hline
			\end{tabular}	
		}{}
	\end{floatrow}
\end{figure}


The table above (left) shows the results of running the minimisation algorithms using approximate partition refinement on the LMC that models the contract signing protocol by Even, Goldreich and Lempel \cite{EvenGL85} with $N=5$ and $L=2$. The perturbed LMCs are obtained by perturbing the probabilities with $\epsilon = 0.0001$. The table above (right) shows the results of running the minimisation algorithm using approximate partition refinement on the LMC that models the contract signing protocol by Even, Goldreich and Lempel \cite{EvenGL85} with $N=5$ and $L=4$. The perturbed LMCs are obtained by perturbing the probabilities with $\epsilon = 0.001$. For both models, the minimisation algorithms successfully recover the quotient of the original model when $\epsilon_2 \gr \epsilon$ ($\epsilon_2 \ge \epsilon$ for the second model). 

