\begin{table}
\centering
\subfloat[Performance of three connection strategies on 27 different settings.]{
\resizebox{\textwidth}{!}{
\begin{tabular}{ c | c c c c | c c c c | c c c c}
    \hline
        & \multicolumn{4}{c|}{CIFAR10} 
        & \multicolumn{4}{c|}{CIFAR100} 
        & \multicolumn{4}{c}{SVHN}\\
                    &  1/4  &  1/2  &  3/4  & 1 
                    &  1/4  &  1/2  &  3/4  & 1 
                    &  1/4  &  1/2  &  3/4  & 1 \\
    \hline
(12,16) m=L
	& 29.30 & 24.27 & 13.63 &  7.23 & 62.48 & 56.85 & 42.97 & 29.14 & 12.99 &  5.66 &  3.80 &  2.03 \\
(12,16) m=N
	& 29.24 & 23.40 & 13.15 &  7.59 & 62.79 & 55.34 & 42.61 & 30.59 & 10.96 &  6.35 &  3.84 &  2.11 \\
(12,16) m=E
	& 29.26 & 21.82 & 13.11 &  7.45 & 61.17 & 53.73 & 41.92 & 30.72 & 12.53 &  5.51 &  3.83 &  2.27 \\
\hline
(12,24) m=L
	& 22.85 & 18.86 & 10.51 &  5.98 & 57.04 & 52.41 & 37.48 & 26.36 &  8.77 &  4.43 &  3.20 &  1.94 \\
(12,24) m=N
	& 25.22 & 19.74 & 10.53 &  6.46 & 56.70 & 51.35 & 37.42 & 26.96 &  8.32 &  4.61 &  3.32 &  2.10 \\
(12,24) m=E
	& 24.51 & 18.45 & 10.66 &  6.56 & 55.93 & 48.65 & 36.88 & 27.80 & 10.30 &  4.64 &  3.24 &  2.05 \\
\hline
(12,32) m=L
	& 20.39 & 13.92 &  9.11 &  5.48 & 53.64 & 48.63 & 34.42 & 24.21 &  7.68 &  4.07 &  2.93 &  1.85 \\
(12,32) m=N
	& 21.68 & 17.26 &  9.48 &  6.00 & 54.05 & 47.32 & 34.29 & 24.70 &  7.10 &  4.40 &  3.12 &  1.92 \\
(12,32) m=E
	& 22.24 & 16.31 &  9.32 &  6.15 & 52.43 & 44.46 & 33.40 & 25.57 &  7.72 &  4.34 &  3.08 &  1.90 \\
\hline
(32,16) m=L
	& 21.40 & 12.23 &  9.36 &  5.96 & 58.92 & 43.82 & 36.66 & 25.32 &  4.78 &  3.61 &  3.42 &  1.97 \\
(32,16) m=N
	& 24.81 & 12.95 &  9.11 &  6.45 & 60.61 & 48.12 & 38.20 & 27.48 &  5.41 &  3.57 &  3.10 &  1.94 \\
(32,16) m=E
	& 19.65 & 12.42 &  9.49 &  6.21 & 51.89 & 42.82 & 38.21 & 26.81 &  5.07 &  3.39 &  2.91 &  1.96 \\
\hline
(32,24) m=L
	& 15.46 &  9.91 &  8.20 &  5.03 & 51.13 & 37.46 & 32.75 & 22.73 &  3.87 &  3.20 &  3.18 &  1.77 \\
(32,24) m=N
	& 21.13 & 12.09 &  7.75 &  5.74 & 54.91 & 43.57 & 33.59 & 25.08 &  4.80 &  3.37 &  2.84 &  1.82 \\
(32,24) m=E
	& 14.80 &  9.76 &  7.75 &  5.43 & 45.42 & 36.92 & 32.74 & 24.80 &   -   &   -   &   -   &   -   \\
\hline
(32,32) m=L
	& 12.86 &  8.58 &  7.10 &  4.81 & 45.99 & 32.87 & 29.46 & 21.77 &  3.68 &  3.00 &  2.85 &  1.76 \\
(32,32) m=N
	& 19.18 & 11.04 &  7.47 &  5.65 & 52.41 & 38.18 & 30.30 & 23.79 &  4.07 &  3.06 &  2.66 &  1.82 \\
(32,32) m=E
	& 13.60 &  8.89 &  7.03 &  4.94 & 41.85 & 34.26 & 30.17 & 23.87 &  3.80 &  2.99 &  2.39 &  1.95 \\
\hline
(52,16) m=L
	& 15.16 & 11.77 &  8.30 &  5.13 & 50.09 & 42.70 & 33.83 & 23.45 &  3.78 &  3.62 &  2.30 &  1.66 \\
(52,16) m=N
	& 22.66 & 11.65 &  7.78 &  6.80 & 58.29 & 47.31 & 35.17 & 27.99 &  5.81 &  3.94 &  2.69 &  1.98 \\
(52,16) m=E
	& 15.91 & 10.85 &  8.16 &  6.09 & 46.29 & 38.96 & 35.00 & 26.58 &  3.93 &  3.12 &  2.54 &  1.85 \\
\hline
(52,24) m=L
	& 12.04 &  9.24 &  6.91 &  4.34 & 42.81 & 36.72 & 29.42 & 20.99 &  3.53 &  3.35 &  2.40 &  1.64 \\
(52,24) m=N
	& 24.18 & 11.60 &  6.92 &  5.83 & 53.92 & 42.65 & 31.97 & 26.07 &  4.96 &  3.72 &  2.59 &  1.90 \\
(52,24) m=E
	& 12.84 &  9.13 &  7.32 &  5.03 & 41.68 & 34.56 & 31.74 & 24.19 &  3.59 &  3.15 &  2.28 &  1.80 \\
\hline
(52,32) m=L
	& 10.76 &  8.20 &  6.34 &  4.56 & 40.76 & 34.27 & 28.16 & 20.58 &  3.30 &  3.08 &  2.25 &  1.72 \\
(52,32) m=N
	& 17.66 &  9.70 &  6.51 &  6.10 & 52.72 & 41.04 & 29.47 & 24.79 &  4.01 &  3.21 &  2.41 &  1.89 \\
(52,32) m=E
	& 11.44 &  8.18 &  6.53 &  4.98 & 38.85 & 32.12 & 29.42 & 23.10 &  3.23 &  2.92 &  2.22 &  1.78 \\
\hline
\hline
\end{tabular}}
}

\subfloat[Percentage of experiments \logdense V1(m=L) is outperformed]{
    \begin{tabular}{c|cccc}
    \hline
Method    & 1/4 & 1/2 & 3/4 & 1\\
    \hline
\input{table_dsm_compare_cnt.tex}
    \hline
    \end{tabular}
    
    \label{tab:dsm_cnt}
}
~
\subfloat[Relative percentage increase in error rate from \logdense V1(m=L) ]{
    \begin{tabular}{c|cccc}
    \hline
Method   & 1/4 & 1/2 & 3/4 & 1  \\
    \hline
\input{table_dsm_compare_diff.tex}
    \hline
    \end{tabular}
    \label{tab:dsm_gap}
}

\caption{\textbf{(a)} Each entry represents the top-1 error rate after 1/4, 1/2, 3/4, and 1 of the total FLOPS are computed on CIFAR10, 100, and SVHN. $(n,g)$ represents the number of feature layers in each of the three dense blocks. $m=L,N,E$ represents the connection strategies \logdense V1, \nearest, and \evenspace, each of which connects to $1+\log (i)$ layers at depth $i$, but the \logdense V1 has $O(\log(L))$ \pbd while the other two have $O(\frac{L}{\log L})$.\logdense V1 outperforms the others clearly. 
\textbf{(b)} Percentage of the 27 experiments that \logdense is outperformed by \nearest and \evenspace, at each cost location. Lower means \logdense is consistently better.
\textbf{(c)} The average over the 27 comparisons of the relative increase in error rate from those of \logdense by \nearest and \evenspace. Higher means \logdense is better with higher margin. 
}
\label{tab:table_dsm_full}
\end{table}
