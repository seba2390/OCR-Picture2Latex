\documentclass{article}

% if you need to pass options to natbib, use, e.g.:
% \PassOptionsToPackage{numbers, compress}{natbib}
% before loading nips_2017
%
% to avoid loading the natbib package, add option nonatbib:
% \usepackage[nonatbib]{nips_2017}

% \usepackage{nips_2017}
\usepackage[final]{nips_2017}
\usepackage[ruled]{algorithm2e}
\usepackage{graphicx}
\usepackage{float}
\usepackage[mathscr]{euscript}

% to compile a camera-ready version, add the [final] option, e.g.:
% \usepackage[final]{nips_2017}

\usepackage[utf8]{inputenc} % allow utf-8 input
\usepackage[T1]{fontenc}    % use 8-bit T1 fonts
\usepackage{hyperref}       % hyperlinks
\usepackage{url}            % simple URL typesetting
\usepackage{booktabs}       % professional-quality tables
\usepackage{amsfonts}       % blackboard math symbols
\usepackage{nicefrac}       % compact symbols for 1/2, etc.
\usepackage{microtype}      % microtypography

\title{Learnings Options End-to-End for Continuous Action Tasks}

% The \author macro works with any number of authors. There are two
% commands used to separate the names and addresses of multiple
% authors: \And and \AND.
%
% Using \And between authors leaves it to LaTeX to determine where to
% break the lines. Using \AND forces a line break at that point. So,
% if LaTeX puts 3 of 4 authors names on the first line, and the last
% on the second line, try using \AND instead of \And before the third
% author name.


\author{Martin Klissarov, Pierre-Luc Bacon, Jean Harb, Doina Precup\\
Reasoning and Learning Lab,\\
McGill University \\
{\tt \{mklissa,pbacon,jharb,dprecup\}@cs.mcgill.ca}
}



% \author{
%   David S.~Hippocampus\thanks{Use footnote for providing further
%     information about author (webpage, alternative
%     address)---\emph{not} for acknowledging funding agencies.} \\
%   Department of Computer Science\\
%   Cranberry-Lemon University\\
%   Pittsburgh, PA 15213 \\
%   \texttt{hippo@cs.cranberry-lemon.edu} \\
%   %% examples of more authors
%   %% \And
%   %% Coauthor \\
%   %% Affiliation \\
%   %% Address \\
%   %% \texttt{email} \\
%   %% \AND
%   %% Coauthor \\
%   %% Affiliation \\
%   %% Address \\
%   %% \texttt{email} \\
%   %% \And
%   %% Coauthor \\
%   %% Affiliation \\
%   %% Address \\
%   %% \texttt{email} \\
%   %% \And
%   %% Coauthor \\
%   %% Affiliation \\
%   %% Address \\
%   %% \texttt{email} \\
% }

% \input{custom}

\usepackage{amssymb,amsmath}
\usepackage{amsthm}
\usepackage{bm}
\def\given{\middle\vert}
\def\grad{\nabla}
\def\transpose{\top}
\def\eqdef{\overset{\underset{\mathrm{def}}{}}{=}}
\def\indicator{\mathds{1}}
\def\ones{\mathds{1}}
%\DeclareMathOperator*{\expectation}{\mathbb{E}}
\def\expectation{\mathbb{E}}
\def\var{\mathrm{Var}}
\def\prob{P}
%\def\defeq{\equiv}
\def\defeq{\dot=}
\newcommand{\deriv}[2][]{\frac{\partial#1}{\partial#2}}
% \newtheorem{theorem}{Theorem}
\newtheorem{prop}{Proposition}
% \newtheorem{assume}{Assumption}
% \newtheorem{lemma}{Lemma}
% \newtheorem{corollary}{Corollary}
\newcommand{\mat}[1]{\mathbf{#1}}%\mathbf{#1}}
\def\eye{I}
\newcommand{\diag}[1]{\text{diag}(#1)}
\def\dtheta{\deriv[]{\theta}}
\newcommand{\card}[1]{|#1|}
\def\vectorize{\text{vec}}

\def\state{s}
\def\State{S}
\def\states{\mathscr{S}}
\def\action{a}
\def\act{\action}
\def\Action{A}
\def\Act{\Action}
\def\actions{\mathscr{A}}
\def\option{o}
\def\opt{\option}
\def\Option{O}
\def\Opt{\Option}
\def\options{\mathscr{O}}

\def\mdp{\mathcal{M}}
\def\error{\Delta}
\def\stationary{\mu}

\def\discount{\gamma}
\def\cardstates{\card{\states}}
\newcommand{\argmax}{\operatornamewithlimits{argmax}}
\def\inv{{-1}}



\begin{document}
% \nipsfinalcopy is no longer used



\maketitle

\begin{abstract}
  We present new results on learning temporally extended actions for continuous tasks, using the options framework (\cite{SuttonPrecupSingh1999}, \cite{Precup2000}). In order to achieve this goal we work with the option-critic architecture~(\cite{Bacon2017}) using a deliberation cost and train it with proximal policy optimization~(\cite{SchulmanWDRK17}) instead of vanilla policy gradient. Results on Mujoco domains are promising, but lead to interesting questions about \textit{when} a given option should be used, an issue directly connected to the use of initiation sets.
  
 
  

\end{abstract}
\vspace*{-8pt}

\section{Introduction}
\vspace*{-8pt}

The options framework (\cite{SuttonPrecupSingh1999}, \cite{Precup2000}) allows a reinforcement learning agent to represent, learn and plan with temporally extended actions. These temporally extended actions consist of a set of internal policies, termination conditions and sometimes initiation sets that allow controlling the number of choices available to an agent. Given a set of options, the agent will learn a policy over options, which is typically viewed as executing in a call-and-return fashion: once this policy chooses an option, the option will execute until it terminates, then the policy over options will make a new choice. Learning options is beneficial as it leads to specialization in the state space, and therefore to potentially reduced complexity in terms of the internal policies of the options. 
The option-critic architecture (\cite{Bacon2017}) provides an agent with an end-to-end algorithm to learn options in order to maximize the expected discounted return, by relying on ideas akin to actor-critic methods.
% Unfortunately, it sometimes leads to degenerated options where the agent terminates the current option at each time step, or where the agent never terminates. To remedy at this situation, the authors make use of a scalar regularizer in the termination gradient, giving rise to temporally extended options. In these papers, it has been proposed that good options are ones that allow an agent to plan and learn faster.
In this work, we exploit the option-critic architecture by combining it to a recent algorithm, Proximal Policy Optimization (PPO)~(\cite{SchulmanWDRK17}), which is very well suited for continuous control tasks and has shown better sample complexity in empirical comparisons. We present results of our approach on a set of environments from the Mujoco framework; our results are consistent with published evaluations which show that learning options provides increased performance, better interpretability and faster learning.  
\vspace*{-8pt}




\section{Background}
\label{gen_inst}
\vspace*{-8pt}


A Markov Decision Process $\mathcal{M}$ is a tuple $ \defeq (\mathcal{S}, \mathcal{A}, \gamma, r, P)$ with $\mathcal{S}$ the state set, $\mathcal{A}$ the action set and the scalar $\gamma\in [0,1)$ the discount factor. The reward function maps states and actions to a scalar reward $r : \states \times \actions \rightarrow Dist(\mathbb{R})$ and the transition matrix $P: \states \times \actions \to Dist(\states)$ specifies the environment's dynamics.
A policy $\pi$ is a set of probability distributions over actions conditioned on states $\pi$: $\states \to \actions$. For a given policy, the value function $V_\pi(s) \defeq \expectation_\pi\left[ \sum_{t=0} \gamma^t r(S_t, A_t) \given S_0 = s\right]$ defines the expected return obtained by following $\pi$. $V_\pi$ satisfies the Bellman equations : $ V_\pi(s) = \sum_{a} \pi\left(a \given s\right)\left( r(s, a) + \gamma \sum_{s'} \prob\left(s' \given s, a\right) V_\pi(s')\right)$.

The policy gradient theorem (\cite{Sutton1999}) provides the gradient of a parametrized stochastic policy $\pi_\theta$ with respect to the expected discounted return from an initial state distribution $d_0 \in \text{dist}(\states)$. For simplicity, we write the policy as $\pi$, making its parametrization ($\theta$) implicit.
\begin{align*}
    \deriv[L(\theta)]{\theta} = \sum_{s} d(s;\theta) \sum_{a} \deriv[\pi\left(a \given s\right)]{\theta}Q_{\pi}(s, a)
\end{align*} 
where  $d(s;\theta) = \sum_{s_0} d(s_0) \sum_{t=0}^{\infty} \gamma^t P_{\pi}(S_t = s | S_0 = s_0)$
is a weighting of states along the trajectories generated by $\pi$ and passing through $s$. 
Using the log-likelihood trick (\cite{Williams1992}),
\begin{align*}
    \deriv[L(\theta)]{\theta} = \expectation\left[ \deriv[\log \pi\left(A_t \given S_t\right)]{\theta} A^{\pi}(S_t, A_t)  \right]
\end{align*}
where $A^{\pi}(S_t, A_t) = Q_{\pi}(S_t, A_t) - V_{\pi}(S_t)$ is the advantage function. The term $V_{\pi}(s_t)$ acts as a \textit{baseline} (\cite{Williams1992,Sutton1999}) which reduces the variance of the resulting
estimator. 
\vspace*{-8pt}

\subsection{Trust region methods and Proximal Policy Optimization (PPO)}
\vspace*{-8pt}

Trust region methods, and in particular the TRPO algorithm (\cite{SchulmanLMJA15}), are second-order methods that maximize a surrogate objective subject to a constraint.
% %
% \begin{align*}
% \max_{\theta}  \expectation \left[ r_t(\theta) \hat A_t  \right] \hspace{2em} \text{subject to :} \hspace{2em} \expectation[KL[\pi_{\theta_{old}}(\cdot|s_t),\pi_{\theta}(\cdot|s_t)]] \leq k
% \end{align*}
% %
%where $r_t(\theta) = \frac{\pi_{\theta}(a_t | s_t)}{\pi_{\theta_{old}}(a_t | s_t)}$. $\theta_{old}$ represents the parameterization of the policy before the update. 
%It is possible to approximately solve this problem by using the conjugate gradient algorithm. 
TRPO has proven useful for continuous control, but it can be computationally expensive and doesn't allow for parameter sharing.

Proximal Policy Optimiation (PPO) achieves the same level of reliability and performance as TRPO while being a first-order method. To do so, it uses an objective with clipped probability ratios, preventing an excessive shift in the probability distribution between updates. This clipping also allows for multiple epochs of minibatch updates on a single sampled trajectory. The clipped surrogate objective is:
\begin{align*}
  \deriv[L(\theta)^{\text{PPO}}]{\theta}  = \expectation\left[\deriv[]{\theta} \min( \rho_t(\theta) A^{\pi}(S_t, A_t), \text{clip}(\rho_t(\theta), 1 - \epsilon, 1+ \epsilon) A^{\pi}(S_t, A_t))\right]
\end{align*}
where $\rho_t(\theta) = \frac{\pi(A_t | S_t)}{\pi_{{old}}(A_t | S_t)}$ is the importance sampling ratio. The authors use the Generalized Advantage Estimation (\cite{SchulmanMLJA15}) to calculate the advantage function $A^{\pi}(S_t, A_t)$.
\vspace*{-8pt}

\subsection{Option-Critic}
\vspace*{-8pt}

The option-critic architecture (\cite{Bacon2017}) is a gradient-based approach for learning intra-option policies as well termination conditions, assuming that all options are available at every state. Moreover, the parameters of the intra-option policies ($\theta_{\pi}$) and the termination function ($\theta_{\beta}$)  are assumed to be independent. The intra-option policy gradient is as follows:
\begin{align*}
    \deriv[L(\theta)]{\theta_\pi} =  \expectation\left[ \deriv[\log \pi\left(\Action_t \given \State_t,\Option_t\right)]{\theta_\pi} Q_\pi(\State_t,\Option_t, \Action_t)\right] \enspace 
\end{align*}
where a baseline (i.e. the above state-option value function $Q_\pi $  parametrized by $\theta_w$) is generally added. However, if the options are learned to optimize returns, in the long run, they will tend to disappear since any MDP can be solved optimally using primitive actions. To avoid this problem, \cite{Harb2018} use the bounded rationality framework (\cite{Simon1969}) and introduce a deliberation cost ($\eta$), interpreted as  a margin of how much better an option should be than the current option in order to replace it. The termination gradient then takes the following form:
\begin{align*}
    \deriv[L(\theta)]{\theta_\beta} = \expectation\left[ -\deriv[\beta(\State_t, \Option_t)]{\theta_\beta} (A^{\pi \beta}(\State_t,\Option_t) + \eta) \right] 
\end{align*}
where $A^{\pi \beta}(\State_t,\Option_t) = Q_\pi(\State_t,\Option_t) - V_\pi(\State_t)$ is the termination advantage function and stems directly from the derivation of the gradient. 
\vspace*{-8pt}

\section{Algorithm}
\vspace*{-8pt}

We introduce the Proximal Policy Option-Critic (PPOC) algorithm which, just like PPO, works in two stages. In the first stage, the agent collects trajectories of different options and computes the advantage functions using Monte-Carlo returns. We then proceed to the optimization stage where, for K optimizer iterations, we choose M tuples and apply the gradients. We also chose to use a stochastic policy over options, parameterized by an independent vector $\theta_{\mu}$ (as opposed to $\epsilon$-greedy) which we learned under the same policy gradient approach. 

\begin{algorithm}[ht]
\DontPrintSemicolon
\SetAlgoLined
\For{iteration=1,2,....}{
    $c_t \leftarrow 0$\;
    $s_t \leftarrow s_0$\;
    Choose $\option_t$ with a softmax policy over options $\mu(\option_t|s_t)$ \\
    \Repeat{T timesteps} {
        Choose $a_t$ according to $\pi(a_t|s_t)$ \;
        Take action $a_t$ in $s_t$, observe $s_{t+1}$, $r_t$\;
        $\hat r_t = r_t - c_t$\;
        \uIf{$\beta$ terminates in $s_{t+1}$}{
        choose new $\option_{t+1}$ according to softmax $\mu(\option_{t+1}|s_{t+1})$\;
        $c_t = \eta$}
        \Else{$c_t = 0$}
    }
    
    Compute the advantage estimates for each timestep; \\
    \For{$\option$=$\option_1$,$\option_2$,....}{
        $\theta_{old} \leftarrow \theta$\;
        \For{K optimizer iterations with minibatches M}{
            $\theta_{\pi}  \leftarrow \theta_{\pi} + \alpha_{\theta_{\pi}} \deriv[L_t(\theta)^{\text{PPO}}]{\theta_\pi} $ \;
            
            $\theta_\beta \leftarrow \theta_\beta - \alpha_{\theta_\beta} \deriv[\beta(s_t)]{\theta_\beta} \left( A(s_t,o_t) + \eta \right)\;$\;
            
            $\theta_{\mu}  \leftarrow \theta_{\mu} + \alpha_{\theta_{\mu}} \deriv[\log\mu(\option_t|s_t)]{\theta_\mu}  A(s_t,\opt_t) $ \;
            
            $\theta_w \leftarrow \theta_w - \alpha_{\theta_w} \deriv[(G_t - Q_\pi(s_t,\option_t))^2]{\theta_w}$\;
            
        }
    }
    
}

%where $ L_t(\theta_\omega)$ = min$( \frac{\pi_{\theta_{\omega}}(a_t | s_t)}{\pi_{\theta_{\omega_{old}}}(a_t | s_t)} \hat A_t$  , clip($\frac{\pi_{\theta_{\omega}}(a_t | s_t)}{\pi_{\theta_{\omega_{old}}}(a_t | s_t)}, 1 - \epsilon, 1+ \epsilon)\hat A_t) $ \;
\caption{Proximal Policy Option Critic (PPOC)}
\end{algorithm}
\section{Experiments}
We performed experiments on locomotion tasks available on OpenAI's Gym (\cite{BrockmanCPSSTZ16}) using the Mujoco simulator (\cite{Todorov2012MuJoCoAP}).  We aim to assess the following: (1) whether the use of options can increase the speed of learning as well as the final performance, (2) the interpretability of the resulting options.

In our experiments, we used as input the vectors defining joint angles, joint velocities, and coordinates of the center of mass. We used two separate networks with 64 hidden units per layer, each containing two layers.\footnote{The code, as well as the values for the hyperparameters, are available here: \url{https://github.com/mklissa/PPOC}} For all the layers we used tanh non-linearity, except for the output which was linear for of value functions and intra-option policies, sigmoid for the termination probability and softmax for the policy over options. The first network was used to output the policy over options $\mu(\option | s)$ and the intra-option policies $\pi(a | s)$, while the second network was used to output the value functions $Q_\pi(s,\option)$ and the termination probabilities $\beta(s)$. The log-standard deviations were parameterized by a vector independent of the input state. 
We used the exact same hyper-parameters as mentioned in \cite{SchulmanWDRK17}, except for the optimizer mini-batch size which was divided by the number of options. We proceeded so in order to avoid training more samples per iteration with the options framework, thus enabling a fair comparison between options and primitive actions. In the case of options, we also divide the reward by 10 to reduce the scale of the value functions, and therefore the termination probability gradient, making it more stable.  We didn't proceed to any hyper-parameters search to improve the results. Our experiments exclusively investigate the merits of using two options and compare the results to the case of primitive actions (no options).

In addition to the classic Mujoco environments, we ran agents in an environment called HopperIceBlock-v0.\footnote{HopperIceBlock-v0 is based on \cite{henderson2017multitask} and is avaialable here: \url{https://github.com/mklissa/gym-extensions}} This environment contained a more explicit compositionality than the original Mujoco environments available on OpenAI's Gym, which simply require  to learn a gait that maximizes speed in a direction. We used Hopper-v1 as a starting point and added some obstacles in the agent's path: solid slippery blocks. The agent had to learn to pass them either by jumping completely over them or by sliding on their surface.

\begin{figure*}[ht]
  \centering

  \includegraphics[width=\textwidth]{figs/score3.png}
  \caption{Results in Mujocco using 12 different random seeds for a total of 1 million steps (each iteration is 2000 steps)}
\end{figure*}




The results are summarized in Fig.1. As expected, using options with a deliberation cost yields better results and faster learning on most environments. It is interesting to note that the increase in performance is not directly proportional to the value of $\eta$. This is due to the different scales of the average returns across  environments, as well as during the course of learning. In the current formulation of the deliberation cost, its value is a hyperparameter that has to be set. It would be useful to explore the possibility of working with a learned value instead of a constant. This is left as future work.

The results that stand out the most are the one on the customized environment. More importantly, the success threshold for the environment is around 1200 points, under that level the agent actually doesn't learn to pass the iceblock and continue its gait.  So, the agent using options is the only one solving this environment. This also led us to investigate how the options are used in this environment as opposed to the classic Mujoco environments.\footnote{Videos from the environments are available at \url{https://www.youtube.com/watch?v=XI_txkRnKjU}} In HopperIceBlock-v0, the interpretability of the options is obvious and greatly helps the performance: one option is used to hop when there is no iceblock nearby, but then when passing over the iceblock, both options are used to complete the specific task. 
% and the other one is used to pass over, or slide on, the iceblock. 
In the case of the classic Mujoco environments, one option is used to gain momentum at the start of the episode and is never used thereafter. Even if the agents using options outperform the agents with primitive actions on classic environments, we can only truly see the benefits of a hierarchical framework when used in the appropriate environment. 

\section{Conclusion}

Our experiments  demonstrate that it is possible to learn options in an end-to-end manner using deep networks on continuous actions environments, and to the best of our knowledge this is the first work to do so. Our results also suggest that the increase in performance is not directly linked to the deliberation cost, which is problematic as it leaves us with the task of finding the right value. For the options framework to be truly end-to-end it would be necessary to learn a value of $\eta$.
More importantly, we have seen that the increase in performance is related to the compositionality of the environment. In the classic Mujoco environments, using options is not as beneficial as using them in a customized environment with a more obvious division in the state-space. This leads to the following question: when should we be using options? This question also points to a fundamental problem in the current options framework: it is necessary for us to manually specify the number of available options. How should one decide on this number? As stated in \cite{Bacon2017}, one way to answer this question would be to reintroduce the notion of initiation sets in the option-critic architecture.

\bibliographystyle{named}
\bibliography{references}
% \documentclass{article}

% if you need to pass options to natbib, use, e.g.:
% \PassOptionsToPackage{numbers, compress}{natbib}
% before loading nips_2017
%
% to avoid loading the natbib package, add option nonatbib:
% \usepackage[nonatbib]{nips_2017}

\usepackage{nips_2017}

% to compile a camera-ready version, add the [final] option, e.g.:
% \usepackage[final]{nips_2017}

\usepackage[utf8]{inputenc} % allow utf-8 input
\usepackage[T1]{fontenc}    % use 8-bit T1 fonts
\usepackage{hyperref}       % hyperlinks
\usepackage{url}            % simple URL typesetting
\usepackage{booktabs}       % professional-quality tables
\usepackage{amsfonts}       % blackboard math symbols
\usepackage{nicefrac}       % compact symbols for 1/2, etc.
\usepackage{microtype}      % microtypography
\usepackage[utf8]{inputenc}
\usepackage{graphicx}
\usepackage{subcaption}
\usepackage{amsmath}

\title{Rapid point-of-care Hemoglobin measurement through low-cost optics and Convolutional Neural Network based validation}

% The \author macro works with any number of authors. There are two
% commands used to separate the names and addresses of multiple
% authors: \And and \AND.
%
% Using \And between authors leaves it to LaTeX to determine where to
% break the lines. Using \AND forces a line break at that point. So,
% if LaTeX puts 3 of 4 authors names on the first line, and the last
% on the second line, try using \AND instead of \And before the third
% author name.

  \author{
  Chris Wu\\
  Athelas Inc. \\
  \texttt{chris@athelas.com}
  \And
  Tanay Tandon\\
  Athelas Inc. \\
  \texttt{tanay@athelas.com}
}
\begin{document}
% \nipsfinalcopy is no longer used
\maketitle
\begin{abstract}
  A low-cost, robust, and simple mechanism to measure hemoglobin would play a critical role in the modern health infrastructure. Consistent sample acquisition has been a long-standing technical hurdle for photometer-based portable hemoglobin detectors which rely on micro cuvettes and dry chemistry. Any particulates (e.g. intact red blood cells (RBCs), microbubbles, etc.) in a cuvette's sensing area drastically impact optical absorption profile, and commercial hemoglobinometers lack the ability to automatically detect faulty samples. We present the ground-up development of a portable, low-cost and open platform with equivalent accuracy to medical-grade devices, with the addition of CNN-based image processing for rapid sample viability prechecks. The developed platform has demonstrated precision to the nearest $0.18[g/dL]$ of hemoglobin, an \(R^{2} = 0.945\) correlation to hemoglobin absorption curves reported in literature, and a 97\% detection accuracy of poorly-prepared samples. We see the developed hemoglobin device/ML platform having massive implications in rural medicine, and consider it an excellent springboard for robust deep learning optical spectroscopy: a currently untapped source of data for detection of countless analytes.
\end{abstract}
\section{Introduction}
Hemoglobin is one of the most common blood tests requested by clinics, and can be used in conjunction with other metrics to diagnose a host of diseases and conditions \cite{mayoclinic}\cite{biomed}, quantify the effects of pharmaceutical drugs \cite{medlineplus2}, and provide a holistic health benchmark \cite{biomed}. As roughly a quarter of the world’s population suffers from a form of hemoglobin deficiency \cite{who}, there are many niches where accessible hemoglobin measurement would fulfill significant unmet need \cite{cdc}\cite{drugsaging}\cite{biomed}.

The widely-adopted approach to point-of-care hemoglobin measurement involves hemolysing whole blood and converting hemoglobin derivatives for single-wavelength absorption measurement, followed by a simple Beer Law calculation \cite{hemocue}\cite{vanzetti}\cite{oshiro}. In present hemoglobin monitors the single biggest risk of inaccurate counting and as a result, incorrect clinical decision making, is incomplete hemolysis due to variable reagent performance. Consequently, light scattering interference from intact cells (lipid membranes) have a drastic effect on spectrophotometric output \cite{lipemia}.

Our intention in developing the presented device platform was twofold: to create an effective and inexpensive hemoglobin solution without the need for hazardous dry reagents such as cyanide (Drabkin's method) or sodium azide \cite{hemocue}\cite{vanzetti}\cite{azide}, and to demonstrate the efficacy of convolutional neural network based validation for elimination of a long-standing error source in hemoglobin measurement. 

\section{Detection platform summary}

The current version of the device has demonstrated: 1) precision to the nearest $0.18[g/dL]$ of Hb, 2) stability of measurement over time, with no fluctuation from a single sample over 10 minutes of continuous output, and 3) \(R^2 = 0.945\) correlation to optical transmission-Hb curve reported in literature (Fig.\ref{fig:HbCorr} image A).

\subsection{Device platform}

\begin{figure*}[!ht]
\centering
    \includegraphics[width=0.8\textwidth]{Images/design.png}
    \caption{Image A) Design concept rendering. Image B) Completed device with blank strip and power adapter inserted.}
    \label{fig:Chassis}
\end{figure*}

\subsubsection{Sensor configuration}
For Hb measurement, a paired $540[nm]$ emitter-detector setup is used for sample transmission calculation. The LED and photodiode were selected to maximize light emission, photocurrent, and Hb absorption.

Image pre-processing for sample viability assurance are captured with a Pi Camera attached to a low-cost optical microscopy setup (separate from device at this point in development). See figure 4 for optical microscopy imaging examples. Both setups have a test strip insertion slot and a plane for allowing optical analysis.  
 
\subsubsection{Signal processing}
An $LTC1050$ Chopper-stabilized op amp is used in a standard transimpedance amplifier configuration for signal processing with a reverse-biased photodiode for signal linearity. Additional components protect sensitive analog signals from various forms of electromagnetic interference, including high frequency noise and LC-tank oscillation due to inherent component properties. As absorption is taken at steady-state, signal bandwidth is not a design concern.

Hb concentration is linearly proportional to optical absorbance ($A$), which is calculated from the negative log of fractional transmission, $Tf/Ti$. From Beer’s Law:
\begin{equation}
C = -Klog(\frac{Tf}{Ti})
\end{equation}
for $K = 1/(el)$, where $e$ is the molar extinction coefficient of hemoglobin, $l$ is the transmission path length ($100[\mu m]$), and $C$ is the concentration of hemoglobin in the sample (in $[g/dL]$).

The device's microcontroller contains a 10-bit ADC with a minimum voltage increment of $5[mV]$. With a swing of $1.5[V]$ the resolution (i.e. the smallest detectable change in sample percent transmission) is 0.333\%. Readjusting $R1$ to maximize full output swing ($5[V]$) will improve resolution in the next device iteration to 0.1[\%T]. For clinically relevant Hb levels and 0.333[\%T] increment size, the resolution is $0.18 [g/dL]$ Hb  (i.e. the difference between consecutive Hb values over a 0.333[\%T] step). This exceeds the precision of predicate devices such as Hemocue, which has an overall bias $\pm$ stdev of -$0.1 \pm 1.6[g/dL]$ compared to lab-grade hematology analyzers \cite{shah}. Of course, this level of precision from the presented device assumes no variability in capillary strip performance: a hefty assumption. 

\begin{figure*}[!ht]
\centering
    \includegraphics[width=0.8\textwidth]{Images/fig1.png}
    \caption{Image A) Plot of literature-reported Hb values for observed \%T measurements (extrapolated from $540[nm]$ molar extinction coefficient \cite{coeff1}\cite{coeff2}\cite{coeff3}) vs. the developed device readout for those \%T values. Strong correlation demonstrates hardware precision and accuracy over the full clinically-relevant range of hemoglobin levels. Image B) PIN photodiode and first amplifier stage used for signal acquisition.}
    \label{fig:HbCorr}
\end{figure*}
\subsection{Capillary test strip}

The test strip consists of a 100uM microchannel coated with sodium lauryl sulfate (SLS), a non-toxic surfactant which serves the dual purpose of lysing RBCs to eliminate turbidity and converting hemoglobin derivatives into a color-stable complex, SLS-Hb.
\begin{figure*}[!ht]
\centering
    \includegraphics[width=0.80\textwidth]{Images/blood.png}
    \caption{Image A is a diagram of the capillary strip design used. Strip with blood sample shown in B. Close-up of a channel demonstrating common issues with dry-reagent capillary strip methods in image C: turbidity at the bottom of the channel due to intact RBCs and air bubbles (white specks) at top. Image D shows the mechanical frame of the sensor input slot with the sample transmission measurement window.}
    \label{fig:blood}
\end{figure*}

In standard practice with SLS, $20[uL]$ of whole blood is added and mixed with $5[mL]$ of $2.08[mmol/L]$ SLS solution for a 0.52 moles SLS to blood sample volume ratio \cite{oshiro}. To adapt for dry-chemistry, $17[\mu mols]$ of SLS in aqueous solution were deposited and spread evenly onto each coverslip, for a total of $34[\mu mols]$ per $66[\mu L]$ available sample volume, maintaining the ratio.

However, since all fluid entering a capillary strip generally shares the same path, depending on viscosity and RBC density, the first units of blood entering the channel may use up reagent at the entrance, leaving none for subsequent units (hence incomplete hemolysis). This issue poses a critical error-source common to all dry capillary strip detection methods, which can be minimized through clever strip design.

A funnel-shaped channel (Fig.\ref{fig:blood} image A) was used to help mitigate the “use up” issue described. Blood deposited at the entrance of the funnel spreads out to cover a larger area before entering the narrow measurement area, having been hemolyzed by reagent in the funnel. While this design was generally effective in eliminating turbidity in the measurement window (Fig.\ref{fig:blood} panel D), variation in RBC density and sample viscosity prevented proper mixing/reaction on occasion, precluding this dry-chemistry approach from large-scale, frequent use with minimal user training. Only an automated viability check to auto detect such errors when present can completely eliminate this risk. 

\subsubsection{CNN-based image processing for sample viability check}

A core component of the presented system is the trained machine learning model for sample validation.

A separate imaging module was developed to analyze intact cells, air bubbles, and ensure appropriate sample prep. A filled strip is inserted into the imaging module, several images are taken across the strip at a scaled magnification, and a trained convolutional neural network determines the viability of the testing strip. 
\begin{figure*}[!ht]
\centering
    \includegraphics[width=1\textwidth]{Images/cnn.jpg}
    \caption{Images A,B,D) Image captures from the optical microscopy setup identified by the model as inadequate for spectrophotometry. Sample A shows debris and color instability. Sample B has image blur and particulates. Sample D contains intact cells which act as light scatterers. Additionally, green boxes identify white blood cells from the sample, demonstrating expandability of model for a vast number of use cases. Image C) Clear view with color stability and no scatterers: ideal for photometric hemoglobin measurement. Data was collected using a labeling interface of all run test strips (bottom of E). 200 strip images were labeled by a human identifying sections of debris and cell boundaries, and the overall image labeled "good" vs. "bad".}
    \label{fig:CNN}
\end{figure*}
The model was adapted from LeNet structure, and trained on 200 human labeled images of good/bad test strips. The images were augmented to expand the training set by 8x (rotational, translational, brightness, and hue transformations). These results were then cross validated on a test set of 100 strip images. 

The binary classification model performed with 97\% accuracy on the task across the non-augmented test set - this included filled strips by clinicians and general users. Similarly, cell segmentation modules were developed to threshold cell boundaries with each candidate cell being trained in a separate Convolutional Neural Net for type classification (White Blood Cell, Red Blood Cell, etc.). 
\section{Conclusion}
Our platform builds off decades of hemoglobin spectroscopy research, while providing unprecedented error detection and reliability thanks to convolutional neural networks. We believe this deep learning approach to on-board quality control can be rapidly scaled to other applications, boosting clinical performance, end diagnoses, and patient safety. 

Given the low-cost, open microcontroller nature of the presented device, we see the platform and model as a powerful combination for deploying in decentralized or rural areas. Untrained test operators have been a major reason point-of-care, self-administered hematology tests have remained unfeasible. However, with robust error correction, the risk of misdiagnosis is greatly reduced and this radical model of personalized medicine is brought within the realm of possibility.

With high-precision analog sensing and CNN based image processing, we have developed the foundation for multi-wavelength spectrophotometry, enabling both structural and molecular sample analyses. We are convinced that the confluence of these distinct, rich sources of data with the speed and versatility of AI will present numerous opportunities to advance the future of healthcare.

\section{Acknowledgements}
We are sincerely thankful to Dhruv Parthasarathy, Deepika Bodapati, Louis Virey, Steve Moffatt, and Sreevaths Kasireddy for funding acquisition and technical input in various aspects of this project.

\medskip

\begin{thebibliography}{9}

\bibitem{mayoclinic}
“Low Hemoglobin Count Causes.” \textit{Mayo Clinic}, Mayo Foundation for Medical Education and Research, 8 May 2015, www.mayoclinic.org/symptoms/low-hemoglobin/basics/causes/sym-20050760.

\bibitem{medlineplus2}
Gersten, Todd. “Drug-Induced Immune Hemolytic Anemia.” \textit{MedlinePlus Medical Encyclopedia}, 1 Feb. 2017, medlineplus.gov/ency/article/000578.htm.

\bibitem{shah}
Shah, N, E A Osea, and G J Martinez. “Accuracy of Noninvasive Hemoglobin and Invasive Point-of-Care Hemoglobin Testing Compared with a Laboratory Analyzer.” \textit{International Journal of Laboratory Hematology}, 2014, pp. 56–61.

\bibitem{who}
WHO. The global prevalence of anaemia in 2011. Geneva: World Health Organization;
2015.

\bibitem{cdc}
“CDC Newsroom: Falls Are Leading Cause of Injury and Death in Older Americans.” \textit{Centers for Disease Control and Prevention}, Centers for Disease Control and Prevention, www.cdc.gov/media/releases/2016/p0922-older-adult-falls.html.

\bibitem{drugsaging}
Duh, MS. “Anaemia and the Risk of Injurious Falls in a Community-Dwelling Elderly Population.” \textit{Drugs Aging}, vol. 4, 2008, pp. 325–334. US National Library of Medicine

\bibitem{biomed}
Yang, Wenfang. “Anemia, Malnutrition and Their Correlations with Socio-Demographic Characteristics and Feeding Practices among Infants Aged 0–18 Months in Rural Areas of Shaanxi Province in Northwestern China: a Cross-Sectional Study.” \textit{BMC Public Health}, 29 Dec. 2012.

\bibitem{hemocue}
Williamsson, Anders. \textit{Capillary Microcuvette}. 7 Oct. 1997.

\bibitem{vanzetti}
Vanzetti G. "An azide-methemoglobin method for hemoglobin determination in blood." \textit{Journal of Laboratory and Clinical Medicine}, 1966, pp. 116-26

\bibitem{azide}
Chang, Soju and Lamm H Steven. "Human Health Effects of Sodium Azide Exposure: A Literature Review and Analysis" \textit{International Journal of Toxicology}, vol. 22, issue 3, 2016, pp. 175 - 186

\bibitem{oshiro}
Oshiro, Iwao. “New Method for Hemoglobin Determination by Using Sodium Lauryl Sulfate (SLS).” \textit{Clinical Biochemistry}, vol. 15, Apr. 1982, pp. 83–88.

\bibitem{coeff1}
N. Kollias, Wellman Laboratories, Harvard Medical School, Boston

\bibitem{coeff2}
W. B. Gratzer, Med. Res. Council Labs, Holly Hill, London

\bibitem{coeff3}
Benesch, Ruth E., et al. “Equations for the Spectrophotometric Analysis of Hemoglobin Mixtures.” \textit{Analytical Biochemistry}, vol. 55, no. 1, 16 May 1973, pp. 245–248., doi:10.1016/0003-2697(73)90309-6.

\bibitem{lipemia}
Creer, Michael H, and Jack Ladenson. “Analytical Errors Due to Lipemia.” \textit{Laboratory Medicine}, vol. 14, no. 6, June 1983, pp. 351–355.

\end{thebibliography}

\end{document}





\clearpage


\end{document}
