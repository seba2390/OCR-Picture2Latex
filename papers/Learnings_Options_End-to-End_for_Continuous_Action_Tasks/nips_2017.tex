\documentclass{article}

% if you need to pass options to natbib, use, e.g.:
% \PassOptionsToPackage{numbers, compress}{natbib}
% before loading nips_2017
%
% to avoid loading the natbib package, add option nonatbib:
% \usepackage[nonatbib]{nips_2017}

% \usepackage{nips_2017}
\usepackage[final]{nips_2017}
\usepackage[ruled]{algorithm2e}
\usepackage{graphicx}
\usepackage{float}
\usepackage[mathscr]{euscript}

% to compile a camera-ready version, add the [final] option, e.g.:
% \usepackage[final]{nips_2017}

\usepackage[utf8]{inputenc} % allow utf-8 input
\usepackage[T1]{fontenc}    % use 8-bit T1 fonts
\usepackage{hyperref}       % hyperlinks
\usepackage{url}            % simple URL typesetting
\usepackage{booktabs}       % professional-quality tables
\usepackage{amsfonts}       % blackboard math symbols
\usepackage{nicefrac}       % compact symbols for 1/2, etc.
\usepackage{microtype}      % microtypography

\title{Learnings Options End-to-End for Continuous Action Tasks}

% The \author macro works with any number of authors. There are two
% commands used to separate the names and addresses of multiple
% authors: \And and \AND.
%
% Using \And between authors leaves it to LaTeX to determine where to
% break the lines. Using \AND forces a line break at that point. So,
% if LaTeX puts 3 of 4 authors names on the first line, and the last
% on the second line, try using \AND instead of \And before the third
% author name.


\author{Martin Klissarov, Pierre-Luc Bacon, Jean Harb, Doina Precup\\
Reasoning and Learning Lab,\\
McGill University \\
{\tt \{mklissa,pbacon,jharb,dprecup\}@cs.mcgill.ca}
}



% \author{
%   David S.~Hippocampus\thanks{Use footnote for providing further
%     information about author (webpage, alternative
%     address)---\emph{not} for acknowledging funding agencies.} \\
%   Department of Computer Science\\
%   Cranberry-Lemon University\\
%   Pittsburgh, PA 15213 \\
%   \texttt{hippo@cs.cranberry-lemon.edu} \\
%   %% examples of more authors
%   %% \And
%   %% Coauthor \\
%   %% Affiliation \\
%   %% Address \\
%   %% \texttt{email} \\
%   %% \AND
%   %% Coauthor \\
%   %% Affiliation \\
%   %% Address \\
%   %% \texttt{email} \\
%   %% \And
%   %% Coauthor \\
%   %% Affiliation \\
%   %% Address \\
%   %% \texttt{email} \\
%   %% \And
%   %% Coauthor \\
%   %% Affiliation \\
%   %% Address \\
%   %% \texttt{email} \\
% }

% \input{custom}

\usepackage{amssymb,amsmath}
\usepackage{amsthm}
\usepackage{bm}
\def\given{\middle\vert}
\def\grad{\nabla}
\def\transpose{\top}
\def\eqdef{\overset{\underset{\mathrm{def}}{}}{=}}
\def\indicator{\mathds{1}}
\def\ones{\mathds{1}}
%\DeclareMathOperator*{\expectation}{\mathbb{E}}
\def\expectation{\mathbb{E}}
\def\var{\mathrm{Var}}
\def\prob{P}
%\def\defeq{\equiv}
\def\defeq{\dot=}
\newcommand{\deriv}[2][]{\frac{\partial#1}{\partial#2}}
% \newtheorem{theorem}{Theorem}
\newtheorem{prop}{Proposition}
% \newtheorem{assume}{Assumption}
% \newtheorem{lemma}{Lemma}
% \newtheorem{corollary}{Corollary}
\newcommand{\mat}[1]{\mathbf{#1}}%\mathbf{#1}}
\def\eye{I}
\newcommand{\diag}[1]{\text{diag}(#1)}
\def\dtheta{\deriv[]{\theta}}
\newcommand{\card}[1]{|#1|}
\def\vectorize{\text{vec}}

\def\state{s}
\def\State{S}
\def\states{\mathscr{S}}
\def\action{a}
\def\act{\action}
\def\Action{A}
\def\Act{\Action}
\def\actions{\mathscr{A}}
\def\option{o}
\def\opt{\option}
\def\Option{O}
\def\Opt{\Option}
\def\options{\mathscr{O}}

\def\mdp{\mathcal{M}}
\def\error{\Delta}
\def\stationary{\mu}

\def\discount{\gamma}
\def\cardstates{\card{\states}}
\newcommand{\argmax}{\operatornamewithlimits{argmax}}
\def\inv{{-1}}



\begin{document}
% \nipsfinalcopy is no longer used



\maketitle

\begin{abstract}
  We present new results on learning temporally extended actions for continuous tasks, using the options framework (\cite{SuttonPrecupSingh1999}, \cite{Precup2000}). In order to achieve this goal we work with the option-critic architecture~(\cite{Bacon2017}) using a deliberation cost and train it with proximal policy optimization~(\cite{SchulmanWDRK17}) instead of vanilla policy gradient. Results on Mujoco domains are promising, but lead to interesting questions about \textit{when} a given option should be used, an issue directly connected to the use of initiation sets.
  
 
  

\end{abstract}
\vspace*{-8pt}

\section{Introduction}
\vspace*{-8pt}

The options framework (\cite{SuttonPrecupSingh1999}, \cite{Precup2000}) allows a reinforcement learning agent to represent, learn and plan with temporally extended actions. These temporally extended actions consist of a set of internal policies, termination conditions and sometimes initiation sets that allow controlling the number of choices available to an agent. Given a set of options, the agent will learn a policy over options, which is typically viewed as executing in a call-and-return fashion: once this policy chooses an option, the option will execute until it terminates, then the policy over options will make a new choice. Learning options is beneficial as it leads to specialization in the state space, and therefore to potentially reduced complexity in terms of the internal policies of the options. 
The option-critic architecture (\cite{Bacon2017}) provides an agent with an end-to-end algorithm to learn options in order to maximize the expected discounted return, by relying on ideas akin to actor-critic methods.
% Unfortunately, it sometimes leads to degenerated options where the agent terminates the current option at each time step, or where the agent never terminates. To remedy at this situation, the authors make use of a scalar regularizer in the termination gradient, giving rise to temporally extended options. In these papers, it has been proposed that good options are ones that allow an agent to plan and learn faster.
In this work, we exploit the option-critic architecture by combining it to a recent algorithm, Proximal Policy Optimization (PPO)~(\cite{SchulmanWDRK17}), which is very well suited for continuous control tasks and has shown better sample complexity in empirical comparisons. We present results of our approach on a set of environments from the Mujoco framework; our results are consistent with published evaluations which show that learning options provides increased performance, better interpretability and faster learning.  
\vspace*{-8pt}




\section{Background}
\label{gen_inst}
\vspace*{-8pt}


A Markov Decision Process $\mathcal{M}$ is a tuple $ \defeq (\mathcal{S}, \mathcal{A}, \gamma, r, P)$ with $\mathcal{S}$ the state set, $\mathcal{A}$ the action set and the scalar $\gamma\in [0,1)$ the discount factor. The reward function maps states and actions to a scalar reward $r : \states \times \actions \rightarrow Dist(\mathbb{R})$ and the transition matrix $P: \states \times \actions \to Dist(\states)$ specifies the environment's dynamics.
A policy $\pi$ is a set of probability distributions over actions conditioned on states $\pi$: $\states \to \actions$. For a given policy, the value function $V_\pi(s) \defeq \expectation_\pi\left[ \sum_{t=0} \gamma^t r(S_t, A_t) \given S_0 = s\right]$ defines the expected return obtained by following $\pi$. $V_\pi$ satisfies the Bellman equations : $ V_\pi(s) = \sum_{a} \pi\left(a \given s\right)\left( r(s, a) + \gamma \sum_{s'} \prob\left(s' \given s, a\right) V_\pi(s')\right)$.

The policy gradient theorem (\cite{Sutton1999}) provides the gradient of a parametrized stochastic policy $\pi_\theta$ with respect to the expected discounted return from an initial state distribution $d_0 \in \text{dist}(\states)$. For simplicity, we write the policy as $\pi$, making its parametrization ($\theta$) implicit.
\begin{align*}
    \deriv[L(\theta)]{\theta} = \sum_{s} d(s;\theta) \sum_{a} \deriv[\pi\left(a \given s\right)]{\theta}Q_{\pi}(s, a)
\end{align*} 
where  $d(s;\theta) = \sum_{s_0} d(s_0) \sum_{t=0}^{\infty} \gamma^t P_{\pi}(S_t = s | S_0 = s_0)$
is a weighting of states along the trajectories generated by $\pi$ and passing through $s$. 
Using the log-likelihood trick (\cite{Williams1992}),
\begin{align*}
    \deriv[L(\theta)]{\theta} = \expectation\left[ \deriv[\log \pi\left(A_t \given S_t\right)]{\theta} A^{\pi}(S_t, A_t)  \right]
\end{align*}
where $A^{\pi}(S_t, A_t) = Q_{\pi}(S_t, A_t) - V_{\pi}(S_t)$ is the advantage function. The term $V_{\pi}(s_t)$ acts as a \textit{baseline} (\cite{Williams1992,Sutton1999}) which reduces the variance of the resulting
estimator. 
\vspace*{-8pt}

\subsection{Trust region methods and Proximal Policy Optimization (PPO)}
\vspace*{-8pt}

Trust region methods, and in particular the TRPO algorithm (\cite{SchulmanLMJA15}), are second-order methods that maximize a surrogate objective subject to a constraint.
% %
% \begin{align*}
% \max_{\theta}  \expectation \left[ r_t(\theta) \hat A_t  \right] \hspace{2em} \text{subject to :} \hspace{2em} \expectation[KL[\pi_{\theta_{old}}(\cdot|s_t),\pi_{\theta}(\cdot|s_t)]] \leq k
% \end{align*}
% %
%where $r_t(\theta) = \frac{\pi_{\theta}(a_t | s_t)}{\pi_{\theta_{old}}(a_t | s_t)}$. $\theta_{old}$ represents the parameterization of the policy before the update. 
%It is possible to approximately solve this problem by using the conjugate gradient algorithm. 
TRPO has proven useful for continuous control, but it can be computationally expensive and doesn't allow for parameter sharing.

Proximal Policy Optimiation (PPO) achieves the same level of reliability and performance as TRPO while being a first-order method. To do so, it uses an objective with clipped probability ratios, preventing an excessive shift in the probability distribution between updates. This clipping also allows for multiple epochs of minibatch updates on a single sampled trajectory. The clipped surrogate objective is:
\begin{align*}
  \deriv[L(\theta)^{\text{PPO}}]{\theta}  = \expectation\left[\deriv[]{\theta} \min( \rho_t(\theta) A^{\pi}(S_t, A_t), \text{clip}(\rho_t(\theta), 1 - \epsilon, 1+ \epsilon) A^{\pi}(S_t, A_t))\right]
\end{align*}
where $\rho_t(\theta) = \frac{\pi(A_t | S_t)}{\pi_{{old}}(A_t | S_t)}$ is the importance sampling ratio. The authors use the Generalized Advantage Estimation (\cite{SchulmanMLJA15}) to calculate the advantage function $A^{\pi}(S_t, A_t)$.
\vspace*{-8pt}

\subsection{Option-Critic}
\vspace*{-8pt}

The option-critic architecture (\cite{Bacon2017}) is a gradient-based approach for learning intra-option policies as well termination conditions, assuming that all options are available at every state. Moreover, the parameters of the intra-option policies ($\theta_{\pi}$) and the termination function ($\theta_{\beta}$)  are assumed to be independent. The intra-option policy gradient is as follows:
\begin{align*}
    \deriv[L(\theta)]{\theta_\pi} =  \expectation\left[ \deriv[\log \pi\left(\Action_t \given \State_t,\Option_t\right)]{\theta_\pi} Q_\pi(\State_t,\Option_t, \Action_t)\right] \enspace 
\end{align*}
where a baseline (i.e. the above state-option value function $Q_\pi $  parametrized by $\theta_w$) is generally added. However, if the options are learned to optimize returns, in the long run, they will tend to disappear since any MDP can be solved optimally using primitive actions. To avoid this problem, \cite{Harb2018} use the bounded rationality framework (\cite{Simon1969}) and introduce a deliberation cost ($\eta$), interpreted as  a margin of how much better an option should be than the current option in order to replace it. The termination gradient then takes the following form:
\begin{align*}
    \deriv[L(\theta)]{\theta_\beta} = \expectation\left[ -\deriv[\beta(\State_t, \Option_t)]{\theta_\beta} (A^{\pi \beta}(\State_t,\Option_t) + \eta) \right] 
\end{align*}
where $A^{\pi \beta}(\State_t,\Option_t) = Q_\pi(\State_t,\Option_t) - V_\pi(\State_t)$ is the termination advantage function and stems directly from the derivation of the gradient. 
\vspace*{-8pt}

\section{Algorithm}
\vspace*{-8pt}

We introduce the Proximal Policy Option-Critic (PPOC) algorithm which, just like PPO, works in two stages. In the first stage, the agent collects trajectories of different options and computes the advantage functions using Monte-Carlo returns. We then proceed to the optimization stage where, for K optimizer iterations, we choose M tuples and apply the gradients. We also chose to use a stochastic policy over options, parameterized by an independent vector $\theta_{\mu}$ (as opposed to $\epsilon$-greedy) which we learned under the same policy gradient approach. 

\begin{algorithm}[ht]
\DontPrintSemicolon
\SetAlgoLined
\For{iteration=1,2,....}{
    $c_t \leftarrow 0$\;
    $s_t \leftarrow s_0$\;
    Choose $\option_t$ with a softmax policy over options $\mu(\option_t|s_t)$ \\
    \Repeat{T timesteps} {
        Choose $a_t$ according to $\pi(a_t|s_t)$ \;
        Take action $a_t$ in $s_t$, observe $s_{t+1}$, $r_t$\;
        $\hat r_t = r_t - c_t$\;
        \uIf{$\beta$ terminates in $s_{t+1}$}{
        choose new $\option_{t+1}$ according to softmax $\mu(\option_{t+1}|s_{t+1})$\;
        $c_t = \eta$}
        \Else{$c_t = 0$}
    }
    
    Compute the advantage estimates for each timestep; \\
    \For{$\option$=$\option_1$,$\option_2$,....}{
        $\theta_{old} \leftarrow \theta$\;
        \For{K optimizer iterations with minibatches M}{
            $\theta_{\pi}  \leftarrow \theta_{\pi} + \alpha_{\theta_{\pi}} \deriv[L_t(\theta)^{\text{PPO}}]{\theta_\pi} $ \;
            
            $\theta_\beta \leftarrow \theta_\beta - \alpha_{\theta_\beta} \deriv[\beta(s_t)]{\theta_\beta} \left( A(s_t,o_t) + \eta \right)\;$\;
            
            $\theta_{\mu}  \leftarrow \theta_{\mu} + \alpha_{\theta_{\mu}} \deriv[\log\mu(\option_t|s_t)]{\theta_\mu}  A(s_t,\opt_t) $ \;
            
            $\theta_w \leftarrow \theta_w - \alpha_{\theta_w} \deriv[(G_t - Q_\pi(s_t,\option_t))^2]{\theta_w}$\;
            
        }
    }
    
}

%where $ L_t(\theta_\omega)$ = min$( \frac{\pi_{\theta_{\omega}}(a_t | s_t)}{\pi_{\theta_{\omega_{old}}}(a_t | s_t)} \hat A_t$  , clip($\frac{\pi_{\theta_{\omega}}(a_t | s_t)}{\pi_{\theta_{\omega_{old}}}(a_t | s_t)}, 1 - \epsilon, 1+ \epsilon)\hat A_t) $ \;
\caption{Proximal Policy Option Critic (PPOC)}
\end{algorithm}
\section{Experiments}
We performed experiments on locomotion tasks available on OpenAI's Gym (\cite{BrockmanCPSSTZ16}) using the Mujoco simulator (\cite{Todorov2012MuJoCoAP}).  We aim to assess the following: (1) whether the use of options can increase the speed of learning as well as the final performance, (2) the interpretability of the resulting options.

In our experiments, we used as input the vectors defining joint angles, joint velocities, and coordinates of the center of mass. We used two separate networks with 64 hidden units per layer, each containing two layers.\footnote{The code, as well as the values for the hyperparameters, are available here: \url{https://github.com/mklissa/PPOC}} For all the layers we used tanh non-linearity, except for the output which was linear for of value functions and intra-option policies, sigmoid for the termination probability and softmax for the policy over options. The first network was used to output the policy over options $\mu(\option | s)$ and the intra-option policies $\pi(a | s)$, while the second network was used to output the value functions $Q_\pi(s,\option)$ and the termination probabilities $\beta(s)$. The log-standard deviations were parameterized by a vector independent of the input state. 
We used the exact same hyper-parameters as mentioned in \cite{SchulmanWDRK17}, except for the optimizer mini-batch size which was divided by the number of options. We proceeded so in order to avoid training more samples per iteration with the options framework, thus enabling a fair comparison between options and primitive actions. In the case of options, we also divide the reward by 10 to reduce the scale of the value functions, and therefore the termination probability gradient, making it more stable.  We didn't proceed to any hyper-parameters search to improve the results. Our experiments exclusively investigate the merits of using two options and compare the results to the case of primitive actions (no options).

In addition to the classic Mujoco environments, we ran agents in an environment called HopperIceBlock-v0.\footnote{HopperIceBlock-v0 is based on \cite{henderson2017multitask} and is avaialable here: \url{https://github.com/mklissa/gym-extensions}} This environment contained a more explicit compositionality than the original Mujoco environments available on OpenAI's Gym, which simply require  to learn a gait that maximizes speed in a direction. We used Hopper-v1 as a starting point and added some obstacles in the agent's path: solid slippery blocks. The agent had to learn to pass them either by jumping completely over them or by sliding on their surface.

\begin{figure*}[ht]
  \centering

  \includegraphics[width=\textwidth]{figs/score3.png}
  \caption{Results in Mujocco using 12 different random seeds for a total of 1 million steps (each iteration is 2000 steps)}
\end{figure*}




The results are summarized in Fig.1. As expected, using options with a deliberation cost yields better results and faster learning on most environments. It is interesting to note that the increase in performance is not directly proportional to the value of $\eta$. This is due to the different scales of the average returns across  environments, as well as during the course of learning. In the current formulation of the deliberation cost, its value is a hyperparameter that has to be set. It would be useful to explore the possibility of working with a learned value instead of a constant. This is left as future work.

The results that stand out the most are the one on the customized environment. More importantly, the success threshold for the environment is around 1200 points, under that level the agent actually doesn't learn to pass the iceblock and continue its gait.  So, the agent using options is the only one solving this environment. This also led us to investigate how the options are used in this environment as opposed to the classic Mujoco environments.\footnote{Videos from the environments are available at \url{https://www.youtube.com/watch?v=XI_txkRnKjU}} In HopperIceBlock-v0, the interpretability of the options is obvious and greatly helps the performance: one option is used to hop when there is no iceblock nearby, but then when passing over the iceblock, both options are used to complete the specific task. 
% and the other one is used to pass over, or slide on, the iceblock. 
In the case of the classic Mujoco environments, one option is used to gain momentum at the start of the episode and is never used thereafter. Even if the agents using options outperform the agents with primitive actions on classic environments, we can only truly see the benefits of a hierarchical framework when used in the appropriate environment. 

\section{Conclusion}

Our experiments  demonstrate that it is possible to learn options in an end-to-end manner using deep networks on continuous actions environments, and to the best of our knowledge this is the first work to do so. Our results also suggest that the increase in performance is not directly linked to the deliberation cost, which is problematic as it leaves us with the task of finding the right value. For the options framework to be truly end-to-end it would be necessary to learn a value of $\eta$.
More importantly, we have seen that the increase in performance is related to the compositionality of the environment. In the classic Mujoco environments, using options is not as beneficial as using them in a customized environment with a more obvious division in the state-space. This leads to the following question: when should we be using options? This question also points to a fundamental problem in the current options framework: it is necessary for us to manually specify the number of available options. How should one decide on this number? As stated in \cite{Bacon2017}, one way to answer this question would be to reintroduce the notion of initiation sets in the option-critic architecture.

\bibliographystyle{named}
\bibliography{references}
% \documentclass{article}

\pdfoutput = 1

% if you need to pass options to natbib, use, e.g.:
% \PassOptionsToPackage{numbers, compress}{natbib}
% before loading nips_2017
%
% to avoid loading the natbib package, add option nonatbib:
% \usepackage[nonatbib]{nips_2017}

\usepackage[final]{nips_2017}

% to compile a camera-ready version, add the [final] option, e.g.:
% \usepackage[final]{nips_2017}

\usepackage[utf8]{inputenc} % allow utf-8 input
\usepackage[T1]{fontenc}    % use 8-bit T1 fonts
\usepackage{hyperref}       % hyperlinks
\usepackage{url}            % simple URL typesetting
\usepackage{booktabs}       % professional-quality tables
\usepackage{amsfonts}       % blackboard math symbols
\usepackage{nicefrac}       % compact symbols for 1/2, etc.
\usepackage{microtype}      % microtypography
\usepackage{amsmath}
\usepackage{svg}
\usepackage{graphicx}
\usepackage{bbm}
\usepackage{array}
\usepackage{subfig}

\title{Label Efficient Learning of Transferable Representations across Domains and Tasks}

% \newcommand{\jh}[1]{\textcolor{green}{JH: #1}}
% \newcommand{\yl}[1]{\textcolor{red}{YL: #1}}
% \newcommand{\alan}[1]{\textcolor{blue}{Alan: #1}}

% The \author macro works with any number of authors. There are two
% commands used to separate the names and addresses of multiple
% authors: \And and \AND.
%
% Using \And between authors leaves it to LaTeX to determine where to
% break the lines. Using \AND forces a line break at that point. So,
% if LaTeX puts 3 of 4 authors names on the first line, and the last
% on the second line, try using \AND instead of \And before the third
% author name.

\author{
  Zelun Luo \\
  Stanford University \\
  \texttt{zelunluo@stanford.edu} \\
  \And
  Yuliang Zou \\
  Virginia Tech \\
  \texttt{ylzou@vt.edu} \\
  \AND
  \hspace{0.8cm} Judy Hoffman \\
  \hspace{0.5cm} University of California, Berkeley \\
  \hspace{0.5cm} \texttt{jhoffman@eecs.berkeley.edu} \\
  \And
  \hspace{-0.4cm} Li Fei-Fei \\
  \hspace{-0.4cm} Stanford University \\
  \hspace{-0.4cm} \texttt{feifeili@cs.stanford.edu} \\
}

\begin{document}
% \nipsfinalcopy is no longer used

\maketitle

\begin{abstract}
We propose a framework that learns a representation transferable across different domains and tasks in a label efficient manner. Our approach battles domain shift with a domain adversarial loss, and generalizes the embedding to novel task using a metric learning-based approach. Our model is simultaneously optimized on labeled source data and unlabeled or sparsely labeled data in the target domain. Our method shows compelling results on novel classes within a new domain even when only a few labeled examples per class are available, outperforming the prevalent fine-tuning approach. In addition, we demonstrate the effectiveness of our framework on the transfer learning task from image object recognition to video action recognition. 
\end{abstract}

% Introduction
\section{Introduction}
Reinforcement learning has achieved great success in areas such as Game-playing \citep{silver2018general,vinyals2019grandmaster}, robotics \cite{kober2013reinforcement}, large language models \citep{ouyang2022training}, etc.
However, due to safety concerns or physical limitations, in some real-world reinforcement learning problems, we must consider additional constraints that may influence the optimal policy and the learning process \citep{garcia2015comprehensive}.
% For example, a robotic arm must not take actions that may cause harm to itself or the environments.
A standard framework to handle such cases is the constrained Markov Decision Process (CMDP) \citep{altman1999constrained}.
Within the CMDP framework, the agent has to maximize
the expected cumulative reward while
obeying a finite number of constraints, which are usually in the form of expected cumulative cost criteria.

However, we are sometimes concerned with the problem with a continuum of constraints.
For example,
the constraints we meet might be time-evolving or subject to uncertain parameters, which
cannot be formulated as an ordinary CMDP
(see Examples \ref{Example_Time_Evolving} and  \ref{Example_Uncertain}).
In this paper we would study a generalized CMDP  
to address the above problem.  Because the constraints are not only infinite-number but also lie
in a continuous set,
the generalization is not trivial. Fortunately, we find that we can borrow the idea behind semi-infinite programming (SIP) \citep{remez1934determination, hettich1993semi} to deal with the semi-infinite constraints.
Accordingly, we propose \emph{semi-infinitely constrained Markov decision processes} (SICMDPs)
as a novel complement to the ordinary CMDP framework.
%More specifically,  an SICMDP model %, we consider 
%contains a continuum of constraints whereas an ordinary CMDP contains a finite number of constraints. 

%This generalization is natural but not trivial. However, we can brows the idea  
%The idea is quite natural and can be backtracked
%to the practice of extending linear programming to linear semi-infinite programming (LSIP) %\cite{remez1934determination, GobernaLSIO1998}.
%In addition, 
%As a complementary approach to the ordinary CMDP framework, 
%SICMDP can be used to model these problems  which cannot be described by a finite number of constraints
%that are not covered by .
%For example,
%the restrictions we consider can be time-evolving or subject to uncertain parameters
%, thus
%cannot be described by a finite number of constraints but a continuum of constraints 
%(see Examples \ref{Example_Time_Evolving} and  \ref{Example_Uncertain}).

We also present two reinforcement learning algorithms to solve SICMDPs called SI-CRL and SI-CPO, respectively.
SI-CRL is a model-based reinforcement learning algorithm designed for tabular cases, and SI-CPO is a policy optimization algorithm for non-tabular cases.
% and analyze its performance both theoretically and empirically.
The main challenge is that we need to deal with a continuum of constraints, thus reinforcement learning algorithms for ordinary CMDPs do not work anymore.
In SI-CRL, we tackle this difficulty by first transforming the reinforcement learning problem to an equivalent LSIP problem, which can then be solved using methods in the LSIP literature like the dual exchange methods \citep{Hu1990,reemtsen1998numerical}.
In SI-CPO, we resort to the idea of cooperative stochastic approximation developed in \cite{lan2020algorithms, wei2020comirror}.
As far as we know, we are the first to introduce tools from semi-infinitely programming (SIP) into the reinforcement learning community for solving constrained reinforcement learning problems.

% To the best of our knowledge, we are the first to apply tools from semi-infinitely programming (SIP) to solve reinforcement learning problems.
Furthermore, we give theoretical analysis for both SI-CRL and SI-CPO.
We decompose the error of SI-CRL into two parts: the statistical error from approximating the true SICMDP with an offline dataset and the optimization error due to the fact that the solution of the LSIP problem obtained by the dual exchange method is inexact.
On the optimization side, we show that the iteration complexity of SI-CRL is $O\left(\left\{\mathrm{diam}(Y)L\sqrt{|\gS|^2|\gA|m}/\left[(1-\gamma)\epsilon\right]\right\}^m\right)$.
On the statistical side, we show that the sample complexity of SI-CRL is $\widetilde O\left(\frac{|S|^2|A|^2}{\epsilon^2(1-\gamma)^3}\right)$ if the offline dataset is generated by a generative model, and $\widetilde O\left(\frac{|S||A|}{\nu_{\min} \epsilon^2(1-\gamma)^3}\right)$ if the dataset is generated by a probability measure $\nu$ as considered in \cite{chen2019information}.
Here $\widetilde O$ means that all logarithm terms are discarded.
For SI-CPO, things become a little more complicated because other than the statistical error and the optimization error, we also need to consider the function approximation error, which comes from imperfect policy parametrizations.
It is shown if the function approximation error can be controlled to $O(\epsilon)$ order, the iteration complexity of SI-CPO is $\widetilde{O}\left(\frac{1}{\epsilon^2(1-\gamma)^6}\right)$ and the sample complexity of SI-CPO is $\widetilde{O}(\frac{1}{\epsilon^4(1-\gamma)^{10}})$.
Here our iteration complexity bound is equivalent to a typical $\widetilde O(1/\sqrt{T})$ global convergence rate.

We perform a set of numerical experiments to illustrate the SICMDP model and validate our proposed algorithms.
Specifically, we examine two numerical examples, namely the discharge of sewage and ship route planning.
Through the discharge of sewage example, we show the advantage of the SICMDP framework over the CMDP baseline obtained by naive discretization in modeling realistic sequential decision-making problems.
Moreover, we demonstrate the effectiveness of the SI-CRL and SI-CPO algorithms in such tabular environments. 
In the ship route planning example, we illustrate the benefits of the SICMDP framework and the ability of the SI-CPO algorithm to address complex continuous control tasks involving continuous state spaces with modern deep reinforcement learning techniques.

% In summary, our contributions are listed as follows.
% First, we present the SICMDP model, which can be viewed as a generalization of the ordinary CMDP model.
% Second, we propose an algorithm to perform reinforcement learning for SICMDPs, which is called SI-CRL, and we believe that we are the first to apply tools from SIP
% to solve reinforcement learning problems.
% Third, we give a theoretical analysis of SI-CRL and identify both its sample complexity and iteration complexity.
% In addition, we perform numerical experiments to illustrate the SICMDP model and validate the SI-CRL algorithm.
% \{This paragraph can be removed!!! \}







% Related Work
\section{Related work}
\textbf{Related work}:
% Object detection related datasets/algo in non-medical domain
% Locally labeled CXR dataset
A few CXR datasets have localized abnormality annotations \cite{shih2019augmenting,filice2020crowdsourcing,jaeger2014two} that are curated manually. These are high quality gold standard ground truth datasets but tend to be smaller in scale (< 30,000 images) and have a narrow coverage, with typically only 1-2 labels. In addition, since most labeling efforts only have abnormality semantics attached, no direct relationships with the affected anatomical locations are available. 

%MEHDI: repeated concepts from above. I am removing the following: 

%The lack of anatomic semantics in the annotation is a limitation for complex multi-modal clinical reasoning work, e.g., differential diagnosis, since clinicians often integrate information along anatomical lines, and for downstream report generation tasks, which often requires describing not only the abnormality but also correctly communicate the location of the abnormalities (and medical devices) to the receiving clinicians. 

Two recent CXR datasets have labels for anatomies described in the reports. In \cite{datta2020dataset}, a small manually annotated dataset (2000 reports) included 10 abnormalities that are individually associated with 29 unique spatial locations (anatomies) at the report level. Another CXR dataset has automatically extracted abnormality and anatomy labels as disconnected concepts that are only correlated at the study level from  160,000 reports using a supervised NLP algorithm \cite{bustos2020padchest}. This was trained on a smaller set of manually annotated data. Neither datasets contain localized annotations for the associated CXR images, nor any comparison relation annotations between sequential exams, both of which are available in the Chest ImaGenome dataset. In Table \ref{tab:related}, we present a comparison of our Chest ImagGenome dataset with other datasets available in the literature.

% Table -- Kashyap

% MEdical imaging datasets to go here: Discussed that we will only focus on cxr datasets that are available for this paper. 
% \caption{\color{red} Kashyap, feel free to continue with the table. We should remove the questionmarks and add a line for our dataset (since all others are not graph). For longer text, using abbreviations and explaining them in the caption often works better. If fill in the values is not possible, it is better to remove the table altogether.}


\begin{table}[t!]
\caption{Summary of existing chest X-ray datasets}
\resizebox{\textwidth}{!}{%
\begin{tabular}{@{}lllllllll@{}}
\toprule
\textbf{Dataset} & \textbf{Annotation Level} & \textbf{Annotation Method} & \textbf{Num Labels} & \textbf{Anatomy Labeled} & \textbf{Graph} & \textbf{Dataset Size} & \textbf{Temporal Labels} & \textbf{Reports} \\ \midrule
SIIM-ACR Pneumothorax Segmentation \cite{filice2020crowdsourcing} & Segmentation & Manual + augmented & 1 & No & No & 12,047 & No & No \\
RSNA Pneumonia Detection Challenge   \cite{shih2019augmenting} & Bounding Boxes & Manual & 1 & No & No & 30,000 & No & No \\
Indiana University Chest X-ray collection \cite{demner2016preparing} & Global & Automated & 10 & No & No & 3,813 & No & Yes \\
NIH CXR dataset \cite{wang2017chestx} & Global & Automated & 14 & No & No & 112,120 & No & No \\
PLCO \cite{team2000prostate} & Global & Automated & 24 & Yes & No & 236,000 & Yes & No \\
Stanford CheXpert \cite{irvin2019chexpert} & Global & Automated & 14 & No & No & 224,316 & No & No \\
MIMIC-CXR \cite{johnson2019mimic} & Global & Automated & 14 & No & No & 377,110 & No & Yes \\
Dutta \cite{datta2020dataset} & Global & Manual & 10 & Yes & Yes & 2,000 & No & Yes \\
PadChest \cite{bustos2020padchest} & Global & Manual + automated & 297 & Yes & No & 160,868 & No & Yes \\
Montgomery County Chest X-ray   \cite{jaeger2014two} & Segmentation & Manual & 1 & Yes & No & 138 & No & No \\
Shenzen Hospital Chest X-ray   \cite{jaeger2014two} & Segmentation & Manual & 1 & Yes & No & 662 & No & No \\  \hline \hline
\textbf{Chest ImaGenome} & Bounding Boxes & Automated & 131 & Yes & Yes & 242,072 & Yes & Yes \\
\bottomrule
\end{tabular}%
}
\label{tab:related}
\vspace{-0.4cm}
\end{table}
% removed (Derived from MIMIC-CXR \cite{johnson2019mimic}) % makes table really small


% Model
\section{Method}
The proposed segmentation-by-detection framework, as depicted in Figure \ref{fig:framework}, consists of a detection module and a segmentation module.
In detection stage, 2D slices (layered box) from the input volume are fed to the RPN. Based on the region proposals obtained from RPN, an attention model (block in orange) is formed. The input volume as well as the attention model are further processed in segmentation stage to get the refined anatomical segmentation. 
\vspace{1em} 

\begin{figure}[t]
\centering
\includegraphics[width=0.95\linewidth]{fig/framework.pdf}
\caption{Schematic representation of the segmentation-by-detection framework. The left part is the detection module while the segmentation module is followed on the right. The blue block denotes the input volume which is 3D ultrasound scan of femoral head. The output segmentation is in red.}
\label{fig:framework}
\end{figure}
% dana could you improve the figure. we can try to think together of better ways 

\noindent\textbf{Detection Module:} 
% dana : here you have to make the clarification that you have ground truth on the boxes (in implementation part)
The detection module follows an RPN architecture, a fully convolutional network which takes image slice as input and outputs object region candidates. 
We use the VGG-16 model as the backbone \cite{simonyan2014very} to learn convolutional features and an $3 \times 3$ spatial window to generate region proposals. At each sliding-window location, 9 anchors are predicted associated with different scales and aspect ratios. The last layer consists of a box-regression (reg) layer and a box-classification (cls) layer in parallel. The reg layer outputs 4 regression offsets, $ t = (t_x,t_y,t_w,t_h)$, denoting a scale-invariant translation as well as log-space height and width shift, where $x,y,w$ and $h$ specify two coordinates of the box center, width and height. The cls layer outputs two scores by softmax, related to probabilities of object and background for each proposal. We assign a positive label (of being object) to candidate which has an Intersection-over-Union (IoU) ratio higher than 0.7 with ground truth box. Note that an image slice may contain multiple object regions or none. 

The loss function of RPN follows the multi-task loss \cite{ren2015faster} which is defined as $L = L_{reg} + L_{cls}$. The regression loss, $L_{reg} = -\log p_{obj}$ is log loss and the classification loss,
\begin{equation} \label{eq:loss}
L_{cls} = \sum_{i \in \{x,y,w,h\}} smooth_{L_1} (t_i - t_i^*)
\end{equation}
is smooth $L_1$ loss where $t_i^*$ denotes the ground truth box for the target object. 
\vspace{1em}

\noindent\textbf{Segmentation Module:}
3D U-Net \cite{cciccek20163d} is utilized in the segmentation module as its outstanding performance in medical image segmentation. The u-shaped architecture consists of two paths: a contracting path, where each layer contains two $3\times3\times3$ convolutions followed by a rectified linear unit (ReLU) and then a max pooling, provides high resolution features. While, the symmetric expanding path for semantically richer features replaces max pooling with a upconvolution $2\times2\times2$ with stride of 2 in each dimension, and then two $3\times3\times3$ convolutions each followed by a ReLU. Skip connections between layers of equal resolution in the contracting path and the expanding path enables context information as well as precise localization.

Different from 3D U-Net, to incorporate the attention model detected by the RPN, our architecture takes as input both the volumetric image data and the candidate RoIs proposed by the RPN, concatenated as 3D volume. 
% dana not sure what you like to say below
% densely annotated
The attention model makes the network to focus on the potential RoIs and can reduce the interference of the surrounding noise.
The anatomical segmentation is then generated from a $1\times1\times1$ convolution which reduces the number of feature maps to the number of labels.  The energy function is computed by a pixel-wise softmax combined with the cross entropy loss.
% dana equation ??

\subsection{System and implementation Details}
The segmentation-by-detection approach adopts a cascade structure with two stages: detection and segmentation. The two networks are trained separately in an end-to-end manner. All the new layers are randomly initialized from zero-mean Gaussian distribution with standard deviations 0.01. Biases are initialized to 0. We use Caffe \cite{jia2014caffe} for the implementation and an NVIDIA Titan X GPU for training.

In the detection stage, we initialize the VGG-16 model by the pre-trained model for ImageNet classification \cite{russakovsky2015imagenet} and further fine-tune the model for our detection task. The input fed to the network are image slices with a fixed size of $184\times96$ and the corresponding ground truth boxes are generated from the annotation in the format of tight bounding boxes surrounding the segmentation contour (as illustrated in Figure \ref{fig:hip} (b), the boundary of white area). To optimize the energy function, stochastic gradient descent (SGD) is used. The global learning rate is set to 0.001, while a momentum of 0.9 and a weight decay of 0.0005 are used. The batch size is set to 256 and each mini-batch only contains the positive anchors for training. The region proposals are obtained from the reg path for each image slice. The attention model is then formed by concatenating all the detected regions, as binary masks, into a volume.

In the segmentation stage, we use the Adam optimizer \cite{kingma2014adam} to learn the network parameters. A global learning rate is set to 0.001 while the two momentum coefficients are set to 0.9 and 0.999 respectively. A batch size of 1 is used due to the memory constraints of the GPU. The network takes the volume data as well as the attention model as input. We train the network for a maximum of 30K iterations and reserve the learned weights with the best performance from every 1K iterations. 
\vspace{1em}

\noindent\textbf{Inference:}
At test time, the 2D slices from an input volume are first fed to the detection module. The attention model is obtained based on the output. Then the volume data as well as the attention model are fed to the segmentation module to get the pixel-wise prediction.




% Experiment
\section{Experiment}
\newcommand{\twomoons}{{\tt Twomoons}}
\newcommand{\gauss}{{\tt Gauss}}
\newcommand{\sculpture}{{\tt Sculpture}}
\newcommand{\baseline}{{\tt Baseline}}
\newcommand{\MM}{{\tt MsgPassing}}
\newcommand{\blackboard}{{\tt Blackboard}}
\newcommand{\ncut}{\text{ncut}}
\newcommand{\chensays}[2][]{\textcolor{blue} {\textsc{Jiecao #1:} \emph{#2}}}

\section{Experiments}
In this section we present experimental results for  graph clustering in the message passing and blackboard models. We will compare the following three algorithms. (1) \baseline: each site sends all the data to the coordinator directly; (2) \MM: our algorithm in the message passing model (Section~\ref{sec:gcmessage}); (3) 
\blackboard: our algorithm in  the blackboard model (Section~\ref{sec:bb}).


%Since both of our algorithms are crucially based on the use of spectral scarification, our main focus in the experiments is to investigate to what extend the quality of the spectral clustering algorithms will be affected by using spectral sparsification, the saving of communication costs by using spectral sparsificaion, ...
%
%
%The goal of this experiment is not to demonstrate the effectiveness of the spectral clustering algorithm. We mainly want to investigate the following, 
%\begin{itemize}
%\item to what extend the quality of clustered results will be affected by using spectral sparsification.
%\item saving of communication costs by using spectral sparsifier.
%\item the affect of constants in algorithms of the message passing/blackboard model.
%\end{itemize}
%
%
%\subsection{The Setup}
%\paragraph{Reference Algorithms}
%We compare different algorithms in our experiment.

%Note that we can also run \MM~ in the blackboard model.

Besides giving the visualized results of these algorithms on various datasets, we also measure the qualities of the results via the {\em normalized cut}, defined as 
\[
\ncut(A_1, \ldots, A_{k}) = \frac{1}{2}\sum_{i\in[k]}\frac{w(A_i, V\backslash A_i)}{\vol(A_i)},
\]
 which is a standard objective function to be minimized for spectral clustering algorithms. 
%We will compare the communication costs of these algorithms in different settings.

%We also compare the total communication costs of different algorithms/models. As the unit does not matter in our case, we normalize all communication costs by the cost of \baseline.  Whenever possible, we will visualize the clustered results.

We implemented the algorithms using multiple languages, including Matlab, Python and C++. Our experiments were conducted on an IBM NeXtScale nx360 M4 server, which is equipped with 2 Intel Xeon E5-2652 v2 8-core processors, 32GB RAM and 250GB local storage.


\subsection{Datasets.}
We test the algorithms in the following real and synthetic datasets, which is visualized in \figref{visualization}.


\begin{figure}[h]
     \centering
     \subfigure[\twomoons]{\includegraphics[width=0.23\textwidth]{twomoons-14000-original.png}\label{fig:twomoons}}
     ~~
     \subfigure[\gauss]{\includegraphics[width=0.23\textwidth]{gauss-10000-original.png}\label{fig:gauss}}
     ~~
     \subfigure[\sculpture]{\includegraphics[width=0.13\textwidth,height=0.16\textwidth]{sculpture-11680-original.jpg}\label{fig:sculpture}}
     \caption{Visualization of the datasets for our experiments.}
     \label{fig:visualization}
\end{figure}



\vspace{-1mm}
\begin{itemize}
\item \twomoons : this dataset contains $n=14,000$ coordinates in $\mathbb{R}^2$. We consider each point to be a vertex. For any two vertices $u, v$, we add an edge with weight $w(u,v) = \exp\{-\|u-v\|_2^2/\sigma^2\}$ with $\sigma = 0.1$ when one vertex is among the $7000$-nearest points of the other.  This construction results in a graph with about $110,000,000$ edges.

\item  \gauss : this dataset contains $n = 10,000$ points in $\mathbb{R}^2$. There are $4$ clusters in this dataset, each generated using a Gaussian distribution. We construct a complete graph as the similarity graph.  For any two vertices $u, v$, we define the weight $w(u,v) = \exp\{-\|u-v\|_2^2/\sigma^2\}$ with $\sigma = 1$. The resulting graph has about $100,000,000$ edges.

\item \sculpture : a photo of \textit{The Greek Slave}~\footnote{Available in e.g., \url{http://artgallery.yale.edu/collections/objects/14794}}. We use an $80\times 150$ version of this photo where each pixel is viewed as a vertex. To construct a similarity graph, we map each pixel to a point in $\mathbb{R}^5$, i.e., $(x, y, r, g, b)$, where the latter three coordinates are the RGB values. For any two vertices $u, v$, we  put an edge between $u, v$ with weight $w(u,v) = \exp\{-\|u-v\|_2^2/\sigma^2\}$ with $\sigma = 0.5$ if one of $u, v$ is among the $5000$-nearest points of the other. This results in a graph with about $70,000,000$ edges.
\end{itemize}
\vspace{-1mm}
In the distributed model edges are randomly partitioned across $s$ sites. 

%\vspace{-1.5mm}



\subsection{Results on clustering quality}
%{\em Quality.} \
\begin{figure*}[ht]
     \centering
     \subfigure[\baseline]{\includegraphics[width=0.2\textwidth]{twomoons-14000-original-clustered.png}\label{fig:twomoons-clustered-original}}
     \subfigure[\MM]{\includegraphics[width=0.2\textwidth]{twomoons-14000-sparsify-clustered-15.png}\label{fig:twomoons-clustered-sparsify}}
     \subfigure[\blackboard]{\includegraphics[width=0.2\textwidth]{twomoons-14000-chain-clustered.png}\label{fig:twomoons-clustered-chain}}
     \caption*{\twomoons, $k = 2$;}

\subfigure[\baseline]{\includegraphics[width=0.2\textwidth]{gauss-10000-original-clustered.png}\label{fig:gauss-clustered-original}}
     \subfigure[\MM]{\includegraphics[width=0.2\textwidth]{gauss-10000-sparsify-clustered-15.png}\label{fig:gauss-clustered-sparsify}}
     \subfigure[\blackboard]{\includegraphics[width=0.2\textwidth]{gauss-10000-chain-clustered.png}\label{fig:gauss-clustered-chain}}
     \caption*{\gauss, $k = 4$}


     \subfigure[\baseline]{\includegraphics[width=0.2\textwidth,height=0.2\textwidth]{sculpture-11680-original-clustered.png}\label{fig:sculpture-clustered-original}}  
     \subfigure[\MM]{\includegraphics[width=0.2\textwidth,height=0.2\textwidth]{sculpture-11680-sparsify-clustered-15.png}\label{fig:sculpture-clustered-sparsify}}
     \subfigure[\blackboard]{\includegraphics[width=0.2\textwidth,height=0.2\textwidth]{sculpture-11680-chain-clustered.png}\label{fig:sculpture-clustered-chain}}
     \caption*{\sculpture, $k = 3$. }


     
     \caption{Visualization of the results on \twomoons, \gauss\ and \sculpture. In the message passing model each site samples $5 n$ edges; in the blackboard model all sites jointly sample $10n$ edges (in \twomoons~ and \gauss) or $20n$ edges (in \sculpture) and the chain has length $18$. $s = 15$.}
     \label{fig:quality-1}
\end{figure*}

We visualize the clustered results for 
the \twomoons, \gauss\ and \sculpture\ in Figure~\ref{fig:quality-1}.
% and visualize the clustered results for \gauss\ and \sculpture in Figure~\ref{fig:quality-2}.
It can be seen that \baseline, \MM\ and \blackboard\ give results of very similar qualities.  For simplicity, here we only present the visualization for $s=15$. Similar results were observed when we varied the values of $s$.  
%\he{To Qin: Do you plan to have two titles (Results \& Quality)?}


% \begin{figure*}[h]
%      \centering
% \subfigure[\baseline]{\includegraphics[width=0.3\textwidth]{gauss-10000-original-clustered.png}\label{fig:gauss-clustered-original}}
%      \subfigure[\MM]{\includegraphics[width=0.3\textwidth]{gauss-10000-sparsify-clustered-15.png}\label{fig:gauss-clustered-sparsify}}
%      \subfigure[\blackboard]{\includegraphics[width=0.3\textwidth]{gauss-10000-chain-clustered.png}\label{fig:gauss-clustered-chain}}
%      \caption*{\gauss, $k = 4$}


%      \subfigure[\baseline]{\includegraphics[width=0.2\textwidth]{sculpture-11680-original-clustered.png}\label{fig:sculpture-clustered-original}}  
%      \subfigure[\MM]{\includegraphics[width=0.2\textwidth]{sculpture-11680-sparsify-clustered-15.png}\label{fig:sculpture-clustered-sparsify}}
%      \subfigure[\blackboard]{\includegraphics[width=0.2\textwidth]{sculpture-11680-chain-clustered.png}\label{fig:sculpture-clustered-chain}}
%      \caption*{\sculpture, $k = 3$. }

%      \caption{Visualization of results on \gauss\ and \sculpture; in the message passing model each site samples $5 n$ edges; in the blackboard model all sites jointly sample $10n$ (in \gauss) or $20n$ (in \sculpture) edges and the chain has length $18$.}
%      \label{fig:quality-2}
% \end{figure*}


We also compare the normalized cut (ncut) values of the clustering results of different algorithms.  The results are presented in Figure \ref{fig:quality}. In all datasets, the ncut values of different algorithms are very close. The ncut value of \MM\ slightly decreases when we increase the value of $s$, while the ncut value of \blackboard\ is independent of $s$.
%We comment that in general, it is difficult to compare \MM\ and \blackboard\ directly because they are affected by different parameters.


\begin{figure*}[!ht]
  \centering
  \subfigure[\twomoons]{\includegraphics[width=0.33\textwidth]{twomoons-14000-ncut.png}\label{fig:twomoons-quality}}\hspace*{-1.1em}
  \subfigure[\gauss]{\includegraphics[width=0.31\textwidth]{gauss-10000-ncut.png}\label{fig:gauss-quality}}\hspace*{-1.1em}
  \subfigure[\sculpture]{\includegraphics[width=0.31\textwidth]{sculpture-11680-ncut.png}\label{fig:sculpture-quality}}\hspace*{-1.1em}
  \subfigure{\includegraphics[width=0.14\textwidth]{legend.png}}
     \caption{Comparisons on normalized cuts. In the message passing model, each site samples $5n$ edges; in each round of the algorithm in the blackboard model, all sites jointly sample $10n$ edges (in \twomoons~and \gauss) or $20n$ edges (in \sculpture) edges and the chain has length $18$.}
     \label{fig:quality}
\end{figure*}

%\textcolor{red}{To Jiecao: Can you put the color lines indicating baseline, message passing, and blackboard within one row in Pic 2? Withthis we can save some space.}

%\vspace{-1.5mm}

\subsection{Results on communication costs} 
\begin{figure*}[!ht]
     \centering
     \subfigure[\twomoons]{\includegraphics[width=0.3\textwidth]{twomoons-14000-communication.png}\label{fig:twomoons-communication}}
     \subfigure[\gauss]{\includegraphics[width=0.3\textwidth]{gauss-10000-communication.png}\label{fig:gauss-communication}}
     \subfigure[\sculpture]{\includegraphics[width=0.3\textwidth]{sculpture-11680-communication.png}\label{fig:sculpture-communication}}


     \subfigure[\twomoons]{\includegraphics[width=0.32\textwidth]{twomoons-14000-communication-2.png}\label{fig:twomoons-communication-2}}
     \subfigure[\gauss]{\includegraphics[width=0.32\textwidth]{gauss-10000-communication-2.png}\label{fig:gauss-communication-2}}
     \subfigure[\sculpture]{\includegraphics[width=0.32\textwidth]{sculpture-11680-communication-2.png}\label{fig:sculpture-communication-2}}
     \caption{Comparisons on communication costs. In the message passing model, each site samples $5n$ edges; in each round of the algorithm in the blackboard model, all sites jointly sample $10n$ (in \twomoons~and \gauss) or $20n$ (in \sculpture) edges and the chain has length $18$. }
     \label{fig:communication}
\end{figure*}

We compare the communication costs of different algorithms in Figure \ref{fig:communication}. We observe that while achieving similar clustering qualities as \baseline, both \MM\ and \blackboard\ are significantly more communication-efficient (by one or two orders of magnitudes in our experiments). We also notice that the value of $s$ does not affect the communication cost of \blackboard, while the communication cost of \MM\ grows almost linearly with $s$; when $s$ is large, \MM\ uses significantly more communication than \blackboard. These confirm our theory.  %In Figure~\ref{fig:mm-const} and Figure~\ref{fig:blackboard-const}   in Appendix~\ref{sec:parameters} we present how the performance of \MM\ and \blackboard\ are affected by their parameters.

%
%
%\vspace{-1.5mm}
%\paragraph{Summary.}  From our experimental results we conclude that \MM\ and \blackboard\ achieve similar clustering quality as the native algorithm \baseline, while significantly reduce the communication cost.  When the number of sites is large, \blackboard\ is more communication efficient than \MM, as predicted by our theory.



\subsection{Parameters in \MM\ and \blackboard}
\label{sec:parameters}

Figure \ref{fig:mm-const} shows in \MM how the value of ncut is affected by the number of sites and the number of edges sampled in each site. 
Here, each site samples $cn$ edges. 
When $c=3$ and $s=1$, the ncut value diverges in all datasets. This is because with such a small $c$, the algorithm does not generate a valid sparsifier. In general, increasing $c$ or $s$ will slightly decrease the ncut value. But once they are above some thresholds, the ncut values of \MM\ and \baseline\ become very close.

Figure \ref{fig:blackboard-const} shows in \blackboard  how the ncut value is affected by the number of iterations and the number of edges sampled. When the number of iterations is set to be $5$, ncut values diverge in all datasets. This is because we cannot expect to generate a valid sparsifier by using such few iterations. It can be seen from \ref{fig:bb-gauss-constant} that for a fixed $c$, performing more iterations will help to reduce ncut values. From the same figure, one can also conclude that for fixed iterations, increasing $c$ also helps to reduce the ncut values.



\begin{figure*}[h!t]
     \centering
     \subfigure[\twomoons]{\includegraphics[width=0.3\textwidth]{twomoons-c.png}\label{fig:mm-twomoons-constant}}
     \subfigure[\gauss~dataset]{\includegraphics[width=0.3\textwidth]{gauss-c.png}\label{fig:mm-gauss-constant}}
     \subfigure[\sculpture]{\includegraphics[width=0.3\textwidth]{sculpture-c.png}\label{fig:mm-sculpture-constant}}
     \caption{The pictures above show the $\ncut$ values with respect to the values of $c$ and $s$ for the \MM\ algorithm. Here  
 each site samples $c n$ edges.}
     \label{fig:mm-const}
\end{figure*}


\begin{figure*}[h!t]
     \centering
     \subfigure[\twomoons]{\includegraphics[width=0.3\textwidth]{twomoons-iter.png}\label{fig:bb-twomoons-constant}}
     \subfigure[\gauss]{\includegraphics[width=0.3\textwidth]{gauss-iter.png}\label{fig:bb-gauss-constant}}
     \subfigure[\sculpture]{\includegraphics[width=0.3\textwidth]{sculpture-iter.png}\label{fig:bb-sculpture-constant}}
     \caption{The pictures above show how the $\ncut$ values are affected by the number of iterations and the value of $c$ for the \blackboard\ algorithm. Here 
all sites jointly sample $c n$ edges. }
     \label{fig:blackboard-const}
\end{figure*}







% Conclusion
\section{Conclusion}

\begin{comment}
\begin{figure}
\includegraphics[width=\linewidth]{figs/beyond_tss_lesion.pdf}
\caption[]{End-to-End runtime lesion study of the entire MNIST dataset and the FMA featurized music dataset. Each of DROP's contributions provides a runtime improvement.}
\label{fig:beyond_lesion}
\end{figure}
\end{comment}



\section{Conclusion}
\label{sec:conclusion}

Advanced data analytics techniques must scale to rising data volumes. 
DR techniques offer a powerful toolkit when processing these datasets, with PCA frequently outperforming popular techniques in exchange for high computational cost. 
In response, we propose DROP, a new dimensionality reduction optimizer. 
DROP combines progressive sampling, progress estimation, and online aggregation to identify high quality low dimensional bases via PCA without processing the entire dataset by balancing the runtime of downstream tasks and achieved dimensionality. 
Thus, DROP provides a first step in bridging the gap between quality and efficiency in end-to-end DR for downstream \red{analytics}. 

%We revisit canonical operators for time series dimensionality reduction and the measurement study of~\cite{keogh-study}, and show that PCA is more effective than popular alternatives in the data mining literature often by a margin of over $2\times$ on average on gold-standard time series benchmark data sets with respect to output data dimension. More surprisingly, we empirically demonstrate that a small number of samples are sufficient to accurately characterize directions of maximum variance and obtain a high-quality low-dimensional transformation.




% Ack
\section*{Acknowledgement}
We would like to start by thanking
our sponsors: Stanford Computer Science Department and Stanford Program in AI-assisted Care
(PAC). Next, we specially thank De-An Huang, Kenji Hata, Serena Yeung, Ozan Sener and all the members of Stanford Vision and Learning Lab for their insightful discussion and feedback. Lastly, we thank all the anonymous reviewers for their valuable comments.

\clearpage
\bibliography{nips_2017.bib}{}
\bibliographystyle{plain}

% Supp
\clearpage
%%% Network Architecture
\section*{Network Architecture}
%% SVHN -> MNIST
\subsection*{(a) SVHN 0-4 $\rightarrow$ MNIST 5-9}
% Encoder
\begin{table}[htbp]
\centering
\caption{Embedding network structure}
\begin{tabular}{|c|c|c|c|c|}
\hline
name & conv1 & pool1 & conv2 & pool2\\
\hline
\hline
layer type & conv-batchnorm-relu & max pool & conv-batchnorm-relu & max pool\\
kernel & 3$\times$3$\times$64 & 2$\times$2 & 3$\times$3$\times$64 & 2$\times$2\\
stride & 1 & 2 & 1 & 2\\
padding & 1 & 0 & 1 & 0\\
\hline
\hline
name & conv3 & pool3 & conv4 & pool4\\
\hline
\hline
layer type & conv-batchnorm-relu & max pool & conv-batchnorm-relu & max pool\\
kernel & 3$\times$3$\times$64 & 2$\times$2 & 3$\times$3$\times$64 & 2$\times$2\\
stride & 1 & 2 & 1 & 2\\
padding & 1 & 0 & 1 & 0\\
\hline
\hline
name & fc1 & fc2 & {} & {}\\
\hline
\hline
layer type & fc-relu & fc & {} & {}\\
kernel & 64$\times$64 & 64$\times$5 & {} & {}\\
\hline
\end{tabular}
\end{table}

% Discriminator
\begin{table}[htbp]
\centering
\caption{Discriminator structure}
\begin{tabular}{|c|c|c|c|c|c|c|}
\hline
name & fc1 & fc2 & fc3 & fc4 & fc5 & fc6\\
\hline
\hline
layer type & fc-relu & fc-relu & fc-relu & fc-relu & fc-relu & fc\\
kernel & 64$\times$64 & 64$\times$5 & 5$\times$500 & 500$\times$500 & 500$\times$500 & 500$\times$1\\
\hline
\end{tabular}
\end{table}
\clearpage

%% Image -> Video
\subsection*{(b) Image object recognition $\rightarrow$ video action recognition}
% Encoder
\ \ \ \ \ \ Embedding network structure: ResNet-18~\footnote{We refer readers to the PyTorch implementation: \url{https://github.com/pytorch/vision/blob/master/torchvision/models/resnet.py}}.

% Decoder
\begin{table}[htbp]
\centering
\caption{Discriminator structure}
\begin{tabular}{|c|c|c|c|c|}
\hline
name & conv1 &  conv2 & conv3\\
\hline
layer type & conv-batchnorm-leaky relu & conv-batchnorm-leaky relu & conv\\
kernel & 3$\times$3$\times$512  & 3$\times$3$\times$512 & 1$\times$1$\times$1\\
stride & 2 & 1 & 1\\
padding & 1 & 1 & 0\\
\hline
\end{tabular}
\end{table}

\subsection*{(c) Ablation: unsupervised domain adaptation}
% Encoder
\begin{table}[htbp]
\centering
\caption{Embedding network structure}
\begin{tabular}{|c|c|c|c|c|}
\hline
name & conv1 & pool1 & conv2 & pool2\\
\hline
\hline
layer type & conv-relu & max pool & conv-relu & max pool\\
kernel & 5$\times$5$\times$20 & 2$\times$2 & 5$\times$5$\times$20 & 2$\times$2\\
stride & 1 & 2 & 1 & 2\\
padding & 0 & 0 & 0 & 0\\
\hline
\hline
name & fc1 & fc2 & {} & {}\\
\hline
\hline
layer type & fc-relu & fc & {} & {}\\
kernel & 800$\times$500 & 500$\times$10 & {} & {}\\
\hline
\end{tabular}
\end{table}

% Discriminator
\begin{table}[htbp]
\centering
\caption{Discriminator structure}
\begin{tabular}{|c|c|c|c|c|c|}
\hline
name & fc1 & fc2 & fc3 & fc4 & fc5\\
\hline
\hline
layer type & fc-relu & fc-relu & fc-relu & fc-relu & fc\\
kernel & 800$\times$500 & 500$\times$10 & 10$\times$500 & 500$\times$500 & 500$\times$1 \\
\hline
\end{tabular}
\end{table}

\end{document}

\clearpage


\end{document}
