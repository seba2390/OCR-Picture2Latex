\documentclass[12pt]{article}
\renewcommand{\baselinestretch}{1.1}
\usepackage[left=2.7cm,bottom=2.7cm,right=2.7cm,top=2.7cm]{geometry}

%% Packages
\usepackage{enumerate}
\usepackage{tabularx}
\usepackage{subfigure}
\usepackage{multirow}
\usepackage{algorithm}
\usepackage{authblk}
\usepackage{algorithmic}
\usepackage{indentfirst}
\usepackage{setspace}
\usepackage{color}
\usepackage[utf8]{inputenc} % allow utf-8 input
\usepackage[T1]{fontenc}    % use 8-bit T1 fonts
\usepackage{hyperref}       % hyperlinks
\usepackage{url}            % simple URL typesetting
\usepackage{booktabs}       % professional-quality tables
\usepackage{amsfonts}       % blackboard math symbols
\usepackage{nicefrac}       % compact symbols for 1/2, etc.
\usepackage{microtype}      % microtypography
\usepackage{lipsum}		% Can be removed after putting your text content
\usepackage{graphicx}
\usepackage{natbib}
\usepackage{doi}
\usepackage[namelimits]{amsmath} 
\usepackage{amssymb}            
\usepackage{amsfonts}       
\usepackage{mathrsfs} 
\usepackage[resetlabels]{multibib}
\usepackage{bm}

\usepackage{graphicx}
\usepackage{epsfig}
\usepackage{enumitem}
\usepackage{url} % not crucial - just used below for the URL
\usepackage{multirow}
\usepackage{epstopdf}
\usepackage{mathtools}

% Convenient notations
\newcommand{\fullset}{\mathcal{F}}
\newcommand{\informsOR}{1}
\newcommand{\informsMOR}{0}

\graphicspath{{}}

\newtheorem{theorem}{Theorem}
\newtheorem{lemma}{Lemma}
\newtheorem{proof}{Proof}
\newtheorem{corollary}{Corollary}
\newtheorem{remark}{Remark}
\providecommand{\keywords}[1]{\textit{\quad Key words }:  #1}
\renewcommand\Affilfont{\small}
\newcommand{\qedsymbol}{\hfill $\blacksquare$}
\begin{document}

\title{Best-Subset Selection in Generalized Linear Models: A Fast and Consistent Algorithm via Splicing Technique}

% \author{Yanhang Zhang\thanks{School of Statistics,
% Renmin University of China,
% \texttt{zhangyh98@ruc.edu.cn}}\ ~~~~
% Junxian Zhu \thanks{Saw Swee Hock School of Public Health, National University of Singapore,
% \texttt{junxian@nus.edu.sg}}\~~~~
% Jin Zhu \thanks{Southern China Center for Statistical Science,
% Department of Statistical Science,
% School of Mathematics,
% Sun Yat-Sen University,
% \texttt{zhuj37@mail2.sysu.edu.cn} }~~~~
% Xueqin Wang\thanks{
% Department of Statistics and Finance/International Institute of Finance,
% 	School of Management,
% 	University of Science and Technology of China,
% 	\texttt{wangxq20@ustc.edu.cn} }}

\author[1]{Junxian Zhu\textsuperscript{*}}
\author[2]{Jin Zhu\textsuperscript{*}}
\author[4]{Borui Tang}
\author[2]{Xuanyu Chen}
\author[3]{Hongmei Lin\textsuperscript{$\dagger$}}
\author[4]{Xueqin Wang\textsuperscript{$\dagger$}}

\affil[1]{\footnotesize Saw Swee Hock School of Public Health, National University of Singapore}
\affil[2]{\footnotesize Southern China Center for Statistical Science, Department of Statistical Science, School of Mathematics, Sun Yat-Sen University}
\affil[3]{\footnotesize School of Statistics and Information, Shanghai University of International Business and Economics \\
  \texttt{hmlin@suibe.edu.cn}}
\affil[4]{\footnotesize Department of Statistics and Finance/International Institute of Finance, School of Management, University of Science and Technology of China \\
 	\texttt{wangxq20@ustc.edu.cn} }

\date{}
\maketitle \sloppy

\begin{abstract}
  In high-dimensional generalized linear models, it is crucial to identify a sparse model that adequately accounts for response variation. Although the best subset section has been widely regarded as the Holy Grail of problems of this type, achieving either computational efficiency or statistical guarantees is challenging. In this article, we intend to surmount this obstacle by utilizing a fast algorithm to select the best subset with high certainty. We proposed and illustrated an algorithm for best subset recovery in regularity conditions. Under mild conditions, the computational complexity of our algorithm scales polynomially with sample size and dimension. In addition to demonstrating the statistical properties of our method, extensive numerical experiments reveal that it outperforms existing methods for variable selection and coefficient estimation. The runtime analysis shows that our implementation achieves approximately a fourfold speedup compared to popular variable selection toolkits like \textsf{glmnet} and \textsf{ncvreg}.
\end{abstract}


\keywords{Best-Subset Selection, Generalized Linear Models, Splicing Technique, Support Recovery Consistency, Polynomial Complexity}

\begingroup\renewcommand\thefootnote{*}
\footnotetext{Equal contribution}
\begingroup\renewcommand\thefootnote{$\dagger$}
\footnotetext{Corresponding author}

% \leavevmode
% \\
% \\
% \\
% \\
% \\
\section{Introduction}
\label{introduction}

AutoML is the process by which machine learning models are built automatically for a new dataset. Given a dataset, AutoML systems perform a search over valid data transformations and learners, along with hyper-parameter optimization for each learner~\cite{VolcanoML}. Choosing the transformations and learners over which to search is our focus.
A significant number of systems mine from prior runs of pipelines over a set of datasets to choose transformers and learners that are effective with different types of datasets (e.g. \cite{NEURIPS2018_b59a51a3}, \cite{10.14778/3415478.3415542}, \cite{autosklearn}). Thus, they build a database by actually running different pipelines with a diverse set of datasets to estimate the accuracy of potential pipelines. Hence, they can be used to effectively reduce the search space. A new dataset, based on a set of features (meta-features) is then matched to this database to find the most plausible candidates for both learner selection and hyper-parameter tuning. This process of choosing starting points in the search space is called meta-learning for the cold start problem.  

Other meta-learning approaches include mining existing data science code and their associated datasets to learn from human expertise. The AL~\cite{al} system mined existing Kaggle notebooks using dynamic analysis, i.e., actually running the scripts, and showed that such a system has promise.  However, this meta-learning approach does not scale because it is onerous to execute a large number of pipeline scripts on datasets, preprocessing datasets is never trivial, and older scripts cease to run at all as software evolves. It is not surprising that AL therefore performed dynamic analysis on just nine datasets.

Our system, {\sysname}, provides a scalable meta-learning approach to leverage human expertise, using static analysis to mine pipelines from large repositories of scripts. Static analysis has the advantage of scaling to thousands or millions of scripts \cite{graph4code} easily, but lacks the performance data gathered by dynamic analysis. The {\sysname} meta-learning approach guides the learning process by a scalable dataset similarity search, based on dataset embeddings, to find the most similar datasets and the semantics of ML pipelines applied on them.  Many existing systems, such as Auto-Sklearn \cite{autosklearn} and AL \cite{al}, compute a set of meta-features for each dataset. We developed a deep neural network model to generate embeddings at the granularity of a dataset, e.g., a table or CSV file, to capture similarity at the level of an entire dataset rather than relying on a set of meta-features.
 
Because we use static analysis to capture the semantics of the meta-learning process, we have no mechanism to choose the \textbf{best} pipeline from many seen pipelines, unlike the dynamic execution case where one can rely on runtime to choose the best performing pipeline.  Observing that pipelines are basically workflow graphs, we use graph generator neural models to succinctly capture the statically-observed pipelines for a single dataset. In {\sysname}, we formulate learner selection as a graph generation problem to predict optimized pipelines based on pipelines seen in actual notebooks.

%. This formulation enables {\sysname} for effective pruning of the AutoML search space to predict optimized pipelines based on pipelines seen in actual notebooks.}
%We note that increasingly, state-of-the-art performance in AutoML systems is being generated by more complex pipelines such as Directed Acyclic Graphs (DAGs) \cite{piper} rather than the linear pipelines used in earlier systems.  
 
{\sysname} does learner and transformation selection, and hence is a component of an AutoML systems. To evaluate this component, we integrated it into two existing AutoML systems, FLAML \cite{flaml} and Auto-Sklearn \cite{autosklearn}.  
% We evaluate each system with and without {\sysname}.  
We chose FLAML because it does not yet have any meta-learning component for the cold start problem and instead allows user selection of learners and transformers. The authors of FLAML explicitly pointed to the fact that FLAML might benefit from a meta-learning component and pointed to it as a possibility for future work. For FLAML, if mining historical pipelines provides an advantage, we should improve its performance. We also picked Auto-Sklearn as it does have a learner selection component based on meta-features, as described earlier~\cite{autosklearn2}. For Auto-Sklearn, we should at least match performance if our static mining of pipelines can match their extensive database. For context, we also compared {\sysname} with the recent VolcanoML~\cite{VolcanoML}, which provides an efficient decomposition and execution strategy for the AutoML search space. In contrast, {\sysname} prunes the search space using our meta-learning model to perform hyperparameter optimization only for the most promising candidates. 

The contributions of this paper are the following:
\begin{itemize}
    \item Section ~\ref{sec:mining} defines a scalable meta-learning approach based on representation learning of mined ML pipeline semantics and datasets for over 100 datasets and ~11K Python scripts.  
    \newline
    \item Sections~\ref{sec:kgpipGen} formulates AutoML pipeline generation as a graph generation problem. {\sysname} predicts efficiently an optimized ML pipeline for an unseen dataset based on our meta-learning model.  To the best of our knowledge, {\sysname} is the first approach to formulate  AutoML pipeline generation in such a way.
    \newline
    \item Section~\ref{sec:eval} presents a comprehensive evaluation using a large collection of 121 datasets from major AutoML benchmarks and Kaggle. Our experimental results show that {\sysname} outperforms all existing AutoML systems and achieves state-of-the-art results on the majority of these datasets. {\sysname} significantly improves the performance of both FLAML and Auto-Sklearn in classification and regression tasks. We also outperformed AL in 75 out of 77 datasets and VolcanoML in 75  out of 121 datasets, including 44 datasets used only by VolcanoML~\cite{VolcanoML}.  On average, {\sysname} achieves scores that are statistically better than the means of all other systems. 
\end{itemize}


%This approach does not need to apply cleaning or transformation methods to handle different variances among datasets. Moreover, we do not need to deal with complex analysis, such as dynamic code analysis. Thus, our approach proved to be scalable, as discussed in Sections~\ref{sec:mining}.









\section{Proposed Approach} \label{sec:method}

Our goal is to create a unified model that maps task representations (e.g., obtained using task2vec~\cite{achille2019task2vec}) to simulation parameters, which are in turn used to render synthetic pre-training datasets for not only tasks that are seen during training, but also novel tasks.
This is a challenging problem, as the number of possible simulation parameter configurations is combinatorially large, making a brute-force approach infeasible when the number of parameters grows. 

\subsection{Overview} 

\cref{fig:controller-approach} shows an overview of our approach. During training, a batch of ``seen'' tasks is provided as input. Their task2vec vector representations are fed as input to \ours, which is a parametric model (shared across all tasks) mapping these downstream task2vecs to simulation parameters, such as lighting direction, amount of blur, background variability, etc.  These parameters are then used by a data generator (in our implementation, built using the Three-D-World platform~\cite{gan2020threedworld}) to generate a dataset of synthetic images. A classifier model then gets pre-trained on these synthetic images, and the backbone is subsequently used for evaluation on specific downstream task. The classifier's accuracy on this task is used as a reward to update \ours's parameters. 
Once trained, \ours can also be used to efficiently predict simulation parameters in {\em one-shot} for ``unseen'' tasks that it has not encountered during training. 


\subsection{\ours Model} 


Let us denote \ours's parameters with $\theta$. Given the task2vec representation of a downstream task $\bs{x} \in \mc{X}$ as input, \ours outputs simulation parameters $a \in \Omega$. The model consists of $M$ output heads, one for each simulation parameter. In the following discussion, just as in our experiments, each simulation parameter is discretized to a few levels to limit the space of possible outputs. Each head outputs a categorical distribution $\pi_i(\bs{x}, \theta) \in \Delta^{k_i}$, where $k_i$ is the number of discrete values for parameter $i \in [M]$, and $\Delta^{k_i}$, a standard $k_i$-simplex. The set of argmax outputs $\nu(\bs{x}, \theta) = \{\nu_i | \nu_i = \argmax_{j \in [k_i]} \pi_{i, j} ~\forall i \in [M]\}$ is the set of simulation parameter values used for synthetic data generation. Subsequently, we drop annotating the dependence of $\pi$ and $\nu$ on $\theta$ and $\bs{x}$ when clear.

\subsection{\ours Training} 


Since Task2Sim aims to maximize downstream accuracy after pre-training, we use this accuracy as the reward in our training optimization\footnote{Note that our rewards depend only on the task2vec input and the output action and do not involve any states, and thus our problem can be considered similar to a stateless-RL or contextual bandits problem \cite{langford2007epoch}.}.
Note that this downstream accuracy is a non-differentiable function of the output simulation parameters (assuming any simulation engine can be used as a black box) and hence direct gradient-based optimization cannot be used to train \ours. Instead, we use REINFORCE~\cite{williams1992simple}, to approximate gradients of downstream task performance with respect to model parameters $\theta$. 

\ours's outputs represent a distribution over ``actions'' corresponding to different values of the set of $M$ simulation parameters. $P(a) = \prod_{i \in [M]} \pi_i(a_i)$ is the probability of picking action $a = [a_i]_{i \in [M]}$, under policy $\pi = [\pi_i]_{i \in [M]}$. Remember that the output $\pi$ is a function of the parameters $\theta$ and the task representation $\bs{x}$. To train the model, we maximize the expected reward under its policy, defined as
\begin{align}
    R = \E_{a \in \Omega}[R(a)] = \sum_{a \in \Omega} P(a) R(a)
\end{align}
where $\Omega$ is the space of all outputs $a$ and $R(a)$ is the reward when parameter values corresponding to action $a$ are chosen. Since reward is the downstream accuracy, $R(a) \in [0, 100]$.  
Using the REINFORCE rule, we have
\begin{align}
    \nabla_{\theta} R 
    &= \E_{a \in \Omega} \left[ (\nabla_{\theta} \log P(a)) R(a) \right] \\
    &= \E_{a \in \Omega} \left[ \left(\sum_{i \in [M]} \nabla_{\theta} \log \pi_i(a_i) \right) R(a) \right]
\end{align}
where the 2nd step comes from linearity of the derivative. In practice, we use a point estimate of the above expectation at a sample $a \sim (\pi + \epsilon)$ ($\epsilon$ being some exploration noise added to the Task2Sim output distribution) with a self-critical baseline following \cite{rennie2017self}:
\begin{align} \label{eq:grad-pt-est}
    \nabla_{\theta} R \approx \left(\sum_{i \in [M]} \nabla_{\theta} \log \pi_i(a_i) \right) \left( R(a) - R(\nu) \right) 
\end{align}
where, as a reminder $\nu$ is the set of the distribution argmax parameter values from the \name{} model heads.

A pseudo-code of our approach is shown in \cref{alg:train}.  Specifically, we update the model parameters $\theta$ using minibatches of tasks sampled from a set of ``seen'' tasks. Similar to \cite{oh2018self}, we also employ self-imitation learning biased towards actions found to have better rewards. This is done by keeping track of the best action encountered in the learning process and using it for additional updates to the model, besides the ones in \cref{ln:update} of \cref{alg:train}. 
Furthermore, we use the test accuracy of a 5-nearest neighbors classifier operating on features generated by the pretrained backbone as a proxy for downstream task performance since it is computationally much faster than other common evaluation criteria used in transfer learning, e.g., linear probing or full-network finetuning. Our experiments demonstrate that this proxy evaluation measure indeed correlates with, and thus, helps in final downstream performance with linear probing or full-network finetuning. 






\begin{algorithm}
\DontPrintSemicolon
 \textbf{Input:} Set of $N$ ``seen'' downstream tasks represented by task2vecs $\mc{T} = \{\bs{x}_i | i \in [N]\}$. \\
 Given initial Task2Sim parameters $\theta_0$ and initial noise level $\epsilon_0$\\
 Initialize $a_{max}^{(i)} | i \in [N]$ the maximum reward action for each seen task \\
 \For{$t \in [T]$}{
 Set noise level $\epsilon = \frac{\epsilon_0}{t} $ \\
 Sample minibatch $\tau$ of size $n$ from $\mc{T}$  \\
 Get \ours output distributions $\pi^{(i)} | i \in [n]$ \\
 Sample outputs $a^{(i)} \sim \pi^{(i)} + \epsilon$ \\
 Get Rewards $R(a^{(i)})$ by generating a synthetic dataset with parameters $a^{(i)}$, pre-training a backbone on it, and getting the 5-NN downstream accuracy using this backbone \\
 Update $a_{max}^{(i)}$ if $R(a^{(i)}) > R(a_{max}^{(i)})$ \\
 Get point estimates of reward gradients $dr^{(i)}$ for each task in minibatch using \cref{eq:grad-pt-est} \\
 $\theta_{t,0} \leftarrow \theta_{t-1} + \frac{\sum_{i \in [n]} dr^{(i)}}{n}$ \label{ln:update} \\
 \For{$j \in [T_{si}]$}{ 
    \tcp{Self Imitation}
    Get reward gradient estimates $dr_{si}^{(i)}$ from \cref{eq:grad-pt-est} for $a \leftarrow a_{max}^{(i)}$ \\
    $\theta_{t, j}  \leftarrow \theta_{t, j-1} + \frac{\sum_{i \in [n]} dr_{si}^{(i)}}{n}$
 }
 $\theta_{t} \leftarrow \theta_{t, T_{si}}$
 }
 \textbf{Output}: Trained model with parameters $\theta_T$. 
 \caption{Training Task2Sim}
 \label{alg:train}  
\end{algorithm}

\section{Theory}
In this section, we give guarantees on our grid-based approach. Suppose there is some underlying distribution $\mathcal{P}$ with corresponding density function $p : \mathbb{R}^d \rightarrow \mathbb{R}_{\ge 0}$ from which our data points $X_{[n]} = \{x_1,...,x_n\}$ are drawn i.i.d. We show guarantees on the density estimator based on the grid cell counts.

We need the following regularity assumptions on the density function. The first ensures that the density function has compact support with smooth boundaries and is lower bounded by some positive quantity, and the other ensures that the density function has smoothness. These are standard assumptions in analyses on density estimation e.g. \cite{gine2002rates,jiang2017uniform,chen2017tutorial,singh2009adaptive}.
\begin{assumption}\label{assumption1}
$p$ has compact support $\mathcal{X} \in \mathbb{R}^d$ and there exists $\lambda_0, r_0, C_0 > 0$ such that $p(x) \ge \lambda_0$ for all $x \in \mathcal{X}$ and $\text{Vol}(B(x, r) \cap \mathcal{X}) \ge C_0 \cdot \text{Vol}(B(x, r))$ for all $x \in \mathcal{X}$ and $0 < r \le r_0$, where $B(x, r) := \{x' \in \mathbb{R}^d: |x-x'| \le r\}$.
\end{assumption}
\begin{assumption}\label{assumption2}
$p$ is $\alpha$-Hölder continuous for some $0 < \alpha \le 1$: i.e. there exists $C_\alpha > 0$ such that $|p(x) - p(x')| \le C_\alpha \cdot |x - x'|^\alpha$ for all $x, x' \in \mathbb{R}^d$.
\end{assumption}

We now give the result, which says that for $h$ sufficiently small depending on $p$ (if $h$ is too large, then the grid is too coarse to learn a statistically consistent density estimator), and $n$ sufficiently large, there will be a high probability finite-sample uniform bound on the difference between the density estimator and the true density. The proof can be found in the Appendix.
\begin{theorem}\label{theorem}
Suppose Assumption~\ref{assumption1} and~\ref{assumption2} hold. Then there exists constants $C, C_{1} > 0$ depending on $p$ such that the following holds.
Let $0 < \delta < 1$, $0 < h < \text{min}\{\left(\frac{\lambda_0}{2\cdot C_\alpha}\right)^{1/\alpha}, r_0\}$, $nh^d \ge C_1$. Let $\mathcal{G}_h$ be a partitioning of $\mathbb{R}^d$ into grid cells of edge-length $h$ and for $x \in \mathbb{R}^d$. Let $G(x)$ denote the cell in $\mathcal{G}_h$ that $x$ belongs to.  Then, define the corresponding density estimator $\widehat{p}_h$ as:
\begin{align*}
    \widehat{p}_h(x) := \frac{|X_{[n]} \cap G(x)|}{n\cdot h^d}.
\end{align*}
Then, with probability at least $1 - \delta$:
\begin{align*}
    \sup_{x \in \mathbb{R}^d} |\widehat{p}_h(x)  - p(x)| \le C\cdot \left( h^\alpha + \frac{\sqrt{\log(1/(h\delta)}}{\sqrt{n\cdot h^d}} \right).
\end{align*}
\end{theorem}


\begin{remark}
In the above result, choosing $h \approx n^{-1/(2\alpha+d)}$ optimizes the convergence rate to $\tilde{O}(n^{-\alpha/(2\alpha+d)})$, which is the minimax optimal convergence up to logarithmic factors for the density estimation problem as established by Tsybakov \cite{tsybakov1997nonparametric,tsybakov2008introduction}.
\end{remark}
In other words, the grid-based approach statistically performs at least as well as any estimator of the density function, including the density estimator used by MeanShift. It is worth noting that while our results only provide results for the density estimation portion of MeanShift++ (i.e. the grid-cell binning technique), we prove the near-minimax optimality of this estimation. This implies that the information contained in the density estimation portion serves as an approximately sufficient statistic for the rest of the procedure, which behaves similarly to MeanShift, which operates on another, also nearly-optimal density estimator. Thus, existing analyses of MeanShift e.g. \cite{arias2016estimation,chen2015convergence,xiang2005convergence,li2007note,ghassabeh2015sufficient,ghassabeh2013convergence,subbarao2009nonlinear} can be adapted here; however, it is known that MeanShift is very difficult to analyze \cite{dasgupta2014optimal} and a complete analysis is beyond the scope of this paper.

\section{Efficient Implementation: Details}\label{sec:efficient-implementation}
We implement the proposed algorithm in a highly efficient \textsf{abess} library
with both Python and R interfaces.
As can be seen from the numerical experiments below,
\textsf{abess} can have competitive or even less running times
compared with other well-known sparse learning software like \textsf{glmnet}.
\textsf{abess} achieves extreme efficiency by leveraging efficient implementation to reduce the time consumption on computing forward and backward sacrifices.
% These include an adaptive grid of tuning parameters, continuation, active set updates, greedy cyclic ordering of
% coordinates, correlation screening, and a careful accounting of floating point operations—some of these
% heuristics (as specified below) appear in prior work
% for deriving highly efficient algorithms for the Lasso (e.g., glmnet).
In the following, we provide a detailed description of the efficient implementation.

%\subsection{Best-Subset Selection Path: Warm-Start Initialization}
%In this part, we describe an efficient procedure to compute
%along a best-subset selection path, i.e., computing the solution of Algorithm~\ref{alg:fbess}
%on a series of support size: $1 < 2 < \cdots < s_{\max}$.
%The procedure is summarized in Algorithm~\ref{alg:abess}.
%% Inspired by the work of \citet{friedman2010regularization},
%% we use the $s_i$ solution as a warm start for $s_{i+1}$.
%Algorithm~\ref{alg:abess} runs Algorithm~\ref{alg:fbess} on various support sizes ranging from small to large.
%% Fortunately, it allows using the output of Algorithm~\ref{alg:fbess} to build a better active set,
%% allowing Algorithm~\ref{alg:fbess} to reach convergence with less splicing iteration.
%Notice that, in Step 4 of Algorithm~\ref{alg:abess},
%we set the initial support set of the next support size as the union of
%the selected subset returned by Algorithm~\ref{alg:fbess} and the variable with the largest forward sacrifice,
%which refers to ``warm-starts initialization''.
%Both existing numerical experiments and theoretical analysis suggest that
%such an initialization is remarkably efficient \citep{friedman2010regularization, barutConditionalSureIndependence2016}.
%From our experience, by setting the initialization as the input of Algorithm~\ref{alg:abess},
%we can reduce the number of iterations in the \textbf{repeat-until} loop in Algorithm~\ref{alg:fbess},
%whittling down the overall runtime.
% \footnote[1]{The adaptivity refers to adapting the warm-start initialization.}
% Algorithm~\ref{alg:abess} describes the procedure mentioned above in detail.
%\begin{algorithm}[htbp]
% \caption{P\underline{a}thwise \underline{Be}st-\underline{S}ubset \underline{S}election for GLM (\textbf{ABESS})}
% \label{alg:abess}
% \begin{algorithmic}[1]
% \REQUIRE A dataset set $\{(\boldsymbol{x}_i, y_i)\}^{n}_{i=1}$ and the maximum support size $s_{\max}$.
% \STATE $\mathcal{A}^0 \leftarrow \{ \arg\max\limits_j | (\frac{\partial l_n( \boldsymbol \beta )}{\partial \boldsymbol \beta} |_{\boldsymbol \beta = {\boldsymbol 0} })_j | \}$
% \FOR {$s = 1, \ldots, s_{\max}$}
% \STATE $(\hat{\boldsymbol \beta}_{s}, \hat{\mathcal{A}}_{s}, \hat{\boldsymbol{d}}_s) \leftarrow \textbf{BESS-GLM}\left( \{(\boldsymbol{x}_i, y_i)\}^{n}_{i=1}, \mathcal{A}^0, k_{\max}, \tau_s \right)$.
% \STATE $\mathcal{A}^0 \leftarrow \hat{\mathcal{A}}_{s} \cup \{ \arg\max\limits_j|(\hat{\boldsymbol{d}}_s)_j| \}$.
% \ENDFOR
% \ENSURE $\{ (\hat{\boldsymbol \beta}_{s_i}, \hat{\mathcal{A}}_{s_i} ) \}_{i=1}^{\max}$.
% \end{algorithmic}
%\end{algorithm}

\subsection{Simplified Convex Optimization and Early Stopping}\label{sec:approximiated-newton-update}

In Algorithm~\ref{alg:fbess}, we need to solve convex optimization problems:
$\tilde{\boldsymbol \beta} \leftarrow \arg\min\limits_{\beta_{\mathcal{I}} = 0} l_n(\boldsymbol\beta ).$
Directly solving it via a convex optimization solver would consume a large time.
To leverage the sparsity nature of this problem, we can turn to solve a simplified problem:
$$\tilde{\boldsymbol \beta}_{\mathcal{A}} \leftarrow \arg\min\limits_{\boldsymbol{\beta}_{\mathcal{A}}}- \sum_{i=1}^n\{y_i \boldsymbol{\beta}_{\mathcal{A}}^\top (\boldsymbol{x}_i)_{\mathcal{A}} - b(\boldsymbol{\beta}_{\mathcal{A}}^\top (\boldsymbol{x}_i)_{\mathcal{A}}) + c(y_i,\phi)\} $$
and pad zero entries to obtain $\tilde{\boldsymbol \beta}$.
The simplified problem solves the regression coefficient on a much smaller dataset, and thus it is computationally appealing.
Since the simplified problem has no closed-form solution, we must perform iterative algorithms to solve it.
The Newton method is one of the most popular methods to solve this problem.
More precisely, we conduct Newton's updates
% \begin{equation}\label{eqn:primary_fit_update}
% \begin{split}
% {\boldsymbol \beta}_{\mathcal{A}}^{m+1} \leftarrow \boldsymbol \beta_{\mathcal{A}}^m - \Big( \left.\frac{\partial^2 l_n( \boldsymbol \beta )}{ (\partial \boldsymbol \beta_{{\mathcal{A}}} )^2 }\right|_{\boldsymbol \beta = \boldsymbol \beta^m} \Big)^{-1} \Big( \left.\frac{\partial l_n( \boldsymbol \beta )}{ \partial \boldsymbol \beta_{{\mathcal{A}}} }\right|_{\boldsymbol \beta = \boldsymbol \beta^m} \Big),
% \end{split}
% \end{equation}
until $\| {\boldsymbol \beta}_{\mathcal{A}}^{m+1} - {\boldsymbol \beta}_{\mathcal{A}}^{m}\|_2 \leq \epsilon$ or $m > k$,
where $\epsilon$ is convergence tolerance and $k$ is the maximal number of Newton's updates.
Generally, setting $\epsilon = 10^{-6}$ and $k = 80$ returns a desirable coefficient estimation.
% In general, the inverse of the second derivative in \eqref{eqn:primary_fit_update} is computationally expensive, so we use its diagonalized version to approximate it. The update then makes the following changes:
% \begin{equation}\label{eqn:fast_primary_fit_update}
% \begin{split}
% {\boldsymbol \beta}_{\tilde{\mathcal{A}} }^{m+1} \leftarrow \boldsymbol \beta_{\tilde{\mathcal{A}} }^m - \rho D \Big( \left.\frac{\partial l_n( \boldsymbol \beta )}{ \partial \boldsymbol \beta_{\tilde{\mathcal{A}}} }\right|_{\boldsymbol \beta = \boldsymbol \beta^m} \Big),
% \end{split}
% \end{equation}
% where $D = \textup{diag}( (\left.\frac{\partial^2 l_n( \boldsymbol \beta )}{ (\partial \boldsymbol \beta_{\tilde{\mathcal{A}_{1}}} )^2 }\right|_{\boldsymbol \beta = \boldsymbol \beta^m} )^{-1}, \ldots, (\left.\frac{\partial^2 l_n( \boldsymbol \beta )}{ (\partial \boldsymbol \beta_{\tilde{\mathcal{A}}_{|A|}} )^2 }\right|_{\boldsymbol \beta = \boldsymbol \beta^m} )^{-1})$
% and $\rho$ is step size.
% While using the approximation increases iteration time, it avoids a high level of computational complexity when computing the matrix inversion.

Notice that not all of the candidate active sets considered in Algorithm~\ref{alg:fbess} associate with a better solution such that $L - l_{n}(\tilde{\boldsymbol{\beta}}) > \tau_s$,
This inspires us to stop Newton's updates on these active candidates early sets to reduce unnecessary Newton's updates.
More precise, after each Newton's update, we check whether the heuristic criterion
$l_1 - (k - m) \times (l_2 - l_1) > L - \tau_s$ holds,
where $l_1 = l_n({\boldsymbol\beta}_{\mathcal{A}}^{m}), l_2 = l_n({\boldsymbol\beta}_{\mathcal{A}}^{m+1})$.
If so, it implies the remaining $k - m$ times Newton update can potentially lead to a lower loss, and thus,
it has no reason to stop Newton's updates; otherwise, we can stop Newton's updates.
The heuristic criterion is designed based on the fact that the convergence rate of Newton's iterative method is quadratic \citep{nocedalNumericalOptimization1999}.
It considerably improves the efficiency of computing forward sacrifices and
retain the high quality of coefficient estimations.

\subsection{Importance-Priority Splicing}\label{sec:importance-priority-splicing}
Considerable speedup is achieved in a high-dimensional regime by employing an importance-priority splicing procedure,
which is motivated by an active set update for the Lasso \citep{lassoscreening2012tibshirani}.
Specifically, a heavy computational burden comes up when
computing forward sacrifices for $p - s$ variables in the inactive set, especially since $p$ is very large.
To alleviate this burden, after completing forward and backward sacrifices through all the variables, we perform the splicing iteration on a small but essential subset: the union of
the selected variables and $d$ ($\ll p$) variables with the more significant forward sacrifices.
Once the splicing iteration converges, we re-compute sacrifices for all the variables and update the important subset.
If the updated important subset is the same as the previous one,
Algorithm~\ref{alg:fbess} is done. Otherwise, the above procedure is repeated.
From our numerical experience, the procedure mentioned above can save lots of time;
meantime, it keeps Algorithm~\ref{alg:fbess} enjoying the statistical properties
presented in Section~\ref{sec:theorical-properties}.

\section{Simulation Studies}\label{sec:simulation}
In this section, we are mainly interested in the empirical performance of the ABESS algorithm on logistic regression and Poisson regression.
Logistic regression is widely used for classification tasks, and Poisson regression is appropriate when the response is a count.
\if0\informsMOR{In ``Additional Simulation'' of Supplementary Material, we }\else{We }\fi
also consider the performance of ABESS algorithm on multi-response linear regression (a.k.a., multi-task learning).
Before formally analyzing the simulation results,
we illustrate our simulation settings in Section~\ref{subsec:setup}.
% This subsection develops parallel with Section \ref{subsec:logistic}.

%In this section, we study the empirical performance of ABESS for GLM on two generalized linear models,
%logistic regression and gamma regression,
%where logistic regression is widely used for classification and
%gamma regression model is useful for modeling positive continuous response variables.
%Before formally studying logistic regression and gamma regression in Section~\ref{subsec:logistic} and Section~\ref{subsec:gamma}, respectively, we illustrate our simulation setting in Section~\ref{subsec:setup}.

\subsection{Setup}\label{subsec:setup}
To synthesize a dataset, we generate multivariate Gaussian realizations $\boldsymbol{x}_1, \ldots, \boldsymbol{x}_n \overset{i.i.d.}{\sim} \mathcal{MVN}(0,\Sigma)$,
where $\Sigma$ is a $p$-by-$p$ covariance matrix.
%We generate i.i.d error $\epsilon\sim N(0,\sigma^2)$.
%Define the signal to noise ratio (SNR) by $SNR = \frac{\beta^{\top}\Sigma\beta}{\sigma^2}$.
We consider two covariance structures for $\Sigma$: the independent structure ($\Sigma$ is an identity matrix)
and the constant structure ($\Sigma_{ij} = \rho^{I(i\neq j)}$ for some positive constant $\rho$). The value of $\rho$ and $p$ will be specified later.
We set the true regression coefficient $\boldsymbol{\beta}^*$ as a sparse vector with $k$ non-zero entries that have equi-spaced indices in $\{1, \ldots, p\}$.
Finally, given a design matrix $\mathbf{X} = (\boldsymbol{x}_1, \ldots, \boldsymbol{x}_n)^\top$ and $\boldsymbol{\beta}^*$,
we draw response realizations $\{y_i\}_{i=1}^n$ according to the GLMs.

We assess our proposal via the following criteria.
First, to measure the performance of subset selection,
we consider the probabilities of covering true active and inactive sets: $\mathbb{P}(\mathcal{A}^* \subseteq \hat{\mathcal{A}})$ and
$\mathbb{P}(\mathcal{I}^* \subseteq \hat{\mathcal{I}})$ (here, $\mathcal{I}^* = (\mathcal{A}^*)^c$).
We also consider exact support recover probability as $\mathbb{P}(\mathcal{A}^* = \hat{\mathcal{A}})$.
Since the probability is unknown, we empirically examine the proportion of recovery for the active set, inactive set, and exact recovery in 200 replications for instead.
As for parameter estimation performance, we examine relative error (ReErr) on parameter estimations:
$\|\hat{\boldsymbol{\beta}}-\boldsymbol{\beta}^*\|_{2} /\|\boldsymbol{\beta}^*\|_{2}$.
Finally, computational efficiency is directly measured by the runtime.

In addition to our proposed algorithms, we compare classical variable selection methods: LASSO \citep{tibshirani1996regression}, SCAD \citep{fan2001variable}, and MCP \citep{zhang2010nearly}.
%, and a recently proposed coordinate descent (CD) method for $\ell_0$-regularized classification \citep{antoine2021l0learn}.
For all these methods, we apply 10-fold cross-validation (CV) and the GIC to select the tuning parameter, respectively.
% For all these methods, we apply 10-fold cross-validation (CV) to select the tuning parameter.
% ABESS also uses generalized information criterion (GIC) \citep{fan2013tuning} because,
% by combining GIC, ABESS can consistently recover $\mathcal{A}^*$ under linear models \citep{zhu2020polynomial}.
The software for these methods is available at R CRAN (\url{https://cran.r-project.org}).
The software of all methods is summarized in Table~\ref{tab:implementation-details}.
All experiments are carried out on an R environment in a Linux platform with Intel(R) Xeon(R) Gold 6248 CPU @ 2.50GHz. 
% Note that, all experiments result are based on 200 random synthetic datasets.
%Ubuntu platform with Intel(R) Xeon(R) Gold 6248 CPU @ 2.50GHz.

% Model selection methods such as cross-validation and information criteria are widely used.
% Recently, \citet{fan2013tuning} explored generalized information criterion (GIC) in tuning parameter selection for
% penalized likelihood methods under GLM.
% Here, we use a GIC-type information criterion to recovery support size, which is defined as:
% $\mathrm{F}(\hat{\boldsymbol \beta}) = l_n( \hat{\boldsymbol \beta} ) + |\text{supp}(\hat{\boldsymbol \beta})| \log(p) \log\log n.$
% Intuitively speaking, the model complexity penalty term $|\text{supp}(\hat{\boldsymbol \beta})| \log p \log\log n$ is set to prevent over-fitting,
% where the term $\log\log n$ with a slow diverging rate is used to prevent under-fitting.
% Combining the Algorithm~\ref{alg:fbess} with GIC, we select the support size that minimizes the $F(\hat{\boldsymbol{\beta}})$.}

% \begin{table}[htbp]
% \caption{Implementation details for all methods.
% The values in the parentheses indicate the version number of R packages.}\label{tab:implementation-details}
% \centering
% \begin{tabular}{ccc}
% \toprule
% Method & Software & Tuning method \\
% \midrule
% ABESS-GIC & abess (0.4.0) & GIC \\
% LASSO-GIC & glmnet (4.1-3) & GIC \\
% SCAD-GIC & ncvreg (3.13.0) & GIC \\
% MCP-GIC & ncvreg (3.13.0) & GIC \\
% CD-GIC & L0Learn (2.0.3) & GIC \\
% ABESS-CV & abess (0.4.0) & 10-folds CV \\
% LASSO-CV & glmnet (4.1-3) & 10-folds CV \\
% SCAD-CV & ncvreg (3.13.0) & 10-folds CV \\
% MCP-CV & ncvreg (3.13.0) & 10-folds CV \\
% CD-CV & L0Learn (2.0.3) & 10-folds CV \\
% \bottomrule
% \end{tabular}
% \end{table}
\begin{table}[htbp]
\caption{Software for all methods.
The values in the parentheses indicate the version number of R packages.The tuning parameter within the MCP/SCAD penalty is fixed at 3/3.7.}\label{tab:implementation-details}
\centering
\if0\informsMOR{
% \begin{tabular}{ccccccc}
% \toprule
% Method & ABESS & LASSO & SCAD & MCP & CD \\
% \midrule
% Software & \textsf{abess} (0.4.0) & \textsf{glmnet} (4.1-3) & \textsf{ncvreg} (3.13.0) & \textsf{ncvreg} (3.13.0) & \textsf{L0Learn} (2.0.3) \\
% Tuning & sparsity $s$ & $\ell_1$ penalty & $\lambda$ & $\lambda$& $\lambda$ \\
% \bottomrule
% \end{tabular}
\begin{tabular}{cccccc}
    \toprule
    Method & ABESS & LASSO & SCAD & MCP \\
    \midrule
    Software & \textsf{abess} (0.4.0) & \textsf{glmnet} (4.1-3) & \textsf{ncvreg} (3.13.0) & \textsf{ncvreg} (3.13.0) \\
    Tuning & sparsity $s$ & $\ell_1$ penalty & $\lambda$ & $\lambda$ \\
    \bottomrule
    \end{tabular}
}\else{
% \begin{tabular}{ccccccc}
% \hline
% Method & ABESS & LASSO & SCAD & MCP & CD \\
% \hline
% Software & \textsf{abess} (0.4.0) & \textsf{glmnet} (4.1-3) & \textsf{ncvreg} (3.13.0) & \textsf{ncvreg} (3.13.0) & \textsf{L0Learn} (2.0.3) \\
% Tuning & sparsity $s$ & $\ell_1$ penalty & {\color{red}SCAD penalty} & {\color{red}MCP penalty} & $\ell_0$ penalty \\
% \hline
% \end{tabular}
\begin{tabular}{cccccc}
\hline
Method & ABESS & LASSO & SCAD & MCP \\
\hline
Software & \textsf{abess} (0.4.0) & \textsf{glmnet} (4.1-3) & \textsf{ncvreg} (3.13.0) & \textsf{ncvreg} (3.13.0) \\
Tuning & sparsity $s$ & $\ell_1$ penalty & {\color{red}SCAD penalty} & {\color{red}MCP penalty} \\
\hline
\end{tabular}
}\fi
\end{table}
% We implement our proposal in an R package abess \citep{zhu-abess-arxiv}.

\subsection{Logistic Regression}\label{subsec:logistic}
% In this subsection, we illustrate the power of ABESS on logistic regression, which is one of the most popular GLMs widely used for classification tasks.
% In terms of logistic regression, the response $y_i$ is a binary variable following a Bernoulli distribution $B(1, p_i)$,
% where $p_i \coloneqq \mathbb{P}(y_i=1)$ is determined by $\log(\frac{p_i}{1-p_i}) = \boldsymbol x_i^\top \boldsymbol{\beta }$.
% Here, the link function is known as the logit function, defined by $logit(p) = \log(\frac{p}{1-p})$.
% As a result, the negative log-likelihood is given by
% \begin{equation*}
% l_n(\boldsymbol\beta) = -\sum_{i=1}^{N}\left\{y_{i} \boldsymbol {x}_i^\top \boldsymbol \beta-\log \left(1+e^{\boldsymbol {x}_i^\top \boldsymbol \beta}\right)\right\}.
% \end{equation*}
% Empirically, we generate $x_i$ and $\beta$ as described in Section \ref{subsec:setup}.
% Binary response $y_i$ is then drawn from the Bernoulli distribution according to (\ref{eqn:formula_binomial}).
% Let $H_j = \sum\limits_{i=1}^{n} \frac{e^{\boldsymbol {x}_i^\top \hat{\boldsymbol \beta}}}{(1 + e^{\boldsymbol {x}_i^\top \hat{\boldsymbol \beta}})^2} x_{ij}^2$ and
% the gradient of $l_n(\boldsymbol{\beta})$ at $\hat{\boldsymbol{\beta}}$ be $\hat{\boldsymbol d} = -\sum\limits_{i=1}^{n}(y_i - \frac{e^{\boldsymbol {x}_i^\top \hat{\boldsymbol \beta}}}{1 + e^{\boldsymbol {x}_i^\top \hat{\boldsymbol \beta}}}) \boldsymbol {x}_i$,
% \eqref{eqn:approx_sacrifice} can be explicit expressed as:
% $\xi_j = H_j (\hat{\boldsymbol{\beta}}_j)^2$ for $j\in \mathcal{A}$ and
% $\zeta_j = H_j^{-1}( \hat{\boldsymbol d}_j )^2$ for $j\in \mathcal{I}$.
% \begin{equation*}
% \begin{aligned}
% % \hat{\boldsymbol d} &= -\sum_{i=1}^{n}(y_i - \frac{e^{\boldsymbol {x}_i^\top \boldsymbol \beta}}{1 + e^{\boldsymbol {x}_i^\top \boldsymbol \beta}}) \boldsymbol {x}_i. \\
% \xi_j
% & = H_j (\hat{\boldsymbol{\beta}}_j)^2, j\in \mathcal{A},\\
% \zeta_j
% & = H_j^{-1}
% ( \hat{\boldsymbol d}_j )^2, j\in \mathcal{I}.
% \end{aligned}
% \end{equation*}
% Given the explicit expression of \eqref{eqn:approx_sacrifice},
% we can conduct Algorithm~\ref{alg:abess} to estimate $\boldsymbol{\beta}$.

The dimension $p$ is fixed as 500 for the logistic regression model. For the constant correlation case, we set $\rho = 0.4$.
The non-zero coefficients $\boldsymbol{\beta}^*_{\mathcal{A}^*}$ are set to be $(2,2,8,8,8,8,10,10,10,10)^\top$. 
Now we compare methods listed in Table~\ref{tab:implementation-details}.
Figures~\ref{fig:rate_binomial} and \ref{fig:ReErr_binomial} present the results on subset selection and parameter estimation when the sample size increases. Out of clarity, we omit the CV results here and defer these results to the Additional Figures in Supplementary Material.


\begin{figure}[htbp]
\centering
\includegraphics[width=1.0\textwidth]{figure/rate_binomial_gic.pdf}
\if0\informsMOR{
\vspace{-30pt}
}\fi
\caption{Performance on subset selection under logistic regression when covariates have independent correlation structure (Upper) and constant correlation structure (Lower), measured by three kinds probabilities: $\mathbb{P}(\mathcal{A}^* \subseteq \hat{\mathcal{A}})$, $\mathbb{P}(\mathcal{I}^* \subseteq \hat{\mathcal{I}})$, and $\mathbb{P}(\mathcal{A}^* = \hat{\mathcal{A}})$ that are presented in Left, Middle and Right panels, respectively.
}
\label{fig:rate_binomial}
\end{figure}
\begin{figure}[htbp]
\centering
\includegraphics[width=0.8\textwidth]{figure/ReErr_binomial_gic.pdf}
\if0\informsMOR{
\vspace{-10pt}
}\fi
\caption{Performance on parameter estimation under logistic regression models when covariance matrices have independent correlation structure (Left) and exponential correlation structure (Right). The $y$-axis is the median of ReErr in a log scale.}
\label{fig:ReErr_binomial}
\end{figure}

As depicted in the left panel of Figure~\ref{fig:rate_binomial}, the probability $\mathbb{P}(\mathcal{A}^* \subseteq \hat{\mathcal{A}})$ approaches 1 as the sample size increases, indicating that all methods, except LASSO in the high correlation setting, can provide a no-false-exclusion estimator when the sample size is sufficiently large. However, when considering $\mathbb{P}(\mathcal{I}^* \subseteq \hat{\mathcal{I}})$, as observed in the middle panel of Figure~\ref{fig:rate_binomial}, the LASSO estimator consistently exhibits false inclusions, and the SCAD/MCP estimator shows false inclusions when the covariates are highly correlated. In contrast, only ABESS guarantees that $\mathbb{P}(\mathcal{I}^* \subseteq \hat{\mathcal{I}})$ approaches 1 for large sample sizes. 

Furthermore, as evident from the right panel of Figure~\ref{fig:rate_binomial}, ABESS accurately recovers the true subset under both correlation settings. While SCAD and MCP can also achieve exact support recovery given a sufficient sample size, ABESS demonstrates support recovery consistency with the smallest sample size, particularly when variables are correlated. It is important to note that although our theory imposes restrictions on the correlation among a small subset of variables (see Assumption~\ref{con:technical-assumption}), our algorithm still performs effectively in the constant correlation setting. This setting (i.e., $\rho=0.4$) violates Assumption~\ref{con:technical-assumption} as the correlation between any two variables exceeds 0.183, which is the maximum acceptable pairwise correlation satisfying Assumption~\ref{con:technical-assumption}.

Moving on to Figure~\ref{fig:ReErr_binomial}, it illustrates the superiority of ABESS in parameter estimation. ABESS visibly outperforms other methods in the small sample size regime and maintains highly competitive performance as the sample size increases. This superiority in parameter estimation is not surprising, as ABESS yields an oracle estimator when the support set is correctly identified. Although SCAD and MCP do not provide algorithmic guarantees for finding the local minimum, they exhibit competitive parameter estimation performance due to their asymptotic unbiasedness. Conversely, the LASSO estimator is biased and performs the worst among all the methods.

%\begin{figure}
%	\centering
%	\includegraphics[width=\textwidth]{figure/Performance_binomial.pdf}
%	\caption{Performance comparison under two correlation structures: independent and exponential. (A) Performance for subset selection, measured by support recover probability. (B) Performance for parameter estimation, measured by median ReErr. (C) Average runtime, measured in seconds. L0Learn is omitted since its runtime is far longer than others.}
%	\label{fig:Performance_binomial}
%\end{figure}

\subsection{Poisson Regression}\label{seubsec:poisson}
% As regard to Poisson regression, the response $y_i$ is a integer variable following a Poisson distribution $\mathcal{P}(\lambda_i)$ where $\lambda_i = \exp(\boldsymbol x_i^\top \boldsymbol{\beta})$.
% As a result, the negative log-likelihood is given by
% \begin{equation*}
% l_n(\boldsymbol\beta) = -\sum_{i=1}^{N}\left\{y_{i} \boldsymbol {x}_i^\top \boldsymbol\beta - e^{\boldsymbol {x}_i^\top \boldsymbol \beta} -\log(y_i!)\right\}.
% \end{equation*}
% Empirically, we generate $x_i$ and $\beta$ as described in Section \ref{subsec:setup}.
% Binary response $y_i$ is then drawn from the Bernoulli distribution according to (\ref{eqn:formula_binomial}).
% Let $H_j = \sum\limits_{i=1}^{n} \exp(\boldsymbol {x}_i^\top \hat{\boldsymbol \beta}) x_{ij}^2$ and
% the gradient of $l_n(\boldsymbol{\beta})$ at $\hat{\boldsymbol{\beta}}$ be $\hat{\boldsymbol d} = -\sum\limits_{i=1}^{n}(y_i - \exp(\boldsymbol {x}_i^\top \hat{\boldsymbol \beta})) \boldsymbol {x}_i$,
% \eqref{eqn:approx_sacrifice} can be explicit expressed as:
% $\xi_j = H_j (\hat{\boldsymbol{\beta}}_j)^2$ for $j\in \mathcal{A}$ and
% $\zeta_j = H_j^{-1}( \hat{\boldsymbol d}_j )^2$ for $j\in \mathcal{I}$.
% Given the explicit expression of \eqref{eqn:approx_sacrifice},
% we can conduct Algorithm~\ref{alg:abess} to estimate $\boldsymbol{\beta}$.


For the Poisson regression model, we consider a fixed $p$ value of 500, and set $\rho = 0.2$ for the constant correlation case. The non-zero coefficients $\boldsymbol{\beta}^*_{\mathcal{A}^*}$ are specified as $(1, 1, 1)^\top$. Figures~\ref{fig:rate_poisson_gic}-\ref{fig:ReErr_poisson_gic} present the evaluation of subset selection and parameter estimation quality. Examining Figures~\ref{fig:rate_poisson_gic}, we observe that for ABESS/SCAD/MCP, the probabilities $\mathbb{P}(\mathcal{A}^* \subseteq \hat{\mathcal{A}})$, $\mathbb{P}(\mathcal{I}^* \subseteq \hat{\mathcal{I}})$, and $\mathbb{P}(\mathcal{A}^* = \hat{\mathcal{A}})$ gradually approach 1 as the sample size $n$ increases. In contrary, the LASSO, regardless of the highest inclusion probability for $\mathcal{A}^*$, still has a chance of including ineffective variables, especially when variables are correlated. Comparing ABESS, SCAD, and MCP, it is evident that ABESS achieves the highest exact selection probability, followed by SCAD and MCP. Similar to the results in logistic regression, ABESS achieves exact selection of the effective variables with the smallest sample size under the constant correlation structure.
Regarding the quality of parameter estimation, the ReErr of all methods reasonably decreases as the sample size $n$ increases. Again, ABESS exhibits the least estimation error in terms of the $\ell_2$-norm, which coincides with the results on logistic regression. It is worth noting that our method demonstrates consistency and polynomial complexity under Poisson regression, even though it violates the sub-Gaussian assumption. This is because the current framework of proofs allows for the relaxation of Assumption~\ref{con:subgaussian} to a sub-exponential distribution assumption, enabling the establishment of similar theoretical properties.

\begin{figure}[htbp]
\centering
\includegraphics[width=1.0\textwidth]{figure/rate_poisson_gic.pdf}
\if0\informsMOR{
\vspace{-30pt}
}\fi
\caption{Performance on subset selection under Poisson regression when covariates have independent correlation structure (Upper) and constant correlation structure (Lower), measured by three kinds probabilities: $\mathbb{P}(\mathcal{A}^* \subseteq \hat{\mathcal{A}})$, $\mathbb{P}(\mathcal{I}^* \subseteq \hat{\mathcal{I}})$, and $\mathbb{P}(\mathcal{A}^* = \hat{\mathcal{A}})$ that are presented in Left, Middle and Right panels, respectively.}
\label{fig:rate_poisson_gic}
\end{figure}
\begin{figure}[htbp]
\centering
\includegraphics[width=0.8\textwidth]{figure/ReErr_poisson_gic.pdf}
\if0\informsMOR{
\vspace{-5pt}
}\fi
\caption{Performance on parameter estimation under Poisson regression models when covariance matrices have independent correlation structure (Left) and exponential correlation structure (Right). The $y$-axis is the median of ReErr in a log scale.}
\label{fig:ReErr_poisson_gic}
\end{figure}

\subsection{Computational analysis}

We compare the runtime of different methods in Table~\ref{tab:implementation-details} for the logistic regression and Poisson regression models in Sections~\ref{subsec:logistic} to \ref{seubsec:poisson}. The runtime results are summarized in Figure~\ref{fig:simu_runtime}, indicating that ABESS demonstrates superior computational efficiency compared to state-of-the-art variable selection methods. For instance, when $n = 3000$, ABESS is at least four times faster than its competitors in logistic regression under an independent correlation structure. Furthermore, regardless of logistic regression or Poisson regression, ABESS exhibits similar computational performance, while other competitors run much faster when the pairwise correlation is higher. Lastly, it is important to note that the runtime of ABESS scales polynomially with sample sizes, aligning with the complexity presented in Theorem~\ref{thm:complexity}.
%In contrast, the runtime of other methods grows more rapidly as the sample size increases
%and appears like a quadratic function of the sample size in the independent scenario.
%Increasing iteration numbers for convergence may lead to this result.
%Moreover, ABESS-GIC is faster than ABESS-CV, demonstrating the superiority of the proposed adaptive parameter tuning procedure.
% Finally, according to the computational comparison presented in Figure~\ref{fig runtime_poisson_gic}, the ABESS has the least runtime and is much faster than the MCP and SCAD when variables are independent.

\begin{figure}[htbp]
\centering
\includegraphics[width=0.8\textwidth]{figure/runtime_binomial_gic.pdf}
\includegraphics[width=0.8\textwidth]{figure/runtime_poisson_gic.pdf}
\if0\informsMOR{
\vspace{-10pt}
}\fi
\caption{Average runtime (measured in seconds) on logistic regression (Upper panel) and Poisson regression (Lower panel). The results on two types of covariances matrix $\Sigma$, the independent correlation structure and constant correlation structure, are presented in the left and right panels, respectively. The error bars represent two times the standard errors.
}
\label{fig:simu_runtime}
\end{figure}

%% For the MOR template, uncomment this line and comment on the code blocks

\if1\informsMOR
{
\input{../appendix_numerical}
}\fi

% \input{realdata}
\mySection{Related Works and Discussion}{}
\label{chap3:sec:discussion}

In this section we briefly discuss the similarities and differences of the model presented in this chapter, comparing it with some related work presented earlier (Chapter \ref{chap1:artifact-centric-bpm}). We will mention a few related studies and discuss directly; a more formal comparative study using qualitative and quantitative metrics should be the subject of future work.

Hull et al. \citeyearpar{hull2009facilitating} provide an interoperation framework in which, data are hosted on central infrastructures named \textit{artifact-centric hubs}. As in the work presented in this chapter, they propose mechanisms (including user views) for controlling access to these data. Compared to choreography-like approach as the one presented in this chapter, their settings has the advantage of providing a conceptual rendezvous point to exchange status information. The same purpose can be replicated in this chapter's approach by introducing a new type of agent called "\textit{monitor}", which will serve as a rendezvous point; the behaviour of the agents will therefore have to be slightly adapted to take into account the monitor and to preserve as much as possible the autonomy of agents.

Lohmann and Wolf \citeyearpar{lohmann2010artifact} abandon the concept of having a single artifact hub \cite{hull2009facilitating} and they introduce the idea of having several agents which operate on artifacts. Some of those artifacts are mobile; thus, the authors provide a systematic approach for modelling artifact location and its impact on the accessibility of actions using a Petri net. Even though we also manipulate mobile artifacts, we do not model artifact location; rather, our agents are equipped with capabilities that allow them to manipulate the artifacts appropriately (taking into account their location). Moreover, our approach considers that artifacts can not be remotely accessed, this increases the autonomy of agents.

The process design approach presented in this chapter, has some conceptual similarities with the concept of \textit{proclets} proposed by Wil M. P. van der Aalst et al. \citeyearpar{van2001proclets, van2009workflow}: they both split the process when designing it. In the model presented in this chapter, the process is split into execution scenarios and its specification consists in the diagramming of each of them. Proclets \cite{van2001proclets, van2009workflow} uses the concept of \textit{proclet-class} to model different levels of granularity and cardinality of processes. Additionally, proclets act like agents and are autonomous enough to decide how to interact with each other.

The model presented in this chapter uses an attributed grammar as its mathematical foundation. This is also the case of the AWGAG model by Badouel et al. \citeyearpar{badouel14, badouel2015active}. However, their model puts stress on modelling process data and users as first class citizens and it is designed for Adaptive Case Management.

To summarise, the proposed approach in this chapter allows the modelling and decentralized execution of administrative processes using autonomous agents. In it, process management is very simply done in two steps. The designer only needs to focus on modelling the artifacts in the form of task trees and the rest is easily deduced. Moreover, we propose a simple but powerful mechanism for securing data based on the notion of accreditation; this mechanism is perfectly composed with that of artifacts. The main strengths of our model are therefore : 
\begin{itemize}
	\item The simplicity of its syntax (process specification language), which moreover (well helped by the accreditation model), is suitable for administrative processes;
	\item The simplicity of its execution model; the latter is very close to the blockchain's execution model \cite{hull2017blockchain, mendling2018blockchains}. On condition of a formal study, the latter could possess the same qualities (fault tolerance, distributivity, security, peer autonomy, etc.) that emanate from the blockchain;
	\item Its formal character, which makes it verifiable using appropriate mathematical tools;
	\item The conformity of its execution model with the agent paradigm and service technology.
\end{itemize}
In view of all these benefits, we can say that the objectives set for this thesis have indeed been achieved. However, the proposed model is perfectible. For example, it can be modified to permit agents to respond incrementally to incoming requests as soon as any prefix of the extension of a bud is produced. This makes it possible to avoid the situation observed on figure \ref{chap3:fig:execution-figure-4} where the associated editor is informed of the evolution of the subtree resulting from $C$ only when this one is closed. All the criticisms we can make of the proposed model in particular, and of this thesis in general, have been introduced in the general conclusion (page \pageref{chap5:general-conclusion}) of this manuscript.





%%%%%%%%%%%%%%%%%%%%%%%%%%%%%%%%%%%%%
\bibliographystyle{unsrtnat}
\bibliography{ABESS}
%%%%%%%%%%%%%%%%%%%%%%%%%%%%%%%%%%%%%

\end{document}



