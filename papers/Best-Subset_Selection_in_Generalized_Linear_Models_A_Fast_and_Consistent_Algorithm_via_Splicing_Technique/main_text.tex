\documentclass[12pt]{article}
\renewcommand{\baselinestretch}{1.1}
\usepackage[left=2.7cm,bottom=2.7cm,right=2.7cm,top=2.7cm]{geometry}

%% Packages
\usepackage{enumerate}
\usepackage{tabularx}
\usepackage{subfigure}
\usepackage{multirow}
\usepackage{algorithm}
\usepackage{authblk}
\usepackage{algorithmic}
\usepackage{indentfirst}
\usepackage{setspace}
\usepackage{color}
\usepackage[utf8]{inputenc} % allow utf-8 input
\usepackage[T1]{fontenc}    % use 8-bit T1 fonts
\usepackage{hyperref}       % hyperlinks
\usepackage{url}            % simple URL typesetting
\usepackage{booktabs}       % professional-quality tables
\usepackage{amsfonts}       % blackboard math symbols
\usepackage{nicefrac}       % compact symbols for 1/2, etc.
\usepackage{microtype}      % microtypography
\usepackage{lipsum}		% Can be removed after putting your text content
\usepackage{graphicx}
\usepackage{natbib}
\usepackage{doi}
\usepackage[namelimits]{amsmath} 
\usepackage{amssymb}            
\usepackage{amsfonts}       
\usepackage{mathrsfs} 
\usepackage[resetlabels]{multibib}
\usepackage{bm}

\usepackage{graphicx}
\usepackage{epsfig}
\usepackage{enumitem}
\usepackage{url} % not crucial - just used below for the URL
\usepackage{multirow}
\usepackage{epstopdf}
\usepackage{mathtools}

% Convenient notations
\newcommand{\fullset}{\mathcal{F}}
\newcommand{\informsOR}{1}
\newcommand{\informsMOR}{0}

\graphicspath{{}}

\newtheorem{theorem}{Theorem}
\newtheorem{lemma}{Lemma}
\newtheorem{proof}{Proof}
\newtheorem{corollary}{Corollary}
\newtheorem{remark}{Remark}
\providecommand{\keywords}[1]{\textit{\quad Key words }:  #1}
\renewcommand\Affilfont{\small}
\newcommand{\qedsymbol}{\hfill $\blacksquare$}
\begin{document}

\title{Best-Subset Selection in Generalized Linear Models: A Fast and Consistent Algorithm via Splicing Technique}

% \author{Yanhang Zhang\thanks{School of Statistics,
% Renmin University of China,
% \texttt{zhangyh98@ruc.edu.cn}}\ ~~~~
% Junxian Zhu \thanks{Saw Swee Hock School of Public Health, National University of Singapore,
% \texttt{junxian@nus.edu.sg}}\~~~~
% Jin Zhu \thanks{Southern China Center for Statistical Science,
% Department of Statistical Science,
% School of Mathematics,
% Sun Yat-Sen University,
% \texttt{zhuj37@mail2.sysu.edu.cn} }~~~~
% Xueqin Wang\thanks{
% Department of Statistics and Finance/International Institute of Finance,
% 	School of Management,
% 	University of Science and Technology of China,
% 	\texttt{wangxq20@ustc.edu.cn} }}

\author[1]{Junxian Zhu\textsuperscript{*}}
\author[2]{Jin Zhu\textsuperscript{*}}
\author[4]{Borui Tang}
\author[2]{Xuanyu Chen}
\author[3]{Hongmei Lin\textsuperscript{$\dagger$}}
\author[4]{Xueqin Wang\textsuperscript{$\dagger$}}

\affil[1]{\footnotesize Saw Swee Hock School of Public Health, National University of Singapore}
\affil[2]{\footnotesize Southern China Center for Statistical Science, Department of Statistical Science, School of Mathematics, Sun Yat-Sen University}
\affil[3]{\footnotesize School of Statistics and Information, Shanghai University of International Business and Economics \\
  \texttt{hmlin@suibe.edu.cn}}
\affil[4]{\footnotesize Department of Statistics and Finance/International Institute of Finance, School of Management, University of Science and Technology of China \\
 	\texttt{wangxq20@ustc.edu.cn} }

\date{}
\maketitle \sloppy

\begin{abstract}
  In high-dimensional generalized linear models, it is crucial to identify a sparse model that adequately accounts for response variation. Although the best subset section has been widely regarded as the Holy Grail of problems of this type, achieving either computational efficiency or statistical guarantees is challenging. In this article, we intend to surmount this obstacle by utilizing a fast algorithm to select the best subset with high certainty. We proposed and illustrated an algorithm for best subset recovery in regularity conditions. Under mild conditions, the computational complexity of our algorithm scales polynomially with sample size and dimension. In addition to demonstrating the statistical properties of our method, extensive numerical experiments reveal that it outperforms existing methods for variable selection and coefficient estimation. The runtime analysis shows that our implementation achieves approximately a fourfold speedup compared to popular variable selection toolkits like \textsf{glmnet} and \textsf{ncvreg}.
\end{abstract}


\keywords{Best-Subset Selection, Generalized Linear Models, Splicing Technique, Support Recovery Consistency, Polynomial Complexity}

\begingroup\renewcommand\thefootnote{*}
\footnotetext{Equal contribution}
\begingroup\renewcommand\thefootnote{$\dagger$}
\footnotetext{Corresponding author}

\section{Introduction}  \label{sec:introduction}

\newcommand\inexpIntro[3]{#1?(#2,#3).}
\newcommand\rinexpIntro[3]{*#1?(#2,#3).}
\newcommand\outexpIntro[3]{#1!(#2,#3).}
\newcommand\outatomIntro[3]{#1!(#2,#3)}

We propose a fully automated method for proving termination of \(\pi\)-calculus processes.
Although there have been a lot of studies on termination analysis for the \(\pi\)-calculus
and related calculi~\cite{Deng06IC,Demangeon07,SangiorgiTermination,KobayashiHybrid,Yoshida04IC,DBLP:journals/jlp/DemangeonHS10,Venet98SAS}, most of them have been rather theoretical,
and there have been surprisingly little efforts in developing  fully automated termination
verification methods and tools based on them. To our knowledge,
Kobayashi's \typical{}~\cite{TyPiCal,KobayashiHybrid} is the only exception that
can prove termination of \(\pi\)-calculus processes (extended with natural numbers)
fully automatically, but its termination analysis is quite limited (see Section~\ref{sec:relatedwork}).

Our method is based on a reduction to termination analysis for sequential programs:
we translate a \(\pi\)-calculus process \(P\) to a sequential program \(S_P\), so that
if \(S_P\) is terminating, so is \(P\). The reduction allows us to use
powerful, mature methods and tools
for termination analysis of sequential programs~\cite{heizmann2016ultimate,freqterm,DBLP:conf/lics/PodelskiR04,Kuwahara2014Termination,DBLP:journals/cacm/CookPR11}.

The idea of the translation is to convert a chain of communications on replicated input
channels to a chain of recursive function calls of the target sequential program.
Let us consider the following Fibonacci process:
\begin{align*}
    & \rinexpIntro{\fib}{n}{r}
        \ifexp{n<2}{ \soutatom{r}{1} \\ &\quad}
                   { \nuexp{s_1} \nuexp{s_2} (\outatomIntro{\fib}{n-1}{s_1} \PAR \outatomIntro{\fib}{n-2}{s_2} \PAR \sinexp{s_1}{x}\sinexp{s_2}{y}\soutatom{r}{x+y}) \\}
    & \PAR \outatomIntro{\fib}{m}{r}
\end{align*}
Here, the process
$\rinexpIntro{\fib}{n}{r} \ldots$ is a function server that computes the \(n\)-th Fibonacci number
in parallel and returns the result to \(r\),
and $\outatom{\fib}{m}{r}$ sends a request for computing the \(m\)-th Fibonacci number;
those who are not familiar with the syntax of the \(\pi\)-calculus may wish to consult
Section~\ref{sec:targetlanguage} first.
To prove that the process above is terminating for any integer \(m\),
it suffices to show that there is no infinite chain of communications on $\fib$:
\[
    \fib(m,r) \to \fib(m_1,r_1) \to \fib(m_2,r_2) \to \cdots.
\]
We convert the process above to the following program:\footnote{The actual translation
  given later is a little more complex.}
\begin{verbatim}
 let rec fib(n) = if n<2 then () else (fib(n-1) [] fib(n-2)) in
 fib(m)
\end{verbatim}
Here, \texttt{[]} represents the non-deterministic choice.
Note that, although the calculation of Fibonacci numbers is not preserved,
for each chain of communications on \texttt{fib}, there is a corresponding
sequence of recursive calls:
\[
\mathtt{fib}(m) \to \mathtt{fib}(m_1) \to \mathtt{fib}(m_2) \to \cdots.
\]
Thus, the termination of the sequential program above implies the termination of
the original process.
As shown in the example above, (i) each communication on a replicated input channel
is converted to a function call, (ii) each communication on a non-replicated input
channel is just removed (or, in the actual translation, replaced by a call of
a trivial function defined by \(f(\seq{x})=(\,)\)), and (iii) parallel composition
is replaced by a non-deterministic choice.
We formalize the translation outlined above and prove its correctness.

The basic translation sketched above sometimes loses too much information.
For example, consider the following process:
\begin{align*}
    & \rinexpIntro{\pre}{n}{r} \soutatom{r}{n-1} \\
    & \PAR \rinexpIntro{f}{n}{r} \ifexp{n<0}{ \soutatom{r}{1} }
                                       { \nuexp{s} (\outatomIntro{\pre}{n}{s} \PAR \sinexp{s}{x}\outatomIntro{f}{x}{r}) } \\
    & \PAR \outatomIntro{f}{m}{r}
\end{align*}
The translation sketched above would yield:
\begin{verbatim}
  let pred(n) = n-1 in
  let rec f(n) = if n<0 then () else (pred(n) [] f(*)) in
  f(m)
\end{verbatim}
Here, \texttt{*} represents a non-deterministic integer: since we have removed
the input $\sinatom{s}{x}$, we do not have information about the value of \( x \).
As a result, the sequential program above is non-terminating, although the original
process is terminating.
To remedy this problem, we also refine the basic translation above by using a refinement
type system for the \(\pi\)-calculus. Using the refinement type system,
we can infer that the value of \(x\) in the original process is less than \(n\),
so that we can refine the definition of \texttt{f} to:
\begin{verbatim}
 let rec f(n) = ... else (pred(n) [] let x=* in assume(x<n);f(x))
\end{verbatim}
The target program is now terminating, from which
we can deduce that the original process is also terminating.
We have implemented an automated tool based on the refined translation above.

The contributions of this paper are summarized as follows.
\begin{itemize}
\item The formalization of the basic translation from the \(\pi\)-calculus
  (extended with integers) to sequential programs, and a proof of its correctness.
\item The formalization of a refined translation based on a refinement type system.
\item An implementation of the refined translation, including automated refinement type
  inference based on CHC solving, and experiments to evaluate the effectiveness of
  our method.
\end{itemize}

The rest of this paper is structured as follows.
Section~\ref{sec:targetlanguage} introduces the source and target languages
of our translation.
Section~\ref{sec:approach} 
formalizes the basic translation, and proves its correctness.
Section~\ref{sec:refinement} refines the basic translation by using a refinement type system.
Section~\ref{sec:implementation} reports an implementation and experiments.
Section~\ref{sec:relatedwork} discusses related work,
and Section~\ref{sec:conclusion} concludes the paper.

The proposed segmentation-by-detection framework, as depicted in Figure \ref{fig:framework}, consists of a detection module and a segmentation module.
In detection stage, 2D slices (layered box) from the input volume are fed to the RPN. Based on the region proposals obtained from RPN, an attention model (block in orange) is formed. The input volume as well as the attention model are further processed in segmentation stage to get the refined anatomical segmentation. 
\vspace{1em} 

\begin{figure}[t]
\centering
\includegraphics[width=0.95\linewidth]{fig/framework.pdf}
\caption{Schematic representation of the segmentation-by-detection framework. The left part is the detection module while the segmentation module is followed on the right. The blue block denotes the input volume which is 3D ultrasound scan of femoral head. The output segmentation is in red.}
\label{fig:framework}
\end{figure}
% dana could you improve the figure. we can try to think together of better ways 

\noindent\textbf{Detection Module:} 
% dana : here you have to make the clarification that you have ground truth on the boxes (in implementation part)
The detection module follows an RPN architecture, a fully convolutional network which takes image slice as input and outputs object region candidates. 
We use the VGG-16 model as the backbone \cite{simonyan2014very} to learn convolutional features and an $3 \times 3$ spatial window to generate region proposals. At each sliding-window location, 9 anchors are predicted associated with different scales and aspect ratios. The last layer consists of a box-regression (reg) layer and a box-classification (cls) layer in parallel. The reg layer outputs 4 regression offsets, $ t = (t_x,t_y,t_w,t_h)$, denoting a scale-invariant translation as well as log-space height and width shift, where $x,y,w$ and $h$ specify two coordinates of the box center, width and height. The cls layer outputs two scores by softmax, related to probabilities of object and background for each proposal. We assign a positive label (of being object) to candidate which has an Intersection-over-Union (IoU) ratio higher than 0.7 with ground truth box. Note that an image slice may contain multiple object regions or none. 

The loss function of RPN follows the multi-task loss \cite{ren2015faster} which is defined as $L = L_{reg} + L_{cls}$. The regression loss, $L_{reg} = -\log p_{obj}$ is log loss and the classification loss,
\begin{equation} \label{eq:loss}
L_{cls} = \sum_{i \in \{x,y,w,h\}} smooth_{L_1} (t_i - t_i^*)
\end{equation}
is smooth $L_1$ loss where $t_i^*$ denotes the ground truth box for the target object. 
\vspace{1em}

\noindent\textbf{Segmentation Module:}
3D U-Net \cite{cciccek20163d} is utilized in the segmentation module as its outstanding performance in medical image segmentation. The u-shaped architecture consists of two paths: a contracting path, where each layer contains two $3\times3\times3$ convolutions followed by a rectified linear unit (ReLU) and then a max pooling, provides high resolution features. While, the symmetric expanding path for semantically richer features replaces max pooling with a upconvolution $2\times2\times2$ with stride of 2 in each dimension, and then two $3\times3\times3$ convolutions each followed by a ReLU. Skip connections between layers of equal resolution in the contracting path and the expanding path enables context information as well as precise localization.

Different from 3D U-Net, to incorporate the attention model detected by the RPN, our architecture takes as input both the volumetric image data and the candidate RoIs proposed by the RPN, concatenated as 3D volume. 
% dana not sure what you like to say below
% densely annotated
The attention model makes the network to focus on the potential RoIs and can reduce the interference of the surrounding noise.
The anatomical segmentation is then generated from a $1\times1\times1$ convolution which reduces the number of feature maps to the number of labels.  The energy function is computed by a pixel-wise softmax combined with the cross entropy loss.
% dana equation ??

\subsection{System and implementation Details}
The segmentation-by-detection approach adopts a cascade structure with two stages: detection and segmentation. The two networks are trained separately in an end-to-end manner. All the new layers are randomly initialized from zero-mean Gaussian distribution with standard deviations 0.01. Biases are initialized to 0. We use Caffe \cite{jia2014caffe} for the implementation and an NVIDIA Titan X GPU for training.

In the detection stage, we initialize the VGG-16 model by the pre-trained model for ImageNet classification \cite{russakovsky2015imagenet} and further fine-tune the model for our detection task. The input fed to the network are image slices with a fixed size of $184\times96$ and the corresponding ground truth boxes are generated from the annotation in the format of tight bounding boxes surrounding the segmentation contour (as illustrated in Figure \ref{fig:hip} (b), the boundary of white area). To optimize the energy function, stochastic gradient descent (SGD) is used. The global learning rate is set to 0.001, while a momentum of 0.9 and a weight decay of 0.0005 are used. The batch size is set to 256 and each mini-batch only contains the positive anchors for training. The region proposals are obtained from the reg path for each image slice. The attention model is then formed by concatenating all the detected regions, as binary masks, into a volume.

In the segmentation stage, we use the Adam optimizer \cite{kingma2014adam} to learn the network parameters. A global learning rate is set to 0.001 while the two momentum coefficients are set to 0.9 and 0.999 respectively. A batch size of 1 is used due to the memory constraints of the GPU. The network takes the volume data as well as the attention model as input. We train the network for a maximum of 30K iterations and reserve the learned weights with the best performance from every 1K iterations. 
\vspace{1em}

\noindent\textbf{Inference:}
At test time, the 2D slices from an input volume are first fed to the detection module. The attention model is obtained based on the output. Then the volume data as well as the attention model are fed to the segmentation module to get the pixel-wise prediction.



\section{Theory}
In this section, we give guarantees on our grid-based approach. Suppose there is some underlying distribution $\mathcal{P}$ with corresponding density function $p : \mathbb{R}^d \rightarrow \mathbb{R}_{\ge 0}$ from which our data points $X_{[n]} = \{x_1,...,x_n\}$ are drawn i.i.d. We show guarantees on the density estimator based on the grid cell counts.

We need the following regularity assumptions on the density function. The first ensures that the density function has compact support with smooth boundaries and is lower bounded by some positive quantity, and the other ensures that the density function has smoothness. These are standard assumptions in analyses on density estimation e.g. \cite{gine2002rates,jiang2017uniform,chen2017tutorial,singh2009adaptive}.
\begin{assumption}\label{assumption1}
$p$ has compact support $\mathcal{X} \in \mathbb{R}^d$ and there exists $\lambda_0, r_0, C_0 > 0$ such that $p(x) \ge \lambda_0$ for all $x \in \mathcal{X}$ and $\text{Vol}(B(x, r) \cap \mathcal{X}) \ge C_0 \cdot \text{Vol}(B(x, r))$ for all $x \in \mathcal{X}$ and $0 < r \le r_0$, where $B(x, r) := \{x' \in \mathbb{R}^d: |x-x'| \le r\}$.
\end{assumption}
\begin{assumption}\label{assumption2}
$p$ is $\alpha$-Hölder continuous for some $0 < \alpha \le 1$: i.e. there exists $C_\alpha > 0$ such that $|p(x) - p(x')| \le C_\alpha \cdot |x - x'|^\alpha$ for all $x, x' \in \mathbb{R}^d$.
\end{assumption}

We now give the result, which says that for $h$ sufficiently small depending on $p$ (if $h$ is too large, then the grid is too coarse to learn a statistically consistent density estimator), and $n$ sufficiently large, there will be a high probability finite-sample uniform bound on the difference between the density estimator and the true density. The proof can be found in the Appendix.
\begin{theorem}\label{theorem}
Suppose Assumption~\ref{assumption1} and~\ref{assumption2} hold. Then there exists constants $C, C_{1} > 0$ depending on $p$ such that the following holds.
Let $0 < \delta < 1$, $0 < h < \text{min}\{\left(\frac{\lambda_0}{2\cdot C_\alpha}\right)^{1/\alpha}, r_0\}$, $nh^d \ge C_1$. Let $\mathcal{G}_h$ be a partitioning of $\mathbb{R}^d$ into grid cells of edge-length $h$ and for $x \in \mathbb{R}^d$. Let $G(x)$ denote the cell in $\mathcal{G}_h$ that $x$ belongs to.  Then, define the corresponding density estimator $\widehat{p}_h$ as:
\begin{align*}
    \widehat{p}_h(x) := \frac{|X_{[n]} \cap G(x)|}{n\cdot h^d}.
\end{align*}
Then, with probability at least $1 - \delta$:
\begin{align*}
    \sup_{x \in \mathbb{R}^d} |\widehat{p}_h(x)  - p(x)| \le C\cdot \left( h^\alpha + \frac{\sqrt{\log(1/(h\delta)}}{\sqrt{n\cdot h^d}} \right).
\end{align*}
\end{theorem}


\begin{remark}
In the above result, choosing $h \approx n^{-1/(2\alpha+d)}$ optimizes the convergence rate to $\tilde{O}(n^{-\alpha/(2\alpha+d)})$, which is the minimax optimal convergence up to logarithmic factors for the density estimation problem as established by Tsybakov \cite{tsybakov1997nonparametric,tsybakov2008introduction}.
\end{remark}
In other words, the grid-based approach statistically performs at least as well as any estimator of the density function, including the density estimator used by MeanShift. It is worth noting that while our results only provide results for the density estimation portion of MeanShift++ (i.e. the grid-cell binning technique), we prove the near-minimax optimality of this estimation. This implies that the information contained in the density estimation portion serves as an approximately sufficient statistic for the rest of the procedure, which behaves similarly to MeanShift, which operates on another, also nearly-optimal density estimator. Thus, existing analyses of MeanShift e.g. \cite{arias2016estimation,chen2015convergence,xiang2005convergence,li2007note,ghassabeh2015sufficient,ghassabeh2013convergence,subbarao2009nonlinear} can be adapted here; however, it is known that MeanShift is very difficult to analyze \cite{dasgupta2014optimal} and a complete analysis is beyond the scope of this paper.

\section{Efficient Implementation: Details}\label{sec:efficient-implementation}
We implement the proposed algorithm in a highly efficient \textsf{abess} library
with both Python and R interfaces.
As can be seen from the numerical experiments below,
\textsf{abess} can have competitive or even less running times
compared with other well-known sparse learning software like \textsf{glmnet}.
\textsf{abess} achieves extreme efficiency by leveraging efficient implementation to reduce the time consumption on computing forward and backward sacrifices.
% These include an adaptive grid of tuning parameters, continuation, active set updates, greedy cyclic ordering of
% coordinates, correlation screening, and a careful accounting of floating point operations—some of these
% heuristics (as specified below) appear in prior work
% for deriving highly efficient algorithms for the Lasso (e.g., glmnet).
In the following, we provide a detailed description of the efficient implementation.

%\subsection{Best-Subset Selection Path: Warm-Start Initialization}
%In this part, we describe an efficient procedure to compute
%along a best-subset selection path, i.e., computing the solution of Algorithm~\ref{alg:fbess}
%on a series of support size: $1 < 2 < \cdots < s_{\max}$.
%The procedure is summarized in Algorithm~\ref{alg:abess}.
%% Inspired by the work of \citet{friedman2010regularization},
%% we use the $s_i$ solution as a warm start for $s_{i+1}$.
%Algorithm~\ref{alg:abess} runs Algorithm~\ref{alg:fbess} on various support sizes ranging from small to large.
%% Fortunately, it allows using the output of Algorithm~\ref{alg:fbess} to build a better active set,
%% allowing Algorithm~\ref{alg:fbess} to reach convergence with less splicing iteration.
%Notice that, in Step 4 of Algorithm~\ref{alg:abess},
%we set the initial support set of the next support size as the union of
%the selected subset returned by Algorithm~\ref{alg:fbess} and the variable with the largest forward sacrifice,
%which refers to ``warm-starts initialization''.
%Both existing numerical experiments and theoretical analysis suggest that
%such an initialization is remarkably efficient \citep{friedman2010regularization, barutConditionalSureIndependence2016}.
%From our experience, by setting the initialization as the input of Algorithm~\ref{alg:abess},
%we can reduce the number of iterations in the \textbf{repeat-until} loop in Algorithm~\ref{alg:fbess},
%whittling down the overall runtime.
% \footnote[1]{The adaptivity refers to adapting the warm-start initialization.}
% Algorithm~\ref{alg:abess} describes the procedure mentioned above in detail.
%\begin{algorithm}[htbp]
% \caption{P\underline{a}thwise \underline{Be}st-\underline{S}ubset \underline{S}election for GLM (\textbf{ABESS})}
% \label{alg:abess}
% \begin{algorithmic}[1]
% \REQUIRE A dataset set $\{(\boldsymbol{x}_i, y_i)\}^{n}_{i=1}$ and the maximum support size $s_{\max}$.
% \STATE $\mathcal{A}^0 \leftarrow \{ \arg\max\limits_j | (\frac{\partial l_n( \boldsymbol \beta )}{\partial \boldsymbol \beta} |_{\boldsymbol \beta = {\boldsymbol 0} })_j | \}$
% \FOR {$s = 1, \ldots, s_{\max}$}
% \STATE $(\hat{\boldsymbol \beta}_{s}, \hat{\mathcal{A}}_{s}, \hat{\boldsymbol{d}}_s) \leftarrow \textbf{BESS-GLM}\left( \{(\boldsymbol{x}_i, y_i)\}^{n}_{i=1}, \mathcal{A}^0, k_{\max}, \tau_s \right)$.
% \STATE $\mathcal{A}^0 \leftarrow \hat{\mathcal{A}}_{s} \cup \{ \arg\max\limits_j|(\hat{\boldsymbol{d}}_s)_j| \}$.
% \ENDFOR
% \ENSURE $\{ (\hat{\boldsymbol \beta}_{s_i}, \hat{\mathcal{A}}_{s_i} ) \}_{i=1}^{\max}$.
% \end{algorithmic}
%\end{algorithm}

\subsection{Simplified Convex Optimization and Early Stopping}\label{sec:approximiated-newton-update}

In Algorithm~\ref{alg:fbess}, we need to solve convex optimization problems:
$\tilde{\boldsymbol \beta} \leftarrow \arg\min\limits_{\beta_{\mathcal{I}} = 0} l_n(\boldsymbol\beta ).$
Directly solving it via a convex optimization solver would consume a large time.
To leverage the sparsity nature of this problem, we can turn to solve a simplified problem:
$$\tilde{\boldsymbol \beta}_{\mathcal{A}} \leftarrow \arg\min\limits_{\boldsymbol{\beta}_{\mathcal{A}}}- \sum_{i=1}^n\{y_i \boldsymbol{\beta}_{\mathcal{A}}^\top (\boldsymbol{x}_i)_{\mathcal{A}} - b(\boldsymbol{\beta}_{\mathcal{A}}^\top (\boldsymbol{x}_i)_{\mathcal{A}}) + c(y_i,\phi)\} $$
and pad zero entries to obtain $\tilde{\boldsymbol \beta}$.
The simplified problem solves the regression coefficient on a much smaller dataset, and thus it is computationally appealing.
Since the simplified problem has no closed-form solution, we must perform iterative algorithms to solve it.
The Newton method is one of the most popular methods to solve this problem.
More precisely, we conduct Newton's updates
% \begin{equation}\label{eqn:primary_fit_update}
% \begin{split}
% {\boldsymbol \beta}_{\mathcal{A}}^{m+1} \leftarrow \boldsymbol \beta_{\mathcal{A}}^m - \Big( \left.\frac{\partial^2 l_n( \boldsymbol \beta )}{ (\partial \boldsymbol \beta_{{\mathcal{A}}} )^2 }\right|_{\boldsymbol \beta = \boldsymbol \beta^m} \Big)^{-1} \Big( \left.\frac{\partial l_n( \boldsymbol \beta )}{ \partial \boldsymbol \beta_{{\mathcal{A}}} }\right|_{\boldsymbol \beta = \boldsymbol \beta^m} \Big),
% \end{split}
% \end{equation}
until $\| {\boldsymbol \beta}_{\mathcal{A}}^{m+1} - {\boldsymbol \beta}_{\mathcal{A}}^{m}\|_2 \leq \epsilon$ or $m > k$,
where $\epsilon$ is convergence tolerance and $k$ is the maximal number of Newton's updates.
Generally, setting $\epsilon = 10^{-6}$ and $k = 80$ returns a desirable coefficient estimation.
% In general, the inverse of the second derivative in \eqref{eqn:primary_fit_update} is computationally expensive, so we use its diagonalized version to approximate it. The update then makes the following changes:
% \begin{equation}\label{eqn:fast_primary_fit_update}
% \begin{split}
% {\boldsymbol \beta}_{\tilde{\mathcal{A}} }^{m+1} \leftarrow \boldsymbol \beta_{\tilde{\mathcal{A}} }^m - \rho D \Big( \left.\frac{\partial l_n( \boldsymbol \beta )}{ \partial \boldsymbol \beta_{\tilde{\mathcal{A}}} }\right|_{\boldsymbol \beta = \boldsymbol \beta^m} \Big),
% \end{split}
% \end{equation}
% where $D = \textup{diag}( (\left.\frac{\partial^2 l_n( \boldsymbol \beta )}{ (\partial \boldsymbol \beta_{\tilde{\mathcal{A}_{1}}} )^2 }\right|_{\boldsymbol \beta = \boldsymbol \beta^m} )^{-1}, \ldots, (\left.\frac{\partial^2 l_n( \boldsymbol \beta )}{ (\partial \boldsymbol \beta_{\tilde{\mathcal{A}}_{|A|}} )^2 }\right|_{\boldsymbol \beta = \boldsymbol \beta^m} )^{-1})$
% and $\rho$ is step size.
% While using the approximation increases iteration time, it avoids a high level of computational complexity when computing the matrix inversion.

Notice that not all of the candidate active sets considered in Algorithm~\ref{alg:fbess} associate with a better solution such that $L - l_{n}(\tilde{\boldsymbol{\beta}}) > \tau_s$,
This inspires us to stop Newton's updates on these active candidates early sets to reduce unnecessary Newton's updates.
More precise, after each Newton's update, we check whether the heuristic criterion
$l_1 - (k - m) \times (l_2 - l_1) > L - \tau_s$ holds,
where $l_1 = l_n({\boldsymbol\beta}_{\mathcal{A}}^{m}), l_2 = l_n({\boldsymbol\beta}_{\mathcal{A}}^{m+1})$.
If so, it implies the remaining $k - m$ times Newton update can potentially lead to a lower loss, and thus,
it has no reason to stop Newton's updates; otherwise, we can stop Newton's updates.
The heuristic criterion is designed based on the fact that the convergence rate of Newton's iterative method is quadratic \citep{nocedalNumericalOptimization1999}.
It considerably improves the efficiency of computing forward sacrifices and
retain the high quality of coefficient estimations.

\subsection{Importance-Priority Splicing}\label{sec:importance-priority-splicing}
Considerable speedup is achieved in a high-dimensional regime by employing an importance-priority splicing procedure,
which is motivated by an active set update for the Lasso \citep{lassoscreening2012tibshirani}.
Specifically, a heavy computational burden comes up when
computing forward sacrifices for $p - s$ variables in the inactive set, especially since $p$ is very large.
To alleviate this burden, after completing forward and backward sacrifices through all the variables, we perform the splicing iteration on a small but essential subset: the union of
the selected variables and $d$ ($\ll p$) variables with the more significant forward sacrifices.
Once the splicing iteration converges, we re-compute sacrifices for all the variables and update the important subset.
If the updated important subset is the same as the previous one,
Algorithm~\ref{alg:fbess} is done. Otherwise, the above procedure is repeated.
From our numerical experience, the procedure mentioned above can save lots of time;
meantime, it keeps Algorithm~\ref{alg:fbess} enjoying the statistical properties
presented in Section~\ref{sec:theorical-properties}.

\section{Simulation}

\subsection{motivation} %
The most efficient way to figure out the answers to the questions we posed in the introduction is to deploy the proposed framework on a real-world platform and analyze how users adopt different and complex privacy policies to optimize their rewards.
However, direct deployment of these strategies and investments is currently impractical due to the following reasons.


Firstly, the most important reason is that such an online experiment may lead to the decline of the recommendation performances as well as the user experience, which harms the platform's revenue.
In the real world, nearly all the companies determine their platform mechanism driven by interest, and the revenues of the platforms are highly correlated with the recommendation performances. 
Therefore, it's nearly impossible to persuade any platform to directly deploy proposed strategies and mechanism online without other benefits.  



Secondly, the experiments are \czq{built} upon several simplifications, mentioned in \cref{assumptions}, which poses challenges towards recommendation model training process.
For example, we assume when a user \czq{adjusts} his data disclosure policy, the recommendation system will forget his un-disclosed data. 
To facilitate such challenges, model unlearning or other privacy-preserving technologies are imposed.
However, in real-world applications, very few the e-commercial platforms have deployed these privacy-preserving technologies during the deep recommendation model training and evaluation processing.
As a result, we may still fail to guarantee the assumptions and simulation methodology becomes a substitution.










In summary, inspired by the success of simulation study on dynamic interactive problems in real-world applications~\cite{Ie:arxiv19:RecSim,krauth2020offline,lucherini2021t,yao2021measuring},  we employ the simulation to study the effects of the proposed framework and the possible game between users and the platform.















\subsection{Simplified Assumptions}
\label{assumptions}

To simplify the simulation process for easier analysis, we make some necessary assumptions to simplify the problem.

\begin{assumption}[Static Assumption] User $i$ optimizes her/his policy on the fixed data $\di{}$ which is not affected by user policy $\pi_i$.
\label{assumption:static}
\end{assumption}

Here static means the user data $\di{}$ is fixed during the simulation, but the disclosed data $\si{}$ produced by different user policies is dynamic. 
It is also the most common setting for recommendation task in research papers~\cite{Rendle:www10:Factorizing,Hidasi:ICLR2016:gru4rec,NCF,kang2018self,Sun:cikm19:BERT4Rec}.
In the simulation, we train the recommendation system $\texttt{M}_{{\scriptscriptstyle \mathcal{S}}}$ on the collected dynamic data $\mathcal{S}$ and validate the recommendation efficiency on a fixed test set. 
In real-world applications, the data $\di{}$, which contains the behavior data from the interaction with the recommender $\texttt{M}_{{\scriptscriptstyle \mathcal{S}}}$, is also dynamically changing with the user's policy $\pi_i$.
It is beyond the scope of this paper and we leave it as the future work. 


\begin{assumption}[Immediate Assumption] The recommendation model $\texttt{M}_{{\scriptscriptstyle \mathcal{S}}}$ can only use the data $\si{}$ currently disclosed by each user $i$.
\label{assumption:forget}
\end{assumption}
The motivation of this assumption is that an untrusted platform can leverage user $i$' all data $\di{}$ if it can use the data disclosed in previous actions.
Without this constraint, the privacy right discussed in this paper is meaningless.
To achieve this, the platform can retrain the model from scratch with new data $\si{'}$ or quick unlearn the data in $\si{}$ then finetune with data $\si{'}$~\cite{cao2015towards,bourtoule2021machine,chen2022recommendation}.


However, the \cref{assumption:forget} also raises a new challenge that the asynchronous changes of user policy bring intractable computation costs for the platform since each time the user changes the disclosed data, the platform needs to update the model.
Here, we make an assumption for simplifying the simulation, assuming all users realize that the platform will cyclically (e.g., once a day) collect their privacy decisions and update recommender systems.
\begin{assumption}
[%
Cyclical Assumption]
Platform cyclically collects user privacy choices, and then the platform updates the model using all newly disclosed data. 
\label{assumption:synchronization}
\end{assumption}



In summary, for easy analysis in simulations, we introduce these assumptions to ignore the time and dynamic effects in this feedback system, just like the traditional recommendation task formulation.
























\subsection{Platform Mechanism Simulation}
\label{sec:plat_mech}
In order to validate the effect of the platform mechanism, we adopt several mechanisms during simulation. 
For easy comparison, we utilize one mechanism at each experiment. %


\subsubsection{\textbf{Data Split Rule}}

\czq{In our simulation, we do not split the profile attribution and the user can determine whether to share all of their attributes.}
For behavior data, we apply ``percentage split'' as $\delta_b$ with different split granularity $p$ (e.g., 1/3) to split the behavior sequence into $1/p$ parts. 
One obvious advantage of ``percentage split'' is that it can normalize the size of user action space and decrease the inconvenience of the interaction between the user and the platform.

\subsubsection{\textbf{Data Disclosure Strategy}}
\label{sec:data_disclose_choice}

As the platforms have certain flexibility to implement different data disclosure strategies, we discuss three representative disclosure strategies used in our study for behavior data in this subsection.
These strategies determine the data disclosure action space $\Pi$ the user can choose.
For profile attributes, we found that all users tend not to disclose them in the experiments since these features are negligible for improving recommendation utility in the presence of behavior data.
Similar result that user profile features contribute very marginal to the recommendation results in the case of strong user behavior modeling on public benchmark datasets has also been reported in other works~\cite{kang2018self,Sun:cikm19:BERT4Rec}.
Thus, in the following study, we mainly focus on modeling only  behavior data.

The ``\textit{separate}'' rule gives the users the control to freely disclose any split personal data.
For this rule, the size of user $i$'s the action space is exponentially expended on the size of the spilt data set $|\delta_b ({\scriptstyle \mathcal{D}_{i,b}})|$, denoted as $2^{|\delta_b ({\scriptstyle \mathcal{D}_{i,b}})|}$. 
However, too many choices might make it difficult for users to make better data disclosure decisions.

Another data disclosure strategy named ``\textit{oldest continuous}'' provides users the choices to disclose continuous behavior data from the beginning time, such as selecting ``the oldest 33\% data''.
In this strategy, to disclose newer behavior data ${\scriptstyle \mathcal{S}_{i,bj}}$, users must also disclose all behavior data before it.
Take an already split data $\delta_b ({\scriptstyle \mathcal{D}_{i,b}}) = \{\scriptstyle \mathcal{S}_{i,b1}, \scriptstyle \mathcal{S}_{i,b2}, \scriptstyle \mathcal{S}_{i,b3}\}$ as an example, the action space provided by oldest continuous strategy is $\Pi = \{[0,0,0], [1,0,0], [1,1,0], [1,1,1]\}$, and its corresponding disclosed data is $\{\varnothing ,
    \{\! {\scriptstyle \mathcal{S}_{i,b1}} \!\},
    \{\! {\scriptstyle \mathcal{S}_{i,b1}} , {\scriptstyle \mathcal{S}_{i,b2}}\! \},$
    $\{ {\scriptstyle \mathcal{S}_{i,b1}}, {\!\scriptstyle \mathcal{S}_{i,b2}},$ ${\scriptstyle \mathcal{S}_{i,b3}}\} \}$.
``\textit{Latest continuous}'' mechanism is similar to ``oldest continuous'', with the only difference in the opposite direction.
The size of these two mechanisms' action spaces is $|\delta_b ({\scriptstyle \mathcal{D}_{i,b}})|$.









\subsection{User Policy Simulation}
\label{sec:user}


In this subsection, we introduce the simulation of user policy in our proposed framework.
As defined in \cref{eq:S_i}, the disclosed data $\si{}$ is result of the platform mechanism $\mathrm{G}$ and user's disclosure policy $\pi_i$. 
Meanwhile, in \cref{eq:updated_rec}, the recommendation utility $\texttt{U}_i( \si{})=\texttt{U}'(\si{},\, \texttt{M}_{{\scriptscriptstyle \mathcal{S}}})$ is also determined by the recommendation model $\texttt{M}_{{\scriptscriptstyle \mathcal{S}}}$, which is \czq{built} upon the all users' disclosed data $\mathcal{S}$. 
The reward of user $i$ may be varied even when $i$ keeps the disclosed data $\si{}$ unchanged since other users might change their disclosed data and the recommender system is changed.
Thus, the expectation rewards are considered rather than immediate value defined in \cref{eq:framework} and we assume all the users are rational and seek for the optimal privacy disclosure action  $\alpha_i^*$ to the optimal expected reward $E[ \texttt{R}_i | \alpha_i ] $ as his policy, i.e., %
\begin{equation}
    \begin{aligned}
    \alpha_i^{*} &= \argmax_{\alpha_i \in \Pi} E[ \texttt{R}_i | \alpha_i ]=  \argmax_{\si{\in [ \Pi \otimes \mathrm{\delta}(\di{}) ] }} E[ \texttt{R}_i(\si{)} ] \\
& =\argmax_{\alpha_i \in \Pi } E\Bigl[ -\lambda_i \texttt{C}_i\bigl( \alpha_i \otimes \mathrm{\delta}(\di{}) \bigr) + \texttt{U}_i\bigl( \alpha_i \otimes \mathrm{\delta}(\di{}) )\bigr) \Bigr].
    \end{aligned}
\label{eq:opt_pi}
\end{equation}

As mentioned before, recommendation utility $\texttt{U}_i$ has been discussed in \cref{sec:platform_obj}.
To study this objective, we need to define the privacy cost function $\texttt{C}_i$ and sensitive weight $\lambda_i$.



\subsubsection{\textbf{Privacy Cost Function}}
\label{sec:privacy_cost}
We simulate every user with the same cost function $\texttt{C}$ and leave the diversity of user privacy sensitivity to the parameter $\lambda_i$. 
Following current experiment specifications in the economics literature~\cite{lin2019valuing,tang2019value}, we model the privacy cost function as a linear summation\footnote{See the Eq. 2 in \cite{lin2019valuing} and the dis-utility from disclosure in the econometric specification session in \cite{tang2019value}.} of disclosed personal data.


\czq{According to the comprehensive survey on privacy value definition \cite{MKT-053}, people will measure the value of their privacy into the intrinsic value of privacy and the instrumental value of privacy.}
\czq{
The intrinsic loss indicates the sake of protecting their intrinsic private data, which measures the valuation on the intrinsic properties such as the education or the income levels. }
\czq{In this work, we denote the intrinsic loss towards the privacy cost on amount of the sharing user profile attributes.}
\czq{The instrumental value of privacy indicates how the transaction efficiency would be affected by sharing user data, especially the data generated in the applications. 
In this work we denote the privacy cost towards the percentage of shard user historical behavior data. 
Therefore, the privacy cost function is described below,}
\begin{equation}
    \texttt{C}_i(\si{})= \beta_i * | {\scriptstyle \mathcal{S}_{i,a}} | + \frac{| {\scriptstyle \mathcal{S}_{i,b}} |}{ | {\scriptstyle \mathcal{D}_{i,b}} |}
    \label{eq:cost_function0}
\end{equation}
\czq{where the first term indicates the intrinsic loss and the second term indicates the instrumental loss.} 
\czq{If user does not tend to disclose profile attribute, such privacy cost function can be simplified to the following format with the instrumental value alone.}
As mentioned in \cref{sec:plat_mech}, user tends not to disclose profile attributes $\scriptstyle \mathcal{D}_{i,a}$ due to no gains in our experiments, so we only consider behavior data here, i.e.,
\begin{equation}
    \texttt{C}_i(\si{})=\texttt{C}(\si{}) =  \frac{| {\scriptstyle \mathcal{S}_{i,b}} |}{ | {\scriptstyle \mathcal{D}_{i,b}} |}
    , %
    \label{eq:cost_function}
\end{equation}
where the $|x|$ is the number of elements in $x$.
Here, the percentage based measurement regards different amount of users' data equally. %


This reduced form specification is not unrealistic as it captures the substitution effect among personal data and incorporates the idea of constant marginal privacy cost. 
One might argue for a higher order functional to capture richer implications. 
However, there is little experimental evidence that the higher order form for privacy cost exists and how the functional form looks like.






\subsubsection{\textbf{Privacy Sensitive Weight}}
\label{sec:user_type}
For user $i$ who disclosed all her/his data (i.e., $\si{} = \di{}$), her/his privacy cost compared to not sharing any data (i.e., $\si{} = \varnothing $) is 
\begin{equation}
    \texttt{C}(\di{}) - \texttt{C}(\varnothing).
\label{eq:privacy_diff}
\end{equation}
Meanwhile, her/his anticipated recommendation utility compared to not sharing any data is:
\begin{equation}
    \texttt{U}(\di{}) - \texttt{U}(\varnothing).
\label{eq:utility_diff}
\end{equation}
We assume all users have accessed to the recommendation utility $\texttt{U}(\di{})=\texttt{U}'(\di{},\texttt{M}_{{\scriptscriptstyle \mathcal{D}}})$ computed on all the data $\di{}$ and the recommendation utility without their data $\texttt{U}(\varnothing)$ before they can take data disclosing actions, which can be regard as a prior knowledge, like the experiences before the platform adopted our framework.
With \cref{eq:privacy_diff} and \cref{eq:utility_diff}, we define the privacy sensitive weight $\lambda_i$ as: 
\begin{equation}
    \lambda_i = w_i  * \frac{\texttt{U}(\di{})  - \texttt{U}(\varnothing) } {  \texttt{C}(\di{}) - \texttt{C}(\varnothing) },
    \label{marginal_define}
\end{equation}
where $w_i$ indicates the diversity of user types towards privacy sensitivity.
The users with $w_i > 1$ is privacy sensitive users, as they will not be willing to disclose the corresponding data $\di{}$ if they only get $\texttt{U}(\di{})$ as before.
While users  with $w_i < 1$ are just the opposite. %
Therefore, the user's privacy sensitive weight is pre-computed, and the $\texttt{U}(\di{})$ can be regarded as the benchmark expectation of the platform.
The formulation of the privacy sensitive weight $\lambda_i$ also meets the idea from \cite{lin2019valuing}, where the heterogeneity from users' social demographic variety should also be explicitly characterized. %



\subsubsection{\textbf{Simulation Algorithm}}

As users behave rationally to find the optimal strategy with a trade-off of exploration and exploitation, it just meets the idea of the reinforcement learning algorithm. 
Therefore, we model each user as a unique agent and apply a multi-agent reinforcement learning method to simulate user possible policy adaptation. 
The recommender system is regarded as the environment to provide feedback, which is built upon the disclosed user data.
All agents' policies are optimized simultaneously by determining their actions, i.e., the disclosed data $\scriptstyle \mathcal{S}^t$ at simulation epoch $t$, which is used to train the recommendation model $\texttt{M}_{{\scriptscriptstyle \mathcal{S}}^{t}}$.
As mentioned before, users tend to find an optimal action over possible action space $\Pi$ to maximize his expected reward, which is determined by all agents in this dynamic MARL environment. 


We assume each user (agent) realizes this situation that the immediate reward is the result of all agents, but no communication or observation among agents is permitted. 
Then, each agent is concerned about her/his own utility and regards the environment as a dynamic system that is partially correlated to herself/himself. 
Now, it is simplified to a Multi-Armed Bandit problem~\cite{katehakis1987multi}.


However, the challenge of the exploration and explication problem also exists in our simulation. 
To address it, we adopt a simple but efficient method, Epsilon Greedy~\cite{sutton2018reinforcement} algorithm, to simulate user's policy $\pi_i$ as following, %
\begin{equation}
     \alpha_i^{t+1} = \left\{
\begin{array}{l l }
\alpha_i \sim \texttt{P}^t_i,
& \text{with possibility } \epsilon \\
\argmax_{\alpha_i} Q_i^t(\alpha_i), & \text{with possibility }  1-\epsilon \\
\end{array} \right. 
    \label{epsilon_greedy}
\end{equation}
where $Q_i^t(\alpha)$ is the user $i$'s estimation value at simulation epoch $t$ on action $\alpha$, and $\texttt{P}^t_i$ denotes a random sample policy.
To conduct an efficient policy exploration, we sample a less explored action with a higher possibility as following,
\begin{equation}
    \texttt{P}^t_i(\alpha) = \frac{ 1/ (N^{t-1}_i(\alpha) +1) } { \sum_{x \in \Pi} 1/(N^{t-1}_i(x) +1) },
    \label{random_rule}
\end{equation}
where $N^{t-1}_i(\alpha)$ represents the total number of action $\alpha$ was taken by user $i$ from start to the last simulation epoch $t{-}1$.
In convenience, we adopt the approximated expected estimation results and 
update it with the residual between the estimation $Q_i^{t-1}(\alpha_i^{t-1})$ and immediate reward  $\texttt{R}_i^{t-1}$ when she/he takes action $\alpha_i^{t-1}$ as following.
\begin{equation*}
       Q_i^t(\alpha) {=} \left\{\!\!\!
\begin{array}{l l}
Q_i^{t-1}(\alpha), &  \text{if } \alpha_i^{t-1} {\neq} \alpha \\
Q_i^{t{-}1}(\alpha) {+} \frac{1}{N^t_i(\alpha)} \bigl(\texttt{R}_i^{t-1}(\! \alpha {\otimes} \mathrm{\delta}(\di{})  ) {-} Q_i^{t{-}1}(\alpha) \bigr), &  \text{if }  \alpha_i^{t{-}1} {=} \alpha \\
\end{array} \right. 
\end{equation*}
where $\texttt{R}_i^{t-1}$ is user $i$-th immediate objective at simulation epoch $t{-}1$, computed by \cref{eq:framework}. 
$Q_i^0(\alpha) $ is the user $i$'s initial expected reward if she/he takes action $\alpha$. 
which is initialized to $0$ as users have no prior about their behaviors on the new dynamic environment.


In our simulation, we set initial $\epsilon=0.5$ for all agents and decay a half during the MARL training processing. The detailed decay epoch is co-related to the size of possible action space $\Pi$.
Here, we define it as $\epsilon = 0.5^{ t /(3 * |\Pi |) }  $, 
where $t$ is the epoch during the reinforcement learning training processing. 

\subsection{\textbf{Discussion}}
To figure out how the platform mechanism affects users' behavior, we turn to the simulation built upon several simplified assumptions. 
One fundamental assumption is the hypothesis of rational man, where users will seek their optimal policies to maximize their objectives. 
However, in the real-world scenarios, human behaviors are also affected by psychological factors, which should also be modeled in future work.
One detailed example is that some users may feel exhausted digging out all the potential privacy choices with the provided platform mechanism.
In our simulation, we assume there remains no mental cost when a user adjusts his policy. 
However, in the reality, some users may refuse to change their policy frequently, especially in complex user interaction applications.
For such situation, a convenient user interface (UI) could be a potential solution to mitigate users' fatigues. 
\czq{
Another important factor is that users may adjust their trust level towards the platform during their exploration. One detailed example is that if the platform or even the recommender system \cite{zhang2022pipattack} is easy to be attacked or the platform will abuse their disclosed data to other applications, they may re-consider their privacy sensitivity. 
Though some works have discussed the utilization of trusted platform or the privacy-preserved recommendation model, the possible effects on user psychological factors might be tackled by a dynamic modeling on the user privacy sensitive weights, which is out of the scope of this work.
}
We simplify the influences of the psychological factors in this work and leave the exploration of psychological effects in mechanism designs and UI designs for future works. 





% \input{realdata}
In this paper, 2D and 3D CNN models were used to generate pelvic sCTs from T1-weighted MR images. Our sCT generation methods were fully automated, requiring no deformable registration or manual segmentation of bone tissues. As shown in Figure~\ref{fig3}, the 2D and 3D CNN models generated high quality sCTs. MAE curves shown in Figure~\ref{fig4} indicated that both models could precisely estimate soft-tissue HU values but had difficulty in reproducing air and high-density bone tissues. 

The MAEs within the body contour across all patients were 40.5 $\pm$ 5.4 HU and 37.6 $\pm$ 5.1 HU for the 2D and 3D models, respectively. The time required for generating a pelvic sCT using our CNN models was about 5.5 s. Our MAE results are comparable to previous studies. Kim $et \ al.$\cite{RN41} presented a voxel-based weighted summation method that produced an MAE of 74.3 $\pm$ 3.9 HU. However, manual contouring of bone tissues required for this method can be tedious and time-consuming. An MAE of 40.5 $\pm$ 8.2 HU was achieved by Dowling $et \ al.$\cite{RN11} using an average MRI-CT atlas from 38 patients. Andreasen $et \ al.$\cite{RN42} reported an MAE of 54 $\pm$ 8 HU using an atlas-based method with pattern recognition, and its prediction time was about 20.8 min. Another random forest model proposed by Andreasen $et \ al.$\cite{RN43} generated sCTs with an MAE of 58 $pm$ 9 HU. A hybrid method suggested by Siversson $et \ al.$ \cite{RN45} obtained an MAE of 36.5 $\pm$ 4.1 HU when ignoring errors introduced by gas cavities. This hybrid method was implemented in the cloud-based commercial software MriPlanner (Spectronic Medical AB, Helsingborg, Sweden), which required 50 to 80 min to generate a sCT.\cite{RN45} The patch-based 3D context-aware generative adversarial network presented by Nie $et \ al.$\cite{RN26} achieved an MAE of 39.0 $\pm$ 4.6 HU. 

Our CNN models reproduced low-density bone as shown in Figure ~\ref{fig4}. The bone-region DSCs were 0.81 $\pm$ 0.04 and 0.82 $\pm$ 0.04 from the 2D and 3D models, respectively. These results are comparable to reported DSC results of 0.79 $\pm$ 0.12\cite{RN10} and 0.91$\pm$0.03{\cite{RN11}}, where the authors compared bone contours manually drawn on the sCT and CT.

It was feasible to train the proposed 3D model with 16 image volumes from scratch. Results of the Wilcoxon signed-rank tests shown in Table~\ref{tab1} demonstrated a statistically significant improvement in overall MAE, bone DSC, and bone precision of the 3D model compared to the 2D model. However, as shown in Figure~\ref{fig4}, the 2D model seemed to perform better in estimating the high-density bone HU values. It should be noted that smaller overall MAEs do not guarantee improved sCT dose calculation and patient positioning performance. While the models performed well, we will continue to acquire more patient data to potentially improve model accuracy and further test model differences.

As this was a retrospective study, the MR image voxel sizes were not matched, resulting in different voxel intensities between images. This may have affected the sCT generation accuracy although we applied intensity normalization. A potential study could examine how voxel size variations affects sCT estimation. 

The proposed 3D model can be implemented on a 12 GB GPU to process volumetric images with dimensions of 256 $\times$ 256 $\times$ 30. More GPU memory would be required to process higher resolution 3D images. Considering the limited access to multi-GPU systems, a 3D architecture with fewer convolutional layers could be considered to deal with higher resolutions. However, the performance could be affected by the reduced parameters and smaller receptive fields of the less complex model. Another approach would be to extract 30-slice sub-volumes from CT and MR images for training the 3D model. The sCT could then be generated by averaging 30-slice sCT sub-volumes produced by the model. 

A number of techniques could be investigated for improving model performance.  Nie $et \ al.$\cite{RN26} showed that introducing an additional adversarial discriminator improved overall sCT quality. The same approach could be adapted in our proposed 2D and 3D CNN models.  Non-rigid deformation\cite{RN44} could also be applied to both CT and MR images in the process of the on-the-fly data augmentation to produce more training pairs. Multiple MR images acquired with different sequences could be fed into models to provide more information for distinguishing different tissues. Multi-GPU systems with more memory would enable the exploration of larger batch sizes for training CNN models, which could reduce variances in gradient estimation and accelerate the training. 



%%%%%%%%%%%%%%%%%%%%%%%%%%%%%%%%%%%%%
\bibliographystyle{unsrtnat}
\bibliography{ABESS}
%%%%%%%%%%%%%%%%%%%%%%%%%%%%%%%%%%%%%

\end{document}



