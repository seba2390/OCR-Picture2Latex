\section{INTRODUCTION}

In the field of Information Retrieval, keyword search and replace is a standard problem. Often we want to either find specific keywords in text, or replace keywords with standardized names.

\noindent For example:

\begin{enumerate}
 \item Keyword Search: Say we have a resume \textit{(D)} of a software engineer, and we have a list of 20K programing skills \textit{corpus = \{java, python, javascript, machine learning, ...\}}. We want to find which of the 20K terms are present in the resume (\textit{corpus $\cap$ D}). 
 
 \item Keyword Replace: Another common use case is when we have a corpus of synonyms (different spellings meaning the same thing) like \textit{corpus = \{ javascript: [‘javascript’, ‘javascripting’, ‘java script’], ...\}} and we want to replace the synonyms with standardized names in the candidate resume.
  
\end{enumerate}

To solve such problems, Regular-Expressions (Regex) are most commonly used. While Regex works well when the number of terms are in 100s, it is, however, considerably slow for 1000s of terms. When the no. of terms are in 10s of thousands and no. of documents are in millions, the run-time will reach a few days. As shown in figure \ref{fig:search}, time taken by Regex to find 15K keywords in 1 document of 10K terms is almost 0.165 seconds. Whereas, for FlashText it is 0.002 seconds. Thus FlashText is 82x faster than Regex for 15K terms.\hfill\break

As the number of terms increase, time taken by Regex grows almost linearly. Whereas time taken by FlashText is almost a constant. In this paper we will discuss the Regex based approach for both keyword search and replace and compare it with FlashText. We will also go through the detailed FlashText algorithm and how it works, and share the code that we used to benchmark FlashText with Regex (which was used to generate the figures in this paper).\hfill\break

\subsection{Regex for keyword search}

Regex as a tool is very versatile and useful for pattern matching. We can search for patterns like '\textbackslash d\{4\}' (which will match any 4 digit number), or keywords like '2017' in a text document. Sample python code (Code \ref{code:ser}) to search '2017' or any 4 digit number in a given string.
\newline
\lstinputlisting[language=Python, caption=Sample python code to search 2017 or any 4 digit number in a given string using Regex., label=code:ser]{codes/regex_search.py}

\begin{figure}[H]
    \centering
    \includegraphics[width=0.6\textwidth]{figures/search.png}
    \caption{Comparing Time taken (y-axis) by Regex and FlashText to find number of terms (x-axis).}
    \label{fig:search}
\end{figure}

Here '\textbackslash b' is used to denote word-boundary, and is used so that 23114 won’t return 2311 as a match. Word-boundary in Regex ('\textbackslash b') matches special characters like \textit{‘space’, ‘period’, ‘new line’, etc.. \{‘ ‘, ‘.’, ‘\textbackslash n’\}}.

\subsection{Regex for keyword replacement}

We can also use Regex tool to replace the matched term with a standardised term. Sample python code (Code \ref{code:rep}) to replace java script with javascript.

\lstinputlisting[language=Python, caption=Sample python code to replace java script with javascript using Regex., label=code:rep]{codes/regex_replace.py}

\begin{figure}[H]
    \centering
    \includegraphics[width=0.6\textwidth]{figures/replace.png}
    \caption{Comparing time taken (y-axis) by Regex and FlashText to replace number of terms (x-axis).}
    \label{fig:replace}
\end{figure}