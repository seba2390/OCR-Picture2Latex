\pagebreak
\documentclass[main.tex]{subfile}
\renewcommand{\thesection}{\Alph{section}} 
\renewcommand{\theequation}{S\arabic{equation}}
\renewcommand{\thefigure}{S\arabic{figure}}
\begin{document}

\textbf{\LARGE 
\begin{center}
Supplementary Information
\end{center}}

\noindent\textbf{\Large Quantum emitters coupled to circular nanoantennas for high brightness quantum light sources}

Hamza A. Abudayyeh $^1$ and Ronen Rapaport $^{1,2}$

$^1$ Racah Institute for Physics, and
$^2$ the Applied Physics Department, The Hebrew University of Jerusalem, Jerusalem 9190401, Israel
\nopagebreak
\\
\\
\\

\setcounter{section}{0}
\setcounter{figure}{0}
\setcounter{equation}{0}


  
\section{Modes of metal dielectric air (MDA) waveguide}

The system considered here is shown in figure \ref{MDAfig}. 
This structure is a special form of the three layer waveguide structure with the substrate as a perfect electric conductor and the cladding is air. 
The solutions of the wave equations that we are interesting in are those that result in a propagating waveguide mode (i.e. in the x direction). 
It is also assumed that the waveguide is homogeneous in the y-direction. 
Therefore the spatial dependence of the electromagnetic fields will have the following form:

\begin{equation}
\label{eq:VecE}
\begin{aligned}
\vec{E}(x,y,z) &=\vec{e}(z) \exp(i\beta x) 
\\
\vec{H}(x,y,z) &=\vec{h}(z) \exp(i\beta x) 
\end{aligned}
\end{equation}
Substituting equations\ref{eq:VecE} into Maxwell's equations results in two sets of solutions for the transverse electric (TE) and transverse magnetic (TM) polarizations.

\subsection*{TE polarization}
For this polarization the nonzero field components are $e_y$, $h_x$ and $h_z$. 
Since we are looking for propagating modes in the waveguide we assume that the fields are evanescent in the air layer. 
Furthermore the parallel electric field must vanish at the PEC surface. 
Thus it is clear that the solutions must take the form of:
\begin{equation}
e_y(z)= 
\begin{cases}
A\sin(\kappa z),  &  0\leq z \leq h \\ 
B\exp\left(-\gamma (z-h)\right),  &  z \geq h
\end{cases}
\end{equation}
Assuming harmonic time dependence, $\kappa $ and $\gamma$ must satisfy :
\begin{equation}
\label{eq:kconservation}
\begin{aligned}
\beta^2  +\kappa^2  &=\frac{n^2\omega^2}{c^2} \\
\beta^2  -\gamma^2 &= \frac{\omega^2}{c^2}
\end{aligned}
\end{equation}
The parallel electric and magnetic fields need to be continuous at the dielectric air boundary which results in the following equations:
\begin{equation}
\label{eq:BCTE}
\begin{aligned}
A\sin(\kappa h)&=B
\\ A\kappa \cos(\kappa h)& = -\gamma B
\end{aligned}
\end{equation}
By dividing these equations we reach to the following transcendental equation:
\begin{equation}
\label{eq:modeTE}
\tan(\kappa h)=-\frac{\kappa}{\gamma}
\end{equation}
By solving equations  \ref{eq:kconservation} and \ref{eq:modeTE} simultaneously one obtains the dispersion relation ($\omega (\beta)$). Clearly this would lead to an infinite number of modes each having a cutoff that occurs when the field is no longer confined to the waveguide layer (i.e. $\gamma \rightarrow 0 $). Therefore the cutoff condition occurs when $\tan(\kappa h)= -\infty$. This can be written in terms of the free-space wavelength $\lambda_0= 2\pi c/\omega$  and the waveguide thickness as: 
\begin{equation}
\label{eq:cutTE}
\frac{h_{cutoff}^{(\nu)}}{\lambda_0}= \frac{2\nu+1}{4\sqrt{n^2 -1}} 
\end{equation}
where $\nu = 0,1,2,... $ is the mode number and $h_{cutoff}^{(\nu)}$ is the minimum waveguide thickness that would support the TE$_{\nu}$ mode.
The electric field of the TE$_\nu$ mode can therefore  be given as:
\begin{equation}
\label{eq:ETe}
\resizebox{0.4\textwidth}{!}{$
E_y^{(\nu)}(x,z)= 
\begin{cases}
E_0\sin(\kappa_\nu z)\exp(i \beta_\nu x),  &  0\leq z \leq h \\ 
E_0 \sin(\kappa_\nu d)\exp\left(-\gamma_\nu (z-h)\right)\exp(i \beta_\nu x),  &  z \geq h
\end{cases}$}
\end{equation}
, and the magnetic field may be found by applying Maxwell's equations.
\begin{figure}[t!]
\centering{\includegraphics[width=0.5\textwidth]{MDA.pdf}}
\caption{Diagram of metal dielectric air waveguide.}
\label{MDAfig}
\end{figure}
\subsection*{TM polarization}
For this polarization the non-vanishing fields are $h_y$, $e_x$, and $e_z$ . Therefore it is easier to solve the equations in terms of the magnetic field $h_y$ and then apply Maxwell's curl equations to find the electric field components.  As before we assume a confined propagating field which leads to the ansatz: 
\begin{equation}
h_y(z)= 
\begin{cases}
A\cos(\kappa z),  &  0\leq z \leq h \\ 
B\exp\left(-\gamma (z-h)\right),  &  z \geq h
\end{cases}
\end{equation}
where the cosine was chosen to force the tangential electric field ($e_x$) to vanish at the conductor surface. The parameters $\kappa$ and $\gamma$ are still related to the propagation constant $\beta$ by equation \ref{eq:kconservation}. By applying the continuity of the tangential electric and magnetic fields at the dielectric air interface we reach to the following conditions: 
\begin{equation}
\label{eq:BCTM}
\begin{aligned}
A\cos(\kappa h)&=B
\\ \frac{A}{n^2}\kappa \sin(\kappa h)& = \gamma B
\end{aligned}
\end{equation}
which yields the transcendental equation: 
\begin{equation}
\label{eq:modeTM}
\tan(\kappa h)=\frac{n^2\gamma}{\kappa}
\end{equation}
Again a mode is cutoff when $\gamma \rightarrow 0 $, or $\tan (\kappa h)=0$. This yields the following condition:
\begin{equation}
\label{eq:cutTM}
\frac{h_{cutoff}^{(\nu)}}{\lambda_0}= \frac{\nu}{2\sqrt{n^2 -1}} 
\end{equation}
with $\nu = 0,1,2,...$ corresponds to the TM$_\nu$ mode. One important difference is that the TM$_0$ mode has no cutoff and therefore is always present regardless of the thickness or wavelength considered.  

We can get the electric field components by applying the curl equations to obtain: 
\begin{equation}
\label{eq:ETM}
\resizebox{0.4\textwidth}{!}{$
\begin{aligned}
E^{(\nu)}_x(x,z)=
\begin{cases}
E_0 \sin(\kappa_\nu z)\exp(i \beta_\nu x),  &  0\leq z \leq h \\ 
E_0 \sin(\kappa_\nu h)\exp(-\gamma_\nu (z-h))\exp(i \beta_\nu x),  &  z \geq h
\end{cases}
\\
E^{(\nu)}_z(x,z)=
\begin{cases}
\frac{-i E_0 \beta_\nu}{\kappa_\nu} \cos(\kappa_\nu z)\exp(i \beta_\nu x),  &  0\leq z < h \\ 
\frac{i E_0 \beta_\nu}{n^2} \sin(\kappa_\nu h)\exp(-\gamma_\nu (z-h))\exp(i \beta_\nu x),  &  z > h
\end{cases}
\end{aligned}
$}
\end{equation}
\section{Incoherent unpolarized dipole sources}
Here we will discuss how the fields and flux of an unpolarized dipole source may be calculated. 
The field by an unpolarized source may be calculated by an incoherent superposition of the fields of randomly oriented oscillating dipoles, i.e.:
\begin{equation}
\label{eq: normincoherent}
\left<|\Psi|^2\right> = \frac{1}{4\pi} \int\int |\Psi(\theta',\phi')|^2 \sin(\theta')d\theta' d\phi' 
\end{equation}
where $\Psi$ is a general vector field that can represent the electric or magnetic field and $<\cdots >$ signifies an average over all dipole oreientations . $\theta'$ and $\phi'$ are the polar and azimuthal angles of the dipole and are not to be confused with the polar and azimuthal angles of the position vector $\vec{r}$ ($\theta$,$\phi$). It is worth noting that all the  fields in this section have implicit dependence on $\vec{r}$ which we dropped from the notation.  
\begin{figure}[t!]
\centering{\includegraphics[width=0.5 \textwidth]{Supp_fig2.pdf}}
\caption{(a) Collection efficiency for the structure scaled to different central wavelengths. Collection efficiency ($\eta$) (b) and Brightness enhancement ($\xi$)  (c) for the optimized structure for dipoles oriented along the x and z axes (solid curves) . The dashed lines represent the curves for dipoles in free space.} 
\label{fig:Supp2}
\end{figure}
Such a randomly oriented dipole can in general be decomposed into three orthogonal dipoles $\vec{P}_x$, $\vec{P}_y$ , and $\vec{P}_z$ oriented along the axes of some coordinate system as follows: 
\begin{equation}
\vec{P}= \vec{P}_x \sin(\theta')\cos(\phi') + \vec{P}_y \sin(\theta')\sin(\phi')+ \vec{P}_z \cos(\theta') 
\end{equation}
We can therefore write the  field of an unpolarized source as:
\begin{equation}
\label{eq: randomdipole}
\vec{\Psi}(\theta',\phi')= \vec{\Psi}_x\sin(\theta')\cos(\phi')+ \vec{\Psi}_y\sin(\theta')\sin(\phi')+ \vec{\Psi}_z\cos(\theta')
\end{equation}
where $\vec{\Psi}_i$ represents the  field due to the dipole $\vec{P_i}$ .  Substituting equation \ref{eq: randomdipole} into equation \ref{eq: normincoherent} and using some integral identities gives the following relation: 
\begin{equation}
\label{eq: averagedfield}
\left<|\Psi|^2\right> = \frac{1}{3}\left( |\Psi_x|^2+|\Psi_y|^2+|\Psi_z|^2 \right)
\end{equation}
This derivation assumes that the orientation of the dipole is completely random i.e. $P_i$ and thus $\Psi_i$ have no dependence on $\theta'$ or $\phi'$ .  Equation \ref{eq: averagedfield} is equally valid for the electric and magnetic field and is therefore valid for the flux: 
\begin{equation}
\label{eq: averagedflux}
\left<F\right> = \frac{1}{3}\left( F_x+F_y+F_z \right)
\end{equation}
For a  system with azimuthal symmetry this reduces to:
\begin{equation}
\label{eq: averagedfluxazimuthal}
\left<F\right> = \frac{2}{3} F_x+\frac{1}{3}F_z 
\end{equation}\section{Circular periodic ('Bullseye') nanoantenna with a central cavity}
In this section we show supplementary results for section 3 in the main text. 

\begin{figure}[t!]
\centering{\includegraphics[width=0.5 \textwidth]{Supp_fig3.pdf}}
\caption{Intensity angular distribution in the yz-plane (a and c) and the collection efficiency (b and d) for various values of the period (a and b) and the dipole height (c and d).  }
\label{fig:Supp3}
\end{figure}

\subsection*{Device geometrical parameters scalability with emitter wavelength:}
One important feature that the design of the nanoantenna possesses is scalability of its geometrical parameters with the emitter's wavelength. 
For this reason we chose in the main text to normalize lengths to the emitter's central wavelength to make the reported parameters general for all relevant emission wavelengths.
In order to display this we plot in figure \ref{fig:Supp2}a the collection efficiency for various central wavelengths between $600$ and $800$ nm where we scale the nanoantenna dimensions to the central wavelength. 
The figure clearly shows that the differences between the various central wavelengths is small and that the overall trend is the same for all wavelengths. 
The differences can be mainly attributed to the wavelength dispersion of the metal optical properties. 
The dispersion of the dielectric in the wavelength range considered can be neglected. 
\subsection*{Device emission dependence on dipole orientation}
We complement the discussion about the dipole orientation in the main text here by displaying the collection efficiency and brightness enhancement (defined in equations 6 and 7 in the main text) for the $P_z$ dipole in figure \ref{fig:Supp2}b and c respectively. 
As compared to the $P_x$ orientation the $P_z$ dipole is both suppressed ($F_z^r \approx 0.45 $ ) and the collection efficiency is low for low numerical apertures. 
This emphasizes the point made in the main text that an unpolarized emitter will emit preferentially with a dipole oriented along the x-axis especially at low NA. 
\subsection*{Device emission dependence on geometrical parameters}
To continue the parameter dependence analysis in the main text, here we discuss the effect of changing the periodicity of the circular grating and the location of the dipole emitter in the polymer layer.
Figure \ref{fig:Supp3} represent the intensity angular distributions and collection efficiency for different values of these parameters. 
The parameters are detuned above an below the optimized parameters ($\Lambda/\lambda_0 =0.86$ and $d/\lambda_0 = 0.29$ ). 
The effect of detuning the periodicity leads to a mismatch between the waveguide propagation constant and the grating Bragg wavevector causing  the formation of sidebands. 
On the other hand changing the dipole location will lead to lower coupling between the dipole emission and the waveguide mode. 
This will have the primary effect of lowering the radiative enhancement as can be seen from figure \ref{fig:Supp3}c.
It will slightly change the collection efficiency of the antenna (see figure \ref{fig:Supp3}d). 
  

\end{document}

