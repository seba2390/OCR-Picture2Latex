\section{Introduction}
Let $k$ be a field. Let $\overline k$ be an algebraic closure of $k$. Let $G$ be a connected affine algebraic $k$-group: we regard $G$ as a $\overline k$-defined algebraic group together with a choice of $k$-structure in the sense of Borel~\cite[AG.~11]{Borel-AG-book}. We say that $G$ is \emph{reductive} if the unipotent radical $R_u(G)$ of $G$ is trivial. Throughout, $G$ is always a connected reductive $k$-group. In this paper, we continue our study of rationality problems for complete reducibility of subgroups of $G$~\cite{Uchiyama-Nonperfect-pre},~\cite{Uchiyama-Nonperfectopenproblem-pre}. By a subgroup of $G$ we mean a (possibly non-$k$-defined) closed subgroup of $G$. Following Serre~\cite[Sec.~3]{Serre-building}:
\begin{defn}
A subgroup $H$ of $G$ is called \emph{$G$-completely reducible over $k$} (\emph{$G$-cr over $k$} for short) if whenever $H$ is contained in a $k$-defined parabolic subgroup $P$ of $G$, then $H$ is contained in a $k$-defined Levi subgroup of $P$. In particular if $H$ is not contained in any proper $k$-defined parabolic subgroup of $G$, $H$ is called \emph{$G$-irreducible over $k$} (\emph{$G$-ir over $k$} for short). 
\end{defn}

So far, most studies on complete reducibility is for complete reducibility over $\overline k$ only; see~\cite{Liebeck-Seitz-memoir},~\cite{Stewart-nonGcr},~\cite{Thomas-irreducible-JOA} for example. We say that a subgroup $H$ of $G$ is $G$-cr if it is $G$-cr over $\overline k$. Not much is known on complete reducibility over $k$ (especially for nonperfect $k$) except a few theoretical results and important examples; see~\cite[Sec.~5]{Bate-geometric-Inventione},~\cite{Bate-cocharacter-Arx},~\cite{Uchiyama-Nonperfect-pre},~\cite{Uchiyama-Nonperfectopenproblem-pre}. In~\cite[Thm.~1.10]{Uchiyama-Separability-JAlgebra},~\cite[Thm.~1.8]{Uchiyama-Classification-pre},~\cite[Thm.~1.2]{Uchiyama-Nonperfect-pre},~\cite[Sec.~6]{Bate-separability-TransAMS}, Bate et al.~and we found several examples of $k$-subgroups of $G$ that are $G$-cr over $k$ but not $G$-cr (or vice versa). All these examples are for $G$ of exceptional type ($E_6$, $E_7$, $E_8$, $G_2$) in $p=2$ and constructions are very intricate. The first main result in this paper is the following:  

\begin{thm}\label{D4example}
Let $k$ be a nonperfect separably closed field of characteristic $2$. Let $G$ be a simple $k$-group of type $D_4$. Then there exists a $k$-subgroup $H$ of $G$ that is $G$-cr over $k$ but not $G$-cr (or vice versa).
\end{thm}

A few comments are in order. First, one can embed $D_4$ inside $E_6$, $E_7$ or $E_8$ as a Levi subgroup. Since a subgroup contained in a $k$-Levi subgroup $L$ of $G$ is $G$-cr over $k$ if and only if it is $L$-cr over $k$ (Proposition~\ref{G-cr-L-cr}), one might argue that our ``new example'' is not really new. However we have checked that our example is different from any example in~\cite[Thm.~1.10]{Uchiyama-Separability-JAlgebra},~\cite[Thm.~1.8]{Uchiyama-Classification-pre},~\cite[Thm.~1.2]{Uchiyama-Nonperfect-pre}. So this is the first such example for classical $G$. Second, the non-perfectness of $k$ is essential in Theorem~\ref{D4example} in view of the following~\cite[Thm.~1.1]{Bate-separable-Paris}:
\begin{prop}\label{paris}
Let $H$ be a subgroup of $G$. Then $H$ is $G$-cr over $k$ if and only if $H$ is $G$-cr over $k_s$ (where $k_s$ is a separable closure of $k$). 
\end{prop}
So in particular if $k$ is perfect, a subgroup of $G$ is $G$-cr over $k$ if and only if it is $G$-cr. Proposition~\ref{paris} is deep: it depends on the recently proved $50$-years-old \emph{center conjecture of Tits} (see Conjecture~\ref{centerconjecture}) in spherical buildings~\cite{Serre-building},~\cite{Tits-colloq},~\cite{Weiss-center-Fourier}. Third, the $k$-definedness of $H$ in Theorem~\ref{D4example} is important. 
Actually it is not difficult to find a $\overline k$-subgroup with the desired property. For our construction of a $k$-defined subgroup $H$, it is essential for $H$ to be $\emph{nonseparable}$ in $G$. We write $\textup{Lie}(G)$ or $\mathfrak g$ for the Lie algebra of $G$. Recall~\cite[Def.~1.1]{Bate-separability-TransAMS}:
\begin{defn}
A subgroup $H$ of $G$ is \emph{nonseparable} if the dimension of $\textup{Lie}(C_G(H))$ is strictly smaller than the dimension of $\mathfrak{c}_{\mathfrak{g}}(H)$ (where $H$ acts on $\mathfrak g$ via the adjoint action). In other words, the scheme-theoretic centralizer of $H$ in $G$ (in the sense of~\cite[Def.~A.1.9]{Conrad-pred-book}) is not smooth. 
\end{defn} 
We exhibit the importance of nonseparability of $H$ in the proof of Theorem~\ref{D4example}. Proper nonseparable $k$-subgroups of $G$ are hard to find, and only handful examples are known~\cite[Sec.~7]{Bate-separability-TransAMS},~\cite[Thm.~1.10]{Uchiyama-Separability-JAlgebra}~\cite[Thm.~1.8]{Uchiyama-Classification-pre},~\cite[Thm.~1.2]{Uchiyama-Nonperfect-pre}. It is known that if $p$ is very good for $G$, every subgroup of $G$ is separable~\cite[Thm.~1.2]{Bate-separability-TransAMS}. Thus, to find a nonseparable subgroup we are forced to work in small $p$. See~\cite{Bate-separability-TransAMS},~\cite{Herpel-smoothcentralizerl-trans} for more on separability. 

In the rest of this section we assume $k$ is algebraically closed. In~\cite{Uchiyama-Separability-JAlgebra}, we asked:
\begin{question}\label{G-cr-M-cr-Q}
Let $H\leq M\leq G$ be a triple of reductive groups with $G$ and $M$ connected. If $H$ is $G$-cr then it is $M$-cr (and vice versa)?
\end{question} 
In general, the answer is no in either direction. It is easy to find a counterexample for the reverse direction: take $H=M=PGL_2$ and $G=SL_3$ in $p=2$ and $H$ sits inside $G$ via the adjoint representation. For more counterexamples, see~\cite{Liebeck-Seitz-memoir},~\cite{Stewart-nonGcr}. A counterexample for the forward direction is hard to find and only a handful such examples are known~\cite[Thm.~1.1]{Uchiyama-Separability-JAlgebra},~\cite[Thm.~1.2]{Uchiyama-Classification-pre},~\cite[Sec.~6]{Bate-separability-TransAMS}. All these examples are for $G$ of exceptional type ($E_6$, $E_7$, $E_8$, $G_2$) in $p=2$. Here is our second main result:
\begin{thm}\label{G-cr-M-cr}
Let $k$ be of characteristic $2$. Let $G$ be simple and of type $D_4$. Then there exists a pair of reductive subgroups $H<M$ of $G$ such that $(G,M)$ is a reductive pair and $H$ is $G$-cr but not $M$-cr.
\end{thm}
Recall that a pair of reductive groups $G$ and $M$ is called a \emph{reductive pair} if $\textup{Lie} M$ is an $M$-module direct summand of $\mathfrak{g}$. See~\cite{Goodbourn-reductivepairs} for more on reductive pairs. For our construction, nonseparablity of $H$ is essential~\cite[Thm.~1.4]{Bate-separability-TransAMS}: 
\begin{prop}
Suppose that $(G,M)$ is a reductive pair. Let $H$ be a subgroup of $M$ such that $H$ is separable in $G$. If $H$ is $G$-cr then $H$ is $M$-cr. 
\end{prop}

Now we move on to a problem with a slightly different flavor. Let $\Gamma$ be a finite group. By a representation of $\Gamma$ in a reductive group $G$, we mean a homomorphism from $\Gamma$ to $G$. We write $\textup{Hom}(\Gamma, G)$ for the set of representations $\rho$ of $\Gamma$ in $G$. The group $G$ acts on $\textup{Hom}(\Gamma, G)$ by conjugation. Let $\Gamma_p$ be a Sylow $p$-subgroup of $G$. In \cite[Sec.~2]{Kulshammer-Donovan-Israel}, K\"ulshammer asked: 
\begin{question}\label{KulshammerQ}
Let $G$ be a reductive algebraic group defined over an algebraically closed field of characteristic $p$. Let $\rho_p \in \textup{Hom}(\Gamma_p, G)$. Then are there only finitely many representations $\rho \in \textup{Hom}(\Gamma, G)$ such that $\rho\mid_{\Gamma_p}$ is $G$-conjugate to $\rho_p$? 
\end{question}
It is known that in general the answer is no. Two counterexamples are known: one in $G$ of type $G_2$~\cite{Bate-QuestionOfKulshammer} and the other in $G$ of type $E_6$~\cite[Thm.~1.14]{Uchiyama-Classification-pre} (both in $p=2$). The third main result in this paper is 
\begin{thm}\label{thmKul}
Let $k$ be of characteristic $2$. Let $G$ be simple of type $D_4$. Then there exists a finite group $\Gamma$ with a Sylow $2$-subgroup $\Gamma_2$ and representations $\rho_a\in \textup{Hom}(\Gamma,G)$ for $a\in k$ such that $\rho_a$ is not conjugate to $\rho_b$ for $a\neq b$ but the restrictions $\rho_a\mid_{\Gamma_2}$ are not pairwise conjugate for all $a\in k$. 
\end{thm} 
We note that nonseparability plays a crucial role in the proof of Theorem~\ref{thmKul}. 
In this paper, the reader will see that seemingly unrelated Questions~\ref{G-cr-M-cr-Q} and~\ref{KulshammerQ} (and the rationality problems for $G$-complete reducibility above and the problem on conjugacy classes below) are related: all our main results concerning these problems (Theorems~\ref{D4example},~\ref{G-cr-M-cr},~\ref{thmKul},~\ref{conjugacy-counterexample}) are based on the same mechanism (nonseparability plus some modifications). However, it is not completely clear yet (at least to the author) how exactly these problems are related. The main purpose of this paper is to give a chance for the reader to look at these problems all in once with a relatively easy example in $G$ of type $D_4$ to stimulate further research on relations between these problems.  

Finally we consider a problem on the number of conjugacy classes. Given $n\in {\mathbb N}$, we let $G$ act on $G^n$ by simultaneous conjugation:
$
g\cdot(g_1, g_2, \ldots, g_n) = (g g_1 g^{-1}, g g_2 g^{-1}, \ldots, g g_n g^{-1}). 
$
In \cite{Slodowy-book}, Slodowy proved the following result, applying Richardson's beautiful tangent space argument~\cite[Sec.~3]{Richardson-Conjugacy-Ann},~\cite[Lem.~3.1]{Richardson-orbits-BullAustralian}. 
\begin{prop}\label{conjugacy}
Let $M$ be a reductive subgroup of a reductive algebraic group $G$ defined over an algebraically closed field $k$. Let $n\in {\mathbb N}$, let $(m_1, \ldots, m_n)\in M^n$ and let $H$ be the subgroup of $M$ generated by $m_1, \ldots, m_n$. Suppose that $(G, M)$ is a reductive pair and that $H$ is separable in $G$. Then the intersection $G\cdot (m_1, \ldots, m_n)\cap M^n$ is a finite union of $M$-conjugacy classes. 
\end{prop}

Proposition~\ref{conjugacy} has many consequences; see~\cite{Bate-geometric-Inventione}, \cite{Slodowy-book}, and \cite[Sec.~3]{Vinberg-invariants-JLT} for example. Here is our main result on conjugacy classes:

\begin{thm}\label{conjugacy-counterexample}
Let $k$ be of characteristic $2$. Let $G$ be simple of type $D_4$. Let $M$ be the subsystem subgroup of type $A_1 A_1 A_1 A_1$. Then there exists $N\in \mathbb{N}$ and a tuple $\mathbf{m}\in M^N$ such that $G\cdot \bold{m} \cap M^N$ is an infinite union of $M$-conjugacy classes. 
\end{thm} 

Here is the structure of the paper. In Section 2, we set out the notation and show some preliminary results. Then in Section 3, we prove our first main result (Theorem~\ref{D4example}) concerning a rationality problem for complete reducibility. In Section 4, we prove some rationality result (Theorem~\ref{centralizerA}) related to the center conjecture. In Section 5, we give a short proof for our second main result on complete reducibility (Theorem~\ref{G-cr-M-cr}) using a recent result from Geometric Invariant Theory (Proposition~\ref{unipotentconjugate}). Then in Section 6, we prove Theorem~\ref{thmKul} giving a new counterexample to the question of K\"ulshammer. Finally in Section 7 we consider a problem on conjugacy classes and prove Theorem~\ref{conjugacy-counterexample}.

