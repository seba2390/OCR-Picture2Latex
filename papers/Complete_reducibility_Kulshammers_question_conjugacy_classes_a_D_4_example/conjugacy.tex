\section{Conjugacy classes (Proof of Theorem~\ref{conjugacy-counterexample})}

\begin{proof}
Let $G$ be as in the hypothesis. Let $\lambda:=(\alpha+2\beta+\gamma+\delta)^{\vee}$. Then $\Psi(R_u(P_\lambda))=\{4,\cdots, 12\}$. Using the commutation relations we have $Z(R_u(P_\lambda))=U_{12}$. Let $n:=n_\alpha n_\gamma n_\delta$. Pick $b\in k$ with $b^3=1$ and $b\neq 1$. Let $t:=(\alpha+\gamma+\delta)^{\vee}(b)$. Define $K:=\langle n, t, U_{12} \rangle$. 
By the same argument as that in the proof of~\cite[Lem.~5.1]{Uchiyama-Separability-JAlgebra} we obtain $C_{P_\lambda}(K)=C_{R_u(P_\lambda)}(K)$ (since $\langle 12, \lambda\rangle=2$). By a standard result there exists $n\in \mathbb{N}$ such that $Z=\langle z_1,\cdots, z_n \rangle$. Now let $M:=\langle L_\lambda, G_{12} \rangle$. Let ${\bf m}:=(n, t, z_1,\cdots, z_n)$ and set $N:=n+2$. Then by the similar argument to that in the proof of~\cite[Lem.~5.1]{Uchiyama-Separability-JAlgebra} yields that $G\cdot {\bf m}\cap P_\lambda(M)^N$ is an infinite union of $P_\lambda(M)$-conjugacy classes. (The crucial thing here is the existence of a curve that is tangent to $\mathfrak{c}_{\mathfrak{g}}(K)$ but not tangent to $C_G(K)$, in other words $K$ is nonseparable in $G$.) Now let $c_\lambda:P_\lambda\rightarrow L_\lambda$ be the canonical projection. Then $c_\lambda(n, t, z_1, \cdots, z_n)=(n,t)$. Since $K_0:=\langle n, t \rangle$ is $L$-ir as shown in the previous section, by~\cite[Prop.~3.5.2]{Stewart-thesis} we are done. 
\end{proof}

