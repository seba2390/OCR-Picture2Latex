\section{Tits' center conjecture}
In~\cite{Tits-colloq}, Tits conjectured the following:
\begin{con}\label{centerconjecture}
Let $X$ be a spherical building. Let $Y$ be a convex contractible simplicial subcomplex of $X$. If $H$ is an automorphism group of $X$ stabilizing $Y$, then there exists a simplex of $Y$ fixed by $H$.   
\end{con}
This so-called center conjecture of Tits was proved by case-by-case analyses by Tits, M\"{u}hlherr, Leeb, and Ramos-Cuevas~\cite{Leeb-Ramos-TCC-GFA},~\cite{Muhlherr-Tits-TCC-JAlgebra},~\cite{Ramos-centerconj-Geo}. Recently uniform proof was given in~\cite{Weiss-center-Fourier}. In relation to the theory of complete reducibility, Serre showed~\cite{Serre-building}:
\begin{prop}\label{SerreContractible}
Let $G$ be a reductive $k$-group. Let $\Delta(G)$ be the building of $G$. If $H$ is not $G$-cr, then the fixed point subcomplex $\Delta(G)^H$ is  convex and contractible. 
\end{prop}
We identify the set of proper $k$-parabolic subgroups of $G$ with $\Delta(G)$ in the usual sense of Tits~\cite{Tits-book}. Note that for a subgroup $H$ of $G$, $N_G(H)(k)$ induces an automorphism group of $\Delta(G)$ stabilizing $\Delta(G)^H$. Thus, combining the center conjecture with Proposition~\ref{SerreContractible} we obtain
\begin{prop}\label{normalizer}
If a subgroup $H$ of $G$ is not $G$-cr over $k$, then there exists a proper $k$-parabolic subgroup of $G$ containing $H$ and $N_G(H)(k)$. 
\end{prop}
Proposition~\ref{normalizer} was an essential tool to prove various theoretical results on complete reducibility over nonperfect $k$ in~\cite{Uchiyama-Nonperfectopenproblem-pre} and~\cite{Uchiyama-Nonperfect-pre}. We have asked the following in~\cite[Rem.~6.5]{Uchiyama-Nonperfectopenproblem-pre}:
\begin{question}\label{centralizerQ}
If $H<G$ is not $G$-cr over $k$, then does there exist a proper $k$-parabolic subgroup of $G$ containing $HC_G(H)$?
\end{question}
The answer is yes if $C_G(H)$ is $k$-defined (or $k$ is perfect). Since in that case the set of $k$ points are dense in $C_G(H)$ (since we assume $k=k_s$) and the result follows from Proposition~\ref{normalizer}. The main result in this section is to present a counterexample to Question~\ref{centralizerQ} when $k$ is nonperfect. 
\begin{thm}\label{centralizerA}
Let $k$ be nonperfect of characteristic $2$. Let $G$ be simple of type $D_4$. Then there exists a non-abelian $k$-subgroup $H$ of $G$ such that $H$ is not $G$-cr over $k$ but $C_G(H)$ is not contained in any proper $k$-parabolic subgroup of $G$. 
\end{thm}
\begin{rem}
Borel-Tits~\cite[Rem.~2.8]{Borel-Tits-unipotent-invent} mentioned that if $k$ is nonperfect of characteristic $2$ and $[k:k^2]>2$, there exists a $k$-plongeable unipotent element $u$ in $G$ of type $D_4$ such that $C_G(u)$ is not contained in any proper $k$-parabolic subgroup of $G$ (with no proof). Note that such $u$ generates a (cyclic) subgroup of $G$ that is not $G$-cr over $k$. (Recall that a unipotent element is called $k$-plongeable if it can be embedded in the unipotent radical of a proper $k$-parabolic subgroup of $G$~\cite{Borel-Tits-unipotent-invent}.) Theorem~\ref{centralizerA} is a nonabelian version of Borel-Tits' result. Also the assumption $[k:k^2]>2$ is not necessary here. 
\end{rem}
\begin{proof}
We keep the same notation from the previous section. Set $n:=n_\alpha n_\gamma n_\delta$,  $t:=(\alpha+\gamma+\delta)^{\vee}(b)$, and $v(\sqrt a):=\epsilon_4(\sqrt a)\epsilon_{11}(\sqrt a)$. Let $H:=\langle n_\alpha n_\gamma n_\delta \epsilon_{12}(a), (\alpha+\gamma+\delta)^{\vee}(b) \rangle$.  Then $H$ is not $G$-cr over $k$. 
We have $H':=v(\sqrt a)^{-1}\cdot H = \langle n, t \rangle$. It is clear that $C_G(H')>G_{12}$. Thus $\langle n, t, G_{12} \rangle < H'C_G(H')$. By running a similar argument as in the proof of Lemma~\ref{uniquepara} in the previous section, we find that the only proper parabolic subgroup of $G$ containing $\langle n, t, U_{12} \rangle$ is $P_{(\alpha+2\beta+\gamma+\delta)^{\vee}}$ (since $n_{12}\cdot 12 = -12$). Clearly $P_{(\alpha+2\beta+\gamma+\delta)^{\vee}}$ does not contain $G_{12}$. Therefore there is no proper parabolic subgroup of $G$ containing $H'C_G(H')$. Thus there is no proper parabolic subgroup of $G$ containing $HC_G(H)$. 
\end{proof}


\section{G-cr vs $M$-cr (Proof of Theorem~\ref{G-cr-M-cr})}
From this section we assume $k$ is algebraically closed. Let $G$ be as in the hypothesis. Let $a, b\in k^{*}$ with $b^3=1$ and $b\neq 1$. Let $H':=\langle n_\alpha n_\gamma n_\delta, (\alpha+\gamma+\delta)^{\vee}(b) \rangle$. Let $v(a):=\epsilon_4(a)\epsilon_{11}(a)$. Define
\begin{equation*}
H:=v(a)\cdot H' = \langle n_\alpha n_\gamma n_\delta \epsilon_{12}(a^2), (\alpha+\gamma+\delta)^{\vee}(b)\rangle.  
\end{equation*}
Then $H$ is $G$-cr (by the same argument as in the previous section). Now let $M:=\langle G_{\alpha}, G_{\gamma}, G_{\delta}, G_{12}\rangle\cong A_1 A_1 A_1 A_1$. 
\begin{prop}
$H$ is not $M$-cr. 
\end{prop}
\begin{proof}
Let $\lambda:=(\alpha+2\beta+\gamma+\delta)^{\vee}$. Then $H<P_\lambda(M)=\langle T, G_{\alpha}, G_{\gamma}, G_{\delta}, U_{12}\rangle$. Let $c_\lambda: P_\lambda\rightarrow L_\lambda$ be the natural projection. Let $v:=(n_\alpha n_\gamma n_\delta \epsilon_{12}(a^2), (\alpha+\gamma+\delta)^{\vee}(b))$. We have
\begin{equation*}
c_\lambda(v)=\lim_{a\rightarrow 0}\lambda(a)\cdot (n_\alpha n_\gamma n_\delta \epsilon_{12}(a^2), (\alpha+\gamma+\delta)^{\vee}(b))= (n_\alpha n_\gamma n_\delta, (\alpha+\gamma+\delta)^{\vee}(b)).
\end{equation*}
We see that $v$ is not $R_u(P_\lambda(M))$-conjugate to $c_\lambda(v)$ since $R_u(P_\lambda (M))=U_{12}$ centralizes $n_\alpha n_\gamma n_\delta$. By Proposition~\ref{unipotentconjugate}, this shows that $H$ is not $M$-cr. 
\end{proof}
  


