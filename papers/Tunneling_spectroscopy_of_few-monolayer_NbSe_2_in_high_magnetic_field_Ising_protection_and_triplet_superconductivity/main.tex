\documentclass[showpacs,superscriptaddress,preprint,prb]{revtex4} 
\usepackage{amsmath,graphicx}
\usepackage{subfigure} 
\usepackage{braket}
\usepackage{subfigure}
\usepackage[usenames]{color}
\usepackage[dvipsnames]{xcolor}
\usepackage{ulem}
\usepackage{xcolor,soul}
%\usepackage[utf8]{inputenc}
\usepackage[T1]{fontenc}
%\usepackage[labelfont=bf,labelsep=space,justification=RaggedRight]{caption}


\begin{document}
\title {Tunnelling spectroscopy of few-monolayer NbSe$_2$ in high magnetic field: triplet superconductivity and Ising protection}
\author{M. Kuzmanović}
\thanks{These two authors contributed equally.}
\affiliation{Laboratoire de Physique des Solides (CNRS UMR 8502), Bâtiment 510, Université Paris-Saclay 91405 Orsay, France}  
\affiliation{QTF Centre of Excellence, Department of Applied Physics, Aalto University School of Science, P.O. Box 15100, 00076 Aalto, Finland}
\author{T. Dvir} 
\thanks{These two authors contributed equally}
\affiliation{Racah Institute of Physics, Hebrew University of Jerusalem, Givat Ram, Jerusalem 91904 Israel}
\affiliation{QuTech and Kavli Institute of Nanoscience, Delft University of Technology, 2600 GA Delft, the Netherlands}
\author{D. LeBoeuf} 
\affiliation{Laboratoire National des Champs Magnétiques Intenses (LNCMI-EMFL), CNRS, UGA, UPS, INSA, Grenoble/Toulouse, France} 
\author{S. Ilić} 
\affiliation{Université Grenoble Alpes, CEA, Grenoble INP, IRIG, PHELIQS, 38000 Grenoble, France}  
\affiliation{Centro de Física de Materiales (CFM-MPC), Centro Mixto CSIC-UPV/EHU, 20018 Donostia-San Sebastián, Spain}
\author{M. Haim} 
\affiliation{Racah Institute of Physics, Hebrew University of Jerusalem, Givat Ram, Jerusalem 91904 Israel} 
\author{D.Möckli} 
\affiliation{Racah Institute of Physics, Hebrew University of Jerusalem, Givat Ram, Jerusalem 91904 Israel} 
\affiliation{Instituto de F\'{i}sica, Universidade Federal do Rio Grande do Sul, 91501-970 Porto Alegre, RS, Brazil}
\author{S. Kramer} 
\affiliation{Laboratoire National des Champs Magnétiques Intenses (LNCMI-EMFL), CNRS, UGA, UPS, INSA, Grenoble/Toulouse, France} 
\author{M. Khodas} 
\affiliation{Racah Institute of Physics, Hebrew University of Jerusalem, Givat Ram, Jerusalem 91904 Israel}
\author{M. Houzet} 
\affiliation{Université Grenoble Alpes, CEA, Grenoble INP, IRIG, PHELIQS, 38000 Grenoble, France}  
\author{J. S. Meyer} 
\affiliation{Université Grenoble Alpes, CEA, Grenoble INP, IRIG, PHELIQS, 38000 Grenoble, France}  
\author{M. Aprili} 
\affiliation{Laboratoire de Physique des Solides (CNRS UMR 8502), Bâtiment 510, Université Paris-Saclay 91405 Orsay, France} 
\author{H. Steinberg} 
\affiliation{Racah Institute of Physics, Hebrew University of Jerusalem, Givat Ram, Jerusalem 91904 Israel}
\author{C. H. L. Quay} 
\affiliation{Laboratoire de Physique des Solides (CNRS UMR 8502), Bâtiment 510, Université Paris-Saclay 91405 Orsay, France}  
	

\begin{abstract}

In conventional Bardeen-Cooper-Scrieffer (BCS) superconductors, Cooper pairs of electrons of opposite spin (i.e. singlet structure) form the ground state. Equal spin triplet pairs (ESTPs), as in superfluid $^3$He, are of great interest for superconducting spintronics and topological superconductivity, yet remain elusive. Recently, odd-parity ESTPs were predicted to arise in (few-)monolayer superconducting NbSe$_2$, from the non-colinearity between the out-of-plane Ising spin-orbit field (due to the lack of inversion symmetry in monolayer NbSe$_2$) and an applied in-plane magnetic field. These ESTPs couple to the singlet order parameter at finite field. Using van der Waals tunnel junctions, we perform spectroscopy of superconducting NbSe$_2$ flakes, of 2--25 monolayer thickness, measuring the quasiparticle density of states (DOS) as a function of applied in-plane magnetic field up to 33T. In flakes $\lesssim$ 15 monolayers thick the DOS has a single superconducting gap. In these thin samples, the magnetic field acts primarily on the spin (vs orbital) degree of freedom of the electrons, and superconductivity is further protected by the Ising field. The superconducting energy gap, extracted from our tunnelling spectra, decreases as a function of the applied magnetic field. However, in bilayer NbSe$_2$, close to the critical field (up to 30T, much larger than the Pauli limit), superconductivity appears to be more robust than expected from Ising protection alone. Our data can be explained by the above-mentioned ESTPs. 

\end{abstract}

%\pacs{73.23.−b, 74.20.Rp, 74.78.−w}

\maketitle  

\section{Introduction}

In both superfluid $^3$He and conventional Bardeen-Cooper-Schrieffer (BCS) superconductors, the ground state is made up of paired spinful entities, respectively nuclei and electrons. While the superfluid $^3$He wavefunction has a spin triplet structure, conventional superconductors are spin singlet~\cite{annett}.

The question thus arises of the possible existence of triplet superconducting pairs, and in particular equal spin triplet pairs (ESTPs, linear combinations of $\ket{\uparrow \uparrow}$ and $\ket{\downarrow \downarrow}$), as have been found in $^3$He-A~\cite{leggett}. ESTPs are intimately related to topological superconductivity and Majorana edge modes~\cite{frolov}. They are also of great interest for superconducting spintronics, as they can carry spin information without dissipation~\cite{ohnishi}.

ESTPs have recently been predicted to arise in (few-)monolayer superconducting 2H-NbSe$_2$ (hereafter NbSe$_2$) in an applied in-plane magnetic field~\cite{mockli2020}, as follows: 

Monolayer transition metal dicalcogenides (TMDs), such as NbSe$_2$, with 2H structure lack in-plane crystal inversion symmetry; this gives rise, via the spin-orbit interaction, to an effective out-of-plane magnetic field $H_{SO}$ known as the `Ising (spin-orbit) field'~\cite{frigeri2004,gorkov2001,saito2016,lu2015}. $H_{SO}$ is momentum-dependent; in particular, it has opposite sign at K and K' points of the hexagonal Brillouin zone~\cite{xu2014}, and a predicted amplitude~\cite{wickramaratne2020} of $\mu_B H_{SO} = E_{SO} \approx 100$ meV in monolayer NbSe$_2$. As it is time-reversal invariant, $H_{SO}$ does not affect the strength of singlet superconductivity; however, it causes Cooper pair spins to point out-of-plane (Figure~\ref{Figure0}(a)) --- unlike conventional superconductors, where Cooper pairs' internal spin axes have no preferred direction.

Thus, an applied in-plane magnetic field $H_{||}$ never completely aligns Cooper pair spins in the plane even when the Zeeman energy $E_Z = \mu_{B}H_{||} \gg E_{SO}$: at zero temperature, the in-plane critical field $H_c$ is expected to diverge logarithmically~\cite{frigeri2004,ilic2017}. In agreement with these expectations, TMD superconductors of (few-)monolayer thicknesses obtained by exfoliation \cite{wang2012} or single-layer deposition ~\cite{zhang2010,fan2015} show critical fields $H_c$ much larger than $\mu_B H_P = \Delta_0/ \sqrt{g}$, the Pauli or paramagnetic limit~\cite{xi2016,liu2018}$^,$ \footnote{At the Pauli limit, the paramagnetic state of spin-aligned quasiparticles becomes more energetically favourable than the superconducting ground state~\cite{clogston1962, fulde1973}. In the absence of spin-orbit coupling, $H_P$ is thus the expected $H_c$ for thin superconductors, where the Meissner effect can be neglected.}. (Here $\mu_B$ is the Bohr magneton, $g$ the Landé g-factor and $\Delta_0$ the superconducting order parameter at zero field.) While inversion symmetry is restored in even-layered NbSe$_2$, the enhancement of $H_c$ persists in bilayer and few-monolayer TMDs, with $H_c$ decreasing monotonically with increasing NbSe$_2$ thickness~\cite{xi2016,barrera2018,prober1980}, and no observation of even-odd effects. This has been attributed to weak inter-layer coupling (compared to $E_{SO}$)~\cite{jones2014} and/or spin-layer locking~\cite{xi2016}.

Both singlet and opposite-spin triplet superconducting order parameters can exist at zero magnetic field, with spin structures respectively $ \Phi_s= \ket{\uparrow \downarrow} - \ket{\downarrow \uparrow}$ and $\Phi_t = \ket{\uparrow \downarrow} + \ket{\downarrow \uparrow}$~\cite{smidman2017}. For $E_{SO} < E_F$ (the Fermi energy), which is the case here, $ \Phi_t$ and $\Phi_s$ decouple, and $\Phi_t$ should not coexist with $\Phi_s$~\cite{rainer1998,haim2020}. ($\Phi_t$ is in any case sensitive to disorder and disappears when the mean free path is shorter than the superconducting coherence length~\cite{mockli2020}.)

The applied in-plane field $H_{||}$, due to its non-colinearity with the Ising field, as well as the momentum dependence of the latter, results in ESTPs with spin structure $\Phi_{tB}= \ket{\downarrow  \downarrow} + \ket{\uparrow \uparrow} $~\cite{mockli2020, tang2020, nakamura2017} (Figure~\ref{Figure0}(b)). $\Phi_{tB}$ is coupled to $\Phi_{s}$ by the in-plane field, and the critical field is affected by their symbiotic relationship~\cite{mockli2019}: $\Phi_{tB}$ enables $\Phi_{s}$ to survive the magnetic field, while $\Phi_{s}$ enables $\Phi_{tB}$ to survive disorder. As a result, in a disordered sample, or even when the temperature $T>T_{ct}$ (the critical temperature associated with $\Phi_{tB}$), the in-plane critical field is higher than it would be for either $\Phi_{s}$ or $\Phi_{tB}$ alone, and the dependence of the superconducting gap $\Delta$ on the applied field is also affected (Figure~\ref{Figure0}(c)).

Very recently, a two-fold anisotropy of critical field, non-linear transport and magneto-resistance was observed in few- and mono-layer NbSe$_2$ devices close to the transition to the normal state~\cite{Hamill_2020,Cho_Lortz_2020}. These results were also interpreted as coming from unconventional superconductivity: $\Phi_{tB}$ triplet components induced by the applied magnetic field and lateral lattice strain can reduce the six-fold rotational symmetry expected from the hexagonal crystal lattice to two-fold symmetry~\cite{Hamill_2020,Cho_Lortz_2020,haim2022}. 



\begin{figure}	
	\includegraphics[width=0.5\textwidth]{figure0.pdf}	
%	\vspace{-15pt}
	\caption{(a) At zero magnetic field, singlet Cooper pairs are composed of electrons at opposite corners of the hexagonal NbSe$_2$ Brillouin zone (K and K' points). Their spins are pinned out-of-plane by the Ising field. (b) An in-plane magnetic field partially aligns electron spins orthogonal to the Ising field, and gives rise to odd-parity equal-spin triplet pairs~\cite{mockli2020}. (c) Theoretical superconducting gap as a function of in-plane magnetic field. Compared to the case where only the Ising field is considered (black line), superconductivity is even more robust to the in-plane magnetic field and the $\Delta$ vs $H_{||}$ curve has a `flattened' shape at intermediate fields (blue line). The triplet component of the order parameter (red line) with spin structure $\Phi_{tB}$  survives disorder through its coupling to the singlet component, which has spin structure $\Phi_{s}$. The temperature used in the calculations is 0.5 $T_c$, the critical temperature.
	% (c,e) Coupling between K/K' and $\Gamma$ pockets can also result in a similar, flattened $\Delta$ vs $H_{||}$, albeit for different values of spin-orbit coupling from that in (d). 
	See Figure~\ref{Figure3} and the text for details and comparison to data.
	}
	\label{Figure0}
\end{figure}


Here we report tunnelling spectroscopy of few-monolayer NbSe$_2$ devices over a wide range of applied in-plane magnetic fields, up to 30 T. As the magnetic field increases, our measurement of the superconducting gap $\Delta$ progressively deviates from the prediction based on pure singlet pairing. We find that this field-induced deviation can be explained by the onset of ESTPs in the form of $\Phi_{tB}$ (Figure~\ref{Figure0}).




\section{Results}

We consider a single-band superconductor, with hole pockets at the K/K' points, and include $\Phi_s$ and $ \Phi_{tB}$ correlations. As mentioned above, $\Phi_{tB}$ is coupled linearly to $\Phi_{s}$, and is expressed even when $\Delta_{tB}<\Delta{s}$. If we neglect inter-valley scattering, and if a finite pairing interaction is present in the $\Phi_{tB}$ channel as suggested by recent density functional theory (DFT) calculations~\cite{wickramaratne2020}, the superconducting energy gap $\Delta$ can be obtained from the quasiclassical theory of superconductivity (cf. Equation S14 in Supp. Info.) :

\begin{equation} \label{singlet-triplet}
\Delta= (E_{SO}\Delta_{s}+ E_Z  \Delta_{tB})/\sqrt{E_{SO}^2+E_{Z}^2},
\end{equation} 

\noindent where $\Delta_{s}$ and $\Delta_{tB}$ are, respectively, the singlet and equal-spin triplet order parameters. 

Here we can see that, compared to the case of $\Phi_{s}$ with Ising protection alone, the coexistence of $\Delta_{tB}$  with $\Delta_{s}$ and the coupling between the two can increase the robustness of superconductivity against an applied in-plane magnetic field. In the case where there is no pairing in the equal-spin triplet channel ($\Delta_{tB} = 0$), $\Delta$ is reduced by the magnetic field through the factor $E_{SO}/\sqrt{E_{SO}^2+E_{z}^2}$ and vanishes asymptotically, giving the afore-mentioned logarithmic divergence of the critical field at zero temperature. To obtain the order parameters $\Delta_{s}$ and $\Delta_{tB}$ at finite temperature and magnetic field, one has to solve two coupled equations self-consistently (cf. Supp. Info. II). 

The quasiclassical theory also gives the density of states (DOS), which is found for $E<E_{SO}$ to be simply the BCS DOS, with the gap as in Equation~\ref{singlet-triplet} (see Equation 14 in the Supp. Info.) --- unlike 2D superconductors with low spin-orbit coupling in in-plane fields, the coherence peak is not Zeeman-split~\cite{meservey1970}. In addition, Ising protection gives a sharp coherence peak in the DOS, regardless of the strength of the triplet coupling or the applied magnetic field. Nevertheless, in the presence of inter-valley scattering ($\tau_{iv}$ being the inter-valley scattering time), Ising protection is reduced (due to averaging over valleys with opposite signs of $H_{SO}$), the DOS is smeared out \cite{haim2020} as in the Abrikosov-Gor'kov theory~\cite{abrikosov,bruno1973}, and the divergence of the critical field at zero temperature is regularised~\cite{ilic2017}. In the limit of strong inter-valley scattering ($E_{SO}^2/\Delta_s \gg 1/\tau_{iv} \gg \Delta_0$) the dependence of $\Delta$ on the applied magnetic field becomes similar to that expected from the Abrikosov-Gor'kov theory with the critical field given by  $\mu_B H_c = E_{SO} \sqrt{2 \Delta \tau_{iv} /\hbar}$. In our experiment, we do not have strong inter-valley scattering as $1/\tau_{iv} \lesssim \Delta_s$. (See Supp. Info. IC)

 \begin{figure}	
	\includegraphics[width=0.75\textwidth]{figure1.pdf}	
%	\vspace{-15pt}
	\caption{Tunnelling spectroscopy of bulk and few-monolayer NbSe$_2$ through van der Waals barriers. (a) Schematic drawing of the tunnel junction: few-monolayer NbSe$_2$, covered with few-monolayer WSe$_2$ (or MoS$_2$) and a Ti/Au electrode. A voltage $V$ is applied between the Ti/Au electrode and the NbSe$_2$ and a current $I$ measured. (b) Differential conductance ($G = dI/dV$) as a function of $V$ across J2 (blue) and $d^2I/dV^2$ vs $V$ of J2 (red). A double peak/dip can be seen in $d^2I/dV^2$, due to the presence of two superconducting gaps. (c) Same as panel (b) for J6. (d)-(h) Colormaps of the magnetic field dependence of the $d^2I/dV^2$ curves for junctions J1--J5. The double dip/peak feature (yellow/blue regions) disappears in thin samples; a single gap is left. Measurements were taken at temperatures of 30-70 mK. }
	\label{Figure1}
\end{figure}


We fabricate tunnel junctions (J1 to J7) on superconducting NbSe$_2$ flakes of $1.2-50$ nm thickness. The tunnel barriers are thin flakes of semiconducting $\mathrm{WSe_2}$ or $\mathrm{MoS_2}$ exfoliated by the van der Waals dry transfer technique described in Ref.~\cite{dvir}. A Ti/Au normal counter electrode is then evaporated on the semiconductor leading to a structure shown schematically in Figure~\ref{Figure1}(a). An ohmic contact to the NbSe$_2$ is also fabricated. The typical surface area of the junction is about 1~µm$^2$ and the resistance in the normal state $>$10~k$\Omega$. The critical temperature $T_c$ decreases from $\sim$7.2 K in the thickest flakes to $\sim$2.6 K in the thinnest ones. 

Using standard lock-in techniques, we first measure the current $I$ and differential conductance $G=dI/dV$ across the junctions as a function of applied bias voltage $V$~\cite{giaever1960} and in-plane magnetic fields $H_{||}$ in dilution refrigerators with base temperatures of 30--70 mK. $G(V)$ is proportional to the DOS convolved with the derivative of the Fermi distribution function \cite{tinkham}. Therefore, in principle, the energy resolution of our spectroscopy is given by the temperature and the integrated voltage noise across the junction. 

Typical $G(V)$ curves are shown for a 25 nm thick sample (J2) and a 6 monolayer sample (J6) in Figure~\ref{Figure1}(b,c), top panels. The main differences between these junctions are: (1) the smaller superconducting gap in the thinner device due to a smaller $T_c$, and (2) the low-energy shoulder, very clearly seen in the thicker junction, is absent in the thinner one. This is even more apparent in the second derivative of the current as a function of the voltage bias, $dG/dV$, in Figure~\ref{Figure1}(b,c): the two peaks in J2, merge to a single peak in J6. This merging was previously observed \cite{dvir, khestanova2018} and we now see that it persists in flakes up to 11 nm ($\approx 15$ monolayer) thick: the two-gap superconductivity of bulk NbSe$_2$~\cite{noat2015} is lost. This is consistent with band structure calculations for bulk and monolayer NbSe$_2$: whereas in the bulk three bands cross the Fermi level~\cite{johannes,yokoya}, and two superconducting gaps have been observed~\cite{dvir}, in the monolayer a single band remains, which crosses the Fermi level twice, resulting in hole pockets at the K/K' and $\Gamma$ points~\cite{wickramaratne2020}. A single-band theory thus seems most suitable for the thinnest flakes. 

Figures~\ref{Figure1}(d-h) show the evolution of the $dG/dV$ curves of five junctions (J1-J5) with increasing in-plane magnetic field. Junctions 1 and 2, the thickest, show similar responses to the applied field: the inner peak shifts to lower energies faster than the outer peak. This is consistent with previous experiments, and is likely due to the 3D character of the Se $p_z$-orbital-derived band at the $\Gamma$ point, which is associated with the smaller superconducting energy gap, as well as its higher diffusion coefficient~\cite{dvir,dvir_nanoletters}. For the thinner junctions, J4 and J5, a single gap persists from zero field up to 9 T.
%in agreement with recent results on MoS$_2$~\cite{costanzo2018}. 

As noted above, the robustness of the gap to applied magnetic fields is expected in thin samples due to Ising protection and drastically reduced orbital depairing (Meissner effect). To significantly reduce the gap and to study the effect of the applied field on the density of states it is necessary to go to even higher fields. 

 \begin{figure}[h!]
	\includegraphics[width=0.5\textwidth]{figure2.pdf}	 
	%\vspace{-15pt}
	\caption{Differential conductance $G = dI/dV$ as a function of the bias voltage $V$ and of the in-plane magnetic field $H_{||}$ of J6 (6 monolayers, top panels) and J7 (bilayer, bottom panels). The tunnelling spectra are normalized by the normal state conductance, $G_N(V)$, measured above $H_c$. (a,d) Colormap of $G(V)$ as a function of field at 1.3 K. The dotted lines indicate the critical fields. (b,e) Horizontal slices of the data in the colormaps (a) and (d) respectively, showing $G(V)$ at different fields, vertically displaced for clarity. The black lines are fits to an Abrikosov-Gor'kov-like (A-G-like) density of states, with the energy gap and A-G broadening parameter as fitting parameters. (The gap is not determined self-consistently.) (c,f) Data at 50mK and zero magnetic field (red lines) together with the fits obtained using a BCS DOS and an effective temperature (black lines). The superconducting gaps obtained from the fits are, respectively, 800$\mu$eV and 400$\mu$eV, while the effective temperatures are respectively 0.9K and 1K.}
	\label{Figure2}
\end{figure}

Therefore, we measure two tunnel junctions (J6, 6 monolayers) and (J7, bilayer) in in-plane magnetic fields of up to 33 T at 1.3 K (pumped liquid helium). Their critical temperatures are, respectively, 5.4 K ($H_P = 10.5$ T) and 2.6 K ($H_P = 5$ T), giving $\Delta/k_B T_c\approx 1.8$, close to the BCS prediction and in agreement with previous studies~\cite{dvir,khestanova2018}. (See Figure S2 and inset.) Finally, the critical in-plane fields are $H_c$ = 18 T for J6 and $H_c$ = 30 T for J7, corresponding respectively to $H_c=1.5 H_P$ and $H_c=6 H_P$. (See Figure~\ref{Figure3}.) These junctions had earlier been characterized at 50mK (dilution refrigerator) at zero magnetic field (Figures~\ref{Figure2}(c) and (f)) --- hard gaps were observed, pointing to tunnelling as the main transport mechanism. These tunnel spectra are well-described by a fit to a BCS density of states, broadened by a $\sim 200\mu eV$ effective temperature. Though higher than the bath temperature, this broadening does not affect the determination of the energy gap, which can be done with high precision. (See Supp. Info. IA and IB for details.)

\section{Discussion}

The evolution of $G(V)$ with the in-plane magnetic field at 1.3K is shown in Figures~\ref{Figure2}(a) (J6) and ~\ref{Figure2}(d) (J7). For clarity, spectra at selected magnetic fields are also shown in Figures~\ref{Figure2}(b) and 2(e) together with an Abrikosov-Gor'kov-like density of states with a field-dependent broadening parameter\cite{abrikosov,srivastava2008}, convolved with a Fermi function to account for the temperature. The fits account very well for the experimental data. 

The superconducting gaps obtained from these fits are shown as a function of the in-plane magnetic field in Figure~\ref{Figure3}, and compared to theory. 

For the six-monolayer device, a simple Ising model accounts for the data reasonably well (Figure~\ref{Figure3}, dashed dark blue line). The fitting parameters are given in the caption of the figure. 

Focusing on the thinner, bilayer device (J7), we see that the Ising theory alone without triplet pairing 
%or K/K'-$\Gamma$ coupling 
fits the data reasonably well up to about 20T, but not close to the critical field, where the superconducting energy gap is more robust than expected (Figure~\ref{Figure3}, dashed dark blue line). 

This key experimental finding is suggestive of a second order parameter, which is revealed as the dominant order parameter disappears \cite{rainer1998}. Indeed introducing a small ESTP component of the gap (triplet model), a better fit of the overall experimental data is obtained (Figure~\ref{Figure3}, brown line). The temperature of the experiment (1.3K) is above the triplet critical temperature ($T_{ct}=0.05T_{cs}=$130mK, obtained from the fit). Therefore, the ESTP order parameter $\Delta_{tB}$ exists only through its coupling with the singlet order parameter $\Delta_s$, and its main effect is to enhance the critical field through the coupling with the singlet order parameter. In addition, the triplet subdominant component also renders the gap vs field dependence more linear (Figure~\ref{Figure0}(c)). 


Our fit also gives $E_{SO}$ = $9.62 T_{cs}$ ($\sim$ 2.2 meV). This is a lower bound for $E_{SO}$, as inter-valley (K-K') scattering is not taken into account. If it is, higher values of $E_{SO}$ will have to be used to arrive at the same $H_{||}^c$, but the shape of the $\Delta(H_{||})$ curve is similar. Further, we note that the shape of $\Delta(H_{||})$ in the triplet model is by construction impervious to intra-valley scattering. Our $E_{SO}$ value is consistent with the upper bound for $E_{SO}$ given by angle-resolved photo-emission spectroscopy (ARPES) measurements, which indicate $E_{SO} \lesssim 20$meV (the measurement resolution), significantly lower than theoretical predictions~\cite{xu2018}. 

For completeness, we also show the Ising theory with strong inter-valley scattering (equivalent to Abrikosov-Gor'kov), where the only fitting parameter is the critical field (Figure~\ref{Figure3}, black line). This does not fit the data at all -- the experimental $\Delta$ is consistently smaller than that predicted by the theory, which also fails to reproduce the `linear' part of the curve at intermediate fields.

At present, much of the literature on quantum transport in few-layer NbSe$_2$ includes only the hole pockets at the K/K' points, as we did, even though there is also a hole pocket at the $\Gamma$ point~\cite{wickramaratne2020}. In Supp. Info. IIC and IID, we consider models which include only $\Phi_s$ and K/K'-$\Gamma$ coupling, and neglect all triplet order parameters. These are found not to fit our data well, given the known level of disorder in the sample, thus strengthening the case for the ESTP interpretation. 

\begin{figure}[h!]
	\includegraphics[width=0.5\textwidth]{figure3.pdf}	
%	\vspace{-15pt}
	\caption{Normalized superconducting gap as a function of the in-plane magnetic field $\Delta(H_{||})$ obtained from the fits of the quasiparticle density of states in Figure~\ref{Figure2}. The error bars were obtained following the procedure described in Supp. Info. IA. The dark blue dashed lines are a fit of experimental data using the Ising theory alone. Here, $E_{SO}=14.45 T_{cs}$ (with $T_{cs}=2.6K$) for the bilayer and $E_{SO}=2.21 T_{cs}$ (with $T_{cs}=5.4K$) for the 6-monolayer. In brown is the Ising theory with an equal-spin triplet component of the order parameter as described in the text. Here, $E_{SO}=9.62 T_{cs}$ and $T_{ct}=0.05 T_{cs}$, with $T_{cs}=2.6K$. 
	%In light blue is the Ising theory with Suhl-Matthias-Walker coupling between K/K' and $\Gamma$ hole pockets. Here $E_{SO}^K = 39.78 T_{cs}$, $E_{SO}^{\Gamma} = 18.59 T_{cs}$ and the inter-pocket coupling is equal to the intra-pocket BCS term. 
	Finally, the solid black line is calculated using the Ising theory with strong disorder (equivalent to the Abrikosov-Gor'kov theory). In all cases, the critical field is constrained to be the experimental one.}
	\label{Figure3}
\end{figure}

Regarding ESTPs, we note that, within the scenarios of Refs.~\cite{Hamill_2020,Cho_Lortz_2020} mentioned earlier, the triplet order parameters allowed by symmetry such as $\Phi_{tB}$ have to be nearly degenerate with the leading singlet order parameter. Attraction in the triplet channel is supported by recent density functional theory (DFT) calculations \cite{wickramaratne2020}; however, there is at present no evidence of near degeneracy between triplet and singlet channels. Our interpretation does not require near degeneracy, and the singlet-triplet coupling comes from a clear microscopic mechanism (the in-plane magnetic field), which is quantitatively accounted for both in the theory and in the analysis of the experimental data.

While previous reports on Andreev spectroscopy experiments have shown a reduction of the gap consistent with field-induced depairing in the presence of Ising protection~\cite{sohn2018}, our hard-gap tunnel junctions allow a nuanced and quantitative analysis of the possible microscopic mechanisms for the enhancement of the critical field, pointing to the presence of equal-spin triplet superconductivity.

Further study at even lower temperatures, independent measurements of $E_{SO}$, independent estimates of the K/K'-$\Gamma$ coupling from theory or experiment, and momentum-selective barriers would be helpful to unambiguously confirm the existence of ESTPs in NbSe$_2$. 


\section{Materials and Methods} Especially at high magnetic fields, special care was taken to ensure that the applied magnetic field is parallel to the flakes. It is aligned to better than $\sim$1$^\circ$. In addition, we checked that the voltage noise due to mechanical vibrations is lower than that from the thermal broadening. This is described in detail in Supp. Info. IA.

\section{Data Availability} The datasets generated and/or analysed during the current study are available from the corresponding author upon reasonable request.
 
\section{Acknowledgements} We acknowledge valuable discussions with Pascal Simon and Freek Massee, and thank the latter for a careful reading of the manuscript. This work was funded by a Maimonides-Israel grant from the Israeli-French High Council for Scientific and Technological Research; JCJC (SPINOES), PIRE (HYBRID) and PRC (TRIPRES) grants from the French Agence Nationale de Recherche; ERC Starting Grant ERC-2014-STG 637928 (TUNNEL); Israel Science Foundation Grants No.s 861/19 and 2665/20, and the Laboratoire d’Excellence LANEF in Grenoble (ANR10-LABX-51-01). T.D. is grateful to the Azrieli Foundation for an Azrieli Fellowship. Part of this work has been performed at the Laboratoire National de Champs Magnétiques Intenses (LNCMI), a member of the European Magnetic Field Laboratry (EMFL).

\bibliographystyle{naturemag}

\bibliography{references}

\end{document}