\documentclass[showpacs,superscriptaddress,onecolumn,prb]{revtex4} 
\usepackage{amsmath,graphicx}
\usepackage{subfigure} 
\usepackage{braket}
\usepackage{subfigure}
\usepackage[usenames]{color}
\usepackage[dvipsnames]{xcolor}
%\usepackage[usenames,dvipsnames]{xcolor}
\usepackage{ulem}
\usepackage[english]{babel}
\usepackage[utf8]{inputenc}
\usepackage[T1]{fontenc}
\usepackage[labelfont=bf,labelsep=space,justification=RaggedRight]{caption}

\usepackage{bbold}
\usepackage{comment}
\usepackage{dcolumn}
\usepackage{bm}
\usepackage[colorlinks=true,linkcolor=blue,citecolor=blue,urlcolor=blue]{hyperref}
\newcommand{\Tr}{\mathop{\rm Tr}\nolimits}
\renewcommand{\bm}{\mathbf}

\newcommand{\subsubsubsection}[1]{\paragraph{#1}\mbox{}\\}
\setcounter{secnumdepth}{4}
\setcounter{tocdepth}{4}


\usepackage{placeins}

\begin{document}

\widetext

\preprint{APS/123-QED}

\title {\large SUPPLEMENTAL MATERIAL for\\[0.25cm] \lq\lq Tunnelling spectroscopy of few-monolayer NbSe$_2$ in high magnetic field: triplet superconductivity and Ising protection\rq\rq}
\author{M. Kuzmanović}
\thanks{These two authors contributed equally}
\affiliation{Laboratoire de Physique des Solides (CNRS UMR 8502), Bâtiment 510, Université Paris-Saclay 91405 Orsay, France}  
\affiliation{QTF Centre of Excellence, Department of Applied Physics, Aalto University School of Science, P.O. Box 15100, 00076 Aalto, Finland}
\author{T. Dvir} 
\thanks{These two authors contributed equally}
\affiliation{Racah Institute of Physics, Hebrew University of Jerusalem, Givat Ram, Jerusalem 91904 Israel}
\affiliation{QuTech and Kavli Institute of Nanoscience, Delft University of Technology, 2600 GA Delft, the Netherlands}
\author{D. LeBoeuf} 
\affiliation{Laboratoire National des Champs Magnétiques Intenses (LNCMI-EMFL), CNRS, UGA, UPS, INSA, Grenoble/Toulouse, France} 
\author{S. Ilić} 
\affiliation{Université Grenoble Alpes, CEA, Grenoble INP, IRIG, PHELIQS, 38000 Grenoble, France}  
\affiliation{Centro de Física de Materiales (CFM-MPC), Centro Mixto CSIC-UPV/EHU, 20018 Donostia-San Sebastián, Spain}
\author{M. Haim} 
\affiliation{Racah Institute of Physics, Hebrew University of Jerusalem, Givat Ram, Jerusalem 91904 Israel} 
\author{D.Möckli} 
\affiliation{Racah Institute of Physics, Hebrew University of Jerusalem, Givat Ram, Jerusalem 91904 Israel} 
\affiliation{Instituto de F\'{i}sica, Universidade Federal do Rio Grande do Sul, 91501-970 Porto Alegre, RS, Brazil}
\author{S. Kramer} 
\affiliation{Laboratoire National des Champs Magnétiques Intenses (LNCMI-EMFL), CNRS, UGA, UPS, INSA, Grenoble/Toulouse, France} 
\author{M. Khodas} 
\affiliation{Racah Institute of Physics, Hebrew University of Jerusalem, Givat Ram, Jerusalem 91904 Israel}
\author{M. Houzet} 
\affiliation{Université Grenoble Alpes, CEA, Grenoble INP, IRIG, PHELIQS, 38000 Grenoble, France}  
\author{J. S. Meyer} 
\affiliation{Université Grenoble Alpes, CEA, Grenoble INP, IRIG, PHELIQS, 38000 Grenoble, France}  
\author{M. Aprili} 
\affiliation{Laboratoire de Physique des Solides (CNRS UMR 8502), Bâtiment 510, Université Paris-Saclay 91405 Orsay, France} 
\author{H. Steinberg} 
\affiliation{Racah Institute of Physics, Hebrew University of Jerusalem, Givat Ram, Jerusalem 91904 Israel}
\author{C. H. L. Quay} 
\affiliation{Laboratoire de Physique des Solides (CNRS UMR 8502), Bâtiment 510, Université Paris-Saclay 91405 Orsay, France}  
	

\maketitle 

\renewcommand{\theequation}{S\arabic{equation}}
\renewcommand{\thefigure}{S\arabic{figure}}
\renewcommand{\bibnumfmt}[1]{[S#1]}
\renewcommand{\citenumfont}[1]{S#1}

The supplementary material is divided into two parts. Part \ref{partI} covers the experimental methods: section \ref{partI:GV} details how the order parameter was extracted from the spectroscopic data and gives an estimate of the uncertainty, while sections \ref{partI:DOSbroadening} and \ref{partI:scattering} comment on the broadening of the density of states and the disorder respectively. Part \ref{partII} presents the details of the theoretical model, where a general overview is given in section \ref{partII:hamiltonian}, followed by the triplet model in section \ref{sec-KK'}. At the end two-pocket models are discussed: 
The possibility of a Suhl-Matthias-Walker coupling is discussed in section \ref{sec:two-pockets-MSW}, and is ruled out on the basis of disorder, and a McMillan coupling is discussed in section \ref{sec-mm}, which is ruled out based on spectroscopic and transport data.

\bigskip

\section{The sample and experimental methods}\label{partI}

The devices reported in this work were fabricated in a similar method to those reported in ref \cite{dvir2018}. First, $\mathrm{NbSe_2}$ was exfoliated within a glovebox with an inert N2 environment onto a silicon chip covered with 285nm of $\mathrm{SiO_2}$. Next, $\mathrm{WSe_2}$ was exfoliated on a PDMS stamp and examined to look for thin flakes. Once a suitable flake was found, it was aligned and brought into contact with a thin flake of $\mathrm{NbSe_2}$. Next, the heterostructure was removed from the glovebox and tunnel contacts (5nm/80nm Ti/Au) were deposited on the $\mathrm{WSe_2}$ barrier. Finally, ohmic contacts (5nm/80nm Ti/Au) where deposited directly on the $\mathrm{NbSe_2}$ after Ar milling to remove oxide layers. The resulting device is shown in figure \ref{fig:sample}. The two junctions focused on in the main text are J7, on top of a 2ML $\mathrm{NbSe_2}$ flake, with a tunneling resistance of $R_1\approx 1.8\mathrm{M\Omega}$, and J6, on top of a 4-8ML $\mathrm{NbSe_2}$ region,  with a resistance of $R_2\approx 12\mathrm{k\Omega}$.

\begin{figure}[h]
	\centering
	\includegraphics[width=0.5\textwidth]{figures/device_fab_new.pdf}
	\caption{
		a. $\mathrm{NbSe_2}$ exfoliated on $\mathrm{SiO_2}$ from a PDMS stamp. Flake thickness is
		determined from the optical contrast. b. $\mathrm{WSe_2}$ exfoliated on a PDMS stamp.
		Flake thickness is determined from the optical contrast. c. $\mathrm{NbSe_2 - WSe_2}$
		heterostructure formed by the deterministic transfer of the WSe 2 . d. The
		final device with multiple tunnel junctions (J6,J7) and ohmic contacts
		(Drain).
	}
	\label{fig:sample}
\end{figure}

As a part of the sample characterization the differential conductance of both junctions was measured at $T=50\mathrm{mK}$ (in a $\mathrm{He^3-He^4}$ dilution refrigerator) (shown in figure 3 of the main text). The extracted values for $\Delta$ are in line with the critical temperature $T_c$, estimated from the $G(V=0)$ temperature dependence - see figure \ref{fig:GvsT}.
This, as well as the data reported in \cite{khestanova2018unusual}, shows that for few-layer samples, which are of main interest here, the order parameter $\Delta$ and the critical temperature $T_c$ follow the BCS prediction $\Delta= 1.76k_bT_c$.

\begin{figure}[h]
	\centering
	\includegraphics[width=0.5\textwidth]{figures/G_vs_T_Tc_J_labels_new_with_turining_point.pdf}
	\caption{The temperature dependence of the zero voltage conductance, $G(V=0)$, for devices J7 (2ML) and J6 (6ML).
		Inset the $\Delta$ vs the estimated $T_c$: for J7 and J6 (red dots), the data from \cite{dvir2018spectroscopy} (blue circles) and the BCS result $\Delta=1.76k_bT_c$ (black line).
	}
	\label{fig:GvsT}
\end{figure}


The high field measurements were performed at the "Laboratoire National des Champs Magnétiques Intenses" in Grenoble, France. The magnet used allowed for magnetic fields up to $H=36\mathrm{T}$.

The sample was cooled down to $T=1.25\mathrm{K}$ in a pumped $\mathrm{He^4}$ cryostat, and the differential conductance was measured using a standard lock-in technique with an excitation of $V_{lock-in}=100\mathrm{\mu V}$. Although the excitation voltage is relatively large, it does not lead to a significant smearing of the obtained spectrum as $eV_{lock-in}\ll3.5k_B T \approx 360\mu eV$ (where $3.5k_b T$ is the FWHM of the Fermi distribution transition). Likewise the electrical noise, generated by the magnet flux noise is also smaller than the temperature - see figure \ref{fig:noise}. The sample was mounted on a rotating stage, the axis of which was perpendicular to the magnetic field. A field of $H=2\mathrm{T}$ was applied and the sample was rotated, while measuring the height of the coherence peaks. By maximizing the peak height the magnetic field was aligned with the plane of the sample, with a maximum deviation of up to $1^\circ$. A misalignment of $\sim$1$^\circ$ (the maximum realistically  possible in our experiment), gives a perpendicular component of the magnetic field smaller than 0.3 T, several times smaller than the perpendicular critical field \cite{tsen2016,haim2020signatures}, thus it should not substantially decrease the observed critical field $H_c$.

\begin{figure}[h]
	\centering
	\includegraphics[width=0.7\textwidth]{figures/noise.pdf}
	\caption{Electrical noise measured at $T=1.2\mathrm{K}$, across  a $R=10\mathrm{k \Omega}$ resistor in a bandwidth of $\mathrm{DC}-32\mathrm{kHz}$, as a function of the magnetic field. The left panel show the noise spectrum, while the right one shows the integrated noise voltage.
	}
	\label{fig:noise}
\end{figure}

\subsection{$G(V)$ curve fitting}\label{partI:GV}

With the main goal of determining the order parameter as a function of the magnetic field, $\Delta(H)$, the following heuristic approach was adopted.
Without a proper microscopic theory to describe the soft gapped spectrum in presence of weak inter-valley scattering, and with limited ability to discern the details of the density of states (DOS) at $T=1.3\mathrm{K}$, the simplest approach to take is to try several different models for the $G(V)$ traces and compare the results. Even though there are countless models that one can utilize the discussion here is limited to the following three: an effective temperature $T^*$ with a BCS DOS, an Abrikosov-Gor'kov DOS (A-G DOS)  ~\cite{AG_DOS}  and a Dynes DOS   ~\cite{Dynes_DOS}. At finite temperatures the $G(V)$ curve is obtained by convolving the DOS with a distribution $\approx 3.5k_B T$ wide (FWHM): \large{$G(V) = \frac{1}{eR} \int_{-\infty}^{\infty} dE N(E) \frac{\partial f(E-V)}{\partial V}$}\normalsize, where $N(E)$ is the normalized density of states.  If $3.5k_B T \approx \Delta$ this leads to a finite conductance at zero voltage bias. The Abrikosov-Gor'kov depairing model lowers the spectral gap below the order parameter $\Delta$. When the depairing energy equals the order parameter, $\alpha(H)=\Delta(H)$ this leads to a gapless state, but even when $\alpha(H)<\Delta(H)$  along with a finite temperature the resulting $G(V)$ trace can be gapless. Lastly, the Dynes DOS is non-zero at $E=0$, directly leading to a soft gap. To illustrate this a simple example is shown in figure \ref{fig:GV_toy_model} - the top panel show the BCS, Abrikosov-Gor'kov and Dynes densities of states, as well as the distribution function $\frac{\partial f(E-V)}{\partial V}$ for some parameter values, while the bottom panel shows the corresponding differential conductance traces. Although the BCS and the Abrikosov-Gor'kov DOS' are fully gapped the resulting spectra are quite similar, and resemble a gapples spectrum.

\begin{figure}[h]
	\centering
	\includegraphics[width=0.5\textwidth]{figures/GV_toy_model.pdf}
	\caption{Top: The BCS density of states (blue, $\Delta=1$), the Abrikosov-Gor'kov one (red, $\Delta=1.15$, $\alpha=0.38$) and the Dynes one (yellow, $\Delta=1.05$, $\Gamma=0.24$). The dashed lines are the derivatives of the Fermi distribution function for $T=0.25\Delta$ (black) and $T=0.4\Delta$ (blue). Bottom: the corresponding $G(V)$ traces. The BCS DOS was convolved with the higher temperature distribution function, while the other two were convolved with the lower temperature one.
	}
	\label{fig:GV_toy_model}
\end{figure}


The parameters for the effective temperature model were the (field dependent) order parameter $\Delta(H)$ and the temperature $T^*(H)$. For the Abrikosov-Gor'kov and Dynes fits the temperature was fixed to $T=1.25\mathrm{K}$, while the depairing and Dynes energies were fitting parameters. The $\Delta(H)$ dependence obtained in this way, for both junctions and all three models, is shown on figure \ref{fig:AG-Dynes-T}.

\begin{figure}[h]
	\centering
	\includegraphics[width=0.7\textwidth]{figures/AG-Dynes-T-comparison.pdf}
	\caption{The extracted order parameters as function of the applied magnetic field using the three different $G(V)$ models, for the J7 (2ML, left panel), and J6 (4-8ML, right panel).
	}
	\label{fig:AG-Dynes-T}
\end{figure}

It is important to note that the values of these extra field dependent parameters, have no physical significance: the gap value is not self-consistently determined, nor should they be interpreted in the context of their usual meaning. They are rather just phenomenological parameters used to describe the obtained $G(V)$ spectra. To illustrate this figure \ref{fig:AG_and_Dynes_SC_vs_fit} shows the self-consistent and experimentally obtained $\Delta$ versus the Abrikosov-Gor'kov and Dynes $\Gamma$ parameters. 

\begin{figure}[h]
	\centering
	\includegraphics[width=0.5\textwidth]{figures/AG_and_Dynes_SC_vs_fit.pdf}
	\caption{The order parameter $\Delta$ as a function of the Abrikosov-Gor'kov depairing (blue) and the Dynes energy (red) for J7, obtained from the fitting (dots) and the self-consistent gap equation (full lines).
	}
	\label{fig:AG_and_Dynes_SC_vs_fit}
\end{figure}


As this approach estimates $\Delta$ not based on the details of the $G(V)$ spectrum, but rather the energy scale of the (spectroscopic) gap, it is important to show that this is a robust feature. To this end figure \ref{fig:H20T_fit} shows the $G(V)$ data from J7 at $H=20\mathrm{T}$, as well as several theoretical traces. The first of which is the Abrikosov-Gor'kov fit, followed by two traces with the same gap, but different depairing values, which demonstrate that the energy scale of the gap is dominantly set by $\Delta$ while the depth of the gap at $V=0$ is set by the depairing. The last trace shows that the gap is significantly different than the one found at $H=0$, countrary to what one might naively infer based on the colorplots shown in figure 3 of the main text.




\begin{figure}[h]
	\centering
	\includegraphics[width=0.5\textwidth]{figures/H20T_fit_vs_Delta0_depairing.pdf}
	\caption{The experimental data from J7 at $H=20\mathrm{T}$ (green circles), the Abrikosov-Gor'kov fit (solid black) and two traces with the same $\Delta$ but different depairings and a trace with $\Delta=\Delta(H=0)$ with a depairing which fits the "depth" of the spectroscopic gap.
	}
	\label{fig:H20T_fit}
\end{figure}

Additionally the error of the $\Delta$ estimation can be performed in the following way: the log-likelihood distribution for the fitting parameters (given the data) is given by 
$$\tilde{p}(\Delta,\tilde{x}) = - \frac{ <(G_i - f(V_i,\Delta,\tilde{x}))^2>_i}{2\sigma^2}$$ where $\tilde{x}$ stands for the additional model-dependent parameters and $< ...>_i$ denotes the average over all of the acquired points. $\sigma$, the noise of the measurement, can be estimated either directly from  data, or  by the root-mean-square error (RMSE) of the fit, both of which give similar results. The log-likelihood is not necessarily quadratic in $\Delta$, as the fitting problem is nonlinear, but close to the maximum-likelihood point it can is approximately quadratic.
Therefore by fitting $\tilde{p}$ near it's maximal point with $-\frac{ (\Delta - \tilde{\Delta})^2}{2\delta_\Delta^2} + c$, where $c$ is related to the RMSE and $\tilde{\Delta}$ is the best fit value, we obtain an estimate of the fit uncertanty $\delta_\Delta$. The Abrikosov-Gor'kov $\Delta(H)$ curve with errors estimated in this way is shown in figure 4 of the main text. We find that the uncertainty of the extracted values of $\Delta$ is roughly equal to the spread of the data, regardless of the $G(V)$ model. The experimental $G(V)$ traces and the Abrikosov-Gor'kov fits are shown on figure 3 of the main text. The fits obtained using the other two models are almost indistinguishable from the Abrikosov-Gor'kov one. 

\subsection{DOS broadening}\label{partI:DOSbroadening}

Here we address the issue of in-gap states and the broadening of the coherence peaks. Indeed, the broadening parameters obtained from the fits are larger than expected from self-consistent Abrikosov-Gor'kov theory \cite{abrikosov1961zh} or a self-consistent Dynes model \cite{herman2016microscopic} (see figure \ref{fig:AG_and_Dynes_SC_vs_fit}). Moreover, at the critical field the broadening parameter is larger than theoretically allowed $2\alpha_c=\Delta(H=0)$. 

There are several possible reasons for this unexpected DOS broadening: 
\begin{enumerate}
\item At the K/K' points, the two spins have slightly different dispersions and densities of states --- this can lead to broadening of the peak~\cite{marganska}. While Ref.~\cite{marganska} assumes superconductivity arising from Coulomb interactions, the same broadening phenomenon will appear for all pairing mechanisms. This is likely the reason why in both our work and that of \cite{khestanova2018unusual}, reproduced here in figure \ref{fig:geim2L}, the effective temperature is much higher than the measurement temperature. This is the case even at (higher) temperatures where there should be good electron-phonon coupling, and where the electron temperature should thus be that of the refrigerator ($\sim$ 100 mK and above). 

Besides this, the following could also contribute to the broadening of the DOS.
\item In-gap states may appear in triplet superconductors because of field-induced gap nodes at the Fermi surface~\cite{fischer2018}. Such nodes would occur along the $\Gamma$-M directions when $E_Z$ (the Zeeman energy) exceeds the order parameter $\Delta_s$. This means that only the $\Gamma$ pockets would be affected and not the K/K' ones. Since electrons tunnel into the K/K' pockets (see Section IIC1), some K/K'-$\Gamma$ coupling would have to be present for such nodes to affect our measured density of states. 
\item Inelastic tunneling; energy loss due to defects in the barrier will broaden the $G(V)$ characteristic~\cite{hlobil,reed}. 
\item Coupling of the charge density waves via phonons either to the quasiparticles, or to amplitude fluctuations (Higgs mode) of the order parameters~\cite{littlewood1982}. 
\end{enumerate}

These various factors are difficult to disentangle; however, none of them significantly affects the energy gap, which can be determined with high precision from the fits, as was shown in this section.

\begin{figure}[h!]
	\centering
	\includegraphics[width=0.5\textwidth]{figures/geim_2L_Dynes.pdf}
	\caption{The bi-layer tunneling spectrum from \cite{khestanova2018unusual} (reproduced with the authors' permission), along with a BCS fit.}
	\label{fig:geim2L}
\end{figure}

\FloatBarrier

\subsection{Estimate of Inter-valley (K/K') and Total Scattering Times}\label{partI:scattering}

\textit{Intervalley Scattering Time:} Ref.~\cite{ilic2017} gives $H_{||}^c$ as a function of temperature for different $\tau_{iv}$, the scattering time between K and K' points, for an Ising superconductor. While the theory in Ref.~\cite{ilic2017} does not include a triplet order parameter or two-pocket coupling, this should not change the order of magnitude of estimates of $H_{||}^c$ for a given $\tau_{iv}$ and vice versa. (This can be seen in Figure 1c of the main text.) In our case $E_{SO}/(k_B T_c) \sim 20$. From Figure 5c in Ref.~\cite{ilic2017}, and using our experimental values for $T$, $T_c$ and $H_{||}^c$ we estimate $h/\tau_{iv} \approx k_B T_{cs}$. $h/\tau_{iv}$ is therefore on the order of $\Delta_s \approx 400 \mu$eV. 

\textit{Total Scattering Time:} In Ref.~\cite{barrera2018}, the mean free path $\ell$ for a bilayer NbSe$_2$ device was estimated to be 17 nm from Hall measurements. Thus, the total scattering time, including all intra- and inter-valley processes, is $\tau = \ell/v_F$. As $v_F \approx 5 \cdot 10^4$ m/s~\cite{dvir2018spectroscopy}, this gives $\tau \approx 340$ fs. Now, our bilayer device is more disordered than that of Ref.~\cite{barrera2018}: whereas $H_c^{||}$ in Ref.~\cite{barrera2018} is the same as in our case, their critical temperature is 5K, as opposed to 2.6K for us. Thus, $1/\tau \approx 10$ meV can be considered a lower limit for the total scattering time $1/\tau$ in our bilayer device. This is consistent with the estimate above of the inter-valley scattering rate, that is to say $1/\tau > 1/\tau_{iv}$. 

From the above two estimates, and from the fact that single-particle $\Gamma-K/K'$ scattering (McMillan coupling) does not seem to be important for our devices (cf. Section IID below), we conclude that intra-pocket (K-K, K'-K' or $\Gamma-\Gamma$) is the dominant disorder-induced scattering.


\section{Theory}\label{partII}

In this appendix, we  calculate the superconducting
energy gap of a monolayer superconductor without an inversion
center and in the presence of an in-plane magnetic field.  
In Sec. II.A, we introduce the appropriate Hamiltonian. In Sec. II. B, we study the density of states in the presence of singlet and triplet pairing in a simplified model restricted to the K/K’ pockets. In Secs. II.C \& II.D, we address the interplay between the K/K’ and $\Gamma$ pockets within two different models taking into account singlet pairing only : in Sec. II.C, we use the so-called Suhl-Matthias-Walker coupling describing inter-pocket pairing or Cooper pair tunneling, whereas in Sec. II.D, we use the so-called MacMillan coupling describing inter-pocket single-particle tunneling or scattering.

%We first present the appropriate Hamiltonian. Then, we obtain the quasi-classical Green functions for the case where there is only intra-valley scattering. We derive expressions for the density of states and the coupled self-consistency equations for the singlet and triplet order parameter in the limit of large spin-orbit coupling~\cite{Ilic-unpub}, which allow us to compute the energy gap as a function of magnetic field and fit the experimental data. The effect of inter-valley disorder is studied using a Landau expansion~\cite{haim2020signatures} to obtain the thermodynamic potential. Minimization of the thermodynamic potential yields the thermodynamic order parameters (OPs) which are used in the calculation of the energy gap. 

\subsection{The model Hamiltonian}\label{partII:hamiltonian}
\label{sec:model} 

The model Hamiltonian is a sum of the free part $H_0$ which defines our band structure model, the disorder Hamiltonian $H_{dis}$ and the interaction $H_{\mathrm{int}}$, 
\begin{equation}
H=H_{0}+ H_{dis} + H_{\mathrm{int}}. \label{eq:8.1}
\end{equation}
Our band structure model includes the Nb derived bands.
The crossing of the Nb derived band with the Fermi level gives rise to a hole-like pocket centered at $\Gamma$ as well as a a pair of hole-like pockets centered at $K(K')$ labeled by the valley index, $\eta=\pm 1$. 
In result, we have
\begin{equation}
H_{0}=\sum_{\eta}\underset{\mathbf{k},s}{\sum}\xi_{\mathbf{k}}^{K}a_{\eta\mathbf{k}s}^{\dagger}a_{\eta\mathbf{k}s}+\sum_{\eta}\underset{\mathbf{k},s,s'}{\sum}\left(\boldsymbol{\gamma}_{\eta}^{K}-\mathbf{B}\right)\cdot\boldsymbol{\sigma}_{ss'}a_{\eta\mathbf{k}s}^{\dagger}a_{\eta\mathbf{k}s'}+\underset{\mathbf{k},s}{\sum}\xi_{\mathbf{k}}^{\Gamma}b_{\mathbf{k}s}^{\dagger}b_{\mathbf{k}s}+\underset{\mathbf{k},s,s'}{\sum}\left(\boldsymbol{\gamma}_{\mathbf{k}}^{\Gamma}-\mathbf{B}\right)\cdot\boldsymbol{\sigma}_{ss'}b_{\mathbf{k}s}^{\dagger}b_{\mathbf{k}s'}.\label{eq:8.2}
\end{equation}
Here $a_{\eta\mathbf{k}s}^{\dagger}$ and $b_{\mathbf{k}s}^{\dagger}$
are the creation operators in the $K/K'$ and $\Gamma$ pockets, respectively,
$\xi_{\mathbf{k}}^{\Gamma\left(K\right)}$ is the spin-independent part of the band energy measured relative to Fermi energy,
$\boldsymbol{\sigma}=\left(\sigma_{x},\sigma_{y},\sigma_{z}\right)$
is the vector of Pauli matrices, $\boldsymbol{\gamma}_{\bar{\eta}}^{K}=-\boldsymbol{\gamma}_{\eta}^{K}$ and $\boldsymbol{\gamma}_{\bar{\mathbf{k}}}^{\Gamma}=-\boldsymbol{\gamma}_{\mathbf{k}}^{\Gamma}$ 
are the anti-symmetric SOC terms arising from the lack of inversion
symmetry and $\mathbf{B}=E_{Z}\hat{x}$ is the Zeeman field, i.e. the magnetic field which
absorbs the prefactor $g\mu_{\mathrm{B}}/2$ that includes the $g$-factor
and Bohr magneton. 
We use the notation $\bar{\eta}=-\eta,\bar{\mathbf{k}}=-\mathbf{k}$.
For the $K$-pocket and $\Gamma$-pocket we consider the Ising SOC of the form,
\begin{align}\label{eq:SOC_KG}
\boldsymbol{\gamma}_{\eta}^{K}=\eta E_{\mathrm{SO}}^{K}\hat{z},\ \ \ \boldsymbol{\gamma}_{\mathbf{k}}^{\Gamma}=E_{\mathrm{SO}}^{\Gamma}\cos\left(3\varphi_{\hat{\mathbf{k}}}\right)\hat{z},
\end{align}
where $\varphi_{\hat{\mathbf{k}}}$ is the angle formed by the momentum unit vector ${\hat{\mathbf{k}}}$ of an electron with the $k_x$ direction. 
Within the $\Gamma$ pocket, the SOC $\boldsymbol{\gamma}_{\mathbf{k}}^{\Gamma}$, Eq.~\eqref{eq:SOC_KG}, changes sign six times at the $\Gamma M$ lines.
In the $K$ and $K'$ pockets, it is constant and antiparallel.
%In our numerical calculations we take $E_{\mathrm{SO}}^{K}/E_{\mathrm{SO}}^{\Gamma}=2.14$ based on  Ref.~\cite{Wickramaratne2020} assuming that this ratio is roughly similar for mono- and bilayer systems.
%Even though the calculation in Ref.~\cite{Wickramaratne2020} is for monolayer we assume it is approximately valid for the two-layered system in the experiment.

%$\gamma^{K}\left(\mathbf{k}\right)=E_{\mathrm{SO}}^{K}\mathrm{sgn}\left[\cos\left(3\varphi_{\mathbf{k}}\right)\right]$
%and for the $\Gamma$ pocket we take $\gamma^{\Gamma}\left(\mathbf{k}\right)=E_{\mathrm{SO}}^{\Gamma}\cos\left(3\varphi_{\mathbf{k}}\right)$.


The presence of randomly
distributed scalar impurities gives rise to a scattering potential,
\begin{align}
H_{dis}(\mathbf{r}) = \sum_{l} U_0(\mathbf{r} - \mathbf{R}_l)    \, ,   
\end{align}
where $\mathbf{R}_l$ is the location of the $l$th impurity scattering center. 
Singlet pairing as well as inter-valley triplet pairing in K/K' pockets is not sensitive to intra-valley scattering. 
To study the effect of inter-valley scattering, we consider a short-range impurity potential such that $U_{0}\left(\mathbf{k}-\mathbf{k}'\right)=U_{0}$. The effect of the impurity potential is described within the self-consistent Born approximation by the appropriate self-energy $\hat{\Sigma}$, which we do not write here explicitly. $\hat{\Sigma}$ is proportional to the scattering rate $1/\tau_{iv}=\pi n_{imp}N_{0}U_{0}^{2}$, where $n_{imp}$ is the impurity density and  $N_{0}$ is the normal state density of states per spin species. Superconductivity within the the $\Gamma$ pocket is sensitive to intra-pocket scattering at finite magnetic fields. We characterize it by a  scattering rate $1/\tau_\Gamma$. As it involves a small momentum transfer, both short-and long-range impurities contribute. Thus we expect $1/\tau_\Gamma\gg1/\tau_{iv}$. In the following, we set $k_{\mathrm{B}}=1$.




The interaction Hamiltonian, $H_{\mathrm{int}}$ in Eq.~\eqref{eq:8.1}, contains the superconducting pairing interactions
\begin{align}
\label{eq:Hint}
H_{\mathrm{int}} & =\frac{1}{2}\sum_{\eta,\eta'}\underset{s_{i},s_{i}'}{\sum}\underset{\mathbf{k},\mathbf{k}'}{\sum}V_{s_{1}s_{2},s_{1}'s_{2}'}^{KK}\left(\mathbf{k},\eta;\mathbf{k}',\eta'\right)a_{\eta\mathbf{k}s_{1}}^{\dagger}a_{\bar{\eta}\bar{\mathbf{k}}s_{2}}^{\dagger}a_{\bar{\eta}'\bar{\mathbf{k}}'s_{2}'}a_{\eta'\mathbf{k}'s_{1}'}+\frac{1}{2}\underset{s_{i},s_{i}'}{\sum}\underset{\mathbf{k},\mathbf{k}'}{\sum}V_{s_{1}s_{2},s_{1}'s_{2}'}^{\Gamma\Gamma}\left(\mathbf{k},\mathbf{k}'\right)b_{\mathbf{k}s_{1}}^{\dagger}b_{\bar{\mathbf{k}}s_{2}}^{\dagger}b_{\bar{\mathbf{k}}'s_{2}'}b_{\mathbf{k}'s_{1}'}\\
 & +\frac{1}{2}\sum_{\eta}\underset{s_{i},s_{i}'}{\sum}\underset{\mathbf{k},\mathbf{k}'}{\sum}V_{s_{1}s_{2},s_{1}'s_{2}'}^{K\Gamma}\left(\mathbf{k},\eta;\mathbf{k}'\right)a_{\eta\mathbf{k}s_{1}}^{\dagger}a_{\bar{\eta}\bar{\mathbf{k}}s_{2}}^{\dagger}b_{\bar{\mathbf{k}}'s_{2}'}b_{\mathbf{k}'s_{1}'}+\frac{1}{2}\sum_{\eta'}\underset{s_{i},s_{i}'}{\sum}\underset{\mathbf{k},\mathbf{k}'}{\sum}V_{s_{1}s_{2},s_{1}'s_{2}'}^{\Gamma K}\left(\mathbf{k};\mathbf{k}',\eta'\right)b_{\mathbf{k}s_{1}}^{\dagger}b_{\bar{\mathbf{k}}s_{2}}^{\dagger}a_{\bar{\eta}'\bar{\mathbf{k}}'s_{2}'}a_{\eta'\mathbf{k}'s_{1}'}.\nonumber 
\end{align}
where $V^{KK}$ and $V^{\Gamma\Gamma}$ are the intra-pocket
pairing interactions in the $K(K')$ and $\Gamma$ pockets, respectively, whereas $V^{K\Gamma}$ $V^{\Gamma K}$ are the inter-pocket pairing interactions.


We introduce the two order parameters (OPs) in the standard way, 
%Performing mean field decoupling of the interaction Hamiltonian $H_{\mathrm{int}}$
%lead to the definition of two OPs
\begin{align}
\label{eq:OP}
\underline{\Delta}_{s_{1}s_{2}}^{K,\eta}\left(\mathbf{k}\right)=\frac{1}{V}\underset{\mathbf{k}',s_{1}',s_{2}'}{\sum}\left[\sum_{\eta'}V_{s_{1}s_{2},s_{1}'s_{2}'}^{KK}\left(\mathbf{k},\eta;\mathbf{k}',\eta'\right)\left\langle a_{\bar{\eta}'\bar{\mathbf{k}}'s_{2}'}a_{\eta'\mathbf{k}'s_{1}'}\right\rangle +V_{s_{1}s_{2},s_{1}'s_{2}'}^{K\Gamma}\left(\mathbf{k},\eta;\mathbf{k}'\right)\left\langle b_{\bar{\mathbf{k}}'s_{2}'}b_{\mathbf{k}'s_{1}'}\right\rangle \right],
\notag \\
\underline{\Delta}_{s_{1}s_{2}}^{\Gamma}\left(\mathbf{k}\right)=\frac{1}{V}\underset{\mathbf{k}',s_{1}',s_{2}'}{\sum}\left[V_{s_{1}s_{2},s_{1}'s_{2}'}^{\Gamma\Gamma}\left(\mathbf{k};\mathbf{k}'\right)\left\langle b_{\bar{\mathbf{k}}'s_{2}'}b_{\mathbf{k}'s_{1}'}\right\rangle +\sum_{\eta'}V_{s_{1}s_{2},s_{1}'s_{2}'}^{\Gamma K}\left(\mathbf{k};\mathbf{k}',\eta'\right)\left\langle a_{\bar{\eta}',\bar{\mathbf{k}}',s_{2}'}a_{\eta',\mathbf{k}',s_{1}'}\right\rangle \right], 
\end{align}
%%%%
where $V$ is the volume of the system, and $\langle \ldots \rangle$ stands for the thermodynamic and quantum mechanical averaging.
They can be represented in the standard matrix form,
%
\begin{equation}
\underline{\Delta}^{\Gamma(K,\eta)}\left(\mathbf{k}\right)
=\left[\psi^{\Gamma(K,\eta)}\left(\mathbf{k}\right)+\mathbf{d}^{\Gamma(K,\eta)}\left(\mathbf{k}\right)\cdot\boldsymbol{\sigma}\right]i\sigma_{2}.\label{eq:1.8}
\end{equation}
Here $^{\Gamma(K,\eta)}\left(\mathbf{k}\right)$ and $\mathbf{d}^{\Gamma(K,\eta)}\left(\mathbf{k}\right)$
parametrize the singlet and triplet components of the OP, respectively.
For simplicity, we assume the singlet OPs to be isotropic,
$\psi^{\Gamma\left(K,\eta\right)}\left(\mathbf{k}\right)=\Delta_{s}^{\Gamma\left(K\right)}$. 


Interactions in Eq.~\eqref{eq:Hint} may have components in both singlet and triplet channels, 
\begin{align}
    V^{\beta\beta'}= V_{s}^{\beta\beta'} + V_{t}^{\beta\beta'}
\end{align}
where $\beta,\beta'$ refer to the pocket index. For the singlet pairing we take 
\begin{align}\label{eq:Vsinglet}
V_{s}^{\beta\beta'}=v^{\beta\beta'}\left[i\sigma_{y}\right]_{s_{1}s_{2}}\left[i\sigma_{y}\right]_{s_{1}'s_{2}'}^{*}.
\end{align}
The hermiticity condition yields $\left[v^{\Gamma \Gamma}\right]^* = v^{\Gamma \Gamma}$, $\left[v^{K K}\right]^* = v^{K K} $, and $\left[v^{\Gamma K}\right]^* = v^{K \Gamma}$.
Furthermore, we take inter-pocket couplings real which makes them equal,
$v^{\Gamma K} = v^{K \Gamma}$, and consider attractive interactions, $v^{\beta\beta'}< 0$ for definitness.
%here we consider only the attractive interactions, $v^{\beta\beta'}< 0$.
%%%%

In this SI, the triplet OP will be considered in the $K/K'$ pockets only, $\mathbf{d}^{K} = \mathbf{d}_\eta$,
\begin{equation}
\mathbf{d}_\eta=\hat{\gamma}_{\eta}\left(\eta_{E1}\hat{x}+\eta_{E2}\hat{y}+\eta_{A}\hat{z}\right)\label{eq:1.9}
\end{equation}
where $\eta_{A}$ and $\eta_{E1(2)}$ are triplet OPs transforming trivially and non-trivially under $D_{3h}$.
Eq.~\eqref{eq:1.9} leads us to the effective interaction,
\begin{align}\label{eq:1.10}
V^{KK}_t(\eta,\eta') & =
\underset{j=1,2}{\sum}v_{t}\left[\hat{\gamma}_{\eta}\sigma_{j}i\sigma_{2}\right]_{s_{1}s_{2}}\left[\hat{\gamma}_{\eta'}\sigma_{j}i\sigma_{2}\right]_{s_{1}'s_{2}'}^{*}
+v_{tz}\left[\hat{\gamma}_{\eta}\sigma_{3}i\sigma_{2}\right]_{s_{1}s_{2}}\left[\hat{\gamma}_{\eta'}\sigma_{3}i\sigma_{2}\right]_{s_{1}'s_{2}'}^{*}.
\end{align}
 
 
As shown in Ref.  \cite{Mockli2019}, the magnetic field couples the singlet order parameter $\Delta_s$ and the equal spin triplet order parameter $\eta_{E2}$.
For the purpose of fitting the data, we assume that singlet pairing is dominant and that the temperature is larger than the critical temperature of all the possible triplet pairings. In that case, we can set $\eta_A=\eta_{E1}=0$ and keep only the singlet and the $\eta_{E2}$-triplet order parameters.

 We define the transition temperature $T_{cs}$ ($T_{ct}$)
by setting $E_Z=E_{\mathrm{SO}}=\Gamma=0$ and keeping only the
$\Delta_s$($\eta_{E2}$) OP in Eq. \eqref{eq:1.8}.
The relation between $T_{cs}$ and $v_{s}$ is $T_{cs}=2\Lambda e^{\gamma_{E}}\pi^{-1}\exp\left[-1/2N\left|v_{s}\right|\right]$,
where $\Lambda$ is a cutoff for the high energy attraction, $N$ is the density of states per spin summed over all superconducting pockets, and $\gamma_{E}$
is Euler's constant. Similarly, for $T_{ct}$, we have $T_{ct}=2\Lambda e^{\gamma_{E}}\pi^{-1}\exp\left[-1/2N\left|v_{t}\right|\right]$.
For the analysis, we use the transition temperatures rather than the interaction
amplitudes as parameters.


The Bogoliubov-de Gennes (BdG) Hamiltonian for electrons in the $K/K'$ and $\Gamma$ pockets reads
\begin{equation}
\label{eq:BdG}
\hat{H}_{\mathrm{BdG}}^{\Gamma\left(K,\eta\right)}=\left[\begin{array}{cc}
\xi_{\mathbf{k}}+\left[\boldsymbol{\gamma}_{\mathbf{k}\left(\eta\right)}^{\Gamma\left(K\right)}-\mathbf{B}\right]\cdot\boldsymbol{\sigma} & \underline{\Delta}^{K\left(\Gamma\right)}\\
\underline{\Delta}^{\Gamma\left(K\right)\dagger} & -\xi_{\mathbf{k}}+\left[\boldsymbol{\gamma}_{\mathbf{k}\left(\eta\right)}^{\Gamma\left(K\right)}+\mathbf{B}\right]\cdot\boldsymbol{\sigma}^{\mathrm{T}}
\end{array}\right]\, .
\end{equation}
Now, we will consider different scenarios: In subsection \ref{sec-KK'}, we will consider both singlet and triplet pairing in a simplified model neglecting the $\Gamma$ pocket.
In subsections \ref{sec:two-pockets-MSW} and \ref{sec-mm}, we consider the effect of the $\Gamma$ pocket, but neglecting triplet pairing.

\subsection{Simplified model without the $\Gamma$ pocket}
\label{sec-KK'}

Focusing on the $K$ pocket with triplet component of the OP, the dispersion relation is determined by the solution of the equation
\begin{equation}
\mathrm{det}\left[E\hat{\sigma}_{0}-\hat{H}_{\mathrm{BdG}}^{K,\eta}\right]=0\label{eq:1.20}
\end{equation}
 for $E$, where $\hat{\sigma}_{0}$ is the $4\times4$ unit matrix. We choose
the phase of the singlet OP $\Delta_s$ to be $0$. The coupling between the singlet and triplet order parameters imposes their relative phases such that $\mathbf{d}_\eta=i\,{\rm sign}(E_Z)\hat{\gamma}_\eta\Delta_{tB}\hat{y}$ with $\Delta_{tB}$ real and positive. The physically
relevant solution for the energy is
\begin{align}
E\left(\xi_{\mathbf{k}}\right)  =&\biggr(\xi_{\mathbf{k}}^{2}+E_{\mathrm{SO}}^{2}+E_Z^{2}+\Delta_s^2+\Delta_{tB}^{2}\label{eq:1.21}\\
 & -2\sqrt{\xi_{\mathbf{k}}^{2}\left(E_{\mathrm{SO}}^2+E_Z^{2}\right)+\left(|E_Z|\Delta_s-E_{\mathrm{SO}}\Delta_{tB}\right)^{2}}\biggr)^{1/2}.\nonumber 
\end{align}
The dispersion $E\left(\xi_{\mathbf{k}}\right)$ has a minimum at
$\xi_{\mathbf{k}}=\sqrt{\rho^2-P^{2}/\rho^{2}}$, where we introduced the notation $\rho=\sqrt{E_{\mathrm{SO}}^{2}+E_Z^2}$ and  $P=|E_Z|\Delta_s-E_{\mathrm{SO}}\Delta_{tB}$, which gives 
the superconducting energy gap
\begin{eqnarray}
\Delta&=&\frac1{\rho}\left(E_{\mathrm{SO}}\Delta_s+|E_Z|\Delta_{tB}\right).\label{eq:1.22}
\end{eqnarray}

\subsubsection{Without inter-valley scattering: Quasiclassical Green functions}

The coupled order parameters and  density of states can be computed using quasiclassical Green functions. In the absence of inter-valley scattering, we obtain~\cite{Ilic-unpub}
\begin{eqnarray}
\nu(E)
&=&2N_0\Re\left[\frac{\omega_n{\rm\, sign}(\Sigma)}{\sqrt2\left[\Sigma-2\rho^2 +{\rm\, sign}(\Sigma)\sqrt{\Sigma^2-4P^2}\right]^{1/2}}\left(1+\frac{|\Sigma|}{\sqrt{\Sigma^2-4P^2}}\right)\right]_{i\omega_n\to E+i\delta},\label{eq-dosf}\\
\Delta_s
&=&2\pi T|v_s|\sum_{\omega_n>0} \frac1{\sqrt2\left[\Sigma-2\rho^2 +\sqrt{\Sigma^2-4P^2}\right]^{1/2}}\left[\Delta_s\left(1+\frac{\Sigma-2E_Z^2}{\sqrt{\Sigma^2-4P^2}}\right)+\Delta_{tB}\frac{2|E_Z|E_{\mathrm{SO}}}{\sqrt{\Sigma^2-4P^2}}\right],\label{eq-s}\\
\Delta_{tB}
&=&2\pi T|v_t|\sum_{\omega_n>0} \frac1{\sqrt2\left[\Sigma-2\rho^2 +\sqrt{\Sigma^2-4P^2}\right]^{1/2}}\left[\Delta_{tB}\left(1+\frac{\Sigma-2E_{\mathrm{SO}}^2}{\sqrt{\Sigma^2-4P^2}}\right)+\Delta_s\frac{2|E_Z|E_{\mathrm{SO}}}{\sqrt{\Sigma^2-4P^2}}\right],\label{eq-t}
\end{eqnarray}
where $\omega_{n}=\pi T\left(2n+1\right)$ are fermionic Matsubara frequencies and we introduced the notation $\Sigma=\omega_n^2+\rho^2+\Delta_s^2+\Delta_{tB}^2$.

The density of states Eq.~\eqref{eq-dosf} displays a superconducting energy gap $\Delta$. Furthermore, there is  a partial \lq\lq mirage\rq\rq\ gap~\cite{Belzig}  centered around $E=\pm\sqrt{\rho^2+\Delta_s^2+\Delta_{tB}^2}$.

In general, the coupled self-consistency equations can be solved numerically. The fit shown in the main text was obtained that way. However, simplifications are possible in the limit $E_{\mathrm{SO}}\gg\Delta_0$, where $\Delta_0$ is the zero-temperature, zero-field singlet order parameter. In that case, Eqs.~\eqref{eq-s} and \eqref{eq-t} may be combined into one equation for the gap $\Delta$,
\begin{eqnarray}
\!\!\frac{\left[2\pi T|v_s|\sum_{\omega_n>0} \frac1{\sqrt{\omega_n^2+\Delta^2}}-1\right]\left[2\pi T|v_t|\sum_{\omega_n>0} \frac1{\sqrt{\omega_n^2+\Delta^2}}-1\right]}{|v_t|E_{\mathrm{SO}}^2\left[2\pi T|v_s|\sum_{\omega_n>0} \frac1{\sqrt{\omega_n^2+\Delta^2}}-1\right]+|v_s|E_Z^2\left[2\pi T|v_t|\sum_{\omega_n>0} \frac1{\sqrt{\omega_n^2+\Delta^2}}-1\right]}&=&2\pi T\sum_{\omega_n>0}\frac1{\sqrt{\omega_n^2+\Delta^2}}\frac1{\omega_n^2+\rho^2}.\;\;\label{eq-gap}
\end{eqnarray}
The density of states  at $|E|\ll E_{\mathrm{SO}}$ acquires a BCS form,
\begin{equation}
\nu(E)=\nu_0\frac{|E|}{\sqrt{E^2-\Delta^2}}\theta(|E|-\Delta).
\end{equation}


\subsubsection{With inter-valley scattering: Landau expansion}


To study the effect of disorder on a qualitative level, we employ a Landau expansion valid close to the critical temperature. 
Considering the model Hamiltonian above and using quasiclassical methods
the difference of the thermodynamic potential in superconducting and
normal state $\Omega$ may be written in the form of a Landau expansion
as
\begin{equation}
\left(V^{2}N_{0}\right)^{-1}\Omega\left(\Delta_s,\Delta_{tB}\right)=\Omega^{\left(2\right)}+\Omega^{\left(4\right)}.\label{eq:1.11}
\end{equation}
Here $\Omega^{\left(2\right)}$ contains the terms quadratic in the
OPs, and $\Omega^{\left(4\right)}$ contains the quartic terms. For $\Omega^{\left(2\right)}$, we have
\begin{align}
\Omega^{\left(2\right)} & =A_{1}\Delta_s^{2}+A_{2}\Delta_{tB}^{2}+2A_{3}\Delta_s\Delta_{tB}. \label{eq:1.12}
\end{align}

Denoting $\tilde{\omega}_{n}=\omega_{n}+\mathrm{sgn}\left(\omega_{n}\right)1/\tau_{iv}$,
the coefficients are given as~\cite{Mockli2020}, 
\begin{equation}
A_{1}=2\pi T\underset{\omega_{n}>0}{\sum}\frac{\tilde{\omega}_{n}E_Z^{2}}{\omega_{n}\left[\tilde{\omega}_{n}\left(E_Z^{2}+\omega_{n}^{2}\right)+\omega_{n}E_{\mathrm{SO}}^{2}\right]}+\ln\frac{T}{T_{cs}},\label{eq:1.13}
\end{equation}
\begin{equation}
A_{2}=2\pi T\underset{\omega_{n}>0}{\sum}\frac{\tau_{iv}^{-1}\left(E_Z^{2}+\omega_{n}^{2}\right)+\omega_{n}E_{\mathrm{SO}}^{2}}{\omega_{n}\left[\tilde{\omega}_{n}\left(E_Z^{2}+\omega_{n}^{2}\right)+\omega_{n}E_{\mathrm{SO}}^{2}\right]}+\ln\frac{T}{T_{ct}},\label{eq:1.14}
\end{equation}
\begin{equation}
A_{3}=2\pi T\underset{\omega_{n}>0}{\sum}\frac{(-E_Z) E_{\mathrm{SO}}}{\tilde{\omega}_{n}\left(E_Z^{2}+\omega_{n}^{2}\right)+\omega_{n}E_{\mathrm{SO}}^{2}}.\label{eq:1.16}
\end{equation}


For succinctness, we do not provide here the full expression for $\Omega^{\left(4\right)}$
and only write the term corresponding to the singlet OP 
\begin{equation}
\Omega_{s}^{\left(4\right)}=-\pi T_{cs}\underset{\omega_{n}'>0}{\sum}D_{1}\left(\omega_{n}'\right)\Delta_s^4,\label{eq:1.17}
\end{equation}
where
\begin{align}
D_{1}\left(\omega\right) & =\frac{1}{2\left[\tilde{\omega}\left(E_Z^{2}+\omega^{2}\right)+\omega E_{\mathrm{SO}}^{2}\right]^{4}}\biggr[-\omega\left(\omega\tilde{\omega}+E_{\mathrm{SO}}^{2}\right)^{4}\label{eq:1.18}\\
 & +2E_Z^{2}\omega\left(\tilde{\omega}^{2}-E_{\mathrm{SO}}^{2}\right)\left(\omega\tilde{\omega}+E_{\mathrm{SO}}^{2}\right)^{2}\nonumber \\
 & +E_Z^{4}\left(3\omega\tilde{\omega}^{4}+2E_{\mathrm{SO}}^{2}\tilde{\omega}^{2}\left(\tau_{iv}^{-1}+\tilde{\omega}\right)-\omega E_{\mathrm{SO}}^{4}\right)\biggr].\nonumber 
\end{align}
and where $\omega_{n}'=\pi T_{cs}\left(2n+1\right)$. In the limit $E_{\mathrm{SO}}=0$ and no triplet OP, $\Delta_{tB}=0$, derivation of the thermodynamic potential \eqref{eq:1.11} by the singlet OP reproduces the self consistency equation found in Ref.~\cite{Maki1964}. Equipped
with the thermodynamic potential, the
OPs are found by the process of minimization. 

\begin{figure}
\includegraphics[scale=0.51]{FigureS1.pdf}
\caption{\label{Fig4} Effect of the magnitude of the spin-orbit coupling: The singlet $\Delta_s$ (solid lines) and triplet $\Delta_{tB}$
(dashed lines) OPs as a function of the field for different values
of $E_{\mathrm{SO}}$ with parameters $T=0.75T_{cs},T_{ct}=0.7T_{cs},\tau_{iv}^{-1}=0.001T_{cs}$.
The inset shows the superconducting gap $\Delta$ for different values
of $E_{\mathrm{SO}}$ and with the same parameters as the main
graph, the solid lines are with the triplet component and the dashed
lines are for a singlet-only superconductor ($T_{ct}=0$). $E_Z,E_{\mathrm{SO}}$
are in units of $T_{cs}$. $\Delta,\Delta_s,\Delta_{tB}$
are normalized to the value of $\Delta=\Delta_s$ at $E_Z=0$.}
\end{figure}

\begin{figure}
\includegraphics[scale=0.51]{FigureS2.pdf}
\caption{\label{Fig5} Effect of the triplet pairing: The singlet $\Delta_s$ (solid lines) and triplet $\Delta_{tB}$
(dashed lines) OPs as a function of the field for different values
of $T_{ct}$ with parameters $T=0.7T_{cs},E_{\mathrm{SO}}=8.0T_{cs},\tau_{iv}^{-1}=0$.
The inset shows the superconducting gap $\Delta$ for different values
of $T_{ct}$ and with the same parameters as the main graph.
The black dashed line is the superconducting gap for a singlet-only superconductor ($T_{ct}=0$). $E_Z,T_{ct}$ are in units of $T_{cs}$.
$\Delta,\Delta_s,\Delta_{tB}$ are normalized
to the value of $\Delta=\Delta_s$ at $E_Z=0$.}
\end{figure}


Using the Landau expansion, we can obtain a qualitative understanding of the way the different
parameters affect  $\Delta_s$, $\Delta_{tB}$ and $\Delta$ as a function of the field. We start with the case of negligible inter-valley scattering. In Fig.~\ref{Fig4}, we see that for large enough $E_{\mathrm{SO}}$
the effect of increasing $E_{\mathrm{SO}}$ is only to stretch the lines for larger
critical fields $E_{Zc}$ but otherwise keeping the shape of lines as
they are. In Fig.~\ref{Fig5} ,we see that the effect of increasing $T_{ct}$
is to obtain larger critical fields $E_{Zc}$ by increasing the triplet component
in the superconducting phase, specifically we see that for larger $T_{ct}$we
get a steeper rise of the triplet component at low fields.


\begin{figure}
\includegraphics[scale=0.51]{FigureS3.pdf}
\caption{\label{Fig6} Effect of disorder:  The singlet $\Delta_s$ (solid lines) and triplet $\Delta_{tB}$
(dashed lines) OPs as a function of the field for different values
of $\tau_{iv}^{-1}$ with parameters $T=0.7T_{cs},E_{\mathrm{SO}}=8.0T_{cs},T_{ct}=0.69T_{cs}$.
The inset shows the superconducting gap $\Delta$ for different values
of $\tau_{iv}^{-1}$ and with the same parameters as the main graph. The red {\tt use black as in Fig. S2 ?}
dashed line is the superconducting gap for a singlet-only superconductor ($T_{ct}=0$), and with $\tau_{iv}^{-1}=2.09T_{cs}$. $E_Z,\tau_{iv}^{-1}$ are in
units of $T_{cs}$. $\Delta,\Delta_s,\Delta_{tB}$
are normalized to the value of $\Delta=\Delta_s$ at $E_Z=0$.}
\end{figure}

We now turn to the effect of disorder. The impurity scattering potential has a broadening effect on the peak of the density of states but does not affect the form of the effective order parameter $\Delta$ appearing in the density of states significantly (though the superconducting energy gap may differ), hence we use \eqref{eq:1.22} as an estimation also in the presence of weak inter-valley scattering.
In the presence of the in-plane magnetic field, the scattering off the scalar impurities causes a spin flip with finite probability, and makes the scalar impurity to behave effectively as a magnetic impurity with a field-dependent concentration. While the problem is captured by the Abrikosov-Gor'kov theory of magnetic impurities~\cite{AG} in some parameter regimes, the general form of the self-consistency equation differs from the standard situation because the spin splitting $E_{\mathrm{SO}}$ intervenes as an additional energy scale.

The parameters corresponding to the lines with the highest critical fields in Figs.~\ref{Fig5} and \ref{Fig6} are identical. The $\tau_{iv}^{-1}$ parameters in Fig.~\ref{Fig6} were chosen so that identical colors in Figs.~\ref{Fig5} and \ref{Fig6} will have approximately the same critical field.   
In Fig.~\ref{Fig6}
we see that by increasing $\tau_{iv}^{-1}$ we negate the affect of
having $T_{ct}>0$. Comparison of the OPs suppression obtained in  Fig.~\ref{Fig5} by decreasing $T_{ct}$ to the suppression obtained in Fig.~\ref{Fig6} by increasing $\tau_{iv}^{-1}$ shows that in the latter process we can retain a relatively steep increase of the triplet OP even for small critical fields, while in Fig.~\ref{Fig5} the decrease in the critical field is accompanied by a faster decrease in slope of the triplet. The reason for this is that, even though increasing $\tau_{iv}^{-1}$ in Fig.~\ref{Fig6} suppresses superconductivity, we still keep a high $T_{ct}$, which strengthens the triplet component, while in  Fig.~\ref{Fig5} the suppression of superconductivity is achieved by direct suppression of the triplet component. The gap contains contributions of both the triplet and singlet order parameters. As the triplet order parameter is affected more strongly by a suppression of $T_{ct}$ than by increase in $\tau_{iv}^{-1}$ the same is true for the gap.
Compared to the triplet order parameter taken separately, the distinction between $\Delta(H)$ in the two cases is less pronounced as long as the singlet order parameter makes a dominant contribution to the gap.



\subsection{Two-pocket superconductivity in NbSe$_2$ - Suhl-Matthias-Walker coupling}\label{sec:two-pockets-MSW}



In few-layer $\mathrm{NbSe_2}$ the $\mathrm{Se}$ derived Fermi pockets, observed in the bulk~\cite{noat2015quasiparticle}, disappear~\cite{Wickramaratne2020}, and only the $\mathrm{Nb}$ derived pockets close to the $K/K'$ and $\Gamma$ points are left. Here we discuss the multi-pocket effects assuming a Suhl-Matthias-Walker\cite{Suhl1959} type coupling between pockets.
This coupling describes the the inter-pocket  Cooper pair tunneling process.


%%%%%%%%%%%%%%%%%%
\subsubsection{OPs in the two-pocket model at zero field, \texorpdfstring{$E_Z =0$}{TEXT}}
%at \texorpdfstring{$E_Z=0$}{TEXT}}
\label{sec:OPs_ZeroB}

The purpose of this section is to elucidate the limitations on the parameters of the two-pocket model imposed by the zero field data (cf. inset of Figure~\ref{fig:GvsT}), where the BCS ratio of the superconducting gap to the critical temperature is found. This implies that, in the case where there are two superconducting gaps, the larger one is measured in the experiment. In addition, in the experiment, quasiparticles tunnel into NbSe$_2$ mainly at the K/K' points, as this is where the energy gap is the smallest in the tunnel barrier material (few-layer WSe$_2$ or MoS$_2$)~\cite{zhu,wang2012}. In the previous work of some of us on bulk NbSe$_2$ with similar barriers, we found a negligible contribution from tunneling into the $\Gamma$ point~\cite{Dvir2017}. Therefore, we consider only tunneling into the K/K' points. 

Specifically, we find that the BCS ratio of the superconducting gap to the critical temperature is achieved either when the $\Gamma$ and $K/K'$ pockets are decoupled or when the interaction amplitudes satisfy the following condition, 
\begin{align}\label{EQ:CONDITION}
    v^{\Gamma\Gamma}+ 2v^{K\Gamma}=2v^{KK} + v^{K\Gamma}\, .
\end{align}
The decoupled pockets are analyzed above, and here we consider the coupled pockets under the condition, \eqref{EQ:CONDITION}.
This relation follows as the number of $K$ pockets is twice as large than the number of $\Gamma$ pockets.
The relation \eqref{EQ:CONDITION} is confirmed numerically in the Fig.~\ref{fig:res1}.
where the ratio of the gap $\Delta_s^K$ to the zero field critical temperature,
\begin{equation}
\label{eq:Tc0_v}
T_{c0}\left(v^{\beta\beta'}\right)=\frac{2\omega_{D}e^{\gamma_{E}}}{\pi}\exp\left(-\frac{v^{KK}+v^{\Gamma\Gamma}/2+\sqrt{2\left(v^{K\Gamma}\right)^{2}+\left(v^{KK}-v^{\Gamma\Gamma}/2\right)^{2}}}{4N_{0}\left[\left(v^{K\Gamma}\right)^{2}-v^{\Gamma\Gamma}v^{KK}\right]}\right).
\end{equation}
is shown as a function of $v^{\Gamma\Gamma}/v^{KK}$ and $v^{K\Gamma}/vv^{KK}$. As the experimentally observed ratio is close to the BCS ratio, we restrict ourselves in the following to this line, which includes, in particular, the case where all interactions are equal.





\begin{figure*}%[htp]
\centering
\includegraphics[width=0.7\textwidth]{pockets/FigPockets1.pdf}
\caption{\label{fig:res1}The OPs at $E_{Z}=0$ for different values of $v^{\Gamma\Gamma}/v^{KK}$
and $v^{K\Gamma}/v^{KK}$, as calculated by numerically
solving the two self-consistency equations. 
The plots show the larger OP, which is $\Delta_{s}^{K}$ in the region of panel $\mathbf{\left(a\right)}$ and $\Delta_{s}^{\Gamma}$ in the region of panel $\mathbf{\left(b\right)}$. We take $N_{0}v^{KK}=-0.5$, and $T/T_{c0}=0.5$ where $T_{c0}$ is given by Eq.~\eqref{eq:Tc0_v} and
the OPs is in units of $T_{c0}$.
}
\end{figure*}

\subsubsection{The case of equal inter- and intra-pocket interactions} \label{sec:equal_amp}

We now consider the case $v^{\Gamma\Gamma}=v^{KK}=v^{K\Gamma}$. 
In this section we show that this case is equivalent to a single pocket. 
It can be shown from the self consistency equations that in this case the OPs are equal $\Delta_{s}^{K}=\Delta_{s}^{\Gamma}=\Delta_{s}$,
and the two self-consistency equations reduce to one:
\begin{equation}
\label{eq:self_lambda_equal_nu}
\pi T\underset{\omega_{n}}{\sum}\left[\frac{\Delta_{s}}{\left|\omega_{n}\right|}-\left(\frac{2}{3}\left\langle f_{0}^{K}\left(\hat{\mathbf{k}},\omega_{n},\Delta_{s}\right)\right\rangle _{\varphi_{\hat{\mathbf{k}}}}+\frac{1}{3}\left\langle f_{0}^{\Gamma}\left(\hat{\mathbf{k}},\omega_{n},\Delta_{s}\right)\right\rangle _{\varphi_{\hat{\mathbf{k}}}}\right)\right]+\Delta_{s}\ln\left(\frac{T}{T_{c0}}\right) =0\, ,
\end{equation}
where $f_{0}^{\Gamma(K)}\left(\hat{\mathbf{k}},\omega_{n},\Delta_{s}\right)$ are quasi-classical Green functions, \cite{haim2020signatures}. 
That is, we effectively have a single pocket with field
dependence which averages over the pair breaking effect at the $K/K'$ and at $\Gamma$ pockets.


The physical picture contained in Eq.~\eqref{eq:self_lambda_equal_nu} is that of a single band, albeit anisotropic.
For this reason, it has been possible to parametrize Eq.~\eqref{eq:self_lambda_equal_nu} by the critical temperature, $T_{c0}$ instead of the pairing amplitude $\lambda$.
In general, this is not possible even close to $T_{c0}$ \cite{Kogan2011}.
Furthermore, the pair breaking equation for the critical field reads,  
\begin{equation}
\label{eq:pair_breaking_lambda_equal_nu}
\ln\left(\frac{T}{T_{c0}}\right)+\left(\frac{2}{3}\mathcal{S}_{K}+\frac{1}{3}\mathcal{S}_{\Gamma}\right)=0\, , 
\end{equation}
where
\begin{equation}
\label{eq:SKG}
\mathcal{S}_{K}=\pi T\underset{\omega_{n}}{\sum}\frac{1}{\left|\omega_{n}\right|}\left(\frac{E_{Z}^{2}}{E_{Z}^{2}+\left(E_{\mathrm{SO}}^{K}\right)^{2}+\omega_{n}^{2}}\right)\, ,\quad 
\mathcal{S}_{\Gamma}=\pi T\underset{\omega_{n}}{\sum}\frac{1}{\left|\omega_{n}\right|}\left(\frac{E_{Z}^{2}}{\sqrt{\left(E_{Z}^{2}+\omega_{n}^{2}\right)\left(E_{Z}^{2}+\left(E_{\mathrm{SO}}^{\Gamma}\right)^{2}+\omega_{n}^{2}\right)}}\right).
\end{equation}
The last equation can be obtained by linearizing Eq.~\eqref{eq:self_lambda_equal_nu}.
Again, it evidently represents the average over the combined Fermi surface made out of $K/K'$ and $\Gamma$ pockets.

\subsubsection{The effect of impurities on the OP}
\label{sec:Impurities}
In the case where all the interactions amplitudes are equal, we saw
that we effectively have a single band comprised of pairing in the
$K/K'$ and $\Gamma$ pockets. Neglecting inter-valley scattering, the presence of impurities can effect
the OP through the $\Gamma$ pairing. This is because the spin-orbit nodes in the $\Gamma$ pocket lead to the 'flattened' shape of the OP as a function of in-plane magnetic field at intermediate fields. (A model with two nodeless pockets cannot reproduce this aspect of the data.)

In order to include this effect, we write the Gorkov equation as
\begin{equation}
\label{eq:Gorkov_imp}
\left[i\omega_{n}\hat{\sigma}_{0}-\hat{H}_{\mathrm{BdG}}^{\Gamma}\left(\mathbf{k}\right)-\hat{\Sigma}^{\left(\Gamma\right)}\left(\omega_{n}\right)\right]\hat{G}^{\Gamma}\left(\mathbf{k},\omega_{n}\right)=\hat{\sigma}_{0}
\end{equation}
where $\hat{\Sigma}^{\left(\Gamma\right)}\left(\omega_{n}\right)$
is the self-energy given withing the self-consistent Born approximation
by \cite{Wickramaratne2021}
\begin{equation}
\label{eq:self_energy}
\hat{\Sigma}^{\left(\Gamma\right)}\left(\omega_{n}\right)=\tau^{-1}_{\Gamma}\int\frac{d\varphi_{\mathbf{k}}}{2\pi}\int\frac{d\xi_{\mathbf{k}}^{\Gamma}}{\pi}\hat{\sigma}_{z}\hat{G}^\Gamma\left(\mathbf{k};\omega_{n}\right)\hat{\sigma}_{z}\, ,
\end{equation}
where $\hat{\sigma}_{z}=\mathrm{diag}\left(\sigma_{0},-\sigma_{0}\right)$
and $\tau^{-1}_{\Gamma}$ is intra $\Gamma$ pocket impurity
scattering rate. We calculate the OP by finding the root of the self
consistency \eqref{eq:self_lambda_equal_nu}, where $f_{0}^{\Gamma}\left(\hat{\mathbf{k}},\omega_{n}\right)$
now includes the effect of impurities as in Eq.~\eqref{eq:Gorkov_imp}.
In order to find the root we must evaluate $\left\langle f_{0}^{\Gamma}\left(\hat{\mathbf{k}},\omega_{n}\right)\right\rangle_{\varphi_{\hat{\mathbf{k}}}}$ for
any given $\Delta_{s}$. This is done by calculating $\hat{\Sigma}^{\left(\Gamma\right)}\left(\omega_{n},\Delta_{s}\right)$
for the given $\Delta_{s}$ by way of iterations using Eqs.~(\ref{eq:Gorkov_imp},\ref{eq:self_energy}),
then inserting $\hat{\Sigma}^{\left(\Gamma\right)}\left(\omega_{n},\Delta_{s}\right)$
into Eq.~\eqref{eq:Gorkov_imp} and finding $\left\langle f_{0}^{\Gamma}\left(\hat{\mathbf{k}},\omega_{n}\right)\right\rangle_{\varphi_{\hat{\mathbf{k}}}} $.
The results for different values
of $\tau^{-1}_{\Gamma}$ are presented in Fig.~\ref{fig:res4}. In our numerical calculations we take $E_{\mathrm{SO}}^{K}/E_{\mathrm{SO}}^{\Gamma}=2.14$ based on  Ref.~\cite{Wickramaratne2020} assuming that this ratio is roughly similar for mono- and bilayer systems.

Without disorder a reasonable fit to the experimental data is achieved (Figure~\ref{fig:res4}). The level of experimental disorder is estimated to be at least $\approx$ 10 meV (cf. Section IC), which is to say tens of $T_{cs}$. The inclusion of such a strong disorder at the level of the self-consistent Born approximation makes the procedure numerically unstable. 
Nevertheless, even at the levels of disorder we are able to incorporate, it can already be seen that the fit becomes progressively worse with increasing disorder (Figure~\ref{fig:res4}). At even higher values of disorder scattering rate, we expect it to continue to get worse compared to the one obtained with the triplet pairing. 
In addition, with increasing disorder, an increase in the value of spin-orbit coupling necessary to obtain the same critical field. Finally, at very strong levels of disorder, i.e. $1/\tau_{\Gamma} \gg \Delta_{SO}^\Gamma$, the $\Gamma$-pocket is described by AG-theory which yields a field-dependence of the OP inconsistent with the data.


\begin{figure*}%[htp]
\centering
\includegraphics[width=0.9\textwidth]{pockets/FigPockets4.pdf}
\caption{\label{fig:res4} 
The OP as a function of $E_Z$. The black and blue lines have $K$ and $\Gamma$ equal pairing interactions  and are calculated by numerically solving Eq.~ \eqref{eq:self_lambda_equal_nu} for the clean case (black line) and with a high impurity scattering rate of $\tau_{\Gamma}^{-1} = 10$ (blue line). The dashed and dotted lines show the OP in the clean case with only the $K$ and $\Gamma$ pocket pairings, respectively. 
These are found by replacing in Eq.~\eqref{eq:self_lambda_equal_nu} $f_{0}^{\Gamma}$ by $f_{0}^{K}$ for the $K$ pocket and replacing $f_{0}^{K}$ with the $f_{0}^{\Gamma}$ for the $\Gamma$ pocket. 
The green dots show the experimental data and the SOC for each line is taken to fit the experimental critical field. 
The parameters  $\tau^{-1}_{\Gamma}$ and $E_{\mathrm{SO}}^{\Gamma},E_{\mathrm{SO}}^{K}$ are in units of $T_{c0}$ and for the black and blue lines we take $E_{\mathrm{SO}}^{K}/E_{\mathrm{SO}}^{\Gamma}=2.14$. 
}
\end{figure*}

\FloatBarrier

\subsection{Two-pocket superconductivity in NbSe$_2$ - McMillan coupling}
\label{sec-mm}

Here we address the possibility that multipocket physics, with a McMillan coupling\cite{mcmillan} plays a significant role in our findings.
As before we, consider the $K(K')$ and $\Gamma$ pockets, originating from the $\mathrm{Nb}$ derived band\cite{Wickramaratne2020}.
Unlike in section \ref{sec:two-pockets-MSW} where two-particle intra-pocket tunneling was discussed, the McMillan coupling considered here amounts to single-particle tunneling.

The theoretical model is presented in section \ref{sec:mcmillan_model},  a comparison with experimental data is given in section \ref{sec:mcmillan_exp}, followed by a short discussion in section \ref{sec:mcmillan_comments}.

\subsubsection{Multipocket model}\label{sec:mcmillan_model}

We account for scattering within and between pockets by the following rates:
\begin{itemize}
	\item $\tau^{-1}$: scattering within the $K,K'$ pockets (intravalley scattering). Note that this type of disorder has no effect on superconductivity due to the Anderson theorem,
	\item $\tau_{iv}^{-1}$: Scattering between $K$ and $K'$ pockets (intervalley scattering),
	\item $\tau_\Gamma^{-1}$: scattering within the $\Gamma$ pocket, 
	\item $\Gamma_{K\Gamma}$, $\Gamma_{\Gamma K}$: scattering from the $K$ to the $\Gamma$ pocket and vice-versa. 
\end{itemize}
The following relation holds for interpocket scattering rates
\begin{equation}
\frac{\Gamma_{K\Gamma}}{\Gamma_{\Gamma K}}=\frac{N_\Gamma}{N_K},
\label{eq2}
\end{equation}
where $N_K$ is the sum of the normal state DoS of the K and K' pockets, and $N_\Gamma$ is the normal state DoS of the $\Gamma$ pocket. Taking into account that DoS per pocket is approximately the same for $K$, $K'$ and $\Gamma$, we have $N_K\approx 2 N_\Gamma$ \cite{sticlet2019topological}. 

We will assume $s$-wave superconductivity, that originates from the $K$ pocket, and is induced by the proximity effect in the $\Gamma$ pocket through interpocket scattering. Furthermore, we will assume diffusive limit in the $\Gamma$ pocket, $\tau_\Gamma^{-1}\gg E_{SO}^\Gamma, E_Z, \Delta_0, \Gamma_{ij}$. If the SOC is sufficiently strong in both pockets, that is, if $E_{SO}^K\gg \Delta_0, \Gamma_{ij}, E_Z, \tau_{iv}^{-1}$, and $(E_{SO}^\Gamma)^2\tau_\Gamma \gg \Delta_0,E_Z,\Gamma_{ij}$, the density of states is given by the Kaiser-Zuckermann formula \cite{kaiser1970mcmillan}, which is a multiband extension of the Abrikosov-Gor'kov theory.  Namely, depairing is both $K$ and $\Gamma$ pockets is captured by the rates quadratic in magnetic field
\begin{equation}
\Gamma_{AG}^K(E_Z)=\frac{E_Z^2}{(E_{SO}^K)^2 \tau_{iv}}, \qquad
\Gamma_{AG}^\Gamma(E_Z)=\frac{E_Z^2}{(E_{SO}^\Gamma)^2 \tau_\Gamma},
\end{equation}
and we can introduce the functions $u_K$ and $u_\Gamma$ that satisfy
\begin{equation}
\frac{\omega_n}{\Delta}=u_K-\frac{\Gamma_{AG}^K(E_Z)}{\Delta}\frac{u_K}{\sqrt{1+u_K^2}}+\frac{\Gamma_{K\Gamma}}{\Delta}\frac{u_K-u_\Gamma}{\sqrt{1+u_\Gamma^2}} = -\frac{\Gamma_{AG}^\Gamma(E_Z)}{\Delta}\frac{u_\Gamma}{\sqrt{1+u_\Gamma^2}}+\frac{\Gamma_{\Gamma K}}{\Delta}\frac{u_\Gamma-u_K}{\sqrt{1+u_K^2}}.
\label{eq4}
\end{equation}
Here, $\Delta$ is the intrinsic gap from the $K$ pocket. Then, the DoS in the superconducting state is given as
\begin{equation}
\nu(E)=\sum_{i=K,\Gamma} N_i \Re\left[\frac{u_i}{\sqrt{1+u_i^2}}\right]_{i\omega_n\to E}.
\end{equation}

The intrinsic gap $\Delta$ satisfies the self-consistency condition
\begin{equation}
\Delta=2\pi T N_K |v_s| \sum_{\omega_n>0}\frac{1}{\sqrt{1+u_K^2}}.
\label{eq6}
\end{equation}
An alternative formulation is given by Eq. \eqref{eqscalt}, parameterized in terms of the critical temperature $T_c$ instead of the coupling constant $v_s$
\begin{equation}
\ln \bigg(\frac{T}{T_c}\bigg)+\frac{\Gamma_{K\Gamma}}{\Gamma_{K\Gamma}+\Gamma_{\Gamma K}}\bigg[\psi \bigg(\frac{1}{2}\bigg)-\psi \bigg(\frac{1}{2}+\frac{\Gamma_{K\Gamma}+\Gamma_{\Gamma K}}{2 \pi  T_c}\bigg)\bigg]=2 \pi T \sum_{\omega_n>0} \bigg[\frac{1}{\Delta}\frac{1}{\sqrt{1+u_K^2}}  - \frac{1}{\omega_n} \bigg].
\label{eqscalt}
\end{equation}
Here, $\psi(x)$ if the digamma function

Close to the phase transition to the normal state, that is, in the limit of  vanishing $\Delta$, it is possible to analytically evaluate Eq. \eqref{eq6}. The self-consistency condition then becomes
\begin{equation}
\ln \bigg(\frac{T}{T_c}\bigg)+\frac{\Gamma_{K\Gamma}}{\Gamma_{K\Gamma}+\Gamma_{\Gamma K}}\bigg[\psi \bigg(\frac{1}{2}\bigg)-\psi \bigg(\frac{1}{2}+\frac{\Gamma_{K\Gamma}+\Gamma_{\Gamma K}}{2 \pi  T_c}\bigg)\bigg]= \sum_{i=\pm} \frac{\alpha_i}{\chi^2}\bigg[
\psi \bigg(\frac{1}{2}+\frac{\beta_i}{2\pi  T}\bigg)-\psi\bigg(\frac{1}{2}\bigg) \bigg)
\bigg].
\label{eq7}
\end{equation}  
Here $T_c$ is the critical temperature. Furthermore, we have introduced  
\begin{align}
\chi= &\,\bigg(4 \Gamma_{K\Gamma}\Gamma_{\Gamma K}+[\Gamma_{K\Gamma}-\Gamma_{\Gamma K}+\Gamma_{AG}^K(E_{Z}^{c})-\Gamma_{AG}^\Gamma(E_Z^{c})]^2\bigg)^{1/2}, \nonumber \\
\alpha_{\pm}= &\, \frac{1}{2}\bigg(-\chi^2\pm\chi [\Gamma_{K\Gamma}-\Gamma_{\Gamma K}+\Gamma_{AG}^K(E_{Z}^c)-\Gamma_{AG}^\Gamma(E_{z}^c)]\bigg), \nonumber  \\
\beta_{\pm}= & \,\frac{1}{2}\bigg(\Gamma_{K\Gamma}+\Gamma_{\Gamma K}+\Gamma_{AG}^K(E_Z^c)+\Gamma_{AG}^\Gamma(E_Z^c)\mp \chi \bigg),
\end{align}
where $E_Z^c$  is the upper critical field. Equation \eqref{eq7} can be used to calculate the $E_Z^c$, provided that the scattering parameters are known. 

\subsubsection{Comparison with the experiment}\label{sec:mcmillan_exp}

The main goal is to obtain the $\Delta(H)$ dependence, based on equation \eqref{eqscalt}. As it stands there are five parameters: the critical temperature $T_c$, the scattering rates $\Gamma_{K\Gamma}$ and $\Gamma_{\Gamma K}$, and the Abrikosov-Gorkov depairing parameters $\Gamma_{AG}^K$ and $\Gamma_{AG}^\Gamma$.
In order to reduce the size of the parameter space, and to ease the exploration, some physically motivated constraints were placed on the parameter values.\\

\subsubsubsection{Zero magnetic field}

At zero field the Abrikosov-Gorkov parameters do not play a role and the order parameter is a function of $\Delta = \Delta(T,T_c,\Gamma_{K\Gamma},\Gamma_{\Gamma K})$.
The temperature was fixed to the experimental one ($T=1.3\mathrm{K}$) and the ratio of the pocket scattering rates is set according to Eq. \eqref{eq2}: $\Gamma_{\Gamma K} = 2\Gamma_{K\Gamma}$. %, based on the pocket structure calculations \cite{Wickramaratne2020} \textcolor{red}{CHECK REF}.\\

With this, at $H=0$ it remains to determine $T_c$, this was done by constraining the value of $\Delta$ to $\approx 390\mathrm{\mu eV}$, as obtained in the experiment  (cf. inset of Figure~\ref{fig:GvsT}). Then we can solve for $T_c$ as a function of $\Gamma_{K\Gamma}$. The result is shown on figure \ref{fig:TcvsGamma}: at zero coupling the ratio $\frac{\Delta}{k_BT_c}$ starts from the BCS value of $\approx1.76$ and grows as the coupling is increased. Therefore for the same value of $\Delta$ the critical temperature of a multipocket superconductor is lower than that of a BCS one.
The densities of states of both pockets, at $H=0$, for select values of $\Gamma_{K\Gamma}$ are shown in figures \ref{fig:dosK} and \ref{fig:dosG}.
\newline

\begin{figure}[h!]
	\centering
	\includegraphics[width=0.5\textwidth]{mutliband/Tc_vs_Gamma.pdf}
	\caption{The dependence of $T_c$ on $\Gamma_{K\Gamma}$, constrained to $\Delta\approx 390\mathrm{\mu eV}$ at $T=1.25\mathrm{K}$.}
	\label{fig:TcvsGamma}
\end{figure}

\begin{figure}[!htb]
	\centering
	\begin{minipage}{.48\textwidth}
		\centering
		\includegraphics[width=\textwidth]{mutliband/dosK.pdf}
		\caption{The DOS in the $K$ pocket vs $\Gamma_{K\Gamma}$.}
		\label{fig:dosK}
	\end{minipage}%
	\begin{minipage}{0.48\textwidth}
		\centering
		\includegraphics[width=\textwidth]{mutliband/dosG.pdf}
		\caption{The DOS in the $\Gamma$ pocket vs $\Gamma_{K\Gamma}$.}
		\label{fig:dosG}
	\end{minipage}
\end{figure}



\subsubsubsection{Finite magnetic field}

At finite fields the values of the depairing parameters in both pockets play a significant role. Here they are parametrized as $\Gamma_{AG}^K(H)=\Gamma_{AG}^K H^2$ and $\Gamma_{AG}^\Gamma(H)=\Gamma_{AG}^\Gamma H^2$. Equation \eqref{eq7} can be used to determine the values of $\Gamma_{AG}^K$ and $\Gamma_{AG}^\Gamma$ such that the critical field matches the experimentally determined one $H_c=30\mathrm{T}$. No matter what the inter-pocket coupling is, a solution will always exist where $\Gamma_{AG}^\Gamma=0$ and $\Gamma_{AG}^K=\Gamma_c^{AG}(\Gamma_{K\Gamma})$, which satisfies the $H_c$ constraint. At zero coupling this is the only solution. Figure \ref{fig:gammaAGHC30} shows $\Gamma_{AG}^\Gamma$ vs $\Gamma_{AG}^K$, for different coupling strengths. Above a certain coupling strength the experimental $H_c$ can reproduced by depairing in either pocket, but below this threshold $H_c=30\mathrm{T}$ can be obtained only with a non-zero value of $\Gamma_{AG}^K$. 

\begin{figure}[h!]
	\centering
	\includegraphics[width=0.5\textwidth]{mutliband/gamma_AG_HC30.pdf}
	\caption{The values of $\Gamma_{AG}^K$ and $\Gamma_{AG}^\Gamma$ which result in $H_c=30\mathrm{T}$, for several values of $\Gamma_{K\Gamma}$ ($T=1.25\mathrm{K}$).}
	\label{fig:gammaAGHC30}
\end{figure}


Consequently, the parameter space can be reduced to just two parameters: $\Gamma_{K\Gamma}$ and $\Gamma_{AG}^K$. We can now proceed and compare the $\Delta(H)$ curves with the experimental one.

In either the limit of zero coupling or very high coupling ($\Gamma_{K\Gamma}>>\Delta$) the field dependence reduces to the Abrikosov-Gorkov one. This makes sense, as at high couplings the bands are indistinguishable (i.e. DOS in both pockets is the same - the green traces in figures \ref{fig:dosK} and \ref{fig:dosG}), so depairing in either pocket can kill superconductivity, and therefore leads to the same $\Delta(H)$ trace.
Figures \ref{fig:deltaH_low} and \ref{fig:deltaH_high} show the zero coupling case (i.e. AG dependence) and the highly coupled one (for several combinations of $(\Gamma_{AG}^K,\Gamma_{AG}^\Gamma)$, which all overlap) respectively.

Next, the case of intermediate coupling will be discussed, for several values of $\Gamma_{K\Gamma}$,  and compared with our experimental data.\\
Figure \ref{fig:gamma260} shows $\Delta(H)$ for $\Gamma_{K\Gamma}=260\mathrm{\mu eV}$: The traces for which the depairing is dominantly in the $K$ pocket (e.g. the blue trace) agree with the data close to $H=0$ and $H=H_c$, but not in between. If the depairing is only in the $\Gamma$ pocket the trace lies below the experimental data for all fields. The purple trace shows the highest value of $\Gamma_G$ which still fits the low and high field data, increasing it further will improve the fit at intermediate fields, but at the expense of the former two.\\
Figure \ref{fig:gamma280} shows $\Delta(H)$ for $\Gamma_{K\Gamma}=280\mathrm{\mu eV}$: The green trace, for which the depairing is solely in the $\Gamma$ pocket, reproduces the experimental data relatively well, except for the bump visible at $H>20\mathrm{T}$. The "revival" of superconductivity, relative to the previous case, can be understood in the following way - the pockets are coupled strongly, superconductivity originates in the $K$ pocket while the depairing is induced in the $\Gamma$ pocket, and a smaller depairing strength is needed to obtain the same critical field. Lower depairing rates in $\Gamma$ (and higher in $K$) all result in traces above the experimental data.\\
Figure \ref{fig:gamma300} shows $\Delta(H)$ for $\Gamma_{K\Gamma}=300\mathrm{\mu eV}$: The same qualitative behavior is observed as for $\Gamma_{K\Gamma}=280\mathrm{\mu eV}$, but with a slightly worse agreement at intermediate fields.
Figure \ref{fig:gamma320} shows $\Delta(H)$ for $\Gamma_{K\Gamma}=320\mathrm{\mu eV}$: all $\Delta(H)$ curves now lie above the experimental data.

We conclude that if the two pocket model is to explain the experimental $\Delta(H)$ curve the appropriate scattering rate is close to $\Gamma_{K\Gamma}\approx280\mathrm{\mu eV}$.

\begin{figure}[!htb]
	\centering
	\begin{minipage}{.48\textwidth}
		\centering
		\vspace*{-4mm}
		\includegraphics[width=\textwidth]{mutliband/delta_vs_H_low_coupling.pdf}
		\caption{$\Delta$ vs $H$ curve for uncoupled pockets. The depairing is in the $K$ pocket.}
		\label{fig:deltaH_low}
	\end{minipage}%
	\begin{minipage}{0.48\textwidth}
		\centering
		\includegraphics[width=\textwidth]{mutliband/delta_vs_H_high_coupling.pdf}
		\caption{$\Delta$ vs $H$ curve for highly coupled pockets. The depairing is either in the $K$ pocket (blue line), the $\Gamma$ pocket (green line), or both (other lines).}
		\label{fig:deltaH_high}
	\end{minipage}
\end{figure}

\begin{figure}[!htb]
	\centering
	\begin{minipage}{.48\textwidth}
		\centering
		\includegraphics[width=\textwidth]{mutliband/gamma260.pdf}
		\caption{$\Delta$ vs $H$ curve for $\Gamma_{K\Gamma}=260\mathrm{\mu eV}$ and $\Gamma_{K\Gamma}=520\mathrm{\mu eV}$.}
		\label{fig:gamma260}
	\end{minipage}%
	\begin{minipage}{0.48\textwidth}
		\centering
		\includegraphics[width=\textwidth]{mutliband/gamma280.pdf}
		\caption{$\Delta$ vs $H$ curve for $\Gamma_{K\Gamma}=280\mathrm{\mu eV}$ and $\Gamma_{K\Gamma}=560\mathrm{\mu eV}$.}
		\label{fig:gamma280}
	\end{minipage}
\end{figure}


\begin{figure}[!htb]
	\centering
	\begin{minipage}{.48\textwidth}
		\centering
		\includegraphics[width=\textwidth]{mutliband/gamma300.pdf}
		\caption{$\Delta$ vs $H$ curve for $\Gamma_{K\Gamma}=300\mathrm{\mu eV}$ and $\Gamma_{K\Gamma}=600\mathrm{\mu eV}$.}
		\label{fig:gamma300}
	\end{minipage}%
	\begin{minipage}{0.48\textwidth}
		\centering
		\includegraphics[width=\textwidth]{mutliband/gamma320.pdf}
		\caption{$\Delta$ vs $H$ curve for $\Gamma_{K\Gamma}=320\mathrm{\mu eV}$ and $\Gamma_{K\Gamma}=640\mathrm{\mu eV}$.}
		\label{fig:gamma320}
	\end{minipage}
\end{figure}

\FloatBarrier

\subsubsection{Comments and discussion}\label{sec:mcmillan_comments}

Based on the analysis presented in this section, a McMillan two-pocket model can, in principle, reproduce the $\Delta(H)$ curve measured in this work, with a large inter-pocket coupling. 
%However this would indicate that the two-pocket character should be prominent in thin ($\approx$ few ML) samples. This goes against the results presented in figure 2 of the main text, as well as the results of \cite{khestanova2018unusual}. 
For the $K\Gamma$ coupling strength required to explain the $\Delta(H)$ data the DOS of the $K$ pocket, which is dominantly probed in tunneling experiments \cite{zhu,wang2012}, is given by the black trace in figure \ref{fig:dosK}. The coherence peak reaches up to $\approx 1.4N_F$, which is comparable to our spectra measured at a finite temperature and with additional broadening, both of which reduce the height of the observed coherence peak. The shoulder, which is clearly present in the theoretical DOS, is absent from tunneling data measured with high-quality junctions - see figure 2c of the main text as well as the bi-layer tunneling spectrum from \cite{khestanova2018unusual}, reproduced here in figure \ref{fig:geim2L}.

Additionally, in thin samples we observe the BCS ratio of $\Delta/T_c\approx 1.76$ (c.f. fig. \ref{fig:GvsT}), while for the coupling needed to fit the $\Delta(H)$ curve the ratio is $\approx 2.1$ (c.f. fig \ref{fig:TcvsGamma}).

Thus, based on spectroscopic and transport data, we argue that the two pocket model with a type McMillan coupling\cite{mcmillan} is an unlikely explanation of the observed data.

\FloatBarrier


\section*{References}
\bibliographystyle{naturemag}
\bibliography{Bibliography.bib}

\end{document}
