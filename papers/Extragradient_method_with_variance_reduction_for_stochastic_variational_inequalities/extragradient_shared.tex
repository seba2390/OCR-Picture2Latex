% SIAM Shared Information Template
% This is information that is shared between the main document and any
% supplement. If no supplement is required, then this information can
% be included directly in the main document.


% Packages and macros go here
\usepackage{amsmath}
\usepackage{latexsym}
\usepackage{amssymb}
\usepackage{lipsum}
\usepackage{amsfonts}
\usepackage{graphicx}
\usepackage{epstopdf}
\usepackage{algorithmic}
\ifpdf
  \DeclareGraphicsExtensions{.eps,.pdf,.png,.jpg}
\else
  \DeclareGraphicsExtensions{.eps}
\fi

%strongly recommended
\numberwithin{theorem}{section}

% Declare title and authors, without \thanks
\newcommand{\TheTitle}{Extragradient method with variance reduction for stochastic variational inequalities} 
\newcommand{\TheAuthors}{A. Iusem, A. Jofr\'e, R. I. Oliveira, and P. Thompson}

% Sets running headers as well as PDF title and authors
\headers{Extragradient method with variance reduction}{\TheAuthors}

% Title. If the supplement option is on, then "Supplementary Material"
% is automatically inserted before the title.
\title{{\TheTitle}\thanks{Submitted to the editors DATE.
}}

% Authors: full names plus addresses.
\author{
Alfredo N. Iusem\thanks{Instituto de Matem\'atica Pura e Aplicada (IMPA), 
Rio de Janeiro, RJ, Brazil.
(\email{iusp@impa.br}).}
\and
Alejandro Jofr\'e\thanks{Centro de Modelamiento Matem\'atico (CMM \& DIM), 
Santiago, Chile.
(\email{ajofre@dim.uchile.cl}).}
\and
Roberto I. Oliveira\thanks{Instituto de Matem\'atica Pura e Aplicada (IMPA), 
Rio de Janeiro, RJ, Brazil.
(\email{rimfo@impa.br}). Roberto I. Oliveira's work was supported by a {\em Bolsa de Produtividade em Pesquisa} from CNPq, Brazil. His work in this article is part of the activities of FAPESP Center for Neuromathematics (grant \#2013/07699-0, FAPESP - S. Paulo Research Foundation).}
\and
Philip Thompson\thanks{Instituto de Matem\'atica Pura e Aplicada (IMPA), 
Rio de Janeiro, RJ, Brazil.
(\email{philip@impa.br}). Philip Thompson's work was supported by a CNPq Doctoral scholarship while he was a PhD student at IMPA with visit appointments at CMM. His work was conducted at IMPA and CMM.}
}

\usepackage{amsopn}
\DeclareMathOperator{\diag}{diag}

%new Commands&theorems:
\newcommand{\esp}{\mathbb{E}}
\newcommand{\prob}{\mathbb{P}}
\newcommand{\var}{\mathbb{V}}
\newcommand{\alg}{\mathcal{F}}
\newcommand{\alge}{\mathcal{G}}
\newcommand{\unit}{\mathbb{I}}
\newcommand{\id}{\mathbb{I}}	
\newcommand{\sol}{\mbox{SOL}}
\newcommand{\sole}{\emph{\mbox{SOL}}}
\newcommand{\vi}{\mbox{VI}}
\newcommand{\vie}{\emph{\mbox{VI}}}
\newcommand{\re}{\mathbb{R}}
\newcommand{\pc}{\mathfrak{p}}
\newcommand{\qc}{\mathfrak{q}}
\DeclareMathOperator*{\covar}{cov}

% Norma bonita!
%\newcommand{\vertiii}[1]{{\left\vert\kern-0.25ex\left\vert\kern-0.25ex\left\vert #1 
%    \right\vert\kern-0.25ex\right\vert\kern-0.25ex\right\vert}}
\newcommand{\vertiii}[1]{{\left\vert\kern-0.4ex #1 
\kern-0.4ex\right\vert}}
\newcommand{\Lpnorm}[1]{\vertiii{\,#1\,}_{p}}
\newcommand{\Lqnorm}[1]{\vertiii{\,#1\,}_{q}}
\newcommand{\Lpcnorm}[1]{\vertiii{\,#1\,}_{\mathfrak{p}}}
\newcommand{\Lqcnorm}[1]{\vertiii{\,#1\,}_{\mathfrak{q}}}	
\newcommand{\Lnorm}[1]{\vertiii{\,#1\,}_{2}}
% Produto interno 
\newcommand{\inner}[2]{\left\langle #1, #2\right\rangle}

\DeclareMathOperator*{\dist}{d}
\DeclareMathOperator*{\diam}{diam}
\newcommand{\argmin}{\mbox{argmin}}
\newcommand{\tang}{\mathbb{T}}
\newtheorem{lema}{Lemma}
\newtheorem{thm}{Theorem}
\newtheorem{cor}{Corollary}
\newtheorem{prop}{Proposition}
\newtheorem{assump}{\textbf{Assumption}}
\newtheorem{algo}{Algorithm}
\newtheorem{rem}{Remark}
\newtheorem{example}{Example}

%%% Local Variables: 
%%% mode:latex
%%% TeX-master: "extragradient_article"
%%% End: 
