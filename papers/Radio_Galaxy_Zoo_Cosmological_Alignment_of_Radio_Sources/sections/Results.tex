\section{Results}
	\label{sec:RES}
 
	Unless stated otherwise, in this section we assume as our null hypothesis the absence of spatial coherence in the orientations of radio sources. 
 
	In Fig.~\ref{fig:Sn} we plot the significance levels (S.L.) of the angular dispersion statistics $S_n$ for three different position angle samples: (1) Radio Galaxy Zoo or RGZ (2) TGSS (3) A subset of the Radio Galaxy Zoo sample, or RGZ II. This last one is designed to mimic the source count and number density of the TGSS sample, by randomly eliminating two thirds of the sources in the RGZ sample. This results in a reduced number count of $10\,088$ and a number density of about $1.5$ deg$^{-2}$. We use this dataset to also confirm that the relations \eqref{eq:Sest} and \eqref{eq:sigmaest} are confirmed up to a margin of $10\%$.
	
	Using Eq. \eqref{eq:Maha} as a statistical test, we obtain $d^2 = 26.15$ for the RGZ sample, corresponding to a p-value $< 0.02$. On the plotted scales this signal is found not to be consistent with the noise. The distribution of $S_{35}$ for the shuffled catalogues (see Sec.~\ref{sec:RandomDatasets}) is plotted in Fig.~\ref{fig:S35}, together with the measured value.
	
	For the other two samples in Fig.~\ref{fig:Sn}, the signal is confirmed to be consistent with the noise (p-value $> 0.05$).
	
	The lower limit for the variable $n$ is set by the condition $n \gg 1$ and in our case we choose $n=15$. On the other hand, the upper limit can reach any value $n < N$, where $N$ is the total number of sources in the sample. For the maximum values of $n$, our choice was motivated by the corresponding angular scales. In Fig.~\ref{fig:sep} we plot the median value of the set of angular scales $\{\varphi_{i, n}\}$ probed as a function of every considered $n$, see Eq.~\eqref{eq:phii}. The errorbars delimit the $68\%$ interval centred on the median. For the RGZ sample the maximum $n=80$ corresponds to $\tilde{\varphi} \approx 2.5^\circ$. For $30\%$ of the sources in our sample, the Radio Galaxy Zoo consensus catalogue contains an optical counterpart with known redshift. Around two thirds of these are spectroscopic and the rest are photometric. Fig.~\ref{fig:z} presents the redshift distribution. The median value is $z = 0.47$ if we consider both classes, and $z = 0.54$ if we consider only spectroscopic redshifts. Assuming a flat $\Lambda$CDM Cosmology and cosmological parameters $\Omega_m=0.31, \Omega_\Lambda = 0.69$; the angular scale of $2.5^\circ$ is equivalent to a comoving scale of around $70-85$ $h_{70}^{-1}$ Mpc at these redshifts. This is the typical length of the longest low-redshift filaments of the cosmic web \citep{Tempel2014}. Since no redshift information is provided for the TGSS sample, we opt for a maximal $n$ corresponding to an angular scale of $\varphi = 5^\circ$.
	
	
		\begin{figure}
			\centering
			\includegraphics[width=.45\textwidth]{files/Sn.pdf}
			\caption{Logarithm of the significance level (S.L.) of the statistics $S_n$ as a function of the number of neighbours $n$ applied to three samples (see text for details). The sample standard deviation of the simulated datasets is also plotted.}
			\label{fig:Sn}
		\end{figure}
		

		\begin{figure}
			\centering
			\includegraphics[width=.45\textwidth]{files/S35.pdf}
			\caption{The distribution of the statistics $S_{35}$ for the $1\,000$ shuffled catalogues of the RGZ sample as presented in Sec.~\ref{sec:RandomDatasets}. The dashed red line marks the highly significant observed value. }
			\label{fig:S35}
		\end{figure}
				
	Of the two physical position angle samples considered, RGZ is the only one containing a signal significantly higher than the noise, consistently above the weak alignment threshold as defined in Section~\ref{sec:S}. Physically we would expect the alignment strength to decrease as a function of $n$. However, in Fig.~\ref{fig:Sn} we can see a minimum of the S.L. located between $n = 35$ and $n=40$, corresponding to an angular scale between $1.5^\circ$ and $2^\circ$ (Fig.~\ref{fig:sep}). This is due to the broader distribution of $d_{i, n}$ for small $n$, which lowers the significance of $S_n$. A similar effect is visible when the same statistic is employed elsewhere \citep[e.g.][]{Lamy2001}.
	
		\begin{figure}
			\centering
			\includegraphics[width=.45\textwidth]{files/sep.pdf}
			\caption{Median of the aperture radii probed by considering the $n$ closest neighbours as a function of $n$. The errorbars delimit the $16$th and $84$th percentile of the distributions. Two of three samples are described in Section~\ref{sec:SS} (TGSS and Radio Galaxy Zoo). The third, RGZ II, is a subsample of the RGZ sample designed to mimic the TGSS lower source density and source count.}
			\label{fig:sep}
		\end{figure}
		
	We use the position of this minimum as an upper bound of the maximal alignment scale. To get an estimate of the physical scales probed by $n=40$, we then use the available redshift information (Fig.~\ref{fig:z}). For the $68\%$ redshift interval quoted in Table \ref{tab:surveys}, the angular size $\varphi = 1.5^\circ$ corresponds to transversal physical sizes in the range $[19, 38]$ Mpc.  These distances roughly correspond to differential redshifts along the line of sight of the order of $\Delta z \sim 0.01$.
	
	If the alignment signal is due to physical proximity we expect these to be the relevant scales. To validate physical proximity as a possible explanation, we confirm that, among the sources with known redshift, $\sim1.5\times10^3$ pairs have an angular separation within $1.5^\circ$ and redshift difference within $0.01$. Since only a third of the RGZ sample has known redshift, we can then estimate the number of physically close pairs as $3\times1.5\times10^3 = 4.5\times10^3$. Because of the large uncertainties on photometric redshifts, this value underestimates the number of real pairs.
	


		\begin{figure}
			\centering
			\includegraphics[width=.45\textwidth]{files/Z.pdf}
			\caption{Redshift distribution of the selected sources in the Radio Galaxy Zoo sample. Around $13\%$ of the sources have photometric redshift and another $17\%$ of them have spectroscopic redshift. }
			\label{fig:z}
		\end{figure}	


	The absence of an alignment signal in TGSS is not surprising. When reduced to similar number densities and source counts the signal is not present in the RGZ sample either.	Number density and source count affect the final signal $S_n$ in different ways. 
	A lower number density has the effect of shifting the signal towards lower $n$. As visible in Fig.~\ref{fig:sep}, the maximum scale $\tilde{\varphi}$ probed with the RGZ sample for $n=80$ corresponds barely to the minimum scale probed with the RGZ II sample.
	
	At the same time, the number count does directly affect the chances of measuring a significant alignment, since the variance is dominated by the shot noise in Eq. \eqref{eq:sigmaest}. Evidently, a change of a factor $3$ in the number of sources $N$ is enough to erase the alignment signal.
	
	The alignment detection discussed above could be contaminated by large radio galaxies, whose lobes are aligned with each other, e.g., along the same position angle, but are counted as separate sources in the RGZ sample.  This can occur because the volunteers are only presented with a $3\arcmin \times 3\arcmin$ field centred on a FIRST catalogue position, so sources larger than that may go unrecognized.  As a rough check on the impact of this potential contamination, we examined the FIRST images of $35$ double-lobed radio galaxies, $3.5\arcmin$ to $10\arcmin$ in extent, drawn from a sample of $6000$  such sources $>1'$ in extent and with secure optical identifications, compiled by one of us \citep[HA, see e.g.,][]{Andernach2012}.  None of these sources appeared in our RGZ sample as two distinct sources.  We therefore conclude that the large source contamination is unlikely to be making a significant contribution, based on a) the low (undetected) probability of having both lobes in our sample, b) and the relative scarcity of large sources in general, ($\sim 3.5\%$ of FRII radio galaxies are $ 1.5\arcmin$, using figure 11 from \citep{Overzier2003}, and c) the fact that our highest significance signal occurs between $1.5$ and $2$ degrees, where there are only a handful of sources so large in the whole sky.  However, the existence of a small fractional population of sources that RGZ volunteers may not find should be investigated further when detailed size distributions are being studied.