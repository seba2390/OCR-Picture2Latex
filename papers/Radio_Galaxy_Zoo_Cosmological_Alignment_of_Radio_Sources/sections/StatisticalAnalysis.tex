\section{Statistical Analysis}
	\label{sec:SA}
	\subsection{Parallel Transport}	
		\label{sec:PT}  	
		
				
		The position angle is a directional quantity defined in the point of the celestial sphere where the corresponding source lies. In order to perform the calculation of the misalignment angle between two directions on a sphere, the notion of parallel transport should be introduced \citep{Jain2004}.
		
		We parametrize the sphere using spherical coordinates $(r, \theta, \phi)$ and we define in every point a natural orthonormal basis dictated by our coordinate system. This set of unit vectors is $(\mathbfit{e}_r, \mathbfit{e}_\theta, \mathbfit{e}_\phi)$, where the three elements point respectively towards the centre of the sphere, northward and eastward.	
		
		A source with position angle $\alpha$, determined up to a rotation of $\pi$ radians, can be identified with the unit vector
		
		\begin{equation}
			\mathbfit{v} = \cos\alpha \; \mathbfit{e}_\theta + \sin \alpha \; \mathbfit{e}_\phi
			\label{eq:v}
		\end{equation}
		
		Since the projection along the line of sight is unknown, we fix this vector to be tangent to the sphere at the point of definition.  
		The vector $\mathbfit{v}$ represents a physical quantity, whereas the definition of position angle $\alpha$ depends on the choice of coordinate system. For example, if parallels and meridians were redefined with respect to a different north pole, the vectors $\mathbfit{e}_\theta$, $\mathbfit{e}_\phi$ and the position angle $\alpha$ would change. However, the vector $\mathbfit{v}$ in Eq. \eqref{eq:v} would still describe the same direction in space.
		On a sphere, parallel transport allows us to define a coordinate-invariant inner product between two vectors, by translating one of them along arcs of great circles connecting the two.  
		
		Let us consider two tangent vectors $\mathbfit{v}_1$ and $\mathbfit{v}_2$ with position angles $\alpha_1$ and $\alpha_2$, defined respectively in $P_1 = (r_1, \theta_1, \phi_1)$ and $P_2 = (r_2, \theta_2, \phi_2)$. Both of these points belong to the same unit sphere ($r_1 = r_2 = 1$). The great circle passing through them lies on a plane perpendicular to $\mathbfit{e}_s$
		
		\begin{equation}
			\mathbfit{e}_s = \frac{\mathbfit{e}_{r_1}\times \mathbfit{e}_{r_2}}{\vert \mathbfit{e}_{r_1} \times \mathbfit{e}_{r_2}\vert}
		\end{equation}
		
		We define $\mathbfit{e}_{t_1}$ and $\mathbfit{e}_{t_2}$ as the tangent vectors of this great circle in the points $P_1$ and $P_2$.
		
		\begin{gather}
			\mathbfit{e}_{t_1} = \mathbfit{e}_{s} \times \mathbfit{e}_{r_1}
			\\
			\mathbfit{e}_{t_2} = \mathbfit{e}_{s} \times \mathbfit{e}_{r_2}
		\end{gather}
		
		We call $\zeta_1$ the angle between $\mathbfit{e}_{t_1}$ and $\mathbfit{e}_{\theta_1}$. Similarly, we define $\zeta_2$ as the angle between $\mathbfit{e}_{t_2}$ and $\mathbfit{e}_{\theta_2}$. Translating the vector $\mathbfit{v}_{1}$ along the great circle maintains the angle with the local tangent vector constant and at the point $P_2$ it results in the translated vector $\mathbfit{v}_1^\prime$ with position angle
		
		\begin{equation}
			\alpha_1^\prime =  \alpha_1 + \zeta_2 - \zeta_1
			\label{eq:alphaprime}
		\end{equation}
		
		Figure~\ref{fig:PT} depicts the vectors involved in the operation. With this in mind, we define the generalized dot product between $\mathbfit{v}_{1}$ and $\mathbfit{v}_{2}$ as the following
		
		\begin{equation}
			\mathbfit{v}_{1} \odot \mathbfit{v}_{2} = \vert \mathbfit{v}_{1} \vert
			\vert \mathbfit{v}_{2} \vert \cos (\alpha_1 - \alpha_2 + \zeta_2 - \zeta_1)
		\end{equation}
		
		Since our dataset is purely directional, we have $\vert \mathbfit{v}_{1} \vert = \vert \mathbfit{v}_{2} \vert = 1$. For the same reason, the inner product is written using the following simplified notation
		
		\begin{equation}
		(\alpha_1, \alpha_2) =  \cos [2(\alpha_1 - \alpha_2 + \zeta_2 - \zeta_1)]
		\label{eq:innerproduct}
		\end{equation}
		
		The factor two is introduced so that the argument of the cosine ranges over the full $-\pi$ to $+\pi$, \citep{Bietenholz1986}. By definition $(\alpha_1, \alpha_2) \in [-1, 1]$, where $+1$ indicates perfect alignment \citep{Jain2004} and $-1$ implies perpendicular directions.
		
		
\begin{figure}
	\centering
	\includegraphics[width=0.45\textwidth]{files/PT.pdf}
	\caption{Two dimensional schematic illustration of parallel transport. The figure displays the arc of great circle passing through the points $P_1$ and $P_2$, with $\mathbfit{e}_{t_1}$ and $\mathbfit{e}_{t_2}$ tangent vectors to curve in these points. Notice that the angle $\theta$ between the tangent vector and $\mathbfit{v}_{1}$ is kept constant when $\mathbfit{v}_{1}$, located at $P_1$, is translated along the curve to the point $P_2$. The figure is taken from \citet{Jain2004}, their figure 1, with the author's permission.}
	\label{fig:PT}
\end{figure}	

	\subsection{Angular Dispersion}
		\label{sec:S}
		Given the $i-$th source, we consider the $n$ sources closest to it (including itself). We call $d_{i,n}$ the dispersion function of their position angles.
		
		\begin{equation}
			d_{i, n}(\alpha) = \frac{1}{n}\sum_{k=1}^{n} (\alpha, \alpha_k)
			\label{eq:d}
		\end{equation}
		
		This quantity is a function of a position angle $\alpha$ located at the point where the $i-$th source lies. We call $\alpha_{\rm{max}}$ the position angle that maximizes the dispersion, which assumes the value
		
		\begin{equation}
			d_{i, n}\big|_{\rm{max}} =  \frac{1}{n} \left[ 
			\left( \sum_{k=1}^{n} \cos 2\alpha_k^\prime\right)^2
			+
			\left( \sum_{k=1}^{n} \sin 2\alpha_k^\prime\right)^2
			\right]^{1/2},
		\end{equation}
		
		where $\alpha_k^\prime$ was defined in Eq.~\eqref{eq:alphaprime} and corresponds to the value of the original position angle $\alpha_k$ after being transported in the $i-$th position. Following \cite{Jain2004}, we regard this maximal value as the measure of the dispersion of the $n$ sources and $\alpha_{\rm{max}}$ as their mean direction. The maximum value allowed for the dispersion is $d_{i, n}|_{\rm{max}} = 1$, corresponding to perfect alignment of the sources. The coordinate-invariance of the inner product (Eq. \ref{eq:innerproduct}) extends to the dispersion.
		
		
		For a sample of $N$ sources we fix a number of nearest neighbours $n$ and we derive the set of dispersions.
		
		\begin{align}
			\{d_{i, n}\big|_{\rm{max}}\} && i =1, \dots, N
		\end{align}
		
		For this set we define the following statistics
		
		\begin{align}
			S_{n} = \frac{1}{N} \sum_{i=1}^{N} d_{i, n}\big|_{\rm{max}},
			\label{eq:S}
		\end{align}

		corresponding to the mean dispersion. $S_n$ measures the average position angle dispersion of the sets containing every source and its $n$ neighbours.  
		If the condition $N \gg n \gg 1$ is satisfied, then $S_{n}$ is expected to be normally distributed. \citeauthor{Jain2004} reports the following form for its variance
		
		\begin{equation}
		\sigma_n^2 = \frac{0.33}{N},
		\label{eq:sigmaest}
		\end{equation} 
		
		where $N$ is the total number of sources in the sample. 
		The quantity $S_n$ can be employed for different values of $n$, although these different measurements are not independent. Because the dispersion $d_{i, n}$ is defined in Eq. \eqref{eq:d} as an average of the $n$ closest neighbours, the presence of a positive alignment for $n^\ast$ neighbours implies a preferential positive signal for every $n>n^\ast$.
		
		The deviation of the dispersion $d_{i, n}|_{\rm{max}}$ from its mean value is not normalized, but is found to be $\propto 1/\sqrt{n}$ \citep{Jain2004}. This is mirrored by $S_n$
		
		\begin{equation}
			S_n \propto \frac{1}{\sqrt{n}}
			\label{eq:Sest}
		\end{equation}
		
		To remove this spurious dependence, we will write the measurements of $S_n$ as one-tailed significance levels when considering multiple values of $n$
	
		\begin{equation}
			S.L. = 1-\Phi \left( \frac{S_n - \braket{S_n}_{MC}}{\sigma_n}\right),
			\label{eq:SL}
		\end{equation}
		
		where $\Phi$ is the cumulative normal distribution function and $\braket{S_n}_{MC}$ is the expected value for $S_n$ in absence of alignment, found through Monte Carlo simulations.
		We then employ the following approximate scale: $\log$ S.L. $< -3.5$, very strong alignment;  $-2.5>\log$ S.L. $> -3.5$, strong alignment; $-1.5>\log$ S.L. $> -2.5$ weak alignment.
		
		For every source (labelled by $i$) we define $\varphi_{i, n}$ as angular radius of the circle containing its $n$ neighbours. We can then define the following set:
		
		\begin{align}
		\{\varphi_{i, n}\} && i =1, \dots, N
		\label{eq:phii}
		\end{align}
		
		The distribution of this set provides information about what angular scale a particular $S_n$ probes. For our purposes we will refer to its median $\tilde{\varphi}(n)$ and the $68\%$ interval around it. 
		
		
		\subsection{Random Datasets}
		\label{sec:RandomDatasets}
		
		To estimate the uncertainties and the significance of a given measurement we use simulated data sets containing only noise. The random data sets ($1\,000$ in total) are generated by shuffling the position angles among different sources to ensure that every configuration is affected by the same position angle distribution and survey geometry. 
		
		For a binned or sampled quantity $W_k$ $k\in \{1\dots N_{bins}\}$ we estimate the covariance matrix as
		
		\begin{equation}
		\Sigma^2_{ij} = \braket{(W_i - \braket{W_i}_{MC} )\cdot (W_j - \braket{W_j}_{MC})}_{MC},
		\label{eq:cov}
		\end{equation}
		
		where all the averages are computed over multiple simulations.
		
		For a multivariate Gaussian random vectors $\bm{x}$ with expected mean $\bm{\mu}$ and covariance matrix $C$ of rank $k$, the $\chi^2$ test is generalized using the Mahalanobis distance squared
		
		\begin{equation}
		d^2 = (\bm{x} - \bm{\mu})^T C^{-1} (\bm{x} - \bm{\mu}),
		\end{equation}
		
		which is chi-square distributed with $k$ degrees of freedom. In our analysis, we define the components of vector $\bm{W}$ as the measurements of the statistics $W$ performed on different scales. We then use as Mahalanobis statistics the following expression:
		
		\begin{equation}
		d^2 = (\bm{W}-<\bm{W}>_{MC})^T (\Sigma^2)^{-1} (\bm{W}-<\bm{W}>_{MC})
		\label{eq:Maha}
		\end{equation}
		
		The alignment analyses performed by \cite{Jain2004, Hutsemekers2014, Taylor2016} are based on statistical tests similar to the position angle/polarization vector mean dispersion $S_n$ defined in Eq. \eqref{eq:S}.  None of the above references take covariance into account when estimating the significance level of the measured dispersion as a function of the angular scale. In this study, the Mahalanobis statistics measures deviation from the noise by taking covariance into account.