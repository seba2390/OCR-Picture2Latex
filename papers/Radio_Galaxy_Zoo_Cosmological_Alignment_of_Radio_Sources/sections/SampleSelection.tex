\section{Sample Selection}
	\label{sec:SS}
	The November 2015 alpha version of the Radio Galaxy Zoo consensus catalogue lists the properties of $85\,151$ radio sources distributed primarily over the footprint of two surveys: FIRST and Australia Telescope Large Area Survey (ATLAS) \citep{Norris2006}. The classification was performed by volunteers, who were presented with radio images from these surveys and the corresponding infrared fields observed by the Wide-field Infrared Survey Explorer \citep[WISE;][]{Wright2010}. They were then asked to match disconnected components corresponding to the same source and recognize the infrared counterpart. A more detailed description of the project is available in \cite{Banfield2015}. 

	The Radio Galaxy Zoo represents a natural choice for our statistical analysis. Whereas components belonging to the same source are usually recognized through self-matching (i.e., cross-matching the source catalogue with itself to identify sources at a certain distance from each other) or human selection, we rely on the additional information provided by human inspection to increase the reliability of the results. Furthermore, the $5\arcsec$ nominal resolution of the FIRST images implies a high number of resolved sources, for which a preferential direction can be defined. Lastly,  the survey covers an area of about $10\,000$ square degrees and allows us to infer general properties of the radio sky, instead of a local statistical anomaly.

	For our second sample, based on the TGSS Alternative Data Release 1, no human-made classification is available. In its place, we opt for automated self-matching. The TGSS ADR1 is based on an independent reprocessing of an original 150 MHz GMRT survey performed between 2010 and 2012 and the corresponding source catalogue, released in 2016, covers $99.5$  per cent of the sky north of $-53^\circ$ declination. A more detailed description is available in \cite{Intema2016}.
	
	\subsection{Radio Galaxy Zoo}
	
			\begin{figure}
				\centering
				\includegraphics[width=0.45\textwidth]{files/exall.pdf}
				\caption{FIRST image for a typical source with morphological features superimposed. The angular extent of the source is about $\mathbf{1\arcmin10\arcsec}$. The red boxes identify the components provided by the Radio Galaxy Zoo, with the crosses indicating surface brightness peaks. The blue ellipses have major and minor axes equal to the FWHM of the fitted Gaussian model in the FIRST catalogue, and the dots are their centres. The red and dashed blue lines are the results of the orthogonal distance regression for the dots and the crosses respectively. In this particular case two or more surface brightness peaks are present and the position angle is extracted from the slope of the red line (see text for more details).}
				\label{fig:exall}
			\end{figure}	
			
			We select extended sources from the Radio Galaxy Zoo consensus catalogue and extract an elongation direction for each of them.We describe this direction with a position angle, defined as the angle east of north in the range $[-\pi/2, +\pi/2]$ between the direction itself and the local meridian. 
			
			To perform the selection and constrain the orientation, we rely on the quantities contained in the November 2015 alpha version of the Radio Galaxy Zoo consensus catalog and, occasionally, on the official FIRST catalogue presented in \cite{Helfand2015b}, version \textsc{14dec17}. Figure~\ref{fig:exall} presents these quantities in graphic form. From the Radio Galaxy Zoo catalogue we extract the areas covered by components belonging to the same source and the peak positions of the source surface brightness contained in these regions (peaks hereafter). From the FIRST catalogue we extract the Gaussian model of the source brightness contained inside the same areas. 
			An additional quantity provided by the Radio Galaxy Zoo for every morphological classification is the consensus level. This is defined as the fraction of users who voted for the specific components configuration and, in this analysis, it is used to rank distinct classifications of the same object. 
			
			Depending on the available data, different procedures are employed to extract the position angle. We define three sub-samples:
			
			\begin{description}
			\item a)  If two or more surface brightness peaks are present for a given source, we define the position angle as the slope of the orthogonal distance linear regression of the peaks, weighted according to their flux densities. Around $80$ per cent of the selected sources belong to this category. An example of such a source is provided in Figure~\ref{fig:exall}.
			
			
			\item b) For sources with only one surface brightness peak in the Radio Galaxy Zoo catalogue, but multiple Gaussian models in the FIRST catalogue, we rely completely on the latter.
			This occurs when components are not seen as separated in the Radio Galaxy Zoo because of the particular automated choice of contour levels. For this sub-sample the centres of the FIRST ellipses, weighted by their integrated flux, are fitted.
			
			\item c) 	If only one surface brightness peak is detected and the FIRST catalogue recognizes only one source inside the single component radio galaxy, we rely on the Gaussian model of the FIRST catalogue and we define the direction as the position angle of the fitted ellipse. In this case, the source must comply with a total of four criteria.
			
			First, sources must meet the conditions required to be included in the Radio Galaxy Zoo sample and be presented to the volunteers. These are aimed at selecting resolved sources with a high signal-to-noise ratio:
			
			\begin{equation}
			\frac{S_{\rm{peak}}}{S_{\rm{int}}} < 1.0 - \left( \frac{0.1}{\log S_{\rm{peak}}}\right) \textrm{ and } SNR>10,
			\label{eq:cond1}
			\end{equation} 
			where $S_{\rm{peak}}$ is the peak brightness in mJy beam$^{-1}$, $S_{\rm{int}}$ is the integrated flux density of the source in mJy and $SNR$ is the signal-to-noise ratio \citep{Banfield2015}.
			
			Secondly, we introduce two additional criteria. The minor axis $m$ of the fitted elliptical Gaussian model should be larger than $2\arcsec$ and the deviation of the ratio between the major and minor axis $r$ from unity should be highly significant:
			
			
			\begin{equation}
			m > 2\arcsec
			\textrm{ and }
			r > 1 + 7\sigma_r.
			\label{eq:cond2}
			\end{equation}
			
			The error on the major and minor axis ratio is overestimated by the quadratic sum
			
			
			\begin{equation}
			\sigma_r = r\sqrt{\left(
				\frac{\sigma_m}{m}
				\right)^2
				+
				\left( 
				\frac{\sigma_m}{M}
				\right)^2},
			\end{equation}
			
			
			where $\sigma_m$ is the empirical uncertainty on both the fitted minor axis $m$ and major axis $M$. The four conditions, \eqref{eq:cond1} and \eqref{eq:cond2}, select extended sources for which an elongation is clearly recognizable.
			
			\end{description}
	
			When both multiple Gaussian models and multiple flux density peaks are available, we choose to prioritize the peaks over the centres. Figure~\ref{fig:exall} provides an example of how the difference between the two fitted position angles is usually small.
	
			
			\begin{figure}
				\centering
				\includegraphics[width=0.5\textwidth]{files/mindisttot.pdf}
				\caption{Distribution of the angular distance between a source in the Radio Galaxy Zoo sample and its closest neighbour, before and after filtering duplicates.}
				\label{fig:mindist}
			\end{figure} 
			
			The release of the Radio Galaxy Zoo consensus catalogue used here includes every classification performed by the volunteers. Because of this, a single source might appear multiple times with different classifications. 
            To filter these duplicate entries we focus our attention on all the recognized components. For every set of overlapping components, we filter out all of the sources they belong to, except for the one with highest consensus level. The effect of this selection process can be seen in Figure~\ref{fig:mindist}, where we plot the distribution of the distances between every source and its closest neighbour. While a natural amount of clustering is expected, we find that almost half of the sources have an extremely close neighbour -- a probable duplicate. After we apply our filter the peak around $0\farcs6$ disappears.
            
 	           
            A second systematic effect inherited from the Radio Galaxy Zoo is the quantisation of the peak positions. To clearly discern its importance, we limit our attention to the sources classified as containing only two peaks and we plot the differential right ascension and declination of every pair (Figure~\ref{fig:PP}). Discretisation is more noticeable in the vertical axis, but a $1\farcs4$ binning effect is visible in both directions. The presence of pixels is caused by numerical approximations in the implementation of the World Coordinates System (WCS).  
			In our analysis, this grid-like disposition of the peaks implies discrete values of the associated position angles. To obtain a continuous distribution of the angles, we smooth out the peak positions by adding a uniformly random value in the range $[-0\farcs7, +0\farcs7]$ to both coordinates before performing the linear regression. This process pushes the influence of the effect to sub-pixel scales, eliminating its impact on the present study. However, further investigation is needed to constrain its causes. 

		 	\begin{figure}
		 		\centering
		 		\includegraphics[width=0.5\textwidth]{files/PP.pdf}
		 		\caption{Relative peak positions for entries classified as containing two peaks. Discretization is evident in the collapsed distributions.}
		 		
		 		\label{fig:PP}
		 	\end{figure}
            For the sake of consistency, the final sample presented in Figure~\ref{fig:RGZPAdist} excludes ATLAS sources and it is limited only to FIRST sources. For the same reason, we also exclude every source positioned above RA $20$ hr and below $4$ hr, since half of the observations in this region were performed after the observing array transitioned to the new JVLA configuration \citep{Helfand2015b}. 
            
			Finally, notice how the original Radio Galaxy Zoo selection in Eq.~\eqref{eq:cond1} does not include an explicit cut for artefacts. During the first run of the Radio Galaxy Zoo classification, the volunteers were presented with $3\arcmin \times3\arcmin$ fields. This corresponds to a maximum distance of $ 3\arcmin \sqrt{2} \approx 4\arcmin 12\arcsec$ between two components. To quantify the contamination from artefacts in our sample, we make use of the column $P(S)$ of the official FIRST catalogue, which indicates the probability of a source to be a sidelobe. We cross-matched our selection with the FIRST catalogue using a search radius of $4\arcmin 12\arcsec$ and we verified that 134 selected sources are part of a field containing possible sidelobes satisfying the condition $P(S)>0.1$. In principle, these artefacts might be recognized as components and influence the value of the position angles. Because of this, we exclude sources with $P(S)>0.1$ from our final Radio Galaxy Zoo sample.
            
            In Figure~\ref{fig:RGZPAdist} we plot the final distribution of the extracted position angles, together with the distributions for the three classes of sources. While we would expect these to be uniform, three peaks are visible around $30^\circ$, $-30^\circ$ and $90^\circ$. In these three directions we recognize the typical pattern that results from the three arms of the observing radio interferometer --- the Very Large Array (VLA). The same effect is visible in the FIRST images and is discussed in \cite{Helfand2015b}, where a three-directional pattern is present in the distribution of the sidelobes around bright sources. The existence of preferential angles may be related to the brightness of the weaker components, although a more detailed analysis would be required to quantify this effect. This will not affect our analysis as long as the effects are non-local.

			\begin{figure}
				\centering
				\includegraphics[width=0.45\textwidth]{files/RGZPATOT.pdf}
				\caption{Position angle distribution of the Radio Galaxy Zoo selection. On top of the total distribution (topmost histogram) the plot contains the distributions of the three sub-samples. From top to bottom: (a) in grey, (b) in red, and (c) in blue. A trimodal systematic effect is visible in the first two.}
				
				\label{fig:RGZPAdist}
			\end{figure}

            A similar pattern is discussed also in other analyses \citep[e.g.,][]{Chang2004, White2007, Demetroullas2015} based on the FIRST survey, where the effect is recognized as non position-dependent. Snapshot surveys are commonly affected by an anisotropic point spread function (PSF) and the connection to the interferometer geometry suggests this origin.
            \cite{Helfand2015b} underlines that particular care was taken in ensuring a constant PSF throughout the different observation epochs of FIRST. In particular, since the hour angle of observation affects the orientation of the pattern in the cleaned images, $90\%$ of the observations were acquired within $1.4$ hr of the local meridian.	
			
			The non-locality of the effect is verified by partitioning the data by both right ascension and declination in four equally populated quadrants. Pairwise, the four position angle distribution are found to be consistent with each other using two-sample Kolmogorov-Smirnov tests. 
            
            This first position angle catalogue contains $30\,059$ sources distributed over an area of about $7\,000$ square degrees, resulting in a number density $\sim4$ deg$^{-2}$.


		\subsection{TGSS Alternative Data Release}
		
			As opposed to the Radio Galaxy Zoo sample, this second position angle sample is based on the product of an automated source extractor. The nominal resolution of $25\arcsec$ for the TGSS images implies a lower number of extended sources with significant elongation compared to FIRST. However, the relatively steep spectrum of radio galaxy lobes and the sensitivity to extended sources of the GMRT allow TGSS to trace the lobes better than FIRST.  Hence, we focus our attention on the identification of double-lobed sources. 
			In Figure~\ref{fig:mindistTGSS} we plot the distance between each entry in the TGSS catalogue and its closest neighbour. The rightmost peak is due to the distribution of uncorrelated radio sources, while the lower peak on the left is caused by multi-component sources. The plot suggests an average distance of $1\arcmin$ between the components of a source of the latter type. A peak around the angular scale of $1\arcmin$ is not present in the Radio Galaxy Zoo catalogue because the pairing was already performed by the volunteers during the classification process.

			\begin{figure}
				\centering
				\includegraphics[width=0.5\textwidth]{files/mindistTGSS.pdf}
				\caption{Distribution of the angular distance between a source in the TGSS catalogue and its closest neighbour. The dashed red line marks the value $1\arcmin12\arcsec$.}
				
				\label{fig:mindistTGSS}
			\end{figure}
			
			We select radio galaxy candidates by self-matching the catalogue with a search radius $1\arcmin12\arcsec$ and imposing a maximum ratio of $10$ between the total fluxes of the two components \citep{VanVelzen2014}.
			To be part of the final sample both components of the pair need to satisfy additional constraints: (1) isolated (i.e., matched only to each other) (2) $SNR>10$. The position angle is then simply that of the line connecting the two components. The search radius we chose corresponds to the local minimum marked in Fig.~\ref{fig:mindistTGSS}. A larger value would introduce an artificial contamination in our double-lobed source catalogue, while a lower value would mean losing part of the genuine sources.
			
			We decide to limit our sample to a portion of the northern hemisphere to minimize the effects of an anisotropic PSF. \cite{Intema2016} reports the synthesized beam to be circular for pointings at declination higher than the GMRT latitude --- about $19^\circ$. Even between declinations of $10^\circ$ and $19^\circ$ the beam is still circular to within $1\%$. Therefore, our final TGSS sample includes only sources with declination above $10^\circ$.
			
			Figure~\ref{fig:TGSSPAdist} shows the position angle distribution of the final TGSS sample. This second position angle catalogue contains $11\,674$ sources distributed over an area of about $17\,000$ square degrees, resulting in a number density $\sim0.7$ deg$^{-2}$. We notice that unlike for the FIRST survey, no particular care was taken with respect to the PSF and its consistency throughout different pointings. However, the  complex geometry of the interferometer and longer integration times compared to FIRST result in a PSF less prone to systematic effects. 
			Table~\ref{tab:surveys} compares the different surveys and samples featured in this section. The difference between the number of sources in the two catalogues produced in this section is due to the different nature of the original surveys and the source selection process. While $85\%$ of the sources in the RGZ sample have size larger than the TGSS resolution ($25\arcsec$), only $55\%$ of them are larger this threshold and have exactly two surface brightness peaks.
			
			We can use the RGZ catalogue to predict the size of the TGSS one. If we account for the different frequencies ($1.4$ GHz for FIRST and $150$ MHz for TGSS) by adopting a nominal spectral index equal to $0.9$ \citep{Vollmer2010} and keeping in mind the sky coverage and angular resolution differences, we find that about $10^4$ sources are expected to be selected by our algorithm. This number is in line with the $11\,674$ sources found in our selection.
			
			\begin{figure}
				\centering
				\includegraphics[width=0.45\textwidth]{files/TGSSPATOT.pdf}
				\caption{Position angle distribution of the TGSS selection. Obvious systematic effects are not present.}
				
				\label{fig:TGSSPAdist}
			\end{figure}


\begin{table*}
\caption{Comparison between the different samples and source catalogs discussed in this paper.}
\label{tab:surveys}
\begin{tabular}{lccccccc}
\hline
Name & Frequency & Median RMS & SNR & Number of & Minimum & Sky & Median Redshift \\
     &           & Noise      & Threshold  & Sources   &  Resolution & Fraction & $68\%$ interval \\
     &           &[mJy beam$^{-1}$] &       &           &            &          &                  \\ 

\hline
				FIRST\,$^a$ &
				$1.4$ GHz & 
				$0.15$& 
				$5$&
				$946\,432$& 
				$5\arcsec\times5\arcsec$& 
				$26\%$&
				$2.2\pm0.9$\,$^b$ 
				\\ 
				Radio Galaxy Zoo\,$^c$  & 
				$1.4$ GHz & 
				$0.15$ & 
				$10$& 
				$82\,187$&  
				$5\arcsec\times5\arcsec$& 
				$22\%$& 
				$0.47_{-0.15}^{+0.21}$\,$^d$ 
				\\
				Radio Galaxy Zoo processed\,$^e$ & 
				$1.4$ GHz &  
				$0.15$& 
				$10$& 
				$30\,059$  &
				$5\arcsec\times5\arcsec$& 
				$19\%$& 
				$0.47_{-0.15}^{+0.20}$\,$^d$ 
				\\
				TGSS\,$^f$& 
				$150$ MHz & 
				$3.5$& 
				$7$& 
				$623\,604$ & 
				$25\arcsec\times25\arcsec$& 
				$90\%$& 
				$-$
				\\
				TGSS processed\,$^e$ & 
				$150$ MHz & 
				$3.5$& 
				$10$& 
				$11\,674$ & 
				$25\arcsec\times25\arcsec$& 
				$42\%$& 
				$-$ \\
\hline
\multicolumn{8}{l}{ $^a$ \cite{Helfand2015b}} \\
\multicolumn{8}{l}{ $^b$ Mean redshift with $68\%$ confidence levels from \cite{Chang2004}}	\\
\multicolumn{8}{l}{ $^c$ \cite{Banfield2015}} \\
\multicolumn{8}{l}{ $^d$ Only $30\%$ of the sample has a human-matched optical counterpart with known redshift} \\
\multicolumn{8}{l}{ $^e$ The selection process, aimed at selecting resolved sources to use in this study, is detailed in Section~\ref{sec:SS}} \\
\multicolumn{8}{l}{ $^f$ \cite{Intema2016}} \\
\end{tabular}
\end{table*}