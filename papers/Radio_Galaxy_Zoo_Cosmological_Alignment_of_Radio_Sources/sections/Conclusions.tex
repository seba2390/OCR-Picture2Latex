\section{Conclusions}
	We constructed two samples of radio galaxies to search for the signature of source alignment: one based on the Radio Galaxy Zoo November 2015 catalogue, and the other on the TGSS Alternative Data Release 1 catalogue.
	
	The RGZ sample is formed by sources present in the FIRST survey and classified by volunteers participating in the Radio Galaxy Zoo collaboration. In this paper, we report marginal evidence of local alignment among radio sources within this sample. The signal is inconsistent with the noise  with a significance level $> 2\sigma$. Its main feature is a  $3.2\sigma$ minimum of the significance level on angular scales between $1.5^\circ$ and $2^\circ$. Assuming a flat $\Lambda$CDM Cosmology and cosmological parameters $\Omega_m=0.31, \Omega_\Lambda = 0.69$, this roughly corresponds to a physical scale in the range $[19, 38]$ Mpc.
	
	By number of sources, RGZ is about six hundred times larger than the set considered by \cite{Taylor2016} and about one hundred times larger than the largest set of quasars considered for the alignment study of quasar polarization vectors \citep{Pelgrims2014}. More detailed investigations of other, even larger samples, with different selection biases (see  Sec.~\ref{sec:SS}) or choices for the scales of interest (see Sec.~\ref{sec:RES}), would be useful.
	
	
	The TGSS sample was obtained from a reprocessed GMRT survey. In this case, no evidence of alignment is found. However, its lower source density means that even if a signal was present, it would not be significant. 
		
	The alignment of astronomical sources has frequently been a topic of interest. Optical galaxies have usually dominated the conversation \citep{Joachimi2015}, which in recent years has seen a resurgence in popularity due to the identification of galaxy alignment as a systematic effect for weak lensing \citep{Kirk2015}. If the alignment of radio galaxies is proved to be connected to the tidally induced alignment of their optical counterparts, radio observations might be used to constrain the intrinsic orientation of galaxies. 
	
	An alternative hypothesis might revolve around the origin of radio-loud AGNs, believed to be associated with galaxy mergers \citep[see, for example,][]{Hardcastle2007, Croton2006, Chiaberge2015}. If mergers play a role in spinning up the supermassive black hole or orienting the accretion disk emitting the jets, a preferential merger direction along the filaments of the large-scale structure could result in the alignment of the jets.
	
	With the new generation of high resolution radio interferometers like the Low Frequency Array (LOFAR) and the Square Kilometre Array (SKA), the cosmological prospects of radio astronomy will be expanded \citep[e.g.,][]{Blake2004a, VanHaarlem2013}. We expect the study of alignment to be part of these efforts. 