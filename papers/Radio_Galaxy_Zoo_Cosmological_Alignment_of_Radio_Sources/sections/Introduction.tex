\section{Introduction}

	In the last two decades, the mutual alignment of optical linear polarizations of quasars over cosmological scales (comoving distance $\geq 100$ $h^{-1}$ Mpc) has been reported \citep{Hutsemekers1998, Lamy2001, Cabanac2005}. Since quasars are rare and non-uniformly distributed, ad hoc statistical tools have been developed over the years to study the phenomenon \citep{Jain2004, Shurtleff2013, Pelgrims2014}. Since the correlation between AGN optical polarization vectors and structural axes has been observed \citep[e.g.,][]{Lyutikov2005, Battye2009}, the coherence of the polarization vectors could be interpreted as an alignment of the nuclei themselves or alignment with respect to an underlying large-scale structure. Confirmation of this came from \cite{Hutsemekers2014}, who considered quasars known to be part of quasar groups and detected an alignment of the polarization vectors either parallel or perpendicular to the large-scale structure they belong to.
		
	Both observational results \citep[e.g.,][]{Tempel2013, Zhang2013, Hirv2016} and tidal torque analytical models \citep[e.g.,][]{Codis2015, Lee2004} suggest the alignment of galaxy spins with respect to the filaments and walls of the large-scale structure. The geometry of the cosmic web influences the spin and shape of galaxies by imparting tidal torques on collapsing proto-halos. The same mechanism might be behind both the alignment of galactic spins and polarization vectors, but the topic is still under discussion \citep{Hutsemekers2014}. The main caveats are the peculiar cosmic evolution of quasars, dominated by feedback, and the implications that an alignment on such large-scales would have for the cosmological principle \citep[see, for example,][]{Zhao2016}.
		
	\cite{Taylor2016} reported local alignment (below the $1^\circ$ scale) of radio galaxies in the ELAIS N1 field observed with the Giant Metrewave Radio Telescope (GMRT) at $610$ MHz. 
	
	Despite the known trend of radio galaxy major axes to be aligned with the optical minor axis rather than the optical
	major axis \citep[e.g.,][]{Andernach1995, Battye2009, Kaviraj2015}, the correlation between the large-scale angular momentum of the galaxy and the angular momentum axis of the material accreting towards the AGN (traced by the jets) is disputed \citep{Hopkins2012}. This makes the tidal torque interpretation of the radio jets alignment nebulous at best. On the other hand, modelling the formation of dominant cluster galaxies suggests that the spin of the black holes powering AGNs is affected by the galactic accretion history and therefore might be aligned with the surrounding large-scale structure \citep{West1994}.
	
	In this work, we attempt to corroborate and extend the results obtained in \cite{Taylor2016} by studying the alignment of radio sources in the maps of the radio sky provided by the two surveys: Faint Images of the Radio Sky at Twenty-centimeters \citep[FIRST;][]{Becker1995a} and TIFR GMRT Sky Survey\footnote{Website: http://tgssadr.strw.leidenuniv.nl/} \citep[TGSS;][]{Intema2016}. In Section~\ref{sec:SS} we construct two catalogues that contain the orientations and coordinates of resolved radio sources. In Section~\ref{sec:SA} we present the statistical instruments we make use of, based on those developed by \cite{Bietenholz1986} and \cite{Jain2004} for the study of quasar optical polarizations. In Section~\ref{sec:RES} we discuss the results of the analysis.
	
	In appendix~\ref{sec:shear} a more sophisticated approach to the study of alignment is presented. The statistics used in there do not however return any significant result.