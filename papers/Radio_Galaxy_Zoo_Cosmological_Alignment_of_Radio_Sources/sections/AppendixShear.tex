	\section{Position Angle as Shear}
		\label{sec:shear}
		In this Appendix, we focus on an approach to the study of the position angles based on an alternative formalism. The study of other directional quantities over large scales through the use of spin-2 spherical harmonics is well established. Examples of such quantities are the polarization $P$ of the CMB or the cosmic shear field $\gamma$ \citep[e.g.,][]{Collaboration2015, Hikage2011}. However, in our attempts, the detailed properties of the position angle datasets forced a sampling of the correlation functions and power spectra that did not allow us to resolve features like the minimum in Fig.~\ref{fig:Sn}.  In particular, the main complications are the partial sky-coverage, the low source density and the predisposition to systematic effects of interferometric measurements.
		
		Although the products presented in this appendix are inconclusive, we describe here our implementation of the cosmic shear statistics, so that it can be applied when suitable samples will become available.
		
		Cosmic shear is usually detected through the analysis of the spin-2 field
		
		\begin{equation}
			\gamma = \gamma_1 + i\gamma_2,
			\label{eq:gamma}
		\end{equation}
		
		where $\gamma_1, \gamma_2$ are defined on a local Cartesian reference frame. Under rotation of an angle $\Phi$ the field transforms as $\gamma\to \gamma~e^{2i\Phi}$. The shear is usually estimated as the ensemble average of galaxy ellipticities $\varepsilon$ \citep{Kirk2015}
		
		\begin{equation}
			\varepsilon =  \frac{1-q}{1+q} (\cos 2\alpha_p + i \sin 2\alpha_p)
			\label{eq:epsi}
		\end{equation}
		\begin{equation}
			\gamma = \braket{\varepsilon}
		\end{equation}
		
		In this definition, $\alpha_p$ is the major axis position angle of the optical galaxy and $q$ is the ratio between the major and minor axes. We define the tangential and cross-component ellipticity $\varepsilon_t$ and $\varepsilon_\times$ with respect to a direction as the projection of the ellipticity in the two  $+/\times$ components: (1) parallel or perpendicular to it (2) oriented at $45^\circ$ or $-45^\circ$. For a direction defined by the polar angle $\Psi$
		
		\begin{gather}
			\epsilon_t = -\operatorname{Re}\{e^{-2i\Psi}\epsilon \}
			\label{eq:et}
			\\
			\epsilon_\times = -\operatorname{Im}\{e^{-2i\Psi}\epsilon\}
			\label{eq:ecross}
		\end{gather}
		
		In our sign convention, a positive $\varepsilon_t$ corresponds to tangential alignment, i.e., the position angle $\alpha_p$ and the direction $\Psi$ are parallel, while a negative value corresponds to radial alignment, i.e., the two are perpendicular \citep{Kilbinger2015}. 
		
				
		The literature contains multiple statistics involving the shear field. In particular, we focus on those described in  \cite{Schneider2002}, \cite{Eifler2010} and implemented by the software \textsc{treecorr}\footnote{\url{https://github.com/rmjarvis/TreeCorr}} \citep{Jarvis2004}. 
		
		When evaluating a two point correlation function, the two components $\gamma_t$ and $\gamma_\times$ are defined with respect to the direction connecting the sources. These components are commonly estimated by neglecting both the curvature of the sphere and the parallel transport operation described in Section~\ref{sec:PT}. Because of this, we limit our analysis in this Section to distances smaller than $5^\circ$, corresponding to about $0.1$ radians.
		
		We introduce the two-point correlation functions
		
		\begin{gather}
			\xi_{tt}(\varphi) = \braket{\gamma_t \gamma_t}
			\label{eq:xitt}
			\\
			\xi_{\times \times}(\varphi) = \braket{\gamma_\times \gamma_\times}
			\label{eq:xicc}
			\\
			\xi_+ (\varphi)= \braket{\gamma_t \gamma_t} + \braket{\gamma_\times \gamma_\times} 
			\\
			\xi_-  (\varphi)= \braket{\gamma_t \gamma_t} - \braket{\gamma_\times \gamma_\times}
		\end{gather}
		
		where the averages are computed over every possible pair of sources with angular distance $\varphi$. The tangential and cross-component shear are defined as in Eq. \eqref{eq:et}, \eqref{eq:ecross}. The two correlation functions $\xi_{tt}$ and $\xi_{\times \times}$ distinguish between different shear configurations, according to the provided definitions of $\gamma_t$ and $\gamma_\times$.
		Furthermore, we define $\overline{\gamma}(\varphi)$ as the mean shear inside a circular aperture of radius $\varphi$. The variance of this quantity can then be estimated directly from the correlation function $\xi_+$
		
		\begin{gather}
			\braket{|\overline{\gamma}|^2}(\varphi) = \int \frac{d\vartheta\vartheta}{2\varphi^2} 
				\xi_+(\vartheta) S_+ \left( \frac{\vartheta}{\varphi}\right)
			\label{eq:Gsq}
		\end{gather}
		
		The definition of the weight function $S_+$ and a more detailed introduction to the top-hat shear dispersion are given by \cite{Schneider2002}.
		
		
		Using the representation introduced in Eq. \eqref{eq:epsi}, the position angle $\alpha$ can be written as
		
		\begin{equation}
			\gamma^\alpha = \cos 2\alpha + i \sin 2 \alpha
		\end{equation} 			
		
		Under a rotation of an angle $\Phi$ the quantity $\gamma^\alpha$ behaves exactly like the shear field, $\gamma^\alpha \to \gamma^\alpha~e^{2i\Phi}$. This justifies the extension to $\gamma^\alpha$ of the statistics defined for $\gamma$. Since we want to study the alignment configuration of the position angles, we should point out that no averaging is involved. In our analysis $\gamma^\alpha$ takes the place of the shear field $\gamma$ and not of the ellipticity $\varepsilon$.    
		
		In the presence of a global systematic effect we rewrite the correlation functions \eqref{eq:xitt} and \eqref{eq:xicc} as
		
		\begin{gather}
			\xi_{tt} (\theta) = \braket{\gamma^\alpha_t \gamma^\alpha_t} - \xi^n_{tt}
			\\
			\xi_{\times \times} (\theta) = \braket{\gamma^\alpha_\times \gamma^\alpha_\times}
			-
			\xi^n_{\times \times},
		\end{gather}
		
		where we subtracted a noise bias, to be estimated through simulated random data sets containing only the noise. The expression for the estimator \eqref{eq:Gsq} must be computed from these unbiased correlation functions.
		
		We do not assume any particular model for our analysis and we set as our primary objective the detection of a positive correlation. In its absence we expect the two-point correlation functions and the dispersion to be consistent with the noise on every scale $\varphi$.
		
		The function $\braket{|\overline{\gamma^\alpha}|^2}(\varphi)$ is closely related to $S_n$ (Eq. \eqref{eq:S}) since both of them estimate the average dispersion (or dispersion squared) of the position angles. The first one considers spherical caps of constant aperture radius $\varphi$, while the second considers caps with a constant number of sources $n$. The dispersion $\braket{|\overline{\gamma^\alpha}|^2}(\varphi)$ has the advantage of probing precise angular scales, but for non-uniformly distributed samples its value can be easily skewed by the sources in low density regions. Another drawback, due to our chosen implementation, is the lack of parallel transport in its computation.  
	
		
		\subsection{Products}
		
		In Fig.~\ref{fig:RGZweak} and~\ref{fig:TGSSweak} we plot the statistics presented in Eq. \eqref{eq:xitt}, \eqref{eq:xicc} and \eqref{eq:Gsq} for the Radio Galaxy Zoo and TGSS samples. The noise bias has already been subtracted. The covariance matrices are generated using the method described in Section~\ref{sec:RandomDatasets}.
		
		Since the diagonal terms in the covariance matrix \eqref{eq:cov} are two orders of magnitude higher than the non-diagonal terms, we can confirm that the measurements of the statistics $\xi_{tt}$ and $\xi_{\times \times}$ for different angular scales are in fact independent. The same is not true for the dispersion $\braket{|\overline{\gamma^\alpha}|^2}$. The reason for this is the same as the one discussed in Section~\ref{sec:S} for the statistics $S_n$. 	
		
		
		The correlation functions $\xi_{tt}(\theta)$ and $\xi_{\times \times}(\theta)$ are consistent with normally distributed noise. This result was checked using common statistical tests: (1) Shapiro-Wilk (2) $\chi^2$ (3) Anderson-Darling (4) two-tailed Kolmogorov-Smirnov. All of them returned p-values $> 0.05$. For the two $\braket{|\overline{\gamma^\alpha}|^2}$ we obtain the Mahalanobis distances $d^2 = 17.28$ and $d^2 = 12.05$. Given the number of degrees of freedom ($k=12$), both correspond to p-values $>0.05$, meaning that these results are also consistent with the noise. 
		
		Nothing conclusive about the alignment configuration can be stated, since both $\xi_{tt}$ and $\xi_{\times \times}$ are consistent with zero. The dispersion $\braket{|\overline{\gamma^\alpha}|^2}$ is also found to be consistent with the noise. This is not unexpected, since the estimator in Eq. \eqref{eq:Gsq} is simply a convolution of $\xi_{+} = \xi_{tt} + \xi_{\times \times}$ and a weight function. If $\xi_{+}$ is found to be largely consistent with zero, the same should be true for $\braket{|\overline{\gamma^\alpha}|^2}$.
		
		
		\begin{figure}
			\includegraphics[width=0.45\textwidth]{files/RGZweak.pdf}
			\caption{Weak lensing statistics for the Radio Galaxy Zoo sample: the two point correlation functions $\xi_{tt}(\varphi)$, $\xi_{\times \times}(\varphi)$ as a function of the distance $\varphi$ and the top-hat shear dispersion $\braket{|\overline{\gamma^\alpha}|^2}(\varphi)$ as a function of the aperture radius $\varphi$.}
			
			\label{fig:RGZweak}
		\end{figure}
		
		\begin{figure}
			\includegraphics[width=0.45\textwidth]{files/TGSSweak.pdf}
			\caption{Weak lensing statistics for the TGSS sample: the two point correlation functions $\xi_{tt}(\varphi)$, $\xi_{\times \times}(\varphi)$ as a function of the distance $\varphi$ and the top-hat shear dispersion $\braket{|\overline{\gamma^\alpha}|^2}(\varphi)$ as a function of the aperture radius $\varphi$.}
			
			\label{fig:TGSSweak}
		\end{figure}		
	
			Finally, the down-crossing of $\xi_{tt}$ around the angular scale of $3\deg$ seems to suggest a change in the configuration of the alignment. The limited number of data points and the overall consistency with zero of the correlation function do not allow for a conclusive statement. However, assuming the downcrossing to be a feature, we can assign a significance to this observation. The probability of obtaining $8$ consecutive positive datapoints is found to be less than $0.005$. 