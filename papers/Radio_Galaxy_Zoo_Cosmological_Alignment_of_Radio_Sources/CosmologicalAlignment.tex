% mnras_template.tex
%
% LaTeX template for creating an MNRAS paper
%
% v3.0 released 14 May 2015
% (version numbers match those of mnras.cls)
%
% Copyright (C) Royal Astronomical Society 2015
% Authors:
% Keith T. Smith (Royal Astronomical Society)

% Change log
%
% v3.0 May 2015
%    Renamed to match the new package name
%    Version number matches mnras.cls
%    A few minor tweaks to wording
% v1.0 September 2013
%    Beta testing only - never publicly released
%    First version: a simple (ish) template for creating an MNRAS paper

%%%%%%%%%%%%%%%%%%%%%%%%%%%%%%%%%%%%%%%%%%%%%%%%%%
% Basic setup. Most papers should leave these options alone.
\documentclass[fleqn,usenatbib]{mnras}

% MNRAS is set in Times font. If you don't have this installed (most LaTeX
% installations will be fine) or prefer the old Computer Modern fonts, comment
% out the following line
%\usepackage{newtxtext,newtxmath}
% Depending on your LaTeX fonts installation, you might get better results with one of these:
%\usepackage{mathptmx}
%\usepackage{txfonts}

% Use vector fonts, so it zooms properly in on-screen viewing software
% Don't change these lines unless you know what you are doing
\usepackage[T1]{fontenc}
\usepackage{ae,aecompl}


%%%%% AUTHORS - PLACE YOUR OWN PACKAGES HERE %%%%%

% Only include extra packages if you really need them. Common packages are:
\usepackage{graphicx}	% Including figure files
\usepackage{amsmath}	% Advanced maths commands
\usepackage{amssymb}	% Extra maths symbols
\usepackage{multicol}   % Multicolumns
\usepackage{bm}         % Vectors

%%%%%%%%%%%%%%%%%%%%%%%%%%%%%%%%%%%%%%%%%%%%%%%%%%

%%%%% AUTHORS - PLACE YOUR OWN COMMANDS HERE %%%%%

% Please keep new commands to a minimum, and use \newcommand not \def to avoid
% overwriting existing commands. Example:
%\newcommand{\pcm}{\,cm$^{-2}$}	% per cm-squared
\newcommand{\braket}[1]{\left\langle #1 \right\rangle}

%%%%%%%%%%%%%%%%%%%%%%%%%%%%%%%%%%%%%%%%%%%%%%%%%%

%%%%%%%%%%%%%%%%%%% TITLE PAGE %%%%%%%%%%%%%%%%%%%

% Title of the paper, and the short title which is used in the headers.
% Keep the title short and informative.
\title[Cosmological Alignment of Radio Sources]{Radio Galaxy Zoo: Cosmological Alignment of Radio Sources}

% The list of authors, and the short list which is used in the headers.
% If you need two or more lines of authors, add an extra line using \newauthor
\author[O. Contigiani et al.]{O. Contigiani,$^{1}$\thanks{E-mail: contigiani@strw.leidenuniv.nl}
F. de Gasperin,$^{1}$
G. K. Miley,$^{1}$ 
L. Rudnick$^{2}$,
H. Andernach,$^{3}$ \newauthor
J. K. Banfield,$^{4, 6}$ 
A. D. Kapi\'nska,$^{5, 6}$
S. S. Shabala$^{7}$,
O. I. Wong$^{5, 6}$
\\
% List of institutions
$^{1}$ Leiden  Observatory, Leiden  University,  P.O. Box 9513, 2300 RA,  Leiden,  the Netherlands
\\
$^{2}$ Minnesota Institute for Astrophysics, University of Minnesota,
116 Church St. SE, Minneapolis, MN 55455
\\
$^{3}$ Departamento de Astronom\'ia, DCNE, Universidad de Guanajuato, Apdo.
Postal 144, CP 36000, Guanajuato, Gto., Mexico
\\
$^{4}$ Research School of Astronomy and Astrophysics, Australian National University, Weston Creek, ACT 2611, Australia
\\
$^{5}$ International Centre for Radio Astronomy Research (ICRAR), The University of Western Australia, 
\\
M468, 35 Stirling Hwy, Crawley WA 6009, Australia
\\
$^{6}$ ARC Centre of Excellence for All-sky Astrophysics (CAASTRO), Australia
\\
$^{7}$ School of Mathematics \& Physics, University of Tasmania, Private Bag 37, Hobart, Tasmania 7001, Australia
}

% These dates will be filled out by the publisher
\date{Accepted XXX. Received YYY; in original form ZZZ}

% Enter the current year, for the copyright statements etc.
\pubyear{2017}

% Don't change these lines
\begin{document}
\label{firstpage}
\pagerange{\pageref{firstpage}--\pageref{lastpage}}
\maketitle

% Abstract of the paper
\begin{abstract}
We study the mutual alignment of radio sources within two surveys, FIRST and TGSS. This is done by producing two position angle catalogues containing the preferential directions of respectively $30\,059$ and $11\,674$ extended sources distributed over more than $7\,000$ and $17\,000$ square degrees. 
The identification of the sources in the FIRST sample was performed in advance by volunteers of the Radio Galaxy Zoo project, while for the TGSS sample it is the result of an automated process presented here.
After taking into account systematic effects, marginal evidence of a local alignment on scales smaller than $2.5^\circ$ is found in the FIRST sample. The probability of this happening by chance is found to be less than $2$ per cent. Further study suggests that on scales up to $1.5^\circ$ the alignment is maximal. For one third of the sources, the Radio Galaxy Zoo volunteers identified an optical counterpart. Assuming a flat $\Lambda$CDM cosmology with $\Omega_m = 0.31, \Omega_\Lambda = 0.69$, we convert the maximum angular scale on which alignment is seen into a physical scale in the range $[19, 38]$ Mpc $h_{70}^{-1}$. This result supports recent evidence reported by Taylor and Jagannathan of radio jet alignment in the $1.4$ deg$^2$ ELAIS N1 field observed with the Giant Metrewave Radio Telescope.  The TGSS sample is found to be too sparsely populated to manifest a similar signal.
\end{abstract}

% Select between one and six entries from the list of approved keywords.
% Don't make up new ones.
\begin{keywords}
galaxies: statistics -- galaxies: jets -- radio continuum: galaxies -- cosmology: observations -- large-scale structure of Universe
\end{keywords}

%%%%%%%%%%%%%%%%%%%%%%%%%%%%%%%%%%%%%%%%%%%%%%%%%%

%%%%%%%%%%%%%%%%% BODY OF PAPER %%%%%%%%%%%%%%%%%%

\IEEEraisesectionheading{\section{Introduction}}

\IEEEPARstart{V}{ision} system is studied in orthogonal disciplines spanning from neurophysiology and psychophysics to computer science all with uniform objective: understand the vision system and develop it into an integrated theory of vision. In general, vision or visual perception is the ability of information acquisition from environment, and it's interpretation. According to Gestalt theory, visual elements are perceived as patterns of wholes rather than the sum of constituent parts~\cite{koffka2013principles}. The Gestalt theory through \textit{emergence}, \textit{invariance}, \textit{multistability}, and \textit{reification} properties (aka Gestalt principles), describes how vision recognizes an object as a \textit{whole} from constituent parts. There is an increasing interested to model the cognitive aptitude of visual perception; however, the process is challenging. In the following, a challenge (as an example) per object and motion perception is discussed. 



\subsection{Why do things look as they do?}
In addition to Gestalt principles, an object is characterized with its spatial parameters and material properties. Despite of the novel approaches proposed for material recognition (e.g.,~\cite{sharan2013recognizing}), objects tend to get the attention. Leveraging on an object's spatial properties, material, illumination, and background; the mapping from real world 3D patterns (distal stimulus) to 2D patterns onto retina (proximal stimulus) is many-to-one non-uniquely-invertible mapping~\cite{dicarlo2007untangling,horn1986robot}. There have been novel biology-driven studies for constructing computational models to emulate anatomy and physiology of the brain for real world object recognition (e.g.,~\cite{lowe2004distinctive,serre2007robust,zhang2006svm}), and some studies lead to impressive accuracy. For instance, testing such computational models on gold standard controlled shape sets such as Caltech101 and Caltech256, some methods resulted $<$60\% true-positives~\cite{zhang2006svm,lazebnik2006beyond,mutch2006multiclass,wang2006using}. However, Pinto et al.~\cite{pinto2008real} raised a caution against the pervasiveness of such shape sets by highlighting the unsystematic variations in objects features such as spatial aspects, both between and within object categories. For instance, using a V1-like model (a neuroscientist's null model) with two categories of systematically variant objects, a rapid derogate of performance to 50\% (chance level) is observed~\cite{zhang2006svm}. This observation accentuates the challenges that the infinite number of 2D shapes casted on retina from 3D objects introduces to object recognition. 

Material recognition of an object requires in-depth features to be determined. A mineralogist may describe the luster (i.e., optical quality of the surface) with a vocabulary like greasy, pearly, vitreous, resinous or submetallic; he may describe rocks and minerals with their typical forms such as acicular, dendritic, porous, nodular, or oolitic. We perceive materials from early age even though many of us lack such a rich visual vocabulary as formalized as the mineralogists~\cite{adelson2001seeing}. However, methodizing material perception can be far from trivial. For instance, consider a chrome sphere with every pixel having a correspondence in the environment; hence, the material of the sphere is hidden and shall be inferred implicitly~\cite{shafer2000color,adelson2001seeing}. Therefore, considering object material, object recognition requires surface reflectance, various light sources, and observer's point-of-view to be taken into consideration.


\subsection{What went where?}
Motion is an important aspect in interpreting the interaction with subjects, making the visual perception of movement a critical cognitive ability that helps us with complex tasks such as discriminating moving objects from background, or depth perception by motion parallax. Cognitive susceptibility enables the inference of 2D/3D motion from a sequence of 2D shapes (e.g., movies~\cite{niyogi1994analyzing,little1998recognizing,hayfron2003automatic}), or from a single image frame (e.g., the pose of an athlete runner~\cite{wang2013learning,ramanan2006learning}). However, its challenging to model the susceptibility because of many-to-one relation between distal and proximal stimulus, which makes the local measurements of proximal stimulus inadequate to reason the proper global interpretation. One of the various challenges is called \textit{motion correspondence problem}~\cite{attneave1974apparent,ullman1979interpretation,ramachandran1986perception,dawson1991and}, which refers to recognition of any individual component of proximal stimulus in frame-1 and another component in frame-2 as constituting different glimpses of the same moving component. If one-to-one mapping is intended, $n!$ correspondence matches between $n$ components of two frames exist, which is increased to $2^n$  for one-to-any mappings. To address the challenge, Ullman~\cite{ullman1979interpretation} proposed a method based on nearest neighbor principle, and Dawson~\cite{dawson1991and} introduced an auto associative network model. Dawson's network model~\cite{dawson1991and} iteratively modifies the activation pattern of local measurements to achieve a stable global interpretation. In general, his model applies three constraints as it follows
\begin{inlinelist}
	\item \textit{nearest neighbor principle} (shorter motion correspondence matches are assigned lower costs)
	\item \textit{relative velocity principle} (differences between two motion correspondence matches)
	\item \textit{element integrity principle} (physical coherence of surfaces)
\end{inlinelist}.
According to experimental evaluations (e.g.,~\cite{ullman1979interpretation,ramachandran1986perception,cutting1982minimum}), these three constraints are the aspects of how human visual system solves the motion correspondence problem. Eom et al.~\cite{eom2012heuristic} tackled the motion correspondence problem by considering the relative velocity and the element integrity principles. They studied one-to-any mapping between elements of corresponding fuzzy clusters of two consecutive frames. They have obtained a ranked list of all possible mappings by performing a state-space search. 



\subsection{How a stimuli is recognized in the environment?}

Human subjects are often able to recognize a 3D object from its 2D projections in different orientations~\cite{bartoshuk1960mental}. A common hypothesis for this \textit{spatial ability} is that, an object is represented in memory in its canonical orientation, and a \textit{mental rotation} transformation is applied on the input image, and the transformed image is compared with the object in its canonical orientation~\cite{bartoshuk1960mental}. The time to determine whether two projections portray the same 3D object
\begin{inlinelist}
	\item increase linearly with respect to the angular disparity~\cite{bartoshuk1960mental,cooperau1973time,cooper1976demonstration}
	\item is independent from the complexity of the 3D object~\cite{cooper1973chronometric}
\end{inlinelist}.
Shepard and Metzler~\cite{shepard1971mental} interpreted this finding as it follows: \textit{human subjects mentally rotate one portray at a constant speed until it is aligned with the other portray.}



\subsection{State of the Art}

The linear mapping transformation determination between two objects is generalized as determining optimal linear transformation matrix for a set of observed vectors, which is first proposed by Grace Wahba in 1965~\cite{wahba1965least} as it follows. 
\textit{Given two sets of $n$ points $\{v_1, v_2, \dots v_n\}$, and $\{v_1^*, v_2^* \dots v_n^*\}$, where $n \geq 2$, find the rotation matrix $M$ (i.e., the orthogonal matrix with determinant +1) which brings the first set into the best least squares coincidence with the second. That is, find $M$ matrix which minimizes}
\begin{equation}
	\sum_{j=1}^{n} \vert v_j^* - Mv_j \vert^2
\end{equation}

Multiple solutions for the \textit{Wahba's problem} have been published, such as Paul Davenport's q-method. Some notable algorithms after Davenport's q-method were published; of that QUaternion ESTimator (QU\-EST)~\cite{shuster2012three}, Fast Optimal Attitude Matrix \-(FOAM)~\cite{markley1993attitude} and Slower Optimal Matrix Algorithm (SOMA)~\cite{markley1993attitude}, and singular value decomposition (SVD) based algorithms, such as Markley’s SVD-based method~\cite{markley1988attitude}. 

In statistical shape analysis, the linear mapping transformation determination challenge is studied as Procrustes problem. Procrustes analysis finds a transformation matrix that maps two input shapes closest possible on each other. Solutions for Procrustes problem are reviewed in~\cite{gower2004procrustes,viklands2006algorithms}. For orthogonal Procrustes problem, Wolfgang Kabsch proposed a SVD-based method~\cite{kabsch1976solution} by minimizing the root mean squared deviation of two input sets when the determinant of rotation matrix is $1$. In addition to Kabsch’s partial Procrustes superimposition (covers translation and rotation), other full Procrustes superimpositions (covers translation, uniform scaling, rotation/reflection) have been proposed~\cite{gower2004procrustes,viklands2006algorithms}. The determination of optimal linear mapping transformation matrix using different approaches of Procrustes analysis has wide range of applications, spanning from forging human hand mimics in anthropomorphic robotic hand~\cite{xu2012design}, to the assessment of two-dimensional perimeter spread models such as fire~\cite{duff2012procrustes}, and the analysis of MRI scans in brain morphology studies~\cite{martin2013correlation}.

\subsection{Our Contribution}

The present study methodizes the aforementioned mentioned cognitive susceptibilities into a cognitive-driven linear mapping transformation determination algorithm. The method leverages on mental rotation cognitive stages~\cite{johnson1990speed} which are defined as it follows
\begin{inlinelist}
	\item a mental image of the object is created
	\item object is mentally rotated until a comparison is made
	\item objects are assessed whether they are the same
	\item the decision is reported
\end{inlinelist}.
Accordingly, the proposed method creates hierarchical abstractions of shapes~\cite{greene2009briefest} with increasing level of details~\cite{konkle2010scene}. The abstractions are presented in a vector space. A graph of linear transformations is created by circular-shift permutations (i.e., rotation superimposition) of vectors. The graph is then hierarchically traversed for closest mapping linear transformation determination. 

Despite of numerous novel algorithms to calculate linear mapping transformation, such as those proposed for Procrustes analysis, the novelty of the presented method is being a cognitive-driven approach. This method augments promising discoveries on motion/object perception into a linear mapping transformation determination algorithm.




\section{Sample Selection}
	\label{sec:SS}
	The November 2015 alpha version of the Radio Galaxy Zoo consensus catalogue lists the properties of $85\,151$ radio sources distributed primarily over the footprint of two surveys: FIRST and Australia Telescope Large Area Survey (ATLAS) \citep{Norris2006}. The classification was performed by volunteers, who were presented with radio images from these surveys and the corresponding infrared fields observed by the Wide-field Infrared Survey Explorer \citep[WISE;][]{Wright2010}. They were then asked to match disconnected components corresponding to the same source and recognize the infrared counterpart. A more detailed description of the project is available in \cite{Banfield2015}. 

	The Radio Galaxy Zoo represents a natural choice for our statistical analysis. Whereas components belonging to the same source are usually recognized through self-matching (i.e., cross-matching the source catalogue with itself to identify sources at a certain distance from each other) or human selection, we rely on the additional information provided by human inspection to increase the reliability of the results. Furthermore, the $5\arcsec$ nominal resolution of the FIRST images implies a high number of resolved sources, for which a preferential direction can be defined. Lastly,  the survey covers an area of about $10\,000$ square degrees and allows us to infer general properties of the radio sky, instead of a local statistical anomaly.

	For our second sample, based on the TGSS Alternative Data Release 1, no human-made classification is available. In its place, we opt for automated self-matching. The TGSS ADR1 is based on an independent reprocessing of an original 150 MHz GMRT survey performed between 2010 and 2012 and the corresponding source catalogue, released in 2016, covers $99.5$  per cent of the sky north of $-53^\circ$ declination. A more detailed description is available in \cite{Intema2016}.
	
	\subsection{Radio Galaxy Zoo}
	
			\begin{figure}
				\centering
				\includegraphics[width=0.45\textwidth]{files/exall.pdf}
				\caption{FIRST image for a typical source with morphological features superimposed. The angular extent of the source is about $\mathbf{1\arcmin10\arcsec}$. The red boxes identify the components provided by the Radio Galaxy Zoo, with the crosses indicating surface brightness peaks. The blue ellipses have major and minor axes equal to the FWHM of the fitted Gaussian model in the FIRST catalogue, and the dots are their centres. The red and dashed blue lines are the results of the orthogonal distance regression for the dots and the crosses respectively. In this particular case two or more surface brightness peaks are present and the position angle is extracted from the slope of the red line (see text for more details).}
				\label{fig:exall}
			\end{figure}	
			
			We select extended sources from the Radio Galaxy Zoo consensus catalogue and extract an elongation direction for each of them.We describe this direction with a position angle, defined as the angle east of north in the range $[-\pi/2, +\pi/2]$ between the direction itself and the local meridian. 
			
			To perform the selection and constrain the orientation, we rely on the quantities contained in the November 2015 alpha version of the Radio Galaxy Zoo consensus catalog and, occasionally, on the official FIRST catalogue presented in \cite{Helfand2015b}, version \textsc{14dec17}. Figure~\ref{fig:exall} presents these quantities in graphic form. From the Radio Galaxy Zoo catalogue we extract the areas covered by components belonging to the same source and the peak positions of the source surface brightness contained in these regions (peaks hereafter). From the FIRST catalogue we extract the Gaussian model of the source brightness contained inside the same areas. 
			An additional quantity provided by the Radio Galaxy Zoo for every morphological classification is the consensus level. This is defined as the fraction of users who voted for the specific components configuration and, in this analysis, it is used to rank distinct classifications of the same object. 
			
			Depending on the available data, different procedures are employed to extract the position angle. We define three sub-samples:
			
			\begin{description}
			\item a)  If two or more surface brightness peaks are present for a given source, we define the position angle as the slope of the orthogonal distance linear regression of the peaks, weighted according to their flux densities. Around $80$ per cent of the selected sources belong to this category. An example of such a source is provided in Figure~\ref{fig:exall}.
			
			
			\item b) For sources with only one surface brightness peak in the Radio Galaxy Zoo catalogue, but multiple Gaussian models in the FIRST catalogue, we rely completely on the latter.
			This occurs when components are not seen as separated in the Radio Galaxy Zoo because of the particular automated choice of contour levels. For this sub-sample the centres of the FIRST ellipses, weighted by their integrated flux, are fitted.
			
			\item c) 	If only one surface brightness peak is detected and the FIRST catalogue recognizes only one source inside the single component radio galaxy, we rely on the Gaussian model of the FIRST catalogue and we define the direction as the position angle of the fitted ellipse. In this case, the source must comply with a total of four criteria.
			
			First, sources must meet the conditions required to be included in the Radio Galaxy Zoo sample and be presented to the volunteers. These are aimed at selecting resolved sources with a high signal-to-noise ratio:
			
			\begin{equation}
			\frac{S_{\rm{peak}}}{S_{\rm{int}}} < 1.0 - \left( \frac{0.1}{\log S_{\rm{peak}}}\right) \textrm{ and } SNR>10,
			\label{eq:cond1}
			\end{equation} 
			where $S_{\rm{peak}}$ is the peak brightness in mJy beam$^{-1}$, $S_{\rm{int}}$ is the integrated flux density of the source in mJy and $SNR$ is the signal-to-noise ratio \citep{Banfield2015}.
			
			Secondly, we introduce two additional criteria. The minor axis $m$ of the fitted elliptical Gaussian model should be larger than $2\arcsec$ and the deviation of the ratio between the major and minor axis $r$ from unity should be highly significant:
			
			
			\begin{equation}
			m > 2\arcsec
			\textrm{ and }
			r > 1 + 7\sigma_r.
			\label{eq:cond2}
			\end{equation}
			
			The error on the major and minor axis ratio is overestimated by the quadratic sum
			
			
			\begin{equation}
			\sigma_r = r\sqrt{\left(
				\frac{\sigma_m}{m}
				\right)^2
				+
				\left( 
				\frac{\sigma_m}{M}
				\right)^2},
			\end{equation}
			
			
			where $\sigma_m$ is the empirical uncertainty on both the fitted minor axis $m$ and major axis $M$. The four conditions, \eqref{eq:cond1} and \eqref{eq:cond2}, select extended sources for which an elongation is clearly recognizable.
			
			\end{description}
	
			When both multiple Gaussian models and multiple flux density peaks are available, we choose to prioritize the peaks over the centres. Figure~\ref{fig:exall} provides an example of how the difference between the two fitted position angles is usually small.
	
			
			\begin{figure}
				\centering
				\includegraphics[width=0.5\textwidth]{files/mindisttot.pdf}
				\caption{Distribution of the angular distance between a source in the Radio Galaxy Zoo sample and its closest neighbour, before and after filtering duplicates.}
				\label{fig:mindist}
			\end{figure} 
			
			The release of the Radio Galaxy Zoo consensus catalogue used here includes every classification performed by the volunteers. Because of this, a single source might appear multiple times with different classifications. 
            To filter these duplicate entries we focus our attention on all the recognized components. For every set of overlapping components, we filter out all of the sources they belong to, except for the one with highest consensus level. The effect of this selection process can be seen in Figure~\ref{fig:mindist}, where we plot the distribution of the distances between every source and its closest neighbour. While a natural amount of clustering is expected, we find that almost half of the sources have an extremely close neighbour -- a probable duplicate. After we apply our filter the peak around $0\farcs6$ disappears.
            
 	           
            A second systematic effect inherited from the Radio Galaxy Zoo is the quantisation of the peak positions. To clearly discern its importance, we limit our attention to the sources classified as containing only two peaks and we plot the differential right ascension and declination of every pair (Figure~\ref{fig:PP}). Discretisation is more noticeable in the vertical axis, but a $1\farcs4$ binning effect is visible in both directions. The presence of pixels is caused by numerical approximations in the implementation of the World Coordinates System (WCS).  
			In our analysis, this grid-like disposition of the peaks implies discrete values of the associated position angles. To obtain a continuous distribution of the angles, we smooth out the peak positions by adding a uniformly random value in the range $[-0\farcs7, +0\farcs7]$ to both coordinates before performing the linear regression. This process pushes the influence of the effect to sub-pixel scales, eliminating its impact on the present study. However, further investigation is needed to constrain its causes. 

		 	\begin{figure}
		 		\centering
		 		\includegraphics[width=0.5\textwidth]{files/PP.pdf}
		 		\caption{Relative peak positions for entries classified as containing two peaks. Discretization is evident in the collapsed distributions.}
		 		
		 		\label{fig:PP}
		 	\end{figure}
            For the sake of consistency, the final sample presented in Figure~\ref{fig:RGZPAdist} excludes ATLAS sources and it is limited only to FIRST sources. For the same reason, we also exclude every source positioned above RA $20$ hr and below $4$ hr, since half of the observations in this region were performed after the observing array transitioned to the new JVLA configuration \citep{Helfand2015b}. 
            
			Finally, notice how the original Radio Galaxy Zoo selection in Eq.~\eqref{eq:cond1} does not include an explicit cut for artefacts. During the first run of the Radio Galaxy Zoo classification, the volunteers were presented with $3\arcmin \times3\arcmin$ fields. This corresponds to a maximum distance of $ 3\arcmin \sqrt{2} \approx 4\arcmin 12\arcsec$ between two components. To quantify the contamination from artefacts in our sample, we make use of the column $P(S)$ of the official FIRST catalogue, which indicates the probability of a source to be a sidelobe. We cross-matched our selection with the FIRST catalogue using a search radius of $4\arcmin 12\arcsec$ and we verified that 134 selected sources are part of a field containing possible sidelobes satisfying the condition $P(S)>0.1$. In principle, these artefacts might be recognized as components and influence the value of the position angles. Because of this, we exclude sources with $P(S)>0.1$ from our final Radio Galaxy Zoo sample.
            
            In Figure~\ref{fig:RGZPAdist} we plot the final distribution of the extracted position angles, together with the distributions for the three classes of sources. While we would expect these to be uniform, three peaks are visible around $30^\circ$, $-30^\circ$ and $90^\circ$. In these three directions we recognize the typical pattern that results from the three arms of the observing radio interferometer --- the Very Large Array (VLA). The same effect is visible in the FIRST images and is discussed in \cite{Helfand2015b}, where a three-directional pattern is present in the distribution of the sidelobes around bright sources. The existence of preferential angles may be related to the brightness of the weaker components, although a more detailed analysis would be required to quantify this effect. This will not affect our analysis as long as the effects are non-local.

			\begin{figure}
				\centering
				\includegraphics[width=0.45\textwidth]{files/RGZPATOT.pdf}
				\caption{Position angle distribution of the Radio Galaxy Zoo selection. On top of the total distribution (topmost histogram) the plot contains the distributions of the three sub-samples. From top to bottom: (a) in grey, (b) in red, and (c) in blue. A trimodal systematic effect is visible in the first two.}
				
				\label{fig:RGZPAdist}
			\end{figure}

            A similar pattern is discussed also in other analyses \citep[e.g.,][]{Chang2004, White2007, Demetroullas2015} based on the FIRST survey, where the effect is recognized as non position-dependent. Snapshot surveys are commonly affected by an anisotropic point spread function (PSF) and the connection to the interferometer geometry suggests this origin.
            \cite{Helfand2015b} underlines that particular care was taken in ensuring a constant PSF throughout the different observation epochs of FIRST. In particular, since the hour angle of observation affects the orientation of the pattern in the cleaned images, $90\%$ of the observations were acquired within $1.4$ hr of the local meridian.	
			
			The non-locality of the effect is verified by partitioning the data by both right ascension and declination in four equally populated quadrants. Pairwise, the four position angle distribution are found to be consistent with each other using two-sample Kolmogorov-Smirnov tests. 
            
            This first position angle catalogue contains $30\,059$ sources distributed over an area of about $7\,000$ square degrees, resulting in a number density $\sim4$ deg$^{-2}$.


		\subsection{TGSS Alternative Data Release}
		
			As opposed to the Radio Galaxy Zoo sample, this second position angle sample is based on the product of an automated source extractor. The nominal resolution of $25\arcsec$ for the TGSS images implies a lower number of extended sources with significant elongation compared to FIRST. However, the relatively steep spectrum of radio galaxy lobes and the sensitivity to extended sources of the GMRT allow TGSS to trace the lobes better than FIRST.  Hence, we focus our attention on the identification of double-lobed sources. 
			In Figure~\ref{fig:mindistTGSS} we plot the distance between each entry in the TGSS catalogue and its closest neighbour. The rightmost peak is due to the distribution of uncorrelated radio sources, while the lower peak on the left is caused by multi-component sources. The plot suggests an average distance of $1\arcmin$ between the components of a source of the latter type. A peak around the angular scale of $1\arcmin$ is not present in the Radio Galaxy Zoo catalogue because the pairing was already performed by the volunteers during the classification process.

			\begin{figure}
				\centering
				\includegraphics[width=0.5\textwidth]{files/mindistTGSS.pdf}
				\caption{Distribution of the angular distance between a source in the TGSS catalogue and its closest neighbour. The dashed red line marks the value $1\arcmin12\arcsec$.}
				
				\label{fig:mindistTGSS}
			\end{figure}
			
			We select radio galaxy candidates by self-matching the catalogue with a search radius $1\arcmin12\arcsec$ and imposing a maximum ratio of $10$ between the total fluxes of the two components \citep{VanVelzen2014}.
			To be part of the final sample both components of the pair need to satisfy additional constraints: (1) isolated (i.e., matched only to each other) (2) $SNR>10$. The position angle is then simply that of the line connecting the two components. The search radius we chose corresponds to the local minimum marked in Fig.~\ref{fig:mindistTGSS}. A larger value would introduce an artificial contamination in our double-lobed source catalogue, while a lower value would mean losing part of the genuine sources.
			
			We decide to limit our sample to a portion of the northern hemisphere to minimize the effects of an anisotropic PSF. \cite{Intema2016} reports the synthesized beam to be circular for pointings at declination higher than the GMRT latitude --- about $19^\circ$. Even between declinations of $10^\circ$ and $19^\circ$ the beam is still circular to within $1\%$. Therefore, our final TGSS sample includes only sources with declination above $10^\circ$.
			
			Figure~\ref{fig:TGSSPAdist} shows the position angle distribution of the final TGSS sample. This second position angle catalogue contains $11\,674$ sources distributed over an area of about $17\,000$ square degrees, resulting in a number density $\sim0.7$ deg$^{-2}$. We notice that unlike for the FIRST survey, no particular care was taken with respect to the PSF and its consistency throughout different pointings. However, the  complex geometry of the interferometer and longer integration times compared to FIRST result in a PSF less prone to systematic effects. 
			Table~\ref{tab:surveys} compares the different surveys and samples featured in this section. The difference between the number of sources in the two catalogues produced in this section is due to the different nature of the original surveys and the source selection process. While $85\%$ of the sources in the RGZ sample have size larger than the TGSS resolution ($25\arcsec$), only $55\%$ of them are larger this threshold and have exactly two surface brightness peaks.
			
			We can use the RGZ catalogue to predict the size of the TGSS one. If we account for the different frequencies ($1.4$ GHz for FIRST and $150$ MHz for TGSS) by adopting a nominal spectral index equal to $0.9$ \citep{Vollmer2010} and keeping in mind the sky coverage and angular resolution differences, we find that about $10^4$ sources are expected to be selected by our algorithm. This number is in line with the $11\,674$ sources found in our selection.
			
			\begin{figure}
				\centering
				\includegraphics[width=0.45\textwidth]{files/TGSSPATOT.pdf}
				\caption{Position angle distribution of the TGSS selection. Obvious systematic effects are not present.}
				
				\label{fig:TGSSPAdist}
			\end{figure}


\begin{table*}
\caption{Comparison between the different samples and source catalogs discussed in this paper.}
\label{tab:surveys}
\begin{tabular}{lccccccc}
\hline
Name & Frequency & Median RMS & SNR & Number of & Minimum & Sky & Median Redshift \\
     &           & Noise      & Threshold  & Sources   &  Resolution & Fraction & $68\%$ interval \\
     &           &[mJy beam$^{-1}$] &       &           &            &          &                  \\ 

\hline
				FIRST\,$^a$ &
				$1.4$ GHz & 
				$0.15$& 
				$5$&
				$946\,432$& 
				$5\arcsec\times5\arcsec$& 
				$26\%$&
				$2.2\pm0.9$\,$^b$ 
				\\ 
				Radio Galaxy Zoo\,$^c$  & 
				$1.4$ GHz & 
				$0.15$ & 
				$10$& 
				$82\,187$&  
				$5\arcsec\times5\arcsec$& 
				$22\%$& 
				$0.47_{-0.15}^{+0.21}$\,$^d$ 
				\\
				Radio Galaxy Zoo processed\,$^e$ & 
				$1.4$ GHz &  
				$0.15$& 
				$10$& 
				$30\,059$  &
				$5\arcsec\times5\arcsec$& 
				$19\%$& 
				$0.47_{-0.15}^{+0.20}$\,$^d$ 
				\\
				TGSS\,$^f$& 
				$150$ MHz & 
				$3.5$& 
				$7$& 
				$623\,604$ & 
				$25\arcsec\times25\arcsec$& 
				$90\%$& 
				$-$
				\\
				TGSS processed\,$^e$ & 
				$150$ MHz & 
				$3.5$& 
				$10$& 
				$11\,674$ & 
				$25\arcsec\times25\arcsec$& 
				$42\%$& 
				$-$ \\
\hline
\multicolumn{8}{l}{ $^a$ \cite{Helfand2015b}} \\
\multicolumn{8}{l}{ $^b$ Mean redshift with $68\%$ confidence levels from \cite{Chang2004}}	\\
\multicolumn{8}{l}{ $^c$ \cite{Banfield2015}} \\
\multicolumn{8}{l}{ $^d$ Only $30\%$ of the sample has a human-matched optical counterpart with known redshift} \\
\multicolumn{8}{l}{ $^e$ The selection process, aimed at selecting resolved sources to use in this study, is detailed in Section~\ref{sec:SS}} \\
\multicolumn{8}{l}{ $^f$ \cite{Intema2016}} \\
\end{tabular}
\end{table*}

\section{Statistical Analysis}
	\label{sec:SA}
	\subsection{Parallel Transport}	
		\label{sec:PT}  	
		
				
		The position angle is a directional quantity defined in the point of the celestial sphere where the corresponding source lies. In order to perform the calculation of the misalignment angle between two directions on a sphere, the notion of parallel transport should be introduced \citep{Jain2004}.
		
		We parametrize the sphere using spherical coordinates $(r, \theta, \phi)$ and we define in every point a natural orthonormal basis dictated by our coordinate system. This set of unit vectors is $(\mathbfit{e}_r, \mathbfit{e}_\theta, \mathbfit{e}_\phi)$, where the three elements point respectively towards the centre of the sphere, northward and eastward.	
		
		A source with position angle $\alpha$, determined up to a rotation of $\pi$ radians, can be identified with the unit vector
		
		\begin{equation}
			\mathbfit{v} = \cos\alpha \; \mathbfit{e}_\theta + \sin \alpha \; \mathbfit{e}_\phi
			\label{eq:v}
		\end{equation}
		
		Since the projection along the line of sight is unknown, we fix this vector to be tangent to the sphere at the point of definition.  
		The vector $\mathbfit{v}$ represents a physical quantity, whereas the definition of position angle $\alpha$ depends on the choice of coordinate system. For example, if parallels and meridians were redefined with respect to a different north pole, the vectors $\mathbfit{e}_\theta$, $\mathbfit{e}_\phi$ and the position angle $\alpha$ would change. However, the vector $\mathbfit{v}$ in Eq. \eqref{eq:v} would still describe the same direction in space.
		On a sphere, parallel transport allows us to define a coordinate-invariant inner product between two vectors, by translating one of them along arcs of great circles connecting the two.  
		
		Let us consider two tangent vectors $\mathbfit{v}_1$ and $\mathbfit{v}_2$ with position angles $\alpha_1$ and $\alpha_2$, defined respectively in $P_1 = (r_1, \theta_1, \phi_1)$ and $P_2 = (r_2, \theta_2, \phi_2)$. Both of these points belong to the same unit sphere ($r_1 = r_2 = 1$). The great circle passing through them lies on a plane perpendicular to $\mathbfit{e}_s$
		
		\begin{equation}
			\mathbfit{e}_s = \frac{\mathbfit{e}_{r_1}\times \mathbfit{e}_{r_2}}{\vert \mathbfit{e}_{r_1} \times \mathbfit{e}_{r_2}\vert}
		\end{equation}
		
		We define $\mathbfit{e}_{t_1}$ and $\mathbfit{e}_{t_2}$ as the tangent vectors of this great circle in the points $P_1$ and $P_2$.
		
		\begin{gather}
			\mathbfit{e}_{t_1} = \mathbfit{e}_{s} \times \mathbfit{e}_{r_1}
			\\
			\mathbfit{e}_{t_2} = \mathbfit{e}_{s} \times \mathbfit{e}_{r_2}
		\end{gather}
		
		We call $\zeta_1$ the angle between $\mathbfit{e}_{t_1}$ and $\mathbfit{e}_{\theta_1}$. Similarly, we define $\zeta_2$ as the angle between $\mathbfit{e}_{t_2}$ and $\mathbfit{e}_{\theta_2}$. Translating the vector $\mathbfit{v}_{1}$ along the great circle maintains the angle with the local tangent vector constant and at the point $P_2$ it results in the translated vector $\mathbfit{v}_1^\prime$ with position angle
		
		\begin{equation}
			\alpha_1^\prime =  \alpha_1 + \zeta_2 - \zeta_1
			\label{eq:alphaprime}
		\end{equation}
		
		Figure~\ref{fig:PT} depicts the vectors involved in the operation. With this in mind, we define the generalized dot product between $\mathbfit{v}_{1}$ and $\mathbfit{v}_{2}$ as the following
		
		\begin{equation}
			\mathbfit{v}_{1} \odot \mathbfit{v}_{2} = \vert \mathbfit{v}_{1} \vert
			\vert \mathbfit{v}_{2} \vert \cos (\alpha_1 - \alpha_2 + \zeta_2 - \zeta_1)
		\end{equation}
		
		Since our dataset is purely directional, we have $\vert \mathbfit{v}_{1} \vert = \vert \mathbfit{v}_{2} \vert = 1$. For the same reason, the inner product is written using the following simplified notation
		
		\begin{equation}
		(\alpha_1, \alpha_2) =  \cos [2(\alpha_1 - \alpha_2 + \zeta_2 - \zeta_1)]
		\label{eq:innerproduct}
		\end{equation}
		
		The factor two is introduced so that the argument of the cosine ranges over the full $-\pi$ to $+\pi$, \citep{Bietenholz1986}. By definition $(\alpha_1, \alpha_2) \in [-1, 1]$, where $+1$ indicates perfect alignment \citep{Jain2004} and $-1$ implies perpendicular directions.
		
		
\begin{figure}
	\centering
	\includegraphics[width=0.45\textwidth]{files/PT.pdf}
	\caption{Two dimensional schematic illustration of parallel transport. The figure displays the arc of great circle passing through the points $P_1$ and $P_2$, with $\mathbfit{e}_{t_1}$ and $\mathbfit{e}_{t_2}$ tangent vectors to curve in these points. Notice that the angle $\theta$ between the tangent vector and $\mathbfit{v}_{1}$ is kept constant when $\mathbfit{v}_{1}$, located at $P_1$, is translated along the curve to the point $P_2$. The figure is taken from \citet{Jain2004}, their figure 1, with the author's permission.}
	\label{fig:PT}
\end{figure}	

	\subsection{Angular Dispersion}
		\label{sec:S}
		Given the $i-$th source, we consider the $n$ sources closest to it (including itself). We call $d_{i,n}$ the dispersion function of their position angles.
		
		\begin{equation}
			d_{i, n}(\alpha) = \frac{1}{n}\sum_{k=1}^{n} (\alpha, \alpha_k)
			\label{eq:d}
		\end{equation}
		
		This quantity is a function of a position angle $\alpha$ located at the point where the $i-$th source lies. We call $\alpha_{\rm{max}}$ the position angle that maximizes the dispersion, which assumes the value
		
		\begin{equation}
			d_{i, n}\big|_{\rm{max}} =  \frac{1}{n} \left[ 
			\left( \sum_{k=1}^{n} \cos 2\alpha_k^\prime\right)^2
			+
			\left( \sum_{k=1}^{n} \sin 2\alpha_k^\prime\right)^2
			\right]^{1/2},
		\end{equation}
		
		where $\alpha_k^\prime$ was defined in Eq.~\eqref{eq:alphaprime} and corresponds to the value of the original position angle $\alpha_k$ after being transported in the $i-$th position. Following \cite{Jain2004}, we regard this maximal value as the measure of the dispersion of the $n$ sources and $\alpha_{\rm{max}}$ as their mean direction. The maximum value allowed for the dispersion is $d_{i, n}|_{\rm{max}} = 1$, corresponding to perfect alignment of the sources. The coordinate-invariance of the inner product (Eq. \ref{eq:innerproduct}) extends to the dispersion.
		
		
		For a sample of $N$ sources we fix a number of nearest neighbours $n$ and we derive the set of dispersions.
		
		\begin{align}
			\{d_{i, n}\big|_{\rm{max}}\} && i =1, \dots, N
		\end{align}
		
		For this set we define the following statistics
		
		\begin{align}
			S_{n} = \frac{1}{N} \sum_{i=1}^{N} d_{i, n}\big|_{\rm{max}},
			\label{eq:S}
		\end{align}

		corresponding to the mean dispersion. $S_n$ measures the average position angle dispersion of the sets containing every source and its $n$ neighbours.  
		If the condition $N \gg n \gg 1$ is satisfied, then $S_{n}$ is expected to be normally distributed. \citeauthor{Jain2004} reports the following form for its variance
		
		\begin{equation}
		\sigma_n^2 = \frac{0.33}{N},
		\label{eq:sigmaest}
		\end{equation} 
		
		where $N$ is the total number of sources in the sample. 
		The quantity $S_n$ can be employed for different values of $n$, although these different measurements are not independent. Because the dispersion $d_{i, n}$ is defined in Eq. \eqref{eq:d} as an average of the $n$ closest neighbours, the presence of a positive alignment for $n^\ast$ neighbours implies a preferential positive signal for every $n>n^\ast$.
		
		The deviation of the dispersion $d_{i, n}|_{\rm{max}}$ from its mean value is not normalized, but is found to be $\propto 1/\sqrt{n}$ \citep{Jain2004}. This is mirrored by $S_n$
		
		\begin{equation}
			S_n \propto \frac{1}{\sqrt{n}}
			\label{eq:Sest}
		\end{equation}
		
		To remove this spurious dependence, we will write the measurements of $S_n$ as one-tailed significance levels when considering multiple values of $n$
	
		\begin{equation}
			S.L. = 1-\Phi \left( \frac{S_n - \braket{S_n}_{MC}}{\sigma_n}\right),
			\label{eq:SL}
		\end{equation}
		
		where $\Phi$ is the cumulative normal distribution function and $\braket{S_n}_{MC}$ is the expected value for $S_n$ in absence of alignment, found through Monte Carlo simulations.
		We then employ the following approximate scale: $\log$ S.L. $< -3.5$, very strong alignment;  $-2.5>\log$ S.L. $> -3.5$, strong alignment; $-1.5>\log$ S.L. $> -2.5$ weak alignment.
		
		For every source (labelled by $i$) we define $\varphi_{i, n}$ as angular radius of the circle containing its $n$ neighbours. We can then define the following set:
		
		\begin{align}
		\{\varphi_{i, n}\} && i =1, \dots, N
		\label{eq:phii}
		\end{align}
		
		The distribution of this set provides information about what angular scale a particular $S_n$ probes. For our purposes we will refer to its median $\tilde{\varphi}(n)$ and the $68\%$ interval around it. 
		
		
		\subsection{Random Datasets}
		\label{sec:RandomDatasets}
		
		To estimate the uncertainties and the significance of a given measurement we use simulated data sets containing only noise. The random data sets ($1\,000$ in total) are generated by shuffling the position angles among different sources to ensure that every configuration is affected by the same position angle distribution and survey geometry. 
		
		For a binned or sampled quantity $W_k$ $k\in \{1\dots N_{bins}\}$ we estimate the covariance matrix as
		
		\begin{equation}
		\Sigma^2_{ij} = \braket{(W_i - \braket{W_i}_{MC} )\cdot (W_j - \braket{W_j}_{MC})}_{MC},
		\label{eq:cov}
		\end{equation}
		
		where all the averages are computed over multiple simulations.
		
		For a multivariate Gaussian random vectors $\bm{x}$ with expected mean $\bm{\mu}$ and covariance matrix $C$ of rank $k$, the $\chi^2$ test is generalized using the Mahalanobis distance squared
		
		\begin{equation}
		d^2 = (\bm{x} - \bm{\mu})^T C^{-1} (\bm{x} - \bm{\mu}),
		\end{equation}
		
		which is chi-square distributed with $k$ degrees of freedom. In our analysis, we define the components of vector $\bm{W}$ as the measurements of the statistics $W$ performed on different scales. We then use as Mahalanobis statistics the following expression:
		
		\begin{equation}
		d^2 = (\bm{W}-<\bm{W}>_{MC})^T (\Sigma^2)^{-1} (\bm{W}-<\bm{W}>_{MC})
		\label{eq:Maha}
		\end{equation}
		
		The alignment analyses performed by \cite{Jain2004, Hutsemekers2014, Taylor2016} are based on statistical tests similar to the position angle/polarization vector mean dispersion $S_n$ defined in Eq. \eqref{eq:S}.  None of the above references take covariance into account when estimating the significance level of the measured dispersion as a function of the angular scale. In this study, the Mahalanobis statistics measures deviation from the noise by taking covariance into account.
	
\section{Results}
\label {Sec:Results}

We present our results for each of the three research questions posed in the introduction. In most cases we found similar trends in Tanzania and Kenya so we report merged findings. We highlight areas where we observed different trends in the two countries. While we often provide the number of responses corresponding to a particular code to give readers a sense of the frequency of each theme in our interviews, we caution that this is a qualitative study and these numbers should not be interpreted as representative of frequencies of beliefs and behaviors in the population.  
\begin {table} [htbp]
\caption{A summary of workarounds by users and agents}
\centering
\begin {tabular} {|p{3cm}|p{2.7cm}|p{1.8cm}|}
\hline
\multicolumn{2}{|l|}{\textbf{Transaction Execution Workarounds}}& \textbf{Changing Component}\\
\hline
\multirow[c]{2}{*}{\textbf{Alternative execution}}& Agent executes transfer& Agent \\ \cline{2-3} & Transact at a distance & User \& Agent\\
\hline
\multirow[c]{2}{*}{\textbf{Alternative services}}& Advance transactions& \multirow[c]{2}{*}{Agent} \\ \cline{2-2} & Credit/Loan services & \\
\hline
\multirow[c]{3}{3cm}{\textbf{Change transaction characteristics}}& Location& \multirow[c]{3}{*}{Transaction} \\ \cline{2-2} & Size & \\ \cline{2-2} & Channel & \\
\hline
\multicolumn{2}{|l|}{\textbf{KYC Workarounds}}& \\
\hline
\multirow[c]{2}{*}{\textbf{Agent-adjusted KYC}}& Relationship-based& User \& Agent \\ \cline{2-3} & Agent complaisance & Agent\\
\hline
\multirow[c]{3}{3cm}{\textbf{Alternative identity proofing}}& Third-party ID& \multirow[c]{3}{*}{User} \\ \cline{2-2} & Reduced KYC & \\ \cline{2-2} & Human recommender & \\
\hline
\multicolumn{2}{|l|}{\textbf{Confirmation Workarounds}}& \\
\hline
\multirow[c]{2}{*}{\textbf{Alternative confirmation}}& Agent confirmation& User \& Agent \\ \cline{2-3} & Recipient confirmation & User\\
\hline
\multicolumn{2}{|l|}{\textbf{Transaction Reversal Workarounds}}& \\
\hline
\textbf{Alternative reversal}& Agent-assisted reversal& Agent\\
\hline
 \end{tabular}
\label{table:tab1workaroundsContinuum}
\end{table}


\subsection{Q1: Practices facilitated by the user-agent relationship, motivations, and privacy/security concerns}
\label{sec:habitsandpractices} 
The practices we observed can be roughly divided into two categories: 1) practices surrounding how mobile money users \textit{choose} an agent, and 2) practices around how mobile money users \textit{use} an agent. 
The latter category was characterized by surprising examples of usage that we term \textit{workarounds}---workflows that are not intended by the MoMo provider, but that clients and agents, either independently or sometimes in a concerted way, work out between themselves. We identify these workarounds (Table \ref{table:tab1workarounds}) according to the stage of the transaction continuum in which they took place:  during the transaction execution, KYC, transaction confirmation, or the optional reversal phase (Table \ref{table:tab1workaroundsContinuum}). In the following sections, we discuss these workarounds and their motivations (Table \ref{table:tab2workaroundmotivations}). Workarounds can additionally (roughly) be categorized by what component of the MoMo environment they modify; \textit{the user}, \textit{the agent}, or \textit{the transaction} itself.  

\begin {table} [htbp]
\caption{Workarounds through the transaction stages}
\centering
\begin {tabular} {|l|l|l|}
\hline
\textbf{Workaround}& \textbf{Ke (n)}& \textbf{Tz (n)}\\
 \hline
 \hline
\multicolumn{3}{|c|}{Transaction Execution Stage}\\
\hline
Proxy, remote and direct transactions& 23& 18\\
Leaving money with the agent& 18& 20\\
Modify transaction size and/or location& 10& 9\\
Advance transactions& 5& 10\\
System circumvention& 7& 6\\
Agent-enabled credit& 2& 5\\
\hline
\multicolumn{3}{|c|}{KYC Stage} \\
\hline
Relationship-based KYC & 20& 1\\
Reduced KYC using ID number only& 19& 0\\
Third-party ID use& 12& 0\\
Complaisant agents& 7& 0\\
\hline
\multicolumn{3}{|c|}{Transaction Confirmation Stage}\\
\hline
Reliance on agent confirmation & 4& 14\\
Get recipient's verbal confirmation& 1& 18\\
Alternative confirmation (Check balance)& & \\
\hline
\multicolumn{3}{|c|}{Post-transaction Execution} \\
\hline
Agent-assisted reversal & 7& 8\\
Multiple transaction  attempts& 1& 3\\
\hline
\multicolumn{3}{|c|}{\textbf{(n): number of respondents who mentioned}} \\
\hline
 \end{tabular}
\label{table:tab1workarounds}
\end{table}

\begin {table*} [htbp]
\caption{Top seven motivations for pursuing workarounds}
\centering
\begin {tabular} {|p{5cm}|l|l|p{9cm}|}
\hline
\textbf{Motivation}& \textbf{Ke (n)}& \textbf{Tz (n)}& \textbf{Sample Quote}\\
 \hline
 \hline
Convenience and time-saving& 22& 26&  ``I had to leave [the money]. I didn’t want to wait because I was in a hurry" (KE19) \\
\hline
Navigating MoMo costs&15 & 21& ``I was trying skip the transaction charges because when you put it into your phone you will be charged the fees to complete that transaction" (TZ20)  \\
\hline
Unavailable service (Network downtime and Insufficient float)& 10& 14& ``I went to an agent and he didn’t have float and so I left money and the [phone] number on a paper and deposited for me later." (TZ11) \\
\hline
Surveillance concerns& 16& 16& ``There is an issue with the withdrawal. I cannot go to someone [an agent] I know because they can send someone to follow me. If you want to withdraw such amounts, you do it in town where nobody knows you" (KE19) \\
\hline
Physical ID challenges& 18& 1& ``[The requirement to have and ID] is an issue because sometimes you forget" (KE31) \\
\hline
Navigating KYC denial of service& 15& 2& ``I do not have an ID, they[the agent] know I do not have an ID. They do not ask me a lot of questions. If the agent sees me, they just deposit the money without asking" (KE32) \\
\hline
Usability issues (Data entry, complex interfaces etc)&15 & 16& ``I was withdrawing money but unfortunately I entered the wrong agent number and so the money went to another agent." (TZ24) \\
\hline
 \end{tabular}
\label{table:tab2workaroundmotivations}
\end{table*}

\subsubsection{Considerations in choice of agents}
While factors like distance from their home, float, and size of transaction were important considerations in choosing an agent, users also considered the perceived security based on the agent's physical location (n=23), and indicated that they preferred to use specific agents regularly (n=58). The top reason for this preference was trust (n=47).  For most, this trust stemmed from knowing the agent and perceiving them as honest. 
``\textit{If you know each other, that means you have built the trust}" (TZ13). 
Other reasons for trust were characterized by a utilitarian perspective---what the user could do with their agent---and were similar to the other reasons for using specific agents regularly like easier recourse, traceability, being assured of float, informal transaction arrangements and perceived confidentiality by the agent (Table \ref{table:regularagents}).
Overall, we observed that users choose agents largely based on security and/or convenience. Unlike the study by Chamboko et al, \cite{chamboko2021role} we did not observe any major considerations around gender. When participants mentioned a preference for agents of a particular gender (n=10) it was generally due to their sense of which agents were more personable.

\begin {table*} [htbp]
\caption{Reasons why users use agents regularly}
\centering
\begin {tabular} {|p{2.7cm}|p{1.5cm}|p{10.8cm}|}
\hline
\textbf{Reason}& \textbf{(n)}& \textbf{Excerpt}\\
\hline
Trust& 47& ``When we are dealing with something like money, the word trust has to be there." (KE07)\\
\hline
Being known& 34& ``When you call them at that time or any other time, they'll just accept your request because they already know that you are their customer" (KE02)\\
\hline
Informal agreements& 25& ``If I have a money issue and I do not have money, at times they lend me." (KE20)

\medskip
``Because sometimes I have to withdraw funds when I am not present to give to someone who is present to keep things going" (TZ12)\\
\hline
Easier recourse& 12& ``Even if I made a mistake, that means I can go back to him and its easy for him to understand me." (TZ11)\\
\hline
Traceability/permanence& 10& ``I know where to find them. Yeah, like not necessarily at their work, [but also] their home" (KE08)

\medskip
``Other agents might do something bad and when you come back the next time you won't find them... Because they just have a small place arranged for this particular service, and they might even move elsewhere. I trust those ones that I can go and buy products from them as well...somebody with a shop and he is doing other transactions" (TZ27)\\
\hline
Easier KYC& 7& ``Because I do not have an ID for this place [Kenyan ID]. So I will go to the one who is further because we know each other." (KE24)\\
\hline
Loyalty& 7& ``There is an agent I prefer because they are faithful to you and you are faithful to them" (KE25)\\
\hline
Assured of float& 6& ``I prefer them because they are reliable with their work. Others would tell you to leave your number to send it later. That is what I don’t like." (TZ31)\\
\hline
Confidentiality& 5& ``The agent is supposed to be someone who can keep secrets because you don’t know what intentions one might have with you if they share information about your transactions" (TZ16)\\
\hline
\multicolumn{3}{|c|}{\textbf{*(n): number of respondents who mentioned}}\\
\hline
\end{tabular}
\label{table:regularagents}
\end{table*}

\subsubsection{Workarounds in the transaction execution phase}
% \mbox{}
% \medskip
Among workarounds that happen during transaction execution, we categorize them as follows: (1) Alternative transaction executions in which either the agent or client does not execute their intended role in the transaction, %(Fig. \ref{figDeposit} and \ref{figWithdraw}), 
(2) agents provide alternative services that are not endorsed by MoMo, and (3) clients modify the transaction characteristics (e.g., size, location).
\mypara{\textbf{Alternative execution with changes in agent/client role}}
\label{lbl:alternativeTxExec}
This first category of workarounds at the transaction execution phase entailed users working with agents to navigate various challenges by having agents either complete the transaction on their behalf, or having them act as transaction proxies. For example, to manage their transaction charges, users completed transactions through direct deposits. In this case the agent directly deposited into the recipient's account (Avoiding steps 1-3 in Figure \ref{P2P}). TZ27 said,  ``\textit{When [the agent] deposits it into my account and I send it from my account, I will be charged, and that’s why sometimes I just decide to ask the agent to do the transaction straight away.}" 
In addition to saving on cost, this workaround transferred to the agent the responsibility for potential mistakes. The transferred responsibility was motivated by the complexities of the reversal process, which we discuss further in Section \ref{workaroundpostexec}. 

Some users also reported leaving money with the agent to execute the transaction later. This was mostly motivated by convenience either because they were in a hurry and/or the MoMo network was unavailable, or when the agent did not have sufficient float (Section \ref{sec:floatdef}). ``\textit{If there is a delay... I will just have to leave the money with my details and phone number}” (KE01). Other times, insufficient float necessitated partial fulfillment of transactions where the user collected part of the cash and got the rest later. Because of the burden of liquidity management, which falls squarely on agents, \cite{eijkman2010bridges} leaving money or getting money later benefits the agent as well, but may be problematic too when users cannot cash out their money \cite{kenya2009mobile}. Some of those who experienced difficulty in using MoMo reported having an agent completing a transaction on their behalf.   


For P2P transactions, agents served as proxies as users made them cash collection points for the intended recipient. In addition to being useful when a MoMo user wanted to transact with a non-registered individual, or when the recipient did not have the required identification---such as when using a SIM card registered under another person, agents acting as proxies was also motivated by the need to save on transaction charges. This workaround circumvented steps 2-4 in Figure \ref{P2P}, by allowing the user to cash out at recipient's agent instead. “\textit{Maybe you have a friend, who needs like a hundred shillings, instead of sending money to the friend, to avoid that transaction cost, you just withdraw in a certain agent then you tell your friend to go and collect}” (KE17).

While the intended design of MoMo involved transacting in person to facilitate KYC, users modified this by sending a third party to complete the transaction on their behalf. Although this sometimes entailed sending the person to make a deposit, more participants indicated that they would send the proxy to collect physical cash once they had remotely initiated the cash-out. Sometimes, the individual who was sent was the intended final recipient. Many participants stated that they completed transactions without being physically present, mostly for convenience's sake.
\mypara{\textit{Privacy and security issues in alternative execution}}Using direct transactions means that the agent received all the transaction notifications and that the user had no means of recourse in the event of a problem as they had no proof of the completed transaction. The alternative execution workarounds also exacerbated the need to share more personal information with both agents and other parties acting as proxies. For example, in direct transactions, Alice has to share with her agent Bob's personal details, and when leaving money with the agent to transact later, users reported leaving their personal details such as their ID, phone numbers, and names with the agent. These were often written ``on a piece of paper.'' Alternatively, users gave a proxy some or all of the following details to share with the agent for purposes of verifying the transaction: their names, ID and phone number, amount transacted, and the transaction confirmation message that they received from the MoMo provider. Remote and proxy transactions therefore thwarted the KYC measures put in place by MoMo providers that require the customer to show up in person at the agent to facilitate identification and authentication (Figure \ref{figDeposit} and \ref{figWithdraw}). As a result, some participants reported that the agent kept their personal details such as their phone number and ID number to facilitate such transactions when they were not physically present. This is similar to a previous finding on cybercafe managers keeping their patrons' passwords \cite{munyendoeighty}.
\mypara{\textbf{Alternative services: agent loans and advance transactions}}
For this second category of workarounds at the transaction execution phase, the findings show that in addition to facilitating CICO services, agents offered loans and advance transactions to users. These interactions were not accompanied by any formal agreements. TZ07 indicated that he gets a loan from the agent to settle employee wages when ``\textit{working at the site and I need to pay the workers and I have shortage of money.}" KE09 even reported that she can get ``\textit{a certain amount of money, and return it at the end of the month.}"

On the other hand, advance transactions allowed the agent to disburse e-money or physical cash before the actual MoMo transaction. ``\textit{When you are in a hurry, you can even go to the agent and take some cash, then withdraw later}" (KE04). KE16 also said that she would call the agent and ``\textit{tell him to deposit some money for me. `When I come back, I will give it to you'.}"  Instead of getting cash, TZ02 shared that they can get an airtime advance.  ``\textit{I can go and tell her to send an airtime bundle to me, she does so and then I will give her the money later.}"
\mypara{\textbf{Modifying transaction characteristics}}
\label{sec:TxModification}
In this third category of workarounds, depending on what challenge they were trying to address, users modified transaction characteristics altering the size, location, or channel of transaction execution. Users reported modifying the transaction size to save on transaction fees. For example, we observed that users in Kenya split their transaction to smaller payments that did not attract any charges, as opposed to sending the whole amount as a lump sum. ``\textit{[Say] I want to send you Ksh 300. If I send you like a hundred separately three times, that will save on cost because sending Ksh 100 is free}" (KE08). In Tanzania the providers imposed fees on all P2P transactions, even of the smallest value. However, we still observed a different way of saving on cost through the use of alternative transaction channels that circumvented the MoMo system. This was enabled by the use of person-to-business (P2B) payment channels instead of the normal withdrawal channels through the agent's store number. In this case, the consumer did not incur any charges---even for large amounts. ``\textit{if you want to withdraw money and you don’t want to be charged a lot they tell you to use \textit{LIPA Number}\footnote{LIPA number is a merchant-specific number that is reserved for settling payment for other goods and services that the agent may provide. The consumer pays no service charge.}}" (TZ18). Users also modified the transaction size and location for privacy and security reasons, which we discuss separately below. 
\medskip
\mypara{{\textit{Privacy/security issues in modification of transaction characteristics}}} Many users were apprehensive about agents knowing their transactions details. ``\textit{I don’t feel good because you can’t know someone’s intentions. [The agent] can know and tell someone else and then you are followed up and robbed}" (TZ32). In prior work, users in Nigeria expressed similar sentiments, reporting the fear of being robbed at the point of cash withdrawal \cite{mogaji2022dark}. 
Participants therefore indicated that they sometimes changed how much they transacted and this afforded them some perceived security/privacy. For KE08, ``\textit{if the transaction is very large, I don’t have to transact it all at once. Maybe I can divide it into two, maybe three times. Maybe do one transaction here, then another in another agent like that. You can do that for security reasons.}" While many users alluded to some fear of robbery, others reported social concerns as a reason for desiring privacy from the agent. For example KE16 said, ``\textit{when you are transacting a small amount, there is shame [especially] when I am a male and I go to a female agent.}"

In addition to changing the transaction size, some users reported changing the location where they transacted to achieve more security/privacy. TZ33 shared that, ``\textit{If for instance this week I have withdrawn a large amount from [one agent], and if next week I am withdrawing almost the same amount of money then, I would not withdraw from [the same agent] this time.}" Some of those who reported feeling  embarrassed about sharing their transaction amount with the agent said that they would travel further to a place where they are not known.  TZ07 explained, ``\textit{I might feel shy to withdraw TShs 6,000\footnote{One US dollar was approximately equivalent to 2340 Tanzanian Shillings at the time of writing this paper.} and would say, `I don’t want my nearest agent to know'.  [So] I will go very far where no one knows about me and withdraw.}" This same participant brought up the benefit of ATMs when compared to transacting at an agent. ``\textit{If I go to withdraw 10,000 at the ATM, it is my secret, or 200,000, it's my secret.}"
\subsubsection{Workarounds in the KYC phase}
\label{Sec:KYCworkarounds}
\mbox{}
\\
KYC practices in Kenya and Tanzania were implemented differently at the time of our study.  In Kenya, MoMo users were required to show their physical ID to the agent whenever they transacted. Tanzania did not have similar requirements at the time of the study. We therefore skipped the questions addressing user perceptions about using  IDs to transact in Tanzania. Thus any sentiments that arose emerged organically as seen in (Table \ref{table:tab1workarounds}). When asked how they felt about the requirement for an ID to transact, KE06 said, ``\textit{It's good to provide IDs because most people are fraudsters.}"  Participants from Tanzania indicated that physical ID was a requirement in the past, but these checks happened rarely at the time of the study, depending on the size of the transaction and the agent. However, they too were concerned about the risks this might present: ``\textit{I am using a SIM card which is not mine... and nobody asks about it. What if I stole that SIM card? Therefore, [the ID] is very important}" \footnote{Up to 18\% of SIM cards in LMICs are registered under a third-party's ID, usually because the people have no ID of their own.}(TZ07). These responses suggest users' beliefs regarding the importance of KYC for security. TZ16 shared a negative experience stemming from lack of identification. ``\textit{I told you of how money was withdrawn from my account. If they required to be shown ID that would not have happened because the SIM card was registered with my name.}"

While users expressed beliefs about the importance of KYC, many of them also felt that KYC was ``cumbersome," ``frustrating," and ``not good, because they require the ID every time." Further analysis of the data revealed that KYC using a physical ID card presented multiple challenges that we summarized as being forgotten, lost, or simply not owning one. As a result, individuals adopted workarounds to navigate the physical ID challenges and the potential denial of service from falling short of requirements. We discuss the two categories of KYC workarounds that our findings surfaced.
\\
\mypara{\textbf{Agent-adjusted KYC}}
By transacting at agents they already used regularly, participants said they avoided  having to show an ID because the agent knew them. KE19 said, ``\textit{It's not a must if the agent knows you... I carry my ID when I want to withdraw money in a place where I am not known, maybe in town or places that are far from home.}" Some agents instead acted out of goodwill, or users looked for agents with lax KYC practices. KE06 urgently needed some money because they were taking their child to school ``...\textit{I went to an agent who refused to help with the transaction because I had no ID. I was upset...I went to another agent to whom I explained the situation to and told him/her that I had memorized the ID number. The agent asked me if I was sure the ID was mine about three times. I said yes...The agent allowed me to transact}." 
\mypara{\textbf{Alternative identity proofing}}
% \medskip
In place of the standard KYC procedures, users and agents sometimes adopted alternative identity proofing in which users: (a) gave their ID number instead of producing the physical ID card, (b) used a third-party ID, or (c) relied on third-party recommenders. KE08 remarked that,  ``\textit{...some agents don’t even bother asking for the physical ID. They just ask for the number.}” Some of those who did not use a SIM card registered in their own name presented the ID of the person under whom the SIM card was registered. ``\textit{I used my cousin’s ID because I had already registered the line with that ID, so I had to show his ID because the name that came was not my name}" (KE15).
When users could not present the third party's physical ID, they either provided the ID number or a copy of the  ID. ``\textit{I was using my friend's ID. I had his photocopy}" (KE16). Two participants from Kenya mentioned ``identifying themselves through other means" like ``\textit{a friend who can confirm that this so and so}" (KE12).
\mypara{{\textit{Privacy/security issues in KYC workarounds}}}
 In addition to agents keeping customer information for purposes of transacting in their absence, as we discussed earlier (\ref{lbl:alternativeTxExec}, the KYC workarounds modified what was provided and how it was provided and these were divergent from the KYC processes as stipulated. Agent-adjusted KYC and using a third-party ID changed what was provided. By using third-party recommenders and an ID number in place of the physical card, users and agents modified how KYC was achieved. The purpose of showing a physical ID in Kenya was to facilitate authentication by the agent, who compared the name on the ID with the name appearing on the agent's transaction summary. This was essentially the name under which the SIM card was registered. Lax KYC procedures by agents exacerbated the privacy and security risks arising from KYC workarounds. The findings show that not all agents implemented KYC in a standard manner. Whether an agent asks or does not ask you for an ID \textit{``depends on the place you usually go to deposit or withdraw money}" (KE05). For example \textit{``in case I am in Kayole,\footnote{Kayole is a low-income neighborhood in Nairobi, Kenya.} I have to have an ID, and when I am in Kibera,\footnote{Kibera is a low-income and largely informal settlement neighborhood in Nairobi, Kenya.} it’s not a must}" (KE20). MoMo users also felt that agents were not always vigilant with regards to implementing KYC. ``\textit{They are not keen}" (KE16). ``\textit{Sometimes I am a boy and I have gone with a girl’s ID}" (KE15) or ``\textit{they would ask `what is your Name?’ I tell them ‘XXX YYY.' They can see I am a male, but I mentioned a female name and nobody cares. Whether I am a thief, I stole her SIM card or whatever, nobody cares and it is actually very concerning}" (TZ07).

\subsubsection{Workarounds at transaction confirmation}
\label{sec:tx_confirmation_workarounds}
\mbox{}
\\
When asked about how they knew that a transaction was complete, the two most common ways users mentioned were the transaction confirmation message by the MoMo provider (n=61) and checking their balance (n=31). However, due to network downtime, ``\textit{...the message sometimes delayed}" (TZ31). Sometimes, only the balance checking option was available. Unfortunately in addition to challenges with timeliness of the message, these two methods were not always dependable. ``\textit{Some messages might be fake}" (TZ27) (We discuss these fraud-related concerns further in Section \ref{sec:fraud}) and ``\textit{to check balance, they charge some amount. It is not free}" (TZ12). As a result, users designed other transaction confirmation practices namely: reliance on agent confirmation and getting the recipient's verbal confirmation.
\mypara{\textbf{Reliance on agent for confirmation}} Users relied on agent communication in many instances. When they were transacting at a distance, some users depended on the agent to let them know ``\textit{if there is a problem.''}  KE05 explained, \textit{``...that’s why I prefer to go to one agent who is close}" (KE05). In remote and direct transactions where the user did not receive the transaction notifications, users waited until the agent told them the name that had appeared on their side of the transaction summary. ``\textit{When I give [the agent] the money to send to someone, when he mentions the recipient’s name and it matches the one that I am sending to, then I know it has gone through}" (TZ06). As an additional precaution, TZ24 said, ``\textit{what I usually do is that I don’t tell the agent the recipient’s name that will appear, I would wait until the agent tells me ‘is it so and so name?’}"
\mypara{\textbf{Getting recipient's verbal confirmation}}In addition to getting confirmation from the agent, some participants also called the recipient to confirm that they had received the money. This was especially common when the user asked the agent to directly top up the recipient's balance (see Section \ref{lbl:alternativeTxExec}). ``\textit{I would call the person I am intending to send money to before [to let them know], that I will be sending money through an agent and not through my number}" (TZ13). %"\textit{when the person who I am sending the money confirms 'Yes, I have seen [the money' then I leave}" (TZ01).
These multiple confirmation methods seemed to be necessary ``...\textit{because there are times that when you go to transfer money to someone else, the name comes up but when you ask the person whether they received it, they say ‘no, I have not yet received it'}" (TZ17).

\subsubsection{Workarounds at the optional transaction reversal phase}
\label{workaroundpostexec}
At the post-execution stage, the main workaround we observed was related to transaction reversal. The need for reversal mostly arose due to challenges in data-entry. The most common mistakes that users cited were entering the wrong agent number and sending to the wrong recipient. For such erroneous transactions, MoMo providers provided a self-serve reversal process. However, these reversal processes were not always clear. ``\textit{I had a challenge reversing it. So I went to the agent and asked if they could help.}" (KE1). They also did not always result in users getting back their money since success depended on the cooperation and honesty of the unintended recipient. If the latter had already withdrawn the money or transferred it out of their mobile wallet, a user could not get it back. Even in instances when the reversal process was successful, users did not get the money back immediately.``\textit{I had to wait for seven days}" (TZ09). In general, the cost of making a mistake was high for users as aptly indicated by KE15, ``\textit{I saw that when you make a mistake, you will lose money so you have to now be keen.}"



\subsection{Q2: Dealing with reduced privacy/security}
\label{sec:enablers}
As seen through the findings in Section \ref{sec:habitsandpractices}, the user workarounds were often accompanied by more risks and reduced security/privacy. As a result users took further measures to mitigate such risks. Based on user responses, we identified two categories of measures:  using a familiar agent, and measures to limit personal risks. 

\subsubsection{Using a familiar agent} 
Close to half of the participants (n=36) believed that the agent agreed to help because they were a customer. Assent by the agent was deemed necessary for ``courtesy" and ``building friendship" because some of these agents were neighbors: ``\textit{...we live close to each other}" (KE24), and because \textit{``If he helped me it’s easy to send other people to his place for transactions}" (TZ09).

While the perception that success was based on a ``customer being king'' philosophy, a majority of participants (n=51) mentioned that knowing the agent gave a sense of security. 
``\textit{I felt it was not good [to leave the agent the money] but, I just trusted anyway...because I usually use that same agent}" (KE01). This concept of knowing the agent as an enabler was also evident in the way users reported ``doing the safe thing'' when they did not know the agent. For example, with regards to waiting to see the transaction confirmation message, TZ09  said, ``\textit{If the agent is a stranger, I wait until I receive it.}"

\subsubsection{Measures to limit personal risk}Participants spoke about actions they took to limit personal risk. Most were similar to those already discussed as workarounds during transaction execution: i.e, changing the transaction size or location depending on the perceived risk.
TZ16 said, ``\textit{...I always look at the amount, I wouldn’t have left one million shillings [for the agent to deposit later] because I would have got sick, if it got lost.}" People who lacked technical literacy and relied on agent assistance mentioned withdrawing all the money from their wallet. While they could not navigate the interface to know their balance they could count the money to ensure it was the full sum of what the sender had indicated. In proxy and remote transactions, users and agents designed alternative ways of limiting risk including calling the agent ahead of the proxy's arrival to authorize them to disburse money to a proxy, or the agent would call the sender. Some users even went ahead and introduced their elected proxies to the agent, and this made authentication in future proxy transactions unnecessary. Users also limited their risk by involving proxies less in transactions that would required them to share information such as their MoMo PIN or to send the proxy with the phone. As a result, most users reported using proxies to collect the cash after initiating the transaction remotely. ``\textit{I sent my brother...because I was just making a deposit and not a withdrawal that would require me to give them the PIN. If I gave him the PIN maybe he can go and withdraw everything}" (KE25). Users also seemed to exercise caution in who they choose as a proxy. More often than not, the proxy was a trusted or known individual.

\subsection{Q3: Implications and challenges of user-agent interactions and  practices} 
\label{sec:risksandchallenges}
Some of the risks and challenges that arise from using MoMo relate to privacy and security. Others stem directly from workarounds, while others are a result of the current MoMo system structure (as noted by \cite{mogaji2022dark}).

\subsubsection{Unofficial and agent-determined fees}
When users worked with the agent to circumvent official transaction costs, they were often subject to ad-hoc agent-determined charges. ``\textit{You may find that he tells you “Please give me Tshs 500 because I have gained nothing from this transaction.” You have to do it because if you do not, next time you won’t be helped}" (TZ01). For some users, there was no other option because the lack of interoperability made the workaround necessary. ``\textit{When you go to send money in another mobile network, the agent asks you to add a certain amount.... They claim that to send money in a different network is a little bit costly}" (TZ34). This risk of being overcharged has been noted before \cite{martin2019mobile,mogaji2022dark}.  

\subsubsection{Reduced data privacy}
As highlighted in previous sections, the alternative ways of transacting such as proxy,  remote, and direct transactions resulted in more sharing of personal and private transaction data (often in unsafe ways such as writing on paper) with multiple parties including agents and proxies to facilitate the various workarounds. 

 \subsubsection{Concerns about disclosed data}We asked participants about how their data is recorded and processed and how they felt about giving personal details at the agents. Close to half (n=29) were concerned about disclosed data and indicated feeling ``insecure." 
 When we asked why they felt this way, most users cited the possibility of agent misconduct including sharing it with third parties like ``politicians" and ``random strangers." ``\textit{I think the agents are using those details wrongly. Maybe that’s why people receive scamming messages}" (TZ15). KE25 added,  ``\textit{...agents can use your ID to register other people.}" TZ27 had experienced the same, ``\textit{I discovered when I was going to register for my SIM cards that there were two other SIM cards that were registered under my ID}." Some of these concerns with agents have been noted before \cite{martin2019mobile}.
For others, the hesitation was purely a matter of privacy. Agents got to know the full names of their customers and some were not happy with this for reasons like not wanting to be called by their full name. 
\mypara{Willingness to share data} Despite these concerns, many participants were willing to share their data (n=45); half of these had previously indicated being concerned about disclosed data. The participants gave various reasons for why they shared their data. Most said that ``it was a requirement" for service access and they ``had no other way." There were those who felt that the information they left including the transaction data trails were important for security so that ``\textit{if an issue happens...a person cannot deny you because you have written [your details] there}" (TZ26).
In contrast to those who identified with the need for security, there are those who said they were not worried because ``the money is theirs" and they ``have not stolen it" or because they believed they were not sharing any sensitive information, where sensitive mostly referred to their PIN number. ``\textit{I get scared of sharing my PIN number but as for such other details like contact number and names, no, I don’t worry.}" (TZ06). The belief that agents were trustworthy and the belief that as users they were anonymous to the agent also made some users confident to give their information. 
``\textit{There is no problem when they know...because I withdraw and leave. They don’t know me}" (KE36).

\subsubsection{Risks from MoMo-related fraud}
\label{sec:fraud}
Some users were concerned about MoMo-related fraud. The types of fraud described by users  were similar to previously-documented fraud \cite{buku2017fraud} \cite{akomea2019control}; many of these were enabled by  social engineering, such as fake transaction reversals and masquerading as a customer care representative or agent. An example of the former is where fraudsters tricked the target user to believing they had sent money erroneously and asked them to send it back. Common reasons that the masquerading caller gave for the call involved resolving account anomalies. As the user complied by giving the required information, the fraudster was able to execute fraud. 
Participants also acknowledged that access to personal information as well as transaction trails facilitated fraud like SIM swaps and other forms of identity theft. For example, users in Kenya strongly associated identity theft frauds with the loss or unauthorized access of a person's ID. ``\textit{If you lose your ID, somebody can replace your SIM card, if that person knows your number, your last withdrawal, and balance, he might call customer care and give these details. After that they will withdraw, and later when you replace your line, you find there is nothing [in your wallet]}" (KE17). 


\section{Experimental Results And Discussion}
\label{sec:results}
The results presented in this section test the performance of the Autoencoder model. We evaluate our model using the performance metrics: accuracy, precision, recall, and F1 score, defined as follow: 

\vspace{-5mm}
\begin{align*}
    Accuracy &= \frac{TP+TN}{TP+TN+FP+FN}
\end{align*}
\vspace{-3mm}
\begin{align*}
    Precision &= \frac{TP}{TP+FP}
\end{align*}
\vspace{-3mm}
\begin{align*}
    Recall &= \frac{TP}{TP+FN}
\end{align*}
\vspace{-3mm}
\begin{align*}
    F1 ~Score &= 2 \times \frac{Precision \times Recall}{Precision + Recall}
\end{align*}

In our experiments, a \textit{positive} outcome means an abnormal activity was detected, whereas a negative outcome means a normal activity was detected.
True Positive (TP) refers to an abnormal activity that was correctly classified as abnormal. 
True Negative (TN) refers to a normal activity that was correctly classified as normal.
False Positive (FP) refers to a normal activity that was misclassified as abnormal.
and False Negative (FN) refers to an abnormal activity that was misclassified as normal.

The success of our model is based on measuring the reconstruction error that is produced by any given data point. Figure \ref{fig:recon} shows an example of reconstructed data overlaid the original data that was inserted into the model. %To be clear, the values shown in this graph are not measurements of the reconstruction loss that are shown in Figure \ref{fig:thresh}. This figure only shows the normalized temperature measurement from each data point, that is why they are not being represented in degrees.
In this figure, extremely severe dips in temperature denoted by the blue line (representing our original data) can be noticed. The data reconstructed by the model, represented by the red line, does not dip as much as the original data. This is because our model was not able to reconstruct these points accurately due to the fact that they are anomalies. The reconstruction loss (i.e. different between the original and the reconstructed data), where the model recognizes normal or abnormal behavior, is shown in Figure \ref{fig:thresh}. The figure shows a visualization of the mean-squared-error (MSE) generated by the model after it was given each data point within the test data set. The dotted red line denotes the threshold determined as mentioned in Section \ref{sec:ml-model}. Each data point's actual label is represented either by blue color to denote a normal behavior or red color to denote an anomaly and every data point that lies above the threshold was classified as anomalous. This figure illustrates our model's capability to detect the majority of anomalies by measuring the MSE produced by each data point.

Overall, as shown in Figure \ref{fig:aeresults}, our model was able to attain high performance with over $90\%$ in all metrics. The precision is lower than the recall metric which shows that the model produced slightly more false positives than false negatives. In a smart farming environment, a higher rate of false positives would not have a dramatic affect on the productivity of day to day operations and would ensure a higher number of anomalous situations are detected. A rather problematic situation would be if there were more false negatives than positives. A user would much prefer receiving an alert when nothing was wrong than not receiving an alert and enabling potential harm to occur to the crops and hardware. In the future, we hope to further decrease the number of false positives and negatives in order to fine-tune an overall more accurate model. This can be done by using more training samples.


% \begin{table}[!t]
%     \caption{Results}
%     \centering
%     \begin{tabular}{| c | c | c | c |}
%     \hline
%     Accuracy & Precision & Recall & F1\\ [0.5ex] % inserts table %heading
%     \hline
    
%     98.98\% & 90\% & 92.95\% & 91.45\% \\
    
%     \hline
%     \end{tabular}
%     \label{table:results}
% \end{table}

\begin{figure}[t!]
    \centering
    \includegraphics[width=8cm]{figures/aeresults-v2.png}
    \caption{Performance metrics for Autoencoder Model}
    \label{fig:aeresults}
\end{figure}


\section{Conclusion and Future Work}
\label{sec:conclusion}
Our approach has shown that smart farming anomaly detection can be done at an extremely accurate level by using an Autoencoder. Our approach would allow vast scalability by only requiring non-anomalous data for training. Greenhouses provide controlled environments that create consistent conditions for crops and data collection. Environments such as this are a perfect use case for our approach since the performance of an Autoencoder can drastically improve when provided with large amounts of non-anomalous data. Our approach shows that it may not be entirely necessary for machine learning professionals that are working on anomaly detection within smart farming to be highly concerned with developing models that are trained using labeled data that contains both normal and anomalous data. 

In the future, we will explore more anomaly detection models in order to optimize the system's performance. Once the best model has been selected, the architecture could be brought online to be used and tested with the added interactions of Internet connectivity. By bringing the system online we will have the ability to alert users of potential threats or anomalous behavior. These alerts could be coupled with actuators such as fertilization, watering, video monitoring, etc. The introduction of cameras can be ``used to calculate biomass development and fertilization status of crops" \cite{Walter6148}. They can also be used to allow the system-user to monitor their property from afar. We plan to introduce photo and video monitoring as one of our next steps to improve security and broaden our scope.

\section{Acknowledgements}
\label{sec:ack}
We thank TTU Shipley Farms for allowing to use greenhouse, and setup smart farm testbed. Dr. Brian Leckie and his group were instrumental in our system and early stages of data collection. We are thankful to Ms. Deepti Gupta to provide helpful guidance on dealing with time-series, correlated data and gave input on our model selection. This research is partially supported by the NSF Grant 2025682 at TTU.


\section*{Acknowledgements}

This publication has been made possible by the participation of more than 7000 volunteers in the Radio Galaxy Zoo project.  The data in this paper are the result of the efforts of
the Radio Galaxy Zoo volunteers.
Their efforts are individually acknowledged at {\url{http://rgzauthors.galaxyzoo.org}}. 


This publication makes use of data product from the Karl G. Jansky Very Large Array. The National Radio Astronomy Observatory is a facility of the National Science Foundation operated under cooperative agreement by Associated Universities, Inc.

This publication makes use of data products from the Wide-field
Infrared Survey Explorer (WISE) and the Spitzer Space Telescope. The WISE is a joint project of the University of California, Los Angeles, and the Jet Propulsion Laboratory/California Institute of Technology, funded by the National Aeronautics and Space Administration. SWIRE is supported by NASA through the SIRTF Legacy Program under contract 1407 with the Jet Propulsion Laboratory. 

We also thank the staff of the GMRT that made possible the observations TGSS is based upon. GMRT is run by the National Centre for Radio Astrophysics of the Tata Institute of Fundamental Research.

FdG is supported by the VENI research programme with project number 1808, which is financed by the Netherlands Organisation for Scientific Research (NWO). Partial support for LR comes from US National Science Foundation grants AST-1211595 and AST-1714205 to the University of Minnesota. HA benefitted from grant DAIP 980/2016-2017 of the University of Guanajuato. Parts of this research were conducted by the Australian Research Council Centre of Excellence for All-sky Astrophysics (CAASTRO), through project number CE110001020.





%%%%%%%%%%%%%%%%%%%%%%%%%%%%%%%%%%%%%%%%%%%%%%%%%%

%%%%%%%%%%%%%%%%%%%% REFERENCES %%%%%%%%%%%%%%%%%%


\bibliographystyle{mnras}
\bibliography{bibcars}


%%%%%%%%%%%%%%%%%%%%%%%%%%%%%%%%%%%%%%%%%%%%%%%%%%

%%%%%%%%%%%%%%%%% APPENDICES %%%%%%%%%%%%%%%%%%%%%

\appendix
	\section{Position Angle as Shear}
		\label{sec:shear}
		In this Appendix, we focus on an approach to the study of the position angles based on an alternative formalism. The study of other directional quantities over large scales through the use of spin-2 spherical harmonics is well established. Examples of such quantities are the polarization $P$ of the CMB or the cosmic shear field $\gamma$ \citep[e.g.,][]{Collaboration2015, Hikage2011}. However, in our attempts, the detailed properties of the position angle datasets forced a sampling of the correlation functions and power spectra that did not allow us to resolve features like the minimum in Fig.~\ref{fig:Sn}.  In particular, the main complications are the partial sky-coverage, the low source density and the predisposition to systematic effects of interferometric measurements.
		
		Although the products presented in this appendix are inconclusive, we describe here our implementation of the cosmic shear statistics, so that it can be applied when suitable samples will become available.
		
		Cosmic shear is usually detected through the analysis of the spin-2 field
		
		\begin{equation}
			\gamma = \gamma_1 + i\gamma_2,
			\label{eq:gamma}
		\end{equation}
		
		where $\gamma_1, \gamma_2$ are defined on a local Cartesian reference frame. Under rotation of an angle $\Phi$ the field transforms as $\gamma\to \gamma~e^{2i\Phi}$. The shear is usually estimated as the ensemble average of galaxy ellipticities $\varepsilon$ \citep{Kirk2015}
		
		\begin{equation}
			\varepsilon =  \frac{1-q}{1+q} (\cos 2\alpha_p + i \sin 2\alpha_p)
			\label{eq:epsi}
		\end{equation}
		\begin{equation}
			\gamma = \braket{\varepsilon}
		\end{equation}
		
		In this definition, $\alpha_p$ is the major axis position angle of the optical galaxy and $q$ is the ratio between the major and minor axes. We define the tangential and cross-component ellipticity $\varepsilon_t$ and $\varepsilon_\times$ with respect to a direction as the projection of the ellipticity in the two  $+/\times$ components: (1) parallel or perpendicular to it (2) oriented at $45^\circ$ or $-45^\circ$. For a direction defined by the polar angle $\Psi$
		
		\begin{gather}
			\epsilon_t = -\operatorname{Re}\{e^{-2i\Psi}\epsilon \}
			\label{eq:et}
			\\
			\epsilon_\times = -\operatorname{Im}\{e^{-2i\Psi}\epsilon\}
			\label{eq:ecross}
		\end{gather}
		
		In our sign convention, a positive $\varepsilon_t$ corresponds to tangential alignment, i.e., the position angle $\alpha_p$ and the direction $\Psi$ are parallel, while a negative value corresponds to radial alignment, i.e., the two are perpendicular \citep{Kilbinger2015}. 
		
				
		The literature contains multiple statistics involving the shear field. In particular, we focus on those described in  \cite{Schneider2002}, \cite{Eifler2010} and implemented by the software \textsc{treecorr}\footnote{\url{https://github.com/rmjarvis/TreeCorr}} \citep{Jarvis2004}. 
		
		When evaluating a two point correlation function, the two components $\gamma_t$ and $\gamma_\times$ are defined with respect to the direction connecting the sources. These components are commonly estimated by neglecting both the curvature of the sphere and the parallel transport operation described in Section~\ref{sec:PT}. Because of this, we limit our analysis in this Section to distances smaller than $5^\circ$, corresponding to about $0.1$ radians.
		
		We introduce the two-point correlation functions
		
		\begin{gather}
			\xi_{tt}(\varphi) = \braket{\gamma_t \gamma_t}
			\label{eq:xitt}
			\\
			\xi_{\times \times}(\varphi) = \braket{\gamma_\times \gamma_\times}
			\label{eq:xicc}
			\\
			\xi_+ (\varphi)= \braket{\gamma_t \gamma_t} + \braket{\gamma_\times \gamma_\times} 
			\\
			\xi_-  (\varphi)= \braket{\gamma_t \gamma_t} - \braket{\gamma_\times \gamma_\times}
		\end{gather}
		
		where the averages are computed over every possible pair of sources with angular distance $\varphi$. The tangential and cross-component shear are defined as in Eq. \eqref{eq:et}, \eqref{eq:ecross}. The two correlation functions $\xi_{tt}$ and $\xi_{\times \times}$ distinguish between different shear configurations, according to the provided definitions of $\gamma_t$ and $\gamma_\times$.
		Furthermore, we define $\overline{\gamma}(\varphi)$ as the mean shear inside a circular aperture of radius $\varphi$. The variance of this quantity can then be estimated directly from the correlation function $\xi_+$
		
		\begin{gather}
			\braket{|\overline{\gamma}|^2}(\varphi) = \int \frac{d\vartheta\vartheta}{2\varphi^2} 
				\xi_+(\vartheta) S_+ \left( \frac{\vartheta}{\varphi}\right)
			\label{eq:Gsq}
		\end{gather}
		
		The definition of the weight function $S_+$ and a more detailed introduction to the top-hat shear dispersion are given by \cite{Schneider2002}.
		
		
		Using the representation introduced in Eq. \eqref{eq:epsi}, the position angle $\alpha$ can be written as
		
		\begin{equation}
			\gamma^\alpha = \cos 2\alpha + i \sin 2 \alpha
		\end{equation} 			
		
		Under a rotation of an angle $\Phi$ the quantity $\gamma^\alpha$ behaves exactly like the shear field, $\gamma^\alpha \to \gamma^\alpha~e^{2i\Phi}$. This justifies the extension to $\gamma^\alpha$ of the statistics defined for $\gamma$. Since we want to study the alignment configuration of the position angles, we should point out that no averaging is involved. In our analysis $\gamma^\alpha$ takes the place of the shear field $\gamma$ and not of the ellipticity $\varepsilon$.    
		
		In the presence of a global systematic effect we rewrite the correlation functions \eqref{eq:xitt} and \eqref{eq:xicc} as
		
		\begin{gather}
			\xi_{tt} (\theta) = \braket{\gamma^\alpha_t \gamma^\alpha_t} - \xi^n_{tt}
			\\
			\xi_{\times \times} (\theta) = \braket{\gamma^\alpha_\times \gamma^\alpha_\times}
			-
			\xi^n_{\times \times},
		\end{gather}
		
		where we subtracted a noise bias, to be estimated through simulated random data sets containing only the noise. The expression for the estimator \eqref{eq:Gsq} must be computed from these unbiased correlation functions.
		
		We do not assume any particular model for our analysis and we set as our primary objective the detection of a positive correlation. In its absence we expect the two-point correlation functions and the dispersion to be consistent with the noise on every scale $\varphi$.
		
		The function $\braket{|\overline{\gamma^\alpha}|^2}(\varphi)$ is closely related to $S_n$ (Eq. \eqref{eq:S}) since both of them estimate the average dispersion (or dispersion squared) of the position angles. The first one considers spherical caps of constant aperture radius $\varphi$, while the second considers caps with a constant number of sources $n$. The dispersion $\braket{|\overline{\gamma^\alpha}|^2}(\varphi)$ has the advantage of probing precise angular scales, but for non-uniformly distributed samples its value can be easily skewed by the sources in low density regions. Another drawback, due to our chosen implementation, is the lack of parallel transport in its computation.  
	
		
		\subsection{Products}
		
		In Fig.~\ref{fig:RGZweak} and~\ref{fig:TGSSweak} we plot the statistics presented in Eq. \eqref{eq:xitt}, \eqref{eq:xicc} and \eqref{eq:Gsq} for the Radio Galaxy Zoo and TGSS samples. The noise bias has already been subtracted. The covariance matrices are generated using the method described in Section~\ref{sec:RandomDatasets}.
		
		Since the diagonal terms in the covariance matrix \eqref{eq:cov} are two orders of magnitude higher than the non-diagonal terms, we can confirm that the measurements of the statistics $\xi_{tt}$ and $\xi_{\times \times}$ for different angular scales are in fact independent. The same is not true for the dispersion $\braket{|\overline{\gamma^\alpha}|^2}$. The reason for this is the same as the one discussed in Section~\ref{sec:S} for the statistics $S_n$. 	
		
		
		The correlation functions $\xi_{tt}(\theta)$ and $\xi_{\times \times}(\theta)$ are consistent with normally distributed noise. This result was checked using common statistical tests: (1) Shapiro-Wilk (2) $\chi^2$ (3) Anderson-Darling (4) two-tailed Kolmogorov-Smirnov. All of them returned p-values $> 0.05$. For the two $\braket{|\overline{\gamma^\alpha}|^2}$ we obtain the Mahalanobis distances $d^2 = 17.28$ and $d^2 = 12.05$. Given the number of degrees of freedom ($k=12$), both correspond to p-values $>0.05$, meaning that these results are also consistent with the noise. 
		
		Nothing conclusive about the alignment configuration can be stated, since both $\xi_{tt}$ and $\xi_{\times \times}$ are consistent with zero. The dispersion $\braket{|\overline{\gamma^\alpha}|^2}$ is also found to be consistent with the noise. This is not unexpected, since the estimator in Eq. \eqref{eq:Gsq} is simply a convolution of $\xi_{+} = \xi_{tt} + \xi_{\times \times}$ and a weight function. If $\xi_{+}$ is found to be largely consistent with zero, the same should be true for $\braket{|\overline{\gamma^\alpha}|^2}$.
		
		
		\begin{figure}
			\includegraphics[width=0.45\textwidth]{files/RGZweak.pdf}
			\caption{Weak lensing statistics for the Radio Galaxy Zoo sample: the two point correlation functions $\xi_{tt}(\varphi)$, $\xi_{\times \times}(\varphi)$ as a function of the distance $\varphi$ and the top-hat shear dispersion $\braket{|\overline{\gamma^\alpha}|^2}(\varphi)$ as a function of the aperture radius $\varphi$.}
			
			\label{fig:RGZweak}
		\end{figure}
		
		\begin{figure}
			\includegraphics[width=0.45\textwidth]{files/TGSSweak.pdf}
			\caption{Weak lensing statistics for the TGSS sample: the two point correlation functions $\xi_{tt}(\varphi)$, $\xi_{\times \times}(\varphi)$ as a function of the distance $\varphi$ and the top-hat shear dispersion $\braket{|\overline{\gamma^\alpha}|^2}(\varphi)$ as a function of the aperture radius $\varphi$.}
			
			\label{fig:TGSSweak}
		\end{figure}		
	
			Finally, the down-crossing of $\xi_{tt}$ around the angular scale of $3\deg$ seems to suggest a change in the configuration of the alignment. The limited number of data points and the overall consistency with zero of the correlation function do not allow for a conclusive statement. However, assuming the downcrossing to be a feature, we can assign a significance to this observation. The probability of obtaining $8$ consecutive positive datapoints is found to be less than $0.005$. 

%If you want to present additional material which would interrupt the flow of the main paper,
%it can be placed in an Appendix which appears after the list of references.

%%%%%%%%%%%%%%%%%%%%%%%%%%%%%%%%%%%%%%%%%%%%%%%%%%


% Don't change these lines
\bsp	% typesetting comment
\label{lastpage}
\end{document}

% End of mnras_template.tex