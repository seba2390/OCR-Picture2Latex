% mnras_template.tex
%
% LaTeX template for creating an MNRAS paper
%
% v3.0 released 14 May 2015
% (version numbers match those of mnras.cls)
%
% Copyright (C) Royal Astronomical Society 2015
% Authors:
% Keith T. Smith (Royal Astronomical Society)

% Change log
%
% v3.0 May 2015
%    Renamed to match the new package name
%    Version number matches mnras.cls
%    A few minor tweaks to wording
% v1.0 September 2013
%    Beta testing only - never publicly released
%    First version: a simple (ish) template for creating an MNRAS paper

%%%%%%%%%%%%%%%%%%%%%%%%%%%%%%%%%%%%%%%%%%%%%%%%%%
% Basic setup. Most papers should leave these options alone.
\documentclass[fleqn,usenatbib]{mnras}

% MNRAS is set in Times font. If you don't have this installed (most LaTeX
% installations will be fine) or prefer the old Computer Modern fonts, comment
% out the following line
%\usepackage{newtxtext,newtxmath}
% Depending on your LaTeX fonts installation, you might get better results with one of these:
%\usepackage{mathptmx}
%\usepackage{txfonts}

% Use vector fonts, so it zooms properly in on-screen viewing software
% Don't change these lines unless you know what you are doing
\usepackage[T1]{fontenc}
\usepackage{ae,aecompl}


%%%%% AUTHORS - PLACE YOUR OWN PACKAGES HERE %%%%%

% Only include extra packages if you really need them. Common packages are:
\usepackage{graphicx}	% Including figure files
\usepackage{amsmath}	% Advanced maths commands
\usepackage{amssymb}	% Extra maths symbols
\usepackage{multicol}   % Multicolumns
\usepackage{bm}         % Vectors

%%%%%%%%%%%%%%%%%%%%%%%%%%%%%%%%%%%%%%%%%%%%%%%%%%

%%%%% AUTHORS - PLACE YOUR OWN COMMANDS HERE %%%%%

% Please keep new commands to a minimum, and use \newcommand not \def to avoid
% overwriting existing commands. Example:
%\newcommand{\pcm}{\,cm$^{-2}$}	% per cm-squared
\newcommand{\braket}[1]{\left\langle #1 \right\rangle}

%%%%%%%%%%%%%%%%%%%%%%%%%%%%%%%%%%%%%%%%%%%%%%%%%%

%%%%%%%%%%%%%%%%%%% TITLE PAGE %%%%%%%%%%%%%%%%%%%

% Title of the paper, and the short title which is used in the headers.
% Keep the title short and informative.
\title[Cosmological Alignment of Radio Sources]{Radio Galaxy Zoo: Cosmological Alignment of Radio Sources}

% The list of authors, and the short list which is used in the headers.
% If you need two or more lines of authors, add an extra line using \newauthor
\author[O. Contigiani et al.]{O. Contigiani,$^{1}$\thanks{E-mail: contigiani@strw.leidenuniv.nl}
F. de Gasperin,$^{1}$
G. K. Miley,$^{1}$ 
L. Rudnick$^{2}$,
H. Andernach,$^{3}$ \newauthor
J. K. Banfield,$^{4, 6}$ 
A. D. Kapi\'nska,$^{5, 6}$
S. S. Shabala$^{7}$,
O. I. Wong$^{5, 6}$
\\
% List of institutions
$^{1}$ Leiden  Observatory, Leiden  University,  P.O. Box 9513, 2300 RA,  Leiden,  the Netherlands
\\
$^{2}$ Minnesota Institute for Astrophysics, University of Minnesota,
116 Church St. SE, Minneapolis, MN 55455
\\
$^{3}$ Departamento de Astronom\'ia, DCNE, Universidad de Guanajuato, Apdo.
Postal 144, CP 36000, Guanajuato, Gto., Mexico
\\
$^{4}$ Research School of Astronomy and Astrophysics, Australian National University, Weston Creek, ACT 2611, Australia
\\
$^{5}$ International Centre for Radio Astronomy Research (ICRAR), The University of Western Australia, 
\\
M468, 35 Stirling Hwy, Crawley WA 6009, Australia
\\
$^{6}$ ARC Centre of Excellence for All-sky Astrophysics (CAASTRO), Australia
\\
$^{7}$ School of Mathematics \& Physics, University of Tasmania, Private Bag 37, Hobart, Tasmania 7001, Australia
}

% These dates will be filled out by the publisher
\date{Accepted XXX. Received YYY; in original form ZZZ}

% Enter the current year, for the copyright statements etc.
\pubyear{2017}

% Don't change these lines
\begin{document}
\label{firstpage}
\pagerange{\pageref{firstpage}--\pageref{lastpage}}
\maketitle

% Abstract of the paper
\begin{abstract}
We study the mutual alignment of radio sources within two surveys, FIRST and TGSS. This is done by producing two position angle catalogues containing the preferential directions of respectively $30\,059$ and $11\,674$ extended sources distributed over more than $7\,000$ and $17\,000$ square degrees. 
The identification of the sources in the FIRST sample was performed in advance by volunteers of the Radio Galaxy Zoo project, while for the TGSS sample it is the result of an automated process presented here.
After taking into account systematic effects, marginal evidence of a local alignment on scales smaller than $2.5^\circ$ is found in the FIRST sample. The probability of this happening by chance is found to be less than $2$ per cent. Further study suggests that on scales up to $1.5^\circ$ the alignment is maximal. For one third of the sources, the Radio Galaxy Zoo volunteers identified an optical counterpart. Assuming a flat $\Lambda$CDM cosmology with $\Omega_m = 0.31, \Omega_\Lambda = 0.69$, we convert the maximum angular scale on which alignment is seen into a physical scale in the range $[19, 38]$ Mpc $h_{70}^{-1}$. This result supports recent evidence reported by Taylor and Jagannathan of radio jet alignment in the $1.4$ deg$^2$ ELAIS N1 field observed with the Giant Metrewave Radio Telescope.  The TGSS sample is found to be too sparsely populated to manifest a similar signal.
\end{abstract}

% Select between one and six entries from the list of approved keywords.
% Don't make up new ones.
\begin{keywords}
galaxies: statistics -- galaxies: jets -- radio continuum: galaxies -- cosmology: observations -- large-scale structure of Universe
\end{keywords}

%%%%%%%%%%%%%%%%%%%%%%%%%%%%%%%%%%%%%%%%%%%%%%%%%%

%%%%%%%%%%%%%%%%% BODY OF PAPER %%%%%%%%%%%%%%%%%%

\section{Introduction}
\label{sec:Introduction}


The goal in top-$\size$ recommendation is to recommend to each
consumer a small set of $\size$ items from a large collection of
items~\cite{cremonesi2010performance}.  For example, Netflix may want
to recommend $\size$ appealing movies to each consumer.  Collaborative
Filtering (CF)~\cite{herlocker2002empirical,lee2012comparative} is a
common top-$\size$ recommendation method.  CF infers user interests by
analyzing partially observed user-item interaction data, such as user
ratings on movies or historical purchase
logs~\cite{kanagal2012supercharging}. The main assumption in CF is that
users with similar interaction patterns have similar interests.


Standard CF methods for top-$\size$ recommendation focus on making  suggestions  that accurately reflect the user's preference history. However, as  observed in previous work,  CF recommendations are generally biased toward  popular items, leading to a rich get richer effect~\cite{vargas2014improving,steck2011item}.  The major reasons for this are \textit{popularity bias} and \textit{sparsity} of CF interaction data (detailed in Section~\ref{sec:related-work}). In a nutshell, to maintain  accuracy, recommendations are generated from the dense regions of the data,  where the popular items lie.  

However,  accurately suggesting popular items, may not be satisfactory for the consumers. For example, in Netflix, an accuracy-focused movie recommender may recommend ``Star Wars: The Force Awakens'' to users who have seen ``Star Wars: Rogue One''.  But, those users are probably already aware of ``The Force Awakens''. Considering additional factors, such as novelty of recommendations,  can lead to more effective suggestions~\cite{cremonesi2010performance,Castells2015,zhang2008avoiding,ziegler2005improving,zhang2012auralist}. 
%Second, accuracy-focused models typically achieve a   overall item-space coverage across their recommendations,  whereas high item-space coverage helps providers of the items increase revenue
%, users satisfaction since they are  likely already aware of or can find these items on their own.  

Focusing on popular items also adversely affects the satisfaction of  the providers of the items. This is because  accuracy-focused models typically achieve a  low overall item space coverage across their recommendations, whereas   high item space coverage helps providers of the items increase their revenue~\cite{vargas2014improving,Castells2015,adomavicius2011maximizing,anderson2006thelongtail, yin2012challenging,adomavicius2012improving}.
%accuracy-focused models typically achieve a

In contrast to the relatively small number of popular items, there are copious  {\it long-tail\/} items that have fewer observations (e.g., ratings) available. More precisely,  using the Pareto  principle (i.e.,~the $80/20$ rule),  long-tail items can be defined as items that generate the lower $20\%$ of observations~\cite{yin2012challenging}. Experimentally we found that these items correspond to almost $85\%$ of the items in several datasets (Sections~\ref{sec:Notation} and \ref{sec:Experiments}). %Table~\ref{tab:DatasetStatsticsSmall})


As previously shown, one way to improve the novelty of top-$\size$ sets is to recommend interesting long-tail items~\cite{cremonesi2010performance,ge2010beyond}.  The intuition  is that since they have fewer observations available,  they are more likely to be unseen~\cite{Kaminskas:2016:DSN:3028254.2926720}.  
 %For example, in online commerce,  newly added items are long-tail items that are yet to be discovered.  
Moreover, long-tail item promotion also results in higher overall coverage of the item space%, which increases profits for providers of the items
~\cite{vargas2014improving,Castells2015,zhang2008avoiding,zhang2012auralist,adomavicius2011maximizing,anderson2006thelongtail,yin2012challenging,jambor2010optimizing}. Because long-tail promotion reduces accuracy~\cite{steck2011item}, there are trade-offs to be explored.


%original submitted to ICDE
%This work studies three aspects of top-$\size$ recommendation: accuracy, novelty, and item-space coverage, and examines their trade-offs. In most previous work, predictions of a base recommendation system are re-ranked to handle their trade-offs~\cite{adomavicius2012improving,jambor2010optimizing,zhang2013personalize,wang2009portfolio}. Due to performance considerations, however, these techniques are not customized per user. For example,  parameters that balance the trade-off between novelty and accuracy are cross-validated at a global level.  This can be detrimental since users have varying preferences for  objectives such as long-tail novelty. We explore how to  automatically infer  user  preference for long-tail novelty, and how to leverage  it to correct  the popularity bias in standard recommender models. Our work does not rely on any additional contextual data, although such data, if available, can help promote newly-added long-tail items~\cite{agarwal2009regression,Saveski:2014:ICR:2645710.2645751}.

This work studies three aspects of top-$\size$ recommendation: accuracy, novelty, and item space coverage, and examines their trade-offs. In most previous work, predictions of a base recommendation algorithm are \textit{re-ranked} to handle these trade-offs~\cite{adomavicius2012improving,jambor2010optimizing,zhang2013personalize,wang2009portfolio}. The re-ranking models are computationally efficient but suffer from two drawbacks. First, due to performance considerations,  parameters that balance the trade-off between novelty and accuracy  are not customized per user. Instead they are cross-validated at a global level.  This can be detrimental since users have varying preferences for  objectives such as long-tail novelty. Second,  the re-ranking methods are often limited to a specific base recommender  that may be sensitive to dataset density. 
As a result, the datasets are pruned and the problem is studied in dense settings~\cite{adomavicius2012improving,ho2014likes}; but real world  scenarios are often sparse~\cite{kanagal2012supercharging,liu2017experimental}.   
% Because  dataset density can impact the performance of most base recommenders (like R-SVD), which in turn affects the performance of the re-ranking model, 

\iffalse
We address these limitations by directly inferring  user  preference for long-tail novelty  from interaction data.  This  allows us to customize the re-ranking  per user, and design a \textit{generic} framework, which resolves the second problem. In particular, since the long-tail novelty preferences are estimated independently of any base  recommender model, we can  plug-in an appropriate base recommender w.r.t. the dataset sparsity.% including ones that are more suitable for sparse settings.  

Modelling  user  preference for  long-tail novelty using only item popularity statistics, e.g., the average popularity of rated items as in~\cite{jugovac2017efficient}, disregards additional information like whether the user found the item interesting and the long-tail preferences of other users  of the items. \iffalse To incorporate them, we introduce the notion of  \emph{item long-tail importance}. Both  user long-tail preferences and item long-tail importance are dependent:  a user has high preference for discovering long-tail items if she is interested in important long-tail items, and an item that is associated with many of these kinds of users is likely to be more important.  We propose a joint optimization framework to directly learn,  from interaction data, both the users' long-tail preferences and the  items' long-tail importance. \fi
We propose an optimization approach that  incorporates  this information and  directly learns,  from interaction data, the users' long-tail novelty preferences.

Next, we use these learned preferences  to design a  top-$\size$ recommendation framework thats is generic, and provides customized balance between accuracy, novelty, and coverage. We refer to it as framework as GANC.  Using GANC, we design a novel algorithm, {\it Ordered Sampling-based Locally Greedy (OSLG)\/}, that relies on the learned long-tail novelty preferences  to scalably correct for popularity bias. Our work does not rely on any additional contextual data, although such data, if available, can help promote newly-added long-tail items~\cite{agarwal2009regression,Saveski:2014:ICR:2645710.2645751}. In summary:
\fi

We address the first limitation by directly inferring  user  preference for long-tail novelty  from interaction data.   Estimating these  preferences  using only item popularity statistics, e.g., the average popularity of rated items as in~\cite{jugovac2017efficient}, disregards additional information, like whether the user found the item interesting or the long-tail preferences of other users  of the items. We propose an approach that  incorporates  this information and  learns the users' long-tail novelty preferences from interaction data.

This approach allows us to customize the re-ranking  per user, and  design a \textit{generic} re-ranking framework, which resolves the second limitation of prior work. In particular, since the long-tail novelty preferences are estimated independently of any base recommender, we can  plug-in an appropriate one w.r.t. different factors, such as the dataset sparsity.

Our top-$\size$ recommendation framework, \textbf{GANC}, is \textbf{G}eneric, and provides customized balance between \textbf{A}ccuracy, \textbf{N}ovelty, and \textbf{C}overage. % Moreover, based on the learned long-tail novelty preferences, we also design a novel algorithm, {\it Ordered Sampling-based Locally Greedy (OSLG)\/}, that relies on the learned long-tail novelty preferences  to scalably correct for popularity bias. 
Our work does not rely on any additional contextual data, although such data, if available, can help promote newly-added long-tail items~\cite{agarwal2009regression,Saveski:2014:ICR:2645710.2645751}. In summary:

%Consider  the following toy example:
\vspace{-0.2cm}
\begin{table}[htb]
\centering
\scriptsize
%\small
\begin{tabular}{ccccccc} 
%\toprule
%&\multirow{2}{*}{}&\multicolumn{7}{c}{Ratings}\\
& & \cellcolor{blue!35}$w_1$ &\cellcolor{blue!18} $w_2$ & $\dots$ &\cellcolor{blue!8} $w_{89}$  &\cellcolor{blue!8} $w_{99}$   
\\
&   &$i_1$&$i_2$&$\dots$&$i_{89}$&$i_{90}$\\ 
\cmidrule(r){3-7} 	 
%\midrule
\cellcolor{red!35}$\theta_1$  &$u_1 $   &5 &   & $\dots$ &  &   \\
\cellcolor{red!28}$\theta_2$  &$u_2$     &5 &    & $\dots$ &  &  \\
 $\theta_3=?$  &$\bf u_3$  &5 &  &   $\dots$ &  &  \\
\cellcolor{red!10}$\theta_4$ & $u_4$  &  &5   & $\dots$ & &\\ 
\cellcolor{red!10}$\theta_5$ & $u_5$  &  & 5  & $\dots$ & &\\ 
$\theta_6=?$  & $\bf u_6$ & &5  &      $\dots$& &  \\ 
 & & $\hdots$  &$\hdots$   &$\hdots$   &$\hdots$   &$\hdots$  \\
%\midrule 
\cmidrule(r){3-7} 	 
\multicolumn{2}{c}{item pop.}  & 3  & 3  & $\dots$ &50&60\\  
%\bottomrule
%$ f_i$    &3  &3  &1  &3  &1  &2  \\  \hline
\end{tabular}
%#.
\caption{Simplified user-item interaction data. The user long-tail novelty preference ($\theta_u$), item long-tail importance weight ($w_i$) are highlighted. Darker colors indicate larger values. } \label{tab:example}
\end{table} 
\vspace{-0.2cm}
\begin{example}  
In Table~\ref{tab:example}, we are interested in estimating $\theta_3$ and $\theta_6$,  the long-tail preference of users $u_3$ and $u_6$ who have each rated a single movie. Additional ratings for other users  are not included here.  Considering only rating information, we observe $i_1$ and $i_2$ are  equally popular $|\mathcal{U}_{i_1}^{\trainset}| = |\mathcal{U}_{i_2}^{\trainset}|=3$, and $r_{31}=5$ and $r_{62}=5$. Using Eq.~\ref{eq:tfidf-risk}  we have $\theta_3 = \theta_6$. However, if we were given the long-tail preferences of the each item's user set, specifically that $u_1$ and $u_2$ have high long-tail preference (darker red), while $u_4$ and $u_5$ have lower long-tail preference (lighter red), we could conclude $i_1$ is a more important long-tail item compared to $i_2$ (indicated by a darker blue shade for $w_1$), and we expect  $\theta_3 \geq \theta_6$.

% On the other hand, if we knew that $u_4$ and $u_5$ have lower long-tail preference, we could conclude $i_2$ is a  less significant long-tail item. Therefore, However, if we  consider the long-tail preferences of other users, we may reason differently.    We need another variable $w_i$ which captures this information. 
%we would conclude that $u_3$ has higher long-tail preference compared to $u_6$, since the users $i_1$ is a more prominent long-tail item. 

% Relying only  on item popularity information, we would  conclude   $u_3$ and $u_6$ have equal long-tail preference, since $i_1$ and $i_2$ are  equally popular. However, considering  the second column,  long-tail preference of users,  long-tail importance for each item,  which captures the long-tail preference of its users. Since  that  both users of $i_1$ have high long-tail preference while  the users of $i_2$ have lower preference,  we may conclude $i_1$ is a more important long-tail item compared to $i_2$. Therefore, $u_3$'s long-tail preference should be at least as large as $u_6$'s preference. Specifically, consider two  items $i_1$ and $i_2$, with the following rating data: $i_1=\{u_1:5, u_2:5, u_3:5 \}$, $i_2=\{u_4:5, u_5:5, u_6:5\}$.  

%Table~\ref{tab:example} shows  simplified rating data. We want an estimate of the long-tail preference of $u_3$ and $u_6$, who have each  rated a single movie.  Relying only  on movie popularity information, we would  conclude   $u_3$ and $u_6$ have similar long-tail preference, since $m_1$ and $m_2$ are  equally popular. However, considering the long-tail preferences of other users of those movies, we may reason differently: since $u_1$ and $u_2$ have high long-tail preference, and $u_4$ and $u_5$ have low long-tail preference, $m_1$ is a more prominent long-tail item compared to $m_2$. Therefore, it is likely that $u_3$ has higher long-tail preference compared to $u_6$.considering the long-tail preferences of other users of those movies, we may reason differently.  For example, 
\label{ex:running}
\end{example}



%------------------------------

\iffalse
\begin{example}
Table~\ref{tab:example} shows rating data for a simplified system. %Note the user-item interaction matrix is sparse.
For this example, we define popular movies as those that have received  three or more ratings; $\{m_1, m_2, m_4\}$ are popular and  $\{m_3, m_5, m_6\}$ are niche movies. We observe $u_1$ and $u_3$  have rated relatively popular movies (risk-averse) while $u_2$ and $u_4$ have rated niche movies (risk-loving). 
\label{ex:running}
\end{example}

\begin{table}[htb]
\centering
\scriptsize
\begin{tabular}{ccccccc} 
\toprule
			&$m_1$ &$m_2$   &$m_3$    &$m_4$   &$m_5$ &$m_6$  \\ \hline 
$u_1 $ &5  &4  & - &-  &-  &-   \\
$u_2$  &-  &-  &-  &-  &5  &5   \\
$u_3$  &-  &4  &-  &5  &-  &-   \\
$u_4$  &-  &-  &3  &-  &-  &4   \\ 
$u_5$  &5  &-  &-  &3  &-  &-   \\ 
$u_6$  &4  &2  &-  &4  &-  &-   \\ 
\bottomrule
%$ f_i$    &3  &3  &1  &3  &1  &2  \\  \hline
\end{tabular}
\caption{User-Movie rating data} \label{tab:example}
\end{table}

It is essential to consider consumer characteristics in designing recommender systems so that they promote long-tail items to the right group of users and spread demand evenly between hit and niche items.  

\fi





%------------------------------
\iffalse
\begin{table}[htb]
\centering
\scriptsize
\begin{tabular}{ccccccc} 
\toprule
			&$m_1$ &$m_2$   &$m_3$    &$m_4$   &$m_5$ &$m_6$  \\ \hline 
$u_1 $ &\textbf{5}  & \textbf{4}  &\textcolor{gray}{ 1.2} &-  &-  &-   \\
$u_2$  &-  &-  &-  &-  & \textbf{5}  &\textbf{5}   \\
$u_3$  &-  &\textbf{4}  &-  &\textbf{5}  &-  &-   \\
$u_4$  &-  &-  &\textbf{3}  &-  &-  &\textbf{4}   \\ 
$u_5$  &\textbf{5}  &-  &-  &\textbf{3}  &-  &-   \\ 
$u_6$  &\textbf{4}  &\textbf{2}  &-  &\textbf{4}  &-  &-   \\ 
\bottomrule
%$ f_i$    &3  &3  &1  &3  &1  &2  \\  \hline
\end{tabular}
\caption{User-Movie rating data} \label{tab:example}
\end{table}
% $\mathcal{P}^1= \{ \mathcal{P}_1^1 \{i_1,i_2,i_3\}, \mathcal{P}_2^1:\{i_2,i_3,i_5\}  \}$
 %$\mathcal{P}^2= \{ \mathcal{P}_1^2: \{i_1,i_2,i_3\}, \mathcal{P}_2^2:\{i_2,i_5,i_6\}  \}$
 %$\mathcal{P}^3= \{ \mathcal{P}_1^3: \{i_7,i_8,i_9\}, \mathcal{P}_2^3:\{i_{10},i_{11},i_{12}\}  \}$
\begin{table}[htb]
\centering
\tiny
\begin{tabular}{ccc} 
\toprule
		&$u_1$&$u_2$  \\ \hline 
$\mathcal{P}^1 $ & $\{i_1,i_2,i_3\}$ & $\{i_2,i_3,i_5\} $ \\
$\mathcal{P}^2$ & $\{i_1,i_2,i_3\}$ & $\{i_2,i_5,i_6\} $ \\
$\mathcal{P}^3$ & $\{i_7,i_8,i_9\}$ & $\{i_{10},i_{11},i_{12} \}$ \\
\bottomrule
%$ f_i$    &3  &3  &1  &3  &1  &2  \\  \hline
\end{tabular}
\caption{Top-$\size$ allocations to users.} \label{tab:paretoExamples}
\end{table}
\fi


\iffalse
When considering long-tail items, it is important to consider consumers' willingness  to explore niche or unpopular items and their propensity towards similar items. In particular, they can be characterized by their  {\it risk degree\/} and {\it focusing degree\/}, respectively.  We compute these estimates  based on historical rating information. The following example further describes these notions in the context of movie rating data. 

\begin{example}  
Table~\ref{tab:example} shows rating data for a simplified system with $6$ users, $6$ movies, and $3$ genres. $m_i^{j}$ implies that movie $m_i$ belongs to genre $j$. Note the user-item interaction matrix is sparse. 
  For this setting, we define popular movies as those that have received  three or more ratings; $\{m_1, m_2, m_4\}$ are popular and  $\{m_3, m_5, m_6\}$ are niche movies. We now profile the users according to their risk and focusing degree. E.g., $u_1$ has rated relatively popular movies belonging to the same genre (risk-averse, high focusing degree); $u_2$ has rated niches movies in the same genre (risk-loving, high focusing degree); $u_3$ has rated popular movies in two different genres (risk-averse, low focusing degree), and $u_4$ has rated niches movies in two different genres (risk-loving, low focusing degree). 
\label{ex:running}
\end{example}
\begin{table}[htb]
\centering
\tiny
\begin{tabular}{ccccccc} 
\toprule
			&$m_1^{1}$ &$m_2^{1}$   &$m_3^{2}$    &$m_4^{3}$   &$m_5^{3}$ &$m_6^{3}$  \\ \hline 
$u_1 $ &5  &4  &-  &-  &-  &-   \\
$u_2$  &-  &-  &-  &-  &5  &5   \\
$u_3$  &-  &4  &-  &5  &-  &-   \\
$u_4$  &-  &-  &3  &-  &-  &4   \\ 
$u_5$  &5  &-  &-  &3  &-  &-   \\ 
$u_6$  &4  &2  &-  &4  &-  &-   \\ 
\bottomrule
%$ f_i$    &3  &3  &1  &3  &1  &2  \\  \hline
\end{tabular}
\caption{User-Movie rating data} \label{tab:example}
\end{table}
It is essential to consider these consumer characteristics in designing recommender systems so that they promote long-tail items to the right group of users and spread demand evenly between the hit and niche items.  
\fi
\iffalse
\begin{center}
\begin{figure*}[tp]
%\scalebox{0.5}{%
\resizebox{1\textwidth}{!}{%
%\small%\addtolength{\tabcolsep}{5pt}% below sums to 8
\begin{tabularx}{1.5\textwidth}{>{\hsize=2.5\hsize}X>{\hsize=2.5\hsize}X>{\hsize=0.5\hsize}X>{\hsize=0.5\hsize}X>{\hsize=0.5\hsize}X>{\hsize=0.5\hsize}X>{\hsize=0.5\hsize}X>{\hsize=0.5\hsize}X}
    \multirow{12}{*}{\includegraphics[scale=0.3]{codeForExample/popularity-movie.png}} & \multirow{12}{*}{\includegraphics[scale=0.3]{codeForExample/scatterplot.png}} & & & & & & \\
%   & &               &       &       &       &       &       \\
    & &\multicolumn{1}{l|}{}               &$m_1^{g1}$   	&$m_2^{g1}$    	&$m_3^{g2}$    &$m_4^{g2}$      &$m_5^{g3}$    \\ \cline{3-8}%\hline
    & &\multicolumn{1}{l|}{u1}          &5  &5  &-  &-   &-  \\
    & &\multicolumn{1}{l|}{u2}    		&-  &-  &4  &4  &5  \\
    & &\multicolumn{1}{l|}{u3}   			&1  &2  &1  &-  &-   \\
    & &\multicolumn{1}{l|}{u4}     		&1  &-  &-  &-  &-  \\
    & &               &       &       &       &       &       \\
    & &               &       &       &       &       &       \\
    & &               &       &       &       &       &       \\
    & &               &       &       &       &       &	\\
    \\
\end{tabularx}}
\caption{User-Movie interaction data a) Popularity-Movie histogram b)Movie genres/clusters c) User-Movie rating data} \label{fig:example}
\end{figure*}
\end{center}
\fi



%We propose a novel approach that allows us to  promote long-tail items in a targeted manner, thereby improving the novelty of top-$\size$ sets, the overall item-space coverage across recommendations, while maintaining reasonable levels of accuracy.

%Next, we integrate these learned preferences  in a generic  top-$\size$ recommendation framework to provide customized balance between accuracy and coverage.

%sequentially make recommendations, while adjusting its parameters with regard to the set of top-$\size$ recommendations made so far. However, since  sequential parameter updates  cause  scalability issues, we propose a sampling based algorithm. This variant of our framework, called {\it Ordered Sampling-based Locally Greedy (OSLG)\/},  allows us to  correct for the popularity bias in recommendations with regard to individual user long-tail preferences. 

%ICDE submission
%Our framework differs with  prior work in the following aspects:  unlike~\cite{adomavicius2011maximizing,adomavicius2012improving,zhang2013personalize,ho2014likes},  the long-tail preference personalization in our framework is learned rather than optimized using cross-validation or parameter tuning. In other words, our personalization method is independent of the underlying base  recommendation models.  Moreover, our framework is  generic. This enables us to  plug-in several base recommenders, and evaluate their  effectiveness without requiring  extensive tuning for the accuracy and coverage trade-off. 


%\vspace{-2.8pt}
\begin{itemize}

\item  We examine various measures for estimating user long-tail novelty preference in Section~\ref{sec:lt-pref} and formulate an optimization problem  to directly learn users' preferences for long-tail  items from interaction data in Section~\ref{sec:learning-lt-pref}. %In addition, we introduce several heuristics for measuring the user preference for less common items from historical rating data.% 

\item  We integrate the user preference estimates into GANC %, a generic re-ranking framework that provides customized balance between accuracy, novelty, and coverage 
(Section~\ref{sec:RiskbasedReranking}), and  introduce {\it Ordered Sampling-based Locally Greedy (OSLG)\/}, a scalable algorithm that relies  on user long-tail preferences to correct the popularity bias (Section~\ref{sec:optimizationAlgorithm}).
%We introduce OSLG, a scalable algorithm that relies  on user long-tail preferences to  maximize item space coverage \textcolor{red}{while maintaining acceptable levels of accuracy} (Section~\ref{sec:optimizationAlgorithm}).

\item   We conduct an extensive empirical study and evaluate performance from  accuracy, novelty, and coverage perspectives (Section~\ref{sec:Experiments}).  We use five  datasets with varying density and difficulty levels. %:  Netflix, MovieTweetings, and MovieLens (100K, 1M, 10M). 
  In contrast to most related work,  our evaluation considers realistic settings that include a large number of infrequent  items and users. %This enables us to study the impact of  data density on the performance trade-offs of several  state of the art top-$\size$ recommendation algorithms. %   %,  and use the all-items ranking protocol~\cite{steck2013evaluation,vargas2014improving}, where performance is measured using all items with train data. to evaluate the performance of several  state of the art top-$\size$ recommendation algorithms 
 
\item Our empirical results confirm that the performance of re-ranking models is impacted by the underlying   base recommender and the dataset density. Our generic approach enables us to easily incorporate a suitable base recommender to devise an effective solution for both dense and sparse settings. In dense settings, we use the same base recommender as existing re-ranking approaches, and we outperform them in accuracy and coverage metrics. For sparse settings, we plug-in a more suitable base recommender, and devise an effective solution that is competitive with existing top-$\size$ recommendation methods in accuracy and novelty. 

%Directly estimating the long-tail novelty preferences allows us to customize re-ranking per user, and  devise a generic framework.   
 
\end{itemize}

Section~\ref{sec:related-work} describes related work. Section~\ref{sec:conclusion} concludes.


\section{Sample Selection}
	\label{sec:SS}
	The November 2015 alpha version of the Radio Galaxy Zoo consensus catalogue lists the properties of $85\,151$ radio sources distributed primarily over the footprint of two surveys: FIRST and Australia Telescope Large Area Survey (ATLAS) \citep{Norris2006}. The classification was performed by volunteers, who were presented with radio images from these surveys and the corresponding infrared fields observed by the Wide-field Infrared Survey Explorer \citep[WISE;][]{Wright2010}. They were then asked to match disconnected components corresponding to the same source and recognize the infrared counterpart. A more detailed description of the project is available in \cite{Banfield2015}. 

	The Radio Galaxy Zoo represents a natural choice for our statistical analysis. Whereas components belonging to the same source are usually recognized through self-matching (i.e., cross-matching the source catalogue with itself to identify sources at a certain distance from each other) or human selection, we rely on the additional information provided by human inspection to increase the reliability of the results. Furthermore, the $5\arcsec$ nominal resolution of the FIRST images implies a high number of resolved sources, for which a preferential direction can be defined. Lastly,  the survey covers an area of about $10\,000$ square degrees and allows us to infer general properties of the radio sky, instead of a local statistical anomaly.

	For our second sample, based on the TGSS Alternative Data Release 1, no human-made classification is available. In its place, we opt for automated self-matching. The TGSS ADR1 is based on an independent reprocessing of an original 150 MHz GMRT survey performed between 2010 and 2012 and the corresponding source catalogue, released in 2016, covers $99.5$  per cent of the sky north of $-53^\circ$ declination. A more detailed description is available in \cite{Intema2016}.
	
	\subsection{Radio Galaxy Zoo}
	
			\begin{figure}
				\centering
				\includegraphics[width=0.45\textwidth]{files/exall.pdf}
				\caption{FIRST image for a typical source with morphological features superimposed. The angular extent of the source is about $\mathbf{1\arcmin10\arcsec}$. The red boxes identify the components provided by the Radio Galaxy Zoo, with the crosses indicating surface brightness peaks. The blue ellipses have major and minor axes equal to the FWHM of the fitted Gaussian model in the FIRST catalogue, and the dots are their centres. The red and dashed blue lines are the results of the orthogonal distance regression for the dots and the crosses respectively. In this particular case two or more surface brightness peaks are present and the position angle is extracted from the slope of the red line (see text for more details).}
				\label{fig:exall}
			\end{figure}	
			
			We select extended sources from the Radio Galaxy Zoo consensus catalogue and extract an elongation direction for each of them.We describe this direction with a position angle, defined as the angle east of north in the range $[-\pi/2, +\pi/2]$ between the direction itself and the local meridian. 
			
			To perform the selection and constrain the orientation, we rely on the quantities contained in the November 2015 alpha version of the Radio Galaxy Zoo consensus catalog and, occasionally, on the official FIRST catalogue presented in \cite{Helfand2015b}, version \textsc{14dec17}. Figure~\ref{fig:exall} presents these quantities in graphic form. From the Radio Galaxy Zoo catalogue we extract the areas covered by components belonging to the same source and the peak positions of the source surface brightness contained in these regions (peaks hereafter). From the FIRST catalogue we extract the Gaussian model of the source brightness contained inside the same areas. 
			An additional quantity provided by the Radio Galaxy Zoo for every morphological classification is the consensus level. This is defined as the fraction of users who voted for the specific components configuration and, in this analysis, it is used to rank distinct classifications of the same object. 
			
			Depending on the available data, different procedures are employed to extract the position angle. We define three sub-samples:
			
			\begin{description}
			\item a)  If two or more surface brightness peaks are present for a given source, we define the position angle as the slope of the orthogonal distance linear regression of the peaks, weighted according to their flux densities. Around $80$ per cent of the selected sources belong to this category. An example of such a source is provided in Figure~\ref{fig:exall}.
			
			
			\item b) For sources with only one surface brightness peak in the Radio Galaxy Zoo catalogue, but multiple Gaussian models in the FIRST catalogue, we rely completely on the latter.
			This occurs when components are not seen as separated in the Radio Galaxy Zoo because of the particular automated choice of contour levels. For this sub-sample the centres of the FIRST ellipses, weighted by their integrated flux, are fitted.
			
			\item c) 	If only one surface brightness peak is detected and the FIRST catalogue recognizes only one source inside the single component radio galaxy, we rely on the Gaussian model of the FIRST catalogue and we define the direction as the position angle of the fitted ellipse. In this case, the source must comply with a total of four criteria.
			
			First, sources must meet the conditions required to be included in the Radio Galaxy Zoo sample and be presented to the volunteers. These are aimed at selecting resolved sources with a high signal-to-noise ratio:
			
			\begin{equation}
			\frac{S_{\rm{peak}}}{S_{\rm{int}}} < 1.0 - \left( \frac{0.1}{\log S_{\rm{peak}}}\right) \textrm{ and } SNR>10,
			\label{eq:cond1}
			\end{equation} 
			where $S_{\rm{peak}}$ is the peak brightness in mJy beam$^{-1}$, $S_{\rm{int}}$ is the integrated flux density of the source in mJy and $SNR$ is the signal-to-noise ratio \citep{Banfield2015}.
			
			Secondly, we introduce two additional criteria. The minor axis $m$ of the fitted elliptical Gaussian model should be larger than $2\arcsec$ and the deviation of the ratio between the major and minor axis $r$ from unity should be highly significant:
			
			
			\begin{equation}
			m > 2\arcsec
			\textrm{ and }
			r > 1 + 7\sigma_r.
			\label{eq:cond2}
			\end{equation}
			
			The error on the major and minor axis ratio is overestimated by the quadratic sum
			
			
			\begin{equation}
			\sigma_r = r\sqrt{\left(
				\frac{\sigma_m}{m}
				\right)^2
				+
				\left( 
				\frac{\sigma_m}{M}
				\right)^2},
			\end{equation}
			
			
			where $\sigma_m$ is the empirical uncertainty on both the fitted minor axis $m$ and major axis $M$. The four conditions, \eqref{eq:cond1} and \eqref{eq:cond2}, select extended sources for which an elongation is clearly recognizable.
			
			\end{description}
	
			When both multiple Gaussian models and multiple flux density peaks are available, we choose to prioritize the peaks over the centres. Figure~\ref{fig:exall} provides an example of how the difference between the two fitted position angles is usually small.
	
			
			\begin{figure}
				\centering
				\includegraphics[width=0.5\textwidth]{files/mindisttot.pdf}
				\caption{Distribution of the angular distance between a source in the Radio Galaxy Zoo sample and its closest neighbour, before and after filtering duplicates.}
				\label{fig:mindist}
			\end{figure} 
			
			The release of the Radio Galaxy Zoo consensus catalogue used here includes every classification performed by the volunteers. Because of this, a single source might appear multiple times with different classifications. 
            To filter these duplicate entries we focus our attention on all the recognized components. For every set of overlapping components, we filter out all of the sources they belong to, except for the one with highest consensus level. The effect of this selection process can be seen in Figure~\ref{fig:mindist}, where we plot the distribution of the distances between every source and its closest neighbour. While a natural amount of clustering is expected, we find that almost half of the sources have an extremely close neighbour -- a probable duplicate. After we apply our filter the peak around $0\farcs6$ disappears.
            
 	           
            A second systematic effect inherited from the Radio Galaxy Zoo is the quantisation of the peak positions. To clearly discern its importance, we limit our attention to the sources classified as containing only two peaks and we plot the differential right ascension and declination of every pair (Figure~\ref{fig:PP}). Discretisation is more noticeable in the vertical axis, but a $1\farcs4$ binning effect is visible in both directions. The presence of pixels is caused by numerical approximations in the implementation of the World Coordinates System (WCS).  
			In our analysis, this grid-like disposition of the peaks implies discrete values of the associated position angles. To obtain a continuous distribution of the angles, we smooth out the peak positions by adding a uniformly random value in the range $[-0\farcs7, +0\farcs7]$ to both coordinates before performing the linear regression. This process pushes the influence of the effect to sub-pixel scales, eliminating its impact on the present study. However, further investigation is needed to constrain its causes. 

		 	\begin{figure}
		 		\centering
		 		\includegraphics[width=0.5\textwidth]{files/PP.pdf}
		 		\caption{Relative peak positions for entries classified as containing two peaks. Discretization is evident in the collapsed distributions.}
		 		
		 		\label{fig:PP}
		 	\end{figure}
            For the sake of consistency, the final sample presented in Figure~\ref{fig:RGZPAdist} excludes ATLAS sources and it is limited only to FIRST sources. For the same reason, we also exclude every source positioned above RA $20$ hr and below $4$ hr, since half of the observations in this region were performed after the observing array transitioned to the new JVLA configuration \citep{Helfand2015b}. 
            
			Finally, notice how the original Radio Galaxy Zoo selection in Eq.~\eqref{eq:cond1} does not include an explicit cut for artefacts. During the first run of the Radio Galaxy Zoo classification, the volunteers were presented with $3\arcmin \times3\arcmin$ fields. This corresponds to a maximum distance of $ 3\arcmin \sqrt{2} \approx 4\arcmin 12\arcsec$ between two components. To quantify the contamination from artefacts in our sample, we make use of the column $P(S)$ of the official FIRST catalogue, which indicates the probability of a source to be a sidelobe. We cross-matched our selection with the FIRST catalogue using a search radius of $4\arcmin 12\arcsec$ and we verified that 134 selected sources are part of a field containing possible sidelobes satisfying the condition $P(S)>0.1$. In principle, these artefacts might be recognized as components and influence the value of the position angles. Because of this, we exclude sources with $P(S)>0.1$ from our final Radio Galaxy Zoo sample.
            
            In Figure~\ref{fig:RGZPAdist} we plot the final distribution of the extracted position angles, together with the distributions for the three classes of sources. While we would expect these to be uniform, three peaks are visible around $30^\circ$, $-30^\circ$ and $90^\circ$. In these three directions we recognize the typical pattern that results from the three arms of the observing radio interferometer --- the Very Large Array (VLA). The same effect is visible in the FIRST images and is discussed in \cite{Helfand2015b}, where a three-directional pattern is present in the distribution of the sidelobes around bright sources. The existence of preferential angles may be related to the brightness of the weaker components, although a more detailed analysis would be required to quantify this effect. This will not affect our analysis as long as the effects are non-local.

			\begin{figure}
				\centering
				\includegraphics[width=0.45\textwidth]{files/RGZPATOT.pdf}
				\caption{Position angle distribution of the Radio Galaxy Zoo selection. On top of the total distribution (topmost histogram) the plot contains the distributions of the three sub-samples. From top to bottom: (a) in grey, (b) in red, and (c) in blue. A trimodal systematic effect is visible in the first two.}
				
				\label{fig:RGZPAdist}
			\end{figure}

            A similar pattern is discussed also in other analyses \citep[e.g.,][]{Chang2004, White2007, Demetroullas2015} based on the FIRST survey, where the effect is recognized as non position-dependent. Snapshot surveys are commonly affected by an anisotropic point spread function (PSF) and the connection to the interferometer geometry suggests this origin.
            \cite{Helfand2015b} underlines that particular care was taken in ensuring a constant PSF throughout the different observation epochs of FIRST. In particular, since the hour angle of observation affects the orientation of the pattern in the cleaned images, $90\%$ of the observations were acquired within $1.4$ hr of the local meridian.	
			
			The non-locality of the effect is verified by partitioning the data by both right ascension and declination in four equally populated quadrants. Pairwise, the four position angle distribution are found to be consistent with each other using two-sample Kolmogorov-Smirnov tests. 
            
            This first position angle catalogue contains $30\,059$ sources distributed over an area of about $7\,000$ square degrees, resulting in a number density $\sim4$ deg$^{-2}$.


		\subsection{TGSS Alternative Data Release}
		
			As opposed to the Radio Galaxy Zoo sample, this second position angle sample is based on the product of an automated source extractor. The nominal resolution of $25\arcsec$ for the TGSS images implies a lower number of extended sources with significant elongation compared to FIRST. However, the relatively steep spectrum of radio galaxy lobes and the sensitivity to extended sources of the GMRT allow TGSS to trace the lobes better than FIRST.  Hence, we focus our attention on the identification of double-lobed sources. 
			In Figure~\ref{fig:mindistTGSS} we plot the distance between each entry in the TGSS catalogue and its closest neighbour. The rightmost peak is due to the distribution of uncorrelated radio sources, while the lower peak on the left is caused by multi-component sources. The plot suggests an average distance of $1\arcmin$ between the components of a source of the latter type. A peak around the angular scale of $1\arcmin$ is not present in the Radio Galaxy Zoo catalogue because the pairing was already performed by the volunteers during the classification process.

			\begin{figure}
				\centering
				\includegraphics[width=0.5\textwidth]{files/mindistTGSS.pdf}
				\caption{Distribution of the angular distance between a source in the TGSS catalogue and its closest neighbour. The dashed red line marks the value $1\arcmin12\arcsec$.}
				
				\label{fig:mindistTGSS}
			\end{figure}
			
			We select radio galaxy candidates by self-matching the catalogue with a search radius $1\arcmin12\arcsec$ and imposing a maximum ratio of $10$ between the total fluxes of the two components \citep{VanVelzen2014}.
			To be part of the final sample both components of the pair need to satisfy additional constraints: (1) isolated (i.e., matched only to each other) (2) $SNR>10$. The position angle is then simply that of the line connecting the two components. The search radius we chose corresponds to the local minimum marked in Fig.~\ref{fig:mindistTGSS}. A larger value would introduce an artificial contamination in our double-lobed source catalogue, while a lower value would mean losing part of the genuine sources.
			
			We decide to limit our sample to a portion of the northern hemisphere to minimize the effects of an anisotropic PSF. \cite{Intema2016} reports the synthesized beam to be circular for pointings at declination higher than the GMRT latitude --- about $19^\circ$. Even between declinations of $10^\circ$ and $19^\circ$ the beam is still circular to within $1\%$. Therefore, our final TGSS sample includes only sources with declination above $10^\circ$.
			
			Figure~\ref{fig:TGSSPAdist} shows the position angle distribution of the final TGSS sample. This second position angle catalogue contains $11\,674$ sources distributed over an area of about $17\,000$ square degrees, resulting in a number density $\sim0.7$ deg$^{-2}$. We notice that unlike for the FIRST survey, no particular care was taken with respect to the PSF and its consistency throughout different pointings. However, the  complex geometry of the interferometer and longer integration times compared to FIRST result in a PSF less prone to systematic effects. 
			Table~\ref{tab:surveys} compares the different surveys and samples featured in this section. The difference between the number of sources in the two catalogues produced in this section is due to the different nature of the original surveys and the source selection process. While $85\%$ of the sources in the RGZ sample have size larger than the TGSS resolution ($25\arcsec$), only $55\%$ of them are larger this threshold and have exactly two surface brightness peaks.
			
			We can use the RGZ catalogue to predict the size of the TGSS one. If we account for the different frequencies ($1.4$ GHz for FIRST and $150$ MHz for TGSS) by adopting a nominal spectral index equal to $0.9$ \citep{Vollmer2010} and keeping in mind the sky coverage and angular resolution differences, we find that about $10^4$ sources are expected to be selected by our algorithm. This number is in line with the $11\,674$ sources found in our selection.
			
			\begin{figure}
				\centering
				\includegraphics[width=0.45\textwidth]{files/TGSSPATOT.pdf}
				\caption{Position angle distribution of the TGSS selection. Obvious systematic effects are not present.}
				
				\label{fig:TGSSPAdist}
			\end{figure}


\begin{table*}
\caption{Comparison between the different samples and source catalogs discussed in this paper.}
\label{tab:surveys}
\begin{tabular}{lccccccc}
\hline
Name & Frequency & Median RMS & SNR & Number of & Minimum & Sky & Median Redshift \\
     &           & Noise      & Threshold  & Sources   &  Resolution & Fraction & $68\%$ interval \\
     &           &[mJy beam$^{-1}$] &       &           &            &          &                  \\ 

\hline
				FIRST\,$^a$ &
				$1.4$ GHz & 
				$0.15$& 
				$5$&
				$946\,432$& 
				$5\arcsec\times5\arcsec$& 
				$26\%$&
				$2.2\pm0.9$\,$^b$ 
				\\ 
				Radio Galaxy Zoo\,$^c$  & 
				$1.4$ GHz & 
				$0.15$ & 
				$10$& 
				$82\,187$&  
				$5\arcsec\times5\arcsec$& 
				$22\%$& 
				$0.47_{-0.15}^{+0.21}$\,$^d$ 
				\\
				Radio Galaxy Zoo processed\,$^e$ & 
				$1.4$ GHz &  
				$0.15$& 
				$10$& 
				$30\,059$  &
				$5\arcsec\times5\arcsec$& 
				$19\%$& 
				$0.47_{-0.15}^{+0.20}$\,$^d$ 
				\\
				TGSS\,$^f$& 
				$150$ MHz & 
				$3.5$& 
				$7$& 
				$623\,604$ & 
				$25\arcsec\times25\arcsec$& 
				$90\%$& 
				$-$
				\\
				TGSS processed\,$^e$ & 
				$150$ MHz & 
				$3.5$& 
				$10$& 
				$11\,674$ & 
				$25\arcsec\times25\arcsec$& 
				$42\%$& 
				$-$ \\
\hline
\multicolumn{8}{l}{ $^a$ \cite{Helfand2015b}} \\
\multicolumn{8}{l}{ $^b$ Mean redshift with $68\%$ confidence levels from \cite{Chang2004}}	\\
\multicolumn{8}{l}{ $^c$ \cite{Banfield2015}} \\
\multicolumn{8}{l}{ $^d$ Only $30\%$ of the sample has a human-matched optical counterpart with known redshift} \\
\multicolumn{8}{l}{ $^e$ The selection process, aimed at selecting resolved sources to use in this study, is detailed in Section~\ref{sec:SS}} \\
\multicolumn{8}{l}{ $^f$ \cite{Intema2016}} \\
\end{tabular}
\end{table*}

\section{Statistical Analysis}
	\label{sec:SA}
	\subsection{Parallel Transport}	
		\label{sec:PT}  	
		
				
		The position angle is a directional quantity defined in the point of the celestial sphere where the corresponding source lies. In order to perform the calculation of the misalignment angle between two directions on a sphere, the notion of parallel transport should be introduced \citep{Jain2004}.
		
		We parametrize the sphere using spherical coordinates $(r, \theta, \phi)$ and we define in every point a natural orthonormal basis dictated by our coordinate system. This set of unit vectors is $(\mathbfit{e}_r, \mathbfit{e}_\theta, \mathbfit{e}_\phi)$, where the three elements point respectively towards the centre of the sphere, northward and eastward.	
		
		A source with position angle $\alpha$, determined up to a rotation of $\pi$ radians, can be identified with the unit vector
		
		\begin{equation}
			\mathbfit{v} = \cos\alpha \; \mathbfit{e}_\theta + \sin \alpha \; \mathbfit{e}_\phi
			\label{eq:v}
		\end{equation}
		
		Since the projection along the line of sight is unknown, we fix this vector to be tangent to the sphere at the point of definition.  
		The vector $\mathbfit{v}$ represents a physical quantity, whereas the definition of position angle $\alpha$ depends on the choice of coordinate system. For example, if parallels and meridians were redefined with respect to a different north pole, the vectors $\mathbfit{e}_\theta$, $\mathbfit{e}_\phi$ and the position angle $\alpha$ would change. However, the vector $\mathbfit{v}$ in Eq. \eqref{eq:v} would still describe the same direction in space.
		On a sphere, parallel transport allows us to define a coordinate-invariant inner product between two vectors, by translating one of them along arcs of great circles connecting the two.  
		
		Let us consider two tangent vectors $\mathbfit{v}_1$ and $\mathbfit{v}_2$ with position angles $\alpha_1$ and $\alpha_2$, defined respectively in $P_1 = (r_1, \theta_1, \phi_1)$ and $P_2 = (r_2, \theta_2, \phi_2)$. Both of these points belong to the same unit sphere ($r_1 = r_2 = 1$). The great circle passing through them lies on a plane perpendicular to $\mathbfit{e}_s$
		
		\begin{equation}
			\mathbfit{e}_s = \frac{\mathbfit{e}_{r_1}\times \mathbfit{e}_{r_2}}{\vert \mathbfit{e}_{r_1} \times \mathbfit{e}_{r_2}\vert}
		\end{equation}
		
		We define $\mathbfit{e}_{t_1}$ and $\mathbfit{e}_{t_2}$ as the tangent vectors of this great circle in the points $P_1$ and $P_2$.
		
		\begin{gather}
			\mathbfit{e}_{t_1} = \mathbfit{e}_{s} \times \mathbfit{e}_{r_1}
			\\
			\mathbfit{e}_{t_2} = \mathbfit{e}_{s} \times \mathbfit{e}_{r_2}
		\end{gather}
		
		We call $\zeta_1$ the angle between $\mathbfit{e}_{t_1}$ and $\mathbfit{e}_{\theta_1}$. Similarly, we define $\zeta_2$ as the angle between $\mathbfit{e}_{t_2}$ and $\mathbfit{e}_{\theta_2}$. Translating the vector $\mathbfit{v}_{1}$ along the great circle maintains the angle with the local tangent vector constant and at the point $P_2$ it results in the translated vector $\mathbfit{v}_1^\prime$ with position angle
		
		\begin{equation}
			\alpha_1^\prime =  \alpha_1 + \zeta_2 - \zeta_1
			\label{eq:alphaprime}
		\end{equation}
		
		Figure~\ref{fig:PT} depicts the vectors involved in the operation. With this in mind, we define the generalized dot product between $\mathbfit{v}_{1}$ and $\mathbfit{v}_{2}$ as the following
		
		\begin{equation}
			\mathbfit{v}_{1} \odot \mathbfit{v}_{2} = \vert \mathbfit{v}_{1} \vert
			\vert \mathbfit{v}_{2} \vert \cos (\alpha_1 - \alpha_2 + \zeta_2 - \zeta_1)
		\end{equation}
		
		Since our dataset is purely directional, we have $\vert \mathbfit{v}_{1} \vert = \vert \mathbfit{v}_{2} \vert = 1$. For the same reason, the inner product is written using the following simplified notation
		
		\begin{equation}
		(\alpha_1, \alpha_2) =  \cos [2(\alpha_1 - \alpha_2 + \zeta_2 - \zeta_1)]
		\label{eq:innerproduct}
		\end{equation}
		
		The factor two is introduced so that the argument of the cosine ranges over the full $-\pi$ to $+\pi$, \citep{Bietenholz1986}. By definition $(\alpha_1, \alpha_2) \in [-1, 1]$, where $+1$ indicates perfect alignment \citep{Jain2004} and $-1$ implies perpendicular directions.
		
		
\begin{figure}
	\centering
	\includegraphics[width=0.45\textwidth]{files/PT.pdf}
	\caption{Two dimensional schematic illustration of parallel transport. The figure displays the arc of great circle passing through the points $P_1$ and $P_2$, with $\mathbfit{e}_{t_1}$ and $\mathbfit{e}_{t_2}$ tangent vectors to curve in these points. Notice that the angle $\theta$ between the tangent vector and $\mathbfit{v}_{1}$ is kept constant when $\mathbfit{v}_{1}$, located at $P_1$, is translated along the curve to the point $P_2$. The figure is taken from \citet{Jain2004}, their figure 1, with the author's permission.}
	\label{fig:PT}
\end{figure}	

	\subsection{Angular Dispersion}
		\label{sec:S}
		Given the $i-$th source, we consider the $n$ sources closest to it (including itself). We call $d_{i,n}$ the dispersion function of their position angles.
		
		\begin{equation}
			d_{i, n}(\alpha) = \frac{1}{n}\sum_{k=1}^{n} (\alpha, \alpha_k)
			\label{eq:d}
		\end{equation}
		
		This quantity is a function of a position angle $\alpha$ located at the point where the $i-$th source lies. We call $\alpha_{\rm{max}}$ the position angle that maximizes the dispersion, which assumes the value
		
		\begin{equation}
			d_{i, n}\big|_{\rm{max}} =  \frac{1}{n} \left[ 
			\left( \sum_{k=1}^{n} \cos 2\alpha_k^\prime\right)^2
			+
			\left( \sum_{k=1}^{n} \sin 2\alpha_k^\prime\right)^2
			\right]^{1/2},
		\end{equation}
		
		where $\alpha_k^\prime$ was defined in Eq.~\eqref{eq:alphaprime} and corresponds to the value of the original position angle $\alpha_k$ after being transported in the $i-$th position. Following \cite{Jain2004}, we regard this maximal value as the measure of the dispersion of the $n$ sources and $\alpha_{\rm{max}}$ as their mean direction. The maximum value allowed for the dispersion is $d_{i, n}|_{\rm{max}} = 1$, corresponding to perfect alignment of the sources. The coordinate-invariance of the inner product (Eq. \ref{eq:innerproduct}) extends to the dispersion.
		
		
		For a sample of $N$ sources we fix a number of nearest neighbours $n$ and we derive the set of dispersions.
		
		\begin{align}
			\{d_{i, n}\big|_{\rm{max}}\} && i =1, \dots, N
		\end{align}
		
		For this set we define the following statistics
		
		\begin{align}
			S_{n} = \frac{1}{N} \sum_{i=1}^{N} d_{i, n}\big|_{\rm{max}},
			\label{eq:S}
		\end{align}

		corresponding to the mean dispersion. $S_n$ measures the average position angle dispersion of the sets containing every source and its $n$ neighbours.  
		If the condition $N \gg n \gg 1$ is satisfied, then $S_{n}$ is expected to be normally distributed. \citeauthor{Jain2004} reports the following form for its variance
		
		\begin{equation}
		\sigma_n^2 = \frac{0.33}{N},
		\label{eq:sigmaest}
		\end{equation} 
		
		where $N$ is the total number of sources in the sample. 
		The quantity $S_n$ can be employed for different values of $n$, although these different measurements are not independent. Because the dispersion $d_{i, n}$ is defined in Eq. \eqref{eq:d} as an average of the $n$ closest neighbours, the presence of a positive alignment for $n^\ast$ neighbours implies a preferential positive signal for every $n>n^\ast$.
		
		The deviation of the dispersion $d_{i, n}|_{\rm{max}}$ from its mean value is not normalized, but is found to be $\propto 1/\sqrt{n}$ \citep{Jain2004}. This is mirrored by $S_n$
		
		\begin{equation}
			S_n \propto \frac{1}{\sqrt{n}}
			\label{eq:Sest}
		\end{equation}
		
		To remove this spurious dependence, we will write the measurements of $S_n$ as one-tailed significance levels when considering multiple values of $n$
	
		\begin{equation}
			S.L. = 1-\Phi \left( \frac{S_n - \braket{S_n}_{MC}}{\sigma_n}\right),
			\label{eq:SL}
		\end{equation}
		
		where $\Phi$ is the cumulative normal distribution function and $\braket{S_n}_{MC}$ is the expected value for $S_n$ in absence of alignment, found through Monte Carlo simulations.
		We then employ the following approximate scale: $\log$ S.L. $< -3.5$, very strong alignment;  $-2.5>\log$ S.L. $> -3.5$, strong alignment; $-1.5>\log$ S.L. $> -2.5$ weak alignment.
		
		For every source (labelled by $i$) we define $\varphi_{i, n}$ as angular radius of the circle containing its $n$ neighbours. We can then define the following set:
		
		\begin{align}
		\{\varphi_{i, n}\} && i =1, \dots, N
		\label{eq:phii}
		\end{align}
		
		The distribution of this set provides information about what angular scale a particular $S_n$ probes. For our purposes we will refer to its median $\tilde{\varphi}(n)$ and the $68\%$ interval around it. 
		
		
		\subsection{Random Datasets}
		\label{sec:RandomDatasets}
		
		To estimate the uncertainties and the significance of a given measurement we use simulated data sets containing only noise. The random data sets ($1\,000$ in total) are generated by shuffling the position angles among different sources to ensure that every configuration is affected by the same position angle distribution and survey geometry. 
		
		For a binned or sampled quantity $W_k$ $k\in \{1\dots N_{bins}\}$ we estimate the covariance matrix as
		
		\begin{equation}
		\Sigma^2_{ij} = \braket{(W_i - \braket{W_i}_{MC} )\cdot (W_j - \braket{W_j}_{MC})}_{MC},
		\label{eq:cov}
		\end{equation}
		
		where all the averages are computed over multiple simulations.
		
		For a multivariate Gaussian random vectors $\bm{x}$ with expected mean $\bm{\mu}$ and covariance matrix $C$ of rank $k$, the $\chi^2$ test is generalized using the Mahalanobis distance squared
		
		\begin{equation}
		d^2 = (\bm{x} - \bm{\mu})^T C^{-1} (\bm{x} - \bm{\mu}),
		\end{equation}
		
		which is chi-square distributed with $k$ degrees of freedom. In our analysis, we define the components of vector $\bm{W}$ as the measurements of the statistics $W$ performed on different scales. We then use as Mahalanobis statistics the following expression:
		
		\begin{equation}
		d^2 = (\bm{W}-<\bm{W}>_{MC})^T (\Sigma^2)^{-1} (\bm{W}-<\bm{W}>_{MC})
		\label{eq:Maha}
		\end{equation}
		
		The alignment analyses performed by \cite{Jain2004, Hutsemekers2014, Taylor2016} are based on statistical tests similar to the position angle/polarization vector mean dispersion $S_n$ defined in Eq. \eqref{eq:S}.  None of the above references take covariance into account when estimating the significance level of the measured dispersion as a function of the angular scale. In this study, the Mahalanobis statistics measures deviation from the noise by taking covariance into account.
	
\section{Results}
The accuracy of proposed method is assessed using $100$+ pairs of image frames with diverse resolutions spanning from $50 \times 50$ to $1000 \times 1000$ pixels, and including different categories (e.g., animals, cars, airplane, people, and abstract images). Additionally, the impact of noisy image frames to the accuracy of the proposed method, is assessed using image frames with up to $70$\% of random noise

The evaluations are designed as it follows
\begin{inlinelist}
	\item \textit{Shape A} is a BMP image, or an abstract shape composed of lines, circles, and random noise drawn using features integrated in the implemented tool
	\item \textit{Shape B} is obtained by $\theta$ degrees rotation of \textit{Shape A} plus a random percentage of noise
	\item \textit{Shape A} and \textit{Shape B} are used as inputs for the proposed method
	\item the central tendency of determined rotations is measured as weighted arithmetic mean among top-3 (WM3) rotations, and it is compared with $\theta$. 
\end{inlinelist}
The assessments are performed with default parameters which are: $\omega = 3$, $\lambda=10$, $\epsilon=10$, and the similarity between two segments is calculated excluding neighbor segments. The results of the experiments are discussed as it follows.

\subsection{Top transformations converge rapidly}
The fundamental argument of iterations is to progressively increase the level of details on the image frame abstraction, and accordingly, iteratively improve the accuracy of the calculated approximated transformations, until a user-defined precision criterion is met. Weighted sample variance among top-3 (WV3) approximated transformations provides a measure of dispersion on top approximations. The WV3 reflects the variability in the top-3 approximated transformations, such that: a small WV3 suggests a very reliable WM3, while a large WV3 reflects an uncertainty about the ``best'' linear mapping transformation. According to the experiments, WV3 gets closer to $1$ in a few iterations which yields (a) rapid convergence among top approximated transformations (this confirms the validation of iteration procedure discussed in Section~\ref{section: Validation}), (b) $\textit{WV3} \approx 1$ in few iterations ($>6$) confirms the accuracy of rapidly converged approximated transformations.


\begin{figure*}[!ht]
	\centering
	\includegraphics[width=0.9\textwidth]{Figures/RotationExample.pdf}
	\caption
	{
		\textit{Shape A} is loaded from a BMP image, and \textit{Shape B} is obtained by $270^\circ$ rotation of \textit{Shape A}. The $\Gamma$ matrices of both shapes at different iterations are presented by circular heatmaps. T: determined transformation, S: standard deviation among top-3 determined transformations, D: difference between actual and determined transformations. The normalized similarity index $J(\Gamma_A, \delta \Gamma_B), \forall\delta \in \Delta$ is plotted using a circular heapmap for all the iterations, see panels A2 and B2.  
	}
	\label{Figure: 270}
\end{figure*}


\subsection{Tuning out the cognitive noise}
Selective and visual attention filter irrelevant stimuli to the subject's task by mechanisms such as habituation and cognitive inhibition. There have been promising efforts to model the ability (e.g.,~\cite{tsotsos1995modeling}) since the \textit{spotlight}~\cite{eriksen1972temporal} and \textit{zoom lens}~\cite{eriksen1986visual} models. Additionally, perceived visual information are function of an observer's distance to an object. This aspect has variety of applications namely is Olivia et al.~\cite{oliva2006hybrid} that incorporates this aspect with hybrid images. A hybrid image is composed of two image frames with low and high spatial frequencies, such that either is perceived as noise as a function of observer's distance to the hybrid image frame. In other words, the image of high spatial frequency is dominant at closer distance, while the image with low frequency is perceived at far distance. Whether the noise is a masked image or it is an irrelevant stimuli, it does not impact the perceived information from an image frame. Therefore, the performance of proposed method in approximating linear mapping transformation using noisy image frames, is assessed by experiments where a percentage of \textit{Shape B} is covered with random noise. 

To this extend, an experiment of four tests, $T1$, $T2$, $T3$, and $T4$ is conducted (see Fig.~\ref{Figure: NoiseImpact}). The tests have \textit{Shape A} in common which is a BMP image of a bee. The \textit{Shape B} is created by $234^\circ$ rotation of \textit{Shape A}, and differs among test in the amount of incorporated random noise. The subject in the \textit{Shape A} (i.e., the bee) is represented by $\approx230$K pixels (of $584$K pixels of the image frame). A portion of $120$K pixels (out of the $\approx230K$ pixels) is subject to random noise. This portion is intentionally chosen to cover the body of the bee which presents the majority of perceptible features of the subject. Given that the pixels are binary and the figure is represented by pixels of value 1 (see Section~\ref{section: Shapre Representation}), the random noise is created by setting the value of a random pixel to $1$ in the subject-to-noise portion of \textit{Shape B}. The random noise is added through an iteration of $0$, $5$K, $50$K, and $500$K random pixel selections (a pixel can be selected multiple times) respectively for $T1$, $T2$, $T3$, and $T4$ (see Fig.~\ref{Figure: NoiseImpact}); such that, the majority of perceptible features on \textit{Shape B} are covered with random noise at $T4$. 

The initial segmentation parameter ($\omega = 3$) provides a limited number of variant initial approximations (see Section~\ref{section: Iteration}). Therefore, the WV3 at first iteration (i.e., $l=3$) of the $T1$, $T2$, $T3$, and $T4$ show relatively high dispersion, which indicate the inconsistency of WM3 (see Fig.~\ref{Figure: NoiseImpact}). The initial approximations are tuned at second iteration (i.e., $l=4$) which improve WV3 tenfold (from $118$ to $18$) for the $T1$, $T2$, and $T3$. Despite of a minor discrepancy, WM3 of the tests $T1$, $T2$, and $T3$ are relatively close to actual transformation (i.e., $234^\circ$). However, the considerable noise of $T4$ prevents its WV3 convergence at the same rate as of $T1$, $T2$, and $T3$ (see Fig.~\ref{Figure: NoiseImpact}). The third iteration (i.e., $l=5$) improves approximations, and it brings WV3 of all the test to a same scale, and accordingly provides reliable WM3. Further iterations squeeze the approximations and reach to $\text{WV3}=1.1$ for all tests at sixth iteration (i.e., $l=8$) which indicates a considerable consistency of WM3. Therefore, the method determines WV3 and WM3 for all test at the same scale, given the considerable amount of noise (specially at $T4$). This confirms that even a low amount of perceptible features of the figures is adequate to tune the initial approximations to reliable approximations. For details of the noise impact on other approximations, refer to Supp. Fig.2.17-20.

\begin{figure}[!t]
	\centering
	\includegraphics[width=\columnwidth]{Figures/NoiseImpact.pdf}
	\caption
	{
		Evaluation of random noise impact on transformation determination.s
	}
	\label{Figure: NoiseImpact}
\end{figure}


\subsection{Image resolution defines maximum number of iterations}
When abstracting an image frame, up until a certain iteration, a segment consists of multiple pixels. However beyond that iteration, a segment might be smaller than a pixel (i.e. one pixel belongs to multiple segments). To determine a segment to which a pixel belongs to, the method rounds the position of the pixel. Therefore, beyond a certain iteration, the rounding procedure could potentially increase the distance between the abstractions of two image frames. In such condition, the WV3 converges up-until a certain iteration, and it is saturated beyond that iteration, and accordingly is the WM3 (see Supp. Fig.2.22-23). Therefore, maximum number of iterations, and accordingly the number of \textit{segments} and \textit{sectors} are the function of shape resolution. 


\subsection{Pin-pointed transformation vs. condensed approximations}
The linear mapping transformation between two shapes is determined either as a single transformation with considerable discrepancy with the rest of the approximations (e.g., panel A on Fig.~\ref{Figure: 270}), or a condensed distribution of approximated transformations around actual transformation (e.g., panel B on Fig.~\ref{Figure: 270}). This behavior originates from the discreet representation of image frames (raster graphics); such that, when drawing a \textit{Shape B} from \textit{Shape A}, a pixel of \textit{Shape A} is mapped to a rounded position on \textit{Shape B}. Therefore, pixels of \textit{Shape A} could overlap as mapped on \textit{Shape B}. For instance, the two pixels at $\langle x_1=4, y_1=4 \rangle$ , $\langle x_2=4, y_2=5 \rangle$ belonging to the segment/sector $V_{nm}$ of \textit{Shape A}, with $70^\circ$ rotation, respectively map to positions $\langle x_1^\prime = -0.562, y_1^\prime = 5.628 \rangle$ and $\langle x_2^\prime = -1.33, y_2^\prime = 6.262 \rangle$. As the coordinates are rounded, the two pixels map to position $\langle -1, 6 \rangle$ belonging to the segment/sector $V_{n'm'}$ of \textit{Shape B}. Therefore, two pixels of \textit{Shape A} map to one pixel on \textit{Shape B} (surjective linear transformation). Accordingly, as abstracting the shapes using aggregation function \textit{count} (see Section~\ref{section: Shape Segmentation}), the abstraction parameters are calculated as it follows: $\gamma_{nm} = 2$ and $\gamma_{n'm'} = 1$ (e.g., see comparison of $\gamma$ value distribution plots on Supp. Fig.2.1-22). Hence, comparing $\gamma_{nm}$ and $\gamma_{n'm'}$ results to $j(\gamma_{nm}, \gamma_{n'm'}) = 0.33$ as opposed to expected $j(\gamma_{nm}, \gamma_{n'm'}) = 1$. Such scenarios prevents ``pin-pointing'' the actual transformation (in this case $70^\circ$) and rather provides a condensed distribution of transformations around actual transformation (e.g., see panel B on Fig.~\ref{Figure: 270}).


\subsection{A small similarity is sufficient to determine a reliable linear mapping approximation}
Ideal scenario for comparing two shapes is when there exist a one-to-one correspondence (injective/surjective) between pixels of two the shapes. However, for variety of reasons discussed as it follows, the rotation function on raster graphics is surjective. For instance, rotation function may map multiple pixels of \textit{Shape A} to one pixel of \textit{Shape B}, causing a percentage of deformation on \textit{Shape B} (e.g., see supp. Fig.2.21), and preventing ``pin-pointing'' actual transformation (as above-discussed). Additionally, shapes are possibly subject to noise, which would prevent one-to-one correspondence between the two shapes (non-surjective). Moreover, \textit{Shape A} may consist of congruent figures (e.g., two side-by-side circles of the same radius), and if \textit{Shape B} is determined by $\theta^\circ$ rotation of \textit{Shape A}, then in addition to $\theta^\circ$, multiple rotation angles may also map the congruent shapes on each other. In such cases, actual transformation is determined using incongruent elements (e.g., saddle area, or pedal of the bicycle on Fig.~\ref{Figure: 270}). Such prominent details not only improve approximations for congruent shapes, but are also advantageous when the majority of the shape is covered by noise (e.g., Fig.\ref{Figure: NoiseImpact}) or is deformed (e.g., Supp. Fig.2.21). 

The method discussed in present study, minimizes the impact of such discrepancies on linear mapping transformation determination, by calculating the similarity of two corresponding segments independently from the rest of the segments (an adaptive neighborhood operation of custom range is optionally enabled). Therefore, a higher similarity between few segments is adequate to determine mapping transformation with considerable accuracy. The experiments on deformed, congruent, and noisy image frames illustrate the accuracy of the proposed method on such scenarios.





\vspace{-1em}
\section{Conclusions}
\vspace{-0.6em}
\label{SEC:CONC}
In this paper, we investigated the impact of workload dependent parameters on the failure ratio of the SSDs under power outage. To this end, we presented a fault injection and failure detection platform which injects the realistic power faults to the under test SSDs. During power failure, SSDs experience the exact voltage drop behavior that occurs during power failures in data centers. The results of our experiments reveal that the failure ratio in SSDs due to power outage is significantly affected by the parameters of the running workloads in the application layer. In addition, we show that failures in SSDs are not only due to volatile DRAM cache but also we observe similar failures in SSDs with disabled internal cache.




\section*{Acknowledgements}

This publication has been made possible by the participation of more than 7000 volunteers in the Radio Galaxy Zoo project.  The data in this paper are the result of the efforts of
the Radio Galaxy Zoo volunteers.
Their efforts are individually acknowledged at {\url{http://rgzauthors.galaxyzoo.org}}. 


This publication makes use of data product from the Karl G. Jansky Very Large Array. The National Radio Astronomy Observatory is a facility of the National Science Foundation operated under cooperative agreement by Associated Universities, Inc.

This publication makes use of data products from the Wide-field
Infrared Survey Explorer (WISE) and the Spitzer Space Telescope. The WISE is a joint project of the University of California, Los Angeles, and the Jet Propulsion Laboratory/California Institute of Technology, funded by the National Aeronautics and Space Administration. SWIRE is supported by NASA through the SIRTF Legacy Program under contract 1407 with the Jet Propulsion Laboratory. 

We also thank the staff of the GMRT that made possible the observations TGSS is based upon. GMRT is run by the National Centre for Radio Astrophysics of the Tata Institute of Fundamental Research.

FdG is supported by the VENI research programme with project number 1808, which is financed by the Netherlands Organisation for Scientific Research (NWO). Partial support for LR comes from US National Science Foundation grants AST-1211595 and AST-1714205 to the University of Minnesota. HA benefitted from grant DAIP 980/2016-2017 of the University of Guanajuato. Parts of this research were conducted by the Australian Research Council Centre of Excellence for All-sky Astrophysics (CAASTRO), through project number CE110001020.





%%%%%%%%%%%%%%%%%%%%%%%%%%%%%%%%%%%%%%%%%%%%%%%%%%

%%%%%%%%%%%%%%%%%%%% REFERENCES %%%%%%%%%%%%%%%%%%


\bibliographystyle{mnras}
\bibliography{bibcars}


%%%%%%%%%%%%%%%%%%%%%%%%%%%%%%%%%%%%%%%%%%%%%%%%%%

%%%%%%%%%%%%%%%%% APPENDICES %%%%%%%%%%%%%%%%%%%%%

\appendix
	\section{Position Angle as Shear}
		\label{sec:shear}
		In this Appendix, we focus on an approach to the study of the position angles based on an alternative formalism. The study of other directional quantities over large scales through the use of spin-2 spherical harmonics is well established. Examples of such quantities are the polarization $P$ of the CMB or the cosmic shear field $\gamma$ \citep[e.g.,][]{Collaboration2015, Hikage2011}. However, in our attempts, the detailed properties of the position angle datasets forced a sampling of the correlation functions and power spectra that did not allow us to resolve features like the minimum in Fig.~\ref{fig:Sn}.  In particular, the main complications are the partial sky-coverage, the low source density and the predisposition to systematic effects of interferometric measurements.
		
		Although the products presented in this appendix are inconclusive, we describe here our implementation of the cosmic shear statistics, so that it can be applied when suitable samples will become available.
		
		Cosmic shear is usually detected through the analysis of the spin-2 field
		
		\begin{equation}
			\gamma = \gamma_1 + i\gamma_2,
			\label{eq:gamma}
		\end{equation}
		
		where $\gamma_1, \gamma_2$ are defined on a local Cartesian reference frame. Under rotation of an angle $\Phi$ the field transforms as $\gamma\to \gamma~e^{2i\Phi}$. The shear is usually estimated as the ensemble average of galaxy ellipticities $\varepsilon$ \citep{Kirk2015}
		
		\begin{equation}
			\varepsilon =  \frac{1-q}{1+q} (\cos 2\alpha_p + i \sin 2\alpha_p)
			\label{eq:epsi}
		\end{equation}
		\begin{equation}
			\gamma = \braket{\varepsilon}
		\end{equation}
		
		In this definition, $\alpha_p$ is the major axis position angle of the optical galaxy and $q$ is the ratio between the major and minor axes. We define the tangential and cross-component ellipticity $\varepsilon_t$ and $\varepsilon_\times$ with respect to a direction as the projection of the ellipticity in the two  $+/\times$ components: (1) parallel or perpendicular to it (2) oriented at $45^\circ$ or $-45^\circ$. For a direction defined by the polar angle $\Psi$
		
		\begin{gather}
			\epsilon_t = -\operatorname{Re}\{e^{-2i\Psi}\epsilon \}
			\label{eq:et}
			\\
			\epsilon_\times = -\operatorname{Im}\{e^{-2i\Psi}\epsilon\}
			\label{eq:ecross}
		\end{gather}
		
		In our sign convention, a positive $\varepsilon_t$ corresponds to tangential alignment, i.e., the position angle $\alpha_p$ and the direction $\Psi$ are parallel, while a negative value corresponds to radial alignment, i.e., the two are perpendicular \citep{Kilbinger2015}. 
		
				
		The literature contains multiple statistics involving the shear field. In particular, we focus on those described in  \cite{Schneider2002}, \cite{Eifler2010} and implemented by the software \textsc{treecorr}\footnote{\url{https://github.com/rmjarvis/TreeCorr}} \citep{Jarvis2004}. 
		
		When evaluating a two point correlation function, the two components $\gamma_t$ and $\gamma_\times$ are defined with respect to the direction connecting the sources. These components are commonly estimated by neglecting both the curvature of the sphere and the parallel transport operation described in Section~\ref{sec:PT}. Because of this, we limit our analysis in this Section to distances smaller than $5^\circ$, corresponding to about $0.1$ radians.
		
		We introduce the two-point correlation functions
		
		\begin{gather}
			\xi_{tt}(\varphi) = \braket{\gamma_t \gamma_t}
			\label{eq:xitt}
			\\
			\xi_{\times \times}(\varphi) = \braket{\gamma_\times \gamma_\times}
			\label{eq:xicc}
			\\
			\xi_+ (\varphi)= \braket{\gamma_t \gamma_t} + \braket{\gamma_\times \gamma_\times} 
			\\
			\xi_-  (\varphi)= \braket{\gamma_t \gamma_t} - \braket{\gamma_\times \gamma_\times}
		\end{gather}
		
		where the averages are computed over every possible pair of sources with angular distance $\varphi$. The tangential and cross-component shear are defined as in Eq. \eqref{eq:et}, \eqref{eq:ecross}. The two correlation functions $\xi_{tt}$ and $\xi_{\times \times}$ distinguish between different shear configurations, according to the provided definitions of $\gamma_t$ and $\gamma_\times$.
		Furthermore, we define $\overline{\gamma}(\varphi)$ as the mean shear inside a circular aperture of radius $\varphi$. The variance of this quantity can then be estimated directly from the correlation function $\xi_+$
		
		\begin{gather}
			\braket{|\overline{\gamma}|^2}(\varphi) = \int \frac{d\vartheta\vartheta}{2\varphi^2} 
				\xi_+(\vartheta) S_+ \left( \frac{\vartheta}{\varphi}\right)
			\label{eq:Gsq}
		\end{gather}
		
		The definition of the weight function $S_+$ and a more detailed introduction to the top-hat shear dispersion are given by \cite{Schneider2002}.
		
		
		Using the representation introduced in Eq. \eqref{eq:epsi}, the position angle $\alpha$ can be written as
		
		\begin{equation}
			\gamma^\alpha = \cos 2\alpha + i \sin 2 \alpha
		\end{equation} 			
		
		Under a rotation of an angle $\Phi$ the quantity $\gamma^\alpha$ behaves exactly like the shear field, $\gamma^\alpha \to \gamma^\alpha~e^{2i\Phi}$. This justifies the extension to $\gamma^\alpha$ of the statistics defined for $\gamma$. Since we want to study the alignment configuration of the position angles, we should point out that no averaging is involved. In our analysis $\gamma^\alpha$ takes the place of the shear field $\gamma$ and not of the ellipticity $\varepsilon$.    
		
		In the presence of a global systematic effect we rewrite the correlation functions \eqref{eq:xitt} and \eqref{eq:xicc} as
		
		\begin{gather}
			\xi_{tt} (\theta) = \braket{\gamma^\alpha_t \gamma^\alpha_t} - \xi^n_{tt}
			\\
			\xi_{\times \times} (\theta) = \braket{\gamma^\alpha_\times \gamma^\alpha_\times}
			-
			\xi^n_{\times \times},
		\end{gather}
		
		where we subtracted a noise bias, to be estimated through simulated random data sets containing only the noise. The expression for the estimator \eqref{eq:Gsq} must be computed from these unbiased correlation functions.
		
		We do not assume any particular model for our analysis and we set as our primary objective the detection of a positive correlation. In its absence we expect the two-point correlation functions and the dispersion to be consistent with the noise on every scale $\varphi$.
		
		The function $\braket{|\overline{\gamma^\alpha}|^2}(\varphi)$ is closely related to $S_n$ (Eq. \eqref{eq:S}) since both of them estimate the average dispersion (or dispersion squared) of the position angles. The first one considers spherical caps of constant aperture radius $\varphi$, while the second considers caps with a constant number of sources $n$. The dispersion $\braket{|\overline{\gamma^\alpha}|^2}(\varphi)$ has the advantage of probing precise angular scales, but for non-uniformly distributed samples its value can be easily skewed by the sources in low density regions. Another drawback, due to our chosen implementation, is the lack of parallel transport in its computation.  
	
		
		\subsection{Products}
		
		In Fig.~\ref{fig:RGZweak} and~\ref{fig:TGSSweak} we plot the statistics presented in Eq. \eqref{eq:xitt}, \eqref{eq:xicc} and \eqref{eq:Gsq} for the Radio Galaxy Zoo and TGSS samples. The noise bias has already been subtracted. The covariance matrices are generated using the method described in Section~\ref{sec:RandomDatasets}.
		
		Since the diagonal terms in the covariance matrix \eqref{eq:cov} are two orders of magnitude higher than the non-diagonal terms, we can confirm that the measurements of the statistics $\xi_{tt}$ and $\xi_{\times \times}$ for different angular scales are in fact independent. The same is not true for the dispersion $\braket{|\overline{\gamma^\alpha}|^2}$. The reason for this is the same as the one discussed in Section~\ref{sec:S} for the statistics $S_n$. 	
		
		
		The correlation functions $\xi_{tt}(\theta)$ and $\xi_{\times \times}(\theta)$ are consistent with normally distributed noise. This result was checked using common statistical tests: (1) Shapiro-Wilk (2) $\chi^2$ (3) Anderson-Darling (4) two-tailed Kolmogorov-Smirnov. All of them returned p-values $> 0.05$. For the two $\braket{|\overline{\gamma^\alpha}|^2}$ we obtain the Mahalanobis distances $d^2 = 17.28$ and $d^2 = 12.05$. Given the number of degrees of freedom ($k=12$), both correspond to p-values $>0.05$, meaning that these results are also consistent with the noise. 
		
		Nothing conclusive about the alignment configuration can be stated, since both $\xi_{tt}$ and $\xi_{\times \times}$ are consistent with zero. The dispersion $\braket{|\overline{\gamma^\alpha}|^2}$ is also found to be consistent with the noise. This is not unexpected, since the estimator in Eq. \eqref{eq:Gsq} is simply a convolution of $\xi_{+} = \xi_{tt} + \xi_{\times \times}$ and a weight function. If $\xi_{+}$ is found to be largely consistent with zero, the same should be true for $\braket{|\overline{\gamma^\alpha}|^2}$.
		
		
		\begin{figure}
			\includegraphics[width=0.45\textwidth]{files/RGZweak.pdf}
			\caption{Weak lensing statistics for the Radio Galaxy Zoo sample: the two point correlation functions $\xi_{tt}(\varphi)$, $\xi_{\times \times}(\varphi)$ as a function of the distance $\varphi$ and the top-hat shear dispersion $\braket{|\overline{\gamma^\alpha}|^2}(\varphi)$ as a function of the aperture radius $\varphi$.}
			
			\label{fig:RGZweak}
		\end{figure}
		
		\begin{figure}
			\includegraphics[width=0.45\textwidth]{files/TGSSweak.pdf}
			\caption{Weak lensing statistics for the TGSS sample: the two point correlation functions $\xi_{tt}(\varphi)$, $\xi_{\times \times}(\varphi)$ as a function of the distance $\varphi$ and the top-hat shear dispersion $\braket{|\overline{\gamma^\alpha}|^2}(\varphi)$ as a function of the aperture radius $\varphi$.}
			
			\label{fig:TGSSweak}
		\end{figure}		
	
			Finally, the down-crossing of $\xi_{tt}$ around the angular scale of $3\deg$ seems to suggest a change in the configuration of the alignment. The limited number of data points and the overall consistency with zero of the correlation function do not allow for a conclusive statement. However, assuming the downcrossing to be a feature, we can assign a significance to this observation. The probability of obtaining $8$ consecutive positive datapoints is found to be less than $0.005$. 

%If you want to present additional material which would interrupt the flow of the main paper,
%it can be placed in an Appendix which appears after the list of references.

%%%%%%%%%%%%%%%%%%%%%%%%%%%%%%%%%%%%%%%%%%%%%%%%%%


% Don't change these lines
\bsp	% typesetting comment
\label{lastpage}
\end{document}

% End of mnras_template.tex