\section{Methodology}\label{sec:method}

The raw sensor data is a collection of time series and ultimately it would best to model the multiple sensors in their native data structure. In the analysis we perform in this paper, we compiled the sensor data into a large table to ease exploration, consisting of heterogeneous and categorical data. The heterogeneity is caused by the wide variety of sensors (wind speed, temperature, telescope pointing) each recorded in specific units. Categorical features emerged because certain measurements values were binned. For instance, due to the unreliability of wind speed measurement, we have binned these values -- wind speed below $5$ knots, $5$-$10$ knots, etc. Similarly, for simplicity, each of the twelve vents have been encoded into either completely open or completely closed. These characteristics induce a discontinuous feature space. Our training data set is thus tabular in nature. 
At hand with our curated data set, we are equipped to work on our two objectives: making accurate predictions of  MegaPrime Image Quality, and use our predictor to, on a per-sample basis, explore the importance of each feature on IQ. 

%The structure introduced by humans when tabulating the data (humans decide, e.g., to put features in columns, observations in rows, and what ordering to use) makes tabulated data different from unstructured audio- or image-based data sets (where sound samples or pixels have a naturally occurring order). 

%In the following we discuss how the structure of our data set influences our exploration of tree-based and neural-network-based inference architectures.

% SF: moved from separate file.
Decision tree-based models ~\citep{decision_trees} and their popular derivatives such as random forests~\citep{randomforests}, and gradient boosted trees \citep{gradient_boosted_decision_trees_gbdt} are well-matched to tabular data. and often are the best performers. Tree-based models select and combine features greedily to whittle down the list of pertinent features to include only the most predictive ones. Feature sparsity and missing data is naturally accommodated by tree models, they simply do not include feature cells containing such values in their splits. We show below our implementation of a variant of gradient boosted tree with uncertainty quantification, and feature exploration.
%In this paper we implement an off-the-shelf model gradient boosting tree model to provide a performance baseline. To extract feature importance and interactions from this tree-based model, we use the SHAP package \citep{shap1,shap2}.

However tree-based models require the human process of feature engineering and are known (e.g. \cite{bengio2010decision}) to poorly generalize.
In contrast to tree-based models, deep neural networks (DNNs) are powerful feature pre-processors. Using back-propagation, they learn a fine-tuned hierarchical representation of data by mapping input features to the output label(s). This allows us to shift our focus from feature engineering to fine-tuning the architecture, designing better loss functions, and generally experimenting with the mechanics of our neural network. In reported comparison cases, DNNs yield improved performance with larger sized datasets \citep{airbnb}. As we will show, our neural network implementation, with the feature engineering steps described above, performs better than the alternative tree-based boosted model. We therefore deepen our analysis of the deep neural networks further: we quantify its probabilistic predictions, and we attempt to model the feature space.

%In this paper we provide results for both neural-network based inference architectures and light gradient boosed machines (LGBM, \citealt{lightgbm}).

% SF: suggestion: remove this paragraph to make it shorter?
%We discuss in this section our modelling process. We begin with a short summary of traditional machine-learning based methods and deep learning methods, and justify our decision to use the latter in this work. We describe the two types of neural networks we utilize in Sections \ref{sec:mdn} and \ref{sec:rvae}. Later, in section~\ref{sec:uncertainty_quantification} we present a case for making probabilistic predictions, with accurate accounting of both \textit{aleatoric} and \textit{epistemic} uncertainties. In Section \ref{sec:DL_probability_calibration}, we describe our post-processing methodology to make both models' prediction errors more accurate. Finally, in Section~\ref{sec:DL_metrics}, we list the metrics we use to evaluate the performance of our models.

%%%%%%%%%%%%%%%%%%%%%%%%%%%%%
%%%%%%%%%%%%%%%%%%%%%%%%%%
%%%%%%%%%%%%
%\subsection{Deep Learning}
%\label{sec:DL}

\iffalse

\begin{figure*}
\begin{subfigure}{0.49\textwidth}
    \centering
    \includegraphics[width=\linewidth]{cfht_mdn.PNG}
    \caption{Finding lower and upper learning rates.}
    \label{fig:mdn}
\end{subfigure}
\begin{subfigure}{0.49\textwidth}
    \centering
    \includegraphics[width=\linewidth]{cfht_rvae.PNG}
    \caption{Calibration Curve}
    \label{fig:vae}
\end{subfigure}
\end{figure*}
\fi

%\subsection{Boosted Trees and Deep Neural Networks}\label{sec:ml_vs_dl}
%\Simon{although this section is very interesting per se, I believe it should probably be shortened quite a bit, summarizing the main advantages of a network w.r.t. to tree-based methods, especially in view of streaming data.}
%While DNNs allow extraction of significantly richer patterns between inputs and output(s), they have earned a reputation for being `black-boxes' -- it can be difficult to explain the series of steps a neural network takes to find the relationships between the input features and the output(s) of interest. This is where tree-based models shine -- they are designed to be agglomeration of `if-then-else' statements, thus making it considerably easier to inspect the relationships both between the input features and the outputs, as well as among the features themselves. They pay a price for increased interpretability with decreased ability for feature extraction\footnote{That is, a machine learning model can only work with the user-provided features, whereas a deep enough neural network, in theory, can construct arbitrary complex combinations of these features if such an exercise results in better prediction of the output variable(s) of interest.}, and often times are not the tools of choice for \emph{unstructured data} -- images, videos, text and audio \citep{tabnet}. 

%This hierarchy, however, is reversed in the case of \emph{structured}, tabular\footnote{Tabular data is considered `structured' because typically a human decides to put the features (columns) and observations (rows) in a certain order, whereas images, audio and languages have a naturally occurring order and do not need to be `structured' by the user.} data, where features are individually meaningful and lack strong multi-scale temporal or spatial structures; in this setting, gradient boosted decision trees (GBDT) \citep{gradient_boosted_decision_trees_gbdt, xgboost, catboost} perform reasonably well.
%preferred for their exmore often than not outperform standard deep models on tabular-style datasets \citep{tabnet}, where features are individually meaningful and lack strong multi-scale temporal or spatial structures; the state-of-the-art performance in problems with tabular heterogeneous data is often achieved by such as gradient boosted decision trees (GBDT) \citep{gradient_boosted_decision_trees_gbdt, xgboost, catboost}.
%This raises two questions: (i) \emph{why} does deep learning perform poorly on tabular data, and (ii) \emph{how} can we fix it? \seb{not exactly sure that DL can be said to perform poorly on tabular data anymore - as it is examplified here.} \yst{agreed. This is an unnecessary shot/comparison} The key to answering question (i) is to understand the ways in which tabular data are fundamentally different from unstructured data:


%Our dataset consists of heterogeneous, sparse, and categorical data.  The dataset is heterogeneous due to the wide variety of sensors (wind speed, temperature, telescope pointing) each recorded in specific  units.  The dataset is sparse due to lost data, sensor failures, and sensor replacement or re-positioning.  The dataset is categorical (rather than continuous) because sensor measurements are often binned, e.g., wind speed below $5$ knots, $5$-$10$ knots, etc.  These characteristics induce a discontinuous feature space.  Tree-based models such as decision trees~\citep{decision_trees}, random forests~\citep{randomforests}, and gradient boosted trees \citep{xgboost, lightgbm} are well-matched to these data characteristics. Tree-based models select and combine features greedily to whittle down the list of pertinent features to include only the most predictive ones. Tree-based models easily handle categorical features using, e.g., one-hot encoding. Further, feature sparsity and the presence of NULL values is easily accommodated by such models,  they simply do not include feature cells containing such values in their splits.  In this paper we implement an off-the-shelf Light GBM model to provide a performance baseline. To extract feature importance and interactions from this tree-based model, we use the SHAP package \citep{shap1,shap2}.%; we expound on this in \S\ref{sec:DL_feature_imp}. 

Tree-based models such as simple decision trees~\citep{decision_trees}, random forests~\citep{randomforests}, and gradient boosted trees \citep{gradient_boosted_decision_trees_gbdt} are well-matched to these data characteristics. Tree-based models select and combine features greedily to whittle down the list of pertinent features to include only the most predictive ones. Further, feature sparsity and missing data is naturally accommodated by tree models, they simply do not include feature cells containing such values in their splits.  In this paper we implement an off-the-shelf model gradient boosting tree model to provide a performance baseline. To extract feature importance and interactions from this tree-based model, we use the SHAP package \citep{shap1,shap2}.%; we expound on this in \S\ref{sec:DL_feature_imp}. 
%\tcr{(SCD: Sankalp, can you work back over the second half of the following paragraph and pass back to me -- starting from "we overcome several deficiencies")}

However tree-based models require the human process of feature engineering and have been shown \citep{bengio2010decision} to poorly generalize.
In contrast to tree-based models, deep neural networks (DNNs) are powerful feature pre-processors. Using back-propagation, they learn a fine-tuned hierarchical representation of data by mapping input features to the output label(s). This allows us to shift our focus from feature engineering to fine-tuning the architecture, designing better loss functions, and generally experimenting with the mechanics of our neural network. As we will show, our neural network with little to no feature engineering \seb{Are we really doing no feature engineering for the MDN?}  performs better than the alternative tree-based boosted model that uses extensive feature engineering. In reported comparison cases, DNNs yield improved performance with larger sized datasets \citep{airbnb}. 

% SF: an interesting case, but not clear to me how DNN is a no-brainer for online-learning, and why we should worry about it in the future for our project.

%Finally, it is more natural to use DNNs in an online-mode, with data points `streaming in' rather than arriving in batches. On the other hand, tree-based models need access to all the data to determine best split points, and hence need to be trained every time a new observation comes in. While we do not use a large dataset (we have $\sim 60,000$ samples), or streaming data, in preparation for our future work on real-time predictions we believe it important to consider neural network models from the start.

%Turning to neural networks, the ideal architecture would be one designed to handle the peculiarities of the tabular data discussed above.  However, for the most part, neural networks are designed for unstructured data. While, as mentioned in Section \ref{sec:relatedWork}, there has been some work done in applying neural networks to tabular data \citep{milli2019nowcasting_paramal},  there remains significant room for improvement.

%In this paper, we overcome several deficiencies of traditional neural networks when applied to tabular data by designing and utilizing a feed-forward mixture density network (MDN) as depicted in Figure \ref{fig:1-to-1_plus_mdn}. %\Simon{Motivation for the MDN should be introduced here. Since we are not using DIMM information, atmospheric seeing acts here as a noise source with complex, heteroschedastic properties that an MDN is good at addressing}
%The VAE is capable of handling missing observations in data, as well as identifying which samples in its training set contain errant values, and hence allows us to use the versions of our input data containing all samples, $\mathcal{D_{F_L,S_L}}$ and $\mathcal{D_{F_S,S_L}}$.
%On the other hand, atmospheric seeing acts as a noise source with complex, heteroschedastic properties that an MDN is good at addressing. We use \texttt{BatchNorm} layers \citep{batchnorm} after the input layer of the MDN so that all features are placed on the same scale and none of them unduly affects the gradients and weights during flow through the network. For accessing feature importance and interactions, we use the recently open-sourced \textsc{Python} package \texttt{pathexplain}\footnote{\url{https://github.com/suinleelab/path_explain}} \citep{explaining_explanations_hessians}. 

\iffalse
\tcb{(SCD: I suggest we deleted the rest of the following})

\tcb{++++++++DELETE REST OF SUBSECTION++++++++++}

Neural networks face extra challenges when being applied to structured data while tree-based models

The structured data sets we work with have a number of characteristics that are important to note.
\begin{itemize}
    \item {\bf Heterogenity:}  Feature come from disparate sources, each with their own units. In our case the features included, e.g., environmental sensor measurements, observatory telemetry, and measured instrumental image quality.
    \item {\bf Sparsity:} Unlike data from audio, video and language, there can be relatively little variation in the values of a column in a table. There are often NULL values where no data was collected, or was lost or corrupted in the collation process. \tcr{(SCD: Sankalp, I don't get the point about the ``relatively little variation'')}
    \item {\bf Categorical:}  Feature are often categorical indicating, e.g., that the wind speed is below $5$ knots, between $5$ and $10$ knots, between $10$ and $20$ knots, or above $20$ knots.  In contrast, image pixel values are typically integers between $0$ and $255$ which is better modeled as a continuous range.
    \item {\bf Correlation:}  A minority of the columns in a table are often responsible for the majority of the predictive power. This is unlike the case for images, for instance, where in spite of local spatial correlations between pixels, it is difficult to claim, e.g., that it is the top-left cluster of them that is most helpful in predicting whether an image contains a cat or a dog. \tcr{(SCD: Simon makes a good comment about images being low-entropy, do we keep this or remove?)}
    \Simon{Here again, I would not define input correlations as specific to tabular data. Natural images e.g. have relatively low entropy - and hence can usually be compressed efficiently - and this in turn implies correlations between pixels}
\end{itemize}

%These issues complicate analysis of tabular data with DNNs.
These differences from unstructured data lead to a sparse and discontinuous high-dimensional feature space, making it difficult to exploit for deep neural networks (DNNs). On the other hand, tree-based models -- decision trees \citep{decision_trees}, random forests \citep{randomforests}, and gradient boosted trees \citep{xgboost, lightgbm} -- naturally profit from this landscape. They are by design able to select and combine features via a greedy heuristic, and thus whittle down the list of pertinent features to just the most predictive ones. They are also able to easily handle categorical features using any number of encoding schemes, the most commonly used one being one-hot encoding. The sparsity of features and the presence of NULL values is not of concern to them either, since they simply do not include feature cells containing such values in their splits.

Given the great fit between tree-based models and tabular data, the obvious question arises -- why even consider neural networks?
\begin{itemize}
    \item DNNs are powerful feature pre-processors. Using back-propagation, they learn a fine-tuned hierarchical representation of data by mapping input features to the output label(s). This allows us to shift our focus from feature engineering to fine-tuning the architecture, designing better loss functions, and generally experimenting with the mechanics of our neural network. As we show in Figures (xxx) and (xxx), our neural network with little to no feature engineering performs better than the alternative tree-based boosted model that uses extensive feature engineering.
    \item There is empirical proof that DNNs result in improved performance with larger sized datasets \citep{airbnb}.
    %\item Deep learning allows us to train systems end-to-end, which unlocks the possibility of using unlabelled samples to better predict the labelled ones (\emph{semi-supervised learning}, \S\ref{sec:DL_SSL}).
    \item %\Simon{This is probably a crucial argument for DNNs, as our input data set is here bound to grow with time}
    It is easier to use DNNs in an online-mode, with data points `streaming in' rather than arriving in batches. This is because tree-based models need access to all the data to determine best split points, and hence need to be trained every time a new observation comes in. While this work does not use streaming data, in preparation for our future work on real-time predictions we believe it best to move to a neural network from the get-go.
    \item Finally, tree-based models have been known to generalize poorly (\sg{Sebastien: you suggested this; can you expand + put in a couple of citations please?}).
\end{itemize}
%\subsection{Deep Learning for Tabular Data}\label{sec:DL_tabnet}
Given the ease of use and conventionally superior performance of tree-based models on tabular data, it is natural to ask--why consider DNNs at all?
\begin{itemize}
    \item DNNs are powerful feature pre-processors. Using back-propagation, they learn a fine-tuned heirarchical representation of data by mapping input features to the output label(s). This allows us to shift our focus from feature engineering to fine-tuning the architecture, designing better loss functions, and generally experimenting with the mechanics of our neural network. This exactly functionality allows us to get away with merely using the pairwise differences of the temperature values in \S\ref{sec:data_cleaning_feature_engineering} as additional, hand-crafted features.
    \item There is empirical proof that DNNs result in improved performance with larger sized datasets \citep{airbnb}.
    \item Deep learning allows us to train systems end-to-end, which unlocks the possibility of using unlabelled samples to better predict the labelled ones (\emph{semi-supervised learning}, \S\ref{sec:DL_SSL}).
    \item It is easier to use DNNs in an online-mode, with data points `streaming in' rather than arriving in batches. This is because tree-based models need access to all the data to determine best split points, and hence need to be trained every time a new observation comes in. While this work does not use streaming data, in preparation for our future work on real-time predictions we believe it best to move to a neural network from the get-go.
\end{itemize}
\fi
%Clearly, the ideal model for working with the kind of tabular data we have from CFHT would be a neural network that is designed to handle the peculiarities of tabular data listed above, and is also easily explainable. As mentioned in Section \ref{sec:relatedWork}, there has been some work done in this field \citep{milli2019nowcasting_paramal}, but there is significant room for improvement. 


%\tcr{(SCD: Sankalp, Should these next two paragraph go here?  The previous two paragraph, respectively, discussed the data + tree-based models, and the NNets.  The next paragraph talks about "deficiencies of tree-bsead models" but the previous paragraph discussed deficiencies / challengaeas of NNets, so the transition is unclear.  I feel the paragraph really is discussing how we surmount the challenges facing the application of NNets to categorical data.  Is this what you're after?  And, if it is, should it go here?  (Maybe, b/c we discussed challenges faced by NNets)}


% SF: commented to move further
%For these reasons, in this paper we design and utilize a feed-forward mixture density network (MDN) as depicted in Figure \ref{fig:1-to-1_plus_mdn}. An MDN is able to capture the complex, heteroschedastic properties of seeing (IQ), and easily able to output uncertainties. 

%\Simon{Motivation for the MDN should be introduced here. Since we are not using DIMM information, atmospheric seeing acts here as a noise source with complex, heteroschedastic properties that an MDN is good at addressing}
%The VAE is capable of handling missing observations in data, as well as identifying which samples in its training set contain errant values, and hence allows us to use the versions of our input data containing all samples, $\mathcal{D_{F_L,S_L}}$ and $\mathcal{D_{F_S,S_L}}$.
%On the other hand, atmospheric seeing acts as a noise source with complex, heteroschedastic properties that an MDN is good at addressing.


% SF: commented to move further
%We use \texttt{BatchNorm} layers \citep{batchnorm} after the input layer of the MDN so that all features are placed on the same scale and none of them unduly affects the gradients and weights during flow through the network. For accessing feature importance and interactions, we use the recently open-sourced \textsc{Python} package \texttt{pathexplain}\footnote{\url{https://github.com/suinleelab/path_explain}} \citep{explaining_explanations_hessians}.

% SF: commented to move further
%In addition, we also utilize a robust variational autoencoder (see Section \ref{sec:rvae}) to study the impact of toggling vents between `ON' and `OFF'. The RVAE allows us to isolate out-of-distribution (OOD) samples -- those data points in the training and test sets which are not drawn from the same underlying distribution as the majority of samples in the training set. This serves two purposes. First, it allows us to analyze the impact of vent-toggling only on those samples that are `similar enough' to the majority of samples in the training set, thus increasing our confidence in downstream IQ prediction on these observations. Second, this allows us to suppress the effect of outliers in our measurements. This can be used for anomaly detection and hence predictive maintenance; we relegate this to future work and briefly discuss its implications in Section \ref{sec:conclusion}.

%We also use an off-the-shelf Light GBM model to serve as a baseline against which to measure the performance and validate the effort spent in creating our network. To extract feature importance and interactions from this tree-based model, we use the SHAP package \citep{shap1,shap2}.%; we expound on this in \S\ref{sec:DL_feature_imp}. 

%Fortunately, the past few years have seen extensive research into designing exactly such DNNs, specialized to beat their ML-based cousins on structured data. While some mimic decision tree ensembles \citep{node}, others focus on developing novel architectures using attention \citep{attention_is_all_you_need} suitable for handling tabular data \citep{selfattention_feature_importance, tabnet}. In this work, we use a modified version of the `TabNet' architecture \citep{tabnet}, with several improvements. TabNet uses a sparse, learnable mask on the input features--a kind of soft feature selection that favors the selection of just features that are important for the sample under consideration. It uses a special type of non-linearity called `sparsemax' \textbf{sparsemax citation}, along with an \emph{attention mechanism} \citep{attention_is_all_you_need} called `self-attention' to zero out the contributions of non-relevant features in through a sequence of several steps. TabNet is also able to naturally normalize features of varying ranges, and thus does not require feature pre-processing. For a more detailed description and a visual overview of their network, we refer the readers to \S$2$ and Figure 4, respectively, of \cite{tabnet}. While TabNet has feature interpretability built-in, it can only output the magnitude of the impact of a feature on the output, and not its direction. In other words, it cannot discriminate a feature positively impacting the MPIQ from a feature negatively impacting it, except in magnitude of their effects. For this reason, we do not use this functionality, and instead utilize the \textrm{pathexplain}\footnote{\url{https://github.com/suinleelab/path_explain}} package to obtain feature importance; we expound on this in \S\ref{sec:DL_feature_imp}.\seb{maybe punt some of the description into the related work section? Also: graph NN}.

\iffalse
In short, our inductive bias is that there are (highly) correlated features, so selecting the minimum set of features is the best strategy.

Fortunately, recent developments in attention and sparse regularization now pave the way to learn more efficiently from structured data.
\fi
%\subsection{Light GBM}\label{sec:ml}
(\sg{waiting on sebastien})

\subsection{Probabilistic Predictions with a Mixture Density Network}
\label{sec:mdn}

Mixture density networks (MDNs) are composed of a neural network, the output of which are the parameters of a mixture model \citep{bishop_mdn}. They are of interest here because the relationship between the feature vectors $\mathbf{x}$ and target labels $\mathbf{y}$ can be thought of stochastic nature. Therefore, MDNs express the probability distribution parameters of MPIQ as a function of the input sensor features. In a one-dimensional mixture model, the overall probability density function (PDF) is a weighted sum of $M$ individual PDF $p_\theta^m(\mathbf{y}|\mathbf{x})$ parameterized by a neural network of parameters $\theta$ \footnote{In their initial form~\cite{bishop_mdn}, MDNs used a Gaussian mixture model (GMM). They can easily be generalized to other distributions.}:

%$p(y)=\sum_{j=0}^{M-1} \alpha_{j} p_{j}(y)$
%is a weighted sum of $M$ PDFs $p_{j}(y)$.  The weights $\Alpha=\left\{\alpha_{0}, \ldots, \alpha_{M-1}\right\}$ are non-negative and $\sum_{j=0}^{M-1} \alpha_{j}=1$.  In a (one-dimensional) GMM each distribution $p_j$ is a Gaussian, parameterized by $\theta_j = (\mu_j, \sigma_j)$ where $\mu_j$ is the mean and $\sigma_j$ the standard deviation.  Writing $\theta = \{\theta_0, \ldots, \theta_{M-1}\}$ we sometimes explicitly indicate the parameterization as

\begin{equation*}
p_\theta(\mathbf{y}|\mathbf{x})=\sum_{m=1}^{M} \alpha_\theta^m(\mathbf{x}) p_\theta^m(\mathbf{y}|\mathbf{x}) \quad \textrm{with} \quad
\sum_{m=1}^{M} \alpha_\theta^m(\mathbf{x})=1.
\end{equation*}

Under the assumption of $N$ independent samples $\mathbf{x}$ from the features distribution, and the corresponding conditional samples $\mathbf{y}$ of MPIQ, we minimize over the negative log-likelihood of the density mixture to obtain the neural network weights:
\begin{equation}
\theta^* = \mathop{\mathrm{argmin}}_{\theta} -\frac{1}{N}\sum_{n=1}^{N} \log p_\theta(\mathbf{y}_n|\mathbf{x}_n)
\label{eqn:mdn_loss}
\end{equation}

To train the neural network we take as the network input the data record of sensor readings, observatory operating conditions, etc.  The network outputs are the (per-vector) mixture model parameters modeling the MPIQ conditional distribution, implicitly parameterized by the neural network. %We ensure that the mixture parameters are properly normalized so as to be valid.  In particular, the mixture weight vectors $\alpha$ is normalized so it sums to one and the standard deviations $\sigma_i$ are constrained to be positive.  %The loss for each data record is the log-likelihood of the corresponding MPIQ reading, calculated according to the corresponding $\Alpha$, $\Theta$. 
In our experiments, we use $\beta$ distributions, and set $M=5$ as it gave sensible results.

%The number of GMM components in an MDN is a hyperparameter that needs to be tuned on the validation set; we leave this exercise to future work and use $m=5$ since it produces sensible results. %Figure~\ref{fig:rvae} provides an overview of the MDN construction.

%Fitting can then typically proceed using expectation-maximisation [20] \yst{Is this correct? For our case, there are no classes, the training should only be a maximum likelihood. Am I missing something here? Or do you mean classical GMM here. I would shorten this.}. \Simon{It looks like what is described above is a density estimation using GMM and not an MDN ? I would expect an MDN to model the distribution of MPIQ conditionally on the inputs ?}
%For our purposes, we have an individual-based model $M,$ with some input $\alpha,$ that produces stochastic realisations $y \sim M(\alpha) .$

\iffalse
\tcr{(SCD:Suggest we delete the rest of this section)}

\tcr{+++++START DELETE++++++++}

We wish to derive a relationship among the inputs $\textbf{x}$, the mixture density weights $\omega_{j}(\textbf{s})$, and the density parameterisations $\theta_{j}(\textbf{x})$. While this could potentially be done with a separate regression for each of the density parameters and weights, this would fail to capture the corresponding relationships that would exist between each parameter and weight. %\Simon{ OK that sounds more like an MDN, which is actually regressing the GMM parameters as a function of the input features. I would therefore directly introduce the MDN above and skip the general discussion on GMMs}.
We can therefore model these using a neural network, which is able to provide flexible fitting for arbitrarily complex relationships. %\Simon{ The following is a bit long and hard to follow, especially in the absence of a figure }
This is exactly what an MDN is -- a mixture model, where the mixture components are modelled using a neural network. Provided that $p_{j}$s are differentiable with respect to $\theta_{j}$s, the loss in Equation (\ref{eqn:mdn_loss}) represents a differentiable function. Standard techniques based on stochastic gradient descent can then be applied in order to optimise the weights of the network with respect to this loss.

In our experiments the GMM has $m=50$ components. The MDN is then used to find the $\mu$s, $\sigma$s, and associated weights $\omega$s. Figure~\ref{fig:rvae} provides an overview of the MDN construction. 

\tcr{(SCD: Will clean this up, but in~(\ref{eqn:mdn_loss}) we use $x$ for the MPIQ, I think in the following the $x_{\rm trn}$ may be the input to the neural net?  Ned to clarify.)} The tuple of inputs and output MPIQ from the training set -- $\mathbf{x_{\rm trn}}$ and $y_{\rm trn}$ -- are passed through a number of hidden densely connected layers in the neural network, which provide a compact representation of the relationship between them. These learned distribution parameters are then passed through a normalisation layer; the weights of the mixture ($\omega$s)are transformed such that they sum to 1, and the scale parameters ($\sigma$s)are transformed so that they are positive. \tcr{(SCD: Probably will need Sankalp to review my edits when I make them here.)} These parameters are used to construct the mixture model, where one can draw samples or calculate statistics, such as mean and variance, for a given input. We note that the choice of the number of components in the mixture model, 50, and the parameterization of the mixture model components itself -- normal -- has not been optimized, and provides an avenue for future research.

\tcr{+++++END DELETE++++++++}
\fi

\subsection{Complementary Predictions and Interpretation with Gradient Boosted Decision Trees}
\label{sec:gbdt}

We complement the MDN IQ predictions by another algorithm to secure our results: a gradient boosted decision tree (GBDT) to predict IQ from the sensor data. This is in fact one of the main reason so much of feature engineering was performed on the sensor data.
A set of consecutive decision trees is fit where each successive model is fit to obtain less overall residuals than the previous ones by weighting more the larger sample residuals. Once converged, we can obtain a final predictions from the trained boosted tree as the weighted mean of all models. The optimization can be performed with gradient descent. Several implementations of this popular algorithm exist and we selected the \textsc{catboost}\footnote{\url{https://catboost.ai/}} one for our modelling, with a loss optimized for both the mean and the variance of the predictions. We first perform nested cross-validation as for the MDN, obtain the best hyper-parameters. We then train ten  GBDT models with the same hyper-parameters with a stochastic optimization, each with a different initialization of the model parameters. Aleatoric and epistemic uncertainties are estimated with a simple ensemble averaging method \citep{malinin2021uncertainty} of each model predictions. We show our results in Figure \ref{fig:MDN_moneyplot_catboost}, and discuss them in detail in Section \ref{sec:results}.

%A trained GBDT model can be probed to explore which decisions were selected at each tree split to ultimately give the best IQ predictions. Our tabular data is quite large with many columns, and after hyper-parameter optimization and regularization, the final solution ended into a fairly large tree which limits a human-readable interpretation. Nerveless, feature ranking and attributions are possible and we explore these results further on.


%%%%%%%%%%%%%%%%%%%%
%\subsection{Mixture Density Robust $\beta$ Variational Autoencoder with $\beta$ Divergence}
\subsection{Density Estimation with a Robust Variational Autoencoder}
\label{sec:rvae}
%\seb{My take: put this section into an appendix}

%From Table \ref{tab:datasets_overview} we observe that $\mathcal{D_{F_S,S_S}}$ and $\mathcal{D_{F_L,S_S}}$ have an order of magnitude {\it fewer} samples than their respective counterparts, $\mathcal{D_{F_S,S_L}}$ and $\mathcal{D_{F_L,S_L}}$. As the training of neural networks is extremely data intensive, this provides strong motivation to find a way to be able to bring the larger data sets to bear on network training. The major hurdle is that the larger data sets have missing values and missing data is not easily handled by neural networks. Our approach is to implement self-supervised learning. By developing an approximation of the underlying multivariate distribution from which the input samples are drawn then, for any given sample with missing observations, we can query the approximate distribution to fill in the observed values. We next describe how we accomplish this using one of the most common tools for self-supervised learning, the autoencoder~\citep{hinton2006reducing_autoencoderintroduction}.

%As a `freebie', we would also have the distribution's best guesses for the missing observations, and we could then proceed with using such a sample in a neural network. This is precisely what we do in what follows.

An autoencoder~\citep{hinton2006reducing_autoencoderintroduction} is a neural network that takes high dimensional input data, encode it into a common efficient representation (usually of lower dimension), and then recreates a full-dimensional approximation at the other end. Through its mapping of the input to a smaller or sparser, and more manageable, but information-dense, latent vector, the autoencoder finds application in many areas including  compression, filtering, and accelerated search. A variational encoder \citep{sgvb, vae_intro_2013, vae2} is a probabilistic version of the autoencoder.  Rather than mapping the input data to a specific (fixed) approximating vector, it maps the input data to the parameters of a probability distribution, e.g., the mean and variance of a Gaussian. VAEs produce a latent variable $\mathbf{z}$, useful for data generation. We refer to the \textit{prior} for this latent variable by $p(\mathbf{z})$, the observed variable (\textit{input}) by $\mathbf{x}$, and its conditional distribution ({\it likelihood}) as $p_{\theta}(\mathbf{x} | \mathbf{z})$. In a Bayesian framework, the relationship between the \textit{input} and the \textit{latent variable} can be fully defined by the \textit{prior}, the \textit{likelihood}, and the \textit{marginal} $p_{\theta}(\mathbf{x})$ as:
\begin{align}
p_{\theta}(\mathbf{x})=\int_{\mathcal{Z}} p_{\theta}(\mathbf{x} | \mathbf{z}) p(\mathbf{z}) d \mathbf{z}. \label{eq:fullDist}
\end{align}

%Given $n$ independent samples of data vectors $x^{(i)}$, the maximum likelihood parameter estimate $\theta^{\ast}$ is the parameter value that maximizes the probability of the data.  Typically we equivalently minimize the negative log-likelihood to the data:
%\begin{align}
%\theta^{*}=\arg \min _{\theta} \sum_{i=1}^{n} - \log p_{\theta}\left(\mathbf{x}^{(i)}\right). \label{eq:MLest}
%\end{align}

It is not not easy to solve Equation~(\ref{eq:fullDist}) as the integration across $\mathbf{z}$ is most often computationally intractable, especially in high dimensions.  To tackle this, variational inference \citep[e.g.][]{Jordan1999} is used to introduce an approximation $q_{\phi}(\mathbf{z}|\mathbf{x})$ to the posterior $p_{\theta}(\mathbf{z}|\mathbf{x})$. In addition to maximizing the probability of generating real data, the goal now is also to minimize the difference between the real $p_{\theta}(\mathbf{z}|\mathbf{x})$ and estimated $q_{\phi}(\mathbf{z}|\mathbf{x})$ {\it posteriors}. We state without proof (see \cite{kingma2019introduction} for detailed derivation):%Expanding the Kullback-Lieibler (KL) divergence $D\left(q_\phi(\mathbf{z}|\mathbf{x}) \| p_\theta(\mathbf{z}|\mathbf{x})\right)$, where the expectation in the KL-divergence is taken only with respect to $\mathbf{z}$ and not the (fixed) $\mathbf{x}$ one finds that:
\begin{align}
    &-\log p_{\theta}(\mathbf{x}) + D_{\rm KL}\left(q_{\phi}(\mathbf{z}|\mathbf{x}\right) \| p_{\theta}(\mathbf{z}|\mathbf{x})) \nonumber \\ 
    &= -\mathbb{E}_{\mathbf{z} \sim q_{\phi}(\mathbf{z}|\mathbf{x})} [ \log p_\theta(\mathbf{x} | \mathbf{z})] + D_{\rm KL}\left(q_\phi(\mathbf{z}|\mathbf{x}) \| p_\theta(\mathbf{z})\right) \label{eq:VAEeq}
\end{align}

The approximating distribution $q$ is chosen to make the right-hand-side of Equation~(\ref{eq:VAEeq}) tractable and differentiable. Taking the right-hand-side as the objective to simultaneously minimize both the divergence term on the left-hand-side (making $q$ a good approximation to $p$) and $-\log p(x)$. This is exactly the loss function that we want to minimize via backpropagation:%This makes each term in Equation~(\ref{eq:MLest}) equal to 
\begin{align}
L_{\mathrm{VAE}}(\theta, \phi; \mathbf{x}) & = -\mathbb{E}_{\mathbf{z} \sim q_{\phi}(\mathbf{z} | \mathbf{x})} \left[\log(p_{\theta}(\mathbf{x} | \mathbf{z}))\right] \nonumber \\
&+D_{\mathrm{KL}}\left(q_{\phi}(\mathbf{z} | \mathbf{x}) \| p_{\theta}(\mathbf{z})\right) \nonumber \\
-L_\mathrm{ELBO} &= L_{\mathrm{REC}}(\theta, \phi; \mathbf{x}) + D_{\mathrm{KL}}(\theta, \phi; \mathbf{x}) \label{eq:VAEeq2}
\end{align}
where $\theta^{*}, \phi^{*} = \mathop{\mathrm{argmin}}_{\theta, \phi} L_{\mathrm{VAE}}$. Since KL-divergence is non-negative, Equation (\ref{eq:VAEeq2}) above can be thought of as the lower bound of $p_{\theta}(\mathbf{x})$, and is the loss function to minimize. It is commonly called the ELBO, short for {\it {evidence based lower bound}}. $L_{\rm{REC}}$ minimizes the difference between input and encoded samples, while $D_{\rm{KL}}$ acts as a regularizer \citep{KLregularizer}.

The typical choice for $q$ (that we also make) is an isotropic conditionally Gaussian distribution whose mean and (diagonal) covariance depend on $\mathbf{x}$.  The result is that the divergence term has a closed-form expression where the mean and variance are learned, for example by using a neural network.  To be able to backpropagate through the first term (the expectation) in the loss function, a reparameterization  is introduced.  For each sample from $\mathbf{x}$ take one sample of $\mathbf{z}$ from the conditionally Gaussian distribution $q_\phi(\mathbf{z}|\mathbf{x})$.  Without loss of generality we can generate an isotropic Gaussian $\mathbf{z}$ by taking a Gaussian source $\boldsymbol{\epsilon} \sim \mathcal{N}(0, \mathrm{\mathbf{I}})$ shifting it by the $\mathbf{x}$-dependent mean $\mathbf{\mu}$ and scaling by the standard deviation $\boldsymbol{\sigma}$ to get $\mathbf{z} = \boldsymbol{\mu}+\boldsymbol{\sigma}\odot \boldsymbol{\epsilon}$, where $\odot$ is the element-wise product.  Approximating the first (expectation) term in the objective with a single term using this value for $\mathbf{z}$ allows one to backpropagate gradients through this objective.  Note that, in the terminology of autoencoders, the $q$ and $p$ functions play the respective roles of encoder and decoder; $q_{\phi}(\mathbf{z} | \mathbf{x})$ generate the latent representation from a data point and  $p_{\theta}(\mathbf{x} | \mathbf{z})$ defines a generative model.

%model the random variable $\mathbf{z}$ as a deterministic variable $\mathbf{z}=\mathcal{T}_{\phi}(\mathbf{x}, \boldsymbol{\epsilon}),$ where $\boldsymbol{\epsilon}$ is an auxiliary independent random variable, and the transformation function $\mathcal{T}_{\phi}$ parameterized by $\phi$ converts $\epsilon$ to $\mathbf{z}$. A common choice of the form of $q_{\phi}(\mathbf{z} | \mathbf{x})$ -- and the one employed in this paper -- is a multivariate Gaussian with a diagonal covariance structure:

%\begin{align}
%\mathbf{z} &\sim q_{\phi}\left(\mathbf{z} | \mathbf{x}^{(i)}\right)=\mathcal{N}\left(\mathbf{z} ; \boldsymbol{\mu}^{(i)}, \boldsymbol{\sigma}^{2(i)} \boldsymbol{I}\right) %\\
%\mathbf{z} &= \boldsymbol{\mu}+\boldsymbol{\sigma} \odot \boldsymbol{\epsilon}, \text { where } \boldsymbol{\epsilon} \sim \mathcal{N}(0, \boldsymbol{I})
%\end{align}
%where $\odot$ refers to element-wise product.

%While this reparameterization works for other types of distributions, if we particularlize to multivariate isotropic Gaussians, we need only learn the mean and variance of the distribution.    The  stochasticity is restricted to the random variable $\epsilon \sim \mathcal{N}(0, \boldsymbol{I})$.


%For instance, we find that recorded temperature values can be glitchy, reaching values as high as $999^{\circ}$ and as low as $-999^{\circ}$, as can be seen from Table \ref{tab:summarystats}. 

The VAE described so far, which we refer to as the `vanilla' VAE, is not the optimal model for our purposes.% for two reasons.  First, the data sets $\mathcal{D_{F_S,S_L}}$ and $\mathcal{D_{F_L,S_L}}$ contain missing observations, where sensor values were not recorded.
This is because our data set $\mathcal{D_{F_S,S_S}}$ %and its derivative $\mathcal{D_{F_L,S_S}}$ 
can contain outliers caused mostly by sensor failures, and sometimes by faulty data processing pipelines. The ELBO for `vanilla' VAE contains a log-likelihood term (first term in RHS of Equation \ref{eq:VAEeq}) that will give high values for low-probability samples \citep{akrami2019robustvae}. %To accommodate these aspects of the data we modify the VAE in two ways. First, following the work of \cite{vae_missingvalues}, we introduce a mask $M$ as an additional input to the encoder, . This mask encodes the positions of the missing observations (`NaN' values), allowing the modified network to impute the missing values. Second,

We state without proof (see \citet{akrami2019robustvae, akrami2020robustvaetabular} for details) that $L_{\rm {REC}}$ for a single sample can be re-written as:
\begin{align}
L_{\rm {REC}}^{(i)} = \mathbb{E}_{\mathbf{z} \sim q_{\phi}(\mathbf{z} | \mathbf{x}^{(i)})}\left[D_{\rm {KL}} \left(\hat{p}(\mathbf{X}) | p_\theta(\mathbf{X}|\mathbf{Z})\right)\right], \label{eq:lrec_mod}
\end{align}
where $\hat{p}(\mathbf{X}) = \frac{1}{N} \sum_{i=1}^{N} \delta (\mathbf{X}- \mathbf{x}^{(i)})$ is the empirical distribution of the input matrix $\mathbf{X}$, and $N$ is the number of samples in a mini-batch. We then substitute the KL-divergence with the $\beta$-cross entropy \citep{ghosh2016robust_variationalbayesianinference} which is considerably more immune to outliers:
\begin{align}
    L_{\rm{REC}, \beta}^{(i)} = \mathbb{E}_{\mathbf{Z} \sim q_{\phi}(\mathbf{Z} | \mathbf{x}^{(i)})}\left[H_{\beta}(\hat{p}(\mathbf{X}),p_{\theta}(\mathbf{X} | \mathbf{Z}))\right], \label{eq:lrec_beta}
\end{align}
where the $\beta$ cross-entropy is given by \cite{formula_betacrossentropy,adaptiveoptics0,rvae_orig}:
\begin{align}
    &H_{\beta}(\hat{p}(\mathbf{X}),p_{\theta}(\mathbf{X} | \mathbf{Z})) = \nonumber \\
    &-\frac{\beta+1}{\beta}\!\int\!\hat{p}(\mathbf{X})\left(p_{\theta}(\mathbf{X} | \mathbf{Z})^{\beta}-1\right) d \mathbf{X}+\int\!p_{\theta}(\mathbf{X} | \mathbf{Z})^{\beta+1} d \mathbf{X} \label{eq:ce_beta_def}
\end{align}
Here $\beta$ is a constant close to 0. This makes the total loss function for a given sample:
\begin{align}
L_{\beta}\left(\theta, \phi ; \mathbf{x}^{(i)}\right) =& \mathbb{E}_{\mathbf{Z} \sim q_{\phi}(\mathbf{Z} | \mathbf{x}^{(i)})}\left[H_{\beta}(\hat{p}(\mathbf{X}),p_{\theta}(\mathbf{X} | \mathbf{Z}))\right] \nonumber \\
%\frac{\beta+1}{\beta}\left(\frac{1}{\left(2 \pi \sigma^{2}\right)^{\beta D / 2}} \exp \left(-\frac{\beta}{2 \sigma^{2}} \sum_{d=1}^{D}\left\|\hat{\mathbf{x}}_{d}-\mathbf{x}_{d}^{(i)}\right\|^{2}\right)-1\right) \nonumber \\
&+ D_{K L}\left(q_{\phi}\left(\mathbf{Z} | \mathbf{x}^{(i)}\right) \| p_{\theta}(\mathbf{Z})\right) \label{eq:loss_beta_prelim}
\end{align}
%\tcr{where $D$ is the number of features. $\hat{\mathbf{x}}_{d}$ is the output from the decoder at the $d^{\rm{th}}$ position, for the input sample $\mathbf{x}$. (SCD: Sankalp there is no $D$ nor $\hat{x}_d$ in the above.)}  \tcb{(SCD: I find the following extremely hard to follow.  It seems you are drawing a single sample $S=1$ below, which is akin to the VAE stuff, can you maybe just say that after the definition of $L_\beta$ and leave it at that?  Should also indicate what value of $\beta$ is used -- that is left open below.)}

\iffalse
\begin{align}
    &H_{\beta}\left(\hat{p}(\mathbf{X}), p_{\theta}(\mathbf{X} | \mathbf{z})\right) = \nonumber \\
    &-\frac{\beta+1}{\beta}\!\int\!\hat{p}(\mathbf{X})\left(p_{\theta}(\mathbf{X} | \mathbf{z})^{\beta}-1\right) d \mathbf{X}+\int\!p_{\theta}(\mathbf{X} | \mathbf{z})^{\beta+1} d \mathbf{X}
\end{align}
\fi

To draw from the continuous $\mathbf{Z}$, we use an empirical estimate of the expectation, and convert the above into a form of the Stochastic Gradient Variational Bayes (SGVB) cost \citep{sgvb} with a single sample $\mathbf{z}^{(j=1)}$ from $\mathbf{Z}$.
\iffalse
:
\begin{align}
L_{\beta}\left(\theta, \phi ; \mathbf{x}^{(i)}\right) = &H_{\beta} \left(\hat{p}(\mathbf{X}),p_{\theta}(\mathbf{X} | \mathbf{z}^{(j)})\right) \nonumber \\
&+ D_{K L}\left(q_{\phi}\left(\mathbf{z}^{(j)} | \mathbf{x}^{(i)}\right) \| p_{\theta}(\mathbf{z}^{(j)})\right) \label{eq:loss_beta_sgvb}
\end{align}
\fi
Next, for each sample we calculate $H_{\beta}(\hat{p}(\mathbf{X}),p_{\theta}(\mathbf{X} | \mathbf{z}^{(1)}))$ when $\mathbf{x}^{(i)} \in [0,1]$. We substitute $\hat{p}(\mathbf{X})=\delta\left(\mathbf{X}-\mathbf{x}^{(i)}\right)$ and model $p_{\theta}(\mathbf{X} | \mathbf{z}^{(1)})$ with a mixture of Beta distributions with weight vector $\mathbf{\omega}$. That is,
\begin{align}
    p_{\theta}(\mathbf{X} | \mathbf{z}^{(1)}) = \sum_{k=1}^{k} \omega_k (\mathbf{X}^{p_k-1})(1-\mathbf{X})^{q_k-1} \times \frac{\Gamma(p_k+q_k)}{\Gamma(p_k)\Gamma(q_k)} \label{eq:pxgivenz_mixturedensity}
\end{align}
Using Equations (\ref{eq:ce_beta_def}) and (\ref{eq:pxgivenz_mixturedensity}), we obtain:
\begin{align}
    &H_{\beta}(\delta(\mathbf{X}-\mathbf{x}^{(i)}),p_{\theta}(\mathbf{X} | \mathbf{Z})) = \nonumber \\
    &-\frac{\beta+1}{\beta} \left(\sum_{d=1}^{D}\left(\sum_{k=1}^{K} \omega_k (\mathbf{x}_{d}^{(i) \cdot p_k-1})(1-\mathbf{x}_{d}^{(i)})^{q_k-1} \Lambda_{d,k}\right)\right) \nonumber \\
    &+\sum_{d=1}^{D} \sum_{k=1}^{K}\frac{\left((p_{d,k}-1)^{1+\beta}+1\right) \left((q_{d,k}-1)^{1+\beta}+1\right)}{(p_{d,k}-1)^{1+\beta}+(q_{d,k}-1)^{1+\beta}+2} \label{eq:ce_final}
\end{align}
where $\Lambda_{d,k} = \frac{\Gamma(p_{d,k}) + \Gamma(q_{d,k})}{\Gamma(p_{d,k})\Gamma(q_{d,k})}$, $D$ is the number of dimensions in a single sample, and $K$ is the number of components in the mixture. Equations (\ref{eq:loss_beta_prelim}) and (\ref{eq:ce_final}) together give us the total loss across all $N$ samples in a given mini-batch:
\begin{align}
L_{\beta}\left(\theta, \phi ; \mathbf{X}\right) = &\frac{1}{N}\sum_{i=1}^{N} \left[H_{\beta}^{(i)}(\hat{p}(\mathbf{X}),p_{\theta}(\mathbf{X} | \mathbf{Z})) \right. \nonumber \\
& \left. + D_{K L}\left(q_{\phi}\left(\mathbf{z}^{(1)} | \mathbf{x}^{(i)}\right) \| p_{\theta}(\mathbf{z}^{(1)})\right)\right], \label{eq:loss_beta_final}
\end{align}
where the superscript $^{(1)}$ implies a single draw from \textbf{z} from \textbf{Z}.

\iffalse
Using empirical estimates of expectation we form the Stochastic Gradient Variational Bayes (SGVB) \citep{sgvb}:
\begin{align}
L_{\beta}\left(\theta, \phi ; \mathbf{x}^{(i)}\right) \approx &\frac{1}{S}\sum_{j=1}^{S}H_{\beta}^{(i)}(\hat{p}(\mathbf{X}),p_{\theta}(\mathbf{X} | \mathbf{z}^{(j)})) \nonumber \\
&+ D_{K L}\left(q_{\phi}\left(\mathbf{Z} | \mathbf{x}^{(i)}\right) \| p_{\theta}(\mathbf{Z})\right),
\end{align}
where $S$ is the number of samples drawn from $q_{\phi}\left(\mathbf{Z} | \mathbf{X}}\right)$. In practice we pick $S=1$; as long as the minibatch size is large enough, this is a good approximation
We pick $\beta$ to be \tcr{xxx} by trial and error. 
\fi

The final, robust variational autoencoder architecture is denoted in the left of Figure \tcr{\ref{fig:rvae_plus_mdn}}.

\iffalse
\begin{figure*}
    \centering
    \includegraphics[width=1\textwidth]{vae.png}
    \caption{Caption}
    \label{fig:rvae}
\end{figure*}
\fi

%\fi


\iffalse
Let us call the underlying, unknown probability distribution $p^{*}(x)$. We would like to estimate it from its independent samples $\mathbf{x}_{1: N}=\left\{\mathbf{x}_{i}\right\}_{i=1}^{N} .$ To formulate inference as an optimization problem, we need to choose an approximating family $\mathcal{Q}$ and an optimization objective $J(q) .$ This objective needs to capture the similarity between $q$ and $p ;$ the field of information theory provides us with a tool for this called the Kullback-Leibler $(\mathrm{KL})$ divergence.

To this end, we consider a parametric model $p(x ; \theta)$ with parameter $\theta,$ and minimize the generalization error measured by the KL divergence $D_{\mathrm{KL}}$ from $p^{*}(x)$ to $p(x ; \theta)$\footnote{The main idea of variational methods is to cast inference as an optimization problem over a class of tractable distributions $\mathcal{Q}$ in order to find a $q \in \mathcal{Q}$ that is most similar to $p$.} :
\begin{align}
D_{\mathrm{KL}}\left(p^{*}(x) \| p(x ; \theta)\right)=\int p^{*}(x) \log \left(\frac{p^{*}(x)}{p(x ; \theta)}\right) d x
\end{align}
However, since $p^{*}(x)$ is unknown in practice, it is replaced with
\begin{align}
D_{\mathrm{KL}}(\hat{p}(x) \| p(x ; \theta))=\mathrm{Const.}-\frac{1}{N} \sum_{i=1}^{N} \ln p\left(x_{i} ; \theta\right)
\end{align}
where $\hat{p}(x)=\frac{1}{N} \sum_{i=1}^{N} \delta\left(x, x_{i}\right),$ is the empirical distribution and $\delta$ is the Dirac delta function. Minimizing this empirical Kullback-Leibler divergence is equivalent to maximum likelihood estimation. Equating the partial derivative of Eq.(2) with respect to $\theta$ to zero, we obtain the following estimating equation:

In our VAE example, we use two small ConvNets for the encoder and decoder networks. In the literature, these networks are also referred to as inference/recognition and generative models respectively. We use tf.keras.Sequential to simplify implementation. Let  and  denote the observation and latent variable respectively in the following descriptions.

Encoder network
This defines the approximate posterior distribution , which takes as input an observation and outputs a set of parameters for specifying the conditional distribution of the latent representation . In this example, we simply model the distribution as a diagonal Gaussian, and the network outputs the mean and log-variance parameters of a factorized Gaussian. We output log-variance instead of the variance directly for numerical stability.

Decoder network
This defines the conditional distribution of the observation , which takes a latent sample  as input and outputs the parameters for a conditional distribution of the observation. We model the latent distribution prior  as a unit Gaussian.

Reparameterization trick
To generate a sample  for the decoder during training, we can sample from the latent distribution defined by the parameters outputted by the encoder, given an input observation . However, this sampling operation creates a bottleneck because backpropagation cannot flow through a random node.

To address this, we use a reparameterization trick. In our example, we approximate  using the decoder parameters and another parameter  as follows:

where  and  represent the mean and standard deviation of a Gaussian distribution respectively. They can be derived from the decoder output. The  can be thought of as a random noise used to maintain stochasticity of . We generate  from a standard normal distribution.

The latent variable  is now generated by a function of ,  and , which would enable the model to backpropagate gradients in the encoder through  and  respectively, while maintaining stochasticity through .

Network architecture
For the encoder network, we use two convolutional layers followed by a fully-connected layer. In the decoder network, we mirror this architecture by using a fully-connected layer followed by three convolution transpose layers (a.k.a. deconvolutional layers in some contexts). Note, it's common practice to avoid using batch normalization when training VAEs, since the additional stochasticity due to using mini-batches may aggravate instability on top of the stochasticity from sampling.
\fi

\subsection{Uncertainty Quantification}
\label{sec:uncertainty_quantification}


\iffalse
\begin{figure*}
\begin{subfigure}{0.49\textwidth}
    \centering
    %\includegraphics[width=.98\linewidth]{figures/cfht_iq_exptime.pdf}
    %\caption{Exposure difference for}
    %\label{fig:exptime_feature}
    \includegraphics[width=.98\linewidth]{figures/find_beta.pdf}
    \caption{Finding optimal $\beta$ for the robust variational autoencoder. The lower the reconstruction loss, the better. The larger the separation between training and noisy dataset reconstruction, the better. $\beta=0.005$ is determined to be the optimal value.\tcr{(SCD: I think we need to discuss what is the difference between uniform and constant noise.  In the caption we only refer to "noisy dataset".  Also in y-legend should be the "sign" and the "NLL" should not be italicised.  Plus the "NLL" in the log should be something like log (|NLL|) I assume -- i.e., include absolute values.)}}
    \label{fig:findbeta_elbo_vs_beta}
\end{subfigure}
\hfill
\begin{subfigure}{0.49\textwidth}
    \centering
    \includegraphics[width=.98\linewidth]{figures/hist_rl_noise_temp.pdf}%hist_noise.pdf}
    \caption{Histogram of reconstruction losses along with the $95^{\rm th}$ percentile cut-off.}
    \label{fig:findbeta_hist0}
\end{subfigure}
\newline
\begin{subfigure}{0.49\textwidth}
    \centering
    \includegraphics[width=.98\linewidth]{figures/hist_mll_noise_temp.pdf}%hist_noise.pdf}
    \caption{Histogram of marginal log likelihoods along with the $5^{\rm th}$ percentile cut-off.}
    \label{fig:findbeta_hist1}
\end{subfigure}
\hfill
\begin{subfigure}{0.49\textwidth}
    \centering
    \includegraphics[width=.98\linewidth]{figures/hist_regret_noise_temp.pdf}%hist_noise.pdf}
    \caption{Histogram of likelihood regrets \citep{likelihood_regret} along with the $95^{\rm th}$ percentile cut-off.  \tcr{(SCD: I think this data is not from Xiao et al, so the caption is misleading -- maybe push that citation to the discussion in text?)}}
    \label{fig:findbeta_hist2}
\end{subfigure}
\caption{In (a), we determine the optimal $\beta$ based on marginal log likelihood's ability to separate the medians of each of the three histograms.  In (b)-(d) we present histograms for three distinct metrics: (b) reconstruction loss, (c) marginal log likelihood, and (d) likelihood regret.  In each plot we plot histograms of the respective metric for three distinct training sets: the actual training set, simulated uniform noise, and simulated constant noise. \tcr{(SCD: what is ``constant'' noise -- seems any oxymoron.)} By injecting synthetically generated noise through an RVAE trained on the training set, we can asses the ability of different metrics to separate in-distribution (ID) from out-of-distribution (OOD) sets. \tcr{(SCD: Is this all detailed in the text?)} }
\label{fig:findbeta}
\end{figure*}
\fi


%\begin{figure*}
%\begin{subfigure}{0.49\textwidth}
%    \centering
%    \includegraphics[width=.98\linewidth]{figures/hist_episstd_features.pdf}
%    \caption{Histograms of uncalibrated epistemic uncertainties from the MDN, for various data sets. Vertical line is the $95^{\rm th}$ percentile value for the training set.}
%    \label{fig:vae_hist_epis}
%\end{subfigure}
%\hfill
%\begin{subfigure}{0.49\textwidth}
%    \centering
%    \includegraphics[width=.98\linewidth]{figures/hist_calepisstd_features.pdf}
%    \caption{Histograms of calibrated epistemic uncertainties from the MDN, for various data sets. Vertical line is the $95^{\rm th}$ percentile value for the training set.}
%    \label{fig:vae_hist_calepis}
%\end{subfigure}
%\newline
%\begin{subfigure}{0.49\textwidth}
%    \centering
%    \includegraphics[width=.98\linewidth]{figures/hist_mll_noise_features.pdf}
%    \caption{Histograms of the pseudo marginal log likelihood (-L$_{\rm ELBO}$) from the RVAE. Vertical line is the $95^{\rm th}$ percentile value for the training set.}
%    \label{fig:vae_hist_mll}
%\end{subfigure}
%\hfill
%\begin{subfigure}{0.49\textwidth}
%    \centering
%    \includegraphics[width=.98\linewidth]{figures/hist_regret_noise_features.pdf}
%    \caption{Histograms of log likelihood regret for the same data sets, from the RVAE. Vertical line is the $95^{\rm th}$ percentile value for the training set.}
%    \label{fig:vae_hist_regret}
%\end{subfigure}
%\newline
%\begin{subfigure}{0.49\textwidth}
%    \centering
%    \includegraphics[width=.98\linewidth]{figures/hist_mll_noise_zoomed_features.pdf}
%    \caption{Zoom-in of \textbf{(c)}, to focus on the hard OoD data sets.}
%    \label{fig:vae_hist_mll_zoomed}
%\end{subfigure}
%\hfill
%\begin{subfigure}{0.49\textwidth}
%    \centering
%    \includegraphics[width=.98\linewidth]{figures/hist_regret_noise_zoomed_features.pdf}
%    \caption{Zoom-in of \textbf{(d)}, to focus on the hard OoD data sets.}
%    \label{fig:vae_hist_regret_zoomed}
%\end{subfigure}
%\label{fig:common_ood_detections}
%\caption{We visualize the discriminative ability of various metrics to separate Out-of-Distribution (OoD) samples from In-Distribution (ID) samples. Surprisingly, we notice that epistemic uncertainties, whether calibrated or uncalibrated, are poor metrics for OoD detection. This was our motivation to design and use the RVAE, which directly captures the likelihood of the training data-generating distribution. \textbf{(c)} and \textbf{(e)} show that the log likelihood is an excellent -- if not perfect -- separator of ID training and test data, from slightly yet certainly OOD synthetic data created by adding Gaussian noise with the indicated $\sigma$ to the normalized (between 0 and 1) training data. In \textbf{(d)} and \textbf{(f)} we calculate the log likelihood regret \citep{likelihood_regret}, as explained in Section \ref{sec:rvae}. Comparing \textbf{(e)} and \textbf{(f)}, we see that regret is a slightly better OoD detector -- the training data histogram is more concentrated, and the modes of the histograms for the two noisy datasets are farther away from the mode of the histogram for the training data set.}
%\end{figure*}

Our predictions will be safer for decision making if for each input vector, in addition to the prediction of IQ, we also predict the degree of (un)certainty. This is especially true since we aim to toggle the twelve vents based on our predictions, which is an expensive manoeuvre -- a configuration of vents that ends up increasing observed IQ as opposed to decreasing it would require re-observation of the target, when CFHT is already oversubscribed by a factor of $\sim 3$. For this reason, we predict a probability density function of MPIQ for every input sample, as described in Section \ref{sec:mdn}.


%Any model mapping a set of covariates to outputs should realistically be probabilistic in nature. In other words, for each input vector associated with an output label, the model should predict not just a measure of central tendency, but also a measure of spread (for example, the mean and standard deviation if one assumes normal distribution for the output variable). This uncertainty naturally arises due to a number of reasons:
%\begin{itemize}
%    \item It is possible that in spite of feature engineering (described in Section \ref{sec:feature_engineering}), our data set does not have every feature required for confident prediction of MPIQ values. This could simply be because the data set with fewer features, $\mathcal{D_{F_S, S_S}}$, is lacking some strongly informative features. As an exaggerated example, consider the task of training a model to predict the digit (from 0 to 9) present in an image. If one were to feed this model only the right half of each image for training, we would intuitively expect it to be very uncertain if an image contains a 3 or an 8.
%    \item Every sensor and measuring device has a certain intrinsic measurement error, which would carry forward to all downstream calculations.
%    \item Any machine learning algorithm is only as good as the data on which it is trained. It might well be the case that a future observation lies in isolation in the multi-dimensional parameter space, i.e., be an outlier. In such a case, a well-behaved model should intuitively be able to detect this \emph{out-of-distribution} behavior.
 %   \item Even if our model is able to satisfactorily predict MPIQ, it is very well possible that there exist other models which can also make predictions just as, if not more, accurately and precisely. It is also possible that for our chosen model, a different data pre-processing scheme or choice of hyperparameters %%, or choice of scheme for balancing data \textbf{insert section}
 %   might result in better predictions.
%\end{itemize}

%Owing to these factors, a highly desirable behavior of any predictive model would be to not only return a point prediction, but also some quantity conveying the model's belief or confidence in its output \citep{kendall2017uncertainties, choi2018uncertainty}. 

Higher error (corresponding to lower model belief or confidence in the estimate) can result from absence of predictive features, error or failure in important sensors, or an input vector that value has drifted from the training distribution. We decompose the sources of predictive uncertainties into two distinct categories: {\it aleatoric} and {\it epistemic}.
%\cite{kendall2017uncertainties}\footnote{The word epistemic comes from the Greek ``episteme", meaning ``knowledge" and corresponds to... Aleatoric comes from the Latin ``aleator", meaning ``dice player"; aleatoric uncertainty is the ``dice player's" uncertainty \citep{gal_thesis}}. 
Aleatoric uncertainty captures the uncertainty inherent to the data generating process. To analogize using an everyday object, this is the entropy associated with an independent toss of a fair coin. Epistemic uncertainty, on the other hand captures the uncertainty associated with improper model-fitting. In contrast to its aleatoric counterpart, given a sufficiently large data set epistemic uncertainty can theoretically be reduced to zero\footnote{The word ``aleatoric'' derives from the Latin ``aleator'' which means ``dice player''. The word ``epistemic'' derives from the Greek ``episteme'' meaning ``knowledge''~\citep{gal_thesis}.}. Aleatoric uncertainty is thus sometimes referred to as {\it irreducible} uncertainty, while epistemic as the {\it reducible} uncertainty. High aleatoric uncertainty can be indicative of noisy measurements or missing informative features, while high epistemic uncertainty for a prediction could be a pointer to the outlier status of the associated input vector.

%and can indeed be explained away given an a sufficiently large training dataset\footnote{If we are given a biased coin where we do not know the probability $p$ of turning up heads, we can determine it to arbitrary precision by flipping the coin an arbitrarily large number of times and counting the number of times it turns up heads.}. 



%measurement noise in filters, or the inherent randomness of a coin flipping experiment. It cannot be reduced by collecting more training data

%\footnote{We know that a fair coin has a 0.5 probability of landing heads, yet we cannot assert with 100\% certainty the outcome of the next flip, no matter how many times we have already flipped the coin.}; however, it can be reduced by collecting more informative features. %\footnote{Uncertainty in galaxy mass estimation in the absence of H and K band information can be drastically reduced by collecting high quality data in these bands.}.


%On the other hand, epistemic uncertainty models the ignorance of the predictive model, and can indeed be explained away given an a sufficiently large training dataset\footnote{If we are given a biased coin where we do not know the probability $p$ of turning up heads, we can determine it to arbitrary precision by flipping the coin an arbitrarily large number of times and counting the number of times it turns up heads.}. 



%Aleatoric uncertainty covers first two of the four issues above, while epistemic uncertainty subsumes the last two. High aleatoric uncertainty can be indicative of noisy measurements or missing informative features, while high epistemic uncertainty could be a pointer to its status as an outlier.

The architecture of the MDN (Section \ref{sec:mdn}) allows us to predict a PDF of MPIQ for each sample. For each sample and mixture model component, let $\mu_m$, $(\sigma_m)^2$, and $\alpha_m$ respectively denote the mean, variance, and normalized weight (weights for all mixture model components must sum to 1) in the mixture model. We obtain the predicted IQ value as the weighted mean of the individual means: 
\begin{equation}
\mu = \sum_{m=1}^M \alpha_m \mu_m \label{eq.calcIQpredict}
\end{equation}
Aleatoric uncertainty is the weighted average of the mixture model variances, calculated as \citep{choi2018uncertainty}:
\begin{equation}
\sigma_{\mathrm{al}}^2 =  \sum_{m=1}^{M} \alpha_m\sigma_m^2, \label{eq.calcAleaUncert}
\end{equation}
while epistemic uncertainty is the weighed variance of the mixture model means: 
\begin{equation}
\sigma_{\mathrm{epis}}^2 = \sum_{m=1}^{M} \alpha_m \mu_m^{2} - \mu^{2} \label{eq.calcEpisUncert}
\end{equation}
The total uncertainty is computed by adding Equations ~(\ref{eq.calcAleaUncert}) and~(\ref{eq.calcEpisUncert}) in quadrature.%  \tcr{(SCD: Note the notation needs to be unified with earlier -- here $\sigma_{i,j}$ while earlier $\sigma^{(i)}$.  Also, we don't use the term "in quadrature" (means something different in communication theory) -- I guess you just are convolving two Guassians so taking the square root of the squares of the standard deviations -- basically Pythoagorean's Theorem?  Is ``in quadrature'' standard in astronomy?)}
\subsection{MuJoCo Model Calibration}
\label{app:model-calibration}

The MuJoCo XML model of the hand requires many parameters, which are then used as the mean of the randomized distribution of each parameter for the environment. Even though substantial randomization is required to achieve good performance on the physical robot, we have found that it is important for the mean of the randomized distributions to correspond to reasonable physical values. We calibrate these values by recording a trajectory on the robot, then optimizing the default value of the parameters to minimize error between the simulated and real trajectory.


To create the trajectory, we run two policies in sequence against each finger. The first policy measures behavior of the joints near their limits by extending the joints of each finger completely inward and then completely outward until they stop moving. The second policy measures the dynamic response of the finger by moving the joints of each finger inward and then outward in a series of oscillations. The recorded trajectory across all fingers lasts a few minutes.


To optimize the model parameters, these trajectories are then replayed as open-loop action sequences in the simulator. 
The optimization objective is to match joint angles after $1$ second of running actions. Parameters 
are adjusted using iterative coordinate descent until the error is minimized. We exclude modifications to the XML
that does not yield improvement over $0.1\%$.


For each joint, we optimize the damping, equilibrium position, static friction loss, stiffness, margin, and the minimum and maximum of the joint range. For each actuator, we optimize the proportional gain, the force range, and the magnitude of backlash in each direction.  Collectively, this corresponds to 264 parameter values.



\subsection{Evaluation metrics}
\label{sup:eval}
Performance was evaluated using precision@$k$ and nDCG@$k$ metrics. Performance was also evaluated using propensity scored metrics, namely propensity scored precision@$k$ and nDCG@$k$ (with $k$ = 1, 3 and 5) for extreme classification. The propensity scoring model and values available on The Extreme Classification Repository~\citep{XMLRepo} were used for the publicly available datasets. For the proprietary datasets, the method outlined in \cite{Jain16} was used. For a predicted score vector $\hat{\mathbf{y}} \in R^L$ and ground truth label vector $\mathbf{y} \in \{0, 1\}^L$, the metrics are defined below. In the following, $p_l$ is propensity score of the label $l$ as proposed in~\citep{Jain16}.
\begin{flalign}
P@k&= \frac{1}{k} {\sum_{l \in rank_k(\hat{\mathbf{y}})}} y_l &
PSP@k&= \frac{1}{k} {\sum_{l \in rank_k(\hat{\mathbf{y}})}} \frac{y_l}{p_l} & \nonumber\\
DCG@k&= \frac{1}{k} {\sum_{l \in rank_k(\hat{\mathbf{y}})}} \frac{y_l}{\log(l+1)} \nonumber
& PSDCG@k&= \frac{1}{k} {\sum_{l \in rank_k(\hat{\mathbf{y}})}} \frac{y_l}{p_l \log(l+1)} & \nonumber\\
nDCG@k&= \frac{DCG@k}{\sum_{l=1}^{\min(k, ||\mathbf{y}||_0)} \frac{1}{\log(l +1) }} \nonumber
 &
PSnDCG@k&= \frac{PSDCG@k}{\sum_{l=1}^{k} \frac{1}{\log l +1 }} &\nonumber,
\end{flalign}

\subsection{Feature Ranking}\label{sec:featureRank}

One of our goals in this work is to understand the physical mechanisms that yield high and low IQ values so that, in the future, we can actuate the observatory to improve the realized IQ.  To accomplish this we need to understand the insights that the ML models decision making processes reveal. To this end, we utilize the methods of integrated Hessians and Shapley values \citep{explaining_explanations_hessians, gilda_mirkwood, gilda_mirkwood_software} for the MDN model. %, and the package \texttt{SHAP} for the machine learning model LGBM.
We use an implementation provided by the \texttt{pathexplainer} software package which compute feature attributions (or importances). The attributions plot ranks the 119 input features, guiding us on how important each feature is, relative to all other ones, in explaining the predicted MPIQ. These enables us to understand the model's decision making process, and to ascertain that the features deemed important by the model make sense physically.
 %On the other hand, the interactions plots help visualize second-order effects -- how one feature interacts with another to simultaneously impact the predicted MPIQ. These measures help us to understand the model's decision making process and help us check that the features deemed important by the model make sense physically.

\subsection{Putting It All Together}\label{sec:putting_it_all_together}

%\seb{This section is very technical, should we just give a high level summary and push the training details onto the appendix?}
\iffalse
As we note above, tabular data has traditionally--and to great success--been analysed using machine learning models, which have shown superior performance over deep learning models. Here, we utilize a host of tree-based machine learning models available in widely known packages, including--Extremely Randomized Trees \citep{extremelyrandomizedtrees}, Random Forests \citep{randomforests}, XGBoost \citep{xgboost}, Gradient Boosting Machines/LightGBM \citep{lightgbm}, and NGBoost \citep{ngboost}. \yst{only discuss the best model? Or if TPOT uses all of these, then you can say we use AutoML which implicitly considers all these variations} We choose to use tree-based models since they are usually more interpretable than linear models due to model-mismatch effects (where the model’s form does not match its true relationships in the data, \cite{model_mismatch_linear_models}. \yst{the last sentence is not clear to me. Please elaborate.} In \citet{shap2}, the authors highlight that even when simpler high-bias linear models achieve high accuracy, low-bias ones (such as tree-based models) are usually preferable, and even more interpretable, since they are likely to better represent the true data-generating mechanism and depend more naturally on their input features. NGBoost is a recently released method based on deicision trees \citep{decision_trees}allows us to automatically obtain both the mean and the standard deviation of predictions for all samples. Instead of using it as the sole machine learning model trained on the CFHT-dataset, we separately train a number of different tree-based models and use their predictions as inputs to an NGBoost model. It has been widely shown that using such an ensemble of models/committee of experts enables one to overcome the limitations of individual models in different parts of the parameter space, and increase generalization \yst{briefly explain how these tree-based methods differ.} \citep{stacking}.
\fi

{\bf Training and test sets:} For both the MDN (Section \ref{sec:mdn}) and the RVAE (Section \ref{sec:rvae}) we partition $\mathcal{D_{F_S,S_S}}$ into two unequally-sized subsets -- a {\it training super-set} containing 90\% of the samples and a {\it test set} containing the rest. We are following a nested cross-validation scenario. We partition the data sets carefully, to ensure that the distribution of MPIQ values in both the test and training sets reflect the distribution in the original data set. To accomplish this we sort the samples by MPIQ values and, starting from the lowest MPIQ value allocate each sample in a round-robin fashion to one of ten buckets generated. We then iterate this process for the training super-set -- again producing a 90-10 split -- to respectively produce the final {\it training set} and the {\it validation set}. We train the models on the training set and record its predictions on the validation and test sets. The validation set guards against over-fitting -- we want our models to learn patterns from the training set, but not to the extent where they fail to generalize to unseen samples. Before making predictions on the test set, we revert the weights of both the MDN and RVAE models to their respective epochs where their respective losses on the validation data set were minimal, as shown in Figure \ref{fig:mdn_training_curve} for the MDN. As a quick reminder, a `prediction' for the MDN is a three-tuple consisting of mean $\mu$, aleatoric uncertainty $\sigma_a$, and epistemic uncertainty $\sigma_e$ for the MPIQ, whereas for the RVAE it is the reconstructed input sample.

{\bf Learning rate and optimizer:} We use a cyclical learning rate scheduler to vary the learning rate from an initial high to a final low value, in multiple cycles; this has been shown to result in a considerably better convergence than using step-wise or constant learning rate schedules \citep{cyclical_learning_rate}. To determine these limits for the MDN and the RVAE, we pick arbitrarily high ($10^{-1}$) and low ($10^{-7}$) limits, exponentially increase the learning rate from the latter to the former in a mere 20 epochs, and evaluate the behavior of the respective loss functions. For the MDN, we determine that at $10^{-3}$ and $10^{-6}$, the loss begins to plateau, as can be seen from Figure \ref{fig:mdn_find_lr}. We thus pick these as the higher and lower limits, respectively, and indicate them by dashed vertical lines. Similarly, from Figure \ref{fig:vae_find_lr} we can see that these limits for the RVAE are $10^{-3}$ and $10^{-5}$. We use the Yogi optimizer \citep{yogi_optimizer} for stochastic gradient descent; this optimizer is an improvement over the commonly used Adam \citep{adam_optimizer}, and we find experimentally that it provides faster convergence. We wrap this optimizer in the Stochastic Weight Averaging optimizer \citep{swa_stochastic_weight_averaging} -- accessible via the \texttt{TensorFlow Addons} library\footnote{\url{https://github.com/tensorflow/addons}} -- and average the model weights every 20 epochs, to overlap with the length of a training cycle. The batch size when using both models is 128.

{\bf Feature normalization and data augmentation:} Finally, we apply strong feature normalization and data augmentation to regularize against over-fitting. Specifically, we use Positional Normalization \citep[PONO]{positional_normalization_pono} layers to capture both the first and second moments of latent feature vectors, and use Momentum Exchange \citep[MoEx]{moex} to mix the moments of one input sample with that of another, to encourage our models to draw out training signal from the moments as well as from the normalized features. In each mini-batch of 128 samples, every feature vector for every sample is added with the feature vector for a randomly picked sample; the probability that this happens is set to 0.5 - this is, half the times, there is no mixing. In case of mixing, the weight assigned to the original sample is picked from a $\beta$ distribution with both concentration parameters set to 100, while the weight of the randomly picked sample is the difference of this from 1 (so that both weights sum to unity). The same random ordering of samples and the same weights are carried over to the model outputs as well (MPIQ for the MDN, the reconstructed input for the RVAE). This augmentation scheme has shown to produce state-of-the-art results, and our own experiments confirm excellent performance. This can be seen in Figure \ref{fig:mdn_training_curve}, where we plot the training and validation losses for one of ten folds; the training loss is significantly higher than the validation loss for a large part of the training process. We insert a PONO layer after each Dense layer in the MDN, and after the penultimate encoding layer in the RVAE. The MoEx layers are inserted before the ultimate Dense layer in the MDN, and the ultimate layer in the RVAE. Each PONO layer is followed by a Group Normalization layer \citep[GN]{group_normalization} with a channel size of 16 (see the MDN in Figure \ref{fig:rvae_plus_mdn}), except when a MoEx layer directly follows the PONO layer, where the former is followed by a Batch Normalization layer \citep{batchnorm}.

{\bf Calibration:} For the MDN, we implement additional steps to calibrate the predicted MPIQ PDFs. We treat each of the 10 training sets (these are obtained after splitting the respective training {\it super-sets} into training and validation sets, as explained at the beginning of this section) as a training {\it super-set}, and the associated validation set as the test set. In other words, we sub-divide the training set into 10 training and validation sets , train on the new training data sets and use the new validation sets as guardrails against over-fitting, and predict MPIQ on the new test sets. After repeating this process a total of 10 times, we now have predictions for the mean and both uncertainties for all samples in the original training set. Finally, we {\it calibrate} our model's predictions on the original test set by using the predictions on the original training set, by following the method described in \cite{crude_probability_calibration}. This is the post-processing step discussed in Section~\ref{sec:DL_probability_calibration}. We repeat this entire process a total of 10 times to cover all samples in $\mathcal{D_{F_S,S_S}}$. We illustrate this workflow in Figure~\ref{fig:cfht_mdn_overview} in Appendix~\ref{sec.workflowFigs}, where in the interest of saving space we show only 3 splits instead of 10.

{\bf RVAE tuning:} For the RVAE, there are a couple of additional considerations. For one, we adopt an annealing methodology to handle the problem of vanishing KL-divergence \citep{cyclical_wkl_annealing}. It is known that the KL-divergence loss term in Equation \ref{eq:VAEeq2} very quickly collapses to 0 if both L$_{\rm REC}$ and L$_{\rm KL}$ are equally weighted. We therefore adopt the methodology suggested by \cite{cyclical_wkl_annealing}: we modify Equation \ref{eq:VAEeq2} by multiplying the second term by a weight scalar W$_{\rm KL}$, and vary this from 0 to 1 in a cyclical fashion, as shown in Figure \ref{fig:vae_weightkl_vs_epoch}. Next, there is the requirement to choose an appropriate $\beta$ in Equation \ref{eq:ce_final}. We choose $\beta=0.005$ based as suggested by \cite{rvae_orig}, and leave the task of finding an optimal $\beta$ to future work. %\sg{(the previous statement will most likely change, as soon as I am able to figure out why all betas give good results and thus how to pick the best among them.)}
Finally, since W$_{\rm KL}$ is annealed with epochs, we need to ensure that our lower and upper learning rates help with convergence for all values of this scalar. From Figure \ref{fig:vae_find_lr}, we see that between learning rates of $10^{-5}$ and $10^{-3}$, the total loss decreases for all values of W$_{\rm KL}$. %since empirical observations show that this offers a good trade-off between the need for high marginal log likelihood for the training samples and low MLL for the OoD samples.reconstruction loss for the training samples

{\bf Overall workflow:} Our overall workflow is as follows:
\begin{enumerate}
    \item For a given train-test split (out of a total of 10) of $\mathcal{D_{F_S,S_S}}$, we use the training set with the MDN, record predictions on the test set, and calibrate them using the methodology described above. We save the weights of the MDN at the epoch of minimum validation loss -- this is shown by the dashed vertical line in Figure \ref{fig:mdn_training_curve}, and for the specific split shown, occurs at epoch 38.
    \item Next, we train the RVAE using the same training set. Similar to the process with the MDN, we revert the model weights back to the epoch of minimum loss, and make predictions on the test set. We gather for the training, validation, and test sets the total loss -L$_{\rm ELBO}$, reconstruction loss L$_{\rm REC}$, and the KL-divergence loss L$_{\rm KL}$. These are plotted in Figures \ref{fig:vae_trainingcurve1} and \ref{fig:vae_trainingcurve2}. We save the $95^{\rm th}$ percentile of -L$_{\rm ELBO, Train}$ as the L2; this is our cut-off between ID and OoD samples. 
    \item Next, we create a small data set of only those samples from the test set where all twelve vents are open. While our goal is to hypothesize the gains in seeing/MPIQ we could have gotten had the vents been in their optimal configuration instead of in the all-open configuration, we believe it is important to be conservative in our estimates. Thus we select only those samples for further processing where we are confident that there were no mechanical malfunctions, high wind conditions, or other system errors that could have prevented the telescope operator from opening all vents.
    \item As a first filter, we select only those samples for which L$_{\rm ELBO, Test} <$ L2, with the intention of filtering out samples for which we are not extremely confident about the ID characteristic.
    \item From this newly created test set, we further only select those samples where our MDN from Step (i) predicts that the true MPIQ is covered by $68^{th}$ percent spread about the median in the predicted MPIQ PDF. This is again enacted in the interest of obtaining conservative predictions downstream.
    \item From the filtered test set in Step (v), we create a permutated data set by toggling all twelve vents ON (==1) and OFF (==0). For a total of 12 vents, this results in $4095$ new samples for each input sample, where the remaining 107 features remain unchanged. The $4096^{\rm th}$ sample is the input test sample itself, since its vents are already in the all-open configuration. For each of the these $4095$ samples, we again apply the same filter as in Step (iv) -- filtering out those vent configurations which, given the training set, are OoD.
    \item Finally, we obtain MPIQ predictions using the MDN for all samples in the permutated data set, created by collation ID permutations for all selected test samples.
\end{enumerate}

{\bf Identifying predictable vent variations \& separating in-distribution from out-of-distribution samples:} In Figure \ref{fig:vae_hist_mll_ood}, we demonstrate our methodology for separating ID samples from OoD ones. As should be expected, most test samples are ID, as are $95\%$ training samples (by definition). A striking yet expected result is that only a very small sample of possible permutations are ID. The reason for this becomes clear from Figure \ref{fig:vae_hist_hds}, where we plot histograms of the different vent configurations in the training set -- 0 on the x-axis corresponds to the all-open configuration, while 1 to all-closed. The vast majority of samples, $\sim80\%$, have all vents closed, while $\sim20\%$ have either all vents or most vents open. Thus the vast majority of samples in the permutated dataset, where the twelve vents can take arbitrary configurations -- say half open and half closed, corresponding to a Hamming distance (x-axis in Figure \ref{fig:vae_hist_mll_ood}) of 0.5 -- are those that the RVAE has not seen before, and thus classifies as OoD.

{\bf Process illustration:} Finally, we illustrate the workflow delineated in Steps (ii) through (vii) above in Figure \ref{fig:cfht_rvae_overview}.

%Using the scheduler thus defined, we train both the RVAE defined in Section \ref{sec:rvae}, and the MDFN defined in Section \ref{sec:mdn}, for 3 cycles, each being 20 epochs long, and evaluate the loss on training and validation splits. In Figure \ref{fig:mdn_training_curve} we visualize this procedure for MDFN, for 1 of 10 training-validation splits, where the training+validation data itself is 1 of 10 folds of the training-test splits of $\mathbf{X_{\rm train}}$ (see Figure \ref{fig:cfht_mdn_overview} for an overview of the workflow). We track the training and validation losses, and revert the model weights back to their values at the epoch corresponding to minimum of the validation loss. For the particular nested split in Figure \ref{fig:mdn_training_curve}, this epoch is number 47, as is indicated by the dashed vertical line. Finally, we apply strong data augmentation to regularize against over-fitting. Specifically, we use the momentum exchange \citep{moex} layer in the penultimate layers of both our networks, with probability $p=1$ and the concentration parameters $\alpha$ and $\beta$ of the $\beta-$ distribution equal to 1 (see xxx for details about the meanings of these parameters).%We use the same process for both $\mathcal{D_{F_S,S_S}}$ and $\mathcal{D_{F_L,S_S}}$.


\iffalse
\begin{algorithm}
\caption{Probabilistic MPIQ Prediction}
\label{algo:putting_it_all_together}
\begin{algorithmic}[1]
\renewcommand{\algorithmicrequire}{\textbf{Input:}}
\renewcommand{\algorithmicensure}{\textbf{Output:}}
%\renewcommand{\algorithmicinitialize}{\textbf{Initialize:}}
\REQUIRE (i) The matrix consisting of environmental and dome measurements $\mathbf{X} = \{\mathbf{x}_{n}^{(1:d)}\}^{N}_{n=1}$ where $N$ is the number of samples, and $d$ is the number of features.\\ Also, (ii) the vector consisting of MPIQ measurements $\mathbf{y} = \{y_{n}\}^{N}_{n=1}$.
\ENSURE (i) predicted probability densities for the input dataset: \tcr{(SCD: I think the following should have $\mu$ and $\sigma_a$ and $\sigma_e$ in it)} \tcb{$\mathbf{y}_\mathrm{pred} =\{y_{n}\vert_\mathrm{pred}^{(1),(2)}\}^{N}_{n=1}$.} \\
%Each predicted PDF is characterized by a mean and a standard deviation. \\
(ii) Feature importance matrix: $\boldsymbol{\mathrm{F}} = \{f_{n}^{(1:d)}\}^{N}_{n=1}$.
\FOR{iteration $i_\mathrm{test}$ in range [1, 10]}
\STATE Divide $\mathbf{X}$ into two subsets, a training subset $\boldsymbol{\mathrm{X_{train}}}\vert_{\hat{i_\mathrm{test}}}$, and a testing subset $\boldsymbol{\mathrm{X_{test}}}\vert_{i_\mathrm{test}}$,into the $9:1$ ratio (90\% training 10\% testing) described above. Here the $\hat{i_{\rm{test}}}$ represents the collation of all \tcr{folds} \tcb{(What is a ``fold''?)} besides $i_{\rm{test}}$. \tcr{(SCD: What is the $\vert_{\hat{i_\mathrm{test}}}$ for?)}
\FOR{iteration $i_\mathrm{val}$ in range [1, 10]}
\STATE Sub-divide the training set into training and validation sets in a $4:1$ ratio \tcr{(SCD: In text says this is alsp 90/10)}; that is, $\boldsymbol{\mathrm{X_{train}}}\vert_{\hat{i_\mathrm{test}}}$ $\rightarrow$ $\boldsymbol{\mathrm{X_{train}}}\vert_{\hat{i_\mathrm{val}},\hat{i_\mathrm{test}}} + \boldsymbol{\mathrm{X_{train}}}\vert_{i_\mathrm{val},\hat{i_\mathrm{test}}}$.
\STATE Train on $\boldsymbol{\mathrm{X_{train}}}\vert_{\hat{i_\mathrm{val}},\hat{i_\mathrm{test}}}$ and obtain predictions $y_{\mathrm{val}}\vert_\mathrm{pred, i_\mathrm{val}}$ by predicting on $\boldsymbol{\mathrm{X_{train}}}\vert_{i_\mathrm{val},\hat{i_\mathrm{test}}}$.
\ENDFOR
\STATE Collate predictions on all 10 validation sets and sample from the resulting PDFs to obtain $3\times50$ values per sample: $50^{th},$ $15.9{\rm th},$ and $84.1^{\rm st}$. \tcr{(SCD: what is 15.9 and 84.1?  Not following here...)} Combine them as described in Section \ref{sec:uncertainty_quantification} to obtain three quantities per sample: mean, epistemic uncertainty, and aleatoric uncertainty.% lower prediction level $\mu - 1.96\sigma$, and upper prediction level $\mu + 1.96\sigma$. %Call this tuned model \textbf{insert model name}$_\mathrm{tuned}$.
\STATE Train the model on $\boldsymbol{\mathrm{X_{train}}}\vert_{\hat{i_\mathrm{test}}}$ and predict on $\boldsymbol{\mathrm{X_{test}}}\vert_{i_\mathrm{test}}$, to obtain three values per sample just like above.
\STATE Use the collated predictions on the super-set of all validation sets obtained from Step 7, to calibrate the predictions on the test set from Step 8, by using the CRUDE method of \cite{crude_probability_calibration}.
\STATE Use the same model trained in Step 8, to obtain feature importances for samples in $\boldsymbol{\mathrm{X_{test}}}\vert_{i_\mathrm{test}}$. Call the matrix $\boldsymbol{\mathrm{F}}_{i_{\rm test}}$.
\ENDFOR
\STATE Collate all predictions from Steps 8 and 9 to obtain uncalibrated and calibrated predictions, respectively, for all samples in $\mathbf{X}$.
\STATE Collate all $\boldsymbol{\mathrm{F}}_{i_{\rm test}}$s to obtain $\boldsymbol{\mathrm{F}}$.
\end{algorithmic} 
\end{algorithm}
\fi


%%%%%%%%%%%%
%\subsection{Semi-supervised}
%\label{sec:semiSupervised}


