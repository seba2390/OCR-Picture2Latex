\begin{itemize}
\item \emph{Wide-field InfraRed Camera (WIRCam):} WIRCam  is the near infrared mosaic imager mounted at the prime focus of the 3.6m CFHT telescope. It saw first light in November 2006 and has been offered for science in all observing semesters since February 2006. 
WIRCam has a field of view of 20 sq.min. The WIRCam focal plane is made of a mosaic of four HAWAII2-RG detectors, each containing 2048 x 2048 pixels, with a sampling of 0.3 arc second per pixel. WIRCam is fitted with an image stabilization unit (ISU) using on-chip guiding. During an exposure, the image stabilization signal is obtained by repeatedly reading out a small, 14x14 pixel region centered on a bright star on each of the four detectors, while the integration continues uninterrupted for the rest of the pixels. The nominal near-infrared seeing of 0.8 arc.sec measured at the CFHT site on Mauna Kea is well sampled by the 0.3 arc.sec pixel scale of WIRCam, making it adequate for most science programs. However, occasionally the seeing has been observed to improve to 0.5 arc.sec or better, under which the PSF becomes undersampled. In order to facilitate science programs which can benefit from such exceptional seeing conditions, WIRCam can use its image stabilization unit to micro-step the image with 0.15 arc.sec sampling. 
\item \emph{Mauna Kea atmospheric monitor (MKAM)} MKAM is a collaborative project between IFA (Institute for Astronomy, University of Hawaii), CFHT and W.M. Keck. The goal of which is to install a seeing monitor on the CFHT site for use by all the observatories on the summit.  Provides a measure of the Fried parameter (r0) at a wavelength of 0.5 microns with a goal of an precision of better than $5\%$ at the observed zenith angle and azimuth and an estimate of r0 corrected to the zenith. The r0 value should be corrected for the finite exposure time of the instrument. Goal of an accuracy of $5\%$ in r0 and the measurements will be from at least six meters above grade.
Fried's parameter r0 quantitatively expresses the image degradation due to atmospheric turbulence.  r0 represents the diameter of the coherent cells in the incident wave at the telescope pupil.
\url{http://articles.adsabs.harvard.edu//full/1981SoPh...69..223R/0000227.000.html}
the MASS profiler is an instrument in addition to the DIMM in MKAM.
MASS Turbulence Profiler. MASS is an instrument to measure the vertical distribution of turbulence in terrestrial atmosphere by analysing the scintillation (twinkling) of bright stars. When stellar light passes through a turbulent layer and propagates down, its intensity fluctuates.
another reference to cfht/rene/r0 from keck...
https://www2.keck.hawaii.edu/optics/kpao/files/KAON/KAON303.pdf
in particular item 2.
and from Tokovonin 
https://iopscience.iop.org/article/10.1086/428930
\end{itemize}

\iffalse
\emph{Echelle SpectroPolarimetric Device (ESPaDOnS)}: ESPaDOnS is a bench-mounted high-resolution echelle spectrograph and spectropolarimeter which was designed to obtain a complete optical spectrum (from 370 to 1,050 nm) in a single exposure with a resolving power of about 68,000 (in spectropolarimetric and 'object+sky' spectroscopic mode) and up to 81,000 (in 'object only' spectroscopic mode). ESPaDOnS was commissioned and fully operational by February 2005. 

\emph{Spectromètre Imageur à Transformée de Fourier pour l'Etude en Long et en Large de raies d'Emission(SITELLE)} SITELLE is the optical imaging Fourier transform spectrometer (IFTS) at CFHT that saw first light in July 2015. It provides integral field unit (IFU) spectroscopic capabilities in the visible spectrum (350 to 900 nm) over an $11' \times 11'$ field of view, with a variable spectral resolution, depending on the requirement of the observer, from R=2 to R>104 (for low to high spectroscopic studies). The SITELLE focal plane is made of 2 cameras, each of them a four output 2048 x 2048 pixels, low noise, CCD231-42 e2v chip that allows attaining 5e noise in 1 second read time and 3.5 electron in 2 seconds. The nominal optical seeing of 0.8 arcsec measured at the CFHT site on Mauna Kea in the R band is well sampled by the 0.32 arc.sec pixel scale of SITELLE, making it adequate for most science programs. SITELLE will therefore provide about 1 million independent spectra at once. 

\emph{SPectropolarimètre InfraROUge (SPIRou)} Near-infrared spectropolarimeter with high radial-velocity precision  SPIRou is fiber-fed from the Cassegrain focus of the CFHT. The SPIRou Cassegain unit includes a polarimeter to change the polarization of the incident light and derive linear or circular polarization states of the observed target. The SPIRou spectrograph is embedded in a cryogenic vessel cooled down to a temperature of 80 K and stabilized at a precision below 2 mK. The spectrograph provides YJHK spectra from 0.95 to 2.35 microns in a single shot, at a spectral resolution of ~75,000. SPIRou will use an Hawaii 4RG detector, which is currently in the making at Teledyne Scientific and Imaging. It will have 4kx4k 15-micron pixels that will be read continuously in order to leave the cryostat temperature undisturbed. Since it is to be delivered after the lab tests in France, the SPIRou team is using an Hawaii H2RG 2kx2k detector with 18 microns pixels for alignments and performance tests. This detector covers a large fraction of the spectral range and gets most of the spectral resolution. \fi

