\section{Related Work}\label{sec:relatedWork}

The summit of Maunakea was selected as the site for CFHT due to its excellent astronomical observing properties: low turbulence, low humidity, low infrared absorption and scattering, excellent weather,  clear nights.  Image quality, or ``seeing'', is quantified using the full-width half-max (FWHM) measure. FWHM, expressed in arcseconds ($''$), is calculated as the ratio of the width of the support of a distribution, measured at half the peak amplitude of the distribution, to  half the peak-amplitude value; smaller FWHM is better.  For example, the FWHM  of the Gaussian distribution is $2 \sqrt{2 \log 2} \sigma$, roughly $2.4$ times the standard deviation $\sigma$.  In our application, FWHM operationally quantifies the degree of blurring of uncrowded and unsaturated images of point sources (such as a star or a quasar) on the central CCDs of a MegaCam frame. %\citep{racine2018}. 
The FWHM measured this way, referred to as image quality or IQ, is an aggregate of multiple sources.\footnote{Note that larger FWHM $\implies$ higher ``seeing'' $\implies$ poorer image quality (IQ) $\implies$ more arc-sec.  So, lower FWHM which equates to a better IQ (fewer arc-sec) is desired.} The main contributors to FWHM / IQ are: imperfections in the optics ($\iqOpt$), turbulence induced by the dome ($\iqDome$), and  atmospheric turbulence ($\iqAtmo$). These contributions are well-modeled as being independent and as combining to form the measured IQ ($\iqMea$) according to
\begin{align}
\iqMea^{5/3} =  \iqOpt^{5/3} + \iqDome^{5/3} + \iqAtmo^{5/3}.
\label{eqn:iq_measured}
\end{align}
If the contributions were modeled by a Gaussian distribution, the exponents in Equation (\ref{eqn:iq_measured}) would be $2 = 6/3$ (because variances of independent Gausian random variables add).  The $5/3^{\rm rd}$ power is due to the spectrum of turbulence which was characterized by Kolmogorov in 1941~\citep{tartarskii}.  We note that while the contribution of the optics is not due to turbulence, we still use the power of $5/3$ in our model for consistency. Finally, we note that of the three contributors we can only influence $\iqDome$ through actuation of various observatory controls.
%\tcb{(SCD: I'm not 100\% sure this is true, can we reduce the optics contribution through better control of mirror temperature?  Also, we use the term "seeing", is seeing a synonmy for IQ contribution?)}
    
While the mean free atmosphere ($\textrm{IQ}_{\textrm{Atmosphere}}$) seeing on Maunakea is estimated to be about $0.43''$~\citep{salmon2009cfht}, in practice, the IQ realized at CFHT is usually worse (i.e. the seeing is higher). Through 40 years of effort by the CFHT staff and  consortium scientists,  $\iqMea$ has steadily decreased, from early values of $2''$ or greater to its current median value of around $0.71''$. Removal of  $\iqOpt$ further reduces this figure to $0.55''$ (see~Figure~\ref{fig:hist_preliminaries}, Equation \ref{eqn:iq_optics_correction}, and Section \ref{sec:feature_engineering}).\footnote{\url{https://www.cfht.hawaii.edu/Science/CFHLS/T0007/T0007-docsu11.html}} In the remainder of this section we discuss prior efforts to quantify IQ and to reduce IQ. Later, in Section~\ref{sec:data} when we discuss our data sources, we return to~(\ref{eqn:iq_measured}) and step through a number of sources of variation in observing conditions (e.g., wavelength of observation, elevation of observation) that we correct to produce a normalized dataset in which IQ measurements from distinct observations can be directly compared.

\iffalse
\begin{figure*}
% SF: removing this plot: already published, and copyright issues.
%\begin{subfigure}{0.49\textwidth}
%    \centering
%    \includegraphics[width=\linewidth]{figures/relatedWorkFig01.png}
%    \caption{Early IQ measurements and derived errors \citep{salmon2009cfht}.}
%    \label{fig:hist_moneyontable}
%\end{subfigure}
%\hfill

%\begin{subfigure}{0.49\textwidth}

\centering
\includegraphics[width=\linewidth]{figures/mkam_seeing_distribution.pdf}
%\caption{}
%\label{fig:hist_mkam}
%\end{subfigure}
\caption{Distribution of seeing measured by the Multiple Aperture Scintillation Sensor (MASS) atmospheric profile, the MKAM Differential Image Motion Monitor (DIMM) and IQ effectively measured at MegaPrime since 2010.}
\label{fig:hist_preliminaries}
\end{figure*}

\fi


\begin{figure*}
\begin{subfigure}{0.99\textwidth}
    \centering
    \includegraphics[width=.98\linewidth]{figures/mpiq+mkam_seeing_distribution.pdf}
    %\caption{Histograms of {\it calibrated epistemic} uncertainty from the MDN, for various data sets. The vertical line is the $95^{\rm th}$ percentile  for the training set.}
    \label{fig:mpiq_mkamiq}
\end{subfigure}
\newline
\begin{subfigure}{0.99\textwidth}
    \centering
    \includegraphics[width=.98\linewidth]{figures/FWHM_history.pdf}
    %\caption{Histograms of {\it calibrated epistemic} uncertainty from the MDN, for various data sets. The vertical line is the $95^{\rm th}$ percentile  for the training set.}
    \label{fig:fwhm_history}
\end{subfigure}
%\centering
%\includegraphics[width=\textwidth]{figures/FWHM_history.pdf}
\caption{Seeing evolution and distribution. \textbf{Upper left:}  %\tcr{ (MASS is best possible -- above ground layer (remove this one), MKAM is without dome (atmosphere 0.55"), Megacam (atmosphere 0.55" + Local 0.43" about 0.75"), so lower bound from green and target of blue)}
Distribution of seeing measured by %the Multiple Aperture Scintillation Sensor (MASS) atmospheric profile,
the MKAM Differential Image Motion Monitor (DIMM) and effective IQ measured at MegaPrime since 2010. Effective MPIQ is the measured IQ less the contributions from optics -- see Equation~(\ref{eqn:iq_measured}) and Section \ref{sec:data_cleaning}. Both the MKAM and the MPIQ curves peak at $\sim 0.55''$; the former contains relatively higher seeing contribution from the ground layer, while the latter includes contribution from the dome itself (see Section \ref{sec:relatedWork} for detailed discussion). Both peaks are higher than the best possible seeing at the site of $\sim0.43''$. \textbf{Upper right:} Distribution of seeing {\it after} the installation of the vents as a function of vent configuation: \textit{all-open} or \textit{all-closed}.  Observe the statistics of the former are much better than those of the latter.  We also plot the MKAM histogram, which is basically unchanged from prior to the installation of the vents.  \textbf{Lower:} Quarterly averaged MegaPrime IQ of the CFHT. The wiggly curve of decaying mean and oscillation amplitude is a model that peaks in midwinter, when the outside air temperature tends to be colder than the dome air. The drop in July of 2014 corresponds to when the vents first started to be used; this is why in the data input to our models we do not use samples from before this month.}%\tcr{(SCD: Think this caption needs to be expanded or discussed in text.  `Cumulative Mass Profile Seeing'' is not discussed, nor is ``MKAM DIMM Seeing''.  Also I suggest we take out the reference to Salmon et al. 2009.  I assume the reference is in there b/c of the comments on the wiggly curve, but I think that is a familiar enough insight that we don't really need to say that.  Note: drop in 2014 may be from when vents first started to be used -- Sankalp will check data -- maybe we should start data set after the median drops.)}\sg{(Sankalp: Sebastien, could you please do this, since these plots were generated using the data you provided?)}}
\label{fig:hist_preliminaries}
\end{figure*}

Early published efforts to quantify and reduce the IQ at CFHT (e.g. \cite{racine1984}) detail campaigns to minimize turbulence inside and around the dome, including analysis and measurements of the opto-mechanical imperfections of the telescope. The team led by Ren\'e Racine estimated that if the in-dome turbulence was corrected and the telescope imperfections were removed, the natural Maunakea seeing would  offer images with IQ below $0.4''$ FWHM on one-quarter of the nights. Later efforts ~\citep{racine1991mirror} used data from the (then) new HRCam, a high-resolution imaging camera at the prime focus of CFHT, to develop a large and homogeneous IQ data set. They correlated their IQ data with thermal sensor data, through which they were able to identify and quantify ``local seeing'' effects.  Their main findings, relevant to our work, are listed below.
 %Mirror seeing is an important source of image spread and amounts
\begin{enumerate}
\item The contribution of mirror optics $\iqOpt$ amounts to about $0.4'' (\Delta T)^{6/5}$ where $\Delta T$ is the temperature difference between the primary mirror and the surrounding air.
\item The dome contribution $\iqDome$ amounts to about $0.1'' (\Delta T)^{6/5}$ where $\Delta T$ is the temperature difference between the air inside and outside of the dome.
%\item Optical aberrations from the CFHT primary mirror and from HRCam imposed (for the HRCam imager) an IQ limit of FWHM $= 0.38''$.
\item The median natural atmospheric seeing at the CFHT site $\iqAtmo$ is $0.43'' \pm 0.05''$.  The 10th and 90th percentiles are roughly $0.25''$ and $0.7''$.
\end{enumerate}

More recent follow-up work is presented in~\cite{salmon2009cfht}.  The authors correlate measured IQ using the (then) new MegaCam with temperature measurements. They analyze $36,520$ MegaCam exposures made in the $u$, $g$, $r$, $i$, and $z$-bands in the three year period between August 2005 and  August 2008.  They find strong dependencies of the measured IQ on temperature gradients. Furthermore, in  Table 4 of~\cite{salmon2009cfht} the authors categorize important factors that contribute to the seeing -- atmosphere, dome, optics -- and provide estimates of their respective contribution.
 % The authors find numerous contributing factors to the seeing.  In  Table~\ref{table:opticsCorrection} we categorize these factors into atmospheric, dome-seeing, and optics.  The right-hand column of the table provides estiamtes of the overall atmospheric / dome seeing / optics contributions to IQ. As discussed in the paper these number are close to, but not exactly equal to what an application of Equation~(\ref{eqn:iq_measured}) would yield due to correlations in the temperature gradients (cf.~Section~5.2 of~\cite{salmon2009cfht}).  Note, if we apply Equation~(\ref{eqn:iq_measured}) to combine the atmospheric and dome-seeing contributions we would an estimate of $0.746''$ for the median seeing value.
As the authors discuss, these estimates update the findings of~\cite{racine1991mirror}.  The most significant findings of~\cite{salmon2009cfht} can be summarized as follows.
\begin{enumerate}
%\item In the convective mode, the thermally-induced image full width at half-maximum intensity (FWHM) grows with the temperature gradient and path length $L$ at the rate of $\simeq 0.2'' \cdot (\Delta T / L)^{6/5} \cdot L^{3/5}$.
%\item For a given $|\Delta T|$, thermal convection is $\simeq 3$ times more detrimental to image quality than is thermal inversion. 
\item The orientation of the dome slit with respect to the wind direction has important effects on IQ.
\item The median dome induced seeing $\iqDome$ before the installation of the vents in 2013 was $0.43''$.
\item The seeing contribution from optics and opto-mechanical imperfections $\iqOpt$ varied from $0.46''$  in the u-band  to $0.28''$  in the i-band. 
\item Atmospheric seeing $\iqAtmo$ at the CFHT site at a wavelength of $500$ nm and an elevation of $17$m above ground was measured using a separate imager.  The median $\iqAtmo$ measured was $0.55''$.  This estimate of atmospheric seeing is independent of effects related to the dome and the optics.
%\item The characteristics value of the outer scale of turbulence is $30$m. 
\end{enumerate}


\iffalse
\begin{table}
\centering
\begin{tabular}{@{}l l l@{}}
    \toprule
    \toprule
    \multicolumn{2}{c}{Individual Components and Values} & Total ('')\\
    \midrule
    Atmosphere & General & \multirow{2}{*}{$\left.\begin{array}{l}
        0.49\\
        0.20\\
        \end{array}\right\rbrace 0.55$} \\
     & Ground layer &  \\
     \\
     & Primary mirror & \multirow{7}{*}{$\left.\begin{array}{l}
        0.09\\
        0.11\\
        0.15\\
        0.08\\
        0.10\\
        0.08\\
        0.25\\
        \end{array}\right\rbrace 0.43$} \\
     & Caisson &  \\
     & Tube &  \\
    Local seeing & Cage &  \\
     & Slit &  \\
     & Wind &  \\
     & Other (dome wake?) &  \\
    \\
     & Primary Mirror & \multirow{3}{*}{$\left.\begin{array}{l}
        0.24\\
        0.08\\
        0.18\\
        \end{array}\right\rbrace 0.33$} \\
    Optics & MegaCam &  \\
     & Other optics, etc.& \\
    \midrule
    Total &  & \hspace{0.5em} 0.85 \\
    \bottomrule
\end{tabular}
\caption{Factors that contribute to median IQ (FWHM in arc-seconds, $500$ nm, at zenith), partitioned by category.  Table is  from~\cite{salmon2009cfht}, cf. Table~4 in that paper.  ``Local seeing'' is what we refer to as ``dome seeing'' in this paper. \tcr{(TODO: Simplify this table just to keep the Atmosphere / Local / Optics of 0.55"  / 0.43" / 0.33" and tie in to Fig 2)}}
\label{table:opticsCorrection}
\end{table}
\fi


\iffalse
\begin{figure}
    \centering
    \includegraphics[width=\linewidth]{figures/relatedWorkFig03.png}
    \caption{Median of tube seeing are plotted against the air temperature differences between the telescope’s caisson central and top end. The plot shows the total seeing, with local contributions removed as well as the final regression fit line and the minimum expected IQ value expected.~\citep{salmon2009cfht}.}
    \label{fig:relWorkFig03}
\end{figure}
\fi

The culminating result of these studies that analyzed the delivered IQ was the December 2012 installation, and July 2014 initial use, of the 12 dome vents depicted Figure~\ref{fig:dome}. Since their installation, CFHT operators have kept all 12 vents completely open as often as possible, barring conditions of mechanical failure and strong winds. As already mentioned, this allows  faster venting of internal air and  equalization of internal and external temperatures.\footnote{\url{http://www.cfht.hawaii.edu/AnnualReports/AR2012.pdf}}  The vent-related improvement in seeing has been dramatic, %Prior to July 2014, the median $\iqMea$ was $0.87''$, and reduced to about $0.79''$. Even more impressive is the median MPIQ achieved when the vents were all open .
with median $\iqMea$ improving from about $0.67''$ to $0.55''$~\footnote{\url{http://www.cfht.hawaii.edu/AnnualReports/AR2014.pdf}}. %\tcr{(SCD: Hey guys, I don't what is up but the median of the right suplot of Figure \ref{fig:hist_preliminaries} is not near to $0.60''$, it is more like $0.85''$ and prior to installation it looks closer to $0.9''$.  Need to reconcile this.  I also put a correlated comment first paragraph right-hand column on page 4 about the current median value (see red font).}

In order to have an external, regularly sampled seeing reference, we use the Maunakea Atmospheric Monitor \citep[MKAM,][]{Skidmore+09,mkam2}. This telescope, dedicated to seeing monitoring, is mounted on top of a weather tower just outside of CFHT. It has a composite instrument, including a Multi Aperture Scintillation Sensor (MASS) and a Differential Image Motion Monitor (DIMM). We only use data from the latter as the former is insensitive to the lower layers of turbulence. DIMMs measure seeing by computing the variance of the relative motion of the images formed by two separate sub-apertures, therefore probing the curvature of the wave-front. This variance can be directly related to the the full-width at half-max (FWHM) of the point spread function (PSF) in long exposures given the wavelength of observation and the sub-apertures diameter \citep[see e.g. ][]{Sarazin+90}. While MKAM measurements are free of the CFHT dome contribution to seeing, they are also sensitive to part of the ground layer contributions to seeing that are not seen by the CFHT instruments, partly due to the lower altitude of the weather tower as compared to the CFHT aperture, and to localized differences in the summit turbulence in the first few meters above ground. MKAM thus serves as a slightly noisy seeing reference for Megacam, free of CFHT dome seeing contributions.

In the left sub-plot of Figure \ref{fig:hist_preliminaries}, we plot histograms of the \textit{corrected} seeing values from MegaCam both before (starting February $2^{\rm nd}$ 2002) and after July $7^{\rm th}$ 2014, when the vents started being used; the latter is the the start date of our data set used in the remainder of this work. We compare these with the seeing distribution from MKAM and MegaCam for observations since 2002. Corrected seeing removes the contribution of the telescope optics from the measured seeing, and is defined in Equation \ref{eqn:iq_optics_correction}. In the top-right plot in Figure \ref{fig:hist_preliminaries}, ``Vents Open'' refers to samples where the 12 vents are either all open, or at most one of them is closed, whereas ``Vents Closed'' incorporates samples where all 12 vents are closed. As can be seen, the introduction of the vents has reduced the median MPIQ by $0.21'' = \left((0.55^{5/3}-0.48^{5/3})^{3/5}\right)$, and the mean MPIQ by $0.27'' = \left((0.61^{5/3}-0.51^{5/3})^{3/5}\right)$. However, there is still money on the table -- the estimated free-air, observatory-free IQ at the CFHT site is estimated to be $\sim0.43-0.44''$ \citep{salmon2009cfht}. This means that there is still a possible improvment in median IQ of $0.15'' = \left((0.48^{5/3}-0.435^{5/3})^{3/5}\right)$, and a mean IQ of $0.23'' = \left((0.52^{5/3}-0.435^{5/3})^{3/5}\right)$. This range of improvement was independently verified by another CFHT team in 2018 \citep{racine2018}. They found that, when open, the dome vents on average reduce IQ by $0.37''$. While this number is significantly larger their either of our estimates of $0.21''$ or $0.27''$, their estimate of degradation of IQ of about $0.20''$, caused by residual eddies induced by thermal differences in the dome, closely matches our own. It is precisely this residual part of $\iqDome$ that we aim to capture in our work. We later show in Figures \ref{fig:common_deltaoptimalreference_vs_nominal_iqs} and \ref{fig:common_deltaoptimalreference_vs_nominal_iqs_robust}, and in Section \ref{sec:resultsRelContribIQ} that our models are indeed able to capture these improvements.

We remind the reader, as mentioned above, that CFHT operators have kept all 12 vents completely open as much as possible.  They have chosen this manner of operation as they had no basis upon which to choose a more varied configuration of the dome vents.  Although, we note that fluid flow modeling conducted during the vent design process predicted that intermediate settings (i.e., neither all-optimal nor all-closed) would optimally reduce internal turbulence \citep{wind_tunnel_test}. By tuning the dome vent configurations, based on current environmental conditions, to a setting between all-open and all-closed, we aim to reduce this residual.

% SF: COULD NOT reproduce this figure, commenting for now.
%\begin{figure*}
%    \centering
%    \includegraphics[width=\linewidth]{figures/relatedWorkFig04.png}
%    \caption{The left panel plots the median 5/3 power differences between the seeing values from a %DIMM mounted in the CFHT dome and the MKAM DIMM against the slit-to-wind azimuth differences for vents CLOSED and  OPEN conditions \citep{bauman2014dome}.  The right panel plots the median 5/3 power differences $\Delta IQ_0$ between the optics-removed $500$ nm zenith IQ values for MegaCam and for the MKAM DIMM  against the slit-to-wind azimuth differences. Typical statistical one-sigma uncertainties are $\pm 0.03''$ \citep{bauman2014dome}.}
%    \label{fig:relWorkFig04}
%\end{figure*}

From a programmatic perspective, our work is a natural extension of~\cite{racine1984, racine1991mirror,salmon2009cfht, racine2018}. While these prior investigations correlated IQ with measurements of temperature gradients, our work tries to relate all of the metrics (not solely the temperature metrics) with the $\iqMea$ through the application of advanced machine learning techniques. Further, rather than only establishing correlation, we also seek to understand whether by actuating the dome parameters under our control we can improve the delivered $\iqMea$. Recent work at the Paranal Observatory by~\cite{milli2019nowcasting_paramal} similarly collected 4 years of sensor data, and trained random forest and neural networks to model and forecast over the short term ($<2$ hrs) the DIMM seeing and the MASS-DIMM atmospheric coherence time and ground layer fraction. Their early results demonstrate good promise, especially for scheduling adaptive optics instruments.
% SF: the difference with us: they predict 3 things, but nothing probabilistic, smaller data set, and not the actual measured IQ.


Finally, we mention recent work~\citep{lyman2020} by the Maunakea Weather Center (MKWC) which takes a macro approach to predict $\iqAtmo$.  The authors tap into large meteorological modeling models.  They start from the NCEP/Global Forecasting System (GFS) which outputs a 3D-grid analyses for standard operational meteorological fields: pressure, wind, temperature, precipitable water, and relative humidity.  Coupling these predictions with advanced analytics and decades of  MKAM DIMM seeing data, \cite{lyman2020} predict the free air contribution ($\iqAtmo$) to seeing on the mountain.  Their work is complementary to ours in that we take in our local sensor measurements to predict (and reduce) the effect of $\iqDome$ on $\iqMea$, while \cite{lyman2020} directly predict $\iqAtmo$. In the long term these two models can be combined to yield improved seeing estimates, forecasts, and decisions.


%Their  work summarizes the experience at MKWC in anticipating optical turbulence at the summit of Maunakea, accrued through years of daily operational forecasting, and continuous comparison between MKWC official forecasts, model guidance, and observational measures of seeing.  They present quantification of the factors that impact seeing, including wind shear, atmospheric stability patterns, and optical turbulence, and document the seasonal and intra-seasonal variations in seeing.  In addition, they claim seeing forecasts  for the Maunakea astronomical observatories at the RMS errors level of $< 0.25''$ since 2012. Furthermore they conclude:
%\begin{enumerate}
%\item The best seeing on average occurs during the summer months, when summit-level conditions are more favorable with lighter winds throughout the atmosphere and less risk of moisture and clouds over the summit.
%\item August and September are months when the contribution from free atmospheric turbulence is lowest because of the more northerly position of a weaker summer jet stream and lighter winds aloft over Hawaii.
%\item On average the contribution of ground layer turbulence is two-thirds to three-quarters of the total seeing, with free atmospheric turbulence contributing the remainder.
%\item Summit winds of $2.5-5$ m/s are associated with the smallest contributions to seeing from the ground layer.
%\item When summit winds are below $2.5$ m/s, shear instability in the ground layer results in bursts of optical turbulence, resulting in intermittent poor seeing.
%\item The summit seeing gradually deteriorates as winds increase above $5$ m/s, with poor seeing and large variability in seeing occurring with summit winds greater than $10$ m/s.
%\item Low-level moisture infrequently reaches the summit, however, when it does the moisture gradients can play a role in degrading seeing by contributing to rapid fluctuations in air density exacerbated by evaporative cooling. In general, summit moisture is accompanied  by less stability in the atmosphere below the summit, often associated with the absence of, or an elevated trade-wind inversion.
%\item When the ground layer turbulence is minimal, free atmosphere turbulence accounts for two-thirds of the total seeing.
%\end{enumerate}


