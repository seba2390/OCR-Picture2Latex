\iffalse
We leverage state-of-the-art machine-learning methods and archival data from the Canada-France-Hawaii Telescope (CFHT) to predict observatory image quality (IQ) from environmental conditions and observatory operating parameters. Drawing on nearly a decade's worth of collected data we develop accurate and interpretable models of the complex dependence between data features and observed IQ for CFHT's wide field camera, MegaCam. Using specialized loss functions and neural network architectures, we predict the  probability distribution function (PDF) of the IQ associated with any given sample. One of our main results is that, based purely on environmental and observatory operating conditions, we can predict the effective MegaPrime IQ to a mean accuracy of $\sim0.07''$.  We further explore how reconfiguration of adjustable observatory parameters can effect improvements in IQ.  In particular, we consider actuation of 12 dome ``vents'', installed in 2013-14, to accelerate the flushing of hot air from the dome, thereby reducing internal air turbulence and improving IQ.  To enable trustworthy predictions of the effect of vent actuation, we first need to differentiate between  uncertainties that arise from the randomness inherent to the input data and those which arise due to imperfect modeling. Using predictions of these uncertainties, in conjunction with probabilistic generative modeling, we identify candidate vent adjustments that are in-distribution and, for the optimal in-distribution vent configuration we calculate the predicted reduction in required observing time.  The IQ prediction reduction, averaged across all samples, is about \sg{$\sim25\%$}. Finally, we use Shapley values to compute robust feature ranking. This allows us to identify the most predictive variables from the sensor data for each observation. Building from this work, our long-term goal is to construct a reliable and robust model that can forecast optimal operating conditions for real-time optimization of IQ.  Such forecasts can be fed into real-time scheduling protocols to accelerate scientific productivity, and into predictive maintenance routines. We anticipate that the data-driven approaches we explore herein will become standard in automating observatory operations and in improving observatory maintenance by the time CFHT's successor, the Maunakea Spectroscopic Explorer (MSE), is installed in the next decade.
\fi
\iffalse
We leverage state-of-the-art machine learning methods and a decade's worth of archival data from the Canada-France-Hawaii Telescope (CFHT) to predict observatory image quality (IQ) from environmental conditions and observatory operating parameters. Specifically, we develop accurate and interpretable models of the complex dependence between data features and observed IQ for CFHT's wide field camera, MegaCam. Our contributions are several-fold. First, we collect, collate and reprocess several disparate data sets gathered by CFHT scientists. Second, we predict probability distribution functions (PDFs) of IQ, and achieve a mean absolute error of $\sim0.07''$ for the predicted medians. Third, we explore data-driven actuation of the 12 dome ``vents'', installed in 2013-14 to accelerate the flushing of hot air from the dome. We leverage epistemic and aleatoric uncertainties in conjunction with probabilistic generative modeling to identify candidate vent adjustments that are in-distribution (ID) and, for the optimal configuration for each ID sample, we predict the reduction in required observing time to achieve a fixed SNR.  On average, the reduction is $\sim12\%$. Finally, we rank sensor data features by Shapley values to identify the most predictive variables for each observation. Our long-term goal is to construct reliable and real-time models that can forecast optimal observatory operating parameters for optimization of IQ. Such forecasts can then be fed into scheduling protocols and predictive maintenance routines. We anticipate that such approaches will become standard in automating observatory operations and maintenance by the time CFHT's successor, the Maunakea Spectroscopic Explorer (MSE), is installed in the next decade.
\fi
We leverage state-of-the-art machine learning methods and a decade's worth of archival data from CFHT to predict observatory image quality (IQ) from environmental conditions and observatory operating parameters. Specifically, we develop accurate and interpretable models of the complex dependence between data features and observed IQ for CFHT's wide-field camera, MegaCam. Our contributions are several-fold. First, we collect, collate and reprocess several disparate data sets gathered by CFHT scientists. Second, we predict probability distribution functions (PDFs) of IQ and achieve a mean absolute error of $\sim0.07''$ for the predicted medians. Third, we explore the data-driven actuation of the 12 dome ``vents'' installed in 2013-14 to accelerate the flushing of hot air from the dome. We leverage epistemic and aleatoric uncertainties in conjunction with probabilistic generative modeling to identify candidate vent adjustments that are in-distribution (ID); for the optimal configuration for each ID sample, we predict the reduction in required observing time to achieve a fixed SNR.  On average, the reduction is $\sim12\%$. Finally, we rank input features by their Shapley values to identify the most predictive variables for each observation. Our long-term goal is to construct reliable and real-time models that can forecast optimal observatory operating parameters to optimize IQ. We can then feed such forecasts into scheduling protocols and predictive maintenance routines. We anticipate that such approaches will become standard in automating observatory operations and maintenance by the time CFHT's successor, the Maunakea Spectroscopic Explorer, is installed in the next decade.
%We delineate the strengths and shortcomings of such an ML-based approach.  

%To the best of our knowledge, this is the first such work identifying data-driven correlations between telescope operating conditions and delivered seeing values on a per-observation basis.
%We use re-sampling techniques to put infrequently observed ranges of IQ on the same footing as those that occur more frequently.

