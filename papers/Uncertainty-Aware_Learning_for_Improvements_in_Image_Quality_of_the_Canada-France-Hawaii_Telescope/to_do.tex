
\section{To Do}\label{sec:to_do}
\begin{itemize}
    \item Better abstract
    \item More comprehensive introduction
    \item Transfer material from Rene Recine's old report to `Related Work' section
    \item Come up with a cool name for the deep-net; helpful phrases: `tabnet', `stacked', `mixture density network', `imputation', `modified', `seeing', `image quality'.
    \item in `related work', make sure to end with what results we should expect from the deep net. this will be a `quasi-ground truth' agaist which to compare results.
    \item in `results' (`discussion'?), compare insights obtained from DL with those expected from above. highlight gains.
    \item make a list of limitations, both from ML perspective and science/seeing perspective
    \item talk to CFHT folks about `future work'.
    \begin{itemize}
        \item end to end pipeline incorporating imputation
        \item higher resolution in vent configuration
        \item explore pairwise feature interaction like in \cite{shap1, shap2}.
        \item propagation of observational errors
        \item feature importance based on errors as opposed to median outputs
    \end{itemize}
    \item Ask Andy to populate `Acknowledgements'
    \item WHAT IF plots: what happens when we change vent configurations?
    \item check with Andy about public vs private nature of data
    \item Include in Appendix analysis highlighting the importance of having probabilistic predictions per data point. Cf. discussion with YST on 07-10-2020.
\end{itemize}
