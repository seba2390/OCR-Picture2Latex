\section{MDN vs. Catboost} \label{sec:app_mdnvscatboost}

In Figures \ref{fig:mdn_vs_cb_train} and \ref{fig:mdn_vs_cb_test}, we plot the predicted probability distribution functions (PDFs) for three samples each from the training and the test sets, using both the mixture density network (MDN) and the {\sc catboost} ensemble. Note that while GBDT (gradient boosted decision tree) is the specific algorithm, and the term we have repeatedly used in the main text of this paper, the specific implementation of it used here is {\sc catboost}\footnote{\url{https://catboost.ai/}}.These samples are those with those with the minimum, median, and maximum measured MPIQ, in both the training and test sets, respectively. We aim to visualize both the difference in performance between the two models at various levels of measured MPIQ, as well as the full PDF for all model components for both models.

We make several observations. First, all 10 components of the {\sc catboost} ensemble have equal weights, whereas each of the 5 components of the MDN can have variable weights depending on the sample. Second, the diversity in model components, both in \textit{x-} and \textit{y-} values, is much larger for MDN than for {\sc catboost}. Third, uncertainties as quantified by the full-width-at-half-maximum are significantly lower for PDFs predicted by MDN than they are for PDFs predicted by {\sc catboost}. Fourth, for both low and high MPIQ values, the $\beta$ likelihoods in MDN components enable a significantly more flexible representation, and hence a more realistic, asymmetric PDF. Specifically, the low-MPIQ ($\sim0.15''$) and high-MPIQ ($\sim2''$) both have quite distinct MDN components (in terms of the predictions) whereas the mid-MPIQ ($\sim0.5''$) MDN predictions are all quite similar, more akin to the {\sc catboost} ensemble. As $\sim0.5''$ is near the mode of the MPIQ distribution (see the top plots in Figure \ref{fig:hist_preliminaries}),  both models are able to capture the relationship between the input features and the output MPIQ values quite well.  In contrast, the behavior of the two models is quite different at the tails of the distribution. The added flexibility if the MDN results in significantly better predictions than the {\sc catboost} ensemble can accomplish.  This is especially the case at low and high MPIQ values, which are close to OoD.

Finally, in Figures \ref{fig:cb_appendix} and \ref{fig:cbonmdn} we highlight the {\sc catboost} ensemble's deficiencies as a predictor, justifying our decision to use MDN for all  tasks in this paper. In Figure \ref{fig:cb_appendix} we see that for all except 14 test samples, the {\sc catboost} ensemble is unable to hypothesize vent configurations that can improve over the all-open baseline.  And, even in cases where improvement is hypothesized, the improvements are minuscule. In Figure \ref{fig:mdncbminusmeasured_vs_measured_density} we explicitly highlight the large difference in performance between the MDN and the {\sc catboost} models, especially when it comes to the tails of the MPIQ distributions.  This is something  we also showed in Figures  \ref{fig:mdn_vs_cb_train} and \ref{fig:mdn_vs_cb_test}. In Figure \ref{fig:deltaoptimalreference_vs_nominal_iqs_all_cb} we check how the {\sc catboost} model performs on test samples with vent configurations designated as optimal by the MDN, and observe, unsurprisingly, poor performance. Recall that from Figure \ref{fig:deltaoptimalreference_vs_nominal_iqs_all_cb} we already know that only for a very small fraction of samples (14 out of $\sim660$) does the {\sc catboost} ensemble predict any reduction in MPIQ at all.

\begin{figure*}
\begin{subfigure}{0.99\textwidth}
    \centering
    \includegraphics[width=.98\linewidth]{figures/pdf_train_idx=11280.pdf}
    \caption{Predictions for training sample with true MPIQ $\sim0.15''$. \textbf{Left:} PDFs for components of the MDN with non-zero weights, along with the final weighted PDF in black. The first, third, and fourth MDN component have non-trivial weights of 0.70, 0.29 and 0.10, respectively, and hence are the only three weights plotted. \textbf{Right:} PDFs for the 10 equally-weighted components of the {\sc catboost} ensemble, along with the final PDF in black.}
    \label{fig:mdn_vs_cb_train_low}
\end{subfigure}
\newline
\begin{subfigure}{0.99\textwidth}
    \centering
    \includegraphics[width=.98\linewidth]{figures/pdf_train_idx=25378.pdf}
    \caption{Similar to Figure \ref{fig:mdn_vs_cb_train_low}, except for training sample with true MPIQ $\sim0.52''$. \textbf{Left:} The first, third, and fourth MDN component have non-trivial weights of 0.77, 0.13 and 0.09, respectively.   \textbf{Right:} PDFs for the 10 equally-weighted components of the {\sc catboost} ensemble, along with the final PDF in black.}
    \label{fig:mdn_vs_cb_train_med}
\end{subfigure}
\newline
\begin{subfigure}{0.99\textwidth}
    \centering
    \includegraphics[width=.98\linewidth]{figures/pdf_train_idx=16919.pdf}
    \caption{Similar to Figures \ref{fig:mdn_vs_cb_train_low} and \ref{fig:mdn_vs_cb_train_med}, except for training sample with true MPIQ $\sim2.0''$. \textbf{Left:} All five components of the MDN have non-zero weights, per the legend. \textbf{Right:} PDFs for the 10 components of the {\sc catboost} ensemble, each with an equal weight of 0.10, along with the final PDF in black.}
    \label{fig:mdn_vs_cb_train_high}
\end{subfigure}
\caption{Probability distribution functions (PDFs) of model components for both  MDN and {\sc catboost} ensembles, for three samples from the training set (at low, medium, and high MPIQ). In each subfigure, the colored curves represent model components PDFs.  The solid black curve in each subfigure is the weighted, final PDF.  The dashed, gray vertical line indicates the true or measured MPIQ.}
\label{fig:mdn_vs_cb_train}
\end{figure*}

\begin{figure*}
\begin{subfigure}{0.99\textwidth}
    \centering
    \includegraphics[width=.98\linewidth]{figures/pdf_test_idx=0.pdf}
    \caption{Predictions for test sample with true MPIQ $\sim0.15''$. \textbf{Left:} PDFs for components of the MDN with non-zero weights, along with the final  PDF in black. \textbf{Right:} PDFs for the 10 components of the {\sc catboost} ensemble each with equal weight, along with the final  PDF in black.}
    \label{fig:mdn_vs_cb_test_low}
\end{subfigure}
\newline
\begin{subfigure}{0.99\textwidth}
    \centering
    \includegraphics[width=.98\linewidth]{figures/pdf_test_idx=3133.pdf}
    \caption{Similar to Figure \ref{fig:mdn_vs_cb_test_low}, except for training sample with true MPIQ $\sim0.52''$.}
    \label{fig:mdn_vs_cb_test_med}
\end{subfigure}
\newline
\begin{subfigure}{0.99\textwidth}
    \centering
    \includegraphics[width=.98\linewidth]{figures/pdf_test_idx=6266.pdf}
    \caption{Similar to Figures \ref{fig:mdn_vs_cb_test_low} and \ref{fig:mdn_vs_cb_test_med}, except for training sample with true MPIQ $\sim1.98''$.}
    \label{fig:mdn_vs_cb_test_high}
\end{subfigure}
\caption{This figure is similar to Figure \ref{fig:mdn_vs_cb_train}, but now for three samples (at low, medium, and high MPIQ) selected from the test set of $\sim6600$ samples (as opposed to the training set in Figure \ref{fig:mdn_vs_cb_train}).}
\label{fig:mdn_vs_cb_test}
\end{figure*}

%%%%%%%%%%%%%%%%%%%%%%%%%%%%%%%%%%%%%%%%%%%%
%%%%%%%%%%%%%%%%%%%%%%%%%%%%%%%%%%%%%%%%%%%%

%In this appendix we present a few figures that we anticipate will help the reader better understand our implementation of the machine learning techniques we use.  In particular, in Figure~\ref{fig:cfht_mdn_overview} we illustrate our MDN training process.  In Figure~\ref{fig:cfht_rvae_overview} we illustrate the approach taken to identify in-distribution vent configurations amongst all possible $2^{12}$ configurations.  Our predictions are restricted to  only in-distribution vent configurations.  Finally, in Figures~\ref{fig:mdn_ancillary} and~\ref{fig:vae_ancillary} we present some details on how we select our learning rate.

\begin{figure*}
    \centering
    \begin{minipage}{.97\textwidth}
        \begin{subfigure}{0.49\textwidth}
            \centering
            \includegraphics[width=.98\linewidth]{figures/hdsvsiqs_and_histhds_cb.pdf}
            \caption{Hamming distance (HD) of optimal vent configurations with reference to the all-open baselines, plotted against measured MPIQ values. We overlay this against a bar-chart showing the percentage of these configurations.}
            \label{fig:hdsvsiqs_and_histhds_cb}
        \end{subfigure}
        \hfill
        \begin{subfigure}{0.49\textwidth}
            \centering
            \includegraphics[width=.98\linewidth]{figures/deltaoptimalreference_vs_nominal_iqs_all_cb.pdf}
            \caption{Improvement in MPIQ for the optimal vent configurations over the all-open baselines. The {\sc catboost} ensemble predicts only 14 test samples to have alternative vent configurations that lower mean predicted MPIQ.}
            \label{fig:deltaoptimalreference_vs_nominal_iqs_all_cb}
        \end{subfigure}
        \newline
        \begin{subfigure}{0.99\textwidth}
            \centering
            \includegraphics[width=.98\linewidth]{figures/barplot_cb.pdf}
            \caption{Change in MPIQ predicted by the {\sc catboost} ensemble (as opposed to the predictions of the MDN model presented in Figure \ref{fig:common_barplot}) for each possible vent configuration for $100$ randomly selected samples, actuating vent configurations across all ID settings for each. Lower is better.}% The general trend is that the higher the measured IQ the more room there is for improvement.}
            \label{fig:barplot_cb}
        \end{subfigure}
        \caption{We use the {\sc catboost} ensemble to find the optimal vent configurations that would result in the lowest MPIQ values for each of the $\sim660$ test samples with all vents open. Our results indicate that by and large, the {\sc catboost} ensemble in unable to find any superior alternative configurations.}
        \label{fig:cb_appendix}
    \end{minipage}

    \begin{minipage}{.97\textwidth}
        \begin{subfigure}{0.49\textwidth}
            \centering
            \includegraphics[width=.98\linewidth]{figures/mdncbminusmeasured_vs_measured_density.pdf}
            \caption{Kernel density estimate plots for the errors in predictions for the $\sim6600$ test samples, from the MDN in blue and the {\sc catboost} ensemble in red.}
            \label{fig:mdncbminusmeasured_vs_measured_density}
        \end{subfigure}
        \hfill
        \begin{subfigure}{0.49\textwidth}
            \centering
            \includegraphics[width=.98\linewidth]{figures/deltaoptimalreference_vs_nominal_iqs_all_cbpredonmdn.pdf}
            \caption{Improvement in MPIQ predicted by the {\sc catboost} ensemble for the optimal vent configurations found by the MDN.}
            \label{fig:deltaoptimalreference_vs_nominal_iqs_all_cbpredonmdn}
        \end{subfigure}
        \caption{{\sc catboost} vs. MDN. In \textbf{(a)}, %derived from Figures \ref{fig:mdn_one_to_one} and \ref{fig:mdn_one_to_one_cat},
        we highlight that {\sc catboost}'s predictions are highly biased at both low and high MPIQ, whereas the MDN is only slightly biased fpr large MPIQ values. %This is the reason we have chosen to use the MDN model, and not the {\sc catboost} model, for representation learning and to make final predictions.
        In \textbf{(b)}, we use {\sc catboost} to predict MPIQ PDFs for the hypothetical samples with optimal vent configurations identified by the MDN. {\sc catboost} predicts an \textit{increase} in MPIQ, further verifying that it does not extrapolate well beyond densely sampled data regions.}
        \label{fig:cbonmdn}
    \end{minipage}
\end{figure*}   
    
