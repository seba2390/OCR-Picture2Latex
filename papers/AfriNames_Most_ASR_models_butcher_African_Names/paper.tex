\documentclass{INTERSPEECH2023}
\interspeechcameraready 

\usepackage{tabularx}

\title{AfriNames: Most ASR models ``butcher" African Names}

\name{
    Tobi Olatunji\textsuperscript{$\dagger$}\thanks{$\dagger$ Equal contribution.}$^{1,2,*}$, Tejumade Afonja$\textsuperscript{$\dagger$}^{3,4,*}$, 
    Bonaventure F. P. Dossou $^{5,6,7,8,*}$, 
    Atnafu Lambebo Tonja$^{9,*}$,
    Chris Chinenye Emezue$^{6,8,10,*}$, 
    Amina Mardiyyah Rufai$^{11,*}$, 
    Sahib Singh$^{12,*}$
}

\address{
    \small
    $^{1}$ Intron Health Inc
    $^{2}$ Georgia Institute of Technology
    $^{3}$ AI Saturdays Lagos
    $^{4}$ CISPA Helmholtz Center for Information Security
    $^{5}$ McGill University
    $^{6}$ Mila Quebec AI Institute
    $^{7}$ Lelapa AI
    $^{8}$ Lanfrica
    $^{9}$ Instituto Politécnico Nacional
    $^{10}$ Technical University of Munich
    $^{11}$ Idiap Research Institute
    $^{12}$ Ford Motor Company
    $^*$Masakhane NLP
}

\email{
    tobi@intron.io, 
    tejumade.afonja@cispa.de, 
    bonaventure.dossou@mila.quebec, 
    atnafu.lambebo@wsu.edu.et, 
    chris.emezue@gmail.com, 
    amina.rufai@idiap.ch, 
    sahibsingh570@gmail.com
}


\begin{document}

\maketitle
 
\begin{abstract}
% 1000 characters. ASCII characters only. No citations.
Useful conversational agents must accurately capture named entities to minimize error for downstream tasks, for example, asking a voice assistant to play a track from a certain artist, initiating navigation to a specific location, or documenting a laboratory result for a patient. However, where named entities such as ``Ukachukwu" (Igbo), ``Lakicia" (Swahili), or ``Ingabire" (Rwandan) are spoken, automatic speech recognition (ASR) models' performance degrades significantly, propagating errors to downstream systems. We model this problem as a distribution shift and demonstrate that such model bias can be mitigated through multilingual pre-training, intelligent data augmentation strategies to increase the representation of African-named entities, and fine-tuning multilingual ASR models on multiple African accents. The resulting fine-tuned models show an 81.5\% relative WER improvement compared with the baseline on samples with African-named entities.
\end{abstract}

% TODO update tags
\noindent\textbf{Index Terms}: Speech recognition, named entity recognition, distribution shift, accented speech




\section{Introduction and Motivation}

Automatic Speech Recognition (ASR) powers voice assistants, which use machine learning and other artificial intelligence techniques to automatically interpret and understand spoken languages for conversational purposes. With the advent of breakthroughs such as 
% Google's Assistant, 
Amazon's Alexa, and Apple's Siri, 
% Samsung Bixby, Microsoft Cortana, 
etc., voice assistant technology has increasingly become a widespread technology with diverse applications  \cite{siegert2021speaker}.
% which range from businesses, service delivery, to healthcare applications, to education, etc. In recent years, the integration of conversational AI into smart devices has become common. 
However, as these devices gain adoption beyond the demographics of their training data, there is a need for more inclusive and robust AI agents with better spoken language understanding (SLU) and accent recognition capabilities
% However, as these devices gain adoption beyond the demographics of their training data, there is a growing need for more robust and inclusive agents with improved spoken language understanding (SLU) and accent recognition 
\cite{desot2019towards, adelani2021masakhaner}\footnote{https://techxplore.com/news/2022-09-effective-automatic-speech-recognition.html}.


%Automatic Speech Recognition (ASR) technology has grown tremendously in recent years, but there still exists a significant issue with its accuracy when it comes to recognizing African names. Despite the diversity of African names, ASR systems often struggle to transcribe them accurately. This issue is rooted in the limited training data used to develop ASR systems, which frequently lack representation from diverse language communities. For example, \cite{adelani2021masakhaner} addressed this challenge by developing NER datasets, models, and evaluations covering ten widely spoken African languages and found that ASR systems performed poorly when transcribing African names, often replacing them with incorrect transcriptions. The authors attributed this issue to the underrepresentation of African languages in the training data used to develop ASR systems.

Useful conversational agents must accurately capture named entities to minimize errors for downstream tasks. For example, in the command, ``Play Billie Jean by Micheal Jackson", conversational agents need to excel at 3 core tasks: Speech Recognition, Named Entity Recognition, and Entity Linking, to appropriately respond to commands. The ASR component of the system must correctly transcribe the speech, laying a good foundation for Named Entity Recognition (NER) \cite{Nguyen_Yu}, which is, in turn, necessary for effective Entity Linking. 

However, in the command ``Play `Trouble Sleep Yanga Wake Am' by Fela Anikulapo Kuti"\footnote{Fela is one of Africa's most legendary artists} spoken by a Nigerian with a thick Yoruba accent, the phonetic and linguistic variability of the heavily accented speech presents a double dilemma for such systems. Firstly, the heavy accent and tonality can be difficult for the system to recognize, and secondly, the use of out-of-vocabulary words can confuse the model, making it nearly impossible for the system to generate a correct response. Siri responds ``I couldn't find `trouble sleep younger we' by Fela and Kolapo Coochie in your library", effectively ``butchering" \footnote{To "butcher" a name means to mispronounce it, resulting in a significant deviation from the correct pronunciation.} the name, typifying the failures of similar agents on out-of-distribution named entities. More examples in Table \ref{tab:failure_examples}.
% 1) The heavily accented/tonal speech; and 2) the out-of-vocabulary words confuse the model making it nearly impossible for the system to divine a plausible response. 

%Recent research demonstrates the growing popularity of End-to-end (E2E) model approaches over hybrid models \cite{li2022recent}. These approaches also established the benefits of unsupervised or semi-supervised pre-trained models, trained on large corpora, as these models have proven capable of achieving human-level accuracy for high-resource languages and decent performance in downstream fine-tuning tasks such as multilingual and low-resource language applications \cite{conneau2020unsupervised, radford2022robust}


We hypothesize that the underrepresentation (and sometimes complete lack of) African named-entities in their training data may partly explain the model bias and eventual ``butchering" of African names by many voice assistants and conversational agents. %We investigate state-of-the-art (SOTA) ASR models' performance on accented African speech with African and non-African named entities. Furthermore, we design a data augmentation strategy to increase the representation of African-named entities in speech corpora and demonstrate the effectiveness of our strategy through fine-tuning experiments on the augmented data.

%We investigate state-of-the-art (SOTA) ASR models' performance on accented African speech with African-named entities by evaluating samples with African and non-African named entities, we design a data augmentation strategy to increase the representation of African-named entities in the speech corpora. We fine-tune ASR models on the augmented data and report results.

Our contributions are as follows:

%\section{Contribution}
%\begin{enumerate}
%    \item First to do speech NER for African accented English ASR
%\item Handcrafted Templating strategy
%\item Automated NER annotation strategy
%\item Improve African names ASR
%\end{enumerate}


%Problem
%ASR models "butcher" our names
%Demonstration of Problem
%We show results of "un-finetuned" ASR models on AfriNER entities in different datasets. -> AfriSpeech, Cv, SautiDB
%Solution
%AfriSpeech dataset
%Demonstration of solution
%Results of models "finetuned" on AfriSpeech dataset



%\begin{enumerate}
%    \item We are the first to investigate speech named-entity recognition for African accented English ASR. We design an effective strategy to evaluate ASR models on speech datasets with no prior NER annotations and demonstrate failures of existing SOTA and commercial ASR models on samples with African named entities
%    \item We create a novel speech corpus rich in African named-entities using a templating framework to augment existing corpora with native African named entities 
 %   \item We finetune pretrained models on the augmented accented speech dataset and show significant improvement in ASR performance and provide our best models as publicly available pre-trained checkpoints.
    
%\end{enumerate}




%\textbf{Another variant}

\begin{enumerate}
    \item We investigate the performance of state-of-the-art (SOTA) ASR models on African named-entities. To do this, we design an effective strategy to evaluate ASR models on speech datasets with no prior NER annotations. Our study highlights the failure of existing SOTA and commercial ASR models on samples with African named-entities
    \item We develop a data augmentation strategy to increase the representation of African-named entities, creating a novel speech corpus rich in African named-entities, and show that by fine-tuning pre-trained models on the augmented accented data, we significantly improve the ability of pre-trained models to recognize African named entities. We open-source the dataset and fine-tuned models\footnote{https://huggingface.co/datasets/tobiolatunji/afrispeech-200}.
    %\item We present AfriSpeech, a novel speech corpus rich in African named-entities, and show that by finetuning pretrained models on the augmented accented AfriSpeech data, we significantly improve the ability of pretrained models to recognize African entities.
    
\end{enumerate}



%NER research using speech data dates back as far as NER using text data (\cite{Nguyen_Yu}). Undeniably, Numerous research advancements have been made in this area, with the majority of them addressing the issues of NER from speech data's low performance in comparison to text, and the large amounts of annotated data required for training(\cite{Galibert_Rosset_Grouin_Zweigenbaum_Quintard,mdhaffar2022end}). A recent research by \cite{porjazovski2020end} showed work on accented speech for NER. 
%\begin{enumerate}
%    \item Voice assistant use cases and their adoption outside their primary markets
    
%\item Recent advances in ASR

%\item ASR performance on named entities
%\item Common african names
%\end{enumerate}

\begin{table*}[t]
\caption{Model behavior examples on native African named entities}
\centering
\small
\begin{tabularx}{\textwidth}{l|X}
\toprule
Model & Sentence\\
\hline
reference & \textbf{Ifeadigo} has been living at \textbf{Kaduna} with his wife \textbf{Chiamaka Orajimetochukwu} \\
\hline
%\multicolumn{2}{l}{\textbf{Commercial APIs}}\\

azure & \underline{if you're diego}. \\

% gcp & \underline{diego} has been living at his wife \\

aws & \underline{if you did good} has been living at \underline{kaduna} with his wife, \underline{she america or raji mo to} \\
%\arrayrulecolor{gray}\hline
%\multicolumn{2}{l}{\textbf{Monolingual finetuning}}\\

%hubert-large-ls960-ft & \underline{ifia di gun} has been living at \underline{cardonal} with his wife \underline{shia maca orraji mo tu truku} \\

% hubert-xlarge-ls960-ft & \underline{ifia di gun} ha been living at \underline{cardona} with his wife \underline{shia macca or ragi mo tu truco} \\

w2v2-lg-960h-lv60-self & \underline{fia digo} has been living at \underline{cadna} with his wife \underline{shi maca orajimo to truco o} \\

% w2v2-lg-960h & \underline{ifia digoun} not been living at \underline{caduna} with his wife \underline{shea macca or ra gi mou toul tru coul} \\

% w2v2-ft-swbd-300h & \underline{ifia digo} has been living at \underline{cadona} with his wife \underline{shi meca or ragimo too tro qo} \\
%\hline
%\multicolumn{2}{l}{\textbf{Multilingual finetuning}}\\

w22-lg-xlsr-53-en & \underline{ifia digu} has been living at \underline{kaduna} with his wife \underline{shiamaka orajimo tutruku} \\

% w2v2-xls-r-1b-english & \underline{ifia digo} has been living at \underline{kaduna} with his wife \underline{shiamaka orajimu tutuku} \\

% whisper-base & \underline{if you are de-goon}, that's when living at \underline{kaduna}, which is wife, \underline{shia makka, or rajimu, to kuku} \\

% whisper-small & \underline{ifya digo} has been living at \underline{kaduna} with his wife, \underline{shiyamaka orajimu tochuku} \\

% whisper-medium & \underline{ifeia digun} has been living at \underline{kaduna} with his wife, \underline{shiamaka or rajimu, to chuku} \\

whisper-large & \underline{ifeardigun} has been living at \underline{kaduna} with his wife, \underline{shiamaka or rajimu, to chukwu} \\

xlsr-general (Ours) & \underline{ifiadigo} has been living at \underline{kaduna} with his wife \underline{chiamaka orajimotochukwu} \\
Whisper-general (Ours) & \underline{ifeadigo} has been living at \underline{kaduna} with his wife \underline{chiamaka orahjimu tochukwu} \\
% Whisper-all (Ours) & \underline{ifeadigo} has been living at \underline{kaduna} with his wife \underline{chiamaka orajimotochukwu} \\
%\arrayrulecolor{black}
\bottomrule
\end{tabularx}
\label{tab:failure_examples}
\end{table*}

\section{Related work}
Developing ASR systems for low-resource languages remains challenging due to the scarcity of training data and resources. As a result, models trained on high-resource languages, such as English, do not perform well on low-resource languages \cite{lepak2021generalisation}. To address this, researchers have proposed several solutions such as cross-lingual representations where the system learns a shared representation for multiple languages \cite{conneau2017word}, data augmentation techniques \cite{feng2021survey}, and fine-tuning ASR model trained on high-resource languages on low-resource languages \cite{anaby2020not}.
Recent SOTA multilingual ASR models such as Whisper \cite{radford2022robust} -- trained on over 680K hours labeled speech samples, including multilingual speech corpora such as Common Voice \cite{commonvoice} -- have significantly improved the ASR landscape, outperforming their monolingual counterparts such as HuBERT \cite{hsu2021hubert}, wavLM \cite{chen2022wavlm}, and wav2vec2 \cite{baevski2020wav2vec} in various downstream tasks. 
Despite these breakthroughs, both open source and commercial ASR systems still exhibit racial bias \cite{koenecke2020racial}, higher error on accented speech \cite{hinsvark2021accented}, and incorrect transcriptions of named entities. Previous studies have highlighted challenges with named entity recognition (NER) for ASR and have investigated various methods to improve NER performance. For instance, French researchers \cite{galliano2009ester} outlined steps for assessing NER in french transcripts of radio broadcasts, while \cite{xiao2021automatic} evaluated Chinese accent ASR on an automatic speech query service (AVQS), highlighting the severe limitations of such systems for Mandarin users with multiple accents. More recently, \cite{mdhaffar2022end, caubriere2020we} attempted to extract semantic information directly from speech signals using a single end-to-end model that learns ASR and NER tasks together. However, none of this work focuses on named entities in African datasets, which presents a new area of research and its unique challenges.

% This work sheds light on the challenges of developing ASR systems for African laguages with a particular focus on named entity recognition. We identify new methods to improve ASR and NER performance on African languages and overcome the challenges. 


% % The onset of multilingual speech corpora like Common Voice \cite{commonvoice} led to the increased trend towards multilingual ASR and the release of large, robust multilingual pre-trained ASR models such as Whisper (90 languages) \cite{radford2022robust} and Wav2vec-xlsr (10 languages) \cite{Babu2022XLSRSC} along with several variations such as SpeechStew \cite{chan2021speechstew} and \cite{ritchie2022large} trained on multilingual speech corpora. These models achieved SOTA performance on many downstream tasks, outperforming monolingual models like HuBERT \cite{hsu2021hubert}, wavLM \cite{chen2022wavlm}, and wav2vec2 \cite{baevski2020wav2vec}, which were pre-trained or fine-tuned exclusively on monolingual corpora such as Librispeech \cite{panayotov2015librispeech}, LibriVox \cite{kearns2014librivox}, SWBD \cite{godfrey1992switchboard}, or WSJ \cite{paul1992design}. Other efforts achieved high ASR performance in other low resource languages such as Indian \cite{indicwav2vec}, Irish \cite{faste2022wav2vec}, Swedish \cite{al2021self}, and Mandarin, Japanese, Arabic, German \cite{yi2020applying}.


% % \subsection{Advances in accented English ASR}
% The challenge isn't limited to low-resource languages. Based on recent survey \cite{hinsvark2021accented}, the linguistic variability of accents presents hard challenges in accented English ASR systems in both data collection and modeling strategies. Promising approaches include training accent-specific models \cite{vergyri2010automatic, najafian2014unsupervised}, data augmentation such as speed perturbation \cite{fukuda2018data}, model generalization through multi-task learning \cite{jain2018improved, radford2022robust, li2018multi}, domain expansion \cite{ghorbani2019domain, houston2020continual}, pronunciation modification \cite{goronzy2004generating, lehr2014discriminative}, adaptation using auxiliary acoustic features \cite{grace2018occam, li2018multi, zhu2020multi}, accent embeddings \cite{viglino2019end, turan2020achieving}, and adversarial training \cite{sun2018domain, chen2020aipnet}. Our work focuses on domain adaptation to accented speech.  %Accent embeddings and adversarial training are more recent techniques that leverage deep learning to address the issue of accented speech recognition. Significant progress made in addressing the challenges posed by linguistic variability holds promise for improving the performance of ASR systems for a wide range of accents.

% % \subsection{Racial Bias in ASR} 
% % Racial bias is an important problem that needs to be addressed to ensure that ASR systems are fair and accessible to all individuals, regardless of racial or ethnic background. A study \cite{koenecke2020racial} found large racial disparities in the performance of five popular commercial ASR systems -- Amazon, Apple, Google, IBM, and Microsoft -- when transcribing structured interviews conducted with 42 white speakers and 73 black speakers. They found that all five ASR systems exhibited substantial racial disparities, with an average word error rate (WER) of 0.35 for black speakers compared with 0.19 for white speakers. This bias is representative of their underlying training data. Most ASR systems work best for native English speakers but their accuracy plummets drastically with non-native English speakers \cite{hassan2022improvement, prasad-jyothi-2020-accents}. Similarly, performance gaps with accented English have been well demonstrated by \cite{doumbouya2021using, siminyu2021ai4d, babirye2022building, ogayo2022building}, with multiple parallel efforts \cite{gutkin2020developing, dossou2021okwugb, afonja2021learning, kamper2011multi} to create accented English datasets. 

% \subsection{Improving ASR for named entities and Speech NER} 
% Many studies have demonstrated progress and challenges in the field of NER for ASR and explored various methods to improve NER performance. French researchers \cite{galliano2009ester} outlined steps for evaluating NER on radio broadcast transcripts.  \cite{xiao2021automatic} evaluated Chinese accented ASR on an Automatic Voice Query Service (AVQS), highlighting the severe limitation of such systems for multi-accented Mandarin users. They improve the final quality of the AVQS system by pairing an end-to-end Transformer-CTC ASR model with fuzzy logic \cite{xiao2021automatic}. \cite{katerenchuk2014improving} and \cite{rangarajan2006detection} leverage prosodic features, \cite{ramabhadran2004use} integrate document metadata to improve NER. More recently, \cite{mdhaffar2022end, caubriere2020we} attempted to extract semantic information directly from speech signals with a single end-to-end model jointly learning ASR and NER tasks.

\section{Methodology}
\begin{figure}[h]
\includegraphics[width=\columnwidth]{images/AfriNames.drawio_new.pdf} %
\centering
\caption{AfriNames dataset augmentation process.}
\label{fig:method}
\end{figure}

\subsection{African Named-Entity Augmentation Workflow}
\textbf{Western vs African-named entities:} We use the term ``Western named entities" to refer to names that are commonly used in Western cultures and languages, such as Laura and Buenos Aires, and that may not have direct translations in African languages \footnote{Due to the influences of colonization and globalization, many Western names have been adopted in African cultures. Therefore, while these names may not have direct translations in African languages, they can still be used and recognized in African contexts. Our work focus specifically on African named entities that are derived from African languages. }. In contrast, we use the term ``African named entities" to denote names, places, and cultural references that are derived from African languages and cultures, and that may not be commonly used or recognized outside of those contexts.

\textbf{Approach:} 
We address the generalization problem as a domain shift, depicted in Figure \ref{fig:method}. Our initial dataset, denoted as $D_1$, consists of Western audio samples $X^{E_{1}}$ and their corresponding transcripts $Y^{E_{1}}$. We employ a pre-trained named entity recognition (NER) model $R_1$ to extract named entities (NEs) from $Y^{E_{1}}$, resulting in the predominantly Western named entity list $E_1$. To inject African named entities, we mask tokens in randomly selected samples from $Y^{E_{1}}$ that match the entities in $E_1$. This process generates the modified dataset $D_1'$ with modified transcripts $Y'^{E_{1}}$. We then randomly insert tokens from a curated African named-entity list $E_2$ to replace the masked tokens in $Y'^{E_{1}}$, creating an augmented dataset $D_2$ with modified transcripts $Y^{E_{2}}$. These transcripts are sent to African crowd-sourced workers for recording, resulting in a new corpus named $D_2'$ with augmented pairs $\{(X^{E_{2}},Y^{E_{2}})\}$. This novel dataset comprises accented audio samples and augmented transcript pairs, combining distributions from $D_1$ and $D_2$ with Anglo-centric named entities $E_1$ and African named entities $E_2$. Next, we use a specialized NER model $R_2$ to annotate all western and African named entities (called $E_3$) present in $D_2$. Using these NER annotations, we select the subset of $D_2$ with NEs. This NE subset $D_3$ (called AfriNER) contains accented speech $X^{E_{3}}$ and corresponding transcripts $Y^{E_{3}}$ with named entities extracted from both $Y^{E_{1}}$ and $Y^{E_{2}}$. Additionally, using curated African NE list $E_2$, we also filter $Y^{E_{3}}$ to create $D_4$ confirmed to contain African NEs (called AfriVal).


% We address the generalization problem as a domain shift, depicted in Figure \ref{fig:method}. Our initial dataset, denoted as $D_1$, consists of Western audio samples $X^{E_{1}}$ and their corresponding transcripts $Y^{E_{1}}$. We employ a pre-trained named entity recognition (NER) model $R_1$ to extract named entities (NEs) from $Y^{E_{1}}$, resulting in the predominantly Western named entity list $E_1$. To inject African named entities, we mask tokens in randomly selected samples from $Y^{E_{1}}$ that match the entities in $E_1$. This process generates the modified dataset $D_1'$ with modified transcripts $Y'^E_{1}$. We then randomly insert tokens from a curated African named-entity list $E_2$ to replace the masked tokens in $Y'^{E_{1}}$, creating an augmented dataset $D_2$ with modified transcripts $Y^{E_{2}}$. These transcripts are sent to African crowd-sourced workers for recording, resulting in a new corpus named $D_2'$. This novel dataset $D_2'$ comprises accented audio samples and augmented transcript pairs $\{(X^{E_{2}},Y^{E_{2}})\}$ along with the audio and transcri$D_1$ and $D_2$ with Anglo-centric named entities $E_1$ and African named entities $E_2$. Next, we use a specialized NER model $R_2$ to annotate all western and African named entities (called $E_3$) present in $D_2$. Using these NER annotations, we select the subset of $D_2$ with NEs. This NE subset $D_3$ (AfriNER) contains accented speech $X^{E_{3}}$ and corresponding transcripts $Y^{E_{3}}$ with named entities extracted from both $Y^{E_{1}}$ and $Y^{E_{2}}$. Additionally, using curated African NE list $E_2$, we also filter $Y^{E_{3}}$ to create $D_4$ confirmed to contain African NEs (AfriValidated).

% A real-world example of $D^{E_{1}}$ is LibriSpeech \cite{panayotov2015librispeech}, a 1,000-hours speech-text dataset from English-only  audiobooks. The resulting ASR model, such as Wav2vec2 \cite{baevski2020wav2vec}, therefore, generalizes poorly to African named entities (Table \ref{tab:failure_examples}).

% We model the generalization problem as a domain shift and illustrate the workflow of the proposed solution in Figure \ref{fig:method}. $D_1$ is a predominantly western dataset $\{(X^{E_{1}},Y^{E_{1}})\}$ with audio and transcript pairs, originating from a distribution $D^{E_{1}}$ induced by Anglo-centric named-entities $E_1$. Model $M_1$ is randomly initialized and trained on $D_1$, learning the mapping $f: X^{E_{1}} \longrightarrow Y^{E_{1}}$, leading to a pretrained model $M_{1}^{E_{1}}$. 
% Pre-trained NER model $R_1$ extracts named entities from $D_1$ transcripts  resulting in predominantly western named entities list $E_1$. Masking $E_1$ tokens from randomly selected samples in $D_1$ produces $D_1'$. Tokens from curated African named-entity list $E_2$ randomly replace masked tokens in $D_1'$. Augmented subset $D_1'$ + $D_1$ creates $D_2$ transcripts which are sent to African crowd-sourced workers for recording to create $D_2'$, a novel corpus of accented audio and augmented transcript pairs originating from a distribution $D^{E_{1}}$ and $D^{E_{2}}$ induced by African and Anglo-centric named-entities $E_1 + E_2 = E_3$. 
% A Specialized NER model $R_2$ extracts African and western named entities $E_3$ from $D_2$ including any African named entities originally in $D_1$. Accented audio recordings of $D_1'$ prompts and $D_1'$ transcripts are isolated to create $D_3$, the subset of $D_2'$ confirmed to contain African named entities (AfriValidated).



%$D^{E_{2}}$ due to the wide distribution shift.
%$M_1$ is an ASR model trained on a predominantly Western dataset $D_1 = \{(X^{E_{1}},Y^{E_{1}})\}$ with audio and transcript pairs, $X^{E_{1}}$ and $Y^{E_{1}}$, originating from a distribution $D^{E_{1}}$ induced by Anglo-centric named-entities $E_1$. The ASR model then learns the mapping $f: X^{E_{1}} \longrightarrow Y^{E_{1}}$, leading to a pretrained model $M_{1}^{E_{1}}$. 
%On the other end of the spectrum are Afro-centric entities $E_2$ like `Fela' and `Ifeadigo'. 
%Pre-trained NER model $R_1$ extracts named entities from the transcripts of a predominantly western speech corpus $D_1$ resulting in a corpus with masked named entities $D_1'$. African-named entities $E_2$ randomly replace masked tokens for a curated subset of $D_1'$. Augmented subset $D_1'$ + $D_1$ creates $D_2$ transcripts which are sent to African crowd-sourced workers for recording. Specialized NER model $R_2$ extracts African and western named entities from $D_2$ which are further filtered by $E_2$ to create $D_3$ the AfriValidated dataset


%Such a dataset does not have  represented as $E_1x$ in speech signal and $E_1y$ in transcripts, learns a function but predicts over a different distribution $D_2$ with African named-entities.
%\begin{equation}\label{eq1}
%    \theta = P(Y_1|X_1,E_1x) 
%\end{equation}


%Chris version............


%We model the generalization problem as a domain shift where an ASR model $M_1$ trained on a predominantly western dataset $D_1 = \{(x_{i},y_{i})\}_{i=1}^{N}$ made of $N$ pairs, $(x_{i},y_{i})$, where $x_i$ is an audio sample and $y_{i}$ is the corresponding transcript. and named-entities $E_1$ represented as $E_1x$ in speech signal and $E_1y$ in transcripts, learns a function (eq \ref{eq1}) but predicts over a different distribution $D_2$ with African named-entities.
%\begin{equation}\label{eq1}
 %   \theta = P(Y_1|X_1,E_1x) 
%\end{equation}


% chris version.............



% is tested on speech from a different data distribution D2 of accented speech with X2, Y2 pairs where X2 = (x1, x2, ..., xT) and Y2=(y1, y2,..., yT) with African named-entities E2.

\subsection{Datasets}
In this study, we primarily explore the AfriSpeech-200 dataset, a 200.91 hours novel accented English speech corpus rich with African-named entities, curated for clinical and general domain ASR using the augmentation process described above. 67,577 prompts were recorded by 2,463 unique crowdsourced African speakers from 13 Anglophone countries across sub-Saharan Africa and the United States. The average audio duration was 10.7 seconds (Table \ref{tab: dataset stats}).

%We explore two additional datasets: (1) \textbf{SautiDB} \cite{afonja2021sautidb}, a dataset of Nigerian accent recordings with 919 audio samples, a sampling rate of 48kHz each, for a total amount of 59 minutes of recordings; (2) \textbf{Common Voice English Accented Dataset}, a subset of English Common Voice (version 10) \cite{commonvoice} with majority American and European English accents removed.
% and (3) \textbf{Medical Speech}\footnote{\url{https://www.kaggle.com/datasets/paultimothymooney/medical-speech-transcription-and-intent}} which contains 6,661 audio utterances for common medical symptoms like \textit{knee pain} or \textit{headache}, for a total of 8 hours of recordings;

%\begin{table}[b]
%\small
%\centering
%\begin{tabular}{l|l|l|l|l|l}
%\hline
%\textbf{Test Samples} & \#n & \#entities & \multicolumn{3}{c}{Categories}\\
% & & & PER & ORG & LOC  \\
%\hline
%AfriSpeech & 2723 & 2478 & 1347 & 398 & 733\\
%SautiDB & 138 & 92& 71 & 3 & 18 \\
% MedSpeech & 622& 2463 & 9 & 0 & 2 & 7 & 0  \\
%CV-En-Accented & 334 & 170 & 76 & 38 & 56  \\

%\hline
%\end{tabular}
%\caption{Dataset entity category counts.}
%\label{tab:splits}
%\end{table}

% \paragraph{\color{red}- Tables for dataset split and samples with NER Tags} 
\begin{table}[h]
\centering
\small
\caption{AfriSpeech-200 Dataset statistics}
\begin{tabular}{l|l|l|l}
\toprule 
%\multicolumn{4}{c}{\textbf{Dataset Stats}} \\
%\hline 
& Train & Dev & Test\\
\midrule
Duration (hrs) & 173.4  &  8.74 & 18.77  \\
\# General domain clips & 21682  &  1407 & 2723  \\
Unique Speakers & 1466   &  247 & 750 \\
Accents & 71 &  45 & 108 \\
%Average Audio duration & 10.7 seconds  \\
\hline
\multicolumn{4}{c}{\textbf{Named Entities Category Counts}}\\
%\multicolumn{4}{c}{\textbf{NER Category Counts}}\\
\hline
% Entities & 20527  &  1233 & 1869  \\
PER & 11011 &  669 & 1064  \\
ORG & 6322 &  372 & 279 \\
LOC & 3194 &  192 & 526  \\
\bottomrule
%\multicolumn{2}{c}{\textbf{Speaker Gender Ratios - \# Clip \%}}\\
%\hline
%Female & 57.11\%  \\
%Male & 42.41\%  \\
%Other/Unknown & 0.48\% \\
%\hline
%\multicolumn{2}{c}{\textbf{Speaker Age Groups - \# Clips}}\\
%\hline
%$<$18yrs & 1,264 (1.88\%) \\
%19-25 & 36,728 (54.58\%)  \\
%26-40 & 18,366 (27.29\%)  \\
%41-55 & 10,374 (15.42\%) \\
%$>$56yrs & 563 (0.84\%)  \\
%\hline
%\multicolumn{2}{c}{\textbf{Clip  Domain - \# Clips}}\\
%\hline
%Clinical & 41,765 (61.80\%)  \\
%General & 25,812 (38.20\%) \\
%\hline
\end{tabular}
\label{tab: dataset stats}
\end{table}

\subsection{AfroAug: African Named-Entity Augmentation}\label{section:name-aug}
%Neural networks learn concepts from training data. Where transcripts or prompts in training data are predominantly Western (e.g. Common Voice and LibriSpeech \cite{ardila2019common, panayotov2015librispeech}) and African-named entities are sparse, such ASR systems fail at correctly transcribing African names like ``Ogochukwu" (Igbo), ``Malaika" (Swahili), or ``Uwimana" (Rwandan), while excellently transcribing Western names like ``Lauren" and ``Bryan"-- representative of the bias in their training corpora. 

To increase the representation of African named entities, we start with a corpus $D_1$ using large open-source predominantly western corpora: Wikitext-103 \cite{merity2016pointer} and scrape African entertainment and news websites to increase the representation of African content. We augment this dataset using two main strategies. 
We curate a list $E_2$ of approximately 100k African names using a database of 90,000 African names from \cite{anderson2013using}, 965 Nigerian Igbo names from \cite{okagbue2017personal}, and 1,000 African names obtained from freely available textbooks, online baby name websites, oral interviews, published articles, and online forums like Instagram and Twitter; and African cities list from Wikipedia \footnote{https://en.wikipedia.org/wiki/List\_of\_cities\_in\_Africa\_by\_population}. We augment $D_1'$ in three key steps:
\begin{enumerate}
    \item \textbf{Named-Entity Extraction with NER Models:} We leverage off-the-shelf pre-trained NER models \cite{conneau2019unsupervised} and annotate all named-entities in corpus $Y^{E_{1}}$ to extract the list $E_1$, tokens tagged with [PER], [LOC], or [ORG]. We mask these tokens $e_i \in E_1$ for a randomly sampled subset of transcripts. 
    \item \textbf{Template Selection:} We manually review, select and validate 140 of these sentences where the replacement of masked tokens with African named entities sounds natural and retains meaning in context. These curated sentences with masked tokens are selected as final templates.
    \item \textbf{Named Entity Replacement:} We randomly (uniformly) replace all [LOC] tags with African cities from $E_2$, and all [PER] and [ORG] tags with African names from $E_2$. We repeat this process 200 times to create text corpus $Y^{E_{2}}$ consisting of 28,000 novel augmented transcripts combined with transcripts from $Y^{E_{1}}$ (100,000+ sentences). $Y^{E_{2}}$ is recorded by crowd-sourced workers.
    % creating a new dataset $D_2' = \{(X^{E_{2}},Y^{E_{2}})\}$ with distribution $D^{E_{2}}$ induced by African and Anglo-centric named entities $E_{2}$. 
    We sample a subset of users from train/dev/test splits for this work. 
    
    %Several studies have demonstrated the utility of "templates" as an effective way to create richer, more expressive training datasets, especially for Question-Answering and prompt engineering \cite{pawar2016question, brown2020language, yao2022prompt} and named entity recognition \cite{DBLP:conf/tsd/DavodyAKK22}. Inspired by this approach, we expand our dataset by creating templates from sentences selected from curated corpora described
\end{enumerate}

A real-world example of $D^{E_{1}}$ is LibriSpeech \cite{panayotov2015librispeech}, a 1,000-hours speech-text dataset from English-only  audiobooks. The resulting ASR model $M_{1}^{E_{1}}$, such as Wav2vec2 \cite{baevski2020wav2vec}, therefore, generalizes poorly to African named entities ${E_{2}}$ (Table \ref{tab:failure_examples}). The pretrained ASR model  $M_{1}^{E_{1}}$ is thus fine-tuned on the new augmented training dataset $D_2'$,  and learns a new mapping $f: X^{E_{2}} \longrightarrow Y^{E_{2}}$ resulting in a more robust model $M_{1}^{E_{2}}$, adapted to the target distribution $D^{E_{2}}$.

%, is finetuned Model $M_1$, thus needs to update its weights to $M_1'$  where $Y'$ and $X'$ represent augmented samples derived from $X$ and $Y$ respectively by randomly replacing named entities from $E_1$ with entities from $E_2$.

%\begin{equation}\label{eq2}
%    \theta = \theta - \alpha\nabla_{\theta}f(\theta, X', Y')
%\end{equation}

% \begin{equation}\label{eq2}
%     \theta_2 = P(Y'|X',E_2) %= \Sigma_i P(y^2_i|X^2,E^2)
% \end{equation}


%Since $D_2$ contains both Western and African-named entities and our primary task is to evaluate NER on African named-entities, to isolate African named-entities, we extract a subset $D_3$ from $D_2$ where exists any $e_i \in E_2$, essentially, the subset of $D_2$ where African-named entities from $E_2$ exists. We note that this process is not perfect as it could omit samples in $D_2$ with African names not captured in $E_2$. However, it guarantees that we can evaluate a subset of test samples with African-named entities.

%\begin{enumerate}
%    \item Candidate Datasets
%    \begin{itemize}
%        \item Fine-tuned
%        \begin{itemize}
%            \item AfriSpeech
%\item Common Voice*
%\item Appen
%\item SautiDB
%\item librispeech

%        \end{itemize}
%        \item Test
%        \begin{itemize}
%            \item AfriSpeech
%\item SautiDB
%\item EduSTT
%\item Prct
%\item lwazi
%\item Nchlt
%\item Common Voice accented
%\item librispeech*

%        \end{itemize}
%    \end{itemize}
%    \item AfriSpeech Templates
%\item Audio recordings
%\item Samples with NER vs non-NER samples

%\end{enumerate}

\section{Experiments}


\subsection{Benchmarks}\label{section:benchmarks}
We compare SOTA open-source pre-trained ASR models: Whisper \cite{radford2022robust}, Wav2vec2 \cite{baevski2020wav2vec}, XLSR \cite{grosman2021xlsr53-large-english}, Hubert \cite{hsu2021hubert}, and WavLM \cite{chen2022wavlm}, with commercial ASR systems. We refer readers to the respective papers for details on pre-training corpora, model architecture, and hyperparameters. We compare 4 model categories: (1) \textbf{Monolingual Models} pre-trained or fine-tuned exclusively on predominantly western transcripts, western English speech, and western named-entities (2) \textbf{Multilingual Models} pre-trained on transcripts from multiple domains, western and accented speech, but with minimal amounts of African named-entities (3) \textbf{Commercial ASR APIs} (4) \textbf{Ours} finetuned on western and African-named entities paired with audios in accented African English.
% (4) \textbf{Models fine-tuned for robustness} on a combination of transcripts with western and African-named entities paired with audios in accented African English.

% \begin{enumerate}
%     \item Monolingual Models pre-trained or fine-tuned exclusively on predominantly western transcripts, western English speech, and western named-entities
%     \item Multilingual Models pre-trained on transcripts from multiple domains, western and accented speech, but with minimal amounts of African named-entities
%     %\item Models fine-tuned on predominantly western transcripts, accented English speech, and predominantly western named entities
%     \item Commercial ASR APIs
%     \item Models fine-tuned for robustness on a combination of transcripts with western and African-named entities paired with audios in accented African English
% \end{enumerate}

\subsection{Fine-tuning}
We select two best-performing open-source models from section \ref{section:benchmarks} and fine-tune them on an accented speech corpus dense with African and western-named entities to achieve robustness to western and African-named entities. We compare pre-trained model performance with fine-tuned checkpoints. Selected model architectures include:
\begin{itemize}
    \item wav2vec2-large-xlsr-53 \cite{grosman2021xlsr53-large-english}: an encoder-decoder architecture with a CNN-based feature extractor, code book, and transformer-based encoder, 378.9M parameters; learning rate of 1e-4.
    \item whisper-medium \cite{radford2022robust}: a decoder-only multi-task architecture, 789.9M parameters; learning rate of 2.5e-4. (We do not fine-tune whisper-large because of computational resource constraints)
\end{itemize}

For each model, we fine-tuned with FP16 \cite{micikevicius2017mixed}, AdamW \cite{loshchilov2019decoupled}, batch size of 16, for 10 epochs, with a linear learning rate decay to zero after a warmup over the first 10\% of iterations. XLSR was trained on a single Tesla T4 GPU with 16GB GPU memory while Whisper was trained on RTX8000 GPU with 48GB GPU memory. Fine-tuning took 24-48 hrs.


%For selected pre-trained and commercial ASR models $M_1$ to $M_n$, as well as fine-tuned models $M_1'$ to $M_n'$, we evaluate WER on samples containing one or more named entities and present single run results in Table \ref{tab:models_benchmarks}. 

%\begin{enumerate}
%    \item Finetune and test on 
%    \begin{itemize}
%        \item western alone (librispeech, common voice, Appen)
%\item accented alone (Afrispeech, SautiDB)
%\item western + african (Common Voice accented)
%    \end{itemize}
%   \item  Benchmarks
%\item Evaluation
%\item Results
%\end{enumerate}

\begin{table*}[t]
\tiny
\centering
\caption{WER results on Afrispeech test samples. \textbf{All} is mean WER across all test samples. \textbf{No-NER} is mean WER across samples with NO predicted named entities (NEs). \textbf{AfriNER} is mean WER across all sentences WITH predicted NEs. \textbf{AfriVal} is mean WER across AfriValidated samples. \textbf{char-AfriNER} and \textbf{char-AfriVal} are mean CER on AfriNER and AfriVal respectively. \textbf{char-AfriNER} and \textbf{char-AfriVal} concatenates the NEs in the predicted and reference transcripts.}

\begin{tabular}{l|l|l|l|l|l|l|l|l}
\toprule 
Model & Params & Training or Finetuning data & \multicolumn{4}{c|}{WER}  & \multicolumn{2}{c}{CER} \\
  & & & All (\#2364) & No-NER  (\#1029) & AfriNER  (\#971) & AfriVal  (\#229) &  char-AfriNER & char-AfriVal \\ 
\midrule
\multicolumn{9}{l}{Baseline}\\
\hline
wav2vec2-large-960h & 317M & Monolingual   & 0.641  & 0.565  & 0.696  & 0.802 & 0.861 & 0.986  \\
\midrule
\multicolumn{6}{l}{Monolingual Fine-tuning: Open-Source SOTA pre-trained Models} & \multicolumn{3}{l} {0.718 Monolingual Mean WER}\\
\hline
wav2vec2-large-960h-lv60-self & 317M & Monolingual   & 0.533  & 0.458  & 0.584  & 0.683  & 0.808 & 0.978  \\

%hubert-large-ls960-ft & 317M & Monolingual   & 0.557  & 0.494  & 0.607  & 0.613  & - & -  \\
hubert-xlarge-ls960-ft & 317M & Monolingual  & 0.562  & 0.487  & 0.613  & 0.701  & 0.803 & 0.986  \\

wavlm-libri-clean-100h-large & 317M & Monolingual   & 0.631  & 0.562  & 0.680  & 0.769  & 0.864 & 0.984  \\

%w2v2-lg-robust-ft-libri-960h & 317M & Monolingual  & 2478& 2364 & 0.3 & 3024 & 0.3 & 5 & 0.3  \\

%wav2vec2-large-robust-swbd-300h & 317M & Monolingual   & 0.733  & 0.660  & 0.796  & 0.844  & - & -  \\
\midrule
\multicolumn{6}{l}{Multilingual Fine-tuning: Open-Source SOTA pre-trained Models } & \multicolumn{3}{l}{0.506 Multilingual Mean WER}\\
\hline
whisper-large & 1550M & Multilingual  & 0.240  & 0.187  & 0.300  & 0.412 & \textbf{0.565} & 0.855 \\
whisper-medium & 769M & Multilingual   & 0.276  & 0.206  & 0.352  & 0.488  & 0.607 & 0.913  \\
%whisper-small & 244M & Multilingual   & 0.330  & 0.257  & 0.405  & 0.462  & - & -  \\
wav2vec2-large-xlsr-53-english & 317M & Multilingual   & 0.506  & 0.447  & 0.550  & 0.617  & 0.772 & 0.965  \\
% wav2vec2-xls-r-1b-english & 317M & Multilingual    & 0.521  & 0.468  & 0.568  & 0.581  & - & -  \\
% nvidia/stt-en-conformer-ctc-large & 118M &  Multi, 10 & AfriSpeech &3024& 2364 & 0.3 & 3024 & 0.3 & 5 & 0.3  \\

\midrule
\multicolumn{6}{l}{Commercial ASR APIs} & \multicolumn{3}{l}{0.588 Commercial Mean WER}\\
\hline
Azure\cite{azure}  & - & -     & 0.340  & 0.273  & 0.402  & 0.509  & 0.674 & 0.946 \\
GCP\cite{gcp}  & - & -     & 0.534  & 0.464  & 0.603  & 0.700  & 0.827 & 0.991 \\
AWS\cite{aws}  & - & -     & 0.354  & 0.279  & 0.426  & 0.556  & 0.735 & 0.970  \\

% (Azure\footnote{\url{https://speech.microsoft.com/portal/speechtotexttool}}, GCP\footnote{\url{https://cloud.google.com/speech-to-text/}}, and AWS\footnote{\url{https://aws.amazon.com/transcribe/}})

\midrule
%\multicolumn{9}{l}{Western named entities fine-tuned on accented speech}\\
%\hline
%wav2vec2-large-xlsr-53-english  & 317M & Multilingual  &  SDB  & 92& 138 & 0.3 & - & - & 15 & 0.3 & 0 & -  \\
%xlsr-53-english-SautiDB & 317M & SDB  & SDB  & 92 & 138 & 0.5 & 48 & - & 90 & 0.157 & 0 & -  & - & -  \\
%xlsr-53-english-CV & 317M & CV  & CV & 170 & 334 & 0.3 & 190 & - & 144 & 0.253 & 3 & -  & - & -  \\

%\midrule
\multicolumn{6}{l}{AfriSpeech Finetuning (Ours)} & \multicolumn{3}{l}{0.160 AfriSpeech Mean WER} \\
\hline
whisper-medium-AfriSpeech & 769M & Monolingual, AfriSpeech     & \textbf{0.186}  & \textbf{0.172}  & \textbf{0.198}  & \textbf{0.108}  & 0.576 & \textbf{0.704} \\
xlsr-53-english-AfriSpeech & 317M & Monolingual, AfriSpeech     & 0.236  & 0.211  & 0.258  & 0.212  & 0.622 & 0.816  \\

% wav2vec2-large-960h-medspeech & 317M & MedSpeech & Medspeech  &  & 622 & 0.3 & 9 & 0.09 & 0 & 0.3  \\
\bottomrule
\end{tabular}

\label{tab:models_benchmarks}
\end{table*}

% Mean WER per model category: Monolingual 0.681; Multilingual 0.479; Commercial 0.554; Ours 0.141 (80.4\% improvement over baseline)

\subsection{Evaluation}
Word Error Rate (WER) and Character Error Rate (CER) are common metrics for evaluating ASR models. WER measures word errors, CER measures character errors. Lower values are better for both.

\subsubsection{AfriNER: Named-Entity Evaluation}\label{section:ner}
To evaluate ASR performance on named entities (NEs), we need a reliable way to identify samples in $Y_2$ with NEs. Ground truth transcripts $Y_2$ contain $E_1$ and $E_2$ entities, jointly called $E_3$. To extract all samples in $Y_2$ with NEs in $E_3$, we run NER inference on all test samples in $Y_2$ using a specialized performant NER model $R_2$ \footnote{https://huggingface.co/masakhane/afroxlmr-large-ner-masakhaner-1.0\_2.0} from \cite{Adelani2022MasakhaNER2A} that jointly predicts the set of African and western named entities $E_3$. We select test sentences where an entity is detected with confidence (score) greater than 0.8. This seemed to be a reasonable threshold based on ad-hoc analysis. $R_2$ is also able to identify unknown African named entities in $Y_2$ not sourced from $E_2$ (but present in $Y_1$). We denote this subset $Y_3$ (Afri-NER). For each model, we compute WER on corresponding model predictions $Z_3$.


%Since ground truth NER labels do not exist for this novel dataset, we identify samples with named entities  To evaluate model 


%$M_{1}^{E_{2}}$ predicted transcripts ($Z_2$) with NEs, we need a reliable way to identify $z_i \in Z_2$ where NEs exist. ASR errors make it unreliable to exactly match tokens from $E_3$ in $Z_2$. Therefore,  We select test sentences where an entity is detected with confidence (score) greater than 0.8. This seemed to be a reasonable threshold based on ad-hoc manual analysis.  $R_2$ is also able to identify African named entities in $Y_2$ not sourced from $E_2$ (present in $Y_1$). Afri-NER ($Z_3$) is the subset of predicted transcripts $Z_2$ where $R_2$ predicts the presence of one or more named entities with a confidence score above 0.8.

%$D_1$ and $D_2'$, $D_2$, therefore, contains sentences with named entities from both $E_1$ and $E_2$, jointly called $E_3$. To evaluate NER on ASR-predicted transcripts from $D_2'$, we need a reliable way to identify named entities in $D_2$ since $D_2$ has no prior ground truth NER annotations. To achieve this, we run NER inference on all test samples in $D_2$ using a specialized performant NER model $R_2$ from \cite{Adelani2022MasakhaNER2A} that jointly predicts the set of African and western named entities $E_3$ in $D_2$. %This process identifies samples from $D_2$ with Western and African named entities.

\subsubsection{Sentence-level AfriValidation: African Named-Entity Validation}\label{section:afriVal}
Our primary goal is to evaluate $M_{1}^{E_{1}}$s and $M_{1}^{E_{2}}$ on transcripts with ``African" NEs. To isolate samples with African NEs, we extract the subset of $Y_2$ from the test partition with any NEs from $E_2$ to create the AfriVal subset. Because these sentences are known to contain African NEs, they are Afri-Validated, and guarantee we can reliably evaluate ASR models on predicted transcripts with African NEs. For each model, we compute WER on corresponding model predictions. % beyond the subset of $D_2$ augmented transcripts. 

% \footnote{WER is based on string edit distance, and penalizes all differences between the model's output and the reference transcript. The lower the value, the better.}
% \footnote{CER is a common metric whose value indicates the percentage of characters that were incorrectly predicted. The lower the value, the better.}

\subsubsection{Character-level AfriValidation}\label{section:char-afriVal}
Since sentence-level WER is impacted by non-NE tokens, we compute CER on NE tokens by isolating them as follows: 1) We run $R_2$ on model predicted transcript $Z_2$ and $Z_3$ to obtain predicted NE tokens with $>0.8$ confidence score. To mitigate the impact of NER errors from $R_2$, for each ground truth and predicted sentence, we concatenate all NER tokens $e_i \in E_3$ from $Y_3$ and all $z_i \in Z_3$ removing all spaces and compute CER.
%Since ground truth NER labels do not exist for this novel dataset, we identify samples with named entities using the process described in \ref{section:ner}. This subset, with named entities from $E_3$, is called Afri-NER. We evaluate Word Error Rate (WER) on corresponding test set model predictions $Z_3$. Furthermore, we extract the subset of $Y_2$ sentences with named entity tokens from $E_2$. This subset is called Afri-Val and evaluate WER on corresponding model predictions. Furthermore, from Afri-Val, we extract named entity tokens from ground truth and model predictions using the process described in \ref{section:afriVal} and compute Character Error Rate (CER). 
For selected pre-trained and commercial ASR models $M_{1}^{E_{1}}$, as well as fine-tuned models $M_{1}^{E_{2}}$, we evaluate WER and CER on samples containing one or more named entities and present single run results in Table \ref{tab:models_benchmarks}.


%we infer NER labels using a specialized NER model \cite{Adelani2022MasakhaNER2A} as described in section \ref{section:afriVal} above. We select test sentences where an entity is detected with confidence (score) greater than 0.8 and evaluate using Word Error Rate (WER). 

%For ``ground truth" transcripts, we extract the subset of $Y_2$ with any named entity from $E_2$ and evaluate Word Error Rate (WER) on corresponding model predictions. Furthermore, we extract named entities $E_3$ from each $y_i in Y_2$ using $R_2$. For each model $M$ with predicted named entities $Z_3$ identified by $R_2$. To mitigate NER errors from $R_2$, for each sentence, we concatenate all $e_i \in E_3$ and all $z_i in Z_3$ and compute Character Error Rate (CER).




\section{Results and Discussion}
\subsection{African named entities are challenging}
The baseline model in Table \ref{tab:models_benchmarks} demonstrates the dominant trend in our results. WER on all samples (column 4, All) improves by 13.6\% (relative) when samples with named entities are EXCLUDED (column 5, No-NER), worsens by 11.7\% (relative) when samples with named entities (western + African) are isolated (column 6, AfriNER). Performance sinks by 29.7\% (relative) on the subset of Afrivalidated examples (column 7, AfriVal)-- samples with African-named entities from $E2$. This pattern is consistent across all model categories except Ours where we observed a 41.9\% (whisper) and 10.2\% (xlsr) relative WER improvement on AfriVal sentences.

\subsection{Training data bias} 
As shown in Table \ref{tab:models_benchmarks}, multilingual/multitask pre-training outperforms monolingual pre-training/fine-tuning. Multilingual/multitask models \cite{radford2022robust, grosman2021xlsr53-large-english, gulati2020conformer} learn more useful representations, are more linguistically diverse, robust, and generalize better to accented speech when compared with monolingual models fine-tuned on datasets (e.g. Librispeech  \cite{panayotov2015librispeech} and Switchboard \cite{godfrey1992switchboard}) with predominantly western NEs and western accents. After fine-tuning on AfriSpeech with African NEs and accented speech, our best model, whisper-medium improves on the baseline by 81.5\% compared to 16.4\% for the pre-trained model.


%\subsection{NER models and AfriValidation}
% Figure \ref{fig:distribution} shows the distribution of named entities in the AfriSpeech dataset. 
%Although NER models are imperfect, we manually validated several outputs that gave us confidence that the specialized NER model \cite{Adelani2022MasakhaNER2A} was in fact correctly identifying African and Western-named entities. Afrivalidation also guaranteed sentences with African-named entities were isolated. However, there are caveats. The African slave name database \cite{anderson2013using} contained western names like George and John which were initially picked up during Afrivalidation. To mitigate this, we limit Afrivalidation to a set of nearly 2k Nigerian names described in section \ref{section:name-aug}. 
% Several [PER], [LOC], and [ORG] examples can be seen in Appendix Tables \ref{apdx:nigeria_ner1} and \ref{apdx:nigeria_ner2}.


\subsection{Multilingual pretraining is insufficient}  
% Appendix Tables \ref{apdx:nigeria_ner1} and \ref{apdx:nigeria_ner2} show several examples with African-named entities, comparing pre-trained vs finetuned versions of our best model, Whisper-medium. 
Despite extensive pretraining on 680k hours of multilingual data (90 languages), the fine-tuned model outperforms the pre-trained model by 77.9\% (relative). Our results demonstrate that multilingual/multi-task pretraining is inadequate as these SOTA models make several mistakes with African-named entities. Fine-tuning results show that our approach is effective in mitigating bias in these large models. 
% Appendix Table \ref{apdx:nigeria_ner1} shows Afrivalidated named entity examples where the pre-trained whisper had a WER $<$ 0.20 (low error) along with the improved transcripts after fine-tuning.  Appendix Table \ref{apdx:nigeria_ner2} shows high error (WER $>$ 0.8) examples along with improved transcripts by the fine-tuned whisper model.

%\subsection{Accented speech on Western named entities} 
%Table \ref{tab:models_benchmarks} shows WER on transcripts with predominantly western named-entities for models finetuned on accented speech. Because these models haven exposed to multiple English accents (including African accents), we expect lower WER on the subset of samples with named entities. %These models show an X\% improvement over the baseline.

%\subsection{Speech characteristics} 
%Maybe look at Spectrograms or MFCCs of samples with western vs African-named entities

\subsection{Character-Level analysis} 
When named entities are isolated as described in Section \ref{section:afriVal}, we observe that our fine-tuned whisper-medium model worsens by 1.9\% (relative) in comparison to the pre-trained whisper-large model (column 8, char-AfriNER). This may be due to the significantly higher number of parameters in whisper-large generalizing better to certain named entities.
However, when evaluated on the Afrivalidated dataset (column 9, char-AfriVal), our fine-tuned whisper-medium model outperforms both pre-trained whisper-large and medium models (relative gain of 17.7\%, and 22.9\% respectively). These results further support our claim that the presence of African-named entities is crucial for achieving better performance in ASR models. 

\subsection{Use of language models} 
Table \ref{tab:failure_examples} shows some of the difficulties with commercial APIs where a language model (LM) is likely used to rescore the raw ASR transcript. This is especially destructive for African-named entities. Because these named entities (e.g.``Ifeadigo") are missing from LM training data, where the probability of sequences with African NEs is effectively zero, and such transcripts are downranked by the algorithm in favor of more likely tokens like ``Diego" as seen in the example in Table \ref{tab:failure_examples}. Prediction score thresholds may also be in use under the hood in these commercial systems, limiting the ASR output where confidence is low resulting in truncated output as seen in Table \ref{tab:failure_examples}. 

\section{Conclusion}
Automatic speech recognition (ASR) for African-named entities is a challenging task for most state-of-the-art (SOTA) ASR models including those trained with multilingual data and multitask objectives. We demonstrate that this bias can be mitigated by fine-tuning these models on accented speech corpora rich in African-named entities, shifting the distribution for robustness in the African context.
% \section{Limitation and Future work}

% \textbf{Limitation and Future work:}
% Table \ref{tab:failure_examples} provides examples where the pre-trained model transcribed some African NEs even without explicit fine-tuning, as well as cases where African NEs are not well transcribed by the fine-tuned model. While this approach works well for African NEs, empirical results are needed to generalize this approach to other races or domains.
% %\section{Limitations \& Future work}



% \subsection{Tables}

% An example of a table is shown in Table~\ref{tab:example}. The caption text must be above the table. Tables must be legible when printed in monochrome on DIN A4 paper; a minimum font size of 8 points is recommended.



%\section{Ethical considerations}


%\subsubsection*{Author Contributions}
%If you'd like to, you may include  a section for author contributions as is done in many journals. This is optional and at the discretion of the authors.

%\subsubsection*{Acknowledgments}
%Use unnumbered third level headings for the acknowledgments. All acknowledgments, including those to funding agencies, go at the end of the paper.


% \chapter{Supplementary Material}
\label{appendix}

In this appendix, we present supplementary material for the techniques and
experiments presented in the main text.

\section{Baseline Results and Analysis for Informed Sampler}
\label{appendix:chap3}

Here, we give an in-depth
performance analysis of the various samplers and the effect of their
hyperparameters. We choose hyperparameters with the lowest PSRF value
after $10k$ iterations, for each sampler individually. If the
differences between PSRF are not significantly different among
multiple values, we choose the one that has the highest acceptance
rate.

\subsection{Experiment: Estimating Camera Extrinsics}
\label{appendix:chap3:room}

\subsubsection{Parameter Selection}
\paragraph{Metropolis Hastings (\MH)}

Figure~\ref{fig:exp1_MH} shows the median acceptance rates and PSRF
values corresponding to various proposal standard deviations of plain
\MH~sampling. Mixing gets better and the acceptance rate gets worse as
the standard deviation increases. The value $0.3$ is selected standard
deviation for this sampler.

\paragraph{Metropolis Hastings Within Gibbs (\MHWG)}

As mentioned in Section~\ref{sec:room}, the \MHWG~sampler with one-dimensional
updates did not converge for any value of proposal standard deviation.
This problem has high correlation of the camera parameters and is of
multi-modal nature, which this sampler has problems with.

\paragraph{Parallel Tempering (\PT)}

For \PT~sampling, we took the best performing \MH~sampler and used
different temperature chains to improve the mixing of the
sampler. Figure~\ref{fig:exp1_PT} shows the results corresponding to
different combination of temperature levels. The sampler with
temperature levels of $[1,3,27]$ performed best in terms of both
mixing and acceptance rate.

\paragraph{Effect of Mixture Coefficient in Informed Sampling (\MIXLMH)}

Figure~\ref{fig:exp1_alpha} shows the effect of mixture
coefficient ($\alpha$) on the informed sampling
\MIXLMH. Since there is no significant different in PSRF values for
$0 \le \alpha \le 0.7$, we chose $0.7$ due to its high acceptance
rate.


% \end{multicols}

\begin{figure}[h]
\centering
  \subfigure[MH]{%
    \includegraphics[width=.48\textwidth]{figures/supplementary/camPose_MH.pdf} \label{fig:exp1_MH}
  }
  \subfigure[PT]{%
    \includegraphics[width=.48\textwidth]{figures/supplementary/camPose_PT.pdf} \label{fig:exp1_PT}
  }
\\
  \subfigure[INF-MH]{%
    \includegraphics[width=.48\textwidth]{figures/supplementary/camPose_alpha.pdf} \label{fig:exp1_alpha}
  }
  \mycaption{Results of the `Estimating Camera Extrinsics' experiment}{PRSFs and Acceptance rates corresponding to (a) various standard deviations of \MH, (b) various temperature level combinations of \PT sampling and (c) various mixture coefficients of \MIXLMH sampling.}
\end{figure}



\begin{figure}[!t]
\centering
  \subfigure[\MH]{%
    \includegraphics[width=.48\textwidth]{figures/supplementary/occlusionExp_MH.pdf} \label{fig:exp2_MH}
  }
  \subfigure[\BMHWG]{%
    \includegraphics[width=.48\textwidth]{figures/supplementary/occlusionExp_BMHWG.pdf} \label{fig:exp2_BMHWG}
  }
\\
  \subfigure[\MHWG]{%
    \includegraphics[width=.48\textwidth]{figures/supplementary/occlusionExp_MHWG.pdf} \label{fig:exp2_MHWG}
  }
  \subfigure[\PT]{%
    \includegraphics[width=.48\textwidth]{figures/supplementary/occlusionExp_PT.pdf} \label{fig:exp2_PT}
  }
\\
  \subfigure[\INFBMHWG]{%
    \includegraphics[width=.5\textwidth]{figures/supplementary/occlusionExp_alpha.pdf} \label{fig:exp2_alpha}
  }
  \mycaption{Results of the `Occluding Tiles' experiment}{PRSF and
    Acceptance rates corresponding to various standard deviations of
    (a) \MH, (b) \BMHWG, (c) \MHWG, (d) various temperature level
    combinations of \PT~sampling and; (e) various mixture coefficients
    of our informed \INFBMHWG sampling.}
\end{figure}

%\onecolumn\newpage\twocolumn
\subsection{Experiment: Occluding Tiles}
\label{appendix:chap3:tiles}

\subsubsection{Parameter Selection}

\paragraph{Metropolis Hastings (\MH)}

Figure~\ref{fig:exp2_MH} shows the results of
\MH~sampling. Results show the poor convergence for all proposal
standard deviations and rapid decrease of AR with increasing standard
deviation. This is due to the high-dimensional nature of
the problem. We selected a standard deviation of $1.1$.

\paragraph{Blocked Metropolis Hastings Within Gibbs (\BMHWG)}

The results of \BMHWG are shown in Figure~\ref{fig:exp2_BMHWG}. In
this sampler we update only one block of tile variables (of dimension
four) in each sampling step. Results show much better performance
compared to plain \MH. The optimal proposal standard deviation for
this sampler is $0.7$.

\paragraph{Metropolis Hastings Within Gibbs (\MHWG)}

Figure~\ref{fig:exp2_MHWG} shows the result of \MHWG sampling. This
sampler is better than \BMHWG and converges much more quickly. Here
a standard deviation of $0.9$ is found to be best.

\paragraph{Parallel Tempering (\PT)}

Figure~\ref{fig:exp2_PT} shows the results of \PT sampling with various
temperature combinations. Results show no improvement in AR from plain
\MH sampling and again $[1,3,27]$ temperature levels are found to be optimal.

\paragraph{Effect of Mixture Coefficient in Informed Sampling (\INFBMHWG)}

Figure~\ref{fig:exp2_alpha} shows the effect of mixture
coefficient ($\alpha$) on the blocked informed sampling
\INFBMHWG. Since there is no significant different in PSRF values for
$0 \le \alpha \le 0.8$, we chose $0.8$ due to its high acceptance
rate.



\subsection{Experiment: Estimating Body Shape}
\label{appendix:chap3:body}

\subsubsection{Parameter Selection}
\paragraph{Metropolis Hastings (\MH)}

Figure~\ref{fig:exp3_MH} shows the result of \MH~sampling with various
proposal standard deviations. The value of $0.1$ is found to be
best.

\paragraph{Metropolis Hastings Within Gibbs (\MHWG)}

For \MHWG sampling we select $0.3$ proposal standard
deviation. Results are shown in Fig.~\ref{fig:exp3_MHWG}.


\paragraph{Parallel Tempering (\PT)}

As before, results in Fig.~\ref{fig:exp3_PT}, the temperature levels
were selected to be $[1,3,27]$ due its slightly higher AR.

\paragraph{Effect of Mixture Coefficient in Informed Sampling (\MIXLMH)}

Figure~\ref{fig:exp3_alpha} shows the effect of $\alpha$ on PSRF and
AR. Since there is no significant differences in PSRF values for $0 \le
\alpha \le 0.8$, we choose $0.8$.


\begin{figure}[t]
\centering
  \subfigure[\MH]{%
    \includegraphics[width=.48\textwidth]{figures/supplementary/bodyShape_MH.pdf} \label{fig:exp3_MH}
  }
  \subfigure[\MHWG]{%
    \includegraphics[width=.48\textwidth]{figures/supplementary/bodyShape_MHWG.pdf} \label{fig:exp3_MHWG}
  }
\\
  \subfigure[\PT]{%
    \includegraphics[width=.48\textwidth]{figures/supplementary/bodyShape_PT.pdf} \label{fig:exp3_PT}
  }
  \subfigure[\MIXLMH]{%
    \includegraphics[width=.48\textwidth]{figures/supplementary/bodyShape_alpha.pdf} \label{fig:exp3_alpha}
  }
\\
  \mycaption{Results of the `Body Shape Estimation' experiment}{PRSFs and
    Acceptance rates corresponding to various standard deviations of
    (a) \MH, (b) \MHWG; (c) various temperature level combinations
    of \PT sampling and; (d) various mixture coefficients of the
    informed \MIXLMH sampling.}
\end{figure}


\subsection{Results Overview}
Figure~\ref{fig:exp_summary} shows the summary results of the all the three
experimental studies related to informed sampler.
\begin{figure*}[h!]
\centering
  \subfigure[Results for: Estimating Camera Extrinsics]{%
    \includegraphics[width=0.9\textwidth]{figures/supplementary/camPose_ALL.pdf} \label{fig:exp1_all}
  }
  \subfigure[Results for: Occluding Tiles]{%
    \includegraphics[width=0.9\textwidth]{figures/supplementary/occlusionExp_ALL.pdf} \label{fig:exp2_all}
  }
  \subfigure[Results for: Estimating Body Shape]{%
    \includegraphics[width=0.9\textwidth]{figures/supplementary/bodyShape_ALL.pdf} \label{fig:exp3_all}
  }
  \label{fig:exp_summary}
  \mycaption{Summary of the statistics for the three experiments}{Shown are
    for several baseline methods and the informed samplers the
    acceptance rates (left), PSRFs (middle), and RMSE values
    (right). All results are median results over multiple test
    examples.}
\end{figure*}

\subsection{Additional Qualitative Results}

\subsubsection{Occluding Tiles}
In Figure~\ref{fig:exp2_visual_more} more qualitative results of the
occluding tiles experiment are shown. The informed sampling approach
(\INFBMHWG) is better than the best baseline (\MHWG). This still is a
very challenging problem since the parameters for occluded tiles are
flat over a large region. Some of the posterior variance of the
occluded tiles is already captured by the informed sampler.

\begin{figure*}[h!]
\begin{center}
\centerline{\includegraphics[width=0.95\textwidth]{figures/supplementary/occlusionExp_Visual.pdf}}
\mycaption{Additional qualitative results of the occluding tiles experiment}
  {From left to right: (a)
  Given image, (b) Ground truth tiles, (c) OpenCV heuristic and most probable estimates
  from 5000 samples obtained by (d) MHWG sampler (best baseline) and
  (e) our INF-BMHWG sampler. (f) Posterior expectation of the tiles
  boundaries obtained by INF-BMHWG sampling (First 2000 samples are
  discarded as burn-in).}
\label{fig:exp2_visual_more}
\end{center}
\end{figure*}

\subsubsection{Body Shape}
Figure~\ref{fig:exp3_bodyMeshes} shows some more results of 3D mesh
reconstruction using posterior samples obtained by our informed
sampling \MIXLMH.

\begin{figure*}[t]
\begin{center}
\centerline{\includegraphics[width=0.75\textwidth]{figures/supplementary/bodyMeshResults.pdf}}
\mycaption{Qualitative results for the body shape experiment}
  {Shown is the 3D mesh reconstruction results with first 1000 samples obtained
  using the \MIXLMH informed sampling method. (blue indicates small
  values and red indicates high values)}
\label{fig:exp3_bodyMeshes}
\end{center}
\end{figure*}

\clearpage



\section{Additional Results on the Face Problem with CMP}

Figure~\ref{fig:shading-qualitative-multiple-subjects-supp} shows inference results for reflectance maps, normal maps and lights for randomly chosen test images, and Fig.~\ref{fig:shading-qualitative-same-subject-supp} shows reflectance estimation results on multiple images of the same subject produced under different illumination conditions. CMP is able to produce estimates that are closer to the groundtruth across different subjects and illumination conditions.

\begin{figure*}[h]
  \begin{center}
  \centerline{\includegraphics[width=1.0\columnwidth]{figures/face_cmp_visual_results_supp.pdf}}
  \vspace{-1.2cm}
  \end{center}
	\mycaption{A visual comparison of inference results}{(a)~Observed images. (b)~Inferred reflectance maps. \textit{GT} is the photometric stereo groundtruth, \textit{BU} is the Biswas \etal (2009) reflectance estimate and \textit{Forest} is the consensus prediction. (c)~The variance of the inferred reflectance estimate produced by \MTD (normalized across rows).(d)~Visualization of inferred light directions. (e)~Inferred normal maps.}
	\label{fig:shading-qualitative-multiple-subjects-supp}
\end{figure*}


\begin{figure*}[h]
	\centering
	\setlength\fboxsep{0.2mm}
	\setlength\fboxrule{0pt}
	\begin{tikzpicture}

		\matrix at (0, 0) [matrix of nodes, nodes={anchor=east}, column sep=-0.05cm, row sep=-0.2cm]
		{
			\fbox{\includegraphics[width=1cm]{figures/sample_3_4_X.png}} &
			\fbox{\includegraphics[width=1cm]{figures/sample_3_4_GT.png}} &
			\fbox{\includegraphics[width=1cm]{figures/sample_3_4_BISWAS.png}}  &
			\fbox{\includegraphics[width=1cm]{figures/sample_3_4_VMP.png}}  &
			\fbox{\includegraphics[width=1cm]{figures/sample_3_4_FOREST.png}}  &
			\fbox{\includegraphics[width=1cm]{figures/sample_3_4_CMP.png}}  &
			\fbox{\includegraphics[width=1cm]{figures/sample_3_4_CMPVAR.png}}
			 \\

			\fbox{\includegraphics[width=1cm]{figures/sample_3_5_X.png}} &
			\fbox{\includegraphics[width=1cm]{figures/sample_3_5_GT.png}} &
			\fbox{\includegraphics[width=1cm]{figures/sample_3_5_BISWAS.png}}  &
			\fbox{\includegraphics[width=1cm]{figures/sample_3_5_VMP.png}}  &
			\fbox{\includegraphics[width=1cm]{figures/sample_3_5_FOREST.png}}  &
			\fbox{\includegraphics[width=1cm]{figures/sample_3_5_CMP.png}}  &
			\fbox{\includegraphics[width=1cm]{figures/sample_3_5_CMPVAR.png}}
			 \\

			\fbox{\includegraphics[width=1cm]{figures/sample_3_6_X.png}} &
			\fbox{\includegraphics[width=1cm]{figures/sample_3_6_GT.png}} &
			\fbox{\includegraphics[width=1cm]{figures/sample_3_6_BISWAS.png}}  &
			\fbox{\includegraphics[width=1cm]{figures/sample_3_6_VMP.png}}  &
			\fbox{\includegraphics[width=1cm]{figures/sample_3_6_FOREST.png}}  &
			\fbox{\includegraphics[width=1cm]{figures/sample_3_6_CMP.png}}  &
			\fbox{\includegraphics[width=1cm]{figures/sample_3_6_CMPVAR.png}}
			 \\
	     };

       \node at (-3.85, -2.0) {\small Observed};
       \node at (-2.55, -2.0) {\small `GT'};
       \node at (-1.27, -2.0) {\small BU};
       \node at (0.0, -2.0) {\small MP};
       \node at (1.27, -2.0) {\small Forest};
       \node at (2.55, -2.0) {\small \textbf{CMP}};
       \node at (3.85, -2.0) {\small Variance};

	\end{tikzpicture}
	\mycaption{Robustness to varying illumination}{Reflectance estimation on a subject images with varying illumination. Left to right: observed image, photometric stereo estimate (GT)
  which is used as a proxy for groundtruth, bottom-up estimate of \cite{Biswas2009}, VMP result, consensus forest estimate, CMP mean, and CMP variance.}
	\label{fig:shading-qualitative-same-subject-supp}
\end{figure*}

\clearpage

\section{Additional Material for Learning Sparse High Dimensional Filters}
\label{sec:appendix-bnn}

This part of supplementary material contains a more detailed overview of the permutohedral
lattice convolution in Section~\ref{sec:permconv}, more experiments in
Section~\ref{sec:addexps} and additional results with protocols for
the experiments presented in Chapter~\ref{chap:bnn} in Section~\ref{sec:addresults}.

\vspace{-0.2cm}
\subsection{General Permutohedral Convolutions}
\label{sec:permconv}

A core technical contribution of this work is the generalization of the Gaussian permutohedral lattice
convolution proposed in~\cite{adams2010fast} to the full non-separable case with the
ability to perform back-propagation. Although, conceptually, there are minor
differences between Gaussian and general parameterized filters, there are non-trivial practical
differences in terms of the algorithmic implementation. The Gauss filters belong to
the separable class and can thus be decomposed into multiple
sequential one dimensional convolutions. We are interested in the general filter
convolutions, which can not be decomposed. Thus, performing a general permutohedral
convolution at a lattice point requires the computation of the inner product with the
neighboring elements in all the directions in the high-dimensional space.

Here, we give more details of the implementation differences of separable
and non-separable filters. In the following, we will explain the scalar case first.
Recall, that the forward pass of general permutohedral convolution
involves 3 steps: \textit{splatting}, \textit{convolving} and \textit{slicing}.
We follow the same splatting and slicing strategies as in~\cite{adams2010fast}
since these operations do not depend on the filter kernel. The main difference
between our work and the existing implementation of~\cite{adams2010fast} is
the way that the convolution operation is executed. This proceeds by constructing
a \emph{blur neighbor} matrix $K$ that stores for every lattice point all
values of the lattice neighbors that are needed to compute the filter output.

\begin{figure}[t!]
  \centering
    \includegraphics[width=0.6\columnwidth]{figures/supplementary/lattice_construction}
  \mycaption{Illustration of 1D permutohedral lattice construction}
  {A $4\times 4$ $(x,y)$ grid lattice is projected onto the plane defined by the normal
  vector $(1,1)^{\top}$. This grid has $s+1=4$ and $d=2$ $(s+1)^{d}=4^2=16$ elements.
  In the projection, all points of the same color are projected onto the same points in the plane.
  The number of elements of the projected lattice is $t=(s+1)^d-s^d=4^2-3^2=7$, that is
  the $(4\times 4)$ grid minus the size of lattice that is $1$ smaller at each size, in this
  case a $(3\times 3)$ lattice (the upper right $(3\times 3)$ elements).
  }
\label{fig:latticeconstruction}
\end{figure}

The blur neighbor matrix is constructed by traversing through all the populated
lattice points and their neighboring elements.
% For efficiency, we do this matrix construction recursively with shared computations
% since $n^{th}$ neighbourhood elements are $1^{st}$ neighborhood elements of $n-1^{th}$ neighbourhood elements. \pg{do not understand}
This is done recursively to share computations. For any lattice point, the neighbors that are
$n$ hops away are the direct neighbors of the points that are $n-1$ hops away.
The size of a $d$ dimensional spatial filter with width $s+1$ is $(s+1)^{d}$ (\eg, a
$3\times 3$ filter, $s=2$ in $d=2$ has $3^2=9$ elements) and this size grows
exponentially in the number of dimensions $d$. The permutohedral lattice is constructed by
projecting a regular grid onto the plane spanned by the $d$ dimensional normal vector ${(1,\ldots,1)}^{\top}$. See
Fig.~\ref{fig:latticeconstruction} for an illustration of the 1D lattice construction.
Many corners of a grid filter are projected onto the same point, in total $t = {(s+1)}^{d} -
s^{d}$ elements remain in the permutohedral filter with $s$ neighborhood in $d-1$ dimensions.
If the lattice has $m$ populated elements, the
matrix $K$ has size $t\times m$. Note that, since the input signal is typically
sparse, only a few lattice corners are being populated in the \textit{slicing} step.
We use a hash-table to keep track of these points and traverse only through
the populated lattice points for this neighborhood matrix construction.

Once the blur neighbor matrix $K$ is constructed, we can perform the convolution
by the matrix vector multiplication
\begin{equation}
\ell' = BK,
\label{eq:conv}
\end{equation}
where $B$ is the $1 \times t$ filter kernel (whose values we will learn) and $\ell'\in\mathbb{R}^{1\times m}$
is the result of the filtering at the $m$ lattice points. In practice, we found that the
matrix $K$ is sometimes too large to fit into GPU memory and we divided the matrix $K$
into smaller pieces to compute Eq.~\ref{eq:conv} sequentially.

In the general multi-dimensional case, the signal $\ell$ is of $c$ dimensions. Then
the kernel $B$ is of size $c \times t$ and $K$ stores the $c$ dimensional vectors
accordingly. When the input and output points are different, we slice only the
input points and splat only at the output points.


\subsection{Additional Experiments}
\label{sec:addexps}
In this section, we discuss more use-cases for the learned bilateral filters, one
use-case of BNNs and two single filter applications for image and 3D mesh denoising.

\subsubsection{Recognition of subsampled MNIST}\label{sec:app_mnist}

One of the strengths of the proposed filter convolution is that it does not
require the input to lie on a regular grid. The only requirement is to define a distance
between features of the input signal.
We highlight this feature with the following experiment using the
classical MNIST ten class classification problem~\cite{lecun1998mnist}. We sample a
sparse set of $N$ points $(x,y)\in [0,1]\times [0,1]$
uniformly at random in the input image, use their interpolated values
as signal and the \emph{continuous} $(x,y)$ positions as features. This mimics
sub-sampling of a high-dimensional signal. To compare against a spatial convolution,
we interpolate the sparse set of values at the grid positions.

We take a reference implementation of LeNet~\cite{lecun1998gradient} that
is part of the Caffe project~\cite{jia2014caffe} and compare it
against the same architecture but replacing the first convolutional
layer with a bilateral convolution layer (BCL). The filter size
and numbers are adjusted to get a comparable number of parameters
($5\times 5$ for LeNet, $2$-neighborhood for BCL).

The results are shown in Table~\ref{tab:all-results}. We see that training
on the original MNIST data (column Original, LeNet vs. BNN) leads to a slight
decrease in performance of the BNN (99.03\%) compared to LeNet
(99.19\%). The BNN can be trained and evaluated on sparse
signals, and we resample the image as described above for $N=$ 100\%, 60\% and
20\% of the total number of pixels. The methods are also evaluated
on test images that are subsampled in the same way. Note that we can
train and test with different subsampling rates. We introduce an additional
bilinear interpolation layer for the LeNet architecture to train on the same
data. In essence, both models perform a spatial interpolation and thus we
expect them to yield a similar classification accuracy. Once the data is of
higher dimensions, the permutohedral convolution will be faster due to hashing
the sparse input points, as well as less memory demanding in comparison to
naive application of a spatial convolution with interpolated values.

\begin{table}[t]
  \begin{center}
    \footnotesize
    \centering
    \begin{tabular}[t]{lllll}
      \toprule
              &     & \multicolumn{3}{c}{Test Subsampling} \\
       Method  & Original & 100\% & 60\% & 20\%\\
      \midrule
       LeNet &  \textbf{0.9919} & 0.9660 & 0.9348 & \textbf{0.6434} \\
       BNN &  0.9903 & \textbf{0.9844} & \textbf{0.9534} & 0.5767 \\
      \hline
       LeNet 100\% & 0.9856 & 0.9809 & 0.9678 & \textbf{0.7386} \\
       BNN 100\% & \textbf{0.9900} & \textbf{0.9863} & \textbf{0.9699} & 0.6910 \\
      \hline
       LeNet 60\% & 0.9848 & 0.9821 & 0.9740 & 0.8151 \\
       BNN 60\% & \textbf{0.9885} & \textbf{0.9864} & \textbf{0.9771} & \textbf{0.8214}\\
      \hline
       LeNet 20\% & \textbf{0.9763} & \textbf{0.9754} & 0.9695 & 0.8928 \\
       BNN 20\% & 0.9728 & 0.9735 & \textbf{0.9701} & \textbf{0.9042}\\
      \bottomrule
    \end{tabular}
  \end{center}
\vspace{-.2cm}
\caption{Classification accuracy on MNIST. We compare the
    LeNet~\cite{lecun1998gradient} implementation that is part of
    Caffe~\cite{jia2014caffe} to the network with the first layer
    replaced by a bilateral convolution layer (BCL). Both are trained
    on the original image resolution (first two rows). Three more BNN
    and CNN models are trained with randomly subsampled images (100\%,
    60\% and 20\% of the pixels). An additional bilinear interpolation
    layer samples the input signal on a spatial grid for the CNN model.
  }
  \label{tab:all-results}
\vspace{-.5cm}
\end{table}

\subsubsection{Image Denoising}

The main application that inspired the development of the bilateral
filtering operation is image denoising~\cite{aurich1995non}, there
using a single Gaussian kernel. Our development allows to learn this
kernel function from data and we explore how to improve using a \emph{single}
but more general bilateral filter.

We use the Berkeley segmentation dataset
(BSDS500)~\cite{arbelaezi2011bsds500} as a test bed. The color
images in the dataset are converted to gray-scale,
and corrupted with Gaussian noise with a standard deviation of
$\frac {25} {255}$.

We compare the performance of four different filter models on a
denoising task.
The first baseline model (`Spatial' in Table \ref{tab:denoising}, $25$
weights) uses a single spatial filter with a kernel size of
$5$ and predicts the scalar gray-scale value at the center pixel. The next model
(`Gauss Bilateral') applies a bilateral \emph{Gaussian}
filter to the noisy input, using position and intensity features $\f=(x,y,v)^\top$.
The third setup (`Learned Bilateral', $65$ weights)
takes a Gauss kernel as initialization and
fits all filter weights on the train set to minimize the
mean squared error with respect to the clean images.
We run a combination
of spatial and permutohedral convolutions on spatial and bilateral
features (`Spatial + Bilateral (Learned)') to check for a complementary
performance of the two convolutions.

\label{sec:exp:denoising}
\begin{table}[!h]
\begin{center}
  \footnotesize
  \begin{tabular}[t]{lr}
    \toprule
    Method & PSNR \\
    \midrule
    Noisy Input & $20.17$ \\
    Spatial & $26.27$ \\
    Gauss Bilateral & $26.51$ \\
    Learned Bilateral & $26.58$ \\
    Spatial + Bilateral (Learned) & \textbf{$26.65$} \\
    \bottomrule
  \end{tabular}
\end{center}
\vspace{-0.5em}
\caption{PSNR results of a denoising task using the BSDS500
  dataset~\cite{arbelaezi2011bsds500}}
\vspace{-0.5em}
\label{tab:denoising}
\end{table}
\vspace{-0.2em}

The PSNR scores evaluated on full images of the test set are
shown in Table \ref{tab:denoising}. We find that an untrained bilateral
filter already performs better than a trained spatial convolution
($26.27$ to $26.51$). A learned convolution further improve the
performance slightly. We chose this simple one-kernel setup to
validate an advantage of the generalized bilateral filter. A competitive
denoising system would employ RGB color information and also
needs to be properly adjusted in network size. Multi-layer perceptrons
have obtained state-of-the-art denoising results~\cite{burger12cvpr}
and the permutohedral lattice layer can readily be used in such an
architecture, which is intended future work.

\subsection{Additional results}
\label{sec:addresults}

This section contains more qualitative results for the experiments presented in Chapter~\ref{chap:bnn}.

\begin{figure*}[th!]
  \centering
    \includegraphics[width=\columnwidth,trim={5cm 2.5cm 5cm 4.5cm},clip]{figures/supplementary/lattice_viz.pdf}
    \vspace{-0.7cm}
  \mycaption{Visualization of the Permutohedral Lattice}
  {Sample lattice visualizations for different feature spaces. All pixels falling in the same simplex cell are shown with
  the same color. $(x,y)$ features correspond to image pixel positions, and $(r,g,b) \in [0,255]$ correspond
  to the red, green and blue color values.}
\label{fig:latticeviz}
\end{figure*}

\subsubsection{Lattice Visualization}

Figure~\ref{fig:latticeviz} shows sample lattice visualizations for different feature spaces.

\newcolumntype{L}[1]{>{\raggedright\let\newline\\\arraybackslash\hspace{0pt}}b{#1}}
\newcolumntype{C}[1]{>{\centering\let\newline\\\arraybackslash\hspace{0pt}}b{#1}}
\newcolumntype{R}[1]{>{\raggedleft\let\newline\\\arraybackslash\hspace{0pt}}b{#1}}

\subsubsection{Color Upsampling}\label{sec:color_upsampling}
\label{sec:col_upsample_extra}

Some images of the upsampling for the Pascal
VOC12 dataset are shown in Fig.~\ref{fig:Colour_upsample_visuals}. It is
especially the low level image details that are better preserved with
a learned bilateral filter compared to the Gaussian case.

\begin{figure*}[t!]
  \centering
    \subfigure{%
   \raisebox{2.0em}{
    \includegraphics[width=.06\columnwidth]{figures/supplementary/2007_004969.jpg}
   }
  }
  \subfigure{%
    \includegraphics[width=.17\columnwidth]{figures/supplementary/2007_004969_gray.pdf}
  }
  \subfigure{%
    \includegraphics[width=.17\columnwidth]{figures/supplementary/2007_004969_gt.pdf}
  }
  \subfigure{%
    \includegraphics[width=.17\columnwidth]{figures/supplementary/2007_004969_bicubic.pdf}
  }
  \subfigure{%
    \includegraphics[width=.17\columnwidth]{figures/supplementary/2007_004969_gauss.pdf}
  }
  \subfigure{%
    \includegraphics[width=.17\columnwidth]{figures/supplementary/2007_004969_learnt.pdf}
  }\\
    \subfigure{%
   \raisebox{2.0em}{
    \includegraphics[width=.06\columnwidth]{figures/supplementary/2007_003106.jpg}
   }
  }
  \subfigure{%
    \includegraphics[width=.17\columnwidth]{figures/supplementary/2007_003106_gray.pdf}
  }
  \subfigure{%
    \includegraphics[width=.17\columnwidth]{figures/supplementary/2007_003106_gt.pdf}
  }
  \subfigure{%
    \includegraphics[width=.17\columnwidth]{figures/supplementary/2007_003106_bicubic.pdf}
  }
  \subfigure{%
    \includegraphics[width=.17\columnwidth]{figures/supplementary/2007_003106_gauss.pdf}
  }
  \subfigure{%
    \includegraphics[width=.17\columnwidth]{figures/supplementary/2007_003106_learnt.pdf}
  }\\
  \setcounter{subfigure}{0}
  \small{
  \subfigure[Inp.]{%
  \raisebox{2.0em}{
    \includegraphics[width=.06\columnwidth]{figures/supplementary/2007_006837.jpg}
   }
  }
  \subfigure[Guidance]{%
    \includegraphics[width=.17\columnwidth]{figures/supplementary/2007_006837_gray.pdf}
  }
   \subfigure[GT]{%
    \includegraphics[width=.17\columnwidth]{figures/supplementary/2007_006837_gt.pdf}
  }
  \subfigure[Bicubic]{%
    \includegraphics[width=.17\columnwidth]{figures/supplementary/2007_006837_bicubic.pdf}
  }
  \subfigure[Gauss-BF]{%
    \includegraphics[width=.17\columnwidth]{figures/supplementary/2007_006837_gauss.pdf}
  }
  \subfigure[Learned-BF]{%
    \includegraphics[width=.17\columnwidth]{figures/supplementary/2007_006837_learnt.pdf}
  }
  }
  \vspace{-0.5cm}
  \mycaption{Color Upsampling}{Color $8\times$ upsampling results
  using different methods, from left to right, (a)~Low-resolution input color image (Inp.),
  (b)~Gray scale guidance image, (c)~Ground-truth color image; Upsampled color images with
  (d)~Bicubic interpolation, (e) Gauss bilateral upsampling and, (f)~Learned bilateral
  updampgling (best viewed on screen).}

\label{fig:Colour_upsample_visuals}
\end{figure*}

\subsubsection{Depth Upsampling}
\label{sec:depth_upsample_extra}

Figure~\ref{fig:depth_upsample_visuals} presents some more qualitative results comparing bicubic interpolation, Gauss
bilateral and learned bilateral upsampling on NYU depth dataset image~\cite{silberman2012indoor}.

\subsubsection{Character Recognition}\label{sec:app_character}

 Figure~\ref{fig:nnrecognition} shows the schematic of different layers
 of the network architecture for LeNet-7~\cite{lecun1998mnist}
 and DeepCNet(5, 50)~\cite{ciresan2012multi,graham2014spatially}. For the BNN variants, the first layer filters are replaced
 with learned bilateral filters and are learned end-to-end.

\subsubsection{Semantic Segmentation}\label{sec:app_semantic_segmentation}
\label{sec:semantic_bnn_extra}

Some more visual results for semantic segmentation are shown in Figure~\ref{fig:semantic_visuals}.
These include the underlying DeepLab CNN\cite{chen2014semantic} result (DeepLab),
the 2 step mean-field result with Gaussian edge potentials (+2stepMF-GaussCRF)
and also corresponding results with learned edge potentials (+2stepMF-LearnedCRF).
In general, we observe that mean-field in learned CRF leads to slightly dilated
classification regions in comparison to using Gaussian CRF thereby filling-in the
false negative pixels and also correcting some mis-classified regions.

\begin{figure*}[t!]
  \centering
    \subfigure{%
   \raisebox{2.0em}{
    \includegraphics[width=.06\columnwidth]{figures/supplementary/2bicubic}
   }
  }
  \subfigure{%
    \includegraphics[width=.17\columnwidth]{figures/supplementary/2given_image}
  }
  \subfigure{%
    \includegraphics[width=.17\columnwidth]{figures/supplementary/2ground_truth}
  }
  \subfigure{%
    \includegraphics[width=.17\columnwidth]{figures/supplementary/2bicubic}
  }
  \subfigure{%
    \includegraphics[width=.17\columnwidth]{figures/supplementary/2gauss}
  }
  \subfigure{%
    \includegraphics[width=.17\columnwidth]{figures/supplementary/2learnt}
  }\\
    \subfigure{%
   \raisebox{2.0em}{
    \includegraphics[width=.06\columnwidth]{figures/supplementary/32bicubic}
   }
  }
  \subfigure{%
    \includegraphics[width=.17\columnwidth]{figures/supplementary/32given_image}
  }
  \subfigure{%
    \includegraphics[width=.17\columnwidth]{figures/supplementary/32ground_truth}
  }
  \subfigure{%
    \includegraphics[width=.17\columnwidth]{figures/supplementary/32bicubic}
  }
  \subfigure{%
    \includegraphics[width=.17\columnwidth]{figures/supplementary/32gauss}
  }
  \subfigure{%
    \includegraphics[width=.17\columnwidth]{figures/supplementary/32learnt}
  }\\
  \setcounter{subfigure}{0}
  \small{
  \subfigure[Inp.]{%
  \raisebox{2.0em}{
    \includegraphics[width=.06\columnwidth]{figures/supplementary/41bicubic}
   }
  }
  \subfigure[Guidance]{%
    \includegraphics[width=.17\columnwidth]{figures/supplementary/41given_image}
  }
   \subfigure[GT]{%
    \includegraphics[width=.17\columnwidth]{figures/supplementary/41ground_truth}
  }
  \subfigure[Bicubic]{%
    \includegraphics[width=.17\columnwidth]{figures/supplementary/41bicubic}
  }
  \subfigure[Gauss-BF]{%
    \includegraphics[width=.17\columnwidth]{figures/supplementary/41gauss}
  }
  \subfigure[Learned-BF]{%
    \includegraphics[width=.17\columnwidth]{figures/supplementary/41learnt}
  }
  }
  \mycaption{Depth Upsampling}{Depth $8\times$ upsampling results
  using different upsampling strategies, from left to right,
  (a)~Low-resolution input depth image (Inp.),
  (b)~High-resolution guidance image, (c)~Ground-truth depth; Upsampled depth images with
  (d)~Bicubic interpolation, (e) Gauss bilateral upsampling and, (f)~Learned bilateral
  updampgling (best viewed on screen).}

\label{fig:depth_upsample_visuals}
\end{figure*}

\subsubsection{Material Segmentation}\label{sec:app_material_segmentation}
\label{sec:material_bnn_extra}

In Fig.~\ref{fig:material_visuals-app2}, we present visual results comparing 2 step
mean-field inference with Gaussian and learned pairwise CRF potentials. In
general, we observe that the pixels belonging to dominant classes in the
training data are being more accurately classified with learned CRF. This leads to
a significant improvements in overall pixel accuracy. This also results
in a slight decrease of the accuracy from less frequent class pixels thereby
slightly reducing the average class accuracy with learning. We attribute this
to the type of annotation that is available for this dataset, which is not
for the entire image but for some segments in the image. We have very few
images of the infrequent classes to combat this behaviour during training.

\subsubsection{Experiment Protocols}
\label{sec:protocols}

Table~\ref{tbl:parameters} shows experiment protocols of different experiments.

 \begin{figure*}[t!]
  \centering
  \subfigure[LeNet-7]{
    \includegraphics[width=0.7\columnwidth]{figures/supplementary/lenet_cnn_network}
    }\\
    \subfigure[DeepCNet]{
    \includegraphics[width=\columnwidth]{figures/supplementary/deepcnet_cnn_network}
    }
  \mycaption{CNNs for Character Recognition}
  {Schematic of (top) LeNet-7~\cite{lecun1998mnist} and (bottom) DeepCNet(5,50)~\cite{ciresan2012multi,graham2014spatially} architectures used in Assamese
  character recognition experiments.}
\label{fig:nnrecognition}
\end{figure*}

\definecolor{voc_1}{RGB}{0, 0, 0}
\definecolor{voc_2}{RGB}{128, 0, 0}
\definecolor{voc_3}{RGB}{0, 128, 0}
\definecolor{voc_4}{RGB}{128, 128, 0}
\definecolor{voc_5}{RGB}{0, 0, 128}
\definecolor{voc_6}{RGB}{128, 0, 128}
\definecolor{voc_7}{RGB}{0, 128, 128}
\definecolor{voc_8}{RGB}{128, 128, 128}
\definecolor{voc_9}{RGB}{64, 0, 0}
\definecolor{voc_10}{RGB}{192, 0, 0}
\definecolor{voc_11}{RGB}{64, 128, 0}
\definecolor{voc_12}{RGB}{192, 128, 0}
\definecolor{voc_13}{RGB}{64, 0, 128}
\definecolor{voc_14}{RGB}{192, 0, 128}
\definecolor{voc_15}{RGB}{64, 128, 128}
\definecolor{voc_16}{RGB}{192, 128, 128}
\definecolor{voc_17}{RGB}{0, 64, 0}
\definecolor{voc_18}{RGB}{128, 64, 0}
\definecolor{voc_19}{RGB}{0, 192, 0}
\definecolor{voc_20}{RGB}{128, 192, 0}
\definecolor{voc_21}{RGB}{0, 64, 128}
\definecolor{voc_22}{RGB}{128, 64, 128}

\begin{figure*}[t]
  \centering
  \small{
  \fcolorbox{white}{voc_1}{\rule{0pt}{6pt}\rule{6pt}{0pt}} Background~~
  \fcolorbox{white}{voc_2}{\rule{0pt}{6pt}\rule{6pt}{0pt}} Aeroplane~~
  \fcolorbox{white}{voc_3}{\rule{0pt}{6pt}\rule{6pt}{0pt}} Bicycle~~
  \fcolorbox{white}{voc_4}{\rule{0pt}{6pt}\rule{6pt}{0pt}} Bird~~
  \fcolorbox{white}{voc_5}{\rule{0pt}{6pt}\rule{6pt}{0pt}} Boat~~
  \fcolorbox{white}{voc_6}{\rule{0pt}{6pt}\rule{6pt}{0pt}} Bottle~~
  \fcolorbox{white}{voc_7}{\rule{0pt}{6pt}\rule{6pt}{0pt}} Bus~~
  \fcolorbox{white}{voc_8}{\rule{0pt}{6pt}\rule{6pt}{0pt}} Car~~ \\
  \fcolorbox{white}{voc_9}{\rule{0pt}{6pt}\rule{6pt}{0pt}} Cat~~
  \fcolorbox{white}{voc_10}{\rule{0pt}{6pt}\rule{6pt}{0pt}} Chair~~
  \fcolorbox{white}{voc_11}{\rule{0pt}{6pt}\rule{6pt}{0pt}} Cow~~
  \fcolorbox{white}{voc_12}{\rule{0pt}{6pt}\rule{6pt}{0pt}} Dining Table~~
  \fcolorbox{white}{voc_13}{\rule{0pt}{6pt}\rule{6pt}{0pt}} Dog~~
  \fcolorbox{white}{voc_14}{\rule{0pt}{6pt}\rule{6pt}{0pt}} Horse~~
  \fcolorbox{white}{voc_15}{\rule{0pt}{6pt}\rule{6pt}{0pt}} Motorbike~~
  \fcolorbox{white}{voc_16}{\rule{0pt}{6pt}\rule{6pt}{0pt}} Person~~ \\
  \fcolorbox{white}{voc_17}{\rule{0pt}{6pt}\rule{6pt}{0pt}} Potted Plant~~
  \fcolorbox{white}{voc_18}{\rule{0pt}{6pt}\rule{6pt}{0pt}} Sheep~~
  \fcolorbox{white}{voc_19}{\rule{0pt}{6pt}\rule{6pt}{0pt}} Sofa~~
  \fcolorbox{white}{voc_20}{\rule{0pt}{6pt}\rule{6pt}{0pt}} Train~~
  \fcolorbox{white}{voc_21}{\rule{0pt}{6pt}\rule{6pt}{0pt}} TV monitor~~ \\
  }
  \subfigure{%
    \includegraphics[width=.18\columnwidth]{figures/supplementary/2007_001423_given.jpg}
  }
  \subfigure{%
    \includegraphics[width=.18\columnwidth]{figures/supplementary/2007_001423_gt.png}
  }
  \subfigure{%
    \includegraphics[width=.18\columnwidth]{figures/supplementary/2007_001423_cnn.png}
  }
  \subfigure{%
    \includegraphics[width=.18\columnwidth]{figures/supplementary/2007_001423_gauss.png}
  }
  \subfigure{%
    \includegraphics[width=.18\columnwidth]{figures/supplementary/2007_001423_learnt.png}
  }\\
  \subfigure{%
    \includegraphics[width=.18\columnwidth]{figures/supplementary/2007_001430_given.jpg}
  }
  \subfigure{%
    \includegraphics[width=.18\columnwidth]{figures/supplementary/2007_001430_gt.png}
  }
  \subfigure{%
    \includegraphics[width=.18\columnwidth]{figures/supplementary/2007_001430_cnn.png}
  }
  \subfigure{%
    \includegraphics[width=.18\columnwidth]{figures/supplementary/2007_001430_gauss.png}
  }
  \subfigure{%
    \includegraphics[width=.18\columnwidth]{figures/supplementary/2007_001430_learnt.png}
  }\\
    \subfigure{%
    \includegraphics[width=.18\columnwidth]{figures/supplementary/2007_007996_given.jpg}
  }
  \subfigure{%
    \includegraphics[width=.18\columnwidth]{figures/supplementary/2007_007996_gt.png}
  }
  \subfigure{%
    \includegraphics[width=.18\columnwidth]{figures/supplementary/2007_007996_cnn.png}
  }
  \subfigure{%
    \includegraphics[width=.18\columnwidth]{figures/supplementary/2007_007996_gauss.png}
  }
  \subfigure{%
    \includegraphics[width=.18\columnwidth]{figures/supplementary/2007_007996_learnt.png}
  }\\
   \subfigure{%
    \includegraphics[width=.18\columnwidth]{figures/supplementary/2010_002682_given.jpg}
  }
  \subfigure{%
    \includegraphics[width=.18\columnwidth]{figures/supplementary/2010_002682_gt.png}
  }
  \subfigure{%
    \includegraphics[width=.18\columnwidth]{figures/supplementary/2010_002682_cnn.png}
  }
  \subfigure{%
    \includegraphics[width=.18\columnwidth]{figures/supplementary/2010_002682_gauss.png}
  }
  \subfigure{%
    \includegraphics[width=.18\columnwidth]{figures/supplementary/2010_002682_learnt.png}
  }\\
     \subfigure{%
    \includegraphics[width=.18\columnwidth]{figures/supplementary/2010_004789_given.jpg}
  }
  \subfigure{%
    \includegraphics[width=.18\columnwidth]{figures/supplementary/2010_004789_gt.png}
  }
  \subfigure{%
    \includegraphics[width=.18\columnwidth]{figures/supplementary/2010_004789_cnn.png}
  }
  \subfigure{%
    \includegraphics[width=.18\columnwidth]{figures/supplementary/2010_004789_gauss.png}
  }
  \subfigure{%
    \includegraphics[width=.18\columnwidth]{figures/supplementary/2010_004789_learnt.png}
  }\\
       \subfigure{%
    \includegraphics[width=.18\columnwidth]{figures/supplementary/2007_001311_given.jpg}
  }
  \subfigure{%
    \includegraphics[width=.18\columnwidth]{figures/supplementary/2007_001311_gt.png}
  }
  \subfigure{%
    \includegraphics[width=.18\columnwidth]{figures/supplementary/2007_001311_cnn.png}
  }
  \subfigure{%
    \includegraphics[width=.18\columnwidth]{figures/supplementary/2007_001311_gauss.png}
  }
  \subfigure{%
    \includegraphics[width=.18\columnwidth]{figures/supplementary/2007_001311_learnt.png}
  }\\
  \setcounter{subfigure}{0}
  \subfigure[Input]{%
    \includegraphics[width=.18\columnwidth]{figures/supplementary/2010_003531_given.jpg}
  }
  \subfigure[Ground Truth]{%
    \includegraphics[width=.18\columnwidth]{figures/supplementary/2010_003531_gt.png}
  }
  \subfigure[DeepLab]{%
    \includegraphics[width=.18\columnwidth]{figures/supplementary/2010_003531_cnn.png}
  }
  \subfigure[+GaussCRF]{%
    \includegraphics[width=.18\columnwidth]{figures/supplementary/2010_003531_gauss.png}
  }
  \subfigure[+LearnedCRF]{%
    \includegraphics[width=.18\columnwidth]{figures/supplementary/2010_003531_learnt.png}
  }
  \vspace{-0.3cm}
  \mycaption{Semantic Segmentation}{Example results of semantic segmentation.
  (c)~depicts the unary results before application of MF, (d)~after two steps of MF with Gaussian edge CRF potentials, (e)~after
  two steps of MF with learned edge CRF potentials.}
    \label{fig:semantic_visuals}
\end{figure*}


\definecolor{minc_1}{HTML}{771111}
\definecolor{minc_2}{HTML}{CAC690}
\definecolor{minc_3}{HTML}{EEEEEE}
\definecolor{minc_4}{HTML}{7C8FA6}
\definecolor{minc_5}{HTML}{597D31}
\definecolor{minc_6}{HTML}{104410}
\definecolor{minc_7}{HTML}{BB819C}
\definecolor{minc_8}{HTML}{D0CE48}
\definecolor{minc_9}{HTML}{622745}
\definecolor{minc_10}{HTML}{666666}
\definecolor{minc_11}{HTML}{D54A31}
\definecolor{minc_12}{HTML}{101044}
\definecolor{minc_13}{HTML}{444126}
\definecolor{minc_14}{HTML}{75D646}
\definecolor{minc_15}{HTML}{DD4348}
\definecolor{minc_16}{HTML}{5C8577}
\definecolor{minc_17}{HTML}{C78472}
\definecolor{minc_18}{HTML}{75D6D0}
\definecolor{minc_19}{HTML}{5B4586}
\definecolor{minc_20}{HTML}{C04393}
\definecolor{minc_21}{HTML}{D69948}
\definecolor{minc_22}{HTML}{7370D8}
\definecolor{minc_23}{HTML}{7A3622}
\definecolor{minc_24}{HTML}{000000}

\begin{figure*}[t]
  \centering
  \small{
  \fcolorbox{white}{minc_1}{\rule{0pt}{6pt}\rule{6pt}{0pt}} Brick~~
  \fcolorbox{white}{minc_2}{\rule{0pt}{6pt}\rule{6pt}{0pt}} Carpet~~
  \fcolorbox{white}{minc_3}{\rule{0pt}{6pt}\rule{6pt}{0pt}} Ceramic~~
  \fcolorbox{white}{minc_4}{\rule{0pt}{6pt}\rule{6pt}{0pt}} Fabric~~
  \fcolorbox{white}{minc_5}{\rule{0pt}{6pt}\rule{6pt}{0pt}} Foliage~~
  \fcolorbox{white}{minc_6}{\rule{0pt}{6pt}\rule{6pt}{0pt}} Food~~
  \fcolorbox{white}{minc_7}{\rule{0pt}{6pt}\rule{6pt}{0pt}} Glass~~
  \fcolorbox{white}{minc_8}{\rule{0pt}{6pt}\rule{6pt}{0pt}} Hair~~ \\
  \fcolorbox{white}{minc_9}{\rule{0pt}{6pt}\rule{6pt}{0pt}} Leather~~
  \fcolorbox{white}{minc_10}{\rule{0pt}{6pt}\rule{6pt}{0pt}} Metal~~
  \fcolorbox{white}{minc_11}{\rule{0pt}{6pt}\rule{6pt}{0pt}} Mirror~~
  \fcolorbox{white}{minc_12}{\rule{0pt}{6pt}\rule{6pt}{0pt}} Other~~
  \fcolorbox{white}{minc_13}{\rule{0pt}{6pt}\rule{6pt}{0pt}} Painted~~
  \fcolorbox{white}{minc_14}{\rule{0pt}{6pt}\rule{6pt}{0pt}} Paper~~
  \fcolorbox{white}{minc_15}{\rule{0pt}{6pt}\rule{6pt}{0pt}} Plastic~~\\
  \fcolorbox{white}{minc_16}{\rule{0pt}{6pt}\rule{6pt}{0pt}} Polished Stone~~
  \fcolorbox{white}{minc_17}{\rule{0pt}{6pt}\rule{6pt}{0pt}} Skin~~
  \fcolorbox{white}{minc_18}{\rule{0pt}{6pt}\rule{6pt}{0pt}} Sky~~
  \fcolorbox{white}{minc_19}{\rule{0pt}{6pt}\rule{6pt}{0pt}} Stone~~
  \fcolorbox{white}{minc_20}{\rule{0pt}{6pt}\rule{6pt}{0pt}} Tile~~
  \fcolorbox{white}{minc_21}{\rule{0pt}{6pt}\rule{6pt}{0pt}} Wallpaper~~
  \fcolorbox{white}{minc_22}{\rule{0pt}{6pt}\rule{6pt}{0pt}} Water~~
  \fcolorbox{white}{minc_23}{\rule{0pt}{6pt}\rule{6pt}{0pt}} Wood~~ \\
  }
  \subfigure{%
    \includegraphics[width=.18\columnwidth]{figures/supplementary/000010868_given.jpg}
  }
  \subfigure{%
    \includegraphics[width=.18\columnwidth]{figures/supplementary/000010868_gt.png}
  }
  \subfigure{%
    \includegraphics[width=.18\columnwidth]{figures/supplementary/000010868_cnn.png}
  }
  \subfigure{%
    \includegraphics[width=.18\columnwidth]{figures/supplementary/000010868_gauss.png}
  }
  \subfigure{%
    \includegraphics[width=.18\columnwidth]{figures/supplementary/000010868_learnt.png}
  }\\[-2ex]
  \subfigure{%
    \includegraphics[width=.18\columnwidth]{figures/supplementary/000006011_given.jpg}
  }
  \subfigure{%
    \includegraphics[width=.18\columnwidth]{figures/supplementary/000006011_gt.png}
  }
  \subfigure{%
    \includegraphics[width=.18\columnwidth]{figures/supplementary/000006011_cnn.png}
  }
  \subfigure{%
    \includegraphics[width=.18\columnwidth]{figures/supplementary/000006011_gauss.png}
  }
  \subfigure{%
    \includegraphics[width=.18\columnwidth]{figures/supplementary/000006011_learnt.png}
  }\\[-2ex]
    \subfigure{%
    \includegraphics[width=.18\columnwidth]{figures/supplementary/000008553_given.jpg}
  }
  \subfigure{%
    \includegraphics[width=.18\columnwidth]{figures/supplementary/000008553_gt.png}
  }
  \subfigure{%
    \includegraphics[width=.18\columnwidth]{figures/supplementary/000008553_cnn.png}
  }
  \subfigure{%
    \includegraphics[width=.18\columnwidth]{figures/supplementary/000008553_gauss.png}
  }
  \subfigure{%
    \includegraphics[width=.18\columnwidth]{figures/supplementary/000008553_learnt.png}
  }\\[-2ex]
   \subfigure{%
    \includegraphics[width=.18\columnwidth]{figures/supplementary/000009188_given.jpg}
  }
  \subfigure{%
    \includegraphics[width=.18\columnwidth]{figures/supplementary/000009188_gt.png}
  }
  \subfigure{%
    \includegraphics[width=.18\columnwidth]{figures/supplementary/000009188_cnn.png}
  }
  \subfigure{%
    \includegraphics[width=.18\columnwidth]{figures/supplementary/000009188_gauss.png}
  }
  \subfigure{%
    \includegraphics[width=.18\columnwidth]{figures/supplementary/000009188_learnt.png}
  }\\[-2ex]
  \setcounter{subfigure}{0}
  \subfigure[Input]{%
    \includegraphics[width=.18\columnwidth]{figures/supplementary/000023570_given.jpg}
  }
  \subfigure[Ground Truth]{%
    \includegraphics[width=.18\columnwidth]{figures/supplementary/000023570_gt.png}
  }
  \subfigure[DeepLab]{%
    \includegraphics[width=.18\columnwidth]{figures/supplementary/000023570_cnn.png}
  }
  \subfigure[+GaussCRF]{%
    \includegraphics[width=.18\columnwidth]{figures/supplementary/000023570_gauss.png}
  }
  \subfigure[+LearnedCRF]{%
    \includegraphics[width=.18\columnwidth]{figures/supplementary/000023570_learnt.png}
  }
  \mycaption{Material Segmentation}{Example results of material segmentation.
  (c)~depicts the unary results before application of MF, (d)~after two steps of MF with Gaussian edge CRF potentials, (e)~after two steps of MF with learned edge CRF potentials.}
    \label{fig:material_visuals-app2}
\end{figure*}


\begin{table*}[h]
\tiny
  \centering
    \begin{tabular}{L{2.3cm} L{2.25cm} C{1.5cm} C{0.7cm} C{0.6cm} C{0.7cm} C{0.7cm} C{0.7cm} C{1.6cm} C{0.6cm} C{0.6cm} C{0.6cm}}
      \toprule
& & & & & \multicolumn{3}{c}{\textbf{Data Statistics}} & \multicolumn{4}{c}{\textbf{Training Protocol}} \\

\textbf{Experiment} & \textbf{Feature Types} & \textbf{Feature Scales} & \textbf{Filter Size} & \textbf{Filter Nbr.} & \textbf{Train}  & \textbf{Val.} & \textbf{Test} & \textbf{Loss Type} & \textbf{LR} & \textbf{Batch} & \textbf{Epochs} \\
      \midrule
      \multicolumn{2}{c}{\textbf{Single Bilateral Filter Applications}} & & & & & & & & & \\
      \textbf{2$\times$ Color Upsampling} & Position$_{1}$, Intensity (3D) & 0.13, 0.17 & 65 & 2 & 10581 & 1449 & 1456 & MSE & 1e-06 & 200 & 94.5\\
      \textbf{4$\times$ Color Upsampling} & Position$_{1}$, Intensity (3D) & 0.06, 0.17 & 65 & 2 & 10581 & 1449 & 1456 & MSE & 1e-06 & 200 & 94.5\\
      \textbf{8$\times$ Color Upsampling} & Position$_{1}$, Intensity (3D) & 0.03, 0.17 & 65 & 2 & 10581 & 1449 & 1456 & MSE & 1e-06 & 200 & 94.5\\
      \textbf{16$\times$ Color Upsampling} & Position$_{1}$, Intensity (3D) & 0.02, 0.17 & 65 & 2 & 10581 & 1449 & 1456 & MSE & 1e-06 & 200 & 94.5\\
      \textbf{Depth Upsampling} & Position$_{1}$, Color (5D) & 0.05, 0.02 & 665 & 2 & 795 & 100 & 654 & MSE & 1e-07 & 50 & 251.6\\
      \textbf{Mesh Denoising} & Isomap (4D) & 46.00 & 63 & 2 & 1000 & 200 & 500 & MSE & 100 & 10 & 100.0 \\
      \midrule
      \multicolumn{2}{c}{\textbf{DenseCRF Applications}} & & & & & & & & &\\
      \multicolumn{2}{l}{\textbf{Semantic Segmentation}} & & & & & & & & &\\
      \textbf{- 1step MF} & Position$_{1}$, Color (5D); Position$_{1}$ (2D) & 0.01, 0.34; 0.34  & 665; 19  & 2; 2 & 10581 & 1449 & 1456 & Logistic & 0.1 & 5 & 1.4 \\
      \textbf{- 2step MF} & Position$_{1}$, Color (5D); Position$_{1}$ (2D) & 0.01, 0.34; 0.34 & 665; 19 & 2; 2 & 10581 & 1449 & 1456 & Logistic & 0.1 & 5 & 1.4 \\
      \textbf{- \textit{loose} 2step MF} & Position$_{1}$, Color (5D); Position$_{1}$ (2D) & 0.01, 0.34; 0.34 & 665; 19 & 2; 2 &10581 & 1449 & 1456 & Logistic & 0.1 & 5 & +1.9  \\ \\
      \multicolumn{2}{l}{\textbf{Material Segmentation}} & & & & & & & & &\\
      \textbf{- 1step MF} & Position$_{2}$, Lab-Color (5D) & 5.00, 0.05, 0.30  & 665 & 2 & 928 & 150 & 1798 & Weighted Logistic & 1e-04 & 24 & 2.6 \\
      \textbf{- 2step MF} & Position$_{2}$, Lab-Color (5D) & 5.00, 0.05, 0.30 & 665 & 2 & 928 & 150 & 1798 & Weighted Logistic & 1e-04 & 12 & +0.7 \\
      \textbf{- \textit{loose} 2step MF} & Position$_{2}$, Lab-Color (5D) & 5.00, 0.05, 0.30 & 665 & 2 & 928 & 150 & 1798 & Weighted Logistic & 1e-04 & 12 & +0.2\\
      \midrule
      \multicolumn{2}{c}{\textbf{Neural Network Applications}} & & & & & & & & &\\
      \textbf{Tiles: CNN-9$\times$9} & - & - & 81 & 4 & 10000 & 1000 & 1000 & Logistic & 0.01 & 100 & 500.0 \\
      \textbf{Tiles: CNN-13$\times$13} & - & - & 169 & 6 & 10000 & 1000 & 1000 & Logistic & 0.01 & 100 & 500.0 \\
      \textbf{Tiles: CNN-17$\times$17} & - & - & 289 & 8 & 10000 & 1000 & 1000 & Logistic & 0.01 & 100 & 500.0 \\
      \textbf{Tiles: CNN-21$\times$21} & - & - & 441 & 10 & 10000 & 1000 & 1000 & Logistic & 0.01 & 100 & 500.0 \\
      \textbf{Tiles: BNN} & Position$_{1}$, Color (5D) & 0.05, 0.04 & 63 & 1 & 10000 & 1000 & 1000 & Logistic & 0.01 & 100 & 30.0 \\
      \textbf{LeNet} & - & - & 25 & 2 & 5490 & 1098 & 1647 & Logistic & 0.1 & 100 & 182.2 \\
      \textbf{Crop-LeNet} & - & - & 25 & 2 & 5490 & 1098 & 1647 & Logistic & 0.1 & 100 & 182.2 \\
      \textbf{BNN-LeNet} & Position$_{2}$ (2D) & 20.00 & 7 & 1 & 5490 & 1098 & 1647 & Logistic & 0.1 & 100 & 182.2 \\
      \textbf{DeepCNet} & - & - & 9 & 1 & 5490 & 1098 & 1647 & Logistic & 0.1 & 100 & 182.2 \\
      \textbf{Crop-DeepCNet} & - & - & 9 & 1 & 5490 & 1098 & 1647 & Logistic & 0.1 & 100 & 182.2 \\
      \textbf{BNN-DeepCNet} & Position$_{2}$ (2D) & 40.00  & 7 & 1 & 5490 & 1098 & 1647 & Logistic & 0.1 & 100 & 182.2 \\
      \bottomrule
      \\
    \end{tabular}
    \mycaption{Experiment Protocols} {Experiment protocols for the different experiments presented in this work. \textbf{Feature Types}:
    Feature spaces used for the bilateral convolutions. Position$_1$ corresponds to un-normalized pixel positions whereas Position$_2$ corresponds
    to pixel positions normalized to $[0,1]$ with respect to the given image. \textbf{Feature Scales}: Cross-validated scales for the features used.
     \textbf{Filter Size}: Number of elements in the filter that is being learned. \textbf{Filter Nbr.}: Half-width of the filter. \textbf{Train},
     \textbf{Val.} and \textbf{Test} corresponds to the number of train, validation and test images used in the experiment. \textbf{Loss Type}: Type
     of loss used for back-propagation. ``MSE'' corresponds to Euclidean mean squared error loss and ``Logistic'' corresponds to multinomial logistic
     loss. ``Weighted Logistic'' is the class-weighted multinomial logistic loss. We weighted the loss with inverse class probability for material
     segmentation task due to the small availability of training data with class imbalance. \textbf{LR}: Fixed learning rate used in stochastic gradient
     descent. \textbf{Batch}: Number of images used in one parameter update step. \textbf{Epochs}: Number of training epochs. In all the experiments,
     we used fixed momentum of 0.9 and weight decay of 0.0005 for stochastic gradient descent. ```Color Upsampling'' experiments in this Table corresponds
     to those performed on Pascal VOC12 dataset images. For all experiments using Pascal VOC12 images, we use extended
     training segmentation dataset available from~\cite{hariharan2011moredata}, and used standard validation and test splits
     from the main dataset~\cite{voc2012segmentation}.}
  \label{tbl:parameters}
\end{table*}

\clearpage

\section{Parameters and Additional Results for Video Propagation Networks}

In this Section, we present experiment protocols and additional qualitative results for experiments
on video object segmentation, semantic video segmentation and video color
propagation. Table~\ref{tbl:parameters_supp} shows the feature scales and other parameters used in different experiments.
Figures~\ref{fig:video_seg_pos_supp} show some qualitative results on video object segmentation
with some failure cases in Fig.~\ref{fig:video_seg_neg_supp}.
Figure~\ref{fig:semantic_visuals_supp} shows some qualitative results on semantic video segmentation and
Fig.~\ref{fig:color_visuals_supp} shows results on video color propagation.

\newcolumntype{L}[1]{>{\raggedright\let\newline\\\arraybackslash\hspace{0pt}}b{#1}}
\newcolumntype{C}[1]{>{\centering\let\newline\\\arraybackslash\hspace{0pt}}b{#1}}
\newcolumntype{R}[1]{>{\raggedleft\let\newline\\\arraybackslash\hspace{0pt}}b{#1}}

\begin{table*}[h]
\tiny
  \centering
    \begin{tabular}{L{3.0cm} L{2.4cm} L{2.8cm} L{2.8cm} C{0.5cm} C{1.0cm} L{1.2cm}}
      \toprule
\textbf{Experiment} & \textbf{Feature Type} & \textbf{Feature Scale-1, $\Lambda_a$} & \textbf{Feature Scale-2, $\Lambda_b$} & \textbf{$\alpha$} & \textbf{Input Frames} & \textbf{Loss Type} \\
      \midrule
      \textbf{Video Object Segmentation} & ($x,y,Y,Cb,Cr,t$) & (0.02,0.02,0.07,0.4,0.4,0.01) & (0.03,0.03,0.09,0.5,0.5,0.2) & 0.5 & 9 & Logistic\\
      \midrule
      \textbf{Semantic Video Segmentation} & & & & & \\
      \textbf{with CNN1~\cite{yu2015multi}-NoFlow} & ($x,y,R,G,B,t$) & (0.08,0.08,0.2,0.2,0.2,0.04) & (0.11,0.11,0.2,0.2,0.2,0.04) & 0.5 & 3 & Logistic \\
      \textbf{with CNN1~\cite{yu2015multi}-Flow} & ($x+u_x,y+u_y,R,G,B,t$) & (0.11,0.11,0.14,0.14,0.14,0.03) & (0.08,0.08,0.12,0.12,0.12,0.01) & 0.65 & 3 & Logistic\\
      \textbf{with CNN2~\cite{richter2016playing}-Flow} & ($x+u_x,y+u_y,R,G,B,t$) & (0.08,0.08,0.2,0.2,0.2,0.04) & (0.09,0.09,0.25,0.25,0.25,0.03) & 0.5 & 4 & Logistic\\
      \midrule
      \textbf{Video Color Propagation} & ($x,y,I,t$)  & (0.04,0.04,0.2,0.04) & No second kernel & 1 & 4 & MSE\\
      \bottomrule
      \\
    \end{tabular}
    \mycaption{Experiment Protocols} {Experiment protocols for the different experiments presented in this work. \textbf{Feature Types}:
    Feature spaces used for the bilateral convolutions, with position ($x,y$) and color
    ($R,G,B$ or $Y,Cb,Cr$) features $\in [0,255]$. $u_x$, $u_y$ denotes optical flow with respect
    to the present frame and $I$ denotes grayscale intensity.
    \textbf{Feature Scales ($\Lambda_a, \Lambda_b$)}: Cross-validated scales for the features used.
    \textbf{$\alpha$}: Exponential time decay for the input frames.
    \textbf{Input Frames}: Number of input frames for VPN.
    \textbf{Loss Type}: Type
     of loss used for back-propagation. ``MSE'' corresponds to Euclidean mean squared error loss and ``Logistic'' corresponds to multinomial logistic loss.}
  \label{tbl:parameters_supp}
\end{table*}

% \begin{figure}[th!]
% \begin{center}
%   \centerline{\includegraphics[width=\textwidth]{figures/video_seg_visuals_supp_small.pdf}}
%     \mycaption{Video Object Segmentation}
%     {Shown are the different frames in example videos with the corresponding
%     ground truth (GT) masks, predictions from BVS~\cite{marki2016bilateral},
%     OFL~\cite{tsaivideo}, VPN (VPN-Stage2) and VPN-DLab (VPN-DeepLab) models.}
%     \label{fig:video_seg_small_supp}
% \end{center}
% \vspace{-1.0cm}
% \end{figure}

\begin{figure}[th!]
\begin{center}
  \centerline{\includegraphics[width=0.7\textwidth]{figures/video_seg_visuals_supp_positive.pdf}}
    \mycaption{Video Object Segmentation}
    {Shown are the different frames in example videos with the corresponding
    ground truth (GT) masks, predictions from BVS~\cite{marki2016bilateral},
    OFL~\cite{tsaivideo}, VPN (VPN-Stage2) and VPN-DLab (VPN-DeepLab) models.}
    \label{fig:video_seg_pos_supp}
\end{center}
\vspace{-1.0cm}
\end{figure}

\begin{figure}[th!]
\begin{center}
  \centerline{\includegraphics[width=0.7\textwidth]{figures/video_seg_visuals_supp_negative.pdf}}
    \mycaption{Failure Cases for Video Object Segmentation}
    {Shown are the different frames in example videos with the corresponding
    ground truth (GT) masks, predictions from BVS~\cite{marki2016bilateral},
    OFL~\cite{tsaivideo}, VPN (VPN-Stage2) and VPN-DLab (VPN-DeepLab) models.}
    \label{fig:video_seg_neg_supp}
\end{center}
\vspace{-1.0cm}
\end{figure}

\begin{figure}[th!]
\begin{center}
  \centerline{\includegraphics[width=0.9\textwidth]{figures/supp_semantic_visual.pdf}}
    \mycaption{Semantic Video Segmentation}
    {Input video frames and the corresponding ground truth (GT)
    segmentation together with the predictions of CNN~\cite{yu2015multi} and with
    VPN-Flow.}
    \label{fig:semantic_visuals_supp}
\end{center}
\vspace{-0.7cm}
\end{figure}

\begin{figure}[th!]
\begin{center}
  \centerline{\includegraphics[width=\textwidth]{figures/colorization_visuals_supp.pdf}}
  \mycaption{Video Color Propagation}
  {Input grayscale video frames and corresponding ground-truth (GT) color images
  together with color predictions of Levin et al.~\cite{levin2004colorization} and VPN-Stage1 models.}
  \label{fig:color_visuals_supp}
\end{center}
\vspace{-0.7cm}
\end{figure}

\clearpage

\section{Additional Material for Bilateral Inception Networks}
\label{sec:binception-app}

In this section of the Appendix, we first discuss the use of approximate bilateral
filtering in BI modules (Sec.~\ref{sec:lattice}).
Later, we present some qualitative results using different models for the approach presented in
Chapter~\ref{chap:binception} (Sec.~\ref{sec:qualitative-app}).

\subsection{Approximate Bilateral Filtering}
\label{sec:lattice}

The bilateral inception module presented in Chapter~\ref{chap:binception} computes a matrix-vector
product between a Gaussian filter $K$ and a vector of activations $\bz_c$.
Bilateral filtering is an important operation and many algorithmic techniques have been
proposed to speed-up this operation~\cite{paris2006fast,adams2010fast,gastal2011domain}.
In the main paper we opted to implement what can be considered the
brute-force variant of explicitly constructing $K$ and then using BLAS to compute the
matrix-vector product. This resulted in a few millisecond operation.
The explicit way to compute is possible due to the
reduction to super-pixels, e.g., it would not work for DenseCRF variants
that operate on the full image resolution.

Here, we present experiments where we use the fast approximate bilateral filtering
algorithm of~\cite{adams2010fast}, which is also used in Chapter~\ref{chap:bnn}
for learning sparse high dimensional filters. This
choice allows for larger dimensions of matrix-vector multiplication. The reason for choosing
the explicit multiplication in Chapter~\ref{chap:binception} was that it was computationally faster.
For the small sizes of the involved matrices and vectors, the explicit computation is sufficient and we had no
GPU implementation of an approximate technique that matched this runtime. Also it
is conceptually easier and the gradient to the feature transformations ($\Lambda \mathbf{f}$) is
obtained using standard matrix calculus.

\subsubsection{Experiments}

We modified the existing segmentation architectures analogous to those in Chapter~\ref{chap:binception}.
The main difference is that, here, the inception modules use the lattice
approximation~\cite{adams2010fast} to compute the bilateral filtering.
Using the lattice approximation did not allow us to back-propagate through feature transformations ($\Lambda$)
and thus we used hand-specified feature scales as will be explained later.
Specifically, we take CNN architectures from the works
of~\cite{chen2014semantic,zheng2015conditional,bell2015minc} and insert the BI modules between
the spatial FC layers.
We use superpixels from~\cite{DollarICCV13edges}
for all the experiments with the lattice approximation. Experiments are
performed using Caffe neural network framework~\cite{jia2014caffe}.

\begin{table}
  \small
  \centering
  \begin{tabular}{p{5.5cm}>{\raggedright\arraybackslash}p{1.4cm}>{\centering\arraybackslash}p{2.2cm}}
    \toprule
		\textbf{Model} & \emph{IoU} & \emph{Runtime}(ms) \\
    \midrule

    %%%%%%%%%%%% Scores computed by us)%%%%%%%%%%%%
		\deeplablargefov & 68.9 & 145ms\\
    \midrule
    \bi{7}{2}-\bi{8}{10}& \textbf{73.8} & +600 \\
    \midrule
    \deeplablargefovcrf~\cite{chen2014semantic} & 72.7 & +830\\
    \deeplabmsclargefovcrf~\cite{chen2014semantic} & \textbf{73.6} & +880\\
    DeepLab-EdgeNet~\cite{chen2015semantic} & 71.7 & +30\\
    DeepLab-EdgeNet-CRF~\cite{chen2015semantic} & \textbf{73.6} & +860\\
  \bottomrule \\
  \end{tabular}
  \mycaption{Semantic Segmentation using the DeepLab model}
  {IoU scores on the Pascal VOC12 segmentation test dataset
  with different models and our modified inception model.
  Also shown are the corresponding runtimes in milliseconds. Runtimes
  also include superpixel computations (300 ms with Dollar superpixels~\cite{DollarICCV13edges})}
  \label{tab:largefovresults}
\end{table}

\paragraph{Semantic Segmentation}
The experiments in this section use the Pascal VOC12 segmentation dataset~\cite{voc2012segmentation} with 21 object classes and the images have a maximum resolution of 0.25 megapixels.
For all experiments on VOC12, we train using the extended training set of
10581 images collected by~\cite{hariharan2011moredata}.
We modified the \deeplab~network architecture of~\cite{chen2014semantic} and
the CRFasRNN architecture from~\cite{zheng2015conditional} which uses a CNN with
deconvolution layers followed by DenseCRF trained end-to-end.

\paragraph{DeepLab Model}\label{sec:deeplabmodel}
We experimented with the \bi{7}{2}-\bi{8}{10} inception model.
Results using the~\deeplab~model are summarized in Tab.~\ref{tab:largefovresults}.
Although we get similar improvements with inception modules as with the
explicit kernel computation, using lattice approximation is slower.

\begin{table}
  \small
  \centering
  \begin{tabular}{p{6.4cm}>{\raggedright\arraybackslash}p{1.8cm}>{\raggedright\arraybackslash}p{1.8cm}}
    \toprule
    \textbf{Model} & \emph{IoU (Val)} & \emph{IoU (Test)}\\
    \midrule
    %%%%%%%%%%%% Scores computed by us)%%%%%%%%%%%%
    CNN &  67.5 & - \\
    \deconv (CNN+Deconvolutions) & 69.8 & 72.0 \\
    \midrule
    \bi{3}{6}-\bi{4}{6}-\bi{7}{2}-\bi{8}{6}& 71.9 & - \\
    \bi{3}{6}-\bi{4}{6}-\bi{7}{2}-\bi{8}{6}-\gi{6}& 73.6 &  \href{http://host.robots.ox.ac.uk:8080/anonymous/VOTV5E.html}{\textbf{75.2}}\\
    \midrule
    \deconvcrf (CRF-RNN)~\cite{zheng2015conditional} & 73.0 & 74.7\\
    Context-CRF-RNN~\cite{yu2015multi} & ~~ - ~ & \textbf{75.3} \\
    \bottomrule \\
  \end{tabular}
  \mycaption{Semantic Segmentation using the CRFasRNN model}{IoU score corresponding to different models
  on Pascal VOC12 reduced validation / test segmentation dataset. The reduced validation set consists of 346 images
  as used in~\cite{zheng2015conditional} where we adapted the model from.}
  \label{tab:deconvresults-app}
\end{table}

\paragraph{CRFasRNN Model}\label{sec:deepinception}
We add BI modules after score-pool3, score-pool4, \fc{7} and \fc{8} $1\times1$ convolution layers
resulting in the \bi{3}{6}-\bi{4}{6}-\bi{7}{2}-\bi{8}{6}
model and also experimented with another variant where $BI_8$ is followed by another inception
module, G$(6)$, with 6 Gaussian kernels.
Note that here also we discarded both deconvolution and DenseCRF parts of the original model~\cite{zheng2015conditional}
and inserted the BI modules in the base CNN and found similar improvements compared to the inception modules with explicit
kernel computaion. See Tab.~\ref{tab:deconvresults-app} for results on the CRFasRNN model.

\paragraph{Material Segmentation}
Table~\ref{tab:mincresults-app} shows the results on the MINC dataset~\cite{bell2015minc}
obtained by modifying the AlexNet architecture with our inception modules. We observe
similar improvements as with explicit kernel construction.
For this model, we do not provide any learned setup due to very limited segment training
data. The weights to combine outputs in the bilateral inception layer are
found by validation on the validation set.

\begin{table}[t]
  \small
  \centering
  \begin{tabular}{p{3.5cm}>{\centering\arraybackslash}p{4.0cm}}
    \toprule
    \textbf{Model} & Class / Total accuracy\\
    \midrule

    %%%%%%%%%%%% Scores computed by us)%%%%%%%%%%%%
    AlexNet CNN & 55.3 / 58.9 \\
    \midrule
    \bi{7}{2}-\bi{8}{6}& 68.5 / 71.8 \\
    \bi{7}{2}-\bi{8}{6}-G$(6)$& 67.6 / 73.1 \\
    \midrule
    AlexNet-CRF & 65.5 / 71.0 \\
    \bottomrule \\
  \end{tabular}
  \mycaption{Material Segmentation using AlexNet}{Pixel accuracy of different models on
  the MINC material segmentation test dataset~\cite{bell2015minc}.}
  \label{tab:mincresults-app}
\end{table}

\paragraph{Scales of Bilateral Inception Modules}
\label{sec:scales}

Unlike the explicit kernel technique presented in the main text (Chapter~\ref{chap:binception}),
we didn't back-propagate through feature transformation ($\Lambda$)
using the approximate bilateral filter technique.
So, the feature scales are hand-specified and validated, which are as follows.
The optimal scale values for the \bi{7}{2}-\bi{8}{2} model are found by validation for the best performance which are
$\sigma_{xy}$ = (0.1, 0.1) for the spatial (XY) kernel and $\sigma_{rgbxy}$ = (0.1, 0.1, 0.1, 0.01, 0.01) for color and position (RGBXY)  kernel.
Next, as more kernels are added to \bi{8}{2}, we set scales to be $\alpha$*($\sigma_{xy}$, $\sigma_{rgbxy}$).
The value of $\alpha$ is chosen as  1, 0.5, 0.1, 0.05, 0.1, at uniform interval, for the \bi{8}{10} bilateral inception module.


\subsection{Qualitative Results}
\label{sec:qualitative-app}

In this section, we present more qualitative results obtained using the BI module with explicit
kernel computation technique presented in Chapter~\ref{chap:binception}. Results on the Pascal VOC12
dataset~\cite{voc2012segmentation} using the DeepLab-LargeFOV model are shown in Fig.~\ref{fig:semantic_visuals-app},
followed by the results on MINC dataset~\cite{bell2015minc}
in Fig.~\ref{fig:material_visuals-app} and on
Cityscapes dataset~\cite{Cordts2015Cvprw} in Fig.~\ref{fig:street_visuals-app}.


\definecolor{voc_1}{RGB}{0, 0, 0}
\definecolor{voc_2}{RGB}{128, 0, 0}
\definecolor{voc_3}{RGB}{0, 128, 0}
\definecolor{voc_4}{RGB}{128, 128, 0}
\definecolor{voc_5}{RGB}{0, 0, 128}
\definecolor{voc_6}{RGB}{128, 0, 128}
\definecolor{voc_7}{RGB}{0, 128, 128}
\definecolor{voc_8}{RGB}{128, 128, 128}
\definecolor{voc_9}{RGB}{64, 0, 0}
\definecolor{voc_10}{RGB}{192, 0, 0}
\definecolor{voc_11}{RGB}{64, 128, 0}
\definecolor{voc_12}{RGB}{192, 128, 0}
\definecolor{voc_13}{RGB}{64, 0, 128}
\definecolor{voc_14}{RGB}{192, 0, 128}
\definecolor{voc_15}{RGB}{64, 128, 128}
\definecolor{voc_16}{RGB}{192, 128, 128}
\definecolor{voc_17}{RGB}{0, 64, 0}
\definecolor{voc_18}{RGB}{128, 64, 0}
\definecolor{voc_19}{RGB}{0, 192, 0}
\definecolor{voc_20}{RGB}{128, 192, 0}
\definecolor{voc_21}{RGB}{0, 64, 128}
\definecolor{voc_22}{RGB}{128, 64, 128}

\begin{figure*}[!ht]
  \small
  \centering
  \fcolorbox{white}{voc_1}{\rule{0pt}{4pt}\rule{4pt}{0pt}} Background~~
  \fcolorbox{white}{voc_2}{\rule{0pt}{4pt}\rule{4pt}{0pt}} Aeroplane~~
  \fcolorbox{white}{voc_3}{\rule{0pt}{4pt}\rule{4pt}{0pt}} Bicycle~~
  \fcolorbox{white}{voc_4}{\rule{0pt}{4pt}\rule{4pt}{0pt}} Bird~~
  \fcolorbox{white}{voc_5}{\rule{0pt}{4pt}\rule{4pt}{0pt}} Boat~~
  \fcolorbox{white}{voc_6}{\rule{0pt}{4pt}\rule{4pt}{0pt}} Bottle~~
  \fcolorbox{white}{voc_7}{\rule{0pt}{4pt}\rule{4pt}{0pt}} Bus~~
  \fcolorbox{white}{voc_8}{\rule{0pt}{4pt}\rule{4pt}{0pt}} Car~~\\
  \fcolorbox{white}{voc_9}{\rule{0pt}{4pt}\rule{4pt}{0pt}} Cat~~
  \fcolorbox{white}{voc_10}{\rule{0pt}{4pt}\rule{4pt}{0pt}} Chair~~
  \fcolorbox{white}{voc_11}{\rule{0pt}{4pt}\rule{4pt}{0pt}} Cow~~
  \fcolorbox{white}{voc_12}{\rule{0pt}{4pt}\rule{4pt}{0pt}} Dining Table~~
  \fcolorbox{white}{voc_13}{\rule{0pt}{4pt}\rule{4pt}{0pt}} Dog~~
  \fcolorbox{white}{voc_14}{\rule{0pt}{4pt}\rule{4pt}{0pt}} Horse~~
  \fcolorbox{white}{voc_15}{\rule{0pt}{4pt}\rule{4pt}{0pt}} Motorbike~~
  \fcolorbox{white}{voc_16}{\rule{0pt}{4pt}\rule{4pt}{0pt}} Person~~\\
  \fcolorbox{white}{voc_17}{\rule{0pt}{4pt}\rule{4pt}{0pt}} Potted Plant~~
  \fcolorbox{white}{voc_18}{\rule{0pt}{4pt}\rule{4pt}{0pt}} Sheep~~
  \fcolorbox{white}{voc_19}{\rule{0pt}{4pt}\rule{4pt}{0pt}} Sofa~~
  \fcolorbox{white}{voc_20}{\rule{0pt}{4pt}\rule{4pt}{0pt}} Train~~
  \fcolorbox{white}{voc_21}{\rule{0pt}{4pt}\rule{4pt}{0pt}} TV monitor~~\\


  \subfigure{%
    \includegraphics[width=.15\columnwidth]{figures/supplementary/2008_001308_given.png}
  }
  \subfigure{%
    \includegraphics[width=.15\columnwidth]{figures/supplementary/2008_001308_sp.png}
  }
  \subfigure{%
    \includegraphics[width=.15\columnwidth]{figures/supplementary/2008_001308_gt.png}
  }
  \subfigure{%
    \includegraphics[width=.15\columnwidth]{figures/supplementary/2008_001308_cnn.png}
  }
  \subfigure{%
    \includegraphics[width=.15\columnwidth]{figures/supplementary/2008_001308_crf.png}
  }
  \subfigure{%
    \includegraphics[width=.15\columnwidth]{figures/supplementary/2008_001308_ours.png}
  }\\[-2ex]


  \subfigure{%
    \includegraphics[width=.15\columnwidth]{figures/supplementary/2008_001821_given.png}
  }
  \subfigure{%
    \includegraphics[width=.15\columnwidth]{figures/supplementary/2008_001821_sp.png}
  }
  \subfigure{%
    \includegraphics[width=.15\columnwidth]{figures/supplementary/2008_001821_gt.png}
  }
  \subfigure{%
    \includegraphics[width=.15\columnwidth]{figures/supplementary/2008_001821_cnn.png}
  }
  \subfigure{%
    \includegraphics[width=.15\columnwidth]{figures/supplementary/2008_001821_crf.png}
  }
  \subfigure{%
    \includegraphics[width=.15\columnwidth]{figures/supplementary/2008_001821_ours.png}
  }\\[-2ex]



  \subfigure{%
    \includegraphics[width=.15\columnwidth]{figures/supplementary/2008_004612_given.png}
  }
  \subfigure{%
    \includegraphics[width=.15\columnwidth]{figures/supplementary/2008_004612_sp.png}
  }
  \subfigure{%
    \includegraphics[width=.15\columnwidth]{figures/supplementary/2008_004612_gt.png}
  }
  \subfigure{%
    \includegraphics[width=.15\columnwidth]{figures/supplementary/2008_004612_cnn.png}
  }
  \subfigure{%
    \includegraphics[width=.15\columnwidth]{figures/supplementary/2008_004612_crf.png}
  }
  \subfigure{%
    \includegraphics[width=.15\columnwidth]{figures/supplementary/2008_004612_ours.png}
  }\\[-2ex]


  \subfigure{%
    \includegraphics[width=.15\columnwidth]{figures/supplementary/2009_001008_given.png}
  }
  \subfigure{%
    \includegraphics[width=.15\columnwidth]{figures/supplementary/2009_001008_sp.png}
  }
  \subfigure{%
    \includegraphics[width=.15\columnwidth]{figures/supplementary/2009_001008_gt.png}
  }
  \subfigure{%
    \includegraphics[width=.15\columnwidth]{figures/supplementary/2009_001008_cnn.png}
  }
  \subfigure{%
    \includegraphics[width=.15\columnwidth]{figures/supplementary/2009_001008_crf.png}
  }
  \subfigure{%
    \includegraphics[width=.15\columnwidth]{figures/supplementary/2009_001008_ours.png}
  }\\[-2ex]




  \subfigure{%
    \includegraphics[width=.15\columnwidth]{figures/supplementary/2009_004497_given.png}
  }
  \subfigure{%
    \includegraphics[width=.15\columnwidth]{figures/supplementary/2009_004497_sp.png}
  }
  \subfigure{%
    \includegraphics[width=.15\columnwidth]{figures/supplementary/2009_004497_gt.png}
  }
  \subfigure{%
    \includegraphics[width=.15\columnwidth]{figures/supplementary/2009_004497_cnn.png}
  }
  \subfigure{%
    \includegraphics[width=.15\columnwidth]{figures/supplementary/2009_004497_crf.png}
  }
  \subfigure{%
    \includegraphics[width=.15\columnwidth]{figures/supplementary/2009_004497_ours.png}
  }\\[-2ex]



  \setcounter{subfigure}{0}
  \subfigure[\scriptsize Input]{%
    \includegraphics[width=.15\columnwidth]{figures/supplementary/2010_001327_given.png}
  }
  \subfigure[\scriptsize Superpixels]{%
    \includegraphics[width=.15\columnwidth]{figures/supplementary/2010_001327_sp.png}
  }
  \subfigure[\scriptsize GT]{%
    \includegraphics[width=.15\columnwidth]{figures/supplementary/2010_001327_gt.png}
  }
  \subfigure[\scriptsize Deeplab]{%
    \includegraphics[width=.15\columnwidth]{figures/supplementary/2010_001327_cnn.png}
  }
  \subfigure[\scriptsize +DenseCRF]{%
    \includegraphics[width=.15\columnwidth]{figures/supplementary/2010_001327_crf.png}
  }
  \subfigure[\scriptsize Using BI]{%
    \includegraphics[width=.15\columnwidth]{figures/supplementary/2010_001327_ours.png}
  }
  \mycaption{Semantic Segmentation}{Example results of semantic segmentation
  on the Pascal VOC12 dataset.
  (d)~depicts the DeepLab CNN result, (e)~CNN + 10 steps of mean-field inference,
  (f~result obtained with bilateral inception (BI) modules (\bi{6}{2}+\bi{7}{6}) between \fc~layers.}
  \label{fig:semantic_visuals-app}
\end{figure*}


\definecolor{minc_1}{HTML}{771111}
\definecolor{minc_2}{HTML}{CAC690}
\definecolor{minc_3}{HTML}{EEEEEE}
\definecolor{minc_4}{HTML}{7C8FA6}
\definecolor{minc_5}{HTML}{597D31}
\definecolor{minc_6}{HTML}{104410}
\definecolor{minc_7}{HTML}{BB819C}
\definecolor{minc_8}{HTML}{D0CE48}
\definecolor{minc_9}{HTML}{622745}
\definecolor{minc_10}{HTML}{666666}
\definecolor{minc_11}{HTML}{D54A31}
\definecolor{minc_12}{HTML}{101044}
\definecolor{minc_13}{HTML}{444126}
\definecolor{minc_14}{HTML}{75D646}
\definecolor{minc_15}{HTML}{DD4348}
\definecolor{minc_16}{HTML}{5C8577}
\definecolor{minc_17}{HTML}{C78472}
\definecolor{minc_18}{HTML}{75D6D0}
\definecolor{minc_19}{HTML}{5B4586}
\definecolor{minc_20}{HTML}{C04393}
\definecolor{minc_21}{HTML}{D69948}
\definecolor{minc_22}{HTML}{7370D8}
\definecolor{minc_23}{HTML}{7A3622}
\definecolor{minc_24}{HTML}{000000}

\begin{figure*}[!ht]
  \small % scriptsize
  \centering
  \fcolorbox{white}{minc_1}{\rule{0pt}{4pt}\rule{4pt}{0pt}} Brick~~
  \fcolorbox{white}{minc_2}{\rule{0pt}{4pt}\rule{4pt}{0pt}} Carpet~~
  \fcolorbox{white}{minc_3}{\rule{0pt}{4pt}\rule{4pt}{0pt}} Ceramic~~
  \fcolorbox{white}{minc_4}{\rule{0pt}{4pt}\rule{4pt}{0pt}} Fabric~~
  \fcolorbox{white}{minc_5}{\rule{0pt}{4pt}\rule{4pt}{0pt}} Foliage~~
  \fcolorbox{white}{minc_6}{\rule{0pt}{4pt}\rule{4pt}{0pt}} Food~~
  \fcolorbox{white}{minc_7}{\rule{0pt}{4pt}\rule{4pt}{0pt}} Glass~~
  \fcolorbox{white}{minc_8}{\rule{0pt}{4pt}\rule{4pt}{0pt}} Hair~~\\
  \fcolorbox{white}{minc_9}{\rule{0pt}{4pt}\rule{4pt}{0pt}} Leather~~
  \fcolorbox{white}{minc_10}{\rule{0pt}{4pt}\rule{4pt}{0pt}} Metal~~
  \fcolorbox{white}{minc_11}{\rule{0pt}{4pt}\rule{4pt}{0pt}} Mirror~~
  \fcolorbox{white}{minc_12}{\rule{0pt}{4pt}\rule{4pt}{0pt}} Other~~
  \fcolorbox{white}{minc_13}{\rule{0pt}{4pt}\rule{4pt}{0pt}} Painted~~
  \fcolorbox{white}{minc_14}{\rule{0pt}{4pt}\rule{4pt}{0pt}} Paper~~
  \fcolorbox{white}{minc_15}{\rule{0pt}{4pt}\rule{4pt}{0pt}} Plastic~~\\
  \fcolorbox{white}{minc_16}{\rule{0pt}{4pt}\rule{4pt}{0pt}} Polished Stone~~
  \fcolorbox{white}{minc_17}{\rule{0pt}{4pt}\rule{4pt}{0pt}} Skin~~
  \fcolorbox{white}{minc_18}{\rule{0pt}{4pt}\rule{4pt}{0pt}} Sky~~
  \fcolorbox{white}{minc_19}{\rule{0pt}{4pt}\rule{4pt}{0pt}} Stone~~
  \fcolorbox{white}{minc_20}{\rule{0pt}{4pt}\rule{4pt}{0pt}} Tile~~
  \fcolorbox{white}{minc_21}{\rule{0pt}{4pt}\rule{4pt}{0pt}} Wallpaper~~
  \fcolorbox{white}{minc_22}{\rule{0pt}{4pt}\rule{4pt}{0pt}} Water~~
  \fcolorbox{white}{minc_23}{\rule{0pt}{4pt}\rule{4pt}{0pt}} Wood~~\\
  \subfigure{%
    \includegraphics[width=.15\columnwidth]{figures/supplementary/000008468_given.png}
  }
  \subfigure{%
    \includegraphics[width=.15\columnwidth]{figures/supplementary/000008468_sp.png}
  }
  \subfigure{%
    \includegraphics[width=.15\columnwidth]{figures/supplementary/000008468_gt.png}
  }
  \subfigure{%
    \includegraphics[width=.15\columnwidth]{figures/supplementary/000008468_cnn.png}
  }
  \subfigure{%
    \includegraphics[width=.15\columnwidth]{figures/supplementary/000008468_crf.png}
  }
  \subfigure{%
    \includegraphics[width=.15\columnwidth]{figures/supplementary/000008468_ours.png}
  }\\[-2ex]

  \subfigure{%
    \includegraphics[width=.15\columnwidth]{figures/supplementary/000009053_given.png}
  }
  \subfigure{%
    \includegraphics[width=.15\columnwidth]{figures/supplementary/000009053_sp.png}
  }
  \subfigure{%
    \includegraphics[width=.15\columnwidth]{figures/supplementary/000009053_gt.png}
  }
  \subfigure{%
    \includegraphics[width=.15\columnwidth]{figures/supplementary/000009053_cnn.png}
  }
  \subfigure{%
    \includegraphics[width=.15\columnwidth]{figures/supplementary/000009053_crf.png}
  }
  \subfigure{%
    \includegraphics[width=.15\columnwidth]{figures/supplementary/000009053_ours.png}
  }\\[-2ex]




  \subfigure{%
    \includegraphics[width=.15\columnwidth]{figures/supplementary/000014977_given.png}
  }
  \subfigure{%
    \includegraphics[width=.15\columnwidth]{figures/supplementary/000014977_sp.png}
  }
  \subfigure{%
    \includegraphics[width=.15\columnwidth]{figures/supplementary/000014977_gt.png}
  }
  \subfigure{%
    \includegraphics[width=.15\columnwidth]{figures/supplementary/000014977_cnn.png}
  }
  \subfigure{%
    \includegraphics[width=.15\columnwidth]{figures/supplementary/000014977_crf.png}
  }
  \subfigure{%
    \includegraphics[width=.15\columnwidth]{figures/supplementary/000014977_ours.png}
  }\\[-2ex]


  \subfigure{%
    \includegraphics[width=.15\columnwidth]{figures/supplementary/000022922_given.png}
  }
  \subfigure{%
    \includegraphics[width=.15\columnwidth]{figures/supplementary/000022922_sp.png}
  }
  \subfigure{%
    \includegraphics[width=.15\columnwidth]{figures/supplementary/000022922_gt.png}
  }
  \subfigure{%
    \includegraphics[width=.15\columnwidth]{figures/supplementary/000022922_cnn.png}
  }
  \subfigure{%
    \includegraphics[width=.15\columnwidth]{figures/supplementary/000022922_crf.png}
  }
  \subfigure{%
    \includegraphics[width=.15\columnwidth]{figures/supplementary/000022922_ours.png}
  }\\[-2ex]


  \subfigure{%
    \includegraphics[width=.15\columnwidth]{figures/supplementary/000025711_given.png}
  }
  \subfigure{%
    \includegraphics[width=.15\columnwidth]{figures/supplementary/000025711_sp.png}
  }
  \subfigure{%
    \includegraphics[width=.15\columnwidth]{figures/supplementary/000025711_gt.png}
  }
  \subfigure{%
    \includegraphics[width=.15\columnwidth]{figures/supplementary/000025711_cnn.png}
  }
  \subfigure{%
    \includegraphics[width=.15\columnwidth]{figures/supplementary/000025711_crf.png}
  }
  \subfigure{%
    \includegraphics[width=.15\columnwidth]{figures/supplementary/000025711_ours.png}
  }\\[-2ex]


  \subfigure{%
    \includegraphics[width=.15\columnwidth]{figures/supplementary/000034473_given.png}
  }
  \subfigure{%
    \includegraphics[width=.15\columnwidth]{figures/supplementary/000034473_sp.png}
  }
  \subfigure{%
    \includegraphics[width=.15\columnwidth]{figures/supplementary/000034473_gt.png}
  }
  \subfigure{%
    \includegraphics[width=.15\columnwidth]{figures/supplementary/000034473_cnn.png}
  }
  \subfigure{%
    \includegraphics[width=.15\columnwidth]{figures/supplementary/000034473_crf.png}
  }
  \subfigure{%
    \includegraphics[width=.15\columnwidth]{figures/supplementary/000034473_ours.png}
  }\\[-2ex]


  \subfigure{%
    \includegraphics[width=.15\columnwidth]{figures/supplementary/000035463_given.png}
  }
  \subfigure{%
    \includegraphics[width=.15\columnwidth]{figures/supplementary/000035463_sp.png}
  }
  \subfigure{%
    \includegraphics[width=.15\columnwidth]{figures/supplementary/000035463_gt.png}
  }
  \subfigure{%
    \includegraphics[width=.15\columnwidth]{figures/supplementary/000035463_cnn.png}
  }
  \subfigure{%
    \includegraphics[width=.15\columnwidth]{figures/supplementary/000035463_crf.png}
  }
  \subfigure{%
    \includegraphics[width=.15\columnwidth]{figures/supplementary/000035463_ours.png}
  }\\[-2ex]


  \setcounter{subfigure}{0}
  \subfigure[\scriptsize Input]{%
    \includegraphics[width=.15\columnwidth]{figures/supplementary/000035993_given.png}
  }
  \subfigure[\scriptsize Superpixels]{%
    \includegraphics[width=.15\columnwidth]{figures/supplementary/000035993_sp.png}
  }
  \subfigure[\scriptsize GT]{%
    \includegraphics[width=.15\columnwidth]{figures/supplementary/000035993_gt.png}
  }
  \subfigure[\scriptsize AlexNet]{%
    \includegraphics[width=.15\columnwidth]{figures/supplementary/000035993_cnn.png}
  }
  \subfigure[\scriptsize +DenseCRF]{%
    \includegraphics[width=.15\columnwidth]{figures/supplementary/000035993_crf.png}
  }
  \subfigure[\scriptsize Using BI]{%
    \includegraphics[width=.15\columnwidth]{figures/supplementary/000035993_ours.png}
  }
  \mycaption{Material Segmentation}{Example results of material segmentation.
  (d)~depicts the AlexNet CNN result, (e)~CNN + 10 steps of mean-field inference,
  (f)~result obtained with bilateral inception (BI) modules (\bi{7}{2}+\bi{8}{6}) between
  \fc~layers.}
\label{fig:material_visuals-app}
\end{figure*}


\definecolor{city_1}{RGB}{128, 64, 128}
\definecolor{city_2}{RGB}{244, 35, 232}
\definecolor{city_3}{RGB}{70, 70, 70}
\definecolor{city_4}{RGB}{102, 102, 156}
\definecolor{city_5}{RGB}{190, 153, 153}
\definecolor{city_6}{RGB}{153, 153, 153}
\definecolor{city_7}{RGB}{250, 170, 30}
\definecolor{city_8}{RGB}{220, 220, 0}
\definecolor{city_9}{RGB}{107, 142, 35}
\definecolor{city_10}{RGB}{152, 251, 152}
\definecolor{city_11}{RGB}{70, 130, 180}
\definecolor{city_12}{RGB}{220, 20, 60}
\definecolor{city_13}{RGB}{255, 0, 0}
\definecolor{city_14}{RGB}{0, 0, 142}
\definecolor{city_15}{RGB}{0, 0, 70}
\definecolor{city_16}{RGB}{0, 60, 100}
\definecolor{city_17}{RGB}{0, 80, 100}
\definecolor{city_18}{RGB}{0, 0, 230}
\definecolor{city_19}{RGB}{119, 11, 32}
\begin{figure*}[!ht]
  \small % scriptsize
  \centering


  \subfigure{%
    \includegraphics[width=.18\columnwidth]{figures/supplementary/frankfurt00000_016005_given.png}
  }
  \subfigure{%
    \includegraphics[width=.18\columnwidth]{figures/supplementary/frankfurt00000_016005_sp.png}
  }
  \subfigure{%
    \includegraphics[width=.18\columnwidth]{figures/supplementary/frankfurt00000_016005_gt.png}
  }
  \subfigure{%
    \includegraphics[width=.18\columnwidth]{figures/supplementary/frankfurt00000_016005_cnn.png}
  }
  \subfigure{%
    \includegraphics[width=.18\columnwidth]{figures/supplementary/frankfurt00000_016005_ours.png}
  }\\[-2ex]

  \subfigure{%
    \includegraphics[width=.18\columnwidth]{figures/supplementary/frankfurt00000_004617_given.png}
  }
  \subfigure{%
    \includegraphics[width=.18\columnwidth]{figures/supplementary/frankfurt00000_004617_sp.png}
  }
  \subfigure{%
    \includegraphics[width=.18\columnwidth]{figures/supplementary/frankfurt00000_004617_gt.png}
  }
  \subfigure{%
    \includegraphics[width=.18\columnwidth]{figures/supplementary/frankfurt00000_004617_cnn.png}
  }
  \subfigure{%
    \includegraphics[width=.18\columnwidth]{figures/supplementary/frankfurt00000_004617_ours.png}
  }\\[-2ex]

  \subfigure{%
    \includegraphics[width=.18\columnwidth]{figures/supplementary/frankfurt00000_020880_given.png}
  }
  \subfigure{%
    \includegraphics[width=.18\columnwidth]{figures/supplementary/frankfurt00000_020880_sp.png}
  }
  \subfigure{%
    \includegraphics[width=.18\columnwidth]{figures/supplementary/frankfurt00000_020880_gt.png}
  }
  \subfigure{%
    \includegraphics[width=.18\columnwidth]{figures/supplementary/frankfurt00000_020880_cnn.png}
  }
  \subfigure{%
    \includegraphics[width=.18\columnwidth]{figures/supplementary/frankfurt00000_020880_ours.png}
  }\\[-2ex]



  \subfigure{%
    \includegraphics[width=.18\columnwidth]{figures/supplementary/frankfurt00001_007285_given.png}
  }
  \subfigure{%
    \includegraphics[width=.18\columnwidth]{figures/supplementary/frankfurt00001_007285_sp.png}
  }
  \subfigure{%
    \includegraphics[width=.18\columnwidth]{figures/supplementary/frankfurt00001_007285_gt.png}
  }
  \subfigure{%
    \includegraphics[width=.18\columnwidth]{figures/supplementary/frankfurt00001_007285_cnn.png}
  }
  \subfigure{%
    \includegraphics[width=.18\columnwidth]{figures/supplementary/frankfurt00001_007285_ours.png}
  }\\[-2ex]


  \subfigure{%
    \includegraphics[width=.18\columnwidth]{figures/supplementary/frankfurt00001_059789_given.png}
  }
  \subfigure{%
    \includegraphics[width=.18\columnwidth]{figures/supplementary/frankfurt00001_059789_sp.png}
  }
  \subfigure{%
    \includegraphics[width=.18\columnwidth]{figures/supplementary/frankfurt00001_059789_gt.png}
  }
  \subfigure{%
    \includegraphics[width=.18\columnwidth]{figures/supplementary/frankfurt00001_059789_cnn.png}
  }
  \subfigure{%
    \includegraphics[width=.18\columnwidth]{figures/supplementary/frankfurt00001_059789_ours.png}
  }\\[-2ex]


  \subfigure{%
    \includegraphics[width=.18\columnwidth]{figures/supplementary/frankfurt00001_068208_given.png}
  }
  \subfigure{%
    \includegraphics[width=.18\columnwidth]{figures/supplementary/frankfurt00001_068208_sp.png}
  }
  \subfigure{%
    \includegraphics[width=.18\columnwidth]{figures/supplementary/frankfurt00001_068208_gt.png}
  }
  \subfigure{%
    \includegraphics[width=.18\columnwidth]{figures/supplementary/frankfurt00001_068208_cnn.png}
  }
  \subfigure{%
    \includegraphics[width=.18\columnwidth]{figures/supplementary/frankfurt00001_068208_ours.png}
  }\\[-2ex]

  \subfigure{%
    \includegraphics[width=.18\columnwidth]{figures/supplementary/frankfurt00001_082466_given.png}
  }
  \subfigure{%
    \includegraphics[width=.18\columnwidth]{figures/supplementary/frankfurt00001_082466_sp.png}
  }
  \subfigure{%
    \includegraphics[width=.18\columnwidth]{figures/supplementary/frankfurt00001_082466_gt.png}
  }
  \subfigure{%
    \includegraphics[width=.18\columnwidth]{figures/supplementary/frankfurt00001_082466_cnn.png}
  }
  \subfigure{%
    \includegraphics[width=.18\columnwidth]{figures/supplementary/frankfurt00001_082466_ours.png}
  }\\[-2ex]

  \subfigure{%
    \includegraphics[width=.18\columnwidth]{figures/supplementary/lindau00033_000019_given.png}
  }
  \subfigure{%
    \includegraphics[width=.18\columnwidth]{figures/supplementary/lindau00033_000019_sp.png}
  }
  \subfigure{%
    \includegraphics[width=.18\columnwidth]{figures/supplementary/lindau00033_000019_gt.png}
  }
  \subfigure{%
    \includegraphics[width=.18\columnwidth]{figures/supplementary/lindau00033_000019_cnn.png}
  }
  \subfigure{%
    \includegraphics[width=.18\columnwidth]{figures/supplementary/lindau00033_000019_ours.png}
  }\\[-2ex]

  \subfigure{%
    \includegraphics[width=.18\columnwidth]{figures/supplementary/lindau00052_000019_given.png}
  }
  \subfigure{%
    \includegraphics[width=.18\columnwidth]{figures/supplementary/lindau00052_000019_sp.png}
  }
  \subfigure{%
    \includegraphics[width=.18\columnwidth]{figures/supplementary/lindau00052_000019_gt.png}
  }
  \subfigure{%
    \includegraphics[width=.18\columnwidth]{figures/supplementary/lindau00052_000019_cnn.png}
  }
  \subfigure{%
    \includegraphics[width=.18\columnwidth]{figures/supplementary/lindau00052_000019_ours.png}
  }\\[-2ex]




  \subfigure{%
    \includegraphics[width=.18\columnwidth]{figures/supplementary/lindau00027_000019_given.png}
  }
  \subfigure{%
    \includegraphics[width=.18\columnwidth]{figures/supplementary/lindau00027_000019_sp.png}
  }
  \subfigure{%
    \includegraphics[width=.18\columnwidth]{figures/supplementary/lindau00027_000019_gt.png}
  }
  \subfigure{%
    \includegraphics[width=.18\columnwidth]{figures/supplementary/lindau00027_000019_cnn.png}
  }
  \subfigure{%
    \includegraphics[width=.18\columnwidth]{figures/supplementary/lindau00027_000019_ours.png}
  }\\[-2ex]



  \setcounter{subfigure}{0}
  \subfigure[\scriptsize Input]{%
    \includegraphics[width=.18\columnwidth]{figures/supplementary/lindau00029_000019_given.png}
  }
  \subfigure[\scriptsize Superpixels]{%
    \includegraphics[width=.18\columnwidth]{figures/supplementary/lindau00029_000019_sp.png}
  }
  \subfigure[\scriptsize GT]{%
    \includegraphics[width=.18\columnwidth]{figures/supplementary/lindau00029_000019_gt.png}
  }
  \subfigure[\scriptsize Deeplab]{%
    \includegraphics[width=.18\columnwidth]{figures/supplementary/lindau00029_000019_cnn.png}
  }
  \subfigure[\scriptsize Using BI]{%
    \includegraphics[width=.18\columnwidth]{figures/supplementary/lindau00029_000019_ours.png}
  }%\\[-2ex]

  \mycaption{Street Scene Segmentation}{Example results of street scene segmentation.
  (d)~depicts the DeepLab results, (e)~result obtained by adding bilateral inception (BI) modules (\bi{6}{2}+\bi{7}{6}) between \fc~layers.}
\label{fig:street_visuals-app}
\end{figure*}


\bibliographystyle{IEEEtran}
\bibliography{mybib}


\end{document}
