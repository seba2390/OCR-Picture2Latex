%% bare_jrnl.tex
%% V1.4b
%% 2015/08/26
%% by Michael Shell
%% see http://www.michaelshell.org/
%% for current contact information.
%%
%% This is a skeleton file demonstrating the use of IEEEtran.cls
%% (requires IEEEtran.cls version 1.8b or later) with an IEEE
%% journal paper.
%%
%% Support sites:
%% http://www.michaelshell.org/tex/ieeetran/
%% http://www.ctan.org/pkg/ieeetran
%% and
%% http://www.ieee.org/

%%*************************************************************************
%% Legal Notice:
%% This code is offered as-is without any warranty either expressed or
%% implied; without even the implied warranty of MERCHANTABILITY or
%% FITNESS FOR A PARTICULAR PURPOSE! 
%% User assumes all risk.
%% In no event shall the IEEE or any contributor to this code be liable for
%% any damages or losses, including, but not limited to, incidental,
%% consequential, or any other damages, resulting from the use or misuse
%% of any information contained here.
%%
%% All comments are the opinions of their respective authors and are not
%% necessarily endorsed by the IEEE.
%%
%% This work is distributed under the LaTeX Project Public License (LPPL)
%% ( http://www.latex-project.org/ ) version 1.3, and may be freely used,
%% distributed and modified. A copy of the LPPL, version 1.3, is included
%% in the base LaTeX documentation of all distributions of LaTeX released
%% 2003/12/01 or later.
%% Retain all contribution notices and credits.
%% ** Modified files should be clearly indicated as such,including  **
%% ** renaming them and changing author support contact information. **
%%*************************************************************************


% *** Authors should verify (and, if needed, correct) their LaTeX system  ***
% *** with the testflow diagnostic prior to trusting their LaTeX platform ***
% *** with production work. The IEEE's font choices and paper sizes can   ***
% *** trigger bugs that do not appear when using other class files.       ***                          ***
% The testflow support page is at:
% http://www.michaelshell.org/tex/testflow/


%\documentclass[journal,comsoc,onecolumn,12pt]{IEEEtran}
%\documentclass[journal,onecolumn,12pt]{IEEEtran}
\documentclass[journal,draftclsnofoot,onecolumn,12pt]{IEEEtran}
% \documentclass[journal]{IEEEtran}
% If IEEEtran.cls has not been installed into the LaTeX system files,
% manually specify the path to it like:
% \documentclass[journal]{../sty/IEEEtran}





% Some very useful LaTeX packages include:
% (uncomment the ones you want to load)


% *** MISC UTILITY PACKAGES ***
%
%\usepackage{ifpdf}
% Heiko Oberdiek's ifpdf.sty is very useful if you need conditional
% compilation based on whether the output is pdf or dvi.
% usage:
% \ifpdf
%   % pdf code
% \else
%   % dvi code
% \fi
% The latest version of ifpdf.sty can be obtained from:
% http://www.ctan.org/pkg/ifpdf
% Also, note that IEEEtran.cls V1.7 and later provides a builtin
% \ifCLASSINFOpdf conditional that works the same way.
% When switching from latex to pdflatex and vice-versa, the compiler may
% have to be run twice to clear warning/error messages.






% *** CITATION PACKAGES ***
%
%\usepackage{cite}
% cite.sty was written by Donald Arseneau
% V1.6 and later of IEEEtran pre-defines the format of the cite.sty package
% \cite{} output to follow that of the IEEE. Loading the cite package will
% result in citation numbers being automatically sorted and properly
% "compressed/ranged". e.g., [1], [9], [2], [7], [5], [6] without using
% cite.sty will become [1], [2], [5]--[7], [9] using cite.sty. cite.sty's
% \cite will automatically add leading space, if needed. Use cite.sty's
% noadjust option (cite.sty V3.8 and later) if you want to turn this off
% such as if a citation ever needs to be enclosed in parenthesis.
% cite.sty is already installed on most LaTeX systems. Be sure and use
% version 5.0 (2009-03-20) and later if using hyperref.sty.
% The latest version can be obtained at:
% http://www.ctan.org/pkg/cite
% The documentation is contained in the cite.sty file itself.






% *** GRAPHICS RELATED PACKAGES ***
%
\ifCLASSINFOpdf
  % \usepackage[pdftex]{graphicx}
  % declare the path(s) where your graphic files are
  % \graphicspath{{../pdf/}{../jpeg/}}
  % and their extensions so you won't have to specify these with
  % every instance of \includegraphics
  % \DeclareGraphicsExtensions{.pdf,.jpeg,.png}
\else
  % or other class option (dvipsone, dvipdf, if not using dvips). graphicx
  % will default to the driver specified in the system graphics.cfg if no
  % driver is specified.
  % \usepackage[dvips]{graphicx}
  % declare the path(s) where your graphic files are
  % \graphicspath{{../eps/}}
  % and their extensions so you won't have to specify these with
  % every instance of \includegraphics
  % \DeclareGraphicsExtensions{.eps}
\fi
% graphicx was written by David Carlisle and Sebastian Rahtz. It is
% required if you want graphics, photos, etc. graphicx.sty is already
% installed on most LaTeX systems. The latest version and documentation
% can be obtained at: 
% http://www.ctan.org/pkg/graphicx
% Another good source of documentation is "Using Imported Graphics in
% LaTeX2e" by Keith Reckdahl which can be found at:
% http://www.ctan.org/pkg/epslatex
%
% latex, and pdflatex in dvi mode, support graphics in encapsulated
% postscript (.eps) format. pdflatex in pdf mode supports graphics
% in .pdf, .jpeg, .png and .mps (metapost) formats. Users should ensure
% that all non-photo figures use a vector format (.eps, .pdf, .mps) and
% not a bitmapped formats (.jpeg, .png). The IEEE frowns on bitmapped formats
% which can result in "jaggedy"/blurry rendering of lines and letters as
% well as large increases in file sizes.
%
% You can find documentation about the pdfTeX application at:
% http://www.tug.org/applications/pdftex





% *** MATH PACKAGES ***
%
%\usepackage{amsmath}
% A popular package from the American Mathematical Society that provides
% many useful and powerful commands for dealing with mathematics.
%
% Note that the amsmath package sets \interdisplaylinepenalty to 10000
% thus preventing page breaks from occurring within multiline equations. Use:
%\interdisplaylinepenalty=2500
% after loading amsmath to restore such page breaks as IEEEtran.cls normally
% does. amsmath.sty is already installed on most LaTeX systems. The latest
% version and documentation can be obtained at:
% http://www.ctan.org/pkg/amsmath





% *** SPECIALIZED LIST PACKAGES ***
%
%\usepackage{algorithmic}
% algorithmic.sty was written by Peter Williams and Rogerio Brito.
% This package provides an algorithmic environment fo describing algorithms.
% You can use the algorithmic environment in-text or within a figure
% environment to provide for a floating algorithm. Do NOT use the algorithm
% floating environment provided by algorithm.sty (by the same authors) or
% algorithm2e.sty (by Christophe Fiorio) as the IEEE does not use dedicated
% algorithm float types and packages that provide these will not provide
% correct IEEE style captions. The latest version and documentation of
% algorithmic.sty can be obtained at:
% http://www.ctan.org/pkg/algorithms
% Also of interest may be the (relatively newer and more customizable)
% algorithmicx.sty package by Szasz Janos:
% http://www.ctan.org/pkg/algorithmicx




% *** ALIGNMENT PACKAGES ***
%
%\usepackage{array}
% Frank Mittelbach's and David Carlisle's array.sty patches and improves
% the standard LaTeX2e array and tabular environments to provide better
% appearance and additional user controls. As the default LaTeX2e table
% generation code is lacking to the point of almost being broken with
% respect to the quality of the end results, all users are strongly
% advised to use an enhanced (at the very least that provided by array.sty)
% set of table tools. array.sty is already installed on most systems. The
% latest version and documentation can be obtained at:
% http://www.ctan.org/pkg/array


% IEEEtran contains the IEEEeqnarray family of commands that can be used to
% generate multiline equations as well as matrices, tables, etc., of high
% quality.




% *** SUBFIGURE PACKAGES ***
%\ifCLASSOPTIONcompsoc
%  \usepackage[caption=false,font=normalsize,labelfont=sf,textfont=sf]{subfig}
%\else
%  \usepackage[caption=false,font=footnotesize]{subfig}
%\fi
% subfig.sty, written by Steven Douglas Cochran, is the modern replacement
% for subfigure.sty, the latter of which is no longer maintained and is
% incompatible with some LaTeX packages including fixltx2e. However,
% subfig.sty requires and automatically loads Axel Sommerfeldt's caption.sty
% which will override IEEEtran.cls' handling of captions and this will result
% in non-IEEE style figure/table captions. To prevent this problem, be sure
% and invoke subfig.sty's "caption=false" package option (available since
% subfig.sty version 1.3, 2005/06/28) as this is will preserve IEEEtran.cls
% handling of captions.
% Note that the Computer Society format requires a larger sans serif font
% than the serif footnote size font used in traditional IEEE formatting
% and thus the need to invoke different subfig.sty package options depending
% on whether compsoc mode has been enabled.
%
% The latest version and documentation of subfig.sty can be obtained at:
% http://www.ctan.org/pkg/subfig




% *** FLOAT PACKAGES ***
%
%\usepackage{fixltx2e}
% fixltx2e, the successor to the earlier fix2col.sty, was written by
% Frank Mittelbach and David Carlisle. This package corrects a few problems
% in the LaTeX2e kernel, the most notable of which is that in current
% LaTeX2e releases, the ordering of single and double column floats is not
% guaranteed to be preserved. Thus, an unpatched LaTeX2e can allow a
% single column figure to be placed prior to an earlier double column
% figure.
% Be aware that LaTeX2e kernels dated 2015 and later have fixltx2e.sty's
% corrections already built into the system in which case a warning will
% be issued if an attempt is made to load fixltx2e.sty as it is no longer
% needed.
% The latest version and documentation can be found at:
% http://www.ctan.org/pkg/fixltx2e


%\usepackage{stfloats}
% stfloats.sty was written by Sigitas Tolusis. This package gives LaTeX2e
% the ability to do double column floats at the bottom of the page as well
% as the top. (e.g., "\begin{figure*}[!b]" is not normally possible in
% LaTeX2e). It also provides a command:
%\fnbelowfloat
% to enable the placement of footnotes below bottom floats (the standard
% LaTeX2e kernel puts them above bottom floats). This is an invasive package
% which rewrites many portions of the LaTeX2e float routines. It may not work
% with other packages that modify the LaTeX2e float routines. The latest
% version and documentation can be obtained at:
% http://www.ctan.org/pkg/stfloats
% Do not use the stfloats baselinefloat ability as the IEEE does not allow
% \baselineskip to stretch. Authors submitting work to the IEEE should note
% that the IEEE rarely uses double column equations and that authors should try
% to avoid such use. Do not be tempted to use the cuted.sty or midfloat.sty
% packages (also by Sigitas Tolusis) as the IEEE does not format its papers in
% such ways.
% Do not attempt to use stfloats with fixltx2e as they are incompatible.
% Instead, use Morten Hogholm'a dblfloatfix which combines the features
% of both fixltx2e and stfloats:
%
% \usepackage{dblfloatfix}
% The latest version can be found at:
% http://www.ctan.org/pkg/dblfloatfix




%\ifCLASSOPTIONcaptionsoff
%  \usepackage[nomarkers]{endfloat}
% \let\MYoriglatexcaption\caption
% \renewcommand{\caption}[2][\relax]{\MYoriglatexcaption[#2]{#2}}
%\fi
% endfloat.sty was written by James Darrell McCauley, Jeff Goldberg and 
% Axel Sommerfeldt. This package may be useful when used in conjunction with 
% IEEEtran.cls'  captionsoff option. Some IEEE journals/societies require that
% submissions have lists of figures/tables at the end of the paper and that
% figures/tables without any captions are placed on a page by themselves at
% the end of the document. If needed, the draftcls IEEEtran class option or
% \CLASSINPUTbaselinestretch interface can be used to increase the line
% spacing as well. Be sure and use the nomarkers option of endfloat to
% prevent endfloat from "marking" where the figures would have been placed
% in the text. The two hack lines of code above are a slight modification of
% that suggested by in the endfloat docs (section 8.4.1) to ensure that
% the full captions always appear in the list of figures/tables - even if
% the user used the short optional argument of \caption[]{}.
% IEEE papers do not typically make use of \caption[]'s optional argument,
% so this should not be an issue. A similar trick can be used to disable
% captions of packages such as subfig.sty that lack options to turn off
% the subcaptions:
% For subfig.sty:
% \let\MYorigsubfloat\subfloat
% \renewcommand{\subfloat}[2][\relax]{\MYorigsubfloat[]{#2}}
% However, the above trick will not work if both optional arguments of
% the \subfloat command are used. Furthermore, there needs to be a
% description of each subfigure *somewhere* and endfloat does not add
% subfigure captions to its list of figures. Thus, the best approach is to
% avoid the use of subfigure captions (many IEEE journals avoid them anyway)
% and instead reference/explain all the subfigures within the main caption.
% The latest version of endfloat.sty and its documentation can obtained at:
% http://www.ctan.org/pkg/endfloat
%
% The IEEEtran \ifCLASSOPTIONcaptionsoff conditional can also be used
% later in the document, say, to conditionally put the References on a 
% page by themselves.




% *** PDF, URL AND HYPERLINK PACKAGES ***
%
%\usepackage{url}
% url.sty was written by Donald Arseneau. It provides better support for
% handling and breaking URLs. url.sty is already installed on most LaTeX
% systems. The latest version and documentation can be obtained at:
% http://www.ctan.org/pkg/url
% Basically, \url{my_url_here}.




% *** Do not adjust lengths that control margins, column widths, etc. ***
% *** Do not use packages that alter fonts (such as pslatex).         ***
% There should be no need to do such things with IEEEtran.cls V1.6 and later.
% (Unless specifically asked to do so by the journal or conference you plan
% to submit to, of course. )
\usepackage{amssymb,amsmath,amsfonts,amsthm}
\usepackage{mathrsfs}
\usepackage{cite}
\usepackage{array}
\usepackage{tabularx}
\usepackage[dvipsnames]{xcolor}
\usepackage{multirow}
\usepackage{adjustbox}
\usepackage{color}
\usepackage{booktabs}
\usepackage{xcolor,colortbl}
\definecolor{lavender}{rgb}{0.9, 0.9, 0.98}
\usepackage{mathtools}
\usepackage{subfigure}
\usepackage{mathtools}
\usepackage{tikz,pgf}
\usetikzlibrary{calc,positioning,mindmap,trees,decorations.pathreplacing}
\usepackage{hyperref}
%\usepackage[dvips]{graphicx}
\usepackage{graphicx}
\usepackage{bbm}
\usepackage[utf8]{inputenc}
% correct bad hyphenation here
\hyphenation{op-tical net-works semi-conduc-tor}
\usepackage{xcolor}
\usepackage{algpseudocode}
\usepackage[linesnumbered,ruled,vlined]{algorithm2e}
\SetKwInput{KwInput}{Input}                % Set the Input
\SetKwInput{KwOutput}{Output}
\SetKwInput{KwInit}{Initialization}
\newtheorem{theorem}{Theorem}
\newtheorem{remark}{Remark}
\newtheorem{corollary}{Corollary}[theorem]

\begin{document}
%
% paper title
% Titles are generally capitalized except for words such as a, an, and, as,
% at, but, by, for, in, nor, of, on, or, the, to and up, which are usually
% not capitalized unless they are the first or last word of the title.
% Linebreaks \\ can be used within to get better formatting as desired.
% Do not put math or special symbols in the title.
\title{Statistically Optimal Beamforming and Ergodic Capacity for RIS-aided MISO  Systems}
%
%
% author names and IEEE memberships
% note positions of commas and nonbreaking spaces ( ~ ) LaTeX will not break
% a structure at a ~ so this keeps an author's name from being broken across
% two lines.
% use \thanks{} to gain access to the first footnote area
% a separate \thanks must be used for each paragraph as LaTeX2e's \thanks
% was not built to handle multiple paragraphs
%

\author{Kali Krishna Kota, M. S. S. Manasa,  %~\IEEEmembership{Student Member,~IEEE,}
        Praful D. Mankar, and Harpreet S. Dhillon% <-this % stops a space
\thanks{K. K. Kota, M. S. S. Manasa, and P. D. Mankar are with the Signal Processing and Communication Research Center, International Institute of Information Technology, Hyderabad 500032, India (e-mail: kali.kota@research.iiit.ac.in, mss.manasa@research.iiit.ac.in, praful.mankar@iiit.ac.in). H. S. Dhillon is with Wireless@VT, Department of ECE, Virginia Tech, Blacksburg, VA (Email:  hdhillon@vt.edu). This paper has been submitted in parts to IEEE GLOBECOM 2023 \cite{kota2022optimal}}}% <-this % stops a space

% note the % following the last \IEEEmembership and also \thanks - 
% these prevent an unwanted space from occurring between the last author name
% and the end of the author line. i.e., if you had this:
% 
% \author{....lastname \thanks{...} \thanks{...} }
%                     ^------------^------------^----Do not want these spaces!
%
% a space would be appended to the last name and could cause every name on that
% line to be shifted left slightly. This is one of those "LaTeX things". For
% instance, "\textbf{A} \textbf{B}" will typeset as "A B" not "AB". To get
% "AB" then you have to do: "\textbf{A}\textbf{B}"
% \thanks is no different in this regard, so shield the last } of each \thanks
% that ends a line with a % and do not let a space in before the next \thanks.
% Spaces after \IEEEmembership other than the last one are OK (and needed) as
% you are supposed to have spaces between the names. For what it is worth,
% this is a minor point as most people would not even notice if the said evil
% space somehow managed to creep in.



% The paper headers
\markboth{}%
{Shell \MakeLowercase{\textit{et al.}}: Bare Demo of IEEEtran.cls for IEEE Journals}
%Journal of \LaTeX\ Class Files,~Vol.~14, No.~8, August~2015
% The only time the second header will appear is for the odd numbered pages
% after the title page when using the twoside option.
% 
% *** Note that you probably will NOT want to include the author's ***
% *** name in the headers of peer review papers.                   ***
% You can use \ifCLASSOPTIONpeerreview for conditional compilation here if
% you desire.




% If you want to put a publisher's ID mark on the page you can do it like
% this:
%\IEEEpubid{0000--0000/00\$00.00~\copyright~2015 IEEE}
% Remember, if you use this you must call \IEEEpubidadjcol in the second
% column for its text to clear the IEEEpubid mark.



% use for special paper notices
%\IEEEspecialpapernotice{(Invited Paper)}




% make the title area
\maketitle

% As a general rule, do not put math, special symbols or citations
% in the abstract or keywords.
\vspace{-1.5cm}
\begin{abstract}
% We consider the joint transmit beamformer and phase shift matrix problem of the downlink of a MISO system aided by a reconfigurable intelligent surface (RIS) that maximizes the mean SNR. We assume the direct and the indirect link via the RIS to the user to be affected by correlated Rician fading. We adopt a sinc-based model from \cite{RIS_Corr_Fad} to capture the correlated fading along both links. We formulate the ergodic capacity maximization problem for this setup to obtain the optimal beamformer and phase shift matrix. However, the ergodic capacity maximization problem is non-convex, so we propose solving the ergodic capacity's upper bound (mean SNR) via an iterative algorithm. Further, we evaluate the obtained beamformer and the phase shift matrix by deriving outage probability and ergodic capacity expressions. Additionally, we derive computationally efficient closed-form expressions for the beamformer and phase shift matrix for special cases of the above-generalized channel model. We also provide closed-form expressions for the outage and ergodic capacity. Consequently, we also derive the diversity order and coding gains of the special cases with closed-form expressions. Finally, we present rigorous numerical analysis in the results discussion to evaluate the iterative and closed-form solutions proposed. Numerical results show that RIS provides significant gains over  - WC is 204, still need to add important info on results 
This paper focuses on optimal beamforming to maximize the mean signal-to-noise ratio (SNR) for a passive reconfigurable intelligent surface (RIS)-aided multiple-input single-output (MISO) downlink system. We consider that both the direct and indirect (through RIS) links to the user experience correlated Rician fading. 
% We adopt a sinc-based model from \cite{RIS_Corr_Fad} to capture this correlated fading. To address the non-convexity of the ergodic capacity maximization problem, we propose an iterative algorithm that focuses on maximizing the upper bound of the ergodic capacity, equivalent to maximizing the mean SNR. 
The assumption of passive RIS imposes the unit modulus constraint, which makes the beamforming problem non-convex. To tackle this issue, we apply semidefinite relaxation (SDR) for obtaining the optimal phase-shift matrix and propose an iterative algorithm to obtain the fixed-point solution for statistically optimal transmit beamforming vector and RIS-phase shift matrix. Further, to measure the performance of the proposed beamforming scheme, we analyze key system performance metrics such as outage probability (OP) and ergodic capacity (EC). 
%{\color{red}But, they are limited by numerical evaluation of the proposed iterative algorithm, which usually is the case in existing works. Such iterative solution-based analysis does not allow to understand the functional dependence of system performance on the parameters such as line-of-sight (LoS) components, correlated fading, number of reflecting elements, number of antennas at the base station, fading factor, etc. Therefore, we derive closed-form expressions for the optimal beamforming vector and phase shift matrix along with OP for special cases of the above-generalized channel model. These expressions are then used to gain useful insights into the system performance and to understand the implications of the proposed solutions. Our numerical analysis shows that the independent and identically distributed ({\rm i.i.d.}) fading is beneficial compared to the correlated case in the presence of LoS components. In fact, we establish this relation analytically for a case with absent LoS. Furthermore, we demonstrate that the maximum mean SNR improves linearly/quadratically with the number of RIS elements in the absence/presence of LoS component under {\rm i.i.d.} fading. }
Just like the existing works, the OP and EC  evaluations rely on the numerical computation of the proposed iterative algorithm, which does not clearly reveal the functional dependence of system performance on key parameters such as line-of-sight (LoS) components, correlated fading, number of reflecting elements, number of antennas at the base station (BS), and fading factor. In order to overcome this limitation, we derive closed-form expressions for the optimal beamforming vector and phase shift matrix along with OP for special cases of the general setup. These expressions are then used to gain useful insights into the system performance and to understand the implications of the proposed solutions. Our analysis reveals that the independent and identically distributed ({\rm i.i.d.}) fading is more beneficial than the correlated case in the presence of LoS components. This fact is analytically established for the setting in which the LoS is blocked. Furthermore, we demonstrate that the maximum mean SNR improves linearly/quadratically with the number of RIS elements in the absence/presence of LoS component under {\rm i.i.d.} fading.

% \textcolor{red}{This paper focuses on optimal beamforming for a passive reconfigurable intelligent surface (RIS) aided multiple-input single-output downlink system. Both direct and indirect links (through the RIS) to the user are considered to have correlated Rician fading. The phase shift matrix problem becomes non-convex due to the unit modulus constraint imposed on the passive RIS. To address this, we employ semidefinite relaxation (SDR) to obtain the optimal phase-shift matrix. We then propose an iterative algorithm to obtain the fixed-point solution for statistically optimal transmit beamforming vector and RIS-phase shift matrix.
% To evaluate the performance of the proposed beamforming scheme, we analyze key metrics such as outage probability and ergodic capacity. However, these metrics are limited by the numerical evaluation of the iterative algorithm, as is common in existing works. This approach hinders the understanding of how system performance depends on parameters such as line-of-sight (LoS) components, correlated fading, number of reflecting elements, number of antennas at the base station, and fading factor. To overcome this, we derive closed-form expressions for the optimal beamforming vector and phase shift matrix under specific cases of the general channel model. By using these expressions, we gain valuable insights into system performance and understand the implications of the proposed solutions. 
% Our numerical analysis reveals that independent and identically distributed (i.i.d.) fading is advantageous compared to the correlated case when LoS components are present. We establish this relationship analytically for a scenario without LoS. Furthermore, we demonstrate that the maximum mean signal-to-noise ratio (SNR) improves linearly/quadratically with the number of RIS elements in the absence/presence of LoS components under i.i.d. fading. }
\end{abstract}
\vspace{-0.4cm}
% Note that keywords are not normally used for peer review papers.
\begin{IEEEkeywords}
\vspace{-0.2cm}
Reconfigurable Intelligent Surfaces, Optimal Beamforming, Statistical Beamforming, Spatially Correlated Channel, Outage Analysis, Ergodic Capacity.
\end{IEEEkeywords}
% For peer review papers, you can put extra information on the cover
% page as needed:
% \ifCLASSOPTIONpeerreview
% \begin{center} \bfseries EDICS Category: 3-BBND \end{center}
% \fi
%
% For peer review papers, this IEEEtran command inserts a page break and
% creates the second title. It will be ignored for other modes.
\IEEEpeerreviewmaketitle

\section{Introduction}
\IEEEPARstart{R}{econfigurable} intelligent surfaces (RIS) is a planar array that consists of many sub-wavelength-sized antenna elements formed of meta-materials. One can control the phases of these elements to change the way they interact with the impinging electromagnetic wave, thereby controlling the local propagation environment to a certain extent \cite{Emil_SigProcessing_2022}. This ability can influence key characteristics of the propagation environment, such as reflection, refraction, and scattering, which were thus far assumed to be uncontrollable in wireless communications systems \cite{CuiTieJun_CodingMetamaterial_2014,LiuYuanwei_RISprinciples_2021}. If configured properly, RISs can reduce the effects of fading and interference by redirecting the impinging signals such that they add constructively at the receiver. The advantages of such technology are low power consumption, high spectral efficiency, improved coverage, and better reliability \cite{PanCunhua_2021_6G,HuangChongwen_2019_EE}. Though RIS is conceptually similar to several existing technologies, such as relays, backscatter communication, and massive multiple-input multiple-output (MIMO), a key differentiating feature is its low-cost implementation and lower power consumption, especially when all the elements are passive (which will be our assumption). Additionally, RIS operates in the full-duplex mode as a passive device  without additional RF chains requirement and energy consumption, which is not the case in the competing technologies mentioned above ~\cite{WuQingging_RIS_comparisons,DiRenzo_RISvsRelay_2020}.

Due to its ability to create a smart propagation environment, RIS is envisioned as an enabling technology for the various applications in future wireless networks such as terahertz communications, simultaneous wireless information exchange, wireless power transfer, non-orthogonal multiple access, physical layer security, etc \cite{pradhan2022robust,PanCunhua_2021_6G,HuangJie_ChannelModeling_2022}. However, including RIS in communication systems comes with its own unique challenges. Particularly important ones are: 1) channel estimation and 2) optimal design of transmit/receive beamformers and RIS phase shift matrix.
The challenges in channel estimation stem from the need for estimation of the cascaded channels (i.e., transmitter-RIS and RIS-receiver) using the composite channel seen by the receiver \cite{ZhengBeixiong_ChEst_Survey} and the lack of active RIS elements to aid  channel estimation  \cite{MiguelDajer_2022}. 
On the other hand, the optimal selection of beamformer and phase shift matrix based on the given knowledge of the channel usually leads to the non-convex optimization formulation mainly because of the {\em unit modulus constraint} for the passive RIS \cite{BasarErtugrul_20219} and often because of the underlying objective function. This paper focuses on optimal beamforming to maximize the transmission capacity.

With little misuse of terminology, we will refer to the jointly optimal selection of the transmit beamformer for the BS and the phase shift matrix for the RIS elements as {\em optimal beamforming} for easier reference. The approaches to optimal beamforming for multi-antenna systems are primarily categorized based on the knowledge of channel state information (CSI) \cite{Goldsmith_2003}. By leveraging the perfect knowledge of  CSI (PSCI) at the transmitter, significant research efforts have been devoted to selecting instantaneously optimal beamforming for time-varying RIS-aided channels with a focus on optimizing a variety of key performance indicators, \cite{WuQingqing_2019_PSCI_TxPow,HuangChongwen_2018_SR,YuXianghao_MISO_PSCI_2019,YuXianghao_MISO_PSCI_2020,NingBoyu_RISMIMO_PSCI_2020}. The authors of \cite{WuQingqing_2019_PSCI_TxPow} applied semi-definite relaxation (SDR) to obtain optimal phase shift matrix that minimizes the transmission power for a single user and multi-user RIS-aided MIMO communication systems, whereas the authors of \cite{HuangChongwen_2018_SR} presented an alternating  majorization-minimization method based algorithm for optimal beamforming to maximize the sum rate. Further, the authors of \cite{YuXianghao_MISO_PSCI_2019,YuXianghao_MISO_PSCI_2020} proposed iterative algorithms such as fixed point iteration, manifold optimization, and branch-and-bound techniques to maximize the {\rm SNR}. Further, \cite{NingBoyu_RISMIMO_PSCI_2020 } proposed a new sum-path-gain maximization criterion to obtain a  suboptimal solution, which is numerically shown to achieve near-optimal RIS-MIMO channel capacity.
%Therefore, it is not always reasonable to assume perfect channel state information (CSI) at the transmitter (particularly for the channels involving RISs) for optimal beamforming. 
While the aforementioned works rely on PCSI, a few studies in the literature also perform instantaneous optimal beamforming for time-varying RIS-aided channel based on imperfect knowledge of CSI (IPCSI) to account  for the error associated with channel estimation. For example,  \cite{ZhiKangda_2022_IPCSI} optimized the sum rate for RIS-aided massive MIMO system with zero-forcing detectors and IPCSI. A projected gradient-based iterative algorithm is presented in \cite{NemanjaStefan_2021_RateOpt_IPCSI} to maximize the transmission rate for  multi-stream multiple-input RIS-aided MIMO systems with IPCSI.
While these works are performance achieving, they all considered instantaneous PCSI/IPCSI knowledge at the transmitter, which is not always feasible and practical, particularly for the RIS channel where the channel estimation itself is a cumbersome task. Moreover, as pointed out above, the formulations of optimal beamforming for RIS-aided systems are non-convex which are often solved through alternating subproblems (for transmit/receiver beamformer and phase shift matrix) based iterative algorithms. Additionally, finding solutions for the optimal phase shift matrix subproblem often relies on solvers like CVX/MOSEK. Because of this, the instantaneous optimal beamforming for RIS-aided systems becomes a computationally challenging task.
Furthermore, the instantaneous requirement of the PCSI/IPCSI feedback and the RIS reconfiguration increases the system complexity and requires complex RIS design for quick reconfigurability. Due to these reasons, it is practical to perform the optimal beamforming 
% (that is applicable for a moderately large time scale) 
using the  statistical knowledge of CSI (SCSI) such that the system performance gets improved in a statistical sense, which is the main theme of this paper. 

In literature, there have been some efforts on the optimal beamforming for RIS-aided systems using the SCSI for a variety of channel models for the {\em direct link} from the BS to user equipment (UE)  and the {\em indirect link} from the BS to UE via RIS.  
%The commonly used channel models for SCSI based beamforming consider line-of-sight (LoS) components along with independent or correlated multipath fading  for RIS channels. 
The authors of \cite{hu2020statistical} considered a RIS-aided MISO system wherein the direct and indirect links  consist of LoS components along with multipath fading. To capture this, the wireless fading along indirect-direct links is modeled using  the Rician-Rician model.  We will refer to the fading along the indirect and direct links using a pair in a similar way throughout this paper.
Therein, the information of the angle of departure (AoD) and the angle of arrival (AoA) from/to BS/RIS defining the responses of LoS components  is appropriately used to obtain the  closed-form expressions of transmit beamformer and phase shift matrix that maximizes the mean SNR (which is equivalent to maximizing the upper bound on EC). The proposed solution requires  solving these closed-form expressions in an alternative fashion to arrive at a fixed-point optimal solution. The authors of \cite{HanYu_2019_SCSI} first obtained an upper bound on EC achieved with PCSI-based maximum ratio transmit beamforming for a large-scale RIS-aided MISO system under Rician-Rayleigh fading. Next, the authors derived a closed-form solution for a statistically optimal phase shift matrix of RIS that maximizes this upper bound on EC. In \cite{GanXu_2021_SCSI}, SCSI-based iterative optimal beamforming algorithms are presented to maximize the sum ergodic capacities for RIS-assisted multi-user MISO uplink and downlink  systems under the Rician-Rician fading scenario. For this scenario, the SCSI-based optimization formulation usually becomes difficult to handle.  Therefore, authors utilized  optimization methods like the alternating direction method of multipliers, fractional programming, and alternating optimization methods for obtaining sub-optimal solutions for power allocation and phase shift matrix.
The authors of \cite{ZhiKangda_2021_SCSI_SPAWC} presented  an approach wherein an IPSCI-based maximum ratio combining at the BS and a SCSI-based RIS-phase shifts configuration is adopted  for the multi-user uplink system under Rician-Rayleigh fading, whereas in  \cite{ZhiKangda_SSCI_mMIMO_2021}, SCSI-based near-optimal RIS-phase shifts are obtained using genetic algorithm when the BS utilizes PSCI-based MRC for multi-user uplink massive MIMO system.
In a similar direction, there exist few additional works on SCSI-based optimal beamforming for a variety of systems, including
multi-user multi-cell downlink system with the direct link absent \cite{LuoCaihong_Cellular_MISO_SSCI_2021}, multi-pair user exchanging information via RIS when the direct links are absent \cite{PengZhangjie_2021_SCSI}, and probabilistic technique  of discrete RIS phase-shit optimization  \cite{pradhan2023probabilistic}. 

In most of the aforementioned and other similar works, the key factor that enables tractable formulations is an assumption of {\rm i.i.d}  fading coefficients, often coupled with the absence of  LoS component along the direct link. However, ensuring independent fading in multi-antenna systems, particularly in RIS-aided systems, may not always be practical. This is especially true when  a large number of RIS elements are placed in a compact uniform planar array (UPA). Thus, it is crucial to consider  correlated fading for designing SCSI-based optimal beamforming for RIS-aided systems. The authors of \cite{Papazafeiropoulos_2022_SCSI_CorrFading} considered SCSI-based maximization of EC with respect to the RIS phase shifts for correlated Rayleigh channel while assuming that the direct link is absent. Further, the authors of \cite{WangJinghe_CorrelatedFading_2021} considered correlated Rician-Rayleigh fading for SCSI-based optimal beamforming to maximize EC of RIS-aided MIMO systems.  The authors proposed an iterative algorithm wherein the  optimal transmit beamforming vector and phase shift matrix are solved alternatively using SDR.
Most of the algorithms presented above rely on alternating between the transmit beamformer and phase shift matrix sub-problems to reach a fixed point solution similar to the PSCI-based approach. Additionally, the optimal phase shift solutions in each iteration often depend on the numerical optimization solver. However, these numerical solutions often hinder further analytical investigation, such as understanding the exact functional dependence of the optimal solutions on key system parameters. {\em Hence, it is equally, or perhaps even more, important to obtain closed-form optimal solutions, which will be the objective of this paper}.

SCSI-based optimal beamforming for RIS-aided systems circumvents the system design complexity issues and often provides comparable performance to the instantaneous PCSI-based beamforming (particularly when a strong LoS component is present). Therefore, to compare its performance with the PCSI-based schemes, it is crucial to characterize the performance of the statistical beamforming scheme analytically. This is another important reason for deriving a closed-form solution for optimal beamforming that lends analytical tractability in the performance analysis.
The performance of the beamformer scheme is usually characterized using the OP and EC. 
The closed-from expressions of OPs are derived for RIS-aided SISO systems under Rician fading with the direct link absent when the phase shift matrices are chosen based on IPCSI  in \cite{BaoTingnan_2022_Outage} and SCSI in \cite{XUPeng_AsymOutProb_2022}. Further, the authors of \cite{VanChien} derived the coverage probabilities for an arbitrary and statistically optimal phase shift matrix for the RIS-aided SISO system under Rayleigh-Rayleigh fading scenario. However, OP and EC analysis under jointly optimal  transmit/receive beamforming and RIS phase shift matrix for general RIS-aided multi-antenna systems is not investigated. In this paper, we focus on the statistically optimal beamforming and its performance analysis for various fading models.  
%In \cite{BaoTingnan_2022_Outage}, closed-form expressions for exact outage probability and ergodic capacity are derived for  RIS-aided SISO system under Ricain fading when the  phase shift matrix is optimally chosen based on IPCSI and the direct link is absent.
%Whereas, the authors of \cite{XUPeng_AsymOutProb_2022} derived an approximate closed-form expression of the outage probability for RIS-aided SISO system under Rician fading when SCSI-based optiaml phase shift matrix is used and direct link is absent. 
%\textcolor{blue}{Along these lines, the authors of \cite{XUPeng_AsymOutProb_2022,SalhabAnasM_AppChGainDist_2021,YangLiang_AppChaGainDist_Diversity_2020} analyze the outage performance for RIS-aided communication systems.  The authors of \cite{XUPeng_AsymOutProb_2022} derive a closed-form expression for the asymptotic outage probability, whereas \cite{SalhabAnasM_AppChGainDist_2021} obtain approximate channel gain distributions for two cases: 1) RIS-aided system and 2) RIS-at-transmitter system. In addition, \cite{YangLiang_AppChaGainDist_Diversity_2020} also derives a closed-form expression for a tight approximation of the channel gain and provides insight into diversity gain for RIS-aided communication systems.}
 %\begin{enumerate}
    %\item The authors of \cite{LuoCaihong_Cellular_MISO_SSCI_2021}  applied the fractional programming algorithm to obtain  SCSI based optimal transmit beamforming for maximizing as approximate  form of sum rate in RIS aided multi-user multi-cell  downlink system when the indirect link is modeled as Rician channel model and the  direct link from BS to UE is absent. 
    %\item The authors \cite{ZhiKangda_SSCI_mMIMO_2021} first obtain an approximation sum rate for RIS aided  multi-user uplink  massive MIMO system  and apply genetic algorithm to maximizing it.  
    %\item In \cite{kota2022optimal}, we derived closed from expressions for statically optimal beamforming that maximizes a tight lower bound on the mean SNR RIS-adided MISO system for the case of i.i.d. Ricain fading along direct and indirect links. In this paper, we extended the analysis by providing optimal algorithms/closed-form solutions for various channel models with correlated and {i.i.d} fading.
    %\item \cite{PengZhangjie_2021_SCSI}: {\em Analysis and optimization for RIS-aided multi-pair communications relying on statistical CSI}\newline
     %{\rm i.i.d.} Rician for indirect link and  no direct link\newline
     %This paper derives an approximate expression for the achievable rate of a RIS-aided multi-pair communication system by assuming the statistical CSI to be known. Further, they propose a genetic algorithm to solve the rate maximization problem to validate the derived expressions. Further, they also show that the Genetic algorithm solution almost approaches the globally optimal solution.   
    %\item \cite{HanYu_2019_SCSI}: {\em Large Intelligent Surface-Assisted Wireless Communication Exploiting Statistical CSI}\\
    %iid Rician for indirect link and iid Raylei for direct link\\
    %This paper presents an optimal phase shift design based on maximizing the ergodic spectral efficiency upper bound and statistical channel state information for a RIS-aided MISO system with large number of reflecting elements. They further formulate the requirement on the number of quantization bits to ensure acceptable ergodic spectral degradation. Simulation results showed the proposed design provided better ergodic spectral efficiency that random phase shifts. 
   % \item  \cite{GanXu_2021_SCSI}: {\em RIS-Assisted Multi-User MISO Communications Exploiting Statistical CSI}\\
    %iid Rician fading for both direct and indirect links\\
    %This paper investigates the ergodic capacity of RIS-assisted multi-user MISO communication system in both the uplink and the downlink based on the statistical information of the channel. For the mentioned system model, the joint transmit beamforming and phase shift matrix design problem is addressed by maximizing the ergodic capacity. Further they propose sub-optimal algorithms to solve the ergodic capacity maximization problem in the uplink and downlink.
    %\item \cite{ZhiKangda_2021_SCSI} or \cite{ZhiKangda_SSCI_mMIMO_2021}: {\em Statistical CSI-Based Design for Reconfigurable Intelligent Surface-Aided Massive MIMO Systems With Direct Links}\\
    %This letter proposes a RIS-aided massive MIMO system transmitting information to multiple, single antenna users with direct links. The optimal phase shifts are obtained through the Genetic algorithm, and a MRC transmitter is used at the base station by assuming that the statistical CSI is known. Simulation results show that the presence of RIS improves the ergodic spectral efficiency significantly.
    %\item \cite{ZhiKangda_2021_SCSI_SPAWC}: {\em Reconfigurable Intelligent Surface-Aided MISO Systems with Statistical CSI: Channel Estimation, Analysis and Optimization}\\
    %MISO, iid Rician for indirect link and iid Rayleigh for direct link\\
    %This paper proposes the design of an RIS-aided multi-user MISO system. Maximal ratio combining is considered at the base station (imperfect CSI knowledge), and the RIS phase shift matrix is designed by using the statistical CSI.   
    %\item  \cite{Papazafeiropoulos_2022_SCSI_CorrFading}: {\em Ergodic Capacity of IRS-Assisted MIMO Systems With Correlation and Practical Phase-Shift Modeling}\\
   %MIMO (massive)?, Rayleigh correlated fading, direct link absent\\
    %This letter focuses on obtaining the RIS phase shifts by maximizing the ergodic capacity of a point-to-point RIS-aided MIMO system under Rayleigh correlated fading scenario. They also assume correlation between the amplitude and phase-shifts of each RIS elements. Further, they derive the pdf of the cascaded channel and ergodic capacity using the obtained optimal phase shifts.     
    %\item \cite{WangJinghe_CorrelatedFading_2021}: {\em Joint Transmit Beamforming and Phase Shift Design for Reconfigurable Intelligent Surface Assisted MIMO Systems}\\
    %MIMO, Correlated fading, Rician channel for indirect link and Rayleigh for direct link.
%\end{enumerate}  
%\subsection{Contributions}
%This paper focuses on designing SCSI-based  optimal beamforming for maximizing the ergodic capacity of a RIS-aided MISO system under the most general channel model which incorporates both the direct and indirect links each comprising of the LoS component and correlated multipath fading. In order to model such a scenario, we consider the correlated Rician-Rician model and assumed the statistical knowledge of CIS is available at the BS.  For this setting, we proposed an iterative algorithm for SCSI-based optimal transmit beamformer and phase shift matrix, and also derive its outage probability and ergodic capacity performances. To the best of the author's knowledge, this is the first paper to consider such a general model for SCSI-based beamforming. The most closest work dealing with a similar problem is \cite{WangJinghe_CorrelatedFading_2021}, but it ignores LoS component along the direct link. In addition, we derive the closed-form expressions or provide computationally efficient optimal beamforming solutions along with their outage and ergodic capacity analyses for various special cases of the above-generalized channel model.  In the following, we summarize the key contributions of this work.

% {\color{blue}This paper focuses on designing SCSI-based optimal beamforming for maximizing the mean SNR (i.e., equivalent to maximizing the upper bound of ergodic capacity) of a RIS-aided MISO system under the correlated Rician-Rician fading model which is the most general fading model as it  incorporates both the direct and indirect links with the LoS component and correlated multipath fading. 
% %In order to model such a scenario, we consider the correlated Rician-Rician model and assumed the statistical knowledge of CIS is available at the BS. 
% For this setting, we proposed an iterative algorithm for SCSI-based optimal beamforming and also derived its outage probability and ergodic capacity performances. In addition, we also derived closed-form expressions for the optimal beamforming and outage probabilities under various special cases of the above generalized  fading model and also provide useful comparative insights related to their  performances. 
% To the best of the author's knowledge, this is the first paper to consider such a general model for SCSI-based beamforming. 
% The closest work dealing with a similar problem is \cite{WangJinghe_CorrelatedFading_2021}, but it ignores the LoS component along the direct link.
% %In addition, we derive the closed-form expressions or provide computationally efficient optimal beamforming solutions along with their outage and ergodic capacity analyses for various special cases of the above-generalized channel model. 
% In the following, we summarize the key contributions of our work.}\\
% %We consider the joint beamformer and phase shift matrix problem that maximizes the ergodic capacity upper bound under various cases considering both correlated and independent fading assumptions. Our paper aims to provide a closed-form solution to the joint problem wherever possible. Following is a comprehensive list of our contributions:

{\em Contributions:} This paper focuses on designing SCSI-based optimal beamforming for maximizing the mean SNR of a RIS-aided MISO system under a correlated Rician-Rician fading model incorporating both direct and indirect links with LoS component and correlated multipath fading. For this setting, we propose an iterative algorithm and also analyzed its OP and EC. In addition, we also derived closed-form expressions for the optimal beamforming and its OP under various special cases of the above generalized fading model. Using these derived expressions, we provide useful insights related to their performance comparisons. {\em To the best of our knowledge, this is the first paper to consider such a general model for SCSI-based beamforming.} The closest work dealing with a similar problem is \cite{WangJinghe_CorrelatedFading_2021}, but it ignores the LoS component along the direct link. In the following, we summarize the key contributions of our work.
% $\bullet$ 
\begin{enumerate}
\item An iterative algorithm is developed for optimal beamforming to maximize the mean SNR under correlated  Rician-Rician fading.%The algorithm alternates over a closed-form transmit beamformer and semi-definite relaxation (SDR) technique based phase shift matrix for a fixed point solutions. 
%We propose the joint beamformer and phase shift matrix design of a generic system model with a correlated Rician fading model along both the direct and indirect links (R1 Corr) that maximizes the mean SNR. For this scenario, we propose to iterate over the transmit beamformer closed-form solution and a solver-based semi-definite relaxation (SDR) solution for the phase shift matrix subproblem to solve the joint problem. Further, we also provide a numerical analysis of the outage and ergodic capacity using this iterative solution.  
~For this beamforming scheme, OP is shown to closely follow square of Rice distribution, which is then utilized to determine EC. The parameters of the outage depend on the numerical evaluation of the proposed algorithm. 
% $\bullet$ 
\item Next, we derived computationally efficient closed-form expressions for optimal beamforming and their OPs for $\rm{i.i.d.}$ Rician-Rayleigh, correlated Rayleigh-Rayleigh and $\rm{i.i.d.}$ Rayleigh-Rayleigh. Furthermore, for $\rm{i.i.d.}$ Rician-Rician case, we maximized a carefully formulated lower bound of the mean SNR to arrive at the closed-form solutions.   
%We also analyze the same system model under the independent fading assumption (R1 IID). Our aim in the paper is to provide a computationally efficient solution to the mean SNR maximization problem. Hence, we propose fixed-point equations for the transmit beamformer and the phase shift matrix as the solution of the lower bound of mean SNR. We further show through numerical analysis that the proposed lower-bound solution is very tight. We also derive closed-form expressions for the outage and ergodic capacity under this case.  
%\item Next, to get an insight into RIS's diversity, we assume that the direct link has no line-of-sight (LoS) component (Rice-Rayleigh - R2 Corr). Under this case, we again provide an iterative solution of the joint problem and the outage and ergodic capacity expressions for a given $\mathbf{f}$ and $\mathbf{Psi}$. Next, as a special case, we derive computationally efficient closed-form solutions for the optimal beamformer, phase shift matrix, outage probability, and ergodic capacity.
%\item We next analyze the case when LoS components are absent in both the links (Rayleigh-Rayleigh) under both correlated and independent fading assumptions (R3 Corr, R3 IID). We derive a closed-form solution of the max mean SNR problem by decoupling $\mathbf{f}$ and $\mathbf{\Psi}$ variables. We deduce that under the R3 Correlated case, all the elements of the RIS provide equal phase shifts. Further, we also conclude that this constant phase shift can be of any value. On the other hand, under the R3 IID case, we derive the mean SNR as a constant dependent on the direct link strength and the number of reflecting elements. 
    %\item Diversity order and coding gain
%\item A thorough numerical analysis on the outage and ergodic capacity performance is presented in the simulation results section. Some key takeaways are: the statistically optimal system in IID vs. correlated fading scenarios depend on the strength of the LoS component. Next, we infer that when $N$ is large enough, the direct link will not improve the achievable capacity any further. We also numerically observe that as $\theta$ increases, the ergodic capacity decreases as the beamwidth of the transmit beamformer increases to take advantage of both the links. We confirm this by plotting the radiation pattern of the obtained statistically optimal beamformer. Finally, we study the impact of the Rician fading factor on the proposed statistical solutions.  
% $\bullet$ 
\item We have shown that the correlated fading outperforms the {\rm i.i.d.} case under Rayleigh-Rayleigh scenario. Next, we present a comparative analysis of {\rm i.i.d.} fading under Rician-Rician, Rician-Rayleigh, and Rayleigh-Rayleigh settings through analytical expressions. 
% \textcolor{red}{In addition, we also present the of SCSI-based optimal beamformer for the limiting cases of the fading factor under different fading models}.\\
%\item A comparative analysis of correlated and independent fading scenarios for the most general case (Rician-Rician), and special cases (Rician-Rayeleigh, Rayleigh-Rayleigh) under it is presented. 
% $\bullet$ 
\item Key takeaways based on our numerical analysis are: 1) the SCSI-based optimal beamforming  performs better under {\rm i.i.d.} fading compared to the correlated scenario in the presence of  LoS component, 2) the direct link presence becomes insignificant when the number of reflecting elements is large, 3) the achievable capacity decreases with the increase in the difference between AoDs of LoS components belonging to direct and indirect links under Rician-Rician fading. 
%Some key takeaways from the simulation results discussion are: The IID scenario performs similar to the correlated fading scenario under the R1 case but outperforms in the R2 case. In general, the IID scenario performs better than the correlated scenario because the channel matrices are well-conditioned. In other words, the channel's spectral range is small, allowing more degrees of freedom for transmission. We do not observe this pattern under the R1 case because of strong LoS components along both links. However, as the LoS component is absent along the direct link under the R2 case, we can see the impact of correlation degrading the achievable capacity performance. Nonetheless, this performance degradation is minor due to the presence of LoS components along the indirect link. On the contrary, in the Rayleigh-Rayleigh case the correlated fading scenario performs better than the independent scenario because of the presence of RIS. This directly indicates the advantage of including RIS in communication systems. Next, we also investigate the relationship between the path-loss ratio between direct and indirect links and the number of reflecting elements numerically. We infer from the simulation results that when $N$ is large enough, the direct link will further not improve the achievable capacity. We also numerically observe the impact of $\theta$ on the statistical performance. We see that as $\theta$ increases, the ergodic capacity decreases due to an increase in the beamwidth of the transmit beamformer that has to take advantage of both links. We further show the radiation pattern of this beamformer as $\theta$ increases. We also study the impact of the Rician fading factor on the proposed statistical solution. 
\end{enumerate}
{\em Notations:} $a^*$ and $|a|$ represent the conjugate and absolute value of $a$. $\left\lVert \mathbf{a} \right\lVert$ and $\mathbf{a}_i$ are the norm and the $i$-th element of vector $\mathbf{a}$, whereas $\mathbf{A}^T$, $\mathbf{A}^H$, $\|\mathbf{A}\|_F$, ${\rm trace}(\mathbf{A})$,  $\mathbf{A}_{i,:}$, $\mathbf{A}_{:,i}$ and $\mathbf{A}_{ij}$ are the transpose, Hermitian, Frobenius norm, trace, $i$-th row, $i$-th column and $ij$-th element of the matrix $\mathbf{A}$, respectively. The notation $\mathbb{C}^{M\times N}$ is the set of  $M \times N$ complex matrices, ${\rm I_M}$ is $M\times M$ identity matrix and $\mathbf{1}_{\rm M}$ is a $M\times 1$ vector with unit elements. $\mathbf{v_A}$ and $\lambda_\mathbf{A}$ are the principal eigenvector and eigenvalue of  $\mathbf{A}$.  $\odot$ is the hadamard product, $\rm{diag}(\mathbf{a})$ is a diagonal matrix such that vector $\mathbf{a}$ forms its diagonal, and $\mathcal{CN}(\boldsymbol{\mu},\mathbf{K})$ denotes complex  Gaussian distribution with mean $\boldsymbol{\mu}$ and covariance matrix $\mathbf{K}$.
%\pagebreak
%\textbf{Channel Modeling} \cite{HuangJie_ChannelModeling_2022} This paper shows the various applications of RIS such as High-frequency band communications, Simultaneous wireless information, and power transfer (SWIPT), Space-air-ground-sea integrated network, RIS-assisted Non-orthogonal Multiple access (NOMA), Physical layer security (PLS), etc. Also, it summarizes the channel measurements observed at sub-6 GHz and mmWave bands in a RIS-aided network. Also, the path loss in the RIS-assisted network is analyzed as a variation of the distance of RIS from the transmitter and receiver. \newline

%\textbf{Survey Papers}
%\cite{8} This paper identifies the major issues and research opportunities on the path to 6G associated with the integration of RISs and other emerging technologies(Section VI).\newline
%\cite{LiuYuanwei_RISprinciples_2021}


%\textbf{Beamforming papers}
%\cite{WangJinghe_CorrelatedFading_2021} Correlated fading.\newline
%\cite{hu2020statistical} This paper deals with joint beamforming in RIS-aided multiple inputs multiple outputs (MIMO) setup with the knowledge of statistical Channel State Information (CSI). The alternating optimization method obtains the optimal solution for active and passive beamforming vectors. \newline
%\cite{6} This paper considers a communication network employing multiple RISs deployed to assist a massive Multi user-MISO system in the presence of a correlated Rayleigh fading channel. Exhaustive search methods are used to find the optimal passive beamforming vectors.\newline
%\cite{YuXianghao_2019} This paper deals with the low-complexity methods for achieving joint beamforming in a RIS-aided multiple-input multi-user network. The adaptive user grouping method is used whose complexity is lesser than the normally used alternating optimization method for beamforming. Here, the BS beamforming is determined based on the zero-forcing (ZF) principle for a fixed RIS phase shift matrix. Then a one-dimensional search is used to find the best beamforming pair.\newline
%\cite{WuQingging_RIS_comparisons} This paper compares RIS with existing technologies like Backscatter, MIMO Relay, and Massive MIMO-based on various parameters like operating mechanism, hardware cost, energy consumption, etc.

%\textcolor{purple}{\cite{3} This paper comprises an experimental setup to show that RIS can be used to increase Signal Interference to noise ratio (SINR) and thereby mitigate multi-path fading which shows that it is an appealing technology for operation multi-operator mobile wireless networks.\newline}

%\textbf{Performance Analysis}
%\cite{5} This paper focuses on deriving the closed-form expression for the Symbol error probability (SEP) in a RIS-assisted Single input single output (SISO) setup  with spatially correlated channel set up assuming that the phases added by The RIS elements are deterministic. A search-based algorithm is proposed which identifies the optimal RIS phase that minimizes the SEP. This optimal equal phase proposed, maximizes the average received SNR and thereby the upper bound of channel capacity. \newline

%\section{Related Works and Motivation}
%\subsection{Perfect CSI}
%\subsubsection{Analysis and Optimization of RIS-Aided Massive
%MIMO with ZF Detectors and Imperfect CSI} This paper analyzes RIS-aided massive MIMO system with zero-forcing detectors and imperfect CSI. First, the uplink achievable rate is formulated that considers a MMSE channel estimator. This sum rate is optimized using the MM algorithm. Simulation results show that by aligining the RIS phase shifts to a user, the rate scaling order approches $\mathcal{O}(\log_2(MN^2)).$  

%\subsubsection{\textbf{Comm Letter}: Robust Transmission Design for RIS-Aided Communications With Both Transceiver Hardware Impairments and Imperfect CSI} This letter considers the joint optimization of the transmit beamformer and the phase shift matrix under imperfect CSI assumption. The optimal beamformer and phase shift matrix are obtained by formulating the constrained transmit power minimization problem. 

%\subsubsection{\textbf{TWC}: Achievable Rate Optimization for MIMO Systems With Reconfigurable Intelligent Surfaces} This paper studies the achievable rate optimization of a multi-stream RIS-aided MIMO system. The authors formulate the joint optimization problem of the covariance matrix of the transmitted signal and the RIS elements and solve it using a projected gradient method-based iterative algorithm.

%\subsection{Statistical CSI}
%\subsubsection{\textbf{TVT} \cite{PengZhangjie_2021_SCSI}: Analysis and optimization for RIS-aided multi-pair communications relying on statistical CSI} This paper derives an approximate expression for the achievable rate of a RIS-aided multi-pair communication system by assuming the statistical CSI to be known. Further, they propose a genetic algorithm to solve the rate maximization problem to validate the derived expressions. Further, they also show that the Genetic algorithm solution almost approaches the globally optimal solution.

%\subsubsection{\textbf{TComm} \cite{GanXu_2021_SCSI}: RIS-Assisted Multi-User MISO Communications Exploiting Statistical CSI}
%WMSE Min, Fractional Programming 
%This paper investigates the ergodic capacity of RIS-assisted multi-user MISO communication system in both the uplink and the downlink based on the statistical information of the channel. For the mentioned system model, the joint transmit beamforming and phase shift matrix design problem is addressed by maximizing the ergodic capacity. Further they propose sub-optimal algorithms to solve the ergodic capacity maximization problem in the uplink and downlink. 

%\subsubsection{\textbf{TVT} \cite{HanYu_2019_SCSI}: Large Intelligent Surface-Assisted Wireless Communication Exploiting Statistical CSI} This paper presents an optimal phase shift design based on maximizing the ergodic spectral efficiency upper bound and statistical channel state information for a RIS-aided MISO system with large number of reflecting elements. They further formulate the requirement on the number of quantization bits to ensure acceptable ergodic spectral degradation. Simulation results showed the proposed design provided better ergodic spectral efficiency that random phase shifts. 

%\subsubsection{\textbf{Comm Letter} \cite{ZhiKangda_2021_SCSI}: Statistical CSI-Based Design for Reconfigurable Intelligent Surface-Aided Massive MIMO Systems With Direct Links}
%This letter proposes a RIS-aided massive MIMO system transmitting information to multiple, single antenna users with direct links. The optimal phase shifts are obtained through the Genetic algorithm, and a MRC transmitter is used at the base station by assuming that the statistical CSI is known. Simulation results show that the presence of RIS improves the ergodic spectral efficiency significantly.

%\subsubsection{\textbf{Invited Paper SPAWC} \cite{ZhiKangda_2021_SCSI_SPAWC}: Reconfigurable Intelligent Surface-Aided MISO Systems with Statistical CSI: Channel Estimation, Analysis and Optimization}
%This paper proposes the design of an RIS-aided multi-user MISO system. Maximal ratio combining is considered at the base station (imperfect CSI knowledge), and the RIS phase shift matrix is designed by using the statistical CSI.    

%\subsubsection{\textbf{Science China} \cite{hu2020statistical}: Statistical CSI-based design for intelligent reflecting surface assisted MISO systems}



%\subsubsection{\textbf{China Communications} \cite{DangJian_2020_SCIS}: Joint Beamforming for Intelligent Reflecting Surface Aided Wireless Communication Using Statistical CSI} Is this a strong paper to cite?

%\subsubsection{\textbf{Comm Letters} \cite{Papazafeiropoulos_2022_SCSI_CorrFading}: Ergodic Capacity of IRS-Assisted MIMO Systems With Correlation and Practical Phase-Shift Modeling}
%This letter focuses on obtaining the RIS phase shifts by maximizing the ergodic capacity of a point-to-point RIS-aided MIMO system under Rayleigh correlated fading scenario. They also assume correlation between the amplitude and phase-shifts of each RIS elements. Further, they derive the pdf of the cascaded channel and ergodic capacity using the obtained optimal phase shifts. 

%\subsubsection{\textbf{TCog} \cite{WangJinghe_CorrelatedFading_2021}: Joint Transmit Beamforming and Phase Shift Design for Reconfigurable Intelligent Surface Assisted MIMO Systems}

%\subsection{Performance Analysis through Outage Probability}
%\subsubsection{\textbf{Comm Letters}: Coverage Probability and Ergodic Capacity of Intelligent Reflecting Surface-Enhanced Communication Systems}
%This letter utilizes the incomplete Gamma distribution and statistical information of Rayleigh channels to derive coverage probability closed-form expressions. These closed-form expressions are for both optimal and arbitrary phase shifts. Further, they derived closed-form expression of ergodic capacity based on a MeijerG function. 

%\subsubsection{\textbf{Comm Letters}: Accurate Performance Analysis of Reconfigurable Intelligent Surfaces Over Rician Fading Channels}
%This letter investigates the performance of RIS-aided communication networks over Rician Fading channel. The authors derive accuarte closed-form approximations of outage probability, channel capacity, etc,. They also provide cloased-form expressions of system diversity and coding gain. 

%\subsubsection{\textbf{WCNC 2022} \cite{BaoTingnan_2022_Outage}: Performance Analysis of RIS-aided Communication Systems over the Sum of Cascaded Rician Fading with imperfect CSI}
%In this paper, the authors study the performance of RIS over the sum of cascaded Rician fading channels under imperfect CSI assumption. They also derive closed-form expressions of outage probability, ergodic capacity and BER.  

%\subsubsection{\textbf{TVT} \cite{XUPeng_AsymOutProb_2022}: Performance Analysis of RIS-Assisted Systems With Statistical Channel State Information}
%This paper investigates the outage performance of a RIS-aided system by assuming the statistical information of CSI to be known. They provide an approximate closed-form expression of the outage probability after deriving the composite channel distribution. They further provide interesting insights into the coding gain via analytical and simulation results.  

% Note that the IEEE does not put floats in the very first column
% - or typically anywhere on the first page for that matter. Also,
% in-text middle ("here") positioning is typically not used, but it
% is allowed and encouraged for Computer Society conferences (but
% not Computer Society journals). Most IEEE journals/conferences use
% top floats exclusively. 
% Note that, LaTeX2e, unlike IEEE journals/conferences, places
% footnotes above bottom floats. This can be corrected via the
% \fnbelowfloat command of the stfloats package.
%\pagebreak
%------------------------------ SYSTEM MODEL ----------------------------%
\vspace{-.3cm}\section{System Model}\vspace{-.2cm}

We consider a RIS-aided MISO communication system consisting of a BS with $M$ antennas, a RIS with $N$ passive antenna elements, and a single antenna UE. The BS can transmit information to the UE by jointly utilizing the direct link (BS-UE) and the indirect link (BS-RIS-UE). We consider a more practical setup wherein the LoS components and multi-path fading are present along both the direct and indirect links. To incorporate this, we model BS-UE, BS-RIS and RIS-UE channels using Rician fading. %Another essential aspect of channel fading is the spatial correlation among the antenna elements.
It is worth noting that the spacing between RIS elements might not ensure the independence of the fading coefficients, especially when arranging a large number of elements in a compact UPA. Therefore, it is crucial to consider correlated fading for designing the SCSI-based optimal beamforming. Our main objective in this paper is to jointly optimize the transmit beamforming vector $\mathbf{f}$ and the RIS phase shift matrix $\mathbf{\Phi}$ by leveraging the statistical knowledge of CSI for the above setup. %In the following sections, we present the details of  the correlated channel model, the received signal model, and the problem formulation. 
% We consider a RIS-aided MISO communication system consisting of a BS with $M$ antennas, RIS with $N$ passive antenna elements, and a single antenna UE. The BS can transmit information to the UE by jointly utilizing the direct link (BS-UE) and the indirect link (BS-RIS-UE) as shown in \autoref{system model}.
% In the RIS-aided communication system, it is reasonable to assume that the RIS is positioned such that it has an LoS path with both the BS and UE. In addition, as a general case, we also assume that the direct link also consists of the LoS component along with the multipath fading.
% In particular, our goal is to jointly optimize the transmit beamforming vector $\mathbf{f}$ and the RIS phase shift matrix $\mathbf{\Phi}$ using the statistical knowledge of CSI. This paper attempts to obtain an optimal beamforming scheme for the above general case to capture the presence of an LoS path along with the multipath fading. Thus, we model all three channels (i.e. BS-UE, BS-RIS, RIS-UE) using Rician fading. Another important aspect to consider for statistical beamforming is the spatial correlation between the antenna elements at BS/RIS. It may be noted that the spacing between the RIS elements may not ensure the independence of their fading coefficients, especially when arranging a large number of RIS elements in a compact uniform planar array (UPA). In the following subsections, we discuss the channel model with correlated fading, the received signal model, and the problem formulation in detail. 
% 
% 
% We consider a RIS-aided MISO communication system consisting of a BS with $M$ antennas, RIS with $N$ passive antenna elements, and a single antenna UE. The BS can transmit information to the UE by jointly utilizing the direct link (BS-UE) and the indirect link (BS-RIS-UE).
% , as shown in \autoref{system model}. 
% \begin{figure}[ht!]
% \centerline{\includegraphics[width=0.4\textwidth]{25April_Figures/Screenshot 2023-06-18 at 9.48.22 PM.png}}\vspace{-.4cm}
% \caption{Illustration of the RIS-aided MISO communication system}
% \label{system model}\vspace{-.5cm}
% \end{figure}
% It is often a reasonable assumption in RIS-aided communication systems that LoS components exist along direct and indirect links. To incorporate this phenomenon, we adopt Rician fading as channel distributions for all the links: BS-UE, BS-RIS, and RIS-UE. Our main objective in this paper is to jointly optimize the transmit beamforming vector $\mathbf{f}$ and the RIS phase shift matrix $\mathbf{\Phi}$ by leveraging the statistical knowledge of channel state information (CSI). Another essential aspect of the wireless channel fading is the spatial correlation among the antenna elements at the BS and RIS. It is worth noting that the spacing between RIS elements might not ensure the independence of the fading coefficients, especially when arranging a large number of elements in a compact uniform planar array (UPA). Therefore, it is crucial to consider the correlated fading for   designing the SCSI-based optimal beamforming.
% {\color{red}In the subsequent subsections, we delve into a detailed discussion of the channel model incorporating correlated fading, the received signal model, and the problem formulation.}
\vspace{-.4cm}\subsection{Spatially Correlated Rician Channel Model}\label{channel model}\vspace{-.2cm}
 We model the BS-UE channel $\mathbf{g}$, BS-RIS channel $\mathbf{H}$, and RIS-UE channel $\mathbf{h}$  using the Rician fading with a factor $K$. Such channels can be expressed as the superposition of a deterministic LoS and spatially correlated random multipath components. The direct link under the correlated fading channel can be expressed as 
\begin{equation}
    \mathbf{g} = \kappa_l\mathbf{\Bar{g}} + \kappa_n\mathbf{\Tilde{g}},\label{directlink}
\end{equation}
where $\kappa_l = \sqrt{\frac{K}{1+K}}$, $\kappa_n = \sqrt{\frac{1}{1+K}}$, $\mathbf{\Tilde{g}}\sim\mathcal{C}\mathcal{N}(0,\mathbf{R}_{\rm BT})$ is the multipath component with covariance matrix $\mathbf{R}_{\rm BT}$ and $\mathbf{\Bar{g}}$ is the deterministic LoS component which is defined by the response of the uniform linear array (ULA) at the BS as $\mathbf{\Bar{g}} = \mathbf{a}_M(\theta_{\rm bd}^{\rm d})$ such that $\theta_{bd}^{\rm d}$ is  AoD along the direct link from BS. The response vector of ULA is given by
% \begin{equation*}
    $\mathbf{a}_M(\theta) = \frac{1}{\sqrt{M}}[1,e^{-j\frac{2 \pi \lambda}{d}\sin(\theta)},\ldots,e^{-j\frac{2 \pi \lambda}{d}(M-1)\sin(\theta)}]^T$, where $d$ is the distance between the antenna elements and $\lambda$ is the operating wavelength.
% \end{equation*}
 Similarly, we express the RIS-UE link as 
\begin{equation}
    \mathbf{h} = \kappa_l\mathbf{\Bar{h}} + \kappa_n\mathbf{\Tilde{h}} ,\label{indirect link_h}
\end{equation}
where $\mathbf{\Tilde{h}}\sim\mathcal{C}\mathcal{N}(0,\mathbf{R}_{\rm RT})$ is the multipath component with covariance matrix $\mathbf{R}_{\rm RT}$ at the RIS transmit end, $\mathbf{\Bar{h}} = \mathbf{a}_N(\theta_{\rm rd})$ and  $\theta_{\rm rd}$ is the AoD from RIS.
Now, the BS-RIS link can be given as
 \begin{equation}
    \mathbf{H} = \kappa_l\mathbf{\Bar{H}} + \kappa_n\mathbf{\Tilde{H}} ,\label{indirect link_H}
\end{equation}
where $\mathbf{\Tilde{H}}\sim\mathcal{C}\mathcal{N}(0,\mathbf{R}_{\rm RR})$ is the multipath component modeled using \textit{double-sided spatial correlation} as $\mathbf{\Tilde{H}} = \mathbf{\Tilde{R}}_{\rm RR}\mathbf{\Tilde{H}}_{\rm W}\mathbf{\Tilde{R}}_{\rm BT}$ such that $\mathbf{R}_{\rm RR} = \mathbf{\Tilde{R}}_{\rm RR}\mathbf{\Tilde{R}}_{\rm RR}^T$, $\mathbf{R}_{\rm BT} = \mathbf{\Tilde{R}}_{\rm BT}\mathbf{\Tilde{R}}_{\rm BT}^T$ and $\mathbf{\Tilde{H}_W}\sim\mathcal{C}\mathcal{N}(0,I)$. $\mathbf{R}_{\rm RR}$ is the correlation matrix at the RIS receive end. $\mathbf{\Bar{H}} = \mathbf{a}_N(\theta_{\rm ra})\mathbf{a}^T_M(\theta_{\rm bd}^{\rm i})$ is the LoS component given by the response matrix of UPA such that $\theta_{\rm ra}$ is the AoA at  RIS and $\theta_{\rm bd}^{\rm i}$ is the AoD at  BS.

To model the covariance matrix for MIMO channel, one can apply the widely used  \emph{Kronecker Separable Model
} \cite{KSM} for capturing the pairwise correlation between the antenna elements. However, recently in \cite{RIS_Corr_Fad}, it is shown that the above-mentioned model is inaccurate for the RIS channel and use the UPA geometry of RIS to derive a new model wherein the correlation between two antenna elements is given by ${\rm sinc}\frac{2d}{\lambda}$. 
% Using this, we model the fading covariance matrix for RIS on receiving and transmitting ends such that its $ij$-th element follows
% \begin{equation*}
%     \mathbf{R}_{{\rm RT},{ij}} = \mathbf{R}_{{\rm RR},{ij}} = {\rm sinc}\frac{2r_{ij}}{\lambda},
% \end{equation*}
% where $r_{ij}$ is the distance between the $i$-th and $j$-th antenna elements of the RIS. Similarly, we model the fading covariance matrix at the BS as
% \begin{equation*}
%     \mathbf{R}_{{\rm BT},{ij}} = {\rm sinc}\frac{2d_{ij}}{\lambda},
% \end{equation*}
% where $d_{ij}$ is the distance between the $i$-th and $j$-th antennas at the BS.
Using this, we model the fading covariance matrices for the BS and for the RIS on receiving and transmitting ends such that their $ij$-th element is
\begin{equation*}
    \mathbf{R}_{{\rm BT},{ij}} = {\rm sinc}\frac{2d_{ij}}{\lambda} \text{~~and~~}\mathbf{R}_{{\rm RT},{ij}} = \mathbf{R}_{{\rm RR},{ij}} = {\rm sinc}\frac{2r_{ij}}{\lambda},
\end{equation*}
where $d_{ij}$ and $r_{ij}$ are the distances between the $i$-th and $j$-th antennas at BS and antenna elements at RIS, respectively.
\vspace{-.4cm}\subsection{Received Signal Model}\vspace{-.2cm}
Given the BS transmits symbol $x$, the signal received at UE is given by 
\begin{equation}
    y = l(d_1,d_2)\mathbf{h}^T\mathbf{\Phi Hf}x + l(d_0)\mathbf{g}^T\mathbf{f}x +n,\label{RxSig}
\end{equation}
where $n \sim \mathcal{N}(0,\sigma_n^2)$ is the complex Gaussian noise, $\mathbf{f}\in\mathbb{C}^N$ is the transmit beamformer, $\mathbf{\Phi}={\rm diag}(\boldsymbol{\psi})$ is the RIS phase shift matrix, and $l(d_1,d_2)$ and $l(d_0)$ are the far field path loss functions for the indirect and direct links, respectively. The transmit power constraint at the BS for $\mathbb{E}[xx^H] = P_s$ implies $\|\mathbf{f}\|^2 = 1$ where $P_s$ is the total transmission power. Further, the consideration of passive elements for RIS implies that the entries of $\boldsymbol{\psi}$ are complex with unit magnitude, i.e., $|\boldsymbol{\psi}_k| = 1 ~~ \forall k = 0,\ldots, N-1$. Also, in a far field scenario, the path loss for the RIS channel follows the ``product of distances" model such that $l(d_1,d_2) = (d_1d_2)^{-\alpha/2}$ \cite{LiuYuanwei_RISprinciples_2021} and BS-UE channel path loss follows $l(d_0) = (d_0)^{-\alpha/2}$, where $d_0$, $d_1$ and $d_2$ are the distances of BS-UE, BS-RIS, and RIS-UE links, respectively, and $\alpha$ is the path loss exponent.

Using \eqref{RxSig}, SNR can be written as
\begin{equation}
    \text{SNR} = \gamma |\mathbf{h}^T\mathbf{\Phi Hf} + \mu \mathbf{g}^T\mathbf{f}|^2,\label{SNR}
\end{equation}
where $\gamma = (d_1d_2)^{-\alpha}\frac{P_s}{\sigma_n^2} $ and $\mu = (\frac{d_0}{d_1d_2})^{-\alpha/2}$ is the path loss ratio (PLR) of direct and indirect links. It may be noted that the PLR captures the strength of the direct link compared to the indirect link and thus it will be an important parameter for the optimal beamforming. % when the direct and indirect links are utilized jointly.
\vspace{-.4cm}\subsection{Problem Formulation}\label{prob from}\vspace{-.2cm}
This paper aims to maximize EC by jointly optimizing the transmit beamforming vector $\mathbf{f}$  and  RIS-phase shift matrix $\mathbf{\Phi}$. For given $\mathbf{f}$ and $\mathbf{\Phi}$, EC  is 
\begin{equation*}
    {\rm C} = \mathbb{E}[\log_2(1 + \gamma|\mathbf{h}^T\mathbf{\Phi Hf} + \mu \mathbf{g}^T\mathbf{f}|^2)].
\end{equation*}
However, the expectation of the log function is difficult to handle in the maximization problem. Thus, we apply Jensen's inequality %\footnote{From Jensen's inequality, we have: $\mathbb{E}[\log(1 + X)] \leq \log(1 + \mathbb{E}[X])$} 
and focus on maximizing the upper bound of capacity as 
\begin{align}
  {\rm C}\leq{\rm C_{ub}} &=  \log_2(1 + \Gamma(\mathbf{f},\mathbf{\Phi})),\text{~~where~~}
  \Gamma(\mathbf{f},\mathbf{\Phi}) = \gamma\mathbb{E}[|\mathbf{h}^T\mathbf{\Phi Hf} + \mu \mathbf{g}^T\mathbf{f}|^2]\label{obj_fun}
\end{align}
represents the mean {\rm SNR}. Henceforth, we will assume $\gamma=1$ without any loss of generality. Thus, the capacity maximization problem can be reformulated using its upper bound as
\begin{subequations}
\begin{align}
    \max_{\mathbf{f},\mathbf{\Phi}} ~~&  \Gamma(\mathbf{f},\mathbf{\Phi}),\label{objective}\\
   \text{s.t.} ~~& \|\mathbf{f}\|^2 = 1,\label{constraint_f}\\
    &|\boldsymbol{\psi}_k| = 1, ~~ \forall k = 0,\ldots,N-1.\label{constraint_phi}%
\end{align}
\label{optimization problem}%
\end{subequations}
\noindent where \eqref{constraint_f} is the unit norm constraint of the transmit beamformer and \eqref{constraint_phi} is the unit magnitude constraint on the passive RIS elements to ensure phase shifts without  amplification/attenuation.
The problem formulation given in \eqref{optimization problem} is non-convex due to \eqref{constraint_phi} and the objective function given in \eqref{obj_fun}. Further, the problem is coupled in terms of $\mathbf{f}$ and $\mathbf{\Phi}$ which makes it further difficult to solve the problem. Hence, to tackle this issue, we solve the optimization problem using alternating subproblems and provide iterative algorithms or closed-form solutions for optimal $\mathbf{f}$ and $\mathbf{\Phi}$ for various fading scenarios in section \ref{optimal beamforming}. Besides, we also decouple $\mathbf{f}$ and $\mathbf{\Phi}$ in some special cases of fading which allows us to get a closed-form solution $\mathbf{f}$ and $\mathbf{\Phi}$. 

In addition, we also analyze the \emph{outage probability} and \emph{ergodic capacity} to characterize the performance of the proposed SCSI-based beamforming schemes.
For a beamforming scheme with optimal beamformer $\mathbf{f}_{\rm opt}$ and phase shift matrix $\mathbf{\Phi}_{\rm opt}$, the OP and EC are given by
% \begin{equation}
%     \text{P}_{\rm out}(\beta) = \mathbb{P}[{\Gamma(\mathbf{f}_{\rm opt},\mathbf{\Phi}_{\rm opt})} \leq \beta],\label{P_out}
% \end{equation}
% and the ergodic capacity is given by
% \begin{equation}
%         \text{EC} = \mathbb{E}[\log_2(1+\Gamma(\mathbf{f}_{\rm opt},\mathbf{\Phi}_{\rm opt}))]=\frac{1}{\ln(2)}\int_0^\infty \frac{1}{1+u}\left(1-\text{P}_{\rm out}(u)\right){\rm d}u.\label{ergodic capacity}
% \end{equation}
\begin{align}
    {\rm P_{out}}(\beta) &= \mathbb{P}[{\Gamma(\mathbf{f}_{\rm opt},\mathbf{\Phi}_{\rm opt})} \leq \beta]~~\text{and}\label{P_out}\\
      {\rm C} = \mathbb{E}[\log_2(1+\Gamma&(\mathbf{f}_{\rm opt},\mathbf{\Phi}_{\rm opt}))]=\frac{1}{\ln(2)}\int_0^\infty \frac{1}{1+u}\left(1-{\rm P_{out}}(u)\right){\rm d}u,\label{ergodic capacity}
\end{align}
respectively. In the next section, we present the algorithms/solutions for optimal beamforming and also perform outage analysis for variants of the channel model discussed in section \ref{channel model}.
%---------------------------- PROBLEM SOLVING-----------------------%
\vspace{-.4cm}\section{Statistically Optimal Beamforming for RIS-aided systems }\label{optimal beamforming}\vspace{-.2cm}
In  our system, the RIS is positioned such that it has an LoS path with both the BS and the UE. Hence, Rician distribution is adopted to model the indirect links $\mathbf{H}$ and $\mathbf{h}$. Besides, we also consider an LoS Path at the direct link which is ignored in \cite{WangJinghe_CorrelatedFading_2021}. This makes the considered RIS-aided MISO system more general. 
For such a system, our objective is to maximize the upper bound on EC as discussed in Section \ref{prob from}. The objective function of problem \eqref{optimization problem}, i.e. the mean {\rm SNR}, can be written as
\begin{align}
   \Gamma(\mathbf{f},\mathbf{\Phi}) = |\kappa_l^2\boldsymbol{\psi}^T\mathbf{Ef} + \mu\kappa_l\mathbf{\Bar{g}}^T\mathbf{f}|^2 + \kappa_l^2\kappa_n^2\boldsymbol{\psi}^H\mathbf{Z}_1\boldsymbol{\psi} + \kappa_n^2[\mu^2+\boldsymbol{\psi}^H\mathbf{Z}_2\boldsymbol{\psi}]\mathbf{f}^H\mathbf{R}_{\rm BT}\mathbf{f},\label{eq:Mean_SNR_R1}
\end{align}
where $\mathbf{E}={\rm diag}(\mathbf{\Bar{h}})\mathbf{\Bar{H}}$, $\mathbf{Z}_1 = \mathbf{R}_{\rm RT}\odot\mathbf{\Bar{H}ff}^H\mathbf{\Bar{H}}^H$ and $\mathbf{Z}_2 = \mathbf{R}_{\rm RR}\odot( \kappa_n^2\mathbf{R}_{\rm RT}+\kappa_l^2\mathbf{\Bar{h}}^*\mathbf{\Bar{h}}^T)$. The proof of \eqref{eq:Mean_SNR_R1} is given in Appendix \ref{AppA}. As mentioned earlier, the non-convex nature of the problem makes it challenging to directly obtain optimal $\mathbf{f}$ and $\mathbf{\Phi}$. Therefore, we tackle this issue by dividing the problem into optimal beamformer and phase shift matrix sub-problems as follows\\
\emph{1) Optimal Beamformer}: For a given phase shift matrix $\mathbf{\Phi}$, the optimization problem with respect to the beamforming vector $\mathbf{f}$ becomes
\begin{subequations}
\begin{align}
    \max_{\mathbf{f}} ~~& \mathbf{f}^H\mathbf{F}\mathbf{f}, \label{subprob_f_R1}\\
\text{s.t.} ~~& \|\mathbf{f}\|^2 = 1,
\end{align}
\end{subequations}
where the objective function follows from \eqref{eq:Mean_SNR_R1} with
\begin{equation}
    \mathbf{F} = \mathbf{F}_1 + \mathbf{F}_2 + \mathbf{F}_3,\label{M}
\end{equation}
such that $\mathbf{F}_1= (\kappa_l^2\boldsymbol{\psi}^T\mathbf{E} + \mu\kappa_l\mathbf{\Bar{g}}^T)^H(\kappa_l^2\boldsymbol{\psi}^T\mathbf{E} + \mu\kappa_l\mathbf{\Bar{g}}^T)$, $\mathbf{F}_2=\kappa_l^2\kappa_n^2\mathbf{\Bar{H}}^H\mathbf{\Phi}^H\mathbf{R}_{\rm RT}\mathbf{\Phi\Bar{H}}$, and $\mathbf{F}_3=\kappa_n^2\mathbf{R}_{\rm BT}[\mu^2+\boldsymbol{\psi}^H\mathbf{Z}_2\boldsymbol{\psi}]$.
% \begin{align}
%     \mathbf{F}_1 = ~~& (\kappa_l^2\boldsymbol{\psi}^T\mathbf{E} + \mu\kappa_l\mathbf{\Bar{g}}^T)^H(\kappa_l^2\boldsymbol{\psi}^T\mathbf{E} + \mu\kappa_l\mathbf{\Bar{g}}^T),\nonumber\\
%     \mathbf{F}_2 = ~~& \kappa_l^2\kappa_n^2\mathbf{\Bar{H}}^H\mathbf{\Phi}^H\mathbf{R}_{\rm RT}\mathbf{\Phi\Bar{H}},\nonumber\\
%     \mathbf{F}_3 = ~~& \kappa_n^2\mathbf{R}_{\rm BT}[\mu^2+\boldsymbol{\psi}^H\mathbf{Z}_2\boldsymbol{\psi}].\nonumber      
%     \end{align} 
It is to be noted that  $\mathbf{F}_1$, $\mathbf{F}_2$, and $\mathbf{F}_3$ are symmetric matrices which implies that $\mathbf{F}$ is also a symmetric matrix. Thus, this optimization problem is equivalent to the Rayleigh quotient maximization, whose solution, i.e., the optimal transmit beamformer, becomes the dominant eigenvector of $\mathbf{F}$ and can be given as 
    \begin{equation}
        \mathbf{f}_{\rm opt} = \mathbf{v_F},\label{OptSol_f_R1}
    \end{equation}
where $\mathbf{v_F}$ is the dominant eigenvector of $\mathbf{F}$.\\
\emph{2) Optimal Phase Shift Matrix}: For a given beamforming vector $\mathbf{f}$, the optimization problem with respect to the phase shift matrix $\mathbf{\Phi} = \rm{diag}(\boldsymbol{\psi})$ becomes
\begin{subequations}
\begin{align}
    \max_{\boldsymbol{\psi}}  ~~& |\boldsymbol{\psi}^H\mathbf{a}+\mathbf{b}|^2 + \boldsymbol{\psi}^H\mathbf{V}\boldsymbol{\psi} \label{subprob_phi_R1},\\
   \text{s.t.} ~~& |\boldsymbol{\psi}_k| = 1 ~~ \forall k = 0,\ldots,N-1,\label{constraint_psi}
\end{align}\label{prob psi}%
\end{subequations}
where the objective function follows by rewriting \eqref{eq:Mean_SNR_R1} with $\mathbf{V}=\kappa_l^2\kappa_n^2\mathbf{Z}_1+\kappa_n^2\mathbf{f}^H\mathbf{R}_{\rm BT}\mathbf{f}\mathbf{Z}_2$, $ \mathbf{a}=\kappa_l^2\mathbf{Ef}$, and $\mathbf{b}= \mu\kappa_l\mathbf{\Bar{g}}^T\mathbf{f}$. %$\mathbf{V} = \mathbf{V}_1 + \mathbf{V}_2$ and
% \begin{align}
%     \mathbf{a} = ~~& \kappa_l^2\mathbf{Ef}, \nonumber\\
%     \mathbf{b} = ~~& \mu\kappa_l\mathbf{\Bar{g}}^T\mathbf{f},\nonumber\\
%     \mathbf{V}_1 = ~~& \kappa_l^2\kappa_n^2\mathbf{Z}_1,\nonumber\\
%     \text{and} \hspace{0.2cm}\mathbf{V}_2 = ~~& \kappa_n^2\mathbf{f}^H\mathbf{R}_{\rm BT}\mathbf{f}\mathbf{Z}_2.\nonumber
% \end{align}
% Since the unit modulus constraint given in \eqref{constraint_psi} is non-convex and reformulate the problem \eqref{prob psi} by introducing an auxiliary variable $\rm t$ as below
Since the above problem is non-convex, we model  \eqref{prob psi} as a semidefinite programming problem by introducing an auxiliary variable as below
\begin{subequations}
\begin{align}
    \max_{\Bar{\boldsymbol{\psi}}} ~~& \Bar{\boldsymbol{\psi}}^H\mathbf{A}\Bar{\boldsymbol{\psi}} + \|\mathbf{b}\|^2,\label{objective_psibar}\\
   \text{s.t.} ~~& |\Bar{\boldsymbol{\psi}_k}| = 1 ~~ \forall k = 0,\ldots,N-1,\label{constraint_psibar}
\end{align}\label{obj psibar}%
\end{subequations}
%\begin{align}
% \text{where}~~ \mathbf{A} = \begin{bmatrix}
%                  \mathbf{aa}^H+\mathbf{V} & \mathbf{ab}^H\\
%                  \mathbf{ba}^H & 0\label{MatrixA}
%                  \end{bmatrix}
%     \hspace{0.3cm}\text{and}\hspace{0.3cm}
%     \Bar{\boldsymbol{\psi}} = \begin{bmatrix}
%                           \boldsymbol{\psi}\\
%                           t
%                           \end{bmatrix}.
% \end{align}
where $\mathbf{A} = \begin{bmatrix}
                 \mathbf{aa}^H+\mathbf{V} & \mathbf{ab}^H\\
                 \mathbf{ba}^H & 0\label{MatrixA}
                 \end{bmatrix}$ and $\Bar{\boldsymbol{\psi}} = [ \boldsymbol{\psi}~~t]^T$.
Further, defining $\mathbf{\Psi} = \Bar{\boldsymbol{\psi}}\Bar{\boldsymbol{\psi}}^H$ such that ${\rm diag}(\mathbf{\Psi}) = 1$ will ensure the constraint in \eqref{constraint_psibar} and also 
allow us to rewrite \eqref{objective_psibar} as ${\rm tr}(\mathbf{A\Psi})$. %However, the problem still remains to be non-convex. 
Next, we use the standard SDR technique to solve the above problem as below  
\begin{subequations}
\begin{align}
    \max_{\mathbf{\Psi}} ~~& {\rm tr}(\mathbf{A\Psi}) ,\\
   \text{s.t.} ~~& \mathbf{\Psi} \succeq 0, \hspace{0.2cm}{\rm diag}(\mathbf{\Psi}) = 1.
\end{align}\label{relaxed_phi}%
\end{subequations}
This problem can be solved by using standard solvers such as CVX/Mosek. Finally, the optimization problem given in \eqref{optimization problem} can now be solved by iterating over the beamforming and phase shift matrix subproblems using \eqref{subprob_f_R1} and \eqref{relaxed_phi}, respectively, as summarized in Algorithm \ref{Alg1}.
\begin{algorithm}\label{Alg1}
% \caption{An algorithm with caption}
\SetKwComment{Comment}{$\triangleright$\ }{}
\KwInput{$\mu$, $\kappa_l$, $\kappa_n$, $\mathbf{\Bar{h}}$, $\mathbf{\Bar{H}}$, $\mathbf{\Bar{g}}$, $\mathbf{R}_{\rm RT}$, $\mathbf{R}_{\rm RR}$, $\mathbf{R}_{\rm BT}$, $\delta$}
  % \KwOutput{$\mathbf{f}_{\rm opt}$ , $\boldsymbol{\psi_{\rm opt}}$}
  \KwInit{$\mathbf{f_0}$ , $\boldsymbol{\psi_0}$}
  %repeat\\ 
\SetKwRepeat{Repeat}{Repeat}{Untill:}
\Repeat{$|\Gamma(\mathbf{f_{i}},\mathbf{\Phi_{i}})-\Gamma(\mathbf{f_{i-1}},\mathbf{\Phi_{i-1}})| \leq \delta$}{Set $\mathbf{f}=\mathbf{f_{i-1}}$ and evaluate $\mathbf{A}$ using \eqref{MatrixA}.\\
  Using $\mathbf{A}$, solve  \eqref{relaxed_phi} for $\mathbf{\Psi}$. %\Comment*[l]{\textcolor{blue}{Standard Solvers}}
  Obtain $\boldsymbol{\psi_i}$ such that $\left[ \boldsymbol{\psi_i~}  t\right]^T\left[ \boldsymbol{\psi_i}^* ~ t^*\right]=\mathbf{\Psi}.$\\
  % $\begin{bmatrix}
  %     \boldsymbol{\psi_i}\\
  %     t
  % \end{bmatrix}
  %  \begin{bmatrix}
  %      \boldsymbol{\psi_i} & t
  % \end{bmatrix}$
  % = $\mathbf{\Psi}.$ \\
  Set $\boldsymbol{\psi}=\boldsymbol{\psi_i}$ and evaluate $\mathbf{F}$ using \eqref{M}.\\
  Obtain $\mathbf{f_{i}}$ using \eqref{OptSol_f_R1} such that $ \mathbf{f_i} = \mathbf{v_F}$. \\
  $\mathbf{i} \gets \mathbf{i}+1$.}
\caption{SCSI-based optimal beamforming for R1 correlated fading}
\end{algorithm}
For the proposed beamforming scheme above, we now present OP and EC. In particular, we analyze the outage for a given $\mathbf{f}$ and $\mathbf{\Phi}$ due to not having closed-form expressions.
% Now, we present the outage and ergodic capacity analysis for the optimal beamforming scheme for  correlated Rician-Rician fading as proposed in Algorithm \ref{Alg1}. As  we don't have closed-form solutions for the optimal beamforming, we will first perform the outage and capacity analysis for given $\mathbf{f}$ and $\mathbf{\Phi}$.
Defining  $\xi_1 = \mathbf{h}^T\mathbf{\Phi Hf}$ and $\xi_2 =  \mathbf{g}^T\mathbf{f}$ will allow to rewrite the OP given in \eqref{P_out}  as
\begin{equation*}
    \text{P}_{\rm out}(\beta) = \mathbb{P}\left[|\xi_1 + \mu \xi_2| \leq \sqrt{\beta/\gamma}\right].
\end{equation*}
 In  Appendix \ref{AppB}, we show that $\xi_1$ (closely) and $\xi_2$ follow complex Gaussian distributions as 
\begin{equation}
\xi_1\sim\mathcal{C}\mathcal{N}(\mu_1,\sigma_1^2)\hspace{0.2cm}\text{and}\hspace{0.2cm} 
    \xi_2\sim\mathcal{C}\mathcal{N}(\mu_2,\sigma_2^2),\label{eq:xi1xi2}
\end{equation}
\begin{align*}
\text{where}~~ \mu_1 =\kappa_l^2\mathbf{\Bar{h}}^T\mathbf{\Phi \bar{H}f} &\text{~and~}\sigma_1^2 =\kappa_l^2\kappa_n^2\boldsymbol{\psi}^H\mathbf{Z}_1\boldsymbol{\psi} + \kappa_n^2\mathbf{f}^H\mathbf{R}_{\rm BT}\mathbf{f}[\boldsymbol{\psi}^H\mathbf{Z}_2\boldsymbol{\psi}],\\ 
    \mu_2 = \kappa_l\mathbf{\Bar{g}}^T\mathbf{f} &\text{~and~}\sigma_2^2 =\kappa_n^2\mathbf{f}^H\mathbf{R}_{\rm BT}\mathbf{f}.
\end{align*} 
% \begin{subequations}
% \begin{align}
%     \xi_1\sim\mathcal{C}\mathcal{N}(\mu_1,\sigma_1^2)~\text{where}~ \mu_1 =\kappa_l^2\mathbf{\Bar{h}}^T&\mathbf{\Phi \bar{H}f} ~\text{and}~\sigma_1^2 =\kappa_l^2\kappa_n^2\boldsymbol{\psi}^H\mathbf{Z}_1\boldsymbol{\psi} + \kappa_n^2\mathbf{f}^H\mathbf{R}_{\rm BT}\mathbf{f}[\boldsymbol{\psi}^H\mathbf{Z}_2\boldsymbol{\psi}],\\ 
% \text{and}~\xi_2\sim\mathcal{C}\mathcal{N}(\mu_2,\sigma_2^2)~&\text{where}~\mu_2 = \kappa_l\mathbf{\Bar{g}}^T\mathbf{f}~\text{and}~\sigma_2^2 =\kappa_n^2\mathbf{f}^H\mathbf{R}_{\rm BT}\mathbf{f}.
% \end{align}\label{eq:xi1xi2}%
% \end{subequations}
Using the independence of $\xi_1$ and $\xi_2$, we get
\begin{equation}
    \xi_1 + \mu\xi_2\sim\mathcal{C}\mathcal{N}(m,\sigma^2)
~~\text{where}~~m = \mu_1+\mu\mu_2,
  \text{~~and~~}  \sigma^2 = \sigma_1^2+\mu^2\sigma_2^2,  \label{parameters_m_sigma}
\end{equation} 
are the mean and variance of $\xi_1 + \mu \xi_2$, respectively.
Further, we also use the fact that the magnitude of a non-zero mean complex Gaussian follows the Rice distribution to obtain
\begin{equation}
    |\xi_1 + \mu\xi_2|\sim{\rm Rice}\left(|m|,\sigma\right).\label{SNR_dist}
\end{equation}
From \eqref{SNR_dist}, we can write OP as given in the following theorem using the  CDF of the Rice distribution with parameters $m$ and $\sigma^2$ given in \eqref{parameters_m_sigma} which are evaluated  using the optimal beamformer $\mathbf{f}_{\rm opt}$ and phase shift matrix $\mathbf{\Phi}_{\rm opt}$ obtained through Algorithm \ref{Alg1}. % will provide the outage performance of the SCSI-based optimal beamforming for this fading scenario as is summarized in the following theorem. 
%Using this, we present the outage performance of the proposed beamforming scheme in Algorithm \ref{Alg1} in  Theorem \ref{Theo1}. 
\begin{theorem}\label{Theo1}
OP of the SCSI-based optimal beamforming scheme for the RIS-aided MISO system under correlated Rician-Rician fading is given by
\begin{align}
    {\rm P_{out}}(\beta)\approx 1-Q_1\left(\frac{|m|}{\sqrt{\sigma/2}},\frac{\sqrt{\beta/\gamma}}{\sqrt{\sigma/2}}\right),\label{Pout_R1}
\end{align}
where $m =  \kappa_l^2\mathbf{\Bar{h}}^T\mathbf{\Phi_{\rm opt} \mathbf{\Bar{H}} f_{\rm opt}} + \mu \kappa_l\mathbf{\Bar{g}}^T\mathbf{f}_{\rm opt}$, $\sigma^2=  \kappa_l^2\kappa_n^2\boldsymbol{\psi}_{\rm opt}^H\mathbf{Z}_1\boldsymbol{\psi}_{\rm opt} + \kappa_n^2\mathbf{f}_{\rm opt}^H\mathbf{R}_{\rm BT}\mathbf{f}_{\rm opt}[\boldsymbol{\psi}_{\rm opt}^H\mathbf{Z}_2\boldsymbol{\psi}_{\rm opt}]+ \mu^2 \kappa_n^2\mathbf{f}_{\rm opt}^H\mathbf{R}_{\rm BT}\mathbf{f}_{\rm opt}$,  
% \begin{align*}
%     m &=  \kappa_l^2\mathbf{\Bar{h}}^T\mathbf{\Phi_{\rm opt} \mathbf{\Bar{H}} f_{\rm opt}} + \mu \kappa_l\mathbf{\Bar{g}}^T\mathbf{f}_{\rm opt},%\label{eq:parameter_m_R1Corr}
%     \\
%     \text{and~~}\sigma^2 &=  \kappa_l^2\kappa_n^2\boldsymbol{\psi}_{\rm opt}^H\mathbf{Z}_1\boldsymbol{\psi}_{\rm opt} + \kappa_n^2\mathbf{f}_{\rm opt}^H\mathbf{R}_{\rm BT}\mathbf{f}_{\rm opt}[\boldsymbol{\psi}_{\rm opt}^H\mathbf{Z}_2\boldsymbol{\psi}_{\rm opt}]+ \mu^2 \kappa_n^2\mathbf{f}_{\rm opt}^H\mathbf{R}_{\rm BT}\mathbf{f}_{\rm opt}%\label{eq:parameter_sigma2_R1Corr},
% \end{align*}
 $\mathbf{f}_{\rm opt}$ and $\mathbf{\Phi}_{\rm opt}$ are solutions of Algorithm \ref{Alg1}, and  $Q_1(\cdot)$ is a Marcum Q-function.
\end{theorem}
% \begin{proof}
%     The proof directly follows from \eqref{SNR_dist} wherein the parameters $m$ and $\sigma^2$ given in \eqref{parameters_m_sigma} are evaluated using the optimal beamformer $\mathbf{f}_{\rm opt}$ and phase shift matrix $\mathbf{\Phi}_{\rm opt}$ that are obtained through Algorithm \ref{Alg1}.
% \end{proof}
Using Theorem \ref{Theo1} and \eqref{ergodic capacity}, we determine EC of the proposed beamforming scheme  in Algorithm \ref{Alg1} in the following corollary.
\begin{corollary}
\label{cor:R1_EC}
EC of the SCSI-based optimal beamforming scheme for the RIS-aided MISO system under correlated Rician-Rician fading is given by
\begin{align}
    {\rm C} \approx \frac{1}{\ln(2)}\int_0^\infty \frac{1}{1+u}Q_1\left(\frac{|m|}{\sqrt{\sigma/2}},\frac{\sqrt{{u}/{\gamma}}}{\sqrt{\sigma/2}}\right){\rm d}u.
\end{align}
where $m$ and $\sigma^2$ are given in Theorem \eqref{Theo1}.
\end{corollary}

Now, we will study the impact of the limiting cases of fading factor $K$ on the SCSI-based optimal  beamforming  and its performance.   \newline
{\em 1) Case $K\to\infty$:} The multipath fading vanishes  as $K$ becomes large for which the resulting channels are described by their deterministic LoS components such that $\mathbf{g}=\mathbf{\Bar{g}}$, $\mathbf{h} = \mathbf{\Bar{h}}$, and $\mathbf{H} = \mathbf{\Bar{H}}$.  For this case, the  {\rm SNR} reduces to a deterministic value for which  \eqref{obj_fun} becomes 
\begin{equation}
\Gamma(\mathbf{f},\mathbf{\Phi}) = |\mathbf{\Bar{h}}^T\mathbf{\Phi\Bar{H}f} + \mu\mathbf{\Bar{g}}^T\mathbf{f}|^2 = |\boldsymbol{\psi}^T\mathbf{Ef} + \mu\mathbf{\Bar{g}}^T\mathbf{f}|^2.\label{eq:R1_Cor_K_inftt}
\end{equation}
 To maximize \eqref{eq:R1_Cor_K_inftt} with constraints \eqref{constraint_f} and \eqref{constraint_phi}, we can easily find
 \begin{align}
     \mathbf{f}_{\rm opt}=\frac{\mathbf{E}^H\boldsymbol{\psi}^*_{\rm opt} + \mu\mathbf{\Bar{g}}^*}{\|\mathbf{E}^H\boldsymbol{\psi}^*_{\rm opt} + \mu\mathbf{\Bar{g}}^*\|}\text{~~and~~}\boldsymbol{\psi}_{\rm opt}=\exp\left(-j\angle{\mathbf{E}\mathbf{f}_{\rm opt}} + j\angle{\mathbf{\Bar{g}}^T\mathbf{f}_{\rm opt}}\right).\label{eq:R1_K_infty}
 \end{align}
 As the optimal $\mathbf{f}_{\rm opt}$ and $\boldsymbol{\psi}_{\rm opt}$ given in \eqref{eq:R1_K_infty} depend on each other, they can be solved alternatively for a fixed point solution.
 \newline {\em 2) Case $K\to 0$:} In this case, the LoS components become insignificant and the channels get completely characterized   by the multipath fading  such that $\mathbf{g}=\mathbf{\tilde{g}}$, $\mathbf{h} = \mathbf{\tilde{h}}$, and $\mathbf{H} = \mathbf{\tilde{H}}$. This is equivalent to the correlated Rayleigh-Rayleigh fading model for which the optimal beamforming and its performance will be presented in  Section \ref{R3}.
 
 %For this case, the mean {\rm SNR} given in \eqref{obj_fun} reduces to 
%      \begin{equation}
%     \Gamma(\mathbf{f}, \mathbf{\Phi}) = \mathbf{f}^H\mathbf{R}_{\rm BT}\mathbf{f}[\boldsymbol{\psi}^H(\mathbf{R}_{\rm RR}\odot\mathbf{R}_{\rm RT})\boldsymbol{\psi}+\mu^2].
% \end{equation}
%     This optimization problem is equivalent to correlated Rayleigh-Rayleigh fading which has been solved in Section \ref{R3}.
% \begin{itemize}
    % \item As $K \rightarrow \infty$:\\
    % \begin{align}
    %    \kappa_l = \sqrt{\frac{K}{1+K}} \rightarrow 1,
    %    \kappa_n = \sqrt{\frac{1}{1+K}} \rightarrow 0,
    %    \text{and}\hspace{0.2cm} \kappa_l^2\kappa_n^2 = \frac{K}{K^2+2K+1} \rightarrow 0
    % \end{align}
    % Now, the channels between BS-UE, RIS-UE and BS-RIS reduces to
    % \begin{align}
    %     \mathbf{g} = \mathbf{\Bar{g}},\hspace{0.2cm}  \mathbf{h} = \mathbf{\Bar{h}}, \text{and}\hspace{0.2cm}  \mathbf{H} = \mathbf{\Bar{H}}
    % \end{align}
    %  For this case, the mean {\rm SNR} given in \eqref{obj_fun} reduces to 
    %  \begin{equation}
    %  \Gamma(\mathbf{f},\mathbf{\Phi}) = \gamma|\kappa_l^2\mathbf{\Bar{h}}^T\mathbf{\Phi\Bar{H}f} + \mu\kappa_l\mathbf{\Bar{g}}^T\mathbf{f}|^2
    % \end{equation}
    % This optimization problem is equivalent to {\rm i.i.d.} Rician-Rician fading which has been solved using an iterative algorithm in \cite{hu2020statistical} and closed-form expressions for $\mathbf{f}$ and $\mathbf{\Phi}$ have been obtained to maximize the lower bound on mean {\rm SNR} in Section \ref{R1_IID}.
%     \item As $K \rightarrow 0$:\\
%     \begin{align}
%        \kappa_l = \sqrt{\frac{K}{1+K}} \rightarrow 0,
%        \kappa_n = \sqrt{\frac{1}{1+K}} \rightarrow 1,
%        \text{and}\hspace{0.2cm} \kappa_l^2\kappa_n^2 = \frac{K}{K^2+2K+1} \rightarrow 0
%     \end{align}
%     Now, the channels between BS-UE, RIS-UE and BS-RIS reduces to
%     \begin{align}
%         \mathbf{g} = \mathbf{\Tilde{g}},\hspace{0.2cm}  \mathbf{h} = \mathbf{\Tilde{h}}, \text{and}\hspace{0.2cm}  \mathbf{H} = \mathbf{\Tilde{H}}
%     \end{align}
%      For this case, the mean {\rm SNR} given in \eqref{obj_fun} reduces to 
%      \begin{equation}
%     \Gamma(\mathbf{f}, \mathbf{\Phi}) = \mathbf{f}^H\mathbf{R}_{\rm BT}\mathbf{f}[\boldsymbol{\psi}^H(\mathbf{R}_{\rm RR}\odot\mathbf{R}_{\rm RT})\boldsymbol{\psi}+\mu^2].
% \end{equation}
%     This optimization problem is equivalent to correlated Rayleigh-Rayleigh fading which has been solved in Section \ref{R3}.
% \end{itemize}


The outage and capacity performances given in Theorem \ref{Theo1} and Corollary \ref{cor:R1_EC} allow us to numerically evaluate the system performance as they rely on $\mathbf{f}_{\rm opt}$ and $\mathbf{\Phi}_{\rm opt}$ which are obtained using Algorithm \ref{Alg1}.  Thus, though they are expressed in a simple analytical form, their evaluation is limited by a computationally complex algorithm. Such numerical solutions may not always provide useful insights into the exact functional dependence of the optimal solution on the key system parameters.
Thus, it is desirable to have a closed-form expression for the optimal beamformer $\mathbf{f}$ and $\mathbf{\Phi}$ so that the outage and capacity performances can be characterized analytically without relying on the numerical evaluation of an algorithm. Additionally, having closed-form expressions for optimal $\mathbf{f}$ and $\mathbf{\Phi}$ will help to reduce the implementation complexity. Motivated by this, we investigate special cases of the channel model given in Section \ref{channel model} for deriving closed-form expressions or computationally efficient solutions  in the following subsections.
\vspace{-.5cm}
\subsection{IID Rician-Rician Fading}\label{R1_IID}
In this subsection, we focus on solving the optimization problem in \eqref{optimization problem} while considering an {\rm i.i.d.} Rician-Rician fading model for direct and indirect links which implies that $\mathbf{\Tilde{g}}\sim\mathcal{C}\mathcal{N}(0,\mathbf{I_M})$, $\mathbf{\Tilde{h}}\sim\mathcal{C}\mathcal{N}(0,\mathbf{I_N})$ and $\mathbf{\Tilde{H}}_{:,i}\sim\mathcal{C}\mathcal{N}(0,\mathbf{I_N})$. For this case, by simplifying the steps given in Appendix \ref{AppA} with identity covariance matrices, we can write the  mean {\rm SNR} given in \eqref{obj_fun} as
\begin{equation}
    \Gamma(\mathbf{f},\mathbf{\Phi}) = |\kappa_l^2\mathbf{\Bar{h}}^T\mathbf{\Phi\Bar{H}f} + \mu\kappa_l\mathbf{\Bar{g}}^T\mathbf{f}|^2 + \kappa_l^2\kappa_n^2\|\mathbf{\Bar{H}f}\|^2 +  (\kappa_l^2\kappa_n^2+\kappa_n^4)N + \mu^2\kappa_n^2.  \label{OptProb_R1_IID}
\end{equation}
 Maximization of mean SNR given in \eqref{OptProb_R1_IID}  has already been addressed by the authors of \cite{hu2020statistical}. 
However, they proposed an alternating optimization-based approach as given in Algorithm \ref{Alg2}. While  Algorithm \ref{Alg2} is computationally efficient compared Algorithm \ref{Alg1}, the system performance analysis is still limited by the numerical evaluation of  the optimal beamforming. 
\begin{algorithm}[t!]\label{Alg2}
\caption{SCSI-based optimal beamforming for IID R1  case \cite{hu2020statistical}}
\SetKwComment{Comment}{$\triangleright$\ }{}
\KwInput{$\mu$, $\kappa_l$, $\kappa_n$, $\mathbf{\Bar{h}}$, $\mathbf{\Bar{H}}$, $\mathbf{\Bar{g}}$, $\delta$}
\KwOutput{$\mathbf{f}_{\rm opt}$ , $\boldsymbol{\psi_{\rm opt}}$}
  \KwInit{$\mathbf{f}_0$ , $\boldsymbol{\psi}_0$}
\SetKwRepeat{Repeat}{Repeat}{Untill:}
\Repeat{$|\Gamma(\mathbf{f}_{i},\mathbf{\Phi}_{i})-\Gamma(\mathbf{f}_{i-1},\mathbf{\Phi}_{i-1})| \leq \delta$}{$\mathbf{Y}=\begin{bmatrix}
      \kappa_l^2\boldsymbol{\psi}_{i-1}^T{\rm diag}(\mathbf{\Bar{h}})\mathbf{\Bar{H}} + \mu\kappa_l\mathbf{\Bar{g}}^T~~
      \kappa_l\kappa_n\mathbf{\Bar{H}}
  \end{bmatrix}^T$ and set $\mathbf{f}_{i}=\mathbf{v_Y}$.\\
  Set $\boldsymbol{\psi}_{i}=\exp\{-j\angle{\left({\rm diag}(\mathbf{\bar{h}})\mathbf{\bar{H}}\mathbf{f}_{i} - \mathbf{\bar{g}}^T\mathbf{f}_{i}\right)}\}$.\\
%For the given transmit beam $\mathbf{f_i}$, calculate the optimal phase shift beam according to (14) in \cite{hu2020statistical}, which yields $\mathbf{\Phi_{i+1}}$.\\
%For the given phase shift beam  $\mathbf{\Phi_{i+1}}$, compute the optimal transmit beam according to (17) in \cite{hu2020statistical}, which yields $\mathbf{f_{i+1}}$.\\
$\mathbf{i} \gets \mathbf{i}+1$.}
\end{algorithm}
% However, they proposed an alternating optimization-based approach which iterates over optimal solutions of $\mathbf{f}$ and $\mathbf{\Phi}$ as given below
% \begin{align}
%     \mathbf{f}=\mathbf{v_Y}\text{~~and~~} \boldsymbol{\psi}=\exp\{-j\angle{\left({\rm diag}(\mathbf{\bar{h}})\mathbf{\bar{H}}\mathbf{f} - \mathbf{\bar{g}}^T\mathbf{f}\right)}\}, \label{Alg2}
% \end{align}
%  where $\mathbf{Y}=\begin{bmatrix}
%       \kappa_l^2\boldsymbol{\psi}^T{\rm diag}(\mathbf{\Bar{h}})\mathbf{\Bar{H}} + \mu\kappa_l\mathbf{\Bar{g}}^T~~
%       \kappa_l\kappa_n\mathbf{\Bar{H}}
%   \end{bmatrix}^T$.
%While  this approach is computationally efficient compared Algorithm \ref{Alg1}, the system performance analysis is still limited by the numerical evaluation of  the optimal beamforming. To tackle this issue, we consider maximizing  a lower bound of mean {\rm SNR} with the goal of obtaining closed-form optimal beamforming expressions for $\mathbf{f}$ and $\mathbf{\Phi}$. 
Thus, to obtain the closed form solution, we  maximize a carefully constructed lower bound of  the mean SNR.  For this, we first optimize it w.r.t. $\mathbf{\Phi}$ as below. \\
%For this, we first obtain $\mathbf{\Phi}_{\rm opt}$ for a given $\mathbf{f}$. Next, we will use $\mathbf{\Phi}_{\rm opt}$ to obtain the maximum of the objective function (with respective to $\mathbf{\Phi}$) and solve for obtaining $\mathbf{f}_{\rm opt}$}.\\
\emph{1) Optimal Phase Shift Matrix}: Using the mean SNR given in \eqref{OptProb_R1_IID}, the optimization problem \eqref{optimization problem} with respect to the phase shift matrix $\mathbf{\Phi}$ for a given $\mathbf{f}$ becomes 
\begin{subequations}
\begin{align}
    \max_{\boldsymbol{\psi}}~~ &|\kappa_l^2 \boldsymbol{\psi}^T{\rm diag}(\mathbf{\bar{h}})\Bar{\mathbf{H}}\mathbf{f} +\mu \kappa_l \bar{\mathbf{g}}^T\mathbf{f}|^2\label{subprob_Phi_R1_IID} \\
    {\rm\text{s.t.}}~~&|\boldsymbol{\psi}_k|=1, ~~\forall k=0,\cdots,N-1,\label{constraint_Phi_R1_IID}
\end{align}
\end{subequations} 
where \eqref{subprob_Phi_R1_IID} follows from \eqref{OptProb_R1_IID}. % and $\bar{\mathbf{h}}^T\mathbf{\Phi} = \boldsymbol{\psi}^T{\rm diag}(\mathbf{\bar{h}})$. 
The above objective function can be upper bounded as
\begin{align*}
     |\kappa_l^2 \boldsymbol{\psi}^T\rm{diag}(\bar{\mathbf{h}}) \bar{\mathbf{H}} \mathbf{f} +\mu \kappa_l \bar{\mathbf{g}}^T \mathbf{f}|^2 \leq  \kappa_l^4 |\boldsymbol{\psi}^T{\rm diag}(\mathbf{\bar{h}}) \bar{\mathbf{H}} \mathbf{f}|^2 +\mu^2 \kappa_l^2 |\bar{\mathbf{g}}^T\mathbf{f}|^2,
\end{align*}
where equality holds when $\boldsymbol{\psi}^T{\rm diag}(\mathbf{\bar{h}})\bar{\mathbf{H}}\mathbf{f} = c  \bar{\mathbf{g}}^T\mathbf{f}$ for a constant $c$. To achieve  equality, we set 
\begin{align}
 \boldsymbol{\psi}^T=c  \bar{\mathbf{g}}^T\mathbf{f} \mathbf{w},\label{psi^t} 
\end{align}
where $\mathbf{w}=\mathbf{f}^H\mathbf{E}^H/\|\mathbf{Ef}\|^2$ is the pseudoinverse of $\mathbf{Ef}$ such that $\mathbf{E}={\rm diag}(\mathbf{\bar{h}})\mathbf{\Bar{H}}$. However, the constraint in \eqref{constraint_Phi_R1_IID} needs to be satisfied. Interestingly, as  will be clear shortly, we have $|\mathbf{w}_i|=|\mathbf{w}_j|$ $\forall i,j$. Thus, we can set $c$ such that  \eqref{constraint_Phi_R1_IID} is ensured. 
Let $\mathbf{e}_n=\mathbf{E}_{n,:}$ be the $n$-th row of $\mathbf{E}$.
By  construction of $\bar{\mathbf{h}}$ and $\bar{\mathbf{H}}$ (see Section \ref{channel model}), we have $\mathbf{e}_n^H\mathbf{e}_n=\mathbf{e}_m^H\mathbf{e}_m$ using which we get
 \begin{align}
 \mathbf{f}^H\mathbf{e}_n^H\mathbf{e}_n\mathbf{f}=\mathbf{f}^H\mathbf{e}_m^H\mathbf{e}_m\mathbf{f} \Rightarrow \|\mathbf{e}_n\mathbf{f}\|^2=\|\mathbf{e}_m\mathbf{f}\|^2.\label{emf}
 \end{align}
 In addition, we also observe that 
\begin{align}
  \|\mathbf{E}\mathbf{f}\|^2=\sum\nolimits_{n=0}^{N-1}|\mathbf{e}_n\mathbf{f}|^2=N|\mathbf{e}_n\mathbf{f}|^2.\label{Ef}
\end{align}
Using \eqref{emf} and \eqref{Ef}, we can write 
\begin{equation}|\bar{\mathbf{g}}^T\mathbf{f}\mathbf{w}_k|=\frac{|\bar{\mathbf{g}}^T\mathbf{f}|}{N|\mathbf{e}_n\mathbf{f}|},\label{gfw}
\end{equation}
where $|\mathbf{w}_k|=\frac{1}{N|\mathbf{e}_n\mathbf{f}|}$ for $\forall k$. 
Finally, by substituting \eqref{gfw} in \eqref{psi^t}, we obtain the optimal RIS phase shift vector with unit magnitude elements as
  \begin{align}
      {\boldsymbol{\psi}}_{\rm opt}^T=\frac{N|\mathbf{e}_n\mathbf{f}|}{|\mathbf{\bar{g}}^T\mathbf{f}|}\mathbf{\bar{g}}^T\mathbf{f}\mathbf{w}\label{OptSol_Phi_R1_IID}.
  \end{align}
\emph{2) Optimal Beamformer}: For a given phase shift matrix $\mathbf{\Phi}_{\rm opt}$, the optimization problem with respect to the beamforming vector $\mathbf{f}$ becomes
\begin{subequations}
\begin{align} 
    \max_{\mathbf{f}}~~ &|\kappa_l^2 \boldsymbol{\psi}_{\rm opt}^T{\rm diag}(\mathbf{\bar{h}}) \bar{\mathbf{H}} \mathbf{f} +\mu \kappa_l \bar{\mathbf{g}}^T\mathbf{f}|^2 + \kappa_l^2\kappa_n^2\|\bar{\mathbf{H}}\mathbf{f}\|^2,\label{subprob_f_R1_IID}\\
   \text{s.t.}~~&\|\mathbf{f}\|=1.
\end{align}   
\end{subequations}
To solve \eqref{subprob_f_R1_IID}, we start by substituting $\boldsymbol{\psi}_{\rm opt}$ in the first term of \eqref{subprob_f_R1_IID} as follows
 \begin{align}
     |\kappa_l^2 {\boldsymbol{\psi}_{\rm opt}}^T\mathbf{Ef} +\mu \kappa_l \mathbf{\bar{g}}^T\mathbf{f}|^2
     &=\kappa_l^4 |{\boldsymbol{\psi}_{\rm opt}}^T\mathbf{Ef}|^2 +\mu^2 \kappa_l^2 |\mathbf{\bar{g}}^T\mathbf{f}|^2+2\kappa_l^3\mu |{\boldsymbol{\psi}_{\rm opt}}^T\mathbf{Ef}||\mathbf{\bar{g}}^T\mathbf{f}|,\nonumber\\
     &\stackrel{(a)}{=}N^2\kappa_l^4|\mathbf{e}_n \mathbf{f}|^2 +\mu^2 \kappa_l^2 |\mathbf{\bar{g}}^T\mathbf{f}|^2+2N\kappa_l^3\mu |\mathbf{e}_n\mathbf{f}||\mathbf{\bar{g}}^T\mathbf{f}|,\label{term1}
 \end{align}
where step (a) follows from $\mathbf{wEf}=1$ and \eqref{psi^t}. 
The second term of \eqref{subprob_f_R1_IID} is simplified as
 % \begin{align}     \|\bar{\mathbf{H}}\mathbf{f}\|^2&=\mathbf{f}^H\bar{\mathbf{H}}^H\bar{\mathbf{H}}\mathbf{f},\nonumber\\
 %     ~~&=\mathbf{f}^H\bar{\mathbf{H}}^H{\rm diag}(\bar{\mathbf{h}})^H{\rm diag}(\bar{\mathbf{h}})\bar{\mathbf{H}}\mathbf{f},\nonumber\\
 %     ~~&=\mathbf{f}^H\mathbf{E}^H\mathbf{Ef},\nonumber\\
 %     ~~&=N|\mathbf{e}_n\mathbf{f}|^2\label{term2}
 % \end{align}
 \vspace{-0.2cm}
 \begin{align}     \|\bar{\mathbf{H}}\mathbf{f}\|^2&=\mathbf{f}^H\bar{\mathbf{H}}^H\bar{\mathbf{H}}\mathbf{f}=\mathbf{f}^H\bar{\mathbf{H}}^H{\rm diag}(\bar{\mathbf{h}})^H{\rm diag}(\bar{\mathbf{h}})\bar{\mathbf{H}}\mathbf{f},=\mathbf{f}^H\mathbf{E}^H\mathbf{Ef}=N|\mathbf{e}_n\mathbf{f}|^2,\label{term2}
 \end{align}
where  the second equality follows from  ${\rm diag}(\bar{\mathbf{h}})^H{\rm diag}(\bar{\mathbf{h}})=\mathbf{I}_{\rm N}$ and the last equality follows from \eqref{Ef}, respectively. Combining \eqref{term1} and \eqref{term2}, the objective given in \eqref{subprob_f_R1_IID} becomes
\begin{align}
w_1|\mathbf{e}_n\mathbf{f}|^2+w_2|\bar{\mathbf{g}}^T\mathbf{f}|^2+w_3|\mathbf{e}_n\mathbf{f}||\bar{\mathbf{g}}^T\mathbf{f}| = 
w_1\mathbf{f}^H\mathbf{E}_1\mathbf{f}+w_2\mathbf{f}^H\mathbf{Gf}+w_3|\mathbf{f}^H\mathbf{E}_2\mathbf{f}|,\label{modified_objfun}
\end{align} 
 where $\mathbf{E}_1=\mathbf{e}_n^H\mathbf{e}_n$, $\mathbf{G}=\bar{\mathbf{g}}^*\bar{\mathbf{g}}^T$, $\mathbf{E}_2=\mathbf{e}_n^H\bar{\mathbf{g}}^T$,  $w_1=N^2\kappa_l^4+N\kappa_l^2\kappa_n^2$, $w_2=\mu^2\kappa_l^2$, and $w_3=2N\mu\kappa_l^3$.\\
 It is to be noted that $\mathbf{E}_1$ and $\mathbf{G}$ are  symmetric and positive semidefinite matrices, whereas the matrix $\mathbf{E}_2$ is a negative definite matrix. Thus, the presence of the third term in \eqref{modified_objfun} makes the problem \eqref{subprob_f_R1_IID} non-convex. For this reason, we ignore the last term in \eqref{modified_objfun} from the maximization problem. This new objective will be equivalent to maximizing the lower bound on the mean {\rm SNR}.  It  is to be noted that this lower bound will be tight due to the following two reasons: 1) $w_1 \gg w_3$ and $w_2 \gg w_3$ and 2) the eigenvalues of  $\mathbf{E}_1$ and $\mathbf{G}$ are larger than the  $|\mathbf{f}^H\mathbf{E}_2\mathbf{f}|$. 
Using these arguments, we simplify the   problem for  maximizing the lower bound of mean {\rm SNR} as 
\vspace{-0.2cm}
\begin{align}
    \max_{\mathbf{f}} ~~& \mathbf{f}^H\mathbf{Zf},\\
   \text{s.t.} ~~& \|\mathbf{f}\|^2 = 1,
\end{align}
where  $\mathbf{Z}=w_1\mathbf{E}_1+w_2\mathbf{G}$ is a symmetric matrix. Thus, this optimization problem is equivalent to the Rayleigh quotient maximization, whose solution, i.e. the optimal beamformer,  becomes the dominant eigenvector of $\mathbf{Z}$ and can be given as 
\vspace{-0.3cm}
\begin{equation}
    \mathbf{f}_{\rm opt} = \mathbf{v_Z}.\label{OptSol_f_R1_iid}
\end{equation}
% For above $\mathbf{f}_{\rm opt}$, the optimal phase shift vector $\boldsymbol{\psi}_{\rm opt}$ becomes
% \begin{align}
% \boldsymbol{\psi}_{\rm opt}^T= \frac{N|\mathbf{e}_n \mathbf{v_Z}|}{|\mathbf{\bar{g}}^T \mathbf{v_Z}|}\mathbf{\bar{g}}^T\mathbf{v_Z}\mathbf{w}=    \frac{\mathbf{\bar{g}}^T \mathbf{v_Z}}{|\mathbf{\bar{g}}^T \mathbf{v_Z}|}\frac{\mathbf{v_Z}^H\mathbf{E}^H}{|\mathbf{e}_n \mathbf{v_Z}|}
% \end{align}
Substituting above $\mathbf{f}_{\rm opt}$ and \eqref{term2} in \eqref{OptSol_Phi_R1_IID} will further simplify the optimal phase shift vector $\boldsymbol{\psi}_{\rm opt}$.
We summarize the optimal beamforming for the independent fading in the following theorem.
\vspace{-0.3cm}
\begin{theorem}\label{Theo2}
    The SCSI-based optimal transmit beamformer and RIS phase shift matrix that maximizes the lower bound of mean SNR given in \eqref{OptProb_R1_IID} under {\rm i.i.d.} Rician-Rician fading are
    \begin{align}
    \mathbf{f}_{\rm opt}=\mathbf{v_Z}\text{~~~and~~~~} {\boldsymbol{\psi}}_{\rm opt}^T= \frac{\mathbf{\bar{g}}^T \mathbf{v_Z}}{|\mathbf{\bar{g}}^T \mathbf{v_Z}|}\frac{\mathbf{v_Z}^H\mathbf{E}^H}{|\mathbf{e}_n \mathbf{v_Z}|},\label{eq:optima_fPhi_R1IIDLB}
    \end{align}
    where $\mathbf{Z}=(N^2\kappa_l^4+N\kappa_l^2\kappa_n^2)\mathbf{e}_n^H\mathbf{e}_n + 2N\mu\kappa_l^3 \bar{\mathbf{g}}^*\bar{\mathbf{g}}^T$.
   % \begin{align}
   %  \mathbf{f}_{\rm opt}=\mathbf{v_Z}\text{~~~and~~~~} {\boldsymbol{\psi}}_{\rm opt}^T=\frac{N|\mathbf{e}_n\mathbf{a}|}{|\mathbf{g}^T\mathbf{a}|}\mathbf{g}^T\mathbf{a}\mathbf{w}
   %  \end{align}
   %  where $\mathbf{a}\in\mathbb{C}^N$ such that $\|\mathbf{a}\|^2 = 1$.
\end{theorem}
Now, we perform outage and capacity analysis for the optimal beamforming scheme given in Theorem \ref{Theo2} for this case of fading scenario in the following corollaries.
\vspace{-0.3cm}
\begin{corollary}
\label{cor:outage_R1IID}
    OP of the SCSI-based optimal beamforming scheme for the RIS-aided MISO system under {\rm i.i.d.} Rician-Rician fading is given by
    \begin{align}
        {\rm P_{out}}(\beta) \approx 1-Q_1\left(\frac{|m|}{\sqrt{\sigma/2}},\frac{\sqrt{{\beta}/{\gamma}}}{\sqrt{\sigma/2}}\right),\label{Pout_R1_IID}
    \end{align}
   % \begin{align*}
   %  \text{where}~~ m&=N\kappa_l^2\mathbf{e}_n\mathbf{v_Z}+ \mu\kappa_l\bar{\mathbf{g}}^T\mathbf{v_Z}, %\label{eq:parameter_nu_R1iid}\\
   %  ~~\text{and}~~\sigma^2= N\kappa_n^2(1+\kappa_l^2|\mathbf{e}_n\mathbf{v_Z}|^2)+\mu^2\kappa_n^2,%\label{eq:parameter_sigma2_R1iid}
   %  \end{align*}
   where $m=N\kappa_l^2\mathbf{e}_n\mathbf{v_Z}+ \mu\kappa_l\bar{\mathbf{g}}^T\mathbf{v_Z}$,  $\sigma^2= N\kappa_n^2(1+\kappa_l^2|\mathbf{e}_n\mathbf{v_Z}|^2)+\mu^2\kappa_n^2$,
    and $\mathbf{Z}$ is given in  \eqref{eq:optima_fPhi_R1IIDLB}.
\end{corollary}
\begin{proof}
From \eqref{SNR_dist}, we have $ |\xi_1 + \mu\xi_2|\sim{\rm Rice}\left(|m|,\sigma\right)$ whose parameters given in \eqref{parameters_m_sigma} becomes  $m=\kappa_l^2\mathbf{\Bar{h}}^T\mathbf{\Phi \bar{H}f}+ \kappa_l\mathbf{\Bar{g}}^T\mathbf{f}$ and $\sigma^2 = \kappa_l^2\kappa_n^2\|\mathbf{\Bar{H}f}\|^2 +  (\kappa_l^2\kappa_n^2+\kappa_n^4)N + \mu^2\kappa_n^2$ for {\rm i.i.d}. fading scenario. Further, substituting $\mathbf{f}_{\rm opt}$ and $\boldsymbol{\psi}_{\rm opt}$ from \eqref{eq:optima_fPhi_R1IIDLB} and simplifying, completes the proof.
\end{proof}
%Further, we can determine the ergodic capacity of the proposed SCSI-based beamforming in Theorem \ref{Theo2} for IID fading scenario using \eqref{ergodic capacity} with outage probability ${\rm P_{out}}(\beta)$ given in Corollary \ref{cor:outage_R1IID}.
\vspace{-0.3cm}
\begin{corollary}
    EC of the SCSI-based beamforming scheme proposed in  Theorem \ref{Theo2} can be determined approximately using \eqref{ergodic capacity} with ${\rm P_{out}}(\beta)$ given in Corollary \ref{cor:outage_R1IID}.
\end{corollary}
\vspace{-0.3cm}
% \begin{corollary}
%     The ergodic capacity of the SCSI-based optimal beamforming scheme for the RIS-aided MISO system under i.i.d Rician-Rician fading is given by
%     \begin{align}
%     {\rm EC}=\frac{1}{\ln(2)}\int_0^\infty \frac{1}{1+u}Q_1\left(\frac{\nu}{\sqrt{2}\sigma},\frac{\sqrt{{u}/{\gamma}}}{\sqrt{2}\sigma}\right){\rm d}u,\label{EC_R1_IID}
%     \end{align}
%     where $\nu$ and $\sigma^2$ are given in \eqref{eq:parameter_nu_R1iid} and \eqref{eq:parameter_sigma2_R1iid}, respectively.
% \end{corollary}
\vspace{-0.3cm}
\subsection{Correlated Rician-Rayleigh fading}
\label{R2}
In this subsection, we assume the LoS component along the direct link to be absent, i.e. $\mathbf{g} = \mathbf{\Tilde{R}}_{\rm BT} \mathbf{\Tilde{g}}_{\rm W}$, reducing the considered model to correlated Rician-Rayleigh fading. This fading scenario with multiple receiving antennas was considered in \cite{WangJinghe_CorrelatedFading_2021} where the authors  proposed an SDR-based iterative algorithm for SCSI-based optimal beamforming. We will present a similar scheme along with its performance analysis. 

The mean SNR given in \eqref{obj_fun} for this fading scenario is given by
\begin{align}
    \Gamma(\mathbf{f}, \mathbf{\Phi}) = |\kappa_l^2\mathbf{\Bar{h}}^T\mathbf{\Phi\Bar{H}f}|^2 + \kappa_l^2\kappa_n^2\boldsymbol{\psi}^H\mathbf{Z}_1\boldsymbol{\psi} +\mathbf{f}^H\mathbf{R}_{\rm BT}\mathbf{f}[\mu^2+\kappa_n^2\boldsymbol{\psi}^H\mathbf{Z}_2\boldsymbol{\psi}].\label{OptProb_R2}
\end{align}
The above expression follows from steps given in Appendix \ref{AppA} with $\mathbf{g}=\mathbf{\tilde{g}}$.
 As $\mathbf{f}$ and $\mathbf{\Phi}$ are coupled, we tackle this scenario by dividing the problem into optimal beamformer and phase shift matrix sub-problems as below.\newline
 \emph{1) Optimal Beamformer}: For a given $\mathbf{\Phi}$, the optimization problem w.r.t $\mathbf{f}$ becomes
 \vspace{-0.3cm}
    \begin{subequations}
    \begin{align}
    \max_{\mathbf{f}} ~~& \mathbf{f}^H\mathbf{F}_{\rm s}\mathbf{f}, \label{subprob_f_R2}\\
   \text{s.t.} ~~& \|\mathbf{f}\|^2 = 1,
    \end{align}
    \end{subequations}
where the objective function follows from \eqref{OptProb_R2} with $\mathbf{F}_{\rm s} = \mathbf{F}_{\rm 1s} + \mathbf{F}_{\rm 2s} + \mathbf{F}_{\rm 3s}$
    % \begin{equation}
    %     \mathbf{F}_{\rm s} = \mathbf{F}_{\rm 1s} + \mathbf{F}_{\rm 2s} + \mathbf{F}_{\rm 3s},
    % \end{equation}
such that $\mathbf{F}_{\rm 1s} = \kappa_l^4\mathbf{E}^H\boldsymbol{\psi}^*\boldsymbol{\psi}^T\mathbf{E}$, $\mathbf{F}_{\rm 2s} = \kappa_l^2\kappa_n^2\mathbf{\Bar{H}}^H\mathbf{\Phi}^H\mathbf{R}_{\rm RT}\mathbf{\Phi\Bar{H}}$, $\mathbf{F}_{\rm 3s} = \mathbf{R}_{\rm BT}[\mu^2+\kappa_n^2\boldsymbol{\psi}^H\mathbf{Z}_2\boldsymbol{\psi}]$.
    % \begin{align}
    %  \mathbf{F}_{\rm 1s} = ~~& (\kappa_l^2\boldsymbol{\psi}^T\mathbf{E})^H(\kappa_l^2\boldsymbol{\psi}^T\mathbf{E}),\\\nonumber
    % \mathbf{F}_{\rm 2s} = ~~& \kappa_l^2\kappa_n^2\mathbf{\Bar{H}}^H\mathbf{\Phi}^H\mathbf{R}_{\rm RT}\mathbf{\Phi\Bar{H}},\\\nonumber
    % \mathbf{F}_{\rm 3s} = ~~& \mathbf{R}_{\rm BT}[\mu^2+\kappa_n^2\boldsymbol{\psi}^H\mathbf{Z}_2\boldsymbol{\psi}].
    % \end{align}
Note that $\mathbf{F}_{\rm 1s}$, $\mathbf{F}_{\rm 2s}$, and $\mathbf{F}_{\rm 3s}$ directly follow from $\mathbf{F}_1$, $\mathbf{F}_2$, and $\mathbf{F}_3$ given in Section \ref{optimal beamforming} by setting $\mathbf{\Bar{g}} = 0$. Thus, the symmetricity of $\mathbf{F}_{\rm s}$ also follows $\mathbf{F}$. Hence, we have
\vspace{-0.3cm}
\begin{equation}
    \mathbf{f}_{\rm opt} = \mathbf{v}_{\mathbf{F}_{\rm s}}.\label{OptSol_f_R2}
\end{equation}
\emph{2) Optimal Phase Shift Matrix}: For a given $\mathbf{f}$, the optimization problem w.r.t  $\mathbf{\Phi}$ becomes
% \begin{subequations}
%     \begin{align}
%     \max_{\boldsymbol{\psi}}  ~~& |\boldsymbol{\psi}^T\mathbf{a}|^2 + \sum_{i=1}^{N}|\boldsymbol{\psi}^T\mathbf{Z}_i|^2, \label{subprob_phi_R2}\\
%    \text{s.t.} ~~& |\mathbf{\psi}_k| = 1 ~~ \forall k = 0,\ldots,N-1,
%     \end{align}
% \end{subequations}
\begin{subequations}
    \begin{align}
    \max_{\boldsymbol{\psi}}  ~~& \boldsymbol{\psi}^H \mathbf{A}_{\rm s} \boldsymbol{\psi}, \label{subprob_phi_R2}\\
   \text{s.t.} ~~& |\mathbf{\psi}_k| = 1 ~~ \forall k = 0,\ldots,N-1,
    \end{align}
\end{subequations}
where $\mathbf{A}_{\rm s}=\kappa_l^4 \mathbf{E}^*\mathbf{f}^* \mathbf{f}^T\mathbf{E}^T + \kappa_l^2 \kappa_n^2 \mathbf{Z}_1 + \kappa_n^2 \mathbf{f}^H\mathbf{R}_{\rm BT} \mathbf{f}~\mathbf{Z}_2$.  %Similar to problem \eqref{obj psibar}, the above problem can be reformulated using SDR in order to solve it using the  convex optimization toolbox. However, this will lead to  a beamforming scheme having a complexity similar to the one presented in Algorithm \ref{Alg1}. 
Next, we reformulate the above problem using SDR as given in \eqref{relaxed_phi} with $\mathbf{A}=\mathbf{A}_{\rm s}$ and $\mathbf{\Psi}=\boldsymbol{\psi\psi}^H$ and obtain optimal  $\boldsymbol{\psi}$ for a given $\mathbf{f}$.  

As the  above solutions of $\mathbf{f}$ and $\mathbf{\Phi}$ are coupled, the optimal beamformer for this fading case can be obtained using Algorithm \ref{Alg1} by simply setting  $\mathbf{F}=\mathbf{F}_{\rm s}$, $\mathbf{A}=\mathbf{A}_{\rm s}$, and $\mathbf{\Psi}=\boldsymbol{\psi\psi}^H$. Further, OP and EC of optimal beamforming under this fading scenario can be evaluated using Theorem \ref{Theo1} and Corollary \ref{cor:R1_EC}, respectively, with modified parameters 
\begin{align}
    m=\kappa_l^2\mathbf{\Bar{h}}^T \mathbf{\Phi}\Bar{\mathbf{H}}\mathbf{f}_{\rm opt} \text{~~and~~} \sigma^2=\kappa_l^2 \kappa_n^2 \boldsymbol{\psi}_{\rm opt} ^H \mathbf{Z}_1 \boldsymbol{\psi}_{\rm opt}  +  ( \mu^2 + \kappa_n^2 \boldsymbol{\psi}_{\rm opt} ^H \mathbf{Z}_2 \boldsymbol{\psi}_{\rm opt} )\mathbf{f}^H_{\rm opt}  \mathbf{R_{\rm BT}} \mathbf{f}_{\rm opt} .
\end{align}
The above parameters can be obtained by modifying the parameters of the distribution of  $\xi_2$ given in \eqref{eq:xi1xi2} with $\mathbf{g}=\mathbf{\tilde{g}}$ and further using it to get mean and variance of $\xi_1+\mu\xi_2$. $\mathbf{f}_{\rm opt}$ and $\boldsymbol{\psi}_{\rm opt}$ are obtained from \autoref{Alg1} with modified parameters as mentioned above. 

% and $\mathbf{Z}_i=\sqrt{\mathbf{\lambda}_i}\mathbf{v}_i$ such that $\mathbf{v}_i$ is the eigen value of $\mathbf{F}_s$.

% Now, we present the outage and ergodic capacity analysis for the optimal beamforming for R2 correlated scheme. However, the lack of closed form solution for the optimal $\mathbf{f}$ and $\mathbf{\Phi}$ restrict the outage analysis here also. Therefore, we will first perform the outage and capacity analysis for given $\mathbf{f}$ and $\mathbf{\Phi}$.
% Outage probability given in \eqref{P_out} can be written as
% \begin{equation}
%     \text{P}_{\rm out}(\beta) = \mathbb{P}\left[|\xi_1 + \mu \xi_2| \leq \sqrt{\beta/\gamma}\right]
% \end{equation}
% where $\xi_1 = \mathbf{h}^T\mathbf{\Phi Hf}$ and $\xi_2 =  \mathbf{g}^T\mathbf{f}$. As shown in Appendix \ref{AppB}, $\xi_1$ and $\xi_2$ follow complex Gaussian distributions as below
% \begin{equation}
%     \xi_1\sim\mathcal{C}\mathcal{N}(\mu_1,\sigma_1^2)
%     \hspace{0.2cm}\text{and}\hspace{0.2cm} 
%     \xi_2\sim\mathcal{C}\mathcal{N}(\mu_2,\sigma_2^2),
% \end{equation}
% where
% \begin{align}
%     \mu_1 = ~~&\kappa_l^2\mathbf{\Bar{h}}^T\mathbf{\Phi} \bar{\mathbf{H}} \mathbf{f},\hspace{0.3cm} \text{and} \hspace{0.3cm}\sigma_1^2 =\kappa_l^2\kappa_n^2\boldsymbol{\psi}^H\mathbf{Z}_1\boldsymbol{\psi} + \kappa_n^2\mathbf{f}^H\mathbf{R}_{\rm BT}\mathbf{f}[\boldsymbol{\psi}^H\mathbf{Z}_2\boldsymbol{\psi}],\\ 
%     \mu_2 = ~~& 0,\hspace{0.3cm} \text{and} \hspace{0.3cm}\sigma_2^2 =\mathbf{f}^H\mathbf{R}_{\rm BT}\mathbf{f}.
% \end{align}
% Again, using the fact that $\xi_1$ and $\xi_2$ are independent, we can say that
% \begin{equation}
%     \xi_1 + \mu\xi_2\sim\mathcal{C}\mathcal{N}(m,\sigma^2),
% \end{equation}
% where 
% \begin{equation}
%   m = \mu_1+\mu\mu_2
%   \hspace{0.2cm}\text{~~and~~}  \sigma^2 = \sigma_1^2+\mu^2\sigma_2^2.  \label{parameters_m_sigma_R2}
% \end{equation}
% Further, we again use the fact that the magnitude of a non-zero mean complex Gaussian follows the Rice distribution to obtain
% \begin{equation}
%     |\xi_1 + \mu\xi_2|\sim{\rm Rice}\left(|m|,\sigma\right).
% \end{equation}
% Therefore, the outage probability for a given $\mathbf{f}$ and $\mathbf{\Phi}$ can be obtained by using the CDF of the Rice distribution. Using this, we present the outage performance of this proposed beamforming scheme in  Theorem \ref{Theo3}. 
% \begin{theorem}\label{Theo3}
% The outage probability of the statistically optimal beamforming scheme for the RIS-aided MISO system under correlated Rician-Rayleigh fading is given by
% \begin{align}
%     {\rm P_{out}}(\beta)=1-Q_1\left(\frac{|m|}{\sqrt{\sigma/2}},\frac{\sqrt{\beta/\gamma}}{\sqrt{\sigma/2}}\right)\label{Pout_R2}
% \end{align}
% where $Q_1(\cdot)$ is a Marcum Q-function and
% \begin{align}
%     m &=  \kappa_l^2\mathbf{\Bar{h}}^T\mathbf{\Phi_{\rm opt} Hf_{\rm opt}} ,\label{eq:parameter_m_R2Corr}\\
%     \text{and~~}\sigma^2 &=  \kappa_l^2\kappa_n^2\boldsymbol{\psi}_{\rm opt}^H\mathbf{Z}_1\boldsymbol{\psi}_{\rm opt} + \mathbf{f}_{\rm opt}^H\mathbf{R}_{\rm BT}\mathbf{f}_{\rm opt}[\kappa_n^2\boldsymbol{\psi}_{\rm opt}^H\mathbf{Z}_2\boldsymbol{\psi}_{\rm opt} + \mu^2]\label{eq:parameter_sigma2_R2Corr},
% \end{align}
% such that $\mathbf{f}_{\rm opt}$ and $\mathbf{\Phi}_{\rm opt}$ are the optimal solutions.
% \end{theorem}
% \begin{proof}
%     The proof directly follows from \eqref{Pout_R2} wherein the parameters $m$ and $\sigma^2$ given in \eqref{parameters_m_sigma_R2} are evaluated using the optimal beamformer $\mathbf{f}_{\rm opt}$ and phase shift matrix $\mathbf{\Phi}_{\rm opt}$ that is obtained.
% \end{proof}
% Using Theorem \ref{Theo3} and \eqref{ergodic capacity}, we determine the ergodic capacity of the proposed beamforming scheme in the following corollary.
% \begin{corollary}
% The ergodic capacity of the statistical optimal beamforming scheme for the RIS-aided MISO system under correlated Rician-Rician fading is given by
% \begin{align}
%     {\rm EC}=\frac{1}{\ln(2)}\int_0^\infty \frac{1}{1+u}Q_1\left(\frac{|m|}{\sqrt{\sigma/2}},\frac{\sqrt{{u}/{\gamma}}}{\sqrt{\sigma/2}}\right){\rm d}u.
% \end{align}
% where $m$ and $\sigma^2$ are given in \eqref{eq:parameter_m_R2Corr} and \eqref{eq:parameter_sigma2_R2Corr}, respectively.
% \end{corollary}
\vspace{-.4cm}
\subsection{IID Rician-Rayleigh Fading}\label{R2_IID}
In this subsection, we consider {\rm i.i.d.} Rician-Rayleigh fading along the indirect-direct links such $\mathbf{g}=\mathbf{\tilde{g}}\sim\mathcal{C}\mathcal{N}(0,\mathbf{I_M})$, $\mathbf{\Tilde{h}}\sim\mathcal{C}\mathcal{N}(0,\mathbf{I_N})$ and $\mathbf{\Tilde{H}}_{:,i}\sim\mathcal{C}\mathcal{N}(0,\mathbf{I_N})$. For this scenario, the mean SNR, i.e. the objective function \eqref{objective}, can be obtained using the steps given in Appendix \ref{AppA} as 
\begin{equation}
    \Gamma(\mathbf{f}, \mathbf{\Phi}) = |\kappa_l^2\mathbf{\Bar{h}}^T\mathbf{\Phi\Bar{H}f}|^2 + \kappa_l^2\kappa_n^2\|\mathbf{\Bar{H}f}\|^2 + N\kappa_l^2\kappa_n^2 + N\kappa_n^4 + \mu^2. 
\end{equation}
By substituting $\mathbf{\Bar{H}} = \mathbf{a}_N(\theta_{\rm ra})\mathbf{a}^T_M(\theta_{\rm bd}^{\rm i})$ and further simplifying, we can write the mean {\rm SNR} as
\begin{align}
    \Gamma(\mathbf{f}, \mathbf{\Phi}) = |\mathbf{a}^H_M(\mathbf{\theta}_{\rm bd}^{\rm i})\mathbf{f}|^2\left[|\kappa_l^2\boldsymbol{\psi}^T{\rm diag}(\mathbf{\bar{h}})\mathbf{a}_N(\mathbf{\theta}_{\rm ra})|^2+\kappa_l^2\kappa_n^2N\right]+ N\kappa_l^2\kappa_n^2 + N\kappa_n^4 + \mu^2.\label{OptProb_R2_IID}
\end{align}
It is clear from \eqref{OptProb_R2_IID} that $\mathbf{f}$ and $\mathbf{\Phi}$ are decoupled and the problem can be equivalently transformed into two independent sub-problems with respect to $\mathbf{f}$ and $\mathbf{\Phi}$. By simple applications of co-phasing and projections, the authors in \cite{hu2020statistical} obtain the optimal beamforming solutions for this fading scenario. But, for completeness, we reconstruct the results in the following theorem.
\vspace{-0.3cm}
\begin{theorem}\label{Theo4}
    The SCSI-based optimal transmit beamformer and RIS phase shift matrix that maximize the mean SNR given in \eqref{OptProb_R2_IID} under {\rm i.i.d.} Rician-Rayleigh fading are
   \begin{align}
    \mathbf{f}_{\rm opt} = \frac{1}{\sqrt{M}}\mathbf{a}_M^H(\mathbf{\theta}_{\rm bd}^{i}),
    \text{~~and~~}\boldsymbol{\psi_{\rm opt}} =  e^{-j\angle{{\rm diag}(\mathbf{\Bar{h}})\mathbf{a}_N(\mathbf{\theta}_{\rm ra})}}.\label{OptSol_R2_IID}
    \end{align}
\end{theorem}
\begin{proof}
    Since \eqref{OptProb_R2_IID} is decoupled in $\mathbf{f}$ and $\mathbf{\Phi}$, we can select $\mathbf{f}$ that maximizes $|\mathbf{a}^H_M(\mathbf{\theta}_{\rm bd}^{\rm i})\mathbf{f}|^2$ and $\mathbf{\Phi}$ that maximizes $|\kappa_l^2\boldsymbol{\psi}^T{\rm diag}(\mathbf{\bar{h}})\mathbf{a}_N(\mathbf{\theta}_{\rm ra})|^2$. For this, one can clearly see that the optimal solutions of $\mathbf{f}$ and $\mathbf{\Phi}$ would be the ones given in \eqref{OptSol_R2_IID}.
\end{proof}
Now, we present OP and EC that are achievable through the beamforming scheme given in Theorem \ref{Theo4} in the following corollaries.
\vspace{-0.3cm}
\begin{corollary}
\label{cor:outage_R2IID}
    OP of the SCSI-based optimal beamforming scheme for the RIS-aided MISO system under {\rm i.i.d.} Rician-Rayleigh fading is given by
    \begin{align}
        {\rm P_{out}}(\beta) \approx 1-Q_1\left(\frac{|m|}{\sqrt{\sigma/2}},\frac{\sqrt{\beta/\gamma}}{\sqrt{\sigma/2}}\right),\label{Pout_R2_IID}
    \end{align}
    % \begin{align*}
    % \text{where}~~~m = \kappa_l^2N\sqrt{M},
    % ~\text{and}\hspace{0.2cm} \sigma^2 = (M + 1) N\kappa_l^2 \kappa_n^2 + N \kappa_n^4 + \mu^2.
    % \end{align*}
    where $m = \kappa_l^2N\sqrt{M}$ and
    $\sigma^2 = (M + 1) N\kappa_l^2 \kappa_n^2 + N \kappa_n^4 + \mu^2$.
\end{corollary}
\begin{proof}
   For $\mathbf{g} = \Tilde{\mathbf{g}}$, the parameters of OP given in \eqref{parameters_m_sigma} becomes  $m=\kappa_l^2\mathbf{\Bar{h}}^T\mathbf{\Phi \bar{H}f}$ and $\sigma^2=\kappa_l^2\kappa_n^2\|\mathbf{\Bar{H}f}\|^2 + N\kappa_l^2\kappa_n^2 + N\kappa_n^4 + \mu^2$. Further, substituting $\mathbf{f}_{\rm opt}$ and $\boldsymbol{\psi}_{\rm opt}$ from \eqref{OptSol_R2_IID} and using $\|\mathbf{\Bar{H}f_{\rm opt}}\|^2 = MN$, we obtain $m$ and $\sigma^2$ as given in \eqref{Pout_R2_IID}.
\end{proof}
\vspace{-0.3cm}
\begin{corollary}
    EC of the SCSI-based  beamforming scheme proposed in  Theorem \ref{Theo4} can be determined approximately using \eqref{ergodic capacity} with  ${\rm P_{out}}(\beta)$ given in Corollary \ref{cor:outage_R2IID}.
\end{corollary}
% \begin{corollary}
%     The ergodic capacity of the statistically optimal beamforming scheme for the RIS-aided MISO system under i.i.d Rician-Rayleigh fading is given by
%     \begin{align}
%     {\rm EC}=\frac{1}{\ln(2)}\int_0^\infty \frac{1}{1+u}Q_1\left(\frac{\nu}{\sqrt{\sigma/2}},\frac{\sqrt{{u}/{\gamma}}}{\sqrt{\sigma/2}}\right){\rm d}u.\label{EC_R2_IID}
%     \end{align}
%     where $m$ and $\sigma^2$ are given in Corollary \ref{cor:outage_R2IID}.
% \end{corollary}
\vspace{-0.7cm}
\subsection{Correlated Rayleigh-Rayleigh Fading}\label{R3}
\vspace{-0.1cm}
In this subsection, we assume that the LoS components along both the direct and indirect links are absent. This reduces the channel model presented in Section \ref{channel model} to a correlated Rayleigh-Rayleigh scenario wherein $\mathbf{H} = \mathbf{\Tilde{H}}$, $\mathbf{h} = \mathbf{\Tilde{h}}$, and $\mathbf{g} = \mathbf{\Tilde{g}}$. It may be noted that this scenario is a special case of the generalized fading model given in Section \ref{channel model}  as it is discussed  earlier to be a limiting case of fading factor, i.e. $K\to 0$. For this scenario, the mean {\rm SNR} becomes
\begin{equation}
    \Gamma(\mathbf{f}, \mathbf{\Phi}) = \mathbf{f}^H\mathbf{R}_{\rm BT}\mathbf{f}[\boldsymbol{\psi}^H(\mathbf{R}_{\rm RR}\odot\mathbf{R}_{\rm RT})\boldsymbol{\psi}+\mu^2].\label{OPtProb_R3}
\end{equation}
The above equation directly follows from the steps given in Appendix \ref{AppA} by setting $\kappa_l = 0$ and $\kappa_n = 1$.
It can be  seen from \eqref{OPtProb_R3} that the terms pertaining to $\mathbf{f}$ and $\mathbf{\Phi}$ are decoupled. 
This is expected as, in the absence of LoS components, the optimal choice of transmit beamforming vector $\mathbf{f}$ will depend on fading covariance matrix associated with BS and the optimal choice of phase shift matrix $\mathbf{\Phi}$ will depend on  fading  covariance matrices associated with RIS. 
Therefore, the optimal choice of $\mathbf{f}$ and $\mathbf{\Phi}$ can be selected independently of each other. 

As the covariance matrix is symmetric, we can set $\mathbf{f}$ equal to $\mathbf{v_{R_{\rm BT}}}$ for maximizing the term $\mathbf{f}^H\mathbf{R}_{\rm BT}\mathbf{f}$ with unit norm constraint \eqref{constraint_f}. For  $\mathbf{f}=\mathbf{v_{R_{\rm BT}}}$, the  maximum value of this term is equal to  $\lambda_{\mathbf{R}_{\rm BT}}$ which is nothing but the maximum eigenvalue value of $\mathbf{R}_{\rm BT}$.
%To maximize $\mathbf{f}^H\mathbf{R}_{\rm BT}\mathbf{f}$, we can set $\mathbf{f}$ straightforwardly as $\mathbf{v_{R_{\rm BT}}}$. This will maximize $\mathbf{f}^H\mathbf{R}_{\rm BT}\mathbf{f}$ to $\lambda_{\rm max}\{\mathbf{R}_{\rm BT}\}$. 
Now, we will maximize the other term corresponding to the phase shift matrix as below
\begin{subequations}
    \begin{align}
    \max_{\boldsymbol{\psi}}  ~~& \boldsymbol{\psi}^H(\mathbf{R}_{\rm RR} \odot \mathbf{R}_{\rm RT})\boldsymbol{\psi}, \label{subprob_phi_R3}\\
   \text{s.t.} ~~& |\mathbf{\psi}_k| = 1 ~~ \forall k = 0,\ldots,N-1,
    \end{align}\label{OptProb_Phi_R3}
\end{subequations}
\vspace{-0.1cm}
The unit modulus constraint makes it difficult to solve the problem directly, as mentioned earlier. However, we could obtain the optimal $\mathbf{\Phi}$ using the fact that the matrix $\mathbf{R}_{\rm RR}\odot\mathbf{R}_{\rm RT}$ in the objective function is a real. 
The objective function \eqref{subprob_phi_R3} can be rewritten as
    \begin{align*}
        \boldsymbol{\psi}^H(\mathbf{R}_{\rm RR} \odot \mathbf{R}_{\rm RT})\boldsymbol{\psi}=& \sum_{i,j}\mathbf{R}_{{\rm RT},{ij}}\mathbf{R}_{{\rm RR},{ij}}\boldsymbol{\psi}_i^H\boldsymbol{\psi}_j
        % = ~~&  \sum_{i\neq j}\mathbf{R}_{{\rm RT},{ij}}\mathbf{R}_{{\rm RR},{ij}}  + \sum_{i\neq j}\mathbf{R}_{{\rm RT},{ij}}\mathbf{R}_{{\rm RR},{ij}}\boldsymbol{\psi}_i^H\boldsymbol{\psi}_j. \nonumber\\
        % %\label{phi_int}
        = {\rm trace}(\mathbf{R}_{\rm RR}\odot\mathbf{R}_{\rm RT}) + \sum_{i\neq j}\mathbf{R}_{{\rm RT},{ij}}\mathbf{R}_{{\rm RR},{ij}}\boldsymbol{\psi}_i^H\boldsymbol{\psi}_j.\label{phi_int} 
    \end{align*}
Note ${\rm trace}(\mathbf{R}_{\rm RR}\odot\mathbf{R}_{\rm RT})$ is a real scalar quantity and is independent of $\mathbf{\Phi}$, whereas the second term is the summation of complex scalars. Thus, we need to co-phase the complex numbers to maximize the second term. To do this, we can simply set $\boldsymbol{\psi}_i = \boldsymbol{\psi}_j$, $\forall i,j=1,\dots,N$.
% \begin{equation}
%     \boldsymbol{\psi}_i = \boldsymbol{\psi}_j~~~~ \forall i,j=1,\dots,N.
% \end{equation}
Hence, the optimal $\mathbf{\Phi}$ can be obtained as $\boldsymbol{\psi}_{{\rm opt},k} = e^{j{\theta}}$ for $k=1,\dots,N$ where $\theta\in[-\pi/2 , \pi/2]$. For this choice of $\boldsymbol{\psi}_{\rm opt}$, maximum value of the objective given in \eqref{OptProb_Phi_R3} becomes
\begin{equation}
    {\rm trace}(\mathbf{R}_{\rm RR}\odot\mathbf{R}_{\rm RT}) + \sum\nolimits_{i\neq j}\mathbf{R}_{\rm RT_{ij}}\mathbf{R}_{\rm RR_{ij}} = \sum\nolimits_{i,j} \mathbf{R}_{\rm RT_{ij}}\mathbf{R}_{\rm RR_{ij}}=\mathbf{1}_{\rm N}^T(\mathbf{R}_{\rm RR}\odot\mathbf{R}_{\rm RT})\mathbf{1}_{\rm N}.\label{Max_Obj_R3}
\end{equation}
For $\mathbf{R}_{\rm RT} = \mathbf{R}_{\rm RR}=\mathbf{R}$, \eqref{Max_Obj_R3} becomes $\|\mathbf{R}\|_{\rm F}^2$. The above results are summarized in  Theorem \ref{Theo5}.
\vspace{-0.3cm}
\begin{theorem}\label{Theo5}
    The SCSI-based optimal transmit beamformer and RIS phase shift matrix that maximize the mean SNR given in \eqref{OptProb_R2_IID} under {\rm i.i.d.} Rayleigh-Rayleigh fading are
   \begin{align}
    \mathbf{f}_{\rm opt} = \mathbf{v_{R_{\rm BT}}}
    \text{~~and~~}\boldsymbol{\psi}_{\rm opt} = \mathbf{1}_{\rm N} e^{j{\theta}},\label{OptSol_R3}
    \end{align}
    \vspace{-0.3cm}
    where $\theta\in[-\pi/2 , \pi/2]$ and the maximum mean {\rm SNR} value is
    \begin{equation}
        \Gamma(\mathbf{f}_{\rm opt},\mathbf{\Phi}_{\rm opt})=\begin{cases}
            \lambda_{\mathbf{R}_{\rm BT}}\left(\mu^2 + \|\mathbf{R}\|_F^2\right), ~~&\text{if~ }\mathbf{R}_{\rm RT} = \mathbf{R}_{\rm RR}=\mathbf{R}\\
             \lambda_{\mathbf{R}_{\rm BT}}\left(\mu^2 +\mathbf{1}_{\rm N}^T(\mathbf{R}_{\rm RR}\odot\mathbf{R}_{\rm RT})\mathbf{1}_{\rm N}\right), ~~&\text{otherwise},
        \end{cases}\label{eq:R3cor_maxSNR}
    \end{equation}
    and $\lambda_{\mathbf{R}_{\rm BT}}$ is the maximum eigen value of $\mathbf{R}_{\rm BT}$.
\end{theorem}
Now, we present OP and EC achievable through the scheme given in Theorem \ref{Theo5} in the following corollaries.
\vspace{-0.3cm}
\begin{corollary}\label{cor:outage_R3cor}
    OP of the SCSI-based optimal beamforming scheme for the RIS-aided MISO system under correlated Rayleigh-Rayleigh fading is given by
    \begin{align}
        {\rm P_{out}}(\beta) \approx 1-Q_1\left(0,\frac{\sqrt{\beta/\gamma}}{\sqrt{\sigma/2}}\right),\label{Pout_R3}
    \end{align}
   %  \begin{align*}
   % \text{where}~~  m =  0%\label{eq:parameter_m_R3}
   %  \text{~~and~~} \sigma^2 =  \Gamma(\mathbf{f}_{\rm opt},\mathbf{\Phi}_{\rm opt}),
   %  \end{align*}
    where  $ m =0$ and $\sigma^2 =  \Gamma(\mathbf{f}_{\rm opt},\mathbf{\Phi}_{\rm opt})$, and  $\Gamma(\mathbf{f}_{\rm opt},\mathbf{\Phi}_{\rm opt})$ is given in \eqref{eq:R3cor_maxSNR}.
\end{corollary}
\begin{proof}
    For $\mathbf{H} = \mathbf{\Tilde{H}}$, $\mathbf{h} = \mathbf{\Tilde{h}}$, and $\mathbf{g} = \mathbf{\Tilde{g}}$, the parameters given in \eqref{parameters_m_sigma} becomes $m = 0$ and $\sigma^2 = \mathbf{f}^H\mathbf{R}_{\rm BT}\mathbf{f}[\boldsymbol{\psi}^H(\mathbf{R}_{\rm RR}\odot\mathbf{R}_{\rm RT})\boldsymbol{\psi}+\mu^2]$. Further, by substituting $\mathbf{f}_{\rm opt}$ and $\mathbf{\boldsymbol{\psi}_{\rm opt}}$ from \eqref{OptSol_R3} and simplifying, we obtain \eqref{Pout_R3}.
\end{proof}
\vspace{-0.3cm}
\begin{corollary}
    EC of the SCSI-based  beamforming scheme proposed in  Theorem \ref{Theo5} can be determined approximately using \eqref{ergodic capacity} with ${\rm P_{out}}(\beta)$ given in Corollary \ref{cor:outage_R3cor}.
\end{corollary}
% \begin{corollary}
%     The ergodic capacity of the statistically optimal beamforming scheme for the RIS-aided MISO system under correlated Rayleigh-Rayleigh fading is given by
%     \begin{align}
%     {\rm EC}=\frac{1}{\ln(2)}\int_0^\infty \frac{1}{1+u}Q_1\left(0,\frac{\sqrt{{u}/{\gamma}}}{\sqrt{\sigma/2}}\right){\rm d}u.\label{EC_R3}
%     \end{align}
%     where $m$ and $\sigma^2$ are given in \eqref{eq:parameter_m_R3} and \eqref{eq:parameter_sigma2_R3}, respectively.
% \end{corollary}
\vspace{-0.6cm}
\subsection{IID Rayleigh-Rayleigh Fading}\label{R3_IID}
Here, we assume that  both the direct and indirect links undergo {\rm i.i.d} multipath fading with the absence of LoS components. This results in the {\rm i.i.d} Rayleigh-Rayleigh fading scenario such that 
$\mathbf{g}\sim\mathcal{C}\mathcal{N}(0,\mathbf{I_M})$, $\mathbf{h}\sim\mathcal{C}\mathcal{N}(0,\mathbf{I_N})$ and $\mathbf{H}_{:,i}\sim\mathcal{C}\mathcal{N}(0,\mathbf{I_N})$. For this, the mean {\rm SNR} reduces to
\vspace{-0.3cm}
\begin{equation}
  \Gamma(\mathbf{f},\mathbf{\Phi})=  \mu^2 + N\label{OptProb_R3_IID},
\end{equation}
which is independent of  $\mathbf{f}$ and $\mathbf{\Phi}$. This implies that the mean {\rm SNR}  is a constant value regardless of the choice of beamforming vector. Therefore, we can set $\mathbf{f}_{\rm opt} \in \mathcal{S}_\mathbf{f} = \{ \mathbf{f} \in \mathbb{C}^M : \|\mathbf{f}\| = 1 \}$ and $\boldsymbol{\psi}_{\rm opt} \in \mathcal{S}_{\boldsymbol{\psi}}  = \{ \boldsymbol{\psi} \in \mathbb{C}^N : |\boldsymbol{\psi}_k| = 1 ; \forall i = 1 \cdots N\}$.
Now, we present the outage performance of this proposed beamforming scheme in  the following corollary. 
\setcounter{theorem}{5}
\setcounter{corollary}{0}
\vspace{-0.3cm}
\begin{corollary}\label{cor:outage_R3IID}
For any $\mathbf{f}\in\mathcal{S}_\mathbf{f}$ and $\boldsymbol{\psi}\in\mathcal{S}_\mathbf{\boldsymbol{\psi}}$, OP for the RIS-aided MISO system under {\rm i.i.d.}. Rayleigh-Rayleigh fading is given by
\begin{align}
    {\rm P_{out}}(\beta) \approx 1-Q_1\left(0,\frac{\sqrt{\beta/\gamma}}{\sqrt{\sigma/2}}\right), \text{~where~}  m =  0 \text{~and~} \sigma^2=\mu^2 + N.\label{Pout_R3_IID}
\end{align}
%where $ m =  0$ and $\sigma^2=\mu^2 + N$. 
% \begin{align*}
%     m =  0 \text{~~and~~}\sigma^2 =  \mu^2 + N.%\label{eq:parameter_sigma2_R3iid}.
% \end{align*}
\end{corollary}
\vspace{-0.3cm}
\begin{proof}
For $\mathbf{{H}} = \mathbf{\Tilde{H}}_{\rm W}$, $\mathbf{h} = \mathbf{\Tilde{h}}_{\rm W}, \mathbf{g} = \mathbf{\Tilde{g}}_{\rm W}$, the parameters given in \eqref{parameters_m_sigma} become $m = 0$ and $\|\mathbf{f}\|^2\left(\mu^2 + \|\boldsymbol{\psi}\|^2\right)$. Further, by substituting $\mathbf{f}_{\rm opt}$ and $\mathbf{\boldsymbol{\psi}_{\rm opt}}$ and simplifying, we obtain \eqref{OptProb_R3_IID}. 
\end{proof}
\vspace{-0.3cm}
\begin{corollary}
    For any $\mathbf{f}\in\mathcal{S}_\mathbf{f}$ and $\boldsymbol{\psi}\in\mathcal{S}_\mathbf{\boldsymbol{\psi}}$,  EC  under {\rm i.i.d.} Rayleigh-Rayleigh fading can be determined approximately using \eqref{ergodic capacity} with ${\rm P_{out}}(\beta)$ given in Corollary \ref{cor:outage_R3IID}.
\end{corollary}
\vspace{-0.3cm}
The optimal transmit beamforming vector $\mathbf{f}_{\rm opt}$, RIS phase shift matrix $\mathbf{\Phi}_{\rm opt}$, and the achievable OP along with its parameters $(m,\sigma^2)$ are summarized in Table \ref{OP_Table} for various fading cases studied in the above subsections. For easy referencing, we refer the Rician-Rician, Rician-Rayleigh, and Rayleigh-Rayleigh fading cases as R1, R2 and R3, respectively. The rows associated with fading cases that have algorithmic and closed-form beamforming solutions are highlighted in different colors.
It may be noted that the parameters $m$ and $\sigma^2$ are  useful to determine the mean SNR as $$\Gamma(\mathbf{f},\mathbf{\Phi})=\mathbb{E}[|\xi_1+\mu\xi_2|^2]=\sigma^2+|m|^2.$$
\begin{table}[h]
\vspace{-.4cm}
\caption{Summary of optimal beamforming and outages}
\label{OP_Table}
\vspace{-0.3cm}
\centering
\fontsize{19pt}{19pt}
% \Huge
\renewcommand{\arraystretch}{2.5}
\centering
\resizebox{\textwidth}{!}{
\begin{tabular}{|c|c|c|c|c|}
%\toprule

\hline
\multicolumn{1}{|c|}{} & \multicolumn{2}{|c|}{\textbf{Optimal Beamforming}} & \multicolumn{2}{|c|}{\textbf{Outage Probability} ${\rm P_{out}}(\beta)=1-Q_1\left(\frac{|m|}{\sqrt{\sigma/2}},\frac{\sqrt{\beta/\gamma}}{\sqrt{\sigma/2}}\right)$} \\
%\cmidrule(rl){1-3} \cmidrule(rl){4-5}
\cline{2-5}
\textbf{Fading Scenario}& {$\mathbf{f}_{\rm opt}$} &{$\mathbf{\Phi_{\rm opt}}$} & {$m$} & {$\sigma^2$}  \\
%\midrule

\hline
\multicolumn{1}{|c|}{\textbf{R1 Corr}} & \multicolumn{2}{|c|}{\Large{Algorithm \autoref{Alg1}}}&  $\kappa_l^2\mathbf{\Bar{h}}^T \mathbf{\Phi}_{\rm opt}\Bar{\mathbf{H}}\mathbf{f}_{\rm opt}  + \mu \kappa_l \Bar{\mathbf{g}}^T\mathbf{f}_{\rm opt}$ & $\kappa_l^2 \kappa_n^2 \boldsymbol{\psi}_{\rm opt}^H \mathbf{Z}_1 \boldsymbol{\psi}_{\rm opt} + \kappa_n^2 ( \mu^2 + \boldsymbol{\psi}_{\rm opt}^H  \mathbf{Z}_2 \boldsymbol{\psi}_{\rm opt})\mathbf{f}_{\rm opt}^H \mathbf{R_{\rm BT}} \mathbf{f}_{\rm opt}$\\


\hline
\multicolumn{1}{|c|}{\textbf{R1 IID} } & \multicolumn{2}{|c|}{\Large{Algorithm}  \ref{Alg2}}& $\kappa_l^2\mathbf{\Bar{h}}^T \mathbf{\Phi}_{\rm opt}\Bar{\mathbf{H}}\mathbf{f}_{\rm opt}  + \mu \kappa_l \Bar{\mathbf{g}}^T\mathbf{f}_{\rm opt}$ & $\kappa^2_l \kappa^2_n \|\Bar{\mathbf{H}}\mathbf{f}_{\rm opt}\|^2 + \kappa^2_n ( \mu^2 + N(\kappa^2_l  + \kappa^2_n))$\\ \hline


\rowcolor{lavender}
\textbf{R1 IID LB} & $\mathbf{v_Z}$ & $ \frac{\mathbf{\bar{g}}^T \mathbf{v_Z}}{|\mathbf{\bar{g}}^T \mathbf{v_Z}|}\frac{\mathbf{v_Z}^H\mathbf{E}^H}{|\mathbf{e}_n \mathbf{v_Z}|}$  & $N \kappa_l^2 \mathbf{e}_n \mathbf{v_Z} + \mu \kappa_l \Bar{\mathbf{g}}^T \mathbf{v_Z}$ & $N \kappa_n^2 (1 + \kappa_l^2 |\mathbf{e}_n \mathbf{v_Z}|^2) + \mu^2 \kappa_n^2$ \\ 

\hline
\multicolumn{1}{|c|}{\textbf{R2 Corr}} & \multicolumn{2}{|c|}{\Large{Algorithm \autoref{Alg1}} with \mbox{\normalsize $\mathbf{F}=\mathbf{F}_{\rm s}$, $\mathbf{A}=\mathbf{A}_{\rm s}$,  $\mathbf{\Psi}=\boldsymbol{\psi\psi}^H$} } & $\kappa_l^2\mathbf{\Bar{h}}^T \mathbf{\Phi}_{\rm opt}\Bar{\mathbf{H}}\mathbf{f}_{\rm opt}$ & $\kappa_l^2 \kappa_n^2 \boldsymbol{\psi}_{\rm opt}^H \mathbf{Z}_1 \boldsymbol{\psi}_{\rm opt} +  ( \mu^2 + \kappa_n^2 \boldsymbol{\psi}_{\rm opt}^H \mathbf{Z}_2 \boldsymbol{\psi}_{\rm opt})\mathbf{f}_{\rm opt}^H \mathbf{R_{\rm BT}} \mathbf{f}_{\rm opt}$ \\
%\multicolumn{1}{|c|}{} & \multicolumn{2}{|c|}{with \mbox{\normalsize $\mathbf{F}=\mathbf{F}_{\rm s}$, $\mathbf{A}=\mathbf{A}_{\rm s}$,  $\mathbf{\Psi}=\boldsymbol{\psi\psi}^H$}} & & \\

\rowcolor{lavender}
\hline
\textbf{R2 IID} & $\frac{\mathbf{a}_M^H(\mathbf{\theta}_{\rm bd}^{i})}{\sqrt{M}}$ & $e^{-j\angle{{\rm diag}(\mathbf{\Bar{h}})\mathbf{a}_N(\mathbf{\theta}_{ra})}}$ & $N \sqrt{M} \kappa_l^2$ & $(M + 1) N\kappa_l^2 \kappa_n^2 + N \kappa_n^4 + \mu^2$ \\ \hline

\rowcolor{lavender}
\textbf{R3 Corr} & $\mathbf{v_{R_{\rm BT}}}$ & $\mathbf{1}_{\rm N} e^{j{\theta}}$ & $0$ & $\lambda_{\mathbf{R}_{\rm BT}} (\mu^2 + \|\mathbf{R}\|_F^2)$  \\ \hline

\rowcolor{lavender}
\textbf{R3 IID} & $\mathbf{f}_{\rm opt} \in \mathcal{S}_\mathbf{f}$ & $\boldsymbol{\psi}_{\rm opt} \in \mathcal{S}_{\boldsymbol{\psi}}$ & $0$ & $\mu^2 + N$  \\
%\bottomrule
\hline


\multicolumn{5}{|c|}{\textbf{*}~~$\mathbf{Z}_1 = \mathbf{R}_{\rm RT}\odot\mathbf{\Bar{H}f_{\rm opt}f_{\rm opt}}^H\mathbf{\Bar{H}}^H,~\mathbf{Z}_2 = \mathbf{R}_{\rm RR}\odot( \kappa_n^2\mathbf{R}_{\rm RT}+\kappa_l^2\mathbf{\Bar{h}}^*\mathbf{\Bar{h}}^T),~\lambda_{\mathbf{R}_{\rm BT}} = \lambda_{\rm max}\{\mathbf{R}_{\rm BT}\},$ $\mathbf{f}_{\rm opt}~\&~\boldsymbol{\psi}_{\rm opt}$ \Large{are obtained using the corresponding algorithms,}} \\


\multicolumn{5}{|c|}{\hspace{-13cm}$\mathbf{Z} ~~~=(N^2\kappa_l^4+N\kappa_l^2\kappa_n^2)\mathbf{e}_n^H\mathbf{e}_n + 2N\mu\kappa_l^3 \bar{\mathbf{g}}^*\bar{\mathbf{g}}^T$, \Large{and}~$\mathbf{e}_n=\mathbf{E}_{n,:}$ is the $n$-th row of $\mathbf{E} = {\rm diag}(\mathbf{\bar{h}})\mathbf{\Bar{H}}.$} \\
\hline
\end{tabular}}
\vspace{-0.5cm}
\end{table}
The following remarks present important insights derived using the  summary given in Table \ref{OP_Table} .
\begin{remark}\label{remark1}
%Form Corollary \ref{cor:outage_R3cor} and \ref{cor:outage_R3IID}
From Table \ref{OP_Table}, it can be noted that the parameter $m=\mathbb{E}[\xi_1+\mu\xi_2]$ is equal to zero for Rayleigh-Rayleigh (R3) fading case as the coefficients of channels  $\mathbf{g}$, $\mathbf{h}$, and $\mathbf{H}$ are zero-mean complex Gaussian distributed. 
Therefore, the mean SNR is given by  
$\Gamma(\mathbf{f},\mathbf{\Phi})=\mathbf{E}[|\xi_1+\mu\xi_2|^2]=\sigma^2$. However, the  maximum mean SNRs achievable via SCSI-based optimal beamforming corresponding to  {\rm i.i.d.} and correlated scenarios of R3 fading case are $\mu^2+N$ and $\lambda_{\mathbf{R}_{\rm BT}}(\mu^2+\|\mathbf{R}\|_F^2)$. 
Thus, using $\lambda_{\mathbf{R}_{\rm BT}}>1$  and $\|\mathbf{R}\|_F^2>N$, it is safe to conclude that the maximum mean SNR under correlated scenario is higher than it is under {\rm i.i.d} scenario.  This is quite evident as the optimal beamforming scheme for correlated scenarios can leverage the information of covariance matrices to maximize the mean SNR.  
However, under {\rm i.i.d.} scenario, the mean SNR does not rely on the choice of beamforming vectors because of the  circularly symmetric fading. Therefore, we can say that the {\rm i.i.d.} and  fully correlated fadings are the extreme scenarios wherein the corresponding achievable mean SNR realizes its extreme values such that
$$\mu^2+N\leq \Gamma(\mathbf{f}_{\rm opt},\mathbf{\Phi}_{\rm opt})\leq M(\mu^2+N^2),$$
where the upper bound corresponds to the fully correlated scenario for which $\lambda_{\mathbf{R}_{\rm BT}}=M$ and $\|\mathbf{R}\|_F^2=N^2$. It is interesting to note  that the maximum mean SNR under correlated R3 fading increases with order  between linear and quadratic in  the number of RIS elements
$N$.  
\end{remark}
\vspace{-0.4cm}
\begin{remark}\label{remark2}
    Using the parameters $m$ and $\sigma^2$ given in Table \ref{OP_Table}, the maximum  mean SNRs achievable via SCSI-based optimal beamforming under {\rm i.i.d.} scenario can be determined as 
    \begin{align*}
       \Gamma(\mathbf{f}_{\rm opt},\mathbf{\Phi}_{\rm opt})= \begin{cases}
            | N \kappa_l^2 \mathbf{e}_n \mathbf{v_Z} + \mu \kappa_l \Bar{\mathbf{g}}^T \mathbf{v_Z}|^2 + N \kappa_n^2 (1 + \kappa_l^2 |\mathbf{e}_n \mathbf{v_Z}|^2) + \mu^2 \kappa_n^2, ~~&\text{for R1 (LB) case}\\
            N^2M\kappa_l^4+ N((M+1)\kappa_l^2\kappa_n^2 + \kappa_n^2)+\mu^2, ~~&\text{for R2 case}\\
            \mu^2+N, ~~&\text{for R3 case}
        \end{cases}%
    \end{align*}
    % It can be seen that the maximum mean SNRs under R1 and R2 cases are quadratic in the number of RIS-elements $N$, whereas  the maximum mean SNRs under R1 is linear in $N$. This is attributed to the fact that the R1 and R2 cases consist the LoS components which allows us to optimally select beamformer for acquiring larger gains.  
    % However, the absence of LoS component under R3 only helps to increase the number of paths with $N$ which in turn causes the mean SNR to improve linear with $N$. Basically, addition of the deterministic LoS components (along indirect link in R2 and along indirect-direct links in R1) helps to increase the mean $m$ and variance $\sigma^2$ parameters of $\xi_1+\mu \xi_2$ with respective to R3 case, as can be verified from Table \ref{OP_Table}. This allows the beamformer to  efficiently  maximize $\Gamma(\mathbf{f},\mathbf{\Phi})=\mathbb{E}[|\xi_1+\mu\xi_2|^2]=\sigma^2+|m|^2$ further with respective to $\mathbf{f}$ and $\mathbf{\Phi}$ for higher gains. Moreover, the maximum mean SNR under R1 case depends on the dots product of vectors $\mathbf{e}_n$ and $\mathbf{\bar{g}}$ with $\mathbf{v_Z}$ which all depends on the AoDs/AoAs of the LoS components of the direct and indirect links.
    
    % Further, it can be seen that the maximum mean SNR under R2 case reduces to that is under R3 case as $K\to 0$. This is expected as R2 and R3 cases becomes the same for $K\to 0$ as the LoS component in R2 becomes significant.
The maximum mean SNRs under R1 and R2 cases increase quadratically with the number of RIS elements $N$, while  it increases linearly with $N$ under R3. This is because R1 and R2 cases include LoS components which allow efficient selection of beamformer for greater gains. In contrast, the absence of LoS components in R3 only increases the number of paths with $N$, resulting in linear improvement in mean SNR with $N$. In other words, the inclusion of deterministic LoS components in R1 and R2 increases the mean $m$ and variance $\sigma^2$  of $\xi_1 + \mu \xi_2$ compared to the R3 case, as can be verified from Table \ref{OP_Table}. This allows the beamformer to efficiently maximize $\Gamma(\mathbf{f},\mathbf{\Phi})=\mathbb{E}[|\xi_1+\mu\xi_2|^2]=\sigma^2+|m|^2$ further w.r.t $\mathbf{f}$ and $\mathbf{\Phi}$. \newline
In addition, it can be seen that the maximum mean SNR under the R1 case depends on the dot products of vectors $\mathbf{e}_n$ and $\mathbf{\bar{g}}$ with $\mathbf{v_Z}$, which are determined by the AoDs/AoAs of the LoS components in the direct and indirect links. 
Furthermore, as expected, the maximum mean SNR in R2 reduces to that of R3 as $K\to 0$, i.e. as the significance of LoS component in R2 diminishes.
       
\end{remark}
% \begin{corollary}
% The ergodic capacity of the statistically optimal beamforming scheme for the RIS-aided MISO system under i.i.d. Rayleigh-Rayleigh fading is given by
%     \begin{align}
%     {\rm EC}=\frac{1}{\ln(2)}\int_0^\infty \frac{1}{1+u}Q_1\left(0,\frac{\sqrt{{u}/{\gamma}}}{\sqrt{\sigma/2}}\right){\rm d}u.\label{EC_R3_IID}
%     \end{align}
%     where $m$ and $\sigma^2$ are given in \eqref{eq:parameter_m_R3iid} and \eqref{eq:parameter_sigma2_R3iid}, respectively.
% \end{corollary}

% \begin{center}
%\begin{table}[ht!]\caption{$\rm{X} = \mathbf{h}^T \mathbf{\Phi} \mathbf{H} \mathbf{f} + \mu \mathbf{g}^T \mathbf{f}$}\label{OP_Table}\centering\fontsize{18pt}{18pt}
% \Huge\renewcommand{\arraystretch}{2.5}\resizebox{\textwidth}{!}{%\begin{tabular}{|c|c|c|c|c|}\hline\textbf{Fading Scenario} & $\mathbf{f}_{\rm opt}$ & $\mathbf{\Psi_{\rm opt}}$ & $m$ & $\sigma^2$ \\ \hline\textbf{R1 Corr} & $\eqref{OptSol_f_R1}$ & \eqref{}&  $\kappa_l^2\mathbf{\Bar{h}}^T \mathbf{\Phi}\Bar{\mathbf{H}}\mathbf{f}  + \mu \kappa_l \Bar{\mathbf{g}}^T\mathbf{f}$ & $\kappa_l^2 \kappa_n^2 \boldsymbol{\psi}^H (\mathbf{R_{\rm RT}} \odot \mathbf{Z}_1) \boldsymbol{\psi} + \kappa_n^2 ( \mu^2 + \boldsymbol{\psi}^H (\mathbf{R_{\rm RR}} \odot \mathbf{Z}_2) \boldsymbol{\psi})\mathbf{f}^H \mathbf{R_{\rm BT}} \mathbf{f}$\\ \hline\textbf{R1 IID} & & \eqref{} & $\kappa_l^2\mathbf{\Bar{h}}^T \mathbf{\Phi}\Bar{\mathbf{H}}\mathbf{f}  + \mu \kappa_l \Bar{\mathbf{g}}^T\mathbf{f}$ & $\kappa^2_l \kappa^2_n \|\Bar{\mathbf{H}}\mathbf{f}\|^2 + \kappa^2_n ( \mu^2 + N(\kappa^2_l  + \kappa^2_n))$\\ \hline\textbf{R1 IID LB} & $\mathbf{v_1}\{\}$ & $\frac{N|\mathbf{e}_n\mathbf{f}|}{|\mathbf{g}^T\mathbf{f}|}\mathbf{g}^T\mathbf{f}\mathbf{w}$ & $N \kappa_l^2 \mathbf{e}_n \mathbf{f}^{\rm{\rm opt}} + \mu \kappa_l \Bar{\mathbf{g}}^T \mathbf{f}^{\rm{\rm opt}}$ & $N \kappa_n^2 (1 + \kappa_l^2 |\mathbf{e}_n \mathbf{f}^{\rm{\rm opt}}|^2) + \mu^2 \kappa_n^2 $ \\ \hline\textbf{R2 Corr} & $\mathbf{v_1}\{\}$ & \eqref{} & $\kappa_l^2\mathbf{\Bar{h}}^T \mathbf{\Phi}\Bar{\mathbf{H}}\mathbf{f}$ & $\kappa_l^2 \kappa_n^2 \boldsymbol{\psi}^H (\mathbf{R_{\rm RT}} \odot \mathbf{Z}_1) \boldsymbol{\psi} +  ( \mu^2 + \kappa_n^2 \boldsymbol{\psi}^H (\mathbf{R_{\rm RR}} \odot \mathbf{Z}_2) \boldsymbol{\psi})\mathbf{f}^H \mathbf{R_{\rm BT}} \mathbf{f}$ \\ \hline\textbf{R2 IID} & $\frac{\mathbf{a}_m^\star(\mathbf{\theta}_{BT})}{\sqrt{M}}$ & $e^{-j\angle{diag(\mathbf{\Bar{h}})\mathbf{a}_n(\mathbf{\theta}_{RR})}}$ & $N \sqrt{M} \kappa_l^2$ & $(M + 1) N\kappa_l^2 \kappa_n^2 + N \kappa_n^4 + \mu^2$ \\ \hline\textbf{R3 Corr} & $\mathbf{v_1}\{\mathbf{R}_{\rm BT}\}$ & $e^{j\theta}*[1~1 \cdots 1]^T; \rm{for~any}~\theta$ & $0$ & $\lambda_{\rm max}\{\mathbf{R}_{\rm BT}\} (\mu^2 + \|\mathbf{R}\|_F^2)$  \\ \hline\textbf{R3 IID} & $\text{any choice of}~\mathbf{f}$ & $\text{any choice of}~\boldsymbol{\psi}$ & $0$ & $\mu^2 + N$ \\ \hline\end{tabular}}\end{table}
% \end{center}



% \section{Correlation better than IID}
% In general, the performance of a system experiencing independent fading is better than the correlated fading scenario. This is because the channel matrix is well-conditioned. In other words, the spectral range of the channel is small, providing more degrees of freedom for transmission. Nonetheless, we often encounter correlated fading in practical scenarios. In our paper, we investigate the statistically optimal RIS-aided communication system in both the independent and correlated fading scenarios under different cases. The first case presented where, both the direct and indirect links experience correlated Rician fading, shows both the scenarios to be performing similarly. This is because of the presence of dominant LoS components. To see the impact of correlation, we assume one of the links (direct) to not have any LoS component (Rayleigh). As expected, the IID scenario performs better as the LoS component is still present along the indirect link. Hence, we study the third case where no LoS components are present along both the links. In this case, we see that the RIS-aided statistically optimal system provides better performance under correlated fading scenario. This may be due the fact that, the statistical design considers correlation in the joint problem, making it less susceptible to performance degradation due to correlation. 
% \newline
% \textcolor{purple}{
% \textbf{\hspace{7cm}Doubts} \newline
% \textbf{PCSI scheme}:\newline
% In general, IID is better than correlation. This is because the channel matrix has small spectral range under IID allowing more transmission degree of freedom. Verify through simulations 
% \newline
% \textbf{SCSI Scheme}:\newline
% Already have the simulations. Need to verify if the explanation regarding LoS.\newline
% \textbf{Strength of LoS component or high $K$}:\newline
% In theory, the presence of a strong LoS component (high $K$) also impacts the achievable capacity as it reduces the rank of the channel matrix. We need to investigate if these same patterns follow in PCSI and SCSI schemes.\newline
% \textbf{Impact of Correlated Diversity Branches in Rician Fading Channels}\newline
% The angle of the LoS component has a lot of impact on the effective K factor at the receiver: This determines, the correlated scenario performing better than IID sometimes.\newline
% \textbf{On capacity of Rician MIMO channels}
% This paper links the number of antennas with the performance in Rician and Rayleigh. \newline
% \textbf{Performance Evaluation and Diversity Analysis of RIS-Assisted Communications Over Generalized Fading Channels in the Presence of Phase Noise}
% }
\vspace{-0.7cm}
\section{Numerical Results and Discussion}\vspace{-.1cm}
This section presents the performance of the proposed SCSI-based beamforming schemes for different fading scenarios and their comparisons with the PCSI-based SNR maximization scheme. 
% Recall that  Rician-Rician, Rician-Rayleigh, and Rayleigh-Rayleigh fading are referred to as R1, R2, and R3, respectively. 
First, we verify the derived outage performances of these schemes via simulations for both {\rm i.i.d.} and correlated scenarios. Then, we will discuss their achievable ECs for various system parameters. For the numerical analysis, we set the parameters as follows: the number of BS antennas $M = 4,$ number of RIS elements $N = 32$, Rician factor $K = 2$, the AoD from  BS to  the user via  direct link $\theta_{\rm{bd}}^{\rm{d}} = 0^\circ$, the AoD from  BS to  RIS $\theta_{\rm{ra}} = \pi/4$, the AoD from  RIS to  user $\theta_{\rm{rd}} = 8 \pi/5$, $\gamma = 1$, and PLR $\mu = 5~\rm{dB}$ unless mentioned otherwise. 
%Fig 1: OP (R1, R2, R3), %Fig 2: OP (R1)
\begin{figure}[ht!]
\centering\vspace{-.5cm}
\begin{minipage}{.4\textwidth}
  \centering
  \includegraphics[width=\textwidth]{25April_Figures/All_OP.eps}
\end{minipage}%
\begin{minipage}{.4\textwidth}
  \centering
  \includegraphics[width=\textwidth]{25April_Figures/R1_OP_new.eps} 
\end{minipage}\vspace{-.2cm}
\caption{Left: OP verification for R1, R2, R3 cases. Right: LB OP accuracy for the R1 case and its comparison with the PCSI scheme. Markers represent the simulation results, whereas the solid lines represent the analytical results.}\vspace{-.5cm}
\label{Fig1}
\end{figure}


Fig. \ref{Fig1} shows the outage performance of the proposed beamforming schemes for different fading scenarios. It may be noted that OPs for R1 {\rm i.i.d.} lower bound, R2 {\rm i.i.d.}, and R3 are derived in \eqref{Pout_R1_IID}, \eqref{Pout_R2_IID}, and \eqref{Pout_R3} respectively. However, OPs for R1 Corr and R2 Corr are derived for given transmit beamformer $\mathbf{f}$ and RIS phase shift matrix $\mathbf{\Phi}$ (refer to Tabel \ref{OP_Table}). Thus, we use the statistically optimal $\mathbf{f}$ and $\mathbf{\Phi}$ obtained through the proposed algorithms to evaluate the outage for the corresponding scenarios. Fig. \ref{Fig1} (Left) shows the accuracy of the derived approximate OP expressions under all fading scenarios where they closely match the simulation results. It can be observed that OP improves under R3, R2, and R1 cases in order. This improvement is because the statistically fixed beamformer performs better in the presence of strong LoS components, which is the case in R1 and R2. Furthermore, it can be seen that the correlated fading scenario provides better OP than {\rm i.i.d.} under R3, whereas the trend is  opposite under R2. This is because the statistical beamformer can exploit the correlated channel under R3, as is evident from \autoref{OP_Table}. In the presence of LoS under R2, the statistically fixed beamformer can utilize the independent fading along with the direct link for higher gains. However, OP is equal for both correlated and {\rm i.i.d.} scenarios under R1. Fig. \ref{Fig1} (Right) verifies the LB OP accuracy derived for the R1 {\rm i.i.d.} case and compares it with the PCSI scheme. The figure shows that the outage LB is tight for $N = 16$ and $N = 32$. As noted above, the SCSI-based scheme performs equally good for correlated and {\rm i.i.d.} scenarios under R1 fading. However, the PCSI scheme provides a better outage under the {\rm i.i.d.} scenario. This is because the instantaneous beamformer under the PCSI scheme with {\rm i.i.d.} scenario utilizes the large spectral range of the channel matrix. 
%Fig 3: EC Vs. N (R1, R2, R3), %Fig 4: EC Vs. N (R1)
 \begin{figure}[ht!]
\centering\vspace{-.5cm}
\begin{minipage}{0.4\textwidth}
  \centering
  \includegraphics[width=\textwidth]{25April_Figures/All_N_new.eps}
\end{minipage}%
\begin{minipage}{0.4\textwidth}
  \centering
  \includegraphics[width=\textwidth]{25April_Figures/R1_N_new.eps}
\end{minipage}\vspace{-.2cm}
\caption{Left: EC vs. $N$ under R1, R2, R3 cases. Right: LB EC accuracy for R1 case and its comparison with the PCSI scheme.}\vspace{-.5cm}
\label{Fig2}
\end{figure}

Fig. \ref{Fig2} shows EC performance with respect to the number of RIS elements $N$ under R1, R2, and R3 cases. In Fig. \ref{Fig2} (Left), we see that EC increases with $N$ in all the cases, as expected. Particularly, the capacity increases rapidly in R1 and R2 cases compared to R3. 
This may be because the SCSI-based scheme  for R1 and R2 cases almost follows  performance trends similar to  the PCSI-based scheme for which the SNR is known to improve quadratically with $N$. Therefore, one can expect that the mean SNR will also improve quadratically in $N$ under R1 and R2 cases. In fact, this is shown to be the case in Remark \ref{remark2} for {\rm i.i.d.} scenario. However, the mean SNR in R3 improves with order between linear  and quadratic   in terms of $N$, as can be verified using Remarks \ref{remark1}.
%\textcolor{red}{This is because the SNR increases in general quadratically in $N$, which is quite well known for the PCSI scheme \cite{RIS_Corr_Fad}. Consequently as we observe the SCSI scheme to follow the PCSI scheme closely, it would naturally help improve the mean SNR quadratically in $N$.
%Thus, the presence of LoS components under the SCSI scheme for R1 and R2 cases would naturally help improve the mean SNR quadratically in $N$. This can be confirmed from \eqref{OptProb_R1_IID} for R1 IID and from \eqref{OptProb_R2_IID} for R2 IID cases. 
%However, the mean SNR in R3 improves linearly with $N$ (because of the absence of LoS), as can be seen from \eqref{OptProb_R3_IID}.} 
Further, it can be observed that the performance difference between R1 and R2 saturates with the increase in $N$. This is because the large $N$ compensates for the loss due to the absence of LoS components along the direct link in R2. In other words, EC gain is contributed by two factors: 1) the number of reflecting elements $N$ and 2) the presence of LoS components. As both of these factors are present in the R1 case, it naturally outperforms the R2 and R3 cases. However, in R2, the LoS component along the direct link is absent, which reduces its capacity performance relatively. Nonetheless, the huge gain achieved by increasing $N$ helps to compensate for this relative loss. Therefore, the difference in gains under R1 and R2 saturates. Furthermore, it can be observed that the capacity performance trends in terms of the correlated and {\rm i.i.d.} scenarios for R1, R2, and R3 cases are similar to their outage performances as discussed in Fig. \ref{Fig1}. Fig. \ref{Fig2} (Right) verifies that the LB EC derived in \eqref{Pout_R1_IID} for the R1 {\rm i.i.d.} case is tight for a wide range of $N$. Moreover, it can be observed that the PCSI scheme provides better capacity than the SCSI scheme, as expected. However, unlike the correlated and {\rm i.i.d.} scenarios of R1 having equal performances under the SCSI-based scheme, R1 {\rm i.i.d.} performs better than R1 correlated case under the PCSI-based scheme. %\textcolor{red}{This is because the independent fading provides higher diversity which can be exploited by the ability of the PCSI scheme for instantaneous beamforming, as mentioned before.}
%%Fig 5: EC Vs. mu (R1)
\begin{figure}[ht!]
\centering\vspace{-.4cm}
\begin{minipage}{.4\textwidth}
  \centering
  \includegraphics[width=\textwidth]{25April_Figures/R1_theta_new.eps}
\end{minipage}%
\begin{minipage}{.4\textwidth}
  \centering
  \includegraphics[width=\textwidth]{25April_Figures/Radiation_Pattern_10dB.eps}
\end{minipage}\vspace{-.2cm}
\caption{Left: EC vs. $\theta$ under R1 for correlated and {\rm i.i.d.} fading scenarios. Right: Radiation pattern for $\mathbf{f}_{\rm opt}$ with $N = 32$.}\vspace{-.6cm}
\label{Fig3}
\end{figure}

Fig. \ref{Fig3} (Left) shows that EC deteriorates  as $\theta$ increases where $\theta = |\theta_{\rm{DBR}} - \theta_{\rm{DBU}}|$ is the difference between AoDs of the direct and indirect links from the BS. This is because the optimal beamformer $\mathbf{f}$ forms a narrow lobe when $\theta$ is small. However, for larger $\theta$, the optimal beamformer $\mathbf{f}$ has a relatively wider beam %two lobes 
to exploit the gains from both the direct and indirect links. This can also be observed from Fig. \ref{Fig3} (Right), where we plot the radiation pattern for an optimal transmit beamformer. The figure verifies that the main lobe beamwidth gradually increases in $\theta$ reducing the overall array gain. However, as the indirect link strength improves by increasing $N$, it can be seen that the drop in capacity with increasing $\theta$ becomes less severe. Further, it can  be observed that the EC LB is tight for any $\theta$, especially for large $N$. 
%Next, to re-emphasize the accuracy of the lower bound, we can see that it is very tight for different values of $N$ and $K$. However, it can be seen that as $\mu$ increases, $N = 16, N = 32$ perform similarly. This is because there is less dependency on the indirect path through RIS when the direct link is strong.
%%Fig 7: EC Vs. d (R1)
\begin{figure}[ht!]
\centering\vspace{-.4cm}
\begin{minipage}{.33\textwidth}
  \centering
  \includegraphics[width=\textwidth]{25April_Figures/R1_mu_new.eps}
\end{minipage}%
\begin{minipage}{.33\textwidth}
  \centering
  \includegraphics[width=\textwidth]{25April_Figures/All_d_Final_new.eps}
\end{minipage}\vspace{-.3cm}
\begin{minipage}{.33\textwidth}
  \centering
  \includegraphics[width=\textwidth]{25April_Figures/All_K_Final2_new.eps}
\end{minipage}
\caption{Left: EC vs. the PLR $\mu$. Center: EC vs. distance between antennas $d$ for $M = 32$. Right: EC vs. Rician factor $K$.}\vspace{-.6cm}
\label{Fig4}
\end{figure}

Fig. \ref{Fig4} (Left) shows EC as a function of PLR $\mu$. It can be seen that the capacity increases with the increase in $\mu$. However, it can also be seen that the SCSI-based scheme performs equally when $\mu$ is large. This is mainly because the strong direct link (i.e., high $\mu$) provides enough capacity such that the presence of RIS cannot improve the mean SNR  further. Furthermore, it may be noted that the LB is accurate for a wide range of $\mu$.  
Fig. \ref{Fig4} (Center) shows the impact of the distance $d$ between consecutive antennas on EC performance under the SCSI-based scheme. Note that the correlation between the antenna elements depends on $d$ (refer to Section \ref{channel model}). Thus, Fig. \ref{Fig4} (Right) captures the impact of correlated fading on the performance of EC. First, we observe that EC under R1 correlated case remains constant as $d$ increases and is similar to the performance of the R1 {\rm i.i.d.} case. Moreover, the capacity under R3 correlated case reduces with increasing $d$ and finally converges to the capacity achieved under the R3 {\rm i.i.d.} case. This also verifies the fact that the correlated setting performs better than the {\rm i.i.d.} scenario under the R3 case, as pointed out before. Interestingly, we can observe that EC of the R3 correlated case gradually drops in a pattern with increase in $d$ and become exactly equal to capacity under the R3 {\rm i.i.d.} case when $d$ is equal to $\frac{\lambda}{2},~\lambda,~\frac{3\lambda}{2},~2 \lambda$. This is because the correlation matrix is constructed using a model based on the Sinc function which goes to zero for the above values of $d$ (meaning the correlated case reduces to  {\rm i.i.d.} for these $d$'s).
%%Fig 8: EC Vs. K (R1, R2, R3)
% \begin{figure}[ht!]\vspace{-.3cm}
% \centering
% %\begin{minipage}{.5\textwidth}
%   \centering
%   \includegraphics[width=.4\textwidth]{25April_Figures/All_K_Final2.eps}
% %\end{minipage}%
% \vspace{-.4cm}
% \caption{EC vs. Rician factor $K$.}
% \label{Fig5}\vspace{-.5cm}
% \end{figure}
Fig. \ref{Fig4} (Right) shows that the EC increases with $K$. This is expected as $K$ indicates the strength of the LoS component that is crucial for achieving capacity under SCSI-based schemes. Furthermore, it can also be seen that the performances of the PCSI- and proposed SCSI-based schemes for the R1 case converge to the same constant value for large $K$. This is because the multi-path fading vanishes as $K \to \infty$, which implies the channel's deterministic nature for which PCSI- and SCSI-based beamforming schemes are the same. The figure also verifies the accuracy of LB EC  for a wide range of $K$.  
\vspace{-0.3cm}
\section{Conclusion}\vspace{-.2cm}
This paper investigates the statistically optimal transmit beamforming and phase shift matrix problem of a RIS-aided MISO  downlink communication system. We considered a generalized fading scenario wherein  both the direct and indirect links consist LoS paths  along with the multi-path components. Moreover, we  considered correlated fading to account for the practical deployment of RISs containing densely packed antenna elements. 
For this setting, we proposed an iterative algorithm  to optimally select the beamformers that  maximize the  mean SNR. In addition, we also derived approximately achievable OP and EC of the proposed algorithm. 
As is usually the case in the literature, this analysis relied on the numerical evaluations of the optimal beamforming solution. In order to get more crisp analytical insights, we further derived closed-form expressions for the optimal beamforming under the absence of LoS components and/or correlated fading along the direct and/or indirect links.     
%Nonetheless, the iterative solution is limited by analysis that can show the functional dependence of parameters such as the Rician fading factor $K$, number of RIS elements $N$, etc. So, we investigate some special fading cases to derive closed-form or computationally efficient solutions. 
%We further derive outage probability and ergodic capacity expressions wherever possible. 
Our analysis revealed some interesting interrelations between different fading scenarios.
For instance, we have shown that the maximum mean SNR improves linearly/quadratically with the number of RIS elements in the absence/presence of LoS component under {\rm i.i.d.} fading. 
Further, we have analytically shown that the statistically optimal beamforming performs better under correlated fading compared to {\rm i.i.d.} case in the absence of LoS paths. 
Our numerical results also show that the correlated fading is advantageous/disadvantageous compared to the  {\rm i.i.d.} case in the absence/presence of LoS paths. 
%Finally, we provide interesting insights into the achievable capacity performance trends and the proposed solutions for various all the fading cases proposed via simulations.    
\vspace{-1cm}\appendix\vspace{-.5cm}
\subsection{Proof of \eqref{eq:Mean_SNR_R1}}\label{AppA}\vspace{-.3cm}
The mean {\rm SNR} for the channel model described in Section \ref{channel model} can be obtained as
\begin{align}
    &\Gamma(\mathbf{f}, \mathbf{\Phi})
    = \mathbb{E}[(\mathbf{h}^T\mathbf{\Phi Hf} + \mu \mathbf{g}^T\mathbf{f})(\mathbf{h}^T\mathbf{\Phi Hf} + \mu \mathbf{g}^T\mathbf{f})^H]\nonumber\\
     % &= \mathbb{E}[((\kappa_l\mathbf{\Bar{h}} + \kappa_n\mathbf{\Tilde{h}})^T\mathbf{\Phi}(\kappa_l\mathbf{\Bar{H}} + \kappa_n\mathbf{\Tilde{H}})\mathbf{f} + \mu(\kappa_l\mathbf{\Bar{g}} + \kappa_n\mathbf{\Tilde{g}})^T\mathbf{f})\times\nonumber\\
     %  &~~((\kappa_l\mathbf{\Bar{h}} + \kappa_n\mathbf{\Tilde{h}})^T\mathbf{\Phi}(\kappa_l\mathbf{\Bar{H}} + \kappa_n\mathbf{\Tilde{H}})\mathbf{f} + \mu(\kappa_l\mathbf{\Bar{g}} + \kappa_n\mathbf{\Tilde{g}})^T\mathbf{f})^H]\nonumber\\
     &\stackrel{(a)}{=} |\kappa_l^2\mathbf{\Bar{h}}^T\mathbf{\Phi}\mathbf{\Bar{H}f} + \kappa_l\mathbf{\Bar{g}}^T\mathbf{f}|^2+ \kappa_l^2\kappa_n^2 \mathbb{E}[|\mathbf{\Bar{h}}^T\mathbf{\Phi}\mathbf{\Tilde{H}f}|^2] + \kappa_l^2\kappa_n^2\mathbb{E}[|\mathbf{\Tilde{h}}^T\mathbf{\Phi}\mathbf{\Bar{H}f}|^2] + \kappa_n^4\mathbb{E}[|\mathbf{\Tilde{h}}^T\mathbf{\Phi}\mathbf{\Tilde{H}f}|^2].\label{ESNR}  
\end{align}
where step (a) is obtained by substituting $\mathbf{g}$, $\mathbf{h}$ and $\mathbf{H}$ from \eqref{directlink},\eqref{indirect link_h} and \eqref{indirect link_H} and simplifying. 
% We simplify these expectations using the following two identities
% \begin{align*}
%   \mathbb{E}[\mathbf{X A}\mathbf{X}^H] &= \rm{trace}\{\mathbf{A}\}\mathbf{I}, \\ 
%   \rm{trace}\{\mathbf{Q}^H \mathbf{A} \mathbf{Q} \mathbf{B}\} &= \mathbf{q}^H (\mathbf{A}^T \odot \mathbf{B}) \mathbf{q}.
% \end{align*}
To  simplify further, we will use the following two identities.
%\begin{itemize}
    \newline$\bullet$ [I-1] For given $\mathbf{A}$ and $\mathbf{X}_{:,i} \in \mathcal{CN}(0, \mathbf{I})$, we have [Reference]
            $\mathbb{E}[\mathbf{X A}\mathbf{X}^H] = \rm{trace}\{\mathbf{A}\}\mathbf{I}.$ 
    %\vspace{-0.6cm}        
    \newline$\bullet$ [I-2] For given $\mathbf{A}$ and $\mathbf{B}$, we have
            $${\rm trace}\{\mathbf{\Phi}^H \mathbf{A} \mathbf{\Phi} \mathbf{B}\} \stackrel{(a)}{=} \sum\nolimits_{i,j}\boldsymbol{\psi}^*_i\boldsymbol{\psi}_j\mathbf{A}_{ij}\mathbf{B}_{ji}\stackrel{(b)}{=}\boldsymbol{\psi}^H(\mathbf{A}^T\odot \mathbf{B})\boldsymbol{\psi},$$
            where steps (a) and (b) follow using $\mathbf{\Phi}={\rm diag}(\boldsymbol{\psi})$ and $\boldsymbol{\psi}^H\mathbf{X}\boldsymbol{\psi}=\sum_{i,j}\boldsymbol{\psi}_i^*\mathbf{X}_{ij}\boldsymbol{\psi}_j$, respectively. 
%\end{itemize} 
% \begin{align}
%     \rm{trace}\{\mathbf{\Phi}^H \mathbf{A} \mathbf{\Phi} \mathbf{B}\} &= \sum_{i,j}\boldsymbol{\psi}^*_i\boldsymbol{\psi}_j\mathbf{A}_{ij}\mathbf{B}_{ji}
%  =\boldsymbol{\psi}^H(\mathbf{A}^T\odot \mathbf{B})\boldsymbol{\psi}
% \end{align}
% where the first equality follows using $\mathbf{\Phi}={\rm diag}(\boldsymbol{\psi})$ and the second equality follows using $\boldsymbol{\psi}^H\mathbf{X}\boldsymbol{\psi}=\sum_{i,j}\boldsymbol{\psi}_i^H\mathbf{X}_{ij}\boldsymbol{\psi}_j$. 
%It is to be noted that $\mathbf{Q}$ is a diagonal matrix, with $\mathbf{q} = \rm{diag}(\mathbf{Q})$. \\
%Now, the second term in RHS of \eqref{ESNR} can be simplified as
% \begin{align}
%     \mathbb{E}[|\mathbf{\Bar{h}}^T\mathbf{\Phi}\mathbf{\Tilde{H}f}|^2] %&=\mathbb{E}[\mathbf{f}^H \Tilde{\mathbf{H}}^H \mathbf{\Phi}^H \Bar{\mathbf{h}}^* \Bar{\mathbf{h}}^T \mathbf{\Phi} \Tilde{\mathbf{H}} \mathbf{f}] 
%     &= \mathbb{E}[\mathbf{f}^H \Tilde{\mathbf{R}}_{\rm BT} \Tilde{\mathbf{H}}^H_W \Tilde{\mathbf{R}}_{\rm RR}\mathbf{\Phi}^H \Bar{\mathbf{h}}^* \Bar{\mathbf{h}}^T \mathbf{\Phi} \Tilde{\mathbf{R}}_{\rm RR} \Tilde{\mathbf{H}}_W \Tilde{\mathbf{R}}_{\rm BT}\mathbf{f}], \nonumber\\
%     &=\mathbf{f}^H\mathbf{R}_{\rm BT}\mathbf{f} \times {\rm trace}(\Tilde{\mathbf{R}}_{\rm RR}\mathbf{\Phi}^H \Bar{\mathbf{h}}^* \Bar{\mathbf{h}}^T \mathbf{\Phi} \Tilde{\mathbf{R}}_{\rm RR}),\nonumber\\
%     &\stackrel{(a)}{=} \mathbf{f}^H\mathbf{R}_{\rm BT}\mathbf{f} \times {\rm trace}({\mathbf{R}}_{\rm RR}\mathbf{\Phi}^H \Bar{\mathbf{h}}^* \Bar{\mathbf{h}}^T \mathbf{\Phi}),\nonumber\\
%     &\stackrel{(b)}{=}\mathbf{f}^H\mathbf{R}_{\rm BT}\mathbf{f} \times \boldsymbol{\psi}^H({\mathbf{R}}_{\rm RR} \odot \mathbf{\Bar{h}}^*\mathbf{\Bar{h}}^T)\boldsymbol{\psi}.\label{Exp1}\\
%     &\hspace{-2.4cm}\text{where   Step (a) follows from the identity I-1, and the  Step (b) from identity I-2. Similarly,}\nonumber \\ 
%     \mathbb{E}[|\mathbf{\Tilde{h}}^T\mathbf{\Phi}\mathbf{\Bar{H}f}|^2] &=\mathbb{E}[\mathbf{f}^H\mathbf{\Bar{H}}^H\mathbf{\Phi}^H\mathbf{\Tilde{h}}^*\mathbf{\Tilde{h}}^T\mathbf{\Phi}\mathbf{\Bar{H}}\mathbf{f}],\nonumber\\  &=\mathbf{f}^H\mathbf{\Bar{H}}^H\mathbf{\Phi}^H\mathbf{R}_{\rm RT}\mathbf{\Phi\Bar{H}f} = {\rm trace}(\mathbf{\Phi}^H\mathbf{R}_{\rm RT}\mathbf{\Phi}\Bar{\mathbf{H}}\mathbf{f}\mathbf{f}^H\mathbf{\Bar{H}}^H),\nonumber\\
%     &=\boldsymbol{\psi}^H(\mathbf{R}_{\rm RT} \odot \mathbf{\Bar{H}f}\mathbf{f}^H\mathbf{\Bar{H}}^H)\boldsymbol{\psi}.\label{Exp2}\\
%     \mathbb{E}[|\mathbf{\Tilde{h}}^T\mathbf{\Phi}\mathbf{\Tilde{H}f}|^2] 
%     &=\mathbb{E}[\mathbf{f}^H \Tilde{\mathbf{H}}^H \mathbf{\Phi}^H \Tilde{\mathbf{h}}^* \Tilde{\mathbf{h}}^T \mathbf{\Phi} \Tilde{\mathbf{H}} \mathbf{f}] = \mathbb{E}[\mathbf{f}^H \Tilde{\mathbf{R}}_{\rm BT} \Tilde{\mathbf{H}}^H_W \Tilde{\mathbf{R}}_{\rm RR}\mathbf{\Phi}^H \Tilde{\mathbf{h}}^* \Tilde{\mathbf{h}}^T \mathbf{\Phi} \Tilde{\mathbf{R}}_{\rm RR} \Tilde{\mathbf{H}}_W \Tilde{\mathbf{R}}_{\rm BT}\mathbf{f}], \nonumber\\
%         &=\mathbf{f}^H\mathbf{R}_{\rm BT}\mathbf{f} \times {\rm trace}(\Tilde{\mathbf{R}}_{\rm RR}\mathbf{\Phi}^H \mathbf{R}_{\rm RT} \mathbf{\Phi} \Tilde{\mathbf{R}}_{\rm RR}) = \mathbf{f}^H\mathbf{R}_{\rm BT}\mathbf{f} \times {\rm trace}({\mathbf{R}}_{\rm RR}\mathbf{\Phi}^H \mathbf{R}_{\rm RT} \mathbf{\Phi}),\nonumber\\
%     &=\mathbf{f}^H\mathbf{R}_{\rm BT}\mathbf{f} \times \boldsymbol{\psi}^H({\mathbf{R}}_{\rm RR} \odot \mathbf{R}_{\rm RT})\boldsymbol{\psi}.\label{Exp3}
% \end{align}

Now, the second term in RHS of \eqref{ESNR} can be simplified as
\begingroup
\allowdisplaybreaks
\begin{align}
    \mathbb{E}[|\mathbf{\Bar{h}}^T\mathbf{\Phi}\mathbf{\Tilde{H}f}|^2] %&=\mathbb{E}[\mathbf{f}^H \Tilde{\mathbf{H}}^H \mathbf{\Phi}^H \Bar{\mathbf{h}}^* \Bar{\mathbf{h}}^T \mathbf{\Phi} \Tilde{\mathbf{H}} \mathbf{f}] 
    &= \mathbb{E}[\mathbf{f}^H \Tilde{\mathbf{R}}_{\rm BT} \Tilde{\mathbf{H}}^H_W \Tilde{\mathbf{R}}_{\rm RR}\mathbf{\Phi}^H \Bar{\mathbf{h}}^* \Bar{\mathbf{h}}^T \mathbf{\Phi} \Tilde{\mathbf{R}}_{\rm RR} \Tilde{\mathbf{H}}_W \Tilde{\mathbf{R}}_{\rm BT}\mathbf{f}], \nonumber\\
     &\stackrel{(a)}{=}\mathbf{f}^H \Tilde{\mathbf{R}}_{\rm BT}{\rm trace}(\Tilde{\mathbf{R}}_{\rm RR}\mathbf{\Phi}^H \Bar{\mathbf{h}}^* \Bar{\mathbf{h}}^T \mathbf{\Phi} \Tilde{\mathbf{R}}_{\rm RR})  \mathbf{I}\Tilde{\mathbf{R}}_{\rm BT}\mathbf{f}, \nonumber\\
    %&=\mathbf{f}^H\mathbf{R}_{\rm BT}\mathbf{f} \times {\rm trace}(\Tilde{\mathbf{R}}_{\rm RR}\mathbf{\Phi}^H \Bar{\mathbf{h}}^* \Bar{\mathbf{h}}^T \mathbf{\Phi} \Tilde{\mathbf{R}}_{\rm RR}),\nonumber\\
    &= \mathbf{f}^H\mathbf{R}_{\rm BT}\mathbf{f} \times {\rm trace}({\mathbf{R}}_{\rm RR}\mathbf{\Phi}^H \Bar{\mathbf{h}}^* \Bar{\mathbf{h}}^T \mathbf{\Phi}),\nonumber\\
    &\stackrel{(b)}{=}\mathbf{f}^H\mathbf{R}_{\rm BT}\mathbf{f} \times \boldsymbol{\psi}^H({\mathbf{R}}_{\rm RR} \odot \mathbf{\Bar{h}}^*\mathbf{\Bar{h}}^T)\boldsymbol{\psi}.\label{Exp1}
\end{align}
\endgroup
where   step (a) follows from the identity I-1, and the  step (b) from identity I-2. Similarly, using I-1 and I-2, we simplify the third and fourth terms in  the RHS of \eqref{ESNR} as below
\begin{align}
    \mathbb{E}[|\mathbf{\Tilde{h}}^T\mathbf{\Phi}\mathbf{\Bar{H}f}|^2] &=\boldsymbol{\psi}^H(\mathbf{R}_{\rm RT} \odot \mathbf{\Bar{H}f}\mathbf{f}^H\mathbf{\Bar{H}}^H)\boldsymbol{\psi},\label{Exp2}\\
   \text{~~and~~} \mathbb{E}[|\mathbf{\Tilde{h}}^T\mathbf{\Phi}\mathbf{\Tilde{H}f}|^2] 
    &=\mathbf{f}^H\mathbf{R}_{\rm BT}\mathbf{f} \times \boldsymbol{\psi}^H({\mathbf{R}}_{\rm RR} \odot \mathbf{R}_{\rm RT})\boldsymbol{\psi}.\label{Exp3}
\end{align}
Further, substituting \eqref{Exp1}, \eqref{Exp2}, and \eqref{Exp3} in \eqref{ESNR}, we obtain \eqref{eq:Mean_SNR_R1}.
\vspace{-.4cm}
\subsection{Distributions of $\xi_1$ and $\xi_2$}\label{AppB}
%{\rm We will add the proof here soon...}
We first derive the means and variances of $\xi_1=\mathbf{h}^T\mathbf{\Phi}\mathbf{H f}$ and $\xi_2=\mathbf{g}^T\mathbf{f}$.
 From the definitions of $\mathbf{H}$, $\mathbf{h}$ and $\mathbf{g}$ given in Section \ref{channel model} and their independence, the means of $\xi_1$ and $\xi_2$ becomes
 \begin{align}
       \mu_1=\mathbb{E}[\xi_1] =  \mathbb{E}[\mathbf{h}^T\mathbf{\Phi}\mathbf{H f}]=\kappa_l^2 \Bar{\mathbf{h}} \mathbf{\Phi} \Bar{\mathbf{H}} \mathbf{f},~~\text{and}~~ \mu_2=\mathbb{E}[\xi_2] = \mathbb{E}[\mathbf{g}^T\mathbf{f}]=\kappa_l\mathbf{\bar{g}}^T\mathbf{f}.  \label{appendixB_mu1mu2}
 \end{align}
 %Here $\mathbf{g}$, $\mathbf{h}$, and $\mathbf{H}$ are independent Gaussian random vectors with distributions $\mathbf{g} \sim \mathcal{CN}(\kappa_l \Bar{\mathbf{g}}, \kappa_n \mathbf{R}_{\rm BT})$, $\mathbf{h} \sim \mathcal{CN}(\kappa_l \Bar{\mathbf{h}}, \kappa_n \mathbf{R}_{\rm RT})$ and $\mathbf{H} = \Tilde{\mathbf{R}}_{\rm RR} \Tilde{\mathbf{H}}_w \Tilde{\mathbf{R}}_{\rm BT}; \Tilde{\mathbf{H}}_{w(:,i)} \sim \mathcal{CN}(0, \mathbf{I})$ respectively. Since Gaussianity is preserved by linear transformation, we conclude that the $\mathbf{h}^T\mathbf{\Phi}$, $\mathbf{H f}$, $\mathbf{g}^T \mathbf{f}$ are Gaussian distributed with means $\mathbb{E}[\mathbf{g}^T\mathbf{f}] = \kappa_l \Bar{\mathbf{g}} \mathbf{f}$, $\mathbb{E}[\mathbf{h}^T\mathbf{\Phi}] = \kappa_l \Bar{\mathbf{h}} \mathbf{\Phi}$, and $\mathbb{E}[\mathbf{H f}] = \kappa_l \Bar{\mathbf{H}} \mathbf{f}$ respectively, and variances the same as above. Thus, we conclude by saying that $\xi_1 = \mathbf{h}^T\mathbf{\Phi}\mathbf{H f}, \xi_2 = \mathbf{g}^T\mathbf{f}$ are non-zero mean, Gaussian distributed, and hence $|\xi_1 + \mu \xi_2|$ is Rice distributed.  
The variance of $\xi_1$  can be obtained as
 \begin{align}
       \sigma_1^2&=\mathbb{E}[\mathbf{h}^T\mathbf{\Phi}\mathbf{H}\mathbf{f}(\mathbf{h}^T\mathbf{\Phi}\mathbf{H}\mathbf{f})^H]-|\mu_1|^2,\nonumber\\
       &=  \kappa_l^2\kappa_n^2 \mathbb{E}[|\mathbf{\Bar{h}}^T\mathbf{\Phi}\mathbf{\Tilde{H}f}|^2] + \kappa_l^2\kappa_n^2\mathbb{E}[|\mathbf{\Tilde{h}}^T\mathbf{\Phi}\mathbf{\Bar{H}f}|^2] + \kappa_l^2\kappa_n^2\mathbb{E}[|\mathbf{\Tilde{h}}^T\mathbf{\Phi}\mathbf{\Tilde{H}f}|^2],\nonumber\\
       &=\kappa_l^2\kappa_n^2\boldsymbol{\psi}^H\mathbf{Z}_1 \boldsymbol{\psi} + \kappa_n^2\mathbf{f}^H\mathbf{R}_{\rm BT}\mathbf{f}[\boldsymbol{\psi}^H\mathbf{Z}_2\boldsymbol{\psi}],\label{appendixB_sigam_1}
 \end{align}
 where the last equality follows using \eqref{Exp1}, \eqref{Exp2} and \eqref{Exp3}. The variance of $\xi_2$ becomes 
 \begin{align}
     \sigma_2^2&=\mathbb{E}[\mathbf{f}^H\mathbf{{g}}^*\mathbf{{g}}^T\mathbf{f}]-|\mu_2|^2=\kappa_n^2\mathbf{f}^H\mathbf{R}_{\rm BT}\mathbf{f}.
 \end{align}
We now comment on the distributions of $\xi_1$ and $\xi_2$. Note that $\xi_2 = \kappa_l \Bar{\mathbf{g}}^T\mathbf{f} + \kappa_n \Tilde{\mathbf{g}}^T\mathbf{f}$ 
where $\Bar{\mathbf{g}}$ is deterministic and $\Tilde{\mathbf{g}} \sim \mathcal{CN}(0, \mathbf{R}_{\rm BT})$. It can be easily shown that $\Tilde{\mathbf{g}}^T\mathbf{f} \sim \mathcal{CN}(0, \mathbf{f}^H\mathbf{R}_{\rm BT}\mathbf{f})$. Thus,  $\xi_2$ becomes complex Gaussian with mean $\mu_2$ and variance $\sigma^2_2$ as given in \eqref{eq:xi1xi2}.
Next, $\xi_1=\mathbf{h}^T\mathbf{\Phi}{\mathbf{H}}\mathbf{f}$ can be expanded as 
$\xi_1  = \kappa_l^2 \Bar{\mathbf{h}}^T\mathbf{\Phi}\Bar{\mathbf{H}}\mathbf{f}  +  \kappa_l \kappa_n \Tilde{\mathbf{h}}^T\mathbf{\Phi}\Bar{\mathbf{H}}\mathbf{f}  +  \kappa_l \kappa_n \Bar{\mathbf{h}}^T\mathbf{\Phi}\Tilde{\mathbf{H}}\mathbf{f}  +  \kappa_n^2\Tilde{\mathbf{h}}^T\mathbf{\Phi}\Tilde{\mathbf{H}}\mathbf{f}$.
Here, the first term is deterministic, whereas the  second and third terms are complex Gaussian as they are linear combinations of elements of $\tilde{\mathbf{h}}$ and $\tilde{\mathbf{H}}$, respectively. 
However, the last term is the sum of products of two zero-mean complex Gaussian random variables as given by 
$$\Tilde{\mathbf{h}}^T\mathbf{\Phi}\Tilde{\mathbf{H}}\mathbf{f} = \sum\nolimits_{i = 1}^N \sum\nolimits_{j = 1}^M \Tilde{\mathbf{h}}_i \Tilde{\mathbf{H}}_{ij} \mathbf{\boldsymbol{\psi}}_i \mathbf{f}_j.$$
The exact distribution of the above form is challenging to derive  as the distribution of the product of two complex Gaussian random variables itself is in a complicated form \cite{Donoughue_GuassianProduct}, which naturally will lead to intractability in further analysis. However, as the above summation includes many terms (mainly when the number of RIS elements is large),  we can apply the central limit theorem to approximate its distribution  using a complex Gaussian. Thus, we can conclude that $\xi_2$ closely follows complex Gaussian distribution with mean $\mu_2$ and variance $\sigma^2_2$ as in \eqref{eq:xi1xi2}.
% % The second order moment of $\xi_1,~\text{and}~\xi_2$ is
% % \begin{subequations}
% %    \begin{align}
% %     &\mathbb{E}[|\xi_1|^2] = |\kappa_l^2 \Bar{\mathbf{h}}^T\mathbf{\Phi}\Bar{\mathbf{H}}^T\mathbf{f}|^2 + \eqref{Exp1} + \eqref{Exp2} + \eqref{Exp3} \label{S_O_Mom_e1},\\
% %     &\mathbb{E}[|\xi_2|^2] = |\kappa \Bar{\mathbf{g}}^T\mathbf{f}|^2 + \kappa_n^2 \mathbf{f}^H \mathbf{R}_{\rm BT} \mathbf{f}\label{S_O_Mom_e2}.
% % \end{align} 
% % \end{subequations}

% % Using \eqref{F_O_Mom}, \eqref{S_O_Mom_e1}, and \eqref{S_O_Mom_e2},
% % \begin{subequations}
% %     \begin{align}
% %       \mathbb{E}[\xi_1 + \mu \xi_2] &= m = \kappa_l^2 \Bar{\mathbf{h}} \mathbf{\Phi} \Bar{\mathbf{H}} \mathbf{f} + \mu \kappa_l \Bar{\mathbf{g}}^T\mathbf{f}, \\    Var[\xi_1 + \mu \xi_2] &= \sigma^2 =  \mathbb{E}[|\xi_1 + \mu \xi_2|^2] - |\mathbb{E}[\xi_1 + \mu \xi_2]|^2,\\
% %       &= \eqref{Exp1} + \eqref{Exp2} + \eqref{Exp3}.
% %     \end{align}
% % \end{subequations}

%  % \begin{align}
%  %     \mathbb{E}[X] =~~& \mathbb{E}[\mathbf{f}^H\mathbf{H}^H\mathbf{\Phi}^H\mathbf{h}^*\mathbf{h}^T\mathbf{\Phi}\mathbf{H}\mathbf{f}]+\mathbb{E}[\mathbf{f}^H\mathbf{g}^*\mathbf{g}^T\mathbf{f}]+\mathbb{E}[\mathbf{h}^T\mathbf{\Phi}\mathbf{H}\mathbf{f}\mathbf{f}^H\mathbf{g}^*]+\mathbb{E}[\mathbf{g}^T\mathbf{f}\mathbf{f}^H\mathbf{H}^H\mathbf{\Phi}^H\mathbf{h}^*]\nonumber\\
%  %      \mathbb{E}[X] =~~& \kappa_L^2\mathbf{\Bar{h}}^T\mathbf{\Phi Hf}+\mu\kappa_l\mathbf{\Bar{g}}^T\mathbf{f}\\
%  %      \mathbb{V}[X] =~~& \mathbb{E}[XX^H] - [\mathbb{E}[X]]^2\nonumber\\
%  %      \mathbb{V}[X] =~~& \kappa_l^2\kappa_n^2\boldsymbol{\psi}^H(\mathbf{R}_{\rm RT}\odot\mathbf{Z_1})\boldsymbol{\psi} + \kappa_n^2\mathbf{f}^H\mathbf{R}_{\rm BT}\mathbf{f}[\boldsymbol{\psi}^H\mathbf{Z}_2\boldsymbol{\psi}] + \mu^2\kappa_n^2\mathbf{f}^H\mathbf{R}_{\rm BT}\mathbf{f}
%  % \end{align}
 
%\ifCLASSOPTIONcaptionsoff\newpage\fi
%\begin{IEEEbiography}{Michael Shell}Biography text here.\end{IEEEbiography}\begin{IEEEbiographynophoto}{John Doe}Biography text here.\end{IEEEbiographynophoto}\begin{IEEEbiographynophoto}{Jane Doe}Biography text here.\end{IEEEbiographynophoto}
\vspace{-0.4cm}
\bibliographystyle{IEEEtran}
\bibliography{Reference}

\end{document}


