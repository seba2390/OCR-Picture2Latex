\section{Related Work}

% \hide{
% We briefly survey the existing work in the field of serverless computing, considering both historical and contemporary work in programming models, functions-as-a-service, state and coordination, workflow languages, and actor languages.

% \kk{In the comments of this section there is a lot of even loosely related work. When we get closer to the deadline we can filter out what seems unnecessary.}
% }
%
\paragraph{Workflows.} 
Many systems have acknowledged the challenges of providing reliable execution guarantees for long-running workflows. Most follow the declarative approach: Netflix's Conductor~\cite{conductor}, Zeebe~\cite{zeebe}, and AWS Step Functions~\cite{aws-step-functions} use a JSON schema for authoring workflows, and Fission Workflows~\cite{fission} use YAML.  Apache Airflow~\cite{airflow} and its productization, Google Cloud Composer~\cite{cloudcomposer}, and Fn Flow~\cite{flow}, are somewhat more code-based, as the schema is constructed in code. However, unlike DF, they do not simply adopt the standard control flow semantics of the host language. Also, none of these systems offer state and synchronization features comparable to entities or critical sections.

\paragraph{Actors.}  Entities in DF, and the instances in the computation model, are inspired by traditional actor systems like Erlang~\cite{armstrong1997development} or Akka~\cite{haller2012integration}, and especially the virtual actors of Orleans~\cite{bykov2011orleans}. The latter support persistence, but may lose actor messages, guaranteeing only "at-most-once" delivery. Similarly, the execution guarantees for Cloudflare's Durable Objects\footnote{Durable Objects closed beta was announced on September 28th, 2020.}~\cite{introduction-to-durable-objects, using-durable-objects} and Lightbend's Akka Serverless\footnote{Akka Serverless was formerly known as Lightbend CloudState.}~\cite{cloudstate}, apply only to a single object; they do not provide causal consistency guarantees or synchronization primitives that span multiple objects, like DF orchestrations.

\paragraph{Engine.} The \sys architecture is inspired by Ambrosia's virtual resiliency \cite{ambrosia}, with the partitions corresponding to immortals. However, instead of a single queue, \sys uses a separate commit log and input queue, which reside in different external services. Also, \sys keeps cold state in storage by virtue of Faster \cite{chandramouli2018faster}, and supports speculation. 

\paragraph{Serverless Computation Model.} The need to augment FaaS with support for state and synchronization has been acknowledged by both the research and industrial communities~\cite{DBLP:conf/cidr/HellersteinFGSS19,DBLP:journals/corr/abs-1902-03383,DBLP:conf/cidr/Schleier-Smith19,8481652}. Jangda et al. \cite{jangda-et-al} present a formal model for FaaS and explain its limitations. They also show how to compose functions using a language called \verb+SPL+. However, unlike our work, they do not combine state, messages, and functions into a single serverless model with reliable and causally consistent execution guarantees. 

Cloudburst~\cite{sreekanti2020cloudburst}, on the other hand, has a similarly unified computation model, and can also, like Netherite, execute efficiently and guarantee causal consistency. However, its programming model does not allow for arbitrary dynamic task-parallel code, like DF, but supports execution of statically specified DAGs only. Similarly, recent work \cite{fault-tolerance-shim,transactional-causal-consistency} investigates how to guarantee causal consistency for serverless applications, but for a workload of transactions over replicated data, not message-passing workflows. The difference is that in our model, only each message-processing step, not the entire workflow, executes transactionally. 

% \paragraph{Serverless Applications. }
%
%\kk{Stateful Serverless Functions on top of stream processing system}
% Stateful functions as a service in action~\cite{akhter2019stateful}.
%
Kappa~\cite{zhang2020kappa} proposes a programming framework for serverless that addresses two issues with serverless functions: the execution time limit and the lack of coordination between lambdas. Kappa is very similar to DF in that it also offers a high-level language programming environment. The main difference with DF is that it doesn't support some advanced features such as error handling and critical sections.
% \kk{It is slightly difficult to understand exactly what their model can and can't do. Also I can't find a citation for this paper since it is published in SoCC this year.}
%
% \kk{Occupy the Cloud: Distributed Computing for the 99}
PyWren~\cite{jonas2017occupy}, mu~\cite{fouladi2017encoding}, and gg~\cite{fouladi2019laptop} all propose simple programming frameworks for developing parallel serverless applications by exploiting the scalability of serverless functions. In contrast, DF is a complete programming solution that supports advanced features (arbitrary composition, critical sections) and also offers strong reliability guarantees.
% PyWren~\cite{jonas2017occupy} proposes a simple programming model for serverless that can be used for distributed data-parallel computations. Compared to DF, their programming model is very simple, only allowing for map reduce computations, and also it only offers an at-least-once guarantee.
% % \kk{From Laptop to Lambda:  Outsourcing Everyday Jobs to Thousands  of Transient Functional Containers}
% gg~\cite{fouladi2019laptop} proposes a serverless solution that allows users to describe dataflows of tasks that are then executed as stateless functions. In contrast to DF, it only supports pure computations that do not communicate with external services. 
% \kk{Encoding, Fast and Slow: Low-Latency Video Processing Using Thousands of Tiny Threads}
% mu~\cite{fouladi2017encoding} proposes a programming framework for serverless applications that allows for composing tasks using state machines. In contrast to DF, 

\hide{
\kk{A fault-tolerance shim for serverless computing.} \cite{sreekanti2020fault}
}

% \kk{It also seems that their system offers weaker reliability guarantees that DF does, but I haven't found a precise example that shows this difference.}


% \paragraph{Exactly and ``Effectively-Once''.}  

% ...

% Several systems such as Apache Pulsar and Kafka acknowledge the difficulty of exactly-once processing an ofter an alternative called ``effectively-once'' processing where producers retransmit messages to consumers until the messages are successfully acknowledged after message deduplication and processing. 

% %$ Apache Kafka and Pular achieve this by storing unique message identifiers alongside the effect of processing that message. 

% \csm{Kafka exactly once is basically effectively once}
% % https://www.confluent.io/blog/exactly-once-semantics-are-possible-heres-how-apache-kafka-does-it/}

% \csm{Effectively once as exactly once in apache pulsar}
% % https://www.splunk.com/en_us/blog/it/effectively-once-semantics-in-apache-pulsar.html}

% \csm{akka exactly once}
% % https://www.lightbend.com/blog/how-akka-works-exactly-once-message-delivery

\hide{
\seb{I moved this under actors.}
\paragraph{Stateful Serverless.} 
Cloudflare's Durable Objects~\cite{introduction-to-durable-objects, using-durable-objects}, like DF entities or Orleans~\cite{bykov2011orleans} virtual actors, are persistent objects. 
\hide{through the use of class definitions in one of their supported host languages (e.g., JavaScript, etc.), similar to DF's Entities.  Each instance of a Durable Objects has a globally unique identifier and is shutdown when idle and recreated on-demand.  However, one notable difference when compared to Entities in DF is that storage accesses require explicitly reading and writing to storage using an API provided by CloudFlare, instead of using instance variables defined within the class to manage state.  These storage accesses observe serializable isolation and may require either abort or rollback under concurrent execution.  }
However, execution guarantees are local only to a single object, therefore they do not support the causal consistency guarantees, nor the synchronization primitives, provided by DF orchestrations.
}

\hide{
\csm{seb: you wanna move this to actors?}\seb{probably. I may also shorten it.. I am glad you figure this all out though}
\csm{I had to watch like 2 50 minute talks lol to barely understand what the hell it was doing.}
\seb{yep, I tried to read some webpages. Too much talking, not enough clarity.}
\csm{Maybe just add a comma after Durable Objects in the actor section and say that both of them don't do synch prim. and consistency guarantees}
\seb{Agreed. }
Lightbend's Akka Serverless\footnote{Akka Serverless was formerly known as Lightbend CloudState.}~\cite{cloudstate} is an open-source on-prem solution for stateful serverless.  Akka Serverless relies on two components: \textit{a.)} a client library for the language that the serverless function will be written in that describes through a programatic API how precisely data should be persisted to cloud storage; and \textit{b.)} a sidecar proxy that runs alongside the serverless function implementation for state management and smart routing of requests.  Similar to Entities in DF and grains in Orleans~\cite{bykov2011orleans}, the sidecar proxy activates (and reclaims) actor instances -- one per serverless function instance for a given unique key -- where state is loaded and kept in memory until reclaimed.  Also similar, Akka Serverless supports an event-sourcing based infrastructure for state management.  However, different from both Entities and Orchestrations in DF, Akka Serverless provides no mechanisms for coordination nor enforcement of causal consistency under failures.
}



\hide
{
%% Programming model.

\seb{We may skip these as they are not very closely related}
\paragraph{Programming Model.}  From the programming model point of view, the Common Gateway Interface (CGI), popular in the mid-to-late 90s, was the first programming model to easily allow the deployment and scaling of web applications, independent of implementation language, where the user only needed to think about their application code.  Widely recognized as the first serverless offering from a cloud provider, Google App Engine provided simplified deployment, operations, and scalability for web applications where the users only had to upload their application code~\cite{google-app-engine}.  Google App Engine had a number of popular spiritual successors, Heroku~\cite{heroku}, Parse~\cite{parse}, and most recently, Firebase~\cite{firebase}.

%% Elastic scalability and pay-as-you-go billing.

\seb{I have added all the citations from the section below to the intro}
\paragraph{Functions-as-a-Service.} Amazon Web Services was the first to provide the modern serverless computing offering, thereby introducing the terminology \textit{Functions-as-a-Service} with the introduction of AWS Lambda~\cite{aws-lambda}.  AWS Lambda was the first service to offer both elastic scalability and pay-as-you-go billing: elastic scalability, providing the ability to scale up or down based on user demand; and pay-as-you-go billing, where billing is performed on a compute time per invocation basis instead of a fixed instance price.  Since then, the other major cloud providers have followed suit: Google subsequently introduced Google Cloud Functions~\cite{google-cloud-functions} for Google Compute Platform; Microsoft subsequently introduced Azure Functions~\cite{azure-functions} for Microsoft Azure.  Several on-prem serverless solutions also exist: Apache OpenWhisk~\cite{openwhisk} is a serverless platform that runs on-prem; and Fission~\cite{fission} is a on-prem serverless solution that is designed specifically for operation with Kubernetes. 

%% State and coordination mechanisms.

\paragraph{State and Coordination.} The need for both stateful functions and coordination primitives has been acknowledged by both the research and industrial communities.~\cite{DBLP:conf/cidr/HellersteinFGSS19,DBLP:journals/corr/abs-1902-03383,DBLP:conf/cidr/Schleier-Smith19,8481652} Several research systems have worked around these deficiencies by using external cloud storage for coordination~\cite{201559, 216767}; however, external cloud storage is both expensive and prohibitively slow.  Recently, Scala's Akka programming framework has introduced a stateful offering for their actors backed by the popular Kubernetes cloud orchestration framework~\cite{akka-stateful-serverless}; while this solves the problem of state management, it still provides no means for coordination between actors.

%% Workflow languages.

\seb{The following section was copied to the beginning and condensed.}
Many systems have acknowledged the challenges around running long-running workflows and reliable execution guarantees.  Apache Airflow~\cite{airflow} provides a Python-based workflow language; it was subsequently productized for cloud-based workflows in Google Cloud Platform as Google Cloud Composer~\cite{cloudcomposer}.  Netflix's Conductor~\cite{conductor} and Zeebe~\cite{zeebe} is another workflow engine developed for cloud services using a language represented as JSON.  In the microservices world, Fission Workflows~\cite{fission} is a Kubernetes-specific solution for running reliable workflows, represented as YAML.  Amazon's Step Functions~\cite{aws-step-functions} attempts to ease the development of long-running workflows composed of any number of Lambda functions, specified using a state machine language embedded in JSON.  Oracle's Fn Flow~\cite{flow} lets developers build long running serverless orchestrations using Java, providing both type safety and the ability to use traditional exception handling for errors.

% %% Actor languages.

\seb{The following section was copied to the beginning and condensed.}
\paragraph{Actor Languages.}  Entities in Durable Functions are very similar to, and inspired by, traditional actor systems (e.g., Erlang~\cite{armstrong1997development}, Akka~\cite{haller2012integration}) where actors control access to their state through message passing and are non-reentrant.  The first system to propose actors backed by cloud storage, was Microsoft's Orleans~\cite{bykov2011orleans}, where actors locally cached their state and periodically persisted checkpoints to durable, cloud storage.  Since this durable cloud storage is available at all of the nodes, actors can be started and stopped on demand in order to respond to user demand.  However, unlike the guarantees provided by orchestrations involving multiple entities, none of these systems provide deadlock freedom nor primitives for multi-object synchronization. 

% %% Serverless on top of streaming systems

% \paragraph{Serverless on top of streaming systems.} 
% \todo[inline]{
% Flink's Stateful functions.
% % https://flink.apache.org/stateful-functions.html 
% Stateful functions as a service in action~\cite{akhter2019stateful}.
% }


% %% Serverless Systems

% \paragraph{Serverless platforms/systems.} 
% There has been a surge of work on serverless execution platforms. These systems (even though they have similarities with our work), are mostly complementary since they can be used to speed up function execution, and do not offer an complete solution for reliable workflow execution. SOCK~\cite{oakes2018sock} is a container system that is optimized for serverless workloads. It is complentary to our work, as it can be used to improve performance of Azure functions 
% SAND~\cite{akkus2018sand} is a serverless platform that offers performance benefits for serverless applications by allocating functions in the same application to the same container, and by introducing a hierarchical communication architecture to reduce communication overhead between functions in the same application. These optimizations are similar to our partitions handling many instances and the internal messaging system. Similarly to the other platforms, SAND can act as a complement to our system.

% \kk{We shouldn't include this is the first submission of the paper since the paper is not available yet.}
% Serverless workflows~\cite{zhang2020workflows}. 
% - They have a similar log-based persistence approach. One of the differences of our works is that we also propose a programming model. 
% - In their work state is external to the execution engine and they just wrap accesses to it. 
% - Their reliability guarantee also seems to be stronger than ours since they don't have speculation. We can phrase the last point in reverse by saying that we get good performance by slightly relaxing the guarantee to allow speculation. 
% - Our programming model allows users to have access to state in workflows without accessing external services, but rather implicitly by defining variables in the workflow.
% - (Maybe) In addition to the above, in their work they look inside tasks (lambda functions), while we see them as black boxes, meaning that our method can be straightforwardly applied to any different language, and also we do not have to alter the execution engine of serverless functions.
% \dajusto{Include Arjun Guha's work on formal foundations for serverless systems. It includes a section on expanding a minimal serverless PL with state and orchestration-like concepts.}

% %% TODO: Service Fabric
% %% TODO: Orleans, Ambrosia
% %% TODO: Add missing citations.

}