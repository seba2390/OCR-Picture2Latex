\documentclass[conference]{IEEEtran}
\vspace{-0.1in}
\section{Neural Program Synthesis from Input-Output Examples}
\vspace{-0.1in}
In programming by example tasks, the program specification is a set of input-output examples~\cite{devlin2017robustfill,bunel2018leveraging}. Specifically, we provide the synthesizer with a set of $K$ input-output pairs $\{(I^{(k)}, O^{(k)})\}_{k=1}^K$ ($\{IO\}^K$ in short). These input-output pairs are annotated with a ground truth program $P^\star$, so that $P^\star(I^{(k)})=O^{(k)}$ for any $k \in \{1, 2, ..., K\}$. To measure the program correctness, we include another set of held-out test cases $\{IO\}_{test}^{K_{test}}$ that differs from $\{IO\}^K$. The goal of the program synthesizer is to predict a program $P$ from $\{IO\}^K$, so that $P(I)=P^\star(I)=O$ for any $(I, O) \in \{IO\}^K + \{IO\}_{test}^{K_{test}}$.

%\label{sec:c-data}
\textbf{C Program Synthesis}. In this work, we make the first attempt of synthesizing C code in a restricted domain from input-output examples only, and we focus on programs for list processing. List processing tasks have been studied in some prior works on input-output program synthesis, but they synthesize programs in restricted domain-specific languages instead of full-fledged popular programming languages~\cite{balog2016deepcoder,odena2020learning,odena2020bustle}. 

Our C code synthesis problem brings new challenges for programming by example. Compared to domain-specific languages, the syntax and semantics of C are much more complicated, which significantly enlarges the program search space. Meanwhile, learning good representations for partially decoded programs also becomes more difficult. In particular, prior neural program synthesizers that utilize per-line interpreters for the programming language to guide the synthesis and representation learning~\cite{chen2018execution,shin2018improving,nye2020representing,Ellis2019WriteEAExtendExecution,odena2020bustle} are not directly applicable to C. Although it is possible to dump some intermediate variable states during C code execution~\cite{campbell2012executable}, since partial C programs are not executable, we are able to obtain all the execution states only until a full C code is generated, which is too late to include them in the program decoding process. In particular, the intermediate execution state is not available when the partial program is syntactically invalid, and this happens more frequently for C due to its syntax design.
\begin{figure}
    \centering
    \includegraphics[width=\textwidth]{fig/c-program-synthesis-crop.pdf}
\caption{\small Illustration of the C program synthesis pipeline. For dataset construction, we develop a random program generator to sample random C programs, then execute the program over randomly generated inputs and obtain the outputs. The input-output pairs are fed into the neural program synthesizer to predict the programs. Note that the synthesized program can be more concise than the original random program.}
\label{fig:ex-c}
\end{figure}


\IEEEoverridecommandlockouts

\usepackage{url}
\usepackage{tabularx}
\usepackage{booktabs}
\usepackage{cite}
\usepackage{amsmath,amssymb,amsfonts}
\usepackage{algorithmic}
\usepackage{enumitem}
\usepackage{graphicx}
\usepackage{textcomp}
\usepackage[table]{xcolor}
\usepackage[table]{xcolor}
\usepackage{array}
\usepackage{balance}
\usepackage{csvsimple} % Required for CSV import
\usepackage{boldline} % Required for bold lines in the table
\usepackage{multirow} % Required for merging cells vertically
\usepackage{booktabs} % used to create professional-looking table lines
% Chat Bubbles
\usepackage{tikz}
\usepackage{etoolbox}
\newcolumntype{L}[1]{>{\raggedright\let\newline\\\arraybackslash}p{#1}} 
\usepackage{roboto}


\definecolor{lightpurple}{RGB}{21, 11, 163}
\definecolor{lightgrey}{RGB}{211, 211, 211}

\tikzstyle{userbubble} = [rectangle, draw, fill=lightgrey!20, text width=.75\columnwidth, font=\roboto\scriptsize, minimum height=1cm, align=left, rounded corners]
\tikzstyle{assistantbubble} = [rectangle, draw, fill=lightpurple!20, text width=.75\columnwidth, font=\roboto\scriptsize, minimum height=1cm, align=left, rounded corners]
                           
\newenvironment{chatbubble}[2]
{
    \ifstrequal{#1}{User}{
        \begin{flushright}
            \begin{tikzpicture}
                \node[userbubble, anchor=east] at (0,0) {
                    \textbf{Human}: #2
                };
            \end{tikzpicture}
        \end{flushright}
  } 
  
  \ifstrequal{#1}{Assistant} {
    \begin{flushleft}
        \begin{tikzpicture}
            \node[assistantbubble, anchor=west] at (0,0){\textbf{Assistant}: #2};
        \end{tikzpicture}
    \end{flushleft}
  } 
  
  
}
{}

\newcommand{\human}[1]{
    \chatbubble{User}{#1}
}
\newcommand{\assistant}[1]{
    \chatbubble{Assistant}{#1}
}
\newcommand{\brackets}[1]{
    \textless#1\textgreater
}
%
% Custom Commands
%
\def\BibTeX{{\rm B\kern-.05em{\sc i\kern-.025em b}\kern-.08em
    T\kern-.1667em\lower.7ex\hbox{E}\kern-.125emX}}
\def\UrlBreaks{\do\/\do-} % Allow breaking at / and -

\begin{document}



\title{Prompts Matter: Insights and Strategies for Prompt Engineering in Automated Software Traceability
  %\thanks{Identify applicable funding agency here. If none, delete this.}
}

\author{
  \IEEEauthorblockN{Alberto D. Rodriguez}
  \IEEEauthorblockA{\textit{College of Engineering} \\
    \textit{University of Notre Dame}\\
    Notre Dame, Indiana \\
    arodri39@nd.edu}
  \and
  \IEEEauthorblockN{Katherine R. Dearstyne}
  \IEEEauthorblockA{\textit{College of Engineering} \\
    \textit{University of Notre Dame}\\
    Notre Dame, Indiana \\
    kdearsty@nd.edu}
  \and
  \IEEEauthorblockN{Jane Cleland-Huang}
  \IEEEauthorblockA{\textit{College of Engineering} \\
    \textit{University of Notre Dame}\\
    Notre Dame, Indiana \\
    JaneClelandHuang@nd.edu}
}
\maketitle

\begin{abstract}
Large Language Models (LLMs) have the potential to revolutionize automated traceability by overcoming the challenges faced by previous methods and introducing new possibilities. However, the optimal utilization of LLMs for automated traceability remains unclear. This paper explores the process of prompt engineering to extract link predictions from an LLM. We provide detailed insights into our approach for constructing effective prompts, offering our lessons learned. Additionally, we propose multiple strategies for leveraging LLMs to generate traceability links, improving upon previous zero-shot methods on the ranking of candidate links after prompt refinement. The primary objective of this paper is to inspire and assist future researchers and engineers by highlighting the process of constructing traceability prompts to effectively harness LLMs for advancing automatic traceability.
\end{abstract}

\begin{IEEEkeywords}
  automated software traceability, large language models, prompt engineering
\end{IEEEkeywords}

\section{Introduction}
\label{sec:intro}
Reinforcement learning has achieved great success in areas such as Game-playing \citep{silver2018general,vinyals2019grandmaster}, robotics \cite{kober2013reinforcement}, large language models \citep{ouyang2022training}, etc.
However, due to safety concerns or physical limitations, in some real-world reinforcement learning problems, we must consider additional constraints that may influence the optimal policy and the learning process \citep{garcia2015comprehensive}.
% For example, a robotic arm must not take actions that may cause harm to itself or the environments.
A standard framework to handle such cases is the constrained Markov Decision Process (CMDP) \citep{altman1999constrained}.
Within the CMDP framework, the agent has to maximize
the expected cumulative reward while
obeying a finite number of constraints, which are usually in the form of expected cumulative cost criteria.

However, we are sometimes concerned with the problem with a continuum of constraints.
For example,
the constraints we meet might be time-evolving or subject to uncertain parameters, which
cannot be formulated as an ordinary CMDP
(see Examples \ref{Example_Time_Evolving} and  \ref{Example_Uncertain}).
In this paper we would study a generalized CMDP  
to address the above problem.  Because the constraints are not only infinite-number but also lie
in a continuous set,
the generalization is not trivial. Fortunately, we find that we can borrow the idea behind semi-infinite programming (SIP) \citep{remez1934determination, hettich1993semi} to deal with the semi-infinite constraints.
Accordingly, we propose \emph{semi-infinitely constrained Markov decision processes} (SICMDPs)
as a novel complement to the ordinary CMDP framework.
%More specifically,  an SICMDP model %, we consider 
%contains a continuum of constraints whereas an ordinary CMDP contains a finite number of constraints. 

%This generalization is natural but not trivial. However, we can brows the idea  
%The idea is quite natural and can be backtracked
%to the practice of extending linear programming to linear semi-infinite programming (LSIP) %\cite{remez1934determination, GobernaLSIO1998}.
%In addition, 
%As a complementary approach to the ordinary CMDP framework, 
%SICMDP can be used to model these problems  which cannot be described by a finite number of constraints
%that are not covered by .
%For example,
%the restrictions we consider can be time-evolving or subject to uncertain parameters
%, thus
%cannot be described by a finite number of constraints but a continuum of constraints 
%(see Examples \ref{Example_Time_Evolving} and  \ref{Example_Uncertain}).

We also present two reinforcement learning algorithms to solve SICMDPs called SI-CRL and SI-CPO, respectively.
SI-CRL is a model-based reinforcement learning algorithm designed for tabular cases, and SI-CPO is a policy optimization algorithm for non-tabular cases.
% and analyze its performance both theoretically and empirically.
The main challenge is that we need to deal with a continuum of constraints, thus reinforcement learning algorithms for ordinary CMDPs do not work anymore.
In SI-CRL, we tackle this difficulty by first transforming the reinforcement learning problem to an equivalent LSIP problem, which can then be solved using methods in the LSIP literature like the dual exchange methods \citep{Hu1990,reemtsen1998numerical}.
In SI-CPO, we resort to the idea of cooperative stochastic approximation developed in \cite{lan2020algorithms, wei2020comirror}.
As far as we know, we are the first to introduce tools from semi-infinitely programming (SIP) into the reinforcement learning community for solving constrained reinforcement learning problems.

% To the best of our knowledge, we are the first to apply tools from semi-infinitely programming (SIP) to solve reinforcement learning problems.
Furthermore, we give theoretical analysis for both SI-CRL and SI-CPO.
We decompose the error of SI-CRL into two parts: the statistical error from approximating the true SICMDP with an offline dataset and the optimization error due to the fact that the solution of the LSIP problem obtained by the dual exchange method is inexact.
On the optimization side, we show that the iteration complexity of SI-CRL is $O\left(\left\{\mathrm{diam}(Y)L\sqrt{|\gS|^2|\gA|m}/\left[(1-\gamma)\epsilon\right]\right\}^m\right)$.
On the statistical side, we show that the sample complexity of SI-CRL is $\widetilde O\left(\frac{|S|^2|A|^2}{\epsilon^2(1-\gamma)^3}\right)$ if the offline dataset is generated by a generative model, and $\widetilde O\left(\frac{|S||A|}{\nu_{\min} \epsilon^2(1-\gamma)^3}\right)$ if the dataset is generated by a probability measure $\nu$ as considered in \cite{chen2019information}.
Here $\widetilde O$ means that all logarithm terms are discarded.
For SI-CPO, things become a little more complicated because other than the statistical error and the optimization error, we also need to consider the function approximation error, which comes from imperfect policy parametrizations.
It is shown if the function approximation error can be controlled to $O(\epsilon)$ order, the iteration complexity of SI-CPO is $\widetilde{O}\left(\frac{1}{\epsilon^2(1-\gamma)^6}\right)$ and the sample complexity of SI-CPO is $\widetilde{O}(\frac{1}{\epsilon^4(1-\gamma)^{10}})$.
Here our iteration complexity bound is equivalent to a typical $\widetilde O(1/\sqrt{T})$ global convergence rate.

We perform a set of numerical experiments to illustrate the SICMDP model and validate our proposed algorithms.
Specifically, we examine two numerical examples, namely the discharge of sewage and ship route planning.
Through the discharge of sewage example, we show the advantage of the SICMDP framework over the CMDP baseline obtained by naive discretization in modeling realistic sequential decision-making problems.
Moreover, we demonstrate the effectiveness of the SI-CRL and SI-CPO algorithms in such tabular environments. 
In the ship route planning example, we illustrate the benefits of the SICMDP framework and the ability of the SI-CPO algorithm to address complex continuous control tasks involving continuous state spaces with modern deep reinforcement learning techniques.

% In summary, our contributions are listed as follows.
% First, we present the SICMDP model, which can be viewed as a generalization of the ordinary CMDP model.
% Second, we propose an algorithm to perform reinforcement learning for SICMDPs, which is called SI-CRL, and we believe that we are the first to apply tools from SIP
% to solve reinforcement learning problems.
% Third, we give a theoretical analysis of SI-CRL and identify both its sample complexity and iteration complexity.
% In addition, we perform numerical experiments to illustrate the SICMDP model and validate the SI-CRL algorithm.
% \{This paragraph can be removed!!! \}






\begin{table*}[!t]
  \label{tab:datasets}
  \centering
  \caption{Datasets}
\begin{tabular}{p{2cm}p{5cm}p{5cm}cp{1.25cm}c p{1.25cm}cp{1.25cm}}


    \toprule
    \textbf{Project Name} & \textbf{Description} & \textbf{Artifacts} & \textbf{Children} & \textbf{Candidates} & \textbf{True} \\
    \midrule \midrule
    % ---
    \multirow{2}{*}{\textbf{CM1 \cite{cm1}}} & The requirements for an instrument a part of NASA's Metric Data Program (MDP). & High-Level Requirements $\rightarrow$ Low-Level Requirements & \multirow{2}{*}{53} & \multirow{2}{*}{265} & \multirow{2}{*}{13} \\ \midrule
    % ---
    \multirow{3}{*}{\textbf{iTrust \cite{Meneely2011_itrust}}} & Open-source electronic health record system. Created at North Carolina State University as a part of a software engineering course. & \multirow{3}{*}{Requirements $\rightarrow$ Java Classes} & \multirow{3}{*}{227} & \multirow{3}{*}{1135} & \multirow{3}{*}{13} \\ \midrule
    % ---
    \multirow{3}{*}{\textbf{Dronology \cite{clelang_huang_dronology}}} & \multirow{3}{6cm}{A system for managing the navigation of \\ UAVs and their communication to the ground control station.} & \textbf{NL}: Requirements $\rightarrow$ Design Definitions & 99 & 495 & 4 \\
    & & & & & \\ 
    & & \textbf{PL}: Design Definitions $\rightarrow$ Java Classes & 458 & 2290 & 48 \\ \bottomrule
  \end{tabular}
  {\newline \newline \raggedright Describes the artifact types in each dataset, the number of children per query, the resulting candidate links across all queries, and how many of those candidates were true links. Dronology is split into two datasets, DronologyNL and DronologyPL, to focus on traces between natural language artifacts and between natural language and programming language artifacts respectively.}
   \vspace{-12pt}
\end{table*}


\section{Related Work}
\label{sec:related}
\textbf{Related work}:
% Object detection related datasets/algo in non-medical domain
% Locally labeled CXR dataset
A few CXR datasets have localized abnormality annotations \cite{shih2019augmenting,filice2020crowdsourcing,jaeger2014two} that are curated manually. These are high quality gold standard ground truth datasets but tend to be smaller in scale (< 30,000 images) and have a narrow coverage, with typically only 1-2 labels. In addition, since most labeling efforts only have abnormality semantics attached, no direct relationships with the affected anatomical locations are available. 

%MEHDI: repeated concepts from above. I am removing the following: 

%The lack of anatomic semantics in the annotation is a limitation for complex multi-modal clinical reasoning work, e.g., differential diagnosis, since clinicians often integrate information along anatomical lines, and for downstream report generation tasks, which often requires describing not only the abnormality but also correctly communicate the location of the abnormalities (and medical devices) to the receiving clinicians. 

Two recent CXR datasets have labels for anatomies described in the reports. In \cite{datta2020dataset}, a small manually annotated dataset (2000 reports) included 10 abnormalities that are individually associated with 29 unique spatial locations (anatomies) at the report level. Another CXR dataset has automatically extracted abnormality and anatomy labels as disconnected concepts that are only correlated at the study level from  160,000 reports using a supervised NLP algorithm \cite{bustos2020padchest}. This was trained on a smaller set of manually annotated data. Neither datasets contain localized annotations for the associated CXR images, nor any comparison relation annotations between sequential exams, both of which are available in the Chest ImaGenome dataset. In Table \ref{tab:related}, we present a comparison of our Chest ImagGenome dataset with other datasets available in the literature.

% Table -- Kashyap

% MEdical imaging datasets to go here: Discussed that we will only focus on cxr datasets that are available for this paper. 
% \caption{\color{red} Kashyap, feel free to continue with the table. We should remove the questionmarks and add a line for our dataset (since all others are not graph). For longer text, using abbreviations and explaining them in the caption often works better. If fill in the values is not possible, it is better to remove the table altogether.}


\begin{table}[t!]
\caption{Summary of existing chest X-ray datasets}
\resizebox{\textwidth}{!}{%
\begin{tabular}{@{}lllllllll@{}}
\toprule
\textbf{Dataset} & \textbf{Annotation Level} & \textbf{Annotation Method} & \textbf{Num Labels} & \textbf{Anatomy Labeled} & \textbf{Graph} & \textbf{Dataset Size} & \textbf{Temporal Labels} & \textbf{Reports} \\ \midrule
SIIM-ACR Pneumothorax Segmentation \cite{filice2020crowdsourcing} & Segmentation & Manual + augmented & 1 & No & No & 12,047 & No & No \\
RSNA Pneumonia Detection Challenge   \cite{shih2019augmenting} & Bounding Boxes & Manual & 1 & No & No & 30,000 & No & No \\
Indiana University Chest X-ray collection \cite{demner2016preparing} & Global & Automated & 10 & No & No & 3,813 & No & Yes \\
NIH CXR dataset \cite{wang2017chestx} & Global & Automated & 14 & No & No & 112,120 & No & No \\
PLCO \cite{team2000prostate} & Global & Automated & 24 & Yes & No & 236,000 & Yes & No \\
Stanford CheXpert \cite{irvin2019chexpert} & Global & Automated & 14 & No & No & 224,316 & No & No \\
MIMIC-CXR \cite{johnson2019mimic} & Global & Automated & 14 & No & No & 377,110 & No & Yes \\
Dutta \cite{datta2020dataset} & Global & Manual & 10 & Yes & Yes & 2,000 & No & Yes \\
PadChest \cite{bustos2020padchest} & Global & Manual + automated & 297 & Yes & No & 160,868 & No & Yes \\
Montgomery County Chest X-ray   \cite{jaeger2014two} & Segmentation & Manual & 1 & Yes & No & 138 & No & No \\
Shenzen Hospital Chest X-ray   \cite{jaeger2014two} & Segmentation & Manual & 1 & Yes & No & 662 & No & No \\  \hline \hline
\textbf{Chest ImaGenome} & Bounding Boxes & Automated & 131 & Yes & Yes & 242,072 & Yes & Yes \\
\bottomrule
\end{tabular}%
}
\label{tab:related}
\vspace{-0.4cm}
\end{table}
% removed (Derived from MIMIC-CXR \cite{johnson2019mimic}) % makes table really small


\section{Experimental Setup}
\label{sec:design}
\section{Scalable Representations for Communication Patterns}
\label{sec:design}

Using the lessons learned in our preliminary studies, along with existing case studies~\cite{isaacs2014combing, Isaacs2016} using idealized unit time, we design a set of strategies for representing communication patterns when there are too many PEs to draw distinct communication lines in Gantt charts. We first describe our design goals. Then, we present our designs. Finally, we discuss initial feedback from experts familiar trace analysis in HPC.


\subsection{Design Goals}

Our goal is to design a representation of communication in execution traces that (1) aids users in recognizing and understanding what communication is occurring in that temporal and logical position in the Gantt chart and (2) is agnostic to the number of processing elements, thereby scaling to larger traces. These goals are derived from usage and scalability limitations noted in prior work~\cite{isaacs2014combing}. 

We limit our focus to scaling in PEs (y-axis) rather than time. Traces are typically explored using a time window, so we focus on that case. Adapting a design or creating a new one for compressed time settings we leave for future work.

Based on our preliminary study (\autoref{sec:prelim}), we chose to focus on offset, ring, and exchange pattern types as stencils require more design consideration even at small scales. 

\subsection{Visualization Design}

Our design process began with open brainstorming on paper, which we include in the supplemental material. We tried a variety of strategies, including linked views and added channels to the traditional Gantt chart encoding rules. However, most of these retained scaling problems, leading us to focus on designs centering on glyphs.

In designing the representation, we considered the saliency of what was to be encoded (e.g., temporal range, pattern type, grouping, stride) and efficacy of available channels, taking into account that the design needs to be incorporated in a Gantt chart. For example, temporal range is set to a horizontal position matching where a pattern would be drawn in a full chart. See supplemental materials for a table containing discussion of channel considerations.

We prioritize the type of pattern before the grouping factor or stride. The rationale is that the pattern type is fixed by the source code while the grouping and stride are often computed from the problem size and number of resources. Therefore, a user will recognize pattern type first before considering other factors. \autoref{fig:abstract_designs} shows the resulting designs.


\begin{figure*}
    \centering
    \begin{subfigure}{0.18\textwidth}
         \centering
         \includegraphics[width=\textwidth]{figures/new-basic-offset.png}
         \caption{Continuous offset pattern}
         \label{fig:noc}
    \end{subfigure}
    \begin{subfigure}{0.18\textwidth}
         \centering
         \includegraphics[width=\textwidth]{figures/new-basic-offset-grouped.png}
         \caption{Grouped offset pattern}
         \label{fig:nog}
    \end{subfigure}
    \begin{subfigure}{0.18\textwidth}
         \centering
         \includegraphics[width=\textwidth]{figures/new-basic-ring.png}
         \caption{Continuous ring pattern}
         \label{fig:nrc}
    \end{subfigure}
    \begin{subfigure}{0.18\textwidth}
         \centering
         \includegraphics[width=\textwidth]{figures/new-basic-ring-grouped.png}
         \caption{Grouped ring pattern}
         \label{fig:nrg}
    \end{subfigure}
    \begin{subfigure}{0.18\textwidth}
         \centering
         \includegraphics[width=\textwidth]{figures/new-basic-exchange.png}
         \caption{Exchange pattern}
         \label{fig:neg}
    \end{subfigure}
    \caption{Examples of our designs for five communication patterns. They are reminiscent of the underlying communication pattern encoding, but not aligned to the underlying chart and agnostic to the number of rows the underlying pattern repeats over. 
    %Note that the angle for our ring pattern is slightly shallower than the angle for offset, this reflects the difference in stride between the two patterns. 
    Grouped representations fill the vertical space to indicate that the repetition continues from the top of row to the bottom.}
    \label{fig:abstract_designs}
\end{figure*}

\vspace{1ex}

\textbf{Encoding Pattern Type.} To encode the pattern type, we started with the overall shape of of the pattern when drawn at small scale with a small stride. Offsets are drawn with angled repeating lines forming a rhombus-like shape. We use a fixed distance between lines and draw as many will fit in the relevant area.

Rings add indicators of the ``wrap-around'' communication. However, unlike fully drawn rings, we only render the protruding segments at the ends of the shape. There are two main rationales for only drawing protruding segments: (1) we want to indicate this is an abstraction and (2) participants in our interviews found the crossing lines difficult to disambiguate. The number of protruding segments is proportional to the stride of a ring. 
%For a ring with a small stride, one segment is added. This increases to a max of four for a very large stride.

Exchange patterns are drawn as a series of symmetrical ``x'' shapes and avoid direct crossings for the same reason as rings. The number of lines in each cross is proportional to stride of the exchange. Short stride exchanges will exchange between only a few PEs, a long stride exchange spans many PEs. Our glyphs approximate this by increasing the number of crossing lines as stride increases.

\vspace{1ex}

\textbf{Encoding Grouping Factor.} To represent grouping, we partition the available area vertically and repeat the pattern type drawing in those partitions. More formally, the encoding rule to show ``grouping" is repetition and alignment on a non-common scale. The number of partitions is determined by the available vertical space in a chart.


\vspace{1ex}

\textbf{Encoding Stride.} We express the notion of stride through the angle of lines used in our pattern types. As people had difficulty with steeply angled lines in our preliminary study, we limit the angles to a range of 15 degrees to 60 degrees. Therefore, these do not match the encoding of a full view. Instead, they hint at the magnitude of distance over which communication is occurring. This allows users to see that there are differences in stride between glyphs, but not necessarily calculate the exact stride visually.

\vspace{1ex}

\textbf{Temporal range.} Rather than show the exact range, we place the glyphs on the x-axis so they are centered in their range. If two structures overlap, they are placed alongside one another. 

\vspace{1ex}

\textbf{Incorporation in Gantt Charts.} These are designed to be used in Gantt charts when exact lines would be too dense to be interpreted. The underlying interval rectangles will still be drawn. The color encoding of these intervals was shown to be a secondary indicator in our preliminary study, so we preserve them. We add a slight blur effect to the background as another signifier that the glyphs are an abstraction and should not be confused for exact lines.



\subsection{Expert Feedback}
\label{sec:expertfeedback}

We sent our designs to two HPC experts for feedback regarding both the designs themselves and the overall approach. Both experts were familiar with idealized unit time representations of traces. The first expert, E1, had previously collaborated on this strategy with the authors but was not involved in any of the work presented here. The second expert, E2, had managed an integration of the strategy into an HPC center's performance tools, referencing the open-source research code~\cite{isaacs2014combing} but using an alternate calculation method and front-end technology.

We sent both experts a short email with a PDF describing the visualizations with comparisons to fully drawn traces and showing how they might be applied in practice, including a few complicated examples such as zoomed-out time and idle processes. (See supplemental materials.) We asked if and how the strategy would be useful and if there were any suggestions or concerns. E1 responded the designs ``definitely look helpful,'' noted the trade off in exactness, and then pointed out figures which led to ambiguities in his view. He also identified a error where the mock-up did not match the underlying trace. E2 noted that stride is less important and wondered how the translation from data to glyph would be calculated. He suggested the strategy might also be helpful for collective communications (e.g., broadcasts, all-to-all, reductions), a set of patterns we did not consider in this work.

We interpreted these responses to suggest the designs were worth further study, particularly E1's ability to interpret well enough to detect an error and E2's interest in further patterns. However, there are design decisions in applying these glyphs in some scenarios, particularly in zoomed-out time, that require refinement. We leave these cases for future iterations and instead focus on how the base designs could be interpreted by a wider range of users in a controlled study.

% It is appearing twice.
% Removing this input
%\section{Metrics}\label{sec:metrics}
\dataset{} can be used to measure performance on three related tasks. % \textit{Factual change detection}:
%: in other words, whether or not a question should be generated. 
% \textit{Discriminating Question Generation} metrics measure how similar generated questions are to annotated questions. \textit{Full System} metrics measure a system's performance on the overall task. 

\paragraph{Factual Change Detection} \label{Metric:Diff Detection}  Given an example consisting of a base passage, target passage, and answer span, the goal is to determine whether there exists a valid differentiating question. In other words, whether there is new information about this answer span that is present in the target passage when compared to the base passage. To measure this, we report accuracy, precision, recall and F1 score over the existence of a differentiating question in our annotations. Note that always predicting no change achieves 60.3\% accuracy but 0\% F1, but random guessing corresponds to 44.1\% F1. 

\paragraph{Discriminating Question Generation} Given a target passage and answer span, write a specific, unambiguous and information-seeking query that can be answered  with the target passage. To measure this, we compare machine generated questions to those that humans verified, edited, or hand wrote. We use two model-free metrics Rouge-1 and Rouge-L \citep{lin2004rouge} which measure the token-level overlap and longest subsequence overlap of the questions, respectively. We also consider two model-based metrics, BLEURT \citep{sellam-etal-2020-learning}, which is a learned evaluation for text similarity based on BERT \citep{devlin-etal-2019-bert}, and a query similarity model \citep{reimers-2019-sentence-bert} trained on Quora Question Pairs \footnote{\href{https://huggingface.co/cross-encoder/quora-roberta-large}{huggingface.co/cross-encoder/quora-roberta-large}}.

Note that we evaluate discriminating question generation despite using a question generation model in our annotation procedure. Note that all of these questions are reviewed by humans and only the very fluent ones are kept. As question generation models vary in which of their productions are very fluent, this set is less trivial than it would initially appear. Nonetheless, we also separate human-written or edited questions and evaluate that set independently. 
% The query similarity metric is based on a T5-XXL model \citep{raffel2020exploring} trained on Quora Question Pairs\footnote{\href{https://www.kaggle.com/c/quora-question-pairs}{https://www.kaggle.com/c/quora-question-pairs}} to predict either ``duplicate'' or ``not duplicate'', where a model gets 1.0 if the question it produces is considered a ''duplicate'' and 0.0 otherwise. 
%Note that these metrics are only computed over the true positive examples, 
%\pj{This is slightly confusing in the Metrics section. There already seems to be some explanation in Results. Maybe we can omit it here.} 
%so question generation models can only be compared easily based on the same set of data; i.e., producing questions based on the same factual change detection model. 

\paragraph{Full System} This is the overall measure of performance on \dataset{}. We reuse the metrics from discriminating question generation, using 0.0 for BLEURT, ROUGE-1, ROUGE-L, and Query Similarity if the factual change detection is incorrect.  
%and is scored in a standard way otherwise. Note that certain systems produce a question or ``None'' in the same output space, while others compose a factual change detection method and question generation method.

\section{Analysis and Prompt Discovery}
\label{sec:results}
\begin{table}[t!]
\centering
\caption{Voice conversion \& F0 manipulation results. MOS results are reported with 95\% confidence interval. VDE, and FFE are reported for F0 manipulation while PER, WER, EER, and MOS are reported for voice conversion. Notice, for VDE, and FFE higher is the better since F0 was flattened.}
\label{tab:conv}

\resizebox{1\columnwidth}{!}{
\begin{tabular}{c@{~} | c@{~} | c@{~}c@{~} | c@{~} | c@{~} ||  c@{~}c@{~} }
\toprule
\multirow{2}{*}{Dataset} & \multirow{2}{*}{Method} & \multicolumn{4}{c||}{Voice Conversion} & \multicolumn{2}{c}{F0 Manipulation} \\
\cmidrule{3-8}
& & PER~$\downarrow$ & WER~$\downarrow$ & EER~$\downarrow$ & MOS~$\uparrow$ & VDE~$\uparrow$ & FFE~$\uparrow$ \\
\midrule
VCTK & GT  & 17.16 & 4.32 & 3.25 & 4.11$\pm$0.29 & -- & -- \\
\midrule 
\multirow{3}{*}{LJ}
% & ASR-TTS   & 50.74  & --     & 66.08 & 32.96 & 1.46 \\
& CPC       & 22.22 	& 16.11 		& 0.46 		& 3.57$\pm$0.15 		& \bf 46.68 & \bf 48.71\\
& HuBERT    & \bf 19.09 & \bf 12.23 & \bf 0.31  & \bf 3.71$\pm$0.24 & 39.20 		& 48.42\\
& VQ-VAE    & 40.88 	& 36.96 		& 9.65 		& 2.90$\pm$0.17 		& 10.54 	& 12.08 \\
\midrule 
\multirow{3}{*}{VCTK} 
% & ASR-TTS   & 68.88  & --    & 41.77 & 13.55 & 6.48 \\
& CPC       &  23.58 		& 15.98 		& \bf 4.83  &  3.42 $\pm$ 0.24 		& \bf 25.29 & \bf 26.97 \\
& HuBERT    &  \bf 20.85 	& \bf 12.72 & 6.01  		& \bf  3.58 $\pm$ 0.28 	& 23.46 	& 26.67 \\
& VQ-VAE    & 36.88  		& 29.44 		& 11.56 		& 3.08 $\pm$ 0.34 		& 7.03  	& 7.80  \\
\bottomrule
\end{tabular}}
\vspace{-0.4cm}
\end{table}

\vspace{-0.1cm}
\section{Results}
\vspace{-0.1cm}
Our results cover
% We report results for 
three different settings: (i) speech reconstruction experiments; (ii) speaker conversion and F0 manipulation; (iii) bitrate analysis with subjective tests for speech codec evaluation. We employ two datasets: LJ~\cite{ljspeech17} single speaker dataset and VCTK~\cite{vctk} multi-speaker dataset. All datasets were resampled to a 16kHz sample rate.

% \paragraph*{Implementation Details.}
% \smallskip
\noindent{\bf Implementation Details\quad} 
\label{sec:impl}
We follow the same setup as in~\cite{lakhotia2021generative}. For CPC, we used the model from~\cite{Riviere2020}, which was trained on a ``clean'' 6k hour sub-sample of the LibriLight dataset~\cite{Kahn2020,Riviere2020}. We extract a downsampled representation from an intermediate layer with a 256-dimensional embedding and a hop size of 160 audio samples. For HuBERT we used a \textsc{Base} 12 transformer-layer model trained for two iterations~\cite{hsu2020hubert} on 960 hours of LibriSpeech corpus~\cite{Panayotov2015}. 
% This model encodes every 320 raw audio samples into a 768-dimensional vector. 
This model downsamples the raw audio $\times320$ into a sequence of 768-dimensional vectors. Similarly to~\cite{lakhotia2021generative}, activations were extracted from the sixth layer.

%CPC: We use a dictionary of 100 units, leading to a bitrate of 700bps.
%HuBERT: A dictionary of 100 units is used, leading to a bitrate of 350bps. 
%VQVE: The VQ-VAE discrete code operates at a bitrate of 800bps.
% For both CPC and HuBERT, the k-means algorithm is applied to convert continuous frames to discrete codes, using the LibriSpeech clean-100h~\cite{Panayotov2015} dataset. 
For CPC and HuBERT, the k-means algorithm is trained on LibriSpeech clean-100h~\cite{Panayotov2015} dataset to convert continuous frames to discrete codes. We quantize both learned representations with $K=100$ centroids. Leading to a bitrate of 700bps for CPC and 350bps for HuBERT.

% VQ-VAE
Similarly to CPC models, we trained the VQ-VAE content encoder model on the ``clean'' 6K hours subset from the LibriLight dataset. We use an encoder operating on the raw signal to extract discrete units, similar to~\cite{jukebox}. In addition, ``random restarts'' were performed when the mean usage of a codebook vector fell below a predetermined threshold. Finally, we used HiFiGAN (architecture and objective) as the decoder instead of a simple convolutional decoder, as it improved the overall audio quality. This model encodes the raw audio into a sequence of discrete tokens from 256 possible tokens~\cite{garbacea2019low} with a hop size of 160 raw audio samples. The VQ-VAE discrete code operates at a bitrate of 800bps. We additionally experimented with 100 discrete units for VQ-VAE, however results were the best for 256. This finding is consistent with~\cite{garbacea2019low}.

% verification model
The speaker verification network uses the architecture proposed in~\cite{heigold2016end}. It was trained on the VoxCeleb2~\cite{voxceleb2} dataset, achieving a 7.4\% Equal Error Rate (EER) for speaker verification on the test split of the VoxCeleb1~\cite{Nagrani17} dataset.

% pitch
Only a single F0 representation is considered across all evaluated models, trained on the VCTK dataset.
% The F0 is extracted from the raw audio using YAAPT~\cite{yaapt} algorithm, using a window size of 20ms and a 5ms hop. 
The F0 is extracted from the raw audio using a window size of 20ms and a 5ms hop. 
As a result, the F0 sequence is sampled at 200Hz. 
% We apply the quantization described at Sec.~\ref{sec:method}, using a pitch codebook of $K'=20$ tokens and an encoder that downsamples the pitch by $\times16$. 
The quantization described at Sec.~\ref{sec:method}, is applied using an F0 codebook of $K'=20$ tokens and an encoder that downsamples the signal by $\times16$. Hence, the discrete F0 representation is sampled at 12.5Hz, leading to a bitrate of 65bps. The final bitrate of the evaluated codecs is the sum of the pitch code bitrate with the content code bitrate.

% \paragraph*{Evaluation Metrics}
% \smallskip
\noindent{\bf Evaluation Metrics\quad} 
We consider both subjective and objective evaluation metrics. For subjective tests, we report the Mean Opinion Scores (MOS). In which human evaluators rate the naturalness of audio samples on a scale of 1--5. Each experiment, included 50 randomly selected samples rated by 30 raters. For objective evaluation, we consider: (i) Equal Error Rate~(EER) as an automatic speaker verification metric obtained using a pre-trained speaker verification network. We report EER between test utterances and enrolled speakers; (ii) Voicing Decision Error (VDE)~\cite{nakatani2008method}, which measures the portion of frames with voicing decision error; (iii) F0 Frame Error (FFE)~\cite{chu2009reducing}, measures the percentage of frames that contain a deviation of more than 20\% in pitch value or have a voicing decision error; (iv) Word Error Rate (WER) and Phoneme Error Rate (PER), proxy metrics to the intelligibility of the generated audio. We used a pre-trained ASR network~\cite{baevski2020wav2vec} on both reconstructed and converted samples to calculate both metrics. %To generate target phonemes, the g2p-en~\cite{g2pE2019} Grapheme2Phoneme module was used.

% \vspace{-0.1cm}
% \smallskip
\noindent{\bf Reconstruction \& Conversion}
% \vspace{-0.1cm}
We start by reporting the reconstruction performance. Results are summarized in Table~\ref{tab:recon}. When considering the intelligibility of the reconstructed signal HuBERT reaches the lowest PER and WER scores across all models, where both CPC and HuBERT are superior to VQ-VAE. However, when considering F0 reconstruction VQ-VAE outperforms both HuBERT and CPC by a significant margin. This results are somewhat intuitive, bearing in mind VQ-VAE objective is to fully reconstruct the input signal. In terms of subjective evaluation, all models reach similar MOS scores, with one exception of CPC on LJ. 

%Notice, since the same F0 units are used for each method, this result implies the VQ-VAE units contain some information about the F0 of the signal, enabling better reconstruction. Regarding speaker information, the CPC gets the lowest EER. 

To better evaluate the disentanglement properties of each method with respect to speaker identity and F0, we conducted an additional set of experiments aiming at speaker conversion and F0 manipulation. For voice conversion, we converted each test utterance into five random target speakers. Next, we employed a speaker verification network, which extracts \emph{d-vector} representation to evaluate speaker-converted utterances' similarity to real speaker utterances (low error-rate indicates good conversion), providing measurement to the speaker identity's disentanglement from the evaluated coding method. The error-rate is reported between converted test utterances and enrolled speakers. For the LJ speech single speaker dataset, we converted samples from the VCTK dataset to the single speaker and enrolled all VCTK speakers together with the single speaker. Results are summarized in Table~\ref{tab:conv} (left). Unlike resynthesis results, on voice conversion CPC and HuBERT outperform VQ-VAE on both LJ and VCTK datasets, indicating VQ-VAE contains more information about the speaker in the encoded units, hence producing more artifacts. Notice, this also affects WER, PER, and the overall subjective quality (MOS). 

Next, to evaluate the presence of F0 in the discrete units, we flattened the F0 units before synthesizing the signal and calculated VDE and FFE with respect to the original F0 values. F0 flattening was done by setting the speakers' mean F0 value across all voiced frames. In this experiment, we expected units that contain F0 information to be better at F0 reconstruction over disentangled units. Results are summarized in Table~\ref{tab:conv} (right). Notice VQ-VAE can still reconstruct the F0 almost at the same level as when using the original F0 as conditioning (5.2 vs 7.03, and 5.59 vs 7.8), in contrast to CPC and HuBERT.

\begin{figure}[t!]
\centering
\includegraphics[width=0.65\columnwidth, trim={50 20 70 20}]{figures/codec_2.pdf}
% \caption{MUSHRA subjective listening test results as a function of bitrate per second for various methods. Purple dots denote the baseline methods, and green dots the proposed SSL based method.} 
\caption{MUSHRA subjective quality results as a function of bitrate per second. Purple dots denote the baseline methods, and green dots the proposed SSL based method.} 
\label{fig:codec}
\vspace{-0.5cm}
\end{figure}

% \vspace{-0.1cm}
% \smallskip
\noindent{\bf Speech Codec}
Our final experiment evaluates the obtained speech units as a low bitrate speech codec. 
% Therefore, we evaluate how the performance varies as a function of the number of discrete units. Changing the number of units is equivalent to varying the bitrate of the encoded signal. 
We use a subjective MUSHRA-type listening test~\cite{series2014method} to measure the perceived quality of the proposed speech codec with regard to its bitrate constraints. In MUSHRA evaluations, listeners are presented with a labeled uncompressed signal for reference, a set of test samples to rate, a copy of the uncompressed reference, and a low-quality anchor. Listeners are asked to rate each test utterance and the copy of the uncompressed reference with respect to the labeled reference in a scale of 1-100.

The experiment is performed on the VCTK dataset~\cite{vctk}. For evaluation, we used 20 utterances from 5 speakers. The set of speakers in the test data is disjoint with those in the training data. For this experiment, HuBERT models with 50, 100, and 200 units were trained as described in Sec.~\ref{sec:impl}. For comparison, we included other speech codecs in our evaluation: Opus~\cite{valin2012definition} wideband at 9 kbps VBR, Codec2~\cite{rowe2011codec} at 2.4 kbps and LPCNet~\cite{valin2019real} operating at 1.6 kbps. The LPCNet model was trained from scratch on the VCTK dataset following the experimental setup in~\cite{valin2019real}. The VQ-VAE model employs the HiFiGAN decoder trained on the LibriLight dataset to match the amount of data reported in~\cite{garbacea2019low}. We compressed the anchor sample with Speex~\cite{valin2016speex} at 4 kbps as a low anchor. Fig.~\ref{fig:codec} depicts the results. HuBERT with 50 units reaches the best MUSHRA score while its bitrate is only 365bps, which is significantly lower than the baseline methods.

\section{Threats to Validity}
\label{sec:threats}
While this initial study provides promising evidence that prompt engineering can enhance LLMs for software traceability tasks, several threats could limit the validity of our findings. First, we evaluated only three open-source projects and only provide a detailed analysis of one, limiting the generalization of our findings. However, we selected projects that spanned multiple domains, artifact types, and sizes to improve generalizability. We also constructed trace queries that were representative of their parent distribution. Second, existing traceability datasets are typically incomplete, as truly considering every candidate link in a project grows $\mathcal{O}(n^2)$ with the number of artifacts. The LLMs identified potential missing traces, but we could not fully validate their accuracy without a project expert. Third, our study used a limited set of LLMs which may not represent the full space of the current state-of-the-art. However, we chose the leading LLMs from our initial explorations with publicly available commercial models. Clearly, there are many extension to this study considering more datasets, different LLMs, and other prompt engineering methods. We leave the full exploration of the problem space to future work and focus on showing the potential these models have towards advancing automated software traceability.

\section{Conclusions and Future Directions}
\label{sec:conclusion}

\begin{comment}
\begin{figure}
\includegraphics[width=\linewidth]{figs/beyond_tss_lesion.pdf}
\caption[]{End-to-End runtime lesion study of the entire MNIST dataset and the FMA featurized music dataset. Each of DROP's contributions provides a runtime improvement.}
\label{fig:beyond_lesion}
\end{figure}
\end{comment}



\section{Conclusion}
\label{sec:conclusion}

Advanced data analytics techniques must scale to rising data volumes. 
DR techniques offer a powerful toolkit when processing these datasets, with PCA frequently outperforming popular techniques in exchange for high computational cost. 
In response, we propose DROP, a new dimensionality reduction optimizer. 
DROP combines progressive sampling, progress estimation, and online aggregation to identify high quality low dimensional bases via PCA without processing the entire dataset by balancing the runtime of downstream tasks and achieved dimensionality. 
Thus, DROP provides a first step in bridging the gap between quality and efficiency in end-to-end DR for downstream \red{analytics}. 

%We revisit canonical operators for time series dimensionality reduction and the measurement study of~\cite{keogh-study}, and show that PCA is more effective than popular alternatives in the data mining literature often by a margin of over $2\times$ on average on gold-standard time series benchmark data sets with respect to output data dimension. More surprisingly, we empirically demonstrate that a small number of samples are sufficient to accurately characterize directions of maximum variance and obtain a high-quality low-dimensional transformation.






\section*{Acknowledgement}The work in this paper has been partially funded by USA National Science Foundation Grants \# SHF-1901059, SHF-1909007, and PFI-TT-2122689.


\begin{table*}
  \centering
  \caption{Investigated Links}
  \small
  \begin{tabularx}{\textwidth}{c L{4cm} X}
    \toprule

    \multicolumn{1}{c}{ID} & \multicolumn{1}{c}{Source} & \multicolumn{1}{c}{Target}\\
    \midrule \midrule
    1 & The DPU-CCM shall implement a mechanism whereby large memory loads and dumps can be accomplished incrementally. & Memory Upload and Download Handling Data can be uploaded to several types of locations, including:
        \begin{itemize}
          \item DRAM
          \item EEPROM
          \item Hardware registers
          \item EEPROM filesystem
        \end{itemize}
        The D-MEM-DAT-UPLD command specifies the target location. If the destination is the EEPROM filesystem, a "block number" is provided in lieu of a memory address, which is used by the DPU FSW to formulate a filename of the form \textit{eefs1:DPU\_blk.\#\#}, where \#\# is the block number. In this case, once the entirety of the uploaded data is received by the DPU FSW, the uploaded data is then written to that file in the EEPROM filesystem. If a file already exists with that name, it is overwritten. The EEPROM filesystem can be reinitialized using the command D-MEM-DISK-INIT. \\
    \midrule
    2 & The DPU-TMALI shall utilize SCM-DCI-SR, along with ERRNO provided by DPU-DCI to decode errors and place them on an error queue for DPU-CCM. & Control and Monitoring the CCM Control Task initializes the DPU FSW. It is the responsibility of the CCM Control Task to establish a successful boot. It does so by blocking on temporary semaphores, each with a 5 second timeout, after spawning the SCU Interface Task and the CCM Command Task. If both of these tasks report a successful initialization by giving the semaphore, the CCM Control Task toggles the BC\_INDEX parameter in EEPROM to indicate a successful boot. If either task does not report a successful initialization, the CCM Control Task disables the watchdog strobe to effect a reboot of the DPU. The rationale for selecting the successful initialization of these two tasks as the definition of a successful boot is that the DPU FSW requires these tasks, as a minimum, to establish ground contact and provide commandability. Once this initialization is complete, the task blocks on a binary semaphore which is given by the SCUI Command ISR upon arrival of the 1 Hz Clock Message. In the event a Clock Message does not arrive, the semaphore will time out after 1.5 seconds. The CCM Control Task remains alive to create and transmit DPU housekeeping at the appropriate intervals, perform various periodic processing tasks, and to process memory dump commands. The final call to ccmErrEnq() is performed in order that if an error occurs in an interrupt service routine, a global variable is set to the value of the errno which is then enqueued into the Error/Event Queue as part of this task’s normal processing. The DPU-CCM shall collect a TASK\_HBEAT from DPU-SCUI, DPU-CCM, DPU-DCX, DPU-TMALI, and DPU-DPA. Non-responsive tasks will be reported in DPU\_HK. \\
    \midrule
    3 & The DPU-CCM shall collect a TASK\_HBEAT from DPU-SCUI, DPU-CCM, DPU-DCX, DPU-TMALI, and DPU-DPA . Non-responsive tasks will be reported in DPU\_HK. & Control and Monitoring Every time the CCM Control executes, it calls ccmPerProcess() to handle periodic processing responsibilities. Such responsibilities include analog to digital conversion updates, DPU task monitoring, ICU heartbeat message production, and watchdog strobe. The ccmHealthChk() function, called by ccmPerProcess() verifies the execution of other tasks by monitoring the amount of time that has elapsed since each task last reported. Other tasks report their execution to the CCM Control Task by calling the function, ccmTaskReport(), providing their task index. Each task has an expected execution frequency, and if a task does not execute as expected, an error is reported in DPU housekeeping. If the Command Dispatch Task fails to report for an extended period, the DPU will execute a reboot, since it is impossible to command the DPU if this task is not executing, otherwise it will strobe the watchdog. \\
    \midrule
    4 & The DPU-TMALI shall utilize SCM\_DCI\_SR, along with ERRNO provided by DPU-DCI to decode errors and place them on an error queue for DPU-CCM. & Error Collection and Reporting The ccmErrEnq() function tracks the last error reported and its frequency of occurrence. Once an error code has been reported it becomes the previously reported error code maintained by ccmErrEnq(). A repetition count is then incremented for each subsequent, consecutively reported, identical instance of this previously reported error. If this error code is reported more than once in one high-rate housekeeping reporting period, then a special error, S\_ccm\_ERR\_REPEAT is enqueued with the repetition count for the error encoded in the least significant byte. This mechanism effectively reduces the potential for housekeeping telemetry to become flooded with a single repeated error. \\ 
    \midrule
    5 & The DPU-CCM shall process real-time non-deferred commands within B ms of receipt from the ICU or the SCU. & The Command and Control CSC provides the core command and control functionality for the system. It includes tasks for initializing the system at bootup, scheduling housekeeping data generation, monitoring other tasks, executing periodic tasks, and receiving and dispatching real-time commands. It maintains data structures for system state, commands, errors and events. \\
    \bottomrule
  \end{tabularx}
  \label{tab:investigated_links}
\end{table*}

\newpage
\balance
\bibliographystyle{IEEEtranS}
\bibliography{references/related}

\end{document}