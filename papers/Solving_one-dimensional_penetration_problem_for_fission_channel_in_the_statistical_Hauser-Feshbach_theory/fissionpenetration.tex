\documentclass[twocolumn,showpacs,preprintnumbers,prc,superscriptaddress]{revtex4-2}
\usepackage{graphicx,color}
\usepackage{float}
\usepackage{bm}
\usepackage[pdftex,colorlinks=true, linkcolor = blue, citecolor=blue,urlcolor=blue, bookmarksnumbered=true, bookmarksopen=true]{hyperref}
\begin{document}

\title{Solving one-dimensional penetration problem for fission channel in the statistical Hauser-Feshbach theory}

\author{T. Kawano}
\email{kawano@lanl.gov}
\affiliation{Los Alamos National Laboratory, Los Alamos, NM 87545, USA}

\author{P. Talou}
\affiliation{Los Alamos National Laboratory, Los Alamos, NM 87545, USA}

\author{S. Hilaire}
\affiliation{CEA, DAM, DIF, F-91297 Arpajon, France}
\affiliation{Universit\'{e} Paris-Saclay, CEA, LMCE, 91680 Bruy\`{e}res-le-Ch\^{a}tel, France}


\date{\today}
\preprint{LA-UR-23-25089}

\begin{abstract}
We solve the Schr\"{o}dinger equation for an arbitrary one-dimensional
potential energy to calculate the transmission coefficient in the
fission channel of compound nucleus reactions. We incorporate the
calculated transmission coefficients into the statistical
Hauser-Feshbach model calculation for neutron-induced reactions on
$^{235,238}$U and $^{239}$Pu. The one-dimensional model reproduces the
evaluated fission cross section data reasonably well considering the
limited number of model parameters involved. A resonance-like
structure appears in the transmission coefficient for a double-humped
fission barrier shape that includes an intermediate well, which is
understood to be a quantum mechanical effect in the fission channel.
The calculated fission cross sections for the neutron-induced
reactions on $^{235,238}$U and $^{239}$Pu all exhibit a similar
structure.
\end{abstract}
%\pacs{}
\maketitle


\section{Introduction}
\label{sec:introduction}
%% \leavevmode
% \\
% \\
% \\
% \\
% \\
\section{Introduction}
\label{introduction}

AutoML is the process by which machine learning models are built automatically for a new dataset. Given a dataset, AutoML systems perform a search over valid data transformations and learners, along with hyper-parameter optimization for each learner~\cite{VolcanoML}. Choosing the transformations and learners over which to search is our focus.
A significant number of systems mine from prior runs of pipelines over a set of datasets to choose transformers and learners that are effective with different types of datasets (e.g. \cite{NEURIPS2018_b59a51a3}, \cite{10.14778/3415478.3415542}, \cite{autosklearn}). Thus, they build a database by actually running different pipelines with a diverse set of datasets to estimate the accuracy of potential pipelines. Hence, they can be used to effectively reduce the search space. A new dataset, based on a set of features (meta-features) is then matched to this database to find the most plausible candidates for both learner selection and hyper-parameter tuning. This process of choosing starting points in the search space is called meta-learning for the cold start problem.  

Other meta-learning approaches include mining existing data science code and their associated datasets to learn from human expertise. The AL~\cite{al} system mined existing Kaggle notebooks using dynamic analysis, i.e., actually running the scripts, and showed that such a system has promise.  However, this meta-learning approach does not scale because it is onerous to execute a large number of pipeline scripts on datasets, preprocessing datasets is never trivial, and older scripts cease to run at all as software evolves. It is not surprising that AL therefore performed dynamic analysis on just nine datasets.

Our system, {\sysname}, provides a scalable meta-learning approach to leverage human expertise, using static analysis to mine pipelines from large repositories of scripts. Static analysis has the advantage of scaling to thousands or millions of scripts \cite{graph4code} easily, but lacks the performance data gathered by dynamic analysis. The {\sysname} meta-learning approach guides the learning process by a scalable dataset similarity search, based on dataset embeddings, to find the most similar datasets and the semantics of ML pipelines applied on them.  Many existing systems, such as Auto-Sklearn \cite{autosklearn} and AL \cite{al}, compute a set of meta-features for each dataset. We developed a deep neural network model to generate embeddings at the granularity of a dataset, e.g., a table or CSV file, to capture similarity at the level of an entire dataset rather than relying on a set of meta-features.
 
Because we use static analysis to capture the semantics of the meta-learning process, we have no mechanism to choose the \textbf{best} pipeline from many seen pipelines, unlike the dynamic execution case where one can rely on runtime to choose the best performing pipeline.  Observing that pipelines are basically workflow graphs, we use graph generator neural models to succinctly capture the statically-observed pipelines for a single dataset. In {\sysname}, we formulate learner selection as a graph generation problem to predict optimized pipelines based on pipelines seen in actual notebooks.

%. This formulation enables {\sysname} for effective pruning of the AutoML search space to predict optimized pipelines based on pipelines seen in actual notebooks.}
%We note that increasingly, state-of-the-art performance in AutoML systems is being generated by more complex pipelines such as Directed Acyclic Graphs (DAGs) \cite{piper} rather than the linear pipelines used in earlier systems.  
 
{\sysname} does learner and transformation selection, and hence is a component of an AutoML systems. To evaluate this component, we integrated it into two existing AutoML systems, FLAML \cite{flaml} and Auto-Sklearn \cite{autosklearn}.  
% We evaluate each system with and without {\sysname}.  
We chose FLAML because it does not yet have any meta-learning component for the cold start problem and instead allows user selection of learners and transformers. The authors of FLAML explicitly pointed to the fact that FLAML might benefit from a meta-learning component and pointed to it as a possibility for future work. For FLAML, if mining historical pipelines provides an advantage, we should improve its performance. We also picked Auto-Sklearn as it does have a learner selection component based on meta-features, as described earlier~\cite{autosklearn2}. For Auto-Sklearn, we should at least match performance if our static mining of pipelines can match their extensive database. For context, we also compared {\sysname} with the recent VolcanoML~\cite{VolcanoML}, which provides an efficient decomposition and execution strategy for the AutoML search space. In contrast, {\sysname} prunes the search space using our meta-learning model to perform hyperparameter optimization only for the most promising candidates. 

The contributions of this paper are the following:
\begin{itemize}
    \item Section ~\ref{sec:mining} defines a scalable meta-learning approach based on representation learning of mined ML pipeline semantics and datasets for over 100 datasets and ~11K Python scripts.  
    \newline
    \item Sections~\ref{sec:kgpipGen} formulates AutoML pipeline generation as a graph generation problem. {\sysname} predicts efficiently an optimized ML pipeline for an unseen dataset based on our meta-learning model.  To the best of our knowledge, {\sysname} is the first approach to formulate  AutoML pipeline generation in such a way.
    \newline
    \item Section~\ref{sec:eval} presents a comprehensive evaluation using a large collection of 121 datasets from major AutoML benchmarks and Kaggle. Our experimental results show that {\sysname} outperforms all existing AutoML systems and achieves state-of-the-art results on the majority of these datasets. {\sysname} significantly improves the performance of both FLAML and Auto-Sklearn in classification and regression tasks. We also outperformed AL in 75 out of 77 datasets and VolcanoML in 75  out of 121 datasets, including 44 datasets used only by VolcanoML~\cite{VolcanoML}.  On average, {\sysname} achieves scores that are statistically better than the means of all other systems. 
\end{itemize}


%This approach does not need to apply cleaning or transformation methods to handle different variances among datasets. Moreover, we do not need to deal with complex analysis, such as dynamic code analysis. Thus, our approach proved to be scalable, as discussed in Sections~\ref{sec:mining}.
The statistical compound nucleus theory describes the probability for
a formed compound nucleus to decay into a channel $a$ by the partial
width $\Gamma_a$, and the Hauser-Feshbach theory~\cite{Hauser1952}
tells us that the energy-average of width $\langle \Gamma_a \rangle$
can be replaced by the optical model transmission coefficient $T_a$ in
the time-reverse process. This is intuitive for particle or
photon-induced reactions, as the interpretation reads the strength to
decay into the channel $a$ is proportional to the compound nucleus
formation probability from the same channel. For the fission channel,
however, the reverse process is not at all trivial. Several
approximations and models are then employed, which significantly
complicate the comparison and interpretation with experimental fission
cross-section data. Studies on the nuclear fission have a long
history, and comprehensive review articles of the fission calculation
are given by Bj\o{}rnholm and Lynn~\cite{Bjornholm1980},
Wagemans~\cite{Wagemans1991}, and more recently Talou and
Vogt~\cite{Talou2023}.


A traditional approach is to calculate a penetrability (transmission
coefficient) through the fission barrier by adopting the
semi-classical Wentzel–Kramers–Brillouin (WKB)
approximation~\cite{Hill1953}.  We often assume that one-dimensional
(1-D) potential energy forms a double-humped fission barrier shape,
which is predicted by the liquid drop model with the microscopic
(shell and pairing energies) corrections, and apply WKB to each of the
barriers separately. By decoupling these two fission barriers, an
effective (net) transmission coefficient $T_f$ through the whole
potential energy is calculated as
\begin{equation}
  T_f = \frac{T_A T_B}{T_A + T_B} \ ,
  \label{eq:effectiveTf}
\end{equation}
where $T_A$ and $T_B$ are the WKB penetrability through the barriers.
Obviously this treatment over-simplifies the fission penetration
problem, as it ignores potential wells between barriers, which gives
rise to the so-called class-II and class-III (in the triple humped
case) states. Some attempts were made in the past to calculate the
fission transmission coefficient by considering the potential well
between barriers. For example, Sin {\it et al.}~\cite{Sin2006,
Sin2016} defined a continuous fission barrier shape and applied WKB
for each segment to calculate the effective transmission
coefficient. Bouland, Lynn, and Talou~\cite{Bouland2013} implemented
the transition states in the class-II well, through which the
penetrability is expressed in terms of the $R$-matrix
formalism. Romain, Morillon, and Duarte~\cite{Romain2016} reported an
anti-resonant transmission due to the class-II and class-III
states. Some recent developments in the fission calculations are
summarized in Ref.~\cite{Talou2023}.


Segmentation of the potential energy along the nuclear elongation
axis, where the inner barrier, class-II state, outer barrier,
class-III states, $\ldots$, are aligned, still implies that the
penetration through the entire potential energy surface is obtained by
assembling its piece-wise components. Although limited to an
analytical expression of potential energy, Cramer and
Nix~\cite{Cramer1970} obtained an exact solution of wave function in
terms of the parabolic-cylinder functions for the double-humped
potential shape. Sharma and Leboeuf~\cite{Sharma1976} extended this
technique to the triple-humped potential barrier case. By solving the
Schr\"{o}dinger equation numerically, an extension of the Cramer-Nix
model to an arbitrary shape of 1-D potential energy is
straightforward. This was reported by Morillon, Duarte, and
Romain~\cite{Morillon2010} and by ourselves~\cite{LA-UR-15-24956},
where the effective transmission coefficient in
Eq.~(\ref{eq:effectiveTf}) is no longer involved.  The solution of
Schr\"{o}dinger equation for 1-D potential is, however, just one of
all the possible fission paths, whereas the dynamical fission process
takes place through any excited states on top of the fission barrier
in a strongly deformed compound nucleus. To calculate the actual
fission transmission coefficient that can be used in the
Hauser-Feshbach theory calculations, we have to take into account the
penetration through the excited states as well.


Eventually we describe the nuclear fission process from two extreme
point of views, namely the compound nucleus evolves through a fixed
albeit large number of fission paths, or the configuration is fully
mixed in the potential well so that the penetration through the
multiple barriers can be totally decoupled as in
Eq.~(\ref{eq:effectiveTf}).


Our approach follows the more general former case; the fission process
takes place along an eigenstate of the compound nucleus, which is
continuous along the nuclear deformation coordinate. In this paper, we
revisit the Cramer-Nix model and its extension to the arbitrary
potential energy shape, and introduce nuclear excitation to calculate
$T_f$. The obtained $T_f$ is used in the Hauser-Feshbach theory to
calculate the fission cross section, which can be compared with
available experimental data. We perform the cross-section calculations
for two distinct cases, the neutron-induced fission on $^{238}$U where
the total excitation energy is still under the fission barrier, and
that for $^{235}$U and $^{239}$Pu where the system energy is higher
than the barrier. In this paper we limit ourselves to the first-chance
fission only, where no neutron emission occurs prior to
fission. However, extension to the multi-chance fission process is not
complicated at all.



\section{Theory}
\label{sec:theory}
%\section{Theory}
In this section, we give guarantees on our grid-based approach. Suppose there is some underlying distribution $\mathcal{P}$ with corresponding density function $p : \mathbb{R}^d \rightarrow \mathbb{R}_{\ge 0}$ from which our data points $X_{[n]} = \{x_1,...,x_n\}$ are drawn i.i.d. We show guarantees on the density estimator based on the grid cell counts.

We need the following regularity assumptions on the density function. The first ensures that the density function has compact support with smooth boundaries and is lower bounded by some positive quantity, and the other ensures that the density function has smoothness. These are standard assumptions in analyses on density estimation e.g. \cite{gine2002rates,jiang2017uniform,chen2017tutorial,singh2009adaptive}.
\begin{assumption}\label{assumption1}
$p$ has compact support $\mathcal{X} \in \mathbb{R}^d$ and there exists $\lambda_0, r_0, C_0 > 0$ such that $p(x) \ge \lambda_0$ for all $x \in \mathcal{X}$ and $\text{Vol}(B(x, r) \cap \mathcal{X}) \ge C_0 \cdot \text{Vol}(B(x, r))$ for all $x \in \mathcal{X}$ and $0 < r \le r_0$, where $B(x, r) := \{x' \in \mathbb{R}^d: |x-x'| \le r\}$.
\end{assumption}
\begin{assumption}\label{assumption2}
$p$ is $\alpha$-Hölder continuous for some $0 < \alpha \le 1$: i.e. there exists $C_\alpha > 0$ such that $|p(x) - p(x')| \le C_\alpha \cdot |x - x'|^\alpha$ for all $x, x' \in \mathbb{R}^d$.
\end{assumption}

We now give the result, which says that for $h$ sufficiently small depending on $p$ (if $h$ is too large, then the grid is too coarse to learn a statistically consistent density estimator), and $n$ sufficiently large, there will be a high probability finite-sample uniform bound on the difference between the density estimator and the true density. The proof can be found in the Appendix.
\begin{theorem}\label{theorem}
Suppose Assumption~\ref{assumption1} and~\ref{assumption2} hold. Then there exists constants $C, C_{1} > 0$ depending on $p$ such that the following holds.
Let $0 < \delta < 1$, $0 < h < \text{min}\{\left(\frac{\lambda_0}{2\cdot C_\alpha}\right)^{1/\alpha}, r_0\}$, $nh^d \ge C_1$. Let $\mathcal{G}_h$ be a partitioning of $\mathbb{R}^d$ into grid cells of edge-length $h$ and for $x \in \mathbb{R}^d$. Let $G(x)$ denote the cell in $\mathcal{G}_h$ that $x$ belongs to.  Then, define the corresponding density estimator $\widehat{p}_h$ as:
\begin{align*}
    \widehat{p}_h(x) := \frac{|X_{[n]} \cap G(x)|}{n\cdot h^d}.
\end{align*}
Then, with probability at least $1 - \delta$:
\begin{align*}
    \sup_{x \in \mathbb{R}^d} |\widehat{p}_h(x)  - p(x)| \le C\cdot \left( h^\alpha + \frac{\sqrt{\log(1/(h\delta)}}{\sqrt{n\cdot h^d}} \right).
\end{align*}
\end{theorem}


\begin{remark}
In the above result, choosing $h \approx n^{-1/(2\alpha+d)}$ optimizes the convergence rate to $\tilde{O}(n^{-\alpha/(2\alpha+d)})$, which is the minimax optimal convergence up to logarithmic factors for the density estimation problem as established by Tsybakov \cite{tsybakov1997nonparametric,tsybakov2008introduction}.
\end{remark}
In other words, the grid-based approach statistically performs at least as well as any estimator of the density function, including the density estimator used by MeanShift. It is worth noting that while our results only provide results for the density estimation portion of MeanShift++ (i.e. the grid-cell binning technique), we prove the near-minimax optimality of this estimation. This implies that the information contained in the density estimation portion serves as an approximately sufficient statistic for the rest of the procedure, which behaves similarly to MeanShift, which operates on another, also nearly-optimal density estimator. Thus, existing analyses of MeanShift e.g. \cite{arias2016estimation,chen2015convergence,xiang2005convergence,li2007note,ghassabeh2015sufficient,ghassabeh2013convergence,subbarao2009nonlinear} can be adapted here; however, it is known that MeanShift is very difficult to analyze \cite{dasgupta2014optimal} and a complete analysis is beyond the scope of this paper.

\subsection{Fission transmission coefficient for double-humped fission barrier}

First we briefly summarize the standard technique to calculate the
fission transmission coefficient $T_f$ for the double-humped fission
barrier. The objective is to emphasize the distinction between the
conventional fission calculation and our approach. The fission barrier
is approximated by an inverted parabola characterized by the barrier
parameters; the heights $V_A$ for the inner barrier and $V_B$ for the
outer barrier, and their curvatures $C_A$ and $C_B$ (the curvature is
often denoted by $\hbar\omega$), as shown schematically in
Fig.~\ref{fig:double_humped_barrier}. By applying the WKB
approximation to the parabolic-shaped barriers, the transmission
coefficient is given by the Hill-Wheeler expression~\cite{Hill1953}
\begin{equation}
  T_i(E)
  = {{1} \over
     {1 + \exp\left(
                2\pi \frac{V_i + E - E_0)}{C_i}
              \right)}} , \qquad i = A, B \ ,
  \label{eq:TfWKB}
\end{equation}
where $E_0$ is the initial excitation energy, $E$ is the nuclear
excitation energies measured from the top of each barrier.  The
``lumped'' transmission coefficient $T$ is the sum of all possible
excited states at $E_k$ for the discrete levels and at $E_x$ in the
continuum,
\begin{eqnarray}
  T_i
  &=& \sum_k T_i(E_k) \nonumber\\
  &+& \int_{E_c}^\infty T_i(E_x) \rho_i(E_x) dE_x , \qquad i = A, B \ ,
  \label{eq:Tlumped}
\end{eqnarray}
where $\rho(E_x)$ is the level density on top of each barrier, and
$E_c$ is the highest discrete state energy. Although we didn't specify
the spin and parity of the compound nucleus, the summation and
integration are performed for the same spin and parity states.  Often
some phenomenological models are applied to $\rho(E_x)$ to take the
nuclear deformation effect into account, which is the so-called
collective enhancement~\cite{Junghans1998}. A standard technique in
calculating fission cross sections, {\it e.g.} as adopted by
Iwamoto~\cite{Iwamoto2007}, assumes typical nuclear deformations at
the inner and outer barriers. Generally speaking the collective
enhancement is model and assumption dependent, which makes fission
model comparison difficult.


When the fission barriers $V_A$ and $V_B$ are fully decoupled, $T_A$
represents a probability to go through the inner barrier, and a
branching ratio from the intermediate state to the outer direction is
$T_B/(T_A+T_B)$.  The effective fission transmission coefficient is
thus given by Eq.~(\ref{eq:effectiveTf}). This expression implies that
the dynamical process in the class-II well is fully adiabatic, and it
virtually forms a semi-stable compound state.  It should be noted that
there is no explicit fission path in this model, since integration
over the excited states in Eq.~(\ref{eq:Tlumped}) is performed before
connecting $T_A$ and $T_B$.


\begin{figure}
 \resizebox{\columnwidth}{!}{\includegraphics{double_humped_barrier.pdf}}
 \caption{Schematic picture of double-humped potential energy
   along the nuclear deformation direction, showing the double-humped
   fission barriers $V_A$ and $V_B$, and the class-I and class-II
   wells between the barriers. The initial compound nucleus state
   is at $E_0$ in class-I, which decays through the states at
   $E_A$ and $E_B$ on top of each barrier.}
 \label{fig:double_humped_barrier}
\end{figure}



\subsection{Fission transmission coefficient for 1-D shape}

\subsubsection{Concatenated parabolas}

The Schr\"{o}dinger equation for an arbitrary one-dimensional (1-D)
potential energy shape can be solved exactly without the WKB
approximation by applying the numerical integration
technique. Although our purpose is to solve problems for any fission
barrier shape, it is still convenient to employ the parabolic
representation to compare with the double-humped barrier
cases. Similar to the three-quadratic-surface parameterization of
nuclear shape~\cite{Nix1969, Moller2009}, the 1-D barrier is
parameterized by smoothly connected parabolas
\begin{equation}
  V(i,x) = V_i + (-1)^i \frac{1}{2} c_i (x - x_i)^2  , \qquad i = 1, 2, \ldots \ ,
  \label{eq:potential}
\end{equation}
where $i$ is the region index for the segmented parabola (odd $i$ for
barriers, and even for wells), $x$ is a dimensionless deformation
coordinate, $c_i = \mu C_i^2 / \hbar^2$, $V_i$ is the top (bottom)
energy of the barrier (well), $x_i$ is the center of each parabola,
and $\mu$ is the inertial mass parameter. Note that the region index
adopted here corresponds to the double-humped case as $A = 1$ and $B =
3$.  Because the deformation coordinate is dimensionless, the
calculated result is insensitive to $\mu$, and we take
\begin{equation}
  \frac{\mu}{\hbar^2} = 0.054 A^{5/3} \qquad \mbox{MeV$^{-1}$}
\end{equation}
as suggested by Cramer and Nix~\cite{Cramer1970}. The region index $i$
runs from 1 to 3 for the double-humped shape, and 5 for the
triple-humped shape. The double-humped case is shown in
Fig.~\ref{fig:connected_barrier} by the solid curve.


By providing the barrier parameters $V_i$ and $C_i$, the junction
point ($\xi_i$) and the parabola center ($x_i$) for each adjacent
region are automatically determined through continuity
relations. Since the abscissa is arbitrary in the 1-D model, we first
fix the center of the first barrier at
\begin{equation}
  x_1 = x_{\rm min} + \sqrt{ \frac{2 V_1}{c_1} } \ ,
\end{equation}
where $x_{\rm min}$ is an arbitrary small offset. The consecutive
central points are given by
\begin{equation}
  x_i = x_{i-1}
      + \sqrt{ \frac{2|V_{i-1} - V_i|(c_{i-1}+c_i)}{c_{i-1}c_i} } \ ,
  \label{eq:center}
\end{equation}
and the junction points are 
\begin{equation}
  \xi_i = \frac{c_i x_i + c_{i+1} x_{i+1}}{c_i + c_{i+1} } \ .
  \label{eq:junction}
\end{equation}
With the central points of Eq.~(\ref{eq:center}) and the junction
points of Eq.~(\ref{eq:junction}), the segmented parabolas in
Eq.~(\ref{eq:potential}) are smoothly concatenated.


In the class-II and/or class-III well between the barriers, it is
possible to add a small imaginary potential that accounts for flux
absorption~\cite{Sin2016}
\begin{equation}
  W(i,x) = 
  \left\{
  \begin{array}{ll}
    \Delta V - W_i & \Delta V \leq W_i \\
    0              & \Delta V  > W_i \\
  \end{array}
  \right. \ ,
\end{equation}
where $\Delta V = V(i,x) - V_i$, $i=2$ for class-II and $i=4$ for
class-III. We assume the potential shape is the same as the real part,
while the strength is given by a parameter $W_i$.


\begin{figure}
 \resizebox{\columnwidth}{!}{\includegraphics{connected_barrier.pdf}}
 \caption{Schematic picture of 1-D potential energy
   along the nuclear deformation direction. 
   The initial compound nucleus state is at $E_0$, 
   which decays through 1-D fission paths,
   {\it e.g.} the trajectories 1, 2, etc.}
 \label{fig:connected_barrier}
\end{figure}


\subsubsection{Solution of 1-D Schr\"{o}dinger equation}

The 1-D Schr\"{o}dinger equation for the fission channel of compound
nucleus at the system energy $E$ is written
as~\cite{Cramer1970}
\begin{equation}
  \frac{d^2}{dx^2}\phi(x)
  + \frac{2\mu}{\hbar^2}
     \left\{ E - \left( V(x) + iW(x) \right) \right\} \phi(x) = 0 \ ,
\end{equation}
where the wave function $\phi(x)$ satisfies the following boundary
condition~\cite{Back1971}
\begin{equation}
  \phi(x) \simeq
  \left\{
  \begin{array}{ll}
    u^{(-)}(kx) - S u^{(+)}(kx) & x > x_{\rm max} \\
    A u^{(-)}(kx) & x < x_{\rm min} \\
  \end{array}
  \right. \ .
  \label{eq:boundary}
\end{equation}
$[x_{\rm min}, x_{\rm max}]$ is the entire range of fission
barrier considered, $k =\sqrt{2\mu E}$ is the wave number, $A$ is the
amplitude of wave function in the class-I well, and
\begin{equation}
  u^{(\pm)}(kx) = \cos(kx) \pm i \sin(kx) \ .
  \label{eq:extfunc}
\end{equation}
The Sch\"{o}dinger equation in the internal region can be solved
numerically by a standard technique such as the Numerov method or
Fox-Goodwin method~\cite{Fox1949}. The solution at the matching point
$x_m$ in the external region ($x_m > x_{\rm max}$) is written as
\begin{equation}
  \psi(x_m) = u^{(-)}(kx_m) - S u^{(+)}(kx_m) \ ,
\end{equation}
and the internal solution $\phi(x_m)$ is smoothly connected with the
external solution at $x_m$. Analog to the scattering matrix element in
the single-channel optical model, the coefficient $S$ is then given by
\begin{equation}
  S = \frac{f u^{(-)}(x_m) - g^{(-)}}{f u^{(+)}(x_m) - g^{(+)}} \ ,
\end{equation}
where 
\begin{equation}
  f \equiv \left. \frac{d\phi / dx}{\phi} \right|_{x_m} \quad \mbox{and}\quad
  g^{(\pm)} \equiv \left. \frac{d u^{(\pm)}}{dx} \right|_{x_m} \  .
\end{equation}


When the potential is real everywhere, the fission transmission
coefficient is given by
\begin{equation}
  T = 1 - |S|^2 \ .
\end{equation}
In the case of the complex potential, the amplitude $A$ in
Eq.~(\ref{eq:boundary}) is given by the normalization factor of the
internal wave function at $x_m$,
\begin{equation}
  A = \left. \frac{u^{(-)} - S u^{(+)}}{\phi} \right|_{x_m} \ ,
\end{equation}
and the transmission coefficient through the barrier is $T_d = |A|^2$.
Because of the loss of flux due to the imaginary potential, $T_d$ is
smaller than $T$, and $T_d$ goes into the statistical Hauser-Feshbach
theory instead of $T$.


\subsubsection{Potential energy for excited states}

Since penetration through the potential defined by
Eq.~(\ref{eq:potential}) is merely one of all the possible fission
paths, we have to aggregate such possible trajectories (paths) to
calculate the lumped transmission coefficient, which is analogous to
Eq.~(\ref{eq:Tlumped}). While the fission penetration for the ground
state takes place through the shape of potential energy in
Eq.~(\ref{eq:potential}), each of the excited states would be
constructed on top of the ground state trajectory. This is a critical
difference between the double-humped and 1-D models, as an adiabatic
intermediate state assumed in the double-humped model conceals an
actual fission path along the deformation coordinate, while it is
explicit in the 1-D model.


To define the fission trajectories for the excited states, one of the
most naive assumptions is that the potential energy is shifted by the
excitation energy $E_x$ as $V(x) = V_0(x) + E_x$, where $V_0$ is the
potential for the ground state.  This, however, ignores distortion of
the eigenstate spectrum in a compound nucleus as it changes shape.  At
the limit of adiabatic change in the nuclear shape, the excitation
energy of each of the eigenstates changes slightly due to shell,
pairing, and nuclear deformation effects.  This results in distortion
of the trajectories, as opposed to a simple shift in energy.



We empirically know that calculated fission cross sections
underestimate experimental data if we simply adopt the level density
$\rho(E_x)$ for an equilibrium shape in the lump-sum of
Eq.~(\ref{eq:Tlumped}). Therefore we often employ some models to
enhance the level densities on top of each of the barriers, which
account for increasing collective degree-of-freedom in a strongly
deformed nucleus. Instead of introducing the collective enhancement in
our 1-D penetration calculation, we assume the excitation energies of
the states will be lowered due to the nuclear deformation. In other
words, the eigenstates in a compound nucleus at relatively low
excitation energies are distorted by deformation effects.  An
illustration of the distortion effect corresponding to a compression
is schematically shown in Fig.~\ref{fig:connected_barrier} by the
dotted curves---trajectories 2 and 3. This trajectory compression
should be mitigated for the higher excitation energies, which is also
phenomenologically known as the damping of collectivity. Although the
compression might depend on the deformation as it changes the pairing
and shell effects, we model the compression in a rather simple way to
eliminate unphysical over-fitting to observed data. We assume the
eigenstates in the compound nucleus are compressed by a factor that
depends on the excitation energy only. Our ansatz reads
\begin{equation}
  \varepsilon_x =
      \left\{ 
        f_0 + \left(1 - e^{-f_1 E_x} \right) \left(1 - f_0\right)
      \right \} E_x , \
  \label{eq:compression}
\end{equation}
where the parameter $f_0$ is roughly 0.8 and the damping
$f_1$ is $\sim 0.2$~MeV$^{-1}$ as shown later. The corresponding fission
trajectory for the excited states is now
\begin{equation}
   V(x) = V_0(x) + \varepsilon_x \ .
\end{equation}
The transmission coefficient for this trajectory is
$T(\varepsilon_x)$, and the lumped transmission coefficient $T_f$ is
given by
\begin{equation}
  T_f  = \sum_k T(\varepsilon_k) 
       + \int_{E_c}^\infty T(\varepsilon_x) \rho(\varepsilon_x) dE_x \ ,
  \label{eq:Tlumped2}
\end{equation}
where the summation and integration are performed for the spin and
parity conserved states. Although the integration range goes to
infinity, or some upper-limit value could be
considered~\cite{Hilaire2003}, this converges quickly with increasing
excitation energy. Generally it is safe to truncate the integration at
$E_x = E_0$.




\section{Results and Discussion}
\label{sec:results}
%% obs-noise = 0.05, derivative-obs-noise = 0.2
\begin{tabular}{llll}
\toprule
            & HIP-GP & SVGP   & Exact GP \\
\midrule
RMSE        & 0.0192 & 0.0192 & 0.0192 \\
Uncertainty & 0.0198 & 0.0206 & 0.0198   \\
\bottomrule
\end{tabular}


\iffalse
% obs-noise = 0.05, derivative-obs-noise = 0.03
\begin{tabular}{llll}
\toprule
            & HIP-GP & SVGP   & Exact GP \\
\midrule
RMSE        & 0.0165 & 0.0165 & 0.0165   \\
Uncertainty & 0.0167 & 0.0175 & 0.0167   \\
\bottomrule
\end{tabular}


% obs-noise = 0.05, derivative-obs-noise = 0.1
\begin{tabular}{llll}
\toprule
            & HIP-GP & SVGP   & Exact GP \\
\midrule
RMSE        & 0.0173 & 0.0172 & 0.0173  \\
Uncertainty & 0.0181 & 0.0189 & 0.0181   \\
\bottomrule
\end{tabular}
\fi

\subsection{Wave function and transmission coefficient for a single fission path}

As an example of the 1-D model, the calculated wave functions for
connected parabolas are shown in Fig.~\ref{fig:wavefunc}, which is for
the $A=239$ system. The assumed barrier heights are $V_1=6.5$,
$V_2=1$, and $V_3=5.5$~MeV, with the curvatures of $C_1=0.6$ and
$C_2=0.4$, and $C_3=0.5$~MeV. We depict the three cases of system
energy $E$; (a) below both of the barriers, (b) between $V_1$ and
$V_3$, and (c) above the both.


Since the 1-D potential penetration problem is invariant whether
numerical integration is performed from the right or left side, the
wave function is normalized to the external function that has unit
amplitude. The penetrability is seen as the amplitude of wave function
inside the potential region. Apparently the wave function penetrates
through the potential barrier when the system has enough energy to
overcome the both barriers $E>V_1$ and $E>V_3$, and it is blocked if
the barrier is higher than the system energy. However, although the
wave function damps rapidly, quantum tunneling is still seen beyond
the barrier.


One of the remarkable differences from the double-humped model in
Eq.~(\ref{eq:effectiveTf}) with the Hill-Wheeler expression of
Eq.~(\ref{eq:TfWKB}) is that the 1-D model sometimes exhibits
resonating behavior due to the penetration through the class-II
well. This was already reported by Cramer and Nix~\cite{Cramer1970} in
their parabolic-cylinder function expression. It should be noted that
this is not an actual compound nucleus resonance, but a sort of the
size effect where the traveling and reflecting waves have accidentally
the same phase. As a result the wave function is amplified
significantly at a resonating energy.


This amplification can be seen easily in the transmission coefficient
as in Fig.~\ref{fig:transmission}.  The top panel is for the same
potential as the one in Fig.~\ref{fig:wavefunc}.  The first sharp
resonance appears below the inner barrier of $V_1=5.5$~MeV, and the
second broader resonance is just above the barrier. We also depicted
the transmission coefficients calculated with the WKB approximation in
Eq.~(\ref{eq:TfWKB}) for the inner and outer barriers. As it is known,
the WKB approximation works reasonably well when the energy is close
to the fission barrier. However, it deviates notably from the 1-D
model when an interference effect of penetrations through the inner
and outer barriers becomes visible.


This effect becomes more remarkable when the inner and outer barriers
have a similar magnitude, which results in a special circumstance that
the penetration and reflection waves are in phase. The bottom panel in
Fig.~\ref{fig:transmission} is the case where these barriers have the
same height of 6.0~MeV. A broad resonance appears just below the
fission barrier, which enhances the fission cross section even if the
compound state is still below the fission barrier. Then the
penetration drops rapidly as the excitation energy decreases. On the
contrary, the penetration by WKB stays higher in the sub-threshold
region. Under these circumstances, the Hill-Wheeler expression may
give unreliable fission cross sections, albeit their magnitude would
be quite small. Nuclear reaction codes sometimes introduce a
phenomenological class-II (and class-III) resonance effect to
compensate for this deficiency~\cite{Talou2023, RIPL3}.


The difference in the WKB curves in Fig.~\ref{fig:transmission} (b) is
due to the curvatures, and both curves approach to $T_A = T_B = 1$
once the system has more than the barrier energy. However, the
effective transmission coefficient becomes 1/2, when
Eq.~(\ref{eq:effectiveTf}) is applied. This is also an important
difference between the double-humped and 1-D models, as the 1-D model
always gives $T = 1$ when the system energy can overcome all the
barriers.


\begin{figure}
 \resizebox{\columnwidth}{!}{\includegraphics{wavefunc_A.pdf}}\\
 \resizebox{\columnwidth}{!}{\includegraphics{wavefunc_B.pdf}}\\
 \resizebox{\columnwidth}{!}{\includegraphics{wavefunc_C.pdf}}
 \caption{Calculated wave functions for the connected parabolas.
   The potential energy is shown by the dot-dashed curves. The solid and dotted
   curves are the normalized wave function (solid for the real part,
   and dotted for the imaginary part). (a) the system energy lies
   below both barrier heights, (b) the energy is between the barriers,
   and (c) is above the barriers case.}
 \label{fig:wavefunc}
\end{figure}


\begin{figure}
 \resizebox{\columnwidth}{!}{\includegraphics{transmission_A.pdf}}\\
 \resizebox{\columnwidth}{!}{\includegraphics{transmission_B.pdf}}
 \caption{Calculated transmission coefficients for the connected parabolas
   as a function of the excitation energy. The top panel 
   (a) is for the potential characterized by $V_1=6.5$, $V_2=1.0$,
   and $V_3=5.5$~MeV, with the curvatures of $C_1=0.6$ and $C_2=0.4$,
   and $C_3=0.5$~MeV, 
   and the bottom panel (b) is for the $V_1=V_3 = 6.0$~MeV case.
   The dashed and dotted curves are the WKB approximation for the inner and 
   outer barriers.}
 \label{fig:transmission}
\end{figure}


\subsection{Fission path through complex potential}

The wave function is absorbed by a potential when a complex class-II
well is given, which results in reduction in the fission transmission
coefficient. When we add a small imaginary part ($W_2 = 0.5$~MeV) to
class-II in the potentials shown in Fig.~\ref{fig:wavefunc}, the
calculated transmission coefficients are compared with the real
potential cases in Fig.~\ref{fig:transmissionImag}. The imaginary
potential acts on the wave function as amplitude damping so that the
asymptotic transmission coefficients at higher energies will be less
than unity. In this case, the asymptotic value of $T_d = |A|^2$ is
$\sim 0.5$, which is determined by $W_2$. The imaginary potential also
shifts the phase of wave function slightly, and the resonance-like
shape is less pronounced.


When a larger imaginary potential is provided, the fission
transmission coefficient goes to almost zero. A physical meaning of
the amplitude damping is not so definite, since the imaginary strength
is arbitrary. This is analogous to the optical model; an incident
particle disappears in the optical potential by its imaginary part
regardless of the nuclear reaction mechanisms. A possible
interpretation is that the system is trapped by a shape isomeric state
that might be long-lived.


\begin{figure}
 \resizebox{\columnwidth}{!}{\includegraphics{transmission_C.pdf}}
 \caption{Calculated transmission coefficients for complex potentials
   as a function of the excitation energy. The thin solid and dotted curves
   are the calculated transmission coefficients same as the one shown
   in Fig.~\ref{fig:transmission}. The thick curves are the transmission
   coefficients when an imaginary strength of 0.5~MeV is added to the
   class-II well.}
 \label{fig:transmissionImag}
\end{figure}



\subsection{Hauser-Feshbach model calculation}

We incorporate the lumped fission transmission coefficient in
Eq.~(\ref{eq:Tlumped2}) into the statistical Hauser-Feshbach model
calculation to demonstrate applicability of the 1-D model in actual
compound nucleus calculations. We do not include the imaginary
potential, so that the calculated results will be tightly constrained
by the potential shape characterized by a limited number of inputs.


The calculation is performed with the CoH$_3$ statistical
Hauser-Feshbach code~\cite{Kawano2019}, which properly combines the
coupled-channels optical model and the statistical Hauser-Feshbach
theory by performing the Engelbrecht-Weidenm\"{u}ller
transformation~\cite{Engelbrecht1973, kawano2015, Kawano2016} of the
optical model penetration matrix~\cite{Satchler1963}. This is
particularly important for nuclear reaction modeling in the actinide
mass region. We employ the coupled-channels optical potential by
Soukhovitskii {\it et al.}~\cite{Soukhovitskii2004} for producing the
neutron penetration matrix and the generalized transmission
coefficients~\cite{Kawano2009}.


To look at the fission channel more carefully, we take some reasonable
model inputs for other reaction channels from literature and do not
attempt to make fine-tuning as the purpose of this study is not a
parameter fitting. Since the curvature parameter $C = \hbar\omega$ is
relatively insensitive to fission cross section calculation, we fix
them to a typical value of 0.6~MeV, and roughly estimate the heights
of inner and outer barriers as well as the trajectory compression
parameters in Eq.~(\ref{eq:compression}) by comparing with
experimental fission cross section data. The class-II depth has also a
moderate impact on the calculation of transmission coefficients as far
as we provide a reasonable value. We fix it to 0.5~MeV. Other model
parameters are set to default internal values in CoH$_3$.  The
$\gamma$-ray strength function is taken from Kopecky and
Uhl~\cite{Kopecky1990} with the M1 scissors mode~\cite{Mumpower2017},
the Gilbert-Cameron composite formula~\cite{Gilbert1965, Kawano2006}
for the level density, and the discrete level data taken from
RIPL-3~\cite{RIPL3}.


First, we perform the statistical model calculations for
neutron-induced reaction on $^{238}$U, where sub-threshold fission may
be seen below about 1~MeV of incident neutron energy.  The ground
state rotational band members, $0^+$, $2^+$, $4^+$, and $6^+$ are
coupled with the deformation parameters taken from the Finite Range
Droplet Model (FRDM)~\cite{Moller1995}. The calculated fission cross
sections are shown in Fig.~\ref{fig:fissionU238} by comparing with the
evaluated fission cross sections in ENDF/B-VIII.0~\cite{ENDF8} and
JENDL-5~\cite{JENDL5}.  The reason of showing the evaluations instead
of actual experimental data is that the evaluated data often include
more experimental information than the direct measurement of $^{238}$U
fission cross section, {\it e.g.}  cross section ratio
measurements. The accuracy of the evaluations is good enough to test
the relevance of this new model. We found that the case of $V_1=6$,
$V_2=0.5$, and $V_3=5$~MeV reasonably reproduce the evaluated fission
cross section in the energy range of our interest. The compression
parameters $f_0=0.8$ and $f_1=0.2$~MeV$^{-1}$ were needed to reproduce
the fission cross section plateau above 2~MeV.  We also calculated the
$V_3=5.5$~MeV case, which produces a more resonance-like structure below
1~MeV, despite the fact that they tend to underestimate the
evaluations on average.


Since the resonating behavior seen in the sub-threshold region
originates from the wave function in between the inner and outer
barriers, their locations and amplitudes strongly depend on the shape
of the potential energy surface.  Because the 1-D potential energy
constructed by smoothly concatenating segmented parabolas is a crude
approximation, we naturally understand that such structure in the
experimental data cannot be predicted exactly by the model unless we
modify the potential shape freely. This being said, the fission cross
sections calculated with the 1-D model in the sub-threshold region are
not so far from reality, which is usually not so obvious in the
Hill-Wheeler case.


\begin{figure}
 \resizebox{\columnwidth}{!}{\includegraphics{fissionU238.pdf}}
 \caption{Calculated fission cross section for neutron-induced reaction
   on $^{238}$U. The barrier height parameters are $V_1=6.0$, $V_2=0.5$,
   and $V_3=5.0$ for the solid curve. The dashed curve is for $V_3=5.5$~MeV.} 
 \label{fig:fissionU238}
\end{figure}


The neutron-induced reaction on $^{235}$U does not have a threshold in
the fission channel. The neutron separation energy is 6.55~MeV, and
the compound nucleus already has enough energy to fission even for a
thermal-energy neutron incident. We adopt the same trajectory
compression parameters, $V_2$, and curvature parameters as those in
the $^{238}$U calculation, and just look for $V_1$ and $V_3$. We found
that the set of $V_1=5.9$ and $V_3=5.7$~MeV gives a reasonable fit to
the experimental $^{235}$U fission cross section, as compared with the
evaluated values in Fig.~\ref{fig:fissionU235}. The resonance-like
structure, which is seen in the sub-threshold fission of $^{238}$U, is
also seen near 60~keV. The evaluated data also show a small bump near
30~keV, which might be attributed to enhancement of the wave-function
amplitude in between the barriers. However, it is hard to claim that
our predicted peak at 60~keV corresponds to the observed bump, as the
potential energy shape is over-simplified in this study.


To show a sensitivity of the inner barrier (or the higher one), a
range of calculated fission cross sections by changing $V_1$ by $\pm
100$~keV is shown by the dashed curves.  More resonance-like structure
appears when $V_1$ is reduced to 5.8~MeV, because there is only a
100~keV difference between $V_1$ and $V_3$.  When the difference is
larger, $V_1+100$~keV, the structure becomes less pronounced.  A
similar sensitivity study was performed by Neudecker {\it el
  al.}~\cite{Neudecker2021}. where a 100--150~keV change in the fission
barrier height changes the calculated fission cross sections by 10\%
or so, while the cross section shape remains the same in the
conventional fission model.


While $V_1$ has such a large sensitivity, the outer barrier (or the
lower one) does not change the calculated fission cross section much,
as far as $V_3$ is lower than $V_1$ by a few hundred keV or
more. Figure~\ref{fig:fissionU235} includes the case of $V_1=5.9$ and
$V_3=5.4$~MeV, where the resonance-like structure is fully washed
out. We do not show the sensitivity of $V_3$ by further lowering the
outer barrier, since these curves are hard to distinguish
anymore. Astonishingly, the calculated fission cross sections remain
almost identical even if $V_3=1$~MeV, which implies that the fission
calculation is totally governed by the single-humped fission barrier
shape.


\begin{figure}
 \resizebox{\columnwidth}{!}{\includegraphics{fissionU235.pdf}}
 \caption{Calculated fission cross section for neutron-induced reaction
   on $^{235}$U. The barrier height parameters are $V_1=5.9$, $V_2=0.5$,
   and $V_3=5.7$~MeV for the solid curve. The dashed curves are the case 
   when $V_1 \pm 100$~keV.} 
 \label{fig:fissionU235}
\end{figure}


Figure~\ref{fig:fissionPu239} shows the calculated fission cross
section of $^{239}$Pu. In this case, it was difficult to obtain a
reasonable fit to the evaluations by employing the same compression
parameters, and a reduction of $f_0$ to 0.55 was needed ($f_1$ is the
same as before). The barrier height parameters are $V_1=5.9$,
$V_2=5.7$~MeV. The resonance-like structure also appears, although it
is not so noticeable like in the uranium cases. We also show the
cross-section band when $V_1\pm 100$~keV. The sensitivity of $V_1$ to
the fission cross section is similar to $^{235}$U. The evaluated cross
sections are roughly covered by the $\pm100$~keV band. However, again,
we emphasize that the objective of the present study is not to fit
perfectly the model calculation to the experimental data but to
demonstrate the fact that the simple 1-D model is potentially capable
of capturing the gross features of the fission reaction process by
producing calculated fission cross sections in reasonable agreement
with experimental data, without the need for a large number of fitting
model parameters.


\begin{figure}
 \resizebox{\columnwidth}{!}{\includegraphics{fissionPu239.pdf}}
 \caption{Calculated fission cross section for neutron-induced reaction
   on $^{239}$Pu. The barrier height parameters are $V_1=5.9$, $V_2=0.5$,
   and $V_3=5.7$~MeV for the solid curve. The dashed curves are the case 
   when $V_1 \pm 100$~keV.} 
 \label{fig:fissionPu239}
\end{figure}


\subsection{Possible refinements}

Although we employed the parameterized potential shape, which is
constructed by segmented parabolas, the experimental fission cross
sections are reasonably reproduced by a few model parameters that
characterize the shape itself. This is already a significant
improvement of the statistical Hauser-Feshbach calculations for
fission compared to the traditional Hill-Wheeler expression for the
double-humped fission barriers connected by
Eq.~(\ref{eq:effectiveTf}). For better reproduction of available
experimental data, as well as prediction of unknown fission cross
sections, we envision further improvement by incorporating a few
theoretical ingredients.


First, the potential energy shape could be taken from the potential
energy surface calculated microscopically~\cite{Goriely2009} or by
semi-microscopic approaches~\cite{Moller2009, Moller2015,
  Verriere2021, Jachimowicz2021}.  Because the potential energy
surface is often defined in a multi-dimensional deformation coordinate
space, either we have to project the surface onto a one-dimensional
axis (it is, however, known that the projection often causes
discontinuity problems~\cite{Dubray2012}), or our 1-D model should be
extended to a set of coupled-equations for the multi-dimensional
coordinate. Second, we should employ a better trajectory compression
model rather than the simple damping of Eq.~(\ref{eq:compression}),
where nuclear deformation effect is ignored, nevertheless it is known
that the single-particle spectrum depends on the nuclear
deformation. Because our trajectory compression model is constant
along the deformation axis, the potential penetration calculation
becomes invariant for exchange of the inner and outer barriers, while
the calculated potential energy surface often indicates that the inner
barrier tends to be higher than the outer barrier for the U and Pu
isotopes. Such a property might be seen by introducing the trajectory
distortion that is deformation dependent.  The nuclear
deformation can be calculated with the full- or semi-microscopic
approaches, where broken symmetries in the nuclear shape are naturally
taken into account. We could estimate possible trajectories by
calculating the microscopic level densities based on the single
particle energies in the deformed one-body potential.


  
\section{Conclusion}
\label{sec:conclusion}
%% \vspace{-0.5em}
\section{Conclusion}
% \vspace{-0.5em}
Recent advances in multimodal single-cell technology have enabled the simultaneous profiling of the transcriptome alongside other cellular modalities, leading to an increase in the availability of multimodal single-cell data. In this paper, we present \method{}, a multimodal transformer model for single-cell surface protein abundance from gene expression measurements. We combined the data with prior biological interaction knowledge from the STRING database into a richly connected heterogeneous graph and leveraged the transformer architectures to learn an accurate mapping between gene expression and surface protein abundance. Remarkably, \method{} achieves superior and more stable performance than other baselines on both 2021 and 2022 NeurIPS single-cell datasets.

\noindent\textbf{Future Work.}
% Our work is an extension of the model we implemented in the NeurIPS 2022 competition. 
Our framework of multimodal transformers with the cross-modality heterogeneous graph goes far beyond the specific downstream task of modality prediction, and there are lots of potentials to be further explored. Our graph contains three types of nodes. While the cell embeddings are used for predictions, the remaining protein embeddings and gene embeddings may be further interpreted for other tasks. The similarities between proteins may show data-specific protein-protein relationships, while the attention matrix of the gene transformer may help to identify marker genes of each cell type. Additionally, we may achieve gene interaction prediction using the attention mechanism.
% under adequate regulations. 
% We expect \method{} to be capable of much more than just modality prediction. Note that currently, we fuse information from different transformers with message-passing GNNs. 
To extend more on transformers, a potential next step is implementing cross-attention cross-modalities. Ideally, all three types of nodes, namely genes, proteins, and cells, would be jointly modeled using a large transformer that includes specific regulations for each modality. 

% insight of protein and gene embedding (diff task)

% all in one transformer

% \noindent\textbf{Limitations and future work}
% Despite the noticeable performance improvement by utilizing transformers with the cross-modality heterogeneous graph, there are still bottlenecks in the current settings. To begin with, we noticed that the performance variations of all methods are consistently higher in the ``CITE'' dataset compared to the ``GEX2ADT'' dataset. We hypothesized that the increased variability in ``CITE'' was due to both less number of training samples (43k vs. 66k cells) and a significantly more number of testing samples used (28k vs. 1k cells). One straightforward solution to alleviate the high variation issue is to include more training samples, which is not always possible given the training data availability. Nevertheless, publicly available single-cell datasets have been accumulated over the past decades and are still being collected on an ever-increasing scale. Taking advantage of these large-scale atlases is the key to a more stable and well-performing model, as some of the intra-cell variations could be common across different datasets. For example, reference-based methods are commonly used to identify the cell identity of a single cell, or cell-type compositions of a mixture of cells. (other examples for pretrained, e.g., scbert)


%\noindent\textbf{Future work.}
% Our work is an extension of the model we implemented in the NeurIPS 2022 competition. Now our framework of multimodal transformers with the cross-modality heterogeneous graph goes far beyond the specific downstream task of modality prediction, and there are lots of potentials to be further explored. Our graph contains three types of nodes. while the cell embeddings are used for predictions, the remaining protein embeddings and gene embeddings may be further interpreted for other tasks. The similarities between proteins may show data-specific protein-protein relationships, while the attention matrix of the gene transformer may help to identify marker genes of each cell type. Additionally, we may achieve gene interaction prediction using the attention mechanism under adequate regulations. We expect \method{} to be capable of much more than just modality prediction. Note that currently, we fuse information from different transformers with message-passing GNNs. To extend more on transformers, a potential next step is implementing cross-attention cross-modalities. Ideally, all three types of nodes, namely genes, proteins, and cells, would be jointly modeled using a large transformer that includes specific regulations for each modality. The self-attention within each modality would reconstruct the prior interaction network, while the cross-attention between modalities would be supervised by the data observations. Then, The attention matrix will provide insights into all the internal interactions and cross-relationships. With the linearized transformer, this idea would be both practical and versatile.

% \begin{acks}
% This research is supported by the National Science Foundation (NSF) and Johnson \& Johnson.
% \end{acks}
We proposed a new model to calculate fission cross sections in the
statistical Hauser-Feshbach framework. Instead of applying the WKB
approximation for uncoupled fission barriers as often done in the
past, we solved the Schr\"{o}dinger equation for a one-dimensional
(1-D) potential model to calculate the penetration probabilities
(transmission coefficients) in the fission channel of compound nucleus
reactions. Because we took continuity of the fission path into
consideration, the expression to combine several penetrabilities for
different barriers, like $T = T_A T_B/(T_A+T_B)$, is no longer
involved in our model. Although the potential shape was parameterized
by smoothly concatenated parabolas for a sake of convenience, the
model can be applied to any arbitrary shape, as we obtain the wave
function by the numerical integration technique.


We showed that a resonance-like structure manifests in the calculated
transmission coefficients for the double-humped fission barrier that
includes a potential well between them, which is understood to be a
quantum mechanical effect in the fission channel. The resonance-like
structure becomes more remarkable when these barriers have a similar
height, where the penetration and reflection waves are in phase. The
structure becomes less sharp when an imaginary part is introduced in
the potential well. The complex potential also absorbs the flux of
fission channel, resulting in lower transmission coefficients.


The 1-D potential model was incorporated into the statistical
Hauser-Feshbach model to calculate neutron-induced reactions on
$^{235,238}$U and $^{239}$Pu. In this case we didn't include the
imaginary part in the potential. In order to calculate the potential
penetration for the excited states, we introduced a simple trajectory
compression model to account for change in the nuclear structure due
to the nuclear deformation. By aggregating the fission transmission
coefficients for all the possible fission paths that are energetically
allowed, calculated fission cross sections for $^{235,238}$U and
$^{239}$Pu were compared with the evaluated data that represent the
experimental cross sections.  We showed that reasonable reproduction
of the data can be obtained by a limited number of model
parameters. Although the detailed structure seen in the experimental
fission cross section is hardly reproduced by the 1-D model due to a
crude approximation for the potential adopted, further improvement
could be made by more careful studies on the potential shape, together
with more realistic trajectory compression models.

  
\section*{Acknowledgments}

TK thanks B. Morillon and P. Romain of CEA for valuable discussions on
this subject.  TK and PT were supported by the Advanced Simulation and
Computing (ASC) Program, National Nuclear Security Administration,
U.S. Department of Energy.  This work was carried out under the
auspices of the National Nuclear Security Administration of the
U.S. Department of Energy at Los Alamos National Laboratory under
Contract No. 89233218CNA000001.

%\bibliography{reference}
%apsrev4-2.bst 2019-01-14 (MD) hand-edited version of apsrev4-1.bst
%Control: key (0)
%Control: author (8) initials jnrlst
%Control: editor formatted (1) identically to author
%Control: production of article title (0) allowed
%Control: page (0) single
%Control: year (1) truncated
%Control: production of eprint (0) enabled
\begin{thebibliography}{41}%
\makeatletter
\providecommand \@ifxundefined [1]{%
 \@ifx{#1\undefined}
}%
\providecommand \@ifnum [1]{%
 \ifnum #1\expandafter \@firstoftwo
 \else \expandafter \@secondoftwo
 \fi
}%
\providecommand \@ifx [1]{%
 \ifx #1\expandafter \@firstoftwo
 \else \expandafter \@secondoftwo
 \fi
}%
\providecommand \natexlab [1]{#1}%
\providecommand \enquote  [1]{``#1''}%
\providecommand \bibnamefont  [1]{#1}%
\providecommand \bibfnamefont [1]{#1}%
\providecommand \citenamefont [1]{#1}%
\providecommand \href@noop [0]{\@secondoftwo}%
\providecommand \href [0]{\begingroup \@sanitize@url \@href}%
\providecommand \@href[1]{\@@startlink{#1}\@@href}%
\providecommand \@@href[1]{\endgroup#1\@@endlink}%
\providecommand \@sanitize@url [0]{\catcode `\\12\catcode `\$12\catcode
  `\&12\catcode `\#12\catcode `\^12\catcode `\_12\catcode `\%12\relax}%
\providecommand \@@startlink[1]{}%
\providecommand \@@endlink[0]{}%
\providecommand \url  [0]{\begingroup\@sanitize@url \@url }%
\providecommand \@url [1]{\endgroup\@href {#1}{\urlprefix }}%
\providecommand \urlprefix  [0]{URL }%
\providecommand \Eprint [0]{\href }%
\providecommand \doibase [0]{https://doi.org/}%
\providecommand \selectlanguage [0]{\@gobble}%
\providecommand \bibinfo  [0]{\@secondoftwo}%
\providecommand \bibfield  [0]{\@secondoftwo}%
\providecommand \translation [1]{[#1]}%
\providecommand \BibitemOpen [0]{}%
\providecommand \bibitemStop [0]{}%
\providecommand \bibitemNoStop [0]{.\EOS\space}%
\providecommand \EOS [0]{\spacefactor3000\relax}%
\providecommand \BibitemShut  [1]{\csname bibitem#1\endcsname}%
\let\auto@bib@innerbib\@empty
%</preamble>
\bibitem [{\citenamefont {Hauser}\ and\ \citenamefont
  {Feshbach}(1952)}]{Hauser1952}%
  \BibitemOpen
  \bibfield  {author} {\bibinfo {author} {\bibfnamefont {W.}~\bibnamefont
  {Hauser}}\ and\ \bibinfo {author} {\bibfnamefont {H.}~\bibnamefont
  {Feshbach}},\ }\bibfield  {title} {\bibinfo {title} {The inelastic scattering
  of neutrons},\ }\href {https://doi.org/10.1103/PhysRev.87.366} {\bibfield
  {journal} {\bibinfo  {journal} {Phys. Rev.}\ }\textbf {\bibinfo {volume}
  {87}},\ \bibinfo {pages} {366} (\bibinfo {year} {1952})}\BibitemShut
  {NoStop}%
\bibitem [{\citenamefont {Bj\o{}rnholm}\ and\ \citenamefont
  {Lynn}(1980)}]{Bjornholm1980}%
  \BibitemOpen
  \bibfield  {author} {\bibinfo {author} {\bibfnamefont {S.}~\bibnamefont
  {Bj\o{}rnholm}}\ and\ \bibinfo {author} {\bibfnamefont {J.~E.}\ \bibnamefont
  {Lynn}},\ }\bibfield  {title} {\bibinfo {title} {The double-humped fission
  barrier},\ }\href {https://doi.org/10.1103/RevModPhys.52.725} {\bibfield
  {journal} {\bibinfo  {journal} {Rev. Mod. Phys.}\ }\textbf {\bibinfo {volume}
  {52}},\ \bibinfo {pages} {725} (\bibinfo {year} {1980})}\BibitemShut
  {NoStop}%
\bibitem [{\citenamefont {Wagemans}(1991)}]{Wagemans1991}%
  \BibitemOpen
  \bibfield  {author} {\bibinfo {author} {\bibfnamefont {C.}~\bibnamefont
  {Wagemans}},\ }\href@noop {} {\emph {\bibinfo {title} {The Nuclear Fission
  Process}}}\ (\bibinfo  {publisher} {CRC Press},\ \bibinfo {year}
  {1991})\BibitemShut {NoStop}%
\bibitem [{\citenamefont {Talou}\ and\ \citenamefont {Vogt}(2023)}]{Talou2023}%
  \BibitemOpen
  \bibfield  {author} {\bibinfo {author} {\bibfnamefont {P.}~\bibnamefont
  {Talou}}\ and\ \bibinfo {author} {\bibfnamefont {R.}~\bibnamefont {Vogt}},\
  }\href {https://doi.org/10.1007/978-3-031-14545-2} {\emph {\bibinfo {title}
  {Nuclear Fission, Theories, Experiments and Applications}}}\ (\bibinfo
  {publisher} {Springer},\ \bibinfo {year} {2023})\BibitemShut {NoStop}%
\bibitem [{\citenamefont {Hill}\ and\ \citenamefont
  {Wheeler}(1953)}]{Hill1953}%
  \BibitemOpen
  \bibfield  {author} {\bibinfo {author} {\bibfnamefont {D.~L.}\ \bibnamefont
  {Hill}}\ and\ \bibinfo {author} {\bibfnamefont {J.~A.}\ \bibnamefont
  {Wheeler}},\ }\bibfield  {title} {\bibinfo {title} {Nuclear constitution and
  the interpretation of fission phenomena},\ }\href
  {https://doi.org/10.1103/PhysRev.89.1102} {\bibfield  {journal} {\bibinfo
  {journal} {Phys. Rev.}\ }\textbf {\bibinfo {volume} {89}},\ \bibinfo {pages}
  {1102 } (\bibinfo {year} {1953})}\BibitemShut {NoStop}%
\bibitem [{\citenamefont {Sin}\ \emph {et~al.}(2006)\citenamefont {Sin},
  \citenamefont {Capote}, \citenamefont {Ventura}, \citenamefont {Herman},\
  and\ \citenamefont {Oblo\ifmmode~\check{z}\else
  \v{z}\fi{}insk\'y}}]{Sin2006}%
  \BibitemOpen
  \bibfield  {author} {\bibinfo {author} {\bibfnamefont {M.}~\bibnamefont
  {Sin}}, \bibinfo {author} {\bibfnamefont {R.}~\bibnamefont {Capote}},
  \bibinfo {author} {\bibfnamefont {A.}~\bibnamefont {Ventura}}, \bibinfo
  {author} {\bibfnamefont {M.}~\bibnamefont {Herman}},\ and\ \bibinfo {author}
  {\bibfnamefont {P.}~\bibnamefont {Oblo\ifmmode~\check{z}\else
  \v{z}\fi{}insk\'y}},\ }\bibfield  {title} {\bibinfo {title} {Fission of light
  actinides: $^{232}\mathrm{Th}$($n$,$f$) and $^{231}\mathrm{Pa}$($n$,$f$)
  reactions},\ }\href {https://doi.org/10.1103/PhysRevC.74.014608} {\bibfield
  {journal} {\bibinfo  {journal} {Phys. Rev. C}\ }\textbf {\bibinfo {volume}
  {74}},\ \bibinfo {pages} {014608} (\bibinfo {year} {2006})}\BibitemShut
  {NoStop}%
\bibitem [{\citenamefont {Sin}\ \emph {et~al.}(2016)\citenamefont {Sin},
  \citenamefont {Capote}, \citenamefont {Herman},\ and\ \citenamefont
  {Trkov}}]{Sin2016}%
  \BibitemOpen
  \bibfield  {author} {\bibinfo {author} {\bibfnamefont {M.}~\bibnamefont
  {Sin}}, \bibinfo {author} {\bibfnamefont {R.}~\bibnamefont {Capote}},
  \bibinfo {author} {\bibfnamefont {M.~W.}\ \bibnamefont {Herman}},\ and\
  \bibinfo {author} {\bibfnamefont {A.}~\bibnamefont {Trkov}},\ }\bibfield
  {title} {\bibinfo {title} {Extended optical model for fission},\ }\href
  {https://doi.org/10.1103/PhysRevC.93.034605} {\bibfield  {journal} {\bibinfo
  {journal} {Phys. Rev. C}\ }\textbf {\bibinfo {volume} {93}},\ \bibinfo
  {pages} {034605} (\bibinfo {year} {2016})}\BibitemShut {NoStop}%
\bibitem [{\citenamefont {Bouland}\ \emph {et~al.}(2013)\citenamefont
  {Bouland}, \citenamefont {Lynn},\ and\ \citenamefont {Talou}}]{Bouland2013}%
  \BibitemOpen
  \bibfield  {author} {\bibinfo {author} {\bibfnamefont {O.}~\bibnamefont
  {Bouland}}, \bibinfo {author} {\bibfnamefont {J.~E.}\ \bibnamefont {Lynn}},\
  and\ \bibinfo {author} {\bibfnamefont {P.}~\bibnamefont {Talou}},\ }\bibfield
   {title} {\bibinfo {title} {{$R$}-matrix analysis and prediction of
  low-energy neutron-induced fission cross sections for a range of {Pu}
  isotopes},\ }\href {https://doi.org/10.1103/PhysRevC.88.054612} {\bibfield
  {journal} {\bibinfo  {journal} {Phys. Rev. C}\ }\textbf {\bibinfo {volume}
  {88}},\ \bibinfo {pages} {054612} (\bibinfo {year} {2013})}\BibitemShut
  {NoStop}%
\bibitem [{\citenamefont {Romain}\ \emph {et~al.}(2016)\citenamefont {Romain},
  \citenamefont {Morillon},\ and\ \citenamefont {Duarte}}]{Romain2016}%
  \BibitemOpen
  \bibfield  {author} {\bibinfo {author} {\bibfnamefont {P.}~\bibnamefont
  {Romain}}, \bibinfo {author} {\bibfnamefont {B.}~\bibnamefont {Morillon}},\
  and\ \bibinfo {author} {\bibfnamefont {H.}~\bibnamefont {Duarte}},\
  }\bibfield  {title} {\bibinfo {title} {{Bruy\`{e}res-le-Ch\^{a}tel} neutron
  evaluations of actinides with the {TALYS} code: The fission channel},\ }\href
  {https://doi.org/10.1016/j.nds.2015.12.003} {\bibfield  {journal} {\bibinfo
  {journal} {Nuclear Data Sheets}\ }\textbf {\bibinfo {volume} {131}},\
  \bibinfo {pages} {222 } (\bibinfo {year} {2016})},\ \bibinfo {note} {special
  Issue on Nuclear Reaction Data}\BibitemShut {NoStop}%
\bibitem [{\citenamefont {Cramer}\ and\ \citenamefont
  {Nix}(1970)}]{Cramer1970}%
  \BibitemOpen
  \bibfield  {author} {\bibinfo {author} {\bibfnamefont {J.~D.}\ \bibnamefont
  {Cramer}}\ and\ \bibinfo {author} {\bibfnamefont {J.~R.}\ \bibnamefont
  {Nix}},\ }\bibfield  {title} {\bibinfo {title} {Exact calculation of the
  penetrability through two-peaked fission barriers},\ }\href
  {https://doi.org/10.1103/PhysRevC.2.1048} {\bibfield  {journal} {\bibinfo
  {journal} {Phys. Rev. C}\ }\textbf {\bibinfo {volume} {2}},\ \bibinfo {pages}
  {1048} (\bibinfo {year} {1970})}\BibitemShut {NoStop}%
\bibitem [{\citenamefont {Sharma}\ and\ \citenamefont
  {Leboeuf}(1976)}]{Sharma1976}%
  \BibitemOpen
  \bibfield  {author} {\bibinfo {author} {\bibfnamefont {R.~C.}\ \bibnamefont
  {Sharma}}\ and\ \bibinfo {author} {\bibfnamefont {J.~N.}\ \bibnamefont
  {Leboeuf}},\ }\bibfield  {title} {\bibinfo {title} {Three-hump potential
  barrier in the $^{234}\mathrm{Th}$ nucleus},\ }\href
  {https://doi.org/10.1103/PhysRevC.14.2340} {\bibfield  {journal} {\bibinfo
  {journal} {Phys. Rev. C}\ }\textbf {\bibinfo {volume} {14}},\ \bibinfo
  {pages} {2340} (\bibinfo {year} {1976})}\BibitemShut {NoStop}%
\bibitem [{\citenamefont {Morillon}\ \emph {et~al.}(2010)\citenamefont
  {Morillon}, \citenamefont {Duarte},\ and\ \citenamefont
  {Romain}}]{Morillon2010}%
  \BibitemOpen
  \bibfield  {author} {\bibinfo {author} {\bibfnamefont {B.}~\bibnamefont
  {Morillon}}, \bibinfo {author} {\bibfnamefont {H.}~\bibnamefont {Duarte}},\
  and\ \bibinfo {author} {\bibfnamefont {P.}~\bibnamefont {Romain}},\
  }\href@noop {} {\emph {\bibinfo {title} {Petits probl\`{e}mes de transmission
  quantique}}},\ \bibinfo {type} {Tech. Rep.}\ (\bibinfo  {institution} {CEA},\
  \bibinfo {year} {2010})\ \bibinfo {note} {private communication}\BibitemShut
  {NoStop}%
\bibitem [{\citenamefont {Kawano}(2015)}]{LA-UR-15-24956}%
  \BibitemOpen
  \bibfield  {author} {\bibinfo {author} {\bibfnamefont {T.}~\bibnamefont
  {Kawano}},\ }\href@noop {} {\emph {\bibinfo {title} {Exact solution of
  fission penetration through arbitrary complex fission barrier}}},\ \bibinfo
  {type} {Tech. Rep.}\ \bibinfo {number} {LA-UR-15-24956}\ (\bibinfo
  {institution} {Los Alamos National Laboratory},\ \bibinfo {year}
  {2015})\BibitemShut {NoStop}%
\bibitem [{\citenamefont {Junghans}\ \emph {et~al.}(1998)\citenamefont
  {Junghans}, \citenamefont {{de Jong}}, \citenamefont {Clerc}, \citenamefont
  {Ignatyuk}, \citenamefont {Kudyaev},\ and\ \citenamefont
  {Schmidt}}]{Junghans1998}%
  \BibitemOpen
  \bibfield  {author} {\bibinfo {author} {\bibfnamefont {A.}~\bibnamefont
  {Junghans}}, \bibinfo {author} {\bibfnamefont {M.}~\bibnamefont {{de Jong}}},
  \bibinfo {author} {\bibfnamefont {H.-G.}\ \bibnamefont {Clerc}}, \bibinfo
  {author} {\bibfnamefont {A.}~\bibnamefont {Ignatyuk}}, \bibinfo {author}
  {\bibfnamefont {G.}~\bibnamefont {Kudyaev}},\ and\ \bibinfo {author}
  {\bibfnamefont {K.-H.}\ \bibnamefont {Schmidt}},\ }\bibfield  {title}
  {\bibinfo {title} {Projectile-fragment yields as a probe for the collective
  enhancement in the nuclear level density},\ }\href
  {https://doi.org/10.1016/S0375-9474(98)00658-7} {\bibfield  {journal}
  {\bibinfo  {journal} {Nuclear Physics A}\ }\textbf {\bibinfo {volume}
  {629}},\ \bibinfo {pages} {635 } (\bibinfo {year} {1998})}\BibitemShut
  {NoStop}%
\bibitem [{\citenamefont {Iwamoto}(2007)}]{Iwamoto2007}%
  \BibitemOpen
  \bibfield  {author} {\bibinfo {author} {\bibfnamefont {O.}~\bibnamefont
  {Iwamoto}},\ }\bibfield  {title} {\bibinfo {title} {Development of a
  comprehensive code for nuclear data evaluation, {CCONE}, and validation using
  neutron-induced cross sections for uranium isotopes},\ }\href
  {https://doi.org/10.1080/18811248.2007.9711857} {\bibfield  {journal}
  {\bibinfo  {journal} {Journal of Nuclear Science and Technology}\ }\textbf
  {\bibinfo {volume} {44}},\ \bibinfo {pages} {687 } (\bibinfo {year}
  {2007})}\BibitemShut {NoStop}%
\bibitem [{\citenamefont {Nix}(1969)}]{Nix1969}%
  \BibitemOpen
  \bibfield  {author} {\bibinfo {author} {\bibfnamefont {J.~R.}\ \bibnamefont
  {Nix}},\ }\bibfield  {title} {\bibinfo {title} {Further studies in the
  liquid-drop theory on nuclear fission},\ }\href
  {https://doi.org/10.1016/0375-9474(69)90730-1} {\bibfield  {journal}
  {\bibinfo  {journal} {Nuclear Physics A}\ }\textbf {\bibinfo {volume}
  {130}},\ \bibinfo {pages} {241 } (\bibinfo {year} {1969})}\BibitemShut
  {NoStop}%
\bibitem [{\citenamefont {M\"oller}\ \emph {et~al.}(2009)\citenamefont
  {M\"oller}, \citenamefont {Sierk}, \citenamefont {Ichikawa}, \citenamefont
  {Iwamoto}, \citenamefont {Bengtsson}, \citenamefont {Uhrenholt},\ and\
  \citenamefont {\AA{}berg}}]{Moller2009}%
  \BibitemOpen
  \bibfield  {author} {\bibinfo {author} {\bibfnamefont {P.}~\bibnamefont
  {M\"oller}}, \bibinfo {author} {\bibfnamefont {A.~J.}\ \bibnamefont {Sierk}},
  \bibinfo {author} {\bibfnamefont {T.}~\bibnamefont {Ichikawa}}, \bibinfo
  {author} {\bibfnamefont {A.}~\bibnamefont {Iwamoto}}, \bibinfo {author}
  {\bibfnamefont {R.}~\bibnamefont {Bengtsson}}, \bibinfo {author}
  {\bibfnamefont {H.}~\bibnamefont {Uhrenholt}},\ and\ \bibinfo {author}
  {\bibfnamefont {S.}~\bibnamefont {\AA{}berg}},\ }\bibfield  {title} {\bibinfo
  {title} {Heavy-element fission barriers},\ }\href
  {https://doi.org/10.1103/PhysRevC.79.064304} {\bibfield  {journal} {\bibinfo
  {journal} {Phys. Rev. C}\ }\textbf {\bibinfo {volume} {79}},\ \bibinfo
  {pages} {064304} (\bibinfo {year} {2009})}\BibitemShut {NoStop}%
\bibitem [{\citenamefont {Back}\ \emph {et~al.}(1971)\citenamefont {Back},
  \citenamefont {Bondorf}, \citenamefont {Otroschenko}, \citenamefont
  {Pedersen},\ and\ \citenamefont {Rasmussen}}]{Back1971}%
  \BibitemOpen
  \bibfield  {author} {\bibinfo {author} {\bibfnamefont {B.~B.}\ \bibnamefont
  {Back}}, \bibinfo {author} {\bibfnamefont {J.~P.}\ \bibnamefont {Bondorf}},
  \bibinfo {author} {\bibfnamefont {G.~A.}\ \bibnamefont {Otroschenko}},
  \bibinfo {author} {\bibfnamefont {J.}~\bibnamefont {Pedersen}},\ and\
  \bibinfo {author} {\bibfnamefont {B.}~\bibnamefont {Rasmussen}},\ }\bibfield
  {title} {\bibinfo {title} {{Fission of U, Np, Pu and Am isotopes excited in
  the (d, p) reaction}},\ }\href {https://doi.org/10.1016/0375-9474(71)90461-1}
  {\bibfield  {journal} {\bibinfo  {journal} {Nuclear Physics A}\ }\textbf
  {\bibinfo {volume} {165}},\ \bibinfo {pages} {449 } (\bibinfo {year}
  {1971})}\BibitemShut {NoStop}%
\bibitem [{\citenamefont {Fox}\ and\ \citenamefont {Goodwin}(1949)}]{Fox1949}%
  \BibitemOpen
  \bibfield  {author} {\bibinfo {author} {\bibfnamefont {L.}~\bibnamefont
  {Fox}}\ and\ \bibinfo {author} {\bibfnamefont {E.~T.}\ \bibnamefont
  {Goodwin}},\ }\bibfield  {title} {\bibinfo {title} {Some new methods for the
  numerical integration of ordinary differential equations},\ }\href
  {https://doi.org/10.1017/s0305004100025007} {\bibfield  {journal} {\bibinfo
  {journal} {Mathematical Proceedings of the Cambridge Philosophical Society}\
  }\textbf {\bibinfo {volume} {45}},\ \bibinfo {pages} {373 } (\bibinfo {year}
  {1949})}\BibitemShut {NoStop}%
\bibitem [{\citenamefont {Hilaire}\ \emph {et~al.}(2003)\citenamefont
  {Hilaire}, \citenamefont {Lagrange},\ and\ \citenamefont
  {Koning}}]{Hilaire2003}%
  \BibitemOpen
  \bibfield  {author} {\bibinfo {author} {\bibfnamefont {S.}~\bibnamefont
  {Hilaire}}, \bibinfo {author} {\bibfnamefont {C.}~\bibnamefont {Lagrange}},\
  and\ \bibinfo {author} {\bibfnamefont {A.~J.}\ \bibnamefont {Koning}},\
  }\bibfield  {title} {\bibinfo {title} {Comparisons between various width
  fluctuation correction factors for compound nucleus reactions},\ }\href
  {https://doi.org/10.1016/S0003-4916(03)00076-9} {\bibfield  {journal}
  {\bibinfo  {journal} {Annals of Physics}\ }\textbf {\bibinfo {volume}
  {306}},\ \bibinfo {pages} {209 } (\bibinfo {year} {2003})}\BibitemShut
  {NoStop}%
\bibitem [{\citenamefont {Capote}\ \emph {et~al.}(2009)\citenamefont {Capote},
  \citenamefont {Herman}, \citenamefont {Oblo\v{z}insk\'{y}}, \citenamefont
  {Young}, \citenamefont {Goriely}, \citenamefont {Belgya}, \citenamefont
  {Ignatyuk}, \citenamefont {Koning}, \citenamefont {Hilaire}, \citenamefont
  {Plujko}, \citenamefont {Avrigeanu}, \citenamefont {Bersillon}, \citenamefont
  {Chadwick}, \citenamefont {Fukahori}, \citenamefont {Ge}, \citenamefont
  {Han}, \citenamefont {Kailas}, \citenamefont {Kopecky}, \citenamefont
  {Maslov}, \citenamefont {Reffo}, \citenamefont {Sin}, \citenamefont
  {Soukhovitskii},\ and\ \citenamefont {Talou}}]{RIPL3}%
  \BibitemOpen
  \bibfield  {author} {\bibinfo {author} {\bibfnamefont {R.}~\bibnamefont
  {Capote}}, \bibinfo {author} {\bibfnamefont {M.}~\bibnamefont {Herman}},
  \bibinfo {author} {\bibfnamefont {P.}~\bibnamefont {Oblo\v{z}insk\'{y}}},
  \bibinfo {author} {\bibfnamefont {P.~G.}\ \bibnamefont {Young}}, \bibinfo
  {author} {\bibfnamefont {S.}~\bibnamefont {Goriely}}, \bibinfo {author}
  {\bibfnamefont {T.}~\bibnamefont {Belgya}}, \bibinfo {author} {\bibfnamefont
  {A.~V.}\ \bibnamefont {Ignatyuk}}, \bibinfo {author} {\bibfnamefont {A.~J.}\
  \bibnamefont {Koning}}, \bibinfo {author} {\bibfnamefont {S.}~\bibnamefont
  {Hilaire}}, \bibinfo {author} {\bibfnamefont {V.~A.}\ \bibnamefont {Plujko}},
  \bibinfo {author} {\bibfnamefont {M.}~\bibnamefont {Avrigeanu}}, \bibinfo
  {author} {\bibfnamefont {O.}~\bibnamefont {Bersillon}}, \bibinfo {author}
  {\bibfnamefont {M.~B.}\ \bibnamefont {Chadwick}}, \bibinfo {author}
  {\bibfnamefont {T.}~\bibnamefont {Fukahori}}, \bibinfo {author}
  {\bibfnamefont {Z.}~\bibnamefont {Ge}}, \bibinfo {author} {\bibfnamefont
  {Y.}~\bibnamefont {Han}}, \bibinfo {author} {\bibfnamefont {S.}~\bibnamefont
  {Kailas}}, \bibinfo {author} {\bibfnamefont {J.}~\bibnamefont {Kopecky}},
  \bibinfo {author} {\bibfnamefont {V.~M.}\ \bibnamefont {Maslov}}, \bibinfo
  {author} {\bibfnamefont {G.}~\bibnamefont {Reffo}}, \bibinfo {author}
  {\bibfnamefont {M.}~\bibnamefont {Sin}}, \bibinfo {author} {\bibfnamefont
  {E.~S.}\ \bibnamefont {Soukhovitskii}},\ and\ \bibinfo {author}
  {\bibfnamefont {P.}~\bibnamefont {Talou}},\ }\bibfield  {title} {\bibinfo
  {title} {{RIPL} - {Reference Input Parameter Library} for calculation of
  nuclear reactions and nuclear data evaluations},\ }\href
  {https://doi.org/10.1016/j.nds.2009.10.004} {\bibfield  {journal} {\bibinfo
  {journal} {Nuclear Data Sheets}\ }\textbf {\bibinfo {volume} {110}},\
  \bibinfo {pages} {3107 } (\bibinfo {year} {2009})}\BibitemShut {NoStop}%
\bibitem [{\citenamefont {Kawano}(2021)}]{Kawano2019}%
  \BibitemOpen
  \bibfield  {author} {\bibinfo {author} {\bibfnamefont {T.}~\bibnamefont
  {Kawano}},\ }\bibfield  {title} {\bibinfo {title} {{CoH$_3$}: The
  coupled-channels and {Hauser-Feshbach} code},\ }\href
  {https://doi.org/10.1007/978-3-030-58082-7} {\bibfield  {journal} {\bibinfo
  {journal} {Springer Proceedings in Physics}\ }\textbf {\bibinfo {volume}
  {254}},\ \bibinfo {pages} {27 } (\bibinfo {year} {2021})},\ \bibinfo {note}
  {{CNR2018}: International Workshop on Compound Nucleus and Related Topics,
  LBNL, Berkeley, CA, USA, September 24 -- 28, 2018, J. Escher, Y. Alhassid,
  L.A. Bernstein, D. Brown, C. Fr\"{o}hlich, P. Talou, W. Younes
  (Eds.)}\BibitemShut {NoStop}%
\bibitem [{\citenamefont {Engelbrecht}\ and\ \citenamefont
  {Weidenm\"{u}ller}(1973)}]{Engelbrecht1973}%
  \BibitemOpen
  \bibfield  {author} {\bibinfo {author} {\bibfnamefont {C.~A.}\ \bibnamefont
  {Engelbrecht}}\ and\ \bibinfo {author} {\bibfnamefont {H.~A.}\ \bibnamefont
  {Weidenm\"{u}ller}},\ }\bibfield  {title} {\bibinfo {title}
  {{Hauser-Feshbach} theory and ericson fluctuations in the presence of direct
  reactions},\ }\href {https://doi.org/10.1103/PhysRevC.8.859} {\bibfield
  {journal} {\bibinfo  {journal} {Phys. Rev. C}\ }\textbf {\bibinfo {volume}
  {8}},\ \bibinfo {pages} {859 } (\bibinfo {year} {1973})}\BibitemShut
  {NoStop}%
\bibitem [{\citenamefont {Kawano}\ \emph {et~al.}(2015)\citenamefont {Kawano},
  \citenamefont {Talou},\ and\ \citenamefont {Weidenm\"uller}}]{kawano2015}%
  \BibitemOpen
  \bibfield  {author} {\bibinfo {author} {\bibfnamefont {T.}~\bibnamefont
  {Kawano}}, \bibinfo {author} {\bibfnamefont {P.}~\bibnamefont {Talou}},\ and\
  \bibinfo {author} {\bibfnamefont {H.~A.}\ \bibnamefont {Weidenm\"uller}},\
  }\bibfield  {title} {\bibinfo {title} {Random-matrix approach to the
  statistical compound nuclear reaction at low energies using the {Monte Carlo}
  technique},\ }\href {https://doi.org/10.1103/PhysRevC.92.044617} {\bibfield
  {journal} {\bibinfo  {journal} {Phys. Rev. C}\ }\textbf {\bibinfo {volume}
  {92}},\ \bibinfo {pages} {044617} (\bibinfo {year} {2015})}\BibitemShut
  {NoStop}%
\bibitem [{\citenamefont {Kawano}\ \emph {et~al.}(2016)\citenamefont {Kawano},
  \citenamefont {Capote}, \citenamefont {Hilaire},\ and\ \citenamefont {Chau
  Huu-Tai}}]{Kawano2016}%
  \BibitemOpen
  \bibfield  {author} {\bibinfo {author} {\bibfnamefont {T.}~\bibnamefont
  {Kawano}}, \bibinfo {author} {\bibfnamefont {R.}~\bibnamefont {Capote}},
  \bibinfo {author} {\bibfnamefont {S.}~\bibnamefont {Hilaire}},\ and\ \bibinfo
  {author} {\bibfnamefont {P.}~\bibnamefont {Chau Huu-Tai}},\ }\bibfield
  {title} {\bibinfo {title} {Statistical {Hauser-Feshbach} theory with
  width-fluctuation correction including direct reaction channels for
  neutron-induced reactions at low energies},\ }\href
  {https://doi.org/10.1103/PhysRevC.94.014612} {\bibfield  {journal} {\bibinfo
  {journal} {Phys. Rev. C}\ }\textbf {\bibinfo {volume} {94}},\ \bibinfo
  {pages} {014612} (\bibinfo {year} {2016})}\BibitemShut {NoStop}%
\bibitem [{\citenamefont {Satchler}(1963)}]{Satchler1963}%
  \BibitemOpen
  \bibfield  {author} {\bibinfo {author} {\bibfnamefont {G.~R.}\ \bibnamefont
  {Satchler}},\ }\bibfield  {title} {\bibinfo {title} {Average compound nucleus
  cross sections in the continuum},\ }\href
  {https://doi.org/10.1016/0031-9163(63)90440-2} {\bibfield  {journal}
  {\bibinfo  {journal} {Physics Letters}\ }\textbf {\bibinfo {volume} {7}},\
  \bibinfo {pages} {55 } (\bibinfo {year} {1963})}\BibitemShut {NoStop}%
\bibitem [{\citenamefont {Soukhovitskii}\ \emph {et~al.}(2004)\citenamefont
  {Soukhovitskii}, \citenamefont {Chiba}, \citenamefont {Lee}, \citenamefont
  {Iwamoto},\ and\ \citenamefont {Fukahori}}]{Soukhovitskii2004}%
  \BibitemOpen
  \bibfield  {author} {\bibinfo {author} {\bibfnamefont {E.~S.}\ \bibnamefont
  {Soukhovitskii}}, \bibinfo {author} {\bibfnamefont {S.}~\bibnamefont
  {Chiba}}, \bibinfo {author} {\bibfnamefont {J.-Y.}\ \bibnamefont {Lee}},
  \bibinfo {author} {\bibfnamefont {O.}~\bibnamefont {Iwamoto}},\ and\ \bibinfo
  {author} {\bibfnamefont {T.}~\bibnamefont {Fukahori}},\ }\bibfield  {title}
  {\bibinfo {title} {Global coupled-channel optical potential for
  nucleon-actinide interaction from {1 keV} to {200 MeV}},\ }\href
  {https://doi.org/10.1088/0954-3899/30/7/007} {\bibfield  {journal} {\bibinfo
  {journal} {Journal of Physics G: Nuclear and Particle Physics}\ }\textbf
  {\bibinfo {volume} {30}},\ \bibinfo {pages} {905 } (\bibinfo {year}
  {2004})}\BibitemShut {NoStop}%
\bibitem [{\citenamefont {Kawano}\ \emph {et~al.}(2009)\citenamefont {Kawano},
  \citenamefont {Talou}, \citenamefont {Lynn}, \citenamefont {Chadwick},\ and\
  \citenamefont {Madland}}]{Kawano2009}%
  \BibitemOpen
  \bibfield  {author} {\bibinfo {author} {\bibfnamefont {T.}~\bibnamefont
  {Kawano}}, \bibinfo {author} {\bibfnamefont {P.}~\bibnamefont {Talou}},
  \bibinfo {author} {\bibfnamefont {J.~E.}\ \bibnamefont {Lynn}}, \bibinfo
  {author} {\bibfnamefont {M.~B.}\ \bibnamefont {Chadwick}},\ and\ \bibinfo
  {author} {\bibfnamefont {D.~G.}\ \bibnamefont {Madland}},\ }\bibfield
  {title} {\bibinfo {title} {Calculation of nuclear reaction cross sections on
  excited nuclei with the coupled-channels method},\ }\href
  {https://doi.org/10.1103/PhysRevC.80.024611} {\bibfield  {journal} {\bibinfo
  {journal} {Phys. Rev. C}\ }\textbf {\bibinfo {volume} {80}},\ \bibinfo
  {pages} {024611} (\bibinfo {year} {2009})}\BibitemShut {NoStop}%
\bibitem [{\citenamefont {Kopecky}\ and\ \citenamefont
  {Uhl}(1990)}]{Kopecky1990}%
  \BibitemOpen
  \bibfield  {author} {\bibinfo {author} {\bibfnamefont {J.}~\bibnamefont
  {Kopecky}}\ and\ \bibinfo {author} {\bibfnamefont {M.}~\bibnamefont {Uhl}},\
  }\bibfield  {title} {\bibinfo {title} {Test of gamma-ray strength functions
  in nuclear reaction model calculations},\ }\href
  {https://doi.org/10.1103/PhysRevC.41.1941} {\bibfield  {journal} {\bibinfo
  {journal} {Phys. Rev. C}\ }\textbf {\bibinfo {volume} {41}},\ \bibinfo
  {pages} {1941 } (\bibinfo {year} {1990})}\BibitemShut {NoStop}%
\bibitem [{\citenamefont {Mumpower}\ \emph {et~al.}(2017)\citenamefont
  {Mumpower}, \citenamefont {Kawano}, \citenamefont {Ullmann}, \citenamefont
  {Krti\ifmmode~\check{c}\else \v{c}\fi{}ka},\ and\ \citenamefont
  {Sprouse}}]{Mumpower2017}%
  \BibitemOpen
  \bibfield  {author} {\bibinfo {author} {\bibfnamefont {M.~R.}\ \bibnamefont
  {Mumpower}}, \bibinfo {author} {\bibfnamefont {T.}~\bibnamefont {Kawano}},
  \bibinfo {author} {\bibfnamefont {J.~L.}\ \bibnamefont {Ullmann}}, \bibinfo
  {author} {\bibfnamefont {M.}~\bibnamefont {Krti\ifmmode~\check{c}\else
  \v{c}\fi{}ka}},\ and\ \bibinfo {author} {\bibfnamefont {T.~M.}\ \bibnamefont
  {Sprouse}},\ }\bibfield  {title} {\bibinfo {title} {Estimation of {$M1$}
  scissors mode strength for deformed nuclei in the medium- to heavy-mass
  region by statistical {Hauser-Feshbach} model calculations},\ }\href
  {https://doi.org/10.1103/PhysRevC.96.024612} {\bibfield  {journal} {\bibinfo
  {journal} {Phys. Rev. C}\ }\textbf {\bibinfo {volume} {96}},\ \bibinfo
  {pages} {024612} (\bibinfo {year} {2017})}\BibitemShut {NoStop}%
\bibitem [{\citenamefont {Gilbert}\ and\ \citenamefont
  {Cameron}(1965)}]{Gilbert1965}%
  \BibitemOpen
  \bibfield  {author} {\bibinfo {author} {\bibfnamefont {A.}~\bibnamefont
  {Gilbert}}\ and\ \bibinfo {author} {\bibfnamefont {A.~G.~W.}\ \bibnamefont
  {Cameron}},\ }\bibfield  {title} {\bibinfo {title} {A composite nuclear-level
  density formula with shell corrections},\ }\href
  {https://doi.org/10.1139/p65-139} {\bibfield  {journal} {\bibinfo  {journal}
  {Can. J. Phys.}\ }\textbf {\bibinfo {volume} {43}},\ \bibinfo {pages} {1446 }
  (\bibinfo {year} {1965})}\BibitemShut {NoStop}%
\bibitem [{\citenamefont {Kawano}\ \emph {et~al.}(2006)\citenamefont {Kawano},
  \citenamefont {Chiba},\ and\ \citenamefont {Koura}}]{Kawano2006}%
  \BibitemOpen
  \bibfield  {author} {\bibinfo {author} {\bibfnamefont {T.}~\bibnamefont
  {Kawano}}, \bibinfo {author} {\bibfnamefont {S.}~\bibnamefont {Chiba}},\ and\
  \bibinfo {author} {\bibfnamefont {H.}~\bibnamefont {Koura}},\ }\bibfield
  {title} {\bibinfo {title} {Phenomenological nuclear level densities using the
  {KTUY05} nuclear mass formula for applications off-stability},\ }\href
  {https://doi.org/10.1080/18811248.2006.9711062} {\bibfield  {journal}
  {\bibinfo  {journal} {Journal of Nuclear Science and Technology}\ }\textbf
  {\bibinfo {volume} {43}},\ \bibinfo {pages} {1 } (\bibinfo {year}
  {2006})}\BibitemShut {NoStop}%
\bibitem [{\citenamefont {M\"{o}ller}\ \emph {et~al.}(1995)\citenamefont
  {M\"{o}ller}, \citenamefont {Nix}, \citenamefont {Myer},\ and\ \citenamefont
  {Swiatecki}}]{Moller1995}%
  \BibitemOpen
  \bibfield  {author} {\bibinfo {author} {\bibfnamefont {P.}~\bibnamefont
  {M\"{o}ller}}, \bibinfo {author} {\bibfnamefont {J.~R.}\ \bibnamefont {Nix}},
  \bibinfo {author} {\bibfnamefont {W.~D.}\ \bibnamefont {Myer}},\ and\
  \bibinfo {author} {\bibfnamefont {W.~J.}\ \bibnamefont {Swiatecki}},\
  }\bibfield  {title} {\bibinfo {title} {Nuclear ground-state masses and
  deformations},\ }\href {https://doi.org/10.1006/adnd.1995.1002} {\bibfield
  {journal} {\bibinfo  {journal} {Atomic Data and Nuclear Data Tables}\
  }\textbf {\bibinfo {volume} {59}},\ \bibinfo {pages} {185 } (\bibinfo {year}
  {1995})}\BibitemShut {NoStop}%
\bibitem [{\citenamefont {Brown}\ \emph {et~al.}(2018)\citenamefont {Brown},
  \citenamefont {Chadwick}, \citenamefont {Capote}, \citenamefont {Kahler},
  \citenamefont {Trkov}, \citenamefont {Herman}, \citenamefont {Sonzogni},
  \citenamefont {Danon}, \citenamefont {Carlson}, \citenamefont {Dunn},
  \citenamefont {Smith}, \citenamefont {Hale}, \citenamefont {Arbanas},
  \citenamefont {Arcilla}, \citenamefont {Bates}, \citenamefont {Beck},
  \citenamefont {Becker}, \citenamefont {Brown}, \citenamefont {Casperson},
  \citenamefont {Conlin}, \citenamefont {Cullen}, \citenamefont {Descalle},
  \citenamefont {Firestone}, \citenamefont {Gaines}, \citenamefont {Guber},
  \citenamefont {Hawari}, \citenamefont {Holmes}, \citenamefont {Johnson},
  \citenamefont {Kawano}, \citenamefont {Kiedrowski}, \citenamefont {Koning},
  \citenamefont {Kopecky}, \citenamefont {Leal}, \citenamefont {Lestone},
  \citenamefont {Lubitz}, \citenamefont {M\'{a}rquez~Dami\'{a}n}, \citenamefont
  {Mattoon}, \citenamefont {McCutchan}, \citenamefont {Mughabghab},
  \citenamefont {Navratil}, \citenamefont {Neudecker}, \citenamefont {Nobre},
  \citenamefont {Noguere}, \citenamefont {Paris}, \citenamefont {Pigni},
  \citenamefont {Plompen}, \citenamefont {Pritychenko}, \citenamefont
  {Pronyaev}, \citenamefont {Roubtsov}, \citenamefont {Rochman}, \citenamefont
  {Romano}, \citenamefont {Schillebeeckx}, \citenamefont {Simakov},
  \citenamefont {Sin}, \citenamefont {Sirakov}, \citenamefont {Sleaford},
  \citenamefont {Sobes}, \citenamefont {Soukhovitskii}, \citenamefont {Stetcu},
  \citenamefont {Talou}, \citenamefont {Thompson}, \citenamefont {van~der
  Marck}, \citenamefont {Welser-Sherrill}, \citenamefont {Wiarda},
  \citenamefont {White}, \citenamefont {Wormald}, \citenamefont {Wright},
  \citenamefont {Zerkle}, \citenamefont {\v{Z}erovnik},\ and\ \citenamefont
  {Zhu}}]{ENDF8}%
  \BibitemOpen
  \bibfield  {author} {\bibinfo {author} {\bibfnamefont {D.~A.}\ \bibnamefont
  {Brown}}, \bibinfo {author} {\bibfnamefont {M.~B.}\ \bibnamefont {Chadwick}},
  \bibinfo {author} {\bibfnamefont {R.}~\bibnamefont {Capote}}, \bibinfo
  {author} {\bibfnamefont {A.~C.}\ \bibnamefont {Kahler}}, \bibinfo {author}
  {\bibfnamefont {A.}~\bibnamefont {Trkov}}, \bibinfo {author} {\bibfnamefont
  {M.~W.}\ \bibnamefont {Herman}}, \bibinfo {author} {\bibfnamefont {A.~A.}\
  \bibnamefont {Sonzogni}}, \bibinfo {author} {\bibfnamefont {Y.}~\bibnamefont
  {Danon}}, \bibinfo {author} {\bibfnamefont {A.~D.}\ \bibnamefont {Carlson}},
  \bibinfo {author} {\bibfnamefont {M.}~\bibnamefont {Dunn}}, \bibinfo {author}
  {\bibfnamefont {D.~L.}\ \bibnamefont {Smith}}, \bibinfo {author}
  {\bibfnamefont {G.~M.}\ \bibnamefont {Hale}}, \bibinfo {author}
  {\bibfnamefont {G.}~\bibnamefont {Arbanas}}, \bibinfo {author} {\bibfnamefont
  {R.}~\bibnamefont {Arcilla}}, \bibinfo {author} {\bibfnamefont {C.~R.}\
  \bibnamefont {Bates}}, \bibinfo {author} {\bibfnamefont {B.}~\bibnamefont
  {Beck}}, \bibinfo {author} {\bibfnamefont {B.}~\bibnamefont {Becker}},
  \bibinfo {author} {\bibfnamefont {F.}~\bibnamefont {Brown}}, \bibinfo
  {author} {\bibfnamefont {R.~J.}\ \bibnamefont {Casperson}}, \bibinfo {author}
  {\bibfnamefont {J.}~\bibnamefont {Conlin}}, \bibinfo {author} {\bibfnamefont
  {D.~E.}\ \bibnamefont {Cullen}}, \bibinfo {author} {\bibfnamefont {M.~A.}\
  \bibnamefont {Descalle}}, \bibinfo {author} {\bibfnamefont {R.}~\bibnamefont
  {Firestone}}, \bibinfo {author} {\bibfnamefont {T.}~\bibnamefont {Gaines}},
  \bibinfo {author} {\bibfnamefont {K.~H.}\ \bibnamefont {Guber}}, \bibinfo
  {author} {\bibfnamefont {A.~I.}\ \bibnamefont {Hawari}}, \bibinfo {author}
  {\bibfnamefont {J.}~\bibnamefont {Holmes}}, \bibinfo {author} {\bibfnamefont
  {T.~D.}\ \bibnamefont {Johnson}}, \bibinfo {author} {\bibfnamefont
  {T.}~\bibnamefont {Kawano}}, \bibinfo {author} {\bibfnamefont {B.~C.}\
  \bibnamefont {Kiedrowski}}, \bibinfo {author} {\bibfnamefont {A.~J.}\
  \bibnamefont {Koning}}, \bibinfo {author} {\bibfnamefont {S.}~\bibnamefont
  {Kopecky}}, \bibinfo {author} {\bibfnamefont {L.}~\bibnamefont {Leal}},
  \bibinfo {author} {\bibfnamefont {J.~P.}\ \bibnamefont {Lestone}}, \bibinfo
  {author} {\bibfnamefont {C.}~\bibnamefont {Lubitz}}, \bibinfo {author}
  {\bibfnamefont {J.~I.}\ \bibnamefont {M\'{a}rquez~Dami\'{a}n}}, \bibinfo
  {author} {\bibfnamefont {C.~M.}\ \bibnamefont {Mattoon}}, \bibinfo {author}
  {\bibfnamefont {E.~A.}\ \bibnamefont {McCutchan}}, \bibinfo {author}
  {\bibfnamefont {S.}~\bibnamefont {Mughabghab}}, \bibinfo {author}
  {\bibfnamefont {P.}~\bibnamefont {Navratil}}, \bibinfo {author}
  {\bibfnamefont {D.}~\bibnamefont {Neudecker}}, \bibinfo {author}
  {\bibfnamefont {G.~P.~A.}\ \bibnamefont {Nobre}}, \bibinfo {author}
  {\bibfnamefont {G.}~\bibnamefont {Noguere}}, \bibinfo {author} {\bibfnamefont
  {M.}~\bibnamefont {Paris}}, \bibinfo {author} {\bibfnamefont {M.~T.}\
  \bibnamefont {Pigni}}, \bibinfo {author} {\bibfnamefont {A.~J.}\ \bibnamefont
  {Plompen}}, \bibinfo {author} {\bibfnamefont {B.}~\bibnamefont
  {Pritychenko}}, \bibinfo {author} {\bibfnamefont {V.~G.}\ \bibnamefont
  {Pronyaev}}, \bibinfo {author} {\bibfnamefont {D.}~\bibnamefont {Roubtsov}},
  \bibinfo {author} {\bibfnamefont {D.}~\bibnamefont {Rochman}}, \bibinfo
  {author} {\bibfnamefont {P.}~\bibnamefont {Romano}}, \bibinfo {author}
  {\bibfnamefont {P.}~\bibnamefont {Schillebeeckx}}, \bibinfo {author}
  {\bibfnamefont {S.}~\bibnamefont {Simakov}}, \bibinfo {author} {\bibfnamefont
  {M.}~\bibnamefont {Sin}}, \bibinfo {author} {\bibfnamefont {I.}~\bibnamefont
  {Sirakov}}, \bibinfo {author} {\bibfnamefont {B.}~\bibnamefont {Sleaford}},
  \bibinfo {author} {\bibfnamefont {V.}~\bibnamefont {Sobes}}, \bibinfo
  {author} {\bibfnamefont {E.~S.}\ \bibnamefont {Soukhovitskii}}, \bibinfo
  {author} {\bibfnamefont {I.}~\bibnamefont {Stetcu}}, \bibinfo {author}
  {\bibfnamefont {P.}~\bibnamefont {Talou}}, \bibinfo {author} {\bibfnamefont
  {I.}~\bibnamefont {Thompson}}, \bibinfo {author} {\bibfnamefont
  {S.}~\bibnamefont {van~der Marck}}, \bibinfo {author} {\bibfnamefont
  {L.}~\bibnamefont {Welser-Sherrill}}, \bibinfo {author} {\bibfnamefont
  {D.}~\bibnamefont {Wiarda}}, \bibinfo {author} {\bibfnamefont
  {M.}~\bibnamefont {White}}, \bibinfo {author} {\bibfnamefont {J.~L.}\
  \bibnamefont {Wormald}}, \bibinfo {author} {\bibfnamefont {R.~Q.}\
  \bibnamefont {Wright}}, \bibinfo {author} {\bibfnamefont {M.}~\bibnamefont
  {Zerkle}}, \bibinfo {author} {\bibfnamefont {G.}~\bibnamefont
  {\v{Z}erovnik}},\ and\ \bibinfo {author} {\bibfnamefont {Y.}~\bibnamefont
  {Zhu}},\ }\bibfield  {title} {\bibinfo {title} {{ENDF/B-VIII.0:} the 8th
  major release of the nuclear reaction data library with {CIELO}-project cross
  sections, new standards and thermal scattering data},\ }\href
  {https://doi.org/10.1016/j.nds.2018.02.001} {\bibfield  {journal} {\bibinfo
  {journal} {Nuclear Data Sheets}\ }\textbf {\bibinfo {volume} {148}},\
  \bibinfo {pages} {1 } (\bibinfo {year} {2018})}\BibitemShut {NoStop}%
\bibitem [{\citenamefont {Iwamoto}\ \emph {et~al.}(2023)\citenamefont
  {Iwamoto}, \citenamefont {Iwamoto}, \citenamefont {Kunieda}, \citenamefont
  {Minato}, \citenamefont {Nakayama}, \citenamefont {Abe}, \citenamefont
  {Tsubakihara}, \citenamefont {Okumura}, \citenamefont {Ishizuka},
  \citenamefont {Yoshida}, \citenamefont {Chiba}, \citenamefont {Otuka},
  \citenamefont {Sublet}, \citenamefont {Iwamoto}, \citenamefont {Yamamoto},
  \citenamefont {Nagaya}, \citenamefont {Tada}, \citenamefont {Konno},
  \citenamefont {Matsuda}, \citenamefont {Yokoyama}, \citenamefont {Taninaka},
  \citenamefont {Oizumi}, \citenamefont {Fukushima}, \citenamefont {Okita},
  \citenamefont {Chiba}, \citenamefont {Sato}, \citenamefont {Ohta},\ and\
  \citenamefont {Kwon}}]{JENDL5}%
  \BibitemOpen
  \bibfield  {author} {\bibinfo {author} {\bibfnamefont {O.}~\bibnamefont
  {Iwamoto}}, \bibinfo {author} {\bibfnamefont {N.}~\bibnamefont {Iwamoto}},
  \bibinfo {author} {\bibfnamefont {S.}~\bibnamefont {Kunieda}}, \bibinfo
  {author} {\bibfnamefont {F.}~\bibnamefont {Minato}}, \bibinfo {author}
  {\bibfnamefont {S.}~\bibnamefont {Nakayama}}, \bibinfo {author}
  {\bibfnamefont {Y.}~\bibnamefont {Abe}}, \bibinfo {author} {\bibfnamefont
  {K.}~\bibnamefont {Tsubakihara}}, \bibinfo {author} {\bibfnamefont
  {S.}~\bibnamefont {Okumura}}, \bibinfo {author} {\bibfnamefont
  {C.}~\bibnamefont {Ishizuka}}, \bibinfo {author} {\bibfnamefont
  {T.}~\bibnamefont {Yoshida}}, \bibinfo {author} {\bibfnamefont
  {S.}~\bibnamefont {Chiba}}, \bibinfo {author} {\bibfnamefont
  {N.}~\bibnamefont {Otuka}}, \bibinfo {author} {\bibfnamefont {J.-C.}\
  \bibnamefont {Sublet}}, \bibinfo {author} {\bibfnamefont {H.}~\bibnamefont
  {Iwamoto}}, \bibinfo {author} {\bibfnamefont {K.}~\bibnamefont {Yamamoto}},
  \bibinfo {author} {\bibfnamefont {Y.}~\bibnamefont {Nagaya}}, \bibinfo
  {author} {\bibfnamefont {K.}~\bibnamefont {Tada}}, \bibinfo {author}
  {\bibfnamefont {C.}~\bibnamefont {Konno}}, \bibinfo {author} {\bibfnamefont
  {N.}~\bibnamefont {Matsuda}}, \bibinfo {author} {\bibfnamefont
  {K.}~\bibnamefont {Yokoyama}}, \bibinfo {author} {\bibfnamefont
  {H.}~\bibnamefont {Taninaka}}, \bibinfo {author} {\bibfnamefont
  {A.}~\bibnamefont {Oizumi}}, \bibinfo {author} {\bibfnamefont
  {M.}~\bibnamefont {Fukushima}}, \bibinfo {author} {\bibfnamefont
  {S.}~\bibnamefont {Okita}}, \bibinfo {author} {\bibfnamefont
  {G.}~\bibnamefont {Chiba}}, \bibinfo {author} {\bibfnamefont
  {S.}~\bibnamefont {Sato}}, \bibinfo {author} {\bibfnamefont {M.}~\bibnamefont
  {Ohta}},\ and\ \bibinfo {author} {\bibfnamefont {S.}~\bibnamefont {Kwon}},\
  }\bibfield  {title} {\bibinfo {title} {Japanese evaluated nuclear data
  library version 5: {JENDL-5}},\ }\href
  {https://doi.org/10.1080/00223131.2022.2141903} {\bibfield  {journal}
  {\bibinfo  {journal} {Journal of Nuclear Science and Technology}\ }\textbf
  {\bibinfo {volume} {60}},\ \bibinfo {pages} {1 } (\bibinfo {year}
  {2023})}\BibitemShut {NoStop}%
\bibitem [{\citenamefont {Neudecker}\ \emph {et~al.}(2021)\citenamefont
  {Neudecker}, \citenamefont {Cabellos}, \citenamefont {Clark}, \citenamefont
  {Grosskopf}, \citenamefont {Haeck}, \citenamefont {Herman}, \citenamefont
  {Hutchinson}, \citenamefont {Kawano}, \citenamefont {Lovell}, \citenamefont
  {Stetcu}, \citenamefont {Talou},\ and\ \citenamefont
  {Vander~Wiel}}]{Neudecker2021}%
  \BibitemOpen
  \bibfield  {author} {\bibinfo {author} {\bibfnamefont {D.}~\bibnamefont
  {Neudecker}}, \bibinfo {author} {\bibfnamefont {O.}~\bibnamefont {Cabellos}},
  \bibinfo {author} {\bibfnamefont {A.~R.}\ \bibnamefont {Clark}}, \bibinfo
  {author} {\bibfnamefont {M.~J.}\ \bibnamefont {Grosskopf}}, \bibinfo {author}
  {\bibfnamefont {W.}~\bibnamefont {Haeck}}, \bibinfo {author} {\bibfnamefont
  {M.~W.}\ \bibnamefont {Herman}}, \bibinfo {author} {\bibfnamefont
  {J.}~\bibnamefont {Hutchinson}}, \bibinfo {author} {\bibfnamefont
  {T.}~\bibnamefont {Kawano}}, \bibinfo {author} {\bibfnamefont {A.~E.}\
  \bibnamefont {Lovell}}, \bibinfo {author} {\bibfnamefont {I.}~\bibnamefont
  {Stetcu}}, \bibinfo {author} {\bibfnamefont {P.}~\bibnamefont {Talou}},\ and\
  \bibinfo {author} {\bibfnamefont {S.}~\bibnamefont {Vander~Wiel}},\
  }\bibfield  {title} {\bibinfo {title} {Informing nuclear physics via machine
  learning methods with differential and integral experiments},\ }\href
  {https://doi.org/10.1103/PhysRevC.104.034611} {\bibfield  {journal} {\bibinfo
   {journal} {Phys. Rev. C}\ }\textbf {\bibinfo {volume} {104}},\ \bibinfo
  {pages} {034611} (\bibinfo {year} {2021})}\BibitemShut {NoStop}%
\bibitem [{\citenamefont {Goriely}\ \emph {et~al.}(2009)\citenamefont
  {Goriely}, \citenamefont {Hilaire}, \citenamefont {Koning}, \citenamefont
  {Sin},\ and\ \citenamefont {Capote}}]{Goriely2009}%
  \BibitemOpen
  \bibfield  {author} {\bibinfo {author} {\bibfnamefont {S.}~\bibnamefont
  {Goriely}}, \bibinfo {author} {\bibfnamefont {S.}~\bibnamefont {Hilaire}},
  \bibinfo {author} {\bibfnamefont {A.~J.}\ \bibnamefont {Koning}}, \bibinfo
  {author} {\bibfnamefont {M.}~\bibnamefont {Sin}},\ and\ \bibinfo {author}
  {\bibfnamefont {R.}~\bibnamefont {Capote}},\ }\bibfield  {title} {\bibinfo
  {title} {Towards a prediction of fission cross sections on the basis of
  microscopic nuclear inputs},\ }\href
  {https://doi.org/10.1103/PhysRevC.79.024612} {\bibfield  {journal} {\bibinfo
  {journal} {Phys. Rev. C}\ }\textbf {\bibinfo {volume} {79}},\ \bibinfo
  {pages} {024612} (\bibinfo {year} {2009})}\BibitemShut {NoStop}%
\bibitem [{\citenamefont {M\"oller}\ and\ \citenamefont
  {Randrup}(2015)}]{Moller2015}%
  \BibitemOpen
  \bibfield  {author} {\bibinfo {author} {\bibfnamefont {P.}~\bibnamefont
  {M\"oller}}\ and\ \bibinfo {author} {\bibfnamefont {J.}~\bibnamefont
  {Randrup}},\ }\bibfield  {title} {\bibinfo {title} {Calculated
  fission-fragment yield systematics in the region
  $74\ensuremath{\le}{Z}\ensuremath{\le}94$ and
  $90\ensuremath{\le}{N}\ensuremath{\le}150$},\ }\href
  {https://doi.org/10.1103/PhysRevC.91.044316} {\bibfield  {journal} {\bibinfo
  {journal} {Phys. Rev. C}\ }\textbf {\bibinfo {volume} {91}},\ \bibinfo
  {pages} {044316} (\bibinfo {year} {2015})}\BibitemShut {NoStop}%
\bibitem [{\citenamefont {Verriere}\ and\ \citenamefont
  {Mumpower}(2021)}]{Verriere2021}%
  \BibitemOpen
  \bibfield  {author} {\bibinfo {author} {\bibfnamefont {M.}~\bibnamefont
  {Verriere}}\ and\ \bibinfo {author} {\bibfnamefont {M.~R.}\ \bibnamefont
  {Mumpower}},\ }\bibfield  {title} {\bibinfo {title} {Improvements to the
  macroscopic-microscopic approach of nuclear fission},\ }\href
  {https://doi.org/10.1103/PhysRevC.103.034617} {\bibfield  {journal} {\bibinfo
   {journal} {Phys. Rev. C}\ }\textbf {\bibinfo {volume} {103}},\ \bibinfo
  {pages} {034617} (\bibinfo {year} {2021})}\BibitemShut {NoStop}%
\bibitem [{\citenamefont {Jachimowicz}\ \emph {et~al.}(2021)\citenamefont
  {Jachimowicz}, \citenamefont {Kowal},\ and\ \citenamefont
  {Skalski}}]{Jachimowicz2021}%
  \BibitemOpen
  \bibfield  {author} {\bibinfo {author} {\bibfnamefont {P.}~\bibnamefont
  {Jachimowicz}}, \bibinfo {author} {\bibfnamefont {M.}~\bibnamefont {Kowal}},\
  and\ \bibinfo {author} {\bibfnamefont {J.}~\bibnamefont {Skalski}},\
  }\bibfield  {title} {\bibinfo {title} {Properties of heaviest nuclei with
  {$98 \ensuremath{\le} Z \ensuremath{\le} 126$} and {$134 \ensuremath{\le} N
  \ensuremath{\le} 192$}},\ }\href {https://doi.org/10.1016/j.adt.2020.101393}
  {\bibfield  {journal} {\bibinfo  {journal} {Atomic Data and Nuclear Data
  Tables}\ }\textbf {\bibinfo {volume} {138}},\ \bibinfo {pages} {101393}
  (\bibinfo {year} {2021})}\BibitemShut {NoStop}%
\bibitem [{\citenamefont {Dubray}\ and\ \citenamefont
  {Regnier}(2012)}]{Dubray2012}%
  \BibitemOpen
  \bibfield  {author} {\bibinfo {author} {\bibfnamefont {N.}~\bibnamefont
  {Dubray}}\ and\ \bibinfo {author} {\bibfnamefont {D.}~\bibnamefont
  {Regnier}},\ }\bibfield  {title} {\bibinfo {title} {Numerical search of
  discontinuities in self-consistent potential energy surfaces},\ }\href
  {https://doi.org/10.1016/j.cpc.2012.05.001} {\bibfield  {journal} {\bibinfo
  {journal} {Computer Physics Communications}\ }\textbf {\bibinfo {volume}
  {183}},\ \bibinfo {pages} {2035 } (\bibinfo {year} {2012})}\BibitemShut
  {NoStop}%
\end{thebibliography}%


\end{document}
