No evidence of a localised excess is visible near the anticipated
Higgs mass $\mH=125.09~\GeV$, 
as shown in Figure~\ref{fig:mllg-low-mass} where the invariant
mass distributions $m_{Z\gamma}$ for the individual categories of the $H\to Z\gamma$ search are
displayed with the background-only fit performed in the range of $115~\GeV < m_{Z\gamma} < 150$~\GeV.
At $\mH = 125.09$~\GeV, the observed $p$-value is \lowmassmHpz\ under the background-only hypothesis, 
in which the dominant contribution
comes from the $\mu\mu$ low \ptt category. The $p$-value corresponds to a local significance of 
\lowmassmHsignificance. 
The expected $p$-value for a SM Higgs boson at $\mH=125.09~\GeV$ is \lowmassmHpzExpSMHiggs, corresponding
to a significance of \lowmassmHsignificanceExpSMHiggs.
The observed 95\% CL limit on \sighbr~ 
is found to be \mhObserved~times the SM prediction, corresponding to
the limit of \mhObservedxsBr. Assuming SM Higgs boson production, the upper limit on
the branching ratio of the Higgs boson decay to $Z\gamma$ is found to be \mhObservedBr.
The expected 95\% CL limit on \sighbr~ assuming no (a SM) Higgs boson decay to $Z\gamma$ is 
\mhExpected\ (\mhExpectedSMHiggs) times the SM prediction.

\begin{figure}
\begin{center}
\subfigure[]{\includegraphics[width=0.375\textwidth]{H2Zy_ZyFast_VBF_BDTG_OS_all_MASS3_data_mc_comparison3.pdf}}%
\subfigure[]{\includegraphics[width=0.375\textwidth]{H2Zy_ZyFast_high_relative_pt_OS_all_MASS3_data_mc_comparison3.pdf}}
\subfigure[]{\includegraphics[width=0.375\textwidth]{H2Zy_ZyFast_high_pTt_OS_ee_MASS3_data_mc_comparison3.pdf}}%
\subfigure[]{\includegraphics[width=0.375\textwidth]{H2Zy_ZyFast_low_pTt_OS_ee_MASS3_data_mc_comparison3.pdf}}
\subfigure[]{\includegraphics[width=0.375\textwidth]{H2Zy_ZyFast_high_pTt_OS_mumu_MASS3_data_mc_comparison3.pdf}}%
\subfigure[]{\includegraphics[width=0.375\textwidth]{H2Zy_ZyFast_low_pTt_OS_mumu_MASS3_data_mc_comparison3.pdf}}
\end{center}
\caption{The invariant $Z\gamma$ mass ($m_{Z\gamma}$) distributions of events satisfying the 
$H\to Z\gamma$ selection in data for the six event categories:
(a) VBF-enriched, (b) high \ptg, (c) $ee$ high \ptt,  
(d) $ee$ low \ptt, (e) $\mu\mu$ high \ptt, and (f) $\mu\mu$ low \ptt.
The points represent the data and the statistical uncertainty. The solid lines show the 
background-only fits to the data, performed independently in each category.
The dashed histogram corresponds to the expected signal
for a SM Higgs boson with $\mH = 125$~\GeV\ 
decaying to $Z\gamma$ with a rate 20 times the SM prediction.
The bottom part of the figures shows the residuals of the data with respect to the background-only fit.
}
\label{fig:mllg-low-mass}
\end{figure}


The invariant mass distributions of events satisfying the 
high-mass selection are
displayed for both categories ($ee$ and $\mu\mu$) in Figure~\ref{fig:mllg-high-mass} and compared to the
background-only fit performed in the fit range $200~\GeV < m_{Z\gamma} < 2500$ \GeV. 
The highest masses measured in the \eeg\ and \mmg\ final states are
1.47~\TeV\ and 1.57~\TeV, respectively.
In the fit range, no significant excess is 
observed with respect to the background-only hypothesis. 
The largest deviation is observed around $\mX = \highmassresonance$ \GeV\
corresponding to a local significance of $\highmasshighestsignificance$.
The global significance, evaluated using the search region of 
[250--2400] \GeV\ in mass, is found to be $\highmasshighestsignificanceglobal$.
The observed and expected upper limits on \sigxbr~as a function of \mX\ are shown in Figure~\ref{fig:limit-high-mass}.
The observed limits vary between \highmasslimitlowmX\ 
and \highmasslimithighmX~for the mass range from 250~\GeV\ to 2.4~\TeV\
at the 95\% CL for a spin-0 resonance produced via gluon--gluon fusion, 
while the expected limits range from 
\highmasslimitlowmXexp\
to \highmasslimithighmXexp\ in the same mass range. For a spin-0 resonance produced via vector-boson fusion, the
limits are up to 4\% lower due to the slightly larger efficiency for this production process.


\begin{figure}
\begin{center}
\subfigure[]{\includegraphics[width=.50\textwidth]{profile_c0.pdf}}%
\subfigure[]{\includegraphics[width=.50\textwidth]{profile_c1.pdf}}
\end{center}
\caption{The invariant $Z\gamma$ mass ($m_{Z\gamma}$) distributions of events satisfying the 
high-mass selection in data for the two event categories:
(a) $ee$ and (b) $\mu\mu$.
The points represent the data and the statistical uncertainty. The solid lines show the
background-only fit to the data, performed independently in each category. 
The bottom part of the figures shows the significance, here defined as the residual of the data with 
respect to 
the background-only fit divided by the statistical uncertainty of the data.}
\label{fig:mllg-high-mass}
\end{figure}


\begin{figure}
\begin{center}
\includegraphics[width=.65\textwidth]{spinzero_limitplot.pdf}
\end{center}
\caption{The observed (solid line) and expected (dashed line) upper limit derived at the 95\% CL
on \sigxbr~at \sqtt~as a function of the high-mass spin-0 resonance's mass, 
assuming production via gluon--gluon fusion and using the narrow width assumption (NWA).
For $m_X> 1.6$~\TeV\ results are derived from ensemble tests in addition to the results obtained 
using closed-form asymptotic formulae.
The shaded regions correspond to the $\pm 1$ and $\pm 2$ standard deviation bands for the expected 
exclusion limit derived using asymptotic formulae.}
\label{fig:limit-high-mass}
\end{figure}



The results are also interpreted in terms of spin-2 resonances. Observed and expected upper 
limits at the 95\% CL
on \sigxbr~are derived and shown in Figure~\ref{fig:limit-spin2-high-mass} for both the $gg$
and $q\bar{q}$ processes. 
The observed limits for the $gg$ ($q\bar{q}$) process
vary between \highmasslowlimitgg~(\highmasslowlimitqq) 
and \highmasshighlimitgg~(\highmasshighlimitqq) for the mass range from 250~\GeV\ to 2.4~\TeV,
while the expected limits range between \highmasslowlimitggexp~(\highmasslowlimitqqexp)
and \highmasshighlimitggexp~(\highmasshighlimitqqexp) in the same mass range. 


\begin{figure}
\begin{center}
\subfigure[]{\includegraphics[width=.50\textwidth]{spin2gg_limitplot.pdf}}%
\subfigure[]{\includegraphics[width=.50\textwidth]{spin2qq_limitplot.pdf}}
\end{center}
\caption{The observed (solid line) and expected (dashed line) upper limit derived at the 95\% CL
on \sigxbr~at \sqtt~as a function of the spin-2
resonance mass produced via (a) gluon--gluon initial states and 
(b) $q\bar{q}$ initial states modelled using the Higgs Characterisation Model (HCM), using
the narrow width assumption (NWA).
For $m_X> 1.6$~\TeV\ results are derived from ensemble tests in addition to the results obtained 
using closed-form asymptotic formulae.
The shaded regions correspond to the $\pm 1$ and $\pm 2$ standard deviation 
bands for the expected exclusion limit derived using asymptotic formulae.}
\label{fig:limit-spin2-high-mass}
\end{figure}

The limits on \sigxbr\ for high-mass resonances are valid for resonances with a natural width that is
small compared to the detector resolution.
