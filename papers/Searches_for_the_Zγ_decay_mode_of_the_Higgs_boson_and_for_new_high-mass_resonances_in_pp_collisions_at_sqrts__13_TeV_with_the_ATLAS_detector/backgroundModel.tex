\subsection{Background modelling}
\label{sec:bckgModel}
The background is mainly composed of non-resonant production of a $Z$ boson in association with
a photon (irreducible background), and of inclusive $Z+$jets events where a jet is misidentified as 
a photon (reducible background), and the relative contributions are determined using data as described
below. Contributions from other sources, such as $t\bar{t}$ production, $W/Z$ events, and, 
for the
$H\to Z\gamma$ search, from other Higgs 
boson decays are expected to be negligible based on studies of simulated events. 
The background exhibits a smoothly falling distribution 
as a function of the invariant mass of the candidate $Z$ boson and photon, \mzg. 

The estimated background composition is used to construct simulated background
samples with the same composition as the data background.
These samples are used in the optimisation
of the selection criteria, the choice of analytical model of the
background shape, and the estimation of the related systematic uncertainties.
The searches rely 
only indirectly on the measured background composition 
since the background shape parameters are determined from the data.

The composition of the background is estimated using a combined binned fit to the calorimeter isolation 
distribution of the photon candidate in the signal region and in a control region enriched in
$Z$+jets background. In the control region, 
photon candidates are required to fail the tight identification, but to pass a modified
loose identification. It differs from the tight identification by removing the requirements on four 
out of nine 
shower shape variables which are the least correlated with the calorimeter isolation~\cite{Aad:2010sp}.
The calorimeter isolation distribution for photons and the contribution of true photons to the control 
region are determined from simulation, 
while the calorimeter isolation distribution for jets is determined 
in the fit and assumed to be the same in the signal and 
control regions. This assumption is supported by extensive studies performed in the context
of earlier analyses~\cite{Aad:2010sp}. The composition is determined in the inclusive 
selection for the $H\to Z\gamma$ search and the fraction of $Z$$+$$\gamma$
events is found to be $0.838\pm 0.005\,\mathrm{(stat.)} \pm 0.031\,\mathrm{(syst.)}$. For the 
high-mass resonance search, the fraction of
$Z$$+$$\gamma$ events is found to be $0.916\pm 0.009\,\mathrm{(stat.)}\,^{+0.013}_{-0.019} 
\,\mathrm{(syst.)}$. 
The systematic uncertainties are estimated by
varying the set of shower shape variables that are removed from the tight identification to
define the modified loose identification~\cite{Aad:2010sp}.
The results of the composition estimate are cross-checked with a
two-dimensional sideband technique~\cite{Aad:2010sp} based on the calorimeter isolation of the photon
candidate and whether or not the photon candidate satisfies the tight identification criteria
(when the photon fails the tight identification it is still required to pass the modified
loose identification), which gives consistent results.

The analytical model of the background and the $m_{Z\gamma}$ range used for the final fit are chosen to limit the bias in the extracted signal 
yield, while at the same time limiting the number of free parameters in the fit to avoid degradation
of the sensitivity~\cite{atlas-higgs}. For
each category used in either analysis, the 
bias (also referred to as spurious signal) is estimated as a function of the signal invariant
mass by performing a signal+background fit to a 
$m_{Z\gamma}$ background-only distribution with small statistical fluctuations. 
The background-only distribution is
constructed from the fast simulation of $Z$$+$$\gamma$ events, and the contribution from 
$Z$+jets events is included by reweighting the $Z$$+$$\gamma$ simulated distribution as follows: 
for each category, the shape of the $Z$+jets contribution is
determined in a data control region defined by requiring that the photon candidate fails to satisfy the identification 
and isolation criteria. To smooth statistical fluctuations in the $Z$+jets shape, a first-order
polynomial is fitted to the ratio of the $Z$+jets and $Z$$+$$\gamma$ shapes, and the smoothed $Z+$jets shape 
is constructed from the fit result and the $Z$$+$$\gamma$ shape. The reweighting of the $Z$$+$$\gamma$ 
distribution to take into account the $Z$+jets contribution is determined from the smoothed
$Z$+jets shape. The normalisation of the $Z$+jets contribution is determined from the number of
events obtained when applying the selection and categorisation to the $Z$$+$$\gamma$ and $Z$+jets simulation
samples and the purity of the inclusive sample, measured as described above.
The spurious signal is required to be less than 40\% (20\%) of the expected statistical
uncertainty in the signal yield, which is dominated by the expected statistical uncertainty of 
the background, in the search for Higgs boson (high-mass resonance) decays to 
$Z\gamma$. The looser requirement for the $H\to Z\gamma$ search is chosen to improve the robustness
of the procedure against statistical fluctuations in the simulated $Z$$+$$\gamma$ event sample.
If two or more considered functions satisfy this requirement, the
function with the fewest number of parameters is chosen. 

For the $H\to Z\gamma$ search, the fit range is also optimised on the basis of the spurious signal
estimates, taking into account the spurious signal and the number of parameters of the chosen functions
in all categories. A fit range from $115~\GeV$ to $150~\GeV$ is selected. 
A second-order Bernstein polynomial
is chosen as the parameterisation for the VBF and high relative $\pt$ categories, and a fourth-order
Bernstein polynomial is chosen for the other categories. For the chosen parameterisation, the largest spurious 
signal obtained in a window of 121--129~\GeV\ is
assigned as a systematic uncertainty in each category associated with the choice of background 
function and ranges from
1.7 events in the VBF category to 25 events in the $\mu\mu$ low \ptt\ category.
The choice of background functions is validated by using second- and third-order polynomials for the
smoothing of the $Z+$jets background shape and by varying the $Z$$+$$\gamma$ purity by $\pm 15\%$.
The large variation of the $Z$$+$$\gamma$ purity is chosen to cover the purity differences between
the different categories and intended to also account for the additional uncertainty in the estimation
of the $Z$+jets invariant mass distribution.

The high-mass resonance search considers as a background model a class of functions~\cite{Aaboud:2016trl} given by

\beq
f_\mathrm{bkg}^k(x; b, {a_k}) = N (1-x^{1/3})^{b}x^{\sum_{j=0}^k a_k \log(x)^j},
\label{eqn:background}
\eeq

where $x= m_{Z\gamma}/\sqrt{s}$, $N$ is a normalisation factor, $k$ determines the number of terms considered in the exponent, and
$b$ and $a_k$ are determined by the fit. 
When testing on the background-only distribution constructed using the simulated $Z$$+$$\gamma$ 
sample taking into account $Z$+jets contributions as discussed before, the spurious signal
criterion is found to be satisfied for the full mass range for $k=0$.
The spurious signal used as an estimate of the systematic uncertainty is parameterised as a smooth function
of the invariant mass. It ranges from 3.6 (6.1) events at 250~\GeV\ to 0.01 (0.005) events at 2.4~\TeV\ for the
$Z\to\mu\mu$ ($Z\to ee$) channel.
The choice of analytic function for the background shape is validated by
using second- and third-order polynomials for the
smoothing of the $Z$+jets background shape, by varying the $Z$$+$$\gamma$ purity by $\pm 5\%$ (motivated by the
range of purities estimated in the two categories), and
by varying the PDFs in the $Z$$+$$\gamma$ simulation 
using the uncertainties associated with the different eigenvectors of the PDF set.

The possibility of needing higher-order functions when fitting
the selected analytical function to the data $m_{Z\gamma}$ distribution
is further investigated by an $F$-test. The test statistic $F$ defined as

\beq
F = \left(\frac{\chi^2_0 - \chi^2_1}{p_1-p_0}\right)/\left(\frac{\chi_1^2}{n-p_1}\right),
\eeq

compares the fit qualities between less and more complex functions.
The $\chi^2_0$ ($\chi^2_1$) is the $\chi^2$ value of a binned fit with the less (more) complex parameterisation,
$p_k$ is the number of free parameters of each fit, and $n$ is the number of bins of the invariant 
mass distribution. Should the probability to find values of $F$ more extreme than the one measured on data 
be less than 5\%, the less complex parameterisation would be rejected in favour of the more complex
parameterisation. 
The binning for the $F$-test is chosen to guarantee a sufficient number of events in
each bin.
For the $H\to Z\gamma$ search, the test is carried out to determine if there is
any indication that a higher-order Bernstein polynomial is required. 
It does not lead to a change in the chosen parameterisation.
For the high-mass search, the test is
performed to determine whether or not the quality of the fit to data is improved significantly if 
using $k=1$. 
The test confirms that the choice of $k=0$ is adequate. The $\chi^2$ per degree of freedom is 1.2 for 30
degrees of freedom (1.1 for 17 degrees of freedom) or better for the chosen parameterisations 
for the $H\to Z\gamma$ (high-mass resonance) search. 

