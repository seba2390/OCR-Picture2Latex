A profile-likelihood-ratio test statistic~\cite{stat} is used to search for
a localised excess over a smoothly falling background in the $m_{Z\gamma}$ distribution
of the data, as well as to quantify its significance and estimate either its
production cross section or signal strength.

The extended unbinned likelihood function $\mathcal{L}(\alpha, \boldsymbol{\theta})$ is given by the
product of a Poisson term, constructed from the number of observed events, $n$,
the expected event yield, $N$, and the probability density function
 of the invariant mass distribution for each candidate event $i$, 
$f_\mathrm{tot}(m_{Z\gamma}^{i}, \alpha,\boldsymbol{\theta})$~\cite{Aaboud:2016trl}:

\beq
\LL\left((\alpha, \boldsymbol{\theta})\Big|\{m_{Z\gamma}^i\}_{i=1..n}\right) = \frac{\mathrm{e}^{-N(\alpha, \boldsymbol{\theta})}N^n(\alpha, \boldsymbol{\theta})}{n!} \prod\limits_{i=1}^nf_\mathrm{tot}(m_{Z\gamma}^i,\alpha,\boldsymbol{\theta}) \times G(\boldsymbol{\theta}),
\eeq
where $\alpha$ represents the parameter of interest and 
$\boldsymbol{\theta}$ are the nuisance parameters. The function $G(\boldsymbol{\theta})$ 
represents the prior constraints on the nuisance parameters.
The expected event yield $N$ is the sum 
of the expected number of signal ($N_{\mathrm{sig}}$), background ($N_{\mathrm{bkg}}$), and
spurious signal ($\boldsymbol{N_\mathrm{spur}}\cdot\boldsymbol{\theta_\mathrm{spur}}$) events, where
$\boldsymbol{\theta_\mathrm{spur}}$ is the nuisance parameter associated with the spurious signal.
For the high-mass resonance search, the parameter of interest is $\alpha=\sigxbr$. 
The $H\to Z\gamma$ search is performed to extract several parameters of interest:
the signal strength $\mu = \sighbr / (\sighbrsm)$,
\sighbr, and \brhzg~assuming $\sigma(pp\to H)_\mathrm{SM}$. 
The signal strength $\mu$ is related to the number of signal events by
$N_{\mathrm{sig}} = L_\mathrm{tot} \times \mu \times (\sighbrsm) \times \varepsilon$, where 
$L_\mathrm{tot}$ is the total integrated luminosity, $\varepsilon$
is the signal efficiency, and  \sighbrsm~is predicted by the SM.
The theoretical uncertainties are
taken into account as described in Section~\ref{sec:sys}.

The probability density function for the invariant mass
($f_\mathrm{tot}(m_{Z\gamma}^{i}, \alpha,\boldsymbol{\theta})$) is built from the probablity 
density functions $f_\mathrm{sig}$ and $f_\mathrm{bkg}$
describing the signal and background invariant mass distributions, respectively:
%

\begin{multline}
f_\mathrm{tot}(m_{Z\gamma}^i,\alpha,\boldsymbol{\theta}) =
\frac{1}{N} \sum_{c} \left\{ \left[ N_{\mathrm{sig}}^{(c)}(\alpha, \boldsymbol{\theta_\mathrm{sig}}) + N_{\mathrm{spur}}^{(c)} \cdot \theta_{\mathrm{spur}}^{(c)} \right] \times f_{\mathrm{sig}}^{(c)}(m_{Z\gamma}^i,\boldsymbol{\theta_\mathrm{sig}}) \right. \\
\left.  + N_{\mathrm{bkg}}^{(c)} \times f_{\mathrm{bkg}}^{(c)}(m_{Z\gamma}^i,\boldsymbol{\theta_\mathrm{bkg}})\right\}.
\end{multline}
%
The index $c$ indicates the category. The $\boldsymbol{\theta_\mathrm{bkg}}$ are nuisance
parameters that determine the shape of the background.
The nuisance parameters associated with the uncertainties
in the signal parameterisation, efficiency and acceptance are denoted by 
$\boldsymbol{\theta_\mathrm{sig}}$. 
Nuisance parameters associated with uncertainties in the event yield or the
 $m_{Z\gamma}$ resolution are assigned log-normal probability density functions, while nuisance 
parameters
associated with the $m_{Z\gamma}$ signal peak position are assigned Gaussian probability 
density functions.  
The nuisance parameters associated 
with the spurious signal, $\boldsymbol{\theta_\mathrm{spur}}$, are assigned 
Gaussian probability density functions. 

The probability that the background can produce a fluctuation greater than or equal to an
excess observed in data 
is quantified by the $p$-value of the $\alpha = 0$ hypothesis,
$p_{0}$, which can also be expressed in terms of number of Gaussian standard deviations and 
provides an estimate of the local significance of a possible deviation from the expected
background. The global significance, corrected for the effect that a deviation can occur
anywhere in the search region, is estimated taking into account the trial 
factors~\cite{Gross:2010qma}.
The compatibility between the data and increasing
non-zero values of $\alpha$ is used to set upper limits at the 95\% CL on
\sighxbr, \brhzg, and the signal strength for $pp\to H\to Z\gamma$, respectively, 
using a modified frequentist 
(CL$_\mathrm{s}$) method~\cite{cls},
by identifying the value $\alpha$ for which the value of CL$_\mathrm{s}$ is equal to 0.05.

The results are derived using closed-form asymptotic formulae~\cite{stat} for masses up to
1.6~\TeV. Due to
the small number of events at large $m_{Z\gamma}$ in the high-mass resonance search, 
the results for $m_X> 1.6$~\TeV\ are derived using ensemble tests. 
The expected cross-section limits obtained from the
asymptotic formulae agree to better than 10\% with those obtained from the ensemble tests
up to $m_X$ of 1.7~\TeV.
At high $m_X$, the observed (expected) central values are underestimated by 35\% (23\%) in the 
asymptotic approach. 

