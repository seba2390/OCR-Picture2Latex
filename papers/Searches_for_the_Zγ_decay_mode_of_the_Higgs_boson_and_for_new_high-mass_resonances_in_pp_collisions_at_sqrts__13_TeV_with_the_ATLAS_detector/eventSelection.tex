\subsection{Event preselection}

Events are required to have at least one primary vertex candidate, determined using the tracks
with transverse momentum $\pt > 400$~\MeV\ 
reconstructed in the ID. The primary vertex candidate with the largest sum of the
squared transverse momenta of the associated tracks ($\sum\pt^{2}$) is considered to be the primary
vertex of the interaction of interest. For $H/X\to Z\gamma$ signal events, the selected primary vertex 
is within 0.3~mm of the true primary interaction vertex for more than 99\% of the events.

The $H/X\to Z(\to\ell\ell)\gamma$ candidate events are selected by requiring two same-flavour 
opposite-charge leptons 
($\ell = e, \mu$) to form a $Z$ boson candidate and at least one photon candidate.

Muon candidates with $|\eta|<2.5$ are reconstructed by combining tracks in the ID with tracks in 
the MS. To extend the acceptance beyond that of the ID, muon candidates are
reconstructed from tracks reconstructed only in the MS up to $|\eta| = 2.7$~\cite{Aad:2016jkr}. 
Muon candidates are required to satisfy the \textit{medium} criterion and have $\pt > 10$~\GeV.
In order to ensure good track quality, the ID tracks associated with muons in $|\eta|<2.5$ are 
required to have at least one hit in the silicon pixel detector and at least five hits in the silicon 
microstrip detector, as well as to extend into the TRT for $0.1<|\eta|<1.9$.
The muon candidates in
$2.5 < |\eta| <2.7$ are required to have hits
in each of the three layers of MS tracking chambers.

Electron candidates are reconstructed from a cluster of energy deposits in neighbouring cells of
the EM calorimeter and a track, matched to the cluster, in the ID. 
They are required to have $\pt > 10$~\GeV\ and be within the fiducial region $|\eta|<2.47$
excluding the candidates in the transition region between the barrel and endcap EM calorimeters,
$1.37 < |\eta| < 1.52$. The electrons are identified with the \textit{medium} likelihood-based 
criterion~\cite{ATLAS-electrons} built from the shower
shapes of the clusters, the number of hits associated with the track in the ID and the quality of the track-cluster 
matching. 

Both the muon and electron candidates are required to be associated with the primary vertex
by requiring the longitudinal impact parameter, $\Delta\zzero$, computed with respect to the
primary vertex position along the beam-line,
to satisfy $\zzsint < 0.5$~mm, where $\theta$ is the polar angle of the track.
In addition the significance of the
transverse impact parameter $\dzero$ calculated with respect to the measured beam-line position must satisfy
$|\dzero| / \sigma_{\dzero} < 3~(5)$ for muons (electrons) where $\sigma_{\dzero}$ is the 
uncertainty in $\dzero$. 

The efficiency of the muon identification is higher than 99\% (60\%) for $\pt> 10$~\GeV\ muons with 
$|\eta|>0.1$ ($|\eta|<0.1$)~(similar to Ref.~\cite{Aad:2016jkr}), while
the efficiency of the electron identification ranges from about 80\% for electrons with 
$\pt$ = 10~\GeV\ to higher than 90\% for electrons with $\pt >$ 50~\GeV\
(similar to Ref.~\cite{ATLAS-electrons}). 
The efficiency is typically
about 5\% higher in the barrel region of the detector than in the endcaps.

The lepton candidates are further required to satisfy additional criteria for track isolation, 
which is defined similarly to the track isolation used in the trigger (see Section~\ref{sec:samples}),
but uses a different track selection and 
a different cone size in some cases.
The track isolation is computed
as the scalar sum of the transverse momenta of all tracks in a cone around the 
lepton candidate with $\pt > 1$~\GeV\ which satisfy loose track-quality criteria and originate from the 
selected primary vertex, excluding the track associated with the lepton candidate. For muon candidates, 
the cone size is chosen to be $\Delta R = 0.3$ for $\pt < 33.3$~\GeV\ and 
$\Delta R = 10 / (\pt/\GeV)$ for $\pt > 33.3$~\GeV. For electron candidates,
the cone size is chosen to be $\Delta R = 0.2$ for $\pt < 50$~\GeV\ and 
$\Delta R = 10 / (\pt/\GeV)$ for $\pt > 50$~\GeV. The requirement
on the track isolation is chosen such that it is 99\% efficient over the full lepton $\pt$ range.

An overlap removal procedure is applied to the selected lepton candidates. 
If two electrons share the same track, or the two electron clusters 
satisfy $|\Delta\eta| < 0.075$ and $|\Delta\phi| < 0.125$, then only the highest-$\pt$ electron
is retained. Electron candidates that are within 
$\Delta R = 0.02$ of a selected muon candidate are also discarded.

Photon candidates are reconstructed from energy clusters in the electromagnetic 
calorimeter. 
Clusters matched to a conversion vertex, reconstructed from either two tracks consistent with a 
vertex originating from a photon conversion or one track that does not have any hits in the innermost 
pixel layer and has an electron-like response in the TRT, are reconstructed as converted photon candidates. 
Clusters without any matching track (clusters with a matching track are reconstructed as electrons
as described above) or conversion vertex are reconstructed as unconverted photon 
candidates~\cite{Aaboud:2016yuq}.
Photon candidates are required to have $\pt > 10$~\GeV\ and $|\eta|<1.37$ or 
$1.52<|\eta|<2.37$. 
The identification of photon candidates is based on the lateral and longitudinal shape of the 
electromagnetic shower~\cite{Aaboud:2016yuq,ATLAS-photonid}. A \textit{loose} identification is used for preselection
and for background studies.

In order to suppress the events arising from FSR processes 
and $H\to\ell\ell^*\to\ell\ell\gamma$ decays,
photon candidates within a $\Delta R = 0.3$ cone around a selected
electron or muon candidate are rejected.

The selection
criteria described in the preceding paragraphs
define the event preselection for the leptons and photons included
in the reconstruction of the
invariant mass of the $\ell\ell$ and $\ell\ell\gamma$ systems.

The event categorisation described in Section~\ref{sec:categorisation} used in the 
search for decays of the Higgs boson to $Z\gamma$
makes use of hadronic jets produced in association with the Higgs boson 
candidate.
Jets are reconstructed using the anti-$k_t$ algorithm~\cite{Cacciari:2008gp} with a radius parameter of
0.4 with three-dimensional topological clusters as input~\cite{Aad:2016upy}. Jets are corrected on an event-by-event
basis for soft energy deposits originating from pile-up interactions~\cite{ATLAS-jetpucorr} and 
calibrated using a combination of simulation- and data-driven correction factors accounting
for the non-compensating response of the calorimeter and energy loss in inactive regions~\cite{Aaboud:2017jcu}.
Jets are required to have a transverse momentum larger than 25~\GeV\ and $|\eta|<4.4$. 
To reduce the contamination from jets produced in pile-up interactions, jets with transverse momentum smaller 
than 60~\GeV\ and contained within the inner detector's acceptance ($|\eta|<2.4$) are required to
pass a selection based on the jet vertex tagging algorithm~\cite{ATLAS-JVT}, which is 92\% efficient
for jets originating from the hard interaction. The jet vertex tagging algorithm is 
based on the tracks associated with the jet which are consistent with originating from the selected primary vertex.
Jet--lepton and jet--photon overlap removal is performed where the jet is removed if the lepton or photon is 
within a cone of size $\Delta R$ = 0.2.


\subsection{Reconstruction of $Z$ candidates and $H/X$ candidates and final selection}
\label{sec:ZHXreco}

The $Z$ boson candidates are reconstructed from two same-flavour opposite-sign leptons satisfying 
the preselection criteria and with an invariant mass $m_{\ell\ell}$ larger than 45~\GeV. 
Leptons are required to be consistent with the objects that triggered the 
event. Trigger efficiency turn-on effects are mitigated by transverse momentum requirements on the leptons
that fired the single-lepton or dilepton trigger. If the event was triggered by a single-lepton trigger, the
transverse momentum is required to be at least 27~\GeV\ for the leading lepton, and at
least 1~\GeV\ higher than the respective trigger threshold in cases where the event was
triggered by one of the higher-threshold triggers. If the event was triggered by a
dilepton trigger, the transverse momentum is required to be 
at least 24~\GeV\ (18~\GeV) for the 
leading muon (electron) and 10~\GeV\ (18~\GeV) for the subleading muon (electron).
For $Z\to\mu\mu$ candidates with an invariant mass between 66 and 89~\GeV, 
the invariant mass resolution of the $Z$ boson candidate
is improved by correcting the muon momenta for collinear FSR by including 
any reconstructed electromagnetic cluster with \pt above 1.5~\GeV\ lying close
to a muon track (with $\Delta R < 0.15$) if the corrected invariant mass is below 
100~\GeV~\cite{Aad:2014aba}.
A constrained kinematic fit is applied to recompute the 
four-momenta of the dilepton pair~\cite{Aad:2014eva} for $Z\to\mu\mu$ and $Z\to ee$ candidates. 
The fit models the lepton energy and momentum
response as a Gaussian distribution for each lepton, and the Gaussian width is given by the expected resolution.
The $Z$ lineshape is used as a constraint with the approximation of the leptons being massless. The
lineshape is modelled by a Breit--Wigner distribution.
After the application of the FSR corrections and the kinematic fit, $Z$ boson candidates are required to have 
an invariant mass within 15~\GeV\
of the $Z$ boson mass, $m_Z = 91.2~\GeV$~\cite{Olive:2016xmw}. If multiple $Z$ boson candidates pass all
requirements, the candidate with the mass closest to the $Z$ boson mass is chosen.
About 0.2\% (0.5\%) of events that pass the final $H\to Z\gamma$ ($X\to Z\gamma$) selection 
have more than one $Z$ boson candidate within the
15~\GeV\ mass window. 

\begin{figure}[b]
\subfigure[]{\includegraphics[width=.50\textwidth]{Zmasscosntraint_125_mm.pdf}}%
\subfigure[]{\includegraphics[width=.50\textwidth]{Zmasscosntraint_125_ee.pdf}}
\caption{Invariant mass distribution, $m_{Z\gamma}$, for the final selection 
before and after application of the final-state radiation corrections ($Z\to\mu\mu$ only) 
and the $Z$ boson mass 
constrained kinematic 
fit for simulated $H\to Z\gamma$ events with $m_H = 125~\GeV$ in the gluon--gluon fusion production mode. 
Events are separated by lepton type, (a) $Z\to\mu\mu$ and (b) $Z\to ee$.}
\label{fig:mllgamma}
\end{figure}


Higgs boson and $X$ candidates are reconstructed from the $Z$ boson candidate and the highest-$\pt$
photon candidate after the preselection.

For the main analyses with the exception of the background studies, the photon candidate used for the 
reconstruction of the $H/X$ candidate is required to pass the \textit{tight} identification~\cite{Aaboud:2016yuq}. 
The efficiency of the tight identification ranges from 67\% (60\%) to 90\% (95\%)
for unconverted (converted) isolated photons from $\pt$ of 15~\GeV\ to 50~\GeV\ and larger.

Photon candidates are furthermore required 
to be isolated from additional activity in the detector. A combined requirement
on the isolation energy in the calorimeter and the inner detector is used. The calorimeter isolation is
computed as the sum of transverse energies of positive-energy topological 
clusters~\cite{Aad:2016upy} in the calorimeter within a cone of $\Delta R =  0.2$ centred around 
the photon shower barycentre. The transverse energy of the photon
candidate is removed and the contributions of the underlying event and pile-up are subtracted based on
the method suggested in Ref.~\cite{Cacciari:2009dp}. The track isolation for a cone size of 
$\Delta R =0.2$ is used and for converted photons the tracks associated with the
conversion are removed. The calorimeter (track) isolation
is required to be less than 6.5\% (5\%) of the photon \pt.
The efficiency of the isolation requirement for photons satisfying the tight identification criteria 
ranges from approximately 60\% for $\pt$ of 15~\GeV\ to more than 90\% for $\pt$ of 40~\GeV\ 
and larger.

For the $H\to Z\gamma$ ($X\to Z\gamma$) search, the photon transverse momentum requirement is tightened to 15~\GeV\
($\pt / m_{Z\gamma} > 0.3$).
  
The invariant mass of the final-state 
particles, $m_{Z\gamma}$, is 
required to satisfy $115~\GeV < m_{Z\gamma} < 170$~\GeV\ for the $H\to Z\gamma$ search 
and $200~\GeV < m_{Z\gamma} < 2500$~\GeV\ for the high-mass resonance search.
Figure~\ref{fig:mllgamma} shows the invariant 
mass distribution for simulated $H\to Z\gamma$ candidates after the final selection
 with and without the lepton momentum 
corrections from the FSR recovery and the kinematic fit. 
Improvements of the \mmmg\ resolution of 3\% are observed for $m_H=125$~\GeV\ from the FSR
recovery.
The kinematic fit improves the \mmmg\ (\meeg) resolution by 7\% (13\%) at the same mass.
For high invariant masses, the \mmmg\ resolution improvement varies from 10\% at 
$\mX=300~\GeV$ to about 50\% for $\mX > 1.5~\TeV$, while the \meeg\ resolution is
improved by 9\% at $\mX=300~\GeV$ and by 3\% or less above $\mX = 500~\GeV$.
The constrained kinematic fit is particularly effective at large
\mX\ for the $Z\to\mu\mu$ final state due to the decreasing precision of the momentum 
measurement for increasing muon \pt.


\subsection{Categorisation}
\label{sec:categorisation}

Events are split into mutually exclusive event categories that are optimised to improve the sensitivity of 
both the $H\to Z\gamma$ and $X\to Z\gamma$ searches.
The event categories separate events on the basis of the expected signal-to-background
ratio and of the expected three-body invariant mass resolution. Different categories
are used in the search for decays of the Higgs boson to $Z\gamma$ 
and the search for high-mass resonances. 


The $H\to Z\gamma$ search uses six exclusive event categories and events are assigned to the categories in
the following order:

\begin{itemize}
\item   \textbf{VBF-enriched} : Events are required to have at least two jets. 
  A boosted decision tree (BDT) that was trained to separate VBF 
  events from other Higgs boson production modes and non-Higgs backgrounds is applied. It uses six kinematic
  variables as input, computed from the $Z$ boson candidate, the photon candidate and the two jets with
  the largest transverse momenta: 
  \begin{itemize}
  \item The invariant mass of the two jets ($m_{jj}$),
  \item The separation of the jets
  in pseudorapidity ($\Delta\eta_{jj}$),
  \item The azimuthal separation of the $Z\gamma$ and the dijet systems
  ($\Delta\phi_{Z\gamma,jj}$),
  \item The component of the transverse momentum of the $Z\gamma$ system 
  that is perpendicular to the difference of the 3-momenta of the $Z$ boson and the photon
  candidate ($\ptt = 2{|p_{x}^{Z} p_{y}^{\gamma} - p_{x}^{\gamma} p_{y}^{Z}|}/{\pT^{Z\gamma}}$),
  \item The smallest $\Delta R$ separation between the $Z$ boson or photon candidate and 
  the two jets ($\Delta R^\mathrm{min}_{Z/\gamma, j}$),
  \item  The difference between the pseudorapidity of the $Z\gamma$ system and the average 
    pseudorapidity of the two jets ($|\eta_{Z\gamma}-(\eta_{j1} + \eta_{j2})/2|$).
  \end{itemize}

  The variable $\ptt$ is strongly correlated with the transverse momentum of the $Z\gamma$ system,
  but has better experimental resolution~\cite{Ackerstaff:1997rc, Vesterinen:2008hx}.
  Any requirement on $\Delta\phi_{Z\gamma,jj}$ effectively vetoes additional jets in the event by 
  restricting the phase space for additional emissions and, to avoid uncontrolled theoretical 
  uncertainties, the BDT does not use shape information for events with $\Delta\phi_{Z\gamma,jj}>2.94$ by
  merging these events into one bin. A minimum value of the BDT output (BDT > 0.82) is required. 
  The expected and
  observed distributions for two input variables, $m_{jj}$ as a typical variable to select
  events with VBF topology and $\Delta\phi_{Z\gamma,jj}$, which serves as an implicit third-jet
  veto, are shown in Figure~\ref{fig:VBFinputs} for
  selected events with at least two jets.
\item \textbf{High relative $\pt$} :
  Events are required to have a high \pt photon, $\pt^\gamma / m_{Z\gamma} > 0.4$.
\item \textbf{$ee$ high $\ptt$} : Events are required to have high $\ptt$ ($\ptt > 40~\GeV$) and a $Z$ boson candidate decay to electrons.
\item \textbf{$ee$ low $\ptt$} : Events are required to have low $\ptt$ ($\ptt < 40~\GeV$) and a $Z$ boson candidate decay to electrons.
\item \textbf{$\mu\mu$ high $\ptt$} : Events are required to have high $\ptt$ ($\ptt > 40~\GeV$) and a $Z$ boson candidate decay to muons.
\item \textbf{$\mu\mu$ low $\ptt$} : Events are required to have low $\ptt$ ($\ptt < 40~\GeV$) and a $Z$ boson candidate decay to muons.
\end{itemize}


\begin{figure}
\subfigure[]{\includegraphics[width=.5\textwidth]{VBF_m_jj.pdf}}%
\subfigure[]{\includegraphics[width=.5\textwidth]{VBF_Dphi_zy_jj_full.pdf}}
\caption{Kinematic variables used in the BDT used to define the VBF-enriched category: 
(a) the invariant mass of the two jets with the highest
transverse momenta, $m_{jj}$ and (b) the azimuthal separation of the $Z\gamma$ and the dijet system,
$\Delta\phi_{Z\gamma, jj}$ for events with at least two jets and 
$115~\GeV < m_{Z\gamma} < 170~\GeV$. 
The observed distribution (normalised to unity) is shown as data points.
The contributions from $Z+\gamma$ events
(obtained from simulation) and the contribution from $Z$+jets 
(obtained from data control regions described in the text) are shown as stacked histograms.
The corresponding expected distributions for Higgs bosons produced via gluon--gluon fusion and vector-boson 
fusion production for $m_H = 125~\GeV$ are shown as open histograms.
The 
$\Delta\phi_{Z\gamma, jj}$ distribution is shown before the suppression of the shape information
for $\Delta\phi_{Z\gamma, jj}>2.94$.}
\label{fig:VBFinputs}
\end{figure}

\begin{table}
\caption{The expected signal efficiency times acceptance, denoted by $\epsilon$, per production mode 
for each category after the 
full event selection, as well as the expected fraction $f$ of each production
process relative to the total signal yield, for simulated
SM Higgs boson production assuming $m_H = 125~\GeV$.
The expected number of signal events per production process is also given.}
\label{tab:cat-eff}
\begin{center}
\begin{tabular}{lSSSSSSSS}
\hline\hline
& \multicolumn{2}{c}{ggF} & \multicolumn{2}{c}{VBF} &\multicolumn{2}{c}{$WH$} & \multicolumn{2}{c}{$ZH$}\\
Category &  $\epsilon$[\%] & $f$[\%] & $\epsilon$[\%] & $f$[\%] & $\epsilon$[\%]  & $f$[\%] & $\epsilon$[\%] & $f$[\%] \\
\hline
VBF-enriched                  & 0.25 & 30.5 & 6.5 & 67.5 & 0.34 & 1.3 & 0.24 & 0.6  \\
High relative $\pt$           & 1.1  & 71.5 & 2.6 & 14.3 & 4.0 & 8.3 & 4.1 & 5.3  \\
$ee$ high $p_{\mathrm{T}t}$     & 1.7 & 80.8 & 2.8 & 11.0 & 3.2 & 4.7 & 3.6 & 3.3  \\
$ee$ low $p_{\mathrm{T}t}$      & 7.1 & 93.2 & 3.6 & 4.1 & 3.7 & 1.5 & 4.2 & 1.1  \\
$\mu\mu$ high $p_{\mathrm{T}t}$   & 2.2 & 80.4 & 3.6 & 11.3 & 4.1 & 4.8 & 4.2 & 3.1  \\
$\mu\mu$ low $p_{\mathrm{T}t}$    & 9.2 & 93.4 & 4.7 & 4.1 & 4.6 & 1.5 & 4.8 & 1.0  \\

\hline                                          
 Total efficiency (\%)   & 21.5 &  & 23.8 &  & 20.2 &  & 21.0 &   \\
\hline                 
Expected events & \multicolumn{2}{c}{35} & \multicolumn{2}{c}{3.3} &\multicolumn{2}{c}{1.0} & \multicolumn{2}{c}{0.7} \\
\hline
\hline
\end{tabular}
\end{center}
\end{table}


For SM $H\to Z(\to\ell\ell)\gamma$ events, the reconstruction and selection efficiency (including
kinematic acceptance) is  21.5\%.
Table~\ref{tab:cat-eff} shows the expected signal efficiency times acceptance for each of the different SM
Higgs boson production processes in each category, as well as the expected relative contribution of
a given production process to each category.
The VBF-enriched category is expected to be about 68\% pure in VBF events. The high relative 
$\pt$
and high $\ptt$ categories are expected to be slightly enriched in VBF and $VH$ events. Overall,
about 40 $H\to Z\gamma$ events are expected to be selected.
Table~\ref{tab:SBtable} summarises for each category: 
the number of selected events from data in the fit range $115~\GeV < m_{Z\gamma} < 150~\GeV$
(see Section~\ref{sec:signalBkgModel}); 
the expected number of events ($S_{90}$) in an interval around the $m_{Z\gamma}$ peak position 
expected to 
contain 90\%  of the SM signal events;
$w_{90}$ defined to be half of the width of the interval; 
the expected $S_{90}/B_{90}$, where $B_{90}$ is the background yield in the same mass window determined from data;
and the expected $S_{90}/\sqrt{S_{90}+B_{90}}$.
The window is constructed so that it includes 45\% of the signal events on either side of the peak position
for $\mH=125.09~\GeV$. The largest fraction of the signal is expected in the low \ptt\ 
categories, which have the smallest expected significance. The VBF-enriched category shows
the largest contribution to the expected significance.



\begin{table}[htbp]
\caption{The number of data events selected in the mass range used for the background fit to the 
$m_{Z\gamma}$ spectrum (115--150~\GeV) per category. In addition, the following numbers are given:
the expected number of Higgs boson signal events in an interval around the peak position for a signal 
of $\mH=125.09~\GeV$, 
expected to contain 90\% of the SM signal ($S_{90}$), the half-width of the $S_{90}$ interval 
($w_{90}$), as well as
the expected signal-to-background ratio in the $S_{90}$ window ($S_{90}/B_{90}$) with 
$B_{90}$ determined from data, and the expected significance estimate $S_{90}/\sqrt{S_{90}+B_{90}}$.}
\label{tab:SBtable}
\begin{center}
\begin{tabular}{lrSccc}
\hline\hline
                 Category  & Events &  $S_{90}$  & $w_{90}$ [GeV] & $S_{90}/B_{90}$ $[10^{-2}$] & $S_{90}/\sqrt{S_{90}+B_{90}}$\\
\hline
VBF-enriched                 & 88   & 1.2  & 3.9 & 9.5 & 0.32 \\        
High relative $\pt$          & 443   & 2.3  & 3.9 & 3.0 & 0.26  \\         
$ee$ high $p_{\mathrm{T}t}$      & 1053  & 3.3  & 3.9 & 1.1 & 0.19 \\         
$ee$ low $p_{\mathrm{T}t}$       & 11707 & 11.2 & 4.2 & 0.3 & 0.18 \\         
$\mu\mu$ high $p_{\mathrm{T}t}$  & 1413  & 4.0  & 3.7 & 1.2 & 0.22 \\         
$\mu\mu$ low $p_{\mathrm{T}t}$   & 16529 & 14.5 & 3.8 & 0.3 & 0.21 \\         
\hline\hline
\end{tabular}
\end{center}
\end{table}

The search for high-mass resonances uses two categories, one for each $Z$ 
boson candidate decay mode, $Z\to ee$
and $Z\to\mu\mu$, to benefit from both the better invariant mass resolution in the electron channel at 
large $m_{Z\gamma}$ and the differences in the systematic uncertainties between electrons and muons.
The invariant mass resolution, measured by the Gaussian width of the signal model (see Section~\ref{sec:signalBkgModel}),
ranges from 2.8~\GeV\ (3.1~\GeV) at $m_X = 250~\GeV$ to 16~\GeV\ (36~\GeV) at $m_X = 2.4~\TeV$
for $Z\to ee$ ($Z\to\mu\mu$).

