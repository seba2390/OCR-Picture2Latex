\section{Discussion}

In this section, we discuss some future extension or improvements on the system. 

As we discussed in Section~\ref{sec:anomalyrelated}, there are many related works for anomaly detection. Some approaches are using adaptive concept drifting, which can self-adapt to the target event without manual tuning~\cite{ma2018robust}. One major burden in our event construction is that we need to build different detection strategies/algorithms for different events, while some events are still needed to be manually tuned or detected by rules. Auto-ML is yet to come for anomaly detection, and the interpretability of the result is also a big challenge for end-to-end usability. The filter range for some events also needs manual tuning which requires large amount of domain expertise and is not robust. As of now, we are investing self-adaptive approaches and building time series metrics anomaly detection platform to reduce the event on-boarding cost. 

Also, our approach, which requires a fair amount of ``one-time" effort, utilizes rules to build links between events. A lot of the rules have similar patterns. Despite that our users prefer to have their own control, rules can be inferred from historical patterns. A great extension to this work is the use of machine learning or grammar inference techniques~\cite{wu2019reinam} to automatically or semi-automatically infer the rules.

%At last, since the framework is robust and generic to many use cases, we are planning to release it as open-source system so that we can allow different kinds of users to customize the system based on their own dependencies/events/domains. This also gives us the chance to further speculate \system's performance under different customized scenarios. 