
In this appendix, we study the behavior of the Dynamic Solow model in the long run by seeking $y \sim y_0 t$, $k_d \sim k_{d0} t$ and $k_s \sim k_{s0} t$ in equations \eqref{eq_res_full_y_dot}-\eqref{eq_res_full_matching} at large values of $t$.

\subsection{Asymptotic Behavior in the Supply-Driven Regime ($k_d>k_s$)}\label{subsec_supply}
We first consider the situation where capital demand exceeds supply, which entails $k=k_{s}$ under the market clearing condition \eqref{eq_res_full_clearing}, and obtain the resulting growth rates. 

For $t\gg1$, the production equation \eqref{eq_res_full_y_dot} becomes
\begin{equation}
	e^{(\rho k_{s0}+\varepsilon - y_0) t} - 1 = \tau_y y_0.
\end{equation}
Consequently, $(\rho k_{s0} + \varepsilon - y_0 )t$ must be constant, which in turn implies that 
\begin{equation}\label{eq_lt_supply_y0}
	y_0 = \rho k_{s0} + \varepsilon,
\end{equation}
with a precision of up to $O(1/t)$. Similarly, capital supply equation \eqref{eq_res_full_ks_dot} yields
\begin{equation}
	k_{s0} = \lambda e^{(y_0-k_{s0})t} - \delta,
\end{equation}
so that $(y_0 - k_{s0})t$ is constant and, therefore, with a precision of up to $O(1/t)$:
\begin{equation}\label{eq_lt_supply_ks0}
    k_{s0} = y_0.
\end{equation}
It follows from equations \ref{eq_lt_supply_y0} and \ref{eq_lt_supply_ks0} that
\begin{equation}\label{y0_solow}
	y_0 = k_{s0} = \frac{\varepsilon}{1-\rho} \equiv R,
\end{equation}
where $R$ denotes the classic Solow growth rate.\footnote{This same result also follows from equation \eqref{eq_bla} for $t{\geq}O(1/\varepsilon)$.} 

To determine the growth rate of capital demand $k_{d0}$, we average equation \eqref{eq_res_full_kd_dot} with respect to time, noting that $\bar{\dot{s}}=0$ since $s$ is bounded:
\begin{equation}\label{eq_lt_supply_kd0}
	k_{d0} = c_2 \bar{s},
\end{equation}
where the bar denotes the time average.

Then we average equation \eqref{eq_res_full_h_dot} while noting that $\bar{\dot{h}}=0$ since $h$ is bounded and that $H(k_s,k_d)=0$ from \eqref{eq_H_switch} (no feedback) to obtain
\begin{equation}\label{eq_lt_supply_h}
	\bar{h} = \overline{\tanh\left(\xi_t\right)} = \tanh\left(\overline{\xi_t}\right) = 0,
\end{equation}
where we have assumed that fluctuations are small to allow us to take averages under the hyperbolic tangent\footnote{This simplifying assumption does not severely restrict applicability as $\tanh{x}$ is approximated reasonably well by a linear function for $-1{\leq}x{\leq}1$ and the noise amplitude is $O(1)$ in \eqref{eq_lt_supply_h}.}. Similarly averaging equation \eqref{eq_res_full_s_dot} leads to
\begin{equation}\label{eq_lt_supply_s}
	\bar{s} = \overline{\tanh\left(\beta_1 s + \beta_2 h\right)} = \tanh\left(\beta_1 \bar{s}\right).
\end{equation}

Equation \eqref{eq_lt_supply_s} has three solutions for $\beta_1>1$: $s=0$, $s_{-}<0$, and $s_{+}>0$, where $s_{-}=-s_{+}$. Our focus is on $s_{-}$ and $s_{+}$ as they correspond to the stable equilibrium points\footnote{For the base case value $\beta_1=1.1$, we have $s_{\pm}\approx\pm 0.5$ from \eqref{eq_lt_supply_s}.}. The system spends most of its time in the attracting regions that surround each of these two equilibria and transits rapidly between them when forced by exogenous noise. In the long run, the time spent in transit is negligible relative to the length of time during which the system is entrapped by the attractors. The attractors have the same strength and are located symmetrically in $s$, thus the system tends to spend an equal amount of time at each of them at large $t$. Therefore, its average position with respect to sentiment $s$ must be zero. More formally, taking $s_{-}$ and $s_{+}$ as the attractors' proxies, we estimate the long-term average sentiment as
\begin{eqnarray}\label{sbar_sup_est}
	\bar{s} = \frac{1}{2}\left(s_{-}+s_{+}\right) = 0. 
\end{eqnarray}
Hence equation \eqref{eq_lt_supply_kd0} yields
\begin{equation}
	k_{d0} = 0.
\end{equation}

This result is intuitively clear: the growth of demand is driven in the long run by average sentiment, which converges to zero because its dynamics are symmetric in the absence of feedback. We conclude that in the supply-driven regime the economy's growth is, as expected, independent of capital demand and matches the classic Solow growth, $y_0=k_{s0}=R$, while capital demand is stagnating ($k_{d0}=0$). We verify these results via numerical simulations in Section \ref{sec_results_asymp}.

%\begin{figure}
%	\centering
%	\begin{subfigure}{.48\textwidth}
%		\centering
%		\includegraphics[width=1.0\linewidth]{figures/fig_appxD_supply_demand_gro%wth.pdf}
%		\caption{
%			Long-term dynamics in the supply case, where $k_s$ grows linearly %according to Eq. \eqref{eq_lt_supply_y0}, until it eventually it catches up to d%emand.
%		}
%		\label{fig_appxD_supply_demand_growth}
%	\end{subfigure}
%	\begin{subfigure}{.48\textwidth}
%		\centering
%		\includegraphics[width=1.0\linewidth]{figures/fig_appxD_supply_sentiment.%pdf}
%		\caption{
%			Convergence of the average sentiment to zero. Light gray are i%ndividual runs, black is the average across runs.	
%		}
%		\label{fig_appxD_supply_sentiment}
%	\end{subfigure}
%	\caption{Visualization of the asymptotic regime where supply is limiting. F%ig. \ref{fig_appxD_supply_demand_growth} shows correctness of theoretical and re%alized paths for a simulation. Fig. \ref{fig_appxD_supply_sentiment} shows tha%t average sentiment converges to zero. Further parameters: $\va%repsilon=2.5e-5$, $\beta_1=1.1$, $\beta_2=1.0$, $c_1=1$, $c_2=2.5e-4$, $\tau%_y=1000$, $\tau_s=250$, $\tau_h=25$, $\tau_\xi=5$, $\lambda=0.15$, $\delt%a=2e-4$}
%\label{fig_appxD_supply}
%\end{figure}

\subsection{Asymptotic Behavior in the Demand-Driven Regime ($k_d<k_s$)}
In the demand-driven regime, the market clearing condition \eqref{eq_res_full_clearing} yields $k=k_{d}$, so that equation \eqref{eq_res_full_y_dot} becomes
\begin{equation}
e^{(\rho k_{d0}+\varepsilon - y_0) t} - 1 = \tau_y y_0.
\end{equation}
Consequently, 
\begin{equation}\label{eq_lt_demand_y0}
y_0 = \rho k_{d0}+\varepsilon,
\end{equation}
with a precision of up to $O(1/t)$. Similarly, equation \eqref{eq_res_full_ks_dot} takes the form: 
\begin{equation}
k_{s0} = \lambda e^{(y_0-k_{s0})t} - \delta e^{k_d-k_s}.
\end{equation}
The term $\delta e^{k_d-k_s}$ can be neglected as it is exponentially small for $k_d<k_s$; therefore, with a precision of up to $O(1/t)$:
\begin{equation}\label{eq_lt_demand_ks0}
y_0 = k_{s0}.
\end{equation}
And finally, averaging equation \eqref{eq_res_full_kd_dot} leads to
\begin{equation}\label{eq_lt_demand_kd0}
	k_{d0} = c_2 \bar{s}.
\end{equation}

We can rewrite equations \eqref{eq_lt_demand_y0} and \eqref{eq_lt_demand_ks0} as
\begin{equation}
y_0 = k_{s0} = R + \rho\left(k_{d0} - R\right).
\end{equation}
It follows that if $\bar{s}>\frac{R}{c_2}$, then the economy's long-term growth exceeds the classic Solow growth rate $R$. For the base case values of $c_2$, $\varepsilon$ and $\rho$ in our model, we find $\bar{s}>0.05$. 

To estimate $\bar{s}$, we must consider three types of characteristic behavior possible in the demand-driven regime: noise-driven, limit cycle and coherence resonance behavior. Noise-driven behavior prevails when feedback is weak. This situation is, in its limit, equivalent to that of the supply-driven regime in which sentiment behaves symmetrically with respect to the origin. Therefore, $\bar{s}\rightarrow0$. Thus, the noise-driven mode generates growth $y_0\rightarrow\varepsilon$, which is lower than $R$.

The growth in the two other modes is studied numerically in Section \ref{sec_results_asymp}. For completeness, we briefly note, first, limit cycles (periodic or stochastic) lead to $\bar{s}\rightarrow0$ and $y_0\rightarrow\varepsilon$ (as the economy tends to spend a half of its time in the region where $s>0$ and the other half where $s<0$) and, second, coherence resonance yields $\bar{s}>0.05$ and $y_0>R$, owing to the attractors' asymmetry caused by technological growth ($\varepsilon>0$) in the presence of economic feedback ($\gamma>0$). 

As a final remark, it follows from \eqref{eq_lt_demand_y0} that, asymptotically, $z\sim z_0t\sim(\rho k_{d0}+\varepsilon - y_0) t \sim  O(1)$. The system's motion is therefore bounded in $z$. Its motion is likewise bounded in $s$ and $h$, which vary between -1 and 1, as, at the boundaries, $\dot{s}$ and $\dot{h}$ are directed into the domain of motion as follows, respectively, from equations \eqref{eq_res_full_s_dot} and \eqref{eq_res_full_h_dot}. Thus, the system's phase trajectories are bounded in the $(s, h, z)$-space.


