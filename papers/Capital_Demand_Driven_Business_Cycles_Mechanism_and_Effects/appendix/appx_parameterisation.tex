In this appendix, we examine the model's parameters and discuss how they affect the behavior of the dynamical system \eqref{eq_kd_shz_system} in the phase space.

We begin with equation \eqref{eq_z_s_dot} that describes sentiment dynamics. Parameter $\beta_1$ defines the relative importance of the herding and random behaviors of firms. In an unforced situation ($\beta_2=0$), the number of stable equilibrium points, to which the firms' sentiment $s$ converges, doubles at $\beta_1=1$ from one to two. For $\beta_1<1$, random behavior prevails since there is a single equilibrium at $s=0$, meaning firms fail to reach a consensus opinion. Conversely, for $\beta_1>1$, herding behavior rules as equation \eqref{eq_z_s_dot} generates a polarized, bi-stable environment with one pessimistic ($s<0$) and one optimistic ($s>0$) equilibrium states. It is sensible to assume $\beta_1\sim1$, otherwise firms would unrealistically behave either randomly or in perfect synchronicity. We set $\beta_1=1.1$, implying a slight prevalence of herding over randomness. In addition, we set $\beta_2=1$ to ensure that analysts' influence on firms' managers likewise appears in the leading order. 

We now consider the information dynamics in \eqref{eq_z_h_dot}. The terms under the hyperbolic tangent describe the impacts of economic growth and exogenous news on the collective opinion of analysts $h$. We assume these two sources of information are of equal importance. Thus, we expect that $\gamma\omega_y=O(1)$ in the feedback term and we model $\xi_t$ as an Ornstein-Uhlenbeck process with an $O(1)$ standard deviation and short decorrelation timescale $\tau_\xi$. Note that $\omega_y\ll1$ and accordingly $\gamma\gg1$. 

Finally, we inspect the economic dynamics in \eqref{eq_z_z_dot}. In this equation, different terms determine leading behaviors on separate timescales. We show in \ref{appx_asymp_derivation} that the last three terms (with technology growth rate $\varepsilon$ estimated on the basis of observed total factor productivity) are in balance in the long run. However, if we consider short timescales, the change in sentiment $\dot{s}$ becomes dominant. Thus, equation \eqref{eq_z_z_dot} can be approximated in the short run as $\dot{z}\sim \rho c_1\dot{s}$ and we set $\rho c_1=1$. We also note that by construction $c_2\ll c_1$ to ensure that the term $c_2s$ does not contribute to capital demand dynamics on short timescales. Hence we expect $c_2\ll 1$.  

As highlighted in Section \ref{sec_model}, there is a segregation of characteristic timescales that emerges naturally from the types of decisions faced by the different agents in the model: $\tau_\xi\ll\tau_h\ll\tau_s\ll\tau_y\ll1/\varepsilon$. This segregation facilitates the transfer of the impact of instantaneous news shocks $\xi_t$ across multiple timescales. The estimates for the timescales are discussed in Section \ref{sec_model_capmkt}.

The parameters $c_2$ and $\gamma$ are central to the system's behavior in the phase space. Increasing $c_2$ stabilizes the system, strengthening convergence towards the stable equilibria and creating a higher barrier between attracting regions. The role of $\gamma$ is twofold. As $\gamma$ grows from zero, its immediate effect is to destabilize the system due to growing economic feedback. However, as $\gamma$ continues to increase, it exerts a stabilizing effect similar to that of $c_2$ because of the term $\gamma c_2$ in the equilibrium condition: 
\begin{equation}\label{eq_kd_equilibrium_condition}
\textrm{arctanh}(s)-\beta_1s = \beta_2\tanh\left(\gamma c_2s + \gamma\varepsilon\right), 
\end{equation}
which follows from equations \eqref{eq_kd_shz_system} for $\dot{h}=\dot{s}=\dot{z}=\xi_t=0$. Consequently, the potential to generate autonomous economic instability is limited. In particular, there exists a critical value\footnote{
	Subject to the values set for the other parameters.
} of $c_2\sim10^{-4}$ below which feedback may generate a limit cycle and above which it does not. Figure \ref{fig_appxC_cycle} depicts the formation and subsequent destruction, for $c_2=10^{-4}$, of the limit cycle as $\gamma$ increases.

\iffalse
\begin{figure}
	\centering
	\begin{subfigure}[b]{0.9\textwidth}   
		\centering
		\includegraphics[width=\textwidth]{figures/fig_appxC_cycle_g350.pdf}
		%\caption{Stable dynamics for $\gamma=350$. Red trajectories converge to the left focus and blue trajectories converge to the right focus.}
		\label{fig_appxC_cycle_g350}
	\end{subfigure}
	%\vskip\baselineskip
	\begin{subfigure}[b]{0.9\textwidth}   
		\centering 
		\includegraphics[width=\textwidth]{figures/fig_appxC_cycle_g1000.pdf}
		%\caption{The equilibria become unstable and a large stable limit cycle emerges for $\gamma=1000$.}    
		\label{fig_appxC_cycle_g1000}
	\end{subfigure}
	%\vskip\baselineskip
	\begin{subfigure}[b]{0.9\textwidth}   
		\centering 
		\includegraphics[width=\textwidth]{figures/fig_appxC_cycle_g4000.pdf}
		%\caption{The left node and the saddle vanish while the limit cycle persists at $\gamma=4000$.}
		\label{fig_appxC_cycle_g4000}
	\end{subfigure}
	%\vskip\baselineskip
	\begin{subfigure}[b]{0.9\textwidth}   
		\centering 
		\includegraphics[width=\textwidth]{figures/fig_appxC_cycle_g15000.pdf}
		%\caption{The dynamics are again stable at $\gamma=15000$. Trajectories converge to the stable focus and the limit cycle disappears.}       
		\label{fig_appxC_cycle_g15000}
	\end{subfigure}
	%\caption{Development of a stable limit cycle with increasing $\gamma$, with $\xi_t=0$, $c_2=1\times10^{-4}$. Other parameters from the base case (\ref{appx_notation}). The left panels show the phase portraits projected on the ($s,z$)-plane and the right panels plot $s(t)$. Classification of equilibrium points is provided in footnote \ref{point notatrion} in Section \ref{sec_kd_limit}. As $\gamma$ increases, a large stable limit cycle emerges and then vanishes, demonstrating the destabilizing effect of $\gamma$ at low values and its stabilizing effect at high values. (a) Two stable foci for $\gamma=350$. (b) The right focus becomes unstable and a stable limit cycle appears for $\gamma=1000$. (c) The left node and the saddle vanish  while the limit cycle persists for $\gamma=4000$. (d) The right focus becomes stable again and the limit cycle disappears for $\gamma=15000$.} 
%	\label{fig_appxC_cycle}
\end{figure}
\fi

\begin{figure}
	\centering
	\includegraphics[width=\textwidth]{figures/fig_appxC_cycle_g.pdf}
	\label{fig_appxC_cycle_g}
	\caption{Development of a stable limit cycle with increasing $\gamma$ for $\xi_t=0$, $c_2=1\times10^{-4}$ and all other parameter values from the base case (\ref{appx_notation}). The left panels show the phase portraits projected on the ($s,z$)-plane and the right panels plot $s(t)$. Classification of equilibrium points is provided in footnote \ref{point notatrion} in Section \ref{sec_kd_limit}. As $\gamma$ increases, a large stable limit cycle emerges and then vanishes, demonstrating the destabilizing effect of $\gamma$ at low values and its stabilizing effect at high values. (i) Stable dynamics for $\gamma=350$: red trajectories converge to the left focus and blue trajectories converge to the right focus. (ii) The equilibria become unstable and a large stable limit cycle emerges for $\gamma=1000$. (iii) The left node and the saddle point vanish while the limit cycle persists at $\gamma=4000$. (iv) The dynamics are again stable at $\gamma=15000$: trajectories converge to the stable focus and the limit cycle disappears.} 
	\label{fig_appxC_cycle}
\end{figure}

In this paper, we argue that realistic economic behaviors cannot be explained by a stochastic limit cycle. Therefore, we proceed to study the system for $c_2\gtrsim10^{-4}$, which ensures a bi-stable configuration without a limit cycle. Figure \ref{fig_appxC_base_c2_dynamics} illustrates that as $c_2$ increases, the barrier between attracting regions grows stronger, resulting in less frequent crossings from one region to the other (i.e. cycle duration increases). We seek $c_2$ at the lower end of this range to reduce cycle duration. %However, the fact that $c_2$ is bounded from below implies a certain minimum on the characteristic duration of business cycles. To obtain shorter cycles, we would have to add an accelerator (e.g. financial market) into the model. 

\begin{figure}
    \centering
    \includegraphics{figures/fig_appxC_base_c2_dynamics.pdf}
    \caption{The effect of $c_2$ on the dynamics of sentiment $s(t)$ for $\xi_t\neq0$. As $c_2$ increases from the base case value $c_2=7\times10^{-4}$ (left) to $c_2=9.5\times10^{-4}$ (right), the barrier separating the two attracting regions grows stronger. The system spends more time captive to the attractors, reducing the frequency of the crossings between them and lengthening the duration of fluctuations. Note that the system tends to stay longer at the expansion attractor (where $s>0$) owing to the asymmetry induced by technological growth $\varepsilon>0$. All other parameters are from the base case (\ref{appx_notation}).}
    \label{fig_appxC_base_c2_dynamics}
\end{figure}

Similarly, the barrier between attracting regions grows stronger as $\gamma$ increases, resulting in infrequent transitions between the attractors. Relatively small values of $\gamma$, however, dampen feedback, leading to weak dynamics and stochastic-like behavior. Accordingly, we focus on values of $\gamma$ between these two extremes. Figure \ref{fig_appxC_base_gamma_dynamics} depicts the dynamics under different values of $\gamma$, with balanced dynamic behaviors for a reasonably wide range thereof.

\begin{figure}
    \centering
    \includegraphics{figures/fig_appxC_base_gamma_dynamics.pdf}
    \caption{The effect of $\gamma$ on the dynamics of sentiment $s(t)$ for $\xi_t\neq0$. At low $\gamma$, the system's behavior is dominated by noise as the barrier between the two attracting regions is weak. As $\gamma$ increases, the barrier grows stronger and the system becomes extremely bi-stable. Reasonably balanced dynamics emerge in the range from $\gamma=1500$ to $\gamma=2500$. Note the asymmetry caused by technological growth becomes exacerbated as $\gamma$ increases in accordance with equation \eqref{eq_kd_equilibrium_condition}. All other parameters are from the base case (\ref{appx_notation}).}
    \label{fig_appxC_base_gamma_dynamics}
\end{figure}

We select $c_2=7\times10^{-4}$ and $\gamma=2000$ for the base case studied in Sections \ref{sec_kd_limit} and \ref{sec_results}. Note that $c_2\ll 1$, $\gamma\gg1$ and $\gamma\omega_y=O(1)$, as expected. Figure \ref{fig_appxC_base_sh_phase} provides the base case phase portrait ($\xi_t=0$) projected on the $(s,h)-$plane, showing attracting regions around the two stable equilibria as well as long trajectories passing near each attractor and ending at the opposite equilibrium. These trajectories, which emerge due to strong feedback ($\gamma\gg1$), allow the economy to transition quickly between expansions and contractions. 

\begin{figure}
    \centering
    \includegraphics[width=\textwidth]{figures/fig_appxC_base_sh_phase.pdf}
    \caption{System dynamics in the base case (\ref{appx_notation}). Left: A phase portrait ($\xi_t=0$) projected on the ($s,h$)-plane. The portrait depicts stable foci, separated by a saddle point, and the large trajectories relevant for regime transitions. Right: A trajectory ($\xi_t\neq0$) projected on the ($s,h$)-plane. The stable foci are at the center of the two attracting regions, within which the trajectory is dense. The transit of the economy between these regions corresponds to regime transitions between contractions and expansions, occurring at much shorter intervals than the periods during which the economy is captive to an attractor. The trajectory was smoothed by a Fourier filter to remove harmonics with periods less than 500 business days for clean visualization.}
    \label{fig_appxC_base_sh_phase}
\end{figure}

\begin{figure}
    \centering
    \includegraphics{figures/fig_appxC_base_epsilon_asymmetry.pdf}
    \caption{The effect of $\varepsilon$ on the dynamics of sentiment $s(t)$ for $\xi_t\neq0$. As $\varepsilon$ increases from the base case value $\varepsilon=2.5\times10^{-5}$ (left) to $\varepsilon=7.5\times10^{-5}$ (right), the system behavior begins to exhibit a stronger asymmetry between the contraction and expansion attractors. All other parameters are from the base case (\ref{appx_notation}).}
    \label{fig_appxC_base_epsilon_asymmetry}
\end{figure}

The attractors are not connected and the economy cannot cross the boundary separating them in the absence of exogenous news shocks $\xi_t$. It takes a random news event to force the economy, entrapped by one attractor, across the boundary. Once it crosses the boundary, the economy finds itself on the long trajectory that takes it swiftly to the other attractor, where the economy remains captive until another news event instigates the next regime transition by again forcing the economy across the boundary. At this point the economy is carried back to the entrapment region where it started. This is the coherence resonance mechanism that is at the heart of the economic fluctuations in our model. 

As a final comment, we note that if $\varepsilon=0$, the equilibrium condition \eqref{eq_kd_equilibrium_condition} is symmetric to $s\to -s$. Technology growth, $\varepsilon>0$, causes an asymmetry\footnote{Note that $\gamma$ amplifies the asymmetry via the term $\gamma\varepsilon$ in equation \eqref{eq_kd_equilibrium_condition}.} wherein the equilibrium at $s>0$ becomes stronger than that at $s<0$ (to the extent that the latter vanishes above a certain threshold). As a result, the system tends to stay longer in the region where economic sentiment is positive, accelerating the economy's long-term growth. The asymmetry, however, vanishes in the limit cycle regime, whether periodic or stochastic (Section \ref{sec_results_asymp}). Figure \ref{fig_appxC_base_epsilon_asymmetry} illustrates this asymmetric behavior.  

