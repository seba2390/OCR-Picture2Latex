\section{Generalization bounds of Recurrent neural networks}\label{sec:maingensec}
The proof has been structured as follows: In section~\ref{sec:Invert_RNN}, we prove thm.~\ref{thm:Invertibility_ESN} where we show that a linear transformation of $\mathbf{h}^{(L)}$ can give back $[\bx^{(1)}, \ldots, \bx^{(L)}]$. The proof follows from a direct application of lemma~\ref{lemma:singlecell_ESN}. Claim~\ref{clam:stabizable_V} shows that the linear matrix at each induction step satisfies a property of stability necessary for the inductive application of lemma~\ref{lemma:singlecell_ESN}.  

In section~\ref{sec:existence}, we first define a pseudo recurrent neural network that stays close to the over parameterized RNN at initialization throughout SGD. We then show in thm.~\ref{thm:existence_pseudo} that there exists a pseudo network which can approximate the target function in concept class. The proof involves breaking correlations among the hidden states and the weight matrices and then we show that the pseudo network concentrates on the desired signal. The above two steps have been divided among the four intermediate claims: ~\ref{claim:simplifybig}, ~\ref{claim:diffftildef}, ~\ref{claim:difftildefphi} and ~\ref{claim:fbacktildeback}.

In section~\ref{sec:optim_general}, we prove theorem~\ref{thm:main_theorem} which shows that RNNs can attain a population risk similar to the target function in the concept class using SGD. First, we show that the pseudo neural network stays close to RNN with small perturbation around initialization in lemmas~\ref{lemma:perturb_NTK_small_output} and~\ref{lemma:perturb_NTK_small}. We then show that there exists a RNN close to random RNN that can approximate the target function in lemma~\ref{lemma:perturb_small_target}. We complete the argument by showing that the SGD can find matrices with training loss close to the optimal in lemma~\ref{lem:trainloss} and then bounding the Rademacher complexity of RNNs with bounds on the movement in the weight matrices in lemma~\ref{lem:radcomp}.


    %First, we give a few definitions about a recurrent neural network (RNN) and the structure of the input sequence to the RNN
    

    \subsection{Invertibility of RNNs at initialization}\label{sec:Invert_RNN}    
    Let $\mathbf{W}^{(k_{b}, k_{e})} = \prod_{k_{b} \ge \ell \ge k_{e}} \mathbf{D}_{(0)}^{(\ell)} \mathbf{W}$, if $k_b \ge k_c$. Otherwise, $\mathbf{W}^{(k_{b}, k_{e})} = \mathbf{I}$. 
Define $\obW^{[\ell]}$ inductively as follows:
% meh
\begin{equation*}
	\obW^{[\ell]} = \left[\mathbf{D}^{(\ell)}_{(0)} \mathbf{W} \obW^{[\ell-1]},\text{ } \mathbf{D}_{(0)}^{(\ell)}\mathbf{A}\right]_r, \quad \text{ for } 2 \le \ell \le L,
\end{equation*}
with $\obW^{[1]} = \mathbf{D}^{(1)}_{(0)} \mathbf{A}$.  We can show that $\obW^{[\ell]} = [\mathbf{W}^{(\ell, 2)} \mathbf{D}_{(0)}^{(1)} \mathbf{A}, \mathbf{W}^{(\ell, 3)} \mathbf{D}_{(0)}^{(2)} \mathbf{A}, \cdots, \mathbf{D}_{(0)}^{(\ell)} \mathbf{A}]_r$ for $\ell \ge 2$, which will be helpful for presentation later on.
%Applying the preceding corollary repeatedly gives us a way to compute $\bx$ given $\bx^{(L)}$:
\begin{theorem}\label{thm:Invertibility_ESN}
	For any $\epsilon_x < \frac{1}{L}$ and a given normalized sequence $\bx^{(1)}, \cdots, \bx^{(L)}$, 
	\begin{align*}
		&\norm{[\bx^{(1)}, \cdots, \bx^{(L)}] - \obW^{[L]\top} \mathbf{h}^{(L)}}_\infty \\&\quad \leq \mathcal{O}\left(L^4 \cdot (\rho^{11} m^{-1/12} + \rho^{7} m^{-1/12} + \rho^{7} m^{-1/4} + \rho^{11} m^{-1/4})  \right) + \mathcal{O}(\rho^2 L^{11/6} \epsilon_x^{5/3})
		%\mathcal{O}(2^{(L + 1)/2} \zeta (1 + L \zeta)) 
	\end{align*}
	with probability at least $1 - e^{-\Omega(\rho^2)}$. 
	%$m \ge \Omega\left(d L^2 \gamma^{-1/2} \zeta^{-2} \ln \frac{1}{\zeta}\right)$, 
	%and Assumption~\ref{as:gamma} holds.
\end{theorem}
\begin{proof}
	The theorem has been restated and proven in theorem~\ref{thm:Invertibility_ESN_proof}.
\end{proof}


\begin{corollary}\label{cor:Invertibility_ESN}
	For a given normalized sequence $\bx$ and any $\varepsilon_x < \frac{1}{L}$, with probability at least $1 - e^{-\Omega(\rho^2)}$ w.r.t. the weights $\mathbf{W}$ and $\mathbf{A}$,
	%\todo{It's not exactly theorem 4.2. Wrte a corollary and refer to that.}
	\begin{align*} 
	&\abs{\overline{\mathbf{W}}^{[L]\top} h^{(L-1)} - \varepsilon_x[\overline{\mathbf{x}}^{(2)}, 
			\cdots, \overline{\mathbf{x}}^{(L-1)}]}  \\&\le \mathcal{O}\left(L^4 \cdot (\rho^{11} m^{-1/12} + \rho^{7} m^{-1/12} + \rho^{7} m^{-1/4} + \rho^{11} m^{-1/4})  \right) + \mathcal{O}(L^{4/3} \varepsilon_x^{2/3}) \\&
		\le  \mathcal{O}(L^4 \rho^{11} m^{-1/12} + \rho^2 L^{11/6} \varepsilon_x^{5/3}), 
	\end{align*}
	where $\overline{{\mathbf{W}}}^{[\ell]}$ is slightly redefined as
	\begin{align*}
		&\obW^{[\ell]} = [\mathbf{D}^{(\ell-1)}_{(0)} \mathbf{W} \obW^{[\ell - 2]},  \mathbf{D}_{(0)}^{(\ell-1)}\mathbf{A}_{[d-1]}]_r, \quad \text{ for all } \ell \ge 4,
	\end{align*}
	with $\obW^{[2]} = \mathbf{D}^{(2)}_{(0)} \mathbf{A}_{[d-1]}$
	and $\mathbf{A}_{[d-1]}  \in \mathbb{R}^{m \times (d-1)}$ denotes the matrix which contains the first $d-1$ columns of the matrix $\mathbf{A}$.
	%Hence,
\end{corollary}
\begin{proof}
	The difference from Thm.~\ref{thm:Invertibility_ESN} is that here we attempt to get the first $d-1$ dimensions of the vectors $\mathbf{x}^{(2)}, \cdots, \mathbf{x}^{(L-1)}$. This leads to a small change in the inversion matrix.
\end{proof}

Note that in the above corollary, $\obW^{[L]} = [\mathbf{W}^{(L, 3)} \mathbf{D}_{(0)}^{(2)}\mathbf{A}_{[d-1]}, \mathbf{W}^{(L, 4)} \mathbf{D}_{(0)}^{(3)}\mathbf{A}_{[d-1]}, \cdots, \mathbf{W}^{(L, L)} 
	\mathbf{D}_{(0)}^{(L-1)}\mathbf{A}_{[d-1]}]_r$, where $\mathbf{W}^{(k_b, k_e)} = \prod_{k_b \ge \ell > k_e} \mathbf{D}_{(0)}^{(\ell)} \mathbf{W}$. We are going to use this definition in the following theorems.
    
   
    \subsection{Existence of good pseudo network}\label{sec:existence}
    We first define a pseudo RNN model, which is shown later to stay close to the RNN model during the gradient descent dynamics. 

\begin{definition}[Pseudo Recurrent Neural Network]
	Given two matrices $\mathbf{W}^{\ast} \in \mathbb{R}^{m \times m}$ and $\mathbf{A}^{\ast} \in \mathbb{R}^{m \times d}$, the output for a pseudo RNN with activation function $\relu$ for a given sequence $\bx$ are given by
	\begin{equation*}
		%\bar{\mathbf{h}}_{(t)} &=  \mathbb{I}_{\mathbf{A}^{(0)T} \mathbf{x}_{(t)} + \mathbf{W}^{(0)T} \bar{\mathbf{h}}_{(t-1)} \ge 0} \mathbf{A}^{T} \mathbf{x}_{(t)} + \mathbf{W}^{T} \bar{\mathbf{h}}_{(t-1)}  \\
		F_s^{(\ell)}(\bx; \mathbf{W}^{\ast}, \mathbf{A}^{\ast})   =  \sum_{i \le \ell} \mathbf{Back}_{i \to \ell, s} \mathbf{D}^{(i)} \left(\mathbf{W}^{\ast} \mathbf{h}^{(i-1)} + \mathbf{A}^{\ast} \mathbf{x}^{(i)}\right) \quad \forall 1 \le \ell \le L,  s \in [\dout ],
		%\sum_{r \in [m]} b_{r, s} \bar{h}_{(t), r} \quad \forall t \ge 1,  s \in [\dout ],
	\end{equation*}
	where $\mathbf{Back}_{i \to \ell, s} = \mathbf{b}_s^{\top} \mathbf{D}^{(\ell)} \mathbf{W} \cdots \mathbf{D}^{(i+1)} \mathbf{W}$. For typographical simplicity, we will denote $F_s^{(\ell)}(\bx; \mathbf{W}^{\ast}, \mathbf{A}^{\ast})$ as $F_s^{(\ell)}$.
	%$\mathbf{D}^{(i)}$ denotes the matrix of activation pattern at RNN unit $i$ at initialization, whose elements are given by
	%\begin{align*}
	%    d^{(i)}_{jk} &= \mathbb{I}_{\mathbf{w}_{k}^{\top} \mathbf{h}^{(i-1)} + \mathbf{a}_{k}^{\top} \mathbf{x}^{(i)} \ge 0} \quad \text{if } k = j\\&= 0; \quad\quad\quad \text{otherwise}, \forall k, j \in [m].
	%\end{align*}
\end{definition}

%%%%%%%%%%%%%%%%%%%%%%%%%%%%%%%%%%%%%%%%%%%%%%%%%%

Now, we show that there exist two matrices $\mathbf{W}^{\ast}$ and $\mathbf{A}^{\ast}$, defined below, such that the pseudo network is close to the concept class under consideration.
\begin{definition}\label{def:existence}
	Define $\mathbf{W}^{\ast}$ and $\mathbf{A}^{\ast}$ as follows.
	\begin{align*}
		\mathbf{W}^{\ast} &= 0 \\
		\mathbf{a}^{*}_{r} &= \frac{\dout }{m} \sum_{s \in [\dout ]} \sum_{r' \in [p]} b_{r, s} b_{r', s}^{\dagger} H_{r', s} \left(\theta_{r', s} \left(\langle \mathbf{w}_{r}, \overline{\mathbf{W}}^{[L]} \mathbf{w}_{r', s}^{\dagger}\rangle\right), \sqrt{m/2} a_{r, d}\right) \mathbf{e}_d, \quad \forall r \in [m],
	\end{align*}
	where
	\begin{equation*} 
		\theta_{r', s} = \frac{\sqrt{m/2}}{\norm[1]{ \overline{\mathbf{W}}^{[L]} \mathbf{w}_{r', s}^{\dagger}}},
	\end{equation*}
	and $\obW^{[L]} = [\mathbf{W}^{(L, 3)} \mathbf{D}_{(0)}^{(2)}\mathbf{A}_{[d-1]}, \mathbf{W}^{(L, 3)} \mathbf{D}_{(0)}^{(2)}\mathbf{A}_{[d-1]}, \cdots, \mathbf{W}^{(L, L)} 
	\mathbf{D}_{(0)}^{(L-1)}\mathbf{A}_{[d-1]}]_r$, where $\mathbf{W}^{(k_b, k_e)} = \prod_{k_b \ge \ell > k_e} \mathbf{D}_{(0)}^{(\ell)} \mathbf{W}$.
	%Here $\mathbf{P}$ denotes a diagonal matrix that projects a vector onto its top $L(d-1)$ coordinates and  $\mathbf{P}^{\perp} = \mathbf{I} - \mathbf{P}$.
\end{definition}
%\todo{There is a dimension mismatch. Pls correct later on.}


In the following theorem, we show that the pseudo RNN can approximate the target concept class, using the weight $\mathbf{W}^{*}$ and $\mathbf{A}^{\ast}$ define above.
\begin{theorem}[Existence of Good Pseudo Network]\label{thm:existence_pseudo}
	The construction of $\mathbf{W}^{*}$ and $\mathbf{A}^{\ast}$ in Definition~\ref{def:existence} satisfies the following. For every normalized input sequence $\mathbf{x}^{(1)}, \cdots, \mathbf{x}^{(L)}$, we have with probability at least $1-e^{-\Omega\left(\rho^{2}\right)}$ over $\mathbf{W}, \mathbf{A}, \mathbf{B},$ it holds for every $s \in [\dout ]$.
	$$
	\begin{array}{l}
		F_{s}^{(L)} \stackrel{\text { def }}{=} \sum_{i=1}^{L} \mathbf{e}_{s}^{\top} \mathbf{Back}_{i \rightarrow L} D^{(i)} \left(\mathbf{W}^{\ast} \mathbf{h}^{(i-1)} + \mathbf{A}^{\ast} \mathbf{x}^{(i)}\right) \\
		= \sum_{r \in[p]} b_{r, s}^{\dagger} \Phi_{r, s} \left(\left\langle  \mathbf{w}_{r, s}^{\dagger}, [\overline{\mathbf{x}}^{(2)}, \cdots, \overline{\mathbf{x}}^{(L-2)}]\right\rangle\right)\\ \pm \mathcal{O}(\dout Lp\rho^2 \varepsilon + \dout L^{17/6} p \rho^4 L_{\Phi} \varepsilon_x^{2/3} + \dout ^{3/2}L^5 p \rho^{11} L_{\Phi} C_{\Phi}  \mathfrak{C}_{\varepsilon}(\Phi, \mathcal{O}(\varepsilon_x^{-1}))  m^{-1/30} ).
		%\left(p \rho^{11} \cdot O\left(\varepsilon_{e}+\mathfrak{C}_{s}(\Phi, 1) \varepsilon_{x}^{1 / 3}+C m^{-0.05}\right)\right)
	\end{array}
	$$
	%We partition the set of $m$ input neurons into $p$ non-overlapping sets.
	%hen, define the weights $\left\{\mathbf{w}^{*}_{k + \frac{lm}{p}}\right\}_{k \in \left[\frac{m}{p}\right], l \in [p]}$ as follows.
	%Construct a matrix $\mathbf{C} \in \mathbb{R}^{p \times m}$ as follows. For each of the functions $f_i(\mathbf{x}) = \phi(\widetilde{\mathbf{a}}_i^{\top} \mathbf{x})$, we can construct vectors $\widetilde{\mathbf{c}}_i$ such that
	%\begin{equation*}
	%   \sup_{\mathbf{x \in \mathcal{X}}} \abs{f_i\left(\mathbf{x}\right) - \sum_{r = i  \frac{m}{2p} }^{(i + 1)  \frac{m}{2p}} \widetilde{c}_{i, r} \sigma\left(\mathbf{a}_r^{(0)\top} \mathbf{x}\right)} \le \varepsilon_i
	%\end{equation*} 
	%holds true with probability at-least $1 - \delta_i$, using \autoref{Thm:Rep_power}. Fix the rows of $\mathbf{C}$ as
	%\begin{align*}
	%   c_{ij} &= \widetilde{c}_{ij}, \quad \text{ if } i  \frac{m}{2p} \le j <  (i + 1)  \frac{m}{2p} \\ &=0 \quad \quad \text{otherwise}, \quad \forall i \in [p].
	%\end{align*}
	%Now,  define the weights $\left\{\mathbf{a}^{*}_{r}\right\}_{r \in [\frac{m}{2}, m-1]}$ as follows.
	%\begin{equation*}
	%\mathbf{a}^{*}_{r} = \frac{2}{m} \sum_{s \in [\dout ]} \sum_{r' \in [p]} b_{r, s} b_{r', s}^{\dagger} H\left(\theta_{r'} \left(\langle \mathbf{w}_{r}, \mathbf{C}^T \mathbf{\widetilde{w}}_{r'}\rangle + \langle \mathbf{a}_{r}^{[d-1]},  \mathbf{\widetilde{a}}_{r'}  \rangle\right), \theta_{r'}  b_{r, s}\right) \mathbf{e}_d,
	%\end{equation*}
	%where $\theta_{r'}$ is given by
	%\begin{equation*} 
	%\theta_{r'} = \frac{\sqrt{m}}{\sqrt{ \norm{\mathbf{a}_{r'}}^2 + \norm{\mathbf{C}^T \mathbf{\widetilde{w}}_{r'}}^2 }}.
	%\end{equation*}
	%Set $\left\{\mathbf{w}_{r}^*\right\}_{r \in [m]}$ and $\left\{\mathbf{a}^{*}_r\right\}_{r \in [\frac{m}{2} - 1]}$ as zero vectors.
	%For each $l \in [p]$, $\mathbf{\widetilde{a'}}_l\in \mathbb{R}^{m}$ is defined by 
	%\begin{equation*}
	%\widetilde{a'}_{l, i + \gamma \frac{m}{p}} =  \sqrt{\frac{\dout }{m}}\widetilde{a}_{l, \gamma}, \quad \forall \gamma \in [p], i \in \left[\frac{m}{p}\right].
	%\end{equation*}  
\end{theorem}

\begin{proof}
	The theorem has been restated and proven in theorem~\ref{thm:existence_pseudo_proof}.
\end{proof}



