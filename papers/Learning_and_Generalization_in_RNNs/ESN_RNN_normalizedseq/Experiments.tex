\section{Experiments}\label{sec:expts}

\iffalse
\begin{center}
 \begin{tabular}{|| c | c | c ||} 
 \hline
 Task & RNN(Relu) & RNN(Tanh)  \\ [0.5ex] 
 \hline\hline
  Counter-2 & 0.15 & 0.1\\
  Counter-3 & 0.02 & 0.02\\
  Counter-4 & 0.06 & 0.02 \\
  Boolean-3 & 0.99 & 0.756 \\
  Boolean-5 & 0.0 & 0.766\\
  Shuffle-2 & 0.966 & 0.60\\
  Shuffle-4 & 0.599 & 0.202\\
  Shuffle-6 & 0.348 & 0.338\\ [1ex]
 \hline
\end{tabular}
\end{center}
\fi 


%\iffalse
\begin{figure}[!ht]
\centering
\iffalse
\includegraphics[width=.45\textwidth]{../ESN_RNN_normalizedseq/RNN_2.png}
\includegraphics[width=.45\textwidth]{../ESN_RNN_normalizedseq/RNN_4.png}
\includegraphics[width=.45\textwidth]{../ESN_RNN_normalizedseq/RNN_8.png}
\fi
%\iffalse
\begin{subfigure}
  \centering
  \includegraphics[width=0.5\linewidth]{ESN_RNN_normalizedseq/RNN_2.png}
%          \vspace{-2\baselineskip}
  \caption{Data dimension: 2}
\end{subfigure}
\begin{subfigure}
  \centering
  \includegraphics[width=0.5\linewidth]{ESN_RNN_normalizedseq/RNN_4.png}
%\vspace{-2\baselineskip}
  \caption{Data dimension: 4}
\end{subfigure}%\hfill
\begin{subfigure}
  \centering
  \includegraphics[width=0.5\linewidth]{ESN_RNN_normalizedseq/RNN_8.png}
%\vspace{-2\baselineskip}
  \caption{Data dimension: 8}
\end{subfigure}%
%\fi
\caption{Invertibitiliy of RNNs at random initialization: Checking behavior of inversion error with number of neurons and the sequence length at different data dimensions.}
\label{fig:RNN_inver}
\end{figure}
%\fi 
\textbf{RNN inversion at random initialization.} We consider a randomly initialized RNN, with the entries of the weights $\mathbf{W}$ and $\mathbf{A}$ randomly picked from the distribution $\mathcal{N}(0, 1)$. Sequences are generated i.i.d. from normal distribution i.e. for each sequence, $\bx^{(i)} \sim N(0, \mathbf{I})$ for each $i \in [L]$. We use SGD with batch size 128, momentum $0.9$ and learning rate $0.1$ to compute the linear matrix $\obW^{[L]}$ so that $\norm[0]{\obW^{[L]} \mathbf{h}^{(L)} - [\bx^{(1)}, \ldots, \bx^{(L)}]}^2$ is minimized. We compute the following two quantities on the test dataset, containing $1000$ sequences: average $L_2$ error given by $\mathbb{E}_{\bx} \frac{\norm[0]{\obW^{[L]} \mathbf{h}^{(L)} - [\bx^{(1)}, \ldots, \bx^{(L)}]}}{\norm[0]{[\bx^{(1)}, \ldots, \bx^{(L)}]}}$ and average $L_\infty$ error given by $\mathbb{E}_{\bx} \norm[0]{\obW^{[L]} \mathbf{h}^{(L)} - [\bx^{(1)}, \ldots, \bx^{(L)}]}_{\infty}$. We plot both the quantities for different settings of data dimension $d$, sequence length $L$ and the number of neurons $m$. $L$ takes values from the set $\{2, 4, 6\}$, $d$ takes from $\{2, 4, 8\}$ and $m$ takes from $\{500, 1000, 2000, 5000, 10000\}$ (Figure~\ref{fig:RNN_inver}). The trends support our bounds in Theorem~\ref{thm:Invertibility_ESN_outline}, i.e. the error increases with increasing $L$ and decreases with increasing $m$. Note that the data distribution is different from the one assumed in normalized sequence Def.~\ref{def:normalized_seq}. It was easier to conduct experiments in the current data setting and a similar statement as Thm.~\ref{thm:Invertibility_ESN_outline} can be given.



\textbf{Performance of RNNs on different regular languages. } We check the performance of RNNs on the formal language recognition task for a wide variety of regular languages. We follow the set-up in \cite{BhattamishraAG20} who conducted experiments on LSTMs etc. but not on RNNs.

%\section{Regular languages} \label{sec:regular}
We consider the regular languages as considered in \cite{BhattamishraAG20}.
Tomita grammars \cite{tomita:cogsci82} contain 7
regular languages representable by DFAs of small
sizes, a popular benchmark for evaluating recurrent models (see references in \cite{BhattamishraAG20}). 
We reproduce the definitions of the Tomita grammars from there verbatim:
Tomita Grammars are 7 regular langauges defined on the alphabet $\Sigma = \{0, 1\}$.
Tomita-1 has the regular expression $1^\ast$.
Tomita-2 is defined by the regular expression $(10)^\ast$.
Tomita-3 accepts the strings where odd number
of consecutive 1s are always followed by an even
number of $0$'s. Tomita-4 accepts the strings that
do not contain three consecutive $0$'s. In Tomita-5 only
the strings containing an even number of $0$'s and
even number of $1$'s are allowed. In Tomita-6 the
difference in the number of $1$'s and $0$'s should be
divisible by 3 and finally, Tomita-7 has the regular
expression $0^\ast 1^\ast 0^\ast 1^\ast$. 

We also check the performance of RNNs on $\mathrm{Parity}$, which contains all languages with strings of the form $(w_1, \ldots, w_L)$ s.t. $w_1 + \ldots + w_L = 1 \mod 2$. Languages $\mathcal{D}_n$ are recursively defined as the set of all strings of the form $(0w1)^{\ast}$, where $w \in \mathcal{D}_{n-1}$, with $\mathcal{D}_0$ containing only $\epsilon$, the empty word. Other languages considered are $(00)^{\ast}$, $(0101)^{\ast}$ and $(00)^{\ast}(11)^{\ast}$. Table~\ref{table:regular} shows the number of examples in train and test data, the range of the length of the strings in the language, and the test accuracy of the RNNs with activation functions $\relu$ and $\tanh$ on the regular languages mentioned above. 

\begin{center}
\begin{table}[!ht]
\centering
\begin{tabular}{|| c | c | c | c | c ||} 
 \hline
 Task & No. of Training/Test examples  & Range of length of strings & RNN(Relu) & RNN(Tanh)  \\ [0.5ex] 
 \hline\hline
 Tomita 1 & 50/100 & [2, 50] & 1.0 & 1.0 \\
 Tomita 2 & 25/50 & [2, 50] & 1.0 & 1.0 \\
 Tomita 3 & 10000/2000 & [2, 50] & 1.0 & 1.0 \\
 Tomita 4 & 10000/2000 & [2, 50] & 1.0 & 1.0 \\
 Tomita 5 & 10000/2000 & [2, 50] & 1.0 & 1.0\\
 Tomita 6 & 10000/2000 & [2, 50] & 1.0 & 1.0\\
 Tomita 7 & 10000/2000 & [2, 50] & 0.259 & 0.99\\
 Parity & 10000/2000 & [2, 50] & 1.0 & 1.0\\
 $\mathcal{D}_2$ & 10000/2000 & [2, 100] & 1.0 & 1.0 \\
 $\mathcal{D}_3$ & 10000/2000 & [2, 100] & 0.99 & 1.0\\
 $\mathcal{D}_4$ & 10000/2000 & [2, 100] & 1.0 & 0.99\\
 $(00)^{\ast}$ & 250/50 & [2, 500] & 1.0 & 1.0\\
 $(0101)^{\ast}$ & 125/25 & [4, 500] & 0.99 & 1.0 \\
 $(00)^{\ast}(11)^{\star}$ & 10000/2000 & [2, 200] & 0.99 & 1.0 
 %\\Dyck-1 & 0.99 & \textbf{0.91} 
 \\[1ex]
 \hline
\end{tabular}
\caption{Performance of RNNs on different regular languages.}
\label{table:regular}
\end{table}
\end{center}

%The test languages consist of Tomita grammars which constitute a popular benchmark, and a number of other languages including $\mathsf{PARITY}$ and cover a variety of capabilities needed to recognize regular languages.
%The details of the regular languages above can be found in \cite{BhattamishraAG20}; we only note that Tomita languages constitute a popular benchmark. 
We vary $m$, the dimension of the hidden state, in the range $[3, 32]$, used RMSProp optimizer~\cite{hinton2014coursera} with the smoothing constant $\alpha = 0.99$ and varied the learning rate in the range $[10^{-2}, 10^{-3}]$. For each language
we train models corresponding to each language
for $100$ epochs and a batch size of $32$. We experimented with two different activations $\relu$ and $\tanh$. 
%The best test accuracies achieved on different languages are given in table~\ref{table:regular} and these were all achieved for $m=32$.
In all but one case (Tomita 7 with ReLU) the test accuracies with near-perfect. This was the case across runs. Tomita 7 results could perhaps be improved by more extensive hyperparameter tuning. 
We train and test on strings of length up to 50, and in a few cases strings of larger lengths (when the number of strings in the language is small). 
%Details are in Appendix~\ref{sec:regular}.




\iffalse
\begin{figure*}

\centering
\includegraphics[width=.45\textwidth]{../ESN_RNN_normalizedseq/RNN_2.png}
\includegraphics[width=.45\textwidth]{../ESN_RNN_normalizedseq/RNN_4.png}
\includegraphics[width=.5\textwidth]{../ESN_RNN_normalizedseq/RNN_8.png}
\caption{Invertibility of RNNs at initialization}
%\label{fig:figure3}
\end{figure*}
%\FloatBarrier


\begin{center}
\begin{table}[!ht]
\centering
\begin{tabular}{|| c | c | c ||} 
 \hline
 Task & RNN(Relu) & RNN(Tanh)  \\ [0.5ex] 
 \hline\hline
 Tomita 1 & 1.0 & 1.0 \\
 Tomita 2 & 1.0 & 1.0 \\
 Tomita 3 & 1.0 & 1.0 \\
 Tomita 4 & 1.0 & 1.0 \\
 Tomita 5 & 1.0 & 1.0\\
 Tomita 6 & 1.0 & 1.0\\
 Tomita 7 & 0.259 & 0.99\\
 Parity & 1.0 & 1.0\\
 $\mathcal{D}_1$ & 1.0 & 1.0 \\
 $\mathcal{D}_2$ & 0.99 & 1.0\\
 $\mathcal{D}_4$ & 1.0 & 0.99\\
 $(aa)^{\ast}$ & 1.0 & 1.0\\
 $(abab)^{\ast}$ & 0.99 & 1.0 \\
 $(aa)^{\ast}(bb)^{\star}$ & 0.99 & 1.0 
 %\\Dyck-1 & 0.99 & \textbf{0.91} 
 \\[1ex]
 \hline
\end{tabular}
\caption{Performance of RNNs on different regular languages.}
\label{table:regular}
\end{table}
\end{center}
\fi
