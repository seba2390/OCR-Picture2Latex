\section{Existence of good pseudo network: proofs}
\subsection{Proof of theorem~\ref{thm:existence_pseudo}}
\begin{definition}[Restating defintion~\ref{def:existence}] 
	Define $\mathbf{W}^{\ast}$ and $\mathbf{A}^{\ast}$ as follows.
	\begin{align*}
		\mathbf{W}^{\ast} &= 0 \\
		\mathbf{a}^{*}_{r} &= \frac{\dout }{m} \sum_{s \in [\dout ]} \sum_{r' \in [p]} b_{r, s} b_{r', s}^{\dagger} H_{r', s} \left(\theta_{r', s} \left(\langle \mathbf{w}_{r}, \overline{\mathbf{W}}^{[L]} \mathbf{w}_{r', s}^{\dagger}\rangle\right), \sqrt{m/2} a_{r, d}\right) \mathbf{e}_d, \quad \forall r \in [m],
	\end{align*}
	where
	\begin{equation*} 
		\theta_{r', s} = \frac{\sqrt{m/2}}{\norm[1]{ \overline{\mathbf{W}}^{[L]} \mathbf{w}_{r', s}^{\dagger}}},
	\end{equation*}
	and $\obW^{[L]} = [\mathbf{W}^{(L, 3)} \mathbf{D}_{(0)}^{(2)}\mathbf{A}_{[d-1]},  \cdots, \mathbf{W}^{(L, L)} 
	\mathbf{D}_{(0)}^{(L-1)}\mathbf{A}_{[d-1]}]_r$, where $\mathbf{W}^{(k_b, k_e)} = \prod_{k_b \ge \ell > k_e} \mathbf{D}_{(0)}^{(\ell)} \mathbf{W}$.
	%Here $\mathbf{P}$ denotes a diagonal matrix that projects a vector onto its top $L(d-1)$ coordinates and  $\mathbf{P}^{\perp} = \mathbf{I} - \mathbf{P}$.
\end{definition}
%\todo{There is a dimension mismatch. Pls correct later on.}
Using Lemma~\ref{lemma:norm_ESN}, we have with probability at least $ 1-e^{-\Omega(\rho^2)}$, for all $\ell \in [L]$ and any vector $\mathbf{u} \in \mathbb{R}^{d}$
\begingroup \allowdisplaybreaks
\begin{align*}
	\left(1 - \frac{1}{100L}\right)^{L} \norm{\mathbf{u}} \le \norm[2]{\mathbf{W}^{(L, \ell+1)} \mathbf{D}^{(\ell)}_{(0)} \mathbf{A} \mathbf{u}} = \norm[2]{\prod_{L \ge \ell' \ge \ell+1} \mathbf{D}_{(0)}^{(\ell')} \mathbf{W} \mathbf{D}_{(0)}^{(\ell)} \mathbf{A} \mathbf{u}} \le \left(1 + \frac{1}{100L}\right)^{L} \norm{\mathbf{u}}.
\end{align*} 
\endgroup
Since, for any vector $\mathbf{u} \in \mathbb{R}^{Ld}$,
\begin{align*}
	\min_{\ell \in [L]} \frac{\norm[2]{\mathbf{W}^{(L, \ell+1)} \mathbf{D}^{(\ell)}_{(0)} \mathbf{A} \mathbf{u}_{\ell d : (\ell + 1)d}}}{\norm[2]{\mathbf{u}_{\ell d : (\ell + 1)d}}} \norm[2]{\mathbf{u}} \le \norm[2]{\mathbf{W}^{[L]} \mathbf{u}} \le \max_{\ell \in [L]} \frac{\norm[2]{\mathbf{W}^{(L, \ell+1)} \mathbf{D}^{(\ell)}_{(0)} \mathbf{A} \mathbf{u}_{\ell d : (\ell + 1)d}}}{\norm[2]{\mathbf{u}_{\ell d : (\ell + 1)d}}} \norm[2]{\mathbf{u}},
\end{align*}
where $\mathbf{u}_{\ell d : (\ell + 1)d} \in \mathbb{R}^{Ld}$ refers to a vector that is equal to the vector $\mathbf{u}$ in the dimensions from $\ell d$ to $(\ell + 1)d$ and $0$ outside, we have
\begin{align*}
	\left(1 - \frac{1}{100L}\right)^{L} \norm[2]{\mathbf{w}_{r', s}^{\dagger}} \le  \norm[2]{\overline{\mathbf{W}}^{[L]} \mathbf{w}_{r', s}^{\dagger}} \le \left(1 + \frac{1}{100L}\right)^{L} \norm[2]{\mathbf{w}_{r', s}^{\dagger}}
\end{align*}
and thus we have $\forall  r' \in [p], s \in [\dout ]$,
\begin{align}
	\left(1 + \frac{1}{100L}\right)^{-L}  \le \sqrt{2/m} \theta_{r', s} = \left(1 - \frac{1}{100L}\right)^{-L}  \label{eq:abstheta_lowerbound}.
\end{align}



\begin{theorem}[Restating theorem~\ref{thm:existence_pseudo}]\label{thm:existence_pseudo_proof}
	The construction of $\mathbf{W}^{*}$ and $\mathbf{A}^{\ast}$ in Definition~\ref{def:existence} satisfies the following. For every normalized input sequence $\mathbf{x}^{(1)}, \cdots, \mathbf{x}^{(L)}$, we have with probability at least $1-e^{-\Omega\left(\rho^{2}\right)}$ over $\mathbf{W}, \mathbf{A}, \mathbf{B},$ it holds for every $s \in [\dout ]$.
	$$
	\begin{array}{l}
		F_{s}^{(L)} \stackrel{\text { def }}{=} \sum_{i=1}^{L} \mathbf{e}_{s}^{\top} \mathbf{Back}_{i \rightarrow L} D^{(i)} \left(\mathbf{W}^{\ast} \mathbf{h}^{(i-1)} + \mathbf{A}^{\ast} \mathbf{x}^{(i)}\right) \\
		= \sum_{r \in[p]} b_{r, s}^{\dagger} \Phi_{r, s} \left(\left\langle  \mathbf{w}_{r, s}^{\dagger}, [\overline{\mathbf{x}}^{(2)}, \cdots, \overline{\mathbf{x}}^{(L-2)}]\right\rangle\right)\\ \pm \mathcal{O}(\dout Lp\rho^2 \varepsilon + \dout L^{17/6} p \rho^4 L_{\Phi} \varepsilon_x^{2/3} + \dout ^{3/2}L^5 p \rho^{11} L_{\Phi} C_{\Phi}  \mathfrak{C}_{\varepsilon}(\Phi, \mathcal{O}(\varepsilon_x^{-1}))  m^{-1/30} ).
		%\left(p \rho^{11} \cdot O\left(\varepsilon_{e}+\mathfrak{C}_{s}(\Phi, 1) \varepsilon_{x}^{1 / 3}+C m^{-0.05}\right)\right)
	\end{array}
	$$
	%We partition the set of $m$ input neurons into $p$ non-overlapping sets.
	%hen, define the weights $\left\{\mathbf{w}^{*}_{k + \frac{lm}{p}}\right\}_{k \in \left[\frac{m}{p}\right], l \in [p]}$ as follows.
	%Construct a matrix $\mathbf{C} \in \mathbb{R}^{p \times m}$ as follows. For each of the functions $f_i(\mathbf{x}) = \phi(\widetilde{\mathbf{a}}_i^{\top} \mathbf{x})$, we can construct vectors $\widetilde{\mathbf{c}}_i$ such that
	%\begin{equation*}
	%   \sup_{\mathbf{x \in \mathcal{X}}} \abs{f_i\left(\mathbf{x}\right) - \sum_{r = i  \frac{m}{2p} }^{(i + 1)  \frac{m}{2p}} \widetilde{c}_{i, r} \sigma\left(\mathbf{a}_r^{(0)\top} \mathbf{x}\right)} \le \varepsilon_i
	%\end{equation*} 
	%holds true with probability at-least $1 - \delta_i$, using \autoref{Thm:Rep_power}. Fix the rows of $\mathbf{C}$ as
	%\begin{align*}
	%   c_{ij} &= \widetilde{c}_{ij}, \quad \text{ if } i  \frac{m}{2p} \le j <  (i + 1)  \frac{m}{2p} \\ &=0 \quad \quad \text{otherwise}, \quad \forall i \in [p].
	%\end{align*}
	%Now,  define the weights $\left\{\mathbf{a}^{*}_{r}\right\}_{r \in [\frac{m}{2}, m-1]}$ as follows.
	%\begin{equation*}
	%\mathbf{a}^{*}_{r} = \frac{2}{m} \sum_{s \in [\dout ]} \sum_{r' \in [p]} b_{r, s} b_{r', s}^{\dagger} H\left(\theta_{r'} \left(\langle \mathbf{w}_{r}, \mathbf{C}^T \mathbf{\widetilde{w}}_{r'}\rangle + \langle \mathbf{a}_{r}^{[d-1]},  \mathbf{\widetilde{a}}_{r'}  \rangle\right), \theta_{r'}  b_{r, s}\right) \mathbf{e}_d,
	%\end{equation*}
	%where $\theta_{r'}$ is given by
	%\begin{equation*} 
	%\theta_{r'} = \frac{\sqrt{m}}{\sqrt{ \norm{\mathbf{a}_{r'}}^2 + \norm{\mathbf{C}^T \mathbf{\widetilde{w}}_{r'}}^2 }}.
	%\end{equation*}
	%Set $\left\{\mathbf{w}_{r}^*\right\}_{r \in [m]}$ and $\left\{\mathbf{a}^{*}_r\right\}_{r \in [\frac{m}{2} - 1]}$ as zero vectors.
	%For each $l \in [p]$, $\mathbf{\widetilde{a'}}_l\in \mathbb{R}^{m}$ is defined by 
	%\begin{equation*}
	%\widetilde{a'}_{l, i + \gamma \frac{m}{p}} =  \sqrt{\frac{\dout }{m}}\widetilde{a}_{l, \gamma}, \quad \forall \gamma \in [p], i \in \left[\frac{m}{p}\right].
	%\end{equation*}  
\end{theorem}




\begin{proof}
	%Using \autoref{Thm:Smooth_H}, we have
	We fix a given normalized sequence $\mathbf{x}^{(1)}, \cdots, \mathbf{x}^{(L)}$ and an index $s \in [\dout ]$.
	The pseudo network for the fixed sequence is given by
	\begingroup \allowdisplaybreaks
	\begin{align}
		F_{s}^{(L)} &= \sum_{i=1}^{L} \mathbf{e}_{s}^{\top} \mathbf{Back}_{i \rightarrow L} D^{(i)} \left(\mathbf{W}^{\ast} \mathbf{h}^{(i-1)} + \mathbf{A}^{\ast} \mathbf{x}^{(i)}\right) \nonumber\\
		&= \frac{\dout }{m} \sum_{i=1}^{L}  \sum_{s' \in [\dout ]} \sum_{r' \in [p]} \sum_{r \in [m]}  b_{r, s'} b_{r', s'}^{\dagger} \mathbf{Back}_{i \to L, r, s} \nonumber\\& \quad \quad \quad \quad H_{r', s'}\Big(\theta_{r', s'} \langle \mathbf{w}_{r}, \overline{\mathbf{W}}^{[L]} \mathbf{w}_{r', s'}^{\dagger}\rangle , \sqrt{m/2} a_{r, d}\Big) \mathbb{I}_{\mathbf{w}_r^{\top} \mathbf{h}^{(i-1)} + \mathbf{a}_r^{\top} \mathbf{x}^{(i)} \ge 0} %\nonumber
		%\\&
		%= \frac{\dout }{m} \sum_{i=1}^{L}  \sum_{s' \in [\dout ]: s' \ne s} \sum_{r' \in [p]} \sum_{r \in [m]}  b_{r, s'} b_{r', s'}^{\dagger} \mathbf{Back}_{i \to L, r, s} \nonumber\\& \quad \quad \quad \quad H_{r', s'}\Big(\theta_{r', s'} \langle \mathbf{w}_{r}, \overline{\mathbf{W}}^{[L]} \mathbf{w}_{r', s'}^{\dagger}\rangle , \sqrt{m/2} a_{r, d}\Big) \mathbb{I}_{\mathbf{w}_r^{\top} \mathbf{h}^{(i-1)} + \mathbf{a}_r^{\top} \mathbf{x}^{(i)} \ge 0}\label{eq:sprimenes} \\&
		%+ \frac{\dout }{m} \sum_{i=1}^{L-1}  \sum_{r' \in [p]} \sum_{r \in [m]}  b_{r, s} b_{r', s}^{\dagger} \mathbf{Back}_{i \to L, r, s} \nonumber\\& \quad \quad \quad \quad H_{r', s}\Big(\theta_{r', s} \langle \mathbf{w}_{r}, \overline{\mathbf{W}}^{[L]} \mathbf{w}_{r', s}^{\dagger}\rangle , \sqrt{m/2} a_{r, d}\Big) \mathbb{I}_{\mathbf{w}_r^{\top} \mathbf{h}^{(i-1)} + \mathbf{a}_r^{\top} \mathbf{x}^{(i)} \ge 0}\label{eq:sprimeesL-1} \\&
		%+ \frac{\dout }{m}   \sum_{r' \in [p]} \sum_{r \in [m]}  b_{r, s} b_{r', s}^{\dagger} \mathbf{Back}_{L \to L, r, s} \nonumber\\& \quad \quad \quad \quad H_{r', s}\Big(\theta_{r', s} \langle \mathbf{w}_{r}, \overline{\mathbf{W}}^{[L]} \mathbf{w}_{r', s}^{\dagger}\rangle , \sqrt{m/2} a_{r, d}\Big) \mathbb{I}_{\mathbf{w}_r^{\top} \mathbf{h}^{(L-1)} + \mathbf{a}_r^{\top} \mathbf{x}^{(L)} \ge 0}\label{eq:sprimeesL}
	\end{align}
	\endgroup
	First of all, we can't show that the above formulation concentrates on the required signal, because of the dependencies of randomness between $\mathbf{W}$, $\mathbf{A}$, $\mathbf{Back}$, $\mathbf{\overline{\mathbf{W}}}^{[L]}$ and $\left\{\mathbf{h}^{(\ell)}\right\}_{\ell \in [L]}$.  To decouple this randomness, we use the fact that ESNs are stable to re-randomization of few rows of the weight matrices and follow the proof technique of Lemma G.3 in \cite{allen2019can}. Choose a random subset $\mathcal{K} \subset[m]$ of size $|\mathcal{K}|=N$. Define the function $F^{(L), \mathcal{K}}_s$ as
	\begin{align*}
		F_{s}^{(L), \mathcal{K}}(\mathbf{h}^{(L-1)}, \mathbf{x}^{(L)}) &\stackrel{\text { def }}{=} \frac{\dout }{m} \sum_{i=1}^{L}  \sum_{s' \in [\dout ]} \sum_{r' \in [p]} \sum_{r \in \mathcal{K}}  b_{r, s'} b_{r', s'}^{\dagger} \mathbf{Back}_{i \to L, r, s} \nonumber\\& \quad \quad \quad \quad H_{r', s'}\Big(\theta_{r', s'} \langle \mathbf{w}_{r}, \overline{\mathbf{W}}^{[L]} \mathbf{w}_{r', s'}^{\dagger}\rangle , \sqrt{m/2} a_{r, d}\Big) \mathbb{I}_{\mathbf{w}_r^{\top} \mathbf{h}^{(i-1)} + \mathbf{a}_r^{\top} \mathbf{x}^{(i)} \ge 0}.  
	\end{align*}
	
	We show the following claim.
	\begin{claim}\label{claim:singlesubset_generalization}
		With probability at least $1-e^{-\Omega(\rho^2)}$, for any $\varepsilon \in (0, \min_{r, s} \frac{1}{C_s(\Phi_{r, s}, \mathcal{O}(\varepsilon_x^{-1}) ) })$,
		\begin{align*}      &\abs{F^{(L), \mathcal{K}}_s(\mathbf{h}^{(L-1)}, \mathbf{x}^{(L)}) - \frac{\dout }{m} \sum_{i=1}^{L}  \sum_{s' \in [\dout ]} \sum_{r' \in [p]} \sum_{r \in \mathcal{K}}  b_{r, s'} b_{r', s'}^{\dagger} \mathbf{Back}_{i \to L, r, s} \Phi_{r', s} \left(\left\langle \mathbf{w}_{r', s}^{\dagger}, [\overline{\mathbf{x}}^{(1)}, \cdots, \overline{\mathbf{x}}^{(L)}]\right\rangle\right)} \\&\le  \mathcal{O}(\dout Lp\rho^8  \mathfrak{C}_{\varepsilon}(\Phi, \mathcal{O}(\varepsilon_x^{-1})) N^{5/3} m^{-7/6}) + \frac{\dout }{m} \cdot \mathcal{O}(\mathfrak{C}_{\varepsilon}(\Phi_{r' s}, \mathcal{O}(\varepsilon_x^{-1})) \rho^2 \sqrt{\dout LpN}) \\& + \frac{\dout LpN}{m} \rho^2 (\varepsilon + \mathcal{O}( L_{\Phi}\rho^5 (N/m)^{1/6}) + \mathcal{O}(\varepsilon_x^{-1} L_{\Phi} L^4 \rho^{11} m^{-1/12} +  L_{\Phi} \rho^2 L^{11/6} \varepsilon_x^{2/3})) + \mathcal{O}(\rho^8\dout Lp N^{7/6} m^{-7/6}) .
		\end{align*}
	\end{claim}
	The above claim has been restated and proven in claim~\ref{claim:singlesubset_generalization_proof}. The above claim states that the function $F^{(L), \mathcal{K}}_s(\mathbf{h}^{(L-1)}, \mathbf{x}^{(L)})$ contains information about the true function.
	
	To complete the proof, we divide the set of neurons into $m/N$ disjoint sets $\mathcal{K}_1, \cdots, \mathcal{K}_{m/N}$, each set is of size $N$. We apply the Claim~\ref{claim:singlesubset_generalization} to each subset $\mathcal{K}_i$ and then add up the errors from each subset. That is, with probability at least $1 - \frac{m}{N}e^{-\Omega(\rho^2)}$,
			\begingroup \allowdisplaybreaks
			\begin{align*}
				&F_s^{(L)}(\mathbf{h}^{(\ell-1)}, \mathbf{x}^{(\ell)}) \\&=  \sum_{j=1}^{m/N}  F_s^{(L),\mathcal{K}_i}(\mathbf{h}^{(\ell-1)}, \mathbf{x}^{(\ell)})\\
				&= \sum_{j=1}^{m/N} \frac{\dout }{m} \sum_{i=1}^{L}  \sum_{s' \in [\dout ]} \sum_{r' \in [p]} \sum_{r \in \mathcal{K}_j}  b_{r, s'} b_{r', s'}^{\dagger} \mathbf{Back}_{i \to L, r, s} \Phi_{r', s} \left(\left\langle \mathbf{w}_{r', s}^{\dagger}, [\overline{\mathbf{x}}^{(2)}, \cdots, \overline{\mathbf{x}}^{(L-1)}]\right\rangle\right) 
				+ \sum_{j=1}^{m/N} error_{\mathcal{K}_j} \\&
				=  \frac{\dout }{m} \sum_{i=1}^{L}  \sum_{s' \in [\dout ]} \sum_{r' \in [p]} \sum_{r \in [m]}  b_{r, s'} b_{r', s'}^{\dagger} \mathbf{Back}_{i \to L, r, s} \Phi_{r', s} \left(\left\langle \mathbf{w}_{r', s}^{\dagger}, [\overline{\mathbf{x}}^{(2)}, \cdots, \overline{\mathbf{x}}^{(L-1)}]\right\rangle\right) 
				+ \sum_{j=1}^{m/N} error_{\mathcal{K}_j}, 
			\end{align*}
			where by Claim~\ref{claim:singlesubset_generalization},
			\begin{align*}
				\abs{error_{\mathcal{K}_i}} &\le \mathcal{O}(\dout Lp\rho^8  \mathfrak{C}_{\varepsilon}(\Phi, \mathcal{O}(\varepsilon_x^{-1})) N^{5/3} m^{-7/6}) + \frac{\dout }{m} \cdot \mathcal{O}(\mathfrak{C}_{\varepsilon}(\Phi_{r' s}, \mathcal{O}(\varepsilon_x^{-1})) \rho^2 \sqrt{\dout LpN}) + \\& + \frac{\dout LpN}{m} \rho^2 (\varepsilon + \mathcal{O}( L_{\Phi}\rho^5 (N/m)^{1/6}) + \mathcal{O}(\varepsilon_x^{-1} L_{\Phi} L^4 \rho^{11} m^{-1/12} +  L_{\Phi} L^{11/6} \rho^2 \varepsilon_x^{2/3})) + \\& \mathcal{O}(\rho^8\dout Lp N^{7/6} m^{-7/6}).
			\end{align*}
			\endgroup
			Thus,
			\begin{align*}
				&\abs{ F_s^{(L)}(\mathbf{h}^{(\ell-1)}, \mathbf{x}^{(\ell)}) - \frac{\dout }{m} \sum_{i=1}^{L}  \sum_{s' \in [\dout ]} \sum_{r' \in [p]} \sum_{r \in [m]}  b_{r, s'} b_{r', s'}^{\dagger} \mathbf{Back}_{i \to L, r, s} \Phi_{r', s} \left(\left\langle \mathbf{w}_{r', s}^{\dagger}, [\overline{\mathbf{x}}^{(2)}, \cdots, \overline{\mathbf{x}}^{(L-1)}]\right\rangle\right) } \\& \le \mathcal{O}(\dout Lp\rho^8  \mathfrak{C}_{\varepsilon}(\Phi, \mathcal{O}(\varepsilon_x^{-1})) N^{2/3} m^{-1/6}) + \frac{\sqrt{\dout ^3Lp}}{\sqrt{N}} \cdot \mathcal{O}(\mathfrak{C}_{\varepsilon}(\Phi_{r' s}, \mathcal{O}(\varepsilon_x^{-1})) \rho^2 )  \\& + \dout Lp \rho^2 (\varepsilon + \mathcal{O}( L_{\Phi}\rho^5 (N/m)^{1/6}) + \mathcal{O}(\varepsilon_x^{-1} L_{\Phi} L^4 \rho^{11} m^{-1/12} +  L_{\Phi} L^{11/6} \rho^2 \varepsilon_x^{2/3})) + \mathcal{O}(\rho^8\dout Lp N^{1/6} m^{-1/6}),
			\end{align*}
			with probability at least $1 - \frac{m}{N}e^{-\Omega(\rho^2)}$.
			Choosing $N = m^{0.2}$, we have
			\begin{align}
				&\abs{ F_s^{(L)}(\mathbf{h}^{(\ell-1)}, \mathbf{x}^{(\ell)}) - \frac{\dout }{m} \sum_{i=1}^{L}  \sum_{s' \in [\dout ]} \sum_{r' \in [p]} \sum_{r \in [m]}  b_{r, s'} b_{r', s'}^{\dagger} \mathbf{Back}_{i \to L, r, s} \Phi_{r', s} \left(\left\langle \mathbf{w}_{r', s}^{\dagger}, [\overline{\mathbf{x}}^{(2)}, \cdots, \overline{\mathbf{x}}^{(L-1)}]\right\rangle\right) }\nonumber \\& \le \mathcal{O}(\dout Lp\rho^8  \mathfrak{C}_{\varepsilon}(\Phi, \mathcal{O}(\varepsilon_x^{-1}))  m^{-1/30}) + \mathcal{O}(\mathfrak{C}_{\varepsilon}(\Phi_{r' s}, \mathcal{O}(\varepsilon_x^{-1})) \rho^2 \sqrt{\dout ^3Lp} m^{-0.1})  \\& + \dout Lp \rho^2 (\varepsilon + \mathcal{O}( L_{\Phi}\rho^5 m^{-2/15}) + \mathcal{O}(\varepsilon_x^{-1} L_{\Phi} L^4 \rho^{11} m^{-1/12} +  L_{\Phi} L^{11/6} 
				\rho^2\varepsilon_x^{2/3})) + \mathcal{O}(\rho^8\dout Lp  m^{-2/15}) \label{eq:prefinalf},
			\end{align}
			with probability at least $1 -  m^{0.8} e^{-\Omega(\rho^2)} \ge 1 - e^{-\Omega(\rho^2)}$.  
			
			Now, in the next claim, we show that the $f$ concentrates on the desired term. 
			\begin{claim}\label{claim:simplifybig}
				With probability exceeding $1 - e^{-\Omega(\rho^2)}$,
				\begin{align*}
					&\Big|  b_{r', s}^{\dagger}  \Phi_{r', s} \left(\left\langle \mathbf{w}_{r', s}^{\dagger}, [\overline{\mathbf{x}}^{(2)}, \cdots, \overline{\mathbf{x}}^{(L-1)}]\right\rangle\right) \\& \quad \quad \quad - \frac{\dout }{m} \sum_{i=1}^{L}  \sum_{s' \in [\dout ]}  \sum_{r \in [m]}  b_{r, s'} b_{r', s'}^{\dagger} \mathbf{Back}_{i \to L, r, s} \Phi_{r', s} \left(\left\langle \mathbf{w}_{r', s}^{\dagger}, [\overline{\mathbf{x}}^{(2)}, \cdots, \overline{\mathbf{x}}^{(L-1)}]\right\rangle\right) \Big| \\&\le  \mathcal{O}(L\dout  \rho C_{\Phi} m^{-0.25}).
				\end{align*}
			\end{claim}
			The claim is restated and proven in claim~\ref{claim:simplifybig_proof}.	
				
			Thus, introducing claim~\ref{claim:simplifybig} in eq.~\ref{eq:prefinalf}, we have
			\begingroup \allowdisplaybreaks
			\begin{align*}
				&\abs{ F_s^{(L)}(\mathbf{h}^{(\ell-1)}, \mathbf{x}^{(\ell)}) - \sum_{r' \in [p]}  b_{r', s}^{\dagger} \Phi_{r', s} \left(\left\langle \mathbf{w}_{r', s}^{\dagger}, [\overline{\mathbf{x}}^{(2)}, \cdots, \overline{\mathbf{x}}^{(L-1)}]\right\rangle\right) } \\& \le \mathcal{O}(\dout Lp\rho^8  \mathfrak{C}_{\varepsilon}(\Phi, \mathcal{O}(\varepsilon_x^{-1}))  m^{-1/30}) + \mathcal{O}(\mathfrak{C}_{\varepsilon}(\Phi_{r' s}, \mathcal{O}(\varepsilon_x^{-1})) \rho^2 \sqrt{\dout ^3Lp} m^{-0.1}) \\& + \dout Lp \rho^2 (\varepsilon + \mathcal{O}( L_{\Phi}\rho^5 m^{-2/15}) + \mathcal{O}(\varepsilon_x^{-1} L_{\Phi} L^4 \rho^{11} m^{-1/12} +  L_{\Phi} L^{11/6} \rho^2 \varepsilon_x^{2/3})) \\& + \mathcal{O}(\rho^8\dout Lp  m^{-2/15})  + \mathcal{O}(Lp\dout  \rho C_{\Phi} m^{-0.25}) \\&
				\le \mathcal{O}(\dout Lp\rho^2 \varepsilon + \dout L^{17/6} p \rho^4 L_{\Phi} \varepsilon_x^{2/3} + \dout ^{3/2} L^5 p \rho^{11} L_{\Phi} C_{\Phi}  \mathfrak{C}_{\varepsilon}(\Phi, \mathcal{O}(\varepsilon_x^{-1}))  m^{-1/30} ) .
			\end{align*}
			\endgroup
		\end{proof}





\subsection{Proof of Claim~\ref{claim:singlesubset_generalization}}

\begin{claim}[Restating claim~\ref{claim:singlesubset_generalization}]\label{claim:singlesubset_generalization_proof}
	With probability at least $1-e^{-\Omega(\rho^2)}$, for any $\varepsilon \in (0, \min_{r, s} \frac{1}{C_s(\Phi_{r, s}, \mathcal{O}(\varepsilon_x^{-1}) ) })$,
	\begin{align*}      &\abs{F^{(L), \mathcal{K}}_s(\mathbf{h}^{(L-1)}, \mathbf{x}^{(L)}) - \frac{\dout }{m} \sum_{i=1}^{L}  \sum_{s' \in [\dout ]} \sum_{r' \in [p]} \sum_{r \in \mathcal{K}}  b_{r, s'} b_{r', s'}^{\dagger} \mathbf{Back}_{i \to L, r, s} \Phi_{r', s} \left(\left\langle \mathbf{w}_{r', s}^{\dagger}, [\overline{\mathbf{x}}^{(1)}, \cdots, \overline{\mathbf{x}}^{(L)}]\right\rangle\right)} \\&\le  \mathcal{O}(\dout Lp\rho^8  \mathfrak{C}_{\varepsilon}(\Phi, \mathcal{O}(\varepsilon_x^{-1})) N^{5/3} m^{-7/6}) + \frac{\dout }{m} \cdot \mathcal{O}(\mathfrak{C}_{\varepsilon}(\Phi_{r' s}, \mathcal{O}(\varepsilon_x^{-1})) \rho^2 \sqrt{\dout LpN}) \\& + \frac{\dout LpN}{m} \rho^2 (\varepsilon + \mathcal{O}( L_{\Phi}\rho^5 (N/m)^{1/6}) + \mathcal{O}(\varepsilon_x^{-1} L_{\Phi} L^4 \rho^{11} m^{-1/12} +  L_{\Phi} \rho^2 L^{11/6} \varepsilon_x^{2/3})) + \mathcal{O}(\rho^8\dout Lp N^{7/6} m^{-7/6}) .
	\end{align*}
\end{claim}

\begin{proof}
We will replace the rows $\left\{\mathbf{w}_{k}, \mathbf{a}_{k}\right\}_{k \in \mathcal{K}}$ of $\mathbf{W}$ and $\mathbf{A}$ with freshly new i.i.d. samples $\widetilde{\mathbf{w}}_{k}, \widetilde{\mathbf{a}}_{k} \sim \mathcal{N}\left(0, \frac{2}{m} \mathbf{I}\right).$ to form new matrices $\widetilde{\mathbf{W}}$ and $\widetilde{\mathbf{A}}$. For the given sequence, we follow the notation of Lemma~\ref{lemma:rerandESN} to denote the hidden states corresponding to the old and the new weight matrices. Let $\widetilde{F}^{(L), \mathcal{K}}_s$ denote the following function:
\begin{align*}
	\widetilde{F}_{s}^{(L), \mathcal{K}}(\widetilde{\mathbf{h}}^{(L-1)}, \mathbf{x}^{(L)}) &\stackrel{\text { def }}{=} \frac{\dout }{m} \sum_{i=1}^{L}  \sum_{s' \in [\dout ]} \sum_{r' \in [p]} \sum_{r \in \mathcal{K}}  b_{r, s'} b_{r', s'}^{\dagger} \widetilde{\mathbf{Back}}_{i \to L, r, s} \nonumber\\& \quad \quad \quad \quad H_{r', s'}\Big(\widetilde{\theta}_{r', s'} \langle \mathbf{w}_{r}, \overline{\widetilde{\mathbf{W}}}^{[L]} \mathbf{w}_{r', s'}^{\dagger}\rangle , \sqrt{m/2} a_{r, d}\Big) \mathbb{I}_{\mathbf{w}_r^{\top} \widetilde{\mathbf{h}}^{(i-1)} + \mathbf{a}_r^{\top} \mathbf{x}^{(i)} \ge 0}, 
\end{align*}
where
\begin{equation*} 
	\widetilde{\theta}_{r', s} = \frac{\sqrt{m/2}}{ \norm[2]{\overline{\widetilde{\mathbf{W}}}^{[L]} \mathbf{w}_{r', s}^{\dagger}}}.
\end{equation*}
Using similar technique used to find the bounds of $\theta_{r', s}$ in eq.~\ref{eq:abstheta_lowerbound}, we ca show that $\forall  r' \in [p], s \in [\dout ]$, w.p. at least $1-e^{-\Omega(\rho^2)}$ over $\widetilde{\mathbf{W}}, \widetilde{\mathbf{A}}$,
\begin{align}
	\left(1 + \frac{1}{100L}\right)^{-L}  \le \sqrt{2/m} \widetilde{\theta}_{r', s} \le  \left(1 - \frac{1}{100L}\right)^{-L}  \label{eq:abstildetheta_lowerbound}.
\end{align}
%\todo{It doesn't directly follow. Add some details here.}
Again, there is one important relation between $\theta_{r', s}$  and $\widetilde{\theta}_{r', s}$ that we will require later on, which we prove in the next claim.
\begin{claim}\label{claim:invtildethetatheta}
	With probability at least $1-e^{-\Omega(\rho^2)}$, for all $r' \in [p], s \in [\dout ]$,
	\begin{align*}
		\abs{\widetilde{\theta}_{r', s} \theta_{r', s}^{-1} - 1} \le \mathcal{O}(\rho^5 (N/m)^{1/6}).
	\end{align*}
\end{claim}
The claim has been restated and proven in claim~\ref{claim:invtildethetatheta_proof}. A simple corollary of the above claim is given below.


\begin{corollary}
	\label{cor:tildetheta_diff_theta}
	With probability at least $1-e^{-\Omega(\rho^2)}$, for all $r' \in [p], s \in [\dout ]$,
	\begin{align*}
		\abs{\widetilde{\theta}_{r', s} - \theta_{r', s}} \le \mathcal{O}(\rho^5 (N/m)^{1/6}).
	\end{align*}
\end{corollary}
The above corollary follows from the bounds on $\theta_{r',s}$ from eq.~\ref{eq:abstheta_lowerbound}.


We will first show that $\widetilde{F}_{s}^{(L), \mathcal{K}}(\widetilde{\mathbf{h}}^{(L-1)}, \mathbf{x}^{(L)})$ and $F_{s}^{(L), \mathcal{K}}(\mathbf{h}^{(L-1)}, \mathbf{x}^{(L)})$ are close. The claim has been restated and proven in claim~\ref{claim:diffftildef_proof}.

\begin{claim}\label{claim:diffftildef}
	With probability at least $1-e^{-\Omega(\rho^2)}$,
	\begingroup\allowdisplaybreaks
	\begin{align*}
		\abs{\widetilde{F}_{s}^{(L), \mathcal{K}}(\widetilde{\mathbf{h}}^{(L-1)}, \mathbf{x}^{(L)}) - F_{s}^{(L), \mathcal{K}}(\mathbf{h}^{(L-1)}, \mathbf{x}^{(L)})} \le \mathcal{O}(\dout Lp\rho^8  \mathfrak{C}_{\varepsilon}(\Phi, \mathcal{O}(\varepsilon_x^{-1})) N^{5/3} m^{-7/6}).
	\end{align*}
	\endgroup
\end{claim}



Now, we show that $\widetilde{F}$ is close to the desired signal in the two claims below. 
	\begin{claim}\label{claim:difftildefphi}
	With probability at least $1-e^{-\Omega(\rho^2)}$,
	\begin{align*}
		\Big|&\widetilde{F}_{s}^{(L), \mathcal{K}}(\widetilde{\mathbf{h}}^{(L-1)}, \mathbf{x}^{(L)}) - \frac{\dout }{m} \sum_{i=1}^{L}  \sum_{s' \in [\dout ]} \sum_{r' \in [p]} \sum_{r \in \mathcal{K}}  b_{r, s'} b_{r', s'}^{\dagger} \widetilde{\mathbf{Back}}_{i \to L, r, s} \Phi_{r', s} \left(\left\langle \mathbf{w}_{r', s}^{\dagger}, [\overline{\mathbf{x}}^{(1)}, \cdots, \overline{\mathbf{x}}^{(L)}]\right\rangle\right)\Big| \\&\le \frac{\dout }{m} \cdot \mathcal{O}(\mathfrak{C}_{\varepsilon}(\Phi_{r' s}, \mathcal{O}(\varepsilon_x^{-1})) \rho^2 \sqrt{\dout LpN}) \\& + \frac{\dout LpN}{m} \rho^2 (\varepsilon + \mathcal{O}( L_{\Phi}\rho^5 (N/m)^{1/6}) + \mathcal{O}(L_{\Phi}\varepsilon_x^{-1}  L^4 \rho^{11} m^{-1/12} +  L_{\Phi} \rho^2 L^{11/6} \varepsilon_x^{2/3})),
	\end{align*}
	for any $\varepsilon \in (0, \min_{r, s} \frac{\sqrt{3}}{C_s(\Phi_{r, s}, \varepsilon_x^{-1})})$.
	%where $\varepsilon' = L_{\Phi} \cdot \mathcal{O}\left(\rho^5 m^{-1/8} \sqrt{L} \right) \cdot \left(2(\sqrt{2} + \rho m^{-0.5})\right)^{L+1} \sqrt{2} (1 + \sqrt{dm^{-1}} + \sqrt{2}\rho m^{-0.5})$.
\end{claim}


\begin{claim}\label{claim:fbacktildeback}
	with probability at least $1 - e^{-\Omega(\rho^2)}$,
	\begin{align*}
		&\Big| \frac{\dout }{m} \sum_{i=1}^{L}  \sum_{s' \in [\dout ]} \sum_{r' \in [p]} \sum_{r \in \mathcal{K}}  b_{r, s'} b_{r', s'}^{\dagger} \widetilde{\mathbf{Back}}_{i \to L, r, s} \Phi_{r', s} \left(\left\langle \mathbf{w}_{r', s}^{\dagger}, [\overline{\mathbf{x}}^{(1)}, \cdots, \overline{\mathbf{x}}^{(L)}]\right\rangle\right) \\& - \frac{\dout }{m} \sum_{i=1}^{L}  \sum_{s' \in [\dout ]} \sum_{r' \in [p]} \sum_{r \in \mathcal{K}}  b_{r, s'} b_{r', s'}^{\dagger} \mathbf{Back}_{i \to L, r, s} \Phi_{r', s} \left(\left\langle \mathbf{w}_{r', s}^{\dagger}, [\overline{\mathbf{x}}^{(1)}, \cdots, \overline{\mathbf{x}}^{(L)}]\right\rangle\right) \Big| \\& \le \mathcal{O}(\rho^8 C_{\Phi} \dout Lp N^{7/6} m^{-7/6}). 
	\end{align*}
\end{claim}
The above two claims have been restated and proven in claim~\ref{claim:difftildefphi_proof} and~\ref{claim:fbacktildeback_proof} respectively.

	Thus, from Claim~\ref{claim:diffftildef}, Claim~\ref{claim:difftildefphi} and Claim~\ref{claim:fbacktildeback}, we have with probability at least $1-e^{-\Omega(\rho^2)}$, for any $\varepsilon \in (0, \min_{r, s} \frac{1}{C_s(\Phi_{r, s}, \mathcal{O}(\varepsilon_x^{-1}) ) })$,
\begingroup \allowdisplaybreaks
\begin{align*}      
	&\abs{F^{(L), \mathcal{K}}_s(\mathbf{h}^{(L-1)}, \mathbf{x}^{(L)}) - \frac{\dout }{m} \sum_{i=1}^{L}  \sum_{s' \in [\dout ]} \sum_{r' \in [p]} \sum_{r \in \mathcal{K}}  b_{r, s'} b_{r', s'}^{\dagger} \mathbf{Back}_{i \to L, r, s} \Phi_{r', s} \left(\left\langle \mathbf{w}_{r', s}^{\dagger}, [\overline{\mathbf{x}}^{(2)}, \cdots, \overline{\mathbf{x}}^{(L-1)}]\right\rangle\right)} \\&\le 
	\abs{F^{(L), \mathcal{K}}_s(\mathbf{h}^{(L-1)}, \mathbf{x}^{(L)}) - \widetilde{F}^{(L), \mathcal{K}}_s(\widetilde{\mathbf{h}}^{(L-1)}, \mathbf{x}^{(L)})} \\& +\abs{\widetilde{F}^{(L), \mathcal{K}}_s(\widetilde{\mathbf{h}}^{(L-1)}, \mathbf{x}^{(L)}) - \frac{\dout }{m} \sum_{i=1}^{L}  \sum_{s' \in [\dout ]} \sum_{r' \in [p]} \sum_{r \in \mathcal{K}}  b_{r, s'} b_{r', s'}^{\dagger} \widetilde{\mathbf{Back}}_{i \to L, r, s} \Phi_{r', s} \left(\left\langle \mathbf{w}_{r', s}^{\dagger}, [\overline{\mathbf{x}}^{(2)}, \cdots, \overline{\mathbf{x}}^{(L-1)}]\right\rangle\right)} \\& + \Big| \frac{\dout }{m} \sum_{i=1}^{L}  \sum_{s' \in [\dout ]} \sum_{r' \in [p]} \sum_{r \in \mathcal{K}}  b_{r, s'} b_{r', s'}^{\dagger} \widetilde{\mathbf{Back}}_{i \to L, r, s} \Phi_{r', s} \left(\left\langle \mathbf{w}_{r', s}^{\dagger}, [\overline{\mathbf{x}}^{(2)}, \cdots, \overline{\mathbf{x}}^{(L-1)}]\right\rangle\right) \\& -  \frac{\dout }{m} \sum_{i=1}^{L}  \sum_{s' \in [\dout ]} \sum_{r' \in [p]} \sum_{r \in \mathcal{K}}  b_{r, s'} b_{r', s'}^{\dagger} \mathbf{Back}_{i \to L, r, s} \Phi_{r', s} \left(\left\langle \mathbf{w}_{r', s}^{\dagger}, [\overline{\mathbf{x}}^{(2)}, \cdots, \overline{\mathbf{x}}^{(L-1)}]\right\rangle\right) \Big|
	\\&
	\le \mathcal{O}(\dout Lp\rho^8  \mathfrak{C}_{\varepsilon}(\Phi, \mathcal{O}(\varepsilon_x^{-1})) N^{5/3} m^{-7/6}) + \frac{\dout }{m} \cdot \mathcal{O}(\mathfrak{C}_{\varepsilon}(\Phi_{r' s}, \mathcal{O}(\varepsilon_x^{-1})) \rho^2 \sqrt{\dout Lp N})  \\& + \frac{\dout LpN}{m} \rho^2 (\varepsilon + \mathcal{O}( L_{\Phi}\rho^5 (N/m)^{1/6}) + \mathcal{O}(\varepsilon_x^{-1} L_{\Phi} L^4 \rho^{11} m^{-1/12} +  L_{\Phi} L^{11/6} \rho^2 \varepsilon_x^{2/3})) + \mathcal{O}(\rho^8\dout Lp N^{7/6} m^{-7/6}) .
\end{align*}
\endgroup
%\todo{Add more details if necessary.}


\end{proof}


























%%%%%%%%%%%%%%%%%%%%%%%%%%%%%%%%%%%%%%%%%%%%%%%%%%%%%%%%%%%%%%%%%%%%%%%%%%%%%%%%%%%%%%%%%%%%%%%%%%%%%%%%%%%%%%%%%%%%%%%%%%%%%%%%%




\subsection{Helping lemmas}
\subsubsection{Function approximation using hermite polynomials}
%%%%%%%%%%%%%%%%%%%%%%%%%%%%%%%%%%%%%%%%%%%%%%%%%%
The following theorem on approximating a smooth function using hermite polynomials has been taken from \cite{allen2019learning} and we will use this theorem to show that pseudo RNNs can approximate the target concept class.
\begin{theorem}[Lemma 6.2 in \cite{allen2019learning}]\label{Thm:Smooth_H}
	For every smooth function $\phi$, every $\varepsilon \in \left(0, \frac{1}{\mathfrak{C}_{s}\left(\phi, 1\right)}\right)$ there exists a $H:\mathbb{R}^2 \to \left(-\mathfrak{C}_\varepsilon\left(\phi, 1\right), \mathfrak{C}_\varepsilon\left(\phi, 1\right)\right)$, satisfying $\abs{H} \le \mathfrak{C}_\varepsilon\left(\phi, 1\right)$, and is $\mathfrak{C}_\varepsilon\left(\phi, 1\right)$-lipschitz continuous in the first variable and for all $x_1 \in (-1, 1)$
	\begin{equation*}
		\left|\mathbb{E}_{\alpha_1, \beta_1, b_0}\left[\mathbb{I}_{\alpha_{1} x_{1}+\beta_{1} \sqrt{1-x_{1}^{2}} + b_0   \geq 0} {H\left(\alpha_{1}, b_0\right)}\right]-\phi\left(x_{1}\right)\right| \leq \varepsilon
	\end{equation*}
	where $\alpha_1, \beta_1\text{ and } b_0 \sim \mathcal{N}\left(0, 1\right)$ are independent random variables.
\end{theorem}

In \cite{allen2019learning}, the function $H$ is shown to be lipschitz continuous in expectation w.r.t. the first variable $\alpha_1$ which follows a normal distribution. However, one can also show that the function $H$ is lipschitz continuous w.r.t. the first variable, even when the variable is perturbed by bounded noise to a variable that does not necessarily follow a gaussian distribution i.e. one can show that
\begin{align*}
	\left|\mathbb{E}_{\alpha_1, \beta_1, b_0 \sim \mathcal{N}(0, 1)} \mathbb{E}_{\theta: \abs{\theta} \le \gamma} \left[H\left(\alpha_{1}, b_0\right) - H\left(\alpha_{1} + \theta, b_0\right)\right]\right| \le \gamma \mathfrak{C}_\varepsilon(\phi, 1). 
\end{align*}
The proof will follow along the similar lines of Claim C.2 in \cite{allen2019learning}. We give a brief overview here. The function $H$ was shown to be a weighted combination of different hermite polynomials. Using the following property of hermite polynomials,
\begin{align*}
	h_i(x+y) = \sum_{k=0}^{i} {i \choose k} x^{i-k} h_k(y),
\end{align*}
we expand the function $H\left(\alpha_{1} + \theta, b_0\right)$ and then, bound each term using the procedure in Claim C.2 of \cite{allen2019learning}.
\todo{Ask Navin, if I have to add more details.}

\begin{corollary}\label{Cor:Smooth_H}
	For any $\sigma > 0$, $r_x > 0$ s.t. $\sigma \ge r_x/10$, $k_0 \ge 0$, and for every smooth function $\phi$, any $\varepsilon \in \left(0, \frac{r_x}{\sigma \mathfrak{C}_{s}\left(\phi, k_0 r_x\right)}\right)$ there exists a $H:\mathbb{R}^2 \to \left(- 
	\frac{\sigma}{r_x}\mathfrak{C}_\varepsilon\left(\phi,  k_0 r_x\right), \frac{\sigma}{r_x} \mathfrak{C}_\varepsilon\left(\phi,  k_0 r_x\right)\right)$, which is $ \frac{\sigma}{r_x}\mathfrak{C}_\varepsilon\left(\phi, k_0 r_x\right)$-lipschitz continuous and for all $x_1 \in (-r_x, r_x)$
	\begin{equation*}
		\left|\mathbb{E}_{\alpha_1, \beta_1, b_0}\left[\mathbb{I}_{\alpha_{1} x_{1}+\beta_{1} \sqrt{r_x^2-x_{1}^{2}} + b_0   \geq 0} {H\left(\alpha_{1}, b_0\right)}\right]-\phi\left(k_0 x_{1}\right)\right| \leq \varepsilon
	\end{equation*}
	where $\alpha_1, \beta_1 \sim \mathcal{N}\left(0, 1\right) \text{ and } b_0 \sim \mathcal{N}\left(0, \sigma^2\right)$ are independent random variables.
\end{corollary}


\begin{lemma}[Function Approximators]\label{Def:Function_approx}
	Let $r_x = \sqrt{2 + (L-2)\varepsilon_x^2}$. For each $\Phi_{r, s}$ and a constant $k_{0, r, s} = \Theta(\frac{1}{\varepsilon_x})$, there exists a function $H_{r, s}$ such that for any $\varepsilon \in (0, \min_{r, s} \frac{r_x}{ C_s(\Phi_{r, s}, k_{0, r, s} r_x)})$,  $H_{r, s}:\mathbb{R}^2 \to \left(-\frac{1}{r_x} \mathfrak{C}_\varepsilon\left(\Phi_{r, s}, k_{0, r, s} r_x\right),\frac{1}{r_x} \mathfrak{C}_\varepsilon\left(\Phi_{r, s}, k_{0, r, s} r_x\right)\right)$, is $\frac{1}{r_x} \mathfrak{C}_\varepsilon\left(\Phi_{r, s},  k_{0, r, s} r_x\right)$-lipschitz continuous, and for all $x_1 \in (-r_x, r_x)$
	\begin{equation*}
		\left|\mathbb{E}_{\alpha_1, \beta_1, b_0}\left[\mathbb{I}_{\alpha_{1} x_{1}+\beta_{1} \sqrt{r_x^2 - x_{1}^{2}} + b_0   \geq 0} {H_{r, s}\left(\alpha_{1}, b_0\right)}\right]-\Phi_{r, s}\left(k_{0, r, s} x_{1} \right)\right| \leq \varepsilon
	\end{equation*}
	where $\alpha_1, \beta_1 \sim \mathcal{N}\left(0, 1\right) \text{ and } b_0 \sim \mathcal{N}\left(0, 1\right)$ are independent random variables.
\end{lemma}
For any $\varepsilon_x \le \frac{1}{L}$, we can see that for all $\Phi_{r, s}$,
$\abs{H_{r, s}} \le \frac{1}{\sqrt{2}} \mathfrak{C}_\varepsilon\left(\Phi_{r, s},  \mathcal{O}(\varepsilon_x^{-1})\right)$ and $H_{r, s}$ is $\frac{1}{\sqrt{2}} \mathfrak{C}_\varepsilon\left(\Phi_{r, s},  \mathcal{O}(\varepsilon_x^{-1})\right)$ lipschitz, for any $\varepsilon \le \frac{\sqrt{3}}{C_s(\Phi_{r, s}, \mathcal{O}(\varepsilon_x^{-1}))}$.
%We use $k_0 = \frac{1}{\varepsilon_x}$, $r_x = \sqrt{2 + (L-2)\varepsilon_x^2}$ and $\sigma = \frac{1}{\sqrt{2}}$. 
%\todo{It doesn't directly follow. Add some details here.}





\subsubsection{Proofs of the helping lemmas}
First, we mention one of the properties on correlations of $\mathbf{Back}_{i \to j}$ matrices, which will be heavily used later on.
\begin{lemma}[Lemma C.1 in \cite{allen2019can}]\label{lemma:backward_correlation}
	For every $\varepsilon_x < 1/L$ and every normalized input sequence, $\mathbf{x}_1, \mathbf{x}_2, ..., \mathbf{x}_{L}$, with probability at least 1 - $e^{-\Omega(\rho^2)}$ over $\mathbf{W}$, $\mathbf{A}$ and $\mathbf{B}$: for every $1 \le i \le j < j' \le L$, 
	\begin{equation*}
		\abs{\langle \mathbf{u}^{\top} \mathbf{Back}_{i \to j}, \mathbf{v}^{\top} \mathbf{Back}_{i \to j'} \rangle} \le \mathcal{O}\left(m^{0.75} \rho\right) \norm{\mathbf{u}} \norm{\mathbf{v}},
	\end{equation*}
	for any two vectors $\mathbf{u}$ and $\mathbf{v}$ in $\mathbb{R}^{\dout }$.
\end{lemma}

%%%%%%%%%%%%%%%%%%%%%%%%%%%%%%%%%%%%%%%%%%%%%%%%%%



%In the following theorem, we show that the pseudo RNN can approximate the target concept class, using the weight $\mathbf{W}^{*}$ and $\mathbf{A}^{\ast}$ define above.


%\begin{proof}
	
	
	
	%\begin{proof}
		
	
		\begin{claim}[Restating claim~\ref{claim:invtildethetatheta}]\label{claim:invtildethetatheta_proof}
			With probability at least $1-e^{-\Omega(\rho^2)}$, for all $r' \in [p], s \in [\dout ]$,
			\begin{align*}
				\abs{\widetilde{\theta}_{r', s} \theta_{r', s}^{-1} - 1} \le \mathcal{O}(\rho^5 (N/m)^{1/6}).
			\end{align*}
		\end{claim}
		
		\begin{proof}
			First of all,
			\begin{align*}
				\sqrt{2/m} \abs{\widetilde{\theta}_{r', s}^{-1} - \theta_{r', s}^{-1} } &= 
				\abs{\norm[2]{\overline{\widetilde{\mathbf{W}}}^{[L]} \mathbf{w}_{r', s}^{\dagger}} - \norm[2]{\overline{\mathbf{W}}^{[L]} \mathbf{w}_{r', s}^{\dagger}}} \\& \le 
				\norm[2]{\overline{\widetilde{\mathbf{W}}}^{[L]} \mathbf{w}_{r', s}^{\dagger}- \overline{\mathbf{W}}^{[L]} \mathbf{w}_{r', s}^{\dagger}}  \\&
				%\le %\norm[2]{\overline{\widetilde{\mathbf{W}}}^{[L]} \mathbf{w}_{r', s}^{\dagger}- \overline{\mathbf{W}}^{[L]}}_2 \norm{\mathbf{w}_{r', s}^{\dagger}}_2 \\&
				\le \max_{\ell \le L} \norm{\left(\widetilde{\mathbf{W}}^{(L, 
						\ell)} - \mathbf{W}^{(L, \ell)}\right) \mathbf{w}_{r', s}^{\dagger}}_2  \\&
				\le \mathcal{O}(\rho^5 (N/m)^{1/6})
			\end{align*}
			where in the pre-final step, we have used Lemma~\ref{lemma:rerandESN} to have w.p. exceeding $1-e^{-\Omega(\rho^2)}$ for any $\ell \le L$,
			\begin{align*}
				\norm{\left(\widetilde{\mathbf{W}}^{(L,
						\ell)} - \mathbf{W}^{(L, \ell)}\right) \mathbf{w}_{r', s}^{\dagger}}_2 &= \norm{\left( \prod_{L \ge \ell' \ge \ell} \widetilde{\mathbf{D}}_{(0)}^{(\ell)} \widetilde{\mathbf{W}}  - \prod_{L \ge \ell' \ge \ell} \mathbf{D}_{(0)}^{(\ell)} \mathbf{W} \right)  \mathbf{w}_{r', s}^{\dagger}}_2\\& \le  \mathcal{O}(\rho^5 (N/m)^{1/6}) \cdot \norm[2]{\mathbf{w}_{r', s}^{\dagger}} = \mathcal{O}(\rho^5 (N/m)^{1/6}).
			\end{align*}
			Hence,
			\begin{align*}
				\abs{ \widetilde{\theta}_{r', s} \theta_{r', s}^{-1} - 1} \le \sqrt{m/2} \abs{\widetilde{\theta}_{r', s} }  \mathcal{O}(\rho^5 (N/m)^{1/6}) \le
				\mathcal{O}(\rho^5 (N/m)^{1/6}),
			\end{align*}
			where we have used the upper bound on $\sqrt{m/2} \abs{\widetilde{\theta}_{r', s} }$ from eq.~\ref{eq:abstildetheta_lowerbound} in the final step. 
		\end{proof}
		
	
		
		%We will first show that $\widetilde{F}_{s}^{(L), \mathcal{K}}(\widetilde{\mathbf{h}}^{(L-1)}, \mathbf{x}^{(L)})$ and $F_{s}^{(L), \mathcal{K}}(\mathbf{h}^{(L-1)}, \mathbf{x}^{(L)})$ are close.
		
		\begin{claim}[Restating claim~\ref{claim:diffftildef}]\label{claim:diffftildef_proof}
			With probability at least $1-e^{-\Omega(\rho^2)}$,
			\begingroup\allowdisplaybreaks
			\begin{align*}
				\abs{\widetilde{F}_{s}^{(L), \mathcal{K}}(\widetilde{\mathbf{h}}^{(L-1)}, \mathbf{x}^{(L)}) - F_{s}^{(L), \mathcal{K}}(\mathbf{h}^{(L-1)}, \mathbf{x}^{(L)})} \le \mathcal{O}(\dout Lp\rho^8  \mathfrak{C}_{\varepsilon}(\Phi, \mathcal{O}(\varepsilon_x^{-1})) N^{5/3} m^{-7/6}).
			\end{align*}
			\endgroup
		\end{claim}
		
		\begin{proof}
			We break the required term into three different terms.
			\begingroup\allowdisplaybreaks
			\begin{align}
				&\abs{\widetilde{F}_{s}^{(L), \mathcal{K}}(\widetilde{\mathbf{h}}^{(L-1)}, \mathbf{x}^{(L)}) - F_{s}^{(L), \mathcal{K}}(\mathbf{h}^{(L-1)}, \mathbf{x}^{(L)})}\nonumber\\&=
				\Big| \frac{\dout }{m} \sum_{i=1}^{L}  \sum_{s' \in [\dout ]} \sum_{r' \in [p]} \sum_{r \in \mathcal{K}}  b_{r, s'} b_{r', s'}^{\dagger} \widetilde{\mathbf{Back}}_{i \to L, r, s} H_{r', s'}\Big(\widetilde{\theta}_{r', s'} \langle \mathbf{w}_{r}, \overline{\widetilde{\mathbf{W}}}^{[L]} \mathbf{w}_{r', s'}^{\dagger}\rangle , \sqrt{m/2} a_{r, d}\Big) \mathbb{I}_{\mathbf{w}_r^{\top} \widetilde{\mathbf{h}}^{(i-1)} + \mathbf{a}_r^{\top} \mathbf{x}^{(i)} \ge 0} \nonumber\\&
				\quad \quad -  \frac{\dout }{m} \sum_{i=1}^{L}  \sum_{s' \in [\dout ]} \sum_{r' \in [p]} \sum_{r \in \mathcal{K}}  b_{r, s'} b_{r', s'}^{\dagger} \mathbf{Back}_{i \to L, r, s}  H_{r', s'}\Big(\theta_{r', s'} \langle \mathbf{w}_{r}, \overline{\mathbf{W}}^{[L]} \mathbf{w}_{r', s'}^{\dagger}\rangle , \sqrt{m/2} a_{r, d}\Big) \mathbb{I}_{\mathbf{w}_r^{\top} \mathbf{h}^{(i-1)} + \mathbf{a}_r^{\top} \mathbf{x}^{(i)} \ge 0} \Big| \nonumber \\&
				\le  \Big| \frac{\dout }{m} \sum_{i=1}^{L}  \sum_{s' \in [\dout ]} \sum_{r' \in [p]} \sum_{r \in \mathcal{K}}  b_{r, s'} b_{r', s'}^{\dagger} \widetilde{\mathbf{Back}}_{i \to L, r, s}  H_{r', s'}\Big(\widetilde{\theta}_{r', s'} \langle \mathbf{w}_{r}, \overline{\widetilde{\mathbf{W}}}^{[L]} \mathbf{w}_{r', s'}^{\dagger}\rangle , \sqrt{m/2} a_{r, d}\Big) \mathbb{I}_{\mathbf{w}_r^{\top} \widetilde{\mathbf{h}}^{(i-1)} + \mathbf{a}_r^{\top} \mathbf{x}^{(i)} \ge 0} \nonumber\\&
				\quad \quad -  \frac{\dout }{m} \sum_{i=1}^{L}  \sum_{s' \in [\dout ]} \sum_{r' \in [p]} \sum_{r \in \mathcal{K}}  b_{r, s'} b_{r', s'}^{\dagger} \mathbf{Back}_{i \to L, r, s}  H_{r', s'}\Big(\widetilde{\theta}_{r', s'} \langle \mathbf{w}_{r}, \overline{\widetilde{\mathbf{W}}}^{[L]} \mathbf{w}_{r', s'}^{\dagger}\rangle , \sqrt{m/2} a_{r, d}\Big) \mathbb{I}_{\mathbf{w}_r^{\top} \widetilde{\mathbf{h}}^{(i-1)} + \mathbf{a}_r^{\top} \mathbf{x}^{(i)} \ge 0} \Big| \label{eq:rerandRNN_1}\\&
				\quad \quad + \Big| \frac{\dout }{m} \sum_{i=1}^{L}  \sum_{s' \in [\dout ]} \sum_{r' \in [p]} \sum_{r \in \mathcal{K}}  b_{r, s'} b_{r', s'}^{\dagger} \mathbf{Back}_{i \to L, r, s}  H_{r', s'}\Big(\widetilde{\theta}_{r', s'} \langle \mathbf{w}_{r}, \overline{\widetilde{\mathbf{W}}}^{[L]} \mathbf{w}_{r', s'}^{\dagger}\rangle , \sqrt{m/2} a_{r, d}\Big) \mathbb{I}_{\mathbf{w}_r^{\top} \widetilde{\mathbf{h}}^{(i-1)} + \mathbf{a}_r^{\top} \mathbf{x}^{(i)} \ge 0} \nonumber\\&
				\quad \quad -  \frac{\dout }{m} \sum_{i=1}^{L}  \sum_{s' \in [\dout ]} \sum_{r' \in [p]} \sum_{r \in \mathcal{K}}  b_{r, s'} b_{r', s'}^{\dagger} \mathbf{Back}_{i \to L, r, s}  H_{r', s'}\Big(\widetilde{\theta}_{r', s'} \langle \mathbf{w}_{r}, \overline{\widetilde{\mathbf{W}}}^{[L]} \mathbf{w}_{r', s'}^{\dagger}\rangle , \sqrt{m/2} a_{r, d}\Big) \mathbb{I}_{\mathbf{w}_r^{\top} \mathbf{h}^{(i-1)} + \mathbf{a}_r^{\top} \mathbf{x}^{(i)} \ge 0} \Big| \label{eq:rerandRNN_2}\\&
				\quad\quad + \Big| \frac{\dout }{m} \sum_{i=1}^{L}  \sum_{s' \in [\dout ]} \sum_{r' \in [p]} \sum_{r \in \mathcal{K}}  b_{r, s'} b_{r', s'}^{\dagger} \mathbf{Back}_{i \to L, r, s}  H_{r', s'}\Big(\widetilde{\theta}_{r', s'} \langle \mathbf{w}_{r}, \overline{\widetilde{\mathbf{W}}}^{[L]} \mathbf{w}_{r', s'}^{\dagger}\rangle , \sqrt{m/2} a_{r, d}\Big) \mathbb{I}_{\mathbf{w}_r^{\top} \mathbf{h}^{(i-1)} + \mathbf{a}_r^{\top} \mathbf{x}^{(i)} \ge 0} \nonumber\\&
				\quad \quad -  \frac{\dout }{m} \sum_{i=1}^{L}  \sum_{s' \in [\dout ]} \sum_{r' \in [p]} \sum_{r \in \mathcal{K}}  b_{r, s'} b_{r', s'}^{\dagger} \mathbf{Back}_{i \to L, r, s}  H_{r', s'}\Big(\theta_{r', s'} \langle \mathbf{w}_{r}, \overline{\mathbf{W}}^{[L]} \mathbf{w}_{r', s'}^{\dagger}\rangle , \sqrt{m/2} a_{r, d}\Big) \mathbb{I}_{\mathbf{w}_r^{\top} \mathbf{h}^{(i-1)} + \mathbf{a}_r^{\top} \mathbf{x}^{(i)} \ge 0} \Big|. \label{eq:rerandRNN_3}
			\end{align}
			\endgroup
			
			We now show that each of the three equations, eq.~\ref{eq:rerandRNN_1}, eq.~\ref{eq:rerandRNN_2} and eq.~\ref{eq:rerandRNN_3} are small. First, we will need a couple of bounds on the terms that appear in the equations.
			\begin{itemize}
				\item Since $b_{r, s'} \sim \mathcal{N}(0, \frac{1}{\dout })$, using the fact~\ref{fact:max_gauss}, we can show that with probability $1-e^{-\Omega(\rho^2)}$,
				$\max_{r, s'} | b_{r, s'} | \le \frac{\rho}{\sqrt{\dout }}.$ 
				\item From the definition of concept class,  $\max_{r',s'} | b_{r', s'}^{\dagger} | \le 1$. 
				\item By the definition of $H$ from def~\ref{Def:Function_approx}, we have $\max_{r',s'} |  H_{r', s'}\Big(\widetilde{\theta}_{r', s'} \langle \mathbf{w}_{r}, \overline{\widetilde{\mathbf{W}}}^{[L]} \mathbf{w}_{r', s'}^{\dagger}\rangle , \sqrt{m/2} a_{r, d}\Big) | \le \mathfrak{C}_{\varepsilon}(\Phi, \mathcal{O}(\varepsilon_x^{-1}))$.
				\item Since $\mathbf{b}_s \sim \mathcal{N}(0, \frac{1}{\dout } \mathbf{I})$, using fact~\ref{lem:chi-squared}, we can show that $\norm{\mathbf{b}_s} \le \mathcal{O}(\frac{\rho}{\sqrt{\dout }})$, w.p. $1-e^{-\rho^2}$. Hence, from lemma~\ref{lemma:norm_ESN}, we have w.p. atleast $1 - e^{-\Omega(\rho^2)}$, for any $1 \le i \le j \le L, s \in [\dout ], r \in [m]$, $\abs{\mathbf{e}_s^{\top} \mathbf{Back}_{i \to j} \mathbf{e}_r} = 
				\abs{\mathbf{b}_s^{\top} \mathbf{D}^{(\ell)} \mathbf{W} \cdots \mathbf{D}^{(i+1)} \mathbf{W}\mathbf{e}_r} \le \norm[2]{\mathbf{b}_s} \norm[2]{\mathbf{D}^{(\ell)} \mathbf{W} \cdots \mathbf{D}^{(i+1)} \mathbf{W}\mathbf{e}_r} \le \mathcal{O}(\frac{\rho}{\sqrt{\dout }})$.
			\end{itemize}
			
			First, let's focus on eq.~\ref{eq:rerandRNN_1}. From Lemma~\ref{lemma:rerandESN}, we have with probability at least $1 - e^{-\Omega(\rho^2)}$,
			\begin{align*}
				\abs{\mathbf{e}_s^{\top} \left(\mathbf{Back_{i \to j}} - \widetilde{\mathbf{Back}}_{i \to j} \right) \mathbf{e}_r} 
				&= \abs{\mathbf{b}_s^{\top} \left( \mathbf{D}^{(j)} \mathbf{W} \cdots \mathbf{D}^{(i)} \mathbf{W} - \widetilde{\mathbf{D}}^{(j)} \widetilde{\mathbf{W}} \cdots \widetilde{\mathbf{D}}^{(i)} \widetilde{\mathbf{W}} \right) \mathbf{e}_r}
				\\&
				\le \norm[2]{\mathbf{b}_s} \norm[2]{\left( \mathbf{D}^{(j)} \mathbf{W} \cdots \mathbf{D}^{(i)} \mathbf{W} - \widetilde{\mathbf{D}}^{(j)} \widetilde{\mathbf{W}} \cdots \widetilde{\mathbf{D}}^{(i)} \widetilde{\mathbf{W}} \right) \mathbf{e}_r}
				\\&\le \mathcal{O}(\rho^7 \dout ^{-1/2} N^{1/6} m^{-1/6}) , \text{ for all }  r \in [m], s \in [\dout ] \text{ and } 1 \le i \le j \le L.
			\end{align*}
			Thus, 
			\begingroup \allowdisplaybreaks
			\begin{align}
				& \Big| \frac{\dout }{m} \sum_{i=1}^{L}  \sum_{s' \in [\dout ]} \sum_{r' \in [p]} \sum_{r \in \mathcal{K}}  b_{r, s'} b_{r', s'}^{\dagger} \widetilde{\mathbf{Back}}_{i \to L, r, s}  H_{r', s'}\Big(\widetilde{\theta}_{r', s'} \langle \mathbf{w}_{r}, \overline{\widetilde{\mathbf{W}}}^{[L]} \mathbf{w}_{r', s'}^{\dagger}\rangle , \sqrt{m/2} a_{r, d}\Big) \mathbb{I}_{\mathbf{w}_r^{\top} \widetilde{\mathbf{h}}^{(i-1)} + \mathbf{a}_r^{\top} \mathbf{x}^{(i)} \ge 0} \nonumber\\&
				-  \frac{\dout }{m} \sum_{i=1}^{L}  \sum_{s' \in [\dout ]} \sum_{r' \in [p]} \sum_{r \in \mathcal{K}}  b_{r, s'} b_{r', s'}^{\dagger} \mathbf{Back}_{i \to L, r, s}  H_{r', s'}\Big(\widetilde{\theta}_{r', s'} \langle \mathbf{w}_{r}, \overline{\widetilde{\mathbf{W}}}^{[L]} \mathbf{w}_{r', s'}^{\dagger}\rangle , \sqrt{m/2} a_{r, d}\Big) \mathbb{I}_{\mathbf{w}_r^{\top} \widetilde{\mathbf{h}}^{(i-1)} + \mathbf{a}_r^{\top} \mathbf{x}^{(i)} \ge 0} \Big|\nonumber\\&
				\le \sum_{i=1}^{L} \sum_{r' \in [p]} \sum_{s' \in [\dout ]} \sum_{r \in \mathcal{K}} \frac{\dout }{m} \abs{\mathbf{e}_s^{\top} \left(\mathbf{Back_{i \to j}} - \widetilde{\mathbf{Back}}_{i \to j} \right) \mathbf{e}_r}  \nonumber\\& \quad \quad \quad \quad \cdot \abs{b_{r, s'} b_{r', s'}^{\dagger}  H_{r', s'}\Big(\widetilde{\theta}_{r', s'} \langle \mathbf{w}_{r}, \overline{\widetilde{\mathbf{W}}}^{[L]} \mathbf{w}_{r', s'}^{\dagger}\rangle , \sqrt{m/2} a_{r, d}\Big) \mathbb{I}_{\mathbf{w}_r^{\top} \widetilde{\mathbf{h}}^{(i-1)} + \mathbf{a}_r^{\top} \mathbf{x}^{(i)} \ge 0}} \nonumber\\&
				\le \sum_{i=1}^{L} \sum_{r' \in [p]} \sum_{s' \in [\dout ]} \sum_{r \in \mathcal{K}} \frac{\dout }{m} \abs{\mathbf{e}_s^{\top} \left(\mathbf{Back_{i \to j}} - \widetilde{\mathbf{Back}}_{i \to j} \right) \mathbf{e}_r}  \nonumber\\& \quad \quad \quad \quad \cdot \abs{b_{r, s'}} \abs{b_{r', s'}^{\dagger}} \abs{  H_{r', s'}\Big(\widetilde{\theta}_{r', s'} \langle \mathbf{w}_{r}, \overline{\widetilde{\mathbf{W}}}^{[L]} \mathbf{w}_{r', s'}^{\dagger}\rangle , \sqrt{m/2} a_{r, d}\Big)} \abs{\mathbb{I}_{\mathbf{w}_r^{\top} \widetilde{\mathbf{h}}^{(i-1)} + \mathbf{a}_r^{\top} \mathbf{x}^{(i)} \ge 0}} \nonumber\\&
				\le \sum_{i=1}^{L} \sum_{r' \in [p]} \sum_{s' \in [\dout ]} \sum_{r \in \mathcal{K}} \frac{\dout }{m} \cdot \mathcal{O}(\dout ^{-1/2} \rho^7 (N/m)^{1/6}) \cdot \frac{\rho}{\sqrt{\dout }} \cdot 1 \cdot \mathfrak{C}_{\varepsilon}(\Phi, \mathcal{O}(\varepsilon_x^{-1})) \cdot 1 \\&
				\le \mathcal{O}( \dout  pL \rho^8 \mathfrak{C}_{\varepsilon}(\Phi_{r', s'}, \mathcal{O}(\varepsilon_x^{-1})) (N/m)^{7/6}) \label{eq:rerandRNN_1SOLVE}.
			\end{align}
			\endgroup
			
			
			Now, we focus on eq.~\ref{eq:rerandRNN_2}. Lemma~\ref{lemma:rerandESN} shows that with probability at least $1 - e^{-\Omega(\rho^2)}$,
			\begin{equation*}
				\abs[0]{\mathbf{w}_r^{\top} (\widetilde{\mathbf{h}}^{(L-1)} -  \mathbf{h}^{(L-1)})} \leq \mathcal{O}\left(\rho^{5} N^{2 / 3} m^{-2 / 3}\right) \quad \text { for every } r \in [m], \ell \in[L].
			\end{equation*}
			
			From lemma~\ref{lemma:norm_ESN}, we have w.p. at least $1 - e^{-\Omega(\rho^2)}$ for any $s \le \frac{\rho^2}{m}$,
			\begin{align*}
				\abs{\left\{r \in [m] \Big| \abs{\mathbf{w}_r^{\top} \mathbf{h}^{(L-1)} + \mathbf{a}_r^{\top} \mathbf{x}^{(L-1)}} \le \frac{s}{\sqrt{m}} \right\}} \le \mathcal{O}(sm).
			\end{align*}
			This can be modified for the subset $\mathcal{K}$,  w.p. at least $1 - e^{-\Omega(\rho^2)}$ for any $s \le \frac{\rho^2}{m}$,
			\begin{align*}
				\abs{\left\{r \in \mathcal{K} \Big| \abs{\mathbf{w}_r^{\top} \mathbf{h}^{(L-1)} + \mathbf{a}_r^{\top} \mathbf{x}^{(L-1)}} \le  \frac{s}{\sqrt{m}} \right\}} \le \mathcal{O}(sN).
			\end{align*}
			
			Thus, w.p. at least $1 - e^{-\Omega(\rho^2)}$,
			\begin{equation*}
				\sum_{r \in \mathcal{K}} \mathbb{I}\left[\abs{\mathbf{w}_r^{\top} \mathbf{h}^{(L-1)} + \mathbf{a}_r^{\top} \mathbf{x}^{(L-1)}} \le \rho^{5} N^{2 / 3} m^{-2 / 3}\right] \le \mathcal{O}(\rho^{5} N^{5 / 3} m^{-1/6}).
			\end{equation*}
			%\todo{Add details here!!}
			
			Hence, that implies  w.p. at least $1 - e^{-\Omega(\rho^2)}$,
			\begingroup \allowdisplaybreaks
			\begin{align*}
				&\sum_{r \in \mathcal{K}} \abs{\mathbb{I}\left[\mathbf{w}_r^{\top} \mathbf{h}^{(L-1)} + \mathbf{a}_r^{\top} \mathbf{x}^{(L-1)} \right] - \mathbb{I}\left[\mathbf{w}_r^{\top} \widetilde{\mathbf{h}}^{(L-1)} + \mathbf{a}_r^{\top} \mathbf{x}^{(L-1)} \right]}\\&
				\le \sum_{r \in \mathcal{K}}\mathbb{I}\left[\abs[0]{\mathbf{w}_r^{\top} \mathbf{h}^{(L-1)} + \mathbf{a}_r^{\top} \mathbf{x}^{(L-1)}} \le \abs[0]{\mathbf{w}_r^{\top} \widetilde{\mathbf{h}}^{(L-1)} + \mathbf{a}_r^{\top} \mathbf{x}^{(L-1)} - \mathbf{w}_r^{\top} \mathbf{h}^{(L-1)} - \mathbf{a}_r^{\top} \mathbf{x}^{(L-1)}} \right] \\&
				= \sum_{r \in \mathcal{K}}\mathbb{I}\left[\abs[0]{\mathbf{w}_r^{\top} \mathbf{h}^{(L-1)} + \mathbf{a}_r^{\top} \mathbf{x}^{(L-1)}} \le \abs[0]{\mathbf{w}_r^{\top} (\widetilde{\mathbf{h}}^{(L-1)} -  \mathbf{h}^{(L-1)}}) \right]
				\\&\le \sum_{r \in \mathcal{K}}\mathbb{I}\left[\abs[0]{\mathbf{w}_r^{\top} \mathbf{h}^{(L-1)} + \mathbf{a}_r^{\top} \mathbf{x}^{(L-1)}} \le \mathcal{O}\left(\rho^{5} N^{2 / 3} m^{-2 / 3}\right) \right] \\&
				\le \mathcal{O}(\rho^{5} N^{5 / 3} m^{-1/6}).
			\end{align*}  
			\endgroup
			Thus, we have in eq.~\ref{eq:rerandRNN_2},
			\begingroup \allowdisplaybreaks
			\begin{align}
				&\Big| \frac{\dout }{m} \sum_{i=1}^{L}  \sum_{s' \in [\dout ]} \sum_{r' \in [p]} \sum_{r \in \mathcal{K}}  b_{r, s'} b_{r', s'}^{\dagger} \mathbf{Back}_{i \to L, r, s}  H_{r', s'}\Big(\widetilde{\theta}_{r', s'} \langle \mathbf{w}_{r}, \overline{\widetilde{\mathbf{W}}}^{[L]} \mathbf{w}_{r', s'}^{\dagger}\rangle , \sqrt{m/2} a_{r, d}\Big) \mathbb{I}_{\mathbf{w}_r^{\top} \widetilde{\mathbf{h}}^{(i-1)} + \mathbf{a}_r^{\top} \mathbf{x}^{(i)} \ge 0} \nonumber\\&
				\quad \quad -  \frac{\dout }{m} \sum_{i=1}^{L}  \sum_{s' \in [\dout ]} \sum_{r' \in [p]} \sum_{r \in \mathcal{K}}  b_{r, s'} b_{r', s'}^{\dagger} \mathbf{Back}_{i \to L, r, s}  H_{r', s'}\Big(\widetilde{\theta}_{r', s'} \langle \mathbf{w}_{r}, \overline{\widetilde{\mathbf{W}}}^{[L]} \mathbf{w}_{r', s'}^{\dagger}\rangle , \sqrt{m/2} a_{r, d}\Big) \mathbb{I}_{\mathbf{w}_r^{\top} \mathbf{h}^{(i-1)} + \mathbf{a}_r^{\top} \mathbf{x}^{(i)} \ge 0} \Big| \nonumber\\&
				\le \max_{r, s} \abs{b_{r, s}} \cdot \max_{r', s'} \abs{b^{\dagger}_{r',s'}} \cdot \max_{i,r,s} \abs{\mathbf{Back}_{i \to L, r, s}} \cdot \max_{r', s', r}\abs{H_{r', s'}\Big(\widetilde{\theta}_{r', s'} \langle \mathbf{w}_{r}, \overline{\widetilde{\mathbf{W}}}^{[L]} \mathbf{w}_{r', s'}^{\dagger}\rangle , \sqrt{m/2} a_{r, d}\Big)} \nonumber\\& \quad\quad\quad\quad
				\cdot \frac{\dout }{m} \sum_{i=1}^{L} \sum_{s' \in [\dout ]} \sum_{r' \in [p]} \sum_{r \in \mathcal{K}} \abs{\mathbb{I}_{\mathbf{w}_r^{\top} \mathbf{h}^{(i-1)} + \mathbf{a}_r^{\top} \mathbf{x}^{(i)} \ge 0} - \mathbb{I}_{\mathbf{w}_r^{\top} \widetilde{\mathbf{h}}^{(i-1)} + \mathbf{a}_r^{\top} \mathbf{x}^{(i)} \ge 0}} \nonumber\\&
				\le \frac{\rho}{\sqrt{\dout }} \cdot 1 \cdot \mathcal{O}(\frac{\rho}{\sqrt{\dout }}) \cdot \mathfrak{C}_{\varepsilon}(\Phi, \mathcal{O}(\varepsilon_x^{-1})) \cdot \frac{\dout ^2Lp }{m} \cdot \mathcal{O}(\rho^4 N^{5/3} m^{1/6}) \le \mathcal{O}(\dout Lp\rho^6 N^{5/3} m^{-7/6} \mathfrak{C}_{\varepsilon}(\Phi, \mathcal{O}(\varepsilon_x^{-1}))) \label{eq:rerandRNN_2SOLVE}. 
			\end{align}
			\endgroup
			
			Now, we focus on eq.~\ref{eq:rerandRNN_3}. We have
			\begingroup \allowdisplaybreaks
			\begin{align}
				&\Big| \frac{\dout }{m} \sum_{i=1}^{L}  \sum_{s' \in [\dout ]} \sum_{r' \in [p]} \sum_{r \in \mathcal{K}}  b_{r, s'} b_{r', s'}^{\dagger} \mathbf{Back}_{i \to L, r, s}  H_{r', s'}\Big(\widetilde{\theta}_{r', s'} \langle \mathbf{w}_{r}, \overline{\widetilde{\mathbf{W}}}^{[L]} \mathbf{w}_{r', s'}^{\dagger}\rangle , \sqrt{m/2} a_{r, d}\Big) \mathbb{I}_{\mathbf{w}_r^{\top} \mathbf{h}^{(i-1)} + \mathbf{a}_r^{\top} \mathbf{x}^{(i)} \ge 0} \nonumber\\&
				\quad \quad -  \frac{\dout }{m} \sum_{i=1}^{L}  \sum_{s' \in [\dout ]} \sum_{r' \in [p]} \sum_{r \in \mathcal{K}}  b_{r, s'} b_{r', s'}^{\dagger} \mathbf{Back}_{i \to L, r, s}  H_{r', s'}\Big(\theta_{r', s'} \langle \mathbf{w}_{r}, \overline{\mathbf{W}}^{[L]} \mathbf{w}_{r', s'}^{\dagger}\rangle , \sqrt{m/2} a_{r, d}\Big) \mathbb{I}_{\mathbf{w}_r^{\top} \mathbf{h}^{(i-1)} + \mathbf{a}_r^{\top} \mathbf{x}^{(i)} \ge 0} \Big| \nonumber\\&
				\le \max_{r,s'} \abs{b_{r, s'}} \cdot \max_{r',s'} \abs{b_{r', s'}^{\dagger}} \cdot \max_{i, r, s} \abs{\mathbf{Back}_{i \to L, r, s}} \cdot \max_{i, r} \abs{\mathbb{I}_{\mathbf{w}_r^{\top} \mathbf{h}^{(i-1)} + \mathbf{a}_r^{\top} \mathbf{x}^{(i)} \ge 0}} \nonumber\\&
				\cdot \frac{\dout }{m} \sum_{i=1}^{L}  \sum_{s' \in [\dout ]} \sum_{r' \in [p]} \sum_{r \in \mathcal{K}} \abs{H_{r', s'}\Big(\theta_{r', s'} \langle \mathbf{w}_{r}, \overline{\mathbf{W}}^{[L]} \mathbf{w}_{r', s'}^{\dagger}\rangle , \sqrt{m/2} a_{r, d}\Big) - H_{r', s'}\Big(\widetilde{\theta}_{r', s'} \langle \mathbf{w}_{r}, \overline{\widetilde{\mathbf{W}}}^{[L]} \mathbf{w}_{r', s'}^{\dagger}\rangle , \sqrt{m/2} a_{r, d}\Big)} \nonumber\\&
				\le \frac{\rho}{\sqrt{\dout }} \cdot 1 \cdot \mathcal{O}(\frac{\rho}{\sqrt{\dout }}) \cdot 1 \\& \cdot \frac{\dout }{m} \sum_{i=1}^{L}  \sum_{s' \in [\dout ]} \sum_{r' \in [p]} \sum_{r \in \mathcal{K}} \cdot \mathfrak{C}_{\varepsilon}(\Phi_{r',s'}, \mathcal{O}(\varepsilon_x^{-1}))  \cdot \abs{\theta_{r', s'} \langle \mathbf{w}_{r}, \overline{\mathbf{W}}^{[L]} \mathbf{w}_{r', s'}^{\dagger}\rangle -  \widetilde{\theta}_{r', s'} \langle \mathbf{w}_{r}, \overline{\widetilde{\mathbf{W}}}^{[L]} \mathbf{w}_{r', s'}^{\dagger}\rangle} \label{eq:rerandRNN_3_preprefinal}\\&
				\le \frac{\rho}{\sqrt{\dout }} \cdot 1 \cdot \mathcal{O}(\frac{\rho}{\sqrt{\dout }}) \cdot 1 \cdot \frac{\dout ^2pL N}{m} \cdot \mathfrak{C}_{\varepsilon}(\Phi, \mathcal{O}(\varepsilon_x^{-1})) \cdot \mathcal{O}(\rho^6 N^{2/3} m^{-2/3} ) \label{eq:rerandRNN_3_prefinal} \\& 
				\le \mathcal{O}(\dout Lp\rho^8  \mathfrak{C}_{\varepsilon}(\Phi, \mathcal{O}(\varepsilon_x^{-1})) N^{5/3} m^{-5/3}) \nonumber,
			\end{align}
			\endgroup
			where we get eq.~\ref{eq:rerandRNN_3_preprefinal} by using the lipschitz continuity of the function $H_{r', s'}$ from def.~\ref{Def:Function_approx}. We get eq.~\ref{eq:rerandRNN_3_prefinal} by bounding the following term:
			\begin{align*}
				&\abs{\theta_{r', s'} \langle \mathbf{w}_{r}, \overline{\mathbf{W}}^{[L]} \mathbf{w}_{r', s'}^{\dagger}\rangle -  \widetilde{\theta}_{r', s'} \langle \mathbf{w}_{r}, \overline{\widetilde{\mathbf{W}}}^{[L]} \mathbf{w}_{r', s'}^{\dagger}\rangle} \\&
				\le \abs{\theta_{r', s'} \left(\langle \mathbf{w}_{r}, \overline{\mathbf{W}}^{[L]} \mathbf{w}_{r', s'}^{\dagger}\rangle -  \langle \mathbf{w}_{r}, \overline{\widetilde{\mathbf{W}}}^{[L]} \mathbf{w}_{r', s'}^{\dagger}\rangle\right)} + \abs{\left(\theta_{r', s'} - \widetilde{\theta}_{r', s'} \right) \langle \mathbf{w}_{r}, \overline{\widetilde{\mathbf{W}}}^{[L]} \mathbf{w}_{r', s'}^{\dagger}\rangle} \\&
				\le \abs{\theta_{r', s'}} \cdot \abs{\left\langle \mathbf{w}_{r}, \left(\overline{\mathbf{W}}^{[L]} -  \overline{\widetilde{\mathbf{W}}}^{[L]} \right) \mathbf{w}_{r', s'}^{\dagger}\right\rangle} + \abs{\theta_{r', s'} - \widetilde{\theta}_{r', s'} } \cdot \abs{ \langle \mathbf{w}_{r}, \overline{\widetilde{\mathbf{W}}}^{[L]} \mathbf{w}_{r', s'}^{\dagger}\rangle} \\&
				\le \mathcal{O}(\rho^6 (N/m)^{2/3}),
			\end{align*}
			where we use the following bounds that are true for all $r \in \mathcal{K}, r' \in [p], s' \in [\dout ]$ w.p. at least $1-e^{-\Omega(\rho^2)}$:
			\begin{itemize}
				\item Eq.~\ref{eq:abstheta_lowerbound} gives an upper bound of $O(1)$ on $|\theta_{r',s'}|$.
				\item Eq.~\ref{eq:tmp_W_MATHCALwl} can be easily modified to get a similar upper bound on $|\langle \mathbf{w}_{r}, (\overline{\mathbf{W}}^{[L]} -  \overline{\widetilde{\mathbf{W}}}^{[L]} ) \mathbf{w}_{r', s'}^{\dagger}\rangle |$.
				\item Cor.~\ref{cor:tildetheta_diff_theta} gives an upper bound on $|\theta_{r',s'} - \widetilde{\theta}_{r',s'}|$.
				\item Since $\norm[2]{\overline{\widetilde{\mathbf{W}}}^{[L]} \mathbf{w}_{r', s'}^{\dagger}} := \sqrt{m/2} \widetilde{\theta}_{r', s'}^{-1}$, we can use Eq.~\ref{eq:abstildetheta_lowerbound} to give an upper bound on the norm. Then, we can use Fact~\ref{fact:max_gauss} to bound $\max_{r \in \mathcal{K}} |\langle \mathbf{w}_r, \overline{\widetilde{\mathbf{W}}}^{[L]} \mathbf{w}_{r', s'}^{\dagger}\rangle| = \frac{\rho}{\sqrt{m}} \cdot \norm[2]{\overline{\widetilde{\mathbf{W}}}^{[L]} \mathbf{w}_{r', s'}^{\dagger}}$.
			\end{itemize}
			%\todo{Add more details here pls.}
		\end{proof}
		%%%%%%%%%%%%%%%%%%%%%%%%%%%%%%%%%%%%%%%%%%%%%%%%%%%%%%%%%%%%%%%%%%%%%%%%%
		\iffalse
		In the following claim, we show that Eq.~\ref{eq:sprimenes} is small with high probability. 
		
		\begin{claim}
			With probability at least $1-e^{-\Omega(\rho^2)}$,
			\begin{align*}
				&\Big|\frac{\dout }{m} \sum_{i=1}^{L}  \sum_{s' \in [\dout ]: s' \ne s} \sum_{r' \in [p]} \sum_{r \in [m]}  b_{r, s'} b_{r', s'}^{\dagger} \mathbf{Back}_{i \to L, r, s} \\& \quad \quad \quad \quad H_{r', s'}\Big(\theta_{r', s'} \langle \mathbf{w}_{r}, \overline{\mathbf{W}}^{[L]} \mathbf{w}_{r', s'}^{\dagger}\rangle , \sqrt{m/2} a_{r, d}\Big) \mathbb{I}_{\mathbf{w}_r^{\top} \mathbf{h}^{(i-1)} + \mathbf{a}_r^{\top} \mathbf{x}^{(i)} \ge 0}\Big| \\&
				\le \mathcal{O}(C_\varepsilon(\Phi, \mathcal{O}( \varepsilon_x^{-1})) p L \dout \rho m^{-0.25}) 
			\end{align*}
		\end{claim}
		
		\begin{proof}
			We first divide the given term into 2 terms.
			\begingroup
			\allowdisplaybreaks
			\begin{align}
				&\Big|\frac{\dout }{m} \sum_{i=1}^{L}  \sum_{s' \in [\dout ]: s' \ne s} \sum_{r' \in [p]} \sum_{r \in [m]}  b_{r, s'} b_{r', s'}^{\dagger} \mathbf{Back}_{i \to L, r, s} \nonumber\\& \quad \quad \quad \quad H_{r', s'}\Big(\theta_{r', s'} \langle \mathbf{w}_{r}, \overline{\mathbf{W}}^{[L]} \mathbf{w}_{r', s'}^{\dagger}\rangle , \sqrt{m/2} a_{r, d}\Big) \mathbb{I}_{\mathbf{w}_r^{\top} \mathbf{h}^{(i-1)} + \mathbf{a}_r^{\top} \mathbf{x}^{(i)} \ge 0}\Big| \nonumber\\&
				\le \Big|\frac{\dout }{m} \sum_{i=1}^{L-1}  \sum_{s' \in [\dout ]: s' \ne s} \sum_{r' \in [p]} \sum_{r \in [m]}  b_{r, s'} b_{r', s'}^{\dagger} \mathbf{Back}_{i \to L, r, s} \nonumber\\& \quad \quad \quad \quad H_{r', s'}\Big(\theta_{r', s'} \langle \mathbf{w}_{r}, \overline{\mathbf{W}}^{[L]} \mathbf{w}_{r', s'}^{\dagger}\rangle , \sqrt{m/2} a_{r, d}\Big) \mathbb{I}_{\mathbf{w}_r^{\top} \mathbf{h}^{(i-1)} + \mathbf{a}_r^{\top} \mathbf{x}^{(i)} \ge 0}\Big|\label{eq:UpuntilL-1}\\&
				+ \Big|\frac{\dout }{m}  \sum_{s' \in [\dout ]: s' \ne s} \sum_{r' \in [p]} \sum_{r \in [m]}  b_{r, s'} b_{r', s'}^{\dagger} \mathbf{Back}_{L \to L, r, s} \nonumber\\& \quad \quad \quad \quad H_{r', s'}\Big(\theta_{r', s'} \langle \mathbf{w}_{r}, \overline{\mathbf{W}}^{[L]} \mathbf{w}_{r', s'}^{\dagger}\rangle , \sqrt{m/2} a_{r, d}\Big) \mathbb{I}_{\mathbf{w}_r^{\top} \mathbf{h}^{(L-1)} + \mathbf{a}_r^{\top} \mathbf{x}^{(L)} \ge 0}\Big|\label{eq:equal2L}
			\end{align}
			\endgroup
			
			We use Lemma~\ref{lemma:backward_correlation} to show that Eq.~\ref{eq:UpuntilL-1} is small. By the definition of $H_{r',s'}$ from Def.~\ref{Def:Function_approx}, we have
			\begingroup \allowdisplaybreaks
			\begin{align*}
				&\Big|\frac{\dout }{m} \sum_{i=1}^{L-1}  \sum_{s' \in [\dout ]: s' \ne s} \sum_{r' \in [p]} \sum_{r \in [m]}  b_{r, s'} b_{r', s'}^{\dagger} \mathbf{Back}_{i \to L, r, s} \\& \quad \quad \quad \quad H_{r', s'}\Big(\theta_{r', s'} \langle \mathbf{w}_{r}, \overline{\mathbf{W}}^{[L]} \mathbf{w}_{r', s'}^{\dagger}\rangle , \sqrt{m/2} a_{r, d}\Big) \mathbb{I}_{\mathbf{w}_r^{\top} \mathbf{h}^{(i-1)} + \mathbf{a}_r^{\top} \mathbf{x}^{(i)} \ge 0}\Big| 
				\\& = \abs{\frac{\dout }{m} \sum_{i=1}^{L-1} \sum_{s' \in [\dout ]: s' \ne s} \sum_{r' \in [p]} H_{r', s'}\Big(\theta_{r', s'} \langle \mathbf{w}_{r}, \overline{\mathbf{W}}^{[L]} \mathbf{w}_{r', s'}^{\dagger}\rangle , \sqrt{m/2} a_{r, d}\Big) \mathbb{I}_{\mathbf{w}_r^{\top} \mathbf{h}^{(i-1)} + \mathbf{a}_r^{\top} \mathbf{x}^{(i)} \ge 0} \left\langle e_{s'}^{\top} \mathbf{Back}_{i \to i}, e_s^{\top} \mathbf{Back}_{i \to L}\right\rangle}
				%\\& \le \max_{r', s'} \abs{H_{r', s'}\Big(\theta_{r', s'} \langle \mathbf{w}_{r}, \overline{\mathbf{W}}^{[L]} \mathbf{w}_{r', s'}^{\dagger}\rangle , \sqrt{m/2} a_{r, d}\Big)} \\& \quad \quad \quad \quad 
				%\cdot \max_r \abs{\mathbb{I}_{\mathbf{w}_r^{\top} \mathbf{h}^{(i-1)} + \mathbf{a}_r^{\top} \mathbf{x}^{(i)} \ge 0}}
				%\\& \quad \quad \quad \quad \cdot \abs{\frac{\dout }{m} \sum_{i=1}^{L-1}  \sum_{s' \in [\dout ]: s' \ne s} \sum_{r' \in [p]} \sum_{r \in [m]}  b_{r, s'} b_{r', s'}^{\dagger} \mathbf{Back}_{i \to L, r, s}} \\&
				%\le \max_{r', s'} C_\varepsilon(\Phi_{r', s'}, \mathcal{O}( \varepsilon_x^{-1})) \cdot 1 \cdot \abs{\frac{\dout }{m} \sum_{i=1}^{L-1}  \sum_{s' \in [\dout ]: s' \ne s} \sum_{r' \in [p]} \sum_{r \in [m]}  b_{r, s'} b_{r', s'}^{\dagger} \mathbf{Back}_{i \to L, r, s}} \\&
				%\le C_\varepsilon(\Phi, \mathcal{O}( \varepsilon_x^{-1})) \cdot 1 \cdot \sum_{i=1}^{L-1}  \sum_{s' \in [\dout ]: s' \ne s} \abs{\frac{\dout }{m} \sum_{r' \in [p]} \sum_{r \in [m]}  b_{r, s'} b_{r', s'}^{\dagger} \mathbf{Back}_{i \to L, r, s}} 
				\\&
				\le C_\varepsilon(\Phi, \mathcal{O}( \varepsilon_x^{-1})) \cdot 1 \cdot \sum_{i=1}^{L-1}  \sum_{s' \in [\dout ]: s' \ne s} \max_{r'}  \abs{b_{r', s'}^{\dagger}}  \sum_{r' \in [p]}  \abs{\frac{\dout }{m}  \sum_{r \in [m]}  b_{r, s'}  \mathbf{Back}_{i \to L, r, s}} \\&
				= C_\varepsilon(\Phi, \mathcal{O}( \varepsilon_x^{-1})) \cdot 1 \cdot \sum_{i=1}^{L-1} \sum_{s' \in [\dout ]: s' \ne s} \max_{r'}   \abs{b_{r', s'}^{\dagger}} \cdot p \cdot  \frac{\dout }{m} \abs{\left\langle \mathbf{e}_{s'}^{\top} \mathbf{Back}_{i \to i},  \mathbf{e}_{s}^{\top} \mathbf{Back}_{i \to L} \right\rangle} \\&
				\le C_\varepsilon(\Phi, \mathcal{O}( \varepsilon_x^{-1})) \cdot 1 \cdot (L-1) p \dout  \mathcal{O}(m^{-0.25} \rho) = \mathcal{O}(C_\varepsilon(\Phi, \mathcal{O}( \varepsilon_x^{-1})) L p \dout \rho m^{-0.25}),
			\end{align*}
			\endgroup
			where we use the definition of $\mathbf{Back}_{i \to i}$ to show that it is equivalent to $\mathbf{B}$ for all $i \in [L]$ and we invoke Lemma~\ref{lemma:backward_correlation} in the pre-final step that incurs a probability of $1 - e^{-\rho^2}$.
			
			We show now that Eq.~\ref{eq:equal2L}  is small. We use the definition of $\mathbf{Back}_{L \to L}$ to show that it is equivalent to $\mathbf{B}$ and then use the fact that the different columns of $\mathbf{B}$ are uncorrelated, since they come from a i.i.d. gaussian distribution.
			\begingroup
			\allowdisplaybreaks
			\begin{align*}
				&\Big|\frac{\dout }{m}   \sum_{s' \in [\dout ]: s' \ne s} \sum_{r' \in [p]} \sum_{r \in [m]}  b_{r, s'} b_{r', s'}^{\dagger} \mathbf{Back}_{L \to L, r, s} \\& \quad \quad \quad \quad H_{r', s'}\Big(\theta_{r', s'} \langle \mathbf{w}_{r}, \overline{\mathbf{W}}^{[L]} \mathbf{w}_{r', s'}^{\dagger}\rangle , \sqrt{m/2} a_{r, d}\Big) \mathbb{I}_{\mathbf{w}_r^{\top} \mathbf{h}^{(L-1)} + \mathbf{a}_r^{\top} \mathbf{x}^{(L)} \ge 0}\Big|
				\\& \le \max_{r', s'} \abs{H_{r', s'}\Big(\theta_{r', s'} \langle \mathbf{w}_{r}, \overline{\mathbf{W}}^{[L]} \mathbf{w}_{r', s'}^{\dagger}\rangle , \sqrt{m/2} a_{r, d}\Big)} \\& \quad \quad \quad \quad 
				\cdot \max_r \abs{\mathbb{I}_{\mathbf{w}_r^{\top} \mathbf{h}^{(L-1)} + \mathbf{a}_r^{\top} \mathbf{x}^{(L)} \ge 0}}
				\\& \quad \quad \quad \quad \cdot \abs{\frac{\dout }{m}   \sum_{s' \in [\dout ]: s' \ne s} \sum_{r' \in [p]} \sum_{r \in [m]}  b_{r, s'} b_{r', s'}^{\dagger} \mathbf{Back}_{L \to L, r, s}}
				\\&\le \max_{r',s'} C_\varepsilon(\Phi_{r', s'}, \mathcal{O}( \varepsilon_x^{-1})) \cdot 1 \cdot \abs{\frac{\dout }{m}   \sum_{s' \in [\dout ]: s' \ne s} \sum_{r' \in [p]} \sum_{r \in [m]}  b_{r, s'} b_{r', s'}^{\dagger} \mathbf{Back}_{L \to L, r, s}} \\ &=
				C_\varepsilon(\Phi, \mathcal{O}( \varepsilon_x^{-1})) \cdot 1 \cdot \abs{\frac{\dout }{m}   \sum_{s' \in [\dout ]: s' \ne s} \sum_{r' \in [p]} \sum_{r \in [m]}  b_{r, s'} b_{r', s'}^{\dagger} b_{r, s}} \\&\le C_\varepsilon(\Phi, \mathcal{O}( \varepsilon_x^{-1})) \cdot 1 \cdot \max_{r'} \abs{b_{r', s'}^{\dagger}} \sum_{r' \in [p]} \sum_{s' \in [\dout ]: s' \ne s}  \abs{\frac{\dout }{m}    \sum_{r \in [m]}  b_{r, s'}  b_{r, s}} \\&
				\le C_\varepsilon(\Phi, \mathcal{O}( \varepsilon_x^{-1})) \cdot 1 \cdot \dout p m^{-1} \cdot 2\sqrt{m} (1 + 2\rho) \\&
				\le 4 C_\varepsilon(\Phi, \mathcal{O}( \varepsilon_x^{-1})) \cdot 1 \cdot \dout p \rho m^{-1/2}.
			\end{align*}
			In the pre-final step, we have used the fact that the dot product between 2 different columns of $\mathbf{B}$, which comes from $\mathcal{N}(0, \mathbb{I})$ is of the order $\sqrt{m}$, w.p. atleast $1-e^{-\Omega(\rho^2)}$.
			\endgroup
			\todo{Add more details here.}
			Hence, 
			\begingroup \allowdisplaybreaks
			\begin{align*}
				&\Big|\frac{\dout }{m} \sum_{i=1}^{L}  \sum_{s' \in [\dout ]: s' \ne s} \sum_{r' \in [p]} \sum_{r \in [m]}  b_{r, s'} b_{r', s'}^{\dagger} \mathbf{Back}_{i \to L, r, s} \\& \quad \quad \quad \quad H_{r', s'}\Big(\theta_{r', s'} \langle \mathbf{w}_{r}, \overline{\mathbf{W}}^{[L]} \mathbf{w}_{r', s'}^{\dagger}\rangle , \sqrt{m/2} a_{r, d}\Big) \mathbb{I}_{\mathbf{w}_r^{\top} \mathbf{h}^{(i-1)} + \mathbf{a}_r^{\top} \mathbf{x}^{(i)} \ge 0}\Big| 
				\\& = \Big|\frac{\dout }{m} \sum_{i=1}^{L-1}  \sum_{s' \in [\dout ]: s' \ne s} \sum_{r' \in [p]} \sum_{r \in [m]}  b_{r, s'} b_{r', s'}^{\dagger} \mathbf{Back}_{i \to L, r, s} \\& \quad \quad \quad \quad H_{r', s'}\Big(\theta_{r', s'} \langle \mathbf{w}_{r}, \overline{\mathbf{W}}^{[L]} \mathbf{w}_{r', s'}^{\dagger}\rangle , \sqrt{m/2} a_{r, d}\Big) \mathbb{I}_{\mathbf{w}_r^{\top} \mathbf{h}^{(i-1)} + \mathbf{a}_r^{\top} \mathbf{x}^{(i)} \ge 0}\Big| 
				\\ &\quad \quad \quad + \Big|\frac{\dout }{m}   \sum_{s' \in [\dout ]: s' \ne s} \sum_{r' \in [p]} \sum_{r \in [m]}  b_{r, s'} b_{r', s'}^{\dagger} \mathbf{Back}_{L \to L, r, s} \\& \quad \quad \quad \quad H_{r', s'}\Big(\theta_{r', s'} \langle \mathbf{w}_{r}, \overline{\mathbf{W}}^{[L]} \mathbf{w}_{r', s'}^{\dagger}\rangle , \sqrt{m/2} a_{r, d}\Big) \mathbb{I}_{\mathbf{w}_r^{\top} \mathbf{h}^{(L-1)} + \mathbf{a}_r^{\top} \mathbf{x}^{(L)} \ge 0}\Big|
				\\&\le  \mathcal{O}(C_\varepsilon(\Phi, \mathcal{O}( \varepsilon_x^{-1})) L p \dout \rho m^{-0.25}) + 4 C_\varepsilon(\Phi, \mathcal{O}( \varepsilon_x^{-1}))  \dout p \rho m^{-1/2} \\&= \mathcal{O}(C_\varepsilon(\Phi, \mathcal{O}( \varepsilon_x^{-1})) L p \dout \rho m^{-0.25}).
			\end{align*}
			\endgroup
		\end{proof}
		
		In the following claim, we show that Eq.~\ref{eq:sprimeesL-1} is small with high probability. The idea is again to invoke Lemma~\ref{lemma:backward_correlation} for bounding correlation between the $s$-th columns of $\mathbf{Back}_{i \to L}$ and $\mathbf{B}$.
		\begin{claim}
			With probability at least $1-e^{-\Omega(\rho^2)}$,
			\begin{align*}
				&\Big|\frac{\dout }{m} \sum_{i=1}^{L-1}  \sum_{r' \in [p]} \sum_{r \in [m]}  b_{r, s} b_{r', s}^{\dagger} \mathbf{Back}_{i \to L, r, s} \\& \quad \quad \quad \quad H_{r', s}\Big(\theta_{r', s} \langle \mathbf{w}_{r}, \overline{\mathbf{W}}^{[L]} \mathbf{w}_{r', s}^{\dagger}\rangle, \sqrt{m/2} a_{r, d}\Big) \mathbb{I}_{\mathbf{w}_r^{\top} \mathbf{h}^{(i-1)} + \mathbf{a}_r^{\top} \mathbf{x}^{(i)} \ge 0}\Big| \\&
				\le \mathcal{O}(C_\varepsilon(\Phi, \mathcal{O}( \varepsilon_x^{-1})) L \rho m^{-0.25}).
			\end{align*}
		\end{claim}
		
		\begin{proof}
			By the definition of $H_{r', s}$ from def.~\ref{Def:Function_approx},
			\begingroup
			\allowdisplaybreaks
			\begin{align*}
				&\Big|\frac{\dout }{m} \sum_{i=1}^{L-1}  \sum_{r' \in [p]} \sum_{r \in [m]}  b_{r, s} b_{r', s}^{\dagger} \mathbf{Back}_{i \to L, r, s} \\& \quad \quad \quad \quad H_{r', s}\Big(\theta_{r', s} \langle \mathbf{w}_{r}, \overline{\mathbf{W}}^{[L]} \mathbf{w}_{r', s}^{\dagger}\rangle, \sqrt{m/2} a_{r, d}\Big) \mathbb{I}_{\mathbf{w}_r^{\top} \mathbf{h}^{(i-1)} + \mathbf{a}_r^{\top} \mathbf{x}^{(i)} \ge 0}\Big| \\&
				\\& \le \max_{r', s} \abs{H_{r', s}\Big(\theta_{r', s} \langle \mathbf{w}_{r}, \overline{\mathbf{W}}^{[L]} \mathbf{w}_{r', s}^{\dagger}\rangle, \sqrt{m/2} a_{r, d}\Big)} \\& \quad \quad \quad \quad 
				\cdot \max_r \abs{\mathbb{I}_{\mathbf{w}_r^{\top} \mathbf{h}^{(i-1)} + \mathbf{a}_r^{\top} \mathbf{x}^{(i)} \ge 0}}
				\\& \quad \quad \quad \quad \cdot \abs{\frac{\dout }{m} \sum_{i=1}^{L-1}   \sum_{r' \in [p]} \sum_{r \in [m]}  b_{r, s} b_{r', s}^{\dagger} \mathbf{Back}_{i \to L, r, s}} \\&
				\le \max_{r'} C_\varepsilon(\Phi_{r', s}, \mathcal{O}(\varepsilon_x^{-1})) \cdot 1 \cdot \abs{\frac{\dout }{m} \sum_{i=1}^{L-1}  \sum_{r' \in [p]} \sum_{r \in [m]}  b_{r, s} b_{r', s}^{\dagger} \mathbf{Back}_{i \to L, r, s}} \\&
				\le C_\varepsilon(\Phi, \mathcal{O}(\varepsilon_x^{-1})) \cdot 1 \cdot \sum_{i=1}^{L-1}  \abs{\frac{\dout }{m} \sum_{r' \in [p]} \sum_{r \in [m]}  b_{r, s} b_{r', s}^{\dagger} \mathbf{Back}_{i \to L, r, s}} \\&
				\le C_\varepsilon(\Phi, \mathcal{O}(\varepsilon_x^{-1})) \cdot 1 \cdot \sum_{i=1}^{L-1}   \max_{r'}  \abs{b_{r', s}^{\dagger}}  \sum_{r' \in [p]}  \abs{\frac{\dout }{m}  \sum_{r \in [m]}  b_{r, s}  \mathbf{Back}_{i \to L, r, s}} \\&
				= C_\varepsilon(\Phi, \mathcal{O}(\varepsilon_x^{-1})) \cdot 1 \cdot \sum_{i=1}^{L-1}\max_{r'}   \abs{b_{r', s}^{\dagger}} \cdot p \cdot  \frac{\dout }{m} \abs{\left\langle \mathbf{e}_{s}^{\top} \mathbf{Back}_{i \to i},  \mathbf{e}_{s}^{\top} \mathbf{Back}_{i \to L} \right\rangle} \\&
				\le C_\varepsilon(\Phi, \mathcal{O}(\varepsilon_x^{-1})) \cdot 1 \cdot (L-1) p  \mathcal{O}(m^{-0.25} \rho) = \mathcal{O}(C_\varepsilon(\Phi, \mathcal{O}(\varepsilon_x^{-1})) L p \rho m^{-0.25}),
			\end{align*}
			\endgroup
			where we use the definition of $\mathbf{Back}_{i \to i}$ to show that it is equivalent to $\mathbf{B}$ for all $i \in [L]$ and we invoke Lemma~\ref{lemma:backward_correlation} in the pre-final step that incurs a probability of $1 - e^{-\Omega(\rho)^2}$.
		\end{proof}
		\fi
		%%%%%%%%%%%%%%%%%%%%%%%%%%%%%%%%%%%%%%%%%%%%%%%%%%%%%%%%%%%%%%%%%%%%%%%%%
		
		
		
		
		
		
		
		%%%%%%%%%%%%%%%%%%%%%%%%%%%%%%%%%%%%%%%%%%%%%%%%%%%%%%%%%%%%%%%%%%%%%%%%%%%%%%%
		\iffalse
		Now, we show that with high probability, Eq.~\ref{eq:sprimeesL} is equal to the desired signal with some noise.
		\begin{claim}
			With probability at least $1 - e^{-\Omega(\rho^2)}$,
			\begin{align*}
				&\frac{\dout }{m}    \sum_{r' \in [p]} \sum_{r \in [m]}  b_{r, s} b_{r', s}^{\dagger} \mathbf{Back}_{L \to L, r, s} \\& \quad \quad \quad \quad H_{r', s}\Big(\theta_{r', s} \langle \mathbf{w}_{r}, \overline{\mathbf{W}}^{[L]} \mathbf{w}_{r', s}^{\dagger}\rangle, \sqrt{m/2} a_{r, d}\Big) \mathbb{I}_{\mathbf{w}_r^{\top} \mathbf{h}^{(L-1)} + \mathbf{a}_r^{\top} \mathbf{x}^{(L)} \ge 0}
				\\&= \sum_{r' \in[p]} b_{r, s}^{\dagger} \Phi_{r', s} \left(\left\langle \mathbf{w}_{r', s}^{\dagger}, [\overline{\mathbf{x}}^{(1)}, \cdots, \overline{\mathbf{x}}^{(L)}]\right\rangle\right) \pm XXX
			\end{align*}
			
			\begin{proof}
				We can write the desired term as follows.
				\begin{align*}
					&\frac{\dout }{m}    \sum_{r' \in [p]} \sum_{r \in [m]}  b_{r, s} b_{r', s}^{\dagger} \mathbf{Back}_{L \to L, r, s} \\& \quad \quad \quad \quad H_{r', s}\Big(\theta_{r', s} \langle \mathbf{w}_{r}, \overline{\mathbf{W}}^{[L]} \mathbf{w}_{r', s}^{\dagger}\rangle, \sqrt{m/2} a_{r, d}\Big) \mathbb{I}_{\mathbf{w}_r^{\top} \mathbf{h}^{(L-1)} + \mathbf{a}_r^{\top} \mathbf{x}^{(L)} \ge 0} \\
					&= \sum_{r' \in [p]} b_{r', s}^{\dagger} \left(\frac{\dout }{m} \sum_{r \in [m]} b_{r, s}^2 H_{r', s}\Big(\theta_{r', s} \langle \mathbf{w}_{r}, \overline{\mathbf{W}}^{[L]} \mathbf{w}_{r', s}^{\dagger}\rangle, \sqrt{m/2} a_{r, d}\Big) \mathbb{I}_{\mathbf{w}_r^{\top} \mathbf{h}^{(L-1)} + \mathbf{a}_r^{\top} \mathbf{x}^{(L)} \ge 0} \right) 
				\end{align*}
				Next, we will try to show the following 3 terms to be close:
				\begin{align}
					&\frac{\dout }{m} \sum_{r \in [m]} b_{r, s}^2 H_{r', s}\Big(\theta_{r', s} \langle \mathbf{w}_{r}, \overline{\mathbf{W}}^{[L]} \mathbf{w}_{r', s}^{\dagger}\rangle, \sqrt{m/2} a_{r, d}\Big) \mathbb{I}_{\mathbf{w}_r^{\top} \mathbf{h}^{(L-1)} + \mathbf{a}_r^{\top} \mathbf{x}^{(L)} \ge 0} \label{eqn:average}\\
					& \mathbb{E}_{b \sim \mathcal{N}(0, 1), \mathbf{a} \sim \mathcal{N}(0, \frac{2}{m}\mathbf{I}_{d}), \mathbf{w} \sim \mathcal{N}(0, \frac{2}{m}\mathbf{I}_{m}} b^2 H_{r', s}\left(\theta_{r', s} \left(\langle \mathbf{w}, \overline{\mathbf{W}}^{[L]} \mathbf{w}_{r', s}^{\dagger}\rangle + \langle \mathbf{a}^{[d-1]},  \mathbf{P}^{\perp}\mathbf{w}_{r', s}^{\dagger}  \rangle\right), \theta_{r', s} a_d\right) \mathbb{I}_{\mathbf{w}^{\top} \mathbf{h}^{(L-1)} + \mathbf{a}^{\top} \mathbf{x}^{(L)} \ge 0} \label{eqn:expect}
					\\& 
					\Phi_{r', s} \left(\left\langle \mathbf{w}_{r', s}^{\dagger}, [\overline{\mathbf{x}}^{(1)}, \cdots, \overline{\mathbf{x}}^{(L)}]\right\rangle\right). \label{eqn:Phi}
				\end{align}
				The major issue in showing that Eq.\ref{eqn:average} is a discrete version of Eq.\ref{eqn:expect}, is due to the coupling involved between the randomness of $\overline{\mathbf{W}}^{[L]}$, $\mathbf{W}$, $\mathbf{A}$ and $\mathbf{h}^{(L)}$.  To decouple this randomness, we use the fact that ESNs are stable to re-randomization of few rows of the weight matrices and follow the proof technique of \cite[Lemma G.3]{allen2019can}. 
				
				Choose a random subset $\mathcal{K} \subset[m]$ of size $|\mathcal{K}|=N,$. Define the function $F^{\mathcal{K}}_s$ as
				\begin{equation*}
					F_{s}^{\mathcal{K}}(\mathbf{h}^{(L-1)}, \mathbf{x}^{(L)}) = \frac{\dout }{m} \sum_{r \in \mathcal{K}} b_{r, s}^2 H_{r', s}\Big(\theta_{r', s} \langle \mathbf{w}_{r}, \overline{\mathbf{W}}^{[L]} \mathbf{w}_{r', s}^{\dagger}\rangle, \sqrt{m/2} a_{r, d}\Big) \mathbb{I}_{\mathbf{w}_r^{\top} \mathbf{h}^{(L-1)} + \mathbf{a}_r^{\top} \mathbf{x}^{(L)} \ge 0}.  
				\end{equation*}
				
				Replace the rows $\left\{\mathbf{w}_{k}, \mathbf{a}_{k}\right\}_{k \in \mathcal{K}}$ of $\mathbf{W}$ and $\mathbf{A}$ with freshly new i.i.d. samples $\widetilde{\mathbf{w}}_{k}, \widetilde{\mathbf{a}}_{k} \sim \mathcal{N}\left(0, \frac{2}{m} \mathbf{I}\right).$ to form new matrices $\widetilde{\mathbf{W}}$ and $\widetilde{\mathbf{A}}$. For the given sequence, we follow the notation of Lemma~\ref{lemma:rerandESN} to denote the hidden states corresponding to the old and the new weight matrices. Let $\widetilde{F}^{\mathcal{K}}_s$ denote the following function:
				\begin{equation*} \widetilde{F}^{\mathcal{K}}_s(\mathbf{h}^{(L-1)}, \mathbf{x}^{(L)}) = \frac{\dout }{m} \sum_{r \in \mathcal{K}} b_{r, s}^2 H_{r', s}\Big(\widetilde{\theta}_{r', s} (\langle \mathbf{w}_{r}, \overline{\widetilde{\mathbf{W}}}^{[L]} \mathbf{w}_{r', s}^{\dagger}\rangle + \langle \mathbf{a}_{r}^{[d-1]},  \mathbf{P}^{\perp}\mathbf{w}_{r', s}^{\dagger}  \rangle), \sqrt{m/2} a_{r, d}\Big) \mathbb{I}_{\mathbf{w}_r^{\top} \mathbf{h}^{(L-1)} + \mathbf{a}_r^{\top} \mathbf{x}^{(L)} \ge 0},
				\end{equation*}
				where
				\begin{equation*} 
					\widetilde{\theta}_{r', s} = \frac{\sqrt{m/2}}{\sqrt{ \norm[1]{\mathbf{P}^{\perp}\mathbf{w}_{r', s}^{\dagger}}^2 + \norm[2]{\overline{\widetilde{\mathbf{W}}}^{[L]} \mathbf{w}_{r', s}^{\dagger}}^2 }}.
				\end{equation*}
				Since from claim~\ref{Claimm:highprob_normW}, we have with probability at least $1-e^{-\Omega(\rho^2)}$, $\norm[2]{\overline{\widetilde{\mathbf{W}}}^{[L]} \mathbf{w}_{r', s}^{\dagger}} \ge 2^{L/2} \sqrt{L} (1 - \rho L m^{-1/2}) \norm{\mathbf{w}_{r', s}^{\dagger}}$ and $\norm[2]{\overline{\widetilde{\mathbf{W}}}^{[L]} \mathbf{w}_{r', s}^{\dagger}} \le 2^{L/2} \sqrt{L} (1 + \rho L m^{-1/2}) \norm{\mathbf{w}_{r', s}^{\dagger}}$, and thus we have $\forall  r' \in [p], s \in [\dout ]$,
				\begin{align}
					\sqrt{2/m} \widetilde{\theta}_{r', s} &= \frac{1}{\sqrt{ \norm[1]{\mathbf{P}^{\perp}\mathbf{w}_{r', s}^{\dagger}}^2 + \norm[2]{\overline{\widetilde{\mathbf{W}}}^{[L]} \mathbf{w}_{r', s}^{\dagger}}^2 }}\nonumber\\&
					\le \frac{1}{\sqrt{  \norm[1]{\mathbf{P}^{\perp}\mathbf{w}_{r', s}^{\dagger}}^2 + 2^{L} L (1 + \rho L m^{-1/2})^2\norm[2]{ \mathbf{w}_{r', s}^{\dagger}}^2 }} \nonumber\\&
					\le \frac{1}{2^{L/2} \sqrt{L} (1 + \rho L m^{-1/2})} \frac{1}{\sqrt{  \norm[1]{\mathbf{P}^{\perp}\mathbf{w}_{r', s}^{\dagger}}^2 + \norm[2]{ \mathbf{w}_{r', s}^{\dagger}}^2 }} \nonumber \\&
					=  \frac{1}{2^{L/2} \sqrt{L} (1 + \rho L m^{-1/2})} \frac{1}{\norm{\mathbf{w}_{r', s}^{\dagger}}} \nonumber\\&
					= 2^{-L/2} L^{-1/2} \frac{1}{1 + \rho L m^{-1/2}} \label{eq:abstildetheta_lowerbound}.
				\end{align}
				Similarly,
				\begingroup
				\allowdisplaybreaks
				\begin{align}
					\sqrt{2/m} \widetilde{\theta}_{r', s} &= \frac{1}{\sqrt{ \norm[1]{\mathbf{P}^{\perp}\mathbf{w}_{r', s}^{\dagger}}^2 + \norm[2]{\overline{\widetilde{\mathbf{W}}}^{[L]} \mathbf{w}_{r', s}^{\dagger}}^2 }}\nonumber\\&
					\le \frac{1}{\sqrt{  \norm[1]{\mathbf{P}^{\perp}\mathbf{w}_{r', s}^{\dagger}}^2 + 2^{L} L (1 - \rho L m^{-1/2})^2\norm[2]{ \mathbf{w}_{r', s}^{\dagger}}^2 }} \nonumber\\&
					\le  \frac{1}{\sqrt{  \norm[1]{\mathbf{P}^{\perp}\mathbf{w}_{r', s}^{\dagger}}^2 + \norm[2]{ \mathbf{w}_{r', s}^{\dagger}}^2 }} 
					=   \frac{1}{\norm{\mathbf{w}_{r', s}^{\dagger}}}
					= 1. \label{eq:abstildetheta_upperbound}
				\end{align}
				\endgroup
				In the following claims, we will bound the following term.
				\begin{align*}
					\abs{F^{\mathcal{K}}_s(\mathbf{h}^{(L-1)}, \mathbf{x}^{(L)}) -\frac{N}{m} \Phi_{r', s} \left(\left\langle \mathbf{w}_{r', s}^{\dagger}, [\overline{\mathbf{x}}^{(1)}, \cdots, \overline{\mathbf{x}}^{(L)}]\right\rangle\right)}
				\end{align*}
				We break the above term into 3 terms, and bound each of these 3 terms.
				\begin{align}
					&\abs{F^{\mathcal{K}}_s(\mathbf{h}^{(L-1)}, \mathbf{x}^{(L)}) -\frac{N}{m} \Phi_{r', s} \left(\left\langle \mathbf{w}_{r', s}^{\dagger}, [\overline{\mathbf{x}}^{(1)}, \cdots, \overline{\mathbf{x}}^{(L)}]\right\rangle\right)} \nonumber\\&\le
					\abs{F^{\mathcal{K}}_s(\mathbf{h}^{(L-1)}, \mathbf{x}^{(L)}) - F^{\mathcal{K}}_s(\widetilde{\mathbf{h}}^{(L-1)}, \mathbf{x}^{(L)})} \label{eq:diffhtildeh}\\&
					\quad + \abs{F^{\mathcal{K}}_s(\widetilde{\mathbf{h}}^{(L-1)}, \mathbf{x}^{(L)}) - \widetilde{F}^{\mathcal{K}}_s(\widetilde{\mathbf{h}}^{(L-1)}, \mathbf{x}^{(L)})} \label{eq:diffftildef}
					\\&
					\quad + \abs{\widetilde{F}^{\mathcal{K}}_s(\widetilde{\mathbf{h}}^{(L-1)}, \mathbf{x}^{(L)}) - \frac{N}{m} \Phi_{r', s} \left(\left\langle \mathbf{w}_{r', s}^{\dagger}, [\overline{\mathbf{x}}^{(1)}, \cdots, \overline{\mathbf{x}}^{(L)}]\right\rangle\right)} \label{eq:difftildefphi}  
				\end{align}
				
				In the following claim, we show that Eq.~\ref{eq:diffhtildeh} is small.
				\begin{claim}\label{claim:diffhtildeh}
					With probability at least $1 - 2me^{-\rho^2} - e^{-\Omega(\rho^2)}$,
					\begin{equation*}
						\abs{F^{\mathcal{K}}_s(\mathbf{h}^{(L-1)}, \mathbf{x}^{(L)}) - F^{\mathcal{K}}_s(\widetilde{\mathbf{h}}^{(L-1)}, \mathbf{x}^{(L)})} \le \mathcal{O}( \mathfrak{C}_{\varepsilon}(\Phi_{r', s}, 1) \rho^{7} N^{5 / 3} m^{-7/6}).
					\end{equation*}
				\end{claim}
				
				\begin{proof}
					\begin{align*}
						&\abs{F^{\mathcal{K}}_s(\mathbf{h}^{(L-1)}, \mathbf{x}^{(L)}) - F^{\mathcal{K}}_s(\widetilde{\mathbf{h}}^{(L-1)}, \mathbf{x}^{(L)})}\\ &= 
						\Big|\frac{\dout }{m} \sum_{r \in \mathcal{K}} b_{r, s}^2 H_{r', s}\Big(\theta_{r', s} \langle \mathbf{w}_{r}, \overline{\mathbf{W}}^{[L]} \mathbf{w}_{r', s}^{\dagger}\rangle, \sqrt{m/2} a_{r, d}\Big)
						\\& \quad\quad\quad\quad\quad\quad\quad\quad\quad
						\cdot \left(\mathbb{I}_{\mathbf{w}_r^{\top} \mathbf{h}^{(L-1)} + \mathbf{a}_r^{\top} \mathbf{x}^{(L)} \ge 0} - \mathbb{I}_{\mathbf{w}_r^{\top} \widetilde{\mathbf{h}}^{(L-1)} + \mathbf{a}_r^{\top} \mathbf{x}^{(L)} \ge 0}\right)\Big| \\&
						\le \max_{r} b^{2}_{r, s} \max_{r} \abs{H_{r', s}\Big(\theta_{r', s} \langle \mathbf{w}_{r}, \overline{\mathbf{W}}^{[L]} \mathbf{w}_{r', s}^{\dagger}\rangle, \sqrt{m/2} a_{r, d}\Big)}  
						\\& \quad \quad \quad \quad \quad \quad \quad \quad \quad
						\abs{\frac{\dout }{m} \sum_{r \in \mathcal{K}} \left(\mathbb{I}_{\mathbf{w}_r^{\top} \mathbf{h}^{(L-1)} + \mathbf{a}_r^{\top} \mathbf{x}^{(L)} \ge 0} - \mathbb{I}_{\mathbf{w}_r^{\top} \widetilde{\mathbf{h}}^{(L-1)} + \mathbf{a}_r^{\top} \mathbf{x}^{(L)} \ge 0}\right)} 
						%\abs{ }
					\end{align*}
					
					From Fact~\ref{fact:max_gauss}, we have w.p. atleast $1 - 2me^{-\rho^2}$,
					\begin{equation*}
						\max_{r} b^{2}_{r, s} = 2 \rho^2.
					\end{equation*}
					
					Since, the function $H$ is a bounded function by Def~\ref{Def:Function_approx}, we have
					\begin{equation*}
						\max_{r} \abs{H_{r', s}\Big(\theta_{r', s} \langle \mathbf{w}_{r}, \overline{\mathbf{W}}^{[L]} \mathbf{w}_{r', s}^{\dagger}\rangle, \sqrt{m/2} a_{r, d}\Big)} \le \mathfrak{C}_{\varepsilon}(\Phi_{r', s}, 1).
					\end{equation*}
					
					
					%W.p. at least $1 - e^{-\Omega(\rho^2)}$,    
					Lemma~\ref{lemma:rerandESN} shows that with probability at least $1 - e^{-\Omega(\rho^2)}$,
					\begin{equation*}
						\abs[0]{\mathbf{w}_r^{\top} (\widetilde{\mathbf{h}}^{(L-1)} -  \mathbf{h}^{(L-1)}} \leq \mathcal{O}\left(\rho^{5} N^{2 / 3} m^{-2 / 3}\right) \quad \text { for every } r \in [m], \ell \in[L].
					\end{equation*}
					
					Also, w.p. at least $1 - e^{-\Omega(\rho^2)}$,
					\begin{equation*}
						\sum_{r \in \mathcal{K}} \mathbb{I}\left[\abs{\mathbf{w}_r^{\top} \mathbf{h}^{(L-1)} + \mathbf{a}_r^{\top} \mathbf{x}^{(L-1)}} \le \rho^{5} N^{2 / 3} m^{-2 / 3}\right] \le \mathcal{O}(\rho^{5} N^{5 / 3} m^{-1/6}).
					\end{equation*}
					\todo{Add details here!!}
					
					Hence, that implies  w.p. at least $1 - e^{-\Omega(\rho^2)}$,
					\begin{align*}
						&\sum_{r \in \mathcal{K}} \abs{\mathbb{I}\left[\mathbf{w}_r^{\top} \mathbf{h}^{(L-1)} + \mathbf{a}_r^{\top} \mathbf{x}^{(L-1)} \right] - \mathbb{I}\left[\mathbf{w}_r^{\top} \widetilde{\mathbf{h}}^{(L-1)} + \mathbf{a}_r^{\top} \mathbf{x}^{(L-1)} \right]}\\&
						\le \sum_{r \in \mathcal{K}}\mathbb{I}\left[\abs[0]{\mathbf{w}_r^{\top} \mathbf{h}^{(L-1)} + \mathbf{a}_r^{\top} \mathbf{x}^{(L-1)}} \le \abs[0]{\mathbf{w}_r^{\top} \widetilde{\mathbf{h}}^{(L-1)} + \mathbf{a}_r^{\top} \mathbf{x}^{(L-1)} - \mathbf{w}_r^{\top} \mathbf{h}^{(L-1)} - \mathbf{a}_r^{\top} \mathbf{x}^{(L-1)}} \right] \\&
						= \sum_{r \in \mathcal{K}}\mathbb{I}\left[\abs[0]{\mathbf{w}_r^{\top} \mathbf{h}^{(L-1)} + \mathbf{a}_r^{\top} \mathbf{x}^{(L-1)}} \le \abs[0]{\mathbf{w}_r^{\top} (\widetilde{\mathbf{h}}^{(L-1)} -  \mathbf{h}^{(L-1)}}) \right]
						\\&\le \sum_{r \in \mathcal{K}}\mathbb{I}\left[\abs[0]{\mathbf{w}_r^{\top} \mathbf{h}^{(L-1)} + \mathbf{a}_r^{\top} \mathbf{x}^{(L-1)}} \le \mathcal{O}\left(\rho^{5} N^{2 / 3} m^{-2 / 3}\right) \right] \\&
						\le \mathcal{O}(\rho^{5} N^{5 / 3} m^{-1/6}).
					\end{align*}            
					
					
					
					%\begin{align}
					%\norm{\mathbf{h}^{(\ell-1)} - \widetilde{h}}
					%    & \sum_{r \in [m]} \abs{ \left(\mathbb{I}_{\mathbf{w}_r^{\top} \mathbf{h}^{(L-1)} + \mathbf{a}_r^{\top} \mathbf{x}^{(L)} \ge 0} - \mathbb{I}_{\mathbf{w}_r^{\top} \widetilde{\mathbf{h}}^{(L-1)} + \mathbf{a}_r^{\top} \mathbf{x}^{(L)} \ge 0}\right)} \le \mathcal{O}(\rho^4 N^{1/3} m^{2/3}).
					%\end{align}
					Combining the above inequalities, we have with probability at least $1 - 2me^{-\rho^2} - e^{-\Omega(\rho^2)}$,
					\begin{equation*}
						\abs{F^{\mathcal{K}}_s(\mathbf{h}^{(L-1)}, \mathbf{x}^{(L)}) - F^{\mathcal{K}}_s(\widetilde{\mathbf{h}}^{(L-1)}, \mathbf{x}^{(L)})} \le \mathcal{O}( \mathfrak{C}_{\varepsilon}(\Phi_{r', s}, 1) \rho^{7} N^{5 / 3} m^{-7/6}).
					\end{equation*}
				\end{proof}
				
				
				In the following claim, we show that Eq.~\ref{eq:diffftildef} is small.      
				\begin{claim}\label{claim:diffftildef}
					With probability at least $1-e^{-\Omega(\rho^2)}$,
					\begin{align*}  &\abs{F^{\mathcal{K}}_s(\widetilde{\mathbf{h}}^{(L-1)}, \mathbf{x}^{(L)}) - \widetilde{F}^{\mathcal{K}}_s(\widetilde{\mathbf{h}}^{(L-1)}, \mathbf{x}^{(L)})} \\&
						\le \frac{N}{m} \cdot 2\rho^2 \cdot 2\mathfrak{C}_{\varepsilon}(\Phi_{r', s}, 1)  \cdot \sqrt{2} \rho \cdot \sqrt{3L} \cdot (2 (\sqrt{2} + \rho m^{-0.5}))^{L} \cdot \sqrt{\frac{2}{m}} \cdot   (\sqrt{N} + \sqrt{d} + \sqrt{2}\rho).
					\end{align*}
				\end{claim}
				
				\begin{proof}
					First of all, we use the definition of $f$ and $\widetilde{F}$ to get
					\begin{align} &\abs{F^{\mathcal{K}}_s(\widetilde{\mathbf{h}}^{(L-1)}, \mathbf{x}^{(L)}) - \widetilde{F}^{\mathcal{K}}_s(\widetilde{\mathbf{h}}^{(L-1)}, \mathbf{x}^{(L)})} \nonumber
						\\&
						= \Big|\frac{\dout }{m} \sum_{r \in \mathcal{K}} b_{r, s}^2 H_{r', s}\Big(\widetilde{\theta}_{r', s} (\langle \mathbf{w}_{r}, \overline{\widetilde{\mathbf{W}}}^{[L]} \mathbf{w}_{r', s}^{\dagger}\rangle + \langle \mathbf{a}_{r}^{[d-1]},  \mathbf{P}^{\perp}\mathbf{w}_{r', s}^{\dagger}  \rangle), \sqrt{m/2} a_{r, d} \Big) \mathbb{I}_{\mathbf{w}_r^{\top} \widetilde{\mathbf{h}}^{(L-1)} + \mathbf{a}_r^{\top} \mathbf{x}^{(L)} \ge 0} \nonumber\\&
						\quad - \frac{\dout }{m} \sum_{r \in \mathcal{K}} b_{r, s}^2 H_{r', s}\Big(\theta_{r', s} \langle \mathbf{w}_{r}, \overline{\mathbf{W}}^{[L]} \mathbf{w}_{r', s}^{\dagger}\rangle, \sqrt{m/2} a_{r, d}\Big) \mathbb{I}_{\mathbf{w}_r^{\top} \widetilde{\mathbf{h}}^{(L-1)} + \mathbf{a}_r^{\top} \mathbf{x}^{(L)} \ge 0}\Big| \nonumber\\&
						\le \frac{N}{m} \cdot \max_{r} b^{2}_{r, s} \mathbb{I}_{\mathbf{w}_r^{\top} \widetilde{\mathbf{h}}^{(L-1)} + \mathbf{a}_r^{\top} \mathbf{x}^{(L)} \ge 0}  \nonumber\\& \cdot \Big|\frac{1}{N}\sum_{r 
							\in \mathcal{K}} \Big[ H_{r', s}\left(\theta_{r', s} (\langle \mathbf{w}_{r}, \overline{\mathbf{W}}^{[L]} \mathbf{w}_{r', s}^{\dagger}\rangle + \langle \mathbf{a}_{r}^{[d-1]},  \mathbf{P}^{\perp}\mathbf{w}_{r', s}^{\dagger}  \rangle)\right) \nonumber\\&\quad\quad\quad\quad\quad\quad\quad\quad
						- H_{r', s}\Big(\widetilde{\theta}_{r', s} (\langle \mathbf{w}_{r}, \overline{\widetilde{\mathbf{W}}}^{[L]} \mathbf{w}_{r', s}^{\dagger}\rangle + \langle \mathbf{a}_{r}^{[d-1]},  \mathbf{P}^{\perp}\mathbf{w}_{r', s}^{\dagger}  \rangle)\Big)\Big]\Big| \label{eq:diff_f_tildef}
					\end{align}
					
					From Fact~\ref{fact:max_gauss}, we have w.p. atleast $1 - 2me^{-\rho^2}$,
					\begin{equation*}
						\max_{r} b^{2}_{r, s} = 2 \rho^2,
					\end{equation*}
					which further implies
					\begin{equation*}
						\max_{r} b^{2}_{r, s} \mathbb{I}_{\mathbf{w}_r^{\top} \widetilde{\mathbf{h}}^{(L-1)} + \mathbf{a}_r^{\top} \mathbf{x}^{(L)} \ge 0} \le \max_{r} b^{2}_{r, s}  = 2 \rho^2.
					\end{equation*}
					Since, the function $H$ is a lipschitz function by Def~\ref{Def:Function_approx}, we have
					\begin{align}
						&\Big|H_{r', s}\left(\theta_{r', s} \left(\langle \mathbf{w}_{r}, \overline{\mathbf{W}}^{[L]} \mathbf{w}_{r', s}^{\dagger}\rangle + \langle \mathbf{a}_{r}^{[d-1]},  \mathbf{P}^{\perp}\mathbf{w}_{r', s}^{\dagger}  \rangle\right), \sqrt{m/2} a_{r, d}\right) \\&
						\quad\quad\quad\quad\quad\quad\quad\quad\quad- H_{r', s}\big(\widetilde{\theta}_{r', s} (\langle \mathbf{w}_{r}, \overline{\widetilde{\mathbf{W}}}^{[L]} \mathbf{w}_{r', s}^{\dagger}\rangle + \langle \mathbf{a}_{r}^{[d-1]},  \mathbf{P}^{\perp}\mathbf{w}_{r', s}^{\dagger}  \rangle), \sqrt{m/2} a_{r, d}\big)\Big| 
						\\& \le \mathfrak{C}_{\varepsilon}(\Phi_{r', s}, 1) \abs{\theta_{r', s} \left(\langle \mathbf{w}_{r}, \overline{\mathbf{W}}^{[L]} \mathbf{w}_{r', s}^{\dagger}\rangle + \langle \mathbf{a}_{r}^{[d-1]},  \mathbf{P}^{\perp}\mathbf{w}_{r', s}^{\dagger}  \rangle\right) - \widetilde{\theta}_{r', s} \left(\langle \mathbf{w}_{r}, \overline{\widetilde{\mathbf{W}}}^{[L]} \mathbf{w}_{r', s}^{\dagger}\rangle + \langle \mathbf{a}_{r}^{[d-1]},  \mathbf{P}^{\perp}\mathbf{w}_{r', s}^{\dagger}  \rangle\right)} \nonumber
					\end{align}
					Using the definition of $\theta_{r', s}$ and $\widetilde{\theta}_{r' ,s}$, we continue as follows.
					\begin{align}
						&\mathfrak{C}_{\varepsilon}(\Phi_{r', s}, 1) \abs{\theta_{r', s} \left(\langle \mathbf{w}_{r}, \overline{\mathbf{W}}^{[L]} \mathbf{w}_{r', s}^{\dagger}\rangle + \langle \mathbf{a}_{r}^{[d-1]},  \mathbf{P}^{\perp}\mathbf{w}_{r', s}^{\dagger}  \rangle\right) - \widetilde{\theta}_{r', s} \left(\langle \mathbf{w}_{r}, \overline{\widetilde{\mathbf{W}}}^{[L]} \mathbf{w}_{r', s}^{\dagger}\rangle + \langle \mathbf{a}_{r}^{[d-1]},  \mathbf{P}^{\perp}\mathbf{w}_{r', s}^{\dagger}  \rangle\right)} \nonumber\\&= \mathfrak{C}_{\varepsilon}(\Phi_{r', s}, 1) 
						\abs{\left\langle [\mathbf{w}_r, \mathbf{a}_r], \frac{[\overline{\mathbf{W}}^{[L]} \mathbf{w}_{r', s}^{\dagger}, \mathbf{P}^{\perp}\mathbf{w}_{r', s}^{\dagger}]}{\norm{[\overline{\mathbf{W}}^{[L]} \mathbf{w}_{r', s}^{\dagger}, \mathbf{P}^{\perp}\mathbf{w}_{r', s}^{\dagger}]}} - \frac{[\overline{\widetilde{\mathbf{W}}}^{[L]} \mathbf{w}_{r', s}^{\dagger}, \mathbf{P}^{\perp}\mathbf{w}_{r', s}^{\dagger}]}{\norm[2]{[\overline{\widetilde{\mathbf{W}}}^{[L]} \mathbf{w}_{r', s}^{\dagger}, \mathbf{P}^{\perp}\mathbf{w}_{r', s}^{\dagger}]}}\right\rangle} \nonumber\\&
						\le \mathfrak{C}_{\varepsilon}(\Phi_{r', s}, 1) 
						\abs{\left\langle [\mathbf{w}_r, \mathbf{a}_r], \frac{[\overline{\mathbf{W}}^{[L]} \mathbf{w}_{r', s}^{\dagger}, \mathbf{P}^{\perp}\mathbf{w}_{r', s}^{\dagger}]}{\norm{[\overline{\mathbf{W}}^{[L]} \mathbf{w}_{r', s}^{\dagger}, \mathbf{P}^{\perp}\mathbf{w}_{r', s}^{\dagger}]}} - \frac{[\overline{\widetilde{\mathbf{W}}}^{[L]} \mathbf{w}_{r', s}^{\dagger}, \mathbf{P}^{\perp}\mathbf{w}_{r', s}^{\dagger}]}{\norm[2]{[\overline{\mathbf{W}}^{[L]} \mathbf{w}_{r', s}^{\dagger}, \mathbf{P}^{\perp}\mathbf{w}_{r', s}^{\dagger}]}}\right\rangle}  \label{eq:stability_barW_1}\\&
						+ \mathfrak{C}_{\varepsilon}(\Phi_{r', s}, 1) 
						\abs{\left\langle [\mathbf{w}_r, \mathbf{a}_r], \frac{[\overline{\widetilde{\mathbf{W}}}^{[L]} \mathbf{w}_{r', s}^{\dagger}, \mathbf{P}^{\perp}\mathbf{w}_{r', s}^{\dagger}]}{\norm{[\overline{\mathbf{W}}^{[L]} \mathbf{w}_{r', s}^{\dagger}, \mathbf{P}^{\perp}\mathbf{w}_{r', s}^{\dagger}]}} - \frac{[\overline{\widetilde{\mathbf{W}}}^{[L]} \mathbf{w}_{r', s}^{\dagger}, \mathbf{P}^{\perp}\mathbf{w}_{r', s}^{\dagger}]}{\norm[2]{[\overline{\widetilde{\mathbf{W}}}^{[L]} \mathbf{w}_{r', s}^{\dagger}, \mathbf{P}^{\perp}\mathbf{w}_{r', s}^{\dagger}]}}\right\rangle}  \label{eq:stability_barW_2}
					\end{align}
					
					Eq.~\ref{eq:stability_barW_1} involves the term
					\begin{align*}
						\frac{[\overline{\mathbf{W}}^{[L]} \mathbf{w}_{r', s}^{\dagger}, \mathbf{P}^{\perp}\mathbf{w}_{r', s}^{\dagger}]}{\norm{[\overline{\mathbf{W}}^{[L]} \mathbf{w}_{r', s}^{\dagger}, \mathbf{P}^{\perp}\mathbf{w}_{r', s}^{\dagger}]}} - \frac{[\overline{\widetilde{\mathbf{W}}}^{[L]} \mathbf{w}_{r', s}^{\dagger}, \mathbf{P}^{\perp}\mathbf{w}_{r', s}^{\dagger}]}{\norm[2]{[\overline{\mathbf{W}}^{[L]} \mathbf{w}_{r', s}^{\dagger}, \mathbf{P}^{\perp}\mathbf{w}_{r', s}^{\dagger}]}}.
					\end{align*}
					
					and  Eq.~\ref{eq:stability_barW_2} involves the term
					\begin{equation*}
						\frac{[\overline{\widetilde{\mathbf{W}}}^{[L]} \mathbf{w}_{r', s}^{\dagger}, \mathbf{P}^{\perp}\mathbf{w}_{r', s}^{\dagger}]}{\norm{[\overline{\mathbf{W}}^{[L]} \mathbf{w}_{r', s}^{\dagger}, \mathbf{P}^{\perp}\mathbf{w}_{r', s}^{\dagger}]}} - \frac{[\overline{\widetilde{\mathbf{W}}}^{[L]} \mathbf{w}_{r', s}^{\dagger}, \mathbf{P}^{\perp}\mathbf{w}_{r', s}^{\dagger}]}{\norm[2]{[\overline{\widetilde{\mathbf{W}}}^{[L]} \mathbf{w}_{r', s}^{\dagger}, \mathbf{P}^{\perp}\mathbf{w}_{r', s}^{\dagger}]}}.
					\end{equation*}
					In the following two claims, we will try to bound the norms of the above two terms.
					
					\begin{claim}\label{claim:stability_barW_1}
						`                   With probability at least $1-e^{-\Omega(\rho^2)}$,
						
						\begin{equation*}
							\norm{\frac{[\overline{\mathbf{W}}^{[L]} \mathbf{w}_{r', s}^{\dagger}, \mathbf{P}^{\perp}\mathbf{w}_{r', s}^{\dagger}]}{\norm{[\overline{\mathbf{W}}^{[L]} \mathbf{w}_{r', s}^{\dagger}, \mathbf{P}^{\perp}\mathbf{w}_{r', s}^{\dagger}]}} - \frac{[\overline{\widetilde{\mathbf{W}}}^{[L]} \mathbf{w}_{r', s}^{\dagger}, \mathbf{P}^{\perp}\mathbf{w}_{r', s}^{\dagger}]}{\norm[2]{[\overline{\mathbf{W}}^{[L]} \mathbf{w}_{r', s}^{\dagger}, \mathbf{P}^{\perp}\mathbf{w}_{r', s}^{\dagger}]}}} \le \sqrt{3L} \cdot (2 (\sqrt{2} + \rho m^{-0.5}))^{L} \cdot \sqrt{\frac{2}{m}} \cdot   (\sqrt{N} + \sqrt{d} + \sqrt{2}\rho).
						\end{equation*}
					\end{claim}
					
					\begin{proof}
						Using the definition of $\overline{\mathbf{W}}^{[L]}$ and $\overline{\widetilde{\mathbf{W}}}^{[L]}$, we have
						\begin{align*}
							\norm{\frac{[\overline{\mathbf{W}}^{[L]} \mathbf{w}_{r', s}^{\dagger}, \mathbf{P}^{\perp}\mathbf{w}_{r', s}^{\dagger}]}{\norm{[\overline{\mathbf{W}}^{[L]} \mathbf{w}_{r', s}^{\dagger}, \mathbf{P}^{\perp}\mathbf{w}_{r', s}^{\dagger}]}} - \frac{[\overline{\widetilde{\mathbf{W}}}^{[L]} \mathbf{w}_{r', s}^{\dagger}, \mathbf{P}^{\perp}\mathbf{w}_{r', s}^{\dagger}]}{\norm[2]{[\overline{\mathbf{W}}^{[L]} \mathbf{w}_{r', s}^{\dagger}, \mathbf{P}^{\perp}\mathbf{w}_{r', s}^{\dagger}]}}} 
							&= \frac{\norm[2]{\overline{\mathbf{W}}^{[L]} \mathbf{w}_{r', s}^{\dagger} -  \overline{\widetilde{\mathbf{W}}}^{[L]} \mathbf{w}_{r', s}^{\dagger}}}
							{\norm{[\overline{\mathbf{W}}^{[L]} \mathbf{w}_{r', s}^{\dagger}, \mathbf{P}^{\perp}\mathbf{w}_{r', s}^{\dagger}]} } \\&
							= \frac{\norm[2]{\left(\overline{\mathbf{W}}^{[L]} -  \overline{\widetilde{\mathbf{W}}}^{[L]}\right)^{\top} \mathbf{w}_{r', s}^{\dagger} }}
							{\norm{[\overline{\mathbf{W}}^{[L]} \mathbf{w}_{r', s}^{\dagger}, \mathbf{P}^{\perp}\mathbf{w}_{r', s}^{\dagger}]} } \\&
							\le \frac{\norm[2]{\overline{\mathbf{W}}^{[L]} -  \overline{\widetilde{\mathbf{W}}}^{[L]}} \norm{\mathbf{w}_{r', s}^{\dagger} }}
							{\norm{[\overline{\mathbf{W}}^{[L]} \mathbf{w}_{r', s}^{\dagger}, \mathbf{P}^{\perp}\mathbf{w}_{r', s}^{\dagger}]} } 
						\end{align*}
						From Claim~\ref{Claimm:change_in_W}, we have with probability at least $1-2e^{-rho^2}$,
						\begin{align*}
							\norm[2]{\overline{\mathbf{W}}^{[L]} -  \overline{\widetilde{\mathbf{W}}}^{[L]}} \le \sqrt{3L} \cdot (2 (\sqrt{2} + \rho m^{-0.5}))^{L} \cdot \sqrt{\frac{2}{m}} \cdot   (\sqrt{N} + \sqrt{d} + \sqrt{2}\rho).
						\end{align*}
						Also, using Claim~\ref{Claimm:highprob_normW}, we have with probability at least $1-e^{-\Omega(\rho^2)}$,
						\begin{align*}
							\norm[2]{[\overline{\mathbf{W}}^{[L]} \mathbf{w}_{r', s}^{\dagger}, \mathbf{P}^{\perp}\mathbf{w}_{r', s}^{\dagger}]} &= \left(\norm[2]{\mathbf{P}^{\perp}\mathbf{w}_{r', s}^{\dagger}}^2 + \norm[2]{\overline{\mathbf{W}}^{[L]} \mathbf{w}_{r', s}^{\dagger}}^2 \right)^{1/2} \\&
							\ge \left(\norm[2]{\mathbf{P}^{\perp}\mathbf{w}_{r', s}^{\dagger}}^2 + \left(2^{L/2} \sqrt{L} (1 -  \rho L m^{-1/2}) \cdot \norm[2]{\mathbf{w}_{r', s}^{\dagger}}\right)^2 \right)^{1/2} \\&
							\ge \left(\norm[2]{\mathbf{P}^{\perp}\mathbf{w}_{r', s}^{\dagger}}^2 + \left(\norm[2]{\mathbf{w}_{r', s}^{\dagger}}\right)^2 \right)^{1/2} = \norm[2]{\mathbf{w}_{r', s}^{\dagger}} = 1,
						\end{align*}
						where in the final step, we have used the fact $\mathbf{P}$ is a projection matrix.
						
						Hence, 
						\begin{align*}
							\norm{\frac{[\overline{\mathbf{W}}^{[L]} \mathbf{w}_{r', s}^{\dagger}, \mathbf{P}^{\perp}\mathbf{w}_{r', s}^{\dagger}]}{\norm{[\overline{\mathbf{W}}^{[L]} \mathbf{w}_{r', s}^{\dagger}, \mathbf{P}^{\perp}\mathbf{w}_{r', s}^{\dagger}]}} - \frac{[\overline{\widetilde{\mathbf{W}}}^{[L]} \mathbf{w}_{r', s}^{\dagger}, \mathbf{P}^{\perp}\mathbf{w}_{r', s}^{\dagger}]}{\norm[2]{[\overline{\mathbf{W}}^{[L]} \mathbf{w}_{r', s}^{\dagger}, \mathbf{P}^{\perp}\mathbf{w}_{r', s}^{\dagger}]}}} &\le  \frac{\norm[2]{\overline{\mathbf{W}}^{[L]} -  \overline{\widetilde{\mathbf{W}}}^{[L]}} \norm{\mathbf{w}_{r', s}^{\dagger} }}
							{\norm{[\overline{\mathbf{W}}^{[L]} \mathbf{w}_{r', s}^{\dagger}, \mathbf{P}^{\perp}\mathbf{w}_{r', s}^{\dagger}]} } \\
							& \le \frac{\norm[2]{\overline{\mathbf{W}}^{[L]} -  \overline{\widetilde{\mathbf{W}}}^{[L]}} \norm{\mathbf{w}_{r', s}^{\dagger} }}
							{\norm{[\overline{\mathbf{W}}^{[L]} \mathbf{w}_{r', s}^{\dagger}, \mathbf{P}^{\perp}\mathbf{w}_{r', s}^{\dagger}]} } \\&
							\le \sqrt{3L} \cdot (2 (\sqrt{2} + \rho m^{-0.5}))^{L} \cdot \sqrt{\frac{2}{m}} \cdot   (\sqrt{N} + \sqrt{d} + \sqrt{2}\rho).
						\end{align*}
					\end{proof}
					
					
					\begin{claim}\label{claim:stability_barW_2}
						With probability at least $1-e^{-\Omega(\rho^2)}$,
						\begin{equation*}
							\norm{\frac{[\overline{\widetilde{\mathbf{W}}}^{[L]} \mathbf{w}_{r', s}^{\dagger}, \mathbf{P}^{\perp}\mathbf{w}_{r', s}^{\dagger}]}{\norm{[\overline{\mathbf{W}}^{[L]} \mathbf{w}_{r', s}^{\dagger}, \mathbf{P}^{\perp}\mathbf{w}_{r', s}^{\dagger}]}} - \frac{[\overline{\widetilde{\mathbf{W}}}^{[L]} \mathbf{w}_{r', s}^{\dagger}, \mathbf{P}^{\perp}\mathbf{w}_{r', s}^{\dagger}]}{\norm[2]{[\overline{\widetilde{\mathbf{W}}}^{[L]} \mathbf{w}_{r', s}^{\dagger}, \mathbf{P}^{\perp}\mathbf{w}_{r', s}^{\dagger}]}}} \le \sqrt{3L} \cdot (2 (\sqrt{2} + \rho m^{-0.5}))^{L} \cdot \sqrt{\frac{2}{m}} \cdot   (\sqrt{N} + \sqrt{d} + \sqrt{2}\rho).
						\end{equation*}
					\end{claim}
					
					
					\begin{proof}
						Using the definition of $\overline{\widetilde{\mathbf{W}}}^{[L]}$ and $\widetilde{\mathbf{W}}^{[L]}$, we have
						\begin{align*}
							&\norm{\frac{[\overline{\widetilde{\mathbf{W}}}^{[L]} \mathbf{w}_{r', s}^{\dagger}, \mathbf{P}^{\perp}\mathbf{w}_{r', s}^{\dagger}]}{\norm{[\overline{\mathbf{W}}^{[L]} \mathbf{w}_{r', s}^{\dagger}, \mathbf{P}^{\perp}\mathbf{w}_{r', s}^{\dagger}]}} - \frac{[\overline{\widetilde{\mathbf{W}}}^{[L]} \mathbf{w}_{r', s}^{\dagger}, \mathbf{P}^{\perp}\mathbf{w}_{r', s}^{\dagger}]}{\norm[2]{[\overline{\widetilde{\mathbf{W}}}^{[L]} \mathbf{w}_{r', s}^{\dagger}, \mathbf{P}^{\perp}\mathbf{w}_{r', s}^{\dagger}]}}} \\&
							= \norm[2]{[\overline{\widetilde{\mathbf{W}}}^{[L]} \mathbf{w}_{r', s}^{\dagger}, \mathbf{P}^{\perp}\mathbf{w}_{r', s}^{\dagger}]} \frac{ \abs{\norm{[\overline{\mathbf{W}}^{[L]} \mathbf{w}_{r', s}^{\dagger}, \mathbf{P}^{\perp}\mathbf{w}_{r', s}^{\dagger}]} -  \norm[2]{[\overline{\widetilde{\mathbf{W}}}^{[L]} \mathbf{w}_{r', s}^{\dagger}, \mathbf{P}^{\perp}\mathbf{w}_{r', s}^{\dagger}]}}}{\norm{[\overline{\mathbf{W}}^{[L]} \mathbf{w}_{r', s}^{\dagger}, \mathbf{P}^{\perp}\mathbf{w}_{r', s}^{\dagger}]} \norm[2]{[\overline{\widetilde{\mathbf{W}}}^{[L]} \mathbf{w}_{r', s}^{\dagger}, \mathbf{P}^{\perp}\mathbf{w}_{r', s}^{\dagger}]}} \\&
							= \frac{\abs{\norm{[\overline{\mathbf{W}}^{[L]} \mathbf{w}_{r', s}^{\dagger}, \mathbf{P}^{\perp}\mathbf{w}_{r', s}^{\dagger}]} -  \norm[2]{[\overline{\widetilde{\mathbf{W}}}^{[L]} \mathbf{w}_{r', s}^{\dagger}, \mathbf{P}^{\perp}\mathbf{w}_{r', s}^{\dagger}]}}}{\norm{[\overline{\mathbf{W}}^{[L]} \mathbf{w}_{r', s}^{\dagger}, \mathbf{P}^{\perp}\mathbf{w}_{r', s}^{\dagger}]} }
						\end{align*}
						Using the fact that for any two vectors $\mathbf{a}$ and $\mathbf{b}$, $\abs{\norm{\mathbf{a}} - \norm{\mathbf{b}}} \le \norm{\mathbf{a} - \mathbf{b}}$, we have    
						\begin{align}
							&
							\frac{\abs{\norm{[\overline{\mathbf{W}}^{[L]} \mathbf{w}_{r', s}^{\dagger}, \mathbf{P}^{\perp}\mathbf{w}_{r', s}^{\dagger}]} -  \norm[2]{[\overline{\widetilde{\mathbf{W}}}^{[L]} \mathbf{w}_{r', s}^{\dagger}, \mathbf{P}^{\perp}\mathbf{w}_{r', s}^{\dagger}]}}}{\norm{[\overline{\mathbf{W}}^{[L]} \mathbf{w}_{r', s}^{\dagger}, \mathbf{P}^{\perp}\mathbf{w}_{r', s}^{\dagger}]} } \nonumber
							\\&\le \frac{\norm{[\overline{\mathbf{W}}^{[L]} \mathbf{w}_{r', s}^{\dagger}, \mathbf{P}^{\perp}\mathbf{w}_{r', s}^{\dagger}] - [\overline{\widetilde{\mathbf{W}}}^{[L]} \mathbf{w}_{r', s}^{\dagger}, \mathbf{P}^{\perp}\mathbf{w}_{r', s}^{\dagger}]}}{\norm{[\overline{\mathbf{W}}^{[L]} \mathbf{w}_{r', s}^{\dagger}, \mathbf{P}^{\perp}\mathbf{w}_{r', s}^{\dagger}]} }\nonumber \\&
							= \frac{\norm[2]{\left(\overline{\mathbf{W}}^{[L]} -  \overline{\widetilde{\mathbf{W}}}^{[L]}\right)^{\top} \mathbf{w}_{r', s}^{\dagger} }}
							{\norm{[\overline{\mathbf{W}}^{[L]} \mathbf{w}_{r', s}^{\dagger}, \mathbf{P}^{\perp}\mathbf{w}_{r', s}^{\dagger}]} } 
							\le \frac{\norm[2]{\overline{\mathbf{W}}^{[L]} -  \overline{\widetilde{\mathbf{W}}}^{[L]}} \norm{\mathbf{w}_{r', s}^{\dagger} }}
							{\norm{[\overline{\mathbf{W}}^{[L]} \mathbf{w}_{r', s}^{\dagger}, \mathbf{P}^{\perp}\mathbf{w}_{r', s}^{\dagger}]} } \label{eq:continue_W_Wbar_2}  
						\end{align}
						
						From Claim~\ref{Claimm:change_in_W}, we have with probability at least $1-2e^{-\rho^2}$,
						\begin{align*}
							\norm[2]{\overline{\mathbf{W}}^{[L]} -  \overline{\widetilde{\mathbf{W}}}^{[L]}} \le \sqrt{3L} \cdot (2 (\sqrt{2} + \rho m^{-0.5}))^{L} \cdot \sqrt{\frac{2}{m}} \cdot   (\sqrt{N} + \sqrt{d} + \sqrt{2}\rho).
						\end{align*}
						Also, using Claim~\ref{Claimm:highprob_normW},  we have with probability at least $1-2e^{-\Omega(\rho^2)}$
						\begin{align*}
							\norm[2]{[\overline{\mathbf{W}}^{[L]} \mathbf{w}_{r', s}^{\dagger}, \mathbf{P}^{\perp}\mathbf{w}_{r', s}^{\dagger}]} &= \left(\norm[2]{\mathbf{P}^{\perp}\mathbf{w}_{r', s}^{\dagger}}^2 + \norm[2]{\overline{\mathbf{W}}^{[L]} \mathbf{w}_{r', s}^{\dagger}}^2 \right)^{1/2} \\&
							\ge \left(\norm[2]{\mathbf{P}^{\perp}\mathbf{w}_{r', s}^{\dagger}}^2 + \left(2^{L/2} \sqrt{L} (1 -  \rho L m^{-1/2}) \cdot \norm[2]{\mathbf{w}_{r', s}^{\dagger}}\right)^2 \right)^{1/2} \\&
							\ge \left(\norm[2]{\mathbf{P}^{\perp}\mathbf{w}_{r', s}^{\dagger}}^2 + \left(\norm[2]{\mathbf{w}_{r', s}^{\dagger}}\right)^2 \right)^{1/2} = \norm[2]{\mathbf{w}_{r', s}^{\dagger}} = 1.
						\end{align*}
						
						Hence, continuing from Eq.~\ref{eq:continue_W_Wbar_2}, we have
						\begin{align*}
							\norm{\frac{[\overline{\widetilde{\mathbf{W}}}^{[L]} \mathbf{w}_{r', s}^{\dagger}, \mathbf{P}^{\perp}\mathbf{w}_{r', s}^{\dagger}]}{\norm{[\overline{\mathbf{W}}^{[L]} \mathbf{w}_{r', s}^{\dagger}, \mathbf{P}^{\perp}\mathbf{w}_{r', s}^{\dagger}]}} - \frac{[\overline{\widetilde{\mathbf{W}}}^{[L]} \mathbf{w}_{r', s}^{\dagger}, \mathbf{P}^{\perp}\mathbf{w}_{r', s}^{\dagger}]}{\norm[2]{[\overline{\widetilde{\mathbf{W}}}^{[L]} \mathbf{w}_{r', s}^{\dagger}, \mathbf{P}^{\perp}\mathbf{w}_{r', s}^{\dagger}]}}} &\le  \frac{\norm[2]{\overline{\mathbf{W}}^{[L]} -  \overline{\widetilde{\mathbf{W}}}^{[L]}} \norm{\mathbf{w}_{r', s}^{\dagger} }}
							{\norm{[\overline{\mathbf{W}}^{[L]} \mathbf{w}_{r', s}^{\dagger}, \mathbf{P}^{\perp}\mathbf{w}_{r', s}^{\dagger}]} } \\
							& \le \frac{\norm[2]{\overline{\mathbf{W}}^{[L]} -  \overline{\widetilde{\mathbf{W}}}^{[L]}} \norm{\mathbf{w}_{r', s}^{\dagger} }}
							{\norm{[\overline{\mathbf{W}}^{[L]} \mathbf{w}_{r', s}^{\dagger}, \mathbf{P}^{\perp}\mathbf{w}_{r', s}^{\dagger}]} } \\&
							\le \sqrt{3L} \cdot (2 (\sqrt{2} + \rho m^{-0.5}))^{L} \cdot \sqrt{\frac{2}{m}} \cdot   (\sqrt{N} + \sqrt{d} + \sqrt{2}\rho).
						\end{align*}
						
					\end{proof}
					
					Thus, Eq.~\ref{eq:stability_barW_1} and Eq.~\ref{eq:stability_barW_2} can be re-written as 
					\begin{align*}
						&\mathfrak{C}_{\varepsilon}(\Phi_{r', s}, 1) 
						\abs{\left\langle [\mathbf{w}_r, \mathbf{a}_r], \mathbf{q}\right\rangle}  \\&
						\mathfrak{C}_{\varepsilon}(\Phi_{r', s}, 1) 
						\abs{\left\langle [\mathbf{w}_r, \mathbf{a}_r], \mathbf{v}\right\rangle}  
					\end{align*}
					where $\mathbf{q}$ and $\mathbf{v}$ are two vectors with norm less than $\sqrt{3L} \cdot (2 (\sqrt{2} + \rho m^{-0.5}))^{L} \cdot \sqrt{\frac{2}{m}} \cdot   (\sqrt{N} + \sqrt{d} + \sqrt{2}\rho)$ from Claims~\ref{claim:stability_barW_1}, ~\ref{claim:stability_barW_2}.
					Using Fact~\ref{fact:max_gauss}, we have w.p. atleast $1 - 2N e^{-\rho^2}$, 
					\begin{align*}
						\max_{r \in [N]} \mathfrak{C}_{\varepsilon}(\Phi_{r', s}, 1) 
						\abs{\left\langle [\mathbf{w}_r, \mathbf{a}_r], \mathbf{q}\right\rangle}, &\le \mathfrak{C}_{\varepsilon}(\Phi_{r', s}, 1)  \cdot \sqrt{2} \rho \cdot \sqrt{3L} \cdot (2 (\sqrt{2} + \rho m^{-0.5}))^{L} \cdot \sqrt{\frac{2}{m}} \cdot   (\sqrt{N} + \sqrt{d} + \sqrt{2}\rho) \\
						\max_{r \in [N]} \mathfrak{C}_{\varepsilon}(\Phi_{r', s}, 1) 
						\abs{\left\langle [\mathbf{w}_r, \mathbf{a}_r], \mathbf{v}\right\rangle} &\le \mathfrak{C}_{\varepsilon}(\Phi_{r', s}, 1)  \cdot \sqrt{2} \rho \cdot \sqrt{3L} \cdot (2 (\sqrt{2} + \rho m^{-0.5}))^{L} \cdot \sqrt{\frac{2}{m}} \cdot   (\sqrt{N} + \sqrt{d} + \sqrt{2}\rho).
					\end{align*}
					Continuing from Eq.~\ref{eq:stability_barW_1} and Eq.~\ref{eq:stability_barW_2}, we have
					\begin{align*}
						&\Big|H_{r', s}\left(\theta_{r', s} \left(\langle \mathbf{w}_{r}, \overline{\mathbf{W}}^{[L]} \mathbf{w}_{r', s}^{\dagger}\rangle + \langle \mathbf{a}_{r}^{[d-1]},  \mathbf{P}^{\perp}\mathbf{w}_{r', s}^{\dagger}  \rangle\right), \sqrt{m/2} a_{r, d}\right) \\&
						\quad\quad\quad\quad\quad\quad\quad\quad\quad- H_{r', s}\big(\widetilde{\theta}_{r', s} (\langle \mathbf{w}_{r}, \overline{\widetilde{\mathbf{W}}}^{[L]} \mathbf{w}_{r', s}^{\dagger}\rangle + \langle \mathbf{a}_{r}^{[d-1]},  \mathbf{P}^{\perp}\mathbf{w}_{r', s}^{\dagger}  \rangle), \sqrt{m/2} a_{r, d}\big)\Big| 
						\\& 
						\le 2\mathfrak{C}_{\varepsilon}(\Phi_{r', s}, 1)  \cdot \sqrt{2} \rho \cdot \sqrt{3L} \cdot (2 (\sqrt{2} + \rho m^{-0.5}))^{L} \cdot \sqrt{\frac{2}{m}} \cdot   (\sqrt{N} + \sqrt{d} + \sqrt{2}\rho).
					\end{align*}
					
					Hence, Eq.`\ref{eq:diff_f_tildef} is further simplified to
					\begin{align*}
						&\abs{F^{\mathcal{K}}_s(\widetilde{\mathbf{h}}^{(L-1)}, \mathbf{x}^{(L)}) - \widetilde{F}^{\mathcal{K}}_s(\widetilde{\mathbf{h}}^{(L-1)}, \mathbf{x}^{(L)})} \\&
						\le \frac{N}{m} \cdot 2\rho^2 \cdot 2\mathfrak{C}_{\varepsilon}(\Phi_{r', s}, 1)  \cdot \sqrt{2} \rho \cdot \sqrt{3L} \cdot (2 (\sqrt{2} + \rho m^{-0.5}))^{L} \cdot \sqrt{\frac{2}{m}} \cdot   (\sqrt{N} + \sqrt{d} + \sqrt{2}\rho)
					\end{align*}
					
				\end{proof}
				
				\begin{claim}\label{Claimm:change_in_W}
					With probability at least $1-8e^{-\rho^2}$,
					\begin{equation*}
						\norm[2]{\overline{\mathbf{W}}^{[L]} -  \overline{\widetilde{\mathbf{W}}}^{[L]}} \le \sqrt{3L} \cdot (2 (\sqrt{2} + \rho m^{-0.5}))^{L} \cdot \sqrt{\frac{2}{m}} \cdot   (\sqrt{N} + \sqrt{d} + \sqrt{2}\rho).
					\end{equation*}
				\end{claim}
				
				\begin{proof}
					Looking at $\overline{\mathbf{W}}^{[L]}$ and $\overline{\widetilde{\mathbf{W}}}^{[L]}$, we have
					\begin{align}
						\norm[2]{\overline{\mathbf{W}}^{[L]} -  \overline{\widetilde{\mathbf{W}}}^{[L]}} &= \left(\sum_{k=0}^{L-1} \norm[2]{ \mathbf{W}^k \mathbf{A} - \widetilde{\mathbf{W}}^k \widetilde{\mathbf{A}}}^2 \right)^{1/2} \nonumber\\
						&=  \left(\sum_{k=0}^{L-1} \norm[2]{ \sum_{k'=1}^{k-1}  \mathbf{W}^{k-k'} \widetilde{\mathbf{W}}^{k'} \mathbf{A} - \mathbf{W}^{k-k'-1} \widetilde{\mathbf{W}}^{k'+1} \mathbf{A} +  \widetilde{\mathbf{W}}^{k} \mathbf{A} - \widetilde{\mathbf{W}}^{k} \widetilde{\mathbf{A}}}^2 \right)^{1/2} \nonumber\\
						&\le \left(\sum_{k=0}^{L-1} \sum_{k'=1}^{k-1} \norm[2]{ \mathbf{W}^{k-k'} \widetilde{\mathbf{W}}^{k'} \mathbf{A} - \mathbf{W}^{k-k'-1} \widetilde{\mathbf{W}}^{k'+1} \mathbf{A}}^2 + \norm[2]{ \widetilde{\mathbf{W}}^{k} \mathbf{A} - \widetilde{\mathbf{W}}^{k} \widetilde{\mathbf{A}}}^2 \right)^{1/2} \nonumber\\&
						%= \left(\sum_{k=0}^{L-1} \sum_{k'=1}^{k+1} \norm[2]{ \widetilde{\mathbf{W}}^{k'-1} \mathbf{W}^{k-k'+1} \mathbf{A} - \widetilde{\mathbf{W}}^{\min(k', k)} \mathbf{W}^{\max(k-k', 0)} \widetilde{\mathbf{A}}}^2 \right)^{1/2} \\&
						= \left(\sum_{k=0}^{L-1} \sum_{k'=1}^{k-1} \norm[2]{ \mathbf{W}^{k-k'-1} \left(\mathbf{W} - \widetilde{\mathbf{W}}\right)  \widetilde{\mathbf{W}}^{k'} \mathbf{A}}^2 + \norm[2]{ \widetilde{\mathbf{W}}^{k} \left(\mathbf{A} - \widetilde{\mathbf{A}}\right) }^2 \right)^{1/2} \\&
						\le  \left(\sum_{k=0}^{L-1} \sum_{k'=1}^{k-1} \norm[2]{ \mathbf{W}^{k-k'-1}}^2 \norm[2]{\left(\mathbf{W} - \widetilde{\mathbf{W}}\right)  \widetilde{\mathbf{W}}^{k'} \mathbf{A}}^2 + \norm[2]{ \widetilde{\mathbf{W}}^{k}}^2 \norm[2]{\mathbf{A} - \widetilde{\mathbf{A}} }^2 \right)^{1/2} \label{eq:diff_W_tildeW}
					\end{align}
					Since, only $N$ rows of $\mathbf{A}$ and $\widetilde{\mathbf{A}}$ are different, $\mathbf{A} - \widetilde{\mathbf{A}}$ behaves as a random  matrix in $\mathbb{R}^{N \times d}$ with gaussian variables from $\mathcal{N}(0, \sqrt{2/m})$.
					Using Fact~\ref{thm:norm_W}, we have w.p. atleast $1 - 2e^{-\rho^2}$
					\begin{equation*}
						\norm[2]{\mathbf{A} - \widetilde{\mathbf{A}}} \le \sqrt{\frac{2}{m}} (\sqrt{d} + \sqrt{N} + \sqrt{2}\rho).
					\end{equation*}
					
					Similarly, since only $N$ rows of $\mathbf{W}$ and $\widetilde{\mathbf{W}}$ are different, $\mathbf{A} - \widetilde{\mathbf{A}}$ behaves as a random  matrix in $\mathbb{R}^{N \times m}$ with gaussian variables from $\mathcal{N}(0, \sqrt{2/m})$. Thus, $\left(\mathbf{W} - \widetilde{\mathbf{W}}\right)  \widetilde{\mathbf{W}}^{k'} \mathbf{A}$ behaves as a matrix with $N$ rows and $d$ columns and by fact~\ref{thm:norm_BA} we have w.p. at least $1 - 2e^{-\rho^2}$,
					\begin{align*}
						\norm[2]{\left(\mathbf{W} - \widetilde{\mathbf{W}}\right)  \widetilde{\mathbf{W}}^{k'} \mathbf{A}} &\le \norm[2]{\left(\mathbf{W} - \widetilde{\mathbf{W}}\right)  \widetilde{\mathbf{W}}^{k'}} \cdot \sqrt{\frac{2}{m}} \cdot (\sqrt{N} + \sqrt{d} + \sqrt{2}\rho) \\
						&\le \norm[2]{\left(\mathbf{W} - \widetilde{\mathbf{W}}\right)  \widetilde{\mathbf{W}}^{k'}} \cdot \sqrt{\frac{2}{m}} \cdot (\sqrt{N} + \sqrt{d} + \sqrt{2}\rho) \\
						&\le \norm[2]{\mathbf{W} - \widetilde{\mathbf{W}}} \norm[2]{\widetilde{\mathbf{W}}^{k'}}  \cdot \sqrt{\frac{2}{m}} \cdot (\sqrt{N} + \sqrt{d} + \sqrt{2}\rho). \\
						&\le \norm[2]{\mathbf{W}} \norm[2]{\widetilde{\mathbf{W}}^{k'}}  \cdot \sqrt{\frac{2}{m}} \cdot (\sqrt{N} + \sqrt{d} + \sqrt{2}\rho) + \norm[2]{\widetilde{\mathbf{W}}^{k' + 1}}  \cdot \sqrt{\frac{2}{m}} \cdot (\sqrt{N} + \sqrt{d} + \sqrt{2}\rho).
					\end{align*}
					
					Also, from Fact~\ref{thm:norm_W}, we can show that with probability exceeding $1 - 6e^{-\rho^2}$,
					\begin{align*}
						\norm{\mathbf{W}} &\le 2 (\sqrt{2} + \rho m^{-0.5})\\
						\norm[2]{\widetilde{\mathbf{W}}} &\le 2 (\sqrt{2} + \rho m^{-0.5})
					\end{align*}
					
					Thus, continuing from Eq.~\ref{eq:diff_W_tildeW}, we have
					\begin{align*}
						\norm[2]{\overline{\mathbf{W}}^{[L]} -  \overline{\widetilde{\mathbf{W}}}^{[L]}} &\le \left(\sum_{k=0}^{L-1} \sum_{k'=1}^{k-1} \norm[2]{ \mathbf{W}^{k-k'-1}}^2 \norm[2]{\left(\mathbf{W} - \widetilde{\mathbf{W}}\right)  \widetilde{\mathbf{W}}^{k'} \mathbf{A}}^2 + \norm[2]{ \widetilde{\mathbf{W}}^{k}}^2 \norm[2]{\mathbf{A} - \widetilde{\mathbf{A}} }^2 \right)^{1/2} \\&
						\le \Big(\sum_{k=0}^{L-1} \sum_{k'=1}^{k-1} \norm[2]{ \mathbf{W} }^{2(k-k'-1)} \norm[2]{\mathbf{W}}^2 \norm[2]{\widetilde{\mathbf{W}}}^{2k'}  \cdot \frac{2}{m} \cdot (\sqrt{N} + \sqrt{d} + \sqrt{2}\rho)^2 \\& \quad\quad\quad\quad 
						+ \norm[2]{ \mathbf{W} }^{2(k-k'-1)} \norm[2]{\widetilde{\mathbf{W}}}^{2(k'+1)}  \cdot \frac{2}{m} \cdot (\sqrt{N} + \sqrt{d} + \sqrt{2}\rho)^2  \\&\quad\quad\quad\quad + \norm[2]{ \widetilde{\mathbf{W}}}^{2k} \cdot \frac{2}{m} \cdot (\sqrt{N} + \sqrt{d} + \sqrt{2}\rho)^2  \Big)^{1/2} \\&
						\le \left(\sum_{k=0}^{L-1} \sum_{k'=1}^{k-1} 3(2 (\sqrt{2} + \rho m^{-0.5}))^{2k} \cdot \frac{2}{m} \cdot (\sqrt{N} + \sqrt{d} + \sqrt{2}\rho)^2   \right)^{1/2}\\&
						\le \sqrt{3L} \cdot (2 (\sqrt{2} + \rho m^{-0.5}))^{L} \cdot \sqrt{\frac{2}{m}} \cdot   (\sqrt{N} + \sqrt{d} + \sqrt{2}\rho).
					\end{align*}
				\end{proof}            
				
				\begin{claim}\label{Claimm:highprob_normW}
					With probability exceeding $1 - e^{-\Omega(\rho^2)}$,
					\begin{align*}
						2^{L/2} \sqrt{L} (1 -  \rho L m^{-1/2}) \cdot \norm[2]{\mathbf{w}_{r', s}^{\dagger}} \le \norm[2]{\overline{\widetilde{\mathbf{W}}}^{[L]} \mathbf{w}_{r', s}^{\dagger}} \le 2^{L/2} \sqrt{L} (1 + \rho L m^{-1/2}) \cdot \norm[2]{\mathbf{w}_{r', s}^{\dagger}}.
					\end{align*}
				\end{claim}
				
				\begin{proof}
					The proof will follow along the similar lines of the proof of Lemma~\ref{lemma:norm_ESN}.
				\end{proof}
				\fi 
				%%%%%%%%%%%%%%%%%%%%%%%%%%%%%%%%%%%%%%%%%%%%%%%%%%%%%%%%%%%%%%%%%%%%%%%%%%%%%
				%Now, we show that $\widetilde{F}$ is close to the desired signal.
				
				%%%%%%%%%%%%%%%%%%%%%%%%%%%%%%%%%%%%%%%%%%%%%%%%%%%%%%%%
				\iffalse
				\begingroup \allowdisplaybreaks
				\begin{align}
					\widetilde{F}_{s}^{(L), \mathcal{K}}(\widetilde{\mathbf{h}}^{(L-1)}, \mathbf{x}^{(L)}) &\stackrel{\text { def }}{=} \frac{\dout }{m} \sum_{i=1}^{L}  \sum_{s' \in [\dout ]} \sum_{r' \in [p]} \sum_{r \in \mathcal{K}}  b_{r, s'} b_{r', s'}^{\dagger} \widetilde{\mathbf{Back}}_{i \to L, r, s} \nonumber\\& \quad \quad \quad \quad H_{r', s'}\Big(\widetilde{\theta}_{r', s'} \langle \mathbf{w}_{r}, \overline{\widetilde{\mathbf{W}}}^{[L]} \mathbf{w}_{r', s'}^{\dagger}\rangle , \sqrt{m/2} a_{r, d}\Big) \mathbb{I}_{\mathbf{w}_r^{\top} \widetilde{\mathbf{h}}^{(i-1)} + \mathbf{a}_r^{\top} \mathbf{x}^{(i)} \ge 0} \nonumber
					\\&
					= \frac{\dout }{m} \sum_{i=1}^{L}  \sum_{s' \in [\dout ]: s' \ne s} \sum_{r' \in [p]} \sum_{r \in \mathcal{K}}  b_{r, s'} b_{r', s'}^{\dagger} \widetilde{\mathbf{Back}}_{i \to L, r, s} \nonumber\\& \quad \quad \quad \quad H_{r', s'}\Big(\widetilde{\theta}_{r', s'} \langle \mathbf{w}_{r}, \overline{\widetilde{\mathbf{W}}}^{[L]} \mathbf{w}_{r', s'}^{\dagger}\rangle , \sqrt{m/2} a_{r, d}\Big) \mathbb{I}_{\mathbf{w}_r^{\top} \widetilde{\mathbf{h}}^{(i-1)} + \mathbf{a}_r^{\top} \mathbf{x}^{(i)} \ge 0}\label{eq:sprimenes} \\&
					+ \frac{\dout }{m} \sum_{i=1}^{L-1}  \sum_{r' \in [p]} \sum_{r \in \mathcal{K}}  b_{r, s} b_{r', s}^{\dagger} \widetilde{\mathbf{Back}}_{i \to L, r, s} \nonumber\\& \quad \quad \quad \quad H_{r', s}\Big(\widetilde{\theta}_{r', s} \langle \mathbf{w}_{r}, \overline{\widetilde{\mathbf{W}}}^{[L]} \mathbf{w}_{r', s}^{\dagger}\rangle , \sqrt{m/2} a_{r, d}\Big) \mathbb{I}_{\mathbf{w}_r^{\top} \widetilde{\mathbf{h}}^{(i-1)} + \mathbf{a}_r^{\top} \mathbf{x}^{(i)} \ge 0}\label{eq:sprimeesL-1} \\&
					+ \frac{\dout }{m}   \sum_{r' \in [p]} \sum_{r \in \mathcal{K}}  b_{r, s} b_{r', s}^{\dagger} \widetilde{\mathbf{Back}}_{L \to L, r, s} \nonumber\\& \quad \quad \quad \quad H_{r', s}\Big(\widetilde{\theta}_{r', s} \langle \mathbf{w}_{r}, \overline{\widetilde{\mathbf{W}}}^{[L]} \mathbf{w}_{r', s}^{\dagger}\rangle , \sqrt{m/2} a_{r, d}\Big) \mathbb{I}_{\mathbf{w}_r^{\top} \widetilde{\mathbf{h}}^{(L-1)} + \mathbf{a}_r^{\top} \mathbf{x}^{(L)} \ge 0}\label{eq:sprimeesL}
				\end{align}
				\endgroup
				
				First, we will show that eq.~\ref{eq:sprimenes} is small in the next claim.
				\begin{claim}
					With probability at least $1-e^{-\Omega(\rho^2)}$,
					\begin{align*}
						&\Big|\frac{\dout }{m} \sum_{i=1}^{L}  \sum_{s' \in [\dout ]: s' \ne s} \sum_{r' \in [p]} \sum_{r \in \mathcal{K}}  b_{r, s'} b_{r', s'}^{\dagger} \widetilde{\mathbf{Back}}_{i \to L, r, s} \\& \quad \quad \quad \quad H_{r', s'}\Big(\widetilde{\theta}_{r', s'} \langle \mathbf{w}_{r}, \overline{\widetilde{\mathbf{W}}}^{[L]} \mathbf{w}_{r', s'}^{\dagger}\rangle , \sqrt{m/2} a_{r, d}\Big) \mathbb{I}_{\mathbf{w}_r^{\top} \widetilde{\mathbf{h}}^{(i-1)} + \mathbf{a}_r^{\top} \mathbf{x}^{(i)} \ge 0}\Big| \\&
						\le ZZZZ%\mathcal{O}(C_\varepsilon(\Phi, \mathcal{O}( \varepsilon_x^{-1})) p L \dout \rho m^{-0.25}) 
					\end{align*}
				\end{claim}
				
				\begin{proof}
					\begingroup \allowdisplaybreaks
					\begin{align*}
						&\Big|\frac{\dout }{m} \sum_{i=1}^{L}  \sum_{s' \in [\dout ]: s' \ne s} \sum_{r' \in [p]} \sum_{r \in \mathcal{K}}  b_{r, s'} b_{r', s'}^{\dagger} \widetilde{\mathbf{Back}}_{i \to L, r, s} \\& \quad \quad \quad \quad H_{r', s'}\Big(\widetilde{\theta}_{r', s'} \langle \mathbf{w}_{r}, \overline{\widetilde{\mathbf{W}}}^{[L]} \mathbf{w}_{r', s'}^{\dagger}\rangle , \sqrt{m/2} a_{r, d}\Big) \mathbb{I}_{\mathbf{w}_r^{\top} \widetilde{\mathbf{h}}^{(i-1)} + \mathbf{a}_r^{\top} \mathbf{x}^{(i)} \ge 0}\Big| \\&
						\le \sum_{i=1}^{L}  \sum_{s' \in [\dout ]: s' \ne s} \sum_{r' \in [p]} \Big|\frac{\dout }{m} \sum_{r \in \mathcal{K}}  b_{r, s'} b_{r', s'}^{\dagger} \widetilde{\mathbf{Back}}_{i \to L, r, s} \\& \quad \quad \quad \quad H_{r', s'}\Big(\widetilde{\theta}_{r', s'} \langle \mathbf{w}_{r}, \overline{\widetilde{\mathbf{W}}}^{[L]} \mathbf{w}_{r', s'}^{\dagger}\rangle , \sqrt{m/2} a_{r, d}\Big) \mathbb{I}_{\mathbf{w}_r^{\top} \widetilde{\mathbf{h}}^{(i-1)} + \mathbf{a}_r^{\top} \mathbf{x}^{(i)} \ge 0}\Big| \\&
					\end{align*}
					\endgroup
				\end{proof}
				
				
				\iffalse
				\begin{proof}
					We first divide the given term into 2 terms.
					\begingroup
					\allowdisplaybreaks
					\begin{align}
						&\Big|\frac{\dout }{m} \sum_{i=1}^{L}  \sum_{s' \in [\dout ]: s' \ne s} \sum_{r' \in [p]} \sum_{r \in [m]}  b_{r, s'} b_{r', s'}^{\dagger} \mathbf{Back}_{i \to L, r, s} \nonumber\\& \quad \quad \quad \quad H_{r', s'}\Big(\theta_{r', s'} \langle \mathbf{w}_{r}, \overline{\mathbf{W}}^{[L]} \mathbf{w}_{r', s'}^{\dagger}\rangle , \sqrt{m/2} a_{r, d}\Big) \mathbb{I}_{\mathbf{w}_r^{\top} \mathbf{h}^{(i-1)} + \mathbf{a}_r^{\top} \mathbf{x}^{(i)} \ge 0}\Big| \nonumber\\&
						\le \Big|\frac{\dout }{m} \sum_{i=1}^{L-1}  \sum_{s' \in [\dout ]: s' \ne s} \sum_{r' \in [p]} \sum_{r \in [m]}  b_{r, s'} b_{r', s'}^{\dagger} \mathbf{Back}_{i \to L, r, s} \nonumber\\& \quad \quad \quad \quad H_{r', s'}\Big(\theta_{r', s'} \langle \mathbf{w}_{r}, \overline{\mathbf{W}}^{[L]} \mathbf{w}_{r', s'}^{\dagger}\rangle , \sqrt{m/2} a_{r, d}\Big) \mathbb{I}_{\mathbf{w}_r^{\top} \mathbf{h}^{(i-1)} + \mathbf{a}_r^{\top} \mathbf{x}^{(i)} \ge 0}\Big|\label{eq:UpuntilL-1}\\&
						+ \Big|\frac{\dout }{m}  \sum_{s' \in [\dout ]: s' \ne s} \sum_{r' \in [p]} \sum_{r \in [m]}  b_{r, s'} b_{r', s'}^{\dagger} \mathbf{Back}_{L \to L, r, s} \nonumber\\& \quad \quad \quad \quad H_{r', s'}\Big(\theta_{r', s'} \langle \mathbf{w}_{r}, \overline{\mathbf{W}}^{[L]} \mathbf{w}_{r', s'}^{\dagger}\rangle , \sqrt{m/2} a_{r, d}\Big) \mathbb{I}_{\mathbf{w}_r^{\top} \mathbf{h}^{(L-1)} + \mathbf{a}_r^{\top} \mathbf{x}^{(L)} \ge 0}\Big|\label{eq:equal2L}
					\end{align}
					\endgroup
					
					We use Lemma~\ref{lemma:backward_correlation} to show that Eq.~\ref{eq:UpuntilL-1} is small. By the definition of $H_{r',s'}$ from Def.~\ref{Def:Function_approx}, we have
					\begingroup \allowdisplaybreaks
					\begin{align*}
						&\Big|\frac{\dout }{m} \sum_{i=1}^{L-1}  \sum_{s' \in [\dout ]: s' \ne s} \sum_{r' \in [p]} \sum_{r \in [m]}  b_{r, s'} b_{r', s'}^{\dagger} \mathbf{Back}_{i \to L, r, s} \\& \quad \quad \quad \quad H_{r', s'}\Big(\theta_{r', s'} \langle \mathbf{w}_{r}, \overline{\mathbf{W}}^{[L]} \mathbf{w}_{r', s'}^{\dagger}\rangle , \sqrt{m/2} a_{r, d}\Big) \mathbb{I}_{\mathbf{w}_r^{\top} \mathbf{h}^{(i-1)} + \mathbf{a}_r^{\top} \mathbf{x}^{(i)} \ge 0}\Big| 
						\\& = \abs{\frac{\dout }{m} \sum_{i=1}^{L-1} \sum_{s' \in [\dout ]: s' \ne s} \sum_{r' \in [p]} H_{r', s'}\Big(\theta_{r', s'} \langle \mathbf{w}_{r}, \overline{\mathbf{W}}^{[L]} \mathbf{w}_{r', s'}^{\dagger}\rangle , \sqrt{m/2} a_{r, d}\Big) \mathbb{I}_{\mathbf{w}_r^{\top} \mathbf{h}^{(i-1)} + \mathbf{a}_r^{\top} \mathbf{x}^{(i)} \ge 0} \left\langle e_{s'}^{\top} \mathbf{Back}_{i \to i}, e_s^{\top} \mathbf{Back}_{i \to L}\right\rangle}
						%\\& \le \max_{r', s'} \abs{H_{r', s'}\Big(\theta_{r', s'} \langle \mathbf{w}_{r}, \overline{\mathbf{W}}^{[L]} \mathbf{w}_{r', s'}^{\dagger}\rangle , \sqrt{m/2} a_{r, d}\Big)} \\& \quad \quad \quad \quad 
						%\cdot \max_r \abs{\mathbb{I}_{\mathbf{w}_r^{\top} \mathbf{h}^{(i-1)} + \mathbf{a}_r^{\top} \mathbf{x}^{(i)} \ge 0}}
						%\\& \quad \quad \quad \quad \cdot \abs{\frac{\dout }{m} \sum_{i=1}^{L-1}  \sum_{s' \in [\dout ]: s' \ne s} \sum_{r' \in [p]} \sum_{r \in [m]}  b_{r, s'} b_{r', s'}^{\dagger} \mathbf{Back}_{i \to L, r, s}} \\&
						%\le \max_{r', s'} C_\varepsilon(\Phi_{r', s'}, \mathcal{O}( \varepsilon_x^{-1})) \cdot 1 \cdot \abs{\frac{\dout }{m} \sum_{i=1}^{L-1}  \sum_{s' \in [\dout ]: s' \ne s} \sum_{r' \in [p]} \sum_{r \in [m]}  b_{r, s'} b_{r', s'}^{\dagger} \mathbf{Back}_{i \to L, r, s}} \\&
						%\le C_\varepsilon(\Phi, \mathcal{O}( \varepsilon_x^{-1})) \cdot 1 \cdot \sum_{i=1}^{L-1}  \sum_{s' \in [\dout ]: s' \ne s} \abs{\frac{\dout }{m} \sum_{r' \in [p]} \sum_{r \in [m]}  b_{r, s'} b_{r', s'}^{\dagger} \mathbf{Back}_{i \to L, r, s}} 
						\\&
						\le C_\varepsilon(\Phi, \mathcal{O}( \varepsilon_x^{-1})) \cdot 1 \cdot \sum_{i=1}^{L-1}  \sum_{s' \in [\dout ]: s' \ne s} \max_{r'}  \abs{b_{r', s'}^{\dagger}}  \sum_{r' \in [p]}  \abs{\frac{\dout }{m}  \sum_{r \in [m]}  b_{r, s'}  \mathbf{Back}_{i \to L, r, s}} \\&
						= C_\varepsilon(\Phi, \mathcal{O}( \varepsilon_x^{-1})) \cdot 1 \cdot \sum_{i=1}^{L-1} \sum_{s' \in [\dout ]: s' \ne s} \max_{r'}   \abs{b_{r', s'}^{\dagger}} \cdot p \cdot  \frac{\dout }{m} \abs{\left\langle \mathbf{e}_{s'}^{\top} \mathbf{Back}_{i \to i},  \mathbf{e}_{s}^{\top} \mathbf{Back}_{i \to L} \right\rangle} \\&
						\le C_\varepsilon(\Phi, \mathcal{O}( \varepsilon_x^{-1})) \cdot 1 \cdot (L-1) p \dout  \mathcal{O}(m^{-0.25} \rho) = \mathcal{O}(C_\varepsilon(\Phi, \mathcal{O}( \varepsilon_x^{-1})) L p \dout \rho m^{-0.25}),
					\end{align*}
					\endgroup
					where we use the definition of $\mathbf{Back}_{i \to i}$ to show that it is equivalent to $\mathbf{B}$ for all $i \in [L]$ and we invoke Lemma~\ref{lemma:backward_correlation} in the pre-final step that incurs a probability of $1 - e^{-\rho^2}$.
					
					We show now that Eq.~\ref{eq:equal2L}  is small. We use the definition of $\mathbf{Back}_{L \to L}$ to show that it is equivalent to $\mathbf{B}$ and then use the fact that the different columns of $\mathbf{B}$ are uncorrelated, since they come from a i.i.d. gaussian distribution.
					\begingroup
					\allowdisplaybreaks
					\begin{align*}
						&\Big|\frac{\dout }{m}   \sum_{s' \in [\dout ]: s' \ne s} \sum_{r' \in [p]} \sum_{r \in [m]}  b_{r, s'} b_{r', s'}^{\dagger} \mathbf{Back}_{L \to L, r, s} \\& \quad \quad \quad \quad H_{r', s'}\Big(\theta_{r', s'} \langle \mathbf{w}_{r}, \overline{\mathbf{W}}^{[L]} \mathbf{w}_{r', s'}^{\dagger}\rangle , \sqrt{m/2} a_{r, d}\Big) \mathbb{I}_{\mathbf{w}_r^{\top} \mathbf{h}^{(L-1)} + \mathbf{a}_r^{\top} \mathbf{x}^{(L)} \ge 0}\Big|
						\\& \le \max_{r', s'} \abs{H_{r', s'}\Big(\theta_{r', s'} \langle \mathbf{w}_{r}, \overline{\mathbf{W}}^{[L]} \mathbf{w}_{r', s'}^{\dagger}\rangle , \sqrt{m/2} a_{r, d}\Big)} \\& \quad \quad \quad \quad 
						\cdot \max_r \abs{\mathbb{I}_{\mathbf{w}_r^{\top} \mathbf{h}^{(L-1)} + \mathbf{a}_r^{\top} \mathbf{x}^{(L)} \ge 0}}
						\\& \quad \quad \quad \quad \cdot \abs{\frac{\dout }{m}   \sum_{s' \in [\dout ]: s' \ne s} \sum_{r' \in [p]} \sum_{r \in [m]}  b_{r, s'} b_{r', s'}^{\dagger} \mathbf{Back}_{L \to L, r, s}}
						\\&\le \max_{r',s'} C_\varepsilon(\Phi_{r', s'}, \mathcal{O}( \varepsilon_x^{-1})) \cdot 1 \cdot \abs{\frac{\dout }{m}   \sum_{s' \in [\dout ]: s' \ne s} \sum_{r' \in [p]} \sum_{r \in [m]}  b_{r, s'} b_{r', s'}^{\dagger} \mathbf{Back}_{L \to L, r, s}} \\ &=
						C_\varepsilon(\Phi, \mathcal{O}( \varepsilon_x^{-1})) \cdot 1 \cdot \abs{\frac{\dout }{m}   \sum_{s' \in [\dout ]: s' \ne s} \sum_{r' \in [p]} \sum_{r \in [m]}  b_{r, s'} b_{r', s'}^{\dagger} b_{r, s}} \\&\le C_\varepsilon(\Phi, \mathcal{O}( \varepsilon_x^{-1})) \cdot 1 \cdot \max_{r'} \abs{b_{r', s'}^{\dagger}} \sum_{r' \in [p]} \sum_{s' \in [\dout ]: s' \ne s}  \abs{\frac{\dout }{m}    \sum_{r \in [m]}  b_{r, s'}  b_{r, s}} \\&
						\le C_\varepsilon(\Phi, \mathcal{O}( \varepsilon_x^{-1})) \cdot 1 \cdot \dout p m^{-1} \cdot 2\sqrt{m} (1 + 2\rho) \\&
						\le 4 C_\varepsilon(\Phi, \mathcal{O}( \varepsilon_x^{-1})) \cdot 1 \cdot \dout p \rho m^{-1/2}.
					\end{align*}
					In the pre-final step, we have used the fact that the dot product between 2 different columns of $\mathbf{B}$, which comes from $\mathcal{N}(0, \mathbb{I})$ is of the order $\sqrt{m}$, w.p. atleast $1-e^{-\Omega(\rho^2)}$.
					\endgroup
					\todo{Add more details here.}
					Hence, 
					\begingroup \allowdisplaybreaks
					\begin{align*}
						&\Big|\frac{\dout }{m} \sum_{i=1}^{L}  \sum_{s' \in [\dout ]: s' \ne s} \sum_{r' \in [p]} \sum_{r \in [m]}  b_{r, s'} b_{r', s'}^{\dagger} \mathbf{Back}_{i \to L, r, s} \\& \quad \quad \quad \quad H_{r', s'}\Big(\theta_{r', s'} \langle \mathbf{w}_{r}, \overline{\mathbf{W}}^{[L]} \mathbf{w}_{r', s'}^{\dagger}\rangle , \sqrt{m/2} a_{r, d}\Big) \mathbb{I}_{\mathbf{w}_r^{\top} \mathbf{h}^{(i-1)} + \mathbf{a}_r^{\top} \mathbf{x}^{(i)} \ge 0}\Big| 
						\\& = \Big|\frac{\dout }{m} \sum_{i=1}^{L-1}  \sum_{s' \in [\dout ]: s' \ne s} \sum_{r' \in [p]} \sum_{r \in [m]}  b_{r, s'} b_{r', s'}^{\dagger} \mathbf{Back}_{i \to L, r, s} \\& \quad \quad \quad \quad H_{r', s'}\Big(\theta_{r', s'} \langle \mathbf{w}_{r}, \overline{\mathbf{W}}^{[L]} \mathbf{w}_{r', s'}^{\dagger}\rangle , \sqrt{m/2} a_{r, d}\Big) \mathbb{I}_{\mathbf{w}_r^{\top} \mathbf{h}^{(i-1)} + \mathbf{a}_r^{\top} \mathbf{x}^{(i)} \ge 0}\Big| 
						\\ &\quad \quad \quad + \Big|\frac{\dout }{m}   \sum_{s' \in [\dout ]: s' \ne s} \sum_{r' \in [p]} \sum_{r \in [m]}  b_{r, s'} b_{r', s'}^{\dagger} \mathbf{Back}_{L \to L, r, s} \\& \quad \quad \quad \quad H_{r', s'}\Big(\theta_{r', s'} \langle \mathbf{w}_{r}, \overline{\mathbf{W}}^{[L]} \mathbf{w}_{r', s'}^{\dagger}\rangle , \sqrt{m/2} a_{r, d}\Big) \mathbb{I}_{\mathbf{w}_r^{\top} \mathbf{h}^{(L-1)} + \mathbf{a}_r^{\top} \mathbf{x}^{(L)} \ge 0}\Big|
						\\&\le  \mathcal{O}(C_\varepsilon(\Phi, \mathcal{O}( \varepsilon_x^{-1})) L p \dout \rho m^{-0.25}) + 4 C_\varepsilon(\Phi, \mathcal{O}( \varepsilon_x^{-1}))  \dout p \rho m^{-1/2} \\&= \mathcal{O}(C_\varepsilon(\Phi, \mathcal{O}( \varepsilon_x^{-1})) L p \dout \rho m^{-0.25}).
					\end{align*}
					\endgroup
				\end{proof}
				\fi
				
				In the following claim, we show that Eq.~\ref{eq:sprimeesL-1} is small with high probability. The idea is again to invoke Lemma~\ref{lemma:backward_correlation} for bounding correlation between the $s$-th columns of $\mathbf{Back}_{i \to L}$ and $\mathbf{B}$.
				
				
				\begin{claim}
					With probability at least $1-e^{-\Omega(\rho^2)}$,
					\begin{align*}
						&\Big|\frac{\dout }{m} \sum_{i=1}^{L-1}  \sum_{r' \in [p]} \sum_{r \in \mathcal{K}}  b_{r, s} b_{r', s}^{\dagger} \widetilde{\mathbf{Back}}_{i \to L, r, s} \\& \quad \quad \quad \quad H_{r', s}\Big(\widetilde{\theta}_{r', s} \langle \mathbf{w}_{r}, \overline{\widetilde{\mathbf{W}}}^{[L]} \mathbf{w}_{r', s}^{\dagger}\rangle, \sqrt{m/2} a_{r, d}\Big) \mathbb{I}_{\mathbf{w}_r^{\top} \widetilde{\mathbf{h}}^{(i-1)} + \mathbf{a}_r^{\top} \mathbf{x}^{(i)} \ge 0}\Big| \\&
						\le YYYY%\mathcal{O}(C_\varepsilon(\Phi, \mathcal{O}( \varepsilon_x^{-1})) L \rho m^{-0.25}).
					\end{align*}
				\end{claim}
				
				\iffalse
				\begin{proof}
					By the definition of $H_{r', s}$ from def.~\ref{Def:Function_approx},
					\begingroup
					\allowdisplaybreaks
					\begin{align*}
						&\Big|\frac{\dout }{m} \sum_{i=1}^{L-1}  \sum_{r' \in [p]} \sum_{r \in [m]}  b_{r, s} b_{r', s}^{\dagger} \mathbf{Back}_{i \to L, r, s} \\& \quad \quad \quad \quad H_{r', s}\Big(\theta_{r', s} \langle \mathbf{w}_{r}, \overline{\mathbf{W}}^{[L]} \mathbf{w}_{r', s}^{\dagger}\rangle, \sqrt{m/2} a_{r, d}\Big) \mathbb{I}_{\mathbf{w}_r^{\top} \mathbf{h}^{(i-1)} + \mathbf{a}_r^{\top} \mathbf{x}^{(i)} \ge 0}\Big| \\&
						\\& \le \max_{r', s} \abs{H_{r', s}\Big(\theta_{r', s} \langle \mathbf{w}_{r}, \overline{\mathbf{W}}^{[L]} \mathbf{w}_{r', s}^{\dagger}\rangle, \sqrt{m/2} a_{r, d}\Big)} \\& \quad \quad \quad \quad 
						\cdot \max_r \abs{\mathbb{I}_{\mathbf{w}_r^{\top} \mathbf{h}^{(i-1)} + \mathbf{a}_r^{\top} \mathbf{x}^{(i)} \ge 0}}
						\\& \quad \quad \quad \quad \cdot \abs{\frac{\dout }{m} \sum_{i=1}^{L-1}   \sum_{r' \in [p]} \sum_{r \in [m]}  b_{r, s} b_{r', s}^{\dagger} \mathbf{Back}_{i \to L, r, s}} \\&
						\le \max_{r'} C_\varepsilon(\Phi_{r', s}, \mathcal{O}(\varepsilon_x^{-1})) \cdot 1 \cdot \abs{\frac{\dout }{m} \sum_{i=1}^{L-1}  \sum_{r' \in [p]} \sum_{r \in [m]}  b_{r, s} b_{r', s}^{\dagger} \mathbf{Back}_{i \to L, r, s}} \\&
						\le C_\varepsilon(\Phi, \mathcal{O}(\varepsilon_x^{-1})) \cdot 1 \cdot \sum_{i=1}^{L-1}  \abs{\frac{\dout }{m} \sum_{r' \in [p]} \sum_{r \in [m]}  b_{r, s} b_{r', s}^{\dagger} \mathbf{Back}_{i \to L, r, s}} \\&
						\le C_\varepsilon(\Phi, \mathcal{O}(\varepsilon_x^{-1})) \cdot 1 \cdot \sum_{i=1}^{L-1}   \max_{r'}  \abs{b_{r', s}^{\dagger}}  \sum_{r' \in [p]}  \abs{\frac{\dout }{m}  \sum_{r \in [m]}  b_{r, s}  \mathbf{Back}_{i \to L, r, s}} \\&
						= C_\varepsilon(\Phi, \mathcal{O}(\varepsilon_x^{-1})) \cdot 1 \cdot \sum_{i=1}^{L-1}\max_{r'}   \abs{b_{r', s}^{\dagger}} \cdot p \cdot  \frac{\dout }{m} \abs{\left\langle \mathbf{e}_{s}^{\top} \mathbf{Back}_{i \to i},  \mathbf{e}_{s}^{\top} \mathbf{Back}_{i \to L} \right\rangle} \\&
						\le C_\varepsilon(\Phi, \mathcal{O}(\varepsilon_x^{-1})) \cdot 1 \cdot (L-1) p  \mathcal{O}(m^{-0.25} \rho) = \mathcal{O}(C_\varepsilon(\Phi, \mathcal{O}(\varepsilon_x^{-1})) L p \rho m^{-0.25}),
					\end{align*}
					\endgroup
					where we use the definition of $\mathbf{Back}_{i \to i}$ to show that it is equivalent to $\mathbf{B}$ for all $i \in [L]$ and we invoke Lemma~\ref{lemma:backward_correlation} in the pre-final step that incurs a probability of $1 - e^{-\Omega(\rho)^2}$.
				\end{proof}
				\fi    
				
				
				\fi    
				%%%%%%%%%%%%%%%%%%%%%%%%%%%%%%%%%%%%%%%%%%%%%%%%%%%%%%%%%%%%%%
				
				%In the following claim, we show that $\widetilde{F}$ is close to the desired signal.
				\begin{claim}[Restating claim~\ref{claim:difftildefphi}]\label{claim:difftildefphi_proof}
					With probability at least $1-e^{-\Omega(\rho^2)}$,
					\begin{align*}
						\Big|&\widetilde{F}_{s}^{(L), \mathcal{K}}(\widetilde{\mathbf{h}}^{(L-1)}, \mathbf{x}^{(L)}) - \frac{\dout }{m} \sum_{i=1}^{L}  \sum_{s' \in [\dout ]} \sum_{r' \in [p]} \sum_{r \in \mathcal{K}}  b_{r, s'} b_{r', s'}^{\dagger} \widetilde{\mathbf{Back}}_{i \to L, r, s} \Phi_{r', s} \left(\left\langle \mathbf{w}_{r', s}^{\dagger}, [\overline{\mathbf{x}}^{(1)}, \cdots, \overline{\mathbf{x}}^{(L)}]\right\rangle\right)\Big| \\&\le \frac{\dout }{m} \cdot \mathcal{O}(\mathfrak{C}_{\varepsilon}(\Phi_{r' s}, \mathcal{O}(\varepsilon_x^{-1})) \rho^2 \sqrt{\dout LpN}) \\& + \frac{\dout LpN}{m} \rho^2 (\varepsilon + \mathcal{O}( L_{\Phi}\rho^5 (N/m)^{1/6}) + \mathcal{O}(L_{\Phi}\varepsilon_x^{-1}  L^4 \rho^{11} m^{-1/12} +  L_{\Phi} \rho^2 L^{11/6} \varepsilon_x^{2/3})),
					\end{align*}
					for any $\varepsilon \in (0, \min_{r, s} \frac{\sqrt{3}}{C_s(\Phi_{r, s}, \varepsilon_x^{-1})})$.
					%where $\varepsilon' = L_{\Phi} \cdot \mathcal{O}\left(\rho^5 m^{-1/8} \sqrt{L} \right) \cdot \left(2(\sqrt{2} + \rho m^{-0.5})\right)^{L+1} \sqrt{2} (1 + \sqrt{dm^{-1}} + \sqrt{2}\rho m^{-0.5})$.
				\end{claim}
				
				\begin{proof}
					We will take the expectation w.r.t. the weights $\left\{ \mathbf{w}_r, \mathbf{a}_r \right\}_{r \in \mathcal{K}}$. The difference between $\widetilde{F}$ and the expected value is given by
					%Using the definition of $\widetilde{\mathbf{Back_{L \to L}}}$, we write the difference between the desired term and its expected value as
					\begingroup \allowdisplaybreaks
					\begin{align}
						&\abs{\widetilde{F}_{s}^{(L), \mathcal{K}}(\widetilde{\mathbf{h}}^{(L-1)}, \mathbf{x}^{(L)}) - \mathbb{E}_{\left\{ \mathbf{w}_r, \mathbf{a}_r \right\}_{r \in \mathcal{K}}} \widetilde{F}_{s}^{(L), \mathcal{K}}(\widetilde{\mathbf{h}}^{(L-1)}, \mathbf{x}^{(L)})} \nonumber\\
						&=\Big|\frac{\dout }{m} \sum_{i=1}^{L}  \sum_{s' \in [\dout ]} \sum_{r' \in [p]} \sum_{r \in \mathcal{K}}  b_{r, s'} b_{r', s'}^{\dagger} \widetilde{\mathbf{Back}}_{i \to L, r, s} H_{r', s}\Big(\widetilde{\theta}_{r', s} \langle \mathbf{w}_{r}, \overline{\widetilde{\mathbf{W}}}^{[L]} \mathbf{w}_{r', s}^{\dagger}\rangle, \sqrt{m/2} a_{r, d}\Big) \mathbb{I}_{\mathbf{w}_r^{\top} \widetilde{\mathbf{h}}^{(L-1)} + \mathbf{a}_r^{\top} \mathbf{x}^{(L)} \ge 0} \nonumber\\&
						\quad\quad\quad\quad -  \mathbb{E}_{[\mathbf{w}, \mathbf{a}] \sim \mathcal{N}(0, \frac{2}{m}\mathbf{I})} \frac{\dout }{m} \sum_{i=1}^{L}  \sum_{s' \in [\dout ]} \sum_{r' \in [p]} \sum_{r \in \mathcal{K}}  b_{r, s'} b_{r', s'}^{\dagger} \widetilde{\mathbf{Back}}_{i \to L, r, s} \nonumber\\& \quad\quad\quad\quad \quad\quad\quad\quad \cdot H_{r', s}\Big(\widetilde{\theta}_{r', s} \langle \mathbf{w}, \overline{\widetilde{\mathbf{W}}}^{[L]} \mathbf{w}_{r', s}^{\dagger}\rangle, \sqrt{m/2} a_{d}\Big) \mathbb{I}_{\mathbf{w}^{\top} \widetilde{\mathbf{h}}^{(L-1)} + \mathbf{a}^{\top} \mathbf{x}^{(L)} \ge 0}\Big| \nonumber\\&
						=\Big|\frac{\dout }{m} \sum_{i=1}^{L}  \sum_{s' \in [\dout ]} \sum_{r' \in [p]} \sum_{r \in \mathcal{K}}  b_{r, s'} b_{r', s'}^{\dagger} \widetilde{\mathbf{Back}}_{i \to L, r, s} H_{r', s}\Big(\widetilde{\theta}_{r', s} \langle \mathbf{w}_{r}, \overline{\widetilde{\mathbf{W}}}^{[L]} \mathbf{w}_{r', s}^{\dagger}\rangle, \sqrt{m/2} a_{r, d}\Big) \mathbb{I}_{\mathbf{w}_r^{\top} \widetilde{\mathbf{h}}^{(L-1)} + \mathbf{a}_r^{\top} \mathbf{x}^{(L)} \ge 0} \nonumber\\&
						\quad\quad\quad\quad -  \frac{\dout }{m} \sum_{i=1}^{L}  \sum_{s' \in [\dout ]} \sum_{r' \in [p]} \sum_{r \in \mathcal{K}}  b_{r, s'} b_{r', s'}^{\dagger} \widetilde{\mathbf{Back}}_{i \to L, r, s} \nonumber\\& \quad\quad\quad\quad \quad\quad\quad\quad \cdot  \mathbb{E}_{\mathbf{w}, \mathbf{a} \sim \mathcal{N}(0, \frac{2}{m}\mathbf{I})} H_{r', s}\Big(\widetilde{\theta}_{r', s} \langle \mathbf{w}, \overline{\widetilde{\mathbf{W}}}^{[L]} \mathbf{w}_{r', s}^{\dagger}\rangle, \sqrt{m/2} a_{d}\Big) \mathbb{I}_{\mathbf{w}^{\top} \widetilde{\mathbf{h}}^{(L-1)} + \mathbf{a}^{\top} \mathbf{x}^{(L)} \ge 0}\Big|, \label{Eqn:conc_H}
					\end{align}
					\endgroup
					where in the final step, we have used the fact that $\widetilde{\mathbf{Back}}$ and $\mathbf{B}$ are independent of  the variables $\left\{ \mathbf{w}_r, \mathbf{a}_r \right\}_{r \in \mathcal{K}}$ w.r.t. which we are taking the expectation.
					
					Note that, the random variable under consideration is a bounded random variable, because: using the fact that $b_{r, s'} \sim \mathcal{N}(0, 1)$, it is bounded by $\rho$ with high probability, $\widetilde{Back}_{i \to L, r, s} = \mathbf{b}_s^{\top} (\prod_{i \le \ell \le L} \mathbf{D}^{(\ell)} \mathbf{W}) \mathbf{e}_r$ is bounded by $\mathcal{O}(\rho)$ using bound on norm of $\mathbf{b}_s$ and Claim~\ref{lemma:norm_ESN}, and the function $H$ is bounded by def.~\ref{Def:Function_approx}.
					\iffalse
					\begin{align*}
						\abs{\frac{\dout }{m} \sum_{i=1}^{L}  \sum_{s' \in [\dout ]} \sum_{r' \in [p]} b_{r, s'} b_{r', s'}^{\dagger} \widetilde{\mathbf{Back}}_{i \to L, r, s}  H_{r', s}\Big(\widetilde{\theta}_{r', s} \langle \mathbf{w}_{r}, \overline{\widetilde{\mathbf{W}}}^{[L]} \mathbf{w}_{r', s}^{\dagger}\rangle , \sqrt{m/2} a_{r, d}\Big) \mathbb{I}_{\mathbf{w}_r^{\top} \widetilde{\mathbf{h}}^{(L-1)} + \mathbf{a}_r^{\top} \mathbf{x}^{(L)} \ge 0}} \le \mathcal{O}(\rho^2 \mathfrak{C}_{\varepsilon}(\Phi, \mathcal{O}(\varepsilon_x^{-1})))
					\end{align*}
					\fi
					Denoting the inequality in eq.~\ref{Eqn:conc_H} as 
					$P(\left\{ \mathbf{w}_r, \mathbf{a}_r \right\}_{r \in \mathcal{K}})$, we get using hoeffding's inequality for bounded variables (fact~\ref{fact:hoeffding})
					\begin{equation*}
						\Pr \left[ P(\left\{ \mathbf{w}_r, \mathbf{a}_r \right\}_{r \in \mathcal{K}}) > \frac{\dout }{m} \cdot  \mathcal{O}(\mathfrak{C}_{\varepsilon}(\Phi_{r' s}, \mathcal{O}(\varepsilon_x^{-1})) \rho^2 \sqrt{\dout LpN}) \right] \le e^{-\rho^2/8}.
					\end{equation*}
					%\todo{Can you add more details here pls.}
					
					Now, we focus on the expected value in eq.~\ref{Eqn:conc_H}.
					For typographical simplicity in the next few steps, we denote the vector $\mathbf{v} =  \widetilde{\mathbf{h}}^{(L-1)}$, vector $\mathbf{q} = \sqrt{m/2} \cdot \mathbf{w}$ and vector $\mathbf{t}=\widetilde{\theta}_{r', s} \cdot \sqrt{2/m} \cdot \overline{\widetilde{\mathbf{W}}}^{[L]} \mathbf{w}_{r', s}^{\dagger}$. Also, let $\mathbf{t}^{\perp}$ denote a vector in the subspace orthogonal to $\mathbf{t}$ that is closest to the vector $\mathbf{v}$.
					
					By the definition of the function $H_{r' ,s}$ from 
					Def.~\ref{Def:Function_approx}, where we use $k_{0, r, s} = \varepsilon_x^{-1} \sqrt{m/2} \theta_{r', s}^{-1}$  for each $r' \in [p], s \in [\dout ]$ in def.~\ref{Def:Function_approx} ($\theta_{r' ,s}$ is defined in def.~\ref{def:existence}),
					we have
					\begingroup
					\allowdisplaybreaks
					\begin{align*}
						&\mathbb{E}_{\mathbf{w}, \mathbf{a} \sim \mathcal{N}(0, \frac{2}{m}\mathbf{I})}  H_{r', s}\Big(\widetilde{\theta}_{r', s} \langle \mathbf{w}, \overline{\widetilde{\mathbf{W}}}^{[L]} \mathbf{w}_{r', s}^{\dagger}\rangle , \sqrt{m/2} a_{d}\Big)  \cdot\mathbb{I} \left[ \langle \mathbf{w},  \widetilde{\mathbf{h}}^{(L-1)}\rangle + a_{d}  \ge 0 \right] \\&
						= \mathbb{E}_{\mathbf{w}, \mathbf{a} \sim \mathcal{N}(0, \frac{2}{m}\mathbf{I})}  H_{r', s}\Big(\widetilde{\theta}_{r', s} \langle \mathbf{w}, \overline{\widetilde{\mathbf{W}}}^{[L]} \mathbf{w}_{r', s}^{\dagger}\rangle , \sqrt{m/2} a_{d}\Big) \cdot\mathbb{I} \left[ \sqrt{m/2}\langle \mathbf{w},  \widetilde{\mathbf{h}}^{(L-1)}\rangle + \sqrt{m/2} a_{d}  \ge 0 \right] \\&
						= \mathbb{E}_{\mathbf{w}, \mathbf{a} \sim \mathcal{N}(0, \frac{2}{m}\mathbf{I})}  H_{r', s}\Big(  \langle \mathbf{q}, \mathbf{t}  \rangle, \sqrt{m/2} a_{d}\Big) \cdot \mathbb{I} \left[\langle \mathbf{v}, \mathbf{q} \rangle + \sqrt{m/2} a_{d}  \ge 0 \right] \\&
						= \mathbb{E}_{\mathbf{w}, \mathbf{a} \sim \mathcal{N}(0, \frac{2}{m}\mathbf{I})}  H_{r', s}\Big(  \langle \mathbf{q}, \mathbf{t}  \rangle, \sqrt{m/2} a_{d}\Big) \cdot  \mathbb{I} \left[\langle \mathbf{v}, \mathbf{t} \rangle \langle \mathbf{q}, \mathbf{t} \rangle +  \sqrt{\norm{\mathbf{v}}^2 - \langle \mathbf{v}, \mathbf{t} \rangle^2}  \langle \mathbf{q}, \mathbf{t}^{\perp} \rangle+ \sqrt{m/2} a_{d}  \ge 0 \right] \\&
						= \Phi_{r', s} \left(\varepsilon_x^{-1} (\sqrt{m/2}\theta_{r', s}^{-1}) \langle\mathbf{t}, \mathbf{v}\rangle\right) \pm \varepsilon 
						\\&
						= \Phi_{r', s} \left( \varepsilon_x^{-1} \theta_{r', s}^{-1} \widetilde{\theta}_{r', s} \langle \overline{\widetilde{\mathbf{W}}}^{[L]} \mathbf{w}_{r', s}^{\dagger}, \mathbf{\widetilde{h}}^{(L-1)} \rangle  \right) \pm \varepsilon \\&
						= \Phi_{r', s} \left( \varepsilon_x^{-1}  \langle \overline{\widetilde{\mathbf{W}}}^{[L]} \mathbf{w}_{r', s}^{\dagger}, \mathbf{\widetilde{h}}^{(L-1)}\rangle  \pm \mathcal{O}(\rho^5 (N/m)^{1/6})  \right) \pm \varepsilon \\&
						= \Phi_{r', s} \left( \varepsilon_x^{-1}  \langle \overline{\widetilde{\mathbf{W}}}^{[L]} \mathbf{w}_{r', s}^{\dagger}, \mathbf{\widetilde{h}}^{(L-1)}  \rangle  \right) \pm \varepsilon \pm \mathcal{O}(L_{\Phi_{r',s}} \rho^5 (N/m)^{1/6}),
						%\\&
						%= \Phi_{r', s} \left( \varepsilon_x^{-1} \langle     \mathbf{w}_{r', s}^{\dagger}, \overline{\widetilde{\mathbf{W}}}^{(L)} \mathbf{\widetilde{h}}^{(L-1)} \rangle \right) \pm \varepsilon \\&
						%= \Phi_{r', s} \left(\varepsilon_x^{-1} \langle     \mathbf{w}_{r', s}^{\dagger}, [\mathbf{x}^{(2)}, \cdots, \mathbf{x}^{(L-1)}] \rangle  \pm \varepsilon' \right) \pm \varepsilon  \\&
						%= \Phi_{r', s} \left( \langle     \mathbf{w}_{r', s}^{\dagger}, [\overline{\mathbf{x}}^{(2)}, \cdots, \overline{\mathbf{x}}^{(L-1)}] \rangle \right) \pm \varepsilon \pm L_{\Phi_{r', s}}\varepsilon',
					\end{align*}
					\endgroup
					where in the pre-final step, we have used claim~\ref{claim:invtildethetatheta} to bound the value of $\theta_{r', s}^{-1} \widetilde{\theta}_{r', s}$ and in the final step, we have used the lipschitz constant of $\Phi_{r', s}$ in the desired range. Corollary~\ref{cor:Invertibility_ESN} shows that with probability at least $1 - e^{-\Omega(\rho^2)}$ w.r.t. the weights $\widetilde{\mathbf{W}}$ and $\widetilde{\mathbf{A}}$, 
					%\todo{It's not exactly theorem 4.2. Wrte a corollary and refer to that.}
					\begin{align*}
						&\abs{\overline{\widetilde{\mathbf{W}}}^{[L]\top} \widetilde{h}^{(L-1)} - \varepsilon_x[\overline{\mathbf{x}}^{(2)}, 
							\cdots, \overline{\mathbf{x}}^{(L-1)}]}  \\&\le \mathcal{O}\left(L^4 \cdot (\rho^{11} m^{-1/12} + \rho^{7} m^{-1/12} + \rho^{7} m^{-1/4} + \rho^{11} m^{-1/4})  \right) + \mathcal{O}(\rho^2 L^{11/6} \varepsilon_x^{2/3}) \\&
						\le  \mathcal{O}(L^4 \rho^{11} m^{-1/12} + \rho^2 L^{11/6} \varepsilon_x^{5/3}). 
					\end{align*}
					%Hence,
					%\begin{equation*}
					%    \langle \mathbf{v}\rangle
					%\end{equation*}
					Thus,
					\begingroup \allowdisplaybreaks
					\begin{align*}
						&\mathbb{E}_{\mathbf{w}, \mathbf{a} \sim \mathcal{N}(0, \frac{2}{m}\mathbf{I})}  H_{r', s}\Big(\widetilde{\theta}_{r', s} \langle \mathbf{w}, \overline{\widetilde{\mathbf{W}}}^{[L]} \mathbf{w}_{r', s}^{\dagger}\rangle , \sqrt{m/2} a_{d}\Big)  \cdot\mathbb{I} \left[ \langle \mathbf{w},  \widetilde{\mathbf{h}}^{(L-1)}\rangle + a_{d}  \ge 0 \right] \\&
						%= \mathbb{E}_{\mathbf{w}, \mathbf{a} \sim \mathcal{N}(0, \frac{2}{m}\mathbf{I})}  H_{r', s}\Big(\widetilde{\theta}_{r', s} \langle \mathbf{w}, \overline{\widetilde{\mathbf{W}}}^{[L]} \mathbf{w}_{r', s}^{\dagger}\rangle , \sqrt{m/2} a_{d}\Big) \cdot\mathbb{I} \left[ \sqrt{m/2}\langle \mathbf{w},  \widetilde{\mathbf{h}}^{(L-1)}\rangle + \sqrt{m/2} a_{d}  \ge 0 \right] \\&
						%= \mathbb{E}_{\mathbf{w}, \mathbf{a} \sim \mathcal{N}(0, \frac{2}{m}\mathbf{I})}  H_{r', s}\Big(  \langle \mathbf{q}, \mathbf{t}  \rangle, \sqrt{m/2} a_{d}\Big) \cdot \mathbb{I} \left[\langle \mathbf{v}, \mathbf{q} \rangle + \sqrt{m/2} a_{d}  \ge 0 \right] \\&
						%= \mathbb{E}_{\mathbf{w}, \mathbf{a} \sim \mathcal{N}(0, \frac{2}{m}\mathbf{I})}  H_{r', s}\Big(  \langle \mathbf{q}, \mathbf{t}  \rangle, \sqrt{m/2} a_{d}\Big) \cdot  \mathbb{I} \left[\langle \mathbf{v}, \mathbf{t} \rangle \langle \mathbf{q}, \mathbf{t} \rangle +  \sqrt{\norm{\mathbf{v}}^2 - \langle \mathbf{v}, \mathbf{t} \rangle^2}  \langle \mathbf{q}, \mathbf{t}^{\perp} \rangle+ \sqrt{m/2} a_{d}  \ge 0 \right] \\&
						%= \Phi_{r', s} \left(\varepsilon_x^{-1} \langle\mathbf{t}, \mathbf{v}\rangle\right) \pm \varepsilon 
						%\\&
						= \Phi_{r', s} \left( \varepsilon_x^{-1} \langle \overline{\widetilde{\mathbf{W}}}^{[L]} \mathbf{w}_{r', s}^{\dagger}, \mathbf{\widetilde{h}}^{(L-1)} \rangle  \right) \pm \varepsilon \pm \mathcal{O}(L_{\Phi_{r', s}} \rho^5 (N/m)^{1/6})
						\\&
						= \Phi_{r', s} \left( \varepsilon_x^{-1} \langle     \mathbf{w}_{r', s}^{\dagger}, \overline{\widetilde{\mathbf{W}}}^{[L]\top} \mathbf{\widetilde{h}}^{(L-1)} \rangle \right) \pm \varepsilon \pm \mathcal{O}(L_{\Phi_{r', s}} \rho^5 (N/m)^{1/6}) \\&
						= \Phi_{r', s} \left(\varepsilon_x^{-1} \langle     \mathbf{w}_{r', s}^{\dagger}, \varepsilon_x[\overline{\mathbf{x}}^{(2)}, \cdots, \overline{\mathbf{x}}^{(L-1)}] \rangle  \pm \varepsilon' \right) \pm \varepsilon \pm \mathcal{O}(L_{\Phi_{r', s}} \rho^5 (N/m)^{1/6})  \\&
						= \Phi_{r', s} \left( \langle     \mathbf{w}_{r', s}^{\dagger}, [\overline{\mathbf{x}}^{(2)}, \cdots, \overline{\mathbf{x}}^{(L-1)}] \rangle \right) \pm \varepsilon \pm \mathcal{O}(L_{\Phi_{r', s}} \rho^5 (N/m)^{1/6}) \pm L_{\Phi_{r', s}}\varepsilon',
					\end{align*}
					\endgroup
					%\todo{There is a major issue in the steps. theta is not equal to one. Pls fix that !!}
					where $\varepsilon \in (0, \min_{r, s} \frac{\sqrt{3}}{C_s(\Phi_{r, s}, \varepsilon_x^{-1})})$ and  $\varepsilon' =  \mathcal{O}(\varepsilon_x^{-1} L^4 \rho^{11} m^{-1/12} + \rho^2 L^{11/6} \varepsilon_x^{2/3}).$ Thus, we have
					\begingroup \allowdisplaybreaks
					\begin{align}
						&\Big|\frac{\dout }{m} \sum_{i=1}^{L}  \sum_{s' \in [\dout ]} \sum_{r' \in [p]} \sum_{r \in \mathcal{K}}  b_{r, s'} b_{r', s'}^{\dagger} \widetilde{\mathbf{Back}}_{i \to L, r, s} \nonumber\\& \quad\quad\quad\quad \quad\quad\quad\quad \cdot  \mathbb{E}_{\mathbf{w}, \mathbf{a} \sim \mathcal{N}(0, \frac{2}{m}\mathbf{I})} H_{r', s}\Big(\widetilde{\theta}_{r', s} \langle \mathbf{w}, \overline{\widetilde{\mathbf{W}}}^{[L]} \mathbf{w}_{r', s}^{\dagger}\rangle, \sqrt{m/2} a_{d}\Big) \mathbb{I}_{\mathbf{w}^{\top} \widetilde{\mathbf{h}}^{(L-1)} + \mathbf{a}^{\top} \mathbf{x}^{(L)} \ge 0} \nonumber\\& - 
						\frac{\dout }{m} \sum_{i=1}^{L}  \sum_{s' \in [\dout ]} \sum_{r' \in [p]} \sum_{r \in \mathcal{K}}  b_{r, s'} b_{r', s'}^{\dagger} \widetilde{\mathbf{Back}}_{i \to L, r, s} \Phi_{r', s} \left( \langle     \mathbf{w}_{r', s}^{\dagger}, [\overline{\mathbf{x}}^{(2)}, \cdots, \overline{\mathbf{x}}^{(L-1)}] \rangle \right) \Big| \\& \le \frac{\dout }{m} \sum_{i=1}^{L}  \sum_{s' \in [\dout ]} \sum_{r' \in [p]} \sum_{r \in \mathcal{K}}  b_{r, s'} b_{r', s'}^{\dagger} \widetilde{\mathbf{Back}}_{i \to L, r, s} \cdot (\varepsilon + \mathcal{O}(L_{\Phi_{r', s}} \rho^5 (N/m)^{1/6}) + L_{\Phi_{r', s}}\varepsilon') \nonumber\\&
						\le \frac{\dout }{m} \sum_{i=1}^{L}  \sum_{s' \in [\dout ]} \sum_{r' \in [p]} \sum_{r \in \mathcal{K}}  \abs{b_{r, s'}} \abs{b_{r', s'}^{\dagger}} \abs{\widetilde{\mathbf{Back}}_{i \to L, r, s}} \cdot (\varepsilon + \max_{r', s} L_{\Phi_{r', s}}\varepsilon' + \mathcal{O}(L_{\Phi_{r', s}} \rho^5 (N/m)^{1/6})) \nonumber\\&
						\le \frac{\dout LpN}{m} \rho^2 (\varepsilon + \mathcal{O}(L_{\Phi} \rho^5 (N/m)^{1/6}) + L_{\Phi}\varepsilon') \label{eq:expectedtheta},
					\end{align}
					\endgroup
					where $\varepsilon \in (0, \min_{r, s} \frac{\sqrt{3}}{C_s(\Phi_{r, s}, \varepsilon_x^{-1})})$ and  $\varepsilon' =  \mathcal{O}(\varepsilon_x^{-1} L^4 \rho^{11} m^{-1/12} + \rho^2 L^{11/6} \varepsilon_x^{2/3}).$ 
					%Thus, the proof follows from adding eq.~\ref{eq:expectedtheta} and eq.~\ref{Eqn:conc_H}.
					Hence, using eq.~\ref{eq:expectedtheta} and eq.~\ref{Eqn:conc_H}, we have w.p. at least $1 - e^{-\Omega(\rho^2)}$,
					\begingroup \allowdisplaybreaks
					\begin{align*}
						\Big|&\widetilde{F}_{s}^{(L), \mathcal{K}}(\widetilde{\mathbf{h}}^{(L-1)}, \mathbf{x}^{(L)}) - \frac{\dout }{m} \sum_{i=1}^{L}  \sum_{s' \in [\dout ]} \sum_{r' \in [p]} \sum_{r \in \mathcal{K}}  b_{r, s'} b_{r', s'}^{\dagger} \widetilde{\mathbf{Back}}_{i \to L, r, s} \Phi_{r', s} \left(\left\langle \mathbf{w}_{r', s}^{\dagger}, [\overline{\mathbf{x}}^{(1)}, \cdots, \overline{\mathbf{x}}^{(L)}]\right\rangle\right)\Big| \\&
						\le \abs{\widetilde{F}_{s}^{(L), \mathcal{K}}(\widetilde{\mathbf{h}}^{(L-1)}, \mathbf{x}^{(L)}) - \mathbb{E}_{\left\{ \mathbf{w}_r, \mathbf{a}_r \right\}_{r \in \mathcal{K}}} \widetilde{F}_{s}^{(L), \mathcal{K}}(\widetilde{\mathbf{h}}^{(L-1)}, \mathbf{x}^{(L)})}  \\&
						+ \Big|\mathbb{E}_{\left\{ \mathbf{w}_r, \mathbf{a}_r \right\}_{r \in \mathcal{K}}} \widetilde{F}_{s}^{(L), \mathcal{K}}(\widetilde{\mathbf{h}}^{(L-1)}, \mathbf{x}^{(L)}) \\& \quad \quad \quad \quad - \frac{\dout }{m} \sum_{i=1}^{L}  \sum_{s' \in [\dout ]} \sum_{r' \in [p]} \sum_{r \in \mathcal{K}}  b_{r, s'} b_{r', s'}^{\dagger} \widetilde{\mathbf{Back}}_{i \to L, r, s} \Phi_{r', s} \left(\left\langle \mathbf{w}_{r', s}^{\dagger}, [\overline{\mathbf{x}}^{(1)}, \cdots, \overline{\mathbf{x}}^{(L)}]\right\rangle\right)\Big| \\&
						\le  \frac{\dout LpN}{m} \rho^2 (\varepsilon + \mathcal{O}(L_{\Phi} \rho^5 (N/m)^{1/6}) + L_{\Phi}\varepsilon') + \frac{\dout }{m} \cdot \mathcal{O}(\mathfrak{C}_{\varepsilon}(\Phi_{r' s}, \mathcal{O}(\varepsilon_x^{-1})) \rho^2 \sqrt{\dout LpN}),
					\end{align*}
					where $\varepsilon \in (0, \min_{r, s} \frac{\sqrt{3}}{C_s(\Phi_{r, s}, \varepsilon_x^{-1})})$ and  $\varepsilon' =  \mathcal{O}(\varepsilon_x^{-1} L^4 \rho^{11} m^{-1/12} + \rho^2 L^{11/6} \varepsilon_x^{2/3}).$ 
					\endgroup
					
				\end{proof}
				
				\begin{claim}[Restating claim~\ref{claim:fbacktildeback}]\label{claim:fbacktildeback_proof}
					With probability at least $1 - e^{-\Omega(\rho^2)}$,
					\begin{align*}
						&\Big| \frac{\dout }{m} \sum_{i=1}^{L}  \sum_{s' \in [\dout ]} \sum_{r' \in [p]} \sum_{r \in \mathcal{K}}  b_{r, s'} b_{r', s'}^{\dagger} \widetilde{\mathbf{Back}}_{i \to L, r, s} \Phi_{r', s} \left(\left\langle \mathbf{w}_{r', s}^{\dagger}, [\overline{\mathbf{x}}^{(1)}, \cdots, \overline{\mathbf{x}}^{(L)}]\right\rangle\right) \\& - \frac{\dout }{m} \sum_{i=1}^{L}  \sum_{s' \in [\dout ]} \sum_{r' \in [p]} \sum_{r \in \mathcal{K}}  b_{r, s'} b_{r', s'}^{\dagger} \mathbf{Back}_{i \to L, r, s} \Phi_{r', s} \left(\left\langle \mathbf{w}_{r', s}^{\dagger}, [\overline{\mathbf{x}}^{(1)}, \cdots, \overline{\mathbf{x}}^{(L)}]\right\rangle\right) \Big| \\& \le \mathcal{O}(\rho^8 C_{\Phi} \dout Lp N^{7/6} m^{-7/6}). 
					\end{align*}
				\end{claim}
				
				\begin{proof}
					From Lemma~\ref{lemma:rerandESN}, we have with probability at least $1 - e^{-\Omega(\rho^2)}$,
					\begin{align*}
						\abs{\mathbf{e}_s^{\top} \left(\mathbf{Back_{i \to j}} - \widetilde{\mathbf{Back}}_{i \to j} \right) \mathbf{e}_r} 
						&= \abs{\mathbf{b}_s^{\top} \left( \mathbf{D}^{(j)} \mathbf{W} \cdots \mathbf{D}^{(i)} \mathbf{W} - \widetilde{\mathbf{D}}^{(j)} \widetilde{\mathbf{W}} \cdots \widetilde{\mathbf{D}}^{(i)} \widetilde{\mathbf{W}} \right) \mathbf{e}_r}
						\\&
						\le \norm[2]{\mathbf{b}_s} \norm[2]{\left( \mathbf{D}^{(j)} \mathbf{W} \cdots \mathbf{D}^{(i)} \mathbf{W} - \widetilde{\mathbf{D}}^{(j)} \widetilde{\mathbf{W}} \cdots \widetilde{\mathbf{D}}^{(i)} \widetilde{\mathbf{W}} \right) \mathbf{e}_r}
						\\&\le \mathcal{O}(\dout ^{-1/2} \rho^7 N^{1/6} m^{-1/6}) , \text{ for all }  r \in [m], s \in [\dout ] \text{ and } 1 \le i \le j \le L.
					\end{align*}
					Hence,
					\begingroup \allowdisplaybreaks
					\begin{align*}
						&\Big| \frac{\dout }{m} \sum_{i=1}^{L}  \sum_{s' \in [\dout ]} \sum_{r' \in [p]} \sum_{r \in \mathcal{K}}  b_{r, s'} b_{r', s'}^{\dagger} \widetilde{\mathbf{Back}}_{i \to L, r, s} \Phi_{r', s} \left(\left\langle \mathbf{w}_{r', s}^{\dagger}, [\overline{\mathbf{x}}^{(1)}, \cdots, \overline{\mathbf{x}}^{(L)}]\right\rangle\right) \\& - \frac{\dout }{m} \sum_{i=1}^{L}  \sum_{s' \in [\dout ]} \sum_{r' \in [p]} \sum_{r \in \mathcal{K}}  b_{r, s'} b_{r', s'}^{\dagger} \mathbf{Back}_{i \to L, r, s} \Phi_{r', s} \left(\left\langle \mathbf{w}_{r', s}^{\dagger}, [\overline{\mathbf{x}}^{(1)}, \cdots, \overline{\mathbf{x}}^{(L)}]\right\rangle\right) \Big|
						\\&
						= \Big| \frac{\dout }{m} \sum_{i=1}^{L}  \sum_{s' \in [\dout ]} \sum_{r' \in [p]} \sum_{r \in \mathcal{K}}  b_{r, s'} b_{r', s'}^{\dagger} \mathbf{e}_r^{\top} \left( \widetilde{\mathbf{Back}}_{i \to L} \Phi_{r', s} - \mathbf{Back}_{i \to L}\right) \mathbf{e}_s \Phi_{r', s} \left(\left\langle \mathbf{w}_{r', s}^{\dagger}, [\overline{\mathbf{x}}^{(1)}, \cdots, \overline{\mathbf{x}}^{(L)}]\right\rangle\right) \Big| \\&
						\le \frac{\dout }{m} \sum_{i=1}^{L}  \sum_{s' \in [\dout ]} \sum_{r' \in [p]} \sum_{r \in \mathcal{K}}  \abs{b_{r, s'}} \cdot \abs{b_{r', s'}^{\dagger}} \cdot  \abs{\Phi_{r', s} \left(\left\langle \mathbf{w}_{r', s}^{\dagger}, [\overline{\mathbf{x}}^{(1)}, \cdots, \overline{\mathbf{x}}^{(L)}]\right\rangle\right)} \cdot \abs{ \mathbf{e}_r^{\top} \left( \widetilde{\mathbf{Back}}_{i \to L} \Phi_{r', s} - \mathbf{Back}_{i \to L}\right) \mathbf{e}_s } \\&
						\le \frac{\dout ^2LpN}{m} \cdot \frac{\rho}{\sqrt{\dout }} \cdot 1 \cdot C_{\Phi} \cdot \mathcal{O}(\dout ^{-1/2} \rho^7 N^{1/6} m^{-1/6}) \\&
						\le \mathcal{O}(\rho^8 C_{\Phi} \dout Lp N^{7/6} m^{-7/6}).
					\end{align*}
					\endgroup
					In the final step, we have used the bounds of different terms as follows.
					we will need a couple of bounds on the terms that appear in the equations.
					\begin{itemize}
						\item Using the fact~\ref{fact:max_gauss}, we can show that with probability $1-e^{-\Omega(\rho^2)}$,
						$\max_{r, s'} | b_{r, s'} | \le \frac{\rho}{\sqrt{\dout }}.$ 
						\item From the definition of concept class,  $\max_{r',s'} | b_{r', s'}^{\dagger} | \le 1$ and $\max_{r', s} \abs{\Phi_{r', s}} \le C_{\Phi}$ in the desired range.
					\end{itemize}
					%\todo{Add some details on the bounds above.}
				\end{proof}
				
				
			
			%\end{proof}
			
			
			
			
		
			
			%Now, in the next claim, we show that the $f$ concentrates on the desired term. 
			\begin{claim}[Restating claim~\ref{claim:simplifybig}]\label{claim:simplifybig_proof}
				With probability exceeding $1 - e^{-\Omega(\rho^2)}$,
				\begin{align*}
					&\Big|  b_{r', s}^{\dagger}  \Phi_{r', s} \left(\left\langle \mathbf{w}_{r', s}^{\dagger}, [\overline{\mathbf{x}}^{(2)}, \cdots, \overline{\mathbf{x}}^{(L-1)}]\right\rangle\right) \\& - \frac{\dout }{m} \sum_{i=1}^{L}  \sum_{s' \in [\dout ]}  \sum_{r \in [m]}  b_{r, s'} b_{r', s'}^{\dagger} \mathbf{Back}_{i \to L, r, s} \Phi_{r', s} \left(\left\langle \mathbf{w}_{r', s}^{\dagger}, [\overline{\mathbf{x}}^{(2)}, \cdots, \overline{\mathbf{x}}^{(L-1)}]\right\rangle\right) \Big| \\&\le  \mathcal{O}(L\dout  \rho C_{\Phi} m^{-0.25}).
				\end{align*}
			\end{claim}
			
			\begin{proof}
				\begingroup \allowdisplaybreaks
				\begin{align*}
					&\Big| \Phi_{r', s} \left(\left\langle \mathbf{w}_{r', s}^{\dagger}, [\overline{\mathbf{x}}^{(2)}, \cdots, \overline{\mathbf{x}}^{(L-1)}]\right\rangle\right) \\& \quad \quad  - \frac{\dout }{m} \sum_{i=1}^{L}  \sum_{s' \in [\dout ]}  \sum_{r \in [m]}  b_{r, s'} b_{r', s'}^{\dagger} \mathbf{Back}_{i \to L, r, s} \Phi_{r', s} \left(\left\langle \mathbf{w}_{r', s}^{\dagger}, [\overline{\mathbf{x}}^{(2)}, \cdots, \overline{\mathbf{x}}^{(L-1)}]\right\rangle\right) \Big| \nonumber\\&
					\le 
					\Big| \Phi_{r', s} \left(\left\langle \mathbf{w}_{r', s}^{\dagger}, [\overline{\mathbf{x}}^{(2)}, \cdots, \overline{\mathbf{x}}^{(L-1)}]\right\rangle\right)  - \frac{\dout }{m} \sum_{r \in [m]}  b_{r, s} b_{r', s}^{\dagger} \mathbf{Back}_{L \to L, r, s} \Phi_{r', s} \left(\left\langle \mathbf{w}_{r', s}^{\dagger}, [\overline{\mathbf{x}}^{(2)}, \cdots, \overline{\mathbf{x}}^{(L-1)}]\right\rangle\right) \Big| \\&
					+ \Big| \ \frac{\dout }{m} \sum_{s'\in [\dout ]: s' \ne s} \sum_{r \in [m]}  b_{r, s'} b_{r', s'}^{\dagger} \mathbf{Back}_{L \to L, r, s} \Phi_{r', s} \left(\left\langle \mathbf{w}_{r', s}^{\dagger}, [\overline{\mathbf{x}}^{(2)}, \cdots, \overline{\mathbf{x}}^{(L-1)}]\right\rangle\right) \Big| \\&
					+ \Big| \ \frac{\dout }{m}  \sum_{i=1}^{L-1} \sum_{s'\in [\dout ]: s' \ne s} \sum_{r \in [m]}  b_{r, s'} b_{r', s'}^{\dagger} \mathbf{Back}_{i \to L, r, s} \Phi_{r', s} \left(\left\langle \mathbf{w}_{r', s}^{\dagger}, [\overline{\mathbf{x}}^{(2)}, \cdots, \overline{\mathbf{x}}^{(L-1)}]\right\rangle\right) \Big| \\&
					+ \Big|  \frac{\dout }{m} \sum_{i=1}^{L-1} \sum_{r \in [m]}  b_{r, s} b_{r', s}^{\dagger} \mathbf{Back}_{i \to L, r, s} \Phi_{r', s} \left(\left\langle \mathbf{w}_{r', s}^{\dagger}, [\overline{\mathbf{x}}^{(2)}, \cdots, \overline{\mathbf{x}}^{(L-1)}]\right\rangle\right) \Big|.
				\end{align*}
				\endgroup
				
				Since, $\mathbf{B} = \mathbf{Back}_{L \to L}$ by definition, we can simplify the above 4 terms as
				\begingroup \allowdisplaybreaks
				\begin{align}
					&\Big| \Phi_{r', s} \left(\left\langle \mathbf{w}_{r', s}^{\dagger}, [\overline{\mathbf{x}}^{(2)}, \cdots, \overline{\mathbf{x}}^{(L-1)}]\right\rangle\right) \\& - \frac{\dout }{m} \sum_{i=1}^{L}  \sum_{s' \in [\dout ]}  \sum_{r \in [m]}  b_{r, s'} b_{r', s'}^{\dagger} \mathbf{Back}_{i \to L, r, s} \Phi_{r', s} \left(\left\langle \mathbf{w}_{r', s}^{\dagger}, [\overline{\mathbf{x}}^{(2)}, \cdots, \overline{\mathbf{x}}^{(L-1)}]\right\rangle\right) \Big| \nonumber\\& 
					\le 
					\Big| \Phi_{r', s} \left(\left\langle \mathbf{w}_{r', s}^{\dagger}, [\overline{\mathbf{x}}^{(2)}, \cdots, \overline{\mathbf{x}}^{(L-1)}]\right\rangle\right)  - \frac{\dout }{m} \sum_{r \in [m]}  b^2_{r, s} b_{r', s}^{\dagger} \Phi_{r', s} \left(\left\langle \mathbf{w}_{r', s}^{\dagger}, [\overline{\mathbf{x}}^{(2)}, \cdots, \overline{\mathbf{x}}^{(L-1)}]\right\rangle\right) \Big| \nonumber\\&
					+ \Big| \ \frac{\dout }{m} \sum_{s'\in [\dout ]: s' \ne s} \sum_{r \in [m]}  b_{r, s'} b_{r', s'}^{\dagger} b_{r, s} \Phi_{r', s} \left(\left\langle \mathbf{w}_{r', s}^{\dagger}, [\overline{\mathbf{x}}^{(2)}, \cdots, \overline{\mathbf{x}}^{(L-1)}]\right\rangle\right) \Big| \nonumber\\&
					+ \Big| \ \frac{\dout }{m}  \sum_{i=1}^{L} \sum_{s'\in [\dout ]: s' \ne s} \sum_{r \in [m]}  \mathbf{Back}_{L \to L, r, s'} b_{r', s'}^{\dagger} \mathbf{Back}_{i \to L, r, s} \Phi_{r', s} \left(\left\langle \mathbf{w}_{r', s}^{\dagger}, [\overline{\mathbf{x}}^{(2)}, \cdots, \overline{\mathbf{x}}^{(L-1)}]\right\rangle\right) \Big| \nonumber\\&
					+ \Big|  \frac{\dout }{m} \sum_{i=1}^{L-1} \sum_{r \in [m]}  \mathbf{Back}_{L \to L, r, s}  b_{r', s}^{\dagger} \mathbf{Back}_{i \to L, r, s} \Phi_{r', s} \left(\left\langle \mathbf{w}_{r', s}^{\dagger}, [\overline{\mathbf{x}}^{(2)}, \cdots, \overline{\mathbf{x}}^{(L-1)}]\right\rangle\right) \Big| \nonumber\\&
					= \Big| \Phi_{r', s} \left(\left\langle \mathbf{w}_{r', s}^{\dagger}, [\overline{\mathbf{x}}^{(2)}, \cdots, \overline{\mathbf{x}}^{(L-1)}]\right\rangle\right)  - \frac{\dout }{m} \sum_{r \in [m]}  b^2_{r, s} b_{r', s}^{\dagger} \Phi_{r', s} \left(\left\langle \mathbf{w}_{r', s}^{\dagger}, [\overline{\mathbf{x}}^{(2)}, \cdots, \overline{\mathbf{x}}^{(L-1)}]\right\rangle\right) \Big| \label{eq:Comptosimple1}\\&
					+ \Big| \ \frac{\dout }{m} \sum_{s'\in [\dout ]: s' \ne s} \sum_{r \in [m]}  b_{r, s'} b_{r', s'}^{\dagger} b_{r, s} \Phi_{r', s} \left(\left\langle \mathbf{w}_{r', s}^{\dagger}, [\overline{\mathbf{x}}^{(2)}, \cdots, \overline{\mathbf{x}}^{(L-1)}]\right\rangle\right) \Big| \label{eq:Comptosimple2}\\&
					+ \Big| \ \frac{\dout }{m}  \sum_{i=1}^{L-1} \sum_{s'\in [\dout ]: s' \ne s}  b_{r', s'}^{\dagger} \Phi_{r', s} \left(\left\langle \mathbf{w}_{r', s}^{\dagger}, [\overline{\mathbf{x}}^{(2)}, \cdots, \overline{\mathbf{x}}^{(L-1)}]\right\rangle\right) \left\langle \mathbf{e}_s^{\top} \mathbf{Back}_{L \to L}, \mathbf{e}_{s'}^{\top} \mathbf{Back}_{i \to L} \right\rangle  \Big| \label{eq:Comptosimple3}\\&
					+ \Big|  \frac{\dout }{m} \sum_{i=1}^{L-1}  b_{r', s}^{\dagger} \Phi_{r', s} \left(\left\langle \mathbf{w}_{r', s}^{\dagger}, [\overline{\mathbf{x}}^{(2)}, \cdots, \overline{\mathbf{x}}^{(L-1)}]\right\rangle\right) \left\langle \mathbf{e}_s^{\top} \mathbf{Back}_{L \to L}, \mathbf{e}_s^{\top} \mathbf{Back}_{i \to L} \right\rangle  \Big| \label{eq:Comptosimple4}.
				\end{align}
				\endgroup
				First, we can use Lemma~\ref{lemma:backward_correlation} to show that both eq.~\ref{eq:Comptosimple3} and eq.~\ref{eq:Comptosimple4} are small.
				\begin{align*}
					&\Big| \ \frac{\dout }{m}  \sum_{i=1}^{L-1} \sum_{s'\in [\dout ]: s' \ne s}  b_{r', s'}^{\dagger} \Phi_{r', s} \left(\left\langle \mathbf{w}_{r', s}^{\dagger}, [\overline{\mathbf{x}}^{(2)}, \cdots, \overline{\mathbf{x}}^{(L-1)}]\right\rangle\right) \left\langle \mathbf{e}_s^{\top} \mathbf{Back}_{L \to L}, \mathbf{e}_{s'}^{\top} \mathbf{Back}_{i \to L} \right\rangle  \Big| \\&
					\le \sum_{i=1}^{L-1} \sum_{s'\in [\dout ]: s' \ne s} \frac{\dout }{m} \cdot \abs{b_{r', s'}^{\dagger}} \cdot \abs{\Phi_{r', s} \left(\left\langle \mathbf{w}_{r', s}^{\dagger}, [\overline{\mathbf{x}}^{(2)}, \cdots, \overline{\mathbf{x}}^{(L-1)}]\right\rangle\right)} \cdot \abs{\left\langle \mathbf{e}_s^{\top} \mathbf{Back}_{L \to L}, \mathbf{e}_{s'}^{\top} \mathbf{Back}_{i \to L} \right\rangle } \\& \le  \mathcal{O}(L\dout \rho C_{\Phi} m^{-0.25}).
				\end{align*}
				
				Also, 
				\begin{align*}
					&\Big|  \frac{\dout }{m} \sum_{i=1}^{L-1}  b_{r', s}^{\dagger} \Phi_{r', s} \left(\left\langle \mathbf{w}_{r', s}^{\dagger}, [\overline{\mathbf{x}}^{(2)}, \cdots, \overline{\mathbf{x}}^{(L-1)}]\right\rangle\right) \left\langle \mathbf{e}_s^{\top} \mathbf{Back}_{L \to L}, \mathbf{e}_s^{\top} \mathbf{Back}_{i \to L} \right\rangle  \Big| \\&
					\le \sum_{i=1}^{L-1}  \frac{\dout }{m} \cdot \abs{b_{r', s}^{\dagger}} \cdot \abs{\Phi_{r', s} \left(\left\langle \mathbf{w}_{r', s}^{\dagger}, [\overline{\mathbf{x}}^{(2)}, \cdots, \overline{\mathbf{x}}^{(L-1)}]\right\rangle\right)} \cdot \abs{\left\langle \mathbf{e}_s^{\top} \mathbf{Back}_{L \to L}, \mathbf{e}_{s}^{\top} \mathbf{Back}_{i \to L} \right\rangle } \\& \le  \mathcal{O}(L \rho C_{\Phi} m^{-0.25}).
				\end{align*}
				
				Since, $\mathbf{b}_{s} \sim \mathcal{O}(0, \frac{1}{\dout }\mathbb{I})$, we can show using using fact~\ref{lem:chi-squared} that with probability at least $1 - e^{-\Omega(\rho^2)}$,
				\begin{align*}
					\abs{\frac{\dout }{m} \sum_{r \in [m]} b_{r, s}^2 - 1} \le \mathcal{O}(\frac{\rho}{\sqrt{m}}).
				\end{align*}   
				Hence, eq.~\ref{eq:Comptosimple1} can be simplified as 
				\begin{align*}
					&\Big| \Phi_{r', s} \left(\left\langle \mathbf{w}_{r', s}^{\dagger}, [\overline{\mathbf{x}}^{(2)}, \cdots, \overline{\mathbf{x}}^{(L-1)}]\right\rangle\right)  - \frac{\dout }{m} \sum_{r \in [m]}  b^2_{r, s} b_{r', s}^{\dagger} \Phi_{r', s} \left(\left\langle \mathbf{w}_{r', s}^{\dagger}, [\overline{\mathbf{x}}^{(2)}, \cdots, \overline{\mathbf{x}}^{(L-1)}]\right\rangle\right) \Big| \\&
					\le \mathcal{O}(\frac{\rho}{\sqrt{m}}) \cdot \abs{b_{r', s}^{\dagger}} \abs{\Phi_{r', s} \left(\left\langle \mathbf{w}_{r', s}^{\dagger}, [\overline{\mathbf{x}}^{(2)}, \cdots, \overline{\mathbf{x}}^{(L-1)}]\right\rangle\right)} 
					\le \mathcal{O}(C_{\Phi}\frac{\rho}{\sqrt{m}}). 
				\end{align*}
				Also, 
				\begin{align*}
					\frac{\dout }{m} \sum_{r \in [m]}  b_{r, s} b_{r, s'} = \frac{1}{2m} \left(\norm{b_{r, s} + b_{r, s'}}^2 - \norm{b_{r, s} - b_{r, s'}}^2\right).    
				\end{align*}
				Since, both $\mathbf{b}_s$ and $\mathbf{b}_{s'}$ are independent gaussian vectors, $\mathbf{b}_{s} + \mathbf{b}_{s'} \sim \mathcal{N}(0, \frac{2}{\dout }\mathbb{I})$ and $\mathbf{b}_{s} - \mathbf{b}_{s'} \sim \mathcal{N}(0, \frac{2}{\dout }\mathbb{I})$. Hence, using fact~\ref{lem:chi-squared} we have with probability at least $1 - e^{-\Omega(\rho^2)}$, for all $s' \in [\dout ]$,
				\begin{align*}
					&\abs{\frac{\dout }{m} \sum_{r \in [m]} (b_{r, s'} + b_{r, s})^2 - 2} \le \mathcal{O}(\frac{\rho}{\sqrt{m}}) \\&
					\abs{\frac{\dout }{m} \sum_{r \in [m]} (b_{r, s'} - b_{r, s})^2 - 2} \le \mathcal{O}(\frac{\rho}{\sqrt{m}}),
				\end{align*}
				and thus
				\begin{align*}
					\abs{\frac{\dout }{m} \sum_{r \in [m]}  b_{r, s} b_{r, s'}} \le \mathcal{O}(\frac{\rho}{\sqrt{m}}).
				\end{align*}
				This can be used to simplify eq.~\ref{eq:Comptosimple4}.
				\begin{align*}
					&\Big|  \frac{\dout }{m} \sum_{s'\in [\dout ]: s' \ne s} \sum_{r \in [m]}  b_{r, s'} b_{r', s'}^{\dagger} b_{r, s} \Phi_{r', s} \left(\left\langle \mathbf{w}_{r', s}^{\dagger}, [\overline{\mathbf{x}}^{(2)}, \cdots, \overline{\mathbf{x}}^{(L-1)}]\right\rangle\right) \Big| \\&
					\le  \sum_{s'\in [\dout ]: s' \ne s} \Big|  \frac{\dout }{m} \sum_{r \in [m]}  b_{r, s'} b_{r', s'}^{\dagger} b_{r, s} \Phi_{r', s} \left(\left\langle \mathbf{w}_{r', s}^{\dagger}, [\overline{\mathbf{x}}^{(2)}, \cdots, \overline{\mathbf{x}}^{(L-1)}]\right\rangle\right) \Big|\\&
					\le \mathcal{O}(\frac{\rho}{\sqrt{m}})  \sum_{s'\in [\dout ]: s' \ne s} \cdot \abs{b_{r', s'}^{\dagger}} \abs{\Phi_{r', s} \left(\left\langle \mathbf{w}_{r', s}^{\dagger}, [\overline{\mathbf{x}}^{(2)}, \cdots, \overline{\mathbf{x}}^{(L-1)}]\right\rangle\right)}\\&
					\le \mathcal{O}(C_{\Phi} \dout  \frac{\rho}{\sqrt{m}}).
				\end{align*}
				Hence, adding everything up, we have with probability exceeding $1 - e^{-\Omega(\rho^2)}$,
				\begin{align*}
					&\Big| \Phi_{r', s} \left(\left\langle \mathbf{w}_{r', s}^{\dagger}, [\overline{\mathbf{x}}^{(2)}, \cdots, \overline{\mathbf{x}}^{(L-1)}]\right\rangle\right)  \\&- \frac{\dout }{m} \sum_{i=1}^{L}  \sum_{s' \in [\dout ]}  \sum_{r \in [m]}  b_{r, s'} b_{r', s'}^{\dagger} \mathbf{Back}_{i \to L, r, s} \Phi_{r', s} \left(\left\langle \mathbf{w}_{r', s}^{\dagger}, [\overline{\mathbf{x}}^{(2)}, \cdots, \overline{\mathbf{x}}^{(L-1)}]\right\rangle\right) \Big| \\&
					\le  \mathcal{O}(C_{\Phi}\frac{\rho}{\sqrt{m}}) + \mathcal{O}(C_{\Phi} \dout  \frac{\rho}{\sqrt{m}}) + \mathcal{O}(L\dout \rho C_{\Phi} m^{-0.25}) +  \mathcal{O}(L \rho C_{\Phi} m^{-0.25})  \\&
					\le \mathcal{O}(L\dout  \rho C_{\Phi} m^{-0.25}).
				\end{align*}
			\end{proof}
			Thus, introducing claim~\ref{claim:simplifybig} in eq.~\ref{eq:prefinalf}, we have
			\begingroup \allowdisplaybreaks
			\begin{align*}
				&\abs{ F_s^{(L)}(\mathbf{h}^{(\ell-1)}, \mathbf{x}^{(\ell)}) - \sum_{r' \in [p]}  b_{r', s}^{\dagger} \Phi_{r', s} \left(\left\langle \mathbf{w}_{r', s}^{\dagger}, [\overline{\mathbf{x}}^{(2)}, \cdots, \overline{\mathbf{x}}^{(L-1)}]\right\rangle\right) } \\& \le \mathcal{O}(\dout Lp\rho^8  \mathfrak{C}_{\varepsilon}(\Phi, \mathcal{O}(\varepsilon_x^{-1}))  m^{-1/30}) + \mathcal{O}(\mathfrak{C}_{\varepsilon}(\Phi_{r' s}, \mathcal{O}(\varepsilon_x^{-1})) \rho^2 \sqrt{\dout ^3Lp} m^{-0.1}) \\& + \dout Lp \rho^2 (\varepsilon + \mathcal{O}( L_{\Phi}\rho^5 m^{-2/15}) + \mathcal{O}(\varepsilon_x^{-1} L_{\Phi} L^4 \rho^{11} m^{-1/12} +  L_{\Phi} L^{11/6} \rho^2 \varepsilon_x^{2/3})) \\& + \mathcal{O}(\rho^8\dout Lp  m^{-2/15})  + \mathcal{O}(Lp\dout  \rho C_{\Phi} m^{-0.25}) \\&
				\le \mathcal{O}(\dout Lp\rho^2 \varepsilon + \dout L^{17/6} p \rho^4 L_{\Phi} \varepsilon_x^{2/3} + \dout ^{3/2} L^5 p \rho^{11} L_{\Phi} C_{\Phi}  \mathfrak{C}_{\varepsilon}(\Phi, \mathcal{O}(\varepsilon_x^{-1}))  m^{-1/30} ) .
			\end{align*}
			\endgroup
		%\end{proof}
		%\end{claim}
		%\end{proof}
		
		\begin{lemma}\label{lemma:norm_WA}
			With probability at least $1-e^{-\Omega(\rho^2)}$,
			\begin{align*}
				& \norm{\mathbf{W}^{\ast}}_F = 0,
				\\
				&\norm{\mathbf{A}^{\ast}}_F \le \mathcal{O}\left(\rho \dout ^{1/2} \frac{\mathfrak{C}_{\varepsilon}(\Phi, \mathcal{O}(\varepsilon_x^{-1}))}{\sqrt{m}}\right).
			\end{align*}
			\begin{proof}
				The norm of $\mathbf{W}^{\ast}$ follows from the fact that it is a zero matrix. From def.~\ref{def:existence}, we have that
				\begin{align*}
					\mathbf{a}^{*}_{r} = \frac{\dout }{m} \sum_{s \in [\dout ]} \sum_{r' \in [p]} b_{r, s} b_{r', s}^{\dagger} H_{r', s} \left(\theta_{r', s} \left(\langle \mathbf{w}_{r}, \overline{\mathbf{W}}^{[L]} \mathbf{w}_{r', s}^{\dagger}\rangle\right), \sqrt{m/2} a_{r, d}\right) \mathbf{e}_d, \quad \forall r \in [m],
				\end{align*}
				where
				\begin{equation*} 
					\theta_{r', s} = \frac{\sqrt{m/2}}{\norm[1]{ \overline{\mathbf{W}}^{[L]} \mathbf{w}_{r', s}^{\dagger}}}.
				\end{equation*}
				Since, there is a dependence between $\mathbf{w}_r$ and $\overline{\mathbf{W}}^{[L]}$, we again need to re-randomize some rows of $\overline{\mathbf{W}}^{[L]}$ as has been done in thm.~\ref{thm:existence_pseudo}. Following the steps as has been done to bound eq.~\ref{eq:rerandRNN_3}, we can get
				\begin{align*}
					\mathbf{A}^{\ast} = \widetilde{\mathbf{A}}^{\ast} + \overline{\mathbf{A}}^{\ast},
				\end{align*}
				where $\norm{\overline{\mathbf{A}}^{\ast}}_F \le \mathcal{O}(\mathfrak{C}_{\varepsilon}(\Phi, \mathcal{O}(\varepsilon_x^{-1})) \rho^6 m^{-5/6})$ with probability at least $1-e^{-\Omega(\rho^2)}$ and for each $r \in [m]$,
				\begin{align*}
					\widetilde{\mathbf{a}}^{*}_{r} = \frac{\dout }{m} \sum_{s \in [\dout ]} \sum_{r' \in [p]} b_{r, s} b_{r', s}^{\dagger} H_{r', s} \left(\theta_{r', s} \left(\langle \mathbf{w}_{r}, \overline{\widetilde{\mathbf{W}}}^{[L]} \mathbf{w}_{r', s}^{\dagger}\rangle\right), \sqrt{m/2} a_{r, d}\right) \mathbf{e}_d, 
				\end{align*}
				where $\overline{\widetilde{\mathbf{W}}}^{[L]}$ doesn't depend on the weight vector $\mathbf{w}_r$. Using the properties of the function $H_{r', s}$ from def.~\ref{Def:Function_approx}, we can show that with probability at least $1-e^{-\Omega(\rho^2)}$,
				\begin{align*}
					\norm{\widetilde{\mathbf{A}}^{\ast}}_F \le \mathcal{O}(\dout ^{1/2} \mathfrak{C}_{\varepsilon}(\Phi, \mathcal{O}(\varepsilon_x^{-1})) \rho m^{-1/2}).
				\end{align*}
				\todo{Add few more details if you have time.}
			\end{proof}
		\end{lemma}