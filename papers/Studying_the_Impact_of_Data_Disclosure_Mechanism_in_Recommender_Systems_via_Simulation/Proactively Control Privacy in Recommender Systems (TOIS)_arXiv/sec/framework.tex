\section{Framework Formulation}






\subsection{Overview} %
\section{Framework}
\label{sec:framework}


\graphicspath{{./}{./images_rss_2015/}{./images_hri_2016/}}


%Under this:
% Shared autonomy assumes $\transition(\stateenv' \given \stateenv, \actionuser, \actionrobot) = \transition(\stateenv' \given \stateenv, 0, \actionrobot)$ - that is, the user does not directly affect the state.
% collab doesn't really care about the user state in cost 


We present our framework for minimizing a cost function for shared autonomy with an unknown user goal. We assume the user's goal is fixed, and they take actions to achieve that goal without considering autonomous assistance. These actions are used to predict the user's goal based on how optimal the action is for each goal (\cref{sec:framework_prediction}). Our system uses this distribution to minimize the expected cost-to-go  (\cref{sec:framework_unknown_goal}). As solving for the optimal action is infeasible, we use hindsight optimization to approximate a solution (\cref{sec:framework_hindsight}). For reference, see \cref{table:variable_definitions} in \cref{sec:variable_definitions} for variable definitions.

\subsection{Cost minimization with a known goal}
\label{sec:framework_known_goal}

We first formulate the problem for a known user goal, which we will use in our solution with an unknown goal. We model this problem as a Markov Decision Process (MDP). 

Formally, let $\stateenv \in \Stateenv$ be the environment state (e.g. human and robot pose). Let $\actionuser \in \Actionuser$ be the user actions, and $\actionrobot \in \Actionrobot$ the robot actions. Both agents can affect the environment state - if the user takes action $\actionuser$ and the robot takes action $\actionrobot$ while in state $\stateenv$, the environment stochastically transitions to a new state $\stateenv'$ through $\transitionallargs$. 

We assume the user has an intended goal $\goal \in \Goal$ which does not change during execution. We augment the environment state with this goal, defined by $\state = \left(\stateenv, \goal\right) \in \Stateenv \times \Goal$. We overload our transition function to model the transition in environment state without changing the goal, $\transition( (\stateenv', g) \given (\stateenv, g), \actionuser, \actionrobot) = \transitionallargs$.

%In our scenario, the user wants to move the robot to one goal in a discrete set of goals $\goal \in \Goal$.
We assume access to a user policy for each goal $\policyuser(\actionuser \given \state) = \policyusergoal(\actionuser \given \stateenv) = p(\actionuser \given \stateenv, \goal)$. We model this policy using the maximum entropy inverse optimal control (MaxEnt IOC) framework of~\citet{ziebart_2008}, where the policy corresponds to stochastically optimizing a cost function $\costuser(\state, \actionuser) = \costusergoal(\stateenv, \actionuser)$. We assume the user selects actions based only on $\state$, the current environment state and their intended goal, and does not model any actions that the robot might take. Details are in \cref{sec:framework_prediction}.

The robot selects actions to minimize a cost function dependent on the user goal and action $\costrobot(\state, \actionuser, \actionrobot) = \costrobotgoal(\stateenv, \actionuser, \actionrobot)$. At each time step, we assume the user first selects an action, which the robot observes before selecting $\actionrobot$. The robot selects actions based on the state and user inputs through a policy $\policyrobot(\actionrobot \given \state, \actionuser) = p(\actionrobot \given \state, \actionuser)$. We define the value function for a robot policy $\vrobot^{\policyrobot}$ as the expected cost-to-go from a particular state, assuming some user policy $\policyuser$:
\begin{align*}
  \vrobot^{\policyrobot}(\state) &= \expctarg{\sumtime \costrobot(\state_t, \actionuser_t, \actionrobot_t) \given \state_0 = \state}\\
  \actionuser_t &\sim \policyuser(\cdot \given \state_t)\\
  \actionrobot_t &\sim \policyrobot(\cdot \given \state_t, \actionuser_t)\\
  \state_{t+1} &\sim \transition(\cdot \given \state_t, \actionuser_t, \actionrobot_t)
\end{align*}
%\vrobot^{\policyrobot}(\state) &= \expctover{\policyuser, \policyrobot, \transition}{\sumtime \costrobot(\state_t, \actionuser_t, \actionrobot_t) \given \state_0 = \state}\\

The optimal value function $\vopt$ is the cost-to-go for the best robot policy:
\begin{align*}
  \vopt(\state) &= \min_{\policyrobot} \vrobot^{\policyrobot}(\state)
\end{align*}

The action-value function $\qopt$ computes the immediate cost of taking action $\actionrobot$ after observing $\actionuser$, and following the optimal policy thereafter:
\begin{align*}
  \qopt(\stateactions) &= \costrobot(\stateactions) + \expctarg{\vopt(\state')}
\end{align*}
%\qopt(\stateactions) &= \costrobot(\stateactions) + \expctover{\state'}{\vopt(\state')}
Where $\state' \sim \transition(\cdot \given \stateactions)$. The optimal robot action is given by $\argmin_\actionrobot \qopt(\stateactions)$.

In order to make explicit the dependence on the user goal, we often write these quantities as:
\begin{align*}
  \vgoal(\stateenv) &= \vopt(\state)\\
  \qgoal(\stateenvactions) &= \qopt(\stateactions)
\end{align*}

Computing the optimal policy and corresponding action-value function is a common objective in reinforcement learning. We assume access to this function in our framework, and describe our particular implementation in the experiments.

%The action-value function $\qrobot$ is defined as the immediate cost of taking action $\actionrobot$ after observing $\actionuser$, plus the cost of following $\policyrobot$ thereafter:
%\begin{align*}
%  \qrobot^{\policyrobot}(\stateactions) &= \costrobot(\stateactions) + \expctover{\state'}{\vrobot^{\policyrobot}(\state')}
%\end{align*}
%
%We define the optimal value and action-value functions as the cost-to-go for the best robot policy:
%\begin{align*}
%  \vopt(\state) &= \min_{\policyrobot} \vrobot^{\policyrobot}(\state)\\
%  \qopt(\stateactions) &= \costrobot(\stateactions) + \expctover{\state'}{\vopt(\state)}
%\end{align*}


\subsection{Cost Minimization with an unknown goal}
\label{sec:framework_unknown_goal}

We formulate the problem of minimizing a cost function with an unknown user goal as a Partially Observable Markov Decision Process (POMDP). A POMDP maps a distribution over states, known as the \emph{belief} $\belief$, to actions. We assume that all uncertainty is over the user's goal, and the environment state is known. This subclass of POMDPs, where uncertainty is constant, has been studied as a Hidden Goal MDP~\citep{fern_2010}, and as a POMDP-lite~\citep{chen_2016}.

In this framework, we infer a distribution of the user's goal by observing the user actions $\actionuser$. Similar to the known-goal setting (\cref{sec:framework_known_goal}), we define the value function of a belief as:
\begin{align*}
  \vrobot^{\policyrobot}(\belief) &= \expctarg{\sumtime \costrobot(\state_t, \actionuser_t, \actionrobot_t)  \given \belief_0 = \belief} \\
  \state_t &\sim \belief_t\\
  \actionuser_t &\sim \policyuser(\cdot \given \state_t)\\
  \actionrobot_t &\sim \policyrobot(\cdot \given \state_t, \actionuser_t)\\
  \belief_{t+1} &\sim \transitionbelief(\cdot \given \belief_t, \actionuser_t, \actionrobot_t)
\end{align*}
%\vrobot^{\policyrobot}(\belief) &= \expctover{\belief, \policyuser, \policyrobot, \transition}{\sumtime \costrobot(\state_t, \actionuser_t, \actionrobot_t)  \given \belief_0 = \belief} \\
Where the belief transition $\transitionbelief$ corresponds to transitioning the known environment state $\stateenv$ according to $\transition$, and updating our belief over the user's goal as described in $\cref{sec:framework_prediction}$. We can define quantities similar to above over beliefs:
\begin{align}
  \vopt(\belief) &= \min_{\policyrobot} \vrobot^{\policyrobot}(\belief) \label{eq:v_belief}\\
  \qopt(\beliefactions) &= \expctarg{\costrobot(\belief, \actionuser, \actionrobot) + \expctover{\belief'}{\vopt(\belief')}} \nonumber
\end{align}
%\qopt(\beliefactions) &= \expctover{\belief}{\costrobot(\belief, \actionuser, \actionrobot) + \expctover{\belief'}{\vopt(\belief')}}%\\&= \expctover{\belief}{\costrobot(\belief, \actionuser, \actionrobot)} + \min_\policyrobot \expctover{\belief}{\expctover{\belief'}{\vopt(\belief')}}


%\begin{align*}
%  \qrobot^{\policyrobot}(\beliefactions) &= \expctover{\belief}{\costrobot(\beliefactions) + \expctover{\belief'}{\vrobot^{\policyrobot}(\belief')}}
%\end{align*}




%As the robot does not know the user's goal a priori, we infer this goal from the user actions $\actionuser$ based on our models $\policyuser(\state)$.

%We can model the robots objective of minimizing a cost function with uncertainty using a Partially Observable Markov Decision Process (POMDP) with uncertainty over the user's goal. A POMDP maps a distribution over states, known as the \emph{belief} $\belief$, to actions. We assume that all uncertainty is over the user's goal, and the environment state is known, as in a Hidden Goal MDP~\citep{fern_2010}. Note that allowing the cost to depend on the observation $\actionuser$ is non-standard, but important for shared autonomy, as prior works suggest that users prefer maintaining control authority~\citep{kim_2012}. Our shared autonomy POMDP is defined by the tuple $\left(\Staterobgoal, \Actionrobot, \transition, \costrobot, \Actionuser, \policyuser, \right)$. The optimal solution to this POMDP minimizes the expected accumulated cost $\costrobot$. As this is intractable to compute, we utilize Hindsight Optimization to select actions, described in \cref{sec:framework_hindsight}.





%%We model the robot as a deterministic dynamical system with transition function $\transition: \Stateenv \times \Actionrobot \rightarrow \Stateenv$. %  where applying action $\actionrobot$ in state $\stateenv$ results in state $\stateenv'$.
%%The user supplies continuous inputs $\actionuser \in \Actionuser$ via an interface (e.g. joystick, mouse). These user inputs map to robot actions through a known deterministic function $\userinputtoaction: \Actionuser \rightarrow \Actionrobot$, corresponding to the effect of \emph{direct teleoperation}.
%
%In our scenario, the user wants to move the robot to one goal in a discrete set of goals $\goal \in \Goal$. We assume access to a stochastic user policy for each goal $\policyusergoal(\stateenv) = p(\actionuser | \stateenv, \goal)$, usually learned from user demonstrations. %Here, the user assumes inputs get mapped directly to actions through $\userinputtoaction$ - thus, they assume direct teleoperation.
%In our system, we model this policy using the maximum entropy inverse optimal control (MaxEnt IOC) framework of~\citet{ziebart_2008}, which assumes the user is approximately optimizing some cost function for their intended goal $g$, $\costusergoal: \Stateenv \times \Actionuser \rightarrow \mathcal{R}$. This model corresponds to a goal specific Markov Decision Process (MDP), defined by the tuple $\left(\Stateenv, \Actionuser, \transition, \costusergoal\right)$. We discuss details in \cref{sec:framework_prediction}. 
%
%Unlike the user, our system does not know the intended goal. We model this with a Partially Observable Markov Decision Process (POMDP) with uncertainty over the user's goal. A POMDP maps a distribution over states, known as the \emph{belief} $\belief$, to actions. Define the system state $\staterobgoal \in \Staterobgoal$ as the robot state augmented by a goal, $\staterobgoal = (\stateenv, \goal)$ and $\Staterobgoal = \Stateenv \times \Goal$. In a slight abuse of notation, we overload our transition function such that $\transition: \Staterobgoal \times \Actionrobot \rightarrow \Staterobgoal$, which corresponds to transitioning the robot state as above, but keeping the goal the same.
%
%In our POMDP, we assume the robot state is known, and all uncertainty is over the user's goal. Observations in our POMDP correspond to user inputs $\actionuser \in \Actionuser$. Given a sequence of user inputs, we infer a distribution over system states (equivalently a distribution over goals) using an observation model $\pomdpohm$. This corresponds to computing $\policyusergoal(\stateenv)$ for each goal, and applying Bayes' rule. We provide details in \cref{sec:framework_prediction}.
%
%The system uses cost function $\costrobot: \Staterobgoal \times \Actionrobot \times \Actionuser \rightarrow \mathcal{R}$, corresponding to the cost of taking robot action $\actionrobot$ when in system state $\staterobgoal$ and the user has input $\actionuser$. Note that allowing the cost to depend on the observation $\actionuser$ is non-standard, but important for shared autonomy, as prior works suggest that users prefer maintaining control authority~\citep{kim_2012}. This formulation enables us to penalize robot actions which deviate from $\userinputtoaction(\actionuser)$. Our shared autonomy POMDP is defined by the tuple $\left(\Staterobgoal, \Actionrobot, \transition, \costrobot, \Actionuser, \pomdpohm \right)$. The optimal solution to this POMDP minimizes the expected accumulated cost $\costrobot$. As this is intractable to compute, we utilize Hindsight Optimization to select actions, described in %\cref{sec:hindsight}.





\subsection{Hindsight Optimization}
\label{sec:framework_hindsight}

Computing the optimal solution for a POMDP with continuous states and actions is generally intractable. Instead, we approximate this quantity through \emph{Hindsight Optimization}~\citep{chong_2000,yoon_2008}, or QMDP~\citep{littman_1995}. This approximation estimates the value function by switching the order of the min and expectation in \cref{eq:v_belief}:
\begin{align*}
  \vhs(\belief) &= \expctover{\belief}{\min_{\policyrobot} \vrobot^{\policyrobot}(\state)}\\
  &= \expctover{\goal}{\vgoal(\stateenv)}\\
  \qhs(\beliefactions) &= \expctover{\belief}{\costrobot(\stateactions) + \expctover{\state'}{\vhs(\state')}}\\
  &= \expctover{\goal}{\qgoal(\stateenvactions)}
\end{align*}

Where we explicitly take the expectation over $\goal \in \Goal$, as we assume that is the only uncertain part of the state.

Conceptually, this approximation corresponds to assuming that all uncertainty will be resolved at the next timestep, and computing the optimal cost-to-go. As this is the best case scenario for our uncertainty, this is a lower bound of the cost-to-go, $\vhs(\belief) \leq \vopt(\belief)$. Hindsight optimization has demonstrated effectiveness in other domains~\citep{yoon_2007, yoon_2008}. However, as it assumes uncertainty will be resolved, it never explicitly gathers information~\citep{littman_1995}, and thus performs poorly when this is necessary.

We believe this method is suitable for shared autonomy for many reasons. Conceptually, we assume the user provides inputs at all times, and therefore we gain information without explicit information gathering. Works in other domains with similar properties have shown that this approximation performs comparably to methods that consider explicit information gathering~\citep{koval_2014}. Computationally, computing $\qhs$ can be done with continuous state and action spaces, enabling fast reaction to user inputs. 
%say that this is a lower bound on cost-to-go?

%Let $\qpomdp(\belief, \actionrobot, \actionuser)$ be the action-value function of the POMDP, estimating the cost-to-go of taking action $\actionrobot$ when in belief $\belief$ with user input $\actionuser$, and acting optimally thereafter. In our setting, uncertainty is only over goals, $\belief(\staterobgoal) = \belief(\goal) = p(\goal | \trajtot)$.

%Let $\qgoal(\staterobot, \actionrobot, \actionuser)$ correspond to the action-value for goal $\goal$, estimating the cost-to-go of taking action $\actionrobot$ when in state $\staterobot$ with user input $\actionuser$, and acting optimally for goal $\goal$ thereafter. The QMDP approximation is~\citep{littman_1995}:
%\begin{align*}
%  \qpomdp(\belief, \actionrobot, \actionuser) &= \sum_{\goal} \belief(\goal) \qgoal(\staterobot, \actionrobot, \actionuser)
%\end{align*}

Computing $\qgoal$ for shared autonomy requires utilizing the user policy $\policyusergoal$, which can make computation difficult. This can be alleviated with the following approximations:
\subsubsection*{Stochastic user with robot}
Estimate $\actionuser$ using $\policyusergoal$ at each time step, e.g. by sampling, and utilize the full cost function $\costrobotgoal(\stateenvactions)$ and transition function $\transitionallargs$ to compute $\qgoal$. This would be the standard QMDP approach for our POMDP.

\subsubsection*{Deterministic user with robot}
Estimate $\actionuser$ as the most likely $\actionuser$ from $\policyusergoal$ at each time step, and utilize the full cost function $\costrobotgoal(\stateenvactions)$ and transition function $\transitionallargs$ to compute $\qgoal$. This uses our policy predictor, as above, but does so deterministically, and is thus more computationally efficient.

\subsubsection*{Robot takes over}
Assume the user will stop supplying inputs, and the robot will complete the task. This enables us to use the cost function $\costrobotgoal(\stateenv, 0, \actionrobot)$ and transition function $\transition(\stateenv' \given \stateenv, 0, \actionrobot)$ to compute $\qgoal$. For many cost functions, we can analytically compute this value, e.g. cost of always moving towards the goal at some velocity. An additional benefit of this method is that it makes no assumptions about the user policy $\policyusergoal$, making it more robust to modelling errors. We use this method in our experiments.

Finally, as we often cannot calculate $\argmax_{\actionrobot} \qhs(\beliefactions)$ directly, we use a first-order approximation, which leads to us to following the gradient of $\qhs(\beliefactions)$.
%In cases were an action exists to assist for all goals, this approximation will take that action. When there aren't any such actions, the output will look similar to a blending between the user control and our assistance strategy, solving for the parameters of blending based on the cost functions. This sort of blending has been shown to be effective in the past~\citep{dragan_2013_assistive}. See \figref{fig:teledata}.


%add something about 1st order approximation for continuous systems?

%Maybe more specifics for our system? 
%-First order approx for qmdp
%-we optimize directly for user's value function
%---actually, we aren't fully solving the POMDP assuming user is optimal

\subsection{User Prediction}
\label{sec:framework_prediction}

In order to infer the user's goal, we rely on a model $\policyusergoal$ to provide the distribution of user actions at state $\stateenv$ for user goal $\goal$. In principle, we could use any generative predictor for this model, e.g.~\citep{koppula_2013, wang_2013_intentioninference}. We choose to use maximum entropy inverse optimal control (MaxEnt IOC)~\citep{ziebart_2008}, as it explicitly models a user cost function $\costusergoal$. We optimize this directly by defining $\costrobotgoal$ as a function of $\costusergoal$.

In this work, we assume the user does not model robot actions. We use this assumption to define an MDP with states $\stateenv \in \Stateenv$ and user actions $\actionuser \in \Actionuser$ as before, transition $\transitionuser(\stateenv' \given \stateenv, \actionuser) = \transition(\stateenv' \given \stateenv, \actionuser, 0)$, and cost $\costusergoal(\stateenv, \actionuser)$. MaxEnt IOC computes a stochastically optimal policy for this MDP.

The distribution of actions at a single state are computed based on how optimal that action is for minimizing cost over a horizon $T$. Define a sequence of environment states and user inputs as $\traj = \left\{ \stateenv_0, \actionuser_0, \cdots, \stateenv_T, \actionuser_T \right\}$. Note that sequences are not required to be trajectories, in that $\stateenv_{t+1}$ is not necessarily the result of applying $\actionuser_t$ in state $\stateenv_t$. Define the cost of a sequence as the sum of costs of all state-input pairs, $\costgoaluser(\traj) = \sum_{t} \costgoaluser(\stateenv_t, \actionuser_t)$. Let $\trajtot$ be a sequence from time $0$ to $t$, and $\trajat{\stateenv}$ a sequence of from time $t$ to $T$, starting at $\stateenv$.

\citet{ziebart_thesis} shows that minimizing the worst-case predictive loss results in a model where the probability of a sequence decreases exponentially with cost, $p(\traj \given \goal) \propto \exp(-\costgoaluser(\traj))$. Importantly, one can efficiently learn a cost function consistent with this model from demonstrations~\citep{ziebart_2008}.

Computationally, the difficulty in computing $p(\traj \given \goal)$ lies in the normalizing constant $\int_{\traj} \exp(-\costgoaluser(\traj))$, known as the partition function. Evaluating this explicitly would require enumerating all sequences and calculating their cost. However, as the cost of a sequence is the sum of costs of all state-action pairs, dynamic programming can be utilized to compute this through soft-minimum value iteration when the state is discrete~\citep{ziebart_2009,ziebart_2012}:
\begin{align*}
  \qgoalsoftt{t}(\stateenv, \actionuser) &= \costgoaluser(\stateenv, \actionuser) + \expctarg{\vgoalsoftt{t+1}(\stateenv')}\\
  \vgoalsoftt{t}(\stateenv) &= \softmin_{\actionuser} \qgoalsoftt{t}(\stateenv, \actionuser)
\end{align*}
Where $\softmin_{x} f(x) = - \log \int_{x} \exp(-f(x)) dx$ and $\stateenv' \sim \transitionuser(\cdot \given \stateenv, \actionuser)$.

The log partition function is given by the soft value function, $\vgoalsoftt{t}(\stateenv) = - \log \int_{\trajat{\stateenv}} \exp\left(-\costgoaluser(\trajat{\stateenv})\right)$, where the integral is over all sequences starting at $\stateenv$ and time $t$. Furthermore, the probability of a single input at a given environment state is given by $\policyuser_t(\actionuser \given \stateenv, \goal) = \exp(\vgoalsoftt{t}(\stateenv) -\qgoalsoftt{t}(\stateenv, \actionuser))$~\citep{ziebart_2009}.

%make more clear that while our user policy doesn't consider robot assistance, it still affects this positive feedback thing
Many works derive a simplification that enables them to only look at the start and current states, ignoring the inputs in between~\citep{ziebart_2012, dragan_2013_assistive}. Key to this assumption is that $\traj$ corresponds to a trajectory, where applying action $\actionuser_t$ at $\stateenv_t$ results in $\stateenv_{t+1}$. However, if the system is providing assistance, this may not be the case. In particular, if the assistance strategy believes the user's goal is $\goal$, the assistance strategy will select actions to minimize $\costusergoal$. Applying these simplifications will result positive feedback, where the robot makes itself more confident about goals it already believes are likely. In order to avoid this, we ensure that the prediction comes from user inputs only, and not robot actions:
\begin{align*}
  p(\traj \given \goal) &= \prod_t \policyuser_t(\actionuser_{t} \given \stateenv_t, \goal)
\end{align*}
%Where the user applied input $\actionuser_t$ at state $\state_t$.
To compute the probability of a goal given the partial sequence up to $t$, we apply Bayes' rule:
\begin{align*}
  p(\goal \given \trajtot) &= \frac{p(\trajtot \given \goal) p(\goal) }{\sum_{\goal'} p(\trajtot \given \goal') p(\goal')}
\end{align*}
This corresponds to our POMDP observation model, used to transition our belief over goals through $\transitionbelief$.


\subsubsection{Continuous state and action approximation}
Soft-minimum value iteration is able to find the exact partition function when states and actions are discrete. However, it is computationally intractable to apply in continuous state and action spaces. Instead, we follow \citet{dragan_2013_assistive} and use a second order approximation about the optimal trajectory. They show that, assuming a constant Hessian, we can replace the difficult to compute soft-min functions $\vgoalsoft$ and $\qgoalsoft$ with the min value and action-value functions $\vgoaluser$ and $\qgoaluser$:
\begin{align*}
  \policyuser_t(\actionuser \given \stateenv, \goal) &= \exp(\vgoaluser(\stateenv) -\qgoaluser(\stateenv, \actionuser))
\end{align*}
Recent works have explored extensions of the MaxEnt IOC model for continuous spaces~\citep{boularias_2011, levine_2012, finn_2016}. We leave experiments using these methods for learning and prediction as future work.


\subsection{Multi-Target MDP}
\label{sec:framework_multitarget}

There are often multiple ways to achieve a goal. We refer to each of these ways as a \emph{target}. For a single goal (e.g. object to grasp), let the set of targets (e.g. grasp poses) be $\target \in \Target$. We assume each target has a cost function $\costtarg$, from which we compute the corresponding value and action-value functions $\vtarg$ and $\qtarg$, and soft-value functions $\vtargsoft$ and $\qtargsoft$. We derive the quantities for goals, $\vgoal, \qgoal, \vgoalsoft, \qgoalsoft$, as functions of these target functions.

We state the theorems below, and provide proofs in the appendix (\cref{sec:mingoal_thms}).

\subsubsection{Multi-Target Assistance}
\label{sec:framework_multigarget_assistance}
We assign the cost of a state-action pair to be the cost for the target with the minimum cost-to-go after this state:
\begin{align}
  \costgoal(\stateenvactions) &= \costtargstar(\stateenvactions) \quad \target* = \argmin_\target \vtarg(\stateenv') \label{eq:goal_target_cost}
\end{align}
Where $\stateenv'$ is the environment state after actions $\actionuser$ and $\actionrobot$ are applied at state $\stateenv$. For the following theorem, we require that our user policy be deterministic, which we already assume in our approximations when computing robot actions in \cref{sec:framework_hindsight}.
\begin{restatable}{theorem}{valfundecompose}
\label{thm:mingoal_assist}
Let $\vtarg$ be the value function for target $\target$. Define the cost for the goal as in \cref{eq:goal_target_cost}. For an MDP with deterministic transitions, and a deterministic user policy $\policyuser$, the value and action-value functions $\vgoal$ and $\qgoal$ can be computed as:
\begin{align*}
  \qgoal(\stateenvactions) &= \qtargstar(\stateenvactions) \qquad \target^* = \argmin_\target \vtarg(\stateenv') \\
  \vgoal(\stateenv) &= \min_\target \vtarg(\stateenv)
\end{align*}
\end{restatable}

\subsubsection{Multi-Target Prediction}
\label{sec:framework_multigarget_prediction}
Here, we don't assign the goal cost to be the cost of a single target $\costtarg$, but instead use a distribution over targets.%based on the cost-to-go.
\begin{restatable}{theorem}{softvalfundecompose}
  \label{thm:mingoal_pred}
  Define the probability of a trajectory and target as $p(\traj, \target) \propto \exp(-\costtarg(\traj))$. Let $\vtargsoft$ and $\qtargsoft$ be the soft-value functions for target $\target$. For an MDP with deterministic transitions, the soft value functions for goal $\goal$, $\vgoalsoft$ and $\qgoalsoft$, can be computed as:
\begin{align*}
  \vgoalsoft(\stateenv) &= \softmin_\target \vtargsoft(\stateenv)\\
  \qgoalsoft(\stateenv, \actionuser) &= \softmin_\target \qtargsoft(\stateenv, \actionuser)
\end{align*}
\end{restatable}

%
%Marginalizing out g:
%\begin{align*}
%  p(a_t | s) &= \sum_g p(a_t, g | s)\\
%  &= \frac{ \sum_g \exp(-Q_g^{t}(s_t, a_t))} {\sum_{g'}\exp(-V_{g'}^{t}(s_{t}))}
%\end{align*}
%
%We can also write this out as:
%\begin{align*}
%  \exp\left( \log\left( p(a_t | s) \right) \right)&= \exp\left( \log\left(  \frac{ \sum_g \exp(-Q_g^{t}(s_t, a_t))} {\sum_{g'}\exp(-V_{g'}^{t}(s_{t}))}\right) \right)\\
%  &= \exp\left( \log\left(  \sum_g \exp(-Q_g^{t}(s_t, a_t)) \right) - \log\left(\sum_{g'}\exp(-V_{g'}^{t}(s_{t})) \right) \right)\\
%  &= \exp\left( \softmin_g V_{g}^{t}(s_{t}) - \softmin_g Q_g^{t}(s_t, a_t)\right)
%\end{align*}
%

\begin{figure}[t]
\centering
 \begin{subfigure}{0.24\textwidth}
   \centering 
   \includegraphics[width=0.97\textwidth, trim=440 250 500 210, clip=true]{rss_multigoal_1.png}
  \caption{}
 \label{fig:multigoal_1}
 \end{subfigure}
 \begin{subfigure}{0.24\textwidth}
   \centering 
   \includegraphics[width=0.97\textwidth, trim=440 250 500 210, clip=true]{rss_multigoal_2.png}
  \caption{}
 \label{fig:multigoal_2}
 \end{subfigure}
 \begin{subfigure}{0.24\textwidth}
   \centering 
   \includegraphics[width=0.97\textwidth, trim=440 250 500 210, clip=true]{rss_multigoal_3_arb.png}
  \caption{}
 \label{fig:multigoal_3_arb}
 \end{subfigure}
 \begin{subfigure}{0.24\textwidth}
   \centering 
   \includegraphics[width=0.97\textwidth, trim=440 250 500 210, clip=true]{rss_multigoal_3_pred.png}
  \caption{}
 \label{fig:multigoal_3_pred}
 \end{subfigure}
 \caption{Value function for a goal (grasp the ball) decomposed into value functions of targets (grasp poses). (\subref{fig:multigoal_1}, \subref{fig:multigoal_2}) Two targets and their corresponding value function $\vtarg$. In this example, there are 16 targets for the goal. (\subref{fig:multigoal_3_arb}) The value function of a goal $\vgoal$ used for assistance, corresponding to the minimum of all 16 target value functions (\subref{fig:multigoal_3_pred}) The soft-min value function $\vgoalsoft$ used for prediction, corresponding to the soft-min of all 16 target value functions.}
 \label{fig:multigoal}
\end{figure}





To study how different platform mechanisms affect users' decisions in information disclosure, we first need to incorporate the users' data disclosure decisions into the recommendation process.
Thus, we propose a privacy aware recommendation framework where users can freely choose which data to disclose with the recommender system.
As illustrated in \cref{fig:framework}, the critical difference between our framework and traditional recommendation is that the platform can only use the sub-data disclosed by the users. 
For example, the user on the left in \cref{fig:privacy_rec} can choose to hide his sensitive demographic attributes (e.g., age, gender, and education) and only discloses the last behavior to the service provider. 


To enjoy the benefits of personalized services, users need to disclose their data to the recommender system to better model them.
Intuitively, more data the recommender system gets, better results the users can get.
However, disclosing data to the platform will increase users' privacy concerns, e.g., data abusing~\cite{Mayer:sp12:Third} and privacy leakage~\cite{zhang2021membership}.
Thus, under the privacy aware setting, users need to make information disclosure decisions based on the trade-off between anticipated privacy risks and potential utilities.
This idea can date back to \textit{Privacy Calculus Theory}~\cite{Laufer:si77:Privacy,Culnan:os99:Information}.

Before going into details, we first define the entire personal data $\di{}$ of user $i \in \mathcal{V}$'s as:
\begin{equation}
\di{} = \{{\scriptstyle \mathcal{D}_{i,a}, \mathcal{D}_{i,b}} \}= \{\{a_{i1},\dots, a_{iK}\}, \{b_{i1},\dots, b_{it_i}\}\},
\label{eq:di}
\end{equation}
where ${\scriptstyle \mathcal{D}_{i,a}}{=}\{a_{i1},\dots, a_{iK}\}$ denotes user $i$'s all profile attributes, $a_{ik}$ denotes the $k$-th profile attribute for user $i$, and $K$ is the number of profile attributes. ${\scriptstyle \mathcal{D}_{i,b}}{=}\{b_{i1},\dots, b_{it_i}\}$ denotes user $i$'s behaviors, $b_{ij}$ is the $j$-th behavior of user $i$, and $t_i$ is the last behavior timestamp. \czq{$\mathcal{V}$ is the set that includes all users in the platform.} %

A rational user is only willing to disclose data when she feels that she gains more from the platform than she loses in data disclosure.
Formally speaking, supposing user $i$ with whole data $\di{}$ currently discloses data $\si{} \subset \di{}$, now she tries to get a better recommendation results via disclosing more data $\si{'} \subset \di{}$ where $|\si{'}| > |\si{}|$, only if
\begin{equation}
    \texttt{U}_i(\si{'}) - \texttt{U}_i(\si{}) > \lambda_i\bigl(\texttt{C}_i( \si{'}) -\texttt{C}_i( \si{})\bigr),
\label{eq:base}
\end{equation}
where $\texttt{U}_i(x)$ denotes the utility that user $i$ can get from the platform with disclosed data $x$, function $\texttt{C}_i(x)$ measures the privacy cost paid by the user $i$ when she discloses the data $x$ to the platform, and $\lambda_i$ is the sensitive weight measuring how much user $i$ cares about her privacy.
Apparently, compared to privacy insensitive users (i.e., small $\lambda_i$), the platform needs to provide more performance improvements to attract privacy sensitive users (i.e., large $\lambda_i$) to disclose their data.
More details can be found in \cref{sec:user_type}.


\subsubsection{\textbf{User Objective.}}
Unlike traditional task settings where users can only passively accept recommendation results (i.e., without tools to optimize their objectives), in our framework, a rational user $i$ tends to maximize her utility $\texttt{U}_i(\si{})$ while minimize the privacy risk $\texttt{C}_i(\si{})$ by control the disclosed data $\si{}$.
The objective function for a specific user $i$ can be formalized as the following:
\begin{equation}
    \texttt{R}_i(\si{}) = -\lambda_i \texttt{C}_i(\si{}) + \texttt{U}_i( \si{}).
\label{eq:framework}
\end{equation}

The linear combination for user objective function follows the initial idea from \textit{Privacy Calculus Theory}~\cite{Laufer:si77:Privacy,Culnan:os99:Information} where both the recommendation performances from the platform and potential privacy cost are considered.
This formulation is also compatible with privacy related research in economics~\cite{farrell2012can,jin2017protecting,lin2019valuing}. %
They studied the micro-foundation on a user's intrinsic and instrumental preferences from disclosing personal information.
In our formulation, user's privacy cost $\texttt{C}_i(\si{})$ corresponds to intrinsic value for personal data (i.e., protecting the data from being obtained by others), while recommendation utility $\texttt{U}_i(\si{})$ corresponds to the instrumental value for personal data.


\subsubsection{\textbf{Platform Objective.}}
\label{sec:platform_obj}
In the proposed framework, the goal of a platform is still to maximize its revenue (e.g., purchases, clicks, or watching time) by improving the users' recommendation utility (e.g., providing more accurate results).
Thus, we define its objective as the summation of all users' recommendation utilities in \cref{eq:framework}\czq{, where $\mathcal{V}$ denotes all users in the platform}:
\begin{equation}
    \texttt{R}_{\text{p}} = \sum_{i \in {\scriptstyle \mathcal{V}}} \texttt{U}_i(\si{}).
\label{eq:platform}
\end{equation}
Considering the utility also depends on the recommendation model, the utility function $\texttt{U}_i(x)$ can be further defined as:
\begin{equation}
    \texttt{U}_i(\si{}) = \texttt{U}(\si{}) = \texttt{U}'(\si{},\, \texttt{M}_{{\scriptscriptstyle \mathcal{S}}}),
    \label{eq:updated_rec}
\end{equation}
where $\texttt{M}_{{\scriptscriptstyle \mathcal{S}}}: \si{} \rightarrow l_i$ ($l_i$ is recommendation results) is a recommendation model trained using all users' disclosed data ${\scriptstyle \mathcal{S}}=\{ {\scriptstyle \mathcal{S}_1},\dots, {\scriptstyle \mathcal{S}_{|\mathcal{V}|}}\}$ and $\texttt{U}'$ represents detailed recommendation utility function.
Here, without loss of generality, we assume that all users share the same utility function.
We will explore the personalized utility function in the future work.


\subsubsection{\textbf{Recommendation Utility Function}}
As shown in \cref{eq:framework} and \cref{eq:platform}, the recommendation utility $\texttt{U}$  occurs in the objective functions of both end users and the platform.
Here, we use the users' satisfaction with the results produced by the recommendation model to measure its utility.
It is worth noting that user satisfaction is still an open problem in \czq{recommender systems}.
\czq{Here, we simply quantify it by the user’s interactions} with the recommendation results, e.g., clicks, watches, and reads.
Based on such feedbacks, we can calculate different quantitative metrics as the utility in our framework, e.g., hit ratio and normalized discounted cumulative gain (NDCG)~\cite{ndcg}.
In this paper, we choose NDCG as the utility function $\texttt{U}$ for all users because of its widespread use \cite{NCF,kang2018self,Sun:cikm19:BERT4Rec}.


In traditional recommendation task, the platform can optimize this objective by only optimizing the model $\texttt{M}_{{\scriptscriptstyle \mathcal{S}}}$ since $\si{}= \di{}$ is a fixed, i.e., all users disclose their whole data.
However, this premise is broken in our proposed framework, where the user's disclosed data $\si{}$ is varying.
Thus, in our new framework, platforms also seek to attract users to share more data in other ways besides optimizing models, such as platform mechanism design.



\subsection{Platform Mechanism}
\label{sec:platform}

As mentioned before, the disclosed data $\si{}$ lives at the heart of the framework.
Ideally, user $i$ can freely choose any data $\si{}$ to disclose with the platform, e.g., choosing any profile attribute $a$ or behavior data $b$ as shown in \cref{fig:privacy_rec}.
However, in practice, such a degree of freedom is difficult to achieve for two reasons.
On the one hand, from the perspective of human-computer interaction, too fine granularity of disclosure choice (e.g., single behavior) can adversely hurt user experience~\cite{Zhang:hcs19:Proactive}.
On the other hand, although the privacy regulations ensure users the right to determine the use of their data, they do not stipulate how the service providers implement this function.


In practice, the platform usually designs some data disclosure mechanisms to provide the end users with several convenient options.
Here, we formulate the platform mechanism $\mathrm{G}=<\delta, \Pi>$ using two components, data split rule $\delta$ and disclosure choice spaces $\Pi$.
The data split rule $\delta$ is regarded as a function that reorganizes the original user data $\di{}$ using different granularity, and $\Pi$ denotes the space of all possible choices the platform provides to the user.
We illustrate a toy example in \cref{fig:mechanism}.

\subsubsection{\textbf{Data Split Rule}}
Since user data usually consists of two different data types (as in \cref{eq:di}), we defined $\delta$ as:
\begin{equation*}
    \delta (x) = \{\delta_a({\scriptstyle \mathcal{D}_{i,a}}), \delta_b({\scriptstyle \mathcal{D}_{i,b}}) \},
\end{equation*}
where $\delta_a$ and $\delta_b$ have similar forms that split the original data into several pieces according to the corresponding granularity and rules:
\begin{equation*}
    \{x_1, x_2,\dots,x_n\} \xrightarrow[\delta_a]{\delta_b}  \{x_1', x_2',\dots,x_m'\}, \,\, %
\end{equation*}
where $m \leq n$ and $x_j'$ is the candidate units for data disclosure. %
According to the segmentation rules, $x_j'$ can be several consecutive data points like $\{x_1, x_2, x_3\}$ or discontinuous random data like $\{x_5, x_{22}\}$.

$\delta_a$ aims to reorganize the user’s profile attributes.
The common approach is to keep original granularity (i.e., user can freely disclose any subset of attributes) or take all attributes as a whole (i.e., disclose all attributes or not).
Formally, it can be instantiated as:
\begin{equation*}
    \begin{aligned}
     \delta_a ({\scriptstyle \mathcal{D}_{i,a}}) &= \{a_{i1},\dots, a_{iK}\}\\
    \text{or}\quad  \delta_a ({\scriptstyle \mathcal{D}_{i,a}}) &= \{\{a_{i1},\dots, a_{iK}\}\}.
    \end{aligned}
\end{equation*}

Similarly, $\delta_b$ aims to transfer a user's original behavior data (e.g., thousands of clicks or more views) to few data disclosure options.
For example, ``percentage split'' with 10\% granularity divides a user's behavior sequence into 10 equal length subsequences, while ``daily split'' divides the user's behaviors by day.
Take ``percentage split'' with 10\% granularity as an example, it can be instantiated as: 
\begin{equation*}
    \delta_b ({\scriptstyle \mathcal{D}_{i,b}}) = \{{\scriptstyle \mathcal{S}_{i,b1}}, {\scriptstyle \mathcal{S}_{i,b2}}, \dots, {\scriptstyle \mathcal{S}_{i,b10}} \},
\end{equation*}
where ${\scriptstyle \mathcal{S}_{i,bj}} = \{b_{i,\lfloor 0.1 t_i*(j-1)\rfloor+1}, \dots, b_{i, \lfloor 0.1 t_i*j\rfloor}\}$ is the $j$-th candidate option of behavior data for user to disclosed.

\subsubsection{\textbf{Data Disclosure Choice Space $\Pi$}}

Assuming the platform has transferred user $i$'s original data $\di{}$ to $\delta(\di{}) = \{{\scriptstyle \mathcal{S}_{i,a1}},\dots,{\scriptstyle \mathcal{S}_{i,an}},$ $ {\scriptstyle \mathcal{S}_{i,b1}}, \dots,{\scriptstyle \mathcal{S}_{i,bm}}\}$, the platform can define data disclosure choice space $\Pi$ on these $m+n$ candidates as:
\begin{equation*}
    \begin{aligned}
    \Pi &= \{\Pi_1,\Pi_2, \dots, \Pi_N\}, \\
    \Pi_j & \sim [o_1, \cdots, o_k, \cdots, o_{n+m}], \quad o_k \in \{0,1\},
    \end{aligned}
\end{equation*}
where $o_k = 1$ denotes disclosing the $k$-th data in $\delta(\di{})$, while $o_k = 0$ means not; $\Pi_j$ is $j-$th data disclosure option that users can take; $N$ is the number of possible choices the platform provides to users.
For example, a full 0 vector $\Pi_j = [0, 0, \dots, 0]$ denotes that users can choose it to do not disclose any data.
More detailed instantiations can be found in \cref{sec:plat_mech}.


\tikzset{
  FARROW/.style={arrows={-{Latex[length=1.25mm, width=1.mm]}}},
  DFARROW/.style={arrows={{Latex[length=1.25mm, width=1.mm]}-{Latex[length=1.25mm, width=1.mm]}}},
  behavior/.style = {circle, fill=monte_carlo, minimum width=1.2em, align=center, inner sep=0, outer sep=0, font=\tiny},
  feature/.style = {circle, fill=salmon, minimum width=1.2em, align=center, inner sep=0, outer sep=0, font=\tiny},
  encoder/.style = {rectangle, fill=Madang!82, minimum width=6em, minimum height=3em, align=center, rounded corners=3},
  emb_layer/.style = {rectangle, fill=languid_lavender!72, minimum width=11em, minimum height=2em, align=center, rounded corners=3},
  project/.style = {rectangle, fill=hint_green, minimum width=7em, minimum height=2.4em, align=center, rounded corners=2},
  ds/.style={
       rectangle split,
       rectangle split part align=base,
       rectangle split horizontal=true,
       rectangle split draw splits=true,
       rectangle split parts=5,
       rectangle split part fill={athens_gray!80, athens_gray!80, athens_gray!80, athens_gray!80, athens_gray!80},
       draw=black, %
       very thin,
       minimum height=1.2em,
       minimum width=2em,
       text width=0.4em,
       inner sep=.5pt,
       text centered,
       font=\tiny,
       text=gray,
       },
}

\begin{figure}
\centering
\resizebox{0.75\linewidth}{!}{
    \begin{tikzpicture}
        
    \node[] (mo)  at (0, 0) {};
    
    \node [feature, below of=mo, node distance=0cm, xshift=-0.9cm] (f1) {$a_{i1}$};
    \node [feature, right of=f1, node distance=0.5cm] (f2) {$a_{i2}$};
    \node [feature, right of=f2, node distance=0.5cm, ] (f3) {$a_{i3}$};
    \node [behavior, right of=f3, node distance=0.8cm, ] (b1) {$b_{i1}$};
    \node [behavior, right of=b1, node distance=0.5cm] (b2) {$b_{i2}$};
    \node [behavior, right of=b2, node distance=0.5cm] (b3) {$b_{i3}$};
    \node [behavior, right of=b3, node distance=0.5cm] (b4) {$b_{i4}$};
    
    \foreach \x in {1,2,3}
    { 
     \node [feature, below of=f\x, node distance=1.2cm ] (nf\x) {$a_{i\x}$};
     \node [behavior, below of=b\x, node distance=1.2cm ] (nb\x) {$b_{i\x}$};
    }
    \node [behavior, below of=b4, node distance=1.2cm ] (nb4) {$b_{i4}$};
    
    \foreach \x in {1,2}
    {
    \draw[densely dotted, line width= 0.7pt] ([xshift=2.5mm] nf\x.north) -> ([xshift=2.5mm]  nf\x.south);
    }
    \draw[densely dotted, line width= 0.7pt] ([xshift=2.5mm] nb2.north) -> ([xshift=2.5mm]  nb2.south);
    
    \draw [decorate, decoration={brace, amplitude=4pt, mirror}] ([xshift=-0.3mm, yshift=-0.3mm] nb1.south west) -- ([xshift=0.3mm, yshift=-0.3mm] nb2.south east) node[midway,yshift=-3mm, font=\tiny] (sb1) {$\scaleto{\mathcal{S}_{i,b1}}{5pt}$};
    \draw [decorate, decoration={brace, amplitude=4pt, mirror}] ([xshift=-0.3mm, yshift=-0.3mm] nb3.south west) -- ([xshift=0.3mm, yshift=-0.3mm] nb4.south east) node[midway,yshift=-3mm, font=\tiny] (sb2) {$\scaleto{\mathcal{S}_{i,b2}}{5pt}$};
    
    \foreach \x in {1,2,3}
    {
    \draw [decorate, decoration={brace, amplitude=4pt, mirror}] ([xshift=-0.3mm, yshift=-0.3mm] nf\x.south west) -- ([xshift=0.3mm, yshift=-0.3mm] nf\x.south east) node[midway,yshift=-3mm, font=\tiny] (sa1) {$\scaleto{\mathcal{S}_{i,a\x}}{5pt}$};
    }
    
    
    \node[densely dotted, draw=black, fit={(f1) (b4) }, inner sep=3, rounded corners=2] (d_box) {};
    \node[left = 0.5mm of d_box, font=\tiny] (di) {$\di{}$};


    \node[densely dotted, draw=black, fit={(nf1) ([xshift=-9mm, yshift=1mm] sa1.south west) (nb4) ([xshift=1.6mm, yshift=1mm] sb2.south east) }, inner sep=2, rounded corners=2] (nd_box) {};
    \node[left = 0.5mm of nd_box, font=\tiny] (di) {$\delta (\di{})$};
    
    \draw[FARROW] (d_box)-> (nd_box) node [pos=0.5, right, black, font=\tiny, align=left] {data\\split}; 
    
    \node[below = -2.5mm of nd_box, font=\tiny, xshift=3cm] (pi1) {$[0,0,0,0,0]$};
    \node[below of = pi1, font=\tiny, node distance =2mm] (pi2) {$\vdots$};
    \node[below of = pi2,  node distance =4mm, font=\tiny] (pi3) {$[1,1,1,0,0]$};
    \node[below of = pi3,  node distance =2.5mm, font=\tiny, text=flamingo] (pi4) {$[0,0,0,1,0]$};
    \node[below of = pi4, font=\tiny, node distance =2mm] (pi5) {$\vdots$};
    \node[below of = pi5,  node distance =4mm, font=\tiny] (pi6) {$[1,1,1,1,1]$};
    
    \node[dotted, draw=black, fit={(pi1) (pi6)}, inner sep=0.5, rounded corners=2, fill opacity=0.3, fill=gallery] (pi) {};
    \node[above = -0.8mm of pi, font=\tiny, xshift=0mm] (pi_l) {choice space $\Pi$};
    
    \node[maninblack, mirrored, minimum size=.2mm, right of=pi4, node distance=12mm] (ui) {};
    \draw[-latex, thin, flamingo] ([xshift=.4mm] ui.west)-> ([xshift=-1.5mm] pi4.east);
    \node [cloud, cloud puffs=7.2, cloud ignores aspect, draw=flamingo, fill=flamingo!42, minimum width=1mm, align=center, inner sep=0pt, above of=ui, node distance=0.75cm] (pii) {$\scaleto{\alpha_i=\Pi_j}{3.5pt}$};
    
    \node [ellipse, scale=0.5, draw=flamingo, fill=flamingo!42, minimum height=1.5mm, minimum width=5mm, above of=ui, node distance=0.95cm] () {};
    \node [ellipse, scale=0.3, draw=flamingo, fill=flamingo!42, minimum height=2mm, minimum width=4mm, above of=ui, node distance=1.05cm] {};
    
    
    \node [behavior, below of=nb4, node distance=2cm, xshift=-1.95cm] (sb1) {$b_{i1}$};
    \node [behavior, right of=sb1, node distance=0.6cm ] (sb2) {$b_{i2}$};
    
    \node[densely dotted, draw=black, fit={(sb2) (sb1)}, inner sep=2, rounded corners=2] (si) {};
    
    \draw[FARROW] (nd_box)-> (si) node [pos=0.4, right, black, font=\tiny, align=left] (pi_a) {$ \Pi_j \otimes \mathrm{\delta}(\di{}) $};
    
     \draw[thin, draw=flamingo] ([xshift=1mm] pi4.west) -- ++(-1, 0) -- ++(0, 0.23)  -- ++(-1.43, 0);
     
    \node[left of = si, font=\tiny, node distance=0.8cm] (si_l) {$\si{}$};

    
    
    
    \end{tikzpicture} }
    \caption{An illustrative example for platform mechanism. The platform firstly split the user's data $\di{}$ (three profile attributes and four behaviors) using the rule $\delta$ (keeping independent for attributes; percentage split with 50\% granularity for behaviors). Then it provides choices $\Pi$ for the user to choose to produce the final disclosed data $\si{}$.}
    \label{fig:mechanism}
\end{figure}



\subsubsection{\textbf{Platform Mechanism Design}}


With \czq{the} mechanism $\mathrm{G}=<\delta,\Pi>$, we can formally define the disclosed data $\si{}$ from user $i$.
Assuming user $i$'s original data $\di{}$ has been spited into candidates $\delta(\di{}) = \{{\scriptstyle \mathcal{S}_{i,1}}, \dots,{\scriptstyle \mathcal{S}_{i,m}}\}$\footnote{Here, we simplify the subscripts for easy description}. 
Then, $\si{}$ can be defined as the union of candidates in $\delta(\di{})$ selected  by a specific choice $ \alpha_i = \Pi_j \in \Pi$:
\begin{equation}
    \si{} = \alpha_i \otimes \mathrm{\delta}(\di{}) = \Bigl\{\scaleto{\bigcup_{\begin{subarray}{l}\,\,\, o_k=1\\\,\,\,  o_k \in \Pi_j\end{subarray}}}{20pt} {\scriptstyle \mathcal{S}_{i,k}} \Bigr\},
\label{eq:S_i}
\end{equation}
where $\delta$ is the platform data split rule and %
action $\alpha_i$ is sampled from user $i$'s privacy disclosure policy $\pi_i$, which decides the data to be disclosed. 
The operator $\otimes$ denotes the aggregation of the selected spilt data based on his choice $\alpha_i$.
\cref{fig:mechanism} illustrates a tiny example of the data disclosure process (i.e., generation process of $\si{}$) of a user with three profile attributes and four behaviors.


With formal definition of $\si{}$, we can  re-write the platform utility $\texttt{R}_{\text{p}}$ in \cref{eq:platform} using \czq{the} platform mechanism $\mathrm{G}$ and model $\texttt{M}_{{\scriptscriptstyle \mathcal{S}}}$ as below,
\begin{align}
     \texttt{R}_{\text{p}} |_{\mathrm{G=<\delta,\Pi>}} {=}  \sum_{i \in \mathcal{V}} \texttt{U}_i'(\si{}, \texttt{M}_{{\scriptscriptstyle \mathcal{S}}}) {=}   \sum_{i \in \mathcal{V}} \texttt{U}_i'(\alpha_i \otimes \mathrm{\delta}(\di{}) , \texttt{M}_{{\scriptscriptstyle \mathcal{S}}}).%
    \label{eq:platform_mechanism_optimization}
\end{align} 

One may figure out some possible optimal solutions towards the platform's best mechanisms.
However, the optimal platform mechanism design is another complex topic, usually considered from the view of game theory, and is out of scopes of this work.
Here, we take the first step, studying the data disclosure decision of users and platform revenues under several common mechanisms.



\subsection{Relationship with Privacy Preservation}


In this subsection, we will discuss the relationship between our work and privacy preservation, and clarify the position of our paper.

First, as claimed in the introduction, our work does not aim to propose a new method to directly protect the users' privacy data from privacy attacks.
The main motivation of this paper is to study the effectiveness of existing privacy mechanisms deployed in the platforms and further explore how different privacy mechanisms affect users' privacy decisions.
For this purpose, our proposed framework is model-agnostic.
The model $\texttt{M}_{\scriptscriptstyle \mathcal{S}}$ used here can be an ordinary recommendation model (e.g., NCF ~\cite{NCF} or GRU4Rec~\cite{Hidasi:ICLR2016:gru4rec}) or privacy-preserving recommendation models~\cite{Chen:TIST2020:Practical}.
For deploying a privacy-preserving model in our framework, we only need to modify the privacy cost function ($\texttt{C}_i$), taking into account the influence of protecting privacy, i.e., less privacy cost than a normal model when disclosing the same amount of data.
In this paper, for the convenience of analysis and considering that privacy-preserving technologies are not widely used in real-world applications, we only studied the proposed framework in ordinary recommendation models.
We leave the exploration of the impact of privacy-preserving technologies on users' privacy cost functions for future work. 

Second, our proposed framework provides users with the power to proactively trade off their privacy cost and recommendation utility.
From the final result point of view, users can optimize their data disclosure decision to discard those data that are not helpful for their recommendation results.
In this way, the proposed framework gives users the tool to achieve data minimization~\cite{Mireshghallah:WWW20:Not,biega2020operationalizing} by themselves, rather than waiting for the platform to implement the data minimization algorithms.
From this perspective, our proposed framework can be seen as  implicitly protecting user privacy.
Even if a privacy attack occurs, only part of users' disclosed data will be leaked.
Besides, giving users control over the recommendation process has also been found be effective in reducing their privacy concerns~\cite{Zhang2014-oa,Chen:CHI18:This}.
