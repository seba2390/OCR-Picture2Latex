
\pdfoutput=1
\documentclass[acmsmall]{acmart}

% OLD PREAMBLE:

% \usepackage{jsen}
% \usepackage{cite}
% \usepackage{amsmath,amssymb,amsfonts, bbm, mathtools}
% \usepackage{algorithm,algorithmic}
% \usepackage{graphicx}
% \usepackage{textcomp}
% \usepackage{wrapfig}
% \usepackage{xfrac}
% \usepackage{stackengine}
% \usepackage{subfigure}
% \def\delequal{\mathrel{\ensurestackMath{\stackon[1pt]{=}{\scriptstyle\Delta}}}}



% \usepackage{color, soul}
% \newcommand{\hlt}[1]{\hl{#1}}
% \newcommand{\red}[1]{\textcolor{red}{#1}}

% \def\BibTeX{{\rm B\kern-.05em{\sc i\kern-.025em b}\kern-.08em
%     T\kern-.1667em\lower.7ex\hbox{E}\kern-.125emX}}
% \markboth{\journalname, VOL. XX, NO. XX, XXXX 2017}
% {Author \MakeLowercase{\textit{et al.}}: Preparation of Papers for IEEE TRANSACTIONS and JOURNALS (February 2017)}
% \definecolor{abstractbg}{rgb}{0.89804,0.94510,0.83137}
% \setlength{\fboxrule}{0pt}
% \setlength{\fboxsep}{0pt}

% NEW PREAMBLE:


\usepackage{amsmath,amsfonts,amssymb,bbm, amsthm, xfrac}
\usepackage{algorithmic}
\usepackage{algorithm}
\usepackage{array, multirow}
% \usepackage[caption=false,font=normalsize,labelfont=sf,textfont=sf]{subfig}
\usepackage{caption, subcaption}
\usepackage{textcomp}
\usepackage{stfloats}
\usepackage{url}
\usepackage{verbatim}
\usepackage{graphicx}
\usepackage{cite}
\usepackage{caption}
\usepackage{subcaption}
\hyphenation{}

\theoremstyle{plain}
\newtheorem{theorem}{Theorem}

\usepackage{color, soul}
\newcommand{\hlt}[1]{\hl{#1}}
\newcommand{\red}[1]{\textcolor{red}{#1}}

\usepackage{color}
\usepackage[dvipsnames]{xcolor}
\usepackage{tikz}

% black
\DeclareRobustCommand{\blackline}{\raisebox{2pt}{\tikz{\draw[-,black!40!Black,solid,line width = 0.9pt](0,0) -- (5mm,0);}}}
\DeclareRobustCommand{\blueline}{\raisebox{2pt}{\tikz{\draw[RoyalBlue,solid,line width = 0.9pt](0,0) -- (5mm,0);}}}
\DeclareRobustCommand{\orangeline}{\raisebox{2pt}{\tikz{\draw[BurntOrange,solid,line width = 0.9pt](0,0) -- (5mm,0);}}}
\DeclareRobustCommand{\greenline}{\raisebox{2pt}{\tikz{\draw[Green,solid,line width = 0.9pt](0,0) -- (5mm,0);}}}
\DeclareRobustCommand{\redline}{\raisebox{2pt}{\tikz{\draw[Red,solid,line width = 0.9pt](0,0) -- (5mm,0);}}}
\DeclareRobustCommand{\purpleline}{\raisebox{2pt}{\tikz{\draw[Purple,solid,line width = 0.9pt](0,0) -- (5mm,0);}}}
% dashed line
\DeclareRobustCommand{\blackdashedline}{\raisebox{2pt}{\tikz{\draw[-,black!40!black,dashed,line width = 0.9pt](0,0) -- (5mm,0);}}}
\DeclareRobustCommand{\bluedashedline}{\raisebox{2pt}{\tikz{\draw[RoyalBlue,dashed,line width = 0.9pt](0,0) -- (5mm,0);}}}
\DeclareRobustCommand{\orangedashedline}{\raisebox{2pt}{\tikz{\draw[BurntOrange,dashed,line width = 0.9pt](0,0) -- (5mm,0);}}}
\DeclareRobustCommand{\greendashedline}{\raisebox{2pt}{\tikz{\draw[Green,dashed,line width = 0.9pt](0,0) -- (5mm,0);}}}
\DeclareRobustCommand{\reddashedline}{\raisebox{2pt}{\tikz{\draw[Red,dashed,line width = 0.9pt](0,0) -- (5mm,0);}}}
\DeclareRobustCommand{\purpledashedline}{\raisebox{2pt}{\tikz{\draw[Purple,dashed,line width = 0.9pt](0,0) -- (5mm,0);}}}
\DeclareRobustCommand{\browndashedline}{\raisebox{2pt}{\tikz{\draw[Brown,dashed,line width = 0.9pt](0,0) -- (5mm,0);}}}
% dotted line
\DeclareRobustCommand{\blackdottedline}{\raisebox{2pt}{\tikz{\draw[-,black!40!black,dotted,line width = 0.9pt](0,0) -- (5mm,0);}}}
\DeclareRobustCommand{\bluedottedline}{\raisebox{2pt}{\tikz{\draw[RoyalBlue,dotted,line width = 0.9pt](0,0) -- (5mm,0);}}}
\DeclareRobustCommand{\orangedottedline}{\raisebox{2pt}{\tikz{\draw[BurntOrange,dotted,line width = 0.9pt](0,0) -- (5mm,0);}}}
\DeclareRobustCommand{\greendottedline}{\raisebox{2pt}{\tikz{\draw[Green,dotted,line width = 0.9pt](0,0) -- (5mm,0);}}}
\DeclareRobustCommand{\reddottedline}{\raisebox{2pt}{\tikz{\draw[Red,dotted,line width = 0.9pt](0,0) -- (5mm,0);}}}
\DeclareRobustCommand{\purpledottedline}{\raisebox{2pt}{\tikz{\draw[Purple,dotted,line width = 0.9pt](0,0) -- (5mm,0);}}}

% point
\DeclareRobustCommand{\tikzcircle}[2][red,fill=red]{\tikz[baseline=-0.5ex]\draw[#1,radius=#2] (0,0) circle ;}
\DeclareRobustCommand{\tikzcirclehollow}[2][red]{\tikz[baseline=-0.5ex]\draw[#1,radius=#2] (0,0) circle;}

\newtheorem{assumption}{Assumption}


\DeclareMathOperator*{\argmax}{arg\,max}
\DeclareMathOperator*{\argmin}{arg\,min}
\newcommand{\si}[1]{{\scriptstyle \mathcal{S}_i#1}}
\newcommand{\di}[1]{{\scriptstyle \mathcal{D}_i#1}}

\newcommand{\czq}[1]{\textcolor{black}{{#1}}}

\renewcommand{\arraystretch}{1.2}


\AtBeginDocument{%
  \providecommand\BibTeX{{%
    \normalfont B\kern-0.5em{\scshape i\kern-0.25em b}\kern-0.8em\TeX}}}

\setcopyright{acmcopyright}
\copyrightyear{2022}
\acmYear{2022}
\acmDOI{XXXXXXX.XXXXXXX}


\acmJournal{TOIS}
\acmVolume{37}
\acmNumber{4}
\acmArticle{111}
\acmMonth{8}





\begin{document}

\title{Studying the Impact of Data Disclosure Mechanism in Recommender Systems via Simulation}

\author{Ziqian Chen}
\email{eric.czq@alibaba-inc.com}
\affiliation{%
  \department{Damo Academy}
  \institution{Alibaba Group}
  \city{Hangzhou}
  \state{Zhejiang}
  \postcode{311121}
  \country{China}
}

\author{Fei Sun}
\authornote{Fei Sun is the corresponding author and now works at ICT, CAS.}
\orcid{0000-0002-6146-148X}
\email{ofey.sf@alibaba-inc.com}
\affiliation{%
  \department{Damo Academy}
  \institution{Alibaba Group}
  \city{Beijing}
  \postcode{100102}
  \country{China}
}

\author{Yifan Tang}
\email{yifan.tang95@gmail.com}
\affiliation{%
  \department{Luohan Academy}
  \institution{Alibaba Group}
  \city{Hangzhou}
  \state{Zhejiang}
  \postcode{311121}
  \country{China}
}

\author{Haokun Chen}
\email{hankel.chk@alibaba-inc.com}
\affiliation{%
  \department{Damo Academy}
  \institution{Alibaba Group}
  \city{Hangzhou}
  \state{Zhejiang}
  \postcode{311121}
  \country{China}
}

\author{Jinyang Gao}
\email{jinyang.gjy@alibaba-inc.com}
\affiliation{%
  \department{Damo Academy}
  \institution{Alibaba Group}
  \city{Hangzhou}
  \state{Zhejiang}
  \postcode{311121}
  \country{China}
}


\author{Bolin Ding}
\email{bolin.ding@alibaba-inc.com}
\affiliation{%
  \department{Damo Academy}
  \institution{Alibaba Group}
  \city{Seattle}
  \state{WA}
  \postcode{98004}
  \country{United States}
}


\begin{abstract}

Recently, privacy issues in web services that rely on users' personal data have raised great attention.
Despite that recent regulations force companies to offer choices for each user to opt-in or opt-out of data disclosure, real-world applications usually only provide an ``all or nothing'' binary option for users to either disclose all their data or preserve all data with the cost of no personalized service.


In this paper, we argue that such a binary mechanism is not optimal for both consumers and platforms.
To study how different privacy mechanisms affect users' decisions on information disclosure and how users' decisions affect the platform's revenue, we propose a privacy aware recommendation framework that gives users fine control over their data. 
In this new framework, users can proactively control which data to disclose based on the trade-off between anticipated privacy risks and potential utilities.
Then we study the impact of different data disclosure mechanisms via simulation with reinforcement learning due to the high cost of real-world experiments.
The results show that the platform mechanisms with finer split granularity and more unrestrained disclosure strategy can bring better results for both consumers and platforms than the ``all or nothing''  mechanism adopted by most real-world applications.


\end{abstract}


\begin{CCSXML}
<ccs2012>
<concept>
<concept_id>10002951.10003317.10003347.10003350</concept_id>
<concept_desc>Information systems~Recommender systems</concept_desc>
<concept_significance>500</concept_significance>
</concept>
  <concept>
  <concept_id>10002978.10003029.10011150</concept_id>
  <concept_desc>Security and privacy~Privacy protections</concept_desc>
  <concept_significance>500</concept_significance>
  </concept>
</ccs2012>
\end{CCSXML}
  
\ccsdesc[500]{Information systems~Recommender systems}
\ccsdesc[500]{Security and privacy~Privacy protections}

\keywords{Recommender System; Privacy; GDPR}



\maketitle

Reinforcement learning has achieved great success in areas such as Game-playing \citep{silver2018general,vinyals2019grandmaster}, robotics \cite{kober2013reinforcement}, large language models \citep{ouyang2022training}, etc.
However, due to safety concerns or physical limitations, in some real-world reinforcement learning problems, we must consider additional constraints that may influence the optimal policy and the learning process \citep{garcia2015comprehensive}.
% For example, a robotic arm must not take actions that may cause harm to itself or the environments.
A standard framework to handle such cases is the constrained Markov Decision Process (CMDP) \citep{altman1999constrained}.
Within the CMDP framework, the agent has to maximize
the expected cumulative reward while
obeying a finite number of constraints, which are usually in the form of expected cumulative cost criteria.

However, we are sometimes concerned with the problem with a continuum of constraints.
For example,
the constraints we meet might be time-evolving or subject to uncertain parameters, which
cannot be formulated as an ordinary CMDP
(see Examples \ref{Example_Time_Evolving} and  \ref{Example_Uncertain}).
In this paper we would study a generalized CMDP  
to address the above problem.  Because the constraints are not only infinite-number but also lie
in a continuous set,
the generalization is not trivial. Fortunately, we find that we can borrow the idea behind semi-infinite programming (SIP) \citep{remez1934determination, hettich1993semi} to deal with the semi-infinite constraints.
Accordingly, we propose \emph{semi-infinitely constrained Markov decision processes} (SICMDPs)
as a novel complement to the ordinary CMDP framework.
%More specifically,  an SICMDP model %, we consider 
%contains a continuum of constraints whereas an ordinary CMDP contains a finite number of constraints. 

%This generalization is natural but not trivial. However, we can brows the idea  
%The idea is quite natural and can be backtracked
%to the practice of extending linear programming to linear semi-infinite programming (LSIP) %\cite{remez1934determination, GobernaLSIO1998}.
%In addition, 
%As a complementary approach to the ordinary CMDP framework, 
%SICMDP can be used to model these problems  which cannot be described by a finite number of constraints
%that are not covered by .
%For example,
%the restrictions we consider can be time-evolving or subject to uncertain parameters
%, thus
%cannot be described by a finite number of constraints but a continuum of constraints 
%(see Examples \ref{Example_Time_Evolving} and  \ref{Example_Uncertain}).

We also present two reinforcement learning algorithms to solve SICMDPs called SI-CRL and SI-CPO, respectively.
SI-CRL is a model-based reinforcement learning algorithm designed for tabular cases, and SI-CPO is a policy optimization algorithm for non-tabular cases.
% and analyze its performance both theoretically and empirically.
The main challenge is that we need to deal with a continuum of constraints, thus reinforcement learning algorithms for ordinary CMDPs do not work anymore.
In SI-CRL, we tackle this difficulty by first transforming the reinforcement learning problem to an equivalent LSIP problem, which can then be solved using methods in the LSIP literature like the dual exchange methods \citep{Hu1990,reemtsen1998numerical}.
In SI-CPO, we resort to the idea of cooperative stochastic approximation developed in \cite{lan2020algorithms, wei2020comirror}.
As far as we know, we are the first to introduce tools from semi-infinitely programming (SIP) into the reinforcement learning community for solving constrained reinforcement learning problems.

% To the best of our knowledge, we are the first to apply tools from semi-infinitely programming (SIP) to solve reinforcement learning problems.
Furthermore, we give theoretical analysis for both SI-CRL and SI-CPO.
We decompose the error of SI-CRL into two parts: the statistical error from approximating the true SICMDP with an offline dataset and the optimization error due to the fact that the solution of the LSIP problem obtained by the dual exchange method is inexact.
On the optimization side, we show that the iteration complexity of SI-CRL is $O\left(\left\{\mathrm{diam}(Y)L\sqrt{|\gS|^2|\gA|m}/\left[(1-\gamma)\epsilon\right]\right\}^m\right)$.
On the statistical side, we show that the sample complexity of SI-CRL is $\widetilde O\left(\frac{|S|^2|A|^2}{\epsilon^2(1-\gamma)^3}\right)$ if the offline dataset is generated by a generative model, and $\widetilde O\left(\frac{|S||A|}{\nu_{\min} \epsilon^2(1-\gamma)^3}\right)$ if the dataset is generated by a probability measure $\nu$ as considered in \cite{chen2019information}.
Here $\widetilde O$ means that all logarithm terms are discarded.
For SI-CPO, things become a little more complicated because other than the statistical error and the optimization error, we also need to consider the function approximation error, which comes from imperfect policy parametrizations.
It is shown if the function approximation error can be controlled to $O(\epsilon)$ order, the iteration complexity of SI-CPO is $\widetilde{O}\left(\frac{1}{\epsilon^2(1-\gamma)^6}\right)$ and the sample complexity of SI-CPO is $\widetilde{O}(\frac{1}{\epsilon^4(1-\gamma)^{10}})$.
Here our iteration complexity bound is equivalent to a typical $\widetilde O(1/\sqrt{T})$ global convergence rate.

We perform a set of numerical experiments to illustrate the SICMDP model and validate our proposed algorithms.
Specifically, we examine two numerical examples, namely the discharge of sewage and ship route planning.
Through the discharge of sewage example, we show the advantage of the SICMDP framework over the CMDP baseline obtained by naive discretization in modeling realistic sequential decision-making problems.
Moreover, we demonstrate the effectiveness of the SI-CRL and SI-CPO algorithms in such tabular environments. 
In the ship route planning example, we illustrate the benefits of the SICMDP framework and the ability of the SI-CPO algorithm to address complex continuous control tasks involving continuous state spaces with modern deep reinforcement learning techniques.

% In summary, our contributions are listed as follows.
% First, we present the SICMDP model, which can be viewed as a generalization of the ordinary CMDP model.
% Second, we propose an algorithm to perform reinforcement learning for SICMDPs, which is called SI-CRL, and we believe that we are the first to apply tools from SIP
% to solve reinforcement learning problems.
% Third, we give a theoretical analysis of SI-CRL and identify both its sample complexity and iteration complexity.
% In addition, we perform numerical experiments to illustrate the SICMDP model and validate the SI-CRL algorithm.
% \{This paragraph can be removed!!! \}





\tikzset{
  FARROW/.style={arrows={-{Latex[length=1.25mm, width=1.mm]}}},
  DFARROW/.style={arrows={{Latex[length=1.25mm, width=1.mm]}-{Latex[length=1.25mm, width=1.mm]}}},
  behavior/.style = {circle, fill=monte_carlo, minimum width=1.2em, align=center, inner sep=0, outer sep=0, font=\tiny},
  feature/.style = {circle, fill=salmon, minimum width=1.2em, align=center, inner sep=0, outer sep=0, font=\tiny},
  encoder/.style = {rectangle, fill=Madang!82, minimum width=6em, minimum height=2em, align=center, rounded corners=3},
  emb_layer/.style = {rectangle, fill=languid_lavender!72, minimum width=11em, minimum height=2em, align=center, rounded corners=3},
  project/.style = {rectangle, fill=hint_green, minimum width=7em, minimum height=2.4em, align=center, rounded corners=2},
  ds/.style={
       rectangle split,
       rectangle split part align=base,
       rectangle split horizontal=true,
       rectangle split draw splits=true,
       rectangle split parts=5,
       rectangle split part fill={athens_gray!80, athens_gray!80, athens_gray!80, athens_gray!80, athens_gray!80},
       draw=black, %
       very thin,
       minimum height=1.2em,
       minimum width=2em,
       text width=0.4em,
       inner sep=.5pt,
       text centered,
       font=\tiny,
       text=gray,
       },
}

\begin{figure}
\centering
\begin{subfigure}[b]{0.48\textwidth}
\resizebox{0.92\linewidth}{!}{
\begin{tikzpicture}
    \node[] (to) at (0, 0) {};
    
    \node[person, female, hair=black, shirt=flamingo, skin=skin, minimum size=0.7cm, below of=to, node distance=1cm] (t_ub) {};
    \node[businessman, mirrored, minimum size=0.7cm, right of=t_ub, node distance=2.8cm] (t_uc) {};
    \node[maninblack, minimum size=0.7cm, left of=t_ub, node distance=2.8cm] (t_ua) {};
    
    \foreach \x/\y in {a/1, b/2, c/3}
    {
    \node [feature, below of=t_u\x, node distance=1.2cm, xshift=-0.9cm] (f_t\x1) {$a_{\scaleto{\y1}{2.5pt}}$};
    \node [right of=f_t\x1, node distance=0.325cm, font=\tiny, align=center] (f_t\x2) {$\cdots$};
     \node [feature, right of=f_t\x2, node distance=0.325cm] (f_t\x3) {$a_{\scaleto{\y K}{2.4pt}}$};
     \node [behavior, right of=f_t\x3, node distance=0.5cm, font=\tiny] (b_t\x1) {$b_{\scaleto{\y1}{2.5pt}}$};
     \node [right of=b_t\x1, node distance=0.325cm, font=\tiny] (b_t\x2) {$\cdots$};
    }
    
    \node [behavior, right of=b_ta2, node distance=0.325cm] (b_ta3) {$b_{\scaleto{1t}{2.5pt}}$};
    \node [behavior, right of=b_tb2, node distance=0.325cm] (b_tb3) {$b_{\scaleto{2t'}{3.2pt}}$};
    \node [behavior, right of=b_tc2, node distance=0.325cm] (b_tc3) {$b_{\scaleto{3t''}{3pt}}$};
     
     \foreach \x in {a, b, c}
     {
     \draw [decorate, decoration={mirror, brace, amplitude=5pt}] ([xshift=-0.5mm, yshift=-0.5mm] f_t\x1.south west) -- ([xshift=0.5mm, yshift=-0.5mm] b_t\x3.south east) node[midway,yshift=0.5cm] (d_t\x) {};
     }
     
     \node [cloud, cloud puffs=12.8, cloud ignores aspect, draw=fern, semithick, fill=hint_green, minimum width=25mm, align=center, inner ysep=4.5pt, below of=to, node distance=4.5cm] (data_t) {data:$\{\mathcal{D}_i\}$};
     
     \node[encoder, below of=data_t, node distance=1.6cm, minimum width=5cm, minimum height=0.8cm,] (rec_model) {recommender $\texttt{M}_{{\scriptscriptstyle \mathcal{D}}}$};
     
     \node[dashed, thick, draw=black, fit={(t_ua) (t_uc) ([yshift=-2mm] f_ta1) ([yshift=-2mm] b_tc3.south east)}, inner sep=5, rounded corners=2] (t_user_side) {};
     
     \draw[FARROW, rounded corners=2] ([yshift=-6mm] d_ta.south) |- (data_t.west) node [pos=0.5, yshift=2mm, right, black, align=left, font=\small ]  { $\mathcal{D}_1$};
     \draw[FARROW, rounded corners=2] ([yshift=-6mm] d_tc.south) |- (data_t.east) node [pos=0.5, yshift=2mm, left, black, align=left, font=\small ]  {$\mathcal{D}_3$};
     \draw[FARROW] ([yshift=-6mm] d_tb.south) -> (data_t.north) node [pos=0.5, right, black, align=left, font=\small ]  {$\mathcal{D}_2$};
     
     \draw[FARROW] ( data_t.south) -> (rec_model.north)  node [pos=0.5, right, black, align=left] (box_c_a){training};
     \draw[FARROW, rounded corners=2] (rec_model.west) -- ++(-1.8, 0) -- ++(0, 4.57) -- (t_user_side.west);  
     \draw[FARROW, rounded corners=2] (rec_model.east) -- ++(1.8, 0) -- ++(0, 4.57) -- (t_user_side.east);  
     \node[left of=rec_model, node distance=3.5cm, yshift=3mm, font=\small] {serving};
     \node[right of=rec_model, node distance=3.5cm, yshift=3mm, font=\small] {serving};
     
\end{tikzpicture}}
\caption{Traditional recommender systems.}
\label{fig:traditional_rec}
\end{subfigure}
\hfil
\begin{subfigure}[b]{0.48\textwidth}
\resizebox{\linewidth}{!}{
    \begin{tikzpicture}
        
    
    \node[person, female, hair=black, shirt=flamingo, skin=skin, minimum size=0.7cm] (ub) at (0,0) {};
    \node[businessman, mirrored, minimum size=0.7cm, right of=ub, node distance=2.8cm] (uc) {};
    \node[maninblack, minimum size=0.7cm, left of=ub, node distance=2.8cm] (ua) {};
    
    
    \node [feature, below of=ua, node distance=0.8cm, xshift=-0.9cm, ] (f_a1) {$a_{\scaleto{11}{2.5pt}}$};
    \node [right of=f_a1, node distance=0.325cm, font=\tiny, align=center] (f_a2) {$\cdots$};
    \node [feature, right of=f_a2, node distance=0.325cm, ] (f_a3) {$a_{\scaleto{1K}{2.4pt}}$};
    \node [behavior, right of=f_a3, node distance=0.5cm, ] (b_a1) {$b_{\scaleto{11}{2.5pt}}$};
    \node [right of=b_a1, node distance=0.325cm, font=\tiny] (b_a2) {$\cdots$};
    \node [behavior, right of=b_a2, node distance=0.325cm] (b_a3) {$b_{\scaleto{1t}{2.5pt}}$};
    
    \node [left of=f_a1, node distance=0.4cm, font=\tiny] (d1) {$\mathcal{D}_1$};
    
    \node [behavior, below of=ua, node distance=2cm] (s_b_a3) {$b_{{\scaleto{1t}{2.5pt}}}$};
    
    
    \node[dotted, draw=black, fit={(f_a1) (b_a3) }, inner sep=2, rounded corners=2] (d_box_a) {};
    \node[dotted, draw=black, fit={(s_b_a3) }, inner sep=2, rounded corners=2] (s_box_a) {};
    \draw[FARROW] (d_box_a) -> (s_box_a) node [pos=0.45, right, black, font=\tiny, align=left]  {data\\ disclosure};
    
    \node [feature, below of=uc, node distance=0.8cm, xshift=-0.9cm] (f_c1) {$a_{\scaleto{31}{2.5pt}}$};
    \node [right of=f_c1, node distance=0.325cm, font=\tiny, align=center] (f_c2) {$\cdots$};
    \node [feature, right of=f_c2, node distance=0.325cm] (f_c3) {$a_{\scaleto{3K}{2.5pt}}$};
    \node [behavior, right of=f_c3, node distance=0.5cm] (b_c1) {$b_{\scaleto{31}{2.5pt}}$};
    \node [right of=b_c1, node distance=0.325cm, font=\tiny] (b_c2) {$\cdots$};
    \node [behavior, right of=b_c2, node distance=0.325cm] (b_c3) {$b_{\scaleto{3t''}{3pt}}$};
    \node [left of=f_c1, node distance=0.4cm, font=\tiny] (d3) {$\mathcal{D}_3$};
    
    \node [feature, below of=f_c1, node distance=1.2cm] (s_f_c1) {$a_{\scaleto{31}{2.5pt}}$};
    \node [below of=f_c2, node distance=1.2cm, font=\tiny, align=center] (s_f_c2) {$\cdots$};
    \node [feature, below of=f_c3, node distance=1.2cm] (s_f_c3) {$a_{\scaleto{3K}{2.5pt}}$};
    \node [behavior, below of=b_c1, node distance=1.2cm] (s_b_c1) {$b_{\scaleto{31}{2.5pt}}$};
    \node [below of=b_c2, node distance=1.2cm, font=\tiny] (s_b_c2) {$\cdots$};
    \node [behavior, below of=b_c3, node distance=1.2cm] (s_b_c3) {$b_{\scaleto{3t''}{3pt}}$};
    
    \node[dotted, draw=black, fit={(f_c1) (b_c3) }, inner sep=2, rounded corners=2] (d_box_c) {};
    \node[dotted, draw=black, fit={(s_f_c1) (s_b_c3) }, inner sep=2, rounded corners=2] (s_box_c) {};
    \draw[FARROW] (d_box_c) -> (s_box_c) node [pos=0.45, right, black, font=\tiny, align=left]  {data\\ disclosure};
    
    
    \node [feature, below of=ub, node distance=0.8cm, xshift=-0.9cm, ] (f_b1) {$a_{\scaleto{21}{2.5pt}}$};
    \node [right of=f_b1, node distance=0.325cm, font=\tiny, align=center] (f_b2) {$\cdots$};
    \node [feature, right of=f_b2, node distance=0.325cm,] (f_b3) {$a_{\scaleto{2K}{2.5pt}}$};
    \node [behavior, right of=f_b3, node distance=0.5cm, ] (b_b1) {$b_{\scaleto{21}{2.5pt}}$};
    \node [right of=b_b1, node distance=0.325cm, font=\tiny] (b_b2) {$\cdots$};
    \node [behavior, right of=b_b2, node distance=0.325cm] (b_b3) {$b_{\scaleto{2t'}{3.2pt}}$};
    \node [left of=f_b1, node distance=0.4cm, font=\tiny] (d2) {$\mathcal{D}_2$};
    
    \node [below of=ub, node distance=2cm, font=\tiny, align=center] (s_b_b2) {$\cdots$};
    \node [behavior, left of=s_b_b2, node distance=0.4cm,] (s_b_b1) {$b_{\scaleto{21}{2.5pt}}$};
    \node [behavior, right of=s_b_b2, node distance=0.4cm,] (s_b_b3) {$b_{\scaleto{2t'}{3.2pt}}$};
    
     \node[dotted, draw=black, fit={(f_b1) (b_b3) }, inner sep=2, rounded corners=2] (d_box_b) {};
    \node[dotted, draw=black, fit={(s_b_b1) (s_b_b3) }, inner sep=2, rounded corners=2] (s_box_b) {};
    \draw[FARROW] (d_box_b) -> (s_box_b) node [pos=0.45, right, black, font=\tiny, align=left]  {data\\ disclosure};
    
    \node [cloud, cloud puffs=12.3, cloud ignores aspect, draw=fern, semithick, fill=hint_green, minimum width=25mm, align=center, inner ysep=0pt, below of=ub, node distance=3.3cm, font=\small] (data_p) {data: $\{\mathcal{S}_1, \dots, \mathcal{S}_{\scaleto{|\mathcal{V}|}{3.2pt}}\}$};
    
    \node[left of = s_box_a, node distance=0.5cm, font=\tiny] {$\mathcal{S}_1$};
    \node[left of = s_b_b1, node distance=5mm, font=\tiny]  {$\mathcal{S}_2$};
    \node[left of = s_f_c1, node distance=5mm, font=\tiny]  {$\mathcal{S}_3$};
    
    \draw[FARROW, rounded corners=2] (s_box_a.south) |- (data_p.west);
    \draw[FARROW, rounded corners=2] ( s_box_c.south) |- (data_p.east);
    \draw[FARROW] (s_box_b.south) -> (data_p.north);
    
    \node[encoder, below of=ub, node distance=4.8cm, minimum width=5cm, minimum height=0.8cm, font=\small] (model_p) {\large recommender $\texttt{M}_{{\scriptscriptstyle \mathcal{S}}}$};
    
    \draw[FARROW] (data_p.south) -> (model_p.north) node [pos=0.5, right, black, align=left, font=\small] (box_c_a){training};
    
    \draw[FARROW, rounded corners=2] (model_p.west) -- ++(-1.8, 0) -- ++(0, 4.8) -- (ua.west); 
    \draw[FARROW, rounded corners=2] (model_p.east) -- ++(1.8, 0) -- ++(0, 4.8) -- (uc.east); 
    \draw[FARROW, rounded corners=2] (model_p.east) -- ++(1.8, 0) -- ++(0, 5.3) -- ++(-3.5, 0) -- (ub); 
    
    \node[left of=model_p, node distance=3.2cm, yshift=2mm, font=\small] {serving};
    \node[right of=model_p, node distance=3.2cm, yshift=2mm, font=\small] {serving};
    
    
    \end{tikzpicture} }
    \caption{Privacy aware recommender systems.}
    \label{fig:privacy_rec}
    \end{subfigure}
    \caption{Illustrative examples for two different recommender system frameworks.}
    \label{fig:framework}
\end{figure}



\section{Simulation}

\subsection{motivation} %
The most efficient way to figure out the answers to the questions we posed in the introduction is to deploy the proposed framework on a real-world platform and analyze how users adopt different and complex privacy policies to optimize their rewards.
However, direct deployment of these strategies and investments is currently impractical due to the following reasons.


Firstly, the most important reason is that such an online experiment may lead to the decline of the recommendation performances as well as the user experience, which harms the platform's revenue.
In the real world, nearly all the companies determine their platform mechanism driven by interest, and the revenues of the platforms are highly correlated with the recommendation performances. 
Therefore, it's nearly impossible to persuade any platform to directly deploy proposed strategies and mechanism online without other benefits.  



Secondly, the experiments are \czq{built} upon several simplifications, mentioned in \cref{assumptions}, which poses challenges towards recommendation model training process.
For example, we assume when a user \czq{adjusts} his data disclosure policy, the recommendation system will forget his un-disclosed data. 
To facilitate such challenges, model unlearning or other privacy-preserving technologies are imposed.
However, in real-world applications, very few the e-commercial platforms have deployed these privacy-preserving technologies during the deep recommendation model training and evaluation processing.
As a result, we may still fail to guarantee the assumptions and simulation methodology becomes a substitution.










In summary, inspired by the success of simulation study on dynamic interactive problems in real-world applications~\cite{Ie:arxiv19:RecSim,krauth2020offline,lucherini2021t,yao2021measuring},  we employ the simulation to study the effects of the proposed framework and the possible game between users and the platform.















\subsection{Simplified Assumptions}
\label{assumptions}

To simplify the simulation process for easier analysis, we make some necessary assumptions to simplify the problem.

\begin{assumption}[Static Assumption] User $i$ optimizes her/his policy on the fixed data $\di{}$ which is not affected by user policy $\pi_i$.
\label{assumption:static}
\end{assumption}

Here static means the user data $\di{}$ is fixed during the simulation, but the disclosed data $\si{}$ produced by different user policies is dynamic. 
It is also the most common setting for recommendation task in research papers~\cite{Rendle:www10:Factorizing,Hidasi:ICLR2016:gru4rec,NCF,kang2018self,Sun:cikm19:BERT4Rec}.
In the simulation, we train the recommendation system $\texttt{M}_{{\scriptscriptstyle \mathcal{S}}}$ on the collected dynamic data $\mathcal{S}$ and validate the recommendation efficiency on a fixed test set. 
In real-world applications, the data $\di{}$, which contains the behavior data from the interaction with the recommender $\texttt{M}_{{\scriptscriptstyle \mathcal{S}}}$, is also dynamically changing with the user's policy $\pi_i$.
It is beyond the scope of this paper and we leave it as the future work. 


\begin{assumption}[Immediate Assumption] The recommendation model $\texttt{M}_{{\scriptscriptstyle \mathcal{S}}}$ can only use the data $\si{}$ currently disclosed by each user $i$.
\label{assumption:forget}
\end{assumption}
The motivation of this assumption is that an untrusted platform can leverage user $i$' all data $\di{}$ if it can use the data disclosed in previous actions.
Without this constraint, the privacy right discussed in this paper is meaningless.
To achieve this, the platform can retrain the model from scratch with new data $\si{'}$ or quick unlearn the data in $\si{}$ then finetune with data $\si{'}$~\cite{cao2015towards,bourtoule2021machine,chen2022recommendation}.


However, the \cref{assumption:forget} also raises a new challenge that the asynchronous changes of user policy bring intractable computation costs for the platform since each time the user changes the disclosed data, the platform needs to update the model.
Here, we make an assumption for simplifying the simulation, assuming all users realize that the platform will cyclically (e.g., once a day) collect their privacy decisions and update recommender systems.
\begin{assumption}
[%
Cyclical Assumption]
Platform cyclically collects user privacy choices, and then the platform updates the model using all newly disclosed data. 
\label{assumption:synchronization}
\end{assumption}



In summary, for easy analysis in simulations, we introduce these assumptions to ignore the time and dynamic effects in this feedback system, just like the traditional recommendation task formulation.
























\subsection{Platform Mechanism Simulation}
\label{sec:plat_mech}
In order to validate the effect of the platform mechanism, we adopt several mechanisms during simulation. 
For easy comparison, we utilize one mechanism at each experiment. %


\subsubsection{\textbf{Data Split Rule}}

\czq{In our simulation, we do not split the profile attribution and the user can determine whether to share all of their attributes.}
For behavior data, we apply ``percentage split'' as $\delta_b$ with different split granularity $p$ (e.g., 1/3) to split the behavior sequence into $1/p$ parts. 
One obvious advantage of ``percentage split'' is that it can normalize the size of user action space and decrease the inconvenience of the interaction between the user and the platform.

\subsubsection{\textbf{Data Disclosure Strategy}}
\label{sec:data_disclose_choice}

As the platforms have certain flexibility to implement different data disclosure strategies, we discuss three representative disclosure strategies used in our study for behavior data in this subsection.
These strategies determine the data disclosure action space $\Pi$ the user can choose.
For profile attributes, we found that all users tend not to disclose them in the experiments since these features are negligible for improving recommendation utility in the presence of behavior data.
Similar result that user profile features contribute very marginal to the recommendation results in the case of strong user behavior modeling on public benchmark datasets has also been reported in other works~\cite{kang2018self,Sun:cikm19:BERT4Rec}.
Thus, in the following study, we mainly focus on modeling only  behavior data.

The ``\textit{separate}'' rule gives the users the control to freely disclose any split personal data.
For this rule, the size of user $i$'s the action space is exponentially expended on the size of the spilt data set $|\delta_b ({\scriptstyle \mathcal{D}_{i,b}})|$, denoted as $2^{|\delta_b ({\scriptstyle \mathcal{D}_{i,b}})|}$. 
However, too many choices might make it difficult for users to make better data disclosure decisions.

Another data disclosure strategy named ``\textit{oldest continuous}'' provides users the choices to disclose continuous behavior data from the beginning time, such as selecting ``the oldest 33\% data''.
In this strategy, to disclose newer behavior data ${\scriptstyle \mathcal{S}_{i,bj}}$, users must also disclose all behavior data before it.
Take an already split data $\delta_b ({\scriptstyle \mathcal{D}_{i,b}}) = \{\scriptstyle \mathcal{S}_{i,b1}, \scriptstyle \mathcal{S}_{i,b2}, \scriptstyle \mathcal{S}_{i,b3}\}$ as an example, the action space provided by oldest continuous strategy is $\Pi = \{[0,0,0], [1,0,0], [1,1,0], [1,1,1]\}$, and its corresponding disclosed data is $\{\varnothing ,
    \{\! {\scriptstyle \mathcal{S}_{i,b1}} \!\},
    \{\! {\scriptstyle \mathcal{S}_{i,b1}} , {\scriptstyle \mathcal{S}_{i,b2}}\! \},$
    $\{ {\scriptstyle \mathcal{S}_{i,b1}}, {\!\scriptstyle \mathcal{S}_{i,b2}},$ ${\scriptstyle \mathcal{S}_{i,b3}}\} \}$.
``\textit{Latest continuous}'' mechanism is similar to ``oldest continuous'', with the only difference in the opposite direction.
The size of these two mechanisms' action spaces is $|\delta_b ({\scriptstyle \mathcal{D}_{i,b}})|$.









\subsection{User Policy Simulation}
\label{sec:user}


In this subsection, we introduce the simulation of user policy in our proposed framework.
As defined in \cref{eq:S_i}, the disclosed data $\si{}$ is result of the platform mechanism $\mathrm{G}$ and user's disclosure policy $\pi_i$. 
Meanwhile, in \cref{eq:updated_rec}, the recommendation utility $\texttt{U}_i( \si{})=\texttt{U}'(\si{},\, \texttt{M}_{{\scriptscriptstyle \mathcal{S}}})$ is also determined by the recommendation model $\texttt{M}_{{\scriptscriptstyle \mathcal{S}}}$, which is \czq{built} upon the all users' disclosed data $\mathcal{S}$. 
The reward of user $i$ may be varied even when $i$ keeps the disclosed data $\si{}$ unchanged since other users might change their disclosed data and the recommender system is changed.
Thus, the expectation rewards are considered rather than immediate value defined in \cref{eq:framework} and we assume all the users are rational and seek for the optimal privacy disclosure action  $\alpha_i^*$ to the optimal expected reward $E[ \texttt{R}_i | \alpha_i ] $ as his policy, i.e., %
\begin{equation}
    \begin{aligned}
    \alpha_i^{*} &= \argmax_{\alpha_i \in \Pi} E[ \texttt{R}_i | \alpha_i ]=  \argmax_{\si{\in [ \Pi \otimes \mathrm{\delta}(\di{}) ] }} E[ \texttt{R}_i(\si{)} ] \\
& =\argmax_{\alpha_i \in \Pi } E\Bigl[ -\lambda_i \texttt{C}_i\bigl( \alpha_i \otimes \mathrm{\delta}(\di{}) \bigr) + \texttt{U}_i\bigl( \alpha_i \otimes \mathrm{\delta}(\di{}) )\bigr) \Bigr].
    \end{aligned}
\label{eq:opt_pi}
\end{equation}

As mentioned before, recommendation utility $\texttt{U}_i$ has been discussed in \cref{sec:platform_obj}.
To study this objective, we need to define the privacy cost function $\texttt{C}_i$ and sensitive weight $\lambda_i$.



\subsubsection{\textbf{Privacy Cost Function}}
\label{sec:privacy_cost}
We simulate every user with the same cost function $\texttt{C}$ and leave the diversity of user privacy sensitivity to the parameter $\lambda_i$. 
Following current experiment specifications in the economics literature~\cite{lin2019valuing,tang2019value}, we model the privacy cost function as a linear summation\footnote{See the Eq. 2 in \cite{lin2019valuing} and the dis-utility from disclosure in the econometric specification session in \cite{tang2019value}.} of disclosed personal data.


\czq{According to the comprehensive survey on privacy value definition \cite{MKT-053}, people will measure the value of their privacy into the intrinsic value of privacy and the instrumental value of privacy.}
\czq{
The intrinsic loss indicates the sake of protecting their intrinsic private data, which measures the valuation on the intrinsic properties such as the education or the income levels. }
\czq{In this work, we denote the intrinsic loss towards the privacy cost on amount of the sharing user profile attributes.}
\czq{The instrumental value of privacy indicates how the transaction efficiency would be affected by sharing user data, especially the data generated in the applications. 
In this work we denote the privacy cost towards the percentage of shard user historical behavior data. 
Therefore, the privacy cost function is described below,}
\begin{equation}
    \texttt{C}_i(\si{})= \beta_i * | {\scriptstyle \mathcal{S}_{i,a}} | + \frac{| {\scriptstyle \mathcal{S}_{i,b}} |}{ | {\scriptstyle \mathcal{D}_{i,b}} |}
    \label{eq:cost_function0}
\end{equation}
\czq{where the first term indicates the intrinsic loss and the second term indicates the instrumental loss.} 
\czq{If user does not tend to disclose profile attribute, such privacy cost function can be simplified to the following format with the instrumental value alone.}
As mentioned in \cref{sec:plat_mech}, user tends not to disclose profile attributes $\scriptstyle \mathcal{D}_{i,a}$ due to no gains in our experiments, so we only consider behavior data here, i.e.,
\begin{equation}
    \texttt{C}_i(\si{})=\texttt{C}(\si{}) =  \frac{| {\scriptstyle \mathcal{S}_{i,b}} |}{ | {\scriptstyle \mathcal{D}_{i,b}} |}
    , %
    \label{eq:cost_function}
\end{equation}
where the $|x|$ is the number of elements in $x$.
Here, the percentage based measurement regards different amount of users' data equally. %


This reduced form specification is not unrealistic as it captures the substitution effect among personal data and incorporates the idea of constant marginal privacy cost. 
One might argue for a higher order functional to capture richer implications. 
However, there is little experimental evidence that the higher order form for privacy cost exists and how the functional form looks like.






\subsubsection{\textbf{Privacy Sensitive Weight}}
\label{sec:user_type}
For user $i$ who disclosed all her/his data (i.e., $\si{} = \di{}$), her/his privacy cost compared to not sharing any data (i.e., $\si{} = \varnothing $) is 
\begin{equation}
    \texttt{C}(\di{}) - \texttt{C}(\varnothing).
\label{eq:privacy_diff}
\end{equation}
Meanwhile, her/his anticipated recommendation utility compared to not sharing any data is:
\begin{equation}
    \texttt{U}(\di{}) - \texttt{U}(\varnothing).
\label{eq:utility_diff}
\end{equation}
We assume all users have accessed to the recommendation utility $\texttt{U}(\di{})=\texttt{U}'(\di{},\texttt{M}_{{\scriptscriptstyle \mathcal{D}}})$ computed on all the data $\di{}$ and the recommendation utility without their data $\texttt{U}(\varnothing)$ before they can take data disclosing actions, which can be regard as a prior knowledge, like the experiences before the platform adopted our framework.
With \cref{eq:privacy_diff} and \cref{eq:utility_diff}, we define the privacy sensitive weight $\lambda_i$ as: 
\begin{equation}
    \lambda_i = w_i  * \frac{\texttt{U}(\di{})  - \texttt{U}(\varnothing) } {  \texttt{C}(\di{}) - \texttt{C}(\varnothing) },
    \label{marginal_define}
\end{equation}
where $w_i$ indicates the diversity of user types towards privacy sensitivity.
The users with $w_i > 1$ is privacy sensitive users, as they will not be willing to disclose the corresponding data $\di{}$ if they only get $\texttt{U}(\di{})$ as before.
While users  with $w_i < 1$ are just the opposite. %
Therefore, the user's privacy sensitive weight is pre-computed, and the $\texttt{U}(\di{})$ can be regarded as the benchmark expectation of the platform.
The formulation of the privacy sensitive weight $\lambda_i$ also meets the idea from \cite{lin2019valuing}, where the heterogeneity from users' social demographic variety should also be explicitly characterized. %



\subsubsection{\textbf{Simulation Algorithm}}

As users behave rationally to find the optimal strategy with a trade-off of exploration and exploitation, it just meets the idea of the reinforcement learning algorithm. 
Therefore, we model each user as a unique agent and apply a multi-agent reinforcement learning method to simulate user possible policy adaptation. 
The recommender system is regarded as the environment to provide feedback, which is built upon the disclosed user data.
All agents' policies are optimized simultaneously by determining their actions, i.e., the disclosed data $\scriptstyle \mathcal{S}^t$ at simulation epoch $t$, which is used to train the recommendation model $\texttt{M}_{{\scriptscriptstyle \mathcal{S}}^{t}}$.
As mentioned before, users tend to find an optimal action over possible action space $\Pi$ to maximize his expected reward, which is determined by all agents in this dynamic MARL environment. 


We assume each user (agent) realizes this situation that the immediate reward is the result of all agents, but no communication or observation among agents is permitted. 
Then, each agent is concerned about her/his own utility and regards the environment as a dynamic system that is partially correlated to herself/himself. 
Now, it is simplified to a Multi-Armed Bandit problem~\cite{katehakis1987multi}.


However, the challenge of the exploration and explication problem also exists in our simulation. 
To address it, we adopt a simple but efficient method, Epsilon Greedy~\cite{sutton2018reinforcement} algorithm, to simulate user's policy $\pi_i$ as following, %
\begin{equation}
     \alpha_i^{t+1} = \left\{
\begin{array}{l l }
\alpha_i \sim \texttt{P}^t_i,
& \text{with possibility } \epsilon \\
\argmax_{\alpha_i} Q_i^t(\alpha_i), & \text{with possibility }  1-\epsilon \\
\end{array} \right. 
    \label{epsilon_greedy}
\end{equation}
where $Q_i^t(\alpha)$ is the user $i$'s estimation value at simulation epoch $t$ on action $\alpha$, and $\texttt{P}^t_i$ denotes a random sample policy.
To conduct an efficient policy exploration, we sample a less explored action with a higher possibility as following,
\begin{equation}
    \texttt{P}^t_i(\alpha) = \frac{ 1/ (N^{t-1}_i(\alpha) +1) } { \sum_{x \in \Pi} 1/(N^{t-1}_i(x) +1) },
    \label{random_rule}
\end{equation}
where $N^{t-1}_i(\alpha)$ represents the total number of action $\alpha$ was taken by user $i$ from start to the last simulation epoch $t{-}1$.
In convenience, we adopt the approximated expected estimation results and 
update it with the residual between the estimation $Q_i^{t-1}(\alpha_i^{t-1})$ and immediate reward  $\texttt{R}_i^{t-1}$ when she/he takes action $\alpha_i^{t-1}$ as following.
\begin{equation*}
       Q_i^t(\alpha) {=} \left\{\!\!\!
\begin{array}{l l}
Q_i^{t-1}(\alpha), &  \text{if } \alpha_i^{t-1} {\neq} \alpha \\
Q_i^{t{-}1}(\alpha) {+} \frac{1}{N^t_i(\alpha)} \bigl(\texttt{R}_i^{t-1}(\! \alpha {\otimes} \mathrm{\delta}(\di{})  ) {-} Q_i^{t{-}1}(\alpha) \bigr), &  \text{if }  \alpha_i^{t{-}1} {=} \alpha \\
\end{array} \right. 
\end{equation*}
where $\texttt{R}_i^{t-1}$ is user $i$-th immediate objective at simulation epoch $t{-}1$, computed by \cref{eq:framework}. 
$Q_i^0(\alpha) $ is the user $i$'s initial expected reward if she/he takes action $\alpha$. 
which is initialized to $0$ as users have no prior about their behaviors on the new dynamic environment.


In our simulation, we set initial $\epsilon=0.5$ for all agents and decay a half during the MARL training processing. The detailed decay epoch is co-related to the size of possible action space $\Pi$.
Here, we define it as $\epsilon = 0.5^{ t /(3 * |\Pi |) }  $, 
where $t$ is the epoch during the reinforcement learning training processing. 

\subsection{\textbf{Discussion}}
To figure out how the platform mechanism affects users' behavior, we turn to the simulation built upon several simplified assumptions. 
One fundamental assumption is the hypothesis of rational man, where users will seek their optimal policies to maximize their objectives. 
However, in the real-world scenarios, human behaviors are also affected by psychological factors, which should also be modeled in future work.
One detailed example is that some users may feel exhausted digging out all the potential privacy choices with the provided platform mechanism.
In our simulation, we assume there remains no mental cost when a user adjusts his policy. 
However, in the reality, some users may refuse to change their policy frequently, especially in complex user interaction applications.
For such situation, a convenient user interface (UI) could be a potential solution to mitigate users' fatigues. 
\czq{
Another important factor is that users may adjust their trust level towards the platform during their exploration. One detailed example is that if the platform or even the recommender system \cite{zhang2022pipattack} is easy to be attacked or the platform will abuse their disclosed data to other applications, they may re-consider their privacy sensitivity. 
Though some works have discussed the utilization of trusted platform or the privacy-preserved recommendation model, the possible effects on user psychological factors might be tackled by a dynamic modeling on the user privacy sensitive weights, which is out of the scope of this work.
}
We simplify the influences of the psychological factors in this work and leave the exploration of psychological effects in mechanism designs and UI designs for future works. 





\subsection{Unsupervised Grammar Induction}

\subsubsection{Setup}\label{sec:LM_setup}
\paragraph{Baselines and Evaluation.} 
For comparison, we include six recent strong models for unsupervised parsing with available open source implementations: StructFormer \cite{DBLP:conf/acl/ShenTZBMC20}, Ordered Neurons~\cite{DBLP:conf/iclr/ShenTSC19}, URNNG~\cite{dblp:conf/naacl/kimrykdm19}, DIORA~\cite{dblp:conf/naacl/drozdovvyim19}, C-PCFG~\cite{kim-etal-2019-compound}, and R2D2~\cite{hu-etal-2021-r2d2}. 
To observe the marginal gain from pretraining, we also include Fast-R2D2 without pretraining denoted as Fast-R2D2$_{\rm w/o}$.
Following~\newcite{htut-etal-2018-grammar}, we train all systems on a training set consisting only of raw text, and evaluate and report the results on an annotated test set. 
As an evaluation metric, we adopt sentence-level unlabeled $F_1$ computed using the script from \newcite{kim-etal-2019-compound}.
We compare against the non-binarized gold trees per convention.
The results of Fast-R2D2 are obtained from 3 runs of each model with different random seeds in pre-training.
The best checkpoint for each system is picked based on scores on the validation set. 
Fast-R2D2 is pretrained with span constraints for the word level but without span constraints for the word-piece level.
To support word-piece level evaluation, 
we convert gold trees to word-piece level trees 
by simply breaking each terminal node into a non-terminal node with its word-pieces as terminals, e.g., (NN discrepancy) into (NN (WP disc) (WP \#\#re) (WP \#\#pan) (WP \#\#cy)).

\paragraph{Environment.} EFLOPS~\cite{DBLP:conf/hpca/DongCZYWFZLSPGJ20} is a highly scalable distributed training system designed by Alibaba. With its optimized hardware architecture and co-designed supporting software tools, including ACCL~\cite{DBLP:journals/micro/DongWFCPTLLRGGL21} and KSpeed (the high-speed data-loading service), it could easily be extended to 10K nodes (GPUs) with linear scalability.

\paragraph{Hyperparameters.} The tree encoder of our model uses 4-layer Transformers with 768-dimensional embeddings, 
3,072-dimensional hidden layer representations, and 12 attention heads. 
The top-down parser of our model uses a 4-layer bidirectional LSTM with 128-dimensional embeddings and 256-dimensional hidden layer. The sampling number $K$ is set to be 256.
Training is conducted using Adam optimization with weight decay using a learning rate of $5 \times 10^{-5}$ for the tree encoder and $1 \times 10^{-2}$ for the top-down parser.
The batch size is set to 64 per GPU for $m$=$4$, though we also limit the maximum total length for each batch, such that excess sentences are moved to the next batch. The limit is set to 1,536. It takes about 120 hours for 60 epochs of training with $m$=$4$ on 8 A100 GPUs.

\paragraph{Data.}  For English, to fully leverage the scalability of Fast-R2D2, we pretrain Fast-R2D2 on WikiText103~\cite{DBLP:conf/iclr/MerityX0S17}
and then fine-tune the model on the Penn Treebank (PTB)~\cite{marcus-etal-1993-building}
for 10 epochs with the same objective.
WikiText103 is split at the sentence level, and sentences longer than 200 after tokenization are discarded (about 0.04‰ of the original data). 
The total number of sentences is 4,089,500, and the average sentence length is 26.97.
For Chinese, we use a subset of Chinese Wikipedia (Simplified Characters) for pretraining, specifically the first 10,000,000 sentences shorter than 150 characters and then fine-tune on Chinese Penn Treebank (CTB) 8.0~\cite{ctb8}.
We test our approach on PTB WSJ data with the standard splits (2--21 for training, 22 for validation, 23 for test) and the same preprocessing as in recent work \cite{kim-etal-2019-compound}, where we discard punctuation and lower-case all tokens. 
To explore the universality of the model across languages, we further evaluate using the CTB,
on which we also remove punctuation.
Note that in all settings, the training and fine-tuning is conducted entirely on raw unannotated text.

\subsubsection{Results and Discussion}

\begin{table}
\newcommand{\invzero}{\hphantom{0}}
\begin{center}
\setlength{\tabcolsep}{3.pt}
\resizebox{0.45\textwidth}{!}{
\begin{tabular}{@{}l|l|l|l|l@{}}
                    &  eval & mem. & \multicolumn{1}{c|}{WSJ}  & \multicolumn{1}{c}{CTB}  \\
Model               & gran. & cplx  &  $F_1(\mu)$ & $F_1(\mu)$\\ \hline \hline
Left Branching (W)  & WD & $O(n)$& \invzero 8.15  & 11.28 \\
Right Branching (W) & WD & $O(n)$& 39.62 & 27.53 \\
Random Trees (W)    & WD & $O(n)$ & 17.76 & 20.17 \\
\hline
URNNG (W)           & WD & $O(n^3)$& 45.4$^\dag$ & ~~--- \\
ON-LSTM (W)         & WD & $O(n)$  & 47.7$^\dag$ & 24.73 \\
DIORA (W)           & WD & $O(n^3)$& 51.4 & ~~---  \\
StructFormer (W)    & WD & $O(n^2)$& 54.0$^\ddagger$ & ~~--- \\
C-PCFG (W)          & WD & $O(n^3)$& 55.2$^\dag$ & 49.95 \\ \hline
R2D2 (WP)           & WD & $O(n)$ & 48.11 & 44.85  \\
Fast-R2D2$^*$(W)$_{\rm w/o}$ & WD & $O(n)$ & 48.24 & 45.24 \\
Fast-R2D2$^*$(WP)$_{\rm w/o}$ & WD & $O(n)$ & 48.89 & 45.26 \\
Fast-R2D2$^*$(WP)  & WD & $O(n)$ & \textbf{57.22} & \textbf{53.13} \\
\hline \hline
R2D2 (WP)           & WP & $O(n)$  & 52.28 & 63.94 \\ 
Fast-R2D2(WP)      & WP & $O(n)$ & 50.20 & \textbf{67.79} \\
Fast-R2D2$^*$(WP)  & WP & $O(n)$& \textbf{53.88} & 67.74 \\ \hline
\end{tabular}
}
\end{center}
\caption{Unsupervised parsing results with words (W) or word-pieces (WP) as input. ``eval gran." is short for evaluation granularity.
        Values marked with $^{\dag}$ are taken from \newcite{kim-etal-2019-compound}, while $^{\ddagger}$ denotes values taken from \newcite{DBLP:conf/acl/ShenTZBMC20}.
        The bottom three systems are all pre-trained or trained 
        at the word-piece level \textbf{without} span constraints and are measured against word-piece level golden trees. ${\rm w/o}$ means without pretraining.}
\label{tbl:constituency_parsing}
\end{table}


Table~\ref{tbl:constituency_parsing} shows the results of all systems with words (W) and word-pieces (WP) as input on the WSJ and CTB test sets. 
When we evaluate all systems on word-level golden trees, 
our Fast-R2D2 performs substantially better than R2D2 across both datasets.
We denote as Fast-R2D2 the method of using the parser to guide the pruning and selecting the best tree using the chart table and as Fast-R2D2$^*$ the system that uses the top-down parser for tree induction with subsequent R2D2 encoding.
Interestingly, the results suggest that Fast-R2D2$^*$ outperforms Fast-R2D2, especially on the WSJ test set.
Additionally, pretrained Fast-R2D2$^*$
outperforms the models specifically designed for grammar induction.

\begin{table}[!htb]
\small
\begin{center}
\setlength{\tabcolsep}{3.5pt}
\resizebox{0.48\textwidth}{!}{ %
\begin{tabular}{@{}ll| l l l l l l@{}}
 & Model  & WD & NNP & VP & SBAR\\\hline \hline
\multirow{5}{*}{\rotatebox[origin=c]{90}{WSJ}} & DIORA (WP)  & 94.63 & 77.83 & 17.30 & 22.16\\
& C-PCFG (W)                  & ~~--- & ~~--- & 41.7$^\dag$ & 56.1$^\dag$ \\
& C-PCFG (WP)                  & 87.35 & 66.44 & 23.63 & 40.40 \\
& R2D2 (WP)    & \textbf{99.76} & \textbf{86.76} & 24.74 & 39.81\\
& Fast-R2D2$^*$ (WP) & 97.67 & 83.44 & \textbf{63.80} & \textbf{65.68} \\ \hline \hline
\multirow{3}{*}{\rotatebox[origin=c]{90}{CTB}} & C-PCFG(WP) &89.34 & 46.74 & 39.53 & ~~---\\
 & R2D2 (WP) & 97.16 & 67.19 & 37.90 & ~~---\\
 & Fast-R2D2$^*$ (WP) & \textbf{97.80} & \textbf{68.57} & \textbf{46.59} & ~~---
 \\ \hline \hline
\end{tabular}
}
\end{center}
\caption{Recall of constituents and words. WD means word.  Values with $^{\dag}$ are taken from \newcite{kim-etal-2019-compound}.}
\label{tbl:unsupervised_chunking}
\end{table}

Following \newcite{dblp:conf/naacl/kimrykdm19} and \newcite{drozdov-etal-2020-unsupervised},
we also compute the recall of constituents when evaluating on word-piece level golden trees.
Besides standard constituents, we also compare the recall of word-piece chunks and proper noun chunks. 
Proper noun chunks are extracted by finding adjacent unary nodes with the same parent and tag NNP. 
Table~\ref{tbl:unsupervised_chunking} reports the recall scores for constituents and words on the WSJ and CTB test sets. 
Compared with the R2D2 baseline, 
our Fast-R2D2 performs slightly worse for small semantic units, 
but significantly better over larger semantic units (such as VP and SBAR) on the WSJ test set.
On the CTB test set, our Fast-R2D2 outperforms R2D2 on all constituents. 

From Tables~\ref{tbl:constituency_parsing}~and~\ref{tbl:unsupervised_chunking}, 
we conclude that Fast-R2D2 overall obtains better results than R2D2 on CTB, while faring slightly worse than R2D2 only for small semantic units on WSJ. We conjecture that this difference stems from differences in  tokenization between Chinese and English. 
Chinese is a character-based language without complex morphology, where collocations of characters are consistent with the language, making it easier for the top-down parser to learn them well. 
In contrast, word-pieces for English are built based on statistics, and individual word-pieces are not necessarily natural semantic units. Thus, there may not be sufficient semantic self-consistency, such that it is harder for a top-down parser with a small number of parameters to fit it well.

\subsection{Downstream Tasks}
We next consider the effectiveness of Fast-R2D2 in downstream tasks. This experiment is not intended to advance the state-of-the-art on the GLUE benchmark but rather to assess to what extent our approach performs respectably against the dominant inductive bias as in conventional sequential Transformers.

\subsubsection{Setup}
\paragraph{Data and Baseline.}
We fine-tune pretrained models on several datasets,
including SST-2, CoLA, QQP, and MNLI from the GLUE benchmark~\cite{wang2018glue}.
As sequential Transformers with their dominant inductive bias remain the norm for numerous NLP tasks, 
we mainly compare Fast-R2D2 with \bert~\cite{devlin2018} as a representative pretrained model based on a sequential Transformer. 
We did not include recursive models such as Gumbel-Tree-LSTMs~\cite{DBLP:conf/aaai/ChoiYL18} and CRvNN~\cite{DBLP:conf/icml/ChowdhuryC21} among our baselines, as they are not pretrained models.
In order to compare the two forms of inductive bias fairly and efficiently,
we pretrain \bert models with 4 layers and 12 layers as well as our Fast-R2D2 from scratch on the WikiText103 corpus following Section~\ref{sec:LM_setup}. 
Considering that longer inputs in the pre-training stage are helpful for BERT’s downstream task performance, we use the original corpus that is not split into sentences as inputs.
For simplicity, Fast-R2D2 is fine-tuned without span constraints.
Following the common settings, we add an MLP layer over the root representation of the R2D2 encoder for single-sentence classification. 
For cross-sentence tasks such as QQP and MNLI, we feed the root representations of the two sentences into the pretrained tree encoder of R2D2 as left and right inputs, 
and also add a new task ID as another input term to the R2D2 encoder. 
Then we feed the hidden output of the new task ID into another MLP layer to predict the final label.
We train all systems across the four datasets for 10 epochs 
with a learning rate of $5\times 10^{-5}$, batch size $64$, and maximum input length $200$.
We validate each model in each epoch and report the best results on development sets.

\begin{table}
\begin{center}
\setlength{\tabcolsep}{1.5pt}
\resizebox{0.48\textwidth}{!}{
\begin{tabular}{l|c|r r|r r}
\multirow{4}{*}{Model} & \multirow{4}{*}{Para.} & \multicolumn{2}{c|}{Single sent.} & \multicolumn{2}{c}{Cross sent.} \\
 &  & \begin{tabular}[c]{@{}l@{}}SST-2\\ (Acc.)\end{tabular} & \begin{tabular}[c]{@{}l@{}}CoLA\\ (Mcc.)\end{tabular} & \begin{tabular}[c]{@{}l@{}}QQP\\ (F1)\end{tabular} & \begin{tabular}[c]{@{}l@{}}MNLI\\m/mm\\ (Acc.)\end{tabular}            \\ \hline \hline
\bert (4L)  & 52M & 84.98 & 17.07 & 84.01 & 73.73/74.63 \\
\bert (12L) & 116M & 90.25 & 40.72 & 87.13 & 80.00/80.41 \\ \hline
R2D2        & 52M & 89.33 & 34.79 & 84.27 &  69.35/68.72 \\ \hline
Fast-R2D2$^\dag$& {\multirow{2}{*}{\begin{tabular}[c]{@{}c@{}}\\52M/\\ 10M\end{tabular}}} & 87.50 & 8.67 & 83.97 & 69.53/69.50 \\
Fast-R2D2$^*\dag$& {} & 88.30 & 10.14 & 84.07 & 69.36/69.11 \\
Fast-R2D2  & {} & 90.25 & 38.45 & 84.35 & 69.36/68.80 \\ 
Fast-R2D2$^*$& {} & 90.71 & 40.11 & 84.32 & 69.64/69.57\\
\hline \hline
\end{tabular}
}
\end{center}
\caption{Downstream results. All systems are pretrained from scratch on WikiText103.
        Para.\ describes the number of parameters for each model. Fast-R2D2 contains the R2D2 encoder and top-down parser, two components with 52M and 10M parameters, respectively.
        Mcc.\ stands for Matthew's correlation coefficient.
        Fast-R2D2 with $\dag$ are models fine-tuned without $\mathcal{L}_\mathrm{bilm}$ for an ablation study.
    }\vspace{-10pt}
\label{tbl:classification}
\end{table}
\subsubsection{Results and Discussion}
Table~\ref{tbl:classification} shows the corresponding scores on SST-2, CoLA, QQPl, and MNLI. 
In terms of the parameter size, our Fast-R2D2 model has 52M and 10M parameters for the R2D2 encoder and top-down parser, respectively.
It is clear that 12-layer \bert is significantly better than 4-layer \bert.
As mentioned in Section~\ref{sec:downstream}, Fast-R2D2 has two options to construct the final tree and representation for a given input sentence:
Fast-R2D2$^*$ uses the output tree from the top-down parser, while Fast-R2D2 uses the best tree inferred by the R2D2 encoder.
Similar to the results for unsupervised parsing, Fast-R2D2$^*$ in classification tasks again outperforms Fast-R2D2.
We hypothesize that trees generated by the top-down parser without Gumbel noise are more stable and reasonable.
Fast-R2D2 significantly outperforms 4-layer \bert and achieves competitive results compared to 12-layer \bert in single sentence classification tasks such as SST-2 and CoLA, but still performs significantly worse in the cross-sentence tasks. 
We believe this is an expected result, as there is no cross-attention mechanism in the inductive bias of Fast-R2D2. 
However, the performance of Fast-R2D2 on classification tasks shows that the inductive bias of R2D2 has higher parameter utilization than sequentially applied Transformers.
Importantly, we demonstrate that a Recursive Neural Network variant with an unsupervised parser can achieve comparable results to pretrained sequential Transformers even with fewer parameters and interpretable intermediate results, 
Hence, our Fast-R2D2 framework provides an alternative for NLP tasks.

\subsection{Speed Evaluation}
To assess the time cost, we mainly compare sequential Transformers and Fast-R2D2 in forced encoding on various sequence length ranges. We randomly select 1,000 sentences for each range from WikiText103 and report the average time consumption on a single A100 GPU. \bert is based on the open source Transformers library\footnote{\url{https://github.com/huggingface/transformers}} and R2D2 is based on the official code in \newcite{hu-etal-2021-r2d2}.\footnote{\url{https://github.com/alipay/StructuredLM_RTDT/tree/r2d2}}

\begin{table}% [htb!]
\small
\begin{center}
\setlength{\tabcolsep}{3.pt}
\resizebox{0.45\textwidth}{!}{
\begin{tabular}{l|rrrr}
\multirow{2}{*}{Model} & \multicolumn{4}{c}{Sequence Length Ranges} \\\cline{2-5}
      & \multicolumn{1}{c|}{0--50} & \multicolumn{1}{l|}{50--100} & \multicolumn{1}{l|}{100--200} & 200--500 \\ 
\hline
\bert (12L) & \multicolumn{1}{r|}{1.36}     & \multicolumn{1}{r|}{1.46}       & \multicolumn{1}{r|}{1.62}        & 2.38 \\ \hline
R2D2  & \multicolumn{1}{r|}{38.06}     & \multicolumn{1}{r|}{173.74}       & \multicolumn{1}{r|}{555.95}        &    ---     \\
Fast-R2D2  & \multicolumn{1}{r|}{4.67} & \multicolumn{1}{r|}{14.91} & \multicolumn{1}{r|}{39.73} & 150.26 \\
Fast-R2D2* & \multicolumn{1}{r|}{1.28} & \multicolumn{1}{r|}{2.96}  & \multicolumn{1}{r|}{5.56}  & 10.70 \\ 
\hline \hline
\end{tabular}
}
\end{center}
\caption{Inference time in seconds for various systems to process 1,000 sentences with a batch size of 50.}
\label{tbl:speed_test}
\end{table}

Table~\ref{tbl:speed_test} shows the inference time in seconds for different systems to process 1,000 sentences with a batch size of 50.
Running R2D2 is time-consuming, since the heuristic pruning method involves substantial memory exchanges between GPU and CPU. 
In Fast-R2D2, we alleviate this problem by using model-guided pruning to accelerate the chart table processing,
in conjunction with a code implementation in CUDA, Fast-R2D2 reduces the inference time significantly. 
Fast-R2D2$^{*}$ further improves the inference speed by running forced encoding in parallel over the binary tree generated by the parser, which is about 30--50 times faster than R2D2 in various ranges. 
Although there is still a gap in speed compared to sequential Transformers, Fast-R2D2$^{*}$ is sufficiently fast for most NLP tasks while producing interpretable intermediate representations.

\textbf{Related work}:
% Object detection related datasets/algo in non-medical domain
% Locally labeled CXR dataset
A few CXR datasets have localized abnormality annotations \cite{shih2019augmenting,filice2020crowdsourcing,jaeger2014two} that are curated manually. These are high quality gold standard ground truth datasets but tend to be smaller in scale (< 30,000 images) and have a narrow coverage, with typically only 1-2 labels. In addition, since most labeling efforts only have abnormality semantics attached, no direct relationships with the affected anatomical locations are available. 

%MEHDI: repeated concepts from above. I am removing the following: 

%The lack of anatomic semantics in the annotation is a limitation for complex multi-modal clinical reasoning work, e.g., differential diagnosis, since clinicians often integrate information along anatomical lines, and for downstream report generation tasks, which often requires describing not only the abnormality but also correctly communicate the location of the abnormalities (and medical devices) to the receiving clinicians. 

Two recent CXR datasets have labels for anatomies described in the reports. In \cite{datta2020dataset}, a small manually annotated dataset (2000 reports) included 10 abnormalities that are individually associated with 29 unique spatial locations (anatomies) at the report level. Another CXR dataset has automatically extracted abnormality and anatomy labels as disconnected concepts that are only correlated at the study level from  160,000 reports using a supervised NLP algorithm \cite{bustos2020padchest}. This was trained on a smaller set of manually annotated data. Neither datasets contain localized annotations for the associated CXR images, nor any comparison relation annotations between sequential exams, both of which are available in the Chest ImaGenome dataset. In Table \ref{tab:related}, we present a comparison of our Chest ImagGenome dataset with other datasets available in the literature.

% Table -- Kashyap

% MEdical imaging datasets to go here: Discussed that we will only focus on cxr datasets that are available for this paper. 
% \caption{\color{red} Kashyap, feel free to continue with the table. We should remove the questionmarks and add a line for our dataset (since all others are not graph). For longer text, using abbreviations and explaining them in the caption often works better. If fill in the values is not possible, it is better to remove the table altogether.}


\begin{table}[t!]
\caption{Summary of existing chest X-ray datasets}
\resizebox{\textwidth}{!}{%
\begin{tabular}{@{}lllllllll@{}}
\toprule
\textbf{Dataset} & \textbf{Annotation Level} & \textbf{Annotation Method} & \textbf{Num Labels} & \textbf{Anatomy Labeled} & \textbf{Graph} & \textbf{Dataset Size} & \textbf{Temporal Labels} & \textbf{Reports} \\ \midrule
SIIM-ACR Pneumothorax Segmentation \cite{filice2020crowdsourcing} & Segmentation & Manual + augmented & 1 & No & No & 12,047 & No & No \\
RSNA Pneumonia Detection Challenge   \cite{shih2019augmenting} & Bounding Boxes & Manual & 1 & No & No & 30,000 & No & No \\
Indiana University Chest X-ray collection \cite{demner2016preparing} & Global & Automated & 10 & No & No & 3,813 & No & Yes \\
NIH CXR dataset \cite{wang2017chestx} & Global & Automated & 14 & No & No & 112,120 & No & No \\
PLCO \cite{team2000prostate} & Global & Automated & 24 & Yes & No & 236,000 & Yes & No \\
Stanford CheXpert \cite{irvin2019chexpert} & Global & Automated & 14 & No & No & 224,316 & No & No \\
MIMIC-CXR \cite{johnson2019mimic} & Global & Automated & 14 & No & No & 377,110 & No & Yes \\
Dutta \cite{datta2020dataset} & Global & Manual & 10 & Yes & Yes & 2,000 & No & Yes \\
PadChest \cite{bustos2020padchest} & Global & Manual + automated & 297 & Yes & No & 160,868 & No & Yes \\
Montgomery County Chest X-ray   \cite{jaeger2014two} & Segmentation & Manual & 1 & Yes & No & 138 & No & No \\
Shenzen Hospital Chest X-ray   \cite{jaeger2014two} & Segmentation & Manual & 1 & Yes & No & 662 & No & No \\  \hline \hline
\textbf{Chest ImaGenome} & Bounding Boxes & Automated & 131 & Yes & Yes & 242,072 & Yes & Yes \\
\bottomrule
\end{tabular}%
}
\label{tab:related}
\vspace{-0.4cm}
\end{table}
% removed (Derived from MIMIC-CXR \cite{johnson2019mimic}) % makes table really small


\section{Conclusions and Future Work}

This paper proposes a privacy aware recommendation framework based on privacy calculus theory to study what will happen if the platform gives users control over their data.
To avoid the great cost in online experiments, we propose to use reinforcement learning to simulate the users' privacy decision making under different platform mechanisms and recommendation models on public benchmark datasets.
The results show a well-designed data disclosure mechanism can perform much better than the popular ``all or nothing'' binary mechanism.
Our work provides some insights to improve current rough solutions in privacy protection regulations, e.g., opt-in under GDPR and opt-out under CCPA.

This paper only takes the first step in studying users' privacy decision making under different platform mechanisms, and several directions remain to be explored.
First, a more complex and accurate privacy cost function can help us better understand users' privacy decision making.
In this work we have modeled different users with their individual privacy sensitivity weights, and one may modify the privacy cost function on the effects of the users' trust towards the platform in the future.
Second, more sophisticated platform mechanisms are also worth exploring.
Recent mechanism design works also turn to the perspectives of deep neural network based mechanism designs, which can be explored with our proposed framework.
Last but not least, deploying online experiments and analyzing users' decisions in real-world can facilitate further researches.



\clearpage

\bibliographystyle{ACM-Reference-Format}
\bibliography{sample-base}



\end{document}
