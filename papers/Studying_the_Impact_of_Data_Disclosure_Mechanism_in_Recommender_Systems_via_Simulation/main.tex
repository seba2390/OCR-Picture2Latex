
\pdfoutput=1
\documentclass[acmsmall]{acmart}


%\usepackage[OT1,T1]{fontenc}

\usepackage[numbers,sort&compress]{natbib}
\renewcommand{\bibfont}{\footnotesize}
%\usepackage{cite}
%\usepackage{mystyle}
%%%%%%%%%%%%%%%%%%%%%%%%%%%%%%%%%%%%
\makeatletter

\usepackage{etex}

%%% Review %%%

\usepackage{zref-savepos}

\newcounter{mnote}%[page]
\renewcommand{\themnote}{p.\thepage\;$\langle$\arabic{mnote}$\rangle$}

\def\xmarginnote{%
  \xymarginnote{\hskip -\marginparsep \hskip -\marginparwidth}}

\def\ymarginnote{%
  \xymarginnote{\hskip\columnwidth \hskip\marginparsep}}

\long\def\xymarginnote#1#2{%
\vadjust{#1%
\smash{\hbox{{%
        \hsize\marginparwidth
        \@parboxrestore
        \@marginparreset
\footnotesize #2}}}}}

\def\mnoteson{%
\gdef\mnote##1{\refstepcounter{mnote}\label{##1}%
  \zsavepos{##1}%
  \ifnum20432158>\number\zposx{##1}%
  \xmarginnote{{\color{blue}\bf $\langle$\arabic{mnote}$\rangle$}}% 
  \else
  \ymarginnote{{\color{blue}\bf $\langle$\arabic{mnote}$\rangle$}}%
  \fi%
}
  }
\gdef\mnotesoff{\gdef\mnote##1{}}
\mnoteson
\mnotesoff








%%% Layout %%%

% \usepackage{geometry} % override layout
% \geometry{tmargin=2.5cm,bmargin=m2.5cm,lmargin=3cm,rmargin=3cm}
% \setlength{\pdfpagewidth}{8.5in} % overrides default pdftex paper size
% \setlength{\pdfpageheight}{11in}

\newlength{\mywidth}

%%% Conventions %%%

% References
\newcommand{\figref}[1]{Fig.~\ref{#1}}
\newcommand{\defref}[1]{Definition~\ref{#1}}
\newcommand{\tabref}[1]{Table~\ref{#1}}
% general
%\usepackage{ifthen,nonfloat,subfigure,rotating,array,framed}
\usepackage{framed}
%\usepackage{subfigure}
\usepackage{subcaption}
\usepackage{comment}
%\specialcomment{nb}{\begingroup \noindent \framed\textbf{n.b.\ }}{\endframed\endgroup}
%%\usepackage{xtab,arydshln,multirow}
% topcaption defined in xtab. must load nonfloat before xtab
%\PassOptionsToPackage{svgnames,dvipsnames}{xcolor}
\usepackage[svgnames,dvipsnames]{xcolor}
%\definecolor{myblue}{rgb}{.8,.8,1}
%\definecolor{umbra}{rgb}{.8,.8,.5}
%\newcommand*\mybluebox[1]{%
%  \colorbox{myblue}{\hspace{1em}#1\hspace{1em}}}
\usepackage[all]{xy}
%\usepackage{pstricks,pst-node}
\usepackage{tikz}
\usetikzlibrary{positioning,matrix,through,calc,arrows,fit,shapes,decorations.pathreplacing,decorations.markings,decorations.text}

\tikzstyle{block} = [draw,fill=blue!20,minimum size=2em]

% allow prefix to scope name
\tikzset{%
	prefix node name/.code={%
		\tikzset{%
			name/.code={\edef\tikz@fig@name{#1 ##1}}
		}%
	}%
}


\@ifpackagelater{tikz}{2013/12/01}{
	\newcommand{\convexpath}[2]{
		[create hullcoords/.code={
			\global\edef\namelist{#1}
			\foreach [count=\counter] \nodename in \namelist {
				\global\edef\numberofnodes{\counter}
				\coordinate (hullcoord\counter) at (\nodename);
			}
			\coordinate (hullcoord0) at (hullcoord\numberofnodes);
			\pgfmathtruncatemacro\lastnumber{\numberofnodes+1}
			\coordinate (hullcoord\lastnumber) at (hullcoord1);
		}, create hullcoords ]
		($(hullcoord1)!#2!-90:(hullcoord0)$)
		\foreach [evaluate=\currentnode as \previousnode using \currentnode-1,
		evaluate=\currentnode as \nextnode using \currentnode+1] \currentnode in {1,...,\numberofnodes} {
			let \p1 = ($(hullcoord\currentnode) - (hullcoord\previousnode)$),
			\n1 = {atan2(\y1,\x1) + 90},
			\p2 = ($(hullcoord\nextnode) - (hullcoord\currentnode)$),
			\n2 = {atan2(\y2,\x2) + 90},
			\n{delta} = {Mod(\n2-\n1,360) - 360}
			in 
			{arc [start angle=\n1, delta angle=\n{delta}, radius=#2]}
			-- ($(hullcoord\nextnode)!#2!-90:(hullcoord\currentnode)$) 
		}
	}
}{
	\newcommand{\convexpath}[2]{
		[create hullcoords/.code={
			\global\edef\namelist{#1}
			\foreach [count=\counter] \nodename in \namelist {
				\global\edef\numberofnodes{\counter}
				\coordinate (hullcoord\counter) at (\nodename);
			}
			\coordinate (hullcoord0) at (hullcoord\numberofnodes);
			\pgfmathtruncatemacro\lastnumber{\numberofnodes+1}
			\coordinate (hullcoord\lastnumber) at (hullcoord1);
		}, create hullcoords ]
		($(hullcoord1)!#2!-90:(hullcoord0)$)
		\foreach [evaluate=\currentnode as \previousnode using \currentnode-1,
		evaluate=\currentnode as \nextnode using \currentnode+1] \currentnode in {1,...,\numberofnodes} {
			let \p1 = ($(hullcoord\currentnode) - (hullcoord\previousnode)$),
			\n1 = {atan2(\x1,\y1) + 90},
			\p2 = ($(hullcoord\nextnode) - (hullcoord\currentnode)$),
			\n2 = {atan2(\x2,\y2) + 90},
			\n{delta} = {Mod(\n2-\n1,360) - 360}
			in 
			{arc [start angle=\n1, delta angle=\n{delta}, radius=#2]}
			-- ($(hullcoord\nextnode)!#2!-90:(hullcoord\currentnode)$) 
		}
	}
}

% circle around nodes

% typsetting math
\usepackage{qsymbols,amssymb,mathrsfs}
\usepackage{amsmath}
\usepackage[standard,thmmarks]{ntheorem}
\theoremstyle{plain}
\theoremsymbol{\ensuremath{_\vartriangleleft}}
\theorembodyfont{\itshape}
\theoremheaderfont{\normalfont\bfseries}
\theoremseparator{}
\newtheorem{Claim}{Claim}
\newtheorem{Subclaim}{Subclaim}
\theoremstyle{nonumberplain}
\theoremheaderfont{\scshape}
\theorembodyfont{\normalfont}
\theoremsymbol{\ensuremath{_\blacktriangleleft}}
\newtheorem{Subproof}{Proof}

\theoremnumbering{arabic}
\theoremstyle{plain}
\usepackage{latexsym}
\theoremsymbol{\ensuremath{_\Box}}
\theorembodyfont{\itshape}
\theoremheaderfont{\normalfont\bfseries}
\theoremseparator{}
\newtheorem{Conjecture}{Conjecture}

\theorembodyfont{\upshape}
\theoremprework{\bigskip\hrule}
\theorempostwork{\hrule\bigskip}
\newtheorem{Condition}{Condition}%[section]


%\RequirePckage{amsmath} loaded by empheq
\usepackage[overload]{empheq} % no \intertext and \displaybreak
%\usepackage{breqn}

\let\iftwocolumn\if@twocolumn
\g@addto@macro\@twocolumntrue{\let\iftwocolumn\if@twocolumn}
\g@addto@macro\@twocolumnfalse{\let\iftwocolumn\if@twocolumn}

%\empheqset{box=\mybluebox}
%\usepackage{mathtools}      % to polish math typsetting, loaded
%                                % by empeq
\mathtoolsset{showonlyrefs=false,showmanualtags}
\let\underbrace\LaTeXunderbrace % adapt spacing to font sizes
\let\overbrace\LaTeXoverbrace
\renewcommand{\eqref}[1]{\textup{(\refeq{#1})}} % eqref was not allowed in
                                       % sub/super-scripts
\newtagform{brackets}[]{(}{)}   % new tags for equations
\usetagform{brackets}
% defined commands:
% \shortintertext{}, dcases*, \cramped, \smashoperator[]{}

\usepackage[Smaller]{cancel}
\renewcommand{\CancelColor}{\color{Red}}
%\newcommand\hcancel[2][black]{\setbox0=\hbox{#2}% colored horizontal cross
%  \rlap{\raisebox{.45\ht0}{\color{#1}\rule{\wd0}{1pt}}}#2}



\usepackage{graphicx,psfrag}
\graphicspath{{figure/}{image/}} % Search path of figures

% for tabular
\usepackage{diagbox} % \backslashbox{}{} for slashed entries
%\usepackage{threeparttable} % threeparttable, \tnote{},
                                % tablenotes, and \item[]
%\usepackage{colortab} % \cellcolor[gray]{0.9},
%\rowcolor,\columncolor,
%\usepackage{colortab} % \LCC \gray & ...  \ECC \\

% typesetting codes
%\usepackage{maple2e} % need to use \char29 for ^
\usepackage{algorithm2e}
\usepackage{listings} 
\lstdefinelanguage{Maple}{
  morekeywords={proc,module,end, for,from,to,by,while,in,do,od
    ,if,elif,else,then,fi ,use,try,catch,finally}, sensitive,
  morecomment=[l]\#,
  morestring=[b]",morestring=[b]`}[keywords,comments,strings]
\lstset{
  basicstyle=\scriptsize,
  keywordstyle=\color{ForestGreen}\bfseries,
  commentstyle=\color{DarkBlue},
  stringstyle=\color{DimGray}\ttfamily,
  texcl
}
%%% New fonts %%%
\DeclareMathAlphabet{\mathpzc}{OT1}{pzc}{m}{it}
\usepackage{upgreek} % \upalpha,\upbeta, ...
%\usepackage{bbold}   % blackboard math
\usepackage{dsfont}  % \mathds

%%% Macros for multiple definitions %%%

% example usage:
% \multi{M}{\boldsymbol{#1}}  % defines \multiM
% \multi ABC.                 % defines \MA \MB and \MC as
%                             % \boldsymbol{A}, \boldsymbol{B} and
%                             % \boldsymbol{C} respectively.
% 
%  The last period '.' is necessary to terminate the macro expansion.
%
% \multi*{M}{\boldsymbol{#1}} % defines \multiM and \M
% \M{A}                       % expands to \boldsymbol{A}

\def\multi@nostar#1#2{%
  \expandafter\def\csname multi#1\endcsname##1{%
    \if ##1.\let\next=\relax \else
    \def\next{\csname multi#1\endcsname}     
    %\expandafter\def\csname #1##1\endcsname{#2}
    \expandafter\newcommand\csname #1##1\endcsname{#2}
    \fi\next}}

\def\multi@star#1#2{%
  \expandafter\def\csname #1\endcsname##1{#2}
  \multi@nostar{#1}{#2}
}

\newcommand{\multi}{%
  \@ifstar \multi@star \multi@nostar}

%%% new alphabets %%%

\multi*{rm}{\mathrm{#1}}
\multi*{mc}{\mathcal{#1}}
\multi*{op}{\mathop {\operator@font #1}}
% \multi*{op}{\operatorname{#1}}
\multi*{ds}{\mathds{#1}}
\multi*{set}{\mathcal{#1}}
\multi*{rsfs}{\mathscr{#1}}
\multi*{pz}{\mathpzc{#1}}
\multi*{M}{\boldsymbol{#1}}
\multi*{R}{\mathsf{#1}}
\multi*{RM}{\M{\R{#1}}}
\multi*{bb}{\mathbb{#1}}
\multi*{td}{\tilde{#1}}
\multi*{tR}{\tilde{\mathsf{#1}}}
\multi*{trM}{\tilde{\M{\R{#1}}}}
\multi*{tset}{\tilde{\mathcal{#1}}}
\multi*{tM}{\tilde{\M{#1}}}
\multi*{baM}{\bar{\M{#1}}}
\multi*{ol}{\overline{#1}}

\multirm  ABCDEFGHIJKLMNOPQRSTUVWXYZabcdefghijklmnopqrstuvwxyz.
\multiol  ABCDEFGHIJKLMNOPQRSTUVWXYZabcdefghijklmnopqrstuvwxyz.
\multitR   ABCDEFGHIJKLMNOPQRSTUVWXYZabcdefghijklmnopqrstuvwxyz.
\multitd   ABCDEFGHIJKLMNOPQRSTUVWXYZabcdefghijklmnopqrstuvwxyz.
\multitset ABCDEFGHIJKLMNOPQRSTUVWXYZabcdefghijklmnopqrstuvwxyz.
\multitM   ABCDEFGHIJKLMNOPQRSTUVWXYZabcdefghijklmnopqrstuvwxyz.
\multibaM   ABCDEFGHIJKLMNOPQRSTUVWXYZabcdefghijklmnopqrstuvwxyz.
\multitrM   ABCDEFGHIJKLMNOPQRSTUVWXYZabcdefghijklmnopqrstuvwxyz.
\multimc   ABCDEFGHIJKLMNOPQRSTUVWXYZabcdefghijklmnopqrstuvwxyz.
\multiop   ABCDEFGHIJKLMNOPQRSTUVWXYZabcdefghijklmnopqrstuvwxyz.
\multids   ABCDEFGHIJKLMNOPQRSTUVWXYZabcdefghijklmnopqrstuvwxyz.
\multiset  ABCDEFGHIJKLMNOPQRSTUVWXYZabcdefghijklmnopqrstuvwxyz.
\multirsfs ABCDEFGHIJKLMNOPQRSTUVWXYZabcdefghijklmnopqrstuvwxyz.
\multipz   ABCDEFGHIJKLMNOPQRSTUVWXYZabcdefghijklmnopqrstuvwxyz.
\multiM    ABCDEFGHIJKLMNOPQRSTUVWXYZabcdefghijklmnopqrstuvwxyz.
\multiR    ABCDEFGHIJKL NO QR TUVWXYZabcd fghijklmnopqrstuvwxyz.
\multibb   ABCDEFGHIJKLMNOPQRSTUVWXYZabcdefghijklmnopqrstuvwxyz.
\multiRM   ABCDEFGHIJKLMNOPQRSTUVWXYZabcdefghijklmnopqrstuvwxyz.
\newcommand{\RRM}{\R{M}}
\newcommand{\RRP}{\R{P}}
\newcommand{\RRe}{\R{e}}
\newcommand{\RRS}{\R{S}}
%%% new symbols %%%

%\newcommand{\dotgeq}{\buildrel \textstyle  .\over \geq}
%\newcommand{\dotleq}{\buildrel \textstyle  .\over \leq}
\newcommand{\dotleq}{\buildrel \textstyle  .\over {\smash{\lower
      .2ex\hbox{\ensuremath\leqslant}}\vphantom{=}}}
\newcommand{\dotgeq}{\buildrel \textstyle  .\over {\smash{\lower
      .2ex\hbox{\ensuremath\geqslant}}\vphantom{=}}}

\DeclareMathOperator*{\argmin}{arg\,min}
\DeclareMathOperator*{\argmax}{arg\,max}

%%% abbreviations %%%

% commands
\newcommand{\esm}{\ensuremath}

% environments
\newcommand{\bM}{\begin{bmatrix}}
\newcommand{\eM}{\end{bmatrix}}
\newcommand{\bSM}{\left[\begin{smallmatrix}}
\newcommand{\eSM}{\end{smallmatrix}\right]}
\renewcommand*\env@matrix[1][*\c@MaxMatrixCols c]{%
  \hskip -\arraycolsep
  \let\@ifnextchar\new@ifnextchar
  \array{#1}}



% sets of number
\newqsymbol{`N}{\mathbb{N}}
\newqsymbol{`R}{\mathbb{R}}
\newqsymbol{`P}{\mathbb{P}}
\newqsymbol{`Z}{\mathbb{Z}}

% symbol short cut
\newqsymbol{`|}{\mid}
% use \| for \parallel
\newqsymbol{`8}{\infty}
\newqsymbol{`1}{\left}
\newqsymbol{`2}{\right}
\newqsymbol{`6}{\partial}
\newqsymbol{`0}{\emptyset}
\newqsymbol{`-}{\leftrightarrow}
\newqsymbol{`<}{\langle}
\newqsymbol{`>}{\rangle}

%%% new operators / functions %%%

\newcommand{\sgn}{\operatorname{sgn}}
\newcommand{\Var}{\op{Var}}
\newcommand{\diag}{\operatorname{diag}}
\newcommand{\erf}{\operatorname{erf}}
\newcommand{\erfc}{\operatorname{erfc}}
\newcommand{\erfi}{\operatorname{erfi}}
\newcommand{\adj}{\operatorname{adj}}
\newcommand{\supp}{\operatorname{supp}}
\newcommand{\E}{\opE\nolimits}
\newcommand{\T}{\intercal}
% requires mathtools
% \abs,\abs*,\abs[<size_cmd:\big,\Big,\bigg,\Bigg etc.>]
\DeclarePairedDelimiter\abs{\lvert}{\rvert} 
\DeclarePairedDelimiter\norm{\lVert}{\rVert}
\DeclarePairedDelimiter\ceil{\lceil}{\rceil}
\DeclarePairedDelimiter\floor{\lfloor}{\rfloor}
\DeclarePairedDelimiter\Set{\{}{\}}
\newcommand{\imod}[1]{\allowbreak\mkern10mu({\operator@font mod}\,\,#1)}

%%% new formats %%%
\newcommand{\leftexp}[2]{{\vphantom{#2}}^{#1}{#2}}


% non-floating figures that can be put inside tables
\newenvironment{nffigure}[1][\relax]{\vskip \intextsep
  \noindent\minipage{\linewidth}\def\@captype{figure}}{\endminipage\vskip \intextsep}

\newcommand{\threecols}[3]{
\hbox to \textwidth{%
      \normalfont\rlap{\parbox[b]{\textwidth}{\raggedright#1\strut}}%
        \hss\parbox[b]{\textwidth}{\centering#2\strut}\hss
        \llap{\parbox[b]{\textwidth}{\raggedleft#3\strut}}%
    }% hbox 
}

\newcommand{\reason}[2][\relax]{
  \ifthenelse{\equal{#1}{\relax}}{
    \left(\text{#2}\right)
  }{
    \left(\parbox{#1}{\raggedright #2}\right)
  }
}

\newcommand{\marginlabel}[1]
{\mbox[]\marginpar{\color{ForestGreen} \sffamily \small \raggedright\hspace{0pt}#1}}


% up-tag an equation
\newcommand{\utag}[2]{\mathop{#2}\limits^{\text{(#1)}}}
\newcommand{\uref}[1]{(#1)}


% Notation table

\newcommand{\Hline}{\noalign{\vskip 0.1in \hrule height 0.1pt \vskip
    0.1in}}
  
\def\Malign#1{\tabskip=0in
  \halign to\columnwidth{
    \ensuremath{\displaystyle ##}\hfil
    \tabskip=0in plus 1 fil minus 1 fil
    &
    \parbox[t]{0.8\columnwidth}{##}
    \tabskip=0in
    \cr #1}}


%%%%%%%%%%%%%%%%%%%%%%%%%%%%%%%%%%%%%%%%%%%%%%%%%%%%%%%%%%%%%%%%%%%
% MISCELLANEOUS

% Modification from braket.sty by Donald Arseneau
% Command defined is: \extendvert{ }
% The "small versions" use fixed-size brackets independent of their
% contents, whereas the expand the first vertical line '|' or '\|' to
% envelop the content
\let\SavedDoubleVert\relax
\let\protect\relax
{\catcode`\|=\active
  \xdef\extendvert{\protect\expandafter\noexpand\csname extendvert \endcsname}
  \expandafter\gdef\csname extendvert \endcsname#1{\mskip-5mu \left.%
      \ifx\SavedDoubleVert\relax \let\SavedDoubleVert\|\fi
     \:{\let\|\SetDoubleVert
       \mathcode`\|32768\let|\SetVert
     #1}\:\right.\mskip-5mu}
}
\def\SetVert{\@ifnextchar|{\|\@gobble}% turn || into \|
    {\egroup\;\mid@vertical\;\bgroup}}
\def\SetDoubleVert{\egroup\;\mid@dblvertical\;\bgroup}

% If the user is using e-TeX with its \middle primitive, use that for
% verticals instead of \vrule.
%
\begingroup
 \edef\@tempa{\meaning\middle}
 \edef\@tempb{\string\middle}
\expandafter \endgroup \ifx\@tempa\@tempb
 \def\mid@vertical{\middle|}
 \def\mid@dblvertical{\middle\SavedDoubleVert}
\else
 \def\mid@vertical{\mskip1mu\vrule\mskip1mu}
 \def\mid@dblvertical{\mskip1mu\vrule\mskip2.5mu\vrule\mskip1mu}
\fi

%%%%%%%%%%%%%%%%%%%%%%%%%%%%%%%%%%%%%%%%%%%%%%%%%%%%%%%%%%%%%%%%

\makeatother

%%%%%%%%%%%%%%%%%%%%%%%%%%%%%%%%%%%%

\usepackage{ctable}
\usepackage{fouridx}
%\usepackage{calc}
\usepackage{framed}
\usetikzlibrary{positioning,matrix}

\usepackage{paralist}
%\usepackage{refcheck}
\usepackage{enumerate}

\usepackage[normalem]{ulem}
\newcommand{\Ans}[1]{\uuline{\raisebox{.15em}{#1}}}



\numberwithin{equation}{section}
\makeatletter
\@addtoreset{equation}{section}
\renewcommand{\theequation}{\arabic{section}.\arabic{equation}}
\@addtoreset{Theorem}{section}
\renewcommand{\theTheorem}{\arabic{section}.\arabic{Theorem}}
\@addtoreset{Lemma}{section}
\renewcommand{\theLemma}{\arabic{section}.\arabic{Lemma}}
\@addtoreset{Corollary}{section}
\renewcommand{\theCorollary}{\arabic{section}.\arabic{Corollary}}
\@addtoreset{Example}{section}
\renewcommand{\theExample}{\arabic{section}.\arabic{Example}}
\@addtoreset{Remark}{section}
\renewcommand{\theRemark}{\arabic{section}.\arabic{Remark}}
\@addtoreset{Proposition}{section}
\renewcommand{\theProposition}{\arabic{section}.\arabic{Proposition}}
\@addtoreset{Definition}{section}
\renewcommand{\theDefinition}{\arabic{section}.\arabic{Definition}}
\@addtoreset{Claim}{section}
\renewcommand{\theClaim}{\arabic{section}.\arabic{Claim}}
\@addtoreset{Subclaim}{Theorem}
\renewcommand{\theSubclaim}{\theTheorem\Alph{Subclaim}}
\makeatother

\newcommand{\Null}{\op{Null}}
%\newcommand{\T}{\op{T}\nolimits}
\newcommand{\Bern}{\op{Bern}\nolimits}
\newcommand{\odd}{\op{odd}}
\newcommand{\even}{\op{even}}
\newcommand{\Sym}{\op{Sym}}
\newcommand{\si}{s_{\op{div}}}
\newcommand{\sv}{s_{\op{var}}}
\newcommand{\Wtyp}{W_{\op{typ}}}
\newcommand{\Rco}{R_{\op{CO}}}
\newcommand{\Tm}{\op{T}\nolimits}
\newcommand{\JGK}{J_{\op{GK}}}

\newcommand{\diff}{\mathrm{d}}

\newenvironment{lbox}{
  \setlength{\FrameSep}{1.5mm}
  \setlength{\FrameRule}{0mm}
  \def\FrameCommand{\fboxsep=\FrameSep \fcolorbox{black!20}{white}}%
  \MakeFramed {\FrameRestore}}%
{\endMakeFramed}

\newenvironment{ybox}{
	\setlength{\FrameSep}{1.5mm}
	\setlength{\FrameRule}{0mm}
  \def\FrameCommand{\fboxsep=\FrameSep \fcolorbox{black!10}{yellow!8}}%
  \MakeFramed {\FrameRestore}}%
{\endMakeFramed}

\newenvironment{gbox}{
	\setlength{\FrameSep}{1.5mm}
\setlength{\FrameRule}{0mm}
  \def\FrameCommand{\fboxsep=\FrameSep \fcolorbox{black!10}{green!8}}%
  \MakeFramed {\FrameRestore}}%
{\endMakeFramed}

\newenvironment{bbox}{
	\setlength{\FrameSep}{1.5mm}
\setlength{\FrameRule}{0mm}
  \def\FrameCommand{\fboxsep=\FrameSep \fcolorbox{black!10}{blue!8}}%
  \MakeFramed {\FrameRestore}}%
{\endMakeFramed}

\newenvironment{yleftbar}{%
  \def\FrameCommand{{\color{yellow!20}\vrule width 3pt} \hspace{10pt}}%
  \MakeFramed {\advance\hsize-\width \FrameRestore}}%
 {\endMakeFramed}

\newcommand{\tbox}[2][\relax]{
 \setlength{\FrameSep}{1.5mm}
  \setlength{\FrameRule}{0mm}
  \begin{ybox}
    \noindent\underline{#1:}\newline
    #2
  \end{ybox}
}

\newcommand{\pbox}[2][\relax]{
  \setlength{\FrameSep}{1.5mm}
 \setlength{\FrameRule}{0mm}
  \begin{gbox}
    \noindent\underline{#1:}\newline
    #2
  \end{gbox}
}

\newcommand{\gtag}[1]{\text{\color{green!50!black!60} #1}}
\let\labelindent\relax
\usepackage{enumitem}

%%%%%%%%%%%%%%%%%%%%%%%%%%%%%%%%%%%%
% fix subequations
% http://tex.stackexchange.com/questions/80134/nesting-subequations-within-align
%%%%%%%%%%%%%%%%%%%%%%%%%%%%%%%%%%%%

\usepackage{etoolbox}

% let \theparentequation use the same definition as equation
\let\theparentequation\theequation
% change every occurence of "equation" to "parentequation"
\patchcmd{\theparentequation}{equation}{parentequation}{}{}

\renewenvironment{subequations}[1][]{%              optional argument: label-name for (first) parent equation
	\refstepcounter{equation}%
	%  \def\theparentequation{\arabic{parentequation}}% we patched it already :)
	\setcounter{parentequation}{\value{equation}}%    parentequation = equation
	\setcounter{equation}{0}%                         (sub)equation  = 0
	\def\theequation{\theparentequation\alph{equation}}% 
	\let\parentlabel\label%                           Evade sanitation performed by amsmath
	\ifx\\#1\\\relax\else\label{#1}\fi%               #1 given: \label{#1}, otherwise: nothing
	\ignorespaces
}{%
	\setcounter{equation}{\value{parentequation}}%    equation = subequation
	\ignorespacesafterend
}

\newcommand*{\nextParentEquation}[1][]{%            optional argument: label-name for (first) parent equation
	\refstepcounter{parentequation}%                  parentequation++
	\setcounter{equation}{0}%                         equation = 0
	\ifx\\#1\\\relax\else\parentlabel{#1}\fi%         #1 given: \label{#1}, otherwise: nothing
}

% hyperlink
\PassOptionsToPackage{breaklinks,letterpaper,hyperindex=true,backref=false,bookmarksnumbered,bookmarksopen,linktocpage,colorlinks,linkcolor=BrickRed,citecolor=OliveGreen,urlcolor=Blue,pdfstartview=FitH}{hyperref}
\usepackage{hyperref}

% makeindex style
\newcommand{\indexmain}[1]{\textbf{\hyperpage{#1}}}
\definecolor{tea_green}{RGB}{214, 234, 193}
\definecolor{hint_green}{RGB}{226,246,209}
\definecolor{Madang}{RGB}{190,235,159}
\definecolor{yellow_green}{RGB}{198,222,119}
\definecolor{link_water}{RGB}{221, 232, 250}
\definecolor{celestial_blue}{RGB}{52, 152, 219}
\definecolor{shakespeare}{RGB}{85, 154, 193}
\definecolor{buttermilk}{RGB}{255,242,174}
\definecolor{chardonnay}{RGB}{250,196,114}
\definecolor{rajah}{RGB}{253,180,98}
\definecolor{fog}{RGB}{213, 193, 234}
\definecolor{melon}{RGB}{254,191,181}
\definecolor{sundown}{RGB}{249, 180, 181}
\definecolor{mona_lisa}{RGB}{246,152,134}
\definecolor{salmon}{RGB}{242,131,107}


\definecolor{saltpan}{RGB}{238, 243, 232}
\definecolor{aqua_spring}{RGB}{232, 243, 232}
\definecolor{tea_green}{RGB}{214, 234, 193}
\definecolor{Madang}{RGB}{190,235,159}
\definecolor{fringy_flower}{RGB}{194, 234, 193}
\definecolor{aero_blue}{RGB}{193, 234, 213}
\definecolor{pixie_green}{RGB}{183,214,170}
\definecolor{french_pass}{RGB}{195,232,246}
\definecolor{ice_cold}{RGB}{169,232,220}
\definecolor{pale_turquoise}{RGB}{172,240,242}
\definecolor{cruise}{RGB}{179,226,205}
\definecolor{sail}{RGB}{163,205,235}
\definecolor{spindle}{RGB}{179,205,227}
\definecolor{link_water}{RGB}{221, 232, 250}
\definecolor{periwinkle}{RGB}{203,213,232}
\definecolor{zanah}{RGB}{220, 233, 213}
\definecolor{frostee}{RGB}{217, 231, 214}
\definecolor{opal}{RGB}{199, 221, 211}
\definecolor{jet_stream}{RGB}{188, 214, 210}
\definecolor{skeptic}{RGB}{153, 187, 167}
\definecolor{hint_green}{RGB}{226,246,209}
\definecolor{snow_flurry}{RGB}{230,245,201}
\definecolor{surf_crest}{RGB}{205,230,208}
\definecolor{yellow_green}{RGB}{198,222,119}
\definecolor{cream}{RGB}{255,255,204}
\definecolor{pale_prim}{RGB}{255,255,179}
\definecolor{spring_sun}{RGB}{242,243,195}
\definecolor{portafino}{RGB}{245,237,160}
\definecolor{buttermilk}{RGB}{255,242,174}
\definecolor{cream_brulee}{RGB}{255, 229, 151}
\definecolor{dairy_cream}{RGB}{254,226,189}
\definecolor{champagne}{RGB}{254,217,166}
\definecolor{chardonnay}{RGB}{250,196,114}
\definecolor{manhattan}{RGB}{226,180,125}
\definecolor{rajah}{RGB}{253,180,98}
\definecolor{early_dawn}{RGB}{252,243,218}
\definecolor{egg_shell}{RGB}{238, 234, 215}
\definecolor{selago}{RGB}{243, 232, 243}
\definecolor{quartz}{RGB}{219,223,238}
\definecolor{fog}{RGB}{213, 193, 234}
\definecolor{languid_lavender}{RGB}{222,203,228}
\definecolor{watusi}{RGB}{254,221,207}
\definecolor{coral_andy}{RGB}{243,204,205}
\definecolor{cosmos}{RGB}{248,209,210}
\definecolor{melon}{RGB}{254,191,181}
\definecolor{azalea}{RGB}{234, 193, 194}
\definecolor{beauty_bush}{RGB}{235, 185, 179}
\definecolor{sundown}{RGB}{249, 180, 181}
\definecolor{mona_lisa}{RGB}{246,152,134}
\definecolor{salmon}{RGB}{242,131,107}


\definecolor{summer_sky}{RGB}{58, 151, 233}
\definecolor{chateau_green}{RGB}{72, 179, 96}
\definecolor{matisse}{RGB}{25, 104, 167}
\definecolor{allports}{RGB}{31, 106, 125}
\definecolor{sun_shade}{RGB}{255, 144, 68}
\definecolor{flamingo}{RGB}{237, 88, 85}
\definecolor{studio}{RGB}{128, 91, 160}



\definecolor{maya_blue}{RGB}{102, 204, 255}
\definecolor{feijoa}{RGB}{178,223,138}
\definecolor{sushi}{RGB}{117, 168, 47}
\definecolor{norway}{RGB}{158, 194, 132}
\definecolor{japanese_laurel}{RGB}{53, 116, 40}
\definecolor{see_green}{RGB}{161,228,195}
\definecolor{monte_carlo}{RGB}{135,204,194}
\definecolor{granny_smith_apple}{RGB}{150,214,150}
\definecolor{moss_green}{RGB}{170,216,176}
\definecolor{chateau_green}{RGB}{72, 179, 96}
\definecolor{opal}{RGB}{164,207,190}
\definecolor{acapulco}{RGB}{117, 170, 148}
\definecolor{viridian}{RGB}{55, 137, 122}
\definecolor{amazon}{RGB}{56, 123, 84}
\definecolor{asparagus}{RGB}{123, 160, 91}
\definecolor{fruit_salad}{RGB}{91, 160, 94}
\definecolor{puerto_rico}{RGB}{72, 179, 150}
\definecolor{mountain_meadow}{RGB}{0, 163, 136}
\definecolor{matisse}{RGB}{25, 104, 167}
\definecolor{allports}{RGB}{31, 106, 125}
\definecolor{astral}{RGB}{55, 111, 137}
\definecolor{spring_leaves}{RGB}{46, 83, 117}
\definecolor{biscay}{RGB}{44, 62, 80}
\definecolor{midnight}{RGB}{0, 29, 50}
\definecolor{amethyst}{RGB}{153, 102, 204}
\definecolor{studio}{RGB}{128, 91, 160}
\definecolor{tapestry}{RGB}{194, 109, 132}
\definecolor{atomic_tangerine}{RGB}{255, 153, 102}
\definecolor{amber}{RGB}{255, 191, 0}
\definecolor{casablanca}{RGB}{244, 178, 84}
\definecolor{california}{RGB}{233, 140, 58}
\definecolor{tomato}{RGB}{255, 97, 56} 
\definecolor{alizarin}{RGB}{233, 58, 64}



\definecolor{linen}{RGB}{251, 239, 227}
\definecolor{double_pearl_lusta}{RGB}{253, 242, 208}
\definecolor{oasis}{RGB}{253, 242, 208}
\definecolor{milan}{RGB}{255, 254, 169}
\definecolor{texas}{RGB}{245, 232, 123}
\definecolor{maize}{RGB}{249, 212, 156}

\definecolor{turmeric}{RGB}{211, 178, 76}
\definecolor{saffron}{RGB}{249,193,62}
\definecolor{my_sin}{RGB}{255, 176, 59}
\definecolor{tree_poppy}{RGB}{246, 154, 27}
\definecolor{jaffa}{RGB}{240, 131, 58}
\definecolor{crusta}{RGB}{254, 127, 44}
\definecolor{tahiti_gold}{RGB}{223, 102, 36}
\definecolor{outrageous_orange}{RGB}{255, 100, 45}
\definecolor{safety_orange}{RGB}{254, 106, 0}


\definecolor{azalea}{RGB}{251, 196, 196}
\definecolor{oyster_pink}{RGB}{238,206,205} 
\definecolor{coral_candy}{RGB}{242,208,205} 
\definecolor{baby_pink}{RGB}{246, 194, 192}
\definecolor{petite_orchid}{RGB}{223, 157, 155}
\definecolor{apricot}{RGB}{241,140,122}
\definecolor{NY_pink}{RGB}{228,136,113}
\definecolor{carmine_pink}{RGB}{231, 76, 60}
\definecolor{deep_carmine_pink}{RGB}{236, 50, 67}

\definecolor{wewak}{RGB}{244, 143, 150}
\definecolor{light_coral}{RGB}{244, 127, 123}
\definecolor{bittersweet}{RGB}{255,111,105}
\definecolor{carnation}{RGB}{245, 80, 86}
\definecolor{flamingo}{RGB}{237, 88, 85}
\definecolor{sunset_orange}{RGB}{242,89,75}
\definecolor{ku_crimson}{RGB}{243, 0, 25}
\definecolor{amaranth}{RGB}{234,46,73}
\definecolor{valencia}{RGB}{214, 87, 70}
\definecolor{chilean_firegongs }{RGB}{215, 87, 44}
\definecolor{mexican_red}{RGB}{170, 41, 37}



\definecolor{napa}{RGB}{163, 154, 137}

\definecolor{athens_gray}{RGB}{236, 240, 241}
\definecolor{gallery}{RGB}{240,240,240}
\definecolor{mercury}{RGB}{230,230,230}
\definecolor{platinum}{RGB}{228,228,228}
\definecolor{silver}{RGB}{191,191,191}
\definecolor{aluminum}{RGB}{153,153,153}
\definecolor{ship_gray}{RGB}{77,77,77}
\definecolor{tuatara}{RGB}{67, 67, 67}

\definecolor{malibu}{RGB}{110, 180, 240}
\definecolor{celestial_blue}{RGB}{52, 152, 219}
\definecolor{curious_blue}{RGB}{41, 128, 185}
\definecolor{french_blue}{RGB}{0, 112, 182}
\definecolor{matisse}{RGB}{25, 104, 167}
\definecolor{shakespeare}{RGB}{85, 154, 193}
\definecolor{seagull}{RGB}{128,177,211}
\definecolor{jelly_bean}{RGB}{45, 126, 150}
\definecolor{venice_blue}{RGB}{87, 135, 105}
\definecolor{boston_blue}{RGB}{68, 147, 161}

\definecolor{turquoise}{RGB}{41,217,194}
\definecolor{java}{RGB}{2,190,196}
\definecolor{riptide}{RGB}{141,211,199}
\definecolor{mountain_meadow}{RGB}{0, 163, 136}
\definecolor{free_speech_aquamarine}{RGB}{0, 156, 114}

\definecolor{cosmic_latte}{RGB}{222, 247, 229}
\definecolor{chinook}{RGB}{163, 232, 178}
\definecolor{padua}{RGB}{121, 189, 143}
\definecolor{ocean_green}{RGB}{79, 176, 112}
\definecolor{pastel_green}{RGB}{107, 227, 135}
\definecolor{chateau_green}{RGB}{69, 191, 85}
\definecolor{RoyalBlue}{RGB}{69, 191, 85}
\definecolor{pigment_green}{RGB}{0, 175, 79}
\definecolor{fern}{RGB}{101,197,117}
\definecolor{killarney}{RGB}{56, 113, 66}

\newtheorem{assumption}{Assumption}


\DeclareMathOperator*{\argmax}{arg\,max}
\DeclareMathOperator*{\argmin}{arg\,min}
\newcommand{\si}[1]{{\scriptstyle \mathcal{S}_i#1}}
\newcommand{\di}[1]{{\scriptstyle \mathcal{D}_i#1}}

\newcommand{\czq}[1]{\textcolor{black}{{#1}}}

\renewcommand{\arraystretch}{1.2}


\AtBeginDocument{%
  \providecommand\BibTeX{{%
    \normalfont B\kern-0.5em{\scshape i\kern-0.25em b}\kern-0.8em\TeX}}}

\setcopyright{acmcopyright}
\copyrightyear{2022}
\acmYear{2022}
\acmDOI{XXXXXXX.XXXXXXX}


\acmJournal{TOIS}
\acmVolume{37}
\acmNumber{4}
\acmArticle{111}
\acmMonth{8}





\begin{document}

\title{Studying the Impact of Data Disclosure Mechanism in Recommender Systems via Simulation}

\author{Ziqian Chen}
\email{eric.czq@alibaba-inc.com}
\affiliation{%
  \department{Damo Academy}
  \institution{Alibaba Group}
  \city{Hangzhou}
  \state{Zhejiang}
  \postcode{311121}
  \country{China}
}

\author{Fei Sun}
\authornote{Fei Sun is the corresponding author and now works at ICT, CAS.}
\orcid{0000-0002-6146-148X}
\email{ofey.sf@alibaba-inc.com}
\affiliation{%
  \department{Damo Academy}
  \institution{Alibaba Group}
  \city{Beijing}
  \postcode{100102}
  \country{China}
}

\author{Yifan Tang}
\email{yifan.tang95@gmail.com}
\affiliation{%
  \department{Luohan Academy}
  \institution{Alibaba Group}
  \city{Hangzhou}
  \state{Zhejiang}
  \postcode{311121}
  \country{China}
}

\author{Haokun Chen}
\email{hankel.chk@alibaba-inc.com}
\affiliation{%
  \department{Damo Academy}
  \institution{Alibaba Group}
  \city{Hangzhou}
  \state{Zhejiang}
  \postcode{311121}
  \country{China}
}

\author{Jinyang Gao}
\email{jinyang.gjy@alibaba-inc.com}
\affiliation{%
  \department{Damo Academy}
  \institution{Alibaba Group}
  \city{Hangzhou}
  \state{Zhejiang}
  \postcode{311121}
  \country{China}
}


\author{Bolin Ding}
\email{bolin.ding@alibaba-inc.com}
\affiliation{%
  \department{Damo Academy}
  \institution{Alibaba Group}
  \city{Seattle}
  \state{WA}
  \postcode{98004}
  \country{United States}
}


\begin{abstract}

Recently, privacy issues in web services that rely on users' personal data have raised great attention.
Despite that recent regulations force companies to offer choices for each user to opt-in or opt-out of data disclosure, real-world applications usually only provide an ``all or nothing'' binary option for users to either disclose all their data or preserve all data with the cost of no personalized service.


In this paper, we argue that such a binary mechanism is not optimal for both consumers and platforms.
To study how different privacy mechanisms affect users' decisions on information disclosure and how users' decisions affect the platform's revenue, we propose a privacy aware recommendation framework that gives users fine control over their data. 
In this new framework, users can proactively control which data to disclose based on the trade-off between anticipated privacy risks and potential utilities.
Then we study the impact of different data disclosure mechanisms via simulation with reinforcement learning due to the high cost of real-world experiments.
The results show that the platform mechanisms with finer split granularity and more unrestrained disclosure strategy can bring better results for both consumers and platforms than the ``all or nothing''  mechanism adopted by most real-world applications.


\end{abstract}


\begin{CCSXML}
<ccs2012>
<concept>
<concept_id>10002951.10003317.10003347.10003350</concept_id>
<concept_desc>Information systems~Recommender systems</concept_desc>
<concept_significance>500</concept_significance>
</concept>
  <concept>
  <concept_id>10002978.10003029.10011150</concept_id>
  <concept_desc>Security and privacy~Privacy protections</concept_desc>
  <concept_significance>500</concept_significance>
  </concept>
</ccs2012>
\end{CCSXML}
  
\ccsdesc[500]{Information systems~Recommender systems}
\ccsdesc[500]{Security and privacy~Privacy protections}

\keywords{Recommender System; Privacy; GDPR}



\maketitle

% !TEX root = ../arxiv.tex

Unsupervised domain adaptation (UDA) is a variant of semi-supervised learning \cite{blum1998combining}, where the available unlabelled data comes from a different distribution than the annotated dataset \cite{Ben-DavidBCP06}.
A case in point is to exploit synthetic data, where annotation is more accessible compared to the costly labelling of real-world images \cite{RichterVRK16,RosSMVL16}.
Along with some success in addressing UDA for semantic segmentation \cite{TsaiHSS0C18,VuJBCP19,0001S20,ZouYKW18}, the developed methods are growing increasingly sophisticated and often combine style transfer networks, adversarial training or network ensembles \cite{KimB20a,LiYV19,TsaiSSC19,Yang_2020_ECCV}.
This increase in model complexity impedes reproducibility, potentially slowing further progress.

In this work, we propose a UDA framework reaching state-of-the-art segmentation accuracy (measured by the Intersection-over-Union, IoU) without incurring substantial training efforts.
Toward this goal, we adopt a simple semi-supervised approach, \emph{self-training} \cite{ChenWB11,lee2013pseudo,ZouYKW18}, used in recent works only in conjunction with adversarial training or network ensembles \cite{ChoiKK19,KimB20a,Mei_2020_ECCV,Wang_2020_ECCV,0001S20,Zheng_2020_IJCV,ZhengY20}.
By contrast, we use self-training \emph{standalone}.
Compared to previous self-training methods \cite{ChenLCCCZAS20,Li_2020_ECCV,subhani2020learning,ZouYKW18,ZouYLKW19}, our approach also sidesteps the inconvenience of multiple training rounds, as they often require expert intervention between consecutive rounds.
We train our model using co-evolving pseudo labels end-to-end without such need.

\begin{figure}[t]%
    \centering
    \def\svgwidth{\linewidth}
    \input{figures/preview/bars.pdf_tex}
    \caption{\textbf{Results preview.} Unlike much recent work that combines multiple training paradigms, such as adversarial training and style transfer, our approach retains the modest single-round training complexity of self-training, yet improves the state of the art for adapting semantic segmentation by a significant margin.}
    \label{fig:preview}
\end{figure}

Our method leverages the ubiquitous \emph{data augmentation} techniques from fully supervised learning \cite{deeplabv3plus2018,ZhaoSQWJ17}: photometric jitter, flipping and multi-scale cropping.
We enforce \emph{consistency} of the semantic maps produced by the model across these image perturbations.
The following assumption formalises the key premise:

\myparagraph{Assumption 1.}
Let $f: \mathcal{I} \rightarrow \mathcal{M}$ represent a pixelwise mapping from images $\mathcal{I}$ to semantic output $\mathcal{M}$.
Denote $\rho_{\bm{\epsilon}}: \mathcal{I} \rightarrow \mathcal{I}$ a photometric image transform and, similarly, $\tau_{\bm{\epsilon}'}: \mathcal{I} \rightarrow \mathcal{I}$ a spatial similarity transformation, where $\bm{\epsilon},\bm{\epsilon}'\sim p(\cdot)$ are control variables following some pre-defined density (\eg, $p \equiv \mathcal{N}(0, 1)$).
Then, for any image $I \in \mathcal{I}$, $f$ is \emph{invariant} under $\rho_{\bm{\epsilon}}$ and \emph{equivariant} under $\tau_{\bm{\epsilon}'}$, \ie~$f(\rho_{\bm{\epsilon}}(I)) = f(I)$ and $f(\tau_{\bm{\epsilon}'}(I)) = \tau_{\bm{\epsilon}'}(f(I))$.

\smallskip
\noindent Next, we introduce a training framework using a \emph{momentum network} -- a slowly advancing copy of the original model.
The momentum network provides stable, yet recent targets for model updates, as opposed to the fixed supervision in model distillation \cite{Chen0G18,Zheng_2020_IJCV,ZhengY20}.
We also re-visit the problem of long-tail recognition in the context of generating pseudo labels for self-supervision.
In particular, we maintain an \emph{exponentially moving class prior} used to discount the confidence thresholds for those classes with few samples and increase their relative contribution to the training loss.
Our framework is simple to train, adds moderate computational overhead compared to a fully supervised setup, yet sets a new state of the art on established benchmarks (\cf \cref{fig:preview}).

\section{Framework}
\label{sec:framework}


\graphicspath{{./}{./images_rss_2015/}{./images_hri_2016/}}


%Under this:
% Shared autonomy assumes $\transition(\stateenv' \given \stateenv, \actionuser, \actionrobot) = \transition(\stateenv' \given \stateenv, 0, \actionrobot)$ - that is, the user does not directly affect the state.
% collab doesn't really care about the user state in cost 


We present our framework for minimizing a cost function for shared autonomy with an unknown user goal. We assume the user's goal is fixed, and they take actions to achieve that goal without considering autonomous assistance. These actions are used to predict the user's goal based on how optimal the action is for each goal (\cref{sec:framework_prediction}). Our system uses this distribution to minimize the expected cost-to-go  (\cref{sec:framework_unknown_goal}). As solving for the optimal action is infeasible, we use hindsight optimization to approximate a solution (\cref{sec:framework_hindsight}). For reference, see \cref{table:variable_definitions} in \cref{sec:variable_definitions} for variable definitions.

\subsection{Cost minimization with a known goal}
\label{sec:framework_known_goal}

We first formulate the problem for a known user goal, which we will use in our solution with an unknown goal. We model this problem as a Markov Decision Process (MDP). 

Formally, let $\stateenv \in \Stateenv$ be the environment state (e.g. human and robot pose). Let $\actionuser \in \Actionuser$ be the user actions, and $\actionrobot \in \Actionrobot$ the robot actions. Both agents can affect the environment state - if the user takes action $\actionuser$ and the robot takes action $\actionrobot$ while in state $\stateenv$, the environment stochastically transitions to a new state $\stateenv'$ through $\transitionallargs$. 

We assume the user has an intended goal $\goal \in \Goal$ which does not change during execution. We augment the environment state with this goal, defined by $\state = \left(\stateenv, \goal\right) \in \Stateenv \times \Goal$. We overload our transition function to model the transition in environment state without changing the goal, $\transition( (\stateenv', g) \given (\stateenv, g), \actionuser, \actionrobot) = \transitionallargs$.

%In our scenario, the user wants to move the robot to one goal in a discrete set of goals $\goal \in \Goal$.
We assume access to a user policy for each goal $\policyuser(\actionuser \given \state) = \policyusergoal(\actionuser \given \stateenv) = p(\actionuser \given \stateenv, \goal)$. We model this policy using the maximum entropy inverse optimal control (MaxEnt IOC) framework of~\citet{ziebart_2008}, where the policy corresponds to stochastically optimizing a cost function $\costuser(\state, \actionuser) = \costusergoal(\stateenv, \actionuser)$. We assume the user selects actions based only on $\state$, the current environment state and their intended goal, and does not model any actions that the robot might take. Details are in \cref{sec:framework_prediction}.

The robot selects actions to minimize a cost function dependent on the user goal and action $\costrobot(\state, \actionuser, \actionrobot) = \costrobotgoal(\stateenv, \actionuser, \actionrobot)$. At each time step, we assume the user first selects an action, which the robot observes before selecting $\actionrobot$. The robot selects actions based on the state and user inputs through a policy $\policyrobot(\actionrobot \given \state, \actionuser) = p(\actionrobot \given \state, \actionuser)$. We define the value function for a robot policy $\vrobot^{\policyrobot}$ as the expected cost-to-go from a particular state, assuming some user policy $\policyuser$:
\begin{align*}
  \vrobot^{\policyrobot}(\state) &= \expctarg{\sumtime \costrobot(\state_t, \actionuser_t, \actionrobot_t) \given \state_0 = \state}\\
  \actionuser_t &\sim \policyuser(\cdot \given \state_t)\\
  \actionrobot_t &\sim \policyrobot(\cdot \given \state_t, \actionuser_t)\\
  \state_{t+1} &\sim \transition(\cdot \given \state_t, \actionuser_t, \actionrobot_t)
\end{align*}
%\vrobot^{\policyrobot}(\state) &= \expctover{\policyuser, \policyrobot, \transition}{\sumtime \costrobot(\state_t, \actionuser_t, \actionrobot_t) \given \state_0 = \state}\\

The optimal value function $\vopt$ is the cost-to-go for the best robot policy:
\begin{align*}
  \vopt(\state) &= \min_{\policyrobot} \vrobot^{\policyrobot}(\state)
\end{align*}

The action-value function $\qopt$ computes the immediate cost of taking action $\actionrobot$ after observing $\actionuser$, and following the optimal policy thereafter:
\begin{align*}
  \qopt(\stateactions) &= \costrobot(\stateactions) + \expctarg{\vopt(\state')}
\end{align*}
%\qopt(\stateactions) &= \costrobot(\stateactions) + \expctover{\state'}{\vopt(\state')}
Where $\state' \sim \transition(\cdot \given \stateactions)$. The optimal robot action is given by $\argmin_\actionrobot \qopt(\stateactions)$.

In order to make explicit the dependence on the user goal, we often write these quantities as:
\begin{align*}
  \vgoal(\stateenv) &= \vopt(\state)\\
  \qgoal(\stateenvactions) &= \qopt(\stateactions)
\end{align*}

Computing the optimal policy and corresponding action-value function is a common objective in reinforcement learning. We assume access to this function in our framework, and describe our particular implementation in the experiments.

%The action-value function $\qrobot$ is defined as the immediate cost of taking action $\actionrobot$ after observing $\actionuser$, plus the cost of following $\policyrobot$ thereafter:
%\begin{align*}
%  \qrobot^{\policyrobot}(\stateactions) &= \costrobot(\stateactions) + \expctover{\state'}{\vrobot^{\policyrobot}(\state')}
%\end{align*}
%
%We define the optimal value and action-value functions as the cost-to-go for the best robot policy:
%\begin{align*}
%  \vopt(\state) &= \min_{\policyrobot} \vrobot^{\policyrobot}(\state)\\
%  \qopt(\stateactions) &= \costrobot(\stateactions) + \expctover{\state'}{\vopt(\state)}
%\end{align*}


\subsection{Cost Minimization with an unknown goal}
\label{sec:framework_unknown_goal}

We formulate the problem of minimizing a cost function with an unknown user goal as a Partially Observable Markov Decision Process (POMDP). A POMDP maps a distribution over states, known as the \emph{belief} $\belief$, to actions. We assume that all uncertainty is over the user's goal, and the environment state is known. This subclass of POMDPs, where uncertainty is constant, has been studied as a Hidden Goal MDP~\citep{fern_2010}, and as a POMDP-lite~\citep{chen_2016}.

In this framework, we infer a distribution of the user's goal by observing the user actions $\actionuser$. Similar to the known-goal setting (\cref{sec:framework_known_goal}), we define the value function of a belief as:
\begin{align*}
  \vrobot^{\policyrobot}(\belief) &= \expctarg{\sumtime \costrobot(\state_t, \actionuser_t, \actionrobot_t)  \given \belief_0 = \belief} \\
  \state_t &\sim \belief_t\\
  \actionuser_t &\sim \policyuser(\cdot \given \state_t)\\
  \actionrobot_t &\sim \policyrobot(\cdot \given \state_t, \actionuser_t)\\
  \belief_{t+1} &\sim \transitionbelief(\cdot \given \belief_t, \actionuser_t, \actionrobot_t)
\end{align*}
%\vrobot^{\policyrobot}(\belief) &= \expctover{\belief, \policyuser, \policyrobot, \transition}{\sumtime \costrobot(\state_t, \actionuser_t, \actionrobot_t)  \given \belief_0 = \belief} \\
Where the belief transition $\transitionbelief$ corresponds to transitioning the known environment state $\stateenv$ according to $\transition$, and updating our belief over the user's goal as described in $\cref{sec:framework_prediction}$. We can define quantities similar to above over beliefs:
\begin{align}
  \vopt(\belief) &= \min_{\policyrobot} \vrobot^{\policyrobot}(\belief) \label{eq:v_belief}\\
  \qopt(\beliefactions) &= \expctarg{\costrobot(\belief, \actionuser, \actionrobot) + \expctover{\belief'}{\vopt(\belief')}} \nonumber
\end{align}
%\qopt(\beliefactions) &= \expctover{\belief}{\costrobot(\belief, \actionuser, \actionrobot) + \expctover{\belief'}{\vopt(\belief')}}%\\&= \expctover{\belief}{\costrobot(\belief, \actionuser, \actionrobot)} + \min_\policyrobot \expctover{\belief}{\expctover{\belief'}{\vopt(\belief')}}


%\begin{align*}
%  \qrobot^{\policyrobot}(\beliefactions) &= \expctover{\belief}{\costrobot(\beliefactions) + \expctover{\belief'}{\vrobot^{\policyrobot}(\belief')}}
%\end{align*}




%As the robot does not know the user's goal a priori, we infer this goal from the user actions $\actionuser$ based on our models $\policyuser(\state)$.

%We can model the robots objective of minimizing a cost function with uncertainty using a Partially Observable Markov Decision Process (POMDP) with uncertainty over the user's goal. A POMDP maps a distribution over states, known as the \emph{belief} $\belief$, to actions. We assume that all uncertainty is over the user's goal, and the environment state is known, as in a Hidden Goal MDP~\citep{fern_2010}. Note that allowing the cost to depend on the observation $\actionuser$ is non-standard, but important for shared autonomy, as prior works suggest that users prefer maintaining control authority~\citep{kim_2012}. Our shared autonomy POMDP is defined by the tuple $\left(\Staterobgoal, \Actionrobot, \transition, \costrobot, \Actionuser, \policyuser, \right)$. The optimal solution to this POMDP minimizes the expected accumulated cost $\costrobot$. As this is intractable to compute, we utilize Hindsight Optimization to select actions, described in \cref{sec:framework_hindsight}.





%%We model the robot as a deterministic dynamical system with transition function $\transition: \Stateenv \times \Actionrobot \rightarrow \Stateenv$. %  where applying action $\actionrobot$ in state $\stateenv$ results in state $\stateenv'$.
%%The user supplies continuous inputs $\actionuser \in \Actionuser$ via an interface (e.g. joystick, mouse). These user inputs map to robot actions through a known deterministic function $\userinputtoaction: \Actionuser \rightarrow \Actionrobot$, corresponding to the effect of \emph{direct teleoperation}.
%
%In our scenario, the user wants to move the robot to one goal in a discrete set of goals $\goal \in \Goal$. We assume access to a stochastic user policy for each goal $\policyusergoal(\stateenv) = p(\actionuser | \stateenv, \goal)$, usually learned from user demonstrations. %Here, the user assumes inputs get mapped directly to actions through $\userinputtoaction$ - thus, they assume direct teleoperation.
%In our system, we model this policy using the maximum entropy inverse optimal control (MaxEnt IOC) framework of~\citet{ziebart_2008}, which assumes the user is approximately optimizing some cost function for their intended goal $g$, $\costusergoal: \Stateenv \times \Actionuser \rightarrow \mathcal{R}$. This model corresponds to a goal specific Markov Decision Process (MDP), defined by the tuple $\left(\Stateenv, \Actionuser, \transition, \costusergoal\right)$. We discuss details in \cref{sec:framework_prediction}. 
%
%Unlike the user, our system does not know the intended goal. We model this with a Partially Observable Markov Decision Process (POMDP) with uncertainty over the user's goal. A POMDP maps a distribution over states, known as the \emph{belief} $\belief$, to actions. Define the system state $\staterobgoal \in \Staterobgoal$ as the robot state augmented by a goal, $\staterobgoal = (\stateenv, \goal)$ and $\Staterobgoal = \Stateenv \times \Goal$. In a slight abuse of notation, we overload our transition function such that $\transition: \Staterobgoal \times \Actionrobot \rightarrow \Staterobgoal$, which corresponds to transitioning the robot state as above, but keeping the goal the same.
%
%In our POMDP, we assume the robot state is known, and all uncertainty is over the user's goal. Observations in our POMDP correspond to user inputs $\actionuser \in \Actionuser$. Given a sequence of user inputs, we infer a distribution over system states (equivalently a distribution over goals) using an observation model $\pomdpohm$. This corresponds to computing $\policyusergoal(\stateenv)$ for each goal, and applying Bayes' rule. We provide details in \cref{sec:framework_prediction}.
%
%The system uses cost function $\costrobot: \Staterobgoal \times \Actionrobot \times \Actionuser \rightarrow \mathcal{R}$, corresponding to the cost of taking robot action $\actionrobot$ when in system state $\staterobgoal$ and the user has input $\actionuser$. Note that allowing the cost to depend on the observation $\actionuser$ is non-standard, but important for shared autonomy, as prior works suggest that users prefer maintaining control authority~\citep{kim_2012}. This formulation enables us to penalize robot actions which deviate from $\userinputtoaction(\actionuser)$. Our shared autonomy POMDP is defined by the tuple $\left(\Staterobgoal, \Actionrobot, \transition, \costrobot, \Actionuser, \pomdpohm \right)$. The optimal solution to this POMDP minimizes the expected accumulated cost $\costrobot$. As this is intractable to compute, we utilize Hindsight Optimization to select actions, described in %\cref{sec:hindsight}.





\subsection{Hindsight Optimization}
\label{sec:framework_hindsight}

Computing the optimal solution for a POMDP with continuous states and actions is generally intractable. Instead, we approximate this quantity through \emph{Hindsight Optimization}~\citep{chong_2000,yoon_2008}, or QMDP~\citep{littman_1995}. This approximation estimates the value function by switching the order of the min and expectation in \cref{eq:v_belief}:
\begin{align*}
  \vhs(\belief) &= \expctover{\belief}{\min_{\policyrobot} \vrobot^{\policyrobot}(\state)}\\
  &= \expctover{\goal}{\vgoal(\stateenv)}\\
  \qhs(\beliefactions) &= \expctover{\belief}{\costrobot(\stateactions) + \expctover{\state'}{\vhs(\state')}}\\
  &= \expctover{\goal}{\qgoal(\stateenvactions)}
\end{align*}

Where we explicitly take the expectation over $\goal \in \Goal$, as we assume that is the only uncertain part of the state.

Conceptually, this approximation corresponds to assuming that all uncertainty will be resolved at the next timestep, and computing the optimal cost-to-go. As this is the best case scenario for our uncertainty, this is a lower bound of the cost-to-go, $\vhs(\belief) \leq \vopt(\belief)$. Hindsight optimization has demonstrated effectiveness in other domains~\citep{yoon_2007, yoon_2008}. However, as it assumes uncertainty will be resolved, it never explicitly gathers information~\citep{littman_1995}, and thus performs poorly when this is necessary.

We believe this method is suitable for shared autonomy for many reasons. Conceptually, we assume the user provides inputs at all times, and therefore we gain information without explicit information gathering. Works in other domains with similar properties have shown that this approximation performs comparably to methods that consider explicit information gathering~\citep{koval_2014}. Computationally, computing $\qhs$ can be done with continuous state and action spaces, enabling fast reaction to user inputs. 
%say that this is a lower bound on cost-to-go?

%Let $\qpomdp(\belief, \actionrobot, \actionuser)$ be the action-value function of the POMDP, estimating the cost-to-go of taking action $\actionrobot$ when in belief $\belief$ with user input $\actionuser$, and acting optimally thereafter. In our setting, uncertainty is only over goals, $\belief(\staterobgoal) = \belief(\goal) = p(\goal | \trajtot)$.

%Let $\qgoal(\staterobot, \actionrobot, \actionuser)$ correspond to the action-value for goal $\goal$, estimating the cost-to-go of taking action $\actionrobot$ when in state $\staterobot$ with user input $\actionuser$, and acting optimally for goal $\goal$ thereafter. The QMDP approximation is~\citep{littman_1995}:
%\begin{align*}
%  \qpomdp(\belief, \actionrobot, \actionuser) &= \sum_{\goal} \belief(\goal) \qgoal(\staterobot, \actionrobot, \actionuser)
%\end{align*}

Computing $\qgoal$ for shared autonomy requires utilizing the user policy $\policyusergoal$, which can make computation difficult. This can be alleviated with the following approximations:
\subsubsection*{Stochastic user with robot}
Estimate $\actionuser$ using $\policyusergoal$ at each time step, e.g. by sampling, and utilize the full cost function $\costrobotgoal(\stateenvactions)$ and transition function $\transitionallargs$ to compute $\qgoal$. This would be the standard QMDP approach for our POMDP.

\subsubsection*{Deterministic user with robot}
Estimate $\actionuser$ as the most likely $\actionuser$ from $\policyusergoal$ at each time step, and utilize the full cost function $\costrobotgoal(\stateenvactions)$ and transition function $\transitionallargs$ to compute $\qgoal$. This uses our policy predictor, as above, but does so deterministically, and is thus more computationally efficient.

\subsubsection*{Robot takes over}
Assume the user will stop supplying inputs, and the robot will complete the task. This enables us to use the cost function $\costrobotgoal(\stateenv, 0, \actionrobot)$ and transition function $\transition(\stateenv' \given \stateenv, 0, \actionrobot)$ to compute $\qgoal$. For many cost functions, we can analytically compute this value, e.g. cost of always moving towards the goal at some velocity. An additional benefit of this method is that it makes no assumptions about the user policy $\policyusergoal$, making it more robust to modelling errors. We use this method in our experiments.

Finally, as we often cannot calculate $\argmax_{\actionrobot} \qhs(\beliefactions)$ directly, we use a first-order approximation, which leads to us to following the gradient of $\qhs(\beliefactions)$.
%In cases were an action exists to assist for all goals, this approximation will take that action. When there aren't any such actions, the output will look similar to a blending between the user control and our assistance strategy, solving for the parameters of blending based on the cost functions. This sort of blending has been shown to be effective in the past~\citep{dragan_2013_assistive}. See \figref{fig:teledata}.


%add something about 1st order approximation for continuous systems?

%Maybe more specifics for our system? 
%-First order approx for qmdp
%-we optimize directly for user's value function
%---actually, we aren't fully solving the POMDP assuming user is optimal

\subsection{User Prediction}
\label{sec:framework_prediction}

In order to infer the user's goal, we rely on a model $\policyusergoal$ to provide the distribution of user actions at state $\stateenv$ for user goal $\goal$. In principle, we could use any generative predictor for this model, e.g.~\citep{koppula_2013, wang_2013_intentioninference}. We choose to use maximum entropy inverse optimal control (MaxEnt IOC)~\citep{ziebart_2008}, as it explicitly models a user cost function $\costusergoal$. We optimize this directly by defining $\costrobotgoal$ as a function of $\costusergoal$.

In this work, we assume the user does not model robot actions. We use this assumption to define an MDP with states $\stateenv \in \Stateenv$ and user actions $\actionuser \in \Actionuser$ as before, transition $\transitionuser(\stateenv' \given \stateenv, \actionuser) = \transition(\stateenv' \given \stateenv, \actionuser, 0)$, and cost $\costusergoal(\stateenv, \actionuser)$. MaxEnt IOC computes a stochastically optimal policy for this MDP.

The distribution of actions at a single state are computed based on how optimal that action is for minimizing cost over a horizon $T$. Define a sequence of environment states and user inputs as $\traj = \left\{ \stateenv_0, \actionuser_0, \cdots, \stateenv_T, \actionuser_T \right\}$. Note that sequences are not required to be trajectories, in that $\stateenv_{t+1}$ is not necessarily the result of applying $\actionuser_t$ in state $\stateenv_t$. Define the cost of a sequence as the sum of costs of all state-input pairs, $\costgoaluser(\traj) = \sum_{t} \costgoaluser(\stateenv_t, \actionuser_t)$. Let $\trajtot$ be a sequence from time $0$ to $t$, and $\trajat{\stateenv}$ a sequence of from time $t$ to $T$, starting at $\stateenv$.

\citet{ziebart_thesis} shows that minimizing the worst-case predictive loss results in a model where the probability of a sequence decreases exponentially with cost, $p(\traj \given \goal) \propto \exp(-\costgoaluser(\traj))$. Importantly, one can efficiently learn a cost function consistent with this model from demonstrations~\citep{ziebart_2008}.

Computationally, the difficulty in computing $p(\traj \given \goal)$ lies in the normalizing constant $\int_{\traj} \exp(-\costgoaluser(\traj))$, known as the partition function. Evaluating this explicitly would require enumerating all sequences and calculating their cost. However, as the cost of a sequence is the sum of costs of all state-action pairs, dynamic programming can be utilized to compute this through soft-minimum value iteration when the state is discrete~\citep{ziebart_2009,ziebart_2012}:
\begin{align*}
  \qgoalsoftt{t}(\stateenv, \actionuser) &= \costgoaluser(\stateenv, \actionuser) + \expctarg{\vgoalsoftt{t+1}(\stateenv')}\\
  \vgoalsoftt{t}(\stateenv) &= \softmin_{\actionuser} \qgoalsoftt{t}(\stateenv, \actionuser)
\end{align*}
Where $\softmin_{x} f(x) = - \log \int_{x} \exp(-f(x)) dx$ and $\stateenv' \sim \transitionuser(\cdot \given \stateenv, \actionuser)$.

The log partition function is given by the soft value function, $\vgoalsoftt{t}(\stateenv) = - \log \int_{\trajat{\stateenv}} \exp\left(-\costgoaluser(\trajat{\stateenv})\right)$, where the integral is over all sequences starting at $\stateenv$ and time $t$. Furthermore, the probability of a single input at a given environment state is given by $\policyuser_t(\actionuser \given \stateenv, \goal) = \exp(\vgoalsoftt{t}(\stateenv) -\qgoalsoftt{t}(\stateenv, \actionuser))$~\citep{ziebart_2009}.

%make more clear that while our user policy doesn't consider robot assistance, it still affects this positive feedback thing
Many works derive a simplification that enables them to only look at the start and current states, ignoring the inputs in between~\citep{ziebart_2012, dragan_2013_assistive}. Key to this assumption is that $\traj$ corresponds to a trajectory, where applying action $\actionuser_t$ at $\stateenv_t$ results in $\stateenv_{t+1}$. However, if the system is providing assistance, this may not be the case. In particular, if the assistance strategy believes the user's goal is $\goal$, the assistance strategy will select actions to minimize $\costusergoal$. Applying these simplifications will result positive feedback, where the robot makes itself more confident about goals it already believes are likely. In order to avoid this, we ensure that the prediction comes from user inputs only, and not robot actions:
\begin{align*}
  p(\traj \given \goal) &= \prod_t \policyuser_t(\actionuser_{t} \given \stateenv_t, \goal)
\end{align*}
%Where the user applied input $\actionuser_t$ at state $\state_t$.
To compute the probability of a goal given the partial sequence up to $t$, we apply Bayes' rule:
\begin{align*}
  p(\goal \given \trajtot) &= \frac{p(\trajtot \given \goal) p(\goal) }{\sum_{\goal'} p(\trajtot \given \goal') p(\goal')}
\end{align*}
This corresponds to our POMDP observation model, used to transition our belief over goals through $\transitionbelief$.


\subsubsection{Continuous state and action approximation}
Soft-minimum value iteration is able to find the exact partition function when states and actions are discrete. However, it is computationally intractable to apply in continuous state and action spaces. Instead, we follow \citet{dragan_2013_assistive} and use a second order approximation about the optimal trajectory. They show that, assuming a constant Hessian, we can replace the difficult to compute soft-min functions $\vgoalsoft$ and $\qgoalsoft$ with the min value and action-value functions $\vgoaluser$ and $\qgoaluser$:
\begin{align*}
  \policyuser_t(\actionuser \given \stateenv, \goal) &= \exp(\vgoaluser(\stateenv) -\qgoaluser(\stateenv, \actionuser))
\end{align*}
Recent works have explored extensions of the MaxEnt IOC model for continuous spaces~\citep{boularias_2011, levine_2012, finn_2016}. We leave experiments using these methods for learning and prediction as future work.


\subsection{Multi-Target MDP}
\label{sec:framework_multitarget}

There are often multiple ways to achieve a goal. We refer to each of these ways as a \emph{target}. For a single goal (e.g. object to grasp), let the set of targets (e.g. grasp poses) be $\target \in \Target$. We assume each target has a cost function $\costtarg$, from which we compute the corresponding value and action-value functions $\vtarg$ and $\qtarg$, and soft-value functions $\vtargsoft$ and $\qtargsoft$. We derive the quantities for goals, $\vgoal, \qgoal, \vgoalsoft, \qgoalsoft$, as functions of these target functions.

We state the theorems below, and provide proofs in the appendix (\cref{sec:mingoal_thms}).

\subsubsection{Multi-Target Assistance}
\label{sec:framework_multigarget_assistance}
We assign the cost of a state-action pair to be the cost for the target with the minimum cost-to-go after this state:
\begin{align}
  \costgoal(\stateenvactions) &= \costtargstar(\stateenvactions) \quad \target* = \argmin_\target \vtarg(\stateenv') \label{eq:goal_target_cost}
\end{align}
Where $\stateenv'$ is the environment state after actions $\actionuser$ and $\actionrobot$ are applied at state $\stateenv$. For the following theorem, we require that our user policy be deterministic, which we already assume in our approximations when computing robot actions in \cref{sec:framework_hindsight}.
\begin{restatable}{theorem}{valfundecompose}
\label{thm:mingoal_assist}
Let $\vtarg$ be the value function for target $\target$. Define the cost for the goal as in \cref{eq:goal_target_cost}. For an MDP with deterministic transitions, and a deterministic user policy $\policyuser$, the value and action-value functions $\vgoal$ and $\qgoal$ can be computed as:
\begin{align*}
  \qgoal(\stateenvactions) &= \qtargstar(\stateenvactions) \qquad \target^* = \argmin_\target \vtarg(\stateenv') \\
  \vgoal(\stateenv) &= \min_\target \vtarg(\stateenv)
\end{align*}
\end{restatable}

\subsubsection{Multi-Target Prediction}
\label{sec:framework_multigarget_prediction}
Here, we don't assign the goal cost to be the cost of a single target $\costtarg$, but instead use a distribution over targets.%based on the cost-to-go.
\begin{restatable}{theorem}{softvalfundecompose}
  \label{thm:mingoal_pred}
  Define the probability of a trajectory and target as $p(\traj, \target) \propto \exp(-\costtarg(\traj))$. Let $\vtargsoft$ and $\qtargsoft$ be the soft-value functions for target $\target$. For an MDP with deterministic transitions, the soft value functions for goal $\goal$, $\vgoalsoft$ and $\qgoalsoft$, can be computed as:
\begin{align*}
  \vgoalsoft(\stateenv) &= \softmin_\target \vtargsoft(\stateenv)\\
  \qgoalsoft(\stateenv, \actionuser) &= \softmin_\target \qtargsoft(\stateenv, \actionuser)
\end{align*}
\end{restatable}

%
%Marginalizing out g:
%\begin{align*}
%  p(a_t | s) &= \sum_g p(a_t, g | s)\\
%  &= \frac{ \sum_g \exp(-Q_g^{t}(s_t, a_t))} {\sum_{g'}\exp(-V_{g'}^{t}(s_{t}))}
%\end{align*}
%
%We can also write this out as:
%\begin{align*}
%  \exp\left( \log\left( p(a_t | s) \right) \right)&= \exp\left( \log\left(  \frac{ \sum_g \exp(-Q_g^{t}(s_t, a_t))} {\sum_{g'}\exp(-V_{g'}^{t}(s_{t}))}\right) \right)\\
%  &= \exp\left( \log\left(  \sum_g \exp(-Q_g^{t}(s_t, a_t)) \right) - \log\left(\sum_{g'}\exp(-V_{g'}^{t}(s_{t})) \right) \right)\\
%  &= \exp\left( \softmin_g V_{g}^{t}(s_{t}) - \softmin_g Q_g^{t}(s_t, a_t)\right)
%\end{align*}
%

\begin{figure}[t]
\centering
 \begin{subfigure}{0.24\textwidth}
   \centering 
   \includegraphics[width=0.97\textwidth, trim=440 250 500 210, clip=true]{rss_multigoal_1.png}
  \caption{}
 \label{fig:multigoal_1}
 \end{subfigure}
 \begin{subfigure}{0.24\textwidth}
   \centering 
   \includegraphics[width=0.97\textwidth, trim=440 250 500 210, clip=true]{rss_multigoal_2.png}
  \caption{}
 \label{fig:multigoal_2}
 \end{subfigure}
 \begin{subfigure}{0.24\textwidth}
   \centering 
   \includegraphics[width=0.97\textwidth, trim=440 250 500 210, clip=true]{rss_multigoal_3_arb.png}
  \caption{}
 \label{fig:multigoal_3_arb}
 \end{subfigure}
 \begin{subfigure}{0.24\textwidth}
   \centering 
   \includegraphics[width=0.97\textwidth, trim=440 250 500 210, clip=true]{rss_multigoal_3_pred.png}
  \caption{}
 \label{fig:multigoal_3_pred}
 \end{subfigure}
 \caption{Value function for a goal (grasp the ball) decomposed into value functions of targets (grasp poses). (\subref{fig:multigoal_1}, \subref{fig:multigoal_2}) Two targets and their corresponding value function $\vtarg$. In this example, there are 16 targets for the goal. (\subref{fig:multigoal_3_arb}) The value function of a goal $\vgoal$ used for assistance, corresponding to the minimum of all 16 target value functions (\subref{fig:multigoal_3_pred}) The soft-min value function $\vgoalsoft$ used for prediction, corresponding to the soft-min of all 16 target value functions.}
 \label{fig:multigoal}
\end{figure}




\section{Simulation Studies}\label{sec:simulation}
In this section, we are mainly interested in the empirical performance of the ABESS algorithm on logistic regression and Poisson regression.
Logistic regression is widely used for classification tasks, and Poisson regression is appropriate when the response is a count.
\if0\informsMOR{In ``Additional Simulation'' of Supplementary Material, we }\else{We }\fi
also consider the performance of ABESS algorithm on multi-response linear regression (a.k.a., multi-task learning).
Before formally analyzing the simulation results,
we illustrate our simulation settings in Section~\ref{subsec:setup}.
% This subsection develops parallel with Section \ref{subsec:logistic}.

%In this section, we study the empirical performance of ABESS for GLM on two generalized linear models,
%logistic regression and gamma regression,
%where logistic regression is widely used for classification and
%gamma regression model is useful for modeling positive continuous response variables.
%Before formally studying logistic regression and gamma regression in Section~\ref{subsec:logistic} and Section~\ref{subsec:gamma}, respectively, we illustrate our simulation setting in Section~\ref{subsec:setup}.

\subsection{Setup}\label{subsec:setup}
To synthesize a dataset, we generate multivariate Gaussian realizations $\boldsymbol{x}_1, \ldots, \boldsymbol{x}_n \overset{i.i.d.}{\sim} \mathcal{MVN}(0,\Sigma)$,
where $\Sigma$ is a $p$-by-$p$ covariance matrix.
%We generate i.i.d error $\epsilon\sim N(0,\sigma^2)$.
%Define the signal to noise ratio (SNR) by $SNR = \frac{\beta^{\top}\Sigma\beta}{\sigma^2}$.
We consider two covariance structures for $\Sigma$: the independent structure ($\Sigma$ is an identity matrix)
and the constant structure ($\Sigma_{ij} = \rho^{I(i\neq j)}$ for some positive constant $\rho$). The value of $\rho$ and $p$ will be specified later.
We set the true regression coefficient $\boldsymbol{\beta}^*$ as a sparse vector with $k$ non-zero entries that have equi-spaced indices in $\{1, \ldots, p\}$.
Finally, given a design matrix $\mathbf{X} = (\boldsymbol{x}_1, \ldots, \boldsymbol{x}_n)^\top$ and $\boldsymbol{\beta}^*$,
we draw response realizations $\{y_i\}_{i=1}^n$ according to the GLMs.

We assess our proposal via the following criteria.
First, to measure the performance of subset selection,
we consider the probabilities of covering true active and inactive sets: $\mathbb{P}(\mathcal{A}^* \subseteq \hat{\mathcal{A}})$ and
$\mathbb{P}(\mathcal{I}^* \subseteq \hat{\mathcal{I}})$ (here, $\mathcal{I}^* = (\mathcal{A}^*)^c$).
We also consider exact support recover probability as $\mathbb{P}(\mathcal{A}^* = \hat{\mathcal{A}})$.
Since the probability is unknown, we empirically examine the proportion of recovery for the active set, inactive set, and exact recovery in 200 replications for instead.
As for parameter estimation performance, we examine relative error (ReErr) on parameter estimations:
$\|\hat{\boldsymbol{\beta}}-\boldsymbol{\beta}^*\|_{2} /\|\boldsymbol{\beta}^*\|_{2}$.
Finally, computational efficiency is directly measured by the runtime.

In addition to our proposed algorithms, we compare classical variable selection methods: LASSO \citep{tibshirani1996regression}, SCAD \citep{fan2001variable}, and MCP \citep{zhang2010nearly}.
%, and a recently proposed coordinate descent (CD) method for $\ell_0$-regularized classification \citep{antoine2021l0learn}.
For all these methods, we apply 10-fold cross-validation (CV) and the GIC to select the tuning parameter, respectively.
% For all these methods, we apply 10-fold cross-validation (CV) to select the tuning parameter.
% ABESS also uses generalized information criterion (GIC) \citep{fan2013tuning} because,
% by combining GIC, ABESS can consistently recover $\mathcal{A}^*$ under linear models \citep{zhu2020polynomial}.
The software for these methods is available at R CRAN (\url{https://cran.r-project.org}).
The software of all methods is summarized in Table~\ref{tab:implementation-details}.
All experiments are carried out on an R environment in a Linux platform with Intel(R) Xeon(R) Gold 6248 CPU @ 2.50GHz. 
% Note that, all experiments result are based on 200 random synthetic datasets.
%Ubuntu platform with Intel(R) Xeon(R) Gold 6248 CPU @ 2.50GHz.

% Model selection methods such as cross-validation and information criteria are widely used.
% Recently, \citet{fan2013tuning} explored generalized information criterion (GIC) in tuning parameter selection for
% penalized likelihood methods under GLM.
% Here, we use a GIC-type information criterion to recovery support size, which is defined as:
% $\mathrm{F}(\hat{\boldsymbol \beta}) = l_n( \hat{\boldsymbol \beta} ) + |\text{supp}(\hat{\boldsymbol \beta})| \log(p) \log\log n.$
% Intuitively speaking, the model complexity penalty term $|\text{supp}(\hat{\boldsymbol \beta})| \log p \log\log n$ is set to prevent over-fitting,
% where the term $\log\log n$ with a slow diverging rate is used to prevent under-fitting.
% Combining the Algorithm~\ref{alg:fbess} with GIC, we select the support size that minimizes the $F(\hat{\boldsymbol{\beta}})$.}

% \begin{table}[htbp]
% \caption{Implementation details for all methods.
% The values in the parentheses indicate the version number of R packages.}\label{tab:implementation-details}
% \centering
% \begin{tabular}{ccc}
% \toprule
% Method & Software & Tuning method \\
% \midrule
% ABESS-GIC & abess (0.4.0) & GIC \\
% LASSO-GIC & glmnet (4.1-3) & GIC \\
% SCAD-GIC & ncvreg (3.13.0) & GIC \\
% MCP-GIC & ncvreg (3.13.0) & GIC \\
% CD-GIC & L0Learn (2.0.3) & GIC \\
% ABESS-CV & abess (0.4.0) & 10-folds CV \\
% LASSO-CV & glmnet (4.1-3) & 10-folds CV \\
% SCAD-CV & ncvreg (3.13.0) & 10-folds CV \\
% MCP-CV & ncvreg (3.13.0) & 10-folds CV \\
% CD-CV & L0Learn (2.0.3) & 10-folds CV \\
% \bottomrule
% \end{tabular}
% \end{table}
\begin{table}[htbp]
\caption{Software for all methods.
The values in the parentheses indicate the version number of R packages.The tuning parameter within the MCP/SCAD penalty is fixed at 3/3.7.}\label{tab:implementation-details}
\centering
\if0\informsMOR{
% \begin{tabular}{ccccccc}
% \toprule
% Method & ABESS & LASSO & SCAD & MCP & CD \\
% \midrule
% Software & \textsf{abess} (0.4.0) & \textsf{glmnet} (4.1-3) & \textsf{ncvreg} (3.13.0) & \textsf{ncvreg} (3.13.0) & \textsf{L0Learn} (2.0.3) \\
% Tuning & sparsity $s$ & $\ell_1$ penalty & $\lambda$ & $\lambda$& $\lambda$ \\
% \bottomrule
% \end{tabular}
\begin{tabular}{cccccc}
    \toprule
    Method & ABESS & LASSO & SCAD & MCP \\
    \midrule
    Software & \textsf{abess} (0.4.0) & \textsf{glmnet} (4.1-3) & \textsf{ncvreg} (3.13.0) & \textsf{ncvreg} (3.13.0) \\
    Tuning & sparsity $s$ & $\ell_1$ penalty & $\lambda$ & $\lambda$ \\
    \bottomrule
    \end{tabular}
}\else{
% \begin{tabular}{ccccccc}
% \hline
% Method & ABESS & LASSO & SCAD & MCP & CD \\
% \hline
% Software & \textsf{abess} (0.4.0) & \textsf{glmnet} (4.1-3) & \textsf{ncvreg} (3.13.0) & \textsf{ncvreg} (3.13.0) & \textsf{L0Learn} (2.0.3) \\
% Tuning & sparsity $s$ & $\ell_1$ penalty & {\color{red}SCAD penalty} & {\color{red}MCP penalty} & $\ell_0$ penalty \\
% \hline
% \end{tabular}
\begin{tabular}{cccccc}
\hline
Method & ABESS & LASSO & SCAD & MCP \\
\hline
Software & \textsf{abess} (0.4.0) & \textsf{glmnet} (4.1-3) & \textsf{ncvreg} (3.13.0) & \textsf{ncvreg} (3.13.0) \\
Tuning & sparsity $s$ & $\ell_1$ penalty & {\color{red}SCAD penalty} & {\color{red}MCP penalty} \\
\hline
\end{tabular}
}\fi
\end{table}
% We implement our proposal in an R package abess \citep{zhu-abess-arxiv}.

\subsection{Logistic Regression}\label{subsec:logistic}
% In this subsection, we illustrate the power of ABESS on logistic regression, which is one of the most popular GLMs widely used for classification tasks.
% In terms of logistic regression, the response $y_i$ is a binary variable following a Bernoulli distribution $B(1, p_i)$,
% where $p_i \coloneqq \mathbb{P}(y_i=1)$ is determined by $\log(\frac{p_i}{1-p_i}) = \boldsymbol x_i^\top \boldsymbol{\beta }$.
% Here, the link function is known as the logit function, defined by $logit(p) = \log(\frac{p}{1-p})$.
% As a result, the negative log-likelihood is given by
% \begin{equation*}
% l_n(\boldsymbol\beta) = -\sum_{i=1}^{N}\left\{y_{i} \boldsymbol {x}_i^\top \boldsymbol \beta-\log \left(1+e^{\boldsymbol {x}_i^\top \boldsymbol \beta}\right)\right\}.
% \end{equation*}
% Empirically, we generate $x_i$ and $\beta$ as described in Section \ref{subsec:setup}.
% Binary response $y_i$ is then drawn from the Bernoulli distribution according to (\ref{eqn:formula_binomial}).
% Let $H_j = \sum\limits_{i=1}^{n} \frac{e^{\boldsymbol {x}_i^\top \hat{\boldsymbol \beta}}}{(1 + e^{\boldsymbol {x}_i^\top \hat{\boldsymbol \beta}})^2} x_{ij}^2$ and
% the gradient of $l_n(\boldsymbol{\beta})$ at $\hat{\boldsymbol{\beta}}$ be $\hat{\boldsymbol d} = -\sum\limits_{i=1}^{n}(y_i - \frac{e^{\boldsymbol {x}_i^\top \hat{\boldsymbol \beta}}}{1 + e^{\boldsymbol {x}_i^\top \hat{\boldsymbol \beta}}}) \boldsymbol {x}_i$,
% \eqref{eqn:approx_sacrifice} can be explicit expressed as:
% $\xi_j = H_j (\hat{\boldsymbol{\beta}}_j)^2$ for $j\in \mathcal{A}$ and
% $\zeta_j = H_j^{-1}( \hat{\boldsymbol d}_j )^2$ for $j\in \mathcal{I}$.
% \begin{equation*}
% \begin{aligned}
% % \hat{\boldsymbol d} &= -\sum_{i=1}^{n}(y_i - \frac{e^{\boldsymbol {x}_i^\top \boldsymbol \beta}}{1 + e^{\boldsymbol {x}_i^\top \boldsymbol \beta}}) \boldsymbol {x}_i. \\
% \xi_j
% & = H_j (\hat{\boldsymbol{\beta}}_j)^2, j\in \mathcal{A},\\
% \zeta_j
% & = H_j^{-1}
% ( \hat{\boldsymbol d}_j )^2, j\in \mathcal{I}.
% \end{aligned}
% \end{equation*}
% Given the explicit expression of \eqref{eqn:approx_sacrifice},
% we can conduct Algorithm~\ref{alg:abess} to estimate $\boldsymbol{\beta}$.

The dimension $p$ is fixed as 500 for the logistic regression model. For the constant correlation case, we set $\rho = 0.4$.
The non-zero coefficients $\boldsymbol{\beta}^*_{\mathcal{A}^*}$ are set to be $(2,2,8,8,8,8,10,10,10,10)^\top$. 
Now we compare methods listed in Table~\ref{tab:implementation-details}.
Figures~\ref{fig:rate_binomial} and \ref{fig:ReErr_binomial} present the results on subset selection and parameter estimation when the sample size increases. Out of clarity, we omit the CV results here and defer these results to the Additional Figures in Supplementary Material.


\begin{figure}[htbp]
\centering
\includegraphics[width=1.0\textwidth]{figure/rate_binomial_gic.pdf}
\if0\informsMOR{
\vspace{-30pt}
}\fi
\caption{Performance on subset selection under logistic regression when covariates have independent correlation structure (Upper) and constant correlation structure (Lower), measured by three kinds probabilities: $\mathbb{P}(\mathcal{A}^* \subseteq \hat{\mathcal{A}})$, $\mathbb{P}(\mathcal{I}^* \subseteq \hat{\mathcal{I}})$, and $\mathbb{P}(\mathcal{A}^* = \hat{\mathcal{A}})$ that are presented in Left, Middle and Right panels, respectively.
}
\label{fig:rate_binomial}
\end{figure}
\begin{figure}[htbp]
\centering
\includegraphics[width=0.8\textwidth]{figure/ReErr_binomial_gic.pdf}
\if0\informsMOR{
\vspace{-10pt}
}\fi
\caption{Performance on parameter estimation under logistic regression models when covariance matrices have independent correlation structure (Left) and exponential correlation structure (Right). The $y$-axis is the median of ReErr in a log scale.}
\label{fig:ReErr_binomial}
\end{figure}

As depicted in the left panel of Figure~\ref{fig:rate_binomial}, the probability $\mathbb{P}(\mathcal{A}^* \subseteq \hat{\mathcal{A}})$ approaches 1 as the sample size increases, indicating that all methods, except LASSO in the high correlation setting, can provide a no-false-exclusion estimator when the sample size is sufficiently large. However, when considering $\mathbb{P}(\mathcal{I}^* \subseteq \hat{\mathcal{I}})$, as observed in the middle panel of Figure~\ref{fig:rate_binomial}, the LASSO estimator consistently exhibits false inclusions, and the SCAD/MCP estimator shows false inclusions when the covariates are highly correlated. In contrast, only ABESS guarantees that $\mathbb{P}(\mathcal{I}^* \subseteq \hat{\mathcal{I}})$ approaches 1 for large sample sizes. 

Furthermore, as evident from the right panel of Figure~\ref{fig:rate_binomial}, ABESS accurately recovers the true subset under both correlation settings. While SCAD and MCP can also achieve exact support recovery given a sufficient sample size, ABESS demonstrates support recovery consistency with the smallest sample size, particularly when variables are correlated. It is important to note that although our theory imposes restrictions on the correlation among a small subset of variables (see Assumption~\ref{con:technical-assumption}), our algorithm still performs effectively in the constant correlation setting. This setting (i.e., $\rho=0.4$) violates Assumption~\ref{con:technical-assumption} as the correlation between any two variables exceeds 0.183, which is the maximum acceptable pairwise correlation satisfying Assumption~\ref{con:technical-assumption}.

Moving on to Figure~\ref{fig:ReErr_binomial}, it illustrates the superiority of ABESS in parameter estimation. ABESS visibly outperforms other methods in the small sample size regime and maintains highly competitive performance as the sample size increases. This superiority in parameter estimation is not surprising, as ABESS yields an oracle estimator when the support set is correctly identified. Although SCAD and MCP do not provide algorithmic guarantees for finding the local minimum, they exhibit competitive parameter estimation performance due to their asymptotic unbiasedness. Conversely, the LASSO estimator is biased and performs the worst among all the methods.

%\begin{figure}
%	\centering
%	\includegraphics[width=\textwidth]{figure/Performance_binomial.pdf}
%	\caption{Performance comparison under two correlation structures: independent and exponential. (A) Performance for subset selection, measured by support recover probability. (B) Performance for parameter estimation, measured by median ReErr. (C) Average runtime, measured in seconds. L0Learn is omitted since its runtime is far longer than others.}
%	\label{fig:Performance_binomial}
%\end{figure}

\subsection{Poisson Regression}\label{seubsec:poisson}
% As regard to Poisson regression, the response $y_i$ is a integer variable following a Poisson distribution $\mathcal{P}(\lambda_i)$ where $\lambda_i = \exp(\boldsymbol x_i^\top \boldsymbol{\beta})$.
% As a result, the negative log-likelihood is given by
% \begin{equation*}
% l_n(\boldsymbol\beta) = -\sum_{i=1}^{N}\left\{y_{i} \boldsymbol {x}_i^\top \boldsymbol\beta - e^{\boldsymbol {x}_i^\top \boldsymbol \beta} -\log(y_i!)\right\}.
% \end{equation*}
% Empirically, we generate $x_i$ and $\beta$ as described in Section \ref{subsec:setup}.
% Binary response $y_i$ is then drawn from the Bernoulli distribution according to (\ref{eqn:formula_binomial}).
% Let $H_j = \sum\limits_{i=1}^{n} \exp(\boldsymbol {x}_i^\top \hat{\boldsymbol \beta}) x_{ij}^2$ and
% the gradient of $l_n(\boldsymbol{\beta})$ at $\hat{\boldsymbol{\beta}}$ be $\hat{\boldsymbol d} = -\sum\limits_{i=1}^{n}(y_i - \exp(\boldsymbol {x}_i^\top \hat{\boldsymbol \beta})) \boldsymbol {x}_i$,
% \eqref{eqn:approx_sacrifice} can be explicit expressed as:
% $\xi_j = H_j (\hat{\boldsymbol{\beta}}_j)^2$ for $j\in \mathcal{A}$ and
% $\zeta_j = H_j^{-1}( \hat{\boldsymbol d}_j )^2$ for $j\in \mathcal{I}$.
% Given the explicit expression of \eqref{eqn:approx_sacrifice},
% we can conduct Algorithm~\ref{alg:abess} to estimate $\boldsymbol{\beta}$.


For the Poisson regression model, we consider a fixed $p$ value of 500, and set $\rho = 0.2$ for the constant correlation case. The non-zero coefficients $\boldsymbol{\beta}^*_{\mathcal{A}^*}$ are specified as $(1, 1, 1)^\top$. Figures~\ref{fig:rate_poisson_gic}-\ref{fig:ReErr_poisson_gic} present the evaluation of subset selection and parameter estimation quality. Examining Figures~\ref{fig:rate_poisson_gic}, we observe that for ABESS/SCAD/MCP, the probabilities $\mathbb{P}(\mathcal{A}^* \subseteq \hat{\mathcal{A}})$, $\mathbb{P}(\mathcal{I}^* \subseteq \hat{\mathcal{I}})$, and $\mathbb{P}(\mathcal{A}^* = \hat{\mathcal{A}})$ gradually approach 1 as the sample size $n$ increases. In contrary, the LASSO, regardless of the highest inclusion probability for $\mathcal{A}^*$, still has a chance of including ineffective variables, especially when variables are correlated. Comparing ABESS, SCAD, and MCP, it is evident that ABESS achieves the highest exact selection probability, followed by SCAD and MCP. Similar to the results in logistic regression, ABESS achieves exact selection of the effective variables with the smallest sample size under the constant correlation structure.
Regarding the quality of parameter estimation, the ReErr of all methods reasonably decreases as the sample size $n$ increases. Again, ABESS exhibits the least estimation error in terms of the $\ell_2$-norm, which coincides with the results on logistic regression. It is worth noting that our method demonstrates consistency and polynomial complexity under Poisson regression, even though it violates the sub-Gaussian assumption. This is because the current framework of proofs allows for the relaxation of Assumption~\ref{con:subgaussian} to a sub-exponential distribution assumption, enabling the establishment of similar theoretical properties.

\begin{figure}[htbp]
\centering
\includegraphics[width=1.0\textwidth]{figure/rate_poisson_gic.pdf}
\if0\informsMOR{
\vspace{-30pt}
}\fi
\caption{Performance on subset selection under Poisson regression when covariates have independent correlation structure (Upper) and constant correlation structure (Lower), measured by three kinds probabilities: $\mathbb{P}(\mathcal{A}^* \subseteq \hat{\mathcal{A}})$, $\mathbb{P}(\mathcal{I}^* \subseteq \hat{\mathcal{I}})$, and $\mathbb{P}(\mathcal{A}^* = \hat{\mathcal{A}})$ that are presented in Left, Middle and Right panels, respectively.}
\label{fig:rate_poisson_gic}
\end{figure}
\begin{figure}[htbp]
\centering
\includegraphics[width=0.8\textwidth]{figure/ReErr_poisson_gic.pdf}
\if0\informsMOR{
\vspace{-5pt}
}\fi
\caption{Performance on parameter estimation under Poisson regression models when covariance matrices have independent correlation structure (Left) and exponential correlation structure (Right). The $y$-axis is the median of ReErr in a log scale.}
\label{fig:ReErr_poisson_gic}
\end{figure}

\subsection{Computational analysis}

We compare the runtime of different methods in Table~\ref{tab:implementation-details} for the logistic regression and Poisson regression models in Sections~\ref{subsec:logistic} to \ref{seubsec:poisson}. The runtime results are summarized in Figure~\ref{fig:simu_runtime}, indicating that ABESS demonstrates superior computational efficiency compared to state-of-the-art variable selection methods. For instance, when $n = 3000$, ABESS is at least four times faster than its competitors in logistic regression under an independent correlation structure. Furthermore, regardless of logistic regression or Poisson regression, ABESS exhibits similar computational performance, while other competitors run much faster when the pairwise correlation is higher. Lastly, it is important to note that the runtime of ABESS scales polynomially with sample sizes, aligning with the complexity presented in Theorem~\ref{thm:complexity}.
%In contrast, the runtime of other methods grows more rapidly as the sample size increases
%and appears like a quadratic function of the sample size in the independent scenario.
%Increasing iteration numbers for convergence may lead to this result.
%Moreover, ABESS-GIC is faster than ABESS-CV, demonstrating the superiority of the proposed adaptive parameter tuning procedure.
% Finally, according to the computational comparison presented in Figure~\ref{fig runtime_poisson_gic}, the ABESS has the least runtime and is much faster than the MCP and SCAD when variables are independent.

\begin{figure}[htbp]
\centering
\includegraphics[width=0.8\textwidth]{figure/runtime_binomial_gic.pdf}
\includegraphics[width=0.8\textwidth]{figure/runtime_poisson_gic.pdf}
\if0\informsMOR{
\vspace{-10pt}
}\fi
\caption{Average runtime (measured in seconds) on logistic regression (Upper panel) and Poisson regression (Lower panel). The results on two types of covariances matrix $\Sigma$, the independent correlation structure and constant correlation structure, are presented in the left and right panels, respectively. The error bars represent two times the standard errors.
}
\label{fig:simu_runtime}
\end{figure}

%% For the MOR template, uncomment this line and comment on the code blocks

\if1\informsMOR
{
\input{../appendix_numerical}
}\fi

\section{Experiment \& Analysis}
\label{sec:exp}
In this section, we first introduce the experimental set-up. Then, we report the performances of baselines and the proposed steep slope loss on ImageNet, followed by comprehensive analyses. 
% At last, we present comprehensive analyses to help better understand the efficacy of the proposed loss.

\noindent\textbf{Experimental Set-Up}.
We use ViT B/16 \cite{Dosovitskiy_ICLR_2021} and ResNet-50 \cite{He_CVPR_2016} as the classifiers, and the respective backbones are used as the oracles' backbones. We denote the combination of oracles and classifiers as \textlangle O, C\textrangle. There are four combinations in total, \ie \textlangle ViT, ViT\textrangle, \textlangle ViT, RSN\textrangle, \textlangle RSN, ViT\textrangle, and \textlangle RSN, RSN\textrangle, where RSN stands for ResNet.
In this work, we adopt three baselines, \ie the cross entropy loss \cite{Cox_JRSS_1972}, focal loss \cite{Lin_ICCV_2017}, and TCP confidence loss \cite{Corbiere_NIPS_2019}, for comparison purposes.

The experiment is conducted on ImageNet \cite{Deng_CVPR_2009}, which consists of 1.2 million labeled training images and 50000 labeled validation images. It has 1000 visual concepts. Similar to the learning scheme in \cite{Corbiere_NIPS_2019}, the oracle is trained with training samples and evaluated on the validation set. During the training process of the oracle, the classifier works in the evaluation mode so training the oracle would not affect the parameters of the classifier. Moreover, we conduct the analyses about how well the learned oracle generalizes to the images in the unseen domains. To this end, we apply the widely-used style transfer method \cite{Geirhos_ICLR_2019} and the functional adversarial attack method \cite{Laidlaw_NeurIPS_2019} to generate two variants of the validation set, \ie stylized validation set and adversarial validation set. \REVISION{Also, we adopt ImageNet-C \cite{Hendrycks_ICLR_2018} for evaluation, which is used for evaluating robustness to common corruptions.}
% Then, the oracle trained with regular training samples would be evaluated with the samples that are in the two unseen domains.

% To understand how the learned oracle work on unseen domains, the oracle is learned with training samples and is evaluated with three types of samples, the samples on the same domain as training samples and the samples on two unseen domains. We base our experiments on ImageNet \cite{Deng_CVPR_2009}, a widely-used large-scale dataset. Except for the training set and the validation set, we use the stylized ImageNet validation set \cite{Geirhos_ICLR_2019} and an ImageNet validation set that are perturbed by the functional adversarial attack technique \cite{Laidlaw_NeurIPS_2019}.
% (Introduce models here.)
% (Introduce hyperpaprameters here.)

The oracle's backbone is initialized by the pre-trained classifier's backbone and trained by fine-tuning using training samples and the trained classifier.
% As the oracle's backbone is initialized by the pre-trained classifier's backbone, the training process of the oracles is equivalent to the process of fine-tuning the initialized oracles.
Training the oracles with all the loss functions uses the same hyperparameters, such as learning rate, weight decay, momentum, batch size, etc.
The details for the training process and the implementation are provided in \appref{sec:implementation}.

For the focal loss, we follow \cite{Lin_ICCV_2017} to use $\gamma=2$,  which leads to the best performance for object detection.
For the proposed loss, we use $\alpha^{+}=1$ and $\alpha^{-}=3$ for the oracle that is based on ViT's backbone, while we use $\alpha^{+}=2$ and $\alpha^{-}=5$ for the oracle that is based on ResNet's backbone.

Following \cite{Corbiere_NIPS_2019}, we use FPR-95\%-TPR, AUPR-Error, AUPR-Success, and AUC as the metrics.
FPR-95\%-TPR is the false positive rate (FPR) when true positive rate (TPR) is equal to 95\%. 
AUPR is the area under the precision-recall curve. 
Specifically, AUPR-Success considers the correct prediction as the positive class, whereas AUPR-Error considers the incorrect prediction as the positive class.
AUC is the area under the receiver operating characteristic curve, which is the plot of TPR versus FPR.
Moreover, we use TPR and true negative rate (TNR) as additional metrics because they assess overfitting issue, \eg TPR=100\% and TNR=0\% imply that the trustworthiness predictor is prone to view all the incorrect predictions to be trustworthy. %due to overfitting.

% \noindent\textbf{Baselines \& Metrics}.
% We adopt widely-used loss functions, \ie cross entropy and focal loss, as the baselines. To comprehensively understand and measure oracles' performance, we use KL divergence and Bhattacharya coefficient to measure the correlation between two feature distributions, use true positive rate (TPR), true negative rate (TNR), accuracy (Acc=$(TP+TN)/Total$), F1 score, precision (P), and recall (R) to measure the efficacy of predicting trustworthiness. Specifically, we add Acc\textsubscript{P} and Acc\textsubscript{N} to understand how much TP and TN contribute to Acc. This is useful when the model overfits the data, \ie classifying all the images as either positives or negatives. Moreover, to differentiate the accuracy of classification from the accuracy of predicting trustworthiness, we denote the classifier's accuracy as C-Acc, and the oracle's accuracy as O-Acc.

% \begin{table}[!t]
	\centering
	\caption{\label{tbl:avg_perf}
	    Averaged performance over the regular ImageNet validation set, the stylized ImageNet validation set, and the adversarial ImageNet validation set. The oracle is trained with the cross entropy (CE) loss, the focal loss, and the proposed steep slope (SS) loss on the ImageNet training set. The resulting oracles w.r.t. each loss are evaluated on the three validation sets. The classifier is used in the evaluation mode in the experiment. $d_{KL}$ represents KL divergence, while $c_{B}$ represents Bhattacharyya coefficient.
	}
	\adjustbox{width=1\columnwidth}{
	\begin{tabular}{L{7ex} C{8ex} C{8ex} C{8ex} C{10ex} C{8ex} C{8ex} C{8ex} C{8ex} C{8ex} C{10ex} C{8ex}}
		\toprule
		Loss & C-Acc & $d_{KL}\uparrow$ & $c_{B}\downarrow$ & TPR & TNR & O-Acc & O-Acc\textsubscript{P} & O-Acc\textsubscript{N} & F1 & P & R  \\
		\cmidrule(lr){1-1} \cmidrule(lr){2-2} \cmidrule(lr){3-3} \cmidrule(lr){4-4} \cmidrule(lr){5-5} \cmidrule(lr){6-6} \cmidrule(lr){7-7} \cmidrule(lr){8-8} \cmidrule(lr){9-9} \cmidrule(lr){10-10} \cmidrule(lr){11-11} \cmidrule(lr){12-12}
		& \multicolumn{11}{c}{Oracle: ViT, classifier: ViT} \\
		\cmidrule(lr){2-12}
		CE & 35.74 & 0.5138 & 0.9983 & 99.98 & 0.04 & 35.78 & 35.74 & 0.04 & 0.4382 & 0.3575 & 0.8444 \\
        Focal & 35.74 & 0.5224 & 0.9972 & 99.23 & 1.30 & 36.22 & 35.43 & 0.78 & 0.4374 & 0.3579 & 0.8403 \\
        SS & 35.74 & 1.0875 & 0.9302 & 73.62 & 47.23 & 63.94 & 29.84 & 34.10 & 0.4727 & 0.4430 & 0.5964 \\ \midrule
		& \multicolumn{11}{c}{Oracle: ResNet, classifier: ViT} \\
		\cmidrule(lr){2-12}
		CE & & & & & & & & & & & \\
		Focal & & & & & & & & & & & \\
		SS & & & & & & & & & & & \\
		\bottomrule	
	\end{tabular}}
\end{table}

% \begin{figure}[!t]
% 	\centering
% 	\subfloat{\includegraphics[width=0.32\textwidth]{fig/sigmoid_imagenet_trfeat}    } \hfill
% 	\subfloat{\includegraphics[width=0.32\textwidth]{fig/focal_imagenet_trfeat}    } \hfill
% 	\subfloat{\includegraphics[width=0.32\textwidth]{fig/steep_imagenet_trfeat}    } \\
% 	\subfloat{\includegraphics[width=0.32\textwidth]{fig/sigmoid_imagenet_valfeat}    } \hfill
% 	\subfloat{\includegraphics[width=0.32\textwidth]{fig/focal_imagenet_valfeat}    } \hfill
% 	\subfloat{\includegraphics[width=0.32\textwidth]{fig/steep_imagenet_valfeat}    } \\
% 	\subfloat{\includegraphics[width=0.32\textwidth]{fig/sigmoid_imagenet_valfeat_sty}    } \hfill
% 	\subfloat{\includegraphics[width=0.32\textwidth]{fig/focal_imagenet_valfeat_sty}    } \hfill
% 	\subfloat{\includegraphics[width=0.32\textwidth]{fig/steep_imagenet_valfeat_sty}    } \\
% 	\subfloat{\includegraphics[width=0.32\textwidth]{fig/sigmoid_imagenet_valfeat_adv}    } \hfill
% 	\subfloat{\includegraphics[width=0.32\textwidth]{fig/focal_imagenet_valfeat_adv}    } \hfill
% 	\subfloat{\includegraphics[width=0.32\textwidth]{fig/steep_imagenet_valfeat_adv}    }
% 	\caption{\label{fig:distribution}
%     	Feature distributions w.r.t. the cross entropy (first column), focal (second column), and the proposed steep slope (third column) losses on the ImageNet training set (second row), ImageNet validation set (first row), stylized ImageNet validation set (third row), and adversarial ImageNet validation set (fourth row). CE stands for cross entropy, while SS stands for steep slope.
%     % 	\REVISION{\textit{Baseline} indicates ResNet GEM.}
%     	}
% \end{figure}

% \begin{table}[!t]
	\centering
	\caption{\label{tbl:perf_rsn_vit}
	    Performances on the regular ImageNet validation set, the stylized ImageNet validation set, and the adversarial ImageNet validation set. In this experiment, ResNet-50 is used for the oracle backbone while ViT is used for the classifier. The classifier is used in the evaluation mode in the experiment.
	}
	\adjustbox{width=1\columnwidth}{
	\begin{tabular}{L{8ex} C{8ex} C{8ex} C{8ex} C{10ex} C{8ex} C{8ex} C{8ex}}
		\toprule
		Loss & Acc$\uparrow$ & FPR-95\%-TPR$\downarrow$ & AURP-Error$\uparrow$ & AURP-Success$\uparrow$ & AUC$\uparrow$ & TPR$\uparrow$ & TNR$\uparrow$ \\
		\cmidrule(lr){1-1} \cmidrule(lr){2-2} \cmidrule(lr){3-3} \cmidrule(lr){4-4} \cmidrule(lr){5-5} \cmidrule(lr){6-6} \cmidrule(lr){7-7} \cmidrule(lr){8-8}
		& \multicolumn{7}{c}{Regular validation set} \\
		\cmidrule(lr){1-1} \cmidrule(lr){2-8}
		CE & 83.90 & 92.58 & 14.59 & 85.57 & 53.78 & 100.00 & 0.00 \\
		Focal & 83.90 & 94.92 & 14.87 & 85.26 & 52.49 & 100.00 & 0.00 \\
		TCP & 83.90 & 91.63 & 14.17 & 86.06 & 55.37 & 100.00 & 0.00 \\
% 		SS & 83.90 & 89.86 & 11.99 & 89.49 & 62.75 & 67.74 & 48.98 \\
		SS & 83.90 & 88.63 & 11.75 & 89.87 & 64.11 & 95.41 & 10.48 \\
% 		& 83.90 & 88.63 & 11.75 & 89.87 & 64.11 & 95.41 & 10.48 \\
%         & 83.90 & 88.72 & 11.76 & 89.85 & 64.01 & 91.10 & 18.51 \\
% 		& 83.90 & 88.25 & 11.54 & 90.24 & 65.23 & 88.23 & 24.25 \\ rsn152
		\midrule
		& \multicolumn{7}{c}{Stylized validation set} \\
		\cmidrule(lr){1-1} \cmidrule(lr){2-8}
		CE & 15.94 & 86.54 & 79.32 & 22.00 & 61.74 & 100.00 & 0.00 \\
		Focal & 15.94 & 95.04 & 85.18 & 14.94 & 48.20 & 100.00 & 0.00 \\
		TCP & 15.94 & 90.82 & 80.69 & 19.96 & 58.34 & 100.00 & 0.00 \\
		SS & 15.94 & 93.80 & 82.10 & 18.19 & 54.18 & 56.94 & 48.88 \\
% 		& 15.94 & 94.27 & 83.09 & 16.97 & 52.35 & 92.13 & 9.03 \\ 52
% 		& 15.94 & 95.76 & 84.28 & 15.74 & 49.02 & 93.83 & 5.52 \\ a62
        \midrule
		& \multicolumn{7}{c}{Adversarial validation set} \\
		\cmidrule(lr){1-1} \cmidrule(lr){2-8}
        CE & & & & & & &  \\
        Focal & & & & & & &  \\
        TCP & & & & & & & \\
        SS & & & & & & &  \\
		\bottomrule	
	\end{tabular}}
\end{table}



% \begin{figure}[!t]
	\centering
	\subfloat[\textlangle ViT, ViT\textrangle]{\includegraphics[width=0.24\textwidth]{fig/risk/risk_vit_vit}} \hfill
	\subfloat[\textlangle ViT, RSN\textrangle]{\includegraphics[width=0.24\textwidth]{fig/risk/risk_vit_rsn}} \hfill
	\subfloat[\textlangle RSN, ViT\textrangle]{\includegraphics[width=0.24\textwidth]{fig/risk/risk_rsn_vit}}
	\hfill
	\subfloat[\textlangle RSN, RSN\textrangle]{\includegraphics[width=0.24\textwidth]{fig/risk/risk_rsn_rsn}}
% 	\subfloat[O: ViT, C: ViT, Loss: TCP]{\includegraphics[width=0.24\textwidth]{fig/tsne/tsne_tcp}    } \hfill
% 	\subfloat[O: ViT, C: ViT, Loss: SS]{\includegraphics[width=0.24\textwidth]{fig/tsne/tsne_steep}    } 
    \\
	\caption{\label{fig:anal_risk}
    	Curves of risk vs. coverage. Selective risk represents the percentage of errors in the remaining validation set for a given coverage.
    	The curves correspond to the oracles used in \tabref{tbl:all_perf_w_std}.
    % 	\REVISION{\textit{Baseline} indicates ResNet GEM.}
    	}
\end{figure}

% \begin{figure}[!t]
	\centering
	\subfloat[O: ViT, C: ViT, Loss: CE]{\includegraphics[width=0.24\textwidth]{fig/tsne/tsne_ce}    } \hfill
	\subfloat[O: ViT, C: ViT, Loss: Focal]{\includegraphics[width=0.24\textwidth]{fig/tsne/tsne_focal}    } \hfill
	\subfloat[O: ViT, C: ViT, Loss: TCP]{\includegraphics[width=0.24\textwidth]{fig/tsne/tsne_tcp}    } \hfill
	\subfloat[O: ViT, C: ViT, Loss: SS]{\includegraphics[width=0.24\textwidth]{fig/tsne/tsne_steep}    } \\
	\caption{\label{fig:anal_tsne}
    	Analysis of t-SNE.
    % 	\REVISION{\textit{Baseline} indicates ResNet GEM.}
    	}
\end{figure}

% \begin{table}[!t]
% 	\centering
% 	\caption{\label{tbl:noise}
% 	    Correctness of oracle on the ImageNet validation set. The oracles are trained with the ImageNet training set. The underlined architecture indicates the architecture of Bayesian network. Leave-out rate indicates the proportion of samples that are ruled out by the oracle. Ideally, it should be equivelant to 1-Acc.
% 	}
% 	\adjustbox{width=1\columnwidth}{
% 	\begin{tabular}{L{10ex} C{12ex} C{12ex} C{9ex} C{9ex} C{9ex} C{9ex} C{9ex} C{9ex} C{9ex}}
% 		\toprule
% 		Dataset & Oracle & Classifier & Acc & O-Acc & O-TP & O-FP & F1 & Precision & Recall \\
% 		\cmidrule(lr){1-1} \cmidrule(lr){2-2} \cmidrule(lr){3-3} \cmidrule(lr){4-4} \cmidrule(lr){5-5} \cmidrule(lr){6-6} \cmidrule(lr){7-7} \cmidrule(lr){8-8} \cmidrule(lr){9-9} \cmidrule(lr){10-10}
% 		Regular & ViT-sigm            & ViT & 83.90 & 83.93 & 83.41 & 15.57 & 0.9121 & 0.8426 & 0.9941    \\
% 		Regular & ViT-Gauss            & ViT & 83.90 & 83.95 & 83.26 & 15.41 & 0.9121 & 0.8438 & 0.9924    \\
% 		Regular & ViT-exp            & ViT & 83.90 & 82.11 &  &  &  &  &     \\  \midrule
% 		Stylized & ViT-sigm            & ViT & 15.93 & 20.62 & 15.36 & 78.79 & 0.2790 & 0.1631 & 0.9639    \\
% 		Stylized & ViT-Gauss            & ViT & 15.93 & 46.28 & 13.01 & 50.79 & 0.3263 & 0.2039 & 0.8163    \\
% 		Stylized & ViT-exp            & ViT & 15.93 & 72.23 &  &  &  &  &     \\ \midrule
% 		Adv & ViT-sigm            & ViT & 7.41 & 11.23 & & & 0.1307 & 0.0762 & 0.5336    \\
% 		Adv & ViT-Gauss            & ViT & 7.41 & 11.15 & 7.14 & 88.79 & 0.1270 & 0.0744 & 0.5088 \\
% 		Adv & ViT-exp            & ViT & 7.41 & 32.57 &  &  &  &  &     \\ 
% 		\bottomrule	
% 	\end{tabular}}
% \end{table}

% \begin{table}[!t]
	\centering
	\caption{\label{tbl:all_perf}
	    Performance on the ImageNet validation set. The averaged scores are computed over three runs. The oracles are trained with the ImageNet training samples. The classifier is used in the evaluation mode in the experiment. Acc is the classification accuracy (\%) and is helpful to understand the proportion of correct predictions. \textit{SS} stands for the proposed steep slope loss.
	   % For example, Acc=83.90\% implies that 83.90\% of predictions is trustworthy and 16.10\% of predictions is untrustworthy.
	}
	\adjustbox{width=1\columnwidth}{
	\begin{tabular}{C{10ex} L{9ex} C{8ex} C{10ex} C{8ex} C{8ex} C{8ex} C{8ex} C{8ex}}
		\toprule
		\textbf{\textlangle O, C\textrangle} & \textbf{Loss} & \textbf{Acc$\uparrow$} & \textbf{FPR-95\%-TPR$\downarrow$} & \textbf{AUPR-Error$\uparrow$} & \textbf{AUPR-Success$\uparrow$} & \textbf{AUC$\uparrow$} & \textbf{TPR$\uparrow$} & \textbf{TNR$\uparrow$} \\
		\cmidrule(lr){1-1} \cmidrule(lr){2-2} \cmidrule(lr){3-3} \cmidrule(lr){4-4} \cmidrule(lr){5-5} \cmidrule(lr){6-6} \cmidrule(lr){7-7} \cmidrule(lr){8-8} \cmidrule(lr){9-9} 
		\multirow{4}{*}{\textlangle ViT, ViT \textrangle} & CE & 83.90 & 93.01 & \textbf{15.80} & 84.25 & 51.62 & \textbf{99.99} & 0.02 \\
		 & Focal \cite{Lin_ICCV_2017} & 83.90 & 93.37 & 15.31 & 84.76 & 52.38 & 99.15 & 1.35 \\
		 & TCP \cite{Corbiere_NIPS_2019} & 83.90 & 88.38 & 12.96 & 87.63 & 60.14 & 99.73 & 0.00 \\
		 & SS & 83.90 & \textbf{80.48} & 10.26 & \textbf{93.01} & \textbf{73.68} & 87.52 & \textbf{38.27} \\
		\midrule
		\multirow{4}{*}{\textlangle ViT, RSN\textrangle} & CE & 68.72 & 93.43 & 30.90 & 69.13 & 51.24 & \textbf{99.90} & 0.20 \\
		 & Focal \cite{Lin_ICCV_2017} & 68.72 & 93.94 & \textbf{30.97} & 69.07 & 51.26 & 93.66 & 7.71 \\
		 & TCP \cite{Corbiere_NIPS_2019} & 68.72 & 83.55 & 23.56 & 79.04 & 66.23 & 94.25 & 0.00 \\
		 & SS & 68.72 & \textbf{77.89} & 20.91 & \textbf{85.39} & \textbf{74.31} & 68.32 & \textbf{67.53} \\
        \midrule
		\multirow{4}{*}{\textlangle RSN, ViT\textrangle} & CE & 83.90 & 93.29 & 14.74 & 85.40 & 53.43 & \textbf{100.00} & 0.00 \\
		 & Focal \cite{Lin_ICCV_2017} & 83.90 & 94.60 & \textbf{14.98} & 85.13 & 52.37 & \textbf{100.00} & 0.00 \\
		 & TCP \cite{Corbiere_NIPS_2019} & 83.90 & 91.93 & 14.12 & 86.12 & 55.55 & \textbf{100.00} & 0.00 \\
         & SS & 83.90 & \textbf{88.70} & 11.69 & \textbf{90.01} & \textbf{64.34} & 96.20 & \textbf{9.00} \\
% 		RSN & ViT & SS & 83.90 & 89.86 & 11.99 & 89.49 & 62.75 & 67.74 & 48.98 \\
        \midrule
        \multirow{4}{*}{\textlangle RSN, RSN\textrangle} & CE & 68.72 & 94.84 & 29.41 & 70.79 & 52.36 & \textbf{100.00} & 0.00 \\
		 & Focal \cite{Lin_ICCV_2017} & 68.72 & 95.16 & \textbf{29.92} & 70.23 & 51.43 & 99.86 & 0.08 \\
		 & TCP \cite{Corbiere_NIPS_2019} & 68.72 & 88.81 & 24.46 & 77.79 & 62.73 & 99.23 & 0.00 \\
         & SS & 68.72 & \textbf{86.21} & 22.53 & \textbf{81.88} & \textbf{67.92} & 79.20 & \textbf{42.09} \\
		\bottomrule	
	\end{tabular}}
\end{table}
\begin{table}[!t]
	\centering
	\caption{\label{tbl:all_perf_w_std}
	    Performance on the ImageNet validation set. The mean and the standard deviation of scores are computed over three runs. The oracles are trained with the ImageNet training samples. The classifier is used in the evaluation mode. Acc is the classification accuracy and is helpful to understand the proportion of correct predictions. \textit{SS} stands for the proposed steep slope loss.
	   % For example, Acc=83.90\% implies that 83.90\% of predictions is trustworthy and 16.10\% of predictions is untrustworthy.
	}
	\adjustbox{width=1\columnwidth}{
	\begin{tabular}{C{10ex} L{10ex} C{8ex} C{10ex} C{10ex} C{10ex} C{10ex} C{10ex} C{10ex}}
		\toprule
		\textbf{\textlangle O, C\textrangle} & \textbf{Loss} & \textbf{Acc$\uparrow$} & \textbf{FPR-95\%-TPR$\downarrow$} & \textbf{AUPR-Error$\uparrow$} & \textbf{AUPR-Success$\uparrow$} & \textbf{AUC$\uparrow$} & \textbf{TPR$\uparrow$} & \textbf{TNR$\uparrow$} \\
		\cmidrule(lr){1-1} \cmidrule(lr){2-2} \cmidrule(lr){3-3} \cmidrule(lr){4-4} \cmidrule(lr){5-5} \cmidrule(lr){6-6} \cmidrule(lr){7-7} \cmidrule(lr){8-8} \cmidrule(lr){9-9} 
		\multirow{4}{*}{\textlangle ViT, ViT \textrangle} & CE & 83.90 & 93.01$\pm$0.17 & \textbf{15.80}$\pm$0.56 & 84.25$\pm$0.57 & 51.62$\pm$0.86 & \textbf{99.99}$\pm$0.01 & 0.02$\pm$0.02 \\
		 & Focal \cite{Lin_ICCV_2017} & 83.90 & 93.37$\pm$0.52 & 15.31$\pm$0.44 & 84.76$\pm$0.50 & 52.38$\pm$0.77 & 99.15$\pm$0.14 & 1.35$\pm$0.22 \\
		 & TCP \cite{Corbiere_NIPS_2019} & 83.90 & 88.38$\pm$0.23 & 12.96$\pm$0.10 & 87.63$\pm$0.15 & 60.14$\pm$0.47 & 99.73$\pm$0.02 & 0.00$\pm$0.00 \\
		 & SS & 83.90 & \textbf{80.48}$\pm$0.66 & 10.26$\pm$0.03 & \textbf{93.01}$\pm$0.10 & \textbf{73.68}$\pm$0.27 & 87.52$\pm$0.95 & \textbf{38.27}$\pm$2.48 \\
		\midrule
		\multirow{4}{*}{\textlangle ViT, RSN\textrangle} & CE & 68.72 & 93.43$\pm$0.28 & 30.90$\pm$0.35 & 69.13$\pm$0.36 & 51.24$\pm$0.63 & \textbf{99.90}$\pm$0.04 & 0.20$\pm$0.00 \\
		 & Focal \cite{Lin_ICCV_2017} & 68.72 & 93.94$\pm$0.51 & \textbf{30.97}$\pm$0.36 & 69.07$\pm$0.35 & 51.26$\pm$0.62 & 93.66$\pm$0.29 & 7.71$\pm$0.53 \\
		 & TCP \cite{Corbiere_NIPS_2019} & 68.72 & 83.55$\pm$0.70 & 23.56$\pm$0.47 & 79.04$\pm$0.91 & 66.23$\pm$1.02 & 94.25$\pm$0.96 & 0.00$\pm$0.00 \\
		 & SS & 68.72 & \textbf{77.89}$\pm$0.39 & 20.91$\pm$0.05 & \textbf{85.39}$\pm$0.16 & \textbf{74.31}$\pm$0.21 & 68.32$\pm$0.41 & \textbf{67.53}$\pm$0.62 \\
        \midrule
		\multirow{4}{*}{\textlangle RSN, ViT\textrangle} & CE & 83.90 & 93.29$\pm$0.53 & 14.74$\pm$0.17 & 85.40$\pm$0.20 & 53.43$\pm$0.28 & \textbf{100.00}$\pm$0.00 & 0.00$\pm$0.00 \\
		 & Focal \cite{Lin_ICCV_2017} & 83.90 & 94.60$\pm$0.53 & \textbf{14.98}$\pm$0.21 & 85.13$\pm$0.24 & 52.37$\pm$0.51 & \textbf{100.00}$\pm$0.00 & 0.00$\pm$0.00 \\
		 & TCP \cite{Corbiere_NIPS_2019} & 83.90 &91.93$\pm$0.49 & 14.12$\pm$0.12 & 86.12$\pm$0.15 & 55.55$\pm$0.46 & \textbf{100.00}$\pm$0.00 & 0.00$\pm$0.00 \\
         & SS & 83.90 & \textbf{88.70}$\pm$0.08 & 11.69$\pm$0.04 & \textbf{90.01}$\pm$0.10 & \textbf{64.34}$\pm$0.16 & 96.20$\pm$0.73 & \textbf{9.00}$\pm$1.32 \\
% 		RSN & ViT & SS & 83.90 & 89.86 & 11.99 & 89.49 & 62.75 & 67.74 & 48.98 \\
        \midrule
        \multirow{4}{*}{\textlangle RSN, RSN\textrangle} & CE & 68.72 & 94.84$\pm$0.27 & 29.41$\pm$0.18 & 70.79$\pm$0.19 & 52.36$\pm$0.41 & \textbf{100.00}$\pm$0.00 & 0.00$\pm$0.00 \\
		 & Focal \cite{Lin_ICCV_2017} & 68.72 & 95.16$\pm$0.19 & \textbf{29.92}$\pm$0.38 & 70.23$\pm$0.44 & 51.43$\pm$0.50 & 99.86$\pm$0.05 & 0.08$\pm$0.03 \\
		 & TCP \cite{Corbiere_NIPS_2019} & 68.72 & 88.81$\pm$0.24 & 24.46$\pm$0.12 & 77.79$\pm$0.29 & 62.73$\pm$0.14 & 99.23$\pm$0.14 & 0.00$\pm$0.00 \\
         & SS & 68.72 & \textbf{86.21}$\pm$0.44 & 22.53$\pm$0.03 & \textbf{81.88}$\pm$0.10 & \textbf{67.92}$\pm$0.11 & 79.20$\pm$2.50 & \textbf{42.09}$\pm$3.77 \\
		\bottomrule	
	\end{tabular}}
\end{table}

\noindent\textbf{Performance on Large-Scale Dataset}. 
The result on ImageNet are reported in \tabref{tbl:all_perf_w_std}. We have two key observations. Firstly, training with the cross entropy loss, focal loss, and TCP confidence loss lead to overfitting the imbalanced training samples, \ie the dominance of trustworthy predictions. Specifically, TPR is higher than 99\% while TNR is less than 1\% in all cases. Secondly, the performance of predicting trustworthiness is contingent on both the oracle and the classifier. When a high-performance model (\ie ViT) is used as the oracle and a relatively low-performance model (\ie ResNet) is used as the classifier, cross entropy loss and focal loss achieve higher TNRs than the loss functions with the other combinations. In contrast, the two losses with \textlangle ResNet, ViT\textrangle~ lead to the lowest TNRs (\ie 0\%). %, compared to the cases with the other combinations.

Despite the combinations of oracles and classifiers, the proposed steep slope loss can achieve significantly higher TNRs than using the other loss functions, while it achieves desirable performance on FPR-95\%-TPR, AUPR-Success, and AUC. This verifies that the proposed loss is effective to improve the generalizability for predicting trustworthiness. Note that the scores of AUPR-Error and TPR yielded by the proposed loss are lower than that of the other loss functions. Recall that AUPR-Error aims to inspect how easy to detect failures and depends on the negated trustworthiness confidences w.r.t. incorrect predictions \cite{Corbiere_NIPS_2019}. The AUPR-Error correlates to TPR and TNR. When TPR is close to 100\% and TNR is close to 0\%, it indicates the oracle is prone to view all the predictions to be trustworthy. In other words, almost all the trustworthiness confidences are on the right-hand side of $p(o=1|\theta,\bm{x})=0.5$. Consequently, when taking the incorrect prediction as the positive class, the negated confidences are smaller than -0.5. On the other hand, the oracle trained with the proposed loss intends to yield the ones w.r.t. incorrect predictions that are smaller than 0.5. In general, the negated confidences w.r.t. incorrect predictions are greater than the ones yielded by the other loss functions. In summary, a high TPR score and a low TNR score leads to a high AUPR-Error.

To intuitively understand the effects of all the loss functions, we plot the histograms of trustworthiness confidences w.r.t. true positive (TP), false positive (FP), true negative (TN), and false negative (FN) in \figref{fig:histogram_part}. The result confirms that the oracles trained with the baseline loss functions are prone to predict overconfident trustworthiness for incorrect predictions, while the oracles trained with the proposed loss can properly predict trustworthiness for incorrect predictions.

% On the other hand, the proposed steep slope loss show better generalizability over the three domains, where TPR is 73.62\% and TNR is 47.23\%. Secondly, the learned oracles exhibit consistent separability over the three domains through the lens of KL divergence and Bhttacharya coefficient. This is aligned with the intuition that a model that work well on a domain is likely to work well on other domains. 

\begin{figure}[!t]
	\centering
	\subfloat[\textlangle ViT, ViT\textrangle + CE]{\includegraphics[width=0.24\textwidth]{fig/hist/ce_vit_vit_val}    } \hfill
	\subfloat[\textlangle ViT, ViT\textrangle + Focal]{\includegraphics[width=0.24\textwidth]{fig/hist/focal_vit_vit_val}    } \hfill
	\subfloat[\textlangle ViT, ViT\textrangle + TCP]{\includegraphics[width=0.24\textwidth]{fig/hist/tcp_vit_vit_val}    } \hfill
	\subfloat[\textlangle ViT, ViT\textrangle +  SS]{\includegraphics[width=0.24\textwidth]{fig/hist/ss_vit_vit_val}    } \\
	\subfloat[\textlangle ViT, RSN\textrangle + CE]{\includegraphics[width=0.24\textwidth]{fig/hist/ce_vit_rsn_val}    } \hfill
	\subfloat[\textlangle ViT, RSN\textrangle + Focal]{\includegraphics[width=0.24\textwidth]{fig/hist/focal_vit_rsn_val}    } \hfill
	\subfloat[\textlangle ViT, RSN\textrangle + TCP]{\includegraphics[width=0.24\textwidth]{fig/hist/tcp_vit_rsn_val}    } \hfill
	\subfloat[\textlangle ViT, RSN\textrangle + SS]{\includegraphics[width=0.24\textwidth]{fig/hist/ss_vit_rsn_val}    } \\
% 	\subfloat[\textlangle RSN, ViT\textrangle + CE]{\includegraphics[width=0.24\textwidth]{fig/hist/ce_rsn_vit_val}    } \hfill
% 	\subfloat[\textlangle RSN, ViT\textrangle + Focal]{\includegraphics[width=0.24\textwidth]{fig/hist/focal_rsn_vit_val}    } \hfill
% 	\subfloat[\textlangle RSN, ViT\textrangle + TCP]{\includegraphics[width=0.24\textwidth]{fig/hist/tcp_rsn_vit_val}    } \hfill
% 	\subfloat[\textlangle RSN, ViT\textrangle + SS]{\includegraphics[width=0.24\textwidth]{fig/hist/ss_rsn_vit_val}    } \\
% 	\subfloat[\textlangle RSN, RSN\textrangle + CE]{\includegraphics[width=0.24\textwidth]{fig/hist/ce_rsn_rsn_val}    } \hfill
% 	\subfloat[\textlangle RSN, RSN\textrangle + Focal]{\includegraphics[width=0.24\textwidth]{fig/hist/focal_rsn_rsn_val}    } \hfill
% 	\subfloat[\textlangle RSN, RSN\textrangle + TCP]{\includegraphics[width=0.24\textwidth]{fig/hist/tcp_rsn_rsn_val}    } \hfill
% 	\subfloat[\textlangle RSN, RSN\textrangle + SS]{\includegraphics[width=0.24\textwidth]{fig/hist/ss_rsn_rsn_val}    } \\
	\caption{\label{fig:histogram_part}
    	Histograms of trustworthiness confidences w.r.t. all the loss functions on the ImageNet validation set.
    	The oracles that are used to generate the confidences are the ones used in \tabref{tbl:all_perf_w_std}. The histograms generated with \textlangle RSN, ViT\textrangle and \textlangle RSN, RSN\textrangle are provided in \appref{sec:histogram}.
    % 	appendix \ref{sec:histogram}.
    % 	the cross entropy (first column), focal loss (second column), TCP confidence loss (third column), and the proposed steep slope loss (fourth column) on the ImageNet validation set.
    % 	\REVISION{\textit{Baseline} indicates ResNet GEM.}
    	}
    \vspace{-1ex}
\end{figure}

% \begin{wrapfigure}{r}{0.5\textwidth}
\begin{table}[!t]
	\centering
	\caption{\label{tbl:perf_mnist}
	    Performance on MNIST and CIFAR-10.
	   % We use the official TCP code, but find out that there are several bugs and we couldn't reproduce the performance reported in their paper, not even close. Below are the best results by fixing a few bugs, according to the technical details in the paper.
	}
	\adjustbox{width=1\columnwidth}{
	\begin{tabular}{C{12ex} L{15ex} C{8ex} C{10ex} C{8ex} C{8ex} C{8ex} C{8ex} C{8ex}}
		\toprule
		\textbf{Dataset} & \textbf{Loss} & \textbf{Acc$\uparrow$} & \textbf{FPR-95\%-TPR$\downarrow$} & \textbf{AUPR-Error$\uparrow$} & \textbf{AUPR-Success$\uparrow$} & \textbf{AUC$\uparrow$} & \textbf{TPR$\uparrow$} & \textbf{TNR$\uparrow$} \\
		\cmidrule(lr){1-1} \cmidrule(lr){2-2} \cmidrule(lr){3-3} \cmidrule(lr){4-4} \cmidrule(lr){5-5} \cmidrule(lr){6-6} \cmidrule(lr){7-7} \cmidrule(lr){8-8} \cmidrule(lr){9-9}
		\multirow{6}{*}{MNIST} & MCP \cite{Hendrycks_ICLR_2017} & 99.10 & 5.56 & 35.05 & \textbf{99.99} & 98.63 & 99.89 & \textbf{8.89} \\
		& MCDropout \cite{Gal_ICML_2016} & 99.10 & 5.26 & 38.50 & \textbf{99.99} & 98.65 & - & - \\
		& TrustScore \cite{Jiang_NIPS_2018} & 99.10 & 10.00 & 35.88 & 99.98 & 98.20 & - & - \\
		& TCP \cite{Corbiere_NIPS_2019} & 99.10 & 3.33 & \textbf{45.89} & \textbf{99.99} & 98.82 & 99.71 & 0.00 \\
		& TCP$\dagger$ & 99.10 & 3.33 & 45.88 & \textbf{99.99} & 98.82 & 99.72 & 0.00 \\
		& SS & 99.10 & \textbf{2.22} & 40.86 & \textbf{99.99} & \textbf{98.83} & \textbf{100.00} & 0.00 \\
		\midrule
		\multirow{6}{*}{CIFAR-10} & MCP \cite{Hendrycks_ICLR_2017} & 92.19 & 47.50 & 45.36 & 99.19 & 91.53 & 99.64 & 6.66 \\
		& MCDropout \cite{Gal_ICML_2016} & 92.19 & 49.02 & 46.40 & \textbf{99.27} & 92.08 & - & - \\
		& TrustScore \cite{Jiang_NIPS_2018} & 92.19 & 55.70 & 38.10 & 98.76 & 88.47 & - & - \\
		& TCP \cite{Corbiere_NIPS_2019} & 92.19 & 44.94 & 49.94 & 99.24 & 92.12 & \textbf{99.77} & 0.00 \\
		& TCP$\dagger$ & 92.19 & 45.07 & 49.89 & 99.24 & 92.12 & 97.88 & 0.00 \\
		& SS & 92.19 & \textbf{44.69 }& \textbf{50.28}  & 99.26 & \textbf{92.22} & 98.46 & \textbf{28.04} \\
		\bottomrule	
	\end{tabular}}
\end{table}
% \end{wrapfigure}

% \begin{figure}[!t]
% 	\centering
% 	\subfloat[Official TCP  plot]{\includegraphics[width=0.45\textwidth]{fig/hist/tcphp_mnist_tefeat}    } \hfill
% 	\subfloat[Proposed with pretrained baseline ]{\includegraphics[width=0.45\textwidth]{fig/hist/steephp_mnist_tefeat}    } \\
% 	\subfloat[TCP with trained baseline]{\includegraphics[width=0.45\textwidth]{fig/hist/tcplp_mnist_tefeat}    } \hfill
% 	\subfloat[Proposed with trained baseline ]{\includegraphics[width=0.45\textwidth]{fig/hist/steeplp_mnist_tefeat}    }
% 	\caption{
%     	Reproduction and comparison.
%     	}
% \end{figure}

\noindent\textbf{Separability between Distributions of Correct Predictions and Incorrect Predictions}.
As observed in \figref{fig:histogram_part}, the confidences w.r.t. correct and incorrect predictions follow Gaussian-like distributions.
Hence, we can compute the separability between the distributions of correct and incorrect predictions from a probabilistic perspective.
% There are two common tools to achieve the goal, \ie Kullback–Leibler (KL) divergence \cite{Kullback_AMS_1951} and Bhattacharyya distance \cite{Bhattacharyya_JSTOR_1946}.
Given the distribution of correct predictions {\small $\mathcal{N}_{1}(\mu_{1}, \sigma^{2}_{1})$} and the distribution of correct predictions {\small $\mathcal{N}_{2}(\mu_{2}, \sigma^{2}_{2})$}, we use the average Kullback–Leibler (KL) divergence {\small $\bar{d}_{KL}(\mathcal{N}_{1}, \mathcal{N}_{2})$} \cite{Kullback_AMS_1951} and Bhattacharyya distance {\small $d_{B}(\mathcal{N}_{1}, \mathcal{N}_{2})$} \cite{Bhattacharyya_JSTOR_1946} to measure the separability. 
More details and the quantitative results are reported in \appref{sec:separability}. 
In short, the proposed loss leads to larger separability than the baseline loss functions. 
This implies that the proposed loss is more effective to differentiate incorrect predictions from correct predictions.

\noindent\textbf{Performance on Small-Scale Datasets}.
We also provide comparative experimental results on small-scale datasets, \ie MNIST \cite{Lecun_IEEE_1998} and CIFAR-10 \cite{Krizhevsky_TR_2009}.
\REVISION{The results are reported in \tabref{tbl:perf_mnist}.}
% The experiment details and results are reported in \appref{sec:mnist}.
The proposed loss outperforms TCP$\dagger$ on metric FPR-95\%-TPR on both MNIST and CIFAR-10, and additionally achieved good performance on metrics AUPR-Error and TNR on CIFAR-10.
This shows the proposed loss is able to adapt to relatively simple data.
\REVISION{More details can be found in \appref{sec:mnist}.}

\noindent\textbf{Generalization to Unseen Domains}.
In practice, the oracle may run into the data in the domains that are different from the ones of training samples.
Thus, it is interesting to find out how well the learned oracles generalize to the unseen domain data.
% To this end, we apply a style transfer method \cite{Geirhos_ICLR_2019} and the functional adversarial attack method \cite{Laidlaw_NeurIPS_2019} to generate the stylized ImageNet validation set and the adversarial ImageNet validation set.
Using the oracles trained with the ImageNet training set (\ie the ones used in \tabref{tbl:all_perf_w_std}), we evaluate it on the stylized ImageNet validation set \cite{Geirhos_ICLR_2019}, adversarial ImageNet validation set \cite{Laidlaw_NeurIPS_2019}, and corrupted ImageNet validation set \cite{Hendrycks_ICLR_2018}.
% and evaluated on the two variants of the validation set.
\textlangle ViT, ViT\textrangle~ is used in the experiment.

The results on the stylized ImageNet, adversarial ImageNet, and ImageNet-C are reported in \tabref{tbl:perf_vit_vit}, \REVISION{More results on ImageNet-C are reported in \tabref{tbl:perf_imagenetc}}.
As all unseen domains are different from the one of the training set, the classification accuracies are much lower than the ones in \tabref{tbl:all_perf_w_std}. 
The adversarial validation set is also more challenging than the stylized validation set \REVISION{and the corrupted validation set}.
As a result, the difficulty affects the scores across all metrics.
The oracles trained with the baseline loss functions are still prone to recognize the incorrect prediction to be trustworthy.
The proposed loss consistently improves the performance on FPR-95\%-TPR, AUPR-Sucess, AUC, and TNR.
Note that the adversarial perturbations are computed on the fly \cite{Laidlaw_NeurIPS_2019}. Instead of truncating the sensitive pixel values and saving into the images files, we follow the experimental settings in \cite{Laidlaw_NeurIPS_2019} to evaluate the oracles on the fly.
Hence, the classification accuracies w.r.t. various loss function are slightly different but are stably around 6.14\%.

% Also, we report the performances on each domain in \tabref{tbl:perf_vit_vit} and \tabref{tbl:perf_rsn_vit}.
% They shows that the cross entropy and focal loss work well on the regular validation set, but work poorly on the stylized and adversarial validation sets. This confirms the overfitting resulted from the learning with the cross entropy and focal loss.

\begin{table}[!t]
	\centering
	\vspace{-1ex}
	\caption{\label{tbl:perf_vit_vit}
	   % Histograms of trustworthiness confidences w.r.t. all the loss functions on the stylized ImageNet validation set (stylized val) and the adversarial ImageNet validation set (adversarial val). \textlangle ViT, ViT\textrangle is used in the experiment and the domains of the two validation sets are different from the one of the training set that is used for training the oracle.
	    Performance on the stylized ImageNet validation set, the adversarial ImageNet validation set, and one (Defocus blur) of validation sets in ImageNet-C. Defocus blus is at at the highest level of severity.
	    \textlangle ViT, ViT\textrangle~ is used in the experiment and the domains of the two validation sets are different from the one of the training set that is used for training the oracle. The corresponding histograms are available in \appref{sec:histogram}. More results on ImageNet-C can be found in \tabref{tbl:perf_imagenetc}.
	   % In this experiment, ViT is used for both the oracle backbone and the classifier. The oracle is trained with the CE loss, the focal loss, and the proposed steep slope loss on the ImageNet training set. The resulting oracles w.r.t. each loss are evaluated on the three validation sets. The classifier is used in the evaluation mode in the experiment.
	}
	\adjustbox{width=1\columnwidth}{
	\begin{tabular}{C{15ex} L{10ex} C{8ex} C{10ex} C{8ex} C{8ex} C{8ex} C{8ex} C{8ex}}
		\toprule
		\textbf{Dataset} & \textbf{Loss} & \textbf{Acc$\uparrow$} & \textbf{FPR-95\%-TPR$\downarrow$} & \textbf{AUPR-Error$\uparrow$} & \textbf{AUPR-Success$\uparrow$} & \textbf{AUC$\uparrow$} & \textbf{TPR$\uparrow$} & \textbf{TNR$\uparrow$} \\
		\cmidrule(lr){1-1} \cmidrule(lr){2-2} \cmidrule(lr){3-3} \cmidrule(lr){4-4} \cmidrule(lr){5-5} \cmidrule(lr){6-6} \cmidrule(lr){7-7} \cmidrule(lr){8-8} \cmidrule(lr){9-9}
% 		& \multicolumn{7}{c}{Regular validation set} \\
% 		\cmidrule(lr){1-1} \cmidrule(lr){2-8}
% 		CE & 83.90 & 92.83 & 15.08 & 84.99 & 52.78 & 100.00 & 0.01 \\
% 		Focal & 83.90 & 92.68 & 14.69 & 85.46 & 53.47 & 99.06 & 1.61 \\
% 		TCP & 83.90 & 88.07 & 12.86 & 87.80 & 60.45 & 99.72 & 1.02 \\
% % 		TCP & 83.90 & 86.45 & 12.12 & 88.95 & 63.39 & 99.07 & 3.06 \\
% 		SS & 83.90 & 80.89 & 10.31 & 92.90 & 73.31 & 88.44 & 35.64 \\
% 		\midrule
% 		& \multicolumn{7}{c}{Stylized validation set} \\
% 		\cmidrule(lr){1-1} \cmidrule(lr){2-8}
		\multirow{4}{*}{Stylized \cite{Geirhos_ICLR_2019}} & CE & 15.94 & 95.52 & 84.18 & 15.86 & 49.07 & \textbf{99.99} & 0.02 \\
		& Focal \cite{Lin_ICCV_2017} & 15.94 & 95.96 & \textbf{85.90} & 14.30 & 46.01 & 99.71 & 0.25 \\
		& TCP \cite{Corbiere_NIPS_2019} & 15.94 & 93.42 & 80.17 & 21.25 & 57.29 & 99.27 & 0.00 \\
% 		& TCP & 15.94 & 93.19 & 78.53 & 24.52 & 60.31 & 95.41 & 6.24 \\
		& SS & 15.94 & \textbf{89.38} & 75.08 & \textbf{34.39} & \textbf{67.68} & 44.42 & \textbf{81.22} \\
        \midrule
% 		& \multicolumn{7}{c}{Adversarial validation set} \\
% 		\cmidrule(lr){1-1} \cmidrule(lr){2-8}
        \multirow{4}{*}{Adversarial \cite{Laidlaw_NeurIPS_2019}} & CE & 6.14 & 94.35 & \textbf{93.70} & 6.32 & 51.28 & \textbf{99.97} & 0.06 \\
        & Focal \cite{Lin_ICCV_2017} & 6.15 & 93.67 & 93.48 & 6.56 & 52.39 & 99.06 & 1.43 \\
        & TCP \cite{Corbiere_NIPS_2019} & 6.11 & 93.94 & 92.77 & 7.55 & 55.81 & 99.71 & 0.00 \\
        & SS  & 6.16 & \textbf{90.07} & 90.09 & \textbf{13.07} & \textbf{65.36} & 87.10 & \textbf{24.33} \\ \midrule
        \multirow{4}{*}{Defocus blur \cite{Hendrycks_ICLR_2018}} & CE & 31.83 & 94.46 & \textbf{68.56} & 31.47 & 50.13 & \textbf{99.15} & 1.07 \\
		& Focal \cite{Lin_ICCV_2017} & 31.83 & 94.98  & 66.87 & 33.24 & 51.28 & 96.70 & 3.26 \\
		& TCP \cite{Corbiere_NIPS_2019} & 31.83 & 93.50 & 64.67 & 36.05 & 54.27 & 96.71 & 4.35 \\
		& SS & 31.83 & \textbf{90.18} & 57.95 & \textbf{48.80} & \textbf{64.34} & 77.79 & \textbf{37.29} \\
		\bottomrule	
	\end{tabular}}
\end{table}

\begin{figure}[!b]
	\centering
	\subfloat[]{\includegraphics[width=0.32\textwidth]{fig/risk/risk_vit_vit} \label{fig:risk_vit}} \hfill
	\subfloat[]{\includegraphics[width=0.30\textwidth]{fig/analysis/loss} \label{fig:abl_loss}} \hfill
	\subfloat[]{\includegraphics[width=0.32\textwidth]{fig/analysis/tpr_tnr} \label{fig:abl_tpr_tnr}} 
	\caption{\label{fig:anal_abl}
    	Analyses based on \textlangle ViT, ViT\textrangle. (a) are the curves of risk vs. coverage. Selective risk represents the percentage of errors in the remaining validation set for a given coverage. (b) are the curves of loss vs. $\alpha^{-}$. (c) are TPR and TNR against various $\alpha^{-}$.
    	}
\end{figure}

\noindent\textbf{Selective Risk Analysis}.
Risk-coverage curve is an important technique for analyzing trustworthiness through the lens of the rejection mechanism in the classification task \cite{Corbiere_NIPS_2019,Geifman_NIPS_2017}. 
In the context of predicting trustworthiness, selective risk is the empirical loss that takes into account the decisions, \ie to trust or not to trust the prediction. 
Correspondingly, coverage is the probability mass of the non-rejected region. As can see in \figref{fig:risk_vit}, the proposed loss leads to significantly lower risks, compared to the other loss functions.
We present the risk-coverage curves w.r.t. all the combinations of oracles and classifiers in \appref{sec:risk}.
They consistently exhibit similar pattern.

\noindent\textbf{Ablation Study}.
In contrast to the compared loss functions, the proposed loss has more hyperparameters to be determined, \ie $\alpha^{+}$ and $\alpha^{-}$.
As the proportion of correct predictions is usually larger than that of incorrect predictions, we would prioritize $\alpha^{-}$ over $\alpha^{+}$ such that the oracle is able to recognize a certain amount of incorrect predictions.
In other words, we first search for $\alpha^{-}$ by freezing $\alpha^{+}$, and then freeze $\alpha^{-}$ and search for $\alpha^{+}$.
\figref{fig:abl_loss} and \ref{fig:abl_tpr_tnr} show how the loss, TPR, and TNR vary with various $\alpha^{-}$. In this analysis, the combination \textlangle ViT, ViT\textrangle~ is used and $\alpha^{+}=1$.
We can see that $\alpha^{-}=3$ achieves the optimal trade-off between TPR and TNR.
We follow a similar search strategy to determine $\alpha^{+}=2$ and $\alpha^{-}=5$ for training the oracle with ResNet backbone.
% With the classifier ViT and the ViT based oracle, we show how the performance vary when $\alpha^{+}$ and $\alpha^{-}$ change.  

\noindent\textbf{Effects of Using $z=\bm{w}^{\top}\bm{x}^{out}+b$}.
Using the signed distance as $z$, \ie $z = \frac{\bm{w}^{\top} \bm{x}^{out}+b}{\|\bm{w}\|}$, has a geometric interpretation as shown in \figref{fig:workflow_a}.
However, the main-stream models \cite{He_CVPR_2016,Tan_ICML_2019,Dosovitskiy_ICLR_2021} use $z=\bm{w}^{\top}\bm{x}^{out}+b$. 
Therefore, we provide the corresponding results in appendix \ref{sec:appd_z}, which are generated by the proposed loss taking the output of the linear function as input.
In comparison with the results of using $z = \frac{\bm{w}^{\top} \bm{x}^{out}+b}{\|\bm{w}\|}$, using $z=\bm{w}^{\top}\bm{x}^{out}+b$ yields comparable scores of FPR-95\%-TPR, AUPR-Error, AUPR-Success, and AUC.
Also, TPR and TNR are moderately different between $z = \frac{\bm{w}^{\top} \bm{x}^{out}+b}{\|\bm{w}\|}$ and $z=\bm{w}^{\top}\bm{x}^{out}+b$, when $\alpha^{+}$ and $\alpha^{-}$ are fixed.
This implies that TPR and TNR are sensitive to $\|\bm{w}\|$. 
% \REVISION{We discuss the reason in \appref{sec:effect_normalization}.}
% 
\REVISION{
This is because the normalization by $\|w\|$ would make $z$ more dispersed in value than the variant without normalization. 
In other words, the normalization leads to long-tailed distributions while no normalization leads to short-tailed distributions. 
Given the same threshold, TNR (TPR) is determined by the location of the distribution of negative (positive) examples and the extent of short/long tails. 
Our analysis on the histograms generated with and without $\|w\|$ normalization verifies this point.
}

% \noindent\textbf{Learning with Class Weights}. We witness the imbalancing characteristics in the learning task for predicting trustworthiness. Table xx shows that one of most common learning techniques with imbalanced data, \ie using class weights, is not effective. The reason is that applying class weights to the loss function, \eg cross entropy, it only scale up the graph along y-axis. However, the long tail regions still slow down the move of the features w.r.t. false positive or false negative towards the well-classified regions.

% \noindent\textbf{Separability between Distributions of Correct Predictions and Incorrect Predictions}.
% As observed in \figref{fig:histogram_part}, the confidences w.r.t. correct and incorrect predictions follow Gaussian-like distributions.
% Hence, we can compute the separability between the distributions of correct and incorrect predictions from a probabilistic perspective.
% % There are two common tools to achieve the goal, \ie Kullback–Leibler (KL) divergence \cite{Kullback_AMS_1951} and Bhattacharyya distance \cite{Bhattacharyya_JSTOR_1946}.
% Given the distribution of correct predictions $\mathcal{N}_{1}(\mu_{1}, \sigma^{2}_{1})$ and the distribution of correct predictions $\mathcal{N}_{2}(\mu_{2}, \sigma^{2}_{2})$, we use the average Kullback–Leibler (KL) divergence $\bar{d}_{KL}(\mathcal{N}_{1}, \mathcal{N}_{2})$ \cite{Kullback_AMS_1951} and Bhattacharyya distance $d_{B}(\mathcal{N}_{1}, \mathcal{N}_{2})$ \cite{Bhattacharyya_JSTOR_1946} to measure the separability. More details and the quantitative results are reported in \appref{sec:separability}. In short, the proposed loss leads to larger separability than the baseline loss functions. This implies that the proposed loss is more effective to differentiate incorrect predictions from correct predictions.

\noindent\textbf{Steep Slope Loss vs. Class-Balanced Loss}.
We compare the proposed loss to the class-balanced loss \cite{Cui_CVPR_2019}, which is based on a re-weighting strategy.
The results are reported in \appref{sec:cbloss}.
Overall, the proposed loss outperforms the class-balanced loss, which implies that the imbalance characteristics of predicting trustworthiness is different from that of imbalanced data classification.

% KL divergence is used to measure the difference between two distributions \cite{Cantu_Springer_2004,Luo_TNNLS_2020}, while Bhattacharyya distance is used to measure the similarity of two probability distributions. Given two Gaussian distributions $\mathcal{N}_{1}(\mu_{1}, \sigma^{2}_{1})$ and $\mathcal{N}_{2}(\mu_{2}, \sigma^{2}_{2})$, we use the averaged KL divergence, \ie $\bar{d}_{KL}(\mathcal{N}_{1}, \mathcal{N}_{2}) = (d_{KL}(\mathcal{N}_{1}, \mathcal{N}_{2}) + d_{KL}(\mathcal{N}_{2}, \mathcal{N}_{1}))/2$, where $d_{KL}(\mathcal{N}_{1}, \mathcal{N}_{2})=\log\frac{\sigma_{2}}{\sigma_{1}}+\frac{\sigma_{1}^{2}+(\mu_{1}-\mu_{2})^{2}}{2\sigma_{2}^{2}}-\frac{1}{2}$ is not symmetrical. On the other hand, Bhattacharyya distance is defined as $d_{B}(\mathcal{N}_{1}, \mathcal{N}_{2})=\frac{1}{4}\ln \left( \frac{1}{4} \left( \frac{\sigma^{2}_{1}}{\sigma^{2}_{2}}+\frac{\sigma^{2}_{2}}{\sigma^{2}_{1}}+2 \right) \right) + \frac{1}{4} \left( \frac{(\mu_{1}-\mu_{2})^{2}}{\sigma^{2}_{1}+\sigma^{2}_{2}} \right)$. A larger $\bar{d}_{KL}$ or $d_{B}$ indicates that the two distributions are further away from each other.


% We hypothesize that $x$ w.r.t. positive and negative samples both follow Gaussian distributions. The discriminativeness of features is an important characteristic that correlates to the performance, \eg accuracy. We are interested in measures of separability of feature distributions, which reflect the discriminativeness from a probabilistic perspective. There are two common tools to achieve the goal, \ie Kullback–Leibler (KL) divergence \cite{Kullback_AMS_1951} and Bhattacharyya distance \cite{Bhattacharyya_JSTOR_1946}. Usually, KL divergence is used to measure the difference between two distributions \cite{Cantu_Springer_2004,Luo_TNNLS_2020}, while Bhattacharyya distance is used to measure the similarity of two probability distributions. Given two Gaussian distributions $\mathcal{N}_{1}(\mu_{1}, \sigma^{2}_{1})$ and $\mathcal{N}_{2}(\mu_{2}, \sigma^{2}_{2})$, we use an averaged KL divergence as in this work, \ie $\bar{d}_{KL}(\mathcal{N}_{1}, \mathcal{N}_{2}) = (d_{KL}(\mathcal{N}_{1}, \mathcal{N}_{2}) + d_{KL}(\mathcal{N}_{2}, \mathcal{N}_{1}))/2$, where $d_{KL}(\mathcal{N}_{1}, \mathcal{N}_{2})$ is the KL divergence between $\mathcal{N}_{1}$ and $\mathcal{N}_{2}$ (not symmetrical). On the other hand, Bhattacharyya distance is defined as $d_{B}(\mathcal{N}_{1}, \mathcal{N}_{2})=\frac{1}{4}\ln \left( \frac{1}{4} \left( \frac{\sigma^{2}_{1}}{\sigma^{2}_{2}}+\frac{\sigma^{2}_{2}}{\sigma^{2}_{1}}+2 \right) \right) + \frac{1}{4} \left( \frac{(\mu_{1}-\mu_{2})^{2}}{\sigma^{2}_{1}+\sigma^{2}_{2}} \right)$. In this work, we use Bhattacharyya coefficient that measures the amount of overlap between two distributions, instead of Bhattacharyya distance. Bhattacharyya coefficient is defined as $c_{B}(\mathcal{N}_{1}, \mathcal{N}_{2}) = \exp(-d_{B}(\mathcal{N}_{1}, \mathcal{N}_{2}))$. $c_{B} \in [0,1]$, where 1 indicates a full overlap and 0 indicates no overlap.

% \noindent\textbf{Semantics Difference between Predicting Trustworthiness and Classification}. As we use ViT for both the oracle and classifier, it is interesting to find out what features are leaned for predicting trustworthiness, in comparison to the features learned for classification. Hence, we compute the $l_{1}$ and $l_{2}$ distances between the features generated by the learned oracle and the features generated by the pre-trained classifier. The features are the inputs to the last layer of ViT, \ie 768-dimensional vectors.

% The mean and standard deviation of distances over all the samples in the training and validation sets are provided in \tabref{tbl:anal_diff}. Note that a smaller distance indicates higher similarity between two features. Overall, the mean of distances w.r.t. the three loss functions are large, but the focal loss yields the smallest averaged distance, which implies that the oracle learned with the focal yields the most similar features as the ones generated by the pre-trained classifier. One of possible reasons is that the focal loss prohibits the oracle training.

% Comparison of classifier backbone and oracle backbone

% Per class accuracy, precision, recall, F1

% \noindent\textbf{Taking Features as Input}
% \figref{fig:anal_featinput} shows the distributions of discriminative features generated by a multi-layer perceptron (MLP),, which plays as an oracle. The MLP takes the features generated by the classifier, instead of images, as input. The MLP-based oracle is training on the training set and is evaluated on the validation set. The figure shows that the oracle barely distinguish between positives and negatives. Because all the features are on the right-hand side of the decision boundary $x=0$.

% focal loss vs proposed

% \begin{figure}[!t]
	\centering
	\subfloat{\includegraphics[width=0.32\textwidth]{fig/analysis/anal_featinput_ce}    } \hfill
	\subfloat{\includegraphics[width=0.32\textwidth]{fig/analysis/anal_featinput_focal}    } \hfill
	\subfloat{\includegraphics[width=0.32\textwidth]{fig/analysis/anal_featinput_ss}    } \\
	\caption{\label{fig:anal_featinput}
    	Analysis of taking the features of the classifier as input to the oracle on the ImageNet validation set. In this experiment, ViT is used for both the oracle backbone and the classifier. The features are 768-dimensional vectors. The classifier is used in the evaluation mode in the experiment.
    % 	\REVISION{\textit{Baseline} indicates ResNet GEM.}
    	}
\end{figure}

% \begin{table}[!t]
	\centering
	\caption{\label{tbl:anal_diff}
	    Analysis of the difference of the output features between the classifier backbone and the oracle backbone in terms of $l_{1}$ and $l_{2}$ distances. The common backbone is ViT. The oracle backbone is trained for predicting trustworthiness, while the classifier backbone is pre-trained for classification.
	}
	\adjustbox{width=1\columnwidth}{
	\begin{tabular}{L{7ex} C{14ex} C{14ex} C{14ex} C{14ex}}
		\toprule
		& \multicolumn{2}{c}{Training} & \multicolumn{2}{c}{Validation} \\
		\cmidrule(lr){2-3} \cmidrule(lr){4-5}
		Loss & $l_{1}$ & $l_{2}$ & $l_{1}$ & $l_{2}$ \\
		\cmidrule(lr){1-1} \cmidrule(lr){2-2} \cmidrule(lr){3-3} \cmidrule(lr){4-4} \cmidrule(lr){5-5}
		CE & 74.0674$\pm$23.9773 & 3.4074$\pm$1.0967 & 78.4107$\pm$24.9338 & 3.6051$\pm$1.1402 \\
        Focal & 29.0901$\pm$8.5641 & 1.3527$\pm$0.3933 & 30.6497$\pm$8.9262 & 1.4240$\pm$0.4100 \\
        SS & 70.1997$\pm$32.8220 & 3.2129$\pm$1.4973 & 77.3162$\pm$33.4536 & 3.5378$\pm$1.5271 \\
		\bottomrule	
	\end{tabular}}
\end{table}

% \noindent\textbf{Ablation Study}. With the classifier ViT and the ViT based oracle, we show how the performance vary when $\alpha^{+}$ and $\alpha^{-}$ change.  

% \noindent\textbf{Generalization to Unseen Classifier}.
% As the oracle is trained by observing what a classifier predicts the label for an image, the knowledge learned in this way highly correlates to the behaviours of the classifier. It is interesting to know how the knowledge learned by the oracle generalizes to other unseen classifiers. To this end, we use the ViT based oracle that is trained with a ViT classifier to predict the trustworthiness of a ResNet-50 on the adversarial validation set, which is the most challenging in the three sets. 
% For the proposed loss, we use $\alpha^{+}=1$ and $\alpha^{-}=3$ for the oracle that is based on ViT's backbone, while we use $\alpha^{+}=2$ and $\alpha^{-}=5$ for the oracle that is based on ResNet's backbone.


\section{Related Work}\label{sec:related}
 
The authors in \cite{humphreys2007noncontact} showed that it is possible to extract the PPG signal from the video using a complementary metal-oxide semiconductor camera by illuminating a region of tissue using through external light-emitting diodes at dual-wavelength (760nm and 880nm).  Further, the authors of  \cite{verkruysse2008remote} demonstrated that the PPG signal can be estimated by just using ambient light as a source of illumination along with a simple digital camera.  Further in \cite{poh2011advancements}, the PPG waveform was estimated from the videos recorded using a low-cost webcam. The red, green, and blue channels of the images were decomposed into independent sources using independent component analysis. One of the independent sources was selected to estimate PPG and further calculate HR, and HRV. All these works showed the possibility of extracting PPG signals from the videos and proved the similarity of this signal with the one obtained using a contact device. Further, the authors in \cite{10.1109/CVPR.2013.440} showed that heart rate can be extracted from features from the head as well by capturing the subtle head movements that happen due to blood flow.

%
The authors of \cite{kumar2015distanceppg} proposed a methodology that overcomes a challenge in extracting PPG for people with darker skin tones. The challenge due to slight movement and low lighting conditions during recording a video was also addressed. They implemented the method where PPG signal is extracted from different regions of the face and signal from each region is combined using their weighted average making weights different for different people depending on their skin color. 
%

There are other attempts where authors of \cite{6523142,6909939, 7410772, 7412627} have introduced different methodologies to make algorithms for estimating pulse rate robust to illumination variation and motion of the subjects. The paper \cite{6523142} introduces a chrominance-based method to reduce the effect of motion in estimating pulse rate. The authors of \cite{6909939} used a technique in which face tracking and normalized least square adaptive filtering is used to counter the effects of variations due to illumination and subject movement. 
The paper \cite{7410772} resolves the issue of subject movement by choosing the rectangular ROI's on the face relative to the facial landmarks and facial landmarks are tracked in the video using pose-free facial landmark fitting tracker discussed in \cite{yu2016face} followed by the removal of noise due to illumination to extract noise-free PPG signal for estimating pulse rate. 

Recently, the use of machine learning in the prediction of health parameters have gained attention. The paper \cite{osman2015supervised} used a supervised learning methodology to predict the pulse rate from the videos taken from any off-the-shelf camera. Their model showed the possibility of using machine learning methods to estimate the pulse rate. However, our method outperforms their results when the root mean squared error of the predicted pulse rate is compared. The authors in \cite{hsu2017deep} proposed a deep learning methodology to predict the pulse rate from the facial videos. The researchers trained a convolutional neural network (CNN) on the images generated using Short-Time Fourier Transform (STFT) applied on the R, G, \& B channels from the facial region of interests.
The authors of \cite{osman2015supervised, hsu2017deep} only predicted pulse rate, and we extended our work in predicting variance in the pulse rate measurements as well.

All the related work discussed above utilizes filtering and digital signal processing to extract PPG signals from the video which is further used to estimate the PR and PRV.  %
The method proposed in \cite{kumar2015distanceppg} is person dependent since the weights will be different for people with different skin tone. In contrast, we propose a deep learning model to predict the PR which is independent of the person who is being trained. Thus, the model would work even if there is no prior training model built for that individual and hence, making our model robust. 

%

\section{Conclusions and Future Work}

This paper proposes a privacy aware recommendation framework based on privacy calculus theory to study what will happen if the platform gives users control over their data.
To avoid the great cost in online experiments, we propose to use reinforcement learning to simulate the users' privacy decision making under different platform mechanisms and recommendation models on public benchmark datasets.
The results show a well-designed data disclosure mechanism can perform much better than the popular ``all or nothing'' binary mechanism.
Our work provides some insights to improve current rough solutions in privacy protection regulations, e.g., opt-in under GDPR and opt-out under CCPA.

This paper only takes the first step in studying users' privacy decision making under different platform mechanisms, and several directions remain to be explored.
First, a more complex and accurate privacy cost function can help us better understand users' privacy decision making.
In this work we have modeled different users with their individual privacy sensitivity weights, and one may modify the privacy cost function on the effects of the users' trust towards the platform in the future.
Second, more sophisticated platform mechanisms are also worth exploring.
Recent mechanism design works also turn to the perspectives of deep neural network based mechanism designs, which can be explored with our proposed framework.
Last but not least, deploying online experiments and analyzing users' decisions in real-world can facilitate further researches.



\clearpage

\bibliographystyle{ACM-Reference-Format}
\bibliography{sample-base}



\end{document}
