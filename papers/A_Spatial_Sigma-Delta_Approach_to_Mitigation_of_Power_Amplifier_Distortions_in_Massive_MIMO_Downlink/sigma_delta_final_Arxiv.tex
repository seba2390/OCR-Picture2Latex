\newif\ifconfver
%\confverfalse      %declaring conference version false
%\confvertrue        %declaring conference version true

\newif\ifplainver  %declare a zplain version
\plainvertrue
%\plainverfalse

\newif\ifhide  %hide something, like a  proof
\hidetrue
%\hidefalse

\ifplainver
    \confverfalse   %automatically disable conf. version argument if it's plain
\fi

\ifconfver
     \documentclass[10pt,twocolumn,twoside]{IEEEtran}
\else
    \ifplainver
        \documentclass[11pt]{article}
        \usepackage{fullpage}
    \else
        \documentclass[12pt,draftcls,onecolumn]{IEEEtran}
    \fi
\fi

\usepackage{calc,amsfonts,amssymb,amsmath,bm,url,color,theorem,graphicx,cite}
\usepackage{psfrag,float}
\usepackage{algorithm}
\usepackage{algorithmic}
\usepackage{soul}
\usepackage{enumerate}
\usepackage{bbm}
\usepackage{multirow}
%\usepackage{subcaption}
\usepackage{array}
\usepackage{epstopdf}
\usepackage{diagbox}
\usepackage{cases}
\usepackage{makecell}
\usepackage{subfigure}

\newlength{\articlesectionshift}%
\setlength{\articlesectionshift}{10pt}%
%\addtolength{\cftsecindent}{\articlesectionshift}%

\let\LaTeXStandardSection\section
\let\LaTeXStandardTheSection\thesection
\let\LaTeXStandardTheSubSection\thesubsection
\let\LaTeXStandardTheSubSubSection\thesubsubsection
\let\LaTeXStandardTheParagraph\theparagraph


\makeatletter
\newcounter{titlecounter}

%\xpretocmd{\maketitle}{\ifnumgreater{\value{titlecounter}}{1}}{\clearpage}{}{} % Well, this is lazy at the end ;-)
%\xpatchcmd{\maketitle}{\let\maketitle\relax\let\@maketitle\relax}{\refstepcounter{titlecounter}\begingroup
%
%
%\makeatother \addtocontents{toc}{\begingroup\addtolength{\cftsecindent}{-\articlesectionshift}}%
%	\addcontentsline{toc}{section}{\protect{\numberline{\thetitlecounter}{\@title~ \@author}}}%
%	\addtocontents{toc}{\endgroup}
%}{%
%	\typeout{Patching was successful}
%}{%
%	\typeout{patching failed}
%}%

\def\@IEEEdestroythesectionargument#1{\LaTeXStandardSection{#1}}%
%\def\@IEEEdestroythesectionargument#1{\section{#1}}%


%\xapptocmd{\maketitle}{%
%	\renewcommand{\thesection}{\LaTeXStandardTheSection}%
%	\renewcommand{\thesubsection}{\LaTeXStandardTheSubSection}%
%	\renewcommand{\thesubsubsection}{\LaTeXStandardTheSubSubSection}%
%	\renewcommand{\theparagraph}{\LaTeXStandardTheParagraph}%
%}{}{}%

\@addtoreset{section}{titlecounter}

%=====

%---
\newlength{\customfigwidth}
\setlength{\customfigwidth}{0.31\textwidth}
%\ifconfver
%	\setlength{\customfigwidth}{.85\linewidth}
%\else
%	\setlength{\customfigwidth}{0.3\textwidth}
%\fi	

\newcolumntype{M}[1]{>{\centering\arraybackslash}m{#1}}

%--- define color
\definecolor{orange}{RGB}{255,107,0}
\def\blue{\color{blue}}
%\def\red{\color{red}}
\def\orange{\color{orange}}
\def\green{\color{green}}
\definecolor{cpink}{rgb}{0.7, 0.11, 0.11}
\def\red{\color{cpink}}

%--- defining the theorem and stuff

\newtheorem{Fact}{Fact}
\newtheorem{Lemma}{Lemma}
\newtheorem{Prop}{Proposition}
\newtheorem{Theorem}{Theorem}
\newtheorem{Def}{Definition}
\newtheorem{Corollary}{Corollary}
\newtheorem{Property}{Property}
\newtheorem{Observation}{Observation}
\newtheorem{Assumption}{A\!\!}
\newtheorem{Asm}{Assumption}
\newtheorem{Exa}{Example}
\newtheorem{Remark}{Remark}


%make sure to keep this for convenience later
\newcommand\bw{\ensuremath{{\bm w}}}
\newcommand\bW{\ensuremath{{\bm W}}}
\newcommand\cW{\ensuremath{\bm{\mathcal{W}}}}
\newcommand\bbw{\ensuremath{{\bar{\bm w}}}}
\newcommand\bq{\ensuremath{{\bm q}}}
\newcommand\bx{\ensuremath{{\bm x}}}
\newcommand\by{\ensuremath{{\bm y}}}
\newcommand\bG{\ensuremath{{\bm G}}}
\newcommand\bh{\ensuremath{{\bm h}}}
\newcommand\bH{\ensuremath{{\bm H}}}
\newcommand\cH{\ensuremath{\bm{\mathcal{H}}}}
\newcommand\bbh{\ensuremath{\bar{\bm h}}}
\newcommand\hbh{\ensuremath{\hat{\bm h}}}
\newcommand\be{\ensuremath{{\bm e}}}
\newcommand\bz{\ensuremath{{\bm z}}}
\newcommand\bp{\ensuremath{{\bm p}}}
\newcommand\bP{\ensuremath{{\bm P}}}
\newcommand\bR{\ensuremath{{\bm R}}}
\newcommand\bT{\ensuremath{{\bm T}}}
\newcommand\bX{\ensuremath{{\bm X}}}
\newcommand\bZ{\ensuremath{{\bm Z}}}
\newcommand\cZ{\ensuremath{\bm{\mathcal{Z}}}}
\newcommand\bC{\ensuremath{{\bm C}}}
\newcommand\bc{\ensuremath{{\bm c}}}
\newcommand\ba{\ensuremath{{\bm a}}}
\newcommand\bba{\ensuremath{\bar{\bm a}}}
\newcommand\bA{\ensuremath{{\bm A}}}
\newcommand\bbA{\ensuremath{\bar{\bm A}}}
\newcommand\bb{\ensuremath{{\bm b}}}
\newcommand\bg{\ensuremath{{\bm g}}}
\newcommand\bB{\ensuremath{{\bm B}}}
\newcommand\blam{\ensuremath{{\bm \lambda}}}
\newcommand\bmu{\ensuremath{{\bm \mu}}}
\newcommand\balp{\ensuremath{{\bm \alpha}}}
\newcommand\bbeta{\ensuremath{{\bm \beta}}}
\newcommand\bXi{\ensuremath{{\bm \Xi}}}
\newcommand\bPi{\ensuremath{{\bm \Pi}}}
\newcommand\bbPi{\ensuremath{\bar{\bm \Pi}}}
\newcommand\bF{\ensuremath{{\bm F}}}
\newcommand\bbF{\ensuremath{\bar{\bm F}}}
\newcommand\bd{\ensuremath{{\bm d}}}
\newcommand\bD{\ensuremath{{\bm D}}}
\newcommand\bu{\ensuremath{{\bm u}}}
\newcommand\bv{\ensuremath{{\bm v}}}
\newcommand\bSig{\ensuremath{{\bm \Sigma}}}
\newcommand\bsig{\ensuremath{{\bm \sigma}}}
\newcommand\beeta{\ensuremath{{\bm \eta}}}
\newcommand\bPhi{\ensuremath{{\bm \Phi}}}
\newcommand\bPsi{\ensuremath{{\bm \Psi}}}
\newcommand\btheta{\ensuremath{{\bm \theta}}}
\newcommand\bTheta{\ensuremath{{\bm \Theta}}}
\newcommand\bomega{\ensuremath{{\bm \omega}}}
\newcommand\bxi{\ensuremath{{\bm \xi}}}
\newcommand\bzeta{\ensuremath{{\bm \zeta}}}

\newcommand\bYh{\ensuremath{{\bm Y}_{\rm H}}}
\newcommand\bYm{\ensuremath{{\bm Y}_{\rm M}}}
\newcommand\byh{\ensuremath{{\bm y}_{\rm H}}}
\newcommand\bym{\ensuremath{{\bm y}_{\rm M}}}
\newcommand\bAm{\ensuremath{{\bm A}_{\rm M}}}
\newcommand\bDh{\ensuremath{{\bm D}_{\rm H}}}
\newcommand\bDm{\ensuremath{{\bm D}_{\rm M}}}
\newcommand\bDx{\ensuremath{{\bm D}_{\rm X}}}
\newcommand\bVh{\ensuremath{{\bm V}_{\rm H}}}
\newcommand\bVm{\ensuremath{{\bm V}_{\rm M}}}
\newcommand\Lh{\ensuremath{L_{\rm H}}}
\newcommand\Mm{\ensuremath{M_{\rm M}}}


\newcommand\bY{\ensuremath{{\bm Y}}}
\newcommand\bV{\ensuremath{{\bm V}}}
\newcommand\bU{\ensuremath{{\bm U}}}
\newcommand\bs{\ensuremath{{\bm s}}}
\newcommand\bS{\ensuremath{{\bm S}}}
\newcommand\bE{\ensuremath{{\bm E}}}

\newcommand{\Rbb}{\mathbb{R}}
\newcommand{\Cbb}{\mathbb{C}}
\newcommand{\Hbb}{\mathbb{H}}
\newcommand{\Sbb}{\mathbb{S}}

\newcommand{\setA}{\mathcal{A}}
\newcommand{\setB}{\mathcal{B}}
\newcommand{\setD}{\mathcal{D}}
\newcommand{\setE}{\mathcal{E}}
\newcommand{\setH}{\mathcal{H}}
\newcommand{\setX}{\mathcal{X}}
\newcommand{\setR}{\mathcal{R}}
\newcommand{\setU}{\mathcal{U}}
\newcommand{\setV}{\mathcal{V}}
\newcommand{\setS}{\mathcal{S}}
\newcommand{\setC}{\mathcal{C}}
\newcommand{\setN}{\mathcal{N}}
\newcommand{\setL}{\mathcal{L}}

\newcommand{\N}{\mathcal{N}}
\newcommand{\CN}{\mathcal{CN}}
\newcommand{\Exp}{\mathbb{E}}
\newcommand{\jj}{\mathfrak{j}}
\newcommand{\dec}{\mathrm{dec}}
\newcommand{\Diag}{\mathrm{Diag}}
\newcommand{\diag}{\mathrm{diag}}

\newcommand\bQ{\ensuremath{{\bm Q}}}
\newcommand\bLam{\ensuremath{{\bm \Lambda}}}
\newcommand\br{\ensuremath{{\bm r}}}

\newcommand\Ree{\ensuremath{{\rm Re}}}
\newcommand\Imm{\ensuremath{{\rm Im}}}
\newcommand{\eps}{\varepsilon}
\newcommand{\bzero}{{\bm 0}}
\newcommand{\bone}{{\bm 1}}
\newcommand{\bI}{{\bm I}}
\newcommand*{\LargerCdot}{\raisebox{-0.25ex}{\scalebox{1.2}{$\cdot$}}}
\newcommand\vvec{\ensuremath{{\rm vec}}}
\newcommand\mmat{\ensuremath{{\rm mat}}}
\newcommand\bigO{\ensuremath{{\mathcal{O}}}}

\newcommand{\lammax}{\lambda_{\rm max}}
\newcommand{\lammin}{\lambda_{\rm min}}
\newcommand{\betamaxi}{\beta^{\sf max}_i}
\newcommand{\betamini}{\beta^{\sf min}_i}
%\newcommand{\hbetamini}{\hat{\beta}^{\sf min}_i}
\newcommand{\hbetamini}{{\delta_i}}
%\newcommand\hbetamin[1]{\hat{\beta}^{\sf min}_{#1}}
\newcommand\hbetamin[1]{{\delta_{#1}}}
\newcommand{\hbetainiti}{\hat{\beta}^{\sf init}_i}

%\newcommand{\hbetai}{{\green \hat{\beta}_i(\bx_{-i})}}
\newcommand{\hbetai}{{\hat{\beta}_i}}
\newcommand{\hRi}{{\hat{\bR}_i}}

\newcommand{\aalp}{{\alpha_i}}
\newcommand\aaalp[1]{{\alpha_{#1}}}
\newcommand{\setI}{\mathcal{I}}

\newcommand\indfn[1]{{{\mathbbm 1}_{#1}}}
%\newcommand\indfn{\ensuremath{{\mathbbm 1}}}
\newcommand\prox{\ensuremath{{\sf prox}}}
\newcommand\LO{\ensuremath{{\sf LO}}}

\newcommand\update{\ensuremath{{\sf UD}}}
\newcommand\bxex{\ensuremath{{\bm x}_{\rm ex}}}
\newcommand\bxkil{\bm x^{k,i,\ell}}
\newcommand\bxki[1]{\bm x^{k,i,#1}}
\newcommand\bxk[2]{\bm x^{k,#1,#2}}
\newcommand\dF[1]{\Delta_{#1}}
\newcommand\dx[1]{e_{#1}}
\newcommand\tdx[1]{\tilde{e}_{#1}}

\newcommand\A[1]{{\sf A#1)}}

\newcommand\sspan{\ensuremath{{\rm span}}}
\newcommand\aff{\ensuremath{{\rm aff}}}
\newcommand\conv{\ensuremath{{\rm conv}}}
\newcommand\bconv{\ensuremath{\overline{\rm conv}}}
\newcommand\var{\ensuremath{{\rm var}}}
\newcommand\cov{\ensuremath{{\rm Cov}}}
\newcommand\svol{\ensuremath{{\rm vol}}}
\newcommand\TVar{\ensuremath{{\rm tvar}}}
\newcommand\tr{\ensuremath{{\rm tr}}}




\hyphenation{op-tical net-works semi-conduc-tor}



\begin{document}


\bibliographystyle{IEEEtran}

%--- I do things quite strangely here to accommodate three style modes.
%--- input title and abstract here; it applies to all modes
%--- it's too complex to do authors or they are input for each mode
\newcommand{\papertitle}{
A Spatial Sigma-Delta Approach to Mitigation of Power Amplifier Distortions in Massive MIMO Downlink
%PA Distortion Effect Mitigation for Massive MIMO Downlink

}

\newcommand{\paperabstract}{
In massive multiple-input multiple-output (MIMO) downlink systems,
the physical implementation of the base stations (BSs) requires the use of cheap and power-efficient power amplifiers (PAs) to avoid high hardware cost and high power consumption.
However, such PAs usually have limited linear amplification ranges.
Nonlinear distortions arising from operation beyond the linear amplification ranges can significantly degrade system performance.
%{\blue Thus, dealing with the PA distortions poses a significant challenge.
%In this paper, we present a new spatial Sigma-Delta ($\Sigma \Delta$) approach to PA distortion effect mitigation.}
Existing approaches to handle the nonlinear distortions, such as digital predistortion (DPD), typically require accurate knowledge, or acquisition, of the PA transfer function.
In this paper, we present a new concept for mitigation of the PA distortions.
%that allows us to bypass such requirement.
Assuming a uniform linear array (ULA) at the BS,
the idea is to apply a Sigma-Delta ($\Sigma \Delta$) modulator to spatially shape
%we apply a $\Sigma \Delta$ modulator that shapes
the PA distortions to the high-angle region.
By having the system operating in the low-angle region,
%the overall effect is that
the received signals are less affected by the PA distortions.
To demonstrate the potential of this spatial $\Sigma \Delta$ approach, we study
the application of our approach to the multi-user MIMO-orthogonal frequency division modulation (OFDM)  downlink scenario.
A symbol-level precoding (SLP) scheme and a zero-forcing (ZF) precoding scheme, with the new design requirement by the spatial $\Sigma \Delta$ approach being taken into account, are developed.
Numerical simulations are performed to show the effectiveness of the developed $\Sigma \Delta$ precoding schemes.
}

%--------

\ifplainver

%    \date{May 30, 2014}

    \title{\papertitle}

    \author{Yatao Liu$^\dag$,  Mingjie Shao$^{\dag\S}$ and Wing-Kin Ma$^\dag$ \\~\\
    $^\dag$Department of Electronic Engineering, The Chinese University of Hong Kong, \\
    Hong Kong SAR, China\\
    $^\S$School of Information Science and Engineering,
Shandong University, Qingdao, China\\~\\
E-mails: ytliu@ee.cuhk.edu.hk, mingjieshao@sdu.edu.cn, wkma@cuhk.edu.hk
\thanks{This work was supported in part by a General Research Fund (GRF) of Hong
Kong Research Grant Council (RGC) under  Project ID CUHK 14208819.}
    }

    \maketitle

    \begin{abstract}
    \paperabstract
    \end{abstract}

\else
    \title{\papertitle}

    \ifconfver \else {\linespread{1.1} \rm \fi


    \author{XXX, XXX, AND XXX
    }
    %}


    \maketitle

    \ifconfver \else
        \begin{center} \vspace*{-2\baselineskip}
        %11th Revision, \today \\[2\baselineskip]
        \end{center}
    \fi

    \begin{abstract}
    \paperabstract
    \end{abstract}

%    \begin{keywords}\vspace{-0.0cm}
%        ...
%    \end{keywords}

    \begin{IEEEkeywords}\vspace{-0.0cm}
       Massive MIMO, MIMO-OFDM, nonlinear distortion, power amplifier, Sigma-Delta modulation, symbol-level precoding.
    \end{IEEEkeywords}

    \ifconfver \else \IEEEpeerreviewmaketitle} \fi

 \fi

\ifconfver \else
    \ifplainver \else
        \newpage
\fi \fi





\maketitle

\section{Introduction}


Over the past decade, massive multiple-input multiple-output (MIMO) systems, which employ large antenna arrays at the base station (BS), have attracted tremendous interest from both academia and industry, due to its potential of offering orders of magnitude improvements in spectral and power efficiencies compared to conventional MIMO systems.
Despite such potential, the practical implementation of massive MIMO requires the use of low-cost and power-efficient power amplifiers (PAs) at the BS;
we should bear in mind that the number of PAs, and the subsequent hardware cost and power consumption, scale with the antenna array size.
%However, such PAs generally exhibit highly nonlinear transfer characteristics
Such PAs generally exhibit nonlinear transfer characteristics, with limited linear amplification ranges; and operation beyond the linear amplification ranges can result in nonlinear distortions to the transmitted signals~\cite{cripps2006rf}.
Researchers have investigated the impact of the PA distortions
on the massive MIMO downlink,
and it has been shown that the PA distortions can lead to substantial performance degradation~\cite{mollen2016waveforms,larsson2018out,anttila2019antenna,moghadam2018energy,moazzen2019performance}.




%previous version-----------------------------------------------------------------------------------------------
%At the same time, other researchers have focused on addressing the PA distortion issue.
%%suppress the PA distortion effect.
%%developing approaches to handle the PA distortions.
%One direction is to
%impose certain power constraints on the PA input signals---such as
%%One direction is to appropriately constrain the power of the PA input signals---such as
%peak power constraints, power back-off, or
%%to make the PA operate in the linear region~\cite{schenk2008rf,jedda2017precoding,spano2017symbol};
%peak-to-average power ratio (PAPR) constraints,
%%to render the signal insensitive to the PA distortions
%to try to keep the PA operating in the linear region~\cite{schenk2008rf,jedda2017precoding,studer2013aware,bao2016efficient,yao2018semidefinite,qin2021low,spano2017symbol};
%and the stringent constant-envelope constraints,
%to allow us to ignore the PA distortions~\cite{mohammed2013per,pan2014constant,shao2019framework,domouchtsidis2020constant}.
%While these approaches are helpful in PA distortion mitigation and also simple in hardware implementation,
%they may result in either low PA output power or difficult signal designs.
%----------------------------------------------------------------------------------------------------------------

As a result, mitigating the PA distortion effects has become an important research problem.
Digital predistortion (DPD) is a promising concept to deal with PA nonlinearity~\cite{ghannouchi2009behavioral,guan2014green,katz2016evolution}. It applies an inverse response, called a predistorter, at the input of the PA to equalize the nonlinear PA transfer characteristics.
In order to do that, DPD requires acquisition of the nonlinear PA transfer function.
There is a certain hardware implementation cost for DPD with each PA~\cite{wood2017system}.
While DPD has been used in small- or medium-scale MIMO, it can be expensive to use in massive MIMO.
%the PA nonlinear distortion effects.
%{\blue To find the inverse response, DPD typically requires additional hardware for each PA, which can be expensive in massive MIMO systems~\cite{wood2017system}.}
%-------------
%DPD requires accurate knowledge, or acquisition, of the PA characteristics, and this can be expensive to do in massive MIMO systems~\cite{katz2016evolution}.
%-------------
As a compromise, one can impose certain power constraints on the PA input signals, such as peak power constraints or
power back-off~\cite{schenk2008rf,jedda2017precoding,spano2017symbol},
peak-to-average power ratio (PAPR) constraints~\cite{studer2013aware,bao2016efficient,yao2018semidefinite,qin2021low,spano2017symbol},
and constant-envelope constraints~\cite{mohammed2013per,pan2014constant,shao2019framework,domouchtsidis2020constant}, to keep the PAs operating in a limited linear amplification region.
This approach does not have additional hardware requirement on top of the PAs (unlike DPD),
but it may result in either lower PA power efficiency (e.g., power back-off) or more difficult signal designs (e.g., those with PAPR constraints).
%Another direction is to devise MIMO precoding schemes that are robust to the PA distortion effects.
%These schemes usually take the PA distortions into account by utilizing convenient PA model approximations.
Another approach is to
take the PA distortions into account when designing the transmitted signals.
%There are mainly two approximation methods.
One can model the PA distortions, together with other hardware impairments (such as phase noise and quantization errors), as additive Gaussian noise, where the noise power scales with the PA input power; see, e.g.,~\cite{bjornson2012optimal,brandt2014weighted,zarei2017multi}.
However,  such Gaussian noise model
can be inaccurate \cite{larsson2018out,anttila2019antenna}.
A more advanced method is to apply the Bussgang theorem to approximate the nonlinear PA model as a linear one---modeling the PA nonlinear effects via signal scalings and uncorrelated additive distortion noise;
%with uncorrelated distortion noise;
see, e.g.,~\cite{aghdam2020distortion,jee2020precoding,jee2021joint}.
%----------------
%{\blue To apply the Bussgang theorem, it requires the knowledge of the PA transfer function at the BS~\cite{zayani2019efficient,aghdam2020distortion,jee2020precoding,jee2021joint}.}
%-----------------


In this work,
we present a different concept for combating the PA distortion effects.
%---------
%which does not require exact knowledge of the PA transfer function at the BS.
%---------
The idea is to apply Sigma-Delta ($\Sigma \Delta$) modulation~\cite{aziz1996overview}.
%The core idea of our approach originally arose from the rationale of Sigma-Delta ($\Sigma \Delta$) modulation~\cite{aziz1996overview},
$\Sigma \Delta$ modulation appears most frequently in analog-to-digital or digital-to-analog converters of temporal signals.
It aims at reducing the quantization noise effects on oversampled, low-frequency, temporal signals.
The principle is to apply a specific feedback loop to shape the quantization noise to high frequency.
%which aims to reduce the quantization noise for oversampled, or low-frequency temporal signals.
%The principle there is to
%apply a $\Sigma \Delta$ modulator with a feedback loop to shape the quantization noise to high frequency.
Consequently, the shaped quantization noise is pushed away from the low-frequency signal in the frequency-domain, and we can remove much of the quantization noise via low-pass filtering.
The reader is referred to the overview paper~\cite{aziz1996overview} and the references therein for details.
More recently, the $\Sigma \Delta$ principle has also been explored in the spatial domain.
Specifically, by utilizing spatial oversampling (i.e., using sub-half-wavelength inter-antenna spacings in a uniform linear antenna array) and a spatial $\Sigma \Delta$ modulator (with feedback loops between adjacent antennas), the quantization noise can be shaped to high spatial frequency (i.e., angle) and the signals in the low-angle region will be less affected by the quantization noise.
This spatial $\Sigma \Delta$ idea has been used for quantization noise reduction in a number of  wireless communication and radar applications, such as
signal detection or channel estimation in the uplink~\cite{corey2016spatial,barac2016spatial,nikoofard2017low,madanayake2017improving,rao2019massive,pirzadeh2020spectral,rao2021massive}, and  beamforming or precoding in the downlink~\cite{scholnik2004spatio,krieger2013dense,shao2019one,shao2020multiuser}.
%In particular, Shao {\it et al.}~\cite{shao2019one} considered the one-bit MIMO precoding problem and developed a spatial $\Sigma \Delta$ modulator to shape the quantization noise in space such that it will have  minimal impact on the received signals at the user side.
These works show the effectiveness of spatial $\Sigma\Delta$ modulation in quantization noise mitigation.



As the key contribution of this work, we advocate to use spatial $\Sigma \Delta$ modulation to handle the PA distortion problem in the downlink transmission.
To the best of our knowledge, this is the first work that explores the use of spatial $\Sigma \Delta$ modulation to combat the PA distortions.
Assuming a uniform linear array (ULA) at the BS, we show that the PA distortions can be effectively mitigated by applying a spatial $\Sigma \Delta$ modulator that shapes the PA distortions to the high-angle region.
%and by employing directional transmit antennas that transmit signals in a low-pass angular sector.
%We further show that the PA distortion mitigation effect can be enhanced by using linear PAs at the last antenna, and such enhancement can provide obvious performance improvement in some cases, as will be demonstrated by simulation results.
It should be emphasized that, different from the $\Sigma \Delta$ modulator in~\cite{shao2019one}, which is implemented in the digital domain, the $\Sigma \Delta$ modulator concept presented in this study is implemented in the analog domain.
%our presented $\Sigma \Delta$ modulator is totally implemented in the analog domain for PA distortion shaping, rather than in the digital domain for quantization noise shaping.
%{\blue It is also worth mentioning that our presented spatial $\Sigma \Delta$ approach requires a gross estimation of the worst PA distortion within certain PA input amplitude range.
%This is different from existing approaches, such as DPD, which require accurate knowledge, or acquisition, of the PA transfer function.
%%By comparison, existing approaches, such as DPD, require accurate knowledge, or acquisition, of the PA transfer function, and this may be expensive to do in massive MIMO systems~\cite{wood2017system,DPD_rep}.
%}
It is worth noting that our spatial $\Sigma \Delta$ approach does not require precise knowledge, or acquisition, of the PA transfer function.
Our approach is based on the assumption that we know the worst-case magnitude of the PA distortion relative to the ideal linear amplification response.
That worst-case magnitude can be a guessed one in practice.


%Then, under the use of the new spatial $\Sigma\Delta$ modulator for PA distortion reduction, we design precoding schemes in the multi-user massive MIMO-OFDM system.

To demonstrate the spatial $\Sigma \Delta$ modulation concept for PA distortion mitigation, we design suitable precoding schemes under this new concept.
Our scenario of interest is that of MIMO-OFDM, which is considered a more realistic scenario than the standard frequency-flat MIMO scenario.
%we apply the spatial $\Sigma \Delta$ approach to multi-user MIMO-OFDM precoding with nonlinear PAs.
As we will show, the spatial $\Sigma \Delta$ modulation concept leads to amplitude constraints with the transmitted signals.
%instantaneous peak power constraints.
%To demonstrate the merits of the spatial $\Sigma \Delta$ approach,
We develop a zero-forcing (ZF) scheme and a symbol-level precoding (SLP) scheme for the precoding design, both assuming quadrature amplitude modulation (QAM) constellations.
%\footnote{The developed precoding schemes can be easily adjusted to handle the phase shift keying (PSK) constellation case based on the derivations in~\cite{shao2020multiuser1}.}
SLP is an emerging precoding paradigm that optimizes symbol-level performance metrics. We formulate our SLP design as a maximum detection probability (DP) problem subject to signal amplitude constraints, which is a large-scale convex problem. To efficiently solve this problem, we custom build an alternating direction method of multipliers (ADMM) algorithm.


%SLP is an advanced multi-user MIMO precoding paradigm that is famous for its superior performance and precise control on signal power/amplitude at the symbol level.
%We should mention some related SLP studies.
Relevant existing SLP studies should be mentioned.
There have been numerous SLP schemes developed for multi-user MIMO, but without OFDM~\cite{spano2017symbol, mohammed2013per,pan2014constant,shao2019framework, masouros2015exploiting,alodeh2015constructive,alodeh2017symbol,liu2021symbol,jacobsson2017quantized,sohrabi2018one}.
There are also some studies for multi-user MIMO-OFDM~\cite{studer2013aware,bao2016efficient,yao2018semidefinite,qin2021low,askerbeyli20191,domouchtsidis2020constant,jacobsson2018nonlinear,tsinos2020symbol}.
The design criteria of the existing SLP schemes include
total power reduction~\cite{masouros2015exploiting,alodeh2015constructive,alodeh2017symbol,liu2021symbol},
per-antenna power minimization~\cite{spano2017symbol,liu2021symbol},
PAPR reduction~\cite{studer2013aware,bao2016efficient,yao2018semidefinite,qin2021low},
one-bit constraints~\cite{jacobsson2017quantized,sohrabi2018one,askerbeyli20191,shao2019framework},
constant-envelope constraints~\cite{mohammed2013per,pan2014constant,shao2019framework,domouchtsidis2020constant},
and phase quantized constant-envelope constraints~\cite{jacobsson2018nonlinear,tsinos2020symbol,shao2019framework}.
%under design criteria such as PAPR reduction~\cite{studer2013aware,bao2016efficient,yao2018semidefinite,qin2021low}, one-bit constraints~\cite{askerbeyli20191}, constant-envelope constraints~\cite{domouchtsidis2020constant} and phase-quantized constant-envelope constraints~\cite{jacobsson2018nonlinear,tsinos2020symbol}.
However, we are unaware of any existing SLP scheme that best fits the requirement of limited signal amplitudes introduced by our spatial $\Sigma \Delta$ approach and under QAM constellations.
This calls for the need to custom design an SLP scheme for our spatial $\Sigma \Delta$ approach.
Our numerical results will show that the proposed scheme, combined with the spatial $\Sigma \Delta$ approach, gives promising performance.
%This necessitates
%This motivates us to develop the new DP maximization SLP design with  signal amplitude constraints, which, combined with the spatial $\Sigma \Delta$ approach, shows good performance in PA distortion mitigation.
%in order to demonstrate the merits of our spatial $\Sigma \Delta$ approach.}
%which plays a prominent role in demonstrating the merits of our spatial $\Sigma \Delta$ approach.}






Our notations are as follows.
We use lowercase letters (e.g., $x$), boldfaced lowercase letters (e.g., $\bx$), and boldfaced capital letters (e.g., $\bX$) to denote scalars, column vectors, and matrices, respectively;
%Boldfaced lowercase and uppercase letters (e.g., $\bx$ and $\bX$) represent column vectors and matrices, respectively.
$\bX^\Tsf$ and $\bX^\Hsf$ represent the transpose and Hermitian transpose of $\bX$, respectively;
%unless specified, $\bx_i$ and $\bar \bx_j^T$ denote the $i$th column and the $j$th row of $\bX$, respectively;
$\jj = \sqrt{-1}$ is the imaginary unit;
$\Re(\bx)$ and $\Im(\bx)$ denote the real and imaginary components of $\bx$, respectively;
$\mathbb{R}$ and $\mathbb{C}$ are the sets of all real and complex numbers, respectively;
$\mathcal{U}{[a,b]}$ denotes the uniform distribution on $[a,b]$;
$\Pi_{\setX}(\bx) \in \arg \min_{\by \in \setX} \|\bx-\by\|_2^2$ is a projection of $\bx$ onto the set $\setX$.



\section{Background}




\subsection{PA Model}
\label{sec:PA_mod}

%\begin{figure}%[htb!]
%	\centering
%	\begin{subfigure}[b]{0.9\linewidth} \includegraphics[width=1\linewidth]{./Figs/PA_response_am.eps}
%		\caption{AM-AM conversion.}\label{fig:PA_response_am}
%	\end{subfigure}\\
%	\begin{subfigure}[b]{0.9\linewidth} \includegraphics[width=1\linewidth]{./Figs/PA_response_pm.eps}
%		\caption{AM-PM conversion.}
%	\end{subfigure}
%	\caption{
%		Illustration of the ideal and modified Rapp PA models.
%	}\label{fig:PA_response}
%\end{figure}

\begin{figure}[t]
\centering
\subfigure[AM-AM conversion.]{
\label{fig:PA_response_am}
\includegraphics[width=0.5\linewidth]{PA_response_am}}
\subfigure[AM-PM conversion.]{
\includegraphics[width=0.5\linewidth]{PA_response_pm}}
\caption{Illustration of the ideal and modified Rapp PA models.}
\label{fig:PA_response}
\end{figure}

Let us start with introducing the PA model.
Consider the widely used memoryless PA model~\cite{schreurs2008rf,ochiai2013analysis}
\begin{equation}\label{eq:PA_general}
	x_{\rm out}(t) = G(x_{\rm in}(t)),
\end{equation}
where
$x_{\rm out}(t)$ and $x_{\rm in}(t)$  are the  complex baseband  PA output and input  signals, respectively;
$G(\cdot)$ is the PA response function, which is assumed to take the form
\[
G(x) = g_a(|x|) e^{\jj \cdot {\rm arg}(x)} e^{\jj \cdot g_p(|x|)},
\]
where $g_a(\cdot)$ and $g_p(\cdot)$ are the amplitude and phase responses, also referred to as
the amplitude-to-amplitude (AM-AM) and amplitude-to-phase (AM-PM) conversions, respectively.


In the ideal case, the PA output should be
a linear amplification of the PA input within the allowable PA output amplitude range.
To be more specific, we desire a linear PA response up to the maximum PA output amplitude, i.e.,
\begin{equation}\label{eq:ideal_pa}
	g_a(r) =
	\begin{cases}
		Ar, & 0 \le r \le r_{\rm max},\\
		Ar_{\rm max}, & r > r_{\rm max},
	\end{cases}
	\quad
	g_p(r) = 0,
\end{equation}
where $A$ is the PA gain, and $Ar_{\rm max}$ is the maximum PA output amplitude, with $r_{\rm max}$ being a reference PA input amplitude;
see Fig.~\ref{fig:PA_response} for an illustration.
However, due to the physics of semiconductors, realistic PAs generally exhibit nonlinear responses.
%are generally not ideal and exhibit nonlinear responses.
In the literature, there are several mathematical models to describe $g_a(\cdot)$ and $g_p(\cdot)$ for realistic PAs~\cite{schreurs2008rf,ochiai2013analysis,3gpp_pa,saleh1981frequency,rapp1991effects}.
A popular model is the modified Rapp model, which was proposed in 3GPP
%3rd Generation Partnership Project (3GPP)
%that is used
for performance evaluation in 5G systems~\cite{3gpp_pa}.
Specifically, the modified Rapp model is expressed as
\begin{equation}\label{eq:rapp}
	g_a(r) \!= \!\frac{Ar}{\left( 1+\left( r/r_{\rm max} \right)^{2\varphi} \right)^{\frac{1}{2\varphi}}}, ~
	g_p(r) \!= \! \frac{B  r^\zeta}{1+\left({r}/{C}\right)^\zeta}~ ({\rm rad}),
\end{equation}
where $\varphi$, $\zeta$, $B$ and $C$ are the fitting parameters;
%the PA gain is $A=16$ and the maximum output amplitude is $Ar_{\rm max} = 1.9$;
see Fig.~\ref{fig:PA_response}.
We see that under such a nonlinear PA model, the PA output is distorted by the PA nonlinearity.
%Further, such nonlinear distortions may result in substantial performance degradation.
Apart from the modified Rapp model,  other commonly used models include the traveling-wave tube amplifier (TWTA) model~\cite{saleh1981frequency} and the solid state power amplifier (SSPA) model~\cite{rapp1991effects}.






\subsection{Prior Art}
\label{sec:prior_art}


In this subsection, we review two conventional approaches to combating the PA distortions.
The first one is {\it power back-off}~\cite{schenk2008rf,jedda2017precoding,spano2017symbol}.
%~\cite{katz2001linearization,huang2014optimal,zavjalov2016influence}.
This method is based on a basic observation of realistic PA responses---the PA response is near-linear for low PA input amplitude;
%the lower the PA input amplitude is, the more linear the PA response will be;
the modified Rapp model shown in Fig.~\ref{fig:PA_response} is an example.
The idea is to reduce the PA input amplitude such that the PA response is near-linear.
%to pursue a linear PA response.
A common way of performing power back-off is to
%constrain the PA input amplitude $r$ by
shrink the PA input amplitude $r$ such that it satisfies
\begin{equation}\label{eq:1db}
	r \le r_{\rm 1dB},
\end{equation}
where $r_{\rm 1dB}$ is the {\it 1dB compression point}, usually recognized as the border between the linear and nonlinear regions of the PA response~\cite{ochiai2013analysis}.
%~\cite{katz2001linearization,ochiai2013analysis,huang2014optimal}.
More precisely, $r_{\rm 1dB}$ is defined as the PA input amplitude, where the difference of the output amplitudes between the linear and the PA response is 1dB, i.e.,
\[
20\log_{10}\frac{Ar_{\rm 1dB}}{g_a(r_{\rm 1dB})} = 1;
\]
see the example in Fig.~\ref{fig:PA_response_am}.
The power back-off method is effective in PA distortion reduction and also simple in implementation,
but such a direct power back-off strategy results in reduced power efficiency~\cite{ochiai2013analysis}.
%but the downside lies in low power efficiency.
%the resulting low PA output power and low power efficiency.

\begin{figure}[t]
	\centering
	\includegraphics[width=0.6\linewidth]{PD}
	\caption{The principle of DPD.}
	\label{fig:PD}
\end{figure}





The second approach is {\it digital predistortion} (DPD)~\cite{ghannouchi2009behavioral,guan2014green,katz2016evolution},
which is currently the most widely applied PA linearization method in wireless BSs.
As illustrated in Fig.~\ref{fig:PD},
DPD works by adding
a nonlinear component, called a predistorter, in front of the nonlinear PA, such that the combined response
of the predistorter and the PA
is an ideal PA response in \eqref{eq:ideal_pa}.
Fig.~\ref{fig:DPD} shows how DPD is realized in practice.
For ease of implementation, the predistorter is implemented in digital domain, and the digital predistorter needs to be continuously adjusted to accommodate the PA response change with time.
%Such adjustment requires information of the PA response in real time, which is usually acquired by assuming a PA model and estimating the model parameters based on the digital PA input and the digitalized PA output feedback.
%{\blue This requires oversampling and considerable computational efforts for estimating the model parameters of a postulated PA transfer function~\cite{wood2017system}.}


Ideally, DPD can achieve excellent PA linearization performance,
but the challenges lie in hardware implementations.
To be specific, DPD requires
%i) a high-resolution ADC for each PA and the ADC sampling rate should be typically three to five times the signal bandwidth to capture the PA distortion behavior;
i) an expensive high-resolution and fast analog-to-digital converter (ADC) at the feedback loop to capture the PA distortion behavior;
ii) an accurate parametric model for the PA transfer function;
iii) a fast training algorithm to update the PA model parameters in real time.
And we have such requirements for each of the PAs.
Therefore, the hardware complexity of DPD can be high in practice, especially in massive MIMO systems with numerous PAs.

\begin{figure}%[t]
	\centering
	\includegraphics[width=0.6\linewidth]{DPD}
	\caption{Block diagram of a general DPD system.}
	\label{fig:DPD}
\end{figure}



\section{Spatial $\Sigma \Delta$ Modulation for PA Distortion Mitigation}
\label{sec:sd_all}

In this section, we present our spatial $\Sigma \Delta$ modulation concept for handling the PA distortion problem.
The idea comes from one-bit $\Sigma \Delta$ noise shaping, which was developed to reduce the quantization noise effects in one-bit MIMO precoding~\cite{shao2019one}.
To explain the idea, we first review the principle of one-bit $\Sigma \Delta$ noise shaping and then describe how we adopt the principle to combat the PA distortion effects.

\subsection{Review of One-Bit ${\Sigma \Delta}$ Noise Shaping}
\label{sec:review_SD}


We review the spatial $\Sigma\Delta$ modulation in \cite{shao2019one}, which was proposed to address the quantization effect caused by one-bit digital-to-analog converters (DACs).
Consider the massive MIMO downlink scenario, where a BS with $N$ antennas transmits signals to multiple users and each transmit antenna is equipped with one-bit DACs for low-cost implementation.
%One-bit DACs can yield quantization noise on the transmitted signals and degrade the communication quality.
%$\Sigma \Delta$ noise shaping aims to mitigate the one-bit quantization effects at the user side by shaping the quantization noise in space.
%The task of one-bit $\Sigma \Delta$ noise shaping is to mitigate the one-bit quantization noise,
%and the general idea is to shape the quantization noise in space such that it will have a minimal effect on the received signals at the user side.
%For convenience of explanation, we assume
%To better explain the idea,
To let the reader see the idea easier, we describe the principle of spatial $\Sigma\Delta$ modulation through real-valued continuous-time transmitted signals, rather than through complex-valued discrete-time transmitted signals as in the original paper~\cite{shao2019one}.
Fig.~\ref{fig:sd_onebit} shows the first-order $\Sigma \Delta$ modulator structure,
where
$x_n(t) \in \Rbb$ is the modulator input,
$b_n(t) \in \Rbb$ is the quantizer input,
${\rm sgn}(\cdot)$ is the signum (or one-bit quantization) function,
$u_n(t) = {\rm sgn}(b_n(t))$ is the quantizer output,
and $q_n(t) \in \Rbb$ is the quantization noise.
We see in Fig.~\ref{fig:sd_onebit} that the input-output relation is
%It is verified that the signal relations in Fig.~\ref{fig:sd_onebit} can be described as
\begin{equation}\label{eq:sd_unt}
u_n(t) = x_n(t) +  q_n(t) -  q_{n-1}(t),
\end{equation}
for $n=1,\dots,N$, where $q_0(t) = 0$.
%As shown in the figure, the quantizer input is designed by feeding back the quantization noise of the previous antenna; specifically,
%\begin{equation}\label{eq:sd_bnt_onebit}
%	b_n(t) = x_n(t) - q_{n-1}(t), ~ q_{n}(t) = u_{n}(t) - b_{n}(t),
%\end{equation}
%for $n=1,\dots,N$, where $q_0(t) = 0$.
%%The signal relation in Fig.~\ref{fig:sd_onebit} can be expressed as
%Rearranging \eqref{eq:sd_bnt_onebit} gives
%\begin{equation}\label{eq:sd_unt}
%	u_n(t) = x_n(t) +  q_n(t) -  q_{n-1}(t).
%\end{equation}
%To show the working principle of the $\Sigma \Delta$ modulator,
As a key assumption in one-bit $\Sigma \Delta$ noise shaping, the transmit antennas are assumed to be arranged as a uniform linear array (ULA).
Let
\begin{equation}\label{eq:angular_response}
	\ba(\theta) = \left[ 1,e^{-\jj\frac{2\pi d}{\lambda}\sin(\theta)}, \dots, e^{-\jj(N-1)\frac{2\pi d}{\lambda}\sin(\theta)} \right]^\Tsf
\end{equation}
be the angular response of the ULA,
where
$\theta \in [-\pi/2,\pi/2]$ is the angle,
$\lambda$ is the carrier wavelength
and $d \le {\lambda}/{2}$ is the inter-antenna spacing.
Assuming unit channel gain, we can express the signal received at angle $\theta$ as
%It is seen from~\eqref{eq:sd_unt}--\eqref{eq:angular_response} that
%the signal received at angle $\theta$ is expressed as\footnote{Here, we assume far-field received signals and unit channel gain.}
\begin{align}
	\ba(\theta)^\Tsf \bu(t) &= \ba(\theta)^\Tsf \bx(t) + \xi_\omega(t), \notag\\
	\xi_\omega(t) &= (1 - e^{-\jj \omega})  Q_\omega(t)  +  q_N(t) e^{- \jj \omega(N-1)}, \label{eq:xi_onebit}
\end{align}
where
$\bu(t) \!\!=\!\! [u_1(t),\dots,u_N(t)]^\Tsf$;
$\bx(t) \!\!=\!\! [x_1(t),\dots,x_N(t)]^\Tsf$;
$\omega \!=\! \frac{2\pi d}{\lambda} \sin(\theta)$ denotes the spatial frequency associated with $\theta$;
$\xi_\omega(t)$ is the received quantization noise with
$Q_\omega(t) \!=\! \sum_{n=1}^{N-1} \! q_n(t) e^{-\jj \omega(n-1)}$.
%The expression in~\eqref{eq:xi_onebit} reveals the working principle of one-bit $\Sigma \Delta$ noise shaping.
We observe from~\eqref{eq:xi_onebit} that the quantization noise term $Q_\omega(t)$ is shaped by a high-pass response $(1 - e^{-\jj \omega})$,
and hence the shaped noise $(1 - e^{-\jj \omega})  Q_\omega(t)$ is expected to be high-frequency noise.
%If we further assume that $|Q_\omega(t)|$ is flat w.r.t. $\omega$,\footnote{This is a common assumption in the $\Sigma \Delta$ literature; see, e.g.,~\cite{shao2019one}.} then the shaped noise $(1 - e^{-\jj \omega})  Q_\omega(t)$ will be high-pass noise,
The noise power should decrease as $|\omega|$ decreases, and $|\omega|$ decreases with $|\theta|$ and $d$.
%, or equivalently $|\theta|$ and/or $d$, decreases.
Therefore, to mitigate the quantization noise effects, it is desired to
\begin{enumerate}[i)]
\item
keep the operating range of user angles $\theta$ small, e.g., $\theta \in [-30^\circ, 30^\circ]$;
%make $|\theta|$ small, which is physically interpreted as applying directional transmit antennas that only transmit signals in a low-pass angular sector, e.g., $[-30^\circ, 30^\circ]$;
\item
use a small inter-antenna spacing $d$, e.g., $d = \lambda/8$.
%which cannot be made arbitrarily small due to practical limitations such as physical antenna size and mutual coupling effects.
\end{enumerate}
It is worth noting that the item i) corresponds to a sectored antenna array scenario;
and that, for ii), we cannot make $d$ arbitrarily small due to physical antenna size and mutual coupling effects.

\begin{figure}%[t]
	\centering
	\includegraphics[width=0.5\linewidth]{sd_onebit}
	\caption{The first-order ${\Sigma \Delta}$ modulator for quantization noise shaping.}
	\label{fig:sd_onebit}
\end{figure}

%The $\Sigma \Delta$ modulator in Fig.~\ref{fig:sd_onebit} may have stability problems.
There may be stability problems with the $\Sigma \Delta$ modulator in Fig.~\ref{fig:sd_onebit}.
Due to the feedback loop, the quantization noise amplitude $|q_n(t)|$ may be accumulated with $n$ and grow large.
%This may incur stability problems with the $\Sigma \Delta$ modulator.
In some extreme cases, we may have $|q_{n}(t)| \to \infty$ as $n \to \infty$.
This phenomenon is referred to as {\it overloading} in the $\Sigma \Delta$ literature.
%It has been shown in~\cite{shao2019one} that
Overloading can be avoided by appropriately constraining the modulator input amplitudes.
%; we summarize the result as follows.

\begin{Fact} \label{fact:no_overloading_onebit}
	{\bf (see, e.g.,~\cite{gray1990quantization,shao2019one})}
	Consider the first-order $\Sigma \Delta$ modulator in Fig.~\ref{fig:sd_onebit}.
	If $|x_n(t)| \le 1$ for all $n$, then it holds that $|q_n(t)| \le 1$ for all $n$.
\end{Fact}

\noindent
Under the no-overloading condition, i.e., $|x_n(t)| \le 1$ for all $n$,
it is common to make the following assumption.
\begin{Asm}\label{asm:q_onebit}
	Consider the first-order $\Sigma \Delta$ modulator in Fig.~\ref{fig:sd_onebit}.
	Under the condition that $|x_n(t)| \le 1$ for all $n$, the quantization noise $q_{n}(t)$ is uniformly distributed on $[-1,1]$ and is independently and identically distributed (i.i.d.) over $n$.
	Also, each $q_{n}(t)$ is independent of any other random variables.
\end{Asm}

\noindent
Under Assumption~\ref{asm:q_onebit}, the power of the received quantization noise $\xi_\omega(t)$ in~\eqref{eq:xi_onebit} is calculated by
\[
\mathbb{E}[|\xi_\omega(t)|^2] =
\frac{4(N-1)}{3}   \sin^2 \left( \frac{\pi d}{\lambda} \sin(\theta)\right) + \frac{1}{3},
\]
which is seen to reduce as $|\theta|$ and/or $d$ decreases.
%The reader is referred to~\cite{shao2019one} for more details of one-bit $\Sigma \Delta$ noise shaping.


\subsection{${\Sigma \Delta}$ PA Distortion Shaping}
\label{sec:sd}

%\begin{figure}%[H]
%	\centering
%	\begin{minipage}[t]{0.48\textwidth}
%		\centering
%		\includegraphics[width=0.8\linewidth]{./Figs/sd.eps}
%		\caption{\small $\Sigma \Delta$ modulator for PA distortion shaping.}
%		\label{fig:sd}
%	\end{minipage}
%	\begin{minipage}[t]{0.48\textwidth}
%		\centering
%		\includegraphics[width=0.8\linewidth]{./Figs/sd_delay.eps}
%		\caption{\small $\Sigma \Delta$ modulator with time-delay components.}
%		\label{fig:sd_delay}
%	\end{minipage}
%\end{figure}
\begin{figure}[t]
	\centering
	\includegraphics[width=0.5\linewidth]{sd}
	\caption{The first-order ${\Sigma \Delta}$ modulator for PA distortion shaping.}
	\label{fig:sd}
\end{figure}

We now adopt the principle of one-bit $\Sigma \Delta$ noise shaping
to handle the PA nonlinear distortions.
The idea is to spatially shape the PA distortions in the same way as how we shape the quantization noise in the previous subsection.
%, instead of the quantization noise, so that the PA distortions
% effect can be mitigated at the user side.
%will have a minimal effect on the received signals at the user side.
%Similar as the structure in Fig.~\ref{fig:sd_onebit},
The first-order $\Sigma \Delta$ modulator structure for PA distortion shaping is shown in Fig.~\ref{fig:sd},
where
$x_n(t) \in \Cbb$ is the modulator input;
$b_n(t) \in \Cbb$ is the PA input;
$u_n(t) = G(b_n(t))$ is the PA output, with $G(\cdot)$ being the PA response function;
% described in Section~\ref{sec:PA_mod};
$A$ is the PA gain;
$q_n(t) \in \Cbb$ is the PA distortion.
Assuming a ULA at the BS
and following similar derivations as in \eqref{eq:sd_unt}--\eqref{eq:xi_onebit},
the signal received at angle $\theta$ is given by
%the signal model under the PA nonlinear distortions and $\Sigma\Delta$ modulation is given by
\begin{align}
	\ba(\theta)^\Tsf \bu(t) &= A\ba(\theta)^\Tsf \bx(t) + \xi_\omega(t), \label{eq:au} \\ %\notag\\
	\xi_\omega(t) &= A(1 - e^{-\jj \omega})  Q_\omega(t)  +  Aq_N(t) e^{- \jj \omega(N-1)}, \label{eq:xi}
\end{align}
where $\xi_\omega(t)$ now represents the $\Sigma \Delta$-shaped PA distortion;
$\omega$ and $Q_\omega(t)$ are defined in the same way as in~\eqref{eq:xi_onebit}.
The PA distortion term $A(1 - e^{-\jj \omega})  Q_\omega(t)$ can be seen as high spatial frequency noise.
By operating in a low-angle region for the users and by using a small inter-antenna spacing $d$, we can mitigate the effects of the PA distortions on the received signals---same as what happens in one-bit $\Sigma \Delta$ noise shaping.
%Similarly, the PA distortion term $Q_\omega(t)$ is shaped to high spatial frequency and
%is expected to be suppressed by using directional antennas that serve a low-pass angular sector
%and by setting a small inter-antenna spacing.
%In this case, the no-overloading condition can also be characterized as modulator input amplitude constraints.
In PA distortion shaping, the no-overloading condition is characterized as follows.
\begin{Fact} \label{Fac:no_overloading}
	Consider the first-order $\Sigma \Delta$ modulator in Fig.~\ref{fig:sd}.
	Given any $\chi > 0$, define
	\begin{equation}\label{eq:C}
	\psi \triangleq \max_{|z| \le \chi}|G(z)/A - z|.
	\end{equation}
	If the modulator input amplitudes are constrained by
	\begin{equation}\label{eq:cont_no_overloading}
		|{x}_{n}(t)| \le   \chi - \psi,
	\end{equation}
    for all $n$, then it holds that $|b_n(t)| \le \chi$ and $|q_{n}(t)| \le \psi$ for all $n$.
\end{Fact}



\noindent
The proof of Fact~\ref{Fac:no_overloading} is relegated to the Appendix.
%The physical meaning of $\psi$ is
The variable $\psi$ in~\eqref{eq:C} describes the largest PA distortion under the PA input amplitude range $|z| \le \chi$.
It remains to specify the choice of $\chi$.
Intuitively speaking, when $\chi$ increases, the signal power tends to increase as $|b_n(t)| \le \chi$;
on the other hand, the PA distortion power also tends to increase as $|q_{n}(t)| \le \psi$ and $\psi$ scales with $\chi$.
Therefore, there is a tradeoff.
%Intuitively speaking, a smaller value of $B$ may result in lower PA distortions (since $C$ decreases as $B$ decreases and $|q_{n}(t)| \le C$) but also lower PA output power (since $|b_n(t)| \le B$ and $|u_n(t)|$ generally decreases as $|b_n(t)|$ decreases), so there is a tradeoff.
In this work, we heuristically choose $\chi = r_{\rm max}$ and it will be shown in our numerical study that such choice empirically works well.
%Under the no-overloading condition in~\eqref{eq:cont_no_overloading},
In the same spirit as Assumption~\ref{asm:q_onebit} in the one-bit DAC case,
we make the following assumption.
\begin{Asm}\label{asm:q}
	Consider the first-order $\Sigma \Delta$ modulator in Fig.~\ref{fig:sd}.
	Under the condition in~\eqref{eq:cont_no_overloading}, the PA distortion $q_{n}(t)$ is i.i.d. over $n$ with $|q_{n}(t)| \sim \mathcal{U}{[0,\psi]}$ and $\arg(q_{n}(t)) \sim \mathcal{U}{[-\pi,\pi]}$.
	Also, each $q_{n}(t)$ is independent of any other random variables.
\end{Asm}

\noindent
Under Assumption~\ref{asm:q}, the power of the $\Sigma \Delta$-shaped PA distortion $\xi_\omega(t)$ in~\eqref{eq:xi} can be shown to be
\[
\mathbb{E}[|\xi_\omega(t)|^2] =
\frac{4(N-1)A^2\psi^2}{3}   \sin^2 \left( \frac{\pi d}{\lambda} \sin(\theta)\right) + \frac{A^2\psi^2}{3}.
\]
We see that the shaped distortion power reduces as $|\theta|$ and/or $d$ decreases.



%\subsection{Spatial $\Sigma \Delta$ Approach with Tail-Removing}
\subsection{A Tail-Removing ${\Sigma \Delta}$ Scheme}
\label{sec:sd_tr}


%The spatial $\Sigma \Delta$ approach has a drawback.
It is observed from \eqref{eq:xi} that the PA distortion $q_N(t)$ at the last antenna cannot be shaped by the first-order $\Sigma \Delta$ modulator.
By our empirical experience, $q_N(t)$ could have a non-negligible effect on the received signals.
To overcome this drawback, we suggest to employ a linear PA at the last antenna to make $q_N(t)=0$ in \eqref{eq:xi}.
The hardware complexity of employing a high-quality linear PA  at only one antenna should be affordable in practice.
In this case, the received signal at angle $\theta$ becomes
\begin{equation}\label{eq:receive_sig_sd}
\begin{aligned}
%\ba(\theta)^\Tsf \bu(t) &= A\ba(\theta)^\Tsf \bx(t) + A(1 - e^{-\jj \omega})  Q_\omega(t),
\ba(\theta)^\Tsf \bu(t) &= A\ba(\theta)^\Tsf \bx(t) + \xi_\omega(t), \\
\xi_\omega(t) &= A(1 - e^{-\jj \omega})  Q_\omega(t),
\end{aligned}
\end{equation}
where all the PA distortions are shaped.
In addition, under Assumption~\ref{asm:q}, the shaped distortion power is
%and the consequent power of the received PA distortion is
\[
\mathbb{E}[|\xi_\omega(t)|^2] \!= \!
\frac{4(N-1)A^2\psi^2}{3}   \sin^2 \left( \frac{\pi d}{\lambda} \sin(\theta)\right).
\]
We will refer to the above scheme as the Tau-Sigma-Delta (T$ \Sigma \Delta$) scheme in the sequel.


\begin{figure}[t]
	\centering
	\includegraphics[width=0.5\linewidth]{sd_delay}
	\caption{ The first-order ${\Sigma \Delta}$ modulator with time-delay components.}
	\label{fig:sd_delay}
\end{figure}







\subsection{Further Discussions}

In this subsection we discuss some practical aspects with the $\Sigma \Delta$ PA distortion shaping concept presented in the previous subsections.
%Let us first discuss the practical aspects of the presented spatial $\Sigma \Delta$ approach.
%In particular, there is a issue in implementing the first-order $\Sigma \Delta$ modulator in Fig.~\ref{fig:sd}.
%The presented spatial $\Sigma \Delta$ approach has an issue in terms of practical implementations.
In practice, the $\Sigma \Delta$ modulator is implemented in analog domain.
The issue arising is that the PAs may introduce certain time delays to the signals.
Let $\bar \tau_n$ be the PA time delay at the $n$th antenna.
Then the delayed PA output feedback
$u_n(t-\bar \tau_n)/A$
%(e.g., $u_1(t-\bar \tau_1)/A$ with $\bar \tau_1$ being the PA delay at the first antenna)
and the PA input $b_n(t)$ will be mismatched.
%, which will result in undesired PA distortion $q_n(t)$.
This issue can be fixed by adding time-delay components in appropriate positions;
see Fig.~\ref{fig:sd_delay} for the structure.
Note that such structure is realizable in practice.
For example, in the traditional feedforward PA linearization method~\cite{cripps2006rf}, a similar structure with time-delay components has been realized.





It is also worthwhile to discuss other $\Sigma \Delta$ modulator structures.
Conceptually, the first-order $\Sigma \Delta$ modulator in Fig.~\ref{fig:sd} can be extended to higher-order $\Sigma \Delta$ modulators,
which aim to provide stronger PA distortion shaping effects~\cite{aziz1996overview,corey2016spatial,shao2020multiuser}.
For example, we can consider the second-order $\Sigma \Delta$ modulator  illustrated in Fig.~\ref{fig:sd_second_order}.
%where the received signal of the T$\Sigma \Delta$ approach is derived as
One can show that for this second-order extension, the T$\Sigma \Delta$ scheme applies linear PAs at the last two antennas and has the received signal given by
\begin{equation}\label{eq:receve_sig_second_sd}
\ba(\theta)^\Tsf \bu(t) = A\ba(\theta)^\Tsf \bx(t) + A(1 - e^{-\jj \omega})^2  \tilde Q_\omega(t),
\end{equation}
where $\tilde Q_\omega(t)=\sum_{n=1}^{N-2}q_n(t)e^{- \jj \omega(n-1)}$.
%See Appendix~\ref{sec:app2} for the detailed derivation of \eqref{eq:receve_sig_second_sd}.
%The detailed derivation of \eqref{eq:receve_sig_second_sd} is relegated to Appendix~\ref{sec:app2}.
Compared with the first-order scheme in~\eqref{eq:receive_sig_sd}, the PA distortion shaping filter in~\eqref{eq:receve_sig_second_sd} is of higher order and has a sharper shape,
%we observe in~\eqref{eq:receve_sig_second_sd} a higher-order PA distortion shaping filter with a sharper shape,
and the PA distortions may be more well suppressed in the low-angle region.
%Therefore, the second-order $\Sigma \Delta$ modulator has the potential to achieve lower PA distortions than the first-order $\Sigma \Delta$ modulator.
Also, the no-overloading condition is given by $|x_n(t)| \le \chi - 3\psi$ for all $n$.
But we should also note that the second-order $\Sigma \Delta$ modulator  requires higher hardware complexity to realize.
%the structure in Fig.~\ref{fig:sd_second_order} and to apply linear PAs at two antennas instead of one antenna (for the T$\Sigma \Delta$ approach).
%Therefore, there exists a trade-off between strong PA distortion shaping and low hardware complexity.
%Following the same spirit, we may consider even higher-order $\Sigma \Delta$ modulator to pursue further PA distortion suppression, but with more expensive hardware.



\begin{figure}[t]
	\centering
	\includegraphics[width=0.5\linewidth]{sd_second_order}
	\caption{The second-order ${\Sigma \Delta}$ modulator.}
	\label{fig:sd_second_order}
\end{figure}







\section{Application to Multi-User MIMO-OFDM Precoding}
\label{sec:application}

In this section, we apply our spatial $\Sigma \Delta$ modulation concept to the multi-user MIMO-OFDM downlink scenario.
%showcasing a practical application, namely multi-user MIMO-OFDM precoding with nonlinear PAs.
The system model under consideration is depicted in Fig.~\ref{fig:sys_mod}, where
%Fig.~\ref{fig:sys_mod} depicts the system model, where
%As depicted in Fig.~\ref{fig:sys_mod},
%We consider a multi-user MIMO-OFDM downlink scenario,
a BS equipped with $N$ antennas simultaneously serves $K$ single-antenna users over a frequency-selective channel.
The modulation scheme is OFDM.
The transmit antennas are assumed to have nonlinear PA effects, and the first-order $\Sigma \Delta$ modulator in Fig.~\ref{fig:sd} is incorporated to combat the PA nonlinear distortion effects.









\subsection{OFDM Signal Model}

To begin with, we describe the implementation of the OFDM signals $x_1(t), \dots, x_N(t)$ in Fig.~\ref{fig:sys_mod}.
For simplicity, we focus on one OFDM block.
% specifically, we consider the zeroth OFDM block.
The ideal OFDM signal we desire to implement takes the form
\begin{equation}\label{eq:OFDM_sig_ideal}
	x^{\sf ideal}_n(t) =  \sum_{p=0}^{M_s-1} z_{n,p} e^{\jj  \frac{2\pi p}{T}t}, \quad  0 \le t < T,
\end{equation}
for $n=1,\dots,N$,
where
$M_s$ is the number of subcarriers,
$z_{n,p} \in \Cbb$ is the signal transmitted at antenna $n$ and subcarrier $p$,
and $T$ is the duration of one OFDM symbol.
Before transmission, a cyclic prefix (CP) is inserted at the beginning of each $x^{\sf ideal}_n(t)$ to avoid inter-symbol interference caused by the frequency-selective channel, i.e.,
\begin{equation}\label{eq:OFDM_sig_ideal_cp}
	x^{\sf ideal}_n(t) = x^{\sf ideal}_n(t+T), \quad -T_{\sf CP} \le t < 0,
\end{equation}
where $T_{\sf CP}$ is the duration of the CP.

\begin{figure*}[t]
	\centering
	\includegraphics[width=0.9\linewidth]{MIMO_OFDM}
	\caption{Multi-user MIMO-OFDM system model with the first-order ${\Sigma \Delta}$ modulator.}
	\label{fig:sys_mod}
\end{figure*}

\begin{figure}[t]
	\centering
	\includegraphics[width=0.6\linewidth]{OFDM_approx}
	\caption{Illustration of the OFDM signal implementation.}
	\label{fig:OFDM_approx}
\end{figure}

In accordance with the LTE and 5G standards~\cite{dahlman20134g,5G_ofdm}, we implement $x^{\sf ideal}_n(t)$ via inverse discrete Fourier transform (IDFT) with oversampling.
%Specifically, we generate the approximated signal of $x^{\sf ideal}_n(t)$
%in \eqref{eq:chapt_ofdm_OFDM_sig_ideal}--\eqref{eq:chapt_ofdm_OFDM_sig_ideal_cp} by
As illustrated in Fig.~\ref{fig:OFDM_approx}, consider approximating $x^{\sf ideal}_n(t)$ in \eqref{eq:OFDM_sig_ideal}--\eqref{eq:OFDM_sig_ideal_cp} by
\begin{equation}\label{eq:OFDM_sig}
	x_n(t) =  \sum_{m=-M_{\sf CP}}^{M-1}  x_{n,m} \Pi(t-mT_s), \quad -T_{\sf CP} \le t < T,
\end{equation}
where
$x_{n,m} = x_n^{\sf ideal}(mT_s)$ is a sample of $x_n^{\sf ideal}(t)$;
$T_s = T/(\kappa M_s)$ is the sampling period, with $\kappa \ge 1$ being the oversampling ratio;
$M_{\sf CP} = T_{\sf CP}/T_s$ is the number of samples for the CP;
$M = \kappa M_s$;
$\Pi(t)$ is a rectangular pulse with width $T_s$, i.e.,
\begin{equation*}
	\Pi(t) =
	\begin{cases}
		1, &\quad 0 \le t < T_s,\\
		0, &\quad {\rm otherwise}.
	\end{cases}
\end{equation*}

\noindent
%Fig.~\ref{fig:chapt_ofdm_OFDM_approx} provides an illustration of such signal approximation, from which we see that
We implement the OFDM signals by generating the approximate signals in~\eqref{eq:OFDM_sig};
the approximation accuracy improves as the oversampling ratio $\kappa$ increases.
In practice, equation~\eqref{eq:OFDM_sig} can be realized by DACs with sample-and-hold operations.
Moreover, according to the equations in~\eqref{eq:OFDM_sig_ideal}--\eqref{eq:OFDM_sig_ideal_cp}, the samples $x_{n,m}$'s in~\eqref{eq:OFDM_sig} are given by
\begin{subnumcases}{x_{n,m} =}
	\sum_{p=0}^{M_s-1} z_{n,p} e^{\jj  \frac{2\pi mp}{M}},   & $0 \le m \le M-1,$ \label{eq:idft_scalar} \\
	x_{n,m+M},   &$-M_{\sf CP} \le m \le -1.$ \label{eq:cp_add}
\end{subnumcases}
Equation~\eqref{eq:idft_scalar} can be rewritten in an equivalent vector form
\begin{equation}\label{eq:idft_vect}
	[x_{n,0},\dots,x_{n,M-1}]^\Tsf = \bF [z_{n,0},\dots,z_{n,M_s-1},0,\dots,0]^\Tsf,
\end{equation}
where
$\bF = [e^{\jj \frac{2\pi mp}{M}}]_{m,p} \in \Cbb^{M \times M}$ is a scaled IDFT matrix.
Equation~\eqref{eq:idft_vect} indicates that the $x_{n,m}$'s can be obtained by performing IDFTs to appropriately zero-padded $z_{n,p}$'s.
%$\{x_{n,m}\}_{m=0}^{M-1}$ can be generated by applying an $M$-point IDFT to an appropriately zero-padded $\{z_{n,v}\}_{v=0}^{M_s-1}$.
For convenience, we will represent~\eqref{eq:idft_vect} as
\begin{equation}\label{eq:bar_bx_n}
	\bar \bx_n = \bF_s \bar \bz_n,
\end{equation}
where
$\bar \bx_n \!\!=\!\! [x_{n,0},\dots,x_{n,M-1}]^\Tsf$, $\bar \bz_n \!\!=\!\! [z_{n,0},\dots,z_{n,M_s-1}]^\Tsf$, and
$\bF_s \in \Cbb^{M \times M_s}$ aggregates the first $M_s$ columns of $\bF$.
%To summarize, combining the IDFT operation in \eqref{eq:bar_bx_n}, the CP addition operation in \eqref{eq:cp_add}, and the signal generation operation in \eqref{eq:OFDM_sig} gives the implementation process of the $x_n(t)$'s  in Fig.~\ref{fig:sys_mod}.
The left-hand side of Fig.~\ref{fig:sys_mod} illustrates the aforementioned implementation process that involves the IDFT operation in \eqref{eq:bar_bx_n}, the CP addition operation in \eqref{eq:cp_add} and the signal generation operation in \eqref{eq:OFDM_sig}.

%\begin{figure}%[H]
%	\centering
%	\begin{minipage}[t]{0.48\textwidth}
%		\centering
%	\includegraphics[width=1\linewidth]{./Figs/OFDM_approx.eps}
%	\caption{\small Illustration of the OFDM signal implementation.}
%	\label{fig:OFDM_approx}
%	\end{minipage}
%	\begin{minipage}[t]{0.48\textwidth}
%		\centering
%		\includegraphics[width=1\linewidth]{./Figs/Multi_path.eps}
%	\caption{\small The multi-path channel.}
%	\label{fig:multi_path}
%	\end{minipage}
%\end{figure}








\subsection{Transmission Signal Model}
%\subsection{SPATIAL $\Sigma\Delta$ APPLICATION}


%We implement the $\Sigma\Delta$ modulator in space for PA distortion  shaping.
%%The idea follows the same spirit as the spatial $\Sigma\Delta$ for quantization noise shaping in Section~\ref{sec:sd_all},
%The idea has been described in Section~\ref{sec:sd_all}-\ref{sec:sd}, where we assume a ULA at the BS operating in a low-angle region.
%The OFDM signals $x_n(t)$'s serve as the $\Sigma \Delta$ modulator input and it follows  from Fig.~\ref{fig:sd} that the $\Sigma \Delta$ modulator output is given by


The OFDM signals $x_n(t)$'s are then fed into the first-order $\Sigma \Delta$ modulator.
We see from Fig.~\ref{fig:sd} that the $\Sigma \Delta$ modulator output is given by
\begin{equation}\label{eq:sys_mod_unt}
	u_n(t) = Ax_n(t) +  A(q_n(t) -  q_{n-1}(t)).
\end{equation}
Following the $\Sigma \Delta$ PA distortion shaping rationale in Section~\ref{sec:sd_all}, we assume a ULA at the BS operating in a low-angle region and using a small inter-antenna spacing.
%a multi-path channel model is assumed for each user;
We also assume that the channel from the BS to each user has $J$ paths;
%a multi-path channel model for each user;
see Fig.~\ref{fig:multi_path} for the scenario.
The channel input-output relation is described by

\begin{figure}[t]
	\centering
	\includegraphics[width=0.8\linewidth]{Multi_path}
	\caption{The multi-path channel.}
	\label{fig:multi_path}
\end{figure}

%The $\Sigma \Delta$ modulator inputs $x_n(t)$'s should satisfy the no-overloading condition in~\eqref{eq:cont_no_overloading}, and the design of such $x_n(t)$'s will be described in the next subsection.
%A multi-path channel model is assumed for each user; see Fig.~\ref{fig:multi_path} for the scenario.
%The channel input-output relation is

\begin{equation}\label{eq:cont_channel_mod}
	{y}_i(t) =  \sum_{j=1}^J \alpha_{i,j} \ba(\theta_{i,j})^\Tsf {\bu}(t - \tau_{i,j}) +  v_i(t), %\quad i=1,\dots,K,
\end{equation}
where
${y}_i(t)$ is the received signal of user $i$;
%$J$ is the number of paths;
$\alpha_{i,j}$,  $\theta_{i,j}$ and $\tau_{i,j}$ are the gain, angle and delay of path $j$ to user $i$, respectively;
$\ba(\theta)$ is the angular response of the ULA defined in~\eqref{eq:angular_response};
$v_i(t)$ is background noise.

%Plugging \eqref{eq:sys_mod_unt} into \eqref{eq:cont_channel_mod} yields the continuous-time system model
%\begin{equation}\label{cont_system_mod}
%{y}_i(t) = \textstyle  A\sum_{j=1}^J \alpha_{i,j} \ba(\theta_{i,j})^\Tsf \bx(t - \tau_{i,j}) + \xi_i(t) +  v_i(t),
%\end{equation}
%where
%\begin{align*}
%\xi_i(t) &\!=\! \textstyle \sum_{j=1}^J \!\sum_{n=1}^{N-1} \! A\alpha_{i,j}  ( 1\!-\!e^{-\jj \omega_{i,j}} ) e^{-\jj \omega_{i,j}(n-1)}q_n( t\!-\!\tau_{i,j} ) \\
%&\qquad + \textstyle \sum_{j=1}^J  A\alpha_{i,j} e^{-\jj \omega_{i,j}(N-1)} q_N(t\!-\!\tau_{i,j})
%\end{align*}
%with $\omega_{i,j} \!=\! \frac{2\pi d}{\lambda} \sin(\theta_{i,j})$,
%is the continuous-time received PA distortion noise.
%{\blue For simplicity of modeling $\xi_{i}(t)$, we make a gross assumption that $q_n( t\!-\!\tau_{i,j} )$  is uncorrelated for different paths $j$.
%This, combined with Assumption~\ref{asm:q}, results in $\mathbb{E}[\xi_i(t)] = 0$ and $\mathbb{E}[|\xi_i(t)|^2] = \sigma_{\xi_i}^2$, where }
%\begin{equation*}
%\begin{aligned}
%\textstyle \sigma_{\xi_i}^2 =  \frac{4(N-1)A^2D^2}{3} & \textstyle \sum_{j=1}^J |\alpha_{i,j}|^2  \sin^2 \left( \frac{\pi d}{\lambda} \sin(\theta_{i,j})\right) \\
%&  \textstyle +  \frac{A^2D^2}{3} \sum_{j=1}^J \left| \alpha_{i,j}\right|^2
%\end{aligned}
%\end{equation*}
%for the $\Sigma \Delta$ scheme in Section~\ref{sec:sd_all}-\ref{sec:sd}, and
%\begin{equation*}
%\begin{aligned}
%\textstyle \sigma_{\xi_i}^2 =   \frac{4(N-1)A^2D^2}{3}  \sum_{j=1}^J |\alpha_{i,j}|^2  \sin^2 \left( \frac{\pi d}{\lambda} \sin(\theta_{i,j})\right)
%\end{aligned}
%\end{equation*}
%for the T$\Sigma \Delta$ scheme in Section~\ref{sec:sd_all}-\ref{sec:sd_tr};
%note that in both cases, $\sigma_{\xi_i}^2$ reduces with either decreasing $|\theta_{i,j}|$ or decreasing $d$.


At the user side,
the received signal $y_i(t)$ is low-pass filtered and then sampled to produce the discrete-time received signal
\begin{equation}\label{eq:yim}
	y_{i,m} = (y_i \otimes \Omega)(mT_s), %\quad  m=-M_{\sf CP},\dots,M-1,
\end{equation}
where
%$p(t) = \frac{1}{T_s}{\rm sinc}(\frac{t}{T_s})$
$\otimes$ denotes the convolution operator and
$\Omega(t)$ is the impulse response of the receive low-pass filter.
%In this work, we assume that $p(t)$ is a root-raised cosine filter with period $T_s$~\cite{mollen2016waveforms}.
%satisfying
%\begin{equation}\label{eq:pt_condition}
%p(t) \ge 0, \quad \int_{-\infty}^\infty |p(t)|^2 dt = 1.
%\end{equation}
Combining \eqref{eq:OFDM_sig}, \eqref{eq:sys_mod_unt}, \eqref{eq:cont_channel_mod} and \eqref{eq:yim} gives the equivalent discrete-time system model
\begin{equation}\label{eq:dist_time_mod}
y_{i,m} = \sum_{\ell=0}^{L-1} \bh_{i,\ell}^\Tsf  \bx_{m-\ell} + \xi_{i,m} +  v_{i,m},
\end{equation}
where
$\bx_m = [x_{1,m},\dots,x_{N,m}]^\Tsf$;
$L$ is the number of channel taps;
$
\bh_{i,\ell} = A\sum_{j=1}^J \alpha_{i,j} \ba(\theta_{i,j})  (\Pi \otimes \Omega)(\ell T_s - \tau_{i,j});
$
$\xi_{i,m}$ is the received PA distortion noise, expressed as
\begin{equation}\label{eq:xiim}
\xi_{i,m} =   \sum_{j=1}^J   (1-e^{-\jj \omega_{i,j}})  \sum_{n=1}^{N-1} \alpha_{i,j} \hat q_{n,m}^{i,j}   +  \sum_{j=1}^J \alpha_{i,j}\hat q_{N,m}^{i,j}
\end{equation}
%\begin{equation}
%	\xi_{i,m} = A  \sum_{j=1}^J \alpha_{i,j} \left(  (1-e^{-\jj \omega_{i,j}}) \sum_{n=1}^{N-1} e^{-\jj \omega_{i,j}(n-1)}\hat q_{n,m}^{i,j} + e^{-\jj \omega_{i,j}(N-1)} \hat q_{N,m}^{i,j} \right)
%\end{equation}
with $\hat q_{n,m}^{i,j} = A  e^{-\jj \omega_{i,j}(n-1)} (q_n \otimes \Omega)(mT_s-\tau_{i,j})$ and
$\omega_{i,j} = \frac{2\pi d}{\lambda} \sin(\theta_{i,j})$;
$v_{i,m}$ is background noise and we assume $v_{i,m} \sim \mathcal{CN}(0,\sigma_v^2)$~\cite{tse2005fundamentals}.
We see from~\eqref{eq:xiim} that each distortion term $\sum_{n=1}^{N-1} \alpha_{i,j} \hat q_{n,m}^{i,j}$ is shaped by a high-pass filter in space.



%$\xi_{i,m} = (\xi_i * p)(mT_s)$ is the received distortion noise;
%$v_{i,m}$ is background noise and follows $\mathcal{CN}(0,\sigma_v^2)$~\cite{tse2005fundamentals}.
%By assuming that the continuous-time received PA distortion noise $\xi_i(t)$ in~\eqref{cont_system_mod} is white noise, one can show that $\xi_{i,m}$ has mean zero and variance $\sigma_{\xi_i}^2$.
%We further make the following assumption.
%
%
%
%
%\begin{Asm}\label{asm:eta}
%	The combined noise term $\eta_{i,m} \triangleq \xi_{i,m} +  v_{i,m}$ in \eqref{eq:dist_time_mod} has $\eta_{i,m} \sim \mathcal{CN}(0,\sigma^2_{\eta_i})$, where $\sigma^2_{\eta_i} = \sigma^2_{\xi_i} + \sigma_v^2$.
%\end{Asm}


Finally, the frequency-domain received signals are obtained by performing CP removal and discrete Fourier transform (DFT) to the $y_{i,m}$'s, i.e.,
%$\bar \br_i   = [r_{i,0}, \dots, r_{i,M_s-1}]^\Tsf$ is obtained by performing CP removal and discrete Fourier transform (DFT) to the $y_{i,m}$'s, i.e.,
\begin{equation}\label{eq:bri}
	\bar \br_i = \bF_s^\Hsf \bar \by_i,
\end{equation}
where
$\bar \br_i  \!\! = \!\! [r_{i,0}, \dots, r_{i,M_s-1}]^\Tsf$ and
$\bar \by_i \!\! = \!\! [y_{i,0}, \dots, y_{i,M-1}]^\Tsf$ are the frequency-domain and CP-free discrete-time received signals for user $i$, respectively.
By combining~\eqref{eq:bar_bx_n}, \eqref{eq:dist_time_mod} and \eqref{eq:bri}, we arrive at the frequency-domain system model
\begin{equation}\label{eq:freq_mod}
	r_{i,p} = \check \bh_{i,p}^\Tsf \bz_p+ \check \xi_{i,p} + \check v_{i,p}, %\quad i=1,\dots,K,~p=0,\dots,M_s-1,
\end{equation}
where
$\check \bh_{i,p} = \sum_{\ell=0}^{L-1} {\bh}_{i,\ell} e^{-\jj \frac{2 \pi \ell (p-1)}{M} }$ is the channel for user $i$ at subcarrier $p$,
$\bz_p = [z_{1,p},\dots,z_{N,p}]^\Tsf$ is the transmitted signal at subcarrier $p$,
$[\check \xi_{i,0}, \dots, \check \xi_{i,M_s-1}]^\Tsf = \bF_s^\Hsf [\xi_{i,0},\dots,\xi_{i,M-1}]^\Tsf$ is the received PA distortion noise, and
$\check v_{i,p} \sim \mathcal{CN}(0,\sigma^2_v)$ is background noise.



%and applying Assumption~\ref{asm:eta} result in the equivalent frequency-domain system model
%\begin{equation}\label{eq:frequency_sys_mod_sd}
%r_{i,v} = \tilde \bh_{i,v}^\Tsf \bz_v + \tilde \eta_{i,v}, %\quad i=1,\dots,K, m=0,\dots,M_s-1,
%\end{equation}
%%for $i=1,\dots,K, m=0,\dots,M_s-1$,
%where
%$\tilde \bh_{i,v} \!= \!\sum_{\ell=0}^{L-1} {\bh}_{i,\ell} e^{-\jj \frac{2 \pi \ell (v-1)}{M} }$ is the frequency-domain channel for user $i$ at subcarrier $v$,
%$\bz_v \!=\! [z_{1,v},\dots,z_{N,v}]^\Tsf$ is the frequency-domain transmitted signal at subcarrier $v$ and
%$\tilde \eta_{i,v} \sim \mathcal{CN}(0,\sigma^2_{\eta_i})$ is noise.










\subsection{MIMO-OFDM Precoding under Signal Amplitude Constraints}
\label{sec:precoding}


%We are interested in the precoding problem under the aforementioned MIMO-OFDM downlink scenario.
%The task is
To complete the picture, we design the multi-user precoding under the effective model \eqref{eq:freq_mod}.
%Under the aforementioned MIMO-OFDM downlink scenario, the BS aims to simultaneously and reliably send data symbols to the users.
To describe,
let $s_{i,p} \in \Cbb$ be the data symbol for user $i$ at subcarrier $p$.
We assume that each $s_{i,p}$ is drawn from a quadrature amplitude modulation (QAM) constellation, i.e.,
\begin{equation*}\label{eq:QAM_cons}
	s_{i,p} \in \setS = \{ s_R + \jj s_I  \mid  s_R, s_I \in \{ \pm 1, \pm 3, \ldots, \pm (2D - 1) \} \},
\end{equation*}
where $D$ is a positive integer (the QAM size is $4D^2$).
Assuming perfect knowledge of the channels $\check \bh_{i,p}$'s at the BS, we aim to design the transmitted signals $\bz_p$'s such that the users will receive their intended data symbols.
More specifically, we want the noise free part of $r_{i,p}$ in~\eqref{eq:freq_mod} to take the form
\begin{equation}\label{eq:symbol_shaping}
	\check \bh_{i,p}^\Tsf \bz_p \approx \beta_i s_{i,p},
\end{equation}
where $\beta_i>0$ denotes a QAM constellation scaling factor for user $i$.
With a larger $\beta_i$, the received signal can be more robust to noise.
The $\beta_i$'s are designed by the BS and each user is assumed to know its $\beta_i$.
As a new design requirement introduced by the spatial $\Sigma \Delta$ approach, the no-overloading condition in~\eqref{eq:cont_no_overloading} must be guaranteed.
%the condition in~\eqref{eq:chapt_sd_cont_no_overloading} must be guaranteed to avoid the overloading effect.
%According to~\eqref{eq:chapt_sd_OFDM_sig}, the condition~\eqref{eq:chapt_sd_cont_no_overloading} amounts to constraining the discrete-time transmitted signals by
It follows from the temporal signal model~\eqref{eq:OFDM_sig} that the no-overloading condition~\eqref{eq:cont_no_overloading} amounts to constraining the discrete-time signals $\bX = [\bx_0, \dots, \bx_{M-1}]$ by
%the signal amplitude constraints
\begin{equation}\label{eq:no_overloading}
	%\| \bZ \bF_s^\Tsf\|_{\rm max} \le \chi - \psi,
\bX \in \setX = \{ \bX \in \Cbb^{N \times M} ~\vert~	\|\bX\|_{\rm max} \le \chi - \psi \},
\end{equation}
where
%$\bX = [\bx_0,\dots,\bx_{M-1}]$;
%$\bZ = [\bz_0,\dots,\bz_{M_s-1}]$ and
$\|\bX\|_{\rm max} = \max_{i,j}|x_{i,j}|$.
%note that here we heuristically choose $\chi = r_{\rm max}$.


%Now we turn to the precoding problem.
%To describe, assuming perfect knowledge of the channels $\tilde \bh_{i,v}$'s at the BS,
%the precoding task is to design the transmitted signals $\bX = [\bx_0,\dots,\bx_{M-1}]$ under the no-overloading condition,
%such that the users will receive their data symbols.
%According to~\eqref{eq:cont_no_overloading} with $\chi=r_{\rm max}$ and~\eqref{eq:OFDM_sig}, the no-overloading condition amounts to constraining $\bX$ by
%\begin{equation}\label{eq:no_overloading}
%	\bX \in \setX = \{ \bX \in \Cbb^{N \times M} ~\vert~  |x_{n,m}| \le r_{\rm max}-D, ~\forall n, m\}.
%\end{equation}
%%More specifically,
%Also, we see from~\eqref{eq:bar_bx_n} that $\bX$ is generated by $\bX = \bZ \bF_s^\Tsf$, where $\bZ = [\bz_0,\dots,\bz_{M_s-1}]$, so the design of $\bX$
%is essentially the design of the frequency-domain signals $\bZ$.
%Let $s_{i,v} \in \Cbb$ be the data symbol for user $i$ at subcarrier $v$ and we assume that each $s_{i,v}$ is drawn from a quadrature amplitude modulation (QAM) constellation, i.e.,
%\begin{equation*}\label{eq:QAM_cons}
%	s_{i,v} \in \setS = \{ s_R + \jj s_I  \mid  s_R, s_I \in \{ \pm 1, \pm 3, \ldots, \pm (2V - 1) \} \},
%\end{equation*}
%where $V$ is a positive integer (the QAM size is $4V^2$).
%We aim to make the noise-free part of the received signal in~\eqref{eq:frequency_sys_mod_sd} take the form
%\begin{equation}\label{eq:pre_goal}
%\tilde \bh_{i,v}^\Tsf  \bz_v \approx \beta_i s_{i,v}, %\quad i=1\dots,K,~m=1,\dots,M_s,
%\end{equation}
%where
%Next we develop two schemes for the precoding design.



\subsubsection{$\Sigma \Delta$ Zero-Forcing}
%\label{sec:zf}

We first develop a zero-forcing (ZF) scheme for the precoding design.
In the ZF scheme, the transmitted signals are designed by the form
\begin{equation}\label{eq:zf}
	\bz_p =
	\frac{1}{\Gamma} \check \bH_p^\dagger \bs_p, \quad p=0,\dots,M_s-1,
\end{equation}
where
$\check \bH_p^\dagger = \check \bH_p^\Hsf (\check \bH_p \check \bH_p^\Hsf)^{-1}$,
$\check \bH_p = [\check \bh_{1,p},\dots,\check \bh_{K,p}]^\Tsf$, $\bs_p = [s_{1,p},\dots,s_{K,p}]^\Tsf$, and $\Gamma$ is a normalization factor to ensure that the no-overloading condition in~\eqref{eq:no_overloading} holds.
%$\|\bX\|_{\rm max} \le r_{\rm max}-D,$.
It is easy to see that such $\Gamma$ is given by
\[
\Gamma = \frac{\| [\check\bH_0^\dagger\bs_0,\dots,\check\bH_{M_s-1}^\dagger\bs_{M_s-1}] \bF_s^\Tsf \|_{\rm max} }{\chi - \psi}.
\]
%Under the ZF scheme~\eqref{eq:zf},
The noise-free received signal is then expressed as
\begin{equation*}
	\check \bh_{i,p}^\Tsf  \bz_p = \frac{1}{\Gamma} s_{i,p},
\end{equation*}
from which we observe that the corresponding QAM constellation scaling factors are
$
\beta_i = {1}/{\Gamma},~ \forall i.
$


%\begin{equation}\label{eq:zf}
%	\bz_v =
%	\frac{1}{\gamma} \tilde \bH_v^\dagger \bs_v, \quad v=0,\dots,M_s-1,
%\end{equation}
%where $\tilde \bH_v^\dagger = \tilde \bH_v^\Hsf (\tilde \bH_v \tilde \bH_v^\Hsf)^{-1}$, $\tilde \bH_v = [\tilde \bh_{1,v},\dots,\tilde \bh_{K,v}]^\Tsf$, $\bs_v = [s_{1,v},\dots,s_{K,v}]^\Tsf$ and $\gamma$ is a normalization factor to ensure the no-overloading condition $\bX \in \setX$.
%It is easy to see that
%\[
%\gamma = \frac{\|\bF_s [\tilde\bH_0^\dagger\bs_0,\dots,\tilde\bH_{M_s-1}^\dagger\bs_{M_s-1}]^\Tsf \|_{\rm max} }{r_{\rm max}-D},
%\]
%where $\|\bX\|_{\rm max} = \max_{i,j}|x_{i,j}|$ denotes the matrix max norm.
%Substituting \eqref{eq:zf} in \eqref{eq:frequency_sys_mod_sd} gives
%\begin{equation*}
%	r_{i,v} = \frac{1}{\gamma} s_{i,v} + \tilde \eta_{i,v},
%\end{equation*}
%from which we observe that the corresponding QAM constellation scalings are
%$
%\beta_i = {1}/{\gamma},~ \forall i.
%$



\subsubsection{$\Sigma \Delta$ Symbol-Level Precoding}
%label{sec:slp}


We next develop a symbol-level precoding (SLP) scheme.
The idea is to jointly design the transmitted signals $\bz_p$'s and the scaling factors $\beta_i$'s to optimize the detection probability (DP) of all data symbols.
%for the goal~\eqref{eq:symbol_shaping}
%and formulate the SLP design as a maximum DP problem.
%In what follows, we show the development details.
%, which can be divided into three main steps.
We begin with characterizing the effective PA distortion noise power
$\mathbb{E}[|\xi_{i,m}|^2]$ for $\xi_{i,m}$ in \eqref{eq:dist_time_mod}.
By noting the difficulty of exactly calculating $\mathbb{E}[|\xi_{i,m}|^2]$, we resort to convenient approximation.
%$\mathbb{E}[|\eta_{i,m}|^2]$ for $\eta_{i,m} = \xi_{i,m} +  v_{i,m}$ in \eqref{eq:dist_time_mod}.
By invoking Fact~\ref{Fac:no_overloading}, we see that under the no-overloading condition,
%Under the no-overloading condition, it is seen from Fact~\ref{Fac:no_overloading} that
it holds that $|q_n(t)| \le \psi$, and consequently the $\hat q_{n,m}^{i,j}$ in~\eqref{eq:xiim} satisfies
$
|\hat q_{n,m}^{i,j}| \le \hat \psi \triangleq A\psi  \int_{-\infty}^\infty |\Omega(t)|dt.
$
Based on this, we consider
\begin{equation*}
\begin{aligned}
&\mathbb{E}[|\xi_{i,m}|^2] \\
 \approx & \sum_{j=1}^J   |1-e^{-\jj \omega_{i,j}}|^2  \sum_{n=1}^{N-1} |\alpha_{i,j}|^2 \mathbb E[|\hat q_{n,m}^{i,j}|^2]   +  \sum_{j=1}^J |\alpha_{i,j}|^2\mathbb E[|\hat q_{N,m}^{i,j}|^2]\\
 \approx & \frac{4(N-1) \hat \psi^2}{3}   \sum_{j=1}^J |\alpha_{i,j}|^2  \sin^2 \left( \frac{\pi d}{\lambda} \sin(\theta_{i,j})\right) +  \frac{\hat \psi^2}{3} \sum_{j=1}^J \left| \alpha_{i,j}\right|^2,
\end{aligned}
\end{equation*}
where the  second line assumes that $\hat q_{n,m}^{i,j}$'s are zero-mean and independent of each other, and the last line assumes that $\hat q_{n,m}^{i,j}$'s are i.i.d. with $|\hat q_{n,m}^{i,j}| \sim \mathcal{U}{[0,\hat \psi]}$ and $\arg(\hat q_{n,m}^{i,j}) \sim \mathcal{U}{[-\pi,\pi]}$.

Then, we assume that the combined noise term $\eta_{i,m} = \xi_{i,m} +  v_{i,m}$ in~\eqref{eq:dist_time_mod} has $\eta_{i,m} \sim \mathcal{CN}(0,\sigma^2_{\eta,i})$, where $\sigma^2_{\eta,i} = \mathbb{E}[|\xi_{i,m}|^2] + \sigma_v^2$.
As a result, by the time-frequency transformation, one can show that the frequency-domain system model in~\eqref{eq:freq_mod} is simplified as
\begin{equation}\label{eq:freq_mod_approx}
	r_{i,p} = \check \bh_{i,p}^\Tsf \bz_p+ \check \eta_{i,p}, %\quad \check \eta_{i,p} \sim \mathcal{CN}(0,\sigma^2_{\eta,i}).
\end{equation}
where $\check \eta_{i,p} = \check \xi_{i,p} + \check v_{i,p}$ and $ \check \eta_{i,p} \sim \mathcal{CN}(0,\sigma^2_{\eta,i})$.
Note that if one consider the T$\Sigma\Delta$
scheme in Section~\ref{sec:sd_tr}, then
\[
\mathbb{E}[|\xi_{i,m}|^2] \approx \frac{4(N-1) \hat \psi^2}{3}   \sum_{j=1}^J |\alpha_{i,j}|^2  \sin^2 \left( \frac{\pi d}{\lambda} \sin(\theta_{i,j})\right),
\]
and the remaining derivation follows the same spirit  as before.
%%Based on this,
%Under the no-overloading condition, we further make the following gross assumption.
%
%\begin{Asm}\label{asm:hatq}
%	Under the condition in~\eqref{eq:no_overloading}, the $\hat q_{n,m}^{i,j}$'s are i.i.d. with $|\hat q_{n,m}^{i,j}| \sim \mathcal{U}{[0,\hat \psi]}$ and $\arg(\hat q_{n,m}^{i,j}) \sim \mathcal{U}{[-\pi,\pi]}$.
%	Also, each $\hat q_{n,m}^{i,j}$ is independent of any other random variables.
%\end{Asm}

%\noindent
%Under Assumption~\ref{asm:hatq}, the received PA distortion noise $\xi_{i,m}$ in~\eqref{eq:xiim} has
%%By applying Assumption~\ref{asm:chapt_sd_hatq} to equation~\eqref{eq:chapt_sd_xiim}, the received PA distortion noise $\xi_{i,m}$
%$\mathbb{E}[\xi_{i,m}] = 0$ and $\mathbb{E}[|\xi_{i,m}|^2] =  \sigma_{\xi,i}^2$, where
%\begin{equation*}
%\begin{aligned}
%\sigma_{\xi,i}^2  = \textstyle \frac{4(N-1) \hat \psi^2}{3}   &\textstyle\sum_{j=1}^J |\alpha_{i,j}|^2  \sin^2 \left( \frac{\pi d}{\lambda} \sin(\theta_{i,j})\right) \\
%& \textstyle +  \frac{\hat \psi^2}{3} \sum_{j=1}^J \left| \alpha_{i,j}\right|^2
%\end{aligned}
%\end{equation*}
%for the $\Sigma \Delta$ scheme in Section~\ref{sec:sd_all}-\ref{sec:sd} and
%\begin{equation*}
%\begin{aligned}
%\sigma_{\xi,i}^2  = \textstyle \frac{4(N-1) \hat \psi^2}{3}   \textstyle\sum_{j=1}^J |\alpha_{i,j}|^2  \sin^2 \left( \frac{\pi d}{\lambda} \sin(\theta_{i,j})\right)
%\end{aligned}
%\end{equation*}
%for the T$\Sigma \Delta$ scheme in Section~\ref{sec:sd_all}-\ref{sec:sd_tr}.
%Note that in both cases, the distortion noise power $\sigma_{\xi,i}^2$ reduces as $|\theta_{i,j}|$ and/or $d$ decreases.


%\noindent
%Under Assumption~\ref{asm:eta},


%Now we characterize the DP under~\eqref{eq:freq_mod_approx}.
Next, we characterize the DP under the signal model in~\eqref{eq:freq_mod_approx}.
Each user detects the symbols by
\begin{equation}\label{eq:dec}
	\hat{s}_{i,p} = {\rm dec} \left( \frac{r_{i,p}}{\beta_i} \right),
\end{equation}
where $\hat{s}_{i,p}$ is the detected symbol of $s_{i,p}$ and ${\rm dec}(\cdot)$ is the decision function corresponding to the QAM constellation set $\setS$.
%Given the data symbol $s_{i,p}$, the DP is defined as the probability of the correct detection $\hat{s}_{i,p} = s_{i,p}$ at the user side, i.e.,
%\begin{equation}\label{eq:dp}
%	\begin{aligned}
%        {\sf DP}_{i,p}=&~ {\sf DP}^R_{i,p} \times {\sf DP}^I_{i,p}, \\
%        		{\sf DP}^R_{i,p}\triangleq &~ {\rm Pr}( \Re(\hat{s}_{i,p}) = \Re(s_{i,p}) \mid s_{i,p} ), \\
%		{\sf DP}^I_{i,p}\triangleq &~{\rm Pr}( \Im(\hat{s}_{i,p}) = \Im(s_{i,p}) \mid s_{i,p} ).
%	\end{aligned}
%\end{equation}
Given the data symbols $s_{i,p}$'s, the DP is defined as the probability of the correct detection of all the symbols, i.e.,
\begin{equation}\label{eq:dp}
	\begin{aligned}
		{\sf DP} &= \textstyle  {\rm Pr}\left( \hat{s}_{i,p} = s_{i,p}, \forall i,p \mid  \{s_{i,p}\}_{i,p} \right) \\
		&= \textstyle \prod_{i=1}^K \prod_{p=0}^{M_s-1} {\rm Pr}( \hat{s}_{i,p} = s_{i,p} \mid  s_{i,p} ) \\
		&= \textstyle \prod_{i=1}^K \prod_{p=0}^{M_s-1} \left( {\sf DP}^R_{i,p} \times {\sf DP}^I_{i,p} \right),
	\end{aligned}
\end{equation}
%\begin{equation}\label{eq:dp}
%	\begin{aligned}
%		{\sf DP} &= \textstyle \prod_{i=1}^K \prod_{p=0}^{M_s-1} {\rm Pr}( \hat{s}_{i,p} = s_{i,p}  ) \\
%		&= \textstyle \prod_{i=1}^K \prod_{p=0}^{M_s-1} {\rm Pr}( \hat{s}_{i,p} = s_{i,p} \mid  s_{i,p} ) {\rm Pr}( s_{i,p} ) \\
%		&\propto \textstyle \prod_{i=1}^K \prod_{p=0}^{M_s-1} {\rm Pr}( \hat{s}_{i,p} = s_{i,p} \mid  s_{i,p} ) \\
%		&= \textstyle \prod_{i=1}^K \prod_{p=0}^{M_s-1} \left( {\sf CDP}^R_{i,p} \times {\sf CDP}^I_{i,p} \right),
%	\end{aligned}
%\end{equation}
where
\begin{equation*}
	\begin{split}
		{\sf DP}^R_{i,p} &\triangleq {\rm Pr}( \Re(\hat{s}_{i,p}) = \Re(s_{i,p}) \mid s_{i,p} ), \\
		{\sf DP}^I_{i,p} &\triangleq {\rm Pr}( \Im(\hat{s}_{i,p}) = \Im(s_{i,p}) \mid s_{i,p} )
	\end{split}
\end{equation*}
%are the conditional detection probabilities (CDPs) of
%Here, $ {\sf DP}^R_{i,p} $and $ {\sf DP}^I_{i,p}$
represent the detection probabilities for the real and imaginary components of $s_{i,p}$, respectively.
Following the derivations in~\cite{shao2019framework,liu2021symbol}, it is shown that
\begin{equation} \label{eq:CDP}
	{\sf DP}^R_{i,p} \!=\!
	\begin{cases}
		\Phi \! \left( \!\frac{\sqrt{2}a_{i,p}^R}{\sigma_{\eta,i}}\! \right) \!-\! \Phi \!\left(\! \frac{\sqrt{2}c_{i,p}^R}{\sigma_{\eta,i}} \!\right),~ | \Re(s_{i,p}) | \!<\! 2D - 1,\\[2ex]
		\Phi \! \left(\! \frac{-\sqrt{2}c_{i,p}^R}{\sigma_{\eta,i}} \!\right), ~ \Re(s_{i,p}) \!=\!  2D - 1, \\[2ex]
		\Phi \! \left(\! \frac{\sqrt{2}a_{i,p}^R}{\sigma_{\eta,i}} \!\right), ~ \Re(s_{i,p}) \!=\!  -2D + 1,
	\end{cases}
\end{equation}
where $\Phi(x) = \frac{1}{\sqrt{2\pi}} \int_{-\infty}^x e^{-{y^2}/{2}}dy$ and
%is the cumulative distribution function of the standard Gaussian distribution and
\begin{equation*}
	\begin{split}
		a_{i,p}^R &=  \beta_i + \beta_i\Re(s_{i,p})  - \Re(\check \bh_{i,p}^\Tsf \bz_p), \\
		c_{i,p}^R &= -\beta_i + \beta_i\Re(s_{i,p})  - \Re(\check \bh_{i,p}^\Tsf \bz_p).
	\end{split}
\end{equation*}
The result in \eqref{eq:CDP} also applies to ${\sf DP}^I_{i,p}$ by replacing ``$R$'' with ``$I$'' and ``$\Re$'' with ``$\Im$''.












%\subsubsection{Problem Formulation}

%After DP characterization, our next step is to formulate the SLP design.
We formulate an SLP formulation that maximizes the DP in \eqref{eq:dp} subject to the no-overloading condition in \eqref{eq:no_overloading}:
\begin{equation}\label{eq:SLP_origin}
	\begin{aligned}
		\max_{  \bm \beta, \bZ, \bX }
		& ~ \textstyle \prod_{i=1}^K \prod_{p=0}^{M_s-1}\left( {\sf DP}^R_{i,p} \times {\sf DP}^I_{i,p} \right) \\
		{\rm s.t.} ~
		&  ~ \bX = \bZ \bF_s^\Tsf,~ \bX \in \setX,~ \bm \beta \ge \bm 0,
	\end{aligned}
\end{equation}
where $\bm \beta = [\beta_1,\dots,\beta_K]^\Tsf$,
$\bZ = [\bz_0,\dots,\bz_{M_s-1}]$,
and $\bX = \bZ \bF_s^\Tsf$ is due to equation~\eqref{eq:bar_bx_n}.
%According to the DP characterization in~\eqref{eq:dp} and the fact that $-\log(\cdot)$ is a decreasing function, problem~\eqref{eq:SLP_origin} can be recast as
A convenient way of handling the objective function in problem~\eqref{eq:SLP_origin} is to apply the $\log$ function over it, which results in
\begin{equation}\label{eq:SLP}
	\begin{aligned}
		\min_{ \bm \beta, \bZ, \bX }
		& ~
		F(\bm \beta, \bZ) \triangleq -
		\sum_{i=1}^{K} \sum_{p=0}^{M_s-1} \left(\log {\sf DP}^R_{i,p} + \log {\sf DP}^I_{i,p} \right) \\
		{\rm s.t.} ~
		& ~ \bX = \bZ \bF_s^\Tsf,~ \bX \in \setX,~ \bm \beta \ge \bm 0.
	\end{aligned}
\end{equation}



\noindent
%Problem~\eqref{eq:SLP} is a large-scale convex problem.
%The large-scale problem nature is due to the fact that $N$ and $M_s$ are usually large in practice, e.g., $N=128$ and $M_s=300$.
Problem~\eqref{eq:SLP} is a large-scale convex problem.
It is large-scale because $M$, $M_s$ and $N$ are large under typical massive MIMO-OFDM system settings;
%Also, in massive MIMO-OFDM systems, $M$, $M_s$ and $N$ are typically large in practice;
for example, when $(M,M_s,N)=(512,350,64)$, the dimensions of both $\bX$ and $\bZ$ will be over $20,000$.
This large-scale problem nature prevents us from applying general convex optimization tools, such as CVX~\cite{grant2008cvx}, to solve the problem, as the computational complexity will be very high.
Therefore, fast algorithms for problem~\eqref{eq:SLP} are desired, which motivates the last step as follows.






As the last step,
% is to develop an efficient algorithm for problem~\eqref{eq:SLP}.
we custom design an alternating direction method of multipliers (ADMM) algorithm for problem~\eqref{eq:SLP}.
According to~\cite{boyd2011distributed}, ADMM is guaranteed to converge to an optimal solution of problem~\eqref{eq:SLP} under some mild conditions.
The development details are as follows.
%To apply ADMM, we rewrite problem~\eqref{eq:SLP} as
%\begin{equation}\label{eq:SLP_ref}
%	\begin{aligned}
%		\min_{ \bm \beta, \bZ, \bX }
%		& ~
%		F(\bm \beta, \bZ) \triangleq
%		\sum_{i=1}^{K} \sum_{p=0}^{M_s-1} \left(-\log {\sf CDP}^R_{i,p} - \log {\sf CDP}^I_{i,p} \right) \\
%		{\rm s.t.} ~
%		& ~  \bZ \bF_s^\Tsf = \bX, ~ \bX \in \setX,~ \bm \beta \ge \bm 0,
%	\end{aligned}
%\end{equation}
%where $\setX \triangleq \{ \bX \in \Cbb^{N \times M} ~\vert~\|\bX\|_{\rm max} \le \chi - \psi\}$.
The augmented Lagrangian of problem~\eqref{eq:SLP} is
\begin{equation*}
	\begin{aligned}
		L_{\rho}(\bm \beta,\bZ,\bX, \bm \Lambda) \!\!=\!\! \textstyle F(\bm \beta, \bZ)\!\! + \!\!  \langle \bX \!\!-\!\! \bZ \bF_s^\Tsf, \bm \Lambda \rangle \!\!+ \!\! \frac{\rho}{2}\| \bX \!\!-\!\! \bZ \bF_s^\Tsf \|_F^2,
	\end{aligned}
\end{equation*}
%\begin{equation*}
%	\begin{aligned}
	%		L_{\rho}(\bm \beta,\bZ,\bX, \bm \Lambda) &= \textstyle F(\bm \beta, \bZ) +  \langle \bX - \bZ \bF_s^*, \bm \Lambda \rangle \\ & \qquad +  \textstyle\frac{\rho}{2}\| \bX - \bZ \bF_s^* \|_F^2,
	%	\end{aligned}
%\end{equation*}
where
$\rho > 0$ is a penalty parameter and
$\bm \Lambda \in \Cbb^{N \times M}$ denotes the dual variables.
%$\mathcal{I}_{\mathcal X}(\bX)$ and $\mathcal{I}_{\bm \beta \ge \bm 0}(\bm \beta)$ are the indicator functions of the sets $\mathcal{X}$ and $\bm \beta \ge \bm 0$, respectively.
%In accordance with the literature~\cite{boyd2011distributed}, the ADMM iterations are
The ADMM iterations are given by
\begin{subequations}\label{eq:admm}
	\begin{align}
		\bX^{k+1} &= \arg\min\limits_{\bX \in \setX} L_{\rho}(\bm \beta^k,\bZ^k,\bX, \bm \Lambda^{k}), \label{eq:admm_pro1}\\
		(\bm \beta^{k+1} ,\bZ^{k+1}) &= \arg\min\limits_{\bm \beta \ge \bm 0} L_{\rho}(\bm \beta,\bZ,\bX^{k+1}, \bm \Lambda^{k}), \label{eq:admm_pro2}\\
		\boldsymbol{\Lambda}^{k+1} &= \boldsymbol{\Lambda}^{k} + \rho (\bX^{k+1} - \bZ^{k+1} \bF_s^\Tsf),\label{eq:admm_pro3}
	\end{align}
\end{subequations}
for $k=0,1,\dots$, where $(\bm \beta^0,\bZ^0,\bX^0, \bm \Lambda^0)$ is the initialization.
It remains to specify the updates in \eqref{eq:admm_pro1} and \eqref{eq:admm_pro2}.
%\begin{algorithm}[H]
%	\caption{ADMM for problem~\eqref{eq:SLP}}
%	\begin{algorithmic}[1]
	%		\STATE
	%		{\bf given} an initial point $(\bm \beta^0,\bZ^0,\bX^0, \bm \Lambda^0)$, a penalty parameter $\rho > 0$, $k=1$.
	%		\REPEAT
	%		\STATE
	%		update $(\bm \beta^{k},\bX^{k}) = \arg\min\limits_{\bm \beta,\bX} L_{\rho}(\bm \beta,\bZ^{k-1},\bX; \bm \Lambda^{k-1})$.
	%		\STATE
	%		update $\bZ^{k} = \arg\min\limits_{\bZ} L_{\rho}(\bm \beta^k,\bZ,\bX^k; \bm \Lambda^{k-1})$.
	%		\STATE
	%		update $\boldsymbol{\Lambda}^{k} = \boldsymbol{\Lambda}^{k-1} - (\bX^k - \bZ^k \bF_s^*)$.
	%		\STATE  $k=k+1$.
	%		\UNTIL some stopping criterion is met.
	%	\end{algorithmic}\label{Al:ADMM}
%\end{algorithm}
The update in~\eqref{eq:admm_pro1} is essentially a projection step
\begin{align*}
	\bX^{k+1} &= \Pi_{\setX} \left( \bZ^{k} \bF_s^\Tsf -  \textstyle \frac{1}{\rho} \bm \Lambda^{k}\right),
	%\beta_i^k &= \arg \min_{\beta \ge 0} ~  \textstyle  \sum_{m=1}^{M_s} f_{i,m}(\beta, \bz_m^{k-1}), ~ \forall i. \label{eq:beta_update}
\end{align*}
%where $\Pi_{\setX}(\cdot)$ is the projection operator onto $\setX$.
where the projection $\Pi_{\setX}(\bx) \triangleq \arg\min_{\by\in \setX}\| \bx -\by \|_2^2$ admits a closed-form expression, i.e., if $\hat \bX = \Pi_{\setX}(\bX)$, then we have
\[
\hat x_{n,m} = \min\{ |x_{n,m}|, \chi - \psi \} e^{\jj \cdot \arg(x_{n,m})}, \quad \forall n,m.
\]
The problem in~\eqref{eq:admm_pro2} is a smooth convex problem with non-negative constraints and we solve the problem by the accelerated proximal gradient (APG) method~\cite{beck2017first}.
For convenience, let us express problem~\eqref{eq:admm_pro2} as
\begin{equation}\label{eq:problem_p}
	\min_{\bm \beta \ge \bm 0} h(\bm \beta, \bZ),
\end{equation}
where $h(\bm \beta, \bZ) = L_{\rho}(\bm \beta,\bZ,\bX^{k+1}, \bm \Lambda^{k})$.
The APG updates for problem~\eqref{eq:problem_p} are
\begin{equation}\label{eq:apg}
	\begin{split}
        \bm \beta^{l+1} &= \max\{ \bm 0, \bm \beta_{\sf ex}^l - \gamma_l \nabla_{\bm \beta}h(\bm \beta_{\sf ex}^l, \bZ_{\sf ex}^l) \}, \\
        \bZ^{l+1} &= \bZ_{\sf ex}^l - \gamma_l \nabla_{\bZ}h(\bm \beta_{\sf ex}^l, \bZ_{\sf ex}^l), \\
		%\bp^{l+1} &= \Pi_{\setP} \left( \tilde \bp^l - \gamma_l \nabla h(\tilde \bp^{l}) \right),
	\end{split}
\end{equation}
for $l = 0,1,\dots$
Here, $\bm \beta_{\sf ex}^l$ and $\bZ_{\sf ex}^l$ are the extrapolated points given by
\begin{align*}
\bm \beta_{\sf ex}^l &=  \textstyle {\bm \beta}^{l} + \frac{\mu_{l-1}-1}{\mu_l} ( \bm \beta^{l} - \bm \beta^{l-1}),\\
\bZ_{\sf ex}^l &=  \textstyle {\bZ}^{l} + \frac{\mu_{l-1}-1}{\mu_l} ( \bZ^{l} - \bZ^{l-1}),
%\tilde \bp^l &=  \textstyle \bp^{l} + \frac{q_{l-1}-1}{q_l} ( \bp^{l} - \bp^{l-1}),
\end{align*}
where
$\bm \beta^{-1} = \bm \beta^{0}$, $\bZ^{-1} = \bZ^0$, $\mu_{-1} = 0$ and $\mu_{l} = \frac{1+\sqrt{1+4\mu_{l-1}^2}}{2}$;
$\gamma_l$ is the step size and is determined by the backtracking line search~\cite{beck2017first}.




We should mention the big-O computational complexity of the ADMM algorithm in \eqref{eq:admm}.
The complexity of performing the updates in \eqref{eq:admm_pro1} and \eqref{eq:admm_pro3} is $\mathcal{O}(NM \log_2 M)$, which mainly comes from computing
%the $M$-point IDFT, or
the inverse fast Fourier transform (IFFT).
For the update in \eqref{eq:admm_pro2}, performing one APG iteration in \eqref{eq:apg} requires $\mathcal{O}(KNM_s)$ floating point operations and $\mathcal{O}(KM_s)$ of computing $\Phi$.
Note that $\Phi$ does not have a closed-form expression; its computation can be done by the {\sf erfc} function in MATLAB and consumes more than one floating point operation.







\section{Simulation Results}
\label{sec:simulation}

In this section, we examine the performance of the developed $\Sigma \Delta$ precoding schemes by numerical simulations.


\subsection{Simulation and Algorithm Settings}

The default simulation settings are as follows.
The IDFT/DFT size is $M=512$.
The number of subcarriers is $M_s=300$.
The symbols $s_{i,p}$'s are uniformly drawn from the QAM constellation.
The adopted PA model is the modified Rapp model in \eqref{eq:rapp} with $A=16$, $\varphi=1.1$, $r_{\rm max}=0.1187$, $B=-345$, $C=0.17$ and $\zeta =4$~\cite{3gpp_pa}.
The inter-antenna spacing is $d = \lambda / 8$.
The numbers of channel paths and taps are $J = 4$ and $L = 20$, respectively.
For each channel path, the gain, angle and delay are generated by $\alpha_{i,j} \sim \frac{1}{\sqrt{J}} \mathcal{CN}(0,1)$,  $\theta_{i,j} \sim \mathcal{U}{[-35^\circ,35^\circ]}$ and $\tau_{i,j} \sim \mathcal{U}{[5T_s,15T_s]}$, respectively.
The receive low-pass filter $\Omega(t)$ in \eqref{eq:yim} is the root-raised cosine filter with period $T_s$ and roll-off factor $0.22$~\cite{mollen2016waveforms}.
All the continuous-time signals are represented by 7 times oversampled discrete signals, i.e., $7$ uniform sampling points for every interval $T_s$.
Unless specified, the results to be presented are averaged results over $1000$ independent channel trials.


\begin{table*}%[ht!]
	\centering
	\renewcommand{\arraystretch}{1.5}
	\caption{Summary of the tested precoding schemes.}\label{tab_schemes}
	\resizebox{\linewidth}{!}{
		\begin{tabular}{c|c|c|c|c|c}
			\hline
			name & description &  constraint for $\bX$ & antennas w/ distortion  &  w/ or w/o  $\Sigma \Delta$   & formulation, method\\
			\hline \hline
			$\Sigma \Delta$ ZF  &  ZF in Section~\ref{sec:precoding} w/ $\Sigma \Delta$ & \eqref{eq:no_overloading}  & all  & w/ $\Sigma \Delta$ & \multirow{6}{*}{\thead{\eqref{eq:zf} w/ specific $\Gamma$,\\ closed-form}}  \\ \cline{1-5}
			T$\Sigma \Delta$ ZF  & ZF in Section~\ref{sec:precoding} w/ T$\Sigma \Delta$  & \eqref{eq:no_overloading} & except the last antenna & w/ $\Sigma \Delta$ &  \\ \cline{1-5}
			ZF w/o $\Sigma \Delta$   & ZF in Section~\ref{sec:precoding} w/o $\Sigma \Delta$  & \eqref{eq:no_overloading} & all & w/o $\Sigma \Delta$ &  \\ \cline{1-5}
			ZF w/o distortion  & ZF in Section~\ref{sec:precoding} w/o distortion & \eqref{eq:no_overloading} & none  & w/o $\Sigma \Delta$ &  \\ \cline{1-5}
			ZF-TP  & total power constrained ZF & \eqref{eq:ZF_TPC} & except the last antenna  & w/o $\Sigma \Delta$ & \\ \cline{1-5}
			%ZF w/o distortion & average power constrained ZF w/o distortion &  $\mathbb{E}[\|\bx_m\|_2^2] = N  r_{\rm max}^2$ & none  & w/o $\Sigma \Delta$ &  \\ \cline{1-5}
			ZF-BO  & ZF w/ power back-off & \eqref{eq:const_1db}  & except the last antenna  & w/o $\Sigma \Delta$ &  \\ \hline
			\hline
			$\Sigma \Delta$ SLP  & SLP in Section~\ref{sec:precoding}  w/ $\Sigma \Delta$ & \eqref{eq:no_overloading} & all  & w/ $\Sigma \Delta$ & \multirow{4}{*}{\thead{\eqref{eq:SLP} w/ specific $\setX$,\\ ADMM in~\eqref{eq:admm}}} \\ \cline{1-5}
			T$\Sigma \Delta$ SLP   & SLP in Section~\ref{sec:precoding} w/ T$\Sigma \Delta$  & \eqref{eq:no_overloading}  & except the last antenna  & w/ $\Sigma \Delta$ &  \\ \cline{1-5}
			%SLP  & SLP in Section~\ref{sec:application}-\ref{sec:precoding} w/o $\Sigma \Delta$ & \eqref{eq:no_overloading} & except the last antenna  & w/o  $\Sigma \Delta$ &  \\ \cline{1-5}
			SLP w/o distortion & SLP in Section~\ref{sec:precoding} w/o distortion & \eqref{eq:no_overloading} & none  & w/o  $\Sigma \Delta$ &  \\ \cline{1-5}
			SLP-BO  & SLP w/ power back-off &  \eqref{eq:const_1db}  & except the last antenna & w/o $\Sigma \Delta$ &   \\ \hline
		\end{tabular}
	}
	%\caption{Summary of the tested precoding schemes.}\label{tab_schemes}
\end{table*}

For clarity, we summarize all the tested precoding schemes in Table~\ref{tab_schemes}.
Three benchmark precoding schemes are considered, namely, ZF with total power constraints (ZF-TP), ZF with power back-off (ZF-BO) and SLP with power back-off (SLP-BO).
The ZF-TP scheme takes the ZF form in~\eqref{eq:zf} and the normalization factor $\Gamma$ is chosen to ensure the total power constraint
\begin{equation}\label{eq:ZF_TPC}
	\Exp[\|\bX\|_F^2] = NMr_{\rm max}^2.
\end{equation}
The ZF-BO scheme also takes the form~\eqref{eq:zf}, but the $\Gamma$ is chosen to satisfy the constraint in \eqref{eq:1db}, such that the PAs will operate in the linear regions.
It follows from~\eqref{eq:OFDM_sig} that the constraint in \eqref{eq:1db} is satisfied if and only if
\begin{equation}\label{eq:const_1db}
	\bX \in \setX_{\rm 1dB} = \{ \bX \in \Cbb^{N \times M} ~\vert~  \|\bX\|_{\rm max} \le r_{\rm 1dB} \},
\end{equation}
and thus the $\Gamma$ is given by
\[
\Gamma = \frac{\| [\check\bH_0^\dagger\bs_0,\dots,\check\bH_{M_s-1}^\dagger\bs_{M_s-1}] \bF_s^\Tsf \|_{\rm max} }{r_{\rm 1dB}}.
\]
Following the same spirit, the SLP-BO scheme is formulated as problem~\eqref{eq:SLP} with $\setX = \setX_{\rm 1dB}$, which is solved by the ADMM algorithm in~\eqref{eq:admm}.
In addition, we consider the ZF scheme and the SLP scheme without PA distortions (and  without $\Sigma \Delta$ modulation); the ZF scheme with PA distortions but without $\Sigma \Delta$ modulation is also included to show the impact of PA distortion.
%The reason of including these schemes is to verify the PA distortion mitigation effect of our presented $\Sigma \Delta$ modulator.



The settings of the ADMM algorithm in \eqref{eq:admm} are as follows.
The initialization for the primal variables $(\bm \beta^0, \bZ^0, \bX^0)$ is chosen as the ZF scheme in Section~\ref{sec:precoding}.
The initialization for the dual variables is $\bm \Lambda^0=\bzero$.
The penalty parameter is fixed as $\rho=500$.
The algorithm is terminated when $|F(\bm \beta^{k+1}, \bm \bZ^{k+1})-F(\bm \beta^{k}, \bm \bZ^{k})| \le 10^{-3}F(\bm \beta^{k}, \bm \bZ^{k})$ and $\|\bX^{k+1} - \bZ^{k+1} \bF_s^\Tsf\|_F^2 \le 10^{-3}$, or when the number of ADMM iterations exceeds $30$.
The APG method in \eqref{eq:apg} for subproblem~\eqref{eq:admm_pro2} is initialized with the previous iterate $(\bm \beta^k, \bZ^k)$ and stops when $\|\bm \beta^{l+1}-\bm \beta^l\|_2^2 +  \|\bZ^{l+1}-\bZ^l\|_F^2\le 10^{-6}$ or when the number of APG iterations exceeds $50$.

\begin{figure}[t]
	\centering
	\subfigure[ZF w/o distortion]{
		\includegraphics[width=0.4\linewidth]{sim_scatter_zf_64_10_1}}
	\subfigure[ZF w/o $\Sigma \Delta$]{
		\includegraphics[width=0.4\linewidth]{sim_scatter_zf_64_10_2}}\\
	\subfigure[$\Sigma \Delta$ ZF]{
		\includegraphics[width=0.4\linewidth]{sim_scatter_zf_64_10_3}}
	\subfigure[T$\Sigma \Delta$ ZF]{
		\includegraphics[width=0.4\linewidth]{sim_scatter_zf_64_10_4}}
	\caption{IQ scatter plots. ${(N,K) = (64,10)}$.}
	\label{fig_scatter_zf_64_10}
\end{figure}


\begin{figure}[t]
	\centering
	\subfigure[ZF w/o distortion]{
		\includegraphics[width=0.4\linewidth]{sim_scatter_zf_56_10_1}}
	\subfigure[ZF w/o $\Sigma \Delta$]{
		\includegraphics[width=0.4\linewidth]{sim_scatter_zf_56_10_2}}\\
	\subfigure[$\Sigma \Delta$ ZF]{
		\includegraphics[width=0.4\linewidth]{sim_scatter_zf_56_10_3}}
	\subfigure[T$\Sigma \Delta$ ZF]{
		\includegraphics[width=0.4\linewidth]{sim_scatter_zf_56_10_4}}
	\caption{IQ scatter plots. ${(N,K) = (56,10)}$.}
	\label{fig_scatter_zf_56_10}
\end{figure}


\subsection{IQ Scatter Plots}

We begin with presenting the in-phase quadrature-phase (IQ) scatter plots.
Fig.~\ref{fig_scatter_zf_64_10} shows the results of the ZF schemes for $(N,K) = (64,10)$.
The red dots represent the constellation points and the blue dots represent the background noise-free received signals at the first subcarrier, i.e., $r_{i,p}/\beta_i$ in~\eqref{eq:dec} with $p=1$ and without background noise.
% (cf. $r_{i,v}/\beta_i$ in~\eqref{eq:dec} with $v=1$ and without background noise).
The dashed lines are the decision boundaries.
We fix one channel and overlay the received signals for $1000$ OFDM blocks.
Comparing ZF w/o distortion and ZF w/o $\Sigma \Delta$, we see that the PA distortions can have a great impact on the received signals.
Comparing ZF w/o $\Sigma \Delta$ and $\Sigma \Delta$ ZF, we see that the spatial $\Sigma \Delta$ approach can help mitigate the PA distortion effect.
Comparing $\Sigma \Delta$ ZF and T$\Sigma \Delta$ ZF, we see that the mitigation effect of the spatial $\Sigma \Delta$ approach can be enhanced by employing a linear PA at the last antenna.


In Fig.~\ref{fig_scatter_zf_56_10}, we reduce the number of transmit antennas from $N=64$ to $N=56$ and show the IQ scatter plots.
In this case, T$\Sigma \Delta$ ZF still works well, but $\Sigma \Delta$ ZF fails to work.
This indicates that when the number of transmit antennas is relatively small, the PA distortion at the last antenna can have a notable effect on the received symbols.














%\begin{figure}%[ht!]
%	\centering
%	\includegraphics[width=1\linewidth]{./Figs/sim_spectrum_zf_32_8.eps}
%	\caption{\small Angular power spectrums of the PA distortions. $(N,K) = (32,8)$.}
%	\label{fig_spectrum_zf}
%\end{figure}

%\begin{figure*}%[htb!]
%	\centering
%	\begin{subfigure}[b]{0.49\linewidth} \includegraphics[width=1\linewidth]{./Figs/sim_ber_zf_16_4.eps}
%		\caption{BERs of the ZF schemes. $(N,K) = (16,4)$.}\label{fig_ber_zf_16_4}
%	\end{subfigure}
%	\begin{subfigure}[b]{0.49\linewidth}	\includegraphics[width=1\linewidth]{./Figs/sim_ber_slp_16_4.eps}
%		\caption{BERs of the SLP schemes. $(N,K) = (16,4)$.}\label{fig_ber_slp_16_4}
%	\end{subfigure}\\
%	\begin{subfigure}[b]{0.49\linewidth} \includegraphics[width=1\linewidth]{./Figs/sim_ber_zf_32_5.eps}
%		\caption{BERs of the ZF schemes. $(N,K) = (32,5)$.}\label{fig_ber_zf_32_5}
%	\end{subfigure}
%	\begin{subfigure}[b]{0.49\linewidth}	\includegraphics[width=1\linewidth]{./Figs/sim_ber_slp_32_5.eps}
%		\caption{BERs of the SLP schemes. $(N,K) = (32,5)$.}\label{fig_ber_slp_32_5}
%	\end{subfigure}\\
%	\begin{subfigure}[b]{0.49\linewidth} \includegraphics[width=1\linewidth]{./Figs/sim_ber_zf_64_6.eps}
%		\caption{BERs of the ZF schemes. $(N,K) = (64,6)$.}\label{fig_ber_zf_64_6}
%	\end{subfigure}
%	\begin{subfigure}[b]{0.49\linewidth}	\includegraphics[width=1\linewidth]{./Figs/sim_ber_slp_64_6.eps}
%		\caption{BERs of the SLP schemes. $(N,K) = (64,6)$.}\label{fig_ber_slp_64_6}
%	\end{subfigure}
%	\caption{
%		BERs of various precoding schemes. $d=\lambda/8$.
%	}\label{fig_ber}
%\end{figure*}



\begin{figure*}
\centering
\subfigure[BERs of the ZF schemes. $(N,K) = (16,4)$.]{
\includegraphics[width=0.45\textwidth]{sim_ber_zf_16_4}\label{fig_ber_zf_16_4}}
\subfigure[BERs of the SLP schemes. $(N,K) = (16,4)$.]{
\includegraphics[width=0.45\textwidth]{sim_ber_slp_16_4}\label{fig_ber_slp_16_4}}
\subfigure[BERs of the ZF schemes. $(N,K) = (32,6)$.]{
\includegraphics[width=0.45\textwidth]{sim_ber_zf_32_6}\label{fig_ber_zf_32_6}}
\subfigure[BERs of the SLP schemes. $(N,K) = (32,6)$.]{
\includegraphics[width=0.45\textwidth]{sim_ber_slp_32_6}\label{fig_ber_slp_32_6}}
\subfigure[BERs of the ZF schemes. $(N,K) = (64,8)$.]{
\includegraphics[width=0.45\textwidth]{sim_ber_zf_64_8}\label{fig_ber_zf_64_8}}
\subfigure[BERs of the SLP schemes. $(N,K) = (64,8)$.]{
\includegraphics[width=0.45\textwidth]{sim_ber_slp_64_8}\label{fig_ber_slp_64_8}}
\caption{BERs under different system settings.}
\label{fig_ber}
\end{figure*}



%\subsection{Angular Power Spectrums}
%
%Secondly, to study the PA distortion power density in the angular domain, we evaluate the angular power spectrums.
%To put it into context,
%we define the angular power spectrum of the PA distortions as
%\[
%\varrho(\theta) = \int_{-T_{\sf CP}}^T |\bm a(\theta)^T (\bu(t)-A\bx(t))|^2 dt,
%\]
%where
%$\ba(\theta)$ is the angular response defined in~\eqref{eq:angular_response};
%$\bu(t)$ is the PA output signal;
%$\bx(t)$ is the modulator input signal if $\Sigma \Delta$ modulator is applied, or the PA input signal otherwise.
%
%
%
%Fig.~\ref{fig_spectrum_zf} shows the results of T$\Sigma \Delta$ nZF and nZF for different inter-antenna spacings $d$.
%We set $(N,K) = (32,8)$.
%We observe that the PA distortions of T$\Sigma \Delta$ nZF exhibit high-pass signals in the angular domain and the distortion power is significantly lower than that of nZF within the angular sector of interest, i.e., $\theta \in [-30^\circ,30^\circ]$.
%We also observe that compared with nZF, the PA distortions of T$\Sigma \Delta$ nZF tend to be more suppressed as $|\theta|$ and/or $d$ decreases.
%The above observations are in agreement with the discussions in Section~\ref{sec:sd_all}.


\begin{figure*}
	\centering
	\subfigure[256-QAM.]{
		\includegraphics[width=0.45\linewidth]{sim_ber_256QAM}}
	\subfigure[1024-QAM.]{
		\includegraphics[width=0.45\linewidth]{sim_ber_1024QAM}}
	\caption{BERs for high-order QAM. ${(N,K) = (56,8)}$.}
	\label{fig_ber_high_QAM}
\end{figure*}

\begin{figure*}
	\centering
	\subfigure[$\varphi = 1.0$]{
		\includegraphics[width=0.45\linewidth]{sim_sspa_rho10}}
	\subfigure[$\varphi = 1.5$]{
		\includegraphics[width=0.45\linewidth]{sim_sspa_rho15}}
	\subfigure[$\varphi = 2.0$]{
		\includegraphics[width=0.45\linewidth]{sim_sspa_rho20}}
	\caption{BERs for different ${\varphi}$. ${(N,K) = (32,5)}$, 256-QAM.}
	\label{fig_ber_sspa}
\end{figure*}

\begin{figure}
	\centering
	\subfigure[64-QAM.]{
		\includegraphics[width=0.45\linewidth]{sim_twta_64QAM}}
	\subfigure[256-QAM.]{
		\includegraphics[width=0.45\linewidth]{sim_twta_256QAM}}
	\caption{BERs under the TWTA model. ${(N,K) = (64,10)}$.}
	\label{fig_ber_twta}
\end{figure}





\subsection{Bit Error Rates}


Now we turn to the bit error rate (BER) results.
Fig.~\ref{fig_ber} shows the BERs of the various precoding schemes under different problem sizes $(N,K)$.
Let us first focus on the case of the small problem size $(N,K)=(16,4)$.
For the ZF schemes as shown in Fig.~\ref{fig_ber_zf_16_4},
we observe that T$\Sigma\Delta$ ZF offers the most promising performance;
specifically, it can achieve similar BER performance as ZF w/o distortion and meanwhile outperform ZF-BO, ZF w/o $\Sigma \Delta$ and $\Sigma \Delta$ ZF.
%it outperforms ZF-BO, ZF and $\Sigma \Delta$ ZF, and the performance gap increases as $1/\sigma_v^2$ increases.
However, $\Sigma \Delta$ ZF suffers from the error floor effect, mainly due to the PA distortion at the last antenna.
For the SLP schemes as shown in Fig.~\ref{fig_ber_slp_16_4}, we observe similar performance behaviors as those in Fig.~\ref{fig_ber_zf_16_4}.
Moreover, T$\Sigma \Delta$ SLP exhibits better performance than T$\Sigma \Delta$ ZF.
But it should be noted that compared with ZF, SLP demands higher computational costs for calling the ADMM algorithm.


%it is seen that the PA distortions can have a significant influence for the ZF precoding without back-off.
%Applying power back-off on ZF, which refers to ZF-BO in the legend, is effective to alleviate the performance degradation caused by the PA distortions.
%Its performance can be better than $\Sigma\Delta$ nZF in the high SNR regime.
%The most promising solution is T$\Sigma\Delta$ nZF, which shows superior BER performance, mainly due to its strong distortion suppression effect.
%Comparing $\Sigma\Delta$ nZF and T$\Sigma\Delta$ nZF, it is seen that the distortion at the last antenna $q_N(t)$ can have nonnegligible influence on the BER performance at the high SNR regime.
%For the SLP schemes as shown in Fig.~\ref{fig_ber}(b), we observe similar performance behaviors  with different strategies including back-off, $\Sigma\Delta$ and T$\Sigma \Delta$.
%Moreover, the advanced  T$\Sigma \Delta$ SLP scheme performs better than T$\Sigma \Delta$ nZF.
%However, it should also be noted that compared with ZF, SLP demands higher computational cost for calling the ADMM algorithm in~\eqref{eq:admm}.

%we see that T$\Sigma \Delta$ nZF outperforms ZF-BO, nZF and $\Sigma \Delta$ nZF, and the performance gap increases as $1/\sigma_v^2$ increases.
%However, in this case $\Sigma \Delta$ nZF suffers from the error floor effect mainly due to the PA distortions at the last antenna.
%For the SLP schemes as shown in Fig.~\ref{fig_ber}(b),
%we observe that T$\Sigma \Delta$ SLP delivers promising performance.
%Specifically, it achieves almost the same BERs as SLP w/o distortion, and meanwhile outperforms T$\Sigma \Delta$ nZF and other benchmark SLP schemes.
%The performance gain of T$\Sigma \Delta$ SLP over SLP demonstrates the effectiveness of the spatial $\Sigma \Delta$ modulator and the performance gain of T$\Sigma \Delta$ SLP over T$\Sigma \Delta$ nZF demonstrates the superiority of SLP over ZF.



%For Fig.~\ref{fig_ber}(a) with $(N,K) = (64,6)$, we observe similar BER performances of $\Sigma \Delta$ nZF, T$\Sigma \Delta$ nZF, nZF and nZF w/o distortion.
%We empirically find that as $N$ increases, the NZF schemes
%tend to have larger QAM constellation scalings $\beta_i$'s, which means that the constellation points are more separated and intuitively it is more unlikely for the PA distortions to cause additional bit errors.
%Therefore, under such settings, the PA distortions may have little impact on the BERs and the $\Sigma \Delta$ modulator cannot help to improve the BERs.


%Fig.~\ref{fig_ber}(a) shows the BERs of the ZF schemes for $(N,K) = (96,10)$ and $d=\lambda/8$.
%We see that E$\Sigma \Delta$ ZF outperforms ZF-BO, NZF and $\Sigma \Delta$ ZF, and the performance gap increases as $1/\sigma_v^2$ increases.
%In this case, $\Sigma \Delta$ ZF suffers from the error floor effect mainly due to the PA distortion at the last antenna.
%Fig.~\ref{fig_ber}(b) shows the BERs of the SLP schemes under the same settings.
%E$\Sigma \Delta$ SLP is seen to deliver promising performance.
%Specifically, it achieves almost the same BERs as SLP w/o distortion, and meanwhile outperforms E$\Sigma \Delta$ ZF and other benchmark SLP schemes.
%The performance gain of E$\Sigma \Delta$ SLP over SLP demonstrates the effectiveness of the $\Sigma \Delta$ modulator and the performance gain of E$\Sigma \Delta$ SLP over E$\Sigma \Delta$ ZF demonstrates the superiority of SLP over ZF.




In Fig.~\ref{fig_ber_zf_32_6}--Fig.~\ref{fig_ber_slp_64_8}, we show the BER performances for larger problem sizes, namely $(N,K)=(32,6)$ and $(N,K)=(64,8)$.
We see from Fig.~\ref{fig_ber_zf_32_6} and Fig.~\ref{fig_ber_zf_64_8} that the performance gap between ZF w/o distortion and ZF w/o $\Sigma \Delta$ tends to vanish as the problem size increases.
This suggests that the PA distortions have less influence on the BER performance.
We empirically find that as the problem size increases, the ZF w/o $\Sigma \Delta$, $\Sigma \Delta$ ZF and T$\Sigma \Delta$ ZF  schemes
tend to have larger QAM constellation scaling factors $\beta_i$'s, which means that the constellation points are more separated and intuitively it is more unlikely for the PA distortions to cause additional bit errors.
Also, from Fig.~\ref{fig_ber_slp_64_8}, we see that the performance gain of T$\Sigma \Delta$ SLP over T$\Sigma \Delta$ ZF can be large for large problem sizes.


%the performance gap between T$\Sigma \Delta$ nZF and nZF tends to decrease as the problem size increases and when $(N,K)=(64,6)$, the two schemes present almost identical BER performance.
%We empirically find that as $N$ increases, the nZF schemes
%tend to have larger QAM constellation scalings $\beta_i$'s, which means that the constellation points are more separated and intuitively it is more unlikely for the PA distortions to cause additional bit errors.
%Therefore, when $N$ is large, the PA distortions may have little impact on the BERs and the $\Sigma \Delta$ modulator cannot help to improve the BER performance.
%We observe from Fig.~\ref{fig_ber}(d) and Fig.~\ref{fig_ber}(f) that T$\Sigma \Delta$ SLP still works well for the larger problem sizes.
%Also, for $(N,K)=(64,4)$, T$\Sigma \Delta$ SLP and $\Sigma \Delta$ SLP are seen to share nearly identical BER performance, which suggests that in such case the PA distortions at the last antenna cannot have much influence on the BERs and there is no need to employ linear PAs at the last antenna.






Next, we consider  higher-order QAM constellations.
Fig.~\ref{fig_ber_high_QAM} shows the BERs for 256-QAM and 1024-QAM constellations under $(N,K) = (56,8)$.
It is seen that both T$\Sigma \Delta$ ZF and T$\Sigma \Delta$ SLP yield good performances, especially when $1/\sigma_v^2$ is large.
Also, compared to the benchmark schemes, T$\Sigma \Delta$ ZF and T$\Sigma \Delta$ SLP tend to be more advantageous for larger QAM sizes, as intuitively higher-order QAM constellations are more vulnerable to distortions and noise.




%\begin{figure}%[H]%[htb!]
%	\centering
%	\begin{subfigure}[b]{1\linewidth} \includegraphics[width=1\linewidth]{./Figs/sim_ber_sspa_rho11.eps}
%		\caption{$\varphi = 1.1$.}
%	\end{subfigure}
%	\begin{subfigure}[b]{1\linewidth}	\includegraphics[width=1\linewidth]{./Figs/sim_ber_sspa_rho16.eps}
%		\caption{$\varphi = 1.6$.}
%	\end{subfigure}
%	\begin{subfigure}[b]{1\linewidth}	\includegraphics[width=1\linewidth]{./Figs/sim_ber_sspa_rho20.eps}
%		\caption{$\varphi = 2.0$.}
%	\end{subfigure}
%	\caption{
%		BERs under different $\varphi$. $(N,K) = (56,8)$, $d=\lambda/8$, 256-QAM.
%	}\label{fig_ber_sspa}
%\end{figure}




%\begin{figure}%[htb!]
%	\centering
%	\begin{subfigure}[b]{1\linewidth} \includegraphics[width=1\linewidth]{./Figs/sim_ber_twta_64QAM.eps}
%		\caption{64-QAM.}
%	\end{subfigure}
%	\begin{subfigure}[b]{1\linewidth}	\includegraphics[width=1\linewidth]{./Figs/sim_ber_twta_256QAM.eps}
%		\caption{256-QAM.}
%	\end{subfigure}
%	\caption{
%		BERs under the TWTA model. $(N,K) = (56,8)$, $d=\lambda/8$.
%	}\label{fig_ber_twta}
%\end{figure}





\subsection{Performance under Other PA Model Parameters or PA Models}

Our developed $\Sigma \Delta$ precoding schemes can work for a variety of PA models.
We first evaluate the performance
%{\blue It is also interesting to evaluate the performance
under other parameters of the modified Rapp model in \eqref{eq:rapp}.
In particular, we fix $A=16$, $r_{\rm max}=0.1187$ and $B=0$, which corresponds to the case without the phase distortion ($g_p(r)=0$), and vary the parameter $\varphi$.
Note that in such case,  larger $\varphi$ means lower PA nonlinearities
and the case with $\varphi \to \infty$ corresponds to the ideal PA model in~\eqref{eq:ideal_pa}.
Fig.~\ref{fig_ber_sspa} presents the BERs for different $\varphi$.
We consider $(N,K) = (32,5)$ and 256-QAM constellation.
It is observed that T$\Sigma \Delta$ SLP (respectively T$\Sigma \Delta$ ZF) outperforms SLP-BO (respectively ZF-BO), and the performance gap reduces with increasing $\varphi$, or with weaker PA nonlinearities.

%This subsection is devoted to testing the BER performance under different PA model parameters or PA models.
%First, we evaluate the BERs under other model parameters of the modified Rapp model.
%In particular, we fix $A=16$, $r_{\rm max}=0.1187$ and $B=0$, which corresponds to the case without the phase distortions ($g_p(r)=0$), and vary the parameter $\varphi$.
%% evaluate the BER performances under different $\varphi$.
%Note that in such case,  larger $\varphi$ means lower PA nonlinearities and the case with $\varphi \to \infty$ corresponds to the ideal PA model.
%Fig.~\ref{fig_ber_sspa} presents the BERs of the ZF schemes for different $\varphi$.
%We consider $(N,K) = (56,8)$, $d=\lambda/8$ and 256-QAM constellation.
%It is observed that T$\Sigma \Delta$ nZF generally outperforms the other ZF schemes with nonlinear distortions, and the performance gap reduces with increasing $\varphi$, or with weaker PA nonlinearites.

Finally, we test the precoding schemes under other PA models.
Specifically, we consider the TWTA model with the following AM-AM and AM-PM conversions
\begin{equation*}
g_a(r) = \frac{Ar}{1+\frac{1}{4}(r/r_{\rm max})^2}, ~
g_p(r) = \frac{\pi}{12} \frac{(r/r_{\rm max})^2}{1+\frac{1}{4}(r/r_{\rm max})^2} ({\rm rad}).
\end{equation*}
Fig.~\ref{fig_ber_twta} shows the BERs for 64-QAM and 256-QAM constellations under the TWTA model with $A=16$ and $r_{\rm max}=0.1187$.
We set $(N,K) = (64,10)$.
It is observed that the T$\Sigma \Delta$ ZF and T$\Sigma \Delta$ SLP schemes can still provide reasonably good performances.



\section{Conclusion}

To conclude, we proposed a spatial $\Sigma \Delta$ modulation concept for
%presented a spatial $\Sigma \Delta$ approach to
PA nonlinear distortion mitigation in the massive MIMO downlink scenario.
While $\Sigma \Delta$ modulation has been used to shape and mitigate quantization noise in temporal DACs/ADCs for decades, its adaptation to mitigate PA distortion in space is new to the best of our knowledge.
Our presented approach can effectively mitigate the PA distortion effects at the user side, under the assumption of a ULA at the BS operating in a low-pass angular sector and using a small inter-antenna spacing.
% when we apply directional transmit antennas serving a low-pass angular sector.
%As the major advantage, our approach does not require exact PA transfer knowledge at the BS (as some existing approaches do).
%and could have lower hardware complexity than the existing DPD method.
%is much simpler to implement than the widely-used DPD method and does not require exact PA transfer knowledge at the BS (as many existing approaches do).
We
%demonstrated the potential of the spatial $\Sigma \Delta$ approach by
applied the spatial $\Sigma \Delta$ approach to the multi-user massive MIMO-OFDM downlink scenario
and custom designed a ZF scheme and an SLP scheme for the precoding design.
Simulation results revealed that the developed $\Sigma \Delta$ precoding schemes can provide promising performance in PA distortion effect suppression.
%are effective in mitigating the PA distortions.
%showed the effectiveness of the developed $\Sigma \Delta$ precoding schemes.



\section*{Appendix: Proof of Fact~\ref{Fac:no_overloading}}


%\subsection{PROOF OF FACT~\ref{Fac:no_overloading}}

%\label{sec:app1}

We prove Fact~\ref{Fac:no_overloading} by induction.
When $n=1$, it is shown that
\begin{align}
	|b_1(t)| &= |{x}_1(t)|  \le \chi - \psi \le \chi, \label{eq:app1_1} \\
	|q_1(t)| &= |G(b_1(t))/A-b_1(t)| \le \psi, \label{eq:app1_2}
\end{align}
where \eqref{eq:app1_2} is due to \eqref{eq:app1_1} and the definition of $\psi$ in~\eqref{eq:C}.
When $n \ge 2$, suppose that $|b_{n-1}(t)| \le \chi$ and $|q_{n-1}(t)| \le \psi$.
Then it holds that
\begin{equation}\label{eq:app1_4}
	\begin{aligned}
		|b_{n}(t)| &= |{x}_n(t) - q_{n-1}(t)| \\
		&\le |{x}_n(t)| + |q_{n-1}(t)| \\
		&\le \chi.
	\end{aligned}
\end{equation}
%where \eqref{eq:app1_3} is due to the fact that $|x+y| \le |x|+|y|$ for all $x,y\in\Cbb$.
By \eqref{eq:app1_4} and \eqref{eq:C}, we have
\[
|q_n(t)| = |G(b_n(t))/A- b_n(t)| \le \psi.
\]
Therefore, $|b_{n}(t)| \le  \chi$ and $|q_{n}(t)| \le  \psi$ hold for all $n$.




\bibliographystyle{IEEEtran}
%\bibliography{refs}
% Template for ASRU-2021 paper; to be used with:
%          spconf.sty  - ICASSP/ICIP LaTeX style file, and
%          IEEEbib.bst - IEEE bibliography style file.
% --------------------------------------------------------------------------
\documentclass{article}
\usepackage{spconf,amsmath,graphicx}

\usepackage[table]{xcolor}
\usepackage{amssymb}
\usepackage{multirow}

% Example definitions.
% --------------------
\def\x{{\mathbf x}}
\def\L{{\cal L}}

% Title.
% ------
\title{Speaker conditioning of acoustic models using affine transformation for multi-speaker speech recognition}
%
% Single address.
% ---------------
\name{Midia Yousefi, John H.L. Hansen}
\address{  Center for Robust Speech Systems (CRSS), Erik Jonsson School of Engineering,\\
University of Texas at Dallas, Richardson, Texas, USA}
%
% For example:
% ------------
%\address{School\\
%	Department\\
%	Address}
%
% Two addresses (uncomment and modify for two-address case).
% ----------------------------------------------------------
%\twoauthors
%  {A. Author-one, B. Author-two\sthanks{Thanks to XYZ agency for funding.}}
%	{School A-B\\
%	Department A-B\\
%	Address A-B}
%  {C. Author-three, D. Author-four\sthanks{The fourth author performed the work
%	while at ...}}
%	{School C-D\\
%	Department C-D\\
%	Address C-D}
%
\begin{document}
%\ninept
%
\maketitle
%
\begin{abstract}

This study addresses the problem of single-channel Automatic Speech Recognition of a target speaker within an overlap speech scenario. In the proposed method, the hidden representations in the acoustic model are modulated by speaker auxiliary  information to recognize only the desired speaker. Affine transformation layers are inserted into the acoustic model network to integrate speaker information with the acoustic features. The speaker conditioning process allows the acoustic model to perform computation in the context of target-speaker auxiliary information. The proposed speaker conditioning method is a general approach and can be applied to any acoustic model architecture. Here, we employ speaker conditioning on a ResNet acoustic model. Experiments on the WSJ corpus show that the proposed speaker conditioning method is an effective solution to fuse speaker auxiliary information with acoustic features for multi-speaker speech recognition, achieving +9\% and +20\% relative WER reduction for clean and overlap speech scenarios, respectively, compared to the original ResNet acoustic model baseline. 

%The affine transformation is parametrized by shifting and scaling coefficients generated based on speaker-specific embedding. 
\end{abstract}
%
\begin{keywords}
 Affine transformation, overlap speech recognition, feature-wise linear modulation, multi-speaker recognition, acoustic modeling
\end{keywords}
%
\section{Introduction}
\label{sec:intro}

%The introduction of deep learning has lead to significant performance improvements in recent single-speaker automatic speech recognition systems \cite{yu2016automatic,li2015robust,vaswani2017attention,chiu2018state}. 

Multi-talker speech recognition is focused on recognizing individual speech sources from overlap speech, and is one main challenge for current ASR systems \cite{chang2020end,watanabe2020chime,yousefi21_interspeech, yousefi2018assessing,barker2018fifth,qian2018past,mirsamadi2019multi,yousefi2020block}. 
%In applications where multi-channel speech recordings are available, spatial information derived from the inter-channel differences \cite{chang2019mimo,chen2018multi,subramanian2021directional}, Direction of Arrival (DoF) \cite{dey2018direction,seltzer2003microphone}, and other beamforming techniques \cite{seltzer2004likelihood,yoshioka2018multi,cho2021convolutional} can help distinguish between speech sources, which makes the problem easier to solve. 
Current solutions for multi-speaker speech recognition  can be categorized into two main approaches: \emph{(i)} performing front-end speech processing based on separation on the overlap speech, then applying ASR to the separated speech signals \cite{boeddeker2018front,yousefi2016supervised,yousefi2020frame,deng2011front, narayanan2014investigation,mirsamadi2014multichannel, yousefi2019probabilistic}; or \emph{(ii)} skipping the explicit separation step and developing a multi-speaker speech recognition system directly using either hybrid \cite{kanda2019acoustic,weng2015deep,kanda2019simultaneous} or end-to-end \cite{seki2018purely,lu2021streaming} ASR frameworks. Recently, an end-to-end multi-speaker speech recognition system was proposed based on Transformers \cite{chang2020end}. This approach achieved considerable improvement at the expense of more computational cost for a reasonable temporal resolution. In another study \cite{chen2017progressive}, overlap speech was considered as a mismatch condition of the clean speech recognition scenario, and teacher-student training was employed for transfer learning from clean to overlap speech. The main drawback of this approach is requiring training sets with parallel clean and overlapped speech, which is difficult to collect in real-world applications \cite{denisov2019end}.
%Authors of \cite{delcroix2018single} proposed alternate SpeakerBeam configurations for target speaker extraction and recognition in single-channel recordings by joint training of two separate networks. However, as reported in \cite{vzmolikova2019speakerbeam} their proposed speaker-aware acoustic model is more effective for multi-channel recordings while it lags behind in single-channel scenario.
%former approaches such as Permutation Invariant Training (PIT) \cite{kolbaek2017multitalker} and Deep Clustering (DP) \cite{isik2016single} in single-channel scenario.
Recently, several studies \cite{kanda2019acoustic,wang2019end,subramanian2020far} have used speaker-specific embeddings to learn a frame-level mask for the target speaker which suppresses interfering speech. Although these approaches use the additional speaker-specific information to guide the ASR system, their main limitation is that they assume only one speaker is active in each Time-Frequency bin. 


To address the challenges of single-channel multi-speaker speech recognition, in this study, we focus on speaker conditioning of the Acoustic Model (AM) by performing an affine transformation.
%which integrates the speaker information with the acoustic features to improve ASR performance. 
In contrast to former approaches which employ speaker embedding to estimate speaker-specific masks, we propose to use speaker embedding to compute parameters of the affine transformation, allowing the acoustic model to conduct its computation in the context of the desired speaker auxiliary information. The proposed speaker conditioning method is a general approach and can be applied to any AM architecture. In this paper, we employ  speaker conditioning on a ResNet acoustic model in hybrid DNN-HMM setup. Experiments are performed on WSJ corpus, achieving +9\% and +20\% relative WER reduction for clean and overlap speech scenarios, compared to the original ResNet acoustic model. The contributions of this paper are threefold:
\begin{itemize}
    \item Proposing speaker conditioning of the ResNet acoustic model using an affine transformation.
    \item Comparing the proposed method with alternate feature-wise  acoustic model transformations such as conditional biasing and middle feature-map fusion.
    \item Evaluating the performance of the proposed speaker conditioned ASR trained on an alternate input feature called Wav2Vec representation. 
\end{itemize}


The remainder of this paper is organized as follows. In Sec.\ref{sec:sys}, the problem is outlined and proposed method described. Sec.\ref{sec:exp}, presents experiments and results. Finally the conclusions are discussed in Sec.\ref{sec:con}. 

\begin{figure*}[t]
\centering
\vspace{-0.9cm}
\includegraphics[width=\linewidth]{models.png}
\vspace{-0.7cm}
\caption{The proposed speaker conditioned ResNet18 acoustic model using Affine Transformation (AT) blocks.}
\label{fig:model}
\vspace{-0.2cm}
\end{figure*}


\section{Single-channel multi-speaker ASR}
\label{sec:sys}

Multi-speaker speech recognition can gain substantial improvement by deploying other sources of information such as speaker identity in addition to acoustic features \cite{denisov2019end, wang2019end}. However, designing an efficient method to fuse a combination of multiple sources (i.e., acoustic features and speaker embeddings) to obtain higher quality and improved information is still a challenging task. Additionally, capturing complex interactions between multiple sources should maintain a balanced compromise between model/network computational cost and performance. A popular method to address this problem is to use a feature-wise transformation \cite{perez2018film} which can model the complex relation between speaker-specific characteristics and acoustic features in a multi-speaker speech scenario to identify and recognize the desired speaker in the mixed speech recording. This transformation can be performed in several manners such as \emph{conditional biasing}, \emph{conditional scaling}, and \emph{conditional affine transformation}. In this section, we focus on a conditional affine transformation  which is a more general approach. The aforementioned conditional biasing and scaling are two specific examples of conditional affine transformation. 

\subsection{Conditional affine transformation} Affine Transformation (AT) influences the output of the acoustic model network by applying a linear modulation to the network's intermediate features. This modulation is parameterized by shifts and coefficients obtained based on speaker-specific embedding. Let $x$ be a context-expanded window of acoustic features for overlap speech, and $y_s$ be a phoneme label or a senone alignment (i.e., from GMM-HMM) for the target-speaker speech signal. DNN acoustic models are used estimate the posterior probability as:
\begin{equation}
   p(y_s|x, s) = DNN(x, z_s),
\end{equation}
where DNN is typically trained to maximize the log probability of the phoneme alignment or minimize the cross-entropy error, and $s$ is the target speaker with an x-vector \cite{snyder2018x} embedding $z_s$. In this study, the original ResNet18 model is considered as our baseline. Next, affine transformation layers are inserted into ResNet18 network to build the speaker conditioned acoustic model. The scale and bias factors of AT are  estimated by a two-layer fully connected network $h$ based on x-vector $z_s$ as:
\begin{equation}
    (\alpha_{i,c},\; \beta_{i,c}) = h(z_s)
\end{equation}
where  $i$ and $c$ refer to the $i$-th data sample in the minibatch, and the $c$-th channel feature map. Once $\alpha_{i,c}$ and $\beta_{i,c}$ are estimated, they are used to modulate the ResNet's intermediate activations $F_{i,c}$ as:
\begin{equation}
      AT(F_{i,c}^{l}| \alpha_{i,c}, \beta_{i,c}, F_{i,c}^{l-1}) = \alpha_{i,c} \odot  F_{i,c}^{l-1} + \beta_{i,c}
\end{equation}
where $AT$ and $l$ represent the Affine Transformation, and network's layer. The proposed speaker conditioned ResNet18 is shown in Fig.\ref{fig:model}.  The speaker embedding x-vector is submitted to the network $h$ to estimate a $[B, 1920]$ matrix which is $(\alpha_{i,c}$, $\beta_{i,c})$ pairs of AT layers. Each AT layer receives two inputs:  the previous layer output, and the $(\alpha_{i,c}$, $\beta_{i,c})$ pair. The dimension of $\alpha_{i,c}$ and $\beta_{i,c}$ is $[B, C]$ each. In the AT layer, each channel of the extracted feature map is scaled by $\alpha_{i,c}$ and shifted by $\beta_{i,c}$ to modulate the feature-map distribution of activations based on the target-speaker embedding.




\begin{table}[h]
\vspace{-0.3cm}
\caption{Comparing our ResNet18 acoustic model baseline with other approaches on WSJ (WER in \%).}
\vspace{-0.3cm}
\begin{minipage}{0.45\textwidth}
\centering
\begin{tabular}{@{\extracolsep{0.1pt}}lc c c} 
\\\hline 
\hline \\
\textit{System}  & \multicolumn{1}{c}{Dev-93} & \multicolumn{1}{c}{Eval-92} \\ 
\hline 
Lee et al. 2021 \cite{lee2021intermediate} &  12.0 & 9.9\\
Higuchi et al. 2020 \cite{higuchi2020mask} &15.4 &	12.1\\
Chi et al. 2020 \cite{chi2020align} & 13.7&	11.4\\
Rouhe et al. 2020 \cite{rouhe2020speaker} & 13.2 &	9.3\\
Sabour et al. 2018 \cite{sabour2018optimal} &- &	9.3\\
Borgholt et al. 2020 \cite{borgholt2020end} &- &	9.3\\
Park et al. 2019 \cite{lee2019simple} & - &	7.8\\
\hline
Our baseline & 12.1	& 7.9

\end{tabular} 
\end{minipage}
\vspace{-0.5cm}
\label{tab:base}
\end{table}











\begin{table*}[t]
\setlength{\extrarowheight}{4pt}
\begin{center}
\vspace{-0.4cm}
\caption{ WER of the proposed speaker conditioned ResNet18 acoustic model with Affine Transformation (AT) in different settings. Each experiment is repeated three times and the average WER is reported.}
%\vspace{-0.2cm}

\begin{tabular}{ |c|c|c|c|c|c|c|c|c|c| }
\cline{2-10}

\multicolumn{1}{c|}{} & \multicolumn{6}{c|}{Simulated overlap speech based on Dev-93} &\multicolumn{3}{c|}{Clean speech} \\ 
\cline{3-6} \cline{7-9} 
\hline
Acoustic model & 0dB & 5dB & 10dB & 15dB & 20dB & 25dB & Dev-93 & Eval-92 & Eval-93 \\
\hline
ResNet18 (baseline) & 65.06&58.29&47.03&34.72&24.36&17.62&12.14&7.92&10.81 \\
ResNet18 + AT (proposed) &63.83&	55.83&	43.30&	29.90&	20.34&	15.15&	11.50&	7.66&	9.64\\
\hline
ResNet18 + AT (bias=0) &63.69&	55.23&	43.57&	30.15&	19.95&	15.19&	11.75&	7.66&	9.57\\
ResNet18 + AT (scale=1)& 65.10& 57.66&	46.39&	32.79&	22.29&	16.65&	12.25&	7.91& 10.57\\
ResNet18 + AT (sigmoid(scale))& 64.63&	56.66&	45.33&	31.78&	21.680&	16.250&	11.993&	7.803&	10.263\\
%ResNet18 + AT (ReLU(scale))& 64.087&	56.197&	44.700&	31.077&	20.617&	15.670&	11.903&	7.700&	10.033\\
ResNet18 + AT (tanh(scale))& 64.53&	56.82&	45.27&	31.64&	21.19&	16.02&	11.99&	7.72&	9.97\\
\hline
ResNet18 + AT (Block1)& 63.68&	55.33&	\textbf{43.31}&29.44&	\textbf{19.51}&\textbf{14.67}&11.56&\textbf{7.49}&	\textbf{9.85}\\
ResNet18 + AT (Block 1-2)&\textbf{63.50}&	\textbf{55.19}&	43.60&	\textbf{29.33}&	19.79&	14.85&	\textbf{11.33}&	7.50&	9.88\\
ResNet18 + AT (Block 1-3)&63.51&	55.29&	43.35&	29.65&	19.87&	14.88&	11.49&	7.55&	9.93\\
ResNet18 + AT (Block 4)& 64.66&	57.66&	47.28&	33.86&	23.27&	16.54&	11.64&	8.19&	10.47\\
\hline
\end{tabular}
\vspace{-0.4cm}
\label{tab:at}
\end{center}
\end{table*}




 %Since this is a difficult optimization problem, we propose to integrate speaker embedding x-vector to leverage the additional information to model the phonemes as:
% \begin{equation}
%     p(y_s|x, s) = DNN_{AI}(x, z_s),
% \end{equation}
% where $DNN_{AI}$ is the Audio-Identity model and $s$ is the target speaker with x-vector embedding $z_s$. 








\section{Experiments and results}
\label{sec:exp}
In this section, we investigate the performance of the proposed speaker conditioning method presented in Fig.\ref{fig:model} on WSJ corpus. 
%Additionally the effect of $\alpha_{i,c}$ and $\beta_{i,c}$  on the final Word-Error-Rate (WER) are studied separately. Also, different scale ranges are considered for the scaling parameter $\alpha_{i,c}$ to understand its role in influencing the performance of the acoustic model. The speaker conditioned acoustic model is compared against several fusion methods such as conditional biasing fusion and middle fusion. Finally, the input MFCC acoustic features are replaced by noise-invariant Wav2Vec representation to train the speaker conditioned acoustic model to further boost the WER. 
In order to conduct the experiments, clean \textit{tr-si284} is used in the training phase for all acoustic models. We partitioned \textit{tr-si284} into a training set $(90\%)$ and a held-out cross-validation set $(10\%)$. ASR performance for different acoustic models are reported in terms of Word-Error-Rate (WER) on clean \textit{dev-93}, \textit{eval-93}, and \textit{eval-92}. Additionally, overlap speech is generated based on \textit{dev-93} by selecting random utterances from random speakers and adding them with Signal-to-Interference Ratio (SIR) ranging from $0$ to $25$dB with increments of $5$dB. The baseline acoustic model is ResNet18 with $3400$ output senones.  The  network parameters  are  updated  by  the  gradients  of the cross entropy loss using Stochastic Gradient Descent (SGD) optimizer with a momentum of $0.9$ and initial learning rate $0.01$. The training process is completed by performing early stopping \cite{zhang2016understanding}. The maximum number of epochs is set to $100$, batch size  $1024$ context-expanded frames; learning rate is decreased by $50\%$ if the cv loss improvement is less than $0.01$ for $3$ successive epochs. The early stopping is performed if no improvement is observed on the cv loss once the learning rate has decayed $6$ times. 13-dim MFCC computed over a $25ms$ window with $10ms$ shift with a $20$ frame context ($10$ frames on each side) is used for training the acoustic model. Consistent with the standard Kaldi recipe for WSJ, we use the trigram language models provided by LDC for WSJ data. In order to minimize the effect of parameter initialization on the acoustic model and final WER, we repeat each experiment three times with different initial parameters.

\begin{table*}[t]
\setlength{\extrarowheight}{4pt}
\begin{center}
\vspace{-0.6cm}
\caption{ WER of the proposed speaker conditioned ResNet18 based on Affine Transformation (AT) compared to other fusion techniques. Each experiment is repeated three times and the average WER is reported.}
\begin{tabular}{ |c|c|c|c|c|c|c|c|c|c| }

\cline{2-10}

\multicolumn{1}{c|}{} & \multicolumn{6}{c|}{Simulated overlap speech based on Dev-93} &\multicolumn{3}{c|}{Clean speech} \\ 
\cline{3-6} \cline{7-9} 
\hline
Signal-to-Interference Ratio & 0dB & 5dB & 10dB & 15dB & 20dB & 25dB & Dev-93 & Eval-92 & Eval-93 \\
\hline
ResNet18 + Conditional biasing  & 64.66&	57.20&	45.74&	32.75&	23.54&	17.87&	13.10&	8.44&	11.22\\
ResNet18 + Middle fusion & 63.94&	57.51&	47.81&	34.02&	23.19&	16.68&	11.81&	8.24&	10.64 \\
\textbf{ResNet18 + AT (proposed)}& \textbf{63.68}&	\textbf{55.33}&	\textbf{43.31}&\textbf{29.44}&	\textbf{19.51}&\textbf{14.67}&\textbf{11.56}&\textbf{7.49}&	\textbf{9.85}\\
\hline
\end{tabular}
\label{tab:com}
\end{center}
\end{table*}




\begin{figure*}
\centering
\begin{tabular}{cc}
\includegraphics[height=3.8cm,width=7cm]{base.png}&
\hspace{1.5cm}
\includegraphics[height=3.8cm,width=7cm]{at.png} 
\end{tabular}
\vspace{-0.2cm}
\label{figur}\caption{ WER of ResNet18 baseline and proposed ResNet18 + AT trained on MFCC and Wav2Vec input features.}
\vspace{-0.2cm}
\label{fig:wav2vec}
\end{figure*}


Performance of the ResNet18 baseline is compared with recent studies in Table \ref{tab:base}. The main purpose of this comparison is to ensure that our baseline achieves a competitive performance compared to recent studies, and it is seen we have a strong starting point for further developing our proposed speaker conditioning technique.  There are other approaches that leverage transfer learning, semi supervised learning, or more advanced language models to achieve further improvement. However, since we focus on speaker conditioning of acoustic model, we train our ResNet18 acoustic model only on \textit{tr-si284}, and use Kaldi for training the language model. The first row of Table \ref{tab:at} presents performance of the baseline on overlap speech, which is severely degraded. Therefore, we build on the ResNet18 acoustic model baseline and apply our Affine Transformation (AT) layers as depicted in Fig.\ref{fig:model}. The results for speaker conditioned ResNet using AT are presented in the second row of Table \ref{tab:at} which shows $+2\%$ relative improvement for severe overlap speech recordings (i.e., $0$dB) and an average of $+5\%$ relative improvement on clean test sets. Since AT effectively performs speaker-adaptation, the trained acoustic model is tuned to the target speaker, therefore, it achieves better performance even on the clean test sets. The maximum relative improvement is achieved for input SIR $20\%$ in which the level of overlap speech is neither too severe nor too easy for the acoustic model; therefore, the  target-speaker auxiliary information can be very helpful in improving performance.  

Moreover, the effect of $\alpha$ and $\beta$ is studied separately by setting $\alpha=1$ and $\beta=0$. The result in Table \ref{tab:at} manifest that the effectiveness of the conditional Affine Transformation can be mainly attributed to the scale coefficient rather than the shift parameter. Therefore, we further investigate the effect of  $\alpha$ by restricting its value to $(0,1)$ using Sigmoid, and $(-1,1)$ using $\tanh$ function. Nevertheless, the results reported in Table \ref{tab:at} reveal that unrestricted $\alpha$ achieves better performance which may be due to the flexibility it provides for the network to learn the range that best suits the data. So far, the AT layers have been applied to all ResNet18 blocks (each dashed rectangular in Fig. \ref{fig:model} is considered as a block). To find the best network depth in which AT layers are most effective, several experiments are conducted with AT only applied to specific individual blocks. Based on these experiments, the AT layers are most effective when applied only to the first block (block1), and least effective when only applied to the last block (block4). However, applying  AT layers to the first two blocks (block1-2) and the first three blocks (block 1-3) did not improve ASR performance, while it differently adds computational cost. To summarize our findings based on the experiments, unrestricted-scale Affine Transformation applied to the initial blocks of the ResNet18 acoustic model achieves the best overall results while simultaneously maintaining the lowest computational cost. 
%This acoustic model is highlighted in bold in Table \ref{tab:at}.

Next, the proposed method is compared with other speaker conditioning techniques in Table \ref{tab:com}. Conditional biasing refers to adding speaker information (x-vector) as a bias to the acoustic features in the first hidden layer. Middle fusion refers to adding the x-vector to the intermediate extracted feature map after the second block. Therefore, the intermediate feature map is conditioned before entering block 3 for extracting further higher-level features adapted to the target speaker. As shown in Table \ref{tab:com}, the proposed speaker conditioning based on Affine Transformation outperforms all other approaches in both clean and overlap speech scenarios. 

So far, the focus of this study has been on designing the acoustic model. However,  performance of the acoustic model can further improve by deploying more robust input features other than MFCC. In the final section, we evaluate the proposed method trained on noise-invariant Wav2Vec features \cite{schneider2019wav2vec}. Wav2Vec representation has been trained on large amounts of unlabeled audio data in an unsupervised manner. Fig. \ref{fig:wav2vec} (left) shows the WER of the baseline ResNet18 trained on MFCC and Wav2Vec features, which manifests the effectiveness of Wav2Vec in reducing the WER across all test sets in the absence of speaker auxiliary information. The highest improvement is achieved for overlap speech with SIR $15$dB, which is $+11\%$ absolute improvement in WER. Fig.\ref{fig:wav2vec} (right) depicts the WER of the proposed speaker conditioned ResNet18 trained on MFCC and Wav2Vec. Similar to the baseline, the speaker conditioned acoustic model benefits from the Wav2Vec features by achieving $+6\%$ absolute improvement in WER for SIR $15$dB. However, due to the availability of speaker information, the acoustic model is less sensitive to the robustness of the input acoustic features, and thus, the amount of improvement from Wav2Vec is less in the proposed speaker conditioned ResNet18 compared to the baseline. In conclusion, the WER across all test sets is improved by using the proposed speaker conditioned acoustic model trained on wav2Vec. For example, on the overlap speech test set with SIR $15$dB, the proposed ResNet18 with Affine Transformation trained on Wav2Vec gains +33\% relative (+11\% absolute) improvement in WER compared to the original ResNet trained on MFCC.


\section{Conclusion}
\label{sec:con}

In this study, we proposed a speaker conditioning method for acoustic modeling in multi-speaker speech recognition. In the proposed method, Affine Transformation layers are inserted into the acoustic model architecture to fuse speaker-specific information with the acoustic model. The proposed speaker conditioned acoustic model  was compared with  other fusion techniques such as early fusion of speaker embedding and middle feature-map fusion. Additionally, the performance of the proposed method was evaluated on alternate input features called Wav2Vec. The results on WSJ corpus clearly demonstrate that the proposed speaker conditioned acoustic model based on affine transformation achieves consistent WER improvement for clean and overlap speech scenarios. 


% Below is an example of how to insert images. Delete the ``\vspace'' line,
% uncomment the preceding line ``\centerline...'' and replace ``imageX.ps''
% with a suitable PostScript file name.
% -------------------------------------------------------------------------
% \begin{figure}[htb]

% \begin{minipage}[b]{1.0\linewidth}
%   \centering
%   \centerline{\includegraphics[width=8.5cm]{image1}}
% %  \vspace{2.0cm}
%   \centerline{(a) Result 1}\medskip
% \end{minipage}
% %
% \begin{minipage}[b]{.48\linewidth}
%   \centering
%   \centerline{\includegraphics[width=4.0cm]{image3}}
% %  \vspace{1.5cm}
%   \centerline{(b) Results 3}\medskip
% \end{minipage}
% \hfill
% \begin{minipage}[b]{0.48\linewidth}
%   \centering
%   \centerline{\includegraphics[width=4.0cm]{image4}}
% %  \vspace{1.5cm}
%   \centerline{(c) Result 4}\medskip
% \end{minipage}
% %
% \caption{Example of placing a figure with experimental results.}
% \label{fig:res}
% %
% \end{figure}


% To start a new column (but not a new page) and help balance the last-page
% column length use \vfill\pagebreak.
% -------------------------------------------------------------------------
%\vfill
%\pagebreak



% 

% References should be produced using the bibtex program from suitable
% BiBTeX files (here: strings, refs, manuals). The IEEEbib.bst bibliography
% style file from IEEE produces unsorted bibliography list.
% -------------------------------------------------------------------------
\bibliographystyle{IEEEbib}
%\bibliography{refs}
%S\bibliographystyle{IEEEtran}
\bibliography{refs}
\end{document}



\end{document}
