To put our results in a broader context, let us briefly discuss alternative methods for 
classical simulation of quantum circuits. Vector-based simulators~\cite{de2007massively,smelyanskiy2016qhipster,haner20170}
represent $n$-qubit quantum states by complex vectors of size $2^n$
stored in a classical memory.
The state vector is updated
upon application of each gate by performing sparse
matrix-vector multiplication. The memory footprint limits 
the method to small number of qubits. For example, 
H{\"a}ner and Steiger~\cite{haner20170}
reported a simulation of
quantum circuits with $n=45$ qubits and a few hundred gates 
using a  supercomputer with $0.5$ petabytes of memory.
In certain special cases 
the memory footprint can be reduced 
by recasting the simulation problem as a 
tensor network contraction~\cite{markov2008simulating,boixo2017simulation,aaronson2016complexity}.
Several tensor-based simulators have been developed~\cite{pednault2017breaking,li2018quantum,chen2018classical} 
for geometrically local  shallow quantum  circuits that include only nearest-neighbor
gates on a 2D grid of qubits~\cite{boixo2018characterizing}.
These methods enabled simulations of systems with more than $100$ qubits~\cite{chen2018classical}.
However, it is expected~\cite{alibaba} that for general (geometrically non-local) circuits 
of size $poly(n)$  the runtime of tensor-based simulators scales as $2^{n-o(n)}$.

In contrast, Clifford simulators described in the present paper are applicable to large-scale circuits
without any locality properties as long as the circuit is dominated by Clifford gates. 
This regime may be important for verification of first fault-tolerant quantum circuits
where  logical non-Clifford gates are expected to be scarce due to their high implementation
cost~\cite{fowler2013surface,jones2013low}.
Another advantage of Clifford simulators is their ability to sample the output
distribution of the circuit (as opposed to computing individual output amplitudes).
This is more close to what one would expect from the actual quantum computer. 
For example, a single run of the heuristic sum-over-Cliffords simulator
described in Section~\ref{heuristic} produces thousands of samples from the (approximate) output distribution. 
In contrast, a single run of a tensor-based simulator typically computes a single amplitude of the
output state.  Thus we believe that our techniques 
extend the reach of classical simulation algorithms complementing
the existing vector- or tensor-based simulators.

%PC:
A version of the sum-over-Cliffords simulator using the Metropolis sampling method is also publicly available
as part of \texttt{Qiskit-Aer}, the classical simulation framework of IBM's quantum
programming suite \texttt{Qiskit}~\cite{Qiskit}. This enables classical simulation and verification of quantum
circuits built in Qiskit on system sizes above $30$ qubits, which quickly become inaccessible with the
default vector-based method. This version also supports parallel processing over the stabilizer state decomposition,
which improves the performance of the Metropolis step.

%SBB:
Let us briefly comment on how simulators based on the stabilizer rank compare
with quasi-probability  methods~\cite{pashayan15,Delfosse15rebits,kocia2017discrete}.
The latter use a discrete Wigner function representation of quantum states
and Monte Carlo sampling
to approximate a given output probability of the target circuit with a small
additive error. Negativity of the Wigner function is an important parameter
that quantifies severity of the ``sign problem" associated with the Monte Carlo sampling.
The negativity also controls the runtime of quasi-probability methods. 
For example, the simulator proposed in~\cite{pashayan15}
has runtime $\epsilon^{-2} M^2$, where $M$ is the negativity and $\epsilon$
is the desired approximation error. In contrast to stabilizer rank simulators,
quasi-probability methods do not directly apply to stabilizer
operations on qubits since the latter are not known to have a non-negative Wigner function 
representation~\cite{Delfosse15rebits,karanjai2018contextuality}.
Furthermore, such methods are not well-suited for sampling the output
distribution since this task requires a small {\em multiplicative} error in 
approximating individual output probabilities. 

Our work leaves several  open questions. 
Since the efficiency of Clifford simulators hinges on the ability to find low-rank
stabilizer decompositions of multi-qubit magic states, 
improved techniques for finding such decompositions are of great interest. 
For example, consider a magic state $|\psi\ra=U|+\ra^{\otimes n}$, where 
$U$ is a diagonal circuit composed of $Z,CZ$, and $CCZ$ gates.
We anticipate that a low-rank exact stabilizer decomposition of $\psi$ can be 
found by computing the {\em transversal number}~\cite{alon1990transversal} of a suitable hypergraph describing
the placement of CCZ gates. Such low-rank decompositions may lead to more efficient
simulation algorithms for Clifford+CCZ circuits. We leave as an open question whether
the stabilizer extent $\xi(\psi)$ is multiplicative under tensor products for general states $\psi$.
Finally, it is of great interest to derive lower bounds on the stabilizer rank
of $n$-qubit magic states scaling exponentially with $n$. 





