In the previous Section we established upper bounds on the approximate stabilizer rank of a state $\psi$ which depend on the the squared $1$-norm $\|c\|_1^2$, where
\[
|\psi\rangle=\sum_{j}c_j |\phi_j\rangle,
\]
is a given stabilizer decomposition.   Recall that the stabilizer extent $\xi(\psi)$  denotes the minimum value of $|| c ||_1^2$ over all stabilizer decompositions of $\psi$. We find that $\xi$ is easier to work with than the approximate stabilizer rank. For any fixed $n$-qubit state $\psi$, $\xi(\psi)$ can be computed using a simple convex optimization program, although the size of this computation scales poorly with $n$. In this section we develop tools that allow us to efficiently compute $\xi(\psi)$ whenever $\psi$ is a tensor product of $1, 2$ and $3$ qubit states. In particular, we prove Proposition~\ref{multi} which establishes that $\xi$ is multiplicative for tensor products of $1$, $2$, and $3$-qubit states. 

In subsection~\ref{Sec_convex_dual} we use standard convex duality to give a characterization of $\xi$ in terms of the \textit{stabilizer fidelity}, defined as the maximum overlap with respect to the set of stabilizer states
\begin{equation}
F( \psi ) := \mathrm{max}_{\phi \in \mathrm{STAB}_n} |\bk{\psi}{\phi}|^2.
\end{equation}

As a consequence, multiplicativity of $\xi$ is directly related to multiplicativity of the stabilizer fidelity. In subsection~\ref{Sec_Fid_Multi} we give sufficient and necessary conditions for multiplicativity of the stabilizer fidelity. In particular, we define the class of \textit{stabilizer-aligned} states for which multiplicativity holds.  In subsection~\ref{Sec_when_stab_aligned} we investigate the class of stabilizer-aligned states and prove that all tensor products of $1,2$ and $3$ qubit states are stabilizer-aligned.  Finally, in section~\ref{Sec_Cstar_multi} we use these results to prove Proposition~\ref{multi}. 

\subsection{Convex duality}
\label{Sec_convex_dual}

Here we show that the optimization of $\xi(\psi)$ can be recast as a dual convex problem and we prove the following:
\begin{theorem}
	\label{thm:witness}
	For any $n$-qubit state $\psi$ we have
	\begin{equation}
	\xi(\psi)=\max_{\omega} \frac{|\bk{\psi}{\omega}|^2}{F(\omega)},
	\label{eq:witness}
	\end{equation}
	where the maximum is over all $n$-qubit states $\omega$. 
\end{theorem}
 Thus any $n$-qubit state $\omega$ can act as a witness to provide a lower bound on $\xi$ and, furthermore, there exists at least one optimal witness state $\omega_{\star}$ which achieves the maximum in Eq.~\eqref{eq:witness}. For example, choosing $\omega=\psi$, we get the lower bound 
\begin{equation}
\xi(\psi) \geq \frac{1}{F(\psi)}.
\label{eq:mirrorwitness}
\end{equation}
For Clifford magic states this lower bound is tight as stated in Proposition~\ref{thm:clifmagic}.  We remark that Thm.~\ref{thm:witness} is a special case of results found in the literature on general resource theories~\cite{regula2017convex}.

\begin{proof}
We shall map the problem into the language of convex optimization and use standard results in that field~\cite{boyd2004convex}.   Using the computation basis $\{ \ket{ \vec{x} } \}$ we can decompose any stabilizer state $\ket{\psi_j} =  \sum_{\vec{x}} M_{\vec{x}, j } \ket{\vec{x}} $. Given a state $\ket{\psi} =  \sum_{\vec{x}}  a_{\vec{x}} \ket{\vec{x}} $, the primal optimization problem can be written as
\begin{align}
	\sqrt{\xi( \psi )}  &  = \mathrm{min}_{\vec{c}}  f(\vec{c})=|| \vec{c} ||_1 \\
	\mbox{such that } & M\vec{c} - \vec{a}	 = 0
\end{align}
This is clearly a convex optimization problem with affine constraints.  Because the coefficient in $\vec{c}$ are complex, rather than real, this is a second order cone problem~\cite{boyd2004convex}.  For any convex optimization problem there exists a dual function
\begin{align}
	g(\nu) &  = \mathrm{inf_{\vec{c}}}  \left( || \vec{c} ||_1 + \nu^{T}(M\vec{c} - \vec{a}) \right) \\
	&= \begin{cases}  -  \nu^{T}\vec{a} & \mbox{ when } ||M^T\nu ||_{\infty} \leq 1 \\
		-  \infty & \mbox{ otherwise }
	\end{cases}
\end{align}
where for any value of the dual variables $\nu$ we have $g(\nu)  \leq \sqrt{\xi( \psi )} $.  The dual optimization problem is the maximisation of $g(\nu)$ over $\nu$ to obtain the best lower bound on  $\sqrt{\xi( \psi )}$.   We can discount the need for two cases by adding  $||M^T \nu  ||_{\infty} \leq 1$ as a constraint, to obtain the problem
\begin{align}
	d^{\star}( \psi )  &  = \mathrm{max}_{\nu}  -  \nu \cdot \vec{a} \\ \nonumber
	\mbox{such that } & || M^T\nu ||_{\infty} \leq 1  ,
\end{align}
or more simply
\begin{align}
	d^{\star}( \psi )  &  = \mathrm{max}_{\nu} \frac{ -  \nu \cdot \vec{a}}{|| M^T\nu ||_{\infty} } .
\end{align}
Because the primal problem has affine constraints, we have strong duality and there must exist a  $\nu_\star$ such that $g(\nu_{\star}) = -  \nu_{\star}^{T}\vec{a}  =  \sqrt{\xi( \psi )} $.   Next, we restate this dual problem in terms of quantum states.   For every $\nu$ we can associate a normalised quantum state
\begin{equation}
\ket{\omega_\nu} : = \frac{1}{||\nu ||_2 } \sum_{\vec{x}} (-\nu^*_{\vec{x}}) \ket{\vec{x}} ,
\end{equation}
so that 
\begin{equation}
\bk{\omega_\nu}{\psi} =  \frac{-  \nu \cdot \vec{a}}{ ||\nu ||_2 } .
\end{equation}
Next we note that 
\begin{equation}
||M^T\nu ||_{\infty} = \frac{\mathrm{Max}_{\ket{\phi} \in \mathrm{STAB}} |	\bk{\omega_\nu}{\phi}   | }{ || \nu ||_2 } = \frac{\sqrt{F(\omega_\nu)}}{|| \nu ||_2}
\end{equation}
Therefore, the dual problem can also be stated as 
\begin{align}
	d^{\star}( \psi )  &  =  \mathrm{max}_{\ket{\omega_{\nu}}}	\frac{\bk{\omega_\nu}{\psi}}{\sqrt{F(\omega_\nu)}}  ,
\end{align}
where the factors $|| \nu ||_2$ have cancelled out.  The optimal $\nu_\star$ gives the optimal  $\ket{\omega_\star}$, which completes the proof.
\end{proof}

\subsection{Stabilizer alignment}
\label{Sec_Fid_Multi}
Combining Theorems \ref{thm:clifmagic} and \ref{thm:randomCvec} we get an upper bound
$\chi_{\delta}(\psi)\leq \delta^{-2} F(\psi)^{-1}$ on the approximate stabilizer rank of any Clifford magic state $\psi$.
We shall be  interested in the case when $\psi$ is a tensor product of 
a large number of few-qubit magic states such as $T$-type or CCZ-type states. For example, the case $\psi=CCZ^{\otimes m}$ is relevant to gadget-based simulation of 
quantum circuits composed of Clifford gates and $m$ CCZ gates. 
This motivates the question of whether the stabilizer fidelity $F(\psi)$ is multiplicative
under tensor product, i.e. \begin{equation}
F(\psi\otimes \phi)\stackrel{?}{=}F(\psi)F(\phi).
\label{eq:fmult}
\end{equation}
Note that $F(\psi\otimes \phi)\ge F(\psi)F(\phi)$ 
since the set of stabilizer states is closed under tensor product. 

Below we define a set of quantum states $\mathcal{S}$ such that 
$F(\phi \otimes \psi) =F(\phi)F(\psi)$ whenever $\phi,\psi\in \mathcal{S}$. Remarkably, this set is also closed under tensor product, that is $\phi\otimes \psi \in \mathcal{S}$ whenever $\phi,\psi\in \mathcal{S}$.
Moreover, we show that the stabilizer fidelity is not multiplicative for all states
$\phi \notin \calS$. More precisely, for any $\phi \notin \calS$ there exists a state $\psi$
such that $F(\phi\otimes \psi)>F(\phi) F(\psi)$. In that sense, our results provide
necessary and sufficient conditions under which the stabilizer fidelity is multiplicative
under tensor product.

To state our results let us generalize the definition of stabilizer fidelity as follows. For each $n\geq 1$ and $0\leq m\leq n$ define a set $S_{n,m}$ which consists of all stabilizer projectors $\Pi$ on $n$ qubits satisfying $\mathrm{Tr}[\Pi]=2^m$. 
\begin{dfn}
For any $n$-qubit state $|\phi\rangle$ define
\[
F_m(\phi)=2^{-m/2} \max_{\Pi\in S_{n,m}} \langle \phi|\Pi|\phi\rangle. \qquad \qquad m=0,\ldots, n.
\]
Let us say that
$\phi$ is  \textit{stabilizer-aligned} if $F_m(\phi)\leq F_0(\phi)$ for all $m$. 
\end{dfn}
Note that in the above $F_0=F$ is the stabilizer fidelity.  Here we investigate the consequences of stabilizer-alignment. Whether or not a given state is stabilizer-aligned is discussed in the following subsection.
\begin{theorem}
Suppose $\phi$ and $\psi$ are stabilizer-aligned. Then $\phi\otimes \psi$ is stabilizer-aligned and
\[
F(\phi\otimes \psi)=F(\phi)F(\psi).
\]
Conversely, suppose $\phi$ is not stabilizer-aligned. 
Let $\phi^{\star}$ be the complex conjugate of $\phi$.
Then 
\[
F(\phi\otimes \phi^{\star})>F(\phi)F(\phi^{\star}).
\]
\label{thm:stabaligned}
\end{theorem}
The theorem implies that the stabilizer fidelity is multiplicative for any
stabilizer-aligned states: 
\begin{corol}
	\label{Fid_Multi}	
Suppose $\psi_1,\ldots,\psi_L$ are stabilizer-aligned quantum states. Then
\[
F(\psi_1\otimes \psi_2\otimes \ldots \otimes \psi_L)=\prod_{j=1}^{L} F(\psi_j).
\]
\label{cor:mul}
\end{corol}
We prove Theorem~\ref{thm:stabaligned} using  characterization of entanglement in tripartite stabilizer states from Ref.~\cite{ghz}:
\begin{lemma}[\cite{ghz}]
\label{lemma:ghz}
	Any pure tripartite  stabilizer state can be transformed by
	local unitary Clifford  operators  to a tensor product of states from the set $\{|0\rangle,|\Psi^{+}\rangle,|\Psi^{+}_3\rangle\}$ where
	\[
	|\Psi^{+}\rangle=\frac{1}{\sqrt{2}}(|00\rangle+|11\rangle) \qquad \qquad |\Psi^{+}_3\rangle=\frac{1}{\sqrt{2}}\left(|000\rangle+|111\rangle\right).
	\]
\end{lemma}
\begin{corol}[\cite{ghz}]
Suppose $\Pi$ be a stabilizer projector describing a bipartite system $AB$. Then there exists a unitary Clifford operator $U=U_A\otimes U_B$ and
integers $a,b,c,d\ge 0$ such that 
\be
\label{PROJ1}
U\Pi U^{-1}
=\sum_{\alpha=1}^{2^a} \sum_{\beta=1}^{2^b} \sum_{\gamma=1}^{2^c}
|\omega_{\alpha\beta\gamma}\ra\la \omega_{\alpha\beta\gamma}|,
\ee
where
\be
\label{PROJ2}
|\omega_{\alpha\beta\gamma}\ra = 2^{-d/2} \sum_{\delta=1}^{2^d}  |\alpha,\gamma,\delta\ra \otimes
|\beta,\gamma,\delta\ra.
\ee
Here $|\alpha,\gamma,\delta\ra$
and $|\beta,\gamma,\delta\ra$ are the computational basis vectors of $A$ and $B$.
\label{corol:ghz}
\end{corol}
\begin{proof}
Let us apply Lemma~\ref{lemma:ghz} to a tripartite stabilizer state
\[
|\Psi\ra= (\Pi \otimes I)2^{-n/2}\sum_{z\in \{0,1\}^n }\; |z\rangle_{AB} \otimes |z\rangle_C ,
\]
where $n=|A|+|B|$ and $C$ is a system of $n$ qubits.
The lemma implies that 
$\Pi$ is equivalent modulo local Clifford operators to a tensor
product of local stabilizer projectors  $|0\ra\la 0|$ and $I=|0\ra\la 0|+|1\ra\la 1|$
as well as bipartite projectors
 $|00\ra\la 00|+|11\ra\la 11|$ 
and $|\Psi^+\ra\la \Psi^+|$ shared between $A$ and $B$.
Let $a$ and $b$ be the number of times $\Pi$ contains
the identity factor on $A$ and $B$ respectively. Let $c$ be the number of times
$\Pi$ contains the projector $|00\ra\la 00|+|11\ra\la 11|$ shared between $A$ and $B$.
Let $d$ be the number of times $\Pi$ contains the EPR projector $|\Psi^+\ra\la \Psi^+|$.
The desired family of states $\omega_{\alpha\beta\gamma}$ is then obtained
by writing each projector $I$ and $|00\ra\la 00|+|11\ra\la 11|$
as a sum of rank-$1$ projectors onto the computational basis vectors. 
\end{proof}
\begin{proof}[Proof of Theorem~\ref{thm:stabaligned}]
To prove the first two claims of the theorem 
it suffices to show that 
	\begin{equation}
	F_m(\phi\otimes \psi)\leq F_0(\phi)F_0(\psi).
	\label{eq:k00}
	\end{equation}
for all $m$.
Indeed, combining Eq.~\eqref{eq:k00} and the obvious bound
 $F_0(\phi)F_0(\psi) \le F_0(\phi\otimes \psi)$ shows that 
$F_m(\phi\otimes \psi)\le F_0(\phi\otimes \psi)$, that is,
$\phi\otimes \psi$ is stabilizer-aligned.
Using Eq.~\eqref{eq:k00} for $m=0$ gives 
multiplicativity of the stabilizer fidelity $F_0(\phi\otimes \psi)=F_0(\phi)F_0(\psi)$. 

Define a bipartite system $AB$ such that $\phi$ and $\psi$ are states of $A$ and $B$.
Let  $\Pi$  be a stabilizer projector of rank $2^m$ such that 
	\[
	F_m(\phi\otimes \psi)=2^{-m/2}\langle \phi \otimes \psi |\Pi|\phi\otimes \psi \rangle.
	\]
We shall write $\Pi$ as a sum of rank-$1$ stabilizer projectors as stated in 
Corollary~\ref{corol:ghz}.
Since local Clifford unitary operators do not change the stabilizer fidelity,
we shall absorb the unitaries $U_A$ and $U_B$
into the states $\phi$ and $\psi$ respectively.
Accordingly, below we set $U=I$.
Consider a single term $\omega_{\alpha\beta\gamma}$ in the decomposition of $\Pi$.
Applying the Cauchy-Schwarz inequality one gets
\be
\label{multi_eq1}
|\la \phi\otimes \psi|\omega_{\alpha\beta\gamma}\ra|^2 = 2^{-d} 
\left| \sum_{\delta=1}^{2^d} \la \phi|\alpha,\gamma,\delta\ra
\cdot \la \psi|\beta,\gamma,\delta\ra \right|^2
\le 2^{-d} \la \phi|\Pi^A_{\alpha\gamma}|\phi\ra
\cdot \la \psi|\Pi^B_{\beta \gamma}|\psi\ra ,
\ee
where we defined stabilizer projectors 
\be
\label{PROJ_AB}
\Pi^A_{\alpha,\gamma} = 
\sum_{\delta=1}^{2^d} |\alpha ,\gamma,\delta\ra\la \alpha ,\gamma,\delta|
\quad \mbox{and} \quad 
 \Pi^B_{\beta,\gamma}=\sum_{\delta=1}^{2^d}
 |\beta ,\gamma,\delta\ra\la \beta ,\gamma,\delta|.
\ee
By assumption, $\psi$ is stabilizer-aligned. Thus 
\be
\label{multi_eq1'}
\max_\gamma 
\la \psi| \sum_{\beta=1}^{2^b} \Pi^B_{\beta \gamma}|\psi\ra \le 2^{(b+d)/2} F_0(\psi).
\ee
Here we noted that $\sum_{\beta=1}^{2^b} \Pi^B_{\beta \gamma}$ is 
a projector of rank $2^{b+d}$ for all $\gamma$.
Combining Eq.~(\ref{multi_eq1},\ref{multi_eq1'}) gives
\be
\label{multi_eq2}
\la \phi\otimes \psi|\Pi|\phi\otimes \psi\ra
=
\sum_{\alpha=1}^{2^a} \sum_{\beta=1}^{2^b} \sum_{\gamma=1}^{2^c}
|\la \phi\otimes \psi|\omega_{\alpha\beta\gamma}\ra|^2
\le 2^{(b-d)/2} F_0(\psi) \cdot  \la \phi| \sum_{\alpha=1}^{2^a} \sum_{\gamma=1}^{2^c}
 \Pi^A_{\alpha,\gamma}|\phi\ra
\ee
The assumption that $\phi$ is stabilizer-aligned gives
\be
\label{multi_eq3}
 \la \phi| \sum_{\alpha=1}^{2^a} \sum_{\gamma=1}^{2^c}
\Pi^A_{\alpha,\gamma}|\phi\ra
\le 
2^{(a+c+d)/2} F_0(\phi).
\ee
Here we noted that $\sum_{\alpha=1}^{2^a} \sum_{\gamma=1}^{2^c}
 \Pi^A_{\alpha,\gamma}$ is a projector of rank $2^{a+c+d}$.
Combining Eqs.~(\ref{multi_eq2},\ref{multi_eq3}) gives
\[
\la \phi\otimes \psi|\Pi|\phi\otimes \psi\ra\le 2^{(a+b+c)/2} F_0(\psi)F_0(\phi).
\]
It remains to notice that $\Pi$ has rank
$2^m$, where $m=a+b+c$.
This establishes Eq.~\eqref{eq:k00}.  


We now prove the converse statement from Theorem \ref{thm:stabaligned}.
\begin{lemma}
	Let $\phi$ be an $n$-qubit state which is not stabilizer-aligned. Then
	\[
	F_0(\phi\otimes \phi^{\star})>F_0(\phi)F_0(\phi^{\star}).
	\]
\end{lemma}
\begin{proof}
	If $\phi$ is not stabilizer-aligned then we have $F_m(\phi)>F_0(\phi)$ for some $m\in \{1,\ldots,n\}$. Let $\Pi$ be a stabilizer projector with
	\[
	F_m(\phi)=\frac{1}{\sqrt{2}^m} \langle \phi|\Pi|\phi\rangle.
	\]
	Let $C$ be an $n$-qubit Clifford such that 
	\[
	\Pi=C\left( |0\rangle\langle0|_{n-m}\otimes I_m\right)C^{\dagger}.
	\]
	Next consider a system of $2n$ qubits and partition them as $[2n]=ABA'B'$ where $|A|=|A'|=n-m$ and $|B|=|B'|=m$. Define a $2n$-qubit stabilizer state
	\[
	|\theta\rangle=C\otimes \alpha |0\rangle_A |\Phi\rangle_{BB'} |0\rangle_{A'} ,
	\] 
	where 
	\[
	|\Phi\rangle_{BB'}=\frac{1}{\sqrt{2}^m}\sum_{z\in \{0,1\}^m}|z\rangle_B|z\rangle_{B'}.
	\]
	Also define a normalized $m$-qubit state
	\[
	|\omega\rangle=\frac{1}{2^{m/4}\sqrt{F_m(\phi)}}\left(\langle 0|_{n-m}\otimes I_m\right) C|\phi\rangle.
	\]
	\begin{align}
		F_0(\phi\otimes \phi^{\star})&\geq \langle \phi\otimes \phi^{\star} |\theta\rangle\langle \theta| \phi\otimes \phi^{\star}\rangle\\
		&=\langle \omega\otimes \omega^\star|\Phi\rangle \langle \Phi|\omega\otimes \omega^\star\rangle 2^m (F_m(\phi))^2\\
		&=(F_m(\phi))^2\\
		&> F_0(\phi) F_0(\phi^{\star}).
	\end{align}
	where in the last line we used the fact that $F_m(\phi)>F_0(\phi)=F_0(\phi^{\star})$.
\end{proof}
\end{proof}

\subsection{Proving and disproving stabilizer alignment}
\label{Sec_when_stab_aligned}

In this section we prove that all states of $n\le 3$ qubits
are stabilizer-aligned.  We also show that 
typical  $n$-qubit states are not stabilizer-aligned for sufficiently large $n$.
An important lemma is the following
\begin{lemma}
	\label{Lem_F_ordering}
For any quantum state $\psi$ we have
$F_m(\psi)\le F_0(\psi)$ for $m=1,2,3$.	 
\end{lemma}
\noindent
It follows immediately that 
\begin{corol}
	\label{Cor_single_qubits}
All states of $n\le 3$ qubits are stabilizer-aligned.  
\end{corol}
Indeed, if we consider $n$-qubit states, it suffices to check
that $F_m(\psi)\le F_0(\psi)$ for $m\le n$.
\begin{corol}
\label{Cor_Fid_Stab_align}	
	If  $F_0(\psi) \geq 1/4$ then $\psi$ is stabilizer-aligned.  
\end{corol}
Indeed, if $m\ge 4$ then  $F_m(\psi) \le 2^{-m/2}\le 1/4\le F_0(\psi)$.

Finally,  we show that Haar-random $n$-qubit states  are not stabilizer-aligned
for sufficiently large $n$.
\begin{claim}
	\label{Claim_Harr}
	Let $\psi$ be a Haar-random $n$-qubit state.  Then
	\[
	\mathrm{Pr}[F_0(\psi\otimes \psi^{\star})\neq F_0 (\psi)F_0(\psi^{\star})]\geq 1-o(1).
	\]
	and so for large enough $n$ a typical state $\psi$ is not stabilizer-aligned.
\end{claim}
Highly structured states on a large number of qubits may be stabilizer-aligned, and for instance it is an open question whether or not all Clifford magic states are stabilizer-aligned. 


\begin{proof}[Proof of Lemma~\ref{Lem_F_ordering}]
First, we claim that 
\be
\label{FFbound}
F_{m-1}(\psi)\ge 2^{-1/2} \left( 1+ \left[\frac{2^m-1}{4^m-1}\right]^{1/2}\right)
\cdot F_m(\psi)
\ee
for all $m\ge 1$. Indeed, consider a fixed $m$ and 
a rank-$2^m$ stabilizer projector $\Pi \in S_{n,m}$ such that 
$F_m(\psi)=2^{-m/2}\la \psi|\Pi|\psi\ra$.
Using the standard stabilizer formalism one can show that 
\[
U\Pi U^{-1} = I^{\otimes m}\otimes |0\ra \la 0|^{\otimes (n-m)}\equiv \Pi'
\]
for some $n$-qubit unitary Clifford operator $U$.
Define a state $|\psi'\ra=U|\psi\ra$.
We have
\[
\Pi'  |\psi' \ra = \Gamma^{1/2}  |\omega \ra \otimes |0^{n-m}\ra
\]
for some $m$-qubit normalized state $|\omega\ra$ and 
$\Gamma=\la \psi'|\Pi'|\psi'\ra=\la \psi|\Pi|\psi\ra$.
Since  $\omega$ is normalized, 
\[
\sum_{P\ne I} \la \omega|P|\omega\ra^2 = 2^m-1,
\]
where the sum runs over all $4^m-1$ non-trivial Pauli operators on $m$ qubits.
Thus there exists an $m$-qubit Pauli
operator $P\ne I$ such that 
\be
\label{goodP}
\la \omega |P|\omega\ra \ge \left(\frac{2^m-1}{4^m-1}\right)^{1/2}.
\ee
Define a stabilizer projector 
\[
\Pi''=
\frac12(I+P)\otimes |0\ra \la 0|^{\otimes (n-m)}\in S_{n,m-1}.
\]
Recalling that $\Gamma=\la \psi|\Pi|\psi\ra=2^{m/2} F_m(\psi)$ we arrive at
\begin{align}
F_{m-1}(\psi)=
F_{m-1}(\psi') & \ge 2^{-(m-1)/2}  \la \psi'|\Pi''|\psi'\ra \\ \nonumber
& =2^{-(m-1)/2} \frac{\Gamma}2 (1+ \la \omega |P|\omega\ra )  \\ \nonumber
& = 2^{-1/2} (1+ \la \omega |P|\omega\ra )\cdot F_m(\psi).
\end{align}
Combining this identity and Eq.~(\ref{goodP}) proves Eq.~(\ref{FFbound}).
Applying Eq.~(\ref{FFbound}) inductively gives
\be
\label{F1bound}
F_0(\psi)\ge  2^{-1/2}(1+\sqrt{1/3}) \cdot F_1(\psi)\approx 1.115 \cdot F_1(\psi),
\ee
\be
\label{F2bound}
F_0(\psi)\ge  2^{-1/2}(1+\sqrt{1/3}) \cdot 2^{-1/2}(1+\sqrt{3/15}) \cdot F_2(\psi)  \approx 1.141 \cdot F_2(\psi),
\ee
\be
\label{F3bound}
F_0(\psi)\ge 2^{-1/2}(1+\sqrt{1/3}) \cdot 2^{-1/2}(1+\sqrt{3/15})
\cdot 2^{-1/2}(1+\sqrt{7/63})\cdot F_3(\psi) 
  \approx 1.076 \cdot F_3(\psi).
\ee
Thus  $F_0(\psi)\ge F_m(\psi)$ for $m=1,2,3$ proving the lemma.
\end{proof}


Next, we prove claim~\ref{Claim_Harr}. \begin{proof}
	Let $w$ be any $n$-qubit state. For Haar-random $\psi$ the probability density function $p(y)$ of $y=|\langle w|\psi\rangle|^2$ does not depend on $w$ and is equal to (equation (9) of Ref.~\cite{random}), 
	\[
	p(y)=(2^n-1)(1-y)^{2^n-2}.
	\]
	Integrating this we obtain the cumulative distribution function
	\[
	\mathrm{Pr}\left[ |\langle w|\psi\rangle|^2\geq x\right]=(1-x)^{2^n-1}\leq \exp(-x(2^{n}-1)).
	\]
	Since an $n$-qubit stabilizer state is specified by $O(n^2)$ bits the cardinality of the set $\mathrm{STAB}_n$ of $n$-qubit stabilizer states is $|\mathrm{STAB}_n|\leq 2^{O(n^2)}$. Choosing $x=n^3/2^n$ and applying a union bound we get
	\[
	\mathrm{Pr} \left[ \left(\max_{w\in \mathrm{STAB}_n} |\langle \psi|w \rangle|^2\right)\geq n^3/2^n\right]\leq e^{-\Omega(n^3)}.
	\]
	This says that with probability very close to 1 a random $\psi$ has $F_0(\psi)=F_0(\psi^{\star})\leq n^3/2^n$. Next suppose $\psi$ has this property. Then
	\[
	F_0(\psi\otimes \psi^{\star})\geq \left|\frac{1}{\sqrt{2}^n}\sum_{z\in \{0,1\}^n} \langle z|\psi\rangle \langle z| \psi^{\star}\rangle \right|^2=\frac{1}{2^n},
	\]
	which is strictly greater than $F_0(\psi)F_0(\psi^{\star})\leq 2^{-2n}(n^3)^2$.
\end{proof}

\begin{figure}[t]
	\centering
	\includegraphics[scale=0.7]{cstar_blochsphere.png}
	\caption{The color indicates the value of $\xi$ for single-qubit states in the first octant of the Bloch sphere. This function controls the upper bound on the approximate stabilizer rank as in Eq.~\eqref{eq:expsingleq}.}
	\label{fig:cstar1qubit}
\end{figure}

\subsection{Multiplicativity of stabilizer extent}
\label{Sec_Cstar_multi}
This subsection considers tensor products  of few-qubit states 
that involve at most three qubits each
and shows that $\xi$ behaves multiplicatively for such products, proving Proposition~\ref{thm:prod}. The proof will draw heavily on Theorem \ref{thm:witness} and Corollary~\ref{Cor_single_qubits}.
\begin{proof}[Proof of Proposition~\ref{thm:prod}]
By Theorem~\ref{thm:witness} there exist witness states $\{ \omega_{\star,1} , \omega_{\star,2}  , \ldots , \omega_{\star,L}  \}$ such that
	\begin{equation}
\frac{|\bk{\psi_j}{\omega_{\star, j}}|^2}{F(\omega_{\star, j})} =  \xi(\psi_j).
\end{equation}
We consider the product witness $\ket{\Omega } =  \bigotimes_j \ket{\omega_{\star,j}}$
for which
\begin{equation}
|\bk{\Psi}{\Omega}|^2 =  \prod_j |\bk{\psi_j}{\omega_{\star,j}}|^2.
\end{equation}
Furthermore,  using Corollary~\ref{Cor_single_qubits} and Theorem \ref{thm:stabaligned} we get
\begin{equation}
	F( \Omega ) = \prod_j  	F( \omega_{\star, j} ) .
\end{equation}
Putting this together yields
\begin{equation}  
\frac{|\bk{\Psi}{\Omega}|^2}{ F( \Omega ) } =  \prod_j  \frac{ |\bk{\psi_j}{\omega_{\star,j}}|^2 }{ F( \omega_{\star, j} ) } = \prod_{j=1}^L  \xi(\psi_j) .
\end{equation}
Thus, using $\Omega$ as a witness, we get
\begin{equation}  
\prod_{j=1}^L  \xi(\psi_j) \leq \xi(\Psi) .
\end{equation}
Furthermore, $\xi$ is inherently sub-multiplicative and so we must have equality. 
\end{proof}
Now let us see how this can be used to bound the approximate stabilizer rank of a product state $\alpha^{\otimes n}$ where $\alpha$ is a single-qubit state. Combining Theorem \ref{thm:prod} with Lemma \ref{lem:randomCvec} we get

\begin{equation}
\chi_{\delta}(\alpha^{\otimes n})\leq \delta^{-1} \xi(\alpha^{\otimes n})=\delta^{-2}(\xi(\alpha))^{n}.
\label{eq:expsingleq}
\end{equation}
Note that since $\alpha$ is a single-qubit state we can easily compute $\xi(\alpha)$ by solving a small convex optimization program.  In Figure \ref{fig:cstar1qubit} we plot $\xi(\alpha)$ as a function of the single-qubit state $\alpha$ on the first octant of the Bloch sphere. 


\begin{figure}[t]
\centering
\includegraphics[scale=0.7]{Comparison.png}
\caption{The approximate stabilizer rank of $|\theta^{\otimes n}\rangle$ is upper bounded as $\chi_\delta(\theta^{\otimes n})\leq \delta^{-2}\xi(\theta)^{n}$, where $\xi(\theta)=(\cos(\theta/2)+\tan(\pi/8)\sin(\theta/2))^2$ is attained by the stabilizer decomposition from Eq.~\eqref{Eq_SingleQubitsDecomp}. The red line shows the function $\xi(\theta)$ for $\theta \in [0,\pi/4]$ and the blue line shows the function $g(\theta)=2^{h_2(\cos^2(\theta/2))}$ where $h_2$ is the binary entropy.  Our upper bound on the approximate stabilizer rank of $\theta^{\otimes n}$ performs better that obtained by a naive expansion in the $0,1$ basis whenever the red line lies below the blue line.}
\label{fig:redblue}
\end{figure}

The maximum value plotted in Figure \ref{fig:cstar1qubit} is $\xi(f)=2/(1+1/\sqrt{3})\approx 1.2679$, which is achieved by the so-called face state $|f\rangle$ which lies in the center of the surface and is defined by
\[
|f\rangle\langle f|=\frac{1}{2}\left(I+\frac{1}{\sqrt{3}}(X+Y+Z)\right).
\]

The single-qubit states in Figure \ref{fig:cstar1qubit} which lie in the $x$-$z$ plane are of the form
\begin{equation}
	\label{Eq_SingleQubitsDecomp}
|\theta\rangle=\cos(\theta/2)|0\rangle+\sin(\theta/2)|1\rangle= \left(\cos(\theta/2)-\sin(\theta/2)\right)|0\rangle+\sqrt{2}\sin(\theta/2)|+\rangle
\end{equation}
for $\theta\in [0,\pi/2]$. In this case, the stabilizer decomposition on the right hand side achieves the optimal value of $\xi$. We can use this example to show that in the general case the upper bound on approximate stabilizer rank given in Theorem \ref{thm:randomCvec} is not tight (for $\delta=O(1)$, say). When $\theta$ is close to $0$ it becomes advantageous to expand $\theta^{\otimes n}$ in the standard $0,1$ basis and truncate amplitudes which are very small. Using this approach one obtains an approximate stabilizer rank scaling as $2^{h_2(\cos^2(\theta/2))}$ where $h_2$ is the binary entropy. In Figure \ref{fig:redblue} we compare the performance of these upper bounds as a function of  $\theta$. 

