It is widely believed that universal quantum computers cannot be efficiently simulated by classical probabilistic algorithms. This belief is partly supported by the fact that
state-of-the-art classical simulators employing modern supercomputers are still limited to a few dozens of qubits~\cite{smelyanskiy2016qhipster,haner20170,pednault2017breaking,chen2018classical}.
At the same time, certain quantum information processing tasks do not require computational universality. For example, quantum error correction
based on stabilizer codes and Pauli noise models~\cite{gottesman1998theory} only requires quantum circuits composed of 
Clifford gates and Pauli measurements--which can be easily simulated  classically 
for thousands of qubits using the Gottesman-Knill theorem~\cite{aaronson04improved,anders2006fast}. Furthermore, it is known that Clifford circuits
can be promoted to universal quantum computation when provided with a plentiful supply of some computational primitive outside the stabilizer operations, such as a non-Clifford gate or magic state~\cite{bravyi2005universal}.  This raises the possibility of simulating quantum circuits with a large number of qubits and few non-Clifford gates.  Aaronson and Gottesman~\cite{aaronson04improved} were the first to propose a classical simulation method covering this situation, with a runtime that scales polynomially with the number of qubits and Clifford gate count but exponentially with the number of non-Clifford gates.  This early work is an intriguing proof of principle but with a very large exponent, limiting potential applications.   


Recent algorithmic improvements have helped tame this exponential scaling by significantly decreasing the size of the exponent.  A first step was made by Garcia, Markov and Cross~\cite{garcia2012efficient,Garcia14moreStabRank}, who proposed and studied the decomposition of states into a superposition of stabilizer states. Bravyi, Smith and Smolin~\cite{Bravyi16stabRank} formalized this into the notion of stabilizer rank. The stabilizer rank $\chi(\psi)$  of 
 a pure state $\psi$ is defined as the smallest integer  $\chi$ such that $\psi$
 can be expressed  as a superposition of $\chi$ stabilizer states. 
It can be thought of as a measure of computational non-classicality analogous the Schmidt rank measure of entanglement.  In particular,  $\chi(\psi)$ quantifies the simulation cost of 
stabilizer operations (Clifford gates and Pauli measurements) applied to the initial state  $\psi$. 

It is known that stabilizer operations augmented with  preparation of certain
single-qubit ``magic states" become computationally universal~\cite{bravyi2005universal}.
In particular, any quantum circuit composed of Clifford gates
and $m$ gates $T=|0\ra\la 0|+e^{i\pi/4}|1\ra\la 1|$ can be implemented by
stabilizer operations acting on the initial state $|\psi\ra =|T\ra^{\otimes m}$,
where $|T\ra \propto |0\ra + e^{i\pi/4} |1\ra$.
Thus the stabilizer rank  $\chi(T^{\otimes m})$ provides an upper bound on the simulation cost
of Clifford+$T$ circuits with $m$ $T$-gates.
The authors of  Ref.~\cite{Bravyi16stabRank} used a numerical search method to 
compute the stabilizer rank $\chi(T^{\otimes m})$ for $m\le 6$ finding that
$\chi(T^{\otimes 6})=7$. The numerical search  becomes impractical for
$m>6$ 
and one instead works with suboptimal decompositions by breaking $m$ magic states up into blocks of six or fewer qubits. This yields a classical simulator of Clifford+$T$ circuits running in time $2^{0.48 m}$ with certain polynomial prefactors~\cite{bravyi2016improved}.
More recently, Ref.~\cite{bravyi2016improved} introduced
an approximate version of the stabilizer rank and a method of constructing  approximate
stabilizer decomposition of the magic states $|T\ra^{\otimes m}$. This led to a simulation algorithm with runtime scaling as $2^{0.23m}$
that  samples the output distribution of the target circuit with  a small statistical error.
In practice, it can simulate single-qubit measurements on the output state of Clifford+$T$ circuits with  $m\le 50$ on a standard laptop~\cite{bravyi2016improved}.  A similar class of simulation methods uses Monte Carlo sampling over quasiprobability distributions, where the distribution can be over either a discrete phase space ~\cite{veitch2012negative,pashayan15,Delfosse15rebits}, over the class of stabilizer states~\cite{Howard17robustness} or over stabilizer operations~\cite{OakRidge17}.  These quasiprobability methods are a natural method for simulating noisy circuits but for pure circuits they appear to be slower than simulation methods based on stabilizer rank.  

Here we present a more general set of tools for finding exact and approximate stabilizer decompositions 
as well as improved simulation algorithms based on such decompositions.
A central theme throughout this paper  is generalizing the results  of Refs.~\cite{Bravyi16stabRank,bravyi2016improved}
beyond the Clifford+$T$ setting.  While Clifford+$T$ is a universal gate set, it requires several hundred $T$ gates to synthesize an arbitrary single qubit gate to a high precision (e.g. below $10^{-10}$ error).   Therefore, it would be impractical to simulate such gates using the Clifford+$T$ framework.  We achieve
significant improvements in the simulation runtime by branching out to more general gate sets
including arbitrary-angle $Z$-rotations and CCZ gates. 
Furthermore, we propose more efficient subroutines for simulating the action of Clifford gates
and Pauli measurements on superpositions of $\chi\gg 1$ stabilizer states. 
In practice, this enables us to perform simulations in the regime $\chi \sim 10^6$
with about 50 qubits
on a laptop computer improving upon $\chi\sim 10^3$ simulations reported in
Ref.~\cite{bravyi2016improved}.
The table provided below summarizes new simulation methods,
simulation tasks addressed by each method, and the runtime scaling. 
\begin{figure}[h]
	\includegraphics[height=6.65cm]{algorithm_summary_table.pdf}
	\caption{Summary of new simulation methods.
	For simplicity, here we restrict the
	attention to quantum circuits composed of Clifford gates and diagonal single-qubit gates
	$R(\theta)=\mathrm{diag}(1,e^{i\theta})$. The $T$-gate can be obtained as a special
	case $T=R(\pi/4)$. We consider strong and weak simulation tasks where the goal is to
	estimate a single output probability (with a small multiplicative error) and
	sample the output probability distribution (with a small  statistical error)
	respectively.
	The runtime scales exponentially with the non-Clifford gate count $m$
	and polynomially with the number of qubits and the Clifford gate count. For simplicity, here
	we ignore the polynomial prefactors. For a detailed description of our simulation methods, 
	see Section~\ref{sec:simulations}.}
	\label{summary_table}
\end{figure}


On the theory side, we establish some general  properties of the approximate stabilizer rank.
Our main tool is a Sparsification Lemma that shows how to 
convert a dense stabilizer decomposition of a given target state 
(that may contain all possible stabilizer states) to a sparse decomposition
that contains fewer stabilizer states. 
The lemma generalizes the method of random linear codes introduced  in Ref.~\cite{bravyi2016improved}
in the context of Clifford+$T$ circuits. 
It allows us to obtain sparse stabilizer decompositions for the output state of
more general quantum circuits directly without using magic state gadgets. 
Combining the Sparsification Lemma and  convex duality arguments, we relate
the approximate stabilizer rank of a state $\psi$ to a stabilizer fidelity $F(\psi)$ defined
as the maximum overlap between $\psi$ and stabilizer states. Central to these calculations is a new quantity called Stabilizer Extent, which quantifies, in an operationally relevant way, how non-stabilizer a state is. We give necessary and sufficient conditions under which the stabilizer fidelity is multiplicative under the tensor product. Finally, we propose a new strategy for proving lower bounds on the stabilizer rank
of the magic states which uses the machinery of ultra-metric matrices~\cite{MMM,NabenVarga}.


As a main application of our simulation algorithms we envision verification of noisy intermediate-size
quantum circuits~\cite{preskill2018quantum}
in the regime when a brute-force classical simulation may be
 impractical~\cite{aharonov2017interactive,morimae2016post,Jozsa2017}.
For example, a quantum circuit composed of Clifford gates and single-qubit $Z$-rotations
with angles  $\theta_1,\ldots,\theta_m$ can be efficiently simulated using our methods
in the regime when only a few of the angles $\theta_a$ are non-zero or if all the angles 
$\theta_a$ are small in magnitude, see Section~\ref{sec:sum_cliffords}.
By fixing the Clifford part of the circuit and varying the rotation angles $\theta_a$ one can
therefore interpolate between the regimes where the circuit output  can and cannot be verified 
classically. From the experimental perspective, single-qubit $Z$-rotations
are often the most reliable elementary operations~\cite{mckay2017efficient}.
Thus one should expect that the circuit output fidelity should not depend significantly on the
choice of the angles $\theta_a$.

The next section provides a more detailed overview of our results.
