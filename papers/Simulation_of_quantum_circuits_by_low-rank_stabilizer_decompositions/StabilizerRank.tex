In this Section, we describe bounds on the exact and approximate stabilizer rank. In subsection \ref{Sec_Symmetric_states}, we give the proof of Theorem~\ref{Thm_many_copies}, which proceeds by establishing an upper bound on the exact stabilizer rank of states symmetric under permutations of certain subsystems. As a consequence we will see that $\chi(\psi^{\otimes t}) \ll \chi(\psi)^t$ for modest $t$. In subsection \ref{Sec_Lemma_Proof} we prove Theorem~\ref{thm:randomCvec} using a Sparsification lemma that allows us to convert exact stabilizer decompositions into approximate stabilizer decompositions (with possibly fewer terms). In subsection \ref{Sec_Clifford_Magic_States}, we study the approximate stabilizer rank of Clifford magic states and establish Proposition \ref{thm:clifmagic}. Finally, in subsection \ref{Sec_ultra} we turn our attention to lower bounds and prove Proposition \ref{prop:lower_bound}.

\subsection{Exact stabilizer rank}
\label{Sec_Symmetric_states}
 Let us denote $\mathrm{Sym}_{n,t}$ as the subspace that is symmetric with respect to swaps between $t$ partitions with each partition holding $n$ qubits.  For instance, any $n$-qubit state $\psi$ satisfies $\psi^{\otimes t} \in \mathrm{Sym}_{n,t}$ for any $t$. Although the symmetric subspace also contains states entangled across these partitions.  Throughout this section we use $\mathrm{dim}( \ldots)$ to denote the dimension of a vector space and  $\mathrm{span}( \ldots)$ to denote the vector space spanned by a set of vectors.  Let us agree that when we write $\mathrm{dim}( \mathbb{S} )$ where $\mathbb{S}$ is a set of vectors (rather than a vector space) this means the dimension of the vector space spanned by $\mathbb{S}$.

This section provides a proof of Thm.~\ref{Thm_many_copies}, though we shall actually prove a more general result regarding the stabilizer rank of a subspace defined as follows
\begin{dfn}
	We define stabilizer rank $\chi(P)$ of a subspace $P$ to be the minimum $\chi$ such that there exists a set of $\chi$ stabilizer states $\mathbb{S}=\{ \phi_1,  \phi_2, \ldots , \phi_\chi \}$ satisfying $P \subset \mathrm{span}[ \mathbb{S} ]$. 
\end{dfn}  
Notice that given a set of stabilizer states $\mathbb{S}$ such that $\mathrm{Sym}_{n,t} \subseteq \mathrm{span}(\mathbb{S}) $, it follows that every element of the space $\mathrm{Sym}_{n,t}$ can be decomposed in terms of $|\mathbb{S}|$ stabilizer states.  Therefore, if $\Psi \in \mathrm{Sym}_{n,t}$ then $\chi(\Psi) \leq \chi(\mathrm{Sym}_{n,t})$.   As a special case, if $\Psi = \psi^{\otimes t}$ then  $\chi(\psi^{\otimes t}) \leq \chi(\mathrm{Sym}_{n,t})$.  Therefore,  Thm.~\ref{Thm_many_copies} follows as a corollary of the following result
\begin{lemma}
	\label{symmetric}
	Consider $\mathrm{Sym}_{n,t}$ for some nonzero $n$ and $t$.  It follows that for all $t \leq 5$ we have
	\begin{equation}
		\label{symmetric_inequality}
	\chi( \mathrm{Sym}_{n,t} ) = \mathrm{dim}[\mathrm{Sym}_{n,t}] = \binom{2^n + t -1}{t} ,
	\end{equation}
	where the round brackets denotes the binomial coefficient.
\end{lemma}
This has the direct and elegant consequence that for all single qubit states $\psi$ we have $\chi( \psi ^{\otimes t}  ) \leq t+1$ whenever $t \leq 5 $. 

\begin{proof}[Proof of Lemma \ref{symmetric}]  First we show that Eq.~\eqref{symmetric_inequality} holds for some $n$ and $t$ whenever there exists a set of stabilizer states $\mathbb{S}$ with the following properties:
	\begin{enumerate}
		 \item every $\Phi \in \mathbb{S}$ satisfies $\Phi \in \mathrm{Sym}_{n,t}$; and
\item $\mathrm{dim}(\mathrm{Sym}_{n,t}) = \mathrm{dim}( \mathbb{S} )$.
	\end{enumerate}	
 For any set of vectors $\mathbb{S}$, there exists a subset $\mathbb{S}' \subseteq \mathbb{S}$  that is a minimal spanning set, with $\mathrm{span}(\mathbb{S}')=\mathrm{span}(\mathbb{S})$ and $|\mathbb{S}'|=\mathrm{dim}(\mathbb{S})$.  Therefore, given a set that spans the symmetric space we can conclude that $\chi(\mathrm{Sym}_{n,t}  ) \leq \mathrm{dim}(\mathbb{S})$.  Furthermore, if $\mathbb{S}$ has the swap invariance property then $\mathrm{span}(\mathbb{S}) \subseteq \mathrm{Sym}_{n,t} $ and $\mathrm{dim}(\mathbb{S}) \leq \mathrm{dim}(\mathrm{Sym}_{n,t})$.   Combining these inequalities gives $\chi(\mathrm{Sym}_{n,t}  ) \leq\mathrm{dim}(\mathrm{Sym}_{n,t})$.  It is obvious that $\mathrm{dim}(\mathrm{Sym}_{n,t}) \leq \chi(\mathrm{Sym}_{n,t}  )$ and so $\chi(\mathrm{Sym}_{n,t}  ) = \mathrm{dim}(\mathrm{Sym}_{n,t})$.  Lastly, the dimension of the symmetric space is well-known and can for example be found in Ref.~\cite{zhu16}. 

Next, it remains to find a set $\mathbb{S}$ with the aforementioned properties for certain values of $n$ and $t$.  We consider sets of stabilizer states of the form $\mathbb{S}_{n,t}=\{ \ket{\phi_j}^{\otimes t} \}_j$ where $\{ \ket{\phi_j} \}_j =: \mathrm{STAB}_n$ is the set of all $n$-qubit stabilizer states.  This ensures property 1.  It remains to show when $\mathbb{S}_{n,t}$ has sufficiently large dimension (property 2).  We observe that the operator
\begin{equation}
	 \sigma_{n,t} :=\frac{1}{|\mathrm{STAB}_n|} \sum_{\psi_j \in \mathrm{STAB}_n} \kb{\psi_j}{\psi_j}^{\otimes t}
\end{equation}	
satisfies
\begin{equation}
	 \mathrm{rank}( \sigma_{n,t} ) = \mathrm{dim}( \mathbb{S}_{n,t}  ) .
\end{equation}
and so property 2 also holds whenever $\mathrm{rank}( \sigma_{n,t} )=\mathrm{dim}( \mathrm{Sym}_{n,t}  )$. 

Let us consider when $t \leq 3$ with no constraints on $n$.  We will use that the stabilizer states form a projective $3$-design~\cite{Webb16,kueng15,zhu16}.  The relevant property of such designs is that for $t \leq 3$ we know
\begin{equation}
	\sigma_{n,t} \propto \Pi_{n,t} ,
\end{equation}
where $\Pi_{n,t}$ is the projector onto $\mathrm{Sym}_{n,t}$.  Therefore, $\mathrm{rank}( \sigma_{n,t} )=\mathrm{rank}( \Pi_{n,t} )  =\mathrm{dim}( \mathrm{Sym}_{n,t}  )$ and the lemma is proven for  the case of $t \leq 3$.

For $t=4$, it is known that the stabilizer states are not a projective $4$-design and so $\sigma_{n,4}$ is not proportional to the symmetric projector~\cite{zhu16}.  However, the stabilizer states ``fail gracefully" to be a projective $4$-design~\cite{zhu16}, such that the deviation of $\sigma_{n,4}$ from $\Pi_{n,4}$ is sufficiently small that we still have $\mathrm{rank}(\sigma_{n,4})=\mathrm{rank}(\Pi_{n,4})$. Ref.~\cite{gross2017schur} extends this result such that we can also deduce the following
\begin{claim}
	\label{RankEq}
	For all $n$ and $t \leq 5$ we have  $\mathrm{rank}(\sigma_{n,t})=\mathrm{rank}(\Pi_{n,t})$.
\end{claim}	
This suffices to prove Lem.~\ref{symmetric}.  In contrast, this proof technique can not extend to $t>5$ due to the stabilizer testing algorithm of Ref~\cite{gross2017schur}. This algorithm shows that there exists a projector $W$ such that $\mathrm{Tr}[W\sigma_{n,6} ]=0$ but  $\mathrm{Tr}[W\Pi_{n,6}] \neq 0$, which entails $\mathrm{rank}(\sigma_{n,6})<\mathrm{rank}(\Pi_{n,6})$.    

Although Claim~\ref{RankEq} can be deduced from Ref.~\cite{gross2017schur}, it is not explicitly shown, so we provide the details here.  Examples 4.27 and 4.28 of Ref.~\cite{gross2017schur},  show that
\begin{align}
	\sigma_{n,4} & \propto   \Pi_{n,4}  + a_{n}\Pi_{n,4}  P_{[4]}^{\otimes n} \Pi_{n,4} , \\ \nonumber 
	\sigma_{n,5} & \propto   \Pi_{n,5}  + b_{n}\Pi_{n,5}  P_{[5]}^{\otimes n} \Pi_{n,5} .
\end{align}
where $a_{n}$ and $b_n$ are positive constants and $P_{[4]}$ and $P_{[5]}$ are projectors onto a stabilizer code 
\begin{align}
	P_{[4]} & = \frac{1}{4}( \id^{\otimes 4} + X^{\otimes 4} + Y^{\otimes 4} + Z^{\otimes 4})  \\ \nonumber
	P_{[5]} & = P_{[4]} \otimes \id 
\end{align}
Since $P_{[4]}$ and $P_{[5]}$ are positive operators, so too are $a_n\Pi_{n,4} P_{[4]}^{\otimes n} \Pi_{n,4}$ and $b_n\Pi_{n,5} P_{[5]}^{\otimes n} \Pi_{n,5}$.  In general, if $M$ and $N$ are positive operators we have $\mathrm{rank}(M+N)\geq \mathrm{rank}(M)$.  Therefore, for $t=4, 5$ we have $\mathrm{rank}(	\sigma_{n,t})\geq \mathrm{rank}( \Pi_{n,t})$, which implies the desired rank equivalence and completes the proof. \end{proof}

\begin{figure}
	\centering
	\includegraphics[width=250pt]{HistoPinkBlue}
	\caption{The exact stabilizer rank (numerically found) for $n$ copies of a single qubit state: for the $T$ state and for generic single qubit states.}
	\label{SingleQubitStates}
\end{figure}

\begin{table}

\centering

	\begin{tabular}{|c||ccccc|} 	  \hline
		&	$t=1$ &  $t=2$ &  $t=3$ & $t=4$ & $t=5$ \\  \hline  \hline
		$n=1$  &	2 &  1.73205 &  1.5874 & 1.49535 & 1.43097 \\ 
		$n=2$  &  4 &  3.16228 & 2.71442 & 2.4323 & 2.23685 \\
		$n=3$ &   8 &  6           & 4.93242 & 4.26215&  3.79966 \\   \hline
	\end{tabular}	
	\caption{Upper bounds on $\chi(\psi^{\otimes t})^{1/t}$ where $\psi$ is an $n$ qubit state.  Asymptotically we have $\chi(\psi^{\otimes N}) \leq (\chi(\psi^{\otimes t})^{1/t})^N$.  Since lower values lead to lower simulation overhead we see a significant advantage in using blocks of size up to 5.} 	\label{Tab_Numbers}
\end{table}

We reflect that we have proved Lem.~\ref{symmetric}, from which Thm.~\ref{Thm_many_copies} follows immediately.  For single qubit states ($n=1$) this entails that 
\begin{equation}
		\chi( \psi^{\otimes t}) \leq t + 1 , \forall t \leq 5 \label{EqOneQubit} .
\end{equation}	 
The rest of this subsection discusses numerical experiments into whether this inequality is tight.

Clearly the bound is loose for stabilizer states since then we have $\chi( \psi^{\otimes t})=1 < t+1$.  However, Clifford magic states are also exceptional for many $t$ values.  Bravyi, Smith and Smolin~\cite{Bravyi16stabRank} discuss the stabilizer rank of single qubit states that are an eigenstate of some Clifford unitary.  For instance, the $\ket{T}$  Clifford magic states are exceptional in that for $ 2 \leq t \leq 4$ we have that $\chi(T^{\otimes t})=t < t+1$, which we illustrate in Fig.~\ref{SingleQubitStates}.  We remark that $\ket{T}$ has the Clifford symmetry $C_T \ket{T}=\ket{T}$ for $C_T = TXT^\dagger$. In total there are 12 single qubit states in the Clifford orbit of $\ket{T}$. An additional class of Clifford symmetric states is the Clifford orbit of the face state $\ket{f}$ 
\begin{equation}
	\kb{f}{f} = \frac{1}{2} \left( \id + \frac{X + Y + Z}{\sqrt{3}} \right) ,
\end{equation}	
which comprises 8 different states.  The face state is an eigenstate of the Clifford $C_F=e^{-i \pi /12}SH$  that cyclically permutes Pauli $X$,$Y$ and $Z$.   Bravyi, Smith and Smolin reported (see conjecture 1 of Ref.~\cite{Bravyi16stabRank}) that $\chi(f^{\otimes t})$ appears to equal $\chi(T^{\otimes t})$, providing another class of states where Eq.~\eqref{EqOneQubit} is not tight.

Next, we ask if there are any other single qubit states for which Eq.~\eqref{EqOneQubit} is not tight.  We proceed by a heuristic, numerical search, extending the search method of Ref.~\cite{Bravyi16stabRank}. To find a decomposition of a state $\ket{\psi}$, we use an objective function $F_{\Psi}\left(\{\ket{\phi_{j}}\}\right)=||\Pi\ket{\Psi}||$ where $\Pi$ is a projector onto $\mathrm{span}\left(\{\ket{\phi_{j}}\}\right)$.   We start by choosing a set of $k$ random stabilizer states $\{\ket{\phi_{j}}\}$, with $k=2$ on the first run.  Random stabilizer states were obtained by generating a random binary matrix, using the algorithm of Garcia et al. to convert it to a canonical stabilizer tableau, and computing the corresponding state vector~\cite{garcia2012efficient}.  Let the value of the objective function at a given timestep be $F$. We update one stabilizer state in the set by applying a random Pauli projector, and evaluate the objective function on the new set $F_{\Psi}\left(\{\ket{\phi_{j}}\}'\right)=F'$. If $F'>F$ then we accept the move, otherwise the new decomposition is accepted with a probability $p=\text{exp}\left[-\beta\left(F-F'\right)\right]$, where $\beta$ is an inverse temperature parameter that decreases as the walk proceeds~\cite{Bravyi16stabRank}.  If $F$ equals 1 at any point in the walk, we halt and conclude $\chi(\Psi) \leq k$.  If $F$ does not converge to unity within a constant number of steps, we increment $k$ and start again.

Random typical states were generated as $\ket{\psi}=U\ket{0}$, where $U$ are Haar random unitaries.  We sampled 1000 Harr random states and numerically estimated the stabilizer rank of $\Psi=\psi^{\otimes t}$ using the above method. In every instance, the best decomposition we found saturated the inequalities of Eq.~\eqref{EqOneQubit}. We also examined conjecture 1 of~\cite{Bravyi16stabRank}, by searching for decompositions of single-qubit Clifford magic states. All decompositions found were below the bound of Eq.~\eqref{EqOneQubit}.

Although these numerical searches were not exhaustive, the results support the hypothesis that Eq.~\eqref{EqOneQubit} is an equality for typical single qubit states.  This supports the conjecture that Eq.~\eqref{EqOneQubit} is tight, if and only if the state has no Clifford symmetries. 

As a closing remark, we comment on consequences of these results for simulation overheads.  If a circuit contains many copies of the same multi-qubit phase gate, simulation overheads are reduced by working with blocks of magic states as shown in Table.~\ref{Tab_Numbers}.


\label{Sec_approx_stab_rank}

\subsection{Sparsification Lemma}
\label{Sec_Lemma_Proof}

Our new bounds  on the approximate stabilizer rank in Theorem~\ref{thm:randomCvec}
are obtained using the following   lemma.
It shows how to convert a stabilizer decomposition of some target state $\psi$ 
with a small $l_1$ norm to 
a sparse stabilizer decomposition of $\psi$.
\begin{lemma}[\bf Sparsification]
	\label{lem:randomCvec}
Let $\psi$ be a normalized $n$-qubit state with a decomposition $\ket{\psi} = \sum_j c_j \ket{\phi_j} $ where all $\phi_j $ are normalized stabilizer states and $c_j \in \mathbb{C}$.  For any integer $k$ there exists a distribution of random quantum states $|\Omega\rangle$ of the form 
$\ket{\Omega}=\frac{\|c\|_1}{k} \sum_{\alpha=1}^k \ket{\omega_\alpha}$ where each $\ket{\omega_\alpha}$ is (up to a global phase) one of the states $\{\ket{\phi_j} \}$ and
\begin{equation}
 \mathbb{E}\left( \, \| \psi -\Omega \|^2\right)  = \frac{\|c \|^2_1}{k},
\label{eq:meanomega}
\end{equation}
where $\|\vec{c} \|_1 := \sum_j |c_j|$ and $\| \psi \| = \sqrt{ \bk{\psi}{\psi} }$.
\end{lemma}	

Theorem~\ref{thm:randomCvec} is a simple corollary of Lemma~\ref{lem:randomCvec}.
Indeed, assume that all $\phi_j$ are stabilizer states. Choosing $k=( \| c \|_1 / \delta )^2 $ we find that the right-hand side is upper-bounded by $\delta^2$.  Therefore there exists at least one $|\Omega\rangle$ (which is manifestly a sum of $k$ stabilizer states) that $\delta$-approximates $|\psi\rangle$.  This proves Theorem~\ref{thm:randomCvec}. 

Note that we can use Markov's inequality and Eq.~\eqref{eq:meanomega} to lower bound the probability that a randomly chosen $\Omega$ is a good approximation to $\psi$, e.g., 

\[
\mathrm{Pr}\left[\| \psi -\Omega \|^2\geq 2\delta^2\right]\leq 1/2  \quad \text{for } \quad k\geq \frac{\|c \|^2_1}{\delta^2}.
\]
Suppose that we randomly choose some $|\Omega\rangle$ as prescribed above. Can we estimate how well it approximates $\psi$? The following Lemma can be used for this purpose.
\begin{lemma}[\bf Sparsification tail bound]
Let $\psi,\Omega,k$ be as in Lemma \ref{lem:randomCvec}.  If we choose $k\geq \frac{\|c \|^2_1}{\delta^2}$ then 
\begin{equation}
\mathbb{E}\left[\langle \Omega|\Omega\rangle-1\right] \leq \delta^2 ,
\label{eq:expomega}
\end{equation}
and
\begin{equation}
\mathrm{Pr}\left[\|\psi-\Omega\|^2\leq \langle \Omega|\Omega\rangle-1+\delta^2\right] \geq 1-2\exp{\left(-\frac{\delta^2}{8F(\psi)}\right)}.
\label{eq:event}
\end{equation}
\label{lem:tailbound}
\end{lemma}

Note that we are interested in cases where the stabilizer fidelity $F(\psi)$ is exponentially small as a function of the number of qubits $n$. In such cases the Lemma states that
\[
\|\psi-\Omega\|^2\leq \langle \Omega|\Omega\rangle-1+\delta^2 ,
\]
with all but vanishingly small probability if $n$ is sufficiently large. Moreover, the quantity $\langle \Omega|\Omega\rangle$ appearing in the above can be approximated to a given relative error using the norm estimation algorithm from Section \ref{Sec_fast_norm} which has runtime scaling linearly with $k$.  


\begin{proof}[Proof of Lemma~\ref{lem:randomCvec}]
Define a probability distribution $p_j := |c_j| / || c ||_1$ and write
\begin{equation}
\ket{\psi} = \|c\|_1\sum_j p_j 	\ket{W_j} 
\end{equation}	
where $\ket{W_j}:= (c_j/|c_j|) 	\ket{\phi_j} $ are normalized stabilizer states. Now define a random variable $|\omega\rangle$ which is equal to $|W_j\rangle$ with probability $p_j$.   Then
\begin{equation}
|\psi\rangle=\|c\|_1\mathbb{E}\left[|\omega\rangle\right].
\end{equation}	
Let $k$ be a positive integer and consider a random state 
\begin{equation}
|\Omega\rangle=\frac{\|c\|_1}{k}\sum_{\alpha=1}^{k} |\omega_\alpha\rangle ,
\label{eq:Omega}
\end{equation}
where $\omega_1,\omega_2,\ldots,\omega_k$ are i.i.d random copies of $|\omega\rangle$.  By construction,  on average we have
\begin{equation}
 \mathbb{E} [ \bk{\psi}{\Omega} ] =  \mathbb{E} [ \bk{\Omega}{\psi} ] = 1
\label{eq:overlap}
\end{equation}
even though for any particular random sample $ \bk{\Omega}{\psi} \neq 1$.  In general, not only will $\Omega$ not be proportional to $\psi$, but $\Omega$  will not be correctly normalized.  However, the normalization can be bounded in expectation as follows
\begin{align}
\mathbb{E}\left[\langle \Omega |\Omega\rangle\right]&=\frac{\|c\|_1^2}{k^2}\mathbb{E}\left[\sum_{\alpha=1}^{k} \langle \omega_\alpha|\omega_\alpha\rangle\right]+\frac{\|c\|_1^2}{k^2}\mathbb{E}\left[\sum_{\alpha\neq \beta} \langle \omega_\alpha|\omega_\beta\rangle\right]\\
&=\|c\|_1^2\frac{\mathbb{E}\left[\langle \omega|\omega\rangle\right]}{k}+\frac{1}{k^2} k(k-1)\\
&\leq 1+\frac{\|c\|_1^2}{k}
\label{eq:omom}
\end{align}
where in the second line we used the fact that $\|c\|_1^2\mathbb{E}\left[\langle \omega_\alpha|\omega_\beta\rangle\right]=\langle \psi|\psi\rangle$ for $\alpha\neq \beta$. 

We are interested in the expected error 
\begin{align}
 \mathbb{E}\left[ \|\psi\rangle -|\Omega\rangle\|^2\right] & = \mathbb{E}\left[ \bk{\Omega}{\Omega} \right]- \mathbb{E}\left[ \bk{\Omega}{\psi} \right]-\mathbb{E}\left[ \bk{\psi}{\Omega} \right]+\mathbb{E}\left[ \bk{\psi}{\psi} \right]
\label{eq:experror}
\end{align} 
Using  $\bk{\psi}{\psi}=1$, Eq.~\eqref{eq:omom} and Eq.~\eqref{eq:overlap} we find 
\begin{align}
  \mathbb{E}\left[ \|\psi\rangle -|\Omega\rangle\|^2\right] & =  \frac{\|c\|_1^2}{k}.
\end{align}
This completes the proof of Lemma ~\ref{lem:randomCvec}.
\end{proof}
\begin{proof}[Proof of Lemma \ref{lem:tailbound}]
Equation \eqref{eq:expomega} follows directly from Eq.~\eqref{eq:omom} and the choice of $k$. Define random variables
\[
X_\alpha=\|c\|_1 \mathrm{Re}(\langle \psi|\omega_\alpha\rangle) \qquad 1\leq\alpha \leq k
\]
and let 
\[
\bar{X}=\frac{1}{k}\sum_{\alpha=1}^{k} X_\alpha=\mathrm{Re}(\langle \psi|\Omega\rangle).
\]
Then
\begin{equation}
\left|\mathrm{Re}(\langle \psi|\Omega\rangle)-1\right|=\left|\bar{X}-\mathbb{E}[\bar{X}]\right|.
\label{eq:repart}
\end{equation}
Now $\bar{X}$ is a sample mean of $k$ independent and identically distributed random variables $X_\alpha$,  each of which is bounded as
\begin{equation}
|X_\alpha|\leq \|c\|_1 |\langle \psi|\omega_\alpha\rangle| \leq \|c\|_1 \sqrt{F(\psi)}
\label{eq:xkbound}
\end{equation}
where in the last inequality we used the definition of stabilizer fidelity. Applying Hoeffding's inequality \cite{hoeffding1963probability} and using Eqs.~(\ref{eq:repart}, \ref{eq:xkbound}) gives
\begin{equation}
\mathrm{Pr}\left[\left|\mathrm{Re}(\langle \psi|\Omega\rangle)-1\right|\geq \frac{\delta^2}{2}\right]\leq 2\exp\left(-\frac{2k\delta^4}{4\left(2\|c\|_1\sqrt{F(\psi)}\right)^2}\right) \leq 2\exp\left(-\frac{\delta^2}{8F(\psi)}\right)
\label{eq:hoeff}
\end{equation}
where we used $k\geq \|c\|_1^2/\delta^2$. Finally, applying the triangle inequality to Eq.~\eqref{eq:experror} gives
\begin{equation}
\|\psi-\Omega\|^2 \leq \langle \Omega|\Omega\rangle-1+2\left|1-\mathrm{Re}(\langle \psi|\Omega\rangle)\right| 
\label{eq:normdiff}
\end{equation}
Combining Eqs.~(\ref{eq:normdiff}, \ref{eq:hoeff}) completes the proof.


\end{proof}

\subsection{Approximate stabilizer rank of Clifford magic states}
\label{Sec_Clifford_Magic_States}
Proposition \ref{thm:clifmagic} asserts that $\xi(\psi)=F(\psi)^{-1}$ when $\psi$ is a Clifford magic state (Def \ref{Dfn_CMS}). In fact, this relation holds for a wider class of $\psi$ and we comment on this at the end of the following proof. Recall that a Clifford magic state $\psi$ is stabilized by a group of Clifford unitaries with generators $Q_j:=V X_j V^\dagger$.  We denote this group as $\mathcal{Q}:=\langle Q_j \rangle = \langle V X_j V^\dagger \rangle$. Here we describe upper bounds on the approximate stabilizer rank of Clifford magic states. We begin with the proof of Proposition \ref{thm:clifmagic}

\begin{proof}[Proof of Proposition \ref{thm:clifmagic}]
From the definition of Clifford magic states, we have
\begin{align}
P_\psi = |\psi\rangle\langle\psi| & = V \frac{1}{2^n} \prod_j (\id + X_j) V^\dagger \\ \nonumber
& = \frac{1}{2^n} \prod_j (\id + Q_j ) \\ \nonumber
& = \frac{1}{| \mathcal{Q} |} \sum_{q \in \mathcal{Q}} q 
\end{align}
Let $\phi_0$ be a stabilizer state such that $|\langle \psi|\phi_0\rangle|^2>0$. Then
\begin{align}
|\psi\rangle & = \frac{\ket{\psi}\bk{\psi}{\phi_0}}{\bk{\psi}{\phi_0}}  \label{eq:cmag1}\\ \nonumber
& = \left[ \frac{1}{|\mathcal{Q}|} \sum_{q\in \mathcal{Q}} q \right]  \frac{\ket{\phi_0}}{\bk{\psi}{\phi_0}} \\ \nonumber
& = \frac{1}{|\mathcal{Q}|\langle \psi|\phi_0\rangle}\sum_{q\in \mathcal{Q}}q|\phi_0\rangle.
\end{align}
Using Eq.~(\ref{eq:cmag1}) and the fact that $q|\phi_0\rangle$ is a stabilizer state for all $q\in \mathcal{Q}$ we immediately obtain
\[
|| \vec{c} ||_1^2 = \frac{1}{|\langle \psi|\phi_0\rangle|^2},
\]
for this decomposition.  To minimise $|| \vec{c} ||_1^2$ it is natural to use the stabilizer state with the larger possible overlap, $F(\psi)= \mathrm{max}_{\phi_0}|\langle \psi|\phi_0\rangle|^2$, which we call the stabilizer fidelity.  Therefore, once we have found a $\phi_0$ attaining the maximum, we have a decomposition achieving $|| \vec{c} ||_1^2 =  F(\psi)^{-1}$.   This discussion suffices to prove that 
\[
\xi(\psi) \leq F(\psi)^{-1}.
\]
To establish the converse consider any stabilizer decomposition 
\[
|\psi\rangle=\sum_{j=1}^{\chi}c_j |\phi_j\rangle.
\]
Taking the inner product with $\psi$ we get
\[
1=\left|\sum_{j=1}^{\chi}c_j \langle \psi |\phi_j\rangle \right|\leq \|c\|_1 \sqrt{F(\psi)},
\]
where we used the fact that $|\langle \psi|\phi_j\rangle|^2\leq F(\psi)$. Squaring the above completes the proof.

More generally, let $\mathcal{Q}$ be \emph{any} subgroup of the Clifford group satisfying $\kb{\psi}{\psi}=|\mathcal{Q}|^{-1}\sum_{q\in\mathcal{Q}} q$ and with exactly one group element (the identity) stabilizing $\ket{\phi_0}$. The above proof goes through unmodified, but admits a wider class of states for which $\xi(\psi)=F(\psi)^{-1}$ including the face state, $\ket{f}$, satisfying
\begin{align}
\kb{f}{f}=\frac{1}{2}\left(\id+\frac{(X+Y+Z)}{\sqrt{3}}\right)=\frac{1}{|\mathcal{Q}|}\sum_{q\in\mathcal{Q}} q
\end{align}
where $\mathcal{Q}=\{\id,C_F,C_F^2\}$ and $C_F=e^{-i \pi /12}SH$ is the Clifford that cyclically permutes Pauli $X$,$Y$ and $Z$. 
\end{proof}
The $\ket{T}^{\otimes n}$ state is the most well known example of a Clifford magic state.  It has been shown (see Lemma~2 of Ref.~\cite{Campbell11} or Lemma~2 of Ref.~\cite{bravyi2016improved}) that  $F(T^{\otimes n})^{-1}=|\bk{+}{T}|^{2n}$ and so $\ket{+}^{\otimes n}$ can be used to generate the decomposition with optimal $\xi(\psi)$.  Combining this with Lemma~\ref{lem:randomCvec} gives the same upper bound on $\chi_\delta(T^{\otimes n})$ as was previously shown in Ref.~\cite{bravyi2016improved}.   However, the techniques are slightly different.  Our Lemma~\ref{lem:randomCvec} randomly selects a subset of terms from the decomposition, whereas Ref.~\cite{bravyi2016improved} randomly select a subset of terms that form a random linear code.  We remark that the random linear code construction also generalises to all Clifford magic states. For any linear code $\mathcal{L} \subseteq \mathbb{F}_2^n$ we can associate a subgroup $\mathcal{Q}_\mathcal{L} \subseteq \mathcal{Q}$.  That is, given a decomposition as in Eq.~\eqref{eq:cmag1} with group $\mathcal{Q}$, we can choose a random subgroup $\mathcal{Q_L} \subseteq \mathcal{Q}$ and define the normalised approximate state 
\begin{equation}
	|\mathcal{L} \rangle \propto \sum_{q\in \mathcal{Q_L}}q|\phi_0\rangle .
\label{eq:cmag}
\end{equation}



Following analogous steps to those in Ref.~\cite{bravyi2016improved}, one can show that this approach gives the same asymptotic scaling of $\chi_\delta$ as in Lemma~\ref{lem:randomCvec}.  While the behaviour of $\chi_\delta$  is identical, it may be easier to implement a simulator working with random subgroups than random subsets. 

As a further example, let us consider the Clifford magic state corresponding to a CCZ (control-control-Z) gate,
\begin{align}
		\ket{CCZ} = CCZ \ket{+}\ket{+}\ket{+} = \frac{1}{\sqrt{8}} \sum_{a,b,c \in \{0,1\}} (-1)^{abc} \ket{a}\ket{b}\ket{c}
\end{align}
This magic state is the ``+1" eigenstate for a group $\mathcal{Q}$ with three generators of the form $CCZ \cdot X_j \cdot CCZ^\dagger$.  More explicitly these generators are
\begin{align}
	Q_1   & =  CCZ \cdot X_1 \cdot CCZ^\dagger  =  X_1  CZ_{2,3} \\ \nonumber
	Q_2  & =  CCZ \cdot X_2 \cdot CCZ^\dagger  =  X_2  CZ_{1,3} \\ \nonumber
	Q_3  & =  CCZ \cdot X_3 \cdot CCZ^\dagger =  X_3  CZ_{1,2} 
\end{align}
where $CZ_{i,j}$ denotes a control-Z between qubits $i$ and $j$.  One can straightforwardly confirm that $F(CCZ)= |\bk{+++}{CCZ}|^2=9/16$, and that
\begin{align}
	\label{CCZ_decomp}
	 \ket{CCZ} = \frac{1}{6}\sum_{Q \in \mathcal{Q}} Q\ket{+++} ,
\end{align}	
has $|| \vec{c} ||_1^2 = 16/9$.  Using this decomposition for many CCZ states shows $ \chi_\delta( CCZ^{\otimes t} ) \leq \delta^{-2} (9/16)^t \sim \delta^{-2} 1.778^t$.  Note that this is slower exponential scaling than obtained by synthesizing each CCZ with 4 $T$-gates and using $ \chi_\delta( T^{\otimes 4t} ) \leq \delta^{-2} 1.884^t$. It is conceivable that a better decomposition exists since $\xi$ only provides an upper bound on the approximate stabilizer rank. 

One could obtain better decompositions if the stabilizer fidelity is not multiplicative, but we show later (see Corollary~\ref{Cor_single_qubits}) that $F(T^{\otimes t})=F(T)^t$ and $F(CCZ^{\otimes t})=F(CCZ)^t$. However, one of the significant open questions remaining from this work is whether stabilizer fidelity is always multiplicative for all Clifford magic states.  Lastly, we remark that one can lift the above stabilizer decomposition to obtain a Clifford unitary decomposition of CCZ that can be used for an approximate sum-over-Cliffords simulator.

\subsection{Lower bound based on ultra-metric matrices}
\label{Sec_ultra}
Previous sections give explicit stabilizer decompositions of states and therefore upper bounds on the stabilizer rank.   Yet we have no techniques that provide lower bounds on the stabilizer rank that scale exponentially with the number of copies.  Here we present results in this direction. Let $|H\rangle=\cos{(\pi/8)}|0\rangle + \sin{(\pi/8)}|1\rangle$ be the magic state which is Clifford equivalent to $\ket{T}$. We would like to approximate $n$ copies of $|H\rangle$ by a low-rank linear combination
of stabilizer states 
\[
|\tilde{x}\rangle =|\tilde{x}_1\rangle \otimes \cdots \otimes |\tilde{x}_n\rangle
\quad \mbox{where} \quad
|\tilde{0}\rangle=|0\rangle \quad \mbox{and} \quad
|\tilde{1}\rangle =|+\rangle.
\]
Here we derive a lower bound on the rank of such approximations stated earlier as Prop.~\ref{prop:lower_bound}.  We first restate this result as follows
\begin{theorem}
\label{thm:main}
Suppose $S\subseteq \{0,1\}^n$ is an arbitrary subset
and $\phi$ is an arbitrary   linear combination of states
$|\tilde{x}\rangle$ with $x\in S$ such that $\|\phi\|=1$.  Then 
\begin{equation}
% \label{eq1}
|S|\ge |\langle H^{\otimes n}|\phi\rangle |^2 \cdot \cos{(\pi/8)}^{-2n}.
\end{equation}
\end{theorem}
\begin{proof}
Let $\chi=|S|$ and $S=\{x^1,x^2,\ldots,x^\chi\}$ for some bit strings $x^i$.
The orthogonal projector onto a linear subspace spanned by
the states $|\tilde{x}^1\rangle,\ldots,|\tilde{x}^\chi\rangle$ has the form
\begin{equation}
\label{eq2}
\Pi=\sum_{i,j=1}^\chi (G^{-1})_{i,j} |\tilde{x}^i\rangle\langle \tilde{x}^j|,
\end{equation}
where $G$ is the Gram matrix defined by
$G_{i,j}=\langle \tilde{x}^i | \tilde{x}^j\rangle = t^{|x^i \oplus x^j|}$,
with $t=2^{-1/2}$. Here and below $\oplus$ denotes addition of bit strings
modulo two.
Noting that $\langle\tilde{x}|H^{\otimes n}\rangle = \cos{(\pi/8)}^{n}$ for all $x$ one gets
\begin{equation}
\label{eq3}
|\langle H^{\otimes n}|\phi\rangle |^2 \le \langle H^{\otimes n} |\Pi|H^{\otimes n}\rangle
=\cos{(\pi/8)}^{2n} \sum_{i,j=1}^\chi (G^{-1})_{i,j}
\le \chi \cos{(\pi/8)}^{2n}.
\end{equation}
The last inequality follows from 
\begin{lemma}
Suppose $x^1,\ldots,x^\chi \in \{0,1\}^n$ are distinct bit strings
and $0<t < 1$ is a real number. 
Let $G$ be a matrix of size $\chi$ with entries
\begin{equation}
\label{Gt}
G_{i,j}=t^{|x^i \oplus x^j|}.
\end{equation}
Then $G$ is invertible and 
\begin{equation}
\label{ubound}
\sum_{i,j=1}^\chi (G^{-1})_{i,j} \le \chi.
\end{equation}
\end{lemma}
\begin{proof}
Let $|1\rangle,|2\rangle,\ldots,|\chi\rangle$ be the basis vectors of $\RR^\chi$
such that $G_{i,j}= \langle i|G|j\rangle$.
We claim that Eq.~(\ref{ubound}) holds whenever one can 
find a family of matrices $G_\sigma$ and probabilities $p_\sigma\ge 0$ such that
\begin{enumerate}
\item[(a)] $G=\sum_\sigma p_\sigma G_\sigma$ and $\sum_\sigma p_\sigma=1$
\item[(b)] $G_\sigma$ is positive definite 
\item[(c)] $0\le \langle i|G_\sigma|j\rangle \le 1$
and $\langle i|G_\sigma|i\rangle=1$ 
\item[(d)]  $\langle i|G_\sigma^{-1}|j\rangle \le 0$ for $i\ne j$ 
\end{enumerate}
Indeed, let $|e\rangle$ be the  all-ones vector, $|e\rangle=\sum_{i=1}^\chi |i\rangle$.
We have to prove that $\langle e|G^{-1}|e\rangle\le \chi$.
Conditions~(a,b) imply that  $G$ is positive definite (and thus invertible). 
Noting that  the function $f(x)=x^{-1}$ is operator convex on the interval $(0,\infty)$
one gets 
\begin{equation}
\label{upper1}
\langle e|G^{-1}|e\rangle \le \sum_\sigma p_\sigma \langle e|G_\sigma^{-1} |e\rangle.
\end{equation}
From conditions~(c,d) one gets
\[
\langle i|G^{-1}_\sigma|j\rangle \le \langle i|G^{-1}_\sigma|j\rangle \langle j|G_\sigma|i\rangle
\]
for $i\ne j$ with the equality for $i=j$. 
Therefore 
\begin{equation}
\label{upper2}
\langle e|G^{-1}_\sigma|e\rangle = \sum_{i,j=1}^\chi \langle i| G^{-1}_\sigma|j\rangle 
\le \sum_{i,j=1}^\chi \langle i| G^{-1}_\sigma|j\rangle  \langle j| G_\sigma|i\rangle  = \mbox{Tr}(G^{-1}_\sigma G_\sigma) =
\mbox{Tr}(I)= \chi.
\end{equation}
Substituting this into Eq.~(\ref{upper1}) gives $\langle e|G^{-1}|e\rangle\le \chi \sum_\sigma p_\sigma =\chi$,
as desired.

It remains to construct the requisite matrices $G_\sigma$.
Our construction is based on the so-called  {\em ultrametric matrices},
see Refs.~\cite{MMM,NabenVarga}.
\begin{dfn}
\label{dfn:UM}
A symmetric real matrix $A$ is called  ultrametric iff
$0\le A_{i,j}<1$ for $i\ne j$,  $A_{i,i}=1$, and 
\begin{equation}
\label{UM1}
A_{i,j} \ge \min{(A_{i,k}, A_{j,k})}
\quad \mbox{for all $i,j,k$}.
\end{equation}
\end{dfn}
The last condition demands that for any triple of elements $A_{i,j}$, $A_{i,k}$, $A_{j,k}$  the two smallest 
elements coincide. 
The following fact was established in Refs.~\cite{MMM,NabenVarga}.
\begin{fact}
\label{fact:UM}
Suppose $A$ is an ultrametric matrix. Then $A$ is invertible and 
positive definite. Furthermore,  $\langle i| A^{-1}|j\rangle \le 0$ for all $i\ne j$.
\end{fact}
Thus it suffices to show that $G$ is a probabilistic mixture of ultrametric matrices.
Indeed, if condition~(a) holds for some ultrametric matrices $G_\sigma$ then 
condition~(c) follows directly from Definition~\ref{dfn:UM}
while conditions~(b,d) follow from Fact~\ref{fact:UM}.

The first step is to equip the Boolean cube $\{0,1\}^n$
with a distance function that obeys an analogue of the ultrametricity condition
Eq.~(\ref{UM1}).
Given a pair of bit strings $x,y\in \{0,1\}^n$,
define $d(x,y)$ as the smallest integer $j\ge 0$ such that 
the last $n-j$ bits of $x$ and $y$ coincide (that is, $x_i=y_i$ for all $i>j$).
We set $d(x,y)=n$ if $x_n\ne y_n$.
Note that $d(x,y)$ is different from the Hamming distance.
For example, $d(101,111)=2$ and $d(101,100)=3$.
By definition $d(x,y)\in [0,n]$ and $d(x,y)=0$ iff $x=y$. 
Furthermore, $d(x,y)$ depends only on $x\oplus y$.
We claim that
\begin{equation}
\label{UM2}
d(x,y)\le \max{\{ d(x,z),d(z,y)\}}
\end{equation}
for any triple of strings $x,y,z$. Indeed,
let $j=\max{\{ d(x,z),d(z,y)\}}$. Then
$x_i=z_i=y_i$ for all $i>j$, that is, 
$d(x,y)\le j$.

Suppose $q_w$ is a normalized probability distribution on the set of integers
$w=0,1,\ldots,n$ such that $q_w>0$ for all $w$. 
Define a $\chi\times \chi$ matrix $A$ such that 
\begin{equation}
\label{Aij}
A_{i,j}= \sum_{w\ge d(x^i,x^j)} \;  q_w.
\end{equation}
Here $x^i$ and $x^j$ are the bit strings from the statement of the lemma.
We claim that $A$ is ultrametric (according to Definition~\ref{dfn:UM}).
Indeed, consider any triple $i,j,k$ as in Eq.~(\ref{UM1}) and assume wlog that
$A_{i,k}\le A_{j,k}$. 
Since the matrix element $A_{i,j}$ is a monotone decreasing function
of the distance $d(x^i,x^j)$, we get $d(x^i,x^k)\ge d(x^j,x^k)$.
Then Eq.~(\ref{UM2}) gives $d(x^i,x^j)\le d(x^i,x^k)$.
Using the monotonicity again one gets $A_{i,j}\ge A_{i,k}=\min{\{A_{i,k},A_{j,k}\}}$,
confirming Eq.~(\ref{UM1}). 
The remaining conditions $0\le A_{i,j}<1$ for $i\ne j$ and $A_{i,i}=1$ follow from the 
assumption that all bit strings $x^i$ are distinct and that $q_w$ is a 
normalized probability distribution.
Thus the matrix $A$ defined by Eq.~(\ref{Aij}) is indeed ultrametric.

We are now ready to define a family of ultrametric matrices $G_\sigma$
such that $G=\sum_\sigma p_\sigma G_\sigma$.
Let us choose the label $\sigma$ as a permutation of $n$ integers, $\sigma \in S_n$.
The distribution $p_\sigma$ will be the uniform distribution on the symmetric group, that is,
$p_\sigma=1/n!$ for all $\sigma\in S_n$.
Given a permutation $\sigma$ and a bit string $x\in \{0,1\}^n$ let $\sigma(x)\in \{0,1\}^n$
be the result of permuting bits of $x$ according to $\sigma$.
We set 
\begin{equation}
\label{Gsigma1}
\langle i|G_\sigma |j\rangle = \sum_{w\ge d(\sigma(x^i),\sigma(x^j))} \;  q_w.
\end{equation}
The same argument as above confirms that $G_\sigma$ is ultrametric for any permutation $\sigma$.
Define 
\begin{equation}
\label{Gsigma2}
G'=\frac1{n!} \sum_{\sigma \in S_n} G_\sigma.
\end{equation}
We claim that  $\langle i|G'|j\rangle = \langle i|G|j\rangle = t^{|x^i\oplus x^j|}$
for a suitable choice of probabilities $q_w$.
Indeed, the identity $d(x,y)=d(0^n,x\oplus y)$ implies that 
a matrix element $\langle i |G_\sigma|j\rangle$ depends only on $x^i\oplus x^j$.
By the symmetry, matrix elements $\langle i| G'|j\rangle$ depend only on the Hamming
weight $h=|x^i\oplus x^j|$. Therefore it suffices to compute
$\langle i |G'|j\rangle$ for the special case when $x^i=0^n$ is the all-zero string
and $x^j$ is any fixed bit string with the Hamming weight $h$, for example,
$x^j=1^h 0^{n-h}$. Then
\begin{equation}
\label{Gsigma3}
\langle i|G'|j\rangle=
\frac1{n!} \sum_{\sigma \in S_n} \; \;  \sum_{w\ge d(0^n,\sigma(1^h0^{n-h}))} \;  q_w.
\end{equation}
By definition of the distance $d(x,y)$ one gets $d(0^n,\sigma(1^h0^{n-h}))\le w$ iff 
$h \le w$ and $\sigma_1,\ldots,\sigma_h\le w$.  The number of such permutations $\sigma$ is ${w \choose h} h! (n-h)!$.
Exchanging the sums over $\sigma$ and $w$ in Eq.~(\ref{Gsigma3})  one gets
\begin{equation}
\label{Gsigma4}
\langle i|G'|j\rangle=
\frac1{n!} \sum_{w=h}^n {w \choose h} h! (n-h)! \,q_w.
\end{equation}
We shall choose $q_w$ as a binomial distribution,
\begin{equation}
\label{qw}
q_w= {n \choose w} t^w (1-t)^{n-w}.
\end{equation}
Substituting Eq.~(\ref{qw}) into Eq.~(\ref{Gsigma4}) 
and introducing a variable $p=w-h$ one gets
\begin{equation}
\label{Gsigma5}
\langle i|G'|j\rangle=
\sum_{p=0}^{n-h}  {n-h \choose p} t^{p+h} (1-t)^{n-h-p}  = t^h.
\end{equation}
By definition, $h=|x^i\oplus x^j|$, so that  $G'=G$ as claimed.
Thus $G$ is indeed a probabilisitic mixture of ultrametric matrices
and the lemma is proved.
\end{proof}
\end{proof}







