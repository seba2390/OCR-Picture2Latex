\label{detailed}
Recall that the Clifford group is a group of unitary $n$-qubit operators
generated by  single-qubit and two-qubit gates
from the set $\{H,S,CX\}$. Here $H$ is the Hadamard gate,
$S=|0\ra\la 0|+i|1\ra\la 1|$ is the phase shift gate, 
and CX=CNOT is the controlled-X gate.
Stabilizer states are $n$-qubit states of the form $|\phi\ra=U|0^n\ra$, where $U$ is a
Clifford operator. We also use $X_j,Y_j,Z_j$ to denote Pauli operators
acting on the $j$-th qubit. Below we also make use of the stabilizer formalism, and refer the unfamiliar reader to the existing literature~\cite{nielsen2002quantum}. 


\subsection{Tools for constructing low-rank stabilizer decompositions}
\label{sec:rank_properties}
In this section we summarize our results pertaining to the stabilizer rank and describe methods of decomposing a state into a superposition of stabilizer states.  A reader interested only in the application for simulation of quantum circuits may wish to proceed to Sections~\ref{sec:subroutines}, \ref{sec:simulations}.

\begin{dfn}[\bf Exact stabilizer rank, $\mathbf{\chi}$~\cite{Bravyi16stabRank}]
Suppose $\psi$ is a pure $n$-qubit state. 
The  exact stabilizer rank $\chi(\psi)$ is the smallest integer $k$ such that 
$\psi$ can be written as 
	\begin{equation}
	\label{srank1}
	\vert  \psi \rangle  = \sum_{\alpha=1}^{k} c_\alpha \vert \phi_\alpha \rangle	,
	\end{equation}	
for some  $n$-qubit stabilizer states $\phi_\alpha$ and some complex coefficients $c_\alpha$. 
\end{dfn}
\noindent
By definition, $\chi(\psi)\ge 1$ for all $\psi$ and $\chi(\psi)=1$ iff $\psi$ is a stabilizer state.

\begin{dfn}[\bf Approximate stabilizer rank, $\mathbf{\chi_\delta}$~\cite{bravyi2016improved}]
Suppose $\psi$ is a pure $n$-qubit state such that $\|\psi\|=1$.
Let  $\delta>0$ be a precision parameter.
The approximate stabilizer rank $\chi_\delta(\psi)$ is the smallest integer $k$ such that 
$\| \psi -\psi'\|\le \delta$ for some state $\psi'$ with exact stabilizer rank $k$.
\end{dfn}
\noindent
Note that this definition differs slightly from the one from Ref.~\cite{bravyi2016improved} which is based on the fidelity.
 Our first result provides an upper bound on the approximate stabilizer rank.
\begin{theorem}[\bf Upper bound on $\mathbf{\chi_\delta}$]
\label{thm:randomCvec}
Let $\psi$ be a normalized $n$-qubit state with a stabilizer decomposition 
$\ket{\psi} = \sum_{\alpha=1}^k c_\alpha 	|\phi_\alpha\ra$
where $\ket{\phi_\alpha}$ are normalized stabilizer states and $c_\alpha\in \mathbb{C}$.  Then
\begin{equation}
\chi_\delta(\psi) \leq 1+ \| \vec{c} \|_1^2 / \delta^2 .
\end{equation}
Here $\| \vec{c}\|_1 \equiv  \sum_{\alpha=1}^k |c_\alpha|$. 
\end{theorem}	
We note that the stabilizer decomposition $\ket{\psi} = \sum_{\alpha=1}^k c_\alpha 
|\phi_\alpha\ra$
in the statement of the theorem 
does not have to be
optimal. For example,  it may include all  stabilizer states. The proof of the theorem is provided in Section~\ref{Sec_approx_stab_rank}. It is constructive in the sense that it provides a method of calculating
a state $\psi'$ which is a superposition of $\chi'\approx \delta^{-2}  \| \vec{c} \|_1^2$ stabilizer states 
such that $\|\psi'-\psi\|\le \delta$. Such a state $\psi'$ is obtained 
using a randomized sparsification method. It works by sampling
$\chi'$ stabilizer states $\phi_\alpha$ from the given stabilizer decomposition of $\psi$ at random
with probabilities proportional to $|c_\alpha|$. The state $\psi'$ is then defined as
a superposition of the sampled states $\phi_\alpha$ with equal weights, 
see the Sparsification Lemma and related discussion in Section~\ref{Sec_Lemma_Proof}. The theorem motivates the following definition. 
\begin{dfn}[\bf Stabilizer Extent, $\xi$]
	\label{Cstar_states}
Suppose $\psi$ is a normalized $n$-qubit state. 
Define the stabilizer extent $\xi(\psi)$ as the minimum of $\| \vec{c} \|^2_1$ over all
stabilizer decompositions  $|\psi\ra=\sum_{\alpha=1}^k c_\alpha |\phi_\alpha\ra$
where $\phi_\alpha$ are normalized stabilizer states. 
\end{dfn}
The theorem immediately implies that 
\begin{equation}
\label{cstar}
\chi_\delta(\psi) \leq 1+\xi(\psi) / \delta^2.
\end{equation}
While it is difficult to compute or prove tight bounds for the exact or approximate stabilizer rank, 
we find that $\xi(\psi)$ is a more amenable quantity that can 
be calculated   for many states  $\psi$ relevant in the context of quantum circuit
simulation. 
In particular, we prove 
\begin{prop}[\textbf{Multiplicativity of Stabilizer Extent}]
	\label{multi}
Let $\{ \psi_1,\psi_2,\ldots, \psi_L \}$ be any set of states 
such that each state $\psi_j$ describes a system of at most three qubits.
	Then
	\begin{equation}
	\xi( \psi_1 \otimes \psi_2  \otimes  \ldots \otimes \psi_L) = \prod_{j=1}^L \xi(\psi_j)  .
	\end{equation}
	\label{thm:prod}
\end{prop}
\noindent
This shows that the upper bound of Theorem~\ref{thm:randomCvec} is multiplicative under tensor product in the case of few-qubit states. It remains open whether $\xi$ is multiplicative on arbitrary collections of states.  

The proof of Proposition~\ref{thm:prod} is provided in Section \ref{Sec_Cstar_multi}.  It uses the fact that standard convex duality provides a characterization of $\xi$ in terms of the following quantity. 
\begin{dfn}[\bf Stabilizer Fidelity, $F$]
The stabilizer fidelity, $F(\psi)$, of a state $\psi$ is 
\begin{equation}
F( \psi ) = \mathrm{max}_{\phi} |\bk{\phi}{\psi}|^2,
\label{eq:stabfid}
\end{equation}
where the maximization is over all normalized stabilizer states $\phi$.
\end{dfn}
Proposition \ref{thm:prod} is obtained as a consequence of new results concerning multiplicativity of the stabilizer fidelity. In particular, we apply the classification of entanglement
in three-partite stabilizer states~\cite{ghz}
to derive conditions for the multiplicativity  of $F(\psi)$. 
More precisely, we define a set of quantum states $\mathcal{S}$ which
we call {\em stabilizer aligned} such that 
$F(\phi \otimes \psi) =F(\phi)F(\psi)$ whenever $\phi,\psi\in \mathcal{S}$. 
A state  $\psi$ is called stabilizer aligned
if  the overlap between $\psi$ and any stabilizer projector of rank $2^k$
is at most $2^{k/2} F(\psi)$. 
Remarkably, the set of stabilizer aligned states is closed under tensor product, that is $\phi\otimes \psi \in \mathcal{S}$ whenever $\phi,\psi\in \mathcal{S}$.
Moreover, we show that the stabilizer fidelity is not multiplicative for all states
$\phi \notin \calS$. That is, for any $\phi \notin \calS$ there exists a state $\psi$
such that $F(\phi\otimes \psi)>F(\phi) F(\psi)$. In that sense, our results provide
necessary and sufficient conditions under which the stabilizer fidelity is multiplicative. 

Proposition \ref{thm:prod} enables computation of $\xi(\psi)$ if $\psi$ is a tensor product of
few-qubit states (that involve at most three qubits).
We now describe another large subclass of states $\psi$ relevant for quantum circuit simulation for which we are able to compute $\xi$. To describe these states, 
recall that any diagonal $t$-qubit gate $V$ can be performed using 
a state-injection gadget that contains only stabilizer operations and consumes an ancillary state
$\ket{V}=V\ket{+}^{\otimes t}$ (see the discussion in Section~\ref{sec:simulations} and Figure~\ref{fig_injection}). Here and below $|+\rangle \equiv (|0\rangle + |1\rangle)/\sqrt{2}$.
The gadget also involves a computational basis measurement over $t$ qubits. Let $\vec{x}\in \{0,1\}^t$ be a string of measurement outcomes.
The desired gate $V$ is performed whenever $\vec{x}=0^t$.
However, given some other outcome $\vec{x}\ne 0^t$,
the gadget implements a gate  $V_\vec{x}=C_\vec{x} V$ where 
\[
C_\vec{x}=\prod_{j\, : \, x_j=1} VX_jV^\dag ,
\]
is the required correction, where $X_j$ is the Pauli $X$ operator acting on the jth qubit.
A special class of unitaries are those where the correction $C_\vec{x}$ is always a Clifford operator.
In this case a unitary gate $V$ is equivalent to the preparation of the ancillary state $|V\ra$
modulo stabilizer operations. 
This motivates the following definition.

\begin{dfn}[\bf Clifford magic states]
\label{Dfn_CMS}
Let $V$ be a diagonal $t$-qubit unitary such that $V X_j V^\dagger$ is a Clifford operator for all 
$j$. Such unitary $V$ is said to belong to the $3^{\mathrm{rd}}$ level of the Clifford hierarchy (see e.g. Ref.~\cite{CliffHier}).   The ancillary state  $\vert V \rangle \equiv V \vert +\rangle^{\otimes t}$ is called
a Clifford magic state. 
\end{dfn}
\noindent
For example, $|T\ra^{\otimes m}$   is a Clifford magic state for any integer $m$.
Note that in general the set of Clifford magic states is closed under tensor product. 
\begin{prop}
Let $\psi$ be a Clifford magic state. Then 
$\xi(\psi)=F(\psi)^{-1}$.
	\label{thm:clifmagic}
\end{prop} 
The proof of Proposition \ref{thm:clifmagic} is provided in Section \ref{Sec_Clifford_Magic_States} where it is extended to a slightly broader class of $\psi$. 

We note that $|T^{\otimes m}\rangle$ is  a Clifford magic state \textit{and} a product state and so either Proposition \ref{thm:prod} or Proposition \ref{thm:clifmagic} could be used along with Eq.~(\ref{cstar}) to upper bound its approximate stabilizer rank. In this way one can easily reproduce the upper bound obtained in Ref.~\cite{bravyi2016improved},
namely, 
\begin{equation}
\chi_\delta(T^{\otimes m}) \le O\left(\delta^{-2} \cos{(\pi/8)}^{-2m}\right).
\label{eq:chiT}
\end{equation}
This stands in sharp contrast with the best known lower bound $\chi(T^{\otimes m})=\Omega(m^{1/2})$ established in Ref.~\cite{Bravyi16stabRank}. It should be expected that the stabilizer rank (either exact or approximate)
of the magic states $T^{\otimes m}$ grows exponentially with $m$ 
in the limit $m\to \infty$. Indeed, the polynomial scaling of $\chi_{\delta}(T^{\otimes m})$ with $m$ for a suitably small constant $\delta$, or $\chi(T^{\otimes m})$, would entail complexity theoretic heresies such as BQP=BPP,  or P=NP~\footnote{By simulating a postselective quantum circuit one could solve 3-SAT using a polynomial number of T-gates, see e.g., Ref. \cite{alibaba}.}.  
Remarkably, we have no techniques for proving unconditional super-polynomial
lower bounds.
Here we made partial progress by solving a simplified problem
where stabilizer decompositions of $T^{\otimes m}$ are restricted to
certain product states. For this simplified setting
we prove a tight lower bound on the approximate stabilizer rank of $T^{\otimes m}$
matching the upper bound of Ref.~\cite{bravyi2016improved}.
 To state our result it is more
convenient to work with the magic state $|H\ra=\cos{(\pi/8)}|0\ra + \sin{(\pi/8)}|1\ra$
which is equivalent to $|T\ra$ modulo Clifford gates.
Ref.~\cite{bravyi2016improved} showed that $|H^{\otimes m}\ra$ admits
an approximate stabilizer decomposition $|H^{\otimes m}\ra \approx \sum_{\alpha=1}^k c_\alpha |\phi_\alpha\ra$
where $k\sim \cos{(\pi/8)}^{-2m}$ and $\phi_\alpha$ are product
stabilizer states of the form 
\begin{equation}
|\tilde{x}\rangle =|\tilde{x}_1\rangle \otimes \cdots \otimes |\tilde{x}_m\rangle
\quad \mbox{where} \quad
|\tilde{0}\rangle=|0\rangle \quad \mbox{and} \quad
|\tilde{1}\rangle =|+\rangle. \label{eqn:Hbasis}
\end{equation}
Here $x_i\in \{0,1\}$.
These are the stabilizer states that achieve the maximum overlap with $|H^{\otimes m}\ra$, see Ref.~\cite{bravyi2016improved}.  Here we prove the following lower bound.
\begin{prop}
\label{prop:lower_bound}
Suppose $S\subseteq \{0,1\}^m$ is an arbitrary subset
and $\psi$ is an arbitrary linear combination of states
$|\vec{\tilde{x}}\rangle$ as in \eqref{eqn:Hbasis} with $\vec{x}\in S$ such that $\|\psi\|=1$.  Then 
\begin{equation}
\label{eq1ultra}
|S|\ge |\langle H^{\otimes m}|\psi\rangle |^2 \cdot \cos{(\pi/8)}^{-2m}.
\end{equation}
\end{prop}
The proof of this result which is given in Section \ref{Sec_ultra} makes use of the machinery of ultra-metric  matrices~\cite{MMM,NabenVarga}. We hope that these techniques may lead to further progress   on lower bounding the stabilizer rank.



We conclude this section by summarizing our results pertaining to the exact stabilizer rank.
Prior work focused exclusively on finding the stabilizer rank of $m$-fold tensor products
of magic state $|T\ra$.
A surprising and counter-intuitive result of Ref.~\cite{Bravyi16stabRank}
is that  for small number of magic states ($m\le 6$) the stabilizer rank $\chi(T^{\otimes m})$
scales linearly with $m$. Meanwhile,   $\chi(T^{\otimes m})$ is expected to 
scale exponentially with $m$ in the  limit $m\to \infty$.
Using a numerical search we observed a sharp jump from $\chi(T^{\otimes 6})=7$
to $\chi(T^{\otimes 7})=12$ indicating a transition from the linear to the exponential scaling
at $m=7$. This poses the question of whether other magic states have a linearly scaling stabilizer rank 
(until some critical $m$ is reached) or if $|T\ra$ is an exceptional state due to its special symmetries.
Here we show that the linear scaling for small $m$ is a generic feature.  
\begin{theorem}[\bf Upper bound on $\mathbf{\chi}$]
	\label{Thm_many_copies}
	Let $\psi$ be an $n$-qubit state and then for all $m \leq 5$ we have
	\begin{equation}
	\chi( \psi^{\otimes m} ) \leq \binom{2^n + m -1}{m}
	\end{equation}
	where the round brackets denote the binomial coefficient.
\end{theorem}
For example, this result shows that for any diagonal single-qubit unitary $V$ the associated magic state $\ket{V}$
obeys $\chi(\ket{V}^{\otimes m})\le m+1$ for $m \leq 5$.  For larger $m$, an exponential scaling is expected.  
The proof of Theorem~\ref{Thm_many_copies} (given in Section \ref{Sec_Symmetric_states}) exploits well-known properties of the symmetric subspace and a recently established fact that $n$-qubit stabilizer states form a 3-design~\cite{Webb16,kueng15}. 

\subsection{Subroutines for manipulating low-rank stabilizer decompositions}
\label{sec:subroutines}
Suppose $U$ is a quantum circuit acting on $n$ qubits.
We consider a classical simulation task where the goal is to
sample a bit string $x\in \{0,1\}^n$ from the probability
distribution $P_U(x)=|\langle x|U|0^n\rangle|^2$ with a small statistical error. 

Suppose we are given an approximate stabilizer decomposition of a state $U\ket{0^n}$:
\be
\label{approxU}
\|U|0^n\ra - |\psi\ra\|\le \delta, \qquad |\psi\ra = \sum_{\alpha=1}^k b_\alpha U_\alpha |0^n\ra
\ee
for some coefficients $b_\alpha$ and some Clifford circuits $U_\alpha$. In Section \ref{algorithms} we give algorithms for the following tasks. These algorithms are the main subroutines used in our quantum circuit simulators.
\begin{enumerate}
\item[\bf (a)] Sample $\vec{x}\in \{0,1\}^n$ from the probability distribution
\be
\label{P(x)normalized}
P(\vec{x})=\frac{|\la \vec{x}|\psi\ra|^2}{\|\psi\|^2}.
\ee
\item[\bf (b)] Estimate the norm $\|\psi\|^2$ with a small multiplicative error.
\end{enumerate}
Note that if $\delta$ is small then $P(\vec{x})$ approximates the true  output distribution $P_U(\vec{x})=|\la \vec{x}|U|0^n\ra|^2$ with a small error. Indeed, Eq.~(\ref{approxU}) gives $\| P - P_U\|_1 \le O(\delta)$.

The tasks (a,b)  are closely related. Using the chain rule for conditional probabilities
one can  reduce the sampling task 
to estimation of marginal probabilities of $P(\vec{x})$.
Any marginal probability can be expressed 
as  $\| \Pi \psi \|^2/\|\psi\|^2$, where $\Pi$
is a tensor product of projectors $|0\ra\la 0|$, $|1\ra\la 1|$, and the identity operators. Note that such projectors map stabilizer states to stabilizer states.
Thus $\Pi |\psi\ra$ admits a stabilizer decomposition with $k$ terms 
that can be easily obtained from Eq.~(\ref{approxU}).
Accordingly, 
task~(a) reduces to a sequence of norm estimations for low-rank stabilizer superpositions,
see Section~\ref{Sec_fast_norm} for details.

Section~\ref{clifford_sim} describes a fast Clifford simulator that transforms a
stabilizer state $U_\alpha|0^n\ra$ 
into a certain canonical form 
which we call a CH-form. 
It is analogous to the stabilizer
tableaux~\cite{aaronson04improved} but includes information about the global phase of a state. 
This allows us to simulate each circuit $U_\alpha$ 
in the superposition Eq.~(\ref{approxU})
independently without destroying information about 
the relative phases.  Our C++ implementation of the simulator
performs approximately $5\times 10^6$ Clifford gates per second
for $n=64$ qubits on a laptop computer.

Section~\ref{heuristic} describes a heuristic algorithm for the task~(a).
We construct a Metropolis-type Markov chain such that $P(x)$ is the unique steady distribution of the chain
(under mild additional assumptions).
We show how to implement each Metropolis step in time $O(k n)$.
Unfortunately, the mixing time of the chain is generally unknown.

Section~\ref{Sec_fast_norm} gives an algorithm for the task~(b).
It exploits the fact that the inner product $\la \phi|\phi'\ra$  between two $n$-qubit stabilizer
states $\phi,\phi'$ can be computed exactly in time $O(n^3)$,
see Ref.~\cite{garcia2012efficient,bravyi2016improved}.
We adapt this inner product algorithm to the CH-form 
of stabilizer states in Section~\ref{Sec_fast_norm}.
The naive method of computing the norm relies on the identity
$\|\psi\|^2=\sum_{\alpha,\beta=1}^k b_\alpha^* b_\beta \la \phi_\alpha|\phi_\beta\ra$, where
$|\phi_\alpha\ra=U_\alpha|0^n\ra$. Evaluating all cross terms using the inner 
product algorithm would take time $O(k^2 n^3)$ which
is impractical for large $k$.
Instead, Ref.~\cite{bravyi2016improved} proposed 
a method of estimating, rather than evaluating, the norm.  
It works by computing inner products between $\psi$ and random stabilizer states
drawn from the uniform distribution. 
This method has runtime $O(k n^3)$
offering a significant speedup in the relevant regime of large rank decompositions. 
Here we propose an improved version of this norm estimation 
method combining both conceptual  and implementation improvements.
The new version of the norm estimation subroutine achieves approximately 
50X speedup compared with Ref.~\cite{bravyi2016improved}.

Section~\ref{Sec_fast_norm}  also
describes a rigorous  algorithm  for the task~(a)
based on the norm estimation and the chain rule for conditional probabilities.
It has runtime $O(k n^6)$ which quickly becomes impractical.
However,  if our goal is to sample only $w$ bits from $P(x)$, 
the runtime is only  $O(k n^3 w^3)$.
Thus the sampling method based on the norm estimation may be practical for small values of $w$.

\subsection{Simulation algorithms}
\label{sec:simulations}
Here we describe how to combine ingredients from previous sections to obtain classical simulation algorithms for quantum circuits.  We consider a circuit
\begin{equation}
U=D_m V_mD_{m-1}V_{m-1}\ldots D_1 V_1 D_0
\label{eq:circuit}
\end{equation}
acting on input state $|0^n\rangle$, where $\{D_j\}$ are Clifford circuits and $\{V_j\}$ are non-Clifford gates.  We discuss three different methods:  gadget-based simulation (using either a fixed-sample or random-sample method as described below) and sum-over-Cliffords simulation.  

Let us first summarize the simulation cost of different methods. The gadget-based methods from Refs.~\cite{Bravyi16stabRank, bravyi2016improved} can be used to simulate quantum circuits Eq.~\eqref{eq:circuit} where $\{V_j\}$ are single-qubit T gates. Using the (random-sample) gadget-based method, the asymptotic cost of sampling from a distribution $\delta$-close in total variation distance to the output distribution $P_U(x)=|\langle x|U|0^n\rangle|^2$ is
\begin{equation}
\tilde{O}\left(\chi_{\delta}\left(|T^{\otimes m}\rangle\right)\right)\leq \tilde{O}\left(\delta^{-2} \xi\left(|T^{\otimes m}\rangle\right)\right)=\tilde{O}\left(\delta^{-2}\left(\cos(\pi/8)\right)^{-2m}\right),
\label{eq:CTscaling}
\end{equation}
where we used Theorem \ref{thm:randomCvec} and Proposition~\ref{thm:prod}, and the $\tilde{O}$-notation suppresses a factor polynomial in $m$, $n$, and $\log(\delta^{-1})$, see Ref.~\cite{bravyi2016improved} for details. 


We will see how the gadget-based approach can be applied in a slightly more general setting where the circuit contains diagonal gates from the third level of the Clifford hierarchy. Then we introduce the sum-over-Cliffords simulation method which can be applied much more generally.  The cost of $\delta$-approximately sampling from the output distribution $P_U$ for the circuit Eq.~\eqref{eq:circuit} using the sum-over-Cliffords method can be upper bounded as
\begin{equation}
\tilde{O}\bigg(\delta^{-2} \prod_{j=1}^{m} \xi(V_j)\bigg)
\label{eq:zrot}
\end{equation}
where the definition of $\xi$ is extended to unitary matrices in a natural way (see below for a formal definition). For example, if each non-Clifford gate is a single-qubit diagonal rotation of the form $V_j=R(\theta_j)=e^{-i(\theta_j/2) Z}$ with $\theta_j\in [0, \pi/2)$ then we will see that $\xi(V_j)=\xi(V_j|+\rangle)$ and the simulation cost is
\[
\tilde{O}\bigg(\delta^{-2} \prod_{j=1}^{m} \xi(V_j|+\rangle)\bigg)=\tilde{O}\bigg(\delta^{-2} \prod_{j=1}^{m} \left(\cos(\theta_j/2)+\tan(\pi/8)\sin(\theta_j/2)\right)^2\bigg).
\]
In the case $\theta_j=\pi/4$ where all non-Cliffords are $T$ gates, we see that the sum-over-Cliffords method achieves the same asymptotic cost Eq.~\eqref{eq:CTscaling} as the gadget-based method from Ref.~\cite{bravyi2016improved}. However the sum-over-Cliffords method is generally preferred because it is simpler to implement and may be slightly faster, as it manipulates stabilizer states of fewer qubits.


\subsubsection{Gadget-based methods}
\label{sec:gadgetbased}
We begin by reviewing the gadget-based methods for simulating circuits expressed over the Clifford+T gate set. A gadget-based simulation directly emulates the operation of a quantum computer that can implement Clifford operations and has access to a supply of magic states.

It is well known that one can perform such a gate on a quantum computer using a state-injection gadget with classical feedforward dependent on measurement outcomes.  In particular, a $t$-qubit gate $V$ can be implemented by a gadget consuming a magic state $\ket{V}=V \ket{+^ t}$, see  Fig.~\ref{fig_injection} for an example. Let $x\in \{0,1\}^t$ be the measurement outcome.  The gadget implements the desired gate $V$ whenever $x=0^t$.  Otherwise, if  ${x} \neq {0^t}$, the gadget implements a gate $V_x=C^\dagger_x V$ where $C_x$ is the required correction.  If $V$ is in the third level of the Clifford hierarchy, the correction $C_x$ is always a Clifford operator and $\ket{V}$ is a Clifford magic state (recall Definition~\ref{Dfn_CMS}).  Formally, postselecting on outcome $x=0^t$ gives
\begin{equation}
V|\psi\rangle= 2^{t/2} (\id \otimes \bra{0}^{\otimes t}) C^{\prime} |\psi\rangle  \ket{V},
\label{eq:singlegate}
\end{equation}
where $C^{\prime}=\left(\prod_{a=1}^{t} \mathrm{CNOT}_{a,a+t}\right)$ is a Clifford unitary.


Now let $U$ from Eq.~\eqref{eq:circuit} be the full circuit to be simulated and suppose $V_j$ is a diagonal $t_j$-qubit gate. Write $\tau=t_1+t_2\ldots+t_m$. If we replace each non-Clifford gate with the corresponding state-injection gadget we obtain a ``gadgetized'' circuit with $n+\tau$ qubits acting on input state $|0^n\rangle|V_1\rangle|V_2\rangle\ldots |V_m\rangle$.  The gadgetized circuit contains $\tau$ extra single-qubit measurements and Clifford gates. If we postselect the measurement outcomes on $0^{\tau}$ we obtain an identity (cf. Eq.~\eqref{eq:singlegate})
\begin{equation}
U|0^n\rangle=2^{\tau/2} (\id \otimes \bra{0}^{\otimes \tau}) C|0^n\rangle|\Psi\rangle\qquad \quad |\Psi\rangle=|V_1\rangle|V_2\rangle\ldots |V_m\rangle
\label{eq:Uidentity}
\end{equation}
where $C$ is an $n+\tau$-qubit Clifford unitary and we have collected together all of the required magic states into the $\tau$-qubit state $\Psi$. We see a renormalisation factor $2^{\tau/2}$ is required to account for post-selection. 

Eq.~\eqref{eq:Uidentity} shows that the output state $U|0^n\rangle$ of interest has exact stabilizer rank equal to that of the magic state $\Psi$, i.e., $\chi(U|0^n\rangle)=\chi(\Psi)$. Indeed, starting from an exact stabilizer decomposition of $\ket{\Psi}$, we can apply $(\id \otimes \bra{0}^{\otimes \tau}) C$ to each stabilizer state in the decomposition and renormalize to obtain an exact stabilizer decomposition of the output state $U\ket{0^n }$.  Once we have computed an exact stabilizer decomposition of $U|0^n\rangle$ we may use the subroutines from Section~\ref{sec:subroutines} to simulate the quantum computation. For example we may sample from the output distribution $P_U$ or compute a given output probability $P_U(x)$. This was the approach taken in Ref.~\cite{Bravyi16stabRank} and here we call this a fixed-sample gadget-based simulator since it postselects on a fixed single measurement outcome.

Note that in the fixed-sample method one must use an exact (rather than approximate) stabilizer decomposition of the resource state $\Psi$. Indeed, in a fixed-sample simulation if $\ket{\Psi_{\mathrm{\delta}}}$ approximates $\ket{\Psi}$ up to an error $\delta$ then the simulation error could be amplified to $2^{\tau/2} \delta$ when substituting in Eq.~\eqref{eq:Uidentity}.  

The random-sample gadget-based simulation method is a different approach that allows us to use approximate stabilizer decompositions within this framework. Here one selects the post-selected measurement outcome $x\in \{0,1\}^\tau$ uniformly at random.  However, now  we have some measurement outcomes other than $x = 0^\tau$ and so have to account for corrections $C_x$.  Clifford corrections are straightforwardly simulated and this is ensured provided each non-Clifford gate $V_j$ in the circuit is diagonal in the computational basis and contained in the third level of the Clifford hierarchy (e.g., the T gate and CCZ gate). This guarantees that the simulation consuming an approximate magic state $|\Psi_\delta\ra$ achieves an average-case simulation error $O(\delta)$, see Ref.~\cite{bravyi2016improved} for details. 

An important distinction between the two gadget-based methods is that the random-sample method allows one to sample from a probability distribution which approximates $P_U$ but--unlike the fixed-sample method-- in general cannot be used to obtain an accurate estimate of an individual output probability $P_U(x)$. 


\begin{figure}
	\centering
	\includegraphics{injection2.pdf}
	\caption{State injection gadgets for single-qubit $T$ gate and general multi-qubit phase gate $V$.  A correction unitary $V X_j V^\dagger$ is required whenever measurement $j$ registers a ``1" outcome.  If all corrections are Clifford then gadgets can be deployed with no additional resource requirements.}
	\label{fig_injection}
\end{figure}

\subsubsection{Sum-over-Cliffords method}
\label{sec:sum_cliffords}
Let $U$ be the quantum circuit Eq.~\eqref{eq:circuit} to be simulated.  We shall construct a sum-over-Cliffords decomposition
\begin{equation}
U=\sum_j c_j K_j
\label{eq:Udecomp}
\end{equation}
where each $K_j$ is a unitary Clifford operator and $c_j$ are some coefficients. This gives
\begin{equation}
U|0^n\rangle=\sum_j c_j K_j|0^n\rangle.
\label{eq:Vphi}
\end{equation}
Applying  Theorem~\ref{thm:randomCvec} one can approximate $U|0^n\ra$ within any desired error $\delta$ by a 
superposition of stabilizer states $\psi$ that contains  
\begin{equation}
k \approx \delta^{-2}  \| \vec{c} \|_1^2
\label{eq:chisim}
\end{equation}
terms.  In this way we can compute an approximate stabilizer decomposition $\psi$ satisfying
\be
\label{approxU1}
\|U|0^n\ra - |\psi\ra\|\le \delta, \qquad |\psi\ra = \sum_{\alpha=1}^k b_\alpha U_\alpha |0^n\ra ,
\ee
for some coefficients $b_\alpha$ and some Clifford circuits $U_\alpha$. Using the methods summarized in the previous section we can then sample from the distribution $P(x)=|\langle x|\psi\rangle|^2$ which $\delta$-approximates the output distribution $P_U$. In particular, one can use either the heuristic Metropolis sampling technique or the rigorous algorithm using norm estimation, which has runtime upper bounded as $O(kn^6)$.


The sum-over-Cliffords decomposition Eq.~\eqref{eq:Udecomp} of $U$ can be obtained by combining decompositions of the constituent non-Clifford gates. If $V_p=\sum_{j} c_j^{(p)} K_j^{(p)}$ for $p=1,2,\ldots, m$, then substituting in Eq.~\eqref{eq:circuit} gives
\[
U=\sum_{j_1,\ldots,j_m} \left(\prod_{p=1}^{m} c_{j_p}^{(p)}\right) D_m K_{j_m}^{(m)}D_{m-1}\ldots D_1 K^{(1)}_{j_1} D_0
\]
which is of the form Eq.~\eqref{eq:Udecomp} with $\|c\|_1^2=\prod_{p=1}^{m} \|c^{(p)}\|_1^2$.  This motivates the following generalization of $\xi$ to unitary operators.

\begin{dfn}[\bf Stabilizer Extent for unitaries, cf. Eq. \ref{eq:zrot}]
	Suppose $W$ is a unitary operator. Define $\xi(W)$ as the minimum of $\| c \|^2_1$ over all
	 decompositions  $W=\sum_j c_j K_j$ where $K_j$ are Clifford unitaries.
\end{dfn}
\noindent
This implies 
\begin{equation}
\xi(U|0^n\rangle)\leq \xi(U) \leq \prod_j \xi(V_j).
\label{eq:cstarU}
\end{equation}

Thus, given $\xi$-optimal decompositions of each non-Clifford gate in the circuit, the asymptotic cost of $\delta$-approximately sampling from $P_U(x)$ using the norm estimation algorithm and the sum-over-Cliffords method is  $\tilde{O}(k)$, and substituting Eq.~\eqref{eq:cstarU} in Eq.~\eqref{eq:chisim} we recover Eq.~\eqref{eq:zrot}.

Note that for any gate $V_j$ which acts on $O(1)$ qubits we may compute a $\xi$-optimal sum-over-Cliffords decomposition in constant time by an exhaustive search. Below we describe decompositions for commonly used non-Clifford gates. We use the following lemma which ``lifts''  a stabilizer decomposition of the resource state $|V\rangle=V|+^t\rangle$ to a sum-over-Cliffords decomposition of $V$.

\begin{lemma}[\bf Lifting lemma]
Suppose $V$ is a diagonal $t$-qubit unitary and
	\begin{equation}
		\label{Eq_Udecomp}
			V|+^t\rangle=\ket{V} = \sum_j c_j \ket{ \phi_j }.
	\end{equation}	
Suppose further that $ \ket{ \phi_j }$ are equatorial stabilizer states so that $ \ket{ \phi_j }= K_j \ket{+^t}$ where $K_j$ is a diagonal Clifford for all $j$. Then
	\begin{equation}
		\label{Eq_UdecompUnitary}
	V = \sum_j c_j K_j ,
\end{equation}	
and therefore $\xi(V) \leq || c ||_1^2$.  Furthermore, if the equatorial stabilizer decomposition Eq.~(\ref{Eq_Udecomp}) achieves the optimal value $\| c \|_1^2 = \xi(\ket{V})$ then  $\xi(\ket{V})=\xi(V)$.
\end{lemma}
\begin{proof}
Since $U$ and $\{K_j\}$ are diagonal in the computational basis we may write
\begin{equation}
V= \sum_{x} e^{i \theta(x) } \kb{x}{x} \qquad K_j = \sum_{x}  e^{i \theta_{j}(x)} \kb{x}{x}
\label{eq:lift1}
\end{equation}
where $\theta, \theta_j$ are functions $\mathbb{F}_2^t \rightarrow \mathbb{R}$.  For all $x\in \{0,1\}^t$ we have
\begin{equation}
\frac{1}{2^{t/2}} e^{i\theta(x)}=\langle x|V|+^t\rangle=\langle x|\sum_{j} c_j K_j|+^t\rangle=\frac{1}{2^{t/2}}  \sum_{j} c_j  e^{i \theta_j(x)}
\label{eq:lift2}
\end{equation}
Combining Eqs.~(\ref{eq:lift1},\ref{eq:lift2}) and cancelling the factors of $2^{-t/2}$ gives Eq.~\eqref{Eq_UdecompUnitary} and the remaining statements of the lemma are immediate corollaries.
\end{proof}

For single-qubit diagonal rotations $R(\theta)=e^{-i(\theta/2) Z}$,  we have
\begin{equation}
	R(\theta) \ket{+} =  \left(\cos(\theta/2)-\sin(\theta/2)\right) \ket{+} +  \sqrt{2} \sin(\theta/2) e^{- i \pi / 4} S \ket{+},
\end{equation}	
which is an optimal decomposition with respect to $\xi$ and is similar to Eq.~\eqref{Eq_SingleQubitsDecomp}.  Therefore, we can use the lifting lemma to obtain an optimal decomposition
\begin{equation}
	R(\theta)  =  \left(\cos(\theta/2)-\sin(\theta/2)\right) \id +  \sqrt{2} e^{-  i \pi / 4} \sin(\theta/2) S \label{eqn:Rtheta}
\end{equation}
and conclude 
\begin{equation}
	\xi( R(\theta)  ) = \xi( R(\theta) \ket{+}   )  = \left(\cos(\theta/2)+\tan(\pi/8)\sin(\theta /2)\right)^2.
\end{equation}

The doubly controlled $Z$ gate (CCZ) is another useful example.  In Section~\ref{Sec_Clifford_Magic_States} we show that
\begin{align}
 \ket{CCZ}   =   \frac{2}{9} &( \id+ CZ_{1,2}X_3 )(\id + CZ_{1,3} X_2)(\id + CZ_{2,3}X_1)\ket{+^3}, \\ \nonumber
=	    \frac{2}{9} & \big(  \id + CZ_{1,2} + CZ_{1,3} + CZ_{2,3} + CZ_{1,2}CZ_{1,3}Z_1 + CZ_{1,2} CZ_{2,3} Z_2   \\ \nonumber 
	&  + CZ_{1,3} CZ_{2,3} Z_3  - CZ_{1,2}CZ_{1,3}CZ_{2,3} Z_1 Z_2 Z_3 \big)  \ket{+^3} ,
\end{align}	
is an optimal decomposition with respect to $\xi$.  Deploying the lifting lemma we have
\begin{align}
	CCZ   = \frac{2}{9} & \big(  \id + CZ_{1,2} + CZ_{1,3} + CZ_{2,3} + CZ_{1,2}CZ_{1,3}Z_1 + CZ_{1,2} CZ_{2,3} Z_2  \label{eqn:CCZ_decomposition}   \\ \nonumber 
	& + CZ_{1,3} CZ_{2,3} Z_3  - CZ_{1,2}CZ_{1,3}CZ_{2,3} Z_1 Z_2 Z_3 \big) ,
\end{align}	
and conclude
\begin{equation}
	\xi( CCZ  ) = \xi( \ket{CCZ}   )  = 16/9 .
\end{equation}
Recall that since this is a Clifford magic state we have  $ \xi( \ket{CCZ}   ) = 1 / F( \ket{CCZ})$ and notice that the stabilizer fidelity is achieved by the equatorial stabilizer state $\ket{+^3}$.  We remark that the above recipe for an optimal sum-over-Cliffords decomposition can be generalised to any Clifford magic state for which the stabilizer fidelity is achieved by some equatorial stabilizer state.

These optimal sum-over-Cliffords decompositions will be used in the numerics of the following Section. 

\subsection{Implementation and simulation results}
\label{sec:numericresults}
In this section we  report numerical results obtained  by simulating two quantum algorithms.
First, we use the sum-over-Cliffords method to simulate 
the Quantum Approximate Optimization  (QAOA) algorithm due to Farhi et al~\cite{farhi2014quantum}. This algorithm allows us to explore the performance of our simulator for circuits containing Cliffords and diagonal rotations. 
This simulation involves $n=50$ qubits, about $60$ non-Clifford gates, and a few hundred Clifford gates. We note that QAOA circuits have been previously used to benchmark classical simulators in Ref.~\cite{fried2017qtorch}. Secondly, we simulate the Hidden Shift algorithm for bent functions due to  Roetteler~\cite{Roetteler09}.
This algorithm  was also used to benchmark the Clifford+$T$ simulator of
Ref.~\cite{bravyi2016improved} which, in the terminology of the previous section, is a gadget-based simulator where sparsification is achieved via suitable choice of a random linear code. We extend this methodology to a Clifford+$CCZ$ simulator of the same circuits. We also simulate the Hidden Shift circuits using the new Sum-over-Cliffords method wherein sparsification is achieved by appealing to the $\xi$ quantity.

\subsubsection{Quantum approximate optimization algorithm}
\begin{figure}[ht]
\centering
\includegraphics[height=7.5cm]{e3lin_n50_d4.pdf}
\caption{The expected value of the cost function $E(\beta,\gamma)$
computed using the Monte Carlo method 
by Van~den~Nest~\cite{nest2009simulating}.
We consider a randomly generated instance of the Max E3LIN2 problem
with $n=50$ qubits and degree $D=4$.
}
\label{fig:QAOA3}
\end{figure}

Here we consider 
the Quantum Approximate Optimization Algorithm applied to the Max E3LIN2 problem~\cite{farhi2014quantum}.
The problem is to maximize  an objective function 
\[
C=\frac12 \; \sum_{1\le u<v<w\le n}\;  d_{uvw} z_u z_v z_w
\]
that depends on $n$ binary variables $z_1,\ldots,z_n\in\{-1,1\}$.
Here $d_{uvw}\in \{0,\pm 1\}$ are some coefficients. 
Let
\[
m=\sum_{u<v<w} |d_{uvw}| 
\]
be the number of non-zero terms in $C$.
Let us say that an instance of the E3LIN2 problem
has degree $D$ if each variable $z_u$
appears in exactly $D$ terms $\pm z_u z_v z_w$
(depending on the values of $n$ and $D$ there could be
one variable that appears in less than $D$ terms).



Following Ref.~\cite{farhi2014quantum} we
consider a family of variational states
\[
|\psi_{\beta,\gamma}\ra = U|0^n\rangle  \qquad \quad U=e^{-i\beta  B} e^{-i\gamma \hat{C}} H^{\otimes n}
\]
where  $\beta,\gamma\in \RR$ are variational parameters,
$B=X_1+\ldots+X_n$ is the transverse field operator,
and $\hat{C}$ is a diagonal operator obtained from $C$
by replacing the variables $z_u$ with the Pauli operators $Z_u$. 
The QAOA algorithm attempts to choose $\beta$ and $\gamma$  maximizing
the expected value of the objective function,
\[
E(\beta,\gamma)=\la \psi_{\beta,\gamma} |\hat{C}|\psi_{\beta,\gamma}\ra.
\]
Once a good choice of $\beta,\gamma$ is made, the QAOA algorithm
samples  $z\in \{-1,1\}^n$  from a probability distribution
$P(z)=|\la z|\psi_{\beta,\gamma}\ra|^2$ by 
preparing the state $|\psi_{\beta,\gamma}\ra$ on a quantum computer 
and measuring each qubit of 
$|\psi_{\beta,\gamma}\ra$.
(In this section we assume that output bits take values $\pm 1$ rather than $0,1$.)
By definition, the expected value of $C(z)$  coincides with $E(\beta,\gamma)$.
By generating sufficiently many samples one can produce a string $z$ such that
$C(z)\ge E(\beta,\gamma)$, see Ref.~\cite{farhi2014quantum} for details.

Our numerical results described  below were obtained for 
a single randomly generated instance of the problem with $n=50$ qubits and degree $D=4$.
We empirically observed that the expected value   $E(\beta,\gamma)$ does not depend significantly
on the choice of the problem instance for fixed $n$ and $D$.
Since the cost function has a symmetry $C(-z)=-C(z)$, finding the maximum and the minimum
values of $C$ are equivalent problems. 

A special feature of the QAOA circuits making them suitable
for benchmarking classical simulators is the ability to verify  that the simulator is working properly.
This is achieved by  computing the expected value $E(\beta,\gamma)$ using two independent
methods and cross checking the final answers. Our first method of computing $E(\beta,\gamma)$ is 
 a classical Monte Carlo algorithm due to Van~den~Nest~\cite{nest2009simulating}. 
It allows one to compute expected values
$\la \omega |F|\omega\ra$, where $F$ is an arbitrary sparse Hamiltonian
and $|\omega\ra$ is a so-called computationally tractable state.
Let us choose  $F=e^{i\beta B} \hat{C} e^{-i\beta B}$
and $|\omega\ra=e^{-i\gamma \hat{C}}|+^{\otimes n}\ra$
so that $\la \omega |F|\omega\ra=E(\beta,\gamma)$.
The algorithm of Ref.~\cite{nest2009simulating}
allows one to estimate $\la \omega |F|\omega\ra$
with an additive error $\epsilon$ in time $O(m^4 \epsilon^{-2})$.
The plot of $E(\beta,\gamma)$  is shown on Fig.~\ref{fig:QAOA3}.

Our second method of computing $E(\beta,\gamma)$
is the sum-over-Cliffords/Metropolis simulator
described in Section~\ref{sec:sum_cliffords}. We used this method to simulate
the QAOA circuit $U$ defined above.  For our choice $n=50$ and $D=4$ 
the unitary  $e^{-i\gamma \hat{C}}$ can be implemented by
a circuit that contains $m=66$ $Z$-rotations $e^{i(\gamma/2)Z}$ 
 and a few hundred Clifford gates. To keep the number of non-Clifford gates sufficiently small we restricted the simulations to the line $\beta=\pi/4$.
As can be seen from Fig.~\ref{fig:QAOA3}, this line contains
a local maximum and a local minimum of $E(\beta,\gamma)$ (we note that $\beta=\pi/4$ is also the choice made by Farhi et al.~\cite{farhi2014quantum}). With this choice the cost function is a function of a single parameter $\gamma$ and we may write
\[
E(\gamma)=\la 0^n|U^\dag \hat{C} U|0^n\ra=\sum_{\vec{z} \in \{0,1\}^n} P_U(\vec{z}) C(\vec{z}).
\]

between the ``exact" value $E(\gamma)$  computed by the Monte Carlo method and
its estimate  $E_{sim}(\gamma)$ obtained using the sum-over-Cliffords/Metropolis  simulator (while the Monte Carlo method is not perfect, we expect the errors to be negligible for our purposes). While the plot only shows $\gamma\geq 0$, note that due to the symmetry of the cost function $C(z)=-C(-z)$ we have $E(\gamma)=-E(-\gamma)$. 
The estimate $E_{sim}(\gamma)$ is defined as  
\[
E_{sim}(\gamma)=\frac1s \sum_{j=1}^s C(\vec{z}^j), \qquad s=4\cdot 10^4
\]
where  $\vec{z}^1,\ldots,\vec{z}^s$ are samples from the distribution $P(\vec{z})$
describing the output of the simulator, see Eq.~(\ref{P(x)normalized}).
Generating all of the data used to produce Fig.~\ref{fig:QAOAfull}a took less than 3 days on a laptop computer, with the most costly data points taking several hours. The number of stabilizer states $k$ used to approximate $U|0^n\ra$ is shown in Fig.~\ref{fig:QAOAfull}b; it was chosen as in Eq.~\eqref{eq:chisim} with $\delta\leq 0.15$ for all values of $\gamma$. This toy example demonstrates that our algorithm is capable of processing superpositions of $k\sim 10^6$ stabilizer states for $n=50$ qubits.


\begin{figure}
	\centering
	\includegraphics[width=\columnwidth]{VecPlotFull.pdf}
	\caption{{\em Classical simulation of the QAOA algorithm:}
		(\textit{a}) Comparison between $E(\gamma)$  and its
		estimate $E_{sim}(\gamma)$ obtained using the sum-over-Cliffords/Metropolis simulator.
		We consider a randomly generated instance of the problem with $n=50$ qubits
		and degree $D=4$.
		For each data point  $10^4$ Metropolis steps 
		were performed to approach the steady distribution $P(\vec{z})$.
		The estimate $E_{sim}(\gamma)$ was obtained by
		averaging the cost function $C(\vec{z})$ over a subsequent $s=4\cdot 10^4$ samples 
		$\vec{x}$  from the output distribution of the simulator.
		Error bars represent the statistical error estimated using
		the MATLAB code due to Wolff~\cite{wolff2004monte} (for estimating errors in Markov chain Monte Carlo data) 
		(\textit{b}) The number of stabilizer states $k$ 
		used by the sum-over-Cliffords simulator was chosen as in Eq.~\eqref{eq:chisim} with $\delta=0.05$ for pink data points and $\delta=0.15$ for blue data points.
	}
	\label{fig:QAOAfull}
\end{figure}


\subsubsection{The hidden shift algorithm}
In this section, we describe the results of simulations applied to a family of quantum circuits that solve the Hidden Shift Problem \cite{van_dam_quantum_2006} for non-linear Boolean functions \cite{Roetteler09}. These circuits are identical to those simulated in \cite{bravyi2016improved} and further details of this quantum algorithm and its circuit instantiation can be found in Section F of the Supplemental Material of \cite{bravyi2016improved}. Briefly, the goal is to learn a hidden shift string $s\in \mathbb{F}_2^{n}$ by measuring the output state $|s\ra$ of the circuit $U$ applied to computational basis input $|0^{\otimes n}\ra$. The number of non-Clifford gates in $U$ can easily be controlled (we may choose any even number of Toffoli gates) and so the exponentially growing overhead in simulation time can be observed.

We will use both the gadget-based method of Section \ref{sec:gadgetbased} and the Sum-over-Cliffords method of \ref{sec:sum_cliffords}. Due to the high number of non-Clifford gates the exact stabilizer rank, $\chi$, is prohibitively high and so some sort of sparsification/approximation must be used, leading to $\chi_\delta$ instead. In principle we could apply the sparsification Lemma \ref{lem:randomCvec} in the gadget-based setting, but we prefer to use the random code method of \cite{bravyi2016improved} to enable a comparison with that work. The simulation timings in Fig.~\ref{fig:HiddenShiftTimes} consist of four trend lines which can be broken down as
\begin{itemize}
\item $T_{GB}$: The gadget-based random code method of \cite{bravyi2016improved}, wherein each Toffoli gate in $U$ is decomposed in terms of a stabilizer circuit using 4 $T$ gadgets. When a gadgetized version of $U$ uses a total of $t$  $|T\ra$-type magic states, then $|T^{\otimes t}\ra$ is approximated by a state $|\mathcal{L}\ra$ where $\mathcal{L} \subseteq \mathbb{F}_2^t$ is a linear subspace i.e., random code (Compare with Eq.~\eqref{eq:cmag}). We then have that $\chi_\delta(|T^{\otimes t}\ra)$ is the number of vectors in $\mathcal{L}$.
\item $CCZ_{GB}$: The gadget-based random code method of \cite{bravyi2016improved}, wherein each Toffoli gate in $U$ is implemented via a $CCZ$ gadget (as discussed e.g., in \cite{Howard17robustness}). When gadgetized $U$ uses a total of $u$  $|CCZ\ra$-type magic states, then $|CCZ^{\otimes u}\ra$ is approximated by a state $|\mathcal{L}\ra$ (see Eq.~\eqref{eq:cmag}) where $\mathcal{L} \subseteq \mathbb{F}_2^{3u}$ is a linear subspace/random code and $\chi_\delta(|CCZ^{\otimes u}\ra)=|\mathcal{L}|$.
\item $T_{SoC}$: The Sum-over-Cliffords method outlined in Sec.~\ref{algorithms}, wherein each Toffoli gate in $U$ is decomposed in terms of a stabilizer circuit using 4 $T$ gates. Each $T$ gate is subsequently decomposed into Clifford gates, $T=c_0I+c_1S$, with weightings as in Eq.~\eqref{eqn:Rtheta}.
\item $CCZ_{SoC}$: The Sum-over-Cliffords method outlined in Sec.~\ref{algorithms}, wherein each Toffoli gate in $U$ written as $CCZ$ which is subsequently decomposed (optimally in terms of $\xi$) into Cliffords as in Eq.~\eqref{eqn:CCZ_decomposition}.
\end{itemize}



The quantity that eventually determines the simulation overhead for both the $T$-based and $CCZ$-based schemes is $F$, the overlap with the closest stabilizer state. Recall $\xi(T)=\xi(|T\rangle)=1/{F(|T\rangle)}$ and likewise for $CCZ$. We have
\begin{align}
F(T) &=|\la +|T\ra|^2=\cos(\pi/8)^2=\frac{1}{2}+\frac{1}{2\sqrt{2}}\approx 0.853, \label{eqn:Tovlap}\\
F(CCZ) &=|\la +^{\otimes 3}|CCZ\ra|^2=\left(\frac{3}{4}\right)^2=\frac{9}{16}. \label{eqn:CCZovlap}
\end{align}



Note that we are using the variable $u$ to denote the number of Toffoli (equivalently $CCZ$) gates in our Hidden Shift circuit. Using the Random Code method, for a target infidelity $\Delta$ we chose a corresponding stabilizer rank $2^k$ where \cite{bravyi2016improved} stipulates
\begin{align}
\log_2 k_T &=\lfloor \log_2\left(4 \cos(\pi/8)^{-8u}/\Delta\right)\rfloor,\\
\log_2 k_{CCZ} &= \lfloor \log_2\left(4 \left(\tfrac{3}{4}\right)^{-2u}/\Delta\right)\rfloor.
\end{align}
Using the Sum-over-Cliffords method, for a target error $\delta$ we chose $k$ as in Lemma~\ref{lem:randomCvec} so that
\begin{align}
k_T &=\left\lfloor \left({\cos(\pi/8)}^{-4u}/\delta\right)^2\right\rfloor,\\
k_{CCZ} &=\lfloor \left(({3}/{4})^{-u}/\delta\right)^2\rfloor.
\end{align}
In either case, we see that there are significant savings to be had by using CCZ gates/states directly versus breaking them down into 4 $T$ gates/states each. For a fixed precision the scaling with $u$ (number of $CCZ$ gates) goes as
\begin{align}
T:&\quad \left(\frac{1}{\cos \pi/8}\right)^{8u} \approx   2^{0.914u}, 
\\ \text{vs.}\quad  
CCZ:&\qquad \left(\frac{16}{9}\right)^u\approx 2^{0.83u}.
\end{align}
This is apparent from the different slopes of the $T$- and $CCZ$- based versions of the simulations in Fig.~\ref{fig:HiddenShiftTimes}. 


Absolute comparisons between the gadget-based and Sum-over-Cliffords method are complicated by various implementation details and the amount of optimization applied to each (i.e., more in the latter case). Broadly speaking, however, we observe that the Sum-over-Cliffords method is as fast, if not faster, than the gadget-based method. This is true \emph{despite the fact that Sum-over-Cliffords is completely general in its applicability} whereas the gadget-based technique is only applicable for non-Clifford gates from the third level of the Clifford hierarchy (i.e. those with state-injection gadgets having Clifford corrections). Not only can Sum-over-Cliffords handle gates outside the third level, its performance often \emph{improves} in such situations. For example, a circuit with many small-angle rotation gates requires a number, $k$, of samples that is smaller as the rotation angle moves away from $\pi/4$ i.e., the $T$ case (recall  Eq.~\eqref{eqn:Rtheta}).

    \begin{figure*}[t!h]
        \centering
        \begin{subfigure}[b]{0.47\textwidth}
            \centering
            \includegraphics[width=\textwidth]{HiddenShiftTimes_RC.pdf}
            \caption[HiddenShift]%
            {{\small Simulation time for Hidden Shift circuits using the gadget-based random code method from Ref.~\cite{bravyi2016improved}.}}    
            \label{fig:HiddenShift_a}
        \end{subfigure}
        \hfill
        \begin{subfigure}[b]{0.47\textwidth}  
            \centering 
            \includegraphics[width=\textwidth]{HiddenShiftTimes_SoC.pdf}
            \caption[]%
            {{\small Simulation time for Hidden Shift circuits using the Sum-over-Cliffords method from \ref{clifford_sim}.}}    
            \label{fig:HiddenShift_b}
        \end{subfigure}
        \vskip\baselineskip
        \begin{subfigure}[b]{0.5\textwidth}   
            \centering 
            \includegraphics[width=\textwidth]{HiddenShiftErrors.pdf}
            \caption[]%
            {{\small Approximation error between the true hidden shift bitstring, $\vec{s}$, and the simulated vector of marginal probabilities, $\vec{\hat{s}}$, for the simulations in Fig.~\ref{fig:HiddenShift_a} and \ref{fig:HiddenShift_b}. The infinity norm gives the largest discrepancy between any individual bit $s_i$ and the corresponding estimate $\hat{s}_i$. Two outlier data points (filled rectangles) whose coordinates are at $(14, 0.304)$ and $(16, 0.512)$, are omitted from this plot for clarity}}    
            \label{fig:HiddenShift_c}
        \end{subfigure}
        \quad
        \begin{subfigure}[b]{0.45\textwidth}   
            \centering 
            \includegraphics[width=\textwidth]{HiddenShiftHistogram.pdf}
            \caption[]%
            {{\small Simulated output, $\vec{\hat{s}}$, versus the true shift string, $\vec{s}$, for the case $T_{SoC}$ with 16 Toffoli gates (i.e corresponding to the open rectangle on the right of \ref{fig:HiddenShift_c}).}}    
            \label{fig:HiddenShift_d}
        \end{subfigure}
        \caption[ The average and standard deviation of critical parameters ]
        {\small Timings and errors for simulations of 40-qubit Hidden Shift circuits with varying numbers of non-Clifford gates. Every Toffoli gate is either recast as a $CCZ$ gate (via Hadamards on the target) or as a circuit comprising 4 $T$ gates and additional Stabilizer operations (\cite{bravyi2016improved}). We fixed precision parameters $\delta=0.3$ and $\Delta=0.3$ for the sum-over-Clifford simulations and gadget-based simulations respectively. Simulations were run on Dual Intel Xeon 1.90GHz processors using Matlab. } 
        \label{fig:HiddenShiftTimes}
    \end{figure*}




