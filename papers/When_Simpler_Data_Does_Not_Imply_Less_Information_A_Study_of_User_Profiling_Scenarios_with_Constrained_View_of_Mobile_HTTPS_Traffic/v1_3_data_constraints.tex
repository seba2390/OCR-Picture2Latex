\section{Data and Constraints}
In this section we describe the scope of our data and the four different scenarios that we have analyzed in our study as a result of applying four constraints to the raw HTTP(S) data.

The data collection was conducted through an web acceleration proxy infrastructure for mobile devices that all the participants in our study configured in their smartphones. Their HTTP(S) traffic went through the proxy when they were connected to the Internet via cellular networks (2G/3G/4G). It is important to mention that the proxy does not archive the content of the HTTP(S) traffic. Instead it keeps the URL, and in the case of HTTPS, only the domain name. Thus, the content analyses conducted in the later part of the study use a limited portion of the pages that could be retrieved solely with the URL, i.e., publicly accessible pages. The private pages such as the HTTPS content, those accessed with credentials or tokens, or those that are personalized through cookies are not retrieved and excluded in the content analyses.

We apply different constraints to the raw HTTP(S) traffic which are reflective of having access to (a) explicit user consent or not; (b) different levels of visibility into the data and (c) different capabilities in technological sophistication. In this regard, we devise four scenarios to infer user traits, attributes and interests from this constrained data. 

As mentioned, there are various entities which can have access to the HTTP(S) traffic and potentially perform profiling. While there are many studies \cite{enck2011study} that deal with mobile apps particularly, and use the data that only they can access (e.g., sensor data, GPS readings, and personal data including contacts and photos), our work takes a different perspective. We deal with a particular data set that is of interest to many entities, i.e., HTTP(S) traffic, but explore various types of profiling entities.

In our study, we focus on analyzing the profiling capability of each scenario separately rather than speculating on possible combinations of different scenarios. It is possible to think of a combination of different scenarios since there are cases where an entity with less constraints can perform an analysis of another entity which has more constraints. For example, a web browser can not only analyze the patterns in the visited URLs but also see the page contents. However, such combinations may not be generalised as the profiling entities are under different circumstances. For example, while the web browser can see the page contents it cannot see the HTTP(S) traffic of other applications where ISPs or proxy servers can see those traffic to some extent. As the combinations can take different forms depending on different situations, we believe that it is essential to first understand each scenario more deeply since the results can also serve as a reference for possible combinations.

% Head 2
\subsection{Profiling Scenarios around HTTP(S) traffic}
Next we describe the four different profiling scenarios, depicted in Figure~\ref{fig:one}.

\begin{figure}
  \includegraphics[scale=0.6]{figures/scenario.png}
  \caption{Four user modeling scenarios by applying four constraints to HTTP(S) mobile traffic.}
  \label{fig:one}
\end{figure}

\paragraph{Scenario 1 (S1) - Profiling based on time-stamps} 
The strictest constraint we assume is such that the S1 data is composed only of the timestamp of the HTTP(S) accesses. It assumes no availability to the content of the HTTP(S) message or processing of the HTTP(S) header. Even though the resulting data seems very simple, the stream of timestamps may carry meaningful behavioral information about the user's habits and routines. For instance, in our collection scenario, S1 data might be reflective of time-periods when the user is outdoors or on-the-move as we only collect and analyze data going through cellular networks\footnote{Note that this is an illustrative example. We do not fully rely on this assumption in our analysis as mobile Internet can be also used indoors (or Wi-Fi outdoors)}. 
Note that all mobile apps on Android could acquire S1 data by tracking when a user is connected to the mobile Internet or to Wi-Fi \footnote{refer to https://developer.android.com/training/basics/network-ops/managing.html}.
(typically in order to prompt updates only on Wi-Fi). We believe that profiling methods that analyze S1 data can be easily built in practice as capturing and storing the timestamps of HTTP(S) traffic is becoming a feasible process for a myriad of players in the Internet ecosystem (for instance, end-devices with limited computing power and storage capacity --such as IoT devices). 

\paragraph{Scenario 2 (S2) - Profiling based on header} 
The second scenario refers to the cases in which only the \textit{header} of the HTTP(S) traffic is available. Whereas the previous constraint only reveals the existence of HTTP(S) traffic, the \textit{type} of user activity becomes accessible under this scenario. The header reveals important information related to the user's activity including the destination address, the amount of data exchanged, and the app used for access in some particular cases (through the user agent field). It evokes profiling scenarios that exploit patterns in the usage of apps or in accessed domain names. The entities that participate in the delivery of HTTP(S) traffic between the user and the destination - such as ISPs, mobile telecommunication operators, and web proxy servers - could analyze this kind of data. Depending on the  platform and versions, there are mobile apps which also have access to the traffic information\footnote{The apps for network usage statistics use such data. An example is available at http://network-connections.mobi/}. 

\paragraph{Scenario 3 (S3) - Profiling based on domain name} 
The third scenario assumes the ability interpret the topical categories of URLs, for example, 'Computers/News and Media' for the domain name 'techcruch.com' Although this may not seem to add much data to the HTTP(S) headers, the difference is significant as it requires merging the HTTP(S) logs with external sources to identify the topical category of the URLs. The scenario is also a critical stage where semantic interpretations of the logs and profiling of preferences become possible. 

Today, there are several online tools and resources that support the categorization of domain names, which makes the implementation of the scenario practical. These tools include open website dictionaries (e.g., Dmoz.org) that are built in a crowd-sourced manner, and machine learning tools that assume a certain topical category and perform a classification of websites with some example pages. As it is impossible to recognize and categorize all individual pages available on the Web, the categorizations are often made at an aggregated-level, such as at the domain name-level. Though the categorizations do not exactly capture the actual delivered content, such tools allow a profiling method to guess an approximate topical category based on popular websites. In fact, this approach is often used in computational advertisement, where the user profiles built from website categories are used to determine the displayed ads or price offerings for each user \cite{castelluccia2014selling}. We therefore make this scenario as a separate one from the aforementioned scenarios. Note that the data in this scenario can also deal with HTTPS traffic given that the domain name is typically visible regardless of the content encryption. 

There are products that demonstrate the types of entities that have access to such information. Prior works on real-time bidding \cite{carrascosa2015always} shows that ad exchange platforms (e.g., doubleclick) perform personalized advertisements based on web browsing patterns. Similar products have been also offered by telecom operators\footnote{Refer to http://www.multichannel.com/news/advanced-advertising/att-drop-internet-preferences-program-gigapower/408181 and https://www.verizonwireless.com/support/verizon-selects/}. 
 In Android, there have been APIs for querying the browsing history of the device\footnote{Though removed from Android 6.0, the permission com.android.browser.permission.READ\_HISTORY\_BOOKMARKS allowed fetching the browser history.}, which allow mobile apps to have access to similar data.

\paragraph{Scenario 4 (S4) - Profiling based on page content} 
The final constraint on the HTTP(S) traffic assumes access to the actual HTTP pages as the full URL is exposed. This type of data enables profiling methods that incorporate advanced content analysis techniques. A profiling entity can implement such a method either by archiving the page content of the HTTP accesses or by storing the full URL and then fetching the documents later for analysis and profiling purposes. 

While this constraint provides the maximum level of detail regarding the \textit{content} that the user has accessed, it is not available for HTTPS traffic where the URL is encrypted and only the domain name is available. Note that collecting and analyzing this type of data requires large-scale storage capabilities. Moreover, explicit user consent may be necessary and strict personal data protection laws might apply to this type of data given the potential sensitivity of the accessed content. In theory, the entities that are mentioned for scenario 3 have the capability to implement this profiling scenario. 

The constraints defined above are specific to mobile HTTP(S) traffic, which is the subject of our study. One may define different constraints for other types of data, user modeling goals and application environments. We leave to future work the exploration of the concept of constrained profiling in other domains beyond mobile HTTP(S) traffic.
