\section{Conclusions and Future Work}
In this paper, we have investigated the modeling capabilities and limitations of applying four different constraints to mobile HTTP(S) data. These constraints might arise due to a variety of reasons, including a lack of visibility of richer datasets by certain entities, and technical and/or legal limitations. We have presented the results of constrained user modeling on HTTP(S) data collected by means of an in-the-wild user study with 61 volunteers for at least 30 days. Our work shows that meaningful user traits and interests can be inferred from constrained data. We believe that our work contributes to the understanding of the trade-offs regarding user modeling, personalization and access to HTTP(S) data, particularly but not limited to a mobile context.

A direct extension of the work is to collect the data directly from mobile devices and conduct a similar study. Different types of data can be collected through the devices, for example, sensor readings, GPS positions, and usage of apps. Possible constraints can be defined in terms of accessing such data, and the impact of the constraints on user modeling can be analyzed. Recent literature suggests the importance of app usage \cite{seneviratne2014predicting}. Although we made an effort to identify app usage information from the HTTP(S) traffic and develop related features, we observed that the HTTP(S) traces do not accurately capture app usages; the domain name and the user-agent fields of the HTTP(S) logs were not enough to identify the app in many cases, and interactions with an app did not trigger an HTTP(S) connection often. Thus, we leave the exploration of the space for future work. 

In future work, it would be interesting to have access to all mobile Internet access data (including Wi-Fi) to carry out a comparative analysis. In addition, we would like to extend the data collection to PC devices at homes and conduct a similar comparative study. 

There are a number of user profile variables that our models could not infer, such as boredom proneness or gender. We plan to investigate this further and possibly design new features that might be more relevant to model these variables. Testing different  machine learning algorithms and conducting a concrete comparison would help understanding the profiling capability of the profiling scenarios. It can extend the findings of the associations between the target variables and the behavioral data. Moreover, observing consistent results across different algorithms would confirm the profiling capability of the scenarios.

Finally, it is important to explore new paradigms for transparency that would inform users not only about which data is being collected, but also how it is being analyzed and for which purposes. More in-depth studies on new negotiation models that can help both service providers and users are also necessary. 

\section{Acknowledgement}
The research leading to these results has received funding partially from the European Union's Horizon 2020 research and innovation programme under grant agreements No 653449 (project TYPES). The paper reflects only the authors' view and the Agency and the Commission are not responsible for any use that may be made of the information it contains.

% Bibliography
\bibliographystyle{ACM-Reference-Format}
\bibliography{v1_references}