\section{Results and Discussion}
We first provide an overview of the results followed by a detailed description of the most predictive features for each target variable. 

\subsection{Results Overview}

\begin{table}
  \includegraphics[scale=0.45]{figures/results3.png}
  \caption{Classification Accuracies of all Models (balanced accuracy between 60\% and 65\% is in white, above 65\% is in grey)}
  \label{tbl:five}
\end{table}

Table~\ref{tbl:five} provides an overview of the results. As the analysis includes a large number of target variables, we only report the cases with a performance competitive with respect to the state-of-the-art performance reported in previous works for similar tasks using mobile data. We report results with a balanced accuracy \footnote{balanced accuracy is a metric for evaluation of classifiers which deals with class imbalance. It is the arithmetic mean of sensitivity ($\frac{\text{true positives}}{\text{true positives} + \text{false negatives}}$)and specificity ($\frac{\text{true negatives}}{\text{true negatives} + \text{false positives}}$).}
between 60 and 65\% (Table~\ref{tbl:five} in white), and with a balanced accuracy
\footnote{The baseline random model refers to a random binary guess accuracy taking into account unbalances in two classes. In addition, to provide a better picture of the classification accuracy we provide also the related confusion matrices (Table~\ref{tbl:five}).
} over 65\% (Table~\ref{tbl:five} in grey). Note that all results have a kappa value between 0.21 and 0.5, meaning that the learned models are 21\% to 50\% better than a baseline random model. While our results do not outperform those reported in the literature, note that our research focuses on examining the profiling capability of the possible scenarios in practice rather than on developing a more accurate method. 

\paragraph{Scenarios 1 and 2} Perhaps the most surprising finding has been the modelling power of the features computed from S1 and S2 data, and particularly from S1 data given that it characterizes overall temporal patterns of online traffic, independently of whether it is user-initiated or not. Even though this might seem to be a very coarse and noisy signal, the results show that it is possible to infer user traits of different nature: personality (extraversion with 69\% balanced accuracy), demographics (educational level with 69\% balanced accuracy) and even a purchasing interest in clothes (with 72\% balanced accuracy) and traveling (with 64\% balanced accuracy). As the non-user initiated traffic is filtered by applying constraint 2, another personality variable, conscientiousness (with 75\% balanced accuracy), is inferred with features from S2 data. 

\paragraph{Scenarios 3 and 4} As expected, having access to the content's semantic information enables us to infer personal preferences. Although this finding may seem obvious, it is important to note that in this work we only analyze the HTTP(S) traffic through the 2G/3G/4G mobile network, which is a subset of the entire online activity of an individual. Despite this limitation, it is possible to infer our participants' level of interest in several product categories: computer \& electronics (67\% balanced accuracy), furniture (75\% balanced accuracy), travel (73\% balanced accuracy), and home appliances at a lower (but better than not-random) balanced accuracy (63\%).

Interestingly, the models built with features from S3 and S4 data did not classify personality traits and demographics better than the models built with features from S1 and S2 data. However, we are careful in reading this finding since there might be other reasons for this performance that are not related to the nature of the data, such as missing entries in the website dictionary (S3 data) or the limitation of the scope of the pages that we could retrieve through the simple HTTP client (S4 data).

While it is difficult to compare the performance for all the target variables to those of the literature, we compare the results for the personality traits and boredom proneness. We additionally present the comparison in Table~\ref{tbl:six}. From the call logs, de Montjoye et al. \cite{de2013predicting} detected all the five personality traits with 49\% to 63\% accuracy (with the baseline being 36\% to 39\%). Smartphone usage and sensor logs were used in \cite{chittaranjan2013mining} to detect the five personality traits with the F-measure ranging between 0.6 and 0.8. Staiano et al \cite{staiano2012friends} explored social network modelling (using both call logs and mobile phone sensors) and detected the five personality traits with 62\% up to 71\% of accuracy (with the baseline accuracy being around 50\%). There is only one study that modelled boredom proneness \cite{matic2015boredom} and achieved the accuracy of 80\% in a binary classification.

\begin{table}
  \includegraphics[scale=0.6]{figures/results_to_soa.png}
  \caption{Comparison to related studies}
  \label{tbl:six}
\end{table}

\subsection{Most Predictive Features per Model}
\begin{table}
  \includegraphics[scale=0.5]{figures/feature_overview.png}
  \caption{Overview of predictive features}
  \label{tbl:seven}
\end{table}
Next we provide a description of the most predictive features and interpret the associations between features and the target variables. Note that we are not able to provide explanations about the meaning of features computed from S3 data as they are produced through LSI, which makes the interpretation difficult. Therefore, we discuss the predictive power of S1, S2 and S4 features. 

\subsubsection{Personality Traits}
\paragraph{Extraversion} Extraverted individuals are defined as sociable, fun-loving and affectionate \cite{goldberg2006international}. The models built with features from all data types are able to classify individuals into \textit{high/low} extroversion with accuracies ranging between 66\% to 72\%. This result corroborates previous work where extraversion has also been inferred with high accuracy \cite{de2013predicting, staiano2014money}. In S1 and S2 data, the most predictive features are related to mobile activity during the night in the weekends (access duration in S1 data and amount of traffic in S2 data), the variability in access patterns in the morning during working days (S1 and S2 data), the distribution parameters of inter-activity durations (S1 and S2 data), and the number of "posts" (accessible only from S2 data). 

Several studies have associated extraversion and overall intensity of mobile phone usage (self-reported) as a means of stimulation \cite{carrascosa2015always, ramirez2010relationship}, which may be directly or indirectly reflected in our features. In addition, extroverts exhibit an increased usage of online leisure services \cite{amichai2008personality}, which may impact the weights of related website categories in S3 data. In terms of S4 data, topics related to health, travel and shopping seem to be the most predictive to classify extraversion; interestingly, the literature has reported an association between travel and shopping and extraversion \cite{hoxter1988tourist, wang2008passion}.

\paragraph{Conscientiousness} Highly conscientious individuals are characterized as efficient and organized \cite{goldberg2006international}. This trait was recognized the best with features from S2 data. The most predictive features capture the duration of morning activity, the amount of traffic during weekends in night hours, and the variability of morning activity during working days.  

The tendency of highly conscientious people to wake up earlier and to have specific diurnal preferences was reported two decades ago in the literature \cite{jiang2013calling} and has been consistently confirmed in recent years \cite{randler2008morningness} associating the characteristic named 'morningness' to this trait. In the technological literature, conscientiousness has been found to be negatively correlated with total Internet usage \cite{landers2006investigation}. Such characteristics of conscientious individuals might be captured in our experiments through the features that quantify morning vs. evening dynamics of mobile traffic, as well as the total traffic consumption. 

The fact that conscientiousness was not accurately classified with features from S1 data but with features from S2 data might suggest that this trait is strongly related to how people actually use the phone as opposed to general patterns of traffic. Recall that the main difference between these two data types is filtering out non-user initiated traffic. 

\paragraph{Openness} Openness to experience (often shortened as 'openness') depicts individuals who are creative, intellectual and insightful \cite{goldberg2006international}. Given this definition, the high classification accuracy of this personality trait from the features in S3 data (website category tags) may not be unexpected. As mentioned, due to the difficulty of reverse engineering the LSI-based features to website categories, we are unable to discuss the meaning of the most predictive features for S3 data (we had 1325 dimensions - categories - in this data when compared to 25 topics in S4 data). 

Openness has been found in the literature to be a predictor for a set of web services \cite{tsao2013big}. These correlations between openness and the types of websites support the accuracy of the models built with features from S3 data, which was higher than that of the models built with features from S4 data despite the fact that the latter relies on a more detailed analysis of the content of the web pages. It could be the case that openness is better captured by quantifying the diversity of the consumed webpages overall than performing in-depth topical analysis of the pages.  

\paragraph{Agreeableness and Neuroticism} Agreeableness reflects individual characteristics that are perceived as kind, sympathetic, cooperative, warm and considerate \cite{goldberg2006international}. Neuroticism is a trait that describes anxious, insecure, and self-pitying individuals \cite{goldberg2006international}. None of the four models was able to accurately infer agreeableness and neuroticism. We are careful in drawing an interpretation from this fact, since the reason might be in the limitation of our features and models, or in the fact that these traits are inherently difficult to capture with mobile online data. 

\subsubsection{Boredom Proneness}
Boredom Proneness was not accurately classified in our experiments. Recent work suggests that finer-grained information of smartphone usage might be necessary to infer this trait (e.g. screen-on events) beyond mobile HTTP(S) traffic \cite{matic2015boredom}.

\subsubsection{Demographics}
Automatic modelling of demographic information of unlabeled users has been also identified as an important user modelling task \cite{koenigstein2011yahoo}. Despite our expectations that the semantic information of browsed web pages would be predictive for this task, only the models built with features from S1 data were able to accurately classify individuals with low/high educational levels with 69\% balanced accuracy. The model relied mostly on features that quantified online activity in the first time periods in the morning and the last time periods in the night. We leave a more thorough investigation of these target variables and interpretation of results to future work. 

\subsubsection{Shopping Interests}
Inferring product interests with a small group of participants has clear limitations, especially considering the population scale that the major players of online behavioral advertising deal with. Our goal was not to target maximum accuracy but to explore the possibility of using different types of data for this purpose. Moreover, we have not found previous work aimed at inferring shopping interests from S1, S2 data that has neither information about websites nor content. Another research question that we were interested in exploring was whether mobile browsing logs (captured in features from S3 and S4 data) would have predictive power for this task. 

The most surprising finding was that two of the shopping categories --'clothes' and 'travel'-- were inferred from features from S1 data with 72\% and 64\% of balanced accuracy, respectively. The most indicative features for these two variables were related to the variability of morning Internet activity both in weekends and in working days, the weekend evening activity (only for 'travel') and the inter-session distribution parameters. Given the competitive performance in inferring extraversion from this data type, a possible explanation might be attributable to the link between extraversion and travel/shopping preferences observed in the literature \cite{huang2010relationship}.

Features from S3 and S4 data exhibited intuitive relations with the target variable, such as the relation of the topic 'food and travel' with the target variable 'travel' and the topic 'technology' with the target variable 'computer and electronics'. 

