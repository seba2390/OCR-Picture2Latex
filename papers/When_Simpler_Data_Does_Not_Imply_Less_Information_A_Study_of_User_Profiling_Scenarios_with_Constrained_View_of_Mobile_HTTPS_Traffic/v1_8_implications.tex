\section{Implications}
\subsection{Possibility of Transparency and Negotiation}
As briefly discussed in the Introduction, we do not aim to draw one-sided interpretations from the results, i.e., either as  warnings about unexpected profiling threats or as solutions to more privacy-preserving profiling methods. We rather believe that the results are implying the possibility of transparency and negotiation between users and service providers. The availability of restricted views suggests a spectrum of compromises, rather than a binary decision where the users grant access to all of their data or to none of it. Previous work suggests \cite{leon2013matters} that explaining the scope of use and offering control to users can affect the willingness to share personal data, and our study provides concrete options for compromise. For example, an online ad exchange which can implement scenario 4 could transparently communicate scenario 3 (i.e., using up to domain names instead of full URLs) as an alternative. The users are then able to consider the trade-off in the relevance of advertisement and the degree of exposure of their browsing behavior. In addition, by choosing scenario 3 explicitly, users can have more trust that their full URL would not be exposed to the service provider. 

On the other hand, previous works showed a number of limitations of having transparency and providing end-users with privacy controls. Nissenbaum's work \cite{nissenbaum2011contextual} elaborates on the paradox of providing too much information to end-users, which discourages them to read and understand it (also make them avoid taking control of their privacy \cite{compano2010policy}), and that simplifying the complexity will inevitably leave out necessary details. Acquisti et al. \cite{acquisti2005privacy} also report various decision biases that often lead to short-term benefits while sacrificing long-term privacy. These works imply that privacy problems would not be addressed with simple transparency policies. We believe that the works further motivate the research on various related topics such as deliberate implementations of transparency goals, design of privacy regulations, development of decision support systems for end-users.

In addition, there is an opportunity for similar research in a variety of application domains, from existing online services and mobile apps to newly emerging ubiquitous systems that deal with highly personal data. For each application area, similar to our work, it will be important to identify the constraints in the application environment and analyze of the impact on the profiling capability. 

\subsection{It is not just about the Data}
With an increased awareness of privacy and the value of personal data, it is becoming important for Internet service providers to communicate to users about the use of personal data. The communication is often focused on explaining the types of collected data. The purpose of the collection is typically described in abstract terms. However, we have shown that it is possible to infer personal traits even from the data that are considered less sensitive (e.g. S1 data) when a proper technique is applied to the data. From the perspective of this work that puts emphasis on transparency, we believe that the communication should not be limited to the collected data but also convey the types of inferences that the service provider will make from the collected data and the purpose for making such inferences.

\subsection{The Power of Inferred User Profiles}
Though the inference of personal traits enables service providers to improve the personalization of their services (e.g., personality-aware friend recommendation \cite{bian2011online}), it may also open the door to manipulations of the users' behavior and decision making. For example, personality and boredom proneness are predictors of several behaviors, including impulsive online buying and gambling. This information may be misused by service providers to exploit human weaknesses to their advantage. Another example is Crystal, a startup that uses automatically inferred personality of email recipients to adapt the content and vocabulary of the answers
\footnote{Refer to https://www.theguardian.com/media/2015/may/19/crystal-knows-best-or-too-much-the-disconcerting-new-email-advice-service}
. These examples have already raised privacy and ethical concerns highlighted by the press. Our study suggests the need for clear codes of conduct by service providers to ensure an ethical treatment of the collected user data and associated inferences.

\subsection{The Importance of Dynamics in HTTP(S) Traffic}
In terms of understanding users through HTTP(S) traffic, significant research efforts have been devoted to inferring traits and preferences from the consumed content or visited destination. In our work, we observed that the time and frequency of HTTP(S) accesses also carry important information about users, such as the timestamp of HTTP(S) accesses (S1 data) enables the inference of certain personality traits (extraversion and conscientiousness), education level, and some shopping interests (clothing). This finding is timely as a large number of diverse services and devices have access such simple data. The results suggests that the potential of the data in terms of revealing personal information should be recognized by both collecting entities and users, and that it is important to clearly communicate the uses of the data.

\subsection{User Modeling in an HTTPS World}
The adoption of HTTPS by online services has been steadily increasing for the past years. Though the number varies depending on the measurement approach, it is commonly observed that a significant portion of the traffic is carried through HTTPS and the portion is growing \cite{naylor2014cost}. While one of the purposes of HTTPS is to protect the user's privacy, our results provide an interesting perspective about how much protection is achieved. According to our classification of the profiling constraints, HTTPS can be interpreted as a protocol to prevent S4 data, as it limits the view of the full URL and the consumed content. However, our results suggest that certain personality traits, education level and shopping interests can be inferred from S1-S3 data, which is visible for all HTTP(S) traffic. This should be considered in the evaluation of HTTPS, and HTTPS itself should not be understood as a full privacy solution. Again, the entities which have access to S3 data (e.g., ISP, middle boxes, WiFi access points) should recognize the sensitivity of the data and reflect on transparent and responsible uses. 