% related work
\section{Related Work}
Many works on personalization and recommendations demonstrate the value of fine-grained online behavioral data for user profiling. Nowadays, many online services (\textit{e.g.} news \cite{liu2010personalized}, music \cite{koenigstein2011yahoo}, and social media feeds \cite{chen2010short}) implement some kind of personalization by using such data. The usage of behavioral data in a wide variety of services reflects how well the data reveals the interests and preferences of people.

Concerns about privacy arise naturally as in some cases the data is informative enough to point out an individual from a pool of users. For example, recent works on de-anonymization found that it is possible to uniquely identify users from movie ratings \cite{narayanan2008robust}, also from mobility traces \cite{de2013unique}. On the other hand, solutions that prevent the identification of individuals have been explored. Research on differential privacy \cite{dwork2008differential} provides means to protect anonymity through generalization or suppression of some attributes of the data \cite{samarati1998protecting} or by adding noise deliberately \cite{iyengar2002transforming}, and at the same time to serve queries up to a certain level of accuracy.

Privacy concerns still remain since the flow of raw behavioral data is complicated, and it is difficult for ordinary users to have awareness or control. A significant portion of Internet services leverage online behavioral data for advertisement as their main monetization strategy. Recent works on analyzing tracking systems in practice reveal a surprising level of exposure of personal online activities to online advertisement platforms \cite{carrascosa2015always}\cite{castelluccia2014selling}. The analyses suggest that a visit to a website is not only visible to the visited website but also to many advertisement platforms. Furthermore, detailed interactions made in the website could be also exposed through cookie-exchange techniques. This information allows service providers to apply sophisticated user-modeling techniques to infer information such as purchase intent \cite{carrascosa2015always}, or price steering \cite{hannak2014measuring}.

The concern intensifies given that such data may reveal or imply other sensitive personal information. Kosinski et al. \cite{kosinski2013private} have analyzed the Facebook Likes of people, and reported that the different Like histories of the people were associated with sensitive personal attributes such as sexual orientation and political views. Lindamood and Kantarcioglu's work \cite{lindamood2009inferring} observed associations between network graph features of Facebook and sensitive personal attributes. Studies have looked at other data as well. Zong et al. \cite{zhong2015you} reported association between the location check-in history and demographic attributes including gender, age, education, and marital status. The kind of apps installed in smartphones was also found to have association with various traits, such as religion, parental status, etc. \cite{seneviratne2014predicting}.

Personality analysis is another area that observed relationships between private information and behavioral data. The study of personality itself is an established area of research, mainly in the psychology literature, and analytical methods based on interviews or questionnaires are already well developed. The 'big-5 personality traits' is a representative example, which measures the personality over the five dimensions,  extroversion, openness, conscientiousness, agreeableness, and neuroticism \cite{goldberg2006international}. Recent studies have found associations between personality and various types of high-level online activities, including overall Internet usage \cite{landers2006investigation}, gaming \cite{jeong2015addictive}, or particular applications such as email, messaging, or social media \cite{tosun2010does}. Ferwerda et al. \cite{ferwerda2016personality} even report the possibility of inferring personality by checking only the disclosed fields of social network accounts. Many works conducted prediction of personality traits with features capturing the technology use, and reported promising performance \cite{de2013predicting, staiano2014money, carrascosa2015always, ramirez2010relationship, tsao2013big}. 

Our work expands the scope of the literature by considering ancillary entities that are unrelated to users' interaction but still have certain level of access to the behavioral data (\textit{e.g.}, mobile apps, WiFi access points, ISP and telecommunication operators, mobile advertising companies). The capacity to profile users is likely to be limited inherently due to the different restrictions imposed on the entities, however, it is unclear what types of and how much profiling  can be performed by them. As the current work is specific to HTTP(S) traffic from mobile phones, we believe there can be additional future works with similar goals as ours but which explore different types of behavioral data.


