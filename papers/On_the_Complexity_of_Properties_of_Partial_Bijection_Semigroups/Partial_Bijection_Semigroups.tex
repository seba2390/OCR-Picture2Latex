\documentclass{amsart}

\usepackage{graphicx,algorithm,algpseudocode,xcolor,stmaryrd,amsaddr,tikz,hyperref}

\algnewcommand\algorithmicinput{\textbf{Input:}}
\algnewcommand\Input{\item[\algorithmicinput]}
\algnewcommand\algorithmicoutput{\textbf{Output:}}
\algnewcommand\Output{\item[\algorithmicoutput]}
\algnewcommand{\LineIf}[2]{\State \algorithmicif\, #1 \,\algorithmicthen\, #2 \,\algorithmicend\ \algorithmicif}
\algnewcommand{\LineForAll}[2]{\State \algorithmicforall\, #1 \,\algorithmicdo\, #2 \,\algorithmicend\ \algorithmicfor}
\algnewcommand{\Accept}{\textbf{accept}}
\algnewcommand{\Reject}{\textbf{reject}}


\newcommand{\Model}{\mathrm{Model}}
\newcommand{\Modelbf}{\mathbf{Model}}
\newcommand{\Fix}{\mathrm{Fix}}
\newcommand{\id}{\mathrm{id}}
\newcommand{\Row}{\mathrm{Row}}
\newcommand{\Null}{\mathrm{Null}}
\newcommand{\End}{\mathrm{End}}
\newcommand{\Span}{\mathrm{span}}
\newcommand{\dom}{\mathrm{dom}}
\newcommand{\image}{\mathrm{image}}
\newcommand{\Inv}{\mathrm{Inv}}
\newcommand{\eq}{\ensuremath\mathrel{=}}
\newcommand{\NL}{\ensuremath\mathsf{NL}}
\newcommand{\NP}{\ensuremath\mathsf{NP}}
\newcommand{\co}{\ensuremath\mathsf{co}}
\newcommand{\PSPACE}{\ensuremath\mathsf{PSPACE}}
\newcommand{\NPSPACE}{\ensuremath\mathsf{NPSPACE}}
\newcommand{\FO}{\ensuremath\mathsf{FO}}
\newcommand{\AC}{\ensuremath\mathsf{AC}}
\newcommand{\LOGSPACE}{\ensuremath\mathsf{L}}
\newcommand{\Poly}{\ensuremath\mathsf{P}}
\newcommand{\gR}{\mathcal{R}}

\newtheorem{theorem}{Theorem}[section]
\newtheorem{lemma}[theorem]{Lemma}
\newtheorem{corollary}[theorem]{Corollary}

\newtheorem{definition}[theorem]{Definition}
\newtheorem{example}[theorem]{Example}
\newtheorem{xca}[theorem]{Exercise}

\theoremstyle{remark}
\newtheorem{remark}[theorem]{Remark}

\numberwithin{equation}{section}
\raggedbottom

\begin{document}

\title{On the Complexity of Properties of Partial Bijection Semigroups}
\date{\today}
\author{Trevor Jack}
\address{NOVAMATH -- Center for Mathematics and Applications \\
Faculdade de Ci\^{e}ncias e Tecnologia \\
Universidade Nova de Lisboa \\
2829-516 Caparica, Portugal}
\email{t.jack@fct.unl.pt}

\thanks{This work was partially supported by the Funda\c{c}\~{a}o para a Ci\^{e}ncia e a Tecnologia (Portuguese Foundation for Science and Technology) through the projects UIDB/00297/2020 (Centro de Matemática e Aplica\c{c}\~{o}es) and PTDC/MAT-PUR/31174/2017.}
\keywords{partial bijection semigroups, inverse semigroups, idempotent membership, algorithms, computational complexity, $\PSPACE$-completeness, $\NL$-algorithms, semigroup identities}
\subjclass[2010]{Primary: 20M20; Secondary 68Q25, 20M18}

\begin{abstract}
We examine the computational complexity of problems in which we are given generators for a partial bijection semigroup and asked to check properties of the generated semigroup. We prove that the following problems are in $\AC^0$: (1) enumerating left and right identities and (2) checking if the semigroup is completely regular. We prove that checking membership of a given idempotent is a PSPACE-complete problem. We also describe a nondeterministic logspace algorithm for checking if an inverse semigroup given by generators satisfies a given semigroup identity that may involve a unary inverse operation.
\end{abstract}
\maketitle
\section{Introduction}

Fleischer and the present author \cite{FJ:CP} investigated the computational complexity of checking whether transformation semigroups given by generators exhibit certain properties. There is a natural and efficient embedding of partial bijection semigroups into transformation semigroups, so the results from \cite{FJ:CP} yield naive upper bounds for the complexity of checking for the same properties in partial bijection semigroups. This paper proves stronger results for some of these properties.

A well-known result by Kozen \cite{KO:LBN} is that checking membership in a transformation semigroup given by generators is a $\PSPACE$-complete problem. This paper proves that checking membership of an idempotent in a partial bijection semigroup given by generators is also $\PSPACE$-complete, which yields several other $\PSPACE$-complete problems as corollaries.

Another result in \cite{FJ:CP} is an $\NL$ algorithm for checking if a transformation semigroup given by generators satisfies a given semigroup identity. This paper extends that result to inverse semigroups, which can be thought of as partial bijection semigroups that contain unique inverses for each of their elements \cite[Thm 5.1.7]{HO:FST}. We allow the given semigroup identity to involve a unary inverse operation and describe an $\NL$ algorithm for determining if an inverse semigroup given by generators satisfies the identity.

\section{Preliminaries} \label{NotationSection}

The {\bf full transformation semigroup} over $[n]$, denoted $T_n$, is the set of all mappings $f \colon [n] \to [n]$, together with function composition. Subsemigroups of the full transformation semigroup are often also referred to as {\bf transformation semigroups}.

The {\bf full partial bijection semigroup} over $[n]$, denoted $I_n$, is the set of all partial bijective mappings $f \colon [n] \to [n]$, together with function composition. For $a,b \in I_n$, we define
\[ \dom(a) := \{q \in [n]: \exists p \in [n](qa = p)\} \]
\[\image(a) := \{q \in [n] : \exists p \in [n] (pa = q)\}\]

Subsemigroups of the full partial bijection semigroup are often also referred to as {\bf partial bijection semigroups}. For other standard definitions used in this paper, please see \cite{HO:FST} and \cite{CP:AT}.

\section{Upper Bounds}
There is a natural representation of a partial bijection $a \in I_n$ as a transformation $a' \in T_{n+1}$, where $xa' = xa$ for $x \in \dom(a)$ and $xa' = n+1$ for $x \not \in \dom(a)$. Thus, partial bijection semigroup problems are at most as difficult as their corresponding transformation semigroup problems. Corollary~\ref{cor:red} follows from applying this fact to results from \cite{FJ:CP}, where relevant definitions for these results can be found. Particularly, the complexity class $\AC^0$ includes any problem that can be described by first-order logic: that is, given generators $A \subseteq I_n$, whether the generated semigroup satisfies a first-order formula quantified over the generators $A$ and the points $[n]$. Note that we are not allowed to quantify over every element in the semigroup.

\begin{corollary} \label{cor:red}
Checking if a partial bijection semigroup given by generators:
\begin{enumerate}
\item is commutative is in $\AC^0$ by \cite[Thm 3.2]{FJ:CP},
\item is a semilattice is in $\AC^0$ by  \cite[Thm 3.3]{FJ:CP},
\item is a group is in $\AC^0$ by  \cite[Thm 3.5]{FJ:CP},
\item has left zeroes, right zeroes, or a zero is in $\NL$ by \cite[Thm 4.6]{FJ:CP},
\item is nilpotent is in $\NL$ by \cite[Thm 4.10]{FJ:CP},
\item is $\gR$-trivial is in $\NL$ by \cite[Thm 4.13]{FJ:CP},
\item has central idempotents is in $\NL$ by \cite[Cor 5.2]{FJ:CP},
\item that is commutative is regular is in $\NL$ by \cite[Cor 5.7]{FJ:CP}.
\end{enumerate}
\end{corollary}

Several problems discussed in \cite{FJ:CP} have tighter upper bounds for partial bijection semigroups than for transformation semigroups. For example, we have $\NL$ algorithms for checking if a transformation semigroup has commuting idempotents and whether the product of any two idempotents is idempotent. But these properties are always true for partial bijection semigroups. Also, we have an $\NL$ algorithm for checking if a transformation semigroup is a band, but we can do better for partial bijection semigroups. Since the idempotents of a partial bijection semigroup commute, a partial bijection semigroup is a band iff it is a semilattice, which can be checked in $\AC^0$ by \cite[Thm 3]{FJ:CP}.

The problem of determining left and right identities is also easier in partial bijection semigroups. Recall that an element $\ell$ (resp. $r$) of a semigroup $S$ is a {\bf left} (resp. {\bf right}) {\bf identity} if $\ell s = s$ (resp. $sr = s$) for all $s \in S$. Left and right identities in transformation semigroups can be enumerated in polynomial time \cite[Thm 6.2 and 6.4]{FJ:CP}. We now show that this problem is in $\AC^0$ for partial bijeciton semigroups by proving the following: (1) verifying that at least one left or right identity exists in a transformation semigorup given by generators is in $\AC^0$ and (2) partial bijection semigroups have at most one left identity and one right identity.

\medskip
{\bf Left Identity Existence}
\begin{itemize}
\item Input: $a_1,\dots,a_k \in T_n$.
\item Problem: Does $\langle a_1,\dots,a_k\rangle$ have a left identity?
\end{itemize}

\begin{theorem} \label{thm:leftid}
Left Identity Existence is in $\AC^0$.
\end{theorem}
\begin{proof}
We claim that the semigroup $S = \langle a_1,\dots,a_k \rangle \leq T_n$ has a left identity iff the following first-order formula holds:
\[ \exists i \in [k] \, \forall j \in [k] \, \forall x,y \in [n]: (x a_i = y a_i \rightarrow x a_j = y a_j) \wedge (x a_i^2 = y a_i^2 \rightarrow x a_i = y a_i) \]

Assume $\ell \in S$ is a left identity. By \cite[Lemma 6.1]{FJ:CP}, $\ell$ is the idempotent power of some generator: $\ell = a_i^\omega$. If $x a_i = y a_i$, then $x a_i^\omega a_j = y a_i^\omega a_j$ and thus $x a_j = y a_j$. If $x a_i^2 = y a_i^2$, then $x a_i^{\omega+1} = y a_i^{\omega+1}$ and thus $x a_i = y a_i$.

Assume the first-order formula holds. Let $a_i$ be the generator given by the existential quantifier and let $\ell = a_i^\omega$ be its idempotent power. Pick any $j \in [k]$ and any $x \in [n]$. Starting with $x\ell^2 = x\ell$ and repeatedly applying the second clause of the first-order formula, we obtain $x\ell a_i = x a_i$. Then the first clause yields $x\ell a_j = x a_j$. Thus, $\ell$ is a left identity.
\end{proof}

\medskip
{\bf Right Identity Existence}
\begin{itemize}
\item Input: $a_1,\dots,a_k \in T_n$.
\item Problem: Does $\langle a_1,\dots,a_k\rangle$ have a right identity?
\end{itemize}

\begin{theorem} \label{thm:rightid}
Right Identity Existence is in $\AC^0$.
\end{theorem}
\begin{proof}
We claim that the semigroup $S = \langle a_1,\dots,a_k\rangle \leq T_n$ has a right identity iff the following first-order formula holds:
\[ \exists i \in [k] \, \forall j \in [k] \, \forall x,y \in [n]: x a_j a_i = y a_j a_i \rightarrow x a_j = y a_j \]

Assume $r \in S$ is a right identity. By \cite[Lemma 6.3]{FJ:CP}, $r$ is the idempotent power of some generator: $r = a_i^\omega$. If $xa_j a_i = y a_j a_i$, then $xa_j a_i^\omega = y a_j a_i^\omega$. Since $a_i^\omega$ is a right identity, then $xa_j = y a_j$.

Assume the first-order formula holds. Let $a_i$ be the generator given by the existential quantifier and let $r = a_i^\omega$ be its idempotent power. Pick any $j \in [k]$ and any $x \in [n]$. Starting with $xa_jr^2 = xa_jr$, we can use the formula to remove copies of $a_i$ until we are left with $xa_jr = xa_j$. Thus, $r$ is a right identity.
\end{proof}

\pagebreak

\begin{corollary}
Given generators of a partial bijection semigroup, enumerating left and right identities of the generated semigroup is in $\AC^0$.
\end{corollary}
\begin{proof}
This follows immediately from Theorems~\ref{thm:leftid} and~\ref{thm:rightid} once we prove that partial bijection semigroups can have at most one left identity and one right identity. Let $S = \langle a_1,\dots,a_k \rangle \leq I_n$. Let $\ell \in S$ be a left identity and $r \in S$ be a right identity. Since $\ell$ and $r$ are idempotents, they fix their domains. Because $x \ell a_i = x a_i$ for every $x \in [n]$ and every $i \in [k]$, the image of $\ell$ must be the union of the domains of the generators. Likewise, $x a_i r = x a_i$ forces the domain of $r$ to be the union of the images of the generators. Thus, left and right identities in partial bijection semigroups are uniquely determined.
\end{proof}

Note that a partial bijection semigroup has a two-sided identity iff it has a left identity and a right identity, so checking for a two-sided identity is also in $\AC^0$.

We now consider the problem of determining if a partial bijection semigroup is completely regular. There are several equivalent characterizations of completely regular semigroups \cite[Prop 4.1.1]{HO:FST}. We say a semigroup is {\bf completely regular} if each of its elements generates a subgroup of the semigroup. Determining if a transformation semigroup given by generators is completely regular is in $\NL$ \cite[Thm 5.6]{FJ:CP}, but we can give a stronger result for partial bijection semigroups.

\medskip
{\bf Completely Regular}
\begin{itemize}
\item Input: $a_1,\dots,a_k \in I_n$.
\item Problem: Is $\langle a_1,\dots,a_k\rangle$ completely regular?
\end{itemize}

\begin{theorem}
Completely Regular is in $\AC^0$.
\end{theorem}
\begin{proof}
Let $S = \langle a_1,\dots,a_k \rangle \leq I_n$. We claim $S$ is completely regular iff the following first-order formula holds:
\[ \forall i,j \in [k] : \dom(a_ia_j) = \dom(a_i) \cap \dom(a_j) \]

Assume $S$ is completely regular and pick any $i,j \in [k]$. We first prove that $\dom(a_ia_j) \subseteq \dom(a_i) \cap \dom(a_j)$. Pick any $x \in \dom(a_ia_j)$. Certainly $x \in \dom(a_i)$. For partial bijection semigroups, an element generates a subgroup iff it is a bijection on its domain, so $\dom(a_ia_j) = \image(a_ia_j)$. Thus, $x \in \image(a_j)$ which in turn forces $x \in \dom(a_j)$.

To prove $\dom(a_i) \cap \dom(a_j) \subseteq \dom(a_ia_j)$, note that $\dom(a_i) = (\dom(a_i) \setminus \dom(a_j)) \cup (\dom(a_i) \cap \dom(a_j))$. Pick any $x \in \dom(a_i) \setminus \dom(a_j)$. Then $x \not \in \image(a_j)$, $x \not \in \image(a_ia_j) = \dom(a_ia_j)$, and thus $xa_i \not \in \dom(a_j)$. This proves that, in addition to being a bijection on its domain, $a_i$ is also a bijection on $\dom(a_i) \setminus \dom(a_j)$. Consequently, $a_i$ is a bijection on $\dom(a_i) \cap \dom(a_j)$ so that for any $x \in \dom(a_i) \cap \dom(a_j)$, we know $xa_i \in \dom(a_i) \cap \dom(a_j)$. Since $xa_i \in \dom(a_j)$, then $x \in \dom(a_ia_j)$.

We now prove that $S$ is completely regular if the first-order formula holds. Setting $i=j$ yields $\dom(a_i^2) = \dom(a_i)$ for each $i \in [k]$. That is, each $a_i$ is a bijection on its domain. Consequently, for each $i\in [k]$ and any $s \in S$, $\dom(a_is) = \dom(a_i) \cap \dom(s)$. This inductively proves for any $b_1,\dots,b_\ell \in \{a_1,\dots,a_k\}$ that $\dom(b_1 \cdots b_\ell) = \bigcap_{i=1}^\ell \dom(b_i)$. Likewise, $\image(sa_i) = \image(s) \cap \image(a_i)$ so that $\image(b_1 \cdots b_\ell) = \bigcap_{i=1}^\ell \image(b_i)$. Since $\dom(b_i) = \image(b_i)$ for each $i \in [\ell]$, then $\dom(b_1 \cdots b_\ell) = \image(b_1 \cdots b_\ell)$ and thus $S$ is completely regular.
\end{proof}

Because the idempotents of a partial bijection semigroup commute, every completely regular partial bijection semigroup is also a Clifford semigroup, yielding the following corollary.

\begin{corollary}
Deciding whether a partial bijection semigroup given by generators is a Clifford semigroup is in $\AC^0$.
\end{corollary}

\section{Idempotent Membership}
We now consider the following problem.

\medskip
{\bf Idempotent Membership}
\begin{itemize}
\item Input: $a_1,\dots,a_k, b \in I_n$ with $bb=b$.
\item Problem: Is $b \in \langle a_1,\dots,a_k \rangle$?
\end{itemize}

We will show this problem is $\PSPACE$-complete by reducing from the Corridor Tiling Problem. For this problem, we are given square tiles $T := \{T_1,\dots,T_k\}$, a set of colors $C := \{1,\dots,c\}$, and a positive integer $m$. Each side of each tile is a color from $C$. The problem is whether there is a way to arrange tiles from  $T$ into a grid of fixed length $m$ and some length $n$ such that adjacent edges have the same color and the edges along the border are all colored $1$ \cite{CHL:DTG}. Formally, for an $m \times n$ grid of tiles, we can refer to the tile in the $i^{th}$ row and $j^{th}$ column as $T_{i,j}$ and think of it as a map $T_{i,j}: \{1,2,3,4\} \mapsto C$ representing colors on its (1) north, (2) east, (3) south, and (4) west edges. We define a grid of tiles to be a {\bf proper tiling} if every outside edge has color 1 and adjacent edges have matching colors. In other words, a tiling is proper if the following conditions are true for every $i,j \geq 1$:
\begin{align*}
T_{1,j}(1) &= T_{i,n}(2) = T_{m,j}(3) = T_{i,1}(4) = 1,\\
T_{i,j}(2) &= T_{i,j+1}(4),\text{ and } T_{i,j}(3) = T_{i+1,j}(1).
\end{align*}
\medskip
{\bf Corridor Tiling Problem}
\begin{itemize}
\item Input: $T = \{T_1,\dots,T_k\}$, $C = \{1,\dots,c\}$, and $m \in \mathbb{N}$.
\item Problem: Is there an $n \in \mathbb{N}$ and an $m \times n$ grid that is a proper tiling?
\end{itemize}

\begin{theorem} \label{idmemthm}
  Idempotent Membership for partial bijection semigroups is $\PSPACE$-complete.
\end{theorem}
\begin{proof}
We reduce the input of the Corridor Tiling Problem to a partial bijection semigroup as follows. Define generators $\{a_{i,j}:i \in [m],j \in [k]\}$ to act on $Q:= \{(q,r):q \in [2m], r \in [c]\}$ in the following way.

\begin{align*}
(i,T_j(1))a_{i,j} &= (i+1,T_j(3)) &\text{ for } i < m \\
(i,T_j(1))a_{i,j} &= (1,T_j(3)) &\text{ for } i = m \text{ and } T_j(3) = 1\\
(m+i,T_j(4))a_{i,j} &= (m+i,T_j(2)) \\
(m+p,r)a_{i,j} &= (m+p,r) &\text{ for } p \in [m], p \neq i
\end{align*}

Note that the size of the domain and image for each $a_{i,j}$ are equal, namely $(m-1)c+1$ for $a_{m,j}$ when $T_j(3) \neq 1$ and $(m-1)c+2$ otherwise. Thus, these are partial bijections. Define an idempotent element $b$ that fixes the domain $\{(1,1)$, $(m+1,1)$, $\dots$, $(2m,1)\}$. We claim there exists a grid that is a proper tiling iff $b \in \langle a_{1,1}, \dots, a_{m,k} \rangle$. The motivation for this reduction is illustrated in Figure~\ref{fig:tile}, where $[\alpha,\beta] \in \{1,\dots,k\}$ represents the index of the tile in the $\alpha^{th}$ row and $\beta^{th}$ column of the proper tiling.

\begin{figure}
\centering
\begin{tikzpicture}[scale=.5]
%-------- First Row
\node[anchor=center] at (6,22) {\small $(1,1)$};

\node[anchor=center] at (12,22) {\small $(1,1)$};

\node[anchor=center] at (16,24) {$\cdots$};

\node[anchor=center] at (20,22) {\small $(1,1)$};

%-------- Second Row
\node[anchor=center, rotate=90] at (2.1,18) {\footnotesize $(m+1,1)$};

\node[anchor=center] at (6,18) {\includegraphics{tile.png}};
\node[anchor=center] at (6,18) {\LARGE $T_{[1,1]}$};
\node[anchor=center, rotate=90] at (4.2,18) {\large $a_{1,[1,1]}$};
\node[anchor=center] at (6,19.9) {\large $a_{1,[1,1]}$};
\node[anchor=center, rotate=90] at (8.2,18) {\tiny $(m+1,T_{[1,1]}(2))$};
\node[anchor=center] at (6,16) {\small $(2,T_{[1,1]}(3))$};

\node[anchor=center] at (12,18) {\includegraphics{tile.png}};
\node[anchor=center] at (12,18) {\LARGE $T_{[1,2]}$};
\node[anchor=center, rotate=90] at (10.2,18) {\large $a_{1,[1,2]}$};
\node[anchor=center] at (12,19.9) {\large $a_{1,[1,2]}$};
\node[anchor=center, rotate=90] at (14.2,18) {\tiny $(m+1,T_{[1,2]}(2))$};
\node[anchor=center] at (12,16) {\small $(2,T_{[1,2]}(3))$};

\node[anchor=center] at (16,18) {$\cdots$};

\node[anchor=center] at (20,18) {\includegraphics{tile.png}};
\node[anchor=center] at (20,18) {\LARGE $T_{[1,n]}$};
\node[anchor=center, rotate=90] at (18.2,18) {\large $a_{1,[1,n]}$};
\node[anchor=center] at (20,19.9) {\large $a_{1,[1,n]}$};
\node[anchor=center, rotate=90] at (22.1,18) {\footnotesize $(m+1,1)$};
\node[anchor=center] at (20,16) {\small $(2,T_{[1,n]}(3))$};

%-------- Third Row
\node[anchor=center, rotate=90] at (2,12) {\footnotesize $(m+2,1)$};

\node[anchor=center] at (6,12) {\includegraphics{tile.png}};
\node[anchor=center] at (6,12) {\LARGE $T_{[2,1]}$};
\node[anchor=center, rotate=90] at (4.2,12) {\large $a_{2,[2,1]}$};
\node[anchor=center] at (6,13.8) {\large $a_{2,[2,1]}$};
\node[anchor=center, rotate=90] at (8.2,12) {\tiny $(m+2,T_{[2,1]}(2))$};
\node[anchor=center] at (6,10) {\small $(3,T_{[2,1]}(3))$};

\node[anchor=center] at (12,12) {\includegraphics{tile.png}};
\node[anchor=center] at (12,12) {\LARGE $T_{[2,2]}$};
\node[anchor=center, rotate=90] at (10.2,12) {\large $a_{2,[2,2]}$};
\node[anchor=center] at (12,13.8) {\large $a_{2,[2,2]}$};
\node[anchor=center, rotate=90] at (14.2,12) {\tiny $(m+2,T_{[2,2]}(2))$};
\node[anchor=center] at (12,10) {\small $(3,T_{[2,2]}(3))$};

\node[anchor=center] at (16,12) {$\cdots$};

\node[anchor=center] at (20,12) {\includegraphics{tile.png}};
\node[anchor=center] at (20,12) {\LARGE $T_{[2,n]}$};
\node[anchor=center, rotate=90] at (18.2,12) {\large $a_{2,[2,n]}$};
\node[anchor=center] at (20,13.8) {\large $a_{2,[2,n]}$};
\node[anchor=center, rotate=90] at (22.1,12) {\footnotesize $(m+2,1)$};
\node[anchor=center] at (20,10) {\small $(3,T_{[2,n]}(3))$};

%-------- Fourth Row
\node[anchor=center] at (3,8.6) {\vdots};
\node[anchor=center] at (6,8.6) {\vdots};
\node[anchor=center] at (12,8.6) {\vdots};
\node[anchor=center] at (16,8.6) {$\ddots$};
\node[anchor=center] at (20,8.6) {\vdots};

%-------- Fifth Row
\node[anchor=center, rotate=90] at (2,5) {\small $(2m,1)$};

\node[anchor=center] at (6,5) {\includegraphics{tile.png}};
\node[anchor=center] at (6,5) {\LARGE $T_{[m,1]}$};
\node[anchor=center, rotate=90] at (4.2,5) {\large $a_{m,[m,1]}$};
\node[anchor=center] at (6,6.8) {\large $a_{m.[m,1]}$};
\node[anchor=center, rotate=90] at (8.3,5) {\tiny $(2m,T_{[m,1]}(2))$};
\node[anchor=center] at (6,3.1) {\small $(1,1)$};

\node[anchor=center] at (12,5) {\includegraphics{tile.png}};
\node[anchor=center] at (12,5) {\LARGE $T_{[m,2]}$};
\node[anchor=center, rotate=90] at (10.2,5) {\large $a_{m,[m,2]}$};
\node[anchor=center] at (12,6.8) {\large $a_{m,[m,2]}$};
\node[anchor=center, rotate=90] at (14.3,5) {\tiny $(2m,T_{[m,2]}(2))$};
\node[anchor=center] at (12,3.1) {\small $(1,1)$};

\node[anchor=center] at (16,5) {$\cdots$};

\node[anchor=center] at (20,5) {\includegraphics{tile.png}};
\node[anchor=center] at (20,5) {\LARGE $T_{[m,n]}$};
\node[anchor=center, rotate=90] at (18.2,5) {\large $a_{m,[m,n]}$};
\node[anchor=center] at (20,6.8) {\large $a_{m,[m,n]}$};
\node[anchor=center, rotate=90] at (22,5) {\small $(2m,1)$};
\node[anchor=center] at (20,3.1) {\small $(1,1)$};
\end{tikzpicture}
\caption{Reducing Tiling Problem to Regular Element} \label{fig:tile}

\begin{flushleft} Each column of the grid is bordered above by the point $(1,1)$. Each of the rows $m+1$, $\dots$, $2m$ are bordered to the left by the points $(m+1,1)$, $\dots$, $(2m,1)$, respectively. The center indicates the tile in each position. The left and top edges of each tile indicate the generator that corresponds to that tile. The generator acts on the points listed immediately above and immediately to the left of the tile and the resulting points are indicated on the bottom and right edges of the tile. \end{flushleft}

\end{figure}

To prove that our defined reduction works, first assume there exists a proper tiling as illustrated in Figure~\ref{fig:tile}. We use the indices of these tiles to build the following:

$$c:= \prod \limits_{\beta = 1}\limits^n b_\beta, \hspace{.5in}\text{ with }\hspace{.5in} b_\beta := \prod \limits_{\alpha=1} \limits^m a_{\alpha,[\alpha,\beta]}$$

The composition $b_\beta$ corresponds to the $\beta^{th}$ column of tiles in Figure~\ref{fig:tile}. We prove $c=b$ by proving the following four claims:

\begin{enumerate}
\item $(p,r) \not \in \dom(c)$ for each $p \in [m]$ and $r \in [c]$ except $(1,1)$. \label{c:tilfor1}
\item $c$ fixes $(1,1)$. \label{c:tilfor2}
\item $(m+p,r) \not \in \dom(c)$ for each $p \in [m]$ and $r \neq 1$. \label{c:tilfor3}
\item $c$ fixes $(m+p,1)$ for each $p \in [m]$. \label{c:tilfor4}
\end{enumerate}

Claim~\ref{c:tilfor1} follows directly from the domain of the first generator $a_{1,[1,1]}$, since $T_{[1,1]}(1) = 1$. To prove Claim~\ref{c:tilfor2}, we prove each $b_\beta$ fixes $(1,1)$.

\begin{align*}
(1,1) \prod \limits_{\alpha \in [m]} a_{\alpha,[\alpha,\beta]} &= (2,T_{[1,\beta]}(3)) \prod \limits_{\alpha=2} \limits^m a_{\alpha,[\alpha,\beta]} &\text{since }T_{[1,\beta]}(1) = 1 \\
&= (m,T_{[m-1,\beta]}(3)) a_{m,[m,\beta]} &\text{since }T_{[\alpha+1,\beta]}(1) = T_{[\alpha,\beta]}(3) \\
&= (1,1) &\text{since }T_{[m,\beta]}(3) = 1
\end{align*}

Since each $b_\beta$ fixes $(1,1)$, so does $c$.

Claim~\ref{c:tilfor3} follows directly from the domain of $b_1$. Because the tiling is proper, $T_{[\alpha,1]}(4) = 1$, so $(m+\alpha,r) \in \dom(a_{\alpha,[\alpha,1]})$ only when $r = 1$.  Also, each $a_{\alpha,[\alpha,1]}$ fixes $(m+p,r)$ for every $p\neq \alpha$. Thus, $(m+p,r) \in \dom(b_1)$ only when $r=1$.

Finally, we prove Claim~\ref{c:tilfor4}.

\begin{align*}
(m+p,1) \prod \limits_{\beta=1}^n b_\beta &= (m+p,T_{[p,1]}(2)) \prod \limits_{\beta=2}^n b_\beta &\text{ since } T_{[p,1]}(4) = 1 \\
&= (m+p,T_{[p,n-1]}(2) &\text{ since } T_{p,\beta+1}(4) = T_{p,\beta}(2) \\
&= (m+p,1) &\text{ since } T_{[p,n]}(2) = 1
\end{align*}

Thus, a proper tiling forces $b \in \langle a_{1,1} , \dots, a_{m,k} \rangle$.

Conversely, assume that $b \in \langle a_{1,1}, \dots, a_{m,k} \rangle$. We prove that the condition $(1,1)b=(1,1)$ forces $b$ to have the following form:
\begin{equation} \label{eq:b}
b = \prod_{\beta=1}^n \prod_{\alpha=1}^m a_{\alpha,[\alpha,\beta]}.
\end{equation}
We also prove that the terms $\prod_{\alpha=1}^m a_{\alpha,[\alpha,\beta]}$ encode columns that are proper tilings for each $\beta \in [n]$. We begin with proving the following claims for any $d = a_{i_1,j_1} \cdots a_{i_\ell,j_\ell}$ that fixes $(1,1)$.

\begin{enumerate}
\item $\ell \geq m$ and $i_\alpha = \alpha$ for each $\alpha \in [m]$, \label{c:col1}
\item $T_{j_1}(1) = 1$, \label{c:col2}
\item $T_{j_p}(3) = T_{j_{p+1}}(1)$ for each $1 \leq p < m$, \label{c:col3}
\item $T_{j_m}(3) = 1$, and \label{c:col4}
\item $(1,1)\prod_{\alpha=1}^m a_{\alpha,j_\alpha}=(1,1)$. \label{c:col5}
\end{enumerate}

Note that the only generators that have $(1,1)$ in their domain are of the form $a_{1,j_1}$ with $T_{j_1}(1) = 1$ and that $(1,1)a_{1,j_1} = (2,T_{j_1}(3))$. Thus, Claim~\ref{c:col2} holds. Also, the only generators with $(2,T_{j_1}(3))$ in their domain are of the form $a_{2,j_2}$ with $T_{j_2}(1) = T_{j_1}(3)$. Thus, Claim~\ref{c:col3} holds for $p=1$. Assume $d = a_{1,j_1} \cdots a_{p,j_p} a_{i_{p+1},j_{p+1}} \cdots a_{i_\ell,j_\ell}$ and $(1,1)a_{1,i_1} \cdots a_{p,i_p} = (p+1,T_{j_p}(3))$ for some $p < m$. Then the $(p+1)^{th}$ generator must be of the form $a_{p+1,j_{p+1}}$ with $T_{j_{p+1}}(1) = T_{j_p}(3)$, inductively proving Claim~\ref{c:col1} and Claim~\ref{c:col3}. Now $(1,1)a_{1,j_1}$ $\cdots$ $a_{m,j_m} = (m,T_{j_{m-1}}(3))a_{m,j_m}$. The only way $(1,1)$ remains in the domain of this composition is if $T_{j_m}(3) = 1$, proving Claim~\ref{c:col4}. In that case, $(m,T_{j_{m-1}}(3))a_{m,j_m} = (1,1)$, proving Claim~\ref{c:col5}.

Because $(1,1)b = (1,1)$, the above claim proves that its first $m$ generators must be of the form $\prod_{\alpha=1}^m a_{\alpha,[\alpha,1]}$ for which $T_{[1,1]},\dots,T_{[m,1]}$ forms a column that is a proper tiling. We now inductively prove Equation~\ref{eq:b}. Assume $b=(\prod_{\beta=1}^p$ $\prod_{\alpha=1}^m a_{\alpha,[\alpha,\beta]})$ $b'$ for some $p$ where $\prod_{\alpha=1}^m a_{\alpha,[\alpha,\beta]}$ satisfies Claims~\ref{c:col1}-\ref{c:col5} for each $\beta \in [p]$. If $b'$ is not the empty word, then since $b$ and each factor $\prod_{\alpha=1}^m a_{\alpha,j_\alpha}$ fixes $(1,1)$, $b'$ must also fix $(1,1)$. Then Claims~\ref{c:col1}-\ref{c:col5} prove that $b'$ begins with yet another composition of the form $\prod_{\alpha=1}^m a_{\alpha,j_\alpha}$ that satisfies those claims. This inductively proves that $b$ has the form $\prod_{\beta = 1}^n \prod_{\alpha=1}^m a_{\alpha,[\alpha,\beta]}$ with terms $\prod_{\alpha=1}^m a_{\alpha,[\alpha,\beta]}$ that encode columns that are proper tilings for each $\beta \in [n]$.

We now prove that the condition $(m+p,1)b=(m+p,1)$ forces the $p^{th}$ row of the corresponding tiling to be a proper tiling, thereby proving that $b \in \langle a_{1,1},\dots,a_{m,k}\rangle$ forces a proper tiling. That is, we prove the following claims.
\begin{enumerate}
\item $T_{[p,1]}(4) = 1$, \label{c:row1}
\item $T_{[p,\beta]}(2) = T_{[p,\beta+1]}(4)$ for each $1 \leq \beta < n$, and \label{c:row2}
\item $T_{[p,n]}(2) = 1$. \label{c:row3}
\end{enumerate}

Since $m < m+p$, no generator will change the first coordinate. Furthermore, the only generators that will change the second coordinate are of the form $a_{p,j}$. Then 
\begin{align*}
(m+p,1)\prod_{\alpha=1}^m a_{\alpha,[\alpha,1]} &= (m+p,1) \prod_{\alpha=p}^m a_{\alpha,[\alpha,1]} \\
&= (m+p,T_{[p,1]}(2)) \prod_{\alpha=p+1}^m a_{\alpha,[\alpha,1]}\\
&= (m+p,T_{[p,1]}(2))
\end{align*}

The second line forces $T_{[p,1]}(4) = 1$, proving Claim~\ref{c:row1}. We prove Claim~\ref{c:row2} by induction, assuming that, for some fixed $q < n$, the claim holds for each $\beta \leq q$ and that the following is true.

$$(m+p,1) \prod_{\beta=1}^q \prod_{\alpha=1}^m a_{\alpha,[\alpha,\beta]} = (m+p, T_{[p,q]}(2))$$
Then,
\begin{align*}
(m+p,1)\prod_{\beta=1}^{q+1} \prod_{\alpha=1}^m a_{\alpha,[\alpha,\beta]} &= (m+p, T_{[p,q]}(2)) \prod_{\alpha=1}^m a_{\alpha,[\alpha,q+1]} \\
&= (m+p, T_{[p,q]}(2)) \prod_{\alpha=p}^m a_{\alpha,[\alpha,q+1]} \\
&= (m+p,T_{[p,q+1]}(2)) \prod_{\alpha=p+1}^m a_{\alpha,[\alpha,q+1]}\\
&= (m+p,T_{[p,q+1]}(2))
\end{align*}

The third line forces $T_{[p,q]}(2) = T_{[p,q+1]}(4)$ proving the induction step for Claim~\ref{c:row2}. Finally, the requirement that $(m+p,1)b=(m+p,1)$ forces Claim~\ref{c:row3} since the induction above shows that $(m+p,1)b=(m+p,T_{[p,n]}(2))$. Thus, $b \in \langle a_{1,1} , \dots, a_{m,k}\rangle$ forces a proper tiling.
\end{proof}

There are natural reductions from the Idempotent Membership to the following problems.

\medskip
{\bf Membership for Partial Bijection Semigroups}
\begin{itemize}
\item Input: $a_1,\dots,a_k,b \in I_n$.
\item Problem: $Is b \in \langle a_1,\dots,a_k\rangle$?
\end{itemize}

\medskip
{\bf Idempotent Membership for Transformation Semigroups}
\begin{itemize}
\item Input: $a_1,\dots,a_k,b \in T_n$ with $bb=b$.
\item Problem: Is $b \in \langle a_1,\dots,a_k\rangle$?
\end{itemize} 

\medskip
{\bf Idempotent Membership for Matrix Semigroups}
\begin{itemize}
\item Input: $a_1,\dots,a_k,b \in \mathbb{F}^{n \times n}$ with $bb=b$.
\item Problem: Is $b \in \langle a_1,\dots,a_k\rangle$?
\end{itemize} 

\medskip
{\bf Membership for Transformation Semigroups}
\begin{itemize}
\item Input: $a_1,\dots,a_k,b \in T_n$.
\item Problem: Is $b \in \langle a_1,\dots,a_k\rangle$?
\end{itemize} 

\medskip
{\bf Membership for Matrix Semigroups}
\begin{itemize}
\item Input: $a_1,\dots,a_k,b \in \mathbb{F}^{n \times n}$.
\item Problem: Is $b \in \langle a_1,\dots,a_k\rangle$?
\end{itemize} 

\begin{corollary}
Membership for Partial Bijection Semigroups, Idempotent Membership for Transformation Semigroups, Idempotent Membership for Matrix Semigroups, Membership for Transformation Semigroups, and Membership for Matrix Semigroups are all PSPACE-complete problems.
\end{corollary}
\begin{proof}
That these problems can be solved in polynomial space is immediate since we can guess generators to compose to be $b$ using only polynomial space to store the composition. PSPACE-hardness is immediate since every partial bijection semigroup $S \leq I_n$ can be represented as a transformation semigroup $S' \leq T_{n+1}$, which can then be represented as a matrix semigroup $S'' \leq \mathbb{F}^{(n+1) \times (n+1)}$.
\end{proof}

\iffalse %------------------------------------------------------------------------------------
\begin{theorem} \label{memthm}
  Membership is $\PSPACE$-complete.
\end{theorem}
\begin{proof}
Denote each $T_i$ as a map $T_i: \{1,2,3,4\} \mapsto [c]$ representing its (1) north, (2) east, (3) south, and (4) west edge colors. We reduce to a partial bijection semigroup as follows. We define $m$ generators for each of the $k$ tiles, $a_{1,0},\dots,a_{k,m-1}$, and one additional generator $a_0$. The underlying set has $2m$ subsets of $c$ points each, which we denote as $(q,r)$ with $q \in \{0,\dots,2m-1\}$ and $r \in [c]$. The set has an additional point $(0,0)$, which will allow $a_0$ to act as a resetting map. The first $m$ blocks will ensure that the tiles match for each length-$m$ column of the grid  The second $m$ blocks will ensure that the tiles match for each of the $m$ rows of the grid. We define the generators as follows:

$$
(q,r)a_{i,j} = \begin{cases}
(q+1,T_i(3)) & \text{ if } q = j  < m - 1 \text{ and } r = T_i(1) \\
(0,0) & \text{ if } q = j  = m - 1 \text{ and } r = T_i(1) \text{ and } T_i(3) = T_0(1)\\
(q,T_i(2)) & \text{ if } q = j+m \text{ and } r = T_i(4) \\
(q,r) & \text{ if }  m \leq q \neq j+m
\end{cases}
$$
$$
(q,r)a_0 = \begin{cases}
(0,T_0(3)) & \text{ if } (q,r) = (0,0) \\
(q,r) & \text{ if } m \leq q
\end{cases}
$$

Define $b$ to transform the domain $\{(0,T_0(3)),(m,T_0(2)),(1+m,T_0(2)),\dots,(2m-1,T_0(2))\}$ as follows: $(0,T_0(3))c = (0,T_0(3))$ and $(j+m,T_0(2))c = (j+m,T_0(4))$ for each $j \in \{0,\dots,m-1\}$. We claim that $c \in \langle a_{1,0},\dots,a_{k,m-1}, b \rangle$ iff there is a tiling satisfying the Corridor Tiling Problem.For the forward direction, note that the only composition that will fix $(0,T_0(3))$ is compositions of sequences of the following form: $a_{i_0,0} \cdots a_{i_{m-1},m-1} b$. Furthermore, for each $j \in \{0,\dots,m-2\}$, $(0,T_0(3))a_{i_0,0} \cdots a_{i_j,j}$ is only in the domain of $a_{i_{j+1},j+1}$ if $T_j(3) = T_{j+1}(1)$. Finally, $(0,T_0(3))a_{i_0,0} \cdots a_{i_{m-1},m-1}$ must equal $(0,0)$ and thus $T_{m-1}(3) = T_0(1)$. This ensures that each column of tiles, corresponding to $a_{i_0,0},\dots,a_{i_{m-1},m-1}$ must have matching colors.

To see that the rows have matching colors, note that, for any sequence $a_{i_0,0} \cdots a_{i_{m-1},m-1}$, each of the $m$ to $2m-1$ blocks of $c$ points are moved by exactly one of the $a_{i_0,0},\dots,a_{i_{m-1},m-1}$ generators. Also, each block is transformed according to the colors of the corresponding tiles. Finally, the domain of each generator has only one point from the block that it does not fix, corresponding to the color on the west side of the tile. So, the condition $(j+m, T_0(2))c=(j+m, T_0(4))$ ensures that the colors of the rows match and that the east and west ends of the grid can be extended by $T_0$ tiles.

For the converse, we can denote the satisfying tiling as $T_{i_1},\dots,T_{i_j}$, ordered by columns from north to south and then rows from west to east. This yields generators $c = (a_{i_1,0}\cdots a_{i_{m-1},m-1})b(a_{i_m,0}\cdots a_{i_{2m-1},m-1})b((a_{i_{j-m},0}\cdots a_{{i_j},m-1})b$. The point $(0,T_0(3))$ will be fixed by each initial composition ending with $b$ and thus fixed by $c$. Also, the definitions of the generators ensures that $(j+m, T_0(2))c= (j+m,T_0(4))$.

We prove \NP-hardness by reducing from the \NP-complete problem Exact Cover. Let $A_1,\dots,A_k$ be subsets of $[n]$ and define $a_1,\dots,a_k,b \in I_{2n}$ as follows:

$$qa_i := \begin{cases}
q+n & \text{if }q \in A_i \\
q & \text{if }q \leq n \text{ or }q-n \not \in A_i
\end{cases}
\hspace{1cm}
qb := \begin{cases}
q+n & \text{if }q \leq n
\end{cases}
$$

We claim that $b \in \langle a_1,\dots,a_k\rangle$ iff there are disjoint sets from $A_1,\dots,A_k$ whose union is $[n]$. Let $A_{i_1},\dots,A_{i_j}$ be such disjoint sets. Then each $q \in [n]$ is moved by exactly one of $a_{i_1},\dots,a_{i_j}$ and fixed by the rest. So, $b=a_{i_1}\cdots a_{i_j}$. Now assume that $a_{i_1}\cdots a_{i_j} = b$. Now assume that $b = a_{i_1}\cdots a_{i_j}$ and note that, for each $q \in [n]$, $qb=q+n$. Then $q$ must be fixed by all except one of $a_{i_1},\dots,a_{i_j}$. Then, $A_{i_1},\dots,A_{i_j}$ forms a disjoint collection whose union is $[n]$.
\end{proof}
\fi %------------------------------------------------------------------------------------

\section{Model Checking}

\cite[Thm 5.1]{FJ:CP} gives an $\NL$ algorithm for checking if a transformation semigroup given by generators satisfies a given semigroup identity, We generalize that result to inverse partial bijection semigroups and semigroup identities that may involve a unary inverse operation. Let $a_1,\dots,a_k$ be partial bijective maps, each defined on subsets of $[n]$, and let $S= \langle a_1,\dots,a_k,a_1^{-1},\dots,a_k^{-1}\rangle$. Let $X^*$ be the free algebra over the variables $X = \{x_1,\dots,x_m,x_1^{-1},\dots,x_m^{-1}\}$. A map $h:X^* \to S$ is a {\bf homomorphism} if $h(ab) = h(a)h(b)$ for each $a,b \in X^*$ and $h(x_i^{-1}) = h(x_i)^{-1}$ for each $i \in [m]$. Let $u$ and $v$ be two elements of $X^*$. We say that an inverse semigroup $S$ {\bf models} $u \eq v$ if $h(u) = h(v)$ holds for each homomorphism $h:X^* \to S$. For a fixed identity $u \eq v$, define the following problem:

\medskip
{$\Modelbf(u \eq v)$}
\begin{itemize}
\item Input: $a_1,\dots,a_k \in I_n$
\item Problem: Does $\langle a_1,\dots,a_k,a_1^{-1},\dots,a_k^{-1} \rangle$ model $u \eq v$?
\end{itemize} 

We will show that this class of problems belongs to $\NL$ by showing that a broader class of problems also belongs to $\NL$. We say that an inverse semigroup $S$ {\bf models} $x_1 = x_1^2,\dots,x_e = x_e^2 \Rightarrow u = v$ if for all homomorphisms $h \colon X^* \to S$ with $h(x_1), \dots, h(x_e)$ idempotent, we have $h(u) = h(v)$.

\medskip
{$\Modelbf(x_1 \eq x_1^2,\dots,x_e \eq x_e^2 \Rightarrow u \eq v)$}
\begin{itemize}
\item Input: $a_1,\dots,a_k \in T_n$
\item Problem: Does $\langle a_1,\dots,a_k \rangle$ model $x_1 \eq x_1^2,\dots,x_e \eq x_e^2 \Rightarrow u \eq v$?
\end{itemize}

\begin{theorem}
  Let $X = \{x_1, \dots, x_m,x_1^{-1},\dots,x_m^{-1}\}$ be a nonempty finite set of variables and let $u, v \in X^*$. Then, $\Model(x_1 \eq x_1^2,\dots,x_e \eq x_e^2 \Rightarrow u \eq v)$ belongs to $\NL$.
  \label{thm:model}
\end{theorem}
%
\begin{proof}
  Let $u \eq x_{i_1}^{f_1} \cdots x_{i_\ell}^{f_\ell}$ and $v \eq x_{j_1}^{g_1} \cdots x_{j_r}^{g_r}$ with $i_1,\dots,i_\ell,j_1,\dots,j_r \in [m]$ and $f_1,\dots,f_\ell,g_1,\dots,g_r \in \{-1,1\}$. We describe an $\NL$ algorithm to test whether an inverse semigroup $S = \langle a_1, \dots, a_k,a_1^{-1},\dots,a_k^{-1} \rangle$ does \emph{not} model
  \begin{align*}
    x_1 \eq x_1^2,\dots,x_e \eq x_e^2 \Rightarrow u \eq v.
  \end{align*}
  
  Since $\NL$ is closed under complementation, this implies that the decision problem $\Model(x_1 \eq x_1^2,\dots,x_e \eq x_e^2 \Rightarrow u \eq v)$ belongs to $\NL$.
  %
  For each $i \in [m]$, we let $P_1(i) = \{p \in [\ell]:i_p = i\}$ and $P_2(i) = \{p \in [r]:j_p = i\}$.
  %
  The algorithm is depicted in Algorithm~\ref{alg:models}.
  %
  Since $\ell + r$ is a constant, the algorithm only requires logarithmic
  space.

  \begin{algorithm}
  \caption{$\co\NL$ algorithm for $\Model(u \eq v)$}
    \label{alg:models}
    \begin{algorithmic}[1]
      \Input{$a_1,\dotsc,a_k \in I_n$}
      \Output{Does $\langle a_1,\dots,a_k,a_1^{-1},\dots,a_k^{-1} \rangle$ \emph{not} model $u \eq v$?}
      \State guess integers $p_1, \dots, p_{\ell+1}, q_1, \dots, q_{r+1} \in [n]$
      \LineIf{$p_1 \ne q_1$ or $p_{\ell+1} = q_{r+1}$}{\Reject}
      \ForAll{$i \in [m]$}
      \LineForAll{$j \in [\ell]$}{$p_j' := p_j$,\, $p_j'' := p_{j+1}$}
      \LineForAll{$j \in [r]$}{$q_j' := q_j$,\, $q_j'' := q_{j+1}$}
        \Repeat
          \State guess $c \in [k]$
          \LineForAll{$j \in P_1(i)$}{
            \LineIf{$f_j = 1$}{$p_j' := p_j'a_c$,\, $p_j'' := p_j''a_c$}
            \LineIf{$f_j = -1$}{$p_{j+1}' := p_{j+1}'a_c,\, p_{j+1}'' := p_{j+1}'' a_c$}}
          \LineForAll{$j \in P_2(i)$}{
            \LineIf{$g_j = 1$}{$q_j' := q_j'a_c$,\, $q_j'' := q_j''a_c$}
            \LineIf{$g_j = -1$}{$q_{j+1}' := q_{j+1}'a_c,\, q_{j+1}'' := q_{j+1}''a_c$}}
         \Until{[$\forall j \in P_1(i) \colon p_j' = p_{j+1}$ if $f_j = 1$ and $p_{j+1}' = p_j$ if $f_j = -1]$ and \\ \hspace{1.35cm} $[\forall j \in P_2(i) \colon q_j' = q_{j+1}$ if $g_j = 1$ and $q_{j+1}' = q_j$ if $g_j = -1]$ and \\ \hspace{1.35cm} $[i \in [e] \Rightarrow (\forall j \in P_1(i) \colon p_j'' = p_{j+1}$ if $f_j = 1$ and $p_{j+1}'' = p_j$ if $f_j = -1)$\\ \hspace{1.35cm} and $(\forall j \in P_2(i) \colon q_j'' = q_{j+1}$ if $g_j = 1$ and $q_{j+1}'' = q_j$ if $g_j = -1)]$}
      \EndFor
      \State\Accept
    \end{algorithmic}
  \end{algorithm}

  The process corresponds to nondeterministically replacing each variable
  in $X$ by an element of $S$ such that the left-hand side and the right-hand
  side of the equation map the point $p_1 = q_1 \in [n]$ to distinct points $p_{\ell+1}, q_{r+1} \in [n]$. A formal correctness proof follows.
  
  First, suppose that the input $S = \langle a_1,\dots,a_k\rangle$ does not model $x_1 \eq x_1^2,\dots,x_e \eq x_e^2 \Rightarrow u \eq v$. This means that there are elements $s_1,\dots,s_m \in S$ such that $s_{i_1}^{f_1}\cdots s_{i_\ell}^{f_\ell} \neq s_{j_1}^{g_1}\cdots s_{j_r}^{g_r}$ and $s_1,\dots,s_e$ are idempotent. Pick a $p_1 \in [n]$ such that $p_1 s_{i_1}^{f_1} \cdots s_{i_\ell}^{f_\ell} \neq p_1 s_{j_1}^{g_1} \cdots s_{j_r}^{g_r}$. Let $q_1 := p_1$. For each $\alpha \in [\ell]$, let $p_\alpha = p_1 s_{i_1}^{f_1} \cdots s_{i_{\alpha-1}}^{f_{\alpha-1}}$. For each $\alpha \in [r]$, let $q_\alpha = q_1 s_{j_1}^{g_1} \cdots s_{j_{\alpha-1}}^{g_{\alpha-1}}$.

  To verify that the algorithm will accept the input, consider any $s_i \in \{s_1,\dots,s_m\}$. Let $s_i = a_{c_1} \cdots a_{c_d}$ with $c_1, \dots, c_d \in [k]$. Lines 6--17 will successively guess the generators and transform the points $p_j'$ and $p_{j+1}'$ for each $j \in P_1(i)$; likewise for $q_j'$ and $q_{j+1}'$ for each $j \in P_2(i)$. When this loop completes, the algorithm will ensure the following. For each $j \in P_1(i)$, $f_i = 1$ implies $p'_j = p_js_i = p_{j+1}$ and $f_i = -1$ implies $p'_{j+1} = p_{j+1}s_i$. Note that, in the latter case, $p_js_i^{-1} = p_{j+1}$, so $p_j = p_{j+1}s_i$ and thus $p'_{j+1} = p_j$. The algorithm works likewise for the points $q_j'$ and $q_{j+1}'$ for each $j \in P_2(i)$. Finally, Lines 16-17 are satisfied since $s_i$ and $s_i^{-1}$ are idempotent.
  
  We now prove that if the algorithm accepts, then $S = \langle a_1,\dots,a_k\rangle$ does not model $x_1 \eq x_1^2,\dots,x_e \eq x_e^2 \Rightarrow u \eq v$.
  %
  Let $p_1,\dots,p_{\ell+1},q_1,\dots,q_{r+1}$ be the guessed integers in Line 1. For each $i \in [m]$, let $s_i = a_{c_1}\cdots a_{c_g}$ be the sequence of guessed generators in Line 7.
  %
  Then for each $j \in P_1(i)$: $p_j s_i = p_{j+1}$ if $f_i=1$ and $p_j s_i^{-1} = p_{j+1}$ if $f_i = -1$. Likewise, for each $j \in P_2(i)$, $q_j s_i = q_{j+1}$ if $g_i=1$ and $q_j s_i^{-1} = q_{j+1}$ if $g_i=-1$. Let $s_i^\omega$ be the idempotent power of $s_i$. Then for each $j \in P_1(i)$ with $j \le e$, we have $p_{j+1}s_i^\omega = p_{j+1} s_i = p_{j+1}$ and for each $j \in P_2(i)$ with $j \le e$, we have $q_{j+1}s_i^\omega = q_{j+1} s_i = q_{j+1}$.
  
  By the definitions of $P_1(i)$ and $P_2(i)$, this demonstrates that $p_1 h(u) = p_{\ell+1}$ and $q_1 h(v) = q_{r+1}$ where $h \colon X^+ \to S$ is the homomorphism defined by $h(x_i) = s_i^\omega$ for all $i \in [e]$ and $h(x_i) = s_i$ for all $i \in \{e+1,\dots,m\}$. By Line~2 of the algorithm, we obtain $h(u) \ne h(v)$, thereby concluding the proof.
\end{proof}

\section{Open Problems}
{\bf Problem 6.1}: What are the lower bounds for the problems mentioned in Corollary~\ref{cor:red} with respect to partial bijection semigroups? Are there tigher upper bounds than the ones provided in Corollary~\ref{cor:red}?

\medskip
{\bf Problem 6.2}: What is the computational complexity of checking membership of an idempotent in an inverse semigroup?

\section{Acknowledgements}
The author would like to thank Peter Mayr and Alan Cain for their valuable comments and contributions.

\begin{thebibliography}{1}

\bibitem{CHL:DTG}
B.~S. Chlebus.
\newblock Domino-tiling games.
\newblock {\em J. Comput. System Sci.}, 32(3):374--392, 1986.
\newblock \href {http://dx.doi.org/10.1016/0022-0000(86)90036-X}
  {\path{doi:10.1016/0022-0000(86)90036-X}}.

\bibitem{CP:AT}
A.~H. Clifford and G.~B. Preston.
\newblock {\em The algebraic theory of semigroups. {V}ol. {I}}.
\newblock Mathematical Surveys, No. 7. American Mathematical Society,
  Providence, R.I., 1961.

\bibitem{FJ:CP}
L.~Fleischer and T.~Jack.
\newblock The complexity of properties of transformation semigroups.
\newblock {\em Internat. J. Algebra Comput.}, 30(3):585--606, 2020.
\newblock \href {http://dx.doi.org/10.1142/S0218196720500125}
  {\path{doi:10.1142/S0218196720500125}}.

\bibitem{HO:FST}
J.~M. Howie.
\newblock {\em Fundamentals of semigroup theory}, volume~12 of {\em London
  Mathematical Society Monographs. New Series}.
\newblock The Clarendon Press, Oxford University Press, New York, 1995.
\newblock Oxford Science Publications.

\bibitem{KO:LBN}
D.~Kozen.
\newblock {\em Lower bounds for natural proof systems}.
\newblock IEEE Comput. Sci., Long Beach, Calif., 1977.

\end{thebibliography}
\end{document}
