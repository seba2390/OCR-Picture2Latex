
%% bare_jrnl.tex
%% V1.4b
%% 2015/08/26
%% by Michael Shell
%% see http://www.michaelshell.org/
%% for current contact information.
%%
%% This is a skeleton file demonstrating the use of IEEEtran.cls
%% (requires IEEEtran.cls version 1.8b or later) with an IEEE
%% journal paper.
%%
%% Support sites:
%% http://www.michaelshell.org/tex/ieeetran/
%% http://www.ctan.org/pkg/ieeetran
%% and
%% http://www.ieee.org/

%%*************************************************************************
%% Legal Notice:
%% This code is offered as-is without any warranty either expressed or
%% implied; without even the implied warranty of MERCHANTABILITY or
%% FITNESS FOR A PARTICULAR PURPOSE!
%% User assumes all risk.
%% In no event shall the IEEE or any contributor to this code be liable for
%% any damages or losses, including, but not limited to, incidental,
%% consequential, or any other damages, resulting from the use or misuse
%% of any information contained here.
%%
%% All comments are the opinions of their respective authors and are not
%% necessarily endorsed by the IEEE.
%%
%% This work is distributed under the LaTeX Project Public License (LPPL)
%% ( http://www.latex-project.org/ ) version 1.3, and may be freely used,
%% distributed and modified. A copy of the LPPL, version 1.3, is included
%% in the base LaTeX documentation of all distributions of LaTeX released
%% 2003/12/01 or later.
%% Retain all contribution notices and credits.
%% ** Modified files should be clearly indicated as such, including  **
%% ** renaming them and changing author support contact information. **
%%*************************************************************************


% *** Authors should verify (and, if needed, correct) their LaTeX system  ***
% *** with the testflow diagnostic prior to trusting their LaTeX platform ***
% *** with production work. The IEEE's font choices and paper sizes can   ***
% *** trigger bugs that do not appear when using other class files.       ***                          ***
% The testflow support page is at:
% http://www.michaelshell.org/tex/testflow/



\documentclass[journal]{IEEEtran}
%
% If IEEEtran.cls has not been installed into the LaTeX system files,
% manually specify the path to it like:
% \documentclass[journal]{../sty/IEEEtran}





% Some very useful LaTeX packages include:
% (uncomment the ones you want to load)


% *** MISC UTILITY PACKAGES ***
%
%\usepackage{ifpdf}
% Heiko Oberdiek's ifpdf.sty is very useful if you need conditional
% compilation based on whether the output is pdf or dvi.
% usage:
% \ifpdf
%   % pdf code
% \else
%   % dvi code
% \fi
% The latest version of ifpdf.sty can be obtained from:
% http://www.ctan.org/pkg/ifpdf
% Also, note that IEEEtran.cls V1.7 and later provides a builtin
% \ifCLASSINFOpdf conditional that works the same way.
% When switching from latex to pdflatex and vice-versa, the compiler may
% have to be run twice to clear warning/error messages.






% *** CITATION PACKAGES ***
%
%\usepackage{cite}
% cite.sty was written by Donald Arseneau
% V1.6 and later of IEEEtran pre-defines the format of the cite.sty package
% \cite{} output to follow that of the IEEE. Loading the cite package will
% result in citation numbers being automatically sorted and properly
% "compressed/ranged". e.g., [1], [9], [2], [7], [5], [6] without using
% cite.sty will become [1], [2], [5]--[7], [9] using cite.sty. cite.sty's
% \cite will automatically add leading space, if needed. Use cite.sty's
% noadjust option (cite.sty V3.8 and later) if you want to turn this off
% such as if a citation ever needs to be enclosed in parenthesis.
% cite.sty is already installed on most LaTeX systems. Be sure and use
% version 5.0 (2009-03-20) and later if using hyperref.sty.
% The latest version can be obtained at:
% http://www.ctan.org/pkg/cite
% The documentation is contained in the cite.sty file itself.






% *** GRAPHICS RELATED PACKAGES ***
%
\ifCLASSINFOpdf
  % \usepackage[pdftex]{graphicx}
  % declare the path(s) where your graphic files are
  % \graphicspath{{../pdf/}{../jpeg/}}
  % and their extensions so you won't have to specify these with
  % every instance of \includegraphics
  % \DeclareGraphicsExtensions{.pdf,.jpeg,.png}
\else
  % or other class option (dvipsone, dvipdf, if not using dvips). graphicx
  % will default to the driver specified in the system graphics.cfg if no
  % driver is specified.
  % \usepackage[dvips]{graphicx}
  % declare the path(s) where your graphic files are
  % \graphicspath{{../eps/}}
  % and their extensions so you won't have to specify these with
  % every instance of \includegraphics
  % \DeclareGraphicsExtensions{.eps}
\fi
% graphicx was written by David Carlisle and Sebastian Rahtz. It is
% required if you want graphics, photos, etc. graphicx.sty is already
% installed on most LaTeX systems. The latest version and documentation
% can be obtained at:
% http://www.ctan.org/pkg/graphicx
% Another good source of documentation is "Using Imported Graphics in
% LaTeX2e" by Keith Reckdahl which can be found at:
% http://www.ctan.org/pkg/epslatex
%
% latex, and pdflatex in dvi mode, support graphics in encapsulated
% postscript (.eps) format. pdflatex in pdf mode supports graphics
% in .pdf, .jpeg, .png and .mps (metapost) formats. Users should ensure
% that all non-photo figures use a vector format (.eps, .pdf, .mps) and
% not a bitmapped formats (.jpeg, .png). The IEEE frowns on bitmapped formats
% which can result in "jaggedy"/blurry rendering of lines and letters as
% well as large increases in file sizes.
%
% You can find documentation about the pdfTeX application at:
% http://www.tug.org/applications/pdftex





% *** MATH PACKAGES ***
%
%\usepackage{amsmath}
% A popular package from the American Mathematical Society that provides
% many useful and powerful commands for dealing with mathematics.
%
% Note that the amsmath package sets \interdisplaylinepenalty to 10000
% thus preventing page breaks from occurring within multiline equations. Use:
%\interdisplaylinepenalty=2500
% after loading amsmath to restore such page breaks as IEEEtran.cls normally
% does. amsmath.sty is already installed on most LaTeX systems. The latest
% version and documentation can be obtained at:
% http://www.ctan.org/pkg/amsmath





% *** SPECIALIZED LIST PACKAGES ***
%
%\usepackage{algorithmic}
% algorithmic.sty was written by Peter Williams and Rogerio Brito.
% This package provides an algorithmic environment fo describing algorithms.
% You can use the algorithmic environment in-text or within a figure
% environment to provide for a floating algorithm. Do NOT use the algorithm
% floating environment provided by algorithm.sty (by the same authors) or
% algorithm2e.sty (by Christophe Fiorio) as the IEEE does not use dedicated
% algorithm float types and packages that provide these will not provide
% correct IEEE style captions. The latest version and documentation of
% algorithmic.sty can be obtained at:
% http://www.ctan.org/pkg/algorithms
% Also of interest may be the (relatively newer and more customizable)
% algorithmicx.sty package by Szasz Janos:
% http://www.ctan.org/pkg/algorithmicx




% *** ALIGNMENT PACKAGES ***
%
%\usepackage{array}
% Frank Mittelbach's and David Carlisle's array.sty patches and improves
% the standard LaTeX2e array and tabular environments to provide better
% appearance and additional user controls. As the default LaTeX2e table
% generation code is lacking to the point of almost being broken with
% respect to the quality of the end results, all users are strongly
% advised to use an enhanced (at the very least that provided by array.sty)
% set of table tools. array.sty is already installed on most systems. The
% latest version and documentation can be obtained at:
% http://www.ctan.org/pkg/array


% IEEEtran contains the IEEEeqnarray family of commands that can be used to
% generate multiline equations as well as matrices, tables, etc., of high
% quality.




% *** SUBFIGURE PACKAGES ***
%\ifCLASSOPTIONcompsoc
%  \usepackage[caption=false,font=normalsize,labelfont=sf,textfont=sf]{subfig}
%\else
%  \usepackage[caption=false,font=footnotesize]{subfig}
%\fi
% subfig.sty, written by Steven Douglas Cochran, is the modern replacement
% for subfigure.sty, the latter of which is no longer maintained and is
% incompatible with some LaTeX packages including fixltx2e. However,
% subfig.sty requires and automatically loads Axel Sommerfeldt's caption.sty
% which will override IEEEtran.cls' handling of captions and this will result
% in non-IEEE style figure/table captions. To prevent this problem, be sure
% and invoke subfig.sty's "caption=false" package option (available since
% subfig.sty version 1.3, 2005/06/28) as this is will preserve IEEEtran.cls
% handling of captions.
% Note that the Computer Society format requires a larger sans serif font
% than the serif footnote size font used in traditional IEEE formatting
% and thus the need to invoke different subfig.sty package options depending
% on whether compsoc mode has been enabled.
%
% The latest version and documentation of subfig.sty can be obtained at:
% http://www.ctan.org/pkg/subfig




% *** FLOAT PACKAGES ***
%
%\usepackage{fixltx2e}
% fixltx2e, the successor to the earlier fix2col.sty, was written by
% Frank Mittelbach and David Carlisle. This package corrects a few problems
% in the LaTeX2e kernel, the most notable of which is that in current
% LaTeX2e releases, the ordering of single and double column floats is not
% guaranteed to be preserved. Thus, an unpatched LaTeX2e can allow a
% single column figure to be placed prior to an earlier double column
% figure.
% Be aware that LaTeX2e kernels dated 2015 and later have fixltx2e.sty's
% corrections already built into the system in which case a warning will
% be issued if an attempt is made to load fixltx2e.sty as it is no longer
% needed.
% The latest version and documentation can be found at:
% http://www.ctan.org/pkg/fixltx2e


%\usepackage{stfloats}
% stfloats.sty was written by Sigitas Tolusis. This package gives LaTeX2e
% the ability to do double column floats at the bottom of the page as well
% as the top. (e.g., "\begin{figure*}[!b]" is not normally possible in
% LaTeX2e). It also provides a command:
%\fnbelowfloat
% to enable the placement of footnotes below bottom floats (the standard
% LaTeX2e kernel puts them above bottom floats). This is an invasive package
% which rewrites many portions of the LaTeX2e float routines. It may not work
% with other packages that modify the LaTeX2e float routines. The latest
% version and documentation can be obtained at:
% http://www.ctan.org/pkg/stfloats
% Do not use the stfloats baselinefloat ability as the IEEE does not allow
% \baselineskip to stretch. Authors submitting work to the IEEE should note
% that the IEEE rarely uses double column equations and that authors should try
% to avoid such use. Do not be tempted to use the cuted.sty or midfloat.sty
% packages (also by Sigitas Tolusis) as the IEEE does not format its papers in
% such ways.
% Do not attempt to use stfloats with fixltx2e as they are incompatible.
% Instead, use Morten Hogholm'a dblfloatfix which combines the features
% of both fixltx2e and stfloats:
%
% \usepackage{dblfloatfix}
% The latest version can be found at:
% http://www.ctan.org/pkg/dblfloatfix




%\ifCLASSOPTIONcaptionsoff
%  \usepackage[nomarkers]{endfloat}
% \let\MYoriglatexcaption\caption
% \renewcommand{\caption}[2][\relax]{\MYoriglatexcaption[#2]{#2}}
%\fi
% endfloat.sty was written by James Darrell McCauley, Jeff Goldberg and
% Axel Sommerfeldt. This package may be useful when used in conjunction with
% IEEEtran.cls'  captionsoff option. Some IEEE journals/societies require that
% submissions have lists of figures/tables at the end of the paper and that
% figures/tables without any captions are placed on a page by themselves at
% the end of the document. If needed, the draftcls IEEEtran class option or
% \CLASSINPUTbaselinestretch interface can be used to increase the line
% spacing as well. Be sure and use the nomarkers option of endfloat to
% prevent endfloat from "marking" where the figures would have been placed
% in the text. The two hack lines of code above are a slight modification of
% that suggested by in the endfloat docs (section 8.4.1) to ensure that
% the full captions always appear in the list of figures/tables - even if
% the user used the short optional argument of \caption[]{}.
% IEEE papers do not typically make use of \caption[]'s optional argument,
% so this should not be an issue. A similar trick can be used to disable
% captions of packages such as subfig.sty that lack options to turn off
% the subcaptions:
% For subfig.sty:
% \let\MYorigsubfloat\subfloat
% \renewcommand{\subfloat}[2][\relax]{\MYorigsubfloat[]{#2}}
% However, the above trick will not work if both optional arguments of
% the \subfloat command are used. Furthermore, there needs to be a
% description of each subfigure *somewhere* and endfloat does not add
% subfigure captions to its list of figures. Thus, the best approach is to
% avoid the use of subfigure captions (many IEEE journals avoid them anyway)
% and instead reference/explain all the subfigures within the main caption.
% The latest version of endfloat.sty and its documentation can obtained at:
% http://www.ctan.org/pkg/endfloat
%
% The IEEEtran \ifCLASSOPTIONcaptionsoff conditional can also be used
% later in the document, say, to conditionally put the References on a
% page by themselves.




% *** PDF, URL AND HYPERLINK PACKAGES ***
%
%\usepackage{url}
% url.sty was written by Donald Arseneau. It provides better support for
% handling and breaking URLs. url.sty is already installed on most LaTeX
% systems. The latest version and documentation can be obtained at:
% http://www.ctan.org/pkg/url
% Basically, \url{my_url_here}.




% *** Do not adjust lengths that control margins, column widths, etc. ***
% *** Do not use packages that alter fonts (such as pslatex).         ***
% There should be no need to do such things with IEEEtran.cls V1.6 and later.
% (Unless specifically asked to do so by the journal or conference you plan
% to submit to, of course. )


% correct bad hyphenation here
\hyphenation{op-tical net-works semi-conduc-tor}
\usepackage{cite}
\usepackage{amsmath,amssymb,amsfonts}
\usepackage{algorithmic}
\usepackage{algorithm}
\usepackage{epsfig}
\usepackage{epstopdf}
\usepackage{graphicx}
\usepackage{textcomp}
\usepackage{xcolor}

\begin{document}
%
% paper title
% Titles are generally capitalized except for words such as a, an, and, as,
% at, but, by, for, in, nor, of, on, or, the, to and up, which are usually
% not capitalized unless they are the first or last word of the title.
% Linebreaks \\ can be used within to get better formatting as desired.
% Do not put math or special symbols in the title.
\title{A Complete Discriminative Tensor Representation Learning for Two-Dimensional Correlation Analysis}
%
%
% author names and IEEE memberships
% note positions of commas and nonbreaking spaces ( ~ ) LaTeX will not break
% a structure at a ~ so this keeps an author's name from being broken across
% two lines.
% use \thanks{} to gain access to the first footnote area
% a separate \thanks must be used for each paragraph as LaTeX2e's \thanks
% was not built to handle multiple paragraphs
%

\author{Lei Gao,~\IEEEmembership{Member,~IEEE,}
        Ling Guan,~\IEEEmembership{Fellow,~IEEE}% <-this % stops a space
\thanks{L. Gao and L. Guan are with the Department of Electrical, Computer and Biomedical Engineering, Ryerson University, Toronto, ON M5B 2K3, Canada (email:iegaolei@gmail.com; lguan@ee.ryerson.ca).}}% <-this % stops a space
%\thanks{J. Doe and J. Doe are with Anonymous University.}% <-this % stops a space
%\thanks{Manuscript received April 19, 2005; revised August 26, 2015.}}

% note the % following the last \IEEEmembership and also \thanks -
% these prevent an unwanted space from occurring between the last author name
% and the end of the author line. i.e., if you had this:
%
% \author{....lastname \thanks{...} \thanks{...} }
%                     ^------------^------------^----Do not want these spaces!
%
% a space would be appended to the last name and could cause every name on that
% line to be shifted left slightly. This is one of those "LaTeX things". For
% instance, "\textbf{A} \textbf{B}" will typeset as "A B" not "AB". To get
% "AB" then you have to do: "\textbf{A}\textbf{B}"
% \thanks is no different in this regard, so shield the last } of each \thanks
% that ends a line with a % and do not let a space in before the next \thanks.
% Spaces after \IEEEmembership other than the last one are OK (and needed) as
% you are supposed to have spaces between the names. For what it is worth,
% this is a minor point as most people would not even notice if the said evil
% space somehow managed to creep in.



% The paper headers
%\markboth{Journal of \LaTeX\ Class Files,~Vol.~14, No.~8, August~2015}%
%{Shell \MakeLowercase{\textit{et al.}}: Bare Demo of IEEEtran.cls for IEEE Journals}
% The only time the second header will appear is for the odd numbered pages
% after the title page when using the twoside option.
%
% *** Note that you probably will NOT want to include the author's ***
% *** name in the headers of peer review papers.                   ***
% You can use \ifCLASSOPTIONpeerreview for conditional compilation here if
% you desire.




% If you want to put a publisher's ID mark on the page you can do it like
% this:
%\IEEEpubid{0000--0000/00\$00.00~\copyright~2015 IEEE}
% Remember, if you use this you must call \IEEEpubidadjcol in the second
% column for its text to clear the IEEEpubid mark.



% use for special paper notices
%\IEEEspecialpapernotice{(Invited Paper)}




% make the title area
\maketitle

% As a general rule, do not put math, special symbols or citations
% in the abstract or keywords.
\begin{abstract}
As an effective tool for two-dimensional data analysis, two-dimensional canonical correlation analysis (2DCCA) is not only capable of preserving the intrinsic structural information of original two-dimensional (2D) data, but also reduces the computational complexity effectively. However, due to the unsupervised nature, 2DCCA is incapable of extracting sufficient discriminatory representations, resulting in an unsatisfying performance. In this letter, we propose a complete discriminative tensor representation learning (CDTRL) method based on linear correlation analysis for analyzing 2D signals (e.g. images). This letter shows that the introduction of the complete discriminatory tensor representation strategy provides an effective vehicle for revealing and extracting the discriminant representations across the 2D data sets, leading to improved results. Experimental results show that the proposed CDTRL outperforms state-of-the-art methods on the evaluated data sets.
\end{abstract}

% Note that keywords are not normally used for peerreview papers.
\begin{IEEEkeywords}
Two-dimensional canonical correlation analysis, two-dimensional linear correlation analysis, discriminative tensor representation learning.
\end{IEEEkeywords}






% For peer review papers, you can put extra information on the cover
% page as needed:
% \ifCLASSOPTIONpeerreview
% \begin{center} \bfseries EDICS Category: 3-BBND \end{center}
% \fi
%
% For peerreview papers, this IEEEtran command inserts a page break and
% creates the second title. It will be ignored for other modes.
\IEEEpeerreviewmaketitle



\section{Introduction}
% The very first letter is a 2 line initial drop letter followed
% by the rest of the first word in caps.
%
% form to use if the first word consists of a single letter:
% \IEEEPARstart{A}{demo} file is ....
%
% form to use if you need the single drop letter followed by
% normal text (unknown if ever used by the IEEE):
% \IEEEPARstart{A}{}demo file is ....
%
% Some journals put the first two words in caps:
% \IEEEPARstart{T}{his demo} file is ....
%% MathType!MTEF!2!1!+-
% feaaguart1ev2aaatCvAUfeBSjuyZL2yd9gzLbvyNv2CaerbuLwBLn
% hiov2DGi1BTfMBaeXatLxBI9gBaerbd9wDYLwzYbItLDharqqtubsr
% 4rNCHbWexLMBbXgBd9gzLbvyNv2CaeHbl7mZLdGeaGqiVu0Je9sqqr
% pepC0xbbL8F4rqqrFfpeea0xe9Lq-Jc9vqaqpepm0xbba9pwe9Q8fs
% 0-yqaqpepae9pg0FirpepeKkFr0xfr-xfr-xb9adbaqaaeGaciGaai
% aabeqaamaabaabauaakabbaaaaaaaacXwyJTgapeqaaiaad2gadaah
% aaWcbeqaaiaaikdaaaaaaa!42F9!

% Here we have the typical use of a "T" for an initial drop letter
% and "HIS" in caps to complete the first word.
\IEEEPARstart{R}{apid} developments in sensory and computing technology have enabled the accessibility of multiple data/information sources representing the same phenomenon from a variety of acquisition techniques and devices. Specifically, multiple data/information sources in images have been playing a vital and central role in two and multidimensional signal processing. Therefore, multiple data/information representation learning is becoming a challenging but increasingly significant research topic in the signal processing and statistics communities [1-2, 21].\\\indent Recently, correlation analysis has drawn more attention in academic and industrial sectors for multiple data/information representation learning [3, 26]. The aim of correlation analysis is to measure and evaluate intrinsic correlation across different data sets. As a typical representation for linear correlation analysis, canonical correlation analysis (CCA) plays important roles and has been applied to signal analysis, visual representation and other tasks [4-5]. In CCA, not only is the correlation taken into consideration, but the canonical characteristic is employed to gain maximal correlation. Consequentially, CCA is widely utilized for cross-modal correlation analysis, such as audiovisual-based emotion recognition [6] and medical imaging analysis [7], etc. Nevertheless, in the field of image and visual computing, 2D data samples are reshaped into one-dimensional vectors before CCA is performed. It is known that the reshaping operation breaks the spatial structural information of 2D data sets and introduces a higher computing complexity [8].\\\indent Therefore, two-dimensional CCA (2DCCA) is presented to address this problem [9]. 2DCCA aims to identify the maximal linear correlation among 2D data sets directly, without going through the reshaping operation. In addition, since the size of covariance matrices for 2D data is smaller than the reshaped vector-based covariance matrices, 2DCCA leads to a lower computational complexity than CCA [9,18]. Assume we have $N$ two dimensional samples with a size of $m \times n$. Then the computational complexity of the traditional CCA is on the order \emph{O$((mn)^{3})$} while 2DCCA only requires a computational complexity of \emph{O$((m)^{3})$} or \emph{O$((n)^{3})$}. Moreover, local two-dimensional canonical correlation analysis (L2DCCA) [17] and two-directional two-dimensional kernel canonical correlation analysis ($(2D)^2KCCA$) [10] are proposed as an extension of 2DCCA. In L2DCCA, the local structural information is introduced to the 2DCCA space, revealing more useful representations between 2D data sets with the computational complexity on the order of \emph{$O((m)^{3}+(N)^{2}m)$} or \emph{$O((n)^{3}+(N)^{2}n)$}. The main purpose of $(2D)^2KCCA$ is to explore the nonlinear correlation between different 2D data sets, in order to achieve better classification performance. However, as far as we know, since most of the existing 2DCCA and related algorithms belong to the unsupervised learning category, they are not able to measure and extract sufficient discriminatory representations across 2D data sets effectively, resulting in an unsatisfying recognition performance. To address the aforementioned issues, a complete discriminant tensor representation learning (CDTRL) is proposed.\\\indent The contributions of this letter are summarized as follows: 1) A discriminant tensor representation learning solution is proposed to explore discriminative representations from 2D data sets. 2) The discriminative representations derived from the range space and the null space of the within-class matrix are utilized jointly to construct a complete discriminant descriptor for 2D correlation analysis. 3) The generality of the proposed CDTRL is validated by two examples. This generic nature guarantees that CDTRL can be used in a broad range of applications.\\\indent The remainder of this letter is organized as follows: A review of related work is presented in Section II. The proposed CDTRL is formulated in Section III. Experimental results and analysis are shown in Section IV. Conclusions are given in Section V.
\section{Related Work}
In this section, we briefly introduce the fundamentals of the 2DCCA and L2DCCA methods, respectively.
\subsection{2DCCA}
Suppose we have two 2D data sets $X$ and $Y$. The samples from $X$ and $Y$ are defined as $ {X_i} \in {R^{m \times n}},{Y_i} \in {R^{p \times q}} (i=1,2,...N)$, where $N$ is the number of samples. The mean matrices of $X$ and $Y$ are calculated as below
\begin{small}
\begin{equation}
{M_X} = 1/N\sum\limits_{i = 1}^N {{X_i}},
{M_Y} = 1/N\sum\limits_{i = 1}^N {{Y_i}},
\end{equation}
\end{small}
and denoted as $\mathop X^{\sim} = X - {M_X} $ and $\mathop Y^{\sim} = Y - {M_Y} $. Then, the purpose of 2DCCA is to find left projected matrices ${L_X}$ \& ${L_Y}$, and right projected matrices ${R_X}$ \& ${R_Y}$, which maximize the correlation between ${L_X}'X^{\sim}R_X$ and ${L_Y}'Y^{\sim}R_Y$ jointly. The optimization of 2DCCA is given in equation (2) [9]
\begin{equation}
\begin{array}{l}
 \arg \max {\mathop{\rm}} ({L_X}^\prime {X^{\sim}}{R_X} \cdot ({L_Y}^\prime {Y^{\sim}}{R_Y})^\prime), \\
 \\
 s.t.{\mathop{\rm var}} ({L_X}^\prime {X^{\sim}}{R_X}) = {\mathop{\rm var}} ({L_Y}^\prime {Y^{\sim}}{R_Y}) = 1, \\
 \end{array}
\end{equation}
where `var' denotes the variance of a given variable. In equation (2), we are capable of finding the solutions to ${L_X}$, ${L_Y}$, ${R_X}$ and ${R_Y}$ by a generalized eigen-value (GEV) algorithm.
\subsection{L2DCCA}
For the L2DCCA method, a manifold algorithm is introduced to 2DCCA to explore the local structural information between the 2D data sets. The weight ${A^X}_{ij}$ between two samples $X_i$ and $X_j$ in $X$ is formulated as
\begin{small}
\begin{equation}
{A^X}_{ij} = \exp (-\frac{{{{\left\| {{X_i} - {X_j}} \right\|}^2_{F}}}}{{{\sigma ^2}}}),
\end{equation}
\end{small}
where ${{{{\left\| {.} \right\|}_{F}}}}$ is the Frobenius norm. The aim of L2DCCA is to find two pairs of projected matrices ${L_{K_X}}$, ${L_{K_Y}}$, ${R_{K_X}}$ and ${R_{K_Y}}$ according to the relation in (4)
\begin{equation}
\begin{array}{l}
 \arg \max {\mathop{\rm}} ({A^X}_{ij}{L_{K_X}}^\prime X^{\sim} {R_{K_X}} \cdot ({A^Y}_{ij}{L_{K_Y}}^\prime Y^{\sim}{R_{K_Y}})^\prime), \\
 \\
 s.t.{\mathop{\rm var}} ({A^X}_{ij}{L_{K_X}}^\prime X^{\sim} {R_{K_X}}) = {\mathop{\rm var}} ({A^Y}_{ij}{L_{K_X}}^\prime {Y^{\sim}} {R_{K_Y}}) = 1. \\
 \end{array}
\end{equation}
Then, solutions to equation (4) are obtained by using the GEV algorithm iteratively.
\section{The Proposed CDTRL Method}
Given two sets of 2D data $X = [{X_1},{X_2},...{X_N}]$ and $Y = [{Y_1},{Y_2},...{Y_N}]$, where $ {X_i} \in {R^{m \times n}},{Y_i} \in {R^{p \times q}} (i=1,2,...N)$. Then, the zero-mean data sets are denoted as $X^{\sim}$ and $Y^{\sim}$, respectively. Assume the size of the two paired project matrices in 2DCCA satisfies the relation: $ L_X \in {R^{m \times {d_1}}}, L_Y \in {R^{p \times {d_1}}}$, $ R_X \in {R^{n \times {d_2} }}$ and $R_Y \in {R^{q \times {d_2} }}$. Then the projections of 2DCCA on the two 2D data sets are expressed as follows
\begin{equation}
 {X_{P}} = {L_X}^\prime {X^{\sim}}{R_X}, {Y_{P}} = {L_Y}^\prime {Y^{\sim}}{R_Y},
\end{equation}
where ${X_{P}} \in {R^{d_1 \times d_2 \times N}}$ and ${Y_{P}} \in {R^{d_1 \times d_2 \times N}}$. Let $n_k$ be the number of samples in the \emph{k}th class and satisfy the following relation
\begin{small}
\begin{equation}
\sum\limits_{k = 1}^c {{n_k} = N},
\end{equation}
\end{small}
where $c$ is the number of classes in $X$ and $Y$.\\\ From equation (5), it is known that there are $N$ paired maps in the ${d_1} \times {d_2}$ plane and their correlation achieves maximum on this plane, which is depicted in Figure 1.\\
\centerline{\includegraphics[height=0.6in,width=1.5in]{fig1.eps}}\\ {Figure. 1 The representtaion of the $N$ paired maps ($X_P$, $Y_P$). }\\

According to Figure. 1, it is observed that $X_P$ and $Y_P$ can be represented by two three-dimensional tensors. Our aim is to find the discriminative representations between the two three-dimensional tensors. However, it is an interesting but challenging research topic to extract the discriminative representations between high dimensional tensors [22-23]. To address this issue, a discriminant tensor strategy is proposed according to the canonical property. According to the canonical property, the correlation satisfies the relation in equation (7)
\begin{equation}
\left\{ \begin{array}{l}
 {\rm{}}({L_{{X}}}^\prime {X^\sim_u}{R_{{X}}} \cdot ({L_{{Y}}}^\prime {Y^\sim_w}{R_{{Y}}})^\prime) = 0, \\
 {\rm{}}({L_{{X}}}^\prime {X^\sim_w}{R_{{X}}} \cdot ({L_{{Y}}}^\prime {Y^\sim_u}{R_{{Y}}})^\prime) = 0, \\
 \end{array} \right.
\end{equation}
where $u,w \in [1,2,...N]$. Therefore, the correlation between different paired maps is 0. Then, the $N$ paired maps ($X_P$, $Y_P$) can be connected as shown in Figure. 2 with the maximum correlation.\\
\centerline{\includegraphics[height=0.8in,width=0.8in]{fig2.eps}}\\ {Figure. 2 The connection of the $N$ paired maps ($X_P$, $Y_P$).}\\\indent

The connected $N$ paired maps are expressed mathematically in equation (8)
\begin{small}
\begin{equation}
F = \left[ \begin{array}{l}
 {X_{P}} \\
 {Y_{P}} \\
 \end{array} \right] = \left[ \begin{array}{l}
 {L_X}^\prime {X^{\sim}}{R_X} \\
 {L_Y}^\prime {Y^{\sim}}{R_Y} \\
 \end{array} \right],
\end{equation}
\end{small}
or in matrix-vector form
\begin{small}
\begin{equation}
\begin{array}{l}
 F = {\left[ {\left( {\begin{array}{*{20}{c}}
   {{L_X}} & {\bf{0}}  \\
   {\bf{0}} & {{L_Y}}  \\
\end{array}} \right)} \right]^{'}}\left[ {\left( {\begin{array}{*{20}{c}}
   {{X^{\sim}}} & {\bf{0}}  \\
   {\bf{0}} & {{Y^{\sim}}}  \\
\end{array}} \right)} \right]\left[ \begin{array}{l}
 {R_X} \\
 {R_Y} \\
 \end{array} \right]. \\
 \end{array}
\end{equation}
\end{small}
Thus, discriminative representation extraction from tensors is accomplished by maximizing the between-class matrix and minimizing the within-class matrix from $N$ connected maps jointly.\\\indent The total mean matrix of $F$ is calculated in equation (10)
\begin{small}
\begin{equation}
{M_F} = \frac{1}{N}\sum\limits_{i = 1}^N {{F_i}},
\end{equation}
\end{small}
and the mean matrix of the $j$th class in $F$ is calculated in (11)
\begin{small}
\begin{equation}
{M_{{F_j}}} = \frac{1}{{{n_j}}}\sum\limits_{s = 1}^{{n_j}} {{F_{js}}},
\end{equation}
\end{small}
where $F_{js}$ represents the \emph{s}th sample in class $j$.\\\indent Afterwards, the within-class and between-class matrices of $N$ samples in ${d_1} \times {d_2}$ space are computed using the following formulas
\begin{small}
\begin{equation}
\begin{array}{l}
 {S_w} = \sum\limits_{j = 1}^c {\sum\limits_{s \in j} {({F_{js}} - {M_F}){{({F_{js}} - {M_F})}^{'}}} },  \\
 {S_b} = \sum\limits_{j = 1}^c {{n_j}({F_j} - {M_{{F_j}}}){{({F_j} - {M_{{F_j}}})}^{'}}}.  \\
 \end{array}
\end{equation}
\end{small}
Then, CDTRL aims to find the optimal left projected matrix $l$ and right projected matrix $r$ as formulated in equation (13)
\begin{small}
\begin{equation}
\begin{array}{l}
 {S_{b,lr}} = \sum\limits_{j = 1}^c {{n_j}{l^{'}}({F_j} - {M_{{F_j}}})r{r^{'}}{{({F_j} - {M_{{F_j}}})}^{'}}l},  \\
 {S_{w,lr}} = \sum\limits_{j = 1}^c {\sum\limits_{s \in j} {{l^{'}}({F_{js}} - {M_F})r{r^{'}}{{({F_{js}} - {M_F})}^{'}}l} }.  \\
 \end{array}
\end{equation}
\end{small}
The proposed CDTRL method is formulated as the following optimization problems to extract the complete discriminative representations corresponding to joint utilization of the range space (14) and the null space (15) of the within-class matrix [19]
\begin{equation}
\begin{array}{l}
 \arg \mathop {\max }\limits_{l,r} \frac{{tr({S_{b,lr}})}}{{tr({S_{w,lr}})}} \\
  = \arg \mathop {\max }\limits_{l,r} \frac{{tr(\sum\limits_{j = 1}^c {{n_j}{l^{'}}({F_j} - {M_{{F_j}}})r{r^{'}}{{({F_j} - {M_{{F_j}}})}^{'}}l} )}}{{tr(\sum\limits_{j = 1}^c {\sum\limits_{s \in j} {{l^{'}}({F_{js}} - {M_F})r{r^{'}}{{({F_{js}} - {M_F})}^{'}}l} } )}} \\
 s.t.\quad{\rm{ }}{S_{w,lr}} \ne {\bf{0}}, \\
 \end{array}
\end{equation}
or
\begin{small}
\begin{equation}
\begin{array}{l}
 \arg \mathop {\max }\limits_{l,r} tr({S_{b,lr}}) \\
  = \arg \mathop {\max }\limits_{l,r} tr(\sum\limits_{j = 1}^c {{n_j}{l^{'}}({F_j} - {M_{{F_j}}})r{r^{'}}{{({F_j} - {M_{{F_j}}})}^{'}}l} )\\
 s.t.\quad{\rm{ }}{S_{w,lr}} = {\bf{0}}, \\
 \end{array}
\end{equation}
\end{small}
where $tr$ denotes the trace operation of a matrix. Since $l$ \& $r$ are two independent matrices and there is no intrinsic relation between them, it is difficult to compute the optimal $l$ \& $r$ simultaneously [20]. To obtain the optimal $l$ and $r$, an iterative strategy is proposed in this letter, which is described as follows. Given an initial value of $r$ (or $l$) and the projected matrix of $l$ (or $r$) is obtained by solving the optimization problem in equation (14) or (15). Afterwards, the projected matrix $r$ (or $l$) is updated with the previous $l$ (or $r$). Thus, $l$ and $r$ are determined by iteratively calculating the ratio between $tr({S_{b,lr}})$ and $tr({S_{w,lr}})$ or $tr({S_{b,lr}})$ until convergence. Then, the computational complexity of CDTRL is on the order of \emph{O$((m)^{3}+(N)^{3})$} or \emph{O$((n)^{3}+(N)^{3})$}. A more detailed description is summarized in the following subsections.
\begin{figure*}[t]
\centering
\includegraphics[height=0.7in,width=5.5in]{fig3.eps}\\ Figure. 3 The diagram of the proposed CDTRL\\
\end{figure*}
\subsection{Update the optimal matrix $l$}
For a given $r$, $S_{w, lr}$ and $S_{b, lr}$ are formulated in (16)
\begin{small}
\begin{equation}
\begin{array}{l}
 {S_{w, lr}} = {l^{'}}{S_{w, lr}}^rl, \\
 {S_{b, lr}} = {l^{'}}{S_{b, lr}}^rl, \\
 \end{array}
\end{equation}
\end{small}
where
\begin{small}
\begin{equation}
\begin{array}{l}
 {S_{w, lr}}^r = \sum\limits_{j = 1}^c {\sum\limits_{s \in j} {({F_{js}} - {M_F})r{r^{'}}{{({F_{js}} - {M_F})}^{'}}} },  \\
 {S_{b, lr}}^r = \sum\limits_{j = 1}^c {{n_j}({F_j} - {M_{{F_j}}})r{r^{'}}{{({F_j} - {M_{{F_j}}})}^{'}}}.  \\
 \end{array}
\end{equation}
\end{small}
Then, using the Lagrange multiplier method, the projected matrix $l$ is obtained by solving the following optimization functions
\begin{small}
\begin{equation}
\begin{array}{l}
{S_{w,lr}}^rl = \lambda {S_{b,lr}}^rl\\
s.t.\quad{\rm{ }}{S_{w,lr}} \ne {\bf{0}}, \\
\end{array}
\end{equation}
\end{small}
or
\begin{small}
\begin{equation}
\begin{array}{l}
\arg \mathop {\max }\limits_l tr({l^{'}}{S_{b,lr}}^rl)\\
s.t.\quad{\rm{ }}{S_{w,lr}} = {\bf{0}}. \\
\end{array}
\end{equation}
\end{small}
\subsection{Update the optimal matrix $r$}
Similarly, with a given $r$, $S_{w, lr}$ and $S_{b, lr}$ are formulated in (20)
\begin{small}
\begin{equation}
\begin{array}{l}
 {S_{w, lr}} = {r^{'}}{S_{w, lr}}^lr, \\
 {S_{b, lr}} = {r^{'}}{S_{b, lr}}^lr, \\
 \end{array}
\end{equation}
\end{small}
where
\begin{small}
\begin{equation}
\begin{array}{l}
 {S_{w, lr}}^l = \sum\limits_{j = 1}^c {\sum\limits_{s \in j}{({F_{js}} - {M_F})l{l^{'}}{{({F_{js}} - {M_F})}^{'}}} },  \\
 {S_{b, lr}}^l = \sum\limits_{j = 1}^c {{n_j}({F_j} - {M_{{F_j}}})l{l^{'}}{{({F_j} - {M_{{F_j}}})}^{'}}}.  \\
 \end{array}
\end{equation}
\end{small}
Again, the projected matrix $r$ is obtained by solving the optimization functions in (22) or (23)
\begin{small}
\begin{equation}
\begin{array}{l}
{S_{w,lr}}^lr = \eta {S_{b,lr}}^lr\\
s.t.\quad{\rm{ }}{S_{w,lr}} \ne {\bf{0}}, \\
\end{array}
\end{equation}
\end{small}
or
\begin{small}
\begin{equation}
\begin{array}{l}
\arg \mathop {\max }\limits_l tr({r^{'}}{S_{b,lr}}^lr)\\
s.t.\quad{\rm{ }}{S_{w,lr}} = {\bf{0}}. \\
\end{array}
\end{equation}
\end{small}
The CDTRL algorithm is summarized in \textbf{Algorithm 1} and the block diagram is depicted in Figure. 3.
\begin{tiny}
\begin{algorithm}[]
\caption{The proposed CDTRL algorithm}
\label{alg:Framwork}
\begin{algorithmic}
\REQUIRE ~~\\
\textbf{*} Given two 2D data sets X and Y with $N$ samples.\\
\ENSURE ~~\\
\STATE \textbf{*} Compute the zero-mean data sets $\mathop X^{\sim}$ and $\mathop Y^{\sim}$.
\STATE \textbf{*} Find projected matrices ${L_X}$, ${L_Y}$, ${R_X}$ and ${R_Y}$ according to equation (2).\\
\STATE \textbf{*} Construct the matrix $F$ based on equation (9).\\
\STATE \textbf{*} Calculate the matrices ${S_{w, lr}}^r$, ${S_{b, lr}}^r$, ${S_{w, lr}}^l$ and ${S_{b, lr}}^l$, respectively.
\STATE \textbf{*} Calculate the $r$ and $l$ according to (18)(22) or (19)(23) until convergence iteratively.
\RETURN the optimal matrices $l$ and $r$.
\end{algorithmic}
\end{algorithm}
\end{tiny}
\section{Experimental Results and Analysis}
To examine the performance of CDTRL, we conduct experiments on AR [11] and FERET [12] face data sets, respectively. The 2D samples in AR database are collected under various conditions, such as different facial emotions, lighting. etc. To verify the generality of the proposed method, 480 samples of 120 subjects are chosen randomly from the AR face database and each of them was normalized to a size of 50 $\times$ 50 pixels. In addition, for each subject, we select four different face images, including one reference sample and the other three samples under different expressions and illumination conditions. %Among them, 12 samples of 3 subjects are given in Figure. 4.\\
%\centerline{\includegraphics[height=0.7in,width=0.8in]{fig1.eps}}\\ {Figure. 4 Facial samples from the AR dataset. (Samples in the first column are references, and the remaining samples are captured under various conditions.)}\\\indent
In the FERET database, 600 samples of 200 subjects with a size of 80 $\times$ 80 pixels are selected. For each person, it contains three samples with different poses (e.g. front, left and right). %Figure. 3 provides some example samples.\\
%\centerline{\includegraphics[height=0.6in,width=0.7in]{fig2.eps}}\\ {Figure. 3 Facial samples from the FERET dataset.}\\\indent
In what follows, we will test the performance of CDTRL under different conditions (e.g. facial expressions and illumination) on the AR database and various facial pose images on the FERET database.

\subsection{Experiments under Different Expressions and Illumination on the AR Database}
In this subsection, we will conduct experiments with the proposed CDTRL and then compare the performances of CCA [5], 2DCCA [9], principal component analysis (PCA) [13], two-dimensional PCA (2DPCA) [14], linear discriminant analysis (LDA) [15], two-dimensional LDA (2DLDA) [16], Local 2DCCA (L2DCCA) [17], discriminative CCA (DCCA) [24] and labeled CCA (LCCA) [25]. Note, for CCA and related algorithms (such as 2DCCA, L2DCCA, DCCA, LCCA and CDTRL), we divide the chosen images into two groups. The first group contains only the reference samples while the remaining images are in the second group. As a result, 360 reference samples (three copies of 120 reference samples to match the samples in the second group) are in the first group $X$ and 360 samples with various conditions are stored in the second group $Y$.\\\indent Since the operations of PCA, LDA, 2DPCA and 2DLDA algorithms do not involve correlation analysis, these methods are either applied directly to the 480 chosen samples (2DPCA and 2DLDA), or to the samples reshaped into one-dimensional vectors (PCA and LDA). For the correlation based methods, 2DCCA, L2DCCA and CDTRL are performed on the 2D data sets X and Y directly while CCA, DCCA and LCCA work on samples reshaped into one-dimensional vectors. To further validate the effectiveness of the proposed method, the leave-one-out cross-validation strategy is utilized and recognition accuracies are tabulated in TABLE I. Viewing the table, it is evident that CDTRL yields performance superior to the others.
\vspace*{-10pt}
\begin{table}[h]
\normalsize
\renewcommand{\arraystretch}{0.75}
\caption{\normalsize{The recognition accuracy with different methods on the AR dataset}}
\setlength{\abovecaptionskip}{0pt}
\setlength{\belowcaptionskip}{10pt}
\centering
\tabcolsep 0.1in
\begin{tabular}{cc}
\hline
\hline
Method & Recognition Accuracy\\
\hline
PCA [13]  &93.33\%\\
2DPCA [14]  &94.17\%\\
LDA [15]  &95.00\%\\
2DLDA [16]  &95.83\%\\
CCA [5]  &95.83\%\\
DCCA [24]  &98.33\%\\
LCCA [25]  &96.67\%\\
2DCCA [9]  &97.50\%\\
L2DCCA [17]  &98.01\%\\
%$(2D)^2$KCCA [10] &98.53\%\\
\textbf{The proposed CDTRL}  &\textbf{100.00\%}\\
\hline
\hline
\end{tabular}
\end{table}
\subsection{Experiments under Different Poses on the FERET Database}
In the FERET database, 600 samples of 200 subjects are chosen. Each subject provides three samples with a size of 20 $\times$ 20 pixels according to three different poses (front, left and right). Then, all 600 samples are utilized to construct the 2D data set $X$ and the wavelet transform [10] is performed twice on each sample in the data set $X$ to generate the corresponding 2D data set $Y$. Moreover, two experimental settings are adopted, front-left and front-right. In the first setting, the front samples are utilized for training while the left samples are for testing. In the second, the front images are still adopted as train samples but the right samples are utilized for testing. Again, since PCA, LDA, 2DPCA and 2DLDA are not able to explore the correlation between the two variable sets, they are applied to the data set $X$ only. On the other hand, 2DCCA, L2DCCA and CDTRL are performed on the 2D data sets $X$ and $Y$ while samples are reshaped into one dimensional vectors for CCA and DCCA. The experimental results are reported in TABLE II, demonstrating better performance of the proposed CDTRL.\\
\vspace*{-10pt}
\begin{table}[h]
\normalsize
\renewcommand{\arraystretch}{0.8}
\caption{\normalsize{The recognition accuracy with different methods on FERET dataset}}
\setlength{\abovecaptionskip}{0pt}
\setlength{\belowcaptionskip}{10pt}
\centering
\tabcolsep 0.1in
\begin{tabular}{ccc}
\hline
\hline
Method & Front-Left & Front-Right \\
\hline
PCA [13]  &77.50\% &75.50\% \\
2DPCA [14]  &78.50\% &76.50\% \\
LDA [15] &67.00\% &65.50\% \\
2DLDA [16] &71.50\% &70.50\% \\
CCA [5] &72.50\% &68.50\% \\
DCCA [24]  &78.50\% &60.50\% \\
2DCCA [9]  &80.50\% &74.50\%\\
L2DCCA [17]  &79.50\% &75.00\%\\
%$(2D)^2$KCCA [10] &97.50\% &96.50\%\\
\textbf{The proposed CDTRL}  &\textbf{83.00\%} &\textbf{78.00\%}\\
\hline
\hline
\end{tabular}
\end{table}

\section{Conclusion}
This letter presents a CDTRL method for linear correlation analysis of 2D data. The main contribution of this letter is to generate the complete discriminative tensor representations across 2D data sets. It is demonstrated that CDTRL is more efficient than 2DCCA and L2DCCA at exploring the linear discriminant correlation between 2D data sets. For the proposed CDTRL method, since 2D data samples are utilized as inputs directly instead of reshaping them into one-dimensional vectors, lower computational complexity is expected. Experimental results show the superiority of the proposed CDTRL method.\\\indent Moreover, one of the worthwhile extensions is to conduct further investigation on the kernelized version of CDTRL based on $(2D)^2KCCA$ [10] to address nonlinear problems in the 2D data representation learning.








%\appendices
%\section{Proof of the First Zonklar Equation}
%Appendix one text goes here.

% you can choose not to have a title for an appendix
% if you want by leaving the argument blank
%\section{}
%Appendix two text goes here.


% use section* for acknowledgment
%\section*{Acknowledgment}


%The authors would like to thank...


% Can use something like this to put references on a page
% by themselves when using endfloat and the captionsoff option.
\ifCLASSOPTIONcaptionsoff
  \newpage
\fi



% trigger a \newpage just before the given reference
% number - used to balance the columns on the last page
% adjust value as needed - may need to be readjusted if
% the document is modified later
%\IEEEtriggeratref{8}
% The "triggered" command can be changed if desired:
%\IEEEtriggercmd{\enlargethispage{-5in}}

% references section

% can use a bibliography generated by BibTeX as a .bbl file
% BibTeX documentation can be easily obtained at:
% http://mirror.ctan.org/biblio/bibtex/contrib/doc/
% The IEEEtran BibTeX style support page is at:
% http://www.michaelshell.org/tex/ieeetran/bibtex/
%\bibliographystyle{IEEEtran}
% argument is your BibTeX string definitions and bibliography database(s)
%\bibliography{IEEEabrv,../bib/paper}
%
% <OR> manually copy in the resultant .bbl file
% set second argument of \begin to the number of references
% (used to reserve space for the reference number labels box)
\begin{thebibliography}{1}

\bibitem{IEEEhowto:kopka}
M. Federici, A. Dutta, P. Forre, N. Kushman, and Z. Akata. ``Learning Robust Representations via Multi-View Information Bottleneck." \emph{2020 International Conference on Learning Representations (Accept)}.
\bibitem{IEEEhowto:kopka}
L. Gao, L. Qi, E. Chen and L. Guan, ``Discriminative multiple canonical correlation analysis for information fusion." \emph{IEEE Trans. on Image Processing}, vol. 27, no. 4, pp. 1951-1965, 2018.
\bibitem{IEEEhowto:kopka}
X. Xing, K. Wang, T. Yan, and Z. Lv. ``Complete canonical correlation analysis with application to multi-view gait recognition." \emph{Pattern Recognition}, vol. 50, pp. 107--117, 2016.
\bibitem{IEEEhowto:kopka}
A. de Cheveigne, G.M. Di Liberto, D. Arzounian, D.D. Wong, J. Hjortkjar, S. Fuglsang, and L.C. Parra. ``Multiway canonical correlation analysis of brain data." \emph{NeuroImage}, vol. 186, pp. 728--740, 2019.
\bibitem{IEEEhowto:kopka}
X. Jing, S. Li, C. Lan, D. Zhang, J. Yang, and Q. Liu. ``Color image canonical correlation analysis for face feature extraction and recognition." \emph{Signal Processing}, vol. 91, no. 8, pp. 2132--2140, 2011.
\bibitem{IEEEhowto:kopka}
L. Gao, L. Qi, and L. Guan. ``Online behavioral analysis with application to emotion state identification." \emph{IEEE Intelligent Systems}, vol. 31, no. 5, pp. 32--39, 2016.
\bibitem{IEEEhowto:kopka}
D. Lin, V.D. Calhoun, and Y. Wang. ``Correspondence between fMRI and SNP data by group sparse canonical correlation analysis." \emph{Medical image analysis}, vol. 18, no. 6, pp. 891--902, 2014.
\bibitem{IEEEhowto:kopka}
N. Sun, Z. Ji, C. Zou, and L. Zhao. ``Two-dimensional canonical correlation analysis and its application in small sample size face recognition." \emph{Neural Computing and Applications}, vol. 19, no. 3, pp. 377--382, 2010.
\bibitem{IEEEhowto:kopka}
S.H. Lee and S. Choi. ``Two-dimensional canonical correlation analysis." \emph{IEEE Signal Process. Lett.}, vol. 14, no. 10,  pp. 735-738, 2007.
\bibitem{IEEEhowto:kopka}
X. Gao, S. Niu, and Q. Sun. ``Two-Directional Two-Dimensional Kernel Canonical Correlation Analysis." \emph{IEEE Signal Processing Letters}, vol. 26, no. 11, pp. 1578--1582, 2019.
\bibitem{IEEEhowto:kopka}
http://www2.ece.ohio-state.edu/~aleix/ARdatabase.html.
\bibitem{IEEEhowto:kopka}
P.J. Phillips, H. Moon, S.A. Rizvi, and P.J. Rauss,``The FERET evaluation methodology for face-recognition algorithms." \emph{IEEE Trans. Pattern Anal. Mach. Intell.}, vol. 22, no. 10, pp. 1090--1104, 2000.
\bibitem{IEEEhowto:kopka}
F. Kherif, and A. Latypova. ``Principal component analysis." \emph{Machine Learning}, pp. 209--225, 2020.
\bibitem{IEEEhowto:kopka}
J. Yang, D. Zhang, A. F. Frangi and Jing-yu Yang. ``Two-dimensional PCA: a new approach to appearance-based face representation and recognition." \emph{IEEE transactions on pattern analysis and machine intelligence}, vol. 26, no. 1, pp. 131-137, 2004.
\bibitem{IEEEhowto:kopka}
P. Deng, H. Wang, T. Li, S. Horng, and X. Zhu. ``Linear discriminant analysis guided by unsupervised ensemble learning." \emph{Information Sciences}, vol. 480, pp. 211--221, 2019.
\bibitem{IEEEhowto:kopka}
 J. Ye, R. Janardan and Q. Li. ``Two-dimensional linear discriminant analysis." \emph{In Advances in neural information processing systems}, pp. 1569-1576, 2005.
\bibitem{IEEEhowto:kopka}
H.X. Wang. ``Local Two-dimensional canonical correlation analysis." \emph{IEEE Signal Process. Lett.}, vol. 17, no. 11, pp. 921--924, 2010.
\bibitem{IEEEhowto:kopka}
L. Gao, and L. Guan. ``A Discriminant Two-Dimensional Canonical Correlation Analysis." \emph{2019 IEEE Canadian Conference of Electrical and Computer Engineering (CCECE)}, pp. 1--4, 2019.
\bibitem{IEEEhowto:kopka}
L. Gao, L. Qi, E. Chen, and L. Guan. ``A fisher discriminant framework based on Kernel Entropy Component Analysis for feature extraction and emotion recognition." \emph{2014 IEEE International Conference on Multimedia and Expo (ICME)}, pp. 1--6, 2014.
\bibitem{IEEEhowto:kopka}
D. Tao, Y. Guo, Y. Li, and X. Gao. ``Tensor rank preserving discriminant analysis for facial recognition." \emph{IEEE transactions on image processing}, vol. 27, no. 1, pp. 325-334, 2018.
\bibitem{IEEEhowto:kopka}
X. Yang, W. Liu, and W. Liu. ``Tensor Canonical Correlation Analysis Networks for Multi-view Remote Sensing Scene Recognition." \emph{IEEE Transactions on Knowledge and Data Engineering (Early Access)}, 2020.
\bibitem{IEEEhowto:kopka}
S. Yang, M. Wang, Z. Feng, Z. Liu, and R. Li. ``Deep sparse tensor filtering network for synthetic aperture radar images classification." \emph{IEEE transactions on neural networks and learning systems}, vol. 29, no. 8, pp. 3919--3924, 2018.
\bibitem{IEEEhowto:kopka}
M. Wang, K. Zhang, X. Pan, and S. Yang. ``Sparse tensor neighbor embedding based pan-sharpening via N-way block pursuit." \emph{Knowledge-Based Systems}, vol. 149, pp. 18--33, 2018.
\bibitem{IEEEhowto:kopka}
L. Gao, L. Qi, E. Chen, and L. Guan. ``Discriminative multiple canonical correlation analysis for information fusion." \emph{IEEE Transactions on Image Processing}, vol. 27, no. 4, pp. 1951--1965, 2018.
\bibitem{IEEEhowto:kopka}
L. Gao, R. Zhang, L. Qi, E. Chen, and L. Guan. ``The labeled multiple canonical correlation analysis for information fusion." \emph{IEEE Transactions on Multimedia}, vol. 21, no. 2, pp. 375--387, 2019.
\bibitem{IEEEhowto:kopka}
X. Yang, W. Liu, W. Liu, and D. Tao. ``A survey on canonical correlation analysis." \emph{IEEE Transactions on Knowledge and Data Engineering(Early Access)}, 2019.

\end{thebibliography}

% biography section
%
% If you have an EPS/PDF photo (graphicx package needed) extra braces are
% needed around the contents of the optional argument to biography to prevent
% the LaTeX parser from getting confused when it sees the complicated
% \includegraphics command within an optional argument. (You could create
% your own custom macro containing the \includegraphics command to make things
% simpler here.)
%\begin{IEEEbiography}[{\includegraphics[width=1in,height=1.25in,clip,keepaspectratio]{mshell}}]{Michael Shell}
% or if you just want to reserve a space for a photo:

%\begin{IEEEbiography}{Michael Shell}
%Biography text here.
%\end{IEEEbiography}

% if you will not have a photo at all:
%\begin{IEEEbiographynophoto}{John Doe}
%Biography text here.
%\end{IEEEbiographynophoto}

% insert where needed to balance the two columns on the last page with
% biographies
%\newpage

%\begin{IEEEbiographynophoto}{Jane Doe}
%Biography text here.
%\end{IEEEbiographynophoto}

% You can push biographies down or up by placing
% a \vfill before or after them. The appropriate
% use of \vfill depends on what kind of text is
% on the last page and whether or not the columns
% are being equalized.

%\vfill

% Can be used to pull up biographies so that the bottom of the last one
% is flush with the other column.
%\enlargethispage{-5in}



% that's all folks
\end{document}


