\section{Conclusion}
\label{sec:conclusion}

We investigated the use of ultrasound tongue imaging for the detection of velar fronting and gliding of /r/ in Scottish English child speakers.
For this task, results indicate that experienced speech and language therapists have near perfect agreement when annotating the correctness of velar speech sounds, but agreement on the correctness of rhotic speech sounds is low.

For automatic error detection, out-of-domain adult data improves results on typically developing speech, but it is less useful when evaluating on disordered speech.
Results indicate that velar fronting error detection benefits more from ultrasound than audio, but we observe the best performance when using both modalities.
In terms of gliding error detection, results are harder to interpret due to low inter-annotator agreement.

Future research should explore techniques to account for speaker, session, and equipment variability with ultrasound equipment, as well as annotation preferences by speech and language therapists.
Nonetheless, the overall performance of the classifier is promising, particularly for velar fronting error detection, with good agreement with experienced speech and language therapists.
This evidence suggests there is potential for systems to be integrated into ultrasound intervention software for automatically quantifying progress during speech therapy.

\section{License and Distribution}

The pronunciation scores obtained in Section \ref{sec:expert_detection} are publicly available in the Ultrasuite Repository\footnote{\url{https://www.ultrax-speech.org/ultrasuite}} and are distributed under Attribution-NonCommercial 4.0 Generic (CC BY-NC 4.0).
A demo of the data preparation and scoring processes with the best performing model in Table \ref{tab:model-results-uxtd} is released as part of the UltraSuite code repository\footnote{\url{https://github.com/UltraSuite}} under Apache License v.2.

\section*{Acknowledgments}
We are grateful to the Speech and Language Therapists who kindly agreed to participate in our data collection.
This work was supported by the Carnegie Trust for the Universities of Scotland (Research Incentive Grant number 008585)
and the EPSRC Healthcare Partnerships grant number EP/P02338X/1 (Ultrax2020 –- \ultraxurl).



