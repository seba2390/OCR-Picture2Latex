\documentclass[final,5p,times,twocolumn,11pt]{elsarticle}

\usepackage{url}
\urlstyle{same}
%% \url is a fragile command,
%% but using a defined url created using \urldef is robust
%% and can be used in a maketitle or a caption
\urldef{\ultraxurl}\url{http://www.ultrax-speech.org}

\usepackage{graphicx}
\graphicspath{{figures/}}

\usepackage{csquotes}
\usepackage{xcolor}
\usepackage{textcomp}
\usepackage{gensymb}
\usepackage{amsmath}
\usepackage{amssymb}
\usepackage{booktabs}
\usepackage{multicol}

\usepackage{hyperref}
\hypersetup{
    colorlinks = true,
    linkbordercolor = {white},
    linkcolor = {blue}
}


\journal{Speech Communication}
\bibliographystyle{apa-bibstyle}\biboptions{authoryear}


\begin{document}
\begin{frontmatter}


\title{Exploiting ultrasound tongue imaging for the\\automatic detection of speech articulation errors}

\author[a]{Manuel Sam Ribeiro}\corref{cor1}
\ead{sam.ribeiro@ed.ac.uk}

\author[b]{Joanne Cleland}
\ead{joanne.cleland@strath.ac.uk}

\author[a]{Aciel Eshky}
\ead{aeshky@ed.ac.uk}

\author[a]{Korin Richmond}
\ead{korin.richmond@ed.ac.uk}

\author[a]{Steve Renals}
\ead{s.renals@ed.ac.uk}

\address[a]{The Centre for Speech Technology Research, University of Edinburgh, UK}
\address[b]{Psychological Sciences and Health, University of Strathclyde, UK}

\cortext[cor1]{Corresponding author}

\begin{abstract}
Speech sound disorders are a common communication impairment in childhood.
Because speech disorders can negatively affect the lives and the development of children, clinical intervention is often recommended.
To help with diagnosis and treatment, clinicians use instrumented methods such as spectrograms or ultrasound tongue imaging to analyse speech articulations.
Analysis with these methods can be laborious for clinicians, therefore there is growing interest in its automation. 
In this paper, we investigate the contribution of ultrasound tongue imaging for the automatic detection of speech articulation errors.
Our systems are trained on typically developing child speech and augmented with a database of adult speech using audio and ultrasound.
Evaluation on typically developing speech indicates that pre-training on adult speech and jointly using ultrasound and audio gives 
the best results with an accuracy of 86.9\%.
To evaluate on disordered speech, we collect pronunciation scores from experienced speech and language therapists, focusing on
cases of velar fronting and gliding of /r/.
The scores show good inter-annotator agreement for velar fronting, but not for gliding errors.
For automatic velar fronting error detection, the best results are obtained when jointly using ultrasound and audio.
The best system correctly detects 86.6\% of the errors identified by experienced clinicians.
Out of all the segments identified as errors by the best system, 73.2\% match errors identified by clinicians.
Results on automatic gliding detection are harder to interpret due to poor inter-annotator agreement, but appear promising.
Overall findings suggest that automatic detection of speech articulation errors has potential to be integrated into ultrasound intervention software for automatically quantifying progress during speech therapy.
\end{abstract}

\begin{keyword}
Speech sound disorders \sep Speech error detection \sep Ultrasound tongue imaging \sep Child speech
\end{keyword}

\end{frontmatter}


% \leavevmode
% \\
% \\
% \\
% \\
% \\
\section{Introduction}
\label{introduction}

AutoML is the process by which machine learning models are built automatically for a new dataset. Given a dataset, AutoML systems perform a search over valid data transformations and learners, along with hyper-parameter optimization for each learner~\cite{VolcanoML}. Choosing the transformations and learners over which to search is our focus.
A significant number of systems mine from prior runs of pipelines over a set of datasets to choose transformers and learners that are effective with different types of datasets (e.g. \cite{NEURIPS2018_b59a51a3}, \cite{10.14778/3415478.3415542}, \cite{autosklearn}). Thus, they build a database by actually running different pipelines with a diverse set of datasets to estimate the accuracy of potential pipelines. Hence, they can be used to effectively reduce the search space. A new dataset, based on a set of features (meta-features) is then matched to this database to find the most plausible candidates for both learner selection and hyper-parameter tuning. This process of choosing starting points in the search space is called meta-learning for the cold start problem.  

Other meta-learning approaches include mining existing data science code and their associated datasets to learn from human expertise. The AL~\cite{al} system mined existing Kaggle notebooks using dynamic analysis, i.e., actually running the scripts, and showed that such a system has promise.  However, this meta-learning approach does not scale because it is onerous to execute a large number of pipeline scripts on datasets, preprocessing datasets is never trivial, and older scripts cease to run at all as software evolves. It is not surprising that AL therefore performed dynamic analysis on just nine datasets.

Our system, {\sysname}, provides a scalable meta-learning approach to leverage human expertise, using static analysis to mine pipelines from large repositories of scripts. Static analysis has the advantage of scaling to thousands or millions of scripts \cite{graph4code} easily, but lacks the performance data gathered by dynamic analysis. The {\sysname} meta-learning approach guides the learning process by a scalable dataset similarity search, based on dataset embeddings, to find the most similar datasets and the semantics of ML pipelines applied on them.  Many existing systems, such as Auto-Sklearn \cite{autosklearn} and AL \cite{al}, compute a set of meta-features for each dataset. We developed a deep neural network model to generate embeddings at the granularity of a dataset, e.g., a table or CSV file, to capture similarity at the level of an entire dataset rather than relying on a set of meta-features.
 
Because we use static analysis to capture the semantics of the meta-learning process, we have no mechanism to choose the \textbf{best} pipeline from many seen pipelines, unlike the dynamic execution case where one can rely on runtime to choose the best performing pipeline.  Observing that pipelines are basically workflow graphs, we use graph generator neural models to succinctly capture the statically-observed pipelines for a single dataset. In {\sysname}, we formulate learner selection as a graph generation problem to predict optimized pipelines based on pipelines seen in actual notebooks.

%. This formulation enables {\sysname} for effective pruning of the AutoML search space to predict optimized pipelines based on pipelines seen in actual notebooks.}
%We note that increasingly, state-of-the-art performance in AutoML systems is being generated by more complex pipelines such as Directed Acyclic Graphs (DAGs) \cite{piper} rather than the linear pipelines used in earlier systems.  
 
{\sysname} does learner and transformation selection, and hence is a component of an AutoML systems. To evaluate this component, we integrated it into two existing AutoML systems, FLAML \cite{flaml} and Auto-Sklearn \cite{autosklearn}.  
% We evaluate each system with and without {\sysname}.  
We chose FLAML because it does not yet have any meta-learning component for the cold start problem and instead allows user selection of learners and transformers. The authors of FLAML explicitly pointed to the fact that FLAML might benefit from a meta-learning component and pointed to it as a possibility for future work. For FLAML, if mining historical pipelines provides an advantage, we should improve its performance. We also picked Auto-Sklearn as it does have a learner selection component based on meta-features, as described earlier~\cite{autosklearn2}. For Auto-Sklearn, we should at least match performance if our static mining of pipelines can match their extensive database. For context, we also compared {\sysname} with the recent VolcanoML~\cite{VolcanoML}, which provides an efficient decomposition and execution strategy for the AutoML search space. In contrast, {\sysname} prunes the search space using our meta-learning model to perform hyperparameter optimization only for the most promising candidates. 

The contributions of this paper are the following:
\begin{itemize}
    \item Section ~\ref{sec:mining} defines a scalable meta-learning approach based on representation learning of mined ML pipeline semantics and datasets for over 100 datasets and ~11K Python scripts.  
    \newline
    \item Sections~\ref{sec:kgpipGen} formulates AutoML pipeline generation as a graph generation problem. {\sysname} predicts efficiently an optimized ML pipeline for an unseen dataset based on our meta-learning model.  To the best of our knowledge, {\sysname} is the first approach to formulate  AutoML pipeline generation in such a way.
    \newline
    \item Section~\ref{sec:eval} presents a comprehensive evaluation using a large collection of 121 datasets from major AutoML benchmarks and Kaggle. Our experimental results show that {\sysname} outperforms all existing AutoML systems and achieves state-of-the-art results on the majority of these datasets. {\sysname} significantly improves the performance of both FLAML and Auto-Sklearn in classification and regression tasks. We also outperformed AL in 75 out of 77 datasets and VolcanoML in 75  out of 121 datasets, including 44 datasets used only by VolcanoML~\cite{VolcanoML}.  On average, {\sysname} achieves scores that are statistically better than the means of all other systems. 
\end{itemize}


%This approach does not need to apply cleaning or transformation methods to handle different variances among datasets. Moreover, we do not need to deal with complex analysis, such as dynamic code analysis. Thus, our approach proved to be scalable, as discussed in Sections~\ref{sec:mining}.
\section{Background and Motivation}

\subsection{IBM Streams}

IBM Streams is a general-purpose, distributed stream processing system. It
allows users to develop, deploy and manage long-running streaming applications
which require high-throughput and low-latency online processing.

The IBM Streams platform grew out of the research work on the Stream Processing
Core~\cite{spc-2006}.  While the platform has changed significantly since then,
that work established the general architecture that Streams still follows today:
job, resource and graph topology management in centralized services; processing
elements (PEs) which contain user code, distributed across all hosts,
communicating over typed input and output ports; brokers publish-subscribe
communication between jobs; and host controllers on each host which
launch PEs on behalf of the platform.

The modern Streams platform approaches general-purpose cluster management, as
shown in Figure~\ref{fig:streams_v4_v6}. The responsibilities of the platform
services include all job and PE life cycle management; domain name resolution
between the PEs; all metrics collection and reporting; host and resource
management; authentication and authorization; and all log collection. The
platform relies on ZooKeeper~\cite{zookeeper} for consistent, durable metadata
storage which it uses for fault tolerance.

Developers write Streams applications in SPL~\cite{spl-2017} which is a
programming language that presents streams, operators and tuples as
abstractions. Operators continuously consume and produce tuples over streams.
SPL allows programmers to write custom logic in their operators, and to invoke
operators from existing toolkits. Compiled SPL applications become archives that
contain: shared libraries for the operators; graph topology metadata which tells
both the platform and the SPL runtime how to connect those operators; and
external dependencies. At runtime, PEs contain one or more operators. Operators
inside of the same PE communicate through function calls or queues. Operators
that run in different PEs communicate over TCP connections that the PEs
establish at startup. PEs learn what operators they contain, and how to connect
to operators in other PEs, at startup from the graph topology metadata provided
by the platform.

We use ``legacy Streams'' to refer to the IBM Streams version 4 family. The
version 5 family is for Kubernetes, but is not cloud native. It uses the
lift-and-shift approach and creates a platform-within-a-platform: it deploys a
containerized version of the legacy Streams platform within Kubernetes.

\subsection{Kubernetes}

Borg~\cite{borg-2015} is a cluster management platform used internally at Google
to schedule, maintain and monitor the applications their internal infrastructure
and external applications depend on. Kubernetes~\cite{kube} is the open-source
successor to Borg that is an industry standard cloud orchestration platform.

From a user's perspective, Kubernetes abstracts running a distributed
application on a cluster of machines. Users package their applications into
containers and deploy those containers to Kubernetes, which runs those
containers in \emph{pods}. Kubernetes handles all life cycle management of pods,
including scheduling, restarting and migration in case of failures.

Internally, Kubernetes tracks all entities as \emph{objects}~\cite{kubeobjects}.
All objects have a name and a specification that describes its desired state.
Kubernetes stores objects in etcd~\cite{etcd}, making them persistent,
highly-available and reliably accessible across the cluster. Objects are exposed
to users through \emph{resources}. All resources can have
\emph{controllers}~\cite{kubecontrollers}, which react to changes in resources.
For example, when a user changes the number of replicas in a
\code{ReplicaSet}, it is the \code{ReplicaSet} controller which makes sure the
desired number of pods are running. Users can extend Kubernetes through
\emph{custom resource definitions} (CRDs)~\cite{kubecrd}. CRDs can contain
arbitrary content, and controllers for a CRD can take any kind of action.

Architecturally, a Kubernetes cluster consists of nodes. Each node runs a
\emph{kubelet} which receives pod creation requests and makes sure that the
requisite containers are running on that node. Nodes also run a
\emph{kube-proxy} which maintains the network rules for that node on behalf of
the pods. The \emph{kube-api-server} is the central point of contact: it
receives API requests, stores objects in etcd, asks the scheduler to schedule
pods, and talks to the kubelets and kube-proxies on each node. Finally,
\emph{namespaces} logically partition the cluster. Objects which should not know
about each other live in separate namespaces, which allows them to share the
same physical infrastructure without interference.

\subsection{Motivation}
\label{sec:motivation}

Systems like Kubernetes are commonly called ``container orchestration''
platforms. We find that characterization reductive to the point of being
misleading; no one would describe operating systems as ``binary executable
orchestration.'' We adopt the idea from Verma et al.~\cite{borg-2015} that
systems like Kubernetes are ``the kernel of a distributed system.'' Through CRDs
and their controllers, Kubernetes provides state-as-a-service in a distributed
system. Architectures like the one we propose are the result of taking that view 
seriously.

The Streams legacy platform has obvious parallels to the Kubernetes
architecture, and that is not a coincidence: they solve similar problems.
Both are designed to abstract running arbitrary user-code across a distributed
system.  We suspect that Streams is not unique, and that there are many
non-trivial platforms which have to provide similar levels of cluster
management.  The benefits to being cloud native and offloading the platform
to an existing cloud management system are: 
\begin{itemize}
    \item Significantly less platform code.
    \item Better scheduling and resource management, as all services on the cluster are 
        scheduled by one platform.
    \item Easier service integration.
    \item Standardized management, logging and metrics.
\end{itemize}
The rest of this paper presents the design of replacing the legacy Streams 
platform with Kubernetes itself.


\section{Data}
\label{sec:data}

We use data from the \textbf{Ultrasuite repository}\footnote{\label{fn:ultrasuite}\url{https://www.ultrax-speech.org/ultrasuite}} \citep{eshky2018ultrasuite}, consisting of synchronised ultrasound and audio data from child speech therapy sessions.
Ultrasuite currently contains three datasets of child speech.
Ultrax Typically Developing (UXTD) includes recordings of 58 typically developing children.
The remaining datasets include recordings from children with speech sound disorders collected over the course of assessment and therapy sessions: Ultrax Speech Sound Disorders  (UXSSD,  8  children)  and  Ultraphonix  (UPX,  20  children). 
Assessment sessions denote recordings at various stages of therapy: baseline (before therapy),  mid-therapy,  post-therapy  (immediately after therapy), and maintenance (several months after therapy).
For the child speech datasets, ultrasound was recorded with an Ultrasonix SonixRP machine using Articulate Assistant Advanced (AAA, \cite{articulate2010articulate}) software at $\sim$120fps with a 135\degree ~field of view.
A single B-Mode ultrasound frame has 412 echo returns for each of 63 scan lines, giving a $63\times412$ \enquote{raw} ultrasound frame capturing a mid-sagittal view of the tongue.
Samples from this data are illustrated in Figure \ref{fig:td_samples}.

To complement the Ultrasuite repository, we use the \textbf{Tongue and Lips corpus}\footnote{Available via the Ultrasuite Repository, see footnote \ref{fn:ultrasuite}.}(TaL, \cite{ribeiro2021tal}).
TaL is a corpus of synchronised ultrasound, audio, and lip videos from 82 adult native speakers of English.
Ultrasound in the TaL corpus was recorded using Articulate Instruments' Micro system \citep{articulate2010articulate} at $\sim$80fps with a 92\degree ~field of view.
TaL used a different transducer than the one in the Ultrasuite data collection.
Because of this, an ultrasound frame of the TaL corpus contains  842  echo  returns  for  each  of  64  scan  lines ($64\times842$ \enquote{raw} ultrasound frame).

\section{Expert speech error detection}
\label{sec:expert_detection}

In this section, our \textbf{goal} is the collection of pronunciation scores for speech segments produced by children with speech sound disorders.
This data is to be used in the evaluation of the automatic error detection systems described in Section \ref{sec:automatic_detection}.
We recruited Speech and Language Therapists with experience using ultrasound visual biofeedback 
and who routinely work with Scottish English-speaking children.
The collection of these scores aimed to simulate the process therapists undergo after collecting data from speech therapy sessions.
We are interested in the processes of velar fronting and gliding of /r/, therefore we focused on productions of /k, g/ (\textit{velars}) and pre-vocalic /r/ (\textit{rhotic}).
We expected SLTs to be able to identify correct and incorrect productions of velars and rhotics with good reliability.

\subsection{Data preparation}

We used data from the eight children available in Ultrasuite's UXSSD dataset.
Children in this dataset were treated for velar fronting, therefore we expected to observe an increasing number of correct velar productions throughout assessment sessions.
Children's response to intervention is reported in \citet{cleland2015using}.
Because intervention for these children did not focus on correcting rhotic productions, this led to an imbalanced set of correct and incorrect rhotic samples.
We pre-selected words containing the target velar and rhotic phones and occurring within prompts of type \enquote{A} (single words) in assessment sessions (baseline, mid-therapy, post-therapy, and maintenance).
We discarded words that contained more than one instance of a target phone (e.g.\ \enquote{cake}) or corrupted word instances (e.g.\ overlapping or unintelligible speech, other background noise, etc).
From the pre-selected word list, we randomly sampled 96 word instances per child.
Samples were balanced across assessment sessions and across velars and rhotics.
For each child, an assessment session contained 24 word instances (12 velars and 12 rhotics).
Where possible, samples were also balanced for the position of the target phone in the word (initial, medial, or final).
The final set of samples consisted of a total of 768 word instances with a vocabulary of 148 words, which
we denote as the \textbf{main set}.

We generated a set of additional samples from one of the speakers available in Ultrasuite's UPX dataset.
We selected speaker 04M, treated for velar fronting and reported to have good improvement after intervention \citep{cleland2019enabling}.
As the speakers in the main set, this speaker was also not treated for gliding.
The sampling process was repeated and a total of 24 word instances were selected (12 velars and 12 rhotics), balanced across assessment sessions.
We denote this set of samples the \textbf{control set}.


\begin{figure}
\includegraphics[width=\columnwidth]{figures_eps/annotation-sample-02M-Post-012A.eps}
\caption{\label{fig:annotation-video-sample} Video frame for the word \enquote{tiger} produced by speaker 02M in a post-therapy assessment session. The annotator is requested to score the target phone /g/ in a word medial position. The vertical red line in the spectrogram indicates the current position of the video.}
\end{figure}


\subsection{Method}
\label{subsec_expert_method}

To annotate the word instances, we recruited 8 annotators.
All annotators were SLTs with more than 4 years of experience and who routinely work with children speaking Scottish English.
Additionally, the annotators had at least 3 years of experience with ultrasound visual biofeedback, with two annotators having more than 10 years of experience.
Each annotator was assigned 96 words from the main set and the 24 words from the control set.
Words taken from the main set were selected such that they were produced by a single child.
Therefore, each SLT annotated data from two children (one main and one control).
For intra-annotator agreement, 20\% of the words (24 samples) were repeated in the annotation list.
Of these, 12 were taken from the control set and 12 from the main set.
Each SLT annotated a total of 144 word instances.

Results were collected via a web interface displaying a video of each word sample separately.
The video contained the spectrogram, ultrasound images of the tongue, and the audio for each word.
Figure \ref{fig:annotation-video-sample} illustrates one video frame extracted from one of the samples.
Annotators were allowed to play videos at normal, half, or quarter speed up to a maximum of 6 total playbacks. 

\begin{figure}
\includegraphics[width=\columnwidth]{figures_eps/annotation_joint_bar.eps}
\caption{\label{fig:annotation-bar-chart} Normalised distribution of primary scores for Velars and Rhotics, ordered chronologically (bottom up) by assessment session: baseline (BL), mid-therapy (Mid), post-therapy (Post), and maintenance (Maint).}
\end{figure}


Annotators were instructed to rate the target phone on a 5-point Likert scale,
where 1 indicates \emph{wrong pronunciation} and 5 indicates \emph{perfect pronunciation}.
The first question requested a score for the target phone (e.g. \enquote{Please rate the velar /k/ in the sample}).
This score is denoted the \textbf{primary score}.
If the annotator scored 3 or lower in the primary score, we requested a \textbf{secondary score}.
The secondary score asked the annotator to rate the target phone with respect to an expected substitution.
(e.g. \enquote{Please rate the target phone for alveolar substitution} or \enquote{Please rate the target phone for gliding substitution}).
An optional field allowed annotators to provide a short comment for each sample.

Given the primary score $s_p$ and the secondary score $s_s$, we determine a \textbf{combined score} $s_c$ defined as $s_c = \log(s_p)-\log(s_s)$.
Because we did not request a secondary score when perfect pronunciation was rated for the target phone, we assumed a value of 1 for $s_s$ when computing the combined score.
The score $s_c$ is positive if there is a preference for the primary class (e.g. velars or rhotics) and negative if there is a preference for the secondary class (e.g. alveolars or glide).
If the annotator gave the same primary and secondary scores, then this uncertainty is represented in $s_c$ with 0.
This preference can be further simplified to produce a \textbf{binary score} $s_b$.
For each sample, positive values of the combined scores are treated as correct productions of the primary class and negative or zero values as incorrect productions.


\subsection{Results}
\label{subsec:annotation_results}

Figure \ref{fig:annotation-bar-chart} shows the normalised distribution of the primary score for all annotated samples, ordered chronologically by session.
The number of incorrect velars across assessment sessions decreases over time, while the distribution for rhotics is more or less stable.
This is expected since intervention for these children focused on velar fronting and production of /r/ was not addressed.

Figure \ref{fig:annotation-primary-secondary-matrix} shows the frequency of primary and secondary scores for velars and rhotics in the main set. 
We remove duplicate samples used for intra-annotator agreement, keeping the score of the first sample to be rated.
For this work, we are primarily interested in the correct production of velars and rhotics and clear cases of substitutions (fronting and gliding).
Cases of correct pronunciations for the expected class are identified by a high primary score (4 or 5).
Cases of velar fronting or gliding are identified by a low primary score (1 or 2) and a high secondary score (4 or 5).
We observe from Figure \ref{fig:annotation-primary-secondary-matrix} that 342 out of the 384 velar samples (89.06\%) fall under one of these two cases.
Of these samples, 248 are correct velars and 94 are alveolar substitutions.
Rhotics include a smaller number of correct productions or gliding (275 out of 384 samples, 71.61\%).
Of these, 122 are marked as a correct production of /r/, while 153 denote cases of gliding.

\begin{figure}[t]
\includegraphics[width=\columnwidth]{figures_eps/annotation_primary_secondary_joint_nodups.eps}
\caption{\label{fig:annotation-primary-secondary-matrix} Frequency of primary and secondary scores given by annotators for Velars and Rhotics in the main set with duplicate entries removed, where 1 indicates wrong pronunciation and 5 indicates perfect pronunciation.}
\end{figure}

\begin{figure*}
\centering
\includegraphics[width=0.95\textwidth]{figures_eps/annotation_binary_cohen.eps}
\caption{\label{fig:annotation-cohen-kappa} Pairwise Cohen's $\kappa$ for the binary score, excluding cases not rated as correct productions or clear substitutions. Diagonal values show $\kappa$ for intra-annotator agreement, while off-diagonal shows pairwise inter-annotator agreement. Each cell is colour-coded according to the levels of agreement proposed by \citet[pp 164-165]{landis1977measurement}.}
\end{figure*}

Some annotators used the optional comment field to elaborate on their score, particularly for incorrect cases that were not instances of velar fronting or gliding.
For velars, some of the cases were reported to be uvular, palatal, or postalveolar realisations, or omitted phones.
For rhotics, most of the non-typical scores indicated deletion of /r/ with some cases reporting a distortion towards a labiodental approximant.

\subsection{Annotator agreement}
\label{subsec:annotator-agreement}

We compute inter-annotator agreement using the control set of samples, rated by all annotators.
Duplicates are removed by choosing the first rating and discarding the second.
Intra-annotator agreement is computed on the 20\% duplicate samples, half from the main set and half from the control set.

To measure global agreement, we use \emph{Krippendorf's $\alpha$} \citep{krippendorff2004content}, which computes annotator 
agreement for multiple annotators and supports several levels of measurements.
We compute $\alpha$ using a difference function for ordinal data for the primary score, a function for interval data for the combined score, and a function for nominal data for the binary score \citep{krippendorff2011computing}.
According to \citet[pp 241-243]{krippendorff2004content}, $\alpha > 0.8$ indicates reliable data, while $0.667 \leqslant \alpha \leqslant 0.8$ indicates moderately reliable data. When $\alpha < 0.667$, the data should be considered unreliable.
Table \ref{tab:annotation-krippendorf} shows $\alpha$ values for primary, combined, and binary scores.
For the binary score, we exclude samples not rated as correct productions or clear substitutions.
Overall annotator agreement appears to be very good for velar samples and poor for rhotic samples.

\begin{table}
\centering
\resizebox{0.65\columnwidth}{!}{%
\begin{tabular}{@{}cccc@{}}
\toprule
            & \textbf{Primary} & \textbf{Combined} & \textbf{Binary} \\ \midrule
All         &  0.601   &  0.578 & 0.579 \\
Velars      &  0.883   &  0.868 & 0.946 \\ 
Rhotics     &  0.210   &  0.117 & 0.050 \\
\bottomrule
\end{tabular}%
}
\caption{Krippendorf's $\alpha$ for primary score $s_p$, combined score $s_c$, and binary score $s_b$ computed for all annotators across all, velar, and rhotic samples. Agreement for binary score excludes samples not rated as correct productions or clear substitutions.}
\label{tab:annotation-krippendorf}
\end{table}


We measure pairwise agreement using Cohen's $\kappa$ \citep{cohen1960coefficient}, which measures agreement between two raters on categorical data.
The $kappa$ statistic is a standardised metric where $\kappa \in [-1, 1]$, with $\kappa = 0$ denoting chance agreement and $\kappa=1$ denoting perfect agreement.
Traditionally, Cohen's $\kappa$ is discussed according to the five agreement levels suggested by \citet[pp 164-165]{landis1977measurement}.
These group values of $\kappa$ into: poor ($\kappa \leq 0.0$), slight ($0.0 < \kappa \leqslant 0.2$), fair ($0.2 < \kappa \leqslant 0.40$), moderate ($0.40 < \kappa \leqslant 0.60$), substantial ($0.60 < \kappa \leqslant 0.80$), and almost perfect ($0.80 < \kappa \leqslant 1.0$) agreement.
Figure \ref{fig:annotation-cohen-kappa} visualises pairwise annotator agreement for the binary score.
Off-diagonal values denote pairwise inter-annotator agreement on the control set, whereas diagonal values denote intra-annotator agreement on the duplicate samples.

According to Figure \ref{fig:annotation-cohen-kappa}, scores provided for the velar samples are very consistent and reliable, with perfect agreement across most raters.
This is observed for inter and intra-annotator scores.
Annotator 1 has substantial agreement with some of the other raters, but not perfect.
Excluding samples rated by annotator 1 leads to improved global inter-annotator agreement for velar samples on the combined score ($\alpha=0.922$).
Results for the rhotic samples, however, indicate a substantial disagreement between annotators.
We observe a perfect agreement for intra-annotator scores across all annotators except annotator 8.
Rhotic agreement between annotators 6, 7, and 8 is higher than other raters, with perfect or moderate agreement.
However, considering only those three raters, global agreement on the combined score is still lower than the moderate reliability threshold for Krippendorf's $\alpha$ ($\alpha=0.631$).
There are various reasons that could explain the agreement discrepancy between velar and rhotic samples.
We provide further insights into these results in Section \ref{sec:discussion}.
However, these results indicate that we should use rhotic scores carefully when evaluating automatic error detection systems in the next section.


\section{Automatic speech error detection}
\label{sec:automatic_detection}

In this section we investigate the automatic detection of speech errors in typically developing and disordered Scottish English child speech.
The proposed system is designed as a tool to be used by Speech and Language Therapists on data collected from speech therapy and assessment sessions.
Therefore, we evaluate model scores with the expert scores for velar fronting and gliding of /r/ provided by therapists in Section \ref{sec:expert_detection}.
Our \textbf{goal} is to investigate the proposed system's ability to simulate expert behaviour in the detection of substitution errors.
Additionally, we aim to analyse the contribution of \emph{ultrasound tongue imaging} and \emph{out-of-domain adult data} on the automatic speech error detection.

\subsection{Data preparation}

\begin{table}[]
\centering
\resizebox{1.0\columnwidth}{!}{%
\begin{tabular}{@{}ccccl@{}}
\toprule
\multicolumn{1}{c}{\textbf{Data set}} & \multicolumn{1}{c}{\textbf{Source}} & \multicolumn{1}{c}{\textbf{Speakers}} & \multicolumn{1}{c}{\textbf{Samples}} & \multicolumn{1}{c}{\textbf{Notes}}                                                                        \\ \midrule
Train                                 & UXTD                                & 45                                    & 8302                                 & Child in-domain train data                                                                                \\
Train                                 & TaL                                 & 81                                    & 81193                                & Adult out-of-domain train data                                                                                  \\
Validation                            & UXTD                                & 5                                     & 534                                  &
Child in-domain validation data                                                                              \\
Test                                  & UXTD                                & 13                                    & 901                                  & Typically developing evaluation set                                                                       \\
Test                                  & UXSSD                               & 8                                     & 768                                  & Disordered speech evaluation set \\ \bottomrule
\end{tabular}%
}
\caption{Datasets used for automatic speech error detection. All sets contain samples from all places of articulation, except the disordered speech evaluation set, which contains the velar (384) and rhotic (384) samples rated by annotators in Section \ref{sec:expert_detection}.}
\label{tab:model-data-sets}
\end{table}


For \textbf{training data}, we use Ultrax Typically Developing (UXTD), which collected data from 58 child speakers.
UXTD includes a subset of utterances with manually-annotated word boundaries for 13 speakers.
We save data from those speakers for evaluation.
From the remaining 45 speakers, we randomly select 40 speakers for training and 5 speakers for validation.
The TaL corpus of adult speech is used as an additional source of training data.
We use the TaL80 dataset, containing data from 81 speakers.

For \textbf{evaluation data}, we use an evaluation set of disordered speech samples and an evaluation set of typically developing speech samples.
The disordered speech samples consist of the \emph{main set} rated by expert SLTs, described in Section \ref{sec:expert_detection}.
This is a set of 768 word instances from the UXSSD dataset.
The typically developing samples are extracted from the UXTD dataset, which includes 220 utterances with manually annotated word boundaries.
These utterances are produced by 13 speakers, disjoint from those in the training and validation sets.
After pre-processing, the typically developing evaluation set consists of 901 phone instances extracted from 866 words with a vocabulary of 153 words.

Because output classes are unbalanced, we control the number of samples per class for the training data.
For classes that are under-represented, we retrieve additional examples.
This is done by perturbing the anchor frame by up to 40ms for under-represented classes.
For classes that are over-represented, we randomly sample 1000 and 10000 examples for the UXTD and TaL sets, respectively.
After balancing and pre-processing, we have a total of 8302 for the UXTD training data.
The TaL corpus is larger than UXTD and has a total of 81193 samples.
Table \ref{tab:model-data-sets} shows the datasets used in this section, and their respective number of speakers and number of samples.

\begin{figure}
\includegraphics[width=\columnwidth]{figures_eps/diagram-sample-creation.eps}
\caption{\label{fig:diagram-sample-build} Sample build process. A sample is constructed around an anchor frame, typically the mid-point of a phone instance. Context windows are fixed at 100ms. Step is set such that 5 context frames are extracted for audio and 4 context frames for ultrasound.}
\end{figure}

The Kaldi speech recognition toolkit \citep{povey2011kaldi} is used to force-align all datasets at the phone level using the reference audio.
For the evaluation data, we constrain the phone alignment to the manually verified word boundaries.
The phone set is a Scottish accent variant of the Combilex lexicon \citep{richmond2009robust, richmond2010generating}.
We discard silence segments and vowels from the phone set and map the remaining phones onto one of nine classes corresponding to place of articulation:
\emph{alveolar, dental, labial, labiovelar, lateral, palatal, postalveolar, rhotic}, and \emph{velar}.
From the training data, we exclude phone instances that do not have parallel audio and ultrasound.
These instances occurred when audio started recording before the ultrasound.
For the UXTD data, there were 91 segments excluded due to early start.

We use Kaldi to extract Mel-frequency cepstral coefficients (MFCCs) for the audio signal.
MFCCs are commonly used for speech recognition, with good results reported for child speech recognition \citep{shivakumar2014improving}.
Waveforms are downsampled to 16KHz and features computed every 10ms over 25ms windows.
We keep 20 cepstral coefficients and append their respective first and second derivatives for a total of 60 features.
A high number of cepstral coefficients is helpful for child speech processing \citep{li2001automatic}.
Ultrasound frames are individually reshaped to $63\times103$ using bi-linear interpolation.
A single sample consists of an anchor frame and a set of context frames.
The anchor frame is fixed at the mid-point of each phone instance and the set of context frames are extracted over a fixed sized window of 100 ms to the left and right of the anchor frame.
Because of the different frame rates, the number of frames in the context window is different for the ultrasound and audio streams.
For the audio, each context window corresponds to 10 frames.
For ultrasound, the context window corresponds to 12 frames for Ultrasuite data and to 8 frames for TaL data.
Over each context window, we extract 5 MFCC frames and 4 ultrasound frames, with the step size set separately to account for the respective frame rates.
Figure \ref{fig:diagram-sample-build} illustrates the sample build process.


\subsection{Model architecture and training}

\begin{figure}
\includegraphics[width=\columnwidth]{figures_eps/diagram-model-architecture.eps}
\caption{\label{fig:diagram-model-architecture}
Convolutional neural network architecture for classifier using ultrasound and audio (MFCCs) inputs.}
\end{figure}


The adopted model architecture, illustrated in Figure \ref{fig:diagram-model-architecture}, largely follows that of earlier work \citep{ribeiro2019speaker, ribeiro2019ultrasound}.
The ultrasound stream is processed by two convolutional layers.
These layers use $5\times5$ kernels with 32 and 64 filters, respectively, and ReLU activation functions.
Each convolutional layer is followed by max-pooling with a $2\times2$ kernel.
The sequence of frames for the audio stream is flattened and processed by a fully-connected layer with rectified linear units.
When using the ultrasound and audio streams, the features are concatenated at this stage.
The batch normalized features are then processed by two fully-connected layers with ReLU activation functions and an output fully-connected layer followed by the softmax function.

Models are optimized via Stochastic Gradient Descent with minibatches of 128 samples and an L2 regularizer with weight 0.1.
We train models on the UXTD data or on the pooled TaL and UXTD data.
When using the UXTD training data, systems are optimized for 200 epochs with a learning rate of 0.1.
With the pooled dataset, systems are optimized for 50 epochs with an identical learning rate of 0.1.
After each epoch, the model is evaluated on the validation data and we keep the best model across all epochs.
We fine-tune systems trained on the pooled data on the UXTD data.
Models that are fine-tuned reduce the learning rate to 0.001 and are optimized for 100 epochs.

\subsection{Scoring}


The output of the classifier is a probability distribution over the nine places of articulation.
To score an input phone instance $x$, we consider an expected class $y$ and a competing class $c$.
The model score $s_m$ is then computed as
%
\begin{equation}
    s_m = \log(p(y|x)) - \log(p(c|x))
\label{eq:model_score}
\end{equation}
%
The expected class may be the canonical phone class, such as a velar or a rhotic.
The competing class is a possible substitution, such as an alveolar or a labiovelar approximant.
If no competing class is given, we can estimate it and compute the phone score with
%
\begin{equation}
c = \underset{q \in \mathcal{Q} - \{ y \}}{\arg\max}\, p(q|x)
\label{eq:model_max_score}
\end{equation}
%
where $\mathcal{Q}$ is the set of places of articulation considered by the model.
This method is related to the Goodness of Pronunciation score \citep{witt2000phone, hu2015improved}.
As with the combined expert score $s_c$ (Section \ref{subsec_expert_method}), 
the magnitude of the model score encodes certainty, whereas the sign encodes preference.
A positive score indicates preference for the expected class, while a negative score indicates preference for the competing class.
We simplify model and combined expert scores onto a binary correct/incorrect label $b(s)$ for error detection according to:
%
\begin{equation}
    b(s) =
    \begin{cases}
        0 & \text{if $s>k$} \\
        1 & \text{if $s\le k$} \\
    \end{cases}
\label{eq:model_binary}
\end{equation}
%
where $s$ is either $s_c$ or $s_m$ and $k$ is a configurable threshold.
Unless otherwise stated, results presented in this work use $k=0$, which treats uncertainty in the model score ($s_m=0$) as an error.
For the purposes of this analysis, uncertainty is not applicable to the combined expert score because we retain only cases that are correct or clear substitutions.

\begin{table*}[t]
\centering
\resizebox{0.89\textwidth}{!}{%
\begin{tabular}{@{}cccccccccc|cc@{}}
\textbf{Training Data} & \textbf{Alveolar} & \textbf{Dental} & \textbf{Labial} & \textbf{Labiovelar} & \textbf{Lateral} & \textbf{Palatal} & \textbf{Postalveolar} & \textbf{Rhotic} & \textbf{Velar} & \textbf{Global} &  \\ \midrule
                       & \multicolumn{10}{c}{\textit{Audio}}    \\ \midrule[.02em]

UXTD                   &  72.36\% &  46.77\%  &  64.00\%  &  52.5\%   &  84.85\%  &  64.52\%  &   68.66\% &   85.44\% &  75.62\%  &  70.81\%  \\
Joint                  &  70.56\%  &  35.14\% &  67.21\%  &  52.11\%  &  74.02\%  &  50.00\%  &   71.64\% &   88.64\% &  70.25\%  &  65.93\%   \\
Joint (+fine-tuning)   &  75.49\%  &  39.73\% &  66.67\%  &  59.42\%  &  80.56\%  &   66.67\% &   70.59\% &   85.71\%  &   77.65\% &  72.03\%   \\ \midrule[.02em]
                       & \multicolumn{10}{c}{\textit{Ultrasound}} \\ \midrule[.02em]
UXTD                   &  78.32\%  &  53.62\%   &  59.21\%   &  74.42\%   &  78.95\%   &  25.93\%  &   50.67\%  &  67.57\%  &  88.27\%   &  69.48\%   \\
Joint                  &  79.38\%  &  60.34\%   &  47.11\%   &  67.35\%   &  84.38\%   &  22.22\%  &   41.00\%  &  73.13\%  &  92.52\%   &  68.37\%   \\
Joint (+fine-tuning)   &  84.05\%  &  61.11\%   &  60.82\%   &  76.00\%   &  84.91\%   &  38.89\%  &   48.81\%  &  79.26\%  &  94.89\%   &  76.14\%   \\ \midrule[.02em]
                       & \multicolumn{10}{c}{\textit{Audio+Ultrasound}}   \\ \midrule[.02em]
UXTD                   &  80.10\%  &  81.08\%   &  83.87\%   &  82.61\%   &  90.18\%   &  65.52\%   &   58.33\%  &  74.83\%  &  89.58\%  &  80.47\%   \\
Joint                  &  83.25\%  &  76.74\%   &  83.33\%   &  63.08\%   &  73.19\%   &  66.67\%   &   76.19\%  &  91.59\%  &  93.45\%  &  81.80\%   \\
Joint (+fine-tuning)   &  87.94\%  &  76.47\%   &  87.78\%   &  72.31\%   &  91.82\%   &  58.82\%   &   75.38\%  &  94.34\%  &  95.58\%  &  \textbf{86.90\%}   \\ \midrule[.02em]
Number of samples      &  196  &  37   &  62   &  46   &  112   &  29   &   84  &  143   &  192  &   901   \\
\bottomrule
\end{tabular}%
}
\caption{Accuracy for the typically developing evaluation set across all places of articulation. Global accuracy is an average of all places of articulation, weighted by the number of samples for each class. The Joint training data denotes the pooled UXTD and TaL data. Highlighted results in bold indicate best overall performance.}
\label{tab:model-results-uxtd}
\end{table*}


\subsection{Results}

We evaluate model performance on the \textbf{typically developing} set.
Table \ref{tab:model-results-uxtd} shows accuracy results, which are computed across examples of all output classes.
Systems trained on the joint UXTD and TaL data underperform when compared with systems trained only on the UXTD data, 
even though there is more training data available.
However, fine-tuning the pre-trained joint model on the UXTD data leads to the best performance.

\begin{table}[t]
\centering
\resizebox{0.90\columnwidth}{!}{%
\begin{tabular}{@{}ccccc@{}}
\toprule
\textbf{Training data} & \textbf{Precision} & \textbf{Recall} & \textbf{F1-Score} & \textbf{Accuracy} \\ \hline
\multicolumn{5}{c}{\textit{Audio}}   \\ \midrule
UXTD                   &  0.384   &  0.524 &  0.443 &  64.1\%   \\
Joint                  &  0.417   &  0.585 &  0.487 &  66.5\%   \\ 
Joint (+fine-tuning)   &  0.393   &  0.537 &  0.454 &  64.9\%   \\ \midrule

\multicolumn{5}{c}{\textit{Ultrasound}}   \\ \midrule
UXTD                   &  0.670   &  0.842  &  0.746 &  84.4\%   \\
Joint                  &  0.702   &  0.805  &  0.750 &  85.4\%   \\ 
Joint (+fine-tuning)   &  0.677   &  0.768  &  0.720 &  83.7\%   \\ \midrule

\multicolumn{5}{c}{\textit{Audio+Ultrasound}}   \\ \midrule
UXTD                   &  0.732   &  0.866   &  \textbf{0.793} &  \textbf{87.7\%}   \\
Joint                  &  0.704   &  0.695   &  0.699 &  83.7\%   \\ 
Joint (+fine-tuning)   &  0.681   &  0.756   &  0.717 &  83.7\%   \\

\bottomrule
\end{tabular}%
}
\caption{Results for velar fronting error detection using the UXSSD velar samples rated by all annotators, except annotator 1.
Scores are computed using \enquote{velar} and \enquote{alveolar} as expected and competing classes respectively.}
\label{tab:model-results-velar}
\end{table}

Comparing systems using only one modality, accuracy results are better for ultrasound when using additional TaL data.
As expected, systems using both audio and ultrasound provide the best results.
Observing accuracy separately for each class, we observe that \emph{labial}, \emph{palatal}, \emph{postalveolar}, or \emph{rhotic} speech sounds have better results when using only audio compared to using only ultrasound.
The remaining speech sounds have better results with ultrasound tongue imaging.
Such differences are expected due to the individual characteristics of speech sounds.
For example, labial sounds do not rely on tongue movement, so they are not expected to benefit much from ultrasound tongue imaging alone.
On the other hand, velar and alveolar sounds have well-defined tongue shapes on the mid-saggital plane, so we would expect ultrasound data to be the primary contributor when identifying them.
We also observe that accuracy improves across all classes when using both modalities as input.
These results meet our expectations that ultrasound and audio complement each other well and that additional out-of-domain training data is beneficial.
Similar findings were reported on related tasks, such as speaker diarisation and word alignment of speech therapy sessions \citep{ribeiro2019ultrasound}.

Table \ref{tab:model-results-velar} shows results for \textbf{velar fronting error detection}.
These are computed over samples identified by annotators as correct velar productions or alveolar substitutions (see Figure \ref{fig:annotation-primary-secondary-matrix}).
We exclude samples rated by annotator 1, due to less than perfect agreement with other annotators.
Results are computed on $b(s)$ using the combined expert score $s_c$ and the model score $s_m$ with an expected \emph{velar} class and a competing \emph{alveolar} class.

We observe that ultrasound is more suited than audio to discriminate between velar and alveolar productions, although systems using both data streams have the best results.
There are no performance improvements to the systems using the joint dataset and fine-tuning when compared to the system using only typically developing child data.
This is an interesting observation, as results on the typically developing dataset indicate that using additional training data and fine-tuning is beneficial.
On the typically developing data, the individual accuracy for the velar and alveolar classes increases with more data and training.
Considering the system using both audio and ultrasound and comparing the UXTD and fine-tuned systems,
velar accuracy increases from 89.58\% to 95.58\% and alveolar accuracy increases from 80.10\% to 87.94\%.
The discrepancy observed between the typically developing set and velar fronting error detection could be attributed to challenges associated with speaker's data.
Speaker performance can vary substantially, particularly when using ultrasound data \citep{ribeiro2019speaker}.
This observation can be further supported by measuring agreement between model and expert binary scores for each of the eight speakers.
Using Cohen's $\kappa$ \citep{cohen1960coefficient}, models and expert scores have near perfect agreement for speakers 1 and 3 ($\kappa > 0.8$), substantial agreement for speakers 4 and 7 ($0.6 < \kappa \leq 0.8$), moderate agreement for speakers 5 and 6 ($0.4 < \kappa \leq 0.6$), and no or slight agreement ($\kappa \leq 0.2$) for speakers 2 and 8.

In Section \ref{subsec:annotator-agreement}, we reported intra- and inter-annotator agreement for the scoring of rhotic productions.
Unlike velars, expert scores for rhotics were shown to have very low inter-annotator agreement.
For this reason, results for \textbf{gliding error detection} should be interpreted carefully.
However, intra-annotator agreement was reliable and consistent for all annotators except annotator 8 (Figure \ref{fig:annotation-cohen-kappa}).
Therefore, we opt to analyse the results for gliding error detection separately for each speaker.

We note from Table \ref{tab:model-results-uxtd} that classification accuracy for rhotics and labiovelars is good across all classifiers on the typically developing evaluation set.
We observe that classification of rhotic instances benefits more from audio (85.71\%) than ultrasound (79.26\%).
The labiovelar class, on the other hand, achieves higher accuracy when using only ultrasound (76.0\%) than when using only audio (59.42\%).
Jointly using audio and ultrasound improves accuracy for rhotics (94.34\%) but not for labiovelars (72.31\%).
The average accuracy for rhotic and labiovelars when using ultrasound and audio is 78.72\% when training only on the UXTD data.
This accuracy slightly decreases when jointly training on the TaL corpus (77.34\%), but improves when fine-tuning on the UXTD data (83.33\%).
These results, however, relate to the typically developing evaluation set.
As observed with the velar case, they may not transfer in the same way to error detection.
Nevertheless, we select the fine-tuned system using both ultrasound and audio to analyse speaker-wise results for gliding error detection.

\begin{table}[t]
\centering
\resizebox{0.95\columnwidth}{!}{%
\begin{tabular}{@{}ccccccc@{}}
\toprule
\textbf{Speaker} & \textbf{N}  & \textbf{Precision} & \textbf{Recall} & \textbf{F1-Score}    & \textbf{Accuracy} & \textbf{Cohen's $\kappa$} \\ \midrule
1       & 41 & 0.778     & 0.467  & 0.583 & 75.6\%  & 0.426 \\
2       & 29 & 1.000     & 0.071  & 0.133 & 55.2\%  & 0.074 \\
3       & 36 & 0.900     & 0.474  & 0.621 & 69.4\%  & 0.404 \\
4       & 11 & 0.833     & 1.000  & 0.909 & 90.9\%  & 0.820 \\
5       & 42 & 0.964     & 0.675  & 0.794 & 66.7\%  & 0.045 \\
6       & 36 & 1.000     & 0.639  & 0.780 & 63.9\%  & 0.000 \\
7       & 45 & 0.500     & 1.000  & 0.667 & 97.8\%  & 0.656 \\
8       & 35 & 0.692     & 0.783  & 0.735 & 62.9\%  & 0.123 \\ \bottomrule
\end{tabular}%
}
\caption{Speaker-wise results for gliding error detection using the UXSSD rhotic samples identified as correct productions or gliding substitutions. Cohen's $\kappa$ is calculated on expert and model binary scores.
Results are generated by the fine-tuned system using both audio and ultrasound, with
scores computed using \enquote{rhotic} and \enquote{labiovelar} as expected and competing classes, respectively.
}
\label{tab:model-results-rhotic}
\end{table}


\begin{figure}[t]
\centering
\includegraphics[width=0.80\columnwidth]{figures_eps/rhotic_confusion_matrices.eps}
\caption{\label{fig:rhotic_confusion_matrices}
Confusion matrices for gliding error detection produced by the fine-tuned system using audio and ultrasound for all 8 speakers in the UXSSD evaluation set. Correct rhotic productions are denoted by $0$ and gliding substitutions are denoted by $1$. }
\end{figure}


Table \ref{tab:model-results-rhotic} shows speaker-wise results for gliding error detection and Figure \ref{fig:rhotic_confusion_matrices} shows their respective confusion matrices.
We observe that the scores given by the expert annotators can vary per speaker.
For example, most samples produced by speakers 5 and 6 were marked as gliding cases by their respective annotators.
The limited number of correct productions influences the calculation of Cohen's $\kappa$, leading to poor agreement even though accuracy and F$_1$ are high.
On the other hand, most samples by speaker 7 were marked as correct instances.
The classifier appears to behave similarly with data from speakers 2 and 7, with most samples classified as correct rhotic instances.
This behaviour is in agreement with the expert for speaker 7, but not for speaker 2.
Most errors produced by the model are Type II errors (false negatives).
This might be due to the lower performance of the competing labiovelar class, as observed on the typically developing set.
Because the classifier is more confident when scoring rhotics, there is a limited number of Type I errors (false positives).
Due to the low inter-annotator agreement, it is not clear whether these differences are due to the scores provided by the annotators or due to challenges associated with speaker or recording variability.

\mySection{Related Works and Discussion}{}
\label{chap3:sec:discussion}

In this section we briefly discuss the similarities and differences of the model presented in this chapter, comparing it with some related work presented earlier (Chapter \ref{chap1:artifact-centric-bpm}). We will mention a few related studies and discuss directly; a more formal comparative study using qualitative and quantitative metrics should be the subject of future work.

Hull et al. \citeyearpar{hull2009facilitating} provide an interoperation framework in which, data are hosted on central infrastructures named \textit{artifact-centric hubs}. As in the work presented in this chapter, they propose mechanisms (including user views) for controlling access to these data. Compared to choreography-like approach as the one presented in this chapter, their settings has the advantage of providing a conceptual rendezvous point to exchange status information. The same purpose can be replicated in this chapter's approach by introducing a new type of agent called "\textit{monitor}", which will serve as a rendezvous point; the behaviour of the agents will therefore have to be slightly adapted to take into account the monitor and to preserve as much as possible the autonomy of agents.

Lohmann and Wolf \citeyearpar{lohmann2010artifact} abandon the concept of having a single artifact hub \cite{hull2009facilitating} and they introduce the idea of having several agents which operate on artifacts. Some of those artifacts are mobile; thus, the authors provide a systematic approach for modelling artifact location and its impact on the accessibility of actions using a Petri net. Even though we also manipulate mobile artifacts, we do not model artifact location; rather, our agents are equipped with capabilities that allow them to manipulate the artifacts appropriately (taking into account their location). Moreover, our approach considers that artifacts can not be remotely accessed, this increases the autonomy of agents.

The process design approach presented in this chapter, has some conceptual similarities with the concept of \textit{proclets} proposed by Wil M. P. van der Aalst et al. \citeyearpar{van2001proclets, van2009workflow}: they both split the process when designing it. In the model presented in this chapter, the process is split into execution scenarios and its specification consists in the diagramming of each of them. Proclets \cite{van2001proclets, van2009workflow} uses the concept of \textit{proclet-class} to model different levels of granularity and cardinality of processes. Additionally, proclets act like agents and are autonomous enough to decide how to interact with each other.

The model presented in this chapter uses an attributed grammar as its mathematical foundation. This is also the case of the AWGAG model by Badouel et al. \citeyearpar{badouel14, badouel2015active}. However, their model puts stress on modelling process data and users as first class citizens and it is designed for Adaptive Case Management.

To summarise, the proposed approach in this chapter allows the modelling and decentralized execution of administrative processes using autonomous agents. In it, process management is very simply done in two steps. The designer only needs to focus on modelling the artifacts in the form of task trees and the rest is easily deduced. Moreover, we propose a simple but powerful mechanism for securing data based on the notion of accreditation; this mechanism is perfectly composed with that of artifacts. The main strengths of our model are therefore : 
\begin{itemize}
	\item The simplicity of its syntax (process specification language), which moreover (well helped by the accreditation model), is suitable for administrative processes;
	\item The simplicity of its execution model; the latter is very close to the blockchain's execution model \cite{hull2017blockchain, mendling2018blockchains}. On condition of a formal study, the latter could possess the same qualities (fault tolerance, distributivity, security, peer autonomy, etc.) that emanate from the blockchain;
	\item Its formal character, which makes it verifiable using appropriate mathematical tools;
	\item The conformity of its execution model with the agent paradigm and service technology.
\end{itemize}
In view of all these benefits, we can say that the objectives set for this thesis have indeed been achieved. However, the proposed model is perfectible. For example, it can be modified to permit agents to respond incrementally to incoming requests as soon as any prefix of the extension of a bud is produced. This makes it possible to avoid the situation observed on figure \ref{chap3:fig:execution-figure-4} where the associated editor is informed of the evolution of the subtree resulting from $C$ only when this one is closed. All the criticisms we can make of the proposed model in particular, and of this thesis in general, have been introduced in the general conclusion (page \pageref{chap5:general-conclusion}) of this manuscript.




% \vspace{-0.5em}
\section{Conclusion}
% \vspace{-0.5em}
Recent advances in multimodal single-cell technology have enabled the simultaneous profiling of the transcriptome alongside other cellular modalities, leading to an increase in the availability of multimodal single-cell data. In this paper, we present \method{}, a multimodal transformer model for single-cell surface protein abundance from gene expression measurements. We combined the data with prior biological interaction knowledge from the STRING database into a richly connected heterogeneous graph and leveraged the transformer architectures to learn an accurate mapping between gene expression and surface protein abundance. Remarkably, \method{} achieves superior and more stable performance than other baselines on both 2021 and 2022 NeurIPS single-cell datasets.

\noindent\textbf{Future Work.}
% Our work is an extension of the model we implemented in the NeurIPS 2022 competition. 
Our framework of multimodal transformers with the cross-modality heterogeneous graph goes far beyond the specific downstream task of modality prediction, and there are lots of potentials to be further explored. Our graph contains three types of nodes. While the cell embeddings are used for predictions, the remaining protein embeddings and gene embeddings may be further interpreted for other tasks. The similarities between proteins may show data-specific protein-protein relationships, while the attention matrix of the gene transformer may help to identify marker genes of each cell type. Additionally, we may achieve gene interaction prediction using the attention mechanism.
% under adequate regulations. 
% We expect \method{} to be capable of much more than just modality prediction. Note that currently, we fuse information from different transformers with message-passing GNNs. 
To extend more on transformers, a potential next step is implementing cross-attention cross-modalities. Ideally, all three types of nodes, namely genes, proteins, and cells, would be jointly modeled using a large transformer that includes specific regulations for each modality. 

% insight of protein and gene embedding (diff task)

% all in one transformer

% \noindent\textbf{Limitations and future work}
% Despite the noticeable performance improvement by utilizing transformers with the cross-modality heterogeneous graph, there are still bottlenecks in the current settings. To begin with, we noticed that the performance variations of all methods are consistently higher in the ``CITE'' dataset compared to the ``GEX2ADT'' dataset. We hypothesized that the increased variability in ``CITE'' was due to both less number of training samples (43k vs. 66k cells) and a significantly more number of testing samples used (28k vs. 1k cells). One straightforward solution to alleviate the high variation issue is to include more training samples, which is not always possible given the training data availability. Nevertheless, publicly available single-cell datasets have been accumulated over the past decades and are still being collected on an ever-increasing scale. Taking advantage of these large-scale atlases is the key to a more stable and well-performing model, as some of the intra-cell variations could be common across different datasets. For example, reference-based methods are commonly used to identify the cell identity of a single cell, or cell-type compositions of a mixture of cells. (other examples for pretrained, e.g., scbert)


%\noindent\textbf{Future work.}
% Our work is an extension of the model we implemented in the NeurIPS 2022 competition. Now our framework of multimodal transformers with the cross-modality heterogeneous graph goes far beyond the specific downstream task of modality prediction, and there are lots of potentials to be further explored. Our graph contains three types of nodes. while the cell embeddings are used for predictions, the remaining protein embeddings and gene embeddings may be further interpreted for other tasks. The similarities between proteins may show data-specific protein-protein relationships, while the attention matrix of the gene transformer may help to identify marker genes of each cell type. Additionally, we may achieve gene interaction prediction using the attention mechanism under adequate regulations. We expect \method{} to be capable of much more than just modality prediction. Note that currently, we fuse information from different transformers with message-passing GNNs. To extend more on transformers, a potential next step is implementing cross-attention cross-modalities. Ideally, all three types of nodes, namely genes, proteins, and cells, would be jointly modeled using a large transformer that includes specific regulations for each modality. The self-attention within each modality would reconstruct the prior interaction network, while the cross-attention between modalities would be supervised by the data observations. Then, The attention matrix will provide insights into all the internal interactions and cross-relationships. With the linearized transformer, this idea would be both practical and versatile.

% \begin{acks}
% This research is supported by the National Science Foundation (NSF) and Johnson \& Johnson.
% \end{acks}

\bibliography{references}

\end{document}