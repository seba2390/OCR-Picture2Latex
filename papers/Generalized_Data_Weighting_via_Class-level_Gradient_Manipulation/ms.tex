\pdfoutput=1
\documentclass{article}

% if you need to pass options to natbib, use, e.g.:
%     \PassOptionsToPackage{numbers, compress}{natbib}
% before loading neurips_2021

% ready for submission
\usepackage[final]{neurips_2021}

% to compile a preprint version, e.g., for submission to arXiv, add add the
% [preprint] option:
%     \usepackage[preprint]{neurips_2021}

% to compile a camera-ready version, add the [final] option, e.g.:
%     \usepackage[final]{neurips_2021}

% to avoid loading the natbib package, add option nonatbib:
%    \usepackage[nonatbib]{neurips_2021}

\usepackage[utf8]{inputenc} % allow utf-8 input
\usepackage[T1]{fontenc}    % use 8-bit T1 fonts
\usepackage{hyperref}       % hyperlinks
\usepackage{url}            % simple URL typesetting
\usepackage{booktabs}       % professional-quality tables
\usepackage{amsfonts}       % blackboard math symbols
\usepackage{nicefrac}       % compact symbols for 1/2, etc.
\usepackage{microtype}      % microtypography
\usepackage{xcolor}         % colors
\usepackage{amsmath}
\usepackage{multirow} 
\usepackage{multicol} 
\usepackage{arydshln}
\usepackage{graphicx}
%\usepackage{todonotes}

\usepackage{caption}

\usepackage{algorithm}
\usepackage{enumitem}

\usepackage{wrapfig}
\usepackage{ulem}
\usepackage{bm}
\usepackage{bbding}
\usepackage{pifont}
\usepackage{wasysym}
\usepackage{amssymb}
\usepackage[noend]{algpseudocode}
\algrenewcommand\alglinenumber[1]{{\sffamily\footnotesize#1}}
\algnewcommand{\LineComment}[1]{\State \(\triangleright\) #1}
\usepackage{todonotes}

%\usepackage[normalem]{ulem}
\newcommand{\canadd}[1]{{\color{blue}  \textbf{(can\_add: #1)}}}
\newcommand{\old}[1]{{\color{green} \sout{#1}}}
\newcommand{\zhengadd}[1]{{\color{red}  \textbf{(zheng\_add: #1)}}}
\newcommand{\erqunadd}[1]{{\color{yellow} \textbf{(erqun\_add:) #1}}}

\title{Generalized Data Weighting via Class-level Gradient Manipulation}

% The \author macro works with any number of authors. There are two commands
% used to separate the names and addresses of multiple authors: \And and \AND.
%
% Using \And between authors leaves it to LaTeX to determine where to break the
% lines. Using \AND forces a line break at that point. So, if LaTeX puts 3 of 4
% authors names on the first line, and the last on the second line, try using
% \AND instead of \And before the third author name.
\newcommand*\samethanks[1][\value{footnote}]{\footnotemark[#1]}

\author{%
  Can Chen\textsuperscript{1}\thanks{Equal contribution; Names listed in alphabetical order.}~, Shuhao Zheng\textsuperscript{1}\samethanks~, Xi Chen\textsuperscript{1}, Erqun Dong\textsuperscript{1}, Xue Liu\textsuperscript{1}, Hao Liu\textsuperscript{2}, Dejing Dou\textsuperscript{3}\\
  \\
  \textsuperscript{1}McGill University, \textsuperscript{2}The Hong Kong University of Science and Technology, \textsuperscript{3}Baidu Research\\
  \texttt{\{can.chen, shuhao.zheng, erqun.dong\}@mail.mcgill.ca}\\ \texttt{xi.chen11@mcgill.ca},
  \texttt{xueliu@cs.mcgill.ca}, \texttt{liuh@ust.hk}, \texttt{doudejing@baidu.com} \\
}

\begin{document}

\maketitle

\begin{abstract}
Label noise and class imbalance are two major issues coexisting in real-world datasets.
%
To alleviate the two issues, state-of-the-art methods reweight each instance by leveraging a small amount of clean and unbiased data.
%
Yet, these methods overlook class-level information within each instance, which can be further utilized to improve performance.
%
To this end, in this paper, we propose \textbf{G}eneralized \textbf{D}ata \textbf{W}eighting (\textbf{GDW}) to simultaneously mitigate label noise and class imbalance by manipulating gradients at the class level.
%
To be specific, GDW unrolls the loss gradient to class-level gradients by the chain rule and reweights the flow of each gradient separately.
%
In this way, GDW achieves remarkable performance improvement on both issues. 
%
Aside from the performance gain, GDW efficiently obtains class-level weights without introducing any extra computational cost compared with instance weighting methods.
%
Specifically, GDW performs a gradient descent step on class-level weights, which only relies on intermediate gradients.
%
Extensive experiments in various settings verify the effectiveness of GDW.
%
For example, GDW outperforms state-of-the-art methods by $2.56\%$ under the $60\%$ uniform noise setting in CIFAR10. 
%
Our code is available at \url{https://github.com/GGchen1997/GDW-NIPS2021}.
\end{abstract}


\pdfoutput=1
\section{Introduction}
%
Real-world classification datasets often suffer from two issues, i.e., label noise~\cite{songLearningNoisyLabels2021} and class imbalance~\cite{heLearningImbalancedData2009}.
%
On the one hand, label noise often results from the limitation of data generation, e.g., sensor errors~\cite{elhadySystematicSurveySensor2018a} and mislabeling from crowdsourcing workers~\cite{tongxiaoLearningMassiveNoisy2015}.
%
Label noise misleads the training process of DNNs and degrades the model performance in various aspects~\cite{alganLabelNoiseTypes2020b,zhuClassNoiseVs2004a,frenayClassificationPresenceLabel2014a}.
%
On the other hand, imbalanced datasets are either naturally long-tailed~\cite{zhaoLongTailDistributionsUnsupervised2012a,vanhornDevilTailsFinegrained2017a} or biased from the real-world distribution due to imperfect data collection~\cite{pavonAssessingImpactClassImbalanced2011a,patelReviewClassificationImbalanced2020a}.
%
Training with imbalanced datasets usually results in poor classification performance on weakly represented classes~\cite{dongClassRectificationHard2017a,cuiClassBalancedLossBased2019,sinhaClassWiseDifficultyBalancedLoss2021a}.
%
Even worse, these two issues often coexist in real-world datasets~\cite{johnsonSurveyDeepLearning2019a}.

%
To prevent the model from memorizing noisy information, many important works have been proposed, including label smoothing~\cite{szegedyRethinkingInceptionArchitecture2016a}, noise adaptation~\cite{goldbergerTrainingDeepNeuralnetworks2017}, importance weighting~\cite{liuClassificationNoisyLabels2014}, GLC~\cite{hendrycksUsingTrustedData2018}, and Co-teach~\cite{hanCoteachingRobustTraining2018a}.
%
Meanwhile, \cite{dongClassRectificationHard2017a,cuiClassBalancedLossBased2019,sinhaClassWiseDifficultyBalancedLoss2021a,linFocalLossDense2020} propose effective methods to tackle class imbalance.
% 
However, these methods inevitably introduce hyper-parameters (e.g., the weighting factor in~\cite{cuiClassBalancedLossBased2019} and the focusing parameter in~\cite{linFocalLossDense2020}), compounding real-world deployment.

%
Inspired by recent advances in meta-learning, 
some works~\cite{renLearningReweightExamples2018,shuMetaWeightNetLearningExplicit2019a,huLearningDataManipulation2019a,wangOptimizingDataUsage2020b} propose to solve both issues by leveraging a clean and unbiased meta set.
%
These methods treat instance weights as hyper-parameters and
dynamically update these weights to circumvent hyper-parameter tuning.
%
Specifically, MWNet~\cite{shuMetaWeightNetLearningExplicit2019a} adopts an MLP with the instance loss as input and the instance weight as output.
%
Due to the MLP, MWNet has better scalability on large datasets compared with INSW~\cite{huLearningDataManipulation2019a} which assigns each instance with a learnable weight.
%
Although these methods can handle label noise and class imbalance to some extent, they cannot fully utilize class-level information within each instance, resulting in the potential loss of useful information.
%
For example, in a three-class classification task, every instance has three logits.
%
As shown in Figure~\ref{fig:motiv}, every logit corresponds to a class-level gradient flow which stems from the loss function and back-propagates.
%
These gradient flows represent three kinds of information: "not cat", "dog", and "not bird".
%
Instance weighting methods~\cite{shuMetaWeightNetLearningExplicit2019a,renLearningReweightExamples2018} alleviate label noise by downweighting all the gradient flows of the instance, which discards three kinds of information simultaneously. 
%
Yet, downweighting the "not bird" gradient flow is a waste of information.
%
Similarly, in class imbalance scenarios, different gradient flows represent different class-level information.
\begin{wrapfigure}[19]{l}{0.5\textwidth}
    \centering
    \includegraphics[scale=0.19]{figures/intro_motivation.png}
    \caption{Motivation for class-level weighting. For a noisy instance (e.g. cat mislabeled as "dog"), all gradient flows are downweighted by instance weighting. Although the gradient flows for "dog" and "not cat" contain harmful information, the gradient flow for "not bird" is still valuable for training, which should not be downweighted.}
    \label{fig:motiv}
\end{wrapfigure}
%
Therefore, it is necessary to reweight instances at the class level for better information usage.

%
To this end, we propose Generalized Data Weighting~(\textbf{GDW}) to tackle label noise and class imbalance by class-level gradient manipulation.
%
Firstly, we introduce class-level weights to represent the importance of different gradient flows and manipulate the gradient flows with these class-level weights.
%
Secondly, we impose a zero-mean constraint on class-level weights for stable training.
%
Thirdly, to efficiently obtain class-level weights, we develop a two-stage weight generation scheme embedded in  bi-level optimization.
%
As a side note, the instance weighting methods~\cite{renLearningReweightExamples2018,shuMetaWeightNetLearningExplicit2019a,huLearningDataManipulation2019a,wangOptimizingDataUsage2020b} can be considered special cases of GDW when class-level weights within any instance are the same.
%
In this way, GDW achieves impressive performance improvement in various settings.

%
To sum up, our contribution is two-fold:
\begin{enumerate}
    %
    \item For better information utilization, we propose GDW, a generalized data weighting method, which better handles label noise and class imbalance. To the best of our knowledge, we are the first to propose single-label class-level weighting on gradient flows.
    %
    \item To obtain class-level weights efficiently, we design a two-stage scheme embedded in a bi-level optimization framework, which does not introduce any extra computational cost.
    %
    To be specific, during the back-propagation we store intermediate gradients, with which we update class-level weights via a gradient descent step.
\end{enumerate}
\pdfoutput=1
\section{Related Works}
\subsection{Non-Meta-Learning Methods for Label Noise}
%
Label noise is a common problem in classification tasks~\cite{zhuClassNoiseVs2004a,frenayClassificationPresenceLabel2014a,alganLabelNoiseTypes2020b}.
%
To avoid overfitting to label noise, \cite{szegedyRethinkingInceptionArchitecture2016a} propose label smoothing to regularize the model.
%
\cite{goldbergerTrainingDeepNeuralnetworks2017,vahdatRobustnessLabelNoise2017a} form different models to indicate the relation between noisy instances and clean instances.
%
\cite{liuClassificationNoisyLabels2014} estimate an importance weight for each instance to represent its value to the model.
%
\cite{hanCoteachingRobustTraining2018a} train two models simultaneously and let them teach each other in every mini-batch.
%
However, without a clean dataset, these methods cannot handle severe noise~\cite{renLearningReweightExamples2018}.
%
\cite{hendrycksUsingTrustedData2018} correct the prediction of the model by estimating the label corruption matrix via a clean validation set, but this matrix is the same across all instances.
%
Instead, our method generates dynamic class-level weights for every instance to improve training.

\subsection{Non-Meta-Learning Methods for Class Imbalance}
%
Many important works have been proposed to handle class imbalance~\cite{elkanFoundationsCostsensitiveLearning2001,chawlaSMOTESyntheticMinority2002a,anandApproachClassificationHighly2010a,dongClassRectificationHard2017a,khanCostSensitiveLearningDeep2018,cuiClassBalancedLossBased2019,kangDecouplingRepresentationClassifier2019,linFocalLossDense2020,sinhaClassWiseDifficultyBalancedLoss2021a}.
%
\cite{chawlaSMOTESyntheticMinority2002a,anandApproachClassificationHighly2010a} propose to over-sample
the minority class and under-sample the majority class.
%
\cite{elkanFoundationsCostsensitiveLearning2001,khanCostSensitiveLearningDeep2018} learn a class-dependent cost matrix to obtain robust representations for both majority and minority classes.
%
\cite{dongClassRectificationHard2017a,cuiClassBalancedLossBased2019,linFocalLossDense2020,sinhaClassWiseDifficultyBalancedLoss2021a} design a
reweighting scheme to rebalance the loss for each class.
%
These methods are quite effective, whereas they need to manually choose loss functions or hyper-parameters.
%
\cite{liuLargeScaleLongTailedRecognition2019,wangLongtailedRecognitionRouting2020} manipulate the feature space to handle class imbalance while introducing extra model parameters.
%
\cite{kangDecouplingRepresentationClassifier2019} decouple representation learning and classifier learning on long-tailed datasets, but with extra hyper-parameter tuning.
%
In contrast, meta-learning methods view instance weights as hyper-parameters and dynamically update them via a meta set to avoid hyper-parameter tuning.

\subsection{Meta-Learning Methods}
%
With recent development in meta-learning~\cite{lakeHumanlevelConceptLearning2015a,franceschiBilevelProgrammingHyperparameter2018a,liuDARTSDifferentiableArchitecture2018}, many important methods have been proposed to handle label noise and class imbalance via a meta set~\cite{wangLearningModelTail2017,jiangMentorNetLearningDataDriven2018a,renLearningReweightExamples2018,liLearningLearnNoisy2019a,shuMetaWeightNetLearningExplicit2019a,huLearningDataManipulation2019a,wangOptimizingDataUsage2020b,alganMetaSoftLabel2021a}.
%
\cite{jiangMentorNetLearningDataDriven2018a} propose MentorNet to provide a data-driven curriculum for the base network to focus on correct instances.
%
To distill effective supervision, \cite{zhangDistillingEffectiveSupervision2020a} estimate pseudo labels for noisy instances with a meta set.
%
To provide dynamic regularization, \cite{vyasLearningSoftLabels2020b,alganMetaSoftLabel2021a} treat labels as learnable parameters and adapt them to the model’s state.
%
Although these methods can tackle label noise, they introduce huge amounts of learnable parameters and thus cannot scale to large datasets.
%
To alleviate class imbalance, \cite{wangLearningModelTail2017} describe a method to learn from long-tailed datasets.
%
Specifically, \cite{wangLearningModelTail2017} propose to encode meta-knowledge into a meta-network and model the tail classes by transfer learning. 
 
\begin{table}
  \caption{Related works comparison. 
  %
"Noise" and "Imbalance" denote whether the method can solve label noise and class imbalance.
%
"Class-level" denotes whether the method utilizes class-level information in each instance, and "Scalability" denotes whether the method can scale to large datasets.}
  \label{tab: comparison_methods}
  \scalebox{0.6}{
  \centering
  \begin{tabular}{@{}c c c c c c c c c c c@{}}
    \toprule
     & Focal \cite{linFocalLossDense2020} & Balanced \cite{cuiClassBalancedLossBased2019} & Co-teaching \cite{hanCoteachingRobustTraining2018a} & GLC \cite{hendrycksUsingTrustedData2018} & L2RW \cite{renLearningReweightExamples2018} & INSW \cite{huLearningDataManipulation2019a} & MWNet \cite{shuMetaWeightNetLearningExplicit2019a} & Soft-label \cite{vyasLearningSoftLabels2020b} & Gen-label \cite{alganMetaSoftLabel2021a}& \textbf{GDW} \\
    \midrule
    Noise & \XSolid & \XSolid & \Checkmark & \Checkmark & \Checkmark & \Checkmark &\Checkmark &\Checkmark &\Checkmark &\Checkmark     \\
    Imbalance & \Checkmark& \Checkmark& \XSolid & \XSolid & \Checkmark &  \Checkmark & \Checkmark & \XSolid & \XSolid & \Checkmark     \\
    Class-level &  \XSolid & \XSolid &  \XSolid & \XSolid &  \XSolid & \XSolid &  \XSolid & \Checkmark &  \Checkmark & \Checkmark  \\
    Scalability & \Checkmark& \Checkmark& \Checkmark & \Checkmark & \Checkmark &  \XSolid & \Checkmark & \XSolid & \XSolid & \Checkmark     \\
    \bottomrule
  \end{tabular}
  }
\end{table}

%
Furthermore, many meta-learning methods propose to mitigate the two issues by reweighting every instance~\cite{renLearningReweightExamples2018,saxenaDataParametersNew2019a,shuMetaWeightNetLearningExplicit2019a,huLearningDataManipulation2019a,wangOptimizingDataUsage2020b}.
%
\cite{saxenaDataParametersNew2019a} equip each instance and each class with a learnable parameter to govern their importance.
%
By leveraging a meta set, \cite{renLearningReweightExamples2018,shuMetaWeightNetLearningExplicit2019a,huLearningDataManipulation2019a,wangOptimizingDataUsage2020b} learn instance weights and model parameters via bi-level optimization to tackle label noise and class imbalance.
%
\cite{renLearningReweightExamples2018} assign weights to training instances only based on their gradient directions.
%
Furthermore, \cite{huLearningDataManipulation2019a} combine reinforce learning and meta-learning, and treats instance weights as rewards for optimization.
%
However, since each instance is directly assigned with a learnable weight, INSW can not scale to large datasets.
%
Meanwhile, \cite{shuMetaWeightNetLearningExplicit2019a,wangOptimizingDataUsage2020b} adopt a weighting network to output weights for instances and use bi-level optimization to jointly update the weighting network parameters and model parameters.
%
Although these methods handle label noise and class imbalance by reweighting instances, a scalar weight for every instance cannot capture class-level information, as shown in Figure~\ref{fig:motiv}.
%
Therefore, we introduce class-level weights for different gradient flows and adjust them to better utilize class-level information.
%

We show the differences between GDW and other related methods in Table \ref{tab: comparison_methods}.



\pdfoutput=1
\section{Method}
\label{method}
\subsection{Notations}
%
In most classification tasks, there is a training set $D_{train}=\{(x_i, y_i)\}_{i=1}^N$ and we assume there is also a clean unbiased meta set $D_{meta}=\{(x_i^v, y_i^v)\}_{i=1}^M$.
%
We aim to alleviate label noise and class imbalance in  $D_{train}$ with $D_{meta}$.
%
The model parameters are denoted as $\boldsymbol \theta$, and the number of classes is denoted as $C$.

\subsection{Class-level Weighting by Gradient Manipulation}
\label{subsec:gradManip}
%
To utilize class-level information, we learn a class-level weight for every gradient flow instead of a scalar weight for all $C$ gradient flows in~\cite{shuMetaWeightNetLearningExplicit2019a}.
%
Denote $\mathcal{L}$ as the loss of any instance.
%
Applying the chain rule, we unroll the gradient of $\mathcal{L}$ w.r.t. $\boldsymbol \theta$ as
\begin{equation}
    \label{eq:chainRule}
    \nabla_{\boldsymbol \theta} \mathcal{L}= \frac{\partial \mathcal{L}}{\partial \mathbf{\boldsymbol \theta}} = \frac{\partial \mathcal{L}}{\partial \mathbf{l}} \frac{\partial \mathbf{l}}{\partial \mathbf{\boldsymbol \theta}} \doteq \mathbf{D}_{1}\mathbf{D}_{2},
\end{equation}
where $\mathbf{l}\in\mathbb{R}^C$ represents the predicted logit vector of the instance.
%
We introduce class-level weights $\boldsymbol \omega \in\mathbb{R}^C$ and denote the $j^{th}$ component of $\boldsymbol \omega$ as $\boldsymbol \omega_j$.
%
To indicate the importance of every gradient flow, we perform an element-wise product $f_{\boldsymbol \omega}(\cdot)$ on $\mathbf{D}_1$ with $\boldsymbol \omega$.
%
After this manipulation, the gradient becomes
\begin{equation}
    \label{eq:newGrad}
    f_{\boldsymbol \omega}\left(\nabla_{\boldsymbol \theta} \mathcal{L}\right) \doteq \left(\boldsymbol \omega \otimes \frac{\partial \mathcal L}{\partial \mathbf{l}}\right) \frac{\partial \mathbf{l}}{\partial \boldsymbol \theta} = \left(\boldsymbol \omega \otimes \mathbf{D}_1\right) \mathbf{D}_2 \doteq \mathbf{D}'_1\mathbf{D}_2,
\end{equation}
where $\otimes$ denotes the element-wise product of two vectors.
%
Note that $\boldsymbol \omega_j$ represents the importance of the $j^{th}$ gradient flow.
%
Obviously, instance weighting is a special case of GDW when elements of $\boldsymbol \omega$ are the same.
%
Most classification tasks~\cite{howardMobileNetsEfficientConvolutional2017a,qinRethinkingSoftmaxCrossEntropy2020a,zhaoBetterAccuracyefficiencyTradeoffs2021a} adopt the \textit{Softmax-CrossEntropy} loss.
%
In this case, we have $\mathbf{D}_1 = \mathbf p - \mathbf y$, where $\mathbf p \in \mathbb{R}^C$ denotes the probability vector output by \textit{softmax} and $\mathbf y\in\mathbb{R}^C$ denotes the one-hot label of the instance~(see Appendix~\ref{section: Appendx_A} for details).

%
As shown in Figure~\ref{fig:motiv}, for a noisy instance (e.g., cat mislabeled as "dog"), instance weighting
methods assign a low scalar weight to all gradient flows of the instance.
%
Instead, GDW assigns class-level weights to different gradient flows by leveraging the meta set.
%
In other words, GDW tries to downweight the gradient flows for "dog" and "not cat", and upweight the gradient flow for "not bird".
%
Similarly, in imbalance settings, different gradient flows have different class-level information.
%
Thus GDW can also better handle class imbalance by adjusting the importance of different gradient flows. 

\subsection{Zero-mean Constraint on Class-level Weights}
\label{subsec:restOn}
%
To retain the \textit{Softmax-CrossEntropy} loss structure, i.e. the $\boldsymbol p -\boldsymbol y$ form, after the manipulation, we impose a zero-mean constraint on $\mathbf{D}'_1$.
%
That is, we analyze the $j^{th}$ element of $\mathbf{D}'_1$~(see Appendix \ref{subsection: Appendix_B.1} for details):
\begin{align}
    \boldsymbol{\omega}_j(\boldsymbol{p}_j-\mathbf{y}_j) =& \boldsymbol{\omega}_t\left({\boldsymbol{p}'_j} - {\mathbf{y}_j}\right) + \left(\sum_k{{\boldsymbol{\omega}_k} {\boldsymbol{p}_k}}-\boldsymbol{\omega}_t\right){\boldsymbol{p}'_j}\label{eq:grad1'},
\end{align}
%
where $\boldsymbol{p}'_j \doteq \frac{\boldsymbol{\omega}_j\boldsymbol{p}_j}{\sum_k \boldsymbol{\omega}_k\boldsymbol{p}_k}$ is the weighted probability, and $\boldsymbol{\omega}_t$ denotes the class-level weight at the target (label) position.
%
We observe that the first term in Eq.~(\ref{eq:grad1'}) satisfies the structure of the gradient of the \textit{Softmax-CrossEntropy} loss, and thus propose to eliminate the second term which messes the structure.
%
Specifically, we let
\begin{equation}
    \label{eq:constraint}
     \sum_k{{\boldsymbol{\omega}_k} {\boldsymbol{p}_k}} - \boldsymbol{\omega}_t = 0\Rightarrow \boldsymbol{\omega}_t = \frac{\sum_{j\neq t}\boldsymbol{\omega}_j \boldsymbol{p}_j}{1 - \boldsymbol{p}_t},
\end{equation}
%
where $\boldsymbol{p}_t$ is the probability of the target class.
%
Note that $\sum_j \boldsymbol{\omega}_j\mathbf{y}_j = \boldsymbol{\omega}_t$, and thus we have
\begin{equation}
    \label{eq:constraint2}
    \sum_j\boldsymbol{\omega}_j(\boldsymbol{p}_j-\mathbf{y}_j) = 0.
\end{equation}
%
This restricts the mean of $\mathbf{D}'_1$ to be zero. 
%
Therefore,  we name this constraint as the \textbf{zero-mean constraint}.
%
With this, we have
\begin{equation}
    \label{eq:d1prime}
\mathbf D_1'= \boldsymbol{\omega}_t \left( \boldsymbol{p}' - \mathbf y\right).
\end{equation}
Eq. (\ref{eq:d1prime}) indicates that $\boldsymbol \omega$ adjust the gradients in two levels, i.e., instance level and class level.
%
Namely, the scalar $\boldsymbol{\omega}_t$ acts as the instance-level weight in previous instance weighting methods~\cite{renLearningReweightExamples2018,shuMetaWeightNetLearningExplicit2019a,huLearningDataManipulation2019a,wangOptimizingDataUsage2020b}, and the $\boldsymbol \omega_j$'s are the class-level weights manipulating gradient flows by adjusting the probability from $\mathbf p$ to $\mathbf p'$. 

\subsection{Efficient Two-stage Weight Generation Embedded in Bi-level Optimization}
\label{subsec:effiBilevel}
%

In this subsection, we first illustrate the three-step bi-level optimization framework in \cite{shuMetaWeightNetLearningExplicit2019a}.
%
Furthermore, we embed a two-stage scheme in the bi-level optimization framework to efficiently obtain class-level weights, with which we manipulate gradient flows and optimize model parameters.

%
\textbf{Three-step Bi-level Optimization.} 
%
Generally, the goal of classification tasks is to obtain the optimal model parameters $\boldsymbol \theta^*$ by minimizing the average loss on $D_{train}$, denoted as $\frac{1}{N}\sum_{i=1}^N l_{train}(x_i, y_i;\boldsymbol \theta)$.
%
As an instance weighting method, \cite{shuMetaWeightNetLearningExplicit2019a} adopt a three-layer MLP parameterized by $\boldsymbol \phi$ as the weighting network and take the loss of the $i^{th}$ instance as input and output a scalar weight $\omega_i$.
%
Then $\boldsymbol \theta^*$ is optimized by minimizing the instance-level weighted training loss:
\begin{equation}
    \label{eq:metaTheta}
    \boldsymbol \theta^*(\boldsymbol \phi) = \mathop{\arg\min}_{\boldsymbol \theta} \frac{1}{N}\sum_{i=1}^N\omega_i(\boldsymbol \phi)l_{train}(x_i, y_i;\boldsymbol \theta).
\end{equation}
%
To obtain the optimal $\omega_i$, they propose to use a meta set as meta-knowledge and minimize the meta-loss to obtain $\boldsymbol \phi^*$:
\begin{equation}
    \label{eq:metaPhi}
    \boldsymbol \phi^* = \mathop{\arg\min}_{\boldsymbol \phi}\frac{1}{M}\sum_{i=1}^M l_{val}(x_i^v, y_i^v;\boldsymbol \theta^*(\boldsymbol \phi)).
\end{equation}
%
Since the optimization for $\boldsymbol \theta^*(\boldsymbol \phi)$ and $\boldsymbol \phi^*$ is nested, they adopt an online strategy to update $\boldsymbol \theta$ and $\boldsymbol \phi$ with a three-step optimization loop for efficiency.
%
Denote the two sets of parameters at the $\tau^{th}$ loop as $\boldsymbol \theta _\tau$ and $\boldsymbol \phi _\tau$ respectively, and then the three-step loop is formulated as:
\begin{enumerate}[label=\textbf{Step \arabic*}]
    \item Update $\boldsymbol \theta_{\tau-1}$ to $\hat {\boldsymbol \theta}_\tau(\boldsymbol \phi)$ via an SGD step on a mini-batch training set by Eq. (\ref{eq:metaTheta}).
    \item With $\hat{\boldsymbol \theta}_\tau (\boldsymbol \phi)$, update $\boldsymbol \phi _{\tau-1}$ to $\boldsymbol \phi _\tau$ via an SGD step on a mini-batch meta set by Eq. (\ref{eq:metaPhi}).
    \item With $\boldsymbol \phi _\tau$, update $\boldsymbol \theta_{\tau-1}$ to $\boldsymbol \theta_\tau$ via an SGD step on the same mini-batch training set by Eq. (\ref{eq:metaTheta}).
\end{enumerate}
%
Instance weights in \textbf{Step 3} are better than those in \textbf{Step 1}, and thus are used to update $\boldsymbol \theta_{\tau-1}$.
%
\begin{figure}
    \centering
    \includegraphics[width=\textwidth]{figures/flow_chart.png}
    \caption{Two-stage Weight Generation. "BP" denotes the back-propagation in \textbf{Step 2} of the bi-level optimization framework. $\mathbf g$ denotes the intermediate gradients w.r.t. $\boldsymbol \omega$. $\ominus$ denotes the minus operator. Note that $\boldsymbol \omega$ is the first-stage (instance-level) weight and $\boldsymbol \omega'$ is the second-stage (class-level) weight.}
    \label{fig:flowChart}
\end{figure}

\textbf{Two-stage Weight Generation.} 
%
To guarantee scalability, we apply the same weighting network in \cite{shuMetaWeightNetLearningExplicit2019a} to obtain weights.
%
To efficiently train $\boldsymbol \phi$ and $\boldsymbol \theta$, we also adopt the three-step bi-level optimization framework.
%
Moreover, we propose an efficient two-stage scheme embedded in \textbf{Step 1-3} to generate class-level weights.
%
This process does not introduce any extra computational cost compared to \cite{shuMetaWeightNetLearningExplicit2019a}.
%
We keep the notations of $\boldsymbol \theta_\tau$ and $\boldsymbol \phi_\tau$ unchanged.

%
The first stage is embedded in \textbf{Step 1}.
%
Explicitly, we obtain the first-stage class-level weights $\boldsymbol \omega_i = \omega_i \boldsymbol 1$, by cloning the output of the weighting network for $C$ times.
%
Then we leverage the cloned weights $\boldsymbol \omega_i$ to manipulate gradients and update $\boldsymbol \theta$ with a mini-batch of training instances:
\begin{equation}
    \label{eq:thetaHat}
    \hat {\boldsymbol\theta}_\tau \left(\boldsymbol \phi _{\tau-1}\right) \leftarrow \boldsymbol \theta_{\tau-1} - \eta_{\boldsymbol \theta} \frac{1}{n}\sum_{i=1}^{n}f_{\boldsymbol \omega _i\left(\boldsymbol \phi _{\tau-1}\right)} \left(\nabla_{\boldsymbol \theta} l_{train}(x_i, y_i;\boldsymbol \theta_{\tau-1})\right),
\end{equation}
where $n$ is the mini-batch size, $\eta_{\boldsymbol \theta}$ is the learning rate of $\boldsymbol \theta$, and $f_{\boldsymbol \omega _i\left(\boldsymbol \phi _{\tau - 1} \right)}(\cdot)$ is the gradient manipulation operation defined in Eq. (\ref{eq:newGrad}).

%
The second stage is embedded in \textbf{Step 2} and \textbf{Step 3}.
%
Specifically in \textbf{Step 2}, GDW optimizes $\boldsymbol \phi$ with a mini-batch meta set:
\begin{equation}
    \label{eq:phi}
    \boldsymbol \phi _{\tau} \leftarrow \boldsymbol \phi _{\tau-1} - \eta_{\boldsymbol \phi} \frac{1}{m} \sum_{i=1}^{m} \nabla_{\boldsymbol \phi _{\tau-1}} l_{meta}(x_i^v, y_i^v;\hat{\boldsymbol \theta} _\tau(\boldsymbol \phi_{\tau-1})),
\end{equation}
where $m$ is the mini-batch size and $\eta_{\boldsymbol \phi}$ is the learning rate of $\boldsymbol \phi$. 
%
During the back-propagation in updating $\boldsymbol \phi_\tau$, GDW generates the second-stage weights using the intermediate gradients $\mathbf g_i$ on $\boldsymbol \omega_i$. 
%
Precisely,
\begin{equation}
    \label{eq:newOmega}
    \boldsymbol \omega'_i = \boldsymbol \omega_i - \eta_{\boldsymbol\omega}  \rm{clip}(\frac{\mathbf{g_i}}{\|\mathbf{g_i}\|_1}, -c, c),
\end{equation}
where $\mathbf{g_i}$ represents $\frac{1}{m}\sum_{i=1}^m\nabla_{\boldsymbol\omega_i} l_{meta}(x_i^v, y_i^v;\hat{\boldsymbol \theta}_\tau(\boldsymbol \phi_{\tau-1}))$ and $c=0.2$ denotes the clip parameter.
%
Then we impose the zero-mean constraint proposed in Eq. (\ref{eq:constraint}) on $\boldsymbol \omega'_i$, which is later used in \textbf{Step 3} to update $\boldsymbol \theta_{\tau-1}$.
%
Note that the two-stage weight generation scheme does not introduce any extra computational cost compared to MWNet because this generation process only utilizes the intermediate gradients during the back-propagation.
%
In \textbf{Step 3}, we use $\boldsymbol \omega'_i$ to manipulate gradients and update the model parameters $\boldsymbol \theta_{\tau-1}$:
\begin{equation}
    \label{eq:newTheta}
    \boldsymbol \theta_\tau \leftarrow\boldsymbol \theta_{\tau-1} - \eta_{\boldsymbol \theta} \frac{1}{n} \sum_{i=1}^{n} f_{\boldsymbol \omega'_{i}}\left(\nabla_{\boldsymbol \theta} l_{train}(x_i, y_i;\boldsymbol \theta_{\tau-1})\right).
\end{equation}
%
The only difference between \textbf{Step 1} and \textbf{Step 3} is that we use $\boldsymbol \omega'_i$ instead of the cloned output of the weighting network $\boldsymbol \omega_i$ to optimize $\boldsymbol \theta$.
%
Since we only introduce $\boldsymbol \phi$ as extra learnable parameters, GDW can scale to large datasets.
%
We summarize GDW in Algorithm \ref{alg:GDW}.
%
Moreover, we visualize the two-stage weight generation process in Figure~\ref{fig:flowChart} for better demonstration.

%
\begin{algorithm}
\caption{Generalized Data Weighting via Class-Level Gradients Manipulation}\label{alg:GDW}
\hspace*{\algorithmicindent} \textbf{Input:} Training set: $D_{train}$, Meta set: $D_{meta}$, batch size $n,m$, \# of iterations $T$ \\
\hspace*{\algorithmicindent} Initial model parameters: $\boldsymbol \theta_0$, initial weighting network parameters: $\boldsymbol \phi_0$\\
\hspace*{\algorithmicindent} \textbf{Output:} Trained model: $\boldsymbol \theta_T$
\begin{algorithmic}[1]
\For{$\tau \leftarrow1$ \textbf{to} $T$}

\State $\{x_i, y_i\}_{i=1}^n\leftarrow$ SampleFrom$(D_{train})$
\State $\{x_i^v, y_i^v\}_{i=1}^m\leftarrow$ SampleFrom$(D_{meta})$
\State Generate $\boldsymbol \omega_i$ from $\mathcal{L}_i$ via the weighting network parameterized by $\boldsymbol \phi _{\tau-1}$
\State Manipulate gradients by Eq. (\ref{eq:newGrad}) and update $\hat{\boldsymbol \theta}_\tau$ by Eq. (\ref{eq:thetaHat})
\State Update $\boldsymbol \phi_\tau$ by Eq. (\ref{eq:phi}); 
\State Update $\boldsymbol \omega_i$ to $\boldsymbol \omega'_i$ by Eq. (\ref{eq:newOmega}) and constrain $\boldsymbol \omega'_i$ by Eq. (\ref{eq:constraint})
\State Manipulate gradients with $\boldsymbol \omega'_i$ by Eq. (\ref{eq:newGrad}) and update $\boldsymbol \theta_\tau$ by Eq. (\ref{eq:newTheta})
\EndFor
\end{algorithmic}
\end{algorithm}




\pdfoutput=1
\section{Experiments}
%
We conduct extensive experiments on classification tasks to examine the performance of GDW.
%
We compare GDW with other methods in the label noise setting and class imbalance setting in Section~\ref{sec:noise} and Section~\ref{sec:imbalance}, respectively.
%
Next, we perform experiments on the real-world dataset Clothing1M~\cite{tongxiaoLearningMassiveNoisy2015} in Section~\ref{sec:real-world}.
%
We conduct further experiments to verify the performance of GDW in the mixed setting, i.e. the coexistence of label noise and class imbalance~(see Appendix~\ref{section: Appendix_F} for details).

\subsection{Label Noise Setting}
\label{sec:noise}
%
\noindent \textbf{Setup.} Following \cite{shuMetaWeightNetLearningExplicit2019a}, we study two settings of label noise: a) Uniform noise: every instance's label uniformly flips to other class labels with probability $p$; b) Flip noise: each class randomly flips to another class with probability $p$.
%
Note that the probability $p$ represents the noise ratio.
%
We randomly select $100$ clean images per class from CIFAR10~\cite{krizhevskyLearningMultipleLayers2009} as the meta set ($1000$ images in total).
%
Similarly, we select a total of $1000$ images from CIFAR100 as its meta set.
%
We use ResNet-32~\cite{heDeepResidualLearning2016} as the classifier model.

%
\noindent \textbf{Comparison methods.} We mainly compare GDW with meta-learning methods: 1) L2RW~\cite{renLearningReweightExamples2018}, which assigns weights to instances based on gradient directions;
2) INSW~\cite{huLearningDataManipulation2019a}, which derives instance weights adaptively from the meta set;
3) MWNet~\cite{shuMetaWeightNetLearningExplicit2019a}; 4) Soft-label~\cite{vyasLearningSoftLabels2020b}, which learns a label smoothing parameter for every instance; 5) Gen-label~\cite{alganMetaSoftLabel2021a}, which generates a meta-soft-label for every instance.
%
We also compare GDW with some traditional methods: 6) BaseModel, which trains ResNet-32 on the noisy training set; 7) Fine-tuning, which uses the meta set to fine-tune the trained model in BaseModel; 8) Co-teaching~\cite{hanCoteachingRobustTraining2018a}; 9) GLC~\cite{hendrycksUsingTrustedData2018}.
%

\noindent \textbf{Training.} Most of our training settings follow \cite{shuMetaWeightNetLearningExplicit2019a} and we use the cosine learning rate decay schedule~\cite{loshchilovSGDRStochasticGradient2016} for a total of $80$ epochs for all methods.
%
See Appendix \ref{section: Appendix_C} for details.

\begin{table}
	\centering
	\caption{Test accuracy on CIFAR10 and CIFAR100 with different uniform noise ratios.}  
	\label{tab:unif_noise_table}
	\scalebox{0.86}{
	\begin{tabular}{ccccccc}
		\specialrule{\cmidrulewidth}{0pt}{0pt}
		\multirow{2}*{Dataset} & \multicolumn{3}{c}{CIFAR10} &  \multicolumn{3}{c}{CIFAR100} \\
		\cmidrule(lr){2-4}
        \cmidrule(lr){5-7}
		%\specialrule{\cmidrulewidth}{0pt}{0pt}
		 & \multicolumn{1}{c}{$0\%$} & \multicolumn{1}{c}{$40\%$}  & \multicolumn{1}{c}{$60\%$}  & \multicolumn{1}{c}{$0\%$}  & \multicolumn{1}{c}{$40\%$}  & \multicolumn{1}{c}{$60\%$}  \\
		\specialrule{\cmidrulewidth}{0pt}{0pt}
		BaseModel & $92.73 \pm 0.37$ & $84.38 \pm 0.32$ & $77.92 \pm 0.29$ & $70.42 \pm 0.54$ & $57.28 \pm 0.80$ & $46.86 \pm 1.54$ \\
		Fine-tuning  & $92.77 \pm 0.37$ & $84.73 \pm 0.47$ & $78.41 \pm 0.31$ & $70.52 \pm 0.57$ & $57.38 \pm 0.87$ & $47.06 \pm 1.47$ \\
		Co-teaching  & $91.54 \pm 0.39$ & $85.26 \pm 0.56$ & $78.90 \pm 6.64$ & $68.33 \pm 0.13$ & \uline{$59.58 \pm 0.83$} & $37.74 \pm 2.60$ \\
		GLC  & $90.85 \pm 0.22$ & $86.12 \pm 0.54$ & \uline{$81.55 \pm 0.60$} & $65.05 \pm 0.59$ & $56.99 \pm 0.82$ & $41.74 \pm 1.98$ \\
		\specialrule{\cmidrulewidth}{0pt}{0pt}
		\specialrule{\cmidrulewidth}{0pt}{0pt}
		L2RW  & $89.70 \pm 0.50$ & $84.66 \pm 1.21$ & $79.98 \pm 1.18$ & $63.40 \pm 1.31$ & $47.06 \pm 4.84$ & $36.02 \pm 2.17$ \\
		INSW  & $92.70 \pm 0.57$ & $84.88 \pm 0.64$ & $78.77 \pm 0.82$ & $70.52 \pm 0.39$ & $57.11 \pm 0.66$ & $48.00 \pm 1.16$ \\
		MWNet  & \textbf{92.95} $\pm$ \textbf{0.33} & $86.46 \pm 0.31$ & $81.14 \pm 0.94$ & \uline{$\textbf{70.64} \pm \textbf{0.31}$} & {$58.37 \pm 0.33$} & \uline{$50.21 \pm 2.98$} \\
		Soft-label  & $92.63 \pm 0.27$ & \uline{$86.52 \pm 0.10$} & $80.94 \pm 0.25$ & $70.50 \pm 0.44$ & $57.48 \pm 0.43$ & $48.18 \pm 0.89$ \\
		Gen-label  & $92.56 \pm 0.56$ & $84.68 \pm 0.57$ & $78.32 \pm 0.94$ & $70.46 \pm 0.37$ & $57.86 \pm 0.50$ & $48.08 \pm 0.98$ \\
		\textbf{GDW}  & \uline{\textbf{92.94} $\pm$ \textbf{0.15}} & \textbf{88.14} $\pm$ \textbf{0.35} & \textbf{84.11} $\pm$ \textbf{0.21} & \textbf{70.65} $\pm$ \textbf{0.52} & \textbf{59.82} $\pm$ \textbf{1.62} & \textbf{53.33} $\pm$ \textbf{3.70} \\
		\specialrule{\cmidrulewidth}{0pt}{0pt}
	\end{tabular}
	}
\end{table}

\begin{table}
	\centering
	\caption{Test accuracy on CIFAR10 and CIFAR100 with different flip noise ratios.}  
	\label{tab:flip_noise_table}
	\scalebox{0.86}{
	\begin{tabular}{ccccccc}
		\specialrule{\cmidrulewidth}{0pt}{0pt}
		\multirow{2}*{Dataset} & \multicolumn{3}{c}{CIFAR10} &  \multicolumn{3}{c}{CIFAR100} \\
		\cmidrule(lr){2-4}
        \cmidrule(lr){5-7}
		%\specialrule{\cmidrulewidth}{0pt}{0pt}
		 & \multicolumn{1}{c}{$0\%$} & \multicolumn{1}{c}{$20\%$}  & \multicolumn{1}{c}{$40\%$}  & \multicolumn{1}{c}{$0\%$}  & \multicolumn{1}{c}{$20\%$}  & \multicolumn{1}{c}{$40\%$}  \\
		\specialrule{\cmidrulewidth}{0pt}{0pt}
		BaseModel & $92.73 \pm 0.37$ &$ 90.14 \pm 0.35$  &  $81.20 \pm 0.93$ &  $70.42 \pm 0.54$ & $64.96 \pm 0.16$ & $49.83 \pm 0.82$ \\
		Fine-tuning  & $92.77 \pm 0.37$  & $90.15\pm 0.36$ & $81.53 \pm 0.96$ & $70.52 \pm 0.57$ & $65.02 \pm 0.22$ & $50.23 \pm 0.71$ \\
		Co-teaching  & $91.54 \pm 0.39$ & $89.27 \pm 0.24$ & $69.77 \pm 3.97$ & $68.33 \pm 0.13$ & $62.96 \pm 0.73$ & $42.54 \pm 1.68$ \\
		GLC  & $90.85 \pm 0.22$ & \uline{$90.22 \pm 0.13$} & \textbf{89.74} $\pm$ \textbf{0.19} & $65.05 \pm 0.59$ & $64.11 \pm 0.40$ & \textbf{63.11} $\pm$ \textbf{0.93} \\
		\specialrule{\cmidrulewidth}{0pt}{0pt}
		\specialrule{\cmidrulewidth}{0pt}{0pt}
		L2RW  & $89.70 \pm 0.50$ & $88.21 \pm 0.49$ & $82.90 \pm 1.27$ &  $63.40 \pm 1.31$ & $55.27 \pm 2.27$ & $45.41 \pm 2.53$ \\
		INSW  & $92.70 \pm 0.57$ &  $89.90 \pm 0.45$ & $80.09 \pm 2.00$ &$70.52 \pm 0.39$ & \uline{$65.32 \pm 0.27$} & $50.13 \pm 0.39$ \\
		MWNet  & \textbf{92.95} $\pm$ \textbf{0.33} &  $89.93 \pm 0.17$ & $85.55 \pm 0.82$ &\uline{$\textbf{70.64} \pm \textbf{0.31}$} & $64.72 \pm 0.68$ & $50.62 \pm 0.46$ \\
		Soft-label  & $92.63 \pm 0.27$ & $90.17 \pm 0.47$ & $85.52 \pm 0.78$ & $70.50 \pm 0.44$ & $65.20 \pm 0.45$ & $50.97 \pm 0.41$ \\
		Gen-label      & $92.56 \pm 0.56$ & $90.18 \pm 0.13$ & $80.93 \pm 1.29$ & $70.46 \pm 0.37$ & $64.94 \pm 0.53$ & $49.93 \pm 0.55$ \\
		\textbf{GDW}  & \uline{\textbf{92.94} $\pm$ \textbf{0.15}} & \textbf{91.05} $\pm$ \textbf{0.26} & \uline{$87.70 \pm 0.37$} & \textbf{70.65} $\pm$ \textbf{0.52} & \textbf{65.41} $\pm$ \textbf{0.75} & \uline{$52.44 \pm 0.79$} \\
		\specialrule{\cmidrulewidth}{0pt}{0pt}
	\end{tabular}
	}
\end{table}


\noindent \textbf{Analysis.} For all experiments, we report the mean and standard deviation over $5$ runs in Table \ref{tab:unif_noise_table} and Table \ref{tab:flip_noise_table}, where 
the best results are in \textbf{bold} and the second-best results are marked by underlines.
%
First, we can observe that GDW outperforms nearly all the competing methods in all noise settings except for the $40\%$ flip noise setting.
%
Under this setting, GLC estimates the label corruption matrix well and thus performs the best, whereas the flip noise assumption scarcely holds in real-world scenarios.
%
Note that GLC also performs much better than MWNet under the $40\%$ flip noise setting as reported in \cite{shuMetaWeightNetLearningExplicit2019a}.
%
Besides, under all noise settings, GDW has a consistent performance gain compared with MWNet, which aligns with our motivation in Figure \ref{fig:motiv}.
%
Furthermore, as the ratio increases from $40\%$ to $60\%$ in the uniform noise setting, the gap between GDW and MWNet increases from $1.68\%$ to $2.97\%$ in CIFAR10 and $1.45\%$ to $3.12\%$ in CIFAR100.
%
Even under $60\%$ uniform noise, GDW still has low test errors in both datasets and achieves more than $3\%$ gain in CIFAR10 and $6\%$ gain in CIFAR100 compared with the second-best method.
%
Last but not least, GDW outperforms Soft-label and Gen-label in all settings.
%
One possible reason is that manipulating gradient flows is a more direct way to capture class-level information than learning labels.

\begin{figure}
\centering
\begin{minipage}[t]{.5\textwidth}
  \centering
    \captionsetup{width=.95\linewidth}
    \includegraphics[width=0.8\columnwidth]{figures/clean_and_noise_w_cifar10.png} 
    \caption{Class-level target weight~($\boldsymbol{\omega}_t$) distribution on CIFAR10 under $40\%$ uniform noise.
    $\boldsymbol{\omega}_t$ of most clean instances are larger than that of most noisy instances, which means $\boldsymbol{\omega}_t$
    can differentiate between clean and noisy instances.
    }
    \label{fig:clean_and_noise_w_cifar10}
\end{minipage}%
\begin{minipage}[t]{.5\textwidth}
  \centering
    \captionsetup{width=.95\linewidth}
    \includegraphics[width=0.8\columnwidth]{figures/exp_verification.png} 
    \caption{The change of class-level weights in an iteration for a noisy instance (cat mislabeled as "dog").  MWNet downweights all gradient flows. In contrast, GDW upweights the "not bird" gradient flow for better information use.}
    \label{fig:exp_verification}
\end{minipage}
\end{figure}

In Figure \ref{fig:clean_and_noise_w_cifar10}, we show the distribution of class-level target weight~($\boldsymbol\omega_t$) on clean and noisy instances in one epoch under the CIFAR10 40\% uniform noise setting.
%
We observe that $\boldsymbol\omega_t$ of most clean instances are larger than that of most noisy instances, which indicates that $\boldsymbol\omega_t$ can distinguish between clean instances and noisy instances.
%
This is consistent with Eq. (\ref{eq:grad1'})  that $\boldsymbol\omega_t$ serves as the instance weight. 

To better understand the changing trend of non-target class-level weights, we visualize the ratio of increased weights in one epoch in Figure \ref{fig:cifar10_unif_ratio_trend} under the CIFAR10 40\% uniform noise setting.
%
Specifically, there are three categories: \textbf{n}on-\textbf{t}arget \textbf{w}eights on \textbf{c}lean instances (\textbf{$\boldsymbol\omega^c_{nt}$}), \textbf{t}rue \textbf{t}arget \textbf{w}eights on \textbf{n}oisy instances (\textbf{$\boldsymbol\omega^{n}_{tt}$}) and \textbf{n}on-\textbf{t}arget (excluding true targets) \textbf{w}eights on \textbf{n}oisy instances ($\boldsymbol\omega^n_{nt}$).
%
Formally, "target weight" means the class-level weight on the label position.
%
"true-target weight" means the class-level weight on the true label position, which are only applicable for noisy instances. "non-target weight" means the class-level weight except the label position and the true label position.
%
For example, as shown in Figure \ref{fig:motiv} where a cat is mislabeled as "dog", the corresponding meanings of the notations are as follows: 1)~$\boldsymbol\omega^n_{t}$ means $\boldsymbol\omega_{dog}$ ("dog" is the target); 2)~$\boldsymbol\omega^n_{tt}$ means $\boldsymbol\omega_{cat}$ ("cat" is the ture target); 3)~$\boldsymbol\omega^n_{nt}$ means $\boldsymbol\omega_{bird}$ ("bird" is one of the non-targets).
%
For a correctly labeled cat, the corresponding meanings are: 1)~$\boldsymbol\omega^c_{t}=\boldsymbol\omega^c_{tt}$ means $\boldsymbol\omega_{cat}$~("cat" is both the target and the ture target); 2)~$\boldsymbol\omega^c_{nt}$ means $\boldsymbol\omega_{dog}$ and $\boldsymbol\omega_{bird}$ ("dog" and "bird" are both non-targets).

%
Note that in Figure \ref{fig:motiv}, $\boldsymbol\omega^n_{tt}$ represents the importance of the "not cat" gradient flow and $\boldsymbol\omega^n_{nt}$ represents the importance of the "not bird" gradient flow.
%
If the cat image in Figure \ref{fig:motiv} is correctly labeled as "cat", then
the two non-target weights $\boldsymbol\omega^c_{nt}$ are used to represent the importance of the "not dog" and the "not bird" gradient flows, respectively.
%
In one epoch, we calculate \textbf{the ratios of} the number of increased $\boldsymbol\omega^c_{nt}$, $\boldsymbol\omega^{n}_{tt}$ and $\boldsymbol\omega^n_{nt}$ \textbf{to} the number of all corresponding weights.
%
$\boldsymbol\omega^c_{nt}$ and $\boldsymbol\omega^n_{nt}$ are expected to increase since their gradient flows contain valuable information,
whereas $\boldsymbol\omega^{n}_{tt}$ is expected to decrease because the "not cat" gradient flow contains harmful information.
%
Figure \ref{fig:cifar10_unif_ratio_trend} aligns perfectly with our expectation.
%
Note that the lines of $\boldsymbol\omega^c_{nt}$ and $\boldsymbol\omega^n_{nt}$ nearly coincide with each other and fluctuate around $65\%$.
%
This means non-target weights on clean instances and noisy instances share the same changing pattern, i.e., around $65\%$ of $\boldsymbol\omega^c_{nt}$ and $\boldsymbol\omega^n_{nt}$ increase.
%
Besides, less than $20\%$ of $\boldsymbol\omega^{n}_{tt}$ increase and thus more than $80\%$ decrease, which means the gradient flows of $\boldsymbol\omega^{n}_{tt}$ contain much harmful information.

%
In Figure \ref{fig:exp_verification}, we show the change of class-level weights in an iteration for a noisy instance, i.e., a cat image mislabeled as "dog".
%
The gradient flows of "not cat" and "dog" contain harmful information and thus are downweighted by GDW.
%
In addition, GDW upweights the valuable "not bird" gradient flow from $0.45$ to $0.63$.
%
By contrast, unable to capture class-level information, MWNet downweights all gradient flows from $0.45$ to $0.43$, which leads to information loss on the "not bird" gradient flow.

\noindent \textbf{Training without the zero-mean constraint.} We have also tried training without the zero-mean constraint in Section \ref{subsec:restOn} and got poor results~(see Appendix \ref{subsection: Appendix_B.2} for details).
%
Denote the \textbf{t}rue \textbf{t}arget as $tt$ and one of the \textbf{n}on-\textbf{t}arget labels as $nt$ ($nt\neq tt$). 
%
Note that the gradient can be unrolled as (see Appendix \ref{subsection: Appendix_B.2} for details):
\begin{equation} \label{eq2}
f_{\boldsymbol \omega}\left(\nabla_{\boldsymbol \theta} \mathcal{L}\right) = \boldsymbol{\omega}_t \sum_j\left({\boldsymbol{p}'_j}-\mathbf{y}_j\right)\frac{\partial \boldsymbol{l}_j}{\partial \boldsymbol \theta} + \left(\sum_k{{\boldsymbol {\omega}_k} {\boldsymbol{p}_k}}-\boldsymbol{\omega}_t\right) \sum_j {\boldsymbol{p}'_j}\frac{\partial \boldsymbol{l}_j}{\partial \boldsymbol \theta}.
\end{equation}

%
If $\sum_k{{\boldsymbol \omega_k} {\boldsymbol p_k}}-\boldsymbol \omega_t$ is positive and the learning rate is small enough, $\left(\sum_k{{\boldsymbol \omega_k} {\boldsymbol p_k}}-\boldsymbol\omega_t\right) \boldsymbol p'_{tt} \frac{\partial \boldsymbol l_{tt}}{\partial \boldsymbol \theta}$ contributes to the decrease of the true target logit $\boldsymbol l_{tt}$  after a gradient descent step.
If negative,  $\left(\sum_k{{\boldsymbol \omega_k} { \boldsymbol p_k}}-\boldsymbol \omega_t\right) \boldsymbol p'_{nt} \frac{\partial \boldsymbol l_{nt}}{\partial \boldsymbol \theta}$ contributes to the increase of the non-target logit $\boldsymbol l_{nt}$.
%
Therefore, without the zero-mean constraint, the second term in Eq. (\ref{eq2}) may hurt the performance of the model regardless of the sign of $\sum_k{{\boldsymbol \omega_k} {\boldsymbol p_k}}-\boldsymbol \omega_t$.
%
Similarly, training without the constraint results in poor performance in other settings.
%
Hence we omit those results in the following subsections.

\begin{figure}
\centering
\begin{minipage}[t]{.5\textwidth}
  \centering
    \includegraphics[width=0.8\columnwidth]{figures/cifar10_unif_ratio_trend.png}
    \captionsetup{width=.95\linewidth}
    \caption{Ratio trend of the number of increased $\boldsymbol\omega^c_{nt}$, $\boldsymbol\omega^{n}_{tt}$, and $\boldsymbol\omega^n_{nt}$ under the CIFAR10 40\% uniform noise setting. 
    %
    Around $65\%$ of $\boldsymbol\omega^c_{nt}$ and $\boldsymbol\omega^n_{nt}$ increase since they contain useful information.
    %
    Besides, less than $20\%$ of $\boldsymbol\omega^{n}_{tt}$ increase and thus more than $80\%$ of $\boldsymbol\omega^{n}_{tt}$ decrease since they contain harmful information.
    }
    \label{fig:cifar10_unif_ratio_trend}
\end{minipage}%
\begin{minipage}[t]{.5\textwidth}
  \centering
    \includegraphics[width=0.82\columnwidth]{figures/cifar10_imb_ratio_trend.png}
    \captionsetup{width=.95\linewidth}
    \caption{Ratio trend of the number of increased $\boldsymbol{\omega}_8$ on $C9$ instances under the Long-Tailed CIFAR10 $\mu=0.1$ setting.
    %
    Less than $10\%$ of $\boldsymbol{\omega}_8$ increase and thus more than $90\%$ decrease.
    %
    A small $\boldsymbol{\omega}_8$ strikes a balance between two kinds of information: "$C8$" and "not $C8$",
    which better handles class imbalance.
    }
    \label{fig:cifar10_imb_ratio_trend}
\end{minipage}
\end{figure}


\subsection{Class Imbalance Setting}
\label{sec:imbalance}

\begin{table}
	\centering
	\caption{Test accuracy on the long-tailed CIFAR10 and CIFAR100 with different imbalance ratios.}  
	\label{tab:class_imbalance}
	\scalebox{0.84}{
	\begin{tabular}{ccccccc}
		\specialrule{\cmidrulewidth}{0pt}{0pt}
		\multirow{2}*{Dataset} & \multicolumn{3}{c}{CIFAR10} &  \multicolumn{3}{c}{CIFAR100} \\
		\cmidrule(lr){2-4}
        \cmidrule(lr){5-7}
		%\specialrule{\cmidrulewidth}{0pt}{0pt}
		 & \multicolumn{1}{c}{$\mu=1$} & \multicolumn{1}{c}{$\mu=0.1$}  & \multicolumn{1}{c}{$\mu=0.01$}  & \multicolumn{1}{c}{$\mu=1$}  & \multicolumn{1}{c}{$\mu=0.1$}  & \multicolumn{1}{c}{$\mu=0.01$}  \\
		\specialrule{\cmidrulewidth}{0pt}{0pt}
		BaseModel & $92.73 \pm 0.37$ & $85.93 \pm 0.57$ & $69.77 \pm 1.13$ & $70.42 \pm 0.54$ & $56.25 \pm 0.49$ & {$37.79 \pm 0.82$} \\
		Fine-tuning  & $92.77 \pm 0.37$ & $82.60 \pm 0.49$ & $59.76 \pm 1.00$ & $70.52 \pm 0.57$ & $55.95 \pm 0.50$ & $37.10 \pm 0.87$ \\
		Focal  & $91.68 \pm 0.49$ & $84.57 \pm 0.83$ & $65.78 \pm 4.02$ & $68.48 \pm 0.38$ & $55.02 \pm 0.51$ & $37.43 \pm 1.00$ \\
		Balanced  & $92.80 \pm 0.47$ & $86.05 \pm 0.46$ & $63.63 \pm 3.60$ & $70.56 \pm 0.56$ & $55.02 \pm 0.80$ & $27.60 \pm 1.39$ \\
		\specialrule{\cmidrulewidth}{0pt}{0pt}
		\specialrule{\cmidrulewidth}{0pt}{0pt}
		L2RW  & $89.70 \pm 0.50$ & $79.11 \pm 3.40$ & $51.15 \pm 7.13$ & $63.40 \pm 1.31$ & $46.28 \pm 4.51$ & $25.86 \pm 5.78$ \\
	    INSW  & $92.70 \pm 0.57$ & \uline{$86.31 \pm 0.28$} & \uline{$70.27 \pm 0.24$} & $70.52 \pm 0.39$ & $55.94 \pm 0.51$ & $37.67 \pm 0.59$ \\
		MWNet            & \textbf{92.95} $\pm$ \textbf{0.33} & $86.17 \pm 0.75$ & $62.70 \pm 1.76$ & \uline{$\textbf{70.64} \pm \textbf{0.31}$} & \uline{$56.49 \pm 1.52$} & \uline{$37.83 \pm 0.86$} \\
		\textbf{GDW}  & \uline{\textbf{92.94} $\pm$ \textbf{0.15}} & \textbf{86.77} $\pm$ \textbf{0.55} & \textbf{71.31} $\pm$ \textbf{1.03} & \textbf{70.65} $\pm$ \textbf{0.52} & \textbf{56.78} $\pm$ \textbf{0.52} & \textbf{37.94} $\pm$ \textbf{1.58} \\
		\specialrule{\cmidrulewidth}{0pt}{0pt}
	\end{tabular}
	}
\end{table}

\noindent \textbf{Setup and comparison methods.}
%
The imbalance factor $\mu \in (0, 1)$ of a dataset is defined as the number of instances in the smallest class divided by that of the largest~\cite{shuMetaWeightNetLearningExplicit2019a}.
%
Long-Tailed CIFAR \cite{krizhevskyLearningMultipleLayers2009} are created
by reducing the number of training instances per class according to an exponential function $n = n_i\mu^{i/(C-1)} $, where $i$ is the class index (0-indexed) and $n_i$ is the original number of training instances.
%
Comparison methods include: 1)~L2RW~\cite{renLearningReweightExamples2018}; 2)~INSW~\cite{huLearningDataManipulation2019a}; 3)~MWNet~\cite{shuMetaWeightNetLearningExplicit2019a}; 4)~BaseModel; 5)~Fine-tuning; 6)~Balanced~\cite{cuiClassBalancedLossBased2019}; 7)~Focal~\cite{linFocalLossDense2020}.

\noindent \textbf{Analysis.} As shown in Table \ref{tab:class_imbalance}, GDW performs best in nearly all settings and exceeds MWNet by $8.6\%$ when the imbalance ratio $\mu$ is $0.01$ in CIFAR10.
%
Besides, INSW achieves competitive performance at the cost of introducing a huge amount of learnable parameters (equal to the training dataset size $N$).
%
Furthermore, we find that BaseModel achieves competitive performance, but fine-tuning on the meta set hurts the model's performance.
%
We have tried different learning rates from $10^{-7}$ to $10^{-1}$ for fine-tuning, but the results are similar.
%
One explanation is that the balanced meta set worsens the model learned from the imbalanced training set.
%
These results align with the experimental results in \cite{huLearningDataManipulation2019a} which also deals with class imbalance.

%
Denote the smallest class as $C9$ and the second smallest class as $C8$ in Long-Tailed CIFAR10 with $\mu=0.1$.
%
Recall that $\boldsymbol \omega_j$ denotes the $j^{th}$ class-level weight.
%
For all $C9$ instances in an epoch, we calculate \textbf{the ratio of} the number of increased $\boldsymbol \omega_8$ \textbf{to} the number of all $\boldsymbol \omega_8$, and then visualize the ratio trend in Figure \ref{fig:cifar10_imb_ratio_trend}.
%
Since $C9$ is the smallest class, instance weighting methods upweight both $\boldsymbol \omega_8$ and $\boldsymbol \omega_9$ on a $C9$ instance.
%
Yet in Figure \ref{fig:cifar10_imb_ratio_trend}, less than $10\%$ of $\boldsymbol \omega_8$ increase and thus more than $90\%$
decrease.
%
This can be explained as follows.
%
There are two kinds of information in the long-tailed dataset regarded to $C8$: "$C8$" and "not $C8$".
%
Since $C8$ belongs to the minority class, the dataset is biased towards the "not $C8$" information.
%
Because $\boldsymbol \omega_8$ represents the importance of "not $C8$", a smaller $\boldsymbol \omega_8$ weakens the "not $C8$" information.
%
As a result, decreased $\boldsymbol \omega_8$ achieves a balance between two kinds of information: "$C8$" and "not $C8$", thus better handling class imbalance at the class level.
%
We have conducted further experiments on imbalanced settings to verify the effectiveness of GDW and see Appendix D for details.


\subsection{Real-world Setting}
\label{sec:real-world}


\noindent \textbf{Setup and training.} The Clothing1M dataset contains one million images from fourteen classes collected from the web~\cite{tongxiaoLearningMassiveNoisy2015}. 
%
Labels are constructed from surrounding texts of images and thus contain some errors.
%
We use the ResNet-18 model pre-trained on ImageNet~\cite{dengImageNetLargescaleHierarchical2009} as the classifier.
%
The comparison methods are the same as those in the label noise setting since the main issue of Clothing1M is label noise~\cite{tongxiaoLearningMassiveNoisy2015}.
%
All methods are trained for $5$ epochs via SGD with a $0.9$ momentum, a $10^{-3}$ initial learning rate, a $10^{-3}$ weight decay, and a $128$ batchsize.
%
See Appendix \ref{section: Appendix_E} for details.

\begin{table}
  \caption{Test accuracy on Clothing1M.}
  \label{tab: Clothing1M}
  \scalebox{0.733}{
  \centering
  \begin{tabular}{c c c c c c c c c c c c}
    \toprule
    Method     & BaseModel  & Fine-tuning & Co-teaching & GLC & L2RW & INSW & MWNet & Soft-label & Gen-label & \textbf{GDW} \\
    \midrule
    Accuracy($\%$) & $65.02$& $67.68$& $68.13$ & $68.60$ & \uline{$68.80$} & $68.25$ &$68.46$ &$68.69$ &$67.64$ & \textbf{69.39}      \\
    \bottomrule
  \end{tabular}
  }
\end{table}


\noindent \textbf{Analysis.} As shown in Table \ref{tab: Clothing1M}, GDW achieves the best performance among all the comparison methods and outperforms MWNet by $0.93\%$.
%
In contrast to unsatisfying results in previous settings, L2RW performs quite well in this setting.
%
One possible explanation is that, compared with INSW and MWNet which update weights iteratively, L2RW obtains instance weights only based on current gradients.
%
As a result, L2RW can more quickly adapt to the model's state, but meanwhile suffers from unstable weights~\cite{shuMetaWeightNetLearningExplicit2019a}.
%
In previous settings, we train models from scratch, which need stable weights to stabilize training.
%
Therefore, INSW and MWNet generally achieve better performance than L2RW.
%
Whereas in this setting, we use the pre-trained ResNet-18 model which is already stable enough.
%
Thus L2RW performs better than INSW and MWNet.

\pdfoutput=1
\section{Conclusion}
Many instance weighting methods have recently been proposed to tackle label noise and class imbalance, but they cannot capture class-level information.
%
For better information use when handling the two issues, we propose GDW to generalize data weighting from instance level to class level by reweighting gradient flows.
%
Besides, to efficiently obtain class-level weights, we design a two-stage weight generation scheme which is embedded in a three-step bi-level optimization framework and leverages intermediate gradients to update class-level weights via a gradient descent step.
%
In this way, GDW achieves remarkable performance improvement in various settings.

%
The limitations of GDW are two-fold.
%
Firstly, the gradient manipulation is only applicable to single-label classification tasks.
%
When applied to multi-label tasks, the formulation of gradient manipulation need some modifications.
%
Secondly, GDW does not outperform comparison methods by a large margin in class imbalance settings despite the potential effectiveness analyzed in Section~\ref{sec:imbalance}.
%
One possible explanation is that better information utilization may not result in performance gain which also depends on various other factors.
\pdfoutput=1
\section{Acknowledgement}
We thank Prof. Hao Zhang from Tsinghua University for helpful suggestions. This research was supported in part by the MSR-Mila collaboration funding. Besides, this research was empowered in part by the computational support provided by Compute Canada (www.computecanada.ca).

\begin{thebibliography}{10}

  \bibitem{songLearningNoisyLabels2021}
  Hwanjun Song, Minseok Kim, Dongmin Park, Yooju Shin, and Jae-Gil Lee.
  \newblock Learning from {{Noisy Labels}} with {{Deep Neural Networks}}: A
    {{Survey}}.
  \newblock {\em arXiv:2007.08199 [cs, stat]}, June 2021.
  
  \bibitem{heLearningImbalancedData2009}
  Haibo He and Edwardo~A. Garcia.
  \newblock Learning from {{Imbalanced Data}}.
  \newblock {\em IEEE Transactions on Knowledge and Data Engineering},
    21(9):1263--1284, September 2009.
  
  \bibitem{elhadySystematicSurveySensor2018a}
  Nancy~E. ElHady and Julien Provost.
  \newblock A {{Systematic Survey}} on {{Sensor Failure Detection}} and
    {{Fault}}-{{Tolerance}} in {{Ambient Assisted Living}}.
  \newblock {\em Sensors}, 18(7):1991, July 2018.
  
  \bibitem{tongxiaoLearningMassiveNoisy2015}
  {Tong Xiao}, {Tian Xia}, {Yi Yang}, {Chang Huang}, and {Xiaogang Wang}.
  \newblock Learning from massive noisy labeled data for image classification.
  \newblock In {\em 2015 {{IEEE Conference}} on {{Computer Vision}} and {{Pattern
    Recognition}} ({{CVPR}})}, pages 2691--2699, {Boston, MA, USA}, June 2015.
    {IEEE}.
  
  \bibitem{alganLabelNoiseTypes2020b}
  G{\"o}rkem Algan and {\.I}lkay Ulusoy.
  \newblock Label {{Noise Types}} and {{Their Effects}} on {{Deep Learning}}.
  \newblock {\em arXiv:2003.10471 [cs]}, March 2020.
  
  \bibitem{zhuClassNoiseVs2004a}
  Xingquan Zhu and Xindong Wu.
  \newblock Class {{Noise}} vs. {{Attribute Noise}}: A {{Quantitative Study}}.
  \newblock {\em Artificial Intelligence Review}, 22(3):177--210, November 2004.
  
  \bibitem{frenayClassificationPresenceLabel2014a}
  Benoit Frenay and Michel Verleysen.
  \newblock Classification in the {{Presence}} of {{Label Noise}}: A {{Survey}}.
  \newblock {\em IEEE Transactions on Neural Networks and Learning Systems},
    25(5):845--869, May 2014.
  
  \bibitem{zhaoLongTailDistributionsUnsupervised2012a}
  Qiuye Zhao and Mitch Marcus.
  \newblock Long-{{Tail Distributions}} and {{Unsupervised Learning}} of
    {{Morphology}}.
  \newblock In {\em Proceedings of {{COLING}} 2012}, pages 3121--3136, {Mumbai,
    India}, December 2012. {The COLING 2012 Organizing Committee}.
  
  \bibitem{vanhornDevilTailsFinegrained2017a}
  Grant Van~Horn and Pietro Perona.
  \newblock The {{Devil}} is in the {{Tails}}: Fine-grained {{Classification}} in
    the {{Wild}}.
  \newblock {\em arXiv:1709.01450 [cs]}, September 2017.
  
  \bibitem{pavonAssessingImpactClassImbalanced2011a}
  Reyes Pav{\'o}n, Rosal{\'i}a Laza, Miguel {Reboiro-Jato}, and Florentino
    {Fdez-Riverola}.
  \newblock Assessing the {{Impact}} of {{Class}}-{{Imbalanced Data}} for
    {{Classifying Relevant}}/{{Irrelevant Medline Documents}}.
  \newblock In Miguel~P. Rocha, Juan M.~Corchado Rodr{\'i}guez, Florentino
    {Fdez-Riverola}, and Alfonso Valencia, editors, {\em 5th {{International
    Conference}} on {{Practical Applications}} of {{Computational Biology}} \&
    {{Bioinformatics}} ({{PACBB}} 2011)}, Advances in {{Intelligent}} and {{Soft
    Computing}}, pages 345--353, {Berlin, Heidelberg}, 2011. {Springer}.
  
  \bibitem{patelReviewClassificationImbalanced2020a}
  Harshita Patel, Dharmendra Singh~Rajput, G~Thippa~Reddy, Celestine Iwendi, Ali
    Kashif~Bashir, and Ohyun Jo.
  \newblock A review on classification of imbalanced data for wireless sensor
    networks.
  \newblock {\em International Journal of Distributed Sensor Networks},
    16(4):1550147720916404, April 2020.
  
  \bibitem{dongClassRectificationHard2017a}
  Qi~Dong, Shaogang Gong, and Xiatian Zhu.
  \newblock Class {{Rectification Hard Mining}} for {{Imbalanced Deep Learning}}.
  \newblock In {\em 2017 {{IEEE International Conference}} on {{Computer Vision}}
    ({{ICCV}})}, pages 1869--1878, {Venice}, October 2017. {IEEE}.
  
  \bibitem{cuiClassBalancedLossBased2019}
  Yin Cui, Menglin Jia, Tsung-Yi Lin, Yang Song, and Serge Belongie.
  \newblock Class-{{Balanced Loss Based}} on {{Effective Number}} of {{Samples}}.
  \newblock In {\em 2019 {{IEEE}}/{{CVF Conference}} on {{Computer Vision}} and
    {{Pattern Recognition}} ({{CVPR}})}, pages 9260--9269, June 2019.
  
  \bibitem{sinhaClassWiseDifficultyBalancedLoss2021a}
  Saptarshi Sinha, Hiroki Ohashi, and Katsuyuki Nakamura.
  \newblock Class-{{Wise Difficulty}}-{{Balanced Loss}} for {{Solving
    Class}}-{{Imbalance}}.
  \newblock In Hiroshi Ishikawa, Cheng-Lin Liu, Tomas Pajdla, and Jianbo Shi,
    editors, {\em Computer {{Vision}} \textendash{} {{ACCV}} 2020}, Lecture
    {{Notes}} in {{Computer Science}}, pages 549--565, {Cham}, 2021. {Springer
    International Publishing}.
  
  \bibitem{johnsonSurveyDeepLearning2019a}
  Justin~M. Johnson and Taghi~M. Khoshgoftaar.
  \newblock Survey on deep learning with class imbalance.
  \newblock {\em Journal of Big Data}, 6(1):27, March 2019.
  
  \bibitem{szegedyRethinkingInceptionArchitecture2016a}
  Christian Szegedy, V.~Vanhoucke, S.~Ioffe, Jonathon Shlens, and Z.~Wojna.
  \newblock Rethinking the {{Inception Architecture}} for {{Computer Vision}}.
  \newblock {\em 2016 IEEE Conference on Computer Vision and Pattern Recognition
    (CVPR)}, 2016.
  
  \bibitem{goldbergerTrainingDeepNeuralnetworks2017}
  J.~Goldberger and E.~{Ben-Reuven}.
  \newblock Training deep neural-networks using a noise adaptation layer.
  \newblock In {\em International {{Conference}} on {{Learning
    Representations}}}, 2017.
  
  \bibitem{liuClassificationNoisyLabels2014}
  Tongliang Liu and Dacheng Tao.
  \newblock Classification with {{Noisy Labels}} by {{Importance Reweighting}}.
  \newblock {\em IEEE Transactions on Pattern Analysis and Machine Intelligence},
    38, November 2014.
  
  \bibitem{hendrycksUsingTrustedData2018}
  Dan Hendrycks, Mantas Mazeika, Duncan Wilson, and Kevin Gimpel.
  \newblock Using {{Trusted Data}} to {{Train Deep Networks}} on {{Labels
    Corrupted}} by {{Severe Noise}}.
  \newblock In {\em Advances in {{Neural Information Processing Systems}}},
    volume~31. {Curran Associates, Inc.}, 2018.
  
  \bibitem{hanCoteachingRobustTraining2018a}
  Bo~Han, Quanming Yao, Xingrui Yu, Gang Niu, Miao Xu, Weihua Hu, Ivor Tsang, and
    Masashi Sugiyama.
  \newblock Co-teaching: Robust training of deep neural networks with extremely
    noisy labels.
  \newblock In {\em Advances in {{Neural Information Processing Systems}}},
    volume~31. {Curran Associates, Inc.}, 2018.
  
  \bibitem{linFocalLossDense2020}
  Tsung-Yi Lin, Priya Goyal, Ross Girshick, Kaiming He, and Piotr Doll{\'a}r.
  \newblock Focal {{Loss}} for {{Dense Object Detection}}.
  \newblock {\em IEEE Transactions on Pattern Analysis and Machine Intelligence},
    42(2):318--327, February 2020.
  
  \bibitem{renLearningReweightExamples2018}
  Mengye Ren, Wenyuan Zeng, Bin Yang, and Raquel Urtasun.
  \newblock Learning to {{Reweight Examples}} for {{Robust Deep Learning}}.
  \newblock In {\em Proceedings of the 35th {{International Conference}} on
    {{Machine Learning}}}, pages 4334--4343. {PMLR}, July 2018.
  
  \bibitem{shuMetaWeightNetLearningExplicit2019a}
  Jun Shu, Qi~Xie, Lixuan Yi, Qian Zhao, Sanping Zhou, Zongben Xu, and Deyu Meng.
  \newblock Meta-{{Weight}}-{{Net}}: Learning an {{Explicit Mapping For Sample
    Weighting}}.
  \newblock In {\em Advances in {{Neural Information Processing Systems}}},
    volume~32. {Curran Associates, Inc.}, 2019.
  
  \bibitem{huLearningDataManipulation2019a}
  Zhiting Hu, Bowen Tan, Russ~R Salakhutdinov, Tom~M Mitchell, and Eric~P Xing.
  \newblock Learning {{Data Manipulation}} for {{Augmentation}} and
    {{Weighting}}.
  \newblock In {\em Advances in {{Neural Information Processing Systems}}},
    volume~32. {Curran Associates, Inc.}, 2019.
  
  \bibitem{wangOptimizingDataUsage2020b}
  Xinyi Wang, Hieu Pham, Paul Michel, Antonios Anastasopoulos, Graham Neubig, and
    J.~Carbonell.
  \newblock Optimizing {{Data Usage}} via {{Differentiable Rewards}}.
  \newblock In {\em International {{Conference}} on {{Machine Learning}}}, 2020.
  
  \bibitem{vahdatRobustnessLabelNoise2017a}
  Arash Vahdat.
  \newblock Toward {{Robustness}} against {{Label Noise}} in {{Training Deep
    Discriminative Neural Networks}}.
  \newblock In {\em Advances in {{Neural Information Processing Systems}}},
    volume~30. {Curran Associates, Inc.}, 2017.
  
  \bibitem{elkanFoundationsCostsensitiveLearning2001}
  Charles Elkan.
  \newblock The foundations of cost-sensitive learning.
  \newblock In {\em Proceedings of the 17th International Joint Conference on
    {{Artificial}} Intelligence - {{Volume}} 2}, {{IJCAI}}'01, pages 973--978,
    {San Francisco, CA, USA}, August 2001. {Morgan Kaufmann Publishers Inc.}
  
  \bibitem{chawlaSMOTESyntheticMinority2002a}
  Nitesh~V. Chawla, Kevin~W. Bowyer, Lawrence~O. Hall, and W.~Philip Kegelmeyer.
  \newblock {{SMOTE}}: Synthetic minority over-sampling technique.
  \newblock {\em Journal of Artificial Intelligence Research}, 16(1):321--357,
    June 2002.
  
  \bibitem{anandApproachClassificationHighly2010a}
  Ashish Anand, Ganesan Pugalenthi, Gary~B. Fogel, and P.~N. Suganthan.
  \newblock An approach for classification of highly imbalanced data using
    weighting and undersampling.
  \newblock {\em Amino Acids}, 39(5):1385--1391, November 2010.
  
  \bibitem{khanCostSensitiveLearningDeep2018}
  Salman~H. Khan, Munawar Hayat, Mohammed Bennamoun, Ferdous~A. Sohel, and
    Roberto Togneri.
  \newblock Cost-{{Sensitive Learning}} of {{Deep Feature Representations From
    Imbalanced Data}}.
  \newblock {\em IEEE Transactions on Neural Networks and Learning Systems},
    29(8):3573--3587, August 2018.
  
  \bibitem{kangDecouplingRepresentationClassifier2019}
  Bingyi Kang, Saining Xie, Marcus Rohrbach, Zhicheng Yan, Albert Gordo, Jiashi
    Feng, and Yannis Kalantidis.
  \newblock Decoupling {{Representation}} and {{Classifier}} for
    {{Long}}-{{Tailed Recognition}}.
  \newblock In {\em International {{Conference}} on {{Learning
    Representations}}}, September 2019.
  
  \bibitem{liuLargeScaleLongTailedRecognition2019}
  Ziwei Liu, Zhongqi Miao, Xiaohang Zhan, Jiayun Wang, Boqing Gong, and Stella~X.
    Yu.
  \newblock Large-{{Scale Long}}-{{Tailed Recognition}} in an {{Open World}}.
  \newblock In {\em 2019 {{IEEE}}/{{CVF Conference}} on {{Computer Vision}} and
    {{Pattern Recognition}} ({{CVPR}})}, pages 2532--2541, {Long Beach, CA, USA},
    June 2019. {IEEE}.
  
  \bibitem{wangLongtailedRecognitionRouting2020}
  Xudong Wang, Long Lian, Zhongqi Miao, Ziwei Liu, and Stella Yu.
  \newblock Long-tailed {{Recognition}} by {{Routing Diverse
    Distribution}}-{{Aware Experts}}.
  \newblock In {\em International {{Conference}} on {{Learning
    Representations}}}, September 2020.
  
  \bibitem{lakeHumanlevelConceptLearning2015a}
  Brenden~M. Lake, Ruslan Salakhutdinov, and Joshua~B. Tenenbaum.
  \newblock Human-level concept learning through probabilistic program induction.
  \newblock {\em Science}, 350(6266):1332--1338, December 2015.
  
  \bibitem{franceschiBilevelProgrammingHyperparameter2018a}
  Luca Franceschi, Paolo Frasconi, Saverio Salzo, Riccardo Grazzi, and
    Massimiliano Pontil.
  \newblock Bilevel {{Programming}} for {{Hyperparameter Optimization}} and
    {{Meta}}-{{Learning}}.
  \newblock In {\em Proceedings of the 35th {{International Conference}} on
    {{Machine Learning}}}, pages 1568--1577. {PMLR}, July 2018.
  
  \bibitem{liuDARTSDifferentiableArchitecture2018}
  Hanxiao Liu, Karen Simonyan, and Yiming Yang.
  \newblock {{DARTS}}: Differentiable {{Architecture Search}}.
  \newblock In {\em International {{Conference}} on {{Learning
    Representations}}}, September 2018.
  
  \bibitem{wangLearningModelTail2017}
  Yu-Xiong Wang, Deva Ramanan, and Martial Hebert.
  \newblock Learning to {{Model}} the {{Tail}}.
  \newblock In {\em Advances in {{Neural Information Processing Systems}}},
    volume~30. {Curran Associates, Inc.}, 2017.
  
  \bibitem{jiangMentorNetLearningDataDriven2018a}
  Lu~Jiang, Zhengyuan Zhou, Thomas Leung, Li-Jia Li, and Li~{Fei-Fei}.
  \newblock {{MentorNet}}: Learning {{Data}}-{{Driven Curriculum}} for {{Very
    Deep Neural Networks}} on {{Corrupted Labels}}.
  \newblock In {\em Proceedings of the 35th {{International Conference}} on
    {{Machine Learning}}}, pages 2304--2313. {PMLR}, July 2018.
  
  \bibitem{liLearningLearnNoisy2019a}
  Junnan Li, Yongkang Wong, Qi~Zhao, and Mohan~S. Kankanhalli.
  \newblock Learning to {{Learn From Noisy Labeled Data}}.
  \newblock In {\em 2019 {{IEEE}}/{{CVF Conference}} on {{Computer Vision}} and
    {{Pattern Recognition}} ({{CVPR}})}, pages 5046--5054, {Long Beach, CA, USA},
    June 2019. {IEEE}.
  
  \bibitem{alganMetaSoftLabel2021a}
  G.~Algan and I.~Ulusoy.
  \newblock Meta {{Soft Label Generation}} for {{Noisy Labels}}.
  \newblock {\em 2020 25th International Conference on Pattern Recognition
    (ICPR)}, 2021.
  
  \bibitem{zhangDistillingEffectiveSupervision2020a}
  Zizhao Zhang, Han Zhang, Sercan~{\"O}. Arik, Honglak Lee, and Tomas Pfister.
  \newblock Distilling {{Effective Supervision From Severe Label Noise}}.
  \newblock In {\em 2020 {{IEEE}}/{{CVF Conference}} on {{Computer Vision}} and
    {{Pattern Recognition}} ({{CVPR}})}, pages 9291--9300, June 2020.
  
  \bibitem{vyasLearningSoftLabels2020b}
  Nidhi Vyas, Shreyas Saxena, and Thomas Voice.
  \newblock Learning {{Soft Labels}} via {{Meta Learning}}.
  \newblock {\em arXiv:2009.09496 [cs, stat]}, September 2020.
  
  \bibitem{saxenaDataParametersNew2019a}
  Shreyas Saxena, Oncel Tuzel, and Dennis DeCoste.
  \newblock Data {{Parameters}}: A {{New Family}} of {{Parameters}} for
    {{Learning}} a {{Differentiable Curriculum}}.
  \newblock In {\em Advances in {{Neural Information Processing Systems}}},
    volume~32. {Curran Associates, Inc.}, 2019.
  
  \bibitem{howardMobileNetsEfficientConvolutional2017a}
  Andrew~G. Howard, Menglong Zhu, Bo~Chen, Dmitry Kalenichenko, Weijun Wang,
    Tobias Weyand, Marco Andreetto, and Hartwig Adam.
  \newblock {{MobileNets}}: Efficient {{Convolutional Neural Networks}} for
    {{Mobile Vision Applications}}.
  \newblock {\em arXiv:1704.04861 [cs]}, April 2017.
  
  \bibitem{qinRethinkingSoftmaxCrossEntropy2020a}
  Zhenyue Qin, Dongwoo Kim, and Tom Gedeon.
  \newblock Rethinking {{Softmax}} with {{Cross}}-{{Entropy}}: Neural {{Network
    Classifier}} as {{Mutual Information Estimator}}.
  \newblock {\em arXiv:1911.10688 [cs, stat]}, September 2020.
  
  \bibitem{zhaoBetterAccuracyefficiencyTradeoffs2021a}
  Shuai Zhao, Liguang Zhou, Wenxiao Wang, Deng Cai, Tin~Lun Lam, and Yangsheng
    Xu.
  \newblock Towards {{Better Accuracy}}-efficiency {{Trade}}-offs: Divide and
    {{Co}}-training.
  \newblock {\em arXiv:2011.14660 [cs]}, March 2021.
  
  \bibitem{krizhevskyLearningMultipleLayers2009}
  A.~Krizhevsky.
  \newblock {\em Learning {{Multiple Layers}} of {{Features}} from {{Tiny
    Images}}}.
  \newblock {{MSc Thesis}}, University of Toronto, 2009.
  
  \bibitem{heDeepResidualLearning2016}
  Kaiming He, Xiangyu Zhang, Shaoqing Ren, and Jian Sun.
  \newblock Deep {{Residual Learning}} for {{Image Recognition}}.
  \newblock In {\em 2016 {{IEEE Conference}} on {{Computer Vision}} and {{Pattern
    Recognition}} ({{CVPR}})}, pages 770--778, June 2016.
  
  \bibitem{loshchilovSGDRStochasticGradient2016}
  Ilya Loshchilov and Frank Hutter.
  \newblock {{SGDR}}: Stochastic {{Gradient Descent}} with {{Warm Restarts}}.
  \newblock {\em International Conference on Learning Representations}, November
    2016.
  
  \bibitem{dengImageNetLargescaleHierarchical2009}
  Jia Deng, Wei Dong, Richard Socher, Li-Jia Li, Kai Li, and Li~{Fei-Fei}.
  \newblock {{ImageNet}}: A large-scale hierarchical image database.
  \newblock In {\em 2009 {{IEEE Conference}} on {{Computer Vision}} and {{Pattern
    Recognition}}}, pages 248--255, June 2009.
  
  \bibitem{kingmaAdamMethodStochastic2015}
  Diederik~P. Kingma and Jimmy Ba.
  \newblock Adam: A {{Method}} for {{Stochastic Optimization}}.
  \newblock {\em International Conference on Learning Representations}, 2015.
  
\end{thebibliography}
  
\pdfoutput=1
\appendix

\section{Derivation of D1}
\label{section: Appendx_A}
%
Denote the logit vector as $\boldsymbol x$, we have
\begin{align}
    \boldsymbol p_j =& \frac{e^{\boldsymbol x_j}}{\sum_k e^{\boldsymbol x_k}} \\
    \mathcal{L} =& -\sum_j  \boldsymbol y_j\log{\boldsymbol p_j}.
\end{align}

For the target~(label) position $t$ we have $\boldsymbol y_t=1$  and
\begin{equation}
    \frac{\partial \mathcal{L}}{\partial  \boldsymbol x_t} = \frac{\partial \mathcal{L}}{\partial \boldsymbol p_t}\frac{\partial \boldsymbol p_t}{\partial \boldsymbol x_t} = -\frac{\sum_k e^{\boldsymbol x_k}}{e^{\boldsymbol x_t}}\frac{e^{\boldsymbol x_t}\sum_k e^{\boldsymbol x_k}- e^{2\boldsymbol x_t}}{\left(\sum_k e^{\boldsymbol x_k}\right)^2} = \frac{e^{\boldsymbol x_t}}{\sum_k e^{\boldsymbol x_k}} - 1 = \boldsymbol p_t - \boldsymbol y_t.
\end{equation}

For any other position $j$~($j \neq t$), we have $\boldsymbol y_j = 0$ and 
\begin{equation}
    \frac{\partial \mathcal{L}}{\partial \boldsymbol x_j} = \frac{\partial \mathcal{L}}{\partial \boldsymbol p_t}\frac{\partial \boldsymbol p_t}{\partial \boldsymbol x_j} = -\frac{\sum_k e^{\boldsymbol x_k}}{e^{\boldsymbol x_t}}\frac{-e^{\boldsymbol x_j+\boldsymbol x_t}}{\left(\sum_k e^{\boldsymbol x_k}\right)^2} = \frac{e^{\boldsymbol x_j}}{\sum_k e^{\boldsymbol x_k}} = \boldsymbol p_j - \boldsymbol y_j.
\end{equation}

Therefore, we can conclude that $\boldsymbol D_1$ = $\boldsymbol p - \boldsymbol y$.


\section{Zero-mean Constraint on Class-level Weights}
\subsection{Derivation}
\label{subsection: Appendix_B.1}
\begin{align}
    \label{eq: ana}
    \boldsymbol{\omega}_j(\boldsymbol{p}_j-\mathbf{y}_j) =& \boldsymbol{\omega}_j\boldsymbol{p}_j-\boldsymbol{\omega}_j\mathbf{y}_j\\
    =& \left(\sum_k \boldsymbol{\omega}_k\boldsymbol{p}_k\right)\frac{\boldsymbol{\omega}_j\boldsymbol{p}_j}{\sum_k \boldsymbol{\omega}_k\boldsymbol{p}_k}-\left(\sum_k \boldsymbol{\omega}_k\mathbf{y}_k\right)\frac{\boldsymbol{\omega}_j\mathbf{y}_j}{\sum_k \boldsymbol{\omega}_k\mathbf{y}_k}\\
    &\text{If $\mathbf{y}_j = 0$, the second term of (19) becomes $0$, therefore can be rewritten as $\boldsymbol{\omega}_t\mathbf{y}_j$} \nonumber\\ 
    =& \left(\sum_k \boldsymbol{\omega}_k\boldsymbol{p}_k\right){\boldsymbol{p}'_j}-\boldsymbol{\omega}_t\mathbf{y}_j\\
    =& \boldsymbol{\omega}_t\left({\boldsymbol{p}'_j} - {\mathbf{y}_j}\right) + \left(\sum_k{{\boldsymbol{\omega}_k} {\boldsymbol{p}_k}}-\boldsymbol{\omega}_t\right){\boldsymbol{p}'_j}.\label{app:eq:grad1'}
\end{align}

\subsection{Training without Zero-mean Constraint}
\label{subsection: Appendix_B.2}
\begin{equation} \label{app:eq2}
\begin{split}
f_{\boldsymbol \omega}\left(\nabla_{\boldsymbol \theta} \mathcal{L}\right) =& f_{\boldsymbol \omega}(\sum_j\frac{\partial \mathcal{L}_i}{\partial \boldsymbol{l}_j} \frac{\partial \boldsymbol{l}_j}{\partial \boldsymbol \theta}) \\
    =& \sum_j \boldsymbol{\omega}_j(\boldsymbol{p}_j-\mathbf{y}_j)\frac{\partial \boldsymbol{l}_j}{\partial \boldsymbol \theta} \\
    =& \boldsymbol{\omega}_{t} \sum_j({\boldsymbol{p}'_j}-\mathbf{y}_j)\frac{\partial \boldsymbol{l}_j}{\partial \boldsymbol \theta} + (\sum_k{{\boldsymbol \omega_k} {\boldsymbol{p}_k}}-\boldsymbol{\omega}_{t}) \sum_j {\boldsymbol{p}'_j}\frac{\partial \boldsymbol{l}_j}{\partial \boldsymbol \theta}.
\end{split}
\end{equation}




\begin{figure}[htbp]
    \centering
    \includegraphics[width=0.8\textwidth]{figures/zero_mean.png}
    \caption{Training curve without zero-mean constraint on CIFAR10 under 40\% uniform noise.}
    \label{fig:zero_mean}
\end{figure}
  
Without zero-mean constraint, the training becomes unstable.
%
We plot the training curve of the CIFAR10 40\% uniform noise setting in Figure~\ref{fig:zero_mean}.

\section{Label Noise Training Setting}
\label{section: Appendix_C}
Following the training setting of \cite{shuMetaWeightNetLearningExplicit2019a}, the classifier network is trained with SGD with a weight decay $5$e-$4$, an initial learning rate of $1$e-$1$ and a mini-batch size of $100$ for all methods. 
%
We use the cosine learning rate decay schedule~\cite{loshchilovSGDRStochasticGradient2016} for a total of 80 epochs.
%
The MLP weighting network is trained with Adam \cite{kingmaAdamMethodStochastic2015} with a fixed learning rate 1e-3 and a weight decay 1e-4.
%
For GLC, we first train 40 epochs to estimate the label corruption matrix and then train another 40 epochs to evaluate its performance.
%
Since Co-teach uses two models, each model is trained for 40 epochs for a fair comparison.
%
We repeat every experiment 5 times with different random seeds (seed=1, 10, 100, 1000, 10000, respectively) for network initialization and label noise generation.
%
We report the average test accuracy over the last 5 epochs as the model's final performance.
%
We use one V100 GPU for all the experiments.

\section{Further Experiments on Imbalanced Setting}
\label{section: Appendix_D}
\subsection{Class-level Weight Analysis}
We conduct one more experiment under the imbalance setting to better verify the interpretability of GDW.
%
As shown in Table~\ref{tab: imb_C}, we report the ratio of the number of increased $\boldsymbol\omega_i$ after gradient update on $C_j$ instances in one epoch, where $C_0$ is the largest class and $C_9$ is the smallest class.

Note that $\boldsymbol \omega_i$ on $C_i$ contains the "is $C_i$" information in the dataset. As a result, $\boldsymbol \omega_i$ on $C_i$ should be large for small classes and small for large classes. As shown above, the ratio of increased $\boldsymbol \omega_i$ on $C_i$ (the diagonal elements) increases from $0.036$ to $0.935$ as $i$ increases from $0$ to $9$.

On the other hand, $\boldsymbol \omega_i$ on $C_j$ ($j \neq i$) contains the "not $C_i$" information in the dataset. If $i$ is a large class, $\boldsymbol \omega_i$ on $C_j$ ($j \neq i$) should be large and vice versa. For $\boldsymbol \omega_i$ ($i=0, 1, 2, 3, 4$), the ratio of increased $\boldsymbol \omega_i$ on $C_j$ ($j \neq i$) are generally larger than $0.5$, and for $\boldsymbol \omega_i$ ($i=5, 6, 7, 8, 9$), the ratio of increased $\boldsymbol \omega_i$ on $C_j$ ($j \neq i$) are generally less than $0.5$. These results align with our analysis on the interpretable information of gradient flows.

\begin{table}[htbp]
    \centering
    \caption{Ratio of increased class-level weights under the imbalance setting.}
    \label{tab: imb_C}
    \begin{tabular}{@{}ccccccccccc@{}}
    \toprule
weight/class & \(C_0\) & \(C_1\) & \(C_2\) & \(C_3\) & \(C_4\) & \(C_5\)
& \(C_6\) & \(C_7\) & \(C_8\) & \(C_9\) \\
\midrule
%\endhead
\(\boldsymbol\omega_0\) & 0.036 & 0.968 & 0.973 & 0.972 & 0.965 & 0.974 & 0.972 &
0.976 & 0.956 & 0.973 \\
\(\boldsymbol\omega_1\) & 0.887 & 0.095 & 0.912 & 0.929 & 0.907 & 0.927 & 0.911 &
0.922 & 0.910 & 0.920 \\
\(\boldsymbol\omega_2\) & 0.848 & 0.844 & 0.141 & 0.839 & 0.822 & 0.845 & 0.818 &
0.847 & 0.829 & 0.802 \\
\(\boldsymbol\omega_3\) & 0.585 & 0.608 & 0.552 & 0.405 & 0.569 & 0.541 & 0.561 &
0.559 & 0.617 & 0.596 \\
\(\boldsymbol\omega_4\) & 0.474 & 0.521 & 0.420 & 0.460 & 0.509 & 0.455 & 0.456 &
0.482 & 0.467 & 0.512 \\
\(\boldsymbol\omega_5\) & 0.291 & 0.261 & 0.288 & 0.252 & 0.309 & 0.701 & 0.303 &
0.267 & 0.297 & 0.257 \\
\(\boldsymbol\omega_6\) & 0.199 & 0.189 & 0.169 & 0.198 & 0.196 & 0.222 & 0.778 &
0.195 & 0.207 & 0.182 \\
\(\boldsymbol\omega_7\) & 0.117 & 0.117 & 0.105 & 0.084 & 0.115 & 0.079 & 0.126 &
0.920 & 0.133 & 0.090 \\
\(\boldsymbol\omega_8\) & 0.115 & 0.124 & 0.178 & 0.185 & 0.184 & 0.174 & 0.191 &
0.181 & 0.862 & 0.137 \\
\(\boldsymbol\omega_9\) & 0.043 & 0.050 & 0.064 & 0.061 & 0.074 & 0.062 & 0.097 &
0.069 & 0.040 & 0.935 \\
\bottomrule
    \end{tabular}
\end{table}
\subsection{Experiments on Places-LT}

\begin{table}[htbp]
    \centering
    \caption{Test accuracy on Places-LT.}
    \label{tab:mix_placeLT}
    \begin{tabular}{@{}ccccc@{}}
\toprule
Method & L2RW & INSW & MW-Net & \textbf{GDW} \\
\midrule
%\endhead
Accuracy (\%) & \(15.08\) & \(17.80\) & \(18.08\) & \textbf{19.17} \\
\bottomrule
    \end{tabular}
    
\end{table}
We have conducted experiments on the Places-LT dataset \cite{liuLargeScaleLongTailedRecognition2019} and compared GDW with other meta-learning-based methods.
%
For all methods, the weight decay is set to $0.001$ and the batchsize is set to $64$. We adopt a $0.01$ initial learning rate and a cosine learning rate decay policy for $10$ epochs. The weight decay is set to $0.001$. The backbone network is ResNet18 and we use the ImageNet pre-trained model for initialization.

As shown in Table~\ref{tab:mix_placeLT}, GDW achieves the best performance among all the comparison methods and outperforms MWNet by $1.09\%$. This improvement is larger than that of CIFAR100. The reason is that GDW can manipulate class-level information and thus performs better on the dataset with a larger number of classes ($365$ in Places-LT and $100$ in CIFAR100). Besides, we can observe that L2RW performs the worst and the reason may be that L2RW suffers from unstable weights~\cite{shuMetaWeightNetLearningExplicit2019a}.

\section{Real-world Training Setting}
\label{section: Appendix_E}
Similar to \cite{shuMetaWeightNetLearningExplicit2019a}, we use the $7$k validation set as the meta set and the origin test set to evaluate the classifier's final performance.
%
For GLC, we first train $2$ epochs to estimate the label corruption matrix and then train another $3$ epochs to evaluate its performance.
%
Since Co-teach uses two models, each model is trained for $3$ epochs for a fair comparison.

\section{Experiments on Mixed Setting}
\label{section: Appendix_F}
We conduct further experiments to verify the performance of GDW in the mixed setting, i.e. the coexistence of label noise and class imbalance.
%
Specifically, we compare GDW with the mostly second-best method MW-Net \cite{shuMetaWeightNetLearningExplicit2019a} under the mixed setting of uniform noise and class imbalance on CIFAR10 and CIFAR100. 
%
As shown in Table~\ref{tab:mix}, GDW demonstrates great performance gain over MW-Net, which means GDW can simultaneously better tackle both problems. 

\begin{table}[htbp]
    \centering
    \caption{Test accuracy under mixed settings.}
    \label{tab:mix}
    \begin{tabular}{@{}ccccc@{}}
    \toprule
    Dataset & Noise Ratio & Imb Factor & MW-Net & GDW \\
    \midrule
    CIFAR10 & \(0.40\) & \(0.10\) & \(71.54\) & \(76.30\) \\
    CIFAR10 &\(0.60\) &\(0.10\) & \(61.62\) & \(70.24\) \\
    CIFAR10 & \(0.40\) & \(0.01\) & \(48.04\) & \(48.53\) \\
    CIFAR10 & \(0.60\) & \(0.01\) & \(39.51\) & \(40.07\) \\
    CIFAR100 &\(0.40\) &\(0.10\) & \(36.10\) & \(38.20\) \\
    CIFAR100 &\(0.60\) &\(0.10\) & \(24.80\) & \(25.40\) \\
    CIFAR100 &\(0.40\)& \(0.01\) & \(21.26\) & \(22.07\) \\
    CIFAR100 &\(0.60\) &\(0.01\) & \(12.75\) & \(14.15\) \\
    \bottomrule
    \end{tabular}
\end{table}
\pdfoutput=1
\section{Checklist}

\begin{enumerate}

\item For all authors...
\begin{enumerate}
  \item Do the main claims made in the abstract and introduction accurately reflect the paper's contributions and scope?
    \answerYes{}
  \item Did you describe the limitations of your work?
    \answerYes{The proposed method can only be applied on classification tasks.}
  \item Did you discuss any potential negative societal impacts of your work?
    \answerNA{}
  \item Have you read the ethics review guidelines and ensured that your paper conforms to them?
    \answerYes{}
\end{enumerate}

\item If you are including theoretical results...
\begin{enumerate}
  \item Did you state the full set of assumptions of all theoretical results?
    \answerYes{See Section~\ref{method}.}
	\item Did you include complete proofs of all theoretical results?
    \answerYes{See Section~\ref{method}.}
\end{enumerate}

\item If you ran experiments...
\begin{enumerate}
  \item Did you include the code, data, and instructions needed to reproduce the main experimental results (either in the supplemental material or as a URL)?
    \answerYes{We only use public datasets and the code is in the supplementary materials.}
  \item Did you specify all the training details (e.g., data splits, hyperparameters, how they were chosen)?
    \answerYes{Most of our settings follow \cite{shuMetaWeightNetLearningExplicit2019a} and other details are in the appendix.}
	\item Did you report error bars (e.g., with respect to the random seed after running experiments multiple times)?
    \answerYes{We repeat all experiments on CIFAR10 and CIFAR100 with five different seeds and the mean and standard deviation are reported. For the Clothing1M dataset, we only run one experiment due to limited resources.}
	\item Did you include the total amount of compute and the type of resources used (e.g., type of GPUs, internal cluster, or cloud provider)?
    \answerYes{We use one V100 GPU. See the appendix for details.}
\end{enumerate}

\item If you are using existing assets (e.g., code, data, models) or curating/releasing new assets...
\begin{enumerate}
  \item If your work uses existing assets, did you cite the creators?
    \answerYes{For dataset, we cite the papers of CIFAR datasets and the Clothing1M dataset. For code, we cite \cite{heDeepResidualLearning2016}.}
  \item Did you mention the license of the assets?
    \answerNA{}
  \item Did you include any new assets either in the supplemental material or as a URL?
    \answerNA{}
  \item Did you discuss whether and how consent was obtained from people whose data you're using/curating?
    \answerNA{}
  \item Did you discuss whether the data you are using/curating contains personally identifiable information or offensive content?
    \answerNA{}
\end{enumerate}

\item If you used crowdsourcing or conducted research with human subjects...
\begin{enumerate}
  \item Did you include the full text of instructions given to participants and screenshots, if applicable?
    \answerNA{}
  \item Did you describe any potential participant risks, with links to Institutional Review Board (IRB) approvals, if applicable?
    \answerNA{}
  \item Did you include the estimated hourly wage paid to participants and the total amount spent on participant compensation?
    \answerNA{}
\end{enumerate}

\end{enumerate}

\end{document}
