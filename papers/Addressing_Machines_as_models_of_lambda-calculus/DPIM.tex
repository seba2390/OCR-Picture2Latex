\documentclass[hidelinks]{lmcs} %%% last changed 2014-08-20
\usepackage[utf8]{inputenc}
\pdfoutput=1

\usepackage{silence}
\WarningFilter{latex}{You have requested package `xypic'}


% LMCS Layouting Macros
\usepackage{lastpage}
\lmcsdoi{18}{3}{10}
\lmcsheading{}{\pageref{LastPage}}{}{}%
{Jul.~02,~2021}{Jul.~29,~2022}{}


%% mandatory lists of keywords
\keywords{Addressing machines, \lam-calculus, combinatory algebras, \lam-models.}

%% read in additional TeX-packages or personal macros here:
%% e.g. \usepackage{tikz}
\usepackage{amssymb}
\usepackage{hyperref}
\usepackage{stmaryrd}
\usepackage{pifont}
\usepackage{xcolor}
\usepackage{mathtools}
\usepackage{proof}
\usepackage{xypic}
\usepackage{booktabs}
%% The amsthm package provides extended theorem environments
\usepackage{amsthm}
%TODO
\newcommand{\todo}[1]{{\color{red}{\bf [TODO]:~{#1}}}}

%THEOREMS
\newtheorem{theorem}{Theorem}
\newtheorem{corollary}{Corollary}
\newtheorem{lemma}{Lemma}
\newtheorem{proposition}{Proposition}
\newtheorem{problem}{Problem}
\newtheorem{definition}{Definition}
\newtheorem{remark}{Remark}
\newtheorem{example}{Example}
\newtheorem{assumption}{Assumption}

%HANS' CONVENIENCES
\newcommand{\define}[1]{\textit{#1}}
\newcommand{\join}{\vee}
\newcommand{\meet}{\wedge}
\newcommand{\bigjoin}{\bigvee}
\newcommand{\bigmeet}{\bigwedge}
\newcommand{\jointimes}{\boxplus}
\newcommand{\meettimes}{\boxplus'}
\newcommand{\bigjoinplus}{\bigjoin}
\newcommand{\bigmeetplus}{\bigmeet}
\newcommand{\joinplus}{\join}
\newcommand{\meetplus}{\meet}
\newcommand{\lattice}[1]{\mathbf{#1}}
\newcommand{\semimod}{\mathcal{S}}
\newcommand{\graph}{\mathcal{G}}
\newcommand{\nodes}{\mathcal{V}}
\newcommand{\agents}{\{1,2,\dots,N\}}
\newcommand{\edges}{\mathcal{E}}
\newcommand{\neighbors}{\mathcal{N}}
\newcommand{\Weights}{\mathcal{A}}
\renewcommand{\leq}{\leqslant}
\renewcommand{\geq}{\geqslant}
\renewcommand{\preceq}{\preccurlyeq}
\renewcommand{\succeq}{\succcurlyeq}
\newcommand{\Rmax}{\mathbb{R}_{\mathrm{max}}}
\newcommand{\Rmin}{\mathbb{R}_{\mathrm{min}}}
\newcommand{\Rext}{\overline{\mathbb{R}}}
\newcommand{\R}{\mathbb{R}}
\newcommand{\N}{\mathbb{N}}
\newcommand{\A}{\mathbf{A}}
\newcommand{\B}{\mathbf{B}}
\newcommand{\x}{\mathbf{x}}
\newcommand{\e}{\mathbf{e}}
\newcommand{\X}{\mathbf{X}}
\newcommand{\W}{\mathbf{W}}
\newcommand{\weights}{\mathcal{W}}
\newcommand{\alternatives}{\mathcal{X}}
\newcommand{\xsol}{\bar{\mathbf{x}}}
\newcommand{\y}{\mathbf{y}}
\newcommand{\Y}{\mathbf{Y}}
\newcommand{\z}{\mathbf{z}}
\newcommand{\Z}{\mathbf{Z}}
\renewcommand{\a}{\mathbf{a}}
\renewcommand{\b}{\mathbf{b}}
\newcommand{\I}{\mathbf{I}}
\DeclareMathOperator{\supp}{supp}
\newcommand{\Par}[2]{\mathcal{P}_{{#1} \to {#2}}}
\newcommand{\Laplacian}{\mathcal{L}}
\newcommand{\F}{\mathcal{F}}
\newcommand{\inv}[1]{{#1}^{\sharp}}
\newcommand{\energy}{Q}
\newcommand{\err}{\mathrm{err}}
\newcommand{\argmin}{\mathrm{argmin}}
\newcommand{\argmax}{\mathrm{argmax}}
%% define non-standard environments BEYOND the ones already supplied
%% here, for example
\theoremstyle{plain}\newtheorem{satz}[thm]{Satz} %\crefname{satz}{Satz}{S\"atze}
%% Do NOT replace the proclamation environments lready provided by
%% your own.

\hyphenation{
stra-te-gy
ana-lo-gous-ly
ele-men-ta-ry
}

\def\bsub{\begin{enumerate}[(i)]}
\def\esub{\end{enumerate}}
\definecolor{red}{rgb}{1, 0, 0}%{0.0, 0.27, 0.13}
%\newcommand{\hr}[1]{{\color{red}#1}}


%% due to the dependence on amsart.cls, \begin{document} has to occur
%% BEFORE the title and author information:

\begin{document}

\title[Instructions]{Addressing Machines as Models of \lam-Calculus}
%\titlecomment{{\lsuper*}OPTIONAL comment concerning the title, \eg,  if a variant or an extended abstract of the paper has appeared elsewhere.}

\author[G.~Della Penna]{Giuseppe Della Penna\rsuper{a}}	%required
\address{Dep.\ of Information Engineering, Computer Science and Mathematics, University of L'Aquila, Italy}	%required
\email{giuseppe.dellapenna@univaq.it}  %optional
%\thanks{thanks 1, optional.}	%optional

\author[B.~Intrigila]{Benedetto Intrigila\rsuper{b}}	%optional
\address{Dipartimento di Ingegneria dell'Impresa, University of Rome ``Tor Vergata'', Italy}	%optional
\email{intrigil@mat.uniroma2.it}  %optional
%\thanks{thanks 2, optional.}	%optional

\author[G.~Manzonetto]{Giulio Manzonetto\lmcsorcid{0000-0003-1448-9014}\rsuper{c}}	%optional chktex 8
\address{Univ. Paris 13, Sorbonne Paris Cit\'e, LIPN, UMR 7030, CNRS, F-93430 Villetaneuse, France.}	%optional
\email{manzonetto@univ-paris13.fr}  %optional
%\thanks{thanks 3, optional.}	%optional

%% etc.

%% required for running head on odd and even pages, use suitable
%% abbreviations in case of long titles and many authors:

%%%%%%%%%%%%%%%%%%%%%%%%%%%%%%%%%%%%%%%%%%%%%%%%%%%%%%%%%%%%%%%%%%%%%%%%%%%

%% the abstract has to PRECEDE the command \maketitle:
%% be sure not to issue the \maketitle command twice!

\begin{abstract}
  \noindent Turing machines and register machines have been used for decades in theoretical computer science as abstract models of computation.
Also the \lam-calculus has played a central role in this domain as it allows to focus on the notion of functional computation, based on the substitution mechanism, while abstracting away from implementation details.
The present article starts from the observation that the equivalence between these formalisms is based on the Church-Turing Thesis rather than an actual encoding of \lam-terms into Turing (or register) machines. The reason is that these machines are not well-suited for modelling \lam-calculus programs.

We study a class of abstract machines that we call \emph{addressing machine} since they are only able to manipulate memory addresses of other machines. The operations performed by these machines are very elementary: load an address in a register, apply a machine to another one via their addresses, and call the address of another machine. We endow addressing machines with an operational semantics based on leftmost reduction and study their behaviour. The set of addresses of these machines can be easily turned into a combinatory algebra.
In order to obtain a model of the full untyped \lam-calculus, we need to introduce a rule that bares similarities with the $\omega$-rule and the rule $\zeta_\beta$ from combinatory logic.
\end{abstract}

\maketitle

%% start the paper here:
\section*{Introduction}
% !TEX root = ../arxiv.tex

Unsupervised domain adaptation (UDA) is a variant of semi-supervised learning \cite{blum1998combining}, where the available unlabelled data comes from a different distribution than the annotated dataset \cite{Ben-DavidBCP06}.
A case in point is to exploit synthetic data, where annotation is more accessible compared to the costly labelling of real-world images \cite{RichterVRK16,RosSMVL16}.
Along with some success in addressing UDA for semantic segmentation \cite{TsaiHSS0C18,VuJBCP19,0001S20,ZouYKW18}, the developed methods are growing increasingly sophisticated and often combine style transfer networks, adversarial training or network ensembles \cite{KimB20a,LiYV19,TsaiSSC19,Yang_2020_ECCV}.
This increase in model complexity impedes reproducibility, potentially slowing further progress.

In this work, we propose a UDA framework reaching state-of-the-art segmentation accuracy (measured by the Intersection-over-Union, IoU) without incurring substantial training efforts.
Toward this goal, we adopt a simple semi-supervised approach, \emph{self-training} \cite{ChenWB11,lee2013pseudo,ZouYKW18}, used in recent works only in conjunction with adversarial training or network ensembles \cite{ChoiKK19,KimB20a,Mei_2020_ECCV,Wang_2020_ECCV,0001S20,Zheng_2020_IJCV,ZhengY20}.
By contrast, we use self-training \emph{standalone}.
Compared to previous self-training methods \cite{ChenLCCCZAS20,Li_2020_ECCV,subhani2020learning,ZouYKW18,ZouYLKW19}, our approach also sidesteps the inconvenience of multiple training rounds, as they often require expert intervention between consecutive rounds.
We train our model using co-evolving pseudo labels end-to-end without such need.

\begin{figure}[t]%
    \centering
    \def\svgwidth{\linewidth}
    \input{figures/preview/bars.pdf_tex}
    \caption{\textbf{Results preview.} Unlike much recent work that combines multiple training paradigms, such as adversarial training and style transfer, our approach retains the modest single-round training complexity of self-training, yet improves the state of the art for adapting semantic segmentation by a significant margin.}
    \label{fig:preview}
\end{figure}

Our method leverages the ubiquitous \emph{data augmentation} techniques from fully supervised learning \cite{deeplabv3plus2018,ZhaoSQWJ17}: photometric jitter, flipping and multi-scale cropping.
We enforce \emph{consistency} of the semantic maps produced by the model across these image perturbations.
The following assumption formalises the key premise:

\myparagraph{Assumption 1.}
Let $f: \mathcal{I} \rightarrow \mathcal{M}$ represent a pixelwise mapping from images $\mathcal{I}$ to semantic output $\mathcal{M}$.
Denote $\rho_{\bm{\epsilon}}: \mathcal{I} \rightarrow \mathcal{I}$ a photometric image transform and, similarly, $\tau_{\bm{\epsilon}'}: \mathcal{I} \rightarrow \mathcal{I}$ a spatial similarity transformation, where $\bm{\epsilon},\bm{\epsilon}'\sim p(\cdot)$ are control variables following some pre-defined density (\eg, $p \equiv \mathcal{N}(0, 1)$).
Then, for any image $I \in \mathcal{I}$, $f$ is \emph{invariant} under $\rho_{\bm{\epsilon}}$ and \emph{equivariant} under $\tau_{\bm{\epsilon}'}$, \ie~$f(\rho_{\bm{\epsilon}}(I)) = f(I)$ and $f(\tau_{\bm{\epsilon}'}(I)) = \tau_{\bm{\epsilon}'}(f(I))$.

\smallskip
\noindent Next, we introduce a training framework using a \emph{momentum network} -- a slowly advancing copy of the original model.
The momentum network provides stable, yet recent targets for model updates, as opposed to the fixed supervision in model distillation \cite{Chen0G18,Zheng_2020_IJCV,ZhengY20}.
We also re-visit the problem of long-tail recognition in the context of generating pseudo labels for self-supervision.
In particular, we maintain an \emph{exponentially moving class prior} used to discount the confidence thresholds for those classes with few samples and increase their relative contribution to the training loss.
Our framework is simple to train, adds moderate computational overhead compared to a fully supervised setup, yet sets a new state of the art on established benchmarks (\cf \cref{fig:preview}).

\section{Preliminaries}\label{sec:pre}
%\documentclass[main]{subfiles}

\begin{document}

\section{Preliminaries}
\label{sec:preliminaries}
%\paragraph{Notation} 
\noindent
For $n \in \N$, we denote $[n] := \{1,\ldots,n\}$ and the vector with all ones as $\1_n \in \R^n$.
%\todo{Any vector $x \in \R^n$ is a column vector and its transpose is denoted by $x^T$. The entries of $x \in \R^n$ will be $x_1, \ldots, x_n$. Inequalities like $x \geq 0$ abbreviate the statement $\forall i \in [n] \, : \, x_i \geq 0$. For $i \in [n]$, we will use $e_i \in \R^n$ to denote the $i$-th standard basis vector (with a $1$ in its $i$-th entry and $0$'s anywhere else).} 
 
%\\
%\todo{When working with sets $\{S^i\}_{i=1}^N$, we denote $\displaystyle \bigtimes_{i=1}^N S^i := S^1 \times \ldots \times S^N$.} \\


\subsection{Multiplayer Games} 
A multiplayer game $G$ specifies (a) the number of players $N \in \N, N \geq 2,$ (b) a set of pure strategies $S^i = [m_i]$ for each player~$i$ where $m_i \in \N, m_i \geq 2,$ and (c) the utility payoffs for each player~$i$ given as a function $u_i: S = S^1 \times \ldots \times S^N \longrightarrow \R$. Throughout this paper, all multiplayer games considered shall have the same number of players $N$ and the same strategy sets $S^1, \ldots, S^N$. Hence, any game $G$ will be determined by its utility functions $\{u_i\}_{i \in [N]}$. The players choose their strategies simultaneously and they cannot communicate with each other. A utility function $u_i$ can be summarized by its pure strategy outcomes for player~$i$, captured as an $N$-dimensional array $\big\{ u_i(\ks) \big\}_{\ks \in S}$.

\begin{ex}
$2$-player games are better known as bimatrix games because their $2$-dimensional payoff arrays in become matrices $A,B \in \R^{m \times n}$.
\end{ex}

As usual, we allow the players to randomize over their pure strategies. Then, player~$i$'s strategy space extends to the set of probability distributions over $S^i$. We identify this set with $\Delta(S^i) := \, \Big\{ s^i = (s_k^i)_k \, \in \R^{m_i} \, \Big| \, s_k^i \geq 0 \, \, \forall k \in [m_i] \, \, \text{and} \, \sum_{k \in [m_i]} s_k^i = 1 \Big\}$ and refer to the probability distributions as mixed strategies. A tuple $\strats = (s^1, \ldots, s^N) \in \Delta(S^1) \times \ldots \times \Delta(S^N) =: \Delta(S)$ of mixed strategies is called a strategy profile in $G$\footnote{Note that in our notation, $\Delta(S)$ is not a simplex of higher dimensions itself but only the product of $N$ simplices.}. The utility payoff of player~$i$ for the strategy profile $\strats$ is defined as the player's utility payoff in expectation 
\[ u_i(\strats) := \sum_{\ks \in S} s_{k_1}^1 \cdot \ldots \cdot s_{k_N}^N \cdot u_i(\ks) \, .\]
The goal of each player is to maximize her utility.


We will abbreviate with $S^{-i}$ the set that consists of all possible pure strategy choices $\ks_{-i} = (k_1, \ldots, k_{i-1},k_{i+1}, \ldots, k_N)$ of the opponent players (resp. $\Delta(S^{-i})$ for the set of mixed strategy choices $\strats^{-i} = (s^1, \ldots, s^{i-1},s^{i+1}, \ldots, s^N)$). We will also often use $u_i(k_i,\ks_{-i})$ instead of $u_i(\ks)$ to stress how player~$i$ can only influence her own strategy when it comes to her payoff (resp. $u_i(s^i,\strats^{-i})$ instead of $u_i(\strats)$).
\begin{defn}
The best response set of player~$i$ to the opponents' strategy choices $\strats^{-i}$ is defined as $\BR_{u_i}(\strats^{-i}) :=  \argmax_{t^i \in \Delta(S^i)} \big\{ \, u_i(t^i,\strats^{-i}) \, \big\}$. 
\end{defn} 
Best response strategies capture the idea of optimal play against the other player's strategy choices. The most popular equilibrium concept in non-cooperative games is based on best responses.
\begin{defn}
A strategy profile $\strats \in \Delta(S)$ to a game $G = \{u_i\}_{i \in [N]}$ is called a \NE{} if for all player~$i \in [N]$ we have $s^i \in \BR_{u_i}(\strats^{-i})$.
\end{defn}
\noindent
By \cite{Nash48}, any multiplayer game $G$ admits at least one \NE{}.

\subsection{Positive Affine Transformations} 

The following lemma is a well-known result for $2$-player games\footnote{ See \cite[Lemma 2.1]{heyman}, \cite[Theorem 5.35]{maschler_solan_zamir_2013}, \cite[Chapter 3]{harsanyi1988general} or \cite[Proposition 3.1]{DynGT}.}:
\begin{lemma}
\label{PAT preserves lemma}
Let $(A,B)$ be a $m \times n$ bimatrix game and take arbitrary $\alpha, \beta >0$ and $c \in \R^n, d \in \R^m$. Define $A' = \alpha A + \1_m c^T$ and $B' = \beta B + d \1_n^T$.

Then the game $(A', B')$ has the same best response sets as the game $(A,B)$. Consequently, both games have the same \NE{} set.
\end{lemma}
The intuition behind this lemma is as follows:
player~$1$ wants to maximize her utility given the strategy that player~$2$ chose. A positive rescaling of $u_1$ will change the utility payoffs but will not change the sets of best response strategies. The same holds true if we add utility payoffs to $u_1$ that are only dependent on the strategy choice $s^2$ of her opponent. In the notation of bimatrix games, this intuition yields that the transformation $A \mapsto \alpha A + \1_m c^T$ does not affect the best response sets of player~$1$. The analogous result holds for player~$2$ and the transformation $B \mapsto \beta B + d \1_n^T$.

Let us generalize PATs to multiplayer games.
\begin{defn}
\label{multiplayer PAT defn}
A positive affine transformation (PAT) specifies for each player~$i$ a scaling parameter $\alpha^i \in \R, \alpha^i >0,$ and translation constants $C^i := ( c_{\ks_{-i}})_{\ks_{-i} \in S^{-i}}$ for each choice of pure strategies from the opponents. 
The PAT $H_{\textnormal{PAT}} = \big\{ \alpha^i, C^i \big\}_{i \in [N]}$ applied to an input game $G = \{u_i\}_{i \in [N]}$ returns the transformed game $H_{\textnormal{PAT}}(G) = \{u_i'\}_{i \in [N]}$ in which (only) the utility functions changed to
\begin{align}
\label{PAT transformed utilities}
\begin{aligned}
u_i' : S &\longrightarrow \R \\
\ks &\longmapsto \alpha_i \cdot u_i(\ks) + c_{\ks_{-i}}^i \, .
\end{aligned}
\end{align}
\end{defn}
We could not find multiplayer PATs defined in the literature, so we came up with the natural generalization above. As shown in Section \ref{sec:bimatrix games}, they indeed generalize the 2-player PATs from Lemma~\ref{PAT preserves lemma} to multiplayer settings. Moreover, multiplayer PATs also preserve the best response sets and \NE{} set.
\begin{lemma}
\label{multiplayer PAT preserves}
Take a PAT $H_{\textnormal{PAT}} = \big\{ \alpha^i, C^i \big\}_{i \in [N]}$ and any game $G = \{u_i\}_{i \in [N]}$. Then, the transformed game $H_{\textnormal{PAT}}(G) = \{u_i'\}_{i \in [N]}$ has the same best response sets as the input game $G$. Consequently, $H_{\textnormal{PAT}}(G)$ also has the same \NE{} set as $G$.
\end{lemma}
\begin{proof}
See \ref{sec:helpinglemmas}.
\end{proof}
PATs have found much success as a tool for simplifying an input game precisely because of this property. We want to investigate which other game transformations also preserve the best response sets or the \NE{} set. If we found more of these transformations, we could use them to, e.g., further increase the class of efficiently solvable games.

\subsection{Game Transformations}

There are various ways in which we could define the concept of a game transformation. Section~\ref{literature review} gives an overview of some definitions from the literature that are useful for different purposes. A key component of PATs are that they operate player-wise and strategy-wise, that is, they do not change the player set nor the players' strategy sets. This allows for a direct comparison of the strategic structure between a game and its PAT-transform. We argue that this is a natural desideratum for a definition of more general game transformation.

\begin{defn}
\label{def game trafo}
A game transformation $H = \{H^i\}_{i \in [N]}$ specifies for each player~$i$ a collection of functions $H^i := \Big\{ h_{\ks}^i : \R \longrightarrow \R \Big\}_{\ks \in S}$, indexed by the different pure strategy profiles $\ks$. \\
The transformation $H$ can then be applied to any $N$-player game $G = \{u_i\}_{i \in [N]}$ to construct the transformed game $H(G) = \{H^i(u_i)\}_{i \in [N]}$ where 
\begin{align}
\label{transformed pure utilities evaluation}
    H^i(u_i) : S \to \R, \quad \ks \mapsto h_{\ks}^i \big( u_i(\ks) \big) \, .
\end{align}
\end{defn}
Observe that the utility payoff of player~$i$ in the transformed game $H(G)$ from the pure strategy outcome $\ks$ is only a function of the utility payoff from \textit{that same} player in \textit{that same} pure strategy outcome of the input game~$G$.

We extend the utility functions $H^i(u_i)$ to mixed strategy profiles $\strats \in \Delta(S)$ as usual through $H^i(u_i)(\strats) := \sum_{\ks \in S} s_{k_1}^1 \cdot \ldots \cdot s_{k_N}^N \cdot h_{\ks}^i \big( u_i(\ks) \big)$. To simplify future notation, we will often use $h_{k_i,\ks_{-i}}^i$ to refer to $h_{\ks}^i$.

\begin{rem}
A multiplayer positive affine transformation $H_{\textnormal{PAT}} = \big\{ \alpha^i, C^i \big\}_{i \in [N]}$ makes a game transformation $H = \{H^i\}_{i \in [N]}$ in the above sense by setting $h_{\ks}^i : \, \, \R \to \R$, $z \mapsto \alpha^i  \cdot z + c_{\ks_{-i}}^i$.
\end{rem}
\begin{defn}
\label{defn NE preserving}
Let $H = \{H^i\}_{i \in [N]}$ be a game transformation. Then we say that $H$ universally preserves \NE{} sets if for all input games $G = \{u_i\}_{i \in [N]}$, the transformed game $H(G) = \{H^i(u_i)\}_{i \in [N]}$ has the same \NE{} set as the input game $G$.
\end{defn}

\begin{defn}
\label{defn BR preserving}
Let map $H^i$ come from a game transformation $H$. Then we say that $H^i$ universally preserves best responses if for all utility functions $u_i : S \longrightarrow \R$ and for all opponents' strategy choices $\strats^{-i} \in \Delta(S^{-i})$:
\begin{equation*}
\BR_{H^i(u_i)}(\strats^{-i}) = \argmax_{t^i \in \Delta(S^i)} \big\{ H^i(u_i)(t^i,\strats^{-i}) \big\} = \argmax_{t^i \in \Delta(S^i)} \big\{ u_i(t^i,\strats^{-i}) \big\} = \BR_{u_i}(\strats^{-i}) \, .
\end{equation*}
\end{defn}

\begin{defn}
\label{defn opponent dependence}
Let map $H^i$ come from a game transformation $H$. Then we say that $H^i$ only depends on the strategy choice of the opponents if for all pure strategy choices $\ks_{-i} \in S^{-i}$ of the opponents, we have the map identities
    \[h_{1, \ks_{-i}}^i = \ldots = h_{m_i, \ks_{-i}}^i\,: \R \to \R \, .\]
\end{defn}

\end{document}
\section{Addressing Machines}\label{sec:machines}
% !TEX root = ../DPIM.tex
% !TEX spellcheck = en-US

In this section we introduce the notion of an \emph{Addressing Machine}.
We first provide some intuitions, then we proceed with the formal description of such machines.
The general structure of an addressing machine is composed by two substructures:
\begin{itemize}
\item the \emph{internal components}, organized as follows:
	\begin{itemize}
	\item a finite number of \emph{internal registers};
	\item an \emph{internal program}.
\end{itemize}
\item the \emph{input-tape}.
\end{itemize}
As the name suggests, the addressing mechanism is central in this formalism.
Each addressing machine is associated with an address, receives a list of addresses in its input-tape and is able to transfer the computation to another machine by calling its address, possibly extending its input-tape.

\subsection{Tapes, Registers and Programs}
We consider fixed a countable set $\Addrs$ of \emph{addresses}, together with a constant $\Null\notin\Addrs$ that we call ``null'' and that corresponds to an uninitialized register.
\begin{defi} We let $\Addrs_\Null = \Addrs\cup\set{\Null}$.
\bsub
\item
	An \emph{$\Addrs$-valued tape} $T$ is a finite (possibly empty) ordered list of addresses $T = [a_1,\dots,a_n]$ with $a_i\in\Addrs$ for all $i \le n$.
	We write $\Tapes$ for the set of all $\Addrs$-valued tapes.
\item
	 Let $a\in\Addrs$ and $T,T'\in\Tapes$. We denote by $\Cons a {T}$ the tape having $a$ as first element and $T$ as tail. We write $\appT{T}{T'}$ for the concatenation of $T$ and $T'$, which is an $\Addrs$-valued tape itself.

\item
	Given an index $i\in\nat$, an $\Addrs_\Null$-valued \emph{register} $R_i$ is a memory-cell capable of storing either $\Null$ or an address $a\in\Addrs$.
 \item Given $\Addrs_\Null$-valued registers $R_0,\dots,R_{n}$ for $n\ge 0$, an address $a\in\Addrs$ and an index $i\in\nat$, we write $\vec R\repl{R_i}{a}$ for the registers $\vec R$ where the value of $R_i$ has been updated:
 \[
 R_0,\dots,R_{i-1},a,R_{i+1},\dots,R_{n}
 \]
Notice that, whenever $i > n$, we assume that $\vec R\repl{R_i}{a} = \vec R$.
\esub
\end{defi}

\noindent
Addressing machines can be seen as having a RISC architecture, since their internal program is composed by only three instructions. We describe the effects of these basic operations on a machine having $r$ internal registers $R_0,\dots,R_{r-1}$.
Therefore, when we say ``if an internal register $R_i$ exists'' we mean that the condition $0\le i< r$ is satisfied.
In the following, $i,j,k\in\nat$ correspond to indices of internal registers:
	\begin{itemize}
	\item $\Load i$: corresponds to the action of reading the first element $a$ from the input-tape $T$, and writing $a$ on the internal register $R_i$. If the input-tape is empty then the machine remains stuck waiting for an input (however, this is not considered as an error state).\\[3pt]
The \emph{precondition} to execute the operation is that the input-tape is non-empty, namely $T = \Cons a T'$; the \emph{postconditions} are that $R_i$, if it exists, contains the address $a$ and the input-tape of the machine becomes $T'$.
	If $R_i$ does not exist, i.e.\ when $i\ge r$, the content of $\vec R$ remains unchanged (i.e., the input element $a$ is read and subsequently thrown away).
	\item $\Apply i j k$: corresponds to the action of reading the contents of $R_i$ and $R_j$, calling an external \emph{application map} on the corresponding addresses $a_1,a_2$, and writing the result in the internal register $R_k$, if it exists.\\[3pt]
The \emph{precondition} is that $R_i,R_j$ exist and are initialized, i.e.\ $R_i,R_j\neq\Null$.
The \emph{postcondition} is that $R_k$, if it exists, contains the address of the machine of address $a_1$ whose input-tape has been extended with $a_2$.
Otherwise the content of $\vec R$ remains unchanged.
	\item
	$\Call i$: transfers the computation to the machine whose address is stored in $R_i$, extending its input-tape with the addresses that are left in $T$.\\[3pt]
	The \emph{precondition} is that $R_i$ exists and is initialized.
	The \emph{postcondition} is that the machine having the address stored in $R_i$ is executed on the extended input-tape.
	\end{itemize}

\noindent
We define what is a syntactically valid program of this language, and  introduce a decision procedure for verifying that the preconditions of each instruction are satisfied when it is executed.
As we will see in Lemma~\ref{lem:correction}, these properties are decidable and statically verifiable.
As a consequence, addressing machines will never give rise to an error at run-time.

\begin{defi}\label{def:progs}\
\bsub
\item\label{def:progs1}
	A \emph{program} $P$ is a finite list of instructions generated by the following grammar (where $\varepsilon$ represents the empty string, and $i,j,k\in\nat$):
	\[
	\begin{array}{lcl}
	\ins{P}&\eqbnf&\Load i;\, \ins{P}\mid \ins{A}\\
	\ins{A}&\eqbnf&\Apply ijk;\, \ins{A}\mid \ins{C}\\
	\ins{C}&\eqbnf&\Call i \mid \varepsilon
	\end{array}
	\]
	In other words a program starts with a list of $\ins{Load}$'s, continues with a list of $\ins{App}$'s and possibly ends with a $\ins{Call}$. Each of these lists may be empty, in particular the empty-program $\varepsilon$ can be generated.
\item\label{def:progs2}
	Given a program $P$, an $r\in\nat$, and a set $\cI\subseteq \set{0,\dots,r-1}$ of indices (representing initialized registers), define $\cI\models^{r} P$ as the least relation closed under the rules:
\[
	\begin{array}{ccccc}
		\infer{\cI\models^{r}\varepsilon}{}
		&&
		\infer{\cI\models^{r} \Apply ijk;\, \ins{A}}{\cI\cup\set{k}\models^{r}  \ins{A} & i,j\in \cI & k<r}
		&&
		\infer{\cI\models^{r} \Load i;\, \ins{P}}{\cI\cup\set{i}\models^{r}  \ins{P} & i< r}
		\\[3pt]
		\infer{\cI\models^{r}\Call i}{i\in \cI}&&
		\infer{\cI\models^{r} \Apply ijk;\, \ins{A}}{\cI\models^{r}  \ins{A} & i,j\in \cI & k\ge r}
		&&
		\infer{\cI\models^{r} \Load i;\, \ins{P}}{\cI\models^{r}  \ins{P} & i \ge r}
	\end{array}
\]
\item Let $r\in\nat$ and $\vec R = R_0,\dots,R_{r-1}$ be $\Addrs_\Null$-valued registers.
	We say that a program $P$ is \emph{valid with respect to $\vec R$} whenever $\cR\models^{r} P$ holds for
	\begin{equation}\label{eq:R}
		\cR = \set {i\st R_i \neq\Null \und 0\le i < r}
	\end{equation}
\esub
\end{defi}

\noindent
Notice that the notion of a valid program is independent from the tape of a machine.

\begin{exas} Consider addresses $a_1, a_2\in\Addrs $, as well as $\Addrs_\Null$-valued registers $R_0 = \Null$, $R_1 = a_1,R_2=a_2,R_3 = \Null$ (so $r = 4$).
In this example, the set $\cR$ of initialized registers as defined in~\eqref{eq:R} is $\cR = \set{1,2}$.
\[
	\begin{array}{lcc}
	P_n&\textrm{Program}&\cR\models^4 P_n\\
	\toprule
	P_0=&\Load 0;\Apply012;\Call 2&\checkmark\\
	P_1=&\Apply 120;\,\Apply 023;\,\Call 3&\checkmark\\
	P_2=&\Load 5;\, \Load 0;\,\Call 0&\checkmark\\
	P_3=&\Load 5;\, \Apply 12{5};\,\Call 2&\checkmark\\
	P_4=&\Apply 012;\,\Call 2&\xmark\\
	P_5=&\Load 0;\,\Call 3&\xmark\\
	P_6=&\Apply 123;\,\Call 5&\xmark\\
	\end{array}
\]
Above we use ``5'' as an index of an unexisting register.
Notice that a program trying to update an unexisting register remains valid (see $P_2,P_3$), the new value is simply discharged.
On the contrary, an attempt at reading the content of an uninitialized ($P_4,P_5$) or unexisting ($P_6$) register invalidates the whole program.
\end{exas}

\begin{nota}\label{nota:aboutprogs}
We use ``$-$'' to indicate an arbitrary index of an unexisting   register. E.g., the program $P_6$ will be written $\Apply 123;\,\Call -$.
We also write $\Load (i_1,\dots,i_k)$ as an abbreviation for $\Load i_1;\,\cdots;\,\Load i_k;$ . By employing all these notations, $P_2$ can be written as  $P_2= \Load (-,0);\Call 0$. % chktex 40 chktex 26
\end{nota}

\begin{lem}\label{lem:correction}
For all $\Addrs_\Null$-valued registers $\vec R$ and program $P$ it is decidable whether $P$ is valid with respect to $\vec R$.
\end{lem}

\begin{proof}
Decidability follows from the fact that the grammar in Definition~\ref{def:progs}\ref{def:progs1} is right-linear, the list of registers $\vec R$ is finite, the rules in Definition~\ref{def:progs}\ref{def:progs2} are syntax-directed and their side conditions are decidable.
%First, notice that the grammar in Definition~\ref{def:progs}\ref{def:progs1} is right-linear, therefore it is decidable whether $P$ is a production.
%Also, $r\in\nat$ therefore $\cR$ is finite and, since $P$ is also finite, the set $\cR$ remains finite during the execution of $\cR\models^{r} P$.
%Decidability follows from these properties, together with the fact that the first instruction of $P$ uniquely determines which rule from Definition~\ref{def:progs}\ref{def:progs2} should be applied (and these rules are exhaustive).
\end{proof}

\subsection{Addressing machines and their operational semantics}

Everything is in place to introduce the definition of an \am.
Thanks to Lemma~\ref{lem:correction} it is reasonable to require that an \am{} has a valid internal program.

\begin{defi}\label{def:AM}
\bsub
\item
	An \emph{addressing machine $\mM$ (with $r$ registers) over $\Addrs$} is given by a tuple:
\[
	\mM = \tuple{\vec R,P,T}
\] where:
\begin{itemize}
\item
	$\vec R = R_0,\dots,R_{r-1}$ are $\Addrs_\Null$-valued registers;
\item
	$P$ is a program valid w.r.t.\ $\vec R$;
\item
	$T$ is an $\Addrs$-valued \emph{(input) tape}.
\end{itemize}
\item
	We write $\mM.r$ for the number of registers of $\mM$, $\mM.\vec R$ for the list of its registers, $\mM.R_i$ for its $i$-th register, $\mM.P$ for the associated program and finally $\mM.T$ for its input tape.
\item
	We say that an addressing machine $\mM$ as above is \emph{stuck}, in symbols $\stuck{\mM}$, whenever its program has shape $\mM.P = \Load i;\ins{P}$ but its input-tape is empty $\mM.T = []$. Otherwise, $\mM$ is \emph{not stuck}, in symbols: $\lnot\stuck{\mM}$.
\item
	The set of all addressing machines over $\Addrs$ will be denoted by $\cM$.
\esub
\end{defi}

\noindent
The machines below will be used as running examples in the next sections.
Intuitively, the \am s $\mK,\mS,\mach{I},\mach{D},\mach{O}$ mimic the behavior of the \lam-terms $\comb{K}$, $\comb{S}$, $\comb{I}$, $\comb{\Delta}$ and $\comb{\Omega}$, respectively. For writing their programs, we adopt the conventions introduced in Notation~\ref{nota:aboutprogs}.

\begin{exas}\label{ex:ilprimoesempiononsiscordamai}
The following are addressing machines.
\bsub
\item\label{ex:ilprimoesempiononsiscordamai1}
	For every $n\in\nat$, define an addressing machine with $n+1$ registers as:
\[
	\mach{x}_n = \tuple{R_0,\dots,R_n,\varepsilon,[]},\textrm{ where }\vec R := \vec \Null.
\]
We call $\mach{x}_0,\mach{x}_1,\mach{x}_2,\dots$ \emph{indeterminate machines} because they share some analogies with variables (they can be used as place holders).
\item
The addressing machine $\mK$ with 1 register $R_0$ is defined by:
\[
	\mK = \tuple{\Null,\RaS (0,-); \Call 0,[]}
\]
\item The addressing machine $\mS$ with 3 registers is defined by:
\[
	\begin{array}{lcl}
	\mS &=& \tuple{\Null,\Null,\Null,P,[]}\textrm{, where:}\\
	\mS.P &=& \RaS (0, 1,2);\,\Apply 020;\\
	&&\Apply 121;\,\Apply 012;\,\Call 2\\
	\end{array}
\]
\item Assume that $k\in\Addrs$ represents the address associated with the \am{} $\mK$.
Define the \am{} $\mach{I}$ as $\mach{I} = \tuple{\Null^3,\mS.P,[k,k]}$.
\item The addressing machine $\mach{D}$ with 1 register is given by:
\[
	\mach{D} = \tuple{\Null,\RaS 0;\,\Apply 000;\,\Call 0,[]}
\]
\item Assume that $d\in\Addrs$ represents the address of the \am{} $\mach{D}$.
Define the \am{} $\mach{O}$ by setting $\mach{O} = \tuple{\Null,\mach{D}.P,[d]}$.
\esub
\end{exas}

\noindent
We now enter into the details of the addressing mechanism which constitutes the core of this formalism.
In an implementation of \am s, it would be reasonable to pick up a fresh address from $\Addrs$ whenever a new machine is constructed and save the correspondence in some address table. See Section~\ref{sec:conclusions} for more implementation details.
To construct a \lam-model, we need a uniform way of associating machines with their addresses.

\begin{defi} Fix a bijective map $\Lookup : \cM \to  \Addrs$ from the set of all \am s over $\Addrs$ to the set $\Addrs$ of addresses.
We call the map $\Lookup(\cdot)$ an \emph{Address Table Map (ATM)}.
\bsub
\item Given $M\in\cM$, we say that $\Lookup \mM$ is the \emph{address of} $\mM$.
\item
	Given an address $a\in\Addrs$, we write $\Lookinv{a}$ for the unique machine having address $a$. In other words, we have $\Lookinv{a} = \mM\iff \Lookup\mM = a$.
\item
	Given $\mM\in\cM$ and $T'\in\Tapes$, we write $\appT{\mM}{T'}$ for the machine
	\[
		\tuple{\mM.\vec R,\mM.P,\appT{\mM.T}{T'}}
	\]
\item
	Define the \emph{application map} $(\App{}{}) : \Addrs\times\Addrs\to \Addrs$ as follows
	\[
		\App{a}{b} = \Lookup (\append{\Lookinv{a}}{b})
	\]
	That is, the \emph{application} of $a$ to $b$ is the unique address $c$ of the \am{} obtained by adding $b$ at the end of the input tape of the \am{} $\Lookinv{a}$.
\esub
\end{defi}

\noindent
Since both $\cM$ and $\Addrs$ are countable sets, there exist $2^{\aleph_0}$ possible choices for an ATM\@.
\begin{rem}\label{rem:forever} Depending on the chosen ATM $\Lookup(-)$, there might exist \am s calling each other, as in $\mM = \tuple{\Lookup\mN,\Call 0,[]}$ and $\mN = \tuple{\Lookup\mM,\Call 0,[]}$, or even countably many machines $(\mM_n)_{n\in\nat}$ satisfying $\mM_n = \tuple{\Lookup\mM_{n+1},\varepsilon,[]}$.
Therefore, in general, the process of recursively dereferencing the addresses stored in the registers (or tape) of a machine might not terminate.
This kind of behaviour is not pathological, rather intrinsic to the notions of addresses and dereference operators.
\end{rem}
In practice, one may desire to work with an ATM performing the association between \am s and their addresses in a computable way.
However, we do not require our ATMs to satisfy any effectiveness conditions since it would be peculiar to propose a model of computation depending on a pre-existing notion of ``computable''. The results presented in this paper are independent from the ATM  under consideration.

\begin{defi}[Small step operational semantics]\label{def:smallstep}
Define a reduction strategy on \am s representing one head-step of computation
\[
	\redh\ \subseteq\cM\to\cM
\]
as the least relation closed under the following rules:
\[
	\begin{array}{lcl}
	\tuple{\vec R,\RaS i;P,\Cons a{T}} &\redh& \tuple{\vec R[R_i := a],P,T},\\
	\tuple{\vec R,\Apply i j k; P,T}&\redh&\tuple{\vec R[R_k := \App{R_i}{R_j}],P,T},\\
	\tuple{\vec R,\Call i,T}&\redh&\appT{\Lookinv {R_i}}{T}.\\
	\end{array}
\]
As usual, we write $\reddh$ for the transitive-reflexive closure of $\redh$.
We say that an \am{} $\mM$ \emph{is in a final state} if there is no $\mN$ such that $\mM\redh \mN$.
We write $\mM\reddh \stuck{\mN}$ whenever $\mM\reddh \mN$ and $\stuck{\mN}$ hold.
When $\mN$ is not important, we simply write $\mM\reddh\stuck{}$. Similarly, $\mM\not\reddh\stuck{}$ means that $\mM$ never reduces to a stuck \am.
\end{defi}

\begin{rem}\label{rem:aboutstuck}\
\bsub
\item Definition~\ref{def:smallstep} is well defined since the validity of a program is preserved by ${\sf h}$-reduction: if $\mM\redh\mN$ and $\mM.P$ is valid w.r.t.\ $\mM.\vec R$ then $\mN.P$ is valid w.r.t.\ $\mN.\vec R$. This follows immediately from Definition~\ref{def:progs}\ref{def:progs2}.
In particular when executing $\RaS i$, or $\Call i$, $R_i$ must be initialized and when executing $\Apply i j k$ we must have $R_i,R_j\neq\Null$.
\item\label{rem:aboutstuck2}
Addressing machines in a final state are either of the form $\tuple{\vec R,\varepsilon,T}$ or $\tuple{\vec R,\Load i;P,[]}$, and in the latter case they are stuck.
\esub
\end{rem}

\begin{lem}\label{lem:about_red}
The reduction strategy $\redh$ enjoys the following properties:
\bsub
\item\label{lem:about_red1}
	 Determinism: $\mM\redh \mN_1 \und \mM\redh \mN_2\ \Rightarrow\ \mN_1 = \mN_2$.
\item\label{lem:about_red2}
	Closure under application: $\forall a\in\Addrs\,.\,\mM\redh \mN\ \Rightarrow\ \append{\mM}{a}\redh \append{\mN}{a}$.
\esub
\end{lem}
\begin{proof} $(i)$ Since the applicable rule from Definition~\ref{def:smallstep}, if any, is uniquely determined by the first instruction on $\mM.P$ and its input-tape $\mM.T$.

$(ii)$ Easy. By cases on the rule applied for deriving $\mM\redh\mN$.
\end{proof}

\begin{exas}\label{ex:somemachines}
For brevity, we sometimes display only the first instruction of the internal program. Take $a,b,c\in\Addrs$.
\bsub
\item We show that $\mach K$ behaves as the first projection:
\[
	\begin{array}{lll}
	\append{\mK}{a,b}&=&\tuple{\Null,\RaS (0, -); \Call 0,[a,b]}\\
	&\redh&\tuple{a, \RaS -; \Call 0,[b]}
	\redh
	\tuple{a, \Call 0,[]}\redh\Lookinv a.\\
	\end{array}
\]
\item We verify that $\mach S$ behaves as the combinator $\comb{S}$ from combinatory logic:
\[
	\begin{array}{lll}
	\append{\mS}{a,b,c}&=&\tuple{\Null^3,\RaS (0,1,2); \cdots,[a,b,c]}\\
	&\reddh&\tuple{a,b,c,\Apply 020; \cdots,[]}\\
	&\redh&\tuple{\App{a}{c},b,c,\Apply 121; \cdots,[]}\\
	&\redh&\tuple{\App{a}{c},\App{b}{c},c,\Apply 012; \cdots,[]}\\
	&\redh&\tuple{\App{a}{c},\App{b}{c},\App{(\App{a}{c})}{(\App{b}{c})},\Call 2; \cdots,[]}\\
	&\redh&\Lookinv {\App{(\App{a}{c})}{(\App{b}{c})}}\\
	\end{array}
\]
\item As expected, $\mach{I}=\append{\mS}{\Lookup \mK,\Lookup \mK}$ behaves as the identity:
\[
	\begin{array}{lll}
	\append{\mach{I}}{a} &=&
	\tuple{\Null^3,\RaS (0,1,2);\cdots,[\Lookup{\mK},\Lookup{\mK},a]}\\
	&\reddh&\tuple{\Lookup{\mK},\Lookup{\mK},a,\Apply 020;\cdots,[]}\\
	&\redh&\tuple{\App{\Lookup{\mK}}{a},\Lookup{\mK},a,\Apply 121;\cdots,[]}\\
	&\redh&\tuple{\App{\Lookup{\mK}}{a},\App{\Lookup{\mK}}{a},a,\Apply 012;\cdots,[]}\\
	&\redh&\tuple{\App{\Lookup{\mK}}{a},\App{\Lookup{\mK}}{a},\App{\App{\Lookup{\mK}}{a}}{(\App{\Lookup{\mK}}{a})},\Call 2;[]}\\
	&\redh&\append{\mK}{a,\App{\Lookup{\mK}}{a}}\\
	&=&\tuple{\Null,\RaS (0,-);\cdots,[a,\App{\Lookup{\mK}}{a}]}\\
	&\reddh&\tuple{a,\App{\Lookup{\mK}}{a},\Call 0,[]}
	\redh\Lookinv{a}\\
	\end{array}
\]
\item Finally, we check that $\mach{O}$ gives rise to an infinite reduction sequence:
\[
	\begin{array}{lll}
	\mach{O} &=& \tuple{\Null,\RaS 0;\,\Apply 000;\,\Call 0,[\Lookup\mach{D}]}\\
	&\redh&\tuple{\Lookup\mach{D},\Apply 000;\,\Call 0,[]}\\
	&\redh&\tuple{\Lookup(\append{\mach{D}}{\Lookup\mach{D}}),\Call 0,[]}\redh \append{\mach{D}}{\Lookup\mach{D}}
	= \mach{O}\reddh\cdots\\
	\end{array}
\]
\esub
\end{exas}

\noindent
Similarly, we can define a big-step operational semantics relating an \am{} $\mM$ with its final result (if any).

\begin{defi}[Big-step semantics]
Define $\mM \goesto \mach{V}$, where $\mM,\mV\in\cM$ and $\mV$ is in a final state, as the least relation closed under the following rules:
\begin{gather*}
	\infer[(\textrm{Stuck})]{\mM \goesto \mM}{
		\mM.P = \RaS i;P'
		&
		\mM.T = []
		}
	\qquad\qquad
	\infer[(\textrm{End})]{\mM \goesto \mM}{
		\mM.P = \varepsilon
		}\\
		\infer[(\textrm{Load})]{\mM \goesto \mV}{
		\mM.P = \RaS i;P'
		&
		\mM.T = \Cons a{T'}
		&
		\tuple{\mM.\vec R\repl{R_i}{a},P',T'}\goesto \mV
		}
		\\
		\infer[(\textrm{App})]{
		\mM\goesto\mV}{\mM.P = \Apply i j k; P'
		&
		a = \App{\mM.R_i}{\mM.R_j}
		&
		\tuple{\mM.\vec R\repl{R_k}{a},\mM.P',\mM.T}\goesto\mV
	}
	\\
		\infer[(\textrm{Call})]{\mM\goesto \mV}{
			\mM.P = \Call i
			&
			\mM' = \Lookup^{-1}(\mM.R_i)
			&
			\append{\mM'}{\mM.T}\goesto \mV
	}
\end{gather*}

\begin{exa} Recall that $\mK.P = \Load (0,-);\,\Call 0$. Notice that we cannot prove $\append{\mK}{a,b}\goesto \Lookinv a$ for an arbitrary $a\in\Addrs$, as we need to ensure that the resulting machine is in a final state.
For this reason, we will use indeterminate machines $\mach{x}_1,\mach{x}_2$ from Example~\ref{ex:ilprimoesempiononsiscordamai}\ref{ex:ilprimoesempiononsiscordamai1}.
\[
	\infer{\append{\mK}{\Lookup \mach{x}_1,\Lookup \mach{x}_2}\goesto \mach{x}_1}{
	\mK.P = \Load 0;\,P';
	&
	\infer{\tuple{\Lookup\mach{x}_1,P',[\Lookup\mach{x}_2]}\goesto \mach{x}_1}{
		P'= \Load -;P'' &
		\infer{\tuple{\Lookup\mach{x}_1,P'',[]}\goesto \mach{x}_1}{
				P''=\Call 0
				&
				R_0 = \Lookup\mach{x}_1
				&
				\infer{\mach{x}_1\goesto \mach{x}_1}{\textrm{(End)}}
		}
	}
	&
	}
\]
\end{exa}
\end{defi}

We now show that the two operational semantics are equivalent on terminating computations.

\begin{prop}\label{prop:equivsem}
For $\mM,\mN\in\cM$, the following are equivalent:
\begin{enumerate}
\item $\mM\reddh \mN\not\redh$;
\item $\mM\goesto\mN$.
\end{enumerate}
\end{prop}

\begin{proof}
(1 $\Rightarrow$ 2) By induction on the length $n$ of the reduction $\mM=\mM_1\redh\mM_2\redh\cdots\redh \mM_n=\mN\not\redh$.

Case $n = 0$. By assumption $\mN$ is in a final state. By Remark~\ref{rem:aboutstuck}\ref{rem:aboutstuck2}, it is either of the form $\mN = \tuple{\vec R,\varepsilon,T}$ or it is stuck $\mN = \tuple{\vec R,\Load i;P,[]}$. In the former case we apply (\textrm{End}), in the latter (\textrm{Stuck}).

Case $n > 1$. Since $\mM_1\redh \mM_2$, we have $\mM_1.P\neq\varepsilon$.
As the length of $\mM_2\reddh \mN$ is $n-1$, by induction hypothesis we have a derivation of $\mM_2\goesto \mN$.
Depending on the first instruction in $\mM_1.P$, we use this derivation to apply the homonymous rule (Load), (App) or (Call) and derive $\mM\goesto \mN$.

(2 $\Rightarrow$ 1) By induction on a derivation of $\mM\goesto \mN$.

Cases (Stuck) or (End). Then, $\mM\reddh\mM=\mN$ by reflexivity of $\reddh$.

Case (Load), i.e.\ $\mM.P=\Load i;P'$. In this case, we have that $\mM\redh \tuple{\mM.\vec R\repl{R_i}{a},P',\mM.T}\reddh \mN$, by induction hypothesis.

Case (App), i.e.\ $\mM.P=\Apply ijk;P'$. Let us call $a = \App{\mM.R_j}{\mM.R_k}$. Then we have $\mM\redh \tuple{\mM.\vec R\repl{R_k}{a},P',\mM.T}\reddh \mN$, by induction hypothesis.

Case (Call), i.e.\ $\mM.P = \Call i$. In this case $\mM\redh \append{\mM'}{\mM.T}$ for $\mM' = \Lookinv{\mM.R_i}$. By induction hypothesis $\append{\mM'}{\mM.T}\reddh \mN$, whence $\mM\reddh \mN$.
\end{proof}


\section{Combinatory Algebras via Evaluation Equivalence}\label{sec:combalg}
% !TEX root = ../DPIM.tex
% !TEX spellcheck = en-US

In this section we show how to construct a combinatory algebra based on the \am s formalism. % chktex 1
Recall that the \am s $\mK$ and $\mS$ have been defined in Example~\ref{ex:ilprimoesempiononsiscordamai}. Consider the algebraic structure % chktex 1
\[
	\cA = (\Addrs,\App{\,}{\,},\Lookup\mach{K},\Lookup\mS)
\]
Since the application $(\App{}{})$ is total, $\cA$ is an applicative structure.
However, it is \emph{not} a combinatory algebra.
For instance, the $\lama$-term $\comb{K}\cons a\cons b$ is interpreted as the address of the machine $\append{\mach{K}}{a,b}$, which is \emph{a priori} different from the address ``$a$'' because no computation is involved.
Therefore, we need to quotient the algebra $\cA$ by an equivalence relation equating at least all addresses corresponding to the same machine at different stages of the execution.

In the following, we denote by $\equiv_{\rel R}$ an arbitrary binary relation on $\cM$. The symbol ${\rel R}$ has no formal meaning, it is simply evocative of a relation.
In the next definition, we are going to associate with every $\equiv_{\rel R}$ two relations, respectively denoted $\simeq_{\rel R}\,\subseteq \Addrs^2$ and $=_{\rel R}\,\subseteq \cM^2$.

\begin{defi}\label{def:inducingequivalences}
Every binary relation $\equiv_{\rel R}\,\subseteq\cM^2$ on \am s induces a relation $\simeq_{\rel R}\,\subseteq \Addrs^2$ defined by % chktex 1
\[
	a\simeq_{\rel R} b\iff \Lookinv{a}\equiv_{\rel R} \Lookinv{b}
\]
which is then extended to:
\bsub
\item $\Addrs_\Null$-valued registers:
\[
	R\simeq_{\rel R} R' \iff (R = \Null = R')\lor (R = a \simeq_{\rel R} b =R');
\]
\item Tuples:
\[
	a_1,\dots,a_n \simeq_{\rel R} b_1,\dots,b_m \iff (n = m)\land (\forall i\in\set{1,\dots,n}\,.\, a_i\simeq_{\rel R} b_i);
\]
(This also applies to tuples of $\Addrs_\Null$-valued registers $\vec R\simeq_{\rel R} \vec R'$.)
\item  $\Addrs$-valued tapes:
	\[
		[a_1,\dots,a_n]\simeq_{\rel R} [b_1,\dots,b_m]\iff \vec a \simeq_{\rel R} \vec b \textrm{ (seen as tuples).}
	\]
\esub
\noindent
In its turn, $\simeq_{\rel R}$ induces a relation $=_{\rel R}\ \subseteq\cM^2$ defined by setting (for all machines $\mM,\mN\in\cM$):
\[
	\mM =_\rel{R} \mN \iff (\mM.\vec R \simeq_{\rel R}\mN.\vec R)\land (\mM.P = \mN.P)\land(\mM.T \simeq_{\rel R}\mN.T)
\]
\end{defi}

In particular, $\mM =_\rel{R} \mN$ entails that $\mM$ and $\mN$ share the same internal program, the number of internal registers, and the length of their input tape.

\begin{lem}\label{lem:equivalence}
If the relation $\equiv_\rel{R}$ is an equivalence then so are $\simeq_\rel{R}$ and $=_\rel{R}$.
\end{lem}
\begin{proof} Assume that $\equiv_\rel{R}$ is an equivalence. Then, the fact that $\simeq_\rel{R}$ is an equivalence follows from its definition since $\Lookinv{\cdot}$ is a bijection. Concerning the relation $=_\rel{R}$, reflexivity, symmetry and transitivity follow immediately from the same properties of $\simeq_{\rel{R}}$ and $=$.
\end{proof}

\begin{defi}\label{def:equiv:Addrs}
Define $\equiva\ \subseteq\cM^2$ as the least equivalence closed under:
\[
	\infer[\redrule]{\mM\equiv_\Addrs	 \mN}{\mM\reddh \mZ =_\Addrs \mN}
\]
\end{defi}
We say that $\mM,\mN$ are \emph{evaluation equivalent} whenever $\mM\equiva\mN$.

\begin{rem}\
\begin{enumerate}[(i)]
\item Reflexivity can be treated as a special case of the rule $\redrule$ since $\mM\reddh\mM=_\Addrs \mM$.
\item It follows from the definition that $=_\Addrs\,\subseteq\ \equiva$ and that $\mM\reddh \mN$ entails $\mM\equiva \mN$.
\end{enumerate}
\end{rem}

\begin{exas}\label{ex:calculs}
From the calculations in Examples~\ref{ex:somemachines}, it follows that
\[
	\begin{array}{lcl}
	\append{\mK}{\Lookup\mach{x}_1,\Lookup\mach{x}_2}&\equiva& \mach{x}_1, \\
	\append{\mS}{\Lookup\mach{x}_1,\Lookup\mach{x}_2,\Lookup\mach{x}_3}&\equiva& \append{(\append{\mach{x}_1}{\Lookup \mach{x}_3})}{\Lookup(\append{\mach{x}_2}{\Lookup \mach{x}_3})}.\\
	\end{array}
\]
\end{exas}

\begin{lem}
The relation $\sima$ is a congruence on $\cA= (\Addrs,\App{\,}{\,},\Lookup\mach{K},\Lookup\mS)$.
\end{lem}
\begin{proof}
By definition $\equiva$ is an equivalence, whence so is $\sima$ by Lemma~\ref{lem:equivalence}.
Let us check that $\sima$ is compatible w.r.t.\ $(\App{}{})$.
Consider $a \sima a'$ and $b\sima b'$.
Call $\mM = \Lookinv{a}$ and $\mN = \Lookinv{a'}$ and proceed by induction on a derivation of $\mM\equiva\mN$, splitting into cases depending on the last applied rule.

\redrule{} By definition, there exists $\mZ\in\cM$ such that $\mM\reddh \mach Z =_\Addrs \mN$. By Lemma~\ref{lem:about_red}\ref{lem:about_red2}, $\append{\mM}{b} \reddh \append{\mZ}{b} =_\Addrs \append{\mN}{b'}$ whence $\App{a}{b}\sima\App{a'}{b'}$.

(Transitivity) and (Symmetry) follow from the induction hypothesis.
\end{proof}

In order to prove that the congruence $\sima$ is non-trivial, we are going to characterize the equivalence $\mM\equiva\mN$ it in terms of confluent reductions.
For this purpose, we extend $\redh$ in such a way that reductions are also possible within registers and elements of the input-tape of an \am.

\begin{defi} Define the reduction relation $\red[c]\,\subseteq\cM^2$ as the least relation containing $\redh$ and closed under the following rules:
\begin{gather*}
\infer[{(\red[i]^R)}]{\tuple{R_0,\dots,R_{r-1},P,T} \red[c] \tuple{\vec R\repl{R_i}{\Lookup\mM},P,T}}{R_i = a\in\Addrs&0\le i<r& \Lookinv{a}\red \mM}\\[3pt]
\infer[{(\red[i]^T)}]{\tuple{\vec R,P,[a_0,\dots,a_n]} \red[c] \tuple{\vec R,P,[a_0,\dots,a_{i-1},\Lookup\mM,a_{i+1},\dots,a_n]}}{0\le i\le n& \Lookinv{a_i}\red \mM}
\end{gather*}
We write $\mM\red[i]\mN$ if $\mN$ is obtained from $\mM$ by directly applying one of the above rules --- this is called an \emph{inner} step of computation.
The transitive and reflexive closure of $\red$ and $\red[i]$ are denoted by $\redd$ and $\redd[i]$, respectively.
\end{defi}

%\begin{lem}\label{lem:CRmoduloA}
%For all $\mM,\mM',\mN\in\cM$, we have:
%\bsub
%\item\label{lem:CRmoduloA1}
%	If $\mM =_\Addrs \mN$, $\mM\redh\mM'$ and $\mM.P\neq\Call i$ for any index $i$, then there exists $\mN'\in\cM$ such that $\mN\redh\mN'$ with $\mM'=_\Addrs\mN'$.
%\item\label{lem:CRmoduloA2} [Proof wrong!]
%$
%	\mM\equiva\mN\iff \exists \mZ_1,\mZ_2\in\cM\,.\,\mM\reddh \mZ_1\und\mN\reddh \mZ_2\und \mZ_1 =_\Addrs \mZ_2.
%$
%\esub
%\end{lem}

%\begin{proof} $(i)$ By cases on $\mM.P\,(=\mN.P)$. Recall that it is valid w.r.t.\ $\mM.\vec R$.

%Case $\mM.P = \Load i;P_1$, then $\mM.T = \Cons {a_1}{T_1}$ for some $a_1\in\Addrs,T_1\in\Tapes$. This entails $\mN = \tuple{\vec R',\mM.P,\Cons {a_2}{T_2}}$ with $\mM.\vec R\sima \vec R'$, $a_1\sima a_2$ and $T_1\sima T_2$.
%Thus, we can take $\mN' = \tuple{\vec R'\repl{R_i}{a_2},\mM.P_1,T_2}$.

%Case $\mM.P = \Apply ijk;P_1$. Then $\mM.R_i = a_1\neq\Null$ and $\mM.R_j = a_2 \neq\Null$ with $i,j<\mM.r=\mN.r$.
%By assumption $\mM.\vec R\sima \mN.\vec R$ and $\mM.T\sima \mN.T$, in particular $\mN.R_i = a_1'\neq\Null,\mN.R_j=a_2'\neq\Null$ with $a_1\sima a_1'$ and $a_2\sima a_2'$. Take $\mN' =\tuple{\mN.\vec R\repl{R_k}{\App{a_1'}{a_2'}},\mM.P_1,\mN.T}$

%%Case $\mM.P = \Call i$. Then $i<\mM.r = \mN.r$, $\mM.R_i = a\in\Addrs$ and $\mM'= \Lookinv{a}$. Moreover $\mN.R_j = a'$ for some $a'\sima a$ and we can take $\mN' = \Lookinv{a'}$.

%$(ii)$ $(\Rightarrow)$ By induction on $\mM\equiva\mN$, by cases on the last rule applied.

%Case \redrule. Then $\mM\reddh \mZ =_\Addrs \mN$ and we can take $\mZ = \mZ_1$ and $\mZ_2 = \mN$.

%Case (Symmetry). Immediate, from the induction hypothesis.

%Case (Transitivity). Assume $\mM\equiva\mach{X}$ and $\mach{X} \equiva \mN$. By induction hypothesis on the former we get $\mM\reddh \mM'$ and $\mach{X}\reddh \mach{X}_1$ with $\mM' =_\Addrs \mach{X}_1$. By induction hypothesis on the latter, $\mach{X}\reddh\mach{X}_2$ and $\mN\reddh\mZ_2$ with $\mach{X}_2\equiva \mZ_2$.
%By Lemma~\ref{lem:about_red}\ref{lem:about_red1} we have, say, $\mach{X}_1\reddh \mach{X}_2$.
%By $(i)$, there exists $\mZ_1$ such that $\mM_1\reddh\mZ_1$ and $\mZ_1 =_\Addrs \mach{X}_2$. In diagrammatic form:
%\[
%	\xymatrix{
%	\mM\ar@{->>}[d]_{\mach c}&\equiva&\mach{X}\phantom{_1}\ar@{->>}[d]_{\mach c}%&\equiva&\mN\ar@{->>}[dd]_{\mach c}\\
%	\mM_1\ar@{->>}[d]_{\mach c}&=_\Addrs&\mach{X}_1\ar@{->>}[d]_{\mach c}&&\\
%	\mZ_1&=_\Addrs&\mach{X}_2&=_\Addrs&\mZ_2\\
%	}
%\]
%By transitivity of $=_\Addrs$, we conclude that $\mZ_1 =_\Addrs \mZ_2$.
%
%$(\Leftarrow)$ From $\mM\reddh \mach{Z}_1\equiva \mZ_2$ it follows $\mM\equiva\mZ_2$. From $\mN\reddh\mZ_2$ we get $\mN\equiva \mZ_2$.
%We conclude by symmetry and transitivity.
%\end{proof}

\begin{lem}[Postponement of inner steps]\label{lem:standardization}~\\ For $\mM,\mN,\mN'\in\cM$, if $\mM\red[i]\mN\red[h]\mN'$ then there exists $\mM'\in\cM$ such that $\mM\red[h]\mM'\redd[i]\mN'$. In diagrammatic form:
\[
\xymatrix{
\mM\ar@{->}[r]^{\mach{i}}\ar@{-->}[d]^{\mach{h}}&\mN\ar@{->}[d]^{\mach{h}}\\
\mM'\ar@{-->>}[r]^{\mach{i}}&\mN'
}
\]
\end{lem}

\begin{proof} By cases analysis over $\mM\red[i]\mN$.
The only interesting case is when the contracted redex is duplicated in $\mN\red[h]\mN'$, namely:

Case $\mM =\tuple{\vec R\repl{R_i}{a},P,T}$, $\mN = \tuple{\vec R\repl{R_i}{b},P,T}$ with  $\mM.P =\mN.P = \Apply ijk;P'$ and $\Lookinv a\red[c]\Lookinv b$.
Assume $i\neq k<\mM.r$ and $i = j$, the other cases being easier.
In this case $\mM' = \tuple{\vec R\repl{R_i}{a}\repl{R_k}{\App{a}{a}},P,T}$, therefore we need 3 inner steps to close the diagram:
\[
	\begin{array}{lcl}
	\mM'&\red[i]&\tuple{\vec R\repl{R_i}{b}\repl{R_k}{\App{a}{a}},P,T}\\
		&\red[i]&\tuple{\vec R\repl{R_i}{b}\repl{R_k}{\App{b}{a}},P,T}\\
			&\red[i]&\tuple{\vec R\repl{R_i}{b}\repl{R_k}{\App{b}{b}},P,T} = \mN'.
	\end{array}
\]
This concludes the proof.
\end{proof}

Morally, the term rewriting system $(\cM,\red[c])$ is orthogonal because $(i)$ the reduction rules defining $\red[c]$ are non-overlapping as $\red[h]$ is deterministic, $(\red[i]^R)$ reduces a register and $(\red[i]^T)$ reduces  one element of the tape; $(ii)$ the terms on the left-hand side of the arrow are linear, as no equality among subterms is required.
Now, it is well-known that orthogonal TRS are confluent, but one cannot apply~\cite[Thm.4.3.4]{terese} directly since we are not exactly dealing with first-order terms (because of the presence of the encoding).

\begin{prop}\label{prop:confluence}
The reduction $\red[c]$ is confluent.
\end{prop}

\begin{proof}[Proof sketch] The Parallel Moves Lemma, which is the key property for proving Theorem~4.3.4 in~\cite{terese} generalizes easily. The rest of the proof follows.
\end{proof}

\begin{lem}\label{lem:onestepisfine}
Let $\mM,\mN\in\cM$.
\begin{enumerate}[(i)]
\item\label{lem:onestepisfine1}
	$\mM\red\mN$ entails $\mM\equiva\mN$.
\item\label{lem:onestepisfine2}
	$\mM\redd\mN$ entails $\mM\equiva\mN$.
\end{enumerate}
\end{lem}

\begin{proof}
\ref{lem:onestepisfine1} By induction on a derivation of $\mM\red\mN$. % chktex 2

Base case $\mM\redh \mN$. Since $\equiva$ is an equivalence then so is $=_\Addrs$, by Lemma~\ref{lem:equivalence}.
In particular $=_\Addrs$ is reflexive, whence $\mN =_\Addrs \mN$. By Definition~\ref{def:equiv:Addrs}, we obtain $\mM\equiva \mN$.

Case ${(\red[i]^R)}$. Then $\mM = \tuple{\vec R[R_i:= \Lookup \mM'],P,T}$ and $\mN = \tuple{\vec R[R_i:= \Lookup \mN'],P,T}$ for some existing register $R_i$ and $\mM',\mN'\in\cM$ such that $\mM' \red \mN'$. By induction hypothesis we get $\mM' \equiva \mN'$, equivalently $\Lookup\mM' \sima \Lookup\mN'$.
From this and reflexivity, it follows $\vec R[R_i:= \Lookup \mM'] \sima \vec R[R_i:= \Lookup \mN']$, $P \sima P$ and $T \sima T$.
Thus $\mM =_\Addrs \mN$, so we conclude because $=_\Addrs\,\subseteq\  \equiva$.

Case ${(\red[i]^T)}$. In this case, we have
\[
	\begin{array}{lcl}
	\mM &=& \tuple{\vec R,P,[a_0,\dots,a_{i-1},\Lookup \mM',a_{i+1}\dots,a_n]}\\
	 \mN &=& \tuple{\vec R,P,[a_0,\dots,a_{i-1},\Lookup \mN',a_{i+1}\dots,a_n]}
	 \end{array}
\]
with $\mM' \red \mN'$. By induction hypothesis we get $\mM' \equiva \mN'$, equivalently $\Lookup\mM' \sima \Lookup\mN'$.
This entails $\mM.T \sima \mN.T$, from which it follows $\mM =_\Addrs \mN$. Conclude as above.

\ref{lem:onestepisfine2}  By induction on the length $n$ of the reduction $\mM\redd\mN$. % chktex 2

Case $n=0$. Then $\mM = \mN$, so we get $\mM\equiva\mN$ by reflexivity.

Case $n>0$. Then $\mM\red\mM'\redd\mN$. By~\ref{lem:onestepisfine1}, we get $\mM \equiva \mM'$.
Since the reduction $\mM'\redd\mN$ is strictly shorter, the induction hypothesis gives $\mM' \equiva \mN$.
Conclude by transitivity.
\end{proof}

\begin{thm}\label{thm:CR}
For $\mM,\mN\in\cM$, we have:
\[
	\mM\equiva \mN\iff \exists \mZ\in\cM\,.\, \mM\redd\mZ\invredd[\mach{c}] \mN
\]
\end{thm}

\begin{proof} $(\Rightarrow)$ By induction on a derivation of $\mM\equiva \mN$.

\redrule{} Assume that $\mM\reddh\mZ=_\Addrs \mN$. From $\mZ=_\Addrs \mN$ we get that $\mZ.r = \mN.r$, $\mZ.\vec R\sima \mN.\vec R$, $\mZ.P = \mN.P$ and $\mZ.T\sima\mN.T$. Note that $\mZ.R_i =\Null$ iff $\mN.R_i = \Null$.
Let us call $\cR$ the set of indices $i$ of, say, $\mZ$ such that $\mZ.R_i\neq\Null$.
By assumption, for every $i\in\cR$, we have $\mZ.R_i =a_i,\mN.R_i = a'_i$ for $a_i\sima a'_i$. Equivalently, $\Lookinv{a_i}\equiva\Lookinv{a'_i}$ holds and its derivation is smaller than $\mM\equiva \mN$. By induction hypothesis, they have a common reduct $\Lookinv{a_i}\redd \mach{X}_i\invredd[\mach{c}]\Lookinv{a'_i}$.
Similarly, calling $\mZ.T = [b_1,\dots,b_n]$ and $\mN.T = [b'_1,\dots,b'_m]$ we must have $m = n$ and $b_j\sima b'_j$ whence the induction hypothesis gives a common reduct $\Lookinv{b_j}\redd \mach{Y}_j \invredd[\mach{c}]\Lookinv{b'_j}$.
Putting all reductions together, we conclude:
\[
\mM\reddh \mZ\redd \tuple{\mZ.\vec R\repl{R_i}{\Lookup{\mach{X}_i}}_{i\in\cR},\mZ.P,[\Lookup{\mach{Y}_1},\dots,\Lookup{\mach{Y}_n}]} \invredd[\mach{c}]\mN
\]


(Transitivity) By induction hypothesis and confluence (Proposition~\ref{prop:confluence}).

(Symmetry) Straightforward from the induction hypothesis.

$(\Leftarrow)$ By Lemma~\ref{lem:onestepisfine}\ref{lem:onestepisfine2} we get $\mM\equiva\mZ$ and $\mN\equiva \mZ$, so we conclude by symmetry and transitivity.
\end{proof}

\begin{prop}\label{prop:cAisnonextcombal}
$\cA_{\,\sima}$ is a non-extensional combinatory algebra.
\end{prop}

\begin{proof} From the calculations in Example~\ref{ex:calculs}, it follows that $\App{\App{\Lookup{\mach{K}}}{a}}{b}\sima a$ and
$\App{\App{\App{\Lookup{\mach{S}}}{a}}{b}}{c}\sima \App{(\App{a}{c})}{(\App{b}{c})}$ hold, for all $a,b,c\in\Addrs$.
Notice that both \am s $\mK$ and $\mS$ are stuck, and $\mK\neq_\Addrs \mS$ since, e.g., $\mK.r\neq\mS.r$.
By Theorem~\ref{thm:CR}, we get $\Lookup\mK\not\sima\Lookup \mS$, whence $\cA_{\,\sima}$ is a combinatory algebra.

To check that $\cA_{\,\sima}$ is not extensional, it is sufficient to exhibit two elements of $\Addrs$ that are extensionally equal, but distinguished modulo $\sima$.
For instance, take $\App{\Lookup\mK}{a}$ and $\App{\Lookup\mK'}{a}$, where $a\in\Addrs$ is arbitrary and $\mK'$ is a different implementation of the combinator $\comb{K}$, namely:
\[
	\begin{array}{lcl}
	\mK' &=& \tuple{\Null,\Null,\RaS (0, 1); \Call 0,[]}, \\
	\mK &=& \tuple{\Null,\RaS (0,-); \Call 0,[]}.
	\end{array}
\]
For all $a,b\in\Addrs$, easy calculations give $\App{\App{\Lookup\mK'}{a}}{b} \sima a$.
Thus, for all $b\in\Addrs$, we have \[
	\App{\App{\Lookup\mK}{a}}{b} \sima a\sima \App{\App{\Lookup\mK'}{a}}{b},
	\]
	whence the two addresses $\App{\Lookup\mK}{a}$ and $ \App{\Lookup\mK'}{a}$ are extensionally equal elements of $\cA_{\,\sima}$.
However, the corresponding \am s are both stuck and $\appT{\mK}{[a]} \neq_\Addrs \appT{\mK'}{[a]}$, because $1 = (\append{\mK}{a}).r \neq (\append{\mK'}{a}).r = 2$.
Since they cannot have a common reduct, we derive $\appT{\mK}{[a]}\not\equiva\appT{\mK'}{[a]}$ by Theorem~\ref{thm:CR}.
We conclude that $\App{\Lookup\mK}{a} \not\sima \App{\Lookup\mK'}{a}$.
\end{proof}

\begin{lem}\label{lem:notlambdalg}
The combinatory algebra $\cA_{\,\sima}$ is not a \lam-model.
\end{lem}

\begin{proof} We need to find $M,N\in\Lambda$ satisfying $M=_\beta N$, while $\cA_{\,\sima}\not\models M = N$.
Take $M = \lam z.(\lam x.x)z =_{\CL} \comb{S(KI)I}$ and $N=\lam x.x =_{\CL} \comb{I}$ where $\comb{I} = \comb{SKK}$.

Recall that $\mach{I} = \mach{\append{S}{\Lookup K,\Lookup K}}$.
Easy calculations give:
\[
	\begin{array}{lll}
	\mach{\append{S}{\App{\Lookup \mK}{\Lookup \mach{I}},\Lookup \mach{I}}} &=&
	\tuple{\Null,\Null,\Null,\RaS 0;\cdots,[\App{\Lookup \mK}{\Lookup \mach{I}},\Lookup\mach{I}]}\\
	&\redh&\tuple{\App{\Lookup \mK}{\Lookup \mach{I}},\Null,\Null,\RaS 1;\cdots,[\Lookup\mach{I}]}\\
	&\redh&\stuck{\tuple{\App{\Lookup \mK}{\Lookup \mach{I}},\Lookup\mach{I},\Null,\RaS 2;\cdots,[]}}
	\end{array}
\]
Similarly,
\[
	\mach{I} = \append{\mach{S}}{\Lookup\mK,\Lookup\mK} \reddh \stuck{\tuple{\Lookup\mach{K},\Lookup\mach{K},\Null,\RaS 2;\cdots,[]}}.
\]
These two machines are both stuck and different modulo $=_\Addrs$ since, e.g., the contents of their register $R_1$ are $\Lookup\mach{I}$ and $\Lookup\mach{K}$ respectively, and it is easy to check that $\Lookup\mach{I}\not\sima \Lookup\mach{K}$.
By Theorem~\ref{thm:CR}, we conclude that $ \App{\App{\Lookup \mS}{(\App{\Lookup\mK}{\Lookup\mach{I}})}}{\Lookup\mach{I}}\not\sima\Lookup\mach{I}$.
\end{proof}


\section{Lambda Models via Applicative Equivalences}\label{sec:lammod}
% !TEX root = ../DPIM.tex
% !TEX spellcheck = en-US

In the previous section we have seen that the equivalence $\sima$, thus $\equiva$, is too weak to give rise to a model of \lam-calculus (Lemma~\ref{lem:notlambdalg}).
The main problem is that a \lam-term $\lam x.M$ is represented as an \am{} performing a ``$\ins{Load}$'' (to read $x$ from the tape) before evaluating the \am{} corresponding to $M$. Since nothing is applied, the tape is empty and the machine gets stuck thus preventing the evaluation of the subterm $M$.
In order to construct a \lam-model we introduce the equivalence $\simea$ below.

\begin{defi}
Define the relation $\equivea$ as the least equivalence satisfying:
\begin{gather*}
	\infer[\redwerule]{\mM \equivea \mN}{\mM\reddh \mZ \eqea \mN}
	\\[3pt]
	\infer[\extrule]{\mM \equivea \mN}{
		\mach{M}\reddh\stuck{\mM'}&
		\mN\reddh\stuck{\mN'}&
		\forall a\in\Addrs\,.\, \append{\mM}{a} \equivea \append{\mN}{a}
	}
\end{gather*}
We say that $\mM$ and $\mN$ are \emph{applicatively equivalent} whenever $\mM\equivea \mN$. Recall that $\simea$ and $\eqea$ are defined in terms of $\equivea$ as described in Definition~\ref{def:inducingequivalences}.
Also in this case, it is easy to check that $\eqea\ \subseteq\ \equivea$ holds.
\end{defi}

\begin{rem}\label{rem:aboutordinals} The rule $\extrule$ shares similarities with the $(\omega)$-rule in \lam-calculus~\cite[Def.~4.1.10]{Bare}, although being more restricted as only applicable to \am{} that eventually become stuck. In particular, both rules have countably many premises, therefore a derivation of $\mM\equivea\mN$ is a well-founded $\omega$-branching tree (in particular, the tree is countable and there are no infinite paths).
Techniques for performing induction ``on the length of a derivation'' in this kind of systems are well-established, see e.g.~\cite{BarendregtTh,IntrigilaS06}. More details about the underlying ordinals will be given in Section~\ref{sec:consistency}.
\end{rem}

\begin{exas}\label{ex:moreexamples} Convince yourself of the following facts.
\bsub
\item\label{ex:moreexamples1}
	As seen in the proof of Lemma~\ref{lem:notlambdalg}, $\mach{I}$ and $\append{\mach{S}}{\App{\Lookup\mK}{\Lookup \mach{I}},\Lookup\mach{I}}$ both reduce to stuck machines.
	For all $a\in\Addrs$, we have that \[\append{\mach{I}}{a}\reddh\Lookinv a\invredd[\mach{h}] \append{\mach{S}}{\App{\Lookup\mK}{\Lookup\mach{I}},\Lookup\mach{I},a}.\]
	By \extrule, they are applicatively equivalent.
\item\label{ex:moreexamples2}
	Since indeterminate machines $\mach{x}_k$ are not stuck, $\mach{x}_m\equivea\mach{x}_n$ entails $m=n$.
\item\label{ex:moreexamples3}
	Let \[\mach{1} = \tuple{\Null^2,\Load (0,1);\Apply010;\Call 0,[]}.\] It is easy to check that, for all $a,b\in\Addrs$, we have $\append{\mach{1}}{a,b}\reddh \append{\Lookinv{a}}{b}\invredd[\mach{h}] \append{\mach{I}}{a,b}$. However, since   $\append{\mach{I}}{\Lookup\mach{x}_n}\reddh \mach{x}_n$ and $\lnot\stuck{\mach{x}_n}$, one cannot apply \extrule{}, whence (intuitively) they are not applicatively equivalent: $\mach{I} \not\equivea \mach{1}$.
\esub
\end{exas}

\noindent
Actually the inequalities claimed in examples~\ref{ex:moreexamples2}-\ref{ex:moreexamples3} above, i.e.\ $\mach{x}_m\not\equivea\mach{x}_n$ for $m\neq n$ and $\mach{I}\not\equivea\mach{1}$, are difficult to prove formally (see Lemma~\ref{lem:about:equivo}\ref{lem:about:equivo2}).

\begin{lem} Let $\mM,\mN\in\cM$ and $a,b\in\Addrs$.
\begin{enumerate}[(i)]
%\item $\forall a,b\in\Addrs\,.\, a\simea b\Rightarrow \append{\mM}{[a]}\equivea \append{\mM}{[b]}$
\item If $\mM\equivea \mN$ then $\append{\mM}{a}\equivea \append{\mN}{a}$.
\item The following rule is derivable:
\[
	\infer[(\mathrm{cong})]{\append{\mM}{a}\equivea \append{\mN}{b}}{\mM \equivea \mN& a \simea b}
\]
\item Therefore, $\simea$ is a congruence on $\cA = (\Addrs,\cdot,\Lookup\mK,\Lookup\mS)$.
\end{enumerate}
\end{lem}

\begin{proof}
%\item By \redwerule since $\append{\mM}{[a]}\eqea \append{\mM}{[b]}$.
$(i)$ By induction on a proof of $\mM\equivea\mN$. Possible cases are:

Case \redwerule. If $\mM\reddh \mZ \eqea \mN$ then $\append{\mM}{a}\reddh\append{\mZ}{a} \eqea \append{\mN}{a}$, by Lemma~\ref{lem:about_red}\ref{lem:about_red2} and the definition of $\eqea$.

Case $\extrule$. Trivial, as the thesis is a premise of this rule.

(Symmetry) and (Transitivity) follow from the induction hypothesis.

$(ii)$ Assume that $\mM\equivea\mN$ and $a\simea b$. Then, we have:
\[
	\begin{array}{lcll}
	\append{\mM}{a}&\eqea&\append{\mM}{b},&\textrm{by reflexivity and }a\simea b,\\
	&\equivea&\append{\mN}{b},&\textrm{by }(i).
	\end{array}
\]
So we conclude by transitivity.

$(iii)$ By Lemma~\ref{lem:equivalence} $\sima$ is an equivalence, by $(ii)$ a congruence.
\end{proof}

We need to show that the congruence $\simea$ is non-trivial, and that the addresses of $\Lookup\mK,\Lookup\mS$ remain distinguished modulo $\simea$.

\begin{lem}\label{lem:about:equivo}
Let $\mM,\mN\in\cM$.
\begin{enumerate}[(i)]
\item\label{lem:about:equivo1}
	If $\mM\equiva \mN$ then $\mM\equivea\mN$.
\item\label{lem:about:equivo2}
	If $\mM\equivea \mN$ and $\mM\reddh\mach{x}_n$ then $\mN\reddh \mach{x}_n$.
\item\label{lem:about:equivo3}
	Hence, the equivalence relation $\simea$ is non-trivial.
\item\label{lem:about:equivo4}
		In particular, $\Lookup \mach{K} \not\simea \Lookup \mach{S}$.
\end{enumerate}
\end{lem}

\begin{proof}
\ref{lem:about:equivo1} % chktex 2
	Easy.

\ref{lem:about:equivo2} % chktex 2
	This proof is the topic of Section~\ref{sec:consistency}.

\ref{lem:about:equivo3} % chktex 2
By~\ref{lem:about:equivo1}, the relation is non-empty. By~\ref{lem:about:equivo2}, $\mach{x}_i \equivea \mach{x}_j $ if and only if $ i = j$, whence there are infinitely many distinguished equivalence classes.

\ref{lem:about:equivo4} % chktex 2
From Example~\ref{ex:somemachines}, we get:
\[
	\begin{array}{lcl}
	\append{\mach{K}}{\Lookup\mach{K},\Lookup\mach{K},\Lookup\mach{x}_1}&\reddh&\tuple{\Lookup{\mach{x}_1,\RaS -;\Call 0,[]}};\\
	\append{\mach{S}}{\Lookup\mach{K},\Lookup\mach{K},\Lookup\mach{x}_1}&\reddh&\mach{x}_1.\\
	\end{array}
\]
For these machines to be $\equivea$-equivalent, the former machine should reduce to $\mach{x}_1$, by~\ref{lem:about:equivo2}, which is impossible since $\tuple{\Lookup{\mach{x}_1,\RaS -;\Call 0,[]}}$ is stuck.
\end{proof}

\subsection{Constructing a \lam-model}

We define an interpretation transforming a \lam-term with free variables $x_1,\dots,x_n$ into an \am{} reading the values of $\vec x$ from its tape.
The definition is inspired from the well-known categorical interpretation of \lam-calculus into a reflexive object of a cartesian closed category.
In particular, variables are interpreted as projections.
See, e.g.,~\cite{Koymans82} or~\cite{Selinger02} for more details.


\begin{defi}[Auxiliary interpretation]\label{def:categoricalint}
Let $M\in\Lama$ and $x_1,\dots,x_n$ be such that $\FV{M}\subseteq\vec x$.
Define $\CInt[\vec x]{-} : \Lama\to\cM$ by induction as follows:
\[
	\begin{array}{lcll}
	\CInt[\vec x]{x_i} &=& \mach{Pr}_i^n,\textrm{ where }1\le i \le n;\\[1ex]
	\CInt[\vec x]{\cons{a}} &=& \mach{Cons}_a^n$, for $a\in\Addrs;\\[1ex]
	\CInt[\vec x]{MN} &=& \tuple{\Null^{n},\Lookup\CInt[\vec x]{M},\Lookup\CInt[\vec x]{N},\Null,\ins{Apply}_n,[]};\\[1ex]
	\CInt[\vec x]{\lam y.M} &=& \CInt[\vec x,y]{M},&\textrm{ assuming wlog that }y\notin\vec x;\\
	\end{array}
\]
where
\[
	\begin{array}{lcl}
	\mach{Pr}_i^n &=& \tuple{\Null,(\RaS -)^{i-1};\RaS 0;(\RaS -)^{n-i-1};\Call 0,[]},\\[1ex]
	\mach{Cons}_a^n &=& \tuple{a,(\RaS -)^{n};\Call 0,[]},\\	[1ex]
	\ins{Apply}_n &=& \RaS (0,\dots,n-1);
					\Apply n 0 n;\cdots;\Apply n {n-1} n;\\
					&&\Apply {n+1} 0 {n+1};\cdots;\Apply {n+1} {n-1} {n+1};\\
                     &&\Apply n {n+1} {n+2};\Call {n+2}.\\
	\end{array}
\]
\end{defi}

\begin{rem}\label{rem:aux_int}
Let $n\in\nat$, and $T = [a_1,\dots,a_n]\in\Tapes$. We have:
\begin{enumerate}[(i)]
\item\label{rem:aux_int1}
	$\appT{\mach{Pr}_i^n}{T}\reddh \Lookinv{a_i}$, for all $i\,(1\le i\le n)$;
\item\label{rem:aux_int2}
	$\appT{\mach{Cons}_b^n}{T}\reddh \Lookinv{b}$, for all $b\in\Addrs$;
\item\label{rem:aux_int3}
	$\tuple{\Null^{n},\Lookup \mM,\Lookup \mN,\Null,\ins{Apply}_n,T} \reddh\append{(\appT{\mM}{T})}{\Lookup(\appT{\mN}{T})}$.
\end{enumerate}
\end{rem}

\noindent
From now on, whenever writing $\CInt{M}$, we assume that $\FV{M}\subseteq\vec x$.
The following are basic properties of the interpretation map defined above.

\begin{lem}\label{lemma:aboutcatint}
Let $M\in\Lam(\Addrs)$, $n\in\nat$, $\vec x = x_1,\dots,x_n$ and $\vec a = a_1,\dots,a_n\in\Addrs$.
\bsub
\item\label{lemma:aboutcatint1}
	$\CInt[\vec x]{M} = \tuple{\vec R,\RaS (i_1,\dots,i_n);P,[]}$ for some $\Addrs_\Null$-valued registers $\vec R$, program $P$ and indices $i_j\in\nat$.
\item\label{lemma:aboutcatint2}
	If $m<n$ then $\append{\CInt[\vec x]{M}}{a_1,\dots,a_m}\reddh \stuck{}$.
\item\label{lemma:aboutcatint3}
	For all $b\in\Addrs$, we have $\append{\CInt[y,\vec x]{M}}{b} \equivea \CInt[\vec x]{M\subst{y}{\cons b}}$.
\item\label{lemma:aboutcatint4}
	In particular, if $y\notin\FV{M}$ then $\append{\CInt[y,\vec x]{M}}{b} \equivea \CInt{M}$.
\item\label{lemma:aboutcatint5}
	$\append{\CInt{M}}{\vec a\,} \equivea \append{\CInt[x_{\sigma(1)},\dots,x_{\sigma(n)}]{M}}{a_{\sigma(1)},\dots,a_{\sigma(n)}}$ for all permutations~$\sigma$ of $\set{1,\dots,n}$.
\esub
\end{lem}

\begin{proof}[Proof of Lemma~\ref{lemma:aboutcatint}]
\ref{lemma:aboutcatint1} % chktex 2
	By a straightforward induction on $M$.

\ref{lemma:aboutcatint2} % chktex 2
	It follows from~\ref{lemma:aboutcatint1}.

\ref{lemma:aboutcatint3} % chktex 2
	We proceed by structural induction on $M$.
	 By~\ref{lemma:aboutcatint2}, if $\vec x\neq\emptyset$ then both \am s reduce to stuck ones, so we can test the applicative equivalence by applying an arbitrary $\vec a$ and conclude using \extrule{} $n$-times.

Case $M = \cons c$. Then $c\subst{y}{\cons b} = c$, and we have:
	\[
	\append{\CInt[y,\vec x]{\cons c}}{b,\vec a} =
	\append{\mach{Cons}_c^{n+1}}{b,\vec a}
	\reddh \Lookinv c\invredd[\mach{h}]\append{\mach{Cons}_c^{n}}{\vec a}.
	\]

Case $M = x_i$ for some $i\,(1\le i\le n)$. Then $x_i\subst{y}{\cons b} = x_i$ and
	\[
	\append{\CInt[y,\vec x]{x_i}}{b,\vec a} = \append{\mach{Pr}_{i+1}^{n+1}}{b,\vec a}
	\reddh \Lookinv{a_i}\invredd[\mach{h}] \append{\mach{Pr}_i^n}{\vec a}
	= \append{\CInt{x_i}}{\vec a}.
	\]

Case $M = y$. Then $y\subst{y}{\cons b} = \cons b$ and we have:
	\[
	\append{\CInt[y,\vec x]{y}}{b,\vec a\,} =
	\append{\mach{Pr}_{1}^{n+1}}{b,\vec a\,}\reddh
	\Lookinv{b}
	\invredd[\mach{h}]
	\append{\mach{Cons}_b^{n}}{\vec a\,} =
	\append{\CInt{\cons b}}{\vec a\,}.
	\]

Case $M = PQ$. Then $(PQ)\subst{y}{\cons b} = (P\subst{y}{\cons b})(Q\subst{y}{\cons b})$ and we have:
\[
	\begin{array}{llll}
	\append{\CInt[y,\vec x]{PQ}}{b,\vec a\,}&=&
	\tuple{\Null^{n+1},\Lookup\CInt[y,\vec x]{P},\Lookup\CInt[y,\vec x]{Q},\Null,\ins{Apply}_{n+1},[b,\vec a\,]}&\\
	&\reddh&\append{\CInt[y,\vec x]{P}}{b,\vec a,\Lookup (\append{\CInt[y,\vec x]{Q}}{b,\vec a\,})}\\
	&\equivea&\append{\CInt{P\subst{y}{\cons b}}}{\vec a,\Lookup (\append{\CInt{Q\subst{y}{\cons b}}}{\vec a\,})},\textrm{ by IH,}\\
	&\invredd[\mach c]&\tuple{\Null^{n},\Lookup\CInt{P\subst{y}{\cons b}},\Lookup\CInt{Q\subst{y}{\cons b}},\Null,\ins{Apply}_n,[\vec a]}&\\
	&=&\CInt{(P\subst{y}{\cons b})(Q\subst{y}{\cons b})}\\
	&=& \CInt{(PQ)\subst{y}{\cons b}}\\
	\end{array}
\]

Case $M = \lam z.P$, wlog $z\notin y,\vec x$, so $(\lam z.P)\subst{y}{\cons b} = \lam z.P\subst{y}{\cons b}$.
	By~\ref{lemma:aboutcatint2} both machines reduce to stuck ones.
	So we have to apply an extra $a_{n+1}\in\Addrs$.
\[
	\begin{array}{lcll}
	\append{\CInt[y,\vec x]{\lam z.P}}{b,\vec a,a_{n+1}}&=&\append{\CInt[y,\vec x,z]{P}}{b,\vec a,a_{n+1}}&\\
	&\equivea&\append{\CInt[\vec x,z]{P\subst{x}{\cons b}}}{\vec a,a_{n+1}},&\textrm{by IH,}\\
	&=& \append{\CInt{\lam z.P\subst{y}{\cons b}}}{\vec a,a_{n+1}}
	\end{array}
\]

\ref{lemma:aboutcatint4} By~\ref{lemma:aboutcatint3}. % chktex 2

\ref{lemma:aboutcatint5} By~\ref{lemma:aboutcatint4}, permuting the substitutions. % chktex 2
\end{proof}


\begin{defi}\label{def:thelambdamodel}
Let $\cS = (\Addrs/_{\simea},\bullet, \Int{-}{-})$, where
\[
	\begin{array}{lcl}
	[a]_{\simea}\bullet [b]_{\simea}&=&[\App{a}{b}]_{\simea},\\[3pt]
	\Int{M}{\rho} &\simea& \Lookup(\append{\CInt{M}}{\rho(x_1),\dots,\rho(x_n)}).\\
	\end{array}
\]
By Lemma~\ref{lemma:aboutcatint}, the definition of $\Int{M}{\rho}$ is independent from the choice of $\vec x$, as long as $\FV{M}\subseteq\vec x$. This is reminiscent of the standard way for defining a syntactic interpretation from a categorical one. (Again, see Koymans's~\cite{Koymans82}.)
\end{defi}

\begin{thm}\label{thm:Sinsonoextslm}
$\cS$ is a syntactic \lam-model.
\end{thm}

\begin{proof} We need to check that the conditions~\ref{def:syntmod1}--\ref{def:syntmod6} from Definition~\ref{def:syntmod} are satisfied by the interpretation function given in Definition~\ref{def:thelambdamodel}.

Take $\vec x = x_1,\dots,x_n$, and write $\rho(\vec x)$ for $\rho(x_1),\dots,\rho(x_n)$.

\ref{def:syntmod1} $\Int{x_i}{\rho} \simea \Lookup(\append{\mach{Pr}_i^n}{\rho(\vec x)})\simea \rho(x_i)$, by Remark~\ref{rem:aux_int}\ref{rem:aux_int1}. % chktex 2

\ref{def:syntmod2} $\Int{\cons a}{\rho} \simea \Lookup(\append{\mach{Cons}_a^n}{\rho(\vec x)})\simea a$, by Remark~\ref{rem:aux_int}\ref{rem:aux_int2}. % chktex 2

\ref{def:syntmod3} In the application case, we have: % chktex 2
\[
	\begin{array}{llll}
	\Int{PQ}{\rho} &\simea&\Lookup(\append{\CInt{PQ}}{\rho(\vec x)})\\
	&=& \tuple{\Null^{n},\Lookup \CInt[\vec x]{P},\Lookup \CInt[\vec x]{Q},\Null,\ins{Apply}_n,[\rho(\vec x)]},&\textrm{by Def.~\ref{def:categoricalint},}\\
	&\simea&\App{\Lookup(\append{\mM}{\rho(\vec x)})}{\Lookup(\append{\mN}{\rho(\vec x)})},&\textrm{by Rem.~\ref{rem:aux_int}\ref{rem:aux_int3},}\\
	&=&\Int{P}{\rho}\bullet \Int{Q}{\rho}
	\end{array}
\]

\ref{def:syntmod4} In the \lam-abstraction case we have, for all $a\in\Addrs$: % chktex 2
\[
	\App{\Int{\lam y.P}{\rho}}{a} \simea \append{\CInt{\lam y.P}}{\rho(\vec x),a}
  	 \simea \append{\CInt[\vec x,y]{P}}{\rho(\vec x),a}\\
	 \simea\Int{P}{\rho\repl{y}{a}}.
\]

\ref{def:syntmod5} This follows from Lemma~\ref{lemma:aboutcatint}\ref{lemma:aboutcatint4}. % chktex 2

\ref{def:syntmod6} By definition $\Int{\lam y.M}{\rho} \simea \Lookup(\append{\CInt[\vec x,y]{M}}{\rho(\vec x)})$ and, by Lemma~\ref{lemma:aboutcatint}\ref{lemma:aboutcatint2}, $\append{\CInt[\vec x,y]{M}}{\rho(\vec x)}$ reduces to a stuck \am. % chktex 2
Similarly, for $\Int{\lam x.N}{\rho}$. We conclude by applying the rule \extrule.
\end{proof}

\begin{rem}
\bsub
\item For closed \lam-terms $M\in\Lamo$, we have $\Int{M}{} = \CInt[]{M}$.
\item It is easy to check that $\Int{\comb{K}}{}\simea\Lookup\mach{K}$ and
$\Int{\comb{S}}{}\simea\Lookup\mach{S}$.
\item More generally, all \am s behaving as the combinator $\comb{K}$ (resp.\ $\comb{S}$) are equated in the model.
\esub
\end{rem}

\begin{lem}\label{lem:Snotext}
The syntactic \lam-model $\cS$ is not extensional.
\end{lem}

\begin{proof} It is enough to check that $\cS\not\models \comb{1} = \comb{I}$. Now, we have:
\[
	\begin{array}{lll}
	\Int{\,\comb{1}\,}{} &=& \tuple{\Null^2,\Lookup\mach{Pr}^2_1,\Lookup\mach{Pr}^2_2,\Null,\ins{Apply}_2,[]};\\
	\Int{\,\comb{I}\ }{} &=& \tuple{\Null,\Load 0;\Call 0,[]}.\\
	\end{array}
\]
By applying an indeterminate machine $\mach{x}_n$, the former reduces to a stuck machine, while the latter reduces to $\mach{x}_n$. By Lemma~\ref{lem:about:equivo}\ref{lem:about:equivo2}, they must be different modulo $\equivea$.
\end{proof}

A difficult problem that arises naturally is the characterization of the \lam-theory induced by the \lam-model $\cS$ defined above.

\begin{prop}\label{prop:aboutThS}
The \lam-theory $\Th{\cS}$ is neither extensional nor sensible.
\end{prop}

\begin{proof} $\Th{\cS}$ is not extensional by Lemma~\ref{lem:Snotext}. To show that it is not sensible, it is enough to check that $\cS\not\models \lam x.\Om = \Om$. Notice that
\[
	\begin{array}{llll}
	\CInt[]{\Om}&=&\tuple{\Lookup\CInt[]{\comb{\Delta}},\Lookup\CInt[]{\comb{\Delta}},\Null,\Apply 012;\Call 2,[]},\\
	&\redh&\append{\CInt[]{\comb{\Delta}}}{\Lookup\CInt[]{\comb{\Delta}}},&\textrm{where:}\\
	\CInt[]{\comb{\Delta}}&=&\tuple{\Null,\Lookup\mach{Pr}^1_1,\Lookup\mach{Pr}^1_1,\Null,\ins{Apply}_1,[]}.\\
	\end{array}
\]
By induction on a derivation of $\mM\equivea \mN$, one checks that $\mM\equivea \mN$ and $\mM\reddh \append{\mach{D}_1}{\mach{D}_2}$ with $\mach{D}_1\simea \mach{D}_2\simea\CInt[]{\comb{\Delta}}$entails $\mN\reddh\append{\mach{D}'_1}{\Lookup\mach{D}'_2}$ for some $\mach{D}'_1\simea\mach{D}'_2\simea\CInt[]{\comb{\Delta}}$. We conclude because the machine $\CInt[]{\lam x.\Om}$ is stuck.
\end{proof}

%We believe that $\Th{\cS}$ is semi-sensible, we conjecture that $\Th{\cS} = \blam$.

\section{Consistency Proof via Ordinal Analysis}\label{sec:consistency}
% !TEX root = ../DPIM.tex
% !TEX spellcheck = en-US
\renewcommand\hole[1]{\llparenthesis#1\rrparenthesis}
\newcommand{\Count}{\omega_1}
\newcommand{\occ}[1]{\mathrm{occ}_\xi(#1)}
\newcommand{\pesc}[1][\alpha]{\approx_{#1}}
\newcommand{\svirg}[1][\alpha]{\sim_{#1}}
\newcommand{\eq}[1][\alpha]{\equiv_{#1}}
\newcommand{\ured}[1][M]{\redh^{\mach{#1}}}
\newcommand{\uredd}[1][M]{\reddh^{\mach{#1}}}
\newcommand{\XX}{\mathbb{X}}
\newcommand{\X}{\mach{X}}
\newcommand{\BB}{\mathbb{B}}
\newcommand{\cMX}{\cM[\XX]^\xi}
\newcommand{\Lup}{\underline{\#}}
\newcommand{\Luinv}[1]{\underline{\#}^{-1}(#1)}

In this section we adapt Barendregt's proof of consistency of $\blam\omega$ (the least \lam-theory closed under the $(\omega)$-rule) to prove Lemma~\ref{lem:about:equivo}\ref{lem:about:equivo2}, which entails the consistency of our system.
First, we need to introduce in our setting the notion of \emph{context} and \emph{underlined reduction}, that are omnipresent techniques in the area of term rewriting systems.

\subsection{Contexts and Underlined Head Reductions}

In \lam-calculus a context is a \lam-term possibly containing occurrences of an algebraic variable, called \emph{hole}, that can be substituted by any \lam-term possibly with capture of free variables.
We will define a \emph{context-machine} similarly, namely as an \am{} possibly having a ``hole'' denoted by $\xi$. Formally, we introduce a new machine having no registers or program, only an empty tape (therefore distinguished from all machines populating $\cM$):
\[
	\xi = \tuple{[]}
\]
We then extend our formalism to include machines working either directly or indirectly with one, or more, occurrences of $\xi$. We wish to ensure the invariant that a machine $\mM$ with no occurrences of $\xi$ maintain as address $\Lookup\mM$ --- for this reason we need to extend the range of addresses in a conservative way.

Consider a countable set $\BB$ of addresses such that $\Addrs\cap\BB = \emptyset$, and write $\XX = \Addrs\cup\BB$ for the set of \emph{extended addresses}. As usual, we set \[\XX_\Null = \XX\cup\set{\Null}.\]

\begin{defi}\label{def:context-machine}
\begin{enumerate}[(i)]
\item
	An \emph{extended machine $\X$} is either of the form
	\begin{itemize}
	\item $\appT{\xi}{T}$ or
	\item $\tuple{\vec R,P,T}$
	\end{itemize}
	where $\vec R$ are $\XX_\Null$-valued registers, $P$ is a valid program, $T\in\Tapes[\XX]$ is an $\XX$-valued tape. We write $\cMX$ for the set of all extended machines.
\item Fix a bijective map $\Lup : \cMX \to \XX$ satisfying $\Lup(\mM)=\Lookup\mM$ for all \am{} $\mM\in\cM$. Write $\Luinv{\cdot} : \XX\to\cMX$ for its inverse.
\item The \emph{number of occurrences} of $\xi$ in $\X\in\cMX$ (resp.\ $R_i$, resp.\ $T$), written $\occ{\X}\in\nat\cup\set{\infty}$ ($\occ{R_i}$, $\occ{T}\in\nat\cup\set{\infty}$), is defined as follows:
\[
	\begin{array}{lcl}
	\occ{\appT{\xi}{T}} &=& 1 + \occ{T};\\[3pt]
	\occ{\tuple{\vec R,P,T}} &=& \occ{T} + \sum_{i=0}^{r-1}\occ{R_i};\\[3pt]
	\occ{[a_1,\dots,a_n]} &=& \occ{\Luinv{a_1}}+\cdots+\occ{\Luinv{a_n}};\\[3pt]
	\occ{R_i}&=&\begin{cases}
	0,&\textrm{if }R_i = \Null,\\
	\occ{\Luinv{a}},&\textrm{if }R_i = a\in\XX.\\
	\end{cases}
	\end{array}
\]
Notice that $\occ{\mM}\in\nat$ entails that $\occ{\mM.R_i},\occ{\mM.T}\in\nat$.
\end{enumerate}
\end{defi}

\noindent
The number of occurrences of $\xi$ in an extended machine $\X$ has been defined to handle the fact that recursively dereferencing all the addresses contained in an extended \am{} might result in a non-terminating process (see Remark~\ref{rem:forever}).

\begin{exas}\label{ex:weird}
The following are examples of extended machines:
\begin{enumerate}[(i)]
\item $\xi$, with $\occ{\xi} = 1$;
\item $\append{\mK}{\Lup{\xi},\Lup{(\append{\xi}{\Lup\xi})}}$, with $\occ{\append{\mK}{\Lup{\xi},\Lup{(\append{\xi}{\Lup\xi})}}} = 3$;
\item\label{ex:weird3} for all $n\in\nat$, $\X_n = \tuple{\Lup\xi,\varepsilon,[\Lup{\X_{n+1}}]}$. In this case, $\occ{\X_0} = \infty$.
\end{enumerate}
\end{exas}

\noindent
As previously mentioned, a key property of contexts in \lam-calculus is that one can plug a \lam-term into the hole and obtain a regular \lam-term.
Similarly, given $\mM\in\cMX$ and $\X\in\cMX$, we can define the \am{} $\X\hole{\mM}$ obtained from $\X$ by recursively substituting (even in the registers/tapes) each occurrence of $\xi$ by $\mM$. However, this operation is well-defined only when $\occ{\X}$ is finite, so we focus on extended machines enjoying this property.

\begin{defi}\
\begin{enumerate}[(i)]
\item A \emph{context-machine} is any $\C\in\cMX$ satisfying $\occ{\C}\in\nat$.
\item Given a context-machine $\C$ and $\mM\in\cM$, define the \am{} $\C\hole{\mM}$ as follows:
\[
	\C\hole{\mM} =\begin{cases}
	\appT{\mM}{T\hole{\mM}},&\textrm{if }\C=\appT{\xi}{T},\\
	\tuple{\vec R\hole{\mM},P,T\hole{\mM}},&\textrm{if }\C=\tuple{\vec R,P,T};\\
	\end{cases}
\]
where (assuming $a\in\XX,T = [a_1,\dots,a_n]\in\Tapes[\XX]$ with $\occ{\Cons a T}\in\nat$):
\[
	\begin{array}{lcl}
	a\hole{\mM} &=& \Lup(\Luinv{a}\hole{\mM});\\[3pt]
	R_i\hole{\mM}&=& \begin{cases}
	\Null&\textrm{if }R_i = \Null,\\
	a\hole{\mM}&\textrm{if }R_i = a;\\[3pt]
	\end{cases}~\\
	T\hole{\mM} &=& [a_1\hole{\mM},\dots,a_n\hole{\mM}].\\[3pt]
	\end{array}
\]
\end{enumerate}
\end{defi}

In the following, when writing $\C\hole{\mM}$ (resp.\ $a\hole{\mM}$, $R_i\hole{\mM}$, $T\hole{\mM}$) we silently assume that the number of occurrences of $\xi$ in $\C$ (resp.\ $a,R_i,T$) is finite.
Let us introduce a notion of reduction for context-machines that allows to mimic the underlined reduction from~\cite{BarendregtTh}. The idea is to decompose a machine $\mN$ as $\mN = \C\hole{\underline{\mM}}$ where $\C$ is a context-machine and $\mM$ the underlined sub-machine.
It is now possible to reduce $\C$ independently from $\mM$ until either the machine reaches a final-state or $\xi$ reaches the head-position. In the latter case, we substitute the head occurrence of $\xi$ by $\mM$, and continue the computation.

\begin{defi}\label{def:weirdreds}\
\bsub
\item\label{def:weirdreds1}
	The head reduction $\redh$ is generalized to extended machines in the obvious way, using $\Lup(\cdot)$ rather than $\Lookup{(\cdot)}$ to compute the addresses.
In particular, the machine $\appT{\xi}{T}\not\redh$ is in final state, but it is not stuck.
\item\label{def:weirdreds2}
	Given $\mM\in\cM$ and $\C\in\cM^\xi$, the \emph{$\mM$-underlined (head-)reduction} $\ured$ is defined by adding to~\ref{def:weirdreds1} the rule
\[
	\appT{\xi}{T}\ured\appT{\mM}{T}.
\]
\esub
\end{defi}

\begin{exas} Let $\C = \append{\mS}{\Lup \xi,\Lup\xi,\Lup \mach{x}_n}$. Then $\C\hole{\mach{K}} = \append{\mS}{\Lookup \mach{K},\Lookup\mach{K},\Lookup \mach{x}_n}$.
\bsub
\item $\C\reddh \append{\xi}{\Lup\mach{x}_n,\Lup{(\append{\xi}{\Lup\mach{x}_n})}}$.
\item$\C\uredd[K] \append{\xi}{\Lup\mach{x}_n,\Lup{(\append{\xi}{\Lup\mach{x}_n})}}
\ured[K]\append{\mach{K}}{\Lup\mach{x}_n,\Lup{(\append{\xi}{\Lup\mach{x}_n})}}\uredd[K] \mach{x}_n$.
\esub
\end{exas}

\begin{lem}\label{lem:chemmeserve}
	For $\C,\C'\in\cMX$ and $\mM,\mN\in\cM$, the following are equivalent:
	\begin{enumerate}
	\item\label{lem:chemmeserve1} $\C\hole{\mM}\reddh \mN$;
	\item\label{lem:chemmeserve2} $\C\reddh^\mM\C'$ and $\C'\hole{\mM} = \mN$.
	\end{enumerate}
\end{lem}

\begin{proof} (\ref{lem:chemmeserve1} $\Rightarrow$~\ref{lem:chemmeserve2})
By induction on the length $n$ of the reduction $\C\hole{\mM}\reddh\mN$.

Case $n = 0$. Trivial, take $\C'=\C$.

Case $n > 0$. Let $\C\hole{\mM}\redh\mN'\reddh\mN$. Split into cases depending on $\C$.

Subcase $\C = \appT{\xi}{T}$, therefore $\C\hole{\mM} = \appT{\mM}{T\hole{\mM}}\redh\mN'$. There are two possibilities:
\begin{itemize}
\item $\mM$ is stuck and $T\neq[]$, say, $T=[a_0,\dots,a_n]$. In this case $\C\hole{\mM} = \tuple{\vec R,\Load i;P,[]}$ and $\mN' = \tuple{\vec R\repl{R_i}{a_0\hole{\mM}},\Load i;P,[a_1\hole{\mM},\dots,a_n\hole{\mM}]}$.
On the other side, $\C\ured \appT{\mM}{T}\ured \C''$ for
\[
	\C'' = \tuple{\vec R\repl{R_i}{a_0},\Load i;P,[a_1,\dots,a_n]}
\]
satisfying $\C''\hole{\mM} = \mN'\reddh\mN$. We conclude by induction hypothesis.
\item $\mM\redh\mM'$. In this case $\mN' = \appT{\mM'}{T\hole{\mM}}$ and $\C\ured \appT{\mM}{T}\ured \C''$ for $\C'' = \appT{\mM'}{T}$ satisfying $\C''\hole{\mM} = \mN'\reddh\mN$. We conclude by induction hypothesis.

Subcase $\C = \tuple{\vec R,P,T}$. By case analysis on $P$. All cases follow easily from the induction hypothesis.
\end{itemize}

(\ref{lem:chemmeserve2} $\Rightarrow$~\ref{lem:chemmeserve1})
By induction on the length $n$ of the reduction $\C\reddh^\mM\C'$.

Case $n=0$. Trivial, take $\mN=\C\hole{\mM}$.

Case $n>0$, i.e.\ $\C\ured\C''\uredd\C'$, where the latter reduction is shorter.

Proceed by case analysis on the shape of $\C$.

Subcase $\C = \appT{\xi}{T}$ and $\C''=\appT{\mM}{T}$.
Then $\mN = \C''\hole{\mM} = \appT{\mM}{T\hole{\mM}} = \C\hole{\mM}$.
Conclude by induction hypothesis.

Subcase $\C = \tuple{\vec R,P,T}$. By case analysis on $P$. All cases follow easily from the induction hypothesis.
\end{proof}

\subsection{Ordinal analysis}

As mentioned in Remark~\ref{rem:aboutordinals}, a derivation of $\mM\equivea\mN$ has the structure of a well-founded $\omega$-branching tree.
Unfortunately, this makes it difficult to prove even simple properties like Lemma~\ref{lem:about:equivo}\ref{lem:about:equivo2}.
We need a more refined system exposing the underlying ordinal and handling the applications of the (Transitivity) rule separately.

\begin{defi}
\begin{enumerate}[(i)]
\item Let $\Count$ be the set of all countable ordinals.
\item If $\pi$ is a derivation of $\mM\equivea\mN$, we define its \emph{length} $\ell(\pi)\in\omega_1$ in the usual inductive way for the rules \redwerule, (Refl.), (Symm.), (Trans.). Concerning the rule $\extrule$ having countably many premises, we set:
\[
	\ell\left(
	\begin{array}{c}
	\infer{\mM \equivea \mN}{
		\mach{M},\mach{N}\reddh\stuck{}&
		\forall a\in\Addrs\,.\, \infer{\append{\mM}{a} \equivea \append{\mN}{a}}{\pi_a}
	}
	\end{array}
	\right) = \sup_{a\in\Addrs}(\ell(\pi_a)+1)
\]
It is easy to check that, if a derivation $\pi$ has premises $(\pi_i)_{i\in \cI}$ for some countable set $\cI$ then $\ell(\pi) > \ell(\pi_i)$ for every $i\in\cI$.
\item For all $\alpha\in\Count$, define $\eq,\svirg,\pesc\,\subseteq\cM^2$ as the least reflexive and symmetric relations closed under the rules of Figure~\ref{fig:Pesiolino}.
\end{enumerate}
\begin{figure}
\begin{gather*}
% General
\infer[(\approx_0)]{\mM\pesc[0] \mN}{\mM\equiva \mN}
\qquad
\infer[(\subseteq^{\approx}_\alpha)]{\mM\svirg\mN}{\mM\pesc\mN}
\qquad
\infer[(\subseteq^{\sim}_\alpha)]{\mM\eq\mN}{\mM\svirg\mN}\\[3pt]
% Pesciolino
\infer[(\approx_\alpha)]{\mM\pesc\mN}{\mM,\mN\reddh\stuck{}&\forall a\in\Addrs,\,\exists\gamma < \alpha\,.\,\append{\mM}{a} \eq[\gamma] \append{\mN}{a}}\\[3pt]
\begin{array}{ccc}
	% Svirgola
	\infer[(R_\alpha^\sim)]{\mM\repl{R_i}{a}\svirg\mM\repl{R_i}{b}}{\Lookinv a\svirg \Lookinv b}
	&\quad&
	\infer[(@_\alpha^\sim)]{\append{\mM}{a}\svirg\append{\mM}{b}}{\Lookinv a\svirg \Lookinv b}\\[3pt]
	%%%%
	\infer[(T_\alpha^\sim)]{\appT{\mM}{T}\svirg\appT{\mN}{T}}{\mM\svirg \mN&T\in\Tapes}
	&&
	\infer[(T_\alpha)]{\appT{\mM}{T}\eq\appT{\mN}{T}}{\mM\eq \mN&T\in\Tapes}\\[3pt]
	%%%%
	\infer[(R_\alpha)]{\mM\repl{R_i}{a}\eq\mM\repl{R_i}{b}}{\Lookinv a\eq \Lookinv b}
	&&
	\infer[(@_\alpha)]{\append{\mM}{a}\eq\append{\mM}{b}}{\Lookinv a\eq \Lookinv b}\\[3pt]
\end{array}~\\
%%%%
\infer[(\le^\approx_\alpha)]{\mM \pesc \mN}{\mM\pesc[\gamma]\mN&\gamma \le \alpha}
\quad
\infer[(\le^\sim_\alpha)]{\mM \svirg \mN}{\mM\svirg[\gamma]\mN&\gamma \le \alpha}
\quad
\infer[(\le_\alpha)]{\mM \eq \mN}{\mM\eq[\gamma]\mN&\gamma \le \alpha}
%%%%
\\
\infer[(\mathrm{Tr}_\alpha)]{\mM \eq \mN}{\mM\eq\mZ&\mZ\eq\mN}\\[-5ex]
\end{gather*}
\caption{Rules satisfied by $\pesc$, $\svirg$ and $\eq$, beyond reflexivity and symmetry.}\label{fig:Pesiolino}
\end{figure}
\end{defi}
The intuitive meanings of the relations $\eq,\svirg,\pesc$ are the following:
\begin{itemize}
\item $\mM\eq\mN\iff\mM\equivea\mN$ is derivable using the rule $\extrule$ at most $\alpha$ times;
\item $\mM\svirg\mN\iff\mM\eq\mN$ is derivable without using transitivity;
\item $\mM\pesc\mN\iff\mM\equivea\mN$ in case $\alpha = 0$. Otherwise, if $\alpha>0$ then
\item $\mM\pesc\mN\iff\mM\svirg\mN$ follows directly from the rule $\extrule$.
\end{itemize}

\noindent
More precisely, the rules $(\approx_0)$, $(\subseteq^{\approx}_\alpha)$, $(\subseteq^{\sim}_\alpha)$ express the fact that $\equiva\,\subseteq\,\pesc\,\subseteq\,\svirg\,\subseteq\,\eq$.
The rule $(\approx_\alpha)$ allows to prove $\mM \pesc \mN$, provided that both machines eventually get stuck and that $\appT{\mM}{[a]} \eq[\gamma_a] \appT{\mN}{[a]}$ is provable for every address $a$, using a smaller ordinal $\gamma_a < \alpha$.
The rules $(R_\alpha)$, $(@_\alpha)$ and $(T_\alpha)$ (resp.\ $(R_\alpha^\sim)$, $(@_\alpha^\sim)$ and $(T_\alpha^\sim)$) represent the contextuality of the relation $\eq$ (resp.\ $\svirg$).
The rules $(\le^\approx_\alpha)$, $(\le^\sim_\alpha)$ and $(\le_\alpha)$ specify that incrementing the ordinal (from top to bottom) is always allowed.
Finally, $(\mathrm{Tr}_\alpha)$ gives the transitivity of $\eq$.

The following lemma describes formally the intuitive meaning discussed above.
\begin{lem}\label{lem:relalphaprops}
Let $\mM,\mN\in\cM$
\begin{enumerate}[(i)]
\item\label{lem:relalphaprops1}\
	 $\mM\equivea\mN\iff\exists\alpha\in\Count\,.\,\mM\eq\mN$.
\item\label{lem:relalphaprops2}\
	$\mM\eq[0]\mN\iff\mM\equiv_\Addrs\mN$.
\item\label{lem:relalphaprops3}\
	$\mM\eq \mN\iff\exists n\ge0, \mZ_1,\dots,\mZ_n\in\cM\,.\, \mM\svirg\mZ_1\svirg\cdots\svirg\mZ_n =\mN$.
\item\label{lem:relalphaprops4}~\\[-3ex]
$
	\begin{array}{ll}
		\mM\svirg\mN\iff&\exists \mach{C}\in\cMX,\mach{M}',\mach{N}'\in\cM\,.\,\\
		&\mM=\mach{C}\hole{\mM'}\land\mN = \mach{C}\hole{\mN'} \land \mM'\pesc\mN'.\\
		\end{array}
	$
\item\label{lem:relalphaprops5}~\\[-2.7ex]
$
	\begin{array}{lcl}
		\mM\pesc\mN\land \alpha\neq 0&\iff&\mM,\mN\reddh\stuck{}\ \land\\
		&&\forall a\in\Addrs,\exists\gamma<\alpha\,.\, \append{\mM}{a}\eq[\gamma]\append{\mN}{a}.\\
		\end{array}
	$
\end{enumerate}
\end{lem}

\begin{proof}\ref{lem:relalphaprops1} $(\Leftarrow)$ Easy.

$(\Rightarrow)$ By induction on the length of a derivation of $\mM\equivea\mN$.

Case \redwerule. I.e., there exists $\mZ\in\cM$ such that $\mM\reddh\mZ\eqea\mN$.
By Theorem~\ref{thm:CR}, we have $\mM\equiva\mZ$ whence $\mM\eq[0]\mZ$ by $(\pesc[0])$, which implies $\mM\eq\mZ$ for all $\alpha\in\Count$ using the rule $(\le_\alpha)$. Now, consider the set
\[
	\cR = \set{ i \st \mZ.R_i \neq\Null} = \set{ i \st \mN.R_i \neq\Null}
\]
Note that $\cR= \set{i_1,\dots,i_k}$ for some $k<\mZ.r_0 (=\mN.r_0)$. For every $i\in\cR$, let $\mZ.R_i = a_i$ and $\mN.R_i = a'_i$. Also, let $\mZ.T = [b_1,\dots,b_m]$ and $\mN.T = [b'_1,\dots,b'_m]$. By assumption, $a_i\simea a'_i$ and $b_j\simea b'_j$ for every $i\in\cR$, and $j\,(1\le j\le m)$.
By induction hypothesis, $\Lookinv{a_i} \eq[\gamma_i] \Lookinv{a'_i}$ and $\Lookinv{b_j} \eq[\delta_j] \Lookinv{b'_j}$. Using the rule $(<_\alpha)$, the same holds for $\eq[\alpha]$ setting $\alpha = \sup_{i\in\cR,1\le j\le m} \set{\gamma_i,\delta_j}$. Putting everything together, we obtain:
\[
	\begin{array}{lcll}
	\mM&\eq&\mZ = \tuple{\mZ.\vec R,P,[b_1,\dots,b_m]}\\
	&\eq&\tuple{\mZ.\vec R\repl{R_{i_1}}{a'_{i_1}},P,[b_1,\dots,b_m]},&\textrm{by $(R_\alpha)$,}\\
	&\eq&\cdots&\qquad\vdots\\
	&\eq&\tuple{\mZ.\vec R[R_i:= a'_i]_{i\in\cR},P,[b_1,\dots,b_m]},&\textrm{by $(R_\alpha)$,}\\
	&=&\tuple{\mN.\vec R,P,[b_1,\dots,b_m]},&\textrm{by definition,}\\
	&\eq&\tuple{\mN.\vec R,P,[b'_1,b_2,\dots,b_m]},&\textrm{by $(T_\alpha)$,}\\
	&\eq&\cdots&\qquad\vdots\\
	&\eq&\tuple{\mN.\vec R,P,[b'_1,\dots,b'_m]},&\textrm{by $(T_\alpha)$,}\\
	&=&\mN,&\textrm{by definition.}\\
	\end{array}
\]
We conclude by applying the transitivity rule $(\mathrm{Tr}_\alpha)$ that $\mM \eq \mN$.

Case \extrule. By induction hypothesis, for every $a\in\Addrs$, there exists $\gamma_a\in\Count$ such that $\append{\mM}{a} \eq[\gamma_a]\append{\mN}{a}$.
For $\gamma = \sup_{a\in\Addrs}\gamma_a$, we get $\append{\mM}{a} \eq[\gamma]\append{\mN}{a}$ by $(\le_\alpha)$. By $(\pesc)$ we get $\mM\pesc\mN$ for $\alpha=\gamma+1\in\Count$, conclude by $(\subseteq^\approx_\alpha)$ and $(\subseteq^\sim_\alpha)$.

(Reflexivity), (Symmetry) and (Transitivity) follow from the respective property of $\eq$.

Concerning items~\ref{lem:relalphaprops2}--\ref{lem:relalphaprops5} the implication $(\Leftarrow)$ is trivial. We analyze $(\Rightarrow)$.

\ref{lem:relalphaprops2} % chktex 2
	By induction on a derivation of $\mM\eq[0]\mN$, using Theorem~\ref{thm:CR}.

\ref{lem:relalphaprops3} % chktex 2
    By induction on a derivation of $\mM\eq\mN$.

    Case $(\subseteq^\sim_\alpha)$. Trivial.

    Case $(R_\alpha)$. I.e., $\mM = \mZ\repl{R_i}{a}$, $\mN = \mZ\repl{R_i}{b}$ and $\Lookinv{a} \eq \Lookinv{b}$. By induction hypothesis, there exist $c_1,\dots,c_k\in\Addrs$ such that
    \[
    	\Lookinv{a}\svirg\Lookinv{c_1}\svirg\cdots\svirg\Lookinv{c_k}=\Lookinv{b}.
    \]
    The case follows by applying the rule $(R^\sim_\alpha)$.

    Case $(@_\alpha)$. Analogous, by applying $(@^\sim_\alpha)$.

    Case $(T_\alpha)$. Analogous, by applying $(T^\sim_\alpha)$.

    Case $(\mathrm{Tr}_\alpha)$. Straightforward from the IH\@.

    Case $(\le_\alpha)$. By IH and $(\le^\sim_\alpha)$.

	Cases (Reflexivity), (Symmetry). Straightforward from the IH\@.

\ref{lem:relalphaprops4} % chktex 2
	By induction on a derivation of $\mM\svirg\mN$.

	Case $(\subseteq^\approx_\alpha)$. Take $\C = \xi$.

	Case $(R^\sim_\alpha)$. I.e., $\mM = \mZ\repl{R_i}{a}$, $\mN = \mZ\repl{R_i}{b}$ and $\Lookinv{a} \svirg \Lookinv{b}$. By induction hypothesis, there exist $\C'\in\cMX$ having address $c = \Lup{\C'}\in\XX$, $\mM',\mN'\in\cM$ such that $\C'\hole{\mM'} = \Lookinv{a}$, $\C'\hole{\mN'} = \Lookinv{b}$ and $\mM' \pesc\mN'$. We conclude by taking $\C = \mZ\repl{R_i}{c}$.

	Case $(@^\sim_\alpha)$. Analogous.

	Case $(T^\sim_\alpha)$. Take $\C = \appT{\C'}{T}$, where $\C'$ is obtained from the IH\@.

	Case $(\le^\sim_\alpha)$. It follows from the IH, by applying $(\le^\sim_\alpha)$ and $(\le^\approx_\alpha)$.

	Cases (Reflexivity), (Symmetry). Straightforward from the IH\@.

\ref{lem:relalphaprops5} Immediate. % chktex 2
\end{proof}

Consider now a scenario where $\C\hole \mM\reddh \C'\hole{\mM}$.
Assuming $\mM\pesc\mN$, one might expect that also $\C\hole \mN\reddh \C'\hole{\mN}$ holds.
In general, this is not the case because $\mM$ and $\mN$ might reach the head position and get control of the computation.
Using the underlined (head-)reduction from Definition~\ref{def:weirdreds}\ref{def:weirdreds2} we can substitute $\mN$ for $\mM$ along the reduction (when it comes in head position) and construct a proof of $\C\hole \mN \eq[\gamma] \C'\hole\mN$ having a lower ordinal $\gamma < \alpha$.

\begin{lem}\label{lem:black_magic}
Let $\alpha > 0$, $\C\in\cMX$, $\mM,\mN\in\cM$ such that $\mM\pesc\mN$.
If $\C\redh^\mM\C'$ and $\C'\hole{\mM}\not\reddh\stuck{}$, then there exists $\gamma < \alpha$ such that $\C\hole \mN \eq[\gamma] \C'\hole\mN$.
\end{lem}

\begin{proof} By cases on the shape of $\C$.

Case $\C = \appT{\xi}{T}$ for some $T\in\Tapes[\XX]$ and $\C' = \appT{\mM}{T}$.
From  $\mM\pesc\mN$ and Lemma~\ref{lem:relalphaprops}\ref{lem:relalphaprops5}, we get that $\mM\reddh{\stuck{\mM'}}$ for some $\mM'\in\cM$.
Since $\C'\hole{\mM} = \appT{\mM}{(T\hole{\mM})}$ cannot reduce to a stuck \am, we must have $T\hole{\mM}\neq[]$.
In other words, $T = [a_0,\dots,a_n]$ for some $n\ge 0$.
Notice that, for all $a_i\in\Tapes[\XX]$, we have $a_i\hole{\mN}\in\Addrs$ (by construction).
By Lemma~\ref{lem:relalphaprops}\ref{lem:relalphaprops5}, there exists $\gamma<\alpha$ such that $\append{\mN}{a_0\hole{\mN}} \eq[\gamma] \append{\mM}{a_0\hole{\mN}}$. By definition:
\[
	\C\hole{\mN} = \appT{\mN}{T\hole{\mN}},\textrm{ and }
	\C'\hole{\mN}=\appT{\mM}{T\hole{\mN}}.
\]
So we construct the proof:
\[
	\infer[(T_\gamma)]{\append{\mN}{a_0\hole{\mN},\dots,a_n\hole{\mN}} \eq[\gamma] \append{\mM}{a_0\hole{\mN},\dots,a_n\hole{\mN}}}{
	\append{\mN}{a_0\hole{\mN}}\eq[\gamma] \append{\mM}{a_0\hole{\mN}}
	}
\]

In all the other cases, $\C\hole{\mN}\to_h\C'\hole{\mN}$, therefore $\C\hole{\mN} \eq[0]\C\hole{\mN}$.
\end{proof}

\begin{cor}\label{cor:black_magic}
Let $n\in\nat$, $\alpha > 0$, $\C\in\cMX$, $\mM,\mN\in\cM$.
If $\C\hole{\mM}\reddh\mach{x}_n$ and $\mM\pesc\mN$ then there exists $\gamma < \alpha$ such that $\C\hole{\mN} \eq[\gamma] \mach{x}_n$.
\end{cor}

\begin{proof} Assume $\C\hole{\mM}\reddh\mach{x}_n$. Equivalently, by Lemma~\ref{lem:chemmeserve}, we have $\C\reddh^\mM\mach{x}_n$.
By definition, there exists $\C_1,\dots,\C_k\in\cMX$ such that
\[
	\C = \C_1\to_h^\mM\cdots\to_h^\mM \C_k = \mach{x}_n
\]
Notice that $\C_i\hole{\mM}\reddh \mach{x}_n$ and, since $\lnot\stuck{\mach{x}_n}$, we have $\C_i\hole{\mN}\not\reddh\stuck{}$.
By Lemma~\ref{lem:black_magic}, there exists $\gamma_1,\dots,\gamma_k<\alpha$ such that $\C_i\hole{\mN}\eq[\gamma_i]\C_{i+1}\hole{\mN}$.
By transitivity $(\mathrm{Tr}_\alpha)$ and $(\le_\alpha)$ we obtain $\mM\eq\mach{x}_n$ for $\alpha = \sup_i{\gamma_i}$.
\end{proof}

\begin{prop} Let $\mM,\mN\in\cM$, $\alpha\in\Count$ and $n\in\nat$.
If $\mM\eq \mN$ and $\mN\reddh\mach{x}_n$ then $\mM\reddh\mach{x}_n$.
\end{prop}

\begin{proof} We proceed by induction on $\alpha$. Since we perform a double induction, the induction hypothesis with respect to this induction is called the $\alpha$-IH ($\alpha$-inductive hypothesis).

Case $\alpha = 0$. By Lemma~\ref{lem:relalphaprops}\ref{lem:relalphaprops2}, we get $\mM\equivea \mN\reddh\mach{x}_n$, so we conclude $\mM\reddh\mach{x}_n$ by confluence (Theorem~\ref{thm:CR}) and $\red[i]$-postponement (Lemma~\ref{lem:standardization}).

Case $\alpha > 0$. By Lemma~\ref{lem:relalphaprops}\ref{lem:relalphaprops3}, there exist $\mZ_1,\dots,\mZ_k$ such that
\begin{equation}\label{eq:svirg}
	\mM \svirg\mZ_1\svirg\cdots\svirg\mZ_k=\mN\reddh\mach{x}_n
\end{equation}
By induction on $k$, we prove that~\eqref{eq:svirg} implies $\mM\reddh\mach{x}_n$.
We call this $k$-IH\@.

Subcase $k =0$. Then $\mM =\mN\reddh\mach{x}_n$ and we are done.

Subcase $k >0$. From the $k$-IH we derive $\mZ_1\reddh\mach{x}_n$.
From $\mM\svirg\mZ_1$ and Lemma~\ref{lem:relalphaprops}\ref{lem:relalphaprops4}, there is a context-machine $\C$ such that $\mM = \C[\mM']$ and $\mZ_1 = \C[\mN']$ with $\mM'\pesc\mN'$ and $\C[\mN']\reddh\mach{x}_n$.
By applying Lemma~\ref{lem:black_magic} we obtain $\C[\mM'] \eq[\gamma] \mach{x}_n$ for some $\gamma<\alpha$.
We conclude by applying the $\alpha$-IH\@.
\end{proof}

From this proposition, Lemma~\ref{lem:about:equivo}\ref{lem:about:equivo2} follows by applying Lemma~\ref{lem:relalphaprops}\ref{lem:relalphaprops1}.

\section{Conclusions and Further Works}\label{sec:conclusions}
\section{Conclusions}
\label{sec:conclusions}

In this paper, we apply shared-workload techniques at the \sql level for
improving the throughput of \qaasl systems without incurring in additional
query execution costs. Our approach is based on query rewriting for grouping
multiple queries together into a single query to be executed in one go. This
results in a significant reduction of the aggregated data access done by the
shared execution compared to executing queries independently.

%execution times and costs of the shared scan operator when
%varying query selectivity and predicate evaluation. We observed that for
%\athena, although the cost only depends on the amount of data read, it is
%conditioned to its ability to use its statistics about the data. In some cases
%a wrong query execution plan leads to higher query execution costs, which the
%end-user has to pay. 

%\bigquery's minimum query execution cost is determined by
%the input size of a query.  However, the query cost can increase depending not
%just in the amount of computation it requires, but in the mix of resources the
%query requires.  

We presented a cost and runtime evaluation of the shared operator driving data access costs. 
Our experimental study using the TPC-H benchmark confirmed the benefits of our
query rewrite approach. Using a shared execution approach reduces significantly
the execution costs. For \athena, we are able to make it 107x cheaper and for
\bigquery, 16x cheaper taking into account Query 10 which we cannot execute,
but 128x if it is not taken into account. Moreover, when having queries that do
not share their entire execution plan, i.e., using a single global plan, we
demonstrated that it is possible to improve throughput and obtain a 10x cost
reduction in \bigquery.

%We followed the TPC systems pricing guideline for
%computing how expensive is to have a TPC-H workload working on the evaluated
%\qaasl systems. The result is that even though we are able to reduce overall
%costs a TPC-H workload in 15x for \bigquery (128x excluding query 10 which we
%could not optimize) and in 107x for \athena, the overall price is at least 10x
%more expensive than the cheapest system price published by the TPC.

There are multiple ways to extend our work. The first is
to implement a full \sql-to-\sql translation layer to encapsulate the proposed
per-operator rewrites.  Another one is to incorporate the initial work on
building a cost-based optimizer for shared execution
\cite{Giannikis:2014:SWO:2732279.2732280} as an external component for \qaasl
systems.  Moreover, incorporating different lines of work (e.g., adding
provenance computation \cite{GA09} capabilities) also based on query
rewriting is part of our future work to enhance our system.


%% in general the use of bibtex is encouraged
\bibliographystyle{alphaurl}
\bibliography{biblio}
%% The Appendices part is started with the command \appendix;
%% appendix sections are then done as normal sections
%\appendix

%\section{Technical Appendix}\label{tech_app}
%\input{include/techapp}


\end{document}
