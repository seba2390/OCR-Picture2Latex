% !TEX root = ../DPIM.tex
% !TEX spellcheck = en-US

In this section we show how to construct a combinatory algebra based on the \am s formalism. % chktex 1
Recall that the \am s $\mK$ and $\mS$ have been defined in Example~\ref{ex:ilprimoesempiononsiscordamai}. Consider the algebraic structure % chktex 1
\[
	\cA = (\Addrs,\App{\,}{\,},\Lookup\mach{K},\Lookup\mS)
\]
Since the application $(\App{}{})$ is total, $\cA$ is an applicative structure.
However, it is \emph{not} a combinatory algebra.
For instance, the $\lama$-term $\comb{K}\cons a\cons b$ is interpreted as the address of the machine $\append{\mach{K}}{a,b}$, which is \emph{a priori} different from the address ``$a$'' because no computation is involved.
Therefore, we need to quotient the algebra $\cA$ by an equivalence relation equating at least all addresses corresponding to the same machine at different stages of the execution.

In the following, we denote by $\equiv_{\rel R}$ an arbitrary binary relation on $\cM$. The symbol ${\rel R}$ has no formal meaning, it is simply evocative of a relation.
In the next definition, we are going to associate with every $\equiv_{\rel R}$ two relations, respectively denoted $\simeq_{\rel R}\,\subseteq \Addrs^2$ and $=_{\rel R}\,\subseteq \cM^2$.

\begin{defi}\label{def:inducingequivalences}
Every binary relation $\equiv_{\rel R}\,\subseteq\cM^2$ on \am s induces a relation $\simeq_{\rel R}\,\subseteq \Addrs^2$ defined by % chktex 1
\[
	a\simeq_{\rel R} b\iff \Lookinv{a}\equiv_{\rel R} \Lookinv{b}
\]
which is then extended to:
\bsub
\item $\Addrs_\Null$-valued registers:
\[
	R\simeq_{\rel R} R' \iff (R = \Null = R')\lor (R = a \simeq_{\rel R} b =R');
\]
\item Tuples:
\[
	a_1,\dots,a_n \simeq_{\rel R} b_1,\dots,b_m \iff (n = m)\land (\forall i\in\set{1,\dots,n}\,.\, a_i\simeq_{\rel R} b_i);
\]
(This also applies to tuples of $\Addrs_\Null$-valued registers $\vec R\simeq_{\rel R} \vec R'$.)
\item  $\Addrs$-valued tapes:
	\[
		[a_1,\dots,a_n]\simeq_{\rel R} [b_1,\dots,b_m]\iff \vec a \simeq_{\rel R} \vec b \textrm{ (seen as tuples).}
	\]
\esub
\noindent
In its turn, $\simeq_{\rel R}$ induces a relation $=_{\rel R}\ \subseteq\cM^2$ defined by setting (for all machines $\mM,\mN\in\cM$):
\[
	\mM =_\rel{R} \mN \iff (\mM.\vec R \simeq_{\rel R}\mN.\vec R)\land (\mM.P = \mN.P)\land(\mM.T \simeq_{\rel R}\mN.T)
\]
\end{defi}

In particular, $\mM =_\rel{R} \mN$ entails that $\mM$ and $\mN$ share the same internal program, the number of internal registers, and the length of their input tape.

\begin{lem}\label{lem:equivalence}
If the relation $\equiv_\rel{R}$ is an equivalence then so are $\simeq_\rel{R}$ and $=_\rel{R}$.
\end{lem}
\begin{proof} Assume that $\equiv_\rel{R}$ is an equivalence. Then, the fact that $\simeq_\rel{R}$ is an equivalence follows from its definition since $\Lookinv{\cdot}$ is a bijection. Concerning the relation $=_\rel{R}$, reflexivity, symmetry and transitivity follow immediately from the same properties of $\simeq_{\rel{R}}$ and $=$.
\end{proof}

\begin{defi}\label{def:equiv:Addrs}
Define $\equiva\ \subseteq\cM^2$ as the least equivalence closed under:
\[
	\infer[\redrule]{\mM\equiv_\Addrs	 \mN}{\mM\reddh \mZ =_\Addrs \mN}
\]
\end{defi}
We say that $\mM,\mN$ are \emph{evaluation equivalent} whenever $\mM\equiva\mN$.

\begin{rem}\
\begin{enumerate}[(i)]
\item Reflexivity can be treated as a special case of the rule $\redrule$ since $\mM\reddh\mM=_\Addrs \mM$.
\item It follows from the definition that $=_\Addrs\,\subseteq\ \equiva$ and that $\mM\reddh \mN$ entails $\mM\equiva \mN$.
\end{enumerate}
\end{rem}

\begin{exas}\label{ex:calculs}
From the calculations in Examples~\ref{ex:somemachines}, it follows that
\[
	\begin{array}{lcl}
	\append{\mK}{\Lookup\mach{x}_1,\Lookup\mach{x}_2}&\equiva& \mach{x}_1, \\
	\append{\mS}{\Lookup\mach{x}_1,\Lookup\mach{x}_2,\Lookup\mach{x}_3}&\equiva& \append{(\append{\mach{x}_1}{\Lookup \mach{x}_3})}{\Lookup(\append{\mach{x}_2}{\Lookup \mach{x}_3})}.\\
	\end{array}
\]
\end{exas}

\begin{lem}
The relation $\sima$ is a congruence on $\cA= (\Addrs,\App{\,}{\,},\Lookup\mach{K},\Lookup\mS)$.
\end{lem}
\begin{proof}
By definition $\equiva$ is an equivalence, whence so is $\sima$ by Lemma~\ref{lem:equivalence}.
Let us check that $\sima$ is compatible w.r.t.\ $(\App{}{})$.
Consider $a \sima a'$ and $b\sima b'$.
Call $\mM = \Lookinv{a}$ and $\mN = \Lookinv{a'}$ and proceed by induction on a derivation of $\mM\equiva\mN$, splitting into cases depending on the last applied rule.

\redrule{} By definition, there exists $\mZ\in\cM$ such that $\mM\reddh \mach Z =_\Addrs \mN$. By Lemma~\ref{lem:about_red}\ref{lem:about_red2}, $\append{\mM}{b} \reddh \append{\mZ}{b} =_\Addrs \append{\mN}{b'}$ whence $\App{a}{b}\sima\App{a'}{b'}$.

(Transitivity) and (Symmetry) follow from the induction hypothesis.
\end{proof}

In order to prove that the congruence $\sima$ is non-trivial, we are going to characterize the equivalence $\mM\equiva\mN$ it in terms of confluent reductions.
For this purpose, we extend $\redh$ in such a way that reductions are also possible within registers and elements of the input-tape of an \am.

\begin{defi} Define the reduction relation $\red[c]\,\subseteq\cM^2$ as the least relation containing $\redh$ and closed under the following rules:
\begin{gather*}
\infer[{(\red[i]^R)}]{\tuple{R_0,\dots,R_{r-1},P,T} \red[c] \tuple{\vec R\repl{R_i}{\Lookup\mM},P,T}}{R_i = a\in\Addrs&0\le i<r& \Lookinv{a}\red \mM}\\[3pt]
\infer[{(\red[i]^T)}]{\tuple{\vec R,P,[a_0,\dots,a_n]} \red[c] \tuple{\vec R,P,[a_0,\dots,a_{i-1},\Lookup\mM,a_{i+1},\dots,a_n]}}{0\le i\le n& \Lookinv{a_i}\red \mM}
\end{gather*}
We write $\mM\red[i]\mN$ if $\mN$ is obtained from $\mM$ by directly applying one of the above rules --- this is called an \emph{inner} step of computation.
The transitive and reflexive closure of $\red$ and $\red[i]$ are denoted by $\redd$ and $\redd[i]$, respectively.
\end{defi}

%\begin{lem}\label{lem:CRmoduloA}
%For all $\mM,\mM',\mN\in\cM$, we have:
%\bsub
%\item\label{lem:CRmoduloA1}
%	If $\mM =_\Addrs \mN$, $\mM\redh\mM'$ and $\mM.P\neq\Call i$ for any index $i$, then there exists $\mN'\in\cM$ such that $\mN\redh\mN'$ with $\mM'=_\Addrs\mN'$.
%\item\label{lem:CRmoduloA2} [Proof wrong!]
%$
%	\mM\equiva\mN\iff \exists \mZ_1,\mZ_2\in\cM\,.\,\mM\reddh \mZ_1\und\mN\reddh \mZ_2\und \mZ_1 =_\Addrs \mZ_2.
%$
%\esub
%\end{lem}

%\begin{proof} $(i)$ By cases on $\mM.P\,(=\mN.P)$. Recall that it is valid w.r.t.\ $\mM.\vec R$.

%Case $\mM.P = \Load i;P_1$, then $\mM.T = \Cons {a_1}{T_1}$ for some $a_1\in\Addrs,T_1\in\Tapes$. This entails $\mN = \tuple{\vec R',\mM.P,\Cons {a_2}{T_2}}$ with $\mM.\vec R\sima \vec R'$, $a_1\sima a_2$ and $T_1\sima T_2$.
%Thus, we can take $\mN' = \tuple{\vec R'\repl{R_i}{a_2},\mM.P_1,T_2}$.

%Case $\mM.P = \Apply ijk;P_1$. Then $\mM.R_i = a_1\neq\Null$ and $\mM.R_j = a_2 \neq\Null$ with $i,j<\mM.r=\mN.r$.
%By assumption $\mM.\vec R\sima \mN.\vec R$ and $\mM.T\sima \mN.T$, in particular $\mN.R_i = a_1'\neq\Null,\mN.R_j=a_2'\neq\Null$ with $a_1\sima a_1'$ and $a_2\sima a_2'$. Take $\mN' =\tuple{\mN.\vec R\repl{R_k}{\App{a_1'}{a_2'}},\mM.P_1,\mN.T}$

%%Case $\mM.P = \Call i$. Then $i<\mM.r = \mN.r$, $\mM.R_i = a\in\Addrs$ and $\mM'= \Lookinv{a}$. Moreover $\mN.R_j = a'$ for some $a'\sima a$ and we can take $\mN' = \Lookinv{a'}$.

%$(ii)$ $(\Rightarrow)$ By induction on $\mM\equiva\mN$, by cases on the last rule applied.

%Case \redrule. Then $\mM\reddh \mZ =_\Addrs \mN$ and we can take $\mZ = \mZ_1$ and $\mZ_2 = \mN$.

%Case (Symmetry). Immediate, from the induction hypothesis.

%Case (Transitivity). Assume $\mM\equiva\mach{X}$ and $\mach{X} \equiva \mN$. By induction hypothesis on the former we get $\mM\reddh \mM'$ and $\mach{X}\reddh \mach{X}_1$ with $\mM' =_\Addrs \mach{X}_1$. By induction hypothesis on the latter, $\mach{X}\reddh\mach{X}_2$ and $\mN\reddh\mZ_2$ with $\mach{X}_2\equiva \mZ_2$.
%By Lemma~\ref{lem:about_red}\ref{lem:about_red1} we have, say, $\mach{X}_1\reddh \mach{X}_2$.
%By $(i)$, there exists $\mZ_1$ such that $\mM_1\reddh\mZ_1$ and $\mZ_1 =_\Addrs \mach{X}_2$. In diagrammatic form:
%\[
%	\xymatrix{
%	\mM\ar@{->>}[d]_{\mach c}&\equiva&\mach{X}\phantom{_1}\ar@{->>}[d]_{\mach c}%&\equiva&\mN\ar@{->>}[dd]_{\mach c}\\
%	\mM_1\ar@{->>}[d]_{\mach c}&=_\Addrs&\mach{X}_1\ar@{->>}[d]_{\mach c}&&\\
%	\mZ_1&=_\Addrs&\mach{X}_2&=_\Addrs&\mZ_2\\
%	}
%\]
%By transitivity of $=_\Addrs$, we conclude that $\mZ_1 =_\Addrs \mZ_2$.
%
%$(\Leftarrow)$ From $\mM\reddh \mach{Z}_1\equiva \mZ_2$ it follows $\mM\equiva\mZ_2$. From $\mN\reddh\mZ_2$ we get $\mN\equiva \mZ_2$.
%We conclude by symmetry and transitivity.
%\end{proof}

\begin{lem}[Postponement of inner steps]\label{lem:standardization}~\\ For $\mM,\mN,\mN'\in\cM$, if $\mM\red[i]\mN\red[h]\mN'$ then there exists $\mM'\in\cM$ such that $\mM\red[h]\mM'\redd[i]\mN'$. In diagrammatic form:
\[
\xymatrix{
\mM\ar@{->}[r]^{\mach{i}}\ar@{-->}[d]^{\mach{h}}&\mN\ar@{->}[d]^{\mach{h}}\\
\mM'\ar@{-->>}[r]^{\mach{i}}&\mN'
}
\]
\end{lem}

\begin{proof} By cases analysis over $\mM\red[i]\mN$.
The only interesting case is when the contracted redex is duplicated in $\mN\red[h]\mN'$, namely:

Case $\mM =\tuple{\vec R\repl{R_i}{a},P,T}$, $\mN = \tuple{\vec R\repl{R_i}{b},P,T}$ with  $\mM.P =\mN.P = \Apply ijk;P'$ and $\Lookinv a\red[c]\Lookinv b$.
Assume $i\neq k<\mM.r$ and $i = j$, the other cases being easier.
In this case $\mM' = \tuple{\vec R\repl{R_i}{a}\repl{R_k}{\App{a}{a}},P,T}$, therefore we need 3 inner steps to close the diagram:
\[
	\begin{array}{lcl}
	\mM'&\red[i]&\tuple{\vec R\repl{R_i}{b}\repl{R_k}{\App{a}{a}},P,T}\\
		&\red[i]&\tuple{\vec R\repl{R_i}{b}\repl{R_k}{\App{b}{a}},P,T}\\
			&\red[i]&\tuple{\vec R\repl{R_i}{b}\repl{R_k}{\App{b}{b}},P,T} = \mN'.
	\end{array}
\]
This concludes the proof.
\end{proof}

Morally, the term rewriting system $(\cM,\red[c])$ is orthogonal because $(i)$ the reduction rules defining $\red[c]$ are non-overlapping as $\red[h]$ is deterministic, $(\red[i]^R)$ reduces a register and $(\red[i]^T)$ reduces  one element of the tape; $(ii)$ the terms on the left-hand side of the arrow are linear, as no equality among subterms is required.
Now, it is well-known that orthogonal TRS are confluent, but one cannot apply~\cite[Thm.4.3.4]{terese} directly since we are not exactly dealing with first-order terms (because of the presence of the encoding).

\begin{prop}\label{prop:confluence}
The reduction $\red[c]$ is confluent.
\end{prop}

\begin{proof}[Proof sketch] The Parallel Moves Lemma, which is the key property for proving Theorem~4.3.4 in~\cite{terese} generalizes easily. The rest of the proof follows.
\end{proof}

\begin{lem}\label{lem:onestepisfine}
Let $\mM,\mN\in\cM$.
\begin{enumerate}[(i)]
\item\label{lem:onestepisfine1}
	$\mM\red\mN$ entails $\mM\equiva\mN$.
\item\label{lem:onestepisfine2}
	$\mM\redd\mN$ entails $\mM\equiva\mN$.
\end{enumerate}
\end{lem}

\begin{proof}
\ref{lem:onestepisfine1} By induction on a derivation of $\mM\red\mN$. % chktex 2

Base case $\mM\redh \mN$. Since $\equiva$ is an equivalence then so is $=_\Addrs$, by Lemma~\ref{lem:equivalence}.
In particular $=_\Addrs$ is reflexive, whence $\mN =_\Addrs \mN$. By Definition~\ref{def:equiv:Addrs}, we obtain $\mM\equiva \mN$.

Case ${(\red[i]^R)}$. Then $\mM = \tuple{\vec R[R_i:= \Lookup \mM'],P,T}$ and $\mN = \tuple{\vec R[R_i:= \Lookup \mN'],P,T}$ for some existing register $R_i$ and $\mM',\mN'\in\cM$ such that $\mM' \red \mN'$. By induction hypothesis we get $\mM' \equiva \mN'$, equivalently $\Lookup\mM' \sima \Lookup\mN'$.
From this and reflexivity, it follows $\vec R[R_i:= \Lookup \mM'] \sima \vec R[R_i:= \Lookup \mN']$, $P \sima P$ and $T \sima T$.
Thus $\mM =_\Addrs \mN$, so we conclude because $=_\Addrs\,\subseteq\  \equiva$.

Case ${(\red[i]^T)}$. In this case, we have
\[
	\begin{array}{lcl}
	\mM &=& \tuple{\vec R,P,[a_0,\dots,a_{i-1},\Lookup \mM',a_{i+1}\dots,a_n]}\\
	 \mN &=& \tuple{\vec R,P,[a_0,\dots,a_{i-1},\Lookup \mN',a_{i+1}\dots,a_n]}
	 \end{array}
\]
with $\mM' \red \mN'$. By induction hypothesis we get $\mM' \equiva \mN'$, equivalently $\Lookup\mM' \sima \Lookup\mN'$.
This entails $\mM.T \sima \mN.T$, from which it follows $\mM =_\Addrs \mN$. Conclude as above.

\ref{lem:onestepisfine2}  By induction on the length $n$ of the reduction $\mM\redd\mN$. % chktex 2

Case $n=0$. Then $\mM = \mN$, so we get $\mM\equiva\mN$ by reflexivity.

Case $n>0$. Then $\mM\red\mM'\redd\mN$. By~\ref{lem:onestepisfine1}, we get $\mM \equiva \mM'$.
Since the reduction $\mM'\redd\mN$ is strictly shorter, the induction hypothesis gives $\mM' \equiva \mN$.
Conclude by transitivity.
\end{proof}

\begin{thm}\label{thm:CR}
For $\mM,\mN\in\cM$, we have:
\[
	\mM\equiva \mN\iff \exists \mZ\in\cM\,.\, \mM\redd\mZ\invredd[\mach{c}] \mN
\]
\end{thm}

\begin{proof} $(\Rightarrow)$ By induction on a derivation of $\mM\equiva \mN$.

\redrule{} Assume that $\mM\reddh\mZ=_\Addrs \mN$. From $\mZ=_\Addrs \mN$ we get that $\mZ.r = \mN.r$, $\mZ.\vec R\sima \mN.\vec R$, $\mZ.P = \mN.P$ and $\mZ.T\sima\mN.T$. Note that $\mZ.R_i =\Null$ iff $\mN.R_i = \Null$.
Let us call $\cR$ the set of indices $i$ of, say, $\mZ$ such that $\mZ.R_i\neq\Null$.
By assumption, for every $i\in\cR$, we have $\mZ.R_i =a_i,\mN.R_i = a'_i$ for $a_i\sima a'_i$. Equivalently, $\Lookinv{a_i}\equiva\Lookinv{a'_i}$ holds and its derivation is smaller than $\mM\equiva \mN$. By induction hypothesis, they have a common reduct $\Lookinv{a_i}\redd \mach{X}_i\invredd[\mach{c}]\Lookinv{a'_i}$.
Similarly, calling $\mZ.T = [b_1,\dots,b_n]$ and $\mN.T = [b'_1,\dots,b'_m]$ we must have $m = n$ and $b_j\sima b'_j$ whence the induction hypothesis gives a common reduct $\Lookinv{b_j}\redd \mach{Y}_j \invredd[\mach{c}]\Lookinv{b'_j}$.
Putting all reductions together, we conclude:
\[
\mM\reddh \mZ\redd \tuple{\mZ.\vec R\repl{R_i}{\Lookup{\mach{X}_i}}_{i\in\cR},\mZ.P,[\Lookup{\mach{Y}_1},\dots,\Lookup{\mach{Y}_n}]} \invredd[\mach{c}]\mN
\]


(Transitivity) By induction hypothesis and confluence (Proposition~\ref{prop:confluence}).

(Symmetry) Straightforward from the induction hypothesis.

$(\Leftarrow)$ By Lemma~\ref{lem:onestepisfine}\ref{lem:onestepisfine2} we get $\mM\equiva\mZ$ and $\mN\equiva \mZ$, so we conclude by symmetry and transitivity.
\end{proof}

\begin{prop}\label{prop:cAisnonextcombal}
$\cA_{\,\sima}$ is a non-extensional combinatory algebra.
\end{prop}

\begin{proof} From the calculations in Example~\ref{ex:calculs}, it follows that $\App{\App{\Lookup{\mach{K}}}{a}}{b}\sima a$ and
$\App{\App{\App{\Lookup{\mach{S}}}{a}}{b}}{c}\sima \App{(\App{a}{c})}{(\App{b}{c})}$ hold, for all $a,b,c\in\Addrs$.
Notice that both \am s $\mK$ and $\mS$ are stuck, and $\mK\neq_\Addrs \mS$ since, e.g., $\mK.r\neq\mS.r$.
By Theorem~\ref{thm:CR}, we get $\Lookup\mK\not\sima\Lookup \mS$, whence $\cA_{\,\sima}$ is a combinatory algebra.

To check that $\cA_{\,\sima}$ is not extensional, it is sufficient to exhibit two elements of $\Addrs$ that are extensionally equal, but distinguished modulo $\sima$.
For instance, take $\App{\Lookup\mK}{a}$ and $\App{\Lookup\mK'}{a}$, where $a\in\Addrs$ is arbitrary and $\mK'$ is a different implementation of the combinator $\comb{K}$, namely:
\[
	\begin{array}{lcl}
	\mK' &=& \tuple{\Null,\Null,\RaS (0, 1); \Call 0,[]}, \\
	\mK &=& \tuple{\Null,\RaS (0,-); \Call 0,[]}.
	\end{array}
\]
For all $a,b\in\Addrs$, easy calculations give $\App{\App{\Lookup\mK'}{a}}{b} \sima a$.
Thus, for all $b\in\Addrs$, we have \[
	\App{\App{\Lookup\mK}{a}}{b} \sima a\sima \App{\App{\Lookup\mK'}{a}}{b},
	\]
	whence the two addresses $\App{\Lookup\mK}{a}$ and $ \App{\Lookup\mK'}{a}$ are extensionally equal elements of $\cA_{\,\sima}$.
However, the corresponding \am s are both stuck and $\appT{\mK}{[a]} \neq_\Addrs \appT{\mK'}{[a]}$, because $1 = (\append{\mK}{a}).r \neq (\append{\mK'}{a}).r = 2$.
Since they cannot have a common reduct, we derive $\appT{\mK}{[a]}\not\equiva\appT{\mK'}{[a]}$ by Theorem~\ref{thm:CR}.
We conclude that $\App{\Lookup\mK}{a} \not\sima \App{\Lookup\mK'}{a}$.
\end{proof}

\begin{lem}\label{lem:notlambdalg}
The combinatory algebra $\cA_{\,\sima}$ is not a \lam-model.
\end{lem}

\begin{proof} We need to find $M,N\in\Lambda$ satisfying $M=_\beta N$, while $\cA_{\,\sima}\not\models M = N$.
Take $M = \lam z.(\lam x.x)z =_{\CL} \comb{S(KI)I}$ and $N=\lam x.x =_{\CL} \comb{I}$ where $\comb{I} = \comb{SKK}$.

Recall that $\mach{I} = \mach{\append{S}{\Lookup K,\Lookup K}}$.
Easy calculations give:
\[
	\begin{array}{lll}
	\mach{\append{S}{\App{\Lookup \mK}{\Lookup \mach{I}},\Lookup \mach{I}}} &=&
	\tuple{\Null,\Null,\Null,\RaS 0;\cdots,[\App{\Lookup \mK}{\Lookup \mach{I}},\Lookup\mach{I}]}\\
	&\redh&\tuple{\App{\Lookup \mK}{\Lookup \mach{I}},\Null,\Null,\RaS 1;\cdots,[\Lookup\mach{I}]}\\
	&\redh&\stuck{\tuple{\App{\Lookup \mK}{\Lookup \mach{I}},\Lookup\mach{I},\Null,\RaS 2;\cdots,[]}}
	\end{array}
\]
Similarly,
\[
	\mach{I} = \append{\mach{S}}{\Lookup\mK,\Lookup\mK} \reddh \stuck{\tuple{\Lookup\mach{K},\Lookup\mach{K},\Null,\RaS 2;\cdots,[]}}.
\]
These two machines are both stuck and different modulo $=_\Addrs$ since, e.g., the contents of their register $R_1$ are $\Lookup\mach{I}$ and $\Lookup\mach{K}$ respectively, and it is easy to check that $\Lookup\mach{I}\not\sima \Lookup\mach{K}$.
By Theorem~\ref{thm:CR}, we conclude that $ \App{\App{\Lookup \mS}{(\App{\Lookup\mK}{\Lookup\mach{I}})}}{\Lookup\mach{I}}\not\sima\Lookup\mach{I}$.
\end{proof}

