% !TEX root = ../DPIM.tex
% !TEX spellcheck = en-US

In the previous section we have seen that the equivalence $\sima$, thus $\equiva$, is too weak to give rise to a model of \lam-calculus (Lemma~\ref{lem:notlambdalg}).
The main problem is that a \lam-term $\lam x.M$ is represented as an \am{} performing a ``$\ins{Load}$'' (to read $x$ from the tape) before evaluating the \am{} corresponding to $M$. Since nothing is applied, the tape is empty and the machine gets stuck thus preventing the evaluation of the subterm $M$.
In order to construct a \lam-model we introduce the equivalence $\simea$ below.

\begin{defi}
Define the relation $\equivea$ as the least equivalence satisfying:
\begin{gather*}
	\infer[\redwerule]{\mM \equivea \mN}{\mM\reddh \mZ \eqea \mN}
	\\[3pt]
	\infer[\extrule]{\mM \equivea \mN}{
		\mach{M}\reddh\stuck{\mM'}&
		\mN\reddh\stuck{\mN'}&
		\forall a\in\Addrs\,.\, \append{\mM}{a} \equivea \append{\mN}{a}
	}
\end{gather*}
We say that $\mM$ and $\mN$ are \emph{applicatively equivalent} whenever $\mM\equivea \mN$. Recall that $\simea$ and $\eqea$ are defined in terms of $\equivea$ as described in Definition~\ref{def:inducingequivalences}.
Also in this case, it is easy to check that $\eqea\ \subseteq\ \equivea$ holds.
\end{defi}

\begin{rem}\label{rem:aboutordinals} The rule $\extrule$ shares similarities with the $(\omega)$-rule in \lam-calculus~\cite[Def.~4.1.10]{Bare}, although being more restricted as only applicable to \am{} that eventually become stuck. In particular, both rules have countably many premises, therefore a derivation of $\mM\equivea\mN$ is a well-founded $\omega$-branching tree (in particular, the tree is countable and there are no infinite paths).
Techniques for performing induction ``on the length of a derivation'' in this kind of systems are well-established, see e.g.~\cite{BarendregtTh,IntrigilaS06}. More details about the underlying ordinals will be given in Section~\ref{sec:consistency}.
\end{rem}

\begin{exas}\label{ex:moreexamples} Convince yourself of the following facts.
\bsub
\item\label{ex:moreexamples1}
	As seen in the proof of Lemma~\ref{lem:notlambdalg}, $\mach{I}$ and $\append{\mach{S}}{\App{\Lookup\mK}{\Lookup \mach{I}},\Lookup\mach{I}}$ both reduce to stuck machines.
	For all $a\in\Addrs$, we have that \[\append{\mach{I}}{a}\reddh\Lookinv a\invredd[\mach{h}] \append{\mach{S}}{\App{\Lookup\mK}{\Lookup\mach{I}},\Lookup\mach{I},a}.\]
	By \extrule, they are applicatively equivalent.
\item\label{ex:moreexamples2}
	Since indeterminate machines $\mach{x}_k$ are not stuck, $\mach{x}_m\equivea\mach{x}_n$ entails $m=n$.
\item\label{ex:moreexamples3}
	Let \[\mach{1} = \tuple{\Null^2,\Load (0,1);\Apply010;\Call 0,[]}.\] It is easy to check that, for all $a,b\in\Addrs$, we have $\append{\mach{1}}{a,b}\reddh \append{\Lookinv{a}}{b}\invredd[\mach{h}] \append{\mach{I}}{a,b}$. However, since   $\append{\mach{I}}{\Lookup\mach{x}_n}\reddh \mach{x}_n$ and $\lnot\stuck{\mach{x}_n}$, one cannot apply \extrule{}, whence (intuitively) they are not applicatively equivalent: $\mach{I} \not\equivea \mach{1}$.
\esub
\end{exas}

\noindent
Actually the inequalities claimed in examples~\ref{ex:moreexamples2}-\ref{ex:moreexamples3} above, i.e.\ $\mach{x}_m\not\equivea\mach{x}_n$ for $m\neq n$ and $\mach{I}\not\equivea\mach{1}$, are difficult to prove formally (see Lemma~\ref{lem:about:equivo}\ref{lem:about:equivo2}).

\begin{lem} Let $\mM,\mN\in\cM$ and $a,b\in\Addrs$.
\begin{enumerate}[(i)]
%\item $\forall a,b\in\Addrs\,.\, a\simea b\Rightarrow \append{\mM}{[a]}\equivea \append{\mM}{[b]}$
\item If $\mM\equivea \mN$ then $\append{\mM}{a}\equivea \append{\mN}{a}$.
\item The following rule is derivable:
\[
	\infer[(\mathrm{cong})]{\append{\mM}{a}\equivea \append{\mN}{b}}{\mM \equivea \mN& a \simea b}
\]
\item Therefore, $\simea$ is a congruence on $\cA = (\Addrs,\cdot,\Lookup\mK,\Lookup\mS)$.
\end{enumerate}
\end{lem}

\begin{proof}
%\item By \redwerule since $\append{\mM}{[a]}\eqea \append{\mM}{[b]}$.
$(i)$ By induction on a proof of $\mM\equivea\mN$. Possible cases are:

Case \redwerule. If $\mM\reddh \mZ \eqea \mN$ then $\append{\mM}{a}\reddh\append{\mZ}{a} \eqea \append{\mN}{a}$, by Lemma~\ref{lem:about_red}\ref{lem:about_red2} and the definition of $\eqea$.

Case $\extrule$. Trivial, as the thesis is a premise of this rule.

(Symmetry) and (Transitivity) follow from the induction hypothesis.

$(ii)$ Assume that $\mM\equivea\mN$ and $a\simea b$. Then, we have:
\[
	\begin{array}{lcll}
	\append{\mM}{a}&\eqea&\append{\mM}{b},&\textrm{by reflexivity and }a\simea b,\\
	&\equivea&\append{\mN}{b},&\textrm{by }(i).
	\end{array}
\]
So we conclude by transitivity.

$(iii)$ By Lemma~\ref{lem:equivalence} $\sima$ is an equivalence, by $(ii)$ a congruence.
\end{proof}

We need to show that the congruence $\simea$ is non-trivial, and that the addresses of $\Lookup\mK,\Lookup\mS$ remain distinguished modulo $\simea$.

\begin{lem}\label{lem:about:equivo}
Let $\mM,\mN\in\cM$.
\begin{enumerate}[(i)]
\item\label{lem:about:equivo1}
	If $\mM\equiva \mN$ then $\mM\equivea\mN$.
\item\label{lem:about:equivo2}
	If $\mM\equivea \mN$ and $\mM\reddh\mach{x}_n$ then $\mN\reddh \mach{x}_n$.
\item\label{lem:about:equivo3}
	Hence, the equivalence relation $\simea$ is non-trivial.
\item\label{lem:about:equivo4}
		In particular, $\Lookup \mach{K} \not\simea \Lookup \mach{S}$.
\end{enumerate}
\end{lem}

\begin{proof}
\ref{lem:about:equivo1} % chktex 2
	Easy.

\ref{lem:about:equivo2} % chktex 2
	This proof is the topic of Section~\ref{sec:consistency}.

\ref{lem:about:equivo3} % chktex 2
By~\ref{lem:about:equivo1}, the relation is non-empty. By~\ref{lem:about:equivo2}, $\mach{x}_i \equivea \mach{x}_j $ if and only if $ i = j$, whence there are infinitely many distinguished equivalence classes.

\ref{lem:about:equivo4} % chktex 2
From Example~\ref{ex:somemachines}, we get:
\[
	\begin{array}{lcl}
	\append{\mach{K}}{\Lookup\mach{K},\Lookup\mach{K},\Lookup\mach{x}_1}&\reddh&\tuple{\Lookup{\mach{x}_1,\RaS -;\Call 0,[]}};\\
	\append{\mach{S}}{\Lookup\mach{K},\Lookup\mach{K},\Lookup\mach{x}_1}&\reddh&\mach{x}_1.\\
	\end{array}
\]
For these machines to be $\equivea$-equivalent, the former machine should reduce to $\mach{x}_1$, by~\ref{lem:about:equivo2}, which is impossible since $\tuple{\Lookup{\mach{x}_1,\RaS -;\Call 0,[]}}$ is stuck.
\end{proof}

\subsection{Constructing a \lam-model}

We define an interpretation transforming a \lam-term with free variables $x_1,\dots,x_n$ into an \am{} reading the values of $\vec x$ from its tape.
The definition is inspired from the well-known categorical interpretation of \lam-calculus into a reflexive object of a cartesian closed category.
In particular, variables are interpreted as projections.
See, e.g.,~\cite{Koymans82} or~\cite{Selinger02} for more details.


\begin{defi}[Auxiliary interpretation]\label{def:categoricalint}
Let $M\in\Lama$ and $x_1,\dots,x_n$ be such that $\FV{M}\subseteq\vec x$.
Define $\CInt[\vec x]{-} : \Lama\to\cM$ by induction as follows:
\[
	\begin{array}{lcll}
	\CInt[\vec x]{x_i} &=& \mach{Pr}_i^n,\textrm{ where }1\le i \le n;\\[1ex]
	\CInt[\vec x]{\cons{a}} &=& \mach{Cons}_a^n$, for $a\in\Addrs;\\[1ex]
	\CInt[\vec x]{MN} &=& \tuple{\Null^{n},\Lookup\CInt[\vec x]{M},\Lookup\CInt[\vec x]{N},\Null,\ins{Apply}_n,[]};\\[1ex]
	\CInt[\vec x]{\lam y.M} &=& \CInt[\vec x,y]{M},&\textrm{ assuming wlog that }y\notin\vec x;\\
	\end{array}
\]
where
\[
	\begin{array}{lcl}
	\mach{Pr}_i^n &=& \tuple{\Null,(\RaS -)^{i-1};\RaS 0;(\RaS -)^{n-i-1};\Call 0,[]},\\[1ex]
	\mach{Cons}_a^n &=& \tuple{a,(\RaS -)^{n};\Call 0,[]},\\	[1ex]
	\ins{Apply}_n &=& \RaS (0,\dots,n-1);
					\Apply n 0 n;\cdots;\Apply n {n-1} n;\\
					&&\Apply {n+1} 0 {n+1};\cdots;\Apply {n+1} {n-1} {n+1};\\
                     &&\Apply n {n+1} {n+2};\Call {n+2}.\\
	\end{array}
\]
\end{defi}

\begin{rem}\label{rem:aux_int}
Let $n\in\nat$, and $T = [a_1,\dots,a_n]\in\Tapes$. We have:
\begin{enumerate}[(i)]
\item\label{rem:aux_int1}
	$\appT{\mach{Pr}_i^n}{T}\reddh \Lookinv{a_i}$, for all $i\,(1\le i\le n)$;
\item\label{rem:aux_int2}
	$\appT{\mach{Cons}_b^n}{T}\reddh \Lookinv{b}$, for all $b\in\Addrs$;
\item\label{rem:aux_int3}
	$\tuple{\Null^{n},\Lookup \mM,\Lookup \mN,\Null,\ins{Apply}_n,T} \reddh\append{(\appT{\mM}{T})}{\Lookup(\appT{\mN}{T})}$.
\end{enumerate}
\end{rem}

\noindent
From now on, whenever writing $\CInt{M}$, we assume that $\FV{M}\subseteq\vec x$.
The following are basic properties of the interpretation map defined above.

\begin{lem}\label{lemma:aboutcatint}
Let $M\in\Lam(\Addrs)$, $n\in\nat$, $\vec x = x_1,\dots,x_n$ and $\vec a = a_1,\dots,a_n\in\Addrs$.
\bsub
\item\label{lemma:aboutcatint1}
	$\CInt[\vec x]{M} = \tuple{\vec R,\RaS (i_1,\dots,i_n);P,[]}$ for some $\Addrs_\Null$-valued registers $\vec R$, program $P$ and indices $i_j\in\nat$.
\item\label{lemma:aboutcatint2}
	If $m<n$ then $\append{\CInt[\vec x]{M}}{a_1,\dots,a_m}\reddh \stuck{}$.
\item\label{lemma:aboutcatint3}
	For all $b\in\Addrs$, we have $\append{\CInt[y,\vec x]{M}}{b} \equivea \CInt[\vec x]{M\subst{y}{\cons b}}$.
\item\label{lemma:aboutcatint4}
	In particular, if $y\notin\FV{M}$ then $\append{\CInt[y,\vec x]{M}}{b} \equivea \CInt{M}$.
\item\label{lemma:aboutcatint5}
	$\append{\CInt{M}}{\vec a\,} \equivea \append{\CInt[x_{\sigma(1)},\dots,x_{\sigma(n)}]{M}}{a_{\sigma(1)},\dots,a_{\sigma(n)}}$ for all permutations~$\sigma$ of $\set{1,\dots,n}$.
\esub
\end{lem}

\begin{proof}[Proof of Lemma~\ref{lemma:aboutcatint}]
\ref{lemma:aboutcatint1} % chktex 2
	By a straightforward induction on $M$.

\ref{lemma:aboutcatint2} % chktex 2
	It follows from~\ref{lemma:aboutcatint1}.

\ref{lemma:aboutcatint3} % chktex 2
	We proceed by structural induction on $M$.
	 By~\ref{lemma:aboutcatint2}, if $\vec x\neq\emptyset$ then both \am s reduce to stuck ones, so we can test the applicative equivalence by applying an arbitrary $\vec a$ and conclude using \extrule{} $n$-times.

Case $M = \cons c$. Then $c\subst{y}{\cons b} = c$, and we have:
	\[
	\append{\CInt[y,\vec x]{\cons c}}{b,\vec a} =
	\append{\mach{Cons}_c^{n+1}}{b,\vec a}
	\reddh \Lookinv c\invredd[\mach{h}]\append{\mach{Cons}_c^{n}}{\vec a}.
	\]

Case $M = x_i$ for some $i\,(1\le i\le n)$. Then $x_i\subst{y}{\cons b} = x_i$ and
	\[
	\append{\CInt[y,\vec x]{x_i}}{b,\vec a} = \append{\mach{Pr}_{i+1}^{n+1}}{b,\vec a}
	\reddh \Lookinv{a_i}\invredd[\mach{h}] \append{\mach{Pr}_i^n}{\vec a}
	= \append{\CInt{x_i}}{\vec a}.
	\]

Case $M = y$. Then $y\subst{y}{\cons b} = \cons b$ and we have:
	\[
	\append{\CInt[y,\vec x]{y}}{b,\vec a\,} =
	\append{\mach{Pr}_{1}^{n+1}}{b,\vec a\,}\reddh
	\Lookinv{b}
	\invredd[\mach{h}]
	\append{\mach{Cons}_b^{n}}{\vec a\,} =
	\append{\CInt{\cons b}}{\vec a\,}.
	\]

Case $M = PQ$. Then $(PQ)\subst{y}{\cons b} = (P\subst{y}{\cons b})(Q\subst{y}{\cons b})$ and we have:
\[
	\begin{array}{llll}
	\append{\CInt[y,\vec x]{PQ}}{b,\vec a\,}&=&
	\tuple{\Null^{n+1},\Lookup\CInt[y,\vec x]{P},\Lookup\CInt[y,\vec x]{Q},\Null,\ins{Apply}_{n+1},[b,\vec a\,]}&\\
	&\reddh&\append{\CInt[y,\vec x]{P}}{b,\vec a,\Lookup (\append{\CInt[y,\vec x]{Q}}{b,\vec a\,})}\\
	&\equivea&\append{\CInt{P\subst{y}{\cons b}}}{\vec a,\Lookup (\append{\CInt{Q\subst{y}{\cons b}}}{\vec a\,})},\textrm{ by IH,}\\
	&\invredd[\mach c]&\tuple{\Null^{n},\Lookup\CInt{P\subst{y}{\cons b}},\Lookup\CInt{Q\subst{y}{\cons b}},\Null,\ins{Apply}_n,[\vec a]}&\\
	&=&\CInt{(P\subst{y}{\cons b})(Q\subst{y}{\cons b})}\\
	&=& \CInt{(PQ)\subst{y}{\cons b}}\\
	\end{array}
\]

Case $M = \lam z.P$, wlog $z\notin y,\vec x$, so $(\lam z.P)\subst{y}{\cons b} = \lam z.P\subst{y}{\cons b}$.
	By~\ref{lemma:aboutcatint2} both machines reduce to stuck ones.
	So we have to apply an extra $a_{n+1}\in\Addrs$.
\[
	\begin{array}{lcll}
	\append{\CInt[y,\vec x]{\lam z.P}}{b,\vec a,a_{n+1}}&=&\append{\CInt[y,\vec x,z]{P}}{b,\vec a,a_{n+1}}&\\
	&\equivea&\append{\CInt[\vec x,z]{P\subst{x}{\cons b}}}{\vec a,a_{n+1}},&\textrm{by IH,}\\
	&=& \append{\CInt{\lam z.P\subst{y}{\cons b}}}{\vec a,a_{n+1}}
	\end{array}
\]

\ref{lemma:aboutcatint4} By~\ref{lemma:aboutcatint3}. % chktex 2

\ref{lemma:aboutcatint5} By~\ref{lemma:aboutcatint4}, permuting the substitutions. % chktex 2
\end{proof}


\begin{defi}\label{def:thelambdamodel}
Let $\cS = (\Addrs/_{\simea},\bullet, \Int{-}{-})$, where
\[
	\begin{array}{lcl}
	[a]_{\simea}\bullet [b]_{\simea}&=&[\App{a}{b}]_{\simea},\\[3pt]
	\Int{M}{\rho} &\simea& \Lookup(\append{\CInt{M}}{\rho(x_1),\dots,\rho(x_n)}).\\
	\end{array}
\]
By Lemma~\ref{lemma:aboutcatint}, the definition of $\Int{M}{\rho}$ is independent from the choice of $\vec x$, as long as $\FV{M}\subseteq\vec x$. This is reminiscent of the standard way for defining a syntactic interpretation from a categorical one. (Again, see Koymans's~\cite{Koymans82}.)
\end{defi}

\begin{thm}\label{thm:Sinsonoextslm}
$\cS$ is a syntactic \lam-model.
\end{thm}

\begin{proof} We need to check that the conditions~\ref{def:syntmod1}--\ref{def:syntmod6} from Definition~\ref{def:syntmod} are satisfied by the interpretation function given in Definition~\ref{def:thelambdamodel}.

Take $\vec x = x_1,\dots,x_n$, and write $\rho(\vec x)$ for $\rho(x_1),\dots,\rho(x_n)$.

\ref{def:syntmod1} $\Int{x_i}{\rho} \simea \Lookup(\append{\mach{Pr}_i^n}{\rho(\vec x)})\simea \rho(x_i)$, by Remark~\ref{rem:aux_int}\ref{rem:aux_int1}. % chktex 2

\ref{def:syntmod2} $\Int{\cons a}{\rho} \simea \Lookup(\append{\mach{Cons}_a^n}{\rho(\vec x)})\simea a$, by Remark~\ref{rem:aux_int}\ref{rem:aux_int2}. % chktex 2

\ref{def:syntmod3} In the application case, we have: % chktex 2
\[
	\begin{array}{llll}
	\Int{PQ}{\rho} &\simea&\Lookup(\append{\CInt{PQ}}{\rho(\vec x)})\\
	&=& \tuple{\Null^{n},\Lookup \CInt[\vec x]{P},\Lookup \CInt[\vec x]{Q},\Null,\ins{Apply}_n,[\rho(\vec x)]},&\textrm{by Def.~\ref{def:categoricalint},}\\
	&\simea&\App{\Lookup(\append{\mM}{\rho(\vec x)})}{\Lookup(\append{\mN}{\rho(\vec x)})},&\textrm{by Rem.~\ref{rem:aux_int}\ref{rem:aux_int3},}\\
	&=&\Int{P}{\rho}\bullet \Int{Q}{\rho}
	\end{array}
\]

\ref{def:syntmod4} In the \lam-abstraction case we have, for all $a\in\Addrs$: % chktex 2
\[
	\App{\Int{\lam y.P}{\rho}}{a} \simea \append{\CInt{\lam y.P}}{\rho(\vec x),a}
  	 \simea \append{\CInt[\vec x,y]{P}}{\rho(\vec x),a}\\
	 \simea\Int{P}{\rho\repl{y}{a}}.
\]

\ref{def:syntmod5} This follows from Lemma~\ref{lemma:aboutcatint}\ref{lemma:aboutcatint4}. % chktex 2

\ref{def:syntmod6} By definition $\Int{\lam y.M}{\rho} \simea \Lookup(\append{\CInt[\vec x,y]{M}}{\rho(\vec x)})$ and, by Lemma~\ref{lemma:aboutcatint}\ref{lemma:aboutcatint2}, $\append{\CInt[\vec x,y]{M}}{\rho(\vec x)}$ reduces to a stuck \am. % chktex 2
Similarly, for $\Int{\lam x.N}{\rho}$. We conclude by applying the rule \extrule.
\end{proof}

\begin{rem}
\bsub
\item For closed \lam-terms $M\in\Lamo$, we have $\Int{M}{} = \CInt[]{M}$.
\item It is easy to check that $\Int{\comb{K}}{}\simea\Lookup\mach{K}$ and
$\Int{\comb{S}}{}\simea\Lookup\mach{S}$.
\item More generally, all \am s behaving as the combinator $\comb{K}$ (resp.\ $\comb{S}$) are equated in the model.
\esub
\end{rem}

\begin{lem}\label{lem:Snotext}
The syntactic \lam-model $\cS$ is not extensional.
\end{lem}

\begin{proof} It is enough to check that $\cS\not\models \comb{1} = \comb{I}$. Now, we have:
\[
	\begin{array}{lll}
	\Int{\,\comb{1}\,}{} &=& \tuple{\Null^2,\Lookup\mach{Pr}^2_1,\Lookup\mach{Pr}^2_2,\Null,\ins{Apply}_2,[]};\\
	\Int{\,\comb{I}\ }{} &=& \tuple{\Null,\Load 0;\Call 0,[]}.\\
	\end{array}
\]
By applying an indeterminate machine $\mach{x}_n$, the former reduces to a stuck machine, while the latter reduces to $\mach{x}_n$. By Lemma~\ref{lem:about:equivo}\ref{lem:about:equivo2}, they must be different modulo $\equivea$.
\end{proof}

A difficult problem that arises naturally is the characterization of the \lam-theory induced by the \lam-model $\cS$ defined above.

\begin{prop}\label{prop:aboutThS}
The \lam-theory $\Th{\cS}$ is neither extensional nor sensible.
\end{prop}

\begin{proof} $\Th{\cS}$ is not extensional by Lemma~\ref{lem:Snotext}. To show that it is not sensible, it is enough to check that $\cS\not\models \lam x.\Om = \Om$. Notice that
\[
	\begin{array}{llll}
	\CInt[]{\Om}&=&\tuple{\Lookup\CInt[]{\comb{\Delta}},\Lookup\CInt[]{\comb{\Delta}},\Null,\Apply 012;\Call 2,[]},\\
	&\redh&\append{\CInt[]{\comb{\Delta}}}{\Lookup\CInt[]{\comb{\Delta}}},&\textrm{where:}\\
	\CInt[]{\comb{\Delta}}&=&\tuple{\Null,\Lookup\mach{Pr}^1_1,\Lookup\mach{Pr}^1_1,\Null,\ins{Apply}_1,[]}.\\
	\end{array}
\]
By induction on a derivation of $\mM\equivea \mN$, one checks that $\mM\equivea \mN$ and $\mM\reddh \append{\mach{D}_1}{\mach{D}_2}$ with $\mach{D}_1\simea \mach{D}_2\simea\CInt[]{\comb{\Delta}}$entails $\mN\reddh\append{\mach{D}'_1}{\Lookup\mach{D}'_2}$ for some $\mach{D}'_1\simea\mach{D}'_2\simea\CInt[]{\comb{\Delta}}$. We conclude because the machine $\CInt[]{\lam x.\Om}$ is stuck.
\end{proof}

%We believe that $\Th{\cS}$ is semi-sensible, we conjecture that $\Th{\cS} = \blam$.
