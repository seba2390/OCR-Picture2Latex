% !TEX root = ../DPIM.tex
% !TEX spellcheck = en-US
\renewcommand\hole[1]{\llparenthesis#1\rrparenthesis}
\newcommand{\Count}{\omega_1}
\newcommand{\occ}[1]{\mathrm{occ}_\xi(#1)}
\newcommand{\pesc}[1][\alpha]{\approx_{#1}}
\newcommand{\svirg}[1][\alpha]{\sim_{#1}}
\newcommand{\eq}[1][\alpha]{\equiv_{#1}}
\newcommand{\ured}[1][M]{\redh^{\mach{#1}}}
\newcommand{\uredd}[1][M]{\reddh^{\mach{#1}}}
\newcommand{\XX}{\mathbb{X}}
\newcommand{\X}{\mach{X}}
\newcommand{\BB}{\mathbb{B}}
\newcommand{\cMX}{\cM[\XX]^\xi}
\newcommand{\Lup}{\underline{\#}}
\newcommand{\Luinv}[1]{\underline{\#}^{-1}(#1)}

In this section we adapt Barendregt's proof of consistency of $\blam\omega$ (the least \lam-theory closed under the $(\omega)$-rule) to prove Lemma~\ref{lem:about:equivo}\ref{lem:about:equivo2}, which entails the consistency of our system.
First, we need to introduce in our setting the notion of \emph{context} and \emph{underlined reduction}, that are omnipresent techniques in the area of term rewriting systems.

\subsection{Contexts and Underlined Head Reductions}

In \lam-calculus a context is a \lam-term possibly containing occurrences of an algebraic variable, called \emph{hole}, that can be substituted by any \lam-term possibly with capture of free variables.
We will define a \emph{context-machine} similarly, namely as an \am{} possibly having a ``hole'' denoted by $\xi$. Formally, we introduce a new machine having no registers or program, only an empty tape (therefore distinguished from all machines populating $\cM$):
\[
	\xi = \tuple{[]}
\]
We then extend our formalism to include machines working either directly or indirectly with one, or more, occurrences of $\xi$. We wish to ensure the invariant that a machine $\mM$ with no occurrences of $\xi$ maintain as address $\Lookup\mM$ --- for this reason we need to extend the range of addresses in a conservative way.

Consider a countable set $\BB$ of addresses such that $\Addrs\cap\BB = \emptyset$, and write $\XX = \Addrs\cup\BB$ for the set of \emph{extended addresses}. As usual, we set \[\XX_\Null = \XX\cup\set{\Null}.\]

\begin{defi}\label{def:context-machine}
\begin{enumerate}[(i)]
\item
	An \emph{extended machine $\X$} is either of the form
	\begin{itemize}
	\item $\appT{\xi}{T}$ or
	\item $\tuple{\vec R,P,T}$
	\end{itemize}
	where $\vec R$ are $\XX_\Null$-valued registers, $P$ is a valid program, $T\in\Tapes[\XX]$ is an $\XX$-valued tape. We write $\cMX$ for the set of all extended machines.
\item Fix a bijective map $\Lup : \cMX \to \XX$ satisfying $\Lup(\mM)=\Lookup\mM$ for all \am{} $\mM\in\cM$. Write $\Luinv{\cdot} : \XX\to\cMX$ for its inverse.
\item The \emph{number of occurrences} of $\xi$ in $\X\in\cMX$ (resp.\ $R_i$, resp.\ $T$), written $\occ{\X}\in\nat\cup\set{\infty}$ ($\occ{R_i}$, $\occ{T}\in\nat\cup\set{\infty}$), is defined as follows:
\[
	\begin{array}{lcl}
	\occ{\appT{\xi}{T}} &=& 1 + \occ{T};\\[3pt]
	\occ{\tuple{\vec R,P,T}} &=& \occ{T} + \sum_{i=0}^{r-1}\occ{R_i};\\[3pt]
	\occ{[a_1,\dots,a_n]} &=& \occ{\Luinv{a_1}}+\cdots+\occ{\Luinv{a_n}};\\[3pt]
	\occ{R_i}&=&\begin{cases}
	0,&\textrm{if }R_i = \Null,\\
	\occ{\Luinv{a}},&\textrm{if }R_i = a\in\XX.\\
	\end{cases}
	\end{array}
\]
Notice that $\occ{\mM}\in\nat$ entails that $\occ{\mM.R_i},\occ{\mM.T}\in\nat$.
\end{enumerate}
\end{defi}

\noindent
The number of occurrences of $\xi$ in an extended machine $\X$ has been defined to handle the fact that recursively dereferencing all the addresses contained in an extended \am{} might result in a non-terminating process (see Remark~\ref{rem:forever}).

\begin{exas}\label{ex:weird}
The following are examples of extended machines:
\begin{enumerate}[(i)]
\item $\xi$, with $\occ{\xi} = 1$;
\item $\append{\mK}{\Lup{\xi},\Lup{(\append{\xi}{\Lup\xi})}}$, with $\occ{\append{\mK}{\Lup{\xi},\Lup{(\append{\xi}{\Lup\xi})}}} = 3$;
\item\label{ex:weird3} for all $n\in\nat$, $\X_n = \tuple{\Lup\xi,\varepsilon,[\Lup{\X_{n+1}}]}$. In this case, $\occ{\X_0} = \infty$.
\end{enumerate}
\end{exas}

\noindent
As previously mentioned, a key property of contexts in \lam-calculus is that one can plug a \lam-term into the hole and obtain a regular \lam-term.
Similarly, given $\mM\in\cMX$ and $\X\in\cMX$, we can define the \am{} $\X\hole{\mM}$ obtained from $\X$ by recursively substituting (even in the registers/tapes) each occurrence of $\xi$ by $\mM$. However, this operation is well-defined only when $\occ{\X}$ is finite, so we focus on extended machines enjoying this property.

\begin{defi}\
\begin{enumerate}[(i)]
\item A \emph{context-machine} is any $\C\in\cMX$ satisfying $\occ{\C}\in\nat$.
\item Given a context-machine $\C$ and $\mM\in\cM$, define the \am{} $\C\hole{\mM}$ as follows:
\[
	\C\hole{\mM} =\begin{cases}
	\appT{\mM}{T\hole{\mM}},&\textrm{if }\C=\appT{\xi}{T},\\
	\tuple{\vec R\hole{\mM},P,T\hole{\mM}},&\textrm{if }\C=\tuple{\vec R,P,T};\\
	\end{cases}
\]
where (assuming $a\in\XX,T = [a_1,\dots,a_n]\in\Tapes[\XX]$ with $\occ{\Cons a T}\in\nat$):
\[
	\begin{array}{lcl}
	a\hole{\mM} &=& \Lup(\Luinv{a}\hole{\mM});\\[3pt]
	R_i\hole{\mM}&=& \begin{cases}
	\Null&\textrm{if }R_i = \Null,\\
	a\hole{\mM}&\textrm{if }R_i = a;\\[3pt]
	\end{cases}~\\
	T\hole{\mM} &=& [a_1\hole{\mM},\dots,a_n\hole{\mM}].\\[3pt]
	\end{array}
\]
\end{enumerate}
\end{defi}

In the following, when writing $\C\hole{\mM}$ (resp.\ $a\hole{\mM}$, $R_i\hole{\mM}$, $T\hole{\mM}$) we silently assume that the number of occurrences of $\xi$ in $\C$ (resp.\ $a,R_i,T$) is finite.
Let us introduce a notion of reduction for context-machines that allows to mimic the underlined reduction from~\cite{BarendregtTh}. The idea is to decompose a machine $\mN$ as $\mN = \C\hole{\underline{\mM}}$ where $\C$ is a context-machine and $\mM$ the underlined sub-machine.
It is now possible to reduce $\C$ independently from $\mM$ until either the machine reaches a final-state or $\xi$ reaches the head-position. In the latter case, we substitute the head occurrence of $\xi$ by $\mM$, and continue the computation.

\begin{defi}\label{def:weirdreds}\
\bsub
\item\label{def:weirdreds1}
	The head reduction $\redh$ is generalized to extended machines in the obvious way, using $\Lup(\cdot)$ rather than $\Lookup{(\cdot)}$ to compute the addresses.
In particular, the machine $\appT{\xi}{T}\not\redh$ is in final state, but it is not stuck.
\item\label{def:weirdreds2}
	Given $\mM\in\cM$ and $\C\in\cM^\xi$, the \emph{$\mM$-underlined (head-)reduction} $\ured$ is defined by adding to~\ref{def:weirdreds1} the rule
\[
	\appT{\xi}{T}\ured\appT{\mM}{T}.
\]
\esub
\end{defi}

\begin{exas} Let $\C = \append{\mS}{\Lup \xi,\Lup\xi,\Lup \mach{x}_n}$. Then $\C\hole{\mach{K}} = \append{\mS}{\Lookup \mach{K},\Lookup\mach{K},\Lookup \mach{x}_n}$.
\bsub
\item $\C\reddh \append{\xi}{\Lup\mach{x}_n,\Lup{(\append{\xi}{\Lup\mach{x}_n})}}$.
\item$\C\uredd[K] \append{\xi}{\Lup\mach{x}_n,\Lup{(\append{\xi}{\Lup\mach{x}_n})}}
\ured[K]\append{\mach{K}}{\Lup\mach{x}_n,\Lup{(\append{\xi}{\Lup\mach{x}_n})}}\uredd[K] \mach{x}_n$.
\esub
\end{exas}

\begin{lem}\label{lem:chemmeserve}
	For $\C,\C'\in\cMX$ and $\mM,\mN\in\cM$, the following are equivalent:
	\begin{enumerate}
	\item\label{lem:chemmeserve1} $\C\hole{\mM}\reddh \mN$;
	\item\label{lem:chemmeserve2} $\C\reddh^\mM\C'$ and $\C'\hole{\mM} = \mN$.
	\end{enumerate}
\end{lem}

\begin{proof} (\ref{lem:chemmeserve1} $\Rightarrow$~\ref{lem:chemmeserve2})
By induction on the length $n$ of the reduction $\C\hole{\mM}\reddh\mN$.

Case $n = 0$. Trivial, take $\C'=\C$.

Case $n > 0$. Let $\C\hole{\mM}\redh\mN'\reddh\mN$. Split into cases depending on $\C$.

Subcase $\C = \appT{\xi}{T}$, therefore $\C\hole{\mM} = \appT{\mM}{T\hole{\mM}}\redh\mN'$. There are two possibilities:
\begin{itemize}
\item $\mM$ is stuck and $T\neq[]$, say, $T=[a_0,\dots,a_n]$. In this case $\C\hole{\mM} = \tuple{\vec R,\Load i;P,[]}$ and $\mN' = \tuple{\vec R\repl{R_i}{a_0\hole{\mM}},\Load i;P,[a_1\hole{\mM},\dots,a_n\hole{\mM}]}$.
On the other side, $\C\ured \appT{\mM}{T}\ured \C''$ for
\[
	\C'' = \tuple{\vec R\repl{R_i}{a_0},\Load i;P,[a_1,\dots,a_n]}
\]
satisfying $\C''\hole{\mM} = \mN'\reddh\mN$. We conclude by induction hypothesis.
\item $\mM\redh\mM'$. In this case $\mN' = \appT{\mM'}{T\hole{\mM}}$ and $\C\ured \appT{\mM}{T}\ured \C''$ for $\C'' = \appT{\mM'}{T}$ satisfying $\C''\hole{\mM} = \mN'\reddh\mN$. We conclude by induction hypothesis.

Subcase $\C = \tuple{\vec R,P,T}$. By case analysis on $P$. All cases follow easily from the induction hypothesis.
\end{itemize}

(\ref{lem:chemmeserve2} $\Rightarrow$~\ref{lem:chemmeserve1})
By induction on the length $n$ of the reduction $\C\reddh^\mM\C'$.

Case $n=0$. Trivial, take $\mN=\C\hole{\mM}$.

Case $n>0$, i.e.\ $\C\ured\C''\uredd\C'$, where the latter reduction is shorter.

Proceed by case analysis on the shape of $\C$.

Subcase $\C = \appT{\xi}{T}$ and $\C''=\appT{\mM}{T}$.
Then $\mN = \C''\hole{\mM} = \appT{\mM}{T\hole{\mM}} = \C\hole{\mM}$.
Conclude by induction hypothesis.

Subcase $\C = \tuple{\vec R,P,T}$. By case analysis on $P$. All cases follow easily from the induction hypothesis.
\end{proof}

\subsection{Ordinal analysis}

As mentioned in Remark~\ref{rem:aboutordinals}, a derivation of $\mM\equivea\mN$ has the structure of a well-founded $\omega$-branching tree.
Unfortunately, this makes it difficult to prove even simple properties like Lemma~\ref{lem:about:equivo}\ref{lem:about:equivo2}.
We need a more refined system exposing the underlying ordinal and handling the applications of the (Transitivity) rule separately.

\begin{defi}
\begin{enumerate}[(i)]
\item Let $\Count$ be the set of all countable ordinals.
\item If $\pi$ is a derivation of $\mM\equivea\mN$, we define its \emph{length} $\ell(\pi)\in\omega_1$ in the usual inductive way for the rules \redwerule, (Refl.), (Symm.), (Trans.). Concerning the rule $\extrule$ having countably many premises, we set:
\[
	\ell\left(
	\begin{array}{c}
	\infer{\mM \equivea \mN}{
		\mach{M},\mach{N}\reddh\stuck{}&
		\forall a\in\Addrs\,.\, \infer{\append{\mM}{a} \equivea \append{\mN}{a}}{\pi_a}
	}
	\end{array}
	\right) = \sup_{a\in\Addrs}(\ell(\pi_a)+1)
\]
It is easy to check that, if a derivation $\pi$ has premises $(\pi_i)_{i\in \cI}$ for some countable set $\cI$ then $\ell(\pi) > \ell(\pi_i)$ for every $i\in\cI$.
\item For all $\alpha\in\Count$, define $\eq,\svirg,\pesc\,\subseteq\cM^2$ as the least reflexive and symmetric relations closed under the rules of Figure~\ref{fig:Pesiolino}.
\end{enumerate}
\begin{figure}
\begin{gather*}
% General
\infer[(\approx_0)]{\mM\pesc[0] \mN}{\mM\equiva \mN}
\qquad
\infer[(\subseteq^{\approx}_\alpha)]{\mM\svirg\mN}{\mM\pesc\mN}
\qquad
\infer[(\subseteq^{\sim}_\alpha)]{\mM\eq\mN}{\mM\svirg\mN}\\[3pt]
% Pesciolino
\infer[(\approx_\alpha)]{\mM\pesc\mN}{\mM,\mN\reddh\stuck{}&\forall a\in\Addrs,\,\exists\gamma < \alpha\,.\,\append{\mM}{a} \eq[\gamma] \append{\mN}{a}}\\[3pt]
\begin{array}{ccc}
	% Svirgola
	\infer[(R_\alpha^\sim)]{\mM\repl{R_i}{a}\svirg\mM\repl{R_i}{b}}{\Lookinv a\svirg \Lookinv b}
	&\quad&
	\infer[(@_\alpha^\sim)]{\append{\mM}{a}\svirg\append{\mM}{b}}{\Lookinv a\svirg \Lookinv b}\\[3pt]
	%%%%
	\infer[(T_\alpha^\sim)]{\appT{\mM}{T}\svirg\appT{\mN}{T}}{\mM\svirg \mN&T\in\Tapes}
	&&
	\infer[(T_\alpha)]{\appT{\mM}{T}\eq\appT{\mN}{T}}{\mM\eq \mN&T\in\Tapes}\\[3pt]
	%%%%
	\infer[(R_\alpha)]{\mM\repl{R_i}{a}\eq\mM\repl{R_i}{b}}{\Lookinv a\eq \Lookinv b}
	&&
	\infer[(@_\alpha)]{\append{\mM}{a}\eq\append{\mM}{b}}{\Lookinv a\eq \Lookinv b}\\[3pt]
\end{array}~\\
%%%%
\infer[(\le^\approx_\alpha)]{\mM \pesc \mN}{\mM\pesc[\gamma]\mN&\gamma \le \alpha}
\quad
\infer[(\le^\sim_\alpha)]{\mM \svirg \mN}{\mM\svirg[\gamma]\mN&\gamma \le \alpha}
\quad
\infer[(\le_\alpha)]{\mM \eq \mN}{\mM\eq[\gamma]\mN&\gamma \le \alpha}
%%%%
\\
\infer[(\mathrm{Tr}_\alpha)]{\mM \eq \mN}{\mM\eq\mZ&\mZ\eq\mN}\\[-5ex]
\end{gather*}
\caption{Rules satisfied by $\pesc$, $\svirg$ and $\eq$, beyond reflexivity and symmetry.}\label{fig:Pesiolino}
\end{figure}
\end{defi}
The intuitive meanings of the relations $\eq,\svirg,\pesc$ are the following:
\begin{itemize}
\item $\mM\eq\mN\iff\mM\equivea\mN$ is derivable using the rule $\extrule$ at most $\alpha$ times;
\item $\mM\svirg\mN\iff\mM\eq\mN$ is derivable without using transitivity;
\item $\mM\pesc\mN\iff\mM\equivea\mN$ in case $\alpha = 0$. Otherwise, if $\alpha>0$ then
\item $\mM\pesc\mN\iff\mM\svirg\mN$ follows directly from the rule $\extrule$.
\end{itemize}

\noindent
More precisely, the rules $(\approx_0)$, $(\subseteq^{\approx}_\alpha)$, $(\subseteq^{\sim}_\alpha)$ express the fact that $\equiva\,\subseteq\,\pesc\,\subseteq\,\svirg\,\subseteq\,\eq$.
The rule $(\approx_\alpha)$ allows to prove $\mM \pesc \mN$, provided that both machines eventually get stuck and that $\appT{\mM}{[a]} \eq[\gamma_a] \appT{\mN}{[a]}$ is provable for every address $a$, using a smaller ordinal $\gamma_a < \alpha$.
The rules $(R_\alpha)$, $(@_\alpha)$ and $(T_\alpha)$ (resp.\ $(R_\alpha^\sim)$, $(@_\alpha^\sim)$ and $(T_\alpha^\sim)$) represent the contextuality of the relation $\eq$ (resp.\ $\svirg$).
The rules $(\le^\approx_\alpha)$, $(\le^\sim_\alpha)$ and $(\le_\alpha)$ specify that incrementing the ordinal (from top to bottom) is always allowed.
Finally, $(\mathrm{Tr}_\alpha)$ gives the transitivity of $\eq$.

The following lemma describes formally the intuitive meaning discussed above.
\begin{lem}\label{lem:relalphaprops}
Let $\mM,\mN\in\cM$
\begin{enumerate}[(i)]
\item\label{lem:relalphaprops1}\
	 $\mM\equivea\mN\iff\exists\alpha\in\Count\,.\,\mM\eq\mN$.
\item\label{lem:relalphaprops2}\
	$\mM\eq[0]\mN\iff\mM\equiv_\Addrs\mN$.
\item\label{lem:relalphaprops3}\
	$\mM\eq \mN\iff\exists n\ge0, \mZ_1,\dots,\mZ_n\in\cM\,.\, \mM\svirg\mZ_1\svirg\cdots\svirg\mZ_n =\mN$.
\item\label{lem:relalphaprops4}~\\[-3ex]
$
	\begin{array}{ll}
		\mM\svirg\mN\iff&\exists \mach{C}\in\cMX,\mach{M}',\mach{N}'\in\cM\,.\,\\
		&\mM=\mach{C}\hole{\mM'}\land\mN = \mach{C}\hole{\mN'} \land \mM'\pesc\mN'.\\
		\end{array}
	$
\item\label{lem:relalphaprops5}~\\[-2.7ex]
$
	\begin{array}{lcl}
		\mM\pesc\mN\land \alpha\neq 0&\iff&\mM,\mN\reddh\stuck{}\ \land\\
		&&\forall a\in\Addrs,\exists\gamma<\alpha\,.\, \append{\mM}{a}\eq[\gamma]\append{\mN}{a}.\\
		\end{array}
	$
\end{enumerate}
\end{lem}

\begin{proof}\ref{lem:relalphaprops1} $(\Leftarrow)$ Easy.

$(\Rightarrow)$ By induction on the length of a derivation of $\mM\equivea\mN$.

Case \redwerule. I.e., there exists $\mZ\in\cM$ such that $\mM\reddh\mZ\eqea\mN$.
By Theorem~\ref{thm:CR}, we have $\mM\equiva\mZ$ whence $\mM\eq[0]\mZ$ by $(\pesc[0])$, which implies $\mM\eq\mZ$ for all $\alpha\in\Count$ using the rule $(\le_\alpha)$. Now, consider the set
\[
	\cR = \set{ i \st \mZ.R_i \neq\Null} = \set{ i \st \mN.R_i \neq\Null}
\]
Note that $\cR= \set{i_1,\dots,i_k}$ for some $k<\mZ.r_0 (=\mN.r_0)$. For every $i\in\cR$, let $\mZ.R_i = a_i$ and $\mN.R_i = a'_i$. Also, let $\mZ.T = [b_1,\dots,b_m]$ and $\mN.T = [b'_1,\dots,b'_m]$. By assumption, $a_i\simea a'_i$ and $b_j\simea b'_j$ for every $i\in\cR$, and $j\,(1\le j\le m)$.
By induction hypothesis, $\Lookinv{a_i} \eq[\gamma_i] \Lookinv{a'_i}$ and $\Lookinv{b_j} \eq[\delta_j] \Lookinv{b'_j}$. Using the rule $(<_\alpha)$, the same holds for $\eq[\alpha]$ setting $\alpha = \sup_{i\in\cR,1\le j\le m} \set{\gamma_i,\delta_j}$. Putting everything together, we obtain:
\[
	\begin{array}{lcll}
	\mM&\eq&\mZ = \tuple{\mZ.\vec R,P,[b_1,\dots,b_m]}\\
	&\eq&\tuple{\mZ.\vec R\repl{R_{i_1}}{a'_{i_1}},P,[b_1,\dots,b_m]},&\textrm{by $(R_\alpha)$,}\\
	&\eq&\cdots&\qquad\vdots\\
	&\eq&\tuple{\mZ.\vec R[R_i:= a'_i]_{i\in\cR},P,[b_1,\dots,b_m]},&\textrm{by $(R_\alpha)$,}\\
	&=&\tuple{\mN.\vec R,P,[b_1,\dots,b_m]},&\textrm{by definition,}\\
	&\eq&\tuple{\mN.\vec R,P,[b'_1,b_2,\dots,b_m]},&\textrm{by $(T_\alpha)$,}\\
	&\eq&\cdots&\qquad\vdots\\
	&\eq&\tuple{\mN.\vec R,P,[b'_1,\dots,b'_m]},&\textrm{by $(T_\alpha)$,}\\
	&=&\mN,&\textrm{by definition.}\\
	\end{array}
\]
We conclude by applying the transitivity rule $(\mathrm{Tr}_\alpha)$ that $\mM \eq \mN$.

Case \extrule. By induction hypothesis, for every $a\in\Addrs$, there exists $\gamma_a\in\Count$ such that $\append{\mM}{a} \eq[\gamma_a]\append{\mN}{a}$.
For $\gamma = \sup_{a\in\Addrs}\gamma_a$, we get $\append{\mM}{a} \eq[\gamma]\append{\mN}{a}$ by $(\le_\alpha)$. By $(\pesc)$ we get $\mM\pesc\mN$ for $\alpha=\gamma+1\in\Count$, conclude by $(\subseteq^\approx_\alpha)$ and $(\subseteq^\sim_\alpha)$.

(Reflexivity), (Symmetry) and (Transitivity) follow from the respective property of $\eq$.

Concerning items~\ref{lem:relalphaprops2}--\ref{lem:relalphaprops5} the implication $(\Leftarrow)$ is trivial. We analyze $(\Rightarrow)$.

\ref{lem:relalphaprops2} % chktex 2
	By induction on a derivation of $\mM\eq[0]\mN$, using Theorem~\ref{thm:CR}.

\ref{lem:relalphaprops3} % chktex 2
    By induction on a derivation of $\mM\eq\mN$.

    Case $(\subseteq^\sim_\alpha)$. Trivial.

    Case $(R_\alpha)$. I.e., $\mM = \mZ\repl{R_i}{a}$, $\mN = \mZ\repl{R_i}{b}$ and $\Lookinv{a} \eq \Lookinv{b}$. By induction hypothesis, there exist $c_1,\dots,c_k\in\Addrs$ such that
    \[
    	\Lookinv{a}\svirg\Lookinv{c_1}\svirg\cdots\svirg\Lookinv{c_k}=\Lookinv{b}.
    \]
    The case follows by applying the rule $(R^\sim_\alpha)$.

    Case $(@_\alpha)$. Analogous, by applying $(@^\sim_\alpha)$.

    Case $(T_\alpha)$. Analogous, by applying $(T^\sim_\alpha)$.

    Case $(\mathrm{Tr}_\alpha)$. Straightforward from the IH\@.

    Case $(\le_\alpha)$. By IH and $(\le^\sim_\alpha)$.

	Cases (Reflexivity), (Symmetry). Straightforward from the IH\@.

\ref{lem:relalphaprops4} % chktex 2
	By induction on a derivation of $\mM\svirg\mN$.

	Case $(\subseteq^\approx_\alpha)$. Take $\C = \xi$.

	Case $(R^\sim_\alpha)$. I.e., $\mM = \mZ\repl{R_i}{a}$, $\mN = \mZ\repl{R_i}{b}$ and $\Lookinv{a} \svirg \Lookinv{b}$. By induction hypothesis, there exist $\C'\in\cMX$ having address $c = \Lup{\C'}\in\XX$, $\mM',\mN'\in\cM$ such that $\C'\hole{\mM'} = \Lookinv{a}$, $\C'\hole{\mN'} = \Lookinv{b}$ and $\mM' \pesc\mN'$. We conclude by taking $\C = \mZ\repl{R_i}{c}$.

	Case $(@^\sim_\alpha)$. Analogous.

	Case $(T^\sim_\alpha)$. Take $\C = \appT{\C'}{T}$, where $\C'$ is obtained from the IH\@.

	Case $(\le^\sim_\alpha)$. It follows from the IH, by applying $(\le^\sim_\alpha)$ and $(\le^\approx_\alpha)$.

	Cases (Reflexivity), (Symmetry). Straightforward from the IH\@.

\ref{lem:relalphaprops5} Immediate. % chktex 2
\end{proof}

Consider now a scenario where $\C\hole \mM\reddh \C'\hole{\mM}$.
Assuming $\mM\pesc\mN$, one might expect that also $\C\hole \mN\reddh \C'\hole{\mN}$ holds.
In general, this is not the case because $\mM$ and $\mN$ might reach the head position and get control of the computation.
Using the underlined (head-)reduction from Definition~\ref{def:weirdreds}\ref{def:weirdreds2} we can substitute $\mN$ for $\mM$ along the reduction (when it comes in head position) and construct a proof of $\C\hole \mN \eq[\gamma] \C'\hole\mN$ having a lower ordinal $\gamma < \alpha$.

\begin{lem}\label{lem:black_magic}
Let $\alpha > 0$, $\C\in\cMX$, $\mM,\mN\in\cM$ such that $\mM\pesc\mN$.
If $\C\redh^\mM\C'$ and $\C'\hole{\mM}\not\reddh\stuck{}$, then there exists $\gamma < \alpha$ such that $\C\hole \mN \eq[\gamma] \C'\hole\mN$.
\end{lem}

\begin{proof} By cases on the shape of $\C$.

Case $\C = \appT{\xi}{T}$ for some $T\in\Tapes[\XX]$ and $\C' = \appT{\mM}{T}$.
From  $\mM\pesc\mN$ and Lemma~\ref{lem:relalphaprops}\ref{lem:relalphaprops5}, we get that $\mM\reddh{\stuck{\mM'}}$ for some $\mM'\in\cM$.
Since $\C'\hole{\mM} = \appT{\mM}{(T\hole{\mM})}$ cannot reduce to a stuck \am, we must have $T\hole{\mM}\neq[]$.
In other words, $T = [a_0,\dots,a_n]$ for some $n\ge 0$.
Notice that, for all $a_i\in\Tapes[\XX]$, we have $a_i\hole{\mN}\in\Addrs$ (by construction).
By Lemma~\ref{lem:relalphaprops}\ref{lem:relalphaprops5}, there exists $\gamma<\alpha$ such that $\append{\mN}{a_0\hole{\mN}} \eq[\gamma] \append{\mM}{a_0\hole{\mN}}$. By definition:
\[
	\C\hole{\mN} = \appT{\mN}{T\hole{\mN}},\textrm{ and }
	\C'\hole{\mN}=\appT{\mM}{T\hole{\mN}}.
\]
So we construct the proof:
\[
	\infer[(T_\gamma)]{\append{\mN}{a_0\hole{\mN},\dots,a_n\hole{\mN}} \eq[\gamma] \append{\mM}{a_0\hole{\mN},\dots,a_n\hole{\mN}}}{
	\append{\mN}{a_0\hole{\mN}}\eq[\gamma] \append{\mM}{a_0\hole{\mN}}
	}
\]

In all the other cases, $\C\hole{\mN}\to_h\C'\hole{\mN}$, therefore $\C\hole{\mN} \eq[0]\C\hole{\mN}$.
\end{proof}

\begin{cor}\label{cor:black_magic}
Let $n\in\nat$, $\alpha > 0$, $\C\in\cMX$, $\mM,\mN\in\cM$.
If $\C\hole{\mM}\reddh\mach{x}_n$ and $\mM\pesc\mN$ then there exists $\gamma < \alpha$ such that $\C\hole{\mN} \eq[\gamma] \mach{x}_n$.
\end{cor}

\begin{proof} Assume $\C\hole{\mM}\reddh\mach{x}_n$. Equivalently, by Lemma~\ref{lem:chemmeserve}, we have $\C\reddh^\mM\mach{x}_n$.
By definition, there exists $\C_1,\dots,\C_k\in\cMX$ such that
\[
	\C = \C_1\to_h^\mM\cdots\to_h^\mM \C_k = \mach{x}_n
\]
Notice that $\C_i\hole{\mM}\reddh \mach{x}_n$ and, since $\lnot\stuck{\mach{x}_n}$, we have $\C_i\hole{\mN}\not\reddh\stuck{}$.
By Lemma~\ref{lem:black_magic}, there exists $\gamma_1,\dots,\gamma_k<\alpha$ such that $\C_i\hole{\mN}\eq[\gamma_i]\C_{i+1}\hole{\mN}$.
By transitivity $(\mathrm{Tr}_\alpha)$ and $(\le_\alpha)$ we obtain $\mM\eq\mach{x}_n$ for $\alpha = \sup_i{\gamma_i}$.
\end{proof}

\begin{prop} Let $\mM,\mN\in\cM$, $\alpha\in\Count$ and $n\in\nat$.
If $\mM\eq \mN$ and $\mN\reddh\mach{x}_n$ then $\mM\reddh\mach{x}_n$.
\end{prop}

\begin{proof} We proceed by induction on $\alpha$. Since we perform a double induction, the induction hypothesis with respect to this induction is called the $\alpha$-IH ($\alpha$-inductive hypothesis).

Case $\alpha = 0$. By Lemma~\ref{lem:relalphaprops}\ref{lem:relalphaprops2}, we get $\mM\equivea \mN\reddh\mach{x}_n$, so we conclude $\mM\reddh\mach{x}_n$ by confluence (Theorem~\ref{thm:CR}) and $\red[i]$-postponement (Lemma~\ref{lem:standardization}).

Case $\alpha > 0$. By Lemma~\ref{lem:relalphaprops}\ref{lem:relalphaprops3}, there exist $\mZ_1,\dots,\mZ_k$ such that
\begin{equation}\label{eq:svirg}
	\mM \svirg\mZ_1\svirg\cdots\svirg\mZ_k=\mN\reddh\mach{x}_n
\end{equation}
By induction on $k$, we prove that~\eqref{eq:svirg} implies $\mM\reddh\mach{x}_n$.
We call this $k$-IH\@.

Subcase $k =0$. Then $\mM =\mN\reddh\mach{x}_n$ and we are done.

Subcase $k >0$. From the $k$-IH we derive $\mZ_1\reddh\mach{x}_n$.
From $\mM\svirg\mZ_1$ and Lemma~\ref{lem:relalphaprops}\ref{lem:relalphaprops4}, there is a context-machine $\C$ such that $\mM = \C[\mM']$ and $\mZ_1 = \C[\mN']$ with $\mM'\pesc\mN'$ and $\C[\mN']\reddh\mach{x}_n$.
By applying Lemma~\ref{lem:black_magic} we obtain $\C[\mM'] \eq[\gamma] \mach{x}_n$ for some $\gamma<\alpha$.
We conclude by applying the $\alpha$-IH\@.
\end{proof}

From this proposition, Lemma~\ref{lem:about:equivo}\ref{lem:about:equivo2} follows by applying Lemma~\ref{lem:relalphaprops}\ref{lem:relalphaprops1}.
