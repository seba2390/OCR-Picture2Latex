\documentclass[11pt]{letter} % Default font size of the document, change to 10pt to fit more text

\usepackage{pdfpages}
\usepackage{graphicx}

\newcommand{\predit}{\textcolor{magenta}}
\newcommand{\lkedit}{\textcolor{blue}}

% Margins
\topmargin=-1in % Moves the top of the document 1 inch above the default
\textheight=9.5in % Total height of the text on the page before text goes on to the next page, this can be increased in a longer letter
\oddsidemargin=-10pt % Position of the left margin, can be negative or positive if you want more or less room
\textwidth=6.5in % Total width of the text, increase this if the left margin was decreased and vice-versa

\let\raggedleft\raggedright % Pushes the date (at the top) to the left, comment this line to have the date on the right

\begin{document}

\begin{letter}

Dear Editor,

We are submitting a manuscript titled ``Learning fluid dynamics using sparse physics-informed discovery of empirical relations" for your consideration as an article in {\em Nature Machine Intelligence}.  
While machine learning has the potential to revolutionize the process of scientific discovery using data-driven methods, there has not yet been a breakthrough enabling wide practical use of such methods.
Our work promises to enable such a breakthrough by offering a systematic way to synthesize new knowledge, presenting it in the form of algebraic or partial differential equations that is familiar to scientists or engineers and easy to interpret and generalize.
Mathematical models in the form of such equations represent the quintessential form of scientific knowledge and are indispensable to our fundamental understanding.

In practice, there are multiple barriers that have prevented the widespread adoption of data-driven model discovery. First, many systems of interest are very high-dimensional, making machine learning computationally intractable without imposing some constraints based on domain knowledge; however, this domain knowledge is usually limited in the first place. Second, although techniques such as deep learning have been able to successfully reproduce observations in many real-world systems, they struggle to produce descriptions that can be easily interpreted by humans. Finally, existing modeling approaches are often extremely sensitive to noise, precluding their applicability to problems where measurements are imperfect. This paper addresses all of these difficulties by proposing a robust and general framework for discovering a {\it complete} set of governing equations for a physical system from observational data, requiring only a minimal set of physical assumptions that hold for a very wide class of problems: {\it smoothness}, {\it locality}, and {\it known symmetries}. Specifically, we show that our approach identifies the evolution equations, physical constraints, and boundary conditions allowing a full characterization of a highly turbulent fluid flow, even in the presence of large amounts of noise. While some studies have previously identified evolution equations with some restrictive assumptions, to our knowledge, this work is the first to also reconstruct constraints and boundary conditions. 

While validated using synthetic data, the proposed approach is applicable to a wide variety of infinite-dimensional physical systems in science and engineering, such as fluid flows and suspensions, active matter, granular media, plasma, excitable media, and so on, fully or partially supplanting time-consuming conventional first-principles modeling approaches. Moreover, our approach provides information about the relative importance of different physical effects and can also serve as a valuable diagnostic tool. Therefore, we expect this paper to be of interest to a broad spectrum of readers of {\it Nature Machine Intelligence}.

\closing{Sincerely,\\ 
\vspace{2mm}

\vspace{2mm}
\fromname{Daniel Gurevich}}

\end{letter}

\end{document}

