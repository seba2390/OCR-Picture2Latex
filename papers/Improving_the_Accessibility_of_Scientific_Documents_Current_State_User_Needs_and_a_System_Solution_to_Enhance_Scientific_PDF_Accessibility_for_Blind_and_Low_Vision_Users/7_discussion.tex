\section{Discussion}
\label{sec:discussion}

In this work, we present the results of several studies that aim to characterize the current state of accessibility for academic paper PDFs, to learn the challenges faced by BLV researchers when reading papers, and to demonstrate how our \scially system that renders PDFs into accessible HTML can be used to mitigate many of these challenges. 

Based on our analysis, the current state of paper accessibility is grim, with an average of \percaccessible of papers across all fields of study satisfying our five assessed accessibility criteria. Though there is some improvement seen over time, we are not optimistic that these improvements are due to authors prioritizing accessibility when writing papers, since the presence of figure alt-text (the only of the five criteria that requires author intervention) remains low. Rather, the commitment to accessibility made by certain typesetting software providers such as Microsoft Word may be responsible for a portion of these improvements. Given the strong correlation between PDF creation software and accessibility compliance, we encourage conferences, publishers, and authors to consider the tools they are using to generate PDFs, and to integrate accessibility requirements during the publication process.

Given the scope and magnitude of the problem, and how PDF is still the dominant file type used for distributing scientific papers, there are clear needs for immediate technological solutions. We propose the \scially system, which integrates several text and vision machine learning models to extract the content from paper PDFs and render this content as HTML. The system adds tags and infers reading order, thereby improving the navigational capabilities of BLV users. Of course, no extractive pipeline is perfect, and we quantify and qualify extraction quality through an evaluation study and user study. Our intrinsic evaluation of extraction quality indicates that most extractions have no major problems affecting readability (86.2\% have no or only some problems). The most common extraction problems are incorrectly extracted or missed section headings, as well as headers, footers, and footnotes being improperly mixed into the body text, which can interrupt reading flow. Participants in our user study responded positively to \scially, preferring its navigational features and tagging to working with PDFs. Though the various types of extraction mistakes made by our system are noted by participants, most participants reported an improvement from their current reading pipeline, and all participants expressed an interest in using the system in the future. 

We present the challenges, coping mechanisms, and positive and negative features identified by participants. We also summarize the collective themes into a set of five design recommendations for other researchers and practitioners looking to design and build systems for accessible reading. The recommendations include (1) matching the document structure to the mental model of the user, (2) tagging all objects within the document appropriately, (3) acting as external memory for the user, (4) indicating known missing data or extraction errors, and (5) reducing verbosity. The first two of these recommendations are related to proper and correct representation of the document structure and in-paper objects. Both are necessary components of an accessible document. The third recommendation is to provide additional navigation features that are otherwise encoded in the visual layout of the document and inaccessible to BLV users. The fourth recommendation is related to error tolerance and user trust. For any machine learning-based document parsing system, errors are inevitable; managing user expectations for these systems is crucial. This recommendation echoes previously published guidelines for human-AI interaction, which suggest communicating to the user the capabilities and limitations of the AI system \citep{Amershi2019GuidelinesFH}. Setting expectations correctly and referring the user back to the original source document when the extractive procedure fails can help mitigate inappropriate reliance on the system. The final recommendation aims to reduce verbosity and the number of keystrokes needed for performing any task, which can speed up the use of such a system.

We hope these design recommendations will facilitate further conversations around the needs of BLV users, and that they may result in systems that ease the reading burden for these users. As one participant puts it, ``reading papers is the hardest part of research'' for researchers who are blind or low vision, and if papers were more accessible, ``there would be more blind researchers.'' It is a duty of the entire community to facilitate this, and to design, prototype, and build systems to support the needs of the BLV research community.

\subsection{Limitations \& Future Work}
\label{sec:future_work}

This work focuses on rendering PDF papers in HTML to improve document navigation and provide a more intuitive reading order. There are many other aspects of accessibility with which we do not contend, such as providing figure alt-text, accessible math, or tagging tables. Future work involves investigating various ways to improve or provide these features automatically, or by harnessing the power of the community to provide some of these features for papers as they are requested. For example, we may integrate element-specific reading features for mathematical equations \citep{Flores2010MathMLTA, Bates2010SpokenMU, Sorge2014TowardsMM, Mackowski2017MultimediaPF} or graphs and charts \citep{Elzer2008AccessibleBC, Engel2017TowardsAC, Engel2019SVGPlottAA}, or create a crowd-sourcing pipeline to solicit alt-text annotations for figures that lack descriptions.

PDF parsing remains an open research problem with many challenges. Our reliance on these technologies necessarily introduce error into our pipeline and system. We attempt to describe and quantify these errors in Appendix~\ref{app:eval_association}, but found no strong correlation between any particular type of error and the overall quality assessment. Unfortunately, this means that there is no obvious mitigation strategy for identifying low-quality extractions before they are shown to users. Further work remains to automatically or semi-automatically identify low-quality parses prior to surfacing them. For example, we could investigate other paper features as predictors of parse quality. With more labeled data, we could also train a neural classifier to identify low-quality parses. 

In this work, we focus on processing PDFs and making them accessible. Some papers are available in XML, HTML, or other structured markup languages; and LaTeX or Word document source can be found for others. Our system could take advantage of these alternatives to PDFs when they are publicly available, for example, by rendering the semantic content of the paper as extracted from these other document representations, as in arXiv Vanity\footnote{\href{http://www.arxiv-vanity.com/}{http://www.arxiv-vanity.com/}} for arXiv LaTeX source or Pubmed Central's PubReader,\footnote{\href{https://www.ncbi.nlm.nih.gov/pmc/about/pubreader/}{https://www.ncbi.nlm.nih.gov/pmc/about/pubreader/}} which renders JATS XML. Though S2ORC \citep{lo-wang-2020-s2orc} contains LaTeX parses derived from arXiv for over 1 million papers, further study is necessary to determine whether these parses are suitable for HTML rendering in our system.

Though we conduct a user study to better understand the challenges of BLV users and their responses to our prototype, the number of participants involved is small. Consequently, we focus on identifying qualitative learnings from these user studies. These learning, when combined with our evaluation and analysis of the current state of scholarly PDF accessibility, provide a more complete portrait of the challenges and issues BLV scholars face when reading papers. To more fully assess the benefits and flaws of our system, a broader user study and testing period is needed. We hope to achieve this in future work.

Lastly, PDFs have been repeatedly called out as being inaccessible, not only for screen readers, but broadly for reading, especially on mobile and other devices with small screen sizes \citep{NielsenPDFStillUnfit}. Dissociating publishing from PDFs continues to be a good goal for the future. In recent years, alternative publication formats have risen in popularity, such as eLife's dual publication in PDF and HTML,\footnote{\href{https://reviewer.elifesciences.org/author-guide/post}{https://reviewer.elifesciences.org/author-guide/post}} the interactive HTML papers at distill.pub,\footnote{\href{https://distill.pub/}{https://distill.pub/}} or the ACM Digital Library's very own dual publication (PDF and HTML) process,\footnote{\href{https://www.acm.org/publications/authors/submissions}{https://www.acm.org/publications/authors/submissions}} which is now available for many of the ACM's computing conferences and journals. We have no doubt that viable alternatives to PDF have and will arise, and encourage the community to explore these options when making publication decisions.