\section{Conclusion}

Based on our findings, most academic papers are inaccessible and significant challenges remain for BLV researchers when interacting with and reading these papers. Though some improvements in accessibility have been seen over time, these changes may not be reflective of author actions directly. In the meantime, we offer a potential solution for the millions of PDFs that have already been published and which still remain the dominant form of distribution for academic papers. We introduce the \scially system for rendering PDFs as accessible HTML documents. The system extracts the content of PDFs, tagging headings and objects and inferring reading order, which results in a more navigable and accessible document. Though the extraction pipeline is imperfect and can result in errors, our evaluation suggests that for the majority of papers, the resulting HTML render has no major problems that impact readability. We confirm these findings in our user study, where all users responded positively to the prototype system, claiming that they would be likely or very likely to use the system were it to be available in the future. Participants described the system as likely to ``become the workflow'' or ``life-changing,'' indicating both a strong favorable response and particular need for these types of solutions.

We do not claim that \scially solves all (or even close to all) accessibility problems for BLV researchers, but it is a step in the right direction.  \scially is a technological solution that can mitigate many of the challenges experienced by BLV researchers at this moment. Though a longer term solution would surely require more dialogue between all stakeholders and a potential revolution in the way in which scholars publish and distribute their research findings, we encourage researchers to prioritize and address these challenges with whatever tools they have in their toolbox right now. We especially encourage others to take into account our findings on the needs and challenges of BLV researchers when designing and engineering new systems and tools for reading the scholarly literature.