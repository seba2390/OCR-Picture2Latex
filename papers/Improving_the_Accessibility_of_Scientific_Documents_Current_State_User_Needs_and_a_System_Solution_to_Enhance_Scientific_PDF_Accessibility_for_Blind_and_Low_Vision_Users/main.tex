%%
%% This is file `sample-manuscript.tex',
%% generated with the docstrip utility.
%%
%% The original source files were:
%%
%% samples.dtx  (with options: `manuscript')
%% 
%% IMPORTANT NOTICE:
%% 
%% For the copyright see the source file.
%% 
%% Any modified versions of this file must be renamed
%% with new filenames distinct from sample-manuscript.tex.
%% 
%% For distribution of the original source see the terms
%% for copying and modification in the file samples.dtx.
%% 
%% This generated file may be distributed as long as the
%% original source files, as listed above, are part of the
%% same distribution. (The sources need not necessarily be
%% in the same archive or directory.)
%%
%% The first command in your LaTeX source must be the \documentclass command.
\documentclass[nonacm,screen,anonymous=false]{acmart}
\usepackage{xspace}
\usepackage{tabularx}
% \usepackage{fancyvrb}
\usepackage{alltt}
% \usepackage{mathabx}
\usepackage{geometry}
\usepackage{array}
\usepackage{enumitem}
\usepackage{caption}
\usepackage{subcaption}
% \usepackage{xeCJK}
\usepackage{ragged2e}

%%
%% \BibTeX command to typeset BibTeX logo in the docs
\AtBeginDocument{%
  \providecommand\BibTeX{{%
    \normalfont B\kern-0.5em{\scshape i\kern-0.25em b}\kern-0.8em\TeX}}}

%% Rights management information.  This information is sent to you
%% when you complete the rights form.  These commands have SAMPLE
%% values in them; it is your responsibility as an author to replace
%% the commands and values with those provided to you when you
%% complete the rights form.
% \setcopyright{acmcopyright}
\setcopyright{none}
\copyrightyear{2021}
% \acmYear{2021}
\acmDOI{}

%% These commands are for a PROCEEDINGS abstract or paper.
% \acmConference[CHI '21]{CHI '21: ACM CHI Conference on Human Factors in Computing Systems}{May 08--13, 2021}{Online}
% \acmBooktitle{CHI '21: ACM CHI Conference on Human Factors in Computing Systems,
%   May 08--13, 2021, Online}
% \acmPrice{15.00}
% \acmISBN{978-1-4503-XXXX-X/18/06}


%%
%% Submission ID.
%% Use this when submitting an article to a sponsored event. You'll
%% receive a unique submission ID from the organizers
%% of the event, and this ID should be used as the parameter to this command.
%%\acmSubmissionID{123-A56-BU3}

%%
%% The majority of ACM publications use numbered citations and
%% references.  The command \citestyle{authoryear} switches to the
%% "author year" style.
%%
%% If you are preparing content for an event
%% sponsored by ACM SIGGRAPH, you must use the "author year" style of
%% citations and references.
%% Uncommenting
%% the next command will enable that style.
%%\citestyle{acmauthoryear}

%%
%% end of the preamble, start of the body of the document source.
\begin{document}

\usepackage{booktabs} 
\usepackage{amsmath,url}
\let\Bbbk\relax
\usepackage{amssymb}
 
\usepackage{amsfonts}
% \usepackage{ctable}
\usepackage{multirow}
% \usepackage{algorithm}
% \usepackage{algpseudocode}
% \usepackage{pifont}
\usepackage{color}
% \usepackage{bbm}
\usepackage{enumitem}
\usepackage{dsfont}
\usepackage{graphicx}
\usepackage{subcaption}
% \let\comment\undefined
% \usepackage[commentmarkup=margin]{changes}
% NOTE: I have to undefined \comment since I want to use the \comment environment
% provided by the verbatim package




\newcommand{\todo}[1]{\textcolor{blue}{\bf #1}}
\newcommand{\fixme}[1]{\textcolor{red}{\bf #1}}

\newcommand{\mc}[3]{\multicolumn{#1}{#2}{#3}}
\newcommand{\mr}[2]{\multirow{#1}{0.10\textheight}{#2}}
% \newcommand{\he}[1]{{\textsf{\textcolor{red}{[From He: #1]}}}}

\newcommand{\mylistbegin}{
  \begin{list}{$\bullet$}
   {
     \setlength{\itemsep}{-2pt}
     \setlength{\leftmargin}{1em}
     \setlength{\labelwidth}{1em}
     \setlength{\labelsep}{0.5em} } }
\newcommand{\mylistend}{
   \end{list}  }

\newcommand{\eg}{\textit{e.g.}}
\newcommand{\xeg}{\textit{E.g.}}
\newcommand{\ie}{\textit{i.e.}}
\newcommand{\etc}{\textit{etc}}
\newcommand{\etal}{\textit{et al.}}
\newcommand{\wrt}{\textit{w.r.t.~}}
\newcommand{\header}[1]{{\vspace{+1mm}\flushleft \textbf{#1}}}
\newcommand{\sheader}[1]{{\flushleft \textit{#1}}}
\newcommand{\CGIR}{\textit{CGIR}}

\newcommand{\floor}[1]{\lfloor #1 \rfloor}
\newcommand{\ceil}[1]{\lceil #1 \rceil}

\newcommand{\bx}{\boldsymbol{x}}
\newcommand{\by}{\boldsymbol{y}}
\newcommand{\ba}{\boldsymbol{a}}
\newcommand{\bw}{\boldsymbol{w}}
\newcommand{\bW}{\boldsymbol{W}}
\newcommand{\bfn}{\boldsymbol{f}}
\newcommand{\blambda}{\boldsymbol{\lambda}}
\newcommand{\btheta}{\boldsymbol{\theta}}

\newcommand{\mcW}{\mathcal{W}}
\newcommand{\mcY}{\mathcal{Y}}
\newcommand{\mcS}{\mathcal{S}}
\newcommand{\mcA}{\mathcal{A}}
\newcommand{\mcV}{\mathcal{V}}
\newcommand{\mcE}{\mathcal{E}}
\newcommand{\mcG}{\mathcal{G}}

\DeclareMathOperator*{\argmax}{arg\,max}
\DeclareMathOperator*{\argmin}{arg\,min}

\let\comment\undefined

%%
%% The "title" command has an optional parameter,
%% allowing the author to define a "short title" to be used in page headers.
\title{Improving the accessibility of scientific documents}
\subtitle{Current state, user needs, and a system solution to enhance scientific PDF accessibility for blind and low vision users}

%%
%% The "author" command and its associated commands are used to define
%% the authors and their affiliations.
%% Of note is the shared affiliation of the first two authors, and the
%% "authornote" and "authornotemark" commands
%% used to denote shared contribution to the research.

\author{Lucy Lu Wang}
\authornote{Denotes equal contribution}
\email{lucyw@allenai.org}
\orcid{0000-0001-8752-6635}
\affiliation{%
  \institution{Allen Institute for AI}
%   \streetaddress{2157 N Northlake Way #110}
  \city{Seattle}
  \state{WA}
  \postcode{98103}
}

\author{Isabel Cachola}
\authornotemark[1]
\authornote{Work done while at the Allen Institute for AI}
\email{icachola@cs.jhu.edu}
\orcid{}
\affiliation{%
  \institution{The Johns Hopkins University}
%   \streetaddress{3400 N. Charles Street}
  \city{Baltimore}
  \state{MD}
  \postcode{21218}
}

\author{Jonathan Bragg}
\email{jbragg@allenai.org}
\orcid{}
\affiliation{%
  \institution{Allen Institute for AI}
%   \streetaddress{2157 N Northlake Way #110}
  \city{Seattle}
  \state{WA}
  \postcode{98103}
}

\author{Evie Yu-Yen Cheng}
\email{eviec@allenai.org}
\orcid{}
\affiliation{%
  \institution{Allen Institute for AI}
%   \streetaddress{2157 N Northlake Way #110}
  \city{Seattle}
  \state{WA}
  \postcode{98103}
}

\author{Chelsea Haupt}
\email{chealseah@allenai.org}
\orcid{}
\affiliation{%
  \institution{Allen Institute for AI}
%   \streetaddress{2157 N Northlake Way #110}
  \city{Seattle}
  \state{WA}
  \postcode{98103}
}

\author{Matt Latzke}
\email{mattl@allenai.org}
\orcid{}
\affiliation{%
  \institution{Allen Institute for AI}
%   \streetaddress{2157 N Northlake Way #110}
  \city{Seattle}
  \state{WA}
  \postcode{98103}
}

\author{Bailey Kuehl}
\email{baileyk@allenai.org}
\orcid{}
\affiliation{%
  \institution{Allen Institute for AI}
%   \streetaddress{2157 N Northlake Way #110}
  \city{Seattle}
  \state{WA}
  \postcode{98103}
}

\author{Madeleine van Zuylen}
\email{madeleinev@allenai.org}
\orcid{}
\affiliation{%
  \institution{Allen Institute for AI}
%   \streetaddress{2157 N Northlake Way #110}
  \city{Seattle}
  \state{WA}
  \postcode{98103}
}

\author{Linda Wagner}
\email{lindaw@allenai.org}
\orcid{}
\affiliation{%
  \institution{Allen Institute for AI}
%   \streetaddress{2157 N Northlake Way #110}
  \city{Seattle}
  \state{WA}
  \postcode{98103}
}

\author{Daniel S. Weld}
\email{danw@allenai.org}
\orcid{}
\affiliation{%
  \institution{Allen Institute for AI}
%   \streetaddress{2157 N Northlake Way #110}
  \city{Seattle}
  \state{WA}
  \postcode{98103}
}
\affiliation{%
  \institution{University of Washington}
%   \streetaddress{1410 NE Campus Parkway}
  \city{Seattle}
  \state{WA}
  \postcode{98103}
}


%%
%% By default, the full list of authors will be used in the page
%% headers. Often, this list is too long, and will overlap
%% other information printed in the page headers. This command allows
%% the author to define a more concise list
%% of authors' names for this purpose.
\renewcommand{\shortauthors}{Wang LL and Cachola I et al}

%%
%% The abstract is a short summary of the work to be presented in the
%% article.
\begin{abstract}
  The majority of scientific papers are distributed in PDF, which pose challenges for accessibility, especially for blind and low vision (BLV) readers. We characterize the scope of this problem by assessing the accessibility of \numpdfs PDFs published 2010--2019 sampled across various fields of study, finding that only \percaccessible of these PDFs satisfy all of our defined accessibility criteria. We introduce the \scially system to offset some of the issues around inaccessibility. \scially incorporates several machine learning models to extract the content of scientific PDFs and render this content as accessible HTML, with added novel navigational features to support screen reader users. An intrinsic evaluation of extraction quality indicates that the majority of HTML renders (87\%) produced by our system have no or only some readability issues. We perform a qualitative user study to understand the needs of BLV researchers when reading papers, and to assess whether the \scially system could address these needs. We summarize our user study findings into a set of five design recommendations for accessible scientific reader systems. User response to \scially was positive, with all users saying they would be likely to use the system in the future, and some stating that the system, if available, would become their primary workflow. We successfully produce HTML renders for over 12M papers, of which an open access subset of 1.5M are available for browsing at \href{https://scia11y.org/}{scia11y.org}.
\end{abstract}

%%
%% The code below is generated by the tool at http://dl.acm.org/ccs.cfm.
%% Please copy and paste the code instead of the example below.
%%
\begin{CCSXML}
<ccs2012>
   <concept>
       <concept_id>10003120.10011738.10011773</concept_id>
       <concept_desc>Human-centered computing~Empirical studies in accessibility</concept_desc>
       <concept_significance>500</concept_significance>
       </concept>
   <concept>
       <concept_id>10003120.10011738.10011776</concept_id>
       <concept_desc>Human-centered computing~Accessibility systems and tools</concept_desc>
       <concept_significance>500</concept_significance>
       </concept>
   <concept>
       <concept_id>10003120.10003121.10003122</concept_id>
       <concept_desc>Human-centered computing~HCI design and evaluation methods</concept_desc>
       <concept_significance>300</concept_significance>
       </concept>
   <concept>
       <concept_id>10003120.10011738.10011774</concept_id>
       <concept_desc>Human-centered computing~Accessibility design and evaluation methods</concept_desc>
       <concept_significance>300</concept_significance>
       </concept>
 </ccs2012>
\end{CCSXML}

\ccsdesc[500]{Human-centered computing~Empirical studies in accessibility}
\ccsdesc[500]{Human-centered computing~Accessibility systems and tools}
\ccsdesc[300]{Human-centered computing~HCI design and evaluation methods}
\ccsdesc[300]{Human-centered computing~Accessibility design and evaluation methods}

%%
%% Keywords. The author(s) should pick words that accurately describe
%% the work being presented. Separate the keywords with commas.
\keywords{accessibility, accessible reader, scientific documents, blind and low vision readers, science of science, user study}

%% A "teaser" image appears between the author and affiliation
%% information and the body of the document, and typically spans the
%% page.
% \begin{teaserfigure}
%   \includegraphics[width=\textwidth]{sampleteaser}
%   \caption{Seattle Mariners at Spring Training, 2010.}
%   \Description{Enjoying the baseball game from the third-base
%   seats. Ichiro Suzuki preparing to bat.}
%   \label{fig:teaser}
% \end{teaserfigure}

%%
%% This command processes the author and affiliation and title
%% information and builds the first part of the formatted document.
\maketitle

%% Sections
\begin{figure}[t]
\begin{center}
   \includegraphics[width=1.0\linewidth]{figures/nas_comp_v3}
\end{center}
   \vspace{-4mm}
   \caption{The comparison between NetAdaptV2 and related works. The number above a marker is the corresponding total search time measured on NVIDIA V100 GPUs.}
\label{fig:nas_comparison}
\end{figure}

\section{Introduction}
\label{sec:introduction}

Neural architecture search (NAS) applies machine learning to automatically discover deep neural networks (DNNs) with better performance (e.g., better accuracy-latency trade-offs) by sampling the search space, which is the union of all discoverable DNNs. The search time is one key metric for NAS algorithms, which accounts for three steps: 1) training a \emph{super-network}, whose weights are shared by all the DNNs in the search space and trained by minimizing the loss across them, 2) training and evaluating sampled DNNs (referred to as \emph{samples}), and 3) training the discovered DNN. Another important metric for NAS is whether it supports non-differentiable search metrics such as hardware metrics (e.g., latency and energy). Incorporating hardware metrics into NAS is the key to improving the performance of the discovered DNNs~\cite{eccv2018-netadapt, Tan2018MnasNetPN, cai2018proxylessnas, Chen2020MnasFPNLL, chamnet}.


There is usually a trade-off between the time spent for the three steps and the support of non-differentiable search metrics. For example, early reinforcement-learning-based NAS methods~\cite{zoph2017nasreinforcement, zoph2018nasnet, Tan2018MnasNetPN} suffer from the long time for training and evaluating samples. Using a super-network~\cite{yu2018slimmable, Yu_2019_ICCV, autoslim_arxiv, cai2020once, yu2020bignas, Bender2018UnderstandingAS, enas, tunas, Guo2020SPOS} solves this problem, but super-network training is typically time-consuming and becomes the new time bottleneck. The gradient-based methods~\cite{gordon2018morphnet, liu2018darts, wu2018fbnet, fbnetv2, cai2018proxylessnas, stamoulis2019singlepath, stamoulis2019singlepathautoml, Mei2020AtomNAS, Xu2020PC-DARTS} reduce the time for training a super-network and training and evaluating samples at the cost of sacrificing the support of non-differentiable search metrics. In summary, many existing works either have an unbalanced reduction in the time spent per step (i.e., optimizing some steps at the cost of a significant increase in the time for other steps), which still leads to a long \emph{total} search time, or are unable to support non-differentiable search metrics, which limits the performance of the discovered DNNs.

In this paper, we propose an efficient NAS algorithm, NetAdaptV2, to significantly reduce the \emph{total} search time by introducing three innovations to \emph{better balance} the reduction in the time spent per step while supporting non-differentiable search metrics:

\textbf{Channel-level bypass connections (mainly reduce the time for training and evaluating samples, Sec.~\ref{subsec:channel_level_bypass_connections})}: Early NAS works only search for DNNs with different numbers of filters (referred to as \emph{layer widths}). To improve the performance of the discovered DNN, more recent works search for DNNs with different numbers of layers (referred to as \emph{network depths}) in addition to different layer widths at the cost of training and evaluating more samples because network depths and layer widths are usually considered independently. In NetAdaptV2, we propose \emph{channel-level bypass connections} to merge network depth and layer width into a single search dimension, which requires only searching for layer width and hence reduces the number of samples.

\textbf{Ordered dropout (mainly reduces the time for training a super-network, Sec.~\ref{subsec:ordered_droput})}: We adopt the idea of super-network to reduce the time for training and evaluating samples. In previous works, \emph{each} DNN in the search space requires one forward-backward pass to train. As a result, training multiple DNNs in the search space requires multiple forward-backward passes, which results in a long training time. To address the problem, we propose \emph{ordered dropout} to jointly train multiple DNNs in a \emph{single} forward-backward pass, which decreases the required number of forward-backward passes for a given number of DNNs and hence the time for training a super-network.

\textbf{Multi-layer coordinate descent optimizer (mainly reduces the time for training and evaluating samples and supports non-differentiable search metrics, Sec.~\ref{subsec:optimizer}):} NetAdaptV1~\cite{eccv2018-netadapt} and MobileNetV3~\cite{Howard_2019_ICCV}, which utilizes NetAdaptV1, have demonstrated the effectiveness of the single-layer coordinate descent (SCD) optimizer~\cite{book2020sze} in discovering high-performance DNN architectures. The SCD optimizer supports both differentiable and non-differentiable search metrics and has only a few interpretable hyper-parameters that need to be tuned, such as the per-iteration resource reduction. However, there are two shortcomings of the SCD optimizer. First, it only considers one layer per optimization iteration. Failing to consider the joint effect of multiple layers may lead to a worse decision and hence sub-optimal performance. Second, the per-iteration resource reduction (e.g., latency reduction) is limited by the layer with the smallest resource consumption (e.g., latency). It may take a large number of iterations to search for a very deep network because the per-iteration resource reduction is relatively small compared with the network resource consumption. To address these shortcomings,  we propose the \emph{multi-layer coordinate descent (MCD) optimizer} that considers multiple layers per optimization iteration to improve performance while reducing search time and preserving the support of non-differentiable search metrics.

Fig.~\ref{fig:nas_comparison} (and Table~\ref{tab:nas_result}) compares NetAdaptV2 with related works. NetAdaptV2 can reduce the search time by up to $5.8\times$ and $2.4\times$ on ImageNet~\cite{imagenet_cvpr09} and NYU Depth V2~\cite{nyudepth} respectively and discover DNNs with better performance than state-of-the-art NAS works. Moreover, compared to NAS-discovered MobileNetV3~\cite{Howard_2019_ICCV}, the discovered DNN has $1.8\%$ higher accuracy with the same latency.


\section{Related work}
\label{sec:related_work}

Accessibility is an essential component of computing, which aims to make technology broadly accessible to as many users as possible, including those with differing sets of abilities. Improvements in usability and accessibility falls to the community, to better understand the needs of users with differing abilities, and to design technologies that play to this spectrum of abilities \citep{Wobbrock2011AbilityBasedDC}.
In computing, significant strides have been made to increase the accessibility of web content. For example, various versions of the Web Content Accessibility Guidelines (WCAG) \citep{Chisholm2001WebCA, Caldwell2008WebCA} and the in-progress working draft for WCAG 3.0,\footnote{\href{https://www.w3.org/TR/wcag-3.0/}{https://www.w3.org/TR/wcag-3.0/}} or standards such as ARIA from the W3C's Web Accessibility Initiative (WAI)\footnote{\href{https://www.w3.org/WAI/standards-guidelines/aria/}{https://www.w3.org/WAI/standards-guidelines/aria/}} have been released and used to guide web accessibility design and implementation. Similarly, positive steps have been made to improve the accessibility of user interfaces and user experience \citep{Peissner2012MyUIGA, Peissner2013UserCI, Thompson2014ImprovingTU, Bigham2014MakingTW}, as well as various types of media content \citep{Mirri2017TowardsAG, Nengroo2017AccessibleI, Gleason2020TwitterAA}. 

We take inspiration from accessibility design principles in our effort to make research publications more accessible to users who are blind and low vision. Blindness and low vision are some of the most common forms of disability, affecting an estimated 3--10\% of Americans depending on how visual impairment is defined \citep{CDCVisionLossBurden}. BLV researchers also make up a representative sample of researchers in the United States and worldwide. A recent Nature editorial pushes the scientific community to better support researchers with visual impairments \citep{NatureCareerColumn2020}, since existing tools and resources can be limited. There are many inherent accessibility challenges to performing research. In this paper, we engage with one of these challenges that affects all domains of study, accessing and reading the content of academic publications. 

BLV users interact with papers using screen readers, braille displays, text-to-speech, and other assistive tools. A WebAIM survey of screen reader users found that the vast majority (75.1\%) of respondents indicate that PDF documents are very or somewhat likely to pose significant accessibility issues.\footnote{\href{https://webaim.org/projects/screenreadersurvey8/}{https://webaim.org/projects/screenreadersurvey8/}} Most paper are published in PDF, which is inherently inaccessible, due in large part to its conflation of visual layout information with semantic content \citep{NielsenPDFStillUnfit, Bigham2016AnUT}. 
\citet{Bigham2016AnUT} describe the historical reasons we use PDF as the standard document format for scientific publications, as well as the barriers the format itself presents to accessibility. Prior work on scientific accessibility have made recommendations for how to make PDFs more accessible \cite{Rajkumar2020PDFAO, Darvishy2018PDFAT}, including greater awareness for what constitutes an accessible PDF and better tooling for generating accessible PDFs. Some work has focused on addressing components of paper accessibility, such as the correct way for screen readers to interpret and read mathematical equations \citep{Flores2010MathMLTA, Bates2010SpokenMU, Sorge2014TowardsMM, Mackowski2017MultimediaPF, Ahmetovic2018AxessibilityAL, Ferreira2004EnhancingTA, Sojka2013AccessibilityII}, describe charts and figures \citep{Elzer2008AccessibleBC, Engel2017TowardsAC, Engel2019SVGPlottAA}, automatically generate figure captions \citep{Chen2019NeuralCG, Qian2020AFS}, or automatically classify the content of figures \citep{Kim2018MultimodalDL}. Other work applicable to all types of PDF documents aims to improve automatic text and layout detection of scanned documents \cite{Nazemi2014PracticalSM} and extract table content \cite{Fan2015TableRD, Rastan2019TEXUSAU}. In this work, we focus on the issue of representing overall document structure, and navigation within that structure. Being able to quickly navigate the contents of a paper through skimming and scanning is an essential reading technique \citep{Maxwell1972SkimmingAS}, which is currently under-supported by PDF documents and PDF readers when reading these documents by screen reader. 

There also exists a variety of automatic and manual tools that assess and fix accessibility compliance issues in PDFs, including the Adobe Acrobat Pro Accessibility Checker\footnote{\href{https://www.adobe.com/accessibility/products/acrobat/using-acrobat-pro-accessibility-checker.html}{https://www.adobe.com/accessibility/products/acrobat/using-acrobat-pro-accessibility-checker.html}}, Common Look\footnote{\href{https://monsido.com/monsido-commonlook-partnership}{https://monsido.com/monsido-commonlook-partnership}}, ABBYY FineReader\footnote{\href{https://pdf.abbyy.com/}{https://pdf.abbyy.com/}}, PAVE\footnote{\href{https://pave-pdf.org/faq.html}{https://pave-pdf.org/faq.html}}, and PDFA Inspector\footnote{\href{https://github.com/pdfae/PDFAInspector}{https://github.com/pdfae/PDFAInspector}}. To our knowledge, PAVE and PDFA Inspector are the only non-proprietary, open-source tools for this purpose. Based on our experiences, however, all of these tools require some degree of human intervention to properly tag a scientific document, and tagging and fixing must be performed for each new version of a PDF, regardless of how minor the change may be.

Guidelines and policy changes have been introduced in the past decade to ameliorate some of the issues around scientific PDF accessibility. Some conferences, such as The ACM CHI Virtual Conference on Human Factors in Computing Systems (CHI) and The ACM SIGACCESS Conference on Computers and Accessibility (ASSETS), have released guidelines for creating accessible submissions.\footnote{See \href{http://chi2019.acm.org/authors/papers/guide-to-an-accessible-submission/}{http://chi2019.acm.org/authors/papers/guide-to-an-accessible-submission/} and \href{https://assets19.sigaccess.org/creating_accessible_pdfs.html}{https://assets19.sigaccess.org/creating\_accessible\_pdfs.html}} The ACM Digital Library\footnote{\href{https://dl.acm.org/}{https://dl.acm.org/}} provides some publications in HTML format, which is easier to make accessible than PDF~\cite{Graells2007EstudioDL}. \citet{Ribera2019PublishingAP} conducted a case study on DSAI 2016 (Software Development and Technologies for Enhancing Accessibility and Fighting Infoexclusion). The authors of DSAI were responsible for creating accessible proceedings and identified barriers to creating accessible proceedings, including lack of sufficient tooling and lack of awareness of accessibility. The authors recommended creating a new role in the organizing committee dedicated to accessible publishing. These policy changes have led to improvements in localized communities, but have not been widely adopted by all academic publishers and conference organizers.

Table~\ref{tab:prior_work} lists prior studies that have analyzed PDF accessibility of academic papers, and shows how our study compares. Prior work has primarily focused on papers published in Human-Computer Interaction and related fields, specific to certain publication venues, while our analysis tries to quantify paper accessibility more broadly.
\citet{Brady2015CreatingAP} quantified the accessibility of 1,811 papers from CHI 2010-2016, ASSETS 2014, and W4A, assessing the presence of document tags, headers, and language. They found that compliance improved over time as a response to conference organizers offering to make papers accessible as a service to any author upon request. \citet{Lazar2017MakingTF} conducted a study quantifying accessibility compliance at CHI from 2010 to 2016 as well as ASSETS 2015,
%\jb{Define acronyms in prev para}
confirming the results of \citet{Brady2015CreatingAP}. They found that across 5 accessibility criteria, the rate of compliance was less than 30\% for CHI papers in each of the 7 years that were studied. The study also analyzed papers from ASSETS 2015, an ACM conference explicitly focused on accessibility, and found that those papers had significantly higher rates of compliance, with over 90\% of the papers being tagged for correct reading order and no criteria having less than 50\% compliance. This finding indicates that community buy-in is an important contributor to paper accessibility.
\citet{Nganji2015ThePD} conducted a study of 200 PDFs of papers published in four disability studies journals, finding that accessibility compliance was between 15-30\% for the four journals analyzed, with some publishers having higher adherence than others. To date, no large scale analysis of scientific PDF accessibility has been conducted outside of disability studies and HCI, due in part to the challenge of scaling such an analysis. We believe such an analysis is useful for establishing a baseline and characterizing routes for future improvement. Consequently, as part of this work, we conduct an analysis of scientific PDF accessibility across various fields of study, and report our findings relative to prior work. 


\begin{table}[t!]
\small
    \centering
    \begin{tabularx}{\linewidth}{L{22mm}L{15mm}L{48mm}L{16mm}L{34mm}}
        \toprule
        \textbf{Prior work} & \textbf{PDFs analyzed} & \textbf{Venues} & \textbf{Year} & \textbf{Accessibility checker} \\
        \midrule
        \citet{Brady2015CreatingAP} & 1811 & CHI, ASSETS and W4A & 2011--2014 & PDFA Inspector \\ [0.5mm]
        \hline \\ [-2.5mm]
        \citet{Lazar2017MakingTF} & 465 + 32 & CHI and ASSETS & 2014--2015 & Adobe Acrobat Action Wizard \\ [0.5mm]
        \hline \\ [-2.5mm]
        \citet{Ribera2019PublishingAP} & 59 & DSAI & 2016 & Adobe PDF Accessibility Checker 2.0 \\ [0.5mm]
        \hline \\ [-2.5mm]
        \citet{Nganji2015ThePD} & 200 & \textit{Disability \& Society}, \textit{Journal of Developmental and Physical Disabilities}, \textit{Journal of Learning Disabilities}, and \textit{Research in Developmental Disabilities} & 2009--2013 & Adobe PDF Accessibility Checker 1.3 \\ [0.6mm]
        \hline \\ [-2.5mm]
        \textbf{\textit{Our analysis}} & \numpdfs & Venues across various fields of study & 2010--2019 & Adobe Acrobat Accessibility Plug-in Version 21.001.20145 \\
        \bottomrule
    \end{tabularx}
    \caption{Prior work has investigated PDF accessibility for papers published in specific venues such as CHI, ASSETS, W4A, DSAI, or various disability journals. Several of these works were conducted manually, and were limited to a small number of papers, while the more thorough analysis was conducted for CHI and ASSETS, two conference venues focused on accessibility and HCI. Our study expands on this prior work to investigate accessibility over \numpdfs PDFs sampled from across different fields of study.
    }
    % \Description{
    % Prior work, PDFs analyzed, Venues, Year, Accessibility checker 
    % Brady et al. [7], 1811, CHI, ASSETS and W4A, 2011--2014, PDFA Inspector 
    % Lazar et al. [23], 465 + 32, CHI and ASSETS, 2014--2015, Adobe Acrobat Action Wizard 
    % Ribera et al. [40], 59, DSAI, 2016, Adobe PDF Accessibility Checker 2.0 
    % Nganji [33], 200, Disability & Society, Journal of Developmental and Physical Disabilities, Journal of Learning Disabilities, and Research in Developmental Disabilities, 2009--2013, Adobe PDF Accessibility Checker 1.3
    % Our analysis, 11397, Venues across various fields of study, 2010--2019, Adobe Acrobat Accessibility Plug-in Version 21.001.20145 
    % }
    \label{tab:prior_work}
\end{table}
\section{Analysis of academic PDF accessibility}
\label{sec:sos}

To capture and better characterize the scope and depth of the problems around academic PDF accessibility, we perform a broad meta-scientific analysis. We aim to measure the extent of the problem (e.g., what proportion of papers have accessible PDFs?), whether the state of PDF accessibility is improving over time (e.g., are papers published in 2019 more likely to be accessible than those published in 2010?), and whether the typesetting software used to create a paper is associated with the accessibility of its PDF (e.g., are papers created using Microsoft Word more or less accessible than papers created with other software?).

Prior studies on PDF accessibility have been limited to papers from specific publication venues such as CHI, ASSETS, W4A, DSAI, and journals in disability research. Notably, these venues are closer to the field of accessible computing, and are consequently more invested in accessibility.\footnote{See submission and accessibility guidelines for ASSETS (\href{https://assets19.sigaccess.org/creating_accessible_pdfs.html}{https://assets19.sigaccess.org/creating\_accessible\_pdfs.html}), CHI (\href{https://chi2021.acm.org/for-authors/presenting/papers/guide-to-an-accessible-submission}{https://chi2021.acm.org/ for-authors/presenting/papers/guide-to-an-accessible-submission}), W4A (\href{http://www.w4a.info/2021/submissions/technical-papers/}{http://www.w4a.info/2021/submissions/technical-papers/}) and DSAI (\href{http://dsai.ws/2020/submissions/}{http://dsai.ws/2020/submissions/}).}  We expand upon this work by investigating accessibility trends across various fields of study and publication venues. Our goal is to characterize the overall state of paper PDF accessibility and identify ongoing challenges to accessibility going forward.

\subsection{Data \& methods}\label{subsec:data-methods}

We sample PDFs from the Semantic Scholar literature corpus \citep{Ammar2018ConstructionOT} for analysis. We construct a dataset of papers by sampling PDFs published in the years of 2010--2019 stratified across the 19 top level fields of study defined by Microsoft Academic Graph \citep{msr:mag1, Shen2018AWS}. Examples of fields include Biology, Computer Science, Physics, Sociology, and others. This dataset allows us to investigate the overall state of PDF accessibility for academic papers, and to study the relationship between field of study and PDF accessibility. 

For each field of study, we sample papers from the top venues by total citation count, along with some documents without venue information, which include things like books and book chapters. The resulting papers come from 1058 unique publication venues; for each field of study, between 29 and 110 publication venues are represented, with Art on the minimum end, and Economics and Computer Science on the maximum end. Each field is represented by an average of 65 different publication venues. The vast majority of documents sampled into our dataset are published papers, rather than preprints or other non-peer-reviewed manuscripts. Publication venues represented in our sample are generally highly reputable journals, for example, \textit{The Lancet} or \textit{Neurology} for Medicine, \textit{The Astrophysical Journal} and \textit{Physical Review Letters} for Physics, or various IEEE publications for Computer Science and Engineering. In some cases, the mapping between publication venue and field of study can be unclear; for example, the publication venue \textit{Mathematical Problems in Engineering} is associated with Mathematics in our sample rather than Engineering. From an examination of the data, classifications seem reasonable and could be justified. We estimate that around 2.2\% of the sample are conference papers, 6.1\% are book chapters, reports, or lecture notes, less than 0.5\% are preprints, and the remaining majority are journal publications. We believe this is a reasonably representative sample of paper-like documents available to scholars and researchers.

We analyze the PDFs in our dataset using the Adobe Acrobat Pro DC PDF accessibility checker.\footnote{\href{https://www.adobe.com/accessibility/products/acrobat/using-acrobat-pro-accessibility-checker.html}{https://www.adobe.com/accessibility/products/acrobat/using-acrobat-pro-accessibility-checker.html}} Though this checker is proprietary and requires a paid license, it is the most comprehensive accessibility checker available and has been used in prior work on accessibility \citep{Lazar2017MakingTF, Ribera2019PublishingAP, Nganji2015ThePD}. Alternatively, non-proprietary PDF parsers such as PDFBox\footnote{\url{https://github.com/apache/pdfbox}} do not consistently extract accessibility criteria from sample PDFs, even when the criteria are met. We also prefer Adobe's checker to PDFA Inspector, used by \citet{Brady2015CreatingAP}, because PDFA Inspector only analyzes three criteria, whereas we are interested in other accessibility attributes as well, like the presence of alt-text.

For each PDF, the Adobe accessibility checker generates a report that includes whether or not the PDF passes or fails tests for certain accessibility features, such as the inclusion of figure alt-text or properly tagged headings for navigation. Because there is no API or standalone application for the Adobe accessibility checker, it can only be accessed through the user interface of a licensed version of Adobe Acrobat Pro. We develop an AppleScript program that enables us to automatically process papers through the Adobe checker. Our program requires a dedicated computer running MacOS and a licensed version of Adobe Acrobat Pro. It takes 10 seconds on average to download and process each PDF, which enables us to scale up our analysis to tens of thousands of papers. Accessibility reports from the checker are saved in HTML format for subsequent analysis.

Each report contains a total of 32 accessibility criteria, marked as ``Passed,'' ``Failed,'' or ``Needs manual check.''\footnote{Please see \href{https://helpx.adobe.com/acrobat/using/create-verify-pdf-accessibility.html}{https://helpx.adobe.com/acrobat/using/create-verify-pdf-accessibility.html} for a description of the accessibility report.}
Following \citet{Lazar2017MakingTF}, we analyze the following five criteria\footnote{For papers containing no tables and/or no figures, the Adobe checker can still return both pass or fail for the Table header and Alt-text criteria respectively. When objects in the PDF are \textit{not} tagged, the checker will fail these criteria even when the paper has no tables and/or no figures. When objects in the PDF \textit{are} tagged and the PDF is accessible, the checker will pass these criteria even when the paper has no tables or no figures.}:

\begin{itemize}
    \item Alt-text: Figures have alternate text.
    \item Table headers: Tables have headers.
    \item Tagged PDF: The document is tagged to specify the correct reading order.
    \item Default language: The document has a specified reading language.
    \item Tab order: The document is tagged with correct reading order, used for navigation with the \texttt{tab} key.
\end{itemize}

\noindent 
% We provide the full dataset with all 32 criteria at \githublink. 
For our analysis, we also report \textit{Total Compliance}, which refers to the sum number of accessibility criteria met (e.g. if a paper has met 3 out of the 5 criteria we specify, then Total Compliance is 3). In some cases, we report the \textit{Normalized Total Compliance}, which is computed as the Total Compliance divided by 5, and can be interpreted as the proportion of the 5 criteria which are satisfied. We also report \textit{\xcompliance{5}}, a binary value of whether a paper has met all 5 criteria we specify (1 if all 5 criteria are met, 0 if any are not met), and the rate of \xcompliance{5} for papers in our dataset.

In addition to running the accessibility checker, we also extract metadata for each PDF, focusing on metadata related to the PDF creation process. PDF metadata are generated by the software used to create each file, and we analyze the associations between different PDF creation software and the accessibility of the resulting PDF document. Our hypothesis is that some classes of software (such as Microsoft Word) produce more accessible PDFs.

\subsection{Accuracy of our automated accessibility checker}
\label{sec:sos_chi}

\begin{table}[t!]
\begin{tabular}{lccc}
    \toprule
    \textbf{Criterion} & \textbf{CHI 2010\citep{Lazar2017MakingTF}} & \textbf{Ours-CHI 2010} & \textbf{Ours-All (\numpdfs)} \\ 
    \midrule
    Alt-text & 3.6\% & 4.0\% & 7.5\%  \\
    Table headers & 0.7\% & 1.0\% & 13.3\% \\
    Tagged PDF & 6.3\% & 7.4\% & 13.4\% \\
    Default language & 2.3\% & 3.0\% & 17.2\% \\
    Tab order & 0.3\% & 1.0\% & 9.3\% \\
    \midrule
    \xcompliance{5} & - & - & 2.4\% \\
    \bottomrule
\end{tabular}
\caption{We reproduce the analysis conducted by \citet{Lazar2017MakingTF} on PDFs of papers published in CHI, showing the percentage of papers that satisfy each of the five accessibility criteria. We find similar compliance rates, indicating that our automated accessibility checker pipeline is comparable to previous analysis methods. We also show the percentage of papers in our full dataset of \numpdfs PDFs that satisfy each criterion, along with the percent that satisfy \xcompliance{5}.
}
\label{tab:chi-results}
% \Description{
% Criterion; CHI 2010 [23]; Ours-CHI 2010; Ours-All (11,397)
% Alt-text; 3.6%; 4.0%; 7.5%
% Table headers; 0.7%; 1.0%; 13.3%
% Tagged PDF; 6.3%; 7.4%; 13.4%
% Default language; 2.3%; 3.0%; 17.2%
% Tab order; 0.3%; 1.0%; 9.3%
% Adobe-5 Compliance; -; -; 2.4
% }
\end{table}

Previous work employed different versions of the Adobe Accessibility Checker to generate paper accessibility reports. To confirm the accuracy of our checker, as well as the automated script we create to perform the analysis, we run our checker on CHI 2010 papers to reproduce the results of \citet{Lazar2017MakingTF}. We identify CHI papers using DOIs reported by the ACM, and resolve these to PDFs in the Semantic Scholar corpus \citep{Ammar2018ConstructionOT}. We identify \numchi CHI papers in the corpus, and generate accessibility reports for these using our automated checker.

Our results shows similar rates of compliance compared to what was measured by \citet{Lazar2017MakingTF} (see Table~\ref{tab:chi-results} for results). This indicates that our automated accessibility checker produces comparable results to previous studies.

\subsection{Proportion of papers with accessible PDFs}
\label{sec:sos_fos}

\begin{figure}[tb!]
  \centering
    \includegraphics[width=0.55\linewidth]{figures/total_compliance_all.png}
  \caption{The distribution of numbers of PDFs in our dataset that meet our defined accessibility compliance criteria. A large majority (8519) of PDFs in our sample meet 0 out of 5 accessibility criteria. Of those meeting 1 criterion (Total Compliance = 1), the most commonly met criterion is Default Language (793 of 1010, 78.5\%). Of those meeting 4 criteria (Total Compliance = 4), the most common missing criterion is Alt-text (396 of 494, 80.2\%).
  } 
  \label{fig:fos-total-compliance}
  \Description{A histogram showing the distribution of total compliance score for our dataset. The majority of PDFs (8519 of 11397) in our sample meet 0 compliance criteria. Small numbers of PDFs meet some criteria, with lower numbers meeting more criteria.}
\end{figure}

\begin{figure}[t!]
  \centering
    \includegraphics[width=0.9\linewidth]{figures/complete_compliance_by_fos.png}
  \caption{Percent of papers per field of study that meet all 5 accessibility criteria defined in \xcompliance{5}. Philosophy, Art, and Psychology have the highest rates of \xcompliance{5} satisfaction while Biology, Mathematics, and Geology have the lowest rates. 
  None of the fields had more than $6.5\%$ of PDFs satisfying \xcompliance{5}. 
  }
  \label{fig:fos-complete-compliance}
  \Description{A bar plot showing the proportion of PDFs in each field of study that satisfy Adobe-5 Compliance (meets all five accessibility criteria we define). Compliance percentage ranges from 6.3\% at the high end to 0.2\% at the low end. At the high end are fields such as philosophy, art, business, and psychology. At the low end are fields like biology, mathematics, and geology.}
\end{figure}

Around 1.6\% of PDFs we attempted to process failed in the Adobe checker (i.e., we could not generate an accessibility report). The accessibility checker most commonly fails because the PDF file is password protected, or the PDF file is corrupt. In both of these cases, the PDF is inaccessible to the user. We exclude these PDFs from subsequent analysis.

Accessibility compliance over all papers is low. Table~\ref{tab:chi-results} shows the percent of papers meeting each of the five criteria, as well as the Adobe-5 Compliance rate associated with this sample of papers. Figure~\ref{fig:fos-total-compliance} shows that the vast majority of papers do not meet any of the five accessibility criteria (8519 papers, 74.7\% do not meet any criteria) and very few (275 papers, 2.4\%) meet all five. Of those PDFs meeting 1 criterion, the most commonly met criterion is Default Language (793 of 1010, 78.5\%). Of those PDFs meeting 4 criteria, the most common \textit{missing} criterion is Alt-text (396 of 494, 80.2\%). In fact, only 854 PDFs (7.5\%) in the whole dataset have alt-text for figures. This is intuitive as Alt-text is the only criterion that \textit{always} requires author input to achieve, while the other four criteria can be derived from the document or automatically inferred, depending on the software used to generate the PDF.
    
As shown in Figure~\ref{fig:fos-complete-compliance}, all fields have an \xcompliance{5} of less than 7\%. The fields with the highest rates of compliance are Philosophy (6.3\%), Art (6.2\%), Business (5.7\%), Psychology (5.7\%), and History (5.3\%) while the fields with the lowest rates of compliance are Geology (0.2\%), Mathematics (0.3\%), and Biology (0.6\%). Fields associated with higher compliance tend to be closer to the humanities, and those with lower levels of compliance tend to be science and engineering fields. The prevalence of different document editing and typesetting software by field of study may explain some of these differences, and we explore these associations in Section~\ref{sec:sos_pdf_headers}.

\subsection{Trends in paper accessibility over time}

\begin{figure}[t!]
  \centering
    \includegraphics[width=0.6\linewidth]{figures/compliance_over_time.png}
  \caption{Accessibility compliance over time (2010-2019). The rate of \xcompliance{5} has remained relatively stable over the last decade, at around 2--3\%. Compliance along several criteria have improved over time, though the largest improvements have been in Default Language, the simplistic criteria to meet. Modest improvements are seen for Table headers, Tagged PDFs, and Tab order. The presence of alt-text has remained stable and lower, around 5--10\%. 
  }
  \label{fig:fos-over-time}
  \Description{A line plot shows changes in compliance rates over time. Adobe-5 Compliance has stayed consistently around 0.02-0.03 since 2010. The proportion of PDFs satisfying the Default language criteria has increased the most over time, from 0.10 in 2010 to 0.27 in 2019. Tagged PDF, Tab order, and Table headers also show improvements. The proportion of PDFs with alt-text has remained fairly consistent over time, between 0.05 and 0.1.}
\end{figure}

We show changes in compliance for all fields of study over time in Figure~\ref{fig:fos-over-time}. With the exception of Default Language, all accessibility criteria demonstrate slowly increasing or stable compliance rates over the past decade, with increases seen in Tagged PDFs and Tab order over time. Default language compliance is increasing most rapidly, from around 10\% compliance in 2010 to more than 25\% in 2019. This may be due to changes in PDF generation defaults in various typesetting software. Though this improvement is good, Default Language is the easiest of the five criteria to bring into compliance, and arguably the least valuable in terms of improving the accessible reading experience. The criterion with the lowest rate of compliance is Alt-text, which has remained stable between 5--10\% and has been lower in recent years. Since Alt-text is the only criterion of the five which always necessitates author intervention, we believe this is a sign that authors have not become more attuned to accessibility needs, and that at least some of the improvements we see over time can be attributed to typesetting software or publisher-level changes. 


\begin{table}[t!]
\begin{tabular}{ll}
    \toprule
        \textbf{Typesetting Software} & \textbf{Count (\%)} \\ 
    \midrule
        Adobe InDesign       & 1591 (14.0\%) \\
        LaTeX                & 1431 (12.6\%) \\
        Arbortext APP        & 1374 (12.1\%) \\
        Microsoft Word       & 1318 (11.6\%) \\
        Printer              & 1021 (9.0\%) \\
    \midrule
        Other                & 4662 (40.9\%) \\
    \bottomrule
\end{tabular}
\caption{Count of papers per Typesetting Software. ``Other'' includes PDFs created with an additional 24 unique software programs, each with counts of less than 350, as well as those created with an unknown typesetting software.}
\label{tab:dist-of-headers}
% \Description{
% Typesetting Software; Count (%)
% Adobe InDesign; 1591 (14.0%)
% LaTeX; 1431 (12.6%)
% Arbortext APP; 1374 (12.1%)
% Microsoft Word; 1318 (11.6%)
% Printer; 1021 (9.0%)
% Other; 4662 (40.9%)
% }
\end{table}

\subsection{Association between typesetting software and paper accessibility}
\label{sec:sos_pdf_headers}

Typesetting software is extracted from PDF metadata and manually canonicalized. We extract values for three metadata fields: \texttt{xmp:CreatorTool}, \texttt{pdf:docinfo:creator\_tool}, and \texttt{producer}. All unique PDF creation tools associated with more than 20 PDFs in our dataset are reviewed and mapped to a canonical typesetting software. For example, the values (\texttt{latex}, \texttt{pdftex}, \texttt{tex live}, \texttt{tex}, \texttt{vtex pdf}, \texttt{xetex}) are mapped to the LaTeX cluster, while the values (\texttt{microsoft}, \texttt{for word}, \texttt{word}) and other variants are mapped to the Microsoft Word cluster. We realize that not all Microsoft Word versions, LaTeX distributions, or other versions of typesetting software within a cluster are equal, but this normalization allows us to generalize over these software clusters. For analysis, we compare the five most commonly observed typesetting software clusters in our dataset, grouping all others into a cluster called \texttt{Other}.

\begin{figure}[t!]
  \centering
    \includegraphics[width=0.5\linewidth]{figures/total_compliance_by_typesetting_software_categories.png}
  \caption{Histograms showing the distribution of Total Compliance scores for each of the top 5 typesetting software, ordered by decreasing mean Total Compliance. Microsoft Word stands out as producing PDFs with significantly higher Total Compliance than other typesetting software. Three of the top five PDF typesetting software clusters, Arbortext APP, Printer, and LaTeX, produce PDFs with low Total Compliance, with the majority of PDFs at 0 compliance. 
  }
  \Description{Five histograms show the distribution of Total Compliance scores for the five most common typesetting software clusters. These are sorted from most compliant to least compliant, in order: Microsoft Word, Adobe InDesign, Arbortext APP, Printer, and LaTeX. Microsoft Word produces many PDFs that satisfy 2 or more criteria, with a peak at Total Compliance = 4. Most PDFs produced by Adobe InDesign satisfy no accessibility criteria, but many satisfy 1 or 2. Arbortext APP, Printer, and LaTeX all produce inaccessible PDFs, with the vast majority of PDFs produced by these software satisfying no accessibility criteria.}
  \label{fig:fos-total-compliance-headers}
\end{figure}

We report the distribution of typesetting software in Table~\ref{tab:dist-of-headers}. The most popular PDF creators are Adobe InDesign, LaTeX, Arbortext APP, Microsoft Word, and Printer. ``Printer'' refers to PDFs generated by a printer driver (by selecting ``Print'' $\rightarrow$ ``Save as PDF'' in most operating systems); unfortunately, creating a PDF through printing provides no indicator of what software was used to typeset the document, and is generally associated with very low accessibility compliance. The ``Other'' category aggregates papers created by all other clusters of typesetting software; each of these clusters is associated with less than 350 PDFs, i.e., the falloff is steep after the Printer cluster. For the following analysis, we present a comparison between the five most common PDF creator clusters.

Figure~\ref{fig:fos-total-compliance-headers} shows histograms of the Total Compliance score for PDFs in the five most common typesetting software clusters. While the vast majority of papers do not meet any accessibility criteria, it is clear that Microsoft Word produces the most accessible PDFs, followed by Adobe InDesign. 
To determine the significance of this difference, we compute the ANOVA and Kruskal-Wallace \citep{Kruskal1952UseOR} statistics with the PDF typesetting software clusters as the sample groups and the Total Compliance as the measurements for the groups. We compute an ANOVA statistic of 2587.1 ($p$ < 0.001) and a Kruskal-Wallace $H$ statistic of 4422.0 ($p$ < 0.001). This indicates a significant difference in the distribution of Total Compliance scores between the five most common PDF typesetting software.

\begin{figure}[t!]
  \centering
    \includegraphics[width=0.64\linewidth]{figures/prop_word_vs_compliance_by_fos.png}
  \caption{There is a strong correlation ($r = 0.89$, $p < 0.001$, 95\% CI shown) between the proportion of PDFs typeset using Microsoft Word and the mean normalized Total Compliance of papers by field of study. Fields such as Business, Philosophy, Sociology, Materials science, and Psychology use Microsoft Word around or over 20\% of the time, and have correspondingly higher mean accessibility compliance. On the other end of the spectrum are fields like Mathematics, Physics, and Medicine, where Microsoft Word is rarely used, and which have very low levels of mean compliance.}
  \label{fig:prop_word_by_fos}
  \Description{A scatter plot shows a positive correlation between the proportion of PDFs typeset using Microsoft Word and the mean normalized total compliance score by field of study. Fields that typeset more using Word have higher Total Compliance scores. The correlation coefficient r is 0.89, and p is less than 0.001. Fields that use Word more and have higher compliance rates include Business, Materials science, Geography, Philosophy, and Sociology. Fields that use Word very little and have low compliance rates include Mathematics, Physics, and Medicine.}
\end{figure}

In Figure~\ref{fig:prop_word_by_fos}, we observe again that usage of Microsoft Word is highly correlated with accessibility compliance. Here, we plot the proportion of Microsoft Word usage per field of study and the corresponding mean normalized Total Compliance rates for those fields. Higher rates of Microsoft Word usage are statistically correlated with higher mean normalized Total Compliance ($r = 0.89$, $p < 0.001$). 

\begin{figure}[t!]
  \centering
    \includegraphics[width=0.6\linewidth]{figures/typesetting_software_over_time.png}
  \caption{The proportion of PDFs typeset by the five most common typesetting software over time. Software such as Adobe InDesign, LaTeX, and Microsoft Word are increasing in popularity over time.}
  \label{fig:software_over_time}
  \Description{A line plot shows the proportion of different typesetting software in our sample between 2010-2019. Microsoft Word, Adobe InDesign, and LaTeX have all increased in proportional usage (all starting at a proportion of 0.07-0.08 in 2010 to 0.15 or above in 2019. The usage of printer drivers to create PDFs has declined from a proportion of 0.12 in 2010 to around 0.05 in 2019. Arbortext APP also shows a modest decline in recent years to below a proportion 0.10.}
\end{figure}

In Figure~\ref{fig:software_over_time}, we show the proportion of usage of each of the five typesetting software over time. In recent years, Adobe InDesign, LaTeX, and Microsoft Word usage are proportionally increasing, while the proportion of Printer-created PDFs is declining. The increase in Adobe InDesign and Microsoft Word have likely driven the increase in rates of Total Compliance over time, since these typesetting software are the most associated with higher accessibility compliance. 


\subsection{Summary of analyses}
\label{sec:sos_summary}

Overall, accessibility compliance over the past decade and across all fields of study have slowly improved. Full compliance based on \xcompliance{5}, however, has remained around 2.4\% on average and does not show trends towards improving. Improvements in several compliance criteria are observed, with Default Language being the most improved, nearing 30\% coverage in 2019. However, Default Language is the easiest criteria to meet, and arguably produces the least amount of accessibility improvement in user experience. Criteria such as Tagged PDFs, Tab order, and Headers show modest improvements over time, though only between 10--15\% of papers in our sample meet any one of these individual criteria. Alt-text compliance is the lowest of our measured criteria, and as the only criterion of the five requiring author intervention in all cases, the lack of alt-text may be indicative of the general lack of author awareness and contribution to accessibility efforts for scientific papers. 

Based on our analysis, typesetting software plays a large role in document accessibility. Of the most common PDF creator software, Microsoft Word appears to produce the most accessibility-compliant PDFs, while LaTeX produces PDFs with the lowest compliance. Microsoft has recently made investments in the accessibility of their Office 365 Suite.\footnote{\href{https://www.microsoft.com/en-us/accessibility/microsoft-365}{https://www.microsoft.com/en-us/accessibility/microsoft-365}} It is clear that software can help increase accessibility compliance by prioritizing accessibility concerns during document creation, and we encourage other developers of typesetting and publishing software to priotize accessibility concerns in their development process. 

Improvements in accessibility compliance have stalled over the past decade, likely because accessibility concerns are considered marginal, and are outside of the awareness of most publishing authors and researchers. Significant changes in the authorial and publication processes are needed to change this status quo, and to increase the accessibility of scientific papers for BLV users going forward. Though we believe and encourage change in the academic paper authorial and publication process in relation to accessibility, the likelihood of rapid improvement is low and these changes will not impact the many millions of academic PDFs that have already been published. Therefore, we introduce a technological solution that may mitigate some of the accessibility challenges of existing paper PDFs, and aim to understand how this solution and others like it could serve the immediate needs of the BLV research community.


\section{Converting PDF to HTML: The \scially Pipeline}
\label{sec:pdf2html}

To address the broad accessibility challenges described in Section \ref{sec:sos}, we propose and prototype a system for extracting semantic content from paper PDFs and re-rendering this content as accessible HTML. HTML is widely accepted as a more accessible document format than PDFs. In the 2019 Access SIGCHI Report, the authors discuss the reasoning behind switching CHI publications to a new HTML5 proceedings format to improve accessibility \citep{Mankoff2019SIGCHI}. 
%\jonathan{CHI 2019 adopted a new HTML 5 proceedings format according to \url{https://dl.acm.org/doi/abs/10.1145/3386280.3386287}}. 
By rendering the content of paper PDFs as HTML, and introducing proper reading order and accessibility features such as section headings, links, and figure tags, we can offset many of the issues of reading from an inaccessible PDF. Our PDF to HTML rendering system is named \scially after the community-adopted numeronym for digital accessibility.\footnote{\href{https://www.a11yproject.com/}{https://www.a11yproject.com/}} 

Figure~\ref{fig:pipeline} provides a schematic for the approach. \scially leverages the two open source PDF processing projects S2ORC \citep{lo-wang-2020-s2orc} and DeepFigures \citep{Siegel2018ExtractingSF}, the Semantic Scholar API,\footnote{\href{https://api.semanticscholar.org/}{https://api.semanticscholar.org/}} and a custom Flask application for rendering the extracted content of the PDF as HTML. The S2ORC project \citep{lo-wang-2020-s2orc} integrates the Grobid machine learning library \citep{Lopez2015GROBIDI} and a custom XML to JSON parser\footnote{Available at \href{https://github.com/allenai/s2orc-doc2json}{https://github.com/allenai/s2orc-doc2json}} to produce a structured representation of paper text. We use a version of the S2ORC pipeline that is based on Grobid v0.6.0. The resulting JSON representation includes metadata fields like title, authors, and affiliations, and paper content fields such as abstract, section headers, body text organized into paragraphs, bibliography entries, and figure and table objects (though not the figure images themselves). The output also contains links between inline citations and figure/table references respectively to bibliography entries and figure/table objects. DeepFigures \citep{Siegel2018ExtractingSF}, on the other hand, leverages a computer vision model to extract images of figures and tables as well as their corresponding captions from the source PDF. 

The outputs of S2ORC and DeepFigures are stitched together to form the HTML render as in Figure~\ref{fig:pipeline}. We place header tags (\texttt{<h1>...</h1>}, \texttt{<h2>...</h2>}) around the title, authors, abstract, section headings, and reference heading. Paragraphs of body text are enclosed in \texttt{<p>...</p>} tags in order within their appropriate sections. Bibliography entries are provided in an unordered list under the reference heading. Figures and tables are enclosed in \texttt{<figure>...} \texttt{</figure>} tags and placement is inferred based on mentions in the text. A figure or table is placed immediately after the paragraph in which its handle is first mentioned (e.g. ``In \underline{Fig. 1}, we show...'' is the first mention of Figure 1 and the figure is placed directly after the paragraph with this mention). Figure and table captions are attached to their corresponding image objects, so that correspondences between the caption text and image are made explicit (in PDFs, this is usually not the case). Any figures or tables which are not mentioned in order in the text are placed in order nonetheless; in other words, if paragraph 1 mentions Figure 1 and paragraph 2 mentions Figure 3, both Figure 1 and 2 will be placed directly following paragraph 1 and Figure 3 following paragraph 2. This ensures that the layout for the HTML render closely approximates the intended reading order. We justify this decision based on user feedback from our pilot study, which is discussed in Section~\ref{sec:user_study}. 

In some cases, we are able to successfully process a PDF through S2ORC to extract textual content but DeepFigures either fails to process the PDF or fails to extract some or all figures from the PDF. To mitigate the cognitive dissonance around figure or table mentions without corresponding figure or table objects, we insert placeholder objects into the HTML render as in Figure~\ref{fig:figure_equations}. For example, if ``Figure 2'' is mentioned in the text but is not successfully extracted by DeepFigures, we would insert a placeholder image for the figure based on the logic described in the previous paragraph along with the text ``Figure 2. Not extracted; please refer to original document.'' Similarly, mathematical equations that we cannot currently extract are acknowledged with the same placeholder text.

\begin{figure}
    \centering
    \includegraphics[width=0.4\textwidth]{figures/fig_success_1882932.png}\includegraphics[width=0.4\textwidth]{figures/placeholder.png}
    \includegraphics[width=0.6\textwidth]{figures/eq_fail.png}
    \caption{A successfully extracted figure from \citet{Nascimento2005VertexCA} is shown with its corresponding figure caption (\textit{top left}). When figures are not extracted and inferred to exist (handle mentioned in text or number between two extracted figures), a placeholder image is shown along with a message referencing the failed extraction (\textit{top right}). Similarly, when an equation is detected to be present in the PDF and not extracted, we insert text signaling the failed extraction and refer the user to the source document (\textit{bottom}).}
    \label{fig:figure_equations}
    \Description{Three subfigures show figure and equation related features in SciA11y. First subfigure shows extracted images from Nascimento and Bioucas-Dias and the associated caption of the figure, as displayed in an HTML <figure> block. Second subfigure shows a placeholder image for a figure that was not successfully extracted by the system. Third subfigure shows the text used as a placeholder for equations that are not extracted. Both the not extracted figure and not extracted equation are followed by the text: "Not extracted; please refer to original document."}
\end{figure}

We add links between inline citations and the corresponding reference entry where possible. We insert links at each inline citation in the body text that link to the corresponding bibliography entry. Following each bibliography entry, we provide links back to the first mention of that entry in each section of the paper in which it was mentioned. For example, if bibliography entry \texttt{[1]} is cited in the ``II. Related Works'' section and the ``III. Methods'' section, we provide two links following the entry in the bibliography to the corresponding citation locations in sections II and III, as in:

\begin{alltt}
    [1] Last name et al. Paper title. Venue. DOI.
        \underline{Link to return to Section II}, \underline{Link to return to Section III}
\end{alltt}

\noindent This allows users to navigate back to their reading location in the document after clicking through to a bibliography entry. A user may otherwise hesitate to resolve a link, because it may result in losing their place and train of thought. Finally, we introduce a table of contents near the beginning of the HTML render to facilitate better understanding of overall document structure. The table of contents includes all section titles, linked to the corresponding sections, as well as figures and tables nested under their respective section headers. The table of contents provides a rapid overview of the structure of the document, and facilitates rapid navigation to the reader's desired sections.

In the current iteration of the HTML render, we do not display author affiliations, footnotes, or mathematical equations due to the difficulty of extracting these pieces of information from the PDF. Though some of the elements are extracted in S2ORC, the overall quality of the extractions for these elements is lower, and is currently insufficient for surfacing in the prototype (see Section~\ref{sec:evaluation} for details). Future work includes investigating the possibility of extracting and exposing these elements, either by improving current models or training new models targeted towards the extraction of specific paper elements.

We leverage the feedback we received during our pilot studies (see Section~\ref{sec:user_study}) to make improvements prior to the main user study. We denote the versions of the prototype as v0.1 (initial version; version seen by P1), v0.2 (version seen by P2), and v0.3 (version seen by all other participants in the main user study). Features implemented in v0.1 include the primary components of the HTML render such as title, authors, abstract, body text with section headers, figures and tables, references, and links between inline citations and references. In v0.1, figures and tables were placed in a separate section following the main body of the paper. Following P1, for version v0.2, we implemented the table of contents, inserted placeholders for objects that we could not extract, and began inserting figures and tables into the body text adjacent to their first mentions. This last change was made in response to P1's feedback that navigating away to figures caused him to lose his reading location. Following P2, for version v0.3, we implemented only minor changes. P2 signaled during his session that URLs in the bibliography were not being correctly extracted, so we patched the data to correctly extract and display URLs in bibliography entries.

Based on our evaluation of the quality of these HTML renders (Section~\ref{sec:evaluation}) and user feedback and response (Section~\ref{sec:user_study}), we believe our approach can dramatically increase the screen reader navigability and accessibility of scientific papers across all disciplines by providing an alternate and more accessible HTML version of these papers. Properly tagged section headings allow for quick navigation and skimming of a paper, links between inline citations and bibliography entries allow users to browse to cited papers without losing their place, and figure tags for figure and table objects allow for direct navigation to these in-paper objects. We now discuss the quality of our PDF extractions (Section~\ref{sec:evaluation}) and user response to the prototype (Section~\ref{sec:user_study}) in detail.
\section{HTML render quality evaluation}
\label{sec:evaluation}

Extracting semantic content from PDF is an imperfect process. Though re-rendering a PDF as HTML can increase a document's accessibility, the process relies on machine learning models that can make mistakes when extracting information. As we glean from user studies, BLV users may have some tolerance for error, but there is an inherent trade-off between errors and perceived trust in the system. We conduct a study to estimate the (1) faithfulness of the HTML renders to the source PDFs, and (2) overall readability of the resulting HTML renders. We define \textit{faithfulness} as how accurately the HTML render represents different facets of the PDF document, such as displaying the correct title, section headers, and figure captions. These facets are measured as the number of errors that are made in rendering, e.g., mistakenly parsing one figure caption into the body text is counted as one error towards that facet. \textit{Readability}, on the other hand, is an ordinal variable meant to capture the overall usability of the parse. Documents are given one of three grades, those with no major problems, some problems, and many problems impacting readability.

To evaluate readability and faithfulness, we first perform open coding on a small sample of document PDFs and corresponding \scially HTML renders. The purpose of this exercise is to identify facets of extraction that impact the ability to read a paper. A rubric is then designed based on these identified facets. The process taken to design the evaluation rubric, the rubric's content, and annotation instructions are detailed in Section~\ref{sec:eval_rubric}. We then annotate a sample of \numeval papers across different fields of study using this rubric. For each category of errors identified during open coding, we compute the overall error rates seen in our sample. We also present the overall assessed readability, reported in aggregate over our sample and by fields of study. Results of this evaluation are presented in Section~\ref{sec:eval_res}.

\subsection{Open coding of document facets}
\label{sec:eval_coding}

One author performed open coding on a sample of papers, comparing the PDF and \scially HTML renders to identify inconsistencies and facets that impact the faithfulness of document representation. Papers are sampled from the Semantic Scholar API\footnote{\href{https://api.semanticscholar.org/}{https://api.semanticscholar.org/}} using various search terms, and selecting the top 3 results for each search term for which a PDF and S2ORC parse are available. Search terms were selected to achieve coverage over different domains, and the top papers are sampled to select for relevant publications. The author stopped sampling papers upon reaching saturation, resulting in 8 search terms and 24 papers. The search terms used were: \texttt{human computer interaction}, \texttt{epilepsy}, \texttt{quasars}, \texttt{language model}, \texttt{influenza epidemiology}, \texttt{anabolic steroids}, \texttt{social networks}, and \texttt{arctic snow cover}.

For each paper, the author evaluated the PDF and HTML render side-by-side, scanning through the document to identify points of difference between the two document representations. Specifically, the author looked for any text in the PDF that is not shown in the HTML, any text from the PDF that is mixed into the main text of the HTML (e.g. figure captions, headers, or footnotes that should be separate from the main text but are mixed in, interrupting the reading flow), and other parsing mistakes (e.g. errors with math, missing lists and tables etc). These observations are detailed qualitatively, and each facet is assessed for its faithfulness to the original PDF document as well as its overall impact on readability.

\subsection{Evaluation rubric}
\label{sec:eval_rubric}

Observations from open coding are coalesced into an evaluation rubric and form for grading the quality and faithfulness of the HTML render.
The evaluation form attempts to capture errors in PDF extraction that affect each of the primary semantic categories identified for proper reading. These semantic categories and common extraction errors are given in Table~\ref{tab:eval_cats}.

Questions in the form are designed to capture each type of faithfulness error, while allowing annotators to qualify their responses. We also include a question to capture the overall readability of the HTML render. Instructions for completing the annotation form are provided in Appendix~\ref{app:eval_instructions}; the final version of the form is replicated in Appendix~\ref{app:eval_instructions}; and the rubric for overall readability evaluation is given in Appendix~\ref{app:quality_rubric}. 

Three authors iterated twice on the content of the evaluation form, until they came to a consensus that all evaluation categories were adequately addressed using a minimum set of questions. Two authors then participated in pilot annotations, where each person independently annotated the same set of five papers sampled from the set labeled by the third author during open coding. Answers to all numeric questions were within $\pm 1$ for these five papers when comparing the two authors' annotations. All three authors discussed discrepancies in overall readability score, iterating on the rubric defined in Appendix~\ref{app:quality_rubric} and coming to a consensus. The finalized form and rubric are used for evaluation.

\begin{table}[t!]
    \small
    \centering
    \begin{tabularx}{\linewidth}{L{20mm}L{65mm}X}
        \toprule
        \textbf{Category} & \textbf{Description} & \textbf{Common errors} \\
        \midrule 
        \textsc{title} & The title and subtitle of the paper & Missing words \newline Extra words \\ 
        \midrule
        \textsc{authors} & A list of authors who wrote the paper; this includes affiliation, though we do not explicitly evaluate affiliation in this study & Missing authors \newline Extra authors \newline Misspellings \\
        \midrule
        \textsc{abstract} & The abstract of the paper & Some text not extracted \newline Other text incorrectly extracted as abstract \\
        \midrule
        \textsc{section headings} & The text of section headings & Some headings not extracted (part of body text) \newline Other text incorrectly extracted as headings \\
        \midrule
        \textsc{body text} & The main text of the paper, organized by paragraph under each section heading & Some paragraphs not extracted (missing) \newline Some text not extracted \newline Other text incorrectly extracted as body text \\
        \midrule
        \textsc{figures} & Images, captions, and alt-text of each figure & Figure not extracted \newline Caption text not extracted (part of body text) \newline Other text incorrectly extracted as caption text \\
        \midrule
        \textsc{tables} & Caption/title and content of each table & Table not extracted (not part of body text) \newline Table not extracted (part of body text) \newline Caption text not extracted (part of body text) \newline Other text incorrectly extracted as caption text  \\
        \midrule
        \textsc{equations} & Mathematical formulas, represented in TeX or Math ML; note: our current pipeline does not extract math & Some equations not extracted \newline Some equations incorrectly extracted \\
        \midrule
        \textsc{bibliography} & Bibliography entries in the reference section & Some bibliography entries not extracted \newline Some bibliography entries incorrectly extracted \newline Other text incorrectly extracted as bibliography \\ 
        \midrule
        \textsc{inline citations} & Inline citations from the body text to papers in the bibliography section & Some inline citations not detected \newline Some inline citations incorrectly linked \\
        \midrule
        \textsc{headers, footers \& footnotes} & Page headers and footers, footnotes, endnotes, and other text that is not a part of the main body of the document & Some headers and footers incorrectly extracted into body text \\
        \bottomrule
    \end{tabularx}
    \caption{Categories of paper objects identified for evaluation along with the common errors seen for each category.}
    \label{tab:eval_cats}
%     \Description{
% Category; Description; Common errors 
% title; The title and subtitle of the paper; Missing words, Extra words  
% authors; A list of authors who wrote the paper; this includes affiliation, though we do not explicitly evaluate affiliation in this study; Missing authors, Extra authors, Misspellings 
% abstract; The abstract of the paper; Some text not extracted, Other text incorrectly extracted as abstract 
% section headings; The text of section headings; Some headings not extracted (part of body text), Other text incorrectly extracted as headings 
% body text; The main text of the paper, organized by paragraph under each section heading; Some paragraphs not extracted (missing), Some text not extracted, Other text incorrectly extracted as body text 
% figures; Images, captions, and alt-text of each figure; Figure not extracted, Caption text not extracted (part of body text), Other text incorrectly extracted as caption text 
% tables; Caption/title and content of each table; Table not extracted (not part of body text), Table not extracted (part of body text), Caption text not extracted (part of body text), Other text incorrectly extracted as caption text  
% equations; Mathematical formulas, represented in TeX or Math ML; note: our current pipeline does not extract math; Some equations not extracted, Some equations incorrectly extracted 
% bibliography; Bibliography entries in the reference section; Some bibliography entries not extracted, Some bibliography entries incorrectly extracted, Other text incorrectly extracted as bibliography  
% inline citations; Inline citations from the body text to papers in the bibliography section; Some inline citations not detected, Some inline citations incorrectly linked 
% headers, footers & footnotes; Page headers and footers, footnotes, endnotes, and other text that is not a part of the main body of the document; Some headers and footers incorrectly extracted into body text 
%     }
\end{table}

Of the categories and errors described in Table~\ref{tab:eval_cats}, our current pipeline does not extract table content and equations. Tables are extracted as images by DeepFigures \citep{Siegel2018ExtractingSF}, which do not contain table semantic information. Regarding equations, we distinguish between inline equations (math written in the body text) and display equations (independent line items that can usually be referenced by number); for this work, we evaluated a small sample of papers for successful extraction of display equations. Though some display equations are recognized, the quality of equation extraction is low, usually resulting in missing tokens or improper math formatting. Therefore, we decided to replace display equations in the prototype with the equation placeholder shown in Figure~\ref{fig:figure_equations}. Since problems with mathematical formulae are among those most mentioned by users in our study, equation extraction is among our most urgent future goals, and we discuss some options in Section~\ref{sec:future_work}.

\subsection{Evaluation results}
\label{sec:eval_res}

We start with the dataset of \numpdfs papers we analyze in Section~\ref{sec:sos},
and subsample 535 documents stratified by field of study. Two expert annotators with undergraduate science training code papers from this sample, with an aim of annotating around 20 papers per field of study. Though we achieve the target number for most fields, we missed this target for some fields closer to the humanities because more of these documents are difficult to manually annotate within our time and resource constraints.
For example, documents are deemed unsuitable for annotation if they are not papers (i.e., they are books, posters, abstracts, etc), if they are too long, or if they are not in English. In these cases, the annotators can skip the document. Detailed guidance on suitability is provided in the annotation instructions (see Appendix~\ref{app:eval_instructions}). 

The two annotators annotated \numeval unique papers and skipped \numevalskipped. The resulting annotated sample consists of papers from 195 unique publication venues. Each paper takes 5--10 minutes to grade. Documents are skipped primarily due to language (paper not in English), length, or the document is not a paper. Inter-annotator agreement is computed over a sample of 20 papers over each of the evaluated facets. We report Cohen's Kappa for categorical questions such as those on the extraction of title, authors, abstract, and bibliography. For numerical questions such as counting the occurrence of extraction errors related to figures, tables, section headings, and body paragraphs etc, we report the intraclass correlation coefficient (ICC) as well as the average difference of values between the two annotators. See Table~\ref{tab:eval_iaa} for these results.
Agreement was high for most element-level annotator questions. Annotators had the highest levels of disagreement on the evaluation of header/footer/footnote errors, section heading errors, and body paragraph errors, likely due to these being text-based and the most numerous; though the average differences reported between annotators on these questions are only between 1-2. Likewise, agreement on overall readability score is modest, at 0.55; we note, however, that neither annotator labeled any paper as having no major readability problems when the other annotator labeled it as having lots of readability problems.

\begin{table}[tb!]
    \centering
    \begin{tabularx}{0.94\linewidth}{L{40mm}L{15mm}L{15mm}L{15mm}L{15mm}L{25mm}}
        \toprule
        Evaluation criteria & Number of classes & Agreement & Cohen's Kappa & ICC & Mean Difference ($\pm$ SD) \\
        \midrule
        Title & 3 & 0.87 & 0.33 & - & - \\
        Authors & 3 & 1.00 & 1.00 & - & - \\
        Abstract & 3 & 0.95 & 0.64 & - & - \\
        \midrule
        Number of figures & - & 1.00 & - & 1.00 & 0.00 $\pm$ 0.00 \\
        Figure extraction errors & - & 0.89 & - & 1.00 & 0.11 $\pm$ 0.31 \\
        Figure caption errors & - & 0.89 & - & 1.00 & 0.11 $\pm$ 0.31 \\
        Number of tables & - & 0.92 & - & 0.98 & 0.12 $\pm$ 0.43 \\
        Table extraction errors & - & 0.89 & - & 0.98 & 0.17 $\pm$ 0.50 \\
        Table caption errors & - & 0.78 & - & 0.94 & 0.33 $\pm$ 0.67 \\
        \midrule
        Header/footer/footnote errors & - & 0.40 & - & 0.60 & 1.88 $\pm$ 2.12 \\
        Section heading errors & - & 0.71 & - & 0.79 & 0.71 $\pm$ 1.70 \\
        Body paragraph errors & - & 0.46 & - & 0.66 & 1.50 $\pm$ 2.22 \\
        \midrule
        Bibliography extraction & 4 & 0.94 & 0.82 & - & - \\
        Inline citation linking & 4 & 0.80 & 0.11 & - & - \\
        \midrule
        Overall score & 3 & 0.55 & 0.07 & - & - \\
        \bottomrule
    \end{tabularx}
    \caption{Inter-rater agreement for evaluation. For categorical questions, such as title, author, abstract, bibliography, inline citation, and overall score, we report the number of classes available for annotation, along with annotator agreement and Cohen's Kappa. For numerical questions, such as the number of each type of extraction error, we report agreement, the intraclass correlation coefficient (ICC), and the average difference and standard deviation of the values between the two annotators.}
    \label{tab:eval_iaa}
%     \Description{
% Evaluation criteria; Number of classes; Agreement; Cohen's Kappa; ICC; Mean Difference (+/- SD) 
% Title; 3; 0.87; 0.33; -; - 
% Authors; 3; 1.00; 1.00; -; - 
% Abstract; 3; 0.95; 0.64; -; - 
% Number of figures; -; 1.00; -; 1.00; 0.00 +/- 0.00 
% Figure extraction errors; -; 0.89; -; 1.00; 0.11 +/- 0.31 
% Figure caption errors; -; 0.89; -; 1.00; 0.11 +/- 0.31 
% Number of tables; -; 0.92; -; 0.98; 0.12 +/- 0.43 
% Table extraction errors; -; 0.89; -; 0.98; 0.17 +/- 0.50 
% Table caption errors; -; 0.78; -; 0.94; 0.33 +/- 0.67 
% Header/footer/footnote errors; -; 0.40; -; 0.60; 1.88 +/- 2.12 
% Section heading errors; -; 0.71; -; 0.79; 0.71 +/- 1.70 
% Body paragraph errors; -; 0.46; -; 0.66; 1.50 +/- 2.22 
% Bibliography extraction; 4; 0.94; 0.82; -; - 
% Inline citation linking; 4; 0.80; 0.11; -; - 
% Overall score; 3; 0.55; 0.07; -; - 
%     }
\end{table}

\begin{figure}[th!]
    \centering
    \includegraphics[width=0.9\linewidth]{figures/all_eval.png}
    \caption{Evaluation results for various document components. Corresponding numbers are provided in Table~\ref{tab:eval_raw_by_element} in Appendix~\ref{app:eval_raw_results}.}
    \label{fig:eval_results}
    \Description{Percent stacked bar plots showing evaluation results. Metadata elements (title, authors, abstract) are extracted correctly the vast majority of the time. Around a quarter of papers don't have figures and two fifths don't have tables. Of those with figures, the majority are extracted correctly, though extraction errors for both images and captions are not infrequent. Of those with tables, table extraction errors and table caption extraction errors are infrequent. Text element errors (header/footer/footnote, section heading, and body paragraph) are the most frequent. For header/footer/footnotes, a bit more than half of all papers have one or more errors. The majority of papers have more than 1 section heading extraction error. A bit less than half of papers have body paragraph errors that affect 1 or more paragraphs. Bibliography extraction and inline citation linking are both good, with the vast majority of papers having all or most entries extracted and linked. For overall readability, more than half of evaluated papers have no major problems, around a third have some problems, and the remaining lots of problems.}
\end{figure}

\begin{figure}[th!]
    \centering
    \includegraphics[width=0.8\linewidth]{figures/eval_fos.png}
    \caption{Overall readability results as proportion of total split by field of study, sorted by the percentage of papers with no major problems. The number of documents analyzed in each field is given, ranging from N=7 (History) to N=39 (Physics). The fields of study with the worst parse quality (Economics, Environmental science, Business, Art, and Political science) tend to be closer to the humanities, and may be due to the under-representation of papers from these fields in the data used to train the PDF parsers we use in our extraction pipeline. Corresponding numbers are provided in Table~\ref{tab:eval_raw_by_fos} in Appendix~\ref{app:eval_raw_results}.}
    \label{fig:eval_fos}
    \Description{Percent stacked bar plots show evaluation results for overall readability split by field of study. The bars are sorted on the field of study from those with the largest proportion of papers having no major problems to those with the least proportion having no major problems. Fields with the highest overall readability include history, engineering, sociology, physics, and chemistry. Fields with the lowest overall readability include political science, art, business, environmental science, and economics. No field stands out as being obviously different from the overall sample, though art, business and economics have the largest proportion of papers that were classified as having lots of readability problems.}
\end{figure}

All results and statistics are reported on the set of \numeval annotated papers. Figure~\ref{fig:eval_results} shows the breakdown of each type of error and the frequency at which it occurs. Metadata elements like title, authors, and abstract are successfully extracted the majority of the time. For figure and table elements, approximately 25\% of papers in our evaluation sample do not include figures, and around 45\% do not have tables. Of those that have figures, the majority (201, 69.1\% of 291) do not have extraction or parsing errors; around half of documents with errors have errors that only relate to one figure. Similarly, the majority of tables and table captions are correctly identified as tables and table captions, and are not incorrectly mixed into the body text. We note that the lack of an error here does not indicate that the table is extracted correctly in an accessible manner, just that it is not incorrectly parsed as body text.

Unsurprisingly, errors in text element parsing are the most prevalent, especially for headers/footers/footnotes and section headings. The most common type of header/footer/footnote error observed are when these texts are mixed into the body text around page breaks, interrupting reading flow. These types of errors are also observed frequently during screen reader use when reading directly from an untagged PDF. For section headings, in particular, the majority of papers have errors; around 67\% of papers have between 1--5 errors (either missed headings or extraneous headings), and 9\% have more than 5 errors. Due to the large number of section headings in papers, parsing errors are more frequent, and unfortunately, these errors impact the ability to properly navigate the HTML parse. 
Errors in body text extraction also negatively impact readability, in this case, select text in the document is being missed completely in the HTML render. We see that though the majority of parses have no body text errors, around 33\% of papers have between 1-5 missing paragraphs. 

Figure~\ref{fig:eval_results}(d) shows grading results for bibliography elements. Our pipeline is quite good at extracting bibliography entries, extracting all or most entries in the vast majority of cases, and successfully linking inline citations to these bibliography entries also in a large majority of cases. When bibliography extraction fails, it tends to fail catastrophically, resulting in no or few extractions.

The overall readability score is provided in Figure~\ref{fig:eval_results}(e). A majority of papers (54.5\%, 210 papers) have no major problems impacting readability. Another 31.7\% (122) of papers have some problems impacting readability, and 13.8\% (53) of papers have lots of readability problems. We are encouraged that a majority of HTML renders have no major problems, though our results necessitate further understanding of the papers with which our extraction pipeline has difficulty. If papers with lots of problems can be identified \textit{a priori}, we can prevent surfacing these low quality parses to the user. We perform some preliminary experiments to identify paper features that are more correlated with readability problems, though no features stood out as being predictive; we present those results in Appendix~\ref{app:eval_association}.

In Figure~\ref{fig:eval_fos}, we show the breakdown of overall readability by field of study,
plotting the proportion of papers per field that are classified as having no major problems, some problems, and lots of problems impacting readability. Many fields have similar distributions compared to the overall evaluation set. However, we note that some fields such as Art, Business, Economics, and Environmental science to some degree, have significantly lower quality extraction results. We posit that this may be due to biases in our PDF extraction pipeline. Some of the machine learning modules we use are primarily trained on paper data from the biomedical and Computer Science domains, where large scale labeled PDF extraction datasets can be found. Humanities-adjacent fields like Art and Business have very different publication norms, and the different layouts and content of papers and documents in these fields may provide additional challenges to our system, resulting in lower quality extraction and rendering. 


\section{User study}
\label{sec:user_study}

We conduct an exploratory user study to better understand the needs of BLV scientists when reading papers, and to assess whether our prototype supports these needs. The study consists of a preliminary questionnaire and semi-structured video interview. Interviews are conducted remotely on Zoom.\footnote{\href{https://zoom.us/}{https://zoom.us/}} All recruitment materials, questionnaires, and the interview plan are reviewed and approved by the internal review board at \allenai. We recruit and interview \numusers users, with a pilot involving two users, and a main study involving four users. Modifications to the prototype between pilots and the main study can be found in Section~\ref{sec:pdf2html}. We report results from all \numusers participants in any analysis that does not involve the prototype, and for analysis that directly involves the prototype, we denote all cases where prototype modifications between the pilot and main study may impact our results.

The inclusion criteria for participants are:

\begin{itemize}
    \item The participant is over 18 years of age;
    \item The participant identifies as blind or low vision;
    \item The participant reads scientific papers regularly (more than 5 per year);
    \item The participant must have used a screen reader to read a paper in the last year; and
    \item The participant must complete the pre-interview questionnaire.
\end{itemize}

Participants were
%\jonathan{i think past tense is a little more clear for things we did} 
recruited through mailing lists, word-of-mouth, and snowball sampling. Prior to each interview, the participant was asked to provide several keywords corresponding to their subject areas of interest, and between 3--5 papers where they experienced difficulty reading the PDF. Among the 3--5 papers, we selected one paper to use for the study, based on the availability of an HTML render, and maximizing the features that would be seen during the user study (e.g., given a choice between a paper with figures and a paper without figures but where both otherwise demonstrate the same paper components, we would select the paper with figures). Each study session was 75 minutes, consisting of three phases:

\begin{itemize}[itemsep=5pt]
    \item[] \textbf{Phase I: Capturing challenges with current work flow} \newline
    The primary research questions we investigate in this phase are: 
    \begin{itemize}[noitemsep, leftmargin=0.4in]
        \item[--] What methods and/or tools do BLV researchers use to assist in reading the literature?
        \item[--] What main accessibility challenges do BLV researchers face?
        \item[--] How do BLV researchers cope with these challenges?
    \end{itemize}
    We first asked the participant to describe their current workflow and the challenges they face when reading papers, clarifying how the user copes with challenges when their workflow does not adequately address the problem. We then asked the participant to demonstrate how they currently read a paper, by opening a paper PDF and walking us through the usage of their tools (PDF viewer, screen reader, magnifier, speech-to-text, etc). Participants kept their computer audio on so we could hear the output of their reader tools. The participant was asked to think aloud and describe their actions when reading the paper. We asked the participant to demonstrate any reading challenges they described in their pre-interview questionnaire. At the end of this phase, we asked the participant to assess how easy or difficult it was to read the paper with their current reading pipeline.
    \item[] \textbf{Phase II: Interaction with prototype} \newline 
    The primary research questions we investigate in this phase are: 
    \begin{itemize}[noitemsep, leftmargin=0.4in]
        \item[--] What features of the HTML render resonated positively with the participant?
        \item[--] What problems can be identified in the HTML render?
    \end{itemize}
    The goal of this phase was to understand how helpful or not helpful the HTML render is to the participant. The participant was asked to interact with an HTML render of the same paper they read in Phase I in the \scially prototype. We first provided an introduction to the prototype, then allowed the participant to proceed uninterrupted for several minutes interacting with the render. The participant was asked to think aloud during their interactions. Towards the end of this phase, we prompted the participant to interact with any features in the HTML render they may have skipped over. At the end of this phase, we asked the participant to assess how easy or difficult it was to read the paper with the HTML render.
    \item[] \textbf{Phase III: Q\&A and discussion} \newline 
    The primary objectives of this phase are to answer the questions:
    \begin{itemize}[noitemsep, leftmargin=0.4in]
        \item[--] How likely is the participant to use the HTML render in the future?
        \item[--] How can the HTML render be improved to best meet the participant's needs moving forward?
    \end{itemize}
    The participant was given further opportunities to ask questions or discuss the prototype. The participant was asked to describe their perceived pros and cons of the prototype, and to provide suggestions of missing features, ordered by priority. We asked the participant whether they would use this prototype if it were available, and if not, what features would need to be implemented to change that decision. 
\end{itemize}

\noindent The interviews were conducted by one author, with two other authors observing and participating during Phase III. All interviews were recorded for followup analysis, and participants were compensated with a \$150 USD gift card for their time. The questions used to guide the semi-structured interview are provided in Appendix~\ref{app:interview_questions}. 

We follow a grounded theory approach to identify themes and concepts from the participant interviews. We first perform open coding to identify relevant concepts, then axial coding to group these concepts under broad themes. These themes are 1) the technologies employed by users, 2) challenges in their current reading pipeline, and 3) mitigation or coping strategies, and in relation to the \scially prototype: 4) positive features, 5) negative features or issues with the prototype, and 6) suggestions for improvement. 
Interviews are selectively coded a second time to identify all concepts falling under each theme. We also employ the same method to code issues raised by participants in the pre-interview questionnaire. 

Themes and concepts are arrived upon by two authors following detailed reading of the interviews. In several cases, we further define attributes associated with some concepts, such as defining whether the technologies used were in relation to opening PDFs, screen reading, or other tasks; or whether the challenges identified affect the whole document, navigation, text, or a particular in-paper element. These delineations are described further in their respective results sections.

\subsection{Study participants}

Participants are graduate students, PhD students, and faculty members from predominantly English-speaking countries, whose primary research areas are in computer science, though also spanning neuroscience and mathematics. We interviewed two participants during the pilot phase and four participants during the main phase of our study. We report findings from all six participants for all themes captured in Phase I of the study. Since only minor changes were made to the prototype between the pilot and main study, we report findings from all participants for Phase II and III as well, making note of features that changed following the pilots. Three of six participants
study human-computer interaction and accessibility, which may be due in part to our sampling methodology, but may also reflect the relevance of accessibility research to BLV researchers. Other study participants conduct research in the areas of machine learning, neuroscience, software engineering, and blockchain. All but one participant reported having more than one year of experience using screen readers. The tools employed by participants are summarized in Table~\ref{tab:user_summary} along with the version of the \scially prototype with which they interacted.

\begin{table}[t!]
    \small
    \centering
    \begin{tabularx}{0.8\linewidth}{lllL{70mm}}
        \toprule
        \textbf{ID} & \textbf{Study} & \textbf{Prototype Version} & \textbf{Current Tools} \\
        \midrule
        P1 & Pilot & v0.1 & NVDA Screen Reader, Adobe Acrobat Reader \\
        \midrule
        P2* & Pilot & v0.2 & Mac Text-to-speech, Mac Magnifying Glass (sighted navigation), Mac Preview \\
        \midrule
        P3 & Main & v0.3 & Braille display, Mac VoiceOver, JAWS/NVDA on Windows, Mac Preview, Adobe Acrobat Reader \\
        \midrule
        P4 & Main & v0.3 & Mac VoiceOver, Mac Preview or Adobe Acrobat Reader \\
        \midrule
        P5 & Main & v0.3 & Microsoft Narrator, Adobe Acrobat Reader \\
        \midrule
        P6 & Main & v0.3 & Braille display, InftyReader, Mac VoiceOver, Mac Preview  \\
        \bottomrule
    \end{tabularx}
    \caption{User study participants, the prototype versions they interacted with, and the tools they currently use for reading papers. *P2 is low vision and uses sighted navigation tools in conjunction with a screen reader.
    }
    \label{tab:user_summary}
%     \Description{
% ID; Study; Prototype Version; Current Tools 
% P1; Pilot; v0.1; NVDA Screen Reader, Adobe Acrobat Reader 
% P2*; Pilot; v0.2; Mac Text-to-speech, Mac Magnifying Glass (sighted navigation), Mac Preview 
% P3; Main; v0.3; Braille display, Mac VoiceOver, JAWS/NVDA on Windows, Mac Preview, Adobe Acrobat Reader 
% P4; Main; v0.3; Mac VoiceOver, Mac Preview or Adobe Acrobat Reader 
% P5; Main; v0.3; Microsoft Narrator, Adobe Acrobat Reader 
% P6; Main; v0.3; Braille display, InftyReader, Mac VoiceOver, Mac Preview  
%     }
\end{table}

\subsection{Study findings}

\subsubsection*{Summary of current experience} 

Of the \numusers participants, three users have experience with screen readers on the Windows OS, such as NVDA, JAWS, and Microsoft Narrator, and three users use VoiceOver on MacOS. Two users use braille display in conjunction with their screen reader. One participant (P2) is low vision and uses a combination of text-to-speech and a magnifying glass to perform sighted navigation; P2's primary reading interaction involves selecting blocks of text in the PDF and using text-to-speech. Adobe Acrobat Reader is the most common software for opening PDFs; though several participants use Preview in MacOS, with one participant (P4) explicitly stating a preference for Preview over Acrobat. One participant uses a proprietary tool called InftyReader, which converts PDFs into ASCII text and math formulas into MathML, which is accessible.

\subsubsection*{Challenges of current PDF reading pipeline}

Table~\ref{tab:current_challenges} lists the challenges recognized by all participants in their current PDF reading pipeline. Some of these challenges affect the entire document, e.g., when a document lacks heading markup, it affects the ability to navigate the whole document. Others pertain to specific elements in PDFs, like inaccessible math formulas or lack of figure alt-text. All six users discussed the inaccessibility of math formulas. Unfortunately, document elements like math, figures, tables, and algorithm blocks are used to convey a significant amount of the information content of a paper, and the inability to access their content can produce negative impacts on the reader's ability to understand the paper.

\begin{table}[t!]
    \centering
    \begin{tabular}{lll}
        \toprule
        \textbf{Issue description} & \textbf{Affects} & \textbf{Raised by user} \\
        \midrule
        Scanned PDFs cannot be read without remediation & Document & P3, P4, P5* \\
        \midrule
        No headings/sub-headings for navigation & Navigation & P1, P3, P5 \\
        Figures are not annotated as figures & Navigation & P1, P5 \\
        Losing cursor focus when switching away from the PDF & Navigation & P1 \\
        Headings are not hierarchical (no sub-headings) & Navigation & P5 \\
        \midrule
        Text is read as single string (no spaces or punctuation) & Text & P1, P4, P5 \\
        Headers/footers/footnotes mixed into text & Text & P1, P4, P5 \\
        Words with ligatures are mispronounced & Text & P1, P3 \\
        Words split at line breaks are mispronounced & Text & P2, P3 \\
        Reading order is incorrect & Text & P3, P5 \\
        Text before and after figures sometimes skipped & Text & P4 \\
        Text on some pages not recognized at all & Text & P4 \\
        \midrule
        Math content is inaccessible & Element & P1, P2, P3, P4, P5, P6 \\
        Tables are inaccessible & Element & P1, P2*, P3, P5, P6 \\
        Figures lack alt-text & Element & P1, P3, P5, P6 \\
        Figure captions are not associated with figures & Element & P1, P5 \\
        Characters or words in figures are read and do not make sense & Element & P4, P5 \\
        Figure alt-text (when provided) is not descriptive & Element & P5 \\
        Code blocks are inaccessible & Element & P2, P4 \\
        \bottomrule
    \end{tabular}
    \caption{Challenges to PDF reading identified by participants during interviews. *Only identified as an issue during pre-interview questionnaire.}
    \label{tab:current_challenges}
%     \Description{
% Issue description; Affects; Raised by user 
% Scanned PDFs cannot be read without remediation; Document; P3, P4, P5* 
% No headings/sub-headings for navigation; Navigation; P1, P3, P5 
% Figures are not annotated as figures; Navigation; P1, P5 
% Losing cursor focus when switching away from the PDF; Navigation; P1 
% Headings are not hierarchical (no sub-headings); Navigation; P5 
% Text is read as single string (no spaces or punctuation); Text; P1, P4, P5 
% Headers/footers/footnotes mixed into text; Text; P1, P4, P5 
% Words with ligatures are mispronounced; Text; P1, P3 
% Words split at line breaks are mispronounced; Text; P2, P3 
% Reading order is incorrect; Text; P3, P5 
% Text before and after figures sometimes skipped; Text; P4 
% Text on some pages not recognized at all; Text; P4 
% Math content is inaccessible; Element; P1, P2, P3, P4, P5, P6 
% Tables are inaccessible; Element; P1, P2*, P3, P5, P6 
% Figures lack alt-text; Element; P1, P3, P5, P6 
% Figure captions are not associated with figures; Element; P1, P5 
% Characters or words in figures are read and do not make sense; Element; P4, P5 
% Figure alt-text (when provided) is not descriptive; Element; P5 
% Code blocks are inaccessible; Element; P2, P4 
%     }
\end{table}

\subsubsection*{Coping mechanisms}

The coping mechanisms employed by BLV researchers to read inaccessible PDFs are wide-ranging, often involving trying tools outside of their primary workflow, soliciting help from others, or in the worst case, giving up and moving on. We describe these in Table~\ref{tab:coping_mechanisms}. Several users reported trying certain tools like alternate PDF readers, browsers, or optical character recognition (OCR), even though the tools usually do not result in a significant improvement over their standard pipeline; when asked why, several participants reported feeling ``hopeful'' that a tool might work (P1) or hoping to get lucky (P3).

Several of these coping mechanisms involved other people. For example, three participants reported needing to ask sighted colleagues or family members to copy text, or to explain select paper content, especially figures and equations. Asking for PDF remediation was also a possibility for several participants; in this process, workers at the researcher's host institution convert a PDF into an accessible format, manually correcting equation representation and writing descriptions for figures. The output of the remediation process is seen as ``ideal'' (P4), but the process takes significant time (several weeks for any PDF) and may not fit into a researcher's schedule and timeline. Additionally, this process may only be available to researchers affiliated with a significantly large and resourced institution, and as P6 discusses, may no longer be a viable option for those who work outside of academia. In some cases, BLV researchers may also message authors directly to gain access to the source documents (P3 and P4). Both LaTeX source and Word documents are more accessible than PDFs, and access to these source documents can greatly improve the ability to read these papers.

Perhaps most disheartening is how often BLV researchers may simply give up in the face of an inaccessible paper. P1 says that by the time he has spent several hours making a paper readable, he may have already lost interest and motivation to read it. When asked how often papers are abandoned, P3 responds 60--70\% of the time. Though P4 does not discuss abandonment directly, P4 shares the following relevant sentiment: ``reading papers is the hardest part of research'' for a BLV researcher, and if papers were more accessible, there would be more blind researchers.

\begin{table}[t!]
    \small
    \centering
    \begin{tabular}{llp{60mm}}
        \toprule
        \textbf{Coping mechanism} & \textbf{Raised by user} & \textbf{What users said} \\
        \midrule
        Give up, abandon the paper & P1, P3, P5 & P3: when asked how often they abandon papers, answers ``60--70\% of the time'' \newline P5: sometimes the only option is to ``sit down and start crying'' (jokingly, though the sentiment is true) \\
        \midrule
        Try other conversion tools & P1, P3, P6 & \\
        \midrule
        Download LaTeX source or Word document if available & P3, P4, P6 & \\
        \midrule
        Ask sighted colleagues or family members to read & P3, P5, P6 & \\
        \midrule
        Ask for remediation / convert to braille & P4, P5, P6 & P4: 10 day turnaround is on the quick side, which is not good enough for research \newline P5: process takes a long time, around 1-2 weeks \\
        \midrule
        Try other PDF readers or browsers & P1, P6 & P1: may try Microsoft Edge browser even though it usually does not help, but he feels ``hopeful'' \\
        \midrule
        Message authors to get source document & P3, P4 & P4: sometimes the author manuscript is accessible but the camera-ready version is not; fault of the conferences and publishers, not the authors \\
        \bottomrule
    \end{tabular}
    \caption{Coping mechanisms discussed by users for dealing with challenging papers.}
    \label{tab:coping_mechanisms}
    \Description{
% Coping mechanism; Raised by user; What users said 
% Give up, abandon the paper; P1, P3, P5; P3: when asked how often they abandon papers, answers "60--70\% of the time", P5: sometimes the only option is to "sit down and start crying" (jokingly, though the sentiment is true) 
% Try other conversion tools; P1, P3, P6; 
% Download LaTeX source or Word document if available; P3, P4, P6; 
% Ask sighted colleagues or family members to read; P3, P5, P6; 
% Ask for remediation / convert to braille; P4, P5, P6; P4: 10 day turnaround is on the quick side, which is not good enough for research, P5: process takes a long time, around 1-2 weeks 
% Try other PDF readers or browsers; P1, P6; P1: may try Microsoft Edge browser even though it usually does not help, but he feels "hopeful" 
% Message authors to get source document; P3, P4; P4: sometimes the author manuscript is accessible but the camera-ready version is not; fault of the conferences and publishers, not the authors 
    }
\end{table}

\subsubsection*{Response to HTML render}

All user interviews were analyzed to extract positive and negative responses to various features or flaws of the prototype. We summarize these features and flaws in Table~\ref{tab:prototype_features}. Among the participants' favorite features are links between inline citations and references (all 6 participants), section headings for navigation (5 participants), the table of contents (4 participants), and figures tagged as figures with associated figure captions (3 participants). Regarding links between inline citations and references, several participants were especially supportive of the return links that allow the reader to return back to their reading context after following a citation link. P3 said that the links acted as external memory, allowing BLV users to essentially ``glance'' at the bibliography and back, like a sighted user might. Similar sentiments were shared by P5 and P6, although P5 also proposed the possibility of preserving the context even further by providing bibliography information inline rather than navigating back and forth between the main text and references section.

Among the negative features observed by participants, most have to do with imperfect extraction, for example, incorrectly extracted headings (3 participants), missed headings (2 participants), and various extraction issues with code blocks, tables, equations, and more. Many of these issues are known and quantified in Section~\ref{sec:evaluation}. Of these issues, problems with heading extraction were most notable, likely because the heading structure is the first element of the document with which the participants interact, and it provides a mental model of the overall document structure. Mistakes in heading extraction are obvious and erode trust in our overall system. As P5 says, ``it's really important that I trust it,'' and errors of this nature, both false positive and false negative extractions, can reduce trust. Similarly, though we describe in our introductory material that our system currently does not extract equations, P6 points out that it is unclear whether the system extracts equations because occasionally math can be found in the body text. This type of conflict between what is described and what is seen can also reduce trust. However, one may be able to build trust even in the face of extraction errors by indicating to the user when content is not extracted; as P4 says regarding the placeholders for not extracted items, ``at least I know there was an equation here.''

\begin{table}[t!]
    \footnotesize
    \centering
    \begin{tabularx}{\linewidth}{llp{60mm}}
        \toprule
        \textbf{Feature} & \textbf{Raised by user} & \textbf{What users said} \\
        \midrule
        \textsc{Positive} & & \\
        \midrule
        Bidirectional links between inline citations and references & P1, P2, P3, P4, P5, P6 & P3: ``very few research teams actually get this and get this right, so well done''; ``crucial piece of the puzzle'' \\
        Headings for easy navigation & P1, P2, P3, P4, P6 & P4: ``Headings are the best thing ever''; makes it very clear what section you are in \\
        Table of contents* & P2, P3, P5, P6 & \\
        Figures are tagged as figures, and captions are associated & P4, P5, P6 & \\
        Can use browser and OS features like find/copy/paste  & P1, P4 & \\
        Simple typography for reading & P2 & \\
        Can interact with headings word-by-word or letter-by-letter & P4 & \\
        Not extracted items are noted as missing & P4 & P4: ``at least I know there was an equation here'' \\
        \midrule
        \textsc{Negative} & & \\
        \midrule
        Some headings extracted incorrectly & P1, P3, P5 & \\
        Some headings missed in extraction & P3, P5 & P5: ``it's really important that i trust it''; ``there [should be] *no* false negatives'' \\
        Code block not extracted & P2, P4 & \\
        Tables are extracted as figures & P2, P6 & \\
        Equations not extracted & P4, P6 & P6: Not sure if this system extracts equations because sometimes there is some math in the body text \\
        Figures placed away from text* & P1 & \\
        No alt-text extracted & P1 & \\
        URLs missing from bibliography entries** & P2 & \\
        Some information not surfaced (keywords, footnotes) & P3 & \\
        Some headers/footers/footnotes mixed in text & P4 & \\
        Headings are not hierarchical & P5 & \\
        \bottomrule
    \end{tabularx}
    \caption{Positive and negative features identified in the prototype. *The feature was implemented or the issue addressed in v0.2 following P1 pilot. **The issue was addressed in v0.3 following P2 pilot.}
    \label{tab:prototype_features}
%     \Description{
% Feature; Raised by user; What users said 
% Positive; 
% Bidirectional links between inline citations and references; P1, P2, P3, P4, P5, P6; P3: "very few research teams actually get this and get this right, so well done"; "crucial piece of the puzzle" 
% Headings for easy navigation; P1, P2, P3, P4, P6; P4: "Headings are the best thing ever"; makes it very clear what section you are in 
% Table of contents*; P2, P3, P5, P6; 
% Figures are tagged as figures, and captions are associated; P4, P5, P6; 
% Can use browser and OS features like find/copy/paste ; P1, P4; 
% Simple typography for reading; P2; 
% Can interact with headings word-by-word or letter-by-letter; P4; 
% Not extracted items are noted as missing; P4; P4: "at least I know there was an equation here" 
% Negative; 
% Some headings extracted incorrectly; P1, P3, P5; 
% Some headings missed in extraction; P3, P5; P5: "it's really important that i trust it"; "there [should be] *no* false negatives" 
% Code block not extracted; P2, P4; 
% Tables are extracted as figures; P2, P6; 
% Equations not extracted; P4, P6; P6: Not sure if this system extracts equations because sometimes there is some math in the body text 
% Figures placed away from text*; P1; 
% No alt-text extracted; P1; 
% URLs missing from bibliography entries**; P2; 
% Some information not surfaced (keywords, footnotes); P3; 
% Some headers/footers/footnotes mixed in text; P4; 
% Headings are not hierarchical; P5; 
%     }
\end{table}



\subsubsection*{Difficulty scale} 

The responses of the users to the difficulty of their current pipeline versus the HTML render are shown in Table~\ref{tab:taskload}. We ask the following question: \textit{On a scale of 1 to 5, how easy or difficult was it to read this paper with the HTML render, and why? (Answers: 1 = Very easy; 2 = Easy; 3 = Neutral; 4 = Difficult; 5 = Very difficult)}

All participants in the main study reported that the HTML render is easier for reading than their current pipeline. Reductions in difficulty rating ranged from 0.5 to 3.0. Most of our participants rated their current pipeline as difficult (4 participants) or neutral (1 participant), with one participant who is low vision (P2) reporting that their current pipeline is easy. During our pilot sessions, users reported that the HTML render was difficult to use. For the main study, users reported the HTML render as neutral or easy to use.

P2 is the only participant to report the HTML render as being more difficult to use than their current pipeline; we note that P2 is sighted and did not engage with most of the navigation features we designed and implemented for screen reader-based navigation. Because P2 primarily interacted with papers through sighted navigation, text highlighting, and text-to-speech, they were able to interact with section headers, figures, tables, and equations in the original PDF using the magnifier tool, and found any missing content in the HTML render to be significantly detrimental to their reading experience.

The overall median difference in difficulty scores between the PDF and HTML render is modest, at 0.75. This modest change may be due to the conflation of interface design and system errors when asking participants to rate the difficulty of use. In general, all users responded very positively to the interface design, especially around the navigational features we introduce. Issues were raised around extraction accuracy and the propagation of these errors to the interface. We may be able to offset some of the latter issues by detecting and removing papers that suffer from more extraction errors, though we leave this to future work.

\begin{table}[tb!]
    \centering
    \begin{tabular}{llp{12mm}p{12mm}lp{15mm}}
    \toprule
        \textbf{ID} & \textbf{Study} & \textbf{Current pipeline} & \textbf{HTML render} & \textbf{Difference} & \textbf{Would use in future} \\
    \midrule
        P1 & Pilot & 4.0 & 4.0 & 0.0 & Yes \\
        P2 & Pilot & 2.0 & 4.0 & \color{red}{-2.0} & Yes \\
    \midrule
        P3 & Main & 3.0 & 2.0 & \color{darkgreen}{1.0} & Yes \\
        P4 & Main & 4.0 & 1.0 & \color{darkgreen}{3.0} & Yes \\
        P5 & Main & 4.0 & 3.0 & \color{darkgreen}{1.0} & Yes* \\
        P6 & Main &4.0 & 3.5 & \color{darkgreen}{0.5} & Yes \\
    \bottomrule
    \end{tabular}
    \caption{Participant ratings on the difficulty scale (1 = very easy, 2 = easy, 3 = neutral, 4 = difficult, 5 = very difficult) and whether they would use the tool in the future. All participants reported a change from more difficult to more easy when moving from their current pipeline to the HTML render except P2, who uses sighted navigation. The median reduction in difficulty score for all participants is 0.75. All participants reported that they would be very likely to use the system in the future were it to be available; P5's response is contingent on improvements in section heading extraction.
    }
    \label{tab:taskload}
%     \Description{
% ID; Study; Current pipeline; HTML render; Difference; Would use in future 
% P1; Pilot; 4.0; 4.0; 0.0; Yes 
% P2; Pilot; 2.0; 4.0; -2.0; Yes 
% P3; Main; 3.0; 2.0; 1.0; Yes 
% P4; Main; 4.0; 1.0; 3.0; Yes 
% P5; Main; 4.0; 3.0; 1.0; Yes* 
% P6; Main &4.0; 3.5; 0.5; Yes 
%     }
\end{table}

\subsubsection*{Future usage}

At the end of each session, we ask users whether they would be likely to use the prototype in the future if it were made publicly available on a range of papers. We ask specifically: \textit{On a scale of 1 to 5, how likely are you to use the HTML render, if it is available to you in the future? (Answers: 1 = Very unlikely, 2 = Unlikely, 3 = Neutral, 4 = Likely, 5 = Very likely)} If the answer is unlikely or neutral, we ask what changes would need to be made to the tool such that they would use it.

All users reported that they would use the prototype in the future.
Five users responded 5, that they would be very likely to use it; one user (P5) responded 3 to the prototype as it currently is, and 5 if some of the issues for heading extraction were addressed. P1, who participated in an early pilot with fewer implemented features, said that this would become a tool in the toolbox, but he would not be able to rely solely on it due to incomplete extractions. P5 expressed a similar sentiment, that in its current state, he may try the prototype system when his current workflow fails, but if issues around heading extraction were addressed, he would be very likely to use it. P3 replies when asked how the system might be integrated into their workflow, ``I think it would become the workflow.'' P4 says ``for unaccessible PDFs, this is life-changing.''


\subsection{Design recommendations}
\label{sec:designrecs}

\begin{figure}[t!]
    \centering
    \includegraphics[width=0.8\linewidth]{figures/designrecs.png}
    \caption{Design recommendations for screen reader friendly paper reading systems. A system should aim to provide the document structure in a way that matches the mental model of the user, and to tag all elements appropriately. These aspects are achievable through proper tagging of a paper, including in PDF format. Additionally, a system should aim to act as external memory for the user, minimizing the amount of cognitive load needed to return to their reading context. To improve trust, a system should indicate when there is known missing data in the extraction or a possibility of missing or incorrect data. Finally, a system should reduce verbosity, ensuring that as few keystrokes as possible are necessary for the user to perform their desired task. 
    }
    \label{fig:design_recs}
    \Description{Five design recommendations. 1. A picture of a building on a circular dark blue background. Text that reads "Document structure should match the mental model of the user; headings should be annotated and hierarchical; reading order should be specified; reading order should match expectations, e.g. starts with title, authors, abstract, introduction and so on." 2. A picture of the word <tag/> on a circular blue background. Text that reads "Objects in the document should be tagged appropriately; headings should be tagged heading 1 through heading 6 as appropriate; figures and tables should be tagged as such; in HTML, lists should be tagged as unordered or ordered lists (<ul> or <ol>) as appropriate." 3. A picture of a brain on a circular yellow background. Text that reads "The system should act as external memory for the user; in-document links can be bidirectional to imitate 'glancing' behavior; bookmarks can be useful for returning to notable passages." 4. A picture of a no entry sign on a circular light gray background. Text that reads "Indicate known missing data and potential errors; some amount of error is tolerable, but these should be indicated to the user; builds user trust in the system." 5. A picture of a pair of lips on a circular dark gray background. Text that reads "Reduce verbosity; remove unnecessary text around links and other introduced text elements or features; minimize key strokes necessary for navigation."}
\end{figure}

We distill our learnings into a set of five design recommendations for BLV user-friendly paper reading systems. Figure~\ref{fig:design_recs} summarizes the following recommendations:

\begin{enumerate}[itemsep=4pt]
    \item[1.] \textbf{Document structure should match the mental model of the user.} Structure is necessary for providing an overview of a document and is essential to navigation. Headings in a paper should be tagged as such and the hierarchy of the headings should match the mental model of the user, i.e., top level headings should be tagged \texttt{<h1>} or \texttt{<h2>}, and lower level headings \texttt{<h3>} through \texttt{<h6>} accordingly. Reading order should be specified, as to not interject non-body text objects into the body text, e.g., headers, footers, and footnotes often disrupt the main flow of text because they visually break paragraphs. Similarly, a user expects a natural flow to a paper, beginning with the title, authors, abstract, introduction etc, and ending with conclusions and references. Papers with various elements interspersed are disruptive of this mental model and can interfere with the reader's understanding of the document.

    \item[2.] \textbf{Objects in the paper should be tagged appropriately.} Self-explanatory. Headings should be tagged as headings, figures as figures, tables as tables, lists as lists and so on. Appropriate tagging allows a user to take advantage of the screen reader's capabilities for navigating to specific types of objects, e.g., most screen readers have shortcuts for navigating headings, and to figures or lists. Proper tagging emulates a sighted user's ability to detect visually distinct objects such as headings, figures, and tables. When objects are not appropriately tagged, a screen reader user must scroll through the whole document each time to identify the desired sections.

    \item[3.] \textbf{The system should act as external memory for the user.} 
    Visual layout can act as a source of external memory for sighted users, who can quickly derive reading context and object types from visual cues. For BLV users, strategies for emulating such external memory can be beneficial. For example, bi-directional navigation for all in-document links are a type of ``glancing'' feature. With this feature, a user no longer needs to commit text to memory in order to rediscover their previous reading context after navigating away. P3, in particular, emphasizes that these features are a ``crucial piece of the puzzle.'' Other memory features like bookmarking or note-taking may also be helpful for returning the user to their reading context.

    \item[4.] \textbf{Indicate known missing data and potential errors.} To facilitate trust in the system, the system should indicate the presence of missing and erroneous data to users. Some degree of fault tolerance is permitted, as long as the overall benefit to the user is greater. However, as these systems rely on statistical methods, extraction quality is rarely perfect. Most users indicate a preference for knowing when the system fails, rather than dealing with the uncertainty of figuring out whether the issue is with the underlying paper, or with the extraction and reading interface.

    \item[5.] \textbf{Reduce verbosity.} Any minimization of unnecessary text and spaces between links can simplify navigation for BLV users. Though these extra commas and spaces may seem innocuous for sighted users, they require extra keystrokes for screen readers. Reduction of unnecessary verbosity around links and introduced features can save time for screen reader users.

\end{enumerate}

The overarching themes of these recommendations are to reduce user cognitive load and improve trust in the system. Regarding cognitive load, interruptions to reading flow for BLV users are especially disruptive, since there are no visual markers to help identify reading context. Paper reading systems for BLV users should therefore attempt to mitigate cognitive load caused by loss of context, by allowing users to quickly navigate back to their reading context when following any links, and by avoiding any disruption of reading flow. Regarding this latter point, properly labeled reading order, headings for navigation, and appropriately tagged objects all contribute to mitigating disruptions. Further, it is also important to remove interjections from headers, footers, footnotes, figure and table captions, and other text, all of which interrupt the natural flow of reading.

Regarding user trust in the system: this should a priority of any system builder. Because PDF extraction and document rendering are imperfect processes, some degree of error is expected. Though all participants in our user study expressed that some degree of error is tolerable, one can mitigate the conversion of errors to distrust by clearly indicating known errors and missing content in the system. For example, in some cases our system is unable to extract a figure caption; if the caption for Figure 3 is not extracted, rather than skipping from Figure 2 to Figure 4 and causing confusion for the reader, it is better to indicate that Figure 3 is missing in the extraction.

A system that responds quickly to user requests is obviously more desirable. However, several participants indicated that some wait time is acceptable, especially if a longer wait time corresponds to a higher quality reading experience. Though we report this finding, we ask readers to take it with a grain of salt. This point may not hold for all or even a majority of users, since several users also remark on the PDF remediation process (which usually takes 1--2 weeks) as being too long to adequately support their research workflow.

Though we derive these design recommendations in the scope of paper reading, they are generalizable to other classes of documents. In fact, several of these design principles echo available guidelines for human-AI interaction \citep{Amershi2019GuidelinesFH}, especially in indicating the capabilities and limitations of the system (recommendation 4). A number of our recommendations are simply good practice, such as exposing the structure of a document and tagging document objects appropriately, and are covered by current guidelines for creating accessible documents. Other recommendations focus on emulating the types of advantages that sighted users derive from layout and visual information, but to implement them in such a way that BLV users can benefit, e.g. using the system as a source of external memory.

\section{Discussion}
\label{sec:discussion}

We introduced our work by identifying a limitation of existing cataloging pipelines: centroiding, deblending, photometry, star/galaxy separation, and incorporation of priors happen in distinct stages.
Uncertainty is typically not propagated between stages.
Any uncertainty estimates these pipelines produce are based on conditional distributions---that is, they are conditional on the output of the previous stages.

We developed a joint model of light sources' centers, colors, fluxes, shapes, and types (star/galaxy).
Whereas previous approaches to cataloging have been framed in algorithmic terms,
statistical formalisms let us characterize our inferences without ambiguity.
Statistical formalisms also make modeling assumptions transparent---whether the assumptions are appropriate ultimately depends on the downstream application.
We highlighted limitations of the model to guide further development.

A model is only useful when it can be applied to data.
We proposed two procedures: one based on MCMC and the other on VI.
Neither MCMC nor VI could be applied to our model without customization.
The need for problem-specific adjustments is a barrier to the broader adoption of both techniques.
With MCMC, for example, we went through several iterations before settling on slice sampling and AIS, including Metropolis-Hastings (MH) and reversible jump~\citep{green1995reversible}.
Compared to slice sampling, we found MH difficult to tune.
We found that reversible-jump MCMC required carefully constructed proposals to jump often enough between the star and galaxy models and was also difficult to tune.

VI required even more problem-specific customization.
Our VI techniques include the following: 1) approximating an integrand with its second-order Taylor expansion; 2) approximating the point-spread function with a mixture of Gaussians; 3) upper bounding the KL divergence between the color and a GMM prior; 4) limiting the variational distribution to a structured mean-field form; 5) limiting the variational distribution to point masses for some parameters; and 6) optimizing the variational lower bound with a variant of Newton's method rather than coordinate ascent. This final technique was particularly laborious, as it involved manually deriving and implementing both gradients and Hessians for a complicated function.

On synthetic data, MCMC was better at quantifying uncertainty, which is likely due to the restrictive form
of the variational distribution.
Additionally, MCMC provided uncertainty estimates for all latent random variables, whereas VI modeled some random variables as point masses---in effect recovering maximum a posteriori (MAP) estimates for them.
However, MCMC was approximately $1000\times$ slower than VI.

On real data, point estimates from VI were not always worse than point estimates from MCMC. Neither procedures' uncertainty estimates were perfectly calibrated for galaxies, suggesting some degree of model misspecification.
Imperfectly calibrated uncertainties can nonetheless be useful, e.g., for flagging particularly unreliable point estimates. Additionally, even if the uncertainties are ignored by downstream analyses, point estimates typically improve when uncertainty is modeled.
For questions requiring calibrated uncertainties, enhancing the galaxy model may help to reduce model misspecification.
Though the galaxy model we use---one with elliptical contours---is standard in astronomy, a more flexible galaxy model shows promise~\citep{regier2015deep}.

For spectrographic targeting, our current catalog should nonetheless
be an improvement over what came before: previously, uncertainty estimates and prior information were ignored. For analysis of subpopulations, however, we stress a key difference between our catalog and traditional astronomical catalogs: our catalog is based on prior information, whereas traditional catalogs are not.
Moreover, though our prior is accurate enough for large-scale cataloging and deblending, it likely is not accurate enough for a final scientific analysis of a particular subpopulation of light sources (e.g. the galaxies with an ``active galactic nucleus'').
For this use case, which is beyond the scope of our work, we suggest two approaches. First, a user can form a Laplace approximation, to ``remove'' our priors from the catalog and replace them with priors that are more suitable for their subpopulation. To facilitate, any catalog generated with our method should also contain parameters of the priors used to generate it.
Catalog users can then apply new priors directly to the catalog, without revisiting the image data; astronomers typically prefer to work with catalogs rather than images because catalogs are so much smaller.

We would prefer that users deal differently, however, with model misspecification that affects their analysis:
instead of trying to work around model misspecification, enhance our model.
Then, rerun our cataloging software, with the new model, on the images. This approach encourages users to adapt the statistical model and the priors to their needs and to treat the catalog as an intermediate data product~\citep{turon2010telescopes}.
While some work would be required to modify our model, the techniques we illustrate in this paper could still be followed to perform inference.
The MCMC procedure makes it particularly straightforward to make changes.

Because astronomical surveys are large (comprising terabytes of data now, and petabytes in the near future), scalability is of paramount concern.
We approximated the posterior for a large image dataset and demonstrated the scaling characteristics necessary to apply approximate Bayesian inference to hundreds of petabytes of images from the next generation of astronomical surveys.
Our optimization procedure found a stationary point, even though doing so required treating the full dataset as a single optimization problem.

Because of the relative ease of deriving and implementing MCMC, it could be a useful tool for trying different models and testing for misspecification prior to implementing VI. In some cases, it may be simpler to expend more computational resources to scale up the MCMC procedure than to implement VI. For the most computationally intensive problems, however, only VI can currently perform approximate inference.

\section{Conclusion}

Based on our findings, most academic papers are inaccessible and significant challenges remain for BLV researchers when interacting with and reading these papers. Though some improvements in accessibility have been seen over time, these changes may not be reflective of author actions directly. In the meantime, we offer a potential solution for the millions of PDFs that have already been published and which still remain the dominant form of distribution for academic papers. We introduce the \scially system for rendering PDFs as accessible HTML documents. The system extracts the content of PDFs, tagging headings and objects and inferring reading order, which results in a more navigable and accessible document. Though the extraction pipeline is imperfect and can result in errors, our evaluation suggests that for the majority of papers, the resulting HTML render has no major problems that impact readability. We confirm these findings in our user study, where all users responded positively to the prototype system, claiming that they would be likely or very likely to use the system were it to be available in the future. Participants described the system as likely to ``become the workflow'' or ``life-changing,'' indicating both a strong favorable response and particular need for these types of solutions.

We do not claim that \scially solves all (or even close to all) accessibility problems for BLV researchers, but it is a step in the right direction.  \scially is a technological solution that can mitigate many of the challenges experienced by BLV researchers at this moment. Though a longer term solution would surely require more dialogue between all stakeholders and a potential revolution in the way in which scholars publish and distribute their research findings, we encourage researchers to prioritize and address these challenges with whatever tools they have in their toolbox right now. We especially encourage others to take into account our findings on the needs and challenges of BLV researchers when designing and engineering new systems and tools for reading the scholarly literature.

% %%
% %% The acknowledgments section is defined using the "acks" environment
% %% (and NOT an unnumbered section). This ensures the proper
% %% identification of the section in the article metadata, and the
% %% consistent spelling of the heading.
\begin{acks}
This work was supported in part by ONR grant N00014-18-1-2193, NSF RAPID grant 2040196, and the University of Washington WRF/Cable Professorship. We thank Jeff Bigham, Leah Findlater, Jon Froehlich, and Venkatesh Potluri for their valuable feedback on study design and recruitment. We thank Oren Etzioni and Doug Raymond for valuable feedback on the project. We thank Bryan Newbold for providing feedback on earlier drafts of the manuscript. We thank Sam Skjonsberg for help with the demo, and Michal Guerquin and Michael Schmitz for feedback on demo deployment. We thank the Semantic Scholar team for assisting with data access and system infrastructure. Finally, we thank the users who participated in our study, who offered invaluable feedback and suggestions.
\end{acks}

% %%
% %% The next two lines define the bibliography style to be used, and
% %% the bibliography file.
\bibliographystyle{ACM-Reference-Format}
\bibliography{a11y}

\newpage
\begin{appendices}

\section{Simple versions of the algorithms}
\label{appendix:simple_algs}

\begin{algorithm}
\caption{Online algorithm}\label{online_simple}
\begin{algorithmic}
\State Initialize \textsc{Student} learning algorithm
\State Initialize expected return $Q(a)=0$ for all $N$ tasks
\For{t=1,\ldots,T}
\State Choose task $a_t$ based on $|Q|$ using $\epsilon$-greedy or Boltzmann policy
\State Train \textsc{Student} using task $a_t$ and observe reward $r_t = x_t^{(a_t)} - x_{t'}^{(a_t)}$
\State Update expected return $Q(a_t) = \alpha r_t + (1 - \alpha) Q(a_t)$
\EndFor
\end{algorithmic}
\end{algorithm}

\begin{algorithm}
\caption{Naive algorithm}\label{naive_simple}
\begin{algorithmic}
\State Initialize \textsc{Student} learning algorithm
\State Initialize expected return $Q(a)=0$ for all $N$ tasks
\For{t=1,...,T}
\State Choose task $a_t$ based on $|Q|$ using $\epsilon$-greedy or Boltzmann policy
\State Reset $D=\emptyset$
\For{k=1,...,K}
\State Train \textsc{Student} using task $a_t$ and observe score $o_t = x_t^{(a_t)}$
\State Store score $o_t$ in list $D$
\EndFor
\State Apply linear regression to $D$ and extract the coefficient as $r_t$
\State Update expected return $Q(a_t) = \alpha r_t + (1 - \alpha) Q(a_t)$
\EndFor
\end{algorithmic}
\end{algorithm}

\begin{algorithm}
\caption{Window algorithm}\label{window_simple}
\begin{algorithmic}
\State Initialize \textsc{Student} learning algorithm
\State Initialize FIFO buffers $D(a)$ and $E(a)$ with length $K$ for all $N$ tasks
\State Initialize expected return $Q(a)=0$ for all $N$ tasks
\For{t=1,\ldots,T}
\State Choose task $a_t$ based on $|Q|$ using $\epsilon$-greedy or Boltzmann policy
\State Train \textsc{Student} using task $a_t$ and observe score $o_t = x_t^{(a_t)}$
\State Store score $o_t$ in $D(a_t)$ and timestep $t$ in $E(a_t)$
\State Use linear regression to predict $D(a_t)$ from $E(a_t)$ and use the coef. as $r_t$
%\State Update expected return $Q(a_t) := r_t$
\State Update expected return $Q(a_t) = \alpha r_t + (1 - \alpha) Q(a_t)$
\EndFor
\end{algorithmic}
\end{algorithm}

\begin{algorithm}
\caption{Sampling algorithm}\label{sampling_simple}
\begin{algorithmic}
\State Initialize \textsc{Student} learning algorithm
\State Initialize FIFO buffers $D(a)$ with length $K$ for all $N$ tasks
\For{t=1,\ldots,T}
\State Sample reward $\tilde{r}_a$ from $D(a)$ for each task (if $|D(a)|=0$ then $\tilde{r}_a=1$)
\State Choose task $a_t = \argmax_a |\tilde{r}_a|$
\State Train \textsc{Student} using task $a_t$ and observe reward $r_t = x_t^{(a_t)} - x_{t'}^{(a_t)}$
\State Store reward $r_t$ in $D(a_t)$
\EndFor
\end{algorithmic}
\end{algorithm}

\newpage
\section{Batch versions of the algorithms}
\label{appendix:batch_algs}

\begin{algorithm}
\caption{Online algorithm}\label{online_batch}
\begin{algorithmic}
\State Initialize \textsc{Student} learning algorithm
\State Initialize expected return $Q(a)=0$ for all $N$ tasks
\For{t=1,\ldots,T}
\State Create prob. dist. $\vec{a_t}=(p_t^{(1)}, ..., p_t^{(N)})$ based on $|Q|$ using $\epsilon$-greedy or Boltzmann policy
\State Train \textsc{Student} using prob. dist. $\vec{a_t}$ and observe scores $\vec{o_t} = (x_t^{(1)}, ..., x_t^{(N)})$
\State Calculate score changes $\vec{r_t} = \vec{o_t} - \vec{o_{t-1}}$
%\State Calculate score change $\hat{r}_t = o_t - o_{t-1}$
%\State Calculate corrected reward $r_t = \hat{r}_t / a_t$ ($a_t$ is prob. dist.)
\State Update expected return $\vec{Q} = \alpha \vec{r_t} + (1 - \alpha) \vec{Q}$
\EndFor
\end{algorithmic}
\end{algorithm}

\begin{algorithm}
\caption{Naive algorithm}\label{online_naive}
\begin{algorithmic}
\State Initialize \textsc{Student} learning algorithm
\State Initialize expected return $Q(a)=0$ for all $N$ tasks
\For{t=1,\ldots,T}
\State Create prob. dist. $\vec{a_t}=(p_t^{(1)}, ..., p_t^{(N)})$ based on $|Q|$ using $\epsilon$-greedy or Boltzmann policy
\State Reset $D(a)=\emptyset$ for all tasks
\For{k=1,\ldots,K}
\State Train \textsc{Student} using prob. dist. $\vec{a_t}$ and observe scores $\vec{o_t} = (x_t^{(1)}, ..., x_t^{(N)})$
\State Store score $o_t^{(a)}$ in list $D(a)$ for each task $a$
\EndFor
\State Apply linear regression to each $D(a)$ and extract the coefficients as vector $\vec{r_t}$
%\State Apply linear regression to each $D(a)$ and extract the coefficients as $\hat{r}_t$
%\State Calculate corrected rewards $r_t = \hat{r}_t / a_t$ ($a_t$ is prob. dist.)
\State Update expected return $\vec{Q} = \alpha \vec{r_t} + (1 - \alpha) \vec{Q}$
\EndFor
\end{algorithmic}
\end{algorithm}

\begin{algorithm}
\caption{Window algorithm}\label{online_window}
\begin{algorithmic}
\State Initialize \textsc{Student} learning algorithm
\State Initialize FIFO buffers $D(a)$ with length $K$ for all $N$ tasks
\State Initialize expected return $Q(a)=0$ for all $N$ tasks
\For{t=1,\ldots,T}
\State Create prob. dist. $\vec{a_t}=(p_t^{(1)}, ..., p_t^{(N)})$ based on $|Q|$ using $\epsilon$-greedy or Boltzmann policy
\State Train \textsc{Student} using prob. dist. $\vec{a_t}$ and observe scores $\vec{o_t} = (x_t^{(1)}, ..., x_t^{(N)})$
\State Store score $o_t^{(a)}$ in $D(a)$ for all tasks $a$
\State Apply linear regression to each $D(a)$ and extract the coefficients as vector $\vec{r_t}$
%\State Apply linear regression to each $D(a)$ and extract the coefficients as $\hat{r}_t$
%\State Calculate corrected rewards $r_t = \hat{r}_t / a_t$ ($a_t$ is prob. dist.)
\State Update expected return $\vec{Q} = \alpha \vec{r_t} + (1 - \alpha) \vec{Q}$
%\State Update expected return $Q = r_t$
\EndFor
\end{algorithmic}
\end{algorithm}

\begin{algorithm}
\caption{Sampling algorithm}\label{online_sampling}
\begin{algorithmic}
\State Initialize \textsc{Student} learning algorithm
\State Initialize FIFO buffers $D(a)$ with length $K$ for all $N$ tasks
\For{t=1,\ldots,T}
\State Sample reward $\tilde{r}_a$ from $D(a)$ for each task (if $|D(a)|=0$ then $\tilde{r}_a=1$)
\State Create one-hot prob. dist. $\vec{\tilde{a}_t}=(p_t^{(1)}, ..., p_t^{(N)})$ based on $\argmax\nolimits_a |\tilde{r}_a|$
\State Mix in uniform dist. : $\vec{a_t} = (1 - \epsilon) \vec{\tilde{a}_t} + \epsilon/N$
\State Train \textsc{Student} using prob. dist. $\vec{a_t}$ and observe scores $\vec{o_t} = (x_t^{(1)}, ..., x_t^{(N)})$
\State Calculate score changes $\vec{r_t} = \vec{o_t} - \vec{o_{t-1}}$
%\State Calculate score change $\hat{r}_t = o_t - o_{t-1}$
%\State Calculate corrected rewards $r_t = \hat{r}_t / a_t$ ($a_t$ is prob. dist.)
\State Store reward $r_t^{(a)}$ in $D(a)$ for each task $a$
\EndFor
\end{algorithmic}
\end{algorithm}

\clearpage
\section{Decimal Number Addition Training Details}
\label{appendix:addition}

Our reimplementation of decimal addition is based on Keras \citep{chollet2015keras}. The encoder and decoder are both LSTMs with 128 units. In contrast to the original implementation, the hidden state is not passed from encoder to decoder, instead the last output of the encoder is provided to all inputs of the decoder. One curriculum training step consists of training on 40,960 samples. Validation set consists of 4,096 samples and 4,096 is also the batch size. Adam optimizer \citep{kingma2014adam} is used for training with default learning rate of 0.001. Both input and output are padded to a fixed size.

In the experiments we used the number of steps until 99\% validation set accuracy is reached as a comparison metric. The exploration coefficient $\epsilon$ was fixed to 0.1, the temperature $\tau$ was fixed to 0.0004, the learning rate $\alpha$ was 0.1, and the window size $K$ was 10 in all experiments.
 
\section{Minecraft Training Details}
\label{appendix:minecraft}

The Minecraft task consisted of navigating through randomly generated mazes. The maze ends with a target block and the agent gets 1,000 points by touching it. Each move costs -0.1 and dying in lava or getting a timeout yields -1,000 points. Timeout is 30 seconds (1,500 steps) in the first task and 45 seconds (2,250 steps) in the subsequent tasks.

For learning we used the \textit{proximal policy optimization} (PPO) algorithm \citep{schulman2017proximal} implemented using Keras \citep{chollet2015keras} and optimized for real-time environments. The policy network used four convolutional layers and one LSTM layer. Input to the network was $40\times 30$ color image and outputs were two Gaussian actions: move forward/backward and turn left/right. In addition the policy network had state value output, which was used as the baseline. Figure \ref{f14} shows the network architecture.

\begin{figure}[h]
  \includegraphics[scale=0.4]{figures/minecraft_network}
\caption{Network architecture used for Minecraft.}
\label{f14}
\end{figure}

For training we used a setup with 10 parallel Minecraft instances. The agent code was separated into runners, that interact with the environment, and a trainer, that performs batch training on GPU, similar to \cite{babaeizadeh2016reinforcement}. Runners regularly update their snapshot of the current policy weights, but they only perform prediction (forward pass), never training. After a fixed number of steps they use FIFO buffers to send collected states, actions and rewards to the trainer. Trainer collects those experiences from all runners, assembles them into batches and performs training. FIFO buffers shield the runners and the trainer from occasional hiccups. This also means that the trainer is not completely on-policy, but this problem is handled by the importance sampling in PPO.

\begin{figure}[h]
  \includegraphics[scale=0.4]{figures/minecraft_training}
\caption{Training scheme used for Minecraft.}
\label{f14}
\end{figure}

During training we also used frame skipping, i.e. processed only every 5th frame. This sped up the learning considerably and the resulting policy also worked without frame skip. Also, we used auxiliary loss for predicting the depth as suggested in \citep{mirowski2016learning}. Surprisingly this resulted only in minor improvements.

For automatic curriculum learning we only implemented the Window algorithm for the Minecraft task, because other algorithms rely on score change, which is not straightforward to calculate for parallel training scheme. Window size was defined in timesteps and fixed to 10,000 in the experiments, exploration rate was set to 0.1.

The idea of the first task in the curriculum was to make the agent associate the target with a reward. In practice this task proved to be too simple - the agent could achieve almost the same reward by doing backwards circles in the room. For this reason we added penalty for moving backwards to the policy loss function. This fixed the problem in most cases, but we occasionally still had to discard some unsuccessful runs. Results only reflect the successful runs.

We also had some preliminary success combining continuous (Gaussian) actions with binary (Bernoulli) actions for "jump" and "use" controls, as shown on figure \ref{f14}. This allowed the agent to learn to cope also with rooms that involve doors, switches or jumping obstacles, see \url{https://youtu.be/e1oKiPlAv74}.

\end{appendices}

\end{document}
\endinput
%%
%% End of file `sample-authordraft.tex'.
