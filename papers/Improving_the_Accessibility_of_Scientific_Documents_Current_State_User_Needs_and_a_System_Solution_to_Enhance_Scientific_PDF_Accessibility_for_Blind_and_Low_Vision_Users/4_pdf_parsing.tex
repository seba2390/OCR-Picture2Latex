\section{Converting PDF to HTML: The \scially Pipeline}
\label{sec:pdf2html}

To address the broad accessibility challenges described in Section \ref{sec:sos}, we propose and prototype a system for extracting semantic content from paper PDFs and re-rendering this content as accessible HTML. HTML is widely accepted as a more accessible document format than PDFs. In the 2019 Access SIGCHI Report, the authors discuss the reasoning behind switching CHI publications to a new HTML5 proceedings format to improve accessibility \citep{Mankoff2019SIGCHI}. 
%\jonathan{CHI 2019 adopted a new HTML 5 proceedings format according to \url{https://dl.acm.org/doi/abs/10.1145/3386280.3386287}}. 
By rendering the content of paper PDFs as HTML, and introducing proper reading order and accessibility features such as section headings, links, and figure tags, we can offset many of the issues of reading from an inaccessible PDF. Our PDF to HTML rendering system is named \scially after the community-adopted numeronym for digital accessibility.\footnote{\href{https://www.a11yproject.com/}{https://www.a11yproject.com/}} 

Figure~\ref{fig:pipeline} provides a schematic for the approach. \scially leverages the two open source PDF processing projects S2ORC \citep{lo-wang-2020-s2orc} and DeepFigures \citep{Siegel2018ExtractingSF}, the Semantic Scholar API,\footnote{\href{https://api.semanticscholar.org/}{https://api.semanticscholar.org/}} and a custom Flask application for rendering the extracted content of the PDF as HTML. The S2ORC project \citep{lo-wang-2020-s2orc} integrates the Grobid machine learning library \citep{Lopez2015GROBIDI} and a custom XML to JSON parser\footnote{Available at \href{https://github.com/allenai/s2orc-doc2json}{https://github.com/allenai/s2orc-doc2json}} to produce a structured representation of paper text. We use a version of the S2ORC pipeline that is based on Grobid v0.6.0. The resulting JSON representation includes metadata fields like title, authors, and affiliations, and paper content fields such as abstract, section headers, body text organized into paragraphs, bibliography entries, and figure and table objects (though not the figure images themselves). The output also contains links between inline citations and figure/table references respectively to bibliography entries and figure/table objects. DeepFigures \citep{Siegel2018ExtractingSF}, on the other hand, leverages a computer vision model to extract images of figures and tables as well as their corresponding captions from the source PDF. 

The outputs of S2ORC and DeepFigures are stitched together to form the HTML render as in Figure~\ref{fig:pipeline}. We place header tags (\texttt{<h1>...</h1>}, \texttt{<h2>...</h2>}) around the title, authors, abstract, section headings, and reference heading. Paragraphs of body text are enclosed in \texttt{<p>...</p>} tags in order within their appropriate sections. Bibliography entries are provided in an unordered list under the reference heading. Figures and tables are enclosed in \texttt{<figure>...} \texttt{</figure>} tags and placement is inferred based on mentions in the text. A figure or table is placed immediately after the paragraph in which its handle is first mentioned (e.g. ``In \underline{Fig. 1}, we show...'' is the first mention of Figure 1 and the figure is placed directly after the paragraph with this mention). Figure and table captions are attached to their corresponding image objects, so that correspondences between the caption text and image are made explicit (in PDFs, this is usually not the case). Any figures or tables which are not mentioned in order in the text are placed in order nonetheless; in other words, if paragraph 1 mentions Figure 1 and paragraph 2 mentions Figure 3, both Figure 1 and 2 will be placed directly following paragraph 1 and Figure 3 following paragraph 2. This ensures that the layout for the HTML render closely approximates the intended reading order. We justify this decision based on user feedback from our pilot study, which is discussed in Section~\ref{sec:user_study}. 

In some cases, we are able to successfully process a PDF through S2ORC to extract textual content but DeepFigures either fails to process the PDF or fails to extract some or all figures from the PDF. To mitigate the cognitive dissonance around figure or table mentions without corresponding figure or table objects, we insert placeholder objects into the HTML render as in Figure~\ref{fig:figure_equations}. For example, if ``Figure 2'' is mentioned in the text but is not successfully extracted by DeepFigures, we would insert a placeholder image for the figure based on the logic described in the previous paragraph along with the text ``Figure 2. Not extracted; please refer to original document.'' Similarly, mathematical equations that we cannot currently extract are acknowledged with the same placeholder text.

\begin{figure}
    \centering
    \includegraphics[width=0.4\textwidth]{figures/fig_success_1882932.png}\includegraphics[width=0.4\textwidth]{figures/placeholder.png}
    \includegraphics[width=0.6\textwidth]{figures/eq_fail.png}
    \caption{A successfully extracted figure from \citet{Nascimento2005VertexCA} is shown with its corresponding figure caption (\textit{top left}). When figures are not extracted and inferred to exist (handle mentioned in text or number between two extracted figures), a placeholder image is shown along with a message referencing the failed extraction (\textit{top right}). Similarly, when an equation is detected to be present in the PDF and not extracted, we insert text signaling the failed extraction and refer the user to the source document (\textit{bottom}).}
    \label{fig:figure_equations}
    \Description{Three subfigures show figure and equation related features in SciA11y. First subfigure shows extracted images from Nascimento and Bioucas-Dias and the associated caption of the figure, as displayed in an HTML <figure> block. Second subfigure shows a placeholder image for a figure that was not successfully extracted by the system. Third subfigure shows the text used as a placeholder for equations that are not extracted. Both the not extracted figure and not extracted equation are followed by the text: "Not extracted; please refer to original document."}
\end{figure}

We add links between inline citations and the corresponding reference entry where possible. We insert links at each inline citation in the body text that link to the corresponding bibliography entry. Following each bibliography entry, we provide links back to the first mention of that entry in each section of the paper in which it was mentioned. For example, if bibliography entry \texttt{[1]} is cited in the ``II. Related Works'' section and the ``III. Methods'' section, we provide two links following the entry in the bibliography to the corresponding citation locations in sections II and III, as in:

\begin{alltt}
    [1] Last name et al. Paper title. Venue. DOI.
        \underline{Link to return to Section II}, \underline{Link to return to Section III}
\end{alltt}

\noindent This allows users to navigate back to their reading location in the document after clicking through to a bibliography entry. A user may otherwise hesitate to resolve a link, because it may result in losing their place and train of thought. Finally, we introduce a table of contents near the beginning of the HTML render to facilitate better understanding of overall document structure. The table of contents includes all section titles, linked to the corresponding sections, as well as figures and tables nested under their respective section headers. The table of contents provides a rapid overview of the structure of the document, and facilitates rapid navigation to the reader's desired sections.

In the current iteration of the HTML render, we do not display author affiliations, footnotes, or mathematical equations due to the difficulty of extracting these pieces of information from the PDF. Though some of the elements are extracted in S2ORC, the overall quality of the extractions for these elements is lower, and is currently insufficient for surfacing in the prototype (see Section~\ref{sec:evaluation} for details). Future work includes investigating the possibility of extracting and exposing these elements, either by improving current models or training new models targeted towards the extraction of specific paper elements.

We leverage the feedback we received during our pilot studies (see Section~\ref{sec:user_study}) to make improvements prior to the main user study. We denote the versions of the prototype as v0.1 (initial version; version seen by P1), v0.2 (version seen by P2), and v0.3 (version seen by all other participants in the main user study). Features implemented in v0.1 include the primary components of the HTML render such as title, authors, abstract, body text with section headers, figures and tables, references, and links between inline citations and references. In v0.1, figures and tables were placed in a separate section following the main body of the paper. Following P1, for version v0.2, we implemented the table of contents, inserted placeholders for objects that we could not extract, and began inserting figures and tables into the body text adjacent to their first mentions. This last change was made in response to P1's feedback that navigating away to figures caused him to lose his reading location. Following P2, for version v0.3, we implemented only minor changes. P2 signaled during his session that URLs in the bibliography were not being correctly extracted, so we patched the data to correctly extract and display URLs in bibliography entries.

Based on our evaluation of the quality of these HTML renders (Section~\ref{sec:evaluation}) and user feedback and response (Section~\ref{sec:user_study}), we believe our approach can dramatically increase the screen reader navigability and accessibility of scientific papers across all disciplines by providing an alternate and more accessible HTML version of these papers. Properly tagged section headings allow for quick navigation and skimming of a paper, links between inline citations and bibliography entries allow users to browse to cited papers without losing their place, and figure tags for figure and table objects allow for direct navigation to these in-paper objects. We now discuss the quality of our PDF extractions (Section~\ref{sec:evaluation}) and user response to the prototype (Section~\ref{sec:user_study}) in detail.