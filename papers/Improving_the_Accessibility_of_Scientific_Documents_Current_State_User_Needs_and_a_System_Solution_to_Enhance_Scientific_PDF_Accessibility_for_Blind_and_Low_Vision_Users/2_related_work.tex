\section{Related work}
\label{sec:related_work}

Accessibility is an essential component of computing, which aims to make technology broadly accessible to as many users as possible, including those with differing sets of abilities. Improvements in usability and accessibility falls to the community, to better understand the needs of users with differing abilities, and to design technologies that play to this spectrum of abilities \citep{Wobbrock2011AbilityBasedDC}.
In computing, significant strides have been made to increase the accessibility of web content. For example, various versions of the Web Content Accessibility Guidelines (WCAG) \citep{Chisholm2001WebCA, Caldwell2008WebCA} and the in-progress working draft for WCAG 3.0,\footnote{\href{https://www.w3.org/TR/wcag-3.0/}{https://www.w3.org/TR/wcag-3.0/}} or standards such as ARIA from the W3C's Web Accessibility Initiative (WAI)\footnote{\href{https://www.w3.org/WAI/standards-guidelines/aria/}{https://www.w3.org/WAI/standards-guidelines/aria/}} have been released and used to guide web accessibility design and implementation. Similarly, positive steps have been made to improve the accessibility of user interfaces and user experience \citep{Peissner2012MyUIGA, Peissner2013UserCI, Thompson2014ImprovingTU, Bigham2014MakingTW}, as well as various types of media content \citep{Mirri2017TowardsAG, Nengroo2017AccessibleI, Gleason2020TwitterAA}. 

We take inspiration from accessibility design principles in our effort to make research publications more accessible to users who are blind and low vision. Blindness and low vision are some of the most common forms of disability, affecting an estimated 3--10\% of Americans depending on how visual impairment is defined \citep{CDCVisionLossBurden}. BLV researchers also make up a representative sample of researchers in the United States and worldwide. A recent Nature editorial pushes the scientific community to better support researchers with visual impairments \citep{NatureCareerColumn2020}, since existing tools and resources can be limited. There are many inherent accessibility challenges to performing research. In this paper, we engage with one of these challenges that affects all domains of study, accessing and reading the content of academic publications. 

BLV users interact with papers using screen readers, braille displays, text-to-speech, and other assistive tools. A WebAIM survey of screen reader users found that the vast majority (75.1\%) of respondents indicate that PDF documents are very or somewhat likely to pose significant accessibility issues.\footnote{\href{https://webaim.org/projects/screenreadersurvey8/}{https://webaim.org/projects/screenreadersurvey8/}} Most paper are published in PDF, which is inherently inaccessible, due in large part to its conflation of visual layout information with semantic content \citep{NielsenPDFStillUnfit, Bigham2016AnUT}. 
\citet{Bigham2016AnUT} describe the historical reasons we use PDF as the standard document format for scientific publications, as well as the barriers the format itself presents to accessibility. Prior work on scientific accessibility have made recommendations for how to make PDFs more accessible \cite{Rajkumar2020PDFAO, Darvishy2018PDFAT}, including greater awareness for what constitutes an accessible PDF and better tooling for generating accessible PDFs. Some work has focused on addressing components of paper accessibility, such as the correct way for screen readers to interpret and read mathematical equations \citep{Flores2010MathMLTA, Bates2010SpokenMU, Sorge2014TowardsMM, Mackowski2017MultimediaPF, Ahmetovic2018AxessibilityAL, Ferreira2004EnhancingTA, Sojka2013AccessibilityII}, describe charts and figures \citep{Elzer2008AccessibleBC, Engel2017TowardsAC, Engel2019SVGPlottAA}, automatically generate figure captions \citep{Chen2019NeuralCG, Qian2020AFS}, or automatically classify the content of figures \citep{Kim2018MultimodalDL}. Other work applicable to all types of PDF documents aims to improve automatic text and layout detection of scanned documents \cite{Nazemi2014PracticalSM} and extract table content \cite{Fan2015TableRD, Rastan2019TEXUSAU}. In this work, we focus on the issue of representing overall document structure, and navigation within that structure. Being able to quickly navigate the contents of a paper through skimming and scanning is an essential reading technique \citep{Maxwell1972SkimmingAS}, which is currently under-supported by PDF documents and PDF readers when reading these documents by screen reader. 

There also exists a variety of automatic and manual tools that assess and fix accessibility compliance issues in PDFs, including the Adobe Acrobat Pro Accessibility Checker\footnote{\href{https://www.adobe.com/accessibility/products/acrobat/using-acrobat-pro-accessibility-checker.html}{https://www.adobe.com/accessibility/products/acrobat/using-acrobat-pro-accessibility-checker.html}}, Common Look\footnote{\href{https://monsido.com/monsido-commonlook-partnership}{https://monsido.com/monsido-commonlook-partnership}}, ABBYY FineReader\footnote{\href{https://pdf.abbyy.com/}{https://pdf.abbyy.com/}}, PAVE\footnote{\href{https://pave-pdf.org/faq.html}{https://pave-pdf.org/faq.html}}, and PDFA Inspector\footnote{\href{https://github.com/pdfae/PDFAInspector}{https://github.com/pdfae/PDFAInspector}}. To our knowledge, PAVE and PDFA Inspector are the only non-proprietary, open-source tools for this purpose. Based on our experiences, however, all of these tools require some degree of human intervention to properly tag a scientific document, and tagging and fixing must be performed for each new version of a PDF, regardless of how minor the change may be.

Guidelines and policy changes have been introduced in the past decade to ameliorate some of the issues around scientific PDF accessibility. Some conferences, such as The ACM CHI Virtual Conference on Human Factors in Computing Systems (CHI) and The ACM SIGACCESS Conference on Computers and Accessibility (ASSETS), have released guidelines for creating accessible submissions.\footnote{See \href{http://chi2019.acm.org/authors/papers/guide-to-an-accessible-submission/}{http://chi2019.acm.org/authors/papers/guide-to-an-accessible-submission/} and \href{https://assets19.sigaccess.org/creating_accessible_pdfs.html}{https://assets19.sigaccess.org/creating\_accessible\_pdfs.html}} The ACM Digital Library\footnote{\href{https://dl.acm.org/}{https://dl.acm.org/}} provides some publications in HTML format, which is easier to make accessible than PDF~\cite{Graells2007EstudioDL}. \citet{Ribera2019PublishingAP} conducted a case study on DSAI 2016 (Software Development and Technologies for Enhancing Accessibility and Fighting Infoexclusion). The authors of DSAI were responsible for creating accessible proceedings and identified barriers to creating accessible proceedings, including lack of sufficient tooling and lack of awareness of accessibility. The authors recommended creating a new role in the organizing committee dedicated to accessible publishing. These policy changes have led to improvements in localized communities, but have not been widely adopted by all academic publishers and conference organizers.

Table~\ref{tab:prior_work} lists prior studies that have analyzed PDF accessibility of academic papers, and shows how our study compares. Prior work has primarily focused on papers published in Human-Computer Interaction and related fields, specific to certain publication venues, while our analysis tries to quantify paper accessibility more broadly.
\citet{Brady2015CreatingAP} quantified the accessibility of 1,811 papers from CHI 2010-2016, ASSETS 2014, and W4A, assessing the presence of document tags, headers, and language. They found that compliance improved over time as a response to conference organizers offering to make papers accessible as a service to any author upon request. \citet{Lazar2017MakingTF} conducted a study quantifying accessibility compliance at CHI from 2010 to 2016 as well as ASSETS 2015,
%\jb{Define acronyms in prev para}
confirming the results of \citet{Brady2015CreatingAP}. They found that across 5 accessibility criteria, the rate of compliance was less than 30\% for CHI papers in each of the 7 years that were studied. The study also analyzed papers from ASSETS 2015, an ACM conference explicitly focused on accessibility, and found that those papers had significantly higher rates of compliance, with over 90\% of the papers being tagged for correct reading order and no criteria having less than 50\% compliance. This finding indicates that community buy-in is an important contributor to paper accessibility.
\citet{Nganji2015ThePD} conducted a study of 200 PDFs of papers published in four disability studies journals, finding that accessibility compliance was between 15-30\% for the four journals analyzed, with some publishers having higher adherence than others. To date, no large scale analysis of scientific PDF accessibility has been conducted outside of disability studies and HCI, due in part to the challenge of scaling such an analysis. We believe such an analysis is useful for establishing a baseline and characterizing routes for future improvement. Consequently, as part of this work, we conduct an analysis of scientific PDF accessibility across various fields of study, and report our findings relative to prior work. 


\begin{table}[t!]
\small
    \centering
    \begin{tabularx}{\linewidth}{L{22mm}L{15mm}L{48mm}L{16mm}L{34mm}}
        \toprule
        \textbf{Prior work} & \textbf{PDFs analyzed} & \textbf{Venues} & \textbf{Year} & \textbf{Accessibility checker} \\
        \midrule
        \citet{Brady2015CreatingAP} & 1811 & CHI, ASSETS and W4A & 2011--2014 & PDFA Inspector \\ [0.5mm]
        \hline \\ [-2.5mm]
        \citet{Lazar2017MakingTF} & 465 + 32 & CHI and ASSETS & 2014--2015 & Adobe Acrobat Action Wizard \\ [0.5mm]
        \hline \\ [-2.5mm]
        \citet{Ribera2019PublishingAP} & 59 & DSAI & 2016 & Adobe PDF Accessibility Checker 2.0 \\ [0.5mm]
        \hline \\ [-2.5mm]
        \citet{Nganji2015ThePD} & 200 & \textit{Disability \& Society}, \textit{Journal of Developmental and Physical Disabilities}, \textit{Journal of Learning Disabilities}, and \textit{Research in Developmental Disabilities} & 2009--2013 & Adobe PDF Accessibility Checker 1.3 \\ [0.6mm]
        \hline \\ [-2.5mm]
        \textbf{\textit{Our analysis}} & \numpdfs & Venues across various fields of study & 2010--2019 & Adobe Acrobat Accessibility Plug-in Version 21.001.20145 \\
        \bottomrule
    \end{tabularx}
    \caption{Prior work has investigated PDF accessibility for papers published in specific venues such as CHI, ASSETS, W4A, DSAI, or various disability journals. Several of these works were conducted manually, and were limited to a small number of papers, while the more thorough analysis was conducted for CHI and ASSETS, two conference venues focused on accessibility and HCI. Our study expands on this prior work to investigate accessibility over \numpdfs PDFs sampled from across different fields of study.
    }
    % \Description{
    % Prior work, PDFs analyzed, Venues, Year, Accessibility checker 
    % Brady et al. [7], 1811, CHI, ASSETS and W4A, 2011--2014, PDFA Inspector 
    % Lazar et al. [23], 465 + 32, CHI and ASSETS, 2014--2015, Adobe Acrobat Action Wizard 
    % Ribera et al. [40], 59, DSAI, 2016, Adobe PDF Accessibility Checker 2.0 
    % Nganji [33], 200, Disability & Society, Journal of Developmental and Physical Disabilities, Journal of Learning Disabilities, and Research in Developmental Disabilities, 2009--2013, Adobe PDF Accessibility Checker 1.3
    % Our analysis, 11397, Venues across various fields of study, 2010--2019, Adobe Acrobat Accessibility Plug-in Version 21.001.20145 
    % }
    \label{tab:prior_work}
\end{table}