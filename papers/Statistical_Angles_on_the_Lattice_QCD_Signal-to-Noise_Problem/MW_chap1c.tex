\section{Meson and Baryon Correlation Functions}\label{sec:corr}

The formal construction of LQCD sketched in Sec.~\ref{sec:lattice} provides a finite-dimensional path integral representation of the QCD partition function
\begin{equation}
  \begin{split}
    Z_{QCD} &= \int \mathcal{D}U\mathcal{D}\bar{q}\mathcal{D}q e^{-S_G(U) - \sum_x \bar{q}(x)D(U;x,x)q(x)} \\
    &= \int \mathcal{D}U e^{-S_G(U)}\det[D(U)].
  \end{split}\label{eq:qcdZ}
\end{equation}
The free energy of the QCD vacuum can be defined straightforwardly as $F_{QCD} = -\partial_\beta \ln Z_{QCD}$, and in principle computed by performing the integral over gauge field configurations.
As discussed above, high-dimensional integrals such as QFT path integrals in large spacetime volumes are most efficiently performed using Monte Carlo techniques where the error after sampling $N$ field configurations decreases as $1/\sqrt{N}$ regardless of the dimensionality of the integral.
Monte Carlo integration relies on sampling the integrand as a probability distribution.
The Wilson Dirac operator, as well as other variants including the clover-improved Dirac operator, obeys the $\gamma_5$-Hermiticity property
\begin{equation}
  \begin{split}
    D(U;x,y) = \gamma_5 D(U;y,x)^\dagger \gamma_5,
  \end{split}\label{eq:g5Hermiticity}
\end{equation}
which guarantees that the determinant inside the path integral weight is real in the presence of an arbitrary gauge field configuration $U$
\begin{equation}
  \begin{split}
    \det[D(U)] = \det[\gamma_5 D(U)^\dagger \gamma_5] = \det[\gamma_5]^2 \det[D(U)^\dagger] = \det[D(U)]^*. 
  \end{split}\label{eq:positivity}
\end{equation}
In the continuum, chiral symmetry guarantees that $\fs{D}(U)$ has eigenstates of opposite chirality with complex conjugate eigenvalues, and therefore that $\det[D(U)]$ is a positive definite product of paired eigenvalues,~\cite{Kaplan:2009yg}
\begin{equation}
  \begin{split}
    \mathcal{P}(U) = e^{-S_G(U)}\det[D(U)] \geq 0.
  \end{split}\label{eq:probdef}
\end{equation}
It is therefore possible to interpret $\mathcal{P}(U)$ as a probability distribution.
At finite lattice spacing, there is no exact chiral symmetry and configurations with $\det D < 0$ are possible.
In $N_f = 2$, that is calculations of two degenerate flavors, the determinant factorizes into a product of equal determinants for the two flavors and is positive semi-definite.
Heavier quarks are less likely to see fluctuations of negative $\det D$, and for the physical strange quark the average determinant phase factor $\avg{\det D/|\det D|}$ is close to one~\cite{Beane:2010em}.
Reweighting and other methods have allowed determinants of light, non-degenerate quarks to be effectively included in Monte Carlo sampling techniques, see Refs.~\cite{Hasenfratz:2008fg,Finkenrath:2013soa} for further discussions.
With a non-negative $\mathcal{P}(U)$ with the same continuum limit as Eq.~\eqref{eq:probdef} constructed, averages of generic observables $\mathcal{O}$ can be identified with the sample means of functions of an ensemble of $N$ random gauge fields $U_i$ distributed according to $\mathcal{P}(U)$,
\begin{equation}
  \begin{split}
    \avg{\mathcal{O}(U)} &= \int \mathcal{D}U \;e^{-S_G(U)}\det[D(U)]\mathcal{O}(U) \\
    &= \int \mathcal{D}U \;\mathcal{P}(U)\;\mathcal{O}(U) \\
    & = \frac{1}{N}\sum_{i=1}^N \mathcal{O}_i + O(N^{-1/2}),
  \end{split}\label{eq:MC}
\end{equation}
where $\mathcal{O}_i = \mathcal{O}(U_i)$ represents the observable calculated in the presence of gauge field configuration $U_i$.


Higher moments of observables dictate statistical properties of Monte Carlo calculations such as their variance.
Parisi first noted that it can be helpful to analyze these moments from a QFT point of view in order to understand the statistical variation of Monte Carlo results~\cite{Parisi:1983ae}.
The variance corresponding to the observable in Eq.~\eqref{eq:MC}, for example, is
\begin{equation}
  \begin{split}
    \text{Var}(\mathcal{O}) &= \avg{\mathcal{O}(U)^2} - \avg{\mathcal{O}(U)}^2 \\
    &= \int \mathcal{D}U\;e^{-S_G(U)}\det[D(U)] \left(\mathcal{O}(U)^2 - \avg{\mathcal{O}(U)}^2\right).
  \end{split}\label{eq:MCvar}
\end{equation}
The Monte Carlo variance is therefore controlled by the QFT operator $(\mathcal{O} - \avg{\mathcal{O}})^2$ encoding quantum fluctuations about the expectation value, and the StN ratio that will be seen in a stochastic calculation can be analytically calculated in the infinite statistics limit,
\begin{equation}
  \begin{split}
    \text{StN}(\mathcal{O}) &= \frac{\avg{\mathcal{O}}}{\sqrt{\text{Var}(\mathcal{O})}}.
  \end{split}\label{eq:MCstn}
\end{equation}
If $\mathcal{O}$ is a complex operator with $\avg{\text{Im}\mathcal{O}} = \avg{\text{Re}\mathcal{O}\text{Im}\mathcal{O}}=0$, which holds for many observables by $C$, then the variance of the real part of a complex random variable can be written
%~\cite{Alexandru:2014hga}
\begin{equation}
  \begin{split}
    \text{Var}(\text{Re}\mathcal{O}) &= \avg{(\text{Re}\mathcal{O})^2} - \avg{\text{Re}\mathcal{O}}^2\\
    &= \frac{1}{2}\avg{|\mathcal{O}|^2} + \frac{1}{2}\avg{\mathcal{O}^2} - \avg{\mathcal{O}}^2.
  \end{split}\label{eq:ComplexVar}
\end{equation}
Lepage first considered the statistics of baryon correlation functions, and by considering the scaling of terms in Eq.~\eqref{eq:ComplexVar} argued there must be exponential degradation of the baryon correlation function StN ratio with Eulcidean time~\cite{Lepage:1989hd}.
Baryon correlation functions analogous to the scalar field correlation functions in Eq.~\eqref{eq:scalar2ptdef}, describe the propagation of a baryon between points separated by a fixed Euclidean spacetime extent.
They also receive additional contributions from excited baryon states that decay exponentially faster than the ground state and can be neglected at large separation.
The conjugate of a baryon correlation function represents propagation of an anti-baryon, and the magnitude-squared appearing in the StN ratio represents propagation of three quarks and three anti-quarks.
The appropriate ground state contains three pions, as shown in Fig.~\ref{fig:lepage}.
At very large Euclidean separations, the root-mean-square magnitude of a baryon correlation function will decay exponentially with a rate $\frac{3}{2}m_\pi$, while the mean decays exponentially at a rate $M_N$.
The StN problem describes the resulting issue that the StN ratio decays at a rate $M_N - \frac{3}{2}m_\pi$.
With physical quark masses, $M_N - \frac{3}{2}m_\pi \sim 0.78\; M_N$.
High-statistics LQCD calculations~\cite{Beane:2009kya,Beane:2010em,Beane:2014oea,Detmold:2014rfa,Detmold:2014hla,Wagman:2016bam,Wagman:2017xfh}, have confirmed that numerical Monte Carlo path integral calculations in LQCD have StN ratios consistent with Parisi-Lepage scaling. 

\begin{figure}[!ht]
  \begin{center}
  \includegraphics[width=.7\columnwidth]{lepage.png}
\end{center}
  \caption{A schematic illustration of Parisi-Lepage scaling.
    At times close to the source, left, the nucleon correlation function, top, decays at a rate set by $M_N$, and it's mean-square, bottom, decays at a rate set by $2M_N$ minus an interaction energy shift.
    Baryon number conservation guarantees that the nucleon is the lightest state contributing to $C_N(t)$, but no conservation law prevents the quarks and antiquarks in $|C_N(t)|^2$ from rearranging to form pions.
    The lowest energy state in $|C_N(t)|^2$ preserving all quark lines explicitly inserted as propagators includes three pions.
  }
  \label{fig:lepage}
\end{figure}

Before discussing Parisi-Lepage scaling and statistics of nucleon correlation functions further, we briefly consider the intimately related scalings of alternative approaches to calculating the properties of dense matter in QCD.
Thermodynamic properties of nuclei and nuclear matter can be determined from free energy calculations if a quark-number chemical potential is introduced.
Large artifacts of the lattice discretization can appear with some definitions of the chemical potential, but can be avoided by recognizing that the chemical potential acts like the time-like component of an imaginary background gauge field that couples to $U(1)_B$~\cite{Hasenfratz:1983ba}.
Inclusion of a quark chemical potential breaks $\gamma_5$-Hermiticity of the Dirac operator.
This breaks reality of the quark determinant in an arbitrary gauge field configuration and introduces a sign problem.
Reweighting methods~\cite{Ferrenberg:1988yz} exploit the freedom to redefine the factorization between a probability distribution and an observable,
and can be used to (inefficiently) avoid sign problems.
To determine the free energy in the presence of a baryon chemical potential $F_{QCD}(\mu_B)$ with reweighting methods, one can sample from the standard $\mu_B=0$ vacuum probability distribution
and include the ratio of determinants with and without chemical potentials as an observable,
\begin{equation}
  \begin{split}
    Z(\mu_B) = e^{-F(\mu_B)V\beta} &= \avg{\frac{\det(D[U,\mu_B])^{N_f}}{\det(D[U,\mu_B=0])^{N_f}}} \\
    &= \sum_{i=1}^N \frac{\det(D[U_i,\mu_B])^{N_f}}{\det(D[U_i,\mu_B=0])^{N_f}}\\
    &\equiv \sum_{i=1}^N Z_i(\mu_B),
  \end{split}\label{eq:ZmuBdef}
\end{equation}
The free energy and hence $Z(\mu_B)$ are real, but $Z(\mu_B)$ is the real part of the average of a random complex function $Z_i(\mu_B)$ sampled for each gauge field configuration $U_i$.
Higher moments of random complex quark determinants have been studied, starting from the observation by Gibbs~\cite{Gibbs:1986ut} that reweighting determinants with baryon chemical potential has an exponentially bad StN problem for $\mu_B > m_\pi / 2$ and continuing with analysis in EFT and random matrix theory~\cite{Cohen:2003kd,Cohen:2003ut,Splittorff:2006fu,Splittorff:2006vj,Splittorff:2007ck,deForcrand:2010ys,Alexandru:2014hga}.
The StN ratio of a Monte Carlo calculation of $Z(\mu_B)$ employing reweighting of complex random functions $Z_i$ can be expressed by Eq.~\eqref{eq:ComplexVar} as~\cite{Alexandru:2014hga}
\begin{equation}
  \begin{split}
    \text{StN}(Z(\mu_B)) &= \frac{\avg{Z_i}}{\sqrt{\text{Var}(Z_i)}}\\
    &= \left( \frac{\avg{|Z_i|^2}}{2\avg{Z_i}^2} + \frac{\avg{Z_i^2}}{2\avg{Z_i}^2} - 1 \right)^{-1/2}.
  \end{split}\label{eq:ZStN}
\end{equation}
The mean-square determinant $\avg{Z_i^2}$ represents $2N_f$ quarks experiencing the same chemical potential as the quarks contributing to $\avg{Z_i}$, and will have a free energy that differs from $2F(\mu_B)$ by energy shifts arising from interactions between the two baryon species,
\begin{equation}
  \begin{split}
    \avg{Z_i^2} &= \int \mathcal{D}U e^{-S_G(U)} \det{D[U,\mu_B]^{2N_f}} =  e^{-\left(2F(\mu_B) - 2\delta F(\mu_B)\right)V\beta},
  \end{split}\label{eq:deltaFdef}
\end{equation}
where the second equality defines the interaction energy shift $\delta F(\mu_B)$.
For two-flavor QCD with up and down quarks, the magnitude of the baryon number determinant is equal to $\det[D(U,\mu_B)]\det[D(U,\mu_B)^*] = \det[D(U,\mu_B)]\det[D(U,-\mu_B)]$, which is equivalent to the determinant describing an isospin chemical potential acting opposite on up and down quarks~\cite{Son:2000xc},
\begin{equation}
  \begin{split}
    \avg{|Z_i|^2} &= \int \mathcal{D}U\;e^{-S_G(U)} \det(D[U,\mu_I])\det(D[U,-\mu_I]) \equiv e^{-2F_{PQ}(\mu_B)V\beta}.
  \end{split}\label{eq:FPQdef}
\end{equation}
The StN ratio of the fermion determinant is therefore given by
\begin{equation}
  \begin{split}
    \text{StN}(Z_i) = \left( \frac{1}{2}e^{[F(\mu_B)-F_{PQ}(\mu_B)]V\beta} + \frac{1}{2}e^{\delta F V \beta} - 1 \right)^{-1},
  \end{split}\label{eq:ZStNexp}
\end{equation}
Due to the bosonic nature of pions compared with the fermionic nature of baryons, the free energy of the pion system is lower than the free energy of the associated baryon system, $F_{PQ}(\mu_B) < F(\mu_B)$~\cite{Cohen:2003ut}.
This implies that reweighting faces an exponentially hard StN problem where the precision of Monte Carlo calculations degrades exponentially with increasing spacetime volume.
The baryon-baryon interaction term is negligible for repulsive interactions, $\delta F <0$, and for attractive interactions, $\delta F > 0$, adds a second source of exponential StN degradation.
For low-density, low-temperature systems in QCD, the phase-quenched term dominates the term arising from baryon-baryon interactions. 
Using $\chi PT$, it can be shown that at large $\beta$ and small $\mu_B$ the phase-quenched free energy difference provides the dominant effect and is given by~\cite{Splittorff:2006fu}
\begin{equation}
  \begin{split}
    F(\mu_B) - F_{PQ}(\mu_B) \simeq  \frac{f_\pi^2 \mu_B^2}{9}\left( 1 - \frac{9m_\pi^2}{4\mu_B^2} \right)^2\theta\left(\mu_B - \frac{3 m_\pi}{2}\right).
  \end{split}\label{eq:ZStNChPT}
\end{equation}
Once $\mu_B$ is large enough to start producing pions in the phase-quenched theory, the StN problem associated with reweighting  $\mu_B \neq 0$ determinants becomes exponentially hard in spacetime volume.
It is interesting to note that in this analysis $\mu_B > \frac{3}{2} m_\pi$ emerges as a natural scale from the phase-quenched theory even though $\mu_B > M_N$ sets the threshold of particle production~\cite{Gibbs:1986ut,Gocksch:1987ha,Cohen:2003kd}.

It is also possible in principle to determine thermodynamic properties of nuclei and nuclear matter using LQCD calculations at fixed baryon number rather than fixed chemical potential, that is working with the canonical ensemble of statistical mechanics instead of the grand canonical ensemble~\cite{Roberge:1986mm,Hasenfratz:1991ax,Kratochvila:2005mk,Alexandru:2005ix,Gattringer:2009wi,Danzer:2012vw,Alexandru:2014hga}.
The partition function at fixed baryon number is the Fourier transform of the partition function with imaginary baryon chemical potential.
Imaginary baryon chemical potential does not lead to a sign problem; however the Fourier transform requires linear combinations with complex coefficients and therefore introduces a sign problem.
This sign problem can be inefficiently avoided by reweighting as above and calculating canonical ensemble quark determinants averaged across a statistical ensemble of gauge fields importance sampled with the QCD vacuum distribution.
A calculation of the StN ratio of canonical ensemble determinants using $\chi$PT and the hadron resonance gas model for baryons shows that the difference in free energy density  between QCD at fixed baryon number and the phase-quenched theory at fixed isospin charge is given by~\cite{Alexandru:2014hga}
\begin{equation}
  \begin{split}
    F(B) - F_{PQ}(B) \simeq \frac{B}{V}\left( M_N - \frac{3}{2}m_\pi \right),
  \end{split}\label{ZStNCanChPT}
\end{equation}
where $B$ is the total baryon number of the system.
The difficulty of solving the canonical ensemble sign problem with reweighting is seen similarly to Eq.~\eqref{eq:ZStNexp} to be exponentially hard in the baryon number density times the spacetime volume, or equivalently exponentially hard in the total baryon number times the inverse temperature $B(M_N - \frac{3}{2}m_\pi)\beta$.
The scale setting this exponential difficulty is again $M_N - \frac{3}{2}m_\pi$, 
the difference between quark contributions to the ground state energy in the full theory containing baryons and phase-quenched theory containing pions.


It is difficult to compute quark determinants with large spacetime volumes,
especially a statistical sample of cold, dense quark determinants whose size is exponentially large in $B(M_N - \frac{3}{2}m_\pi)\beta$.
To compute the low-temperature physics of hadrons, 
it is standard to calculate ensembles of gauge fields distributed according to the QCD thermal vacuum probability distribution $e^{-S_G(U)}\det[D(U)]$
and then calculate hadronic correlation functions as observables, as in Eq.~\eqref{eq:MC}.
By Parisi-Lepage scaling, the StN problem associated with the correlation function of a single baryon is exponentially hard in $(M_N - \frac{3}{2}m_\pi)t$.
The NPLQCD collaboration verified Parisi-Lepage scaling in LQCD calculations and extended it to nuclei where up to corrections from interactions it is seen that the StN problem associated with correlation functions of baryon number $B$ is $B(M_N - \frac{3}{2}m_\pi)t$~\cite{Beane:2009kya}.
Using hadronic correlation functions, it is possible to probe low-temperature hadronic physics where $B(M_N - \frac{3}{2}m_\pi)\beta \gg 1$ and partition function methods would face a severe StN problem,
while only including non-zero baryon charge on a fraction $t/\beta \ll 1$ of the spacetime volume such that $B(M_N - \frac{3}{2}m_\pi)t \lesssim 1$ and the Parisi-Lepage StN problem is manageable.
It is also helpful that statistical ensembles of $N$ correlation functions for a number of different hadrons can all be computed on the same ensemble of gauge field configurations and correlated differences can be computed between them.


Once a gauge field ensemble distributed according to the vacuum distribution $e^{-S_G(U)}\det[D(U)]$ is generated, correlation functions must be defined and computed in the representation of QCD where the quark path integral has been performed analytically.
This can be readily accomplished if the interpolating operator is a product of local quark fields.
Since the QCD action in a fixed gluon field configuration is quadratic in the action of the quark fields,
correlation functions of products of quark fields in a fixed gauge field background
can be expressed a products of quark propagators in the same gauge field background
by performing free-fermion contractions.
Quark propagators in a given gauge field configuration are given by the inverse of the Dirac operator in that gauge field configuration,
\begin{equation}
  \begin{split}
    S(U_i;x,0) &= \int \mathcal{D}\bar{q}\mathcal{D}q e^{-\sum_y \bar{q}(y)D(U;y,y)q(y)} \bar{q}(x)q(0) \\
    &= D(U_i;x,0)^{-1}, \\
  \end{split}\label{propdef}
\end{equation}
Constructing the inverse of a large sparse numerical matrix can be performed with iterative Krylov solvers such as conjugate gradient and optimized using methods such as deflation~\cite{Stathopoulos:2007zi}.

The average quark propagator vanishes by Elitzur's theorem because it is not gauge invariant.
Non-vanishing expectation values only arise for color-singlet functions of quark propagators.
Such two-point functions describe for instance the propagation of mesons built with quark-antiquark sources and baryons and nuclei built from multi-quark sources.
A simple interpolating operator for the $\pi^+$, a pseudoscalar with quantum numbers of an anti-up and a down quark, is given by $\pi^+(x) = \bar{d}^a(x) \gamma_5 u^a(x)$,
where $u$ and $d$ are up and down quark fields and $a,b,\cdots$ denote $\mathfrak{su}(3)$ fundamental indices.
The pion correlation function computed with these operators can be expressed in terms of propagators as
\begin{equation}
  \begin{split}
    G_\pi(\v{x},t) &= \avg{\pi^+(\v{x},t)\pi^-(0)}\\
    &= \avg{\left( \bar{d}^b(x)\gamma_5 u^b(x)\right) \left( \bar{d}^a(0)\gamma_5 u^a(0) \right)}\\
    &= \sum_{i=1}^N \tr_s \left[S_u^{ab}(U_i;x,0)\gamma_5 S_d^{ba}(U_i;0,x)\gamma_5\right]\\
    &= \sum_{i=1}^N \tr\left[ S_u(U_ix,0) S_d^\dagger(U_i;x,0) \right],
  \end{split}\label{picorrden}
\end{equation}
where in the second line $\tr_s$ denotes a trace over spin and we use matrix notation for propagator spin contractions and in the third line $\tr = \tr_c \tr_s$ denotes a trace over color and spin. 

Interpolating fields for baryons are slightly more complicated.
The non-relativistic quark model provides a useful guide for constructing QCD interpolating operators, and for instance proton interpolating operators can be constructed with spin-singlet diquarks as
\begin{equation}
  \begin{split}
    p(x) = \varepsilon_{ijk}(u_i^T C \gamma_5 d_j)u_k,
  \end{split}\label{eq:protoninterp}
\end{equation}
where $i,j,k,\cdots$ represent $\mathfrak{su}(3)$ fundamental indices and $u$ and $d$ are fields representing up and down quarks.
Correlation functions of spin-1/2 operators are spin matrices, and it is convenient to analyze them by projecting out particle spin and parity components.
In accordance with the axial anomaly, positive-parity baryons are lighter than negative-parity baryons.
Projectors onto spin-up and spin-down positive-parity states $\Gamma_{+\uparrow}$ and $\Gamma_{+\downarrow}$ are useful for isolating the lowest-energy states in baryon propagators.
In a chiral basis with
\begin{equation}
  \begin{split}
    \gamma_i = -\sigma_2\otimes \sigma_i, \hspace{20pt}     \gamma_4 = \gamma_1\otimes 1, \hspace{20pt} \gamma_5 = \gamma_1\gamma_2\gamma_3\gamma_4 = \sigma_3 \otimes 1, \hspace{20pt}C = \gamma_2\gamma_4 = \sigma_3\otimes i\sigma_2,
  \end{split}\label{gammadef}
\end{equation}
these projectors are given by
%\footnote{The basis used in Chroma is found by taking $\gamma_2\rightarrow -\gamma_2$ and then decrementing all indices by one. Note that this means that the usual positive and negative helicity Dirac spinors $u_\uparrow = \sqrt{m}(1,0,1,0)$ and $u_\downarrow = \sqrt{m}(0,1,0,1)$ have their helicities reversed in Chroma.}
\begin{equation}
  \begin{split}
    \Gamma_{+\uparrow/\downarrow} &= \frac{1}{2}(1 \pm i \gamma_5\gamma_3\gamma_4)(1 + \gamma_4).
  \end{split}\label{spinprojdef}
\end{equation}
Defining spin-0 color-\textbf{6} diquarks as
\begin{equation}
  \begin{split}
    \text{diq}_{\alpha\beta}^{c^\prime c}(A,B) = \varepsilon^{abc}\varepsilon^{a^\prime b^\prime c^\prime} A_{\gamma \alpha}^{ab}B_{\gamma \beta}^{a^\prime b^\prime},
  \end{split}\label{eq:diquarkdef}
\end{equation}
Zero-momentum, positive parity baryon correlation functions are explicitly defined as
\begin{equation}
  \begin{split}
    G_{p}(t) &= \sum_\v{x} \tr\left\lbrace \Gamma S_u^{ab}(U_i;\v{x},t;0) \tr_s\text{diq}^{ba}\left[ C\gamma_5 S_u(U_i;\v{x},t;0), S_d(U_i;\v{x},t;0)C\gamma_5 \right]\right\rbrace\\
    &\hspace{20pt} +\tr\left\lbrace \Gamma S_u^{ab}(U_i;\v{x},t;0) \text{diq}^{ba}\left[ S_u(U_i;\v{x},t;0) C\gamma_5, C\gamma_5 S_d(U_i;\v{x},t;0)  \right] \right\rbrace.
  \end{split}\label{eq:protoncorr}
\end{equation}

The spectrum of hadronic masses and binding energies can be extracted from two-point correlation functions with the same techniques as the scalar filed two-point functions of Sec.~\ref{sec:lattice}.
Multi-particle bound states can be constructed from interpolating operators that are products of the single-particle interpolating operators projected to the correct quantum numbers.
Binding energies can be directly measured from the large-time behavior of Euclidean correlation functions.
As discussed in Sec.~\ref{sec:lattice}, the phase shifts of scattering states are related to the energy levels of multi-particle Euclidean correlation functions in a finite volume.
Basic ingredients to models of nuclear forces such as the deuteron binding energy and neutron-neutron scattering length can be computed directly from QCD using the large-time behavior of finite volume LQCD multi-baryon correlation functions~\cite{Luscher:1985dn,Luscher:1986pf,Luscher:1990ux,Aoki:2002in,Beane:2003da}.


\begin{figure}
  \begin{center}
  \includegraphics[width=\columnwidth]{ppsummary.png}
\end{center}
  \caption{
    Results from the LQCD calculation of proton-proton fusion in Ref.~\cite{Savage:2016kon}. 
    EFT($\fs{\pi}$) is used to separate long-distance axial interactions that can be described as an isolated nucleon interacting with a $W^\pm$ boson from short-distance axial interactions that require more detailed QCD input about the structure of nuclear interactions, for example include a $W^\pm$ boson interacting with a pion exchanged between the nucleons.
    Both the proton-proton fusion cross-section and the finite-volume energy shift of a two-nucleon system in a background axial field depend on a poorly experimentally constrained low-energy constant in EFT($\fs{\pi}$) called $L_{1A}$.
    A calculation of the two-nucleon finite volume energy shift in LQCD, right, therefore allows an extraction of $L_{1A}$ and in turn the proton-proton fusion cross-section.
    Results for $L_{1A}$ shown are consistent with phenomenological determinations within uncertainties, shown as statistical, fitting systematic, scale-setting systematic, and quark mass extrapolation uncertainties.
    The quark mass extrapolation uncertainty in particular requires refinement with additional calculations at lighter values of the quark masses than the $m_\pi \sim 800$ MeV ensembles used here.
    Further details about the LQCD ensembles used for this production are given in Ref.~\cite{Orginos:2015aya}.
  }
  \label{fig:plateau}
\end{figure}

Following early calculations of pion scattering~\cite{Gupta:1993rn,Fiebig:1999hs,Liu:2001ss,Yamazaki:2004qb,Beane:2005rj,Aoki:2005uf}, nucleon-nucleon scattering was computed from Euclidean correlation functions in the quenched approximation~\cite{Gupta:1993rn,Fukugita:1994na}, and then in fully dynamical $N_f=2+1$ LQCD by the NPLQCD collaboration~\cite{Beane:2006mx}.
%Further studies have been performed by the NPLQCD~\cite{}, CP-PACS~\cite{}, HALQCD~\cite{}, and CalLatt~\cite{} collaborations.
Calculations initially have been performed at heavy quark masses where the StN problem is less severe.
A silver lining from this is that early calculations have now begun to explore and understand the dependence of seemingly fine-tuned parameters in nuclear physics on the underlying parameters of the SM.
Early studies of two- and three-baryon systems determined that light nuclei exist with heavy quark masses and that binding energies are generally larger than with physical quark masses~\cite{Beane:2009gs,Beane:2010em,Yamazaki:2009ua,Beane:2009py,Beane:2009kya,Beane:2011pc,Beane:2011iw,Beane:2011zpa,Beane:2012vq,Yamazaki:2012hi,Beane:2013br,Yamazaki:2015asa,Yamazaki:2015vjn,Orginos:2015aya}.
In particular, the dineutron is unbound in nature but bound at heavier quark masses~\cite{Beane:2011iw}.
Detailed studies of nuclear correlation functions for nuclei with atomic number $A=2-5$ performed by the NPLQCD collaboration using ensembles tuned to heavy quark masses where $m_\pi \sim 800$ MeV~\cite{Beane:2012vq,Beane:2013br},
the PACS-CS collaboration in the quenched approximation~\cite{Yamazaki:2009ua,Yamazaki:2011nd}, with $m_\pi \sim 500$ MeV~\cite{Yamazaki:2012hi}, and $m_\pi \sim 300$ MeV~\cite{Yamazaki:2015asa,Yamazaki:2015vjn},
and HALQCD~\cite{Doi:2011gq,Inoue:2011nq}, but see Ref.~\cite{Savage:2016egr} for a critique of the HALQCD potential method.


Studies of nuclear structure and reactions can also be performed using LQCD calculation of multi-baryon correlation functions.
The magnetic moments of light nuclei at $m_\pi \sim 800$ MeV and $m_\pi \sim 450$ MeV obey shell-model like relations, suggesting that nuclei are describable as collections of interacting nucleons rather than structureless blobs of quarks and gluons~\cite{Chang:2015qxa,Detmold:2015daa,Parreno:2016fwu}.
Electromagnetic background field calculations were also used to postdict the measured cross-section of the electromagnetic radiative capture reaction $np\rightarrow d\gamma$~\cite{Beane:2015yha} by relating finite volume energy shifts to poorly known parameters in EFT($\fs{\pi}$) with the formalism of Detmold and Savage~\cite{Detmold:2004qn}.
Weak nuclear reactions have also recently been studied in lattice QCD, and the rates of proton-proton fusion and tritium $\beta$-decay have been determined at $m_\pi \sim 800$ MeV by the NPLQCD collaboration using LQCD in conjunction with EFT($\fs{\pi}$), see Fig.~\ref{fig:plateau}~\cite{Savage:2016kon}.
Calculations of the kinematically forbidden doubly-weak reaction $nn\rightarrow pp$~\cite{Shanahan:2017bgi,Tiburzi:2017iux} have identified and made preliminary LQCD explorations of previously overlooked isotensor polarizability effects that add significant theoretical uncertainties to interpretation of experimental searches for neutrinoless double-$\beta$ decay.
Understanding the emergence of nuclear structure and reactions from quarks and gluons will require studies similarly combining LQCD with EFT in order to incorporate known infrared nuclear physics and quantify the effects of short-distance physics specific to QCD.


As the number of quark fields included in a correlation function is increased, the number of contractions needed to express the correlation function in terms of propagators increases factorially.
Faster contraction algorithms exploiting symmetries of nuclear interpolating operators and the Grassmannian nature of quark fields lead to dramatic accelerations in the efficiency of multi-baryon contraction algorithms~\cite{Detmold:2012eu,Doi:2012xd}.
Calculations of up to $A=28$ have been performed~\cite{Detmold:2012eu}, but exponential decrease in statistical precision from the StN problem and other challenges have so far kept detailed studies of all but the lightest nuclei out of reach.


Methods have been developed that are able to exponentially reduce the severity of the StN problem in simple theories, and recently in LQCD.
In particular, the hierarchical integration approach of Ref.~\cite{Luscher:2001up} exploits the locality of QCD to express Wilson loops and other observables as products of factors that only depend on the gauge field on a subset of the lattice.
By calculating the average observable as a product of sub-lattice averages, the overall variance is reduced.
In a two-level hierarchical integration scheme, for instance, taking the product of averages instead of the average of products leads to  effectively $N^{-1}$ instead of $N^{-1/2}$ error scaling, or equivalently reduces the exponential scale of StN degradation for nuclei from $B(M_N - \frac{3}{2}m_\pi)$ to $\frac{B}{2}(M_N - \frac{3}{2}m_\pi)$.
Gluonic observables have been calculated using hierarchical integration~\cite{Meyer:2002cd,DellaMorte:2007zz,DellaMorte:2008jd,DellaMorte:2010yp,Vera:2016xpp}, and recently
C{\`e}, Giusti, and Schaefer have applied hierarchical integration to baryon correlation functions in quenched QCD~\cite{Ce:2016idq}
and unquenched fermion determinants~\cite{Ce:2016ajy}. 
The success of hierarchical integration schemes can be physically understood with the ideas that fluctuations in fields at widely separated spacetime points are approximately uncorrelated and averaging over uncorrelated fluctuations add little signal but a lot of noise.


Other investigations of the StN problem have focused on understanding the probability distributions of noisy observables.
It is lucidly argued by Hamber, Marinari, Parisi and Rebbi in Appendix B of Ref.~\cite{Hamber:1983vu} and further explained by Guagnelli, Marinari, and Parisi~\cite{Guagnelli:1990jb} that probably distributions of single-particle correlation functions are sensitive to the effects of multi-particle interactions.
Assuming a model of two-body forces, the $n$-th moment of a correlation functions receives a contribution proportional to the mass of the particle that scales with $n$, and a contribution arising from two-body interactions that scales like $n(n-1)/2$ times the binding energy or finite volume energy shift for a scattering state.
This pattern of moments for the average mass is satisfied for a correlation function that is log-normally distributed and whose $n$-th moment is therefore given by $e^{\mu n + \sigma^2 n^2/2}$.
Since pions experience perturbative two-body forces and even weaker multi-body forces, pion correlation functions are predicted by these arguments to have a log-normal distribution where the variance can be quantitatively related in $\chi$PT to the $\pi-\pi$ scattering length.

Endres, Kaplan, Lee, and Nicholson~\cite{Endres:2011jm} found that log-normal correlation functions are ubiquitous in Monte Carlo calculations of non-relativistic unitary fermions, and
developed a cumulant expansion that determines the average correlation function of a log-normally distribution sample more precisely than a calculation of the sample mean.
They presented statistical and mean-field arguments suggesting log-normal distributions are generic for QFT correlation functions,
and the cumulant expansion is used in further studies of unitary fermions~\cite{Endres:2011er,Endres:2011mm,Lee:2011sm,Endres:2012cw}.
DeGrand~\cite{DeGrand:2012ik} observed that meson, baryon, and gauge-field correlation functions in 
$SU(N_c)$ gauge theories with a range of $N_c$ are also approximately log-normal at early times where 
imaginary parts of correlation functions can be neglected. 

At large time separations, a nucleon correlation function calculated in a generic gauge configuration is complex,
and a log-normal distribution provides a poor fit to the real part.
Parisi-Lepage analysis was extended to higher moments of the correlation function distribution by Savage~\cite{Savage:2010misc},
who showed that all odd moments are exponentially suppressed compared to even moments at late times.
Kaplan noted that a stable distribution provides a reasonable fit to the real part of the late time nucleon distribution~\cite{davidkaplanLuschertalk}.
Correlation function distributions have been studied analytically in the Nambu-Jona-Lasinio 
model~\cite{Grabowska:2012ik,Nicholson:2012xt}, where it was found that real correlation functions were approximately log-normal but complex correlation functions in a physically equivalent formulation of the theory were broad and symmetric at late times with qualitative similarities to the QCD nucleon distribution. 

Chapter~\ref{chap:statistics} describes the statistics of baryon correlation functions at large times, where they must be treated as complex.
Complex correlation functions for the nucleon and other hadrons are found to be well-described by an approximately uncorrelated product of a log-normal magnitude and a wrapped normal phase factor.
From the behavior of the phase, the nucleon StN problem described by Parisi-Lepage scaling is found to follow directly from the sign problem associated with the nucleon correlation function phase.
The time evolution of the log-magnitude and the phase is further shown to resemble a heavy-tailed random walk where steps are only correlated over hadronic timescales.
Building on these observations, a new estimator based on correlation function ratios is proposed for which StN degradation only appears in a time paramter that can be treated independently from the source-sink separation time.

A variant on this technique callled phase reweighting is discussed in Chapter~\ref{chap:PR}.
Exploratory studies are conducted for meson, baryon, and two-baryon systems.
Phase-reweighted results with large time separations inacessible to standard analysis techniques 
are found to give consistent results with comparable precision to standrd techniques,
and possibilities for expanding the scope and improving the precision of phase reweighting are highlighted.

Chapter~\ref{chap:mesons} discusses isovector meson correlation functions.
These meson correlation functions are real but non-positive definite and face a ``sign'' rather than a ``phase'' problem.
Similar statistical distributions are found to describe the real parts of meson and baryon correlation functions,
and the asymptotic time dependence of moments of this distribution is shownw to possess generic features.
Applications of phase reweighting and ratio-based estimators are discussed.


