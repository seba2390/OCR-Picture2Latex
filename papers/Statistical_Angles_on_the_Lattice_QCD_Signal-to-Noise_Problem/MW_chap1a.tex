\section{Quarks and Gluons}\label{sec:qcd}

Quantum field theories generically describe correlations between various sorts of matter throughout spacetime that
are consistent with the general postulates of quantum mechanics and special relativity.
A particular QFT is specified by the number and symmetry transformation properties of fields included to represent these various sorts of matter.
The SM includes fields representing
spin one-half fermions called quarks and leptons,
spin one gauge bosons called gluons, photons, $W^\pm$, and $Z^0$ bosons,
and a spin-zero Higgs boson
experimentally discovered recently at the LHC~\cite{Aad:2012tfa,Chatrchyan:2012xdj}.
The SM's properties are highly constrained by $SU(3)_C\times SU(2)_L\times U(1)_Y$ gauge invariance,
%\cite{Yang:1954ek},
where the photon, $W^\pm$, and $Z^0$ are associated with the $SU(2)_L\times U(1)_Y$ electroweak gauge group~\cite{Glashow:1961tr,Salam:1964ry,Weinberg:1967tq,Glashow:1970gm}
and the gluons are associated with the $SU(3)_C$ QCD gauge group~\cite{Fritzsch:1972jv,Fritzsch:1973pi,Politzer:1973fx,Gross:1973id}.

The effects of heavy particles in QFT decouple from low-energy physics~\cite{Appelquist:1974tg}, and through renormalization\footnote{See Ref.~\cite{Collins:105730} for a clear and comprehensive review of renormalization in QFT with further references to the original literature.} 
%~\cite{Wilson:1973jj}
can be implicitly included in an EFT 
that only explicitly includes fields for particles with mass lower than a freely chosen renormalization scale $\mu$.
The benefit of considering a low-energy EFT is typically that low-energy dynamics are described more simply; the cost is that only dynamics at energy scales $\lesssim \mu$ are accurately described.\footnote{Introductions to EFT for a wide range of particle, nuclear, atomic, and condensed matter systems can be found in~\cite{Buras:1998raa,Kaplan:2005es,Bedaque:2002mn,Beane:2000fx,Kaplan:1996nv,Scherer:2002tk,Bernard:1995dp,Braaten:2004rn}.}
Nuclear energy scales are typically measured in MeV, and nuclei should be well-described by an EFT of the SM that includes explicit fields for the gluons, photons, and light quarks and leptons valid for $\mu \ll 80$ GeV, though this remains to be verified experimentally. 
The explicit gauge group of this low-energy EFT is $SU(3)_C\times U(1)_{EM}$ where the photon is associated with the gauge group $U(1)_{EM}$ of quantum electrodynamics (QED).
Quantum fluctuations in the photon field do not change the qualitative behavior of classical photons in many systems and can often be included as perturbative corrections to classical electromagnetism.\footnote{Important cases where QED must be treated non-perturbatively include bound states held together by non-perturbative Coulomb forces~\cite{Caswell:1985ui}, near-threshold scattering states of charged particles~\cite{Kong:1999sf}, and charged particles in finite volumes~\cite{Beane:2014qha}.}
Leptons only interact with quarks`, gluons, and one another through photon exchange, and so as a further simplification the interactions of quarks and gluons with leptons and photons can be perturbatively expanded about the non-interacting limit.
Up to $O(\alpha)$ corrections, the low-energy physics of light nuclei should be accurately described by QCD.

SM matter can exist in qualitatively different thermodynamic phases.
Normal matter exists in a phase
where QCD is confining\footnote{Precisely defining confinement is subtle, see Ref.~\cite{Greensite:2003bk}.} and
electroweak interactions are screened by the Higgs mechanism~\cite{Higgs:1964ia,Higgs:1964pj,Higgs:1966ev,Englert:1964et}.
Other regions of the SM phase diagram
where matter has dramatically different properties exist,
for instance the deconfined quark-gluon plasma experimentally created at RHIC~\cite{Arsene:2004fa,Back:2004je,Adcox:2004mh} and the LHC~\cite{Aamodt:2008zz}, 
which LQCD calculations predict forms above $T\sim 160$ MeV~\cite{McLerran:1980pk,Kuti:1980gh,Engels:1980ty,Kajantie:1981wh,Kogut:1982rt,Pisarski:1983ms,Celik:1983wz,Aoki:2006we,Borsanyi:2010bp,Bazavov:2011nk,Bhattacharya:2014ara}.
Extremely dense astrophysical environments such as the interior of neutron stars may also contain matter in exotic phases of the SM~\cite{Collins:1974ky,Alford:1998mk,Alford:1997zt,Rapp:1997zu,Schafer:1998ef}.
Calculating the properties of cold, dense, strongly interacting matter from first principles
but is an essential step towards understanding nuclei, nuclear matter, and neutron stars
as highly entangled, emergent states
of quarks, gluons, and other SM fields.
Reliable calculations of the QCD equation of state for cold, dense matter will be required to understand gravitational wave signals from neutron star mergers~\cite{Flanagan:2007ix,Damour:2012yf}, 
a timely goal for nuclear theory now that gravitational waves from binary black hole mergers have been observed at LIGO~\cite{Abbott:2016blz}.
LQCD calculations can describe cold, dense, strongly interacting matter in principle,
but have long been obstructed by the sign problem in practice~\cite{Gibbs:1986ut}.

In it's confined phase, the lowest-energy states in QCD describe color-singlet particles:
mesons that have the conserved charges of a quark-antiquark pair,
baryons that have the conserved charges of three quarks,
bound states such as nuclei,
glueballs,
and other exotic particles.
Gluon number is not conserved in QCD, so in addition to the minimum number of quarks that a hadron must have by symmetry, 
hadrons consistent of an indeterminate, fluctuating number of gluons and quark-antiquark pairs.
An unphysical but illuminating version of QCD with $N_f$ massless quark fields has the global symmetry group $SU(N_f)_L\times SU(N_f)_R\times U(1)_Q^{N_f}$.
The $N_f$ copies of $U(1)_Q$ are associated with conservation of the total quark minus antiquark number for each quark flavor.
The QCD action is also invariant under flavor-singlet axial transformations, but the quantum theory is not and quark number is not separately conserved for positive and negative chirality quarks~\cite{Adler:1969gk,Bell:1969ts}.
Weak interactions mix quark flavors, and $U(1)_Q^{N_f}$ is broken to the subgroup $U(1)_B$ that acts identically on all quark flavors when electroweak interactions are included.
The conserved charge $B$ associated with $U(1)_B$ is called baryon number and is equal to one-third the total quark number.
Non-perturbative electroweak effects can mix baryons with leptons, breaking baryon number $U(1)_B$ and lepton number $U(1)_L$ to the subgroup $U(1)_{B-L}$, but violations of baryon number are negligible at accessible energies~\cite{tHooft:1976up}.

Chiral symmetry $SU(N_f)_L\times SU(N_f)_R$ transformations mix different flavors and chiralities of massless quarks.
Not all of these transformations are associated with conserved charges.
The low-temperature QCD vacuum is only invariant under a subgroup $SU(N_f)_V \subset SU(N_f)_L\times SU(N_f)_R$ of transformations that do not mix chirality~\cite{Nambu:1961fr,Nambu:1961tp,Vafa:1983tf,Vafa:1984xg}.
Symmetries that not preserved by the vacuum are said to be spontaneously broken and are associated with massless particles instead of conservation laws.
In nature, quarks are massive and $SU(N_f)_L\times SU(N_f)_R$ is explicitly broken.
When considering the dynamics of massless Nambu-Goldstone bosons, explicit $SU(N_f)\times SU(N_f)_R$ breaking by quark masses can be understood as a small perturbation leading to approximately massless pseudo-Nambu-Goldstone bosons.
The up and down quarks are the lightest quarks, and $SU(2)_L\times SU(2)_R$ is a good approximate symmetry where $SU(2)_V$ can be associated with approximate conservation of isospin.
The lightest mesons, the pions, can be accurately described as pseudo-Goldstone bosons in chiral perturbation theory ($\chi$PT), the low-energy EFT of spontaneously broken chiral symmetry~\cite{Weinberg:1978kz,Gasser:1983yg,Gasser:1984gg}.
The strange quark is heavier, but is still light compared to hadronic scales and the pions, kaons, and eta can be understood as an octet of pseudo-Nambu-Goldstone bosons in three-flavor $\chi$PT~\cite{Gasser:1984gg}.
Since the pion is the lightest state in QCD, it's Compoton wavelength $m_\pi^{-1}$ sets the longest correlation length in the QCD vacuum.
$m_{\pi}^{-1}$ also sets correlation length for widely separated color-singlet hadrons and the long-distance behavior of nuclear forces.
Low-energy baryon dynamics are described by EFT($\fs{\pi}$), an EFT that accurately describes baryon-baryon scattering with energy and momentum transfer much less than $m_\pi$.
Attempts to combine pions and nucleons into a convergent EFT for low-energy nuclear physics have a long history~\cite{Gasser:1987rb,Jenkins:1990jv,Weinberg:1990rz,Weinberg:1991um,Kaplan:1996xu,Kaplan:1998tg,Kaplan:1998we,Bedaque:1998kg,Fleming:1999ee,Beane:2001bc}, and are still under active investigation~\cite{Epelbaum:2008ga}.

The quark and gluon fields of QCD are tensors with components representing each spin, color, and flavor.
Lorentz spinor fields $q(x)$ representing quarks transform in the fundamental representation of the algebra $\mathfrak{su}(3)_C$, spinor fields $\bar{q}(x)$ representing  transform in the antifundamental representation, and Lorentz vector fields $G_\mu(x)$ representing gluons transform in the adjoint representation.
With $N_f$ quark flavors explicitly represented and $N_c=3$ colors, $q(x)$ is a $4 N_c N_f$ component vector with three color states, four spin states, and $N_f$ flavor states, while each of the four spin components of $G_\mu(x)$ is as a $N_c\times N_c$ anti-Hermitian matrix.
In Minkowski spacetime,
QCD states are represented by vectors in a Hilbert space
and time evolution is described by a unitary operator $e^{-i H_{QCD} t}$, where $H_{QCD}$ is the QCD Hamiltonian and $t$ is the duration of the time evolution.
The probability that an initial quantum state $\ket{I}$ prepared at time $t=0$ dynamically evolves into a final quantum state $\ket{F}$ at time $t$ is the squared magnitude of the amplitude $\mbraket{F}{e^{-iH_{QCD}t}}{I}$.
Denoting the initial state field configurations by $q_I(x),\;\bar{q}_I(x)$ and $G_I(x)$ and the final state field configurations by $q_F(x)$, $\bar{q}_F(x)$, and $G_F(x)$, this generic amplitude can be represented by the path integral
\begin{equation}
  \begin{split}
    \mbraket{F}{e^{-iH_{QCD}t}}{I} = \int_{q=q_I,\bar{q}=\bar{q}_I,G=G_I}^{q=q_F,\bar{q}=\bar{q}_F,G=G_F} \mathcal{D}G\mathcal{D}q\mathcal{D}\bar{q} e^{iS_{QCD}[q,\bar{q},G]},
  \end{split}\label{eq:pathintegraldef}
\end{equation}
where $S_{QCD}$ is the Minkowski-space QCD action
\begin{equation}
  \begin{split}
    S_{QCD}(q,\bar{q},G) = \int d^4x \left[ \frac{1}{2g^2}\tr\left(G_{\mu\nu}(x)G^{\mu\nu}(x)\right)  + \bar{q}(x)\left(\fs{D} - m_q\right)q(x)\right].
  \end{split}\label{eq:minkowskiactiondef}
\end{equation}
The QCD action involves the covariant derivative
\begin{equation}
  D_\mu = \partial_\mu + G_\mu,
  \label{eq:covariantderivativedef}
\end{equation}
responsible for parallel transport of quark color vectors, as well as the gluon field strength tensor
\begin{equation}
  \begin{split}
    G_{\mu\nu} &= [D_\mu, D_\nu] = \partial_\mu G_\nu - \partial_\nu G_\mu + [G_\mu, G_\nu],\\
\end{split}
  \label{eq:fieldstrengthdef}
\end{equation}
The Dirac operator is $\fs{D} = \gamma^\mu D_\mu$ where $\gamma^\mu$ represent a Lorentz vector of $4\times 4$ spin matrices satisfying $\{\gamma_\mu,\gamma_\mu\} = 2g_{\mu\nu}$ with $g_{\mu\nu} = \text{diag}(-1,1,1,1)$. 
Quarks are taken to be in mass eigenstates where the quark mass matrix $m_q$ is diagonal in spin, flavor, and color. 
The mass of each quark flavor is an input parameter of QCD. 
$g$ denotes the bare gauge coupling, an input parameter whose value, as discussed below, sets the scale of hadronic and nuclear masses and energies in physical units.
Formally defining the path integral measure and divergent-looking oscillatory integrals in Eq.~\eqref{eq:pathintegraldef} requires non-perturbative regularization with the methods of lattice field theory and is deferred to Sec.~\ref{sec:lattice}.

Real-time QCD path integrals cannot be calculated exactly or numerically with known methods because of the sign problem,  
but perturbation theory and other approximations can be used to understand semi-classical fluctuations that illuminate the structure of QCD~\cite{Hooft:1972fi}.
The zero-coupling limit of QCD is a free field theory that can be solved exactly.
Expanding the integrands of path integrals about the zero field configuration (or another saddle point of the action) in powers of the QCD coupling constant provides a method of deriving asymptotic expansions to path integrals valid when the QCD coupling is weak~\cite{tHooft:1977xjm}.
The utility of a weak-coupling expansion for so-called strong interactions may not be obvious upon first glance, but
at very short distances such as the interaction region of a high energy collision, the QCD interactions of quarks become perturbatively weak~\cite{Politzer:1973fx,Gross:1973id}.

This property, known as asymptotic freedom, arises from perturbative quantum fluctuations of the gluon field.
Perturbative quantum fluctuations give rise to vacuum polarization effects that modify the effective color charge appearing for example in the chromoelectric Coulomb's law, see Fig.~\ref{fig:qcdmagnet} and for further discussion Refs.~\cite{Appelquist:1977tw,opac-b1131978}.
The modifications are simplest in momentum space, where a well-known calculation shows that the effective coupling $\alpha_s(\mu^\prime) = g(\mu^\prime)^2/(4\pi)$ evaluated at an energy scale $\mu^\prime$ is related to the effective coupling at a different energy scale $\mu$ by~\cite{Politzer:1973fx,Gross:1973id}
\begin{equation}
  \begin{split}
    \alpha_s(\mu^\prime) = \frac{\alpha_s(\mu)}{1 + \frac{\alpha_s(\mu)}{4\pi}(\frac{11}{3}N_c - \frac{2}{3}N_f )\ln(\mu^2/\mu^{\prime 2})},
  \end{split}\label{eq:alphas}
\end{equation}
where $N_c=3$ is the number of colors and $N_f$ is the number of quark fields explicitly included in the theory.
Comparison of the evolution of $\alpha_s(\mu)$ to Eq.~\eqref{eq:alphas} and it's higher-order corrections to results from experiments obtained at a variety of scales provides strong experimental support for QCD at high energies, as summarized in Ref.~\cite{Bethke:2009jm}.
The renormalization group provides useful tools for relating theories that describe different length scales.\footnote{Wen changing renormalization scales from $\mu$ to $\mu^\prime$, the renormalization group can be used to re-sum logarithms such as the one appearing in Eq.~\eqref{eq:alphas} that become large when $\mu/\mu^\prime$ becomes sufficiently large (or sufficiently small). Without this re-summation, perturbation theory fails~\cite{Collins:105730}.}

\begin{figure}[!ht]
  \begin{center}
  \includegraphics[width=.7\columnwidth]{vacuumoneloop.png}
  \end{center}
  \caption{
    Perturbative fluctuations of the QED vacuum, left, and QCD vacuum, right, about a source of charge in the vacuum.
    Fluctuations of electron-positron pairs in QED are more likely to be oriented with the opposite charge closer to the test charge at center,
    and the QED vacuum accordingly acts as a dielectric where charge is slightly screened at large distances compared to classical expectations.
    In QCD, quark-antiquark pairs act similarly, but fluctuations of gluon fields also contribute.
    A gluon source, right, induces vacuum fluctuations from gluons in orthogonal color orientations that behave like charged vector particle loops.
    Both the QED and QCD configurations shown act as paramagnets as well as dielectrics if the spins of the fluctuations are aligned.
    Pauli blocking raises the energy of each spin-aligned fermion, but no such effect occurs for bosons.
    Explicit calculation shows that paramagnetic effects lead to an overall decrease in the vacuum energy if color charge is effectively enhanced, rather than screened, at large distances~\cite{Appelquist:1977tw,opac-b1131978}.}
  \label{fig:qcdmagnet}
\end{figure}

The one-loop running coupling diverges at a scale
\begin{equation}
  \begin{split}
  \Lambda_{QCD} = \mu \exp\left[ \frac{-1}{\alpha_s(\mu)}\left(\frac{2\pi}{11N_c/3 - 2N_f/3}\right) \right].
  \end{split}\label{eq:LambdaQCDdef}
\end{equation}
Before $\mu$ reaches $\Lambda_{QCD}$ in a limit from above, $\alpha_s(\mu)$ becomes large, neglected two-loop corrections to Eq.~\eqref{eq:alphas} become as important as one-loop corrections, and Eq.\eqref{eq:alphas} becomes unreliable.
The divergence in the one-loop result is not a physical divergence; however, Eq.~\eqref{eq:LambdaQCDdef} allows a physical length scale to be defined from the dimensionless running coupling evaluated at a given scale.
By construction, $\frac{d}{d\mu}\Lambda_{QCD}=0$, so $\Lambda_{QCD}$ is renormalization scale invariant.
It is also independent of the renormalization scheme used to relate $\alpha_s(\mu)$ to a physical observable at the level of perturbative accuracy considered.
This emergence of a physical, dimensionful scale that only depends on the dimensionless gauge coupling, $N_c$, and $N_f$ is known as dimensional transmutation.
This is a non-perturbative phenomenon, signaled by the fact that a perturbative expansion in $\alpha_s(\mu)$ of the RHS of Eq.~\eqref{eq:LambdaQCDdef} vanishes to all orders in $\alpha_s(\mu)$.
For energies and momenta much larger than $\Lambda_{QCD}$, or distances and times much smaller than $\Lambda_{QCD}^{-1}$, the QCD running coupling is small and weak-coupling perturbation theory applies.
The predictive accuracy of weak-coupling expansions for high-energy QCD is demonstrated by comparing the best-fit $\alpha_s(\mu)$ determined from comparing perturbative QCD predictions to the results of various experiments and various $\mu$~\cite{Bethke:2009jm}.
For low-energies and large distances, the QCD running coupling is large and perturbation theory is unreliable.

At larger distances than $\Lambda_{QCD}^{-1}$, the effective potential between static color sources rises approximately linearly rather than falling according to Coulomb's law.
This adds an infinite energy cost to isolating a static color charge, and the static charge is said to be confined.
Confinement of static color charges can be analytically demonstrated for the strong-coupling limit of LQCD~\cite{Wilson:1974sk}.
In the strong coupling limit a linearly rising potential between static color charges arises from gluon field configurations resembling tubes of color flux joining the charges.
There is currently no analytic proof that QCD is confining outside the strong-coupling limit, but a wealth of non-perturbative LQCD results demonstrate that the spectrum of QCD describes bound and scattering states of color-singlet particles.
This is in accordance with experimental non-observation of isolated quarks or gluons.

\begin{figure}[ht!]
  \begin{center}
  \includegraphics[width=\columnwidth]{zoom-18.png}
  \end{center}
  \caption{
    A visualization of the topological charge density of a generic LQCD vacuum gluon configuration, courtesy Daniel Trewartha.
    Cooling has been applied to the gauge field configuration to average over ultraviolet fluctuations.
    The remaining three-dimensional structures that are apparent are localized regions of large positive, orange, or large negative, blue, topological charge density~\cite{Trewartha:2015ida}.}
  \label{fig:topcharge}
\end{figure}

It is noteworthy that Eq.~\eqref{eq:minkowskiactiondef} is not the most general local action comprised of quark and gluon fields that is consistent with Poincar{\'e} invariance and $SU(3)_C$ gauge invariance.
First, there are an infinite number of higher dimension operators that could be included in the action.
The renormalizability of the SM shows that these higher dimensional interactions will not be generated by perturbative fluctuations of SM fields.
Higher dimensional operators can arise when the SM is considered as an EFT for BSM physics, but
renormalization group arguments and EFT suggest that interactions involving higher dimensional operators are suppressed by $(p/\Lambda_{BSM})^n$ where $p$ is a relevant physical momentum scale, $\Lambda_{BSM} \gtrsim 10$ TeV is the cutoff scale where BSM physics could give rise to such interactions, and $n$ is the dimension of the operator.
Second, QCD is not the most general action containing dimension four operators.
Since the full SM does not respect the discrete symmetry of $CP$, an additional $CP$ violating term is expected to appear in the QCD action,
\begin{equation}
  \begin{split}
    S_{\theta} &= -  \bar{\theta}  \frac{N_f}{32\pi^2} \int d^4 x\; \varepsilon^{\mu\nu\alpha\beta}\tr\left[ G_{\mu\nu}(x)G_{\alpha\beta}(x) \right]
  \end{split}\label{eq:thetaactiondef}
\end{equation}
where $\bar{\theta}$ is a free parameter.\footnote{Strictly, we assume a basis where the determinant of the quark mass matrix is real and positive. An anomalous $U(1)_A$ transformation can be used to remove an overall phase from the quark matrix while introducing a shift in the vacuum angle $\theta$. The notation $\bar{\theta}$ is reserved for the vacuum angle in a basis where all $CP$ violation has been shifted from the quark mass matrix to the vacuum angle~\cite{coleman:1988aspects}.}
Spatial integrals of the topological charge density appearing in Eq.~\eqref{eq:thetaactiondef} are constrained to be integers, and count non-perturbative excitations of the gluon field localized in both space and time called instantons.
Single instanton field configurations are associated with saddle points of the (Euclidean) action and can be analyzed semi-classically~\cite{Belavin:1975fg,tHooft:1976fv}.
The thermal vacuum of QCD resembles a gas of instantons at high temperatures~\cite{Gross:1980br}, and it is expected that non-perturbative features of QCD are associated with instantons and multi-instanton gluon field configurations~\cite{Schafer:1996wv}.
Random samples of the topological charge density generated from smoothed LQCD gauge field ensembles suggest the non-perturbative QCD vacuum can be intuitively described as a fluctuating medium with regions of large positive or negative topological charge density localized in spacetime reminiscent of instantons and anti-instantons~\cite{Chu:1994vi,Gattringer:2001ia}.
While local fluctuations in the topological charge density appear commonplace in the QCD vacuum, the non-observation of $CP$ violating strong interactions shows that $\bar{\theta}$ in Eq.~\eqref{eq:thetaactiondef} must be very small or vanish.
Non-zero $\bar{\theta}$ would introduce $CP$ violating interactions such as a neutron electric dipole moment, and experimental constraints place stringent bounds $\bar{\theta} \lesssim 10^{-9}$ ~\cite{He:1990qa,Baker:2006ts}.
LQCD calculations needed to relate experimental observation or bounds of electric dipole moments to rigorous constraints on particular $CP$ violating BSM theory are underway~\cite{Shintani:2005xg,Guo:2015tla,Shintani:2015vsx,Bhattacharya:2016zcn,Gupta:2017anz}.
LQCD calculations can also directly probe theoretically how $\bar{\theta}\neq 0$ affects the physics of QCD.
However, such LQCD calculations are difficult because setting $\bar{\theta}\neq 0$ introduces a sign problem, see Ref.~\cite{Cai:2016eot} for a recent discussion.

