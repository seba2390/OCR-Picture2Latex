\chapter{Meson Statistics and Real-Valued Sign Problems}\label{chap:mesons}
 
%
%Lepage-Savage scaling is shown to be a generic property of complex correlation functions, and to apply to isovector meson correlation functions.
%Most tellingly, the sample mean of an ensemble of $N$ isovector meson correlation functions show systematic deviations from the true average unless $\avg{cos \theta_i} \geq 1/\sqrt{N}$ in accordance with the expectations of circular statistics~\cite{Fisher:1995}.
%Phase reweighting has been used to the StN problem for the $\rho^+$ in Chatper~\ref{chap:PR}, and its success as well as the failure of the ratio estimator introduced in Chapter~\ref{chap:statistics} and further explored in this chapter.
%The ground state energies of other isovector meson channels, previously inaccessible to techniques based on spectroscopy in the golden window, are precisely extracted from the noise region of phase-reweighted correlation functions.
%%%%%%%%%%%%

%%%%%%%%%%%%%%%%%%%%%%%%%%%%%%%%%%%%%%%%%%%%%%%
The statistical observations of Chapter~\ref{chap:statistics} begin with a magnitude-phase decomposition of correlation functions.
This is a suitable starting point for baryon and multi-baryon correlation functions, which are generically complex.
It is a less obvious starting point for analyzing correlation functions that are real but non-positive definite.
Such correlation functions face a true ``sign problem'' rather than a ``phase problem.''
Each $C_i$ in a Monte Carlo ensemble of $i=1,\cdots,N$ real but non-positive-definite correlation functions is described by a positive-definite magnitude and a binary valued sign,
\begin{equation}
  \begin{split}
    C_i = |C_i|e^{i\theta_i},\hspace{20pt} e^{i\theta_i} \in \{+1,-1\}.
  \end{split}\label{eq:binary}
\end{equation}
In this chapter, the binary-valued signs of meson correlation functions in QCD are analyzed and found to be consistently described by circular statistics.
Real-valued isovector correlation functions are shown to be statistically well-described by the real part of the complex-log-normal distribution introduced in Chapter~\ref{chap:statistics}.
The correlation-function-ratio estimator introduced in Chapter~\ref{chap:statistics} and phase-reweighted techniques introduced in Chapter~\ref{chap:PR} are both applied to isovector meson correlation functions.
Phase reweighting allows precise ground-state energy results to be extracted from the noise region; analyzing correlation function ratios does not.
As a demonstration of the utility of phase reweighting and exploration of meson phenomenology in the world of heavy quarks where LQCD simulations are currently performed, phase-reweighted ground-state energies are extracted from the noise region for several channels that have proven difficult to access with GW spectroscopy.

The $\rho^+$ correlation function introduced in Eq.~\eqref{eq:blockcon} is part of a more general class of isovector meson correlation functions associated with the pseudoscalar $\pi$, the $1^{--}$ vector $\rho(770)$ the scalar $a_0(980)$, the $1^{++}$ vector $a_1(1260)$, and the $1^{+-}$ vector $b_1(1235)$ resonances~\cite{Olive:2016xmw}. 
At light quark masses, these channels include narrow and broad resonances as well as a large multiplicity of scattering states above $\pi \pi$ and $K\bar{K}$ thresholds. 
The resonance structure of these states has been studied intensively in recent years~\cite{Aoki:2007rd,Dudek:2009qf,Dudek:2010wm,Lang:2011mn,Aoki:2011yj,Edwards:2011jj,Thomas:2011rh,Dudek:2011tt,Dudek:2012ag,Dudek:2012gj,Dudek:2012xn,Pelissier:2012pi,Dudek:2013yja,Wilson:2014cna,Bolton:2015psa,Briceno:2015dca,Wilson:2015dqa,Dudek:2016cru,Briceno:2016kkp,Briceno:2016mjc},
A wide range of single- and multi-particle interpolating operators,
extensive formalism relating scattering parameters to finite volume energy shifts~\cite{Luscher:1990ux,Luscher:1990ck,Rummukainen:1995vs,Lellouch:2000pv,Beane:2003da,Feng:2004ua,He:2005ey,Kim:2005gf,Christ:2005gi,Bernard:2008ax,Lage:2009zv,Bernard:2010fp,Luu:2011ep,Briceno:2012yi,Briceno:2012rv,Hansen:2012bj,Hansen:2012tf,Gockeler:2012yj,Dudek:2012xn,Guo:2012hv,Leskovec:2012gb,Briceno:2013bda,Briceno:2013lba,Briceno:2014oea,Hansen:2014eka,Dudek:2014qha,Wilson:2014cna,Wilson:2015dqa,Bali:2015gji,Briceno:2016xwb,Hansen:2017mnd}
variational methods for identifying excited states~\cite{Michael:1985ne,Luscher:1990ck,Dudek:2007wv,Blossier:2009kd},
and techniques for efficiently computing disconnecting diagrams such as distillation~\cite{Peardon:2009gh},
are needed to extract physical resonance parameters from LQCD.

At larger quark masses, the pion mass is closer to the masses of the other isovector mesons
and channels with multi-pion ground states at light quark masses
may instead have single-hadron ground states associated with stable isovector mesons.
At the heavy quark masses used in this work with $m_\pi \sim 450$ MeV and $m_\pi \sim 800$ MeV,
it is expected that some or all of the low-lying isovector mesons can be described as compact bound states~\cite{Briceno:2016xwb}
and therefore have
finite-volume ground-state energies that are exponentially close to their infinite-volume counterparts.
Extending the use of phase-reweighting to meson spectroscopy at light quark masses
will require new variational methods for phase-reweighted correlation functions 
in conjunction with the existing sophisticated tools of LQCD resonance calculations above.

LQCD results in this chapter employ a larger set of the $m_\pi \sim 450$ MeV correlation function ensembles generated by the College of William and Mary/JLab lattice group and the NPLQCD collaboration previously employed above that includes three spacetime volumes $L^3\times \beta$ of dimensions $24^3\times 64$, $32^3 \times 96$ and $48^3\times 96$, see Ref.~\cite{Orginos:2015aya}. Correlation functions with $m_\pi \sim 800$ MeV and spacetime dimension $32^3 \times 48$ are further employed in this chapter.

\section{A Magnitude-Sign Decomposition}
%%%%%%%%%%%%%%%%%%%%%%%%%%%%%%%%%%%%%%%%%%%%%%%%

Isovector meson states can be constructed with simple interpolating operators such as $[\bar{d}\Gamma u](x)$ obeying $[\bar{d}\Gamma u]^\dagger = \bar{d} \Gamma^\dagger u$ where $\Gamma$ is a Dirac spin matrix. 
LQCD calculations require an ensemble of $i=1,\cdots,N$ correlation functions $C_i^\Gamma$ calculated in QCD-vacuum importance sampled gauge field configurations whose average determines the QCD correlation function $G^\Gamma$.
With this choice of interpolating operators, isovector correlation functions are given in terms of quark propagators by
\begin{equation}
  \begin{split}
    G^\Gamma (t) &= \avg{C_i^\Gamma(t)} = \frac{1}{N}\sum_{i=1}^N C_i^\Gamma(t) \\ 
    &= \sum_\v{x} \int \mathcal{D}\bar{q}\mathcal{D}q\; e^{-\sum_y \bar{q}(y)D(U_i,y;y;)q(y)} [\bar{d}\Gamma u](\v{x},t) [\bar{u} \Gamma^\dagger d](0)\\
    &= \sum_\v{x} \tr\left[ S_u(U_i;\v{x},t;0)\Gamma^\dagger S_d(U_i;0;\v{x},t) \right]  \\
    &= \sum_\v{x} \tr\left[ S_u(U_i;\v{x},t;0)\Gamma^\dagger \gamma_5 S_d(U_i;\v{x},t;0)^\dagger \gamma_5 \Gamma \right] ,
  \end{split}\label{Cdef}
\end{equation}
where the trace is over spin and color indices and the last equality uses $\gamma_5$-Hermiticity.
For isoscalar mesons, single propagator traces corresponding to quark-line-disconnected diagrams also contribute to the correlation function. 
Disconnected diagrams have different statistical behavior than the quark-line-connected diagrams considered here, and a detailed discussion of the statistics and phase reweighting of disconnected diagrams is left to future work. 
In the isospin limit where $S_u = S_d \equiv S$,
\begin{equation}
  \begin{split}
    C_i^\Gamma(t)^\dagger = \sum_\v{x} \tr\left[ S(U_i;\v{x},t;0) \gamma_5 \Gamma S(U_i;\v{x},t;0)^\dagger \Gamma^\dagger \gamma_5  \right].
  \end{split}\label{eq:reality}
\end{equation}
Each meson correlation function in a Monte Carlo ensemble is therefore real, provided
\begin{equation}
  \gamma_5 \Gamma \gamma_5 = \pm \Gamma^\dagger,
  \label{g5Hermmat}
\end{equation}
which holds for $\Gamma \in \{\gamma_\mu, \gamma_5, \gamma_\mu \gamma_5\}$ but not for complex linear combinations such as $\gamma_1 + i \gamma_2$. 
Vector-meson correlation functions with interpolating operators corresponding to $J_z = \pm1$ are complex in a generic gauge field configuration while those corresponding to $J_z = 0$ are real, suggesting that reality of isovector meson correlation function is a property of a particular choice of interpolating operator and not a fundamental property of the state. 
One is always free to use complex linear combinations of interpolating operators in a QFT, so real correlation functions can generically be considered to be a special case of complex correlation functions.
Still, the heuristic picture of a complex phase taking a random walk on the unit circle from Chapter.~\ref{chap:statistics} is obscure when applied to signed real numbers and StN expectations associated with this picture should be investigated further.


The log-magnitude $R^\Gamma_i$ and phase $\theta^\Gamma_i$ associated with $C_i^\Gamma$ are formally defined by
Effective masses for the magnitude and phase can be defined as in Chapter~\ref{chap:statistics}. 
The phase effective mass $M^\Gamma_\theta = \ln\left(\avg{\cos \theta^\Gamma_i(t)}/\avg{\cos \theta^\Gamma_i(t+1)}\right)$ is determined by the average phase.
If there are $N_+$ positive sign correlation functions and $N_-$ negative sign correlation functions, then the average phase is simply
\begin{equation}
  \begin{split}
    \avg{e^{i\theta_\Gamma(t)}} = \avg{\cos(\theta_\Gamma(t))} = \frac{N_+ - N_-}{N},
  \end{split}\label{eq:counting}
\end{equation}
so the effective mass contribution of a pure sign is simply determined by a counting problem.
The average phase of Eq.~\eqref{eq:counting} for the $\rho^+$ correlation function is shown in Fig.~\ref{fig:RhoThCirc}.
At small times the average phase is nearly one.
At intermediate time the average phase decreases exponentially, and in the $m_\pi\sim 450$ MeV ensemble becomes close to zero at large times.


\begin{figure}[!ht] \centering
  \includegraphics[width=\columnwidth]{rhodecomp.png}
  \caption{Effective mass and bootstrap uncertainties for the rho magnitude-sign decomposition. The left plot shows $M^\rho(t)$ for the $m_\pi \sim 450$ MeV ensemble in blue and for the $m_\pi \sim 800$ MeV ensemble in black. The middle plot the magnitude effective mass $m_R^{\rho}(t)$ in orange for the $m_\pi \sim 450$ MeV ensemble and in purple for the $m_\pi \sim 800$ MeV ensemble. The right plot shows the phase (sign) effective mass $M_\theta^{\rho}(t)$ in green for the $m_\pi \sim 450$ MeV ensemble and brown for the $m_\pi \sim 800$ MeV ensemble. The middle plot includes red lines at $m_\pi$ for both pion masses, with the shorter line corresponding to the smaller $m_\pi \sim 800$ MeV lattice. The right-hand plot similarly includes red lines at $M_\rho - m_\pi$, with $M_\rho$ the central value from golden widow fits to the full correlation functions shown left.
  }
  \label{fig:RhoEM}
\end{figure}


Parisi-Lepage analysis predicts that all isovector mesons other than the pion face an exponentially hard StN problem,
\begin{equation}
  \begin{split}
    \text{StN}(C_i^\Gamma) \sim e^{-M_\Gamma^{StN}t} \sim \frac{\avg{C_i^\Gamma}}{\sqrt{\avg{|C_i^\Gamma|^2}}} \sim e^{-(M_\Gamma - m_\pi)t}.
  \end{split}\label{eq:mesonstn}
\end{equation}
The nucleon StN problem is identified as a sign problem above by noting that an exponentially-decaying phase inherently faces an exponentially hard StN problem and observing that the average phase of the nucleon correlation function decays at a rate equal to the StN-degradation rate.
Similarly, isovector meson correlation functions can be decomposed into a magnitude and phase (sign) as
\begin{equation}
  \begin{split}
    C^\Gamma_i(t) = e^{R^\Gamma_i(t) + i\theta^\Gamma_i(t)},
  \end{split}\label{eq:mesondef}
\end{equation}
and the Parisi-Lepage StN problem for mesons can be identified as a sign problem if
\begin{equation}
  \begin{split}
    \frac{\avg{e^{i\theta^\Gamma_i(t)}}}{\sqrt{\avg{\left|e^{i\theta^\Gamma_i(t)}\right|^2}}} = \avg{e^{i\theta^\Gamma_i(t)}} \sim e^{-M_\Gamma^{StN} t}.
  \end{split}\label{eq:phaseStN}
\end{equation}
Thermal artifacts associated with ``backwards-propagating'' states have more sizable effects on meson correlation function than baryon correlation functions
because in the meson cause the backwards states are degenerate with the forwards states, see e.g. Ref~\cite{Beane:2009kya},
and it is helpful to define an effective mass taking time reflection symmetry into account as
\begin{equation}
  \begin{split}
    M_\Gamma(t) = \text{arccosh}\left[ \frac{\avg{C_i^\Gamma(t+1)} + \avg{C_i^\Gamma(t-1)}}{2\avg{C_i^\Gamma(t)}} \right].
  \end{split}\label{mGamma}
\end{equation}
Effective masses for magnitude and phase contributions can be defined analogously as
\begin{equation}
  \begin{split}
    M_R^\Gamma(t) &= \text{arccosh}\left[ \frac{\avg{e^{R_i^\Gamma(t+1)}} + \avg{e^{R_i^\Gamma(t-1)}}}{2\avg{e^{R_i^\Gamma(t)}}} \right], \\ 
    M_\theta^\Gamma(t) &=  \text{arccosh}\left[ \frac{\avg{\cos \theta_i^\Gamma(t+1)} + \avg{\cos \theta_i^\Gamma(t-1)}}{2\avg{\cos \theta_i^\Gamma(t)}} \right].
  \end{split}\label{mGammaRTh}
\end{equation}
LQCD results for the $\rho^+$-meson are shown in Fig.~\ref{fig:RhoEM}, and the $\rho^+$ mass can be determined by fitting the correlation function or effective mass in the intermediate-time plateau region $t=13\rightarrow 20$ assuming ground-state dominance.
Results with $m_\pi \sim 450$ MeV show that at large times $M_R^\rho$ approaches $m_\pi$ with no StN problem and $M_\theta^\rho$ approaches $M_\rho - m_\pi$ with a severe StN problem.
Results for $m_\pi \sim 800$ MeV are consistent but do not show a visible plateau and indicate that a larger time direction is needed to observe the approach of the magnitude and phase effective masses to their expected asymptotic values.
Correlations between the magnitude and phase are negligible at small times but statistically significant at large times.
For $t \gtrsim 20$, the ratio
\begin{equation}
  \begin{split}
    \frac{\text{Cov}(e^{R_i^\Gamma},e^{i\theta_i^\Gamma})}{\avg{e^{R_i^\Gamma + i \theta_i^\Gamma}}} = 1 - \frac{\avg{e^{iR_i^\Gamma}}\avg{e^{i\theta^\Gamma_i}}}{G_\Gamma},
  \end{split}\label{eq:magphasecov}
\end{equation}
plateaus to a constant value of $0.319(14)(4)$.
This constant-time behavior is consistent with $M_\Gamma = M_R^\Gamma + M_\theta^\Gamma$, and explicit calculation shows that $M_\Gamma - M_R^\Gamma - M_\theta^\Gamma$ is consistent with zero for $t \gtrsim 20$.
These observations indicate that $M_\Gamma$ can be decomposed into magnitude and phase effective mass contributions, and that the meson StN problem arises from re-weighting the sign problem inherent to calculation of the average meson sign in complete analogy to the case of the nucleon phase.

The uncertainties associated with $M_\rho$, $M_R^\rho$, and $M_\theta^\rho$ are shown in Fig.~\ref{fig:RhoEMErrors}.
Agreement between LQCD results and Parisi-Lepage scaling predictions is visible at intermediate times,
but at the largest times the $m_\pi \sim 450$ MeV variance of $M_\rho$ stops growing exponentially and approaches a constant.
This unphysical time-dependence is consistent with observations of the nucleon above
and signals the onset of a noise region where the results of standard estimators systematically deviate from QCD.
The existence of such a noise region is predicted by circular statistics,
and its appearance in $\rho^+$-meson results suggests that the perspective of circular statistics might be useful for real but non-positive correlation functions.

\begin{figure}[!t] \centering
  \includegraphics[width=\columnwidth]{rhodecompvar.png}
  \caption{The left panel shows the variance of the effective mass $M_\rho(t)$ shown in Fig.~\ref{fig:RhoEM} determined by bootstrap resampling. Uncertainties on the results shown are determined by further bootstrap resampling of variance results. The middle panel shows the variance of $M_R^\rho(t)$ analogously, and the right panel shows the variance of $M_\theta^\rho(t)$. Results for $m_\pi \sim 450$ MeV and $m_\pi \sim 800$ MeV with $L=32$ are both shown and color-coled as in Fig.~\ref{fig:RhoEM}. The red lines in the left plot show predictions of exponential StN degradation satisfying Parisi-Lepage scaling $M_\rho^{StN} = M_\rho - m_\pi$, where the overall normalization has been fixed by one intermediate time chosen to be $t=22$.
  }
  \label{fig:RhoEMErrors}
\end{figure}

Since $e^{i \theta_i^\Gamma} \in \{\pm 1\}$ is a random variable with unit magnitude, $\theta_i^\Gamma$ can be interpreted as a discrete circular random variable with $\theta_i^\Gamma \in \{0,\ \pi\}$.
Standard theorems of circular statistics apply to discrete circular random variables~\cite{Fisher:1995,jammalamadaka2001topics,Mardia:2009},
including the result that standard parameter inference based on trigonometric moments generically breaks down due to finite-sample-size effects unless
\begin{equation}
  \begin{split}
    \frac{1}{N}\sum_{i=1}^N \cos\theta_i(t) > \frac{1}{\sqrt{N}}.
  \end{split}\label{eq:circstatsbound}
\end{equation}
It is derived explicitly in Chapter~\ref{chap:statistics} that Eq.~\eqref{eq:circstatsbound} must be met in order to reliably calculate the average phase of a random variable drawn from a wrapped normal distribution, which is empirically shown to provide a good fit to baryon correlation functions. 
Eq.~\eqref{eq:circstatsbound} holds for other common distributions in circular statistics, and on general grounds the statistical ensemble size necessary to distinguish any sufficiently broad circular distribution from a uniform distribution grows with the width of the distribution.
Fig.~\ref{fig:RhoThCirc} demonstrates explicitly that the average phase of $\rho^+$ correlation functions in LQCD begin deviating from exponential time-dependence and eventually reach a constant value.
The constant value eventually reached by the phase is seen in Fig.~\ref{fig:RhoThCirc} to decrease as $1/\sqrt{N}$ as the sample size is increased.
This verifies that Eq.~\eqref{eq:circstatsbound} applies to real but non-positive correlation functions with sign problems such as the $C_i^\Gamma$.
Similar results are expected to apply to any quantum Monte Carlo calculation with a sign problem.


\begin{figure}[!t] \centering
  \includegraphics[width=.45\columnwidth]{avcos.png} \hspace{10pt}
  \includegraphics[width=.45\columnwidth]{avcoslog.png}
  \caption{Results for the $\rho^+$ average phase $\avg{cos \theta_i^\rho}$, equal to the fractional difference between the number of positive and negative correlation functions as in Eq.~\eqref{eq:counting}. Results using the $m_\pi\sim 450$ MeV ensemble are shown in green on both plots, with a logarithmic scale used on right. Results for the $m_\pi \sim 800$ MeV ensemble are shown in purple on left. In addition to the green points corresponding to $N= 50,000$ correlation functions, there are also results shown in brown for a subset of $N=5000$ correlation functions and results shown in yellow for $N=500$ correlation functions. Corresponding lines in each color are shown at $1/\sqrt{N}$. The circular statistics bound of Eq.~\eqref{eq:circstatsbound} predicts that estimates of the average phase will be systematically biased for $\avg{\cos \theta_i^\rho} < 1/\sqrt{N}$.
    } 
  \label{fig:RhoThCirc}
\end{figure}


\section{Non-Positive Correlation Functions Moments}

The scaling of higher moments of meson correlation functions can be understood analogously to the Lepage-Savage scaling of higher moments of baryon correlation functions.
Even moments of baryon correlation functions $\avg{|C_i^N|^{2n}}$ are associated with states of zero baryon number, as described in Ref.~\cite{Beane:2014oea}.
In the absence of non-zero baryon number charge the lightest states contributing to $\avg{|C_i^N|^{2n}}$ will be multi-pion rather than multi-nucleon states. 
Conversely, odd moments $\avg{|C_i^N|^{2n}C_i^N}$ are associated with states containing one baryon as well as $3n$ pions described by a partial quenched theory of $nN_f$ valence quark flavors.
In general, nucleon correlation function moments have time dependence that in the absence of hadronic interactions is given by
\begin{equation}
  \begin{split}
    \avg{|C_i^N|^{2n} (C_i^N)^B} \sim e^{-3n m_\pi t - B M_N t}.
  \end{split}\label{eq:baryonLS}
\end{equation}
It follows that the real parts of baryon correlation functions have increasingly broad and symmetric distributions at large times.

For $\rho^+$ correlation functions, $G$-parity plays a role analogous to baryon number above and ensures that states with the quantum numbers of an odd number of $\rho^+$ mesons cannot be described as multi-pion states. 
Even moments of $C_i^\rho$ are not distinguished from multi-pion correlation functions by any conserved charge and therefore have large-time scaling controlled by the pion mass,
\begin{equation}
  \begin{split}
     \avg{|C_i^\rho|^{2n}}\sim e^{-2 n m_\pi},
  \end{split}\label{eq:rhoeven}
\end{equation}
By $G$-parity, odd moments of $C_i^\rho$ have large-time scaling influenced by the $\rho^+$-meson mass,
\begin{equation}
  \begin{split}
    \avg{|C_i^\rho|^{2n+1}} \sim e^{-2n m_\pi - M_\rho}.
  \end{split}\label{eq:rhoodd}
\end{equation}
As in the nucleon case, QCD inequalities ensuring the pion is the lightest state in the QCD spectrum  guarantee that $\rho^+$ correlation functions become increasingly broad and symmetric at large times~\cite{Weingarten:1983uj,Witten:1983ut}. 
Other single-meson states distinguishable from single-pion states by some quantum number (parity for the $a_0$ and $a_1$; charge conjugation for the $b_1$) similarly have even moments whose large-time decay is set by the pion mass and odd moments who large-time decay is set by the mass of the meson in question.
Isovector meson correlation functions other than pion correlation functions are therefore increasingly broad and symmetric at large times. 

Lepage-Savage scaling can also be proven generically for the real parts of complex correlation functions.
Consider a correlation function ensemble $C_i$ with $Q$ quark lines and ground-state mass $M$ satisfying $M \geq \frac{Q}{2}m_\pi$, and for the purposes of estimating large-time scaling ignore interactions between hadrons.
Applying trigonometric identities relating $\cos^n(\theta)$ and $\cos(n\theta)$ to the phase gives
\begin{equation}
  \begin{split}
    \avg{(\text{Re} C_i)^n} &= \avg{e^{nR_i}\cos^n(\theta_i)} \\
        &= \begin{cases} \frac{1}{2^{n-1}}\sum_{k=0}^{(n-1)/2} {n \choose k} \avg{e^{n R_i}\cos\left( (n-2k)\theta_i \right)} , 
        & n\text{ odd} \\ 
        \frac{1}{2^n}{n\choose n/2} \avg{e^{n R_i}} + \frac{1}{2^{n-1}}\sum_{k=0}^{(n-2)/2} {n \choose k} \avg{e^{n R_i}\cos\left( (n-2k)\theta_i \right)} , 
        & n\text{ even} \end{cases}.
    \end{split}\label{eq:LSgeneral1}
\end{equation}
Each term in the above sums can be re-grouped as 
\begin{equation}
  \begin{split}
    \avg{e^{2k R_i}e^{(n-2k)R_i}\cos( (n-2k) \theta_i)} = \avg{|C_i|^{2k}C_i^{n-2k}} \sim e^{-k Q m_\pi t - (n-2k) M t},
  \end{split}
\end{equation}
where the right-most relation holds because $|C_i|^{2k}$ contains $kQ$ conserved quarks and $kQ$ separately conserved antiquarks and therefore has time-dependence $\avg{|C_i|^{2k}} \sim e^{-k Q m_\pi t}$.
Since $M \geq \frac{Q}{2}m_\pi$, terms with larger $k$ decay exponentially slower with increasing time than terms with smaller $k$.
The sum is therefore dominated by the term with largest $k$.
For odd $n$, the dominant term at large times has $k= (n-1)/2$ and decays as $e^{-(\frac{n-1}{2}) Q m_\pi t - M t}$,
while for even $n$, the dominant term has $k = (n-2)/2$ and decays as $e^{-(\frac{n}{2} - 1) Q m_\pi t - 2 M t}$.
The additional term besides the sums appearing in Eq.~\eqref{eq:LSgeneral1} dominates for even $n$, giving
\begin{equation}
  \begin{split}
    \avg{C_i^n}  &\sim \begin{cases} e^{-\frac{(n-1)Q}{2}m_\pi t - M t} , & n\text{ odd} \\
      e^{-\frac{Q n}{2} m_\pi t} , & n\text{ even} \end{cases}.
  \end{split}\label{eq:LSgeneral}
\end{equation}
It follows that the real parts of complex correlation functions for generic hadrons in LQCD become increasingly broad and symmetric at late times, with ratios of odd moments to even moments decreasing at a rate fixed (up to hadronic interaction energy shifts) by the number of valence quarks fields used to construct the correlation function.
Analogous results apply to complex or real but non-positive correlation functions in quantum Monte Carlo calculations more generally.
%LQCD results exhibiting the scalings predicted by Eq.~\eqref{eq:LSgeneral} are shown in Fig.~

%\section{Non-Positive Correlation Function Distributions}
%
%The $\rho^+$ magnitude is less-well described by a log-normal distribution than the nucleon, and the $\rho^+$ log-magnitude is better fit by a stable distribution with statistically significant skew as shown in Fig.~\ref{Rho450RHistograms}.
%The same property are found if one considers the distribution of magnitudes of real parts of nucleon correlation functions rather than the distribution of magnitudes of full nucleon correlation functions.
%For the baryon case, this skewness arises because the magnitude (absolute value) of the real part is not equal to the magnitude, and in particular
%\begin{equation}
%  \begin{split}
%    \ln |\text{Re} C_i|^2 = \frac{1}{2}\ln \left( \frac{1}{4}C_i^2 + \frac{1}{4}(C_i^\dagger)^2 + \frac{1}{2}|C_i|^2  \right).
%  \end{split}\label{eq:remag}
%\end{equation}
%Since the contributions from $(C_i^N)^2$ are exponentially smaller that the contributions from $|C_i^N|^2$, the net result should be a distribution peaked close to $-{3}{2}m_\pi$ but skewed towards larger negative values.  
%This expectation matches the observations of Fig.~\ref{Rho450Histograms}.  
%It is not obvious that similar results should apply to the $\rho^+$ magnitude distribution because $|C_i^\Gamma|^2 = (C_i^\Gamma)^2$, yet similar similar skewness is observed for large-time distributions of $\ln|C_i^\rho|$ as for $\ln(|\text{Re} C_i^N|)$ in Fig.~\ref{Rho450Histograms}.
%
%\begin{figure}[!ht]
%  \centering
%  \includegraphics[width=\columnwidth]{MF5.png}
%  \caption{
%    Normalized histograms of $R_{\gamma_z}(t)$  for  $m_\pi \sim 450$ MeV. Blue curves correspond to best-fit normal distributions, purple curves to best fit stable distributions.
%  }
%  \label{Rho450RHistograms}
%\end{figure}
%
%
%This suggests that at large times $C_i^\Gamma$ statistically behaves like the real part of a complex correlation function with whose magnitude receives roughly equal contributions from real and imaginary parts.
%Since baryon correlation functions were shown in Chapter~\ref{chap:statistics} to have similar time dependence as an uncorrelated product of a log-normal magnitude and a wrapped-normal phase factor,
%meson correlation functions should be described by the real part of this distribution.
%To define this distribution concretely, suppose $X + i Y = e^{R + i\theta}$ is a \emph{complex-log-normal} random variable, that is a complex product of a log-normally distributed magnitude and a (symmetric) wrapped-normally distributed phase factor,
%\begin{equation}
%  \begin{split}
%    \mathcal{P}(R, \theta; \mu_R, \sigma_R, \sigma_\theta) = \sum_{n=-\infty}^\infty \frac{1}{(2\pi)^{3/2}\sigma_R}e^{-(R - \mu_R)^2/(2\sigma_R^2) - n^2 \theta^2 /(2 \sigma_\theta^2)}.
%  \end{split}\label{PCLN}
%\end{equation}
%The probability distribution for the real part of a complex-log-normal random variable, marginalized over the imaginary part, is then
%\begin{equation}
%  \begin{split}
%    \mathcal{P}(X; \mu_R, \sigma_R, \sigma_\theta) = \frac{1}{(2\pi)^{3/2}}\int_{-\infty}^\infty \frac{dY}{X^2 + Y^2}\sum_{n=-\infty}^\infty e^{-(\sqrt{X^2+Y^2} - \mu_R)^2/(2\sigma_R^2) - n^2 \arg(X+i Y)^2/(2\sigma_\theta^2)}.
%  \end{split}\label{PCLNX}
%\end{equation}
%The sum can be expressed as an elliptic-$\vartheta$ function by Poisson summation and the resulting integral and be calculated numerically.
%Given a sample of random variables $X_i + i Y_i = e^{R_i + i\theta_i}$, parameter inference for the complex log-normal can be performed analytically,
%\begin{equation}
%  \begin{split}
%    \mu_R &= \frac{1}{N}\sum_{i=1}^N R_i, \\
%    \sigma_R^2 &= \frac{1}{N}\sum_{i=1}^N (R_i - \mu_R)^2, \\
%    \sigma_\theta^2 &= - \ln\left[ \left(\frac{1}{N}\sum_{i=1}^N \cos \theta_i \right)^2 + \left(\frac{1}{N}\sum_{i=1}^N \sin \theta_i \right)^2 \right].
%  \end{split}\label{eq:CLNparams}
%\end{equation}
%Complex-log-normal distributions with parameters computed by Eq.~\eqref{eq:CLNparams} give a good fit to LQCD baryon correlation functions at all times, as discussed in Chapter~\ref{chap:statistics}.
%
%Complex-log-normal distributions with parameters computed by Eq.~\eqref{eq:CLNparams} do not give a good fit to $\rho^+$-meson correlation functions,
%as seen in Fig.~\ref{fig:fixed_hist}.
%The same poor agreement is seen if one applies Eq.~\eqref{eq:CLNparams} to the real parts of nucleon correlation functions rather than to the full complex correlation function.
%This suggests that the isovector meson correlation functions at hand are ``missing'' an uncorrelated imaginary part of equal magnitude.
%Such an imaginary part can be constructed by considering complex linear combinations of random pairs of vectors from the ensemble,
%\begin{equation}
%  \begin{split}
%    C_{i^\prime}^\Gamma(t) &= C_i(t) + i C_j(t),\hspace{20pt} i = 1,\cdots,\frac{N}{2},\ j = \frac{N}{2}+1,\cdots,N.
%  \end{split}\label{eq:Ciprime}
%\end{equation}
%The log-magnitude $R_{i^\prime}$ of $C_{i^\prime}$ is found to be much more normally distributed than $R_i$,
%and in particular appears symmetric at large times.
%Further, applying Eq.~\eqref{eq:CLNparams} to $C_{i^\prime}$ provides a complex-log-normal distribution whose real part is in good agreement with results for $\text{Re}C_{i^\prime}$ and therefore with results for $\text{Re}C_i$,
%as shown in Fig.~\ref{fig:fixed_hist}.
%
%
%
%
%
%

%\section{Pion Statistics}
%
%The pion is quite different, because taking $\Gamma = \gamma_5$ in Eq.~\eqref{eq:reality} gives
%\begin{equation}
%  \begin{split}
%    C^{\gamma_5}_i(t) &= \sum_\v{x} \tr\left[ S_u(\v{x},t;0;U) S_d(\v{x},t;0;U)^\dagger \right],
%  \end{split}\label{Cg5def}
%\end{equation}
%which in the limit of degenerate up and down quarks is just the sum of the propagator eigenvalue magnitudes squared.
%The pion correlation function is positive definite in the isospin limit, and has exactly zero StN problem associated with a sign problem.
%It is still possible for the pion to have a StN problem arising from interactions; this occurs anytime the lowest energy in the two-pion state setting the exponential decay of the variance might differ from the energy of two non-interacting pions.
%Such an effect is visibly apparent in the $m_\pi \sim 800$ MeV ensemble, where precision is seen to increase with time.
%\begin{equation}
%  \begin{split}
%    m_{\pi}^{StN} &= \begin{cases} 0.002(3)  & m_\pi \sim 450 \text{ MeV}  \\ -0.023(4) & m_\pi \sim 800 \text{ MeV} \end{cases} \\\\
%    &= \begin{cases}  3(5) \text{ MeV} & m_\pi \sim 450 \text{ MeV}  \\ -31(6) \text{ MeV} & m_\pi \sim 800 \text{ MeV} \end{cases}
%  \end{split}\label{mpiStN}
%\end{equation}
%It is noteworthy that interactions provide a mechanism for changing exponential StN scaling distinct from the StN behavior associated with complex observable average phases.
%In particular, StN problems associated with negative interaction are not affected by phase reweighting.
%
%
%\begin{figure}[!ht] \centering
%  \includegraphics[width=\columnwidth]{MF6.png}
%  \caption{Effective mass and uncertainties for the pion. The left plots shows $m_{\gamma_5}(t) = m_\pi(t)$ for the $m_\pi \sim 450$ MeV ensemble in orange and for the $m_\pi \sim 800$ MeV ensemble in purple. The middle plot shows a close-up of $m_\pi(t)$ for only the $m_\pi \sim 450$ MeV ensemble. The plateau from $15 \leq t \leq 81$ is shown as a shaded band whose width is equal to the statistical uncertainty of a constant fit in this region. The right plot shows the variance of the effective mass for $450$ MeV and $800$ MeV in orange and purple respectively. In all cases, standard errors are calculated with bootstrap methods.
%  }
%  \label{PionEM}
%\end{figure}
%
%
%The sign and magnitude of the pion StN problem depend on the nature of interactions in the variance correlation function.
%The $\pi^+$ variance correlation function receives contribution from just one of the contractions contribution to the $\pi^+ - \pi^-$ correlation function. 
%The ``gluon-exchange'' diagram contributes, but the ``quark-exchange'' diagram is absent from the $\pi^+$ variance since it is just the product of one-pion contractions averaged over gauge fields. 
%These diagrams can be analyzed in $\chi$PT and the gluon-exchange diagram is predicted to vanish at leading order~\cite{Sharpe:1992pp,Gupta:1993rn}.
%This suggests the pion should have very mild StN scaling associated with interaction energy shifts.
%
%
%
%With the benefit of 25 years of LQCD progress since Ref.~\cite{Guagnelli:1990jb}, we can revisit these seminal studies of the distribution of the pion correlation function with 1000 times the statistics on volumes 8 times larger with 3 times larger time directions, not to mention dynamical rather than quenched fermions.
%
%Much more is visible on volumes that 
%
%, consistent with observables of very mild StN scaling~\cite{Sharpe:1992pp,Gupta:1993rn}.
%
%
%
%
%% ADD DISCUSSION OF VOLUME-SCALING
%%\cite{Hamber:1983vu,Guagnelli:1990jb}
%
%It was noted for the nucleon in Chapter~\ref{chap:statistics} that both the log-magnitude and phase evolve in time with random walk behavior in the sense that their time derivatives tend towards time-independent distributions at large time. On timescales largeer than $1/m_\pi$ where correlation functions evaluated at different points are approximately decorrelated, the time evolution of the log-magnitude and phase can therefore be viewed as repeatedly taking approximately independent, identically distributed steps in a random walk. The distribution for the random steps is this large-time distribution for the time derivative. Study of the nucleon showed further that the time derivative distributions are heavy tailed, non-Gaussian stable distributions that lead to random walks described by superdiffusive L{\'e}vy flights rather than Brownian motion.
%
%Histograms of the $\rho^+$ log-magnitude velocity are shown in Figs.~\ref{Rho450dRdtHistograms}-\ref{Rho800dRdtHistograms}. After small times, heavy-tails clearly develop in the distribution that are well-described by a stable distribution but poorly described by a Gaussian. At large times, the distribution appears to approach a time-independent shape that is symmetric but heavy-tailed, as in the case of the nucleon. This approach to a time independent distribution allows us to again interpret the log-magnitude as taking a random walk across hadronic timescales where individual steps can be considered nearly decorrelated. 
%
%\begin{figure}[!ht]
%  \centering
%  \includegraphics[width=.3\columnwidth]{Pictures/Histogram_Rho450dRdt_8.pdf}\hspace{10pt}
%  \includegraphics[width=.3\columnwidth]{Pictures/Histogram_Rho450dRdt_20.pdf} \hspace{10pt}
%  \includegraphics[width=.3\columnwidth]{Pictures/Histogram_Rho450dRdt_32.pdf} \hspace{10pt}
%  \caption{
%    Histograms of $dR_{\gamma_z}/dt$ for the $m_\pi \sim 450$ MeV ensemble.
%  }
%  \label{Rho450dRdtHistograms}
%\end{figure}
%\begin{figure}[!ht]
%  \centering
%  \includegraphics[width=.3\columnwidth]{Pictures/Histogram_Rho800dRdt_8.pdf}\hspace{10pt}
%  \includegraphics[width=.3\columnwidth]{Pictures/Histogram_Rho800dRdt_14.pdf} \hspace{10pt}
%  \includegraphics[width=.3\columnwidth]{Pictures/Histogram_Rho800dRdt_20.pdf} \hspace{10pt}
%  \caption{
%    Histograms of $dR_{\gamma_z}/dt$ for the $m_\pi \sim 800$ MeV ensemble.
%  }
%  \label{Rho800dRdtHistograms}
%\end{figure}
%
%Maximum likelihood fits of the log-magnitude velocity to a stable distribution are shown in Fig.~\ref{RhodRdtStable}. These fits again suggest that the $m_\pi \sim 800$ MeV data is approaching a similarly shaped large-time distribution to the $m_\pi \sim 450$ MeV data but has insufficient time to plateau. For the $m_\pi \sim 450$ MeV ensemble, it is possible to clearly identify a large-time plateau in stable distribution parameter estimates. It is noteworthy that similar structures are seen in $m_\pi \sim 450$ MeV and $m_\pi \sim 800$ MeV data, and it is plausible that the asymptotic value of $\mu$ is sensitive to the value of the quark mass while the other parameters are largely insensitive. 
%Results for large-time parameters for the $m_\pi \sim 450$ MeV ensemble are shown in Fig.~\ref{RhoDeltaRStable}. The $\rho^+$ is more heavy-tailed than the nucleon, with
%
%\begin{equation}
%  \begin{split}
%    \alpha\left( \Delta R_\rho(t\rightarrow \infty, \Delta t \sim 0.12\text{ fm}) \right) \rightarrow \begin{cases} 1.276(2)(1), & m_\pi \sim 450 \text{ MeV}. \\
%       \end{cases},
%  \end{split}
%  \label{RhoalphaDeltaR}
%\end{equation}
%
%\begin{figure}[!ht]
%  \centering
%  \includegraphics[width=.2\columnwidth]{Pictures/RhodRdtAlpha.pdf} \hspace{10pt}
%  \includegraphics[width=.2\columnwidth]{Pictures/RhodRdtBeta.pdf}\hspace{10pt}
%  \includegraphics[width=.2\columnwidth]{Pictures/RhodRdtMu.pdf}\hspace{10pt}
%  \includegraphics[width=.2\columnwidth]{Pictures/RhodRdtGamma.pdf}\hspace{10pt}
%  \caption{
%    Maximum likelihood estimates for maximum likelihood fits of $\frac{dR_{\gamma_5}}{dt}$ to the stable distribution parametrized in Eq.~\eqref{PSdef}-\eqref{PhiSdef}.
%  The associated uncertainties are estimated via bootstrap.
%}
%  \label{RhodRdtStable}
%\end{figure}
%
%\begin{figure}[!ht]
%  \centering
%  \includegraphics[width=.45\columnwidth]{Pictures/RhoDeltaRAlpha.pdf}\hspace{10pt}
%  \includegraphics[width=.45\columnwidth]{Pictures/RhoDeltaRGamma.pdf}\hspace{10pt}
%  \caption{
%    Maximum likelihood estimates for the index of stability, $\alpha\left(\Delta R_{\gamma_z}(t, \Delta t)\right)$ and width $\gamma\left( \Delta R_{\gamma_z}(t, \Delta t) \right)$, in the large-time 
%  plateau region as a function of $\Delta t$ for the $m_\pi \sim 450$ MeV ensemble. Analogous results for the $m_\pi \sim 800$ MeV ensemble would require new simulations with a largeer time direction.
%  }
%  \label{RhoDeltaRStable}
%\end{figure}
%
%
%For the pion conversely only very mild deviations from the $\alpha=2$ case of a Gaussian distribution can be seen. This is consistent with the intuitive picture of heavy-tails arising from strong interactions. The pion is a relatively weakly interacting particle due to its Goldstone nature, and so we expect it to be much closer to the $\alpha = 2$ limit of Brownian motion that the strongly interacting and very heavy-tailed nucleon log-magnitude.  In accordance with the large time random walk picture, the stable distribution fit parameters for $dR_{\gamma_5}/dt$ plateau to constant values at large times. The index of stability is given by
%\begin{equation}
%  \begin{split}
%    \alpha\left( \Delta R(t\rightarrow \infty, \Delta t \sim 0.12\text{ fm}) \right) \rightarrow \begin{cases} 1.9654(38)(31) & m_\pi \sim 450 \text{ MeV} \\
%      1.965(4)(3), & m_\pi \sim 800 \text{ MeV} \end{cases},
%  \end{split}
%  \label{PionalphaDeltaR}
%\end{equation}
%which should be contracted with $\alpha =  1.639(4)(1)$ for the nucleon at $m_\pi \sim 450$ MeV and the $\rho^+$ results above. 
%Results for the large-time values of the index of stability and width are shown for a variety of $\Delta t$ in Fig.~\ref{PionDeltaRStable}. The index of stability is nearly flat and nearly Gaussian for all $\Delta t$ considered. The width grows sub-linearly with $\Delta t$ in both cases, but noticeably faster for the heavier quark mass ensemble. It is noteworthy that the widths of the pion distributions at these two different quark masses nearly coincide at $\Delta t = 1$ but have distinct scaling with $\Delta t$.
%\begin{figure}[!ht]
%  \centering
%  \includegraphics[width=.45\columnwidth]{Pictures/PionDeltaRAlpha.pdf}\hspace{10pt}
%  \includegraphics[width=.45\columnwidth]{Pictures/PionDeltaRGamma.pdf}\hspace{10pt}
%  \caption{
%    Maximum likelihood estimates for the index of stability, $\alpha\left(\Delta R_{\gamma_5}(t, \Delta t)\right)$ and width $\gamma\left( \Delta R_{\gamma_5}(t, \Delta t) \right)$, in the large-time 
%  plateau region as a function of $\Delta t$.  Associated uncertainties are estimated with bootstrap methods.
%  }
%  \label{PionDeltaRStable}
%\end{figure}
%
%
%


\section{Phase Reweighting Non-Positive Correlation Functions}


\begin{figure}[!t]
  \centering
  \includegraphics[width=\columnwidth]{RhoRatio2DEM.png} 
  \caption{Results for $\tilde{M}_\rho(t,\Delta t)$ with $m_\pi\sim 450$ MeV, left, and with $m_\pi \sim 800$ MeV, right.}
  \label{Rho2DWEM}
\end{figure}

The previous sections suggest that real but non-positive meson correlation functions have similar statistical behavior to the real parts of baryon correlation functions.
A significant difference between (real) meson and (complex) baryon correlation functions is that ratio-based estimators sampling $C_i(t)/C_i(t-\Delta t)$ analogous to those introduced in Chapter~\ref{chap:statistics} do not remove exponential StN degradation from meson correlations, as shown in
Fig.~\ref{Rho2DWEM}.
The correlation-function-ratio effective mass 
\begin{equation}
  \begin{split}
    \tilde{M}_\Gamma = \ln\left[ \frac{\avg{C_i(t)}}{\avg{C_(t-\Delta t)}} \right] - \ln\left[ \frac{\avg{C_i(t+1)}}{\avg{C_i(t-\Delta t)}}\right]
  \end{split}\label{MGammadef}
\end{equation}
is noisy for $t\gtrsim 15$ when the sign begins contributing appreciably to the effective mass.
A similar problem arises for estimators sampling ratios of the real parts of nucleon correlation functions.
For the real part of the nucleon correlation function, this can be understood because
\begin{equation}
  \begin{split}
    \frac{\text{Re}C_i(t)}{\text{Re}C_i(t-\Delta t)} = e^{R_i(t) - R_i(t-\Delta t)}\left( \frac{e^{i\theta_i(t) - i\theta_i(t-\Delta t)} + e^{-i\theta_i(t) - i \theta_i(t-\Delta t)}}{1 + e^{-2i\theta(t-\Delta t)}} \right).
  \end{split}\label{eq:reratio}
\end{equation}
The first term in the numerator of Eq.~\eqref{eq:reratio} samples a phase difference $e^{i \Delta \theta_i(t, \Delta t)}$ associated with a $\Delta t$ step random walk of the phase.
The second term includes a sum of phases that
constructively rather than destructively interfere
and should have an exponential StN problem in $t$.
It is not obvious that ratios of intrinsically real but non-positive correlation functions must behave similarly,
but Fig.~\ref{Rho2DWEM} shows clear evidence for such an exponential StN problem in $t$.


\begin{figure}[!t]
  \centering
  \includegraphics[width=\columnwidth]{RhoRTh2DEM.png} 
  \caption{Results for the $\rho$-meson magnitude and phase ratio-based effective masses $\tilde{M}_R^\rho$, left, and $\tilde{M}_\theta^\rho$, right.}
  \label{Rho2DRThWEM}
\end{figure}

As shown in Fig.~\ref{Rho2DRThWEM}, large noise at large $t$ and small $\Delta t$ in correlation-function-ratio effective masses for $C_i^\rho(t) / C_i^\rho(t-\Delta t)$ is also present in the effective mass for $|C_i^\rho(t)| / |C_i^\rho(t-\Delta t)|$ but not in the effective mass for $e^{i\theta_i^\rho(t) - i\theta_i^\rho(t-\Delta t)}$.
This suggests that explicitly reweighting $C_i^\rho(t)$ by the inverse phase $e^{-i\theta_i(t-\Delta t)}$ (the sign of $C_i(t-\Delta t)$)
faithfully represents a random walk of length $\Delta t$ for the phase,
and motivates the introduction of phase-reweighted correlation functions as defined Chapter~\ref{chap:PR},
\begin{equation}
  \begin{split}
    G^\theta_\Gamma(t) = \avg{ C_i^\Gamma (t) e^{-i\theta_i^\Gamma(t)}} .
  \end{split}\label{eq:mesonPR}
\end{equation}
%Intuitively, multiplying by $C_i^{-1}(t-\Delta t)$ removes exponential StN degradation for $C_i(t)$ by cancelling all phase fluctuations accrued from time 0 to $t-\Delta t$ so that the variance only corresponds to $\Delta t$ steps in the L{\'e}vy of the phase. 
%Phase reweighting introduces a factor of $e^{-i\theta_i(t-\Delta t)}$ for the destructive phase interference, but does not include the heavy-tailed random variable $|C_i(t-\Delta t)^{-1}|$.
To partially account for thermal artifacts associated with backwards-propagating states, a symmetrized phase-reweighted effective mass is employed for mesons that is defined as
\begin{equation}
  \begin{split}
    M_\Gamma^\theta(t,\Delta t) = \text{ArcCosh} \frac{G_\Gamma^\theta(t - 1, \Delta t - 1) + G_\Gamma^\theta(t+1, \Delta t + 1)}{G_\Gamma^\theta(t, \Delta t)}.
  \end{split}\label{MGammaACdef}
\end{equation}
As shown in the $\rho^+$-meson results of Chap.~\ref{chap:PR}, phase reweighting tames the StN problem for real but non-positive meson correlation functions as well as complex baryon correlation functions.


%$|C^{-1}(t-\Delta t)|$ is a heavy-tailed, approximately log-normal random variable with mean and variable increasing exponentially in time, and from a statistical point of view it is plausible that $|C_i^{-1}(t-\Delta t)|$ might add significant variance to improved estimators.

\begin{figure}[!t]
  \centering
  \includegraphics[width=\columnwidth]{Rho2DEM.png} 
  \caption{Results for $M_\rho^\theta(t,\Delta t)$ for the $m_\pi \sim 450$ MeV ensemble left and $m_\pi\sim 800$ MeV ensemble, left. Results should be compared with those for $\tilde{M}(t,\Delta t)$ in Fig.~\ref{Rho2DWEM}.}
  \label{Rho2DEM}
\end{figure}

%The magnitude of a phase-reweighted correlation function $G^\theta_\Gamma$ is identical to the magnitude of the corresponding QCD correlation function $G_\Gamma$.
%Still, the corresponding phase-reweighted effective mass $M^\theta_\Gamma$ is not equal to the true effective mass $M_\Gamma$ at large $t$ and small $\Delta t$.
%Bias in $M^\theta_\Gamma$ can be thought of as arising from excited-state contributions that are exponentially suppressed by $\Delta t$ but independent of $t$, as described in Chapter~\ref{chap:PR}.
%The magnitude is associated with pion physics and therefore near-zero modes of the Dirac operator, and it is possible that excitations in the magnitude decay on slower time scales than
%excitations in the phase.
%Ground-state saturation might be achieved by considering phase reweighted correlation functions at very large $t$ and extrapolating towards unbiased $\Delta t \rightarrow t$ results using data at modest $\Delta t$, even in cases where extrapolations along the $t=\Delta t$ line corresponding to the standard correlation function are overwhelmed by noise before reaching unbiased ground state results.
%For the $\rho^+$ meson, the results of Chap.~\ref{chap:PR} are consistent with this argument.
%The magnitude does not reach ground state saturation until $t \gtrsim 25$, as shown in Fig.~\ref{fig:PRrho}, but ground state saturation for $m^\theta_\rho$  for $\Delta t \gtrsim 6$.

\begin{figure}[!t]
  \centering
  \includegraphics[width=\columnwidth]{PRa0.png} \\
  \includegraphics[width=\columnwidth]{PRa1.png} \\
  \includegraphics[width=\columnwidth]{PRb1.png}
  \caption{Results for $M_\Gamma(t)$ and $M_\Gamma^\theta(t,\Delta t)$ for the $m_\pi \sim 450$ MeV ensemble for the $a_0$, top, $a_1$, middle, and $b_1$, channels. The left panel shows traditional effective mass $M_\Gamma(t)$ results. The middle panel shows $M_\Gamma^\theta(t,\Delta t)$ as a function of $t$ for a variety of fixed $\Delta t = 1 \rightarrow 9$. The right panel shows large-$t$ plateau results for $M_\Gamma^\theta$ as a function of $\Delta t$ and constant-plus-exponential fits used to extract the $\Delta t \rightarrow t \rightarrow \infty$ mass as in Chapter~\ref{chap:PR}.
    }
  \label{fig:PRmesons}
\end{figure}


\begin{figure}[!t]
  \centering
  \includegraphics[width=\columnwidth]{PRrho24.png} \\
  \includegraphics[width=\columnwidth]{PRrho32.png} \\
  \includegraphics[width=\columnwidth]{PRrho48.png}
  \caption{Results for $M_\rho(t)$ and $M_\rho^\theta(t,\Delta t)$ for the $m_\pi \sim 450$ MeV ensemble similar to those of Fig.~\ref{fig:PRmesons} are shown for three different spacetime volumes: $48^3\times 96$, top, $32^3\times 96$, middle, and $24^3\times 64$, bottom. Different sized statistical ensembles are used for each volume, see the main text for more details.}
  \label{fig:PRrho}
\end{figure}

Ground-state saturation is not achieved in standard analyses of isovector meson channels besides the $\pi$ and $\rho^+$ in the LQCD calculations considered here, as seen in Fig.~\ref{fig:PRmesons}.
This suggests that simple $\bar{q}\Gamma q$ interpolating operators have poor onto the signal ground state relative to their overlap onto the noise ground state in these channels.
While this obstructs standard determinations of the ground-state energies of these channels, ground-state saturation of phase-reweighted effective masses occurs after the magnitude and the variance correlation function (dominated by $\avg{|C_i|^2}$) have reached their ground states.
Phase-reweighted ground-state saturation improves with larger overlap onto the variance ground state and should not be obstructed by small relative signal-ground-state overlap to variance-ground-state overlap in $G_\Gamma$.
%This suggests the entire ground-state isovector meson spectrum might be accessible to phase-reweighted calculations with large $t \gtrsim 25$ and modest $\Delta t$, as was found to reproduce known results for the $\rho^+$.
Figs.~\ref{fig:PRmesons} shows that phase-reweighted ground state saturation can be achieved in all isovector meson channels with large $t\gtrsim 25$ and modest $\Delta t \gtrsim 6 - 10$.

Fig.~\ref{fig:PRrho} shows phase-reweighted results for the $\rho^+$-meson employing the three spacetime volumes $L^3 \times \beta$ of dimension $24^3 \times 64$, $32^3\times 96$, and $48^3\times 96$ with $m_\pi \sim 450$ MeV.
Smaller uncertainties on the $L=48$ ensemble arise from higher statistics $N=600,000$ (including time reversed correlation functions) compared to the $L=32$ ensemble with $N=40,000$.
The $L=24$ ensemble includes a very large sample size of $N=2,880,000$, but
gains less statistical precision by the spatial average involved in momentum projection than the larger volumes.
Larger uncertainties compared to the $L=32$ ensemble also arise from the smaller time direction $\beta = 64$, which restricts the region of usable phase-reweighted correlation function data with $t\gtrsim 25$ to a small number of points.
The fraction of timeslices after variance-ground-state saturation and therefore useful for phase-reweighted calculations increases as the time extent of the lattice is increased.
Since there is no maximum $t$ where phase-reweighted results become overwhelmed by noise, phase-reweighted calculations could be performed more precisely on lattices with larger time extents.

\begin{figure}[!t]
  \centering
  \includegraphics[width=.6\columnwidth]{meson_spec.png}
  \caption{A compilation of $M_\rho^\theta$ results for the $m_\pi \sim 450$ MeV ensemble and all three spatial volumes. Smaller error bars show statistical uncertainty, while large error bars show statistical and systematic uncertainties added in quadrature. Black dashed lines show $2m_\pi$ and $3m_\pi$ for reference, while dashed lines in each color show the non-interacting $p$-wave energy shift $\sqrt{(2m_\pi)^2 + (2\pi/L)^2}$ for the corresponding volume.}
  \label{fig:mesonspec}
\end{figure}

Ground-state isovector meson mass results in all channels from the three volumes are compiled in Fig.~\ref{fig:mesonspec}.
%Several interesting results on the phenomenology of QCD with $m_\pi \sim 450$ MeV are suggested. 
$M_\rho$ is found to be volume-independent within uncertainties over the range of volumes considered here.
In each case $M_\rho$ is below the non-interacting $p$-wave scattering state energy $\sqrt{ (2m_\pi)^2 + (2\pi/L)^2}$ expected for a finite volume state that would associated with an unbound $\pi\pi$ resonance in infinite volume.
The $\rho^+$ is still above the infinite-volume $\pi\pi$ threshold, $M_\rho > 2m_\pi$, and more detailed studies of $\pi\pi$ scattering are required to assess the resonant nature of the $\rho^+$ at $m_\pi \sim 450$ MeV.
Studies by the Hadspec Collaboration in Refs.~\cite{Dudek:2012gj,Dudek:2012xn}
%Detailed resonance studies by the Hadspec Collaboration indicate that
indicate that the $\rho^+$ is slightly unbound at $m_\pi \sim 391$ MeV.
%with $M_\rho = 864(19)(6) MeV$
Other meson masses are found to be volume-independent over the range of volumes considered, consistent with the scalings expected for compact bound states.
At these values of the quark masses $M_{a_0} < 2m_\pi$, and therefore the $a_0$ is lighter than the (not precisely known) $K^0\overline{K}^0$ and $\pi\eta$ thresholds.
Furhter studies combining variantional methods, resonance formalism, and phase reweighting are needed
to asses whether the $a_0$-meson is bound at $m_\pi \sim 450$ MeV and better understand light hadron phenomenology with heavier than physical quark masses.


These results demonstrate that phase reweighting can be used to predict ground-state energies of correlation functions without a GW.
The precision of phase reweighted results is limited by the size of the lattice time direction, and in particular no useful results are found for the $m_\pi \sim 800$ MeV ensembles with smaller time directions.
This motivates the generation of LQCD ensembles with very large time directions
where phase reweighting may provide ground-state energy results
for multi-nucleon systems without a GW.
The results of this chapter also suggest that phase reweighting may be applied to real but non-positive correlation functions that appear in GFMC and other many-body methods applicable to particle, nuclear, and condensed matter physics as well as to complex correlation functions describing multi-baryon systems in LQCD.


%\section{Conclusion}
%
%Real but non-positive meson correlation functions show similar statistical behavior to the real parts of complex baryon correlation functions.
%Circular statistics bounds are observed to restrict unbiased results to small times where $\avg{\cos(\theta_i)} > 1/\sqrt{N}$ and lead to the existence of a noise region where results are biased by finite sample size effects at sufficiently large times.
%The magnitude of all meson correlation functions is observed to decay exponentially at large times with a rate equal to $m_\pi$, and the average phase is observed to decay exponentially with a rate equal to $m_\Gamma - m_\pi$.
%Lepage-Savage scaling leading to broad, symmetric correlation function distributions at large times is shown to be generic for non-positive correlation functions.
%The distributions of real but non-positive meson correlation functions are found to be approximately described by complex log-normal distributions at all times once similar modifications to parameter inference are made as those needed to fit baryon correlation functions using only the real part. 
%The pion log-magnitude time evolves consistently with Brownian motion, while other meson correlation functions display signatures of heavy-tailed L\{'e}vy flights similarly to baryons.
%
%Volume dependence of the distributions of meson correlation functions is found to be consistent with the expectations of Hamber \emph{et al}?
%
%Phase reweighting is shown to tame the StN problem in all meson channels, in contrast to improved estimators based on correlation function ratios that are shown to fail when applied to real but non-positive meson correlation functions and when applied to only the real parts of baryon correlation functions.
%In all isovector meson channels, the magnitude has a very slow approach to ground-state saturation at $t\gtrsim 25$ that can be intuitively understood by their connection to pion physics and near-zero modes.
%Conversely, phase-reweighted effective mass extrapolations show ground-state saturation at much smaller values of $\Delta t \gtrsim 6-10$.
%This allows ground-state energies to be extracted reliably in all isovector meson channels, including those where ground-state saturation is not achieved before $t=\Delta t$ standard effective mass results enter their noise regions and become unreliable.
%Results of phase-reweighted effective mass extrapolations on three different physical volumes suggest that the $\rho^+$ and $a_0$ are bound states with these values of the quark masses with $m_{a_0} < m_\rho$ and are consistent with bound states in all isovector meson channels.
%These predictions employing phase reweighting and simple $\overline{q}\Gamma q$ interpolating operators for the meson spectrum at $m_\pi \sim 450$ MeV await verification by more sophisticated variational methods.







