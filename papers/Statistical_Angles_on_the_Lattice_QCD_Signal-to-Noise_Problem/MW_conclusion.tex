\chapter{Conclusion and Outlook}\label{chap:conc}

LQCD is emerging as a tool for precise calculations of hadrons and nuclei
that make no uncontrolled assumptions besides the validity of the Standard Model.
Recent calculations of mesons and the nucleon have been performed with physical quark masses,
and in the multi-baryon sector LQCD calculations of electromagnetic and weak fusion reactions have been performed.
In the near future LQCD calculations will be used to predict the structure of nuclei in terms of quarks and gluons at the EIC,
predict electroweak reaction rates relevant for stellar fusion and neutrino-nucleus scattering,
and reliably connect experimental searches for fundamental symmetry violation such as neutrinoless double-beta decay to theoretical bounds on beyond the Standard Model physics.
Matching LQCD calculation to EFTs of nucleons will allow three-body forces and other poorly-known aspects of the nuclear force to be constrained from first principles theory, supporting many-body methods used to study the structure and reactions of larger nuclei.
LQCD calculations can in principle also study large nuclei and the equation of state of dense matter
relevant for understanding the structure of matter in the interior of neutron stars
and the gravitational waves emitted from their collisions.
Practical application of LQCD to systems of large baryon number has been impeded by the sign and StN problems.
This thesis has presented statistical observations of baryon correlation functions relevant for understanding the baryon StN problem as a sign problem afflicting generic complex correlation functions, and has also presented new statistical analysis techniques where
StN degradation is independent of source-sink separation time $t$
and appears instead in a tunable control parameter $\Delta t$.

Chapter~\ref{chap:statistics} presents evidence that nucleon correlation functions are statistically described by an approximately decorrelated product of a log-normal magnitude and wrapped normal phase factor.
The nucleon correlation function magnitude is found to have no StN problem and has the large-time scaling 
$\langle |C(t)| \rangle  \sim e^{-\frac{3}{2}m_\pi t}$. 
The nucleon log-magnitude, $R(t)$, is approximately described by a normal distribution with linearly increasing mean and almost constant variance. 
The complex phase, which gives the direct importance sampling of $C(t)$ a sign problem,  has the large-time scaling of approximately 
$\langle e^{i\theta(t)} \rangle \sim e^{-(M_N - \frac{3}{2}m_\pi)t}$.
This shows that the StN problem arising from solving the sign problem associated with the phase by reweighting is the Parisi-Lepage StN problem.


Building on the observation that $\Delta \theta_i(t,\Delta t)$ has constant width at large times, 
a new estimator for baryon effective masses is studied 
that relies on statistical sampling of correlation function ratios and therefore phase differences.
This estimator has a StN ratio that is constant in $t$, the source-sink separation time, and the StN problem instead leads to an exponentially degrading StN ratio in $\Delta t$, the difference between the numerator and denominator sink times.
  The independence of $t$ and $\Delta t$ in this estimator allows similarly precise results to be extracted from all sufficiently large $t$ rather than from a window of intermediate $t$, as with traditional estimators.
  The new estimator effectively includes $\Delta t$ timesteps of time evolution following $t-\Delta t$ timesteps of dynamical source improvement and it includes a systematic uncertainty that must be eliminated by extrapolating to the limit $ \Delta t \rightarrow t \rightarrow \infty$.
  The systematic uncertainty of the new estimator is expected to decrease as $e^{-\delta E \Delta t}$ for large $\Delta t$, where $\delta E$ is the energy gap between the ground state and the first excited state with appropriate quantum numbers and appreciable overlap with the effective source at $t - \Delta t$.
  Statistical uncertainties increase with increasing $\Delta t$ as $\sim e^{2(M_N - \frac{3}{2}m_\pi)\Delta t}$.
  For $\Delta t \gtrsim \frac{\ln(N)}{2(M_N - \frac{3}{2}m_\pi)}$ additional systematic uncertainties associated with finite-sample-size effects in statistical inference of circular random variables leads to unreliable results in the same way that $t \gtrsim \frac{\ln (N)}{2(M_N - \frac{3}{2}m_\pi)}$ leads to unreliable results in the noise region of standard estimators.


Chapter~\ref{chap:PR} introduces phase reweighting, a refined method of constructing estimators based on sampling phase differences
whose bias is guaranteed to vanish in a well-defined limit.
Reweighting each correlation in a statistical ensemble with a phase factor from the same correlation function at an earlier time
can be intuitively thought of as reducing the number of steps in the random walk of the phase to a fixed interval,
and provides phase-reweighted correlation functions with constant StN ratios in time separation between the source and sink.
Systematic uncertainties in phase reweighting can be understood as arising from excited state contamination, and in particular excitations of the vacuum arising from the boundary introduced by phase reweighting.
Phase reweighting is demonstrated to give accurate results for the $\rho^+$, nucleon, and $\Xi^-\Xi^-$ systems, and in particular the bias in the $\Xi^-\Xi^-$ binding energy is much smaller than the single-hadron bias and consistent with zero within the uncertainties of the calculation.

Chapter~\ref{chap:mesons} applies the statistical tools and phase-reweighting techniques of the previous chapters to isovector meson correlation functions.
These provide an interesting test case because they are real but non-positive.
The moments of generic non-positive correlation functions are proven to have similar scaling properties to the Lepage-Savage scaling of real parts of baryon correlation functions.
The sign of meson correlation functions acts as a discrete circular random variable,
and the bound $\avg{\cos\theta_i^\Gamma} > 1/\sqrt{N}$ for unbiased parameter inference of circular random variables is found to apply to real but non-positive meson correlation functions.
Correlation-function-ratio estimators are found to fail for isovector meson correlation functions in the same way that they fail when applied to only the real parts of baryon correlation functions.
Phase reweighting is further motivated by its comparative success, and is shown to tame the StN problem
facing real but non-positive correlation functions analogously to complex correlation functions.
This extends the scope of possible applications of phase reweighting to many real but non-positive correlation functions with sign problems appearing in a broad array of quantum Monte Carlo calculations in particle, nuclear, and condensed matter physics.

LQCD calculations of multi-baryon systems without a golden window stand to benefit from phase reweighting.
This thesis has discussed re-analysis of existing correlation functions obtained from Monte Carlo ensembles 
optimized for small- and intermediate-time analysis.
The relative precision of phase reweighting compared to the precision of standard techniques is expected to increase with the size of the time direction.
Especially since plateaus in phase-reweighted effective masses are visible only at large $t$ or sometimes not visible at any $t$ in the calculations at hand,
new Monte Carlo ensembles optimized for large-time analysis
will be necessary for phase-reweighted results
to achieve significant gains in precision compared to standard techniques.
This productions and other explorations of complex correlation function statistics and future applications of phase reweighting are underway.
