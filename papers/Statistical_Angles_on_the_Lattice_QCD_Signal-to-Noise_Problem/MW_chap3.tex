\chapter{Phase Reweighting}\label{chap:PR}
 
%%%%%%%%%%%%

%%%%%%%%%%%%%%%%%%%%%%%%%%%%%%%%%%%%%%%%%%%%%%%

The signal-to-noise problem leads to exponentially degrading precision in LQCD calculations of multi-baryon systems
that is even more severe than in the single-baryon calculations described above~\cite{Beane:2010em}.
Many of the interesting statistical features~\cite{Parisi:1983ae,Hamber:1983vu,Lepage:1989hd,Beane:2009kya,Beane:2009gs,Beane:2010em,Endres:2011jm,Endres:2011er,Endres:2011mm,Lee:2011sm,DeGrand:2012ik,Grabowska:2012ik,Nicholson:2012xt,Beane:2014oea} of single-nucleon correlator functions are shared by multi-nucleon correlation functions.
In particular the logarithms of generic LQCD correlation functions we have studied exhibit characteristics of L{\'e}vy Flights associated with 
heavy-tailed Stable Distributions.
Generic multi-baryon correlation functions appear to be well-described by an uncorrelated product of a log-normally distributed magnitude and a wrapped normally distributed phase factor.
The average magnitude of a nuclear correlation function is proportional to $\sim e^{-B \frac{3}{2}m_\pi}$, while the average phase factor is proportional to $\sim e^{-B(m_N-3 m_\pi /2)}$.
By the same logic of Sec.~\ref{sec:decomposition}, the StN problem in multi-baryon correlation functions can be identified as arising from reweighting a sign problem.
The baryon number sign problem is spacetime extensive in every form it arises, see Sec.~\ref{sec:corr}, and so can be mitigated by restricting the time interval, $\Delta t$, over which the system contains non-zero particle number prior to measurement.
This restriction, used to construct the improved estimator of Sec.~\ref{sec:estimator}, neglects correlations across distances larger than $\Delta t$
and creates a bias in ground-state energies that decreases exponentially with increasing $\Delta t$.

This chapter introduces a refined estimator for the analysis of LQCD correlation functions that
exploits these ideas and
permits the extraction of 
ground-state energies from the noise region.
Through phase reweighting, this  estimator provides an exponential improvement in the StN ratio, but it also introduces a bias that must be 
systematically removed through extrapolation.  
This technique is similar to that used in Green's Function Monte Carlo (GFMC) methods  applied to  
nuclear many-body systems where the phase 
of the wavefunction is held fixed until the system is close to its ground state, at which 
point the phase is released for final evolution~\cite{Zhang:1995zz,Zhang:1996us,Wiringa:2000gb,Carlson:2014vla}. 
Similar techniques are also used in Lattice Effective Field Theory (LEFT) calculations in which a Wigner-symmetric Hamiltonian, 
emerging from the large-N$_c$ limit of QCD~\cite{Kaplan:1995yg}, 
is used for initial time evolution before asymmetric perturbations  are added that introduce a sign problem~\cite{Lahde:2015ona}.
Phase reweighting shares physical similarities, and possibly formal connections, 
to the approximate factorization of domain-decomposed quark propagators recently suggested and explored by 
C$\grave{e}$, Giusti and Schaefer~\cite{Ce:2016idq,Ce:2016ajy,Ce:2016qto}.


%
\begin{figure}[!t]
	\includegraphics[width=0.6 \columnwidth]{PhaseRW_corr.png}
        \centering
		\caption{
	\label{fig:PRWcorrs} 
  The $\rho^+$-meson phase-reweighted correlation function $G^\theta_\rho(t,\Delta t)$ is a product
  of quark propagators forming $C_i^\rho(t)$, shown as solid lines,
  and a phase factor $e^{-i\theta_i^\rho(t-\Delta t)}$, shown as dashed propagator lines with reversed quark-charge arrows.
        %A diagrammatic representation of the $\rho^+$-meson phase-reweighted correlation function $G^\theta_\rho(t,\Delta t)$.
        %Solid lines indicate  quark propagators forming $C_i(t)$.
        %Dashed lines indicate that only the phase of the correlation function is included, and quark-charge arrows are reversed for 
        %the phase-reweighing factor
        %$e^{-i\theta_i(t-\Delta t)}$.
        Gluon lines indicate that phase reweighting introduces correlations
        associated with excitations produced at $t-\Delta t$
        and lead to bias when $\Delta t \neq t$.
        For momentum-projected correlation functions, 
        excitations involving correlated interactions between 
        $C_i^\rho(t)$ and $e^{-i\theta_i^\rho(t-\Delta t)}$ are suppressed by the spatial volume.
        $G^\theta_\rho(t,\Delta t)$ effectively includes a non-local source
        whose magnitude is dynamically refined for $t - \Delta t$ steps
        while the phase is held fixed (shaded region)
        before the full system is evolved for the last $\Delta t$ steps of propagation.
	}		
\end{figure}
%


%LQCD calculations involve ensembles of a large number, $N$, of correlation functions $C_i(t)$, each calculated from a 
%source on a particular gauge field configuration.
%Expectation values $G(t)=\langle C_i(t) \rangle$ can be computed from sample averages $G(t) = {1\over N}\sum C_i(t)$ 
%across field configurations importance sampled from the QCD vacuum probability distribution.
%The ground-state energy of correlation functions can be accurately determined
%from the late-time behavior of $G(t)$, but for generic correlation functions the StN problem
%restricts the extraction of precise ground-state energy measurements to early and intermediate times.

The  phase-reweighted correlation function is defined by
%
\begin{eqnarray}
G^{\theta}(t,\Delta t) & = & 
\langle e^{-i \theta_i (t-\Delta t)} \ C_i(t)  \rangle
\  \ ,
\label{eq:PRWdef}
\end{eqnarray}
%
where $\theta_i (t-\Delta t) = \text{arg}[ C_i(t-\Delta t) ]$.
Phase reweighting resembles limiting the approximate L{\'e}vy Flight of the correlation function phase 
to $\Delta t$ steps at large times, suggesting that $G^\theta(t,\Delta t)$ has a StN ratio that decreases exponentially with 
$\Delta t$ but is constant in $t$.
In the limit that $\Delta t \rightarrow t$, the reweighting factor approaches unity and $G^{\theta}(t,t) = G(t)$.
The exact correspondence $G^\theta(t,t)=G(t)$ 
gives phase reweighting
an advantage over our previously suggested estimator~\cite{Wagman:2016bam}  involving 
multiplication by $C_i^{-1}(t-\Delta t)$ rather than $e^{-i\theta_i(t-\Delta t)}$.
Phase reweighting also leads to more precise ground-state energy extractions than estimators involving reweighting with $C_i^{-1}(t-\Delta t)$;
multiplication by the heavy-tailed variable $|C_i(t-\Delta t)^{-1}|$ leads to increased variance.



Dynamical correlations between $C_i(t)$ and $e^{-i\theta_i(t-\Delta t)}$
lead to differences in  ground-state energies extracted from $G^\theta(t,\Delta t)$ and $G(t)$ for $t\neq \Delta t$. 
Locality suggests that these correlations should decrease exponentially with increasing $\Delta t$
at a rate controlled by the longest correlation length in the theory.
At asymptotically large $\Delta t$,
one-pion-exchange correlations are expected to provide the largest contributions to the bias.
These contributions will be
suppressed by factors involving the spatial volume in products of a momentum-projected correlation function with a momentum-projected phase factor.
Excitations involving the $\sigma$ meson, correlated two-pion exchange, and other light
excitations that do not change the quantum numbers of the system
are not volume-suppressed and
may dominate at small $\Delta t$.
Near-threshold bound states may have complicated small $\Delta t$ bias that is sensitive to the size of the spatial volume.



%
\begin{figure}[!t]
  \centering
	\includegraphics[width=0.7 \columnwidth]{rho_EM_comp_final2.png}
		\caption{
	\label{fig:emps} 
  The upper panel shows the $\rho^+$ effective mass from the LQCD ensemble of Ref.~\cite{Orginos:2015aya}.
	The lower panel shows $M_\rho^\theta(t,\Delta t)$ with a range of fixed $\Delta t$'s.
        Temporal structure at larger times arises from proximity to the midpoint of the lattice at $t=48$.
        The highlighted interval $t=28\rightarrow 43$ is used for correlated $\chi^2$ minimization fits of $M_\rho^\theta$.
	Masses and times are given in lattice units.
	}		
\end{figure}
%

The construction of $G^\theta$ is generic for any correlation function,
and is schematically depicted for the $\rho^+$ meson in Fig.~\ref{fig:PRWcorrs}.
In the plateau region of the  $\rho^+$ correlation function, the average of the  magnitude is approximately proportional to $e^{-M_\pi t}$, 
while the average of the phase factor\footnote{
The phases of isovector meson correlation functions are restricted to be discrete values $\theta_\rho = 0,\ \pi$ 
when interpolating operators in a Cartesian spin basis are used.  In forthcoming work, we demonstrate that circular statistics 
applies to real but non-positive isovector meson correlation functions.
} is approximately proportional to  $e^{-(M_\rho-M_\pi) t}$.
$G^\theta(t,\Delta t)$ is a product of these~two averages plus corrections arising from correlations between 
$C_i(t)$ and $e^{-i\theta_i(t-\Delta t)}$, and so at large $t$ and $\Delta t$ it is expected to have the form 
%
\begin{eqnarray}
G^{\theta}(t,\Delta t) & \sim &
e^{-M_\pi (t-\Delta t) } e^{-M_\rho \Delta t} \left(\alpha+\beta e^{-\delta M_{\rho} \Delta t } +  ... \right),
\;\;\;\;
\label{eq:PRWtdep}
\end{eqnarray}
%
where $M_\rho +\delta M_{\rho}$ is the energy of the lowest-lying excited state of the $\rho^+$
leading to appreciable correlations between $C_i(t)$ and $e^{-i\theta_i(t-\Delta t)}$,
and $\alpha$ and $\beta$ are overlap factors that cannot be determined with general arguments but can be calculated with LQCD.
The ellipses denote further-suppressed contributions from higher-lying states.
A phase-reweighted effective mass can be defined as
$M^\theta=\log\left( G^{\theta}(t,\Delta t) /G^{\theta}(t+1,\Delta t+1) \right)$, which reduces to the standard effective mass definition when $\Delta t\rightarrow t$.
For the $\rho^+$ meson, the form of the correlation function given in Eq.~(\ref{eq:PRWtdep}) leads to
%
\begin{eqnarray}
M_\rho^\theta(t, \Delta t) & = & M_\rho \ +\  c\ \delta M_{\rho} e^{-\delta M_{\rho} \Delta t}\ +\ ...
\ \ ,
\label{eq:PRWEMP}
\end{eqnarray}
%
at large $t$,
where $c=\beta/\alpha$ and the ellipses denote higher order contributions which are exponentially suppressed with
$\Delta t$ and standard excited state contributions that are exponentially suppressed with $t$.



%
\begin{figure}[!t]
  \centering
	\includegraphics[width=0.7 \columnwidth]{rho_3d.png}
		\caption{
	\label{fig:2DEMP} 
        The $\rho^+$ meson phase-reweighted effective mass for all $\Delta t \leq t$. 
        The standard effective mass in the upper panel of Fig.~\ref{fig:emps} corresponds to
        $M^\theta_\rho(t,t)$, a projection along the line $t = \Delta t$ indicated.
        The bottom panel of Fig.~\ref{fig:emps} 
        shows $M^\theta_\rho(t,\Delta t)$ on lines of constant $\Delta t$ parallel to the $t$ axis indicated. 
	}		
\end{figure}
%

LQCD calculations of $M_\rho^\theta$ summarized in Figs.~\ref{fig:emps}-\ref{fig:Rhoextrap}
permit precise numerical study of small $\Delta t$ bias and $\Delta t \rightarrow t$ extrapolation.
These calculations employ $N\sim 130,000$ correlation functions
previously computed by the NPLQCD collaboration~ 
from smeared sources and point sinks on an 
ensemble of 2889 isotropic-clover gauge-field configurations 
at a pion mass of $M_\pi \sim 450~{\rm MeV}$
generated jointly by the College of William and Mary/JLab lattice group and by the NPLQCD collaboration, see Ref.~\cite{Orginos:2015aya} for further details.
The spacetime extent of the lattices is $48^3\times 96$ at a lattice spacing of $a\sim 0.117(1)~{\rm fm}$.
For all of the correlation functions examined in this work,  
momentum projected blocks are derived from quark propagators originating from smeared sources localized about a site in the lattice volume, 
as detailed in previous works by the NPLQCD collaboration, e.g. Ref.~\cite{Beane:2006mx,Orginos:2015aya}.
For instance, the blocks associated with the $\rho^+$ meson are
%
\begin{eqnarray}
{\cal B}^{(\rho^+)}_\mu ({\bf p},t; x_0) &  = & 
\sum_{\bf x} e^{i{\bf p}\cdot {\bf x}}\ \overline{S}_d({\bf x},t;x_0)\gamma_\mu S_u({\bf x},t;x_0).\;\;
  \label{eq:blockdef}
\end{eqnarray}
%
Correlations functions are derived by contracting the blocks with local interpolating fields~\cite{Detmold:2012eu},
e.g., 
%
\begin{eqnarray}
C^{(\rho^+;\mu)}({\bf p},t; x_0) & = & 
{\rm Tr}\left[\ 
{\cal B}^{(\rho^+)}_\mu ({\bf p},t; x_0)  \gamma^\mu
\ \right]
 ,
 \label{eq:blockcon}
\end{eqnarray}
%
where the trace is over color and spin. It is the phases of contracted momentum-projected blocks that have been used to form phase-reweighted correlation functions.
Expressions similar to those in eqs.~(\ref{eq:blockdef}) and (\ref{eq:blockcon}) are used for the nucleon and two-nucleon 
systems~\cite{Beane:2006mx,Orginos:2015aya}.


%
\begin{figure}[!t]
  \centering
	\includegraphics[width=0.7 \columnwidth]{rho_moneyplot.png}
		\caption{
	\label{fig:Rhoextrap} 
        The $\rho^+$ mass extracted from large-time phase-reweighted correlation functions.
        %The blue points correspond to averages of $M_\rho^\theta$ over $t=30\rightarrow 40$ as a function of $\Delta t$.
	The light-brown shaded region corresponds to the $68\%$ confidence region associated with 	
        three-parameter (constant plus exponential)  fits to Eq.~\eqref{eq:PRWEMP}.
        The dashed lines show the extrapolated $M_\rho^\theta$ result
        including statistical and systematic uncertainties
        described in the main text.
        The gray horizontal band corresponds to a determination of the $\rho^+$ mass from the plateau region~\cite{Orginos:2015aya}.
        The purple line corresponds to the $\pi\pi$ non-interacting $p$-wave energy.	}		
\end{figure}
%

At large $t$ and small $\Delta t$, bias in $M_\rho^\theta$ is consistent with Eq.~\ref{eq:PRWEMP}.
At intermediate $\Delta t$, $M_\rho^\theta$ approaches a value consistent  with the $\pi\pi$ non-interacting $p$-wave energy $\sqrt{(2M_\pi)^2 + (2\pi/L)^2}$. 
At large $\Delta t$, $M_\rho^\theta$ approaches a lower-energy plateau consistent with the $\rho^+$ mass extracted from a $t=\Delta t$ plateau $t = 18 \rightarrow 28$.
The suppression of $\rho^+$ bound state contributions  compared to $\pi\pi$ scattering states contributions to $C_i(t)e^{-i\theta_i(t-\Delta t)}$ is found to be less severe in smaller volumes.
The energy gap between the bound and scattering states also increases in smaller volumes.
In accord with these arguments, the non-monotonic $\Delta t$ behavior visible in Fig.~\ref{fig:Rhoextrap} is not seen with $V=32^3$ or $V=24^3$.
$M_\theta^\rho$ is consistent with the $\rho^+$ mass determined in Ref.~\cite{Orginos:2015aya} for $\Delta t \gtrsim 5$ in these smaller volumes.
Variational methods employing phase reweighted correlation functions
with multiple interpolating operators
may be required 
to reliably distinguish closely spaced energy levels with large spatial volumes.


%
\begin{figure}[!t]
  \centering
	\includegraphics[width=0.7 \columnwidth]{nucleon_moneyplot.png}
		\caption{
	\label{fig:Nextrap} 
	The large-time nucleon phase-reweighted effective mass with statistical and systematic extrapolation errors shown
        with light-brown bands and dashed lines as in Fig.~\ref{fig:Rhoextrap}.
        The gray horizontal band corresponds to golden window result of Ref~\cite{Orginos:2015aya} obtained with four times higher statistics.
	}		
\end{figure}
%


The nucleon mass does not appear to have complications from low-lying excited states
and the large time phase-reweighted nucleon effective mass
derived from $\sim 100,000$ sources with $V=32^3$~\cite{Orginos:2015aya}
approaches its intermediate time plateau value at large $\Delta t$.
Small $\Delta t$ bias is well-described with a constant plus exponential form, and
the nucleon excited state gap can be 
extracted across a range of fitting regions as $\delta M_N = 786(44)(25)$~MeV, 
where the first uncertainty is statistical from a correlated $\chi^2$-minimization fit of $M_N^\theta(t,\Delta t)$ 
to Eq.~\eqref{eq:PRWEMP}
with $\Delta t = 2\rightarrow 10$ and $t = 30\rightarrow 40$  and the second uncertainty is a systematic  
determined from the variation in central value when the fitting region is changed to be $\Delta t = 1\rightarrow 10$ or $\Delta t = 3\rightarrow 10$. 
This result is consistent with a naive extrapolation $M_\sigma \sim 830$ MeV of the $\sigma$-meson mass 
determined at $M_\pi \sim 391$~MeV
~\cite{Briceno:2016mjc}. 
Results for strange-baryon excited-state masses from phase-reweighted effective mass extrapolations  are also 
consistent with the $\sigma$-meson mass in one- and two-baryon systems,
for instance $\delta M_{\Xi} = 822(44)(71)$~MeV and $\delta M_{\Xi\Xi(\si)} = 908(265)(82)$~MeV.






The $\Xi^-\Xi^-(\si)$ has slower StN degradation than a two-nucleon system
and is considered here for a first investigation of phase-reweighted baryon-baryon binding energies.
The $\Xi^-\Xi^-(\si)$ binding energy was
determined 
by
the NPLQCD collaboration
to be
$B_{\Xi\Xi (\si)}= 15.4(1.0)(1.4)~{\rm MeV}$
for the gauge field configurations considered here
using the correlation function production and sink-tuning~\cite{Beane:2009kya,Beane:2009gs,Beane:2010em} described for the deuteron and di-neutron in Ref.~\cite{Orginos:2015aya}.\footnote{$B_{\Xi\Xi(\si)} = -M_{\Xi\Xi(\si)}+2M_\Xi$
approaches the $\Xi\Xi(\si)$ binding energy in the infinite volume limit.
In finite volume $B_{\Xi\Xi(\si)}$ differs from the infinite-volume binding energy
by corrections that are exponentially suppressed by the binding momentum.
}
Results for $\Xi^-\Xi^-(\si)$ using the $\sim 100,000$ correlation function ensemble described above 
for constant 
fits to the phase reweighted binding energy with $t = 28\rightarrow 43$, $\Delta t = 1,2,3\rightarrow 6$ 
give 
$B_{\Xi\Xi(\si)}=15.8(3.5)(2.6)$ MeV.
Consistency between golden window results and phase-reweighted results with large $t$ and all $\Delta t \gtrsim 1$
suggests
a high degree of cancellation at all $\Delta t$ between excited state effects in 
one- and two-baryon phase reweighted effective masses.
$B_{\Xi\Xi(\si)}(t, \Delta t = 0) $, which only involves correlation function magnitudes,
plateaus to $7.1(0.6)(0.8)\text{ MeV}$.
Phase effects modify this magnitude result
by an amount on the order of nuclear energy scales rather than hadronic mass scales,
providing encouraging evidence that extrapolations involving modest $\Delta t$ can accurately determine nuclear binding energies in the noise region.
The precision of phase-reweighted results scales with the number of points in the noise region,
and could be increased on lattices of longer temporal extent then those used in this work ($\sim 11.2 {\rm fm}$).



%
\begin{figure}[!t]
  \centering
	\includegraphics[width=0.65 \columnwidth]{xixi_moneyplot.png}
		\caption{
	\label{fig:XiXiBindextrap} 
	The $\Xi^-\Xi^- (\si)$ phase-reweighted binding energy with statistical and systematic extrapolation errors shown
      with light-brown bands and dashed lines as in Fig.~\ref{fig:Rhoextrap}.
        The gray horizontal band corresponds to the golden window result of Ref.~\cite{Orginos:2015aya}, obtained with four times higher statistics.
	}		
\end{figure}
%

Phase reweighting 
allows 
energy levels
to be extracted
from LQCD correlation functions
at times larger than the
golden window
accessible to standard techniques involving source and sink optimization~\cite{Beane:2009kya,Beane:2009gs,Beane:2010em,Detmold:2014rfa,Detmold:2014hla}.
It is expected that these methods
will permit the extraction of ground-state energies in systems
without a golden window.
The phase-reweighting method
is equivalent to a dynamical source improvement
in which the phase is held fixed while the magnitude of the hadronic correlation function
is evolved into its ground state,
and then the phase is released to provide a source for subsequent time slices.
The bias introduced by phase reweighting can be removed by extrapolation
but suffers from a StN problem
that can be viewed as arising from evolution of the dynamically improved source.
Generalizations of the phase-reweighting methods presented here
may allow for reaction rates,
operator matrix elements,
and other observables to be extracted from phase-reweighted correlation functions.
Further study is planned of the $\Delta t \rightarrow t$ extrapolation 
and applications of phase reweighting to hadronic and nuclear systems.
Mesonic systems in particular are discussed in the next chapter.



\begin{table}[!ht]
\begin{center}
\begin{minipage}[!ht]{16.5 cm}
\end{minipage}
\setlength{\tabcolsep}{1em}
\resizebox{.7\linewidth}{!}{%
\def\arraystretch{.7}%
\begin{tabular}{|c| l | l | l |}
\hline
$\Delta t$    & \qquad $M^\theta_{\rho^+}$  & \qquad $M_N^\theta$          & \qquad $B_{\Xi\Xi(\si)}$     \\
\hline
\hline
$1$ & \quad 0.40872(21) &\quad 0.61209(50)&\quad  -0.0081(15)           \\
$2$ & \quad 0.47392(30) &\quad 0.65278(66)&\quad  -0.0096(24)           \\
$3$ & \quad 0.50841(40) &\quad 0.67861(88)&\quad   -0.0083(36)          \\
$4$ & \quad 0.52722(52) &\quad 0.6951(12)&\quad    -0.0089(62)         \\
$5$ & \quad 0.53774(67) &\quad 0.7057(16)&\quad   -0.003(11)          \\
$6$ & \quad 0.54284(84) &\quad 0.7135(22)&\quad    0.003(16)          \\
$7$ & \quad 0.5446(11) &\quad 0.7193(30)&\qquad\qquad       -      \\
$8$ & \quad 0.5449(15) &\quad 0.7225(41)&\qquad\qquad       -      \\
$9$ & \quad 0.5446(19) &\quad 0.7235(56)&\qquad\qquad       -      \\
$10$ & \quad 0.5439(23) &\quad 0.7259(76)&\qquad\qquad       -      \\
$11$ & \quad 0.5421(30) &\quad 0.723(10) & \qquad \qquad -      \\
$12$ & \quad 0.5395(37) &\quad 0.725(14) &  \qquad \qquad -      \\
$13$ & \quad 0.5368(47) &\qquad \qquad -  &  \qquad \qquad -      \\
$14$ & \quad 0.5359(58) &\qquad \qquad -  &  \qquad \qquad -      \\
$15$ & \quad 0.5321(71) &\qquad \qquad -  &  \qquad \qquad -      \\
$16$ & \quad 0.5271(83) &\qquad \qquad -  &  \qquad \qquad -      \\
$17$ & \quad 0.5215(95) &\qquad \qquad -  &  \qquad \qquad -      \\
$18$ & \quad 0.519(11) &\qquad \qquad -  &  \qquad \qquad -      \\
$19$ & \quad 0.518(12) &\qquad \qquad -  &  \qquad \qquad -      \\
$20$ & \quad 0.517(12) &\qquad \qquad -  &  \qquad \qquad -      \\
$21$ & \quad 0.516(12) &\qquad \qquad -  &  \qquad \qquad -      \\
$22$ & \quad 0.515(12) &\qquad \qquad -  &  \qquad \qquad -      \\
$23$ & \quad 0.510(12) &\qquad \qquad -  &  \qquad \qquad -      \\
$24$ & \quad 0.512(13) &\qquad \qquad -  &  \qquad \qquad -      \\
$25$ & \quad 0.513(13) &\qquad \qquad -  &  \qquad \qquad -      \\
\hline
PR Ground  & 0.5222(60)(27)   &  0.7220(33)(11) &    -0.0096(22)(11)         \\
PR Excited  & 0.5508(11)(7)   &  \qquad \qquad - &    \qquad \qquad -         \\
\hline 
GW Ground  &  0.5248(14)(15) & 0.72551(35)(26)&    -0.00909(59)(83)         \\
GW $\pi\pi$  &  0.547997(78)(14) & \qquad \qquad - &    \qquad \qquad -         \\
\hline 
\end{tabular}
}
\begin{minipage}[t]{16.5 cm}
\vskip 0.0cm
\tiny
\noindent
  \caption{
Phase-reweighted (PR) effective masses of the $\rho^+$, nucleon and the effective 
energy difference between $\Xi\Xi(\si)$ and two $\Xi$'s
derived from  eq.~(\ref{eq:PRWdef}). 
The extrapolated PR ground values are taken from three-parameter constant plus exponential correlated $\chi^2$-minimization fits for
$M_N^\theta$ and one-parameter constant fits for $B_{\Xi\Xi(\si)}^\theta$
with statistical uncertainties for fits starting at $\Delta t = 2$ and systematic uncertainties
defined from variation of the $\Delta t$ fitting window as described in the main text.
PR data is taken from $t = 28 \rightarrow 43$ for the $\rho^+$ and  $\Xi\Xi(\si)$
and $t = 31 \rightarrow 40$ for the nucleon.
For the $\rho^+$, the region $\Delta t = 2 \rightarrow 10$ is used to constrain the first scattering state for the PR excited state result, 
while the region $\Delta t = 16\rightarrow 25$ is used to constrain the ground state.
Golden window (GW) ground refers to the ground-state energy determinations using the short and intermediate
time plateau regions described in Ref.~\cite{Orginos:2015aya}.
GW $\pi\pi$ refers to the non-interacting $p$-wave energy shift $\sqrt{(2M_\pi)^2 + (2\pi/L)^2}$ using $M_\pi$ and $L$ for the $48^3$ ensemble described in the main text.
}
\label{tab:masses}
\end{minipage}
\end{center}
\end{table}     




