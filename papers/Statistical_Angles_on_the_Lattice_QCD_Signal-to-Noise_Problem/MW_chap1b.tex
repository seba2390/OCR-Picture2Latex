\section{Lattice Field Theory}\label{sec:lattice}


In continuous spacetime, any field can occupy an infinite number of momentum states. 
For free fields, each of these momentum modes acts as a harmonic oscillator and contributes to the vacuum energy. 
When quantum fluctuations of this infinite number of modes are considered in perturbative expansions about the free field vacuum,
divergences arise in quantities such as the vacuum energy density.
There are many regularization schemes that can be used to control these divergences and verify that perturbative relationships between physical observables in QFT are finite.
Renormalization group arguments and universality suggest that different regularizations of the same QFT approach the same continuum limit when the regularization scale cutting off high-momentum, ultraviolet (UV) fluctuations is is taken to infinity.

To describe the confinement of quarks into hadrons with QCD, non-perturbative quantum effects beside fluctuations about the free field vacuum must be computed. 
To include non-perturbative quantum fluctuations in path integrals, the measure $\mathcal{D}U\mathcal{D}\bar{q}\mathcal{D}q$ schematically presented in Eq.~\eqref{eq:pathintegraldef} must be concretely defined. 
Lattice regularization is the only known non-perturbative regulator for UV divergences in QCD path integrals.\footnote{Construction of a lattice regularization of chiral gauge theories necessary for a non-perturbative regulator for electroweak interactions is a longstanding challenge that has seen exciting recent development~\cite{Grabowska:2015qpk,Grabowska:2016bis}.}
In lattice field theory, continuous spacetime is replaced by a discrete lattice of points.\footnote{Textbook introductions to lattice field theory can be found in Refs.\cite{creutz:1983quarks,Gupta:1997nd,smit:2002introduction,montvay:1997quantum}. The scalar field theory construction briefly sketched in this section most closely follows the presentation of Montvay and M{\"u}nster~\cite{montvay:1997quantum}.} 
Each spacetime point is associated with a quantum mechanical Hilbert space.
For a theory involving only a single complex scalar field $\varphi(x)$, this Hilbert space can be defined in the coordinate basis where field operators $\hat{\varphi}(x)$ act by
\begin{equation}
  \begin{split}
    \hat{\varphi}(x)\ket{\varphi(x)} = \ket{\varphi(x)}\varphi(x),
  \end{split}\label{eq:scalarstatedef}
\end{equation}
and states are normalized as
\begin{equation}
  \begin{split}
    \braket{\varphi(x)}{\varphi(x^\prime)} = \delta(\text{Re}\varphi(x) - \text{Re}\varphi(x^\prime))\delta(\text{Im}\varphi(x) - \text{Im}\varphi(x^\prime)).
  \end{split}\label{eq:scalarnormdef}
\end{equation}
The overall state of the system at time $t$ is specified the full Hilbert space, a tensor product of the Hilbert space at each spatial point with states 
\begin{equation}
  \begin{split}
    \ket{\varphi_t} &= \prod_\v{x}\ket{\varphi(\v{x},t)}. 
  \end{split}\label{eq:scalarproduct}
\end{equation}
The dynamics of the system are fully specified by the form of the Hamiltonian $H_\varphi$, a Hermitian operator defined on the full Hilbert space.
The path integral formulation of the theory is instead specified by an action $S_\varphi$ and a path integral measure
\begin{equation}
  \begin{split}
    \mathcal{D}\varphi^\dagger\mathcal{D}\varphi = \prod_x d\text{Re}\varphi(x)d\text{Im}\varphi(x).
  \end{split}\label{eq:scalarmeasuredef}
\end{equation}
Hilbert space matrix elements are related to path integrals by
\begin{equation}
  \begin{split}
    \mbraket{\varphi_t}{e^{-i\hat{H}_\varphi t}}{\varphi_0} = \int_{\varphi_0}^{\varphi_t} \mathcal{D}\varphi^\dagger \mathcal{D}\varphi\;e^{iS_\varphi}.
  \end{split}\label{eq:scalarpathintegraldef}
\end{equation}
To make the path integral finite, the lattice field theory can be defined on a spacetime volume of spatial volume $L^3$ and time extent $\beta$.
Eq.~\eqref{eq:scalarpathintegraldef} then requires a large but finite number $2L^3\beta$ of integrals over real variables. 

Stochastic integration methods broadly called Monte Carlo techniques are often useful for evaluating high-dimensional integrals.
These techniques rely on sampling field configurations rather than enumerating and evaluating the contributions of all field configurations.
The error of estimating a well-behaved integral from a random sample of $N$ field configurations with Monte Carlo techniques scales as $1/\sqrt{N}$ independently of the dimension of the integrals.
Minkowksi space path integrals are not well-behaved, and the oscillatory integrand $e^{iS_\varphi}$ leads to a sign problem obstructing Monte Carlo evaluation of real-time path integrals.
The action is extensive, and $e^{iS_\varphi}$ is $O(1)$ for generic points in field configuration space.
UV fluctuations that do not contribute significantly to the average path integral make contributions of the same magnitude as semi-classical field configurations expected to dominate the path integral.
In a finite statistical sample, it is difficult to accurately reproduce these cancellations in the average path integral relying on interference between the amplitudes of different field configurations.

Monte Carlo simulations can preference field configurations making dominant contributions to path integrals by using importance sampling,
in which field configurations are sampled from a distribution that weights configurations with larger contributions to the integral with higher probability than configurations that make smaller contributions.
Importance sampling relies on being able to treat the integrand under consideration as a probability distribution.
To compute the average real part, field configurations could be drawn from an importance sampled probability distribution approximately proportional to $|\cos(iS_\varphi)|$ separately in regions of positive and  negative $\cos(iS_\varphi)$.
This will efficiently sample field configurations from the regions giving dominant positive and dominant negative contributions.
However, without knowing the nodal structure of $e^{iS_\varphi}$ or the relative proportion of positive-valued and negative-valued configurations in field space, it is impossible to know the relative degree of cancellation between separately importance sampled positive and negative contributions.
Importance sampling cannot be applied to integrals of the pure phase $e^{iS_\varphi}$.
The sign problem similarly obstructs real-time simulation of hadrons in QCD.

Instead, standard practice is to consider analytically continuing the path integral to imaginary time $t = i x^0$.
This changes Minkowski spacetime to Euclidean spacetime with metric $(++++)$, and correspondingly changes the isometry group of flat spacetime to $SO(4)$.
The latter affects fields with spin non-trivially, but for the case of a scalar field the only charge to the action is the relative sign of the kinetic and potential terms.
For simplicity, we only consider isotropic lattices where the lattice spacing is identical in all space and time directions and unless otherwise specified will work in units where the lattice spacing is set to unity.
There is no unique way to define a lattice derivative, and one simple choice of the free scalar field action is
\begin{equation}
  \begin{split}
  S_\varphi &= \sum_{x} \varphi^\dagger(x) \left( \sum_\mu \partial_\mu^\dagger\partial_\mu + m^2 \right)\varphi(x)\\
  &= \sum_{x;\mu} |\varphi(x+\hat{\mu}) -  \varphi(x)|^2 + \sum_x m^2|\varphi(x)|^2.
  \label{eq:scalareuclideanaction}
\end{split}
\end{equation}
Here and below the same notation is used for the Minkowski and Euclidean action.
The only free parameter in Eq.~\eqref{eq:scalareuclideanaction} is the bare mass in lattice units $m$.
In particular the value of the lattice spacing is not an independent parameter.
Results from a Monte Carlo calculation are dimensionless numbers that can be interpreted as results in lattice units
and cannot be immediately converted to physical units.
Agreement between a prediction of the lattice field theory and an experimental observable, or some other specified constraint, must assumed in order to set the lattice scale in physical units.
Other observables can then be predicted in physical units.
In gauge theories, this plays the role of coupling constant renormalization or equivalently of determining $\Lambda_{QCD}$ (appropriately defined non-perturbatively) in lattice units.

After transforming to Euclidean spacetime, the path integral weight changes from $e^{iS_\varphi}$ to $e^{-S_\varphi}$.
Minima of the Euclidean action are associated with solutions of the classical field equations of motion, and the weight $e^{-S_\varphi}$ exponentially damps UV fluctuations with larger action than classical field configurations of minimum action.
The path integral weight is real and non-negative for a scalar field in Euclidean spacetime, and it can therefore be interpreted as a probability distribution.
In a finite spacetime volume with $L^3 \beta$ lattice points and field boundary conditions specified, configurations can sampled using a Markov process of sequential random updates that is constructed to have a non-zero probability of reaching any configuration from any other configuration with enough random update steps.
The desired scalar field vacuum distribution can be generated from this Markov process using for instance the Metropolis algorithm where updates from field configuration $\varphi_i$ to $\varphi_{i+1}$ that decrease the action are accepted and included in a statistical ensemble, but updates that increase the action are only accepted with a probability equal to $e^{-S_\varphi(\varphi_{i+1})+S_\varphi(\varphi_i)}$.
If periodic boundary conditions are chosen in the time direction (anti-periodic for fermions), then $e^{-S_\varphi}$ is precisely the Boltzmann distribution describing the equilibrium state of the QFT in the language of statistical mechanics.
The partition function of statistical mechanics can be identified with the analytic continuation of the vacuum-to-vacuum amplitude to Euclidean spacetime,
\begin{equation}
  \begin{split}
    Z = \int_{\varphi_\beta=\varphi_0}\mathcal{D}\varphi^\dagger \mathcal{D}\varphi e^{-S_\varphi}.
  \end{split}\label{eq:scalarZdef}
\end{equation}
By analogy with statistical mechanics, the length of the time direction $\beta$ can be identified with the inverse physical temperature of the system.
One can intuitively think that the system is immersed in a non-zero temperature heatbath made up of its periodic images separated by distance $\beta$.

There is no distinction between space and time directions in Euclidean spacetime, and including periodic boundary conditions in spatial directions can be described as coupling the system to a heatbath of its periodic images to form a cubic lattice with the original volume acting as the unit cell.
For a system with a finite correlation length, $m_{\pi}^{-1}$ for a system of isolated hadrons in QCD, effects of interactions with these periodic images will be exponentially small in the ratio of the box size to the correlation length, $e^{-m_\pi L}/L$~\cite{Luscher:1985dn}.
%\footnote{Finite volume effects for $D$ compact dimensions are generally suppressed by $e^{-m_\pi L}/L^{(D-1)/2}$.}
Multi-particle scattering states of strongly interacting hadrons receive additional power law finite volume corrections that encode the probability of particles with finite-range interactions scattering within the finite volume~\cite{Huang:1957im,Luscher:1986pf}.
Infinite volume Euclidean correlation functions describing scattering states are independent of the scattering phase shift away from kinematic thresholds~\cite{Maiani:1990ca}, but
L{\"u}scher showed that power law finite volume corrections in QFT depend on the specific form of the interaction and can be used to extract the phase shifts of hadronic scattering states from finite volume correlation functions~\cite{Luscher:1986pf}.
The original work of L{\"u}scher has been extended to increasingly more complex systems, and the finite volume effects of strong interactions in two, three, and more hadron systems~\cite{Luscher:1990ux,Luscher:1990ck,Rummukainen:1995vs,Lellouch:2000pv,Beane:2003da,He:2005ey,Kim:2005gf,Christ:2005gi,Lage:2009zv,Luu:2011ep,Briceno:2012yi,Briceno:2012rv,Hansen:2012bj,Hansen:2012tf,Gockeler:2012yj,Briceno:2013bda,Briceno:2013lba,Hansen:2014eka,Briceno:2016xwb,Hansen:2017mnd}.
Loosely bound systems, which for these purposes includes light nuclei, also receive additional finite volume corrections that are exponentially small in the binding momentum of the system~\cite{Beane:2003da}.
In the deuteron, a light nucleus with the quantum numbers of a spin-triplet neutron-proton bound state, the binding momentum is much smaller than $m_\pi$ and these near-threshold scattering effects are the much larger than $e^{-m_\pi L}$ effects.

The correspondence Eq.~\eqref{eq:scalarZdef} between statistical mechanics and Euclidean QFT for a free scalar field
can be established through the construction of a transfer matrix, a bounded, self-adjoint operator $\hat{T}_\varphi$ defined on the full Hilbert space by
\begin{equation}
  Z = \tr(\hat{T}_\varphi^{\beta}) = \int \mathcal{D}\varphi^\dagger\mathcal{D}\varphi \mbraket{\varphi_0}{\hat{T}_\varphi}{\varphi_{\beta-1}}\mbraket{\varphi_{\beta-1}}{\cdots\vphantom{\hat{T}_\varphi}}{\varphi_1}\mbraket{\varphi_1}{\hat{T}_\varphi}{\varphi_0},
  \label{eq:scalarZsplit}
\end{equation}
where the second equality follows by inserting complete sets of coordinate basis states with the normalization of Eq.~\eqref{eq:scalarstatedef} at each time-slice.
The path integral representation of the free-field partition function follows if a bounded, self-adjoint operator $\hat{T}_\varphi$ can be defined that has matrix elements
\begin{equation}
  \mbraket{\varphi_{t+1}}{\hat{T}}{\varphi_t} =  e^{-L[\varphi_{t+1},\varphi_t]},
  \label{eq:scalarT}
\end{equation}
where 
\begin{equation}
  L[\varphi_{t+1},\varphi_t] = \sum_\v{x} |\varphi(\v{x},t+1) - \varphi(\v{x})|^2 + \frac{1}{2}\sum_\v{x} V[\varphi_{t+1}] + \frac{1}{2}\sum_\v{x} V[\varphi_t],
  \label{eq:scalarL}
\end{equation}
\begin{equation}
  V[\varphi_t] = \sum_k |\varphi(\v{x} + \uv{k},t) - \varphi(\v{x},t)|^2 + m^2 |\varphi(\v{x},t)|^2.
  \label{eq:scalarV}
\end{equation}
Note that locality of the action plays an essential role in allowing the path integral weight to be decomposed into a product of factors depending on time-slice pairs $\{\varphi_t,\varphi_{t+1}\}$ in Eq.~\eqref{eq:scalarZsplit}.
Textbooks on lattice field theory explicitly demonstrate the construction of the scalar field transfer matrix in terms of field operators and their conjugate momenta~\cite{creutz:1983quarks,smit:2002introduction,montvay:1997quantum}.

If the partition function trace is computed in the eigenbasis of $\hat{T}$, guaranteed to exist by self-adjointness, it can be expressed as
\begin{equation}
  \begin{split}
    Z = \tr(\hat{T}^\beta) = \sum_n e^{-E_n \beta}\sim e^{-E_0 \beta},
  \end{split}\label{Zspectral}
\end{equation}
where $e^{-E_n}$ denotes the $n$-th eigenvalue of the transfer matrix. 
If the transfer matrix is bounded, than $E_n \geq 0$.
Minus the log of the largest eigenvalue is the smallest $E_n$ and is denoted $E_0$.
Proportionality at large values of a time parameter up to exponential corrections is denoted as $\sim$ in Eq.\eqref{Zspectral} and below, and allows $E_0$ to be determined from the partition function as
\begin{equation}
  \begin{split}
    E_0 = \lim_{\beta\rightarrow \infty} -\partial_\beta \ln Z = \lim_{\beta\rightarrow\infty} \frac{1}{\beta}\ln\left[ \frac{Z(\beta)}{Z(\beta+1)} \right],
  \end{split}\label{ZEM}
\end{equation}
where the first continuous-time definition and second discrete-time definition are both independent of the overall normalization of $Z$.
$E_0$ can be identified with the free energy of the zero temperature vacuum.
The $E_n$ for $n>0$ similarly can be identified with the free energies of excited states of the vacuum that contribute to the free energy at non-zero temperature.
The spectral representation of the transfer matrix in Eq.~\eqref{Zspectral} allows Euclidean time evolution to be represented by sums of exponentials dominated by a single ground-state exponential decay at large times.
If transfer matrix $\hat{T}$ for a generic QFT is given, then a Hamiltonian can be defined through
\begin{equation}
  \begin{split}
    \hat{T} = e^{-\hat{H}}.
  \end{split}\label{THdef}
\end{equation}
It can be shown through Osterwalder-Schrader reflection positivity~\cite{Osterwalder:1973dx} that the Euclidean Hamiltonian operator coincides with the Hamiltonian operator defined in Minkowski space.

By inserting additional fields in the path integral representing $Z$, sources creating states other than the vacuum can be introduced.
If these sources carry conserved charge, then the theory will evolve according to the spectrum of the Hamiltonian in the sector of the theory with appropriate quantum numbers.
If thermal boundary conditions are employed, then then initial and final state scalar fields must have the same quantum numbers.
Euclidean correlation functions are defined as traces of the product of transfer matrices with additional operators, and can be non-vanishing if these operators have zero net charge.
The Euclidean-spacetime propagator for the scalar field is defined as the two-point correlation function
\begin{equation}
  \begin{split}
    G_\varphi(\v{p},t) = \sum_\v{x} e^{-i\v{p}\cdot \v{x}} \avg{\varphi(\v{x},t)\varphi^\dagger(0)} = \sum_\v{x} e^{-i\v{p}\cdot \v{x}} \int \mathcal{D}\varphi^\dagger\mathcal{D}\varphi \;e^{-S_\varphi}\varphi(\v{x},t)\varphi^\dagger(0).
  \end{split}\label{eq:scalar2ptdef}
\end{equation}
Existence of a transfer matrix guarantees a spectral representation for arbitrary Euclidean correlation functions comprised of products of local field operators, for example
\begin{equation}
  \begin{split}
    G_\varphi(\v{p}=0,t) &= \sum_\v{x}   \tr(\hat{T}^{\beta-t} \hat{\varphi}(\v{x})\hat{T}^t \hat{\varphi}^\dagger(0)) \\
    &= V \int \mathcal{D}\varphi^\dagger \mathcal{D}\varphi \mbraket{\varphi_0}{\hat{T}^{\beta-t}}{\varphi_t} \mbraket{\varphi_t}{\vphantom{\hat{T}^t}\hat{\varphi}(\v{x})}{\varphi_t} \mbraket{\varphi_t}{\hat{T}^t}{\varphi_0} \mbraket{\varphi_0}{\vphantom{\hat{T}^t}\hat{\varphi}^\dagger(0)}{\varphi_0}\\
    &= \sum_{n,m} |Z_{nm}^\varphi|^2 e^{-E_m(\beta-t)} e^{-E_n t} \\
    &\sim e^{-M_\varphi t}
  \end{split}
\end{equation}
where $Z_{nm}^\varphi$ describes the amplitude of the field operator to annihilate state $m$ and create state $n$, that is $Z_{nm}^\varphi=V^{1/2} \mbraket{n}{\hat{\varphi}(\v{x})}{m}$, and $M_\varphi$ is the ground-state energy in the single-particle state created by $\varphi^\dagger$.
Excited state effects are exponentially small in $t$, and thermal effects are exponentially small in $\beta$ (or often $\beta - t$).
Both can be neglected when $t \ll \beta$ is much larger than the relevant energy gap to the first excited state in the spectrum.
$M_\varphi$ can be straightforwardly determined from the large-time behavior of the Euclidean correlation function by forming an effective mass analogous to Eq.~\eqref{ZEM}.

LQCD was first constructed by Wilson~\cite{Wilson:1974sk} in the path integral formalism with $\mathfrak{su}(3)$ valued gauge fields $A_\mu$ replaced by $SU(3)$ valued gauge fields $U_\mu$ sometimes called gauge links,
\begin{equation}
  U_\mu(x) = \mathcal{P}\exp\left( \int_x^{x+\hat{\mu}} A_\mu(x)dx\mu \right),
\end{equation}
where $A_\mu$ again is anti-Hermitian.
Introducing plaquettes
\begin{equation}
  U_{\mu\nu}(x) = U_\mu(x)U_\nu(x+\hat{\mu})U_\mu^\dagger(x+\hat{\nu})U_\nu^\dagger(x) = U_{\nu\mu}^\dagger(x),
\end{equation}
for $\mu\neq\nu$ and $U_{\mu\mu}(x)=0$, the Wilson action for pure Yang-Mills theory is defined by
\begin{equation}
  S_G(U)  = \frac{1}{g^2}\sum_{x;\mu,\nu}\text{Tr} \left[ 1 - U_{\mu\nu}(x) \right].
\end{equation}
Defining the path integral measure $\mathcal{D}U$ to be the $SU(3)$ Haar measure, gauge invariance of the Wilson action $S_G$ guarantees that only color-singlet states contribute to the QCD partition function.
At each spacetime point in the lattice, coordinate basis states can be defined for the spatial components of the gauge field.
The path integral formulation includes time-like gauge fields $A_0$ that naively give rise to negative-norm states and cannot appear in the full Hilbert space defined from the tensor product as in the scalar example.
Time-like gauge links arise in the Hilbert space formulation of LQCD as projectors that restrict the action of the transfer matrix to the gauge singlet sector of the full Hilbert space.
The gluon part of the physical Hilbert space of LQCD is the gauge singlet projection of the coordinate basis eigenstates of spatial link operators $\ket{U_k(x)}$.

Defining quarks in LQCD is more complicated because the naive discretized action for a Dirac fermion in $D$ spacetime dimensions actually describes $2^D$ degenerate fermions.
This fermion doubling problem was first solved by Wilson, who demonstrated that adding higher dimensional operators to the action that are irrelevant in the continuum limit is enough to break this degeneracy and give all but one fermion a large mass on the order of the inverse lattice spacing.
The Wilson quark action is given by
\begin{equation}
  \begin{split}
    S_F(q,\bar{q},U) &= \sum_{x;\mu} \bar{q}(x)\left[\frac{1}{2}(D_\mu -D_\mu^\dagger)\gamma_\mu + \frac{1}{2} D_\mu^\dagger D_\mu\right]q(x) + \sum_x \bar{q}(x)m_q q(x), \\
\end{split}
\label{eq:wilsonquarkdef}
\end{equation}
where
\begin{equation}
  \begin{split}
    D_\mu q(x) &= U_\mu(x)q(x+\hat{\mu}) - q(x),\\
    D_\mu^\dagger q(x) &= U_\mu^\dagger(x-\hat{\mu})q(x-\hat{\mu}) - q(x),\\
    \sum_\mu D_\mu^\dagger D_\mu q(x) &= 8\,q(x) - \sum_\mu\left[ U_\mu(x)q(x+\hat{\mu}) + U_\mu^\dagger(x-\hat{\mu})q(x-\hat{\mu}) \right].
\end{split}
\label{eq:quarkgaugecovariant}
\end{equation}
Because the naive fermion action represented by the first two terms in  Eq.~\eqref{eq:wilsonquarkdef} is linear in derivatives, its discrete Fourier transform will involve a sine of the momentum.
Sine vanishes at both zero and $\pi$, and so the naive fermion propagator includes poles corresponding to isolated particles of mass $m_q$ both at momentum zero and $\pi$.
These spurious poles with non-zero momentum represent the $2^D-1$ fermion doublers, and are removed when the last term in Eq.~\eqref{eq:wilsonquarkdef} is included in the fermion propagator.
An explicit construction of the LQCD transfer matrix with Wilson's quark and gluon actions was given by L{\"u}scher~\cite{Luscher:1977} and shows that the Wilson term can be incorporated through a gluon field dependent renormalization of the quark states.

The Wilson action demonstrates that it can be helpful to add higher dimensional terms to the LQCD action.
These terms are irrelevant in the continuum limit and so leave the physics under study unchanged, but they can be chosen to reduce the size of systematic errors associated with non-zero lattice spacing.
Lattice artifacts can be studied in EFT by considering the Symanzik action, the most general action including higher dimensional operators consistent with the symmetries of the discretized lattice field theory~\cite{Symanzik:1983dc}.
Interactions terms in the Symanzik action are organized in a derivative expansion so that lattice artifacts can be computed as a power series in ratios of low-energy scales to the lattice cutoff.
Adding higher-dimensional terms to the LQCD action will effectively shift the renormalized values of couplings in the Symanzik action.
Tuning the coefficients of higher-dimensional operators explicitly included in the action can be used to cancel the effects of quantum fluctuations and drive the renormalized couplings associated with particular lattice artifacts to zero or another specified value.
By tuning all the renormalized low-energy constants parametrizing lattice artifacts at a given order in the derivative expansion to zero, the parametric scaling of lattice artifacts in low-energy observables can be systematically improved from linear to quadratic or better.


Many different LQCD groups use many different discretized QCD actions, and in particular difficulties related to chiral symmetry and additive mass renormalization\footnote{See Ref.~\cite{Kaplan:2009yg} for a review of lattice fermions and chiral symmetry.} motivated the construction of staggered~\cite{Kogut:1974ag}, domain wall~\cite{Kaplan:1992bt,Shamir:1993zy,Furman:1994ky}, and overlap~\cite{Narayanan:1993sk,Narayanan:1994gw,Neuberger:1997fp} quark actions.
Some improved actions can be shown to not possess a positive-definite transfer matrix~\cite{Luscher:1984is}.
It is expected, and assumed below, that all discretized LQCD actions are in the same universality class and approach the same continuum limit fixed point, and that
results from LQCD discretizations such as domain wall fermions that do not possess a positive-definite transfer matrix only differ by perturbatively small amounts from the well-defined results of the Wilson action.
Performing calculations at multiple lattice spacings and preferably with multiple different discretized actions allows systematic uncertainties related to these and other lattice artifacts to be quantified.


Exploratory LQCD calculations discussed in Chapters~\ref{chap:statistics}-\ref{chap:PR} were generated by the NPLQCD collaboration and use 
the improved L{\"u}scher-Weize gauge action action~\cite{Luscher:1984xn} and
the clover-improved quark action including a term of the form $\bar{q}\sigma_{\mu\nu}G_{\mu\nu}q$ constructed by Sheikholeslami and Wohlert~\cite{Sheikholeslami:1985ij}.
See Ref.~\cite{Orginos:2015aya} for further details.

%Computing large statistical ensembles of gauge field configurations and hadronic correlations functions is challenging, and it is worth highlighting that even with state-of-the-art algorithms, large-scale parallel supercomputing and research at the interface of physics and computing continues to be essential part of the success LQCD, see Ref.~\cite{Boyle:2017wul} for a recent review.
%Most modern techniques for gauge field ensemble generation are based on the hybrid Monte Carlo algorithm~\cite{Duane:1987de,Gottlieb:1987mq,Sexton:1992nu,Hasenbusch:2001ne,Luscher:2005rx,Clark:2006fx}, see Refs.~\cite{Jung:2010jt,Luscher:2010ae} for further discussion and references.
