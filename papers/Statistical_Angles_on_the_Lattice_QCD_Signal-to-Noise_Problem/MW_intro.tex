\chapter{Introduction}
 
%%%%%%%%%%%%
Fluctuations of quantum fields throughout spacetime lead to a seemingly classical universe ultimately governed by probabilistic laws.
Quantum field theory (QFT) describes statistical distributions for experimental results consistent with the laws of quantum mechanics and special relativity.
Experiments to date are well-described by a quantum field theory (QFT) called the Standard Model (SM) of particle physics in conjunction with general relativity.
There are notable exceptions,
namely neutrino masses, dark matter, dark energy, and  quantum gravity, 
and understanding the nature of these exceptions is a central goal of modern physics.
Another aim is to understand the connection between microscopic fields and macroscopic matter.
These are not independent goals, and relating macroscopic properties of matter to statistical functions of SM fields allows SM predictions to be precisely tested and new physics beyond the Standard Model (BSM) to be potentially discovered.
Connecting microscopic fields and macroscopic observables also allows matter in extreme environments to be theoretically understood and its properties predicted;
for example high-frequency gravitational waves emitted from neutron star collisions
that are expected to be observed at LIGO
reflect in part the microscopic equation of state of quarks and gluons.

The electron field in the SM is associated with quantum states containing definite numbers of point-like charged particles called electrons.
The quark field is superficially similar to three copies of the electron field with distinct charges called colors.
However, strong quantum fluctuations in the vacuum confine quarks into color-neutral composite objects called hadrons:
mesons such as pions, baryons such as protons and neutrons, and their bound states such as atomic nuclei.
Much effort in nuclear and particle physics is directed towards understanding the connection between experimentally observable hadrons and nuclei and the theory of fluctuating quark and gluon fields.
Establishing this connection quantitatively allows
reliable theoretical calculations of nuclear structure and reactions
to be made without further assumptions besides the validity of the SM.
Reliable predictions can in turn be used to 
better understand exotic astrophysical environments such as neutron stars and supernovae,
improve models of nuclear forces and theoretical inputs to models of fusion and fission in stars and reactors,
and inform experimental searches for BSM physics relying on precise measurements of hadrons and nuclei.

In high-energy particle collisions, the SM can be accurately solved using perturbative techniques 
where quantum fluctuations are assumed to make small corrections to classical results.
In low energy density regions such as everyday materials,
and even at higher energy density regions like the center of the sun, 
strong nuclear interactions described in the SM by quantum chromodynamics (QCD) cannot be treated perturbatively.
The only known method for reliably solving QCD non-perturbatively for generic low-energy systems is lattice QCD (LQCD).
In LQCD, quark and gluon fields are stochastically sampled on a discrete set of points forming a spacetime lattice using Monte Carlo techniques.
Physical observables are identified with functions of the quark and gluon fields averaged over quantum fluctuations.
Predictions can be directly compared with experiment
in order to identify and understand strong interaction effects in particle and nuclear physics.
LQCD predictions can also be taken as input to effective field theory (EFT) and quantum many-body methods
that can then predict the properties of other systems with controlled uncertainties.

Broad nuclear theory efforts will benefit from the ability of LQCD to
reliably calculate properties of strongly interacting matter 
that are difficult or impossible to access experimentally.
Stellar fusion and other astrophysical reactions responsible for the synthesis of heavy elements take place at high temperatures difficult to create on earth, 
and in some cases fusion cross-sections must be determined phenomenologically by nuclear theory.
%The first LQCD calculation of proton-proton fusion, the start of the stellar fusion reaction chain powering the sun, was recently performed.
%Uncertainties in this exploratory LQCD study are larger than theoretical uncertainties from other methods,
%but with larger computing resources and improved algorithms
%it is possible to systematically improve the precision of LQCD far beyond the theoretical accuracy of other nuclear theory methods.
Precise LQCD calculations of electroweak fusion reactions would allow these cross-sections to be determined from first principles.
Calculations of other electroweak reactions including neutrino-nucleus scattering would have applications to experimental neutrino efforts such as DUNE and non-proliferation efforts.
LQCD can also provide predictions of the quark and gluon structure of nuclei that can be tested by and used to interpret results from a proposed electron-ion collider (EIC).
Precise LQCD calculations of even simple systems can be used to determine poorly-known parameters in nuclear many-body models,
and for example precise determinations of three-nucleon forces would improve the accuracy of nuclear structure and reaction calculations of heavy nuclei at FRIB.
Future LQCD studies of fusion reactions involving more complex systems
can accurately determine poorly known nuclear reaction rates
needed as inputs to nuclear many-body models 
that in turn inform
macroscopic models of supernovae, neutron stars, and reactors.

Nuclear and hadronic experiments searching for BSM physics need LQCD calculations
to reliably connect experimental results
to constraints on BSM theories.
Dark matter direct detection experiments and searches and for fundamental symmetry violation, 
including neutrinoless double-beta decay, neutron electric dipoles moments, proton decay, and neutron-antineutron oscillations, 
require accurate QCD predictions in order to reliably constrain BSM theory.
In fundamental symmetry searches, QCD results are not needed to see a signal of BSM physics, but they are needed to turn experimental results into quantitative predictions to be verified by other experiments and to establish reliable bounds on theory from null results.

In these and other applications of LQCD to particle and nuclear physics,
precise calculations can be performed for few-particle systems,
but calculations of nuclear matter and of large nuclei face infamous obstacles
called the sign problem and the signal-to-noise (StN) problem respectively.
The sign problem refers to issues in
numerically calculating integrals of oscillatory functions.
It arises in Monte Carlo calculations
where results are sensitive to delicate cancellations between opposite sign contributions that are only apparent with high statistics. 
The StN problem refers to the issue of exponential precision loss in calculations of protons, neutrons, and nuclei.
The sign and StN problems have so far obstructed LQCD calculations of the equation of state of cold, dense matter inside neutron stars and the structure and reactions of large nuclei.


This thesis presents statistical observations
to better understand
and statistical techniques to tame the StN problem for baryon and multi-baryon correlation functions in LQCD.
Building on observations that the nucleon StN problem can be associated with a random walk in the phase of complex correlation functions,
new statistical estimators are proposed that possess constant, rather than exponentially degrading, precision but have a bias that must be removed by extrapolation.
These estimators are shown to reproduce LQCD results for single- and multi-particle systems using only late-time correlation functions too noisy to be analyzed by previous methods.

Chapter~\ref{chap:statistics} describes observations of the statistical distributions of nucleon correlation functions in LQCD.
By considering a decomposition of correlation functions into magnitude and phase, the StN problem is shown to follow from the sign problem obstructing Monte Carlo sampling of non-positive definite functions.
The probability distributions of the magnitudes and phases are correlation functions are shown to possess interesting structure,
and in particular time evolution of the phase of the correlation function is observed to resemble a L{\'e}vy flight on the unit circle.
Empirical evidence for heavy-tailed distributions resembling stable distributions is found for 
differences of phases at dynamically correlated times.
Observations from this chapter were previously described in:

\begin{enumerate}
\item M.~L.~Wagman and M.~J.~Savage, ``Taming the Signal-to-Noise Problem in Lattice QCD by Phase Reweighting,'' arXiv:1704.07356 [hep-lat].
\end{enumerate}

Chapter~\ref{chap:PR} describes phase reweighting, an improved statistical estimator for complex correlation functions
motivated by the preceding observations.
Phase reweighting removes the exponential degradation of precision arising from the StN problem,
at the cost of introducing a bias.
The bias can be removed by extrapolating to a well-defined limit.
As this limit is approached the bias becomes exponentially smaller but precision becomes exponentially worse.
First results of phase reweighting for meson, baryon, and multi-baryon systems are shown to be encouraging.
Phase reweighting and its first applications previously appear in:

\begin{enumerate}
\item M.~L.~Wagman and M.~J.~Savage, ``On the Statistics of Baryon Correlation Functions in Lattice QCD,'' arXiv:1611.07643 [hep-lat].
\end{enumerate}


Chapter~\ref{chap:mesons} describes the statistics of real but sometimes negative meson correlation functions in LQCD.
The real part of the wrapped-normal log-normal distribution introduced in Chapter~\ref{chap:statistics} provides a good description of isovector meson correlation functions.
Lepage-Savage scaling is shown to be a generic property of complex correlation functions, and to apply to isovector meson correlation functions.
The sample mean of an ensemble of $N$ isovector meson correlation functions show systematic deviations from the true average unless $\avg{cos \theta_i} \geq 1/\sqrt{N}$ in accordance with the expectations of circular statistics~\cite{Fisher:1995}.
These observations suggest that the picture of L{\'e}vy Flights on the unit circle used to motivate phase reweighting applies, and real but non-positive-definite meson correlation functions can be viewed as projections of complex correlation functions with similar statistical behavior to the baryons.
Phase reweighting has been used to the StN problem for the $\rho^+$ in Chatper~\ref{chap:PR}, and its success as well as the failure of the ratio estimator introduced in Chapter~\ref{chap:statistics} and further explored in this chapter.
The ground state energies of other isovector meson channels, previously inaccessible to techniques based on spectroscopy in the golden window, are precisely extracted from the noise region of phase-reweighted correlation functions.

The remainder of this chapter provides background on QCD in Sec.~\ref{sec:qcd}, lattice field theory in Sec.~\ref{sec:lattice}, and meson and baryon correlation functions in Sec.~\ref{sec:corr} that is helpful for the subsequent chapters.
%Mesons, bound states of a quark and an antiquark, are found to have statistical behavior consistent with expectations from circular statistics despite sampling only binary-valued random signs.
%Symmetry considerations describing the time evolution of higher moments of the real parts of meson correlation functions are introduced and verified.
%The spacetime structure of correlation function probability distributions is explored, and connections are found between mesons and baryons, bound states of three quarks such as protons and neutrons.
%The role of non-perturbative QCD dynamics including instantons and chiral symmetry breaking in statistical distributions is briefly explored.
%Results from this chapter appear in:


