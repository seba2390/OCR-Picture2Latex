\section {Controller Design for The Network Power Balance Case}
\label{sec:npb}

In the per-node balance case, individual control areas rebalance power within their own areas
after disturbances. However, in many circumstances, it may be more efficient for all control 
areas to eliminate power imbalance of the overall system in a coordinated manner. This can 
be modeled as the condition:
\bq
\label{eq:balance.net}
\sum\nolimits_j P^g_j & = &  \sum\nolimits_j \left( P^l_j + p_j \right)
\eq
In this case the tie-line flows may not be restored to their pre-disturbance values.
To ensure that they are within operational limits, the constraints \eqref{eq:lineConstraint}
are imposed.

Even though the philosophy of the controller design as well as the proofs are similar to the per-node case, 
the details are much more complicated.   
Our presentation will however be brief where there is no confusion.

%In this situation, we have the following assumption
%\bi
%\item[A4:]  For $\forall j\in N$, $\sum\nolimits_j(\underline{P}^g_j - \overline{P}^l_j )<\sum\nolimits_j p_j$, $\sum\nolimits_j(\overline{P}^g_j - \underline{P}^l_j ) > \sum\nolimits_j p_j $.
%In addition, in the steady state, for $\forall j\in N$, the generation ($\underline P_j^g, \overline P_j^g$) and controllable load constraints ($\underline P_j^l, \overline P_j^l$)  cannot be reached  with the powers of tie lines connected to area $j$ ($\underline P_{ij}, \overline P_{ij}$, $i\in N_j$)  reaching their limits simultaneously.
%\ei 

%Similar to A1, assumption A4 is to guarantee that: 1) the network should has power margin;2) each area should has power margin when balances the load demands. Otherwise, the network or some area is small signal unstable.

\subsection{Control goals}
In the network power balance case, the control goals are formalized as the following 
optimization problem.
\begin{subequations}
        \bq
        \text{NBO:}\min &\!\!\!\!\!\!& \sum \limits_j \frac{\alpha_j}{2} \left(P^g_j\right)^2 
        +  \sum \limits_j \frac{\beta_j}{2} \left(P^l_j\right)^2
% \nonumber\\     
%         & \!\!\!\!\!\! &
         +  \sum \limits_j \frac{D_j}{2} \omega_j^2 +  \sum \limits_j \frac{z_j^2}{2}
          \nonumber\\     
        \label{eq:opt.2o}
        \\
        \text{over} & \!\!\!\!\!\! & x := (\theta, \phi, \omega, P^g, P^l) \text{ and }
        u := (u^g, u^l)
        \nonumber
        \\ 
        \text{s. t.}  
        & \!\!\!\!\!\! & \eqref{eq:OpConstraints.1},   
        \nonumber
        \\
        & \!\!\!\!\!\! & 
        P^g_j = P^l_j  + p_j +U_j(\theta, \omega) \quad j\in N
        \label{eq:opt.2a}
        \\
        & \!\!\!\!\!\! & 
        P^g_j = P^l_j  + p_j +\hat{U}_j(\phi) \quad j\in N
        \label{eq:opt.2b}
        \\
        & \!\!\!\!\!\! & \underline{\theta}_{ij} \le  \phi_i-\phi_j \le \overline{\theta}_{ij}
        \label{eq:opt.2c}, \quad (i,j) \in E 
        \\
        & \!\!\!\!\!\! & P^g_j \ = \ u^g_j, \quad j\in N
        \label{eq:opt.2d}
        \\
        & \!\!\!\!\!\! & P^l_j \ = \ u^l_j, \quad j\in N
        \label{eq:opt.2e}
        \eq
where $\alpha_j>0$, $\beta_j>0$ are constant weights; $z_j$ is a shorthand defined for convenience as        
\bqn
        z_j & := & P^g_j-P^l_j-p_j-\hat{U}_j(\phi)
\eqn
$U(\theta, \omega) := D\omega + CBC^T\theta$, and $\hat{U}(\phi)  :=  CBC^T \phi$.

While $\theta$ represents the phase angles in the physical power network,
$\phi$ is a cyber quantity that can be interpreted as virtual phase angles
(see remarks below).
The matrices $D$, $C$ and $B$ are defined as in the previous section.
\label{eq:opt.2}
\end{subequations}

As in the per-node  case, we define the variables $\tilde\theta_{ij}:=\theta_i - \theta_j$,
or in vector form, $\tilde\theta := C^T\theta$.
As we fix $\theta_0 := 0$ to be a reference angle under assumption A1, 
$\tilde\theta = C^T\theta$ defines a bijection between $\theta$ and $\tilde\theta$. 
Similarly we define 
$\tilde\phi_{ij}:=\phi_i - \phi_j$ or $\tilde{\phi}:=C^T\phi$, and $\phi_0 := 0$ so 
there is a bijection between $\phi$ and $\tilde\phi$.  
Note that both $\tilde{\theta}$ and $\tilde \phi$ are restricted to the column space of $C^T$. 
We will use $(\theta, \phi)$  and $(\tilde{\theta}, \tilde{\phi})$ interchangeably.
For instance we will abuse notation and write $\hat U(\phi):=CBC^T \phi$ or 
$\hat U(\tilde\phi):= CB\tilde \phi$.

 
% Note that we have abused and use the notation $\phi_{ij}=\phi_i-\phi_j$. 


We now summarize some of the interesting properties of NBO \eqref{eq:opt.2}
that will be proved formally in the next two subsections.  
We first compare NBO \eqref{eq:opt.2} with PBO in \cite{Wang:DistributedFrequency} for
the per-node balance case.
\begin{remark}[Comparison of NBO and PBO]
        \bee
        \item  Intuitively the network balance condition \eqref{eq:balance.net} is a relaxation 
        of the per-node balance condition (3) in \cite{Wang:DistributedFrequency}, and hence we
        expect that the optimal cost of NBO lower bounds that of PBO.   This is indeed the case, as we now argue.
		Constraint \eqref{eq:opt.2b} implies that any
        feasible point of \eqref{eq:opt.2} has $z_j=0$ and hence these two optimization
        problems have the same objective function.  
        % Both aims to recover the nominal frequency while minimizing the regulation costs.  
        Their variables and constraints 
        are different in that  PBO directly enforces the per-node balance  
        condition while NBO \eqref{eq:opt.2} has the additional 
        variable $\phi$ and constraints \eqref{eq:opt.2b}\eqref{eq:opt.2c}.
        Any optimal point $(\theta^*, \omega^*, P^{g*}, P^{l^*})$ for PBO
        however defines a feasible point $(\theta^*, \phi, \omega^*, P^{g*}, P^{l*})$ 
        for NBO \eqref{eq:opt.2} with the same cost where $\phi=\theta^*$.
        The point $(\theta^*, \phi, \omega^*, P^{g*}, P^{l*})$ satisfies 
        \eqref{eq:opt.2b}\eqref{eq:opt.2c} 
        because $(\theta^*, \omega^*, P^{g*}, P^{l^*})$ satisfies \eqref{eq:opt.2a},
        $\omega^*=0$ and $\theta^*_i = \theta^*_j$ by Theorem 2 in \cite{Wang:DistributedFrequency}, and
        $\underline{\theta}_{ij} \leq 0 \leq \overline{\theta}_{ij}$ by assumption A1.
     
        \item
        Even though any feasible point of \eqref{eq:opt.2} has $z_j=0$, the 
        objective function is augmented with $z_j^2$ to improve convergence 
        (see \cite{Feijer:Stability}).         
        \eee
\end{remark}

Even though neither the network balance condition \eqref{eq:balance.net} nor the
line limits \eqref{eq:lineConstraint} are explicitly enforced in \eqref{eq:opt.2}, 
they are satisfied at optimality (Theorem \ref{thm:6} below).
Indeed, the virtual phase angles $\phi$  
and the conditions \eqref{eq:opt.2a}--\eqref{eq:opt.2c} are carefully designed to
enforce these conditions as well as to restore the nominal frequency $\omega^*=0$
at optimality.  This technique is previously used in \cite{Mallada-2017-OLC-TAC}.
\begin{remark}[Virtual phase angles $\phi$]
\bee
        \item
        Under mild conditions, $\omega^*=0$ at optimality for both PBO and NBO.  
%        For PBO, this is a consequence of the per-node power balance requirement
%        \eqref{eq:balance.node} and the constraint \eqref{eq:opt.1a}; see Lemma 
%        \ref{lemma:1}.
        For NBO, this is a consequence of the constraint \eqref{eq:opt.2b} on $\phi$; 
        see Lemma \ref{lemma:6}.
        Summing \eqref{eq:opt.2b} over 
        all $j\in N$ also implies the network  balance condition \eqref{eq:balance.net}
        since $\textbf{1}^T \hat U(\phi) = \textbf{1}^T CBC^T\phi = 0$.

        \item
        In PBO, $\theta^*_i = \theta^*_j$ at optimality (i.e., tie-line flows are restored
        $P_{ij}^*=0$) and $U(\theta^*, \omega^*)=0$.  This does not necessarily hold in NBO.  
        However $\phi$ is regarded as virtual phase angles because, at optimality, 
        $\phi^*$ differs from the real phase angles $\theta^*$ only by a constant, 
        $\phi^*-\theta^*=\textbf{1}(\phi_0-\theta_0)$ (Lemma \ref{lemma:6} in
        the appendix).
        Hence $C^T\phi^*-C^T\theta^*=C^T\cdot\textbf{1}(\phi_0-\theta_0)=0$,  implying 
        $\tilde\phi_{ij}^*=\tilde\theta_{ij}^*$. 
        Then the constraints \eqref{eq:opt.2c} are exactly the flow constraints \eqref{eq:lineConstraint}.
        In other words, we impose the flow constraints on $\tilde{\theta}$ indirectly by enforcing such
        constraints on the virtual angle  $\tilde{\phi}$.  .
\eee
\end{remark}



\subsection{Distributed controller}

Our  control laws are: 
\begin{subequations}
        \bq
        \dot \lambda_j & = &  \gamma^{\lambda}_j \left( P^g_j(t) -  P^l_j(t) - p_j-\hat{U}_j(\tilde\phi(t))\right),
        \  j\in N
\label{eq:control.2a}\\
    \dot {\eta}^+_{ij}&=&\gamma^{\eta}_{ij}[\tilde\phi_{ij}(t)-\overline{\theta}_{ij}]^+_{{\eta}^+_{ij}}\qquad\quad\ \qquad \forall (i,j)\in E
\label{eq:control.2b}\\
    \dot {\eta}^-_{ij}&=&\gamma^{\eta}_{ij}[\underline{\theta}_{ij}-\tilde\phi_{ij}(t)]^+_{{\eta}^-_{ij}}\qquad\quad\ \qquad \forall (i,j)\in E
\label{eq:control.2c}\\
        \dot{\tilde\phi}_{ij} &= & \gamma^{\tilde\phi}_{ij}\left( B_{ij}[\lambda_i(t)-\lambda_j(t)+z_i(t)-z_j(t)]\right.
        \nonumber\\
        &&\left.  \ + \ \eta^-_{ij}(t)-\eta^+_{ij}(t)\right)\qquad\qquad\ \forall (i,j)\in E
\label{eq:control.2d}\\
        u^g_j(t) & = & \left[ 
        P^g_j(t) - \gamma^g_j \left( \alpha_j P^g_j(t) + \omega_j(t)+z_j(t) + \lambda_j(t) \right) \right]_{\underline P^g_j}^{\overline P^g_j}
        \nonumber\\
        &&  \ \, + \ {\omega_j(t)}/{R_j}, \qquad\qquad\qquad\ \ \, j\in N
\label{eq:control.2e}\\
        u^l_j(t) & = & \left[ 
        P^l_j(t) - \gamma^l_j \left( \beta_j P^l_j(t) - \omega_j(t)-z_j(t) - \lambda_j(t)\right)
        \right]_{\underline P^l_j}^{\overline P^l_j}
         \nonumber\\
& & \qquad\qquad\qquad\qquad\qquad\qquad\  j\in N		\label{eq:control.2f}
        \eq
where $\gamma^{\lambda}_j, \gamma^{\eta}_{ij}, \gamma^{\tilde\phi}_{ij}, \gamma^g_j, \gamma^l_j$ are positive constants. For any $x_i, a_i \in \mathbb R$, the operator $[x_i]^+_{a_i}$  is defined by 
        \label{eq:control.2}
\end{subequations}
\bqn
[x_i]^+_{a_i}&:=&\left\{
\begin{array}{ll}
x_i& \text{if} \ a_i>0 \ \text{or} \  x_i>0;\\
0,& \text{otherwise}.
\end{array}
 \right.
\eqn

For a vector case, $[x]^+_a$ is defined accordingly componentwise \cite{cherukuri:asymptotic}.

Here we assume that each node $i$ updates a set of internal states
$(\lambda_i(t), \eta^+_{ij}(t), \eta^-_{ji}(t), \tilde\phi_{ij}(t) )$ according
to \eqref{eq:control.2a}--\eqref{eq:control.2d}.\footnote{For each (directed)
link $(i,j)\in E$ we assume that only node $i$ 
maintains the variables $(\eta^+_{ij}(t), \eta^-_{ji}(t), \tilde\phi_{ij}(t) )$.
In practice, node $j$ will probably maintain symmetric variables to reduce 
communication burden or for other reasons outside our mathematical model here.}
In contrast to the \emph{completely decentralized} control derived in the 
per-node balance case,  here the control is \emph{distributed} where each 
node $i$ updates $(\lambda_i(t), \eta^+_{ij}(t), \eta^-_{ij}(t) )$ using only local 
measurements or computation but requires the information $(\lambda_j(t), z_j(t))$
from its neighbors $j$ to update $\tilde\phi_{ij}(t)$.
Note that $z_i(t)$ is not a variable but a shorthand for (function)
$P^g_j(t)-P^l_j(t)-p_j-\hat{U}_j(\tilde\phi(t))$.
The control inputs $(u^g_i(t), u^l_i(t) )$ 
in \eqref{eq:control.2e}\eqref{eq:control.2f} are
functions of the network state $(P^g_i(t), P^l_i(t), \omega_i(t) )$ and the
internal state $(\lambda_i(t), \eta^+_{ij}(t), \eta^-_{ij}(t), \tilde\phi_{ij}(t) )$. 
We write $u^g_j$ and $u^l_j$ as functions of 
$(P^g_j, P^l_j, \omega_j, \lambda_j)$: for $j\in N$
\begin{subequations}
        \bq
        u^g_j(t) & := & u^g_j \left( P^g_j(t), \omega_j(t), \lambda_j(t), z_j(t)\right)
        \label{eq:control.2a'}
        \\
        u^l_j(t) & := & u^l_j \left( P^l_j(t), \omega_j(t), \lambda_j(t), z_j(t) \right)
        \label{eq:control.2b'}
        \eq
        where the functions are defined by the right-hand sides of
        \eqref{eq:control.2e}\eqref{eq:control.2f}.
        \label{eq:control.2'}
\end{subequations}

Now we comment on the implementation of the control \eqref{eq:control.2}. 

\begin{remark}[Implementation]
\bee
         \item As discussed above, communication is needed only between neighboring
         nodes (areas) to update the variables $\tilde\phi_{ij}(t)$.
%        The internal variables $(\lambda_j(t), \phi_{ij}, \eta^+_{ij}, \eta^-_{ij})$ in \eqref{eq:control.2a}--\eqref{eq:control.2d} are cyber quantities that are
%        computed at each node $j$, where  $(\lambda_j(t),  \eta^+_{ij}, \eta^-_{ij})$ can be computed at node $j$ based on local information while $\phi_{ij}$ requires to know $\lambda_i(t)$ and $z_i(t)$ of its neighboring nodes $i\in N$. 
%        We can also directly compute $z_i(t)$ by differentiating $\lambda_i(t)$, hence further reducing the communication requirement. 
%        Furthermore, according to the definition of $z_i$, we have $\dot {\lambda}_i= \gamma^{\lambda}_i z_i$. Hence, if the constant $\gamma^{\lambda}_i$ is known to node $j$ in advance, $z_i$ can be calculated from $\lambda_i$, with no need to exchange between neighbouring nodes. \slow{But does node $j$ needs $z_i$ to update $\phi_{ij}$, and 
%        if so, how does $j$ get it without communication?}
%    \fliu{I think we made a mistake, since we do need the information of $\phi_i$. So the controller should require to exchange the information  of both $\lambda_i$ and $\phi_i$. Zhaojian, I remember we had discussed this issue before? }
%       \wang{I agree. both $\lambda_i$ and $\phi_i$ should be exchanged. In addition, I think "the requirement to communication can be minimized" should be removed, as there is no definition about what is the minimal communication.} 
        \item
        Similar to the per-node power balance case, we can avoid measuring the load change $p_j$  
        by using \eqref{eq:model.1b} and the definition of $z_j(t)$ to replace \eqref{eq:control.2a} with
        \bqn
        z_j(t) & = & 
         M_j \dot \omega_j + D_j\omega_j(t) - \sum_{i: i\rightarrow j} \! P_{ij}(t)
        +  \sum_{k: j\rightarrow k} \! P_{jk}(t) 
        \nonumber \\
          && \ - \ \hat{U}_j(\tilde\phi(t))
           \nonumber\\
        \dot\lambda_j &=& \gamma^{\lambda}_j \, z_j(t)
        \eqn
\eee
\end{remark}


\newcounter{TempEqCnt}
\setcounter{TempEqCnt}{\value{equation}}
\setcounter{equation}{9}
\begin{figure*}[!t]
	
	%	\begin{equation}
	
	\bq
	L_2(x; \rho) & \!\!\!\! = \!\!\!\! &  
	\frac{1}{2} \left( \sum_{j\in N} \alpha_j \left(P^g_j\right)^2 + \sum_{j\in N} \beta_j \left(P^l_j\right)^2
	+ \sum_{j\in N}  D_j \omega_j^2+ \sum_j z_j^2 \right)
	%	\nonumber\\
	%	&&
	\ + \ \sum_{j\in N} \mu_j \left(P^g_j - P^l_j  - p_j -  D_j \omega_j 
	+ \sum_{i: i\rightarrow j}  B_{ij} \theta_{ij} -  \sum_{k: j\rightarrow k} \! B_{jk}\theta_{jk}  \!\right)
	\nonumber\\ 
	& & 
	+\ \sum_{j\in N} \lambda_j \left(
	P^g_j - P^l_j  - p_j  
	+ \sum_{i: i\rightarrow j} B_{ij} \phi_{ij}
	- \sum_{k: j\rightarrow k} B_{jk} \phi_{jk} \!\right) 
	%	\nonumber\\
	%	&&
	\ + \sum_{(i,j)\in E} \eta^-_{ij} \left(\underline{\theta}_{ij}-\phi_{ij}(t)\right)
	\ + \sum_{(i,j)\in E} \eta^+_{ij} \left(\phi_{ij}(t)-\overline{\theta}_{ij}\right)
	\label{eq:defL.2}
	\eq
%\slow{Should $L_2$ in \eqref{eq:defL.2} be $\cdots 
%\sum_{j\in N} \lambda_j \left(
%	P^g_j - P^l_j  - p_j  
%	+ \sum_{i: i\rightarrow j} B_{ij} \phi_{ij}
%	- \sum_{k: j\rightarrow k} B_{jk} \phi_{jk} \!\right) \cdots$?}
	
	%	\end{equation}
	%\setcounter{equation}{\value{mytempeqncnt}}
	
	%%%%%%%%%
	\hrulefill
	\vspace*{2pt}
\end{figure*}  
\setcounter{equation}{\value{TempEqCnt}} 

\subsection{Design rationale}
\label{subsec:design2}
The controller design \eqref{eq:control.2} is also motivated by a (partial) primal-dual algorithm for
\eqref{eq:opt.2}, as for the per-node power balance case.   
% We first review the form of a standard primal-dual algorithm and then explain that {the closed-loop dynamics
%	\eqref{eq:model.1}\eqref{eq:control.2} carry out such an algorithm for 
%	\eqref{eq:opt.2} in real time over the closed-loop system.} 

\vspace{0.1in}
\noindent
\textbf{Primal-dual algorithms.}
The optimization problem in the network balance case differs from that in the per-node balance case
in the inequalities \eqref{eq:opt.2c} on $\tilde\phi$.
Consider a general constrained convex optimization with inequality constraints:
\bqn
\min_{x\in X} \ \ f(x) & s. t. & g(x) = 0, \ \ h(x)\le 0
\eqn
where $f:\mathbb R^n\rightarrow \mathbb R$, $g:\mathbb R^n\rightarrow \mathbb R^{k_1}$, $h:\mathbb R^n\rightarrow \mathbb R^{k_2}$,
and $X\subseteq \mathbb R^n$ is closed and convex. 
Here an inequality constraint $h(x)\le 0$ is imposed explicitly. 
Let $\rho_1\in\mathbb R^{k_1}$ be the Lagrange multiplier associated
with the equality constraint $g(x)=0$, $\rho_2\in\mathbb R^{k_2}$ that associated with the inequality 
constraint $h(x)\le 0$, and $\rho := (\rho_1, \rho_2)$.  
Define the Lagrangian $L(x; \rho) := f(x) + \rho_1^T g(x)+ \rho_2^T h(x)$.
A standard primal-dual algorithm takes the form:
%The controller design \eqref{eq:control.2} is also motivated by a partial primal-dual algorithm for
%\eqref{eq:opt.2} that dualizes the constraints \eqref{eq:lineConstraint} and \eqref{eq:opt.2b}, 
%as we  explain. 
%The corresponding Lagrangian of \eqref{eq:opt.2} is defined by \eqref{eq:defL.2}. Denote $x:=(P^g, p^l, \omega, \theta,  \phi)$ as the primal variables, 
%$\rho_1 := (\lambda, \mu )$, $\rho_2 := (\eta^+, \eta^- )$ and $\rho := (\rho_1, \rho_2 )$ are Lagrange multipliers of \eqref{eq:opt.2}.
%% and $(\underline\gamma, \overline\gamma, \underline\nu, \overline\nu)$ are nonnegative.  
%
%Similar to (\ref{eq:pma.1}), a primal-dual algorithm with inequality constraints takes the form
%\begin{subequations}
%        \bq
%        \dot x & = & - \Gamma_x\, \frac{\partial L_2}{\partial x} \left(x(t), \rho(t) \right)
%        \label{eq:primaldual.2a}
%        \\
%        \dot \rho & = &   \Gamma_\rho\, \left [ \frac{\partial L_2}{\partial \rho}(x(t), \rho(t)) \right]^+_{\rho}
%        \label{eq:primaldual.2b}
%        \eq
%        \label{eq:primaldual.2}
%\end{subequations}
\begin{subequations}
	\bq
	x(t+1) & := & \text{Proj}_X \left( x(t) \ - \ \Gamma^x\, \nabla_x L(x(t); \rho(t)) \right)
\label{eq:primaldual.2a}
	\\
	\rho_1(t+1) & := & \rho_1(t) \ + \ \Gamma^{\rho_1}\, \nabla_{\rho_1} L(x(t); \rho(t))
	\label{eq:primaldual.2b}
\\
	\rho_2(t+1) & := & \left(\rho_2(t) \ + \  \Gamma^{\rho_2}\, \nabla_{\rho_2} L(x(t); \rho(t))\right)^+
\label{eq:primaldual.2c}
	\eq
where $\Gamma^x, \Gamma^{\rho_1}, \Gamma^{\rho_2}$ are strictly positive diagonal gain matrices.
Here, if $a$ is a scalar then $(a)^+ := \max\{a, 0\}$ and if $a$ is a vector then 
$(a)^+$ is defined accordingly componentwise.
For a dual algorithm, \eqref{eq:primaldual.2a} is replaced by 
	\bq
	x(t) & := & \min_{x\in X}\, L(x; \rho(t))
	\label{eq:da.2b}
	\eq
As for the per-node balance case, all variables in $x(t)$ are updated according to
\eqref{eq:primaldual.2a} except $\omega(t)$ which is updated according to \eqref{eq:da.2b},
as we see below.
\label{eq:primaldual.2}
\end{subequations}

The set $X$ in \eqref{eq:primaldual.2a} is defined by the constraints 
\eqref{eq:OpConstraints.1}:
\bq
\!\!\!\!\!
X & \!\!\!\!\ := \!\!\!\! & \left\{(P^g, P^l) : 
(\underline{P}^g, \underline{P}^l)  \ \leq \ (P^g, P^l) \ \leq \
(\overline{P}^g, \overline{P}^l)  \right\}
\label{eq:defX}
\eq

\vspace{0.1in}
\noindent
\textbf{Controller \eqref{eq:control.2} design.} Let $\rho_1 := (\lambda, \mu)$ be
 the Lagrange multipliers associated with constraints \eqref{eq:opt.2b} and \eqref{eq:opt.2a}
 respectively,
 $\rho_2 := (\eta^+, \eta^-)$ the  multipliers associated with constraints \eqref{eq:opt.2c}, 
 and $\rho:=(\rho_1, \rho_2)$.
Define the Lagrangian of \eqref{eq:opt.2} by (\ref{eq:defL.2}). 
Note that it is only a function of $(x, \rho)$ and independent of $u := (u^g, u^l)$ as we 
treat $u$ as a function of $(x, \rho)$
defined by the right-hand sides of \eqref{eq:control.2e}\eqref{eq:control.2f}.
%The set $X$ in \eqref{eq:primaldual.2a} or \eqref{eq:da.2b} is the same as (\ref{eq:defX}).

The closed-loop dynamics \eqref{eq:model.1}\eqref{eq:control.2} carry out 
        an approximate primal-dual algorithm \eqref{eq:primaldual.2} for 
        solving \eqref{eq:opt.2} in real time over the coupled physical power network and cyber computation. 
Since the reasoning is similar to the per-node balance case, we only provide a summary.
Rewrite the Lagrangian $L_2$ in  vector form 
\setcounter{equation}{10} 
\bq
\label{eq:defL.21}
L_2(x; \rho) &  =& \frac{1}{2} \left( (P^g)^T A^g P^g + (P^l)^T A^l P^l + \omega^T D \omega + z^Tz \right)
\nonumber \\
&& \ + \ \lambda^T \!\! \left( P^g - P^l - p - CB\tilde\phi \right) 
\nonumber \\
& & \ + \ \mu^T \!\! \left( P^g - P^l - p -D\omega - C B\tilde \theta \right)
\nonumber \\
&& \ + \ (\eta^+)^T \left(\tilde \phi-\overline{\theta} \right) \ + \ (\eta^-)^T\left(\underline{\theta}-\tilde\phi\right)
\eq
where $A^l := \diag(\beta_j, j\in N)$, $B:=\diag(B_{ij}, (i,j)\in E)$.
%\slow{Should $L_2$ be
%\bq
%\label{eq:defL.21}
%L_2&  =& \frac{1}{2} \left( (P^g)^T A^g P^g + (P^l)^T A^l P^l + \omega^T D \omega + z^Tz \right)
%\nonumber \\
% && \ + \ \lambda^T \!\! \left( P^g - P^l - p - CB\tilde\phi \right) 
% \nonumber \\
% & & \ + \ \mu^T \!\! \left( P^g - P^l - p -D\omega - C B\tilde \theta \right)
%\nonumber \\
%&& \ + \ (\eta^+)^T \left(\tilde \phi-\overline{\theta} \right) \ + \ (\eta^-)^T\left(\underline{\theta}-\tilde\phi\right)
%\eq
%}

First, the control \eqref{eq:control.2b}\eqref{eq:control.2c} can be interpreted as
a continuous-time version of the dual update \eqref{eq:primaldual.2c}
on the dual variable $\rho_2 := (\eta^+(t), \eta^-(t))$:
\begin{subequations}
\bqn
\dot \eta^+ & = & \Gamma^{\eta} \, \left[ \nabla_{\eta^+} L_2 (x(t); \rho(t)) \right]^+_{\eta^+(t)}
\\
\dot \eta^- & = & \Gamma^{\eta^-} \, \left[ \nabla_{\eta^-} L_2 (x(t); \rho(t)) \right]^+_{\eta^-}
\eqn
where $\Gamma^{\eta} := \diag(\gamma^{\eta}_{ij}, (i,j)\in E)$.

Second, the control \eqref{eq:control.2a} carries out the dual update 
\eqref{eq:primaldual.2b} on $\lambda(t)$:  
        \bq
        \dot\lambda & = &  \Gamma^{\lambda}\ \nabla_{\lambda} L_2 (x(t), \rho(t))
        \label{eq:dual.2d}
        \eq
where $\Gamma^{\lambda} := \diag(\gamma^{\lambda}_j, j\in N)$.  
The swing dynamic \eqref{eq:model.1b} carries out  
the dual update \eqref{eq:primaldual.2b} on $\mu(t)$ because, 
as in the per-node balance case, we can identify $\mu(t) \equiv \omega(t)$ so that
        \bq
        \dot\mu & = & \dot\omega \ \ = \ \ M^{-1}\ \nabla_\mu L_2 (x(t); \rho(t))
        \label{eq:dual.2c}
        \eq
where $M := \diag(M_j, j\in N)$.  

Finally we  show that \eqref{eq:model.1a},
\eqref{eq:model.1c}, \eqref{eq:model.1d},  and \eqref{eq:control.2d} implement a
mix of the primal updates \eqref{eq:primaldual.2a} and \eqref{eq:da.2b} on the primal variables
$x := (\tilde{\theta}(t ); \tilde{\phi}(t ); \omega(t ); P^g(t ); P^l(t ))$.
Setting $\omega(t) \equiv \mu(t)$ is equivalent to the primal update \eqref{eq:da.2b}
on $\omega(t)$, as in the per-node balance case.   Moreover the
control laws \eqref{eq:control.2e}\eqref{eq:control.2f} are then equivalent to 
\begin{align}
        T^g\dot P^g & \ =\  \left[ P^g(t) - \, \Gamma^g\, \nabla_{P^g} L_2 (x(t), \rho(t))
        \right]_{\underline P^g}^{\overline P^g} \ - \ P^g(t)
        \label{eq:primal.2a} \\    
        T^l\dot P^l & \ =\  \left[ P^l(t) - \, \Gamma^l\, \nabla_{P^l} L_2 (x(t), \rho(t))
        \right]_{\underline P^l}^{\overline P^l} \ - \ P^l(t)    
        \label{eq:primal.2b} 
\end{align}
i.e., the generator and controllable load at each node $j$ carry out the primal update 
\eqref{eq:primaldual.2a}.  
For $(\tilde\theta, \tilde\phi)$, \eqref{eq:model.1a} and \eqref{eq:control.2d} are equivalent
to the primal update \eqref{eq:primaldual.2a}:
        \bq
        \dot{ \tilde{\theta}}& = & - B^{-1}\nabla_{\tilde\theta} L_2 (x(t), \rho(t))
        \label{eq:primal.2c}
\\
        \dot {\tilde{\phi}} & = & -\Gamma^{\tilde{\phi}} \ \nabla_{\tilde{\phi}} L_2 (x(t), \rho(t))
        \label{eq:primal.2d}
        \eq
        where $\Gamma^{\tilde{\phi}} := \diag(\gamma^{\tilde\phi}_{ij}, (i,j)\in E)$
        \label{eq:primaldual.3}. 
\end{subequations}


\subsection{Optimality of equilibrium point}
\label{subsec:optimality.1}

In this subsection, we address the optimality of the equilibrium point of the closed-loop system \eqref{eq:model.1}\eqref{eq:control.2}. 
Given an $(x, \rho) := \left( (\tilde{\theta}, {\tilde{\phi}}, \omega,  P^{g}, P^{l}), (\lambda, \mu), (\eta^-, \eta^+)\right)$, recall that
the control input $u(x, \rho_1, \rho_2)$ is given by \eqref{eq:control.2'}.
\begin{definition}
        \label{def:ep.2}
        A point $(x^*, \rho^*) := ( \tilde{\theta}^*, {\tilde \phi}^*, \omega^*,  P^{g*}, P^{l*}, \lambda^*, \eta^{+*}, $ $ \eta^{-*}, \mu^* )$
        is an \emph{equilibrium point} or an \emph{equilibrium} of the closed-loop system 
        \eqref{eq:model.1}\eqref{eq:control.2} if 
        \bee
        \item The right-hand side of \eqref{eq:model.1} vanishes at $x^*$ and $u(x^*, \rho^*)$.  
        \item The right-hand side of \eqref{eq:control.2a}--\eqref{eq:control.2d} vanishes at $(x^*, \rho^*)$.
        \eee
\end{definition}

\begin{definition}
        A point $(x^*, \rho^*)$ is \emph{primal-dual optimal} if $(x^*, u(x^*, \rho^*))$ is optimal 
        for \eqref{eq:opt.2} and $\rho^*$ is optimal for its dual problem.
\end{definition}

We  make the following assumption:
\bi
\item[\textbf{A2:}]  
     The problem (\ref{eq:opt.2}) is feasible. 
\ei 
 
 The following theorem characterizes the correspondence between the equilibrium of the closed-loop system \eqref{eq:model.1}\eqref{eq:control.2} and the primal-dual optimal solution of  \eqref{eq:opt.2}. 
\begin{theorem}
        \label{thm:5}
        Suppose  A2  holds.   A point  $(x^*, \rho^*)$ is primal-dual optimal if
        and only if  $(x^*, \rho^*)$ is an equilibrium of  closed-loop 
        system \eqref{eq:model.1}\eqref{eq:control.2}  satisfying \eqref{eq:OpConstraints.1}
        and $\mu^*=0$.
\end{theorem}

Next result says that, at equilibrium, the network balance condition \eqref{eq:balance.net} and 
line limits \eqref{eq:lineConstraint} are satisfied and the nominal frequency is
restored.  Moreover the equilibrium is unique.
\begin{theorem}
        \label{thm:6}
        Suppose  A1 and A2 hold.  Let $(x^*, \rho^*)$ be primal-dual optimal. 
        Then
        \bee
%        \item for $(i,j)\in E$, if $\underline \theta _{ij}<\phi^* _{ij}<\overline \theta_{ij}$, $\lambda_i^*=\lambda_j^*$.
        \item The equilibrium
        $(x^*, \mu^*)$ is unique, with $(\theta^*, \phi^*)$ being unique
        up to (equilibrium) reference angles $(\theta_0, \phi_0)$.
%        \item
%        $\rho^*$ is unique with the condition that $\sum\nolimits_j(\underline{P}^g_j - \overline{P}^l_j )<\sum\nolimits_j p_j$,  $\sum\nolimits_j(\overline{P}^g_j - \underline{P}^l_j ) > \sum\nolimits_j p_j $, $\forall j\in N$ and in the steady state, for $\forall j\in N$, the generation ($\underline P_j^g, \overline P_j^g$) and controllable load constraints ($\underline P_j^l, \overline P_j^l$)  cannot be reached  with   tie lines powers connected to area $j$ ($\underline P_{ij}, \overline P_{ij}$, $i\in N_j$)  reaching their limits simultaneously.        
        % \item $x^*$ achieves per-node power balance \eqref{eq:balance.node} and
        %               satisfies the operational constraints \eqref{eq:OpConstraints.1};
        \item The nominal frequency is restored, i.e., $\omega^*_j=0$ for all $j\in N$;
        moreover $\tilde\phi^*_{ij} = \tilde\theta^*_{ij}$ for all $(i,j)\in E$.
        \item The network balance condition \eqref{eq:balance.net} is satisfied by $x^*$.
        \item The line limits \eqref{eq:lineConstraint} are satisfied by $x^*$, implying
        $\underline{P}_{ij} \le P_{ij} \le \overline{P}_{ij}$ on every tie line $(i,j)\in E$.
%        \item If $\underline{P}^g_j<P^{g*}_j<\overline{P}^g_j$  and
%        $\underline{P}^l_j<P^{l*}_j<\overline{P}^l_j$ then, at optimality, the marginal 
%        generation regulation cost is equal to the marginal load regulation cost at
%        node $j$, i.e., $\alpha_j P^{g*}_j = -\beta_j P^{l*}_j = -\lambda_j^*$.
        \eee 
\end{theorem}
Theorem \ref{thm:6} shows that the equilibrium point has a simple yet intuitive structure. 
Moreover, Theorem \ref{thm:6} implies that the closed-loop system can autonomously 
eliminate congestions on tie lines. This feature has important implications. It means our distributed frequency control is capable of serving as a corrective re-dispatch without the coordination of dispatch centers if a congestion 
arises.   This can enlarge the feasible region for economic dispatch,
since corrective re-dispatch has been naturally taken into account. 

The proofs of Theorem \ref{thm:5} and \ref{thm:6} are given in Appendix A. 


\subsection{Asymptotic stability}
 In this subsection, we address the asymptotic stability of the closed-loop 
 system \eqref{eq:model.1}\eqref{eq:control.2}, under an additional assumption:
 \bi
 \item[\textbf{A3:}] The initial state of the closed-loop system  \eqref{eq:model.1}\eqref{eq:control.2} is finite, and $p^g_j(0)$, $p^l_j(0)$  satisfy constraint \eqref{eq:OpConstraints.1}. 
 \ei
 
 As in the per-node balance case the closed-loop system \eqref{eq:model.1}\eqref{eq:control.2} 
satisfies constraint \eqref{eq:OpConstraints.1} even during transient.
\begin{lemma}
        \label{lemma:bounded.2}
        Suppose A1 and A3 hold. 
        Then constraint \eqref{eq:OpConstraints.1} is satisfied for all  $t>0$, i.e. 
        $(P^g(t), P^l(t))\in X$ for all $t\geq 0$ where $X$ is defined in \eqref{eq:defX}.    
\end{lemma}
The proof is exactly the same as that for Lemma 3 in \cite{Wang:DistributedFrequency} and omitted.

%As for the network balance case, we set $\mu(t)\equiv \omega(t)$ and use $\tilde\theta, \tilde\phi$
%instead of $\theta,\phi$ since there is a bijection between these variables once we fix 
%$\theta_0(t) \equiv \phi_0(t) \equiv 0$ as the reference.   We denote the vector of state variables 
%by $w:=(\tilde {\theta}, \omega, P^g, P^l, \lambda, \eta^+, \eta^-, \tilde{\phi})$.
% 


Similar to the per-node balance case, we first rewrite the closed-loop system using states 
$\tilde{\theta}, \tilde{\phi}$ instead of $\theta, \phi$ (they are equivalent under assumption A1).
Setting $\mu\equiv \omega$, the closed-loop system \eqref{eq:model.1}\eqref{eq:control.2} 
is equivalent to (in vector form):
        \begin{subequations}
\bq
                \dot {\tilde \theta}(t)&=&   C^T\omega(t) \\
\label{eq:closedloop.2a}
                \nonumber
                \dot \omega (t)&=&M^{-1}\left(P^g(t)-P^l(t)-p-D\omega(t)-CB\tilde \theta(t) \right ) \label{eq:closedloop.2b}\\
                \label{eq:model.2b}\\
                \dot{P}^g(t)&=&(T^g)^{-1}\left ( -P^g(t)+\hat u^g(t) \right)
\label{eq:closedloop.2c}\\ 
                \dot{P}^l(t)&=&(T^l)^{-1}\left ( -P^l(t)+\hat u^l(t) \right )
\label{eq:closedloop.2d}\\
\dot \eta^+(t)&=& \Gamma^{\eta}[\tilde\phi(t)-\overline \theta]^+_{\eta^+}
\label{eq:closedloop.2f}\\
\dot \eta^-(t)&=& \Gamma^{\eta}[\underline \theta-\tilde\phi(t)]^+_{\eta^-}
\label{eq:closedloop.2g}\\
                \dot{\lambda}(t)&=&\Gamma^{\lambda} \left (P^g(t)-P^l(t)-p-CB\tilde \phi(t) \right)
\label{eq:closedloop.2e}\\
\dot {\tilde \phi}(t)&=& \Gamma^{\tilde \phi} \left ( BC^T\lambda(t)
+BC^Tz(t)+\eta^-(t)-\eta^+(t)
\label{eq:closedloop.2h}
\right) 
\eq
where $z(t) := P^g(t) - P^l(t) - p - CB\tilde \phi(t)$ and
\label{eq:closedloop.2}
\end{subequations}
\bqn
	\hat u^g(t) & = & \left[ 
	P^g(t) - \Gamma^g \left( A^g P^g_j(t) + \omega(t)+z(t) + \lambda(t) \right) \right]_{\underline P^g}^{\overline P^g}
	\\
	\hat u^l_j(t) & = & \left[ 
	P^l(t) - \Gamma^l \left( A^l P^l_j(t) - \omega(t)-z(t) - \lambda(t)\right)
	\right]_{\underline P^l}^{\overline P^l}
\eqn
%where, $A^g:=\diag\{\alpha_j, j\in N\}$ and $A^l:=\diag\{\beta_j, j\in N\}$.
Denote $w:=(\tilde {\theta}, \omega, P^g, P^l, \lambda, \eta^+, \eta^-, \tilde{\phi})$.

Note that the right-hand sides of \eqref{eq:closedloop.2f}\eqref{eq:closedloop.2g}
are discontinuous due to projection to the nonnegative quadrant for 
$(\eta^{+}(t), \eta^-(t))$.  The system
\eqref{eq:closedloop.2} is called a projected dynamical system and we adopt the
concept of Caratheodory solutions for such a system where a trajectory $(w(t), t\geq 0)$
is called a Caratheodory solution, or just a solution, to \eqref{eq:closedloop.2} if it is 
absolutely continuous in $t$ and satisfies \eqref{eq:closedloop.2} almost everywhere.
The result in \cite[Theorems 2 and 3]{DupuisNagurney1993} implies that, given any initial
state, there exists a unique solution trajectory to the closed-loop 
system \eqref{eq:closedloop.2} as the unprojected system is Lipschitz and
the nonnegative quadrant is closed and convex.
See \cite[Theorem 3.1]{Monica:Existence} for extension of this result to 
the Hilbert space.

With regard to system \eqref{eq:closedloop.2}, we first define two sets, $\sigma^+$ and $\sigma^-$, as follows \cite{Feijer:Stability}.
\bqn
\sigma^+ &:=& \{(i,j)\in E \, | \, \eta^+_{ij}=0,         \, \tilde{\phi}_{ij}-\overline{\theta}_{ij}<0
\}\\
\sigma^- &:=& \{(i,j)\in E \, | \, \eta^-_{ij}=0,         \, \underline{\theta}_{ij}-\tilde{\phi}_{ij}<0
\}
\eqn
Then \eqref{eq:control.2b} and \eqref{eq:control.2c}
are equivalent to
\begin{subequations}
	\bq
	\dot\eta^+_{ij} &=& \left\{
	\begin{array}{ll}
		\gamma^{\eta}_{ij}(\tilde{\phi}_{ij}-\overline{\theta}_{ij}),
		& \text{if} (i,j) \notin \sigma^+ ;\\
		0,
		& \text{if} (i,j) \in \sigma^+ .        \end{array}
	\right.\\
	\dot\eta^-_{ij} &=& \left\{
	\begin{array}{ll}
		\gamma^{\eta}_{ij}(\underline{\theta}_{ij}-\tilde{\phi}_{ij}),
		& \text{if} (i,j) \notin \sigma^- ;\\
		0,
		& \text{if} (i,j) \in \sigma^- .        \end{array}
	\right.
	\eq
	\label{eq:eta}
\end{subequations}

In a fixed $\sigma^+, \sigma^-$, define $F(w)$. 
% \slow{I think $F$ must be continuously differentiable, so cannot include projection 
% of $\eta^+, \eta^-$.} 
 \bq
  F(w)&=&\left [ 
   \begin{array}{l}
               -B^{1/2}C^T\omega \\
               -M^{-1/2}\left(P^g-P^l-p-D\omega-CB\tilde \theta \right )\\ 
               (T^g)^{-1} \left(A^g P^g+\omega+z+\lambda\right)\\
               (T^l)^{-1} \left( A^l P^l-\omega-z-\lambda\right)\\ 
               -(\Gamma^{\eta})^{1/2}[\tilde\phi-\overline \theta]^+_{\eta^+} \\
               -(\Gamma^{\eta})^{1/2}[\underline \theta-\tilde\phi]^+_{\eta^-} \\
               -(\Gamma^{\lambda})^{1/2} \left (P^g-P^l-p-CB\tilde
\phi \right)\\               
              -(\Gamma^{\tilde{\phi}})^{1/2} \left ( BC^T\lambda+BC^Tz+\eta^- - \eta^+ \right)
    \end{array}  \right ] 
         \label{eq:Fz.2}
 \eq
%        where $\Gamma ^{\tilde \theta}:=\diag(\sqrt {B^{-1}_{ij}}, (i,j)\in E)$; $\Gamma ^{\omega}:=\diag(\sqrt {M^{-1}_j}, j\in N)$; $\Gamma ^{g}:=\diag((T^g_j)^{-1}, j\in N)$; $\Gamma ^{l}:=\diag((T^l_j)^{-1}, j\in N)$; $\Gamma ^{\eta}:=\diag(\sqrt{\gamma^{\eta}_{ij}}, (i,j)\in E)$; $\Gamma ^{\lambda}:=\diag(\sqrt{\gamma^{\lambda}_j}, j\in N)$; $\Gamma ^{\tilde \phi}:=\diag(\sqrt{\gamma^{\tilde
%\phi}_j}, j\in N)$. 
If $\sigma^+$ and $\sigma^-$ do not change, $F(w)$ is continuously differentiable in $w$.
%\slow{Let's use the same notation as much as possible between per-node and
%network balance cases.  We can either change the notations in per-node case
%or in network case, e.g., $\Gamma$'s.}
        
        %Note that $\tilde \theta(t):=C^T\theta(t)$. Then we still have
        %\bqn
        %\dot {\tilde \theta}& =& -\Gamma^{\tilde \theta} \nabla_{\tilde \theta}L
        %\eqn
        
       Similarly, we define 
        $
        S   :=  \mathbb R^{m+n+1}\times X \times \mathbb R^{2m+n+1+m}
        $, where the closed convex set $X$ is defined in \eqref{eq:defX}.
%        \slow{Can/should we re-order \eqref{eq:closedloop.2e} in \eqref{eq:closedloop.2} and
%        everything following that (e.g., definition of $F(w)$), so that
%        $w:=(\tilde {\theta}, \omega, P^g, P^l, \eta^+, \eta^-, \lambda, \tilde{\phi})$?
%        If not, we will re-order $S$ to be 
%        $S   :=  \mathbb R^{m+n+1}\times X \times \mathbb R^{n+1} \times \mathbb R^{2m}_+ \times \mathbb R^{m}$.
%        }
         Then for any $w$ we define the projection of $w-F(w)$ onto  $S$ as  
$$H(w) :=  \text{Proj}_S (w-F(w)) := \ \arg\min_{y\in S} \| y - (w-F(w)) \|_2 $$
Then the closed-loop system \eqref{eq:closedloop.2} is equivalent to 
\bq
\label{eq:model.4}
\dot w(t)&=&\Gamma_2 (H(w(t))-w(t))
\eq
where the positive definite gain matrix is :
\begin{align}
	\Gamma_2\  =\ \diag\ \bigg(B^{-1/2}, M^{-1/2}, (T^g)^{-1},  (T^l)^{-1},\nonumber\\
	 (\Gamma^{\lambda})^{1/2}, (\Gamma^{\eta})^{1/2}, (\Gamma^{\eta})^{1/2}, (\Gamma^{\tilde \phi})^{1/2} \bigg) \nonumber
\end{align}

Note that the projection operation $H$ has an effect only on $(P^g; P^l)$ and Lemma \ref{lemma:bounded.2} indicates that $w(t )\in  S$ for all $t>0$, justifying the equivalence of \eqref{eq:closedloop.2} and (\ref{eq:model.4}).

A point $w^*\in S$ is an equilibrium of the closed-loop system (\ref{eq:model.4}) if and only if it is a fixed point of the projection
$H(w^*)=w^*$. 
Let $E_2:=\{\ w\in S\ |\ H(w(t))-w(t)=0\ \}$ be the set of equilibrium points. Then we have the following theorem.

\begin{theorem}
	\label{thm:stability.22}
	Suppose A1, A2 and A3 hold.
	Starting from any initial point $w(0)$, 
	$w(t)$ remains in a bounded set for all $t$ 
	and $w(t)\rightarrow w^*$ as $t\rightarrow\infty$ for some equilibrium $w^*\in E_2$ that
	is optimal for problem (\ref{eq:opt.2}).
\end{theorem}

For any equilibrium point $w^*$, we define the following  function taking the same form as the per-node case.
\begin{align}
	\label{eq:lyapunov.2}
	\tilde{V}_2(w)&=-(H(w)-w)^TF(w)
	-\frac{1}{2} ||H(w)-w||^2_2 \nonumber\\
	&\quad +\frac{1}{2}k(w-w^*)^T\Gamma_2^{-2}(w-w^*) 
\end{align}
where   $k$ is small enough such that $\Gamma_2-k\Gamma_2^{-1} > 0$ is strictly positive definite. 

For any fixed $\sigma^+$ and $\sigma^-$, $\tilde{V}_2$ is continuously differentiable as $F(w)$ is continuously differentiable in this situation. Similar to $V_1(w)$ used in Part I of the paper, we know $ \tilde{V}_2(w) \geq 0$ on $S$ and $ \tilde{V}_2(w)=0$ holds only at any equilibrium $w^* = H(w^*)$\cite{Fukushima:Equivalent}. Moreover, $\tilde{V}_2$ is nonincreasing for fixed $\sigma^+$ and $ \sigma^-$, as we prove in  Appendix \ref{appd.thm8}. 

It is worth to note that the index sets $\sigma^+$ and $\sigma^-$ may change sometimes, resulting in discontinuity of $\tilde{V}_2(w)$. To circumvent such an issue, we slightly modify the definition of $V_2(w)$ at the discontinuous points as:
\begin{enumerate}
	\item $V_2(w) := \tilde{V}_2(w)$, if $\tilde{V}_2(w)$ is continuous at $w$;
	\item $V_2(w) := \limsup\limits_{v\to w} \tilde{V}_2(v)$, if  $\tilde{V}_2(w)$ is discontinuous at $w$.
\end{enumerate}
Then $V_2(w)$ is upper semi-continuous in $w$, and  $ V_2(w) \geq 0$ on $S$ and $ V_2(w)=0$ holds only at any equilibrium $w^* = H(w^*)$. 
As  $V_2(w)$ is not differentiable for $w$ at discontinuous points, we use the Clarke gradient as the gradient at these points \cite{clarke:optimization}.

%further define the gradient of $V_2(w)$ as follows. 
%\begin{enumerate}
%	\item $\frac{\partial V_2}{\partial w}(w):= \frac{\partial\tilde V_2}{\partial w}(w)$, if $\tilde{V}_2(w)$ is continuous at $w$;
%	\item $\frac{\partial V_2}{\partial w}(w) := \limsup\limits_{v\to w} \left(\frac{ \tilde{V}_2(v) - \tilde{V}_2(w)}{v-w}\right)$, if  $\tilde{V}_2(w)$ is discontinuous at $w$.
%\end{enumerate}
%\wang{I think 2) should be $\frac{\partial V_2}{\partial w}(w) := \lim\limits_{v\to w} \bigg(\frac{ \limsup\limits_{v\to w}\tilde{V}_2(v) - \tilde{V}_2(w)}{v-w}\bigg)$, but it is so strange.}
%
%Here, we just use $\frac{\partial V_2}{\partial w}(w(t_k^-))$ as its Clarke gradient at $w(t_k)$ \cite{clarke:optimization}.

Note that $\tilde{V}_2$ is continuous almost everywhere except the switching points. Hence both $V_2(w)$ is \emph{nonpathological} \cite{Bacciotti:Nonpathological,bacciotti:stability}. With these definitions and notations above, we can prove  Theorem \ref{thm:stability.22}. The detail of proof is provided in  Appendix \ref{appd.thm8}.



       

