
\section{Network model}

%\subsection{Network model}

We summarize the notation used in Part I \cite{Wang:DistributedFrequency}. The power network is model by a directed graph ${G}:=(N, E)$ 
where  $N=\{0,1,2,...n\}$ is the set of nodes (control areas) and
$E\subseteq N\times N$ is the set of edges (tie lines).  If a pair of
nodes $i$ and $j$ are connected by a tie line
directly, we denote the tie line by $(i,j)\in E$.
Let $m:= |E|$ denote the number of tie lines.
Use $(i,j)\in E$
or $i\rightarrow j$ interchangeably to denote a directed edge from $i$ to $j$.
Assume the graph is connected
and node $0$ is a reference node.
For each node $j\in N$, $\theta_j(t)$ denotes the rotor angle at node $j$ at time
$t$ and $\omega_j(t)$ is the  frequency.
$P_j^g(t)$ denotes the (aggregate) generation at node
$j$ at time $t$ and $u^g_j(t)$ is its generation control command.
$P^l_j(t)$ denotes the (aggregate) controllable load and $u^l_j(t)$ is its
load control command.  $p_j$ is the (aggregate) uncontrollable load. 

%We adopt a second-order linearized model to describe the frequency dynamics
%of each node, and two first-order inertia equations to describe the dynamics of
%power generation regulation and load regulation at each node.  We assume
%the tie lines are lossless and adopt the DC power flow model.
The power system dynamics for each node $j\in N$ is
\begin{subequations}
	\begin{align}
	\dot \theta_j & =  \omega_j(t)
	\label{eq:model.1a}
	\\
	M_j \dot \omega_j & =   P^g_j(t) - P^{l}_j(t) - p_j -D_j \omega_j(t)
	\nonumber
	\\
	&  + \sum_{i: i\rightarrow j} \! B_{ij} (\theta_i(t) - \theta_j(t))
	-  \sum_{k: j\rightarrow k} \! B_{jk}(\theta_j(t) - \theta_k(t))
	\label{eq:model.1b}
	\\
	T^g_j \dot P^g_j & =  - P^g_j(t) + u^g_j(t) - {\omega_j(t)}/{R_j}
	\label{eq:model.1c}
	\\
	T^l_j \dot P^{l}_j & =  - P^{l}_j(t) + u^l_j(t)
	\label{eq:model.1d}
	\end{align}
	where $D_j>0$ are damping constants, $R_j>0$ are droop parameters,
	and $B_{jk}>0$ are line parameters that depend on the reactance of the line $(j,k)$.
	Let  $x := (\theta, \omega, P^g, P^l)$ denote the state of the network
	and $u := (u^g, u^l)$ denote the control.
%	\footnote{Given
%		a collection of $x_i$ for $i$ in a certain set $A$, $x$ denotes the column vector
%		$x := (x_i, i\in A)$ of a proper dimension with $x_i$ as its components.}
	\label{eq:model.1}
\end{subequations}

%Our goal is to design feedback control laws for the generation command
%$u^g(x(t))$ and load control $u^l(x(t))$.
The capacity constraints are:
\begin{subequations}
        \bq
                \underline{P}^g_j & \leq \ P^g_j(t) \ \leq \overline{P}^g_j, \quad j\in N
        \label{eq:OpConstraints.1a}\\
                \underline{P}^l_j & \leq \ P^l_j(t) \ \leq \overline{P}^l_j, \quad j\in N
        \label{eq:OpConstraints.1b}
        \eq        
        Here \eqref{eq:OpConstraints.1a}  and \eqref{eq:OpConstraints.1b} are  \emph{hard limits} on the regulation capacities of generation and controllable load at each node, which should not be violated at any time even during transient.
     \label{eq:OpConstraints.1}
\end{subequations}
%Hence these constraints are satisfied not only at equilibrium, but also during transient.

The system operates in a steady state initially, i.e., the generation
and the load are balanced and the frequency is at its nominal value. All variables
represent {deviations} from their nominal or scheduled values so that,
e.g., $\omega_j(t)=0$ means the frequency is at its nominal value. 

In this paper, all nodes cooperate to rebalance power over the entire network after a disturbance.  The power flows $P_{ij}$ on the tie lines may deviate from their scheduled values and we require that they satisfy line limits,
i.e., 
\bqn
\underline{P}_{ij} \ \le  \ P_{ij} \ \le \ \overline{P}_{ij} \qquad\quad \forall (i,j)\in E
\eqn
for some upper and lower bounds $\underline{P}_{ij}, \overline{P}_{ij}$. 

In DC approximation the power flow on line $(i,j)$ is given by 
$P_{ij} =B_{ij}(\theta_i - \theta_j)$.  Hence line flow constraints in the per-node
balance case is $\theta_i = \theta_j$ for all $(i,j)\in E$ and in the network balance
case is: 
\bq
\label{eq:lineConstraint}
\underline{\theta}_{ij} \ \le \ \theta_{i} - \theta_j \ \le \ \overline{\theta}_{ij} \qquad\quad \forall (i,j)\in E
\eq
where $\underline{\theta}_{ij}=\underline{P}_{ij}/B_{ij}$, $\overline{\theta}_{ij}=\overline{P}_{ij}/B_{ij}$ 
are lower and upper bounds on angle differences. 

As the generation $P^g_j$ and load $P^l_j$ in each area can increase or decrease, 
and a line flow $P_{ij}$ can in either direction, we make the following assumption.
\bi
\item[\textbf{A1:}] 
\bee
\item $\underline{P}^g_j < 0 < \overline{P}^g_j$ and $\underline{P}^l_j <0 < \overline{P}^l_j$
for $\forall j\in N$.
\item $\underline{\theta}_{ij} \leq 0 \leq \overline{\theta}_{ij}$ for $(i,j)\in E$.
\item $\theta_0(t):=0$ and $\phi_0(t):=0$ for all $t\geq 0$.
\eee
\ei  
Here $\phi$ is a vector variable that represents virtual phase angles in the
network balance case in Section \ref{sec:npb}.
The assumption $\theta_0 \equiv \phi_0 \equiv 0$ amounts to using $(\theta_0(t), \phi_0(t))$ as reference
angles.  It is made merely for notational convenience: as we will see, the equilibrium point will be
unique with this assumption (or unique up to reference angles without this assumption).


