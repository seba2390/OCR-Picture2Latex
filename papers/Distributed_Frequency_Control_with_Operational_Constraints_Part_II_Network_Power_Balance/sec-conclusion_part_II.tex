
\section{Concluding Remarks}
In this paper, we have devised a distributed optimal frequency control in the network balance case, which can autonomously restore the nominal frequencies after unknown load disturbances while minimizing the regulation costs. The capacity constraints on the generations and  controllable loads can also be satisfied even during transient. In addition, congestions can be eliminated automatically, implying tie-line powers can be remained within given ranges. Only neighborhood communication is required in this case. Like the per-node case, here the closed-loop system again carries out a primal-dual algorithm with saturation to solve the associated optimal problem. To cope with the discontinuity introduced due to enforcing different types of constraints, we have constructed a nonpathological Lyapunov function to prove the asymptotically stability of the closed-loop systems. Simulations on a modified Kundur's power system validate the effectiveness of our controller. 

This approach is also applicable to other problem involving frequency regulation, e.g. standalone microgrid or demand side management. We highlight two crucial implications of our work: First, our distributed frequency control is capable of serving as an automatically corrective re-dispatch without the coordination of dispatch center when certain congestion happens; Second, the feasible region of economic decisions can be enlarged  benefiting from the corrective re-dispatch. 
In this sense, our work may provide a systematic way to bridge the gap between  the (secondary) frequency control in a fast timescale  and the economic dispatch in a slow timescale, hence breaking the traditional hierarchy of the power system frequency control and economic dispatch.
    
    
 
    
    
    