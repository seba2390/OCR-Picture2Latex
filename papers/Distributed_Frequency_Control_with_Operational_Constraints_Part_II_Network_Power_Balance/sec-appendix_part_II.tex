\newpage
\appendices

\makeatletter
\@addtoreset{equation}{section}
\@addtoreset{theorem}{section}
\makeatother

\section{Proofs of Theorem \ref{thm:5} and Theorem \ref{thm:6}}

\renewcommand{\theequation}{A.\arabic{equation}}
\renewcommand{\thetheorem}{A.\arabic{theorem}}

%The rest of this subsection is devoted to the proof of Theorems \ref{thm:5} 
%and Theorem \ref{thm:6}.
We start with a lemma.
\begin{lemma}
	\label{lemma:6}
	Suppose $(x^*, u^*)$ is optimal for \eqref{eq:opt.2}.  Then
	\bee
	\item  $\omega^* = 0$, i.e., the nominal frequency is restored; 
	\item  the network balance condition \eqref{eq:balance.net} is satisfied by $x^*$;
	\item  $\phi^*-\theta^* = (\phi^*_0-\theta^*_0)\,\textbf{1}$;
	\item  $\underline{\theta}_{ij} \le \theta^*_{ij} \le \overline{\theta}_{ij}$, 
		i.e., the line limits \eqref{eq:lineConstraint} are satisfied.
	\eee
\end{lemma}
\begin{proof}
%	The proof of 1) follows a similar argument as that of Lemma \ref{lemma:1},
%	making use of \eqref{eq:opt.2a}\eqref{eq:opt.2b} instead of \eqref{eq:balance.node}.
	Suppose $(x^*, u^*)$ is optimal but $\omega^*\neq 0$.
	Then \eqref{eq:opt.2b} implies
	\bq
	P^{g*} - P^{l*} - p & = &  CBC^T\phi^*
	\label{eq:lemma.6.1}
	\eq
	Consider $\hat x := (\hat\theta, \phi^*, \hat\omega, P^{g*}, P^{l*})$ with
	$\hat\theta:=\phi^*$ and $\hat\omega:=0$.  Then $\hat x$ satisfies
	\eqref{eq:opt.2a}\eqref{eq:opt.2b} due to \eqref{eq:lemma.6.1}.  
	Hence $(\hat x, u^*)$ is feasible for \eqref{eq:opt.2} but has a strictly lower
	cost, contradicting the optimality of $(x^*, u^*)$.  Hence $\omega^*=0$.
	
	Multiplier both sides of \eqref{eq:lemma.6.1} by $\textbf{1}^T$ yields
	the network balance condition \eqref{eq:balance.net}, proving 2).
		
	To prove 3), setting $\omega^*=0$ in \eqref{eq:opt.2a} and combining
	with \eqref{eq:opt.2b} yield
	\bqn
	CBC^T\theta^* &=& P^{g*}-P^{l*}-p \ \ = \ \ CBC^T\phi^*
	\eqn
	Since $CBC^T$ is 
	an $(n+1)\times (n+1)$ matrix with rank $n$, its null space has dimension 1
	and is spanned by $\textbf{1}$ because $C^T\textbf{1} = 0$.
	Hence $CBC^T(\phi^*-\theta^*)=0$ implies that 
	$\phi^*-\theta^* = (\phi^*_0-\theta^*_0)\,\textbf{1}$. 
	%Let 
	%\bqn
	%\tilde \phi :=&\{\phi_{ij},\ \forall %(i,j)\in
	%E\} 
	%\eqn 
	%and
	%\bqn
	%\tilde \theta :=& \{\theta_{ij},\ \forall %(i,j)\in
	%E\}
	%\eqn
	%Then we have $\tilde \phi=C^T\phi$ and $\tilde \theta=C^T\theta$.
	To prove 4), note that $\tilde \phi=C^T\phi$ and $\tilde \theta=C^T\theta$
	and hence 
	\bqn
	\tilde{\phi}^*-\tilde{\theta}^* & = & C^T({\phi^*}-{\theta^*})
			\ \ = \ \ (\phi^*_0-\theta^*_0)C^T\textbf{1}  \ \ = \ \ 0
	\eqn
	i.e. $\tilde{\phi}^*=\tilde{\theta}^*$.   We conclude from \eqref{eq:opt.2c} that
	$\underline{\theta}_{ij} \le \theta^*_{ij} \le \overline{\theta}_{ij}$. 
	This completes 	the proof.
\end{proof}

We have the following result.
\begin{lemma}
	\label{lemma:7}
	Suppose $(x^*, \rho^*)$ is primal-dual optimal.  Then
	\bqn
	u^{g*}_j \ = \
	P^{g*}_j & = & \left[ 
	P^{g*}_j - \gamma^g_j \left( \alpha_j P^{g*}_j + \omega^*_j + z^*_j+\lambda_j^* \right)
	\right]_{\underline P^g_j}^{\overline P^g_j}
	\nonumber \\
	u^{l*}_j \ = \
	P^{l*}_j & = & \left[ 
	P^{l*}_j - \gamma^l_j \left( \beta_j P^{l*}_j - \omega^*_j -z^*_j- \lambda_j^* \right)
	\right]_{\underline P^l_j}^{\overline P^l_j}
	\eqn
	for any $\gamma^g_j>0$ and $\gamma^l_j>0$.
\end{lemma}
\begin{comment}
\begin{proof}
	Since \eqref{eq:opt.2} is (strictly) convex
	with linear constraints, strong duality holds.
	Hence $(x^*,\rho^*) $ is primal-dual optimal
	if and only if it satisfies the KKT condition:\
	$(x^*, u(x^*,\rho^*))$ is primal feasible and
	\bq
	x^*&=&\arg \min_x \{L_2(x;\rho^*)|(x,u(x,\rho^*))
	\ \text{satisfies} \eqref{eq:OpConstraints.1},
	\eqref{eq:opt.2d},\eqref{eq:opt.2e}\}\nonumber\\   \label{eq:lamma.7}
	\eq
	
	The proof is the same as the proof of Lemma
	\ref{lemma:2}, thus is omitted here.
\end{proof}
\end{comment}

Lemma \ref{lemma:7} shows that the saturation of control input does not impact the optimal solution of optimization problem \eqref{eq:opt.2}. 

%\slow{Maybe omit Lemma \ref{lemma:7} since it is not directly
%referenced.  It's used in the detailed proof of the last step of Theorem \ref{thm:5}.}\wang{Agree. Similar argument has been introduced in the Appendix A.}
%\slow{Revisit.}
With Lemma \ref{lemma:bounded.2}, Lemma \ref{lemma:6}
and Lemma \ref{lemma:7}, we now can prove Theorem \ref{thm:5} and Theorem \ref{thm:6}.
\begin{proof}[Proof of Theorem \ref{thm:5}]
	\noindent
	$\Rightarrow$: 
	Suppose $(x^*, \rho^*)$ is primal-dual optimal. 
	Then $x^*$ satisfies the operational constraints \eqref{eq:OpConstraints.1}.
	Moreover the right-hand side of \eqref{eq:model.1} vanishes because:
	\bi
	\item $\dot\theta = 0$ since $\omega^*=0$ from Lemma \ref{lemma:6}.
	\item $\dot\omega = 0$ since constraint \eqref{eq:opt.2a} holds for $x^*$.
	% and $x^*$ achieves network power balance \eqref{eq:balance.net}.
	\item $\dot P^g =\dot P^l=0$ since $\omega^*= 0$ and $x^*$ 
	satisfies \eqref{eq:opt.2d} and \eqref{eq:opt.2e}.
	\item $\dot \lambda = 0$ since \eqref{eq:opt.2b} holds for $x^*$.
	\item $\dot \eta^+ = \dot \eta^- = 0$ since \eqref{eq:opt.2c} holds for $x^*$.
	\item From \eqref{eq:primal.2d} we have 
		\bqn
		\dot {\tilde{\phi}} & = & -\Gamma^{\tilde{\phi}} \ \nabla_{\tilde{\phi}} L_2 (x^*, \rho^*)
		\eqn
		Since $(x^*, \rho^*)$ is a saddle point,
		we must have $\nabla_{\tilde{\phi}} L_2 (x^*, \rho^*)=0$, implying
		$\dot {\tilde{\phi}}=0$.
	\ei 
	Hence $(x^*, \rho^*)$ is an equilibrium of the closed-loop system 
	\eqref{eq:model.1}\eqref{eq:control.2} that satisfies 
	the operational constraints \eqref{eq:OpConstraints.1}.   Moreover 
	$\mu^* = \omega^* = 0$ since
	$
	\frac{\partial L_2}{\partial \omega_j}(x^*, \rho^*)  = 
	D_j ( \omega^*_j - \mu^*_j) = 0
	$
	and $D_j>0$ for all $j\in N$.
	
	
	\vspace{0.07in}
	\noindent
	$\Leftarrow$: Suppose now $(x^*, \rho^*)$ is an equilibrium of the closed-loop 
	system \eqref{eq:model.1}\eqref{eq:control.2} that satisfies \eqref{eq:OpConstraints.1}
	and $\mu^*=0$.
	Since \eqref{eq:opt.2} is convex with linear constraints, 
	$(x^*, \rho^*)$ is a primal-dual optimal if and only if 
	$(x^*, u(x^*, \rho^*))$ is primal feasible and satisfies 
	\bq
	x^*& \!\!\!\!\!\!\!\! = \!\!\!\!\!\!\!\! &\arg \min_x \{L_2(x;\rho^*)|(x,u(x,\rho^*))
	\ \text{satisfies} \eqref{eq:OpConstraints.1},
	\eqref{eq:opt.2d},\eqref{eq:opt.2e}\}\nonumber\\  
	\label{eq:lamma.7}
	\eq
	This is because $\nabla_{\rho_1} L_2(x^*, \rho^*)=0$ since
	$\dot\mu = \dot\lambda = 0$, $\eta^{+*}\geq 0$, $\eta^{-*}\geq 0$,
	 and the complementary slackness condition
	$\eta^{+*}_{ij}(\tilde\phi_{ij}^*-\overline{\theta}_{ij})=0$,
	$\eta^{-*}_{ij}(\underline{\theta}_{ij}-\tilde\phi_{ij}^*)=0$ is satisfied
	since $\dot{\eta}^+ = \dot{\eta}^- =0$.
	
	To show that $(x^*, u(x^*, \rho^*))$ is primal feasible, note that since 
	$(x^*, u(x^*, \rho^*))$ is an equilibrium of \eqref{eq:model.1}, it satisfies 
	$\omega^*=0$ and hence \eqref{eq:opt.2d}\eqref{eq:opt.2e}, in
	addition to \eqref{eq:OpConstraints.1}.
	% We have proved in Theorem \ref{thm:1} that $w^*=0$ implies $\theta^*=0$
	% and hence \eqref{eq:model.1b} implies $P^{g*}_j = p_j - P^{l*}_j$ for all $j\in N$,
	% which is \eqref{eq:balance.node}.   
	Moreover $\dot\omega=0$ means $(x^*, u(x^*, \rho^*))$ satisfies
	\eqref{eq:opt.2a}, $\dot \lambda = 0$ implies \eqref{eq:opt.2b}, 
	$\dot \eta^+ = \dot \eta^- = 0$ implies \eqref{eq:opt.2c}.
	
	To show that $(x^*, \rho^*)$ satisfies 
	\eqref{eq:lamma.7}, note that 
	\eqref{eq:opt.2d}\eqref{eq:opt.2e} and \eqref{eq:control.2e}\eqref{eq:control.2f}
	imply that
	\bqn
	P^{g*}_j & = & \left[ 
	P^{g*}_j - \gamma^g_j \left( \alpha_j P^{g*}_j + \omega^*_j +z^{*}_{_{j}} +\lambda_j^* \right)
	\right]_{\underline P^g_j}^{\overline P^g_j}
	\nonumber \\
	P^{l*}_j & = & \left[ 
	P^{l*}_j - \gamma^l_j \left( \beta_j P^{l*}_j - \omega^*_j -z^{*}_{_{j}} - \lambda_j^* \right)
	\right]_{\underline P^l_j}^{\overline P^l_j}
	\eqn
	The rest of the proof follows the same line of argument as that in Theorem 1 in \cite{Wang:DistributedFrequency}.
	This proves that $(x^*,\rho^*)$ is primal-dual optimal and completes the 
	proof of Theorem \ref{thm:5}.
\end{proof}

Next we prove Theorem \ref{thm:6}.
\begin{proof}[Proof of Theorem \ref{thm:6}]
	Let $(x^*, \rho^*) = (\tilde{\theta}^*$$, \tilde \phi^*,$ $
	\omega^*,$ $ P^{g*},$ $ P^{l*},$ $\lambda^*,$$ \eta^{+*},
	\eta^{-*}, \mu^*)$
	be primal-dual optimal. 
%	
%	If $\underline \theta _{ij}<\phi _{ij}<\overline \theta_{ij}$, 
%	Why is this true?  If it does not hold, are $\eta^+, \eta^-$ unique?
%	 $\eta_{ij}^+= \eta_{ij}^-=0 $. 
%	
%	Noticing that $z=0$
%	at optimality, 1) is a direct deduction of $\dot\phi_{ij}=0$.
%	
%	The uniqueness of $x^*, \mu^*$ in 2) is the same as Theorem \ref{thm:1}. 
%	
%	{For $\forall j\in N$, if $P_j^{g*}$ or $P_j^{gl*}$ does not reach the limits, then $\lambda_j^*$ is unique. Why ?}
%	
%	{Otherwise, there must exist $(i,j)\in E$ with $\underline \theta _{ij}<\phi _{ij}^*<\overline \theta_{ij}$. Why?} 
%	
%	Then, $\lambda_j^*=\lambda_i^*$ from 1). If $\sum\nolimits_j(\underline{P}^g_j - \overline{P}^l_j )<\sum\nolimits_j p_j$,  $\sum\nolimits_j(\overline{P}^g_j - \underline{P}^l_j ) > \sum\nolimits_j p_j $, there must exist at least one $j\in N$, where $\underline{P}^g_j < P_j^{g*} < \overline{P}^g_j$ or $\underline{P}^l_j < P_j^{l*} < \overline{P}^l_j$ holds. This implies that $\lambda_j^*$ is unique. Therefore, $\lambda_j^*$ is unique for $\forall j\in N$.		
	For the uniqueness of $x^*$, $(\omega^*, P^{g*}, P^{l*})$ are unique because
	the objective function in \eqref{eq:opt.2} is strictly convex in $(\omega, P^{g}, P^{l})$.
	Hence $\mu^*= \omega^*$ is unique as well.
	Assumption A1 that $\phi^*_0:=0$ and \eqref{eq:lemma.6.1} imply that $\phi^*$ is 
	uniquely determined by the equilibrium $(\omega^*, P^{g*}, P^{l*})$.  
	Since $\theta^* - \phi^* = (\theta^*_0-\phi^*_0)\textbf{1}$, assumption A1 that 
	$\theta^*_0:=0$ then implies that $\theta^*$ is unique.
	This proves the uniqueness of $(x^*, \mu^*)$.
	
	The remaining three parts of the theorem follow from Lemma \ref{lemma:6}.
\end{proof}



\section{Proof Theorem \ref{thm:stability.22}}
\label{appd.thm8}
\renewcommand{\theequation}{B.\arabic{equation}}
\renewcommand{\thetheorem}{B.\arabic{theorem}}

\newcounter{TempEqCnt2}
\setcounter{TempEqCnt2}{\value{equation}}
\setcounter{equation}{1}
\newcounter{mytempeqncnt2}
    \begin{figure*}[!t]
    \normalsize
        		%\setcounter{mytempeqncnt}{\value{equation}}
        		%\setcounter{equation}{5}
    \begin{equation}
        \label{eq:Q}
        Q=\\
        	\left[
        	\begin{array}{llllllll}
        		0  & -BC^T   & 0           & 0           & 0        & 0     & 0 & 0 \\
        		CB &   D     & -I_{|N|}    & I_{|N|}     & 0        & 0     & 0 & 0 \\
        		0  &I_{|N|}  & A^g+I_{|N|} & -I_{|N|}    & 0        & 0     & 0 & -CB \\
        		0  &-I_{|N|} & -I_{|N|}    & A^l+I_{|N|} & 0        & 0     & 0 & CB \\
        		0  &0        & -I_{|N|}    & I_{|N|}     & 0        & 0     & 0 & CB \\
        		0  &0        & 0           &   0         & 0        & 0     & 0 & -I_{|E|-|\sigma^+|} \\
        		0  &0        & 0           &   0         & 0        & 0     & 0 & I_{|E|-|\sigma^-|} \\
        		0  &0        & -BC^T        &  BC^T      & -BC^T    &I_{|E|-|\sigma^+|}     &-I_{|E|-|\sigma^-|} & BC^TCB \\
        	\end{array} 
        		\right]
      \end{equation}
        		%\setcounter{equation}{\value{mytempeqncnt}}
        		\hrulefill
        		\vspace*{4pt}
      \end{figure*}  
\setcounter{equation}{\value{TempEqCnt2}} 

        
We start with a lemma.
        
        \begin{lemma}
        	\label{lemma:decreasing.2}
        	Suppose A1, A4 and A5 hold. Then 
        	\begin{enumerate}
        		\item $\dot V_2(w(t))\leq 0 $ in a fixed $\sigma^+$ and $\sigma^-$.
        		\item The trajectory $w(t)$ is bounded, i.e., there exists $\overline w$ such
        		that $\|w(t)\|\leq \overline w$ for all $t\geq 0$.
        	\end{enumerate}  	
%        	$V_n(v)$ is always decreasing along system \eqref{eq:model.4}.
        \end{lemma}
        
\begin{proof}[Proof of Lemma \ref{lemma:decreasing.2} ] 
	Given fixed $\sigma^+$, $\sigma^-$, for all $(i,j)\notin \sigma^+$, $(i,j)\notin \sigma^-$, we have
	\begin{subequations}
		\begin{align}
		\dot V_2(w)&\le k(H(w)-w^*)^T\cdot \Gamma_2^{-1} (H(w)-(w-F(w)))	\label{eq:ProofThm.2a}	\\
		& -(H(w)-w)^T\Gamma_2\cdot Q\cdot\Gamma_2(H(w)-w)\label{eq:ProofThm.2b}\\
		& - \left( H(w) - (w - F(w)) \right)^T(\Gamma_2-k\Gamma_2^{-1}) (w-H(w))  \label{eq:ProofThm.2c}\\
		& -k(w-w^*)^T\cdot \Gamma_2^{-1} F(w)	\label{eq:ProofThm.2d}
		\end{align}
		where $Q$  is a semi-definite positive matrix and $\Gamma_2 Q=\nabla_wF(w)$, which given in \eqref{eq:Q}. Here, the subscript
		of $I$ means its dimension, and $|A|$ means the cardinality of set $A$. 	           
	\end{subequations}
	
%	\fliu{In the previous definition, we consider all $(i,j)$ in $V_2$. But here we exclude those $(i,j)\in\sigma^+$ and $(i,j)\in\sigma^-$. Should we add some description to make them  consistent?   }
	
	Given fixed $\sigma^+$ and $\sigma^-$, $F(w)$ is continuous differentiable. In this case, (\ref{eq:ProofThm.2a}) and (\ref{eq:ProofThm.2c}) are nonpositive due to as discussed in the proof of Theorem 4 in \cite{Wang:DistributedFrequency}. (\ref{eq:ProofThm.2b}) is also nonpositive as $Q$ is semi-definite positive (see Eq. \eqref{eq:Q}).     
	
	Next, we  prove that (\ref{eq:ProofThm.2d}) is nonpositive. 
	Similar to the per-node case, substituting $\mu(t)\equiv \omega(t)$ into the Lagrangian $L_2(x, \rho)$ in \eqref{eq:defL.21} we obtain a function $\hat{L}_2(w)$ defined as follows.  
%	\slow{What is the difference between $\hat L_2$ and $L_2$ below?}\wang{There is no $\mu$ in $\hat{L}_2$, similar to $\hat{L}_1$.}
	\begin{align}
	%	    \hat{L}(w)&\quad =\hat L_2(\tilde {\theta}, \omega, P^g, P^l, \lambda, \eta^+, \eta^-, \tilde{\phi})  \nonumber\\   
	\hat{L}_2(w)& \quad := 
	L_2 \left(\tilde {\theta}, \omega, P^g, P^l, \lambda, \mu, \eta^+, \eta^-, \tilde{\phi} \right)_{\mu=\omega} 
	\nonumber
	\\ &\quad =  
	\frac{1}{2} \left( (P^g)^T A^g P^g + (P^l)^T A^l P^l - \omega^T D \omega+z^Tz \right )\nonumber
	\\
	& \quad + \ \lambda^T \!\! \left( P^g - P^l - p -CB\tilde{\phi}\right) + \ \omega^T \!\! \left( P^g - P^l - p - C B\tilde \theta \right)\nonumber\\
	&\quad +(\eta^-)^T\left(\underline{\theta}-\tilde\phi\right) + (\eta^+)^T \left(\tilde \phi-\overline{\theta}\right)\nonumber
	\end{align}
	In addition, denote $w_1=(\tilde\theta, P^g, P^l, \tilde{\phi})$, $w_2=(\lambda, \omega, \eta^+, \eta^-)$. Then $\hat L_2$ is convex in $w_1$ and concave in $w_2$.  
	
	In $F(w)$,  $[\tilde\phi-\overline \theta]^+_{\eta^+}$ and $[\underline \theta-\tilde\phi]^+_{\eta^-} $ have unknown dimensions (up to $\sigma^+$ and $\sigma^-$). Fortunately, we have 
	\begin{align}
	(\eta^+-\eta^{+*})^T[\tilde\phi-\overline \theta]^+_{\eta^+}&\le(\eta^+-\eta^{+*})^T(\tilde\phi-\overline \theta)\nonumber\\
	&=(\eta^+-\eta^{+*})^T\nabla_{\eta^+} \hat L_2\nonumber
	\end{align}
	where the inequality holds since $\eta_{ij}^{+}=0\le \eta_{ij}^{+*}$ and $\tilde\phi_{ij}-\overline \theta_{ij} < 0$ for $(i,j)\in \sigma^+$, i.e., $(\eta_{ij}^+-\eta_{ij}^{+*}) \cdot (\tilde{\phi}_{ij}-\overline{\theta}_{ij})\ge0$. Similarly, 
	\begin{align}
	(\eta^--\eta^{-*})^T[\underline \theta-\tilde\phi]^+_{\eta^-}&\le(\eta^--\eta^{-*})^T(\underline \theta-\tilde\phi) \nonumber\\
	& = (\eta^--\eta^{-*})^T\nabla_{\eta^-} \hat L_2.\nonumber
	\end{align}
	
	Consequently, it can be verified that 
	\bqn
	(w-w^*)^T\Gamma_2^{-1} F(w) & \!\!\!\! \le \!\!\!\! & (w-w^*)^T \nabla_{w}\hat{L}_2(w) \nonumber\\
	& \!\!\!\!= \!\!\!\! &\ 
	(w-w^*)^T\begin{bmatrix} \ \ \nabla_{w_1}\hat L_2 \\ - \nabla_{w_2} \hat L_2 \end{bmatrix}\!
	(w_1, w_2)
	\eqn
	where, $\nabla_{w_1} \hat{L}_2 := \begin{bmatrix}
	\nabla_{\tilde\theta} \hat L_2 \\
	\nabla_{P^g} \hat L_2 \\
	\nabla_{P^l} \hat L_2 \\
	\nabla_{\tilde{\phi}} \hat L_2 
	\end{bmatrix}$   and $\nabla_{w_2} \hat{L}_2 := \begin{bmatrix}
	\nabla_{\lambda} \hat L_2 \\
	\nabla_{\omega} \hat L_2\\
	\nabla_{\eta^+} \hat L_2\\
	\nabla_{\eta^-} \hat L_2
	\end{bmatrix}$.  
	
	\setcounter{equation}{2}
	Then we have
	\begin{align}
	&-k(w-w^*)^T\Gamma_2^{-1} F(w)\le \ -k(w-w^*)^T\cdot \Gamma_2^{-1} F(w) \nonumber \\
	= &\  -k(w_1-w_1^*)^T \nabla_{w_1}\hat L_2(w_1,w_2) + k(w_2-w_2^*)^T \nabla_{w_2}\hat L_2(w_1,w_2) \nonumber \\
	\le & \ k \left(\hat L_2(w_1^*,w_2) - \hat L_2(w_1,w_2) +\hat L_2(w_1,w_2) - \hat L_2(w_1,w^*_2)\right) \nonumber \\
	= &  \ k\left( \underbrace{\hat L_2(w_1^*,w_2) - \hat L_2(w^*_1,w^*_2)}_{\le 0} + \underbrace{\hat L_2(w^*_1,w^*_2) - \hat L_2(w_1,w^*_2)}_{\le 0}\right)\nonumber \\
	\le & \ 0
	\label{saddle point}	
	\end{align}
	where the first inequality holds because $\hat L_2$ is convex in $w_1$ and concave in $w_2$ and the second inequality follows
	because $(w^*_1, w^*_2)$ is a saddle point.    
	Therefore (\ref{eq:ProofThm.2d}) is nonpositive, proving the first assertion. 
	
	To prove the second assertion, we further investigate the situation that $\sigma^+$ or  $\sigma^-$ changes. 
	We only consider the set $\sigma^+$ since it is the same to  $\sigma^-$. We have the following observations:
	\begin{itemize}
		\item The set $\sigma^+$ is reduced, which only happens when $\tilde{\phi}_{ij}-\overline{\theta}_{ij}$ goes through zero, from negative to positive. Hence an extra term will be added to $V_2$. As this term is initially zero, there is no discontinuity of $V_2$ in this case.
		\item The set $\sigma^+$ is enlarged when $\eta_{ij}^+$ goes to zero from positive while $\tilde{\phi}_{ij}<\overline{\theta}_{ij}$. Here $V_2$ will lose a positive term $(\gamma_{ij}^\eta)^2(\tilde{\phi}_{ij}-\overline \theta_{ij})^2/2$, causing discontinuity.
	\end{itemize}
	
	 
	In the context, we conclude that $V_2$ is always nonincreasing along the trajectory even when $\sigma^+$ or $\sigma^-$ changes and discontinuity occurs. 

	To prove that the trajectory $w(t)$ is bounded
	note that \cite[Theorem 3.1]{Fukushima:Equivalent} proves that 
	$\hat V_2(w) := - \left( H(w)-w \right)^T F(w)  \, - \, \frac{1}{2} ||H(w)-w||^2_2$ satisfies
	$\hat V_2(w) \geq 0$ over $S$. 	
	Therefore, we have 
	\bqn
	\frac{1}{2}k(w(t)-w^*)^T\Gamma_2^{-2}(w(t)-w^*) &\!\!\! \le \!\!\! & V_2(w(t)) \ \le \ V_2(w(0))
	\eqn
	indicating the trajectory $w(t)$ is bounded.  
\end{proof}

\begin{lemma}
        		\label{lemma:decreasing.3}
        		Suppose A1, A4 and A5 hold. Then 
        		\begin{enumerate}
        			\item The trajectory $w(t)$  converges to the largest weakly invariant subset $W_2^*$ contained in $W_2:=\{w\in S | \dot V_2(w)=0 \}$.
        			\item Every point $w^*\in W_2^*$ is an equilibrium point of \eqref{eq:model.4}.
        		\end{enumerate}  	
        		%        	$V_n(v)$ is always decreasing along system \eqref{eq:model.4}.
        	\end{lemma}
        
        \begin{proof} [Proof of Lemma \ref{lemma:decreasing.3} ] 
        	Given an initial point $w(0)$ there is a compact set $\Omega_0 := \Omega(w(0)) \subset S$ such that $w(t)\in\Omega_0$ 
        	for $t\geq 0$ and $\dot V_2(w) \leq 0$ in $\Omega_0$. 
        	
        	Invoking the proof of Lemma \ref{lemma:decreasing.2}, $V_2$ is radially unbounded and positively definite except at equilibrium. As $V_2$ and $\dot{V}_2$ are nonpathological,   we conclude that any trajectory $w(t)$ starting from $\Omega_0$ converges to the largest weakly invariant subset $W_2^*$ contained in $W_2=\{\ w\in \Omega_0\ |\ \dot V_2(w) =0\ \}$ \cite[Proposition 3]{Bacciotti:Nonpathological}, proving  the first assertion.
        	
        For the second assertion, We fix $w(0) \in W_2^*$ and then prove that $w(0)$ must be an equilibrium point. 
        		
       	From (\ref{eq:ProofThm.2b}), direct computing yields 
        	\begin{align}
        	\dot V_{2}(w(t)) &\le  -\dot{\omega}^TD \dot {\omega}-(\dot {P}^g)^T A^g \dot {P}^g \nonumber\\
        	&-(\dot {P}^l)^T A^l \dot {P}^l -(CB\dot{\tilde \phi}-\dot {P}^g+\dot {P}^l)^T(CB\dot{\tilde \phi}-\dot {P}^g+\dot {P}^l)\nonumber\\
        	&\le 0
        	\label{dot V2}
        	\end{align}
       Since $A^g$, $A^l$ and $D$ are positively definite diagonal matrices, $\dot{V}_2(w)=0$ holds  only when $ \dot P^g = \dot P^l = \dot \omega=0$. Therefore, for any $w(0)  \in W_2^*$, the trajectory $w(t)$ satisfies 
       \bq
       \label{eq:dotV=0}
       \dot P^g(t) \ = \ \dot P^l(t) \ = \ \dot \omega(t) \ = \ 0, \qquad t\ge 0
       \eq
Hence $P^g(t)$, $P^l(t)$ and $\omega(t)$ are all constants due to the boundedness property guaranteed by  Lemma \ref{lemma:decreasing.2}.

   On the other hand, for $\dot V_2(w)=0$, both terms in  (\ref{saddle point}) have to be zero, implying that 
        	$$\hat L_2(w_1(t),w_2^*) = \hat L_2(w^*_1,w^*_2)$$
        	must hold in $W_2$. Differentiating with respect to $t$ gives
        	\begin{align}
	        	\left(\frac{\partial}{\partial w_1}\hat L_2(w_1(t),w_2^*)\right)^T\cdot\dot w_1(t)=0        	=-\dot{\tilde{\phi}}^T(\Gamma^{\tilde \phi}) ^{-1}\dot{\tilde{\phi}}
        	\end{align}
        	The second equality holds due to Eq. \eqref{eq:dotV=0} and \eqref{eq:closedloop.2h}. Then we can conclude $\dot{\tilde{\phi}}=0$ immediately, implying $\tilde{\phi}$ is also constant in $W^*$ due to its boundedness.        	
        	
%        	Let $W_2 :=\{w\in \Omega_0 | \ \dot P^g= \dot P^l= \dot \omega=\dot{\tilde \phi} =0\}$.
%        	Then \eqref{dot V2} and \eqref{dot V22} imply that $w\in W_2$ if and only if $\dot V_2(w) = 0$.
%        	
%        	
%        	For the second assertion, fix any $w(0)\in W_2^*$. 
	Invoking the close-loop dynamics  \eqref{eq:closedloop.2},  $\dot{\tilde{\theta}}(t)$, $\dot{\eta}^+(t)$, $\dot{\eta}^-(t)$ and $\dot{\lambda}(t)$ must be constants in $W_2^*$ as $P^{g}(t), P^l(t), \omega(t)$ and ${\tilde{\phi}}(t)$ are all constants. Then we conclude that $\dot{\tilde{\theta}}(t)= \dot{\eta}^+(t)=\dot{\eta}^-(t)=\dot{\lambda}(t)=0$ holds for all $t\ge 0$ due to the boundedness property of $w(t)$ (Lemma \ref{lemma:decreasing.2}).
       	This implies that any $w(0)\in W_2^*$ must be an equilibrium point,  completing the proof.
        \end{proof}
   
\begin{proof}[Proof of Theorem \ref{thm:stability.22} ] 
		 Fix any initial state $w(0)$ and consider the trajectory $(w(t), t\geq 0)$ of the
		 closed-loop system close-loop dynamics  \eqref{eq:closedloop.2}.		 
		 As mentioned in the proof of Lemma \ref{lemma:decreasing.3},
		 $w(t)$ stays entirely in a compact set $\Omega_0$.   Hence there exists an infinite 
		 sequence of time instants ${t_k}$ such that $w(t_k)\to \hat {w}^*$ as $t_k\to\infty$, 
		 for some $\hat w^* \in W_2^*$. Lemma \ref{lemma:decreasing.3} guarantees that
		 $\hat w^*$ is an equilibrium point of the closed-loop system \eqref{eq:closedloop.2}, and hence $\hat w^*=H(\hat{w}^*)$.
		  Thus, using this specific equlibrium point $\hat {w}^*$ in the definition of $V_2$, we have 
%		 \begin{align}
%		 	V_2^* = \lim_{t\to \infty} V_2(t) = \lim_{t_k\to \infty} V_2(t_k) = V_2(\lim_{t_k\to \infty} w(t_k)) = V_2(\hat w^*) = 0
%		 	\nonumber
%		 \end{align}
		 \begin{align}
			 V_2^* =\lim\limits_{t\to \infty} V_2(w(t)) &= \lim\limits_{t_k\to \infty} V_2(w(t_k)) \nonumber\\
				 &\qquad=\lim\limits_{w(t_k) \to \hat w^*} V_2\big( w(t_k)\big) = V_2(\hat w^*) = 0 \nonumber
		 \end{align}
		 Here, the first equality uses the  fact that
		 $V_2(t)$ is nonincreasing in $t$; the second equality uses the fact that
		 $t_k$ is the infinite sequence of $t$; the third equality uses the fact that $w(t)$ is
		 absolutely continuous in $t$; the fourth equality is due to the upper semi-continuity of $V_2(w)$, and the last equality holds as $\hat {w}^*$ is an equilibrium point of $V_2$. 
		 
		 The quadratic term $(w-\hat w^*)^T\Gamma_2^{-2}(w-\hat w^*)$
		 in $V_2$ then implies that $w(t)\to \hat {w}^*$ as $t\to \infty$, which completes the proof. 
\end{proof}         

      
%        Based on Lemma \ref{lemma:eta} and Lemma \ref{lemma:decreasing.2}, we can further prove Theorem \ref{thm:stability.22}. Similar to the proof of Theorem \ref{thm:stability.2}, we only need to prove that $\dot P^g= \dot P^l= \dot \omega= \dot{\tilde \phi} = 0$ implies that $\dot v(t)=0$.
%        
%        \begin{proof}[Proof of Theorem \ref{thm:stability.22}]
%        	$\Rightarrow$ 1). 	As explained previously, in terms of the equivalence between the closed-loop system \eqref{eq:model.1}\eqref{eq:control.2} and system \eqref{eq:model.4}, we only need to prove system \eqref{eq:model.4} is asymptotically stable. 
%        	
%        	Let $v(\infty):=\lim\limits _{t\rightarrow+\infty}(v(t))$. Then according to Lemma \ref{lemma:decreasing.2},  we have $(P^g(\infty), P^l (\infty), \omega (\infty), \tilde \phi (\infty))$
%        	are all finite constant vectors on $Z^+_n$.
%        	
%        	We first prove $\dot {\tilde \theta} (\infty)=0$. Since $\dot \omega(\infty)=0$, \eqref{eq:closedloop.2b}
%        	yields
%        	\bqn
%        	P^g(\infty)-P^l(\infty)-p-D\omega(\infty)-CB\tilde \theta(\infty)=0
%        	\eqn
%        	Differentiating the equation gives $CB \dot{\tilde\theta}(\infty)=CBC^T \dot{\theta}(\infty)=0$.
%        	Hence $\dot{\theta}(\infty)=\dot{\theta}_0(\infty)\cdot \textbf{1}$ and $\dot {\tilde \theta}(\infty)= C^T\dot {\theta}(\infty)=0$.
%        	
%        	We then prove $\dot \lambda =0$. Note that $(P^g(\infty), P^l (\infty), \omega (\infty), \tilde \phi (\infty))$
%        	are finite constant vectors on $Z^+_n$ and $p$ is a finite constant vector. Therefore, $ \lambda_j(t) (\forall j\in N)$ on the set $Z_b^+ $ satisfies 
%        	\bq
%        	\label{eq:converge.2}
%        	\dot{\lambda}_j(\infty)=\gamma_j^{\lambda}\left(P^g_j(\infty) -P^l_j(\infty)-p_j-\tilde{U}(\phi(\infty))\right)
%        	\eq
%        	Thus we have   
%        	\bq
%        	(\gamma^{\lambda}_j)^{-1} \cdot \dot \lambda_j(\infty)=z_j(\infty)=c_j 
%        	\label{eq:zc-constant}
%        	\eq
%        	where $c_j$ is a finite constant.
%        	
%        	We will show $c_j$ are identical  for all$j\in N$. To this end,         
%        	we assume for the sake of contradiction that $c_j $ are not identical.
%        	Then without loss of generality we let $c_1 \le c_2 \le c_3... \le c_{|N|}$.   
%        	
%        	a) In case $c_1>0$. We define a set $S_1:=\{j\in N_1 \, |\, c_j=c_1\}$. $S_1$ is the set of  all the neighboring nodes of node
%        	1, which have the same "$c $" values as node 1. Then we  recursively define 
%        	\bqn
%        	S_2 &:=& \{j\in N_i \,| \, i\in S_1, \ c_j=c_1\}\\
%        	\cdots\\
%        	S_m &:=& \{j\in N_i \, | \, i\in S_{m-1} , \ c_j=c_1\}. 
%        	\eqn    
%        	
%        	The recursive process repeats until $S_m=S_{m-1}$. As the total number of nodes are finite, this process
%        	must stop in finite steps. Let $S=\cup^m_{i=1} \ S_i$. Then we have $\dot{\lambda}_j=c_1$ for all $j\in S$. Moreover, all the nodes in $S$ are connected. 
%        	
%        	Consider \eqref{eq:closedloop.2h}. Since $\dot {\tilde \theta}(\infty)=0$,
%        	for any $(i,j)\in E$, we have
%        	\bqn
%        	0&=&B_{ij}(\lambda_i(\infty)-\lambda_j(\infty))+B_{ij}(z_i(\infty)-z_j(\infty))\\
%        	&&+\eta^-_{ij}(\infty)-\eta^+_{ij}(\infty)
%        	\eqn   
%        	Differentiating this equation yields
%        	\bq
%        	0&=&B_{ij}(\dot{\lambda}_i(\infty)-\dot{\lambda}_j(\infty))+\dot{\eta}^-_{ij}(\infty)-\dot{\eta}^+_{ij}(\infty)
%        	\label{eq:lambda-eta}
%        	\eq
%        	
%        	Consider $(i,j)\in S$, there is $\dot {\lambda}_i(\infty)=\dot {\lambda}_j(\infty)=c_1>0$. Then we have $\dot{\eta}^-_{ij}(\infty)=\dot{\eta}^+_{ij}(\infty)$.
%        	Since $\phi_{ij}(\infty)$ is a constant, according to Lemma \ref{lemma:eta}, there must be $\dot{\eta}^-_{ij}(\infty)=\dot{\eta}^+_{ij}(\infty)=0$.
%        	It means there is no congestion between any two neighboring nodes in $S$. 
%        	Moreover, since $c_1>0$, $\lambda_j(\infty)$
%        	goes to $+\infty$ for all $j\in S$. Hence, according  to \eqref{eq:closedloop.2f} and \eqref{eq:closedloop.2f} we  have $P^{g}_j(\infty)=\underline{P}^g_j$ and $P^{l}_j(\infty)=\overline{P}^l_j$ for all $j\in S$.
%        	
%        	Next we define two additional sets:
%        	\bqn
%        	U &:=& \{j\in N_i \,|\, i\in S, \ c_j < c_1 \}\\
%        	U_c&:=& \{(i,j)\in E \, |\, i\in S, \ j\in U_1 \}
%        	\eqn
%        	
%        	Then $U_c$ is indeed a cut set that separates $S$ and $U$.  Moreover, according to \eqref{eq:lambda-eta},
%        	for any $(i,j)\in U_c$, we have 
%        	\bqn
%        	\dot{\eta}^+_{ij}(\infty)-\dot{\eta}^-_{ij}(\infty)=B_{ij}(\dot{\lambda}_i(\infty)-\dot{\lambda}_j(\infty))>0
%        	\eqn   
%        	
%        	Furthermore, in terms of Lemma \ref{lemma:eta}, there must be $\dot{\eta}^+_{ij}>0$ and $\dot{\eta}^+_{ij}=0$.
%        	This implies $\phi_{ij}(\infty)>\overline{\theta}_{ij}(\infty)$ is always true for all the edges connecting
%        	$U$ and $S$. Hence, the constraints \eqref{eq:opt.2c} are violated for all edge $(i,j)\in U_c$.
%        	
%        	Note that  $P^{g}_j(\infty)=\underline{P}^g_j$ and $P^{l}_j(\infty)=\overline{P}^l_j$ for all $j\in S$.
%        	This means both the generation $P^g_j$ and controllable load $P^l_j$ have reached their maximum regulation
%        	capabilities. On the other hand, all the tie lines connected $S$ have been overloaded. As a sequence,
%        	it is impossible to eliminate all the violations of \eqref{eq:opt.2c} by adjusting $P^g_j$ and $P^l_j$
%        	in any case. This implies the original NBO problem \eqref{eq:opt.2} is not feasible, which contradicts
%        	to Assumption A3. Therefore all $c_j$ must be identical. 
%        	
%        	b) In case $c_1 \le 0$. Then there must be $c_{|k|}<0$. Then we can start constructing $S$ from $c_{|N|}$.
%        	Similar to the process for the case of $c_1>0$, it also follows contradiction. 
%        	
%        	In the context of a) and b), we can conclude that $c_j$ must be identical for all $j\in N$. 
%        	
%        	Then we further claim that $c_j$ must be zero, as we explain. 
%        	If $c_j>0$, then all $\lambda_j(\infty)$ goes to  $+\infty$, while if $c_j<0$ then all $\lambda_j(\infty)$
%        	goes to $-\infty$.
%        	Both the cases mean that the NBO problem \eqref{eq:opt.2} is not feasible, which is in contradiction
%        	to Assumption A3. This implies $\dot \lambda_j(\infty)=z_j(\infty)=0$  holds on $Z^+_n$ for all $j\in N$. 
%        	
%        	Next we prove $\dot{\eta}^+_{ij}(\infty)=\dot{\eta}^+_{ij}(\infty)=0$ for all $(i,j)\in E$. Since $\dot {\lambda}_i(\infty)=\dot {\lambda}_j(\infty)=0$ , this can
%        	directly obtained according to \eqref{eq:lambda-eta} and Lemma \ref{lemma:eta}.     
%        	
%        	Now we have $\dot {\tilde \theta}(\infty)=\dot \omega(\infty)=\dot {P}^g(\infty)=\dot {P}^l(\infty)=\dot \lambda(\infty)=\dot {\eta}^+(\infty)=\dot {\eta}^-(\infty)=\dot {\tilde \phi}(\infty)=0$, indicating that system \eqref{eq:model.4} is asymptotically stable. The
% equivalence between \eqref{eq:model.4} and the closed-loop system \eqref{eq:model.1}\eqref{eq:control.2} implies the latter is also asymptotically stable. Then from Theorem \ref{thm:5} we conclude 1) is true.  
%        	      
%      

        
       


%\newpage
%\bibliographystyle{unsrt}
% \bibliography{../../PowerRef-201202}




\end{document}
