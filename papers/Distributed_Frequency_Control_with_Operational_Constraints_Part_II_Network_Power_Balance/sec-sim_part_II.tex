\section{Case studies}
\subsection{System configuration}
A four-area system based on Kundur's four-machine, two-area system \cite{Fang:Design} \cite{Kundur:Power} is used to test our optimal frequency controller. There are one (aggregate) generator (Gen1$\sim$Gen4), one controllable (aggregate) load (L1c$\sim$L4c) and one uncontrollable (aggregate) load (L1$\sim$L4) in each area, which is shown in Fig.\ref{fig:system}. The parameters of generators and controllable loads are given in Table \ref{tab:SysPara}. The total uncontrollable load in each area are identically 480MW. At time $t=10s$, we add step changes on the uncontrollable loads in four areas to test the performance of our controllers. 

All the simulations are implemented in PSCAD \cite{website:PSCAD} with 8GB memory and 2.39 GHz CPU.  We use the detailed electromagnetic transient model of three-phase synchronous machines to simulate generators with both governors and exciters.  The uncontrollable load L1-L4 are modelled by the fixed load in PSCAD, while controllable load L1c-L4c are formulated by the self-defined controlled current source. The closed-loop system diagram is shown in Fig.\ref{fig:control2}. We need measure loacal frequency, generation, controllable load and tie-line power flows to compute control demands. Only $\tilde{\phi}_{ij}$ are exchanged between neighbors. All variables are added by their initial steady state values to explicitly show the actual values.
\begin{figure}[htp]
        \centering
        \includegraphics[width=0.45\textwidth]{topology.pdf}
        \caption{Four-area power system}
        \label{fig:system}
\end{figure}

\begin{figure}[htp]
	\centering
	\includegraphics[width=0.4\textwidth]{control_partII.pdf}
	\caption{Closed-loop system diagram}
	\label{fig:control2}
\end{figure}



\begin{table}[http]
        %\renewcommand{\arraystretch}{1.3}
        \centering
        \caption{\\ \textsc{System parameters}}
        \label{tab:SysPara}
        \begin{tabular}{c c c c c c c}
                \hline 
                Area $j$ & $D_j$ & $R_j$ & $\alpha_j$ & $\beta_j$ &$T^g_j$ &$T^l_j$\\
                \hline
                1 & 0.04  & 0.04  & 2    & 2.5  & 4    & 4 \\
                2 & 0.045  & 0.06  & 2.5  & 4    & 6    & 5 \\
                3 & 0.05  & 0.05  & 1.5  & 2.5  & 5    & 4 \\
                4 & 0.055  & 0.045 & 3    & 3    & 5.5  & 5 \\
                \hline
        \end{tabular}
\end{table}



\subsection {Network Power Balance Case}

In this case, the generations in each area are initiated as (560.9, 548.7, 581.2, 540.6) MW and the controllable loads (70.8, 89.6, 71.3, 79.4) MW. The load changes are identical to those in Table \ref{tab:DistConstraintsnet}, which are also unknown to the controllers. We  use method in Remark 4 to estimate the load changes. Operational constraints on generations, controllable loads and tie lines are shown in Table \ref{tab:DistConstraintsnet}. 

\begin{table}[htb]
	%\renewcommand{\arraystretch}{1.3}
	\centering
	\caption{\\ \textsc {Capacity limits in network case}}
	\label{tab:DistConstraintsnet}
	\begin{tabular}{c c c c c}
		\hline 
		& Area 1 & Area 2 & Area 3& Area 4\\
		\hline
		[$\underline{P}^g_j$,$\overline{P}^g_j$] (MW) & [550, 710] & [530, 680] & [550, 700]& [530, 670]\\
		\hline
		[$\underline{P}^l_j$,$\overline{P}^l_j$] (MW) & [20, 80] & [60, 100] & [20, 80] & [35, 80]\\
		\hline
		\hline
		Tie line & (2,1) & (3,1) & (3,2) & (4,2)\\
		\hline
		[$\underline P_{ij}$,$\overline P_{ij}$] (MW) & [-65, 65] & [-65, 65] & [-65, 65] & [-65, 65]\\
		\hline
	\end{tabular}
\end{table}

\subsubsection {Stability and optimality}
The dynamics of local frequencies  and tie-line power flows are illustrated in  Fig.\ref{fig:stabnet}. The frequencies are well restored in all four control areas while the tie line powers are remained within their acceptable ranges. The generations and controllable loads are different from that before disturbance, indicating that the system is stabilized at a new steady state. The resulting equilibrium point is  given in Table IV, which is identical to the optimal solution of  \eqref{eq:opt.2} computed by centralized optimization using CVX. These simulation results confirm that our controller can autonomously guarantee the frequency stability while achieving optimal operating point in the overall system. 

\begin{figure}[htp]
	\centering
	\includegraphics[width=0.5\textwidth]{fig6.pdf}
	\caption{Dynamics of frequency (left) and tie-line flows (right) in network balance case}
	\label{fig:stabnet}
\end{figure}

\begin{table}[htb]
	%\renewcommand{\arraystretch}{1.3}
	\centering
	\caption{\\ \textsc {Equilibrium Points}}
	\label{tab:eps}
	\begin{tabular}{c c c c c}
		\hline 
		& Area 1 & Area 2 & Area 3& Area 4\\
		\hline
		$P^{g*}_j$ (MW)  & 620  & 596  & 660  & 580\\
		$P^{l*}_j$ (MW)  & 23.6 & 59.8 & 23.6 & 39.7\\
		\hline
		\hline
		Tie line      & (2,1) & (3,1) & (3,2) & (4,2)\\
		\hline
		$P_{ij}^*$ (MW) & -39.94 & 13.35 & 53.27 & -59.6\\
		\hline
	\end{tabular}
\end{table}

\subsubsection{Dynamic performance}
In this subsection, we analyze the impacts of operational (capacity and  line power) constraints on the dynamic property. Similarly, we compare the  responses of  frequency controllers with and without considering input saturations. The trajectories of mechanical power of turbines and controllable loads are shown in Fig.\ref{fig:dynamicnet.mec} and Fig.\ref{fig:dynamicnet.2}, respectively. In this case, the system frequency is restored, and the same optimal equilibrium point is achieved. 

\begin{figure}[htp]
	\centering
	\includegraphics[width=0.5\textwidth]{network_pm.pdf}
	\caption{Mechanical outputs with(left)/without(right) capacity constraints}
	\label{fig:dynamicnet.mec}
\end{figure}

%\begin{figure}[htp]
%	\centering
%	\includegraphics[width=0.5\textwidth]{fig7.pdf}
%	\caption{Generators' outputs with(left)/without(right) capacity constraints}
%	\label{fig:dynamicnet.1}
%\end{figure}

\begin{figure}[htp]
	\centering
	\includegraphics[width=0.5\textwidth]{fig8.pdf}
	\caption{Controllable loads with(left)/without(right) capacity constraints}
	\label{fig:dynamicnet.2}
\end{figure}

\subsubsection {Congestion analysis}
In this scenario, we reduce tie-line power constraints to $\overline{P}_{ij}=-\underline{P}_{ij}=50$MW, which causes congestions in tie-line (2,3) and  (2,4). The steady states under the distributed control are listed in Table \ref{tab:epcong}. Note that $P_j^{g*} - {p_j} + P_j^{l*} - \sum\limits_{k:j \to k} {{P_{jk}^*}}  + \sum\limits_{i:i \to j} {{P_{ij}^*}}  = 0$ hold in each area. 

The dynamics of tie-line powers in two different scenarios shown in Fig.\ref{fig:congestion} indicate that (2,4) reaches the limit in steady state. However, by adopting the proposed fully distributed optimal frequency control, the congestion is eliminated and all the tie line powers remain within the limits.  Thus, congestion control is achieved optimally in a distributed manner.

\begin{table}[htb]
	%\renewcommand{\arraystretch}{1.3}
	\centering
	\caption{\\ \textsc {Simulation Results with Congestion}}
	\label{tab:epcong}
	\begin{tabular}{c c c c c}
		\hline 
		& Area 1 & Area 2 & Area 3& Area 4\\
		\hline
		$P^{g*}_j$ (MW) & 618   & 595  & 658  & 585\\
		$P^{l*}_j$ (MW) & 25.1  & 60.7 & 25.1 & 34.9\\
		\hline
		\hline
		Tie line  & (2,1) & (3,1) & (3,2) & (4,2)\\
		\hline
		$P_{ij}^*$ (MW) & -36.4 & 13.1 & 49.5 & -49.9\\
		\hline
	\end{tabular}
\end{table}

\begin{figure}[htp]
	\centering
	\includegraphics[width=0.5\textwidth]{fig10.pdf}
	\caption{Tie line power with(left)/without(right) capacity constraints}
	\label{fig:congestion}
\end{figure}
