\section{Introduction}
In Part I of the paper we have investigated the optimal frequency control of multi-area power system with operational constraints \cite{Wang:DistributedFrequency}. In that case, the tie-line powers are required to be unchanged in the steady state after load disturbances, which implies the power mismatch in each area has to be balanced individually. It is referred to as the per-node power balance case. In Part II of the paper, we consider the transmission congestion in the distributed optimal frequency design. 

The per-node balance case in Part I mainly considers the situation where the power delivered from one area to another is fixed, e.g. contract power, which should not be violated in normal operation. However, in some circumstances, control areas may cooperate for better frequency recovery or regulation cost reduction. In this case, power mismatch may be  balanced by all generations and controllable loads among all control areas in cooperation. Similar situations also appear in one control area with multiple generators and controllable loads that cooperate to eliminate the power mismatch in the area. It is referred to as the network power balance case. Compared with the per-node balance case, the most challenging problem in this case is that the tie-line powers may change and congestions may occur. In addition, local information is not sufficient and neighboring information turns to be helpful. As for the constraints, the tie-line power constraints are not hard limits, which only need to be satisfied at equilibrium. The capacity limits on the generations and  controllable loads also required to be satisfied both in  steady state and during transient.

In the recent  literature of frequency control, tie-line power constraints are considered in \cite{Mallada-2017-OLC-TAC, Stegink:aunifying, Stegink:aport,Yi:Distributed, zhao:aunified, zhang:real}. In \cite{Mallada-2017-OLC-TAC},  tie-line power constraints are included in the load-side secondary frequency control. A virtual variable is used to estimate the tie-line power, whose value is identical to the tie-line power at equilibrium. In \cite{Stegink:aunifying, Stegink:aport}, an optimal economic dispatch problem including  tie-line power constraints is formulated, then the solution dynamics derived from a primal-dual algorithm is shaped as a port-Hamiltonian form. The power system dynamics also have a port-Hamiltonian form, which are interconnected with the solution dynamics to constitute a closed-loop Hamiltonian system. Then, the optimality and stability are proved. In \cite{zhao:aunified}, a unified method is proposed for  primary and secondary frequency control, where the congestion management is implemented in the secondary control. In \cite{zhang:real}, a real-time control framework is proposed for tree power networks, where transmission capacities are considered. 

% Moreover the closed-loop system remains asymptotically stable and converges to the unique optimal point.

% \fliu{The current presentation looks quite clear for me. Should we mention the \emph{approximate} primal-dual in Introduction?}
% 
% \slow{Comparison of per-node and network case is in Remark 3.  Refer to \cite{LiZhaoChen2016}.}
 
Similar to the per-node balance case, hard limits, such as capacity constraints of power injections on buses, are enforced only in the steady-state in the literature, which may fail if such constraints are violated in transient. Here we construct a fully distributed control to recover nominal frequency while eliminating congestion. Differing from the literature, it enforces regulation capacity constraints not only at equilibrium, but also during transient. We show that the controllers together with the physical dynamics serve as  primal-dual updates with saturation for solving the optimization problem. 
The optimal solution of the optimization problem and the equilibrium point of closed-loop system are identical. 

The  enforcement of capacity constraints during transient and tie-line power limits simultaneously makes the stability proof difficult. Specifically, the Lyapunov function  is not continuous anymore, as in the per-node case in Part I of the paper. In this situation, the conventional LaSalle's invariance principle does not apply. To overcome the difficulty, we construct a nonpathological Lyapunov function to mitigate the impacts of nonsmooth dynamics. The salient features of the controller are: 
\bee
\item \emph{Control goals:} the controller restores the nominal frequency and balance the power mismatch in the whole system after unknown load disturbance while minimizing the regulation costs; 

\item \emph{Constraints:} the regulation capacity constraints are always enforced even during transient and the congestions can be eliminated automatically; 

\item \emph{Communication:} only neighborhood communication is needed in the  network balance case;

% \slow{Comment that per-node balance case means that
%no communication is needed between control areas.  It does not imply that the generators within
%the same area will not need communication to restore the nominal frequency.
%Check and cite MIT algorithm that requires no communication to restore nominal frequency.}
%
%\fliu{You're right. Within a control area, there should be a requirement of communication. But I am not sure which paper you mean...maybe Zhaojian know it? }
%
%\wang{I find two papers may be related to the problem,"Distributed Frequency Control in Power Grids under limited communication" and "Modeling the Impact of Communication Loss on the power grid under emergency control". Both of them deal with the communication failure problem. Are they right? I think it is not necessary to mention communication within one area as we use the aggregate model. How about if the reviewer wants to know, we cite them then? }

\item \emph{Measurement:} the controller is adaptive to unknown load disturbances automatically with no need of load measurement. 
\eee

The rest of this paper is organized as follows. In Section II, we describe our model. Section III formulates the optimal frequency control problem in the network balance case, presents the distributed  frequency controller and proves  the optimality, uniqueness and stability of the closed-loop equilibrium. Simulation results are given in Section IV. Section V concludes the paper.




