%Options: KBD=VOID; STD=VOID; LANG=ENGLISH; FMT=LATEX; HYPHEN=default;
\documentclass[aps,11pt,tightenlines,notitlepage,superscriptaddress,longbibliography,nofootinbib]{revtex4-1}
%\documentclass[11pt]{article}
\usepackage[final]{epsfig}
\usepackage{amssymb}
\usepackage{latexsym}
\usepackage{wrapfig}
\usepackage{amsmath}
\usepackage{booktabs}
\usepackage{threeparttable}
\usepackage{xcolor}
\usepackage{amsthm}
\usepackage{comment}
\usepackage{graphicx, caption, subcaption, enumitem}
\usepackage{esint}
\usepackage{natbib}
% set page format suitable for 8.5 by 11 paper

\setlength{\oddsidemargin}{-0.25in}
\setlength{\evensidemargin}{-0.25in}
\setlength{\textwidth}{7.0in}
\setlength{\topmargin}{-0.5in}
\setlength{\textheight}{9.0in}
\setlength{\belowcaptionskip}{2mm}




% define new commands
\newcommand{\R}{\mathbb{R}}
\newcommand{\Z}{\mathbb{Z}}
\newcommand{\bfa}{{\bf a}}
\newcommand{\bfb}{{\bf b}}
\newcommand{\bfc}{{\bf c}}
\newcommand{\bfd}{{\bf d}}
\newcommand{\bfe}{{\bf e}}
\newcommand{\bff}{{\bf f}}
\newcommand{\bfg}{{\bf g}}
\newcommand{\bfh}{{\bf h}}
\newcommand{\bfk}{{\bf k}}
\newcommand{\bfm}{{\bf m}}
\newcommand{\bfn}{{\bf n}}
\newcommand{\bfo}{{\bf o}}
\newcommand{\bfp}{{\bf p}}
\newcommand{\bfq}{{\bf q}}
\newcommand{\bfr}{{\bf r}}
\newcommand{\bfs}{{\bf s}}
\newcommand{\bft}{{\bf t}}
\newcommand{\bfu}{{\bf u}}
\newcommand{\bfv}{{\bf v}}
\newcommand{\bfw}{{\bf w}}
\newcommand{\bfx}{{\bf x}}
\newcommand{\bfy}{{\bf y}}
\newcommand{\tbfy}{{\tilde{\bfy}}}
\newcommand{\bfz}{{\bf z}}
\newcommand{\bfA}{{\bf A}}
\newcommand{\bfB}{{\bf B}}
\newcommand{\bfC}{{\bf C}}
\newcommand{\bfD}{{\bf D}}
\newcommand{\bfF}{{\bf F}}
\newcommand{\bfG}{{\bf G}}
\newcommand{\bfH}{{\bf H}}
\newcommand{\bfI}{{\bf I}}
\newcommand{\bfK}{{\bf K}}
\newcommand{\bfM}{{\bf M}}
\newcommand{\bfP}{{\bf P}}
\newcommand{\bfQ}{{\bf Q}}
\newcommand{\bfR}{{\bf R}}
\newcommand{\bfS}{{\bf S}}
\newcommand{\bfT}{{\bf T}}
\newcommand{\bfU}{{\bf U}}
\newcommand{\bfV}{{\bf V}}
\newcommand{\bfW}{{\bf W}}
\newcommand{\bfX}{{\bf X}}
\newcommand{\bfY}{{\bf Y}}
\newcommand{\vphi}{{\varphi}}
\newcommand{\eps}{{\varepsilon}}
\newcommand{\Nhat}{\hat{\mbox{\tiny {\bf N}}}}
\newcommand{\ehat}{\hat{\bf e}}
\newcommand{\nhat}{\hat{\bf n}}
\newcommand{\uhat}{\hat{\bf u}}
\newcommand{\phihat}{\hat{\varphi}}
\newcommand{\beq}{\begin{equation}}
\newcommand{\eeq}{\end{equation}}
\newcommand{\beqs}{\begin{eqnarray}}
\newcommand{\eeqs}{\end{eqnarray}}
\newcommand{\calA}{{\cal A}}
\newcommand{\calB}{{\cal B}}
\newcommand{\calC}{{\cal C}}
\newcommand{\calD}{{\cal D}}
\newcommand{\calE}{{\cal E}}
\newcommand{\calF}{{\cal F}}
\newcommand{\calG}{{\cal G}}
\newcommand{\calH}{{\cal H}}
\newcommand{\calI}{{\cal I}}
\newcommand{\calL}{{\cal L}}
\newcommand{\calM}{{\cal M}}
\newcommand{\calN}{{\cal N}}
\newcommand{\calP}{{\cal P}}
\newcommand{\calS}{{\cal S}}
\newcommand{\calT}{{\cal T}}
\newcommand{\calU}{{\cal U}}
\newcommand{\calV}{{\cal V}}
\newcommand{\calW}{{\cal W}}
\newcommand{\sint}{-\hspace{-4mm}\int} % slashed integral, math
\newcommand{\ssint}{-\hspace{-3mm}\int} % slashed integral, text
\newtheorem{theorem}{Theorem}[section]
\newtheorem{corollary}{Corollary}[section]
\newtheorem{lemma}{Lemma}[section]
\newtheorem{proposition}{Proposition}[section]
\newtheorem{remark}{Remark}[section]
\newtheorem{definition}{Definition}[section]
\newtheorem{conjecture}{Conjecture}[section]

%
\DeclareMathOperator{\sign}{sign}
\DeclareMathOperator{\Tr}{Tr}
\DeclareMathOperator{\rank}{rank}
\DeclareMathOperator{\Mod}{mod}
\DeclareMathOperator{\curl}{curl}
\DeclareMathOperator{\sym}{sym}
\DeclareMathOperator{\modTry}{mod}
\DeclareMathOperator{\dist}{dist}

\begin{document}

\begin{center}
\Large
%{\textbf{Universal design principles for shape-morphing with origami}} \\
%or\\
{\textbf{Supplemental Material (SM): Continuum field theory for the deformations of planar kirigami}}
\vspace{2mm}
\large

\vspace{2mm}
Yue Zheng, Imtiar Niloy, Paolo Celli,  Ian Tobasco and Paul Plucinsky
\end{center}

\renewcommand{\thesection}{SM.\arabic{section}} 
\renewcommand{\thefigure}{S\arabic{figure}} 
\renewcommand{\thepage}{S\arabic{page}}  
\renewcommand{\theequation}{S\arabic{equation}}  

\tableofcontents

\vspace{.5cm}
\noindent \textbf{Notation.}\;We refer to equations and figures from the main text as Eq.\;(1), Fig.\;1 and so on. The SM versions are distinguished by an ``S", as in Eq.\;(S1), Fig.\;S1.  

\section{Unit cells of four quad panels and four parallelogram slits}
\label{s:UC}

In this section, we derive all possible unit cells in quad-kirigami composed of four convex quad panels and four parallelogram slits.  The result is based on the geometric argument in Fig.\;\ref{fig:GeomArg}, which establishes a series of necessary conditions that eventually leads to the overall characterization. 

We start with two generic quad panels connected at a corner (Fig.\;\ref{fig:GeomArg}(a)). We will show that these two panels \textit{seed} the entire pattern.  First, observe that the two panels must repeat along the vector indicated by $\mathbf{t}$ (Fig.\;\ref{fig:GeomArg}(b)), since our unit cell consists of four quad panels. As the slits are assumed to be parallelograms, we can conclude their shapes from two of their sides (Fig.\;\ref{fig:GeomArg}(c)).  We now have identified the four parallelogram slits of our unit cell. The rest of the pattern is obtained through a tessellation (Fig.\;\ref{fig:GeomArg}(d)), in this case along the vector $\mathbf{s}$. This identifies the final two panels of the unit cell (Fig.\;\ref{fig:GeomArg}(e)), which turn out to be a rotation of the seed by $180^{\circ}$.  

\begin{figure}[h!]
\centering
\includegraphics[width =\linewidth]{GeomArg.pdf}
\caption{Geometric argument for constructing unit cells with quad panels and parallelogram slits. }
\label{fig:GeomArg}
\end{figure}

As a word of caution, we note that it is possible for a given seed to initiate an invalid tessellation, in the sense that two of the resulting panels overlap. One can avoid this by counter-rotating the seeded panels until a valid configuration is obtained. %simply   In such cases, one can always take the two seeded panels and rotate them relative to each other to obtain a valid configuration with those panel shapes. 


\section{Mechanism deformations}\label{sec:KinAppend}

Here we  describe all possible planar and \textit{mechanism deformations} of our kirigami, i.e., those that  rotate and translate the panels in the plane while preserving the pattern's topology. Coarse-graining leads to the effective description in Eq.\;(3-4) of the main text. We specialize to kirigami with rhombi-slits at the end.

\subsection{Identifying mechanisms}
We first study a single unit cell of four panels $\mathcal{P}_i$, labeled as shown in Fig.\;\ref{fig:DescribeKin}, with each $\mathbf{x}_{ij}$ indicating the corner points connecting adjacent  panels. 
The rigid deformations of this cell have the form 
\begin{equation}
\begin{aligned}\label{eq:Rigid0}
\mathbf{y}_i(\mathbf{x}) = \mathbf{R}_{i} \mathbf{x}+ \mathbf{c}_i, \quad \mathbf{x} \in \mathcal{P}_i, \quad i = 1,\ldots,4
\end{aligned}
\end{equation}
for appropriately chosen 2D rotations $\mathbf{R}_i$ and 2D translations $\mathbf{c}_i$.  In particular, to preserve the topology, the deformations must satisfy 
\begin{equation}
\begin{aligned}\label{eq:connect}
\mathbf{y}_{1}(\mathbf{x}_{41}) = \mathbf{y}_4(\mathbf{x}_{41}), \quad \mathbf{y}_2(\mathbf{x}_{12}) =\mathbf{y}_1(\mathbf{x}_{12}), \quad \mathbf{y}_{3}(\mathbf{x}_{32}) = \mathbf{y}_{2}(\mathbf{x}_{32}), \quad \mathbf{y}_{4}(\mathbf{x}_{43}) = \mathbf{y}_{3}(\mathbf{x}_{43}).
\end{aligned}
\end{equation}
By routine manipulation, this restriction is equivalent to prescribing three of the four translations as 
\begin{equation}
\begin{aligned}\label{eq:c123}
&\mathbf{c}_1 = (\mathbf{R}_4 - \mathbf{R}_1) \mathbf{x}_{41} + \mathbf{c}_4, \\
&\mathbf{c}_2 = (\mathbf{R}_4 - \mathbf{R}_1) \mathbf{x}_{41} + (\mathbf{R}_1 - \mathbf{R}_2) \mathbf{x}_{12} +  \mathbf{c}_4, \\
&\mathbf{c}_3 = (\mathbf{R}_4 - \mathbf{R}_1) \mathbf{x}_{41} + (\mathbf{R}_1 - \mathbf{R}_2)  \mathbf{x}_{12} + (\mathbf{R}_2 - \mathbf{R}_3) \mathbf{x}_{32} + \mathbf{c}_4
\end{aligned}
\end{equation}
and solving the \textit{loop compatibility condition}
\begin{equation}
\begin{aligned}\label{eq:connectFinal}
\mathbf{R}_1( \mathbf{x}_{12} - \mathbf{x}_{41}) + \mathbf{R}_{2}(\mathbf{x}_{32} - \mathbf{x}_{12}) + \mathbf{R}_3(\mathbf{x}_{43} - \mathbf{x}_{32}) + \mathbf{R}_4 (\mathbf{x}_{41} - \mathbf{x}_{43}) = \mathbf{0}.
\end{aligned}
\end{equation}
Concerning the nature of  (\ref{eq:connectFinal}), we note that each term in the sum is a vector describing one side of the deformed slit.  Summing the terms, as shown, traverses the sides in a loop that must return back to the starting point for the slit to remain closed.
The analysis, so far, is completely general and applies regardless of the shape of the slits.

\begin{figure}[t]
\centering
\includegraphics[width = 4in]{DescribeKin.png}
\caption{Notation for rigidly deforming a unit cell of the kirigami.}
\label{fig:DescribeKin}
\end{figure}


The next step is to solve for the rotations $\mathbf{R}_i, i = 1,\ldots, 4$, and build on this solution to construct the rigid panel deformations. Here, as it is consistent with the patterns in the main text, we simplify to the case of a parallelogram slit given by the vectors
\begin{equation}
\begin{aligned}\label{eq:parallelProp}
\mathbf{u} = \mathbf{x}_{12} - \mathbf{x}_{41} = -(\mathbf{x}_{43} - \mathbf{x}_{32}) , \quad \mathbf{v} = \mathbf{x}_{32} - \mathbf{x}_{12} = -( \mathbf{x}_{41} - \mathbf{x}_{43}).
\end{aligned}
\end{equation}
%i.e., that opposite sides have equal length and are parallel.  
The loop condition reduces to 
\begin{equation}
\begin{aligned}\label{eq:loopParallel}
(\mathbf{R}_1 - \mathbf{R}_3) \mathbf{u} + (\mathbf{R}_2 - \mathbf{R}_4) \mathbf{v} = \mathbf{0}
\end{aligned}
\end{equation}
and we read off the solution $\mathbf{R}_1 = \mathbf{R}_3$ and $\mathbf{R}_2 = \mathbf{R}_4$. That this is the only solution for the motion of the parallelogram slit follows from elementary geometry (e.g., by a routine application of the law of cosines). 
Thus, the rigid deformations of the unit cell are parameterized by rotations that satisfy
\begin{equation}\label{eq:rotations-result}
\begin{aligned}
\mathbf{R}_1 = \mathbf{R}_3 = \mathbf{R}(\gamma + \xi), \quad \mathbf{R}_2 = \mathbf{R}_4 = \mathbf{R}(\gamma - \xi)
\end{aligned}
\end{equation}
for two angles $\gamma$ and $\xi$, and translations $\mathbf{c}_{i}$, $i=1,2,3$ obeying (\ref{eq:c123}).  The angle $\gamma$ rotates the unit cell as a whole. The angle $\xi$ describes how its slit opens and closes. The slit angle $\angle \mathbf{x}_{41}\mathbf{x}_{12} \mathbf{x}_{32}$ in Fig.\;\ref{fig:DescribeKin} is deformed to $\angle \mathbf{x}_{41}\mathbf{x}_{12} \mathbf{x}_{32} + 2 \xi$ under our parameterization. 

%What is less obvious is that these are the only types of rotations that solve (\ref{eq:loopParallel}).  This result essentially follows from the law of cosines:  Suppose we fix the rotation $\mathbf{R}_1$ and $\mathbf{R}_2$. This means that we know two sides of the deformed slit and the angle between those two sides.  We also know what the length of the other two sides should be.  Using the law of cosines, this is enough information to determine the shape of the convex quadrilateral slit after deformation (assuming such a shape exists).  Algebraically, this means that $\mathbf{R}_3$ and $\mathbf{R}_4$ are uniquely determined given $\mathbf{R}_1$ and $\mathbf{R}_2$ (or there is no convex quad-slit solution). But we have already shown that $\mathbf{R}_1 = \mathbf{R}_3$ and $\mathbf{R}_2 = \mathbf{R}_4$ is a solution to the loop condition above.  So this must be the only solution. 


\begin{figure}[b]
\centering
\includegraphics[width = 7in]{MechDef.pdf}
\caption{Kirigami patterns with parallelogram slits  can deform as a mechanism in the plane.}
\label{fig:MechDef}
\end{figure}

Iterating the result of (\ref{eq:rotations-result}) yields a characterization of all rigid mechanism deformations of the kirigami: its panel rotations must alternate between $\mathbf{R}(\gamma + \xi)$ and $\mathbf{R}(\gamma -\xi)$ throughout. For a given set of alternating rotations, the corresponding panel translations that preserve the pattern's topology are uniquely prescribed up to an overall translation. Since $\xi$ is an angle describing how the slits open and close, it has a maximum possible value $\xi^{+} \in [0, \pi/2]$ and a minimum possible value $\xi^{-} \in [-\pi/2, 0]$ beyond which continued deformation results in  panels overlapping. In a valid mechanism, $\xi \in [\xi^{-}, \xi^+]$. Fig.\;\ref{fig:MechDef} depicts this result for the pattern in Fig.\;\ref{fig:GeomArg}(e).  The deformation continuously evolves as a function of $\xi$, distorting the slits while keeping the panels rigid until eventually one set of slits closes (at $\xi=\xi^-$ or $\xi=\xi^+$). %Hence, the deformations describe a mechanism parameterized by $\xi \in [\xi^{-}, \xi^+]$; we consistently use the term \textit{mechanism deformations} to describe this family of deformations. %All parallelogram slit patterns admit a mechanism. This property is key to them exhibiting a rich family of effective deformations under frustration.




\subsection{Coarse-grained description of mechanisms} In the main text, we coarse-grained the mechanism deformation discussed above to derive its \textit{effective deformation gradient} 
\begin{equation}
    \begin{aligned}
    \mathbf{F}_{\text{eff}} = \mathbf{R}(\gamma) \mathbf{A}(\xi).
    \end{aligned}
\end{equation}
To reiterate, this formula captures the bulk behavior of the mechanism  parameterized by the angles $\gamma$ and $\xi$. It applies to any kirigami pattern composed of a unit cell of four quad panels and four parallelograms slits. Recall the vectors $\mathbf{s}_i$, $\mathbf{t}_i$, $i = 1,\ldots,4$, from Fig.\;2.  From these vectors and Eq.\;(1), we find that 
\begin{equation}
    \begin{aligned}\label{eq:AxiExplicit}
    \mathbf{A}(\xi) = \mathbf{a}(\xi) \otimes \mathbf{s} + \mathbf{b}(\xi) \otimes \mathbf{W} \mathbf{s}
    \end{aligned}
\end{equation}
where $\mathbf{W} = \mathbf{R}(\pi/2)$, and 
\begin{equation}
\begin{aligned}\label{eq:abKin}
&\mathbf{a}(\xi) = \mathbf{R}(-\xi) \mathbf{a}_1 + \mathbf{R}(\xi) \mathbf{a}_2, \quad \mathbf{b}(\xi) = \mathbf{R}(-\xi) \mathbf{b}_1 + \mathbf{R}(\xi) \mathbf{b}_2.
\end{aligned}
\end{equation}
The vectors $\mathbf{a}_{1,2}$ and $\mathbf{b}_{1,2}$ encode the shape of the reference pattern explicitly via
\begin{equation}
\begin{aligned}\label{eq:abShape}
&\mathbf{a}_1 = |\mathbf{s}|^{-2} \mathbf{s}_{12}, && \mathbf{b}_1 =   \tfrac{1}{ (\mathbf{W} \mathbf{s} \cdot \mathbf{t})} \mathbf{t}_{14}    - \tfrac{(\mathbf{s} \cdot \mathbf{t})}{|\mathbf{s}|^2 (\mathbf{W} \mathbf{s} \cdot \mathbf{t})} \mathbf{s}_{12},  \\
&\mathbf{a}_2 = |\mathbf{s}|^{-2} \mathbf{s}_{34}, 
&& \mathbf{b}_2 =   \tfrac{1}{ (\mathbf{W} \mathbf{s} \cdot \mathbf{t})} \mathbf{t}_{23}    - \tfrac{(\mathbf{s} \cdot \mathbf{t})}{|\mathbf{s}|^2 (\mathbf{W} \mathbf{s} \cdot \mathbf{t})} \mathbf{s}_{34}
\end{aligned}
\end{equation} 
where $\mathbf{s}_{ij} = \mathbf{s}_i + \mathbf{s}_j$ and $\mathbf{t}_{ij} = \mathbf{t}_i + \mathbf{t}_j$. 


\subsection{Simplification to rhombi-slits} The examples in the main text focus on the special case of kirigami with rhombi-slits. Such patterns correspond to certain special choices of $\mathbf{s}_i,\mathbf{t}_i$:
\begin{equation}
    \begin{aligned}\label{eq:sVtVExplicit}
&2\mathbf{s}_1 = (1-\lambda_3) \mathbf{e}_1 -  a_r (1-\lambda_2) \mathbf{e}_2 , && 2\mathbf{s}_4 =(1-\lambda_3) \mathbf{e}_1 +  a_r (1-\lambda_2) \mathbf{e}_2 , \\
&2\mathbf{s}_2 =  \lambda_3 \mathbf{e}_1 + a_r \lambda_2 \mathbf{e}_2,  &&  2\mathbf{s}_3 =  \lambda_3 \mathbf{e}_1 - a_r \lambda_2 \mathbf{e}_2 , \\
&2\mathbf{t}_1 =  a_r \lambda_2 \mathbf{e}_2 + (1- \lambda_1) \mathbf{e}_1,    && 2\mathbf{t}_2 =  a_r \lambda_2 \mathbf{e}_2 - (1- \lambda_1) \mathbf{e}_1, \\
&2 \mathbf{t}_3 = a_r (1-\lambda_2) \mathbf{e}_2 + (1-\lambda_3) \mathbf{e}_1,   && 2\mathbf{t}_4 = a_r (1-\lambda_2) \mathbf{e}_2 - (1-\lambda_3) \mathbf{e}_1, 
    \end{aligned}
\end{equation}
for $\lambda_1,\ldots, \lambda_4$ taking values in $[0,1]$, and an aspect ratio $a_r >0$. We use $\mathbf{e}_1, \mathbf{e}_2$ for the standard Cartesian basis vectors on $\mathbb{R}^2$. These formulas follow by comparing the vectors $\mathbf{s}_1,\ldots, \mathbf{t}_4$ defining the unit cell in Fig.\;2 to the $(\lambda_1, \ldots, \lambda_4,a_r)$-parameterization of the rhombi-slit cell in Fig.\;3. Observe that $\mathbf{s} = \sum_{i=1,\ldots,4} \mathbf{s}_i = \mathbf{e}_1$ and $\mathbf{t} =\sum_{i=1,\ldots,4} \mathbf{t}_i = a_r \mathbf{e}_2$. The explicit formula for $\mathbf{A}(\xi)$ in Eq.\;(5) is obtained by substituting  (\ref{eq:sVtVExplicit}) into (\ref{eq:AxiExplicit}-\ref{eq:abShape}) and performing routine algebra. 

In fact, the formulas in (\ref{eq:sVtVExplicit}) describe \textit{all} kirigami patterns with a unit cell of four quad panels and four rhombi-slits. This follows from the argument in SM.1 (see Fig.\;\ref{fig:GeomArg}(a-c)), as we explain.  We desire rhombi-slits and each such slit will have only one free side length, rather than the two free lengths of a generic parallelogram. This restricts the seed: its second quad \textit{must} be a reflection of its first quad across the vector $\mathbf{W}\mathbf{t}$, to obtain rhombi slits. Since the seed quads are mirrored, the $180^{\circ}$ rotation from before is now a reflection across $\mathbf{t}$. Hence, $\mathbf{s}$ is parallel to $\mathbf{W} \mathbf{t}$, i.e., the lattice vectors $\mathbf{s}$ and $\mathbf{t}$ are orthogonal.  Without loss of generality, we can rotate and rescale the pattern to make $\mathbf{s}$ into $\mathbf{e}_1$. It follows that $\mathbf{t} = a_r \mathbf{e}_2$ for some $a_r >0$. We then obtain a unit cell exactly as described in Fig.\;3, consistent with (\ref{eq:sVtVExplicit}). 

\section{Coarse-graining soft modes}
In this section, we derive the effective PDE in Eq.\;(3) by constructing a general, locally mechanistic soft mode. The idea is to allow spatial variations in the mechanisms of the previous section; an effective deformation $\mathbf{y}_\text{eff}(\mathbf{x})$ emerges in the bulk. As stated in the main text, we find that $\mathbf{y}_\text{eff}(\mathbf{x})$ describes a soft mode when it solves the PDE. We enunciate a basic numerical procedure based on our construction to find the panel motions. It is used in the theory portions of Fig.\;4 of the main text. We end with a discussion of the discrepancies between theory and experiment in our approach.

\subsection{Closing the gaps in a soft mode}\label{s:Gaps}

We start by fixing our notation with reference to Fig.\;\ref{fig:Ansatz}. Fix a general periodic quad-kirigami pattern as in Section \ref{s:UC}, whose unit cell has four convex quad panels and four parallelogram slits. The system we study is made up of a large but finite number of kirigami unit cells selected from this pattern, within a two-dimensional reference domain $\Omega\subset\mathbb{R}^2$ (Fig.\;\ref{fig:Ansatz}(a)).
Non-dimensionalizing by the system size, we take our panels and unit cells to have (non-dimensional) widths $\sim \ell \ll 1$. Each cell is labeled by its lower left corner point, which we consistently write with an overbar, as in $\bar{\mathbf{x}}$. The four panels in the cell at $\bar{\mathbf{x}}$ are then labeled counter-clockwise as $\mathcal{P}_{1, \bar{\mathbf{x}}}^{(\ell)}, \ldots, \mathcal{P}_{4,\bar{\mathbf{x}}}^{(\ell)}$ (Fig.\;\ref{fig:Ansatz}(b)). 

We consider panel deformations made up of counter-rotations and translations in the form
\begin{equation}
\begin{aligned}\label{eq:ansatzRigid}
&\mathbf{y}_{i,\bar{\mathbf{x}}}^{(\ell)}(\mathbf{x}) = \mathbf{R}(\gamma(\bar{\mathbf{x}}) + \sigma_i \xi(\bar{\mathbf{x}})) (\mathbf{x} - \bar{\mathbf{x}}) + \mathbf{y}_{\text{eff}}(\bar{\mathbf{x}}) + \ell \mathbf{d}_i(\bar{\mathbf{x}}),\quad \mathbf{x} \in \mathcal{P}_{i,\bar{\mathbf{x}}}^{(\ell)}
\end{aligned}
\end{equation} 
for $i = 1,\ldots,4$, and where $\sigma_1 = \sigma_3 = 1$ and $\sigma_2 = \sigma_4 = -1$. The effective deformation $\mathbf{y}_\text{eff}(\mathbf{x})$ describes a macroscopic motion we think of as occurring in a smooth manner on the whole domain $\Omega$, even though it is only sampled at the cell reference points $\bar{\mathbf{x}}$. Similarly, we refer to macroscopic angle functions $(\gamma(\mathbf{x}), \xi(\mathbf{x}))$ and panel translation functions $\mathbf{d}_1(\mathbf{x}), \ldots, \mathbf{d}_4(\mathbf{x})$.   %occur with low energy panels of the pattern by iterating on this cell deformation. The right neighbor to the $\bar{\mathbf{x}}$-cell, for instance, is deformed rigidly by $\mathbf{y}^{(\ell)}_{i,\bar{\mathbf{x}}+ \ell \mathbf{s}}(\mathbf{x}), \mathbf{x} \in \mathcal{P}_{i,\bar{\mathbf{x}}+ \ell \mathbf{s}}^{(\ell)}$, $i = 1,\ldots, 4$, and so on. 
\begin{figure}[t]
\centering
\includegraphics[width =\linewidth]{Ansatz.png}
\caption{Notation for deriving the effective description and corresponding simulation method.}
\label{fig:Ansatz}
\end{figure}
In this ansatz, initially connected panels can and generically will have gaps open up between them as in Fig.\;\ref{fig:Ansatz}(c,f). We will choose the panel translations $\mathbf{d}_i(\mathbf{x})$ to close the gaps within each deformed unit cell, thus enforcing a locally mechanistic response. Then, the only gaps that occur are between neighboring unit cells (Fig.\;\ref{fig:Ansatz}(c)). As we shall see, these gaps are consistent with a soft mode (and, in particular, are of order $\sim\ell^2$) provided $\mathbf{y}_\text{eff}(\mathbf{x})$ solves the effective PDE.

To help organize the calculation, we label the corner points of the $\bar{\mathbf{x}}$-cell as $\mathbf{x}_{ij}^{(\ell)}(\bar{\mathbf{x}})$ (Fig.\;\ref{fig:Ansatz}(d)). We then follow the calculation from section \ref{sec:KinAppend} using the replacements 
$\mathbf{R}_i \mapsto \mathbf{R}_i(\bar{\mathbf{x}})$, $\mathbf{c}_i \mapsto \mathbf{y}_{\text{eff}}(\bar{\mathbf{x}}) - \mathbf{R}_i(\bar{\mathbf{x}})\bar{\mathbf{x}} + \ell \mathbf{d}_{i}(\bar{\mathbf{x}})$ and $\mathbf{x}_{ij} \mapsto \mathbf{x}_{ij}^{(\ell)}(\bar{\mathbf{x}})$ to compare (\ref{eq:ansatzRigid}) and (\ref{eq:Rigid0}).
Copying the result of (\ref{eq:c123}), we see that the $\bar{\mathbf{x}}$-cell remains gap-free if and only if 
\begin{equation}
\begin{aligned}\label{eq:ciResults}
\ell ( \mathbf{d}_1(\bar{\mathbf{x}}) - \mathbf{d}_4(\bar{\mathbf{x}})) = \Big( \mathbf{R}(\gamma(\bar{\mathbf{x}}) + \sigma_4 \xi(\bar{\mathbf{x}})) -  \mathbf{R}(\gamma(\bar{\mathbf{x}}) + \sigma_1 \xi(\bar{\mathbf{x}})) \Big) \big( \mathbf{x}_{41}^{(\ell)}(\bar{\mathbf{x}})  - \bar{\mathbf{x}}\big), \\
\ell ( \mathbf{d}_2(\bar{\mathbf{x}}) - \mathbf{d}_1(\bar{\mathbf{x}})) = \Big( \mathbf{R}(\gamma(\bar{\mathbf{x}}) + \sigma_1 \xi(\bar{\mathbf{x}})) -  \mathbf{R}(\gamma(\bar{\mathbf{x}}) + \sigma_2 \xi(\bar{\mathbf{x}})) \Big) \big( \mathbf{x}_{12}^{(\ell)}(\bar{\mathbf{x}})  - \bar{\mathbf{x}}\big), \\
\ell ( \mathbf{d}_3(\bar{\mathbf{x}}) - \mathbf{d}_2(\bar{\mathbf{x}})) = \Big( \mathbf{R}(\gamma(\bar{\mathbf{x}}) + \sigma_2 \xi(\bar{\mathbf{x}})) -  \mathbf{R}(\gamma(\bar{\mathbf{x}}) + \sigma_3 \xi(\bar{\mathbf{x}})) \Big) \big( \mathbf{x}_{32}^{(\ell)}(\bar{\mathbf{x}})  - \bar{\mathbf{x}}\big).
\end{aligned}
\end{equation}
There are six constraints for the eight degrees of freedom in question. Following Fig.\;\ref{fig:Ansatz}(d,e), we have 
\begin{equation}
\begin{aligned}\label{eq:xijParams}
\mathbf{x}^{(\ell)}_{41}(\bar{\mathbf{x}}) = \bar{\mathbf{x}} + \ell(\mathbf{t}_1 + \mathbf{t}_2 + \mathbf{t}_3 + \mathbf{s}_3), \quad \mathbf{x}^{(\ell)}_{12}(\bar{\mathbf{x}}) = \bar{\mathbf{x}} + \ell \mathbf{t}_1, \quad \mathbf{x}^{(\ell)}_{32}(\bar{\mathbf{x}}) = \bar{\mathbf{x}} + \ell ( \mathbf{s}_1 + \mathbf{s}_2).
\end{aligned}
\end{equation}
Combining (\ref{eq:ciResults}) and (\ref{eq:xijParams}) gives the constraints 
\begin{equation}
\begin{aligned}\label{eq:ciPrescribe}
&\mathbf{d}_1(\mathbf{x}) = \Big( \mathbf{R}(\gamma(\mathbf{x}) - \xi(\mathbf{x}) )-  \mathbf{R}(\gamma(\mathbf{x}) +  \xi(\mathbf{x})) \Big) \big( \mathbf{t}_1 + \mathbf{t}_2 + \mathbf{t}_3  + \mathbf{s}_3\big) + \mathbf{d}_4(\mathbf{x}), \\
&\mathbf{d}_2(\mathbf{x}) = \Big( \mathbf{R}(\gamma(\mathbf{x}) - \xi(\mathbf{x}) )-  \mathbf{R}(\gamma(\mathbf{x}) +  \xi(\mathbf{x})) \Big) \big(  \mathbf{t}_2 + \mathbf{t}_3 + \mathbf{s}_3\big) + \mathbf{d}_4(\mathbf{x}) , \\
&\mathbf{d}_3(\mathbf{x}) = \Big( \mathbf{R}(\gamma(\mathbf{x}) - \xi(\mathbf{x}) )-  \mathbf{R}(\gamma(\mathbf{x}) +  \xi(\mathbf{x})) \Big) \big(  \mathbf{t}_2 + \mathbf{t}_3 + \mathbf{s}_1 + \mathbf{s}_2 +  \mathbf{s}_3\big)  + \mathbf{d}_4(\mathbf{x}).
\end{aligned}
\end{equation}
Again, these equations ensure that the panels within each unit cell remain connected upon deformation. Enforcing them prescribes three of the panel translation fields $\mathbf{d}_i(\mathbf{x})$, but not all four.


Next, we attend to the gaps between unit cells, which are given by
\begin{equation}
\begin{aligned}\label{eq:gapsDef}
&\mathbf{g}_{14}^{(\ell)}(\bar{\mathbf{x}}) = \mathbf{y}^{(\ell)}_{1, \bar{\mathbf{x}} + \ell \mathbf{s}} (\mathbf{x}^{(\ell)}_{14}(\bar{\mathbf{x}}))  -  \mathbf{y}^{(\ell)}_{4, \bar{\mathbf{x}}} (\mathbf{x}^{(\ell)}_{14}(\bar{\mathbf{x}})), \quad 
\mathbf{g}_{23}^{(\ell)}(\bar{\mathbf{x}}) = \mathbf{y}^{(\ell)}_{2, \bar{\mathbf{x}} + \ell \mathbf{s}} (\mathbf{x}^{(\ell)}_{23}(\bar{\mathbf{x}}))  - \mathbf{y}^{(\ell)}_{3, \bar{\mathbf{x}}}(\mathbf{x}^{(\ell)}_{23}(\bar{\mathbf{x}})) , \\
&\mathbf{g}_{21}^{(\ell)}(\bar{\mathbf{x}}) = \mathbf{y}^{(\ell)}_{2, \bar{\mathbf{x}} + \ell \mathbf{t}} (\mathbf{x}^{(\ell)}_{21}(\bar{\mathbf{x}}))  -  \mathbf{y}^{(\ell)}_{1, \bar{\mathbf{x}}} (\mathbf{x}^{(\ell)}_{21}(\bar{\mathbf{x}})), \quad \mathbf{g}_{34}^{(\ell)}(\bar{\mathbf{x}}) = \mathbf{y}^{(\ell)}_{3, \bar{\mathbf{x}} + \ell \mathbf{t}} (\mathbf{x}^{(\ell)}_{34}(\bar{\mathbf{x}}))  -  \mathbf{y}^{(\ell)}_{4, \bar{\mathbf{x}}} (\mathbf{x}^{(\ell)}_{34}(\bar{\mathbf{x}}))
\end{aligned}
\end{equation}
as in Fig.\;\ref{fig:Ansatz}(c,d,f). An inspired Taylor expansion will lead to the PDE for $\mathbf{y}_\text{eff}(\mathbf{x})$. Recall the deformed $\bar{\mathbf{x}}$-cell remains connected, by (\ref{eq:ciPrescribe}). Copy it twice and translate the copies along $\ell \mathbf{R}(\gamma(\bar{\mathbf{x}})) \mathbf{A}(\xi(\bar{\mathbf{x}})) \mathbf{s}$ and $\ell \mathbf{R}(\gamma(\bar{\mathbf{x}})) \mathbf{A}(\xi(\bar{\mathbf{x}})) \mathbf{t}$ to produce a perfectly connected ``reference mechanism'' (recall in Fig.\;2, $\mathbf{s}_{\text{def}} = \mathbf{R}(\gamma) \mathbf{A}(\xi)\mathbf{s}$ and $\mathbf{t}_{\text{def}} = \mathbf{R}(\gamma) \mathbf{A}(\xi) \mathbf{t}$). Using this reference mechanism to compare, we see that 
\begin{equation}
\begin{aligned}\label{eq:firstAwesome}
&\mathbf{y}^{(\ell)}_{4, \bar{\mathbf{x}}} (\mathbf{x}^{(\ell)}_{14}(\bar{\mathbf{x}})) = \mathbf{y}^{(\ell)}_{1, \bar{\mathbf{x}}}( \mathbf{x}^{(\ell)}_{14}(\bar{\mathbf{x}})) + \ell  \mathbf{R}(\gamma(\bar{\mathbf{x}})) \mathbf{A}(\xi(\bar{\mathbf{x}})) \mathbf{s}, 
 &\mathbf{y}^{(\ell)}_{3, \bar{\mathbf{x}}} (\mathbf{x}^{(\ell)}_{23}(\bar{\mathbf{x}})) = \mathbf{y}^{(\ell)}_{2, \bar{\mathbf{x}}}(\mathbf{x}^{(\ell)}_{23}(\bar{\mathbf{x}})) + \ell  \mathbf{R}(\gamma(\bar{\mathbf{x}})) \mathbf{A}(\xi(\bar{\mathbf{x}})) \mathbf{s}, \\
 &\mathbf{y}^{(\ell)}_{1, \bar{\mathbf{x}}} (\mathbf{x}^{(\ell)}_{21}(\bar{\mathbf{x}})) = \mathbf{y}^{(\ell)}_{2, \bar{\mathbf{x}}}( \mathbf{x}^{(\ell)}_{21}(\bar{\mathbf{x}})) + \ell  \mathbf{R}(\gamma(\bar{\mathbf{x}})) \mathbf{A}(\xi(\bar{\mathbf{x}})) \mathbf{t}, 
 &\mathbf{y}^{(\ell)}_{4, \bar{\mathbf{x}}} (\mathbf{x}^{(\ell)}_{34}(\bar{\mathbf{x}})) = \mathbf{y}^{(\ell)}_{3, \bar{\mathbf{x}}}(\mathbf{x}^{(\ell)}_{34}(\bar{\mathbf{x}})) + \ell  \mathbf{R}(\gamma(\bar{\mathbf{x}})) \mathbf{A}(\xi(\bar{\mathbf{x}})) \mathbf{t}.
\end{aligned}
\end{equation}
On the other hand by Taylor expansion, 
\begin{equation}
\begin{aligned}\label{eq:secondAwesome}
\mathbf{y}^{(\ell)}_{i,\bar{\mathbf{x}} + \ell \mathbf{s}}(\mathbf{x}) = \mathbf{y}^{(\ell)}_{i,\bar{\mathbf{x}}}(\mathbf{x}) +  \ell \nabla \mathbf{y}_{\text{eff}}(\bar{\mathbf{x}}) \mathbf{s}  + O(\ell^2), \\
\mathbf{y}^{(\ell)}_{i,\bar{\mathbf{x}} + \ell \mathbf{t}}(\mathbf{x}) = \mathbf{y}^{(\ell)}_{i,\bar{\mathbf{x}}}(\mathbf{x}) +  \ell \nabla \mathbf{y}_{\text{eff}}(\bar{\mathbf{x}}) \mathbf{t}  + O(\ell^2)
\end{aligned}
\end{equation}
for any $\mathbf{x} \in \mathcal{P}_{i,\bar{\mathbf{x}}}^{(\ell)}$. Substituting (\ref{eq:firstAwesome}) and (\ref{eq:secondAwesome}) into (\ref{eq:gapsDef}), we conclude the following formulas for the gaps between unit cells:  
\begin{equation}
\begin{aligned}\label{eq:gapsCalc}
&\mathbf{g}_{14}^{(\ell)}(\bar{\mathbf{x}})  = \ell\big(\nabla \mathbf{y}_{\text{eff}}(\bar{\mathbf{x}}) - \mathbf{R}(\gamma(\bar{\mathbf{x}})) \mathbf{A}(\xi(\bar{\mathbf{x}}))  \big) \mathbf{s}  + O(\ell^2), 
&\mathbf{g}_{23}^{(\ell)}(\bar{\mathbf{x}})  = \ell\big(\nabla \mathbf{y}_{\text{eff}}(\bar{\mathbf{x}}) - \mathbf{R}(\gamma(\bar{\mathbf{x}})) \mathbf{A}(\xi(\bar{\mathbf{x}}))  \big) \mathbf{s}  + O(\ell^2), \\
&\mathbf{g}_{21}^{(\ell)}(\bar{\mathbf{x}})  = \ell\big(\nabla \mathbf{y}_{\text{eff}}(\bar{\mathbf{x}}) - \mathbf{R}(\gamma(\bar{\mathbf{x}})) \mathbf{A}(\xi(\bar{\mathbf{x}}))  \big) \mathbf{t}  + O(\ell^2), 
&\mathbf{g}_{34}^{(\ell)}(\bar{\mathbf{x}})  = \ell\big(\nabla \mathbf{y}_{\text{eff}}(\bar{\mathbf{x}}) - \mathbf{R}(\gamma(\bar{\mathbf{x}})) \mathbf{A}(\xi(\bar{\mathbf{x}}))  \big) \mathbf{t}  + O(\ell^2).
\end{aligned}
\end{equation}
These formulas prescribe the gaps between the unit cell at $\bar{\mathbf{x}}$ and its neighbors above and to the right (Fig.\;\ref{fig:Ansatz}(c)). Letting $\bar{\mathbf{x}}$ vary throughout the unit cells recovers all the gaps. Thus, the gaps in our prescription (\ref{eq:ansatzRigid}) are of order $\sim \ell^2$ if and only if the effective PDE holds:
\begin{equation}
\begin{aligned}\label{eq:effectDescribe}
\nabla \mathbf{y}_{\text{eff}}(\mathbf{x}) = \mathbf{R}(\gamma(\mathbf{x})) \mathbf{A}(\xi(\mathbf{x})).
\end{aligned}
\end{equation}

\subsection{A numerical procedure for plotting panel motions}\label{s:Simulations}
The preceding description can be used to visualize the panel motions of a soft mode. Given a solution $(\mathbf{y}_{\text{eff}}(\mathbf{x}), \gamma(\mathbf{x}), \xi(\mathbf{x}))$ of the effective PDE (\ref{eq:effectDescribe}), we substitute it and the formulas for each $\mathbf{d}_i(\mathbf{x})$ from (\ref{eq:ciPrescribe}) into the rigid panel motion ansatz from (\ref{eq:ansatzRigid}). To generate the theoretical results in Fig.\;4, we take $\mathbf{d}_4(\mathbf{x}) = \mathbf{0}$, though in general this choice could be optimized to improve the gaps at order $\sim \ell^2$. (That this optimization does not improve the gaps at a lower order follows from (\ref{eq:gapsCalc}).)  By this choice, each panel motion $\mathbf{y}_{i,\bar{\mathbf{x}}}^{(\ell)}(\mathbf{x})$ in (\ref{eq:effectDescribe}) is fully assigned: we achieve a recipe for deforming all panels in the pattern by rotations and translations approximating the effective deformation $\mathbf{y}_{\text{eff}}(\mathbf{x})$. The gaps between unit cells are at most $O(\ell^2)$, and are negligible compared to the unit cell widths in the limit of a large number of cells. %So, this procedure provides appropriate approximations of the deformations of finely patterned kirigami. 


\subsection{Elastic energy}

Having evaluated the gaps in our ansatz for the panel motions, we now determine when the construction yields a \textit{soft mode}, which by definition has stored elastic energy much smaller than bulk elasticity scaling. We do so by introducing a simple spring model for the energy, and coarse-graining it in the doubly asymptotic limit mentioned in the main text.  The effective PDE (\ref{eq:effectDescribe}) emerges to set the bulk energy; we conclude it must  be solved for our ansatz to generate a soft mode.

Consider the ansatz in (\ref{eq:ansatzRigid}-\ref{eq:gapsCalc}), this time without necessarily enforcing the PDE for $\mathbf{y}_\text{eff}(\mathbf{x})$. While the construction is purely geometric, one can enrich it with  distortions that \textit{elastically close the gaps}. Two sources of energy result: the panels and hinges (i) stretch to preserve the pattern's topology, and (ii) bend to account for the counter-rotations. Adopting a simple elastic spring model as in \cite{Deng2020}, we define the energy
\begin{equation}
    \begin{aligned}
    \mathcal{E}_{\text{SM}}^{(\ell,\delta)} = \sum_{\bar{\mathbf{x}}} \sum_{i\sim j} \frac{1}{2}E \Big( |\mathbf{g}_{ij}^{(\ell)}(\bar{\mathbf{x}})|^2  + \delta^2 w_{ij}(\xi(\bar{\mathbf{x}})) \Big)  
    \end{aligned}
\end{equation}
under the rigid ansatz (\ref{eq:ansatzRigid}). The first term describes a linear spring with zero rest length linking the corner points of adjacent panels; it accounts for the forces needed to close gaps of length $|\mathbf{g}_{ij}^{(\ell)}(\bar{\mathbf{x}})|$.  The second term describes a (possibly nonlinear) torsional spring energy $w_{ij}(\xi)$, and accounts for the bending moments in the hinges.  Per Fig.\;\ref{fig:Fab}(b), the experimental patterns come with hinges of length $d \approx (1/10)\ell$ and height $h \approx (1/40)\ell$; in the model, we lump these dimensions into a single characteristic length $\delta \ll \ell$ and take the hinge energy to be $\sim \delta^2$. Consistent with our experiments in which the panels and hinges are made of the same material, we let $E >0$ denote a single characteristic elastic modulus. As usual, we use $\bar{\mathbf{x}}$ to denote a choice of unit cell in the  reference domain $\Omega$. The inner sum over $i\sim j$ then collects the spring energies associated to the eight vertices of the $\bar{\mathbf{x}}$-cell shown in Fig.\;\ref{fig:Ansatz}(d). 

Having defined an energy, we coarse-grain it in a limit 
\begin{equation}
    \begin{aligned}\label{eq:asymtotics}
    \ell \rightarrow 0 \quad \text{with $\delta \equiv \delta(\ell)$ chosen to satisfy} \quad \frac{\delta(\ell)}{\ell} \rightarrow 0.
    \end{aligned}
\end{equation}
This requires that the hinges remain asymptotically negligible in size relative to the panels, while the number of panels grows. We additionally assume that the quantities $\mathbf{y}_{\text{eff}}(\mathbf{x}), \gamma(\mathbf{x})$ and $\xi(\mathbf{x})$ from our ansatz are chosen independently of $\ell$ (see, however, \ref{sec:Comparison} for a brief discussion of how to relax this constraint). By the formulas for the gaps in (\ref{eq:gapsCalc}) and the definitions 
\begin{equation}
   w(\xi):= \sum_{i \sim j} w_{ij}(\xi)  \quad\text{and}\quad Q_{\text{bulk}}(\mathbf{G}) :=2\Tr\Big( \mathbf{G}^T\mathbf{G} ( \mathbf{s} \otimes \mathbf{s} + \mathbf{t} \otimes \mathbf{t})\Big),
\end{equation} 
we have that
\begin{equation}
    \begin{aligned}
    \mathcal{E}_{\text{SM}}^{(\ell,\delta(\ell))} &=\sum_{\bar{\mathbf{x}}}E \Big(|\mathbf{g}_{21}^{(\ell)}(\bar{\mathbf{x}})|^2 + |\mathbf{g}_{34}^{(\ell)}(\bar{\mathbf{x}})|^2  + |\mathbf{g}_{14}^{(\ell)}(\bar{\mathbf{x}})|^2 + |\mathbf{g}_{23}^{(\ell)}(\bar{\mathbf{x}})|^2 + \delta^2 w(\xi(\bar{\mathbf{x}})) \Big)  \\
    &= \sum_{\bar{\mathbf{x}}}E \Big\{2\ell^2  \sum_{\mathbf{v} \in \{\mathbf{s},\mathbf{t}\}}|\big(\nabla \mathbf{y}_{\text{eff}}(\bar{\mathbf{x}}) - \mathbf{R}(\gamma(\bar{\mathbf{x}})) \mathbf{A}(\xi(\bar{\mathbf{x}}))  \big) \mathbf{v} + O(\ell)|^2  + \delta^2 w(\xi(\bar{\mathbf{x}})) \Big\} \\
    &=\sum_{\bar{\mathbf{x}}} \ell^2 E \Big\{  Q_{\text{bulk}}\Big(\nabla \mathbf{y}_{\text{eff}}(\bar{\mathbf{x}}) - \mathbf{R}(\gamma(\bar{\mathbf{x}})) \mathbf{A}(\xi(\bar{\mathbf{x}}))  \Big) + \frac{\delta^2}{\ell^2}w(\xi(\bar{\mathbf{x}})) + O(\ell) \Big\}.
    \end{aligned}
\end{equation}
Recognize, in view of Fig.\;\ref{fig:Ansatz}(a-b), that each unit cell covers an area  $\ell^2 |\mathbf{s} \cdot \mathbf{W} \mathbf{t}|$. Applying the Riemann integration formula   $\sum_{\bar{\mathbf{x}}}\ell^2 f(\bar{\mathbf{x}}) \to \frac{1}{|\mathbf{s} \cdot \mathbf{W} \mathbf{t}|}\int_{\Omega} f(\mathbf{x})\, dA$ yields for the coarse-grained energy
\begin{equation}
    \begin{aligned}\label{eq:takeLimit}
    \lim_{\ell \rightarrow 0}\, \mathcal{E}_{\text{SM}}^{(\ell,\delta(\ell))} &= \lim_{\ell \rightarrow 0}\, \frac{E}{|\mathbf{s} \cdot \mathbf{W}\mathbf{t}|} \int_{\Omega}  \Big\{ Q_{\text{bulk}}\Big(\nabla \mathbf{y}_{\text{eff}}(\mathbf{x}) - \mathbf{R}(\gamma(\mathbf{x})) \mathbf{A}(\xi(\mathbf{x}))  \Big) + \frac{\delta(\ell)^2}{\ell^2} w(\xi(\mathbf{x})) +O(\ell)\Big\}\, dA  \\
    &= \frac{E}{|\mathbf{s} \cdot \mathbf{W}\mathbf{t}|}\int_{\Omega} Q_{\text{bulk}}\Big(\nabla \mathbf{y}_{\text{eff}}(\mathbf{x}) - \mathbf{R}(\gamma(\mathbf{x})) \mathbf{A}(\xi(\mathbf{x}))  \Big)\,dA.
    \end{aligned}
\end{equation}

The bulk elastic energy just derived vanishes if and only if the effective PDE (\ref{eq:effectDescribe}) holds.  This explains why the effective PDE captures the soft modes of our kirigami patterns. When specialized to conformal kirigami, (\ref{eq:takeLimit}) recovers the bulk part of the elastic energy from \cite{czajkowski2022conformal}, up to a choice of elastic moduli. %The higher order effects in their theory are, however, lost in our result due to the coarse-graining.


\subsection{Comparing experimental soft modes to the coarse-grained theory}\label{sec:Comparison}
The effective PDE  (\ref{eq:effectDescribe}) is fundamentally an asymptotic constraint. It describes the leading order behaviors of soft kirigami modes in the limit of a large number of panels and with negligible hinges (per (\ref{eq:asymtotics})). At small but finite $\ell$ and $\delta/\ell$, the actual deformation of the kirigami can be expected to exhibit slight deviations from our ansatz. We address these deviations now. %soft modes associated to any physical system with small, but finite, . 

Consider, for instance, a refinement of the panel motions in the ansatz (\ref{eq:ansatzRigid}) obtained by replacing the $\ell$-independent fields $(\mathbf{y}_{\text{eff}}(\mathbf{x}) , \gamma(\mathbf{x}), \xi(\mathbf{x}))$ with ones that are allowed to depend on the number of panels and perhaps even the hinge sizes:
\begin{equation}
    \begin{aligned}\label{eq:perturb}
    \mathbf{y}_{\text{eff}}(\mathbf{x}) &\mapsto \mathbf{y}_{\text{eff}}^{(\ell,\delta)}(\mathbf{x}) = \mathbf{y}_{0}(\mathbf{x}) + \ell \mathbf{y}_1(\mathbf{x}) + \frac{\delta^2}{\ell^2} \mathbf{y}_2(\mathbf{x}),  \\
    \mathbf{\gamma}(\mathbf{x}) &\mapsto \gamma^{(\ell,\delta)}(\mathbf{x}) = \gamma_0(\mathbf{x}) + \ell \gamma_1(\mathbf{x}) + \frac{\delta^2}{\ell^2} \gamma_2(\mathbf{x}), \\ 
    \xi(\mathbf{x}) &\mapsto \xi^{(\delta,\ell)}(\mathbf{x}) = \xi_0(\mathbf{x}) + \ell \xi_1(\mathbf{x}) + \frac{\delta^2}{\ell^2} \xi_2(\mathbf{x}).
    \end{aligned}
\end{equation}
After updating the previous asymptotic analysis to account for these replacements, one finds a coarse-grained energy nearly identical to that of (\ref{eq:takeLimit}), the only difference being that the term inside $Q_{\text{bulk}}(\cdot)$ is replaced by $\nabla \mathbf{y}_{0}(\mathbf{x}) - \mathbf{R}(\gamma_0(\mathbf{x})) \mathbf{A}(\xi_0(\mathbf{x})).$ Setting $\nabla \mathbf{y}_{0}(\mathbf{x}) = \mathbf{R}(\gamma_0(\mathbf{x})) \mathbf{A}(\xi_0(\mathbf{x}))$ to produce a soft mode leaves room for perturbations in the effective description which now scale with the parameters:
\begin{equation}
\begin{aligned}\label{eq:PDEApprox}
\Big | \nabla \mathbf{y}_{\text{eff}}^{(\ell,\delta)}(\mathbf{x}) - \mathbf{R}(\gamma^{(\ell,\delta)}(\mathbf{x})) \mathbf{A}(\xi^{(\ell,\delta)}(\mathbf{x}))\Big | \sim \max \Big\{ \ell, \frac{\delta^2}{\ell^2} \Big\}.
\end{aligned}
\end{equation}
The choice of the remaining perturbations in (\ref{eq:perturb}) is a delicate matter involving minimizing the residual elastic energy of a general  pattern. The  derivation of this energy is the subject of ongoing research; for more in the case of conformal kirigami, see \cite{czajkowski2022conformal}. 
%Physically, the $\ell$-perturbation slightly relaxes strain-gradient elasticity, whereas the $\delta^2/\ell^2$-perturbation slightly relaxes hinge bending. 
  %A similar analysis for the general kirigami presented here would be interesting, though it is not an easy task and well beyond the scope of this work. 

Even without settling the exact form of the residual energy, we can still address the order of magnitude of the discrepancies between our experiments and theory. The patterns in Fig.\;4 are $16\times16$ unit cells, giving $\ell \sim 1/16\mbox{-}1/32$. Per Fig.\;\ref{fig:Fab},  $\delta^2/\ell^2 \sim \hat{d}\cdot \hat{h} = 1/400$. So, $\ell \gg \delta^2/\ell^2$ in our samples.  By (\ref{eq:PDEApprox}), the deformation gradients in the experiments can reasonably deviate from the effective PDE constraint (\ref{eq:effectDescribe}) on the order of $3\mbox{-}6\%$ of the magnitude of the field variables $\mathbf{y}_{\text{eff}}(\mathbf{x}), \gamma(\mathbf{x})$ and $\xi(\mathbf{x})$ used in our simulations. This expectation is  consistent with the results   in the main text. 




 



%\textit{Bulk elasticity dominates --}\;Now, imagine a many-celled kirigami pattern that is loaded into the fully nonlinear regime and approximating some smooth effective deformation $\mathbf{y}_{\text{eff}}(\mathbf{x})$  (e.g., all experimental examples shown in the main text).  Given the analysis above, we see that the bulk elastic term inside the bracket is $O(1)$ unless there is a pair of angle fields $\gamma(\mathbf{x})$ and $\xi(\mathbf{x})$ that satisfy $\nabla \mathbf{y}_{\text{eff}}(\mathbf{x}) \approx \mathbf{R}(\gamma(\mathbf{x})) \mathbf{A}(\xi(\mathbf{x}))$ on most of $\Omega$ and are consistent with the loading.  Conversely, the hinge energy on such loading is always $O(\delta^2/\ell^2) \ll O(1).$  So the bulk term dominates, and, as such, is always driving the response of the kirigami pattern to (approximately) satisfy the constrained theory in (\ref{eq:effectDescribe}), which is Eq.\;(3) in the main text (with Eq.\;(4) is implied). 

%\textit{Mechansims are not always best --}\;Next, imagine loading a many-celled kirigami in a manner consistent with a mechanism deformation, i.e., a homogeneous effective deformation with gradient $\mathbf{F}_{\text{eff}} = \mathbf{A}(\xi_{0})$.  Note that this deformation actuates all the hinges uniformly. However, the hinges typically prefer to be less actuated or not actuated at all if possible; that is, one expects $w(\xi)$ to be minimized at $\xi = 0$ and monotonically increasing as $|\xi|$ increases in a typical examples. So if there are other exotic soft modes that  1) solve the effective theory, 2) are consistent with the loading, and 3) relieve slit actuation, then these modes will reduce the hinge energy while keeping the bulk energy zero. Thus, if the reduction in hinge energy -- likely $O(E\delta^2/\ell^2)$ -- dominates the $O(E \ell)$ term in the energy in (\ref{eq:secondEnergy}), then  mechanism deformation is not the preferred state. 

%\textit{Deviations from the effective theory --}\;Finally, we are reminded that the exact effective description in (\ref{eq:effectDescribe}) is  the collection of smooth fields $(\mathbf{y}_{\text{eff}}(\mathbf{x}), \gamma(\mathbf{x}), \xi(\mathbf{x}))$ that make the bulk elasticity in (\ref{eq:secondEnergy}) vanish identically. In comparing this theory with experiments, we expect some slight deviation. Below, we argue that the slit actuation should be slightly muted compared to the exact effective theory due to hinge elasticity. Consider, for instance, a solution to the effective theory $(\mathbf{y}_{\text{eff}}(\mathbf{x}), \gamma(\mathbf{x}), \xi(\mathbf{x}))$ but add a small perturbation $\xi_{\text{rel}}(\mathbf{x}) = \xi(\mathbf{x}) + \delta \xi(\mathbf{x})$ to the slit actuation. Then, the energy becomes
%\begin{equation}
  

 
 


\section{The compatibility condition for effective deformations}

The previous sections concerned the derivation of the effective PDE
\begin{equation}\label{eq:effPDE}
    \nabla \mathbf{y}_{\text{eff}}(\mathbf{x}) = \mathbf{R}(\gamma(\mathbf{x})) \mathbf{A}(\xi(\mathbf{x}))
\end{equation}
relating $\mathbf{y}_\text{eff}(\mathbf{x})$ to the angle fields $(\gamma(\mathbf{x}),\xi(\mathbf{x}))$. The rest of this supplement is about the analysis of this PDE and its solutions. We start in this section by deriving the compatibility condition 
\begin{equation}\label{eq:compat1}
\nabla \gamma(\mathbf{x}) = \boldsymbol{\Gamma}(\xi(\mathbf{x})) \nabla \xi(\mathbf{x})  
\end{equation}
as a consequence of (\ref{eq:effPDE}). Actually, on simply connected domains $\Omega$, this condition is not only necessary but also sufficient for the fields $(\gamma(\mathbf{x}),\xi(\mathbf{x}))$ to admit some effective deformation $\mathbf{y}_\text{eff}(\mathbf{x})$ solving (\ref{eq:effPDE}). Indeed, we shall derive it by enforcing the curl-free nature of $\nabla\mathbf{y}_\text{eff}(\mathbf{x})$. Sufficiency follows since a curl-free tensor field on a simply connected domain is always the gradient of a vector field.  %the curl-free condition on the effective description leading to [cite main text eqn]. 

Let $\mathbf{y}_{\text{eff}}(\mathbf{x})$, $\gamma(\mathbf{x})$ and $\xi(\mathbf{x})$ satisfy the effective PDE (\ref{eq:effPDE}) on $\Omega\subset\mathbb{R}^2$. Since partial derivatives  commute, i.e., $\partial_1\partial_2 \mathbf{y}_{\text{eff}}(\mathbf{x}) = \partial_2 \partial_1 \mathbf{y}_{\text{eff}}(\mathbf{x})$,
\begin{equation}
\begin{aligned}\label{eq:curlFree}
\partial_1\Big( \mathbf{R}(\gamma(\mathbf{x})) \mathbf{A}(\xi(\mathbf{x})) \mathbf{e}_2\Big) - \partial_2 \Big( \mathbf{R}(\gamma(\mathbf{x})) \mathbf{A}(\xi(\mathbf{x})) \mathbf{e}_1 \Big) = \mathbf{0}.
\end{aligned}
\end{equation}
Using the chain and product rules, there follows
\begin{equation}
\begin{aligned}\label{eq:chainRule}
\mathbf{R}(\gamma(\mathbf{x})) \Big[ \partial_1 \gamma(\mathbf{x}) \mathbf{W}  \mathbf{A}(\xi(\mathbf{x})) \mathbf{e}_2 +  \partial_1 \xi(\mathbf{x})   \mathbf{A}'(\xi(\mathbf{x})) \mathbf{e}_2 -  \partial_2 \gamma(\mathbf{x}) \mathbf{W}  \mathbf{A}(\xi(\mathbf{x})) \mathbf{e}_1 -   \partial_2 \xi(\mathbf{x})   \mathbf{A}'(\xi(\mathbf{x})) \mathbf{e}_1\Big]  = \mathbf{0}
\end{aligned}
\end{equation}
where $\mathbf{W} = \mathbf{R}(\pi/2)$. Note  $\partial_1 \gamma(\mathbf{x}) \mathbf{e}_2 - \partial_2 \gamma(\mathbf{x}) \mathbf{e}_1 = \mathbf{W} \nabla \gamma(\mathbf{x})$ and  $\partial_1 \xi(\mathbf{x}) \mathbf{e}_2 - \partial_2 \xi(\mathbf{x}) \mathbf{e}_1 = \mathbf{W} \nabla \xi(\mathbf{x})$.  Hence, (\ref{eq:chainRule}) re-writes as 
\begin{equation}
\begin{aligned}\label{eq:firstRewrite}
\mathbf{W} \mathbf{A}(\xi(\mathbf{x})) \mathbf{W} \nabla \gamma(\mathbf{x}) + \mathbf{A}'(\xi(\mathbf{x})) \mathbf{W} \nabla \xi(\mathbf{x}) = \mathbf{0},
\end{aligned}
\end{equation}
after pre-multiplying by $\mathbf{R}^T(\gamma(\mathbf{x}))$. Writing $\mathbf{W} \mathbf{A}(\xi) \mathbf{W}$ in the Cartesian basis shows that 
\begin{equation}
    \begin{aligned}\label{eq:coolIdent}
    -\mathbf{W} \mathbf{A}(\xi) \mathbf{W} = -\left(\begin{array}{cc} 0 & -1 \\ 
    1 & 0  \end{array}\right) \left(\begin{array}{cc} A_{11}(\xi) & A_{12}(\xi) \\ 
    A_{21}(\xi) & A_{22}(\xi)  \end{array}\right)\left(\begin{array}{cc} 0 & -1 \\ 
    1 & 0  \end{array}\right) = \left(\begin{array}{cc} A_{22}(\xi) & -A_{21}(\xi) \\ 
    -A_{12}(\xi) & A_{11}(\xi)  \end{array}\right) := \text{cof} \mathbf{A}(\xi)
    \end{aligned}
\end{equation}
for the \textit{cofactor} matrix of $\mathbf{A}(\xi)$.
Of course, $\det \nabla \mathbf{y}_{\text{eff}}(\mathbf{x})  = \det \mathbf{A}(\xi(\mathbf{x}))>0$ in all physically relevant cases. Then,  $\text{cof }\mathbf{A}(\xi) = \big(\det \mathbf{A}(\xi) \big) \mathbf{A}^{-T}(\xi)$, where $\mathbf{A}^{-T}(\xi)$ is the well-defined inverse transpose of $\mathbf{A}(\xi)$. By making use of all these identities, the equation in (\ref{eq:firstRewrite}) becomes
\begin{equation}
\begin{aligned}\label{eq:finalCompat}
\nabla \gamma(\mathbf{x}) =\underbrace{\Big[  \frac{\mathbf{A}^T(\xi(\mathbf{x})) \mathbf{A}'(\xi(\mathbf{x})) \mathbf{W}}{\det \mathbf{A}(\xi(\mathbf{x}))} \Big]}_{:= \boldsymbol{\Gamma}(\xi(\mathbf{x}))} \nabla \xi(\mathbf{x}),
\end{aligned}
\end{equation}
which is the desired compatibility condition (\ref{eq:compat1}).


%{\color{red}\textit{Interpretation as the fitting condition for a locally mechanistic response --} To get a sense of the physical nature of this compatibility condition, consider any kirigami pattern from Section \ref{s:UC} with many cells filling a planar reference domain $\Omega$, \textit{but} with the unit cells disconnected from one another. Note that one can sample from a smooth function $\xi(\mathbf{x})$ on $\Omega$ to open up the slits of each unit cell by a pure mechanism deformation. The question is whether this  actuation is amenable to fitting a locally mechanistic response: Precisely, can one coordinate relative rotations between all the actuated cells so that the gaps between neighboring cells are small relative to their size? Eq.\;(\ref{eq:finalCompat}) is exactly the fitting rule. It says that $\xi(\mathbf{x})$ is amenable to such fitting if $\boldsymbol{\Gamma}(\xi(\mathbf{x})) \nabla \xi(\mathbf{x})$ is the gradient of a scalar field. In such instances, $\nabla \gamma(\mathbf{x}) = \boldsymbol{\Gamma}(\xi(\mathbf{x})) \nabla \xi(\mathbf{x})$ is the sampling rule for the relative rotations between the cells.  } 


%We have just derived that, for sufficiently smooth $\mathbf{y}_{\text{eff}}(\mathbf{x})$, $\gamma(\mathbf{x})$ and $\xi(\mathbf{x})$ that satisfy $\nabla \mathbf{y}_{\text{eff}}(\mathbf{x}) = \mathbf{R}(\gamma(\mathbf{x})) \mathbf{A}(\xi(\mathbf{x}))$ and $\det \nabla \mathbf{y}_{\text{eff}}(\mathbf{x}) > 0$, the angles necessarily satisfy the PDE system in (\ref{eq:finalCompat}). Going the other direction, we assume that (\ref{eq:finalCompat}) holds on a simply connected domain $\Omega$ and is physical in the sense that $\det \mathbf{A}(\xi(\mathbf{x})) > 0$ for the solution.  It then follows that (\ref{eq:curlFree}) holds, i.e., that  $\mathbf{R}(\gamma(\mathbf{x})) \mathbf{A}(\xi(\mathbf{x}))$ is curl-free on this domain. By standard results in the theory of PDEs and continuum mechanics, this curl-free condition implies the existence of a vector field $\mathbf{y}_{\text{eff}}(\mathbf{x})$ such that $\nabla \mathbf{y}_{\text{eff}}(\mathbf{x}) = \mathbf{R}(\gamma(\mathbf{x})) \mathbf{A}(\xi(\mathbf{x}))$ on $\Omega$.  Moreover, $\mathbf{y}_{\text{eff}}(\mathbf{x})$ is unique up to a translation. 

%In summary, solving the compatibility condition in (\ref{eq:finalCompat}) is both necessary and sufficient for the existence of sufficiently smooth effective deformations that describe the kirigami.

\section{PDE type and a universal link to the Poisson's ratio}\label{s:PDEClass}

Here, we classify the general quad-based kirigami patterns from the main text using the concept of PDE type. In particular, we find the type of the compatibility relations relating the angle fields:
\begin{equation}\label{eq:forPDEType}
\nabla \gamma(\mathbf{x}) = \boldsymbol{\Gamma}(\xi(\mathbf{x})) \nabla \xi(\mathbf{x}).  
\end{equation}
Since we are in two dimensions, this system comprises two first order scalar PDEs in the unknowns $(\gamma(\mathbf{x}),\xi(\mathbf{x}))$. It is therefore equivalent to a single second order nonlinear scalar PDE in, say, $\xi(\mathbf{x})$. Following the standard procedure, we determine the type of this equation as \textit{elliptic}, \textit{hyperbolic}, or \textit{parabolic} depending on the coefficients of its linearization about a solution \cite[Ch.\;III; Sec.\;1.3]{courant2008methods}. 

Note in the examples from Fig.\;4, the PDE type remains constant (it is either elliptic or hyperbolic). In general, however, the type can vary across  $\Omega$. We first illustrate this procedure on a mechanism deformation, and then discuss general soft modes. We end by finding the link between the PDE type of (\ref{eq:forPDEType}) and the effective Poisson's ratio of our kirigami, in the general case stated in the main text.

\subsection{Definition of  PDE type}
As recalled above, the PDE type of (\ref{eq:forPDEType}) is defined through  linearization. To help explain this procedure, we first demonstrate it on a pure mechanism. Then, we obtain the type for a general soft mode.

Consider a mechanism deformation given by the constant angle functions $(\gamma_0(\mathbf{x}),\xi_0(\mathbf{x}))\equiv(\gamma_0,\xi_0)$. Substituting the perturbations $\gamma(\mathbf{x}) =  \gamma_0 + \delta \gamma(\mathbf{x})$ and $\xi(\mathbf{x}) = \xi_0 + \delta \xi(\mathbf{x})$ into (\ref{eq:forPDEType}) and collecting terms at leading order yields the linear system
\begin{equation}
    \begin{aligned}
    \nabla \delta \gamma(\mathbf{x}) = \boldsymbol{\Gamma}(\xi_0) \nabla \delta \xi(\mathbf{x}).
    \end{aligned}
\end{equation}
Taking the curl of the lefthand side to eliminate $\delta\gamma(\mathbf{x})$ gives a second order scalar PDE for $\delta\xi(\mathbf{x})$:
\begin{equation}
    \begin{aligned}\label{eq:linearPDESupp}
    \nabla \cdot \Big( \mathbf{W} \boldsymbol{\Gamma}(\xi_0) \nabla \delta \xi(\mathbf{x})\Big)= 0.
    \end{aligned}
\end{equation}
Introducing  $\text{sym}\, \mathbf{A} = \frac{1}{2}(\mathbf{A} + \mathbf{A}^T)$, this becomes 
\begin{equation}
    \begin{aligned}\label{eq:linPDE}
    \big(\sym (\mathbf{W}\boldsymbol{\Gamma}(\xi_0))\big)_{ij} \partial_i \partial_j \delta \xi(\mathbf{x}) = 0
    \end{aligned}
\end{equation}
in index notation with repeated indices summed. Eq.\;(\ref{eq:linPDE}) gives the linearized PDE in standard form. 
Its type is determined via the discriminant of its coefficients \cite{courant2008methods}. Equivalently, it is determined by 
\begin{equation}
    \begin{aligned}\label{eq:sigmaPDE}
    \sigma_{\text{PDE}}(\xi_0) := \det\Big( \text{sym} \big( \mathbf{W} \boldsymbol{\Gamma}(\xi_0)\big)\Big).
    \end{aligned}
\end{equation}
If $\sigma_{\text{PDE}}(\xi_0) >0$, the PDE is said to be of \textit{elliptic} type; if $\sigma_{\text{PDE}}(\xi_0) <0$, it is called \textit{hyperbolic};  if $\sigma_{\text{PDE}}(\xi_0) =0$, it is parabolic. While this classification is made in reference to the linearized PDE (\ref{eq:linPDE}), it is understood to apply to the nonlinear PDE (\ref{eq:forPDEType}) as well.

Returning to generalities, we now linearize the  system (\ref{eq:forPDEType}) about an arbitrary solution $(\gamma_0(\mathbf{x}), \xi_0(\mathbf{x}))$ and determine its type. Taking $\gamma(\mathbf{x}) = \gamma_0(\mathbf{x}) + \delta \gamma(\mathbf{x})$ and $\xi(\mathbf{x}) = \xi_0(\mathbf{x}) + \delta \xi(\mathbf{x})$ gives the linear system
\begin{equation}
    \begin{aligned}\label{eq:linPDE2}
    \nabla \delta \gamma(\mathbf{x}) = \boldsymbol{\Gamma}(\xi_0(\mathbf{x})) \nabla \delta \xi(\mathbf{x}) + \delta \xi(\mathbf{x}) \boldsymbol{\Gamma}'(\xi_0(\mathbf{x})) \nabla \xi_0 (\mathbf{x}) 
    \end{aligned}
\end{equation}
for the angle perturbations $(\delta\gamma(\mathbf{x}),\delta\xi(\mathbf{x}))$. Its highest order terms involve their gradients, and are  the  spatially-varying versions of the analogous terms in (\ref{eq:linPDE}). 
%Again, this PDE can be solved for $\delta \gamma(\mathbf{x})$ if and only if the righthand side is curl-free:
%\begin{equation}
%    \begin{aligned}\label{eq:getFirstLine}
%    &\mathbf{e}_1 \cdot \boldsymbol{\Gamma}(\xi_0(\mathbf{x})) \partial_2 \nabla \delta \xi(\mathbf{x}) - \mathbf{e}_2 \cdot %\boldsymbol{\Gamma}(\xi_0(\mathbf{x})) \partial_1 \nabla \delta \xi(\mathbf{x})  \\
%    &\quad + \mathbf{e}_1 \cdot \partial_2\Big(\boldsymbol{\Gamma}(\xi_0(\mathbf{x}))\Big) \nabla \delta \xi(\mathbf{x}) -  %\mathbf{e}_2 \cdot \partial_1\Big(\boldsymbol{\Gamma}(\xi_0(\mathbf{x}))\Big) \nabla \delta \xi(\mathbf{x})  \\
%    &\quad + \mathbf{e}_1 \cdot \partial_2 \big( \delta \xi(\mathbf{x}) \boldsymbol{\Gamma}'(\xi_0(\mathbf{x})) \nabla \xi_0 %(\mathbf{x}) \big) - \mathbf{e}_2 \cdot \partial_1 \big( \delta \xi(\mathbf{x}) \boldsymbol{\Gamma}'(\xi_0(\mathbf{x})) %\nabla \xi_0 (\mathbf{x}) \big)= 0.
%    \end{aligned}
%\end{equation}
%Now recognize that, structurally, this equation is simply a second order linear PDE in $\delta \xi(\mathbf{x})$ of the form
Again, we can take the curl of (\ref{eq:linPDE2}) to eliminate $\delta \gamma(\mathbf{x})$ in favor of $\delta\xi(\mathbf{x})$. The end result is a second order linear scalar PDE 
\begin{equation}
 \begin{aligned}\label{eq:large2ndOrder}
    &c_{11}(\mathbf{x}) \partial_1 \partial_1 \delta \xi(\mathbf{x}) + c_{22}(\mathbf{x}) \partial_2 \partial_2 \delta \xi(\mathbf{x}) + c_{12}(\mathbf{x}) \partial_1 \partial_2 \delta \xi(\mathbf{x}) + b_1(\mathbf{x}) \partial_1 \delta \xi(\mathbf{x}) + b_2(\mathbf{x}) \partial_2 \delta \xi(\mathbf{x}) + a(\mathbf{x}) \delta \xi(\mathbf{x}) = 0
    \end{aligned}
\end{equation}
with coefficients $c_{ij}(\mathbf{x})$, $b_i(\mathbf{x})$ and $a(\mathbf{x})$ depending on  $\xi_0(\mathbf{x})$ and $\nabla \xi_0(\mathbf{x})$. It is classified through its highest order derivatives, which involve the coefficients $c_{ij}(\mathbf{x})$. As a moment's consideration will show, these terms are identical to those of (\ref{eq:linPDE}), except  that the previously constant mechanism angle $\xi_0$ is now replaced by a spatially-varying one $\xi_0(\mathbf{x})$. Thus, the PDE type is determined by  $\sigma_{\text{PDE}}(\xi_0(\mathbf{x}))$ from (\ref{eq:sigmaPDE}). The dependence on $\mathbf{x}$ highlights the possibility that this type may vary throughout $\Omega$.

In summary, the PDE system (\ref{eq:forPDEType}) governing soft modes of our kirigami patterns is classified according to its linearization about a solution $(\gamma_0(\mathbf{x}), \xi_0(\mathbf{x}))$, using the determinant 
\begin{equation}
    \begin{aligned}\label{eq:sigmaPDE-general}
    \sigma_{\text{PDE}}(\xi_0(\mathbf{x})) := \det\Big( \text{sym}\big(\mathbf{W} \boldsymbol{\Gamma}(\xi_0(\mathbf{x}))\big)\Big).
    \end{aligned}
\end{equation}
It is \textit{elliptic} where $\sigma_{\text{PDE}}(\xi_0(\mathbf{x})) >0$, \textit{hyperbolic} where $\sigma_{\text{PDE}}(\xi_0(\mathbf{x})) <0$, and \textit{parabolic} where $\sigma_{\text{PDE}}(\xi_0(\mathbf{x})) =0$. It can be of mixed type in general, but can also be of one type throughout $\Omega$ as demonstrated by the rhombi-slit examples in the main text (see Fig.\;4). %Its type gives a hint as to the properties of its solutions, e.g., waves manifest in the hyperbolic case. 

%Motivated by this observation, we classify kirigami deformations based on our effective continuum field theory as follows:
%\begin{definition}
%Consider the coarse-grained description of a planar kirigami pattern with a unit cell of four quad panels and four %parallelogram slits. Let $\mathbf{y}_{\emph{eff}}(\mathbf{x}),\gamma(\mathbf{x})$ and $\xi(\mathbf{x})$ be a solution to the %effective description 
%\begin{equation}
%\begin{aligned}\label{eq:solutionClassification}
%\nabla \mathbf{y}_{\emph{eff}}(\mathbf{x}) = \mathbf{R}(\gamma(\mathbf{x})) \mathbf{A}(\xi(\mathbf{x})), \quad \nabla %\gamma(\mathbf{x}) = \boldsymbol{\Gamma}(\xi(\mathbf{x})) \nabla \xi(\mathbf{x}), \quad \mathbf{x} \in \Omega,  
%\end{aligned}
%\end{equation}
%describing the kirigami's response when there are many cells filling a planar reference domain $\Omega$.  We classify the %response as follows:
%\begin{itemize}[leftmargin=*]
%    \item \textbf{Elliptic kirigami.}\;The solution in (\ref{eq:solutionClassification}) satisfies  %$\sigma_{\emph{PDE}}(\xi(\mathbf{x})) > 0$ on all of $\Omega$.
%    \item \textbf{Hyperbolic kirigami.}\;The solution, instead, satisifies $\sigma_{\emph{PDE}}(\xi(\mathbf{x})) < 0$ on all %of $\Omega$. 
%    \item \textbf{Mixed-type kirigami.}\;The solution satisfies $\sigma_{\emph{PDE}}(\xi(\mathbf{x})) >0$ on some part of %$\Omega$ and $\sigma_{\emph{PDE}}(\xi(\mathbf{x})) <0$ on another part. 
%\end{itemize}
%\end{definition}

\subsection{Definition of the effective Poisson's ratio}
Next, we define the effective Poisson's ratio of our general kirigami patterns, including the rhombi-slit ones from the main text as a special case. After identifying the Poison's ratio, we go on to link it to the PDE type in the next section. 

We require a formula for the linear strain of a perturbation. 
As in the main text, we expand about a mechanism using $\mathbf{y}_{\text{eff}}(\mathbf{x}) = \mathbf{A}(\xi_0)\mathbf{x} + \mathbf{u}(\mathbf{A}(\xi_0) \mathbf{x})$, where $\mathbf{u}(\mathbf{y})$ is a small displacement field. It is coupled to the angles $\gamma(\mathbf{x})  = \delta \gamma(\mathbf{x})$ and $\xi(\mathbf{x}) = \xi_0 + \delta \xi(\mathbf{x})$ through the PDE (\ref{eq:effPDE}). (Again, we set $\gamma_0=0$ to remove the free global rotation.) Notice that 
\begin{equation}
    \begin{aligned}
    &\nabla \mathbf{y}_{\text{eff}}(\mathbf{x}) = \mathbf{A}(\xi_0) + \nabla \mathbf{u}(\mathbf{A}(\xi_0) \mathbf{x}) \mathbf{A}(\xi_0),\\
    & \mathbf{R}(\gamma(\mathbf{x})) \mathbf{A}(\xi(\mathbf{x})) = \mathbf{A}(\xi_0) + \delta \gamma(\mathbf{x}) \mathbf{W} \mathbf{A}(\xi_0) + \delta \xi(\mathbf{x}) \mathbf{A}'(\xi_0) 
    \end{aligned}
\end{equation}
to first order in the perturbation.
Since the lefthand sides above are required to be equal by the effective PDE (\ref{eq:effPDE}),   
\begin{equation}
 \begin{aligned}\label{eq:uDisp}
 \nabla \mathbf{u}(\mathbf{A}(\xi_0)\mathbf{x}) = \delta \gamma(\mathbf{x}) \mathbf{W} + \delta \xi(\mathbf{x}) \mathbf{A}'(\xi_0) \mathbf{A}^{-1}(\xi_0)
 \end{aligned}
\end{equation}
to leading order. The linear strain  $\boldsymbol{\varepsilon}(\mathbf{y}) = \sym \nabla \mathbf{u}(\mathbf{y})$ is then given to leading order by
\begin{equation}
    \begin{aligned}\label{eq:firstAux} 
    \boldsymbol{\varepsilon}(\mathbf{A}(\xi_0)\mathbf{x}) = \delta \xi(\mathbf{x}) \sym \big(\mathbf{A}'(\xi_0) \mathbf{A}^{-1}(\xi_0)\big).
    \end{aligned}
\end{equation}
Since this strain is symmetric, it can be represented in the principle directions of strain space as 
\begin{equation}
    \begin{aligned}\label{eq:spectralAux}
    \boldsymbol{\varepsilon}(\mathbf{A}(\xi_0)\mathbf{x})  = \delta \xi(\mathbf{x}) \Big( \varepsilon_1(\xi_0) \mathbf{v}_1(\xi_0) \otimes \mathbf{v}_1(\xi_0) + \varepsilon_2(\xi_0) \mathbf{v}_2(\xi_0) \otimes \mathbf{v}_2(\xi_0)\Big).
    \end{aligned}
\end{equation}
To fix the notation, we let $\{\mathbf{v}_{1}(\xi_0), \mathbf{v}_{2}(\xi_0)\}$ be a righthanded orthonormal basis with $\mathbf{v}_1(\xi_0) \cdot \mathbf{e}_1 > 0$ and $\mathbf{v}_1(\xi_0) \cdot \mathbf{e}_2 \geq 0$.  
Given all this, we can now define the effective Poisson's ratio of our kirigami patterns as
\begin{equation}
    \begin{aligned}
    \nu_{21}(\xi_0) := - \frac{\varepsilon_2(\xi_0)}{\varepsilon_1(\xi_0)}.
    \end{aligned}
\end{equation}
This makes concrete the brief description of the general Poisson's ratio in the main text, and recovers the definition in the rhombi-slit case as well. Note its value depends on the slit actuation, $\xi_0$.  

Importantly, away from singular points where $\varepsilon_1(\xi_0) = 0$, the Poisson's ratio satisfies 
\begin{equation}
    \begin{aligned}
    \sign( \nu_{21}(\xi_0) ) = -\sign( \varepsilon_1(\xi_0) \varepsilon_2(\xi_0)) = -\sign\Big[\det \Big( \sym\big(\mathbf{A}'(\xi_0) \mathbf{A}^{-1}(\xi_0)\big) \Big) \Big].
    \end{aligned}
\end{equation}
This follows from combining the righthand sides of  (\ref{eq:firstAux}) and (\ref{eq:spectralAux}).  Thus, the kirigami pattern's auxeticity at actuation $\xi_0$ is determined by the sign of  
\begin{equation}
    \begin{aligned}\label{eq:sigmaAux}
    \sigma_{\text{Aux}}(\xi_0) := \det \Big( \sym\big(\mathbf{A}'(\xi_0) \mathbf{A}^{-1}(\xi_0)\big)\Big).
    \end{aligned}
\end{equation}
When $\sigma_{\text{Aux}}(\xi_0) > 0$, the pattern has a negative Poisson's ratio and is auxetic; when $\sigma_{\text{Aux}}(\xi_0) < 0$, it has a positive Poisson's ratio and a standard, non-auxetic response. %We turn to relate this criterion to the PDE type.

\subsection{Link between  PDE type and the Poisson's ratio}
We are now in a position to substantiate the claim from the main text that the PDE type of (\ref{eq:forPDEType}) is linked in general to the effective Poisson's ratio of the kirigami.
Formally, we establish the following universal relationship between (\ref{eq:sigmaPDE}) and (\ref{eq:sigmaAux}):
\begin{equation}
    \begin{aligned}\label{eq:universal}
    \sign\big(\sigma_{\text{Aux}}(\xi)\big) =  \sign \big( \sigma_{\text{PDE}}(\xi)\big) \quad \text{when $\det \mathbf{A}(\xi) >0$}.
    \end{aligned}
\end{equation}
Note $\det \nabla \mathbf{y}_{\text{eff}}(\mathbf{x}) = \det \mathbf{A}(\xi(\mathbf{x})) > 0$  for all physically relevant deformations. The identity (\ref{eq:universal}) shows that our kirigami patterns deform auxetically if and only if their effective PDEs are of elliptic type. Conversely, a hyperbolic PDE type corresponds to a standard, non-auxetic Poisson's ratio.  %Per Eq.\;(10), the pattern was found to be auxetic if and only if it is elliptic, and not auxetic if and only if it is hyperbolic. Here, we show that this link is, in fact, universal --- it holds for all quad-kirigami whose unit cells have four convex quad panels and four parallelogram slits (as identified in Section \ref{s:UC}). To make this result precise, we must first make clear the notion of effective Poisson's ratio for a general kirigami pattern.   %The general result is as follows: the kirigami is elliptic if and only if it is auxetic, and hyperbolic if and only if it is \textit{not} auxetic. 

The desired identity actually follows from a stronger result regarding the eigenvalues of the symmetric tensors in the definitions of $\sigma_\text{Aux}(\xi)$ and $\sigma_\text{PDE}(\xi)$. The fact is that the eigenvalues of $\sym \big( \mathbf{W} \boldsymbol{\Gamma}(\xi) \big)$
 and $\sym \big(\mathbf{A}'(\xi)  \mathbf{A}^{-1}(\xi)\big)$ are of \textit{opposite signs}, provided $\det \mathbf{A}(\xi(\mathbf{x})) > 0$. This is clear in the rhombi-slit case, where the matrices are diagonal (as follows from the formulas in Eq.\;(5) and (6)). But in the general case, it is not obvious. To verify it, we show that the quadratic forms 
\begin{equation}
    \begin{aligned}
    &Q_{\text{PDE}}(\mathbf{v}) = \mathbf{v} \cdot  \sym \big( \mathbf{W} \boldsymbol{\Gamma}(\xi) \big) \mathbf{v}, \\
    &Q_{\text{Aux}}(\mathbf{v}) = \mathbf{v} \cdot  \sym \big(\mathbf{A}'(\xi)  \mathbf{A}^{-1}(\xi)\big) \mathbf{v}
    \end{aligned}
\end{equation}
are negatives of each other, in suitable coordinates. Since the eigenvalues of the matrices in question are nothing other than the maximum and minimum of $Q_\text{PDE}(\mathbf{v})$ and $Q_\text{Aux}(\mathbf{v})$ amongst unit vectors $|\mathbf{v}|=1$ \cite{lax07}, this establishes the result. %, we have 
%\begin{equation}
%    \begin{aligned}\label{eq:eigSigma}
%    &\sign\big(\sigma_{\text{PDE}}(\xi)\big) = %\sign\Big(\min_{\text{unit vectors }\mathbf{v}} %Q_{\text{PDE}}(\mathbf{v}) \Big) \sign %\Big(\max_{\text{unit vectors }\mathbf{w}} %Q_{\text{PDE}}(\mathbf{w}) \Big), \\
%    &\sign\big(\sigma_{\text{Aux}}(\xi)\big) = %\sign\Big(\min_{\text{unit vectors }\mathbf{v}} %Q_{\text{Aux}}(\mathbf{v}) \Big) \sign %\Big(\max_{\text{unit vectors }\mathbf{w}} %Q_{\text{Aux}}(\mathbf{w}) \Big).
%    \end{aligned}
%\end{equation}
%We establish the result in (\ref{eq:universal}) by manipulating these quadratic forms.

Observe that 
\begin{equation}
    \begin{aligned}
    Q_{\text{PDE}}(\mathbf{v}) = \mathbf{v} \cdot \mathbf{W} \boldsymbol{\Gamma}(\xi) \mathbf{v}, \quad Q_{\text{Aux}}(\mathbf{v}) = \mathbf{v} \cdot \mathbf{A}'(\xi) \mathbf{A}^{-1}(\xi) \mathbf{v}
    \end{aligned}
\end{equation}
since $\mathbf{v} \cdot \mathbf{B} \mathbf{v} = \mathbf{0}$ whenever $\mathbf{B}$ is a skew tensor. Using the first of these, we get that 
\begin{equation}
    \begin{aligned}
    Q_{\text{PDE}}(\mathbf{W}\tilde{\mathbf{v}}) = -\frac{1}{\det \mathbf{A}(\xi)}\tilde{\mathbf{v}}  \cdot \Big( \mathbf{A}^T(\xi) \mathbf{A}'(\xi) \Big) \tilde{\mathbf{v}}
    \end{aligned}
\end{equation}
by the definition of $\boldsymbol{\Gamma}(\xi)$ in (\ref{eq:finalCompat}). Similarly,  \begin{equation}
    \begin{aligned}\label{eq:QAuxFinal}
    Q_{\text{Aux}}\big( \tfrac{\mathbf{A}(\xi) \tilde{\mathbf{v}}}{|\mathbf{A}(\xi) \tilde{\mathbf{v}}|}\big) = \frac{1}{|\mathbf{A}(\xi)\tilde{\mathbf{v}}|^2} \tilde{\mathbf{v}} \cdot \mathbf{A}^T(\xi) \mathbf{A}'(\xi) \tilde{\mathbf{v}} 
    \end{aligned}
\end{equation}
for any $\tilde{\mathbf{v}} \neq \mathbf{0}$. The resulting forms involve the \textit{same} tensor $\mathbf{A}^T(\xi) \mathbf{A}'(\xi)$. So whenever $\det \mathbf{A}(\xi)> 0$, 
\begin{equation}
    \begin{aligned}
    &\sign\Big( \min_{|\mathbf{v}|=1}\, Q_{\text{PDE}}(\mathbf{v})\Big) = -\sign \Big(\min_{|\tilde{\mathbf{v}}|=1}\, \tilde{\mathbf{v}}\cdot \mathbf{A}^T (\xi) \mathbf{A}'(\xi) \tilde{\mathbf{v}} \Big), \\
    &\sign\Big( \min_{|\mathbf{v}|=1}\, Q_{\text{Aux}}(\mathbf{v})\Big) = \sign \Big(\min_{|\tilde{\mathbf{v}}|=1}\, \tilde{\mathbf{v}}\cdot \mathbf{A}^T (\xi) \mathbf{A}'(\xi) \tilde{\mathbf{v}} \Big).
    \end{aligned}
\end{equation}
The same identities hold with maximization in place of minimization. Hence, (\ref{eq:universal}) is proved.










 





\section{Exact solutions of the effective PDE}

In this section, we provide a detailed description of the  nonlinear PDE solutions behind the theory portions of Fig.\;4 in the main text. The reader may also wish to consult Section \ref{s:Simulations} which presents the numerical method we use to plot the panel motions.

\subsection{Simple wave solutions}
%We show that a wave-like response is possible when the geometric Poisson's ratio $\nu_{21}(\xi)$ is positive on some interval of opening angles $\xi$.
We start by constructing the nonlinear wave response of our hyperbolic rhombi-slit kirigami, under an assumption that its Poisson's ratio $\nu_{21}(\xi)$ is positive on an interval of slit actuation angles $\xi$. These solutions are used to plot the theory half of Fig.\;4(b).

Consider solving the compatibility equations
\begin{equation}\label{eq:compaptHyper}
\nabla \gamma(\mathbf{x}) = \boldsymbol{\Gamma}(\xi(\mathbf{x})) \nabla \xi(\mathbf{x})
\end{equation}
with $\nu_{21}(\xi(\mathbf{x})) > 0$. Since we are dealing with rhombi-slits, $\boldsymbol{\Gamma}(\xi) = \Gamma_{12}(\xi) \mathbf{e}_1 \otimes \mathbf{e}_2 + \Gamma_{21}(\xi) \mathbf{e}_2 \otimes \mathbf{e}_1$ for  $\Gamma_{12}(\xi)  = -\mu_1'(\xi)/\mu_{2}(\xi)$ and $\Gamma_{21}(\xi) = \mu_2'(\xi)/ \mu_1(\xi)$. Its eigenvectors and corresponding eigenvalues are
\begin{equation}
    \label{eq:eigen}
    \mathbf{v}^{\pm}(\xi) = \mathbf{e}_1  \pm \tfrac{\sqrt{\Gamma_{12}(\xi)\Gamma_{21}(\xi)}}{\Gamma_{12}(\xi)}\mathbf{e}_2\quad\text{and}\quad  
   \Gamma^{\pm}(\xi) = \pm \sqrt{\Gamma_{12}(\xi) \Gamma_{21}(\xi)}
\end{equation}
and the condition that $\nu_{21}(\xi(\mathbf{x}))>0$ requires that $\Gamma_{12}(\xi(\mathbf{x})) \Gamma_{21}(\xi(\mathbf{x})) > 0$. Note this guarantees a real-valued eigensystem (another way of defining hyperbolicity). 
To solve (\ref{eq:compaptHyper}), we look for a solution satisfying $\gamma(\mathbf{x}) = f(\xi(\mathbf{x}))$ for a function $f(\xi)$. This is inspired by the notion of a  \textit{simple wave} solution, which is of the  form $(\gamma(\mathbf{x}),\xi(\mathbf{x})) = \mathbf{f}(\theta(\mathbf{x}))$ for functions $\mathbf{f}(\theta)$ and $\theta(\mathbf{x})$ \cite[Ch.\;11.2.1]{evans10}. Here, we use  $\mathbf{f}(\theta)=(f(\theta),\theta)$. 

In order for the ansatz $\gamma(\mathbf{x}) = f(\xi(\mathbf{x}))$ to solve the PDE system (\ref{eq:compaptHyper}), we must require that 
\begin{equation}
\begin{aligned}\label{eq:reduceCompat}
f'(\xi(\mathbf{x})) \nabla \xi(\mathbf{x}) = \boldsymbol{\Gamma}(\xi(\mathbf{x})) \nabla \xi(\mathbf{x}).
\end{aligned}
\end{equation}
This condition requires $\nabla \xi(\mathbf{x})$ to be parallel to an  eigenvector $\mathbf{v}^{\pm}(\xi(\mathbf{x}))$. Then, $f'(\xi)$ is recovered from the corresponding eigenvalue.   Since the eigenvectors form an orthonormal basis, it is equivalent to solve 
\begin{equation}
\begin{aligned}
\begin{cases}\label{eq:methodCharac}
\Gamma_{12}(\xi(\mathbf{x})) \Gamma_{21}(\xi(\mathbf{x})) > 0,\\
\nabla \xi(\mathbf{x}) \cdot \mathbf{W} \mathbf{v}^{\sigma}(\xi(\mathbf{x})) = 0
\end{cases}
\end{aligned}
\end{equation}
with $\sigma= +$ or $-$ and with $\mathbf{W} = \mathbf{R}(\pi/2)$. For various boundary conditions,  (\ref{eq:methodCharac}) has a unique solution, which covers some or all of the specimen domain $\Omega$. This solution can be obtained analytically by the method of characteristics:  $\xi(\mathbf{x})$ is found to be constant along a family of straight \textit{characteristic lines}, each of which is perpendicular to $\mathbf{v}^{\sigma}(\xi(\mathbf{x}))$. If such a solution exists in only part of the domain $\tilde{\Omega} \subset \Omega$, we can attempt to cover the rest of $\Omega$ using other simple wave solutions (or other solutions altogether). In Fig.\;4(b), we use four simple wave solutions in the four corners of the sample, along with a constant actuation region where $\xi(\mathbf{x})\equiv \xi_0$ in the middle portion.

Before going on to describe the simple wave solutions behind Fig.\;4(b), we explain how to recover the rotation angle $\gamma(\mathbf{x})$. Suppose we have a simple wave solution per (\ref{eq:methodCharac}) on a sub-domain $\tilde{\Omega} \subset \Omega$ covered by characteristic lines.  Then, considering the equation (\ref{eq:reduceCompat}) and our ansatz $\gamma(\mathbf{x}) = f(\xi(\mathbf{x}))$, we find that 
\begin{equation}
\begin{aligned}\label{eq:gammaSolve}
\gamma(\mathbf{x}) = \int_0^{\xi(\mathbf{x})} \Gamma^{\sigma}(s) ds + \gamma_0, \quad \mathbf{x}\in\tilde{\Omega}
\end{aligned}
\end{equation}
for some real-valued constant $\gamma_0$. Since $\xi(\mathbf{x})$ remains constant along characteristic lines, so does $\gamma(\mathbf{x})$.

To obtain Fig.\;4(b), we first construct a simple wave solution to match the experiment in the upper left quadrant of the pattern.  We take this region to be the domain $(0,1)^2$ with the panels scaled appropriately on this domain without loss of generality.  The solutions on the other three quadrants are then obtained by mirror symmetry. Since the experiment appears to indicate a rarefaction fan-type shape, we parameterize our simple wave solutions using their boundary data along the line segment $\mathcal{L} := \{ s \mathbf{e}_2 \colon s \in (0,1)\}$, and consider data $\xi=\xi_{\text{b}}$ corresponding to a sharp increase in the opening angle at the origin.  Specifically, we use  $\xi_{\text{b}}(s) = \xi_0 \exp(-\lambda s)$, $s \in (0,1)$ for $\lambda \gg 1$ and $\xi_0 \in (0, 0.235 \pi)$. This choice gives  $\xi_{\text{b}}(0) = \xi_0$, $\xi_{\text{b}}(1)  \approx 0$, with a sharp increase in actuation near the origin.  We then set 
\begin{equation}
\begin{aligned}
\xi\big( s \mathbf{e}_2 + t \mathbf{W} \mathbf{v}^{+}( \xi_{\text{b}}(s)) \big) = \xi_{\text{b}}(s) \quad \text{ for } \quad   s \mathbf{e}_2 + t \mathbf{W} \mathbf{v}^{+}( \xi_{\text{b}}(s)) \in (0,1)^2
\end{aligned}
\end{equation}
to obtain our simple wave solution. This prescription solves (\ref{eq:methodCharac}) with characteristic lines that sweep from left to right throughout the domain. They start from a line parallel to $\mathbf{e}_2$, and tilt towards a line parallel to $\mathbf{e}_1$ for a value of $\xi_0 = 0.235\pi$.  The value of $\xi_0 = 0.11 \pi$ matches the bulk deformation inside the (essentially uniform) middle diamond region.  This value sets the rightmost characteristic, which is parallel to $\mathbf{W} \mathbf{v}^{+}( \xi_{\text{b}}(0) = 0.11\pi)  \approx 0.7 \mathbf{e}_1 + \mathbf{e}_2$. We  choose $\lambda = 10$  for the simulated solution; other large values of $\lambda$ do not alter the solution noticeably.  Again, we apply $\xi(\mathbf{x}) \equiv \xi_0 = 0.11\pi$ on the regions of $(0,1)^2$ not described by the simple wave solution and its mirror reflections. %This choice gives the full expression for $\xi(\mathbf{x})$ solving (\ref{eq:methodCharac})  on $(0,1)^2$ for $\sigma = +$. 
Having constructed $\xi(\mathbf{x})$, we find $\gamma(\mathbf{x})$ using (\ref{eq:gammaSolve}) with $\gamma_0 = -\int_0^{\xi_b(0)} \Gamma^{+}(s) ds$. This prescription solves (\ref{eq:compaptHyper}) on $(0,1)^2$  and mirroring the result  produces a solution to the full compatibility condition (\ref{eq:compaptHyper}) for the example in Fig.\;4(b). We plot the result using the numerical method discussed in Section \ref{s:Simulations}.



\subsection{Conformal map solutions}
We now consider rhombi-slit kirigami under the assumption that $\alpha = - \beta$.  From the formula for $\mathbf{A}(\xi)$ in the main text, the effective PDE reduces in this setting to
\begin{equation}
    \begin{aligned}\label{eq:conformalCase}
        \nabla \mathbf{y}_{\text{eff}}(\mathbf{x}) = \mu(\xi(\mathbf{x}))  \mathbf{R}(\gamma(\mathbf{x})) 
    \end{aligned}
\end{equation}
where $\mu_1(\xi) = \mu_2(\xi) = \mu(\xi) = \cos \xi - \alpha \sin \xi$.
Instead of using the compatibility condition to determine the angles $(\gamma(\mathbf{x}), \xi(\mathbf{x}))$, we simply observe that such effective deformations are conformal maps in the plane. So, they can be found using complex analysis. 


Let $z = x_ 1 + i x_2$ denote a complex number and consider any complex analytic function  $f(z)$  on a domain $\Omega$ in the complex plane.  Then  $\mathbf{y}_{\text{c}}(\mathbf{x}) = \text{Re}[f(z)] \mathbf{e}_1 + \text{Im}[f(z)] \mathbf{e}_2$ has $\nabla \mathbf{y}_{\text{c}}(\mathbf{x}) = \mu_{\text{c}}(\mathbf{x}) \mathbf{R}(\gamma(\mathbf{x}))$ for a 2D rotation $\mathbf{R}(\gamma(\mathbf{x}))$ and dilatation $\mu_{\text{c}}(\mathbf{x})$. It is an effective deformation obeying (\ref{eq:conformalCase}) if we can match  $\mu_c(\mathbf{x})$ to an angle $\xi(\mathbf{x})$ such that $\mu(\xi(\mathbf{x})) = \mu_c(\mathbf{x})$.  This is  possible as long as $\mu_c(\mathbf{x})$  does not stray too far from $1$ (how far exactly depends only on the choice of $\alpha = -\beta$).

\textit{Optimization framework for conformal kirigami\;--\;}We generate conformal deformations $\mathbf{y}_{c}(\mathbf{x}; \boldsymbol{\tau}, \boldsymbol{\delta}, \boldsymbol{\kappa})$ as above, based on the family of complex polynomials of the form $f(z) = \sum_{k =1,\ldots, N} \tau_{k} (z - \delta_k  - i \kappa_k)^k$ with real valued parameters $\boldsymbol{\tau} = (\tau_1, \ldots, \tau_N)$, $\boldsymbol{\delta} = (\delta_1, \ldots, \delta_N)$ and $\boldsymbol{\kappa} = (\kappa_1, \ldots, \kappa_N)$.  To compare with the experiments, we notice that a proxy for boundary data is the list of 2D vectors $\bar{\mathbf{y}}_{1}, \ldots, \bar{\mathbf{y}}_{M}$ that represent the center points of each boundary panel in a deformed sample. Prior to deformation, these panels are also described by an analogous list of 2D vectors  $\bar{\mathbf{x}}_1, \ldots, \bar{\mathbf{x}}_M$. After extracting these lists, we optimize 
\begin{equation}
\begin{aligned}
\min_{\boldsymbol{\tau}, \boldsymbol{\delta}, \boldsymbol{\kappa}  \in \mathbb{R}^{N}} \sum_{m = 1}^{M} | \mathbf{y}_{\text{c}}(\bar{\mathbf{x}}_m; \boldsymbol{\tau},\boldsymbol{\delta}, \boldsymbol{\kappa}) -  \bar{\mathbf{y}}_{m}|^2
\end{aligned}
\end{equation}
 using \verb+fminunc+ in MATLAB. From the optimal $\mathbf{y}_{\text{eff}}(\mathbf{x}) =  \mathbf{y}_{\text{c}}(\mathbf{x}; \boldsymbol{\tau}_{\text{opt}}, \boldsymbol{\delta}_{\text{opt}}, \boldsymbol{\kappa}_{\text{opt}})$, we obtain the angles $\xi(\mathbf{x})$ and $\gamma(\mathbf{x})$. Finally, with $\mathbf{y}_{\text{eff}}(\mathbf{x}), \gamma(\mathbf{x}), \xi(\mathbf{x})$ known, we plot the results using the numerical method described in Section \ref{s:Simulations}.

This optimization procedure is extremely versatile. We used it to produce the simulation in Fig.\;4(e), based on the extracted boundary data from the RS center-pulling experiment (same figure). We have also done a number of other comparisons of different boundary data for the RS example. Fig.\;\ref{fig:Conformal} shows three additional comparisons between theory and experiment using the optimization procedure. All simulations in the conformal case were carried out with $N = 5$ ($15$ parameters are optimized) with the reference configuration  $\boldsymbol{\tau^0} = (1,0,0,0,0)$, $\boldsymbol{\delta^0} = \boldsymbol{0}$ and $\boldsymbol{\kappa^0} = \boldsymbol{0}$ chosen as the input data to the optimization.

\begin{figure}[h!]
\centering
\includegraphics[width =\linewidth]{Conformal.pdf}
\caption{Comparison between experiments (top) and simulations (bottom) for conformal kirigami.}
\label{fig:Conformal}
\end{figure}


\subsection{One-dimensional solutions}
Here we construct some universal solutions for both hyperbolic and elliptic rhombi-slit kirigami, via the one-dimensional ansatz $\xi(\mathbf{x}) = f(\mathbf{x} \cdot \mathbf{t})$ with a unit vector $\mathbf{t}$. The annular solutions in Fig.\;4(c) and (f) use the results of this section.

Enforcing the usual compatibility conditions (\ref{eq:compaptHyper}), we find for $\xi(\mathbf{x}) = f(\mathbf{x} \cdot \mathbf{t})$ that \begin{equation}
\begin{aligned}
\nabla \gamma(\mathbf{x}) = f'(\mathbf{x} \cdot \mathbf{t}) \Gamma_{12}(f(\mathbf{x} \cdot \mathbf{t})) (\mathbf{t} \cdot \mathbf{e}_2) \mathbf{e}_1 +  f'(\mathbf{x} \cdot \mathbf{t}) \Gamma_{21}(f(\mathbf{x} \cdot \mathbf{t})) (\mathbf{t} \cdot \mathbf{e}_1) \mathbf{e}_2.
\end{aligned}
\end{equation}
Since the partial derivatives $\partial_1 \partial_2 \gamma(\mathbf{x})$ and $\partial_2 \partial_1 \gamma(\mathbf{x})$ commute, we obtain that
\begin{equation}
\begin{aligned}\label{eq:ODEformula}
\Big( f'(s) \Gamma_{12}(f(s)) (\mathbf{t} \cdot \mathbf{e}_2)^2  - f'(s) \Gamma_{21}(f(s))  (\mathbf{t} \cdot \mathbf{e}_1)^2 \Big)' &= 0 \\
\Rightarrow \quad f'(s)\Big(  \Gamma_{12}(f(s)) (\mathbf{t} \cdot \mathbf{e}_2)^2 - \Gamma_{21}(f(s))  (\mathbf{t} \cdot \mathbf{e}_1)^2\Big) &= c_0
\end{aligned}
\end{equation}
for some constant $c_0 \in \mathbb{R}$. This last equation is an ODE in $f(s)$ that can be solved for a suitable initial condition $f(0)$. 

Taking one such solution, we recover its $\gamma(\mathbf{x})$ as follows. If $\mathbf{t} \cdot \mathbf{e}_2 = 0$, then  $\nabla \gamma(\mathbf{x}) = -c_0 \mathbf{e}_2$, yielding $\gamma(\mathbf{x}) = - c_0 (\mathbf{x} \cdot \mathbf{e}_2)+ d_0$ for some constant $d_0$.  Otherwise,
\begin{equation}
\begin{aligned}
\nabla \gamma(\mathbf{x}) =  \frac{c_0}{(\mathbf{t} \cdot \mathbf{e}_2)} \mathbf{e}_1  + f'(\mathbf{x} \cdot \mathbf{t})  \Gamma_{21}(f(\mathbf{x} \cdot \mathbf{t})) \frac{(\mathbf{t} \cdot \mathbf{e}_1)}{(\mathbf{t} \cdot \mathbf{e}_2)} \mathbf{t}
\end{aligned}
\end{equation}
and hence
\begin{equation}
\begin{aligned}
\gamma(\mathbf{x}) = \frac{c_0}{(\mathbf{t} \cdot \mathbf{e}_2)} (\mathbf{x} \cdot \mathbf{e}_1) +  \frac{(\mathbf{t} \cdot \mathbf{e}_1)}{(\mathbf{t} \cdot \mathbf{e}_2)}  \int_0^{f(\mathbf{x} \cdot \mathbf{t})} \Gamma_{21}(s) ds  + d_0
\end{aligned}
\end{equation}
for some constant $d_0$.  In either case, we obtain an expression for $\gamma(\mathbf{x})$.  %that solves the compatibility condition $\nabla \gamma(\mathbf{x}) = \boldsymbol{\Gamma}(\xi(\mathbf{x})) \nabla \xi(\mathbf{x})$ when $\xi(\mathbf{x}) = f(\mathbf{x} \cdot \mathbf{t})$.

The annular solutions in the main text use $\mathbf{t} = \mathbf{e}_1$ or $\mathbf{e}_2$. We solve the ODE in (\ref{eq:ODEformula}) with $f(0) = 0$, and select the constant $c_0$ such that the resulting $f(s)$ is monotonically increasing on the interval $(0,1)$.

\section{Specimen fabrication and data extraction}
Our kirigami specimens are cut out of 1.5 mm-thick natural rubber sheets (McMaster-Carr 8633K71) using an 80 Watt Epilog Fusion Pro 32 laser cutter. To avoid burning the specimens and to produce clean cuts, the laser cutter is focused on the bottom face of the rubber sheet. The specimens are painted with white primer paint in order to facilitate the image processing procedure and to create high contrast with the black background.

The specific kirigami patterns discussed in the main article need to undergo slight modifications before being processed by the laser cutter. In particular, ideal point-like joints are replaced with compliant flexure-like hinges. This modification to the design is illustrated in Fig.~\ref{fig:Fab} for the hyperbolic sample. Fig.~\ref{fig:Fab}(a) represents the idealized pattern and Fig.~\ref{fig:Fab}(b) represents the pattern that is sent to the laser cutter.
In Fig.~\ref{fig:Fab}(b), we also indicate the normalized dimensions of the hinges: $\hat{d}=1/10$ and $\hat{h}=1/40$ as shown. %{\color{red} Note that the presence of elastic joints inevitably yields discrepancies with the purely-kinematic theoretical solutions. These discrepancies are particularly accentuated in Fig.~4(b) and (e), where we pull a few locations along the boundary and where we try to find theoretical solutions that match the experimental results. Conversely, discrepancies are minimized in the case of the annular deformations of Fig.~4(c) and (f), where the experiments are designed to match the theory. This is done by exactly matching the boundary slit openings coming from the theoretical solution, and by letting the inner portion of the specimen deform accordingly.}
%
\begin{figure}[h]
\centering
\includegraphics[scale = 1]{Fig_Sup_Fab.pdf}
\caption{(a) Idealized unit cell for the hyperbolic sample, with its characteristic dimensions. (b) A portion of the pattern used to fabricate an analog to (a); the thin red lines overlaid to the pattern represent the idealized unit cell. The only difference between (b) and (a) is the addition of flexure-like hinges of non-dimensional width $\hat{d}$ and length $\hat{h}$.}
\label{fig:Fab}
\end{figure}
%

The images of the samples are recorded by means of a FLIR 5-megapixel 35 fps camera with Edmund Optics lenses. The samples are anchored to the supporting plate via pins or double-sided tape. Quantitative information on the deformation is extracted through digital image processing in MATLAB. After converting the images to binary (using the \verb+imbinarize+ function), we obtain the centroid $\textbf{c}_i$, semi-major axis $\textbf{a}_i$, and semi-minor axis $\textbf{b}_i$ of a generic slit $i$ using the \verb+regionprops+ function. Then, we calculate $\xi_i$ from $\tan{\xi_i}=a_i/b_i$ and take $\gamma_i$ as the inclination of the major axis with respect to the horizontal; since we know the dimensions of the panels in the sample, we plot an idealized unit cell, deformed by an angle $\xi_i$ and rotated by $\gamma_i$, centered at the slit centroid $\textbf{c}_i$. We repeat this process for each slit. In Fig.\;4 of the main text, panels are only superimposed onto the right-half of each specimen and are colored according to the calculated value of $\xi_i$.


\bibliographystyle{apsrev4-1} % Tell bibtex which bibliography style to use
% \bibliographystyle{ieeetr}
\bibliography{bib}



\end{document}