%% ****** Start of file template.aps ****** %
%%
%%
%%   This file is part of the APS files in the REVTeX 4 distribution.
%%   Version 4.0 of REVTeX, August 2001
%%
%%
%%   Copyright (c) 2001 The American Physical Society.
%%
%%   See the REVTeX 4 README file for restrictions and more information.
%%
%
% This is a template for producing manuscripts for use with REVTEX 4.0
% Copy this file to another name and then work on that file.
% That way, you always have this original template file to use.
%
% Group addresses by affiliation; use superscriptaddress for long
% author lists, or if there are many overlapping affiliations.
% For Phys. Rev. appearance, change preprint to twocolumn.
% Choose pra, prb, prc, prd, pre, prl, prstab, or rmp for journal
%  Add 'draft' option to mark overfull boxes with black boxes
%  Add 'showpacs' option to make PACS codes appear
\documentclass[aps,prl,twocolumn,superscriptaddress]{revtex4-1}  
 % for review and submission
%\documentclass[aps,superscriptaddress,groupedaddress]{revtex4}  % for double-spaced preprint
\usepackage{graphicx}  % needed for figures
\usepackage{dcolumn}   % needed for some tables
\usepackage{bm}        % for math
\usepackage{amssymb}   % for math
\usepackage[final]{epsfig}
\usepackage{amssymb}
\usepackage{latexsym}
\usepackage{wrapfig}
\usepackage{amsmath}
\usepackage{booktabs}
\usepackage{threeparttable}
\usepackage{xcolor}
\usepackage{amsthm}
\usepackage{lipsum}
\usepackage{hyperref}
\usepackage{titlesec,enumitem}
% avoids incorrect hyphenation, added Nov/08 by SSR
\hyphenation{ALPGEN}
\hyphenation{EVTGEN}
\hyphenation{PYTHIA}

% define new commands
\newcommand{\R}{\mathbb{R}}
\newcommand{\Z}{\mathbb{Z}}
\newcommand{\bfa}{{\bf a}}
\newcommand{\bfb}{{\bf b}}
\newcommand{\bfc}{{\bf c}}
\newcommand{\bfd}{{\bf d}}
\newcommand{\bfe}{{\bf e}}
\newcommand{\bff}{{\bf f}}
\newcommand{\bfg}{{\bf g}}
\newcommand{\bfh}{{\bf h}}
\newcommand{\bfk}{{\bf k}}
\newcommand{\bfm}{{\bf m}}
\newcommand{\bfn}{{\bf n}}
\newcommand{\bfo}{{\bf o}}
\newcommand{\bfp}{{\bf p}}
\newcommand{\bfq}{{\bf q}}
\newcommand{\bfr}{{\bf r}}
\newcommand{\bfs}{{\bf s}}
\newcommand{\bft}{{\bf t}}
\newcommand{\bfu}{{\bf u}}
\newcommand{\bfv}{{\bf v}}
\newcommand{\bfw}{{\bf w}}
\newcommand{\bfx}{{\bf x}}
\newcommand{\bfy}{{\bf y}}
\newcommand{\tbfy}{{\tilde{\bfy}}}
\newcommand{\bfz}{{\bf z}}
\newcommand{\bfA}{{\bf A}}
\newcommand{\bfB}{{\bf B}}
\newcommand{\bfC}{{\bf C}}
\newcommand{\bfD}{{\bf D}}
\newcommand{\bfF}{{\bf F}}
\newcommand{\bfG}{{\bf G}}
\newcommand{\bfH}{{\bf H}}
\newcommand{\bfI}{{\bf I}}
\newcommand{\bfK}{{\bf K}}
\newcommand{\bfM}{{\bf M}}
\newcommand{\bfP}{{\bf P}}
\newcommand{\bfQ}{{\bf Q}}
\newcommand{\bfR}{{\bf R}}
\newcommand{\bfS}{{\bf S}}
\newcommand{\bfT}{{\bf T}}
\newcommand{\bfU}{{\bf U}}
\newcommand{\bfV}{{\bf V}}
\newcommand{\bfW}{{\bf W}}
\newcommand{\bfX}{{\bf X}}
\newcommand{\bfY}{{\bf Y}}
\newcommand{\vphi}{{\varphi}}
\newcommand{\eps}{{\varepsilon}}
\newcommand{\Nhat}{\hat{\mbox{\tiny {\bf N}}}}
\newcommand{\ehat}{\hat{\bf e}}
\newcommand{\nhat}{\hat{\bf n}}
\newcommand{\uhat}{\hat{\bf u}}
\newcommand{\phihat}{\hat{\varphi}}
\newcommand{\beq}{\begin{equation}}
\newcommand{\eeq}{\end{equation}}
\newcommand{\beqs}{\begin{eqnarray}}
\newcommand{\eeqs}{\end{eqnarray}}
\newcommand{\calA}{{\cal A}}
\newcommand{\calB}{{\cal B}}
\newcommand{\calC}{{\cal C}}
\newcommand{\calD}{{\cal D}}
\newcommand{\calE}{{\cal E}}
\newcommand{\calF}{{\cal F}}
\newcommand{\calG}{{\cal G}}
\newcommand{\calH}{{\cal H}}
\newcommand{\calI}{{\cal I}}
\newcommand{\calL}{{\cal L}}
\newcommand{\calM}{{\cal M}}
\newcommand{\calN}{{\cal N}}
\newcommand{\calP}{{\cal P}}
\newcommand{\calS}{{\cal S}}
\newcommand{\calT}{{\cal T}}
\newcommand{\calU}{{\cal U}}
\newcommand{\calV}{{\cal V}}
\newcommand{\calW}{{\cal W}}
\newcommand{\sint}{-\hspace{-4mm}\int} % slashed integral, math
\newcommand{\ssint}{-\hspace{-3mm}\int} % slashed integral, text
\newtheorem{theorem}{Theorem}[section]
\newtheorem{corollary}{Corollary}[section]
\newtheorem{lemma}{Lemma}[section]
\newtheorem{proposition}{Proposition}[section]
\newtheorem{remark}{Remark}[section]
\newtheorem{definition}{Definition}[section]
\begin{document}

% The following information is for internal review, please remove them for submission
%\widetext
%\leftline{Version xx as of \today}
%\leftline{Primary authors: Joe E. Physics}
%\leftline{To be submitted to (PRL, PRD-RC, PRD, PLB; choose one.)}
%\leftline{Comment to {\tt d0-run2eb-nnn@fnal.gov} by xxx, yyy}
%\centerline{\em D\O\ INTERNAL DOCUMENT -- NOT FOR PUBLIC DISTRIBUTION}

% the following line is for submission, including submission to the arXiv!!
%\hspace{5.2in} \mbox{Fermilab-Pub-04/xxx-E}

\title{Continuum field theory for the deformations of planar kirigami}
%Coarse-graining the mechanics of planar kirigami
%The effective deformations of frustrated kirigami in the plane } %tessellations in the plane}
%Alternate titles:
% Systematic coarse-graining of frustrated kirigami in the plane
% A universal approach to coarse-graining kirigami in the plane
% On elliptic and hyperbolic kirigami in the plane
% 
%\input author_list.tex       % D0 authors (remove the first 3 lines
                             % of this file prior to submission, they
                             % contain a time stamp for the authorlist)
                             % (includes institutions and visitors)
\author{Yue Zheng}
\affiliation{Aerospace and Mechanical Engineering, University of Southern California, Los Angeles, CA 90014, USA}
%\altaffiliation[Also at ]{Department of Aerospace Engineering and Mechanics, University of Minnesota.}%Lines break automatically or can be forced with \\
\author{Imtiar Niloy}%
\affiliation{Civil Engineering, Stony Brook University, Stony Brook, NY 11794, USA}
\author{Paolo Celli}%
\affiliation{Civil Engineering, Stony Brook University, Stony Brook, NY 11794, USA}
\author{Ian Tobasco}
\email{tobasco@uic.edu}
\affiliation{Mathematics, Statistics and Computer Science, University of Illinois at Chicago, Chicago, IL 60607, USA}
\author{Paul Plucinsky}
\email{plucinsk@usc.edu}
\affiliation{Aerospace and Mechanical Engineering, University of Southern California, Los Angeles, CA 90014, USA}%
\date{\today}


\begin{abstract}
Mechanical metamaterials exhibit exotic properties at the system level, that emerge from the interactions of many nearly rigid building blocks. Determining these emergent properties theoretically has remained an open challenge outside of a few select examples. %, including Miura-Ori origami~\cite{Nassar2017, Wei2013} and the classic rotating squares kirigami, the latter of which was recently shown to approximate conformal maps~\cite{czajkowski2022conformal}. 
Here, for a large class of periodic and planar kirigami, we provide a coarse-graining rule linking the design of the panels and slits to the kirigami's macroscale deformations. The procedure gives a system of nonlinear partial differential equations (PDE) expressing geometric compatibility of angle functions related to the motion of individual slits.  Leveraging known solutions of the PDE, we present excellent agreement between simulations and experiments across kirigami designs. The results reveal a surprising nonlinear wave response that persists even at large boundary loads, the existence of which is determined completely by the Poisson's ratio of the unit cell.
\end{abstract}

%\pacs{}
\maketitle
Mechanical metamaterials are solids with exotic %/unconventional/extraordinary
properties arising primarily from the geometry and topology of their mesostructures.
Recent studies have focused on creating metamaterials with unexpected shape-morphing capabilities \cite{Mullin_PRL_2007, Bertoldi_NATREVMATS_2017}, as this property is advantageous in applications spanning robotics, bio-medical devices, and space structures \cite{rafsanjani2019programming,kuribayashi2006self,velvaluri2021origami,zirbel2013accommodating}. A natural motif in this setting is a design that exhibits a mechanism~\cite{Pellegrino_IJSS_1986, Hutchinson2006,milton2013complete} or floppy mode~\cite{Lubensky2015}: the pattern, when idealized as an assembly of rigid elements connected along perfect hinges, can be activated by a continuous motion at zero energy. Yet mechanisms, even when carefully designed, rarely occur as a natural response to loads \cite{Coulais2018}. Instead, the complex elastic interplay of a metamaterial's building blocks results in an exotic soft mode of deformation. Characterizing soft modes is a difficult problem. 
Linear analysis hints at a rich field theory \cite{alibert2003truss,abdoul2018strain}, the nonlinear version of which has been uncovered only in a few  examples. 
Miura-Origami \cite{Schenk2013}, for instance, takes on a saddle like shape under bending, a feature linked to its auxetic behavior in the plane \cite{Wei2013}. The Rotating Squares (RS) \cite{grima2000auxetic} pattern exhibits domain wall motion \cite{Deng2020} and was recently linked to conformal soft modes \cite{czajkowski2022conformal}. 

In this Letter, we go far beyond any one example to establish a general coarse-graining rule determining the exotic, nonlinear soft modes of a large class of mechanism-based mechanical metamaterials inspired by kirigami. Our method includes the RS pattern as a special case, illuminating the particular nature of its conformal response. In general, we find a dichotomy between kirigami systems that respond by a nonlinear wave-like motion, and others including conformal kirigami that do not. We turn to introduce the specific systems treated here, and to describe our theoretical and experimental results. %Our results draw a clear link between system-level properties and the design of the pattern enabling its exotic soft response, which we demonstrate in experiments of various kirigami devices. %allowing us to tune the design to exhibit a surprising nonlinear wave-like response, which we then validate experimentally.

\begin{figure}
\centering
\includegraphics[scale = 1]{Fig_Intro.pdf}
\caption{Response of planar kirigami to the heterogeneous loading conditions indicated by the arrows. (a) Rotating Squares pattern; (b) another pattern featuring rhombi slits. Insets depict representative unit cells in reference and deformed configurations, with actuation angles $\xi$ and $\gamma$.}
\label{fig:IntroFig}
\end{figure}

\textit{Setup and overview of results\;--}\;Kirigami traditionally describes an elastic sheet with a pattern of cuts and folds \cite{Callens2017, Sussman_PNAS_2015, Wang2017}. More recently, the term has come to include cut patterns that, by themselves, produce complex deformations both in and out-of-plane \cite{ Cho_PNAS_2014, Rafsanjani2016, Tang2017, Blees2015, Rafsanjani2017, Dias_SOFTMATTER_2017, Konakovic2018, Celli2018, Choi2019}.\;Here, we study the 2D response of patterns with  repeating unit cells of four convex quadrilateral panels and four parallelogram slits. These patterns form a large model system for mechanism-based kirigami  \cite{Yang2018geometry, Singh2021,dang2022theorem}; their pure mechanism deformations are unit-cell periodic and counter-rotate the panels. Fig.\;\ref{fig:IntroFig} shows two examples, with the familiar RS pattern in (a). Each kirigami is free to deform as a mechanism under the loading, yet curiously neither does. Instead, exotic soft modes reveal themselves in the response.



%We focus on ``planar kirigami".\;In three dimensions, kirigami describes a pattern of cuts and folds~\cite{Callens2017, Castle2014, Sussman_PNAS_2015, Wang2017}. For planar media, this term is instead attributed to sheets with engineered cut patterns, which can deform either in-plane~\cite{} or out-of-plane~\cite{Blees2015, Rafsanjani2017, Dias_SOFTMATTER_2017, Konakovic2018, Celli2018, Choi2019} depending on the loads. Building on kinematic observations in \cite{Yang2018geometry, Choi2020compact, Singh2021}, our model system for planar kirigami is a pattern composed of a repeating unit cell of four convex quadrilateral panels and four parallelogram slits, and our focus is restricted to the planar (2D) deformations of these patterns. Importantly, as we show below, all patterns with this panel-slit geometry  exhibit a planar mechanism, and their mechanism deformations are unit-cell periodic and obtained by counter-rotating the panels. Fig.\;\ref{fig:IntroFig} shows two such examples, including the familiar RS pattern in (a), being pulled at the center at both ends. Neither exhibits its characteristic periodic mechanism response, even though it is free to do so.




%% ALTERNATIVE TO THE 1ST PARAGRAPH:
% Mechanical metamaterials are mesostructured solids comprising periodic or non-periodic arrays of unit cells. When subjected to external loads, the complex interactions between unit cells yield exotic global properties that can span both static and dynamic realms, and small and large deformation regimes. Metamaterials are particularly attractive for their shape-morphing capacity. The natural motif in this setting is a design that exhibits a mechanism~\cite{Pellegrino_IJSS_1986, Hutchinson2006} or floppy mode~\cite{Lubensky2015}: this pattern, when idealized as an assembly of rigid elements connected by perfect hinges, can be activated to achieve a continuous motion at zero energy. Mechanisms are often achieved  through designs inspired by  kirigami \cite{Callens2017}. For planar media, this term is usually attributed to sheets with engineered cut patterns, which can deform either in-plane~\cite{Cho_PNAS_2014, Tang2015, Rafsanjani2016, Tang2017, Deng2020} or out-of-plane~\cite{Blees2015, Rafsanjani2017, Dias_SOFTMATTER_2017, Konakovic2018, Celli2018, Choi2019}. Here, we focus on the 2D response of planar kirigami and on heterogeneous loading conditions, whose effects have only been analyzed for some well-known patterns.

%The most common modeling strategy to capture the mechanics of planar kirigami and metamaterials is by reducing them to networks of springs~\cite{Coulais2018, Deng2020}. However, there has been a growing interest in developing continuum models to elucidate the complex interplay between elasticity and preferred modes of deformation in these systems when subjected to non-uniform loading~\cite{Reis2018}. Examples are a theory for metamaterials in which holes are considered as geometric charges~\cite{Bar-Sinai2020}, and a modeling framework by Czajkowski et al., that rests on the assumption that metamaterials deform following conformal maps and is therefore only applicable to specific mesostructures~\cite{czajkowski2022conformal}.






 %Elastic frustration in kirigami systems is often modeled by spring-networks \cite{} based on ideas in \cite{}, yet this modeling leaves opaque a preference for global shape under non-ideal loading conditions. 
What determines soft modes? The key insight is that each unit cell is  approximately mechanistic, yielding a bulk actuation that varies slowly from cell to cell. To characterize the response, then, one must solve the geometry problem of ``fitting together'' many nearly mechanistic cells. %In the limit of a large number of panels, thi such geometric compatibility conditions} %If the deformation of a typical unit cell is approximately mechanistic, the answer comes down to enforcing geometric compatibility of a locally mechanistic response. 
Coarse-graining this problem, we derive a continuum field theory coupling the kirigami's \textit{macroscopic} or \textit{effective deformation} to the local motion of its unit cells. For each cell, we track the opening angle  $2\xi$ of its deformed slit, along with an angle $\gamma$ giving the cell's overall rotation as in Fig.\;\ref{fig:IntroFig}. %However, these deformations change gradually from cell to cell, breaking periodicity, notably in a way that appears consistent with some underlying smooth effective deformation. 
We derive a system of partial differential equations (PDEs) relating these angles, whose coefficients depend nonlinearly on $\xi$ as well as on the unit cell design. Solving this system exactly, we demonstrate an excellent match with experiments of different designs. 

Our theory divides  planar kirigami into two generic classes, which we term  \textit{elliptic} and \textit{hyperbolic} based on the so-called type of the coarse-grained PDE \cite{courant2008methods,evans10}. %\;The names are based on the governing PDE's ``type", a notion from the qualitative study of differential equations \cite{courant2008methods}.\;
Elliptic kirigami shows a characteristic decay in actuation away from  loads. In contrast, hyperbolic kirigami deforms with persistent actuation, via a nonlinear wave response. Surprisingly, this dichotomy turns out to be directly related to the Poisson's ratio of the unit cell---elliptic kirigami is auxetic, while hyperbolic kirigami is not. This result serves as a powerful demonstration of our continuum field theory, and adds to the emerging literature connecting Poisson's ratio to the qualitative behavior of  mechanical metamaterials \cite{Wei2013, nassar2017curvature,lebee2018fitting,rocklin2017transformable}.

%We identify this class and describe its linear and nonlinear waves, taking inspiration from analogies with hydrodynamics.

%Actually, this pattern is \textit{conformal} case --- kirigami whose  macroscopic deformations are locally dilational, angle-preserving maps.} The RS pattern is {\color{red}conformal} \cite{czajkowski2022conformal}, but so are many other patterns, as we explain.  

%{\color{red}
%In the process,  %It also serves as a .} %implied by geometric compatibility. %that we now  that we Altogether, our results comprise a general method for deriving and understanding the possible continuum behaviors of planar kirigami.   

%In this Letter, we build on this idea, showing that a continuum field theory, which couples an effective deformation to the unit cell kinematics through $\gamma$ and $\xi$, characterizes all possible shapes that  planar kirigami can achieve by a locally mechanistic response.
 
 %Our method is general; it not only confirms the conformal hypothesis  of the RS example from Fig.\ 1, but  describes the response of any pattern in our model system for planar kirigami. In addition, our method does not require understanding the precise form or details of the local elastic interactions between neighboring panels, but instead is based on pure geometry and a general formulation of kinematic compatibility of the effective deformation.




\textit{Coarse-graining planar kirigami\;--}\;We begin by introducing a general kirigami pattern consisting of a periodic array of unit cells, each having four quad panels and four parallelogram slits as in Fig.\;\ref{fig:RigidDef}(a). The most general setup is  as follows: start by selecting a seed of two quad panels connected at a corner point, rotate a copy of this seed $180^{\mathrm{o}}$, and connect it to the original seed to form a unit cell. Provided the resulting panels are disjoint, tessellating this unit cell along a Bravais lattice with basis vectors $\mathbf{s} = \mathbf{s}_1 + \mathbf{s}_2 + \mathbf{s}_3 + \mathbf{s}_4$ and $\mathbf{t} = \mathbf{t}_1 + \mathbf{t}_2 + \mathbf{t}_3 + \mathbf{t}_4$ gives a viable pattern. For an explanation of why this procedure is exhaustive, see supplemental section SM.1~\cite{suppl}. We fix one such pattern  and coarse-grain its kinematics. %The resulting pattern is determined by the vectors $\mathbf{s}_i, \mathbf{t}_i$, $i = 1,\ldots, 4$. 
 %and a simple geometric argument (Appendix A) explains why this procedure is exhaustive within  this recipe produces the full catalogue of designs in our model system for planar kirigami.
 
 \begin{figure}
\centering
\includegraphics[scale=1]{Fig_Def.pdf}
\caption{Coarse-graining a mechanism. (a) Vectors $\mathbf{s}_i,\mathbf{t}_i$ define the unit cell, which tessellates along $\mathbf{s}$ and $\mathbf{t}$ to produce the pattern. In a mechanism, panels rotate by the indicated rotation matrices $\mathbf{R}(\gamma \pm \xi)$.  (b) Coarse-graining through the lattice defines the effective deformation gradient $\mathbf{F}_{\text{eff}}$. Soft modes agree locally with this picture.}
\label{fig:RigidDef}
\end{figure}



%We now fix a kirigami pattern seeded as above and study its kinematics. 
First, we consider mechanisms. As our kirigami has parallelogram slits, its pure mechanism deformations are given by an alternating array of panel rotations specified by the  rotation matrices $\mathbf{R}(\gamma \pm\xi)$ in Fig.\;\ref{fig:RigidDef}(a). 
%This fact follows essentially from the parallelogram identity (Appendix B). 
(The angles $\gamma$ and $\xi$ agree with those of Fig.~\ref{fig:IntroFig}.) %when specialized in the case of a pure mechanism. 
To coarse-grain, we view the deformation as distorting the underlying Bravais lattice: from the top half of the figure, the original lattice vectors $\mathbf{s}$ and $\mathbf{t}$ deform to 
\begin{equation}
\begin{aligned}\label{eq:defBravais}
&\mathbf{s}_{\text{def}} = \mathbf{R}(\gamma) \big( \mathbf{R}(-\xi)  (\mathbf{s}_{1} + \mathbf{s}_2) + \mathbf{R}(\xi) ( \mathbf{s}_{3} + \mathbf{s}_4) \big),\\
&\mathbf{t}_{\text{def}} = \mathbf{R}(\gamma) \big( \mathbf{R}(-\xi)  (\mathbf{t}_{1} + \mathbf{t}_4) + \mathbf{R}(\xi)  (\mathbf{t}_{2} + \mathbf{t}_3) \big).
\end{aligned}
\end{equation}
In turn, this distortion can be encoded into the two-by-two matrix $\mathbf{F}_\text{eff}$ defined by $\mathbf{F}_{\text{eff}} \mathbf{s} = \mathbf{s}_{\text{def}}$ and $\mathbf{F}_{\text{eff}} \mathbf{t} = \mathbf{t}_{\text{def}}$, concretely linking Fig.\;\ref{fig:RigidDef}(a) and (b). We call $\mathbf{F}_\text{eff}$ the \textit{coarse-grained} or \textit{effective deformation gradient} associated with the mechanism. Evidently,
\begin{equation}
\begin{aligned}\label{eq:Feff}
\mathbf{F}_{\text{eff}} = \mathbf{R}(\gamma) \mathbf{A}(\xi)
\end{aligned}
\end{equation}
for a shape tensor $\mathbf{A}(\xi)$ that depends only on $\xi$ and on the vectors $\mathbf{s}_i$ and $\mathbf{t}_i$ defining the unit cell. This tensor will be made explicit in the examples to come (see SM.2~\cite{suppl} for the general formula).
%In words, the mechanism deforms the unit cell through alternating rotations, then tessellates it along the deformed Bravais lattice with the new basis vectors $\mathbf{s}_{\text{def}}$ and $\mathbf{t}_{\text{def}}$. 





Having coarse-grained the pattern's mechanisms, we now extend our viewpoint to its \textit{exotic soft modes of deformation}, whose elastic energy scaling is less than bulk. Specifically, we consider elastic effects accounting for the finite size and distortion of the inter-panel hinges, and show in SM.3~\cite{suppl} that the energy per unit area of the kirigami vanishes with an increasing number of cells provided its effective deformation $\mathbf{y}_{\text{eff}}(\mathbf{x})$ obeys 
%\begin{comment}
%is proportional (by an elastic modulus) to
%\begin{equation}
%   \int_{\Omega} Q_{\text{bulk}}\Big(\nabla \mathbf{y}_{\text{eff}}(\mathbf{x}) - %\mathbf{R}(\gamma(\mathbf{x})) \mathbf{A}(\xi(\mathbf{x}))  \Big)\,dA
%\end{equation}
%where $y_\text{eff}(x)$ gives the bulk, effective deformation of the kirigami.
%Importantly, this limit holds in a doubly asymptotic limit in which the panel lengths %are much smaller than the system size, and the hinges are much smaller in size than %the panels. 
%that enrich the kirigami's kinematics and show in SM.3~\cite{suppl} that certain slow variations of $\mathbf{F}_{\text{eff}}$ over the pattern admit an elastic energy scaling \textit{far less than bulk} (in a  limit of finely patterned kirigami).} 
%\end{comment}
\begin{equation}
\begin{aligned}\label{eq:effectiveDescription}
\nabla \mathbf{y}_{\text{eff}}(\mathbf{x}) = \mathbf{R}(\gamma(\mathbf{x})) \mathbf{A}(\xi(\mathbf{x})).
\end{aligned}
\end{equation} 
While this PDE is trivially solved by the pure mechanisms in (\ref{eq:Feff}), it admits many other exotic solutions whose effective deformation gradients $\nabla\mathbf{y}_\text{eff}(\mathbf{x})$ and angle fields $\gamma(\mathbf{x})$ and $\xi(\mathbf{x})$ vary across the sample. 
The PDE characterizes soft modes in a doubly asymptotic limit of finely patterned kirigami, in which the hinges are small relative to the panels and the number of panels is large.
%result characterizes the soft modes of all kirigami with four panels ...

As gradients are curl-free, it follows by taking the curl of (\ref{eq:effectiveDescription}) that (SM.4~\cite{suppl})
\begin{equation}
\begin{aligned}\label{eq:compat} 
 \nabla \gamma(\mathbf{x}) = \boldsymbol{\Gamma}(\xi(\mathbf{x})) \nabla \xi(\mathbf{x}) 
 \end{aligned}
 \end{equation}
 for  $\boldsymbol{\Gamma}(\xi) =  \frac{\mathbf{A}^T(\xi)\mathbf{A}'(\xi) }{\det \mathbf{A}(\xi)}  \mathbf{R}(\tfrac{\pi}{2})$.
 %that  directly relates changes in the slit openings across the pattern to changes in the rotations of the 4-panel cells.} 
Eq.\;(\ref{eq:compat}) is a PDE reflecting the  geometric constraint that every closed loop in the kirigami must remain closed. This PDE can sometimes be solved analytically for the angle fields, as we do in the examples below, but in general we imagine it will be solved numerically. After finding $\gamma(\mathbf{x})$ and $\xi(\mathbf{x})$,  $\mathbf{y}_{\text{eff}}(\mathbf{x})$ can be recovered from  (\ref{eq:effectiveDescription})  uniquely up to a translation.  Eqs.\;(\ref{eq:effectiveDescription}-\ref{eq:compat}) furnish a complete  effective description of the locally mechanistic kinematics of any planar kirigami with a unit cell of four quad panels and four parallelogram slits.
 
 
% \textit{Rhombi-slit kirigami\;--\;} 

%In the remainder of this Letter, we explore the implications of  the effective description (\ref{eq:effectiveDescription}-\ref{eq:compat}) on system level properties, as reflected by the patterns response to mechanical frustration; specifically, whether the pattern supports wave-like behavior or not. Simultaneously, we validate the theory against experiments to demonstrate its predictive power. 

\textit{Linear analysis, PDE type and Poisson's ratio\;--} %waves and auxetic kirigami\;--}
While the effective description (\ref{eq:effectiveDescription}-\ref{eq:compat}) is nonlinear, we can start to learn its implications for kirigami soft modes by linearizing about a pure mechanism. 
We do so first for the class of rhombi-slit kirigami, whose shape tensors $\mathbf{A}(\xi)$ are diagonal. This simplification greatly clarifies the exposition without compromising the generality of our results; we treat general patterns at the end of this section. %, including the ones in Fig.\;\ref{fig:IntroFig}, and 
%{\color{red}We state the results for general patterns.} %Rhombiincludes the patterns in Fig.\;\ref{fig:IntroFig} {\color{red}and} corresponds to diagonal shape tensors $\mathbf{A}(\xi)$
%This class .  
% This seemingly dramatic simplification of the design space will allow us to concretely illustrate the predictions of our theory.
 
 
 %} %Nevertheless, the results we obtain below extend \textit{mutatis mutandis} to the general kirigami patterns above, albeit with more tedious linear algebraic work (see SM.5~\cite{suppl}}). 
 
Per Fig.\;\ref{fig:PRatio}, a rhombi-slit kirigami is defined by parameters  $\lambda_{1}, \ldots, \lambda_4$ that can take any value in $[0,1]$, and an aspect ratio $a_r >0$: 
\begin{equation}
\begin{aligned}\label{eq:rhombiSlits}
&\mathbf{A}(\xi) = \mu_1(\xi) \mathbf{e}_1 \otimes \mathbf{e}_1 + \mu_2(\xi) \mathbf{e}_2 \otimes \mathbf{e}_2, \\
&\mu_1(\xi) = \cos \xi - \alpha \sin \xi,\quad \mu_2(\xi)  = \cos \xi + \beta \sin \xi, \\
&\alpha = a_r (\lambda_4 - \lambda_2), \quad\beta =  a_r^{-1} (\lambda_1 - \lambda_3).
\end{aligned}
\end{equation}
Note $\alpha$ and $\beta$ encode the geometry of the unit cell, $\mu_{1}(\xi)$ and $\mu_2(\xi)$ give the stretch or contraction of its sides under a mechanism, and  $\mathbf{e}_{1}$ and $\mathbf{e}_2$ are unit vectors along the initial slit axes, as in Fig.\;\ref{fig:PRatio}. Finally,  $\boldsymbol{\Gamma}(\xi)$ in (\ref{eq:compat}) satisfies 
\begin{equation}
    \begin{aligned}\label{eq:Gamma}
    \boldsymbol{\Gamma}(\xi) = \Gamma_{12}(\xi) \mathbf{e}_1 \otimes \mathbf{e}_2 + \Gamma_{21}(\xi) \mathbf{e}_2 \otimes \mathbf{e}_1
    \end{aligned}
\end{equation} 
for $\Gamma_{12}(\xi)  = -\mu_1'(\xi)/\mu_{2}(\xi)$ and $\Gamma_{21}(\xi) = \mu_2'(\xi)/ \mu_1(\xi)$. 
%$\Gamma_{ij}(\xi)  = -\mu_j'(\xi)/\mu_{i}(\xi)$. 
Eqs.\;(\ref{eq:rhombiSlits}-\ref{eq:Gamma}) follow from (\ref{eq:defBravais}-\ref{eq:Feff}) after choosing appropriate $\mathbf{s}_i$ and $\mathbf{t}_i$ (SM.2~\cite{suppl}).


Proceeding perturbatively, we write $\xi(\mathbf{x})=\xi_0+\delta\xi(\mathbf{x})$ and $\gamma(\mathbf{x})=\delta\gamma(\mathbf{x})$ for small angles $\delta \xi(\mathbf{x})$ and $\delta \gamma (\mathbf{x})$, and let $\mathbf{y}_{\text{eff}}(\mathbf{x}) = \mathbf{A}(\xi_0)\mathbf{x} + \mathbf{u}(\mathbf{A}(\xi_0)\mathbf{x})$ for a small  displacement $\mathbf{u}(\mathbf{y})$ about a pure mechanism with constant slit opening angle $2\xi_0$. 
(Taking $\gamma_0=0$ eliminates a free global rotation.) 
Expanding (\ref{eq:effectiveDescription}) to linear order and computing the strain $\boldsymbol{\varepsilon}(\mathbf{y}) =\tfrac{1}{2} (\nabla \mathbf{u}(\mathbf{y}) + \nabla \mathbf{u}^T(\mathbf{y}))$ yields
\begin{equation}
\begin{aligned}\label{eq:strain}
&\boldsymbol{\varepsilon}(\mathbf{A}(\xi_0)\mathbf{x}) = \delta \xi(\mathbf{x}) \left(\begin{array}{cc} \varepsilon_1(\xi_0) & 0 \\ 0 & \varepsilon_2(\xi_0) \end{array}\right)
\end{aligned}
\end{equation}
with $\varepsilon_i(\xi_0) = \mu_i'(\xi_0)/\mu_i(\xi_0)$, $i=1,2$. Similarly, expanding (\ref{eq:compat}) to linear order and taking its curl gives that
\begin{equation}
    \begin{aligned}\label{eq:linPDE}
    0=\big(\Gamma_{21}(\xi_{0})\partial_{1}^{2} -\Gamma_{12}(\xi_{0})\partial_{2}^{2}\big)\delta\xi(\mathbf{x}).
    \end{aligned}
\end{equation}
Both equations must hold for the  perturbation to be consistent with the effective theory. 

 \begin{figure}
\centering
\includegraphics[scale = 1]{Fig_PR2.pdf}
\caption{Effective Poisson's ratio as a function of slit actuation $\xi$ for different rhombi-slit kirigami. The plot fixes $\alpha = -0.9$ and varies $\beta$ from $0$ to $0.9$. The RS pattern on the lower left sits at the lower extreme $\beta = 0.9$. It is purely dilational ($\nu_{21} = -1$) and is auxetic for all $\xi$. The upper extreme $\beta=0$ arises with the design on the upper left, which is non-auxetic $(\nu_{21}>0)$ for all relevant $\xi >0$. Some designs transition between auxetic and non-auxetic behavior as a function of $\xi$.}%fact dictates each pattern's response to loads, even far into the nonlinear regime (see Fig.\;\ref{fig:FinalFig}).}
\label{fig:PRatio}
\end{figure}


 \begin{figure*}
\centering
\includegraphics[scale = 1]{Fig_Final2.pdf}
\caption{Comparison between theory and experiments of rhombi-slit kirigami. (a,d) Two $16\times16$ cell patterns prior to deformation, with opposite Poisson's ratios and types. Top row is non-auxetic and hyperbolic. Bottom row is auxetic and elliptic. (b,e) Left entries are experimental samples pulled along their centerlines. Right entries show simulations, based on exact solutions of the effective theory. (c,f) Annular deformations produced experimentally (left) and using the theory (right). Colormaps indicate the slit actuation angle $\xi(\mathbf{x})$, extracted from the experiment using the procedure in SM.7~\cite{suppl}.}
\label{fig:FinalFig}
\end{figure*} 

The ratio of principal strains in (\ref{eq:strain}) defines an \textit{effective Poisson's ratio} which turns out to be  directly related to the coefficients in (\ref{eq:linPDE}):
\begin{equation}\label{eq:firstPoissons}
    \begin{aligned}
    \nu_{21}(\xi_0) := -\frac{\varepsilon_2(\xi_0)}{\varepsilon_1(\xi_0)} = \frac{\Gamma_{21}(\xi_{0})}{\Gamma_{12}(\xi_0)}\frac{\mu_1^2(\xi_0)}{\mu_2^2(\xi_0)}.
    \end{aligned}
\end{equation}
This link has remarkable implications.\;Writing  (\ref{eq:linPDE})  as $\partial_2^2 \delta \xi(\mathbf{x}) = \tfrac{\mu_2^2(\xi_0)}{\mu_1^2(\xi_0)}\nu_{21}(\xi_0) \partial_1^2 \delta \xi(\mathbf{x})$ and applying standard PDE theory, we discover that the overall structure of the perturbations is governed by the sign of the Poisson's ratio, i.e., by whether the pattern is auxetic or not:
\begin{equation}
    \begin{aligned}
    \begin{cases}\label{eq:Poissons}
    \nu_{21}(\xi_0) < 0 \quad \text{ elliptic and auxetic},   \\
    \nu_{21}(\xi_0) > 0 \quad  \text{ hyperbolic and non-auxetic}.
    \end{cases}
    \end{aligned}
\end{equation}
This criterion is visualized in Fig.~3.



%The effective Poisson's ratio is still defined via the first expression in (\ref{eq:Poissons}), after putting the strain tensor $\textbf{\boldsymbol{\varepsilon}(\mathbf{A}(\xi_0)\mathbf{x}})$ in a principal frame (that generically varies with $\xi_0$). While the argument is essentially the same, the linear algebraic details are a bit cumbersome and are provided in SM.5~\cite{suppl}.} %Curiously, a similar link {\color{red}An effective Poisson's ratio, similar to (\ref{eq:firstPoissons}), couples to fine details of exotic soft modes for Miura-Ori \cite{Schenk2013, Wei2013,lebee2018fitting} and egg-box origami \cite{nassar2017curvature}. Here too, for planar kirigami, this coupling emerges and has} %The argument (in SM.5~\cite{suppl}) is more involved, since $\mathbf{A}(\xi)$ generically has off-diagonal components.} 

 

The terms hyperbolic and elliptic come from PDE theory where an equation's type, found by linearization, informs the structure of its solutions \cite{courant2008methods,evans10}. Here in the hyperbolic case,  (\ref{eq:linPDE}) is the classical wave
equation with wave speed $c=\tfrac{\mu_2(\xi_0)}{\mu_1(\xi_0)}\sqrt{\nu_{21}(\xi_0)}$, the $x_{1}$- and $x_{2}$-coordinates being like ``space'' and ``time''. Linearization predicts waves for small loads; motivated by this, we go on below to construct a branch of nonlinear wave solutions describing the hyperbolic kirigami in Fig.\;\ref{fig:IntroFig}(b).
In contrast, the RS pattern in Fig.\;\ref{fig:IntroFig}(a) is auxetic and so is elliptic.  Instead of waves, elliptic kirigami shows a decay in  actuation away from  loads. We highlight the strong maximum principle of elliptic PDEs \cite{evans10}: the maximum and minimum actuation in an elliptic kirigami must occur only at its boundary, lest it deform by a constant mechanism. No such principle holds for hyperbolic kirigami. 
 
Remarkably, the same coupling in (\ref{eq:Poissons}) between Poisson's ratio and PDE type holds for the general quad-based kirigami patterns treated here. 
 We sketch the main ideas to provide clarity on this important result (see SM.5~\cite{suppl} for details). 
 Linearizing about a mechanism as before leads to a strain   $\boldsymbol{\varepsilon}(\mathbf{A}(\xi_0)\mathbf{x})$ with eigenvalues $\delta \xi(\mathbf{x}) \varepsilon_i(\xi_0)$, $i=1,2$. 
 Passing to a principle frame, the effective Poisson's ratio  of the  pattern---which dictates its auxeticity---is still given by the first expression in (\ref{eq:firstPoissons}).
 Eq.\;(\ref{eq:linPDE}) becomes a general second order linear PDE $c_{ij}(\xi_0)\partial^2_{ij}\delta\xi(\mathbf{x}) =0$ (summation implied). It is elliptic or hyperbolic according to the sign of the discriminant of its coefficients. A coordinate transformation reveals (\ref{eq:Poissons}). 
 
\textit{Nonlinear analysis and examples\;--}\;The previous linear analysis addresses the character of the kirigami's response nearby a pure mechanism, but does not prescribe it at  finite loads. We now present several exact solutions of the nonlinear system (\ref{eq:effectiveDescription}-\ref{eq:compat}) which capture the deformations of the kirigami in Fig.\;\ref{fig:FinalFig}, far into the nonlinear response. Our solutions are based on known results from PDE theory, which we detail in SM.6~\cite{suppl} and summarize here. %{\color{red} We compare the deformations of two rhombi-slit kirigami patterns to exact solutions of the fully nonlinear effective description in (\ref{eq:effectiveDescription}-\ref{eq:compat}) in Fig.\;\ref{fig:FinalFig}. We consider two designs, $(\alpha, \beta) = (-0.9,0)$ and $(-0.9,0.9)$, consistent positive and negative Poisson's ratio, respectively throughout actuation. For each specimen, we conduct both center pulling  (Fig.\;\ref{fig:FinalFig}(b,e)) and annular type (Fig.\;\ref{fig:FinalFig}c,f) displacement tests.}  
Using them, we simulate the panel motions with an ansatz that rotates and translates the panels to fit the solution. Due to the finiteness of the sample, one may expect slight deviations between theory and experiment, which scale with the relative panel size. See SM.3~\cite{suppl} for more details.   %Our theory and simulations are validated against two experimental samples in Fig.\;\ref{fig:FinalFig}. %While the deformations are far from linear here, we show that their properties are still largely governed by whether the kirigami device is auxetic or not. 


\noindent\textit{(i)\;Nonlinear waves\;--\;}%Just as in the linear analysis above, the existence of nonlinear wave-like responses depends on the hyperbolicity of the kirigami device. 
 Fig.\;\ref{fig:FinalFig}(a)  shows the $\alpha=-0.9$, $\beta=0$ pattern from the top left of Fig.\;\ref{fig:PRatio}, which remains non-auxetic, thus hyperbolic, for $\xi \in  (0, 0.235 \pi)$.  The hyperbolicity is borne out through the existence of nonlinear \textit{simple wave solutions} to (\ref{eq:compat}),  defined by the criteria that $\xi=\xi(\theta(\mathbf{x}))$ and $\gamma = \gamma(\theta(\mathbf{x}))$ for a scalar function $\theta(\mathbf{x})$. As a result, the angles vary across envelopes of straight line segments called characteristic curves. The term ``simple wave'' comes from compressible gas dynamics, where the same functional form governs gas densities varying next to regions of constant density \cite{courant1999supersonic}. For kirigami, simple waves alleviate slit openings next to regions of uniform actuation in response to loads. 
 
The left part of Fig.\;\ref{fig:FinalFig}(b) shows the experimental specimen pulled at its left and right ends along its centerline.  Slits open by an essentially constant amount in a central diamond region (orange), and recede towards the specimen's corners. Note the  ``fanning out" of contours of constant slit actuation from where the loads are applied. A simulation based on simple wave solutions matches these features on the right of Fig.\;\ref{fig:FinalFig}(b). Its straight line contours are characteristic curves. 


\noindent\textit{(ii)\;Conformal maps\;--\;}Recent work~\cite{czajkowski2022conformal} has noted the relevance of conformal maps for kirigami. Adding to this discussion, and as an example of our more general elliptic class, we note using  \eqref{eq:rhombiSlits} that the only rhombi-slit kirigami designs that deform conformally ($\mu_1(\xi)=\mu_2(\xi)$ for all $\xi$ by definition \cite{do2016differential}) have $\alpha=-\beta$ and $\nu_{21}(\xi) = -1$.  This includes the RS pattern in Fig.\;\ref{fig:FinalFig}(d), fabricated according to the lower left $\alpha = -0.9$ design in Fig.\;\ref{fig:PRatio}. We highlight the RS pattern due to its dramatic shape-morphing. Conformal mappings are basic examples in complex analysis \cite{brown2009complex}, enabling  numerous solutions to (\ref{eq:compat}).

The left part of Fig.\;\ref{fig:FinalFig}(e) shows the RS pattern pulled at its left and right ends. Its slits open up dramatically at the loading points and remain closed at the corners: the largest and smallest openings are at the boundary, per the maximum principle. Contours of constant slit actuation form arcs around these points. On the right of Fig.\;\ref{fig:FinalFig}(e), we fit the deformed boundary of the pattern to a conformal map. The simulation recovers the locations where the slits are most open and closed, and qualitatively matches their variations in the bulk. 
%{\color{red}We attribute the deviations from experiments here and in the hyperbolic case) to us matching a coarse-grained theory to a finite sample.  
%As  explained in SM.4~\cite{suppl}, experimental samples are free deviate from the PDE constraint (\ref{eq:effectiveDescription}) on a scale $\sim$ ``cell-length"/``sample-length". This scale is  small but non-negligible for the samples in Fig.\;\ref{fig:FinalFig}.}


\noindent\textit{(iii)\;Annuli\;--\;}Though one may think of hyerperbolic and elliptic kirigami as a dichotomy, and this is true as far as auxeticity is concerned, we close by pointing out the existence of some special effective deformations that are ``universal" in that they occur for both.  One example is the annular deformation in Fig.\;\ref{fig:FinalFig}(c) and (f), which arises from  (\ref{eq:compat}) under the condition that $\xi(\mathbf{x})$ is either only a function of $x_1$ or of $x_2$.  All rhombi-slit kirigami patterns are capable of this deformation, as we demonstrate using the previous hyperbolic (c) and elliptic (f) designs. Note unlike the previous examples, these experiments are done using pure displacement boundary conditions.

%{\color{red} Unlike previous examples, these experiments are carried out by matching the boundary openings from the theoretical solution. This allows to filter out the effects of hinge elasticity, and reduces discrepancies with the theory.}

%The apparent differences are likely the result of higher-order elastic effects not captured by the effective theory. For instance, the panels are connected at small hinges, which undergo significant distortion due to the counter-rotations inherent to the mechanism. Suppressing the slit-actuation slightly can reduce the stresses in the hinges, albeit by potentially violating the PDE description slightly. This notwithstanding, we are attempting to match strains at large deformations in a pattern with significant strain-gradients -- the strains vary from $0$ to 35$\%$ across the sample. So the comparison between the effective theory and the experiment really is quite reasonable in its agreement. 

%\textit{Annular deformations\;--\;}The hyperbolic kirigami in  Fig.\;\ref{fig:FinalFig}(a) and the conformal kirigami in (d) exhibit contrasting responses when pulled along the centerline. As our final example, we show that this need not always be the case -- some effective deformations for planar kirigami are ``universal" in that they depend only superficially on the reference parameters of the unit cell. The idea is to look for solutions in which $\xi$ varies only in one dimension by writing $\xi(\mathbf{x}) = f(\mathbf{x} \cdot \mathbf{t})$ for a unit vector $\mathbf{t}$. In this setting, the compatibility condition in (\ref{eq:compat}) reduces to a first order ODE of the form $f'(s) = c_0g(f(s),\mathbf{t})$, for some constant $c_0$ and an explicit function $g(\cdot)$ derived in Appendix \ref{}. This ODE can be solved numerically for fairly generic families  of $(f(0),c_0, \mathbf{t})$.  In particular, $\mathbf{t} = \mathbf{e}_1$ or $\mathbf{e}_2$ corresponds to annuli deformations regardless of the parameters $\alpha$ and $\beta$. In Fig.\;\ref{fig:FinalFig}(c) and (f), we find striking agreement between the  annuli deformations produced theoretically and experimentally for the two patterns discussed previously. 

%\textit{Discussion\;--}\;


%This Letter derived a coarse-grained theory for the exotic soft modes of planar kirigami systems made of repeating panels and slits, and demonstrated excellent agreement between exact solutions and experiments. The theory entails a set of nonlinear PDEs governing the kirigami's effective deformation and its actuation angles, and shows how to categorize kirigami designs into two general types---hyperbolic and elliptic. 
%Hyperbolic kirigami deforms by a nonlinear wave-like response supporting regions of constant maximum actuation in its bulk; elliptic kirigami, which includes conformal kirigami, is instead characterized by long-range decay of its actuation away from loads. For rhombi-slit designs, these behaviors are directly linked to the effective Poisson's ratio of the unit cell: elliptic designs are auxetic, and hyperbolic ones are not. That  the Poisson's ratio should predict such fine details of a metamaterial's  response is not obvious from its definition. %as the ratio of incremental extensions in perpendicular directions.

%Whereas hyperbolic kirigami deforms by a nonlinear wave-like response, elliptic kirigami does not. The former is not governed by the maximum principle and can exhibit large regions of essentially constant actuation, whereas the latter actuates by an amount that is maximized where loads are applied and decays smoothly away from those points.  
%Surprisingly, these behaviors are linked to the sign of the Poisson's ratio in an especially simple way: elliptic kirigami, including conformal kirigami, are auxetic; hyperbolic kirigami are not. Thus, the sign of the Poisson's ratio predicts whether or not it supports a nonlinear wave-like response.

\textit{Discussion --} Looking forward, while our emphasis here was on the derivation of  coarse-grained PDEs expressing bulk geometric constraints, we set aside the important question of the forces underlying them. Understanding the inter-panel forces more closely should eventually lead to a complete continuum theory predicting exactly which exotic soft mode will arise in response to a given load. Our results show that the effective PDE system (\ref{eq:effectiveDescription}-\ref{eq:compat}) plays the dominant, constraining role. This is consistent with the conformal elasticity of Ref.~\cite{czajkowski2022conformal}.

More broadly, we expect that an effective PDE of a geometric origin exists to constrain the bulk behavior of mechanical metamaterials beyond kirigami. Such PDEs have been found for certain origami designs \cite{nassar2017curvature,lebee2018fitting}, via a differential geometric argument akin to our passage from (\ref{eq:effectiveDescription}) to (\ref{eq:compat}). In origami, one also finds a surprising coupling between the Poisson's ratio of the mechanisms and certain fine features of  exotic soft modes. Are such couplings universal? What about the role of heterogeneity \cite{Dudte2016,Celli2018,Choi2019,dang2022inverse}? Can coarse-graining lead to constitutive models for mechanical metamaterials, common to practical engineering \cite{khajehtourian2021continuum,mcmahan2021effective}, or to effective descriptions of their dynamics~\cite{Deng2017}? While there are many avenues left to explore, our work on the soft modes of planar kirigami highlights new physics and is a convincing step towards the discovery of a continuum theory for mechanical metamaterials at large.

%insight not easily explained by conventional discrete (spring-network) modeling. Coarse-graining also identifies the natural variables for continuum constitutive modeling \cite{khajehtourian2021continuum,mcmahan2021effective}, which is at the heart of practical engineering.  All told, our results comprise a framework for deriving and understanding the continuum behavior of mechanic metamaterials, with many avenues for further exploration.  %As such, effective continuum theories may become a new paradigm by which to explain the exotic properties in mechanical metamaterials.


%Violate the PDE and pay the price in bulk elastic energy; satisfy it and the elastic energy is much less than bulk. 
%At the same time, there appears to be an interesting and delicate elastic interplay at higher order between stretching and bending of hinges, which likely serves as a selection criteria under loads for which multiple PDE solutions are possible. So still much to explore. 
%{\color{red}[Paul: find the big picture, boil down to essential points]} The coarse-graining idea should also generalize.  We expect that origami systems \cite{Schenk2013,Wei2013,nassar2017curvature}, for instance, have a coarse-grained theory linking the design and kinematics of their unit cell to first and second fundamental forms describing their preferred shapes under bending. Finally, ...Previous PDEs have been derived in the context of some origami examples before \cite{}. The paradigm in mechanical metamaterials research has been to understand the purely geometric response of a unit cell, then resort to numerics by way of discrete modeling using spring-networks. 
%Yet, our results on effective descriptions of planar kirigami --- its unexpected wave-like response and the link to Poisson's ratio ---  provide new insights into the physics of these systems by means of coarse-graining, suggesting perhaps an update to this paradigm.

%We didn't talk about elasticity. The method derived here for coarse-graining generalizes easily to other situations. Similar ideas should apply to treat origami systems [cite Nassar, Maha]. At the heart of the method is the systematic use of the Cauchy-Born

%The Bravais lattice interpretation of the kirigami enjoys a certain familiarity to models of crystalline solids \cite{}.  In that setting, one often relates the distortion of crystals to continuum deformations through the Cauchy-Born rule, i.e., an assertion that the mapping between reference and deformed Bravais lattices is a continuum deformation gradient that well-describes the bulk motion of atoms in a crystal. Here, we apply the Cauchy-Born rule analogously to the mechanism deformations of planar kirigami. Specifically,  we consolidate the details embedded in the unit cell into an effective deformation gradient $\mathbf{F}_{\text{eff}}$ that takes the original Bravais lattice to the deformed lattice via $\mathbf{F}_{\text{eff}} \mathbf{s} = \mathbf{s}_{\text{def}}$ and $\mathbf{F}_{\text{eff}} \mathbf{t} = \mathbf{t}_{\text{def}}$.  

%punchline! What is the effective behavior of a  would expect the effective properties of mechanical meta-materials to arise from the. The paradigm has been to understand the purely geometric viewpoint of meta-materials and result to numerics. Our results point towards an update to this paradigm involving coarse-graining the behavior of a unit cell and enforcing compatibility at the macro-scale. 

















\textbf{Acknowledgment.} YZ and PP acknowledge support through PP's start-up package at University of Southern California. IN and PC acknowledge the support of the Research Foundation for the State University of New York, and thank Megan Kam of iCreate for fabrication support. IT was supported by NSF award DMS-2025000.




\bibliographystyle{apsrev4-1} % Tell bibtex which bibliography style to use
% \bibliographystyle{ieeetr}
\bibliography{bib}

\end{document}