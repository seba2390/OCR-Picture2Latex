
\documentclass[a4paper,11pt]{article}
%\pdfoutput=1 % if your are submitting a pdflatex (i.e. if you have
             % images in pdf, png or jpg format)

\usepackage{jcappub} % for details on the use of the package, please
                     % see the JCAP-author-manual
\usepackage[T1]{fontenc}
\usepackage{amsmath}
\usepackage{amsfonts}
\usepackage{amssymb}
\usepackage{mathrsfs}

\usepackage{graphicx}
\usepackage{color}
\usepackage{soul}
\usepackage{subfig}
\usepackage[normalem]{ulem}
\usepackage[utf8]{inputenc}
%\usepackage{authblk}
\usepackage{aas_macros}
%\usepackage{epstopdf}
\usepackage[T1]{fontenc} % if needed
\newcommand {\del}[1] {\textcolor{red}{\sout{#1}}}
\renewcommand{\vec}[1]{\boldsymbol{#1}}
\newcommand{\me}{\mathrm{e}}
\newcommand{\mi}{\mathrm{i}}
\newcommand{\dif}{\mathrm{d}}
\newcommand{\tabincell}[2]{\begin{tabular}{@{}#1@{}}#2\end{tabular}}

\usepackage{floatrow}
\DeclareFloatFont{tiny}{\scriptsize}
\floatsetup[table]{font=tiny}


\renewcommand\arraystretch{2}

\title{\boldmath A scenario  for the anisotropy of galactic cosmic rays related to nearby source and local regular magnetic field}

\author[a]{Ai-feng Li,}
\author[b,c]{Qiang Yuan,}
\author[d,1]{Wei Liu,\note{Corresponding author.}}
\author[d,e,1]{ and Yi-qing Guo \note{Corresponding author.}}



\affiliation[a]{College of Information Science and Engineering, Shandong Agricultural University, Taian 271018, China}
\affiliation[b]{Key Laboratory of Dark Matter and Space Astronomy, Purple Mountain Observatory, Chinese Academy of Sciences, Nanjing 210008, China}
\affiliation[c]{School of Astronomy and Space Science, University of Science and Technology of China, Hefei 230026, China}
\affiliation[d]{Key Laboratory of Particle Astrophysics, Institute of High Energy Physics, Chinese Academy of Sciences, Beijing 100049, China}
\affiliation[e]{University of Chinese Academy of Sciences, Beijing 100049, China}


\emailAdd{liuwei@ihep.ac.cn}
\emailAdd{guoyq@ihep.ac.cn}

\abstract{
In our recent work, we build a propagation scenario to simultaneously explain the  spectra and anisotropy of cosmic rays (CRs) by considering spatially dependent  propagation (SDP) model  and nearby Geminga supernova remnant (SNR) source. 
But the phase of anisotropy is still inconsistent with the experimental data. 
Recent observations of  CR anisotropy  show that the phase  is  consistent with local regular magnetic field (LRMF) observed by  Interstellar Boundary Explorer (IBEX) below 100 TeV,  
which indicates that diffusion along LRMF is important.
In this work, we  further introduce the LRMF and take into account the effect of corresponding anisotropic diffusion to explain the anisotropy of CRs.
We find that when the diffusion coefficient  perpendicular to the LRMF is much smaller than the parallel one, the phase of anisotropy  points to $\sim R.A.= 3^h$, which accords with experimental observation below 100 TeV.
We also analyze the influence of the ratio of perpendicular and parallel  diffusion coefficient  on the anisotropy  and the energy dependence of the ratio.
The results illustrate that with the decrease of perpendicular diffusion,  
the anisotropic phase changes from the direction of nearby source to the LRMF  below 100 TeV, meanwhile it changes from the galactic center (GC)  to opposite direction of LRMF above 100 TeV. 
When the perpendicular diffusion coefficient grows faster than the parallel one with energy, the diffusion approaches to the isotropic at high energy, the phase of anisotropy shifts from the LRMF to the GC above 100 TeV.
This could be helpful to ascertain the energy dependence of diffusion coefficients. 
%Overall,  we provide a good scenario for  anisotropy of CRs related to nearby Geminga SNR source and LRMF, in the energy region from 100 GeV to PeV.

}


\begin{document}
\maketitle
\flushbottom

\section{Introduction}
\label{sec:intro}

%It is widely believed that CRs below PeV originate from Galactic sources, presumably supernova remnants (SNRs) \citep{1934CoMtW...3...79B}. CRs can be accelerated to form non-thermal power law spectra, $dN/d{\cal R} \propto {\cal R}^{-\nu}$ by the diffusive shock acceleration mechanism of SNRs, with ${\cal R}$ is rigidity and $\nu$ is an injection power index \citep{2014NuPhS.256...65C,  2015ARNPS..65..245F, 2015A&ARv..23....3D,  2015CRPhy..16..674H}.
%After CRs are released by their sources,  they enter the Milky Way and  interact with irregular magnetic fields and interstellar gas, which could be described by a diffusion process and the diffusion coefficient $D({\cal R}) \propto {\cal R}^\delta$, with $\delta \sim 0.3 - 0.6$  inferred from the Boron-to Carbon ratio. After the diffusion transportation, CR spectrum further softens  into $\phi \propto {\cal R}^{-\nu -\delta}   $\citep{2017PhRvD..95h3007Y} .

%which the CR particles are assumed to isotropically diffuse, 

%In particular, It can't explain the hardening of CR spectra above a few hundred GeV/nucleon or  the anisotropy of CRs. 

Owing to the deflection of the Galactic magnetic field (GMF), CRs lose their original direction and become nearly isotropic.
However, small CR anisotropy with relative amplitudes at the order of $10^{-4} \sim 10^{-3}$ is still observed  at a wide energy range from 100 GeV to hundreds of TeV.
Numerous experiments such as Tibet AS$\gamma$   
\citep{2006Sci...314..439A,  Amenomori:2010yr,  Amenomori:2017jbv},
Super-Kamiokande \citep{
	2007PhRvD..75f2003G}, Milagro \citep{ 2008PhRvL.101v1101A, 2009ApJ...698.2121A}, IceCube/Ice-Top \citep{ 2010ApJ...718L.194A,
	2011ApJ...740...16A, 2012ApJ...746...33A, 2013ApJ...765...55A, 2016ApJ...826..220A}, ARGO-YBJ \citep{2013PhRvD..88h2001B, 2015ApJ...809...90B}, HAWC \citep{2014ApJ...796..108A}
provided the high-precision and two dimensional observations of CR anisotropy,
which show a complex relationship with energy.
Less than 100 TeV, the amplitude increases  first and then decreases, but increases again above 100 TeV. At the same time the phase reverses from $\sim R.A.= 3^h$ to the GC at an energy of $100 \sim 300 $ TeV.
The CR anisotropy may arise due to several causes. 
It may result from the uneven distribution of CR sources such as SNRs \citep{2006APh....25..183E, 2012JCAP...01..010B}
or be due to the LRMF which governs the propagation of CRs nearby the solar system \citep {2014Sci...343..988S, 2015PhRvL.114b1101M, 2016PhRvL.117o1103A}. 
%the anisotropy can also be induced through both large-scale and local magnetic field configurations,
The anisotropy can also be a pure kinetic effect called the Compton-Getting effect arising from the relative motion between the observer and the CR plasma \citep{1935PhRv...47..817C,1968Ap&SS...2..431G}.


Conventional CR propagation model  has been successfully explained
many observed phenomena, such as broken power-law energy spectra, secondary-to-primary ratios, the diffuse gamma-ray distribution, etc.
%However, It has also been challenged by more and more observations, including the hardening of CR spectra at ${\cal R} \sim 200~GV$ \citep{2019SciA....5.3793A, 2021PhRvL.126t1102A, 2015PhRvL.114q1103A, 2017PhRvL.119y1101A,2017ApJ...839....5Y} and  the anisotropy of CRs, etc. In particular, conventional CR propagation model   
However it fails to  explain  the anisotropy of CRs \citep{2012JCAP...01..011B,       2017PhRvD..96b3006L}. 
it is expected 
that the amplitude of anisotropy is proportional to the diffusion coefficient $D({\cal R}) \propto {\cal R}^\delta$, with $\delta \sim 0.3 - 0.6$ .
Thus, one can expect that the anisotropy amplitude will increase with energy.
But the expected anisotropy amplitude is larger than the observation  above hundreds of TeV \citep{2012JCAP...01..011B}. 
Furthermore, conventional CR propagation model expects the anisotropic phase point to GC, because SNRs are mostly in the inner galactic disk region. Nevertheless, the observed  phase of anisotropy varies with the energy, especially the phase below 100 TeV is obviously deviated from the GC. 


Based on the conventional propagation model, various improvements have been explored to explain the anisotropy problem, including  nearby source \citep {2006APh....25..183E, 2012JCAP...01..010B, 2013APh....50...33S, 2017PhRvD..96b3006L}, spatially dependent diffusion process \citep{2012PhRvL.108u1102E, 2012ApJ...752L..13T, 2016ApJ...819...54G}, ensemble fluctuations of CR sources, etc. 
Our recent work attempted to explain  anisotropy of CRs by involve the nearby  source  under SDP model\citep {2019JCAP...10..010L, 2019JCAP...12..007Q}. We found that  the anisotropic phase of CRs  is sensitive to the direction of the nearby source.
Because Geminga SNR is located close to the direction of anti GC and far from the galactic disk and it may suppress the expected large amplitude, so it is a prime candidate \citep {2017Sci...358..911A}. Involve the nearby Geminga SNR source can explain the spectra and amplitude of anisotropy very well, however it couldn't account for the phase of anisotropy which is inconsistent with the observation below 100 TeV \citep {2019JCAP...10..010L}.

In galaxy, there is  GMF  with the order of $ \sim 3\mu G$, that have two components including regular and the turbulent part. A PeV CR particle  with a nucleon number of $Z$ move in a circular motion  in GMF with a Larmor radius $r_L \simeq 0.4  E_{PeV}/(ZB_{3\mu G})$ pc \citep{2016PhRvL.117o1103A, 2017PrPNP..94..184A,  2016MNRAS.457.3975S}.
The Larmor radius of PeV CRs is much smaller than their scattering length.
The propagation of CRs is anisotropic when the regular magnetic field strength is comparable to or beyond that of the turbulent one.
Therefore, the anisotropy  may be induced by LRMF nearby solar system below PeV\citep {2009ApJ...703L..90B}.
%Therefore LRMF nearby solar system can be responsible for the anisotropy of CRs below PeV 
%The propagation of CRs in a regular magnetic field is anisotropic.
%The CR particles propagate along the parallel  and the perpendicular to the magnetic field with different diffusion coefficients.  
%The ratio of $D_\perp$to $D_\|$ is approximately equal to the ratio of turbulent to regular magnetic field.$(D_\|)$, $(D_\perp)$
Recently Interstellar Boundary Explorer  (IBEX) performed observations of neutral particles which unveiled LRMF \citep {2013ApJ...776...30F}. 
%The LRMF around the solar system inferred from IBEX  \citep {2013ApJ...776...30F} experiment
The LRMF is  along $(l,b) = (210.5^\circ,-57.1^\circ)$ with an uncertainty of $\sim 1.5^\circ$  within $20$ pc. The observation of anisotropic phase of CRs is coincident with  the direction of LRMF below 100 TeV.
Several literatures \citep {2014Sci...343..988S, 2015PhRvL.114b1101M, 2016PhRvL.117o1103A, 2020ApJ...892....6L} studied the relationship between anisotropy and LRMF.
% governs the propagation of CRs, which may be responsible for the anisotropy.

The deflecting effect of the magnetic field on the CR particles combined with nearby sources may explain the amplitude and phase of the anisotropy.
To demonstrate  this view, we introduce LRMF and it's corresponding anisotropic diffusion simultaneously considering the contribution of nearby Geminga SNR source under SDP model to account for anisotropy of CRs.
% considering the contribution  of both nearby Geminga SNR source and the LRMF and it's corresponding anisotropic diffusion  under spatially dependent diffusion model.
% based on the previous work \citep{2019JCAP...10..010L}.  
In this work, we first calculate the proton, helium and all-particle energy spectra to match the experimental data. 
Then we analyze the anisotropy of CRs and compare the results with recent work \citep{2019JCAP...10..010L}. 
We find that when the diffusion coefficient perpendicular to the LRMF is much smaller than the parallel one, the phase of anisotropy clearly points to LRMF, which is consistent with the experimental data.
We also investigate the influence of the ratio of perpendicular and parallel  diffusion coefficient  on the anisotropy  and the energy dependence of the ratio.
This could be helpful to ascertain the ratio and  energy dependence of diffusion coefficients.


%The rest of the paper is organized as follows. In Section 2, spatially dependent diffusion, local source and the anisotropic diffusion  and large-scale dipole anisotropy are introduced in detail.  
%The anisotropic diffusion  configuration in regular magnetic field is also described in detail.
%Section 3 first presents the results of proton, helium, all particles' spectrum compared with relevant observations, and provides the anisotropy of CRs with different diffusion coefficients, and  compares anisotropy of CRs with different ratios of perpendicular to parallel diffusion coefficients and different power index of  diffusion coefficent with rigidity dependence. Section 4 is reserved for the summary and conclusion.


\section{ Model}\label{sec: Model}
\subsection{Spatially dependent diffusion}

The region where CRs diffuse in the Milky Way is called a magnetic halo, which is usually approximated as a cylinder with its radial boundary equal to the 
Galactic radius, i.e. $R = 20$ kpc and its half thickness $z_h$ is about a few kpc. $z_h$  is  usually  determined by fitting the B/C ratio along with diffusion coefficient \citep {2007ARNPS..57..285S}. Both CR sources and the interstellar medium 
are usually assumed to be concentrated near the Galactic disk, whose average thickness $z_s$  is roughly $200$ pc. 

HAWC observations have recently that the diffusion coefficient of CRs near the galactic disk is at least two orders of magnitude smaller than the conventional one \citep{2017Sci...358..911A}.
Therefore, we work in aSDP frame \citep {2012ApJ...752L..13T, 2016ApJ...819...54G, 2016ChPhC..40a5101J}, whose diffusion coefficient are different in inner halo and outer halo. The parameterized diffusion coefficient we adopt is  \citep {2018PhRvD..97f3008G, 2018ApJ...869..176L}.
\begin{align}
\ D_{xx}( r,z,R) &=
D_0F(r,z)({\dfrac{\cal R}{ {\cal R}_0}})^{{\delta_0}F(r,z)}
\label{eq:D_{xx}}
\end{align}
$F(r,z)$ is parameterized as
\begin{align}
\begin{split}
F(r,z)= \left \{
\begin{array}{ll}
g(r,z)+[1-g(r,z)]{(\dfrac{z}{ \xi z_0})}^n,       &  |z|\leq\xi z_0 \\
1, & |z|>\xi z_0\\
\end{array}
\right.
\end{split}
\end{align}
where $g(r,z) = N_m/[1+f(r,z)]$, and $f(r,z)$ is the source density
distribution. The spatial
distribution of sources takes the form of SNR distribution
\citep {1996A&AS..120C.437C}. $f(r,z) \propto (r/r_\odot)^{1.69}
\exp[-3.33(r -r_\odot)/r_\odot] \exp(-|z|/z_s)$,
where $r_\odot = 8.5$ kpc and $z_s = 0.2$ kpc. In this work, we adopt numerical package DRAGON to solve the transport equation \citep {2008AIPC.1085..380E}.
The corresponding transport
parameters are shown in Table \ref{tab:transport}. 
\begin{table}
	\begin{center}
		\caption{parameters of SDP model.}
		\begin{tabular}{|ccccccccc|}
			\hline
			& $D_0$~~~    &   $\delta_0$~~~     &    $N_{\rm m}$~~~    &    $\xi$~~~   &    $n$~~~  &    $v_A$~~~    &    $z_h$ &  \\
			& [${\rm cm}^2 \cdot {\rm s}^{-1}$] & & & & & [${\rm km}\cdot{\rm s}^{-1}$] & [kpc]  &\\
			\hline
			& $4.87 \times 10^{28}$  & 0.58  & 0.62    & 0.1   & 4          & 6        &  5    &      \\
			\hline
		\end{tabular}
		\label{tab:transport}
	\end{center}
\end{table}

The injection spectrum of background sources is assumed to be a power-law
of rigidity with a high-energy exponential cutoff, $q({\cal R}) \propto
{\cal R}^{-\nu} \exp(-{\cal R}/{\cal R}_{\rm c})$. The cutoff rigidity
of each element could be either $Z$- or $A$-dependent. 

\subsection{Local source}

Green's function method is adopted to calculate the propagation of particles from the local source, assuming a spherical geometry with 
infinite boundary conditions \citep{2012JCAP...01..011B, 2013APh....50...33S}. 
The  CR density of local source as a function of space, rigidity, and time is described as
\begin{equation}
\phi(r,{\cal R},t)=\frac{q_{\rm inj}({\cal R})}{(\sqrt{2\pi}\sigma)^3}
\exp\left(-\frac{r^2}{2\sigma^2}\right),
\end{equation}
where $q_{\rm inj}({\cal R})\delta(t)\delta({\bf r})$ is the instantaneous injection spectrum of a point source, $\sigma({\cal R},t)=\sqrt{2D({\cal R})t}$ 
is the effective diffusion length within time $t$, $D({\cal R})$ is the 
diffusion coefficient which was adopted as the value nearby the solar system. The injection spectrum is aslo parameterized
as a cutoff power-law form, $q_{\rm inj}({\cal R})=q_0{\cal R}^{-\alpha}
\exp(-{\cal R}/{\cal R}'_{\rm c})$.
The normalization $q_0$ is obtained
through fitting to the CR energy spectra.
In this work, we selected the Geminga SNR source $(l,b= 194.3^\circ,-13^\circ)$, whose distance and age are $d = 330$ pc and $\tau = 3.4 \times 10^5$ years respectively \citep {2017Sci...358..911A, 2007Ap&SS.308..225F}. 
The corresponding injection parameters are given in table \ref{tab:para_inj}.
\begin{table*}
	\begin{center}
		\begin{tabular}{|c|ccc|ccc|}
			\hline
			& \multicolumn{3}{c|}{Background} & \multicolumn{3}{c|}{Local source} \\
			\hline
			%\toprule[1.5pt]
			Element & Normalization$^\dagger$ & ~~~$\nu$~~~  & ~~~$\mathcal R_{c}$~~~ & ~~~$q_0$~~~~~ & ~~~~~$\alpha$~~~ & ~~~${\cal R}'_c$~~~ \\
			\hline
			& $[({\rm m}^2\cdot {\rm sr}\cdot {\rm s}\cdot {\rm GeV})^{-1}]$ & & [PV] & [GeV$^{-1}$] & &  [TV] \\
			\hline
			p   & $1.91\times 10^{-2}$    & 2.34   &  7  & $8.28\times 10^{52}$  & 2.16 & 25 \\
			He & $1.43\times 10^{-3}$   & 2.27     &  7  & $2.35\times 10^{52}$  & 2.08  &  25  \\
			C   & $6.15\times 10^{-5}$   & 2.31    &  7  & $7.2\times 10^{50}$    & 2.13 &  25  \\
			N   & $7.67\times 10^{-6}$   & 2.34    &  7  & $1.13\times 10^{50}$  & 2.13 &   25  \\
			O   & $8.20\times 10^{-5}$   & 2.36    &  7  & $1.11\times 10^{51}$ & 2.13  &   25  \\
			Ne & $8.05\times 10^{-6}$   & 2.28   &  7  & $1.13\times 10^{50}$ & 2.13  &  25 \\
			Mg & $1.62\times 10^{-5}$   & 2.39     &  7  & $1.08\times 10^{50}$ & 2.13  &  25  \\
			Si & $1.28\times 10^{-5}$     & 2.37   &  7  & $1.05\times 10^{50}$ & 2.13  &   25  \\
			Fe & $1.23\times 10^{-5}$    & 2.29    &  7  & $2.20\times 10^{50}$ & 2.13   &  25  \\
			\hline
			%\bottomrule[1.5pt]
		\end{tabular}\\
		$^\dagger${The normalization is set at total energy $E = 100$ GeV.}
	\end{center}
	\caption{Injection parameters of the background and local source.}
	\label{tab:para_inj}
\end{table*}


\subsection{ Anisotropic diffusion}

Under the anisotropic diffusion model,
diffusion of CRs consists of  two components: parallel  and  perpendicular to the magnetic field. 
The diffusion tensor $D_{ij}$ associated with the magnetic field is written as
\begin{equation}
D_{ij}\,\equiv\,D_\perp\delta_{ij}\,+\,\big(D_\|-D_\perp\big)b_ib_j ~, ~ ~  
b_i = \dfrac{B_i}{|\vec{B}|}
\label{eq:D_ij_1}
\end{equation}
Where $b_i$ is the $i$-th component of the unit vector  \citep {1999ApJ...520..204G}.
Here $D_{\parallel}$ and $D_{\perp}$ are the diffusion coefficients aligned parallel and perpendicular to the regular magnetic field, respectively.

CR particles propagate in a magnetic field, $B=\delta B +B_0$, that is the sum  of a regular  $B_0$ field and a turbulent $\delta B$ field. Mean free path of the CR particles in the TeV region is $\sim20$ pc which is larger than the Lamore radius. According to   the quasi-linear theory 
\citep {1966ApJ...146..480J, 1968PhRvL..21...44J},  $\delta B(k)^2=\int{d^3\delta B(k)^2\ll|B_0|^2} $ at this scale.
The ratio between the perpendicular and parallel diffusion coefficients is expected to be as
% equation \ref{eq:Dration}% by
\begin{equation}
\dfrac{D_\perp}{D_\parallel} \sim \mathcal{F}(k) \sim \dfrac{\delta B(k)^2}{B_0^2} \ll 1 ~,
\label{eq:Dration}
\end{equation}
where $\mathcal{F}(k)$ is the normalized power of the turbulent modes with wave number $k$ \citep {2017JCAP...10..019C, 2015PhRvL.114b1101M, 2015ApJ...799..157B}.
However,  if $\delta B/B_0\sim1$, $D_\perp\simeq D_\|$, then the diffusion of CRs for higher energies is isotropic.
This ratio need to be calculated with the numerical simulation. 
%It was shown that  the perpendicular diffusion always has a steeper rigidity dependence than the parallel component \cite{2007JCAP...06..027D}.  %2016MNRAS.457.3975S}.

In this work, we adopt two different rigidity dependents $D_\parallel$ and $D_\perp$    following  \citep{2017JCAP...10..019C},
\begin{align}
D_\parallel &= D_{0\parallel} \left(\frac{\cal R}{{\cal R}_0} \right)^{\delta_\|} ~, \\
D_\perp &=\,D_{0\perp} \left(\frac{\cal R}{{\cal R}_0} \right)^{\delta_\perp} = \varepsilon D_{0\parallel} \left(\frac{\cal R}{{\cal R}_0} \right)^{\delta_\perp} ~,
\label{eq:DparaDperp}
\end{align}
which $\varepsilon = \dfrac{D_{0\perp}}{D_{0\parallel} }$ is the ratio between perpendicular and parallel diffusion coefficient at the reference rigidity ${\cal R}_{0}$. 
%As discussed above, in order to make a simulation physically meaningful,  
The parameters $\delta_\parallel$, $\delta_\perp$, ${\cal R}_{0}$, and $\varepsilon$ need to be carefully chosen to keep $D_\perp\lesssim D_\parallel$ within the concerned energy region(GeV - PeV). 



\subsection{ Large-scale dipole anisotropy}
In a simple model of isotropic
diffusion, the amplitude of the dipole anisotropy is usually defined with observed relative difference between the fluxes in the maximum and
minimum directions $\phi_{\rm max}$ and $\phi_{\rm max}$   
as \citep{2015PhRvL.114b1101M}
\begin{equation}
|\delta| = \dfrac{(\phi_{\rm max}-\phi_{\rm min})}{(\phi_{\rm max}+\phi_{\rm min})}=\dfrac{3D \cdot \nabla \psi}{v \psi} ~,
\end{equation}
It can be seen that the amplitude of the dipole anisotropy is proportional to the spatial gradient of the CR density $\nabla \psi$ and the diffusion coefficient $D$.

In anisotropic diffusion model,  its vector form can be written as \citep{2016PhRvL.117o1103A, 2020ApJ...892....6L}
\begin{equation}
{\delta} =  \dfrac{3}{v \psi} D_{ij}  \dfrac{ \partial \psi}{ \partial x_j} ~.
\end{equation}
The rigidity dependence of the  dipole amplitude results from both diffusion tensor  $D_{ij}$ and $\dfrac{ \nabla \psi}{ \psi}$.
For GeV–PeV CRs, %$D_\parallel>D_\perp$, 
parallel diffusion predominates, the anisotropic phase tends to the direction of magnetic field.
In order to study the influence of two diffusion coefficients ($D_\perp$ and $D_\parallel$ ) on anisotropy, we would choose several  sets of  parameters ($\varepsilon,~  \delta_\parallel,~  \delta_\perp$) to calculate the anisotropy.


\section{Results and Discussion}\label{sec:results}

\begin{figure*}[!ht]
	\includegraphics[width=0.48\textwidth]{Proton.eps}
	\includegraphics[width=0.48\textwidth]{Helium.eps}
	\caption{Energy spectra of protons (left) and helium nuclei (right). The data 
		points are taken from 
		DAMPE\citep {2019SciA....5.3793A, 2021PhRvL.126t1102A},
		AMS-02 \citep{2015PhRvL.114q1103A, 2017PhRvL.119y1101A},
		CREAM-III \citep{2017ApJ...839....5Y}, NUCLEON \citep{2017JCAP...07..020A}, KASCADE \citep{2005APh....24....1A} and KASCADE-Grande \citep{2013APh....47...54A} respectively.  The blue lines are the background fluxes, and the red lines are the fluxes from a nearby Geminga SNR source respectively. The black lines  represent the sum contributions of the background and nearby Geminga SNR source.
	}
	\label{fig:Fig1}
\end{figure*}


\begin{figure*}
	\centering
	\includegraphics[width=0.7\textwidth]{spec_all.eps}
	\caption{The all-particle spectra multiplied by $E^{2.6}$.
		The data points are taken from ref  \citep{2003APh....19..193H}.
		The solid lines with different colors are the model predictions of different mass groups, and the black solid line is the total contribution.
	}
	\label{fig:all_spec}
\end{figure*}

The LRMF within 20 pc has been measured accurately by IBEX experiment, which is much smaller than the size of the Milky Way \citep {2013ApJ...776...30F, 2017JCAP...10..019C}.
Therefore, when we simulate the CRs background using DRARGON public numerical packages \citep {2008AIPC.1085..380E}, the CRs  is approximately isotropic diffusion in the Milky Way. 
First, we calculate the proton and helium spectra. Figure \ref{fig:Fig1} shows the energy spectra of protons and helium obtained for the model calculation, with the red, blue and black lines representing the contributions from the local source, the background sources and the sum of them, respectively. It can be seen that the contribution of nearby Geminga SNR source can simultaneously account for the spectral hardening features at $\sim$ 200 GeV, and softening features at  $\sim$ 10 TeV. 
Through adding different compositions together, we  get the all-particle spectrum as shown in  Figure \ref{fig:all_spec}, which is well consistent with the observational data.
\begin{figure*}
	\includegraphics[width=0.98\textwidth]{amp_phcompare.eps}
	\caption{When $\varepsilon=0.01$, $\Delta\delta=0.32$, energy dependences of the amplitude (left) and phase (right)	of the dipole anisotropies together with the contribution from nearby Geminga SNR source and LRMF. 
		The red broken line indicates the result \citep{2019JCAP...10..010L}.
	}
	\label{fig:amp_ph}
\end{figure*}

The anisotropy of CRs depends on the sum of the CR flows from the background  and the local source. %after being anisotropically diffused in the LRMF.  
In this work, the parameters of parallel diffusion coefficient $D_\parallel$ are set as those shown in Table \ref{tab:transport}.
The CRs in the 100 GeV - PeV energy region require $D_\parallel> D_\perp$, therefore
we  set $\varepsilon=0.01$ and  $\Delta\delta=\delta_\perp-\delta_\parallel=0.32$, where $\Delta\delta$ is the difference between  $\delta_\perp$ and $\delta_\parallel$. 
Figure  \ref{fig:amp_ph} shows the comparison of the anisotropy of CRs with and without LRMF \citep{2019JCAP...10..010L},
%In Figure  \ref{fig:amp_ph}, 
where the solid black line presents anisotropy of the introduced LRMF and it's anisotropic diffusion effect combined with   nearby Geminga SNR source, and red broken line is  the result of the work \citep{2019JCAP...10..010L}, respectively. 
For $E<100$ TeV,  anisotropic phase with the LRMF clearly points to LRMF ($\sim R.A.= 3^h$ ), which is better in accordance with the experimental observation than the result without magnetic field. The nearby source dominates the observed anisotropies, although its flux is sub-dominant. The position of nearby source and the direction of LRMF together determine the anisotropic phase.  The phase flips at 100 TeV from  $\sim R.A.= 3^h$ to GC. For $E\gtrsim100$ TeV, the contribution from the local source and LRMF decrease significantly, and background become dominant instead. The phase of the anisotropy turns to the  GC, since CR sources are more abundant in the inner Galaxy.

\begin{figure*}

	\includegraphics[width=0.98\textwidth]{diffepsilon.eps}
	\caption{ Energy dependences of the amplitude (left) and phase (right)	of the  anisotropies for $\Delta\delta (\delta_\perp-\delta_\parallel)=0.32$ together with the contribution from nearby Geminga SNR source and local regular magnetic field.  The three black lines correspond to the results for three different $\varepsilon ( = 0.1,~ 0.01,~ 0.001)$.
	}
	\label{fig:diffepsilon}
\end{figure*}

Anisotropy of CRs is related to the diffusion coefficient, which is determined by the parameters $\varepsilon$, $\delta_\parallel$, $\delta_\perp$, and ${\cal R}_{0}$.
To study the induced anisotropy changes of CRs by the ratio of  perpendicular and parallel diffusion coefficients, we 
calculate the anisotropy of different $\varepsilon$ (0.1,  0.01,  0.001) at   $\Delta\delta = 0.32$.
% when  perpendicular  diffusion coefficient is not larger than parallel  one.
Figure \ref{fig:diffepsilon} shows the anisotropies of three different  $\varepsilon$ and demonstrates the amplitude and phase of anisotropy are significantly related to $\varepsilon$.
At the same energy, the amplitude of anisotropy increases with increasing of  $\varepsilon$, which is due to the amplitude is proportional to the diffusion coefficient.  
As  $\varepsilon$ decreases from 0.1 to 0.001, the phase of anisotropy shifts from the position of the nearby source to the LRMF below 100 TeV, that results from the parallel diffusion gradually  increases and the perpendicular  diffusion weakens.
At the same time, the phase of anisotropy shifts from the GC to opposite direction of  LRMF above  100 TeV. 
%In summary, both the phase and the amplitude agree well with the experimental data at about $\varepsilon=0.01$.

\begin{figure*}
	\includegraphics[width=0.98\textwidth]{diffdelta.eps}
	\caption{
		Energy dependences of the amplitude (left) and phase (right)	of the  anisotropies for $\varepsilon =0.01$ together with the contribution from nearby Geminga SNR source and LRMF.  The three black lines correspond to the results for three different $\Delta\delta (\delta_\perp-\delta_\parallel)( = 0, ~ 0.3, ~ 0.8)$.
	}
	\label{fig:diffdelta}
\end{figure*}	

In addition to  $\varepsilon$, we also study the  relationship between anisotropy and $\Delta\delta$.  Figure \ref{fig:diffdelta} shows the amplitude and phase of the anisotropy with different $\Delta\delta$ ( 0, 0.3 and 0.8) at $\varepsilon$=0.01. %imply the anisotropy features are sensitive to $\Delta\delta$ (Figure \ref{fig:diffdelta})
%, which is similar to $\varepsilon$. 
From  Figure \ref{fig:diffdelta} (left), we can see that with the increase of $\Delta\delta$, the  anisotropic amplitude of CRs rises, which is due to the diffusion coefficient become larger.
% as $\Delta\delta$ increases 
Figure \ref{fig:diffdelta} (right) illustrates that with the increase of $\Delta\delta$, anisotropic phase of CRs changes from  the LRMF to the direction of the nearby source below 100 TeV, while it changes from the opposite direction of LRMF to GC above 100 TeV.  It can be attributed to that the diffusion gradually transforms from anisotropic to the isotropic,  when perpendicular diffusion coefficient grows faster than the parallel one.
The current  experimental  observation around PeV energy range are still in dispute. The AS$\gamma$ measurement points to the GC, while the ICECUBE observation points to the LRMF. Further experimental measurements of anisotropic phase, such as LHAASO, will help to understand the energy dependence of the diffusion coefficient.


\section{Summary}\label{sec:summary}

The observation of CR  anisotropy  is hard to explain for a long time.
Our recent work attempted to account for the anisotropy of  CRs through the contribution of nearby Geminga SNR source under the SDP model. However the  anisotropic phase is about $\sim R.A.= 5^h$ below 100 TeV, which disagrees with  the experimental data.  
%Recently IBEX performed observations of neutral particles which unveiled the direction of LRMF nearby the solar system. 
LRMF observed by IBEX  is coincident with observational phase of anisotropy, which indicates that diffusion along LRMF is important. Here, we further take into account the  LRMF and the corresponding effect of anisotropic diffusion on CR anisotropy. 

We find that when the diffusion coefficient  perpendicular to the LRMF is much smaller than the parallel one, i.e. the CR propagation is principally along the magnetic field, the phase of anisotropy  changes from the direction of local source to the LRMF, which conforms with the observations below 100 TeV.

We further study the influence of the ratio of the perpendicular and parallel diffusion coefficients on anisotropy. When the perpendicular diffusion coefficient becomes smaller, the phase of anisotropy gradually shifts from the local source to the LRMF below 100 TeV, while it shifts from the GC to the opposite direction of LRMF above 100 TeV.

We also investigate the energy dependence of the ratio. When the perpendicular diffusion coefficient grows faster than the parallel one with energy, the diffusion approaches to the isotropic at high energy, the phase of anisotropy changes from the LRMF to the GC above 100 TeV. However the current measurements at that energy range are still in dispute. The Tibet AS$\gamma$ measurement points to the GC, while the ICECUBE observation points to the LRMF. We hope that the CR experiments, for example LHAASO, could measure the anisotropy phase precisely, which could be helpful to ascertain the energy dependence of diffusion coefficients.

Overall,  we provide a good scenario for  anisotropy of CRs related to nearby Geminga SNR source and LRMF inferred from IBEX, in the energy region from 100 GeV to PeV.




\section*{Acknowledgements}
This work is supported by the National Key $R\&D$ Program of China grant (2018YFA0404202) and  the National Natural Science Foundation of China (11963004, 11635011, 11875264, U1831208, U1738205, U2031110) and Shandong Province  Natural Science Foundation ( ZR2020MA095). 




\bibliographystyle{unsrt_update}
\bibliography{ref1}

\end{document}
