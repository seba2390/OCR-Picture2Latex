\vspace{-3mm}
\section{Symbolic MEC Decomposition}\label{sec:mecs}

\vspace{-1mm}
In this section we present a succinct description of the basic symbolic
algorithm for MEC decomposition and then present the main ideas
for the improved algorithm.

\vspace{-0.5mm}
\smallskip\noindent{\em Basic symbolic algorithm for MEC decomposition.} 
The basic symbolic algorithm for MEC decomposition maintains a set of identified
MECs and a set of candidates for MECs, initialized with the SCCs of the MDP.
Whenever a candidate is considered, either
(a)~it is identified as a MEC or
(b)~it contains vertices with outgoing random edges, which are then removed 
together with their random attractor from the candidate, and the SCCs
of the remaining sub-MDP are added to the set of candidates.
We refer to the algorithm as~\ref{alg:mecbasic}.

\vspace{-0.5mm}
\begin{prp}\label{prp:basicmecs}
Algorithm~\ref{alg:mecbasic} correctly computes the MEC decomposition of MDPs and requires $O(n^2)$ symbolic steps.
\end{prp}

\vspace{-0.5mm}
\smallskip\noindent{\em Improved symbolic algorithm for MEC decomposition.}
The improved symbolic algorithm for MEC decomposition uses the ideas of symbolic
lock-step search presented in Section~\ref{sec:lss}.
Informally, when considering a candidate that lost a few edges from the remaining
graph, we use the symbolic lock-step search to identify some bottom SCC.
We refer to the algorithm as~\ref{alg:mecimpr}.
Since all the important conceptual ideas regarding the symbolic lock-step search 
are described in Section~\ref{sec:lss}, we relegate the technical details to Appendix~\ref{sec:appmecs}.
We summarize the main result (proof in Appendix~\ref{sec:appmecs}).

\vspace{-0.5mm}
\begin{thm}[Improved Algorithm for MEC]\label{thm:improvedmecs}
Algorithm~\ref{alg:mecimpr} correctly computes the MEC decomposition of MDPs and requires $O(n\cdot \sqrt{m})$ symbolic steps.
\end{thm}
