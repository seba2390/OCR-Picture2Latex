\section{MDPs with Streett Objectives}\label{sec:mdps}

\smallskip\noindent{\bf Basic Symbolic Algorithm.}
We refer to the basic symbolic algorithm for MDPs with Streett objectives
as~\ref{alg:streettmdpbasic}, which is similar to the algorithm for graphs,
with SCC computation replaced by MEC computation.
The pseudocode of Algorithm~\ref{alg:streettmdpbasic} together with its
detailed description is presented in Appendix~\ref{sec:appmdps}. 

\begin{prp}\label{prp:basicmdps}
Algorithm~\ref{alg:streettmdpbasic} correctly computes the almost-sure winning set
in MDPs with Streett objectives and requires $O(n^2 \cdot \min(n,k))$ symbolic steps.
\end{prp}

\begin{rmk}
The above bound uses the basic symbolic MEC decomposition algorithm. 
Using our improved symbolic MEC decomposition algorithm, the above bound
could be improved to 
$O(n \cdot \sqrt{m} \cdot \min(n,k))$.
\end{rmk}

\smallskip\noindent{\bf Improved Symbolic Algorithm.}
We refer to the improved symbolic algorithm for MDPs with Streett objectives as~\ref{alg:streettmdpimpr}.
First we present the main ideas for the improved symbolic algorithm. Then we explain
the key differences compared to the improved symbolic algorithm for graphs. A thorough description
with the technical details and proofs is presented in Appendix~\ref{sec:appmdps}.
\begin{compactitem}
\item First, we improve the algorithm by interleaving the symbolic MEC computation with the detection
of bad vertices~\cite{ChatterjeeDHL16,Loitzenbauer16}. This allows to replace the
computation of MECs in each iteration of the while-loop with the
computation of SCCs and an additional random attractor computation.
\begin{compactitem}
\item {\em Intuition of interleaved computation.}
Consider a candidate for a good end-component $S$ after a random attractor
to some bad vertices is removed from it. After the removal of the random attractor,
the set $S$ does not have random vertices with outgoing edges. Consider that further $\bad(S) = \emptyset$ holds.
If $S$ is strongly connected and contains an edge, then it is a good end-component.
If $S$ is not strongly connected, then $\mdp[S]$ contains at least two SCCs and some
of them might have random vertices with outgoing edges. Since end-components are strongly connected and
do not have random vertices with outgoing edges, we have that (1) every good end-component is completely
contained in one of the SCCs of $\mdp[S]$ and (2) the random vertices of an SCC with outgoing edges
and their random attractor do not intersect with any good end-component (see Lemma~\ref{lem:eccontained}).
\item {\em Modification from basic to improved algorithm.}
We use these observations to modify the basic algorithm as follows:
First, for the sets that are candidates for good end-components, we do not maintain the property
that they are end-components, but only that they do not have random vertices with outgoing edges
(it still holds that every maximal good end-component is either already identified or 
contained in one of the candidate sets). Second, for a candidate set $S$, we repeat the removal of 
bad vertices until $\bad(S) = \emptyset$ holds before we continue with the 
next step of the algorithm. This allows us to make progress after the removal of bad vertices
by computing all SCCs (instead of MECs) of the remaining sub-MDP.
If there is only one SCC, then this is a good end-component (if it contains 
at least one edge). Otherwise (a) we remove from each SCC
the set of random vertices with outgoing edges and their random attractor
and (b) add the remaining vertices of each SCC as a new candidate set.

\end{compactitem}
\item Second, as for the improved symbolic algorithm for graphs, we use 
the symbolic lock-step search to quickly identify a top or bottom SCC every time
a candidate has lost a small number of edges since the last time its superset
was identified as being strongly connected. The symbolic lock-step search
is described in detail in Section~\ref{sec:lss}.
\end{compactitem}

\vspace{2mm}
Using interleaved MEC computation and lock-step search leads to a similar algorithmic
structure for Algorithm~\ref{alg:streettmdpimpr} as for our improved symbolic algorithm for graphs
(Algorithm~\ref{alg:streettgraphimpr}). The key differences are as follows:
First, the set of candidates for good end-components is initialized with the MECs of
the input graph instead of the SCCs. Second, whenever bad vertices are removed
from a candidate, also their random attractor is removed.
Further, whenever a candidate is partitioned into its SCCs, for each SCC, the
random attractor of the vertices with outgoing random edges is removed.
Finally, whenever a candidate $S$ is separated into
$C$ and $S \setminus C$ via symbolic lock-step search, the random attractor of
the vertices with outgoing random edges is removed from $C$, and the random
attractor of $C$ is removed from $S$.

\begin{thm}[Improved Algorithm for MDPs]\label{thm:improvedmdps}
Algorithm~\ref{alg:streettmdpimpr} correctly computes the almost-sure winning set
in MDPs with Streett objectives and requires \mbox{$O(n \cdot \sqrt{m \log n})$}
symbolic steps.
\end{thm}
