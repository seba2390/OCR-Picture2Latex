\vspace{-3mm}
\section{Graphs with Streett Objectives}\label{sec:graphs}

\vspace{-1mm}
\smallskip\noindent{\bf Basic Symbolic Algorithm.}
Recall that for a given graph (with $n$ vertices) and a Streett objective (with $k$ 
target pairs) each non-trivial strongly connected subgraph without bad vertices is
a good component. The basic symbolic algorithm for graphs with Streett objectives 
repeatedly removes bad vertices from each SCC and then recomputes the SCCs 
until all good components are found. The winning set then consists of the vertices
that can reach a good component. We refer to this
algorithm as~\ref{alg:streettgraphbasic}. For the pseudocode and more details
see Appendix~\ref{sec:appgraphs}.

\vspace{-0.5mm}
\begin{prp}\label{prp:basicgraphs}
Algorithm~\ref{alg:streettgraphbasic} correctly computes the winning set in
graphs with Streett objectives and requires $O(n \cdot \min(n,k))$ symbolic steps.
\end{prp}

\vspace{-0.5mm}
\smallskip\noindent{\bf Improved Symbolic Algorithm.}
In our improved symbolic algorithm we replace the recomputation of all SCCs
with the search for a new top or bottom SCC with Procedure~\ref{proc:lockstep} 
from vertices that have lost adjacent edges whenever there are not too many such 
vertices. We present the improved symbolic algorithm for graphs with Streett objectives
in more detail as it also conveys important intuition for the MDP case.
The pseudocode is given in Algorithm~\ref{alg:streettgraphimpr}.

\vspace{-0.5mm}
\smallskip\noindent{\em Iterative Refinement of Candidate Sets.}
The improved algorithm maintains a set~$\goodC$ of 
already identified good components that is initially empty
and a set~$\mathcal{\ec}$ of candidates for good components that is initialized with 
the SCCs of the input graph~$G$. The difference to the basic algorithm lies
in the properties of the vertex sets maintained in $\mathcal{\ec}$ and the way we identify 
sets that can be separated from each other without destroying a good component.
In each iteration one vertex set~$S$ is removed from $\mathcal{\ec}$ and, after 
the removal of bad vertices from the set, either identified as a good component or 
split into several candidate sets.
By Lemma~\ref{lem:eccontained} and Corollary~\ref{cor:geccontained}
the following invariant is maintained throughout the algorithm for the sets
 in $\goodC$ and $\mathcal{\ec}$.

\vspace{-0.5mm}
\begin{invariant}[Maintained Sets]\label{inv:gccontained}
The sets in $\mathcal{\ec} \cup \goodC$ are pairwise disjoint and for every good 
component~$\scc$ of $G$ there exists a set $Y \supseteq \scc$
such that either $Y \in \mathcal{\ec}$ or $Y \in \goodC$.
\end{invariant}

\vspace{-0.5mm}
\smallskip\noindent{\em Lost Adjacent Edges.}
In contrast to the basic algorithm, the subgraph induced by a set~$S$ contained 
in~$\mathcal{\ec}$ is not necessarily strongly connected. Instead, we remember
vertices of~$S$ that have lost adjacent edges since the last time a superset
of $S$ was determined to induce a strongly connected subgraph; vertices that lost 
incoming edges are contained in~$H_S$ and vertices that lost 
outgoing edges are contained in~$T_S$. 
In this way we maintain Invariant~\ref{inv:HT} throughout the algorithm,
which enables us to use Procedure~\ref{proc:lockstep} with the running time 
guarantee provided by Theorem~\ref{thm:lss}.

\begin{algorithm2e}[t!]
	\SetAlgoRefName{StreettGraphImpr}
	\caption{Improved Alg. for Graphs with Streett Obj.}
	\label{alg:streettgraphimpr}
	\SetKwInOut{Input}{Input}
	\SetKwInOut{Output}{Output}
	\BlankLine
	\Input{graph $G = (V, E)$ and Streett pairs 
	$\SP= \{(L_i, U_i) \mid 1 \le i \le k\}$
	}
	\Output
	{
	$\win{G, \Streett{\SP}}$
	}
	\BlankLine
	$\mathcal{\ec} \gets \sccalg(G)$; $\goodC \gets \emptyset$\label{l:gimpr:initstart}\;
	\lForEach{$\scc \in \mathcal{\ec}$}{
		$H_\scc \gets \emptyset$; $T_\scc \gets \emptyset$\label{l:gimpr:initend}
	}
	\While{$\mathcal{\ec} \ne \emptyset$\label{l:gimpr:whilestart}}{
		remove some~$S \in \mathcal{\ec}$ from $\mathcal{\ec}$\;
		$\badv \gets \bigcup_{1 \le i \le k : U_i \cap S = \emptyset} (L_i \cap S)$\label{l:gimpr:badstart}\;
		\While{$\badv \ne \emptyset$\label{l:gimpr:innerw}}{
			$S \gets S \setminus \badv$\;
			$H_S \gets (H_S \cup \post(\badv)) \cap S$\;
			$T_S \gets (T_S \cup \pre(\badv)) \cap S$\;
			$\badv \gets \bigcup_{1 \le i \le k : U_i \cap S = \emptyset} (L_i \cap S)$\;
			\label{l:gimpr:badend}
		}
		\If(\tcc*[h]{$G[S]$ contains at least one edge}){$\post(S) \cap S \ne \emptyset$\label{l:p1}}{ 
			\lIf{$\lvert H_S \rvert + \lvert T_S \rvert = 0$}{
				$\goodC \gets \goodC \cup \set{S}$\label{l:gimpr:good1}
			}\ElseIf{$\lvert H_S \rvert + \lvert T_S \rvert \ge \sqrt{m / \log n}$\label{l:gimpr:basicstart}}{
				delete $H_S$ and $T_S$\;
				$\mathcal{\scc} \gets \sccalg(G[S])$\;
				\lIf{$\lvert \mathcal{\scc} \rvert = 1$}{
					$\goodC \gets \goodC \cup \set{S}$\label{l:gimpr:good2}
				}\Else{
					\lForEach{$\scc \in \mathcal{\scc}$}{
						$H_\scc \gets \emptyset$; $T_\scc \gets \emptyset$
					}
                    $\mathcal{\ec} \gets \mathcal{\ec} \cup \mathcal{\scc}$\;
				}
				\label{l:gimpr:basicend}
			}\Else{\label{l:gimpr:lssbeg}
				($\scc$, $H_S$, $T_S$) $\gets $\ref{proc:lockstep}($G$, $S$, $H_S$, $T_S$)\;
				\lIf{$\scc = S$}{
					$\goodC \gets \goodC \cup \set{S}$\label{l:gimpr:good3}
				}\Else(\tcc*[h]{separate $\scc$ and $S \setminus \scc$}){
					$S \gets S \setminus \scc$\;
					$H_\scc \gets \emptyset$; $T_\scc \gets \emptyset$\;
					$H_S \gets (H_S \cup \post(\scc)) \cap S$\label{l:p2}\;
					$T_S \gets (T_S \cup \pre(\scc)) \cap S$\label{l:p3}\;
					$\mathcal{\ec} \gets \mathcal{\ec} \cup \set{S} \cup \set{\scc}$\;
				\label{l:gimpr:lssend}
				}
			}
		}
		\label{l:gimpr:whileend}
	}
	\Return{$\reachG{G}{\bigcup_{\scc \in \goodC} \scc}$}\;
\end{algorithm2e}

\vspace{-0.5mm}
\smallskip\noindent{\em Identifying SCCs.}
Let $S$ be the vertex set removed from $\mathcal{\ec}$ in a fixed iteration of 
Algorithm~\ref{alg:streettgraphimpr} after the removal of bad vertices in 
the inner while-loop. First note that if $S$ is strongly connected and contains
at least one edge, then it is a good component. If the set $S$ was already identified
as strongly connected in a previous iteration, i.e., $H_S$ and $T_S$ are empty,
then $S$ is identified as a good component in line~\ref{l:gimpr:good1}.
If many vertices of $S$ have lost adjacent edges since the last time a super-set
of $S$ was identified as a strongly connected subgraph, then 
the SCCs of $G[S]$ are determined 
as in the basic algorithm. To achieve the optimal 
asymptotic upper bound, we say that many vertices of $S$ have 
lost adjacent edges when we have 
$\lvert H_S \rvert + \lvert T_S \rvert \ge \sqrt{m / \log n}$, while lower 
thresholds are used in our experimental results.
Otherwise, if not too many vertices of~$S$ lost
adjacent edges, then we  start a symbolic \emph{lock-step search} for top
SCCs from the vertices of~$H_S$ and for bottom SCCs from the vertices
of~$T_S$ using Procedure~\ref{proc:lockstep}. 
The set returned by the procedure is either a top or a bottom SCC $\scc$ of
$G[S]$ (Theorem~\ref{thm:lss}). Therefore we can from now on consider $\scc$
and $S \setminus \scc$ separately, maintaining
Invariants~\ref{inv:HT} and~\ref{inv:gccontained}. 

\vspace{-0.5mm}
\smallskip\noindent{\em Algorithm~\ref{alg:streettgraphimpr}.} A succinct description of the 
pseudocode is as follows: 
Lines~\ref{l:gimpr:initstart}--\ref{l:gimpr:initend} initialize the set of 
candidates for good components with the SCCs of the input graph.
In each iteration of the main while-loop one candidate is considered and the following
operations are performed:
(a)~lines~\ref{l:gimpr:badstart}--\ref{l:gimpr:badend} iteratively remove all bad vertices; if afterwards
the candidate is still strongly connected (and contains at least one edge), it is identified as a good component
in the next step; otherwise it is partitioned into new candidates in one of the following ways:
(b)~if many vertices lost adjacent edges, lines~\ref{l:gimpr:basicstart}--\ref{l:gimpr:basicend} partition the
candidate into its SCCs (this corresponds to an iteration of the basic algorithm);
(c)~otherwise, lines~\ref{l:gimpr:lssbeg}--\ref{l:gimpr:lssend} use symbolic 
lock-step search to partition the candidate into one of its SCCs and the remaining vertices.
The while-loop terminates when no candidates are left.
Finally, vertices that can reach some good component are returned.
We have the following result (proof in Appendix~\ref{sec:appgraphs}).

\vspace{-0.5mm}
\begin{thm}[Improved Algorithm for Graphs]\label{thm:improvedgraphs}
Algorithm~\ref{alg:streettgraphimpr} correctly
computes the winning set in graphs with Streett objectives and requires
$O(n \cdot \sqrt{m \log n})$ symbolic steps.
\end{thm}
