\vspace{-4mm}
\section{Introduction}\label{sec:intro}
\vspace{-1mm}

In this work we present faster symbolic algorithms for graphs and Markov decision 
processes (MDPs) with strong fairness objectives.
For the fundamental model-checking problem, the input consists of a {\em model} and 
{\em a specification}, and the algorithmic verification problem is to check whether 
the model {\em satisfies} the specification. 
We first describe the specific model-checking problem we consider and then our 
contributions. 

\vspace{-0.5mm}
\smallskip\noindent{\em Models: Graphs and MDPs.}
Two standard models for reactive systems are graphs and Markov decision processes (MDPs).
Vertices of a graph represent states of a reactive system, edges represent transitions 
of the system, and infinite paths of the graph represent non-terminating trajectories 
of the reactive system.
MDPs extend graphs with probabilistic transitions that represent reactive systems with 
uncertainty. 
Thus graphs and MDPs are the de-facto model of reactive systems with nondeterminism,
and nondeterminism with stochastic aspects, respectively~\cite{ClarkeBook,baierbook}.

\vspace{-0.5mm}
\smallskip\noindent{\em Specification: Strong Fairness (aka Streett) Objectives.}
A basic and fundamental property in the analysis of reactive systems is the 
{\em strong fairness condition}, which informally requires that if events are enabled infinitely often, then 
they must be executed infinitely often.
More precisely, the strong fairness conditions (aka Streett objectives) consist of 
$k$ types of requests and corresponding grants, and the objective requires that for each 
type if the request happens infinitely often, then the corresponding grant must also happen
infinitely often. 
After safety, reachability, and liveness, the strong fairness condition is one of the most
standard properties that arise in the analysis of reactive systems, and chapters 
of standard textbooks in verification are devoted to it (e.g., 
\cite[Chapter~3.3]{ClarkeBook},~\cite[Chapter~3]{MPProgress},~\cite[Chapters~8,~10]{AlurHenzingerBook}). 
Moreover, all $\omega$-regular objectives can be described by Streett objectives, e.g., 
LTL formulas and non-deterministic $\omega$-automata can be translated to
deterministic Streett automata~\cite{Safra88} and efficient translation
has been an active research area~\cite{ChatterjeeGK13,EsparzaK14,KomarkovaK14}. 
Thus Streett objectives are a canonical class of objectives that arise in verification.

\vspace{-0.5mm}
\smallskip\noindent{\em Satisfaction.} 
The basic notions of satisfaction for graphs and MDPs are as follows:
For graphs the notion of satisfaction requires that there is a trajectory (infinite path) 
that belongs to the set of paths described by the Streett objective.
For MDPs the satisfaction requires that there is a policy to resolve the nondeterminism 
such that the Streett objective is ensured almost-surely (with probability~1).
Thus the algorithmic model-checking problem of graphs and MDPs with Streett objectives is a 
core problem in verification.

\vspace{-0.5mm}
\smallskip\noindent{\em Explicit vs Symbolic Algorithms.}
The traditional algorithmic studies consider {\em explicit} algorithms that 
operate on the explicit representation of the system. In contrast, 
{\em implicit} or {\em symbolic} algorithms only use a set of predefined operations 
and do not explicitly access the system~\cite{ClarkeGP99}.
The significance of symbolic algorithms in verification is as follows:
to combat the state-space explosion, large systems must be succinctly represented 
implicitly and then symbolic algorithms are scalable, 
whereas explicit algorithms do not scale as it is computationally too expensive
to even explicitly construct the system.

\vspace{-0.5mm}
\smallskip\noindent{\em Relevance.} 
In this work we study symbolic algorithms for graphs and MDPs with Streett objectives.
Symbolic algorithms for the analysis of graphs and MDPs are at the heart of many state-of-the-art
tools such as SPIN, NuSMV for graphs~\cite{SPIN,NUSMV} and PRISM, LiQuor, Storm for MDPs~\cite{PRISM,LIQUOR,STORM}.
Our contributions are related to the algorithmic complexity of graphs and MDPs with 
Streett objectives for symbolic algorithms. 
We first present previous results and then our contributions.

\vspace{-0.5mm}
\smallskip\noindent{\em Previous Results.}
The most basic algorithm for the problem for graphs is based on repeated SCC (strongly
connected component) computation, and informally can be described as follows:
for a given SCC, (a)~if for every request type that is present in the SCC 
the corresponding grant type is also present in the SCC, then the SCC is identified 
as ``good'', (b)~else vertices of each request type that has no corresponding
grant type in the SCC are removed, and the algorithm recursively proceeds
on the remaining graph.
Finally, reachability to good SCCs is computed. 
The current best-known symbolic algorithm for SCC computation requires $O(n)$ symbolic
steps, for graphs with $n$ vertices~\cite{GentiliniPP08}, and moreover, the algorithm is optimal~\cite{ChatterjeeDHL18}.
For MDPs, the SCC computation has to be replaced by MEC (maximal end-component) computation,
and the current best-known symbolic algorithm for MEC computation requires $O(n^2)$ symbolic steps.
While there have been several explicit algorithms for graphs with Streett 
objectives~\cite{HenzingerT96,ChatterjeeHL15}, MEC computation~\cite{ChatterjeeH11,ChatterjeeH12,ChatterjeeH14}, 
and MDPs with Streett objectives~\cite{ChatterjeeDHL16}, as well as symbolic algorithms for MDPs with 
B\"uchi objectives~\cite{ChatterjeeHJS13}, the current best-known bounds for symbolic algorithms 
with Streett objectives are obtained from the basic algorithms, which are $O(n \cdot \min(n,k))$ 
for graphs and $O(n^2\cdot \min(n,k))$ for MDPs, where $k$ is the number of types of request-grant pairs.

\vspace{-0.5mm}
\smallskip\noindent{\em Our Contributions.}
In this work our main contributions are as follows:
\begin{compactitem}
\item We present a symbolic algorithm that requires $O(n \cdot \sqrt{m \log n})$ symbolic steps,
both for graphs and MDPs, where $m$ is the number of edges. 
In the case $k=O(n)$, the previous worst-case bounds are quadratic ($O(n^2)$) for graphs and 
cubic ($O(n^3)$) for MDPs.
In contrast, we present the first sub-quadratic symbolic algorithm both for graphs as well as MDPs.
Moreover, in practice, since most graphs are sparse (with $m=O(n)$), the worst-case bounds of 
our symbolic algorithm in these cases are $O(n \cdot \sqrt{n\log n})$.
Another interesting contribution of our work is that we also present an $O(n \cdot \sqrt{m})$ symbolic 
steps algorithm for MEC decomposition, which is relevant for our results as well as of 
independent interest, as MEC decomposition is used in many other algorithmic problems 
related to MDPs.
Our results are summarized in Table~\ref{tab:symb:comparison}.

\item While our main contribution is theoretical, based on the algorithmic insights we 
also  present a new symbolic algorithm implementation for graphs and MDPs with Streett objectives. 
We show that the new algorithm improves (by around 30\%) the basic algorithm 
on several academic benchmark examples from the VLTS benchmark suite~\cite{VLTS}.
\end{compactitem}

\vspace{-1mm}
\begin{table}
\renewcommand{\arraystretch}{1.3}
\caption{Symbolic algorithms for Streett objectives and MEC decomposition.}\label{tab:symb:comparison}
\vspace{0.5mm}
\centering
\begin{tabular}{@{}llll@{}}
\toprule
& \multicolumn{2}{c}{Symbolic Operations}\\
\cmidrule{2-4}
Problem & Basic Algorithm & Improved Algorithm & Reference \\
\midrule
Graphs with Streett  & $O(n \cdot \min(n, k))$ & \boldmath$\mathbf{O(n \sqrt{m \log n})}$ & Theorem~\ref{thm:improvedgraphs}\\
MDPs with Streett  & $O(n^2 \cdot \min(n, k))$ & \boldmath$\mathbf{O(n \sqrt{m \log n})}$ & Theorem~\ref{thm:improvedmdps} \\
MEC decomposition & $O(n^2)$ & \boldmath$\mathbf{O(n \sqrt{m})}$ & Theorem~\ref{thm:improvedmecs} \\
\bottomrule
\end{tabular}
\end{table}
\setlength\intextsep{0em}
\smallskip

\vspace{-0.5mm}
\smallskip\noindent{\em Technical Contributions.}
The two key technical contributions of our work are as follows:
\begin{compactitem}
\item \emph{Symbolic Lock Step Search:} We search for newly emerged SCCs by a 
local graph exploration around vertices that lost adjacent edges. 
In order to find small new SCCs first, all searches are conducted ``in parallel'', i.e., in lock-step, 
and the searches stop as soon as the first one finishes successfully. 
This approach has successfully been used to improve explicit algorithms~\cite{HenzingerT96,ChatterjeeJH03,ChatterjeeH12,ChatterjeeDHL16}.
Our contribution is a non-trivial symbolic variant (Section~\ref{sec:lss}) 
which lies at the core of the theoretical improvements.

\item \emph{Symbolic Interleaved MEC Computation:}
For MDPs the identification of vertices that have to be removed can be interleaved with the computation of MECs such that in each iteration the computation of SCCs instead of MECs is sufficient to 
make progress~\cite{ChatterjeeDHL16}. We present a symbolic variant of this interleaved computation. 
This interleaved MEC computation is the basis for applying the lock-step search to MDPs.
\end{compactitem}
