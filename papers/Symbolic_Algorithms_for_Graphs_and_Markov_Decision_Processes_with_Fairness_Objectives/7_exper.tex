\section{Experiments}\label{sec:exper}

We present a basic prototype implementation of our algorithm and compare against 
the basic symbolic algorithm for graphs and MDPs with Streett objectives.

\smallskip\noindent{\em Models.}
We consider the academic benchmarks from the VLTS benchmark suite~\cite{VLTS}, which gives
representative examples of systems with nondeterminism, and has been used in previous
experimental evaluation (such as~\cite{BarnatCP11,ChatterjeeHJS13}).

\smallskip\noindent{\em Specifications.}
We consider random LTL formulae and use the tool Rabinizer~\cite{KomarkovaK14} to 
obtain deterministic Rabin automata. Then the negations of the formulae
give us Streett automata, which we consider as the specifications.

\smallskip\noindent{\em Graphs.}
For the models of the academic benchmarks, we first compute SCCs, as all 
algorithms for Streett objectives compute SCCs as a preprocessing step. 
For SCCs of the model benchmarks we consider products with the specification Streett
automata, to obtain graphs with Streett objectives, which are the benchmark
examples for our experimental evaluation. The number of transitions in the
benchmarks ranges from $300$K to $5$Million.

\smallskip\noindent{\em MDPs.}
For MDPs, we consider the graphs obtained as above and consider a 
fraction of the vertices of the graph as random vertices, which is 
chosen uniformly at random. 
We consider $10\%$, $20\%$, and $50\%$ of the vertices as random vertices
for different experimental evaluation.

\setlength{\abovecaptionskip}{1pt}
\begin{figure}[t]
\centering
\includegraphics[width=0.5\textwidth]{fig/Sgraph}
\caption{Results for graphs with Streett objectives.}
\label{fig:graphs}
\end{figure}
\setlength{\abovecaptionskip}{6pt}
\begin{figure}[H]
\begin{center}
	\vspace{-2mm}
     \subfloat[10\% random vertices\label{fig:mdp10}]{
       \includegraphics[width=0.5\textwidth]{fig/Smdp10}
     }
     \subfloat[20\% random vertices\label{fig:mdp20}]{
       \includegraphics[width=0.5\textwidth]{fig/Smdp20}
     }
     \hfill
	\vspace{-2mm}
     \subfloat[50\% random vertices\label{fig:mdp50}]{
       \includegraphics[width=0.5\textwidth]{fig/Smdp50}
     }
\caption{Results for MDPs with Streett objectives.}
\label{fig:mdps}
\end{center}
\end{figure}

\smallskip\noindent{\em Experimental evaluation.}
In the experimental evaluation we compare the number of symbolic steps 
(i.e., the number of $\pre/\post$ operations\footnote{Recall that the basic
set operations are cheaper to compute, and asymptotically at most the
number of $\pre/\post$ operations in all the presented algorithms.})
executed by the algorithms, the comparison of running time yields similar
results and is provided in Appendix~\ref{sec:appexper}.
As the initial preprocessing step is the same for all the algorithms
(computing all SCCs for graphs and all MECs for MDPs), the comparison
presents the number of symbolic steps executed after the preprocessing.
The experimental results for graphs are shown in Figure~\ref{fig:graphs} and
the experimental results for MDPs are shown in Figure~\ref{fig:mdps} (in each
figure the two lines represent equality and an order-of-magnitude improvement, respectively).

\smallskip\noindent{\em Discussion.}
Note that the lock-step search is the key reason for theoretical 
improvement, however, the improvement relies on a large number of Streett 
pairs. In the experimental evaluation, the LTL formulae generate
Streett automata with small number of pairs, which after the product
with the model accounts for an even smaller fraction of pairs as compared to
the size of the state space. This has two effects:
\begin{compactitem}
\item In the experiments the lock-step search is performed for a much smaller 
parameter value ($O(\log n)$ instead of the theoretically optimal  bound of 
$\sqrt{m/\log n}$), and leads to a small improvement.
\item For large graphs, since the number of pairs is small as compared to the 
number of states, the improvement over the basic algorithm is minimal.
\end{compactitem}
In contrast to graphs, in MDPs even with small number of pairs as 
compared to the state-space, the interleaved MEC computation has
a notable effect on practical performance, and we observe performance improvement 
even in large MDPs.
