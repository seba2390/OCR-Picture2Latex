\vspace{-3mm}
\section{Definitions}\label{sec:defin}

\vspace{-2mm}
\subsection{Basic Problem Definitions}

\vspace{-1mm}
\smallskip\noindent{\em Markov Decision Processes \upbr{MDPs} and Graphs.}
An MDP~$\mdp = ((V, E), \allowbreak(\vp, \vr), \allowbreak\trans)$ consists of a
finite directed graph $G = (V, E)$ with a set of $n$ vertices~$V$ and a set of $m$ 
edges~$E$, a partition of the vertices into 
\emph{player~1 vertices} $\vp$ and \emph{random vertices} $\vr$, and a 
probabilistic transition function~$\trans$. We call an edge $(u,v)$ with
$u \in \vp$ a \emph{player~1 edge} and an edge $(v, w)$ with $v \in \vr$ a
\emph{random edge}. For $v \in V$ we define $\In(v)=\{w \in V \mid (w,v) \in E\}$
and $\Out(v)=\{w \in V \mid (v,w) \in E\}$.
The probabilistic transition function is a function from $\vr$ to $\mathcal{D}(V)$, 
where $\mathcal{D}(V)$ is the set of probability distributions over $V$ and 
a random edge $(v, w) \in E$ if and only if $\trans(v)[w] > 0$.
Graphs are a special case of MDPs with $\vr = \emptyset$.

\vspace{-0.5mm}
\smallskip\noindent{\em Plays and Strategies.} 
A \emph{play} or infinite path in $\mdp$ is an infinite sequence $\pat = \langle v_0, 
v_1, v_2, \ldots \rangle$ such that $(v_i, v_{i+1}) \in E$ for all $i \in \mathbb{N}$;
we denote by $\Pat$ the set of all plays.
A player~1 \emph{strategy}~$\straa: V^* \cdot \vp \rightarrow V$ is a function that 
assigns to every finite prefix~$\pat \in V^* \cdot \vp$ of a play that ends in a 
player~1 vertex~$v$ a successor vertex $\straa(\pat) \in V$ such that 
$(v, \straa(\pat)) \in E$; we denote by $\Straa$ the 
set of all player~1 strategies. A strategy is \emph{memoryless} if we have 
$\straa(\pat) = \straa(\pat')$ for any $\pat, \pat' \in V^* \cdot \vp$ that 
end in the same vertex $v \in \vp$.

\vspace{-0.5mm}
\smallskip\noindent{\em Objectives.}
An \emph{objective} $\obj$ is a subset of $\Pat$ said to be winning
for player~1. We say that a  play $\pat \in \Pat$
\emph{satisfies the objective} if $\pat \in \obj$. For a vertex set~$\target
\subseteq V$ the \emph{reachability objective} is the set of infinite paths
that contain a vertex of $\target$, i.e., 
$\Reach{\target} = \set{\sseq \in \Pat \mid \exists j \ge 0: v_j \in \target}$.
Let $\Inf(\pat)$ for $\pat \in \Pat$ denote the set of vertices that occur
infinitely often in $\pat$. Given a set $\SP$ of $k$ pairs $(L_i, U_i)$ of vertex
sets $L_i, U_i \subseteq V$ with $1 \le i \le k$, the \emph{Streett objective}
is the set of infinite paths for which it holds
\emph{for each} $1 \le i \le k$ that whenever a vertex of $L_i$ occurs
infinitely often, then a vertex of $U_i$ occurs infinitely often, i.e.,
$\Streett{\SP} = \set{\pat \in \Pat \mid L_i \cap \Inf(\pat) = \emptyset
\text{ or } U_i \cap \Inf(\pat) \ne \emptyset \text{ for all } 1 \le i \le k}$.

\vspace{-0.5mm}
\smallskip\noindent{\em Almost-Sure Winning Sets.}
For any measurable set of plays $A \subseteq \Pat$ we denote by
$\pr{\straa}{v}{A}$ the probability that a play starting at $v \in V$
belongs to $A$ when player~1 plays strategy~$\straa$. 
A strategy~$\straa$ is \emph{almost-sure \upbr{a.s.} winning} from a vertex
$v \in V$ for an objective $\obj$ if $\pr{\straa}{v}{\obj} = 1$.
The \emph{almost-sure winning set} $\as{\mdp, \obj}$
of player~1 is the set of vertices for which player~1 has an
almost-sure winning strategy. In graphs the existence of an almost-sure
winning strategy corresponds to the existence of a play in the objective,
and the set of vertices for which player~1 has an (almost-sure) winning
strategy is called the \emph{winning set} $\win{\mdp, \obj}$ of player~1.

\vspace{-0.5mm}
\smallskip\noindent{\em Symbolic Encoding of MDPs.}
Symbolic algorithms operate on sets of vertices, which are usually described by 
Binary Decision Diagrams (\textsc{bdd}s)~\cite{Lee59,Akers78}.
In particular Ordered Binary Decision Diagrams~\cite{Bryant85} (\textsc{Obdd}s) 
provide a canonical symbolic representation of Boolean functions. 
For the computation of almost-sure winning sets of MDPs it is sufficient to encode MDPs 
with \textsc{Obdd}s and one additional bit that denotes whether a vertex is 
in~$\vp$ or~$\vr$.

\vspace{-0.5mm}
\smallskip\noindent{\em Symbolic Steps.}
One symbolic step corresponds to one primitive operation as supported by 
standard symbolic packages like \textsc{CuDD}~\cite{Somenzi15}.
In this paper we only allow the same basic \emph{set-based symbolic operations} as 
in~\cite{RaviBS00,GentiliniPP03,BloemGS06,ChatterjeeHJS13}, namely set operations and
the following one-step symbolic operations for a set of vertices $Z$:
(a)~the one-step predecessor operator
$\pre(Z)=\{v \in V \mid \Out(v) \cap Z \ne \emptyset\};$
(b)~the one-step  successor operator
\mbox{$\post(Z)=\{v \in V \mid \In(v) \cap Z \ne \emptyset\};$}
and (c)~the one-step \emph{controllable} predecessor operator
$\cpre{R}(Z) = \left\{ v \in \vp \mid \Out(v) \subseteq Z \right\}
\cup  \left\{ v \in \vr \mid \Out(v) \cap Z \ne \emptyset \right\};$
i.e., the $\cpre{R}$ operator computes all vertices such that the successor belongs to~$Z$ with
positive probability. This operator can be defined using the $\pre$ operator
and basic set operations as follows:
$\cpre{R}(Z)= \pre(Z) \setminus (\vp \cap \pre(V \setminus Z))\,$.
We additionally allow cardinality computation and picking an 
arbitrary vertex from a set as in~\cite{ChatterjeeHJS13}.

\vspace{-0.5mm}
\smallskip\noindent{\em Symbolic Model.} Informally, a symbolic algorithm does not 
operate on explicit representation of the transition function of a graph, 
but instead accesses it through $\pre$ and $\post$ operations.
For explicit algorithms, a $\pre/\post$ operation on a set of vertices (resp., a single vertex)
requires $O(m)$ (resp., the order of indegree/outdegree of the vertex) time. 
In contrast, for symbolic algorithms $\pre/\post$ operations are considered unit-cost.
Thus an interesting algorithmic question is whether better algorithmic bounds can be obtained
considering $\pre/\post$ as unit operations.
Moreover, the basic set operations are computationally less expensive 
(as they encode the relationship between the state variables) compared to the $\pre/\post$ 
symbolic operations (as they encode the transitions and thus the relationship between the 
present and the next-state variables). 
In all presented algorithms, the number of set operations is asymptotically at most the 
number of $\pre/\post$ operations. 
Hence in the sequel we focus on the number of $\pre/\post$ operations of algorithms.

\vspace{-0.5mm}
\smallskip\noindent{\em Algorithmic Problem.}
Given an MDP~$\mdp$ (resp. a graph~$G$) and a set of Streett pairs $\SP$, the problem
we consider asks for a symbolic algorithm to compute the almost-sure winning
set $\as{\mdp, \Streett{\SP}}$ (resp. the winning set $\win{G, \Streett{\SP}}$), which
is also called the \emph{qualitative analysis} of MDPs (resp. graphs).

\vspace{-2mm}
\subsection{Basic Concepts related to Algorithmic Solution}

\vspace{-1mm}
\smallskip\noindent{\em Reachability.}
For a graph~$G = (V, E)$ and a set of vertices $S \subseteq V$ 
the set $\reachG{G}{S}$ is the set of vertices of $V$ that \emph{can 
reach} a vertex of $S$ within~$G$, and it can be identified with at most
$\lvert \reachG{G}{S} \setminus S \rvert + 1$ many $\pre$ operations.

\vspace{-0.5mm}
\smallskip\noindent{\em Strongly Connected Components.}
For a set of vertices $S \subseteq V$ we denote by 
$G[S] = (S, E \cap (S \times S))$ the subgraph of the graph~$G$ induced by 
the vertices of~$S$. An induced subgraph~$G[S]$ is strongly connected 
if there exists a path in~$G[S]$ between every pair of vertices of $S$.
A \emph{strongly connected component \upbr{SCC}} of~$G$ is a set of 
vertices~$\scc\subseteq V$ such that the induced subgraph~$G[\scc]$ is 
strongly connected and $\scc$ is a maximal set in~$V$ with this property.
We call an SCC \emph{trivial} if it only contains a single 
vertex and no edges; and \emph{non-trivial} otherwise. The SCCs of~$G$ partition
its vertices and can be found in $O(n)$ symbolic steps~\cite{GentiliniPP08}.
A bottom SCC~$\scc$ in a directed graph~$G$ is an SCC with no edges from 
vertices of~$\scc$ to vertices of~$V \setminus \scc$, i.e., an SCC without
\emph{outgoing} edges. Analogously, a top SCC~$\scc$ is an SCC with no \emph{incoming}
edges from~$V \setminus \scc$.
For more intuition for bottom and top SCCs, consider the graph in which 
each SCC is contracted into a single vertex (ignoring edges within
an SCC). In the resulting directed acyclic graph the sinks represent the
bottom SCCs and the sources represent the top SCCs.
Note that every graph has at least one bottom and at least one top SCC.
If the graph is not strongly connected, then there exist at least one
top and at least one bottom SCC that are disjoint and thus one of them 
contains at most half of the vertices of~$G$.

\vspace{-0.5mm}
\smallskip\noindent{\em Random Attractors.}
In an MDP~$\mdp$ the \emph{random attractor} $\at{R}{\mdp}{W}$  of a set of vertices 
$W$ is defined as $\at{R}{\mdp}{W} =
\bigcup_{j \ge 0} Z_j$ where $Z_0 = W$ and $Z_{j+1} = Z_j \cup \cpre{R}(Z_j)$
for all $j > 0$. The attractor can be computed with at most
$\lvert \at{R}{\mdp}{W} \setminus  W \rvert + 1$ many $\cpre{R}$ operations.

\vspace{-0.5mm}
\smallskip\noindent{\em Maximal End-Components.}
Let $\ec$ be a vertex set without outgoing random edges, i.e.,
with $\Out(v) \subseteq \ec$ for all $v \in \ec \cap \vr$.
A sub-MDP of an MDP~$\mdp$ induced by a vertex set $\ec \subseteq V$
without outgoing random edges is defined as 
$\mdp[\ec] = ((\ec, E \cap (\ec \times \ec), (\vp \cap \ec, 
\vr \cap \ec), \trans)$. Note that the requirement that $\ec$ has
no outgoing random edges
is necessary in order to use the same probabilistic transition function~$\trans$.
An \emph{end-component} \upbr{EC} of an MDP~$\mdp$ is a set of 
vertices $\ec \subseteq V$ such that \upbr{a} $\ec$ has no outgoing random 
edges, i.e., $\mdp[\ec]$ is a valid sub-MDP, \upbr{b} the induced sub-MDP $\mdp[\ec]$
is strongly connected, and \upbr{c} $\mdp[\ec]$ contains at least one edge.
Intuitively, an end-component is a set of vertices for which player~1 can ensure
that the play stays within the set and almost-surely reaches all the vertices in the 
set (infinitely often). An end-component is a \emph{maximal end-component} \upbr{MEC} 
if it is maximal under set inclusion.
An end-component is \emph{trivial} if it consists of a single vertex \upbr{with
a self-loop}, otherwise it is \emph{non-trivial}.
The \emph{MEC decomposition} of an MDP consists of all MECs of the MDP.

\vspace{-0.5mm}
\smallskip\noindent{\em Good End-Components.}
All algorithms for MDPs with Streett objectives are based on finding 
good end-components, defined below. Given the union of all good end-components, the 
almost-sure winning set is obtained by computing the 
almost-sure winning set for the reachability objective
with the union of all good end-components as the target set.
The correctness of this approach is shown in \cite{ChatterjeeDHL16,Loitzenbauer16}
(see also~\cite[Chap.~10.6.3]{baierbook}). For Streett objectives a good end-component
is defined as follows. In the special case of graphs they are called good components.

\begin{dfn}[Good end-component]
	Given an MDP $\mdp$ and a set $\SP= \{(L_\idxs, U_\idxs) \mid 1 \le \idxs \le k\}$ of
	target pairs, a \emph{good end-component} is an 
	end-component $\ec$ of $\mdp$ such that for each $1 \le \idxs \le k$ either 
	$L_\idxs \cap \ec = \emptyset$ or $U_\idxs \cap \ec \ne \emptyset$.
	A maximal good end-component is a good end-component that is maximal with respect to set inclusion.
\end{dfn}

\begin{lem}[{Correctness of Computing Good End-Components~\cite[Corollary~2.6.5, Proposition~2.6.9]{Loitzenbauer16}}]
\label{lem:gec}
	For an MDP~$\mdp$ and a set~$\SP$ of target pairs, let $\mathcal{\ec}$ be the set of all maximal good end-components.
	Then $\as{\mdp, \Reach{\bigcup_{\ec \in \mathcal{\ec}}\ec}}$ is equal to $\as{\mdp, \Streett{\SP}}$.
\end{lem}

\vspace{-0.5mm}
\smallskip\noindent{\em Iterative Vertex Removal.}
All the algorithms for Streett objectives maintain vertex sets that are 
candidates for good end-components. For such a vertex set~$S$ we (a) 
refine the maintained sets according to the SCC decomposition of $\mdp[S]$
and (b) for a set of vertices~$W$ for which we know that it cannot be contained 
in a good end-component, we remove its random attractor from $S$. The following lemma 
shows the correctness of these operations.

\vspace{-0.5mm}
\begin{lem}[{Correctness of Vertex Removal~\cite[Lemma~2.6.10]{Loitzenbauer16}}]
\label{lem:eccontained}
	Given an MDP $\mdp = ((V, E), (\vp, \vr), \trans)$, let $\ec$ be an end-component with $X \subseteq S$ for 
	some $S \subseteq V$. Then 
	\begin{compactitem}
	 \item[\upbr{a}]$\ec \subseteq \scc$ for one SCC~$\scc$ of $\mdp[S]$ and
	 \item[\upbr{b}] $\ec \subseteq S \setminus \at{R}{\mdp'}{W}$ for each
     $W \subseteq V \setminus \ec$ and each sub-MDP~$\mdp'$ containing~$\ec$.
	\end{compactitem}
\end{lem}

\vspace{-0.5mm}
Let $\ec$ be a good end-component. Then $\ec$ is an end-component and for each index $\idxs$,
$\ec \cap U_\idxs = \emptyset$ implies $\ec \cap L_\idxs = \emptyset$ . 
Hence we obtain the following corollary.

\vspace{-0.5mm}
\begin{cor}[{\cite[Corollary~4.2.2]{Loitzenbauer16}}]\label{cor:geccontained}
Given an MDP $\mdp$, let $\ec$ be a \emph{good} end-component with 
$X \subseteq S$ for some $S \subseteq V$. 
For each $i$ with $S \cap U_i = \emptyset$ it holds that
$\ec \subseteq S \setminus \at{R}{\mdp[S]}{ L_i \cap S}$.
\end{cor}

\vspace{-0.5mm}
For an index~$\idxs$ with $S \cap U_\idxs = \emptyset$ we call the 
vertices of $S \cap L_\idxs$ \emph{bad vertices}. The set of all bad 
vertices $\bad(S) = \bigcup_{1 \le i \le k} \set{v \in L_i \cap S \mid 
U_i \cap S = \emptyset}$ can be computed with $2 k$ set operations.
