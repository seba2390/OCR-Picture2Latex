\let\oldprintchaptertitle=\printchaptertitle
\renewcommand{\printchaptertitle}[1]{%
	\vspace*{-75pt}
	\oldprintchaptertitle{#1}
}%
\myChapterStar{Résumé}{}{section}
\let\printchaptertitle=\oldprintchaptertitle
Dans cette thèse, nous nous intéressons à la conception et à l'implémentation des systèmes workflows distribués, dédiés à l'automatisation des processus opérationnels dits administratifs. Nous proposons une approche de mise en oeuvre de tels systèmes en nous appuyant sur les concepts de systèmes multi-agents, d'architecture Pair à Pair (P2P, égal à égal), d'architecture orientée service et d'édition coopérative de documents structurés (artefacts). 
En effet, nous développons des outils mathématiques permettant à tout concepteur de systèmes workflows, d'exprimer chaque processus administratif qu'il désire automatiser, sous forme d'une grammaire attribuée dont les symboles représentent les tâches à exécuter, les productions spécifient un ordonnancement de celles-ci, et les instances (les arbres de dérivation qui lui sont conformes) représentent les différents scénarii d'exécution menant aux différents états qui matérialisent l'accomplissement des objectifs de l'entreprise. Le modèle grammatical obtenu est ensuite introduit dans un système P2P que nous proposons, et qui est chargé d'assurer l'exécution complètement décentralisée d'instances du processus sous-jacent. 
Ledit système orchestre l'exécution d'une instance de processus sous forme d'une chorégraphie durant laquelle, divers agents logiciels pilotés par des agents humains (acteurs), se coordonnent à l'aide d'artefacts qu'ils éditent collégialement. Les artefacts échangés représentent la mémoire du système: ils donnent des informations sur les tâches déjà exécutées, sur celles prêtes à l'être et sur leurs exécutants. Les agents logiciels sont autonomes et identiques: ils exécutent le même et unique protocole à chaque fois qu'ils reçoivent un artefact. Ce protocole leur permet d'identifier les tâches qu'ils doivent exécuter dans l'immédiat, de les exécuter, de mettre à jour l'artefact et de le diffuser si nécessaire, pour la continuation du processus d'exécution. 
En outre, les acteurs n'ont potentiellement qu'une perception partielle des processus dans lesquels ils sont impliqués. Ce qui autorise donc une possible exécution confidentielle de certaines tâches en pratique: cette propriété permet d'offrir une gestion automatique des processus administratifs se rapprochant un peu plus, de leur gestion non informatisée.

\vspace{1cm}
\noindent\textbf{Mots clés:} Workflows Administratifs, Artefacts, Pair à Pair, Réplique Partielle, Gestion des Processus Opérationnels.

\myCleanStarChapterEnd

