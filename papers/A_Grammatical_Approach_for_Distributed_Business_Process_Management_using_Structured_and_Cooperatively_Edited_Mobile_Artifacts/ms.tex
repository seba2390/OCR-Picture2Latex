\documentclass[12pt, a4paper, openany, oneside]{memoir}
\usepackage[T1]{fontenc}
\usepackage[latin1]{inputenc}
% Import du style (obligatoire)
% Un template bilingue pour la production des mémoires et thèses dans le département de Mathématiques-Informatique.
% Ce template est conforme aux recommandations de l'école doctorale
%
% Ce fichier est le fichier de style principal. Vous y trouverez toutes les définitions des commandes personnalisées
%
% @author Zekeng Ndadji Milliam Maxime
\newcommand\hmmax{0}
\newcommand\bmmax{0}
\usepackage{pifont}
\usepackage{pslatex}
%\usepackage{charter}
%\usepackage{mathptmx}
\usepackage{yfonts}
\usepackage[mathscr]{euscript}
\usepackage{latexsym}
\usepackage{stmaryrd}
\usepackage{amssymb}
\usepackage{amsmath}
\usepackage{graphicx}
\usepackage{pst-all}
\usepackage{verbatim} 
\usepackage{fancyvrb}%Pour utiliser l'environnement "Verbatim"
\usepackage{mathrsfs}
\usepackage{vmargin}
\usepackage{titletoc}
\usepackage{bbding}
\usepackage{wasysym}
%\usepackage{eufrak}
\usepackage{ifthen}
\usepackage[final]{pdfpages}
%\usepackage{natbib}
\usepackage[natbibapa]{apacite}
\usepackage{morewrites}
\usepackage{tocbibind}
\usepackage{multicol}
 
\setpapersize{A4}
\DeclareMathAlphabet{\mathpzc}{OT1}{pzc}{m}{it}

\newcommand{\mathbox}[1]{\mbox{{\small \mbox{$ #1 $}}}}
\newcommand{\sta}[3]{\mathbox{#1 \stackrel{#2}{\longrightarrow} #3}}
\newcommand\QEDBox
{{\leavevmode\unskip\nobreak\hfil\penalty50\hskip.75cm%
    \hbox{} \nobreak\hfil $\Box$ \parfillskip=0pt
    \finalhyphendemerits=0
    \par
}}
\def\qed{\QEDBox}
\makeatletter
% *************** Définitions de quelques couleurs ***************
\usepackage{color}
\usepackage{colortbl}

\definecolor{greenyellow}   {cmyk}{0.15, 0   , 0.69, 0   }
\definecolor{yellow}        {cmyk}{0   , 0   , 1   , 0   }
\definecolor{goldenrod}     {cmyk}{0   , 0.10, 0.84, 0   }
\definecolor{dandelion}     {cmyk}{0   , 0.29, 0.84, 0   }
\definecolor{apricot}       {cmyk}{0   , 0.32, 0.52, 0   }
\definecolor{peach}         {cmyk}{0   , 0.50, 0.70, 0   }
\definecolor{melon}         {cmyk}{0   , 0.46, 0.50, 0   }
\definecolor{yelloworange}  {cmyk}{0   , 0.42, 1   , 0   }
\definecolor{orange}        {cmyk}{0   , 0.61, 0.87, 0   }
\definecolor{burntorange}   {cmyk}{0   , 0.51, 1   , 0   }
\definecolor{bittersweet}   {cmyk}{0   , 0.75, 1   , 0.24}
\definecolor{redorange}     {cmyk}{0   , 0.77, 0.87, 0   }
\definecolor{mahogany}      {cmyk}{0   , 0.85, 0.87, 0.35}
\definecolor{maroon}        {cmyk}{0   , 0.87, 0.68, 0.32}
\definecolor{brickred}      {cmyk}{0   , 0.89, 0.94, 0.28}
\definecolor{red}           {cmyk}{0   , 1   , 1   , 0   }
\definecolor{orangered}     {cmyk}{0   , 1   , 0.50, 0   }
\definecolor{rubinered}     {cmyk}{0   , 1   , 0.13, 0   }
\definecolor{wildstrawberry}{cmyk}{0   , 0.96, 0.39, 0   }
\definecolor{salmon}        {cmyk}{0   , 0.53, 0.38, 0   }
\definecolor{carnationpink} {cmyk}{0   , 0.63, 0   , 0   }
\definecolor{magenta}       {cmyk}{0   , 1   , 0   , 0   }
\definecolor{violetred}     {cmyk}{0   , 0.81, 0   , 0   }
\definecolor{rhodamine}     {cmyk}{0   , 0.82, 0   , 0   }
\definecolor{mulberry}      {cmyk}{0.34, 0.90, 0   , 0.02}
\definecolor{redviolet}     {cmyk}{0.07, 0.90, 0   , 0.34}
\definecolor{fuchsia}       {cmyk}{0.47, 0.91, 0   , 0.08}
\definecolor{lavender}      {cmyk}{0   , 0.48, 0   , 0   }
\definecolor{thistle}       {cmyk}{0.12, 0.59, 0   , 0   }
\definecolor{orchid}        {cmyk}{0.32, 0.64, 0   , 0   }
\definecolor{darkorchid}    {cmyk}{0.40, 0.80, 0.20, 0   }
\definecolor{purple}        {cmyk}{0.45, 0.86, 0   , 0   }
\definecolor{plum}          {cmyk}{0.50, 1   , 0   , 0   }
\definecolor{violet}        {cmyk}{0.79, 0.88, 0   , 0   }
\definecolor{royalpurple}   {cmyk}{0.75, 0.90, 0   , 0   }
\definecolor{blueviolet}    {cmyk}{0.86, 0.91, 0   , 0.04}
\definecolor{periwinkle}    {cmyk}{0.57, 0.55, 0   , 0   }
\definecolor{cadetblue}     {cmyk}{0.62, 0.57, 0.23, 0   }
\definecolor{cornflowerblue}{cmyk}{0.65, 0.13, 0   , 0   }
\definecolor{midnightblue}  {cmyk}{0.98, 0.13, 0   , 0.43}
\definecolor{navyblue}      {cmyk}{0.94, 0.54, 0   , 0   }
\definecolor{royalblue}     {cmyk}{1   , 0.50, 0   , 0   }
\definecolor{blue}          {cmyk}{1   , 1   , 0   , 0   }
\definecolor{cerulean}      {cmyk}{0.94, 0.11, 0   , 0   }
\definecolor{cyan}          {cmyk}{1   , 0   , 0   , 0   }
\definecolor{processblue}   {cmyk}{0.96, 0   , 0   , 0   }
\definecolor{skyblue}       {cmyk}{0.62, 0   , 0.12, 0   }
\definecolor{turquoise}     {cmyk}{0.85, 0   , 0.20, 0   }
\definecolor{tealblue}      {cmyk}{0.86, 0   , 0.34, 0.02}
\definecolor{aquamarine}    {cmyk}{0.82, 0   , 0.30, 0   }
\definecolor{bluegreen}     {cmyk}{0.85, 0   , 0.33, 0   }
\definecolor{emerald}       {cmyk}{1   , 0   , 0.50, 0   }
\definecolor{junglegreen}   {cmyk}{0.99, 0   , 0.52, 0   }
\definecolor{seagreen}      {cmyk}{0.69, 0   , 0.50, 0   }
\definecolor{green}         {cmyk}{1   , 0   , 1   , 0   }
\definecolor{forestgreen}   {cmyk}{0.91, 0   , 0.88, 0.12}
\definecolor{pinegreen}     {cmyk}{0.92, 0   , 0.59, 0.25}
\definecolor{limegreen}     {cmyk}{0.50, 0   , 1   , 0   }
\definecolor{yellowgreen}   {cmyk}{0.44, 0   , 0.74, 0   }
\definecolor{springgreen}   {cmyk}{0.26, 0   , 0.76, 0   }
\definecolor{olivegreen}    {cmyk}{0.64, 0   , 0.95, 0.40}
\definecolor{rawsienna}     {cmyk}{0   , 0.72, 1   , 0.45}
\definecolor{sepia}         {cmyk}{0   , 0.83, 1   , 0.70}
\definecolor{brown}         {cmyk}{0   , 0.81, 1   , 0.60}
\definecolor{tan}           {cmyk}{0.14, 0.42, 0.56, 0   }
\definecolor{gray}          {cmyk}{0   , 0   , 0   , 0.50}
\definecolor{black}         {cmyk}{0   , 0   , 0   , 1   }
\definecolor{white}         {cmyk}{0   , 0   , 0   , 0   } 

\usepackage{memhfixc}


% *************** Style de chapitre et de section ***************
\newcommand{\myPrintChapterLabel}[1]{
	\ifthenelse{\equal{#1}{A}}
		{\myAppendixLabel} 
		{\ifthenelse{\equal{#1}{B}}
			{\myAppendixLabel} 
			{\ifthenelse{\equal{#1}{C}}
				{\myAppendixLabel} 
				{\ifthenelse{\equal{#1}{D}}
					{\myAppendixLabel} 
					{\ifthenelse{\equal{#1}{E}}
						{\myAppendixLabel} 
						{\ifthenelse{\equal{#1}{F}}
							{\myAppendixLabel} 
							{\ifthenelse{\equal{#1}{G}}
								{\myAppendixLabel} 
								{\myChapterLabel}}}}}}} 
}

\newcommand{\StyleFolder}{Template-Style}
\newcommand{\ChapterStylesFolder}{\StyleFolder/chapter-styles}
\newcommand{\SectionStylesFolder}{\StyleFolder/section-styles}
\newcommand{\FooterHeaderStylesFolder}{\StyleFolder/footer-header-styles}
\newcommand{\MinitocStylesFolder}{\StyleFolder/minitoc-styles}
\newcommand{\BibTableStylesFolder}{\StyleFolder/bib-table-styles}
\newcommand{\UserCommandsFolder}{\StyleFolder/user-commands}
\newcommand{\TocStylesFolder}{\StyleFolder/toc-styles}
\newcommand{\myChapterStyle}[1]{
	\chapterstyle{#1}
}


\makechapterstyle{fieldset}{%
    \renewcommand{\chapnamefont}{\LARGE\sffamily}%
    \renewcommand{\chapnumfont}{\fontsize{80pt}{0pt}\sffamily}%
    \renewcommand{\chaptitlefont}{\fontsize{25pt}{30pt}\Huge\bfseries}%
	% Impression du texte chapitre ou annexe
    \renewcommand{\printchaptertitle}[1]{%
		  \vspace*{-45.3pt}
		  \begin{center}
			  \chaptitlefont %\hrule height 1.5pt
			  \begin{center}\textcolor{black}{\textsc{{##1}}}\end{center}
			  \vspace{-2mm}
			  \textcolor{black}{\hrule height 2.5pt}%
		  \end{center}
        }%
		\renewcommand{\printchaptername}{%
			\vspace*{-100.3pt}
		}
	% Impression du numéro de chapitre
	\renewcommand{\printchapternum}{%
		\begin{center}
			\chapnumfont{\textcolor{blue}{$\mathpzc{\thechapter}$}}
		\end{center}
		\vspace{0mm}
		\parbox[c]{.07\textwidth}{
			\centering
			\textcolor{black}{\hrule height 2.5pt}
		}
		\parbox[c]{.2\textwidth}{
			\centering
			\textcolor{blue}{\textsc{\myPrintChapterLabel{\thechapter}}}
		}
		\parbox[c]{.71\textwidth}{
			\centering
			\textcolor{black}{\hrule height 2.5pt}
		}
		\vspace{0mm}
	}%
}

\makechapterstyle{titleontopright}{%
    \renewcommand{\chapnamefont}{\LARGE\sffamily}%
    \renewcommand{\chapnumfont}{\fontsize{60pt}{0pt}\sffamily\bfseries}%
    \renewcommand{\chaptitlefont}{\Huge\bfseries}%
	% Impression du texte chapitre ou annexe
    \renewcommand{\printchaptertitle}[1]{%
		  \vspace*{-28.3pt}
        \chaptitlefont \hrule height 1.0pt
        \begin{flushright}\textcolor{black}{{##1}}\end{flushright}
		  \hrule height 1.0pt%
        }%
		\renewcommand{\printchaptername}{%
			\begin{flushright}
				\normalsize \chapnamefont
				$~\mathit{\myPrintChapterLabel{\thechapter}}$
			\end{flushright}
		}
	% Impression du numéro de chapitre
    \renewcommand{\printchapternum}{%
        %\begin{flushright}
				\vspace*{-40.3pt}  
				\hspace{15,0cm}\chapnumfont{$\mathit{\thechapter}$}
		  %\end{flushright}%
        }%
}

\makechapterstyle{bringhurst}{%
\renewcommand{\chapterheadstart}{}
\renewcommand{\printchaptername}{}
\renewcommand{\chapternamenum}{}
\renewcommand{\printchapternum}{}
\renewcommand{\afterchapternum}{}
\renewcommand{\printchaptertitle}[1]{%
\raggedright\Large\scshape\MakeLowercase{##1}}
\renewcommand{\afterchaptertitle}{%
\vskip\onelineskip \hrule\vskip\onelineskip}
}
%
\newlength{\headindent}
\newlength{\rightblock}
\makechapterstyle{southall}{%
\setlength{\headindent}{36pt}
\setlength{\rightblock}{\textwidth}
\addtolength{\rightblock}{-\headindent}
\setlength{\beforechapskip}{2\baselineskip}
\setlength{\afterchapskip}{5\baselineskip}
\setlength{\midchapskip}{0pt}
\renewcommand{\chaptitlefont}{\huge\rmfamily\raggedright}
\renewcommand{\chapnumfont}{\chaptitlefont}
\renewcommand{\printchaptername}{}
\renewcommand{\chapternamenum}{}
\renewcommand{\afterchapternum}{}
\renewcommand{\printchapternum}{%
\begin{minipage}[t][\baselineskip][b]{\headindent}
{\vspace{0pt}\chapnumfont%%%\figureversion{lining}
\thechapter}
\end{minipage}}
\renewcommand{\printchaptertitle}[1]{%
\hfill\begin{minipage}[t]{\rightblock}
{\vspace{0pt}\chaptitlefont ##1\par}\end{minipage}}
\renewcommand{\afterchaptertitle}{%
\par\vspace{\baselineskip}%
\hrulefill \par\nobreak\noindent \vskip\afterchapskip}
}
%
\makechapterstyle{chappell}{
\setlength\beforechapskip{0pt}
\renewcommand*\chapnamefont{\large\centering}
\renewcommand*\chapnumfont{\large}
\renewcommand*\printchapternonum{%
\vphantom{\printchaptername}%
\vphantom{\chapnumfont 1}%
\afterchapternum
\vskip -\onelineskip}
\renewcommand*\chaptitlefont{\Large\itshape}
\renewcommand*\printchaptertitle[1]{%
\hrule\vskip\onelineskip\centering\chaptitlefont ##1}
}
%

\makechapterstyle{fieldset-rev}{%
    \renewcommand{\chapnamefont}{\LARGE\sffamily}%
    \renewcommand{\chapnumfont}{\fontsize{80pt}{0pt}\sffamily}%
    \renewcommand{\chaptitlefont}{\fontsize{25pt}{30pt}\Huge\bfseries}%
	% Impression du texte chapitre ou annexe
    \renewcommand{\printchaptertitle}[1]{%
		  \vspace*{-45.3pt}
		  \begin{center}
			  \chaptitlefont %\hrule height 1.5pt
			  \begin{center}\textcolor{black}{\textsc{{##1}}}\end{center}
			  \vspace{-4mm}
			  \textcolor{black}{\hrule height 1.5pt}%
			  \vspace{1mm}
			  \textcolor{black}{\hrule height 4.5pt}
		  \end{center}
        }%
		\renewcommand{\printchaptername}{%
			\vspace*{-100.3pt}
		}
	% Impression du numéro de chapitre
	\renewcommand{\printchapternum}{%
		\begin{center}
			\chapnumfont{\textcolor{blue}{$\mathpzc{\thechapter}$}}
		\end{center}
		\vspace{0mm}
		\parbox[c]{.07\textwidth}{
			\centering
			\textcolor{black}{\hrule height 2.5pt}
		}
		\parbox[c]{.2\textwidth}{
			\centering
			\textcolor{blue}{\textsc{\myPrintChapterLabel{\thechapter}}}
		}
		\parbox[c]{.71\textwidth}{
			\centering
			\textcolor{black}{\hrule height 2.5pt}
		}
		\vspace{0mm}
	}%
}
%


% *************** Nouvelle taille *******************************
\newcommand{\timesContentFontSize}{\fontsize{13pt}{18pt}}

%--- Niveau 1: section
 
\newcommand{\FonteSectionI}{\sffamily\bfseries\raggedright\fontsize{16pt}{20.7pt}\selectfont}%

%\renewcommand{\thesection}{\arabic{section}}%
\renewcommand{\section}{%
   \par\vspace{20pt}
   %\hrule height 0.5mm
   \vspace{1.5mm}
   \renewcommand{\@seccntformat}[1]{\fontsize{16pt}{20.7pt}\thesection.\hspace{0.7em}}
   \@startsection{section}  % nom de l'inter
   {1}%                     % niveau de l'inter
   {0pt}%                   % l'indentation du titre et du texte suivant
   {4pt}% beforeskip %
   {6pt}% afterskip
   {\FonteSectionI}%        % style
}

%--- Niveau 2: sous-section
\newcommand{\FonteSectionII}{\sffamily\bfseries\raggedright\fontsize{15pt}{19.4pt}\selectfont}%

\renewcommand{\thesubsection}{\thesection.\arabic{subsection}}%
\renewcommand{\subsection}{%
\vspace{3mm}
  \renewcommand{\@seccntformat}[1]%
               {{\fontsize{15pt}{19.4pt}\thesubsection.\hspace{0.7em}}}%
  \@startsection%
   {subsection}%            % nom de l'inter
   {2}%                     % niveau de l'inter
   {0pt}%                   % l'indentation du titre et du texte suivant
   {3pt}
   {5pt}
   {\FonteSectionII}}%      % style

%--- Niveau 3: sous-sous-section 

\newcommand{\FonteSectionIII}{\sffamily\bfseries\fontsize{14pt}{18.2pt}\raggedright\selectfont}%

\renewcommand{\thesubsubsection}{\thesubsection.\arabic{subsubsection}}%
\renewcommand{\subsubsection}{%
\vspace{2mm}
  \renewcommand{\@seccntformat}[1]{%
               {\sffamily\bfseries\fontsize{14pt}{18.2pt}\thesubsubsection.\hspace{0.7em}}}%
  \@startsection%
   {subsubsection}%         % nom de l'inter
   {3}%                     % niveau de l'inter
   {0pt}%                   % l'indentation du titre et du texte suivant
   {3pt}
   {3pt}
   {\FonteSectionIII}}%     % style

% <Alinéas>----------------------------------------------------------------

% Disable single lines at the start of a paragraph
\clubpenalty = 10000
% Disable single lines at the end of a paragraph 
\widowpenalty = 10000
\displaywidowpenalty = 10000



\makeevenhead{ruled}{\small\textsc{\rightmark}}{}{\thepage}
\makeoddhead{ruled}{\small\textsc{\rightmark}}{}{\thepage}
\makeoddfoot{ruled}{\hrule height 0.3mm \small\textsc{\phdthesislabel}}{}{\small\textsc{\studentlab}}
\makeevenfoot{ruled}{\hrule height 0.3mm \small\textsc{\phdthesislabel}}{}{\small\textsc{\studentlab}}

% Notes de bas de page
%\newcommand{\FonteNoteBasPage}{\footnotesize\sffamily}
%\renewcommand{\footnotesize}{\FonteNoteBasPage}
\addtolength{\skip\footins}{6pt} 

\renewcommand{\footnoterule}{%
	\par %\vspace*{-12.3pt}
	\noindent\rule{3.5cm}{0.6pt}\vspace*{6pt} 
}

\setlength{\footnotesep}{3pt} % Espace vertical avant chaque note (strut)

\newcommand{\@Myfnmark}{
      \mbox{\fontsize{8}{11}\sffamily\arabic{footnote}. }%
}

\renewcommand{\@makefntext}[1]{%
      \noindent\@Myfnmark#1%
}%

\def\@thefnmark{\arabic{footnote}}


%\setlrmarginsandblock{3.5cm}{2.5cm}{*}
%\setulmarginsandblock{2.5cm}{*}{1}
%\checkandfixthelayout



% *************** Style table de matière et listes de figures ***************
\settocdepth{subsubsection}
\setsecnumdepth{subsubsection}
\maxsecnumdepth{subsubsection}
\settocdepth{subsubsection}
\maxtocdepth{subsubsection}

\renewcommand{\chapternumberline}[1]{
% Impression du texte chapitre ou annexe
\hspace{-0.4cm}\textbf{
	$\mathit{\myPrintChapterLabel{#1}}$
	$\mathpzc{{#1}}~\RHD$}
}

\newcommand{\lof}{false}
\renewcommand{\numberline}[1]{
\ifthenelse{\equal{\lof}{false}}{\hspace{-0.6cm}$\mathrm{{#1}}$ -} {$\mathrm{{#1}}$ -}
}

\let\oldcontentsline=\contentsline
\renewcommand{\contentsline}[4]{
	\vspace{0.8mm}
	\oldcontentsline{#1}{#2}{#3}{#4}
	\vspace{0.8mm}
}



\newcommand{\minitoclevel}{section}
\newcommand{\minitocstyle}{titleontopright}

% Insertion d'une mini table de matière
\newcommand{\myMiniToc}[2]{
	\ifthenelse{\equal{#1}{}}
	{\renewcommand{\minitoclevel}{section}}
	{\renewcommand{\minitoclevel}{#1}}
	\startcontents[chapters]
	\ifthenelse{\equal{\minitocstyle}{fieldset}}
	{\myMiniTocFieldset{#2}}
	{
		\ifthenelse{\equal{\minitocstyle}{titleontopright}}
		{\myMiniTocTitleOnTopRight{#2}}
		{}
	}
}%

% Minitoc de type titleontop
\newcommand{\myMiniTocTitleOnTopRight}[1]{
	\vspace{-0.7mm}
	\vspace{20pt}
	%\hspace{-22pt}
	\begin{minipage}{\textwidth}
	\begin{flushright}\noindent\textcolor{black}{\textbf{#1}} \vspace{5pt} \hrule height 0.08mm \end{flushright}
	\par
	\printcontents[chapters]{}{1}{}
	\par
	\begin{flushright} \vspace{0pt} \hrule height 0.08mm \vspace{30pt} \end{flushright}
	\end{minipage}
}

% Minitoc de type fieldset
\newcommand{\myMiniTocFieldset}[1]{
	%\vspace{-0.7mm}
	\vspace{5pt}
	%\hspace{-22pt}
	\begin{minipage}{\textwidth}
		\parbox[c]{.07\textwidth}{
			\centering
			\textcolor{black}{\hrule height 0.4mm}
		}
		\parbox[c]{.2\textwidth}{
			\centering
			\textcolor{black}{\textsc{\textbf{#1}}}
		}
		\parbox[c]{.71\textwidth}{
			\centering
			\textcolor{black}{\hrule height 0.4mm}
		}
		%\vspace{-10pt} 
		\par
		\printcontents[chapters]{}{1}{}
		\par
		\begin{flushright} 
			\vspace{0pt} 
			\hrule height 0.4mm 
			\vspace{30pt} 
		\end{flushright}
	\end{minipage}
}

\newcommand{\myMiniTocClearPage}[2]{
	\myMiniToc{#1}{#2}
	\clearpage
}

\newcommand{\myMiniTocStyle}[1]{
	\renewcommand{\minitocstyle}{#1}
}



\newcommand{\myBibliography}[2]{
	\bibliographystyle{#1}
	\bibliography{#2}
	\myCleanStarChapterEnd
}

% Bibligraphie
\renewenvironment{thebibliography}[1]{%BEGIN
   \myChapterStar{\myBibliographyTitle}{}{true}\label{biblio}%
   \begin{myBiblio}
  }{%END
   \end{myBiblio}
}

\def\bibi[#1]{\item[\@biblabel{#1}\hfill]} % @ special
\newenvironment{myBiblio}{%BEGIN
   \list{}{
         \usecounter{enumiv}%
         \let\p@enumiv\@empty
         \renewcommand\theenumiv{\arabic{enumiv}}%
         \renewcommand\newblock{\hskip .11em \@plus.33em \@minus.07em}%
         %% dimensions horizontales
         \setlength{\leftmargin}{0mm}%%%
         %\setlength{\itemindent}{-3mm}%%%
         \setlength{\labelsep}{2mm}%%%
         \setlength{\labelwidth}{10mm}%%%
         %% dimensions verticales
          \setlength{\topsep}{0pt}%
          \setlength{\parskip}{6pt}%
          \setlength{\itemsep}{5pt}%
          \setlength{\partopsep}{0pt}%
          \setlength{\parsep}{3pt}%
         \sloppy\clubpenalty4000\widowpenalty4000%
         \sfcode`\.=\@m
         }%
  }{%END
      \def\@noitemerr{\@latex@warning{Empty 'thebibliography' environment}}
      %\FonteTexte%
      \endlist%
}

\renewcommand{\cite}[1]{\Citep{#1}}

% Tableaux
\usepackage[format=hang,font=small,labelfont=bf,textfont=it,skip=5pt,labelsep=endash]{caption}
\captionsetup[table]{name=Table,position=top}
\captionsetup[figure]{name=Figure,position=bottom}
\newcommand{\tocsetted}{false}

% Des redéfinitions supplémentaires
\let\oldmainmatter=\mainmatter
\renewcommand{\mainmatter}{
	\oldmainmatter
	% Numéroter les chapitres en chiffres romains
	\renewcommand{\thechapter}{\Roman{chapter}}
	% Numeroter les tableaux en chiffres romains
	\renewcommand{\thetable}{\Roman{table}}
	\renewcommand{\thefigure}{\arabic{figure}}
	
	\counterwithout*{figure}{chapter}
	\counterwithout*{table}{chapter}
}

\let\oldappendix=\appendix
\renewcommand{\appendix}{
	\oldappendix
	% Numeroter les tableaux en chiffres romains
	\renewcommand{\thetable}{\Roman{table}}
	\renewcommand{\thefigure}{\arabic{figure}}
	
	\counterwithout*{figure}{chapter}
	\counterwithout*{table}{chapter}
}



% Quelques raccourcis utiles
% *************** Nouvelles Commandes ***************
\newcommand{\myChapterLabel}{Chapter}
\newcommand{\myAppendixLabel}{Appendix}
\newcommand{\lifa}{Laboratoire d'Informatique Fondamentale et Appliquée (LIFA)}
\newcommand{\myBibliographyTitle}{Bibliography}
\newcommand{\losname}{List of Symbols}
\newcommand{\loaname}{List of Acronyms}

\newcommand{\doctypethesis}{Thèse de Doctorat en}
\newcommand{\doctypemaster}{Mémoire de Master en}
\newcommand{\doctype}{\ifthenelse{\equal{\doclevel}{\master}}{\doctypemaster}{\doctypethesis}}
\newcommand{\phd}{PhD}
\newcommand{\master}{Master}
\newcommand{\doclevel}{\phd}
\newcommand{\level}[1]{
	\renewcommand{\doclevel}{#1}
}
\newcommand{\phdthesislabel}{\doctype $~$ \studentspeciality $~$, Université de Dschang}
\newcommand{\studentspeciality}{\computerScience}
\newcommand{\speciality}[1]{
	\renewcommand{\studentspeciality}{#1}
}
\newcommand{\computerScience}{Informatique}
\newcommand{\mathematics}{Mathématiques}
\newcommand{\studentlab}{LIFA}
\newcommand{\lab}[1]{
	\renewcommand{\studentlab}{#1}
}

% Environnement personnalisé de description
\newcommand{\myDescription}[2]{
\par\vspace{0.35cm}\noindent\textbf{#1}
\begin{list}{}{}
	\item \noindent #2
\end{list}
\vspace{2pt}
}

\newcommand{\myTableOfContents}[1]{
	\ifthenelse{\equal{\tocsetted}{false}}
	{\clearpage}{}
	\mySaveMarks
	\ifthenelse{\equal{#1}{}}{}
	{\renewcommand{\contentsname}{#1}}
	\addcontentsline{toc}{section}{\myNumberLine{\contentsname}}
	\renewcommand{\leftmark}{\contentsname}
	\renewcommand{\rightmark}{\contentsname}
	\tableofcontents*
	\myCleanStarChapterEnd
	\renewcommand{\tocsetted}{true}
}

\newcommand{\myTableOfContentsStar}[1]{
	\ifthenelse{\equal{\tocsetted}{false}}
	{\clearpage}{}
	\mySaveMarks
	\ifthenelse{\equal{#1}{}}{}
	{\renewcommand{\contentsname}{#1}}
	\renewcommand{\leftmark}{\contentsname}
	\renewcommand{\rightmark}{\contentsname}
	\tableofcontents*
	\myCleanStarChapterEnd
	\renewcommand{\tocsetted}{true}
}

\newcommand{\myListOfSymbols}[1]{
	\ifthenelse{\equal{#1}{}}{}
	{\renewcommand{\losname}{#1}}
	\myChapterStar{\losname}{}{section}
	\begin{center}
	\begin{tabular}[t]{rp{5mm}p{12cm}}
		$EC$ & &  The Editor in Chief in the running example \\
		$AE$ & &  The Associated Editor in the running example \\
		$R1$ & &  The first Referee in the running example \\
		$R2$ & &  The second Referee in the running example \\
		$\mathbb{G}$ & &  A grammatical model of workflow \\
		$t_{i_f}$ & & A global artefact obtained after merging a set of artefacts
	\end{tabular}
\end{center}


	\myCleanStarChapterEnd
	\renewcommand{\tocsetted}{true}
}

\newcommand{\myListOfSymbolsStar}[1]{
	\ifthenelse{\equal{#1}{}}{}
	{\renewcommand{\losname}{#1}}
	\myChapterStar{\losname}{}{false}
	\begin{center}
	\begin{tabular}[t]{rp{5mm}p{12cm}}
		$EC$ & &  The Editor in Chief in the running example \\
		$AE$ & &  The Associated Editor in the running example \\
		$R1$ & &  The first Referee in the running example \\
		$R2$ & &  The second Referee in the running example \\
		$\mathbb{G}$ & &  A grammatical model of workflow \\
		$t_{i_f}$ & & A global artefact obtained after merging a set of artefacts
	\end{tabular}
\end{center}


	\myCleanStarChapterEnd
	\renewcommand{\tocsetted}{true}
}

\newcommand{\myListOfAcronyms}[1]{
	\ifthenelse{\equal{#1}{}}{}
	{\renewcommand{\loaname}{#1}}
	\myChapterStar{\loaname}{}{section}
	\begin{center}
	\begin{tabular}[t]{rp{5mm}p{10cm}}
		AST & & Abstract Syntax Tree \\
		BPM & & Business Process Management \\
		BPMN & & Business Process Model and Notation \\
		CDML & & Component Description Meta Language \\
		CSCW & & Computer-Supported Cooperative Work \\
		P2P & & Peer to Peer \\
		DS(E)L & & Domain Specific (Embeded) Language \\
		DTD & & Document Type Definition \\
		GMAWfP & & a Grammatical Model of Administrative Workflow Process \\
		GMWf & & Grammatical Model of Workflow \\
		LSAWfP & & a Language for the Specification of Administrative Workflow Processes \\
		(L)WfE & & (Local) Workflow Engine \\
		P2PTinyWfMS & & a Peer-to-Peer Tiny Workflow Management System \\
		SOA & & Service Oriented Architecture \\
		SON & & Shared-Overlay Network \\
		TinyCE & & a Tiny Cooperative Editor \\
		WfM(S) & & Workflow Management (System) \\
		WF-Net & & Workflow Net \\
		(WS-)BPEL & & (Web Services) Business Process Execution Language \\
		XML & & eXtensible Markup Language \\
		YAWL & & Yet Another Workflow Language \\
	\end{tabular}
\end{center}


	\myCleanStarChapterEnd
	\renewcommand{\tocsetted}{true}
}

\newcommand{\myListOfAcronymsStar}[1]{
	\ifthenelse{\equal{#1}{}}{}
	{\renewcommand{\loaname}{#1}}
	\myChapterStar{\loaname}{}{false}
	\begin{center}
	\begin{tabular}[t]{rp{5mm}p{10cm}}
		AST & & Abstract Syntax Tree \\
		BPM & & Business Process Management \\
		BPMN & & Business Process Model and Notation \\
		CDML & & Component Description Meta Language \\
		CSCW & & Computer-Supported Cooperative Work \\
		P2P & & Peer to Peer \\
		DS(E)L & & Domain Specific (Embeded) Language \\
		DTD & & Document Type Definition \\
		GMAWfP & & a Grammatical Model of Administrative Workflow Process \\
		GMWf & & Grammatical Model of Workflow \\
		LSAWfP & & a Language for the Specification of Administrative Workflow Processes \\
		(L)WfE & & (Local) Workflow Engine \\
		P2PTinyWfMS & & a Peer-to-Peer Tiny Workflow Management System \\
		SOA & & Service Oriented Architecture \\
		SON & & Shared-Overlay Network \\
		TinyCE & & a Tiny Cooperative Editor \\
		WfM(S) & & Workflow Management (System) \\
		WF-Net & & Workflow Net \\
		(WS-)BPEL & & (Web Services) Business Process Execution Language \\
		XML & & eXtensible Markup Language \\
		YAWL & & Yet Another Workflow Language \\
	\end{tabular}
\end{center}


	\myCleanStarChapterEnd
	\renewcommand{\tocsetted}{true}
}

\newcommand{\myListOfFigures}[1]{
	\ifthenelse{\equal{\tocsetted}{false}}
	{\clearpage}{}
	\mySaveMarks
	\ifthenelse{\equal{#1}{}}{}
	{\renewcommand{\listfigurename}{#1}}
	\addcontentsline{toc}{section}{\myNumberLine{\listfigurename}}
	\renewcommand{\leftmark}{\listfigurename}
	\renewcommand{\rightmark}{\listfigurename}
	\renewcommand{\lof}{true}
	\listoffigures*
	\renewcommand{\lof}{false}
	\myCleanStarChapterEnd
	\renewcommand{\tocsetted}{true}
}

\newcommand{\myListOfFiguresStar}[1]{
	\ifthenelse{\equal{\tocsetted}{false}}
	{\clearpage}{}
	\mySaveMarks
	\ifthenelse{\equal{#1}{}}{}
	{\renewcommand{\listfigurename}{#1}}
	\renewcommand{\leftmark}{\listfigurename}
	\renewcommand{\rightmark}{\listfigurename}
	\renewcommand{\lof}{true}
	\listoffigures*
	\renewcommand{\lof}{false}
	\myCleanStarChapterEnd
	\renewcommand{\tocsetted}{true}
}

\newcommand{\myListOfTables}[1]{
	\ifthenelse{\equal{\tocsetted}{false}}
	{\clearpage}{}
	\mySaveMarks
	\ifthenelse{\equal{#1}{}}{}
	{\renewcommand{\listtablename}{#1}}
	\addcontentsline{toc}{section}{\myNumberLine{\listtablename}}
	\renewcommand{\leftmark}{\listtablename}
	\renewcommand{\rightmark}{\listtablename}
	\renewcommand{\lof}{true}
	\listoftables*
	\renewcommand{\lof}{false}
	\myCleanStarChapterEnd
	\renewcommand{\tocsetted}{true}
}

\newcommand{\myListOfTablesStar}[1]{
	\ifthenelse{\equal{\tocsetted}{false}}
	{\clearpage}{}
	\mySaveMarks
	\ifthenelse{\equal{#1}{}}{}
	{\renewcommand{\listtablename}{#1}}
	\renewcommand{\leftmark}{\listtablename}
	\renewcommand{\rightmark}{\listtablename}
	\renewcommand{\lof}{true}
	\listoftables*
	\renewcommand{\lof}{false}
	\myCleanStarChapterEnd
	\renewcommand{\tocsetted}{true}
}

\newcommand{\myListOfAlgorithms}[1]{
	\ifthenelse{\equal{\tocsetted}{false}}
	{\clearpage}{}
	\mySaveMarks
	\ifthenelse{\equal{#1}{}}{}
	{\renewcommand{\listalgorithmname}{#1}}
	\addcontentsline{toc}{section}{\myNumberLine{\listalgorithmname}}
	\renewcommand{\leftmark}{\listalgorithmname}
	\renewcommand{\rightmark}{\listalgorithmname}
	%\renewcommand{\lof}{true}
	\listofalgorithms
	%\renewcommand{\lof}{false}
	\myCleanStarChapterEnd
	\renewcommand{\tocsetted}{true}
}

\newcommand{\myListOfAlgorithmsStar}[1]{
	\ifthenelse{\equal{\tocsetted}{false}}
	{\clearpage}{}
	\mySaveMarks
	\ifthenelse{\equal{#1}{}}{}
	{\renewcommand{\listalgorithmname}{#1}}
	\renewcommand{\leftmark}{\listalgorithmname}
	\renewcommand{\rightmark}{\listalgorithmname}
	\renewcommand{\lof}{true}
	\listofalgorithms
	\renewcommand{\lof}{false}
	\myCleanStarChapterEnd
	\renewcommand{\tocsetted}{true}
}

\newcommand{\myChapter}[2]{
	\chapter[#2]{#1}
}

\newcommand{\shortTitle}{}

\newcommand{\mySaveMarks}{
	\let\oldleftmark=\leftmark
	\let\oldrightmark=\rightmark
}

\newcommand{\myNumberLine}[1]{
	\hspace{-0.55cm}#1
}

\newcommand{\myChapterNumberLine}[1]{
	\hspace{-0.25cm}#1
}

\newcommand{\myChapterStar}[3]{
	\mySaveMarks
	\ifthenelse{\equal{#2}{}}
	{\renewcommand{\shortTitle}{#1}}
	{\renewcommand{\shortTitle}{#2}}
	\renewcommand{\leftmark}{\shortTitle}
	\renewcommand{\rightmark}{\shortTitle}
	\chapter*{#1}
	\ifthenelse{\equal{#3}{false}}
	{}
	{
		\ifthenelse{\equal{#3}{}}
		{\addcontentsline{toc}{chapter}{\myChapterNumberLine{\shortTitle}}}
		{
			\ifthenelse{\equal{#3}{true}}
			{\addcontentsline{toc}{chapter}{\myChapterNumberLine{\shortTitle}}}
			{
				\ifthenelse{\equal{#3}{chapter}}
				{\addcontentsline{toc}{#3}{\myChapterNumberLine{\shortTitle}}}
				{\addcontentsline{toc}{#3}{\myNumberLine{\shortTitle}}}
			}
		}
	}
}

\newcommand{\mySection}[2]{
	\resumecontents[chapters]
	\ifthenelse{\equal{#2}{}}
	{
		\renewcommand{\shortTitle}{#1}
		\Needspace{5\baselineskip}
		\section{#1}
	}
	{
		\renewcommand{\shortTitle}{#2}
		\Needspace{5\baselineskip}
		\section[#2]{#1}
	}
	\hrule height 0.5mm
	\vspace{5mm}
	\stopcontents[chapters]
}

\newcommand{\mySectionStar}[3]{
	\resumecontents[chapters]
	\ifthenelse{\equal{#2}{}}
	{\renewcommand{\shortTitle}{#1}}
	{\renewcommand{\shortTitle}{#2}}
	\renewcommand{\rightmark}{\shortTitle}
	\section*{#1}
	\hrule height 0.5mm
	\vspace{5mm}
	\ifthenelse{\equal{#3}{false}}
	{}
	{
		\ifthenelse{\equal{#3}{}}
		{\addcontentsline{toc}{section}{\myNumberLine{\shortTitle}}}
		{
			\ifthenelse{\equal{#3}{true}}
			{\addcontentsline{toc}{section}{\myNumberLine{\shortTitle}}}
			{
				\ifthenelse{\equal{#3}{chapter}}
				{\addcontentsline{toc}{#3}{\myChapterNumberLine{\shortTitle}}}
				{\addcontentsline{toc}{#3}{\myNumberLine{\shortTitle}}}
			}
		}
	}
	\stopcontents[chapters]
}

\newcommand{\mySubSection}[2]{
	\ifthenelse{\equal{\minitoclevel}{section}}{}
	{\resumecontents[chapters]}
	\ifthenelse{\equal{#2}{}}
	{
		%\renewcommand{\shortTitle}{#1}
		\subsection{#1}
	}
	{
		%\renewcommand{\shortTitle}{#2}
		\subsection[#2]{#1}
	}
	\stopcontents[chapters]
}

\newcommand{\mySubSectionStar}[3]{
	\ifthenelse{\equal{\minitoclevel}{section}}{}
	{\resumecontents[chapters]}
	\ifthenelse{\equal{#2}{}}
	{\renewcommand{\shortTitle}{#1}}
	{\renewcommand{\shortTitle}{#2}}
	\renewcommand{\rightmark}{\shortTitle}
	\subsection*{#1}
	\ifthenelse{\equal{#3}{false}}
	{}
	{
		\ifthenelse{\equal{#3}{}}
		{\addcontentsline{toc}{subsection}{\myNumberLine{\shortTitle}}}
		{
			\ifthenelse{\equal{#3}{true}}
			{\addcontentsline{toc}{subsection}{\myNumberLine{\shortTitle}}}
			{
				\ifthenelse{\equal{#3}{chapter}}
				{\addcontentsline{toc}{#3}{\myChapterNumberLine{\shortTitle}}}
				{\addcontentsline{toc}{#3}{\myNumberLine{\shortTitle}}}
			}
		}
	}
	\stopcontents[chapters]
}

\newcommand{\mySubSubSection}[2]{
	\ifthenelse{\equal{\minitoclevel}{subsubsection}}
	{\resumecontents[chapters]}{}
	\ifthenelse{\equal{#2}{}}
	{
		\renewcommand{\shortTitle}{#1}
		\subsubsection{#1}
	}
	{
		\renewcommand{\shortTitle}{#2}
		\subsubsection[#2]{#1}
	}
	\stopcontents[chapters]
}

\newcommand{\mySubSubSectionStar}[3]{
	\ifthenelse{\equal{\minitoclevel}{subsubsection}}
	{\resumecontents[chapters]}{}
	\ifthenelse{\equal{#2}{}}
	{\renewcommand{\shortTitle}{#1}}
	{\renewcommand{\shortTitle}{#2}}
	\renewcommand{\rightmark}{\shortTitle}
	\subsubsection*{#1}
	\ifthenelse{\equal{#3}{false}}
	{}
	{
		\ifthenelse{\equal{#3}{}}
		{\addcontentsline{toc}{subsubsection}{\myNumberLine{\shortTitle}}}
		{
			\ifthenelse{\equal{#3}{true}}
			{\addcontentsline{toc}{subsubsection}{\myNumberLine{\shortTitle}}}
			{
				\ifthenelse{\equal{#3}{chapter}}
				{\addcontentsline{toc}{#3}{\myChapterNumberLine{\shortTitle}}}
				{\addcontentsline{toc}{#3}{\myNumberLine{\shortTitle}}}
			}
		}
	}
	\stopcontents[chapters]
}

\newcommand{\myRestoreMarks}{
	\let\leftmark=\oldleftmark
	\let\rightmark=\oldrightmark
}

\newcommand{\myCleanStarChapterEnd}{
	\clearpage
	\myRestoreMarks
}

\newcommand{\currentlanguage}{english}

\newcommand{\switchLanguage}[1]{
	\renewcommand{\currentlanguage}{#1}
	\ifthenelse{\equal{#1}{français}}
	{
		\usepackage[frenchb]{babel}
		\renewcommand{\myChapterLabel}{Chapitre}
		\renewcommand{\myAppendixLabel}{Annexe}
		\renewcommand{\lifa}{Laboratoire d'Informatique Fondamentale et Appliquée (LIFA)}
		\renewcommand{\myBibliographyTitle}{Bibliographie}
		\renewcommand{\phdthesislabel}{Thèse de Doctorat en $~$ \studentspeciality $~$, Université de Dschang}
		\renewcommand{\computerScience}{Informatique}
		\renewcommand{\mathematics}{Mathématiques}
		\renewcommand{\studentlab}{URIFIA}
		\renewcommand{\doctypethesis}{Thèse de Doctorat en}
		\renewcommand{\doctypemaster}{Mémoire de Master en}
		\renewcommand{\losname}{Liste des Symboles}
		\renewcommand{\loaname}{Liste des Acronymes}
	}{
		\usepackage[english]{babel}
		\renewcommand{\myChapterLabel}{Chapter}
		\renewcommand{\myAppendixLabel}{Appendix}
		\renewcommand{\lifa}{Laboratoire d'Informatique Fondamentale et Appliquée (LIFA)}
		\renewcommand{\myBibliographyTitle}{Bibliography}
				\renewcommand{\phdthesislabel}{PhD Thesis in $~$ \studentspeciality $~$, University of Dschang}
		\renewcommand{\computerScience}{Computer Science}
		\renewcommand{\mathematics}{Mathematics}
		\renewcommand{\studentlab}{URIFIA}
		\renewcommand{\doctypethesis}{PhD Thesis in}
		\renewcommand{\doctypemaster}{Master Report in}
		\renewcommand{\losname}{List of Symbols}
		\renewcommand{\loaname}{List of Acronyms}
	}
}


\newcommand{\documentType}[1]{
	\ifthenelse{\equal{#1}{numerical}}
	{
		% *************** Activation des liens hypertexte ***************
		\ifpdf
			\pdfcompresslevel=9
				\usepackage[plainpages=false,pdfpagelabels,bookmarksnumbered,%
				colorlinks=true,%
				linkcolor=blue,%
				citecolor=blue,%
				filecolor=forestgreen,%
				urlcolor=midnightblue,%
				pdftex,%
				unicode]{hyperref}
			\pdfimageresolution=600
			\usepackage{thumbpdf} 
		\else
			\usepackage{hyperref}
		\fi
	}
	{
		% *************** Activation des liens hypertexte ***************
		\ifpdf
			\pdfcompresslevel=9
				\usepackage[plainpages=false,pdfpagelabels,bookmarksnumbered,%
				colorlinks=true,%
				linkcolor=black,%
				citecolor=black,%
				filecolor=black,%
				urlcolor=black,%
				pdftex,%
				unicode]{hyperref}
			\pdfimageresolution=600
			\usepackage{thumbpdf} 
		\else
			\usepackage{hyperref}
		\fi
	}
}


\letcountercounter{sidefootnote}{footnote}


\DeclareMathVersion{normal2}

% Choisissez le type de document
% Les deux choix possibles sont numerical (copie à diffuser numériquement - usage abondant de couleur) et physical (copie à imprimer - moins de couleur)
\documentType{numerical}

% Choississez la langue de votre thèse (obligatoire)
% Les choix possibles sont english et français
\switchLanguage{english}

% Import de vos définitions et de votre style
% Un template bilingue pour la production des mémoires et thèses dans le département de Mathématiques-Informatique.
% Ce template est conforme aux recommandations de l'école doctorale
%
% Ce fichier est conçu pour accueillir vos imports (\usepackage) et vos propres définitions d'environnements latex et/ou de style
% Consulter le fichier style.tex pour savoir ce qui a déja été importé
%
% @author Zekeng Ndadji Milliam Maxime

\usepackage{algorithm}
\usepackage{algpseudocode}
\usepackage{listings}% http://ctan.org/pkg/listings  Pour ecrire des math dans un verbatim \begin{lstlisting} ... \end{lstlisting}
\lstset{
  basicstyle=\ttfamily,
  mathescape
}

\renewcommand{\listalgorithmname}{List of Algorithms}
\floatname{algorithm}{Algorithm}
%\renewcommand{\algorithmicreturn}{\textbf{retourne}}
%\renewcommand{\algorithmicprocedure}{\textbf{procédure}}
%\newcommand{\Not}{\textbf{non}\ }
%\newcommand{\AlgAnd}{\textbf{et}\ }
%\newcommand{\Or}{\textbf{ou}\ }
\newcommand{\To}{\textbf{to}\ }
\let\oldComment=\Comment
\renewcommand{\Comment}[1]{\oldComment{{\scriptsize#1}}}
%\renewcommand{\algorithmicrequire}{\textbf{Entrée:}}
%\renewcommand{\algorithmicensure}{\textbf{Sortie:}}
%\renewcommand{\algorithmiccomment}[1]{\{#1\}}
%\renewcommand{\algorithmicend}{\textbf{fin}}
%\renewcommand{\algorithmicif}{\textbf{si}}
%\renewcommand{\algorithmicthen}{\textbf{alors}}
%\renewcommand{\algorithmicelse}{\textbf{sinon}}
%\renewcommand{\algorithmicfor}{\textbf{pour}}
%\renewcommand{\algorithmicforall}{\textbf{pour tout}}
%\renewcommand{\algorithmicdo}{\textbf{faire}}
%\renewcommand{\algorithmicwhile}{\textbf{tant que}}
\algdef{SE}[DOWHILE]{DoWhile}{EndDoWhile}{\algorithmicdo}[1]{\algorithmicwhile\ #1}%
\newcommand{\algorithmicelsif}{\algorithmicelse\ \algorithmicif}
\newcommand{\algorithmicendif}{\algorithmicend\ \algorithmicif}
\newcommand{\algorithmicendfor}{\algorithmicend\ \algorithmicfor}


%Définition de nouveaux environnements de type théorème
\newtheorem{theorem}{Theorem}
\newtheorem{definition}[theorem]{Definition}
\newtheorem{proposition}[theorem]{Proposition}
\newtheorem{lemma}[theorem]{Lemma}
\newtheorem{example}[theorem]{Example}
\newtheorem{remark}[theorem]{Remark}
\newtheorem{corollary}[theorem]{Corollary}
\newtheorem{problem}[theorem]{Problem}

%Environnements de preuve
\newenvironment{proof}[1][{\textbf{Proof}}]{
	\par
	\normalfont
	\topsep6\p@\@plus6\p@ \trivlist
	\item[\hskip\labelsep\itshape
	#1\@addpunct{.}]\ignorespaces
}{%
	\qed\endtrivlist
}
\newenvironment{preuve}[1][{\textbf{Preuve}}]{
	\par
	\normalfont
	\topsep6\p@\@plus6\p@ \trivlist
	\item[\hskip\labelsep\itshape
	#1\@addpunct{.}]\ignorespaces
}{%
	\qed\endtrivlist
}

\usepackage[framemethod=TikZ]{mdframed}
\mdfdefinestyle{MyFrame}{%
	linecolor=white,
	outerlinewidth=0pt,
	roundcorner=0pt,
	innertopmargin=4pt,
	innerbottommargin=4pt,
	innerrightmargin=4pt,
	innerleftmargin=4pt,
	leftmargin=4pt,
	rightmargin=4pt
	%backgroundcolor=gray!50!white
}

\usepackage{csquotes}

\renewcommand{\BBA}{and}








% Thèse ou mémoire (\phd | \master)
\level{\phd}

% Spécialité de la thèse ou du mémoire: entrez la votre
\speciality{\computerScience}

% Initiales du laboratoire
\lab{URIFIA}

% Style des titres de chapitres (fieldset | titleontopright | default | section | hangnum | companion | article | demo | veelo | bringhurst | southall | chappell)
\myChapterStyle{fieldset-rev}
% Style de la minitoc (fieldset | titleontopright)
\myMiniTocStyle{fieldset}

\title{A Grammatical Approach to Peer-to-Peer Cooperative Editing on a Service-Oriented Architecture}
%\title{Yet Another Approach to Facilitate Workflow Design and Distributed Execution using Structured and Cooperatively Edited Mobile Artifacts}
\author{ZEKENG NDADJI Milliam Maxime}
\date{\today}

\begin{document}
{
	% Taille de la police du texte
	\timesContentFontSize

	% Code pour inclure des documents au format PDF
	\includepdf[pages=-, offset=72 -72]{Couverture.pdf} %Insertion de la couverture
	%\includepdf[pages=-, offset=72 -72]{Originalite.pdf} %Attestation d'originalité
	%\includepdf[pages=-, offset=72 -72]{Correction.pdf} %Attestaion de correction

	\pagestyle{ruled}
	\nouppercaseheads
	\normalfont
	

	\frontmatter
	
	% Inclusion des fichiers de dédicaces et de remerciements
	%%\myChapterStar{Titre}{Titre court}{Ajouter à la table des matières? (false|true|chapter|section|subsection|subsubsection -chapter par défaut-)}
\myChapterStar{Declaration of Authorship}{}{false}

I, the undersigned student \textsc{ZEKENG NDADJI Milliam Maxime}, registration number CM04-10SCI1755, declare that this thesis submitted in fulfilment of the requirements for obtaining the degree of \textit{Doctor/PhD} in Computer Science, option \textit{Software Engineering}, speciality \textit{Business Process Management}, is the result of my own work unless otherwise referenced or acknowledged.

~

\begin{center}
	\textbf{The Student}
	
	~
	
	~
	
	~
	
	\textbf{\textsc{ZEKENG NDADJI Milliam Maxime}}
	
	Date: ...................................
\end{center}

~

\begin{center}
	\textbf{The Co-Directors}
\end{center}

~

~

~

%\begin{multicols}{2}
\hspace{-7mm}
\parbox[c]{.5\textwidth}{
	%\begin{minipage}
		\begin{center}
			\textbf{\textsc{TCHOUP\'E TCHENDJI Maurice}}
			
			\textit{Senior Lecturer, University of Dschang, Cameroon}
			
			Date: ...................................
		\end{center}
	%\end{minipage}
}
	%\columnbreak
\parbox[c]{.5\textwidth}{
	%\begin{minipage}
		\begin{center}
			\textbf{\textsc{TAYOU DJAMEGNI Cl\'ementin}}
			
			\textit{Full Professor, University of Dschang, Cameroon}
			
			Date: ...................................
		\end{center}
	%\end{minipage}
}
%\end{multicols}





	%\myChapterStar{Titre}{Titre court}{Ajouter à la table des matières? (false|true|chapter|section|subsection|subsubsection -chapter par défaut-)}
\myChapterStar{Dedication}{}{section}
\vspace*{5cm}
\begin{flushright}
\textit{I dedicate this work to all those who one day, saw the efforts they had put into a project (especially an intense love relationship) being wiped out without a logical explanation and who took it upon themselves, to ride the waves unleashed by these storms, came out ten times stronger and developed an incredible desire to live rather than die. Like your daily efforts, this work is in large part the result of the knowledge that suffering has brought to me and thus, it contributes to prove Friedrich Nietzsche's aphorism: "\textbf{what doesn't kill you, makes you stronger}".}
\end{flushright}


	%\myChapterStar{Titre}{Titre court}{Ajouter à la table des matières? (false|true|chapter|section|subsection|subsubsection -chapter par défaut-)}
\myChapterStar{Acknowledgements}{}{section}
Along the paths I followed during this thesis, I have gained a unique experience and, I hope, the necessary maturity to aspire to the title of Doctor/PhD. Throughout these years of work, I have come to realise that producing a thesis is far from being a solitary labour. In order to go through this tunnel, one must be under the benevolent escort of the Almighty Lord and of souls of good faith. I have benefited from a multiform accompaniment of many persons, both physical and moral. I would like them to find in this section, the expression of my deep gratitude. I therefore thank very warmly:

~

\noindent- The \textsc{LORD} \textit{our God, creator of heaven and earth}: in addition to my days, he offers me every day, the necessary grace and strength to continue to glorify him. Glory be to you, Holy Father!

~

\noindent- Dr \textsc{TCHOUP{\'E} TCHENDJI Maurice}, \textit{Senior Lecturer at the Department of Mathematics and Computer Science of the Faculty of Science of the University of Dschang}: he took the time to mentor me throughout this work, inculcating in me his sense of commitment and organisation. Sir, I reiterate that you are my model.

~

\noindent- Pr \textsc{TAYOU DJAMEGNI Cl{\'e}mentin}, \textit{Head of the Computer Engineering Department of the Institute of Technology - Fotso Victor of Bandjoun}: it was under difficult conditions that he agreed to supervise this work and to reward us with his legendary positivism. I'm more than honoured to be one of your students, Sir.

~

\noindent- All the imminent members of my pre-hearing and defence juries: they agreed to objectively evaluate this work. Thank you for the chance you are giving me, gentlemen.

~

\noindent- Dr \textsc{PARIGOT Didier}, \textit{Senior Researcher on Programming Languages at INRIA}: he accompanied me during this thesis by his reviews, his points of view, his recommendations and his know-how in terms of software programming. I learned a lot from your multiple contributions, Sir.

~

\noindent- The \textit{INRIA/LIRIMA FUCHSIA Associate Research Team}: they welcomed me when I needed it most and gave me a chance to express myself. Working with you guys is a lifelong dream.

~

\noindent- The \textit{lecturers at the University of Dschang}: they contributed enormously to my education. As education is priceless, I can only express my gratitude.

~

\noindent- My \textit{parents}, \textsc{NDADJI Emmanuel} and \textsc{MAFFOZEMTSOP Marie}: they gave me everything. I love you infinitely.

~

\noindent- My \textit{brothers, sisters, friends and colleagues (Nani, Chance, Alex, Arnold, Brice, Ari\`ege, Brel, Doris, Audrey, Fabrice, Rodrigue, Virginie, Emeric, Nestor, Preston, Yann, Max, Ange, Lionel, etc.)}: I live only by you and for you (uh, for me too ;-)).

~

\noindent- My dear \textsc{BANGUKET T. Nina}: she proofread and corrected the typos in all my English documents. Now I can write a correct sentence in English, thanks to you.




\myCleanStarChapterEnd


	
	% Génération de la table des matières (avec "Table of Contents" comme titre), de la liste des symboles, de la liste des acronymes, de la liste des tableaux et de la liste des figures
	% Des alternatives à ces commandes sont dispos: il suffit juste d'ajouter le suffixe "Star" pour ne pas les mettre dans la table de matière
	\myTableOfContents{Table of Contents}
	
	% Resumé et abstract
	  In this paper, we explore the connection between secret key agreement and secure omniscience within the setting of the multiterminal source model with a wiretapper who has side information. While the secret key agreement problem considers the generation of a maximum-rate secret key through public discussion, the secure omniscience problem is concerned with communication protocols for omniscience that minimize the rate of information leakage to the wiretapper. The starting point of our work is a lower bound on the minimum leakage rate for omniscience, $\rl$, in terms of the wiretap secret key capacity, $\wskc$. Our interest is in identifying broad classes of sources for which this lower bound is met with equality, in which case we say that there is a duality between secure omniscience and secret key agreement. We show that this duality holds in the case of certain finite linear source (FLS) models, such as two-terminal FLS models and pairwise independent network models on trees with a linear wiretapper. Duality also holds for any FLS model in which $\wskc$ is achieved by a perfect linear secret key agreement scheme. We conjecture that the duality in fact holds unconditionally for any FLS model. On the negative side, we give an example of a (non-FLS) source model for which duality does not hold if we limit ourselves to communication-for-omniscience protocols with at most two (interactive) communications.  We also address the secure function computation problem and explore the connection between the minimum leakage rate for computing a function and the wiretap secret key capacity.
  
%   Finally, we demonstrate the usefulness of our lower bound on $\rl$ by using it to derive equivalent conditions for the positivity of $\wskc$ in the multiterminal model. This extends a recent result of Gohari, G\"{u}nl\"{u} and Kramer (2020) obtained for the two-user setting.
  
   
%   In this paper, we study the problem of secret key generation through an omniscience achieving communication that minimizes the 
%   leakage rate to a wiretapper who has side information in the setting of multiterminal source model.  We explore this problem by deriving a lower bound on the wiretap secret key capacity $\wskc$ in terms of the minimum leakage rate for omniscience, $\rl$. 
%   %The former quantity is defined to be the maximum secret key rate achievable, and the latter one is defined as the minimum possible leakage rate about the source through an omniscience scheme to a wiretapper. 
%   The main focus of our work is the characterization of the sources for which the lower bound holds with equality \textemdash it is referred to as a duality between secure omniscience and wiretap secret key agreement. For general source models, we show that duality need not hold if we limit to the communication protocols with at most two (interactive) communications. In the case when there is no restriction on the number of communications, whether the duality holds or not is still unknown. However, we resolve this question affirmatively for two-user finite linear sources (FLS) and pairwise independent networks (PIN) defined on trees, a subclass of FLS. Moreover, for these sources, we give a single-letter expression for $\wskc$. Furthermore, in the direction of proving the conjecture that duality holds for all FLS, we show that if $\wskc$ is achieved by a \emph{perfect} secret key agreement scheme for FLS then the duality must hold. All these results mount up the evidence in favor of the conjecture on FLS. Moreover, we demonstrate the usefulness of our lower bound on $\wskc$ in terms of $\rl$ by deriving some equivalent conditions on the positivity of secret key capacity for multiterminal source model. Our result indeed extends the work of Gohari, G\"{u}nl\"{u} and Kramer in two-user case.
	%\myChapterStar{}{}{false}
\begin{center}
\chaptitlefont %\hrule height 1.5pt
\begin{center}\textcolor{black}{\textsc{Une Approche Grammaticale d'Edition Coop\'erative entre pairs sur une Architecture Orient\'ee Service}}
\end{center}
\end{center}
\vspace{8mm}
\textcolor{black}{\hrule height 4.5pt}%
\vspace{1mm}
\textcolor{black}{\hrule height 1.5pt}

	%\let\oldprintchaptertitle=\printchaptertitle
\renewcommand{\printchaptertitle}[1]{%
	\vspace*{-75pt}
	\oldprintchaptertitle{#1}
}%
\myChapterStar{Résumé}{}{section}
\let\printchaptertitle=\oldprintchaptertitle
Dans cette thèse, nous nous intéressons à la conception et à l'implémentation des systèmes workflows distribués, dédiés à l'automatisation des processus opérationnels dits administratifs. Nous proposons une approche de mise en oeuvre de tels systèmes en nous appuyant sur les concepts de systèmes multi-agents, d'architecture Pair à Pair (P2P, égal à égal), d'architecture orientée service et d'édition coopérative de documents structurés (artefacts). 
En effet, nous développons des outils mathématiques permettant à tout concepteur de systèmes workflows, d'exprimer chaque processus administratif qu'il désire automatiser, sous forme d'une grammaire attribuée dont les symboles représentent les tâches à exécuter, les productions spécifient un ordonnancement de celles-ci, et les instances (les arbres de dérivation qui lui sont conformes) représentent les différents scénarii d'exécution menant aux différents états qui matérialisent l'accomplissement des objectifs de l'entreprise. Le modèle grammatical obtenu est ensuite introduit dans un système P2P que nous proposons, et qui est chargé d'assurer l'exécution complètement décentralisée d'instances du processus sous-jacent. 
Ledit système orchestre l'exécution d'une instance de processus sous forme d'une chorégraphie durant laquelle, divers agents logiciels pilotés par des agents humains (acteurs), se coordonnent à l'aide d'artefacts qu'ils éditent collégialement. Les artefacts échangés représentent la mémoire du système: ils donnent des informations sur les tâches déjà exécutées, sur celles prêtes à l'être et sur leurs exécutants. Les agents logiciels sont autonomes et identiques: ils exécutent le même et unique protocole à chaque fois qu'ils reçoivent un artefact. Ce protocole leur permet d'identifier les tâches qu'ils doivent exécuter dans l'immédiat, de les exécuter, de mettre à jour l'artefact et de le diffuser si nécessaire, pour la continuation du processus d'exécution. 
En outre, les acteurs n'ont potentiellement qu'une perception partielle des processus dans lesquels ils sont impliqués. Ce qui autorise donc une possible exécution confidentielle de certaines tâches en pratique: cette propriété permet d'offrir une gestion automatique des processus administratifs se rapprochant un peu plus, de leur gestion non informatisée.

\vspace{1cm}
\noindent\textbf{Mots clés:} Workflows Administratifs, Artefacts, Pair à Pair, Réplique Partielle, Gestion des Processus Opérationnels.

\myCleanStarChapterEnd


	\clearpage
	
	%\myListOfSymbols{}
	\myListOfAcronyms{}
	\myListOfTables{}
	\myListOfFigures{}
	\myListOfAlgorithms{}
	
	
	% *********** Partie principale ***********
	\mainmatter
	
	% Inclusion des différents chapitres
	% \leavevmode
% \\
% \\
% \\
% \\
% \\
\section{Introduction}
\label{introduction}

AutoML is the process by which machine learning models are built automatically for a new dataset. Given a dataset, AutoML systems perform a search over valid data transformations and learners, along with hyper-parameter optimization for each learner~\cite{VolcanoML}. Choosing the transformations and learners over which to search is our focus.
A significant number of systems mine from prior runs of pipelines over a set of datasets to choose transformers and learners that are effective with different types of datasets (e.g. \cite{NEURIPS2018_b59a51a3}, \cite{10.14778/3415478.3415542}, \cite{autosklearn}). Thus, they build a database by actually running different pipelines with a diverse set of datasets to estimate the accuracy of potential pipelines. Hence, they can be used to effectively reduce the search space. A new dataset, based on a set of features (meta-features) is then matched to this database to find the most plausible candidates for both learner selection and hyper-parameter tuning. This process of choosing starting points in the search space is called meta-learning for the cold start problem.  

Other meta-learning approaches include mining existing data science code and their associated datasets to learn from human expertise. The AL~\cite{al} system mined existing Kaggle notebooks using dynamic analysis, i.e., actually running the scripts, and showed that such a system has promise.  However, this meta-learning approach does not scale because it is onerous to execute a large number of pipeline scripts on datasets, preprocessing datasets is never trivial, and older scripts cease to run at all as software evolves. It is not surprising that AL therefore performed dynamic analysis on just nine datasets.

Our system, {\sysname}, provides a scalable meta-learning approach to leverage human expertise, using static analysis to mine pipelines from large repositories of scripts. Static analysis has the advantage of scaling to thousands or millions of scripts \cite{graph4code} easily, but lacks the performance data gathered by dynamic analysis. The {\sysname} meta-learning approach guides the learning process by a scalable dataset similarity search, based on dataset embeddings, to find the most similar datasets and the semantics of ML pipelines applied on them.  Many existing systems, such as Auto-Sklearn \cite{autosklearn} and AL \cite{al}, compute a set of meta-features for each dataset. We developed a deep neural network model to generate embeddings at the granularity of a dataset, e.g., a table or CSV file, to capture similarity at the level of an entire dataset rather than relying on a set of meta-features.
 
Because we use static analysis to capture the semantics of the meta-learning process, we have no mechanism to choose the \textbf{best} pipeline from many seen pipelines, unlike the dynamic execution case where one can rely on runtime to choose the best performing pipeline.  Observing that pipelines are basically workflow graphs, we use graph generator neural models to succinctly capture the statically-observed pipelines for a single dataset. In {\sysname}, we formulate learner selection as a graph generation problem to predict optimized pipelines based on pipelines seen in actual notebooks.

%. This formulation enables {\sysname} for effective pruning of the AutoML search space to predict optimized pipelines based on pipelines seen in actual notebooks.}
%We note that increasingly, state-of-the-art performance in AutoML systems is being generated by more complex pipelines such as Directed Acyclic Graphs (DAGs) \cite{piper} rather than the linear pipelines used in earlier systems.  
 
{\sysname} does learner and transformation selection, and hence is a component of an AutoML systems. To evaluate this component, we integrated it into two existing AutoML systems, FLAML \cite{flaml} and Auto-Sklearn \cite{autosklearn}.  
% We evaluate each system with and without {\sysname}.  
We chose FLAML because it does not yet have any meta-learning component for the cold start problem and instead allows user selection of learners and transformers. The authors of FLAML explicitly pointed to the fact that FLAML might benefit from a meta-learning component and pointed to it as a possibility for future work. For FLAML, if mining historical pipelines provides an advantage, we should improve its performance. We also picked Auto-Sklearn as it does have a learner selection component based on meta-features, as described earlier~\cite{autosklearn2}. For Auto-Sklearn, we should at least match performance if our static mining of pipelines can match their extensive database. For context, we also compared {\sysname} with the recent VolcanoML~\cite{VolcanoML}, which provides an efficient decomposition and execution strategy for the AutoML search space. In contrast, {\sysname} prunes the search space using our meta-learning model to perform hyperparameter optimization only for the most promising candidates. 

The contributions of this paper are the following:
\begin{itemize}
    \item Section ~\ref{sec:mining} defines a scalable meta-learning approach based on representation learning of mined ML pipeline semantics and datasets for over 100 datasets and ~11K Python scripts.  
    \newline
    \item Sections~\ref{sec:kgpipGen} formulates AutoML pipeline generation as a graph generation problem. {\sysname} predicts efficiently an optimized ML pipeline for an unseen dataset based on our meta-learning model.  To the best of our knowledge, {\sysname} is the first approach to formulate  AutoML pipeline generation in such a way.
    \newline
    \item Section~\ref{sec:eval} presents a comprehensive evaluation using a large collection of 121 datasets from major AutoML benchmarks and Kaggle. Our experimental results show that {\sysname} outperforms all existing AutoML systems and achieves state-of-the-art results on the majority of these datasets. {\sysname} significantly improves the performance of both FLAML and Auto-Sklearn in classification and regression tasks. We also outperformed AL in 75 out of 77 datasets and VolcanoML in 75  out of 121 datasets, including 44 datasets used only by VolcanoML~\cite{VolcanoML}.  On average, {\sysname} achieves scores that are statistically better than the means of all other systems. 
\end{itemize}


%This approach does not need to apply cleaning or transformation methods to handle different variances among datasets. Moreover, we do not need to deal with complex analysis, such as dynamic code analysis. Thus, our approach proved to be scalable, as discussed in Sections~\ref{sec:mining}.
	%\myChapter{Titre}{Titre court}
\myChapter{A State of the Art in Business Process Management: the Artifact-Centric Modelling}{}
\label{chap1:artifact-centric-bpm}
\myMiniToc{section}{Contents}
% If no minitoc then
% \startcontents[chapters]
% \leavevmode
% \\
% \\
% \\
% \\
% \\
\section{Introduction}
\label{introduction}

AutoML is the process by which machine learning models are built automatically for a new dataset. Given a dataset, AutoML systems perform a search over valid data transformations and learners, along with hyper-parameter optimization for each learner~\cite{VolcanoML}. Choosing the transformations and learners over which to search is our focus.
A significant number of systems mine from prior runs of pipelines over a set of datasets to choose transformers and learners that are effective with different types of datasets (e.g. \cite{NEURIPS2018_b59a51a3}, \cite{10.14778/3415478.3415542}, \cite{autosklearn}). Thus, they build a database by actually running different pipelines with a diverse set of datasets to estimate the accuracy of potential pipelines. Hence, they can be used to effectively reduce the search space. A new dataset, based on a set of features (meta-features) is then matched to this database to find the most plausible candidates for both learner selection and hyper-parameter tuning. This process of choosing starting points in the search space is called meta-learning for the cold start problem.  

Other meta-learning approaches include mining existing data science code and their associated datasets to learn from human expertise. The AL~\cite{al} system mined existing Kaggle notebooks using dynamic analysis, i.e., actually running the scripts, and showed that such a system has promise.  However, this meta-learning approach does not scale because it is onerous to execute a large number of pipeline scripts on datasets, preprocessing datasets is never trivial, and older scripts cease to run at all as software evolves. It is not surprising that AL therefore performed dynamic analysis on just nine datasets.

Our system, {\sysname}, provides a scalable meta-learning approach to leverage human expertise, using static analysis to mine pipelines from large repositories of scripts. Static analysis has the advantage of scaling to thousands or millions of scripts \cite{graph4code} easily, but lacks the performance data gathered by dynamic analysis. The {\sysname} meta-learning approach guides the learning process by a scalable dataset similarity search, based on dataset embeddings, to find the most similar datasets and the semantics of ML pipelines applied on them.  Many existing systems, such as Auto-Sklearn \cite{autosklearn} and AL \cite{al}, compute a set of meta-features for each dataset. We developed a deep neural network model to generate embeddings at the granularity of a dataset, e.g., a table or CSV file, to capture similarity at the level of an entire dataset rather than relying on a set of meta-features.
 
Because we use static analysis to capture the semantics of the meta-learning process, we have no mechanism to choose the \textbf{best} pipeline from many seen pipelines, unlike the dynamic execution case where one can rely on runtime to choose the best performing pipeline.  Observing that pipelines are basically workflow graphs, we use graph generator neural models to succinctly capture the statically-observed pipelines for a single dataset. In {\sysname}, we formulate learner selection as a graph generation problem to predict optimized pipelines based on pipelines seen in actual notebooks.

%. This formulation enables {\sysname} for effective pruning of the AutoML search space to predict optimized pipelines based on pipelines seen in actual notebooks.}
%We note that increasingly, state-of-the-art performance in AutoML systems is being generated by more complex pipelines such as Directed Acyclic Graphs (DAGs) \cite{piper} rather than the linear pipelines used in earlier systems.  
 
{\sysname} does learner and transformation selection, and hence is a component of an AutoML systems. To evaluate this component, we integrated it into two existing AutoML systems, FLAML \cite{flaml} and Auto-Sklearn \cite{autosklearn}.  
% We evaluate each system with and without {\sysname}.  
We chose FLAML because it does not yet have any meta-learning component for the cold start problem and instead allows user selection of learners and transformers. The authors of FLAML explicitly pointed to the fact that FLAML might benefit from a meta-learning component and pointed to it as a possibility for future work. For FLAML, if mining historical pipelines provides an advantage, we should improve its performance. We also picked Auto-Sklearn as it does have a learner selection component based on meta-features, as described earlier~\cite{autosklearn2}. For Auto-Sklearn, we should at least match performance if our static mining of pipelines can match their extensive database. For context, we also compared {\sysname} with the recent VolcanoML~\cite{VolcanoML}, which provides an efficient decomposition and execution strategy for the AutoML search space. In contrast, {\sysname} prunes the search space using our meta-learning model to perform hyperparameter optimization only for the most promising candidates. 

The contributions of this paper are the following:
\begin{itemize}
    \item Section ~\ref{sec:mining} defines a scalable meta-learning approach based on representation learning of mined ML pipeline semantics and datasets for over 100 datasets and ~11K Python scripts.  
    \newline
    \item Sections~\ref{sec:kgpipGen} formulates AutoML pipeline generation as a graph generation problem. {\sysname} predicts efficiently an optimized ML pipeline for an unseen dataset based on our meta-learning model.  To the best of our knowledge, {\sysname} is the first approach to formulate  AutoML pipeline generation in such a way.
    \newline
    \item Section~\ref{sec:eval} presents a comprehensive evaluation using a large collection of 121 datasets from major AutoML benchmarks and Kaggle. Our experimental results show that {\sysname} outperforms all existing AutoML systems and achieves state-of-the-art results on the majority of these datasets. {\sysname} significantly improves the performance of both FLAML and Auto-Sklearn in classification and regression tasks. We also outperformed AL in 75 out of 77 datasets and VolcanoML in 75  out of 121 datasets, including 44 datasets used only by VolcanoML~\cite{VolcanoML}.  On average, {\sysname} achieves scores that are statistically better than the means of all other systems. 
\end{itemize}


%This approach does not need to apply cleaning or transformation methods to handle different variances among datasets. Moreover, we do not need to deal with complex analysis, such as dynamic code analysis. Thus, our approach proved to be scalable, as discussed in Sections~\ref{sec:mining}.
\mySection{Key Principles of Business Process Management}{}
\label{chap1:sec:bpm-def-key-principles}
%\subsection*{Contexte et définitions}
%\label{sec:contexte}

\mySubSection{Some Business Process Management Basic Concepts}{}
\label{chap1:sec:bpm-def-history}
Research in the CSCW field focuses on the role of computers in collaborative work \cite{schimdt1992taking}. These have given rise to numerous softwares called \textit{CSCW systems} or \textit{groupware}. CSCW systems communicate through networks and provide functionalities facilitating exchanges, coordination, collaboration and co-decision between the actors of a given collaborative work; they thus defy the space and time constraints to which collaborative work is subjected. Indeed, with the help of such systems, actors can either operate on the same site and thus manipulate the same objects (\textit{centralised approach}), or they can operate on geographically distant sites (\textit{distributed approach}); in this case, the objects they manipulate are replicated on the different sites and synchronised at the appropriate time \cite{johansen1988groupware, grudin1994computer, penichet2007classification}. In the same vein, they can act at the same time (\textit{synchronous approach}) or at completely different times and sometimes independently of the actions carried out by others (\textit{asynchronous approach}) \cite{johansen1988groupware, grudin1994computer, penichet2007classification}. 
CSCW systems are often referred to as \textit{workflow systems}. However, it should be noted that workflow is an extension and a generalisation of CSCW to business processes' automation.

\mySubSubSection{Some Definitions}{}
\label{chap1:sec:bpm-basic-concepts-def}
A \textit{business process} can be informally defined as a set of tasks ordered following a specific pattern and whose execution produces a service or a particular business goal \cite{workflow95}. When such a process is managed electronically, it is called \textit{workflow}. The purpose of workflow is to streamline, coordinate and control business processes in an organised, distributed and computerised environment. The peer-review validation of an article in a scientific journal is a common example of business process. Descriptions of it can be found in \cite{peerReview02, van2001proclets, badouel14}. As in most literature works, most of the time, we will use the terms "business process" and "workflow" as synonyms in the rest of this manuscript.

The \textit{Workflow Management Coalition}\footnote{The growing reputation of workflow led to the creation, in 1993, of the \textit{Workflow Management Coalition} (WfMC) as the organisation responsible for developing standards in this field. Official website of the WfMC: \url{https://www.wfmc.org/}.} (WfMC) \cite{workflowModel} defines \textit{Workflow Management} (WfM) as the modelling and computer management of all the tasks and different actors involved in executing a business process. WfM is achieved using \textit{Workflow Management Systems} (WfMS): these are complex systems with the aim of automating at best workflows by providing an appropriate framework to facilitate collaboration between actors involved in business processes' execution \cite{workflow95, van2013business, van2015business, dumas2018fundamental}. WfMS are composed of logically orchestrated tools to specificy, to optimise, to automate and to monitor business processes \cite{workflowModel, dumas2018fundamental}. Technically, the management of a process according to WfM is done in two phases \cite{divitini2001inter}:
\begin{enumerate}
	\item the \textit{process modelling phase}: the process is studied and then specified using a language (usually graphical) called \textit{workflow language}. The resulting specification is called \textit{workflow model};
	\item the \textit{process instantiation and execution phase}: the workflow model is introduced into a WfMS which then instantiates and orchestrates the execution of the underlying process.
\end{enumerate}
Since WfMS are pre-engineered standalone systems, WfM simplifies business processes' automation to their specifications in \textit{workflow languages}.

WfM primarily focuses on business processes' automation. It is not fundamentally concerned with other issues such as the analysis, the verification and the management (maintenance) of workflow (models) unlike BPM, which made these its foundation \cite{van2016don}. BPM is the discipline that combines knowledge from information technology and knowledge from management sciences and applies this to operational business processes \cite{van2013business}. BPM can be seen as an extension of WfM as it primarily supports WfM and provide additional tools to improve business processes. For this, we have chosen to use the expression BPM rather than WfM (which tends to disappear) in the context of this work. It should be noted however that our contributions (chapter \ref{chap2:structured-editing-artifact-type} and \ref{chap3:choreography-workflow-design-execution}) could be perfectly presented as part of restricted WfM (we are not interested in the management and improvement of workflow models).



\mySubSubSection{An Introductive Example of Business Process}{}
\label{chap1:sec:running-example}
BPM is an important technology because it simplifies the automation of business processes which are the foundation of how companies and organisations operate. Business processes can be found everywhere. The examples are diverse and include the following:
\begin{itemize}
\item The design and development of a software by a team (especially when members are geographically dispersed) \cite{theseImine};
\item The simultaneous writing of a scientific paper or the documentation of a product by several researchers (cooperative editing) \cite{theseImine};
\item The follow-up of a medical file \cite{workflow07};
\item The student registration process in a faculty;
\item The withdrawal of a large sum of money from a bank teller;
\item The procedure for taking holidays in a government institution;
\item The procedure for claiming damages from an insurance company;
\item The peer-review process \cite{peerReview02, van2001proclets, badouel14}.
\end{itemize}

The peer-review process presents all the characteristics of the type of processes (\textit{administrative processes}) studied in this manuscript. Then, we will use it as an illustrative example along the whole manuscript (running example). Our description of this process is inspired by those made in \cite{peerReview02, van2001proclets, badouel14}: 
\begin{example}
	\textbf{The peer-review process (running example)}:\\
	The process is triggered when the editor in chief receives a paper for validation submitted by one of the authors who participated in its drafting.
	\begin{itemize}
		\item After receipt, the editor in chief performs a pre-validation after which, he can accept or reject the submission for various reasons (subject of minor interest, submission not within the journal scope, non-compliant format, etc.);
		\item If the submission is rejected, he writes a report then notifies the corresponding author and the process ends; in the other case, he chooses an associated editor and sends him the paper for the continuation of the validation; 
		\item The associated editor prepares the manuscript, forms a referees committee (two members in our case) and then triggers the peer-review evaluation process;
		\item Each referee reads, seriously evaluates the paper and sends back a message and a report to the associated editor;
		\item After receiving reports from all the referees, the associated editor takes a decision and informs the editor in chief who sends the final decision to the corresponding author.
	\end{itemize}
\end{example}

From this description, it is easy to identify all the tasks to be executed, their sequencing, actors involved and the tasks assigned to them. For this case, four actors are involved: an editor in chief ($EC$) who is responsible for initiating the process, an associated editor ($AE$) and two referees ($R1$ and $R2$).
A summary of tasks assignment is presented in table \ref{tableau:tachesExecutant}. We have associated symbols with tasks so that we can easily manipulate them in diagrams. 
\begin{table}[ht]
	\caption{Exhaustive tasks list of a paper validation process in a scientific journal and their respective performers.}
	\label{tableau:tachesExecutant}
	\begin{tabular}[t]{|m{8.4cm}|m{2.7cm}|m{2.63cm}|}
		\hline
		\textbf{Tasks} & \textbf{Associated Symbols}  & \textbf{Executors} \\
		\hline
		Receipt, pre-validation of a submitted paper and possible choice of an associated editor to lead peer-review evaluation & $A$  & $EC$\\
		\hline
		Drafting of a pre-validation report informing on the reasons for the immediate rejection of the paper & $B$ & $EC$ \\
		\hline
		Sending the final decision (acceptance or rejection of the paper) to the author & $D$ & $EC$ \\
		\hline
		Study, eventually formatting of the paper for the examination by a committee & $C$ & $AE$ \\
		\hline
		Constitution of the reading committee (selection of referees) and triggering the peer-review evaluation & $E$ & $AE$ \\
		\hline
		Decision making (paper accepted or rejected) from referees evaluations & $F$ & $AE$ \\
		\hline
		Evaluation of the manuscript by the first (resp. second) referee & $G1$ (resp. $G2$) & $R1$ (resp. $R2$) \\
		\hline
		Drafting of the after evaluation report by the first (resp. second) referee & $H1$ (resp. $H2$) & $R1$ (resp. $R2$) \\
		\hline
		Writing the message according to evaluation by the first (resp. second) referee & $I1$ (resp. $I2$) & $R1$ (resp. $R2$) \\
		\hline
	\end{tabular}
\end{table}




\mySubSubSection{Workflow Typology}{}
\label{chap1:sec:workflow-typology}
The authors of \cite{workflow95} conduct a very interesting study on the classification of workflows in which, they report the lack of a commonly accepted approach to categorising workflows. There are therefore several approaches to workflow classification in the literature.

The classification of workflows according to the nature and behaviour of automated processes is one of the most commonly found in the literature. According to it, workflows are divided into three groups: \textit{production} workflows, \textit{administrative} workflows and \textit{ad-hoc} workflows \cite{mcCready, van1998application}. Production workflows are those that automate highly structured processes that undergo very little (or no) change over time: all the scenarios are known in advance and most of the tasks are carried out by systems. This is the case for processes in industrial production lines. 
Administrative workflows apply to variable processes for which all cases are known; that is, tasks are predictable and their sequencing rules are simple and clearly defined. In these, changes are more frequent than with production workflows and human actors are more involved in the execution of tasks. In particular, this type of workflow brings considerable added value to public administration organisations whose business is focused on administrative routines \cite{boukhedouma2015adaptation}. Our running example, the peer-review process, is an administrative process. In the work presented in this manuscript, we are interested in this type of workflows.
These are opposite of ad-hoc workflows, which automate occasional processes for which it is not always possible to define the set of rules in advance. Processes are therefore only partially specified and may undergo many updates over time.

The workflows' classification made in \cite{workflow95} is orthogonal to the above-mentioned one (they can be used together); it is more concerned with tasks' automation degree. The authors classify workflows based on a measurement system, represented by a continuum ranging from \textit{human-oriented} workflows to \textit{system-oriented} workflows as shown in figure \ref{chap1:fig:dimitrios-classification}.
\begin{figure}[ht!]
	\noindent
	\makebox[\textwidth]{\includegraphics[scale=0.94]{./Chap1/images/dimitrios-classification.png}}
	\caption{Classification of workflows according to whether they are human-oriented or system-oriented (source \cite{workflow95}).}
	\label{chap1:fig:dimitrios-classification}
\end{figure}
The first type (human-oriented workflows) includes workflows in which humans collaborate to perform tasks and to coordinate themselves; in these, humans are responsible for ensuring the validity and consistency of the exchanged data and of the workflow's results. The second type of workflows (system-oriented workflows) refers to those in which the use of computer systems to perform tasks is unavoidable because, they involve complex data and computationally-intensive operations. According to this classification system, human-oriented workflows are the ones we are interested in.

In \cite{dumas2005process}, the authors refine the two above classification frameworks. Concerning the refinement of the one that classifies workflows according to the nature and behaviour of automated processes \cite{mcCready}, their classification framework distinguishes \textit{unframed}, \textit{ad hoc framed}, \textit{loosely framed}, and \textit{tightly framed} workflows. 
A workflow is said to be unframed if there is no explicit workflow model associated with it; its execution is strongly conducted by its actors. When actors play a crucial role (no longer limited to the simple execution of tasks, but also including the explicit choice of the control flow, the adjustment of control and data flows, etc.) in the execution of a workflow, it is said to be \textit{user-centric} \cite{badouel2015active}. This is the case for workflows being automated by groupware\footnote{Groupware systems are computer-based systems that support groups of people engaged in a common task (or goal) and that provide an interface to a shared environment \cite{ellis1991groupware}.}. 
In the case of ad hoc framed workflows, workflow models are defined a priori but, they frequently change. 
A workflow is said to be loosely framed when it is defined by a workflow model describing the "right way of doing things", while allowing its actual executions to deviate from this way; this is the preferred type of workflow handled by \textit{Case Management Systems} \cite{van2013business} (see sec. \ref{chap1:sec:gag}).
Finally, a tightly framed process is one which consistently follows a defined process model.

Concerning the classification framework of \cite{workflow95}, authors of \cite{dumas2005process} refine it and consider three types of workflows: \textit{Person-to-Person}, \textit{Person-to-Application}, and \textit{Application-to-Application} workflows. 
Person-to-Person workflows are those for whom all the tasks require human intervention. Application-to-Application workflows are their opposite; in these, all the tasks are executed by software systems. Person-to-Application workflows are in the middle; they involve both human-oriented tasks and system-oriented tasks. Pratically, most of workflows are of this category.

Nowadays, some scientific works require increasingly complex and data-intensive simulations and analysis. Scientific data management is therefore a major challenge \cite{bell2009beyond} with a high level of complexity. Workflow technologies are increasingly used to manage this complexity \cite{juveGideon}. These are responsible for scheduling computational tasks on distributed resources, managing dependencies between tasks and staging data sets in and out of runtime sites. The resulting workflows are called \textit{scientific workflows} and are usually based on a middleware infrastructure (\textit{Grid} or \textit{Cloud}). Ideally, the scientist should be able to integrate almost any scientific data resource into such a workflow during analysis, inspect and visualise the data on-the-fly as it is computed, make parameter changes as needed and re-run only the affected components, and capture sufficient metadata in the final products so that, scientific workflow executions help to explain the results and make them reproducible. Thus, a scientific workflow system becomes a scientific problem-solving environment, adapted to an increasingly distributed and service-oriented infrastructure (Grid or Cloud) \cite{ludascher2006scientific}.

There are many other types of workflows in the literature. We can mention on the fly, \textit{service-oriented} workflows \cite{piccinelli2003service, yongyi2009research}, \textit{structured} workflows \cite{kiepuszewski2000structured, eder2002meta, liu2005analysis}, etc. We do not present them here because they are not of great interest to the work we are doing for this thesis. We invite the interested reader to take a look at the few works mentioned above.


\mySubSection{Business Process Management Lifecycle and Key Activities}{}
\label{chap1:sec:bpm-key-activities-concerns}
A high-level view of the BPM discipline reveals that, its lifecycle consists of three phases on which it is possible to iterate indefinitely: the \textit{(re)design}, \textit{implement/configure}, and \textit{run \& adjust} phases \cite{van2013business} (see fig. \ref{chap1:fig:bpm-lifecycle}). 
During its lifecycle, four key activities namely \textit{model}, \textit{enact}, \textit{analyse}, and \textit{manage} (see fig. \ref{chap1:fig:bpm-key-concerns}) are carried out \cite{van2013business}. 
In this section, we examine what is done during these different activities; we mainly focus on the \textit{"model"} and the \textit{"enact"} activities: they are the only ones common to BPM and WfM and thus, they are of relevant interest for the work presented in this manuscript. 

\mySubSubSection{Business Process Management Lifecycle}{}
\label{chap1:sec:bpm-lifecycle}
\begin{figure}[ht!]
	\noindent
	\makebox[\textwidth]{\includegraphics[scale=0.5]{./Chap1/images/bpm-lifecycle.png}}
	\caption{The three phases of BPM's lifecycle (source \cite{van2013business}).}
	\label{chap1:fig:bpm-lifecycle}
\end{figure}

The automation of a given process using BPM starts with its modelling using one or more workflow languages \cite{dumas2018fundamental}. This \textit{"model" activity} is initiated during the \textit{"(re)design" phase} of the BPM lifecycle. The workflow models obtained during this activity can be analysed (the \textit{"analyse" activity}) either by simulations or by using model checking\footnote{Model checking is an automated technique that, given a finite-state model of a system and a formal property, systematically checks whether this property holds for (a given state in) that model \cite{baier2008principles}.} algorithms (to verify models' soundness): this type of analysis is said to be \textit{model-based}. 
As shown in figure \ref{chap1:fig:bpm-lifecycle}, the (re)design phase is followed by the \textit{"implement/configure" phase} in which, the workflow models obtained in the previous phase are converted, if necessary, into executable workflow models and then, used to configure the process execution environment (the WfMS): this is where the \textit{"model" activity} ends. 
After the \textit{"implement/configure" phase}, comes the \textit{run \& adjust phase}. During this last phase, the workflow is instantiated, executed and managed (adjusted) according to the scenarios foreseen when modelling the underlying process and when designing the host WfMS: these are the purposes of the \textit{"enact"} and \textit{"manage"} activities. Moreover, when a workflow instance is running, produced and logged data can be analysed (to discover possible bottlenecks, waste, and deviations) for possible improvement of its corresponding workflow model: this other type of analysis/monitoring is said to be \textit{data-based}; during the last decade, \textit{process mining} \cite{van2011process} has emerged as one of the leading techniques conducting \textit{data-based analysis}. If enough possible improvements to the workflow model are detected, the cycle can restart to apply them.
\begin{figure}[ht!]
	\noindent
	\makebox[\textwidth]{\includegraphics[scale=0.54]{./Chap1/images/bpm-key-concerns.png}}
	\caption{The four key activities of BPM (source \cite{van2013business}).}
	\label{chap1:fig:bpm-key-concerns}
\end{figure}



\mySubSubSection{The "Model" Activity}{}
\label{chap1:sec:bpm-model-activity}
\noindent\textbf{\textit{Basic concepts}}

Process modelling is a crucial activity in WfM/BPM. As mentioned above (sec. \ref{chap1:sec:bpm-lifecycle}), it is done using dedicated languages called \textit{workflow languages}. Several workflow languages have already been developed. Among the most well-known are the BPMN standard \cite{BPMN} based on statecharts, the UML activity diagrams language \cite{booch2000guide}, the WF-Net (\textit{Workflow Net}) language \cite{wil2003business} which uses a formalism derived from that of Petri nets, the YAWL language \cite{van2005yawl} which is an extension of WF-Net and so forth. 
Some of these languages (BPM, UML activity diagrams) are \textit{informal} (i.e. they do not have a well-defined semantics and do not allow for analysis \cite{zur2013much, van2013business}) while others (WF-Net, YAWL) are based on powerful mathematical (\textit{formal}) tools (Petri nets). Nevertheless, they all allow to express in a diagram (called a \textit{worklow model}), the tasks that make up a given process and the control flow between them. More precisely, workflow languages allow to describe the behaviour of processes through the representation (among others) \cite{grigori2001elements} of :
\begin{itemize}
	\item Tasks that make up the main part of the process;
	\item Information and resources relating to the various tasks;
	\item Sequences or dependencies between those tasks;
	\item Trigger and termination events for the tasks.
\end{itemize}

Tasks are the base of any workflow; a \textit{task} is the smallest unit of hierarchical decomposition of a process. A task represents any work that is performed within a process. It consumes time, one or more resources, requires one or more input objects and produces one or more output objects. You can find examples of tasks in our running example (sec. \ref{chap1:sec:running-example}). 

From a workflow point of view, the term \textit{resource} refers to a system or a human who can execute a task. It is also known as \textit{actor}, \textit{participant}, \textit{stakeholder}, \textit{agent} or \textit{user} depending on the context. Resources can be grouped according to various characteristics, to form either a \textit{role} or an \textit{organisational unit} \cite{grigori2001elements}. A \textit{role} is a group of resources with the same functional capabilities, while an \textit{organisational unit} is a set of resources (or class of resources) that belong to the same structure (department, team, service, cell, etc.).

~

\noindent\textbf{\textit{Routing patterns}}

To achieve its objectives, any workflow language must, for a given process, allow to express at least its tasks and their \textit{routing} (\textit{control flow}). The task control flow is commonly referred to as the \textit{lifecycle (process) model} of the process under study \cite{divitini2001inter, hull2009facilitating}. There are a number of routing patterns identified in the literature as basic ones: these are \textit{sequential}, \textit{parallel}, \textit{alternative} or \textit{conditional} and \textit{iterative} routings (see fig. \ref{chap1:fig:basic-routing}) \cite{van1998application}.
\begin{figure}[ht!]
	\noindent
	\makebox[\textwidth]{\includegraphics[scale=4.5]{./Chap1/images/basic-routing.png}}
	\caption{Four basic routing constructs (source \cite{van1998application}).}
	\label{chap1:fig:basic-routing}
\end{figure}
\begin{itemize}
	\item Sequential routing expresses the fact that tasks must be executed one after the other (task $A$ before tasks $B$ and $C$ in figure \ref{chap1:fig:basic-routing}(a));
	\item Parallel routing is used to specify the potentially concurrent execution of certain tasks. Tasks $B$ and $C$ in figure \ref{chap1:fig:basic-routing}(b) can be executed at the same time; in this case, tasks $A$ and $D$ are considered as \textit{gateways}: $A$ is said to be an \textit{AND-Split} gateway while $D$ is an \textit{AND-Join} gateway;
	\item With alternative routing, one can model a \textit{decision}: i.e. the choice to execute one task rather than another at a given time. In figure \ref{chap1:fig:basic-routing}(c), tasks $B$ and $C$ cannot be both executed; for this case, $A$ is called and \textit{OR-Split} gateway and $D$ is an \textit{OR-Join} gateway;
	\item In some cases, it is necessary to execute a task multiple times.  In figure \ref{chap1:fig:basic-routing}(d) task $B$ is executed one or more times.
\end{itemize}

The search for more advanced and expressive routing patterns has been the subject of many studies \cite{van2012workflow, borger2012approaches}. The interested reader is invited to consult the few references mentioned above to find out more. 

When a given workflow language only allows to specify the routing of the processes' tasks, when it is not interested in modelling the consumed and produced data during tasks execution, and when it only gives a secondary role to the processes' users, it is said to be \textit{process-centric}. This is the case for all the previously mentioned languages (BPMN, WF-Net, UML activity diagrams and YAWL). This type of workflow language is often referred to as "\textit{traditional workflow language}".

~

\noindent\textbf{\textit{Examples of workflow models}}

Figure \ref{chap1:fig:comparing-workflow-languages} shows the orchestration diagrams corresponding to the graphical description of the peer-review process (see its textual description in sec. \ref{chap1:sec:running-example}) using the process-centric notations BPMN and WF-Net. The graphical notations equivalent to sequential flow, \{And, Or\}-Splits and \{And, Or\}-Joins are well represented. Each diagram resumes the \textit{main scenarios} of this process.
\begin{figure}[ht!]
	\noindent
	\makebox[\textwidth]{\includegraphics[scale=0.2]{./Chap1/images/comparingWorkflowLanguages-peerReview.png}}
	\caption{Orchestration diagrams of the peer-review process.}
	\label{chap1:fig:comparing-workflow-languages}
\end{figure}


\mySubSubSection{The "Enact" Activity}{}
\label{chap1:sec:bpm-enact-activity}
\noindent\textbf{\textit{Overview}}

The "enact" activity takes as input, the workflow models (specifications) obtained during the model activity. If these models are executable (i.e. they have been coded in more technical languages taking into account implementation details) then they are directly introduced into a WfMS suitably installed at the different workflow execution sites; otherwise, they are first converted into executable models then, they are introduced into the WfMS. There are several languages for producing executable workflow models. These are usually proprietary and provided by WfMS designers. Of these languages, \textit{(Web Services) Business Process Execution Language} ((WS-)BPEL) is the standard\footnote{BPEL is standardised by the OASIS consortium. OASIS website: \url{https://www.oasis-open.org/}. BPEL Specification (PDF version): \url{https://docs.oasis-open.org/wsbpel/2.0/OS/wsbpel-v2.0-OS.pdf}.} and is well compatible with BPMN \cite{white2005using, ouyang2006bpmn, leymann2010bpel}.

Once the WfMS is properly configured using workflow models, it can create workflow instances and properly orchestrate their execution. To do this, WfMS must coordinate (according to workflow models) the execution of a set of tools and applications offering various services. In the 1990s, the WfMC developed and proposed an architectural \textit{reference model} for the implementation of WfMS \cite{workflowModel} (see fig. \ref{chap1:fig:wfms-reference-model}). The latter structures and describes precisely, the expected functionalities of a WfMS.

~

\noindent\textbf{\textit{The reference model}}

The WfMC reference model is a centralised architectural model in which the main component is called \textit{workflow enactment service}. The workflow enactment service is responsible for controlling the executions of workflow instances. It is composed of several \textit{workflow engines}. A given workflow engine handles some parts of workflows and also manages some of their resources \cite{van2013business, dumas2018fundamental}. 
\begin{figure}[ht!]
	\noindent
	\makebox[\textwidth]{\includegraphics[scale=0.6]{./Chap1/images/wfms-reference-model.png}}
	\caption{Reference model of the Workflow Management Coalition (source \cite{workflowModel}).}
	\label{chap1:fig:wfms-reference-model}
\end{figure}
According to the reference model, WfMS must provide tools to facilitate their configurations using workflow models: therein, these tools are referred to as \textit{process definition tools}. Process definition tools are connected to the WfMS core (the workflow enactment service) via \textit{Interface 1}. In order to execute tasks, users use \textit{workflow client applications} that communicate with the WfMS via \textit{Interface 2}. When necessary, a given workflow engine invokes other applications via \textit{Interface 3}. The \textit{administration and monitoring tools} connected via \textit{Interface 5}, are used to monitor and control the workflows. Finally, the WfMS can be connected to other WfMS using \textit{Interface 4}. A considerable effort has been made to standardise the five interfaces shown in figure \ref{chap1:fig:wfms-reference-model}. These efforts led to the production of languages (exchange formats) such as \textit{Workflow Process Definition Language} (WPDL), \textit{XML\footnote{XML: eXtensible Markup Language.} Process Definition Language} (XPDL) and BPEL.

The reference model has been very successful. Firstly, because to this day, it perfectly orchestrates the different tools used for the design and execution of workflows. Secondly, because it has served as the basic model for a very large number of WfMS in the industry. Examples include ActionWorkflow \cite{actionWorkflow}, FlowMark \cite{flowmark}, Staffware \cite{staffware}, InConcert \cite{inConcert}, etc. Because the reference model is a centralised approach (client-server architecture), it has the advantage of facilitating a good mastery of the technologies used in the production of WfMS. Also, the implementation of (generally lightweight) client applications and the overall maintenance of WfMS (which is limited to the maintenance of the central server) are much simpler \cite{theseKanzow}. However, systems based on a client-server architecture show some limitations because of the centralisation of workflow management. Their main weaknesses are: \textit{the (non) fault tolerance}, \textit{the (difficult) scalability} and the strong dependency of the system vis a vis the central server, which stores data, controls and thus, represents \textit{a possible point of congestion} \cite{junYan06, fakas04}. Concretely \cite{junYan06},
\begin{enumerate}
	\item The client-server architecture allows centralised coordination of workflows with little use of the computing potential on the client side. Workflow systems based on such an architecture are very cumbersome. In application areas where several workflow instances need to be executed in parallel, the centralised server can be overloaded with heavy computations and intensive communications when the system load increases, thus becoming a potential bottleneck. 
	\item Client-server systems are vulnerable to server failures. The centralised server is commonly viewed as a single point of congestion in the system. Its malfunction can cause the entire system to shut down. 
	\item The limited scalability of the client-server architecture prevents the WfMS based on it, from dealing with the ever-changing work environment. This also raises difficulties in system configuration, as any changes to the system, such as the admission of new actors, may require changes and updates to the centralised workflow server, which is very impractical and inefficient. Therefore, these WfMS are particularly unsuitable for application areas where workflow actors are required to join and leave frequently.
	\item An important and crucial element of any workflow system is to allow actors to maintain their autonomy and control. However, workflow actors in a client-server-based WfMS are exclusively controlled by centralised servers. A serious problem is that, a large number of actors working on the "lightweight client side" may not be able to exercise their control, decision-making and problem-solving capabilities.
\end{enumerate}

Knowing that various actors involved in a given business process are very often spread over remote sites, the reference model does not seem to be very suitable for efficiently implementing cooperation among them, as would systems based on a distributed architectural model be.
In order to meet the shortcomings of the reference model, several works \cite{theseKanzow, junYan06, fakas04, theseImine, SON} have focused on the production of distributed WfMS built on top of peer-to-peer (P2P) architectures. This approach has also been successful since, systems such as ADEPT \cite{adept} and METEOR$_{2}$ \cite{meteor} have been designed over years \cite{theseKanzow}.












\mySection{Peer to Peer Business Process Management}{}
\label{chap1:sec:p2p-bpm}
Knowing that workflows are naturally distributed, they can sometimes involve resources from different organisations. Within each organisation, WfMS must therefore be built with a strong emphasis on (sometimes inter-organisational) cooperation; this differs from the idea in which classical information systems have often been built. However, even if WfMS must be \textit{interoperable} to facilitate cooperation, they must also ensure the \textit{autonomy} and the \textit{confidentiality} of actors and organisations involved in workflow execution; because, though organisations are aware of the need and necessity to participate in cooperation, they wish to protect their expertise in order to ensure sufficient confidentiality on their local data and local processes \cite{boukhedouma2015adaptation}. 
The main challenge for WfMS designers over the last two decades, has therefore been to build WfMS capable of both ensuring the agility of organisations and fostering the interconnection of business processes, while preserving their autonomy and the confidentiality of their local processes and data.

The production of fully distributed WfMS proved to be an effective solution to this challenge \cite{meilin1998workflow}. This has been made more feasible with the advent of new concepts such as the \textit{Multiagent} paradigm and the \textit{Service-Oriented Architecture} (SOA). In this section, we take a look at how the distributed workflow management approach works, and some of the decentralised WfMS that have been developed for this purpose


\mySubSection{The Advent of the Multiagent and Service-Oriented Concepts}{}
\label{chap1:sec:agent-soa-soc-concepts}

\mySubSubSection{The Multiagent Concept}{}
\label{chap1:sec:agent-concept}
The \textit{agent} and \textit{multiagents systems} concepts emerged in the 1980s. These concepts have generated lots of excitement in different research communities mainly because, they form the basis of a new paradigm for designing and implementing software systems that operates in distributed and open environments\footnote{\textit{"An open system is one in which the structure of the system itself is capable of dynamically changing. The characteristics of such a system are that its components are not known in advance; can change over time; and can consist of highly heterogeneous agents implemented by different people, at different times, with different software tools and techniques. Perhaps the best-known example of a highly open software environment is the internet. The internet can be viewed as a large, distributed information resource, with nodes on the network designed and implemented by different organisations and individuals."} (Katia P. Sycara, \citeyearpar{sycara1998multiagent})}, such as the internet \cite{sycara1998multiagent}. One of the best-known and most famous definitions of the agent concept was formulated by Jacques Ferber \citeyearpar{ferber1997systemes} and states that : \textit{an agent is a physical or logical entity capable of acting upon itself and its environment, which has a partial representation of that environment, which, in a multiagent system, can communicate with other agents, and whose behaviour is the consequence of its observations, knowledge and interactions with other agents}.

Multiagent systems have brought a new way to look at distributed systems and have provided a path to more robust intelligent applications \cite{deloach2001multiagent}. The challenge of the multiagent concept is to build distributed systems in which the nodes (agents), endowed with great autonomy, high reactivity and communicating using an asynchronous messaging system (they therefore possess cooperation and deliberation/decision capabilities \cite{theseKanzow}), can appear and disappear at any time without paralysing the system. A multiagent system is characterised as follows \cite{sycara1998multiagent}:
\begin{enumerate}
	\item Each agent has incomplete information or capabilities for solving problems and thus, has a limited viewpoint;
	\item There is no system global control;
	\item Data are decentralised;
	\item Computation is asynchronous.
\end{enumerate}
Such properties for a multiagent system provide it with several capabilities that have mainly attracted researchers and professionals to the multiagent paradigm. Among these capabilities we can distinguish the following \cite{sycara1998multiagent}: 
\begin{itemize}
	\item The capability to solve problems that are too large and difficult to handle by a centralised agent/server, because of resource limitations, or the sheer risk of having one centralised entity that could be a performance bottleneck or could fail at critical times;
	\item The capability to allow for the interconnection and interoperation of multiple existing legacy systems;
	\item The capability to provide solutions to problems that can naturally be regarded as a society of autonomous interacting components-agents;
	\item The capability to provide solutions that efficiently use information sources that are spatially distributed;
	\item The capability to provide solutions in situations	where expertise is distributed.
\end{itemize}

The above mentioned capabilities of multiagent systems are in line with the desired capabilities of a distributed WfMS. This has motivated the growing use of concepts developed for multiagent systems, in the production of such WfMS. However, the vocabulary used by the designers of distributed WfMS is not always identical to the multiagent jargon, and it is sometimes necessary to abstract the proposed systems to exhibit the multiagent concepts involved in their design and implementation.



\mySubSubSection{The Service-Oriented Architecture}{}
\label{chap1:sec:soa-concept}
\noindent\textbf{\textit{Basic concepts}} 

\textit{Service-Oriented Architecture} (SOA) has spread rapidly as a result of its growing success, and has been widely accepted as a supporting architecture for information systems because of its pivotal concept of \textit{service} \cite{boukhedouma2015adaptation}. Service is the essential concept of SOA and can be defined as \textit{"self-describing, platform-agnostic computational elements that support rapid, low-cost composition of distributed applications. Services perform functions, which can be anything from simple requests to complicated business  processes. Services allow organisations to expose their core competencies programmatically over a network using standard languages and protocols, and be implemented via a self-describing interface based on open standards"} (Mike P. Papazoglou, \citeyearpar{papazoglou2003service}). For MacKenzie et al. \citeyearpar{mackenzie2006reference}, the term service combines the following related ideas :
\begin{itemize}
	\item The offer to perform work for another;
	\item The capability to perform work for another;
	\item The specification of the offered work.
\end{itemize}

From a purely technological point of view, a service is a software component represented by two separate elements: its interface, which allows defining the access modalities to the service (name of the service and the parameters of the public operations defining the signatures of the operations) and its implementation \cite{boukhedouma2015adaptation}. Services are offered by \textit{service providers} (see fig. \ref{chap1:fig:basic-soa}); these are organisations that procure the service implementations, supply their service interfaces and provide related technical and business support \cite{papazoglou2003service}. Service interfaces are available for their searching, their binding, and their invocation by \textit{service consumers} (see fig. \ref{chap1:fig:basic-soa}) \cite{mackenzie2006reference}. Clients of services (service consumers) can be other solutions or applications within an enterprise or clients outside the enterprise. Service providers must therefore provide a distributed computing infrastructure for both intra and cross-enterprise application integration and collaboration. To satisfy these requirements, provided services should be \cite{papazoglou2003service}:
\begin{itemize}
	\item \textit{Technology neutral}: they must be able to be invoked by clients coded with various technologies having a few standards as a common denominator;
	\item \textit{Loosely coupled}: they must not require neither knowledge nor internal structures or conventions (context) at the service consumer or service provider side;
	\item \textit{Transparent from the point of view of their location}: one should be able to locate and invoke the services irrespective of their real location. To do so, the use of a \textit{service registry} where services interfaces and location information are stored, is required (see fig. \ref{chap1:fig:basic-soa}).
\end{itemize}

SOA is an architectural style, a logical way of designing a software system to provide services to either end-user applications or other services distributed in a network through published and discoverable interfaces. Basically, SOA defines an interaction between software agents as an exchange of messages between service consumers (clients) and service providers (see fig. \ref{chap1:fig:basic-soa}).
\begin{figure}[ht!]
	\noindent
	\makebox[\textwidth]{\includegraphics[scale=0.6]{./Chap1/images/basic-soa.png}}
	\caption{The basic Service-Oriented Architecture (source \cite{papazoglou2003service}).}
	\label{chap1:fig:basic-soa}
\end{figure}
In SOA, the exchange of messages between agents can be \textit{synchronous} or \textit{asynchronous}.

In the synchronous model, the service consumer invokes a service and expects a result. The invoked service is then designed to immediately return a result and is the only service involved. This model operates similarly to remote procedure call technologies such as Remote Method Invocation (RMI) but with a much loosely coupling between the service consumer and its provider.

In the asynchronous model (which is generally a particular form of \textit{publish/subscribe}\footnote{Publish/Subscribe is a communication paradigm well adapted to the loosely coupled nature of distributed interaction in large-scale applications; with systems based on its interaction scheme, subscribers register their interest in an event, or a pattern of events, and are subsequently asynchronously notified of events generated by publishers \cite{eugster2003many}.}), a given service consumer $A$ expresses its desire to be aware of the execution state of a given service $b$, published (provided) by a service provider $B$, by subscribing to it. During this subscription, it provides information about one or more services $(a_i)$ that it also provides and that must be invoked when the execution state of $b$ has changed. Several services are thus involved, and each agent is generally both a service provider and a service consumer.

SOA has been designed to facilitate the implementation of distributed applications based on Peer-to-Peer architectures (nodes/agents communicate directly without going through a central server) and in which, the skills of each agent are exposed, discoverable and invocable by the others but, the technique and technology used by each agent is confidential. This setting completes the concept of agents to answer correctly to the challenges of distributed WfMS: hence the very increasing use of the concept of service in workflow systems. Actually, some currents of thought claim that SOA was designed to facilitate the automation of business processes and thus, the design of distributed WfMS. This is the case of Hurwitz et al. \citeyearpar{hurwitz2009service} who define SOA as : \textit{"a software architecture for building applications that implement business processes or services by using a set of loosely coupled black-box components orchestrated to deliver a well-defined level of service"}.

~

\noindent\textbf{\textit{Shared-data Overlay Network}} 

As the use of SOA in P2P applications escalates, there is a proliferation of tools to facilitate the design and implementation of these new applications \cite{kaur2013design}. \textit{Shared-data Overlay Network} (SON) \cite{SON} is one of those tools. SON is a middleware offering several \textit{Domain Specific Languages} (DSL) to facilitate the implementation of P2P systems whose components communicate in an asynchronous manner by services invocations. SON combines the powerful concepts of \textit{Component-Based Software Engineering}\footnote{Component-Based Software Engineering is an emerging paradigm of software development whose goal is, composing applications with plug \& play software components on the frameworks; so, to realise software reuse by changing both software architecture and software process \cite{aoyama1998new}.}, \textit{Service-Oriented Computing} and \textit{P2P Computing} in its engineering. 

By using SON middleware, the P2P application designer (software developer) is able not only to specify applications in component-based service model, but also to perform an effective code generation. In fact (see fig. \ref{chap1:fig:son-model}), the software developer defines using a dedicated DSL called \textit{Component Description Meta Language} (CDML), for each component, a set of services (input, internal and output). Then, he only implements the code of the components, i.e., the methods that implement the defined services. Afterwards, a code generation tool, called \textit{Component Generator}, generates a set of Java source files that implement the so-called \textit{container of the component}. These Java files are compiled together with the implementation code to generate a standalone and ready-to-use component. 
Thus, software developers are assisted and have greater ease in developing component and service-based P2P applications. These facilities allow them to focus more on the business logic and to defer to SON, the management of the runtime requirements (e.g., communication mechanisms, instantiation and connection of components, service discovery, etc.).
\begin{figure}[ht!]
	\noindent
	\makebox[\textwidth]{\includegraphics[scale=0.7]{./Chap1/images/son-model.png}}
	\caption{Overview of a P2P application development process with SON (source \cite{SON}).}
	\label{chap1:fig:son-model}
\end{figure}
We use SON to implement prototypes of some of the models presented in this manuscript.


\mySubSection{Some Existing Distributed WfMS}{}
\label{chap1:sec:some-p2p-wfms}
In this section, we briefly present some existing approaches to distributed workflow management. As mentioned in \cite{theseKanzow}, these approaches have the following characteristics :
\begin{itemize}
	\item They are based on distributed entities that can communicate;
	\item These entities act autonomously, locally and thus, influence the further execution of the process (through their local actions, they choose the next actions to be executed);
	\item Each entity has a confidential local state;
	\item Each entity has only a partial view of the system's overall state at a given time;
	\item Workflow execution results from the automated interaction between the different entities.
\end{itemize}
The approaches to distributed workflow management presented here, can be divided into two categories :
\begin{enumerate}
	\item The first category contains those in which, data and controls are partially distributed and the WfMS is based on a client-server architecture;
	\item The second one is concerned with those in which WfMS, data and controls are fully distributed.
\end{enumerate}

\mySubSubSection{Some Partially Distributed WfMS}{}
\label{chap1:sec:partially-distributed-wfms}
\noindent\textbf{\textit{ADEPT (Advanced Decision Environment for Process Tasks) \cite{adept}}}

The ADEPT project was designed to automate flexible workflows at British Telecom\footnote{Nowadays British Telecom is renamed BT Group and remains the leader in the fixed telephony sector (source Wikipedia - \url{https://fr.wikipedia.org/wiki/BT_Group} - visited the 07/03/2020).}. Its main goal is to allocate resources to business processes using agents. According to its logic, workflow tasks are executed by agents acting as cooperating actors in a system supervised by one or more statically or dynamically assigned servers.
\begin{figure}[ht!]
	\noindent
	\makebox[\textwidth]{\includegraphics[scale=0.7]{./Chap1/images/adept.png}}
	\caption{An ADEPT environment (source \cite{adept}).}
	\label{chap1:fig:adept}
\end{figure}

Each agent is capable of providing one or more services. Services can be atomic (reduced to the execution of a single task) or composite (resulting from the combination of several other services, using operators that define the execution constraints - parallel, sequential, etc. -). If an agent needs the services of another agent, they must enter into an agreement called \textit{Service Level Agreement}\footnote{A Service Level Agreement is a formal contract used to guarantee that consumers' service quality expectation can be achieved \cite{wu2012service}.}. To facilitate the negotiation of agreements between agents, ADEPT provides a negotiation protocol and a service description language. Technically, the service description language allows agents to expose their services so that, they can be discovered by other agents which can then initiate negotiations for the use of those services, through the negotiation protocol.

Figure \ref{chap1:fig:adept} shows the architecture of ADEPT on an example of a workflow in which, four agents (marketing team, design team, sales team and legal department) collaborate to achieve business goals. Other publications on the ADEPT project may be useful for its understanding \cite{reichert1998adept, dadam2000clinical, reichert2003adept}.

~

\noindent\textbf{\textit{EvE (an Event-driven Distributed Workflow Execution Engine) \cite{eve}}}

According to the EvE approach, the distributed execution of workflows is done by event communication between agents (called \textit{brokers}) in charge of executing tasks. These agents perform tasks and create events in response to the occurrence of other events. EvE is based on a multi-server architecture in which, each server manages an entire cluster (a local network). However, this multi-server architecture is made transparent for the different agents thanks to \textit{adapters}; they can communicate independently of their respective domains. EvE provides amongst others the following services: 
\begin{itemize}
	\item Agents managed by servers and distributed across the network, capable of detecting events and executing tasks assigned to them; thereby, generating new events that are notified to other agents thanks to an inter-server communication mechanism that has been set up;
	\item A data warehouse in which information about agents, runtime data and \textit{Event - Condition - Action}\footnote{ECA is a paradigm that specifies the desired behaviour for reactive systems (i.e. systems that maintain ongoing interactions with their environments \cite{manna2012temporal}). In such a system centered around the ECA paradigm, when an event occurs, a condition is evaluated (by a querying mechanism) and the system takes corresponding action \cite{almeida2005modular}.} (ECA) event handling rules are stored. The information stored in the warehouse can be updated dynamically without the need to restart the system;
	\item Logging services for failure analysis and recovery. EvE supports exception notification, alerts and has the ability to resume execution after temporary disconnections;
\end{itemize}
\begin{figure}[ht!]
	\noindent
	\makebox[\textwidth]{\includegraphics[scale=0.6]{./Chap1/images/eve.png}}
	\caption{The workflow execution process in EvE (source \cite{eve}).}
	\label{chap1:fig:eve}
\end{figure}

The execution of a workflow starts as soon as an event is generated by a broker. The local EVE-server (its manager) then performs event detection and ECA-rule execution. Within the execution of each rule, task assignment determines responsible brokers, which are then notified and subsequently react as defined by their ECA-rules. Particularly, brokers can generate new events, which again are handled by EVE-servers, and so on transitively (see fig. \ref{chap1:fig:eve}).


\mySubSubSection{Some Fully Distributed WfMS}{}
\label{chap1:sec:fully-distributed-wfms}
\noindent\textbf{\textit{METEOR$_2$ (Managing End-To-End OpeRations 2) \cite{das1997orbwork, meteor}}}

The METEOR$_2$ project is a continuation of the METEOR \cite{krishnakumar1995managing} effort. It is intended to reliably support coordination of users and automated tasks in real-world multi-enterprise heterogeneous computing environments. Key capabilities of the METEOR$_2$ WfMS include a comprehensive toolkit for building workflows and supporting high-level process modelling, detailed workflow specification and automatic code generation for its workflow enactment systems. 
\begin{figure}[ht!]
	\noindent
	\makebox[\textwidth]{\includegraphics[scale=0.8]{./Chap1/images/meteor2.png}}
	\caption{The METEOR$_2$ architecture (source \cite{das1997orbwork}).}
	\label{chap1:fig:meteor2}
\end{figure}

METEOR$_2$ introduces concepts to represent each workflow as a set of tasks, task managers, processing entities and interfaces, in order to execute them in a completely distributed manner. Figure \ref{chap1:fig:meteor2} shows the various modules in METEOR$_2$ and their interaction. As can be seen in the picture, METEOR$_2$ includes a workflow designer that is used to create workflow models in a dedicated language. Once created, workflow models are stored in a workflow model repository. METEOR$_2$ also includes a workflow code generator that can read a stored workflow model and generate a convenient specific distributed workflow application. The generated application called the \textit{runtime system}, consists of a set of communicating agents called \textit{task managers} and their associated tasks, web-based user interfaces, a distributed recovery mechanism, a distributed scheduler and various monitoring components. All these workflow component are \textit{Common Object Request Broker Architecture}\footnote{CORBA is a standard middleware for distributed object systems. In its paradigm, a client application wishing to perform an operation on a server object, sends a request. The request is received by an Object Request Broker (ORB), responsible for all of the mechanisms required to find the object implementation for the request, to prepare the object implementation to receive the request, and to communicate the data making up the request to the server object. A server object accessible by CORBA is referred to as a CORBA object \cite{houlding2004system}.} (CORBA) objects and thus, they possess communication capabilities.

~

\noindent\textbf{\textit{The "Web Workflow Peer" Approach \cite{fakas04}}}

The approach proposed by Fakas and Karakostas is based on the concepts of \textit{Web Workflow Peer Directory} (WWPD) and \textit{Web Workflow Peer} (WWP). WWPD is an active directory system that maintains a list of all peers (WWP) that are available to participate in web workflow processes. It allows peers to register with the system and offer their services and resources to other peers. During the execution of workflow processes, the WWPD assists WWP to locate other WWP and use their services and resources. In their setting, key functionalities and data are distributed among WWP. The architecture is completely decentralised as no central workflow engine is employed to coordinate the process execution. The WWP encapsulates the necessary knowledge to perform the activities that are assigned to it and also to delegate some of the process execution to other WWP. The only centralised feature is the WWPD.
\begin{figure}[ht!]
	\noindent
	\makebox[\textwidth]{\includegraphics[scale=0.8]{./Chap1/images/wwp.png}}
	\caption{A P2P workflow architecture (source \cite{fakas04}).}
	\label{chap1:fig:wwp}
\end{figure}

A WWP is a processing unit with an interface that is exposed on the Web and which can be accessed using Internet protocols. Its interface describes different types of processing capabilities, each corresponding to a workflow activity. When combined, such activities form a workflow process. A WWP that initiates and administers the process is called the \textit{Administrator Peer}. Other WWP delegated to carry out workflow activities are called the \textit{Participating Peers} (see fig. \ref{chap1:fig:wwp}). In practice, all peers are capable of becoming administrators in different workflow process instances. WWP use mobile documents called \textit{Workflow Process Description} as communication medium. Segments of those documents move from site to site and conveys structural information about the running workflow instance.

Workflow process administration is achieved by employing a notification mechanism. For instance, at the completion of an activity the WWP notifies the Administrator Peer so that, an updated status of the process instance is maintained. Similarly, upon expiration of an activity deadline, the  Administrator Peer notifies the WWP responsible for the expired activity. As far as we know, there is still no real workflow system based on this promising architecture.


~

\noindent\textbf{\textit{SwinDeW (Swinburne Decentralised Workflow) \cite{junYan06}}}

Combining workflow and P2P concepts, SwinDeW \cite{junYan06} has been designed as a special P2P system, which provides workflow management support in a truly decentralised way. SwinDeW adopts a flat, flexible and loosely coupled structure with an intentional absence of both a centralised device for data storage, and a centralised control engine for coordination. SwinDeW offers several distributed protocols, especially for the definition, instantiation and execution of processes.
\begin{figure}[ht!]
	\noindent
	\makebox[\textwidth]{\includegraphics[scale=0.5]{./Chap1/images/swindew.png}}
	\caption{A high-level view of SwinDeW's architecture (source \cite{junYan06}).}
	\label{chap1:fig:swindew}
\end{figure}

The SwinDeW system is defined as four layers (see fig. \ref{chap1:fig:swindew}). The top layer is the application layer; it defines application-related functions to fulfil workflows. \textit{Workflow Participant Software} (WfPS) is an application that provides interfaces to interact with a workflow participant and other WfPS, requesting services and responding to requests. Core services of the workflow system are provided at the service layer, which include the peer management service, the process definition service, the process enactment service, and the monitoring and administration service. The data layer consists of distributed Data Repositories (DR) that store workflow-related information. Finally, the monitoring and administration service provides supervisory capabilities and status monitoring.

In SwinDeW, a peer is given by a WfPS and a set of DR (see fig. \ref{chap1:fig:swindew-wfps}). Each peer resides on a physical machine, enabling direct communication with other peers in order to carry out the workflow. A peer is a self-managing entity that is associated with and operates on behalf of a workflow participant. From the functional perspective, the WfPS of a peer consists of three software components :
\begin{enumerate}
	\item A user component which serves as  a "bridge" between the associated workflow participant and the workflow environment;
	\item A task component that is in charge of the execution of tasks conducted by the associated participant;
	\item A flow component which helps to fit an individual task into the workflow. It deals with data dependency and control dependency among tasks by handling incoming and outgoing messages.
\end{enumerate}
A peer (agent) consists also in a set of four DR : the peer repository, the resource and tool repository, the task repository, and the process repository.
\begin{enumerate}
	\item A given peer repository stores an organisational model that represents organisational entities and their relationships. This repository represents a user's view of the completely defined organisational model;
	\item A resource and tool repository stores part of the resource model, which represents non human resources such as machines, external hardware, tools, etc.
	\item A task repository stores a set of active task instances, which represent the work allocated to the associated workflow participant in the context of process instances;
	\item A process repository stores a partial process definition distributed to the considered peer.
\end{enumerate}
\begin{figure}[ht!]
	\noindent
	\makebox[\textwidth]{\includegraphics[scale=0.5]{./Chap1/images/swindew-wfps.png}}
	\caption{Structure of a peer in SwinDeW (source \cite{junYan06}).}
	\label{chap1:fig:swindew-wfps}
\end{figure}

Workflow processes in SwinDeW are defined by a definition peer, which is associated with an authorised participant such as a process engineer. The resulting workflow models are stored in a distributed manner, in the process repositories of various peers. To avoid the distribution of too large workflow models, SwinDeW uses a "\textit{know what you should know}" policy to partition these models and thus, to configure each peer only with the partitions of the models that are of interest to it.

In SwinDeW, a workflow instance is executed under the management of the workflow system. Once such an instance is created, a peer network is also constructed for carrying out this process instance. Various task instances are scheduled to enact at different sites, step by step. The execution of a task depends on the satisfaction of two conditions : the \textit{information condition}, which defines the start condition of a task from the data dependency
perspective (a task can be executed only after essential input data are available), and the \textit{control condition}, which indicates the start condition of a task from the control dependency perspective (a task can be executed only after some relevant work has been logically completed). Peers collaborate with one another through direct message exchange, to properly schedule the execution of various task instances. There are two kinds of messages flowing between peers: \textit{information messages} and \textit{control messages}, which are structured in XML format. When a peer receives messages from its predecessor peers directly, it evaluates the information and control conditions of the task instance independently, starts working when both the conditions are satisfied, and notifies its successor peers directly by delivering information messages and control messages after the task instance is completed. The successor peers repeat the same procedure until the completion of the whole process instance. This approach is then fully distributed; moreover, it has an implementation.


\mySection{Artifact-Centric Business Process Management}{}
\label{chap1:sec:data-aware-bpm}
%\subsection*{Contexte et définitions}
%\label{sec:contexte}
Emerged in the early 2000s, the \textit{artifact-centric} paradigm of BPM is one of those that has been much studied over the last two decades. This paradigm has been pioneered by IBM \cite{nigam2003business} and revisited in several works such as \cite{abi2016towards, deutsch2014automatic, hull2009facilitating, lohmann2010artifact, assaf2017continuous, assaf2018generating, boaz2013bizartifact, lohmann2011artifact, estanol2012artifact}; it proposes a new approach to BPM by focusing on both automated processes (tasks and their sequencing) and data manipulated through the concept of "\textit{business artifact}" (\textit{artifact-centric modelling}). In this section we present the key concepts of the artifact-centric paradigm as well as some artifact-centric frameworks from the literature.

\mySubSection{Artifact-Centric BPM Basic Concepts}{}
\label{chap1:sec:artifact-centric-bpm-key-concept}

\mySubSubSection{The Aim of Artifact-Centric BPM}{}
\label{chap1:sec:aim-artifact-centric-bpm}
In order to be able to better model workflows, process modelling should include a specification of the order in which tasks are executed (control flow), the way data are processed (data flow), and how different branches in distributed and inter-organisational business processes and services are invoked and coordinated (message flow) \cite{lohmann2011artifact}. These three conceptual models of workflows are also known as the \textit{process}, the \textit{informational} and the \textit{organisational} models \cite{divitini2001inter}. Traditional approaches (BPMN, YAWL, BPEL, etc.) to BPM are process-centric (they are also said to be \textit{imperative}): they generally offer two different views on business processes: 
\begin{enumerate}
	\item Collaboration diagrams (sometimes called interconnected models) that emphasise the local control flow of each participant of the process;
	\item Choreography diagrams (interaction models) that describe the process from the point of view of the messages that are exchanged among the participants.
\end{enumerate}
Traditional approaches thus express workflow models by means of diagrams which define how a workflow is supposed to operate, but give little importance (or none at all) to the information produced as a consequence of the process execution: data are treated as second-class citizens.

To precisely remedy this, researchers have developed the artifact-centric \cite{nigam2003business} approach to the design and execution of business processes. Artifact-centric models do not specify processes as a sequence of tasks to be executed or messages to be exchanged (i.e. imperatively), but from the point of view of the data objects (called \textit{business artifacts} or simply \textit{artifacts}) that are manipulated throughout the course of the process (i.e. declaratively) \cite{lohmann2011artifact}. They rely on the assumption that any business needs to record details of what it produces in terms of concrete information. Artifacts are proposed as a means to record this information. They model key business-relevant entities which are updated by a set of services (specified by pre and postconditions) that implement business process tasks. This approach has been successfully applied in practice and it provides a simple and robust structure for workflow modelling \cite{estanol2012artifact}.



\mySubSubSection{How the Artifact-Centric Approach to BPM Works}{}
\label{chap1:sec:artifact-centric-bpm-approach}
According to the artifact-centric paradigm, BPM takes place in two main phases guided by the concept of artifact. In order to automate a given process, the designer must first of all focus on artifact modelling; i.e. he must provide data structures capable of storing and logically conveying the information produced during the execution of workflows. Then, these artifacts models will be introduced into a WfMS supporting artifact-centric execution of workflows for the enactment.

~

\noindent\textbf{\textit{What is an artifact ?}}

\cite{nigam2003business} define an artifact as \textit{"a concrete, identifiable, self-describing chunk of information that can be used by a business person to actually run a business"}. Artifacts are business-relevant objects that are created, evolve, and (typically) archived as they pass through the workflow \cite{hull2009facilitating}; they represent key conceptual objects of workflow that evolve as they move through an enterprise. Artifacts are modelled through \textit{artifacts types} (or \textit{models}). An expected characteristic of artifacts is that they should be self-describing: this requirement allows a business person to be able to look at an artifact and determine if he or she can work on it. Toward this end, an artifact type should include both:
\begin{enumerate}
	\item an \textit{information model} (or "data schema"), for holding information about the artifact as it moves through the process, from creation to archival storage; and
	\item a \textit{lifecycle model} (or "lifecycle schema"), which describes how and when tasks (activities or services) might be invoked on the artifacts as they move through the process.
\end{enumerate}
A prototypical example of an artifact type is the "air courier package delivery", whose information model can hold data about a package including sender, receiver, steps occurring in transport and billing activity; and whose lifecycle model would specify the possible ways that the delivery service and billing might be carried out \cite{hull2013data, hull2009facilitating}.

Several approaches of modelling the lifecycle of artifacts have been studied in the literature. The most commonly used approach is that in which, some form of \textit{finite state machines} (automata) \cite{hull2009facilitating} are used to specify lifecycles. Other variants presenting the lifecycle of an artifact by a Petri net \cite{lohmann2010artifact}, logical formulae depicting legal successors of a state \cite{damaggio2012artifact} have also been proposed.


~

\noindent\textbf{\textit{The artifact-centric execution}}

Artifact-centric models can be executed by artifact-centric WfMS. This new type of WfMS put stress on how artifacts are created, updated and exchanged between various actors. In these, artifacts are considered as \textit{adaptive documents} that conveys all the information concerning a particular execution case of a given process, from its inception in the system to its termination. In particular, this information provides details on the case's execution status as well as on its lifecycle (a representation of the possible evolutions of this status). To do this, during the execution of a given process, the actions carried out by each of the actors (agents) have the effect of updating (\textit{editing}) the artifacts involved in that execution. If the process is cooperative, the artifact representing it will be updated by several agents: it is said to be cooperatively edited and thus, the execution of a given business process according to the artifact-centric approach, can be assimilated to the \textit{cooperative editing} of documents.

Two major trends in the artifact-centric modelling approach have been developed: \textit{orchestration} and \textit{choreography} \cite{hull2009facilitating}. 
\begin{enumerate}
	\item Orchestration suggests the creation of centralised systems (usually called \textit{artifact hubs}), coordinated by an \textit{orchestrator} whose role is to facilitate interaction between actors while ensuring that business goals are met.
	\item Choreography-oriented approaches get rid of the orchestrator, and model actors as autonomous agents coordinating with artifacts and communicating in a P2P manner, to accomplish business goals. In these, each agent focuses on achieving a local business goal and the achievement of the global business goal is the result of aggregating results from different local business goals.
\end{enumerate}
Compared to choreography, orchestration reduces the agents' autonomy by making the orchestrator the main controller of interactions. Also, the orchestrated approach does not scale well. A limitation of the choreography-oriented approach is the lack of a single synchronisation point from which, it is possible to know the process's (actual) global execution state. Despite this, we agree with \cite{lohmann2010artifact} that, this completely decentralised approach is the one that best fits the modelling of the intrinsically distributed nature of business processes.




\mySubSection{Some Existing Artifact-Centric BPM Frameworks}{}
\label{chap1:sec:existing-artifact-centric-bpm}
There are several frameworks in the literature, that implement artifact-centric concepts. We briefly present some of them in this section. We begin by presenting purely artifact-centric approaches; then we look at an even more flexible model, recently developed as a \textit{data-centric} solution for case management.


\mySubSubSection{Some Purely Artifact-Centric BPM Frameworks}{}
\label{chap1:sec:purely-artifact-centric-bpm}
\noindent\textbf{\textit{Proclets \cite{van2001proclets}}}

The concept of \textit{proclets} was introduced to specify business processes in which, objects' lifecycles can be modelled at different levels of granularity and cardinality. A proclet can be seen as a lightweight workflow process equipped with a knowledge base that contains information on previous interactions; it is thus equipped with an explicit lifecycle or active documents (i.e., documents aware of tasks and processes): in this setting then, a proclet is both an agent and an artifact. Proclets can find each other using a \textit{naming service}, and communicate with each other to exchange messages through \textit{channels} (see fig. \ref{chap1:fig:proclet}). 
\begin{figure}[ht!]
	\noindent
	\makebox[\textwidth]{\includegraphics[scale=0.7]{./Chap1/images/proclet.png}}
	\caption{Graphical representation of the proclet-based framework (source \cite{van2001proclets}).}
	\label{chap1:fig:proclet}
\end{figure}

In the proclet-based framework, the lifecycle of proclet instance is described by a \textit{proclet class} used as artifact type. Like an ordinary workflow model, a proclet class describes the order in which tasks can/need to be executed for individual instances of the class. Proclet classes are specified using a graphical language based on a sub-class of Petri nets so-called \textit{class of sound WF-nets}.

Proclets are well-suited to deal with settings in which several instances of data objects are involved. Proclets are considered to be distributed and autonomous enough to decide how to interact with the other proclets: thus, the proclet-based framework does not model proclets' locations. Moreover, the execution model of proclets is similar to choreography; the interoperation of proclets is not managed or facilitated by a centralised hub.


~

\noindent\textbf{\textit{Artifact hosting \cite{hull2009facilitating}}}

Hull et al. extend the artifact-centric model proposed by Nigam and Caswell \citeyearpar{nigam2003business}, to provide an interoperation framework in which data (artifacts) are hosted on central infrastructures named \textit{artifact-centric hubs}. Data hosted in artifact-centric hubs can be read and written by agents. This model is between choreography and orchestration because, agents are all connected to the hub but are not coordinated by a particular orchestrator. Unlike traditional orchestration schemes, the hub enables the participating agent to be pro-active, and serves primarily as a shared resource for coordinating activities. Participating agents can access information about the running artifact instances, can progress those instances along their lifecycles, and can subscribe to events in order to be alerted about significant steps in the progress of artifacts through their lifecycles. Security mechanisms are proposed for controlling access to data hosted in the hub.
\begin{figure}[ht!]
	\noindent
	\makebox[\textwidth]{\includegraphics[scale=0.5]{./Chap1/images/artifact-hub.png}}
	\caption{An example of interoperation using an artifact-centric hub (source \cite{hull2009facilitating}).}
	\label{chap1:fig:artifact-hub}
\end{figure}

Figure \ref{chap1:fig:artifact-hub} illustrates an example of six groups of agents (potentially organisations) coordinating using artifacts that are managed in a centralised hub. These are the agents related to the resources (\textit{Candidates}, the \textit{Human Resources Organisation}, the \textit{Hiring Organisations}, the \textit{Evaluators}, the \textit{Travel Provider} and the \textit{Reimbursement}) that carry out the employee hiring process in a given enterprise.


~

\noindent\textbf{\textit{Artifact-centric choreographies \cite{lohmann2010artifact}}}

Lohmann and Wolf \citeyearpar{lohmann2010artifact} provide a more choreography-like framework for artifact-centric interoperation. They abandon the fact of having a single artifact hub \cite{hull2009facilitating} and they introduce the idea of having several agents which operate on artifacts. Some of those artifacts are \textit{mobile} (their location may change over time); thus, the authors provide a systematic approach for modelling artifact location and its impact on the accessibility of actions using a Petri net. Their model was designed with the conviction that by making explicit who is accessing an artifact and where the artifact is located, one will be able to automatically generate an interaction model that can serve as a contract between agents, and which make sure that global goal states specified on artifacts are reached. They thus propose an approach to automatically derive such an interaction model.



~

\noindent\textbf{\textit{Declarative choreographies \cite{sun2012declarative}}}

In \cite{sun2012declarative}, the authors are also interested in choreographies. More precisely, they develop a language allowing to model (in a declarative manner) the collaboration between several actors (the choreography) and a distributed algorithm allowing the execution of the choreographies specified in their language. Their choreography language has four distinct features :
\begin{enumerate}
	\item Each type of actor is an artifact schema with a selected sub-part of its information model visible to choreography specification.
	\item Correlations between actor types and instances are explicitly specified, along with cardinality constraints on correlated instances (e.g. each Order instance may correlate with exactly one Payment instance and multiple Vendor instances).
	\item Messages can include data; data in both messages and artifacts can be used in choreography constraints.
	\item The language is declarative and uses logic rules based on a mix of first-order logic and a set of binary temporal operators from DecSerFlow\footnote{DecSerFlow: \textit{a Declarative Service Flow Language} is a graphical, extendible language for expressing process models in a declarative way; it captures what is the high-level process behaviour without expressing how it is procedurally executed, hence giving a concise and easily interpretable feedback to the business manager \cite{lamma2007learning}.} \cite{van2006decserflow}.
\end{enumerate}
In particular, Skolem\footnote{Thoralf Albert Skolem (1887-1963) is a Norwegian mathematician and logician. He is particularly known for his work in mathematical logic and set theory which now bears his name, such as the L\"{o}wenheim-Skolem theorem or the notion of skolemisation (source, Wikipedia: \url{https://en.wikipedia.org/wiki/Thoralf_Skolem}, visited the 02/04/2020).} notations are used to both reference correlated actor instances and to manipulate dependencies among messages. 





\mySubSubSection{A Guarded Attribute Grammars Based Framework to Data-Centric Case Management}{}
\label{chap1:sec:gag}
\noindent\textbf{\textit{What is case management ?}}

Highly important processes in organisations that have a tremendous impact on the success and add the most value, involve a high degree of knowledge work: they are driven by users' decisions (\textit{user-centric}) making it difficult to specify them into a set of activities with precedence relations at design-time (they are said to be \textit{knowledge-intensive}). Because knowledge-intensive processes are subject of frequent exceptions, traditional BPM solutions are not able to support them sufficiently \cite{hauder2014research}. \textit{Adaptive Case Management} (ACM) is gaining interest among researchers and practitioners as an  emerging paradigm to master situations in which adaptions have to be made at run-time (unpredictable situations) by so called knowledge workers. In contrast to traditional BPM, the ACM paradigm is not dictating knowledge workers a predefined course of action, but provides them with the required information at the right time (they are \textit{data-centric}) and authorises them to make decisions on their own \cite{hauder2014research}. 

The notion of \textit{case} in the ACM context, is closely related to the concept of artifact. Both involve the notion of a conceptual entity that progresses through time, according to some set of guidelines or lifecycle schema, and both taking advantage of a growing set of data accumulated over the case instance lifecycle \cite{hull2013data}. ACM can be seen as an extension of the artifact-centric paradigm in which, the flexibility of workflow models (types of artifacts) is highly valued; therefore, both users and data are treated as first-class citizens.

There is a growing research interest in the ACM paradigm and several models have already emerged. Guard-Stage-Milestone (GSM) \cite{hull2011business, damaggio2013equivalence}, a declarative model of the lifecycle of artifacts, was recently introduced and has been adopted as a basis of \textit{Case Management Model and Notation} (CMMN), the OMG standard for ACM. The GSM model defines Guards, Stages and Milestones to control the enabling, enactment and completion of (possibly hierarchical) activities; it then allows for dynamic creation of subtasks (the stages), and handles data attributes. However, interaction with users are modelled as incoming messages from the environment, or as events from low-level (atomic) stages. In this way, users do not explicitly contribute to the choice of a flow for a process.

Recently, Badouel et al. \citeyearpar{badouel14, badouel2015active} have proposed a more user-centric and data-driven ACM model (AWGAG) based on the concepts of Active Workspaces (AW) and Guarded Attribute Grammars (GAG). We are particularly interested in this model because, it incorporates concepts that we will manipulate throughout this manuscript. These include the concepts of \textit{grammars} as artifact types, \textit{structured documents} (trees) as artifacts, \textit{artifact editing}, etc.

~

\Needspace{5\baselineskip}
\noindent\textbf{\textit{AWGAG \cite{badouel2015active}}}

The AWGAG model of collaborative systems is centered on the notion of user's workspace. It assume that the workspace of a user is given by a map. It is a tree used to visualise and organise tasks in which, the user is involved together with information used for their resolution. The workspace of a given user may, in fact, consist of several maps where each map is associated with a particular service offered by the user. In short, one can assume that a user offers a unique service so that any workspace can be identified with its graphical representation as a map.
\begin{figure}[ht!]
	\noindent
	\makebox[\textwidth]{\includegraphics[scale=0.7]{./Chap1/images/aw-clinician.png}}
	\caption{Active workspace of a clinician (source \cite{badouel2015active}).}
	\label{chap1:fig:aw-clinician}
\end{figure}

As an example, figure \ref{chap1:fig:aw-clinician} shows a map that represents the workspace of a clinician acting in the context of a disease surveillance system. The service provided by the clinician is identifying the symptoms of influenza in a patient, clinically examining the patient, eventually placing him under therapeutic care, declaring the suspect cases to the disease surveillance center, and monitoring the patient based on subsequent requests from the epidemiologist or the biologist.

Each call to this service, namely when a new patient comes to the clinician, creates a new tree rooted at the central node of the map. This tree is an artifact that represents a structured document for recording information about the patient all along being taken over in the system. Initially, the artifact is reduced to a single (open) node that bears information about the name, age and sex of the patient. An open node, graphically identified by a question mark, represents a pending task that requires the clinician's attention. In this example the initial task of a given artifact is to clinically examine the patient. This task is refined into three subtasks: clinical assessment, initial care, and case declaration.

In the AWGAG model, a task is interpreted as a problem to be solved, that can be completed by refining it into sub-tasks using business rules. A business rule is modelled by a production $P: s_0 \rightarrow s_1 \ldots s_n$ expressing that task $s_0$ can be reduced to subtasks $s_1$ to $s_n$. For instance, the production 
\begin{itemize}
	\item[] $patient \rightarrow clinicalAssessment \ initialCare \ caseDeclaration$
\end{itemize}
states that, a task of sort $patient$, the axiom of the grammar associated with the service provided by the clinician, can be refined by three subtasks whose sorts are respectively $clinicalAssessment$, $initialCare$, and $caseDeclaration$. If several productions with the same left-hand side $s_0$ exist, then the choice of a particular production corresponds to a decision made by the user. In the example, the clinician has to decide whether the case under investigation has to be declared to the disease surveillance center or not. This decision can be reflected by the following two productions:
\begin{itemize}
	\item[] $suspectCase: caseDeclaration \rightarrow followUp$
	\item[] $benignCase: caseDeclaration \rightarrow$
\end{itemize}
If the case is reported as suspect, then the clinician will have to follow up the case according to further requests of the biologist or the epidemiologist. On the contrary (i.e. the clinician has described the case as benign), the case is closed with no follow up actions.

AWGAG  model considers artifacts as trees whose nodes are sorted and whose productions are taken into a grammar (GAG). The lifecycle of an artifact is implicitly given by the set of productions of the underlying GAG:
\begin{enumerate}
	\item The artifact initially associated with a case, is reduced to a single open node.
	\item An open node $X$ of sort $s$ can be refined by choosing a production $P: s \rightarrow s_1 \ldots s_n$ that fits its sort. The open node $X$ becomes a closed node under the decision of applying production $P$ to it. In doing so, task $s$ associated with $X$ is replaced by $n$ subtasks $s_1$ to $s_n$, and new open nodes $X_1$ of sort $s_1$ to $X_n$ of sort $s_n$, are created accordingly: the artifact is said to be edited (see fig. \ref{chap1:fig:aw-artifact-edition}).
	\item The case reaches completion when its associated artifact is closed, i.e. it no longer contains open nodes.
\end{enumerate}
\begin{figure}[ht!]
	\noindent
	\makebox[\textwidth]{\includegraphics[scale=0.5]{./Chap1/images/aw-artifact-edition.png}}
	\caption{Artifact edition in AWGAG (source \cite{badouel2015active}).}
	\label{chap1:fig:aw-artifact-edition}
\end{figure}

Additional information are attached to open nodes using \textit{attributes}, to model the interactions and data exchanged between the various tasks associated with them. For that, each sort comes equipped with a set of \textit{inherited} attributes and a set of \textit{synthesised} attributes where: inherited attributes represents input data, i.e. necessary data for the associated task to be executed, while synthesised attributes represents output data, i.e. data that are produced after the task being executed. This formalism puts emphasis on a declarative (logical) decomposition of tasks to avoid over-constrained schedules. Indeed, business rules do not prescribe any ordering on task executions. Ordering of tasks depend on the exchanged data and are therefore determined at runtime. In this way, the AWGAG model allows as much concurrency as possible in the execution of the current pending tasks.

Furthermore, a given AWGAG model is flexible and can incrementally be designed: one can initially let the designer manually develop large parts of the map, and progressively improve the automation of the process by refining the classification of the nodes, and introducing new business rules when recurrent patterns of activities are detected. The AWGAG model presents great properties such as \textit{distribution} and \textit{soundness}; these properties and more details are discussed in \cite{badouel2015active}. Several implementations and extensions of the AWGAG model are currently being carried out.





% \vspace{-0.5em}
\section{Conclusion}
% \vspace{-0.5em}
Recent advances in multimodal single-cell technology have enabled the simultaneous profiling of the transcriptome alongside other cellular modalities, leading to an increase in the availability of multimodal single-cell data. In this paper, we present \method{}, a multimodal transformer model for single-cell surface protein abundance from gene expression measurements. We combined the data with prior biological interaction knowledge from the STRING database into a richly connected heterogeneous graph and leveraged the transformer architectures to learn an accurate mapping between gene expression and surface protein abundance. Remarkably, \method{} achieves superior and more stable performance than other baselines on both 2021 and 2022 NeurIPS single-cell datasets.

\noindent\textbf{Future Work.}
% Our work is an extension of the model we implemented in the NeurIPS 2022 competition. 
Our framework of multimodal transformers with the cross-modality heterogeneous graph goes far beyond the specific downstream task of modality prediction, and there are lots of potentials to be further explored. Our graph contains three types of nodes. While the cell embeddings are used for predictions, the remaining protein embeddings and gene embeddings may be further interpreted for other tasks. The similarities between proteins may show data-specific protein-protein relationships, while the attention matrix of the gene transformer may help to identify marker genes of each cell type. Additionally, we may achieve gene interaction prediction using the attention mechanism.
% under adequate regulations. 
% We expect \method{} to be capable of much more than just modality prediction. Note that currently, we fuse information from different transformers with message-passing GNNs. 
To extend more on transformers, a potential next step is implementing cross-attention cross-modalities. Ideally, all three types of nodes, namely genes, proteins, and cells, would be jointly modeled using a large transformer that includes specific regulations for each modality. 

% insight of protein and gene embedding (diff task)

% all in one transformer

% \noindent\textbf{Limitations and future work}
% Despite the noticeable performance improvement by utilizing transformers with the cross-modality heterogeneous graph, there are still bottlenecks in the current settings. To begin with, we noticed that the performance variations of all methods are consistently higher in the ``CITE'' dataset compared to the ``GEX2ADT'' dataset. We hypothesized that the increased variability in ``CITE'' was due to both less number of training samples (43k vs. 66k cells) and a significantly more number of testing samples used (28k vs. 1k cells). One straightforward solution to alleviate the high variation issue is to include more training samples, which is not always possible given the training data availability. Nevertheless, publicly available single-cell datasets have been accumulated over the past decades and are still being collected on an ever-increasing scale. Taking advantage of these large-scale atlases is the key to a more stable and well-performing model, as some of the intra-cell variations could be common across different datasets. For example, reference-based methods are commonly used to identify the cell identity of a single cell, or cell-type compositions of a mixture of cells. (other examples for pretrained, e.g., scbert)


%\noindent\textbf{Future work.}
% Our work is an extension of the model we implemented in the NeurIPS 2022 competition. Now our framework of multimodal transformers with the cross-modality heterogeneous graph goes far beyond the specific downstream task of modality prediction, and there are lots of potentials to be further explored. Our graph contains three types of nodes. while the cell embeddings are used for predictions, the remaining protein embeddings and gene embeddings may be further interpreted for other tasks. The similarities between proteins may show data-specific protein-protein relationships, while the attention matrix of the gene transformer may help to identify marker genes of each cell type. Additionally, we may achieve gene interaction prediction using the attention mechanism under adequate regulations. We expect \method{} to be capable of much more than just modality prediction. Note that currently, we fuse information from different transformers with message-passing GNNs. To extend more on transformers, a potential next step is implementing cross-attention cross-modalities. Ideally, all three types of nodes, namely genes, proteins, and cells, would be jointly modeled using a large transformer that includes specific regulations for each modality. The self-attention within each modality would reconstruct the prior interaction network, while the cross-attention between modalities would be supervised by the data observations. Then, The attention matrix will provide insights into all the internal interactions and cross-relationships. With the linearized transformer, this idea would be both practical and versatile.

% \begin{acks}
% This research is supported by the National Science Foundation (NSF) and Johnson \& Johnson.
% \end{acks}






	\mathversion{normal2}
	\myChapter{A Workflow for Structured Documents' Cooperative Editing : Key Principles and Algorithms}{}
\label{chap2:structured-editing-artifact-type}
\myMiniToc{section}{Contents}
% If no minitoc then
% \startcontents[chapters]
% \leavevmode
% \\
% \\
% \\
% \\
% \\
\section{Introduction}
\label{introduction}

AutoML is the process by which machine learning models are built automatically for a new dataset. Given a dataset, AutoML systems perform a search over valid data transformations and learners, along with hyper-parameter optimization for each learner~\cite{VolcanoML}. Choosing the transformations and learners over which to search is our focus.
A significant number of systems mine from prior runs of pipelines over a set of datasets to choose transformers and learners that are effective with different types of datasets (e.g. \cite{NEURIPS2018_b59a51a3}, \cite{10.14778/3415478.3415542}, \cite{autosklearn}). Thus, they build a database by actually running different pipelines with a diverse set of datasets to estimate the accuracy of potential pipelines. Hence, they can be used to effectively reduce the search space. A new dataset, based on a set of features (meta-features) is then matched to this database to find the most plausible candidates for both learner selection and hyper-parameter tuning. This process of choosing starting points in the search space is called meta-learning for the cold start problem.  

Other meta-learning approaches include mining existing data science code and their associated datasets to learn from human expertise. The AL~\cite{al} system mined existing Kaggle notebooks using dynamic analysis, i.e., actually running the scripts, and showed that such a system has promise.  However, this meta-learning approach does not scale because it is onerous to execute a large number of pipeline scripts on datasets, preprocessing datasets is never trivial, and older scripts cease to run at all as software evolves. It is not surprising that AL therefore performed dynamic analysis on just nine datasets.

Our system, {\sysname}, provides a scalable meta-learning approach to leverage human expertise, using static analysis to mine pipelines from large repositories of scripts. Static analysis has the advantage of scaling to thousands or millions of scripts \cite{graph4code} easily, but lacks the performance data gathered by dynamic analysis. The {\sysname} meta-learning approach guides the learning process by a scalable dataset similarity search, based on dataset embeddings, to find the most similar datasets and the semantics of ML pipelines applied on them.  Many existing systems, such as Auto-Sklearn \cite{autosklearn} and AL \cite{al}, compute a set of meta-features for each dataset. We developed a deep neural network model to generate embeddings at the granularity of a dataset, e.g., a table or CSV file, to capture similarity at the level of an entire dataset rather than relying on a set of meta-features.
 
Because we use static analysis to capture the semantics of the meta-learning process, we have no mechanism to choose the \textbf{best} pipeline from many seen pipelines, unlike the dynamic execution case where one can rely on runtime to choose the best performing pipeline.  Observing that pipelines are basically workflow graphs, we use graph generator neural models to succinctly capture the statically-observed pipelines for a single dataset. In {\sysname}, we formulate learner selection as a graph generation problem to predict optimized pipelines based on pipelines seen in actual notebooks.

%. This formulation enables {\sysname} for effective pruning of the AutoML search space to predict optimized pipelines based on pipelines seen in actual notebooks.}
%We note that increasingly, state-of-the-art performance in AutoML systems is being generated by more complex pipelines such as Directed Acyclic Graphs (DAGs) \cite{piper} rather than the linear pipelines used in earlier systems.  
 
{\sysname} does learner and transformation selection, and hence is a component of an AutoML systems. To evaluate this component, we integrated it into two existing AutoML systems, FLAML \cite{flaml} and Auto-Sklearn \cite{autosklearn}.  
% We evaluate each system with and without {\sysname}.  
We chose FLAML because it does not yet have any meta-learning component for the cold start problem and instead allows user selection of learners and transformers. The authors of FLAML explicitly pointed to the fact that FLAML might benefit from a meta-learning component and pointed to it as a possibility for future work. For FLAML, if mining historical pipelines provides an advantage, we should improve its performance. We also picked Auto-Sklearn as it does have a learner selection component based on meta-features, as described earlier~\cite{autosklearn2}. For Auto-Sklearn, we should at least match performance if our static mining of pipelines can match their extensive database. For context, we also compared {\sysname} with the recent VolcanoML~\cite{VolcanoML}, which provides an efficient decomposition and execution strategy for the AutoML search space. In contrast, {\sysname} prunes the search space using our meta-learning model to perform hyperparameter optimization only for the most promising candidates. 

The contributions of this paper are the following:
\begin{itemize}
    \item Section ~\ref{sec:mining} defines a scalable meta-learning approach based on representation learning of mined ML pipeline semantics and datasets for over 100 datasets and ~11K Python scripts.  
    \newline
    \item Sections~\ref{sec:kgpipGen} formulates AutoML pipeline generation as a graph generation problem. {\sysname} predicts efficiently an optimized ML pipeline for an unseen dataset based on our meta-learning model.  To the best of our knowledge, {\sysname} is the first approach to formulate  AutoML pipeline generation in such a way.
    \newline
    \item Section~\ref{sec:eval} presents a comprehensive evaluation using a large collection of 121 datasets from major AutoML benchmarks and Kaggle. Our experimental results show that {\sysname} outperforms all existing AutoML systems and achieves state-of-the-art results on the majority of these datasets. {\sysname} significantly improves the performance of both FLAML and Auto-Sklearn in classification and regression tasks. We also outperformed AL in 75 out of 77 datasets and VolcanoML in 75  out of 121 datasets, including 44 datasets used only by VolcanoML~\cite{VolcanoML}.  On average, {\sysname} achieves scores that are statistically better than the means of all other systems. 
\end{itemize}


%This approach does not need to apply cleaning or transformation methods to handle different variances among datasets. Moreover, we do not need to deal with complex analysis, such as dynamic code analysis. Thus, our approach proved to be scalable, as discussed in Sections~\ref{sec:mining}.
\mySection{Basic Concepts on Cooperative Editing Workflows}{}
\label{chap2:sec:cooperative-editing-concepts}
Cooperative editing is a work of (hierarchically) organised groups, that operate according to a schedule involving delays and a division of labor (coordination). 
Like any CSCW, cooperative editing is subject to spatial and temporal constraints. Thus, one distinguishes distributed or not, and synchronous or asynchronous cooperative editing. When distributed, the various editing sites are geographically dispersed and each of them has a local copy of the document to be edited; systems that support such an edition should offer algorithms for data replication \cite{Yasushi2005} and for the fusion of updates. When asynchronous, various co-authors get involved at different times to bring their different contributions.

A cooperative editing workflow goes generally, from the creation of the document to edit, to the production of the final document through the alternation and repetition of distribution, editing and synchronisation phases. The literature is full of several cooperative editing workflows and of their management systems. We present a few in this section.


\mySubSection{Real-Time Cooperative Editing Workflows}{}
\label{chap2:sec:real-time-cooperative-editing}
In these generally centralised systems (Etherpad\footnote{Official website of Etherpad: \url{http://www.etherpad.org/}, visited the 04/04/2020.} \cite{epad}, Google Docs\footnote{Google Docs is accessible online at \url{https://www.docs.google.com/}, visited the 04/04/2020.}, Framapad\footnote{Get more information on Framapad at \url{http://www.framasoft.org/}, visited the 04/04/2020.}, Fidus Writer\footnote{Official website of Fidus Writer: \url{https://www.fiduswriter.org/}, visited the 04/04/2020.} \cite{fiduswriter}, etc.), the original document is created by a co-author on the central server. The latter then invites his colleagues to join him for the editing; they therefore connect to the editing session usually identified by a URL (distribution phase, although the document is generally not really duplicated). During an editing session (synchronous editing phase), all connected co-authors work on a single copy of the document but in different contexts. When the integration is automatic, changes performed by one of them are immediately (automatically) propagated to be incorporated into the basic document (synchronisation phase), and the latter is then redistributed to others. The changes are saved progressively and the server usually keeps multiple versions of the document.

The majority of real-time editors use the model of operational transformations \cite{theseOster, theseMounir}. Their architectures are therefore based on the one defined by this model. Meaning that, they distinguish two main components: an \textit{integration algorithm}, responsible for the receipt, dissemination and execution of operations and a \textit{set of processing functions} that are responsible for "merging" updates by serialising two concurrent operations.

\mySubSection{Asynchronous Cooperative Editing Workflows}{}
\label{chap2:sec:async-cooperative-editing}
This edit mode is distinguished by real distribution phases in which, the document to be edited is replicated on different sites, using appropriate algorithms \cite{Yasushi2005}. A co-author may then contribute at any time, by editing his local copy of the document. Here, we focus on a few asynchronous cooperative editors operating in client-server mode.

~

\noindent\textbf{\textit{Wikiwikiweb (Wikis)}}

Wikis \cite{wikiwikiweb} are a family of collaborative editors for editing web pages from a browser. To edit a page on a Wiki, one must duplicate it and contribute. After editing, he just have to save it and to publish a new version of that page. In a competing editing case, it is the last published version which will be visible. Even though it is still possible to access the previously published versions, there is no guarantee that a new version of the page preserves intentions (incorporates aspects) of previous versions. For this aspect, a Wiki can be seen much more as a web page version manager.

~

\noindent\textbf{\textit{CVS (Concurrent Versions System)}}

Under CVS \cite{cvs}, versions of a document are managed in a space called repository, and each user has a personal workspace.
To edit a document, the user must create a replica in his workspace. He will amend this replica, then will release a new version of the document in the repository. In case the document is concurrently edited by several authors and at least one update has already been published, the author wishing to publish a new update, will be forced to consult and integrate all previous updates through  dedicated tools, integrated in CVS.

~

\Needspace{5\baselineskip}
\noindent\textbf{\textit{SVN (Subversion)}}

SVN\footnote{Check more about SVN at \url{http://www.subversion.apache.org/}, visited the 04/04/2020.} \cite{svn} was created to replace CVS. Its main goal was to propose a better implementation of CVS. So as CVS, SVN relies on an optimistic protocol of concurrent access management: the \textit{copy-edit-merge} paradigm. SVN provides many technical changes like a new commit algorithm, the management of metadata versions, new user commands and many others features.

~

\noindent\textbf{\textit{Git}}

The main purpose of Git\footnote{Official website of Git: \url{https://www.git-scm.com/}, visited the 04/04/2020.} is the management of various files in a content tree considered as a deposit (all files of a source code for example). To edit a deposit, a given user connects to it and clones (forks). He obtains a copy of that deposit, modifies it locally through a set of commands provided by Git. Then he offers his contribution to primary maintainer which can validate it and thus, merges it with the original deposit. During this operation, new versions of modified files are created in the main repository. It is therefore possible under Git, to access any revision of a given file. 



\mySubSection{Badouel and Tchoup\'e's Cooperative Editing Workflow}{}
\label{chap2:sec:badouel-tchoupe-cooperative-editing}
Badouel and Tchoup\'e \citeyearpar{badouelTchoupeCmcs} proposed a workflow for cooperative editing of structured documents (those with regular structures defined by grammatical models such as DTD, XML schema \cite{xml2000}, etc.), based on the concept of "view". The authors use context free grammars as documents models. A document is thus, a derivation tree for a given grammar.

The lifecycle of a document in their workflow can be sketched as follows: initially, the document to edit ($t$) is in a specific state (initial state); various co-authors who are potentially located in distant geographical sites, get a copy of $t$ which they locally edit. For several reasons (confidentiality, security, efficiency, etc. \cite{tchoupeAtemkeng2}), a given co-author "$i$" does not necessarily have access to all the grammatical symbols that appear in the tree (document);  only a subset of them can be considered relevant for him: that is his \textit{view} ($\mathcal{V}_i$). The locally edited document, is therefore a \textit{partial replica} (denoted $t_{\mathcal{V}_i}$) of the original document. This one is obtained by \textit{projection} ($\pi$) of the original document with regard to the view of the considered co-author ($t_{\mathcal{V}_i}=\pi_{\mathcal{V}_i}(t)$). The edition is asynchronous and local documents obtained are called updated partial replicas denoted by $t_{\mathcal{V}_i}^{maj}$.

Badouel and Tchoup\'e focus only on the positive edition: edited documents are only increasing; thus, the co-authors cannot remove portions of the document when a synchronisation has already been performed. To both ensure that property, and be able to tell a co-author where he shall contribute, the documents being edited are represented by trees with \textit{buds} that indicate the only places where editions are possible.
Buds are typed; a \textit{bud of sort $X$} is a leaf node labelled $X_\omega$: it can only be edited (extended in a subtree) by using a \textit{$X$-production} (production with $X$ as left hand side).

When a synchronisation point is reached, all contributions $t_{\mathcal{V}_i}^{maj}$ of different co-authors are merged in a single global document $t_f$. To ensure that the merging is always possible (convergence), Badouel and Tchoup\'e assume that on each site, the editions are controlled by a local grammar. These local grammars are obtained from the global one, by projection along the corresponding views \cite{badouelTchoupeCmcs, tchoupeAtemkeng2}.
\begin{figure}[ht!]
	\noindent
	\makebox[\textwidth]{\includegraphics[scale=0.65]{Chap2/images/editionBPMNEn.png}}
	\caption{A BPMN orchestration diagram sketching a cooperative editing workflow of a structured document according to Badouel and Tchoup\'e.}
	\label{chap2:fig:badouel-tchoupe-workflow}
\end{figure}

Figure \ref{chap2:fig:badouel-tchoupe-workflow} gives an overview, with a BPMN orchestration diagram, of the structured documents' cooperative editing workflow according to Badouel and Tchoup\'e's proposal; at site 1, operations of (re)distribution and merging of the document in accordance with a (global) model $G$, are realised; at sites 2 and 3, edition of partial replicas in accordance with (local) models $G_1$ and $G_2$, derived by projecting the global documents model $G$, are done.

In summary, the workflow of Badouel and Tchoup\'e is different from the others with its concept of "view" and by the fact that, it exclusively manipulates (partial) structured documents.




\mySection{Tree Automata for Extracting Consensus from Partial Replicas of a Structured Document}{}
\label{chap2:sec:grammatical-cooperative-editing}


In this section, we will better study Badouel and Tchoup\'e work on structured editing. We will adopt and adapt the different mathematical tools they proposed, to produce a more general algorithm for merging partial replicates, by taking into account the cases where these would be in conflict.

\mySubSection{Structured Cooperative Editing and Notion of Partial Replication}{} 
\label{chap2:sec:structured-cooperative-editing-partial-rep}

\mySubSubSection{Structured Document, Editing and Conformity}{}
\label{chap2:sec:structured-document-edition-conformity}

In the XML community, the document model is typically specified using a DTD or a XML Schema \cite{xml2000}. It is shown that these DTD are equivalent to (regular) grammars with special characteristics called   \textit{XML grammars} \cite{XMLG}. The (context free) grammars are therefore a generalisation of the DTD and, on the basis of the studies they have undergone, mainly as formal models for the specification of programming languages, they provide an ideal framework for the formal study of the transformations involved in XML technologies. That's why we use them in our work as tools for specifying the structure of documents.

We are only interested in the structure of the documents regardless of their contents and the attributes they may contain.  
We will therefore represent the abstract structure of a structured document by a tree, and its model by an abstract context free grammar; a valid structured document will then be a derivation tree for this grammar. 
A context free grammar defines the structure of its instances (the documents that are conform to it) by means of productions. 
A production, generally denoted $p: X_0 \rightarrow X_1 \ldots X_n$, is comparable in this context, to a structuring rule which shows how the symbol $X_0$, located in the left part of the production, is divided into a sequence of other symbols $X_1 \ldots X_n$, located on its right side. More formally, 

\begin{definition}
An \textbf{abstract context free grammar} is given by $\mathbb{G}=\left(\mathcal{S},\mathcal{P},A\right)$
composed of a finite set $\mathcal{S}$ of \textbf{grammatical symbols} or \textbf{sorts} corresponding to the different \textbf{syntactic categories} involved, a particular grammatical symbol $A\in\mathcal{S}$ called \textbf{axiom}, and a finite set $\mathcal{P}\subseteq\mathcal{S}\times\mathcal{S}^{*}$ of \textbf{productions}. 
A production $P=\left(X_{P(0)},X_{P(1)}\cdots X_{P(n)}\right)$ is denoted $P:X_{P(0)}\rightarrow X_{P(1)}\cdots X_{P(n)}$ and $\left|P\right|$ 
denotes the length of the right hand side of $P$. A production with the symbol $X$ as left part is called a \textit{X-production}.
\end{definition}

For certain treatments on trees (documents), it is necessary to designate precisely a particular node. Several indexing techniques exist, among them, the so-called \textit{Dynamic Level Numbering} \cite{boe04} based on identifiers with variable lengths, inspired by the \textit{Dewey} decimal classification (see fig. \ref{chap2:fig:indexed-tree}). According to this indexing system, a tree can be defined as follows:

\begin{definition}
A \textbf{tree} whose nodes are labelled in an alphabet $\mathcal{S}$, is a partial map $t:\mathbb{N}^*\rightarrow \mathcal{S}$, whose domain $\mathit{Dom}(t)\subseteq \mathbb{N}^*$ is a prefix closed set such that, for all $u\in \mathit{Dom}(t)$, the set $\{i\in\mathbb{N}~|~u\cdot i\in\mathit{Dom}(t)\}$ is a non-empty interval of integers $[1,\cdots,n]\cap\mathbb{N}$ ($\epsilon \in \mathit{Dom}(t) \mathit{~is~ the~ root~ label}$); the integer $n$ is the \textbf{arity} of the node whose address is $u$. 
$t(u)$ is the value (label) of the node in $t$, whose address is $u$.
%Un arbre $t_{w}$ est \textbf{un sous-arbre} de $t$ de racine le noeud d'adresse $w \in \mathit{Dom}(t)$ a pour domaine $\mathit{Dom}(t_{w})=\{u ~|~ w.u \in \mathit{Dom}(t)\} \mbox{ et }  t_{w}(u) = t(w.u)$
If $t_1,\cdots,t_n$ are trees and $a\in \mathcal{S}$, we denote $t = a [t_1,\ldots,t_n]$, the tree $t$ of domain $\mathit{Dom}(t)=\{\varepsilon\}\cup\{i\cdot u~|~1\leq i\leq n\, ,\; u\in \mathit{Dom}(t_i)\}$ with $t(\varepsilon)=a$, and $t(i\cdot u)=t_i(u)$. 

%L'arbre vide sera noté \textit{nil} et $\mathit{next} ~t= [t_1, \cdots, t_n]$ est la liste des sous-arbres de l'arbre $t = a[t_1,\ldots,t_n]$. 
\end{definition}
\begin{figure}[ht!]
	\noindent
	\makebox[\textwidth]{\includegraphics[scale=0.45]{Chap2/images/arbreIndexe.png}}
	\caption{Example of an indexed tree.}
	\label{chap2:fig:indexed-tree}
\end{figure}

Let $t$ be a document and $\mathbb{G}=\left(\mathcal{S},\mathcal{P},A\right)$ a grammar. $t$ is a derivation tree for $\mathbb{G}$ if its root node is labelled by the axiom $A$ of $\mathbb{G}$, and if for all internal node $n_0$ labelled by the sort $X_0$, and whose sons $n_1, \ldots n_n$, are respectively labelled by the sorts $X_1,\ldots, X_n$, there is one production $P \in \mathcal{P}$ such that, $P:X_0\rightarrow X_1\cdots X_n$ and $\left|P\right|=n$. 
It is also said in this case, that $t$ belongs to the language generated by $\mathbb{G}$ from the symbol $A$, and it is denoted $t \in \mathscr{L}{\left( \mathbb{G}, ~A \right)}$ or  $t\therefore \mathbb{G}$.

%\begin{comment}
There is a bijective correspondence between the set of derivation trees of one grammar and all its \textit{Abstract Syntax Tree} (\textit{AST}). In an AST, nodes are labelled by the names of the productions. 

\begin{definition}
The set $AST(\mathbb{G},X)$ of \textbf{abstract syntax trees} according to the grammar $\mathbb{G}$ associated with grammatical symbol $X$, consists of trees in the form $P[t_1,\ldots,t_n]$ where $P$ is a production such that, $X=X_{P(0)}$, $n=|P|$ and $t_i\in AST(\mathbb{G},X_i), ~X_{i} = X_{P(i)}$ for all $1\leq i\leq n$. 
%Les arbres de syntaxe abstraite sont donc les termes pour la signature dont les sortes sont les symboles grammaticaux et dont les opérateurs sont les productions de la grammaire où la production $P:X_{P(0)}\rightarrow X_{P(1)}\cdots X_{P(n)}$  est vue comme un opérateur d'arité $X_{P(1)}\times\cdots\times X_{P(n)}\rightarrow X_{P(0)}$.
\end{definition}
AST are used to show that a given tree, labelled with grammatical symbols, is an instance of a given grammar.
%\end{comment}

A structured document being edited, is represented by a tree containing \textit{buds} (or \textit{open nodes}) which indicate in a tree, the only places where editions (i.e updates) are possible\footnote{Note that, we are interested only in the \textit{positive edition} based on a partial optimistic replication \cite{Yasushi2005} of edited documents. In fact, the edited documents are only increasing: there is no possible erasure as soon as a synchronisation has been performed.}.
Buds are typed; a \textit{bud of sort $X$} is a leaf node labelled by $X_\omega$ (see fig. \ref{chap2:fig:tree-with-bud}): it can only be edited (i.e extended to a subtree) by using an \textit{X-production}. Thus, a structured document being edited and that have the grammar $\mathbb {G} = (\mathcal {S}, \mathcal {P}, A) $ as model, is a derivation tree for the extended grammar 
$\mathbb{G}_{\Omega}=(\mathcal{S}\cup\mathcal{S}_{\omega},\mathcal{P}\cup\mathcal{S}_{\Omega},A)$, obtained from $\mathbb {G} $ as follows: for all sort $X$, we not only add in the set $\mathcal{S}$ of sorts a new sort $X_{\omega}$, but we also add  a new $\varepsilon$-production $X_{\Omega} : X_{\omega} \rightarrow \varepsilon$ in the set $\mathcal {P}$ of productions; so we have: $\mathcal{S}_{\omega}=\{X_{\omega},~ X\in\mathcal {S}\}$ and $\mathcal{S}_{\Omega} = \{X_{\Omega} : X_{\omega} \rightarrow \varepsilon,~ X_{\omega} \in \mathcal{S}_{\omega}\}$.
\begin{figure}[ht!]
	\noindent
	\makebox[\textwidth]{\includegraphics[scale=0.45]{Chap2/images/documentBourgeons.png}}
	\caption{Example of a tree that contains buds.}
	\label{chap2:fig:tree-with-bud}
\end{figure}


When we look at the productions of a grammar, we can notice that each sort is associated with a set of productions. From this point of view therefore, we can consider a grammar as an application
\[
	 gram : symb \rightarrow [(prod,~[symb])]
\]
which associates to each sort, a list of pairs formed by a production name and the list of sorts in the right hand side of this production. Such an observation suggests that a grammar can be interpreted as a (descending) tree automaton that can be used for recognition or for the generation of its instances.

\begin{definition}
\label{chap2:def:tree-automaton}
A (descending) \textbf{tree automaton} defined on $\Sigma$, is a quadruplet $\mathcal{A}=(\Sigma,Q,R,q_0)$ of a 
 set $\Sigma$ of symbols %avec arité  (signature)
; its elements are the nodes' labels of the trees to be generated (or recognised), a set $Q$ of states, a particular state $q_0 \in Q$ called initial state, and a finite set $R\subseteq  Q \times \Sigma \times Q^*$ of transitions.
\begin{itemize}
	\item An element of $R$ is denoted $q\rightarrow \left( \sigma, [ q_1,\cdots,q_n]\right)$ or in an equivalent way  $q\stackrel{\sigma}{\rightarrow}(q_{1},\ldots,q_{n})$: intuitively, $[ q_1,\cdots,q_n]$ is the list of states accessible from $q$ by crossing a transition labelled $\sigma$.
	\item If $q\stackrel{\sigma_1}{\rightarrow}\left( q_{1}^1, \cdots, q_{n_1}^1\right), \cdots, q\stackrel{\sigma_k}{\rightarrow}\left( q_{1}^k, \cdots, q_{n_k}^k\right)$ denotes the set of transitions associated to the state $q$, we denote \textbf{$next~q=[\left( \sigma_1,[ q_{1}^1, \cdots, q_{n_1}^1]\right),\cdots, \left( \sigma_k,[ q_{1}^k, \cdots, q_{n_k}^k]\right)]$}, the list that consists of pairs $\left( \sigma_i,[ q_{1}^i, \cdots, q_{n_i}^i]\right)$. 
	A transition of the form $q\rightarrow(\sigma,[\;])$, is called \textit{final transition} and a state possessing this transition is a \textit{final state}.
\end{itemize}
\end{definition}


One can interpret a grammar $\mathbb{G}=\left(\mathcal{S},\mathcal{P},A\right)$ as a (descending) tree automaton \cite{Comon} $\mathcal{A}=(\Sigma,Q,R,q_0)$ considering that:
\begin{itemize}
	\item[(1)] $\Sigma=\mathcal{P}$ is the type of labels of the nodes forming the tree to recognise; 
	\item[(2)] $Q=\mathcal{S}$ is the type of states and, 
	\item[(3)] $q\rightarrow \left( \sigma,[ q_1,\cdots,q_n]\right)$ is a transition of the automaton when the pair $\left(\sigma,[q_1,\cdots,q_n] \right)$ appears in the list $(gram~~ q)$\footnote{Reminder: $gram$ is the application obtained by abstraction of $\mathbb{G}$ and have as type : $gram : symb \rightarrow [(prod,~[symb])]$.}.
\end{itemize}
We note $\mathcal{A}_{\mathbb{G}} $ the tree automaton derived from $\mathbb{G}$.

To obtain the set $AST_\mathcal{A}$ of \textit{AST} generated by a tree automaton $\mathcal{A}$ from an initial state $q_0$, one must:
\begin{itemize}
	\item[(1)] Create a root node $r$, associate the initial state $q_0$ and add it to the set $AST_\mathcal{A}$ initially empty;
	\item[(2)] Remove from $AST_\mathcal{A}$ an AST $t$ under construction, i.e. with at least one leaf node $node$ unlabelled. Let $q$ be the state associated to $node$.
	For all transition $q\stackrel{\sigma}{\rightarrow} \left(q_1,\cdots,q_n\right)$   of $\mathcal{A}$, add in $AST_\mathcal{A}$ the trees $t'$ which are replicas of $t$ in which, the node $node$ has been substituted by a node $node'$ labelled $\sigma$ and possessing $n$ (unlabelled) sons, each associated to a (distinct) state $q_i$ of $[ q_1,\cdots,q_n]$; 
	\item[(3)] Iterate step (2) until he obtains trees with all the leaf nodes labelled (they are consequently associated to the final states of $\mathcal{A}$): these are \textit{AST}.
\end{itemize}
We note $\mathcal{A} \models t \triangleright q$ the fact that the tree automaton $\mathcal{A}$ accepts the tree $t$ from the initial state $q$, and
 $\mathscr{L}(\mathcal{A}, q)$ (tree language) the set of trees generated by the automaton $\mathcal{A}$ from the initial state $q$. 
Thus, $ \left(\mathcal{A}  \models t \triangleright q \right) \Leftrightarrow \left( t \in \mathscr{L}(\mathcal{A}, q) \right)$.


As for automata on words, one can define a synchronous product on tree automata to obtain the automaton recognising the intersection, the union, etc., of regular tree languages \cite{Comon}. We introduce below the definition of the synchronous product of $k$ tree automata whose adaptation will be used in the next section for the derivation of the consensual automaton. 

\begin{definition}
\label{chap2:def:synchronous-product}
\textbf{Synchronous product of $k$ automata:}\\
Let $\mathcal{A}_1=\left(\Sigma,Q^{(1)},R^{(1)},q_{0}^{(1)}\right), \ldots , \mathcal{A}_k=\left(\Sigma,Q^{(k)},R^{(k)},q_{0}^{(k)}\right) $ be $k$ tree automata. The synchronous product of these $k$ automata $\mathcal{A}_1 \otimes \cdots \otimes \mathcal{A}_k$ denoted $\otimes_{i=1}^{k} \mathcal{A}^{(i)}$, is the automaton $\mathcal{A}_{(sc)}=(\Sigma,Q,R,q_{0})$ defined as follows: 
\begin{itemize}
	\item[\textbf{(a)}] Its states are vectors of states : $Q =Q^{(1)}\times\cdots\times Q^{(k)}$; 
	\item[\textbf{(b)}] Its initial state is the vector formed by the initial states of the different automata : $q_{0}=\left(q_{0}^{(1)},\cdots, q_{0}^{(k)}\right)$; 
	\item[\textbf{(c)}] Its transitions are given by :\\
				$\left(q^{(1)}, \ldots, q^{(k)}\right)$ $\stackrel{a}{\rightarrow}\left(\left(q^{(1)}_1,\ldots,q^{(k)}_1\right),\ldots,\left(q^{(1)}_n,\ldots,q^{(k)}_n\right)\right)$ $\Leftrightarrow$\\ 
				   $\left( q^{(i)}\stackrel{a}{\rightarrow}\left(q^{(i)}_1,\ldots,q^{(i)}_n\right) \quad \forall i,~ 1\leq i\leq k \right)$
\end{itemize}
\end {definition}

\mySubSubSection{Notions of View, Projection, Reverse Projection and Merging}{}
\label{chap2:sec:view-projection-expansion-merging}
\noindent\textbf{\textit{View, associated projection and merging}}

The derivation tree giving the (global) representation of a structured document edited in a cooperative way, makes visible the set of grammatical symbols of the grammar that participated in its construction. As we mentioned in section \ref{chap2:sec:badouel-tchoupe-cooperative-editing} above, for reasons of confidentiality (accreditation degree), a co-author manipulating such a document will not necessarily have access to all of these grammatical symbols; only a subset of them can be considered relevant for him: it is his \textit{view}. A view $\mathcal{V}$ is then a subset of grammatical symbols ($\mathcal{V} \subseteq \mathcal{S}$). 
%Intuitivement,  il s'agit des sortes visibles par un co-auteur dans la représentation globale (arbre de dérivation) du document considéré. 

 
A partial replica of $t$ according to the view $\mathcal{V}$, is a partial copy of $t$ obtained by deleting in $t$, all the nodes labelled by symbols that are not in $\mathcal{V}$. 
Figure \ref{chap2:fig:partial-view} shows a document $t$ (center) and two partial replicas $t_{v_1}$ (left) and $t_{v_2}$ (right) obtained respectively by projections from the views $\mathcal{V}_1 = \{A,B\}$ and $\mathcal{V}_2 = \{A,C\}$. 
\begin{figure}[ht!]
	\noindent
	\makebox[\textwidth]{\includegraphics[scale=0.5]{Chap2/images/docEtRepliques.png}}
	\caption{Example of projections made on a document and partial replicas obtained.}
	\label{chap2:fig:partial-view}
\end{figure}


Practically, a partial replica is obtained via a \textit{projection} operation denoted $\pi$. We therefore denote $\pi_{\mathcal{V}}(t)= t_{\mathcal{V}}$ the fact that $t_{\mathcal{V}}$ is a partial replica obtained by projection of $t$ according to the view $\mathcal{V}$.

Note $t_{\mathcal{V}_i} \leq t_{\mathcal{V}_i}^{maj}$ the fact that, the document $t_{\mathcal{V}_i}^{maj}$ is an update of the document $t_{\mathcal{V}_i}$, i.e. $t_{\mathcal{V}_i}^{maj}$ is obtained from $t_{\mathcal{V}_i}$ by replacing some of its buds by trees.
In an asynchronous cooperative editing process, there are synchronisation points\footnote{Recall that a synchronisation point can be defined statically or triggered by a co-author as soon as certain properties are satisfied.} in which, one tries to merge all contributions $t_{\mathcal{V}_i}^{maj}$ of the various co-authors to obtain a single comprehensive document $t_f$\footnote{It may happen that the edition must be continued after the merging (this is the case if there are still buds in the merged document): one must redistribute to each of the $n$ co-authors a (partial) replica $t_{\mathcal{V}_i}$ of $t_f$ such that  $t_{\mathcal{V}_i} = \pi_{\mathcal{V}_i}(t_f)$ for the continuation of the editing process.}. A merging algorithm that does not incorporate conflict management and relies on a solution to the \textit{reverse projection} problem was given by Badouel and Tchoup\'e.

~

\noindent\textbf{\textit{Partial replica and reverse projection (expansion)}}

The \textit{reverse projection} (also called the \textit{expansion}) of an updated partial replica $t_{\mathcal{V}_i}^{maj}$ relatively to a given grammar $\mathbb{G}=\left(\mathcal{S},\mathcal{P},A\right)$, is the set $T_{i\mathcal{S}}^{maj}$ of documents conform to $\mathbb{G}$, that admit $t_{\mathcal{V}_i}^{maj}$ as partial replica according to ${\mathcal{V}_i}$:
\begin{itemize}
	\item[] $ T_{i\mathcal{S}}^{maj} = \left\{t_{i\mathcal{S}}^{maj} \therefore \mathbb{G}~ | ~ \pi_{\mathcal{V}_i}\left(t_{i\mathcal{S}}^{maj}\right) = t_{\mathcal{V}_i}^{maj} \right\}
	$
\end{itemize} 

A solution to the problem of evaluating the expansion of a given partial replica using tree automata, was proposed by Badouel and Tchoup\'e; in that solution, productions of the grammar $\mathbb{G}$ are used, to bind to a view $\mathcal{V}_i \subseteq \mathcal{S}$ a tree automaton $\mathcal{A}^{(i)}$ such as, the trees it recognises from an initial state built from $t_{\mathcal{V}_i}^{maj}$, are exactly those having this partial replica as projection according to the view $\mathcal{V}_i$:
$ T_{i\mathcal{S}}^{maj} = \mathscr{L}\left(\mathcal{A}^{(i)},~q_{t_{\mathcal{V}_i}}\right) $. 
Practically, they have considered that a state $q$ of the automaton $\mathcal{A}^{(i)}$ is a pair $\left(Tag ~X, \;ts\right)$ where $X$ is a grammatical symbol, $ts$ is a forest (tree set), and \textit{Tag} is a label that is either \textit{Open} or \textit{Close}, and indicates whether the concerned state $q$ can be used to generate a \textit{closed} node or a \textit{bud}. The states of $\mathcal{A}^{(i)}$ are typed: a state of the form $\left(Tag ~X, \;ts\right)$ is of type $X$. We also have a function named \textit{typeState} which, when applied to a state, returns its type\footnote{ $typeState :: state\rightarrow symb$ \\
 $.~~~typeState ~\left(Open ~X, \;ts\right) = X$\\
 $.~~~typeState ~\left(Close ~X, \;ts\right) = X$
	}.
A transition from one state $q$, is of one of the forms $\left(Close~X, \;ts \right) \rightarrow \left(p, \;[q_1, \ldots, q_n]\right)$ or $\left(Open ~X, \;[\;]\right) \rightarrow \left(X_\omega, \;[\;]\right)$. 
A transition of the form $\left(Close~X, \;ts \right) \rightarrow \left(p, \;[q_1, \ldots, q_n]\right)$ is used to generate AST of type $X$ (whose root is labelled by a \textit{X-production}) admitting ''$ts$'' as projection according to the view ${\mathcal{V}_i}$ if $X$ does not belong to ${\mathcal{V}_i}$, and ''$x [ ts ]$'' otherwise.
Similarly, a transition of the form $\left(Open ~X, \;[\;]\right) \rightarrow \left(X_\omega, \;[\;]\right)$ is used to generate a single AST reduced to a bud of type $X$. 

The interested reader may consult \cite{badouelTchoupeCmcs} for a more detailed description of the process of associating a tree automaton with a view and the section \ref{chap2:sec:consensus-illustration} for an illustration.

\documentclass[a4paper,12pt]{article}

\usepackage{relsize}
\usepackage{epigraph}
\usepackage{nicefrac}
\usepackage[margin=1in]{geometry}
\usepackage{fullpage}
\usepackage[T1]{fontenc}
%\usepackage{style}
%\usepackage{framed}
%\usepackage[parfill]{parskip}
%\usepackage{libertine} 
%\usepackage[libertine]{newtxmath}

\usepackage{hyperref}
\usepackage{color}
\hypersetup{colorlinks=true,linkcolor=blue,citecolor=blue}
\usepackage{float}
%\floatstyle{boxed}
%\newfloat{algorithm}{t}{lop}
%\usepackage{natbib}
%\usepackage{tikz}
\usepackage{varwidth}

\newcommand{\q}{$``?"$}

\usepackage{mathtools}
\usepackage{float}
\usepackage[titletoc,title]{appendix}
\usepackage[classfont=sanserif,langfont=roman,funcfont=italic]{complexity}
\usepackage{caption}
%\usepackage{dsfont}
\usepackage{comment}
\usepackage{enumerate}
\usepackage{amsmath, amsthm, amstext,  graphicx,amsopn} %comment
\usepackage{amssymb}
\usepackage{subfigure}
\usepackage{tcolorbox}
%\usepackage{titlesec}
%\usepackage{lmodern}
%\titlespacing*{\section}{0pt}{3pt}{0pt}
%\titlespacing*{\subsection}{0pt}{2pt}{0pt}
%\titlespacing*{\subsubsection}{0pt}{1pt}{0pt}

\bibliographystyle{alpha}

\newtheorem{definition}{Definition}
\newtheorem{proposition}{Proposition}
\newtheorem{example}{Example}
\newtheorem{property}{Property}
\newtheorem{corollary}{Corollary}
\newtheorem{note}{Note}
\newtheorem{todo}{TODO}
\newtheorem{claim}{Claim}
\newtheorem{theorem}{Theorem}
\newtheorem{lemma}{Lemma}
\newtheorem{observation}{Observation}
\newtheorem{conjecture}{Conjecture}
\newtheorem{remark}{Remark}
%\usepackage[T1]{fontenc}
% compact list spacing
\usepackage{enumitem}
\usepackage{bbm}
%\setlist{topsep=0pt,parsep=0pt,partopsep=0pt,itemsep=0pt}
\newcommand{\simina}[1]{{\color{red}\noindent\textbf{Note: }\marginpar{****}\textit{{#1}}}}%

\DeclareMathOperator{\Ex}{\mathbb{E}}% expected value


%\newcommand{\footnoteremember}[2]{

\DeclareMathOperator*{\argmax}{arg\,max}

%%%%%%%%%!!!!!!!!!!Imported from Warnke!!!!!!%%%%%%%%%%%%%%%%%%%%%%%%%%
%\usepackage{mathpazo}
\newcommand\set[1]{\ensuremath{\{#1\}}}
\newcommand\bigset[1]{\ensuremath{\bigl\{#1\bigr\}}}
\newcommand\Bigset[1]{\ensuremath{\Bigl\{#1\Bigr\}}}
\newcommand\biggset[1]{\ensuremath{\biggl\{#1\biggr\}}}
\newcommand\lrset[1]{\ensuremath{\left\{#1\right\}}}
\newcommand\xpar[1]{(#1)}
\newcommand\bigpar[1]{\bigl(#1\bigr)}
\newcommand\Bigpar[1]{\Bigl(#1\Bigr)}
\newcommand\biggpar[1]{\biggl(#1\biggr)}
\newcommand\Biggpar[1]{\Biggl(#1\Biggr)}
\newcommand\lrpar[1]{\left(#1\right)}
\newcommand\bigsqpar[1]{\bigl[#1\bigr]}
\newcommand\Bigsqpar[1]{\Bigl[#1\Bigr]}
\newcommand\biggsqpar[1]{\biggl[#1\biggr]}
\newcommand\lrsqpar[1]{\left[#1\right]}
\newcommand\xcpar[1]{\{#1\}}
\newcommand\bigcpar[1]{\bigl\{#1\bigr\}}
\newcommand\Bigcpar[1]{\Bigl\{#1\Bigr\}}
\newcommand\biggcpar[1]{\biggl\{#1\biggr\}}
\newcommand\lrcpar[1]{\left\{#1\right\}}

\newcommand\abs[1]{|#1|}
\newcommand\bigabs[1]{\bigl|#1\bigr|}
\newcommand\Bigabs[1]{\Bigl|#1\Bigr|}
\newcommand\biggabs[1]{\biggl|#1\biggr|}
\newcommand\lrabs[1]{\left|#1\right|}

\newcommand\oY{{\overline Y}}
\newcommand\oy{{\overline y}}
\newcommand\ty{{\tilde{y}}}


\newcommand\cB{{\mathcal B}}
\newcommand\cC{{\mathcal C}}
\newcommand\cD{{\mathcal D}}
\newcommand\cE{{\mathcal E}}
\newcommand\cF{{\mathcal F}}
\newcommand\cG{{\mathcal G}}

\newcommand\cI{{\mathcal I}}
\newcommand\cM{{\mathcal M}}
\newcommand\cQ{{\mathcal Q}}
\newcommand\cR{{\mathcal R}}

\newcommand\cT{{\mathcal T}}

\newcommand\fR{{\mathfrak R}}

\newcommand{\uv}{{\underline v}}
\newcommand{\uu}{{\underline u}}
\newcommand{\uc}{{\underline c}}

\newcommand\tc{t_{\mathrm{c}}}
\newcommand\tci[1][i]{t_{\mathrm{c},{#1}}}
\newcommand\tcx{t_{\mathrm{b}}}
\newcommand\tcs{t_{\mathrm{s}}}
\newcommand{\ts}{t}

\newcommand{\St}{{\tilde{S}}}

\newcommand{\cTv}[1][{v,\ts}]{{\mathcal T}_{{#1}}}
\newcommand{\cVv}[1][{v,\ts}]{{\mathcal V}_{{#1}}}
\newcommand{\tpv}[1][{v,\ts}]{{\mathfrak T}_{{#1}}}
\newcommand{\tpvV}[1][{v,\ts}]{{\mathfrak V}_{{#1}}}
\newcommand\bpt[1][{\ts}]{{\mathfrak T}_{{#1}}}
\newcommand\bpV[1][{\varphi,\ts}]{{\mathfrak V}_{{#1}}}
\newcommand\bp[1][{\varphi,\ts}]{{\mathfrak X}_{{#1}}}
\newcommand\tF{{\widetilde F}}
\newcommand\norm[1]{||#1||}

%\newcommand{\indic}[1]{\mathbbm{1}_{\{{#1}\}}}



%%%%%%%%%%%%%%%%%%%%%%%%%%%%%%%%%%%%%%%%%%%%%%%%%%%%%%%%%%%%%%%%%%%%%%%

%\title{Optimal Communication Cost of Consensus with Low Memory}
\title{Consensus with Bounded Space and Minimal Communication}
%\title{Consensus Protocol with Minimal Communication for Low Memory Nodes}
\author{Simina Br\^anzei\thanks{Purdue University, USA. E-mail: \url{simina.branzei@gmail.com}.} \and Yuval Peres\thanks{E-mail: \url{yuval@yuvalperes.com}.}}

\date{}

\begin{document}
\maketitle

\begin{abstract}
	Population protocols are a fundamental model in distributed computing, where many nodes with bounded memory and computational power have random pairwise interactions over time. This model has been studied in a rich body of literature aiming to understand the tradeoffs between the memory and time needed to perform computational tasks.
	
	We study the population protocol model focusing on the communication complexity needed to achieve consensus with high probability. When the number of memory states is $s = O(\log \log{n})$, the best upper bound known was given by a protocol with $O(n \log{n})$ communication, while the best lower bound was $\Omega(n \log(n)/s)$ communication. %~\cite{comm_init}.
	
	We design a protocol that shows the lower bound is sharp, solving an open problem from \cite{comm_init}. When each agent has  $s=O(\log{n}^{\theta})$ states of memory, with $\theta \in (0,1/2)$, consensus can be reached in time $O(\log(n))$ with $O(n \log{(n)}/s)$ communications with high probability.
\end{abstract}

\newpage 
\section{Introduction}
Population protocols are a basic model in distributed computing introduced in~\cite{AADFP04}, where a collection of agents with bounded memory have random pairwise interactions over time. Population protocols can be used to model colonies of insects such as ants and bees, flocks of birds~\cite{AADFP04,DFGR06}, chemical reaction networks~\cite{CardelliHDC17}, gene regulatory networks~\cite{BB04}, wireless sensor networks~\cite{DV12}, and opinion formation in social networks~\cite{PVV09} (see overview in~\cite{AAEGR17}). 

In a population protocol there are $n$ agents, each of which is represented as a finite state machine and has a Poisson clock with unit rate. When the clock of an agent rings, the node wakes up and gets matched with a random other node. The two nodes update their states as a function of both of their previous states. %The goal is to jointly compute a function of the initial state of the system. 
In the consensus (majority) problem, each agent starts with an initial belief bit and the goal is to have all the agents learn, through interactions, the belief with higher initial count. Note this model is equivalent to a discrete model where in each round, a random node wakes up and is matched to another random node. \footnote{The equivalence between the continuous and discrete models holds up to rescaling the time.}

The population protocol model can be used to provide insights into how eusocial insects solve problems such as searching for a new home (reaching consensus), foraging for food, or allocating tasks to workers~\cite{Radeva17}.
In the case of chemical reaction networks, assumptions about chemical solutions are very similar to the rules that define population protocols: molecules can be seen as agents that have pairwise interactions (collisions), the next pair to interact is chosen randomly (as in a well-mixed solution), each molecule has a finite number of states, and each interaction can update the state of one or both molecules (see~\cite{CardelliHDC17}).
More recent experimental evidence suggests that population protocols can be implemented at the molecular level by DNA nucleotides and are equivalent to computations carried out by living cells to function correctly (see discussion in ~\cite{AAEGR17} and ~\cite{CDSPCSS13,CC12}).

The consensus problem has been studied extensively from a computational point of view, aiming to understand the resources required to reach consensus, such as time and memory~\cite{AngluinAER07,DraiefV12,AAEGR17,AlistarhDKSU17,AlistarhAG18,AlistarhG18,BerenbrinkGK20}. One of the simplest consensus protocols is the \emph{three state protocol}~\cite{AAE08}, where every node has only $3$ states of memory, labeled ``$0$”, ``$1$”, and ``$?$”. Each node starts by initializing its state to their initial belief bit. When the clock of a node $i$ rings, the node wakes up and gets matched with another random node $j$, with the following update rules. If both nodes have the same belief in $\{0,1\}$, then they keep their beliefs. However, if node $i$’s belief is $b \in \{0,1\}$ and node $j$’s belief is ``$?$'', then node $j$ updates its belief to $b$. If on the other hand nodes $i$ and $j$ have opposing beliefs in $\{0,1\}$, then node $j$ changes their belief to ``$?$'' while node $i$ keeps its belief. The three state protocol converges to the correct majority bit with high probability in time $O(\log n)$ and communication $O(n \log{n})$. \cite{5062181} gives an elegant approach for studying the three state protocol, by considering the deterministic system obtained when the number of nodes $n$ goes to infinity and studying the resulting differential equations. %Their analysis implies that the three state protocol reaches consensus in $O(\log{n})$ time.

In many distributed settings, such as blockchain, setting up a communication channel between the nodes is expensive, so light communication is crucial for the efficiency of a protocol~\cite{comm_init}. Minimizing the communication complexity (or cost) requires that nodes do not communicate each time they wake up. Rather, when a node wakes up it decides first whether to communicate or not. If it decides to communicate, then it gets matched with a random other node as usual and they exchange states. If it decides to not communicate, then the node can update its internal state by itself. For example, a useful type of update that a node can do individually is to increase a counter that tells it how many rounds it has been since it last communicated. Thus the goal is to minimize the number of communications required to reach consensus with high probability. \footnote{Thus each time two nodes communicate will count as one communication, regardless of the amount of information passed through the channel.}

The study of communication in population protocols was initiated in~\cite{comm_init}, which showed that when the number $s$ of memory states of a node is $s > \log \log{n}$, then consensus can be reached with $O(n)$ communication. When $s < \log \log{n}$, the upper and lower bounds did not match; the best upper bound known is given by an algorithm with $O(n \log{n})$ communication, while the lower bound is $\Omega(n \log(n)/s)$ communication. 

Our main result solves the open problem from ~\cite{comm_init} by showing that the lower bound is tight. We design and analyze a protocol that achieves consensus with $O(n\log(n)/s)$ communication in $O(\log{n})$ time w.h.p., where $s = O(\log \log(n)^3)$ is the number of memory states per agent. Note the three state protocol described above also runs in time $O(\log{n})$ and in fact this is the minimum possible time for \emph{any} consensus protocol that is correct.

\begin{theorem}[Main Theorem] \label{thm:main}
	Suppose that each node has $s=O(\log{n}^{\theta})$ states of memory where $\theta \in (0,1/2)$. Then with high probability, consensus can be reached in time $O(\log(n))$ with $O(n \log{(n)}/s)$ communications.
\end{theorem}

\begin{remark}
When $s < \log{\log{n}}$, the upper bound from Theorem~\ref{thm:main} matches up to constants the lower bound from \cite{comm_init}. For $s < \left( \log{\log{n}}\right)^3$, Theorem~\ref{thm:main} improves the state of the art, while for $s >  \left( \log{\log{n}}\right)^3$ there is an algorithm with $O(n)$ communication. 
\end{remark}

\paragraph{Algorithm} %\label{sec:protocol}
Our first main contribution is to design the following consensus algorithm, which achieves the communication and time bounds of Theorem~\ref{thm:main}:

%\noindent \emph{\bf Step 1:} Run the three state protocol for $s$ steps. This will bring the fraction of incorrect leaders, question mark leaders, and incorrect non-leaders down to small constants (TBD). \\

%\noindent \emph{\bf Step 2:}
\begin{itemize}
	\item Self-select a number of $n/s$ \emph{leaders}\footnote{This step can be implemented in a decentralized way by using the randomness in the interactions. See ~\cite{comm_init} for how this is done. Note this has a negligible effect on the protocol and proofs work the same.}. The rest of the nodes will be \emph{followers}.
	\item Each follower keeps a counter that can take a value from $\{1, \ldots, 8 s+1\}$. The counter value of each follower is initialized uniformly at random from $\{1, \ldots, 8s\}$. We denote by bin $j$ the set of followers with counter value $j$.
	The followers are divided in two groups:
	\begin{itemize}
		\item \emph{Informed nodes:} with counter value in $\{1, \ldots, 8 s \}$. When the clock of such a node rings, the node increases its counter value by one.
		\item \emph{Uninformed nodes:} with counter value $8 s+1$. When the clock of such a node rings, they communicate, and if they reach an informed node, then they adopt its bit and counter value.
	\end{itemize}
	\item Each leader has a belief, which can be $0$, $1$, or \q. When the clock of a leader rings, the leader acts as follows depending on its belief:
	\begin{itemize}
		\item 0 or 1: If the leader gets matched with an informed follower, then the leader flips a fair coin. If the coin turns heads, then the leader pushes, and if it turns tails, then the leader pulls from the follower.
		
		In case of a pull, if the bits match, then the leader keeps its belief, and if they don't match, then the leader switches to \q.
		
		\item \q: If the leader meets an informed node, then it adopts the informed node's bit.
	\end{itemize}
\end{itemize}

To understand this protocol, we study the following deterministic approximation, which can be understood as taking the number of nodes $n$ to $\infty$ while keeping the parameter $s$ fixed. The deterministic system will be given by some differential equations that we analyze and then show the random system will closely approximate it for large enough $n$. We keep track of the following fractions, the denominator for all of which is the number of nodes $n$ (and then take the limit of $n \to \infty$):
\begin{description}
	\item[$\bullet$] $\alpha(t)$: Fraction of nodes that are leaders with the incorrect bit at time $t$.
	\item[$\bullet$] $\delta(t)$: Fraction of nodes that are leaders and have belief \q.
	\item[$\bullet$] $\beta_j(t)$: Fraction of nodes that are followers, informed, and have the incorrect bit and counter value $j$.
	\item[$\bullet$] $\beta(t) = \sum_{j=1}^{8s} \beta_j(t)$: Fraction of nodes that are followers, informed, and have the incorrect bit.
	\item[$\bullet$] $\gamma_j(t)$: Fraction of nodes that are followers with counter value $j$.
	\item[$\bullet$] $u(t)$: Fraction of nodes that are uninformed.
\end{description}

We let $\widetilde{\alpha}(j)$ be the number of leaders with the incorrect bit after $j$ clock rings in the random system given by our protocol.
The execution of our protocol for $n=6000$ nodes with $s=5$ is illustrated in Figure 1. 
%\begin{figure}[h!] 
%	\label{fig:random_system_prob=0.3}
%	\centering
%	\includegraphics[scale=1.5]{./pics/alpha_delta_c=0_49_s=5_random.png}
%	\caption{The fraction of leaders with the wrong bit (in red) and the fraction of nodes that are informed followers with the wrong bit (in blue) over time, where $n = 6000$, $s=5$, and the initial minority fraction is $0.49$. The $X$ axis shows the number $j$ of clock rings. The values are normalized, so the formula for the red curve is $\frac{\widetilde{\alpha}(j)}{n} \cdot s$ and for the blue one is $\frac{\widetilde{\beta}(j)}{n} \cdot \frac{s}{s-1}$.}
%\end{figure}

\begin{figure}[h!]
	\centering
	\subfigure[$\frac{\widetilde{\alpha}(t)}{n} \cdot s$ (in red) and $\frac{\widetilde{\beta}(t)}{n} \cdot \frac{s}{s-1}$  (in blue) over time.]
	{
		\includegraphics[scale=0.56]{alpha_beta_random_n=3000_s=5_c=0_45.png}
	}
	\subfigure[${{\alpha}(t)} \cdot s$ (in red) and ${{\beta}(t)} \cdot \frac{s}{s-1}$ (in blue) over time.]
	{
		\includegraphics[scale=0.56]{alpha_beta_deterministic_n=3000_s=5_c=0_45.png}
	}
	\subfigure[$\frac{\widetilde{\delta}(t)}{n} \cdot s$ over time.]
	{
		\includegraphics[scale=1.5]{delta_random_n=3000_s=5_c=0_45.png}
	}
	\subfigure[${{\delta}(t)} \cdot s$ over time.]
	{
		\includegraphics[scale=1.5]{delta_deterministic_n=3000_s=5_c=0_45.png}
	}
	\caption{The random and deterministic system for $n=3000$, $s=5$, and initial minority $45\%$ (that is, $\rho=0.1$).}
\end{figure}


\begin{comment}
\begin{todo}
	\begin{itemize}
	\item Say in most models, the node that wakes up always communicates - they get matched to random other node.
	\item After the protocol explain how the nodes are chosen as leaders (see other paper). Explain this introduces a small error that doesn't have an effect on the rest of the proof.
	\item Emphasize that the nodes communicate at rate $1/s$ and yet the algorithm still completes in time $log(n)$. In the proof there are two crucial things to show - the protocol completes in time $log(n)$ and the number of uninformed stays at $1/s$.
	\item Add overview that describes informally the three phases and what each one achieves.
	\item For the picture in the intro, show random and deterministic side by side.
	\item Plot random system at increments of $n * step$, where $step$ is the grid size in the deterministic system.
	\end{itemize}
\end{todo}
\end{comment}

\subsection{Overview}

To prove Theorem 1, it suffices to show that w.h.p. consensus is reached in time $O(\log(n)$ and the rate of communication is $O(1/s)$. i.e. only one in $\Omega(s)$ clock rings leads to a communication. The latter claim will follow if we show that there at most $O(n/s)$ uninformed nodes throughout, since the only nodes that initiate communications are the $n/s$ leaders and the uninformed nodes.

\medskip

The proof of Theorem 1 consists of three phases:
\begin{description} 
\item[I.]	From near tie to a dominant majority in time $\Theta(s)$;
\item[II.]	Exponential decay of minority at rate $\Omega(1/s)$ for $\Theta(s^2)$ time units;
\item[III.]	Exponential decay of minority at constant rate for $\Theta(\log n)$ time units.
\end{description}
Each phase is first proved for a deterministic approximation to the process that solves a system of $\Theta(s)$ coupled differential equations. Then martingale arguments, described further below, enable us to analyze the random system.
The system of differential equations is too involved to solve directly, so four different potential functions are used to control it.

\medskip

{\bf Phase I} is the most delicate. The key to this phase is to change variables and consider the relative advantage of the majority among decisive leaders and informed followers. Thus the new variables are 
$$\xi=\frac{\left(1/s-\delta-\alpha\right)-\alpha}{1/s-\delta} \quad \text{and} \quad 
\eta_j=\frac{(\gamma_j-\beta_j)-\beta_j}{\gamma_j} \,.
$$
It is quite intuitive (and easy to verify, see Proposition 1) that the minimal relative advantage
$$ \Phi = \min\Bigl\{  \xi, \eta_1, \ldots, \eta_{8s} \Bigr\}$$
is nondecreasing, but to bound the convergence time this must be made more quantitative. 

In our protocol, followers adopt the belief bits sent to them, but leaders only flip their belief bits after pulling contrary bits from two or more followers. This the relative advantage of the majority among leaders grows faster than the corresponding advantage among followers. Our proof uses this to define a modified potential function $\Psi$, see (\ref {psi_fun}), that grows at rate $\Omega(s/rho)$ as long as $\Phi$ is bounded away from 1, where $\rho$ is the initial value of $\Phi$. We deduce in Proposition \ref{prop:phase1_progress} that $\Phi$ will exceed a prescribed level $\lambda<1$ by time $T_1=O\left(\frac{s}{\rho(1-\lambda)}\right)$. 
	
	\medskip 
	
	{\bf Phase II}  starts at time $T_1$. In this phase (using a bound on $\beta$ obtained in phase I) we show that a linear potential function in the original variables 
	$$\Lambda_2=\alpha+\frac{\delta}{16}+\frac{\beta}{4}$$
	exhibits exponential decay at rate $\Omega(1/s)$.
	
	\medskip

{\bf Phase III} starts at time $T_2=\Theta(s^2)$. At this time, $\beta$ is exponentially small in $s$ (via phase II). This allows us to use a modified
potential function $\Lambda_3$ (a linear combination of $\alpha, \delta$ and the $\beta_j$)  that exhibits true exponential decay. We deduce that $\delta$ and all the minority fractions $\alpha,\beta_j$ will be well below $1/n$ by time $O(\log n)$. 

To handle the random fluctuations in phases I and II we use the differential equation method for approximating stochastic processes developed by Kurtz~\cite{kurtz} and Wormald~\cite{wormald}, in the optimized version presented by Warnke~\cite{warnke}. Such approximations cannot be used in phase III, because the approximation error between the random process and the deterministic one scaled by $n$ is of order $\Omega(n^{1/2})$.

In the main text, we focus on the case where the initial advantage $\rho=\Phi(0)$ is a positive constant. However, even if $\rho$ tends to zero with $n$, the protocol works and reaches consensus on the majority value w.h.p provided 
$\rho>n^{-\sigma}$ for some $\sigma \in (0,1/2)$. See the discussion in the concluding remarks.

%\noindent \emph{Model}: Set of nodes $[n] = \{1, \ldots, n\}$. Each node $i$ is an automaton with $s$ states and has an initial bit $b_i$. The goal is to reach consensus with high probability on the correct majority of the initial bits, where a fraction $p > 1/2 $ of nodes have the majority belief. We study the problem in the population protocol model, where each node $i$ has a Poisson clock with unit rate. When its clock rings, the node can change its state and has the option to communicate with another node. If $i$ does want to communicate, then it gets matched with a uniformly chosen node $j$, case in which they exchange and update their states.




%\begin{todo}
%	In the above plot the memory is actually $20 \cdot s$.
%\end{todo}

\subsection{Related literature} \label{sec:further_related_work}

\cite{AngluinAER07} studies the computational power of population protocols and gives a precise characterizations of the class of predicates that can be computed in a stable way on complete interaction graphs (i.e. where each agent can get matched with any other agent).

The tradeoffs between time and memory required to reach consensus have been studied in a rich body of literature. \cite{AAEGR17} studies majority and leader election, showing a  unified lower bound for the two problems that relates the space available per node with the time complexity achievable by a protocol. In particular, solving these tasks when each agents has $O(\log \log{n})$ memory states takes $\Omega(n/\polylog(n))$ time in expectation. On the other hand both tasks can be solved in polylogarithmic time with $O(\log(n)^2)$ memory states per agent. \cite{AlistarhAG18} shows a lower bound of $\Omega(\log{n})$ states for any consensus protocol that stabilizes in time $O(n^{1-c})$, for any constant $c > 0$. This result is complemented by a consensus protocol that uses $O(\log{n})$ states and stabilizes in $O(\log^{2}{n})$ time.

\cite{PVV09} studied the three state protocol and suggested a symmetric variant that converges faster, where both the initiator and responder update their states. 
Their paper shows the differential equations that characterize the three state protocol and inspired the approach in the present paper.
\cite{DraiefV12} studies the problem of exact consensus, where the protocol must converge to the correct final state with probability $1$ and the goal is bound the expected convergence time. Their work uses four memory states per agent to design a protocol that achieves consensus with probability $1$ from any starting configuration. %and shows an upper bound on the expected convergence time for all interaction graphs.

Plurality consensus is the problem where initially each node has a color chosen from a set $\{1, \ldots, k\}$ and one of the colors has an initial advantage. The goal is that all the nodes eventually adopt the color found in highest proportion initially; bounds on the convergence time were given in~\cite{BecchettiCNPS15,BecchettiCNPST17}.

In the leader election problem, all the agents start from the same state, and the goal is that they reach an outcome where exactly one agent is in a special leader state. The time and memory required for leader election in population protocols were studied in~\cite{DotyS18}, which showed a lower bound of $\Omega(n)$ on the time required to select a leader with probability $1$, when each agent has a constant amount of space. \cite{AlistarhG15} showed that  poly-logarithmic stabilization time can be achieved by allowing $O(\log^3{n})$ states rather than constant number of states per agent. \cite{SOIKM20} designed a protocol that runs in $O(\log{n})$ parallel time in expectation with $O(\log{n})$ states per agent. 
\cite{BerenbrinkGK20} designs and analyzes a population protocol that uses $\Theta(\log \log{n})$ states per agent, and elects a leader in $O(n \log{n})$ interactions in expectation. 

\cite{MSA19} studies the proportion computation problem, where each agent starts in one of two states $A$ and $B$, and the goal is that each agent learns the fraction of agents that initially started in state $A$.
\cite{CGKK+15} studies Lotka-Volterra type of dynamics in the population protocol model,
where agents have types and the update rule is such that when an agent $a$ of type  $i$ interacts with an agent $b$ of type $j$ with $ a$ as the initiator, then $b$’s type becomes $i$ with probability $P_{ij}$. This update rule is a simple variant of the Lotka-Volterra update rule and can be used to model opinion dynamics in social networks. The work in \cite{CGKK+15} shows that any such protocol converges in time polynomial in $n$ when the interaction graph is complete, while convergence time can be exponential when interaction is restricted (e.g. the graph is a star).


\section{Deterministic system} \label{sec:deterministic_system_def}

We start by studying the deterministic dynamical system obtained by taking $n \to \infty$ in the algorithm. Recall the variables defined earlier for the deterministic system. We introduce a few auxiliary variables and deduce the differential equations for the deterministic system.

\medskip 

\noindent \textbf{Notation}: Let $\Gamma(t)$ denote the fraction of informed followers at time $t$ $$\Gamma(t) = 1 - \frac{1}{s}- u(t) = \sum_{j=1}^{8s} \gamma_{j}(t)$$ and $R(t)$ the rate at which informed nodes depart each bin $j$ (see equation \ref{eq:def_beta_j_dot}) $$R(t) = 1 + \frac{1}{2s} - \frac{\delta(t)}{2} - u(t).$$

\begin{lemma} \label{lem:def_deterministic}
The deterministic system obtained by taking $n \to \infty$ in the algorithm can be described by the following differential equations:
\begin{align}
\dot{\alpha} & = - \frac{\alpha}{2} \cdot \left(\Gamma - \beta \right) + \delta \cdot \beta \label{alpha_dot} \\
\dot{\delta} & = \frac{\alpha}{2} \cdot \left(\Gamma - \beta \right) + \frac{1/s - \alpha - \delta}{2} \cdot \beta - \delta \cdot \Gamma \\
%\dot{\beta} & = u \cdot \beta + \frac{\alpha}{2} \cdot \left(\Gamma - \beta \right) - \beta_{c \cdot s} - \frac{1/s - \alpha - \delta}{2} \cdot \beta \\
\dot{u} & = \gamma_{c \cdot s} - u \cdot \Gamma \label{u_dot} \\
\label{eq:def_beta_j_dot}
\dot{\beta}_j & = \beta_{j-1} - \beta_j \cdot R, \; \; \forall j \in\{2, \ldots, 8s\}\\
\label{eq:def_beta_1_dot}
\dot{\beta}_1 & = -\beta_1 \cdot R + \frac{\alpha}{2} \cdot \Gamma \\
\dot{\gamma}_j & = \gamma_{j-1} - \gamma_j \cdot R, \; \; \forall j \in \{2, \ldots, 8s\} \label{gamma_j_dot} \\
\dot{\gamma}_1 & = - \gamma_1 \cdot R + \left(\frac{1}{2s} - \frac{\delta}{2}\right) \cdot \Gamma \label{gamma_1_dot}
\end{align}
\end{lemma}
\begin{proof}
	Note that the fraction of correct leaders is $1/s - \alpha(t) - \delta(t)$ and the fraction of correctly informed followers is $1 - 1/s - u(t) - \beta(t)$.
	
	The identity for $\dot{\alpha}$ holds since the first term in the update for $\dot{\alpha}(t) $ counts the expected number of leaders with the incorrect bit at time $t$ that pull from an informed node with the correct bit. The second term counts leaders with \q that pull from an informed node with the wrong bit.
	
	The identity for $\dot{\delta}$ holds since the first term counts wrong leaders that pulled from a correct informed node, the second term counts correct leaders that pulled from an incorrect informed node, and the last term counts leaders with \q that pulled from an informed node.
	
	%The identity for $\dot{\beta}$ holds since the first term represents uninformed nodes that pull from an incorrectly informed follower, the second term is due to leaders that push the incorrect bit to followers that are either uninformed or have the correct bit, the third term is due to followers whose informed status expires, the fourth term is due to leaders that push the correct bit to an informed follower with the wrong bit.
	
	For $\dot{\beta}_1$ we get the following identities, where the first term counts the followers with counter value $1$ that get a clock tick and increase their counter to 2, the second term counts the followers that get pushed a correct bit from the leaders (and so leave the set of followers with the wrong bit and counter value 1), the third term counts the followers that get pushed the wrong bit from leaders with incorrect information, and the last term counts uninformed followers that pull the wrong bit from followers with counter value 1. Thus
	\begin{align}
	\dot{\beta}_1 & = - \beta_1 - \frac{1/s - \delta - \alpha}{2} \cdot \beta_1 + \frac{\alpha}{2} \cdot \left( \Gamma - \beta_1 \right) + u \cdot \beta_1 \\
	& = -\beta_1 \cdot \Bigl( 1 + \frac{1}{2s} - \frac{\delta}{2} - u\Bigr) +  \frac{\alpha}{2} \cdot \Gamma \\
	& = -\beta_1 \cdot R + \frac{\alpha}{2} \cdot \Gamma
	\end{align}
	
	For $\dot{\beta}_j$, where $j \in \{2, \ldots, 8s\}$, we get the update rule
	\begin{align}
	\dot{\beta}_j & = \beta_{j-1} - \frac{1/s - \delta}{2} \cdot \beta_j + u \beta_j, \; \; \forall j \in\{2, \ldots, 8s\} \notag \\
	& = \beta_{j-1} - \beta_j \cdot R
	\end{align}
	
	For $\dot{\gamma}_j$, where $j \in \{2, \ldots, 8s\}$, we get the update rule
	\begin{align}
	\dot{\gamma}_j & = \gamma_{j-1} - \gamma_j \cdot \Bigl( 1 + \frac{1}{2s} -  \frac{\delta}{2} - u\Bigr) \notag \\
	& = \gamma_{j-1} - \gamma_j \cdot R
	\end{align}
	
	For $\dot{\gamma}_1$, we get the update rule
	\begin{align}
	\dot{\gamma}_1 & =  - \gamma_1 + \left(\frac{1}{2s} - \frac{\delta}{2} \right) \cdot \left( \Gamma - \gamma_1 \right) + u \cdot \gamma_1 \notag \\
	 &  = - \gamma_1 \cdot R + \left(\frac{1}{2s} - \frac{\delta}{2}\right) \cdot \Gamma
	\end{align}
\end{proof}

\begin{figure}[h!]
	\centering
	\subfigure[$\alpha$ (in red) and $\beta$ (in blue) over time. The values are normalized, so the plot shows in fact $\alpha(t) \cdot s$ and $\beta(t) \cdot s/(s-1)$.]
	{
		\includegraphics[scale=0.56]{alpha_beta_deterministic_c=0_49_s=5.png}
		\label{fig:alpha_beta_deterministic}
	}
	\subfigure[$\delta$ over time]
	{
		\includegraphics[scale = 0.56]{delta_deterministic_c=0_49_s=5.png}
		\label{fig:delta_deterministic}
	}
	\subfigure[$\gamma_j$ over time]
	{
		\includegraphics[scale=0.56]{gamma_deterministic_c=0_49_s=5.png}
		\label{fig:gamma_deterministic}
	}
	\subfigure[$\beta_j(t)$ over time]
	{
		\includegraphics[scale=0.56]{beta_deterministic_c=0_49_s=5.png}
		\label{fig:beta_j_deterministic}
	}
\subfigure[$\log \left(\frac{\alpha}{1/s - \delta} \right)$ and $\log\left(\frac{\beta_j(t)}{\gamma_j(t)}\right)$, for $j = \{1, \ldots, 8s\}$ over time. Note that $\log \left(\frac{\alpha}{1/s - \delta} \right)$ - shown in blue - is ahead of all the other variables in the plot.]
{
	\includegraphics[scale=0.7]{log_ratio_beta_gamma_j_deterministic_c=0_49_s=5.png}
	\label{fig:log_ratio_beta_gamma_j_deterministic}
}
	\caption{The deterministic system for $s=5$ and initial minority $49\%$ (that is, $\rho=0.02$).}
	\label{fig:alpha_beta_gamma_delta_deterministic}
\end{figure}


\begin{figure}[h!]
	\centering
	\subfigure[The advantage variables $\xi$ (in red) and $y$ (in blue) over time.]
	{
		\includegraphics[scale=0.56]{xi_y_deterministic_c=0_49_s=5.png}
		\label{fig:xi_y_deterministic}
	}
	\subfigure[The advantage variables $\eta_j$ over time, for $j \in \{2, \ldots, 8s\}$.]
	{
		\includegraphics[scale = 0.56]{eta_j_deterministic_c=0_49_s=5.png}
		\label{fig:eta_j_deterministic}
	}
	\caption{The deterministic system for $s=5$ and initial minority $49\%$ (that is, $\rho=0.02$).}
	\label{fig:xi_eta_y_deterministic}
\end{figure}

We show the variables that define the deterministic system are bounded from below. For $\alpha$, we will study the variable $w := 1/s - \delta - \alpha$, while for $\beta_j$ we study $w_j := \gamma_j - \beta_j$. Then $\dot{w}_j = w_{j-1} - w_j R$.

\begin{lemma} \label{lem:bounded_differential}
Suppose $0 < \alpha(0) < 1/s - \delta(0)$ and $0 < \beta_j(0) < \gamma_j(0)$ for all $j$.
The following functions have a strictly positive derivative
\begin{align} \label{eq:positive_derivatives}
g_{\alpha}(t) & = \alpha(t) \cdot e^{\frac{t}{2}(1-\frac{1}{s})}, \, \, g_{\delta}(t) = \delta(t)\cdot  e^{t(1-\frac{1}{s})} , \, \,
g_{\beta_j}(t) = \beta_j(t) \cdot e^{t(1+\frac{1}{2s})} , \, \,
g_{u}(t) = u(t) \cdot e^{t(1-\frac{1}{s})} \notag \\
g_{\gamma_j}(t) & =  \gamma_j(t) \cdot e^{t(1+\frac{1}{2s})} , \, \,
g_{w}(t) = w(t) \cdot e^{\frac{t}{2} (1 - \frac{1}{s})} , \, \,
g_{w_j}(t) = w_j(t) \cdot e^{t(1+\frac{1}{2s})}
\end{align}
In particular, for all $t > 0$, all the variables $\alpha(t),\delta(t),\beta_j(t),u(t),\gamma_j(t),w(t),w_j(t)$ are strictly positive  and $\alpha(t) > e^{\frac{-t}{2}(1-\frac{1}{s})} \cdot \alpha(0)$.
\end{lemma}
\begin{proof}
We will use the first violation method.

Let $t \geq 0$ be the first time at which the derivative of a function in (\ref{eq:positive_derivatives}) is non-positive. Then at time $t$, we have that $u(t),\delta(t) \geq 0$ and all the other variables are strictly positive, i.e.
\begin{align}  \label{eq:positive_vars}
\alpha(t) > 0, \delta(t) \geq 0, \beta_j(t) > 0, u(t) \geq 0, \gamma_j(t) > 0, \Gamma(t) > 0, w(t) > 0, w_j(t)>0
\end{align}
We will check that all the functions in (\ref{eq:positive_derivatives}) have strictly positive derivatives at time $t$, which will give a contradiction.

\medskip

\noindent \emph{Case 1}: Consider $g_{\alpha}'(t)$. Then by (\ref{eq:positive_vars}),
\begin{align}
g_{\alpha}'(t) & = e^{\frac{t}{2}(1 - \frac{1}{s})} \cdot \left[ \frac{\alpha(t)}{2} \left( u(t) + \beta(t)\right) + \delta(t) \beta(t) \right] > 0\notag
\end{align}

\medskip

\noindent \emph{Case 2}: Consider $g_{\delta}'(t)$. Then by (\ref{eq:positive_vars})
\begin{align}
g_{\delta}'(t) & = e^{t(1-\frac{1}{s})} \cdot \left[\alpha(t) \beta(t) + \frac{\alpha(t)}{2} \Gamma(t) + \left(\frac{1}{s} - \delta(t)\right) \frac{\beta(t)}{2} + \delta(t)u(t) \right] > 0, \notag
\end{align}
where we used the fact that
%$w(t) > 0$, so $1/s - \delta(t) - \alpha(t) > 0$, and thus
 $1/s - \delta(t)> w(t)> 0$.

\medskip

\noindent \emph{Case 3}: Consider $g_{\beta_1}'(t)$. Then by (\ref{eq:positive_vars})
\begin{align}
g_{\beta_1}'(t) & = e^{t(1+\frac{1}{2s})} \cdot \left[ \beta_1(t) \cdot \left(u(t) + \frac{\delta(t)}{2} \right) + \frac{\alpha(t)}{2} \Gamma(t) \right] > 0 \notag
\end{align}

\medskip

\noindent \emph{Case 4}: Consider $g_{\beta_j}'(t)$, for $j \geq 2$. Then by (\ref{eq:positive_vars})
\begin{align}
g_{\beta_j}'(t) & = e^{t(1+\frac{1}{2s})} \cdot \left[ \beta_{j-1}(t) + \beta_j(t) \cdot \left(u(t) + \frac{\delta(t)}{2} \right) \right] > 0\notag
\end{align}

\medskip

\noindent \emph{Case 5}: Consider $g_{u}'(t)$. Using the identity $1 - 1/s - \Gamma(t) = u(t)$ and (\ref{eq:positive_vars}),
\begin{align}
g_{u}'(t) & = e^{t(1 - \frac{1}{s})} \cdot \left[\gamma_{cs}(t) + u(t)^2 \right] > 0 \notag
\end{align}

\medskip

\noindent \emph{Case 6}: Consider $g_{\gamma_1}'(t)$. Since $w(t) > 0$, we get that $1/s - \delta(t) > 0$, so
\begin{align}
g_{\gamma_1}'(t) & = e^{t(1+\frac{1}{2s})} \cdot \left[ \gamma_1(t) \cdot \left(u(t) + \frac{\delta(t)}{2}\right) + \frac{\Gamma(t)}{2} \left(\frac{1}{s} - \delta(t)\right) \right] > 0 \notag
\end{align}

\noindent \emph{Case 7}: Consider $g_{\gamma_j}'(t)$ for $j \geq 2$. Then by (\ref{eq:positive_vars})
\begin{align}
g_{\gamma_j}'(t) &=  e^{t(1+\frac{1}{2s})} \cdot \left[ \gamma_{j-1}(t) + \gamma_j(t) \cdot \left(u(t) + \frac{\delta(t)}{2}\right) \right] > 0 \notag
\end{align}

\medskip

\noindent \emph{Case 8}: Consider $g_{w}'(t)$. Since $w_j > 0$ for all $j$, we obtain that $\Gamma(t) - \beta(t) > 0$. Moreover, $1 - 1/s - \beta(t) \geq 1 - 1/s - u(t) - \beta(t) = \Gamma(t) - \beta(t) > 0$. Then by (\ref{eq:positive_vars})
\begin{align}
g_{w}'(t) & = e^{\frac{t}{2}(1-\frac{1}{s})} \cdot \left[ \delta(t) \cdot \left(\Gamma(t) - \beta(t)\right) + \frac{w(t)}{2}\left(1 - \frac{1}{s} - \beta(t)\right) \right] > 0 \notag
\end{align}

\medskip

\noindent \emph{Case 9}: Consider $g_{w_j}'(t)$ for $j \geq 2$. Then by (\ref{eq:positive_vars})
\begin{align}
g_{w_j}'(t) & = e^{t(1 + \frac{1}{2s})} \cdot \left[ w_j(t) \cdot \left(u(t) + \frac{\delta(t)}{2} \right) + w_{j-1}(t) \right] > 0
\end{align}
\medskip

\noindent \emph{Case 10}: Consider $g_{w_1}'(t)$. Then at time $t$ it still is the case that $w_1(t) > 0$; all the other variables are strictly positive as well. Then by (\ref{eq:positive_vars})
\begin{align}
g_{w_1}'(t) &= e^{t(1 + \frac{1}{2s})} \cdot \left[  w_1(t) \cdot \left(u(t) + \frac{\delta(t)}{2}\right) + w(t) \cdot \frac{\Gamma(t)}{2} \right] > 0
\end{align}
Thus all the functions in the lemma statement must have strictly positive derivatives.
\end{proof}

Let $T > 0 $ be an upper bound on time. Define a domain large enough to contain all the values that the variables may take:
\begin{align} \label{def:domain_0}
D_0 = & \Bigl\{(\vec{y}) \mid \vec{y} =
\left(\alpha, \{\beta_j\}, \{\gamma_j\}, \delta)\}\right); 0 < \alpha < \frac{1}{s} ; 0 < \beta_j < \gamma_j ; \sum_{j=1}^{8s} \gamma_j < 1 ; \frac{-1}{s} < \delta < \frac{1}{s} \Bigr\}
\end{align}

We use the following existence theorem, adapted from Hurewicz~\cite{hurewitz_book}.

\medskip

\noindent \textbf{Theorem} [\cite{hurewitz_book}, Theorem 11, pp. 32].
\emph{If in some bounded $a$-dimensional domain $D_0$, the function $F:D_0 \to {\mathbb R}$ satisfies a Lipschitz condition, then the solution of $\frac{dy}{dt}=F(y)$ passing through any point of $D_0$ may be uniquely extended arbitrarily close to the boundary of $D_0$.}

\medskip

\begin{corollary} \label{cor:hurewicz}
Suppose $0 < \alpha(0) < 1/s - \delta(0)$ and $0 < \beta_j(0) < \gamma_j(0)$ for all $j$. Then the system of differential equations in (\ref{alpha_dot}--\ref{gamma_1_dot}) with these initial conditions has a unique solution for all times $t \geq 0$.
\end{corollary}
\begin{proof}
Equation~(\ref{u_dot}) follows from equations (\ref{gamma_j_dot}) and (\ref{gamma_1_dot}) using the identity $u(t) = 1 - 1/s - \Gamma(t)$.
Lemma~\ref{lem:bounded_differential} and the preceding theorem ensure that for any $T > 0$, the system (\ref{alpha_dot}--\ref{gamma_1_dot}) has a unique solution in $[0,T)$.
\end{proof}

\section{Phase I} \label{sec:phase_1}

We start by studying the advantage of the correct nodes over the incorrect ones in several categories (leaders, nodes with counter value $j$, for $j=1, \ldots, 8s$), which are tracked by the following variables:
\begin{itemize}
	\item $\xi = \frac{1/s - \delta - 2\alpha}{1/s - \delta}$: the normalized advantage of the correct leaders over the incorrect leaders at time $t$, counted among the group of leaders without a question mark.
	\item $\eta_j = \frac{\gamma_j - 2 \beta_j}{\gamma_j}$, for $j = 1, \ldots, 8s$: the advantage of informed nodes with counter value $j$ with the correct bit over the nodes with the incorrect bit and counter $j$ at time $t$.
	\item $\eta = \frac{\Gamma - 2\beta}{\Gamma}$: the normalized advantage of the informed followers with the correct bit over the ones with the incorrect bit. Note that $\eta$ is a convex combination of the $\eta_j$'s, thus it can alternatively be written as:
	\begin{align} \label{convex_y}
	\eta = \sum_{j=1}^{8s} \frac{\gamma_j \cdot \eta_j}{\Gamma }
	\end{align}
\end{itemize}

We will show in Proposition \ref{prop:phase1_progress} that the minimum of these advantages is increasing with a rate large enough to guarantee sufficient progress in the first $O(s)$ steps of the protocol.

\begin{lemma}
The derivatives of ${\xi}$, ${\eta}_j$, and ${y}$ are given by:
\begin{align}
\dot{\xi} & = \frac{\delta \cdot \Gamma}{1/s - \delta} \cdot \bigl(\eta - \xi \bigr) + \frac{\Gamma}{4} \cdot \bigl(1 - \xi^2 \bigr) \cdot \eta \label{xi_dot_def} \\
\dot{\eta_j} & =  \frac{\gamma_{j-1}}{\gamma_j} \cdot  \bigl(\eta_{j-1} - \eta_j\bigr), \; \forall j = 2, \ldots, 8s \; \; \label{eta_j_dot_def} \\
\dot{\eta_1}(t) & = \frac{\Gamma}{2 \gamma_1} \cdot (1/s - \delta) \cdot \bigl(\xi - \eta_1\bigr) \label{eta_1_dot_def}\\
%\dot{\eta} & = \frac{\bigl(1/s - \delta \bigr)}{2} \cdot \bigl(\xi - y\bigr) + \frac{\gamma_{8s}}{\Gamma} \cdot \bigl(y - \eta_{8s} \bigr) \label{y_dot_def}
\end{align}
\end{lemma}
\begin{proof}
By definition, ${\xi} = 1 - \frac{2\alpha}{1/s - \delta}$. Rewriting $\alpha$, $\beta$, and $\beta_j$ in terms of the advantage variables $\xi$, $\eta$, and $\eta_j$ gives:
\begin{align}
2 \beta_j & = (1 - \eta_j) \gamma_j , \; \; \forall j \in \{1, \ldots, 8s\}  \notag \\
2 \alpha & = (1 - \xi) \left(\frac{1}{s} - \delta\right) \; \mbox{ and } \;
2 \beta = (1 - \eta) \Gamma
\end{align}
Differentiating $\xi$ gives:
\begin{align}
\left(\frac{1}{s} - \delta\right)^2 \cdot \dot{\xi} & = - 2 \dot{\alpha} \cdot \left(\frac{1}{s} - \delta\right) + \left(\frac{1}{s} - \delta\right)' \cdot 2 \alpha \notag \\
& = \left[ \alpha(\Gamma - \beta) - 2 \delta \beta \right] \cdot \left(\frac{1}{s} - \delta\right) - \alpha \cdot \left[ \alpha(\Gamma - \beta) + \beta \left(\frac{1}{s} - \delta - \alpha\right) - 2 \delta \Gamma \right] \notag \\
%& = \alpha \left[ (\Gamma - 2 \beta) \cdot \left(\frac{1}{s} - \delta - \alpha \right) + 2 \delta \Gamma \right] - 2 \delta \beta \left(\frac{1}{s} - \delta\right)  \notag \\
& = \left(\frac{1}{s} - \delta\right) \cdot \left[ \frac{1 - \xi}{2} \left( \eta \Gamma \left(\frac{1}{s} - \delta\right) \frac{1 + \xi}{2} + 2 \delta \Gamma \right) - \delta(1 - \eta) \Gamma \right]
\end{align}
Thus
\begin{align}
\left(\frac{1}{s} - \delta\right) \dot{\xi} & = (1 - \xi) \delta \Gamma - \delta(1 - \eta) \Gamma + \frac{1 - \xi^2}{4} \eta \Gamma \left(\frac{1}{s} - \delta\right) \notag \\
& = \delta \Gamma ( \eta - \xi) + \frac{1 - \xi^2}{4} \eta \Gamma  \left(\frac{1}{s} - \delta\right)
\end{align}
Dividing both sides by $\left(\frac{1}{s} - \delta\right)$ gives
\begin{align} \label{xi_dot_final}
\dot{\xi} = \frac{\delta \Gamma}{\left(\frac{1}{s} - \delta\right)} (\eta - \xi) + \frac{1 - \xi^2}{4} \cdot \Gamma \eta
\end{align}
Differentiating $\eta_j$ gives:
\begin{align} \label{eta_j_dot_def_direct}
\gamma_j^2 \cdot \dot{\eta}_j = - 2 \dot{\beta}_j \cdot \gamma_j + 2 \beta_j \cdot \dot{\gamma}_j
\end{align}
For $j \geq 2$, equation (\ref{eta_j_dot_def_direct}) can be rewritten as
\begin{align}
\gamma_j^2 \cdot \dot{\eta}_j & = -2 \gamma_j \cdot (\beta_{j-1} - \beta_j \cdot R) + 2 \beta_j \cdot (\gamma_{j-1} - \gamma_j \cdot R) \notag \\
& = - 2 \gamma_j \cdot \beta_{j-1} + 2 \beta_j \cdot \gamma_{j-1} \notag \\
& = - \gamma_j \cdot (1 - \eta_{j-1}) \cdot \gamma_{j-1} + \gamma_{j-1} \cdot (1 - \eta_j) \cdot \gamma_j \notag \\
& = \gamma_j \cdot \gamma_{j-1} \cdot \Bigl( \eta_{j-1} - \eta_j \Bigr)
\end{align}
Therefore $\dot{\eta}_j = \frac{\gamma_{j-1}}{\gamma_j} \cdot \Bigl( \eta_{j-1} - \eta_j\Bigr)$ as required.

\medskip

For $\eta_1$, the derivative satisfies the identities
\begin{align}
\gamma_1^2 \cdot  \dot{\eta}_1 & = -2 \dot{\beta}_1  \gamma_1 + 2 \beta_1  \dot{\gamma}_1 \\
& = - \gamma_1 \cdot \bigl[ -2 \beta_1 \cdot R + \alpha \cdot \Gamma \bigr] + 2 \beta_1 \cdot \Bigl[ - \gamma_1 \cdot R + \left( \frac{1}{2s} - \frac{\delta}{2} \right) \Gamma \Bigr] \\
& = - \alpha \Gamma \cdot \gamma_1 + \beta_1 \Gamma \left(\frac{1}{s} - \delta \right) \\
& = \frac{\Gamma}{2} \cdot \Bigl[ -2 \alpha \gamma_1 + 2 \beta_1 \left(\frac{1}{s} - \delta \right)  \Bigr] \\
& = \frac{\Gamma}{2} \cdot \Bigl[ - \gamma_1 (1 - \xi) \left(\frac{1}{s} - \delta \right)  + (1 - \eta_1) \gamma_1 \left(\frac{1}{s} - \delta \right)  \Bigr]
\end{align}
Dividing by $\gamma_1$ gives
$$
\gamma_1 \cdot \dot{\eta}_1 = \frac{\Gamma}{2} \left(\frac{1}{s} - \delta \right) \cdot \Bigl[ - (1 - \xi) + (1 - \eta_1) \Bigr] \implies \dot{\eta}_1 = \frac{\Gamma}{2} \left(\frac{1}{s} - \delta \right) \cdot \Bigl[ \xi - \eta_1 \Bigr]
$$
For $\dot{\gamma}_1$, we get the update rule
\begin{align}
\dot{\gamma}_1 & = - \gamma_1 \cdot \left(1 + \frac{1}{2s} - \frac{\delta}{2} - u\right) + \left(\frac{1}{2s} - \frac{\delta}{2}\right) \cdot \Gamma \notag \\
& = - \gamma_1 \cdot R + \left(\frac{1}{2s} - \frac{\delta}{2}\right) \cdot \Gamma
\end{align}
\end{proof}


\begin{proposition} \label{prop:monotone_phi}
The function $\Phi$ given by
\begin{align} \label{phi_fun_basic}
\Phi = \min\Bigl\{  \xi, \eta_1, \ldots, \eta_{8s} \Bigr\}
\end{align}
is weakly increasing.
\end{proposition}
\begin{proof}
Note that $\eta_j(0) = \xi(0) = \rho$.
We consider a few cases, depending on which term in the definition of $\Phi$ realizes the minimum at time $t$. We have a few cases:
\begin{itemize}
	\item If $\Phi(t) = \xi(t)$, then $\eta(t) \geq \xi(t)$ by equation (\ref{convex_y}), so $\dot{\xi}(t) \geq 0$.
	\item If $\Phi(t) = \eta_j(t)$, for some $j \in \{2, \ldots, 8s\}$, then  $\eta_{j-1}(t) \geq \eta_j(t)$. By definition of $\dot{\eta_j}(t)$ this implies $\dot{\eta_j}(t) \geq 0$.
	\item If $\Phi(t) = \eta_1(t)$, then $\xi(t) \geq \eta_1(t)$. By definition of $\dot{\eta_1}(t)$ this implies $\dot{\eta_1}(t) \geq 0$.
\end{itemize}

\medskip

It follows that the right derivative of $\Phi(t)$ is non-negative, so $\Phi$ is non-decreasing, since $\Phi$ is a continuous function (see page 208, solved exercise 19 in \cite{hardy_book}).

% [\textcolor{blue}{TODO:} Add reference from here - https://math.stackexchange.com/questions/2047701/right-derivative-of-continuous-function-nonnegative-implies-increasing-function]
\end{proof}

\begin{proposition} \label{ub:delta}
The fraction of leaders with question mark is bounded by
$\delta(t) < 0.2/s$ for all $t$.
\end{proposition}
\begin{proof}
At the beginning of the protocol, $\delta(0) = 0$. Let $t$ be the minimum time at which $\delta(t) = 0.2/s$. We show that at this time, $\dot{\delta}(t) < 0$, and this will yield a contradiction.

First, rewrite $\dot{\delta}(t)$ as follows:
\begin{align} \label{delta_ub}
\dot{\delta}(t) & = \frac{\alpha(t)}{2} \cdot \left(\Gamma(t) - \beta(t) \right) + \frac{1/s - \alpha(t) - \delta(t)}{2} \cdot \beta(t) - \delta(t) \cdot \Gamma(t) \notag \\
& = \frac{\alpha(t)}{2} \cdot \left(\Gamma(t) - 2\beta(t) \right) + \frac{1/s - 0.2/s}{2} \cdot \beta(t) - \frac{0.2}{s} \cdot \Gamma(t) \notag \\
& = \Gamma(t) \cdot \left(\frac{\alpha(t)}{2} - \frac{0.2}{s}\right) - \left(\alpha(t) - \frac{0.4}{s}\right) \cdot \beta(t) \notag \\
& = \frac{\Gamma(t)- 2 \beta(t)}{2} \cdot \left(\alpha(t) - \frac{0.4}{s} \right)
\end{align}
Proposition~\ref{prop:monotone_phi} implies $\xi(t) \geq \rho$ and $\eta_j(t) \geq \rho$ at all times $t$. Then $\Gamma(t) - 2\beta(t) \geq 0$ and
\begin{align}
\xi(t) = \frac{1/s - \delta(t) - 2 \alpha(t)}{1/s - \delta(t)} = 1 -  \frac{2 \alpha(t)}{0.8/s} = 1 - \frac{s \cdot \alpha(t)}{0.4} \geq \rho \iff \alpha(t) - \frac{0.4}{s} \leq - \frac{0.4 \rho}{s} < 0
\end{align}
Replacing in (\ref{delta_ub}), we obtain $\dot{\delta}(t) < 0$. It follows that $\delta(t) < 0.2/s$ at all times $t$.
\end{proof}

\begin{lemma} \label{lem:R_lb}
Recall $R(t) = 1 + \frac{1}{2s} - \frac{\delta(t)}{2} - u(t)$.
If $s \geq 32$ and $u(t) \leq \frac{1}{se^2} \left(1 + \frac{3}{s}\right)$, then $R(t) \geq e^{\frac{1}{4s}}$.
\end{lemma}
\begin{proof}
Using the bound on $\delta(t)$ from Proposition~\ref{ub:delta} and the bound on $u(t)$ in the lemma statement, when $s \geq 32$ we get
\begin{align} %\label{eq:R_lb}
R(t) & \geq 1 + \frac{1}{2s} - \frac{0.1}{s} - \frac{e^{-2}}{s} \cdot \left( 1 + \frac{3}{s} \right) \notag \\
& = 1 + \frac{0.4}{s} - \frac{e^{-2}}{s} \cdot \left(1 + \frac{3}{s} \right) \notag \\
& > 1 + \frac{1}{4s} + \frac{1}{16s^2} > e^{\frac{1}{4s}} \notag
\end{align}
%%% > e^{\frac{1}{4s}}$ for $s \geq 30$.
The penultimate inequality is equivalent to $s > \frac{1/16 + 4 e^{-2}}{0.4 - e^{-2} - 0.25}$ and the last inequality follows since $e^x < 1 + x + x^2$ for $x < 1$.
\end{proof}

\begin{lemma} \label{ub:u_gamma}
Let $s \geq 32$ and $\epsilon \in (0, 1/2)$. If the following upper bounds hold for $\gamma_j$ and $u$ at time $t = t_0$, then they hold at all times $t \geq t_0$:
\begin{align}
\label{eq:ub_gammaj_phase1} \gamma_j(t) & < \frac{1 - \epsilon}{s} \cdot e^{-\frac{j}{4s}}, \; \forall j \in \{1, \ldots, 8s \}  \\
\label{eq:ub_u_phase2} u(t) & < \frac{1 - \epsilon}{s  e^{2}}  \left(1 + \frac{3}{s}\right)
\end{align}
\end{lemma}
\begin{proof}
The set of times $t$ at which of one or more of the inequalities in (\ref{eq:ub_gammaj_phase1}--\ref{eq:ub_u_phase2}) is violated is closed, so if this set is nonempty, it has a minimal element $t>t_0$.  We will derive a contradiction by establishing that the derivative of the first variable that violates the inequality at time $t$ is negative at that time. (The variables are ordered $\gamma_1,\ldots,\gamma_{8s},u$.)
	
\medskip

\noindent \emph{Case 1} : The first variable to violate the inequalities in (\ref{eq:ub_gammaj_phase1}--\ref{eq:ub_u_phase2}) is $\gamma_1$.
Using the bounds on $\delta(t)$ and $R(t)$, the derivative of $\gamma_1(t)$ at the time where $\gamma_1(t) = \frac{1-\epsilon}{s} \cdot e^{-\frac{1}{4s}}$ is
\begin{align}
\dot{\gamma}_1 & = - \gamma_1(t) \cdot R(t) + \left( \frac{1}{2s} - \frac{\delta(t)}{2} \right) \Gamma(t) \notag \\
& < - \frac{1-\epsilon}{s} \cdot e^{-\frac{1}{4s}} \cdot e^{\frac{1}{4s}} + \frac{1}{2s} < 0, \mbox{ since } \epsilon \in (0, 1/2)
\end{align}
Therefore if $\gamma_1(t)$ reaches the upper bound at time $t$, then its derivative at that time is strictly negative, which contradicts the definition of $t$ as the first time where the upper bound for $\gamma_1$ is reached.

\medskip

\noindent \emph{Case 2} : The first variable to violate the inequalities in (\ref{eq:ub_gammaj_phase1}--\ref{eq:ub_u_phase2}) is $\gamma_j$, for some $j \geq 2$.
Then
\begin{align} \label{ub_gamma_j_case1}
\dot{\gamma}_j(t) & = \gamma_{j-1}(t) - \gamma_j(t) \cdot R(t) \notag \\
& < \frac{1-\epsilon}{s} \cdot e^{-\frac{j-1}{4s}} - \frac{1-\epsilon}{s} \cdot e^{-\frac{j}{4s}} \cdot R(t) \notag \\
& = \frac{1-\epsilon}{s} \cdot e^{-\frac{j}{4s}} \cdot \Bigl[ e^{\frac{1}{4s}} - R(t) \Bigr]
\end{align}

Since (\ref{eq:ub_u_phase2}) holds before time $t$, the assumption of  Lemma~\ref{lem:R_lb} holds at time $t$, so $R(t) > e^{\frac{1}{4s}}$ when $s \geq 32$.
Thus $\dot{\gamma}_j(t) < 0$ by (\ref{ub_gamma_j_case1}), contradicting         the assumption of Case 2.

\medskip

\noindent \emph{Case 3} : The first variable to violate the inequalities in (\ref{eq:ub_gammaj_phase1}--\ref{eq:ub_u_phase2}) is $u$.
Then we get
\begin{align}
\dot{u}(t) & = \gamma_{8s}(t) - u(t) \cdot \Gamma(t) \notag \\
& < \frac{1-\epsilon}{se^{2}} \Bigl[ 1 - \left(1+\frac{3}{s}\right) \cdot \Gamma(t) \Bigr] \notag \\
& < \frac{1-\epsilon}{se^{2}} \Bigl[ 1 - \left(1+\frac{3}{s}\right)\cdot \left( 1 - \frac{2}{s} \right) \Bigr] \notag \\
& = \frac{1-\epsilon}{se^{2}} \Bigl[ 6/s^2-1/s  \Bigr] <0 \,, \mbox{ for } s > 6 \notag
\end{align}
This is in contradiction with the assumption of Case 3.

Thus there is no time $t$ at which one of the upper bounds in (\ref{eq:ub_gammaj_phase1}--\ref{eq:ub_u_phase2}) is violated.
\end{proof}

\begin{lemma} \label{gamma_ratio_lb}
For each $j \geq 2$, the ratio of consecutive $\gamma_j$'s is bounded as follows:
$$\frac{\gamma_{j-1}(t)}{\gamma_j(t)} > 1/2, \; \forall t \geq 0$$
\end{lemma}
\begin{proof}
	We use the method of first violation. Thus we look at the first time $t$ where the inequalities are violated; if multiple inequalities are violated at the same time $t$, then consider the smallest index $j$ with the property that $\gamma_{j-1}(t)/\gamma_j(t) = 1/2$.
If $j = 2$, then
\begin{align}
\left(\frac{\gamma_{1}(t)}{\gamma_{2}(t)} \right)' & = \frac{\gamma_2(t) \cdot \dot{\gamma}_1(t) - \gamma_1(t) \cdot \dot{\gamma}_2(t)}{\gamma_2(t)^2} \notag \\
& = \frac{\gamma_2(t) \cdot \Bigl(-\gamma_1(t) R(t) + \left(\frac{1}{2s} - \frac{\delta(t)}{2}\right)\Gamma(t)\Bigr) - \gamma_1(t) \Bigl(\gamma_1(t) - \gamma_2(t) R(t) \Bigr)}{\gamma_2(t)^2} \notag \\
& = \frac{\left(\frac{1}{2s} - \frac{\delta(t)}{2}\right)\Gamma(t)}{\gamma_2(t)} - \left( \frac{\gamma_1(t)}{\gamma_2(t)}\right)^2 \geq \frac{0.4}{s} \cdot \frac{\Gamma(t)}{\gamma_2(t)} - \left( \frac{\gamma_1(t)}{\gamma_2(t)}\right)^2 \notag \\
& \geq 0.4 \cdot \left(\frac{s-2}{s}\right) - \frac{1}{4} > 0
\end{align}
Thus the derivative of $\gamma_1/\gamma_2$ at time $t$ is strictly positive, which means that the value of $\gamma_1/\gamma_2$ couldn't have reached $1/2$ at time $t$. Since $t$ was the time of the first violation, this means there is no violation.

If
$j \geq 3$, then
\begin{align}
\left(\frac{\gamma_{j-1}(t)}{\gamma_j(t)} \right)' & = \frac{\gamma_{j}(t) \cdot \dot{\gamma}_{j-1}(t) - \gamma_{j-1}(t) \cdot \dot{\gamma}_j(t)}{\gamma_j(t)^2} \notag \\
& = \frac{\gamma_j(t)  \Bigl( \gamma_{j-2}(t) - \gamma_{j-1}(t) \cdot R(t)\Bigr)}{\gamma_j(t)^2} - \frac{\gamma_{j-1}(t)  \Bigl( \gamma_{j-1}(t) - \gamma_{j}(t) \cdot R(t)\Bigr)}{\gamma_j(t)^2}  \notag \\
& = \frac{\gamma_{j-1}(t)}{\gamma_j(t)} \cdot \Bigl[ \frac{\gamma_{j-2}(t)}{\gamma_{j-1}(t)} - \frac{\gamma_{j-1}(t)}{\gamma_j(t)} \Bigr]
\end{align}
When $\gamma_{j-1}(t) / \gamma_j(t) = 1/2$, we have
\begin{align}
\left(\frac{\gamma_{j-1}(t)}{\gamma_j(t)} \right)' & = \frac{\gamma_{j-1}(t)}{\gamma_j(t)} \cdot \Bigl[ \frac{\gamma_{j-2}(t)}{\gamma_{j-1}(t)} - \frac{\gamma_{j-1}(t)}{\gamma_j(t)} \Bigr] \notag \\
& = \frac{1}{2} \cdot \Bigl( \frac{\gamma_{j-2}(t)}{\gamma_{j-1}(t)} - \frac{1}{2} \Bigr) > 0
\end{align}
This means that the value of $\gamma_{j-1}/\gamma_j$ could not have dropped to $1/2$ at time $t$, in contradiction with $t$ being the time of the first violation.
It follows that $\gamma_{j-1}(t)/\gamma_j(t) > 1/2$ for all $j \geq 2$ at all times $t$.
\end{proof}

\medskip

By definition of $\Gamma$ and the upper bound on $u$ in (\ref{ub:u_gamma}),
\begin{align} \label{eq:Gamma_lb_ub}
1 - \frac{2}{s} < \Gamma(t) \leq 1 - \frac{1}{s}\,.
\end{align}

\begin{proposition} \label{prop:phase1_progress}
Let $\lambda \in (0, 1)$. There exists $C = C(\lambda, \rho) \leq \frac{144}{\rho (1 - \lambda)}$ and a time $T_{1} < C \cdot s$ so that $\xi(T_{1}) > \lambda$ and $\eta_j(T_{1}) > \lambda$ for all $j$. Consequently, at all times  $t \geq T_{1}$, we have $2 \beta(t) < 1 - \lambda$.
\end{proposition}
\begin{proof}
	From the expressions for the derivatives of $\eta_j$ and $\xi$, at a high level it can be observed that each $\eta_{j}$ follows $\eta_{j-1}$ when $j \geq 2$, $\eta_1$ follows $\xi$, and $\xi$ is pulled towards $\eta$ but in addition has a positive drift given by the term $\frac{\Gamma}{4} \eta (1 - \xi^2)$. This motivates the following definition:
	\begin{align} \label{psi_fun}
	\Psi(t) = \min\Bigl\{ \xi(t), \eta_1(t) + \epsilon \cdot v_1, \ldots, \eta_{8s}(t) + \epsilon \cdot v_{8s} \Bigr\},
	\end{align}
	where $\epsilon, v_1, \ldots, v_{8s}$ are free parameters that will be set later, where $0 < v_1 < \ldots < v_{8s}$. Note $\Psi(0) = \rho$.
	
	Let $\lambda_1 = (1 + \lambda)/2$ and $T_1 = T_1(\lambda) = \min\{t > 0 \; : \; \Psi(t) = \lambda_1\}$. The goal is to bound from below the right derivative $\Psi_{+}'(t)$ for $t < T_{1}$.
	
	%Let $\epsilon = \frac{\rho}{8} (1 - \lambda^2)$.
\medskip
	
	We consider a few cases, depending on which term in the definition of $\Psi$ realizes the minimum at time $t$:
	
	\smallskip
	
	\noindent \textbf{\emph{Case 1}}. The minimum is $\xi(t)$, that is, $\Psi(t) = \xi(t) < \lambda_1$. Then $\eta_j(t) + \epsilon v_j \geq \xi(t)$, so $\eta_j(t) \geq \xi(t) - \epsilon v_{8s}$ and $\eta(t) \geq \xi(t) - \epsilon v_{8s}$. Then
	\begin{align} \label{psi_plus_lb_case1}
	\Psi_{+}'(t) & \geq \dot{\xi}(t) = \frac{\delta(t)\Gamma(t)}{1/s - \delta(t)} \cdot \Bigl[ \eta(t) - \xi(t) \Bigr] + \frac{\Gamma(t)}{4} \left( 1 - \xi(t)^2 \right) \eta(t) \notag \\
	& \geq \frac{0.2/s}{0.8/s} \cdot (-\epsilon v_{8s}) + \frac{1}{8}(1 - \lambda_1^2)\rho
	\end{align}
	Let $\epsilon = \frac{\rho}{8}(1 - \lambda_1^2)$. Then for $v_{8s}=2$, the inequality in (\ref{psi_plus_lb_case1}) implies
	\begin{align}
	\Psi_{+}'(t) & \geq \epsilon \left( 1 - \frac{v_{8s}}{4}\right) = \frac{\epsilon}{2}
	\end{align}
	
	\noindent \textbf{\emph{Case 2}}. The minimum is $\Psi(t) = \eta_j(t) + \epsilon v_j$ for some $j \in \{2, \ldots, 8s\}$, so $\eta_j + \epsilon v_j \leq \eta_{j-1} + \epsilon v_{j-1}$.
	The right derivative of $\Psi$ is lower bounded by
	\begin{align}
	\Psi_{+}'(t) & \geq \dot{\eta}_j(t) = \frac{\gamma_{j-1}(t)}{\gamma_j(t)} \cdot \Bigl[ \eta_{j-1}(t) - \eta_j(t)\Bigr] \geq \frac{\epsilon}{2} \left( v_j - v_{j-1} \right)
	\end{align}	
	
	\noindent \textbf{\emph{Case 3}}. The minimum is $\Psi(t) = \eta_1(t) + \epsilon v_1$. Then the right derivative of $\Psi$ is lower bounded by
	\begin{align}
	\Psi_{+}'(t) & \geq \dot{\eta}_1(t) = \frac{\Gamma(t)}{2 \gamma_1(t)} \cdot \Bigl[ \frac{1}{s} - \delta(t) \Bigr] \left( \xi(t) - \eta_1(t) \right) \geq \frac{1 - 2/s}{2/s} \cdot \frac{0.8}{s} \cdot \epsilon v_1 \notag \\
	& \geq \frac{s}{4} \cdot \frac{0.8}{s} \cdot \epsilon v_1  = \frac{\epsilon v_1}{5}
	\end{align}
	
	By setting the $v_j$'s to satisfy the lower bounds with equality, i.e. $$
	\frac{v_1}{5} = \frac{v_2 - v_1}{2} = \ldots = \frac{v_{8s} - v_{8s-1}}{2}
	$$
we get $v_j  = \frac{2(3+ 2j)}{3+16s}$ for each $j \geq 1$.
\medskip
	
\noindent From cases $(1-3)$, it follows that at all times $t < T_{1}$, the right derivative $\Psi_{+}'(t) $ is large, i.e. it is bounded from below by $\epsilon/(9s)$. Then
	$$
	T_{1} \leq \frac{\lambda_1 - \rho}{(\epsilon/(9s))} \leq \frac{9s}{\epsilon}
	$$
	Moreover, at any time $t \geq T_{1}$,
	$$
	\min\{\xi(t), \eta_1(t), \ldots, \eta_{8s}(t)\} \geq \lambda_1 - 2 \epsilon > \lambda
	$$
	Taking
	$$
	C = C(\lambda, \rho) = 9/\epsilon = \frac{72}{\rho \cdot \left(1 - \left(\frac{1 + \lambda}{2}\right)^2\right)} \leq \frac{144}{\rho (1 - \lambda)}
	$$
	works. Moreover, since $y$ is a weighted average of $\eta_j$, for all $t > T_1$ we have
	$$\lambda < \eta(t) = \frac{\Gamma(t) - 2 \beta(t)}{\Gamma(t)} = 1 - \frac{2\beta(t)}{\Gamma(t)}\, .$$
	Therefore for all $t > T_1$ we have
	$
	2 \beta(t) < \Gamma(t)  \cdot (1 - \lambda) < 1 - \lambda
	$
as required.
\end{proof}

\section{Phases II and III: Exponential Decay of Errors} \label{sec:phases_2and3}

We divide the time after phase I in two phases, II and III. During phase II, we bound the fraction of wrong nodes by $\exp(-c_2 \cdot t/s)$ for some constant $c_2 > 0$. During phase III, we bound the fraction of wrong nodes by $\exp(-c_3 \cdot t)$ for some constant $c_3 > 0$.
Both of these will be handled in a similar way, by considering a potential function that is a linear combination of $\alpha$, $\delta$, and $\beta_j$ for $j \in \{1, \ldots, 8s+1\}$, with different coefficients for each of the two phases. The variable $\beta_{8s+1}(t)$ represents the fraction of nodes that are uninformed and have the wrong bit at time $t$. The derivative of $\beta_{8s+1}$ is
\begin{align} \label{eq:derivative_beta_8s1}
\dot{\beta}_{8s+1} = \beta_{8s} - \beta_{8s+1}\cdot \Gamma
\end{align}

The existence and uniqueness theorem (Corollary~\ref{cor:hurewicz}) applies when equation (\ref{eq:derivative_beta_8s1}) is added to the system of differential equations (\ref{alpha_dot}--\ref{gamma_1_dot}).

\subsection{Linear Potential Function}

Define
\begin{align} \label{def:Lambda}
\Lambda(t) = \alpha(t) + d \cdot \delta(t) + \sum_{j=1}^{8s+1} a_j \cdot \beta_j(t),
\end{align}
where $d, a_1, \ldots, a_{8s+1} \geq 0 $ have to be determined. We will require $\Lambda$ to satisfy the inequality $\dot{\Lambda}(t) \leq -\zeta \cdot \Lambda(t)$ for some constant $\zeta > 0$ and each time $t$ in the range of phases $II$ and $III$, respectively.

Bounding $\dot{\Lambda}$ using equations (\ref{alpha_dot}-\ref{eq:def_beta_1_dot}) and the inequality $\Gamma \leq 1$, we get
\begin{small}
\begin{align} %\label{eq:ub_psi_dot}
\dot{\Lambda}(t) & \leq - \frac{\alpha(t)}{2} \cdot \Bigl(\Gamma(t) - \beta(t)\Bigr) + \delta(t) \beta(t) + d \cdot \frac{\alpha(t)}{2} \cdot \Gamma(t) + \frac{ d \beta(t)}{2s} - \frac{d \delta(t)}{2}  \notag \\
& \; \; \; \; + a_1  \Bigl[ -\beta_1(t)  R(t) + \frac{\alpha(t)}{2} \Bigr] + \sum_{j=2}^{8s} a_j  \Bigl[\beta_{j-1}(t) - \beta_j(t)  R(t) \Bigr] + a_{8s+1}   \Bigl[\beta_{8s}(t) - \beta_{8s+1}(t) \Gamma(t) \Bigr] \notag \\
& = \frac{\alpha(t)}{2}  \Bigl[ {\beta(t)} - 1 + \frac{2}{s} + {d} + {a_1} \Bigr] + \delta(t)  \Bigl[ \beta(t) - \frac{d}{2} \Bigr] + \sum_{j=1}^{8s} \beta_j(t)  \Bigl[\frac{d}{2s} - a_j  R(t) + a_{j+1} \Bigr] \notag
\end{align}
\end{small}

To ensure that $\dot{\Lambda}(t) \leq - \zeta \cdot \Lambda(t) = - \zeta \cdot \Bigl[ \alpha(t) + d \cdot \delta(t) + \sum_{j=1}^{8s} a_j \cdot \beta_j(t) \Bigr]$, it suffices that
%\begin{align}
%-z \Lambda(t) =  -z \cdot \Bigl[ \alpha(t) + d \cdot \delta(t) + %\sum_{j=1}^{8s} a_j \cdot \beta_j(t) \Bigr]
%\end{align}
%For (\ref{eq:ub_psi_dot}) to hold, it suffices that the following constraints are met
\begin{align} \label{eq:phase2_constraints}
(a) \; \; \; & \frac{d}{2s} - a_j \cdot R(t) + a_{j+1} \leq - \zeta \cdot a_j, \; \forall j \leq 8s  \notag \\
& \iff a_{j+1} \leq a_j \cdot \Bigl(R(t) - \zeta \Bigr) - \frac{d}{2s}\, , \; \forall j \leq 8s \notag \\
(b) \; \; \; & \beta(t) - \frac{d}{2}  \leq - \zeta \cdot d  \iff \beta(t) \leq \left( \frac{1}{2} - \zeta \right) d \notag\\
(c) \; \; \; & \frac{\beta(t)}{2} - \frac{1 - 2/s}{2} + \frac{d}{2} + \frac{a_1}{2} \leq - \zeta \iff 2 \zeta + \beta(t) + d + a_1 \leq 1 - \frac{2}{s} \notag \\
(d) \; \; \; & - a_{8s+1} \cdot \Gamma \leq - a_{8s+1} \cdot \zeta
\end{align}

\subsection{Phase II} \label{sec:phase_2}

Let $\Lambda_{2}$ be the potential function for phase II, following the template in (\ref{def:Lambda}). The goal is to set the coefficients in $\Lambda_{2}$ so that  $\dot{\Lambda}_{2}(t) \leq -\zeta_{2} \cdot \Lambda_{2}(t)$, for some constant $\zeta_{2}> 0$.

Recall that at the end of Phase I we obtain that $\beta(t) \leq 1/64$ for $t \geq T_1$ when $\lambda = 31/32$ in Proposition~\ref{prop:phase1_progress}. Then there exists $C_1 > 0$ such that
\begin{align}  %\label{eq:ub_T1_precise}
T_1 \leq \frac{C_1 s}{\rho}, \mbox{ where } C_1 \leq 144 \cdot 32\,.
\end{align}
Also $R(t) \geq e^{\frac{1}{4s}} \geq 1 + \frac{1}{4s}$ from Lemma \ref{lem:R_lb} and $\Gamma(t) \geq 1 - 2/s$ from (\ref{eq:Gamma_lb_ub}).

\medskip

Set $a_j= a$, $\forall j \in \{1, \ldots, 8s\}$, and $a_{8s+1} = 0$, so part (d) of (\ref{eq:phase2_constraints}) holds.

For part (a) of (\ref{eq:phase2_constraints}) to hold, it suffices that $\frac{d}{2s} \leq a \left(\frac{1}{4s} - \zeta_{2} \right)$.
Let $\zeta_{2} = \frac{1}{8s}$ and $d = a/4$.

For part (b) of (\ref{eq:phase2_constraints}) to hold, it suffices that
$$
\beta(t) \leq \frac{d}{2} \cdot \left( 1 - \frac{1}{4s} \right)
$$
This holds when $\beta(t) \leq a/16$.

Finally, for part (c) of (\ref{eq:phase2_constraints}) to hold, it suffices that
$$
\beta(t) \leq \frac{1}{2} - \frac{5a}{4} - \frac{1}{4s}
$$
Let $a = 1/4$. Then the requirement on $\beta$ is $\beta(t) \leq 1/64$, which holds for all $t \geq T_1$.
Thus
\begin{align} \label{eq:lambda2_definition_exact_coeffs}
\Lambda_2(t) = \alpha(t) + \frac{\delta(t)}{16} + \frac{\beta(t)}{4}
\end{align}
It follows that $\dot{\Lambda}_{2}(t) \leq - \zeta_{2} \cdot \Lambda_{2}(t) = - \frac{\Lambda_{2}(t)}{8s}$, for all $t \geq T_1$.
\begin{align} \label{eq:lambda2_beyond_T1}
\Lambda_{2}(t) \leq \Lambda_{2}(T_1) \cdot \exp\left(-\frac{t - T_1}{8s}\right), \mbox{ for all } t \geq T_1\,.
\end{align}
Also note that $\Lambda_{2}(T_1) \leq 1$. We deduce that for $t \geq T_1$ and $j \leq 8s$
\begin{align}
%\alpha(t) & \leq  e^{-\frac{t}{8s}}  \\
%d \cdot \delta(t)& = \frac{\delta(t)}{16} \leq \Lambda(T_1) \cdot e^{-\frac{t}{8s}} \implies \delta(t) \leq 16 \cdot e^{-\frac{t}{8s}} \\
a_j \cdot \beta_j(t) & = \frac{\beta_j(t)}{4} \leq \Lambda_{2}(t)
%\Lambda(T_1) \cdot \exp\left(-\frac{t-T_1}{8s} \right)
\implies \beta_j(t) \leq 4 \cdot \exp\Bigl(-\frac{t-T_1}{8s}\Bigr)
\end{align}
Thus $\beta$ can be bounded by
\begin{align}
\beta(t) = \sum_{j=1}^{8s} \beta_j(t) \leq  32s \cdot \exp\Bigl(-\frac{t-T_1}{8s}\Bigr)
\end{align}
In particular, if we take $T_2 = T_1 + 64 s^2$, then for all $t \geq T_2$
\begin{align} \label{eq:beta_ub_end_phase2}
\beta(t) \leq 32 s \cdot \exp\Bigl( -8s \Bigr)
\end{align}

\subsection{Phase III} \label{sec:phase_3}

Let $\Lambda_{3}$ be the potential function for phase III, following the template in (\ref{def:Lambda}). The goal is to set the coefficients in $\Lambda_{3}$ so that $\dot{\Lambda}_{3}(t) \leq -\zeta_{3} \cdot \Lambda_{3}(t)$, for some constant $\zeta_{3}> 0$.
The constraints needed to ensure that $\dot{{\Lambda}}_{3}(t) \leq - \zeta_{3} \cdot {\Lambda}_{3}(t)$ for $t \geq T_2$ are given in (\ref{eq:phase2_constraints}).

To satisfy part (a) of (\ref{eq:phase2_constraints}), set $\zeta_{3} = 1/2 - 2/s$ and ${a}_1 = 1/s$, and require that $\beta(t) + {d} \leq 1/s$. This also ensures that constraint $(d)$ is met.

\medskip

Part (b) of (\ref{eq:phase2_constraints}) is satisfied when $\beta(t) \leq 2 {d}/s$. Part (c) of (\ref{eq:phase2_constraints}) is satisfied when $ {d} \leq  {a}_j$ and $ {a}_{j+1} =  {a}_j/2$, since these imply
$$
{a}_{j+1} + \frac{ {d}}{2s} \leq  {a}_j \cdot \left( \frac{1}{2} + \frac{1}{2s} \right) =  {a}_j \cdot (1 - \zeta_{3})
$$
Thus let
\begin{align}
{a}_j & =  {a}_1 \cdot 2^{1-j} = \frac{1}{s} \cdot 2^{1-j} \notag \\
 {d} & = {a}_{8s} = \frac{1}{s} \cdot 2^{1-8s} \notag \\
\end{align}
These require that $\beta(t) \leq 2^{2-8s}/s^2$, which holds by (\ref{eq:beta_ub_end_phase2}) for all $t \geq T_2$.

\medskip

Recall $T_2 = O(s^2)$. For any $\theta > 0$, let $T_3 = T_2 + \frac{1 + \theta}{\zeta_{3}} \ln{n}$. Then $\Lambda_{3}(T_2) \leq 1$ implies
\begin{align}
\Lambda_{3}(T_3) \leq \exp \Bigl( -\zeta_{3} \cdot (T_3 - T_2) \Bigr) = n^{-1 - \theta}
\end{align}

Thus for all $j \leq 8s+1$,
$$\beta_j(T_3) \leq \Lambda_{3}(T_3) /  {a}_j \leq s \cdot  2^{j-1} n^{-1 - \theta}\,.$$

This implies that
\begin{align}
\alpha(T_3) + \beta(T_3) + \beta_{8s+1}(T_3) \leq n^{-1-\theta} \cdot \Bigl( 1 + s \cdot 2^{8s+1} \Bigr) = o\left(\frac{1}{n}\right)
\end{align}

\section{Random System}

Let $\widetilde{\alpha}(j) = $ number of leaders after $j$ rings in the finite system. Similarly, define variables $\widetilde{\delta}(j), \widetilde{\gamma}_1(j), \ldots, \widetilde{\gamma}_{8s}(j), \widetilde{\beta}_1(j), \ldots,  \widetilde{\beta}_{8s+1}(j)$. Also define
\begin{align}
\widetilde{\Gamma}(j) & = \sum_{i=1}^{8s} \widetilde{\gamma}_i(j) \notag \\
\widetilde{u}(j) & = n - \frac{n}{s} - \widetilde{\Gamma}(j) \notag \\
\widetilde{R}(j) & = n + \frac{n}{2s} - \frac{\widetilde{\delta}}{2} - \widetilde{u}(j) \notag
\end{align}

\medskip

Our goal is to show that the random system is closely approximated by the deterministic system analyzed in sections \ref{sec:deterministic_system_def}--\ref{sec:phases_2and3}. A key tool will be the following theorem by Warnke~\cite{warnke}, which refines earlier results by Kurtz~\cite{kurtz} and Wormald~\cite{wormald}. We include the main theorem in ~\cite{warnke} for the special case where the derivatives do not depend on time.

\begin{theorem}[\cite{warnke}]\label{thm:DEM}%
	Given integers~$a,n \ge 1$ and a bounded domain~$\mathcal{D} \subseteq \mathbb{R}^{a}$, let $F_k~:~\mathcal{D}~\to~\mathbb{R}$ be $L$-Lipschitz-continuous functions \footnote{With respect to the $\ell^{\infty}$ norm on $\mathbb{R}^a$.} on~$\mathcal{D}$ and $M=\sup\{F_k(y) \mid k \in [a], y \in \mathcal{D}\}$.
	Suppose $\{\mathcal{F}_i\}_{i \ge 0}$ are increasing $\sigma$-fields and
	$(Y_k(i))_{k=1}^{a}$ are $\mathcal{F}_i$-measurable random variables for all $i \ge 0$ with $Y_k(0)=n \cdot {y}_k^* $ where~$({y}_1^*, \ldots,  {y}_a^*) \in \mathcal{D}$. Moreover, assume that for all $i \geq 0$ and $1 \le k \le a$, the following conditions hold whenever~$\bigl(\frac{Y_1(i)}{n},...,\frac{Y_a(i)}{n}\bigr) \in \mathcal{D}$:
	%
	\vspace{-0.25em}%
	\begin{enumerate}%
		\itemsep 0.125em \parskip 0em  \partopsep=0pt \parsep 0em
		\item[(i)]\label{dem:trend}%
		$\Bigl|\mathbb{E}\bigl[Y_k(i+1)-Y_k(i) \mid \mathcal{F}_{i}\bigr]-F_k\bigpar{\frac{Y_1(i)}{n},...,\frac{Y_a(i)}{n}} \Bigr| \le \lambda_0$,
		\item[(ii)]\label{dem:bounded}%
		$\bigabs{Y_k(i+1)-Y_k(i)}\le 1$. %, and
		%\item[(iii)]\label{dem:init}%
		 \end{enumerate}\vspace{-0.125em}%

	Let $T>0$ and $\lambda \ge \lambda_0 \min\{T,L^{-1}\} + M/n$. Let $(y_k(t))_{1 \le k \le a}$ be the unique solution for $t\le T$ of the system of differential equations
	\begin{equation}\label{dem:sol}
	y'_k(t) =F_k\bigpar{y_1(t), \ldots, y_a(t)} \quad \text{ with } \quad y_k(0) = {y}_k^* \qquad \text{for~$1 \le k \le a\,.$}
	\end{equation}
	 If $(y_1(t), \ldots y_a(t))$ has~$\ell^{\infty}$-distance at least~$3 e^{L T} \lambda$ from the boundary~of~$\mathcal{D}$ for all~$t \in [0,T]$, then
with probability at least $1-2a \cdot \exp(\frac{-n\lambda^2}{8T})$, we have
	\begin{equation}\label{dem:error}
	\max_{0 \le i \le T n} \max_{1 \le k \le a}\bigabs{Y_k(i)-n \cdot y_k\bigpar{\tfrac{i}{n}}} \; \leq \; 3 e^{L T} \lambda n \,.
	\end{equation}
\end{theorem}

\medskip

\begin{remark} \label{rmk:det_derivatives_map}
	Let $y = \left(\alpha, \beta_1, \ldots, \beta_{8s+1}, \gamma_1, \ldots, \gamma_{8s}, \delta\right) \in [0,1]^{16s+3}.$
	Recall $u = 1 - \frac{1}{s} - \sum_{j=1}^{8s} \gamma_j$, $\Gamma = \sum_{j=1}^{8s} \gamma_{j}$ and $R = 1 + \frac{1}{2s} - \frac{\delta}{2} - u.$
	Then Lemma~\ref{lem:def_deterministic} can be summarized by describing functions that give the time derivatives of the coordinates of $y$:
	\begin{align}
	& F_{\alpha}(y) = - \frac{\alpha}{2} \cdot \left(\Gamma - \beta \right) + \delta \cdot \beta \label{F_alpha} \\
	& F_{\delta}(y) = \frac{\alpha}{2} \cdot \left(\Gamma - \beta \right) + \frac{1/s - \alpha - \delta}{2} \cdot \beta - \delta \cdot \Gamma \\
	\label{F_beta_j}
	& F_{\beta_j}(y) = \beta_{j-1} - \beta_j \cdot R, \; \; \forall j \in\{2, \ldots, 8s\}\\
	\label{F_beta_1}
&	F_{\beta_1}(y) = -\beta_1 \cdot R + \frac{\alpha}{2} \cdot \Gamma \\
	\label{F_beta_8s1}
	& F_{\beta_{8s+1}}(y) = \beta_{8s} - \beta_{8s+1}\cdot \Gamma \\
	& F_{\gamma_j}(y) = \gamma_{j-1} - \gamma_j \cdot R, \; \; \forall j \in \{2, \ldots, 8s\} \label{F_gamma_j} \\
	& F_{\gamma_1}(y) = - \gamma_1 \cdot R + \left(\frac{1}{2s} - \frac{\delta}{2}\right) \cdot \Gamma \label{F_gamma_1}
	\end{align}
\end{remark}

In the discrete system, the derivatives are replaced by the differences in the values of variables at time $i+1$ and $i$, for each $i \in \mathbb{N}$.
Define $$\widetilde{{Y}}(i) = \left(\widetilde{\alpha}(i), \widetilde{\beta}_1(i), \ldots,  \widetilde{\beta}_{8s+1}(i),  \widetilde{\gamma}_1(i), \ldots, \widetilde{\gamma}_{8s}(i),\widetilde{\delta}(i)\right).$$

Let $\mathcal{F}_i$ be the history after $i$ clock rings for the process $\widetilde{Y}$. The next lemma verifies condition (i) from Theorem~\ref{thm:DEM}.

\begin{lemma} \label{lem:change_expectation_Y}
The expected change in $\widetilde{\alpha}$ is approximated by
\begin{align} \label{eq:exp_change_F_alpha}
\Bigl|\Ex \left[\widetilde{\alpha}(i+1) - \widetilde{\alpha}(i) \mid \mathcal{F}_i \right]  - F_{\alpha}({\widetilde{Y}(i)}/{n}) \Bigr| \leq \frac{1}{n}
\end{align}
More generally, for each $k = 1, \ldots 16s+3$,
\begin{align} \label{eq:expected_change_Y_tilde}
\Bigl| \Ex \left[\widetilde{Y}_k(i+1) - \widetilde{Y}_k(i) \mid \mathcal{F}_i \right]  - F_{{y}_k}({\widetilde{Y}(i)}/{n}) \Bigr| \leq \frac{1}{n}
\end{align}
\end{lemma}
\begin{proof}
%The expected value of $\widetilde{Y}(i+1)$ given $\widetilde{Y}(i)$ is as follows
The expected change in the number of leaders with the incorrect bit is given by
\begin{align} \label{eq:alpha_1step}
\Ex\left[\widetilde{\alpha}(i+1) - \widetilde{\alpha}(i) \mid \mathcal{F}_i \right] & = - \frac{\widetilde{\alpha}(i)}{2n} \cdot \frac{\widetilde{\Gamma}(i)-\widetilde{\beta}(i)}{n-1} + \frac{\widetilde{\delta}(i)}{n} \cdot \frac{\widetilde{\beta}(i)}{n-1}
\end{align}
The first term in the update (\ref{eq:alpha_1step}) represents the probability that the node selected at step $i+1$ is a leader with the incorrect bit that pulls from an informed node with the correct bit. The second term is the probability that the node selected is a leader with \q that pulls from an informed node with the wrong bit.

\medskip

On the other hand, by Remark~\ref{rmk:det_derivatives_map}
\begin{align} \label{eq:alpha_1step_Ytilde}
F_{\alpha}(\widetilde{Y}(i)/n) = - \frac{\widetilde{\alpha}(i)}{2n} \cdot \frac{\widetilde{\Gamma}(i) - \widetilde{\beta}(i)}{n} + \frac{\widetilde{\delta}(i)}{n} \cdot \frac{\widetilde{\beta}(i)}{n}
\end{align}
Subtracting (\ref{eq:alpha_1step_Ytilde}) from (\ref{eq:alpha_1step}) implies (\ref{eq:exp_change_F_alpha}).

%\Ex\left[\widetilde{u}(i+1) - \widetilde{u}(i) \mid \mathcal{F}_i \right] & = \frac{\widetilde{\gamma}_{cs}(i)}{n} - \frac{\widetilde{u}(i)}{n} \cdot \frac{\widetilde{\Gamma}(i)}{n-1} \notag \\

The expected change in the number of leaders with \q is given by
\begin{align}
\Ex\left[\widetilde{\delta}(i+1) - \widetilde{\delta}(i) \mid \mathcal{F}_i \right] & = \frac{\widetilde{\alpha}(i)}{2n} \cdot \frac{\widetilde{\Gamma}(i)-\widetilde{\beta}(i)}{n-1} + \frac{n/s - {\widetilde{\alpha}(i)} - {\widetilde{\delta}(i)}}{2n} \cdot \frac{\widetilde{\beta}(i)}{n-1} - \frac{\widetilde{\delta}(i)}{n} \cdot \frac{\widetilde{\Gamma}(i)}{n-1} \notag
\end{align}
The first term is the probability that a wrong leader pulled from a correctly informed node, the second term is the probability that a correct leader pulled from an incorrectly informed node, and the last term is the probability that a leader with \q pulled from an informed node.

\medskip

On the other hand, by Remark~\ref{rmk:det_derivatives_map} and Lemma~\ref{lem:def_deterministic}
\begin{align}
F_{\delta}(\widetilde{Y}(i)/n) = \frac{\widetilde{\alpha}(i)}{2n} \cdot \frac{\widetilde{\Gamma}(i)-\widetilde{\beta}(i)}{n}  + \frac{n/s - {\widetilde{\alpha}(i)} - {\widetilde{\delta}(i)}}{2n} \cdot \frac{\widetilde{\beta}(i)}{n} - \frac{\widetilde{\delta}(i)}{n} \cdot \frac{\widetilde{\Gamma}(i)}{n} \notag
\end{align}
Subtracting the last two displayed equations yields (\ref{eq:expected_change_Y_tilde}) for $\delta$.

The expected change in the number of informed nodes in bin $1$ with the wrong bit is
\begin{small}
	\begin{align}
	\Ex\left[\widetilde{\beta}_1(i+1) - \widetilde{\beta}_1(i) \mid \mathcal{F}_i \right] & = - \frac{\widetilde{\beta}_1(i)}{n}- \frac{n/s - \widetilde{\delta}(i) - \widetilde{\alpha}(i)}{2n} \cdot \frac{\widetilde{\beta}_1(i)}{n-1} + \frac{\widetilde{\alpha}(i)}{2n} \cdot \frac{\widetilde{\Gamma}(i) - \widetilde{\beta}_1(i)}{n-1}  + \frac{\widetilde{u}(i)}{n} \cdot \frac{\widetilde{\beta}_1(i)}{n-1} \notag
	\end{align}
\end{small}
The first term represents the probability the selected node is a follower with counter value $1$ that increases its counter value, thus leaving bin $1$. The second term is the probability the selected node is a follower that gets pushed a correct bit from the a leader (and so leaves the set of followers with the wrong bit and counter value 1). The third term is the probability the selected node is a follower that gets pushed the wrong bit from a leader with incorrect information, and the last term is the probability that an uninformed follower pulls the wrong bit from a follower with counter value 1. Comparing this to $F_{\beta_1}(\widetilde{Y}(i)/n)$ using Lemma~\ref{lem:def_deterministic} yields (\ref{eq:expected_change_Y_tilde}) for $\beta_1$.

\medskip

The expected change in the number of informed nodes in bin $j \in \{2, \ldots, 8s\}$ with the wrong bit is
\begin{small}
\begin{align}
\Ex\left[\widetilde{\beta}_j(i+1) - \widetilde{\beta}_j(i) \mid \mathcal{F}_i \right] & =
 \frac{\widetilde{\beta}_{j-1}(i)}{n} -  \frac{\widetilde{\beta}_{j}(i)}{n}- \frac{n/s - \widetilde{\delta}(i)}{2n} \cdot \frac{\widetilde{\beta}_j(i)}{n-1} + \frac{\widetilde{u}(i)}{n} \cdot \frac{\widetilde{\beta}_j(i)}{n-1}, \; \; \forall j \in\{2, \ldots, 8s\} \notag
 \end{align}
 \end{small}
 The first and second term represent the probability the selected node is a wrongly informed node from bin $j-1$ and bin $j$, respectively. The third term is the probability the selected node is a leader that pushes its bit to a wrong node in bin $j$, and the last term is the probability that an uniformed node pulls from an incorrect node in bin $j$.
 Comparing this to $F_{\beta_{j}}(\widetilde{Y}(i)/n)$ using Lemma~\ref{lem:def_deterministic} yields (\ref{eq:expected_change_Y_tilde}) for $\beta_j$.

\medskip

The expected change in the number of uninformed nodes with the wrong bit is 
\begin{small}
	\begin{align}
	\Ex\left[\widetilde{\beta}_{8s+1}(i+1) - \widetilde{\beta}_{8s+1}(i) \mid \mathcal{F}_i \right] & =
	\frac{\widetilde{\beta}_{8s}(i)}{n} -  \frac{\widetilde{\beta}_{8s+1}(i)}{n} \cdot \frac{\widetilde{\Gamma}(i)}{n-1} \notag
	\end{align}
\end{small}
The first term represents the probability the selected node is a wrongly informed node from bin $8s$ and the second is the probability that an uninformed node with the wrong bit is selected and contacts an informed node.
Comparing this to $F_{\beta_{8s+1}}(\widetilde{Y}(i)/n)$ using (\ref{eq:derivative_beta_8s1}) yields (\ref{eq:expected_change_Y_tilde}) for $\beta_{8s+1}$.

\medskip

%The third term is the probability the selected node is a leader that pushes its bit to a wrong node in bin $j$, and the last term is the probability that an uniformed node pulls from an incorrect node in bin $j$.
%Comparing this to $F_{\beta_j}(\widetilde{Y}(i)/n)$ using Lemma~\ref{lem:def_deterministic} yields (\ref{eq:expected_change_Y_tilde}) for $\beta_j$.


The arguments for $\gamma_j$ are similar to those for $\beta_j$, so we only include the expected changes:
\begin{small}
\begin{align}
\Ex\left[\widetilde{\gamma}_j(i+1) - \widetilde{\gamma}_j(i) \mid \mathcal{F}_i \right] & = \frac{\widetilde{\gamma}_{j-1}(i)}{n} - \frac{\widetilde{\gamma}_j(i)}{n} - \frac{\widetilde{\gamma}_j(i)}{n-1} \cdot \Bigl( \frac{1}{2s} -  \frac{\widetilde{\delta}(i)}{2n} - \frac{\widetilde{u}(i)}{n}\Bigr) \; \; \forall j \in\{2, \ldots, 8s\} \notag \\
\Ex\left[\widetilde{\gamma}_1(i+1) - \widetilde{\gamma}_1(i) \mid \mathcal{F}_i \right] & = - \frac{\widetilde{\gamma}_1(i)}{n} + \left(\frac{1}{2s} - \frac{\widetilde{\delta}(i)}{2n} \right) \cdot  \frac{\widetilde{\Gamma}(i)-\widetilde{\gamma}_1(i)}{n-1} + \frac{\widetilde{u}(i)}{n} \cdot \frac{\widetilde{\gamma}_1(i)}{n-1} \notag
\end{align}
\end{small}
\end{proof}

\begin{corollary} \label{cor:warnke_bounds_beta_j}
	Let $ T \leq \frac{\ln{n}}{240s}$.
	Suppose $\alpha(0) = \widetilde{\alpha}(0) / n $ and similarly for the other variables. There exists an event $A$ where $\mathbb{P}(A) \geq 1 - e^{-2n^{1/3}}$, such that on $A$ the following inequalities hold:
	\begin{align}
	& \max_{0 \leq i \leq Tn} \mid \widetilde{\alpha}(i) - n \cdot \alpha\left({i}/{n}\right)| \leq 3 n^{7/8} \notag \\
	& \max_{0 \leq i \leq Tn} \mid \widetilde{\beta}_j(i) - n \cdot \beta_j\left({i}/{n}\right)| \leq 3 n^{7/8}, \mbox{ for each } j \in \{1, \ldots, 8s+1\}\notag  \\
	& \max_{0 \leq i \leq Tn} \mid \widetilde{\delta}(i) - n \cdot \delta\left({i}/{n}\right)| \leq 3 n^{7/8} \notag \\
	& \max_{0 \leq i \leq Tn} \mid \widetilde{\gamma}_j(i) - n \cdot \gamma_j\left({i}/{n}\right)| \leq 3 n^{7/8}, \mbox{ for each } j \in \{1, \ldots, 8s\} \notag
	\end{align}
\end{corollary}
\begin{proof}
	We apply the differential equation method (Theorem~\ref{thm:DEM}). The variables are defined in Remark~\ref{rmk:det_derivatives_map}.
Let $a = 16s+3$ and consider the following domain:
\begin{align}
\mathcal{D}  = \Bigl\{ \bigl(\alpha, \delta, \beta_1, \ldots, \beta_{8s+1}, \gamma_1, \ldots, \gamma_{8s} \bigr) \mid
\alpha, \delta, \beta_j, \gamma_j \in [-1, 1] \; \forall j,  \sum_{j=1}^{8s+1} \beta_j \in (-1,1), \sum_{j=1}^{8s} \gamma_j \in (-1,1)  \Bigr\}
\end{align}
Examining the functions in Remark~\ref{rmk:det_derivatives_map}, it can be seen that $M = \sup\{F_k(y) \mid k \in [a], y \in \mathcal{D}\} \leq 4$ and the Lipschitz constant is $L = 30s$.

By Lemma~\ref{lem:change_expectation_Y}, condition (i) of Theorem~\ref{thm:DEM} holds with $\lambda_0 = 1/n$. Let $\lambda = n^{-1/4}$. Then by definition of $\mathcal{D}$ and Lemma~\ref{lem:bounded_differential}, the distance from the boundary is at least $1/s$ for all $t \in [0, T]$. Since
$
T \leq {\ln{n}}/(240s) %\leq \frac{1}{30s} \ln{\left(\frac{n^{1/8}}{3s}\right)},
$
we get that $3e^{LT}\lambda = 3e^{30s \cdot T} \cdot n^{-1/4} \leq 3 e^{\ln\left({n^{1/8}}\right)} n^{-1/4} \leq 3 n^{-1/8} < 1/s$.
Then with probability at least $1 - \exp\left(-2n^{1/3}\right)$, we have
\begin{align}% \label{dem:error}
\max_{0 \le i \le T n} \max_{1 \le k \le a}\bigabs{Y_k(i)-n \cdot y_k\bigpar{\tfrac{i}{n}}} \; \leq \; 3 n^{7/8} \,.
\end{align}
\end{proof}

\begin{proposition} \label{prop:warnke_app}
For each $i \in \{n T_2, \ldots, 4n \ln{n}\}$, let $G_i$ be the event where the following constraints hold:
	\begin{align}
	& \widetilde{\beta}(i) \leq \frac{n}{s^2} \cdot {2^{2-8s}} \\
	& \frac{\widetilde{\delta}(i)}{2} + \widetilde{u}(i) \leq \frac{n-1}{2s}
	\end{align}
Define
$G = \bigcap_{i = nT_2}^{4n \ln{n}} G_i\,.$
Then $\Pr(G^c) \leq e^{-n^{1/3}}$.
\end{proposition}
\begin{proof}
Recall $T_2 = T_1 + 64 s^2 \leq \frac{C_1 s}{\rho} + 64 s^2$, where $C_1 \leq 144 \cdot 32$.
We apply the differential equation method iteratively on time intervals
\begin{align}
\{\ell nT_2, \ldots, (\ell+1)nT_2\}, \mbox{ for } \ell \in \{0, 1,\ldots, \ln{n}\}\,.
\end{align}
At the beginning of each $\ell$-th iteration, we reset the deterministic system to match the random one; we denote the $\ell$-th deterministic system set this way by $\alpha^{[\ell]}(t)$, where $t \in \{\ell T_2, \ldots, (\ell +1)T_2\}$. The initial condition is $\alpha^{[\ell ]}(\ell T_2) := \widetilde{\alpha}(\ell nT_2)/n$. The other variables are similarly set. Note that $\alpha^{[0]} = \alpha$. An example can be seen in the next figure.

\begin{figure}[h!]
	\centering
	\subfigure[Fraction of leaders with the wrong bit]
	{
		\includegraphics[scale=1.52]{staircase_alpha_n=2500_T=2000_c=0_48.png}
	}
	\subfigure[Fraction of informed followers with the wrong bit]
	{
		\includegraphics[scale=1.52]{staircase_beta_n=2500_T=2000_c=0_48.png}
	}
	\caption{The fraction of leaders with the wrong bit, where the deterministic system is reset periodically to restart from the random system ($\frac{\widetilde{\alpha}}{n} \cdot s$ shown in red and $\frac{\widetilde{\beta}}{n} \cdot \frac{s}{s-1}$ in blue). The $X$ axis shows the time, $n = 2500$, $s=5$, and the initial minority fraction is $0.49$}
\end{figure}


By Corollary~\ref{cor:warnke_bounds_beta_j}, there exists an event $A_{\ell}$ with $\mathbb{P}(A_{\ell}) \geq 1 - e^{-2n^{1/3}}$, so that on $A_{\ell}$ the following inequalities hold at all times $i \in \{\ell nT_2, \ldots, (\ell +1)nT_2\}$:
\begin{align} \label{eq:beta_close_beta_tilde}
& \mid \widetilde{\alpha}(i) - n \cdot \alpha^{[\ell]}(i/n) \mid \leq 3 n^{7/8} \notag \\
& \mid \widetilde{\delta}(i) - n \cdot \delta^{[\ell]}(i/n) \mid \leq 3 n^{7/8} \notag \\
& \mid \widetilde{\beta}(i) - n \cdot \beta^{[\ell]}(i/n) \mid \leq 24s \cdot n^{7/8} \notag \\
& \mid \widetilde{\Gamma}(i) - n \cdot \Gamma^{[\ell]}(i/n) \mid \leq 24s \cdot n^{7/8}
\end{align}
Recall $\Lambda_2(t) = \alpha(t) + \delta(t)/16 + \beta(t)/4$.
Summing up the inequalities in (\ref{eq:beta_close_beta_tilde}) with suitable weights, we get that on the event $A_{\ell}$ the next inequality holds
\begin{align} \label{eq:tilde_lambda_deterministic_interval_bound}
\mid \widetilde{\Lambda}_2(i) - n \cdot \Lambda_2^{[\ell]}(i/n) \mid \leq 7s \cdot n^{7/8}
\end{align}
The proof of inequality (\ref{eq:lambda2_beyond_T1}) gives
\begin{align} \label{eq:lambda2_beyond_T1_kth_system}
\Lambda_{2}^{[\ell]}(t) \leq \Lambda_{2}^{[\ell ]}(\ell T_2) \cdot \exp\left(-\frac{t - \ell T_2}{8s}\right), \mbox{ for all } t \geq \ell T_2\,.
\end{align}
By (\ref{eq:lambda2_beyond_T1}), it follows that $\Lambda_{2}^{[0]}(T_2) \leq e^{-8s}$. Therefore, on the event $A_0$, we have
\begin{align} \label{eq:lambda2_base_case_k=1}
\Lambda_{2}^{[1]}(T_2) = \frac{\widetilde{\Lambda}_2(nT_2)}{n} \leq \exp(-8s) + 7s \cdot n^{-1/8}\,.
\end{align}
Let $A^* = \bigcap_{\ell =0}^{\ln(n)} A_{\ell }$. Note that $\Pr((A^*)^c) \leq (1 + \ln{n}) \cdot e^{-2n^{1/3}} \leq e^{-n^{1/3}}$.
We use induction on $\ell  = 1, \ldots, \ln{n}$ to show that on $A^*$ we have
\begin{align} \label{eq:lambda2_k_induction_bound}
\Lambda_2^{[\ell ]}(t) \leq \exp(-8s) + 7s \cdot n^{-1/8}, \mbox{ for all } \ell  = 1, \ldots, \ln{n} \mbox{ and } t \in [\ell T_2, (\ell +1)T_2]
\end{align}%, for all $k = 1, \ldots, \ln{n} $ and $t \in [kT_2, (k+1)T_2]$.
The base case $\ell =1$ holds by (\ref{eq:lambda2_base_case_k=1}). We assume (\ref{eq:lambda2_k_induction_bound}) holds for $\ell $ and derive it for $\ell +1$. By applying (\ref{eq:lambda2_beyond_T1_kth_system}), we obtain
\begin{align}  \label{eq:lambda2_k+1_induction_bound_initialTk}
\Lambda_2^{[\ell ]}((\ell +1) T_2) \leq \Bigl(\exp(-8s) + 7s \cdot n^{-1/8} \Bigr) \cdot \exp(-8s) \leq \exp(-8s)
\end{align}
Then on $A^*$ we have
\begin{align}\label{eq:lambda2_k+1_induction_bound_initialTk+1}  \Lambda_2^{[\ell +1]}((\ell +1)T_2) & = \frac{\widetilde{\Lambda}_2((\ell +1)nT_2)}{n} \notag \\
& \leq \Lambda_2^{[\ell ]}((\ell +1) T_2) + 7s \cdot n^{-1/8} \notag \\
& \leq \exp(-8s) + 7s \cdot n^{-1/8}
\end{align}
The induction claim follows from (\ref{eq:lambda2_beyond_T1_kth_system}) and (\ref{eq:lambda2_k+1_induction_bound_initialTk+1}). Note that $T_2 \geq 4$, thus for each $i \in \{nT_2, \ldots, 4n\ln{n}\}$, on $A^*$ we have
\begin{align} \label{eq:beta_delta_tilde_i_ub}
 & \widetilde{\beta}(i) \leq 4\left(\exp(-8s) + 7s \cdot n^{-1/8}\right) n \leq \frac{n}{s^2} \cdot 2^{2-8s}  \notag \\
& \widetilde{\delta}(i) \leq 16\left(\exp(-8s) + 7s \cdot n^{-1/8}\right) n\,.
\end{align}

Let $\epsilon_0 > \epsilon_1 > \ldots > \epsilon_{\;\ln{n}} > 0$, where $\epsilon_0 = 1/50$. For $\ell  > 0$, define $\epsilon_{\ell}= \epsilon_{\ell -1} - 1/(60 \ln{n})$. We will show by induction on $\ell = 0, \ldots, \ln{n}$ that on $A^*$, the following inequalities hold
\begin{align}
\gamma_j^{[\ell ]}(t) & < \frac{1 - \epsilon_{\ell }}{s} \cdot e^{-\frac{j}{4s}}, \; \forall j \in \{1, \ldots, 8s \}, \forall t \geq \ell T_2 \\
u^{[\ell ]}(t) & < \frac{1 - \epsilon_{\ell }}{s  e^{2}}  \left(1 + \frac{3}{s}\right), \forall t \geq \ell T_2
\end{align}
The base case $\ell =0$ follows by Lemma~\ref{ub:u_gamma} since $\gamma_j(0) = 1/(8s)$ and $u(0) = 0$. Assume it holds for $\ell -1$ and derive it for $\ell $. Recall $u(t) = 1 - 1/s - \Gamma(t)$. Then by (\ref{eq:beta_close_beta_tilde}), on $A^*$ we have
\begin{align}
\Bigl | u^{[\ell ]}(\ell T_2) - u^{[\ell -1]}(\ell T_2) \Bigr |  = \Bigl | \frac{\widetilde{u}(n \ell T_2)}{n} - u^{[\ell -1]}(\ell T_2) \Bigr | \leq 24s n^{-1/8} \le \frac{\epsilon_{\ell } - \epsilon_{\ell -1}}{se^2},
\end{align}
and similarly for $\gamma_j^{[\ell ]}(\ell T_2)$.
Applying Lemma~\ref{ub:u_gamma} again completes the induction step.
By (\ref{eq:beta_close_beta_tilde}), for each $i \in \{\ell nT_2, \ldots, (\ell +1)nT_2\}$, on $A^*$ we have
\begin{align} \label{eq:u_tilde_i_ub}
\widetilde{u}(i) \leq n \cdot u^{[\ell ]}(i/n) + 24s \cdot n^{7/8} \leq \frac{n}{se^2} \left(1 + \frac{3}{s} \right) + 24s \cdot n^{7/8}
\end{align}
Combining (\ref{eq:beta_delta_tilde_i_ub}) and (\ref{eq:u_tilde_i_ub}) implies that  $A^* \subseteq G$. Since $\Pr(A^*) \geq 1- e^{-n^{1/3}}$, this completes the argument.
\end{proof}



\begin{proposition} \label{prop:exp_lambda_3_tilde}
For each integer $i \geq 0$, let $\widetilde{\Lambda}_{3}(i) = \widetilde{\alpha}(i) + d \cdot \widetilde{\delta}(i) + \sum_{j=1}^{8s+1} a_j \cdot \widetilde{\beta}_j(i)$, where $d =1/s \cdot 2^{1-8s}$ and $a_j = 1/s \cdot 2^{1-j}$ for all $j$. Then for each $i \geq n T_2$
$$
\Ex\Bigl[\widetilde{\Lambda}_{3}(i+1) - \widetilde{\Lambda}_{3}(i) \Bigr] \leq - \frac{\zeta_{3}}{n} \cdot \Ex\left[\widetilde{\Lambda}_{3}(i)\right] + e^{-n^{1/3}}, \quad \mbox{for } \zeta_{3} = 1/2 - 2/s\,.
$$
\end{proposition}
\begin{proof}
By Lemma~\ref{lem:change_expectation_Y}, the change in the number of leaders with the wrong bit is
\begin{align} \label{eq:tilde_alpha_ub}
\Ex\left[ \widetilde{\alpha}(i+1) - \widetilde{\alpha}(i) \mid \mathcal{F}_i \right] & = \frac{\widetilde{\alpha}(i)}{2n(n-1)} \cdot \left[{-\widetilde{\Gamma}(i) + \widetilde{\beta}(i)}\right] + \frac{\widetilde{\delta}(i)\widetilde{\beta}(i)}{n(n-1)}
\end{align}
Recall the definition of $G_i$ from Proposition~\ref{prop:warnke_app}.
On $G_i$, since $\widetilde{\Gamma}(i) > n/2$,   Lemma~\ref{lem:change_expectation_Y} gives the following bound on the number of \q nodes:
\begin{align}  \label{eq:tilde_delta_ub}
\Ex\left[ \widetilde{\delta}(i+1) - \widetilde{\delta}(i) \mid \mathcal{F}_i \right] & \leq \frac{\widetilde{\alpha}(i)}{2n(n-1)} \cdot \left[{\widetilde{\Gamma}(i) - \widetilde{\beta}(i)}\right] + \frac{\widetilde{\beta}(i)}{2s(n-1)}  - \frac{\widetilde{\delta}(i)}{2(n-1)}
\end{align}

For the number of informed nodes in bin $1$ with the wrong bit, Lemma~\ref{lem:change_expectation_Y} gives the upper bound:
\begin{align} \label{eq:tilde_beta_1_ub}
\Ex\left[ \widetilde{\beta}_1(i+1) - \widetilde{\beta}_1(i) \mid \mathcal{F}_i \right] & \leq -\frac{\widetilde{\beta}_1(i)}{n} - \frac{\widetilde{\beta}_1(i)}{2s(n-1)}  + \frac{\widetilde{\alpha}(i)\widetilde{\Gamma}(i)}{2n(n-1)} + \frac{\widetilde{u}(i)\widetilde{\beta}_1(i)}{n(n-1)}
\end{align}

On $G_i$, since $\widetilde{\delta}(i) + 2 \widetilde{u}(i) \leq \frac{n-1}{2s}$, Lemma~\ref{lem:change_expectation_Y} gives the following upper bound on the number of informed nodes in bin $j \in\{2, \ldots, 8s\}$ with the wrong bit:
\begin{align} \label{eq:tilde_beta_j_ub}
\Ex\left[ \widetilde{\beta}_j(i+1) - \widetilde{\beta}_j(i) \mid \mathcal{F}_i \right] & \leq \frac{\widetilde{\beta}_{j-1}(i) - \widetilde{\beta}_{j}(i)}{n} -  \frac{\widetilde{\beta}_j(i)}{2n(n-1)} \cdot \frac{n}{2s}
\end{align}

Finally, for the number of uninformed nodes with the wrong bit, Lemma \ref{lem:change_expectation_Y} gives the bound:
\begin{align} \label{eq:tilde_beta_8s1_ub}
\Ex\left[ \widetilde{\beta}_{8s+1}(i+1) - \widetilde{\beta}_{8s+1}(i) \mid \mathcal{F}_i \right] & \leq \frac{\widetilde{\beta}_{8s}(i)}{n} -  \frac{\widetilde{\beta}_{8s+1}(i)}{n} \cdot \frac{\widetilde{\Gamma}(i)}{n-1}
\end{align}

By adding inequality (\ref{eq:tilde_alpha_ub}), inequality (\ref{eq:tilde_delta_ub}) multiplied by $d$, inequality (\ref{eq:tilde_beta_1_ub}) multiplied by $a_1$, inequality (\ref{eq:tilde_beta_j_ub}) multiplied by $a_j$, and inequality (\ref{eq:tilde_beta_8s1_ub}) multiplied by $a_{8s+1}$, we obtain that on $G_i$ the conditional expectation of $\widetilde{\Lambda}_3$ can be bounded by:
\begin{small}
\begin{align} \label{eq:lambda_3_tilde_ub}
\mathbbm{1}_{G_i} \cdot \Ex\left[ \widetilde{\Lambda}_{3}(i+1) - \widetilde{\Lambda}_{3}(i) \mid \mathcal{F}_i \right] & \leq \mathbbm{1}_{G_i} \cdot \Bigl\{ \frac{\widetilde{\alpha}(i)}{2n(n-1)}  \left[\widetilde{\Gamma}(i)  (d+a_1 - 1) + \widetilde{\beta}(i)  (1 - d) \right] + \frac{\widetilde{\delta}(i)}{n(n-1)} \left( \widetilde{\beta}(i) - d  \frac{n}{2} \right) \Bigr. \notag \\
&  \Bigl. + \sum_{j=1}^{8s} \frac{\widetilde{\beta}_j(i)}{n} \left[ a_{j+1} - a_j \left( 1 + \frac{1}{2s} \right) + \frac{dn}{2s(n-1)} \right] - \frac{\widetilde{\beta}_{8s+1}(i)}{2n} \Bigr\}
\end{align}
\end{small}

Using $\widetilde{\Gamma}(i) \geq (1 - \frac{2}{s})n$ and $d + a_1 < 2/s$, we get that on $G_i$ the conditional expectation can further be bounded as follows:
\begin{small}
\begin{align}
\mathbbm{1}_{G_i} \cdot \Ex\left[ \widetilde{\Lambda}_{3}(i+1) - \widetilde{\Lambda}_{3}(i) \mid \mathcal{F}_i \right] & \leq \mathbbm{1}_{G_i} \cdot
\Bigl\{ \frac{\widetilde{\alpha}(i)}{2n} \left[ - \left(1 - \frac{2}{s}\right)^2 + \frac{1}{s^2}\right] +  \frac{\widetilde{\delta}(i)}{n} \left[ - \frac{d}{2}\left(1 - \frac{4}{s}\right) \right] \Bigr. \notag \\
& \Bigl. + \sum_{j=1}^{8s+1} \frac{\widetilde{\beta}_j(i)}{n} \cdot \left[ - \frac{a_j}{2} \left( 1 - \frac{1}{s} \right) \right] \Bigr\} \notag \\
& \leq - \frac{\zeta_3}{n} \cdot \widetilde{\Lambda}_3(i) \notag
\end{align}
\end{small}
Taking expectation over all histories $\mathcal{F}_i$, we get the following inequality:
%and dividing into two cases, for $G_i$ and $G_i^c$, gives [\textcolor{red}{TODO: Explain why Warnke gives $P(G_i^c) < e^{-n...}$.}]
\begin{small}
\begin{align}
\Ex\left[ \widetilde{\Lambda}_{3}(i+1) - \widetilde{\Lambda}_{3}(i) \right] & = \mathbb{P}(G_i) \Ex\left[\widetilde{\Lambda}_{3}(i+1) - \widetilde{\Lambda}_{3}(i) \mid G_i \right] + \mathbb{P}(G_i^c) \Ex\left[\widetilde{\Lambda}_{3}(i+1) - \widetilde{\Lambda}_{3}(i) \mid G_i^c  \right] \notag \\
& \leq \frac{-\zeta_3}{n} \cdot \Ex\left[\widetilde{\Lambda}_3(i) \right] + e^{-n^{1/3}}\,. \notag
\end{align}
\end{small}
\end{proof}

The next proposition completes the proof of Theorem~\ref{thm:main}. We use the following lemma.

\begin{lemma}[Theorem A.1.15 in \cite{AlonSpencer_book}] \label{lem:alonspencer}
	Let $J$ have Poisson distribution with mean $\mu$. For $\epsilon > 0$
	\begin{align}
	\mathbb{P}\left[J \leq \mu(1-\epsilon)\right] & \leq e^{-\epsilon^2 \mu/2}, \notag \\
	\mathbb{P}\left[J \geq \mu(1+\epsilon)\right] & \leq \Bigl[ e^{\epsilon} (1 + \epsilon)^{-(1+\epsilon)}\Bigr]^{\mu}. \notag
	\end{align}
\end{lemma}

\begin{proposition}
Given $\theta > 0$, let $T_4 = T_4(\theta) = \frac{1 + 4\theta}{\zeta_{3}} \ln{n}$. Then with probability at least $1 - n^{-\theta/2}$, by time $T_4$ the system has reached consensus and the total communication until consensus is $O(\frac{n \log{n}}{s})$.
\end{proposition}
\begin{proof}
By Proposition~\ref{prop:exp_lambda_3_tilde}, for all $i \in \{nT_2, \ldots, 4n \ln{n}\}$, we have
\begin{small}
\begin{align}
\Ex\left[ \widetilde{\Lambda}_{3}(i+1) \right] & \leq  \left(1 - \frac{\zeta_3}{n} \right)\Ex\left[ \widetilde{\Lambda}_{3}(i) \right] + e^{-n^{1/3}}\,.
\end{align}
\end{small}
By definition of $\widetilde{\Lambda}_3$, we have that $ \widetilde{\Lambda}_{3}(i) \leq n$ for all $i$.
By induction on $i \geq nT_2$, it follows that
\begin{small}
\begin{align} \label{eq:lambda_3_ub_t2}
\Ex\left[ \widetilde{\Lambda}_{3}(i) \right] & \leq  \left(1 - \frac{\zeta_3}{n}\right)^{i-nT_2} \cdot n + (i-nT_2) \cdot e^{-n^{1/3}}\,.
\end{align}
\end{small}
Let $T_3 = T_2 + \frac{1 + \theta}{\zeta_3} \ln{n}$ as in Section~\ref{sec:phase_3}.
For $i = nT_3$, inequality (\ref{eq:lambda_3_ub_t2}) gives
\begin{small}
	\begin{align}
	\Ex\left[ \widetilde{\Lambda}_{3}(i) \right] & \leq  n\left(1 - \frac{\zeta_3}{n}\right)^{n(T_3 - T_2)}  + n(T_3 - T_2) \cdot e^{-n^{1/3}} \notag \\
	& \leq n e^{-(1+\theta)\log{n}} + \frac{1 + \theta}{\zeta_3} n \ln{n} \cdot e^{-n^{1/3}} < 2n^{-\theta}, \; \mbox{for large } n\,.
	\end{align}
\end{small}
If after $i$ clock rings there is no consensus, then at least one of $\widetilde{\alpha}(i)$, $\widetilde{\delta}(i)$, $\widetilde{\beta}_1(i), \ldots, \widetilde{\beta}_{8s+1}(i)$ is positive, so $\widetilde{\Lambda}_3(i) \geq \min\{1, d, a_1, \ldots, a_{8s+1}\} = a_{8s+1}$. Using Markov's inequality we get
\begin{align} \label{eq:prob_no_consensus}
\mathbb{P}(\mbox{no consensus after } n T_3 \mbox{ rings})
\leq \mathbb{P}\left(\widetilde{\Lambda}_3(n T_3) \geq a_{8s+1} \right) \leq
a_{8s+1}^{-1} \Ex\left( \widetilde{\Lambda}_3(n T_3) \right) \leq s 2^{8s} \cdot 2 n^{-\theta}
\end{align}

Let $J(t)$ denote the number of clock rings by time $t$. Then $J(t)$ has Poisson distribution with parameter $nt$. By Lemma~\ref{lem:alonspencer}, where $\mu = nT_4$ and $\epsilon = 1 - T_3/T_4$, we have
\begin{align} \label{eq:prob_few_clock_rings}
\mathbb{P}(J(T_4) \leq n T_3) & \leq \exp(- \left( 1 - \frac{T_3}{T_4}\right)^2 \cdot nT_4) \leq \exp\left(-\left(\frac{2\theta}{1 + 4 \theta}\right)^2 \cdot n\left(\frac{1+4\theta}{\zeta_3}\right) \ln{n}\right) \notag \\
& \leq \exp\left(- \frac{4 \theta^2}{\zeta_3(1+4\theta)} \cdot n \ln{n}\right) \,.
\end{align}

\smallskip

Combining (\ref{eq:prob_no_consensus}) and (\ref{eq:prob_few_clock_rings}) implies that
\begin{align} \label{eq:prob_no_consensus_ub_by_T4}
\mathbb{P}(\mbox{no consensus by time } T_4) & \leq \mathbb{P}(\mbox{no consensus after } n T_3 \mbox{ rings}) + \mathbb{P}\left(J(T_4) \leq n T_3\right)  \notag \\
& \leq 2s 2^{8s} n^{-\theta}  + \exp(-n\theta^2)
\end{align}

Let $L_i = \textbf{1}_{\{\mbox{i-th clock ring is a leader}\}}$ and $\Upsilon_i = \textbf{1}_{\{\mbox{i-th clock ring is uninformed}\}}$. Let $N_{C}(j)$ denote the number of communications by time $j$. Observe that the sequence $$M_j = \sum_{i=1}^{m} \Bigl(L_i - \Ex[L_i \mid \mathcal{F}_{i-1}] +  \Upsilon_i -\Ex[\Upsilon_i \mid \mathcal{F}_{i-1}]\Bigr) = \sum_{i=1}^{m} \left(L_i - \frac{1}{s} +  \Upsilon_i -\frac{\widetilde{u}(i)}{n}\right)$$ is a martingale, as the increments have mean zero given the past. Moreover, the increments are in $[-1, 1]$. Applying Azuma-Hoeffding gives
$\mathbb{P}\left(M_j \geq \frac{j}{2s}\right) \leq e^{-\frac{j}{8s^2}}$. Then we have
\begin{align} \label{eq:comm_by_nT_3_ub}
\mathbb{P}\left( N_C(nT_3) \geq \frac{2nT_3}{s} \right) & \leq \mathbb{P}\left( \sum_{i=1}^{n T_3} (L_i + \Upsilon_i) \geq \frac{2n T_3}{s} \right) \notag \\
& \leq \mathbb{P}\left(\sum_{i=1}^{n T_3} \left(L_i - \frac{1}{s} + \Upsilon_i - \frac{\widetilde{u}(i)}{n}\right) \geq \frac{ n T_3}{2s} \right) +  \mathbb{P}\left(\sum_{i=1}^{n T_3} \frac{\widetilde{u}(i)}{n} \geq \frac{n T_3}{2s} \right) \notag \\
& \leq e^{-\frac{n T_3}{8s^2}} + e^{-n^{1/3}} \leq e^{-{n}} + e^{-n^{1/3}} \,.
\end{align}
Combining inequalities (\ref{eq:prob_no_consensus_ub_by_T4}) and (\ref{eq:comm_by_nT_3_ub}) proves the proposition.
\end{proof}

\section{Concluding Remarks}

Recall the potential function $\Phi$ defined in Phase I. Next we discuss the extension of Theorem 1 to the setting where the initial advantage $\rho=\Phi(0)$ is allowed to tend to zero as $n \to \infty$.
\begin{corollary} Suppose $\rho=\Phi(0)>n^{-\sigma}$ for some $\sigma \in (0,1/2)$.
Then the protocol in Theorem 1 reaches consensus on the majority belief w.h.p in  time to $O(\log{n} +s|\log\rho|)$, and the total communication until consensus is $O(n/s \cdot \log(n)+ n |\log \rho|)$ w.h.p.
\end{corollary}
\begin{proof}
Let $\epsilon=\rho/16$ and $v_j= v_j  = \frac{2(3+ 2j)}{3+16s}$ for each $j \geq 1$. Recall the potential function from (\ref{psi_fun}):
\begin{align} 
\Psi(t) = \min\Bigl\{ \xi(t), \eta_1(t) + \epsilon \cdot v_1, \ldots, \eta_{8s}(t) + \epsilon \cdot v_{8s} \Bigr\}.
\end{align}
The proof of Proposition \ref{prop:phase1_progress} yields a lower bound for the right derivative
$\Psi_{+}'(t)  \ge \epsilon/(9s)=\rho/(144s)$, hence $t_1=288s$ satisfies
$\Psi(t_1) \ge 3\rho$, whence $\Phi(t_1)  \ge 5\rho/2>2\rho$. 

Iterating, we deduce that
$t_*=144s \log_2(1/\rho)$ satisfies $\Phi(t_*) \ge 1/2$. 
(Note that $\Psi$ changes its form in each round of the iteration, but $\Phi$ does not.) 
The assumption that $\rho>n^{-\sigma}$ for some $\sigma \in (0,1/2)$ is needed in order to apply Warnke's Theorem and conclude that the random system also satisfies $\tilde{\Phi}(t_1)>2\rho$ and reaches relative advantage at least $1/2$ by time $O(s\log(1/\rho))$.
\end{proof}

%TODO: Figure out what to do with these vars:
%\begin{itemize}
%	\item $\widehat{\alpha}(t) = $ fraction of leaders at time $t$ out of $n$ in the finite stochastic system.
%	\item
%	\item $J(t) = $ number of rings by time $t$. Note: can show $|J(t) - nt|= O(n^{1/2 + \epsilon})$ w.h.p.
%\end{itemize}


\bibliographystyle{natbib}
\bibliography{consensus_bib}

\end{document}




~

\noindent\textbf{\textit{Linearisation of a structured document}}

To simplify the presentation, we represent in the following, trees by their linearisation in the form of a Dyck word. To do this, we associate a (various) pair of brackets to each grammatical symbol and the linearisation of a tree is obtained by performing a Depth First Search (DFS) of the resulting tree (see fig. \ref{chap2:fig:tree-linearisation}).
\begin{figure}[ht!]
	\noindent
	\makebox[\textwidth]{\includegraphics[scale=0.6]{Chap2/images/linearisation.png}}
	\caption{Linearisation of a tree ($tv1$): the Dyck symbols '$($' and '$)$' (resp. '$[$' and '$]$') have been associated with the grammatical symbol $A$ (resp. $B$).}
	\label{chap2:fig:tree-linearisation}
\end{figure}


~

\noindent\textbf{\textit{The transition schemas for the view $\{A, B\}$}}

A list of trees (forest) is represented by the concatenation of their linearisation. We use the opening parenthesis '(' and the closing one ')' to represent Dyck symbols associated with the visible symbol $A$, and the opening bracket '[' and the closing one ']' to represent those associated with the visible symbol $B$. Each transition of the automata associated to partial replicas according to the view $\{A, B\}$ is conform to one of the transition schemas\footnote{We do not represent the whole set of transition schemas in this example; only the useful subset for reconciliation of closed documents is shown here because the documents to reconcile in this example are all closed (has no buds). To consider buds, one should, for each visible sort $X$, associate a new pair of Dyck symbols to the bud of type $X$ then, derive the new schemas.} in table \ref{chap2:table:trans-schem-a-b}.
\begin{table}[ht]
	\centering
	\caption{The transition schemas for the view $\{A, B\}$.}
	\label{chap2:table:trans-schem-a-b}
	\begin{tabular}[t]{lcll}
	$\langle A,w_{1} \rangle$ & $\longrightarrow$ & $(P_{1}, [\langle C,u \rangle, \langle B,v \rangle])$ & if $w_{1}=u[v]$ \\
	
	$\langle A,w_{2} \rangle$ & $\longrightarrow$ & $(P_{2}, [\textit{ }])$ & if $w_{2}=\epsilon$\\
	
	$\langle B,w_{3} \rangle$ & $\longrightarrow$ & $(P_{3}, [\langle C,u \rangle, \langle A,v \rangle])$ & if $w_{3}=u(v)$\\
	
	$\langle B,w_{4} \rangle$ & $\longrightarrow$ & $(P_{4}, [\langle B,u \rangle, \langle B,v \rangle])$ & if $w_{4}=[u][v]$\\
	
	$\langle C,w_{5} \rangle$ & $\longrightarrow$ & $(P_{5}, [\langle A,u \rangle, \langle C,v \rangle])$ & if $w_{5}=(u)v$\\
	
	$\langle C,w_{6} \rangle$ & $\longrightarrow$ & $(P_{6}, [\langle C,u \rangle, \langle C,v \rangle])$ & if $w_{6}=uv\neq\epsilon$\\
	
	$\langle C,w_{7} \rangle$ & $\longrightarrow$ & $(C_\omega,[\,])$ & if $w_{7}=\epsilon$ \\
	\end{tabular}
\end{table}
These schemas are obtained from the grammar productions \cite{badouelTchoupeCmcs}, and the pairs $\langle X,w_{i} \rangle$ are states where $X$ is a grammatical symbol and $w_{i}$ a forest encoded in the Dyck language. The first schema for example, states that the AST generated from the state $\langle A,w_{1} \rangle$ are those obtained using the production $P_{1}$ to create a tree of the form $P_{1}[t_{1}, t_{2}]$; $t_{1}$ and $t_{2}$ being generated respectively from the states $\langle C,u \rangle$ and $\langle B,v \rangle$ such that $w_{1}=u[v]$. The state $\langle C,w_{7} \rangle$ with $w_{7}=\epsilon$ being an exit state, the rule $\langle C,w_{7} \rangle$ $\longrightarrow$ $(C_{\omega}, [\,])$ linked to the production $P_{7}$ states that, the AST generated from the state $\langle C,w_{7} \rangle$ is reduced to a bud of type $C$ ($C$ is the symbol located at the left hand side of $P_{7}$).


~

\noindent\textbf{\textit{Construction of the automaton $\mathcal{A}^{(1)}$ associated to $tv1$}}

Having associated Dyck symbols '(' and ')' (resp. '[' and ']') to the grammatical symbol $A$ (resp. $B$), the linearisation of the partial replica $tv1$ (fig. \ref{chap2:fig:consensus-workflow}(c)) gives $(([[()()][()]])[()])$. As $A$ is the axiom of the grammar, the initial state of the automaton $\mathcal{A}^{(1)}$ is $q_{0}^{1}=\langle A,([[()()][()]])[()] \rangle$. When considering only the  states accessible from $q_{0}^{1}$ and by applying the previous schema of transition, we obtain the automaton in table \ref{chap2:table:automaton-a-b}, for the replica $tv1$ (fig. \ref{chap2:fig:consensus-workflow}(c)).
\begin{table}[ht]
	\caption{The tree automaton associated to the replica $tv1$.}
	\label{chap2:table:automaton-a-b}
	\begin{tabular}[t]{|lcp{5.7cm}|lcp{5cm}|}
	\hline
	$q_{0}^{1}$ & $\longrightarrow$ & $(P_{1}, [q_{1}^{1}, q_{2}^{1}])$ & & with & $q_{1}^{1}=\langle C,([[()()][()]]) \rangle$ and $q_{2}^{1}=\langle B,() \rangle$\\
	
	%\end{tabular}
	%\begin{tabular}[t]{|lcp{5.cm}|lcp{5cm}|}
	\hline
	$q_{1}^{1}$ & $\longrightarrow$ & $(P_{5}, [q_{3}^{1}, q_{4}^{1}])$ & & with & $q_{3}^{1}=\langle A,[[()()][()]] \rangle$ and $q_{4}^{1}=\langle C,\epsilon \rangle$\\
	\hline
	%\end{tabular}
	%\begin{tabular}[t]{lcp{5.3cm}lcp{5cm}}
	
	$q_{1}^{1}$ & $\longrightarrow$ & $(P_{6}, [q_{4}^{1}, q_{1}^{1}])$ | $(P_{6}, [q_{1}^{1}, q_{4}^{1}])$ & & &\\
	\hline
	%\end{tabular}
	%\begin{tabular}[t]{lcp{5.3cm}lcp{5cm}}
	
	$q_{2}^{1}$ & $\longrightarrow$ & $(P_{3}, [q_{4}^{1}, q_{5}^{1}])$ & & with & $q_{5}^{1}=\langle A,\epsilon \rangle$\\
	\hline
	%\end{tabular}
	%\begin{tabular}[t]{lcp{5.3cm}lcp{5cm}}
	
	$q_{3}^{1}$ & $\longrightarrow$ & $(P_{1}, [q_{4}^{1}, q_{6}^{1}])$ & & with & $q_{6}^{1}=\langle B,[()()][()] \rangle$\\
	\hline
	%\end{tabular}
	%\begin{tabular}[t]{lcp{5.3cm}lcp{5cm}}
	
	$q_{4}^{1}$ & $\longrightarrow$ & $(C_\omega,[\,])$ & & &\\
	\hline
	%\end{tabular}
	%\begin{tabular}[t]{lcp{5.3cm}lcp{5cm}}
	
	$q_{5}^{1}$ & $\longrightarrow$ & $(P_{2}, [\,])$ & & &\\
	\hline
	%\end{tabular}
	%\begin{tabular}[t]{lcp{5.3cm}lcp{5cm}}
	
	$q_{6}^{1}$ & $\longrightarrow$ & $(P_{4}, [q_{7}^{1}, q_{2}^{1}])$ & & with & $q_{7}^{1}=\langle B,()() \rangle$\\
	\hline
	%\end{tabular}
	%\begin{tabular}[t]{lcp{5.3cm}lcp{5cm}}
	
	$q_{7}^{1}$ & $\longrightarrow$ & $(P_{3}, [q_{8}^{1}, q_{5}^{1}])$ & & with & $q_{8}^{1}=\langle C,() \rangle$\\
	\hline
	%\end{tabular}
	%\begin{tabular}[t]{lcp{5.3cm}lcp{5cm}}
	
	$q_{8}^{1}$ & $\longrightarrow$ & $(P_{5}, [q_{5}^{1}, q_{4}^{1}])$ & & &\\
	\hline
	%\end{tabular}
	%\begin{tabular}[t]{lcp{5.3cm}lcp{5cm}}
	
	$q_{8}^{1}$ & $\longrightarrow$ & $(P_{6}, [q_{8}^{1}, q_{4}^{1}])$ | $(P_{6}, [q_{4}^{1}, q_{8}^{1}])$ & & &\\
	\hline
	\end{tabular}
\end{table}
The state $q_{4}^{1}=\langle C,\epsilon \rangle$ is the only exit state of  $\mathcal{A}^{(1)}$.
It is easy to verify that the document of figure \ref{chap2:fig:consensus-workflow}(f) resulting from the reverse projection of $tv1$, belongs to the language accepted by the automaton $\mathcal{A}^{(1)}$.

~

\noindent\textbf{\textit{Construction of the automaton $\mathcal{A}^{(2)}$ associated to $tv2$}}

As before, by associating to the grammatical symbol $C$ (resp. $A$) the Dyck symbols '[' and ']' (resp. '(' and ')'), we obtain the transition schemas for the automata associated to the partial replicas according to the view $\{A, C\}$ (see table \ref{chap2:table:trans-schem-a-c}).
\begin{table}[ht]
	\centering
	\caption{The transition schemas for the view $\{A, C\}$.}
	\label{chap2:table:trans-schem-a-c}
	\begin{tabular}[t]{lcll}
	
	$\langle A,w_{1} \rangle$ & $\longrightarrow$ & $(P_{1}, [\langle C,u \rangle, \langle B,v \rangle])$ & if $w_{1}=[u]v$\\
	
	$\langle A,w_{2} \rangle$ & $\longrightarrow$ & $(P_{2}, [\textit{ }])$ & if $w_{2}=\epsilon$\\
	$\langle B,w_{3} \rangle$ & $\longrightarrow$ & $(P_{3}, [\langle C,u \rangle, \langle A,v \rangle])$ & if $w_{3}=[u](v)$\\
	
	$\langle B,w_{4} \rangle$ & $\longrightarrow$ & $(P_{4}, [\langle B,u \rangle, \langle B,v \rangle])$ & if $w_{4}=uv\neq\epsilon$\\
	$\langle B,w_{5} \rangle$ & $\longrightarrow$ & $(B_\omega,[\,])$ & if $w_{5}=\epsilon$\\
	
	$\langle C,w_{6} \rangle$ & $\longrightarrow$ & $(P_{5}, [\langle A,u \rangle, \langle C,v \rangle])$ & if $w_{6}=(u)[v]$\\
	
	
	$\langle C,w_{7} \rangle$ & $\longrightarrow$ & $(P_{6}, [\langle C,u \rangle, \langle C,v \rangle])$ & if $w_{7}=[u][v]$\\
	
	$\langle C,w_{8} \rangle$ & $\longrightarrow$ & $(P_{7},[\textit{ }])$ & if $w_{8}=\epsilon$\\
	\end{tabular}
\end{table}

The linearisation of the partial replica $tv2$ (fig. \ref{chap2:fig:consensus-workflow}(e)) is $([([\textit{ }][\textit{ }]()[\textit{ }]())[\textit{ }]][[\textit{ }][\textit{ }]]())$. The automaton $\mathcal{A}^{(2)}$ associated to this replica has as initial state $q_{0}^{2}=\langle A,[([\textit{ }][\textit{ }]()[\textit{ }]())[\textit{ }]][[\textit{ }][\textit{ }]]() \rangle$ and its transitions are the ones in table \ref{chap2:automaton-a-c}. 
\begin{table}[ht]
	\caption{The tree automaton associated to the replica $tv2$.}
	\label{chap2:automaton-a-c}
	\begin{tabular}[t]{|lcp{5.7cm}|lcp{5cm}|}
	\hline
	$q_{0}^{2}$ & $\longrightarrow$ & $(P_{1}, [q_{1}^{2}, q_{2}^{2}])$ & & with & $q_{1}^{2}=\langle C,([\,][\,]()[\,]())[\,] \rangle$ and $q_{2}^{2}=\langle B,[[\,][\,]]() \rangle$\\
	\hline
	%\end{tabular}
	%\begin{tabular}[t]{lcp{5.3cm}lcp{5cm}}
	
	$q_{1}^{2}$ & $\longrightarrow$ & $(P_{5}, [q_{3}^{2}, q_{4}^{2}])$ & & with & $q_{3}^{2}=\langle A,[\,][\,]()[\,]() \rangle$ and $q_{4}^{2}=\langle C,\epsilon \rangle$\\
	\hline
	%\end{tabular}
	%\begin{tabular}[t]{lcp{5.3cm}lcp{5cm}}
	
	$q_{2}^{2}$ & $\longrightarrow$ & $(P_{3}, [q_{5}^{2}, q_{6}^{2}])$ & & with & $q_{5}^{2}=\langle C,[\,][\,] \rangle$ and $q_{6}^{2}=\langle A,\epsilon \rangle$\\
	\hline
	%\end{tabular}
	%\begin{tabular}[t]{lcp{5.3cm}lcp{5cm}}
	
	$q_{3}^{2}$ & $\longrightarrow$ & $(P_{1}, [q_{4}^{2}, q_{7}^{2}])$ & & with & $q_{7}^{2}=\langle B,[\,]()[\,]() \rangle$\\
	\hline
	%\end{tabular}
	%\begin{tabular}[t]{lcp{5.3cm}lcp{5cm}}
	
	$q_{4}^{2}$ & $\longrightarrow$ & $(P_{7}, [\,])$ & & &\\
	
	\hline
	%\end{tabular}
	%\begin{tabular}[t]{lcp{5.3cm}lcp{5cm}}
	
	$q_{5}^{2}$ & $\longrightarrow$ & $(P_{6}, [q_{4}^{2}, q_{4}^{2}])$ & & &\\
	
	\hline
	%\end{tabular}
	%\begin{tabular}[t]{lcp{5.3cm}lcp{5cm}}
	
	$q_{6}^{2}$ & $\longrightarrow$ & $(P_{2}, [\,])$ & & &\\
	
	\hline
	%\end{tabular}
	%\begin{tabular}[t]{lcp{5.3cm}lcp{5cm}}
	
	$q_{7}^{2}$ & $\longrightarrow$ & $(P_{4}, [q_{8}^{2}, q_{7}^{2}])$ | $(P_{4}, [q_{9}^{2}, q_{10}^{2}])$ | $(P_{4}, [q_{11}^{2}, q_{11}^{2}])$ | $(P_{4}, [q_{12}^{2}, q_{13}^{2}])$ | $(P_{4}, [q_{7}^{2}, q_{8}^{2}])$ & & with & $q_{8}^{2}=\langle B,\epsilon \rangle$, $q_{9}^{2}=\langle B,[\,] \rangle$, $q_{10}^{2}=\langle B,()[\,]() \rangle$, $q_{11}^{2}=\langle B,[\,]() \rangle$, $q_{12}^{2}=\langle B,[\,]()[\,] \rangle$ and $q_{13}^{2}=\langle B,() \rangle$\\
	
	\hline
	%\end{tabular}
	%\begin{tabular}[t]{lcp{5.3cm}lcp{5cm}}
	
	$q_{8}^{2}$ & $\longrightarrow$ & $(B_\omega,[\,])$ & & &\\
	
	\hline
	%\end{tabular}
	%\begin{tabular}[t]{lcp{5.3cm}lcp{5cm}}
	
	$q_{9}^{2}$ & $\longrightarrow$ & $(P_{4}, [q_{8}^{2}, q_{9}^{2}])$ | $(P_{4}, [q_{9}^{2}, q_{8}^{2}])$ & & &\\
	
	\hline
	%\end{tabular}
	%\begin{tabular}[t]{lcp{5.3cm}lcp{5cm}}
	
	$q_{10}^{2}$ & $\longrightarrow$ & $(P_{4}, [q_{8}^{2}, q_{10}^{2}])$ | $(P_{4}, [q_{13}^{2}, q_{11}^{2}])$ | $(P_{4}, [q_{14}^{2}, q_{13}^{2}])$ | $(P_{4}, [q_{10}^{2}, q_{8}^{2}])$ & & with & $q_{14}^{2}=\langle B,()[\,]    \rangle$\\
	
	\hline
	%\end{tabular}
	%\begin{tabular}[t]{lcp{5.3cm}lcp{5cm}}
	
	$q_{11}^{2}$ & $\longrightarrow$ & $(P_{3}, [q_{4}^{2}, q_{6}^{2}])$ & & &\\
	
	\hline
	%\end{tabular}
	%\begin{tabular}[t]{lcp{5.3cm}lcp{5cm}}
	
	$q_{12}^{2}$ & $\longrightarrow$ & $(P_{4}, [q_{8}^{2}, q_{12}^{2}])$ | $(P_{4}, [q_{9}^{2}, q_{14}^{2}])$ | $(P_{4}, [q_{11}^{2}, q_{9}^{2}])$ | $(P_{4}, [q_{12}^{2}, q_{8}^{2}])$ & & &\\
	
	\hline
	%\end{tabular}
	%\begin{tabular}[t]{lcp{5.3cm}lcp{5cm}}
	
	$q_{13}^{2}$ & $\longrightarrow$ & $(P_{4}, [q_{8}^{2}, q_{13}^{2}])$ | $(P_{4}, [q_{13}^{2}, q_{8}^{2}])$ & & &\\
	
	\hline
	%\end{tabular}
	%\begin{tabular}[t]{lcp{5.3cm}lcp{5cm}}
	
	$q_{14}^{2}$ & $\longrightarrow$ & $(P_{4}, [q_{8}^{2}, q_{14}^{2}])$ | $(P_{4}, [q_{13}^{2}, q_{9}^{2}])$ | $(P_{4}, [q_{14}^{2}, q_{8}^{2}])$ & & &\\
	\hline
	\end{tabular}
\end{table}
The state $q_{8}^{2}=\langle B,\epsilon \rangle$ is the only exit state of the automaton $\mathcal{A}^{(2)}$. 


~

\noindent\textbf{\textit{Construction of the consensus automaton $\mathcal{A}_{(sc)}$}}

By application of synchronous product of several tree automata described in section \ref{chap2:sec:consensus-calculation} (\textit{construction of the consensus automaton}) to the automata $\mathcal{A}^{(1)}$ and $\mathcal{A}^{(2)}$, the consensual automaton $\mathcal{A}_{(sc)}=\mathcal{A}^{(1)}\otimes\mathcal{A}^{(2)}$ has $q_{0}=(q_{0}^{1}, q_{0}^{2})$ as initial state. $\mathcal{A}^{(1)}$ has a transition from $q_{0}^{1}$ to $[q_{1}^{1}, q_{2}^{1}]$ labelled $P_{1}$. Similarly, $\mathcal{A}^{(2)}$ has a transition from $q_{0}^{2}$ to $[q_{1}^{2}, q_{2}^{2}]$ labelled $P_{1}$. So, we have in $\mathcal{A}_{(sc)}$, a transition labelled $P_{1}$ for accessing states $[q_{1}=(q_{1}^{1}, q_{1}^{2}), q_{2}=(q_{2}^{1}, q_{2}^{2})]$ from the initial state $q_{0}=(q_{0}^{1}, q_{0}^{2})$. Following this principle, we construct the consensual automaton in table \ref{chap2:table:automaton-a-b-c}. 
\begin{table}[ht]
	\caption{The consensual tree automaton.}
	\label{chap2:table:automaton-a-b-c}
	\begin{tabular}[t]{|lcp{5.7cm}|lcp{5cm}|}
	\hline
	& & & & & $q_{0}=(q_{0}^{1}, q_{0}^{2})$\\
	\hline
	$q_{0}$ & $\longrightarrow$ & $(P_{1}, [q_{1}, q_{2}])$ & & with & $q_{1}=(q_{1}^{1}, q_{1}^{2})$ and $q_{2}=(q_{2}^{1}, q_{2}^{2})$\\
	\hline
	%\end{tabular}
	%\begin{tabular}[t]{lcp{5.3cm}lcp{5cm}}
	
	$q_{1}$ & $\longrightarrow$ & $(P_{5}, [q_{3}, q_{4}])$ & & with & $q_{3}=(q_{3}^{1}, q_{3}^{2})$ and $q_{4}=(q_{4}^{1}, q_{4}^{2})$\\
	
	\hline
	%\end{tabular}
	%\begin{tabular}[t]{lcp{5.3cm}lcp{5cm}}
	
	$q_{2}$ & $\longrightarrow$ & $(P_{3}, [q_{5}, q_{6}])$ & & with & $q_{5}=(q_{4}^{1}, q_{5}^{2})$ and $q_{6}=(q_{5}^{1}, q_{6}^{2})$\\
	
	\hline
	%\end{tabular}
	%\begin{tabular}[t]{lcp{5.3cm}lcp{5cm}}
	
	$q_{3}$ & $\longrightarrow$ & $(P_{1}, [q_{4}, q_{7}])$ & & with & $q_{7}=(q_{6}^{1}, q_{7}^{2})$\\
	
	\hline
	%\end{tabular}
	%\begin{tabular}[t]{lcp{5.3cm}lcp{5cm}}
	
	$q_{4}$ & $\longrightarrow$ & $(P_{7}, [\,])$ & & &\\
	
	\hline
	%\end{tabular}
	%\begin{tabular}[t]{lcp{5.3cm}lcp{5cm}}
	
	$q_{5}$ & $\longrightarrow$ & $(P_{6}, [q_{8}, q_{8}])$ & & with & $q_8=(q_{s1}, q^2_4)$ and $q_{s1}=\langle Open~C,[\,] \rangle $\\
	
	\hline
	%\end{tabular}
	%\begin{tabular}[t]{lcp{5.3cm}lcp{5cm}}
	
	$q_{6}$ & $\longrightarrow$ & $(P_{2}, [\,])$ & & &\\
	
	\hline
	%\end{tabular}
	%\begin{tabular}[t]{lcp{5.3cm}lcp{5cm}}
	
	$q_{7}$ & $\longrightarrow$ & $(P_{4}, [q_{9}, q_{10}])$ | $(P_{4}, [q_{11}, q_{12}])$ | $(P_{4}, [q_{13}, q_{14}])$ | $(P_{4}, [q_{15}, q_{16}])$ | $(P_{4}, [q_{17}, q_{18}])$ & & with & $q_{9}=(q_{7}^{1}, q_{8}^{2})$, $q_{10}=(q_{2}^{1}, q_{7}^{2})$, $q_{11}=(q_{7}^{1}, q_{9}^{2})$, $q_{12}=(q_{2}^{1}, q_{10}^{2})$, $q_{13}=(q_{7}^{1}, q_{11}^{2})$, $q_{14}=(q_{2}^{1}, q_{11}^{2})$, $q_{15}=(q_{7}^{1}, q_{12}^{2})$, $q_{16}=(q_{2}^{1}, q_{13}^{2})$, $q_{17}=(q_{7}^{1}, q_{7}^{2})$ and $q_{18}=(q_{2}^{1}, q_{8}^{2})$\\
	\hline
	%\end{tabular}
	
	%\begin{tabular}[t]{|lcp{6.56cm}|lcp{4.99cm}|}
	%\hline
	$q_{8}$ & $\longrightarrow$ & $(P_{7}, [\,])$ & & &\\
	
	\hline
	%\end{tabular}
	%\begin{tabular}[t]{lcp{5.3cm}lcp{5cm}}
	
	$q_{9}$ & $\longrightarrow$ & $(P_{3}, [q_{19}, q_{20}])$ & & with & $q_{19}=(q_{8}^{1}, q_{s1})$ and $q_{20}=(q_{5}^{1}, q_{s2})$, $q_{s2}=\langle Open~A,[\,] \rangle $\\
	
	\hline
	%\end{tabular}
	%\begin{tabular}[t]{lcp{5.3cm}lcp{5cm}}
	
	$q_{13}$ & $\longrightarrow$ & $(P_{3}, [q_{21}, q_{6}])$ & & with & $q_{21}=(q_{8}^{1}, q_{4}^{2})$\\
	
	\hline
	%\end{tabular}
	%\begin{tabular}[t]{lcp{5.3cm}lcp{5cm}}
	
	$q_{14}$ & $\longrightarrow$ & $(P_{3}, [q_{4}, q_{6}])$ & & &\\
	
	\hline
	%\end{tabular}
	%\begin{tabular}[t]{lcp{5.3cm}lcp{5cm}}
	
	$q_{18}$ & $\longrightarrow$ & $(P_{3}, [q_{22}, q_{20}])$ & & with & $q_{22}=(q_{4}^{1}, q_{s1})$\\
	
	\hline
	%\end{tabular}
	%\begin{tabular}[t]{lcp{5.3cm}lcp{5cm}}
	
	$q_{19}$ & $\longrightarrow$ & $(P_{5}, [q_{20}, q_{22}])$ | $(P_{6}, [q_{19}, q_{22}])$ | $(P_{6}, [q_{22}, q_{19}])$ & & &\\
	\hline
	%\end{tabular}
	%\begin{tabular}[t]{lcp{5.3cm}lcp{5cm}}
	
	$q_{20}$ & $\longrightarrow$ & $(P_{2}, [\,])$ & & &\\
	\hline
	%\end{tabular}\\
	
	%\begin{tabular}[t]{|p{8.43cm}| p{6.5cm}|}
	%\hline
	$q_{10}$ & $\longrightarrow$ & $(B_\omega, [\,])$ & & &\\ 
	\hline
	$q_{11}$ & $\longrightarrow$ & $(B_\omega, [\,])$ & & &\\
	\hline
	$q_{12}$ & $\longrightarrow$ & $(B_\omega, [\,])$ & & &\\
	\hline
	$q_{15}$ & $\longrightarrow$ & $(B_\omega, [\,])$ & & &\\
	\hline
	$q_{16}$ & $\longrightarrow$ & $(B_\omega, [\,])$ & & &\\
	\hline
	$q_{17}$ & $\longrightarrow$ & $(B_\omega, [\,])$ & & &\\
	\hline
	$q_{21}$ & $\longrightarrow$ & $(C_\omega, [\,])$ & & &\\
	\hline
	$q_{22}$ & $\longrightarrow$ & $(C_\omega, [\,])$ & & &\\
	\hline
	\end{tabular}
\end{table}


The states $\{q_{10}, q_{11}, q_{12}, q_{15}, q_{16}, q_{17}, q_{21}, q_{22}\}$ are the exit states of the automaton $\mathcal{A}_{(sc)}$. They are states whose composite states are either in conflict (for example $q_{10}=(q_{2}^{1}, q_{7}^{2})$ et $q_{2}^{1} \curlyveeuparrow q_{7}^{2}$), or are all exit states (for example $q_{22}=(q_{4}^{1}, q_{s1}))$.

The use of the function that generates the simplest AST (with buds) of a tree language from its automaton \cite{badouelTchoupeCmcs} on $\mathcal{A}_{(sc)}$, produces \textit{four} AST whose derivation trees (the consensus) are shown in figure \ref{chap2:fig:consensus-example-trees}.
\begin{figure}[ht!]
	\noindent
	\makebox[\textwidth]{\includegraphics[scale=0.3]{Chap2/images/consensusTrees.png}}
	\caption{Consensual trees generated from the automaton $\mathcal{A}_{(sc)}$}
	\label{chap2:fig:consensus-example-trees}
\end{figure}





\mySection{A Software Architecture for Centralised Management of Structured Documents in a Cooperative Editing Workflow}{}
\label{chap2:sec:architecture-cooperative-editing}

In this section, we will focus on the implementation of a system that can support cooperative editing as perceived by Badouel and Tchoup\'e. This effort is motivated by the fact that: 
\begin{enumerate}
\item \textit{This type of editing workflow applies to structured documents}: 
this leads to the fact that, one can locally perform validations in accordance with a local model derived from the global one;
\item \textit{This type of editing workflow is particularly compatible with administrative workflows}: concepts of "view" and partial replica introduced by Badouel and Tchoupé, make that the type of workflow they offer is particularly adapted for the specification of many administrative processes. Consider, for example, the process "tracking a medical record in a health center with the reception and consultation services": the aforesaid record can be modelled as a structured document in which the members of the host service (reception) cannot view and/or modify certain information contained therein; those information, requesting the expertise of the consulting staff for example. Therefore, one can associate views to each of these services. It is left only to specify the medical record's circuit and an editing workflow of the type described in the previous section is obtained;
\item \textit{A generic architectural model describing precisely an approach for the implementation of this type of workflow does not exist}: the only prototype \cite{artTinyCE} which was designed around the concepts handled (view, partial replica, merging, etc.) for this type of workflow, was more of a graphic tool (editor) for the experimentation of concepts and algorithms presented in \cite{badouelTchoupeCmcs}; workflow management is not addressed in it: this tool cannot be used to specify an editing workflow, it does not support routing or storage of artifacts, nothing is done concerning monitoring, etc., yet these concerns are among the most important to be taken care of by a workflow management infrastructure \cite{ima}.
\end{enumerate}


\mySubSection{The Proposed Architecture}{}
\label{chap2:sec:archi-proposed-architecture}

\mySubSubSection{Overall Operations}{}
\label{chap2:sec:archi-overall-operations}
The architecture that we propose is composed of three tiers: some \textit{clients}, a \textit{central server} and several \textit{administration tools}. We consider that, each participant in a given workflow has a client. Initially, the workflow owner (comparable to a deposit owner in Git) connects to the server from his client. He creates his workflow by specifying all necessary informations (the workflow name, the overall grammar, different participants, their rights and their views, the basic document and the workflow's circuit), then triggers the process. Next, participants concerned by the newly created workflow receive an alert message from the system, inviting them to participate. Each participant must therefore connect himself to the server to obtain a partial replica of the workflow model (encoded in a specification file written in a dedicated DSL) and state (his local document model, a partial replica of the initial document, etc.) according to his rights and his view on the given workflow. A given participant performs his duties and submits his local (partial) replica to the central server which performs synchronisations as soon as possible and the process continues (see fig. \ref{chap2:fig:badouel-tchoupe-workflow}) until the end.  For specific needs (authentication, access to corporate data, etc.), clients and server may require the intervention of an administration tool (database, paperwork and many others). These three tiers are interconnected around a middleware as presented in figure \ref{chap2:fig:architecture}.
\begin{figure}[ht!]
	\noindent
	\makebox[\textwidth]{\includegraphics[scale=0.3]{Chap2/images/architectureEn.png}}
	\caption{A software architecture (three-tiers) for centralised management of structured documents' cooperative editing workflows.}
	\label{chap2:fig:architecture}
\end{figure}


\mySubSubSection{Server Architecture}{}
\label{chap2:sec:archi-server-architecture}
The server is responsible for the storage, restoration, execution and monitoring of workflows. Its architecture is based on three basic elements as shown in figure \ref{chap2:fig:architecture}(a) : its \textit{model}, \textit{storage module} and its \textit{runtime engine}.
\begin{enumerate}
\item \textit{The model}: it is the one orchestrating all the tasks supported by the server. It consists of a workflow engine, a set of parsers and three communication interfaces (the interface with the middleware, that with the storage module and the one with the runtime engine).
\item \textit{The storage module}: it is responsible for the storage of workflows. Like CVS, it maintains a main repository for each workflow. The repository space of a given workflow includes its specification file written in a DSL. There are also (global) document versions showing the state of the workflow at given times. These versions of the underlying documents, facilitate the control and monitoring of workflows.
\item \textit{The runtime engine}: it consists of implementations of projection, expansion and consensual merging algorithms. These implementations are used by the workflow engine in the realisation of these tasks. A runtime engine written entirely in Haskell, was proposed in \cite{artTinyCE}. However, it is quite rigid and almost impossible to adapt to the architecture presented here. To this end, we present in section \ref{chap2:sec:archi-tinyce-v2-cross-fertilisation}, a more flexible version of the latter.
\end{enumerate}

\mySubSubSection{Client Architecture}{}
\label{chap2:sec:archi-client-architecture}
The client (figure \ref{chap2:fig:architecture}(b)) is also based on three entities: a \textit{model}, an \textit{editing engine} and a \textit{storage module}. The model is responsible for organising and controlling the execution of tasks and user commands. For each new local workflow, the model generates an editing environment which is used by the editing engine to provide conventional facilities of structured document editors (compliance check, syntax highlighting, graphical editing of documents presentations, etc.). Each workflow is locally represented by a specification file and by one structured document representing the current perception of the overall workflow from the current local site. When reaching synchronisation phases, the local structured document is forwarded to the server site, where it is merged with others, in one structured document representing the current state of the overall workflow : it is therefore, a coordination support between the workflow engines of the client and of the server.


\mySubSubSection{The Middleware}{}
\label{chap2:sec:archi-middleware}
The middleware is responsible for the interaction between different tiers of our architecture. It must be designed so that, the coupling between these tiers is as weak as possible. One can for this purpose, consider a SOA in which:
\begin{itemize}
\item Our clients are service clients;
\item The server is a service provider for clients and a client of services offered by the administration tools;
\item The administration tools are service providers.
\end{itemize}
With such an architecture, we can guarantee the independence of each tier and thus, an easier maintenance.


\mySubSection{TinyCE v2}{}
\label{chap2:sec:archi-tinyce-v2}

\mySubSubSection{Presentation of TinyCE v2}{}
\label{chap2:sec:archi-tinyce-v2-presentation}
Due to its technical nature and to the number of technologies it needs for its instantiation, the architecture presented above has not yet been fully implemented. However, many of its components have already been implemented and tested in a test project called TinyCE v2\footnote{TinyCE v2 is a more advanced version of TinyCE \cite{artTinyCE}.} (a Tiny Cooperative Editor version 2).

TinyCE v2 is an editor prototype providing graphic and cooperative editing of the abstract structure of structured documents. It is used following a networked client-server model. Its user interface offers to the user, facilities for the creation of workflows (documents, grammars, actors and views (see fig. \ref{chap2:fig:workflow-creation})), edition and validation of partial replicas (see fig. \ref{chap2:fig:user-connexion}). 
Moreover, this interface also offers the functionality to experiment the concepts of projection, expansion and consensual merging (see fig. \ref{chap2:fig:workflow-merging}). TinyCE v2 is designed using Java and Haskell languages. It offers several implementations of our architecture concepts namely: parsers, storage modules, server's runtime engine, workflow engines and communication interfaces.
\begin{figure}[ht!]
	\noindent
	\makebox[\textwidth]{\includegraphics[scale=0.85]{Chap2/images/WorkflowCreationEn.png}}
	\caption{Some screenshots showing the creation process of a cooperative editing workflow in TinyCE v2.}
	\label{chap2:fig:workflow-creation}
\end{figure}

\begin{figure}[ht!]
	\noindent
	\makebox[\textwidth]{\includegraphics[scale=0.78]{Chap2/images/connexionUtilisateurEn.png}}
	\caption{Some screenshots of TinyCE v2 showing the authentication window of a co-author (Auteur1) as well as those displaying the various local and remote workflows in which he is implicated.}
	\label{chap2:fig:user-connexion}
\end{figure}

\begin{figure}[ht!]
	\noindent
	\makebox[\textwidth]{\includegraphics[scale=0.55]{Chap2/images/exemplefusionEn.png}}
	\caption{An illustration of consensual merging in TinyCE v2.}
	\label{chap2:fig:workflow-merging}
\end{figure}

\mySubSubSection{Java-Haskell Cross-Fertilisation in TinyCE v2}{}
\label{chap2:sec:archi-tinyce-v2-cross-fertilisation}
As in \cite{artTinyCE}, the runtime engine of TinyCE v2 exploits the possibility offered by Java, to run an "external program".  Indeed, we designed an interface of TinyCE v2 (runtime interface) capable of launching a Haskell interpreter (GHCi - Glasgow Haskell Compiler interactive\footnote{Official website of GHC: \url{http://www.haskell.org/ghc/}, visited the 04/04/2020.} - in this case) and make it execute various commands. When creating a workflow, TinyCE v2 generates a Haskell program file (.hs), containing data types and functions necessary to achieve the operations of projection, expansion and consensual merging on the structured document representing the state of that workflow. In this way, we considerably reduce the use frequency of parsers presented in \cite{artTinyCE}. The functions are more open to changes as they are contained in a text file and not in a compiled program as in \cite{artTinyCE}. In fact, the main differences between our Java-Haskell cross-fertilisation approach and the one of \cite{artTinyCE} are almost the same that drive the debates on interpreted and compiled languages; our approach is likened to interpreted languages and that of \cite{artTinyCE}, to compiled languages. So, even though our approach can present security risks (that can be addressed using PKI (Public Key Infrastructure) and standard encryption systems like AES (Advanced Encryption Standard), RSA (Rivest Shamir Adleman), etc.), it has the advantage of being portable and easier to maintain. 
\begin{comment}
We present a brief summary of this approach below.

~

\noindent\textbf{\textit{Implementation of the Runtime Interface}}

Haskell's GHC implementation is not just a compiler. It is therefore possible to create functions (programs) and have them executed (interpreted) in interactive mode thanks to its GHCi module. A Haskell program under GHC is a well-designed file with the extension \textit{.hs} similar to the one below, stored for example in a file named \textit{helloGHC.hs}:
\begin{Verbatim}[fontsize=\small, numbers=left, numbersep=8pt]
module HelloGHC where
sayHello name = "Hello "++name
\end{Verbatim}

To execute this program in interactive mode, one just has to load the module with the command \Verb|:load| or its shortcut \Verb|:l| then to call the function \Verb|sayHello| in the following way (where \Verb|xxx| is an argument):
\begin{Verbatim}[fontsize=\small, numbers=left, numbersep=8pt]
:load "helloGHC.hs"
sayHello xxx
\end{Verbatim}

Java allows to launch any executable program from a java code, and thus, to launch the GHCi interpreter of GHC. To execute a program in Java one can use the class \Verb|ILanguageRunner| defined in algorithm \ref{chap2:algo:java-interpreter}. 
\begin{algorithm}
\small
\caption{A generic Java class to run any executable file.}
\label{chap2:algo:java-interpreter}
\begin{Verbatim}[fontsize=\small, numbers=left, numbersep=8pt]
import java.io.*;
public class ILanguageRunner{
	protected Process process = null;
	protected String language;
	protected String command;
	public ILanguageRunner(String language, String command){
		this.language = language;
		this.command = command;
	}
	public void startExecProcess() throws IOException{
		ProcessBuilder processBuilder = new ProcessBuilder(command);
		process = processBuilder.start();
	}
	public void killExecProcess(){
		if(process != null)
			process.destroy();
	}
	public void setExecCode(String code) throws IOException{...}
	public void getExecErrors() throws IOException{...}
	public String getExecResult() throws IOException{...}
	public String executeCode(String code) throws IOException{
		startExecProcess();
		setExecCode(code);
		getExecErrors();
		String result = getExecResult();
		killExecProcess();
		return result;
	}
}
\end{Verbatim}
\end{algorithm}
In this class,
\begin{itemize}
	\item the attributes \Verb|process|, \Verb|language| and \Verb|command| (algorithm \ref{chap2:algo:java-interpreter}, lines 3, 4 and 5) respectively represent the Java process allowing to launch the executable, the current language\footnote{Let's note that this code is designed for the execution of all types of interactive interpreters and not only for the execution of GHCi.} and the path to the executable;
	\item the method \Verb|startExecProcess| (algorithm \ref{chap2:algo:java-interpreter}, lines 10 to 13) builds the Java process from the current command. The method \Verb|killExecProcess| (algorithm \ref{chap2:algo:java-interpreter}, lines 14 to 17) destroys the process when it exists;
\end{itemize}
\begin{algorithm}
\small
\caption{The method setExecCode of the class ILanguageRunner.}
\label{chap2:algo:java-interpreter-setExecCode}
\begin{Verbatim}[fontsize=\small, numbers=left, numbersep=8pt]
public void setExecCode(String code) throws IOException{
	OutputStream stdin = process.getOutputStream();
	OutputStreamWriter stdinWriter = new OutputStreamWriter(stdin);
	try{
		stdinWriter.write(code);
	}finally{
		try{stdinWriter.close();}catch(IOException e){}
		try{stdin.close();}catch(IOException e){}
	}
}
\end{Verbatim}
\end{algorithm}
\begin{itemize}
	\item the method \Verb|setExecCode| (algorithm \ref{chap2:algo:java-interpreter-setExecCode}) makes it possible to write the code to be executed on the standard input of the interpreter (this is the passing of arguments to the executed program);
\end{itemize}
\begin{algorithm}
\small
\caption{The method getExecErrors of the class ILanguageRunner.}
\label{chap2:algo:java-interpreter-getExecErrors}
\begin{Verbatim}[fontsize=\small, numbers=left, numbersep=8pt]
public void getExecErrors() throws IOException{
	InputStream stdout = process.getErrorStream();
	InputStreamReader stdoutReader = new InputStreamReader(stdout);
	BufferedReader stdoutBuffer = new BufferedReader (stdoutReader);
	StringBuffer errorBuffer = null;
	try{
		String line = null;
		while((line = stdoutBuffer.readLine()) != null){
			if (errorBuffer == null)
				errorBuffer = new StringBuffer();
			errorBuffer.append(line);
		}
	}finally{
		try{stdoutBuffer.close();}catch(IOException e){}
		try{stdoutReader.close();}catch(IOException e){}
		try{stdout.close();}catch(IOException e){}
	}
	if(errorBuffer != null)
		throw new IOException(errorBuffer.toString());
}
\end{Verbatim}
\end{algorithm}
\begin{itemize}
	\item the method \Verb|getExecErrors| (algorithm \ref{chap2:algo:java-interpreter-getExecErrors}) makes it possible to recover the possible errors occurring during the execution of the code defined by \Verb|setExecCode| (the errors are returned in the form of a Java exception);
	\item the method \Verb|getExecResult| (algorithm \ref{chap2:algo:java-interpreter-getExecResult}) makes it possible to recover the possible results of the execution of the code defined by \Verb|setExecCode|;
\end{itemize}
\begin{algorithm}
\small
\caption{The method getExecResult of the class ILanguageRunner.}
\label{chap2:algo:java-interpreter-getExecResult}
\begin{Verbatim}[fontsize=\small, numbers=left, numbersep=8pt]
public String getExecResult() throws IOException{
	InputStream stdout = process.getInputStream();
	InputStreamReader stdoutReader = new InputStreamReader(stdout);
	BufferedReader stdoutBuffer = new BufferedReader(stdoutReader);
	StringBuffer resultBuffer = null;
	try{
		String line = null;
		while((line = stdoutBuffer.readLine()) != null){
			if (resultBuffer == null)
				resultBuffer = new StringBuffer();
			resultBuffer.append(line);
		}
	}finally{
		try{stdoutBuffer.close();}catch(IOException e){}
		try{stdoutReader.close();}catch(IOException e){}
		try{stdout.close();}catch(IOException e){}
	}
	if(resultBuffer != null)
		return resultBuffer.toString();
	return null;
}
\end{Verbatim}
\end{algorithm}
\begin{itemize}
	\item finally, the method \Verb|executeCode| (algorithm \ref{chap2:algo:java-interpreter}, lines 21 to 28) executes any code. It starts a process, defines the code, recovers the errors if there are any, otherwise it recovers the result and then destroys the process.
\end{itemize}

A use of the above class \Verb|ILanguageRunner| for the execution of Haskell codes in interactive mode by GHCi can be done through the class \Verb|HaskellRunner| which inherits from \Verb|ILanguageRunner| and whose code is given in algorithm \ref{chap2:algo:java-haskel-interpreter}.
\begin{algorithm}
\small
\caption{A specialised Java class to run a Haskell program.}
\label{chap2:algo:java-haskel-interpreter}
\begin{Verbatim}[fontsize=\small, numbers=left, numbersep=8pt]
import java.io.IOException;

public final class HaskellRunner extends ILanguageRunner{
	public HaskellRunner(){
		super("Haskell", "ghci");
	}
	@Override
	public String getExecResult() throws IOException{
		String execResult = super.getExecResult(), 
		       tmpString;
		if(execResult == null)
			return null;
		String[] tab = execResult.split("Prelude> ");
		if(tab.length == 2)
			tab = tab[1].split("\\*[a-zA-Z0-9_]{1,}>");
		StringBuilder goodResult = new StringBuilder();
		for(int i = 1; i < tab.length - 1; i++){
			tmpString = tab[i].trim();
			if(!tmpString.isEmpty()){
				goodResult.append(tmpString);
				goodResult.append("\n");
			}
		}
		return goodResult.toString();
	}
}
\end{Verbatim}
\end{algorithm}

In this class, the name of the command has been set to \Verb|ghci| (this implies that the path of the directory containing the GHCi program must be written in the \Verb|path| environment variable) and that of the language to \Verb|Haskell|. We redefined the method \Verb|getExecResult| to better format the result by removing the superfluous strings added by our current version of GHCi\footnote{HaskellPlatform-2014.2.0.0-i386}.

~

\noindent\textbf{\textit{Example of the Runtime Interface Use}}

One can run the Haskell engine with the code of algorithm \ref{chap2:algo:java-haskel-interpreter-example}. 
\begin{algorithm}
\small
\caption{Running a Haskell function with the proposed "Haskell interpreter".}
\label{chap2:algo:java-haskel-interpreter-example}
\begin{Verbatim}[fontsize=\small, numbers=left, numbersep=8pt]
import java.io.IOException;
public class HelloTinyCE {
	public static void main(String[] args){
		HaskellRunner runner = new HaskellRunner();
		String nom = "\"Ange Frank, Chris Maxime and Yann Alex\"";
		String commande = ":load helloGHC.hs\nsayHello " + nom;
		try{
			String resultat = runner.executeCode(commande);
			System.out.println("The execution result is : " + 
								resultat);
		}catch(IOException ex){
			System.err.println("Some errors occured when" +
					" executing commands.");
		}
	}
}
\end{Verbatim}
\end{algorithm}
Executing this code gives the following result:
\begin{Verbatim}[fontsize=\small, numbers=left, numbersep=8pt]
The execution result is : "Hello Ange Frank, Chris Maxime and Yann Alex"
\end{Verbatim}
If we replace the code on line 7 by the code \textit{String commande = ":load helloGHC.hs $\setminus$nsayHello";} then the result is now the following:
\begin{Verbatim}[fontsize=\small, numbers=left, numbersep=8pt]
Some errors occured when executing commands.
\end{Verbatim}

A version of the class \Verb|HaskellRunner| plays the role of "Haskell interpreter" within TinyCE v2. It allows a bidirectional communication between the TinyCE v2 model (coded in Java) and Haskell following a text-based protocol (well-formed strings following an XML-like coding) that we have implemented.
\end{comment}






% \vspace{-0.5em}
\section{Conclusion}
% \vspace{-0.5em}
Recent advances in multimodal single-cell technology have enabled the simultaneous profiling of the transcriptome alongside other cellular modalities, leading to an increase in the availability of multimodal single-cell data. In this paper, we present \method{}, a multimodal transformer model for single-cell surface protein abundance from gene expression measurements. We combined the data with prior biological interaction knowledge from the STRING database into a richly connected heterogeneous graph and leveraged the transformer architectures to learn an accurate mapping between gene expression and surface protein abundance. Remarkably, \method{} achieves superior and more stable performance than other baselines on both 2021 and 2022 NeurIPS single-cell datasets.

\noindent\textbf{Future Work.}
% Our work is an extension of the model we implemented in the NeurIPS 2022 competition. 
Our framework of multimodal transformers with the cross-modality heterogeneous graph goes far beyond the specific downstream task of modality prediction, and there are lots of potentials to be further explored. Our graph contains three types of nodes. While the cell embeddings are used for predictions, the remaining protein embeddings and gene embeddings may be further interpreted for other tasks. The similarities between proteins may show data-specific protein-protein relationships, while the attention matrix of the gene transformer may help to identify marker genes of each cell type. Additionally, we may achieve gene interaction prediction using the attention mechanism.
% under adequate regulations. 
% We expect \method{} to be capable of much more than just modality prediction. Note that currently, we fuse information from different transformers with message-passing GNNs. 
To extend more on transformers, a potential next step is implementing cross-attention cross-modalities. Ideally, all three types of nodes, namely genes, proteins, and cells, would be jointly modeled using a large transformer that includes specific regulations for each modality. 

% insight of protein and gene embedding (diff task)

% all in one transformer

% \noindent\textbf{Limitations and future work}
% Despite the noticeable performance improvement by utilizing transformers with the cross-modality heterogeneous graph, there are still bottlenecks in the current settings. To begin with, we noticed that the performance variations of all methods are consistently higher in the ``CITE'' dataset compared to the ``GEX2ADT'' dataset. We hypothesized that the increased variability in ``CITE'' was due to both less number of training samples (43k vs. 66k cells) and a significantly more number of testing samples used (28k vs. 1k cells). One straightforward solution to alleviate the high variation issue is to include more training samples, which is not always possible given the training data availability. Nevertheless, publicly available single-cell datasets have been accumulated over the past decades and are still being collected on an ever-increasing scale. Taking advantage of these large-scale atlases is the key to a more stable and well-performing model, as some of the intra-cell variations could be common across different datasets. For example, reference-based methods are commonly used to identify the cell identity of a single cell, or cell-type compositions of a mixture of cells. (other examples for pretrained, e.g., scbert)


%\noindent\textbf{Future work.}
% Our work is an extension of the model we implemented in the NeurIPS 2022 competition. Now our framework of multimodal transformers with the cross-modality heterogeneous graph goes far beyond the specific downstream task of modality prediction, and there are lots of potentials to be further explored. Our graph contains three types of nodes. while the cell embeddings are used for predictions, the remaining protein embeddings and gene embeddings may be further interpreted for other tasks. The similarities between proteins may show data-specific protein-protein relationships, while the attention matrix of the gene transformer may help to identify marker genes of each cell type. Additionally, we may achieve gene interaction prediction using the attention mechanism under adequate regulations. We expect \method{} to be capable of much more than just modality prediction. Note that currently, we fuse information from different transformers with message-passing GNNs. To extend more on transformers, a potential next step is implementing cross-attention cross-modalities. Ideally, all three types of nodes, namely genes, proteins, and cells, would be jointly modeled using a large transformer that includes specific regulations for each modality. The self-attention within each modality would reconstruct the prior interaction network, while the cross-attention between modalities would be supervised by the data observations. Then, The attention matrix will provide insights into all the internal interactions and cross-relationships. With the linearized transformer, this idea would be both practical and versatile.

% \begin{acks}
% This research is supported by the National Science Foundation (NSF) and Johnson \& Johnson.
% \end{acks}

	\mathversion{normal}
	\mathversion{normal2}
	\myChapter{A Choreography-like Workflow Design and Distributed Execution Framework Based on Structured Mobile Artifacts' Cooperative Editing}{}
\label{chap3:choreography-workflow-design-execution}
\myMiniToc{section}{Contents}
% If no minitoc then
% \startcontents[chapters]
% \leavevmode
% \\
% \\
% \\
% \\
% \\
\section{Introduction}
\label{introduction}

AutoML is the process by which machine learning models are built automatically for a new dataset. Given a dataset, AutoML systems perform a search over valid data transformations and learners, along with hyper-parameter optimization for each learner~\cite{VolcanoML}. Choosing the transformations and learners over which to search is our focus.
A significant number of systems mine from prior runs of pipelines over a set of datasets to choose transformers and learners that are effective with different types of datasets (e.g. \cite{NEURIPS2018_b59a51a3}, \cite{10.14778/3415478.3415542}, \cite{autosklearn}). Thus, they build a database by actually running different pipelines with a diverse set of datasets to estimate the accuracy of potential pipelines. Hence, they can be used to effectively reduce the search space. A new dataset, based on a set of features (meta-features) is then matched to this database to find the most plausible candidates for both learner selection and hyper-parameter tuning. This process of choosing starting points in the search space is called meta-learning for the cold start problem.  

Other meta-learning approaches include mining existing data science code and their associated datasets to learn from human expertise. The AL~\cite{al} system mined existing Kaggle notebooks using dynamic analysis, i.e., actually running the scripts, and showed that such a system has promise.  However, this meta-learning approach does not scale because it is onerous to execute a large number of pipeline scripts on datasets, preprocessing datasets is never trivial, and older scripts cease to run at all as software evolves. It is not surprising that AL therefore performed dynamic analysis on just nine datasets.

Our system, {\sysname}, provides a scalable meta-learning approach to leverage human expertise, using static analysis to mine pipelines from large repositories of scripts. Static analysis has the advantage of scaling to thousands or millions of scripts \cite{graph4code} easily, but lacks the performance data gathered by dynamic analysis. The {\sysname} meta-learning approach guides the learning process by a scalable dataset similarity search, based on dataset embeddings, to find the most similar datasets and the semantics of ML pipelines applied on them.  Many existing systems, such as Auto-Sklearn \cite{autosklearn} and AL \cite{al}, compute a set of meta-features for each dataset. We developed a deep neural network model to generate embeddings at the granularity of a dataset, e.g., a table or CSV file, to capture similarity at the level of an entire dataset rather than relying on a set of meta-features.
 
Because we use static analysis to capture the semantics of the meta-learning process, we have no mechanism to choose the \textbf{best} pipeline from many seen pipelines, unlike the dynamic execution case where one can rely on runtime to choose the best performing pipeline.  Observing that pipelines are basically workflow graphs, we use graph generator neural models to succinctly capture the statically-observed pipelines for a single dataset. In {\sysname}, we formulate learner selection as a graph generation problem to predict optimized pipelines based on pipelines seen in actual notebooks.

%. This formulation enables {\sysname} for effective pruning of the AutoML search space to predict optimized pipelines based on pipelines seen in actual notebooks.}
%We note that increasingly, state-of-the-art performance in AutoML systems is being generated by more complex pipelines such as Directed Acyclic Graphs (DAGs) \cite{piper} rather than the linear pipelines used in earlier systems.  
 
{\sysname} does learner and transformation selection, and hence is a component of an AutoML systems. To evaluate this component, we integrated it into two existing AutoML systems, FLAML \cite{flaml} and Auto-Sklearn \cite{autosklearn}.  
% We evaluate each system with and without {\sysname}.  
We chose FLAML because it does not yet have any meta-learning component for the cold start problem and instead allows user selection of learners and transformers. The authors of FLAML explicitly pointed to the fact that FLAML might benefit from a meta-learning component and pointed to it as a possibility for future work. For FLAML, if mining historical pipelines provides an advantage, we should improve its performance. We also picked Auto-Sklearn as it does have a learner selection component based on meta-features, as described earlier~\cite{autosklearn2}. For Auto-Sklearn, we should at least match performance if our static mining of pipelines can match their extensive database. For context, we also compared {\sysname} with the recent VolcanoML~\cite{VolcanoML}, which provides an efficient decomposition and execution strategy for the AutoML search space. In contrast, {\sysname} prunes the search space using our meta-learning model to perform hyperparameter optimization only for the most promising candidates. 

The contributions of this paper are the following:
\begin{itemize}
    \item Section ~\ref{sec:mining} defines a scalable meta-learning approach based on representation learning of mined ML pipeline semantics and datasets for over 100 datasets and ~11K Python scripts.  
    \newline
    \item Sections~\ref{sec:kgpipGen} formulates AutoML pipeline generation as a graph generation problem. {\sysname} predicts efficiently an optimized ML pipeline for an unseen dataset based on our meta-learning model.  To the best of our knowledge, {\sysname} is the first approach to formulate  AutoML pipeline generation in such a way.
    \newline
    \item Section~\ref{sec:eval} presents a comprehensive evaluation using a large collection of 121 datasets from major AutoML benchmarks and Kaggle. Our experimental results show that {\sysname} outperforms all existing AutoML systems and achieves state-of-the-art results on the majority of these datasets. {\sysname} significantly improves the performance of both FLAML and Auto-Sklearn in classification and regression tasks. We also outperformed AL in 75 out of 77 datasets and VolcanoML in 75  out of 121 datasets, including 44 datasets used only by VolcanoML~\cite{VolcanoML}.  On average, {\sysname} achieves scores that are statistically better than the means of all other systems. 
\end{itemize}


%This approach does not need to apply cleaning or transformation methods to handle different variances among datasets. Moreover, we do not need to deal with complex analysis, such as dynamic code analysis. Thus, our approach proved to be scalable, as discussed in Sections~\ref{sec:mining}.
\mySection{Overview of the Artifact-Centric Model Presented in this Thesis}{}
\label{chap3:sec:model-overview}
In this section, a brief description of the artifact-centric model studied in this chapter is given. Furthermore, an overview of the distributed execution of the peer-review process using this model is presented.

\mySubSection{A Description of the Artifact-Centric Model Presented in this Thesis}{}
\label{chap3:sec:model-description}
We outlined in this chapter's introduction that the presented artifact-centric model is based on the asynchronous structured cooperative editing techniques proposed in the work of Badouel et al. As in these works, an artifact is represented by a tree containing "\textit{open nodes}" on some of its leaves, materialising the tasks to be executed or being executed and, an attributed grammar called the \textit{Grammatical Model of Workflow} (GMWf) is used as \textit{artifact type}. The symbols of a given GMWf represent the process tasks and each of its productions represents a scheduling of a subset of these tasks; intuitively, a production given by its left and right hand sides, specifies how the task on the left hand side precedes (must be executed before) those on the right hand side (see sec. \ref{chap3:sec:artifacts-structure}).
When a task is executed on a given site, the corresponding open node in the artifact is closed accordingly (it is said to be \textit{closed}) and the data produced during execution are filled in its attributes; then, one of the GMWf's production having the considered task as left hand side is chosen by the local actor to expand the open node into a subtree highlighting in the form of new open nodes, the new tasks to be executed: this is what editing an artifact consists of.
\begin{figure}[ht!]
	\noindent
	\makebox[\textwidth]{\includegraphics[scale=0.27]{./Chap3/images/protocole.png}}
	\caption{An overview of the artifact-centric BPM model presented in this chapter.}
	\label{chap3:fig:overview-protocol}
\end{figure}

We are especially interested on administrative processes and the approach we propose for their automation is declined in two steps: derivation of different models (target artifacts and their model, accreditations, etc.) from a textual description of the process and, implementation of a choreography between the agents communicating by asynchronous exchange of artifacts for its execution. More precisely, from the observation that one can analyse the textual description of an administrative business process to exhibit all the possible execution scenarios leading to its business goals, we propose to model each of these scenarios by an annotated tree in which, each node corresponds to a task of the process assigned to a given agent, and each hierarchical decomposition (a node and its sons) represents a scheduling of these tasks: these annotated trees are called \textit{target artifacts}. From these target artifacts, are derived a GMWf (\textit{artifact type}) which contains both the \textit{information model} (modelled by its attributes) and the \textit{lifecycle model} (thanks to the set of its productions) which are two essential notions of the artifact-centric modelling paradigm \cite{hull2013data}. Once the GMWf is obtained, we propose to add organisational information called \textit{accreditations} in this chapter; they aim, as in \cite{badouelTchoupeCmcs, theseTchoupe, tchoupeAtemkeng2, tchoupeZekeng2016, tchoupeZekeng2017, zekengTchoupe2018}, is to enrich the notion of access to different parts of artifacts, by offering a simple mechanism for modelling the generally different perceptions that actors have on processes and their data. With the couple (GMWf, accreditations), each autonomous agent is configured (see fig. \ref{chap3:fig:overview-protocol} (1)) and is ready to proceed to the decentralised execution of the studied process.

The actual execution is a choreography in which the agents are reactive autonomous software components, communicating in P2P mode and are  driven by human agents (actors) in charge of executing tasks. An agent's reaction to the reception of a message (an artifact) consists in the execution of a five-step protocol clearly described in this chapter (see sec. \ref{chap3:sec:the-protocols}). 
This protocol allows it to: (1) \textit{merge} the received artifact with the one it hosts locally in order to consider all updates, (2) \textit{project} the artifact resulting from the merger in order to hide the parts to which the local actor may not have access and highlight the tasks to be locally executed, (3) make the local actor \textit{execute} the revealed tasks and thus edit the potentially partial replica of the artifact obtained after the projection, (4) integrate the new updates into the artifact through an operation called \textit{expansion-pruning} and finally, (5) \textit{diffuse} the updated artifact to other sites for further execution of the process if necessary. 
The agents' operational capabilities allow that, for the execution of a given process, an artifact created by one of them (initially reduced to an open node), moves from site to site to indicate tasks that are ready to be executed at the appropriate time and to provide necessary data (created by other agents) for that execution; the mobile artifact, cooperatively edited by agents, thus "grows" as it transits through the distributed system so formed (see fig. \ref{chap3:fig:overview-protocol} (2)).


\mySubSection{The Running Example: the Peer-Review Process}{}
\label{chap3:sec:running-example}

\mySubSubSection{Description of the Peer-Review Process}{}
\label{chap3:sec:peer-review-description}
The peer-review process \cite{peerReview02} is a common example of administrative business process. We presented a brief description of it inspired by those made in \cite{peerReview02, van2001proclets, badouel14}, in chapter \ref{chap1:artifact-centric-bpm}, section \ref{chap1:sec:running-example}. Described in this way, we will use the peer-review process as an illustrative example in this chapter.

Lets recall that from that description, we have identified all the tasks to be executed, their sequencing, actors involved and the tasks assigned to them. Precisely, four actors are involved: an editor in chief ($EC$) who is responsible for initiating the process, an associated editor ($AE$) and two referees ($R1$ and $R2$).
A summary of tasks assignment was presented in table \ref{tableau:tachesExecutant} (page \pageref{tableau:tachesExecutant}), and orchestration diagrams using BPMN and WF-Net were also presented in figure \ref{chap1:fig:comparing-workflow-languages} (page \pageref{chap1:fig:comparing-workflow-languages}).


\mySubSubSection{Overview of the Peer-Review Process Artifact-Centric Execution using the Model Presented in this Thesis}{}
\label{chap3:sec:peer-review-overview}
To run the peer-review process described above according to the artifact-centric model presented in this chapter, four agents controlled by four actors (\textit{the editor in chief}, \textit{the associated editor} and the \textit{two referees}) will be deployed. 
Each of them will be pre-configured using a global Grammatical Model of Workflow (GMWf) and a set of accreditations. As we will see later, the global GMWf is used as model of artifacts and formally describes all the process tasks to be executed as well as their execution order (see fig. \ref{chap1:fig:comparing-workflow-languages}), and the accreditations set specifies the permissions (\textit{reading}, \textit{writing} and \textit{execution}) of each of the four actors relative to these tasks. 
After the pre-configuration of agents, each of them will derive (by projection \cite{tchoupeAtemkeng2}) a local GMWf which will locally guide the execution of the tasks to guarantee the confidentiality of some workflow data (contained in a mobile artifact) and the consistency of local updates with the global GMWf.

The artifact-centric execution of a scientific paper validation workflow will be triggered on the editor in chief's site, by introducing (in this site) an artifact (an annotated tree) reduced to its root node. Each node of the artifact represents a task and encapsulates an attribute containing its execution status. Therefore at a given time, the whole artifact contains information on already executed tasks and on data produced during their execution; it also exhibits tasks that are ready to be executed.
The analysis of this artifact by the local agent will highlight the expected contributions from the editor in chief. Guided by the local GMWf, he (here tasks are executed by a human) will perform tasks resulting in the consistent updating of the artifact's local copy; meaning that new nodes will be added to the artifact and some of its existing nodes will be updated: this is what we call editing an artifact. Then, this (updated) copy will be immediately analysed by the local agent to determine whether the currently managed process scenario is complete (this is the case when the artifact local copy structure matches one of the target artifacts: all causally dependent tasks have been executed) or not: in this case, all sites on which execution must continue are determined and an execution request is addressed to each of them (the artifact is sent to them).
Figure \ref{chap3:fig:overview-example} sketches an overview of exchanges that can take place between the four agents of the peer-review process when validating a scientific paper using the model presented in the present chapter. The scenario presented there, corresponds to the one in which the paper is pre-validated by the editor in chief and therefore, is analysed by a peer review committee. Note that there may be situations where multiple copies of the artifact are updated in parallel; this is notably the case when they are present on site 3 (first referee) and 4 (second referee).
\begin{figure}[ht!]
	\noindent
	\makebox[\textwidth]{\includegraphics[scale=0.28]{./Chap3/images/figOverviewExample.png}}
	\caption{An overview of the artifact-centric execution of the peer-review process using the model presented in this chapter.}
	\label{chap3:fig:overview-example}
\end{figure}

\mySection{Modelling Artifacts}{}
\label{chap3:sec:modelling-artifacts}

\mySubSection{Artifacts' Structure}{}
\label{chap3:sec:artifacts-structure}
Let's consider an administrative process $\mathcal{P}_{op}$ to be automated. The set $\left\{ \mathcal{S}_{op}^1,\ldots,\mathcal{S}_{op}^k \right\}$ of $\mathcal{P}_{op}$ execution scenarios is known in advance and so, $\mathcal{P}_{op}$ can be specified as any oriented graph with tools like BPMN or as a petri net with tools like YAWL. 
Moreover, each execution scenario of $\mathcal{P}_{op}$ can be modelled using an annotated tree $t_i$. Indeed, starting from the fact that a given scenario $\mathcal{S}_{op}^i$ consists of a set $\mathbb{T}_n = \{X_1, \ldots, X_n\}$ of $n$ (non-recursive) tasks to be executed in a specific order (in parallel or in sequence), one can represent $\mathcal{S}_{op}^i$ as a tree $t_i$ in which each node (labelled $X_i$) potentially corresponds to a task $X_i$ of $\mathcal{S}_{op}^i$, and each hierarchical decomposition (a node and its sons) corresponds to a scheduling: the task associated with the parent node must be executed before those associated with the son nodes; the latter must be executed according to an order - parallel or sequential - that can be specified by particular annotations. Indeed, it is enough to have two annotations "$\fatsemi$" (is sequential to) and "$\parallel$" (is parallel to) to be applied to each hierarchical decomposition. The annotation "$\fatsemi$" (resp. "$\parallel$") reflects the fact that the tasks associated with the son nodes of the decomposition must (resp. can) be executed in sequence (resp. in parallel).

Considering the running example (the peer-review process), the two scenarios that make it up can be modelled using the two annotated trees in figure \ref{chap3:fig:global-artefacts}. In particular, we can see that the tree $art_1$ shows how the task "Receipt and pre-validation of a submitted paper" assigned to the editor in chief ($EC$), and associated with the symbol $A$ (see table \ref{tableau:tachesExecutant}, page \pageref{tableau:tachesExecutant}), must be executed before tasks associated with the symbols $B$ and $D$, that are to be executed in sequence. This annotated tree represents the scenario where the paper received by the editor in chief, is immediately rejected.
\begin{figure}[ht!]
	\noindent
	\makebox[\textwidth]{\includegraphics[scale=0.6]{./Chap3/images/artefactsGlobaux.png}}
	\caption{Target artifacts of a peer-review process.}
	\label{chap3:fig:global-artefacts}
\end{figure}


\mySubSection{Target Artifacts and Grammatical Model of Workflow}{}
\label{chap3:sec:target-artifacts-and-gmwf}
In this chapter, we use the expression \textit{target artifact} to designate the annotated tree $t_i$ modelling a given scenario $\mathcal{S}_{op}^i$ of a given administrative process $\mathcal{P}_{op}$. From the set of target artifacts of a given process, it is possible to derive an abstract grammar\footnote{It is enough to consider the set of target artifacts as a regular tree language: there is therefore a (abstract) grammar to generate them.} that can be enriched to serve as a \textit{artifact type} as defined in \cite{hull2009facilitating}: it is this grammar that we designate by the expression \textit{Grammatical Model of Workflow (GMWf)}.

Let's consider the set $\left\{t_1,\ldots,t_k\right\}$ of target artifacts modelling the $k$ execution scenarios of a given process $\mathcal{P}_{op}$ of $n$ tasks ($\mathbb{T}_n = \{X_1, \ldots, X_n\}$). Each $t_i$ is a derivation tree for an abstract grammar (a GMWf) $\mathbb{G}=\left(\mathcal{S},\mathcal{P},\mathcal{A}\right)$ whose set of symbols is $\mathcal{S}=\mathbb{T}_n$ (all process tasks) and each production $p \in \mathcal{P}$ reflects a hierarchical decomposition contained in at least one of the target artifacts. Each production is therefore exclusively of one of the following two forms: $p: X_0 \rightarrow X_1 \fatsemi \ldots \fatsemi X_n$ or  $p: X_0 \rightarrow X_1 \parallel \ldots \parallel X_n$. The first form $p: X_0 \rightarrow X_1 \fatsemi \ldots \fatsemi X_n$ (resp. the second form $p: X_0 \rightarrow X_1 \parallel \ldots \parallel X_n$) means that task $X_0$ must be executed before tasks $\left\{X_1,\ldots,X_n\right\}$, and these must be (resp. these can be) executed in sequence (resp. in parallel). A GMWf can therefore be formally defined as follows:
\begin{definition}
	\label{defGMWf1}
	A \textbf{Grammatical Model of Workflow} (GMWf) is defined by $\mathbb{G}=\left(\mathcal{S},\mathcal{P},\mathcal{A}\right)$
	where:
	\begin{itemize}
	\item $\mathcal{S}$ is a finite set of \textbf{grammatical symbols} or \textbf{sorts} corresponding to various \textbf{tasks} to be executed in the studied business process; 
	\item $\mathcal{A}\subseteq\mathcal{S}$ is a finite set of particular symbols called \textbf{axioms}, representing tasks that can start an execution scenario (roots of target artifacts), and 
	\item $\mathcal{P}\subseteq\mathcal{S}\times\mathcal{S}^{*}$ is a finite set of \textbf{productions} decorated by the annotations "$\fatsemi$" (is sequential to) and "$\parallel$" (is parallel to): they are \textbf{precedence rules}. 
	A production $P=\left(X_{P(0)},X_{P(1)},\cdots, X_{P(|P|)}\right)$ is either of the form $P: X_0 \rightarrow X_1 \fatsemi \ldots \fatsemi X_{|P|}$, or of the form $P: X_0 \rightarrow X_1 \parallel \ldots \parallel X_{|P|}$ and $\left|P\right|$ 
	designates the length of $P$'s right-hand side.
	A production with the symbol $X$ as left-hand side is called a \textit{X-production}.
	\end{itemize}
\end{definition}

Let's illustrate the notion of GMWf by considering the one generated from an analysis of the target artifacts obtained in the case of the peer-review process (see fig. \ref{chap3:fig:global-artefacts}). The derived GMWf is  $\mathbb{G}=\left(\mathcal{S},\mathcal{P},\mathcal{A}\right)$ in which, the set $\mathcal{S}$ of grammatical symbols is
$\mathcal{S}=\{A, B, C, D, E, F, G1, G2, H1, H2, I1, I2\}$ (see table \ref{tableau:tachesExecutant});
the only initial task (axiom) is $A$ (then $\mathcal{A}=\{A\}$) and the set $\mathcal{P}$ of productions is:
\[ 
\begin{array}{l|l|l|l}
P_{1}:\; A\rightarrow B\fatsemi D & \; P_{2}:\; A\rightarrow C\fatsemi D\; & \; P_{3}:\; C\rightarrow E\fatsemi F\; & \; P_{4}:\; E\rightarrow G1\parallel G2    \\
P_{5}:\; G1\rightarrow H1 \fatsemi I1 & \; P_{6}:\; G2\rightarrow H2 \fatsemi I2\; & \; P_{7}:\; B\rightarrow \varepsilon\; & \; P_{8}:\; D\rightarrow \varepsilon  \\
P_{9}:\; F\rightarrow \varepsilon & \; P_{10}:\; H1\rightarrow \varepsilon & \; P_{11}:\; I1\rightarrow \varepsilon\; & \; P_{12}:\; H2\rightarrow \varepsilon  \\
P_{13}:\; I2\rightarrow \varepsilon &  &  &   \\
\end{array}
\]

For some administrative business processes, there may be special cases where it is not possible to strictly schedule the tasks of a scenario using the two (only) forms of productions selected for GMWf. For example, this is the case for the scenario of a four-task process with tasks $A, B, C$ and $D$, where the task $A$ precedes all others, the tasks $B$ and $C$ can be executed in parallel and precede $D$. 
In these cases, the introduction of a certain number of new symbols known as \textit{(re)structuring symbols} (not associated with tasks) can make it possible to produce a correct scheduling that respects the form imposed on productions. For the previous example, the introduction of a new symbol $S$ allows us to obtain the following productions: $p_1: A \rightarrow S \fatsemi D$, $p_2: S \rightarrow B \parallel C$, $p_3:B \rightarrow \epsilon$, $p_4:C \rightarrow \epsilon$ and $p_5:D \rightarrow \epsilon$, which properly model the required scheduling. 
To deal with such cases, the previously given GMWf definition (definition \ref{defGMWf1}) is slightly adapted by integrating the (re)structuring symbols; the resulting definition is as follows:
\begin{definition} 
	\label{defGMWf2}
	A \textbf{Grammatical Model of Workflow} (GMWf) is defined by $\mathbb{G}=\left(\mathcal{S},\mathcal{P},\mathcal{A}\right)$
	wherein, 
	$\mathcal{P}$ and $\mathcal{A}$ refer to the same purpose as in definition \ref{defGMWf1}, 
	$\mathcal{S}=\mathcal{T} \cup \mathcal{T}_{Struc}$ 
	is a finite set of \textbf{grammatical symbols} or \textbf{sorts} in which, those of $\mathcal{T}$ correspond to \textbf{tasks} of the studied business process, while those of $\mathcal{T}_{Struc}$ are (re)structuring symbols.
\end{definition}

\mySubSection{Artifact Type and Artifact Edition}{}
\label{chap3:sec:gmwf-as-artifact-type}
As formalised in definition \ref{defGMWf2}, a GMWf perfectly models the tasks and control flow of administrative processes (lifecycle model). To remain faithful to the artifact-centric philosophy, the GMWf definition must be adjusted to be able to use it as an artifact type. In particular, it is necessary to equip it with tools allowing to represent the information model (the data) of processes as well as the dynamic (evolutionary) character of artifacts.

\mySubSubSection{Modelling the Information Model of Processes with GMWf}{}
\label{chap3:sec:gmwf-information-model}
The structure of the consumed and produced data by business processes differs from one process to another. It is therefore not easy to model them using a general type, although several techniques to do so have emerged in recent years \cite{badouel14}. For the work presented in this chapter, tackling the data structure of automated processes has no proven interest because, it does not bring any added value to the proposed model: a representation of these data using a set of variables is largely sufficient.

To represent the potential consumed and produced data by the tasks of a process modelled using GMWf, we use the notion of \textit{attribute} embedded in the nodes associated with tasks. To take them into account, we adjust for the last time, the definition of GMWf. We thus attach to each symbol, an attribute named $status$, allowing to store all the data of the associated task; its precise type is left to the discretion of the process designer. However, for the purposes of this work, we will consider it a string. The new definition of GMWf is thus the following one:
\begin{definition} 
	\label{defGMWf3}
	A \textbf{Grammatical Model of Workflow} (GMWf) is defined by $\mathbb{G}=\left(\mathcal{S},\mathcal{P},\mathcal{A}\right)$
	wherein, 
	$\mathcal{S}$, $\mathcal{P}$ and $\mathcal{A}$ refer to the same purpose as in definition \ref{defGMWf2}.
	Each grammatical symbol $X\in\mathcal{S}$ is associated with an attribute named \textbf{\textit{status}} of type string, that can be updated when task $X$ is executed; $\textbf{X.status}$ provides access (read and write) to its content.
\end{definition}

A GMWf is therefore ultimately an attributed grammar whose instances represent the different execution scenarios of the underlying business process. In artifact-centric models, the artifact used as a communication medium between agents executing the tasks, must represent at each moment, the execution state of the underlying process. As defined up to now, GMWf models do not satisfy this second concern: they cannot therefore be used as artifact types. We will now equip them with tools to allow them to endow their instances (the artifacts) with the ability to report about the execution state of the process they represent.

\mySubSubSection{Artifact Type}{}
\label{chap3:sec:artifact-type}
For each task, it is important to know whether or not it has already been executed; if not, it is also important to know whether or not it is ready to be executed. Recall also that, we model the execution of processes as the desynchronised cooperative editing of mobile artifacts (which are exchanged by agents). This implies that the artifact-centric model of this chapter considers that, an artifact is a structured document that is initially empty, and which is completed as it circulates between the agents. Contrary to the models in the literature, at each moment, the artifact thus contains only a (potentially empty) part of the lifecycle model of the process. This is why we have chosen not to represent it as a (tree) state machine but rather as an annotated tree that is incrementally built in accordance with an attributed grammar.

Concretely, an \textit{artifact} is an annotated tree that potentially contains buds (this is the equivalent of the notion of structured document being edited as presented in chapter \ref{chap2:structured-editing-artifact-type}). A \textit{bud} or \textit{open node} is a typed leaf node indicating in an artifact, a place where an edition is possible; i.e. a node associated with a task that has not yet been executed. A bud can be unlocked (\textit{unlocked bud}) or locked (\textit{locked bud}) depending on whether the task associated with it is ready to be executed\footnote{A task is ready to be executed if all the tasks that precede it according to the precedence constraint set have already been executed and the agent that currently holds the mobile artifact, have the necessary accreditation to trigger its execution.} or not. More formally, a \textit{bud of type $X \in \mathcal{S}$} is a leaf node labelled either by $X_{\overline{\omega}}$ or by $X_\omega$ depending on its state (\emph{locked} or \emph{unlocked}). An artifact containing no buds is said to be \textit{closed}. Such an artifact, symbolises the end of tasks execution with respect to the agent hosting the artifact. An example of an artifact related to the peer-review process and containing buds is shown in figure \ref{chap3:sec:artifact-with-buds}. In this one, we can see that the tasks associated with symbols $A$ and $C$ have already been executed. Task $E$ is ready to be executed while tasks $F$ and $D$ are not ready to be executed yet.
\begin{figure}[ht!]
	\noindent
	\makebox[\textwidth]{\includegraphics[scale=0.4]{./Chap3/images/documentBourgeons.png}}
	\caption{An intentional representation of an annotated artifact containing buds.}
	\label{chap3:sec:artifact-with-buds}
\end{figure}

From the thus given definition of (mobile) artifact, it is clear that an artifact is updated only at the level of its leaves and therefore, it only "grows" (positive editing). Knowing moreover that, the correct and complete execution of a given administrative process corresponds to the execution of one of its scenarios, we deduce that: for a process $\mathcal{P}_{op}$ whose GMWf is $\mathbb{G}=\left(\mathcal{S},\mathcal{P},\mathcal{A}\right)$, a given mobile artifact, is a prefix to one of its target artifacts. Thus, the type (model) of this artifact is a grammar $\mathbb{G}_{\Omega}=(\mathcal{S}\cup\mathcal{S}_{\omega},\mathcal{P}\cup\mathcal{S}_{\Omega},\mathcal{A}\cup\mathcal{A}_{\omega})$ obtained by extending $\mathbb{G}$ (for bud recognition and recognition of all possible prefixes of target artifacts) as follows:
\begin{enumerate}
\item For all sort $X$, add in the set $\mathcal{S}$ of sorts, two new sorts $X_{\overline{\omega}}$ and $X_{\omega}$;

\item For all new sort $X_{\omega}$ added to $\mathcal{S}$, add in the set $\mathcal {P}$ of productions two new $\varepsilon$-productions $X_{\Omega} : X_{\omega} \rightarrow \varepsilon$ and $X_{\overline{\Omega}} : X_{\overline{\omega}} \rightarrow \varepsilon$; 
we then have: $\mathcal{S}_{\omega}=\{X_{\overline{\omega}}, ~X_{\omega},~ X\in\mathcal {S}\}$, $ \mathcal{A}_{\omega}=\{X_{\overline{\omega}}, ~X_{\omega},~ X\in\mathcal{A}\} $ $ and $ $\mathcal{S}_{\Omega} = \{X_{\Omega} : X_{\omega} \rightarrow \varepsilon, ~X_{\overline{\Omega}} : X_{\overline{\omega}} \rightarrow \varepsilon,~ X_{\overline{\omega}}~and~X_{\omega} \in \mathcal{S}_{\omega}\}$.
\end{enumerate}

\mySubSubSection{Artifact Edition}{}
\label{chap3:sec:artifact-edition}
If we still consider a running process $\mathcal{P}_{op}$ whose GMWf is $\mathbb{G}=\left(\mathcal{S},\mathcal{P},\mathcal{A}\right)$, then, the \textit{editing} of an artifact $t$ circulating between agents consists of developing one or more of its buds into a subtree while updating their \textit{status} attributes. Concretely, for a bud ${X_\omega}$ of the said artifact one can:
\begin{enumerate}
\item Execute the task associated with $X$; 

\item Choose an $X$-production $P \in \mathcal{P}$ to be used for the development of ${X_\omega}$;

\item If $P$ is of the form $P:~X \rightarrow X_1 \parallel X_2 \parallel \ldots \parallel X_{|P|}$ (resp. $P:~X \rightarrow X_1 \fatsemi X_2 \fatsemi \ldots \fatsemi X_{|P|}$) then, create $|P|$ buds $X_{1\omega}, X_{2\omega}, \ldots, X_{|P|\omega}$ (resp. $X_{1\omega}, X_{2\overline{\omega}}, \ldots, X_{|P|\overline{\omega}}$) respectively of type $X_1, X_2, \ldots , X_{|P|}$, and replace in the artifact, the considered bud ${X_\omega}$ by the parallel (resp. sequential) subtree $X[X_{1\omega}, X_{2\omega}, \ldots, X_{|P|\omega}]$\footnote{The tree coded by $X[X_{1\omega}, X_{2\omega}, \ldots, X_{|P|\omega}]$ is the one whose root is labelled $X$ and has as sons, $|P|$ nodes labelled by $X_{1\omega}, X_{2\omega}, \ldots, X_{|P|\omega}$ respectively.} (resp. $X[X_{1\omega}, X_{2\overline{\omega}}, \ldots, X_{|P|\overline{\omega}}]$);
%4) if $P$ is of the form $P:~X \rightarrow X_1 \fatsemi X_2 \fatsemi \ldots \fatsemi X_{|P|}$ then create $|P|$ buds $X_{1\omega}, X_{2\overline{\omega}}, \ldots, X_{|P|\overline{\omega}}$ respectively of type $X_1, X_2, \ldots , X_{|P|}$ and replace in the artifact, the considered bud ${X_\omega}$ by the sequential subtree $X[X_{1\omega}, X_{2\overline{\omega}}, \ldots, X_{|P|\overline{\omega}}]$;

\item Update the execution status of the task associated with $X$: $X.status$ $=$ $"bla~bla~\ldots"$.
Updating $t$ results in an artifact $t^{maj}$ and we note $t \leq t^{maj}$.
\end{enumerate}

Although it is obvious, it seems important to clarify that the editing of an artifact is only a consequence of process tasks' execution by actors located on agents. We can therefore imagine this scenario for the peer-review process (see fig. \ref{chap3:fig:artifact-edition}): the associated editor who received a request from the editor in chief to peer-review a given article, has executed task $C$ (i.e. he has appraised the paper and formatted it to prepare the peer-review). Through a dedicated tool (a specialised editor), he has been invited to submit a report on the execution of the said task (via the filling of a form for example). The submission of that report, in which he may have provided a copy of the formatted paper as well as comments for referees, will cause (in background) the update of the artifact. The retrieved data will thus be stored in the $status$ attribute of task $C$ and the bud $C_\omega$ will be extended into a subtree as described above (the only production available for this purpose is $P_{3}: C\rightarrow E\fatsemi F$).
\begin{figure}[ht!]
	\noindent
	\makebox[\textwidth]{\includegraphics[scale=0.4]{./Chap3/images/artifactEdition.png}}
	\caption{An example of artifact edition: the bud $C_\omega$ is extended in a subtree.}
	\label{chap3:fig:artifact-edition}
\end{figure}

\mySection{Agent and choreography}{}
\label{chap3:sec:agents-and-choreography}

Now that we have formally defined the structure and the editing model of artifacts, let's focus on the structure of agents that oversee the execution of tasks and update the artifact accordingly, as well as on the artifact-centric choreography implemented between them. 

\mySubSection{Relations between Agent, Actor and Choreography}{}
\label{chap3:sec:agent-definition}
We borrowed the term \textit{agent} from \cite{lohmann2010artifact}. In the case of our study, an \textit{agent} (which we also call a \textit{peer}) is a software component, installed at a given site, piloted by a human agent called \textit{actor} (the tasks of the processes we handle are executed by humans) and capable of interacting with other agents by service invocation (message exchange). An agent is completely \textit{autonomous}: i.e. it encapsulates all the data and functions necessary for the execution of the tasks assigned to it, or precisely, tasks assigned to the actor piloting it. The agent is \textit{reactive}: it reacts in the same way to each message it receives by executing a well-defined protocol that goes from the analysis of the received message (artifact) to the possible transmission of other messages. As announced in the introduction, each message contains a collectively edited \textit{mobile artifact}. For the execution of a given process, the choreography is therefore a result of the messages (artifact replicas) exchanges between the agents involved and of the reaction of the latter to the reception of messages.

\mySubSection{Structure of an Agent}{}
\label{chap3:sec:agent-structure}

An agent is built to be able to fully manage the lifecycle (creation, storage, edition/execution) of a given business process' artifacts. Thus, an agent is made up of three major software components: a local workflow engine (LWfE), a specialised graphical editor and a storage device (see fig. \ref{chap3:fig:simplify-architecture-peer}).
\begin{figure}[ht!]
	\noindent
	\makebox[\textwidth]{\includegraphics[scale=0.45]{./Chap3/images/architectureSimplifieDUnPair.png}}
	\caption{Simplified architecture of an agent.}
	\label{chap3:fig:simplify-architecture-peer}
\end{figure}


\mySubSubSection{The Local Workflow Engine}{}
\label{chap3:sec:the-local-workflow-engine}

The local workflow engine (LWfE) is the main component of an agent. 
It receives messages from other agents and reacts by executing a well defined protocol (see sec. \ref{chap3:sec:the-protocols}). It communicates with the other agent's engines via its communication interface that exposes four services : two input services or \textit{provided services} (\textit{returnTo} and \textit{forwardTo}) connected to two corresponding output services or \textit{required services} (\textit{returnTo} and \textit{forwardTo}) so that: 

\begin{itemize}
	\item The invocation by an agent $j$ of the service \textit{forwardTo} offered by an agent $i$, causes on $i$, the execution of its corresponding input service \textit{forwardTo}. This service makes it possible to send a \textbf{request} from agent $j$ to agent $i$. The request contains the replica of the mobile artifact located on agent $j$. This artifact must contain buds to be completed (executed) by actor $A_i$ (the human agent piloting agent $i$). 
	\item The invocation by an agent $i$ of the service \textit{returnTo} offered by an agent $j$, causes the execution on $j$, of its corresponding input service \textit{returnTo}. This service allows agent $i$ to return the \textbf{response} to a request previously received from agent $j$. As the request, the response contains the replica of the mobile artifact located on agent $i$.
\end{itemize}


\mySubSubSection{The Storage Device}{}
\label{chap3:sec:the-storage-device}

A database (DB) of documents (a JSON\footnote{JavaScript Object Notation, \url{http://www.json.org}, \url{https://www.mongodb.com}, visited the 04/04/2020.} DB for example) is used by the LWfE to store an agent's configuration and data (especially artifacts) that it handles. 	
	
\mySubSubSection{The Specialised Editor}{}
\label{chap3:sec:the-specialised-editor}
	
Each agent provides a specialised editor (preferably WYSIWYG\footnote{What You See Is What You Get.}) that allows its actor (the pilot) to execute tasks. More precisely, the specialised editor allows the actor to view the tasks that are assigned to him, those ready to be executed, and when he has executed a task, it gives him the means to record an execution report. Any (editing) action carried out by the local actor via the specialised editor, causes the consistent update (as presented in section \ref{chap3:sec:artifact-edition}) of the mobile artifact local replica.

The specialised editor is particularly important as it guarantees controlled access to the artifact. Indeed, as announced in the introduction and following the steps of \cite{hull2009facilitating}, for reasons of confidentiality/security, actors do not necessarily have the right to access all information relating to the execution of a process in which they are involved. It is therefore important to provide a mechanism for regulating access to this information (stored in the artifact). In our case, we define this mechanism under the name \textit{accreditation} and we include it in the configuration of an agent in the same way as the GMWf of the studied process.


\mySubSection{Concepts of Accreditation, Partial Replica of an Artifact and Local GMWf}{}
\label{chap3:sec:accreditation-partial-replica}

\mySubSubSection{Concept of Accreditation}{}
\label{chap3:sec:accreditation}

Let's consider a process $\mathcal{P}_{op}$ and its GMWf $\mathbb{G}=\left(\mathcal{S},\mathcal{P},\mathcal{A}\right)$. The accreditation of an agent provides information on the rights (permissions) its actor has on each sort (task) of $\mathbb{G}$. 
To simplify, the nomenclature of rights manipulated here is inspired by the one used in Unix-like operating systems. Three types of accreditation are then defined: accreditation in reading \textit{(r)}, in writing \textit{(w)} and in execution \textit{(x)}. 
\begin{enumerate}
	\item \textit{Accreditation in reading \textit{(r)}}: when an agent is accredited in reading on a sort $X$, its actor has the right to know if the associated task is executed. Moreover, he can access its execution status.
	We call an agent's (actor's) \textbf{\textit{view}} the set of sorts on which it (he) is accredited in reading.
	\item \textit{Accreditation in writing \textit{(w)}}: when an agent is accredited in writing on a sort $X$, its actor can execute the associated task. 
	Note that a task can be executed only by exactly one actor: for a given sort, a single agent is accredited in writing; this is an important point of the model which guarantees the absence of execution conflicts. 
	Since the dedicated editors for "updating artifacts" are of type WYSIWYG (see sec. \ref{chap3:sec:the-specialised-editor}), any agent accredited in writing on a symbol must therefore be accredited in reading on it. 
	\item \textit{Accreditation in execution \textit{(x)}}: an agent accredited in execution on a sort $X$ is authorised to ask the agent which is accredited in writing on it, to execute it. Note that this request can be made without the agent being accredited in reading\footnote{In fact, as we will see later (sec. \ref{chap3:sec:the-protocols} (\textit{the diffusion protocol}, page \pageref{chap3:sec:execution-protocol-diffusion})), it is an automatic task (of the agent) that sends the execution request and not the actor.} on the considered sort.
\end{enumerate}

\noindent More formally, an accreditation is defined as follows:

\begin{definition} \label{defSyllabaire}
	An \textbf{accreditation} $\mathcal{A}_{A_i}$ defined on the set $\mathcal{S}$ of grammatical symbols for an agent $i$ piloted by an actor $A_i$, is a triplet $\mathcal{A}_{A_i}=\left(\mathcal{A}_{A_i(r)},\mathcal{A}_{A_i(w)},\mathcal{A}_{A_i(x)}\right)$ such that, 
	$\mathcal{A}_{A_i(r)} \subseteq \mathcal{S}$ also called \textbf{view} of $i$ (or \textbf{view} of $A_i$), is the set of symbols on which $i$ is accredited in reading, 
	$\mathcal{A}_{A_i(w)} \subseteq \mathcal{A}_{A_i(r)}$ is the set of symbols on which $i$ is accredited in writing and  
	$\mathcal{A}_{A_i(x)} \subseteq \mathcal{S}$ is the set of symbols on which $i$ is accredited in execution.
\end{definition}


The accreditations of various agents must be produced by the workflow designer just after modelling the scenarios in the form of target artifacts. From the task assignment for the peer-review process in the running example (see table \ref{tableau:tachesExecutant}), it follows that the accreditation in writing of the editor in chief is $\mathcal{A}_{EC(w)}=\{A, B, D\}$, that of the associated editor is $\mathcal{A}_{AE(w)}=\{C, E, F\}$ and that of the first (resp. the second) referee is $\mathcal{A}_{R_1(w)}=\{G1, H1, I1\}$ (resp. $\mathcal{A}_{R_2(w)}=\{G2, H2, I2\}$).
Even more, since the editor in chief can only perform the task $D$ if the task $C$ is already executed (see artifacts $art_1$ and $art_2$, fig. \ref{chap3:fig:global-artefacts}), in order for the editor in chief to be able to ask the associated editor to perform this task, it (the agent) must be accredited in execution on it; so we have $\mathcal{A}_{EC(x)}=\{C\}$.
Moreover, in order to be able to access all the information on the peer-review evaluation of a paper (task $C$) and to summarise the right decision to send to the author, the editor in chief must be able to consult the reports (tasks $I1$ and $I2$) and the messages (tasks $H1$ and $H2$) of the different referees, as well as the final decision taken by the associated editor (task $F$). These tasks, added to $\mathcal{A}_{EC(w)}$\footnote{Recall that in our case, we use WYSIWYG tools and therefore, one can only execute what he see.} constitute the set $\mathcal{A}_{EC(r)}=\mathcal{V}_{EC}=\{A, B, C, D, H1, H2, I1, I2, F\}$ of tasks on which, it is accredited in reading. By doing so for each of other agents, we deduce the accreditations represented in table \ref{tableau:vuesActeurs}.
\begin{table}[ht]
	\centering
	\caption{Accreditations of the different agents taking part in the peer-review process.}
	\label{tableau:vuesActeurs}
	\begin{tabular}[t]{|m{3.5cm}|m{10.3cm}|}
		\hline
		\textbf{Agent} & \textbf{Accreditation} \\
		\hline
		Editor in Chief ($EC$) & $\mathcal{A}_{EC}=\left(\{A, B, C, D, H1, H2, I1, I2, F\}, \{A, B, D\}, \{C\}\right)$ \\
		\hline
		Associated Editor ($AE$) & $\mathcal{A}_{AE}=\left(\{A, C, E, F, H1, H2, I1, I2\}, \{C, E, F\}, \{G1, G2\}\right)$ \\
		\hline
		First referee ($R1$) & $\mathcal{A}_{R1}=\left(\{C, G1, H1, I1\}, \{G1, H1, I1\}, \emptyset\right)$ \\
		\hline
		Second referee ($R2$) & $\mathcal{A}_{R2}=\left(\{C, G2, H2, I2\}, \{G2, H2, I2\}, \emptyset\right)$ \\
		\hline
	\end{tabular}
\end{table}

~

\noindent\textbf{\textit{In summary, what is the workflow model ?}}

To summarise, we state that in the artifact-centric model presented in this chapter, an administrative process $\mathcal{P}_{op}$ is completely specified using a triplet $\mathbb{W}_f=\left(\mathbb{G}, \mathcal{L}_{P_k}, \mathcal{L}_{\mathcal{A}_k} \right)$ called \textit{a Grammatical Model of Administrative Workflow Process} (GMAWfP) and composed of: a GMWf, a list of actors (agents) and a list of their accreditations. 
The GMWf is used to describe all the tasks of the studied process and their scheduling, while the list of accreditations provides information on the role played by each actor involved in the process execution.  
A GMAWfP can then be formally defined as follows:
\begin{definition}
	\label{defMGSPWfA}
	A \textbf{Grammatical Model of Administrative Workflow Process} (GMAWfP) $\mathbb{W}_f$ for a given business process, is a triplet $\mathbb{W}_f=\left(\mathbb{G}, \mathcal{L}_{P_k}, \mathcal{L}_{\mathcal{A}_k} \right)$
	wherein $\mathbb{G}$ is the studied process (global) GMWf, $\mathcal{L}_{P_k}$ is the set of $k$ agents taking part in its execution and $\mathcal{L}_{\mathcal{A}_k}$ represents the set of these agents' accreditations. 
\end{definition}


\mySubSubSection{Concept of Partial Replica of an Artifact}{}
\label{chap3:sec:partial-replica}

To effectively ensure that actors only have access to information of proven interest to them, each agent let them access only to a potentially partial replica $t_{\mathcal{V}_i}$ of the mobile artifact $t$. 
The $t$'s partial replicas are obtained by projections according to the views of each actor. 
A partial replica $t_{\mathcal{V}_i}$ of $t$ according to the \textit{view} $\mathcal{V}_{A_i} =  \mathcal{A}_{A_i(r)} $, is a partial copy of $t$ obtained by means of the so-called \textit{projection operator} denoted $\pi$ as presented below. 

Technically, the projection $t_{\mathcal{V}_i}$ of an artifact $t$ according to the view $\mathcal{V}_{i} = \mathcal{A}_{A_i(r)}$ is obtained by deleting in $t$ all nodes whose types do not belong to $\mathcal{V}_{i}$. In our case, the main challenges in this operation are:
\begin{enumerate}
	\item[\textbf{(1)}] nodes of $t_{\mathcal{V}_i}$ must preserve the previously existing execution order between them in $t$,
	\item[\textbf{(2)}] $t_{\mathcal{V}_i}$ must be build by using exclusively the only two forms of production retained for GMWf and
	\item[\textbf{(3)}] $t_{\mathcal{V}_i}$ must be unique in order to ensure the continuation of process execution (see sec. \ref{chap3:sec:the-protocols}).
\end{enumerate}

The projection operation is noted $\pi$. Inspired by the one proposed in \cite{badouelTchoupeCmcs}, it projects an artifact by preserving the hierarchy (father-son relationship) between nodes of the artifact (it thus meets challenge \textbf{(1)}); but in addition, it inserts into the projected artifact when necessary, new additional \textit{(re)structuring symbols } (accessible in reading and writing by the agent for whom the projection is made). This enables it to meet challenge \textbf{(2)}. The details of how to accomplish the challenge \textbf{(3)} are outlined immediately after the algorithm (algorithm \ref{chap3:algo:artifact-projection}) is presented.
\begin{figure}[ht!]
	\noindent
	\makebox[\textwidth]{\includegraphics[scale=0.35]{./Chap3/images/docEtRepliques.png}}
	\caption{Example of projections made on an artifact and partial replicas obtained.}
	\label{chap3:fig:partial-replicas}
\end{figure}

Figure \ref{chap3:fig:partial-replicas} illustrates the projection of an artifact of the peer-review process relatively to the $R1$ (first referee) and $EC$ (Editor in Chief) agent views. Note the presence in $t_{\mathcal{V}_{EC}}$ of new (re)structuring symbols (in gray). These last ones make it possible to avoid introducing in $t_{\mathcal{V}_{EC}}$, the production $p: C \rightarrow H1 \fatsemi I1 \parallel H2 \fatsemi I2 \fatsemi F$ whose form does not correspond to the two forms of production retained for the GMWf writing\footnote{Note that this production specifies in its right-hand side that we must have parallel and sequential treatments.
Inserting $S1$, $S2$ and $S3$ allows to rewrite $p$ in four productions $p1: C \rightarrow S1 \fatsemi F$, $p2: S1 \rightarrow S2 \parallel S3$, $p3: S2 \rightarrow H1 \fatsemi I1$ and $p4: S3 \rightarrow H2 \fatsemi I2$.}.

~

\noindent\textbf{\textit{The algorithm}}

Let's consider an artifact $t$ and note by $n=X\left[t_1,\ldots,t_m\right]$ a node of $t$ labelled with the symbol $X$ and having $m$ sub-artifacts $t_1,\ldots,t_m$. Note also by $p_n$, the production of the GMWf that was used to extend node $n$; the type of $p_n$ is either \textit{sequential} (i.e. $p_n: X \rightarrow X_1 \fatsemi \ldots \fatsemi X_m$ where $X_1,\ldots,X_m$ are the roots of the sub-artifacts $t_1,\ldots,t_m$) or \textit{parallel} ($p_n: X \rightarrow X_1 \parallel \ldots \parallel X_m$). 
Concretely, to project $t$ according to a given view $\mathcal{V}$ (i.e to find $\mathit{projs_t}=\pi_{\mathcal{V}}\left(t \right)$), the recursive processing presented in algorithm \ref{chap3:algo:artifact-projection}, is applied to the root node $n=X\left[t_1,\ldots,t_m\right]$ of $t$.

\begin{algorithm}
\small
\caption{Algorithm to project a given artifact according to a given view.}
\label{chap3:algo:artifact-projection}
\begin{mdframed}[style=MyFrame]
	%\begin{itemize}[leftmargin=*]
	{\large\textbullet} $~$ \textbf{If symbol $X$ is visible ($X \in \mathcal{V}$)} then :
	%\begin{enumerate}[leftmargin=*]
	
	\textbf{1.}$~$ $n$ is kept in the artifact;
	
	\textbf{2.}$~$ For each sub-artifact $t_i$ of $n$, having node $n_i=X_i\left[t_{i_1},\ldots,t_{i_k}\right]$ as root (of which $p_{n_i}$ is the production that was used to extend it), the following processing is applied :
	%\begin{itemize}[leftmargin=*]
	
	$~~$\textbf{a.}$~$ The projection of $t_i$ according to $\mathcal{V}$ is done. We obtain the list $\mathit{projs_{t_i}} = \pi_{\mathcal{V}}\left(t_i \right) = \left\{t_{i_{\mathcal{V}_1}},\ldots,t_{i_{\mathcal{V}_l}}\right\}$;
	
	$~~~~$\textbf{b.}$~$ If the type of $p_{n_i}$ is the same as the type of $p_{n}$ or the projection of $t_i$ has produced no more than one artifact ($\left|\mathit{projs_{t_i}}\right| \leq 1$), we just replace $t_i$ by artifacts $t_{i_{\mathcal{V}_1}},\ldots,t_{i_{\mathcal{V}_l}}$ of the list $\mathit{projs_{t_i}}$;
	
	$~~~~$ Otherwise, a new (re)structuring symbol $S_i$ is introduced and we replace the sub-artifact $t_i$ with a new artifact $new\_t_i$ whose root node is $n_{t_i}=S_i\left[t_{i_{\mathcal{V}_1}},\ldots,t_{i_{\mathcal{V}_l}}\right]$;
	%\end{itemize}
	
	\textbf{3.}$~$ If the list of new sub-artifacts of $n$ contains only one element $t_1$ having $n_1=S_1\left[t_{1_{\mathcal{V}_1}},\ldots,t_{1_{\mathcal{V}_l}}\right]$ (with $S_1$ a newly created (re)structuring symbol) as root node, we replace in this one, $t_1$ by the sub-artifacts $t_{1_{\mathcal{V}_1}},\ldots,t_{1_{\mathcal{V}_l}}$ of $n_1$. This removes a non-important (re)structuring symbol $S_1$.
	%\end{enumerate}
	
	\noindent{\large\textbullet} $~$ \textbf{Else}, $n$ is deleted and the result of the projection ($\mathit{projs_t}$) is the union of the projections of each of its sub-artifacts: $\mathit{projs_t} = \pi_{\mathcal{V}}\left(t \right) = \bigcup^{m}_{i=1} \pi_{\mathcal{V}}\left(t_i \right)$
	%\end{itemize}
\end{mdframed}
\end{algorithm}

Note that the algorithm described here can return several artifacts (a forest). To avoid that it produces a forest in some cases and thus meet challenge \textbf{(3)}, we make the following assumption: 
\begin{displayquote}
\textit{GMAWfP manipulated in this work are such that all agents are accredited in reading on the GMWf axioms (\textbf{axioms' visibility assumption}).}
\end{displayquote}
The designer must therefore ensure that all agents are accredited in reading on all GMWf axioms. To do this, after modelling a process $\mathcal{P}_{op}$ and obtaining its GMWf $\mathbb{G}=\left(\mathcal{S},\mathcal{P},\mathcal{A}\right)$, it is sufficient (if necessary) to create a new axiom $A_{\mathbb{G}}$ on which, all actors will be accredited in reading, and to associate it with new unit productions\footnote{A production of a context free grammar is a \textit{unit production} if it is on the form $A \rightarrow B$, where $A$ and $B$ are non-terminal symbols.} $pa : A_{\mathbb{G}} \rightarrow X_a$ where, $X_a \in \mathcal{A}$ is a symbol labelling the root of a target artifact.
Moreover, the designer of the GMWf must statically choose the agent responsible for initiating the process. This agent will therefore be the only one to possess an accreditation in writing on the new axiom $A_{\mathbb{G}}$.

\begin{proposition}
	\label{propositionStabiliteProjArt}
	For all GMAWfP $\mathbb{W}_f=\left(\mathbb{G}, \mathcal{L}_{P_k}, \mathcal{L}_{\mathcal{A}_k} \right)$ verifying the axioms' visibility assumption, the projection of an artifact $t$ which is conform to its GMWf ($t \therefore \mathbb{G}$) according to a given view $\mathcal{V}$, results in a single artifact $t_{\mathcal{V}}=\pi_{\mathcal{V}} \left(t\right)$ (stability property of $\pi$).
\end{proposition}

\begin{proof}
	Let's show that $\pi_{\mathcal{V}} \left(t\right)$ produces a single tree $t_{\mathcal{V}}$ which is an artifact. 
	Note that the only case in which the projection of an artifact $t$ according to a view $\mathcal{V}$ produces a forest, is when the root node of $t$ is associated with an invisible symbol $X$ ($X \notin \mathcal{V}$). Knowing that $t \therefore \mathbb{G}$ and that $\mathbb{W}_f$ validates the axioms' visibility assumption, it is deduced that the root node of $t$ is labelled by one of the axioms $A_{\mathbb{G}}$ of $\mathbb{G}$ and that $A_{\mathbb{G}} \in \mathcal{V}$ (hence the uniqueness of the produced tree). Since the projection operation preserves the form of productions, it is concluded that $t_{\mathcal{V}}=\pi_{\mathcal{V}} \left(t\right)$ is an artifact.
\end{proof}

A Haskell implementation of this projection algorithm is introduced in appendix \ref{appendice1:algorithms-implementations} of this manuscript. Another implementation in Java has also been proposed and integrated into the prototype that we will present in section \ref{chap3:sec:p2ptinywfms} of this chapter.


\mySubSubSection{The Need of a Local GMWf}{}
\label{chap3:sec:local-gmwf}
Since the artifact copy manipulated at a specific site is a potentially partial replica of the mobile (global) artifact, and since its editing depends on the agent's perception (view) of the process, it becomes crucial to provide each agent with a local GMWf. The latter will serve in addition to preserve the possible confidentiality of certain tasks and data, to guide the local actions of updating the artifact in order to ensure the convergence of the system to a coherent business goal state.
The local GMWf of an agent can be derived by projecting the global GMWf $\mathbb{G}$ according to the view $\mathcal{V}_i$ of its pilot (\textbf{\textit{GMWf projection}}). This projection is carried out using $\Pi$ operator and the GMWf obtained is noted $\mathbb{G}_{\mathcal{V}_i}=\Pi_{\mathcal{V}_i}\left(\mathbb{G}\right)$.

~

\noindent\textbf{\textit{A naive algorithm for non-recursive GMWf projection}}

The goal of this algorithm is to derive by projection of a given GMWf $\mathbb{G}=\left(\mathcal{S},\mathcal{P},\mathcal{A}\right)$ according to a view $\mathcal{V}$, a local GMWf $\mathbb{G}_{\mathcal{V}} = \left(\mathcal{S}_{\mathcal{V}},\mathcal{P}_{\mathcal{V}}, \mathcal{A}_{\mathcal{V}}\right)$ (we note $\mathbb{G}_{\mathcal{V}} = \Pi_{\mathcal{V}}\left(\mathbb{G} \right)$). The proposed algorithm is algorithm \ref{chap3:algo:gmwf-projection}.

\begin{algorithm}
\small
\caption{Algorithm to project a given GMWf according to a given view.}
\label{chap3:algo:gmwf-projection}
\begin{mdframed}[style=MyFrame]
	
	\noindent\textbf{1.}$~$ First of all, it is necessary to generate all the target artifacts denoted by $\mathbb{G}$ (see note (1) below); 
	we thus obtain a set $arts_{\mathbb{G}}=\left\{t_1,\ldots,t_n\right\}$;
	
	\noindent\textbf{2.}$~$ Then, each of the target artifacts must be projected according to $\mathcal{V}$. We thus obtain a set $arts_{\mathbb{G}_{\mathcal{V}}} = \left\{t_{\mathcal{V}_1},\ldots,t_{\mathcal{V}_m}\right\}$ (with $m \leq n$ because there may be duplicates; %\footnote{Deux artefacts cibles différents peuvent avoir la même projection suivant une vue $\mathcal{V}$ donnée.}; 
	in this case, only one copy is kept) of artifacts partial replicas;
	
	\noindent\textbf{3.}$~$ Then, collect the different (re)structuring symbols appearing in artifacts of $arts_{\mathbb{G}_{\mathcal{V}}}$, making sure to remove duplicates (see note (2) below) 
	and to consequently update the artifacts and the set $arts_{\mathbb{G}_{\mathcal{V}}}$. We thus obtain a set $\mathcal{S}_{\mathcal{V}_{Struc}}$ of symbols and a final set $arts_{\mathbb{G}_{\mathcal{V}}} = \left\{t_{\mathcal{V}_1},\ldots,t_{\mathcal{V}_l}\right\}$ (with $l \leq m$) of artifacts. These are exactly the only ones that must be conform to the searched GMWf $\mathbb{G}_{\mathcal{V}}$. So we call them, \textit{local target artifacts for the view $\mathcal{V}$};
	
	\noindent\textbf{4.}$~$ At this stage, it is time to collect all the productions that made it possible to build each of the \textit{local target artifacts for the view $\mathcal{V}$}. We obtain a set $\mathcal{P}_{\mathcal{V}}$ of distinct productions.\\
	\textbf{The searched local GMWf $\mathbb{G}_{\mathcal{V}} = \left(\mathcal{S}_{\mathcal{V}},\mathcal{P}_{\mathcal{V}}, \mathcal{A}_{\mathcal{V}}\right)$ is such as}:
	%\begin{itemize}
	
	$~~$\textbf{a.}$~$ its set of symbols is $\mathcal{S}_{\mathcal{V}} = \mathcal{V} \cup \mathcal{S}_{\mathcal{V}_{Struc}}$;
	
	$~~$\textbf{b.}$~$ its set of productions is $\mathcal{P}_{\mathcal{V}}$;
	
	$~~$\textbf{c.}$~$ its axioms are in $\mathcal{A}_{\mathcal{V}} = \mathcal{A}$
	%\end{itemize}
	%\end{enumerate}
	
	~
	
	\noindent\textit{\textbf{Note (1):}$~$ To generate all the target artifacts denoted by a GMWf $\mathbb{G}=\left(\mathcal{S},\mathcal{P},\mathcal{A}\right)$, one just has to use the set of productions to generate the set of artifacts having one of the axiom $A_{\mathbb{G}}\in \mathcal{A}$ as root. In fact, for each axiom $A_{\mathbb{G}}$, it should be considered that every $A_{\mathbb{G}}$-production $P=\left(A_{\mathbb{G}},X_1\cdots X_n\right)$ induces artifacts $\left\{t_1, \ldots, t_m\right\}$ such as: the root node of each $t_i$ is labelled $A_{\mathbb{G}}$ and has as its sons, a set of artifacts $\left\{t_{i_1},\ldots,t_{i_n}\right\}$, part of the Cartesian product of the sets of artifacts generated when considering each symbol $X_1,\cdots, X_n$ as root node.}
	
	\noindent\textit{\textbf{Note (2):}$~$ In this case, two (re)structuring symbols are identical if for all their appearances in nodes of the different artifacts of $arts_{\mathbb{G}_{\mathcal{V}}}$, they induce the same local scheduling.}
\end{mdframed}
\end{algorithm}

Figure \ref{chap3:fig:gmwf-projection} illustrates the research of a local model $\mathbb{G}_{\mathcal{V}_{EC}}$ such as $\mathbb{G}_{\mathcal{V}_{EC}} = \Pi_{\mathcal{V}_{EC}}\left(\mathbb{G}\right)$ with $\mathcal{V}_{EC}=\mathcal{A}_{EC(r)}=\{A, B, C, D, H1, H2, I1, I2, F\}$. Target artifacts generated from $\mathbb{G}$ (fig. \ref{chap3:fig:gmwf-projection}(b)) are projected to obtain two \textit{local target artifacts for the view $\mathcal{V}_{EC}$} (fig. \ref{chap3:fig:gmwf-projection}(c)). 
From the local target artifacts thus obtained, the searched GMWf is produced (fig. \ref{chap3:fig:gmwf-projection}(d)).
\begin{figure}[ht!]
	\noindent
	\makebox[\textwidth]{\includegraphics[scale=0.3]{./Chap3/images/projectionMGWf.png}}
	\caption{Example of projection of a GMWf according to a given view.}
	\label{chap3:fig:gmwf-projection}
\end{figure}

The GMWf projection algorithm presented here only works for GMWf that do not allow recursive symbols\footnote{It is only in this context that all the target artifacts can be enumerated.}. We therefore assume that:
\begin{displayquote}
\textit{For the execution model presented in this chapter, the manipulated GMAWfP are those whose GMWf do not contain recursive symbols (\textbf{non-recursive GMWf assumption})}.
\end{displayquote} 
Therefore, it is no longer possible to express iterative routing between process tasks (in the general case); except in cases where the maximum number of iterations is known in advance. This algorithm has some interesting properties and the interested reader will find an introduction to its Haskell implementation in appendix \ref{appendice1:algorithms-implementations}.

\begin{proposition}
	\label{propositionStabiliteProjGMWf}
	For all GMAWfP $\mathbb{W}_f=\left(\mathbb{G}, \mathcal{L}_{P_k}, \mathcal{L}_{\mathcal{A}_k} \right)$ verifying the axioms' visibility and the non-recursivity of GMWf assumptions, the projection of its GMWf $\mathbb{G}=\left(\mathcal{S},\mathcal{P},\mathcal{A}\right)$ according to a given view $\mathcal{V}$, is a GMWf $\mathbb{G}_{\mathcal{V}} = \Pi_{\mathcal{V}}\left(\mathbb{G} \right)$ for a GMAWfP $\mathbb{W}_{f_{\mathcal{V}}}$ verifying the assumptions of axiom visibility and non-recursivity of GMWf (stability property of $\Pi$).
\end{proposition}

\begin{proof}
	Let's show that $\mathbb{G}_{\mathcal{V}} = \Pi_{\mathcal{V}}\left(\mathbb{G} \right)$ is a GMWf for a new GMAWfP $\mathbb{W}_{f_{\mathcal{V}}}=\left(\mathbb{G}_{\mathcal{V}}, \mathcal{L}_{P_k}, \mathcal{L}_{\mathcal{A}_{\mathcal{V}_k}} \right)$ that verifies the assumptions of axioms' visibility and non-recursivity of GMWf. 
	As $\mathbb{W}_f=\left(\mathbb{G}, \mathcal{L}_{P_k}, \mathcal{L}_{\mathcal{A}_k} \right)$ validates the non-recursivity of GMWf assumption, the set of target artifacts ($arts_{\mathbb{G}}=\left\{t_1,\ldots,t_n\right\}$) that it denotes is finite and can therefore be fully enumerated. Knowing further that $\mathbb{W}_f$ validates the axioms' visibility assumption, it is deduced that the set $arts_{\mathbb{G}_{\mathcal{V}}} = \left\{t_{\mathcal{V}_1}=\pi_{\mathcal{V}}\left(t_1\right), \ldots,t_{\mathcal{V}_n}=\pi_{\mathcal{V}}\left(t_n\right)\right\}$ is finite and the root node of each artifact $t_{\mathcal{V}_i}$ is associated with an axiom $A_{\mathbb{G}} \in \mathcal{A}$ (see proposition \ref{propositionStabiliteProjArt}). $\mathbb{G}_{\mathcal{V}}$ being built from the set $arts_{\mathbb{G}_{\mathcal{V}}}$, its axioms $\mathcal{A}_{\mathcal{V}}=\mathcal{A}$ are visible to all actors and its productions are only of the two forms retained for GMWf. In addition, each new (re)structuring symbol ($S \in \mathcal{S}_{\mathcal{V}_{Struc}}$)) is created and used only once to replace a symbol that is not visible and not recursive (by assumption) when projecting artifacts of $arts_{\mathbb{G}}$. The new symbols are therefore not recursive. By replacing in $\mathcal{L}_{\mathcal{A}_k}$ the view $\mathcal{V}$ by $\mathcal{V} \cup \mathcal{S}_{\mathcal{V}_{Struc}}$, one obtains a new set $\mathcal{L}_{\mathcal{A}_{\mathcal{V}_k}}$ of accreditations for a new GMAWfP $\mathbb{W}_{f_{\mathcal{V}}}=\left(\mathbb{G}_{\mathcal{V}}, \mathcal{L}_{P_k}, \mathcal{L}_{\mathcal{A}_{\mathcal{V}_k}} \right)$ verifying the assumptions of axioms' visibility and non-recursivity of GMWf.
\end{proof}


\begin{proposition}
	\label{propositionCoherenceArtefact}
	For all GMAWfP $\mathbb{W}_f=\left(\mathbb{G}, \mathcal{L}_{P_k}, \mathcal{L}_{\mathcal{A}_k} \right)$ verifying the axioms' visibility and the non-recursivity of GMWf assumptions, the projection of an artifact $t$ which is conform to the GMWf $\mathbb{G}$ according to a given view $\mathcal{V}$, is an artifact which is conform to the projection of $\mathbb{G}$ according to $\mathcal{V}$ $\left(\forall t, ~t \therefore \mathbb{G} \Rightarrow \pi_{\mathcal{V}}\left(t\right) \therefore \Pi_{\mathcal{V}}\left(\mathbb{G} \right)\right)$.
\end{proposition}

\begin{proof}
	Knowing that the considered GMAWfP $\mathbb{W}_f=\left(\mathbb{G}, \mathcal{L}_{P_k}, \mathcal{L}_{\mathcal{A}_k} \right)$ verifies the axioms' visibility and the non-recursivity of GMWf assumptions, it is deduced that the set of its target artifacts $arts_{\mathbb{G}}$ (those who helped to build its GMWf $\mathbb{G}$) is finite and any artifact that is conform to its GMWf $\mathbb{G}$ is a target artifact $\left( \forall t, ~t \therefore \mathbb{G} \Leftrightarrow t \in arts_{\mathbb{G}} \right)$. Therefore, considering a given artifact $t$ such that $t$ is conform to $\mathbb{G}$ ($t \therefore \mathbb{G}$), one knows that it is a target artifact ($t \in arts_{\mathbb{G}}$) and its projection according to a given view $\mathcal{V}$ produces a single artifact $t_{\mathcal{V}}=\pi_{\mathcal{V}}\left(t\right)$ (see "stability property of $\pi$", proposition \ref{propositionStabiliteProjArt}) such as $t$ and $t_{\mathcal{V}}$ have the same root (one of the axioms $A_{\mathbb{G}} \in \mathcal{A}$ of $\mathbb{G}$). Since $t$ is a target artifact, its projection $t_{\mathcal{V}}$ (through the renaming of some potential (re)structuring symbols) is part of the set $arts_{\mathbb{G}_{\mathcal{V}}}$ of artifacts that have generated $\mathbb{G}_{\mathcal{V}} = \Pi_{\mathcal{V}}\left(\mathbb{G} \right)$ by applying the projection principle described in the algorithm \ref{chap3:algo:gmwf-projection}. Therefore, the productions involved in the construction of $t_{\mathcal{V}}$ are all included in the set of productions of the GMWf $\mathbb{G}_{\mathcal{V}} = \Pi_{\mathcal{V}}\left(\mathbb{G} \right)$. As the set of axioms of $\mathbb{G}_{\mathcal{V}}$ is $\mathcal{A}_{\mathcal{V}} = \mathcal{A}$, it is deduced that $A_{\mathbb{G}} \in \mathcal{A}_{\mathcal{V}}$ and concluded that $t_{\mathcal{V}} \therefore \mathbb{G}_{\mathcal{V}}$.
\end{proof}

\begin{proposition}
	\label{propositionReciproqueCoherenceArtefact}
	Consider a GMAWfP $\mathbb{W}_f=\left(\mathbb{G}, \mathcal{L}_{P_k}, \mathcal{L}_{\mathcal{A}_k} \right)$ verifying the axioms' visibility and the non-recursivity assumptions. For all artifact $t_{\mathcal{V}}$ which is conform to $\Pi_{\mathcal{V}}\left(\mathbb{G} \right)$, it exists at least one artifact $t$ which is conform to $\mathbb{G}$ such that $t_{\mathcal{V}}=\pi_{\mathcal{V}}\left(t\right)$ $\left(\forall t_{\mathcal{V}}, ~t_{\mathcal{V}} \therefore \Pi_{\mathcal{V}}\left(\mathbb{G} \right) \Rightarrow \exists t, ~t \therefore \mathbb{G} ~and~ t_{\mathcal{V}}=\pi_{\mathcal{V}}\left(t\right) \right)$.
\end{proposition}

\begin{proof}
	With proposition \ref{propositionStabiliteProjGMWf} ("stability property of $\Pi$") it has been shown that the projection $\mathbb{G}_{\mathcal{V}} = \Pi_{\mathcal{V}}\left(\mathbb{G} \right)$ according to the view $\mathcal{V}$ of a GMWf $\mathbb{G}$ verifying the axioms' visibility and the non-recursivity assumptions, is a GMWf verifying the same assumptions. On this basis and using similar reasoning to that used to prove the proposition \ref{propositionCoherenceArtefact}, it's been determined that an artifact $t_{\mathcal{V}}$ that is conform to $\mathbb{G}_{\mathcal{V}}$, is one of its target artifacts (\textit{local target artifact for the view $\mathcal{V}$}): i.e, $t_{\mathcal{V}} \in arts_{\mathbb{G}_{\mathcal{V}}}$. Referring to the projection process which made it possible to obtain $\mathbb{G}_{\mathcal{V}}$, it is determined that the set $arts_{\mathbb{G}_{\mathcal{V}}}$ is exclusively made up of the projections of the set $arts_{\mathbb{G}}=\left\{t_1,\ldots,t_n\right\}$ of $\mathbb{G}$'s target artifacts. $t_{\mathcal{V}}$ is therefore the projection of at least one target artifact $t_i \in arts_{\mathbb{G}}$ of $\mathbb{G}$ $\left(t_{\mathcal{V}}=\pi_{\mathcal{V}}\left(t_i\right)\right)$. Knowing that $\forall t, ~t \therefore \mathbb{G} \Leftrightarrow t \in arts_{\mathbb{G}}$ (see proof of proposition \ref{propositionCoherenceArtefact}), it is deduced that $t_i \therefore \mathbb{G}$ and the proof of this proposition is made.
\end{proof}


By applying the GMWf projection algorithm presented above to the running example, one obtain the productions listed in table \ref{tableau:gramLocales} for the different agents respectively. Let us note that this algorithm simply project each target artifact according to the view of the considered agent, then gather the productions in the obtained partial replicas while removing the duplicates. In the illustrated case here, we have considered an update of the GMWf of the peer-review process so that it validates the axioms' visibility assumption (see sec. \ref{chap3:sec:partial-replica}).
\begin{table}[h]
	\centering
	\caption{Local GMWf productions of all the agents involved in the peer-review process.}
	\label{tableau:gramLocales}
	\begin{tabular}[t]{|m{3.5cm}|m{10.5cm}|}
		\hline
		\textbf{Agent} & \textbf{Productions of local GMWf} \\
		\hline
		Editor in Chief ($EC$) &
		\[ 
		\begin{array}{l|l|l}
		P_{1}:\; A_{\mathbb{G}}\rightarrow A & \; P_{2}:\; A\rightarrow B\fatsemi D\; & \; P_{3}:\; A\rightarrow C\fatsemi D  \\
		P_{4}:\; C\rightarrow S1\fatsemi F & \; P_{5}:\; S1\rightarrow S2\parallel S3\; & \; P_{6}:\; S2\rightarrow H1 \fatsemi I1  \\
		P_{7}:\; S3\rightarrow H2 \fatsemi I2 & \; P_{8}:\; B\rightarrow \varepsilon\; & \; P_{9}:\; D\rightarrow \varepsilon \\
		P_{10}:\; F\rightarrow \varepsilon & \; P_{11}:\; H1\rightarrow \varepsilon\; & \; P_{12}:\; I1\rightarrow \varepsilon  \\
		P_{13}:\; H2\rightarrow \varepsilon & \; P_{14}:\; I2\rightarrow \varepsilon \; &   \\
		\end{array}
		\]
		\\
		\hline
		Associated Editor ($AE$) & 
		\[ 
		\begin{array}{l|l|l}
		P_{1}:\; A_{\mathbb{G}}\rightarrow A & \; P_{2}:\; A\rightarrow C \; & \; P_{3}:\; C\rightarrow E\fatsemi F  \\
		P_{4}:\; E\rightarrow S1\parallel S2 & \; P_{5}:\; S1\rightarrow H1\fatsemi I1 \; & \; P_{6}:\; S2\rightarrow H2\fatsemi I2  \\
		P_{7}:\; H1\rightarrow \varepsilon & \; P_{8}:\; I1\rightarrow \varepsilon \; & \; P_{9}:\; H2\rightarrow \varepsilon \\
		P_{10}:\; I2\rightarrow \varepsilon & \; P_{11}:\; F\rightarrow \varepsilon \; & \; P_{12}:\; A_{\mathbb{G}}\rightarrow \varepsilon \\
		\end{array}
		\]
		\\
		\hline
		First referee ($R1$) & 
		\[ 
		\begin{array}{l|l|l}
		P_{1}:\; A_{\mathbb{G}}\rightarrow C & P_{2}:\; C\rightarrow G1\; & \; P_{3}:\; G1\rightarrow H1\fatsemi I1 \\
		P_{4}:\; H1\rightarrow \varepsilon & P_{5}:\; I1\rightarrow \varepsilon \; & \; P_{6}:\; A_{\mathbb{G}}\rightarrow \varepsilon \\
		\end{array}
		\]
		\\
		\hline
		Second referee ($R2$) & 
		\[ 
		\begin{array}{l|l|l}
		P_{1}:\; A_{\mathbb{G}}\rightarrow C & P_{2}:\; C\rightarrow G2\; & \; P_{3}:\; G2\rightarrow H2\fatsemi I2 \\
		P_{4}:\; H2\rightarrow \varepsilon & P_{5}:\; I2\rightarrow \varepsilon \; & \; P_{6}:\; A_{\mathbb{G}}\rightarrow \varepsilon \\
		\end{array}
		\]
		\\
		\hline
	\end{tabular}
\end{table}



\mySubSection{The Artifact-Centric Choreography}{}
\label{chap3:sec:execution-model}

In this section, we are interested in the actual execution of a process $\mathcal{P}_{op}$ whose GMWf is $\mathbb{G}=\left(\mathcal{S},\mathcal{P},\mathcal{A}\right)$.

\mySubSubSection{Initial Configuration of an Agent}{}
\label{chap3:sec:initial-configuration-of-a-peer}

Each agent $i$ taking part in the choreography, has a single identifier (its ID). For a proper execution, it manages a local copy of the process' global GMWf $\mathbb{G}$, accreditations of various agents involved and its local GMWf $\mathbb{G}_{i}$. In addition, it handles a list $RET_i$ of agents who have made requests and whose answers are yet to be sent, as well as two queues: $REQ_i$ which stores requests waiting to be executed, and $ANS_i$ which temporally stores answers received from agents to which requests were previously made. A local copy $t_i$ of the mobile artifact and its (potentially partial) replica $t_{\mathcal{V}_{i}}$ are also handled by agent $i$.


\mySubSubSection{The Execution Choreography and Agent's Behaviour}{}
\label{chap3:sec:architecture-of-a-peer}
The execution of an instance of the process is triggered when an artifact $t$ is introduced into the system (on the appropriate agent); this artifact is in fact an unlocked bud of the type of one axiom $A_\mathbb{G} \in \mathcal{A}$ (initial task) of the (global) GMWf $\mathbb{G}$.

An artifact that arrives on a given agent is either a request or a response to a request; depending on the case, it is inserted in the appropriate queue ($REQ_i$ or $ANS_i$).
As soon as possible\footnote{For instance at the end of the local replica completion or after a given time interval.}, the artifact is removed from the queue, merged with the local copy (if it exists) and is then completed as needed.
Completing an artifact consists of executing in a coherent way, the various tasks it imposes, i.e. those on which the current agent is accredited in writing.

At the end of the completion on an artifact, if its configuration shows that it must be completed by other agents (this is the case if the artifact contains buds created by the current agent and whose agent accredited in writing, are remote), replicas of the artifact are sent to the said agents by invoking the service \textit{forwardTo}.
Otherwise, the artifact is complete (it contains no more buds), or semi-complete (it contains buds that had been created by other agents and on which, the current agent is not accredited in writing); in which case, a replica is returned to the agent from which the artifact was previously received by invoking the service \textit{returnTo}.


The execution of the process ends when all the tasks constituting a scenario of the process have been executed. In this case, the artifact that is cooperatively edited is complete (closed) on the agent where the process was triggered.



\mySubSubSection{The Protocols}{}
\label{chap3:sec:the-protocols}

The activity that takes place on an agent in relation to the handling of a given artifact, breaks down into five sub-activities (see fig. \ref{chap3:fig:peer-architecture}); each of them is managed by a dedicated protocol. These activities are the following: 

\begin{itemize}
	\item \textit{creation} (initialisation of a new case) or \textit{receipt-merger} of a replica.
	\item \textit{replication}: it consists in the extraction (from the local replica) of the partial replica that the local agent has to complete (manage its execution).
	\item \textit{execution}: it consists in the extension by the local actor (via the specialised editor) of the buds for which he is accredited in writing.
	\item \textit{expansion-pruning}: it consists in the reconstruction by expansion of the local (global) replica from the updated local partial replica.
	\item \textit{diffusion}: it corresponds to the return of the response to a request, or to the sending of requests.
\end{itemize}
\begin{figure}[ht!]
	\noindent
	\makebox[\textwidth]{\includegraphics[scale=0.28]{./Chap3/images/architecturePair.png}}
	\caption{Activity of an agent in the system.}
	\label{chap3:fig:peer-architecture}
\end{figure}

The management protocols for these different activities are described in the following paragraphs.

~

\noindent\textbf{\textit{The Receipt-Merger Protocol}}

An artifact is received either when a new case is initialised or after a request or a response is delivered. In all cases, a merge using an adaptation of the algorithm in \cite{badouelTchoupeCmcs} is performed. The goal in this step is to update the local copy $t_i$ of the global artifact from those received (the $(t^{maj}_j)_{1 \leq j \neq i \leq k}$ contained in queues $REQ_i$ and $ANS_i$). 
For that purpose (see algorithm \ref{algorithmeFusion}), we merge $t_i$ with each artifact $t^{maj}_j$ from the requests queue (algorithm \ref{algorithmeFusion}, lines 2 to 6) and from the responses queue (algorithm \ref{algorithmeFusion}, lines 7 to 10) until they are empty. 
At each merge, $t_i$ is updated (algorithm \ref{algorithmeFusion}, lines 3 and 8). 
For each received request, the identity of the sender is kept in the list $RET_i$ (algorithm \ref{algorithmeFusion}, line 5) to be able to return a response at the end of the request processing.
Note that during the merge, some previously locked buds can be unlocked: this is the case if all the tasks that precede them have been executed.

\begin{algorithm}
\small
\caption{Merger protocol executed by an agent $i$.\label{algorithmeFusion}}
\begin{algorithmic}[1]
	\Procedure{Merger}{}
		\For{$req : REQ_i$}\Comment{While there is a request}
			\State $t_i \gets merge(t_i,~\textrm{\textit{req.artifact}})$\Comment{We merge the artifact of the request with $t_i$}
			\State delete $req$ from $REQ_i$
			\State $enqueue(req.sender,~RET_i)$\Comment{And we add the request sender in $RET_i$ queue}
		\EndFor
		\For{$ans : ANS_i$}
			\State $t_i \gets merge(t_i,~\textrm{\textit{ans.artifact}})$\Comment{We merge the artifact of the answer with $t_i$}
			\State delete $ans$ from $ANS_i$
		\EndFor
	\EndProcedure
\end{algorithmic}
\end{algorithm}

~

\Needspace{5\baselineskip}
\noindent\textbf{\textit{The Replication Protocol}}

Replication is done just after the merge. The objective here is to update the local partial replica $t_{\mathcal{V}_{i}}$ from the local (global) artifact $t_i$.
To do this (see algorithm \ref{algorithmeReplication}), the local workflow engine proceeds as follows:
\begin{itemize}
	\item It realises the expansion\footnote{It is important to note that the expansion algorithm used here only returns one artifact (see \textit{the expansion-pruning protocol}, page \pageref{chap3:sec:execution-protocol-expansion-pruning}), unlike the one presented in \cite{badouelTchoupeCmcs} which generates a potentially infinite family of artifacts represented by a tree automaton. This uniqueness is guaranteed by the fact that the expansion of $t_{\mathcal{V}_{i}}$ into $t^{maj}_i$ is done using a three-way approach (\textit{three-way merge} \cite{tomMens}). In fact, the expansion is carried out based on the grammatical model $\mathbb{G}$ and on the view $\mathcal{V}_{i}$, but also on the prefix $t_i$ of (the local global artifact replica) $t^{maj}_i$.} of the partial replica $t _{\mathcal{V}_{i}}$ to obtain a global artifact $t^{maj}_i$ which integrates all the updates made during the previous execution (algorithm \ref{algorithmeReplication}, line 2). This operation is necessary, since at the end of the previous expansion, there may have been a pruning (see \textit{the expansion-pruning protocol}, page \pageref{chap3:sec:execution-protocol-expansion-pruning}) which removed from the global artifact local copy $t_i$, some updates contained in $t_{\mathcal{V}_{i}}$;
	\item Then, it merges $t_i$ and $t^{maj}_i$ in one artifact $t_{i_f}$ (algorithm \ref{algorithmeReplication}, line 3);
	\item Finally, it realises the projection of $t_{i_f}$ relatively to the view $\mathcal{V}_{i}$ to obtain the new version of $t_{\mathcal{V}_{i}}=\pi_{\mathcal{V}_i}(t_{i_f})$ (algorithm \ref{algorithmeReplication}, line 4).
\end{itemize}

\begin{algorithm}
\small
\caption{Replication protocol executed by an agent $i$.\label{algorithmeReplication}}
\begin{algorithmic}[1]
	\Procedure{Replication}{}
		\State $t^{maj}_i \gets expand(t_{\mathcal{V}_{i}}, ~t_i, ~\mathcal{V}_{i}, ~\mathbb{G})$
		\State $t_{i_f} \gets merge(t_i,~t^{maj}_i)$
		\State $t_{\mathcal{V}_{i}} \gets \textrm{\textit{projection}}(t_{i_f}, ~\mathcal{V}_{i}, ~\mathbb{G}_{i})$
	\EndProcedure
\end{algorithmic}
\end{algorithm}

~

\noindent\textbf{\textit{The Execution Protocol}}

This protocol (algorithm \ref{algorithmeExecution}) is executed after the production of the local partial replica $t_{\mathcal{V}_{i}}$ by an agent $i$. It is executed by the local actor through the specialised editor, in order to extend the (unlocked) buds of $t_{\mathcal{V}_{i}}$ on which, he is accredited in writing.

The execution of the artifact's local replica by the agent $i$, must be done "as far as possible" by respecting the scheduling (sequential or parallel) of the tasks. Indeed, during the extension of a bud, if there is unlocking or creation of new unlocked buds on which the current agent is accredited in writing, its actor must extend/execute them; this is the purpose of the \textit{while} loop in algorithm \ref{algorithmeExecution}. In addition, the extension of buds whose type $S$ corresponds to a (re)structuring symbol, is automatically done by the local workflow engine when the local GMWf has only one $S$-production.

\begin{algorithm}
\small
\caption{Execution protocol executed by an agent $i$.\label{algorithmeExecution}}
\begin{algorithmic}[1]
	\Procedure{Execution}{}
		\While{$not~isEmpty(buds \gets nextLocalUnlockedBuds(t_{\mathcal{V}_{i}},~\mathcal{A}_{A_i(w)}))$}\Comment{While there are tasks that can be concurrently executed by the actor $A_i$ of agent $i$ in the partial replica $t_{\mathcal{V}_{i}}$}
			\State $bud \gets prompt("Choose~a~task~to~execute",~buds)$\Comment{Actor $A_i$ chooses the task (bud) to execute}
			\State $prods \gets localExecutionPossibilities(bud.type)$\Comment{The specialized editor (agent) generates and activates the set of execution possibilities according to the current (local) configuration}
			\State $choice \gets prompt("Choose~an~execution~possibility",~prods)$\Comment{Actor $A_i$ executes the selected task and provide feedback through the specialized editor}
			\State $t_{\mathcal{V}_{i}} \gets \textrm{\textit{updateArtifact}}(choice,~t_{\mathcal{V}_{i}})$\Comment{Then $t_{\mathcal{V}_{i}}$ is updated accordingly}
		\EndWhile
	\EndProcedure
\end{algorithmic}
\end{algorithm}


~

\noindent\textbf{\textit{The Expansion-Pruning Protocol}}
\label{chap3:sec:execution-protocol-expansion-pruning}

After completion of the partial replica $t_{\mathcal{V}_{i}}$, the updates must be propagated to the local (global) replica $t_i$ of the artifact. This makes it possible to highlight (if they exist) the tasks for which requests must be made, or to determine if answers to requests can be returned.
Algorithm \ref{algorithmeExpansion} allows addressing this concern. For that, the expansion of the local updated partial replica $t_{\mathcal{V}_{i}}$ is made (algorithm \ref{algorithmeExpansion}, line 2) to obtain the global artifact $t^{maj}_i$ which integrates all the contributions made by actor $A_i$ during the previous local execution phase.

Note that the artifact $t^{maj}_i$ may have so called \textit{upstairs buds}\footnote{Intuitively, a node ${n_{X_{\bar{\omega}}}}$ associated with the task $X$ is an \textit{upstair bud} if, $X$ (not already executed) precedes at least one task $Y$ made visible (and naturally not already executed) on the site of an agent $i$, $i$ not having any accreditation in reading on $X$.} (these are the internal nodes of $t^{maj}_i$ that do not belong to $\mathcal{V}_{i}=\mathcal{A}_{A_i (r)}$, and which are not in $t_i$ - see fig. \ref{chap3:fig:execution-figure-2}). 
To prevent and manage this situation, a pruning of $t^{maj}_i$ is performed (algorithm \ref{algorithmeExpansion}, line 3) to ensure compliance with task-related precedence constraints of executions. 
To do this, for every path of $t^{maj}_i$ starting from the root, we prune at the level of the first (upstairs) bud encountered; it must appear unlocked if all the tasks that precede it have already been executed. The artifact obtained after this phase is the new version of $t_i$, and represents the current state of the process execution from the point of view of agent $i$. 

\begin{algorithm}
\small
\caption{Expansion-Pruning protocol executed by an agent $i$.\label{algorithmeExpansion}}
\begin{algorithmic}[1]
	\Procedure{Expansion-Pruning}{}
		\State $t^{maj}_i \gets expand(t_{\mathcal{V}_{i}}, ~t_i, ~\mathcal{V}_{i}, ~\mathbb{G})$
		\State $t_{i} \gets pruning(t^{maj}_i, ~t_i)$%\Comment{We merge the artefact of the request with $t_i$}
	\EndProcedure
\end{algorithmic}
\end{algorithm}

~

\noindent\textbf{\textit{A three-way merging expansion algorithm}}\label{three-way-merge}: consider an (global) artifact under execution $t$, and $t_{\mathcal{V}}=\pi_{\mathcal{V}}\left(t\right)$ its partial replica on the site of an actor $A_i$ whose view is $\mathcal{V}$. Consider the partial replica $t_{\mathcal{V}}^{maj} \geq t_{\mathcal{V}}$ obtained by developing some unlocked buds of $t_{\mathcal{V}}$ as a result of $A_i$'s contribution. The expansion problem consists in finding an (global) artifact under execution $t_f$, which integrates nodes of $t$ and $t_{\mathcal{V}}$. To solve this problem made difficult by the fact that $t$ and $t_{\mathcal{V}}$ are conform to two different models ($\mathbb{G}$ and $\mathbb{G}_{\mathcal{V}} = \Pi_{\mathcal{V}} \left(\mathbb{G} \right)$), we perform a three-way merge {\cite{tomMens}. We merge the artifacts $t$ and $t_{\mathcal{V}}$ using a (global) target artifact $t_g$ such that: 
\begin{enumerate}
	\item[\textbf{(a)}] $t$ is a prefix of $t_g$ ($t \leq t_g$)
	\item[\textbf{(b)}] $t_{\mathcal{V}}^{maj}$ is a prefix of the partial replica of $t_g$ according to $\mathcal{V}$ $\left(t_{\mathcal{V}}^{maj} \leq \pi_{\mathcal{V}}\left(t_g \right)\right)$
\end{enumerate}	
The proposed algorithm proceeds in two steps.
	
~
	
\noindent\textit{Step 1 - Search for the merging guide $t_g$}:
the search of a merging guide is done by the algorithm \ref{chap3:algo:search-guide}.

\begin{algorithm}
\small
\caption{Algorithm to search a merging guide.}
\label{chap3:algo:search-guide}
\begin{mdframed}[style=MyFrame]
	\noindent\textbf{1.}$~$ First of all, we have to generate the set $arts_{\mathbb{G}}=\left\{t_1,\ldots,t_n\right\}$ of target artifacts denoted by $\mathbb{G}$;
	
	\noindent\textbf{2.}$~$ Then, we must filter this set to retain only the artifacts $t_i$ admitting $t$ as a prefix (criterion \textbf{(a)}) and whose projections according to $\mathcal{V}$ ($t_{i_{\mathcal{V}_j}}$) admit $t_{\mathcal{V}}^{maj}$ as a prefix (criterion \textbf{(b)}). It is said that an artifact $t_a$ (whose root node is $n_a=X_a[t_{a_1},\ldots,t_{a_l}]$) is a prefix of a given artifact $t_b$ (whose root node is $n_b=X_b[t_{b_1},\ldots,t_{b_m}]$) if and only if the root nodes $n_a$ and $n_b$ are of the same types (i.e $X_a=X_b$) and:
	%\begin{itemize}[leftmargin=*]
	
		$~~$\textbf{a.}$~$ The node $n_a$ is a bud or,
		
		$~~$\textbf{b.}$~$ The nodes $n_a$ and $n_b$ have the same number of sub-artifacts (i.e $l=m$), the same type of scheduling for the sub-artifacts and each sub-artifact $t_{a_i}$ of $n_a$ is a prefix of the sub-artifact $t_{b_i}$ of $n_b$.
	%\end{itemize}
	
	\noindent We obtain the set $guides=\left\{t_{g_1},\ldots,t_{g_k}\right\}$ of artifacts that can guide the merging;
	
	\noindent\textbf{3.}$~$ Finally, we randomly select an element $t_g$ from the set $guides$.
	%\end{enumerate}
\end{mdframed}
\end{algorithm}


~

\noindent\textit{Step 2 - Merging $t$, $t_{\mathcal{V}}^{maj}$ and $t_g$}:
we want to find an artifact $t_f$ that includes all the contributions already made during the workflow execution. The structure of the searched artifact $t_f$ is the same as that of $t_g$: hence the interest to use $t_g$ as a guide. The merging is carried out by the algorithm \ref{chap3:algo:three-way-merge}.

\begin{algorithm}
\small
\caption{Three-way merging algorithm.}
\label{chap3:algo:three-way-merge}
\begin{mdframed}[style=MyFrame]
	\noindent A prefixed depth path of the three artifacts ($t$, $t_{\mathcal{V}}^{maj}$ and $t_g$) is made simultaneously until there is no longer a node to visit in $t_g$. Let $n_{t_i}$ (resp. $n_{t_{\mathcal{V}_j}^{maj}}$ and $n_{t_{g_k}}$) be the node located at the address $w_i$ (resp. $w_j$ and $w_k$) of $t$ (resp. $t_{\mathcal{V}}^{maj}$ and $t_g$) and currently being visited. If nodes $n_{t_i}$, $n_{t_{\mathcal{V}_j}^{maj}}$ and $n_{t_{g_k}}$ are such that (\textbf{processing}):
	
	%\begin{itemize}[leftmargin=*]
	\noindent\textbf{1.}$~$ $n_{t_{\mathcal{V}_j}^{maj}}$ is associated with a (re)structuring symbol (fig. \ref{chap3:fig:expansion-pattern}(d)) then: we take a step forward in the depth path of $t_{\mathcal{V}}^{maj}$ and we resume processing;
	
	\noindent\textbf{2.}$~$ $n_{t_i}$, $n_{t_{\mathcal{V}_j}^{maj}}$ and $n_{t_{g_k}}$ exist and are all associated with the same symbol $X$ (fig. \ref{chap3:fig:expansion-pattern}(a) and \ref{chap3:fig:expansion-pattern}(b)) then:
	%\begin{enumerate}[leftmargin=*]
	we insert $n_{t_{\mathcal{V}_j}^{maj}}$ (it is the most up-to-date node) into $t_f$ at the address $w_k$; 
	if $n_{t_{\mathcal{V}_j}^{maj}}$ is a bud then we prune (delete sub-artifacts) $t_g$ at the address $w_k$; 
	we take a step forward in the depth path of the three artifacts and we resume processing.
	%\end{enumerate}
	
	\noindent\textbf{3.}$~$ $n_{t_i}$, $n_{t_{\mathcal{V}_j}^{maj}}$ and $n_{t_{g_k}}$ exist and are respectively associated with symbols $X_i$, $X_j$ and $X_k$ such that $X_k \neq X_i$ and $X_k \neq X_j$ (fig. \ref{chap3:fig:expansion-pattern}(e)) then: 
	%\begin{enumerate}[leftmargin=*]
	we add $n_{t_{g_k}}$ in $t_f$ at address $w_k$. This is an upstair bud; 
	we take a step forward in the depth path of $t_g$ and we resume processing.
	%\end{enumerate}
	
	\noindent\textbf{4.}$~$ $n_{t_i}$ (resp. $n_{t_{\mathcal{V}_j}^{maj}}$) and $n_{t_{g_k}}$ exist and are associated with the same symbol $X$ (fig. \ref{chap3:fig:expansion-pattern}(c) and \ref{chap3:fig:expansion-pattern}(f)) then: 
	%\begin{enumerate}[leftmargin=*]
	we insert $n_{t_i}$ (resp. $n_{t_{\mathcal{V}_j}^{maj}}$) into $t_f$ at the address $w_k$;
	if $n_{t_i}$ (resp. $n_{t_{\mathcal{V}_j}^{maj}}$) is a bud, we prune $t_g$ at the address $w_k$; 
	we take a step forward in the depth path of the artifacts $t$ (resp. $t_{\mathcal{V}}^{maj}$) and $t_g$, then we resume processing.
	%\end{enumerate}
	%\end{itemize}
\end{mdframed}
\end{algorithm}

\begin{figure}[ht!]
	\noindent
	\makebox[\textwidth]{\includegraphics[scale=0.27]{./Chap3/images/expansionMergePattern.png}}
	\caption{Some scenarios to be managed during the expansion.}
	\label{chap3:fig:expansion-pattern}
\end{figure}

As for the other key algorithms, a Haskell implementation of this expansion-pruning algorithm is introduced in appendix \ref{appendice1:algorithms-implementations}.

\begin{proposition}
	\label{propositionAtLeastOneGuide}
	For any update $t_{\mathcal{V}}^{maj}$ in accordance with a GMWf $\mathbb{G}_{\mathcal{V}} = \Pi_{\mathcal{V}}\left(\mathbb{G} \right)$, of a partial replica $t_{\mathcal{V}}=\pi_{\mathcal{V}}\left(t\right)$ obtained by projecting (according to the view $\mathcal{V}$) an artifact $t$ being executed in accordance with the GMWf $\mathbb{G}$ of a GMAWfP verifying the axioms' visibility and the non-recursivity assumptions, there is at least one target artifact (the three-way merge guide) $t_g \in arts_{\mathbb{G}}$ of $\mathbb{G}$ such as:
	\begin{enumerate}
		\item[\textbf{(a)}] $t$ is a prefix of $t_g$ ($t \leq t_g$)
		\item[\textbf{(b)}] $t_{\mathcal{V}}^{maj}$ is a prefix of the partial replica of $t_g$ according to $\mathcal{V}$ $\left(t_{\mathcal{V}}^{maj} \leq \pi_{\mathcal{V}}\left(t_g \right)\right)$
	\end{enumerate}	
\end{proposition}	

\begin{proof}
	Thanks to the proposals \ref{propositionStabiliteProjGMWf}, \ref{propositionCoherenceArtefact} and the artifact editing model used here (see sec. \ref{chap3:sec:artifact-edition}) it is established that since the artifact $t$ being executed in accordance with $\mathbb{G}$ is a prefix of a non-empty set of $\mathbb{G}$'s target artifacts $arts_{\mathbb{G}}^{'} = \left\{t_{1}^{'},\ldots, t_{n}^{'}\right\}$ ($\forall 1 \leq i \leq n, ~t \leq t_{i}^{'}$), its projection $t_{\mathcal{V}}$ according to the view $\mathcal{V}$ is a prefix of a non-empty set $arts_{\mathbb{G}_{\mathcal{V}}}^{'} = \left\{t_{{\mathcal{V}}_{1}}^{'},\ldots, t_{{\mathcal{V}}_{m}}^{'}\right\}$ of $\mathbb{G}_{\mathcal{V}} = \Pi_{\mathcal{V}}\left(\mathbb{G} \right)$'s local target artifacts for the said view ($\forall 1 \leq j \leq m, ~t_{\mathcal{V}} \leq t_{{\mathcal{V}}_{j}}^{'}$): elements of $arts_{\mathbb{G}}^{'}$ are potential merging guides candidates that all verify the property \textbf{(a)}. In addition, using the propositions \ref{propositionStabiliteProjGMWf} and \ref{propositionReciproqueCoherenceArtefact}, it is established that each element of $arts_{\mathbb{G}_{\mathcal{V}}}^{'}$ is the projection of at least one element of $arts_{\mathbb{G}}^{'}$ according to the view $\mathcal{V}$ \textbf{(1)}. Given that $t_{\mathcal{V}}^{maj}$ is obtained by developing buds of $t_{\mathcal{V}}$ in accordance with $\mathbb{G}_{\mathcal{V}}$, it is inferred that $t_{\mathcal{V}}^{maj}$ is a prefix of a non-empty subset $arts_{\mathbb{G}_{\mathcal{V}}}^{maj} \subseteq arts_{\mathbb{G}_{\mathcal{V}}}^{'}$ of local target artifacts for the view $\mathcal{V}$ \textbf{(2)}. With the proposition \ref{propositionReciproqueCoherenceArtefact} once again, it is determined that for each artifact $t_{{\mathcal{V}}_{j}}^{'} \in arts_{\mathbb{G}_{\mathcal{V}}}^{maj}$, there is at least one artifact $t_{g_j}$ that is conform to $\mathbb{G}$ such as $t_{{\mathcal{V}}_{j}}^{'} = \pi_{\mathcal{V}}\left(t_{g_j} \right)$: this new set $arts_{\mathbb{G}}^{maj} = \left\{t_{g_1},\ldots, t_{g_k}\right\}$ is made up of potential merging guides candidates that all verify the property \textbf{(b)}. Results \textbf{(1)} and \textbf{(2)} show that $arts_{\mathbb{G}}^{maj}$ and $arts_{\mathbb{G}}^{'}$ are not disjoint. As a consequence, the set $guides= arts_{\mathbb{G}}^{maj} \cap arts_{\mathbb{G}}^{'}$ of potential merging guides that all verify both property \textbf{(a)} and \textbf{(b)} is not empty.
\end{proof}

\begin{corollary}
	\label{propositionUniqueExpansion}
	For an artifact $t$ being executed in accordance with a GMWf $\mathbb{G}$ of a GMAWfP verifying the axioms' visibility and the non-recursivity assumptions, and an update $t_{\mathcal{V}}^{maj} \geq t_{\mathcal{V}}$ of its partial replica $t_{\mathcal{V}}=\pi_{\mathcal{V}}\left(t\right)$ according to the view $\mathcal{V}$, the expansion of $t_{\mathcal{V}}^{maj}$ contains at least one artifact and the expansion-pruning algorithm presented here returns one and only one artifact.
\end{corollary}

\begin{comment}
\begin{proof}
	The proof of this corollary derives from the proof of the proposition \ref{propositionAtLeastOneGuide} and from the fact that in the last instruction of the algorithm \ref{chap3:algo:search-guide}, an artifact is randomly selected an returned from a non-empty set of potential guides.
\end{proof}
\end{comment}
This result (corollary \ref{propositionUniqueExpansion}) derives from the proof of the proposition \ref{propositionAtLeastOneGuide} (\textit{there is always at least one artifact in the expansion of $t_{\mathcal{V}}^{maj}$ under the conditions of corollary \ref{propositionUniqueExpansion}}) and from the fact that in the last instruction of the algorithm \ref{algo:search-guide}, an artifact is randomly selected an returned from a non-empty set of potential guides (\textit{only one of the expansion artifacts is used in the three-way merging}).

~

\noindent\textbf{\textit{The Diffusion Protocol}}
\label{chap3:sec:execution-protocol-diffusion}

After expansion-pruning, the local workflow engine must examine whether requests need to be sent to other agents (this is the case if $t_i$ still have unlocked buds created on its site\footnote{An agent only requests the execution of a bud if it was created on its site.}, on which the current agent is not accredited in writing) or, if responses are to be returned (this is the case if $t_i$ is complete or semi-complete).

To build the list of requests to diffuse, the local workflow engine scans the artifact $t_i$ produced by expansion-pruning and builds the list of required agents (those to receive a request) from buds\footnote{Normally (due to assumptions of our model) at this stage, for each unlocked bud of $t_i$, agent $i$ is accredited in execution on the associated task. Any other situation would be a design flaw.} (algorithm \ref{algorithmeDiffusion}, lines 2 to 7). If the required agents list is not empty (the artifact is not complete), it sends a request to each agent in the list (algorithm \ref{algorithmeDiffusion}, lines 8 to 12). Otherwise, if there are agents who have previously made requests, it sends responses instead (algorithm \ref{algorithmeDiffusion}, lines 13 to 18).

\begin{algorithm}
\small
\caption{Diffusion protocol executed by an agent $i$.\label{algorithmeDiffusion}}
\begin{algorithmic}[1]
	\Procedure{Diffusion}{}
		\For{$bud : unlockedBuds(t_i)$ and $bud$ have been created by $i$}%\Comment{For each unlocked bud in the updated artefact}
			\If{$bud.type \in \mathcal{A}_{A_i(x)}$}%\Commnent{If peer $i$ can ask the execution of this task}
				\State $agent \gets \textrm{\textit{executorOf}}(bud.type)$
				\State $enqueue(agent, requiredAgents)$
			\EndIf
			\EndFor
			\If {$not~isEmpty(requiredAgents)$}
				\State $req \gets new~Request(i,~t_i)$
			  \For{$agent : requiredAgents$}
					\State $invoqueService("forwardTo",~req,~agent)$
				\EndFor		
			\ElsIf{$not~isEmpty(RET_i)$}
				\State $ans \gets new~Answer(i,~t_i)$
			 	\For{$agent : RET_i$}
					\State $invoqueService("returnTo",~ans,~agent)$
					\State $delete~agent~from~RET_i$
				\EndFor
			\Else
				\State $alert("The~process~execution~is~terminated.")$
			\EndIf
	\EndProcedure
\end{algorithmic}
\end{algorithm}


\mySection{Illustrating the Choreography on the Peer-Review Process}{}
\label{chap3:sec:choreograpghy-illustration}

The execution of an instance of our running example begins when under the editor in chief's action (via the GUI of the specialised editor), an unlocked bud of type $A_{\mathbb{G}}$ is created on his site.
Figure \ref{chap3:fig:execution-figure-1} which must be read following the direction of the arrows it contains, summarises the state\footnote{The state of an agent $i$ at a given moment is given by the values of variables $REQ_i$, $ANS_i$, $RET_i$ and the replicas $t_i$ and $t_{\mathcal{V}_i}$.} of the agent $EC$ (editor in chief site) before and after the event creating the artifact; it also illustrates the running of the five-step protocol on the agent $EC$.
\begin{figure}[ht!]
	\noindent
	\makebox[\textwidth]{\includegraphics[scale=0.28]{./Chap3/images/executionFigure1.png}}
	\caption{Beginning of the peer-review process on the editor in chief's site.}
	\label{chap3:fig:execution-figure-1}
\end{figure}

As soon as a bud of type $A_{\mathbb{G}}$ is created, the local workflow engine extends it using the unique $A_{\mathbb{G}}$-production ($P_{1}:A_{\mathbb{G}} \rightarrow A$) of the local GMWf. This results in the creation of a bud of type $A$ that the editor in chief must extend via the specialised editor by choosing an $A$-production.
For this scenario, it is assumed that he chooses the production $P_{3}:A \rightarrow C \fatsemi D$.
The task ($A$) executed by the latter is shown in green colour on figure \ref{chap3:fig:execution-figure-1}.
The newly created tasks ($C$ and $D$) appear in the form of locked buds (the locked buds are shown in red colour) because the editor in chief is not accredited in writing on $C$ and, since $D$ is linked to $C$ by a sequential scheduling constraint, it can only be executed when all tasks ($C$, $E$, $F$, $G1$, $G2$, $H1$, $H2$, $I1$, $I2$) preceding it will have been executed.
After expansion-pruning, the only required agent is the associated editor (responsible for executing task $C$): a request is sent to it by invoking the service \textit{forwardTo}.
\begin{figure}[ht!]
	\noindent
	\makebox[\textwidth]{\includegraphics[scale=0.28]{./Chap3/images/executionFigure2.png}}
	\caption{Continuation of the peer-review process execution on the associated editor's site; the latter receives the request formulated by the editor in chief.}
	\label{chap3:fig:execution-figure-2}
\end{figure}

The event that triggers the workflow execution on the site of associated editor (fig. \ref{chap3:fig:execution-figure-2}) is the receipt of the request sent by the editor in chief.
In the artifact sent by the latter, there are buds ($C_{\overline{\omega}}$ and $D_{\overline{\omega}}$).
After merging, the bud $C_{\omega}$ is unlocked (the unlocked buds are shown in blue). It indicates the only place where the contribution of the associated editor is expected.
During the execution phase, the local artifact partial replica is updated by the associated editor via the productions $P_{3}: C\rightarrow E\fatsemi F$, $P_{4}: E\rightarrow S1\parallel S2$, $P_{5}: S1\rightarrow H1\fatsemi I1$ and $P_{6}: S2\rightarrow H2\fatsemi I2$ of his local GMWf. 
At the end of this phase, buds of types $H1, H2, I1$ and $I2$ appear locked not only because they are constrained by a sequential scheduling (case of $I1$ and $I2$), but especially because of the presence of \textit{upstairs buds} (the upstairs buds are represented in orange colour).
Indeed, tasks $G1$ and $G2$ made visible after the expansion are upstairs buds because they must be executed before the tasks of type $H1$ and $H2$. So, there is pruning at $G1$ and $G2$ before sending (in parallel) the artifact to both referees.
\begin{figure}[ht!]
	\noindent
	\makebox[\textwidth]{\includegraphics[scale=0.28]{./Chap3/images/executionFigure3.png}}
	\caption{Continuation of the peer-review process execution on the first referee's site: the request of the associated editor arrives at the first referee.}
	\label{chap3:fig:execution-figure-3}
\end{figure}

Figure \ref{chap3:fig:execution-figure-3} illustrates how the protocol takes place on the site of one of the referees (the first referee). After the contribution of the latter, no new bud is created: no request is formulated. It is rather a response corresponding to the request previously received from the associated editor which is returned by invoking the service $returnTo$.
\begin{figure}[ht!]
	\noindent
	\makebox[\textwidth]{\includegraphics[scale=0.28]{./Chap3/images/executionFigure4.png}}
	\caption{Continuation of the peer-review process execution: the associated editor receives answers from referees, to requests that he has previously made.}
	\label{chap3:fig:execution-figure-4}
\end{figure}

The execution protocol is unrolled again on the site of the associated editor following events related to the reception of responses from the two referees (fig. \ref{chap3:fig:execution-figure-4}). We choose to treat these responses simultaneously; but we could do otherwise and obtain the same result. 
At merge, since the subtree rooted in $E$ is closed, the bud $F_{\overline{\omega}}$ is unlocked and the associated editor extends it through production $P_{11}: F \rightarrow \varepsilon$. Having no request to make, the answer to the request previously received from the editor in chief is returned.
\begin{figure}[ht!]
	\noindent
	\makebox[\textwidth]{\includegraphics[scale=0.28]{./Chap3/images/executionFigure5.png}}
	\caption{Continuation and end of the peer-review process execution: the editor in chief receives a response containing referees' contributions, from the associated editor.}
	\label{chap3:fig:execution-figure-5}
\end{figure}

The editor in chief receives the response from associated editor and once again runs the execution protocol (fig. \ref{chap3:fig:execution-figure-5}). 
After its contribution (on the node $D$), the artifact obtained after expansion-pruning is closed and the execution of the process ends successfully.

\mySection{Experimentation}{}
\label{chap3:sec:p2ptinywfms}

In this section, we present and experiment \textit{P2PTinyWfMS} (a Peer-to-Peer Tiny Workflow Management System), an experimental prototype system implemented according to the approach proposed in this chapter.

\mySubSection{P2PTinyWfMS: an Experimental Prototype System}{}
\textit{P2PTinyWfMS} is a tool developed in Java under Eclipse\footnote{Official website of Eclipse: \url{https://www.eclipse.org}, visited the 04/04/2020.} and dedicated to the distributed execution of administrative workflows specified using GMWf. 
In accordance with the agent's architecture of this chapter (see fig. \ref{chap3:fig:peer-architecture}), \textit{P2PTinyWfMS} has a front-end for displaying and graphically editing artifacts manipulated during the execution of a business process (see fig. \ref{chap3:sec:p2ptinywfms-1} and \ref{chap3:sec:p2ptinywfms-3}), as well as a communication module built from SON\footnote{SON is available under Eclipse from a family of SmartTools plugins.}. 

Let's recall that SON (Shared-data Overlay Network) \cite{SON} is a middleware offering several DSL to facilitate the implementation of P2P systems whose components communicate by service invocations. 
Component Description Meta Language (CDML) is the DSL provided by SON to specify among other things the services required and provided by each peers; from a CDML specification, SON generates Java code for allowing peers to communicate.
The following listing shows the contents of the CDML file used in the case of \textit{P2PTinyWfMS} to specify the four services that its instances expose; they are: two input services (\textit{inForwardTo} - lines 6 to 9 -, and \textit{inReturnTo} - lines 10 to 13 -) and two output services (\textit{outForwardTo} - lines 14 to 17 - and \textit{outReturnTo} - lines 18 to 21 -). These services take as argument an artifact corresponding to either a request or a response.

\begin{Verbatim}[frame=lines,fontsize=\scriptsize, numbers=left, numbersep=8pt, label=CDML file: specification of required and provided services of P2PTinyWfMS]
<?xml version="1.0" encoding="ISO-8859-1"?>
<component name="p2pTinyWfMS" type="p2pTinyWfMS" extends="inria.communicationprotocol"
 ns="p2pTinyWfMS">
  <containerclass name="P2pTinyWfMSContainer"/>
  <facadeclass name="P2pTinyWfMSFacade" userclassname="P2pTinyWfMS"/>
  <input name="forwardTo" method="inForwardTo">
    <attribute name="request" 
     javatype="smartworkflow.dwfms.lifa.miu.util.p2pworkflow.PeerToPeerWorkflowRequest"/>
  </input>
  <input name="returnTo" method="inReturnTo">
    <attribute name="response" 
     javatype="smartworkflow.dwfms.lifa.miu.util.p2pworkflow.PeerToPeerWorkflowResponse"/>
  </input>
  <output name="forwardTo" method="outForwardTo">
    <attribute name="request" 
     javatype="smartworkflow.dwfms.lifa.miu.util.p2pworkflow.PeerToPeerWorkflowRequest"/>
  </output>
  <output name="returnTo" method="outReturnTo">
    <attribute name="response" 
     javatype="smartworkflow.dwfms.lifa.miu.util.p2pworkflow.PeerToPeerWorkflowResponse"/>
  </output>
</component>
\end{Verbatim}


\mySubSection{Executing our Running Example under P2PTinyWfMS}{}
SON offers a DSL (the "\textit{.world}" files) for the description of the deployment of a distributed system whose components have been specified by a CDML file. In order to execute our running example, we deployed four instances of \textit{P2PTinyWfMS} identified by $EC$, $AE$, $R1$ and $R2$ respectively. As explained in section \ref{chap3:sec:choreograpghy-illustration}, each instance is initially equipped with the global GMWf as well as accreditations of various agents from which it derives its local GMWf by projection.

Figures \ref{chap3:sec:p2ptinywfms-1}, \ref{chap3:sec:p2ptinywfms-3} and \ref{chap3:sec:p2ptinywfms-6} are screen shots with some highlights of the workflow's distributed execution.
We have the tab "\textit{Workflow overview}" presenting at the beginning of the execution, various tasks, agents, target artifacts etc., on the editor in chief's site (fig. \ref{chap3:sec:p2ptinywfms-1}). We also have the tabs "\textit{Workflow execution}" of the sites of the associated editor (fig. \ref{chap3:sec:p2ptinywfms-3}) and of the editor in chief (fig. \ref{chap3:sec:p2ptinywfms-6}) that present the artifacts resulting from their execution after receiving a request from the editor in chief (resp. after receiving a response from the associated editor).
\begin{figure}[ht!]
	\noindent
	\makebox[\textwidth]{\includegraphics[scale=0.43]{./Chap3/images/p2ptinywfms-1.png}}
	\caption{P2pTinyWfMS on the editor in chief's site: presentation of the GMWf (the tasks and their relations, the actors and their accreditations).}
	\label{chap3:sec:p2ptinywfms-1}
\end{figure}

\begin{figure}[ht!]
	\noindent
	\makebox[\textwidth]{\includegraphics[scale=0.43]{./Chap3/images/p2ptinywfms-3.png}}
	\caption{P2pTinyWfMS on the associated editor's site: receipt of editor in chief's request, execution of tasks, expansion-pruning, and diffusion.}
	\label{chap3:sec:p2ptinywfms-3}
\end{figure}

\begin{figure}[ht!]
	\noindent
	\makebox[\textwidth]{\includegraphics[scale=0.43]{./Chap3/images/p2ptinywfms-6.png}}
	\caption{P2pTinyWfMS on the editor in chief's site: reception of the associated editor's response, execution of tasks, expansion-pruning and end of the case.}
	\label{chap3:sec:p2ptinywfms-6}
\end{figure}

\mySection{Related Works and Discussion}{}
\label{chap3:sec:discussion}

In this section we briefly discuss the similarities and differences of the model presented in this chapter, comparing it with some related work presented earlier (Chapter \ref{chap1:artifact-centric-bpm}). We will mention a few related studies and discuss directly; a more formal comparative study using qualitative and quantitative metrics should be the subject of future work.

Hull et al. \citeyearpar{hull2009facilitating} provide an interoperation framework in which, data are hosted on central infrastructures named \textit{artifact-centric hubs}. As in the work presented in this chapter, they propose mechanisms (including user views) for controlling access to these data. Compared to choreography-like approach as the one presented in this chapter, their settings has the advantage of providing a conceptual rendezvous point to exchange status information. The same purpose can be replicated in this chapter's approach by introducing a new type of agent called "\textit{monitor}", which will serve as a rendezvous point; the behaviour of the agents will therefore have to be slightly adapted to take into account the monitor and to preserve as much as possible the autonomy of agents.

Lohmann and Wolf \citeyearpar{lohmann2010artifact} abandon the concept of having a single artifact hub \cite{hull2009facilitating} and they introduce the idea of having several agents which operate on artifacts. Some of those artifacts are mobile; thus, the authors provide a systematic approach for modelling artifact location and its impact on the accessibility of actions using a Petri net. Even though we also manipulate mobile artifacts, we do not model artifact location; rather, our agents are equipped with capabilities that allow them to manipulate the artifacts appropriately (taking into account their location). Moreover, our approach considers that artifacts can not be remotely accessed, this increases the autonomy of agents.

The process design approach presented in this chapter, has some conceptual similarities with the concept of \textit{proclets} proposed by Wil M. P. van der Aalst et al. \citeyearpar{van2001proclets, van2009workflow}: they both split the process when designing it. In the model presented in this chapter, the process is split into execution scenarios and its specification consists in the diagramming of each of them. Proclets \cite{van2001proclets, van2009workflow} uses the concept of \textit{proclet-class} to model different levels of granularity and cardinality of processes. Additionally, proclets act like agents and are autonomous enough to decide how to interact with each other.

The model presented in this chapter uses an attributed grammar as its mathematical foundation. This is also the case of the AWGAG model by Badouel et al. \citeyearpar{badouel14, badouel2015active}. However, their model puts stress on modelling process data and users as first class citizens and it is designed for Adaptive Case Management.

To summarise, the proposed approach in this chapter allows the modelling and decentralized execution of administrative processes using autonomous agents. In it, process management is very simply done in two steps. The designer only needs to focus on modelling the artifacts in the form of task trees and the rest is easily deduced. Moreover, we propose a simple but powerful mechanism for securing data based on the notion of accreditation; this mechanism is perfectly composed with that of artifacts. The main strengths of our model are therefore : 
\begin{itemize}
	\item The simplicity of its syntax (process specification language), which moreover (well helped by the accreditation model), is suitable for administrative processes;
	\item The simplicity of its execution model; the latter is very close to the blockchain's execution model \cite{hull2017blockchain, mendling2018blockchains}. On condition of a formal study, the latter could possess the same qualities (fault tolerance, distributivity, security, peer autonomy, etc.) that emanate from the blockchain;
	\item Its formal character, which makes it verifiable using appropriate mathematical tools;
	\item The conformity of its execution model with the agent paradigm and service technology.
\end{itemize}
In view of all these benefits, we can say that the objectives set for this thesis have indeed been achieved. However, the proposed model is perfectible. For example, it can be modified to permit agents to respond incrementally to incoming requests as soon as any prefix of the extension of a bud is produced. This makes it possible to avoid the situation observed on figure \ref{chap3:fig:execution-figure-4} where the associated editor is informed of the evolution of the subtree resulting from $C$ only when this one is closed. All the criticisms we can make of the proposed model in particular, and of this thesis in general, have been introduced in the general conclusion (page \pageref{chap5:general-conclusion}) of this manuscript.




% \vspace{-0.5em}
\section{Conclusion}
% \vspace{-0.5em}
Recent advances in multimodal single-cell technology have enabled the simultaneous profiling of the transcriptome alongside other cellular modalities, leading to an increase in the availability of multimodal single-cell data. In this paper, we present \method{}, a multimodal transformer model for single-cell surface protein abundance from gene expression measurements. We combined the data with prior biological interaction knowledge from the STRING database into a richly connected heterogeneous graph and leveraged the transformer architectures to learn an accurate mapping between gene expression and surface protein abundance. Remarkably, \method{} achieves superior and more stable performance than other baselines on both 2021 and 2022 NeurIPS single-cell datasets.

\noindent\textbf{Future Work.}
% Our work is an extension of the model we implemented in the NeurIPS 2022 competition. 
Our framework of multimodal transformers with the cross-modality heterogeneous graph goes far beyond the specific downstream task of modality prediction, and there are lots of potentials to be further explored. Our graph contains three types of nodes. While the cell embeddings are used for predictions, the remaining protein embeddings and gene embeddings may be further interpreted for other tasks. The similarities between proteins may show data-specific protein-protein relationships, while the attention matrix of the gene transformer may help to identify marker genes of each cell type. Additionally, we may achieve gene interaction prediction using the attention mechanism.
% under adequate regulations. 
% We expect \method{} to be capable of much more than just modality prediction. Note that currently, we fuse information from different transformers with message-passing GNNs. 
To extend more on transformers, a potential next step is implementing cross-attention cross-modalities. Ideally, all three types of nodes, namely genes, proteins, and cells, would be jointly modeled using a large transformer that includes specific regulations for each modality. 

% insight of protein and gene embedding (diff task)

% all in one transformer

% \noindent\textbf{Limitations and future work}
% Despite the noticeable performance improvement by utilizing transformers with the cross-modality heterogeneous graph, there are still bottlenecks in the current settings. To begin with, we noticed that the performance variations of all methods are consistently higher in the ``CITE'' dataset compared to the ``GEX2ADT'' dataset. We hypothesized that the increased variability in ``CITE'' was due to both less number of training samples (43k vs. 66k cells) and a significantly more number of testing samples used (28k vs. 1k cells). One straightforward solution to alleviate the high variation issue is to include more training samples, which is not always possible given the training data availability. Nevertheless, publicly available single-cell datasets have been accumulated over the past decades and are still being collected on an ever-increasing scale. Taking advantage of these large-scale atlases is the key to a more stable and well-performing model, as some of the intra-cell variations could be common across different datasets. For example, reference-based methods are commonly used to identify the cell identity of a single cell, or cell-type compositions of a mixture of cells. (other examples for pretrained, e.g., scbert)


%\noindent\textbf{Future work.}
% Our work is an extension of the model we implemented in the NeurIPS 2022 competition. Now our framework of multimodal transformers with the cross-modality heterogeneous graph goes far beyond the specific downstream task of modality prediction, and there are lots of potentials to be further explored. Our graph contains three types of nodes. while the cell embeddings are used for predictions, the remaining protein embeddings and gene embeddings may be further interpreted for other tasks. The similarities between proteins may show data-specific protein-protein relationships, while the attention matrix of the gene transformer may help to identify marker genes of each cell type. Additionally, we may achieve gene interaction prediction using the attention mechanism under adequate regulations. We expect \method{} to be capable of much more than just modality prediction. Note that currently, we fuse information from different transformers with message-passing GNNs. To extend more on transformers, a potential next step is implementing cross-attention cross-modalities. Ideally, all three types of nodes, namely genes, proteins, and cells, would be jointly modeled using a large transformer that includes specific regulations for each modality. The self-attention within each modality would reconstruct the prior interaction network, while the cross-attention between modalities would be supervised by the data observations. Then, The attention matrix will provide insights into all the internal interactions and cross-relationships. With the linearized transformer, this idea would be both practical and versatile.

% \begin{acks}
% This research is supported by the National Science Foundation (NSF) and Johnson \& Johnson.
% \end{acks}



	\mathversion{normal}
	%\input{Chap4/Chap4}
	
	%Ainsi de suite
	
	% \vspace{-0.5em}
\section{Conclusion}
% \vspace{-0.5em}
Recent advances in multimodal single-cell technology have enabled the simultaneous profiling of the transcriptome alongside other cellular modalities, leading to an increase in the availability of multimodal single-cell data. In this paper, we present \method{}, a multimodal transformer model for single-cell surface protein abundance from gene expression measurements. We combined the data with prior biological interaction knowledge from the STRING database into a richly connected heterogeneous graph and leveraged the transformer architectures to learn an accurate mapping between gene expression and surface protein abundance. Remarkably, \method{} achieves superior and more stable performance than other baselines on both 2021 and 2022 NeurIPS single-cell datasets.

\noindent\textbf{Future Work.}
% Our work is an extension of the model we implemented in the NeurIPS 2022 competition. 
Our framework of multimodal transformers with the cross-modality heterogeneous graph goes far beyond the specific downstream task of modality prediction, and there are lots of potentials to be further explored. Our graph contains three types of nodes. While the cell embeddings are used for predictions, the remaining protein embeddings and gene embeddings may be further interpreted for other tasks. The similarities between proteins may show data-specific protein-protein relationships, while the attention matrix of the gene transformer may help to identify marker genes of each cell type. Additionally, we may achieve gene interaction prediction using the attention mechanism.
% under adequate regulations. 
% We expect \method{} to be capable of much more than just modality prediction. Note that currently, we fuse information from different transformers with message-passing GNNs. 
To extend more on transformers, a potential next step is implementing cross-attention cross-modalities. Ideally, all three types of nodes, namely genes, proteins, and cells, would be jointly modeled using a large transformer that includes specific regulations for each modality. 

% insight of protein and gene embedding (diff task)

% all in one transformer

% \noindent\textbf{Limitations and future work}
% Despite the noticeable performance improvement by utilizing transformers with the cross-modality heterogeneous graph, there are still bottlenecks in the current settings. To begin with, we noticed that the performance variations of all methods are consistently higher in the ``CITE'' dataset compared to the ``GEX2ADT'' dataset. We hypothesized that the increased variability in ``CITE'' was due to both less number of training samples (43k vs. 66k cells) and a significantly more number of testing samples used (28k vs. 1k cells). One straightforward solution to alleviate the high variation issue is to include more training samples, which is not always possible given the training data availability. Nevertheless, publicly available single-cell datasets have been accumulated over the past decades and are still being collected on an ever-increasing scale. Taking advantage of these large-scale atlases is the key to a more stable and well-performing model, as some of the intra-cell variations could be common across different datasets. For example, reference-based methods are commonly used to identify the cell identity of a single cell, or cell-type compositions of a mixture of cells. (other examples for pretrained, e.g., scbert)


%\noindent\textbf{Future work.}
% Our work is an extension of the model we implemented in the NeurIPS 2022 competition. Now our framework of multimodal transformers with the cross-modality heterogeneous graph goes far beyond the specific downstream task of modality prediction, and there are lots of potentials to be further explored. Our graph contains three types of nodes. while the cell embeddings are used for predictions, the remaining protein embeddings and gene embeddings may be further interpreted for other tasks. The similarities between proteins may show data-specific protein-protein relationships, while the attention matrix of the gene transformer may help to identify marker genes of each cell type. Additionally, we may achieve gene interaction prediction using the attention mechanism under adequate regulations. We expect \method{} to be capable of much more than just modality prediction. Note that currently, we fuse information from different transformers with message-passing GNNs. To extend more on transformers, a potential next step is implementing cross-attention cross-modalities. Ideally, all three types of nodes, namely genes, proteins, and cells, would be jointly modeled using a large transformer that includes specific regulations for each modality. The self-attention within each modality would reconstruct the prior interaction network, while the cross-attention between modalities would be supervised by the data observations. Then, The attention matrix will provide insights into all the internal interactions and cross-relationships. With the linearized transformer, this idea would be both practical and versatile.

% \begin{acks}
% This research is supported by the National Science Foundation (NSF) and Johnson \& Johnson.
% \end{acks}
	
	%************ Bibliographie ***************
	% La charte de l'école doctorale recommande un style dans lequel les citations seront de la forme (NomAuteur, Année) ou (NomAuteur et al., Année)
	%\myBibliography{style}{url du fichier .bib}
	\myBibliography{apacite}{bibliography}
	
	% *********** Annexes *********************
	\appendix
	
	\myChapter{Implementation of Some Important Algorithms Presented in this Thesis}{}
\label{appendice1:algorithms-implementations}
\mySaveMarks

We had thought to present in this appendix, a Haskell implementation of the projection algorithms proposed in chapter \ref{chap3:choreography-workflow-design-execution} of this thesis. However, these are far too voluminous and their presentation here will not be very readable. We have therefore decided to present only the main data types here. We have hosted the rest of the produced Haskell code on the public Git repository accessible via this link: \url{https://github.com/MegaMaxim10/my-thesis-projection-algorithms}.


\mySectionStar{Haskell Type for Tags}{}{false}
Let's start by defining the tags for the node types (sequential or parallel). More clearly, in a given artifact, a node $A$ is tagged with \Verb|Seq| (resp. \Verb|Par|) when its sub-artifacts are executed in sequence (resp. potentially in parallel), i.e. the production used for its extension is a sequential (resp. parallel) one. A node with at most one sub-artifact is always tagged with \Verb|Seq|.
\begin{Verbatim}[frame=lines, fontsize=\small, numbers=left, numbersep=8pt]
data ProductionTag x = Seq x | Par x deriving (Eq, Show)
\end{Verbatim}
The \Verb|untagProduction| function below clears a given symbol of its tag (\Verb|Seq| or \Verb|Par|):
\begin{Verbatim}[frame=lines, fontsize=\small, numbers=left, numbersep=8pt]
untagProduction:: ProductionTag x -> x
untagProduction (Seq x) = x
untagProduction (Par x) = x
\end{Verbatim}

\mySubSectionStar{Definition of tags (\textit{closed}, \textit{locked}, \textit{unlocked} or \textit{upstair}) for symbols}{}{false}
In an artifact: a closed node is tagged \Verb|Closed|, an unlocked bud is tagged \Verb|Unlocked|, a locked bud is tagged \Verb|Locked| and an upstair bud is tagged \Verb|Upstair| (only found after expansion).
\begin{Verbatim}[frame=lines, fontsize=\small, numbers=left, numbersep=8pt]
data NodeTag x = Closed x | Locked x | Unlocked x | Upstair x deriving (Eq, Show)
\end{Verbatim}
The \Verb|untagNode| function below clears a given symbol of its tag (\Verb|Closed|, \Verb|Unlocked|, \Verb|Locked| or \Verb|Upstair|):
\begin{Verbatim}[frame=lines, fontsize=\small, numbers=left, numbersep=8pt]
untagNode:: NodeTag x -> x
untagNode (Closed x) = x
untagNode (Locked x) = x
untagNode (Unlocked x) = x
untagNode (Upstair x) = x
\end{Verbatim}

\mySubSectionStar{Definition of tags for symbol types (structuring or standard)}{}{false}
The symbols of a given artifact $t$ are either those of the grammatical model $\mathbb{G}$ denoting $t$, or (re)structuring symbols introduced to preserve some important properties of our model (mainly, the form of productions used in GMWf): in this case, the symbols of $\mathbb{G}$ are said to be standard and are tagged with \Verb|Standard| while the (re)structuring symbols are tagged with \Verb|Structural|.
\begin{Verbatim}[frame=lines, fontsize=\small, numbers=left, numbersep=8pt]
data SymbolTag x = Structural x | Standard x deriving (Eq, Show)
\end{Verbatim}
As the previous "untag" functions, the \Verb|untagSymbol| function below clears a given symbol of its tag (\Verb|Structural| or \Verb|Standard|):
\begin{Verbatim}[frame=lines, fontsize=\small, numbers=left, numbersep=8pt]
untagSymbol:: SymbolTag x -> x
untagSymbol (Structural x) = x
untagSymbol (Standard x) = x
\end{Verbatim}


\mySectionStar{Haskell Type for Artifacts}{}{false}
Recursively, we consider that an artifact is given by its root node (\Verb|nodeLabel|) and the list of its sub-artifacts (\Verb|sonsList|) tagged either by \Verb|Seq| (to indicate that they are executed in sequence) or by \Verb|Par| (to indicate that they are potentially executed in parallel). We do not consider empty artifacts. The corresponding Haskell type is as follows:
\begin{Verbatim}[frame=lines, fontsize=\small, numbers=left, numbersep=8pt]
data Artifact a = Node {
                       nodeLabel:: a, 
                       sonsList:: ProductionTag [Artifact a]
                  } deriving Eq
\end{Verbatim}
Here is an example of artifact encoded in this type. It corresponds to the target artifact $art_1$ in the figure \ref{chap3:fig:global-artefacts}:
\begin{Verbatim}[frame=lines, fontsize=\small, numbers=left, numbersep=8pt]
art1 = Node (Closed "Ag") (
           Seq [
             Node (Closed "A") (
               Seq [
                 Node (Closed "B") (Seq []), 
                 Node (Closed "D") (Seq [])
               ])
           ])
\end{Verbatim}


\mySectionStar{Haskell Type for GMWf}{}{false}
Let's start by presenting a type for productions: a production is given by its left hand side (\Verb|lhs|) consisting of one symbol and by its right hand side (\Verb|rhs|) consisting of several symbols.
\begin{Verbatim}[frame=lines, fontsize=\small, numbers=left, numbersep=8pt]
data Production symb = Prod {lhs:: symb, rhs:: [symb]} deriving Eq
\end{Verbatim}

Finally, a GMWf is given by the set of symbols and the set of productions constituting it. The productions are tagged either by \Verb|Seq| or by \Verb|Par|:
\begin{Verbatim}[frame=lines, fontsize=\small, numbers=left, numbersep=8pt]
data GMWf a = GMWf {
                   symbols:: [a], 
                   productions:: [ProductionTag (Production a)]
              } deriving (Eq, Show)
\end{Verbatim}


These are the main data types that we have defined and which are manipulated by the different projection functions that are available in our Git repository\footnote{Our Git repository: \url{https://github.com/MegaMaxim10/my-thesis-projection-algorithms}}. They are included with some test cases that one will be able to immediately experiment.


\myRestoreMarks


	\myChapter{List of Publications Issued from the Work Presented in this Thesis}{}
\label{appendice2:article-appendice}
\mySaveMarks
\mySectionStar{Journal Papers}{}{false}
\mySubSectionStar{Published}{}{false}
\begin{enumerate}
	\item Milliam Maxime Zekeng Ndadji, Maurice Tchoup{\'e} Tchendji, Cl{\'e}mentin Tayou Djamegni and Didier Parigot. "\textit{A new Domain-Specific Language and Methodology based on Scenarios, Grammars and Views, for Administrative Processes Modelling.}" ParadigmPlus, Volume 1, Number 3, 2020, 1-22.
	\item Maurice Tchoup{\'e} Tchendji and Milliam Maxime Zekeng Ndadji. "\textit{Tree Automata for Extracting Consensus from Partial Replicas of a Structured Document.}" Journal of Software Engineering and Applications 10.05 (2017): 432-456.
\end{enumerate}

\mySubSectionStar{Under Review}{}{false}
\begin{enumerate}
	\item Maurice Tchoup{\'e} Tchendji, Milliam Maxime Zekeng Ndadji and Didier Parigot. "\textit{A Grammatical Approach for Administrative Workflow Design and their Distributed Execution using Structured and Cooperatively Edited Mobile Artifacts.}" Software and Systems Modeling, Springer (\textbf{submitted}).
	\item Milliam Maxime Zekeng Ndadji, Maurice Tchoup{\'e} Tchendji, Cl{\'e}mentin Tayou Djamegni and Didier Parigot. "\textit{A Projection-Stable Grammatical Model for the Distributed Execution of Administrative Processes with Emphasis on Actors' Views.}" Journal of King Saud University - Computer and Information Sciences, Elsevier (\textbf{submitted}).
\end{enumerate}

\mySectionStar{Communications in Conferences}{}{false}
\mySubSectionStar{Published}{}{false}
\begin{enumerate}
	\item Milliam Maxime Zekeng Ndadji, Maurice Tchoup{\'e} Tchendji, Cl{\'e}mentin Tayou Djamegni and Didier Parigot. "\textit{A Grammatical Model for the Specification of Administrative Workflow using Scenario as Modelling Unit.}" H. Florez and S. Misra (eds) Applied Informatics. ICAI 2020. Communications in Computer and Information Science, vol 1277, Springer, Cham, 2020. pages 131-145.
	\item Milliam Maxime Zekeng Ndadji, Maurice Tchoup{\'e} Tchendji, Cl{\'e}mentin Tayou Djamegni and Didier Parigot. "\textit{A Language for the Specification of Administrative Workflow Processes with Emphasis on Actors' Views.}" Gervasi O. et al. (eds) Computational Science and Its Applications - ICCSA 2020. ICCSA 2020. Lecture Notes in Computer Science, vol 12254, Springer, Cham, 2020. pages 231-245.
	\item Milliam Maxime Zekeng Ndadji, Maurice Tchoup{\'e} Tchendji and Didier Parigot. "\textit{A Projection-Stable Grammatical Model to Specify Workflows for their P2P and Artifact-Centric Execution.}" CRI'2019 - Conf{\'e}rence de Recherche en Informatique. Dec 2019, Yaound{\'e}, Cameroon. (hal-02375958).
	\item Milliam Maxime Zekeng Ndadji and Maurice Tchoup{\'e} Tchendji. "\textit{A Software Architecture for Centralized Management of Structured Documents in a Cooperative Editing Workflow.}" Innovation and Interdisciplinary Solutions for Underserved Areas. Lecture Notes of the Institute for Computer Sciences, Social Informatics and Telecommunications Engineering (LNICST), Springer, Cham, 2018. pages 279-291.
	\item Maurice Tchoup{\'e} Tchendji and Milliam Maxime Zekeng Ndadji. "\textit{R{\'e}conciliation par consensus des mises {\`a} jour des r{\'e}pliques partielles d'un document structur{\'e}.}" CARI 2016 Proceedings, volume 1, 2016. pages 84-96.
\end{enumerate}


\myRestoreMarks

\begin{comment}
\includepdf[pages={1}, offset=72 -72]{Appendices/Articles/pplus-board.pdf}
\includepdf[pages=-, offset=72 -72]{Appendices/Articles/ICAI-2020-Extended.pdf}

\includepdf[pages={1}, offset=72 -72]{Appendices/Articles/JSEA_10_01_Content_2017011916111320.pdf}
\includepdf[pages=-, offset=72 -72]{Appendices/Articles/JSEA_2017052615402522.pdf}

%\includepdf[pages=-, offset=72 -72]{Appendices/Articles/mainApprocheP2PSOA-Elsevier.pdf}

%\includepdf[pages=-, offset=72 -72]{Appendices/Articles/CRI-2019-Extended.pdf}

\includepdf[pages={1,144-158}, offset=72 -72]{Appendices/Articles/ICAI-2020.pdf}

\includepdf[pages={1,269-283}, offset=72 -72]{Appendices/Articles/ICCSA-2020.pdf}

\includepdf[pages={4}, offset=72 -72]{Appendices/Articles/covers.pdf}
\includepdf[pages=-, offset=72 -72]{Appendices/Articles/mainprojectionmgwfarima.pdf}

\includepdf[pages={1,281-293}, offset=72 -72]{Appendices/Articles/InterSol2017-Book.pdf}

\includepdf[pages={2,97-109}, offset=72 -72]{Appendices/Articles/CARI2016.pdf}
\end{comment}




	%Ainsi de suite
}
\end{document}
