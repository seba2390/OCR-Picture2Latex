% Un template bilingue pour la production des mémoires et thèses dans le département de Mathématiques-Informatique.
% Ce template est conforme aux recommandations de l'école doctorale
%
% Ce fichier est conçu pour accueillir vos imports (\usepackage) et vos propres définitions d'environnements latex et/ou de style
% Consulter le fichier style.tex pour savoir ce qui a déja été importé
%
% @author Zekeng Ndadji Milliam Maxime

\usepackage{algorithm}
\usepackage{algpseudocode}
\usepackage{listings}% http://ctan.org/pkg/listings  Pour ecrire des math dans un verbatim \begin{lstlisting} ... \end{lstlisting}
\lstset{
  basicstyle=\ttfamily,
  mathescape
}

\renewcommand{\listalgorithmname}{List of Algorithms}
\floatname{algorithm}{Algorithm}
%\renewcommand{\algorithmicreturn}{\textbf{retourne}}
%\renewcommand{\algorithmicprocedure}{\textbf{procédure}}
%\newcommand{\Not}{\textbf{non}\ }
%\newcommand{\AlgAnd}{\textbf{et}\ }
%\newcommand{\Or}{\textbf{ou}\ }
\newcommand{\To}{\textbf{to}\ }
\let\oldComment=\Comment
\renewcommand{\Comment}[1]{\oldComment{{\scriptsize#1}}}
%\renewcommand{\algorithmicrequire}{\textbf{Entrée:}}
%\renewcommand{\algorithmicensure}{\textbf{Sortie:}}
%\renewcommand{\algorithmiccomment}[1]{\{#1\}}
%\renewcommand{\algorithmicend}{\textbf{fin}}
%\renewcommand{\algorithmicif}{\textbf{si}}
%\renewcommand{\algorithmicthen}{\textbf{alors}}
%\renewcommand{\algorithmicelse}{\textbf{sinon}}
%\renewcommand{\algorithmicfor}{\textbf{pour}}
%\renewcommand{\algorithmicforall}{\textbf{pour tout}}
%\renewcommand{\algorithmicdo}{\textbf{faire}}
%\renewcommand{\algorithmicwhile}{\textbf{tant que}}
\algdef{SE}[DOWHILE]{DoWhile}{EndDoWhile}{\algorithmicdo}[1]{\algorithmicwhile\ #1}%
\newcommand{\algorithmicelsif}{\algorithmicelse\ \algorithmicif}
\newcommand{\algorithmicendif}{\algorithmicend\ \algorithmicif}
\newcommand{\algorithmicendfor}{\algorithmicend\ \algorithmicfor}


%Définition de nouveaux environnements de type théorème
\newtheorem{theorem}{Theorem}
\newtheorem{definition}[theorem]{Definition}
\newtheorem{proposition}[theorem]{Proposition}
\newtheorem{lemma}[theorem]{Lemma}
\newtheorem{example}[theorem]{Example}
\newtheorem{remark}[theorem]{Remark}
\newtheorem{corollary}[theorem]{Corollary}
\newtheorem{problem}[theorem]{Problem}

%Environnements de preuve
\newenvironment{proof}[1][{\textbf{Proof}}]{
	\par
	\normalfont
	\topsep6\p@\@plus6\p@ \trivlist
	\item[\hskip\labelsep\itshape
	#1\@addpunct{.}]\ignorespaces
}{%
	\qed\endtrivlist
}
\newenvironment{preuve}[1][{\textbf{Preuve}}]{
	\par
	\normalfont
	\topsep6\p@\@plus6\p@ \trivlist
	\item[\hskip\labelsep\itshape
	#1\@addpunct{.}]\ignorespaces
}{%
	\qed\endtrivlist
}

\usepackage[framemethod=TikZ]{mdframed}
\mdfdefinestyle{MyFrame}{%
	linecolor=white,
	outerlinewidth=0pt,
	roundcorner=0pt,
	innertopmargin=4pt,
	innerbottommargin=4pt,
	innerrightmargin=4pt,
	innerleftmargin=4pt,
	leftmargin=4pt,
	rightmargin=4pt
	%backgroundcolor=gray!50!white
}

\usepackage{csquotes}

\renewcommand{\BBA}{and}






