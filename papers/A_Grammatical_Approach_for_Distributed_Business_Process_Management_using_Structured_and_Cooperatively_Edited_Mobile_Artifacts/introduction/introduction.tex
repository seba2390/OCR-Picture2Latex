%\myChapterStar{Titre}{Titre court}{Ajouter à la table des matières? (false|true|chapter|section|subsection|subsubsection -chapter par défaut-)}
\myChapterStar{General Introduction}{}{true}
\label{chap0:introduction}
\myMiniToc{section}{Contents}
% If no minitoc then
% \startcontents[chapters]

\mySectionStar{The Emergence of Business Process Management}{}{true}
\label{chap0:sec:bpm-emergence}
Business Process Management (BPM) has received considerable attention in recent years due to its potential for significantly increasing productivity and saving costs. It is defined by Wil M. P. Van Der Aalst \citeyearpar{van2013business} as \textit{"the discipline that combines knowledge from information technology and knowledge from management sciences and applies this to operational business processes"}. BPM aims to improve business processes by focusing on their automation, their analysis, their involvement in decision-making operations (management) and the organisation of work. Hence, BPM is often accompanied by software to manage, control and support business processes: these software systems are called \textit{Workflow Management Systems} (WfMS) \cite{workflow95, ima}.

The BPM discipline emerged in the 1980s in a more restrictive form known as \textit{Workflow Management} (WfM). Before WfM was developed, information systems were built from scratch; it means that, all components of such systems had to be programmed, including data storage and retrieval \cite{van1998application}. Software vendors soon realised that many information systems had similar data management requirements. This generic functionality was therefore outsourced to a data management system. Subsequently, the generic functionality related to user interaction (forms, buttons, graphics, etc.) was outsourced to user interfaces generators. The trend to outsource recurring functionalities to generic tools has continued in different areas. It is in this context that WfM has been introduced. Precisely, a WfMS automatically manages the process-related aspects \cite{workflow95, van2013business} of information systems. The aim of WfM in the design of information systems is to simplify as much as possible, the modelling and management of the business processes they automate: traditionally, this modelling and management are reduced to the specification of processes using simple graphical languages called \textit{workflow languages}.

Across the years, WfM has evolved into BPM. While WfM focused primarily on the automation of business processes, i.e. it was not fundamentally interested in other issues such as the analysis, verification and maintenance of their specifications, BPM made it its foundation \cite{van2016don}. With this evolution, many tools and techniques have been developed and have allowed the BPM field to mature. Today, its relevance is recognised by practitioners (users, managers, analysts, consultants and software developers) and academics. The availability of many BPM systems (WfMS) and a series of BPM-related conferences is proof of this \cite{van2013business}.

With the evolution of WfM and the development of new concepts related to the implementation of collaborative software systems, namely \textit{Peer-to-Peer (P2P) computing}, \textit{multiagent} paradigm and \textit{Service-Oriented Architecture} (SOA), the way of designing and implementing WfMS has also evolved. There has been a shift from centralised systems implemented according to a reference architectural model \cite{workflowModel}, to fully distributed systems offering decentralised workflow execution \cite{meilin1998workflow, junYan06, fakas04}. New paradigms of specification and management of business processes have been developed too. Among these, we find the \textit{document-centric} \cite{krishnan2001xdoc, marchetti2005xflow, badouelTchoupeCmcs}, the \textit{email-based} \cite{burkhart2012context, gazze2012workmail}, the \textit{database-based} \cite{actionWorkflow, miao2011realization}, the \textit{artifact-centric} \cite{nigam2003business, hull2009facilitating, lohmann2010artifact}, the \textit{data-centric} \cite{damaggio2013equivalence, badouel2015active} paradigms, etc. The \textit{artifact-centric} paradigm has been the subject of several studies over the last two decades \cite{abi2016towards, deutsch2014automatic, hull2009facilitating, lohmann2010artifact, assaf2017continuous, assaf2018generating, boaz2013bizartifact, lohmann2011artifact, estanol2012artifact}; it has been very successful because it has enabled the development of much more flexible workflow languages, that treat process data as first-class citizens, as opposed to the existing languages (BPMN - \textit{Business Process Model and Notation} \cite{BPMN} -, YAWL - \textit{Yet Another Workflow Language} \cite{van1998application, van2005yawl} -, etc.), that were only concerned with process tasks scheduling and assignment to actors. All these concepts' evolution and this involvement of many technologies in workflow systems' engineering, has made the BPM field, one of the most attractive for software vendors and software engineering researchers.


\mySectionStar{The Mitigated Use of Business Process Management}{}{true}
\label{chap0:sec:bpm-mitigated-use}
The BPM discipline has quickly established itself as an indispensable solution to the process automation needs of large firms, which often involve production lines \cite{van2016don}. Indeed, the workflows encountered in this context are generally highly structured and their tasks are almost entirely automated (executed by machines); this greatly "simplifies" their management by BPM. Another field of application in which BPM tends to naturally impose itself, is that of science. Indeed, the management (storage, distribution, computation, analysis, etc.) of generally very voluminous scientific data, involves several human and material resources that are often distributed across organisations \cite{bell2009beyond}. With the help of cloud computing, BPM in this context, serves to organise these resources for a more efficient management of scientific data \cite{juveGideon, ludascher2006scientific}. In these two application contexts, the systems' complexity due to the large amount of data to be managed and to the time-consuming and intensive computations requiring computer support, as well as the multiplicity and the distributivity of the involved resources, have somehow "imposed" the use of BPM.

Although in its current state BPM could be applied very successfully in the context of organisations with so-called administrative business processes (i.e. processes whose tasks are often semi-automated and therefore require the expertise of human agents - they are then more complex to automate using generic frameworks) \cite{boukhedouma2015adaptation}, there is less enthusiasm for it \cite{dumas2015models, van2016don}; information systems using classical database management concepts and tailored to the use cases involved, are often preferred. In some cases, such information systems are inspired by some BPM concepts or they embed workflow engines to make limited use of them: this is the case of Enterprise Resource Planning (ERP) systems such as SAP S/4HANA\footnote{Official website of SAP: \url{https://www.sap.com/}, visited the 19/03/2020.} and Oracle Fusion Applications (OFA)\footnote{Official website of Oracle ERP: \url{https://www.oracle.com/applications/erp/}, visited the 19/03/2020.} \cite{van2016don, van2013business}. Therefore, the application of "pure" BPM is still limited to specific industries such as banking and insurance. This mitigated use of BPM for the automation of quite common administrative processes can be explained by the following factors:
\begin{itemize}
	\item \textit{Building process management systems is considerably more "tricky" than building information systems using classical database management}. In database-based information systems, a specific number of processes whose behaviours are known in advance are designed and automated, whereas in process management, systems are designed and implemented to offer a generic management of an arbitrary set of processes with behavioural similarities. To quote Wil M. P. Van Der Aalst \citeyearpar{van2013business}, "\textit{BPM is multifaceted, complex, and difficult to demarcate}"; it is therefore not accessible to everyone and requires from its practitionners : great capacities of reasoning, logical abstraction, generalisation and architectural design \cite{workflow95}.
	\item \textit{Existing BPM solutions are too abstract and generic}. As a result, compared to traditional information systems, they are less suitable for certain applications such as the exclusive management of administrative processes, that have a variety of specifications depending on organisations \cite{borger2012approaches, zur2013much, van2013business}.
	\item \textit{There is no real consensus on BPM implementation techniques and tools}. BPM is composed by a multitude of paradigms and tools. Even if an effort of standardisation has been made in recent years, WfMS vendors seem to prefer the development of multiple proprietary solutions \cite{van2013business}; in our opinion, this has the effect of reducing BPM credibility.
\end{itemize} 

Because they offer enormous benefits in terms of time saving, implementation cost and system complexity management, we believe that there is a need to better adapt workflow solutions for administrative processes' management, which seem to be the most frequently encountered \cite{mcCready, van1998application, dumas2005process}. Naturally, we are not the first to take an interest in this issue; the authors of \cite{dumas2015models} and \cite{van2016don} have established that the potential obstacle to the popularisation of workflow solutions, is the imbalance of research work on the different BPM aspects. Indeed, they believe that research has focused too much on BPM specific artifacts (such as process models) rather than on improving organisations' business processes: this does not meet the real use and needs of BPM practitioners. They therefore propose to address now the issue of business process improvement, in order to give even more reasons to organisations regarding the choice of BPM for their processes management. This perspective is shared by a plurality of researchers; hence their growing interest in the new field of \textit{process mining}\footnote{Process mining is one of the leading techniques conducting workflows' event data automatic analysis, for possible improvement of its corresponding workflow model \cite{van2013business}.} \cite{van2011process}. 

Although the idea of improving business processes by analysing the data produced during their execution is interesting, we believe that it does not answer the question that we are struggling with: \textit{how can we get information system users (organisations and software vendors) to systematically opt for BPM technology to automate their processes} ? Indeed, data analysis is posterior to the users' choice of a technology to produce these data. Moreover, the question of improving business processes seems really interesting for use cases where these processes are very complex; it is not always the case for administrative processes. 




\mySectionStar{Our Global Vision}{}{true}
\label{chap0:sec:our-global-vision}
Because the benefits of BPM are now widely recognised and unanimously accepted, we believe that it would be more interesting to tackle the problem of its popularisation for administrative business processes automation, by improving its technology as it stands at present. Therefore, it would be a matter of:
\begin{enumerate}
	\item Making more accessible, the automation of administrative business processes using BPM. We believe that if the implementation of workflow systems for administrative business processes is simplified, then more and more software vendors will choose it to produce organisations' information systems.
	\item Adapt BPM technology so that it is less abstract and evasive, so that it is closer to that of classical information systems, and so that it can respond to specific problems of organisations.
\end{enumerate}

We believe that one way to achieve these goals, is to apply the \textit{domain-specific software engineering} \cite{bryant2010domain} to BPM and thus, to focus on a kind of \textit{Domain-Specific BPM}: that is the application of BPM techniques in specific domains of activity, thus respecting the constraints imposed by them, and offering a framework which best fits the needs of practitioners in these domains. In this thesis therefore, we must use BPM exclusively for the automation of administrative business processes: it is not a matter of reinventing the wheel (even if some new concepts are added) but rather of reproducing and adapting in the more constrained context of administrative business processes management, what is already being done in BPM, while introducing new concepts and making judicious choices to achieve the desired goals.

As far as we know, there are no other scientific studies that have focused exclusively on the automation of administrative processes. On the other hand, there is a growing effort among workflow solution providers to popularise administrative BPM. Precisely, they increasingly offer to organisations, cloud-based and flexible process management solutions such as Metatask\footnote{Official website of Metatask: \url{https://metatask.io/}, visited the 29/03/2020.}, Samepage\footnote{Official website of Samepage: \url{https://www.samepage.io/}, visited the 29/03/2020.}, Digital Business Transformation Suite\footnote{Official website of Digital Business Transformation Suite: \url{https://www.interfacing.com/}, visited the 29/03/2020.}, Favro\footnote{Official website of Favro: \url{https://learn.favro.com/}, visited the 29/03/2020.}, etc. Let us mention that there are studies focused on adapting BPM technologies to scientific data management exclusively (Domain-specific BPM): these gave birth to the field of scientific workflows \cite{bell2009beyond, juveGideon, ludascher2006scientific}.



\mySectionStar{The Challenge Addressed in this Thesis}{}{true}
\label{chap0:sec:thesis-challenge}
In this thesis, we are interested in the automation of administrative business processes (exclusively) using BPM technology. The idea is to use the most up-to-date paradigms already developed in the implementation of workflow systems, to produce a new approach for the specification and execution of administrative business processes. Nowadays, the most common paradigms and concepts used in the implementation of workflow management approaches include:
\begin{itemize}
	\item \textit{Artifact-centric paradigm}: it is the most successful BPM paradigm of the last two decades; it simplifies the specification of a business process, to the instantiation of a data structure called artifact, whose state evolves into a business goal state when executing process instances.
	\item \textit{Cooperative edition of documents}: in the artifact-centric paradigm, an artifact can be seen as a structured document containing the execution state of a process instance at a given time. Because the evolution of the state contained in it is the consequence of actions carried out by the different actors involved in the considered instance's execution, this execution can be assimilated to the cooperative edition of structured documents.
	\item \textit{Peer to Peer computing}: workflow systems increasingly rely on P2P architectures; as opposed to centralised systems, they facilitate scalability and fault tolerance \cite{fakas04, junYan06, theseMounir}.
	\item \textit{Multiagent paradigm}: the multiagent paradigm was developed to facilitate the creation of distributed systems, especially those based on P2P architectures. It is also increasingly used in workflow systems, where process tasks are now executed by agents that communicate through messages; these agents have a high degree of autonomy to allow better decentralisation of process management.
	\item \textit{Service-Oriented Architecture}: in decentralised workflow management systems based on P2P architectures, the concept of SOA is generally used to define communication protocols between agents and to increase their autonomy by implementing a loose coupling between them.
\end{itemize}

We combine these concepts to produce a workflow solution that best suits the automation of administrative business processes. Because this project is much too voluminous to be addressed in the context of a single PhD thesis, we will focus herein, only on the fundamental aspects: administrative process modelling and their distributed execution. One could summarise the main objective of this thesis by saying that it focuses on:
\begin{displayquote}
\textit{The proposal of a new artifact-centric framework, facilitating the modelling of administrative business processes and the completely decentralised execution of the resulting workflows; this completely decentralised execution being provided by a P2P system conceived as a set of agents communicating asynchronously by service invocation so that, the execution of a given workflow instance is technically assimilated to the cooperative edition of (mobile) structured documents called artifacts.}
\end{displayquote}
This justifies the title of this thesis. We should however admit that, a title like "\textit{\textbf{yet another approach to facilitate administrative workflows design and distributed execution using structured and cooperatively edited mobile artifacts}}" would certainly have better reflected the work done\footnote{Although this is the title that best suits our work, academic constraints have "forced" us to keep the one currently in use (A Grammatical Approach to Peer-to-Peer Cooperative Editing on a Service-Oriented Architecture). Indeed, it is with this last one that we applied for a thesis in our doctoral school; at the time when we were certain of the judicious title for our work, the legal texts governing the PhD thesis in our institution, no longer allowed us to make changes to the title of our thesis.}.



\mySectionStar{A Synoptic View of our Methodology and Engineering}{}{true}
\label{chap0:sec:engineering-overview}
\begin{figure}[ht!]
	\noindent
	\makebox[\textwidth]{\includegraphics[scale=0.16]{./introduction/images/methodology.png}}
	\caption{A synoptic view of our methodology.}
	\label{chap0:fig:overview-methodology}
\end{figure}

Figure \ref{chap0:fig:overview-methodology} summarises the methodology and chronology we used to complete this thesis work. We started by studying a plethora of concepts including, the key concepts presented in chapter \ref{chap1:artifact-centric-bpm}. Then, we reinforced our knowledge on some other concepts, especially those related to the implementation of distributed systems (see fig. \ref{chap0:fig:overview-methodology}(1)). Having a better knowledge on BPM, especially on the artifact-centric paradigm, we undertook the design of a new artifacts' model; the search for such a model led us to an extensive study of structured cooperative editing workflows. In particular, we studied the structured documents asynchronous cooperative editing model proposed by Badouel and Tchoup\'e \citeyearpar{badouelTchoupeCmcs} so well that, we extended it with three new contributions (see chapter \ref{chap2:structured-editing-artifact-type}). Based on the document model manipulated by Badouel and Tchoup\'e, we derived an artifact model (an attributed grammar) that can be used to specify administrative business processes. We then designed a P2P system based on communicating agents using a service-oriented model, capable of executing in a completely decentralised manner, administrative processes specified using the proposed artifact model. To demonstrate the concreteness of our artifact-centric model, we finally implemented a fully functional prototype system allowing to experiment the proposed models on various administrative processes (see chapter \ref{chap3:choreography-workflow-design-execution}). As a result, we count at least six major contributions in this thesis:
\begin{enumerate}
	\item An extension of the merge algorithm proposed by Badouel and Tchoup\'e for the cooperative edition of a structured document, so that it can be applied in the more general case where edition conflicts might appear. It should be noted that this contribution was initiated in our Master's work before being fully matured during our first year of thesis. Some of the elements presented here are therefore part of our Master's dissertation;
	
	\item A generic system architecture that can be used to produce workflow systems for the cooperative editing of structured documents based on the Badouel and Tchoup\'e extended model;
	
	\item A workflow system prototype referred to as TinyCE v2 (a Tiny Cooperative Editor version 2), coded in Java and Haskell following the proposed system architecture and a novel cross-fertilisation protocol. It allowed us to test all the proposed algorithms related to the merge of documents' replicas;
	
	\item Another tree-based model of "business artifact" for administrative processes modelling, which makes it possible to better assimilate them to structured documents edited cooperatively;
	
	\item A choreography-oriented artifact-centric workflow execution model in which geographically dispersed agents execute the same and unique update (editing of artifact upon receipt) and diffusion (dissemination of updates) protocol;
	
	\item A prototype of a distributed system referred to as P2PTinyWfMS (a Peer-to-Peer Tiny Workflow Management System), allowing to fully experiment the artifact-centric approach investigated in this thesis.
\end{enumerate}

More concretely, in the asynchronous cooperative editing workflow as perceived by Badouel and Tchoup\'e, each of the co-authors receives in the different phases of the editing process, a copy of the edited document to insert his contribution. 
Since the collectively edited document is structured, it may in some cases, be preferable for reasons of confidentiality for example, that a co-author has access only to certain information; meaning that, he only has access to certain parts of the document, belonging to certain given types (\textit{sorts}\footnote{A \textit{sort} is a datum used to define the structuring rules (syntax) in a document model. Example: a \textit{non-terminal symbol} in a context free grammar, an \textit{ELEMENT} in a \textit{Document Type Definition} (DTD).}) of the document model. Thus, the replica $t_{\mathcal{V}_i}$ edited by co-author $c_i$ from the site $i$ may be only a \textit{partial replica} of the (global\footnote{We designate by \textit{global document} or simply \textit{document} when there is no ambiguity, the document including all parts.}) document $t$, obtained via a \textit{projection operation}, which conveniently eliminates from the global document $t$, parts which are not accessible to the co-author in question. Badouel and Tchoup\'e call \textit{view} of a co-author, the set of \textit{sorts} that he can access. 

When asynchronous local editions are done on partial replicas, it can be assumed that each co-author has on his site a local document model guiding him in his edition. This local model can help to ensure that for any update $t_{\mathcal{V}_i}^{maj}$ of a partial replica $t_{\mathcal{V}_i}$ (conform to the considered model), there is at least one document $t_c$ conform to the global model so that $t_{\mathcal{V}_i}^{maj}$ is a partial replica of $t_c$: for this purpose, the local model should be coherent towards the global one\footnote{Intuitively, a local model of document is \textit{coherent} towards a global model if any partial document $t_{\mathcal{V}_i}$ that is conform to it, is the partial replica of at least one (global) document $t_c$ conform to the global model.}. Thus, because of the edition's asynchronism, the only inconsistencies that we can have when the synchronisation time arrives are those from the concurrent edition of the same node\footnote{Manipulated documents are structured, they can be intentionally represented by a tree. Intuitively, a node is an identifiable part in the document (a section, a subsection, an image, a table, \ldots): it is the instance of a sort.}  (in the point of view of the global document) by several co-authors: the partial replicas concerned are said to be in conflict. 
In its first contribution, this thesis proposes an approach of detection and resolution of such conflicts by \textit{consensus},
using a tree automaton said of consensus, to represent all the documents that are the consensus of competing editions realised on different partial replicas.

A structured document $t$ is intentionally represented by a tree that possibly contains buds\footnote{A \textit{bud} is a leaf node of a tree indicating that an edition must be done at that level in the tree. Edit a bud consists to replace it by a subtree using the productions of the grammar of the document.}. Intuitively, synchronise or merge consensually the updates $t_{\mathcal{V}_1}^{maj}, \ldots, t_{\mathcal{V}_n}^{maj}$ of $n$ partial replicas of a document $t$, consists in finding a document $t_{c}$ conform to the global model, integrating all the nodes of $t_{\mathcal{V}_i}^{maj}$ not in conflict and in which, all the conflicting nodes are replaced by buds. Consensus documents are therefore the largest prefixes without conflicts in merged documents. 
Technically, the process for obtaining the documents forming part of the consensus is: 
(1) For each update $t_{\mathcal{V}_i}^{maj}$ of a partial replica $t_{\mathcal{V}_i}$, we associate a tree automaton with \textit{exit states} $\mathcal{A}^{(i)}$ recognising the trees (conform to the global model) for which $t_{\mathcal{V}_i}^{maj}$ is a  projection. 
(2) The consensual automaton $\mathcal{A}_{(sc)}$ generating the consensus documents is obtained by performing a \textit{synchronous product} of the automata $\mathcal{A}^{(i)}$ with a commutative and associative operator noted $\otimes$  that we define. It is such that: $\mathcal{A}_{(sc)}=\otimes\mathcal{A}^{(i)}$. 
(3) It only remains to generate the set of trees (or those most representative) accepted by the automaton $\mathcal{A}_{(sc)}$, to obtain the consensus documents.

The concept of structured document as perceived by Badouel and Tchoup\'e can be adapted to correspond to that of artifact in the sense of artifact-centric workflow systems. In this sense, a structured document can be seen as an artifact (annotated tree) that can be exchanged between the different agents (actors) involved in the execution of a given particular business process case; during its life, it is edited appropriately to make the system converge towards the achievement of one of the considered process's business goals. Its buds materialise the tasks to be executed or which are being executed and, an attributed grammar called the \textit{Grammatical Model of Workflow} (GMWf) is used as \textit{artifact type}. The sorts of a given GMWf represent the process tasks and each of its productions represents a scheduling of a subset of these tasks.
When a task is executed on a given site, the corresponding bud in the artifact is closed accordingly; then, one of the GMWf's production having the considered task as left hand side is choosen by the local stakeholder to expand the bud into a subtree highlighting in the form of new buds, the new tasks to be executed. To enrich the notion of access to different parts of artifacts, we add to GMWf, organisational information called \textit{accreditations} (similar to views) that offer a simple mechanism for modelling the generally different perceptions that actors have on processes and their data. The couple (GMWf, accreditations) is then the proposed model of "business artifact".

The execution of an administrative process's instance modelled using the couple (GMWf, accreditations), is a choreography in which the agents are reactive autonomous software components, communicating in peer-to-peer mode and driven by human agents (actors) in charge of executing tasks. An agent's reaction to the reception of a message (an artifact) consists in the execution of a five-step protocol clearly described in this thesis. 
This protocol allows it to: (1) \textit{merge} the received artifact with the one it hosts locally in order to consider all updates, (2) \textit{project} the artifact resulting from the merger in order to hide the parts to which the local actor may not have access and highlight the tasks to be locally executed, (3) make the local actor \textit{execute} the revealed tasks and thus edit the potentially partial replica of the artifact obtained after the projection, (4) integrate the new updates into the artifact through an operation called \textit{expansion-pruning} and finally, (5) \textit{diffuse} the updated artifact to other sites for further execution of the process if necessary. 
The agents' operational capabilities allow that, for the execution of a given process, an artifact created by one of them (initially reduced to an open node), moves from site to site to indicate tasks that are ready to be executed at the appropriate time and to provide necessary data (created by other agents) for that execution; the mobile artifact, cooperatively edited by agents, thus "grows" as it transits through the distributed system so formed.



\mySectionStar{The Organisation of this Manuscript}{}{true}
\label{chap0:sec:manuscript-organisation}
The rest of this thesis manuscript consists of four chapters and two appendices organised as follows:

~

\noindent\textbf{Chapter \ref{chap1:artifact-centric-bpm} - A State of the Art in Business Process Management: the Artifact-Centric Modelling}: we present some basic concepts related to the field of BPM, P2P, SOA, as well as the multiagent and artifact-centric paradigms. We also do a survey of some P2P and artifact-centric workflow management systems.


~

\noindent\textbf{Chapter \ref{chap2:structured-editing-artifact-type} - A Workflow for Structured Documents' Cooperative Editing : Key Principles and Algorithms}: after a brief presentation of Badouel and Tchoup\'e's asynchronous cooperative edition model, we present our algorithm for reconciling partial replicas of a structured document as well as a generic architecture of administrative workflow management systems.


~

\noindent\textbf{Chapter \ref{chap3:choreography-workflow-design-execution} - A Choreography-like Workflow Design and Distributed Execution Framework Based on Structured Mobile Artifacts Cooperative Editing}: we propose a new choreography-like approach to address administrative workflow design and their distributed execution. We also present a fully functional prototype system built according to the proposed approach.


~

\noindent\textbf{General Conclusion}: we summarise our work and present some possible venues for its further development.


~

\noindent\textbf{Appendix \ref{appendice1:algorithms-implementations} - Implementations of Some Important Algorithms Presented in this Thesis}: we present an implementation of all this thesis' key algorithms in the Haskell language.


~

\noindent\textbf{Appendix \ref{appendice2:article-appendice} - List of Scientific Communications Issued from the Work Presented in this Thesis}: we list the various scientific papers we produced during this thesis.



\myCleanStarChapterEnd

