\mySection{Modelling Artifacts}{}
\label{chap3:sec:modelling-artifacts}

\mySubSection{Artifacts' Structure}{}
\label{chap3:sec:artifacts-structure}
Let's consider an administrative process $\mathcal{P}_{op}$ to be automated. The set $\left\{ \mathcal{S}_{op}^1,\ldots,\mathcal{S}_{op}^k \right\}$ of $\mathcal{P}_{op}$ execution scenarios is known in advance and so, $\mathcal{P}_{op}$ can be specified as any oriented graph with tools like BPMN or as a petri net with tools like YAWL. 
Moreover, each execution scenario of $\mathcal{P}_{op}$ can be modelled using an annotated tree $t_i$. Indeed, starting from the fact that a given scenario $\mathcal{S}_{op}^i$ consists of a set $\mathbb{T}_n = \{X_1, \ldots, X_n\}$ of $n$ (non-recursive) tasks to be executed in a specific order (in parallel or in sequence), one can represent $\mathcal{S}_{op}^i$ as a tree $t_i$ in which each node (labelled $X_i$) potentially corresponds to a task $X_i$ of $\mathcal{S}_{op}^i$, and each hierarchical decomposition (a node and its sons) corresponds to a scheduling: the task associated with the parent node must be executed before those associated with the son nodes; the latter must be executed according to an order - parallel or sequential - that can be specified by particular annotations. Indeed, it is enough to have two annotations "$\fatsemi$" (is sequential to) and "$\parallel$" (is parallel to) to be applied to each hierarchical decomposition. The annotation "$\fatsemi$" (resp. "$\parallel$") reflects the fact that the tasks associated with the son nodes of the decomposition must (resp. can) be executed in sequence (resp. in parallel).

Considering the running example (the peer-review process), the two scenarios that make it up can be modelled using the two annotated trees in figure \ref{chap3:fig:global-artefacts}. In particular, we can see that the tree $art_1$ shows how the task "Receipt and pre-validation of a submitted paper" assigned to the editor in chief ($EC$), and associated with the symbol $A$ (see table \ref{tableau:tachesExecutant}, page \pageref{tableau:tachesExecutant}), must be executed before tasks associated with the symbols $B$ and $D$, that are to be executed in sequence. This annotated tree represents the scenario where the paper received by the editor in chief, is immediately rejected.
\begin{figure}[ht!]
	\noindent
	\makebox[\textwidth]{\includegraphics[scale=0.6]{./Chap3/images/artefactsGlobaux.png}}
	\caption{Target artifacts of a peer-review process.}
	\label{chap3:fig:global-artefacts}
\end{figure}


\mySubSection{Target Artifacts and Grammatical Model of Workflow}{}
\label{chap3:sec:target-artifacts-and-gmwf}
In this chapter, we use the expression \textit{target artifact} to designate the annotated tree $t_i$ modelling a given scenario $\mathcal{S}_{op}^i$ of a given administrative process $\mathcal{P}_{op}$. From the set of target artifacts of a given process, it is possible to derive an abstract grammar\footnote{It is enough to consider the set of target artifacts as a regular tree language: there is therefore a (abstract) grammar to generate them.} that can be enriched to serve as a \textit{artifact type} as defined in \cite{hull2009facilitating}: it is this grammar that we designate by the expression \textit{Grammatical Model of Workflow (GMWf)}.

Let's consider the set $\left\{t_1,\ldots,t_k\right\}$ of target artifacts modelling the $k$ execution scenarios of a given process $\mathcal{P}_{op}$ of $n$ tasks ($\mathbb{T}_n = \{X_1, \ldots, X_n\}$). Each $t_i$ is a derivation tree for an abstract grammar (a GMWf) $\mathbb{G}=\left(\mathcal{S},\mathcal{P},\mathcal{A}\right)$ whose set of symbols is $\mathcal{S}=\mathbb{T}_n$ (all process tasks) and each production $p \in \mathcal{P}$ reflects a hierarchical decomposition contained in at least one of the target artifacts. Each production is therefore exclusively of one of the following two forms: $p: X_0 \rightarrow X_1 \fatsemi \ldots \fatsemi X_n$ or  $p: X_0 \rightarrow X_1 \parallel \ldots \parallel X_n$. The first form $p: X_0 \rightarrow X_1 \fatsemi \ldots \fatsemi X_n$ (resp. the second form $p: X_0 \rightarrow X_1 \parallel \ldots \parallel X_n$) means that task $X_0$ must be executed before tasks $\left\{X_1,\ldots,X_n\right\}$, and these must be (resp. these can be) executed in sequence (resp. in parallel). A GMWf can therefore be formally defined as follows:
\begin{definition}
	\label{defGMWf1}
	A \textbf{Grammatical Model of Workflow} (GMWf) is defined by $\mathbb{G}=\left(\mathcal{S},\mathcal{P},\mathcal{A}\right)$
	where:
	\begin{itemize}
	\item $\mathcal{S}$ is a finite set of \textbf{grammatical symbols} or \textbf{sorts} corresponding to various \textbf{tasks} to be executed in the studied business process; 
	\item $\mathcal{A}\subseteq\mathcal{S}$ is a finite set of particular symbols called \textbf{axioms}, representing tasks that can start an execution scenario (roots of target artifacts), and 
	\item $\mathcal{P}\subseteq\mathcal{S}\times\mathcal{S}^{*}$ is a finite set of \textbf{productions} decorated by the annotations "$\fatsemi$" (is sequential to) and "$\parallel$" (is parallel to): they are \textbf{precedence rules}. 
	A production $P=\left(X_{P(0)},X_{P(1)},\cdots, X_{P(|P|)}\right)$ is either of the form $P: X_0 \rightarrow X_1 \fatsemi \ldots \fatsemi X_{|P|}$, or of the form $P: X_0 \rightarrow X_1 \parallel \ldots \parallel X_{|P|}$ and $\left|P\right|$ 
	designates the length of $P$'s right-hand side.
	A production with the symbol $X$ as left-hand side is called a \textit{X-production}.
	\end{itemize}
\end{definition}

Let's illustrate the notion of GMWf by considering the one generated from an analysis of the target artifacts obtained in the case of the peer-review process (see fig. \ref{chap3:fig:global-artefacts}). The derived GMWf is  $\mathbb{G}=\left(\mathcal{S},\mathcal{P},\mathcal{A}\right)$ in which, the set $\mathcal{S}$ of grammatical symbols is
$\mathcal{S}=\{A, B, C, D, E, F, G1, G2, H1, H2, I1, I2\}$ (see table \ref{tableau:tachesExecutant});
the only initial task (axiom) is $A$ (then $\mathcal{A}=\{A\}$) and the set $\mathcal{P}$ of productions is:
\[ 
\begin{array}{l|l|l|l}
P_{1}:\; A\rightarrow B\fatsemi D & \; P_{2}:\; A\rightarrow C\fatsemi D\; & \; P_{3}:\; C\rightarrow E\fatsemi F\; & \; P_{4}:\; E\rightarrow G1\parallel G2    \\
P_{5}:\; G1\rightarrow H1 \fatsemi I1 & \; P_{6}:\; G2\rightarrow H2 \fatsemi I2\; & \; P_{7}:\; B\rightarrow \varepsilon\; & \; P_{8}:\; D\rightarrow \varepsilon  \\
P_{9}:\; F\rightarrow \varepsilon & \; P_{10}:\; H1\rightarrow \varepsilon & \; P_{11}:\; I1\rightarrow \varepsilon\; & \; P_{12}:\; H2\rightarrow \varepsilon  \\
P_{13}:\; I2\rightarrow \varepsilon &  &  &   \\
\end{array}
\]

For some administrative business processes, there may be special cases where it is not possible to strictly schedule the tasks of a scenario using the two (only) forms of productions selected for GMWf. For example, this is the case for the scenario of a four-task process with tasks $A, B, C$ and $D$, where the task $A$ precedes all others, the tasks $B$ and $C$ can be executed in parallel and precede $D$. 
In these cases, the introduction of a certain number of new symbols known as \textit{(re)structuring symbols} (not associated with tasks) can make it possible to produce a correct scheduling that respects the form imposed on productions. For the previous example, the introduction of a new symbol $S$ allows us to obtain the following productions: $p_1: A \rightarrow S \fatsemi D$, $p_2: S \rightarrow B \parallel C$, $p_3:B \rightarrow \epsilon$, $p_4:C \rightarrow \epsilon$ and $p_5:D \rightarrow \epsilon$, which properly model the required scheduling. 
To deal with such cases, the previously given GMWf definition (definition \ref{defGMWf1}) is slightly adapted by integrating the (re)structuring symbols; the resulting definition is as follows:
\begin{definition} 
	\label{defGMWf2}
	A \textbf{Grammatical Model of Workflow} (GMWf) is defined by $\mathbb{G}=\left(\mathcal{S},\mathcal{P},\mathcal{A}\right)$
	wherein, 
	$\mathcal{P}$ and $\mathcal{A}$ refer to the same purpose as in definition \ref{defGMWf1}, 
	$\mathcal{S}=\mathcal{T} \cup \mathcal{T}_{Struc}$ 
	is a finite set of \textbf{grammatical symbols} or \textbf{sorts} in which, those of $\mathcal{T}$ correspond to \textbf{tasks} of the studied business process, while those of $\mathcal{T}_{Struc}$ are (re)structuring symbols.
\end{definition}

\mySubSection{Artifact Type and Artifact Edition}{}
\label{chap3:sec:gmwf-as-artifact-type}
As formalised in definition \ref{defGMWf2}, a GMWf perfectly models the tasks and control flow of administrative processes (lifecycle model). To remain faithful to the artifact-centric philosophy, the GMWf definition must be adjusted to be able to use it as an artifact type. In particular, it is necessary to equip it with tools allowing to represent the information model (the data) of processes as well as the dynamic (evolutionary) character of artifacts.

\mySubSubSection{Modelling the Information Model of Processes with GMWf}{}
\label{chap3:sec:gmwf-information-model}
The structure of the consumed and produced data by business processes differs from one process to another. It is therefore not easy to model them using a general type, although several techniques to do so have emerged in recent years \cite{badouel14}. For the work presented in this chapter, tackling the data structure of automated processes has no proven interest because, it does not bring any added value to the proposed model: a representation of these data using a set of variables is largely sufficient.

To represent the potential consumed and produced data by the tasks of a process modelled using GMWf, we use the notion of \textit{attribute} embedded in the nodes associated with tasks. To take them into account, we adjust for the last time, the definition of GMWf. We thus attach to each symbol, an attribute named $status$, allowing to store all the data of the associated task; its precise type is left to the discretion of the process designer. However, for the purposes of this work, we will consider it a string. The new definition of GMWf is thus the following one:
\begin{definition} 
	\label{defGMWf3}
	A \textbf{Grammatical Model of Workflow} (GMWf) is defined by $\mathbb{G}=\left(\mathcal{S},\mathcal{P},\mathcal{A}\right)$
	wherein, 
	$\mathcal{S}$, $\mathcal{P}$ and $\mathcal{A}$ refer to the same purpose as in definition \ref{defGMWf2}.
	Each grammatical symbol $X\in\mathcal{S}$ is associated with an attribute named \textbf{\textit{status}} of type string, that can be updated when task $X$ is executed; $\textbf{X.status}$ provides access (read and write) to its content.
\end{definition}

A GMWf is therefore ultimately an attributed grammar whose instances represent the different execution scenarios of the underlying business process. In artifact-centric models, the artifact used as a communication medium between agents executing the tasks, must represent at each moment, the execution state of the underlying process. As defined up to now, GMWf models do not satisfy this second concern: they cannot therefore be used as artifact types. We will now equip them with tools to allow them to endow their instances (the artifacts) with the ability to report about the execution state of the process they represent.

\mySubSubSection{Artifact Type}{}
\label{chap3:sec:artifact-type}
For each task, it is important to know whether or not it has already been executed; if not, it is also important to know whether or not it is ready to be executed. Recall also that, we model the execution of processes as the desynchronised cooperative editing of mobile artifacts (which are exchanged by agents). This implies that the artifact-centric model of this chapter considers that, an artifact is a structured document that is initially empty, and which is completed as it circulates between the agents. Contrary to the models in the literature, at each moment, the artifact thus contains only a (potentially empty) part of the lifecycle model of the process. This is why we have chosen not to represent it as a (tree) state machine but rather as an annotated tree that is incrementally built in accordance with an attributed grammar.

Concretely, an \textit{artifact} is an annotated tree that potentially contains buds (this is the equivalent of the notion of structured document being edited as presented in chapter \ref{chap2:structured-editing-artifact-type}). A \textit{bud} or \textit{open node} is a typed leaf node indicating in an artifact, a place where an edition is possible; i.e. a node associated with a task that has not yet been executed. A bud can be unlocked (\textit{unlocked bud}) or locked (\textit{locked bud}) depending on whether the task associated with it is ready to be executed\footnote{A task is ready to be executed if all the tasks that precede it according to the precedence constraint set have already been executed and the agent that currently holds the mobile artifact, have the necessary accreditation to trigger its execution.} or not. More formally, a \textit{bud of type $X \in \mathcal{S}$} is a leaf node labelled either by $X_{\overline{\omega}}$ or by $X_\omega$ depending on its state (\emph{locked} or \emph{unlocked}). An artifact containing no buds is said to be \textit{closed}. Such an artifact, symbolises the end of tasks execution with respect to the agent hosting the artifact. An example of an artifact related to the peer-review process and containing buds is shown in figure \ref{chap3:sec:artifact-with-buds}. In this one, we can see that the tasks associated with symbols $A$ and $C$ have already been executed. Task $E$ is ready to be executed while tasks $F$ and $D$ are not ready to be executed yet.
\begin{figure}[ht!]
	\noindent
	\makebox[\textwidth]{\includegraphics[scale=0.4]{./Chap3/images/documentBourgeons.png}}
	\caption{An intentional representation of an annotated artifact containing buds.}
	\label{chap3:sec:artifact-with-buds}
\end{figure}

From the thus given definition of (mobile) artifact, it is clear that an artifact is updated only at the level of its leaves and therefore, it only "grows" (positive editing). Knowing moreover that, the correct and complete execution of a given administrative process corresponds to the execution of one of its scenarios, we deduce that: for a process $\mathcal{P}_{op}$ whose GMWf is $\mathbb{G}=\left(\mathcal{S},\mathcal{P},\mathcal{A}\right)$, a given mobile artifact, is a prefix to one of its target artifacts. Thus, the type (model) of this artifact is a grammar $\mathbb{G}_{\Omega}=(\mathcal{S}\cup\mathcal{S}_{\omega},\mathcal{P}\cup\mathcal{S}_{\Omega},\mathcal{A}\cup\mathcal{A}_{\omega})$ obtained by extending $\mathbb{G}$ (for bud recognition and recognition of all possible prefixes of target artifacts) as follows:
\begin{enumerate}
\item For all sort $X$, add in the set $\mathcal{S}$ of sorts, two new sorts $X_{\overline{\omega}}$ and $X_{\omega}$;

\item For all new sort $X_{\omega}$ added to $\mathcal{S}$, add in the set $\mathcal {P}$ of productions two new $\varepsilon$-productions $X_{\Omega} : X_{\omega} \rightarrow \varepsilon$ and $X_{\overline{\Omega}} : X_{\overline{\omega}} \rightarrow \varepsilon$; 
we then have: $\mathcal{S}_{\omega}=\{X_{\overline{\omega}}, ~X_{\omega},~ X\in\mathcal {S}\}$, $ \mathcal{A}_{\omega}=\{X_{\overline{\omega}}, ~X_{\omega},~ X\in\mathcal{A}\} $ $ and $ $\mathcal{S}_{\Omega} = \{X_{\Omega} : X_{\omega} \rightarrow \varepsilon, ~X_{\overline{\Omega}} : X_{\overline{\omega}} \rightarrow \varepsilon,~ X_{\overline{\omega}}~and~X_{\omega} \in \mathcal{S}_{\omega}\}$.
\end{enumerate}

\mySubSubSection{Artifact Edition}{}
\label{chap3:sec:artifact-edition}
If we still consider a running process $\mathcal{P}_{op}$ whose GMWf is $\mathbb{G}=\left(\mathcal{S},\mathcal{P},\mathcal{A}\right)$, then, the \textit{editing} of an artifact $t$ circulating between agents consists of developing one or more of its buds into a subtree while updating their \textit{status} attributes. Concretely, for a bud ${X_\omega}$ of the said artifact one can:
\begin{enumerate}
\item Execute the task associated with $X$; 

\item Choose an $X$-production $P \in \mathcal{P}$ to be used for the development of ${X_\omega}$;

\item If $P$ is of the form $P:~X \rightarrow X_1 \parallel X_2 \parallel \ldots \parallel X_{|P|}$ (resp. $P:~X \rightarrow X_1 \fatsemi X_2 \fatsemi \ldots \fatsemi X_{|P|}$) then, create $|P|$ buds $X_{1\omega}, X_{2\omega}, \ldots, X_{|P|\omega}$ (resp. $X_{1\omega}, X_{2\overline{\omega}}, \ldots, X_{|P|\overline{\omega}}$) respectively of type $X_1, X_2, \ldots , X_{|P|}$, and replace in the artifact, the considered bud ${X_\omega}$ by the parallel (resp. sequential) subtree $X[X_{1\omega}, X_{2\omega}, \ldots, X_{|P|\omega}]$\footnote{The tree coded by $X[X_{1\omega}, X_{2\omega}, \ldots, X_{|P|\omega}]$ is the one whose root is labelled $X$ and has as sons, $|P|$ nodes labelled by $X_{1\omega}, X_{2\omega}, \ldots, X_{|P|\omega}$ respectively.} (resp. $X[X_{1\omega}, X_{2\overline{\omega}}, \ldots, X_{|P|\overline{\omega}}]$);
%4) if $P$ is of the form $P:~X \rightarrow X_1 \fatsemi X_2 \fatsemi \ldots \fatsemi X_{|P|}$ then create $|P|$ buds $X_{1\omega}, X_{2\overline{\omega}}, \ldots, X_{|P|\overline{\omega}}$ respectively of type $X_1, X_2, \ldots , X_{|P|}$ and replace in the artifact, the considered bud ${X_\omega}$ by the sequential subtree $X[X_{1\omega}, X_{2\overline{\omega}}, \ldots, X_{|P|\overline{\omega}}]$;

\item Update the execution status of the task associated with $X$: $X.status$ $=$ $"bla~bla~\ldots"$.
Updating $t$ results in an artifact $t^{maj}$ and we note $t \leq t^{maj}$.
\end{enumerate}

Although it is obvious, it seems important to clarify that the editing of an artifact is only a consequence of process tasks' execution by actors located on agents. We can therefore imagine this scenario for the peer-review process (see fig. \ref{chap3:fig:artifact-edition}): the associated editor who received a request from the editor in chief to peer-review a given article, has executed task $C$ (i.e. he has appraised the paper and formatted it to prepare the peer-review). Through a dedicated tool (a specialised editor), he has been invited to submit a report on the execution of the said task (via the filling of a form for example). The submission of that report, in which he may have provided a copy of the formatted paper as well as comments for referees, will cause (in background) the update of the artifact. The retrieved data will thus be stored in the $status$ attribute of task $C$ and the bud $C_\omega$ will be extended into a subtree as described above (the only production available for this purpose is $P_{3}: C\rightarrow E\fatsemi F$).
\begin{figure}[ht!]
	\noindent
	\makebox[\textwidth]{\includegraphics[scale=0.4]{./Chap3/images/artifactEdition.png}}
	\caption{An example of artifact edition: the bud $C_\omega$ is extended in a subtree.}
	\label{chap3:fig:artifact-edition}
\end{figure}
