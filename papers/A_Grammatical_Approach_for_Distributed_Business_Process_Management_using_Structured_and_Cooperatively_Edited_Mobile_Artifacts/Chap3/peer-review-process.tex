\mySection{Overview of the Artifact-Centric Model Presented in this Thesis}{}
\label{chap3:sec:model-overview}
In this section, a brief description of the artifact-centric model studied in this chapter is given. Furthermore, an overview of the distributed execution of the peer-review process using this model is presented.

\mySubSection{A Description of the Artifact-Centric Model Presented in this Thesis}{}
\label{chap3:sec:model-description}
We outlined in this chapter's introduction that the presented artifact-centric model is based on the asynchronous structured cooperative editing techniques proposed in the work of Badouel et al. As in these works, an artifact is represented by a tree containing "\textit{open nodes}" on some of its leaves, materialising the tasks to be executed or being executed and, an attributed grammar called the \textit{Grammatical Model of Workflow} (GMWf) is used as \textit{artifact type}. The symbols of a given GMWf represent the process tasks and each of its productions represents a scheduling of a subset of these tasks; intuitively, a production given by its left and right hand sides, specifies how the task on the left hand side precedes (must be executed before) those on the right hand side (see sec. \ref{chap3:sec:artifacts-structure}).
When a task is executed on a given site, the corresponding open node in the artifact is closed accordingly (it is said to be \textit{closed}) and the data produced during execution are filled in its attributes; then, one of the GMWf's production having the considered task as left hand side is chosen by the local actor to expand the open node into a subtree highlighting in the form of new open nodes, the new tasks to be executed: this is what editing an artifact consists of.
\begin{figure}[ht!]
	\noindent
	\makebox[\textwidth]{\includegraphics[scale=0.27]{./Chap3/images/protocole.png}}
	\caption{An overview of the artifact-centric BPM model presented in this chapter.}
	\label{chap3:fig:overview-protocol}
\end{figure}

We are especially interested on administrative processes and the approach we propose for their automation is declined in two steps: derivation of different models (target artifacts and their model, accreditations, etc.) from a textual description of the process and, implementation of a choreography between the agents communicating by asynchronous exchange of artifacts for its execution. More precisely, from the observation that one can analyse the textual description of an administrative business process to exhibit all the possible execution scenarios leading to its business goals, we propose to model each of these scenarios by an annotated tree in which, each node corresponds to a task of the process assigned to a given agent, and each hierarchical decomposition (a node and its sons) represents a scheduling of these tasks: these annotated trees are called \textit{target artifacts}. From these target artifacts, are derived a GMWf (\textit{artifact type}) which contains both the \textit{information model} (modelled by its attributes) and the \textit{lifecycle model} (thanks to the set of its productions) which are two essential notions of the artifact-centric modelling paradigm \cite{hull2013data}. Once the GMWf is obtained, we propose to add organisational information called \textit{accreditations} in this chapter; they aim, as in \cite{badouelTchoupeCmcs, theseTchoupe, tchoupeAtemkeng2, tchoupeZekeng2016, tchoupeZekeng2017, zekengTchoupe2018}, is to enrich the notion of access to different parts of artifacts, by offering a simple mechanism for modelling the generally different perceptions that actors have on processes and their data. With the couple (GMWf, accreditations), each autonomous agent is configured (see fig. \ref{chap3:fig:overview-protocol} (1)) and is ready to proceed to the decentralised execution of the studied process.

The actual execution is a choreography in which the agents are reactive autonomous software components, communicating in P2P mode and are  driven by human agents (actors) in charge of executing tasks. An agent's reaction to the reception of a message (an artifact) consists in the execution of a five-step protocol clearly described in this chapter (see sec. \ref{chap3:sec:the-protocols}). 
This protocol allows it to: (1) \textit{merge} the received artifact with the one it hosts locally in order to consider all updates, (2) \textit{project} the artifact resulting from the merger in order to hide the parts to which the local actor may not have access and highlight the tasks to be locally executed, (3) make the local actor \textit{execute} the revealed tasks and thus edit the potentially partial replica of the artifact obtained after the projection, (4) integrate the new updates into the artifact through an operation called \textit{expansion-pruning} and finally, (5) \textit{diffuse} the updated artifact to other sites for further execution of the process if necessary. 
The agents' operational capabilities allow that, for the execution of a given process, an artifact created by one of them (initially reduced to an open node), moves from site to site to indicate tasks that are ready to be executed at the appropriate time and to provide necessary data (created by other agents) for that execution; the mobile artifact, cooperatively edited by agents, thus "grows" as it transits through the distributed system so formed (see fig. \ref{chap3:fig:overview-protocol} (2)).


\mySubSection{The Running Example: the Peer-Review Process}{}
\label{chap3:sec:running-example}

\mySubSubSection{Description of the Peer-Review Process}{}
\label{chap3:sec:peer-review-description}
The peer-review process \cite{peerReview02} is a common example of administrative business process. We presented a brief description of it inspired by those made in \cite{peerReview02, van2001proclets, badouel14}, in chapter \ref{chap1:artifact-centric-bpm}, section \ref{chap1:sec:running-example}. Described in this way, we will use the peer-review process as an illustrative example in this chapter.

Lets recall that from that description, we have identified all the tasks to be executed, their sequencing, actors involved and the tasks assigned to them. Precisely, four actors are involved: an editor in chief ($EC$) who is responsible for initiating the process, an associated editor ($AE$) and two referees ($R1$ and $R2$).
A summary of tasks assignment was presented in table \ref{tableau:tachesExecutant} (page \pageref{tableau:tachesExecutant}), and orchestration diagrams using BPMN and WF-Net were also presented in figure \ref{chap1:fig:comparing-workflow-languages} (page \pageref{chap1:fig:comparing-workflow-languages}).


\mySubSubSection{Overview of the Peer-Review Process Artifact-Centric Execution using the Model Presented in this Thesis}{}
\label{chap3:sec:peer-review-overview}
To run the peer-review process described above according to the artifact-centric model presented in this chapter, four agents controlled by four actors (\textit{the editor in chief}, \textit{the associated editor} and the \textit{two referees}) will be deployed. 
Each of them will be pre-configured using a global Grammatical Model of Workflow (GMWf) and a set of accreditations. As we will see later, the global GMWf is used as model of artifacts and formally describes all the process tasks to be executed as well as their execution order (see fig. \ref{chap1:fig:comparing-workflow-languages}), and the accreditations set specifies the permissions (\textit{reading}, \textit{writing} and \textit{execution}) of each of the four actors relative to these tasks. 
After the pre-configuration of agents, each of them will derive (by projection \cite{tchoupeAtemkeng2}) a local GMWf which will locally guide the execution of the tasks to guarantee the confidentiality of some workflow data (contained in a mobile artifact) and the consistency of local updates with the global GMWf.

The artifact-centric execution of a scientific paper validation workflow will be triggered on the editor in chief's site, by introducing (in this site) an artifact (an annotated tree) reduced to its root node. Each node of the artifact represents a task and encapsulates an attribute containing its execution status. Therefore at a given time, the whole artifact contains information on already executed tasks and on data produced during their execution; it also exhibits tasks that are ready to be executed.
The analysis of this artifact by the local agent will highlight the expected contributions from the editor in chief. Guided by the local GMWf, he (here tasks are executed by a human) will perform tasks resulting in the consistent updating of the artifact's local copy; meaning that new nodes will be added to the artifact and some of its existing nodes will be updated: this is what we call editing an artifact. Then, this (updated) copy will be immediately analysed by the local agent to determine whether the currently managed process scenario is complete (this is the case when the artifact local copy structure matches one of the target artifacts: all causally dependent tasks have been executed) or not: in this case, all sites on which execution must continue are determined and an execution request is addressed to each of them (the artifact is sent to them).
Figure \ref{chap3:fig:overview-example} sketches an overview of exchanges that can take place between the four agents of the peer-review process when validating a scientific paper using the model presented in the present chapter. The scenario presented there, corresponds to the one in which the paper is pre-validated by the editor in chief and therefore, is analysed by a peer review committee. Note that there may be situations where multiple copies of the artifact are updated in parallel; this is notably the case when they are present on site 3 (first referee) and 4 (second referee).
\begin{figure}[ht!]
	\noindent
	\makebox[\textwidth]{\includegraphics[scale=0.28]{./Chap3/images/figOverviewExample.png}}
	\caption{An overview of the artifact-centric execution of the peer-review process using the model presented in this chapter.}
	\label{chap3:fig:overview-example}
\end{figure}
