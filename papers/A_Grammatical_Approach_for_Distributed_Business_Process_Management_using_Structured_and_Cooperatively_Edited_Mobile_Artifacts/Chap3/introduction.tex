\mySection{Introduction}{}
\label{chap3:sec:introduction}
%\subsection*{Contexte et définitions}
%\label{sec:contexte}

As outlined in chapter \ref{chap1:artifact-centric-bpm}, section \ref{chap1:sec:artifact-centric-bpm-approach}, the execution of a given business process according to the artifact-centric approach can be assimilated to the cooperative editing of documents. Indeed, in IBM's work \cite{nigam2003business}, an artifact (also called "\textit{adaptive document}") is considered as a document that conveys all the information concerning a particular case of execution of a given process, from its inception into the system to its termination. In particular, this information provides details on the execution status of the case as well as on its lifecycle (a representation of the possible evolutions of this status). 
To do this, during the execution of a given process, the actions carried out by each of the stakeholders (agents) have the effect of updating (\textit{editing}) the artifacts involved in that execution. If the process is cooperative, the artifact representing it will be updated by several agents: it is said to be cooperatively edited (\textit{cooperative editing}). 
%In choreography-oriented approaches (see chapter \ref{chap1:artifact-centric-bpm}, Section \ref{chap1:sec:artifact-centric-bpm-approach}), in order to increase parallelism of execution, this cooperative editing is distributed, asynchronous and deals with copies (replicas) of the artifact that are subsequently merged at the appropriate time.

In this chapter, we propose a new artifact-centric approach to BPM. In this one, artifacts are seen as structured documents (annotated trees) that can be exchanged between the different agents involved in the execution of a given business process particular case (it is in this sense that they are said to be mobile); during their life, they are edited appropriately to make the system converge towards the achievement of one of the considered process's business goals. The approach presented in this chapter is based on the asynchronous structured cooperative editing techniques proposed in the work of Badouel et al. \cite{badouelTchoupeCmcs, theseTchoupe, tchoupeAtemkeng2} and extended in chapter \ref{chap2:structured-editing-artifact-type} \cite{tchoupeZekeng2016, tchoupeZekeng2017, zekengTchoupe2018} of this manuscript.  

\noindent The major contributions of this chapter are as follows:	
\begin{enumerate}
\item The proposal of another tree-based model of "business artifact", which makes it possible to better assimilate them to structured documents edited cooperatively;

\item The proposal of a choreography-oriented artifact-centric execution model in which agents execute the same and unique update (editing of artifact upon receipt) and diffusion (dissemination of updates) protocol;

\item The proposal of a prototype of a distributed system allowing to fully experiment the approach investigated in this chapter.
\end{enumerate}
  
In the rest of this chapter, in section \ref{chap3:sec:model-overview}, we present an overview of the studied artifact-centric model and the distributed execution of the peer-review process used as a running example in this manuscript. In section \ref{chap3:sec:modelling-artifacts}, we introduce and formally define the concepts of artifact and artifact-type (GMWf). We then present in section \ref{chap3:sec:agents-and-choreography}, the internal structure (architecture and features) of an agent, the notion of accreditation as well as the new artifact-centric and completely decentralised execution model of administrative processes that we propose. Illustrations of section \ref{chap3:sec:agents-and-choreography}'s algorithms are given in section \ref{chap3:sec:choreograpghy-illustration} in order to facilitate their understanding. In the same vein, a prototype system allowing to fully experiment the approach investigated in this chapter is presented in section \ref{chap3:sec:p2ptinywfms}. In section \ref{chap3:sec:discussion}, we discuss the obtained results as well as a positioning of these results in relation to those in the literature. The section \ref{chap3:sec:conclusion} is devoted to the conclusion.

