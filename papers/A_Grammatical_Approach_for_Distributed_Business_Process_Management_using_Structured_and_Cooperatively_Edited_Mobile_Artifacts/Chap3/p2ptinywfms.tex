\mySection{Experimentation}{}
\label{chap3:sec:p2ptinywfms}

In this section, we present and experiment \textit{P2PTinyWfMS} (a Peer-to-Peer Tiny Workflow Management System), an experimental prototype system implemented according to the approach proposed in this chapter.

\mySubSection{P2PTinyWfMS: an Experimental Prototype System}{}
\textit{P2PTinyWfMS} is a tool developed in Java under Eclipse\footnote{Official website of Eclipse: \url{https://www.eclipse.org}, visited the 04/04/2020.} and dedicated to the distributed execution of administrative workflows specified using GMWf. 
In accordance with the agent's architecture of this chapter (see fig. \ref{chap3:fig:peer-architecture}), \textit{P2PTinyWfMS} has a front-end for displaying and graphically editing artifacts manipulated during the execution of a business process (see fig. \ref{chap3:sec:p2ptinywfms-1} and \ref{chap3:sec:p2ptinywfms-3}), as well as a communication module built from SON\footnote{SON is available under Eclipse from a family of SmartTools plugins.}. 

Let's recall that SON (Shared-data Overlay Network) \cite{SON} is a middleware offering several DSL to facilitate the implementation of P2P systems whose components communicate by service invocations. 
Component Description Meta Language (CDML) is the DSL provided by SON to specify among other things the services required and provided by each peers; from a CDML specification, SON generates Java code for allowing peers to communicate.
The following listing shows the contents of the CDML file used in the case of \textit{P2PTinyWfMS} to specify the four services that its instances expose; they are: two input services (\textit{inForwardTo} - lines 6 to 9 -, and \textit{inReturnTo} - lines 10 to 13 -) and two output services (\textit{outForwardTo} - lines 14 to 17 - and \textit{outReturnTo} - lines 18 to 21 -). These services take as argument an artifact corresponding to either a request or a response.

\begin{Verbatim}[frame=lines,fontsize=\scriptsize, numbers=left, numbersep=8pt, label=CDML file: specification of required and provided services of P2PTinyWfMS]
<?xml version="1.0" encoding="ISO-8859-1"?>
<component name="p2pTinyWfMS" type="p2pTinyWfMS" extends="inria.communicationprotocol"
 ns="p2pTinyWfMS">
  <containerclass name="P2pTinyWfMSContainer"/>
  <facadeclass name="P2pTinyWfMSFacade" userclassname="P2pTinyWfMS"/>
  <input name="forwardTo" method="inForwardTo">
    <attribute name="request" 
     javatype="smartworkflow.dwfms.lifa.miu.util.p2pworkflow.PeerToPeerWorkflowRequest"/>
  </input>
  <input name="returnTo" method="inReturnTo">
    <attribute name="response" 
     javatype="smartworkflow.dwfms.lifa.miu.util.p2pworkflow.PeerToPeerWorkflowResponse"/>
  </input>
  <output name="forwardTo" method="outForwardTo">
    <attribute name="request" 
     javatype="smartworkflow.dwfms.lifa.miu.util.p2pworkflow.PeerToPeerWorkflowRequest"/>
  </output>
  <output name="returnTo" method="outReturnTo">
    <attribute name="response" 
     javatype="smartworkflow.dwfms.lifa.miu.util.p2pworkflow.PeerToPeerWorkflowResponse"/>
  </output>
</component>
\end{Verbatim}


\mySubSection{Executing our Running Example under P2PTinyWfMS}{}
SON offers a DSL (the "\textit{.world}" files) for the description of the deployment of a distributed system whose components have been specified by a CDML file. In order to execute our running example, we deployed four instances of \textit{P2PTinyWfMS} identified by $EC$, $AE$, $R1$ and $R2$ respectively. As explained in section \ref{chap3:sec:choreograpghy-illustration}, each instance is initially equipped with the global GMWf as well as accreditations of various agents from which it derives its local GMWf by projection.

Figures \ref{chap3:sec:p2ptinywfms-1}, \ref{chap3:sec:p2ptinywfms-3} and \ref{chap3:sec:p2ptinywfms-6} are screen shots with some highlights of the workflow's distributed execution.
We have the tab "\textit{Workflow overview}" presenting at the beginning of the execution, various tasks, agents, target artifacts etc., on the editor in chief's site (fig. \ref{chap3:sec:p2ptinywfms-1}). We also have the tabs "\textit{Workflow execution}" of the sites of the associated editor (fig. \ref{chap3:sec:p2ptinywfms-3}) and of the editor in chief (fig. \ref{chap3:sec:p2ptinywfms-6}) that present the artifacts resulting from their execution after receiving a request from the editor in chief (resp. after receiving a response from the associated editor).
\begin{figure}[ht!]
	\noindent
	\makebox[\textwidth]{\includegraphics[scale=0.43]{./Chap3/images/p2ptinywfms-1.png}}
	\caption{P2pTinyWfMS on the editor in chief's site: presentation of the GMWf (the tasks and their relations, the actors and their accreditations).}
	\label{chap3:sec:p2ptinywfms-1}
\end{figure}

\begin{figure}[ht!]
	\noindent
	\makebox[\textwidth]{\includegraphics[scale=0.43]{./Chap3/images/p2ptinywfms-3.png}}
	\caption{P2pTinyWfMS on the associated editor's site: receipt of editor in chief's request, execution of tasks, expansion-pruning, and diffusion.}
	\label{chap3:sec:p2ptinywfms-3}
\end{figure}

\begin{figure}[ht!]
	\noindent
	\makebox[\textwidth]{\includegraphics[scale=0.43]{./Chap3/images/p2ptinywfms-6.png}}
	\caption{P2pTinyWfMS on the editor in chief's site: reception of the associated editor's response, execution of tasks, expansion-pruning and end of the case.}
	\label{chap3:sec:p2ptinywfms-6}
\end{figure}
