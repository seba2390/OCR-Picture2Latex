\mySection{Illustrating the Choreography on the Peer-Review Process}{}
\label{chap3:sec:choreograpghy-illustration}

The execution of an instance of our running example begins when under the editor in chief's action (via the GUI of the specialised editor), an unlocked bud of type $A_{\mathbb{G}}$ is created on his site.
Figure \ref{chap3:fig:execution-figure-1} which must be read following the direction of the arrows it contains, summarises the state\footnote{The state of an agent $i$ at a given moment is given by the values of variables $REQ_i$, $ANS_i$, $RET_i$ and the replicas $t_i$ and $t_{\mathcal{V}_i}$.} of the agent $EC$ (editor in chief site) before and after the event creating the artifact; it also illustrates the running of the five-step protocol on the agent $EC$.
\begin{figure}[ht!]
	\noindent
	\makebox[\textwidth]{\includegraphics[scale=0.28]{./Chap3/images/executionFigure1.png}}
	\caption{Beginning of the peer-review process on the editor in chief's site.}
	\label{chap3:fig:execution-figure-1}
\end{figure}

As soon as a bud of type $A_{\mathbb{G}}$ is created, the local workflow engine extends it using the unique $A_{\mathbb{G}}$-production ($P_{1}:A_{\mathbb{G}} \rightarrow A$) of the local GMWf. This results in the creation of a bud of type $A$ that the editor in chief must extend via the specialised editor by choosing an $A$-production.
For this scenario, it is assumed that he chooses the production $P_{3}:A \rightarrow C \fatsemi D$.
The task ($A$) executed by the latter is shown in green colour on figure \ref{chap3:fig:execution-figure-1}.
The newly created tasks ($C$ and $D$) appear in the form of locked buds (the locked buds are shown in red colour) because the editor in chief is not accredited in writing on $C$ and, since $D$ is linked to $C$ by a sequential scheduling constraint, it can only be executed when all tasks ($C$, $E$, $F$, $G1$, $G2$, $H1$, $H2$, $I1$, $I2$) preceding it will have been executed.
After expansion-pruning, the only required agent is the associated editor (responsible for executing task $C$): a request is sent to it by invoking the service \textit{forwardTo}.
\begin{figure}[ht!]
	\noindent
	\makebox[\textwidth]{\includegraphics[scale=0.28]{./Chap3/images/executionFigure2.png}}
	\caption{Continuation of the peer-review process execution on the associated editor's site; the latter receives the request formulated by the editor in chief.}
	\label{chap3:fig:execution-figure-2}
\end{figure}

The event that triggers the workflow execution on the site of associated editor (fig. \ref{chap3:fig:execution-figure-2}) is the receipt of the request sent by the editor in chief.
In the artifact sent by the latter, there are buds ($C_{\overline{\omega}}$ and $D_{\overline{\omega}}$).
After merging, the bud $C_{\omega}$ is unlocked (the unlocked buds are shown in blue). It indicates the only place where the contribution of the associated editor is expected.
During the execution phase, the local artifact partial replica is updated by the associated editor via the productions $P_{3}: C\rightarrow E\fatsemi F$, $P_{4}: E\rightarrow S1\parallel S2$, $P_{5}: S1\rightarrow H1\fatsemi I1$ and $P_{6}: S2\rightarrow H2\fatsemi I2$ of his local GMWf. 
At the end of this phase, buds of types $H1, H2, I1$ and $I2$ appear locked not only because they are constrained by a sequential scheduling (case of $I1$ and $I2$), but especially because of the presence of \textit{upstairs buds} (the upstairs buds are represented in orange colour).
Indeed, tasks $G1$ and $G2$ made visible after the expansion are upstairs buds because they must be executed before the tasks of type $H1$ and $H2$. So, there is pruning at $G1$ and $G2$ before sending (in parallel) the artifact to both referees.
\begin{figure}[ht!]
	\noindent
	\makebox[\textwidth]{\includegraphics[scale=0.28]{./Chap3/images/executionFigure3.png}}
	\caption{Continuation of the peer-review process execution on the first referee's site: the request of the associated editor arrives at the first referee.}
	\label{chap3:fig:execution-figure-3}
\end{figure}

Figure \ref{chap3:fig:execution-figure-3} illustrates how the protocol takes place on the site of one of the referees (the first referee). After the contribution of the latter, no new bud is created: no request is formulated. It is rather a response corresponding to the request previously received from the associated editor which is returned by invoking the service $returnTo$.
\begin{figure}[ht!]
	\noindent
	\makebox[\textwidth]{\includegraphics[scale=0.28]{./Chap3/images/executionFigure4.png}}
	\caption{Continuation of the peer-review process execution: the associated editor receives answers from referees, to requests that he has previously made.}
	\label{chap3:fig:execution-figure-4}
\end{figure}

The execution protocol is unrolled again on the site of the associated editor following events related to the reception of responses from the two referees (fig. \ref{chap3:fig:execution-figure-4}). We choose to treat these responses simultaneously; but we could do otherwise and obtain the same result. 
At merge, since the subtree rooted in $E$ is closed, the bud $F_{\overline{\omega}}$ is unlocked and the associated editor extends it through production $P_{11}: F \rightarrow \varepsilon$. Having no request to make, the answer to the request previously received from the editor in chief is returned.
\begin{figure}[ht!]
	\noindent
	\makebox[\textwidth]{\includegraphics[scale=0.28]{./Chap3/images/executionFigure5.png}}
	\caption{Continuation and end of the peer-review process execution: the editor in chief receives a response containing referees' contributions, from the associated editor.}
	\label{chap3:fig:execution-figure-5}
\end{figure}

The editor in chief receives the response from associated editor and once again runs the execution protocol (fig. \ref{chap3:fig:execution-figure-5}). 
After its contribution (on the node $D$), the artifact obtained after expansion-pruning is closed and the execution of the process ends successfully.
