\mySection{Summary}{}
\label{chap1:sec:summary}
%\subsection*{Contexte et définitions}
%\label{sec:contexte}
In this chapter, we have provided an overview of the basic concepts related to BPM. To this end, we have provided clear and concise definitions of numerous notions. We presented the concept of WfMS, how it works and the current challenges in the production of such systems. Knowing that the main current challenges in the production of WfMS are to provide them with the flexibility to better cope with the distributed nature of the workflows they manage, and extend their expressiveness so that they can address as first-class citizens, other perspectives of workflows such as data and users, we presented new paradigms to BPM and a few approaches to P2P BPM.

In addition, we focused on the artifact-centric paradigm for BPM and established that, according to it, the execution of a given workflow can be seen as the cooperative editing of one or more documents called artifacts. In these cases, it is preferable that the manipulated documents are structured; they can be exchanged between the different actors in the workflows' execution (they are said to be mobile) to serve as a support to help their coordination, but also, indirectly, to be the fruit of their cooperation: this is the research axis that we followed in this thesis.

To make our task easier, it would be a good idea to look at structured documents' cooperative editing workflows: it is the subject of the next chapter. Since Badouel and Tchoup\'e have theorised an asynchronous cooperative editing model for structured documents of which, some of the concepts were brilliantly taken up in the development of the AWGAG model, it is their editing model that will be the main subject of our study.



