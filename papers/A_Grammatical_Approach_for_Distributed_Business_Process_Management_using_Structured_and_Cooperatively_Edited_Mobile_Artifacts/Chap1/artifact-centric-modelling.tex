\mySection{Artifact-Centric Business Process Management}{}
\label{chap1:sec:data-aware-bpm}
%\subsection*{Contexte et définitions}
%\label{sec:contexte}
Emerged in the early 2000s, the \textit{artifact-centric} paradigm of BPM is one of those that has been much studied over the last two decades. This paradigm has been pioneered by IBM \cite{nigam2003business} and revisited in several works such as \cite{abi2016towards, deutsch2014automatic, hull2009facilitating, lohmann2010artifact, assaf2017continuous, assaf2018generating, boaz2013bizartifact, lohmann2011artifact, estanol2012artifact}; it proposes a new approach to BPM by focusing on both automated processes (tasks and their sequencing) and data manipulated through the concept of "\textit{business artifact}" (\textit{artifact-centric modelling}). In this section we present the key concepts of the artifact-centric paradigm as well as some artifact-centric frameworks from the literature.

\mySubSection{Artifact-Centric BPM Basic Concepts}{}
\label{chap1:sec:artifact-centric-bpm-key-concept}

\mySubSubSection{The Aim of Artifact-Centric BPM}{}
\label{chap1:sec:aim-artifact-centric-bpm}
In order to be able to better model workflows, process modelling should include a specification of the order in which tasks are executed (control flow), the way data are processed (data flow), and how different branches in distributed and inter-organisational business processes and services are invoked and coordinated (message flow) \cite{lohmann2011artifact}. These three conceptual models of workflows are also known as the \textit{process}, the \textit{informational} and the \textit{organisational} models \cite{divitini2001inter}. Traditional approaches (BPMN, YAWL, BPEL, etc.) to BPM are process-centric (they are also said to be \textit{imperative}): they generally offer two different views on business processes: 
\begin{enumerate}
	\item Collaboration diagrams (sometimes called interconnected models) that emphasise the local control flow of each participant of the process;
	\item Choreography diagrams (interaction models) that describe the process from the point of view of the messages that are exchanged among the participants.
\end{enumerate}
Traditional approaches thus express workflow models by means of diagrams which define how a workflow is supposed to operate, but give little importance (or none at all) to the information produced as a consequence of the process execution: data are treated as second-class citizens.

To precisely remedy this, researchers have developed the artifact-centric \cite{nigam2003business} approach to the design and execution of business processes. Artifact-centric models do not specify processes as a sequence of tasks to be executed or messages to be exchanged (i.e. imperatively), but from the point of view of the data objects (called \textit{business artifacts} or simply \textit{artifacts}) that are manipulated throughout the course of the process (i.e. declaratively) \cite{lohmann2011artifact}. They rely on the assumption that any business needs to record details of what it produces in terms of concrete information. Artifacts are proposed as a means to record this information. They model key business-relevant entities which are updated by a set of services (specified by pre and postconditions) that implement business process tasks. This approach has been successfully applied in practice and it provides a simple and robust structure for workflow modelling \cite{estanol2012artifact}.



\mySubSubSection{How the Artifact-Centric Approach to BPM Works}{}
\label{chap1:sec:artifact-centric-bpm-approach}
According to the artifact-centric paradigm, BPM takes place in two main phases guided by the concept of artifact. In order to automate a given process, the designer must first of all focus on artifact modelling; i.e. he must provide data structures capable of storing and logically conveying the information produced during the execution of workflows. Then, these artifacts models will be introduced into a WfMS supporting artifact-centric execution of workflows for the enactment.

~

\noindent\textbf{\textit{What is an artifact ?}}

\cite{nigam2003business} define an artifact as \textit{"a concrete, identifiable, self-describing chunk of information that can be used by a business person to actually run a business"}. Artifacts are business-relevant objects that are created, evolve, and (typically) archived as they pass through the workflow \cite{hull2009facilitating}; they represent key conceptual objects of workflow that evolve as they move through an enterprise. Artifacts are modelled through \textit{artifacts types} (or \textit{models}). An expected characteristic of artifacts is that they should be self-describing: this requirement allows a business person to be able to look at an artifact and determine if he or she can work on it. Toward this end, an artifact type should include both:
\begin{enumerate}
	\item an \textit{information model} (or "data schema"), for holding information about the artifact as it moves through the process, from creation to archival storage; and
	\item a \textit{lifecycle model} (or "lifecycle schema"), which describes how and when tasks (activities or services) might be invoked on the artifacts as they move through the process.
\end{enumerate}
A prototypical example of an artifact type is the "air courier package delivery", whose information model can hold data about a package including sender, receiver, steps occurring in transport and billing activity; and whose lifecycle model would specify the possible ways that the delivery service and billing might be carried out \cite{hull2013data, hull2009facilitating}.

Several approaches of modelling the lifecycle of artifacts have been studied in the literature. The most commonly used approach is that in which, some form of \textit{finite state machines} (automata) \cite{hull2009facilitating} are used to specify lifecycles. Other variants presenting the lifecycle of an artifact by a Petri net \cite{lohmann2010artifact}, logical formulae depicting legal successors of a state \cite{damaggio2012artifact} have also been proposed.


~

\noindent\textbf{\textit{The artifact-centric execution}}

Artifact-centric models can be executed by artifact-centric WfMS. This new type of WfMS put stress on how artifacts are created, updated and exchanged between various actors. In these, artifacts are considered as \textit{adaptive documents} that conveys all the information concerning a particular execution case of a given process, from its inception in the system to its termination. In particular, this information provides details on the case's execution status as well as on its lifecycle (a representation of the possible evolutions of this status). To do this, during the execution of a given process, the actions carried out by each of the actors (agents) have the effect of updating (\textit{editing}) the artifacts involved in that execution. If the process is cooperative, the artifact representing it will be updated by several agents: it is said to be cooperatively edited and thus, the execution of a given business process according to the artifact-centric approach, can be assimilated to the \textit{cooperative editing} of documents.

Two major trends in the artifact-centric modelling approach have been developed: \textit{orchestration} and \textit{choreography} \cite{hull2009facilitating}. 
\begin{enumerate}
	\item Orchestration suggests the creation of centralised systems (usually called \textit{artifact hubs}), coordinated by an \textit{orchestrator} whose role is to facilitate interaction between actors while ensuring that business goals are met.
	\item Choreography-oriented approaches get rid of the orchestrator, and model actors as autonomous agents coordinating with artifacts and communicating in a P2P manner, to accomplish business goals. In these, each agent focuses on achieving a local business goal and the achievement of the global business goal is the result of aggregating results from different local business goals.
\end{enumerate}
Compared to choreography, orchestration reduces the agents' autonomy by making the orchestrator the main controller of interactions. Also, the orchestrated approach does not scale well. A limitation of the choreography-oriented approach is the lack of a single synchronisation point from which, it is possible to know the process's (actual) global execution state. Despite this, we agree with \cite{lohmann2010artifact} that, this completely decentralised approach is the one that best fits the modelling of the intrinsically distributed nature of business processes.




\mySubSection{Some Existing Artifact-Centric BPM Frameworks}{}
\label{chap1:sec:existing-artifact-centric-bpm}
There are several frameworks in the literature, that implement artifact-centric concepts. We briefly present some of them in this section. We begin by presenting purely artifact-centric approaches; then we look at an even more flexible model, recently developed as a \textit{data-centric} solution for case management.


\mySubSubSection{Some Purely Artifact-Centric BPM Frameworks}{}
\label{chap1:sec:purely-artifact-centric-bpm}
\noindent\textbf{\textit{Proclets \cite{van2001proclets}}}

The concept of \textit{proclets} was introduced to specify business processes in which, objects' lifecycles can be modelled at different levels of granularity and cardinality. A proclet can be seen as a lightweight workflow process equipped with a knowledge base that contains information on previous interactions; it is thus equipped with an explicit lifecycle or active documents (i.e., documents aware of tasks and processes): in this setting then, a proclet is both an agent and an artifact. Proclets can find each other using a \textit{naming service}, and communicate with each other to exchange messages through \textit{channels} (see fig. \ref{chap1:fig:proclet}). 
\begin{figure}[ht!]
	\noindent
	\makebox[\textwidth]{\includegraphics[scale=0.7]{./Chap1/images/proclet.png}}
	\caption{Graphical representation of the proclet-based framework (source \cite{van2001proclets}).}
	\label{chap1:fig:proclet}
\end{figure}

In the proclet-based framework, the lifecycle of proclet instance is described by a \textit{proclet class} used as artifact type. Like an ordinary workflow model, a proclet class describes the order in which tasks can/need to be executed for individual instances of the class. Proclet classes are specified using a graphical language based on a sub-class of Petri nets so-called \textit{class of sound WF-nets}.

Proclets are well-suited to deal with settings in which several instances of data objects are involved. Proclets are considered to be distributed and autonomous enough to decide how to interact with the other proclets: thus, the proclet-based framework does not model proclets' locations. Moreover, the execution model of proclets is similar to choreography; the interoperation of proclets is not managed or facilitated by a centralised hub.


~

\noindent\textbf{\textit{Artifact hosting \cite{hull2009facilitating}}}

Hull et al. extend the artifact-centric model proposed by Nigam and Caswell \citeyearpar{nigam2003business}, to provide an interoperation framework in which data (artifacts) are hosted on central infrastructures named \textit{artifact-centric hubs}. Data hosted in artifact-centric hubs can be read and written by agents. This model is between choreography and orchestration because, agents are all connected to the hub but are not coordinated by a particular orchestrator. Unlike traditional orchestration schemes, the hub enables the participating agent to be pro-active, and serves primarily as a shared resource for coordinating activities. Participating agents can access information about the running artifact instances, can progress those instances along their lifecycles, and can subscribe to events in order to be alerted about significant steps in the progress of artifacts through their lifecycles. Security mechanisms are proposed for controlling access to data hosted in the hub.
\begin{figure}[ht!]
	\noindent
	\makebox[\textwidth]{\includegraphics[scale=0.5]{./Chap1/images/artifact-hub.png}}
	\caption{An example of interoperation using an artifact-centric hub (source \cite{hull2009facilitating}).}
	\label{chap1:fig:artifact-hub}
\end{figure}

Figure \ref{chap1:fig:artifact-hub} illustrates an example of six groups of agents (potentially organisations) coordinating using artifacts that are managed in a centralised hub. These are the agents related to the resources (\textit{Candidates}, the \textit{Human Resources Organisation}, the \textit{Hiring Organisations}, the \textit{Evaluators}, the \textit{Travel Provider} and the \textit{Reimbursement}) that carry out the employee hiring process in a given enterprise.


~

\noindent\textbf{\textit{Artifact-centric choreographies \cite{lohmann2010artifact}}}

Lohmann and Wolf \citeyearpar{lohmann2010artifact} provide a more choreography-like framework for artifact-centric interoperation. They abandon the fact of having a single artifact hub \cite{hull2009facilitating} and they introduce the idea of having several agents which operate on artifacts. Some of those artifacts are \textit{mobile} (their location may change over time); thus, the authors provide a systematic approach for modelling artifact location and its impact on the accessibility of actions using a Petri net. Their model was designed with the conviction that by making explicit who is accessing an artifact and where the artifact is located, one will be able to automatically generate an interaction model that can serve as a contract between agents, and which make sure that global goal states specified on artifacts are reached. They thus propose an approach to automatically derive such an interaction model.



~

\noindent\textbf{\textit{Declarative choreographies \cite{sun2012declarative}}}

In \cite{sun2012declarative}, the authors are also interested in choreographies. More precisely, they develop a language allowing to model (in a declarative manner) the collaboration between several actors (the choreography) and a distributed algorithm allowing the execution of the choreographies specified in their language. Their choreography language has four distinct features :
\begin{enumerate}
	\item Each type of actor is an artifact schema with a selected sub-part of its information model visible to choreography specification.
	\item Correlations between actor types and instances are explicitly specified, along with cardinality constraints on correlated instances (e.g. each Order instance may correlate with exactly one Payment instance and multiple Vendor instances).
	\item Messages can include data; data in both messages and artifacts can be used in choreography constraints.
	\item The language is declarative and uses logic rules based on a mix of first-order logic and a set of binary temporal operators from DecSerFlow\footnote{DecSerFlow: \textit{a Declarative Service Flow Language} is a graphical, extendible language for expressing process models in a declarative way; it captures what is the high-level process behaviour without expressing how it is procedurally executed, hence giving a concise and easily interpretable feedback to the business manager \cite{lamma2007learning}.} \cite{van2006decserflow}.
\end{enumerate}
In particular, Skolem\footnote{Thoralf Albert Skolem (1887-1963) is a Norwegian mathematician and logician. He is particularly known for his work in mathematical logic and set theory which now bears his name, such as the L\"{o}wenheim-Skolem theorem or the notion of skolemisation (source, Wikipedia: \url{https://en.wikipedia.org/wiki/Thoralf_Skolem}, visited the 02/04/2020).} notations are used to both reference correlated actor instances and to manipulate dependencies among messages. 





\mySubSubSection{A Guarded Attribute Grammars Based Framework to Data-Centric Case Management}{}
\label{chap1:sec:gag}
\noindent\textbf{\textit{What is case management ?}}

Highly important processes in organisations that have a tremendous impact on the success and add the most value, involve a high degree of knowledge work: they are driven by users' decisions (\textit{user-centric}) making it difficult to specify them into a set of activities with precedence relations at design-time (they are said to be \textit{knowledge-intensive}). Because knowledge-intensive processes are subject of frequent exceptions, traditional BPM solutions are not able to support them sufficiently \cite{hauder2014research}. \textit{Adaptive Case Management} (ACM) is gaining interest among researchers and practitioners as an  emerging paradigm to master situations in which adaptions have to be made at run-time (unpredictable situations) by so called knowledge workers. In contrast to traditional BPM, the ACM paradigm is not dictating knowledge workers a predefined course of action, but provides them with the required information at the right time (they are \textit{data-centric}) and authorises them to make decisions on their own \cite{hauder2014research}. 

The notion of \textit{case} in the ACM context, is closely related to the concept of artifact. Both involve the notion of a conceptual entity that progresses through time, according to some set of guidelines or lifecycle schema, and both taking advantage of a growing set of data accumulated over the case instance lifecycle \cite{hull2013data}. ACM can be seen as an extension of the artifact-centric paradigm in which, the flexibility of workflow models (types of artifacts) is highly valued; therefore, both users and data are treated as first-class citizens.

There is a growing research interest in the ACM paradigm and several models have already emerged. Guard-Stage-Milestone (GSM) \cite{hull2011business, damaggio2013equivalence}, a declarative model of the lifecycle of artifacts, was recently introduced and has been adopted as a basis of \textit{Case Management Model and Notation} (CMMN), the OMG standard for ACM. The GSM model defines Guards, Stages and Milestones to control the enabling, enactment and completion of (possibly hierarchical) activities; it then allows for dynamic creation of subtasks (the stages), and handles data attributes. However, interaction with users are modelled as incoming messages from the environment, or as events from low-level (atomic) stages. In this way, users do not explicitly contribute to the choice of a flow for a process.

Recently, Badouel et al. \citeyearpar{badouel14, badouel2015active} have proposed a more user-centric and data-driven ACM model (AWGAG) based on the concepts of Active Workspaces (AW) and Guarded Attribute Grammars (GAG). We are particularly interested in this model because, it incorporates concepts that we will manipulate throughout this manuscript. These include the concepts of \textit{grammars} as artifact types, \textit{structured documents} (trees) as artifacts, \textit{artifact editing}, etc.

~

\Needspace{5\baselineskip}
\noindent\textbf{\textit{AWGAG \cite{badouel2015active}}}

The AWGAG model of collaborative systems is centered on the notion of user's workspace. It assume that the workspace of a user is given by a map. It is a tree used to visualise and organise tasks in which, the user is involved together with information used for their resolution. The workspace of a given user may, in fact, consist of several maps where each map is associated with a particular service offered by the user. In short, one can assume that a user offers a unique service so that any workspace can be identified with its graphical representation as a map.
\begin{figure}[ht!]
	\noindent
	\makebox[\textwidth]{\includegraphics[scale=0.7]{./Chap1/images/aw-clinician.png}}
	\caption{Active workspace of a clinician (source \cite{badouel2015active}).}
	\label{chap1:fig:aw-clinician}
\end{figure}

As an example, figure \ref{chap1:fig:aw-clinician} shows a map that represents the workspace of a clinician acting in the context of a disease surveillance system. The service provided by the clinician is identifying the symptoms of influenza in a patient, clinically examining the patient, eventually placing him under therapeutic care, declaring the suspect cases to the disease surveillance center, and monitoring the patient based on subsequent requests from the epidemiologist or the biologist.

Each call to this service, namely when a new patient comes to the clinician, creates a new tree rooted at the central node of the map. This tree is an artifact that represents a structured document for recording information about the patient all along being taken over in the system. Initially, the artifact is reduced to a single (open) node that bears information about the name, age and sex of the patient. An open node, graphically identified by a question mark, represents a pending task that requires the clinician's attention. In this example the initial task of a given artifact is to clinically examine the patient. This task is refined into three subtasks: clinical assessment, initial care, and case declaration.

In the AWGAG model, a task is interpreted as a problem to be solved, that can be completed by refining it into sub-tasks using business rules. A business rule is modelled by a production $P: s_0 \rightarrow s_1 \ldots s_n$ expressing that task $s_0$ can be reduced to subtasks $s_1$ to $s_n$. For instance, the production 
\begin{itemize}
	\item[] $patient \rightarrow clinicalAssessment \ initialCare \ caseDeclaration$
\end{itemize}
states that, a task of sort $patient$, the axiom of the grammar associated with the service provided by the clinician, can be refined by three subtasks whose sorts are respectively $clinicalAssessment$, $initialCare$, and $caseDeclaration$. If several productions with the same left-hand side $s_0$ exist, then the choice of a particular production corresponds to a decision made by the user. In the example, the clinician has to decide whether the case under investigation has to be declared to the disease surveillance center or not. This decision can be reflected by the following two productions:
\begin{itemize}
	\item[] $suspectCase: caseDeclaration \rightarrow followUp$
	\item[] $benignCase: caseDeclaration \rightarrow$
\end{itemize}
If the case is reported as suspect, then the clinician will have to follow up the case according to further requests of the biologist or the epidemiologist. On the contrary (i.e. the clinician has described the case as benign), the case is closed with no follow up actions.

AWGAG  model considers artifacts as trees whose nodes are sorted and whose productions are taken into a grammar (GAG). The lifecycle of an artifact is implicitly given by the set of productions of the underlying GAG:
\begin{enumerate}
	\item The artifact initially associated with a case, is reduced to a single open node.
	\item An open node $X$ of sort $s$ can be refined by choosing a production $P: s \rightarrow s_1 \ldots s_n$ that fits its sort. The open node $X$ becomes a closed node under the decision of applying production $P$ to it. In doing so, task $s$ associated with $X$ is replaced by $n$ subtasks $s_1$ to $s_n$, and new open nodes $X_1$ of sort $s_1$ to $X_n$ of sort $s_n$, are created accordingly: the artifact is said to be edited (see fig. \ref{chap1:fig:aw-artifact-edition}).
	\item The case reaches completion when its associated artifact is closed, i.e. it no longer contains open nodes.
\end{enumerate}
\begin{figure}[ht!]
	\noindent
	\makebox[\textwidth]{\includegraphics[scale=0.5]{./Chap1/images/aw-artifact-edition.png}}
	\caption{Artifact edition in AWGAG (source \cite{badouel2015active}).}
	\label{chap1:fig:aw-artifact-edition}
\end{figure}

Additional information are attached to open nodes using \textit{attributes}, to model the interactions and data exchanged between the various tasks associated with them. For that, each sort comes equipped with a set of \textit{inherited} attributes and a set of \textit{synthesised} attributes where: inherited attributes represents input data, i.e. necessary data for the associated task to be executed, while synthesised attributes represents output data, i.e. data that are produced after the task being executed. This formalism puts emphasis on a declarative (logical) decomposition of tasks to avoid over-constrained schedules. Indeed, business rules do not prescribe any ordering on task executions. Ordering of tasks depend on the exchanged data and are therefore determined at runtime. In this way, the AWGAG model allows as much concurrency as possible in the execution of the current pending tasks.

Furthermore, a given AWGAG model is flexible and can incrementally be designed: one can initially let the designer manually develop large parts of the map, and progressively improve the automation of the process by refining the classification of the nodes, and introducing new business rules when recurrent patterns of activities are detected. The AWGAG model presents great properties such as \textit{distribution} and \textit{soundness}; these properties and more details are discussed in \cite{badouel2015active}. Several implementations and extensions of the AWGAG model are currently being carried out.




