\mySection{Key Principles of Business Process Management}{}
\label{chap1:sec:bpm-def-key-principles}
%\subsection*{Contexte et définitions}
%\label{sec:contexte}

\mySubSection{Some Business Process Management Basic Concepts}{}
\label{chap1:sec:bpm-def-history}
Research in the CSCW field focuses on the role of computers in collaborative work \cite{schimdt1992taking}. These have given rise to numerous softwares called \textit{CSCW systems} or \textit{groupware}. CSCW systems communicate through networks and provide functionalities facilitating exchanges, coordination, collaboration and co-decision between the actors of a given collaborative work; they thus defy the space and time constraints to which collaborative work is subjected. Indeed, with the help of such systems, actors can either operate on the same site and thus manipulate the same objects (\textit{centralised approach}), or they can operate on geographically distant sites (\textit{distributed approach}); in this case, the objects they manipulate are replicated on the different sites and synchronised at the appropriate time \cite{johansen1988groupware, grudin1994computer, penichet2007classification}. In the same vein, they can act at the same time (\textit{synchronous approach}) or at completely different times and sometimes independently of the actions carried out by others (\textit{asynchronous approach}) \cite{johansen1988groupware, grudin1994computer, penichet2007classification}. 
CSCW systems are often referred to as \textit{workflow systems}. However, it should be noted that workflow is an extension and a generalisation of CSCW to business processes' automation.

\mySubSubSection{Some Definitions}{}
\label{chap1:sec:bpm-basic-concepts-def}
A \textit{business process} can be informally defined as a set of tasks ordered following a specific pattern and whose execution produces a service or a particular business goal \cite{workflow95}. When such a process is managed electronically, it is called \textit{workflow}. The purpose of workflow is to streamline, coordinate and control business processes in an organised, distributed and computerised environment. The peer-review validation of an article in a scientific journal is a common example of business process. Descriptions of it can be found in \cite{peerReview02, van2001proclets, badouel14}. As in most literature works, most of the time, we will use the terms "business process" and "workflow" as synonyms in the rest of this manuscript.

The \textit{Workflow Management Coalition}\footnote{The growing reputation of workflow led to the creation, in 1993, of the \textit{Workflow Management Coalition} (WfMC) as the organisation responsible for developing standards in this field. Official website of the WfMC: \url{https://www.wfmc.org/}.} (WfMC) \cite{workflowModel} defines \textit{Workflow Management} (WfM) as the modelling and computer management of all the tasks and different actors involved in executing a business process. WfM is achieved using \textit{Workflow Management Systems} (WfMS): these are complex systems with the aim of automating at best workflows by providing an appropriate framework to facilitate collaboration between actors involved in business processes' execution \cite{workflow95, van2013business, van2015business, dumas2018fundamental}. WfMS are composed of logically orchestrated tools to specificy, to optimise, to automate and to monitor business processes \cite{workflowModel, dumas2018fundamental}. Technically, the management of a process according to WfM is done in two phases \cite{divitini2001inter}:
\begin{enumerate}
	\item the \textit{process modelling phase}: the process is studied and then specified using a language (usually graphical) called \textit{workflow language}. The resulting specification is called \textit{workflow model};
	\item the \textit{process instantiation and execution phase}: the workflow model is introduced into a WfMS which then instantiates and orchestrates the execution of the underlying process.
\end{enumerate}
Since WfMS are pre-engineered standalone systems, WfM simplifies business processes' automation to their specifications in \textit{workflow languages}.

WfM primarily focuses on business processes' automation. It is not fundamentally concerned with other issues such as the analysis, the verification and the management (maintenance) of workflow (models) unlike BPM, which made these its foundation \cite{van2016don}. BPM is the discipline that combines knowledge from information technology and knowledge from management sciences and applies this to operational business processes \cite{van2013business}. BPM can be seen as an extension of WfM as it primarily supports WfM and provide additional tools to improve business processes. For this, we have chosen to use the expression BPM rather than WfM (which tends to disappear) in the context of this work. It should be noted however that our contributions (chapter \ref{chap2:structured-editing-artifact-type} and \ref{chap3:choreography-workflow-design-execution}) could be perfectly presented as part of restricted WfM (we are not interested in the management and improvement of workflow models).



\mySubSubSection{An Introductive Example of Business Process}{}
\label{chap1:sec:running-example}
BPM is an important technology because it simplifies the automation of business processes which are the foundation of how companies and organisations operate. Business processes can be found everywhere. The examples are diverse and include the following:
\begin{itemize}
\item The design and development of a software by a team (especially when members are geographically dispersed) \cite{theseImine};
\item The simultaneous writing of a scientific paper or the documentation of a product by several researchers (cooperative editing) \cite{theseImine};
\item The follow-up of a medical file \cite{workflow07};
\item The student registration process in a faculty;
\item The withdrawal of a large sum of money from a bank teller;
\item The procedure for taking holidays in a government institution;
\item The procedure for claiming damages from an insurance company;
\item The peer-review process \cite{peerReview02, van2001proclets, badouel14}.
\end{itemize}

The peer-review process presents all the characteristics of the type of processes (\textit{administrative processes}) studied in this manuscript. Then, we will use it as an illustrative example along the whole manuscript (running example). Our description of this process is inspired by those made in \cite{peerReview02, van2001proclets, badouel14}: 
\begin{example}
	\textbf{The peer-review process (running example)}:\\
	The process is triggered when the editor in chief receives a paper for validation submitted by one of the authors who participated in its drafting.
	\begin{itemize}
		\item After receipt, the editor in chief performs a pre-validation after which, he can accept or reject the submission for various reasons (subject of minor interest, submission not within the journal scope, non-compliant format, etc.);
		\item If the submission is rejected, he writes a report then notifies the corresponding author and the process ends; in the other case, he chooses an associated editor and sends him the paper for the continuation of the validation; 
		\item The associated editor prepares the manuscript, forms a referees committee (two members in our case) and then triggers the peer-review evaluation process;
		\item Each referee reads, seriously evaluates the paper and sends back a message and a report to the associated editor;
		\item After receiving reports from all the referees, the associated editor takes a decision and informs the editor in chief who sends the final decision to the corresponding author.
	\end{itemize}
\end{example}

From this description, it is easy to identify all the tasks to be executed, their sequencing, actors involved and the tasks assigned to them. For this case, four actors are involved: an editor in chief ($EC$) who is responsible for initiating the process, an associated editor ($AE$) and two referees ($R1$ and $R2$).
A summary of tasks assignment is presented in table \ref{tableau:tachesExecutant}. We have associated symbols with tasks so that we can easily manipulate them in diagrams. 
\begin{table}[ht]
	\caption{Exhaustive tasks list of a paper validation process in a scientific journal and their respective performers.}
	\label{tableau:tachesExecutant}
	\begin{tabular}[t]{|m{8.4cm}|m{2.7cm}|m{2.63cm}|}
		\hline
		\textbf{Tasks} & \textbf{Associated Symbols}  & \textbf{Executors} \\
		\hline
		Receipt, pre-validation of a submitted paper and possible choice of an associated editor to lead peer-review evaluation & $A$  & $EC$\\
		\hline
		Drafting of a pre-validation report informing on the reasons for the immediate rejection of the paper & $B$ & $EC$ \\
		\hline
		Sending the final decision (acceptance or rejection of the paper) to the author & $D$ & $EC$ \\
		\hline
		Study, eventually formatting of the paper for the examination by a committee & $C$ & $AE$ \\
		\hline
		Constitution of the reading committee (selection of referees) and triggering the peer-review evaluation & $E$ & $AE$ \\
		\hline
		Decision making (paper accepted or rejected) from referees evaluations & $F$ & $AE$ \\
		\hline
		Evaluation of the manuscript by the first (resp. second) referee & $G1$ (resp. $G2$) & $R1$ (resp. $R2$) \\
		\hline
		Drafting of the after evaluation report by the first (resp. second) referee & $H1$ (resp. $H2$) & $R1$ (resp. $R2$) \\
		\hline
		Writing the message according to evaluation by the first (resp. second) referee & $I1$ (resp. $I2$) & $R1$ (resp. $R2$) \\
		\hline
	\end{tabular}
\end{table}




\mySubSubSection{Workflow Typology}{}
\label{chap1:sec:workflow-typology}
The authors of \cite{workflow95} conduct a very interesting study on the classification of workflows in which, they report the lack of a commonly accepted approach to categorising workflows. There are therefore several approaches to workflow classification in the literature.

The classification of workflows according to the nature and behaviour of automated processes is one of the most commonly found in the literature. According to it, workflows are divided into three groups: \textit{production} workflows, \textit{administrative} workflows and \textit{ad-hoc} workflows \cite{mcCready, van1998application}. Production workflows are those that automate highly structured processes that undergo very little (or no) change over time: all the scenarios are known in advance and most of the tasks are carried out by systems. This is the case for processes in industrial production lines. 
Administrative workflows apply to variable processes for which all cases are known; that is, tasks are predictable and their sequencing rules are simple and clearly defined. In these, changes are more frequent than with production workflows and human actors are more involved in the execution of tasks. In particular, this type of workflow brings considerable added value to public administration organisations whose business is focused on administrative routines \cite{boukhedouma2015adaptation}. Our running example, the peer-review process, is an administrative process. In the work presented in this manuscript, we are interested in this type of workflows.
These are opposite of ad-hoc workflows, which automate occasional processes for which it is not always possible to define the set of rules in advance. Processes are therefore only partially specified and may undergo many updates over time.

The workflows' classification made in \cite{workflow95} is orthogonal to the above-mentioned one (they can be used together); it is more concerned with tasks' automation degree. The authors classify workflows based on a measurement system, represented by a continuum ranging from \textit{human-oriented} workflows to \textit{system-oriented} workflows as shown in figure \ref{chap1:fig:dimitrios-classification}.
\begin{figure}[ht!]
	\noindent
	\makebox[\textwidth]{\includegraphics[scale=0.94]{./Chap1/images/dimitrios-classification.png}}
	\caption{Classification of workflows according to whether they are human-oriented or system-oriented (source \cite{workflow95}).}
	\label{chap1:fig:dimitrios-classification}
\end{figure}
The first type (human-oriented workflows) includes workflows in which humans collaborate to perform tasks and to coordinate themselves; in these, humans are responsible for ensuring the validity and consistency of the exchanged data and of the workflow's results. The second type of workflows (system-oriented workflows) refers to those in which the use of computer systems to perform tasks is unavoidable because, they involve complex data and computationally-intensive operations. According to this classification system, human-oriented workflows are the ones we are interested in.

In \cite{dumas2005process}, the authors refine the two above classification frameworks. Concerning the refinement of the one that classifies workflows according to the nature and behaviour of automated processes \cite{mcCready}, their classification framework distinguishes \textit{unframed}, \textit{ad hoc framed}, \textit{loosely framed}, and \textit{tightly framed} workflows. 
A workflow is said to be unframed if there is no explicit workflow model associated with it; its execution is strongly conducted by its actors. When actors play a crucial role (no longer limited to the simple execution of tasks, but also including the explicit choice of the control flow, the adjustment of control and data flows, etc.) in the execution of a workflow, it is said to be \textit{user-centric} \cite{badouel2015active}. This is the case for workflows being automated by groupware\footnote{Groupware systems are computer-based systems that support groups of people engaged in a common task (or goal) and that provide an interface to a shared environment \cite{ellis1991groupware}.}. 
In the case of ad hoc framed workflows, workflow models are defined a priori but, they frequently change. 
A workflow is said to be loosely framed when it is defined by a workflow model describing the "right way of doing things", while allowing its actual executions to deviate from this way; this is the preferred type of workflow handled by \textit{Case Management Systems} \cite{van2013business} (see sec. \ref{chap1:sec:gag}).
Finally, a tightly framed process is one which consistently follows a defined process model.

Concerning the classification framework of \cite{workflow95}, authors of \cite{dumas2005process} refine it and consider three types of workflows: \textit{Person-to-Person}, \textit{Person-to-Application}, and \textit{Application-to-Application} workflows. 
Person-to-Person workflows are those for whom all the tasks require human intervention. Application-to-Application workflows are their opposite; in these, all the tasks are executed by software systems. Person-to-Application workflows are in the middle; they involve both human-oriented tasks and system-oriented tasks. Pratically, most of workflows are of this category.

Nowadays, some scientific works require increasingly complex and data-intensive simulations and analysis. Scientific data management is therefore a major challenge \cite{bell2009beyond} with a high level of complexity. Workflow technologies are increasingly used to manage this complexity \cite{juveGideon}. These are responsible for scheduling computational tasks on distributed resources, managing dependencies between tasks and staging data sets in and out of runtime sites. The resulting workflows are called \textit{scientific workflows} and are usually based on a middleware infrastructure (\textit{Grid} or \textit{Cloud}). Ideally, the scientist should be able to integrate almost any scientific data resource into such a workflow during analysis, inspect and visualise the data on-the-fly as it is computed, make parameter changes as needed and re-run only the affected components, and capture sufficient metadata in the final products so that, scientific workflow executions help to explain the results and make them reproducible. Thus, a scientific workflow system becomes a scientific problem-solving environment, adapted to an increasingly distributed and service-oriented infrastructure (Grid or Cloud) \cite{ludascher2006scientific}.

There are many other types of workflows in the literature. We can mention on the fly, \textit{service-oriented} workflows \cite{piccinelli2003service, yongyi2009research}, \textit{structured} workflows \cite{kiepuszewski2000structured, eder2002meta, liu2005analysis}, etc. We do not present them here because they are not of great interest to the work we are doing for this thesis. We invite the interested reader to take a look at the few works mentioned above.


\mySubSection{Business Process Management Lifecycle and Key Activities}{}
\label{chap1:sec:bpm-key-activities-concerns}
A high-level view of the BPM discipline reveals that, its lifecycle consists of three phases on which it is possible to iterate indefinitely: the \textit{(re)design}, \textit{implement/configure}, and \textit{run \& adjust} phases \cite{van2013business} (see fig. \ref{chap1:fig:bpm-lifecycle}). 
During its lifecycle, four key activities namely \textit{model}, \textit{enact}, \textit{analyse}, and \textit{manage} (see fig. \ref{chap1:fig:bpm-key-concerns}) are carried out \cite{van2013business}. 
In this section, we examine what is done during these different activities; we mainly focus on the \textit{"model"} and the \textit{"enact"} activities: they are the only ones common to BPM and WfM and thus, they are of relevant interest for the work presented in this manuscript. 

\mySubSubSection{Business Process Management Lifecycle}{}
\label{chap1:sec:bpm-lifecycle}
\begin{figure}[ht!]
	\noindent
	\makebox[\textwidth]{\includegraphics[scale=0.5]{./Chap1/images/bpm-lifecycle.png}}
	\caption{The three phases of BPM's lifecycle (source \cite{van2013business}).}
	\label{chap1:fig:bpm-lifecycle}
\end{figure}

The automation of a given process using BPM starts with its modelling using one or more workflow languages \cite{dumas2018fundamental}. This \textit{"model" activity} is initiated during the \textit{"(re)design" phase} of the BPM lifecycle. The workflow models obtained during this activity can be analysed (the \textit{"analyse" activity}) either by simulations or by using model checking\footnote{Model checking is an automated technique that, given a finite-state model of a system and a formal property, systematically checks whether this property holds for (a given state in) that model \cite{baier2008principles}.} algorithms (to verify models' soundness): this type of analysis is said to be \textit{model-based}. 
As shown in figure \ref{chap1:fig:bpm-lifecycle}, the (re)design phase is followed by the \textit{"implement/configure" phase} in which, the workflow models obtained in the previous phase are converted, if necessary, into executable workflow models and then, used to configure the process execution environment (the WfMS): this is where the \textit{"model" activity} ends. 
After the \textit{"implement/configure" phase}, comes the \textit{run \& adjust phase}. During this last phase, the workflow is instantiated, executed and managed (adjusted) according to the scenarios foreseen when modelling the underlying process and when designing the host WfMS: these are the purposes of the \textit{"enact"} and \textit{"manage"} activities. Moreover, when a workflow instance is running, produced and logged data can be analysed (to discover possible bottlenecks, waste, and deviations) for possible improvement of its corresponding workflow model: this other type of analysis/monitoring is said to be \textit{data-based}; during the last decade, \textit{process mining} \cite{van2011process} has emerged as one of the leading techniques conducting \textit{data-based analysis}. If enough possible improvements to the workflow model are detected, the cycle can restart to apply them.
\begin{figure}[ht!]
	\noindent
	\makebox[\textwidth]{\includegraphics[scale=0.54]{./Chap1/images/bpm-key-concerns.png}}
	\caption{The four key activities of BPM (source \cite{van2013business}).}
	\label{chap1:fig:bpm-key-concerns}
\end{figure}



\mySubSubSection{The "Model" Activity}{}
\label{chap1:sec:bpm-model-activity}
\noindent\textbf{\textit{Basic concepts}}

Process modelling is a crucial activity in WfM/BPM. As mentioned above (sec. \ref{chap1:sec:bpm-lifecycle}), it is done using dedicated languages called \textit{workflow languages}. Several workflow languages have already been developed. Among the most well-known are the BPMN standard \cite{BPMN} based on statecharts, the UML activity diagrams language \cite{booch2000guide}, the WF-Net (\textit{Workflow Net}) language \cite{wil2003business} which uses a formalism derived from that of Petri nets, the YAWL language \cite{van2005yawl} which is an extension of WF-Net and so forth. 
Some of these languages (BPM, UML activity diagrams) are \textit{informal} (i.e. they do not have a well-defined semantics and do not allow for analysis \cite{zur2013much, van2013business}) while others (WF-Net, YAWL) are based on powerful mathematical (\textit{formal}) tools (Petri nets). Nevertheless, they all allow to express in a diagram (called a \textit{worklow model}), the tasks that make up a given process and the control flow between them. More precisely, workflow languages allow to describe the behaviour of processes through the representation (among others) \cite{grigori2001elements} of :
\begin{itemize}
	\item Tasks that make up the main part of the process;
	\item Information and resources relating to the various tasks;
	\item Sequences or dependencies between those tasks;
	\item Trigger and termination events for the tasks.
\end{itemize}

Tasks are the base of any workflow; a \textit{task} is the smallest unit of hierarchical decomposition of a process. A task represents any work that is performed within a process. It consumes time, one or more resources, requires one or more input objects and produces one or more output objects. You can find examples of tasks in our running example (sec. \ref{chap1:sec:running-example}). 

From a workflow point of view, the term \textit{resource} refers to a system or a human who can execute a task. It is also known as \textit{actor}, \textit{participant}, \textit{stakeholder}, \textit{agent} or \textit{user} depending on the context. Resources can be grouped according to various characteristics, to form either a \textit{role} or an \textit{organisational unit} \cite{grigori2001elements}. A \textit{role} is a group of resources with the same functional capabilities, while an \textit{organisational unit} is a set of resources (or class of resources) that belong to the same structure (department, team, service, cell, etc.).

~

\noindent\textbf{\textit{Routing patterns}}

To achieve its objectives, any workflow language must, for a given process, allow to express at least its tasks and their \textit{routing} (\textit{control flow}). The task control flow is commonly referred to as the \textit{lifecycle (process) model} of the process under study \cite{divitini2001inter, hull2009facilitating}. There are a number of routing patterns identified in the literature as basic ones: these are \textit{sequential}, \textit{parallel}, \textit{alternative} or \textit{conditional} and \textit{iterative} routings (see fig. \ref{chap1:fig:basic-routing}) \cite{van1998application}.
\begin{figure}[ht!]
	\noindent
	\makebox[\textwidth]{\includegraphics[scale=4.5]{./Chap1/images/basic-routing.png}}
	\caption{Four basic routing constructs (source \cite{van1998application}).}
	\label{chap1:fig:basic-routing}
\end{figure}
\begin{itemize}
	\item Sequential routing expresses the fact that tasks must be executed one after the other (task $A$ before tasks $B$ and $C$ in figure \ref{chap1:fig:basic-routing}(a));
	\item Parallel routing is used to specify the potentially concurrent execution of certain tasks. Tasks $B$ and $C$ in figure \ref{chap1:fig:basic-routing}(b) can be executed at the same time; in this case, tasks $A$ and $D$ are considered as \textit{gateways}: $A$ is said to be an \textit{AND-Split} gateway while $D$ is an \textit{AND-Join} gateway;
	\item With alternative routing, one can model a \textit{decision}: i.e. the choice to execute one task rather than another at a given time. In figure \ref{chap1:fig:basic-routing}(c), tasks $B$ and $C$ cannot be both executed; for this case, $A$ is called and \textit{OR-Split} gateway and $D$ is an \textit{OR-Join} gateway;
	\item In some cases, it is necessary to execute a task multiple times.  In figure \ref{chap1:fig:basic-routing}(d) task $B$ is executed one or more times.
\end{itemize}

The search for more advanced and expressive routing patterns has been the subject of many studies \cite{van2012workflow, borger2012approaches}. The interested reader is invited to consult the few references mentioned above to find out more. 

When a given workflow language only allows to specify the routing of the processes' tasks, when it is not interested in modelling the consumed and produced data during tasks execution, and when it only gives a secondary role to the processes' users, it is said to be \textit{process-centric}. This is the case for all the previously mentioned languages (BPMN, WF-Net, UML activity diagrams and YAWL). This type of workflow language is often referred to as "\textit{traditional workflow language}".

~

\noindent\textbf{\textit{Examples of workflow models}}

Figure \ref{chap1:fig:comparing-workflow-languages} shows the orchestration diagrams corresponding to the graphical description of the peer-review process (see its textual description in sec. \ref{chap1:sec:running-example}) using the process-centric notations BPMN and WF-Net. The graphical notations equivalent to sequential flow, \{And, Or\}-Splits and \{And, Or\}-Joins are well represented. Each diagram resumes the \textit{main scenarios} of this process.
\begin{figure}[ht!]
	\noindent
	\makebox[\textwidth]{\includegraphics[scale=0.2]{./Chap1/images/comparingWorkflowLanguages-peerReview.png}}
	\caption{Orchestration diagrams of the peer-review process.}
	\label{chap1:fig:comparing-workflow-languages}
\end{figure}


\mySubSubSection{The "Enact" Activity}{}
\label{chap1:sec:bpm-enact-activity}
\noindent\textbf{\textit{Overview}}

The "enact" activity takes as input, the workflow models (specifications) obtained during the model activity. If these models are executable (i.e. they have been coded in more technical languages taking into account implementation details) then they are directly introduced into a WfMS suitably installed at the different workflow execution sites; otherwise, they are first converted into executable models then, they are introduced into the WfMS. There are several languages for producing executable workflow models. These are usually proprietary and provided by WfMS designers. Of these languages, \textit{(Web Services) Business Process Execution Language} ((WS-)BPEL) is the standard\footnote{BPEL is standardised by the OASIS consortium. OASIS website: \url{https://www.oasis-open.org/}. BPEL Specification (PDF version): \url{https://docs.oasis-open.org/wsbpel/2.0/OS/wsbpel-v2.0-OS.pdf}.} and is well compatible with BPMN \cite{white2005using, ouyang2006bpmn, leymann2010bpel}.

Once the WfMS is properly configured using workflow models, it can create workflow instances and properly orchestrate their execution. To do this, WfMS must coordinate (according to workflow models) the execution of a set of tools and applications offering various services. In the 1990s, the WfMC developed and proposed an architectural \textit{reference model} for the implementation of WfMS \cite{workflowModel} (see fig. \ref{chap1:fig:wfms-reference-model}). The latter structures and describes precisely, the expected functionalities of a WfMS.

~

\noindent\textbf{\textit{The reference model}}

The WfMC reference model is a centralised architectural model in which the main component is called \textit{workflow enactment service}. The workflow enactment service is responsible for controlling the executions of workflow instances. It is composed of several \textit{workflow engines}. A given workflow engine handles some parts of workflows and also manages some of their resources \cite{van2013business, dumas2018fundamental}. 
\begin{figure}[ht!]
	\noindent
	\makebox[\textwidth]{\includegraphics[scale=0.6]{./Chap1/images/wfms-reference-model.png}}
	\caption{Reference model of the Workflow Management Coalition (source \cite{workflowModel}).}
	\label{chap1:fig:wfms-reference-model}
\end{figure}
According to the reference model, WfMS must provide tools to facilitate their configurations using workflow models: therein, these tools are referred to as \textit{process definition tools}. Process definition tools are connected to the WfMS core (the workflow enactment service) via \textit{Interface 1}. In order to execute tasks, users use \textit{workflow client applications} that communicate with the WfMS via \textit{Interface 2}. When necessary, a given workflow engine invokes other applications via \textit{Interface 3}. The \textit{administration and monitoring tools} connected via \textit{Interface 5}, are used to monitor and control the workflows. Finally, the WfMS can be connected to other WfMS using \textit{Interface 4}. A considerable effort has been made to standardise the five interfaces shown in figure \ref{chap1:fig:wfms-reference-model}. These efforts led to the production of languages (exchange formats) such as \textit{Workflow Process Definition Language} (WPDL), \textit{XML\footnote{XML: eXtensible Markup Language.} Process Definition Language} (XPDL) and BPEL.

The reference model has been very successful. Firstly, because to this day, it perfectly orchestrates the different tools used for the design and execution of workflows. Secondly, because it has served as the basic model for a very large number of WfMS in the industry. Examples include ActionWorkflow \cite{actionWorkflow}, FlowMark \cite{flowmark}, Staffware \cite{staffware}, InConcert \cite{inConcert}, etc. Because the reference model is a centralised approach (client-server architecture), it has the advantage of facilitating a good mastery of the technologies used in the production of WfMS. Also, the implementation of (generally lightweight) client applications and the overall maintenance of WfMS (which is limited to the maintenance of the central server) are much simpler \cite{theseKanzow}. However, systems based on a client-server architecture show some limitations because of the centralisation of workflow management. Their main weaknesses are: \textit{the (non) fault tolerance}, \textit{the (difficult) scalability} and the strong dependency of the system vis a vis the central server, which stores data, controls and thus, represents \textit{a possible point of congestion} \cite{junYan06, fakas04}. Concretely \cite{junYan06},
\begin{enumerate}
	\item The client-server architecture allows centralised coordination of workflows with little use of the computing potential on the client side. Workflow systems based on such an architecture are very cumbersome. In application areas where several workflow instances need to be executed in parallel, the centralised server can be overloaded with heavy computations and intensive communications when the system load increases, thus becoming a potential bottleneck. 
	\item Client-server systems are vulnerable to server failures. The centralised server is commonly viewed as a single point of congestion in the system. Its malfunction can cause the entire system to shut down. 
	\item The limited scalability of the client-server architecture prevents the WfMS based on it, from dealing with the ever-changing work environment. This also raises difficulties in system configuration, as any changes to the system, such as the admission of new actors, may require changes and updates to the centralised workflow server, which is very impractical and inefficient. Therefore, these WfMS are particularly unsuitable for application areas where workflow actors are required to join and leave frequently.
	\item An important and crucial element of any workflow system is to allow actors to maintain their autonomy and control. However, workflow actors in a client-server-based WfMS are exclusively controlled by centralised servers. A serious problem is that, a large number of actors working on the "lightweight client side" may not be able to exercise their control, decision-making and problem-solving capabilities.
\end{enumerate}

Knowing that various actors involved in a given business process are very often spread over remote sites, the reference model does not seem to be very suitable for efficiently implementing cooperation among them, as would systems based on a distributed architectural model be.
In order to meet the shortcomings of the reference model, several works \cite{theseKanzow, junYan06, fakas04, theseImine, SON} have focused on the production of distributed WfMS built on top of peer-to-peer (P2P) architectures. This approach has also been successful since, systems such as ADEPT \cite{adept} and METEOR$_{2}$ \cite{meteor} have been designed over years \cite{theseKanzow}.











