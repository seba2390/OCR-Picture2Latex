\mySection{Introduction}{}
\label{chap1:sec:introduction}
%\subsection*{Contexte et d�finitions}
%\label{sec:contexte}
The work we are doing in this thesis falls within the domain of Computer-Supported Cooperative Work (CSCW); it is a sub-domain of Software Engineering. Software engineering can be seen as a field of engineering that enables the design, the implementation and the maintenance of quality software systems \cite{barais2005construire}. 
Research in the field of software engineering has led to the implementation of new design and even programming paradigms (e.g. \textit{object-oriented programming} \cite{meyer2000conception}, \textit{component-based programming} \cite{heineman2001component}, etc.), new methods and processes (\textit{Object Modeling Technique - OMT -} \cite{rumbaugh1991object}, \textit{Unified Modeling Language - UML -} \cite{booch2000guide}, etc.), new technologies (\textit{\textbf{workflow}}, etc.), etc. The importance of software and the ever-increasing complexity of systems keep the field of software engineering among the most important in computer science.

Since the start of the 1980s, workflow technology knows an ever-growing success near companies and researchers in the field of computer-aided production. 
This success can be justified by the fact that, workflow enables firms to reduce their production costs as well as to quickly and easily develop new products and services: their competitiveness is therefore increased. 
Workflow technology offers methods and tools (notations, management systems, etc.) for the specification, optimisation, automation and monitoring of business processes \cite{workflow95, van2015business}.   
Workflow technology tools are logically orchestrated within complex systems called \textit{Workflow Management Systems} (WfMS) \cite{workflowModel, ima}. 
The purpose of WfMS is not only to automate at best workflows, but also to provide an appropriate framework for facilitating collaboration between actors involved in the execution of a given business process. 

The search for Workflow Management (WfM) / Business Process Management\footnote{BPM can be considered as an extension of classical Workflow Management (WfM) systems and approaches \cite{van2015business}.} (BPM) techniques has been densely conducted over the past two decades and a clear interest has been given to the \textit{artifact-centric} paradigm \cite{nigam2003business} proposed by International Business Machines Corporation (IBM). This one, revisited in several works \cite{abi2016towards, deutsch2014automatic, hull2009facilitating, lohmann2010artifact, assaf2017continuous, assaf2018generating, boaz2013bizartifact, lohmann2011artifact, estanol2012artifact}, proposes a new approach to BPM by focusing on both automated processes (tasks and their sequencing) and data manipulated through the concept of "\textit{business artifact}" (\textit{artifact-centric modelling}). 
%This contrasts with traditional BPM approaches (BPMN - \textit{Business Process Model and Notation}\footnote{BPMN was initiated by the \textit{Business Process Management Initiative} (BPMI) which merged with \textit{Object Management Group} (OMG) in 2005.} \cite{BPMN} -, YAWL - \textit{Yet Another Workflow Language}\footnote{YAWL allows processes to be represented using an extension of WF-Net (workflow net), a formalism derived from that of \textit{Petri Nets}.} \cite{van1998application, van2005yawl} -, BPEL - \textit{Business Process Execution Language}\footnote{BPEL allows to formalise the behaviour of business processes by choreographing web services.} \cite{jordan2007web} -) which are mainly concerned only with task scheduling and messages to be exchanged \cite{van2013business} thus, treating data as second-class citizens.

In this chapter which serves as a state of the art, we present some key notions related to BPM in general and to the artifact-centric paradigm in particular, in order to make it easier to better apprehend the concepts handled in this manuscript. 
In Section \ref{chap1:sec:bpm-def-key-principles}, we define some basic concepts such as \textit{business process}, \textit{workflow}, \textit{workflow management}, etc., then we present some workflow classification approaches as well as an overview of their standardised automation using BPM. 
In section \ref{chap1:sec:p2p-bpm}, we conduct a review of Peer to Peer (P2P) approaches to BPM. 
%we briefly present business process modelling languages such as BPMN, WF-Net (Workflow Net), YAWL, etc., and centralized WfMS such as ActionWorkflow \cite{actionWorkflow}, FlowMark \cite{flowmark}, Staffware \cite{staffware}, InConcert \cite{inConcert} etc. 
In section \ref{chap1:sec:data-aware-bpm}, we focus on the artifact-centric paradigm to BPM; we present the two general approaches to its implementation (\textit{orchestration} and \textit{choreography}) as well as some frameworks implementing it from the literature. 
Section \ref{chap1:sec:summary} is dedicated to a summary of the explored concepts and a smooth transition to the next chapter.

