%\myChapterStar{Titre}{Titre court}{Ajouter � la table des mati�res? (false|true|chapter|section|subsection|subsubsection -chapter par d�faut-)}
\myChapterStar{General Conclusion}{}{true}
\label{chap5:general-conclusion}
\myMiniToc{section}{Contents}
% If no minitoc then
% \startcontents[chapters]

We hereby summarise the work reported and presented in this document. To this, we associate a critical analysis of our models and methodological choices. Finally, we present some research avenues that can be explored following the work in this thesis manuscript.

\mySectionStar{Recall of this Thesis' Challenge and of our Methodological Choices}{}{true}
In this thesis, we focused on the automation of business processes using the technological framework offered by the BPM domain. 
We have contributed to the ambition of making more accessible, administrative business processes automation through this technology. We were guided by the aim of increasing its success in business sectors using administrative processes as it has been in other sectors such as science, banking and insurance, which are governed by much more programmable processes. 
We thought that a first step towards achieving this great ambition was to rely on the current (most up-to-date) BPM paradigms and tools, to design and implement a new BPM framework that would be tailor-made for the management of administrative business processes. 
Having identified the artifact-centric BPM, structured cooperative edition, P2P computing, multiagent system and SOA concepts as being the hot topics in the implementation of BPM, we set the following goal for this thesis:
\begin{displayquote}
	\textit{The proposal of a new artifact-centric framework, facilitating the modelling of administrative business processes and the completely decentralised execution of the resulting workflows; this completely decentralised execution being provided by a P2P system conceived as a set of agents communicating asynchronously by service invocation so that, the execution of a given workflow instance is technically assimilated to the cooperative edition of (mobile) structured documents called artifacts.}
\end{displayquote}

To achieve this goal, we have chosen to base our work on the structured document cooperative editing model developed by Badouel and Tchoup\'e a decade earlier. They proposed an approach based on grammatical models, to manage the lifecycle of a document collegially edited by actors on geographically distant sites. In their model, each actor has a potentially partial view (obtained by projection) of the edited document. The latter is used as an interface between the different actors of the system. When an actor receives a document he must know from its content, what he can do and/or what he has to do about it. The information contained in a document can only grow over its lifetime in the system. Since the system is distributed and under the assumption that several actors can contribute concurrently, it is possible that at any given time there may be several potentially partial replicas of the document in the system. Therefore, it was necessary not only to address the problems related to the coherence of views in order to ensure the feasibility of synchronisation/merge, and to ensure the system's convergence towards a coherent end-state, but also to provide algorithms for the merging of partial replicas. Badouel and Tchoup\'e did this brilliantly, making their work a solid foundation for modelling workflow systems.

We adapted Badouel and Tchoup\'e's document model to define an artifact model. Like them, we have therefore chosen to use grammars as our basic mathematical instrument. In the same vein, we made the choice to model workflow systems in which actors have potentially only a partial perception of the processes they execute. We believe that this configuration is relevant to many administrative business processes. For example, in a peer-review process, a reviewer does not necessarily need to know if another reviewer has been contacted for the expertise of the article entrusted to him; and even if so, he should not necessarily know if the latter has already returned his report, etc. 
Similarly, when organising a journey for a Head of State, not all actors (secret services, civil office, doctor, presidential guard, etc.) have access to the same information which may include for example, tasks to be executed, their dates and states of execution, etc.



\mySectionStar{A Critical Analysis of the Performed Work}{}{true}
Chronologically, we started by better understanding Badouel and Tchoup\'e's model in order to extend it so that it takes into account, conflicts detection and resolution. We then embarked on the construction of a workflow system based on this model by first proposing a generic architecture of such systems and an experimental system prototype based on it. We finished by proposing an artifact-centric framework for the completely decentralised management of administrative business processes. The results we can claim are the following:

~

\noindent\textbf{An algorithm for reconciling partial replicas of a structured document}: we gave a solution to the conflicts that arose when merging partial replicas of a structured document by developing reconciliation and control techniques adapted to the modelling of Badouel and Tchoup\'e. To address this problem, we have expressed it as that of the consensual merging of $k$ updated partial replica $(t_{\mathcal{V}_i}^{maj})_{1 \leq i\leq k}$ (according to $k$ views $(\mathcal{V}_i)_{1 \leq i\leq k}$) whose global model is given by a grammar $\mathbb{G}=\left(\mathcal{S},\mathcal{P},A\right)$, which consists in: finding the largest documents $t^{maj}_{\mathcal{S}}$ conforming to $\mathbb{G}$ such that, for any document $t$ conforming to $\mathbb{G}$ and admitting $t_{\mathcal{V}_i}^{maj}$ as projection along the view ${\mathcal{V}_i}$, $t^{maj}_{\mathcal{S}}$ and $t$ are eventually updates each for other. 
The solution we have proposed is as follows: (1) We associate a tree automaton with \textit{exit states} $\mathcal{A}^{(i)}$ for each update $t_{\mathcal{V}_i}^{maj}$ of a partial replica $t_{\mathcal{V}_i}$; this automaton recognises the trees (conform to the global model) for which $t_{\mathcal{V}_i}^{maj}$ is a projection. (2) We perform a \textit{synchronous product} of the automata $\mathcal{A}^{(i)}$ with a commutative and associative operator noted $\otimes$ that we define to obtain the consensual automaton $\mathcal{A}_{(sc)}$ generating the consensus documents: $\mathcal{A}_{(sc)}=\otimes\mathcal{A}^{(i)}$. (3) We obtain the consensus documents by generating the set of trees accepted by the automaton $\mathcal{A}_{(sc)}$.



~

\noindent\textbf{A software architecture for the implementation of workflow systems}: we proposed a generic architecture that could facilitate the implementation of workflow systems as modelled by Badouel and Tchoup\'e. The proposed architecture is composed of three tiers: \textit{clients}, a \textit{central server} and \textit{administration tools}. These three tiers are interconnected around a middleware that facilitates service-oriented interfacing between them.



~

\noindent\textbf{TinyCE v2}: based on the proposed system architecture, we have built a workflow system prototype referred to as TinyCE v2. The latter was coded in Java and Haskell following a cross-fertilisation protocol that we presented in this manuscript, and it allowed us to test all the proposed algorithms related to the reconciliation of documents' partial replicas.



~

\noindent\textbf{A grammar-based language for the artifact-centric modelling of administrative processes}: we proposed a new tool (a language) that allows to specify any administrative business process using a triplet $\mathbb{W}_f=\left(\mathbb{G}, \mathcal{L}_{P_k}, \mathcal{L}_{\mathcal{A}_k} \right)$ composed of: 
a grammatical model (GMWf) $\mathbb{G}$, a list of actors $\mathcal{L}_{P_k}$ and a list of accreditations $\mathcal{L}_{\mathcal{A}_k}$. 
The GMWf is an attributed grammar used to describe all the tasks (by means of its symbols or sorts) of the studied process and the precedence of execution between them (by means of its productions); it is used as artifact type.
The list of accreditations provides information on the role played by each actor involved in the process execution; it is through accreditations that one is able to model the potentially partial perceptions that different actors have on the processes and their data.


~

\noindent\textbf{A distributed workflow system and a fully decentralised execution model of administrative business processes}: we proposed a multiagent-like distributed system in which autonomous software agents, based on a same and unique architecture presented in this manuscript, can exchange artifacts to communicate through service invocations, so as to orchestrate the fully decentralised execution of a given administrative business process instance modelled using the proposed grammar-based language. Each time a given agent receives an artifact, it executes a unique five-step protocol allowing it to identify tasks that are ready to be executed on its site, to allow their execution by the local actor and to diffuse if necessary, the updated artifact. To ensure the successful completion of this execution model on a given process case, the initial configuration of agents must be such as to guarantee the coherence of their respective accreditations (views). We have therefore proposed projection algorithms to derive agent-specific models that allow them to control their actions in order to ensure the system's convergence towards a business goal state. We have also investigated some mathematical properties of these algorithms. 


~

\noindent\textbf{P2PTinyWfMS}: we have finally produced a prototype of a distributed system providing an artifact-centric management of administrative workflows according to our models. In this one, we have implemented all our algorithms in Java language and we have tested them on some process examples with convincing results.


~


The work we have done and presented in this manuscript is not perfect: neither in the applied methodology, nor in its presentation, and even less in its scientific contributions. The first criticisms that we can make of it are the following:

~

\noindent\textbf{The diversity of our algorithms' presentation formats}: we didn't just use pseudo code to present our algorithms. Sometimes we used code (Haskell and Java) and other times we wrote them as an arbitrary and ordered set of instructions written in natural language; in these cases we have neglected the more frequently used and more precise mathematical notations. This methodological choice of presentation can indeed be confusing for the reader. We justify it, however, by our desire to be precise, concise and as simple to understand as possible. We have written each of our algorithms in all the formats used in this manuscript before selecting for each of them, the format that seemed to us the clearest and simplest to present and understand.


~

\noindent\textbf{Conflict management}: we have chosen to use a single conflict resolution strategy: that of rejecting conflicting contributions and asking for new ones from contributors. This seems to us rather restrictive but it was a necessary (not necessarily wise) choice for a complete automation of the process of merging partial replicas of a structured document. However, in practice, it would be more appropriate to propose, following the example of Git, several conflict management strategies using a participant as an actor (coordinator) of this resolution.



~

\noindent\textbf{Insistence on manipulating user views}: although we have already justified the choice to take user views into consideration in our models by explaining their contribution to both security and accuracy, it is no less true that they do not always have a positive impact on our work. They have made it a little more complex and have led us to make slightly restrictive assumptions such as the non-recursive GWMf assumption that guarantees their projection. We could make the use of these views more flexible by restricting it to only censorship and not to the complete deletion of sensitive data; this would certainly allow us to overcome some assumptions.




~

\noindent\textbf{The weak study of our execution model's properties}: apart from the isolated study of the properties of a few of its algorithms, we did not study some properties of our decentralised workflow execution model as it is usual in similar BPM studies. We probably pay the consequences of our not completely formal and uncommon (but specific to artifact-centric models, it is one of their often mentioned limits) presentation that mixes the artifact modelling with its execution. A clear separation of these two aspects would certainly allow us to better study them in an isolated and more conventional way.



~

\noindent\textbf{Still as theoretical as ever}: obviously, we have the ambition to produce concrete systems that can be used in production environments. We are still a long way from that. For the two types of systems studied in this thesis, we have only produced prototypes that allow us to provide experimental proof of concepts. Theoretical studies on the Badouel and Tchoup\'e model being already quite advanced, it would be time to start implementing these concrete environments.



\mySectionStar{Some Perspectives}{}{true}
The perspectives presented here are classified into categories according to their priority, to better guide the reader wishing to continue the work. The categories includes :
\begin{itemize}
	\item \textbf{Short-term}: to indicate that the perspective is almost unavoidable and its results will be a real plus to the overall vision we have; it is therefore necessary to work on it as soon as possible;
	\item \textbf{Mid-term}: to indicate that the perspective's results will be pratical and usable but not a necessity;
	\item \textbf{Long-term}: to indicate that the perspective is optional and its theoretical results would only help give credibility to our work.
\end{itemize}
Here are now some interesting avenues for the continuation of our work that come to mind:

~

\noindent\textbf{A Language for the Specification of Administrative Workflow based on Attributed Grammars (LSAWfP) (priority: short-term)}: it is obvious that process modelling is a crucial phase of BPM \cite{dumas2018fundamental}. Despite the many efforts made in producing process modelling tools, existing tools (languages) are not commonly accepted. They are mainly criticised for their inability to specify both the tasks making up the processes and their scheduling (their lifecycle models), the data they manipulate (their information models) and their organisational models. Process modelling in these languages often results in a single task graph; such a graph can quickly become difficult to read and maintain. Moreover, these languages are often too general (they have a very high expressiveness); this makes their application to specific types of processes complex: especially for administrative processes. 
One can generalise the artifact specification model presented in this thesis, in order to provide a new language for administrative processes modelling that allows designers to specify the lifecycle, information and organisational models of such processes using a mathematical tool based on a variant of attributed grammars. Therefore, the approach imposed by the new language will certainly require the designer to subdivide his process into scenarios, then to model each scenario individually using a simple task graph (an annotated tree) from which a grammatical model will be further derived. At each moment then, the designer will manipulate only a scenario of the studied process: this seems more intuitive and modular because it will allow to produce task graphs that will be more refined and therefore, more readable and easier to maintain.


~

\noindent\textbf{A Scenario-Oriented Scheme for Administrative Business Processes Modelling (priority: mid-term)}: in the BPM community, researchers and professionals in the field have little interest in the "how" to model business processes to the benefit of the "with what" to model them. As a result, there is a plethora of workflow modelling languages but very few methods \cite{dumas2018fundamental}. The question on the method to be used to successfully model a given process is however crucial when we know that BPM reduces the automation of the said process to its specification in a particular workflow language: a well carried-out specification produces a quality workflow system. Because the modelling language that can be extracted from the work of this thesis seems to be adapted to a process modelling philosophy centered on the notion of scenario, it would be interesting to propose a method that would accompany it. This one would present the steps to be followed to succeed in its scenario-oriented modelling of administrative workflow systems.



~

\noindent\textbf{Verification of workflows specified using our models (priority: short-term)}: one of the BPM activities is the formal analysis/verification of the specifications produced using a given workflow language. Proposing and/or adapting a verification method for workflows designed using our models seems to be an interesting avenue of research especially since several similar works have done the same \cite{badouel2015active, van1997verification, van2000workflow}. To this end, it will certainly be necessary to deeply study our models in order to highlight their mathematical properties. These properties will then make it possible to identify and present the criteria that must be verified by a specification in order to be qualified as correct (sound).




~

\noindent\textbf{A tool to help in the specification of administrative processes with a scenario-oriented approach (priority: short-term)}: since the models we have proposed are new, it would be wise to propose a tool to assist in their use in practice. In addition to being a guide, such a tool should simplify the creation of process models and possibly validate them according to pre-established correction criteria. Moreover, it will be able to provide several DSL for saving the specifications as well as several modules for exporting them in more conventional notations (BPMN, YAWL, etc.).




~

\noindent\textbf{A framework to generate administrative business processes' specific decentralised execution simulators (priority: mid-term)}: as implemented, the WfMS prototype P2PTinyWfMS can be used as a foundation for the production of a tool that generates simulators of the completely decentralised execution of administrative processes specified using our models. The new framework can be based on the models found in \cite{tchembe2019ad}; then, it will be implemented to generate a simulation environment tailored to a given administrative process. Still in a generative logic, another approach would be to use the recent concepts of model-driven engineering to produce a simulator based on integrated development environments such as Eclipse: one could for example use a GEMOC approach \cite{bousse2016execution, combemale2017language}.



~

\noindent\textbf{Extension of our decentralised execution model for recursive GMWf and monitoring support (priority: short-term)}: we already mentioned this in the discussion section of chapter \ref{chap3:choreography-workflow-design-execution}. It would be interesting to introduce into our multiagent system, new types of agents to monitor the execution of processes. In addition, we could also make the use of views more flexible by modifying their impact on projection operations. For example, we could redefine the artifact projection operation so that it no longer erases nodes but censors them; that is, it replaces them with symbols that help in the specification of control flows (restructuring symbols for instance) carrying no information on the process. This will preserve the possibility of offering only a potentially partial view of the processes and their data to actors while allowing the use of recursive symbols in GMWf.



~

\noindent\textbf{Concrete implementation of WfMS supporting our models (priority: short-term)}: eventually, it will also be necessary to propose implementations of the WfMS presented here. Naturally, the implemented system will have to cover all phases and activities of the BPM lifecycle while focusing on the automation of administrative business processes. A study of recent BPM systems to ensure interoperability and an openness of the implemented system on the cloud will certainly be interesting avenues to explore.



~

\noindent\textbf{Study of each proposed formal tool's properties in order to identify more precisely the class of workflows to which they apply (priority: long-term)}: another, more theoretical, line of research would be to formally analyse the proposed BPM approach to characterise the class of workflows to which it can actually be applied. We assumed that we were only interested in administrative workflows; however, the proposed models are quite general and could well be applied to other classes of workflows. The study carried out in this perspective should therefore determine these classes and the conditions under which the proposed model automates them.



\myCleanStarChapterEnd

