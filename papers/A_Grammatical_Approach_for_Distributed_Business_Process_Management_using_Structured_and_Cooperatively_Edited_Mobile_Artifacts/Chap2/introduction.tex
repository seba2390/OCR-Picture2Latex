\mySection{Introduction}{}
\label{chap2:sec:introduction}
The purpose of this chapter is to introduce the main concepts related to asynchronous cooperative editing of structured documents. In an asynchronous cooperative editing workflow, several authors located on geographically distant sites coordinate to edit asynchronously the same structured document. In such editing workflows (see fig. \ref{chap2:fig:badouel-tchoupe-workflow}), the desynchronised editing phases in which each co-author edits on his site, his copy of the document, alternate with the synchronisation-redistribution phases in which, the different contributions (local replicas) are merged (on a dedicated site) into a single document, which is then redistributed to the various co-authors for the continuation of the edition. This pattern is repeated until the document is completely edited.

Badouel and Tchoup\'e \citeyearpar{badouelTchoupeCmcs} have theorised an asynchronous cooperative editing workflow in which, stakeholders (several subsystems - sites - distributed across a network) work by editing and exchanging (partial) replicas of documents representing their perceptions (views) at any given time. Therefore, each subsystem (actor) has a partial view of the edited document at any given time, and the current (global) document is given by the merging of different (partial) documents from the various subsystems. In their model, collaborations between actors can be divided into three sequential phases (see fig. \ref{chap2:fig:badouel-tchoupe-workflow}): 
\begin{itemize}
	\item The \textit{distribution phase} where global structured document (an artifact) is replicated to each subsystem;
	\item The \textit{editing phase} in which local processes of subsystems are executed, inducing an update of the local replica of the global document; 
	\item The \textit{synchronisation phase} in which the various local documents updated are merged into a global document.
\end{itemize}

In this chapter, we are mainly interested in the model proposed by Badouel and Tchoup\'e. However, we are not just doing a systematic review of literature of their model. We subtly introduce three contributions which further validate their model, and polish the path towards our goal of producing a completely decentralised model for the automation of administrative workflows :
\begin{enumerate}
	\item First of all, we extend the merge algorithm proposed by them so that, it can be applied in the more general case where conflicts might appear. To this end, we propose a consensus reconciliation algorithm that generates conflict-free maximum prefixes of the documents resulting from the merging of several conflicting replicates.
	\item Second, we propose a generic system architecture that can be used to produce workflow systems for the cooperative editing of structured documents based on their model.
	\item Finally, we propose a cooperative editing system prototype based on the proposed architecture and coded by cross-fertilisation of Java and Haskell.
\end{enumerate}

In the rest of this chapter, in section \ref{chap2:sec:cooperative-editing-concepts}, we will present some basic concepts related to cooperative editing of documents, as well as some existing cooperative editing systems. In section \ref{chap2:sec:grammatical-cooperative-editing}, we will present the main concepts of Badouel and Tchoup\'e's model as well as the reconciliation algorithm that we propose. In section \ref{chap2:sec:architecture-cooperative-editing}, will be presented, the generic architecture of workflow systems that we propose, as well as a prototype system built according to it. We will conclude this chapter in section \ref{chap2:sec:conclusion}.
