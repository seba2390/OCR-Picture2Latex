\mySection{Summary}{}
\label{chap2:sec:conclusion}
This chapter was devoted to the study of asynchronous cooperative editing concepts in general, and to the study of notions related to the model introduced by Badouel and Tchoup\'e for structured documents cooperative editing \cite{badouelTchoupeCmcs}. In order to be more efficient, we have chosen to study these models with new contributions, including: an algorithm for reconciling potentially conflicting partial replicas of a structured document, and a generic architecture for designing workflow systems that can be modelled as structured cooperative editing systems in the sense of Badouel and Tchoup\'e. The correction of the proposed algorithms has been demonstrated. Implementations of these have been made, notably in TinyCE v2, the cooperative editor prototype implemented in Java and Haskell according to the new architecture proposed in this chapter.

The aim of these studies was to familiarise us with some key mathematical tools, in particular: grammars, views, projection/replication algorithms, fusion/reconciliation algorithms, etc. These mathematical tools are proving to be effective in the implementation of artifact-centric BPM models as demonstrated by Badouel et al. with the AWGAG model \cite{badouel14, badouel2015active}. For this purpose, they will form the foundation of the new artifact-centric model for the completely decentralised design and execution of administrative workflows (assimilated to the asynchronous cooperative edition of mobile structured documents) on a SOA that we propose in the next chapter.