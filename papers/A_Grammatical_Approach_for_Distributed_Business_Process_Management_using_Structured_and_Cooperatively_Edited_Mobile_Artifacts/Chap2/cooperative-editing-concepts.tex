\mySection{Basic Concepts on Cooperative Editing Workflows}{}
\label{chap2:sec:cooperative-editing-concepts}
Cooperative editing is a work of (hierarchically) organised groups, that operate according to a schedule involving delays and a division of labor (coordination). 
Like any CSCW, cooperative editing is subject to spatial and temporal constraints. Thus, one distinguishes distributed or not, and synchronous or asynchronous cooperative editing. When distributed, the various editing sites are geographically dispersed and each of them has a local copy of the document to be edited; systems that support such an edition should offer algorithms for data replication \cite{Yasushi2005} and for the fusion of updates. When asynchronous, various co-authors get involved at different times to bring their different contributions.

A cooperative editing workflow goes generally, from the creation of the document to edit, to the production of the final document through the alternation and repetition of distribution, editing and synchronisation phases. The literature is full of several cooperative editing workflows and of their management systems. We present a few in this section.


\mySubSection{Real-Time Cooperative Editing Workflows}{}
\label{chap2:sec:real-time-cooperative-editing}
In these generally centralised systems (Etherpad\footnote{Official website of Etherpad: \url{http://www.etherpad.org/}, visited the 04/04/2020.} \cite{epad}, Google Docs\footnote{Google Docs is accessible online at \url{https://www.docs.google.com/}, visited the 04/04/2020.}, Framapad\footnote{Get more information on Framapad at \url{http://www.framasoft.org/}, visited the 04/04/2020.}, Fidus Writer\footnote{Official website of Fidus Writer: \url{https://www.fiduswriter.org/}, visited the 04/04/2020.} \cite{fiduswriter}, etc.), the original document is created by a co-author on the central server. The latter then invites his colleagues to join him for the editing; they therefore connect to the editing session usually identified by a URL (distribution phase, although the document is generally not really duplicated). During an editing session (synchronous editing phase), all connected co-authors work on a single copy of the document but in different contexts. When the integration is automatic, changes performed by one of them are immediately (automatically) propagated to be incorporated into the basic document (synchronisation phase), and the latter is then redistributed to others. The changes are saved progressively and the server usually keeps multiple versions of the document.

The majority of real-time editors use the model of operational transformations \cite{theseOster, theseMounir}. Their architectures are therefore based on the one defined by this model. Meaning that, they distinguish two main components: an \textit{integration algorithm}, responsible for the receipt, dissemination and execution of operations and a \textit{set of processing functions} that are responsible for "merging" updates by serialising two concurrent operations.

\mySubSection{Asynchronous Cooperative Editing Workflows}{}
\label{chap2:sec:async-cooperative-editing}
This edit mode is distinguished by real distribution phases in which, the document to be edited is replicated on different sites, using appropriate algorithms \cite{Yasushi2005}. A co-author may then contribute at any time, by editing his local copy of the document. Here, we focus on a few asynchronous cooperative editors operating in client-server mode.

~

\noindent\textbf{\textit{Wikiwikiweb (Wikis)}}

Wikis \cite{wikiwikiweb} are a family of collaborative editors for editing web pages from a browser. To edit a page on a Wiki, one must duplicate it and contribute. After editing, he just have to save it and to publish a new version of that page. In a competing editing case, it is the last published version which will be visible. Even though it is still possible to access the previously published versions, there is no guarantee that a new version of the page preserves intentions (incorporates aspects) of previous versions. For this aspect, a Wiki can be seen much more as a web page version manager.

~

\noindent\textbf{\textit{CVS (Concurrent Versions System)}}

Under CVS \cite{cvs}, versions of a document are managed in a space called repository, and each user has a personal workspace.
To edit a document, the user must create a replica in his workspace. He will amend this replica, then will release a new version of the document in the repository. In case the document is concurrently edited by several authors and at least one update has already been published, the author wishing to publish a new update, will be forced to consult and integrate all previous updates through  dedicated tools, integrated in CVS.

~

\Needspace{5\baselineskip}
\noindent\textbf{\textit{SVN (Subversion)}}

SVN\footnote{Check more about SVN at \url{http://www.subversion.apache.org/}, visited the 04/04/2020.} \cite{svn} was created to replace CVS. Its main goal was to propose a better implementation of CVS. So as CVS, SVN relies on an optimistic protocol of concurrent access management: the \textit{copy-edit-merge} paradigm. SVN provides many technical changes like a new commit algorithm, the management of metadata versions, new user commands and many others features.

~

\noindent\textbf{\textit{Git}}

The main purpose of Git\footnote{Official website of Git: \url{https://www.git-scm.com/}, visited the 04/04/2020.} is the management of various files in a content tree considered as a deposit (all files of a source code for example). To edit a deposit, a given user connects to it and clones (forks). He obtains a copy of that deposit, modifies it locally through a set of commands provided by Git. Then he offers his contribution to primary maintainer which can validate it and thus, merges it with the original deposit. During this operation, new versions of modified files are created in the main repository. It is therefore possible under Git, to access any revision of a given file. 



\mySubSection{Badouel and Tchoup\'e's Cooperative Editing Workflow}{}
\label{chap2:sec:badouel-tchoupe-cooperative-editing}
Badouel and Tchoup\'e \citeyearpar{badouelTchoupeCmcs} proposed a workflow for cooperative editing of structured documents (those with regular structures defined by grammatical models such as DTD, XML schema \cite{xml2000}, etc.), based on the concept of "view". The authors use context free grammars as documents models. A document is thus, a derivation tree for a given grammar.

The lifecycle of a document in their workflow can be sketched as follows: initially, the document to edit ($t$) is in a specific state (initial state); various co-authors who are potentially located in distant geographical sites, get a copy of $t$ which they locally edit. For several reasons (confidentiality, security, efficiency, etc. \cite{tchoupeAtemkeng2}), a given co-author "$i$" does not necessarily have access to all the grammatical symbols that appear in the tree (document);  only a subset of them can be considered relevant for him: that is his \textit{view} ($\mathcal{V}_i$). The locally edited document, is therefore a \textit{partial replica} (denoted $t_{\mathcal{V}_i}$) of the original document. This one is obtained by \textit{projection} ($\pi$) of the original document with regard to the view of the considered co-author ($t_{\mathcal{V}_i}=\pi_{\mathcal{V}_i}(t)$). The edition is asynchronous and local documents obtained are called updated partial replicas denoted by $t_{\mathcal{V}_i}^{maj}$.

Badouel and Tchoup\'e focus only on the positive edition: edited documents are only increasing; thus, the co-authors cannot remove portions of the document when a synchronisation has already been performed. To both ensure that property, and be able to tell a co-author where he shall contribute, the documents being edited are represented by trees with \textit{buds} that indicate the only places where editions are possible.
Buds are typed; a \textit{bud of sort $X$} is a leaf node labelled $X_\omega$: it can only be edited (extended in a subtree) by using a \textit{$X$-production} (production with $X$ as left hand side).

When a synchronisation point is reached, all contributions $t_{\mathcal{V}_i}^{maj}$ of different co-authors are merged in a single global document $t_f$. To ensure that the merging is always possible (convergence), Badouel and Tchoup\'e assume that on each site, the editions are controlled by a local grammar. These local grammars are obtained from the global one, by projection along the corresponding views \cite{badouelTchoupeCmcs, tchoupeAtemkeng2}.
\begin{figure}[ht!]
	\noindent
	\makebox[\textwidth]{\includegraphics[scale=0.65]{Chap2/images/editionBPMNEn.png}}
	\caption{A BPMN orchestration diagram sketching a cooperative editing workflow of a structured document according to Badouel and Tchoup\'e.}
	\label{chap2:fig:badouel-tchoupe-workflow}
\end{figure}

Figure \ref{chap2:fig:badouel-tchoupe-workflow} gives an overview, with a BPMN orchestration diagram, of the structured documents' cooperative editing workflow according to Badouel and Tchoup\'e's proposal; at site 1, operations of (re)distribution and merging of the document in accordance with a (global) model $G$, are realised; at sites 2 and 3, edition of partial replicas in accordance with (local) models $G_1$ and $G_2$, derived by projecting the global documents model $G$, are done.

In summary, the workflow of Badouel and Tchoup\'e is different from the others with its concept of "view" and by the fact that, it exclusively manipulates (partial) structured documents.



