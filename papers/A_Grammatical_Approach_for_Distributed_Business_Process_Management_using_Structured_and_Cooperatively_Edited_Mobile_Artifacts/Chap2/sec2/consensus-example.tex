
~

\noindent\textbf{\textit{Linearisation of a structured document}}

To simplify the presentation, we represent in the following, trees by their linearisation in the form of a Dyck word. To do this, we associate a (various) pair of brackets to each grammatical symbol and the linearisation of a tree is obtained by performing a Depth First Search (DFS) of the resulting tree (see fig. \ref{chap2:fig:tree-linearisation}).
\begin{figure}[ht!]
	\noindent
	\makebox[\textwidth]{\includegraphics[scale=0.6]{Chap2/images/linearisation.png}}
	\caption{Linearisation of a tree ($tv1$): the Dyck symbols '$($' and '$)$' (resp. '$[$' and '$]$') have been associated with the grammatical symbol $A$ (resp. $B$).}
	\label{chap2:fig:tree-linearisation}
\end{figure}


~

\noindent\textbf{\textit{The transition schemas for the view $\{A, B\}$}}

A list of trees (forest) is represented by the concatenation of their linearisation. We use the opening parenthesis '(' and the closing one ')' to represent Dyck symbols associated with the visible symbol $A$, and the opening bracket '[' and the closing one ']' to represent those associated with the visible symbol $B$. Each transition of the automata associated to partial replicas according to the view $\{A, B\}$ is conform to one of the transition schemas\footnote{We do not represent the whole set of transition schemas in this example; only the useful subset for reconciliation of closed documents is shown here because the documents to reconcile in this example are all closed (has no buds). To consider buds, one should, for each visible sort $X$, associate a new pair of Dyck symbols to the bud of type $X$ then, derive the new schemas.} in table \ref{chap2:table:trans-schem-a-b}.
\begin{table}[ht]
	\centering
	\caption{The transition schemas for the view $\{A, B\}$.}
	\label{chap2:table:trans-schem-a-b}
	\begin{tabular}[t]{lcll}
	$\langle A,w_{1} \rangle$ & $\longrightarrow$ & $(P_{1}, [\langle C,u \rangle, \langle B,v \rangle])$ & if $w_{1}=u[v]$ \\
	
	$\langle A,w_{2} \rangle$ & $\longrightarrow$ & $(P_{2}, [\textit{ }])$ & if $w_{2}=\epsilon$\\
	
	$\langle B,w_{3} \rangle$ & $\longrightarrow$ & $(P_{3}, [\langle C,u \rangle, \langle A,v \rangle])$ & if $w_{3}=u(v)$\\
	
	$\langle B,w_{4} \rangle$ & $\longrightarrow$ & $(P_{4}, [\langle B,u \rangle, \langle B,v \rangle])$ & if $w_{4}=[u][v]$\\
	
	$\langle C,w_{5} \rangle$ & $\longrightarrow$ & $(P_{5}, [\langle A,u \rangle, \langle C,v \rangle])$ & if $w_{5}=(u)v$\\
	
	$\langle C,w_{6} \rangle$ & $\longrightarrow$ & $(P_{6}, [\langle C,u \rangle, \langle C,v \rangle])$ & if $w_{6}=uv\neq\epsilon$\\
	
	$\langle C,w_{7} \rangle$ & $\longrightarrow$ & $(C_\omega,[\,])$ & if $w_{7}=\epsilon$ \\
	\end{tabular}
\end{table}
These schemas are obtained from the grammar productions \cite{badouelTchoupeCmcs}, and the pairs $\langle X,w_{i} \rangle$ are states where $X$ is a grammatical symbol and $w_{i}$ a forest encoded in the Dyck language. The first schema for example, states that the AST generated from the state $\langle A,w_{1} \rangle$ are those obtained using the production $P_{1}$ to create a tree of the form $P_{1}[t_{1}, t_{2}]$; $t_{1}$ and $t_{2}$ being generated respectively from the states $\langle C,u \rangle$ and $\langle B,v \rangle$ such that $w_{1}=u[v]$. The state $\langle C,w_{7} \rangle$ with $w_{7}=\epsilon$ being an exit state, the rule $\langle C,w_{7} \rangle$ $\longrightarrow$ $(C_{\omega}, [\,])$ linked to the production $P_{7}$ states that, the AST generated from the state $\langle C,w_{7} \rangle$ is reduced to a bud of type $C$ ($C$ is the symbol located at the left hand side of $P_{7}$).


~

\noindent\textbf{\textit{Construction of the automaton $\mathcal{A}^{(1)}$ associated to $tv1$}}

Having associated Dyck symbols '(' and ')' (resp. '[' and ']') to the grammatical symbol $A$ (resp. $B$), the linearisation of the partial replica $tv1$ (fig. \ref{chap2:fig:consensus-workflow}(c)) gives $(([[()()][()]])[()])$. As $A$ is the axiom of the grammar, the initial state of the automaton $\mathcal{A}^{(1)}$ is $q_{0}^{1}=\langle A,([[()()][()]])[()] \rangle$. When considering only the  states accessible from $q_{0}^{1}$ and by applying the previous schema of transition, we obtain the automaton in table \ref{chap2:table:automaton-a-b}, for the replica $tv1$ (fig. \ref{chap2:fig:consensus-workflow}(c)).
\begin{table}[ht]
	\caption{The tree automaton associated to the replica $tv1$.}
	\label{chap2:table:automaton-a-b}
	\begin{tabular}[t]{|lcp{5.7cm}|lcp{5cm}|}
	\hline
	$q_{0}^{1}$ & $\longrightarrow$ & $(P_{1}, [q_{1}^{1}, q_{2}^{1}])$ & & with & $q_{1}^{1}=\langle C,([[()()][()]]) \rangle$ and $q_{2}^{1}=\langle B,() \rangle$\\
	
	%\end{tabular}
	%\begin{tabular}[t]{|lcp{5.cm}|lcp{5cm}|}
	\hline
	$q_{1}^{1}$ & $\longrightarrow$ & $(P_{5}, [q_{3}^{1}, q_{4}^{1}])$ & & with & $q_{3}^{1}=\langle A,[[()()][()]] \rangle$ and $q_{4}^{1}=\langle C,\epsilon \rangle$\\
	\hline
	%\end{tabular}
	%\begin{tabular}[t]{lcp{5.3cm}lcp{5cm}}
	
	$q_{1}^{1}$ & $\longrightarrow$ & $(P_{6}, [q_{4}^{1}, q_{1}^{1}])$ | $(P_{6}, [q_{1}^{1}, q_{4}^{1}])$ & & &\\
	\hline
	%\end{tabular}
	%\begin{tabular}[t]{lcp{5.3cm}lcp{5cm}}
	
	$q_{2}^{1}$ & $\longrightarrow$ & $(P_{3}, [q_{4}^{1}, q_{5}^{1}])$ & & with & $q_{5}^{1}=\langle A,\epsilon \rangle$\\
	\hline
	%\end{tabular}
	%\begin{tabular}[t]{lcp{5.3cm}lcp{5cm}}
	
	$q_{3}^{1}$ & $\longrightarrow$ & $(P_{1}, [q_{4}^{1}, q_{6}^{1}])$ & & with & $q_{6}^{1}=\langle B,[()()][()] \rangle$\\
	\hline
	%\end{tabular}
	%\begin{tabular}[t]{lcp{5.3cm}lcp{5cm}}
	
	$q_{4}^{1}$ & $\longrightarrow$ & $(C_\omega,[\,])$ & & &\\
	\hline
	%\end{tabular}
	%\begin{tabular}[t]{lcp{5.3cm}lcp{5cm}}
	
	$q_{5}^{1}$ & $\longrightarrow$ & $(P_{2}, [\,])$ & & &\\
	\hline
	%\end{tabular}
	%\begin{tabular}[t]{lcp{5.3cm}lcp{5cm}}
	
	$q_{6}^{1}$ & $\longrightarrow$ & $(P_{4}, [q_{7}^{1}, q_{2}^{1}])$ & & with & $q_{7}^{1}=\langle B,()() \rangle$\\
	\hline
	%\end{tabular}
	%\begin{tabular}[t]{lcp{5.3cm}lcp{5cm}}
	
	$q_{7}^{1}$ & $\longrightarrow$ & $(P_{3}, [q_{8}^{1}, q_{5}^{1}])$ & & with & $q_{8}^{1}=\langle C,() \rangle$\\
	\hline
	%\end{tabular}
	%\begin{tabular}[t]{lcp{5.3cm}lcp{5cm}}
	
	$q_{8}^{1}$ & $\longrightarrow$ & $(P_{5}, [q_{5}^{1}, q_{4}^{1}])$ & & &\\
	\hline
	%\end{tabular}
	%\begin{tabular}[t]{lcp{5.3cm}lcp{5cm}}
	
	$q_{8}^{1}$ & $\longrightarrow$ & $(P_{6}, [q_{8}^{1}, q_{4}^{1}])$ | $(P_{6}, [q_{4}^{1}, q_{8}^{1}])$ & & &\\
	\hline
	\end{tabular}
\end{table}
The state $q_{4}^{1}=\langle C,\epsilon \rangle$ is the only exit state of  $\mathcal{A}^{(1)}$.
It is easy to verify that the document of figure \ref{chap2:fig:consensus-workflow}(f) resulting from the reverse projection of $tv1$, belongs to the language accepted by the automaton $\mathcal{A}^{(1)}$.

~

\noindent\textbf{\textit{Construction of the automaton $\mathcal{A}^{(2)}$ associated to $tv2$}}

As before, by associating to the grammatical symbol $C$ (resp. $A$) the Dyck symbols '[' and ']' (resp. '(' and ')'), we obtain the transition schemas for the automata associated to the partial replicas according to the view $\{A, C\}$ (see table \ref{chap2:table:trans-schem-a-c}).
\begin{table}[ht]
	\centering
	\caption{The transition schemas for the view $\{A, C\}$.}
	\label{chap2:table:trans-schem-a-c}
	\begin{tabular}[t]{lcll}
	
	$\langle A,w_{1} \rangle$ & $\longrightarrow$ & $(P_{1}, [\langle C,u \rangle, \langle B,v \rangle])$ & if $w_{1}=[u]v$\\
	
	$\langle A,w_{2} \rangle$ & $\longrightarrow$ & $(P_{2}, [\textit{ }])$ & if $w_{2}=\epsilon$\\
	$\langle B,w_{3} \rangle$ & $\longrightarrow$ & $(P_{3}, [\langle C,u \rangle, \langle A,v \rangle])$ & if $w_{3}=[u](v)$\\
	
	$\langle B,w_{4} \rangle$ & $\longrightarrow$ & $(P_{4}, [\langle B,u \rangle, \langle B,v \rangle])$ & if $w_{4}=uv\neq\epsilon$\\
	$\langle B,w_{5} \rangle$ & $\longrightarrow$ & $(B_\omega,[\,])$ & if $w_{5}=\epsilon$\\
	
	$\langle C,w_{6} \rangle$ & $\longrightarrow$ & $(P_{5}, [\langle A,u \rangle, \langle C,v \rangle])$ & if $w_{6}=(u)[v]$\\
	
	
	$\langle C,w_{7} \rangle$ & $\longrightarrow$ & $(P_{6}, [\langle C,u \rangle, \langle C,v \rangle])$ & if $w_{7}=[u][v]$\\
	
	$\langle C,w_{8} \rangle$ & $\longrightarrow$ & $(P_{7},[\textit{ }])$ & if $w_{8}=\epsilon$\\
	\end{tabular}
\end{table}

The linearisation of the partial replica $tv2$ (fig. \ref{chap2:fig:consensus-workflow}(e)) is $([([\textit{ }][\textit{ }]()[\textit{ }]())[\textit{ }]][[\textit{ }][\textit{ }]]())$. The automaton $\mathcal{A}^{(2)}$ associated to this replica has as initial state $q_{0}^{2}=\langle A,[([\textit{ }][\textit{ }]()[\textit{ }]())[\textit{ }]][[\textit{ }][\textit{ }]]() \rangle$ and its transitions are the ones in table \ref{chap2:automaton-a-c}. 
\begin{table}[ht]
	\caption{The tree automaton associated to the replica $tv2$.}
	\label{chap2:automaton-a-c}
	\begin{tabular}[t]{|lcp{5.7cm}|lcp{5cm}|}
	\hline
	$q_{0}^{2}$ & $\longrightarrow$ & $(P_{1}, [q_{1}^{2}, q_{2}^{2}])$ & & with & $q_{1}^{2}=\langle C,([\,][\,]()[\,]())[\,] \rangle$ and $q_{2}^{2}=\langle B,[[\,][\,]]() \rangle$\\
	\hline
	%\end{tabular}
	%\begin{tabular}[t]{lcp{5.3cm}lcp{5cm}}
	
	$q_{1}^{2}$ & $\longrightarrow$ & $(P_{5}, [q_{3}^{2}, q_{4}^{2}])$ & & with & $q_{3}^{2}=\langle A,[\,][\,]()[\,]() \rangle$ and $q_{4}^{2}=\langle C,\epsilon \rangle$\\
	\hline
	%\end{tabular}
	%\begin{tabular}[t]{lcp{5.3cm}lcp{5cm}}
	
	$q_{2}^{2}$ & $\longrightarrow$ & $(P_{3}, [q_{5}^{2}, q_{6}^{2}])$ & & with & $q_{5}^{2}=\langle C,[\,][\,] \rangle$ and $q_{6}^{2}=\langle A,\epsilon \rangle$\\
	\hline
	%\end{tabular}
	%\begin{tabular}[t]{lcp{5.3cm}lcp{5cm}}
	
	$q_{3}^{2}$ & $\longrightarrow$ & $(P_{1}, [q_{4}^{2}, q_{7}^{2}])$ & & with & $q_{7}^{2}=\langle B,[\,]()[\,]() \rangle$\\
	\hline
	%\end{tabular}
	%\begin{tabular}[t]{lcp{5.3cm}lcp{5cm}}
	
	$q_{4}^{2}$ & $\longrightarrow$ & $(P_{7}, [\,])$ & & &\\
	
	\hline
	%\end{tabular}
	%\begin{tabular}[t]{lcp{5.3cm}lcp{5cm}}
	
	$q_{5}^{2}$ & $\longrightarrow$ & $(P_{6}, [q_{4}^{2}, q_{4}^{2}])$ & & &\\
	
	\hline
	%\end{tabular}
	%\begin{tabular}[t]{lcp{5.3cm}lcp{5cm}}
	
	$q_{6}^{2}$ & $\longrightarrow$ & $(P_{2}, [\,])$ & & &\\
	
	\hline
	%\end{tabular}
	%\begin{tabular}[t]{lcp{5.3cm}lcp{5cm}}
	
	$q_{7}^{2}$ & $\longrightarrow$ & $(P_{4}, [q_{8}^{2}, q_{7}^{2}])$ | $(P_{4}, [q_{9}^{2}, q_{10}^{2}])$ | $(P_{4}, [q_{11}^{2}, q_{11}^{2}])$ | $(P_{4}, [q_{12}^{2}, q_{13}^{2}])$ | $(P_{4}, [q_{7}^{2}, q_{8}^{2}])$ & & with & $q_{8}^{2}=\langle B,\epsilon \rangle$, $q_{9}^{2}=\langle B,[\,] \rangle$, $q_{10}^{2}=\langle B,()[\,]() \rangle$, $q_{11}^{2}=\langle B,[\,]() \rangle$, $q_{12}^{2}=\langle B,[\,]()[\,] \rangle$ and $q_{13}^{2}=\langle B,() \rangle$\\
	
	\hline
	%\end{tabular}
	%\begin{tabular}[t]{lcp{5.3cm}lcp{5cm}}
	
	$q_{8}^{2}$ & $\longrightarrow$ & $(B_\omega,[\,])$ & & &\\
	
	\hline
	%\end{tabular}
	%\begin{tabular}[t]{lcp{5.3cm}lcp{5cm}}
	
	$q_{9}^{2}$ & $\longrightarrow$ & $(P_{4}, [q_{8}^{2}, q_{9}^{2}])$ | $(P_{4}, [q_{9}^{2}, q_{8}^{2}])$ & & &\\
	
	\hline
	%\end{tabular}
	%\begin{tabular}[t]{lcp{5.3cm}lcp{5cm}}
	
	$q_{10}^{2}$ & $\longrightarrow$ & $(P_{4}, [q_{8}^{2}, q_{10}^{2}])$ | $(P_{4}, [q_{13}^{2}, q_{11}^{2}])$ | $(P_{4}, [q_{14}^{2}, q_{13}^{2}])$ | $(P_{4}, [q_{10}^{2}, q_{8}^{2}])$ & & with & $q_{14}^{2}=\langle B,()[\,]    \rangle$\\
	
	\hline
	%\end{tabular}
	%\begin{tabular}[t]{lcp{5.3cm}lcp{5cm}}
	
	$q_{11}^{2}$ & $\longrightarrow$ & $(P_{3}, [q_{4}^{2}, q_{6}^{2}])$ & & &\\
	
	\hline
	%\end{tabular}
	%\begin{tabular}[t]{lcp{5.3cm}lcp{5cm}}
	
	$q_{12}^{2}$ & $\longrightarrow$ & $(P_{4}, [q_{8}^{2}, q_{12}^{2}])$ | $(P_{4}, [q_{9}^{2}, q_{14}^{2}])$ | $(P_{4}, [q_{11}^{2}, q_{9}^{2}])$ | $(P_{4}, [q_{12}^{2}, q_{8}^{2}])$ & & &\\
	
	\hline
	%\end{tabular}
	%\begin{tabular}[t]{lcp{5.3cm}lcp{5cm}}
	
	$q_{13}^{2}$ & $\longrightarrow$ & $(P_{4}, [q_{8}^{2}, q_{13}^{2}])$ | $(P_{4}, [q_{13}^{2}, q_{8}^{2}])$ & & &\\
	
	\hline
	%\end{tabular}
	%\begin{tabular}[t]{lcp{5.3cm}lcp{5cm}}
	
	$q_{14}^{2}$ & $\longrightarrow$ & $(P_{4}, [q_{8}^{2}, q_{14}^{2}])$ | $(P_{4}, [q_{13}^{2}, q_{9}^{2}])$ | $(P_{4}, [q_{14}^{2}, q_{8}^{2}])$ & & &\\
	\hline
	\end{tabular}
\end{table}
The state $q_{8}^{2}=\langle B,\epsilon \rangle$ is the only exit state of the automaton $\mathcal{A}^{(2)}$. 


~

\noindent\textbf{\textit{Construction of the consensus automaton $\mathcal{A}_{(sc)}$}}

By application of synchronous product of several tree automata described in section \ref{chap2:sec:consensus-calculation} (\textit{construction of the consensus automaton}) to the automata $\mathcal{A}^{(1)}$ and $\mathcal{A}^{(2)}$, the consensual automaton $\mathcal{A}_{(sc)}=\mathcal{A}^{(1)}\otimes\mathcal{A}^{(2)}$ has $q_{0}=(q_{0}^{1}, q_{0}^{2})$ as initial state. $\mathcal{A}^{(1)}$ has a transition from $q_{0}^{1}$ to $[q_{1}^{1}, q_{2}^{1}]$ labelled $P_{1}$. Similarly, $\mathcal{A}^{(2)}$ has a transition from $q_{0}^{2}$ to $[q_{1}^{2}, q_{2}^{2}]$ labelled $P_{1}$. So, we have in $\mathcal{A}_{(sc)}$, a transition labelled $P_{1}$ for accessing states $[q_{1}=(q_{1}^{1}, q_{1}^{2}), q_{2}=(q_{2}^{1}, q_{2}^{2})]$ from the initial state $q_{0}=(q_{0}^{1}, q_{0}^{2})$. Following this principle, we construct the consensual automaton in table \ref{chap2:table:automaton-a-b-c}. 
\begin{table}[ht]
	\caption{The consensual tree automaton.}
	\label{chap2:table:automaton-a-b-c}
	\begin{tabular}[t]{|lcp{5.7cm}|lcp{5cm}|}
	\hline
	& & & & & $q_{0}=(q_{0}^{1}, q_{0}^{2})$\\
	\hline
	$q_{0}$ & $\longrightarrow$ & $(P_{1}, [q_{1}, q_{2}])$ & & with & $q_{1}=(q_{1}^{1}, q_{1}^{2})$ and $q_{2}=(q_{2}^{1}, q_{2}^{2})$\\
	\hline
	%\end{tabular}
	%\begin{tabular}[t]{lcp{5.3cm}lcp{5cm}}
	
	$q_{1}$ & $\longrightarrow$ & $(P_{5}, [q_{3}, q_{4}])$ & & with & $q_{3}=(q_{3}^{1}, q_{3}^{2})$ and $q_{4}=(q_{4}^{1}, q_{4}^{2})$\\
	
	\hline
	%\end{tabular}
	%\begin{tabular}[t]{lcp{5.3cm}lcp{5cm}}
	
	$q_{2}$ & $\longrightarrow$ & $(P_{3}, [q_{5}, q_{6}])$ & & with & $q_{5}=(q_{4}^{1}, q_{5}^{2})$ and $q_{6}=(q_{5}^{1}, q_{6}^{2})$\\
	
	\hline
	%\end{tabular}
	%\begin{tabular}[t]{lcp{5.3cm}lcp{5cm}}
	
	$q_{3}$ & $\longrightarrow$ & $(P_{1}, [q_{4}, q_{7}])$ & & with & $q_{7}=(q_{6}^{1}, q_{7}^{2})$\\
	
	\hline
	%\end{tabular}
	%\begin{tabular}[t]{lcp{5.3cm}lcp{5cm}}
	
	$q_{4}$ & $\longrightarrow$ & $(P_{7}, [\,])$ & & &\\
	
	\hline
	%\end{tabular}
	%\begin{tabular}[t]{lcp{5.3cm}lcp{5cm}}
	
	$q_{5}$ & $\longrightarrow$ & $(P_{6}, [q_{8}, q_{8}])$ & & with & $q_8=(q_{s1}, q^2_4)$ and $q_{s1}=\langle Open~C,[\,] \rangle $\\
	
	\hline
	%\end{tabular}
	%\begin{tabular}[t]{lcp{5.3cm}lcp{5cm}}
	
	$q_{6}$ & $\longrightarrow$ & $(P_{2}, [\,])$ & & &\\
	
	\hline
	%\end{tabular}
	%\begin{tabular}[t]{lcp{5.3cm}lcp{5cm}}
	
	$q_{7}$ & $\longrightarrow$ & $(P_{4}, [q_{9}, q_{10}])$ | $(P_{4}, [q_{11}, q_{12}])$ | $(P_{4}, [q_{13}, q_{14}])$ | $(P_{4}, [q_{15}, q_{16}])$ | $(P_{4}, [q_{17}, q_{18}])$ & & with & $q_{9}=(q_{7}^{1}, q_{8}^{2})$, $q_{10}=(q_{2}^{1}, q_{7}^{2})$, $q_{11}=(q_{7}^{1}, q_{9}^{2})$, $q_{12}=(q_{2}^{1}, q_{10}^{2})$, $q_{13}=(q_{7}^{1}, q_{11}^{2})$, $q_{14}=(q_{2}^{1}, q_{11}^{2})$, $q_{15}=(q_{7}^{1}, q_{12}^{2})$, $q_{16}=(q_{2}^{1}, q_{13}^{2})$, $q_{17}=(q_{7}^{1}, q_{7}^{2})$ and $q_{18}=(q_{2}^{1}, q_{8}^{2})$\\
	\hline
	%\end{tabular}
	
	%\begin{tabular}[t]{|lcp{6.56cm}|lcp{4.99cm}|}
	%\hline
	$q_{8}$ & $\longrightarrow$ & $(P_{7}, [\,])$ & & &\\
	
	\hline
	%\end{tabular}
	%\begin{tabular}[t]{lcp{5.3cm}lcp{5cm}}
	
	$q_{9}$ & $\longrightarrow$ & $(P_{3}, [q_{19}, q_{20}])$ & & with & $q_{19}=(q_{8}^{1}, q_{s1})$ and $q_{20}=(q_{5}^{1}, q_{s2})$, $q_{s2}=\langle Open~A,[\,] \rangle $\\
	
	\hline
	%\end{tabular}
	%\begin{tabular}[t]{lcp{5.3cm}lcp{5cm}}
	
	$q_{13}$ & $\longrightarrow$ & $(P_{3}, [q_{21}, q_{6}])$ & & with & $q_{21}=(q_{8}^{1}, q_{4}^{2})$\\
	
	\hline
	%\end{tabular}
	%\begin{tabular}[t]{lcp{5.3cm}lcp{5cm}}
	
	$q_{14}$ & $\longrightarrow$ & $(P_{3}, [q_{4}, q_{6}])$ & & &\\
	
	\hline
	%\end{tabular}
	%\begin{tabular}[t]{lcp{5.3cm}lcp{5cm}}
	
	$q_{18}$ & $\longrightarrow$ & $(P_{3}, [q_{22}, q_{20}])$ & & with & $q_{22}=(q_{4}^{1}, q_{s1})$\\
	
	\hline
	%\end{tabular}
	%\begin{tabular}[t]{lcp{5.3cm}lcp{5cm}}
	
	$q_{19}$ & $\longrightarrow$ & $(P_{5}, [q_{20}, q_{22}])$ | $(P_{6}, [q_{19}, q_{22}])$ | $(P_{6}, [q_{22}, q_{19}])$ & & &\\
	\hline
	%\end{tabular}
	%\begin{tabular}[t]{lcp{5.3cm}lcp{5cm}}
	
	$q_{20}$ & $\longrightarrow$ & $(P_{2}, [\,])$ & & &\\
	\hline
	%\end{tabular}\\
	
	%\begin{tabular}[t]{|p{8.43cm}| p{6.5cm}|}
	%\hline
	$q_{10}$ & $\longrightarrow$ & $(B_\omega, [\,])$ & & &\\ 
	\hline
	$q_{11}$ & $\longrightarrow$ & $(B_\omega, [\,])$ & & &\\
	\hline
	$q_{12}$ & $\longrightarrow$ & $(B_\omega, [\,])$ & & &\\
	\hline
	$q_{15}$ & $\longrightarrow$ & $(B_\omega, [\,])$ & & &\\
	\hline
	$q_{16}$ & $\longrightarrow$ & $(B_\omega, [\,])$ & & &\\
	\hline
	$q_{17}$ & $\longrightarrow$ & $(B_\omega, [\,])$ & & &\\
	\hline
	$q_{21}$ & $\longrightarrow$ & $(C_\omega, [\,])$ & & &\\
	\hline
	$q_{22}$ & $\longrightarrow$ & $(C_\omega, [\,])$ & & &\\
	\hline
	\end{tabular}
\end{table}


The states $\{q_{10}, q_{11}, q_{12}, q_{15}, q_{16}, q_{17}, q_{21}, q_{22}\}$ are the exit states of the automaton $\mathcal{A}_{(sc)}$. They are states whose composite states are either in conflict (for example $q_{10}=(q_{2}^{1}, q_{7}^{2})$ et $q_{2}^{1} \curlyveeuparrow q_{7}^{2}$), or are all exit states (for example $q_{22}=(q_{4}^{1}, q_{s1}))$.

The use of the function that generates the simplest AST (with buds) of a tree language from its automaton \cite{badouelTchoupeCmcs} on $\mathcal{A}_{(sc)}$, produces \textit{four} AST whose derivation trees (the consensus) are shown in figure \ref{chap2:fig:consensus-example-trees}.
\begin{figure}[ht!]
	\noindent
	\makebox[\textwidth]{\includegraphics[scale=0.3]{Chap2/images/consensusTrees.png}}
	\caption{Consensual trees generated from the automaton $\mathcal{A}_{(sc)}$}
	\label{chap2:fig:consensus-example-trees}
\end{figure}

