
In this section, we will better study Badouel and Tchoup\'e work on structured editing. We will adopt and adapt the different mathematical tools they proposed, to produce a more general algorithm for merging partial replicates, by taking into account the cases where these would be in conflict.

\mySubSection{Structured Cooperative Editing and Notion of Partial Replication}{} 
\label{chap2:sec:structured-cooperative-editing-partial-rep}

\mySubSubSection{Structured Document, Editing and Conformity}{}
\label{chap2:sec:structured-document-edition-conformity}

In the XML community, the document model is typically specified using a DTD or a XML Schema \cite{xml2000}. It is shown that these DTD are equivalent to (regular) grammars with special characteristics called   \textit{XML grammars} \cite{XMLG}. The (context free) grammars are therefore a generalisation of the DTD and, on the basis of the studies they have undergone, mainly as formal models for the specification of programming languages, they provide an ideal framework for the formal study of the transformations involved in XML technologies. That's why we use them in our work as tools for specifying the structure of documents.

We are only interested in the structure of the documents regardless of their contents and the attributes they may contain.  
We will therefore represent the abstract structure of a structured document by a tree, and its model by an abstract context free grammar; a valid structured document will then be a derivation tree for this grammar. 
A context free grammar defines the structure of its instances (the documents that are conform to it) by means of productions. 
A production, generally denoted $p: X_0 \rightarrow X_1 \ldots X_n$, is comparable in this context, to a structuring rule which shows how the symbol $X_0$, located in the left part of the production, is divided into a sequence of other symbols $X_1 \ldots X_n$, located on its right side. More formally, 

\begin{definition}
An \textbf{abstract context free grammar} is given by $\mathbb{G}=\left(\mathcal{S},\mathcal{P},A\right)$
composed of a finite set $\mathcal{S}$ of \textbf{grammatical symbols} or \textbf{sorts} corresponding to the different \textbf{syntactic categories} involved, a particular grammatical symbol $A\in\mathcal{S}$ called \textbf{axiom}, and a finite set $\mathcal{P}\subseteq\mathcal{S}\times\mathcal{S}^{*}$ of \textbf{productions}. 
A production $P=\left(X_{P(0)},X_{P(1)}\cdots X_{P(n)}\right)$ is denoted $P:X_{P(0)}\rightarrow X_{P(1)}\cdots X_{P(n)}$ and $\left|P\right|$ 
denotes the length of the right hand side of $P$. A production with the symbol $X$ as left part is called a \textit{X-production}.
\end{definition}

For certain treatments on trees (documents), it is necessary to designate precisely a particular node. Several indexing techniques exist, among them, the so-called \textit{Dynamic Level Numbering} \cite{boe04} based on identifiers with variable lengths, inspired by the \textit{Dewey} decimal classification (see fig. \ref{chap2:fig:indexed-tree}). According to this indexing system, a tree can be defined as follows:

\begin{definition}
A \textbf{tree} whose nodes are labelled in an alphabet $\mathcal{S}$, is a partial map $t:\mathbb{N}^*\rightarrow \mathcal{S}$, whose domain $\mathit{Dom}(t)\subseteq \mathbb{N}^*$ is a prefix closed set such that, for all $u\in \mathit{Dom}(t)$, the set $\{i\in\mathbb{N}~|~u\cdot i\in\mathit{Dom}(t)\}$ is a non-empty interval of integers $[1,\cdots,n]\cap\mathbb{N}$ ($\epsilon \in \mathit{Dom}(t) \mathit{~is~ the~ root~ label}$); the integer $n$ is the \textbf{arity} of the node whose address is $u$. 
$t(u)$ is the value (label) of the node in $t$, whose address is $u$.
%Un arbre $t_{w}$ est \textbf{un sous-arbre} de $t$ de racine le noeud d'adresse $w \in \mathit{Dom}(t)$ a pour domaine $\mathit{Dom}(t_{w})=\{u ~|~ w.u \in \mathit{Dom}(t)\} \mbox{ et }  t_{w}(u) = t(w.u)$
If $t_1,\cdots,t_n$ are trees and $a\in \mathcal{S}$, we denote $t = a [t_1,\ldots,t_n]$, the tree $t$ of domain $\mathit{Dom}(t)=\{\varepsilon\}\cup\{i\cdot u~|~1\leq i\leq n\, ,\; u\in \mathit{Dom}(t_i)\}$ with $t(\varepsilon)=a$, and $t(i\cdot u)=t_i(u)$. 

%L'arbre vide sera noté \textit{nil} et $\mathit{next} ~t= [t_1, \cdots, t_n]$ est la liste des sous-arbres de l'arbre $t = a[t_1,\ldots,t_n]$. 
\end{definition}
\begin{figure}[ht!]
	\noindent
	\makebox[\textwidth]{\includegraphics[scale=0.45]{Chap2/images/arbreIndexe.png}}
	\caption{Example of an indexed tree.}
	\label{chap2:fig:indexed-tree}
\end{figure}

Let $t$ be a document and $\mathbb{G}=\left(\mathcal{S},\mathcal{P},A\right)$ a grammar. $t$ is a derivation tree for $\mathbb{G}$ if its root node is labelled by the axiom $A$ of $\mathbb{G}$, and if for all internal node $n_0$ labelled by the sort $X_0$, and whose sons $n_1, \ldots n_n$, are respectively labelled by the sorts $X_1,\ldots, X_n$, there is one production $P \in \mathcal{P}$ such that, $P:X_0\rightarrow X_1\cdots X_n$ and $\left|P\right|=n$. 
It is also said in this case, that $t$ belongs to the language generated by $\mathbb{G}$ from the symbol $A$, and it is denoted $t \in \mathscr{L}{\left( \mathbb{G}, ~A \right)}$ or  $t\therefore \mathbb{G}$.

%\begin{comment}
There is a bijective correspondence between the set of derivation trees of one grammar and all its \textit{Abstract Syntax Tree} (\textit{AST}). In an AST, nodes are labelled by the names of the productions. 

\begin{definition}
The set $AST(\mathbb{G},X)$ of \textbf{abstract syntax trees} according to the grammar $\mathbb{G}$ associated with grammatical symbol $X$, consists of trees in the form $P[t_1,\ldots,t_n]$ where $P$ is a production such that, $X=X_{P(0)}$, $n=|P|$ and $t_i\in AST(\mathbb{G},X_i), ~X_{i} = X_{P(i)}$ for all $1\leq i\leq n$. 
%Les arbres de syntaxe abstraite sont donc les termes pour la signature dont les sortes sont les symboles grammaticaux et dont les opérateurs sont les productions de la grammaire où la production $P:X_{P(0)}\rightarrow X_{P(1)}\cdots X_{P(n)}$  est vue comme un opérateur d'arité $X_{P(1)}\times\cdots\times X_{P(n)}\rightarrow X_{P(0)}$.
\end{definition}
AST are used to show that a given tree, labelled with grammatical symbols, is an instance of a given grammar.
%\end{comment}

A structured document being edited, is represented by a tree containing \textit{buds} (or \textit{open nodes}) which indicate in a tree, the only places where editions (i.e updates) are possible\footnote{Note that, we are interested only in the \textit{positive edition} based on a partial optimistic replication \cite{Yasushi2005} of edited documents. In fact, the edited documents are only increasing: there is no possible erasure as soon as a synchronisation has been performed.}.
Buds are typed; a \textit{bud of sort $X$} is a leaf node labelled by $X_\omega$ (see fig. \ref{chap2:fig:tree-with-bud}): it can only be edited (i.e extended to a subtree) by using an \textit{X-production}. Thus, a structured document being edited and that have the grammar $\mathbb {G} = (\mathcal {S}, \mathcal {P}, A) $ as model, is a derivation tree for the extended grammar 
$\mathbb{G}_{\Omega}=(\mathcal{S}\cup\mathcal{S}_{\omega},\mathcal{P}\cup\mathcal{S}_{\Omega},A)$, obtained from $\mathbb {G} $ as follows: for all sort $X$, we not only add in the set $\mathcal{S}$ of sorts a new sort $X_{\omega}$, but we also add  a new $\varepsilon$-production $X_{\Omega} : X_{\omega} \rightarrow \varepsilon$ in the set $\mathcal {P}$ of productions; so we have: $\mathcal{S}_{\omega}=\{X_{\omega},~ X\in\mathcal {S}\}$ and $\mathcal{S}_{\Omega} = \{X_{\Omega} : X_{\omega} \rightarrow \varepsilon,~ X_{\omega} \in \mathcal{S}_{\omega}\}$.
\begin{figure}[ht!]
	\noindent
	\makebox[\textwidth]{\includegraphics[scale=0.45]{Chap2/images/documentBourgeons.png}}
	\caption{Example of a tree that contains buds.}
	\label{chap2:fig:tree-with-bud}
\end{figure}


When we look at the productions of a grammar, we can notice that each sort is associated with a set of productions. From this point of view therefore, we can consider a grammar as an application
\[
	 gram : symb \rightarrow [(prod,~[symb])]
\]
which associates to each sort, a list of pairs formed by a production name and the list of sorts in the right hand side of this production. Such an observation suggests that a grammar can be interpreted as a (descending) tree automaton that can be used for recognition or for the generation of its instances.

\begin{definition}
\label{chap2:def:tree-automaton}
A (descending) \textbf{tree automaton} defined on $\Sigma$, is a quadruplet $\mathcal{A}=(\Sigma,Q,R,q_0)$ of a 
 set $\Sigma$ of symbols %avec arité  (signature)
; its elements are the nodes' labels of the trees to be generated (or recognised), a set $Q$ of states, a particular state $q_0 \in Q$ called initial state, and a finite set $R\subseteq  Q \times \Sigma \times Q^*$ of transitions.
\begin{itemize}
	\item An element of $R$ is denoted $q\rightarrow \left( \sigma, [ q_1,\cdots,q_n]\right)$ or in an equivalent way  $q\stackrel{\sigma}{\rightarrow}(q_{1},\ldots,q_{n})$: intuitively, $[ q_1,\cdots,q_n]$ is the list of states accessible from $q$ by crossing a transition labelled $\sigma$.
	\item If $q\stackrel{\sigma_1}{\rightarrow}\left( q_{1}^1, \cdots, q_{n_1}^1\right), \cdots, q\stackrel{\sigma_k}{\rightarrow}\left( q_{1}^k, \cdots, q_{n_k}^k\right)$ denotes the set of transitions associated to the state $q$, we denote \textbf{$next~q=[\left( \sigma_1,[ q_{1}^1, \cdots, q_{n_1}^1]\right),\cdots, \left( \sigma_k,[ q_{1}^k, \cdots, q_{n_k}^k]\right)]$}, the list that consists of pairs $\left( \sigma_i,[ q_{1}^i, \cdots, q_{n_i}^i]\right)$. 
	A transition of the form $q\rightarrow(\sigma,[\;])$, is called \textit{final transition} and a state possessing this transition is a \textit{final state}.
\end{itemize}
\end{definition}


One can interpret a grammar $\mathbb{G}=\left(\mathcal{S},\mathcal{P},A\right)$ as a (descending) tree automaton \cite{Comon} $\mathcal{A}=(\Sigma,Q,R,q_0)$ considering that:
\begin{itemize}
	\item[(1)] $\Sigma=\mathcal{P}$ is the type of labels of the nodes forming the tree to recognise; 
	\item[(2)] $Q=\mathcal{S}$ is the type of states and, 
	\item[(3)] $q\rightarrow \left( \sigma,[ q_1,\cdots,q_n]\right)$ is a transition of the automaton when the pair $\left(\sigma,[q_1,\cdots,q_n] \right)$ appears in the list $(gram~~ q)$\footnote{Reminder: $gram$ is the application obtained by abstraction of $\mathbb{G}$ and have as type : $gram : symb \rightarrow [(prod,~[symb])]$.}.
\end{itemize}
We note $\mathcal{A}_{\mathbb{G}} $ the tree automaton derived from $\mathbb{G}$.

To obtain the set $AST_\mathcal{A}$ of \textit{AST} generated by a tree automaton $\mathcal{A}$ from an initial state $q_0$, one must:
\begin{itemize}
	\item[(1)] Create a root node $r$, associate the initial state $q_0$ and add it to the set $AST_\mathcal{A}$ initially empty;
	\item[(2)] Remove from $AST_\mathcal{A}$ an AST $t$ under construction, i.e. with at least one leaf node $node$ unlabelled. Let $q$ be the state associated to $node$.
	For all transition $q\stackrel{\sigma}{\rightarrow} \left(q_1,\cdots,q_n\right)$   of $\mathcal{A}$, add in $AST_\mathcal{A}$ the trees $t'$ which are replicas of $t$ in which, the node $node$ has been substituted by a node $node'$ labelled $\sigma$ and possessing $n$ (unlabelled) sons, each associated to a (distinct) state $q_i$ of $[ q_1,\cdots,q_n]$; 
	\item[(3)] Iterate step (2) until he obtains trees with all the leaf nodes labelled (they are consequently associated to the final states of $\mathcal{A}$): these are \textit{AST}.
\end{itemize}
We note $\mathcal{A} \models t \triangleright q$ the fact that the tree automaton $\mathcal{A}$ accepts the tree $t$ from the initial state $q$, and
 $\mathscr{L}(\mathcal{A}, q)$ (tree language) the set of trees generated by the automaton $\mathcal{A}$ from the initial state $q$. 
Thus, $ \left(\mathcal{A}  \models t \triangleright q \right) \Leftrightarrow \left( t \in \mathscr{L}(\mathcal{A}, q) \right)$.


As for automata on words, one can define a synchronous product on tree automata to obtain the automaton recognising the intersection, the union, etc., of regular tree languages \cite{Comon}. We introduce below the definition of the synchronous product of $k$ tree automata whose adaptation will be used in the next section for the derivation of the consensual automaton. 

\begin{definition}
\label{chap2:def:synchronous-product}
\textbf{Synchronous product of $k$ automata:}\\
Let $\mathcal{A}_1=\left(\Sigma,Q^{(1)},R^{(1)},q_{0}^{(1)}\right), \ldots , \mathcal{A}_k=\left(\Sigma,Q^{(k)},R^{(k)},q_{0}^{(k)}\right) $ be $k$ tree automata. The synchronous product of these $k$ automata $\mathcal{A}_1 \otimes \cdots \otimes \mathcal{A}_k$ denoted $\otimes_{i=1}^{k} \mathcal{A}^{(i)}$, is the automaton $\mathcal{A}_{(sc)}=(\Sigma,Q,R,q_{0})$ defined as follows: 
\begin{itemize}
	\item[\textbf{(a)}] Its states are vectors of states : $Q =Q^{(1)}\times\cdots\times Q^{(k)}$; 
	\item[\textbf{(b)}] Its initial state is the vector formed by the initial states of the different automata : $q_{0}=\left(q_{0}^{(1)},\cdots, q_{0}^{(k)}\right)$; 
	\item[\textbf{(c)}] Its transitions are given by :\\
				$\left(q^{(1)}, \ldots, q^{(k)}\right)$ $\stackrel{a}{\rightarrow}\left(\left(q^{(1)}_1,\ldots,q^{(k)}_1\right),\ldots,\left(q^{(1)}_n,\ldots,q^{(k)}_n\right)\right)$ $\Leftrightarrow$\\ 
				   $\left( q^{(i)}\stackrel{a}{\rightarrow}\left(q^{(i)}_1,\ldots,q^{(i)}_n\right) \quad \forall i,~ 1\leq i\leq k \right)$
\end{itemize}
\end {definition}

\mySubSubSection{Notions of View, Projection, Reverse Projection and Merging}{}
\label{chap2:sec:view-projection-expansion-merging}
\noindent\textbf{\textit{View, associated projection and merging}}

The derivation tree giving the (global) representation of a structured document edited in a cooperative way, makes visible the set of grammatical symbols of the grammar that participated in its construction. As we mentioned in section \ref{chap2:sec:badouel-tchoupe-cooperative-editing} above, for reasons of confidentiality (accreditation degree), a co-author manipulating such a document will not necessarily have access to all of these grammatical symbols; only a subset of them can be considered relevant for him: it is his \textit{view}. A view $\mathcal{V}$ is then a subset of grammatical symbols ($\mathcal{V} \subseteq \mathcal{S}$). 
%Intuitivement,  il s'agit des sortes visibles par un co-auteur dans la représentation globale (arbre de dérivation) du document considéré. 

 
A partial replica of $t$ according to the view $\mathcal{V}$, is a partial copy of $t$ obtained by deleting in $t$, all the nodes labelled by symbols that are not in $\mathcal{V}$. 
Figure \ref{chap2:fig:partial-view} shows a document $t$ (center) and two partial replicas $t_{v_1}$ (left) and $t_{v_2}$ (right) obtained respectively by projections from the views $\mathcal{V}_1 = \{A,B\}$ and $\mathcal{V}_2 = \{A,C\}$. 
\begin{figure}[ht!]
	\noindent
	\makebox[\textwidth]{\includegraphics[scale=0.5]{Chap2/images/docEtRepliques.png}}
	\caption{Example of projections made on a document and partial replicas obtained.}
	\label{chap2:fig:partial-view}
\end{figure}


Practically, a partial replica is obtained via a \textit{projection} operation denoted $\pi$. We therefore denote $\pi_{\mathcal{V}}(t)= t_{\mathcal{V}}$ the fact that $t_{\mathcal{V}}$ is a partial replica obtained by projection of $t$ according to the view $\mathcal{V}$.

Note $t_{\mathcal{V}_i} \leq t_{\mathcal{V}_i}^{maj}$ the fact that, the document $t_{\mathcal{V}_i}^{maj}$ is an update of the document $t_{\mathcal{V}_i}$, i.e. $t_{\mathcal{V}_i}^{maj}$ is obtained from $t_{\mathcal{V}_i}$ by replacing some of its buds by trees.
In an asynchronous cooperative editing process, there are synchronisation points\footnote{Recall that a synchronisation point can be defined statically or triggered by a co-author as soon as certain properties are satisfied.} in which, one tries to merge all contributions $t_{\mathcal{V}_i}^{maj}$ of the various co-authors to obtain a single comprehensive document $t_f$\footnote{It may happen that the edition must be continued after the merging (this is the case if there are still buds in the merged document): one must redistribute to each of the $n$ co-authors a (partial) replica $t_{\mathcal{V}_i}$ of $t_f$ such that  $t_{\mathcal{V}_i} = \pi_{\mathcal{V}_i}(t_f)$ for the continuation of the editing process.}. A merging algorithm that does not incorporate conflict management and relies on a solution to the \textit{reverse projection} problem was given by Badouel and Tchoup\'e.

~

\noindent\textbf{\textit{Partial replica and reverse projection (expansion)}}

The \textit{reverse projection} (also called the \textit{expansion}) of an updated partial replica $t_{\mathcal{V}_i}^{maj}$ relatively to a given grammar $\mathbb{G}=\left(\mathcal{S},\mathcal{P},A\right)$, is the set $T_{i\mathcal{S}}^{maj}$ of documents conform to $\mathbb{G}$, that admit $t_{\mathcal{V}_i}^{maj}$ as partial replica according to ${\mathcal{V}_i}$:
\begin{itemize}
	\item[] $ T_{i\mathcal{S}}^{maj} = \left\{t_{i\mathcal{S}}^{maj} \therefore \mathbb{G}~ | ~ \pi_{\mathcal{V}_i}\left(t_{i\mathcal{S}}^{maj}\right) = t_{\mathcal{V}_i}^{maj} \right\}
	$
\end{itemize} 

A solution to the problem of evaluating the expansion of a given partial replica using tree automata, was proposed by Badouel and Tchoup\'e; in that solution, productions of the grammar $\mathbb{G}$ are used, to bind to a view $\mathcal{V}_i \subseteq \mathcal{S}$ a tree automaton $\mathcal{A}^{(i)}$ such as, the trees it recognises from an initial state built from $t_{\mathcal{V}_i}^{maj}$, are exactly those having this partial replica as projection according to the view $\mathcal{V}_i$:
$ T_{i\mathcal{S}}^{maj} = \mathscr{L}\left(\mathcal{A}^{(i)},~q_{t_{\mathcal{V}_i}}\right) $. 
Practically, they have considered that a state $q$ of the automaton $\mathcal{A}^{(i)}$ is a pair $\left(Tag ~X, \;ts\right)$ where $X$ is a grammatical symbol, $ts$ is a forest (tree set), and \textit{Tag} is a label that is either \textit{Open} or \textit{Close}, and indicates whether the concerned state $q$ can be used to generate a \textit{closed} node or a \textit{bud}. The states of $\mathcal{A}^{(i)}$ are typed: a state of the form $\left(Tag ~X, \;ts\right)$ is of type $X$. We also have a function named \textit{typeState} which, when applied to a state, returns its type\footnote{ $typeState :: state\rightarrow symb$ \\
 $.~~~typeState ~\left(Open ~X, \;ts\right) = X$\\
 $.~~~typeState ~\left(Close ~X, \;ts\right) = X$
	}.
A transition from one state $q$, is of one of the forms $\left(Close~X, \;ts \right) \rightarrow \left(p, \;[q_1, \ldots, q_n]\right)$ or $\left(Open ~X, \;[\;]\right) \rightarrow \left(X_\omega, \;[\;]\right)$. 
A transition of the form $\left(Close~X, \;ts \right) \rightarrow \left(p, \;[q_1, \ldots, q_n]\right)$ is used to generate AST of type $X$ (whose root is labelled by a \textit{X-production}) admitting ''$ts$'' as projection according to the view ${\mathcal{V}_i}$ if $X$ does not belong to ${\mathcal{V}_i}$, and ''$x [ ts ]$'' otherwise.
Similarly, a transition of the form $\left(Open ~X, \;[\;]\right) \rightarrow \left(X_\omega, \;[\;]\right)$ is used to generate a single AST reduced to a bud of type $X$. 

The interested reader may consult \cite{badouelTchoupeCmcs} for a more detailed description of the process of associating a tree automaton with a view and the section \ref{chap2:sec:consensus-illustration} for an illustration.
