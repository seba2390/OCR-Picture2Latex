\documentclass[]{article}
\usepackage{hyperref}

% Using the caption package allows us to support captions that contain "itemize" environments.
% The font=small option makes the text of captions smaller than the main text.
% Marie asked us to do this.
\usepackage[font=small]{caption}


\setcounter{secnumdepth}{3} % Number subsubsections, because we reference them,
% so the reader needs numbers to find the correct place.

\usepackage{amsfonts}
\usepackage{natbib}
\usepackage{breakcites}
\usepackage{authblk}
\usepackage{amsmath}
\usepackage{graphicx}
\usepackage{bm}
\usepackage{listings}

% MIT_TEMPLATE
\newif\ifMIT
%\MITtrue % uncomment to use MIT template
\MITfalse

%%%%% NEW MATH DEFINITIONS %%%%%

\usepackage{amsmath,amsfonts,bm}
\usepackage{xifthen}

% Highlight a newly defined term
\newcommand{\newterm}[1]{{\bf #1}}

\def\eps{{\epsilon}}


% Utility for ticks 
\newcommand{\cmark}{\ding{51}}%
\newcommand{\xmark}{\ding{55}}%

% Theorem styles 
\theoremstyle{definition}
\newtheorem{theorem}{Theorem}[section]
\newtheorem{definition}{Definition}[section]
% \newtheorem{remark}{Remark}[theorem] %numbered remark
\newtheorem*{remark}{Remark} %unnumbered remark
\newtheorem{lemma}{Lemma}[section]
\newtheorem{prop}{Proposition}[section]
\newtheorem{corollary}{Corollary}[theorem]
\newtheorem{conjecture}{Conjecture}[section]
\newtheorem{assumption}{Assumption}[section]

\newtheorem{manualtheoreminner}{Theorem}
\newenvironment{manualtheorem}[1]{%
  \renewcommand\themanualtheoreminner{#1}%
  \manualtheoreminner
}{\endmanualtheoreminner}


% Math helper - standard function
\DeclareMathOperator*{\argmax}{arg\,max}
\DeclareMathOperator*{\argmin}{arg\,min}
\DeclareMathOperator{\support}{support}
\DeclareMathOperator{\MAX}{MAX}
\DeclareMathOperator{\term}{\texttt{term}}
\DeclareMathOperator*{\logsumexp}{log-sum-exp}
\DeclareMathOperator*{\TV}{TV}
\newcommand{\norm}[1]{\left\lVert#1\right\rVert}
\DeclarePairedDelimiter\set\{\}
\DeclarePairedDelimiter\abs{\lvert}{\rvert}%
\newcommand*{\mytop}{\mathrel{\scalebox{0.5}{$\top$}}}
\newcommand*{\mybot}{\mathrel{\scalebox{0.5}{$\bot$}}}
\newcommand*{\mydiese}{\mathrel{\scalebox{0.5}{$\#$}}}
\newcommand*{\myplus}{\mathrel{\scalebox{0.5}{$+$}}}
\newcommand*{\myminus}{\mathrel{\scalebox{0.5}{$-$}}}
\newcommand*{\bmg}{\bm{\gamma}}
\newcommand*{\bml}{\bm{\lambda}}

% MDP notation
\renewcommand{\S}{\mathcal{S}}
\newcommand{\X}{\mathcal{X}}
\newcommand{\A}{\mathcal{A}}
\newcommand{\T}{\mathcal{T}}
\newcommand{\M}{\mathcal{M}}
\newcommand{\B}{\mathcal{B}}
\newcommand{\Bset}{\mathfrak{B}}
\newcommand{\Dist}{\mathscr{P}}
\newcommand{\D}{\mathcal{D}}
\newcommand{\Real}{\mathbb{R}}
\renewcommand{\P}{\mathcal{P}}
\newcommand{\E}{\mathop{\mathbb{E}}}
\renewcommand{\H}{\mathcal{H}}
% \newcommand{\R}{\mathcal{R}}
% \newcommand{\C}{\mathcal{C}}

% Extended MDP notation
\newcommand{\Pstar}{p^{\star}}
\newcommand{\Rstar}{\bm{r}^{\star}}
\newcommand{\Cstar}{C^{\star}}
% \newcommand{\rmax}{\textsc{Rmax}}
\newcommand{\rmax}{r_{\mytop}}
\newcommand{\cmax}{\textsc{Cmax}}

\newcommand{\mstar}{m^{\star}}
\newcommand{\mhat}{\hat{m}}
\newcommand{\mopt}{m^{\star}}

\newcommand{\Phat}{\hat{p}}
\newcommand{\Rhat}{\hat{\bm{r}}}
\newcommand{\Chat}{\hat{C}}

% Math helper - custom function
\newcommand{\expwrtpi}[1]{\E_{\pi} [\sum_{t=0}^{\infty} \gamma^t #1(s_t, a_t)]}
\newcommand{\expangle}[1]{\langle #1  \rangle}

% helper function for return and constraints

% for value function, takes arguments:
% #1: policy 
% #2: the function of interest, R or C_i
% #3 (optional): the MDP for which this is estimated
\newcommand{\V}[3]{ %
    \ifthenelse{\isempty{#3}}%
    {V^{#1}(#2)}% #3 is empty 
    {V^{#1}_{#3}(#2)}%
}

\newcommand{\Q}[3]{
    \ifthenelse{\isempty{#3}}
    {Q^{#1}(#2)}% #3 is empty 
    {Q^{#1}_{#3}(#2)}%
}


\newcommand{\Adv}[3]{
    \ifthenelse{\isempty{#3}}
    {A^{#1}(#2)}% #3 is empty 
    {A^{#1}_{#3}(#2)}%
}

% careful diff notation
% 1: pi
% 2: R/C
% 3: M
\newcommand{\J}[3]{
    \ifthenelse{\isempty{#3}}
    {\mathcal{J}^{#1}_{#2}}% #3 is empty -> eg V^{\pi}(x ; R)
    {\mathcal{J}^{#1}_{#3,#2}}% -? eg V^{\pi}_{M}(x ; C)
    % {J_{#2}(#1)}% #3 is empty 
    % {J_{#2}(#1, #3)} %
}



\newcommand{\MRkern}{%
  \mkern-6.5mu
  \mathchoice{}{}{\mkern0.2mu}{\mkern0.5mu}%
}

% for value function, takes arguments:
% #1: policy 
% #2: the function of interest, R or C_i
% #3 (optional): the MDP for which this is estimated
% #4: variables to be given input (x) or (x,a)
\newcommand{\val}[4]{ %
    \ifthenelse{\isempty{#3}}%
    {v^{#1}_{#2}(#4)}% #3 is empty -> eg V^{\pi}(x ; R)
    {v^{#1}_{#3,#2}(#4)}% -? eg V^{\pi}_{M}(x ; C)
    % {V^{#1}_{#3}(#4 ;#2)}% -? eg V^{\pi}_{M}(x ; C)
    % {V_{#2}(#4 ; #1)}% #3 is empty -> eg V_R(x ; \pi)
    % {V_{#2}(#4 ;#1, #3)}% -? eg V_C(x ; \pi, M)
    % {#2 \MRkern V^{#1}_{#3}(#4)}% -? eg V^{\pi}_{M}(x ; C) # combines the letter V and R together
}

\newcommand{\qval}[4]{
    \ifthenelse{\isempty{#3}}
    {q^{#1}_{#2}(#4)}% #3 is empty -> eg V^{\pi}(x ; R)
    {q^{#1}_{#3,#2}(#4)}% -? eg V^{\pi}_{M}(x ; C)
    % {Q^{#1}(#4 ; #2)}% #3 is empty -> eg Q^{\pi}(x,a ; R)
    % {Q^{#1}_{#3}(#4 ;#2)}% -? eg Q^{\pi}_{M}(x,a ; C)
    % {Q_{#2}(#4 ; #1)}% #3 is empty -> eg Q_R(x,a ; \pi)
    % {Q_{#2}(#4 ;#1, #3)}% -? eg Q_C(x,a ; \pi, M)
}
\DeclareMathOperator*{\advantage}{Adv}

\newcommand{\adv}[4]{
    \ifthenelse{\isempty{#3}}
    {\advantage^{#1}_{#2}(#4)}% #3 is empty -> eg V^{\pi}(x ; R)
    {\advantage^{#1}_{#3,#2}(#4)}% -? eg V^{\pi}_{M}(x ; C)
    % {A^{#1}(#4 ; #2)}% #3 is empty -> eg Q^{\pi}(x,a ; R)
    % {A^{#1}_{#3}(#4 ;#2)}% -? eg Q^{\pi}_{M}(x,a ; C)
    % {A_{#2}(#4 ; #1)}% #3 is empty -> eg A_R(x,a ; \pi)
    % {A_{#2}(#4 ;#1, #3)}% -? eg A_C(x,a ; \pi, M)
}




\newcommand{\ci}{C}

\newcommand{\pib}{\pi_{b}}
\newcommand{\piopt}{\pi^{*}}
\newcommand{\pie}{\pi_{t}}

\newcommand{\lR}{\lambda_{R}}
\newcommand{\lC}{\lambda_{C}}
\newcommand{\ephi}{e_{\phi}}

\newcommand{\pr}{\text{Pr}}
\newcommand{\IS}{\text{IS}}
\newcommand{\CI}{\text{CI}}


% SPIBB symbols 
\newcommand{\EpsPib}{(\pi_b, e, \epsilon)}

\title{NIPS 2016 Tutorial:\\ Generative Adversarial Networks}

\author{Ian Goodfellow}
\affil{OpenAI, {\tt ian@openai.com}}

\renewcommand\Authands{, and }


\date{}


% Make \[ \] math have equation numbers
\DeclareRobustCommand{\[}{\begin{equation}}
\DeclareRobustCommand{\]}{\end{equation}}

\begin{document}

\newlength{\figwidth}
\setlength{\figwidth}{26pc}

\maketitle

\begin{abstract}
\label{sec:abstract}

%% 1. what is the problem 
Scientific applications that run on leadership computing facilities often face the challenge 
of being unable to fit leading science cases onto accelerator devices due to memory constraints 
(memory-bound applications).
%
% 2. what is your solution 
In this work, the authors studied one such US Department of Energy mission-critical condensed matter 
physics application, Dynamical Cluster Approximation (DCA++), and this paper discusses how device memory-bound challenges were successfully reduced  by proposing an effective 
``all-to-all'' communication method---a ring communication algorithm. 
%
This implementation takes advantage of acceleration on GPUs and remote direct memory access (RDMA) for fast data exchange between GPUs. 
%
\\Additionally, the ring algorithm was optimized with sub-ring communicators
and multi-threaded support to further reduce communication overhead and 
expose more concurrency, respectively.
%
% 3. What's the cherry-picked evaluation result you want to mention
The computation and communication were also analyzed 
by using the Autonomic Performance Environment for Exascale 
(APEX) profiling tool,  and this paper further discusses the 
performance trade-off for the ring algorithm implementation. 
%
The memory analysis on the ring algorithm shows that the allocation size for the authors' most 
memory-intensive data structure per GPU is now reduced to $1/p$ of the original size, where $p$ is the number of GPUs in the ring communicator.
%
The communication analysis suggests that 
the distributed Quantum Monte Carlo execution time grows linearly as sub-ring size increases, and the cost of messages passing through the network interface connector could be a limiting factor.


%
% \todoRed{Ronnie: Next sentence needs rewrite, too much information about Green's function that no one knows in the abstract; recommend generalizing.} \emph {However, DCA++ is currently facing memory-bound challenge as 
% a larger device array $G_t$ is limited by device memory size, where
% $G_t$ is a two-particle Green's function that allows condensed matter
% scientists to explore larger and more complex (higher fidelity)
% physics cases.}

\end{abstract}

\keywords{DCA++, Quantum Monte Carlo, GPU Remote Direct Memory Access, memory-bound issue, exascale machines}


\section{Introduction}  \label{sec:introduction}

\newcommand\inexpIntro[3]{#1?(#2,#3).}
\newcommand\rinexpIntro[3]{*#1?(#2,#3).}
\newcommand\outexpIntro[3]{#1!(#2,#3).}
\newcommand\outatomIntro[3]{#1!(#2,#3)}

We propose a fully automated method for proving termination of \(\pi\)-calculus processes.
Although there have been a lot of studies on termination analysis for the \(\pi\)-calculus
and related calculi~\cite{Deng06IC,Demangeon07,SangiorgiTermination,KobayashiHybrid,Yoshida04IC,DBLP:journals/jlp/DemangeonHS10,Venet98SAS}, most of them have been rather theoretical,
and there have been surprisingly little efforts in developing  fully automated termination
verification methods and tools based on them. To our knowledge,
Kobayashi's \typical{}~\cite{TyPiCal,KobayashiHybrid} is the only exception that
can prove termination of \(\pi\)-calculus processes (extended with natural numbers)
fully automatically, but its termination analysis is quite limited (see Section~\ref{sec:relatedwork}).

Our method is based on a reduction to termination analysis for sequential programs:
we translate a \(\pi\)-calculus process \(P\) to a sequential program \(S_P\), so that
if \(S_P\) is terminating, so is \(P\). The reduction allows us to use
powerful, mature methods and tools
for termination analysis of sequential programs~\cite{heizmann2016ultimate,freqterm,DBLP:conf/lics/PodelskiR04,Kuwahara2014Termination,DBLP:journals/cacm/CookPR11}.

The idea of the translation is to convert a chain of communications on replicated input
channels to a chain of recursive function calls of the target sequential program.
Let us consider the following Fibonacci process:
\begin{align*}
    & \rinexpIntro{\fib}{n}{r}
        \ifexp{n<2}{ \soutatom{r}{1} \\ &\quad}
                   { \nuexp{s_1} \nuexp{s_2} (\outatomIntro{\fib}{n-1}{s_1} \PAR \outatomIntro{\fib}{n-2}{s_2} \PAR \sinexp{s_1}{x}\sinexp{s_2}{y}\soutatom{r}{x+y}) \\}
    & \PAR \outatomIntro{\fib}{m}{r}
\end{align*}
Here, the process
$\rinexpIntro{\fib}{n}{r} \ldots$ is a function server that computes the \(n\)-th Fibonacci number
in parallel and returns the result to \(r\),
and $\outatom{\fib}{m}{r}$ sends a request for computing the \(m\)-th Fibonacci number;
those who are not familiar with the syntax of the \(\pi\)-calculus may wish to consult
Section~\ref{sec:targetlanguage} first.
To prove that the process above is terminating for any integer \(m\),
it suffices to show that there is no infinite chain of communications on $\fib$:
\[
    \fib(m,r) \to \fib(m_1,r_1) \to \fib(m_2,r_2) \to \cdots.
\]
We convert the process above to the following program:\footnote{The actual translation
  given later is a little more complex.}
\begin{verbatim}
 let rec fib(n) = if n<2 then () else (fib(n-1) [] fib(n-2)) in
 fib(m)
\end{verbatim}
Here, \texttt{[]} represents the non-deterministic choice.
Note that, although the calculation of Fibonacci numbers is not preserved,
for each chain of communications on \texttt{fib}, there is a corresponding
sequence of recursive calls:
\[
\mathtt{fib}(m) \to \mathtt{fib}(m_1) \to \mathtt{fib}(m_2) \to \cdots.
\]
Thus, the termination of the sequential program above implies the termination of
the original process.
As shown in the example above, (i) each communication on a replicated input channel
is converted to a function call, (ii) each communication on a non-replicated input
channel is just removed (or, in the actual translation, replaced by a call of
a trivial function defined by \(f(\seq{x})=(\,)\)), and (iii) parallel composition
is replaced by a non-deterministic choice.
We formalize the translation outlined above and prove its correctness.

The basic translation sketched above sometimes loses too much information.
For example, consider the following process:
\begin{align*}
    & \rinexpIntro{\pre}{n}{r} \soutatom{r}{n-1} \\
    & \PAR \rinexpIntro{f}{n}{r} \ifexp{n<0}{ \soutatom{r}{1} }
                                       { \nuexp{s} (\outatomIntro{\pre}{n}{s} \PAR \sinexp{s}{x}\outatomIntro{f}{x}{r}) } \\
    & \PAR \outatomIntro{f}{m}{r}
\end{align*}
The translation sketched above would yield:
\begin{verbatim}
  let pred(n) = n-1 in
  let rec f(n) = if n<0 then () else (pred(n) [] f(*)) in
  f(m)
\end{verbatim}
Here, \texttt{*} represents a non-deterministic integer: since we have removed
the input $\sinatom{s}{x}$, we do not have information about the value of \( x \).
As a result, the sequential program above is non-terminating, although the original
process is terminating.
To remedy this problem, we also refine the basic translation above by using a refinement
type system for the \(\pi\)-calculus. Using the refinement type system,
we can infer that the value of \(x\) in the original process is less than \(n\),
so that we can refine the definition of \texttt{f} to:
\begin{verbatim}
 let rec f(n) = ... else (pred(n) [] let x=* in assume(x<n);f(x))
\end{verbatim}
The target program is now terminating, from which
we can deduce that the original process is also terminating.
We have implemented an automated tool based on the refined translation above.

The contributions of this paper are summarized as follows.
\begin{itemize}
\item The formalization of the basic translation from the \(\pi\)-calculus
  (extended with integers) to sequential programs, and a proof of its correctness.
\item The formalization of a refined translation based on a refinement type system.
\item An implementation of the refined translation, including automated refinement type
  inference based on CHC solving, and experiments to evaluate the effectiveness of
  our method.
\end{itemize}

The rest of this paper is structured as follows.
Section~\ref{sec:targetlanguage} introduces the source and target languages
of our translation.
Section~\ref{sec:approach} 
formalizes the basic translation, and proves its correctness.
Section~\ref{sec:refinement} refines the basic translation by using a refinement type system.
Section~\ref{sec:implementation} reports an implementation and experiments.
Section~\ref{sec:relatedwork} discusses related work,
and Section~\ref{sec:conclusion} concludes the paper.

\documentclass[12pt]{article}
\pagestyle{plain}

\pdfoutput=1

\usepackage{color}
\usepackage{graphicx}
\usepackage{amsmath}
\usepackage{amssymb}
\usepackage{xspace}
\usepackage{slashed} %
\usepackage[small]{subfigure}
\usepackage[numbers,compress]{natbib}
\usepackage{epigraph}
\usepackage{tikzsymbols} %

 \usepackage{etoolbox}
\makeatletter
\newlength\epitextskip
\pretocmd{\@epitext}{\em}{}{}
\apptocmd{\@epitext}{\em}{}{}
\patchcmd{\epigraph}{\@epitext{#1}\\}{\@epitext{#1}\\[\epitextskip]}{}{}
\makeatother

\setlength\epigraphrule{0pt}
\setlength\epitextskip{2ex}
\setlength\epigraphwidth{1.2\textwidth}

\usepackage[hyperfootnotes=false]{hyperref}

\usepackage{mciteplus} 

\setlength{\topmargin}{-.3in} \setlength{\oddsidemargin}{.0in}
\setlength{\textheight}{8.5in} \setlength{\textwidth}{6.35in}
\setlength{\footnotesep}{\baselinestretch\baselineskip}
\newlength{\abstractwidth}
\setlength{\abstractwidth}{\textwidth}
\addtolength{\abstractwidth}{-6pc}
\renewcommand{\title}[1]{\vbox{\center\bf{\Large{#1}}}\vspace{5mm}}
\renewcommand{\author}[1]{\vbox{\center#1}\vspace{5mm}}
\newcommand{\address}[1]{\vbox{\center\em#1}}
\newcommand{\email}[1]{\vbox{\center\tt#1}\vspace{5mm}}
\newcommand{\sbreak}{
    \begin{center}
        $\blacklozenge$$\blacklozenge$$\blacklozenge$$\blacklozenge$
    \end{center}
}
\newcommand{\be}{\begin{equation}}
\newcommand{\ee}{\end{equation}}
\newcommand{\bi}{\begin{itemize}}
\newcommand{\ei}{\end{itemize}}
\newcommand{\DR}[1]{{\sl{\textcolor{blue}{#1}}}}


\newcommand{\ket}[1]{| #1 \rangle }
\newcommand{\bra}[1]{\langle #1 |}

\newcommand{\intlim}[3]{\int_{#1}^{#2}\! \df #3 \,}

\newcommand{\img}{\mathrm{i}}
\renewcommand{\Im}{\mathrm{Im}}
\newcommand{\tr}{\text{tr}}

\newcommand{\OO}{\mathcal{O}}
\newcommand{\CM}{\mathcal{M}}
\newcommand{\CW}{\mathcal{W}}
\newcommand{\CC}{\mathcal{C}}
\newcommand{\CL}{\mathcal{C}}
\newcommand{\CD}{\mathcal{D}}
\newcommand{\CB}{\mathcal{B}}

\newcommand{\lads}{\ell_{AdS}}
\newcommand{\TFD}{|TFD\rangle}
\newcommand{\T}[1]{\langle #1 \rangle_\beta}

\renewcommand{\le}{\left}
\newcommand{\ri}{\right}
\newcommand{\Ti}[3]{#1_{#2}^{(#3)}} %
\newcommand{\bias}[2]{\Ti{b}{#1}{#2}} %
\newcommand{\tia}[3]{#1_{#2; #3}} %
\newcommand{\W}[2]{\Ti{W}{#1}{#2}} %
\newcommand{\x}[2]{\tia{x}{#1}{#2}} %


\begin{document}

\begin{titlepage}
\rightline{MIT-CTP/5269}
\begin{center}
\hfill \\
\hfill \\
\vskip .5cm

\title{Why is AI hard and Physics simple?}

\author{Daniel A. Roberts}

\address{
    Center for Theoretical Physics and Department of Physics,\\ Massachusetts
  Institute of Technology,\\ Cambridge, MA 02139, USA
 \\~\\
 Salesforce, Cambridge, MA 02139, USA
}

\email{drob@mit.edu}


\end{center}



\begin{abstract}


We discuss why AI is hard and why physics is simple. %
We discuss how physical intuition and the approach of theoretical physics can be brought to bear on the field of artificial intelligence and specifically machine learning. We suggest that the underlying project of machine learning and the underlying project of physics are strongly coupled through the principle of sparsity, and 
we call upon theoretical physicists to work on AI \emph{as physicists}.
As a first step in that direction, we discuss an upcoming book on the principles of deep learning theory that attempts to realize this approach \cite{Principles}.


\end{abstract}


\end{titlepage}


\tableofcontents

\section{Some Questions}\label{sec:introduction}
\epigraph{[T]he mathematical formulation of the physicist's often crude experience leads in an uncanny number of cases to an amazingly accurate description of a large class of phenomena.}{Eugene Wigner, ``The Unreasonable Effectiveness of Mathematics in the Natural Sciences''}

\noindent{}In a famous essay \cite{wigner1960unreasonable}, Eugene Wigner observed the important role that mathematical structure plays in describing physical theories. Wigner was both a mathematician and a physicist, and most of the engagement with his essay
focuses on the ``mathematical formulation'' part of the quote, specifically on the puzzle of why mathematics and physics should have anything to do with each other.

Our engagement will be different. Ignoring the mathematician's appeal to rigorous (re)formulation, we focus on the latter part of the quote: ``The physicist's often crude experience leads in an uncanny number of cases to an amazingly accurate description of a large class of phenomena.''


Why is this the case? What can physicists bring to other fields that have a natural description in terms of mathematics, such as artificial intelligence? %
Can they bring anything that isn't already known by mathematicians?
Given the accumulated success of the mathematical formulation of physics, perhaps we might hope to also find 
physics in other useful mathematical structures?




In this essay, we will attempt to identify the aspects of artificial intelligence (AI) that make it so hard (\S\ref{sec:ai-hard}). 
In particular, our focus will be on machine learning approaches to artificial intelligence in which functions are \emph{trained} rather than explicitly \emph{programmed}.
We will find that the difficulty is in precisely understanding the sparse mathematical structure 
that must be imposed upon any prospective 
machine learning model in order to make it tractable. This structure can be somewhat understood through a combination of intuition and analogy to human intelligence, but so far has appeared to resist any principled theoretical approach.%

Setting that aside for a moment, we will then comment on why a ``physicist's crude experience'' leads to such accurate %
and predictive theories (\S\ref{sec:physics-simple}). The physicist's engagement with math is not that of the mathematician, because the physicist is constrained by both theory \emph{and} experiment. It is precisely the interplay between the two that gives the physicist's  mathematical formulations such power in their descriptive abilities. In short, we will explain 
why physics is simple enough to enable immense progress in understanding
phenomena in nature.\footnote{
    Please note that the notion of \emph{simplicity} to which we are appealing is not at all meant to suggest that physics is \emph{trivial}. Instead, we mean it as a compliment: we have the utmost respect for the work of physicists. This essay is an apologia for physicists doing machine learning \emph{qua physicists}; it is meant to interrogate what it is about the approach or perspective of physicists that allows them to reach so far in explaining fundamental phenomena and then consider whether that same approach could be applied more broadly to understanding intelligence.
}



 
In many ways, machine learning and physics are two sides of the same coin.
The program of machine learning is the selection and fitting of mathematical models to describe some desired phenomena usually encoded in a probability distribution.  Traditionally, that is also a description of the program of physics, provided we specify that the phenomena are \emph{natural}.  

Perhaps with foresight, Wigner speaks of the physicist's success in describing phenomena, without bothering to limit that class to those that are natural.\footnote{
    One could argue that the field of condensed matter physics is mainly engaged in studying \emph{unnatural} phenomena. And thermodynamics grew out of the investigation of the \emph{artificial machines} of the steam age, eventually finding an underlying microscopic description in statistical mechanics. 
    All together, this is rather suggestive that the language and methods of physics might be particularly well suited for learning principles of machine learning. 
    We thank Sho Yaida for making and developing this point about physics and unnatural phenomena.
}  
It's our thesis that the tools and intuitions of the physicist can have a much broader domain of applicability, and 
we hope that more theoretical physicists will be open to making contributions to the field of AI by studying machine learning models \emph{as physicists}. Given the importance of tying such theoretical work directly to experiment, we anticipate that such efforts could have a direct impact on models at the forefront of AI research. 


In light of our discussion of simplicity in physics, 
in \S\ref{sec:do} we 
speculate on the ways in which 
physics 
approaches to understanding
intelligence 
might be useful, and  in \S\ref{sec:ETofDL} we connect our ideas to the approach of an upcoming book \cite{Principles} that attempts to %
develop some principles of 
deep learning 
theory 
as a concrete  
example of what such a contribution could look like.







\subsection{Why is AI hard?}\label{sec:ai-hard}


\noindent{}Artificial Intelligence is hard because there is no such thing as a free lunch. 

This is the well-known oft-quoted statement 
that, when you average over all possible machine-learning problems, 
all learning algorithms are equivalent \cite{wolpert1996lack,wolpert1997no}.

For example, the number of possible $n$-pixel images, with each pixel being either black or white, is $2^n$: an exponentially large number. If we give each image one of two labels, then the number of possible ways of labeling the set of images is $2^{2^n}$: a doubly-exponentially large number. To get a sense of what this means, for even modest values of $n$ this is a absurdly large number: since $2^{2^{9}} \sim 10^{154}$, there are a preposterously greater number of ways to classify the set of 9-bit black and white images than there are atoms in the observable universe $\sim 10^{80}$ \cite{eddington2012philosophy}. 

With that in mind, if each image could in principle receive a different label without leveraging any notion of structure or relationships between the images -- i.e.~if the labels don't correlate in any way with the detailed properties or \emph{features} of the image -- then the best one could hope to do is memorize each image with its label. 
This is the essence of the no-free-lunch theorem: \emph{(i)} 
the task of memorizing images with their labels grows so absurdly quickly that it's more or less impossible for any interesting problem size, and \emph{(ii)} it is impossible to devise a scheme or algorithm that does better than such explicit memorization.

  
This seems pretty bleak. One reading of this theorem seems to be that no matter how we try to improve our methods and tools for machine learning, we can never do better than random. A learning algorithm that predicts the label $y$ on a previously unseen image $x$ and another algorithm that predicts the opposite label $\neg y$ will do equally as well in the long run if 
we average over all possible unseen data points $(x,y)$. 
Thus, we might as well just make a random prediction; in other words, pure and complete \emph{generalization} is impossible.



Yet humans learn very efficiently \cite{gopnik1999scientist,griffiths2006optimal,vul2014one}, and machine learning -- in particular realized via  artificial-neural-network-based function approximation: \emph{deep learning} --  appears to work as well \cite{lecun2015deep,ImageNet2012,goodfellow2014generative,mikolov2013efficient,mikolov2013distributed,BERT2018,radford2019language,Brown2020LanguageMA,mnih2015human,silver2016mastering,silver2018general,berner2019dota,starcraft,muzero,alphafold2prelim}. Once you 
return from absorbing the content in both citation dumps, you have to admit that there's a bit of tension between the no-free-lunch theorem and the apparent success of intelligence, both human and artificial.




In fact, the no-free-lunch theorem does not mean that there is no point in trying; rather, it implies that there's no \emph{best} learning algorithm \emph{only} if we average over all possible inputs to our problem.
For example, in our discussion above the label of an image $x$ is equally likely to be $y$ and $\neg y$ only if we average over all logical possibilities of such labelings. 

But such averaging is clearly silly; any label that's meaningful to a human \emph{does} correlate with the properties of the image: \Cat is a \texttt{cat}. It has whiskers and pointy ears, just like many other images of cats. Knowing the right label for this input is absolutely useful in predicting the labels of future images, since the labels are not at all random. Instead, the actual learning problems that we care about -- which are often the ones related to human intelligence -- have quite a lot of structure.





\subsubsection*{Human learning}

If we return to our exponential -- $2^n$ -- counting of the number of different $n$-pixel images,
to a human a huge fraction of these images will look like utter \emph{noise}. 
Therefore, whatever learning algorithm humans use to learn to characterize, understand, and interpret images can only ever apply to a small fraction of the possible images, and so only the tiniest fraction of the doubly-exponential potential labelings will ever be learnable by a human. Learning to classify only within that subset escapes the no-free-lunch theorem, and so humans are demonstrably able to learn things \cite{gopnik1999scientist,griffiths2006optimal,vul2014one}.



For such human learning, it has been argued that both pattern matching -- e.g.~``this image looks very similar to this other one that I've seen before, so it must also be a cat'' -- and domain specific knowledge -- e.g.~``cats have whiskers and fur and pointy ears, and so this image of an object with whiskers, fur, and pointy ears must also be a cat'' -- are essential when generalizing from sparse data \cite{tenenbaum2006theory}. Together, these make up 
the \emph{inductive bias} of a learning algorithm.
Such a bias makes \emph{a priori} assumptions about the nature of the problem, such as which features of the data are useful.

For instance, images may contain objects. Objects are made up of large components, each of which may be made up of subcomponents; recall again the relationship between whiskers and pointy ears, and the way we recognize that \Cat is a \texttt{cat}.
Images that have such a structure 
are easy to classify, both because we often understand how the components make up any object that we care about, and also because organizing objects in terms of a sequence of such \emph{representations} makes it easy for us to recognize statistical patterns among  different images at different levels of coarse-graining. 
Without this hierarchical structure, any classification task would be hopeless.


\subsubsection*{Machine learning}

Even if some of the images within the enormous set of \emph{other} images have some kind of interesting structure, a human wouldn't be able to recognize it -- in this sense, \emph{noise} is just a catch-all concept for images that are without meaning to humans. Our inductive biases only let us find \emph{human-meaningful} representations.   So how do we tell such models what gives meaning to humans?



From this perspective, the task of coming up with a learning algorithm is closely tied to the task of modeling the world. One approach suggests that we should rely on human intuition and proposes building artificial models that try to learn the way that humans learn \cite{tenenbaum2011grow,lake2017building}
rather than relying on unstructured algorithms designed to recognize statistical patterns in data. 

However, the approach based on mere ``pattern recognition'' is wildly successful \cite{ImageNet2012,goodfellow2014generative,mikolov2013efficient,mikolov2013distributed,BERT2018,radford2019language,Brown2020LanguageMA,mnih2015human,silver2016mastering,silver2018general,berner2019dota,starcraft,muzero,alphafold2prelim} despite the fact that machine learning algorithms can often find patterns in data without any human-meaningful structure. For instance, many deep learning models can memorize completely random labelings of noise \cite{zhang2016understanding}, a task that would be utterly impossible for a human.
This leaves us to wonder: what is the inductive bias of such models if they succeed at tasks that are otherwise impossible for humans, and by what mechanism are they able to %
escape the no-free-lunch theorem? 

This is why (understanding how) AI (works) is hard.


















\subsection{Why is Physics simple?}\label{sec:physics-simple}




\noindent{}Physics is simple because there is no such thing as a free lunch.\footnote{Yes, that's the same lunch. Just a different interpretation. You could say that this is the unknown never-quoted statement 
that, when you average over all possible \emph{physics-learning} problems, 
all physicists are equivalent \emph{[citation needed]}.
}





That is, the reason the laws of physics are even learnable at all is because the mathematical models or \emph{theories} that offer good descriptions of the universe are especially simple models within the general frameworks that we use to enumerate the possible theories of physics. 
This is another way to interpret Wigner's observation: the laws of physics actually have an (unreasonable?) lack of \emph{algorithmic complexity}. 

Before directly addressing why our laws of physics really had to be simple in the way that they are, let's first understand where this lack of algorithmic complexity comes from. 


\subsubsection*{Interacting particles}


The principal framework used to express different theories of fundamental physics is known as
\emph{quantum field theory}. %
To specify a particular theory within the framework, we first enumerate the 
degrees of freedom -- i.e.~the different elementary particles -- and then determine how they \emph{interact} with each other. 
A general interaction involves the creation of some particles and the destruction of some other particles at a particular point in space and time.

For example, the theory of \emph{quantum electrodynamics} is a quantum field theory that describes how matter and light interact. The basic matter particles are the electron and the positron, and the particle of light is called the photon. The strength of their interaction is set by the electric charge, which in our universe is more or less given by the dimensionless number $1/137$.  A cartoon example of such an interaction, called a Feynman diagram, 
is shown in Fig.~\ref{fig:feynman} in which an electron and positron are destroyed and two photons are created. Making use of this and many similar diagrams, an ensemble
 of grad students could predict the result of essentially any experiment involving light and matter.


\begin{figure}[h]
\begin{center}
\includegraphics[scale=.4]{ep.pdf}
\caption{An interaction involving $4$ particles: an electron $(e^-)$ and positron $(e^+)$ are destroyed, and 2 photons $(\gamma)$ are created.}
\label{fig:feynman}
\end{center}
\end{figure}





The most generic theory expressed in the quantum field theory framework would have all possible combinations of interactions between all possible particles. For instance, if we have $p$ different types of particles, then
there are $\binom{p}{2}\sim p^2$ potential interactions involving $2$ particles, $\binom{p}{3}\sim p^3$ interactions involving $3$ particles, $\binom{p}{4}\sim p^4$ interactions involving $4$ particles, and so on.\footnote{We're lying here since in principle these particles can also \emph{self-interact}. (This just means that the quantum analog of the classical probability distributions that describes such particles has non-Gaussian statistics.) Taking this into account, the left-hand side of these $\sim$ expressions is false, but the right-hand side is still approximately correct.
}
Each of these interactions has a strength or \emph{coupling}, 
and a generic quantum field theory would have random nonzero values for the strength of each of these interactions.






In fact, things can get quite complicated as quantum field theories sometimes have an infinite number of degrees of freedom per particle type -- you might recognize from familiar everyday phenomena like \emph{turning up the lights} --  and a priori the strength of these interactions between all the degrees of freedom could vary in space and in time. So in some sense, the discussion in the previous paragraph was a vast under-counting of the number of potential couplings!

\subsubsection*{Typicality}


In many cases we can still effectively model such systems with a finite number $N$ degrees of freedom.\footnote{For instance, the number of states accessible at a given energy is finite, so systems with a bounded energy can be modeled with a finite number of degrees of freedom. If a concrete model is helpful for you, imagine a system of $N$ spins.}
In the absence of any guiding principle, each degree of freedom in a \emph{typical} theory can either participate or not in an interaction. Using the binomial theorem to add up all these interactions, this more or less gives $\binom{N}{2} + \binom{N}{3}  + \ldots + \binom{N}{N}\approx 2^N$ different potential interactions between the $N$ degrees of freedom. Thus, generically the number of possible interactions and therefore parameters will scale as an exponential in the size of the system.\footnote{
    If you're familiar with the Ising model, then perhaps the following is helpful: if we have $N$ spins, and there's no restriction on the interactions, then there's $2^N$ possible independent interaction terms in the Hamiltonian based on whether a particular spin is present or not in the interaction term. This is akin to a completely random Hamiltonian on $N$ degrees of freedom.
} %

As we will see, such typical theories with a number of parameters that scales exponentially with the number of degrees of freedom do not provide a good model of the physics in our universe. Concretely, the number of experiments that we'd have to perform in order to learn all those parameters is about as large as the total number of experiments we could ever perform on such a system. But if we have to perform just about every possible experiment, then such a theory has no predictive power! 

In other words, the results of one experiment would not really correlate with the results of another. Such a theory has no generalization property: learning the parameters of such a theory is tantamount to simply memorizing a look-up table with the outcomes of all such experiments.
This means that such theories are about as complicated as they could be, and the algorithmic complexity required to specify them essentially renders them useless. 

Don't worry if this reminds you a bit of the discussion in \S\ref{sec:ai-hard} -- that's typical.















\subsubsection*{Sparsity}

The key error we just made was assuming the ``absence of any guiding principle'' in determining our theory. The most important such principle is \emph{sparsity}: the number of parameters defining a theory should be far less than the number of experiments that such a theory describes. The fact that the relevant quantum field theories for the laws of physics in our universe are sparse is why they are even useful.

In the most general application of the principle of sparsity, we limit the number of degrees of freedom that can participate in any interaction. For instance, for a theory describing $N$ degrees of freedom we might limit interactions to $k$ degrees of freedom at a time, for $k \ll N$. If the cartoon in Fig.~\ref{fig:feynman} is still helpful, this would mean that we'd only need to consider diagrams with $k$ or fewer legs.

Now, let's add up the number of interactions in such a sparse theory. If the number of degrees of freedom is large ($N \gg 1$) then such a sum is dominated by the $\binom{N}{k}\sim N^k/k!$  term: the number of parameters only scales polynomially in the number of degrees of freedom.

When a sparse quantum field theory is a good model for some natural phenomenon, it's incredibly useful. If we perform the fractionally small number of experiments needed to learn all the parameters, we can then predict the result of the huge fraction of experiments we've yet to perform. From an algorithmic complexity standpoint it is somewhat miraculous that we can compress our huge look-up table of experiment/outcome into such an efficient description. 
In many senses, this type of compression is precisely what we mean when 
we say that physics enables us to understand a given phenomenon.



















\subsubsection*{Locality}
The general sparsity property that we just discussed is sometimes called \emph{$k$-locality}. This emphasizes that no more than $k$ of the $N$ degrees of freedom can communicate in any given type of interaction. We impose such sparseness on our theories because it's consistent with our experimental observations.\footnote{
    The phenomenon of \emph{universality} provides an explanation for this $k$-locality, though not for spatial locality. See footnote~\ref{footnote:universality} for a brief discussion of how universality is manifested in deep learning.
}
However, it's actually far too weak a condition: typical theories that have only a sparse $k$-local structure imposed don't actually look anything like the theories that usefully describe our universe. 

What we actually observe experimentally is much stronger than $k$-locality. 
In particular, our natural intuition about the world suggests the degrees of freedom should be arranged in some manner such that some are closer together than others, and that this closeness is somehow related to their ability to interact and exchange information. This is intuitive; an experiment performed on the moon should not influence experiments performed on earth, at least not until signals from the moon containing information about such an experiment can reach the earth. 

What we need is to also impose \emph{spatial locality}. Rather than letting the $N$ degrees of freedom interact however they like so long as they're quarantined in group sizes of $k$ or fewer, we first arrange all $N$ in a lattice and then say that they can only interact with their neighbors. For instance, if we have a $1$-dimensional lattice -- a line -- each of the $N$ degrees of freedom has a left and right neighbor. More generally, if we have a $d$-dimensional hypercube then each degree of freedom will have $2d$ neighbors. To count the number of distinct interactions, we simply add up the links on the lattice and find $Nd$ total interactions: the number of possible interactions is linear in the number of degrees of freedom. 



In fact, the entire concept of \emph{spatial dimension} is only meaningful because of the spatial locality of interactions! If degrees of freedom were placed at different spatial locations -- i.e.~the moon and earth -- but they could talk to each other as easily as if we placed them next to each other, then the whole notion of space itself is meaningless.\footnote{
    Relatedly, locality gives meaning to moon-landing conspiracists everywhere.
} The reason \emph{space} is a useful abstraction is because \emph{locality} is imposed on quantum field theories.\footnote{
Einstein was famously bothered by a property of quantum mechanics that naively seemed to allow actions in one place to have an effect on actions somewhere else at the same instant.
You've likely heard of this as \emph{spooky action at a distance}, though to Einstein it would have probably sounded more like \emph{spukhafte Fernwirkung} \cite{einsteinborn}.
In a paper with Podolsky and Rosen \cite{Einstein:1935rr}, Einstein was worried that this uniquely quantum phenomenon -- which is now known as \emph{entanglement} -- would violate the principle of locality. 

In particular, any local 
quantum field theory -- according to the principle of relativity -- has a rigid notion of causality that says that the outcome of an experiment at a particular point $x$ and time $t$ cannot at all influence the outcome another experiment at a different point $x'$ and time $t'$, unless the absolute distance between these experiments is less than the time interval between them times the speed of light: $|x-x'| < c|t-t'|$ \cite{Streater:1989vi}. We now appreciate that entanglement \emph{(a)} does not violate causality and \emph{(b)} arises from completely local interactions \cite{bell1964einstein}. Quantum effects are just sometimes weird and unintuitive.

Amusingly, gravitational objects known as \emph{wormholes} -- discussed by Einstein and Rosen around the same time \cite{Einstein:1935tc}  -- have a history of being confused with entanglement. A wormhole is a special type of geometry that provide a shortcut through space so that vastly separated regions can be more closely connected together. It turns out that those of us that were confused weren't actually confused, and the joke was on everyone else: both entanglement and wormholes are actually manifestations of the same underlying phenomenon \cite{Maldacena:2013xja}.
}






\subsubsection*{Translation invariance}

Our final guiding principle is called \emph{translation invariance}. Translation invariance is a \emph{symmetry} imposed on 
quantum field theories that are used to model phenomena that respect
the following empirical observation: the outcome of any experiment is generally unaffected by where it is performed. The same general laws that describe physics here on earth also describe physics in the same way on the moon and also describe physics in the same way outside our galaxy. 

To implement this principle, we require that the strength of any local interaction should be the same anywhere.\footnote{In the language of machine learning, local translationally-invariant quantum field theories enforce a very rigid notion of parameter sharing!} In other words, all the parameters have to be the same.\footnote{
    As a specific example, arranging $N$ spins into a $d$-dimensional hypercube and imposing translation invariance would leave us with the standard $d$-dimensional Ising model and only $1$ parameter.
} 
 Concretely, the electric charge we measure while at rest on the earth will be the same as the electric charge we'd measure at rest on the moon: $\approx 1/137$.

Let's spend a moment to take stock of this huge reduction in algorithmic complexity. We started with a \emph{typical} theory of $N$ degrees of freedom that requires an exponential -- i.e.~$2^{O(N)}$ -- number of parameters to specify. By imposing the most generic notion of sparsity, \emph{$k$-locality}, we reduced this to a polynomial -- i.e.~$O(N^k)$ -- number of parameters. Then by imposing $d$-dimensional \emph{spatial locality}, we reduced this to a linear -- i.e.~$O(N)$ number of parameters. Finally, by imposing \emph{translational invariance}, we could specify our theory with an $O(1)$  number of parameters. For such theories, the number of parameters no longer grows with the number of degrees of freedom: once we measure the electric charge, we can more or less predict the outcome of any other experiment involving light and matter.\footnote{That is, if we ignore the mass of the electron, which is small enough that in many cases doing so would be completely justified. Otherwise, we have an additional parameter to measure: in some sense, the electron mass can be thought of as the strength of a 2-particle ``interaction'' between an electron and a positron and \emph{no} photons.\label{footnote:electron-mass}}
The fact that the universe seems to obey these constraints is very powerful indeed!






 












\subsubsection*{Complexity}
To conclude our discussion of simplicity in physics, let's discuss how nontrivial complexity can arise from a sparse mathematical model. 



In any sparse theory, we can study how small perturbations to some observables can grow and affect the rest of the system \cite{Lieb:1972wy,Sekino:2008he,Maldacena:2015waa}. Sparse theories have a preferred basis: the set of variables in which the formulation of the theory is sparse is special. In terms of such variables, the description of the theory is extremely simple.

In a theory with spatial locality, it's simple to perform an experiment by disturbing a local degree of freedom and observe how the system responds. The particulars of such a disturbance will grow in space and slowly spread to change the  overall state of the system: this is a quantum manifestation of chaos and the butterfly effect \cite{Larkin:1969abc,Maldacena:2015waa}.

An important property of any system -- which sometimes goes by the name \emph{thermalization} -- is how quickly it returns to equilibrium after such a disturbance. Depending on how detailed an experiment you perform, this question has a variety of different answers, ranging from a time that is independent of the number of degrees of freedom $N$ to a time that scales with the size of the system and tracks how long it takes for the disturbance to cover the entire system. If you want to find the system in a \emph{truly} random state, you might have to actually wait a time that is exponential in the number of degrees of freedom \cite{Susskind:2015toa}.

Despite this long timescale, the fine details of a random state as compared to a simple-but-perturbed state are extremely difficult to measure. In the simple-but-perturbed state, the correlations between what the original state counterfactually would have been and what the disturbed state actually is quickly grow to become more and more complicated as the difference spreads throughout the system.  However, because we can only perform local experiments, 
it's actually often difficult for the results of any such experiment to determine whether a given state is truly from a random theory or just approximately random and built up from a sparse theory \cite{Sekino:2008he,Hayden:2007cs,Lashkari:2011yi,Brandao:2012qcp,Brown:2012gy}.
That is, as far as we can actually tell, a system can be arbitrarily complicated even as it continues to change meaningfully for a long time.\footnote{For a different perspective on simplicity and complexity in physics, with connections to spin glasses and string theory, see the introduction of \cite{Denef:2011ee}.}




By contrast, without locality, there's no spatial structure, and without sparsity, there's no real structure at all. In such a completely generic theory, 
there aren't any special set of observables for experimenters to measure, track, or query.
In these systems, any disturbance of a single degree of freedom immediately grows to affect the entire system: the equilibration time is completely independent of the system size, regardless of how big it is! 
Thus, %
it's not really meaningful to observe \emph{anything}, at least in the usual sense of what we mean. In these systems, nothing happens!\footnote{The global statistical properties of these systems can have interesting time evolution see e.g. \cite{Cotler:2016fpe}. However, such observables are so overly complicated that they do not correspond to any experiments that %
could be practically performed on such systems.
} 

\subsubsection*{Physics learning}

The reason sparsity and more specifically locality are so important is because they provide structure to the space of physics theories. In fact, physical constraints due to these principles play an important role in setting which experiments can be practically carried out or algorithmically which computations are possible in the universe.\footnote{
It would be interesting to consider such constraints when assessing concerns about human existential risk due to artificial intelligence, see e.g.~\cite{bostrom2014superintelligence}.  In particular, when you ``imagine'' an agent with unbounded computational resources you should also make sure to imagine what kind of universe such an agent would have to live in. (Such a universe is entirely unlike our own \cite{Bekenstein:1980jp,tHooft:1993gx,Susskind:1994vu,Sekino:2008he,Shenker:2013pqa,Maldacena:2015waa,Brown:2015bva,Brown:2015lvg}.) For an even more speculative discussion on how physical limits to information theory may constrain or place computational limits on thinking agents, see \cite{roberts2018causality}.
}
Just as the inductive bias of human learning enables us to find \emph{human-meaningful} structure in images, sparsity and locality make up the inductive bias that lets us learn physics.\footnote{
Relatedly, since the way in which we typically interact with the world is classical -- i.e.~devoid of distinctly quantum effects -- quantum mechanics is naturally unintuitive.}  

In other words, we impose sparseness on our theories because it's consistent with our experimental observations as local observers that themselves must be described by the theory. You could say: sparseness is what the universe finds when it introspects.
Since humans are local entities within the universe, the sort of experiments that we can reason about theoretically and easily perform practically -- the sort of experiments that we have any intuition about at all -- are all based on local correlations. No wonder that the theories that we learn are designed to predict the results of local experiments!













What about all those other experiments? More or less, they are the analogues of \emph{noise}: they have no meaning to local observers like us humans. Many of them involve correlating a huge fraction of the $N \gg 1$ degrees of freedom of the system across vast regions of space. Some other experiments are impossible to perform in any reasonable amount of time by any observer within a system.\footnote{An example of such an experiment that relates to an important black hole information problem \cite{Almheiri:2012rt} is given in \cite{Harlow:2013tf}.
}  
Thus, when the observers are part of the system they are trying to learn about, the lack of algorithmic complexity of such a theory imposed by locality can privilege certain experiments and prevent others from being performed. 
The results of all such essentially impossible experiments are heavily constrained by our algorithmically simple theories of physics, though we'll never be able to check for sure!





































This brings us back to the no-free-lunch theorem: the catch is that here the ``average over all possible \emph{physics-learning} problems'' is like an average over all possible experiments. And it turns out that the set of experiments that 
 actually are interesting is a tiny subset of the total number of logically possible experiments. 

 This is why physics is (able to offer) simple (explanations for seemingly 
 complicated natural phenomena).






























































































































\subsection{What can Physics do for AI?}\label{sec:do}
In summary, \emph{physics learning} -- just like \emph{human learning} more generally -- is possible because of sparsity. In the strongest but most speculative sense, such a principle is a precondition for there to be local observers capable of performing experiments. But it's still not clear from this discussion how to apply this to \emph{machine learning}.






As we mentioned, one approach to understanding machine learning is through human learning \cite{tenenbaum2011grow,lake2017building}. This is the approach of the cognitive scientist.
Such research could lead to understanding high-level principles of intelligence applicable across many models of intelligent systems, regardless of their underlying mechanisms. If the goal of machine learning is an artificial human-level intelligence, then these sorts of \emph{top-down} considerations can be incredibly useful for guiding the direction and focus of AI research.


By contrast, the physics approach is \emph{bottom up}: the focus is on understanding how the algorithmic description of a model is connected to the model's output or experimental results.
Starting from the definition of a model in terms of its microscopic degrees of freedom -- e.g.~the weights and biases of a deep neural network -- what is the high-level behavior of the model? How are these microscopic variables connected to the \emph{latent variables} in which the model has a simple description? Such an approach isn't directly concerned with the general properties of -- or constraints guiding -- human intelligence, but rather focuses on directly understanding how the specification of a model can be connected to its ability to complete a task.
In this way, the physics approach can provide a natural complement to the cognitive science approach.




Applying this bottom-up approach directly to humans is far too complicated; the path from microscopic physics through biology and neuroscience to psychology is probably too daunting for any entity short of a superintelligence\dots

Applying this bottom-up approach to machine learning, however, is likely just right. We propose that physicists interested in working out aspects of a theory of intelligence could make progress by working out a theory of current successful machine learning models through the lens of physics.
On the one hand, by studying current state-of-the-art machine learning frameworks we can study algorithms that share some properties with human intelligence.  On the other hand, the output and behavior of these algorithms aren't separated from their microscopic descriptions -- e.g.~their PyTorch \cite{paszke2017automatic} implementations -- by something as messy as biology. 



The reason to be hopeful is that for these models to work at all given the no-free-lunch theorem, there must be structure. 
By trying to understand which models within a given machine learning framework -- e.g.~deep learning -- lead to sparse descriptions, perhaps we can better understand
what selects such latent descriptions from a bottom-up sparsity perspective and connect them to the top-down human perspective. %
At which point, perhaps we'll all sit at the same lunch table. 




















































\section{The Unreasonable Effectiveness of Deep Learning}\label{sec:ETofDL}
While the discussion of physics has been varying levels of concrete, the discussion of AI has been entirely in the abstract. In an attempt to restore the balance, 
in the rest of the essay we'll focus on explaining how the framework of \emph{deep learning} can be interpreted through the lens of some of the physics principles that we've  been outlining.


While artificial intelligence and deep learning have become nearly synonymous, we haven't really discussed the specifics of deep learning at all. Deep learning is a branch of machine learning based on artificial neural networks. Such networks compute a very flexible set of functions that turn out to be excellent for mathematically modeling the sorts of tasks 
that we associate with human intelligence. 

To that point, the list of AI breakthroughs based on the deep learning framework includes problems in computer vision \cite{ImageNet2012,goodfellow2014generative}, natural language processing and generation \cite{mikolov2013efficient,mikolov2013distributed,BERT2018,radford2019language,Brown2020LanguageMA}, and the mastery of many games by deep reinforcement learning: beginning with a general purpose algorithm capable of learning to play a set of 49 Atari games \cite{mnih2015human}, and growing to include Go \cite{silver2016mastering,silver2018general}, chess without a specialized algorithm \cite{silver2018general}, Dota II \cite{berner2019dota}, Starcraft \cite{starcraft}, and even a unifying algorithm for handling Go, chess, shogi, and 57 Atari games without any prior knowledge of the dynamics of those games \cite{muzero}. 
Most recent -- and perhaps most exciting  -- is a deep learning solution to the protein folding problem in biology \cite{alphafold2prelim}.

However, deep learning's effectiveness is puzzling.
In defiance of the cognitive science perspective, domain specific knowledge has often proved neutral if not harmful in guiding the approach of practitioners. While the choice of the specific model -- the neural network architecture -- 
is often essential, within a class of architectures that work the most important considerations leading to success always seem to be the size of the model -- huge -- and the amount of training data -- as much as possible \cite{sutton_2019}.\footnote{For instance, in the domain of natural language modeling the advent of the transformer architecture \cite{attention2017} has arguably rendered much of the expertise of natural language processing researchers obsolete. Instead, the most important ingredient of success is ever larger models and ever more data \cite{kaplan2020scaling}.
}
Moreover, unlike human learning models, deep learning models that have sufficiently many parameters can also memorize random labelings of random data \cite{zhang2016understanding}.
In short, attempts to apply top-down analysis to explain the success of deep learning often come up short. 




This suggests there might be something to gain by a physicist's bottom-up analysis. 
In \S\ref{sec:introduction}, we argued that \emph{sparsity} is an essential guiding principle of physics theory. With that discussion in mind, in this section we will similarly argue that sparsity is a guiding principle of deep learning theory.

The material presented here is an overview of the perspective taken in an upcoming book with Yaida and Hanin \cite{Principles}.























\subsubsection*{Interacting neurons}\label{sec:EFT}

\epigraph{On being asked, ``How is Perceptron performing today?'' I am often tempted to respond, ``Very well, thank you, and how are Neutron and Electron behaving?''}{Frank Rosenblatt, inventor of the perceptron and also the Perceptron  \cite{rosenblatt1961principles}.}

\noindent{}A very natural framework for studying deep learning is closely related to the sort of quantum field theory that we described in \S\ref{sec:physics-simple}, with the first principal difference that deep learning is a classical statistical theory, not a quantum theory, and the second principal difference that the statistical variables in deep learning are not organized into fields. 

At the highest level, this doesn't matter. For both deep learning and quantum field theory there is an organizing principle called \emph{effective theory} that allows us to make progress.

To begin, we need to specify the degrees of freedom in deep learning, just as we did before when discussing quantum field theory. Rather than Electrons and Neutrons (or rather, electrons and neutrons), in this case we have neurons.\footnote{
    The perceptron model, invented by Rosenblatt, was the first artificial neural network with learnable weights \cite{rosenblatt1958perceptron}.
} 

Neurons take a weighted sum of input signals and then fire if that sum is above a certain threshold.\footnote{
    More generally, the weighted sum of input signals minus the threshold is acted on by a scalar \emph{activation function}. For the \emph{perceptron} activation function \cite{mcculloch1943logical}, this corresponds to the simplistic notion of firing.
} In a typical neural network, many such neurons are organized into layers, and deep learning places a particular emphasis on networks that are iteratively composed of many layers with a similar structure. 

\begin{figure}[h]
\begin{center}
\includegraphics[scale=.9]{mlp.pdf}
\caption{Depiction of a simple multilayer neural network. The circles represent the neurons in the hidden layers, the black dot at the top represents the network output, and the lines indicate the connections between the neurons. This four-layer network takes a $2 \times 2$ image $x$ and computes the function $f(x)$ after passing the input through 3 hidden layers. The input is transformed into the output at layer $L=4$ through a sequence of intermediate signals, $s^{(1)}$, $s^{(2)}$, and $s^{(3)}$.
}
\label{fig:mlp}
\end{center}
\end{figure}





An example of such a multilayer structure is shown in Fig.~\ref{fig:mlp}. In the network represented by this figure, the function $f(x)$ is computed by sequentially passing signals through the layers of neurons. The input $x$ is transformed into the signal $s^{(1)}$ by the five neurons in the first hidden layer, then into a sequence of signals $s^{(2)}$ and $s^{(3)}$ by the second and third hidden layers, respectively, before being transformed into the output $f(x) \equiv s^{(4)}$ at the single-neuron final layer (black dot at the top). Each neuron $i$ in a layer $\ell$ forms a component of the signal in that layer, e.g.~$s_i^{(\ell)}$. Moreover, an important quantity of interest is the number of layers or \emph{depth} of the network.

The strength of the connections between the layers can be individually modified by a group of parameters called \emph{weights}. The firing threshold of each neuron is set by another set of parameters called \emph{biases}. Together, the weights and biases make up the model parameters.
In a typical learning algorithm,  these parameters are iteratively adjusted until the network output approximates some desired function on a sequence of training examples $\mathcal{D} = \{x_1, \dots, x_{N_\mathcal{D}} \}$. 

On the one hand, after such a \emph{training} procedure, the graph gives a simple way to compute the function $f(x)$ given the settings of the weights and biases. On the other hand, knowing the network architecture and the trained values of the model parameters doesn't really offer any insight into how the function works. In terms of these 
variables, the network behaves like a \emph{black box}.



Consider the language model GPT-3, which has 175 billion parameters \cite{Brown2020LanguageMA}. Assuming the values of these parameters are stored as four-byte floats, that's 700 gigabytes! If someone hands you those 700 GB and asks you to predict the output of the model on a given input $x$, the best thing to do would be to 
explicitly compute the function $f(x)$ using an implementation of the model. If someone asks you \emph{why} the output $f(x)$ is given on an input $x$, you're in trouble; these microscopic variables -- the model parameters -- are just not the right set of variables in which the computation is easy to understand.\footnote{This is like being given a description of a person's \emph{connectome} and asked to predict what such a person would say if prompted with the input $x$; the best you can hope to do is actually go find the person and ask them $x$!}



However, there's another way of thinking about what the network is doing.\footnote{
    More precisely, by sampling the weights and biases from an initialization distribution, we are \emph{inducing} a distribution on the network outputs as well as the on the hidden layer signals. Then, to analyze this \emph{ensemble}, we formally integrate out the weights and biases. 
} For the output neuron defining the function, there's some marginal probability that it fires on a particular input: $p(f | x)$. 
More generally, for the training dataset $\mathcal{D} = \{x_1, \dots, x_{N_\mathcal{D}} \}$, there's a joint probability distribution encoding the probabilities that the output fires on a selection of inputs: %
\be\label{eq:network-output}
p(f | \mathcal{D}) \equiv p\le(f(x_1), \dots, f(x_{N_\mathcal{D}}) \ri) \, . 
\ee
This \emph{output distribution} encodes the correlations of the network output when the network is evaluated on different inputs.
Even more generally, we could consider a sequence of $L$ \emph{neural distributions} 
\be\label{eq:neural-distribution-sequence}
p(s^{(1)} | \mathcal{D}),~ p(s^{(2)} | \mathcal{D}),~   \dots,~ p(s^{(L)} | \mathcal{D})\, ,
\ee 
characterizing how the input data $\mathcal{D}$ is transformed into the network output $f(x) \equiv s^{(L)}$ at layer $L$ through the signals $s^{(\ell)}$ computed by the $\ell$-th layer neurons in each of the hidden layers.


In a way that can be made quite precise \cite{Principles}, we can interpret these correlations in these neural distributions as arising from \emph{interactions} between the neurons just as we described quantum field theory in terms of the interactions of elementary particles. Working out the form of each of the distributions in this sequence lets us understand how a network transforms input into output; it is tantamount to opening the black box of deep learning. So, as you can imagine, it's naively a pretty difficult thing to do.








Mirroring our discussion of the quantum field theory degrees of freedom before, the most generic neural network expressed in the deep learning framework would have all possible interactions between all the different neurons. Focusing  on just the network output for a moment, if we have $N_\mathcal{D}$ different inputs in our dataset,  then there are $\binom{N_{\mathcal{D}}}{2}\sim N_\mathcal{D}^2$ potential interactions involving the output neuron evaluated on two inputs, $\binom{N_{\mathcal{D}}}{3}\sim N_\mathcal{D}^3$ potential interactions involving the output neuron evaluated on three different inputs, $\binom{N_{\mathcal{D}}}{4}\sim N_\mathcal{D}^4$ potential interactions on four different inputs, and so on. Each of these interactions has a strength or \emph{data-dependent coupling} which encodes the patterns of correlations in the input data at the network output after being transformed by the neural network.\footnote{
    These couplings should not be confused with the \emph{parameters} -- the weights and biases -- of the network. Instead, the data-dependent couplings give a alternate description of the underlying model.
}

Adding up all these interactions, this more or less gives $2^{O(N_\mathcal{D})}$ different neural interactions, meaning that a generic description of the output distribution of the network in terms of these interactions will scale exponentially with the size of the dataset. Such a description is certainly not sparse: depending on how the total number of weights and biases compares to $2^{O(N_\mathcal{D})}$, this is extremely likely to be even more complicated than the microscopic description of the network in terms of the model parameters!



















































\subsubsection*{Sparsity in the thermodynamic limit} %


One of the most powerful tools in theoretical physics is a certain asymptotic limit where the number of degrees of freedom of a system is taken to be very large. 
This idea has many different manifestations and as a result carries a number of different names: the thermodynamic limit, the semi-classical limit, and large-$N$. %

Given the many different avatars, there are many different ways to think about such a limit, but the central idea is that when the number of degrees of freedom of a system are formally taken to be infinite, the system's interactions can turn off 
and 
as a consequence
\emph{fluctuations} of the degrees of freedom are heavily suppressed.\footnote{
    Among the many systems for which this limit leads to a simple description, a familiar one is the thermodynamic description of gas in a room in terms of the ideal gas law.
}
This is essentially a physics instantiation of the central limit theorem.

In the thermodynamic limit complicated interacting systems often exhibit extremely simple behavior, and deep learning is no exception. In 1996, Neal described such a limit for neural networks with only a single hidden layer by taking the number of neurons in the hidden layer to be infinite \cite{neal1996priors}. 
This same limit was more or less employed in the multilayer setting by numerous authors \cite{poole2016exponential,raghu2017expressive, schoenholz2016deep,lee2018deep,g2018gaussian,jacot2018neural}. In this case, the number of neurons $N_\ell$ in each hidden layer $\ell$ is taken to be infinite, while the overall depth of the network $L$ is held fixed.









This \emph{infinite-width limit} provides a simple and tractable \emph{toy model} of deep learning. In particular, in this limit all the neural interactions turn off. 
This means that  the sequence of neural distributions \eqref{eq:neural-distribution-sequence} -- and in particular the output distribution \eqref{eq:network-output} -- will each converge to multivariate Gaussian distributions.
For each of these distributions, the pattern of correlation is determined entirely by the distribution's pairwise covariance matrix.\footnote{
    In the language of footnote~\ref{footnote:electron-mass}, we can alternatively think of this covariance as defining a $2$-neuron interaction just as we thought of the electron mass as a $2$-particle interaction between an electron and positron. The point is that such a $2$-particle correlation is exceptionally simple and allows the degrees of freedom to become disentangled: there always exists a basis in which their \emph{statistical independence} is manifest.
}

Thus, the infinite-width limit leads to exceptionally sparse representations: we've reduced the number of data-dependent couplings required to describe the network's output distribution from scaling exponentially in the size of the dataset $\sim2^{O(N_\mathcal{D})}$ to scaling quadratically $\sim O(N_\mathcal{D}^2)$. Naively, while such models seem \emph{overparameterized} -- potentially containing more parameters $N \to \infty$ than training data $N_\mathcal{D}$ -- %
in terms of the data-dependent couplings, they are actually sparse!

Further theoretical analysis, however, shows that this limit is too simple: these networks only permit a very simplified notion of learning in which the features used to determine the network output are fixed before any training begins \cite{jacot2018neural,chizat2018note, li2019towards}. Instead, only the coefficients of a linear function of those fixed random features get modified during training, severely constraining the classes of functions that can be learned.
To understand why this is problematic, let us recall our discussion of \emph{human learning} in \S\ref{sec:ai-hard}. There, we argued that understanding data in terms of a sequence of 
representations was an essential component of human learning;
a similar mechanism is supposed to be an essential component of \emph{deep learning} as well \cite{lecun2015deep,Rob}.






In the typical discussion of representation learning, we start with the fine-grained representation of an input such as an image in terms of its pixels:  $x =\Cat$. For a classification task, a network might output a coarse-grained representation of that image: $f(x) = \texttt{cat}$. In between, the signals at the hidden-layer neurons $s^{(\ell)}$ form intermediate representations. For instance, the initial layers can act as oriented edge detectors, while the deeper layers form more and more coarse-grained representations, organizing human-meaningful sub-features such as \texttt{fur} and \texttt{whiskers} into higher-level features like a \texttt{face}.










However, the intermediate representations in infinite-width networks are fixed from the start, completely independent of the training data. 
In a sense, the behavior of networks in this infinite-width limit is very similar to the behavior of networks without any hidden layers. By virtue of being shallow, such networks don't contain intermediate representations of their input data, and their output is always described by a Gaussian distribution. In other words, in the infinite-width limit networks are neither \emph{deep} nor do they \emph{learn} representations.  

The lack of \emph{representation learning} in the infinite-width limit indicates the breakdown of its usefulness as a toy model. %
This breakdown thus hints at the need to go beyond such a limit in order 
to describe deep networks at any nontrivial depth.























\subsubsection*{Sparsity and complexity in the \texorpdfstring{$1/N$}{1/N} expansion}%
\epigraph{Everything should be as simple as it can be, but not simpler.}{Roger Sessions, simplifying a much longer quote of Albert Einstein  \cite{sessions_1950}.}


\noindent{}Toy models -- like the infinite-width network -- are invented to strip away irrelevant details in order to more easily understand the essential properties of an underlying phenomenon.\footnote{
All mathematical models are toy models or approximations, in one way or another. For example, consider the Heliocentric model of the solar system with its fixed sun and non-interacting planets, or consider the Ising model as a simplistic theoretical description of ferromagnetism. The Standard Model of particle physics \cite{Glashow:1961tr,Weinberg:1967tq,Salam:1968rm} -- perhaps the most successful mathematical theory of natural phenomena in terms of the detailed correspondence of its theoretical predictions to the outcomes of experiments -- is just a very rich toy model: 
it does not include the force of gravity.
Typically, the adjective ``toy'' is reserved for overly-simplistic models.
}
When a toy model breaks down, it indicates the correspondence between the mathematical model and the underlying physical reality diverges.

For this reason, it's sometimes useful to consider a sequence of toy models with increasing sophistication.\footnote{
    For instance, Newtonian gravity, general relativity, and string theory are a sequence of increasingly sophisticated models of gravity.
} 
The probing of a rigid framework 
    with a sequence of such models leads to a built-in type of \emph{transfer learning} between the models; it's often clear from the start -- or after a quick calculation -- how a particular model can be related to others in the sequence.
 Moreover, we can often
concretely connect the 
simplifying properties used to construct these models
to the situations in which they break down.  
This is tantamount to actually understanding the role that such properties play across a broad class of models living in the underlying framework.


















The program of constructing such a sequence of models in the framework of deep learning was first begun by Yaida in \cite{Yaida2019}. To define these models, we will make use
of an ancient trick in physics for studying small deviations from idealized behavior: 
\emph{perturbation theory}. 
Traditionally, perturbation theory applies to systems that have an infinitesimally-small dimensionless parameter $(\epsilon \ll 1)$ in terms of which any quantity of interest can be Taylor expanded: $f(\epsilon) \approx f(0) + \epsilon f'(0) +  \epsilon^2 f''(0)/2 + \dots$.
In this case we will also perform a magic trick -- for it is really something of a sleight-of-hand -- by realizing that perturbation theory can also be applied to systems that have an infinitely-large dimensionless parameter $(N \gg 1)$ by defining $\epsilon \equiv 1/N$. This is sometimes called the \emph{$1/N$ expansion} \cite{tHooft:1973jz}.





In particular, the $1/N$ expansion lets us back off of the infinite-width thermodynamic limit and compute \emph{finite-width corrections} to the network output distribution \eqref{eq:network-output} for networks that have a finite number of neurons per layer. Concretely, let's imagine the number of the neurons $N_\ell$ in each of the hidden layers $\ell = 1, \dots, L-1$ are all equal: $N_\ell \equiv N $. Then, we can compute the network output distribution by working out a perturbation series in the small parameter $1/N$, expanding around the $N\to \infty$ infinite-width toy model. Each toy model in the sequence is then defined according to the order in $1/N$ at which the perturbation series is truncated. By retaining additional higher-order corrections, we get richer toy models, asymptotically approaching a complete description of the underlying network.\footnote{
    The sequence of toy models in the $1/N$ expansion should not be confused with the sequence of neural distributions \eqref{eq:neural-distribution-sequence} computed by the layers of a network. In particular, each toy model gives an explicit representation of all the distributions \eqref{eq:neural-distribution-sequence} truncated to a fixed power of $1/N$.
}

Importantly, the finite-width corrected output distributions are no longer multivariate Gaussian distributions. Each order in $1/N$ turns on an additional set of neural interactions: the $O(1/N)$ correction allows interactions between collections of four neurons, each evaluated on potentially different inputs; the $O(1/N^2)$ correction allows interactions between collections of six neurons, each evaluated on potentially different inputs; and generally the $O(1/N^k)$ correction would allow interactions between collections of $(2k+2)$ neurons.\footnote{
    For a technical reason, there are only interactions between even numbers of neurons.
}

Accordingly, these distributions are defined in terms of more and more intricate patterns of correlation. At leading order in $1/N$, describing the four-neuron interaction requires $\sim N_{\mathcal{D}}^4$ data-dependent couplings, and generally describing a $(2k+2)$-neuron interaction requires $\sim N_{\mathcal{D}}^k$ data-dependent couplings.
Thus, these models populate the entire spectrum from sparse to complex! 

Despite the increasing complexity, these models also have \emph{less} naive parameters -- less weights and less biases -- than their infinite-width counterparts. Taking $N$ to be finite reduces the number of neurons per layer -- from infinite to finite -- but makes the effective description -- in terms of the data-dependent couplings encoding the pattern of neuron-neuron correlations -- more complicated and less sparse! %

Going forward our focus will be on the simplest \emph{nearly-Gaussian} model at $O(1/N)$,  including only the leading finite-width corrections. Such a model is still quite sparse -- requiring $\sim N_{\mathcal{D}}^4$ data-dependent couplings -- but also is potentially  rich enough to capture representation learning in deep networks.





























\subsubsection*{Representation learning from renormalization group flow}\label{sec:rg}

To investigate representation learning in our finite-width model, we need to introduce one final tool from theoretical physics: \emph{renormalization group flow} \cite{PhysRevB.4.3174,PhysRevB.4.3184}. The Wilsonian renormalization group (RG) flow equations detail how the strength of interactions can change with the scale at which an experiment is carried out. Recall from our discussion of quantum electrodynamics in \S\ref{sec:physics-simple} that we kept specifying the strength of the electric charge, $\approx 1/137$,  \emph{at rest}. More precisely, this coupling can change or \emph{run} depending on the distance or energy scale at which matter and light are interacting.




The \emph{flow} in \emph{RG flow} is generated by a repeated marginalization over the microscopic fine-grained degrees of freedom of the system, which results in an \emph{effective theory} of coarse-grained observables.
In the language of RG flow, interactions that grow with the flow are called \emph{relevant}, and interactions that decrease in strength are called \emph{irrelevant}.
By focusing only on the growing relevant interactions, the effective theory provides an efficient way to study the dynamics of the course-grained degrees of freedom.






In the framework of deep learning, we can derive analogous flow equations -- in this case discrete recursions -- that detail how the neural interactions run as we change the depth of the network from $L$ layers to $(L+1)$ layers \cite{Yaida2019}. Moreover, these recursions tell us how to relate the sequence of neural distributions \eqref{eq:neural-distribution-sequence} describing the hidden-layer representations. Such recursions constrain the data-dependent couplings of these distributions and thus describe how the pattern of correlation in the data evolves from the input through the hidden-layer representations to the network output.

From this perspective, it's easy to make the connection between these recursions and RG flow concrete. To compute the layer $(\ell+1)$-th distribution $p(s^{(\ell+1)}|\mathcal{D})$, in the sequence of neural distributions \eqref{eq:neural-distribution-sequence}, we first construct the interlayer joint neural distribution between neurons in the $(\ell+1)$-th layer and the $\ell$-th layer,  
$p(s^{(\ell+1)}, s^{(\ell)}|\mathcal{D})$, and then marginalize over the neurons in the $\ell$-th layer $s^{(\ell)}$. Repeating such a marginalization induces an RG flow, integrating out the fine-grained representations of the shallower layers in favor of the coarse-grained representations in the deeper layers. 

In the context of deep learning, we call this \emph{representation group flow}, or \emph{RG flow} for short \cite{Principles}. Such an RG flow is precisely what we were interested in from the onset: the flow lets us understand how the microscopic variables, such as the pixels of the input image, are transformed into intermediate representations in the hidden layers, and then how those hidden-layer representations determine the network output.

With enough effort, the discrete RG flow equations can be worked out for any of the toy models described by the $1/N$ expansion. With orthogonal effort, such equations can be analyzed. In particular, by considering the asymptotic limit of large network depth $L$, we can understand the behavior of networks that are both deep and wide.








Solving the discrete RG flow equations at large $L$, we find that the finite-width corrections are \emph{relevant}, growing increasingly larger as the depth of the network increases. Explicitly, we find that leading $1/N$ corrections also scale linearly with $L$. This suggests that the proper perturbative parameter defining our sequence of toy models actually should have been the depth-to-width ratio of the network: $L/N$. This aspect ratio $L/N$ behaves as an \emph{emergent scale} or \emph{cutoff}, controlling the importance of the finite-width corrections to the infinite-width limit.



Moreover, this analysis lets us understand what went wrong in the infinite-width limit. Recall in that limit that we effectively took the number of neurons $N$ in each hidden layer to be infinite while holding the overall depth of the network $L$ fixed. This means that the ratio vanishes: $L/N \to 0$. Formally, in this limit networks behave as if they are not at all deep. Accordingly, our naive notion of what it means for a realistic network to be wide $N \gg 1$ was wrong; instead we should have been comparing width to depth.
Now we see that clearly that the proper way to study corrections to the infinite-width limit is large depth $L$ and large width $N$, with their ratio $L/N$ held fixed.\footnote{
    A very similar thing happens for the \emph{strong interactions} in physics. In the limit of Yang-Mills theory as the number of colors goes to infinity $(N\to\infty)$, you get a very different limit depending on how the Yang-Mills coupling $g$ scales with $N$ \cite{tHooft:1973jz}. Depending on this scaling, you can get a trivial free theory -- analogous to the infinite-width limit -- or a theory where the interactions are too strong to admit a $1/N$ expansion. Naturally, with the correct scaling you find string theory \cite{Maldacena:1997re}. 
}

This RG flow analysis also underscores the importance of paying attention anytime a dimensionless scale grows infinitely large or becomes perturbatively small. Further analysis of the details of initialization and training shows that scales such as width $N$ and depth $L$ are typically responsible for the extreme range of optimal hyperparameter tunings found by deep learning practitioners \cite{Principles}. The varying of such parameters over many orders of magnitude signals to an effective theorist that the cutoff of a theory is not properly understood. In particular, most hyperparameters -- appropriately constructed -- should be order-one quantities, robust to small changes and applicable across a variety of different architecture widths and depths.



Of course, the lack of representation learning in the infinite-width limit was also indicative of the breakdown of the infinite-width limit. By carefully studying the leading $L/N$ finite-width corrections, we can incorporate the effects of depth at finite width. In this limit, not only do we see nontrivial representation learning, we find that it's \emph{relevant} \cite{Principles}. In other words, we explicitly can see how \emph{deepness} leads to representation \emph{learning}!























\subsubsection*{Simplicity from criticality}



So why is such a simple toy model -- in particular, the leading finite-width nearly-Gaussian model -- so effective at describing deep learning? Are the other richer models that incorporate higher-order corrections from the $1/N$ expansion capturing other qualitative features of deep learning that the simplest finite-width model is missing? 

We can make progress on these questions by appealing to the physics principle of \emph{criticality}. First, note that for iterative maps like the one that defines the neural network shown in Fig.~\ref{fig:mlp}, exponential growth or decay of signals is generic. Such behavior is actually quite harmful to the performance of the network: exponential decay means that input signals effectively disappear after some characteristic depth scale, while exponential growth leads to numerical instability.\footnote{
    Concretely, the network targets are typically $O(1)$ numbers, so exponential behavior makes training exponentially difficult \cite{poole2016exponential,raghu2017expressive,schoenholz2016deep}.
} 


However, systems tuned to a \emph{critical point} exhibit self-similar behavior under RG flow. Thus, the principle of criticality suggests that we should search for nontrivial fixed points of the RG flow equations in which the relevant observables do not behave exponentially.  In \cite{Principles} we explain how to tune the initialization distributions of the weights and biases of the network order by order in the $1/N$ expansion in order to reach criticality.\footnote{
One consequence of renormalization group flows to nontrivial fixed points is the phenomenon of \emph{universality} \cite{Kadanoff:1971pc}. The idea is that the sequence of distributions describing different systems at various levels of coarse-graining can converge under RG flow, despite having potentially very different fine-grained microscopic descriptions. The set of such systems that behave similarly are said to form a \emph{universality class}.

Similarly for \emph{representation group flows}, we find that the sequence of neural distributions \eqref{eq:neural-distribution-sequence} can also converge under RG flow for networks employing different activation functions \cite{Principles}. For activation functions within a universality class, the details 
become irrelevant, with the behavior of observables explicitly depending on only a few Taylor coefficients of the function. \label{footnote:universality}
}










With that in mind, one feature of critical systems is the manifestation of exceptionally large \emph{fluctuations} on the order of the size of the system.\footnote{
     Recalling our discussion of the thermodynamic limit, it's clear that there's some tension between this notion of criticality and the infinite-width limit in which  fluctuations are supposed to be suppressed. This offers another way of thinking about the infinite-width limit: it's generically an unstable fixed point of the RG flow.
}  In the context of deep learning, this means large instantiation-to-instantiation fluctuations between explicit initializations of a network's weights and biases. The RG flow analysis for networks at criticality shows that these fluctuations have a typical size that scales with the emergent cutoff $\sim L/N$. When such fluctuations are large, destructive exponential behavior is generic in any particular instantiation. 


Thus, on the one hand, to suppress such fluctuations we require $L/N < 1$. On the other hand, to get nontrivial representation learning, we require $L/N$ nonzero. That is, successful networks are both wide and deep, with their aspect ratio reasonably small but nonzero. In this regime, the simplest finite-width-corrected \emph{nearly-Gaussian} model thus captures the essence of deep learning, realizing the paraphrased Einstein's principle of simple, but not too simple.












\subsubsection*{Deep learning: an effective theory approach}


To review, our goal was to find a physicist's bottom-up formalism for understanding the principles of deep learning theory. In the infinite-width limit, we observed that all the neural interactions turned off, leading to a very sparse toy model of neural networks with limited use. This further let us explain how \emph{overparameterized} networks with more tunable parameters than training data are secretly really simple in terms of their sparse \emph{data-dependent coupling} descriptions.



Backing off the strict infinite-width limit, we found a nontrivial expansion in the depth-to-width ratio $L/N$ that lets us systematically incorporate interactions into the theory at the cost of a more complicated analysis. Non-intuitively, reducing the naive number of parameters -- by taking finite $N$ %
 -- is actually a means of making our theoretical description \emph{less sparse}!


Then, the RG flow analysis of the $1/N$ expansion gave an effective theory understanding of representation learning, while criticality offered insight into the tuning of hyperparameters. All together, this makes our effective theory approach \cite{Principles} extremely useful for understanding real neural networks in practice.






































































































\section{The Future}\label{sec:future}





\epigraph{TANSTAAFL! -- \emph{Robert A. Heinlein \cite{heinlein}.}}{}\vspace{-1\baselineskip}


\noindent{}There might not be free lunches, but there will be lunch specials. 



\section*{Acknowledgments}
We are grateful to Yasaman Bahri, John Frank, Yoni Kahn, Yann LeCun, Kyle Mahowald, George Musser, Adrienne Rothschilds, David Schwab, Douglas Stanford, DJ Strouse, Josh Tenenbaum, and Jesse Thaler for discussions and feedback. We are especially grateful to Boris Hanin and Sho Yaida for collaboration, discussions, and for providing extensive feedback on multiple drafts.
This essay was brought to you in the limit of $L/N$ held fixed, and 
by readers like you.
Thank you for your time.







\mciteSetMidEndSepPunct{}{\ifmciteBstWouldAddEndPunct.\else\fi}{\relax}
\bibliographystyle{utphys}
\bibliography{why}{}

\end{document}

\section{How do generative models work? How do GANs compare to others?}
\label{sec:tree}

We now have some idea of what generative models can do and why it might be
desirable to build one.
Now we can ask: how does a generative model actually work? And in particular,
how does a GAN work, in comparison to other generative models?

\subsection{Maximum likelihood estimation}

To simplify the discussion somewhat, we will focus on generative models
that work via the principle of \newterm{maximum likelihood}.
Not every generative model uses maximum likelihood.
Some generative models do not use maximum likelihood by default, but
can be made to do so (GANs fall into this category).
By ignoring those models that do not use maximum likelihood, and
by focusing on the maximum likelihood version of models that do not
usually use maximum likelihood, we can eliminate some of the more 
distracting differences between different models.

The basic idea of maximum likelihood is to define a model that provides
an estimate of a probability distribution, parameterized by parameters
$\vtheta$.
We then refer to the \newterm{likelihood} as the probability that the model
assigns to the training data: $\prod_{i=1}^m \pmodel\left(\vx^{(i)}; \vtheta \right),$
for a dataset containing $m$ training examples $\vx^{(i)}$.

The principle of maximum likelihood simply says to choose the parameters for the model
that maximize the likelihood of the training data.
This is easiest to do in log space, where we have a sum rather than a product
over examples.
This sum simplifies the algebraic expressions for the derivatives of the likelihood
with respect to the models, and when implemented on a digital computer, is less
prone to numerical problems, such as underflow resulting from multiplying together
several very small probabilities.

\begin{align}
\vtheta^* =& \argmax_\vtheta \prod_{i=1}^m \pmodel\left(\vx^{(i)}; \vtheta \right) \\
  =& \argmax_\vtheta \log \prod_{i=1}^m \pmodel\left(\vx^{(i)}; \vtheta \right) \label{eq:log} \\
          =& \argmax_\vtheta \sum_{i=1}^m \log \pmodel\left(\vx^{(i)}; \vtheta \right).
\end{align}

In \eqref{eq:log}, we have used the property that $\argmax_v f(v) = \argmax_v \log f(v)$ for 
positive $v$, because the logarithm is a function that increases everywhere and does not change
the location of the maximum.

The maximum likelihood process is illustrated in \figref{fig:mle}.

We can also think of maximum likelihood estimation as minimizing the
\newterm{KL divergence} between the data generating distribution and the
model:
\begin{equation}
\vtheta^* = \argmin_\vtheta \KL\left( \pdata(\vx) \Vert \pmodel(\vx ; \vtheta) \right).
\label{eq:kl}
\end{equation}
If we were able to do this precisely, then if $\pdata$ lies within the family of distributions
$\pmodel(\vx ; \vtheta)$, the model would recover $\pdata$ exactly.
In practice, we do not have access to $\pdata$ itself, but only to a training set
consisting of $m$ samples from $\pdata$.
We uses these to define $\ptrain$, an \newterm{empirical distribution} that places mass only
on exactly those $m$ points, approximating $\pdata$.
Minimizing the KL divergence between $\ptrain$ and $\pmodel$ is exactly equivalent to maximizing
the log-likelihood of the training set.

\begin{figure}
\centering
\includegraphics[width=\textwidth]{mle.pdf}
\caption{The maximum likelihood process consists of taking several samples from
  the data generating distribution to form a training set, then pushing up on the
  probability the model assigns to those points, in order to maximize the likelihood
  of the training data.
  This illustration shows how different data points push up on different parts of
  the density function for a Gaussian model applied to 1-D data.
  The fact that the density function must sum to $1$ means that we cannot simply
  assign infinite likelihood to all points; as one point pushes up in one place
  it inevitably pulls down in other places.
  The resulting density function balances out the upward forces from all the data
  points in different locations.
}
\label{fig:mle}
\end{figure}

For more information on maximum likelihood and other statistical estimators,
see chapter 5 of \citet{Goodfellow-et-al-2016}.


\subsection{A taxonomy of deep generative models}

If we restrict our attention to deep generative models that work by maximizing
the likelihood, we can compare several models by contrasting the ways that they
compute either the likelihood and its gradients, or approximations to these
quantities.
As mentioned earlier, many of these models are often used with principles other
than maximum likelihood, but we can examine the maximum likelihood variant of
each of them in order to reduce the amount of distracting differences between
the methods.
Following this approach, we construct the taxonomy shown in \figref{fig:tree}.
Every leaf in this taxonomic tree has some advantages and disadvantages.
GANs were designed to avoid many of the disadvantages present in pre-existing
nodes of the tree, but also introduced some new disadvantages.

\begin{figure}
\centering
\includegraphics[width=\textwidth]{fuck_arxiv_tree.pdf}
\caption{
Deep generative models that can learn via the principle of maximim likelihood
differ with respect to how they represent or approximate the likelihood.
On the left branch of this taxonomic tree, models construct an explicit density,
$\pmodel(\vx; \vtheta)$, and thus an explicit likelihood which can be maximized.
Among these explicit density models, the density may be computationally tractable,
or it may be intractable, meaning that to maximize the likelihood it is necessary
to make either variatioanl approximations or Monte
Carlo approximations (or both).
On the right branch of the tree, the model does not explicitly represent a
probability distribution over the space where the data lies.
Instead, the model provides some way of interacting less directly with this
probability distribution.
Typically the indirect means of interacting with the probability distribution is
the ability to draw samples from it.
Some of these implicit models that offer the ability to sample from the distribution
do so using a Markov Chain; the model defines a way to stochastically transform
an existing sample in order to obtain another sample from the same distribution.
Others are able to generate a sample in a single step, starting without any input.
While the models used for GANs can sometimes be constructed to define an explicit
density, the training algorithm for GANs makes use only of the model's ability to
generate samples.
GANs are thus trained using the strategy from the rightmost leaf of the tree:
using an implicit model that samples directly from the distribution represented
by the model.
}
\label{fig:tree}
\end{figure}

\subsection{Explicit density models}

In the left branch of the taxonomy shown in \figref{fig:tree} are models that define
an explicit density function $\pmodel(\vx ; \vtheta)$.
For these models, maxmimization of the likelihood is straightforward; we simply plug
the model's definition of the density function into the expression for the likelihood,
and follow the gradient uphill.

The main difficulty present in explicit density models is designing a model that can
capture all of the complexity of the data to be generated 
while still maintaining computational tractability.
There are two different strategies used to confront this challenge:
(1) careful construction of models whose structure guarantees their tractability,
as described in \secref{sec:explicit_tractable},
and (2) models that admit tractable approximations to the likelihood and its
gradients, as described in \secref{sec:approx}.

\subsubsection{Tractable explicit models}
\label{sec:explicit_tractable}

In the leftmost leaf of the taxonomic tree of \figref{fig:tree} are the models
that define an explicit density function that is computationally tractable.
There are currently two popular approaches to tractable explicit density models:
fully visible belief networks and nonlinear independent components analysis.

\paragraph{Fully visible belief networks}
\newterm{Fully visible belief networks} \citep{Frey96,Frey98} or FVBNs are models that use the chain
rule of probability to decompose a probability distribution over an $n$-dimensional vector $\vx$
into a product of one-dimensional probability distributions:
\[
\pmodel(\vx) = \prod_{i=1}^n \pmodel\left(\evx_i \mid \evx_1, \dots, \evx_{i-1} \right).
\]
FVBNs are, as of this writing, one of the three most popular
approaches to generative modeling, alongside GANs and variational autoencoders.
They form the basis for sophisticated generative models from DeepMind, such as
WaveNet \citep{aaron-wavenet-2016}. WaveNet is able to generate realistic human speech.
The main drawback of FVBNs is that samples must be generated
one entry at a time: first $\evx_1$, then $\evx_2$, etc., so the cost of generating
a sample is $O(n)$.
In modern FVBNs such as WaveNet, the distribution over each $\evx_i$ is computed by a deep
neural network, so each of these $n$ steps involves a nontrivial amount of computation.
Moreover, these steps cannot be parallelized.
WaveNet thus requires two minutes of computation time to generate one second of audio,
and cannot yet be used for interactive conversations.
GANs were designed to be able to generate all of $\vx$ in parallel, yielding greater
generation speed.

\paragraph{Nonlinear independent components analysis}
Another family of deep generative models with explicit density functions is based on
defining continuous, nonlinear transformations between two different spaces.
For example, if there is a vector of latent variables $\vz$ and a continuous, differentiable,
invertible transformation
$g$ such that $g(\vz)$ yields a sample from the model in $\vx$ space,
then
\begin{equation}
  \label{eq:change-of-variable}
  p_x(\vx) = p_z(g^{-1}(\vx)) \left| \mathrm{det}
  \left( \frac{\partial g^{-1}(\vx)} {\partial \vx}\right) \right|. 
\end{equation}
The density $p_x$ is tractable if the density $p_z$ is tractable
and the determinant of the Jacobian of $g^{-1}$ is tractable.
In other words, a simple distribution over $\vz$ combined with
a transformation $g$ that warps space in complicated ways can yield
a complicated distribution over $\vx$, and if $g$ is carefully designed,
the density is tractable too.
Models with nonlinear $g$ functions date back at least to
~\citet{deco1995higher}.
The latest member of this family is real NVP \citep{dinh2016density}.
See \figref{fig:nvp} for some visualizations of ImageNet samples
generated by real NVP.
The main drawback to nonlinear ICA models is that they impose restrictions
on the choice of the function $g$. In particular, the invertibility requirement
means that the latent variables $\vz$ must have the same dimensionality as $\vx$.
GANs were designed to impose very few requirements on $g$, and, in particular,
admit the use of $\vz$ with larger dimension than $\vx$.

\begin{figure}
\centering
\includegraphics[width=\textwidth]{fig_imnet_64_samples}
\caption{Samples generated by a real NVP model trained on 64x64 ImageNet images.
Figure reproduced from \citet{dinh2016density}.}
\label{fig:nvp}
\end{figure}

For more information about the chain rule of probability used to define FVBNs
or about the effect of deterministic transformations on probability densities
as used to define nonlinear ICA models, see chapter 3 of \citet{Goodfellow-et-al-2016}.

In summary, models that define an explicit, tractable density are highly
effective, because they permit the use of an optimization algorithm directly
on the log-likelihood of the training data.
However, the family of models that provide a tractable density is limited,
with different families having different disadvantages.

\subsubsection{Explicit models requiring approximation}
\label{sec:approx}

To avoid some of the disadvantages imposed by the design requirements of models
with tractable density functions, other models have been developed that still
provide an explicit density function but use one that is intractable, requiring
the use of approximations to maximize the likelihood.
These fall roughly into two categories: those using deterministic approximations,
which almost always means variational methods, and those using stochastic approximations,
meaning Markov chain Monte Carlo methods.

\paragraph{Variational approximations}
Variational methods define a lower bound
\[ \mathcal{L}(\vx; \vtheta) \leq \log \pmodel(\vx; \vtheta). \]
A learning algorithm that maximizes $\mathcal{L}$ is guaranteed to obtain at least
as high a value of the log-likelihood as it does of $\mathcal{L}$.
For many families of models, it is possible to define an $\mathcal{L}$ that is computationally
tractable even when the log-likelihood is not.
Currently, the most popular approach to variational learning in deep generative models
is the \newterm{variational autoencoder} \citep{Kingma-arxiv2013,Rezende-et-al-ICML2014} or VAE.
Variational autoencoders are one of the three approaches to deep generative modeling that are
the most popular as of this writing, along with FVBNs and GANs.
The main drawback of variational methods is that,
when too weak of an approximate posterior distribution or too weak of a prior distribution is used,
\footnote{
  Empirically, VAEs with highly flexible priors or highly flexible approximate posteriors
  can obtain values of $\mathcal{L}$ that are near their own log-likelihood
  \citep{kingma2016improving,chen2016variational}.
  Of course, this is testing the gap between the objective and the bound at the maximum of the bound;
  it would be better, but not feasible, to test the gap at the maximum of the objective.
  VAEs obtain likelihoods that are competitive with other methods, suggesting that they are also
  near the maximum of the objective.
  In personal conversation, L. Dinh and D. Kingma have conjectured that a family of models
  \citep{Dinh-et-al-arxiv2014,rezende2015variational,kingma2016improving,dinh2016density}
  usable as VAE priors or approximate posteriors are universal approximators.
  If this could be proven, it would establish VAEs as being asymptotically consistent.
}
even with a perfect optimization algorithm and infinite training data, the gap
between $\mathcal{L}$ and the true likelihood can result in $\pmodel$ learning something other than
the true $\pdata$.
GANs were designed to be unbiased, in the sense that with a large enough model and infinite data,
the Nash equilibrium for a GAN game corresponds to recovering $\pdata$ exactly.
In practice, variational methods often obtain very good likelihood, but are regarded as producing
lower quality samples.
There is not a good method of quantitatively measuring sample quality, so this is a subjective opinion,
not an empirical fact.
See \figref{fig:vae_samples} for an example of some samples drawn from a VAE.
While it is difficult to point to a single aspect of GAN design and say that it results in
better sample quality, GANs are generally regarded as producing better samples.
Compared to FVBNs, VAEs are regarded as more difficult to optimize, but GANs are not
an improvement in this respect.
For more information about variational approximations, see chapter 19 of
\citet{Goodfellow-et-al-2016}.

\begin{figure}
  \centering
  \includegraphics[width=\textwidth]{cifar_vae}
  \caption{Samples drawn from a VAE trained on the CIFAR-10 dataset.
    Figure reproduced from \citet{kingma2016improving}.
  }
  \label{fig:vae_samples}
\end{figure}

\paragraph{Markov chain approximations}
Most deep learning algorithms make use of some form of stochastic approximation,
at the very least in the form of using a small number of randomly selected training
examples to form a minibatch used to minimize the expected loss.
Usually, sampling-based approximations work reasonably well as long as a fair sample
can be generated quickly (e.g. selecting a single example from the training set
is a cheap operation) and as long as the variance across samples is not too high.
Some models require the generation of more expensive samples, using Markov chain
techniques.
A Markov chain is a process for generating samples by repeatedly drawing a sample
$\vx' \sim q(\vx' \mid \vx).$
By repeatedly updating $\vx$ according to the transition operator $q$, Markov chain
methods can sometimes guarantee that $\vx$ will eventually converge to a sample from
$\pmodel(\vx)$.
Unfortunately, this convergence can be very slow, and there is no clear way to test
whether the chain has converged, so in practice one often uses $\vx$ too early, before
it has truly converged to be a fair sample from $\pmodel$.
In high-dimensional spaces, Markov chains become less efficient.
Boltzmann machines \citep{Fahlman83,Ackley85,Hinton-Boltzmann,Hinton86a} are a
family of generative models that rely on Markov chains both to train the model
or to generate a sample from the model.
Boltzmann machines were an important part of the deep learning renaissance beginning
in 2006 \citep{Hinton06,hinton2007learning} but they are now used only very rarely,
presumably mostly because the underlying Markov chain approximation techniques have
not scaled to problems like ImageNet generation.
Moreover, even if Markov chain methods scaled well enough to be used for training,
the use of a Markov chain to generate samples from a trained model is undesirable
compared to single-step generation methods because the multi-step Markov chain
approach has higher computational cost.
GANs were designed to avoid using Markov chains for these reasons.
For more information about Markov chain Monte Carlo approximations, see chapter 18 of
\citet{Goodfellow-et-al-2016}.
For more information about Boltzmann machines, see chapter 20 of the same book.

Some models use both variational and Markov chain approximations.
For example, deep Boltzmann machines make use of both types of
approximation \citep{SalHinton09}.

\subsection{Implicit density models}

Some models can be trained without even needing to explicitly define a density
functions.
These models instead offer a way to train the model while interacting only
indirectly with $\pmodel$, usually by sampling from it.
These constitute the second branch, on the right side, of our taxonomy of
generative models depicted in \figref{fig:tree}.

Some of these implicit models based on drawing samples from $\pmodel$ define
a Markov chain transition operator that must be run several times to obtain
 a sample from the model.
From this family, the primary example is the \newterm{generative stochastic network}
\citep{Bengio-et-al-ICML-2014}.
As discussed in \secref{sec:approx}, Markov chains often fail to scale to high
dimensional spaces, and impose increased computational costs for using the
generative model. GANs were designed to avoid these problems.

Finally, the rightmost leaf of our taxonomic tree is the family of implicit models
that can generate a sample in a single step.
At the time of their introduction, GANs were the only notable member of this family,
but since then they have been joined by additional models based on
kernelized moment matching \citep{Li-et-al-2015,dziugaite2015training}.

\subsection{Comparing GANs to other generative models}

In summary, GANs were designed to avoid many disadvantages associated with other generative
models:
\begin{itemize}
  \item They can generate samples in parallel, instead of using runtime proportional to the
    dimensionality of $\vx$. This is an advantage relative to FVBNs.
  \item The design of the generator function has very few restrictions. This is an advantage
    relative to Boltzmann machines, for which few probability distributions admit tractable
    Markov chain sampling, and relative to nonlinear ICA, for which the generator must be
    invertible and the latent code $\vz$ must have the same dimension as the samples
    $\vx$.
  \item No Markov chains are needed. This is an advantage relative to Boltzmann machines and GSNs.
  \item No variational bound is needed, and specific model families usable within the GAN
    framework are already known to be universal approximators, so GANs are already known
    to be asymptotically consistent.
    Some VAEs are conjectured to be asymptotically consistent, but this is not yet proven.
  \item GANs are subjectively regarded as producing better samples than other methods.
\end{itemize}
At the same time, GANs have taken on a new disadvantage: training them requires finding
the Nash equilibrium of a game, which is a more difficult problem than optimizing an
objective function.

\section{How do GANs work?}

We have now seen several other generative models and explained that GANs do not
work in the same way that they do. But how do GANs themselves work?

\subsection{The GAN framework}

The basic idea of GANs is to set up a game between two players.
One of them is called the \newterm{generator}.
The generator creates samples that are intended to come from the
same distribution as the training data.
The other player is the \newterm{discriminator}.
The discriminator examines samples to determine whether they are real
or fake.
The discriminator learns using traditional supervised learning techniques,
dividing inputs into two classes (real or fake).
The generator is trained to fool the discriminator.
We can think of the generator as being like a counterfeiter, trying to
make fake money, and the discriminator as being like police, trying to
allow legitimate money and catch counterfeit money.
To succeed in this game, the counterfeiter must learn to make money that
is indistinguishable from genuine money, and the generator network must
learn to create samples that are drawn from the same distribution as the
training data.
The process is illustrated in \figref{fig:framework}.

\begin{figure}
\includegraphics[width=\textwidth]{framework}
\caption{The GAN framework pits two adversaries against each other in a game.
Each player is represented by a differentiable function controlled by a set
of parameters.
Typically these functions are implemented as deep neural networks.
The game plays out in two scenarios.
In one scenario, training examples $\vx$ are randomly sampled from the training
set and used as input for the first player, the discriminator, represented
by the function $D$. The goal of the discriminator is to output the probability
that its input is real rather than fake, under the assumption that half of the
inputs it is ever shown are real and half are fake.
In this first scenario, the goal of the discriminator is for $D(\vx)$ to be
near 1.
In the second scenario, inputs $\vz$ to the generator are randomly sampled from
the model's prior over the latent variables.
The discriminator then receives input $G(\vz)$, a fake sample created by the
generator.
In this scenario, both players participate. The discriminator strives to make
$D(G(z))$ approach 0 while the generative strives to make the same quantity
approach 1.
If both models have sufficient capacity, then the Nash equilibrium of this game
corresponds to the $G(\vz)$ being drawn from the same distribution as the training
data, and $D(\vx) = \frac{1}{2}$ for all $\vx$.
}
\label{fig:framework}
\end{figure}

Formally, GANs are a structured probabilistic model (see chapter 16 of
\citet{Goodfellow-et-al-2016} for an introduction to structured probabilistic
models) containing latent variables $\vz$ and observed variables $\vx$.
The graph structure is shown in \figref{fig:graph}.

The two players in the game are represented by two functions, each of which
is differentiable both with respect to its inputs and with respect to its
parameters.
The discriminator is a function $D$ that takes $\vx$ as input and uses
$\vtheta^{(D)}$ as parameters.
The generator is defined by a function $G$ that takes $\vz$ as input and
uses $\vtheta^{(G)}$ as parameters.

Both players have cost functions that are defined in terms of both players'
parameters.
The discriminator wishes to minimize $J^{(D)}\left( \vtheta^{(D)}, \vtheta^{(G)} \right)$
and must do so while controlling only $\vtheta^{(D)}$.
The generator wishes to minimize $J^{(G)}\left( \vtheta^{(D)}, \vtheta^{(G)} \right)$
and must do so while controlling only $\vtheta^{(G)}$.
Because each player's cost depends on the other player's parameters, but each player
cannot control the other player's parameters, this scenario is most
straightforward to describe as a game rather than as an optimization problem.
The solution to an optimization problem is a (local) minimum, a point in parameter space
where all neighboring points have greater or equal cost.
The solution to a game is a Nash equilibrium.
Here, we use the terminology of local differential Nash equilibria \citep{ratliff2013characterization}.
In this context, a Nash equilibrium is a tuple $(\vtheta^{(D)}, \vtheta^{(G)} )$
that is a local minimum of $J^{(D)}$ with respect to $\vtheta^{(D)}$ and a local
minimum of $J^{(G)}$ with respect to $\vtheta^{(G)}$.

\paragraph{The generator}
The generator is simply a differentiable function $G$.
When $\vz$ is sampled from some simple prior distribution,
$G(\vz)$ yields a sample of $\vx$ drawn from $\pmodel$.
Typically, a deep neural network is used to represent $G$.
Note that the inputs to the function $G$ do not need to correspond
to inputs to the first layer of the deep neural net; inputs may
be provided at any point throughout the network.
For example, we can partition $\vz$ into two vectors $\vz^{(1)}$
and $\vz^{(2)}$, then feed $\vz^{(1)}$ as input to the first layer
of the neural net and add $\vz^{(2)}$ to the last layer of the neural
net. If $\vz^{(2)}$ is Gaussian, this makes $\vx$ conditionally Gaussian
given $\vz^{(1)}$.
Another popular strategy is to apply additive or multiplicative noise to
hidden layers or concatenate noise to hidden layers of the neural net.
Overall, we see that there are very few restrictions on the design of
the generator net.
If we want $\pmodel$ to have full support on $\vx$ space we need the dimension
of $\vz$ to be at least as large as the dimension of $\vx$, and $G$ must be
differentiable, but those are the only requirements.
In particular, note that any model that can be trained with the nonlinear
ICA approach can be a GAN generator network.
The relationship with variational autoencoders is more complicated;
the GAN framework can train some models that the VAE framework cannot and vice
versa, but the two frameworks also have a large intersection.
The most salient difference is that, if relying on standard backprop,
VAEs cannot have discrete variables at the input to the generator,
while GANs cannot have discrete variables at the output of the generator.

\paragraph{The training process}
The training process consists of simultaneous SGD.
On each step, two minibatches are sampled: a minibatch of $\vx$ values from
the dataset and a minibatch of $\vz$ values drawn from the model's prior over
latent variables.
Then two gradient steps are made simultaneously:
one updating $\vtheta^{(D)}$ to reduce $J^{(D)}$
and one updating $\vtheta^{(G)}$ to reduce $J^{(G)}$.
In both cases, it is possible to use the gradient-based optimization algorithm
of your choice.
Adam \citep{kingma2014adam} is usually a good choice.
Many authors recommend running more steps of one player than the other, but
as of late 2016, the author's opinion is that the protocol that works the best
in practice is simultaneous gradient descent, with one step for each player.


\begin{figure}
  \centering
  \includegraphics{graph}
  \caption{The graphical model structure of GANs, which is also shared with
    VAEs, sparse coding, etc.
    It is directed graphical model where every latent variable influences
    every observed variable.
    Some GAN variants remove some of these connections.
  }
  \label{fig:graph}
\end{figure}

\subsection{Cost functions}

Several different cost functions may be used within the GANs framework.

\subsubsection{The discriminator's cost, $J^{(D)}$}

All of the different games designed for GANs so far use the same cost for the
discriminator, $J^{(D)}$. They differ only in terms of the cost used for the
generator, $J^{(G)}$.

The cost used for the discriminator is:
\begin{equation}
  J^{(D)}(\vtheta^{(D)}, \vtheta^{(G)}) = -\frac{1}{2} \E_{\vx \sim \pdata} \log D(\vx) - \frac{1}{2} \E_{\vz} \log \left(1 - D\left( G(z)\right) \right).
  \label{eq:discriminator_cost}
\end{equation}

This is just the standard cross-entropy cost that is minimized when training a standard binary classifier
with a sigmoid output.
The only difference is that the classifier is trained on two minibatches of data; one coming from the
dataset, where the label is $1$ for all examples, and one coming from the generator, where the label
is $0$ for all examples.

All versions of the GAN game encourage the discriminator to minimize \eqref{eq:discriminator_cost}.
In all cases, the discriminator has the same optimal strategy.
The reader is now encouraged to complete the exercise in \secref{sec:opt_d} and review its solution
given in \secref{sec:opt_d_soln}. This exercise shows how to derive the optimal discriminator strategy
and discusses the importance of the form of this solution.

We see that by training the discriminator, we are able to obtain an estimate of the ratio
\[
  \frac{\pdata(\vx)}{\pmodel(\vx)}
\]
at every point $\vx$.
Estimating this ratio enables us to compute a wide variety of divergences and their gradients.
This is the key approximation technique that sets GANs apart from variational autoencoders
and Boltzmann machines.
Other deep generative models make approximations based on lower bounds or Markov chains;
GANs make approximations based on using supervised learning to estimate a ratio of two densities.
The GAN approximation is subject to the failures of supervised learning: overfitting and underfitting.
In principle, with perfect optimization and enough training data, these failures can be overcome.
Other models make other approximations that have other failures.

Because the GAN framework can naturally be analyzed with the tools of game theory,
we call GANs ``adversarial.'' But we can also think of them as cooperative, in
the sense that the discriminator estimates this ratio of densities and then freely
shares this information with the generator.
From this point of view, the discriminator is more like a teacher instructing the
generator in how to improve than an adversary.
So far, this cooperative view has not led to any particular change in the development
of the mathematics.

\subsubsection{Minimax}
\label{sec:minimax}

So far we have specified the cost function for only the discriminator.
A complete specification of the game requires that we specify a cost function also
for the generator.

The simplest version of the game is a \newterm{zero-sum game}, in which the sum of all player's
costs is always zero.
In this version of the game,
\begin{equation}
J^{(G)} = - J^{(D)}.
\label{eq:minimax}
\end{equation}

Because $J^{(G)}$ is tied directly to $J^{(D)}$, we can summarize the entire game with a
\newterm{value function} specifying the discriminator's payoff:
\[ V\left(\vtheta^{(D)}, \vtheta^{(G)} \right) = - J^{(D)} \left(\vtheta^{(D)}, \vtheta^{(G)} \right).\]

Zero-sum games are also called \newterm{minimax} games because their solution involves minimization
in an outer loop and maximization in an inner loop:
\[ \vtheta^{(G)*} = \argmin_{\vtheta^{(G)}} \max_{\vtheta^{(D)}} V\left(\vtheta^{(D)}, \vtheta^{(G)} \right) . \]

The minimax game is mostly of interest because it is easily amenable to theoretical analysis.
\citet{Goodfellow-et-al-NIPS2014-small} used this variant of the GAN game to show that learning in
this game resembles minimizing the Jensen-Shannon divergence between the data and the model distribution,
and that the game converges to its equilibrium if both players' policies can be updated directly in
function space.
In practice, the players are represented with deep neural nets and updates are made in parameter space,
so these results, which depend on convexity, do not apply.

\subsubsection{Heuristic, non-saturating game}
\label{sec:heuristic}

The cost used for the generator in the minimax game (\eqref{eq:minimax}) is useful for theoretical analysis,
but does not perform especially well in practice.

Minimizing the cross-entropy between a target class and a classifier's predicted distribution
is highly effective because the cost never saturates when the classifier has the wrong output.
The cost does eventually saturate, approaching zero, but only when the classifier has already
chosen the correct class.

In the minimax game, the discriminator minimizes a cross-entropy, but the generator maximizes
the same cross-entropy.
This is unfortunate for the generator, because when the discriminator successfully rejects
generator samples with high confidence, the generator's gradient vanishes.

To solve this problem, one approach is to continue to use cross-entropy minimization for the
generator.
Instead of flipping the sign on the discriminator's cost to obtain a cost for the generator,
we flip the target used to construct the cross-entropy cost.
The cost for the generator then becomes:
\[
  J^{(G)} = -\frac{1}{2} \E_\vz \log D(G(\vz))
\]

In the minimax game, the generator minimizes the log-probability of the discriminator being correct.
In this game, the generator maximizes the log-probability of the discriminator being mistaken.

This version of the game is heuristically motivated, rather than being motivated by a theoretical
concern.
The sole motivation for this version of the game is to ensure that each player has a strong
gradient when that player is ``losing'' the game.

In this version of the game, the game is no longer zero-sum, and cannot be described with a single
value function.

\subsubsection{Maximum likelihood game}
\label{sec:mle_gan}

We might like to be able to do maximum likelihood learning with GANs, which would mean minimizing
the KL divergence between the data and the model, as in \eqref{eq:kl}.
Indeed, in \secref{sec:tree}, we said that GANs could optionally implement maximum likelihood,
for the purpose of simplifying their comparison to other models.

There are a variety of methods of approximating \eqref{eq:kl} within the GAN
framework.
\citet{Goodfellow-ICLR2015} showed that using
\[ J^{(G)} = -\frac{1}{2} \E_z \exp\left( \sigma^{-1} \left( D(G(\vz)) \right) \right), \]
  where $\sigma$ is the logistic sigmoid function, is equivalent to minimizing \eqref{eq:kl},
  under the assumption that the discriminator is optimal.
  This equivalence holds in expectation; in practice, both stochastic gradient descent on the KL
  divergence and the GAN training procedure will have some variance around the true expected
  gradient due to the use of sampling (of $\vx$ for maximum likelihood and $\vz$ for GANs)
  to construct the estimated gradient.
  The demonstration of this equivalence is an exercise (\secref{sec:mle_exercise}
  with the solution in \secref{sec:mle_soln}).


  Other methods of approximating maximum likelihood within the GANs framework are possible.
  See for example \citet{nowozin2016f}.
  
 

  \subsubsection{Is the choice of divergence a distinguishing feature of GANs?}
  \label{sec:which_divergence}

  As part of our investigation of how GANs work, we might wonder exactly what it is
  that makes them work well for generating samples.

  Previously, many people (including the author) believed that GANs produced sharp,
  realistic samples because they minimize the Jensen-Shannon divergence while
  VAEs produce blurry samples because they minimize the KL divergence between the
  data and the model.

  The KL divergence is not symmetric; minimizing $\KL(\pdata \Vert \pmodel)$
  is different from minimizing $\KL(\pmodel \Vert \pdata)$.
  Maximum likelihood estimation performs the former; minimizing the Jensen-Shannon
  divergence is somewhat more similar to the latter.
  As shown in \figref{fig:kl}, the latter might be expected to yield better samples
  because a model trained with this divergence would prefer
  to generate samples that come only from modes in the training distribution
  even if that means ignoring some modes, rather than
  including all modes but generating some samples that do not come 
  from any training set mode.

  \begin{figure}
    \centering
    \includegraphics[width=\figwidth]{kl}
    \caption{
The two directions of the KL divergence are not equivalent.
The differences are most obvious when the model has too little
capacity to fit the data distribution.
Here we show an example of a distribution over one-dimensional
data $x$.
In this example, we use a mixture of two Gaussians as the data
distribution, and a single Gaussian as the model family.
Because a single Gaussian cannot capture the true data distribution,
the choice of divergence determines the tradeoff that the model makes.
On the left, we use the maximum likelihood criterion.
The model chooses to average out the two modes, so that it places high
probability on both of them.
On the right, we use the reverse order of the arguments to the KL
divergence, and the model chooses to capture only one of the two modes.
It could also have chosen the other mode; the two are both local minima
of the reverse KL divergence.
We can think of $\KL( \pdata \Vert \pmodel )$ as preferring to place
high probability everywhere that the data occurs,
and $\KL( \pmodel \Vert \pdata )$ as preferrring to place low probability
wherever the data does not occur.
From this point of view, one might expect
$\KL( \pmodel \Vert \pdata )$
to yield more visually pleasing samples,
because the model will not choose to generate unusual samples lying
between modes of the data generating distribution.
    }
    \label{fig:kl}
  \end{figure}

Some newer evidence suggests that the use of the Jensen-Shannon divergence does
not explain why GANs make sharper samples:
\begin{itemize}
  \item It is now possible to train GANs using maximum likelihood, as described in
    \secref{sec:mle_gan}.
    These models still generate sharp samples, and still select a small number of modes.
    See \figref{fig:fgan}.
  \item
    GANs often choose to generate from very few modes; fewer than the limitation
    imposed by the model capacity.
    The reverse KL prefers to generate from {\em as many modes of the data distribution as the model is able to};
    it does not prefer fewer modes in general.
    This suggests that the mode collapse is driven by a factor other than the choice 
    of divergence.
\end{itemize}

\begin{figure}
\centering
\includegraphics[width=\figwidth]{fgan}
\caption{
The f-GAN model is able to minimize many different divergences.
Because models trained to minimize $\KL(\pdata \Vert \pmodel)$
still generate sharp samples and tend to select a small number
of modes, we can conclude that the use of the Jensen-Shannon
divergence is not a particular important distinguishing
characteristic of GANs and that it does not explain why their
samples tend to be sharp.
}
\label{fig:fgan}
\end{figure}

Altogether, this suggests that GANs choose to generate a small number of modes
due to a defect in the training procedure, rather than due to the divergence
they aim to minimize.
This is discussed further in \secref{sec:mode_collapse}.
The reason that GANs produce sharp samples is not entirely clear.
It may be that the family of models trained using GANs is different from the family
of models trained using VAEs (for example, with GANs it is easy to make models
where $\vx$ has a more complicated distribution than just an isotropic Gaussian
conditioned on the input to the generator).
It may also be that the kind of approximations that GANs make have different
effects than the kind of approximations that other frameworks make.

\subsubsection{Comparison of cost functions}

We can think of the generator network as learning by a strange kind of reinforcement
learning.
Rather than being told a specific output $\vx$ it should associate with each $\vz$,
the generator takes actions and receives rewards for them.
In particular, note that $J^{(G)}$ does not make reference to the training data
directly at all; all information about the training data comes only through what
the discriminator has learned. (Incidentally, this makes GANs resistant to overfitting,
because the generator has no opportunity in practice to directly copy training examples)
The learning process differs somewhat from traditional reinforcement learning because
\begin{itemize}
 \item The generator is able to observe not just the output of the reward function but
   also its gradients.
 \item The reward function is non-stationary; the reward is based on the discriminator
   which learns in response to changes in the generator's policy.
   \end{itemize}

In all cases, we can think of the sampling process that begins with the selection of
a specific $\vz$ value as an episode that receives a single reward, independent of the
actions taken for all other $\vz$ values.
The reward given to the generator is a function of a single scalar value,
$D(G(\vz))$.
We usually think of this in terms of cost (negative reward).
The cost for the generator is always monotonically decreasing in $D(G(\vz))$ but
different games are designed to make this cost decrease faster along different parts of the curve.

\Figref{fig:comparison} shows the cost response curves as functions of $D(G(\vz))$ for three
different variants of the GAN game.
We see that the maximum likelihood game gives very high variance in the cost, with most of the
cost gradient coming from the very few samples of $\vz$ that correspond to the samples that
are most likely to be real rather than fake.
The heuristically designed non-saturating cost has lower sample variance, which may explain
why it is more successful in practice.
  This suggests that variance reduction techniques could be an important research area
  for improving the performance of GANs, especially GANs based on maximum likelihood.

\begin{figure}
\centering
\includegraphics[width=\figwidth]{comparison}
\caption{
  The cost that the generator receives for generating a samples $G(\vz)$ depends only on
  how the discriminator responds to that sample.
  The more probability the discriminator assigns to the sample being real, the less cost
  the generator receives.
  We see that when the sample is likely to be fake, both the minimax game and the maximum
  likelihood game have very little gradient, on the flat left end of the curve.
  The heuristically motivated non-saturating cost avoids this problem.
  Maximum likelihood also suffers from the problem that nearly all of the gradient comes
  from the right end of the curve, meaning that a very small number of samples dominate
  the gradient computation for each minibatch.
  This suggests that variance reduction techniques could be an important research area
  for improving the performance of GANs, especially GANs based on maximum likelihood.
  Figure reproduced from \citet{Goodfellow-ICLR2015}.
}
\label{fig:comparison}
\end{figure}



\subsection{The DCGAN architecture}

Most GANs today are at least loosely based on the DCGAN architecture \citep{radford2015unsupervised}.
DCGAN stands for ``deep, convolution GAN.'' Though GANs were both deep and convolutional prior to
DCGANs, the name DCGAN is useful to refer to this specific style of architecture.
Some of the key insights of the DCGAN architecture were to:
\begin{itemize}
  \item Use batch normalization \citep{Ioffe+Szegedy-2015} layers in most layers of both the discriminator and the generator,
        with the two minibatches for the discriminator normalized separately.
        The last layer of the generator and first layer of the discriminator are not batch normalized,
        so that the model can learn the correct mean and scale of the data distribution.
        See \figref{fig:dcgan}.
  \item The overall network structure is mostly borrowed from the all-convolutional net \citep{Springenberg2015}.
        This architecture contains neither pooling nor ``unpooling'' layers.
        When the generator needs to increase the spatial dimension of the representation
        it uses transposed convolution with a stride greater than 1.
  \item The use of the Adam optimizer rather than SGD with momentum.
\end{itemize}

\begin{figure}
\centering
\includegraphics[width=\textwidth]{dcgan}
  \caption{The generator network used by a DCGAN. Figure reproduced from \citet{radford2015unsupervised}.}
\label{fig:dcgan}
\end{figure}

Prior to DCGANs, LAPGANs \citep{denton2015deep} were the only version of GAN
that had been able to scale to high resolution images.
LAPGANs require a multi-stage generation process in which multiple GANs
generate different levels of detail in a Laplacian pyramid representation
of an image.
DCGANs were the first GAN model to learn to generate high resolution images
in a single shot.
As shown in \figref{fig:dcgan_lsun}, DCGANs are able to generate high quality
images when trained on restricted domains of images, such as images of bedrooms.
DCGANs also clearly demonstrated that GANs learn to use their latent code
in meaningful ways, with simple arithmetic operations in latent space
having clear interpretation as arithmetic operations on semantic attributes
of the input, as demonstrated in \figref{fig:dcgan_face_arithmetic}.


\begin{figure}
  \centering
  \includegraphics[width=\figwidth]{dcgan_lsun}
  \caption{Samples of images of bedrooms generated by a DCGAN trained on the LSUN dataset.}
  \label{fig:dcgan_lsun}
\end{figure}

\begin{figure}
\centering
$
\vcenter{
    \hbox{%
\includegraphics[width=.15\figwidth]{man_with_glasses.png} %
    }
}
\vcenter{\hbox{-}} %
\vcenter{\hbox{
\includegraphics[width=.15\figwidth]{man_without_glasses.png} %
}}
\vcenter{\hbox{+}} %
\vcenter{\hbox{
\includegraphics[width=.15\figwidth]{woman_without_glasses.png} %
}}
\vcenter{\hbox{=}} %
\vcenter{\hbox{
\includegraphics[width=.45\figwidth]{woman_with_glasses.png}
}
}
$
\caption{DCGANs demonstrated that GANs can learn a distributed representation
that disentangles the concept of gender from the concept of wearing
glasses. If we begin with the representation of the concept of a
man with glasses, then subtract the vector representing the concept
of a man without glasses, and finally add the vector representing
the concept of a woman without glasses, we obtain the vector representing
the concept of a woman with glasses. The generative model correctly
decodes all of these representation vectors to images that may be
recognized as belonging to the correct class.
Images reproduced from \citet{radford2015unsupervised}.
}
\label{fig:dcgan_face_arithmetic}
\end{figure}

\subsection{How do GANs relate to noise-contrastive estimation and maximum likelihood?}

While trying to understand how GANs work, one might naturally wonder about
how they are connected to \newterm{noise-constrastive estimation} (NCE)
\citep{Gutmann+Hyvarinen-2010}.
Minimax GANs use the cost function from NCE as their value function, so the
methods seem closely related at face value.
It turns out that they actually learn very different things, because the
two methods focus on different players within this game.
Roughly speaking, the goal of NCE is to learn the density model within the
discriminator, while the goal of GANs is to learn the sampler defining
the generator.
While these two tasks seem closely related at a qualitative level, the
gradients for the tasks are actually quite different.
Surprisingly, maximum likelihood turns out to be closely related to NCE,
and corresponds to playing a minimax game with the same value function,
but using a sort of heuristic update strategy rather than gradient descent
for one of the two players.
The connections are summarized in \figref{fig:nce}.

\begin{figure}
\centering
\includegraphics[width=\figwidth]{nce}
\caption{
  \citet{Goodfellow-ICLR2015} demonstrated the following
  connections between minimax GANs, noise-contrastive estimation, and
  maximum likelihood: all three can be interpreted as strategies
  for playing a minimax game with the same value function.
  The biggest difference is in where $\pmodel$ lies.
  For GANs, the generator is $\pmodel$, while for NCE and MLE,
  $\pmodel$ is part of the discriminator.
  Beyond this, the differences between the methods lie in the update
  strategy.
  GANs learn both players with gradient descent.
  MLE learns the discriminator using gradient descent, but has a heuristic
  update rule for the generator.
  Specifically, after each discriminator update step, MLE copies the density model learned inside the discriminator and 
  converts it into a sampler to be used as the generator.
  NCE never updates the generator; it is just a fixed source of noise.
}
\label{fig:nce}
\end{figure}

\section{Tips and Tricks}

Practitioners use several tricks to improve the performance of GANs.
It can be difficult to tell how effective some of these tricks are;
many of them seem to help in some contexts and hurt in others.
These should be regarded as techniques that are worth trying out,
not as ironclad best practices.

NIPS 2016 also featured a workshop on adversarial training, with
an invited talk by Soumith Chintala called "How to train a GAN."
This talk has more or less the same goal as this portion of the tutorial,
with a different collection of advice.
To learn about tips and tricks not included in this tutorial, check
out the GitHub repository associated with Soumith's talk:

\url{https://github.com/soumith/ganhacks}


\subsection{Train with labels}

Using labels in any way, shape or form almost always results in a dramatic
improvement in the subjective quality of the samples generated by the model.
This was first observed by \citet{denton2015deep}, who built class-conditional
GANs that generated much better samples than GANs that were free to generate
from any class.
Later, \citet{salimans2016improved} found that sample quality improved
even if the generator did not explicitly incorporate class information; training
 the discriminator to recognize specific classes of real objects is sufficient.

 It is not entirely clear why this trick works.
 It may be that the incorporation of class information gives the training
 process useful clues that help with optimization.
 It may also be that this trick gives no objective improvement in sample quality,
 but instead biases the samples toward taking on properties that the human
 visual system focuses on.
 If the latter is the case, then this trick may not result in a better model
 of the true data-generating distribution, but it still helps to create media
 for a human audience to enjoy and may help an RL agent to carry out tasks
 that rely on knowledge of the same aspects of the environment that are relevant
 to human beings.

 It is important to compare results obtained using this trick only to other
 results using the same trick; models trained with labels should be compared
 only to other models trained with labels, class-conditional models should
 be compared only to other class-conditional models.
 Comparing a model that uses labels to one that does not is unfair and an
 uninteresting benchmark, much as a convolutional model can usually be expected
 to outperform a non-convolutional model on image tasks.

\subsection{One-sided label smoothing}
\label{sec:label_smooth}

GANs are intended to work when the discriminator estimates a ratio of two
densities, but deep neural nets are prone to producing highly confident
outputs that identify the correct class but with too extreme of a probability.
This is especially the case when the input to the deep network is adversarially
constructed; the classifier tends to linearly extrapolate and produce
extremely confident predictions \citep{Goodfellow-2015-adversarial}.

To encourage the discriminator to estimate soft probabilities rather than
to extrapolate to extremely confident classification, we can use a technique
called \newterm{one-sided label smoothing} \citep{salimans2016improved}.

Usually we train the discriminator using \eqref{eq:discriminator_cost}.
We can write this in TensorFlow \citep{tensorflow} code as:
\begin{lstlisting}
d_on_data = discriminator_logits(data_minibatch)
d_on_samples = discriminator_logits(samples_minibatch)
loss = tf.nn.sigmoid_cross_entropy_with_logits(d_on_data, 1.) + \
       tf.nn.sigmoid_cross_entropy_with_logits(d_on_samples, 0.)
\end{lstlisting}

The idea of one-sided label smoothing is to replace the target for the real examples
with a value slightly less than one, such as .9:

\begin{lstlisting}
loss = tf.nn.sigmoid_cross_entropy_with_logits(d_on_data, .9) + \
       tf.nn.sigmoid_cross_entropy_with_logits(d_on_samples, 0.)
\end{lstlisting}

This prevents extreme extrapolation behavior in the discriminator; if it learns
to predict extremely large logits corresponding to a probability approaching $1$
for some input, it will be penalized and encouraged to bring the logits back
down to a smaller value.

It is important to not smooth the labels for the fake samples.
Suppose we use a target of $1-\alpha$ for the real data and a target of $0+\beta$
for the fake samples.
Then the optimal discriminator function is
\[ D^*(\vx) = \frac{(1-\alpha) \pdata(\vx) + \beta \pmodel(\vx)} { \pdata(\vx) + \pmodel(\vx) }.\]

When $\beta$ is zero, then smoothing by $\alpha$ does nothing but scale down the optimal value
of the discriminator.
When $\beta$ is nonzero, the shape of the optimal discriminator function changes.
In particular, in a region where $\pdata(\vx)$ is very small and $\pmodel(\vx)$ is larger,
$D^*(\vx)$ will have a peak near the spurious mode of $\pmodel(\vx)$.
The discriminator will thus reinforce incorrect behavior in the generator; the generator
will be trained either to produce samples that resemble the data or to produce samples
that resemble the samples it already makes.

One-sided label smoothing is a simple modification of the much older label smoothing
technique, which dates back to at least the 1980s.
\citet{Szegedy-et-al-2015} demonstrated that label smoothing is an excellent regularizer
in the context of convolutional networks for object recognition.
One reason that label smoothing works so well as a regularizer is that it does not
ever encourage the model to choose an incorrect class on the training set, but only
to reduce the confidence in the correct class.
Other regularizers such as weight decay often encourage some misclassification
if the coefficient on the regularizer is set high enough.
\citet{wardefarley2016} showed that label smoothing can help to reduce vulnerability to
adversarial examples, which suggests that label smoothing should help the discriminator
more efficiently learn to resist attack by the generator.

\subsection{Virtual batch normalization}

Since the introduction of DCGANs, most GAN architectures have involved some form
of batch normalization.
The main purpose of batch normalization is to improve the optimization of the model,
by reparameterizing the model so that the mean and variance of each feature are determined
by a single mean parameter and a single variance parameter associated with that feature,
rather than by a complicated interaction between all of the weights of all of the layers
used to extract the feature.
This reparameterization is accomplished by subtracting the mean and dividing by the standard
deviation of that feature on a minibatch of data.
It is important that the normalization operation is {\em part of the model},
so that back-propgation computes the gradient of features that are defined to always
be normalized.
The method is much less effect if features are frequently renormalized after learning
without the normalization defined as part of the model.

Batch normalization is very helpful, but for GANs has a few unfortunate side effects.
The use of a different minibatch of data to compute the normalization statistics
on each step of training results in fluctuation of these normalizing constants.
When minibatch sizes are small (as is often the case when trying to fit a large generative
model into limited GPU memory) these fluctuations can become large enough that they
have more effect on the image generated by the GAN than the input $\vz$ has.
See \figref{fig:bad_batchnorm} for an example.

\begin{figure}
\centering
\includegraphics[width=\figwidth]{bad_batchnorm}
\caption{
Two minibatches of sixteen samples each, generated by a generator network using
batch normalization.
These minibatches illustrate a problem that occurs occasionally when using batch
normalization: fluctuations in the mean and standard deviation of feature values
in a minibatch can have a greater effect than the individual $\vz$ codes for individual
images within the minibatch.
This manifests here as one minibatch containing all orange-tinted samples and the other
containing all green-tinted samples.
The examples within a minibatch should be independent from each other, but in this
case, batch normalization has caused them to become correlated with each other.
}
\label{fig:bad_batchnorm}
\end{figure}

\citet{salimans2016improved} introduced techniques to mitigate this problem.
\newterm{Reference batch normalization} consists of running the network twice:
once on a minibatch of \newterm{reference examples} that are sampled once at the
start of training and never replaced, and once on the current minibatch of examples
to train on.
The mean and standard deviation of each feature are computed using the reference
batch. The features for both batches are then normalized using these computed statistics.
A drawback to reference batch normalization is that the model can overfit to the
reference batch. To mitigate this problem slightly, one can instead use
\newterm{virutal batch normalization}, in which the normalization statistics for each
example are computed using the union of that example and the reference batch.
Both reference batch normalization and virtual batch normalization have the property
that all examples in the training minibatch are processed independently from each other,
and all samples produced by the generator (except those defining the reference batch)
are i.i.d.

\subsection{Can one balance $G$ and $D$?}

Many people have an intuition that it is necessary to somehow balance the two players
to prevent one from overpowering the other.
If such balance is desirable and feasible, it has not yet been demonstrated in any
compelling fashion.

The author's present belief is that GANs work by estimating the ratio of the data density
and model density. This ratio is estimated correctly only when the discriminator is
optimal, so it is fine for the discriminator to overpower the generator.

Sometimes the gradient for the generator can vanish when the discriminator becomes
too accurate.
The right way to solve this problem is not to limit the power of the discriminator,
but to use a parameterization of the game where the gradient does not vanish
(\secref{sec:heuristic}).

Sometimes the gradient for the generator can become very large if the discriminator
becomes too confident. Rather than making the discriminator less accurate, a better
way to resolve this problem is to use one-sided label smoothing (\secref{sec:label_smooth}).

The idea that the discriminator should always be optimal in order to best estimate
the ratio would suggest training the discriminator for $k > 1$ steps every time
the generator is trained for one step. In practice, this does not usually result in a
clear improvement.

One can also try to balance the generator and discriminator by choosing the model
size.
In practice, the discriminator is usually deeper and sometimes has more filters
per layer than the generator.
This may be because it is important for the discriminator to be able to correctly
estimate the ratio between the data density and generator density, but it may
also be an artifact of the mode collapse problem---since the generator tends not
to use its full capacity with current training methods, practitioners presumably
do not see much of a benefit from increasing the generator capacity.
If the mode collapse problem can be overcome, generator sizes will presumably
increase. It is not clear whether discriminator sizes will increase proportionally.



\section{Research Frontiers}

GANs are a relatively new method, with many research directions still
remaining open.

\subsection{Non-convergence}

The largest problem facing GANs that researchers should try to resolve is the issue
of non-convergence.

Most deep models are trained using an optimization algorithm that seeks out a low
value of a cost function.
While many problems can interfere with optimization, optimization algorithms usually
make reliable downhill progress.
GANs require finding the equilibrium to a game with two players.
Even if each player successfully moves downhill on that player's update,
the same update might move the other player uphill.
Sometimes the two players eventually reach an equilibrium, but in other scenarios
they repeatedly undo each others' progress without arriving anywhere useful.
This is a general problem with games not unique to GANs, so a general solution
to this problem would have wide-reaching applications.

To gain some intuition for how gradient descent performs when applied to games
rather than optimization, the reader is encouraged to solve the exercise in
\secref{sec:xy_exercise} and review its solution in \secref{sec:xy_soln} now.

Simultaneous gradient descent converges for some games but not all of them.

In the case of the minimax GAN game (\secref{sec:minimax}),
\citet{Goodfellow-et-al-NIPS2014-small} showed that simultaneous gradient
descent converges {\em if the updates are made in function space}.
In practice, the updates are made in parameter space, so the convexity
properties that the proof relies on do not apply.
Currently, there is neither a theoretical argument that GAN games should
converge when the updates are made to parameters of deep neural networks,
nor a theoretical argument that the games should not converge.

In practice, GANs often seem to oscillate, somewhat like what happens in
the toy example in \secref{sec:xy_soln}, meaning that they progress from
generating one kind of sample to generating another kind of sample without
eventually reaching an equilibrium.

Probably the most common form of harmful non-convergence encountered in the GAN
game is mode collapse.

\subsubsection{Mode collapse}
\label{sec:mode_collapse}

Mode collapse, also known as \newterm{the Helvetica scenario}, is a problem that occurs
when the generator learns to map several different input $\vz$ values to the same output
point.
In practice, complete mode collapse is rare, but partial mode collapse is common.
Partial mode collapse refers to scenarios in which the
generator makes multiple images that contain the same color or texture themes,
or multiple images containing different views of the same dog.
The mode collapse problem is illustrated in \figref{fig:mode_collapse}.

Mode collapse may arise because the maximin solution to the GAN game is different
from the minimax solution.
When we find the model
\[ G^* = \min_G \max_D V(G,D), \]
$G^*$ draws samples from the data distribution.
When we exchange the order of the min and max and find
\[ G^* = \max_D \min_G V(G, D), \]
the minimization with respect to the generator now lies in the inner
loop of the optimization procedure.
The generator is thus asked to map every $\vz$ value to the single $\vx$
coordinate that the discriminator believes is most likely to be real rather than fake.
Simultaneous gradient descent does not clearly privilege $\min \max$ over $\max \min$
or vice versa. We use it in the hope that it will behave like $\min \max$ but it
often behaves like $\max \min$.


\begin{figure}
\centering
\includegraphics[width=\figwidth]{mode_collapse}
\caption{
An illustration of the mode collapse problem on a two-dimensional toy dataset.
In the top row, we see the target distribution $\pdata$ that the model should
learn. It is a mixture of Gaussians in a two-dimensional space.
In the lower row, we see a series of different distributions learned over time
as the GAN is trained.
Rather than converging to a distribution containing all of the modes in the
training set, the generator only ever produces a single mode at a time, cycling
between different modes as the discriminator learns to reject each one.
Images from \citet{metz2016unrolled}.
}
\label{fig:mode_collapse}
\end{figure}

As discussed in \secref{sec:which_divergence}, mode collapse does not seem to be
caused by any particular cost function.
It is commonly asserted that mode collapse is caused by the use of Jensen-Shannon
divergence, but this does not seem to be the case, because GANs that minimize
approximations of $\KL(\pdata \Vert \pmodel)$ face the same issues, and because
the generator often collapses to even fewer modes than would be preferred by the
Jensen-Shannon divergence.

Because of the mode collapse problem, applications of GANs are often limited to
problems where it is acceptable for the model to produce a small number of 
distinct outputs, usually tasks where the goal is to map some input to one of
many acceptable outputs.
As long as the GAN is able to find a small number of these acceptable outputs,
it is useful.
One example is text-to-image synthesis, in which the input is a caption for an
image, and the output is an image matching that description.
See \figref{fig:text2im} for a demonstration of this task.
In very recent work, \citet{reedgenerating} have shown that other models have
higher output diversity than GANs for such tasks (\figref{fig:low_diversity}),
but StackGANs \citep{zhang2016stackgan} seem to have higher output diversity than previous GAN-based
approaches (\figref{fig:stackgan}).


\begin{figure}
\centering
\includegraphics[width=\textwidth]{text2im}
\caption{
Text-to-image synthesis with GANs.
Image reproduced from \citet{reed2016generative}.
}
\label{fig:text2im}
\end{figure}

\begin{figure}
  \centering
  \includegraphics[width=\textwidth]{low_diversity}
  \caption{GANs have low output diversity for text-to-image
    tasks because of the mode collapse problem.
    Image reproduced from \citet{reedgenerating}.
  }
  \label{fig:low_diversity}
\end{figure}

\begin{figure}
  \centering
  \includegraphics[width=\textwidth]{stackgan}
  \caption{StackGANs are able to achieve higher output
    diversity than other GAN-based text-to-image models.
    Image reproduced from \citet{zhang2016stackgan}.}
    \label{fig:stackgan}
  \end{figure}

The mode collapse problem is probably the most important issue with GANs that
researchers should attempt to address.

One attempt is \newterm{minibatch features} \citep{salimans2016improved}.
The basic idea of minibatch features is to allow the discriminator to compare
an example to a minibatch of generated samples and a minibatch of real samples.
By measuring distances to these other samples in latent spaces, the discriminator
can detect if a sample is unusually similar to other generated samples.
Minibatch features work well.
It is strongly recommended to directly copy the Theano/TensorFlow code released
with the paper that introduced them, since small changes in the definition of the
features result in large reductions in performance.

Minibatch GANs trained on CIFAR-10 obtain excellent results, with most samples
being recognizable as specific CIFAR-10 classes (\figref{fig:minibatch_cifar}).
When trained on $128 \times 128$ ImageNet, few images are recognizable as belonging
to a specific ImageNet class (\figref{fig:minibatch_imagenet}).
Some of the better images are cherry-picked into \figref{fig:cherry}.

\begin{figure}
  \centering
  \includegraphics[width=\figwidth]{minibatch_cifar}
  \caption{
    Minibatch GANs trained on CIFAR-10 obtain excellent results, with most samples
    being recognizable as specific CIFAR-10 classes.
    (Note: this model was trained with labels)
  }
  \label{fig:minibatch_cifar}
\end{figure}

\begin{figure}
  \centering
  \includegraphics[width=\figwidth]{minibatch_imagenet}
  \caption{
    Minibatch GANS trained with labels on $128 \times 128$ ImageNet produce images
    that are occasionally recognizable as belonging to specific
    classes.
  }
  \label{fig:minibatch_imagenet}
\end{figure}

\begin{figure}
  \centering
  \includegraphics[width=\figwidth]{cherry}
  \caption{
    Minibatch GANs sometimes produce very good images when trained on $128 \times 128$
    ImageNet, as demonstrated by these cherry-picked examples.
  }
  \label{fig:cherry}
\end{figure}

Minibatch GANs have reduced the mode collapse problem enough that other problems, such
as difficulties with counting, perspective, and global structure become the most obvious
defects (\figref{fig:counting}, \figref{fig:perspective}, and \figref{fig:structure},
respectively).
Many of these problems could presumably be resolved by designing better model architectures.


\begin{figure}
  \centering
  \includegraphics[width=\figwidth]{counting}
  \caption{
    GANs on $128\times 128$ ImageNet seem to have trouble with counting, often generating
    animals with the wrong number of body parts.
  }
  \label{fig:counting}
\end{figure}

\begin{figure}
  \centering
  \includegraphics[width=\figwidth]{perspective}
  \caption{
    GANs on $128\times 128$ ImageNet seem to have trouble with the idea of three-dimensional
    perspective, often generating images of objects that are too flat or highly axis-aligned.
    As a test of the reader's discriminator network, one of these images is actually real.
  }
  \label{fig:perspective}
\end{figure}

\begin{figure}
  \centering
  \includegraphics[width=\figwidth]{structure}
  \caption{
    GANs on $128\times 128$ ImageNet seem to have trouble coordinating global structure,
    for example, drawing ``Fallout Cow,'' an animal that has both quadrupedal and bipedal structure.
  }
  \label{fig:structure}
\end{figure}

Another approach to solving the mode collapse problem is \newterm{unrolled GANs} \citep{metz2016unrolled}.
Ideally, we would like to find $G^* = \argmin_G \max_D V(G, D)$.
In practice, when we simultaneously follow the gradient of $V(G, D)$ for both players, we essentially ignore
the $\max$ operation when computing the gradient for $G$.
Really, we should regard $\max_D V(G,D)$ as the cost function for $G$, and we should back-propagate through
the maximization operation.
Various strategies exist for back-propagating through a maximization operation, but many, such as those
based on implicit differentiation, are unstable.
The idea of unrolled GANs is to build a computational graph describing $k$ steps of learning
in the discriminator, then backpropagate through all $k$ of these steps of learning
when computing the gradient on the generator.
Fully maximizing the value function for the discriminator takes tens of thousands of steps,
but \citet{metz2016unrolled} found that unrolling for even small numbers of steps, like 10 or fewer,
can noticeably reduce the mode dropping problem.
This approach has not yet been scaled up to ImageNet.
See \figref{fig:unrolled} for a demonstration of unrolled GANs on a toy problem.

\begin{figure}
  \centering
  \includegraphics[width=\figwidth]{unrolled}
  \caption{Unrolled GANs are able to fit all of the modes of a mixture of Gaussians
    in a two-dimensional space. Image reproduced from \citet{metz2016unrolled}.
  }
  \label{fig:unrolled}
\end{figure}

\subsubsection{Other games}

If our theory of how to understand whether a continuous, high-dimensional non-convex game 
will converge could be improved, or if we could develop algorithms that converge 
more reliably than simultaneous gradient descent, several application areas besides GANs
would benefit.
Even restricted to just AI research, we find games in many scenarios:
\begin{itemize}
  \item Agents that literally play games, such as AlphaGo \citep{silver2016mastering}.
  \item Machine learning security, where models must resist adversarial examples \citep{Szegedy-ICLR2014,Goodfellow-2015-adversarial}.
  \item Domain adaptation via domain-adversarial learning \citep{ganin2015domain}.
  \item Adversarial mechanisms for preserving privacy \citep{edwards2015censoring}.
  \item Adversarial mechanisms for cryptography \citep{abadi2016learning}.
\end{itemize}
This is by no means an exhaustive list.

\subsection{Evaluation of generative models}

Another highly important research area related to GANs is that it is not clear
how to quantitatively evaluate generative models.
Models that obtain good likelihood can generate bad samples, and models that
generate good samples can have poor likelihood.
There is no clearly justified way to quantitatively score samples.
GANs are somewhat harder to evaluate than other generative models because
it can be difficult to estimate the likelihood for GANs
(but it is possible---see \citet{wu2016quantitative}).
\citet{Theis2015d} describe many of the difficulties with evaluating generative models.

\subsection{Discrete outputs}

The only real requirement imposed on the design of the generator by the GAN framework
is that the generator must be differentiable.
Unfortunately, this means that the generator cannot produce discrete data, such
as one-hot word or character representations.
Removing this limitation is an important research direction that could unlock the
potential of GANs for NLP.
There are at least three obvious ways one could attack this problem:
\begin{enumerate}
  \item Using the REINFORCE algorithm \citep{Williams-1992}.
  \item Using the concrete distribution \citep{maddison2016concrete} or Gumbel-softmax \citep{jang2016categorical}.
  \item Training the generate to sample continuous values that can be decoded to discrete ones (e.g., sampling
    word embeddings directly).
\end{enumerate}

\subsection{Semi-supervised learning}
\label{sec:ssl}

A research area where GANs are already highly successful is the use of generative
models for semi-supervised learning, as proposed but not demonstrated in the original
GAN paper \citep{Goodfellow-et-al-NIPS2014-small}.

GANs have been successfully applied to semi-supervised learning at least since the introduction
of CatGANs \citep{springenberg2015unsupervised}.
Currently, the state of the art in semi-supervised learning on MNIST, SVHN, and CIFAR-10
is obtained by \newterm{feature matching GANs} \citep{salimans2016improved}.
Typically, models are trained on these datasets using 50,000 or more labels,
but feature matching GANs are able to obtain good performance
with very few labels.
They obtain state of the art performance within several categories for different
amounts of labels, ranging from 20 to 8,000.

The basic idea of how to do semi-supervised learning with feature matching GANs
is to turn a classification problem with $n$ classes into a classification problem
with $n+1$ classes, with the additional class corresponding to fake images.
All of the real classes can be summed together to obtain the probability of the
image being real, enabling the use of the classifier as a discriminator within
 the GAN game.
 The real-vs-fake discriminator can be trained even with unlabeled data, which
 is known to be real, and with samples from the generator, which are known to
 be fake.
 The classifier can also be trained to recognize individual real classes on the limited
 amount of real, labeled examples.
 This approach was simultaneously developed by \citet{salimans2016improved}
 and \citet{odena2016semi}. The earlier CatGAN used an $n$ class discriminator
 rather than an $n+1$ class discriminator.

 Future improvements to GANs can presumably be expected to yield further
 improvements to semi-supervised learning.

 \subsection{Using the code}

 GANs learn a representation $\vz$ of the image $\vx$.
 It is already known that this representation can capture useful high-level
 abstract semantic properties of $\vx$, but it can be somewhat difficult
 to make use of this information.

One obstacle to using $\vz$ is that it can be difficult to obtain
$\vz$ given an input $\vx$.
\citet{Goodfellow-et-al-NIPS2014-small} proposed but did not demonstrate
using a second network analogous to the generator to sample from $p(\vz \mid \vx)$,
much as the generator samples from $p(\vx)$.
So far the full version of this idea, using a fully general neural network as the
encoder and sampling from an arbitrarily powerful approximation of $p(\vz \mid \vx)$,
has not been successfully demonstrated,
but
\citet{donahue2016adversarial} demonstrated how to train a deterministic encoder,
and \citet{dumoulin2016adversarially} demonstrated how to train an encoder
network that samples from a Gaussian approximation of the posterior.
Futher research will presumably develop more powerful stochastic encoders.

Another way to make better use of the code is to train the code to be more useful.
InfoGANs \citep{chen2016infogan} regularize some entries in the code vector with
an extra objective function that encourages them to have high mutual information
with $\vx$. Individual entries in the resulting code then correspond to specific
semantic attributes of $\vx$, such as the direction of lighting on an image of
a face.

\subsection{Developing connections to reinforcement learning}
\label{sec:rl_connections}

Researchers have already identified connections between GANs and
actor-critic methods \citep{pfau2016connecting}, inverse reinforcement learning \citep{finn2016connection},
and have applied GANs to imitation learning \citep{ho2016generative}.
These connections to RL will presumably continue to bear fruit, both for GANs
and for RL.


\section{Plug and Play Generative Networks}

Shortly before this tutorial was presented at NIPS, a new generative model
was released. This model, plug and play generative networks \citep{nguyen2016plug}, has dramatically
improved the diversity of samples of images of ImageNet classes that can be
produced at high resolution.

PPGNs are new and not yet well understood.
The model is complicated, and most of the recommendations about how to design the model
are based on empirical observation rather than theoretical understanding.
This tutorial will thus not say too much about exactly how PPGNs work, since this will
presumably become more clear in the future.

As a brief summary, PPGNs are basically an approximate Langevin sampling approach to
generating images with a Markov chain.
The gradients for the Langevin sampler are estimated using a denoising autoencoder.
The denoising autoencoder is trained with several losses, including a GAN loss.

Some of the results are shown in \figref{fig:ppgn}.
As demonstrated in \figref{fig:recons}, the GAN loss is crucial for obtaining high quality images.

\begin{figure}
\centering
\includegraphics[width=\figwidth]{ppgn}
\caption{PPGNs are able to generate diverse, high resolution images from ImageNet
classes. Image reproduced from \citet{nguyen2016plug}.}
\label{fig:ppgn}
\end{figure}


\begin{figure}
\centering
\includegraphics[width=\figwidth]{recons}
\caption{The GAN loss is a crucial ingredient of PPGNs. Without it, the denoising autoencoder
used to drive PPGNs does not create compelling images.}
\label{fig:recons}
\end{figure}



\section{Exercises}

This tutorial includes three exercises to check your understanding.
The solutions are given in \secref{sec:solutions}.

\subsection{The optimal discriminator strategy}
\label{sec:opt_d}

As described in \eqref{eq:discriminator_cost}, the goal of the discriminator is to minimize
\begin{equation}
  J^{(D)}(\vtheta^{(D)}, \vtheta^{(G)}) = -\frac{1}{2} \E_{\vx \sim \pdata} \log D(\vx) - \frac{1}{2} \E_{\vz} \log \left(1 - D\left( G(z) \right) \right)
\end{equation}
with respect to $\vtheta^{(D)}$.
Imagine that the discriminator can be optimized in function space, so the value of
$D(\vx)$ is specified independently for every value of $\vx$.
What is the optimal strategy for $D$?
What assumptions need to be made to obtain this result?

\subsection{Gradient descent for games}
\label{sec:xy_exercise}

Consider a minimax game with two players that each control a single scalar value.
The minimizing player controls scalar $x$ and the maximizing player controls
scalar $y$.
The value function for this game is
\[ V(x, y) = x y .\]

\begin{itemize}
  \item Does this game have an equilibrium? If so, where is it?
  \item Consider the learning dynamics of simultaneous gradient descent.
        To simplify the problem, treat gradient descent as a continuous time
        process.
        With an infinitesimal learning rate, gradient descent is described by a system of partial differential equations:
        \begin{align}
          \frac{\partial x}{\partial t} &= - \frac{\partial}{\partial x} V\left( x(t), y(t) \right) \\
          \frac{\partial y}{\partial t} &= \frac{\partial}{\partial y} V\left( x(t), y(t) \right).
        \end{align}
        Solve for the trajectory followed by these dynamics.
\end{itemize}



\subsection{Maximum likelihood in the GAN framework}
\label{sec:mle_exercise}

In this exercise, we will derive a cost that yields (approximate) maximum likelihood learning within the GAN framework.
Our goal is to design $J^{(G)}$ so that, if we assume the discriminator is optimal, the expected gradient of $J^{(G)}$
will match the expected gradient of $\KL(\pdata \Vert \pmodel)$.

The solution will take the form of:
\[
  J^{(G)} = \E_{\vx \sim p_g} f(\vx) .
\]

The exercise consists of determining the form of $f$.


\section{Solutions to exercises}
\label{sec:solutions}

\subsection{The optimal discriminator strategy}
\label{sec:opt_d_soln}

Our goal is to minimize
\begin{equation}
  J^{(D)}(\vtheta^{(D)}, \vtheta^{(G)}) = -\frac{1}{2} \E_{\vx \sim \pdata} \log D(\vx) - \frac{1}{2} \E_{\vz} \log \left(1 - D\left( G(z) \right) \right)
\end{equation}
in function space, specifying $D(\vx)$ directly.

We begin by assuming that both $\pdata$ and $\pmodel$ are nonzero everywhere.
If we do not make this assumption, then some points are never visited during training,
and have undefined behavior.

To minimize $J^{(D)}$ with respect to $D$, we can write down the functional derivatives with
respect to a single entry $D(\vx)$, and set them equal to zero:
\[
\frac{\delta} {\delta D(\vx)} J^{(D)} = 0.
\]
By solving this equation, we obtain
\[
D^*(\vx) = \frac{ \pdata(\vx) } {\pdata(\vx) + \pmodel(\vx) }.
\]

Estimating this ratio is the key approximation mechanism used by GANs.

The process is illustrated in \figref{fig:ratio}.

\begin{figure}
\centering
\includegraphics[width=\figwidth]{ratio}
\caption{
An illustration of how the discriminator estimates a ratio of
densities.
In this example, we assume that both $z$ and $x$ are one dimensional
for simplicity.
The mapping from $z$ to $x$ (shown by the black arrows) is non-uniform so that $\pmodel(x)$
(shown by the green curve) is
greater in places where $z$ values are brought together more densely.
The discriminator (dashed blue line) estimates the ratio between the data density (black dots)
and the sum of the data and model densities.
Wherever the output of the discriminator is large, the model density is too low, and wherever
the output of the discriminator is small, the model density is too high.
The generator can learn to produce a better model density by following the discriminator uphill;
each $G(z)$ value should move slightly in the direction that increases $D(G(z))$.
Figure reproduced from \citet{Goodfellow-et-al-NIPS2014-small}.
}
\label{fig:ratio}
\end{figure}


\subsection{Gradient descent for games}
\label{sec:xy_soln}

The value function
\[ V(x, y) = x y \]
is the simplest possible example of a continuous function with a saddle point.
It is easiest to understand this game by visualizing the value function in three
dimensions, as shown in \figref{fig:xy}.

\begin{figure}
\centering
\includegraphics[width=\figwidth]{xy}
\caption{A three-dimensional visualization of the value function $V(x,y) = xy$.
  This is the canonical example of a function with a saddle point, at $x=y=0$.
}
\label{fig:xy}
\end{figure}

The three dimensional visualization shows us clearly that there is a saddle point
at $x=y=0$. This is an equilibrium of the game. We could also have found this point
by solving for where the derivatives are zero.

Not every saddle point is an equilibrium; we require that an infinitesimal perturbation
of one player's parameters cannot reduce that player's cost.
The saddle point for this game satisfies that requirement.
It is something of a pathological equilibrium because the value function is constant
as a function of each player's parameter when holding the other player's parameter
fixed.

To solve for the trajectory taken by gradient descent, we take the derivatives, and find that
\begin{align}
  \frac{\partial x}{\partial t} = - y(t) \\
  \frac{\partial y}{\partial t} = x(t). \label{eq:dy}
\end{align}
Diffentiating \eqref{eq:dy}, we obtain
\[
  \frac{\partial^2 y}{\partial t^2} = \frac{\partial x}{\partial t} = -y(t).
\]
Differential equations of this form have sinusoids as their set of basis functions
of solutions.
Solving for the coefficients that respect the boundary conditions, we obtain
\begin{align}
  x(t) = x(0) \cos(t) - y(0) \sin(t) \\
  y(t) = x(0) \sin(t) + y(0) \cos(t).
\end{align}

These dynamics form a circular orbit, as shown in \figref{fig:orbit}.
In other words, simultaneous gradient descent with an infinitesimal learning rate
will orbit the equilibrium forever, at the same radius that it was initialized.
With a larger learning rate, it is possible for simultaneous gradient descent to
spiral outward forever.
Simultaneous gradient descent will never approach the equilibrium.

\begin{figure}
  \center
  \includegraphics[width=\figwidth]{orbit}
  \caption{Simultaneous gradient descent with infinitesimal learning rate
    will orbit indefinitely at constant radius when applied to $V(x,y) = xy$,
    rather than approaching the equilibrium solution at $x=y=0$.
  }
  \label{fig:orbit}
\end{figure}

For some games, simultaneous gradient descent does converge, and for others,
such as the one in this exercise, it does not.
For GANs, there is no theoretical prediction as to whether simultaneous
gradient descent should converge or not.
Settling this theoretical question, and developing algorithms guaranteed to
converge, remain important open research problems.

\subsection{Maximum likelihood in the GAN framework}
\label{sec:mle_soln}

We wish to find a function $f$ such that the expected gradient of 
\begin{equation}
  J^{(G)} = \E_{\vx \sim p_g} f(\vx)
  \label{eq:cost_per_sample}
\end{equation}
is equal to the expected gradient of 
$\KL(\pdata \Vert p_g)$.

First we take the derivative of the KL divergence with respect to a parameter $\theta$:
\begin{equation}
  \frac{\partial}{\partial \theta} \KL(\pdata \Vert p_g) = - \E_{\vx \sim \pdata} \frac{\partial}{\partial \theta} \log p_g(\vx) .
\label{eq:mle_gradient}
\end{equation}

We now want to find the $f$ that will make the derivatives of \eqref{eq:cost_per_sample} match \eqref{eq:mle_gradient}.
We begin by taking the derivatives of \eqref{eq:cost_per_sample}:
\[
  \frac{\partial}{\partial \theta} J^{(G)} = \E_{\vx \sim p_g} f(x) \frac{\partial}{\partial \theta} \log p_g(\vx).
\]
To obtain this result, we made two assumptions:
\begin{enumerate}
  \item We assumed that $p_g(\vx) \geq 0$ everywhere so that we were able to use the identity $p_g(\vx) = \exp( \log p_g(\vx) ).$
  \item We assumed that we can use Leibniz's rule to exhange the order of differentiation and integration (specifically, that both the function and its derivative are continuous, and that the function vanishes for infinite values of $\vx$).
\end{enumerate}

We see that the derivatives of $J^{(G)}$ come very near to giving us what we want; the only problem is that
the expectation is computed by drawing samples from $p_g$ when we would like it to be computed by drawing
samples from $\pdata$.
We can fix this problem using an importance sampling trick; by setting $f(x) = \frac{\pdata(\vx)}{p_g(\vx)}$
we can reweight the contribution to the gradient from each generator sample to compensate for it having
been drawn from the generator rather than the data.

Note that when constructing $J^{(G)}$ we must {\em copy} $p_g$ into $f(x)$ so that $f(x)$ has a derivative of
zero with respect to the parameters of $p_g$.
Fortunately, this happens naturally if we obtain the value of $\frac{\pdata(\vx)}{p_g(\vx)}$.

From \secref{sec:opt_d_soln}, we already know that the discriminator estimates the desired ratio.
Using some algebra, we can obtain a numerically stable implementation of $f(\vx)$.
If the discriminator is defined to apply a logistic sigmoid function at the output layer,
with $D(\vx) = \sigma( a(\vx) )$, then $f(x) = - \exp(a(\vx))$.

This exercise is taken from a result shown by \citet{Goodfellow-ICLR2015}.
From this exercise, we see that the discriminator estimates a ratio of densities
that can be used to calculate a variety of divergences.

\section{Conclusion}

GANs are generative models that use supervised learning to approximate an intractable cost
function, much as Boltzmann machines use Markov chains to approximate their cost and VAEs
use the variational lower bound to approximate their cost.
GANs can use this supervised ratio estimation technique to approximate many cost functions, including the KL divergence used for maximum
likelihood estimation.

GANs are relatively new and still require some research to reach their new potential.
In particular, training GANs requires finding Nash equilibria in high-dimensional,
continuous, non-convex games.
Researchers should strive to develop better theoretical understanding and better training
algorithms for this scenario.
Success on this front would improve many other applications, besides GANs.

GANs are crucial to many different state of the art image generation and manipulation systems,
and have the potential to enable many other applications in the future.

\section*{Acknowledgments}
The author would like to thank the NIPS organizers for inviting him to
present this tutorial.
Many thanks also to those who commented on his Twitter and Facebook posts
asking which topics would be of interest to the tutorial audience.
Thanks also to D. Kingma for helpful discussions regarding the description of VAEs.
Thanks to Zhu Xiaohu, Alex Kurakin and Ilya Edrenkin for spotting typographical errors in the
manuscript.

\bibliography{biblio}
\bibliographystyle{natbib}

\end{document}
