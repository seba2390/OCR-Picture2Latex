% ****** Start of file apssamp.tex ******
%
%   This file is part of the APS files in the REVTeX 4.1 distribution.
%   Version 4.1r of REVTeX, August 2010
%
%   Copyright (c) 2009, 2010 The American Physical Society.
%
%   See the REVTeX 4 README file for restrictions and more information.
%
% TeX'ing this file requires that you have AMS-LaTeX 2.0 installed
% as well as the rest of the prerequisites for REVTeX 4.1
%
% See the REVTeX 4 README file
% It also requires running BibTeX. The commands are as follows:
%
%  1)  latex apssamp.tex
%  2)  bibtex apssamp
%  3)  latex apssamp.tex
%  4)  latex apssamp.tex
%
\documentclass[%
 reprint,
superscriptaddress,
%groupedaddress,
%unsortedaddress,
%runinaddress,
%frontmatterverbose, 
%preprint,
%showpacs,preprintnumbers,
%nofootinbib,
%nobibnotes,
%bibnotes,
 amsmath,amssymb,
 aps,
%pra,
%prb,
%rmp,
%prstab,
%prstper,
%floatfix,
]{revtex4-1}

\usepackage{graphicx}% Include figure files
\usepackage{dcolumn}% Align table columns on decimal point
\usepackage{bm}% bold math
\usepackage{amsmath}
%\usepackage{authblk}

%\usepackage{hyperref}% add hypertext capabilities
%\usepackage[mathlines]{lineno}% Enable numbering of text and display math
%\linenumbers\relax % Commence numbering lines

%\usepackage[showframe,%Uncomment any one of the following lines to test 
%%scale=0.7, marginratio={1:1, 2:3}, ignoreall,% default settings
%%text={7in,10in},centering,
%%margin=1.5in,
%%total={6.5in,8.75in}, top=1.2in, left=0.9in, includefoot,
%%height=10in,a5paper,hmargin={3cm,0.8in},
%]{geometry}

\begin{document}

%\preprint{APS/123-QED}

\title{Supplementary Material for: Suppression of electron thermal conduction by whistler turbulence in a sustained thermal gradient}

\author{G. T. Roberg-Clark}
\email{grc@umd.edu}
\affiliation{Department of Physics, University of Maryland College Park, College Park, MD 20740, USA}
\author{J. F. Drake}%
\email{drake@umd.edu}
\affiliation{Department of Physics, University of Maryland College Park, College Park, MD 20740, USA}
\affiliation{Institute for Physical Science and Technology, University of Maryland, College Park, MD 20742, USA}
\affiliation{Institute for Research in Electronics and Applied Physics, University of Maryland, College Park, MD 20742, USA}
\affiliation{Joint Space-Science Institute (JSI), College Park, MD 20742, USA}
\author{C. S. Reynolds}%
\email{chris@astro.umd.edu}
\affiliation{Department of Astronomy, University of Maryland College Park, College Park, MD 20740, USA}
\affiliation{Joint Space-Science Institute (JSI), College Park, MD 20742, USA}
\author{M. Swisdak}%
\email{swisdak@umd.edu}
\affiliation{Department of Physics, University of Maryland College Park, College Park, MD 20740, USA}
\affiliation{Institute for Research in Electronics and Applied Physics, University of Maryland, College Park, MD 20742, USA}
\affiliation{Joint Space-Science Institute (JSI), College Park, MD 20742, USA}

\date{\today}

\maketitle

\textit{Isotropization of the distribution function}. Here we demonstrate strong scattering of the electron distribution function for the large simulation quoted in the main paper with $L_{x}=4 L_{0}$, $\beta_{e0h}=64$. The distribution function is sampled from a thin band around the center of the domain at $L=L_{x}/2$, averaged over all $y$. In fig. \ref{fig:S1}a the initial (highly anisotropic) $f_{0}$ at $t=0$ is shown in $v_{x} - v_{y}$ space, with $v_{z}$ dependence integrated out. By the end of the simulation (fig. \ref{fig:S1}b) the electron distribution function is much more isotropic. This is accomplished via the resonant overlap mechanism mentioned in the main paper.

\renewcommand{\thefigure}{S\arabic{figure}}

\begin{figure}[b]
    \centering
    \includegraphics[scale=.55]{FigS1.pdf}
    \caption{Evidence for isotropization of the distribution function by whistler scattering by late time in the center of the simulation domain. (a) $f(t=0)$ as a function of $v_{x}$ and $v_{y}$. (b) $f(t=800\Omega_{e0}^{-1})$ }
    \label{fig:S1}
\end{figure}

\end{document}
%