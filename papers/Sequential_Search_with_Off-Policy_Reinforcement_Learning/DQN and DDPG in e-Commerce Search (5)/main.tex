%%
%% This is file `sample-manuscript.tex',
%% generated with the docstrip utility.
%%
%% The original source files were:
%%
%% samples.dtx  (with options: `manuscript')
%% 
%% IMPORTANT NOTICE:
%% 
%% For the copyright see the source file.
%% 
%% Any modified versions of this file must be renamed
%% with new filenames distinct from sample-manuscript.tex.
%% 
%% For distribution of the original source see the terms
%% for copying and modification in the file samples.dtx.
%% 
%% This generated file may be distributed as long as the
%% original source files, as listed above, are part of the
%% same distribution. (The sources need not necessarily be
%% in the same archive or directory.)
%%
%% The first command in your LaTeX source must be the \documentclass command.
%%%% Small single column format, used for CIE, CSUR, DTRAP, JACM, JDIQ, JEA, JERIC, JETC, PACMCGIT, TAAS, TACCESS, TACO, TALG, TALLIP (formerly TALIP), TCPS, TDSCI, TEAC, TECS, TELO, THRI, TIIS, TIOT, TISSEC, TIST, TKDD, TMIS, TOCE, TOCHI, TOCL, TOCS, TOCT, TODAES, TODS, TOIS, TOIT, TOMACS, TOMM (formerly TOMCCAP), TOMPECS, TOMS, TOPC, TOPLAS, TOPS, TOS, TOSEM, TOSN, TQC, TRETS, TSAS, TSC, TSLP, TWEB.
% \documentclass[acmsmall]{acmart}

%%%% Large single column format, used for IMWUT, JOCCH, PACMPL, POMACS, TAP, PACMHCI
% \documentclass[acmlarge,screen]{acmart}

%%%% Large double column format, used for TOG
% \documentclass[acmtog, authorversion]{acmart}

%%%% Generic manuscript mode, required for submission
%%%% and peer review


\documentclass[sigconf]{acmart}
\usepackage{subfigure}
% \usepackage{CJKutf8}
\usepackage{algorithm}
\usepackage{algorithmic}
\usepackage{amsmath}
\usepackage{microtype}
\usepackage{balance}
% \usepackage[linesnumbered,ruled,vlined]{algorithm2e}

% \usepackage{CJK} % this gives errors.
\newcommand{\multirows}[1]{\begin{tabular}{@{}c@{}}#1\end{tabular}}
\newcommand{\cI}{\mathcal{I}}
\newcommand{\cU}{\mathcal{U}}
\newcommand{\cC}{\mathcal{C}}
\newcommand{\cL}{\mathcal{L}}
\newcommand{\cK}{\mathcal{K}}

\newcommand{\R}{\mathbb{R}}
\newcommand{\Z}{\mathbb{Z}}
\DeclareMathOperator{\sign}{sign}

%% NOTE that a single column version may be required for 
%% submission and peer review. This can be done by changing
%% the \doucmentclass[...]{acmart} in this template to 
%% \documentclass[manuscript,screen]{acmart}
%% 
%% To ensure 100% compatibility, please check the white list of
%% approved LaTeX packages to be used with the Master Article Template at
%% https://www.acm.org/publications/taps/whitelist-of-latex-packages 
%% before creating your document. The white list page provides 
%% information on how to submit additional LaTeX packages for 
%% review and adoption.
%% Fonts used in the template cannot be substituted; margin 
%% adjustments are not allowed.
%%
%%
%% \BibTeX command to typeset BibTeX logo in the docs
\AtBeginDocument{%
  \providecommand\BibTeX{{%
    \normalfont B\kern-0.5em{\scshape i\kern-0.25em b}\kern-0.8em\TeX}}}

%% Rights management information.  This information is sent to you
%% when you complete the rights form.  These commands have SAMPLE
%% values in them; it is your responsibility as an author to replace
%% the commands and values with those provided to you when you
%% complete the rights form.
\setcopyright{acmcopyright}
\copyrightyear{2021}
\acmYear{2021}
\acmDOI{10.1145/1122445.1122456}

%% These commands are for a PROCEEDINGS abstract or paper.
\acmConference[CIKM '21] {Proceedings of the 30th ACM International Conference on Information and Knowledge Management}{November 1--5, 2021}{Virtual Event, Australia.}
\acmBooktitle{Proceedings of the 30th ACM Int'l Conf. on Information and Knowledge Management (CIKM '21), November 1--5, 2021, Virtual Event, Australia}
\acmPrice{15.00}
\acmISBN{978-1-4503-8446-9/21/11}
\acmDOI{10.1145/3459637.3481954}
% Authors, replace the red X's with your assigned DOI string during the rightsreview eform process.

\settopmatter{printacmref=true}


%%
%% Submission ID.
%% Use this when submitting an article to a sponsored event. You'll
%% receive a unique submission ID from the organizers
%% of the event, and this ID should be used as the parameter to this command.
%%\acmSubmissionID{123-A56-BU3}

%%
%% The majority of ACM publications use numbered citations and
%% references.  The command \citestyle{authoryear} switches to the
%% "author year" style.
%%
%% If you are preparing content for an event
%% sponsored by ACM SIGGRAPH, you must use the "author year" style of
%% citations and references.
%% Uncommenting
%% the next command will enable that style.
%%\citestyle{acmauthoryear}

%%
%% end of the preamble, start of the body of the document source.
\begin{document}
\fancyhead{}

%%
%% The "title" command has an optional parameter,
%% allowing the author to define a "short title" to be used in page headers.
% \title{Off-Policy Reinforcement Learning in e-Commerce Sequential Search with Online Incremental Update}
%Lingfei: after the whole paper, I feel that online incremental update is not the key innovation. So we may consider the following title to highlight two new pieces: 1) Sequential Search (a new problem definition) and 2) RL for SS. 
\title{Sequential Search with Off-Policy Reinforcement Learning}

%%
%% The "author" command and its associated commands are used to define
%% the authors and their affiliations.
%% Of note is the shared affiliation of the first two authors, and the
%% "authornote" and "authornotemark" commands
%% used to denote shared contribution to the research.


\author{Dadong Miao}
% \email{midaodadong@jd.com}
\author{Yanan Wang}
% \email{wangyanan11@jd.com}
\author{Guoyu Tang}
% \email{tangguoyu@jd.com}
\affiliation{%
  \institution{JD.com}
  \streetaddress{JD Building, No. 18 Kechuang 11 Street, BDA}
  \city{Beijing}
  \postcode{101111}
  \country{People's Republic of China}
}

\author{Lin Liu}
% \email{liulin1@jd.com}
\author{Sulong Xu}
% \email{xusulong@jd.com}
\author{Bo Long}
% \email{bo.long@jd.com}
\affiliation{%
  \institution{JD.com}
  \streetaddress{JD Building, No. 18 Kechuang 11 Street, BDA}
  \city{Beijing}
  \postcode{101111}
  \country{People's Republic of China}
}


\author{Yun Xiao}
% \email{xiaoyun1@jd.com}
\author{Lingfei Wu}
% \email{lwm@mail.wm.edu}
\author{Yunjiang Jiang}
% \email{yunjiangster@gmail.com}
\affiliation{%
  \institution{JD.com Silicon Valley R\&D Center}
  \streetaddress{675 E Middlefield Road}
  \city{Mountain View}
  \state{CA}
  \postcode{94043}
  \country{USA}
}
\renewcommand{\shortauthors}{Dadong Miao, et al.}
% \author{Anonymous Authors}
% \renewcommand{\shortauthors}{Anonymous Authors, et al.}

%%
%% The abstract is a short summary of the work to be presented in the
%% article.
% \begin{abstract}
% We explore the use of off-policy reinforcement learning in multi-session e-commerce search. We design the training data in a maximally contiguous fashion to put a user's entire behavior sequence in a single minibatch.
% % give the model simultaneous access to the users' entire behavior sequence.
% In addition to attention network based on long range behavior sequence, a versatile RNN backbone is designed to capture near term behavior sequence, on which we also build a DDPG network.
% % to ensure long term reward signals are fully captured in the back-propagation. 
% As a novel optimization step, we fit multiple short user sequences in a single RNN pass within a training batch, by solving a greedy knapsack problem on the fly. 

% Empirically we demonstrate the effectiveness of the RNN backbone against its attention-based DIN counterpart, in both offline and online metrics. The addition of reinforcement learning further elevates these metrics to new heights. Besides the core metrics for e-commerce business, online experiments demonstrate significant gains in a variety of secondary metrics, most notably result diversity and level of personalization. The use of hidden state daily update further boosts online core metrics by a wide margin.
% \end{abstract}

%Lingfei: after the whole paper, I feel that online incremental update is not the key innovation. So we may consider the following title to highlight two new pieces: 1) Sequential Search (a new problem definition) and 2) RL for SS. 
\begin{abstract}
Recent years have seen a significant amount of interests in Sequential Recommendation (SR), which aims to understand and model the sequential user behaviors and the interactions between users and items over time. Surprisingly, despite the huge success Sequential Recommendation has achieved, there is little study on Sequential Search (SS), a twin learning task that takes into account a user's current and past search queries, in addition to behavior on historical query sessions.
% a user's long-range behavior sequence as well as their near-term behavior sequence over time. 
The SS learning task is even more important than the counterpart SR task for most of E-commence companies due to its much larger online serving demands as well as traffic volume.

% To address this challenge, in this paper, we first present a new learning task, Sequential Search in order to well capture a user's long-range and short-term behaviors for more accurate and personalized search.
To this end, we propose a highly scalable hybrid learning model that consists of an RNN learning framework leveraging all features in short-term user-item interactions, and an attention model utilizing selected item-only features from long-term interactions.
As a novel optimization step, we fit multiple short user sequences in a single RNN pass within a training batch, by solving a greedy knapsack problem on the fly. 
Moreover, we explore the use of off-policy reinforcement learning in multi-session personalized search ranking. Specifically, we design a pairwise Deep Deterministic Policy Gradient model that efficiently captures users' long term reward in terms of pairwise classification error.
Extensive ablation experiments demonstrate significant improvement each component brings to its state-of-the-art baseline, on a variety of offline and online metrics.     
\end{abstract}


%%
%% The code below is generated by the tool at http://dl.acm.org/ccs.cfm.
%% Please copy and paste the code instead of the example below.
%%
\begin{CCSXML}
<ccs2012>
<concept>
<concept_id>10010147.10010257.10010293.10010294</concept_id>
<concept_desc>Computing methodologies~Neural networks</concept_desc>
<concept_significance>500</concept_significance>
</concept>
<concept>
<concept_id>10002951.10003317.10003338.10003343</concept_id>
<concept_desc>Information systems~Learning to rank</concept_desc>
<concept_significance>500</concept_significance>
</concept>
</ccs2012>
\end{CCSXML}

\ccsdesc[500]{Computing methodologies~Neural networks}

\ccsdesc[500]{Information systems~Learning to rank}


%%
%% Keywords. The author(s) should pick words that accurately describe
%% the work being presented. Separate the keywords with commas.
\keywords{sequential-search, RNN, reinforcement-learning, actor-critic}


%% A "teaser" image appears between the author and affiliation
%% information and the body of the document, and typically spans the
%% page.
% \begin{teaserfigure}
%   \includegraphics[width=\textwidth]{sampleteaser}
%   \caption{Seattle Mariners at Spring Training, 2010.}
%   \Description{Enjoying the baseball game from the third-base
%   seats. Ichiro Suzuki preparing to bat.}
%   \label{fig:teaser}
% \end{teaserfigure}

%%
%% This command processes the author and affiliation and title
%% information and builds the first part of the formatted document.
\maketitle


Neural networks are powerful models that excel at a wide range of tasks.
However, they are notoriously difficult to interpret and extracting explanations 
    for their predictions is an open research problem. Linear models, in contrast, are generally considered interpretable, because
    the \emph{contribution} 
    (`the weighted input') of every dimension to the output is explicitly given.
Interestingly, many modern neural networks implicitly model the output as a linear transformation of the input;
    a ReLU-based~\cite{nair2010rectified} neural network, e.g.,
    is piece-wise linear and the output thus a linear transformation of the input, cf.~\cite{montufar2014number}.
    However, due to the highly non-linear manner in which these linear transformations are `chosen', the corresponding contributions per input dimension do not seem to represent the learnt model parameters well, cf.~\cite{adebayo2018sanity}, and a lot of research is being conducted to find better explanations for the decisions of such neural networks, cf.~\cite{simonyan2013deep,springenberg2014striving,zhou2016CAM,selvaraju2017grad,shrikumar2017deeplift,sundararajan2017axiomatic,srinivas2019full,bach2015pixel}.
    
In this work, we introduce a novel network architecture, the \textbf{Convolutional Dynamic Alignment Networks (CoDA-Nets)}, {for which the model-inherent contribution maps are faithful projections of the internal computations and thus good `explanations' of the model prediction.} 
There are two main components to the interpretability of the CoDA-Nets. 
    First, the CoDA-Nets are \textbf{dynamic linear}, i.e., they compute their outputs through a series of input-dependent linear transforms, which are based on our novel \mbox{\textbf{Dynamic Alignment Units (DAUs)}}. 
        As in linear models, the output can thus be decomposed into individual input contributions, see Fig.~\ref{fig:teaser}.
    Second, the DAUs are structurally biased to compute weight vectors that \textbf{align with \mbox{relevant} patterns} in their inputs. 
In combination, the CoDA-Nets thus inherently  
produce contribution maps that are `optimised for interpretability': 
since each linear transformation matrix and thus their combination is optimised to align with discriminative features, the contribution maps reflect the most discriminative features \emph{as used by the model}.

With this work, we present a new direction for building inherently more interpretable neural network architectures with high modelling capacity.
In detail, we would like to highlight the following contributions:
\begin{enumerate}[wide, label={\textbf{(\arabic*)}}, itemsep=-.5em, topsep=0em, labelwidth=0em, labelindent=0pt]
    \item We introduce the Dynamic Alignment Units (DAUs), which 
    improve the interpretability of neural networks and have two key properties:
    they are 
    \emph{dynamic linear} 
    and align their weights with discriminative input patterns.
    \item Further, we show that networks of DAUs \emph{inherit} these two properties. In particular, we introduce Convolutional Dynamic Alignment Networks (CoDA-Nets), which are built out of multiple layers of DAUs. As a result, the \emph{model-inherent contribution maps} of CoDA-Nets highlight discriminative patterns in the input.
    \item We further show that the alignment of the DAUs can be promoted 
    by applying a `temperature scaling' to the final output of the CoDA-Nets. 
    \item We show that the resulting contribution maps 
    perform well under commonly employed \emph{quantitative} criteria for attribution methods. Moreover, under \emph{qualitative} inspection, we note that they exhibit a high degree of detail.
    \item Beyond interpretability, 
    CoDA-Nets are performant classifiers and yield competitive classification accuracies on the CIFAR-10 and TinyImagenet datasets.
\end{enumerate}
\section{Related Work}
\label{sec:related_work}


\subsection{User Behavior Modeling}
\label{subsec:rw:user}

User behavior modeling is an important topic in industrial ads, search, and recommendation system. A notable pioneering work that leverages the power of neural network is provided by Youtube Recommendation \cite{covington2016deep}. User historical interactions with the system are embedded first, and sum-pooled into fixed width input for downstream multi-layer perception. 

Follow-up work starts exploring the sequential nature of these interactions. Among these, earlier work exploits sequence models such as RNN \cite{hidasi2015session}, while later work starting with \cite{li2017neural} mostly adopts attention between the target example and user historical behavior sequence, notably DIN \cite{zhou2018deep} and KFAtt\cite{liu2020kalman}. 

More recently, self-attention \cite{kang2018self} and graph neural net \cite{wu2019session,pang2021heterogeneous} have been successfully applied in the sequential recommendation domain. 




% User Behavior Modeling Methods contains kinds of Methods:Pooling-based, Attention-based和Sequence-based。Pooling-based\cite{covington2016deep}和Attention-based\cite{zhou2018deep}, \cite{vaswani2017attention} Methods会将用户行为看成无序的集合,其中Pooling-based将所有用户行为看做是等价的。Sequence-based\cite{hidasi2015session}则可以根据用户行为序列提取信息,具有时间属性。

% ps:
% \cite{covington2016deep}: Deep Neural Networks for YouTube Recommendations.
% \cite{zhou2018deep}: Deep interest network for click-through rate prediction.
% \cite{vaswani2017attention}: Attention is all you need.
% \cite{hidasi2015session}: Session-based recommendations with recurrent neural networks.
% \label{subsec:rw:behavior}

\subsection{RNN in search and recommendation}
\label{subsec:rw:rnn}

While attention excels in training efficiency, RNN still plays a useful role in settings like incremental model training and updates. 
Compared to DNN, RNN is capable of taking the entire history of a user into account, effectively augmenting the input feature space. Furthermore, it harnesses the sequential nature of the input data efficiently, by constructing a training example at every event in the sequence, rather than only at the last event \cite{zhang2014sequential}. Since the introduction of Attention in \cite{vaswani2017attention}, however, RNN starts to lose its dominance in the sequential modeling field, mainly because of its high serving latency. 

We argue however RNN saves computation in online serving, since it propagates the user hidden state in a forward only manner, which is friendly to incremental update. In the case when user history can be as long as thousands of sessions, real time attention computation can be highly impractical, unless mitigated by some approximation strategies \cite{drachsler2008personal}. The latter introduces additional complexity and can easily lose accuracy. 

Most open-source implementations of reinforcement learning framework for search and recommendation system implicitly assume an underlying RNN backbone \cite{chen2019top}. The implementation however typically simplifies the design by only feeding a limited number of ID sequences into the RNN network \cite{zhao2019deep}. 

\cite{zhang2019deep} contains a good overview of existing RNN systems in Search / Recommendations. In particular, they are further divided into those with user identifier and those without. In the latter case, the largest unit of training example is a single session from which the user makes one or more related requests. While in the former category, a single user could come and go multiple times over a long period of time, thus providing much richer contexts to the ranker. It is the latter scenario that we focus on in this paper. To the best of our knowledge, such settings are virtually unexplored in the search ranking setting.




% RNN具备记忆能力,当涉及到连续的、与上下文相关的任务时,它比其他神经网络具有更大的优势。论文1\cite{zhang2019deep}Deep learning based recommender system: A survey and new perspectives里面将RNN在推荐系统中的应用分为三种:Session-based Recommendation without User Identifier, Sequential Recommendation with User Identifier和Feature Representation Learning with RNNs。其中不要求用户注册和登录的应用或网站没有用户标识,这些系统通常使用第一种方式通过the session或cookie获取用户的短期偏好。可以获取到用户标识的系统则使用第二种方式建模序列推荐任务,同时RNN也可作为一种特征表示的学习方式。RNN在搜索中的应用与在推荐场景中类似\cite{zhang2014sequential}。

\subsection{Deep Reinforcement Learning}
\label{subsec:rw:rl}

While the original reinforcement learning idea was proposed more than 3 decades ago, there has been a strong resurgence of interest in the past few years, thanks in part to its successful application in playing Atari games \cite{mnih2013playing}, DeepMind's AlphaGo \cite{silver2017mastering} and in text generation domains \cite{chen2019reinforcement,gong2019reinforcement}. Both lines of work achieve either super-human level or current state-of-the-art performance on a wide range of indisputable metrics.  

Several important technical milestones include Double DQN \cite{van2016deep} to mitigate over-estimation of Q value, and \cite{schaul2015prioritized}, which introduces experience replay. However, most of the work focuses on settings like gaming and robotics. We did not adopt experience replay in our work because of its large memory requirement, given the billion example scale at which we operate. 

The application in personalized search and recommendation has been more recent. Majority of the work in this area focuses on sequential recommendation such as \cite{zhao2018deep} as well as ads placement within search and recommendation results \cite{zhao2021dear}.

An interesting large scale off-policy recommendation work is presented in \cite{chen2019top} for youtube recommendation. They make heuristic correction of the policy gradient calculation to bridge the gap between on-policy and off-policy trajectory distributions. We tried it in our problem with moderate offline success, though online performance was weaker, likely because our changing user queries make the gradient adjustment less accurate.

Several notable works in search ranking include \cite{hu2018reinforcement} which takes an on-policy approach and \cite{xu2020reinforcement} which uses pairwise training examples similar to ours. However both works consider only a single query session, which is similar to the sequential recommendation setting, since the query being fixed can be treated as part of the user profile. In contrast, our work considers the user interactions on a search platform over an extended period of time, which typically consist of hundreds of different query sessions. 




% Several notable works include xyz (dawei yin, . Despite the dissimilarity between search ranking and recommendation tasks, namely the existence of a query, the problem setup is often quite similar. In particular, for learning to rank problems, typically only a single query session is considered. Many of the ranking related RL papers also focus on relevance learning, with a notable exception of xyz (Hu from alibaba). The latter however also considers a single query as the entire user history. Thus to the best of our knowledge, multi-query cross session reinforcement learniing has not been fully explored under RL.

% We also mention the work in the personalized Ads targeting domain, from ByteDance. Since it deals with the mixture of Ads and recommendation results, it does not directly address the homogeneous ranking problem that we face.


% \begin{itemize}
% \item Value-Based RL —DQN及其改进算法
% \begin{itemize}
% \item 论文1\cite{mnih2013playing}: Playing Atari with Deep Reinforcement Learning【深度强化学习开山之作】第一次提出了DQN算法,是深度强化学习真正意义的开山之作,算法用卷积神经网络构造Q网络,网络输入经过处理后的最近4帧游戏画面,输出在这种状态下执行各种动作的Q函数值。在绝大部分游戏上,DQN超过了之前最好的算法,在部分游戏上,甚至超过了人类玩家的水平;
% \item 论文2\cite{mnih2015human}: Human-level control through deep reinforcement learning 对论文1进行改进,构建目标Q网络,目标Q网络和Q网络之间周期性同步参数,提升了算法的收敛性;
% \item 论文3\cite{van2016deep}:  Deep reinforcement learning with double q-learning【Duoble DQN】 使用当前值网络的参数θ选择最优动作,用目标值网络的参数θ-评估该最优动作,将动作选择和策略评估分离,降低了过高估计Q值的风险;
% \item 论文4\cite{schaul2015prioritized}: Prioritized experience replay【Prioritized replay 样本采样方式优化】基于优先级采样的DQN,是对经验回放机制的改进,为经验池中的每个样本计算优先级,增大有价值的训练样本在采样时的概率。加快收敛速度和提升效果;
% \item 论文5\cite{wang2016dueling}: Dueling network architectures for deep reinforcement learning【Dueling networks】将CNN卷积层之后的全连接层替换为两个分支,其中一个分支拟合状态价值(state values)函数V(s),另外一个分支拟合动作优势(action advantages)函数A(s,a)。最后将两个分支的输出值相加,形成Q函数值。这种改进能够更准确的估计Q值。
% \item 论文6\cite{hausknecht2015deep}:Deep recurrent q-learning for partially observable MDPs. 在CNN的卷积层之后加入LSTM单元,记住之前的信息。
% \item 论文7\cite{hessel2018rainbow}: Rainbow: Combining Improvements in Deep Reinforcement Learning 整合了DQN的诸多优化。包括Double Q-learning,Prioritized replay,Dueling networks,Multi-step learning,Distributional RL 和 Noisy Nets.
% \item DQN适用范围:DQN是求每个action的\(max_aQ(s,a)\),适用于低维,离线的动作空间,在连续空间不适用。
% \end{itemize}
% \item Policy-Based RL算法
% \begin{itemize}
% \item  policy based RL 直接对策略建模,通过reward来直接对策略进行更新,使得累计回报最大。适用于连续的动作空间,但是无法衡量策略究竟是不是最优,策略评估高方差。
% \item  最开始的reinforcement算法\cite{williams1992simple} Simple Statistical Gradient-Following Algorithms for Connectionist Reinforcement Learning[1992, Williams] 
% \item  策略梯度算法\cite{sutton1999policy} Policy Gradient Methods for Reinforcement Learning with Function Approximation[2000, Sutton]
% \end{itemize}

% \item Actor-Critic算法
% \begin{itemize}
% \item 结合Policy-Based RL和Value-Based RL方法,基本上解决了高维状态与动作空间的问题,并使性能有明显的提升。但原始的Actor-Critic方法对于复杂问题可能会不稳定。
% \item DPG算法\cite{silver2014deterministic}:Deterministic Policy Gradient Algorithms【DPG】
% \item DDPG算法\cite{lillicrap2015continuous}:Continuous Control With Deep Reinforcement Learning【DDPG】
% \item TD3算法\cite{fujimoto2018addressing}:Addressing Function Approximation Error in Actor-Critic Methods【Twin Delayed Deep Deterministic policy gradient TD3】
% \end{itemize}

% \end{itemize}

% \label{subsec:rw:rl}

% \subsection{Reinforcement Learning in search and recommendation、}
% \begin{itemize}
% \item 传统的搜索/推荐算法
% \begin{itemize}
% \item 认为搜索/推荐是一个静态的过程;
% \item 建模即时reward,仅考虑当前的商品是否被点击/购买,忽略长期价值;
% \end{itemize}
% \item 强化学习在推荐中的应用
% \begin{itemize}
% \item  MDP推荐系统\cite{shani2005mdp}:An MDP-Based Recommender System
% \item  RL推荐系统(最大化长期价值)\cite{theocharous2015ad}:Personalized ad recommendation systems for life-time value optimization with guarantees
% \begin{itemize}
% \item 把个性化⼴告推荐系统定义为强化学习问题,最⼤化⽣命周期值(life-time value)
% \end{itemize}
% \item  DRN新闻推荐\cite{zheng2018drn}:DRN:A deep reinforcement learning framework for news recommendation
% \begin{itemize}
% \item 基于【DQN】的推荐算法
% \item 主要贡献:1.提出基于deep Q-learning的推荐框架,该框架可以明确建模长期reward(MAB-based方法不更清晰地给出future reward,MDP-based方法不适用于大规模数据);2.引入用户活跃度(用户返回APP的情况),作为点击/不点击标签的补充,从而获取更多的的用户反馈信息;3.加入了探索策略(采用Dueling Bandit Gradient Descent方法挑选当前推荐环境下候选items),为用户寻找新的有吸引力的新闻;
% \end{itemize}
% \item Listwise推荐\cite{zhao2017deep}:Deep reinforcement learning for list-wise recommendations
% \begin{itemize}
% \item 基于【DDPG】的推荐算法
% \item 主要缺点:使用全联接网络表示用户状态,不能很好的建模用户和商品之间的关系
% \end{itemize}
% \item Pagewise推荐\cite{zhao2018deep}:Deep Reinforcement Learning for Page-wise Recommendations
% \begin{itemize}
% \item 基于【DDPG】的推荐算法
% \item 主要贡献:生成每个推荐物品的同时,也决定每个推荐维度在二维屏幕上的位置
% \end{itemize}
% \item 考虑正负反馈的推荐\cite{zhao2018recommendations}:Recommendations with Negative Feedback via
% Pairwise Deep Reinforcement Learning
% \begin{itemize}
% \item 基于【DQN】的推荐算法
% \item 主要贡献:考虑负反馈以及商品的偏序关系,并将这种偏序关系建模到DQN的loss函数中。用户跳过或者是没有任何行为的商品,不仅能够影响用户的行为,还可以让我们更好的了解用户的偏好。
% \end{itemize}
% \item User-Item Interactions Modeling\cite{liu2018deep} : Deep Reinforcement Learning based Recommendation with Explicit User-Item Interactions Modeling
% \begin{itemize}
% \item 基于【DDPG】的推荐算法
% \item 主要贡献:状态表征模块设计,文中强调了状态表征的重要性,并设计了三种状态表征模块;DRR-p:商品之间组pair,pair内item embedding相乘,和原始商品信息concat;DRR-u:用户embedding和item embedding相乘,同时concat商品pair相乘结果;DRR-ave:DRR-u基础上item处考虑position weight,并经过average pooling;
% \item 不足之处:虽然设计了状态表征模块,但是仅使用了用户最近的N个正反馈商品
% \end{itemize}
% \end{itemize}

% \item 强化学习在搜索中的应用
% \begin{itemize}
% \item 阿里RL搜索排序\cite{hu2018reinforcement}:Reinforcement learning to rank in e-commerce search engine: Formalization, analysis, and application
% \end{itemize}

% \end{itemize}

\label{subsec:rw:rl_search}

% mainly low rank approximation and pq based.

\section{Dynamic Alignment Networks}
\label{subsec:alignment}


In this section, we present our novel type of network architecture: the Convolutional Dynamic Alignment Networks (CoDA-Nets). For this, we first introduce Dynamic Alignment Units (DAUs) as the basic building blocks of CoDA-Nets and discuss two of their key properties in sec.~\ref{subsec:align_units}. Concretely, we show that these units linearly transform their inputs with dynamic (input-dependent) weight vectors and, additionally, that they are biased to align these weights with the input during optimisation. 
We then discuss how DAUs can be used for classification (sec.~\ref{subsec:classification}) and how we build performant networks out of multiple layers of convolutional DAUs (sec.~\ref{subsec:coda}). Importantly, the resulting \emph{linear decompositions} of the network outputs are optimised to align with discriminative patterns in the input, making them highly suitable for interpreting the network predictions. 

In particular, we structure this section around the following \textbf{three important properties} (\colornum{P1-P3}) of the DAUs:
\\[.25em]
\colornum{P1: Dynamic linearity.} The DAU output $o$ is computed as a dynamic (input-dependent) linear transformation of the input $\vec x$, such that \mbox{$o=\vec w(\vec x)^T\vec x=\sum_jw_j(\vec x)x_j$}. 
Hence, 
$o$ can be decomposed into contributions  
from individual input dimensions, which are given by $w_j(\vec x)x_j$ for dimension $j$.
\\[.5em]
\colornum{P2: Alignment maximisation.} Maximising the average output of a single DAU over a set of inputs $\vec x_i$ 
maximises the alignment between inputs $\vec x_i$ and the weight vectors $\vec w(\vec x_i)$. As the modelling capacity of $\vec w(\vec x)$ is restricted, $\vec w(\vec x)$ will encode the most frequent patterns in the set of inputs $\vec x_i$.
\\[.5em]
\colornum{P3: Inheritance.} When combining multiple DAU layers to form a \mbox{Dynamic} Alignment Network (DA-Net), the properties \colornum{P1} and \colornum{P2} are \emph{inherited}. In particular, DA-Nets are dynamic linear (\colornum{P1}) and maximising the last layer's output induces an output maximisation in the constituent DAUs (\colornum{P2}).
\\[.5em]
These properties increase the interpretability
of a DA-Net, such as a CoDA-Net (sec.~\ref{subsec:coda}) for the following reasons.
First, the output of a DA-Net can be decomposed into contributions from the individual input dimensions, similar to linear models (cf.~Fig.~\ref{fig:teaser}, \colornum{P1} and \colornum{P3}).
Second, we note that optimising a neural network for classification applies a maximisation to the outputs of the last layer for every sample. 
This maximisation aligns the dynamic weight vectors $\vec w(\vec x)$ of the constituent DAUs of the DA-Net with their respective inputs (cf.~Fig.~\ref{fig:alignment}, \colornum{P2} and \colornum{P3}).

 Importantly, the weight vectors will align with the \emph{discriminative} patterns in their inputs when optimised for classification as we show in sec.~\ref{subsec:classification}.
As a result, the model-inherent contribution maps of CoDA-Nets are optimised to align well with \emph{discriminative input patterns} in the input image 
and the interpretability of our models thus forms part of the global optimisation procedure.
\begin{figure}[t!]
    \centering
    \hspace{-.25em}
    \begin{subfigure}[b]{0.48\textwidth}
    \includegraphics[width=\textwidth]{resources/DAUs-v3.pdf}
     \end{subfigure}
    \caption{\small
        For different inputs $\vec x$, we visualise the linear weights and contributions (for the single layer, see eq.~\eqref{eq:contrib_1}, for the CoDA-Net eq.~\eqref{eq:contrib}) for the ground truth label $l$ and the strongest non-label output $z$. 
    As can be seen, the weights align well with the input images.
    The first three rows are based on a single DAU layer, the last three on a 5 layer CoDA-Net. The first two samples (rows) per model are correctly classified and the last one is misclassified. }
    \label{fig:alignment}
\end{figure}
\subsection{Dynamic Alignment Units}
\label{subsec:align_units}
We define the Dynamic Alignment Units (DAUs) by
\begin{align}
    \label{eq:au}
    \text{DAU}(\vec x) = g(\mat a \mat b\vec x +\vec b)^T \vec x = \vec w(\vec x)^T\, \vec x\quad \textbf{.}
\end{align}
% 
Here, $\vec x\in\mathbb R^{d}$ is an input vector, $\mat a\in\mathbb R^{d\times r}$ and $\mat b \in \mathbb R^{r\times d}$ are trainable transformation matrices, $\vec b\in\mathbb R^{d}$ a trainable bias vector, and \mbox{$g(\vec u)=\alpha(||\vec u||)\vec u$} is a non-linear function that scales the norm of its input. {In contrast to using a single matrix $\mat m \in\mathbb R^{d\times d}$, using $\mat{ab}$ allows us to control the maximum rank $r$ of the transformation and to reduce the number of parameters}; we will hence refer to $r$ as the rank of a DAU. 
%
As can be seen by the right-hand side of eq.~\eqref{eq:au}, the DAU linearly transforms the input $\vec x$ (\colornum{P1}). At the same time, given the quadratic form ($\vec x^T\mat B^T\mat A^T\vec x$) and the  rescaling function $\alpha(||\vec u||)$, the output of the DAU is a non-linear function of its input. In this work, we focus our analysis on 
two choices for $g(\vec u)$ in particular\footnote{
In preliminary experiments we observed comparable behaviour over a range of different normalisation functions such as, e.g., L1 normalisation.}, namely rescaling to unit norm ($\text{L2}$) and the squashing function ($\text{SQ}$, see \cite{sabour2017dynamic}):
\begin{align}
    \label{eq:nonlin}
    \text{L2}(\vec u) = \frac{\vec u}{||\vec u||_2} \;\;\text{and}\;\;
    \text{SQ}(\vec u) = \text{L2}(\vec u) \times \frac{||\vec u||^2_2}{1+||\vec u||_2^2}
\end{align}
Under these rescaling functions, the norm of the weight vector is upper-bounded: $||\vec w(\vec x)|| \leq 1$. Therefore, the output of the DAUs is upper-bounded by the norm of the input:
\begin{align}
    \text{DAU}(\vec x) = 
    ||\vec w(\vec x)|| \hspace{.2em} ||\vec x|| \cos(\angle(\vec x, \vec w(\vec x)))\leq ||\vec x||
    \label{eq:bound}
\end{align}
As a corollary, for a given input $\vec x_i$, the DAUs can only achieve this upper bound if $\vec x_i$ is an eigenvector (EV) of the linear transform $\mat{AB}\vec x+ \vec b$. Otherwise, the cosine in eq.~\eqref{eq:bound} will not be maximal\footnote{
Note that $\vec w(\vec x)$ is proportional to $\mat{ab}\vec x + \vec b$. The cosine in eq.~\eqref{eq:bound}, in turn, is maximal if and only if $\vec w(\vec x_i)$ is proportional to $\vec x_i$ and thus, by transitivity, if $\vec x_i$ is proportional to $\mat{ab}\vec x_i + \vec b$. This means that $\vec x_i$ has to be an EV of $\mat{ab}\vec x +\vec b$ to achieve maximal output.}. 
As can be seen in eq.~\eqref{eq:bound}, maximising the average output of a DAU over a set of inputs $\{\vec x_i|\,i=1, ..., n\}$
maximises the alignment between $\vec w(\vec x)$ and $\vec x$ (\colornum{P2}).
In particular, it optimises the parameters of the DAU such that the \emph{most frequent input patterns} are encoded as EVs in the linear transform $\mat{ab}\vec x + \vec b$, similar to an $r$-dimensional PCA decomposition ($r$ the rank of $\mat{ab}$). In fact, as discussed in the supplement, the optimum of the DAU maximisation solves a low-rank matrix approximation~\cite{eckart1936approximation} problem similar to singular value decomposition.
\begin{figure}[t!]
    \centering
    \includegraphics[height=6.5em]{resources/evs.pdf}
    \caption{\small Eigenvectors (EVs) of \tmat{AB} after maximising the output of a rank-3 DAU over a set of noisy samples of 3 MNIST digits. Effectively, the DAUs encode the most frequent components in their EVs, similar to a principal component analysis (PCA).
    }
    \label{fig:EVs}
\end{figure}
%
As an illustration of this property, in Fig.~\ref{fig:EVs} we show the 3 EVs\footnote{Given $r=3$, the EVs maximally span a 3-dimensional subspace.} of matrix $\mat{ab}$ (with rank $r=3$, bias $\vec b=\vec 0$) after optimising a DAU over a set of $n$ noisy samples of 3 specific MNIST~\cite{lecun2010mnist} images; for this, we used $n=3072$ and zero-mean Gaussian noise. As expected, the EVs of \tmat{ab} encode the original, noise-free images, since this on average maximises the alignment (eq.~\eqref{eq:bound}) between the weight vectors $\vec w(\vec x_i)$ and the input samples $\vec x_i$ over the dataset.
%
%

\subsection{DAUs for classification}
\label{subsec:classification}
{DAUs can be used directly for classification by applying $k$ DAUs in parallel to obtain an output \mbox{$\hat{\vec y}(\vec x)=\left[\text{DAU}_1(\vec x), ..., \text{DAU}_k(\vec x)\right]$}. 
Note that this is a linear transformation $\hat{\vec y}(\vec x)$$=$$\mat W(\vec x) \vec x$, with each row in $\mat w$$\in$$\mathbb R^{k \times d}$ corresponding to the weight vector $\vec w_j^T$ of a specific DAU $j$.
In particular, consider 
a dataset $\mathcal D = \{(\vec x_i, \vec y_i)|\, \vec x_i\in\mathbb R^d, \vec y_i\in\mathbb R^k\}$ of $k$ classes with `one-hot' encoded labels $\vec y_i$ for the inputs $\vec x_i$.
To optimise the DAUs as classifiers on $\mathcal D$,} we can apply a sigmoid non-linearity to each DAU output and optimise the loss function $\mathcal L = \sum_i\text{BCE}(\sigma(\hat{\vec y}_i), \vec y_i)$, where \text{BCE} denotes the binary cross-entropy and $\sigma$ applies the sigmoid function to each entry in $\hat{\vec y}_i$. Note that for a given sample, \text{BCE} either maximises (DAU for correct class) or minimises (DAU for incorrect classes) the output of each DAU. Hence, this classification loss will still maximise the (signed) cosine between the weight vectors $\vec w(\vec x_i)$ and $\vec x_i$. 

To illustrate this property, in Fig.~\ref{fig:alignment} (top) we show the weights $\vec w(\vec x_i)$ for several samples of the digit `3' after optimising the DAUs for classification on a noisy MNIST dataset; the first two are correctly classified, the last one is misclassified as a `5'. As can be seen, the weights align with the respective input (the weights for different samples are different). However,  different parts of the input are either positively or negatively correlated with a class, which is reflected in the weights: for example, the extended stroke on top of the `3' in the misclassified sample is assigned \emph{negative weight} and, since the background noise is \emph{uncorrelated} with the class labels, it is not represented in the weights. 

In a classification setting, the DAUs {thus} encode \emph{the most frequent discriminative patterns} in the linear transform $\mat{ab}\vec x + \vec b$ such that the dynamic weights $\vec w(\vec x)$ align well with these patterns.
Additionally, since the output for class $j$ is a linear transformation of the input (\colornum{P1}), we can compute the contribution vector $\vec s_j$ containing the per-pixel contributions to this output by the element-wise product ($\odot$)
\begin{align}
\label{eq:contrib_1}
    \vec s_j(\vec x_i) = \vec w_j(\vec x_i)\odot\vec x_i\quad ,
\end{align}
 see Figs.~\ref{fig:teaser} and
\ref{fig:alignment}. 
Such linear decompositions constitute the model-inherent `explanations' which we evaluate in sec.~\ref{sec:results}.
\subsection{Convolutional Dynamic Alignment Networks}
\label{subsec:coda}
The modelling capacity of a single layer of DAUs is limited, similar to a single linear classifier. However, DAUs can be used as the basic building block for deep convolutional neural networks, which yields powerful classifiers. Importantly, in this section we show that such a Convolutional Dynamic Alignment Network (CoDA-Net) inherits the properties (\colornum{P3}) of the DAUs by maintaining both the dynamic linearity (\colornum{P1}) as well as the alignment maximisation (\colornum{P2}). For a convolutional dynamic alignment layer, each filter is modelled by a DAU, similar to dynamic local filtering layers~\cite{jia2016dynamic}. Note that the output of such a layer is also a dynamic linear transformation of the input to that layer, since a convolution is equivalent to a linear layer with certain constraints on the weights, cf.~\cite{convlin}. We include the implementation details in the supplement.
Finally, at the end of this section, we highlight an important difference between output maximisation and optimising for classification with the {BCE} loss. In this context we discuss the effect of \emph{temperature scaling} and present the loss function we optimise in our experiments.

\myparagraph{Dynamic linearity (\colornum{P1}).} In order to see that the linearity is maintained, we note that the successive application of multiple layers of DAUs also results in a dynamic linear mapping. Let $\mat W_l$ denote the linear transformation matrix produced by a layer of DAUs and let $\vec a_{l-1}$ be the input vector to that layer; as mentioned before, each row in the matrix $\mat w_l$ corresponds to the weight vector of a single DAU\footnote{
Note that this also holds for convolutional DAU layers. Specifically, each row in the matrix $\mat w_l$ corresponds to a single DAU applied to exactly one spatial location in the input and the input with spatial dimensions is vectorised to yield $\vec a_{l-1}$. For further details, we kindly refer the reader to~\cite{convlin} and the implementation details in the supplement of this work.}. As such, the output of this layer is given by 
\begin{align}
    \vec a_l = \mat W_l (\vec a_{l-1}) \vec a_{l-1}\quad .
\end{align}
In a network of DAUs, the successive linear transformations can thus be collapsed. In particular, \emph{for any pair of activation vectors} $\vec{a}_{l_1}$ and $\vec{a}_{l_2}$ with ${l_1}<{l_2}$, the vector $\vec{a}_{l_2}$ can 
    be expressed as a linear transformation of $\vec{a}_{l_1}$:
\begin{align}
\label{eq:collapse}
    \vec{a}_{l_2} &= \mat{W}_{{l_1}\rightarrow {l_2}} \left(\vec{a}_{l_1}\right)\vec{a}_{l_1} \quad 
        \\{with} \quad \mat{W}_{{l_1}\rightarrow {l_2}}\left(\vec{a}_{l_1}\right) &= \textstyle\prod_{k={l_1}+1}^{l_2} \mat{W}_k \left(\vec{a}_{k-1}\right)\quad \text{.}
\end{align}
For example, the matrix $\mat W_{0\rightarrow L}(\vec{a}_0 = \vec{x}) = \mat W(\vec{x})$ models the linear transformation from the input to the output space, see Fig.~\ref{fig:teaser}.
Since this linearity holds between any two layers, the $j$-th entry of any activation vector $\vec a_l$ in the network can be decomposed into input contributions via:
    \begin{align}
    \label{eq:contrib}
        \vec{s}_{j}^l(\vec x_i) = \left[\mat W_{0\rightarrow l} (\vec{x}_i)\right]_j^T \odot \vec x_i\quad \text{,}
    \end{align}
    with $[\mat W]_j$ the $j$-th row in the matrix.
%

\myparagraph{Alignment maximisation (\colornum{P2}).}
Note that the output of a CoDA-Net is bounded independent of the network parameters: since each DAU operation can---independent of its parameters---at most reproduce the norm of its input (eq.~\eqref{eq:bound}), the linear concatenation of these operations necessarily also has an upper bound which does not depend on the parameters.
Therefore, in order to achieve maximal outputs on average (e.g., the class logit over the subset of images of that class), all DAUs in the network need to produce weights $\vec w (\vec a_l)$ that align well with the class features. In other words, the weights will align with discriminative patterns in the input.
For example, in Fig.~\ref{fig:alignment} (bottom), we visualise the `global matrices' $\mat W_{0\rightarrow L}$ and the corresponding contributions (eq.~\eqref{eq:contrib}) for a $L=5$ layer CoDA-Net. As before, the weights align with discriminative patterns in the input and do not encode the uninformative noise.
%
%
%
%

\myparagraph[0]{Temperature scaling and loss function.} 
\begin{figure}[t]
    \centering
    \includegraphics[width=.45\textwidth]{resources/Temperature_qualitative.pdf}
    \caption{\small By lowering the upper bound (cf.~eq.~\eqref{eq:bound}), the correlation maximisation in the DAUs can be emphasised.
    We show contribution maps for a model trained with different temperatures.
    }
    \label{fig:scaling}
\end{figure}
So far we have assumed that minimising the {BCE} loss for a given sample is equivalent to applying a maximisation or minimisation loss to the individual outputs of a CoDA-Net. While this is in principle correct, {BCE} introduces an additional, non-negligible effect: \emph{saturation}. Specifically, it is possible for a CoDA-Net to achieve a low {BCE} loss without the need to produce well-aligned weight vectors. As soon as the classification accuracy is high and the outputs of the networks are large, the gradient---and therefore the \emph{alignment pressure}---will vanish. This effect can, however, easily be mitigated:
 as discussed in the previous paragraph, the output of a CoDA-Net is upper-bounded \textit{independent of the network parameters}, since each individual DAU in the network is upper-bounded. 
By scaling the network output with a temperature parameter $T$ such that 
    $\hat{\vec y} (\vec x) = T^{-1} \mat W_{0\rightarrow L}(\vec x)\,\vec x$, 
we can explicitly decrease this upper bound and thereby increase the \emph{alignment pressure} in the DAUs by avoiding the early saturation due to {BCE}.
In particular, the lower the upper bound is, the stronger the induced DAU output maximisation should be, since the network needs to accumulate more signal to obtain large class logits (and thus a negligible gradient). This is indeed what we observe both qualitatively, cf.~Fig.~\ref{fig:scaling}, and quantitatively, cf.~Fig.~\ref{fig:localisation} (right column).
Alternatively, the representation of the network's computation as a linear mapping allows to directly regularise what properties these linear mappings should fulfill. For example, we show in the supplement that by regularising the absolute values of the matrix $\mat W_{0\rightarrow L}$, we can induce sparsity in the signal alignments, which can lead to sharper heatmaps.
%
The overall loss for an input $\vec x_i$ and the target vector $\vec y_i$ is thus computed as 
    \begin{align}
        \label{eq:loss}
        \mathcal{L}(\vec x_i, \vec y_i) &= 
        \text{BCE}(\sigma(T^{-1} \mat W_{0\rightarrow L}(\vec x_i)\,\vec{x}_i + {\vec{b}}_0)\,,\, \vec{y}_i) \\&+ 
        \lambda \langle | \mat W_{0\rightarrow L}(\vec x_i) |\rangle\quad \text{.}
    \end{align}
    Here, $\lambda$ is the strength of the regularisation, $\sigma$ applies the sigmoid activation to each vector entry,
    ${\vec{b}}_0$ is a fixed bias term, and $\langle|\mat W_{0\rightarrow L}(\vec x_i)|\rangle$ refers to the mean over the absolute values of 
        all entries in the matrix $\mat W_{0\rightarrow L}(\vec x_i)$.
%
%
%
\subsection{Implementation details}
\label{subsec:details}
\myparagraph[-.25]{Shared matrix \tmat b.} In our experiments, we opted to share the matrix $\mat b\in \mathbb R^{r\times d}$ between all DAUs in a given layer. This increases parameter efficiency by having the DAUs share a common $r$-dimensional subspace and still fixes the maximal rank of each DAU to the chosen value of $r$. 

\myparagraph[-.25]{Input encoding.} 
In sec.~\ref{subsec:align_units}, we showed that the norm-weighted cosine similarity between the dynamic weights and the layer inputs is optimised and the output of a DAU is at most the norm of its input. This favours pixels with large RGB values, since these have a larger norm and can thus produce larger outputs in the maximisation task. To mitigate this bias, we add the negative image as three additional color channels and thus encode each pixel in the input %is encoded 
as 
\mbox{[$r$, $g$, $b$, $1-r$, $1-g$, $1-b$]}, with $r, g, b\in [0, 1]$.

% \section{Analysis}
We look at users with long historical behavior sequence and found the DDPG model to be 

TODO: 添加类目新用户,老用户
\section{Experiment}
\label{sec:experiment}
\subsection{Evaluation Setup}
% \subsection{Setup}
% \label{sec:setup}
% Our primary evaluation is carried out on an in-house dataset collected from the 1 month of search log. This has the advantage of being 
% % We used one industrial scale in-house dataset and one public Amazon review dataset. The former has the advantage of being 
% directly reflected in online experiments, which is the most important way to validate our idea in an industrial setting. 
% The latter is better suited for reproducibility; furthermore we release our method to generate RL compatibility training and test data from the original public data. 
% We believe this will serve as a useful benchmark for other RL related studies in the e-commerce context.


% talk about details of the two datasets, evaluation metrics, training setup, parameters, implementation in Tensorflow, and so on.

\subsubsection{Training Data Generation}
We collect 30 days of training data from our in-house search log. Table~\ref{tab:in-house-data} summarizes its basic statistics. The total number of examples in DIN-S (pre-RNN) training is 200m, while under the RNN/S3DDPG data format, we have 6m variable length sessions instead. While the majority of users only have a single session, the number of sessions per user can go as high as 100. This makes our knapsack session packing algorithm~\ref{alg:knapsack_rnn} a key step towards efficient training. 
\begin{table}[htbp]
\centering
\caption{In-house data statistics.}
\small
\begin{tabular}{c|c|c|c}
\hline
statistics & mean & minimum & maximum \\
\hline
Number of unique users & 3788232 & - & - \\
\hline
sessions per user &	13.42 & 1 & 113 \\
\hline
items per session & 26.97 & 1 & 499 \\
\hline
Features per (query, item) & 110 & - & - \\
\hline
\end{tabular}
\label{tab:in-house-data}
\end{table}

% The variance of number of items across different query sessions is quite high. This is characteristic of e-commerce search, which resembles recommendation streams more than web search. 
A characteristic of e-commerce search sessions is the huge variance in browsing depth (number of items in a session). In fact, some power user can browse content up to thousands of items within a single session. 
The short sessions (such as the minimum number of 2 items in the table) are due to lack of relevant results. 



% \subsubsection{Training Data Generation}
% For the DIN baseline training, we generate pointwise dataset from $N = 30$ days of search log. 
Each DIN-S training example consists of a single query and a single item under the query session. To leverage users' historic sequence information, the data also includes the item id, category id, shop id, and brand id of the historical sequence of clicked / purchased / carted items by the current user. The sequence is truncated at a maximum length of 500 for online serving efficiency. 

For RNN and S3DDPG, each example consists of a pair of items under the same query. In order to keep the training data compact, i.e., without expanding all possible item pairs, the training data adopts the User Session Input format (Section~\ref{subsubsec:data_format}). To ensure all sessions under a user are contained within each minibatch, and ordered chronologically, the session data is further sorted by session id as primary key and session timestamp as secondary key during the data generation mapreduce job.


% To further ensure each minibatch contains entire history of a user, the data is constructed so that sessions under the same user appear contiguously in the data. This is achieved by relying on sorting by the primary query key and secondary timestamp key in the mapreduce reducer phase. 

During training, a random pair of items is sampled from each session, with one positive label (purchased) and one negative label (viewed/clicked only). Thus each minibatch consists of $\sum_{u=1}^B |S_u|$ item pairs, where $S_u$ stands for the set of all sessions under user $u$ and $B$ is the minibatch size, in terms of number of users. 

\subsubsection{Offline Evaluation}
We evaluate all models on one day of search log data beyond the training period. For RRNN and S3DDPG, however, 
% the evluation set also includes the previous 29 days of data, for a total of $N = 30$ days. This is 
% To evaluate the baseline DIN-S model, we simply take one day of search log data beyond the training period and construct an evaluation set similar to training, that is, each example consists of a query and an item, with binary label being whether or not the item was purchased under that query.
% For the RNN and S3DDPG evaluation, 
% we construct the evaluation set in the User Session Input format similar to training. In particular, all sessions under a single user are ordered chronologically and continguously within the data set. in order to capture the historical information,
we also include $N-1$ days prior to the last day, for a total of $N = 30$ days. The first $29$ days are there to build the user state vector only. Their labels are needed for user state aggregation during RNN forward evolution. Only labels from the last day sessions are used in the evaluation metrics, to prevent any leakage between training and validation.
% Labels from earlier days are needed only for aggregation of the user states during the RNN forward evolution. 

\subsubsection{Offline Evaluation Metrics}
While cross entropy loss \eqref{eq:reward_definition} and square loss \eqref{eq:dqn_td_loss} are used during training of S3DDPG, for hold-out evaluation, we aim to assess the ability of the model to generalize forward in time. Furthermore even though the training is performed on sampled item pairs, in actual online serving, the objective is to optimize ranking for an entire session worth of items, whose number of can reach the hundreds. Thus we mainly look at session-wise metrics such as Session AUC or NDCG. Session AUC in particular is used to decide early stopping of model training: 

\begin{align} \label{eq:session_auc}
    \text{Session AUC}(\eta, \lambda) := \sum_{u = 1}^B \sum_{t = 1}^{|S_u|} \rm{AUC}(\eta_{u, t}, \lambda_{u, t}),
\end{align}
where $\eta_{u, t}$ denotes the list of model predictions for \textbf{all items} within the session $(u, t)$ and $\lambda_{u, t}$ the corresponding binary item purchase labels. This is in contrast with training, where $\eta_{u, t}$, $\lambda_{u, t}$ denote predictions and labels for a randomly chosen positive / negative item pair. 

The following standard definition of ROC AUC is used in \eqref{eq:session_auc} above. For two vectors $\boldsymbol{p}, \boldsymbol{t} \in \R^n$, where $t_i \in \{0, 1\}$:
\begin{align*}
\rm{AUC}(\boldsymbol{p}, \boldsymbol{t}) := \frac{\sum_{1 \leq i < j \leq n} \sign(p_i - p_j) \sign(t_i - t_j)}{n(n-1) / 2},
\end{align*}
where $\sign(x) = x / |x|$ for $x \neq 0$ and $\sign(0) = 0$.

% We systematically extract a model checkpoint that attains the highest session AUC among all saved checkpoints.

NDCG is another popular metric in search ranking, intended to judge full page result relevance \cite{distinguishability2013theoretical}. It again takes the model predictions (which can be converted into ranking positions) as well as corresponding labels for all items, and compute a position-weighted average of the label, normalized by its maximal possible value:
\begin{align*}
    \rm{NDCG}(\boldsymbol{p}, \boldsymbol{t}) = \sum_{i=1}^n \frac{2^{t_i} - 1}{\log_2(i + 1)} / \sum_{i=1}^{\sum_j t_j} \frac{2^{t_i} - 1}{\log_2(i + 1)}.
\end{align*}

% where $\rm{DCG}$ and $\rm{IDCG}$ are in turn defined by
% \begin{align}
% \rm{DCG}(\boldsymbol{p}, \boldsymbol{t}) &:= \sum_{i=1}^n \frac{2^{t_i} - 1}{\log_2(i + 1)} \\
% \rm{IDCG}(\boldsymbol{p}, \boldsymbol{t}) &:= \sum_{i \leq n: t_i = 1} \frac{2^{t_i} - 1}{\log_2(i + 1)}.
% \end{align}

\subsubsection{Online Metrics}
For e-commerce search, there are essentially three types of core online metrics. 
\begin{itemize}
    \item GMV stands for gross merchandise value, which measures the total revenue generated by a platform. Due to the variation of A/B bucket sizes, it is often more instructive to consider GMV per user.
    \item CVR stands for conversion rate and essentially measures the number of purchases per click. Again this is averaged over the number of users.
    \item CTR is simply click-through rate, which measures number of clicks per query request. We do not consider this metric in our online experiments since it is not directly optimized by our models.
\end{itemize}
% \subsubsection{Online Serving}
% The ultimate objective of introducing the sequential search context as well as reinforcement learning architecture is to promote better user experience and personalized result relevance, measured primarily through a set of core metrics such as conversion rate and GMV. Therefore model online serving is a crucial component of the overall evaluation.

% The serving of the baseline DIN model is straightforward. First the model gets loaded into the system. Then upon a query request from a given user, the search engine computes all the required features for all retrieved items under the query, and sends them to the DIN model. The bulk of the feature preparation and transmission lies in the item features (either 1-sided or jointly 2-sided with query and user information), as well as the user behaviorial sequence features, which however are capped at the length of 500. Once the scores are computed for all items, they are sent back to the search engine and sorted to produce a ranking among the retrieved results.

% RNN and S3DDPG models are served essentially identically. Therefore most of the serving infrastructure (Figure~\ref{fig:incremental_update}) has been built when we launched the RNN model. There are two major differences with the DIN or non-RNN types of models:
% \begin{itemize}
%     \item In addition to the regular input features required by the model, RNN models also receive a numeric vector for the previous user state. 
%     \item This user vector is updated in real time after every user query request, so that when the same user comes back next time, he or she gets served a model with an updated state vector.
% \end{itemize}
% The updating of the user vector takes the following two inputs
% \begin{itemize}
%     \item A list of state vector candidates returned by the model, one for each item scored.
%     \item The user's behavioral feedback upon viewing the ranked results in the search page; for our purpose, which items got purchased, if any. 
% \end{itemize}
% If no item was purchased within the previous query session, the user state stays the same. Otherwise it gets updated with the candidate state vector corresponding to the purchased item.

\subsection{Evaluation Results}


\begin{table}[htbp] 
\centering
\caption{Offline Metrics}
\begin{tabular}{c|c|c}
\hline
Model name & Session AUC & NDCG \\
\hline
DNN & 0.6765 & 0.5104 \\
DIN-S & 0.6875 & 0.5200 \\
RNN & 0.6915 & 0.5272 \\
S3DDPG & 0.6968 & 0.5307 \\
\hline
\end{tabular}
\label{tab:offline_metrics}
\end{table}
We present both Session AUC and NDCG for the 4 models listed in Table~\ref{tab:offline_metrics}. The DNN baseline simply aggregates the user sequential features through sum-pooling, all of which are id embeddings. The successive improvements are consistent between the two session-wise metrics: RNN improved upon DIN-S by about $0.4\%$ in Session AUC and $0.7\%$ in NDCG, while S3DDPG further improves upon RNN by another $0.5\%$ in Session AUC and $0.4\%$ in NDCG. The overall gain of S3DDPG is around a full $1\%$ in either metrics from the DIN-S baseline, and $2\%$ from the DNN baseline.


Table~\ref{tab:user_group_evaluation} highlights the gain of S3DDPG over the myopic RNN baseline on a variety of user subsets. For instance, along the dimension of users' past session counts, S3DDPG shows a significantly stronger performance for more seasoned users in both validation metrics. Another interesting dimension is whether the current query belongs to a completely new category of shopping intent compared to the users' past search experience. Users who issue such queries in the evaluation set are labeled ``Category New Users". Along that dimension, we see that S3DDPG clearly benefits more than RNN from similar queries searched in the past.

While there are a number of hyperparameters associated with reinforcement learning models in general, the most important one is arguably the discount factor $\gamma$ parameter. We choose $\gamma = 0.8$ for all our S3DDPG experiments, since we found little improvement when switching to other $\gamma$ value. Another important parameter specific to actor-critic style architecture is the relative weight $\mu$ between PG loss and TD loss. Interestingly, as we shift weight from TD to PG loss (increasing $\mu$), there is a noticeable trend of improvement in both AUC and NDCG, as shown in Figure~\ref{fig:hyperparameter}. This suggests the effectiveness of maximizing the long term cumulative reward directly, even at the expense of less strict enforcement of the Bellman equation through the TD loss. When $\mu$ is set to $1$, however, training degenerates, as the Q-value optimized by PG loss is not bound to the actual reward (cross entropy loss) any more.

\begin{figure}
    \centering
    %\includegraphics[width=0.9\linewidth]{SessionAUC_µ.png}
    \includegraphics[width=\linewidth]{mu.png}
    \caption{influence of hyperparameter $\mu$}
    \label{fig:hyperparameter}
\end{figure}

Finally we conduct 3 sets of online A/B tests, each over a timespan of 2 weeks. The overall metric improvements are reported in Table~\ref{tab:online_metrics}. The massive gain from DNN to DIN-S is expected, since the DNN baseline, with sum-pooling of the sequential features, is highly ineffective at using the rich source of sequential data. Nonetheless we also see modest to large improvements between RNN and DIN-S and between S3DDPG and RNN respectively, in all core business metrics. Figures~\ref{fig:daily_ucvr} present the daily UCVR metric comparison for the last 2 sets of A/B tests.  Aside from a single day of traffic variation, both RNN and S3DDPG show consistent improvement over their respective baselines. 
\begin{figure}
    \centering
%    \includegraphics[width=\linewidth]{UCVR-S3DDPG-RNN.png}
%     \caption{UCVR (conversion rate/unique user) delta in \% from DIN-S and RNN models over 14 days.}
%     \label{fig:rnn_v_DIN-S}
% \end{figure}
% \begin{figure}
%     \centering
    \includegraphics[width=\linewidth]{UCVR_ONLINE.png}
    \caption{Daily UCVR \% improvement for online A/B tests over 14 days.}
    \label{fig:daily_ucvr}
\end{figure}
% The gain from DIN to RNN is relatively easy to understand: by giving the model the full spectrum of input features from previous sessions in the near-term (30 days), RNN is equipped with much more context information about the user than the DIN model, which only gets a few categorical features of the past interacted items. 
% % The gain from RNN to S3DDPG is much more elusive. After all, they both have the same set of input features, and RNN, with its GRU kernel, is already a time-proven method for dealing with sequential data. Furthermore, the vast majority of users in the evaluation set only have a single session on the last day.

% The offline metrics gain from RNN to S3DDPG mainly comes from the removal of a weighting scheme used in both DIN and RNN, intended to promote items with higher prices, since besides user conversion rate, another important metric is GMV (gross merchandise volume), which measure the the total revenue generated by users' shopping activities on the search platform. While the price weighting scheme was very helpful in boosting GMV per user for the baseline DIN model as well as the RNN model, we found in online experiments that removing it did not cause any negative effect on GMV/user. Quite surprisingly, it in fact led to huge boost of both GMV/user as well as the closely related RPM (revenue-per-mil) metrics, both of which increased by nearly $2\%$ in relative term compared to RNN with price weighting, at a significance level of p-value $\leq 1\%$. 

% In fact, the trick lies in a technical detail mentioned earlier in Section~\ref{sec:method}. Namely, for both DIN and RNN, we introduced a weighting scheme to promote higher priced items within the search results. This was effective in boosting GMV and related revenue metrics, without sacrificing too much on click-through and conversion rate. For S3DDPG, however, this was removed. The hope was that reinforcement learning with its emphasis on long term reward is able to bring in more user conversion that would offset the short-term loss in per item revenue.

% Indeed from the online metrics Table~\ref{tab:online_metrics}, we see highly significant gains in GMV / user 


% As mentioned in Section~\ref{subsec:rnn:pairwise}, 

% Experiments to report:
% 1. DIN
% 2. RNN
% 3. DDPG:
% 3a. no target Q
% 3b. target Q
% 4. DQN: not convergent
% 5. 4 reward experiments
% 6. Analysis: long/short validation set
% 7. off-policy TopK
% 8. gamma grid search
% 9. TD + PG no tuning
% 10. target actor 



% \begin{table}[htbp]
% \centering
% \caption{Hyperparameter µ}
% \begin{tabular}{c|c|c}
% \hline
% Parameter & Session AUC & NDCG \\
% \hline
% 0 & -- & -- \\
% \hline
% 0.1 & 0.6887 & 0.5226 \\
% \hline
% 0.2 & 0.6946 & 0.5284 \\
% \hline
% 0.3 & 0.6939 & 0.5281 \\
% \hline
% 0.4 & 0.6941 & 0.5281 \\
% \hline
% 0.5 & 0.6968 & 0.5307 \\
% \hline
% 0.6 & 0.6959 & 0.5307 \\
% \hline
% 0.7 & 0.6970 & 0.5304 \\
% \hline
% 0.8 & 0.6979 & 0.5311 \\
% \hline
% 0.9 & 0.6985 & 0.5329 \\
% \hline
% 1.0 & -- & -- \\
% \hline
% \end{tabular}
% \end{table}

% \begin{table}[htbp] \label{tab:offline_metrics}
% \centering
% \caption{S3DDPG vs RNN for Category New Users}
% \begin{tabular}{c|c|c|c}
% \hline
% Data Type & Model pairs & Session AUC & NDCG \\
% \hline
% New User & S3DDPG vs RNN & +0.3195\% & +0.1412\% \\
% \hline
% Old User & S3DDPG vs RNN & +0.6584\% & +0.6688\%\\
% \hline
% \end{tabular}
% \end{table}

\begin{table}[htbp]
\centering
\caption{S3DDPG vs RNN for different user evaluation groups}
\begin{tabular}{c|c|c}
\hline
User Group  & Session AUC & NDCG \\
\hline
Past Session Count < 5  & +0.6534\% & +0.2418\% \\
Past Session Count >= 5  & +0.9544\% & +0.9092\%\\
Category New Users & +0.3195\% & +0.1412\% \\
Category Old Users & +0.6584\% & +0.6688\% \\
\hline
\end{tabular}
 \label{tab:user_group_evaluation}
\end{table}

\begin{table}[htbp]
\centering
\caption{Online Metrics}
\begin{tabular}{c|c|c|c}
\hline
Model pairs & GMV/user & CVR/user & RPM \\
\hline
DIN-S vs DNN & +4.05\%(1e-3) & +3.51\%(1e-3) & +4.08\%(5e-3) \\
RNN vs DIN-S & +0.60\%(5e-3) & +1.58\%(9e-3) & +0.49\%(8e-3) \\
S3DDPG vs RNN & +1.91\%(8e-3) & +0.78\%(1e-2) & +1.94\%(9e-3) \\
\hline
\end{tabular}
 \label{tab:online_metrics}
\end{table}
\vspace{-10pt}
% \begin{itemize}
% \item 离线实验指标
% \begin{itemize}
% \item RNN vs. DNN
% \begin{itemize}
% \item auc +0.94\%
% \end{itemize}
% \item DQN(gamma=0.8) vs. DNN
% \begin{itemize}
% \item auc +1.51\%
% \end{itemize}
% \end{itemize}
% \begin{itemize}
% \item DQN(gamma=0.8) vs. RNN
% \begin{itemize}
% \item auc +0.56\%
% \item results for other gamma 
% % (see Diagram~\ref{dia:gamma_sweep}
% \end{itemize}
% \end{itemize}
% \end{itemize}
% 在线实验指标
% \begin{itemize}
% \item RNN vs. DNN(线上观察7天)
% \begin{itemize}
% \item UV价值(搜索GMV/搜索用户) : +2.40\%
% \item UCVR(订单行/搜索用户): +3.09\%
% \item GMV: +1.99\%
% \item RPM(1000*搜索GMV/搜索量): +2.54\%
% \end{itemize}
% \item DDPG vs. RNN(线上观察7天)
% \begin{itemize}
% \item UV价值:+1.91\%
% \item UCVR:+0.78\%
% \item GMV:+1.77\%
% \item RPM:+1.94\%
% \item 流动性:+4.24\%
% \item 流动性: 使用rbo距离计算每个query今天的昨天TOP20 SKU排序的diff率,多天取平均。
% \item CID4基尼不纯度:0.95\%
% \item CID4基尼不纯度:TOP60坑位有曝光的CID4的gini不纯度
% \end{itemize}
% \item AB实验指标
% \end{itemize}


\label{sec:industrial}

% \subsection{Public Dataset Experiment}
% \label{sec:public}
% We take the Amazon review dataset as our starting point. Each reviewer is identified through his or her reviewer id. Since every review event contains a timestamp, we can naturally form a temporal sequence. Unlike our in-house dataset, however, there is no analogue of a session (with multiple items) in the dataset. Instead, we simply consider the sequence of review events. To take advantage of full recurrence as proposed, we include expensive features such as item titles, past reviews, etc, in addition to asin\_id (item id) and category names.

% \subsection{Case Study}
\label{sec:case}



\section{Conclusions and Future Work}\label{section-conclusion}
In this work, we have systematically studied different key notions and results concerning anti-unification of unordered goals, i.e. sets of atoms. We have defined different anti-unification operators and we have studied several desirable characteristics for a common generalization, namely optimal cardinality (lcg), highest $\tau$-value (msg) and variable dataflow optimizations. For each case we have provided detailed worst-case time complexity results and proofs. An interesting case arises when one wants to minimize the number of generalization variables or constrain the generalization relations so as they are built on injective substitutions. In both cases, computing a relevant generalization becomes an NP-complete problem, results that we have formally established.
In addition, we have proven that an interesting abstraction -- namely $k$-swap stability which was introduced in earlier work -- can be computed in polynomially bounded time, a result that was only conjectured in  earlier work. 

Our discussion of dataflow optimization in Section~\ref{section-relation-2} essentially corresponds to a reframing of what authors of related work sometimes call the \textit{merging} operation in rule-based anti-unification approaches as in~\cite{Baumgartner2017}. Indeed, if the "store" manipulated by these approaches contains two anti-unification problems with variables generalizing the same terms, then one can "merge" the two variables to produce their most specific generalization. If the merging is exhaustive, this technique results in a generalization with as few different variables as possible. In this work we isolated dataflow optimization from that specific use case and discussed it as an anti-unification problem in its own right.

While anti-unification of goals in logic programming is not in itself a new subject, to the best of our knowledge our work is the first systematic treatment of the problem in the case where the goals are not sequences but unordered sets. Our work is motivated by the need for a practical (i.e. tractable) generalization algorithm in this context. The current work provides the theoretical basis behind these abstractions, and our concept of $k$-swap stability is a first attempt that is worth exploring in work on clone detection such as~\cite{clones}. 

Other topics for further work include adapting the $k$-swap stable abstraction from the $\preceq^\iota$ relation to dealing with the $\sqsubseteq^\iota$ relation. 
A different yet related topic in need of further research is the question about what anti-unification relation is best suited for what applications. For example, in our own work centered around clone detection in Constraint Logic Programming, anti-unification is seen as a way to measure the distance amongst predicates in order to guide successive syntactic transformations. Which generalization relation is best suited to be applied at a given moment and whether this depends on the underlying constraint context remain open questions that we plan to investigate in the future. 

%The main results of this paper are the polynomial algorithms solving specific anti-unification problems, along with several worst-case time complexity results and proofs. 

% have made efforts to extend the classical anti-unification concepts to the case where the artefacts to generalize are unordered goals. We have done this by considering different levels of atomic abstraction through different generalization relations. W





%Throughout the paper, we have introduced four generalization relations. Figure~\ref{fig-interconnexion} shows how the four relations are linked on a conceptual level. $\sqsubseteq$ is the most general relation as generalization is defined with any substitution. Restricting the definition to injective substitutions or to renamings yields more specific relations, the intersection of which is relation $\preceq^\iota$ where variables are generalized through injective renamings. 

%\begin{figure}[htbp]
%	\begin{center}
%		\begin{tikzpicture}[x=0.75pt,y=0.75pt,yscale=-1,xscale=1]
%		%uncomment if require: \path (0,300); %set diagram left start at 0, and has height of 300
%		
%		%Shape: Ellipse [id:dp330479544492589] 
%		\draw   (150.05,144.64) .. controls (150.05,72.29) and (186.09,13.64) .. (230.55,13.64) .. controls (275,13.64) and (311.05,72.29) .. (311.05,144.64) .. controls (311.05,216.99) and (275,275.64) .. (230.55,275.64) .. controls (186.09,275.64) and (150.05,216.99) .. (150.05,144.64) -- cycle ;
%		%Shape: Ellipse [id:dp3140532351606715] 
%		\draw   (165,87.62) .. controls (165,64.99) and (193.75,46.64) .. (229.22,46.64) .. controls (264.69,46.64) and (293.45,64.99) .. (293.45,87.62) .. controls (293.45,110.26) and (264.69,128.61) .. (229.22,128.61) .. controls (193.75,128.61) and (165,110.26) .. (165,87.62) -- cycle ;
%		%Shape: Ellipse [id:dp9580468020324391] 
%		\draw   (261.25,53.46) .. controls (285.77,54.29) and (304.38,90.37) .. (302.81,134.06) .. controls (301.24,177.74) and (280.08,212.49) .. (255.56,211.67) .. controls (231.03,210.85) and (212.43,174.76) .. (214,131.08) .. controls (215.57,87.39) and (236.72,52.64) .. (261.25,53.46) -- cycle ;
%		
%		% Text Node
%		\draw (172,266) node   {$\sqsubseteq $};
%		% Text Node
%		\draw (235,216) node   {$\preceq $};
%		% Text Node
%		\draw (170,125) node   {$\sqsubseteq^\iota $};
%		% Text Node
%		\draw (254,92) node   {$\preceq^\iota $};
%		\end{tikzpicture}
%	\end{center}
%	\caption{The interconnexions of four generalization relations}
%	\label{fig-interconnexion}
%\end{figure}

%Figure~\ref{fig-interconnexion} shows how the four relations are linked on a conceptual level. When needed in concrete applications, the right generalization operator (or an abstraction) should be used; this of course depends on whether or not the atomic structure should be generalized and the variable dataflow preserved. 

%Future work will focus on the use of such generalization operators in the purpose of applying synctatic transformations on predicates in such a way that the structural distance between them decreases; such a synctatic distance can be evaluated over the most specific generalization of the predicates under scrutiny.


\bibliographystyle{ACM-Reference-Format}
\balance
\bibliography{references}
\citestyle{acmauthoryear}




\end{document}
\endinput
%%
%% End of file `sample-authordraft.tex'.
