%%
%% This is file `sample-manuscript.tex',
%% generated with the docstrip utility.
%%
%% The original source files were:
%%
%% samples.dtx  (with options: `manuscript')
%% 
%% IMPORTANT NOTICE:
%% 
%% For the copyright see the source file.
%% 
%% Any modified versions of this file must be renamed
%% with new filenames distinct from sample-manuscript.tex.
%% 
%% For distribution of the original source see the terms
%% for copying and modification in the file samples.dtx.
%% 
%% This generated file may be distributed as long as the
%% original source files, as listed above, are part of the
%% same distribution. (The sources need not necessarily be
%% in the same archive or directory.)
%%
%% The first command in your LaTeX source must be the \documentclass command.
%%%% Small single column format, used for CIE, CSUR, DTRAP, JACM, JDIQ, JEA, JERIC, JETC, PACMCGIT, TAAS, TACCESS, TACO, TALG, TALLIP (formerly TALIP), TCPS, TDSCI, TEAC, TECS, TELO, THRI, TIIS, TIOT, TISSEC, TIST, TKDD, TMIS, TOCE, TOCHI, TOCL, TOCS, TOCT, TODAES, TODS, TOIS, TOIT, TOMACS, TOMM (formerly TOMCCAP), TOMPECS, TOMS, TOPC, TOPLAS, TOPS, TOS, TOSEM, TOSN, TQC, TRETS, TSAS, TSC, TSLP, TWEB.
% \documentclass[acmsmall]{acmart}

%%%% Large single column format, used for IMWUT, JOCCH, PACMPL, POMACS, TAP, PACMHCI
% \documentclass[acmlarge,screen]{acmart}

%%%% Large double column format, used for TOG
% \documentclass[acmtog, authorversion]{acmart}

%%%% Generic manuscript mode, required for submission
%%%% and peer review


\documentclass[sigconf]{acmart}
\usepackage{subfigure}
% \usepackage{CJKutf8}
\usepackage{algorithm}
\usepackage{algorithmic}
\usepackage{amsmath}
\usepackage{microtype}
\usepackage{balance}
% \usepackage[linesnumbered,ruled,vlined]{algorithm2e}

% \usepackage{CJK} % this gives errors.
\newcommand{\multirows}[1]{\begin{tabular}{@{}c@{}}#1\end{tabular}}
\newcommand{\cI}{\mathcal{I}}
\newcommand{\cU}{\mathcal{U}}
\newcommand{\cC}{\mathcal{C}}
\newcommand{\cL}{\mathcal{L}}
\newcommand{\cK}{\mathcal{K}}

\newcommand{\R}{\mathbb{R}}
\newcommand{\Z}{\mathbb{Z}}
\DeclareMathOperator{\sign}{sign}

%% NOTE that a single column version may be required for 
%% submission and peer review. This can be done by changing
%% the \doucmentclass[...]{acmart} in this template to 
%% \documentclass[manuscript,screen]{acmart}
%% 
%% To ensure 100% compatibility, please check the white list of
%% approved LaTeX packages to be used with the Master Article Template at
%% https://www.acm.org/publications/taps/whitelist-of-latex-packages 
%% before creating your document. The white list page provides 
%% information on how to submit additional LaTeX packages for 
%% review and adoption.
%% Fonts used in the template cannot be substituted; margin 
%% adjustments are not allowed.
%%
%%
%% \BibTeX command to typeset BibTeX logo in the docs
\AtBeginDocument{%
  \providecommand\BibTeX{{%
    \normalfont B\kern-0.5em{\scshape i\kern-0.25em b}\kern-0.8em\TeX}}}

%% Rights management information.  This information is sent to you
%% when you complete the rights form.  These commands have SAMPLE
%% values in them; it is your responsibility as an author to replace
%% the commands and values with those provided to you when you
%% complete the rights form.
\setcopyright{acmcopyright}
\copyrightyear{2021}
\acmYear{2021}
\acmDOI{10.1145/1122445.1122456}

%% These commands are for a PROCEEDINGS abstract or paper.
\acmConference[CIKM '21] {Proceedings of the 30th ACM International Conference on Information and Knowledge Management}{November 1--5, 2021}{Virtual Event, Australia.}
\acmBooktitle{Proceedings of the 30th ACM Int'l Conf. on Information and Knowledge Management (CIKM '21), November 1--5, 2021, Virtual Event, Australia}
\acmPrice{15.00}
\acmISBN{978-1-4503-8446-9/21/11}
\acmDOI{10.1145/3459637.3481954}
% Authors, replace the red X's with your assigned DOI string during the rightsreview eform process.

\settopmatter{printacmref=true}


%%
%% Submission ID.
%% Use this when submitting an article to a sponsored event. You'll
%% receive a unique submission ID from the organizers
%% of the event, and this ID should be used as the parameter to this command.
%%\acmSubmissionID{123-A56-BU3}

%%
%% The majority of ACM publications use numbered citations and
%% references.  The command \citestyle{authoryear} switches to the
%% "author year" style.
%%
%% If you are preparing content for an event
%% sponsored by ACM SIGGRAPH, you must use the "author year" style of
%% citations and references.
%% Uncommenting
%% the next command will enable that style.
%%\citestyle{acmauthoryear}

%%
%% end of the preamble, start of the body of the document source.
\begin{document}
\fancyhead{}

%%
%% The "title" command has an optional parameter,
%% allowing the author to define a "short title" to be used in page headers.
% \title{Off-Policy Reinforcement Learning in e-Commerce Sequential Search with Online Incremental Update}
%Lingfei: after the whole paper, I feel that online incremental update is not the key innovation. So we may consider the following title to highlight two new pieces: 1) Sequential Search (a new problem definition) and 2) RL for SS. 
\title{Sequential Search with Off-Policy Reinforcement Learning}

%%
%% The "author" command and its associated commands are used to define
%% the authors and their affiliations.
%% Of note is the shared affiliation of the first two authors, and the
%% "authornote" and "authornotemark" commands
%% used to denote shared contribution to the research.


\author{Dadong Miao}
% \email{midaodadong@jd.com}
\author{Yanan Wang}
% \email{wangyanan11@jd.com}
\author{Guoyu Tang}
% \email{tangguoyu@jd.com}
\affiliation{%
  \institution{JD.com}
  \streetaddress{JD Building, No. 18 Kechuang 11 Street, BDA}
  \city{Beijing}
  \postcode{101111}
  \country{People's Republic of China}
}

\author{Lin Liu}
% \email{liulin1@jd.com}
\author{Sulong Xu}
% \email{xusulong@jd.com}
\author{Bo Long}
% \email{bo.long@jd.com}
\affiliation{%
  \institution{JD.com}
  \streetaddress{JD Building, No. 18 Kechuang 11 Street, BDA}
  \city{Beijing}
  \postcode{101111}
  \country{People's Republic of China}
}


\author{Yun Xiao}
% \email{xiaoyun1@jd.com}
\author{Lingfei Wu}
% \email{lwm@mail.wm.edu}
\author{Yunjiang Jiang}
% \email{yunjiangster@gmail.com}
\affiliation{%
  \institution{JD.com Silicon Valley R\&D Center}
  \streetaddress{675 E Middlefield Road}
  \city{Mountain View}
  \state{CA}
  \postcode{94043}
  \country{USA}
}
\renewcommand{\shortauthors}{Dadong Miao, et al.}
% \author{Anonymous Authors}
% \renewcommand{\shortauthors}{Anonymous Authors, et al.}

%%
%% The abstract is a short summary of the work to be presented in the
%% article.
% \begin{abstract}
% We explore the use of off-policy reinforcement learning in multi-session e-commerce search. We design the training data in a maximally contiguous fashion to put a user's entire behavior sequence in a single minibatch.
% % give the model simultaneous access to the users' entire behavior sequence.
% In addition to attention network based on long range behavior sequence, a versatile RNN backbone is designed to capture near term behavior sequence, on which we also build a DDPG network.
% % to ensure long term reward signals are fully captured in the back-propagation. 
% As a novel optimization step, we fit multiple short user sequences in a single RNN pass within a training batch, by solving a greedy knapsack problem on the fly. 

% Empirically we demonstrate the effectiveness of the RNN backbone against its attention-based DIN counterpart, in both offline and online metrics. The addition of reinforcement learning further elevates these metrics to new heights. Besides the core metrics for e-commerce business, online experiments demonstrate significant gains in a variety of secondary metrics, most notably result diversity and level of personalization. The use of hidden state daily update further boosts online core metrics by a wide margin.
% \end{abstract}

%Lingfei: after the whole paper, I feel that online incremental update is not the key innovation. So we may consider the following title to highlight two new pieces: 1) Sequential Search (a new problem definition) and 2) RL for SS. 
\begin{abstract}
Recent years have seen a significant amount of interests in Sequential Recommendation (SR), which aims to understand and model the sequential user behaviors and the interactions between users and items over time. Surprisingly, despite the huge success Sequential Recommendation has achieved, there is little study on Sequential Search (SS), a twin learning task that takes into account a user's current and past search queries, in addition to behavior on historical query sessions.
% a user's long-range behavior sequence as well as their near-term behavior sequence over time. 
The SS learning task is even more important than the counterpart SR task for most of E-commence companies due to its much larger online serving demands as well as traffic volume.

% To address this challenge, in this paper, we first present a new learning task, Sequential Search in order to well capture a user's long-range and short-term behaviors for more accurate and personalized search.
To this end, we propose a highly scalable hybrid learning model that consists of an RNN learning framework leveraging all features in short-term user-item interactions, and an attention model utilizing selected item-only features from long-term interactions.
As a novel optimization step, we fit multiple short user sequences in a single RNN pass within a training batch, by solving a greedy knapsack problem on the fly. 
Moreover, we explore the use of off-policy reinforcement learning in multi-session personalized search ranking. Specifically, we design a pairwise Deep Deterministic Policy Gradient model that efficiently captures users' long term reward in terms of pairwise classification error.
Extensive ablation experiments demonstrate significant improvement each component brings to its state-of-the-art baseline, on a variety of offline and online metrics.     
\end{abstract}


%%
%% The code below is generated by the tool at http://dl.acm.org/ccs.cfm.
%% Please copy and paste the code instead of the example below.
%%
\begin{CCSXML}
<ccs2012>
<concept>
<concept_id>10010147.10010257.10010293.10010294</concept_id>
<concept_desc>Computing methodologies~Neural networks</concept_desc>
<concept_significance>500</concept_significance>
</concept>
<concept>
<concept_id>10002951.10003317.10003338.10003343</concept_id>
<concept_desc>Information systems~Learning to rank</concept_desc>
<concept_significance>500</concept_significance>
</concept>
</ccs2012>
\end{CCSXML}

\ccsdesc[500]{Computing methodologies~Neural networks}

\ccsdesc[500]{Information systems~Learning to rank}


%%
%% Keywords. The author(s) should pick words that accurately describe
%% the work being presented. Separate the keywords with commas.
\keywords{sequential-search, RNN, reinforcement-learning, actor-critic}


%% A "teaser" image appears between the author and affiliation
%% information and the body of the document, and typically spans the
%% page.
% \begin{teaserfigure}
%   \includegraphics[width=\textwidth]{sampleteaser}
%   \caption{Seattle Mariners at Spring Training, 2010.}
%   \Description{Enjoying the baseball game from the third-base
%   seats. Ichiro Suzuki preparing to bat.}
%   \label{fig:teaser}
% \end{teaserfigure}

%%
%% This command processes the author and affiliation and title
%% information and builds the first part of the formatted document.
\maketitle

% !TEX root = 0-20SPAWC_SAND.tex
% DO NOT REMOVE THE ABOVE COMMENT!
\section{Introduction}
%
Millimeter-wave (mmWave) and massive multi-user (MU) multiple-input multiple-output (MIMO) will be  core technologies for future wireless systems~\cite{larsson14a, rappaport15a}.
%
The combination of these technologies enables simultaneous communication to multiple user equipments (UEs) at unprecedentedly high data rates. 
%
These advantages come at the cost of significantly increased power consumption, implementation complexity, and system costs. A viable solution to address these challenges is the use of low-resolution data converters combined with sophisticated but efficient baseband processing algorithms in all-digital basestations (BS) architectures~\cite{dutta2019case,jacobsson17b,li17b,mo16b,panagiotis20}. 

\subsection{Channel Estimation with Low-Resolution Data Converters}
%
Coarse quantization of the received baseband samples, due to the use of low-resolution analog-to-digital converters (ADCs) at the BS, together with the high path loss at mmWave or terahertz (THz) frequencies~\cite{rappaport15b, gao16}, renders the acquisition of accurate channel estimates a challenging task.
%
Fortunately,  wave propagation at mmWave or THz frequencies is directional~\cite{akdeniz14a} and channels typically consist only of a few dominant propagation paths~\cite{rappaport13a,rappaport15a}. Both of these properties cause the channel vectors to be sparse in the beamspace domain, which can be exploited to perform denoising that improves reliability of data transmission~\cite{alkhateeb14a,mo14b,tang13,brady13,ghods19a}. 

Practical sparsity-exploiting channel denoising methods for mmWave massive MU-MIMO systems must exhibit low computational complexity due to the large number of BS antenna elements and the potentially large number of UEs that commmunicate simultaneously.
%
A low-complexity mmWave channel denoising algorithm called BEACHES (short for beamspace channel estimation) has been proposed recently in~\cite{ghods19a}. This method has orders-of-magnitude lower complexity than state-of-the-art denoising methods, such as atomic norm minimization (ANM)~\cite{bhaskar13} and Newtonized orthogonal matching pursuit (NOMP) \cite{mamandipoor16}. 
%
However, all of these existing denoising methods perform poorly when denoising channel vectors that were acquired through low-resolution data converters. 
%
Channel estimation with 1-bit ADCs has been analyzed in~\cite{li17b,li16a,jacobsson17b,mollen16c,studer16a}. 
%
Beamspace sparsity of mmWave channels has been exploited to denoise channel vectors from 1-bit measurements in \cite{mo16b, huang19,kaushik18}.
%
However, all of these denoising methods exhibit high complexity, ignore beamspace sparsity, and/or require a number of parameters that must be adapted to the instantaneous propagation conditions, such as the number of dominant propagation paths.  


\subsection{Contributions}
%
We propose low-complexity channel estimation algorithms for mmWave massive MU-MIMO systems that operate with 1-bit data converters.
%
By using a Bussgang-like decomposition~\cite{bussgang52a} of the 1-bit measurement process, our methods adapt the optimal denoising parameters to the channel's instantaneous sparsity via Stein's unbiased risk estimate (SURE).
%
We propose two methods that build upon BEACHES put forward in~\cite{ghods19a} and a novel method, referred to as Sparsity-Adaptive oNe-bit Denoiser (SAND), which automatically tunes two algorithm parameters to minimize the channel estimation mean-square error (MSE). 
%
To demonstrate the efficacy of our channel estimation algorithms, we perform  MSE and bit error rate (BER) simulations with  line-of-sight (LoS) and non-LoS mmWave channels in a massive MU-MIMO system. 

\subsection{Notation}
%
Lowercase and uppercase boldface letters denote column vectors and matrices, respectively. 
%
The $k$th entry of the vector~$\bma$ is~$a_k$; 
the real and imaginary parts are $\realindex{\bma} = \realpart{\bma}$ and $\imagindex{\bma} = \imagpart{\bma}$, respectively. 
For a matrix $\bA$, 
its transpose and Hermitian transpose are $\bA^\Tran$ and $\bA^\Herm$, respectively. 
A complex Gaussian vector $\bma$ with mean $\bmm$ and covariance $\bK$ is written as $\bma\sim\CN(\bmm, \bK)$.
%
Expectation is denoted by $\smolE{\cdot}$. 
%

\myparagraph[0]{Interpretability.}
In order to make machine learning models more interpretable, a variety of techniques has been developed.
On the one hand, 
    research has been undertaken to develop model-agnostic explanation methods for which the model behaviour
    under different inputs is analysed; this includes among others \cite{lundberg2017unified,petsiuk2018rise,lime}.
    While their generality and the applicability to any model are advantageous,
    these methods typically require evaluating the respective model several times and are therefore costly
    approximations of model behaviour.
    %
On the other hand,
    many techniques that explicitly take advantage of the internal computations have been proposed for explaining
    the model predictions, including, for example, \cite{simonyan2013deep,springenberg2014striving,zhou2016CAM,selvaraju2017grad,shrikumar2017deeplift,sundararajan2017axiomatic,srinivas2019full,bach2015pixel}.\\
    %
In contrast to techniques that aim to explain models \emph{post-hoc},
some recent work has focused on designing new types of network architectures, which are \emph{inherently} more interpretable.
Examples of this are the prototype-based neural networks~\cite{chen2019looks}, the BagNet~\cite{brendel2018approximating}
and the self-explaining neural networks (SENNs)~\cite{melis2018towards}.
Similarly to our proposed architectures,
    the SENNs and the BagNets derive their explanations 
    from a linear decomposition of the output into contributions from the input (features).
This \emph{dynamic linearity}, i.e., the property that the output is computed via some form of an input-dependent linear mapping, is additionally shared by the entire model family of piece-wise linear networks (e.g., ReLU-based networks). In fact, the contribution maps of the CoDA-Nets are conceptually similar to  evaluating the `Input$\times$Gradient' (IxG), cf.~\cite{adebayo2018sanity}, on piece-wise linear models, which also yields a linear decomposition in form of a contribution map.
However, in contrast to the piece-wise linear functions, we combine this \emph{dynamic linearity} with a structural bias towards an alignment between the contribution maps and the discriminative patterns in the input. This results in explanations of much higher quality, whereas IxG on piece-wise linear models has been found to yield unsatisfactory explanations of model behaviour~\cite{adebayo2018sanity}.

\myparagraph{Architectural similarities.} In our CoDA-Nets, the convolutional kernels are dependent on the specific patch that they are applied on; i.e., a CoDA-Layer might apply different filters at every position in the input. As such, these layers can be regarded as an instance of dynamic local filtering layers as introduced in~\cite{jia2016dynamic}.
Further, our dynamic alignment units (DAUs) share some high-level similarities to attention networks, cf.~\cite{xu2015show}, in the sense that each DAU has a limited budget to distribute over its dynamic weight vectors (bounded norm), which is then used to compute a weighted sum. However, whereas in attention networks the weighted sum is typically computed over vectors, which might even differ from the input to the attention module, a DAU outputs a \emph{scalar} which is a weighted sum of all scalar entries in the input. Moreover, we note that at their optimum (maximal average output over a set of inputs), the DAUs solve a constrained low-rank matrix approximation problem~\cite{eckart1936approximation}. While low-rank approximations have been used for increasing parameter efficiency in neural networks, cf.~\cite{yu2017compressing}, this concept has to the best of our knowledge not been used in order to endow neural networks with a structural bias towards finding low-rank approximations of the input for increased interpretability in classification tasks. Lastly, the CoDA-Nets 
are related to capsule networks. However, whereas in classical capsule networks the activation vectors of the capsules directly serve as input to the next layer, in CoDA-Nets the corresponding vectors are used as convolutional filters. 
We include a detailed comparison in the supplement. 
\section{Method}
\label{sec:method}
We introduce the main model architecture in this section. Each subsection forms the foundation for the next one. Section~\ref{sec:din} describes an attention based network inspired by \cite{zhou2018deep} that exploits the correlation between the current ranking task and the entire historical sequence of items for which the user has expressed interest. Section~\ref{sec:rnn} details an elaborate Recurrent Neural Net backbone that can efficiently handle a batch of uneven sized sequences, and how it is used to capture more recent user interactions with the search engine. The output embedding of the attention network is simply fed as an input into every timestamp of the recurrence. Finally we discuss how to build an actor-critic style reinforcement learning model on top of the RNN structure in section~\ref{sec:ddpg}. 

\subsection{Attention for Long-Term Session Sequence}
\label{sec:din}
As mentioned in section~\ref{subsec:rw:user}, the attention network in the Deep Interest Network (DIN) model is a natural way to incorporate user history into personalized recommendation. To adapt to the search ranking setting, we introduce Search Ranking Deep Interest Network (DIN-S), which makes the following adjustments on top of DIN:
\begin{itemize}
    \item Query side features are introduced alongside the focus item features, to participate in attention with historical item sequence.
    \item To account for the possibility that the current query request is not correlated with any of the user's past actioins, a zero score is appended to the regular attention scores before getting the softmax weights. This is illustrated by the zero attention unit in Figure~\ref{fig:din}.
\end{itemize}https://www.overleaf.com/project/5fed74e5fac2600586cb62da
% The only difference in personalized search is to add query related features to the attend list. 
\begin{figure}
    \centering
    \includegraphics[width=\linewidth]{DIN.png}
    \centering
    \caption{DIN-S architecture.}
    \label{fig:din}
\vspace{-5pt}
\end{figure}

The overall architecture of DIN-S is outlined in Diagram~\ref{fig:din}. Due to the nature of algorithmic iterations within an industrial setting, DIN-S is not only one of our quality comparison baselines, but also one of the major components in our proposed final architecture.
We have also tried DIEN \cite{zhou2019deep} and other follow up works. Despite the better results reported in papers, we found little incremental improvement in our own systems. However our design of the RNN backbone model shares some similarity with the approach in DIEN, and indeed both ours and the DIEN work use attention and RNN together. A key difference, however, is that our RNN training data and algorithmic design uses all features available in previous interactions by the user, including both item or query/user one-sided as well as two-sided features. By contrast, the RNN (GRU) in DIEN appears to be just an extension of the co-existing Attention network, taking mostly item-side only categorical features.

\subsection{RNN for Near-Term Sequential Search}
\label{sec:rnn}
In order to compare with the baseline DIN-S (attention + MLP) model fairly and conveniently, we build a so-called RNN backbone that can wrap around any base model architecture. The logic is outline in the bottom half of Diagram~\ref{fig:rnn}. To summarize, for any base model $M$, the RNN backbone introduces a new feature vector $H_t$, the hidden state, concatenated to the output of $M$. The output of each time iteration of the RNN model is another vector $H_{t+1}$, which serves both as the input to downstream networks, as well as the hidden state input for the next time iteration. 

\subsubsection{Contiguous Session-Based Data Format}
\label{subsubsec:data_format}
While DIN-S can be trained in a pointwise / pairwise fashion, our implementation of RNN tries to pool all relevant information together in the data by 
\begin{itemize}
    \item arranging all items within a query session in a single training example. In our case we used tsv (tab-separated values) format. Thus the number of columns in each row is variable, depending on the number of items under the session.
    \item placing all query sessions belonging to the same user contiguously to ensure they are loaded altogether in a mini-match.
\end{itemize}
\begin{figure}
    \centering
    \includegraphics[width=\linewidth]{DDPG-UserSessionTSV.png}
    \centering
    \caption{User Session Input Format. $B$ stands for batch size. Each row represents a single TSV row in the input data. The numbers of columns = number of query features + num of items $\times$ number of item features.}
    \label{fig:user_session_tsv}
\vspace{-5pt}
\end{figure}
As illustrated in Figure~\ref{fig:user_session_tsv}, each mini-batch thus contains a 4d tensor $B$ whose elements are indexed by $(u, t, i, f)$, which stand for users, sessions (time-ordered), items, and features respectively. We assume that features are all dense or have been converted to fixed width dense format, through either embedding sum-pooling or 
attention-pooling from the DIN-S base model. We will use $B_{u, t}$ to denote the 2d slices of $B$ containing all $(i, f)$ values. 
% \vspace{-20pt}

\subsubsection{RNN Model Architecture}
\begin{figure}
    \centering
    \includegraphics[width=\linewidth]{RNN.png}
    \centering
    \caption{RNN architecture.}
    \label{fig:rnn}
\end{figure}
Let $\omega_{u, t}, H_{u, t}$ stand for the regular output and hidden state output of the RNN network for user $u$ and session $t$. The RNN network can thus be described by a function $F$ with the following signature
\begin{align} \label{eq:rnn_kernel}
    (\omega_{u, t+1}, H_{u, t+1}) = F(B_{u, t + 1}, H_t).
\end{align}
This is the most general form of an RNN network. All RNN variants such as LSTM, GRU obey the above signature of $F$.

Recall now $B_{u, t}$ is a 2d tensor, with dimension given by (the number of items, number of features). The same is true of the output tensor $\omega_{u,t}$. The hidden state $H_{u, t}$ however has \textbf{no item dimension}: it is a fixed width vector for a given user after a given session. For simplicity, our choice of $F$ simply computes $H_{u, t}$ as a weighted average of the output $\omega_{u, t}$. More precisely,
\begin{align}
    \omega_{u, t+1}, S_{u, t+1} &= \rm{GRU}(\omega_{u, t}, S_{u, t}) \\
    H_{u, t+1, f} &= \frac{1}{|\cC_{u, t}|} \sum_{j \in \cC_{u, t}} \omega_{u, t+1, j, f}. 
\end{align}
Here $\cC_{u, t}$ stands for the set of items in user $u$'s session $t$ that were purchased. Those sessions without purchases are excluded from our training set, since under the above framework, 
\begin{enumerate}
    \item the user hidden state would not be updated;
    \item the final pairwise training label contains no information.
\end{enumerate} 
The vast majority of the remaining sessions contain exactly 1 purchased item.

This completes the description of the RNN evolution of the state vector under a listwise input format, where all items in a session are used. For training efficiency, however, we adopt a pairwise setup, where 2 items are sampled from each query session, and exactly one of them has been purchased. Thus we can think of each session as consisting of exactly 2 items. Since the hidden state is a weighted average over only the purchased items, pairwise sampling preserves all the information for the hidden state vector in a single RNN step, provided the session contains only a single purchased item, which is more than $90\%$ of the cases.

Lastly, the RNN model outputs a single logit $\eta_{u, t, i}$ for each item $i$ chosen within the user session $(u, t)$, by passing the RNN output vector $\omega_{u, t, i} \in \R^d$ of dimension $d$ through a multi-layer perception $P$ of dimensions $[1024, 256, 64, 1]$:
\begin{align} \label{eq:rnn_output}
    \eta_{u, t, i} = P(O(u, t, i)),  \quad P: \R^d \to \R.
\end{align}
The corresponding label is a binary indicator $\lambda_{u, t, i} \in \{1, 0\}$, which denotes whether the item was purchased or not.

\subsubsection{Pairwise Loss} \label{subsec:rnn:pairwise}
Unlike clicks or mouse hover actions, each page session in e-commerce search typically receives at most one \textbf{purchase}. Thus we are confronted with severe positive and negative label imbalance. To address this problem, we choose pairwise loss in our modeling, which samples a purchased item from the current session at random, and matches it with a random item that is viewed or clicked but not purchased. 

The exact sampling procedure is described in Algorithm~\ref{alg:pairwise_sampling}. Note that as long as the session is non-empty, the procedure will always output a pair. There are occasional edge cases when all items are purchased, in which case we output two purchased items. Alternatively, such perfect sessions can be filtered from the training set.

\begin{algorithm}[t]
\caption{Pairwise sampling from a query session.}
\label{alg:pairwise_sampling}
\begin{algorithmic}[1]
\REQUIRE{a list of $N > 0$ labels: $\lambda_1, \ldots, \lambda_N \in \{0, 1\}$}
\ENSURE{two indices $1 \leq a, b \leq N$, s.t. $\lambda_a = \max \lambda_i$ and $\lambda_b = \min \lambda_i$}
\STATE Compute $\lambda_{\min} := \min_i \lambda_i$ and $\lambda_{\max} := \max_i \lambda_i$.
\STATE Construct the list of admissible pairs $A := \{(a, b) \in [N]^2: \lambda_a = \lambda_{\max}, \lambda_b = \lambda_{\min}\}$ 
\STATE Output a uniformly random element $(a, b)$ from $A$.
\end{algorithmic}
\end{algorithm}
% Our choice of pairwise loss is motivated by the disparity between positive and negative examples within each session: typically users only make a single purchase out of a page of 10 results. 
The final loss function on an input session $(u, t)$ is given by the following standard sigmoid cross entropy formula:

\begin{align} \label{eq:logloss}
    &\cL(B_{u, t}, \lambda_{u, t}) = -\lambda_{u, t} \log \sigma(\eta_{u, t}) - (1-\lambda_{u, t}) \log (1 - \sigma(\eta_{u, t}))
\end{align}
where 
\begin{itemize}
\item $B_{u, t}$ stands for all the features available to the model for a given user session $(u, t)$.
\item $\lambda_{u, t} := \lambda_{u, t, a, b} = \frac{\lambda_{u, t, a}}{\lambda_{u, t, a} + \lambda_{u, t, b}} \in \{0, 1\}$, depending on whether a purchase was made on item $a$ or $b$ within the user session $(u, t)$. 
\item $\eta_{u, t} := \eta_{u, t, a, b} = \eta_{u, t, a} - \eta_{u, t, b}$ is simply the difference between the model outputs for the two items $a$ and $b$, which can be interpreted as the log-odds that the purchase was made on the first item.
\item $a, b$ are a pair of random item indices within the current session, chosen according to Algorithm~\ref{alg:pairwise_sampling}, where item $a$ is purchased while item $b$ is not.
\item $\sigma(\eta_{u, t})$ transforms the pairwise logit $\eta_{u, t}$ through the sigmoid function $\sigma: x \mapsto (1 + e^{-x})^{-1}$, and can be interpreted as the model predicted probability that item $a$ is purchased, given exactly one of item $a$ and $b$ is purchased.
\end{itemize}

% \\
%     &\frac{\ell_{u, t, i}}{\ell_{u, t, i} + \ell_{u, t, j}} \log \frac{e_{u, t, i}}{e_{u, t, i} + e_{u, t, j}} + \frac{\ell_{u, t, j}}{\ell_{u, t, i} + \ell_{u, t, j}} \log \frac{e_{u, t, j}}{e_{u, t, i} + e_{u, t, j}}
\subsubsection{Knapsack Sequence Packing}
Since the numbers of historical sessions vary widely across different users, the naive implementation of the above 4d representation can be computationally quite wasteful due to excessive zero padding. We thus adopt a knapsack strategy (Algorithm~\ref{alg:knapsack_rnn}) to fit multiple short user session sequences into the maximum length seen in the current mini-batch. 

\begin{algorithm}[t]
\caption{Parallel RNN via Knapsack Packing}
\label{alg:knapsack_rnn}
\begin{algorithmic}[1]
\REQUIRE{a list of $N$ (user, session) indices: $\cI = \{(u_1, 1), \ldots, (u_1, T_1), \ldots, (u_n, T_n)\}$}
\REQUIRE{input feature vectors associated with each (user, session) pair: $\{B_{u, t} \in \R^D: (u, t) \in \cI\}$}
\REQUIRE{an expensive RNN kernel $\tilde{O}: \R^{2D} \to \R^{2D}$}
\ENSURE{efficient computation of $\{\omega_{u, t} := O(B_{u, t}): (u, t) \in \cI\}$}
\STATE Apply the greedy knapsack strategy (Algorithm~\ref{alg:greedy_knapsack}) to get a mapping $m: (u, t) \mapsto (u', t')$, as well as the 2d array $S := \{S_{u', t'}\}$ that encodes the starting positions of the subsequences.
\STATE Construct a new input features $B'$ according to $B'_{u', t'} = B_{u, t}$.
\STATE Zero pad the missing entries of $B'$, for vectorized processing.
\STATE Compute $\omega' := O'(B', S)$ for all packed users in parallel.
\STATE Rerrange $\omega'_{u', t'}$ into the original user sequences $\omega_{u, t}$ via the inverse map $m^{-1}: (u', t') \mapsto (u, t)$.
\end{algorithmic}
\end{algorithm}
To break down Algorithm~\ref{alg:knapsack_rnn}, we introduce a few terminologies:
\begin{definition}
For a given RNN kernel $\tilde{O}: \R^D \times \R^D \to \R^D \times \R^D$, its associated \textbf{sequence map} $O: \R^{D \times T} \to \R^{D \times T}$, $(B_1, \ldots, B_T) \mapsto (\omega_1, \ldots, \omega_T)$ is given inductively by 
\begin{align*}
    (\omega_1, H_{u, 1}) &:= \tilde{O}(B_{u, 0}, H_{u, 0}) \\
    (\omega_{t+1}, H_{u, t+1})  &:= \tilde{O}(\omega_t, H_{u, t}) \quad \text{for } t \leq T -1.
\end{align*}
The initial hidden state is typically chosen to be the all zero vector: $H_{u, 0} = \vec{0}$.
\end{definition}

Note that after applying the knapsack packing Algorithm~\ref{alg:greedy_knapsack}, the maximum length of all the sequences stays the same. The total number of sequences, however, is reduced, by an average factor of 20x. As a result, some new sequence now contains multiple old sequences, arranged contiguously from the left. In such cases, we do not want the hidden states to propagate across sequences. Thus we introduce the following extended RNN sequence map that takes into account the old sequence boundary information:
\begin{definition}
Given an RNN kernel $\tilde{O}$ as above, and a 2d indicator array $\{S_t \in \{0, 1\}: 1 \leq t \leq T\}$ denoting the starting positions of sub-sequences within each user sequence, the \textbf{boundary-aware sequence map} 
\begin{align*}
    O': \R^{D \times T} \times  \{0, 1\}^{D \times T} \to \R^{D \times T}, \quad (B_1, \ldots, B_T) \mapsto (\omega_1, \ldots, \omega_T)
\end{align*} 
is defined via the following inductive formula
\begin{align*}
    (\omega_1, H_{u, 1}) &:= \tilde{O}(B_{u, 0}, H_{u, 0}) \\
    (\omega_{t + 1}, H_{u, t + 1}) &:= 
    \begin{cases}
      \tilde{O}(B_{u, t}, H_{u, 0}) & \text{if  $S_{t+1} = 1$ }\\
      \tilde{O}(\omega_t, H_{u, t}) & \text{otherwise}
    \end{cases}  
\end{align*}
\end{definition}
 
% TODO(jyj): check table stats to ensure logical soundness here.
% The numbers of items per session typically do not vary by much, and have a mean around 10; indeed we count users' next page request under the same query as a separate session. Thus there is little marginal benefit to apply knapsack packing along the item dimension. Furthermore, under the pairwise training strategy (Section~\ref{subsec:rnn:pairwise}, each session consists of exactly 2 items, making the item dimension already uniform across sessions. 

Overall the session knapsack strategy saves about 20x compute and speed up CPU training time by about 3x. Note that during online serving, knapsack is not needed since we deal with one user at a time.
\begin{algorithm}[t]
\caption{Greedy Knapsack Sequence Packing.}
\label{alg:greedy_knapsack}
\begin{algorithmic}[1]
\REQUIRE{A nonempty index set $\cI := \{(u_1, 1), \ldots, (u_1, T_1), \ldots, (u_n, T_n)\}$.}
\ENSURE{an index map $m: \cI \to \cU' \times [T']$, where $|\cU'| \leq n$ is the packed user index set and $T' \leq \max_i T_i$.}
\ENSURE{a 2d array $S_{u', t'}$ indicating the start positions of subsequences in each packed user sequence.}  
\STATE Set $T' := \max_{i=1}^n T_i$.
\STATE Initialize $U := [n] \setminus \{i\}$.
\STATE Initialize the list of knapsacks $\cK \leftarrow []$.
\STATE Initialize $S_{u', t'}$ to the all $0$ 2d array.
\WHILE {$U \neq \emptyset$}
    \STATE Pop a longest user sequence from $U$, say $u_j$.
    \IF {$T_j + \sum_{k \in \cK_i} T_k < T'$ for some $i \leq |\cK|$}
        \STATE Define $m(j, \ell) := (i, \ell + \sum_{k \in \cK_i} T_k)$ for $\ell < T_j$.
        \STATE Set $S_{i, \sum_{k \in \cK_i} T_k} \leftarrow 1$.
        \STATE Append $j$ to the end of $K_i$.
    \ELSE
        \STATE Append $[j]$ to the end of $\cK$.
        \STATE Define $m(j, \ell) := (|\cK|, \ell)$
        \STATE Set $S_{|\cK|, 1} \leftarrow 1$.
    \ENDIF
\ENDWHILE

\end{algorithmic}
\end{algorithm}
\subsection{DDPG for Near-Term Future Sessions}
\label{sec:ddpg}

While attention and RNN are capable of leveraging past sequential data, they fall short of predicting or optimizing future user behavior several steps in advance. This is not surprising because the former are essentially trained in a supervised approach, where the target is simply the next session. To optimize trajectories of several future sessions, we naturally turn to the vast repertoire of reinforcement learning (RL) techniques. 

As mentioned in Section~\ref{subsec:rw:rnn}, unlike the vast majority of RL literature in search and recommendation, our trajectory of agent (ranker) / environment (user) interaction is not confined within a single query session. Instead the user continues to type new queries, over a span of weeks or months. Thus the environment changes from one session to the next. However a key assumption here is that the different manifestations of the environment (user) share an underlying preference theme, as a single user's shopping tastes are strongly correlated across multiple shopping categories or intents.

Another important difference between our sequential session setup and the single session setup in other works is that each step of S3DDPG needs to rank a list of tens or hundreds of items, rather than just picking the top K from the remaining candidate pool. Due to the combinatorial explosion associated with ranking tasks, it becomes infeasible to treat the set of permutations of the items as our action space. Instead we take the vector output of the RNN network, along with the actor network prediction, as the action, which lives in a continuous space.
% Table~\ref{tab:modeling_differences} summarizes the key areas where S3DDPG departs from existing RL work in search and recommendation.

% \begin{table}[htbp]
% \centering
% \caption{Modeling Difference}
% \small
% \begin{tabular}{c|c|c}
% \hline
% Key Areas & Our Work & Typical Other Work \\
% \hline
% Single Step Task & \begin{tabular}{@{}c@{}}Independent \\ Ranking\end{tabular} & \begin{tabular}{@{}c@{}} Masked Top-K \\ Retrieval \end{tabular} \\
% \hline
% Explicit Actions & 10! permutations & Candidate Items \\
% \hline
% Implicit Actions & RNN output vector & Candidate Items \\
% \hline
% \begin{tabular}{@{}c@{}} Environment \\
% (User Intent) \end{tabular} & Dynamic & Static \\
% \hline
% \end{tabular}
% \label{tab:modeling_differences}
% \end{table}


\subsubsection{S3DDPG Network}

\begin{figure}
    \centering
    % \includegraphics[width=\linewidth]{DDPG.png}
    \includegraphics[scale=0.5]{ddpg_new.png}
    \centering
    \caption{S3DDPG architecture.}
    \label{fig:ddpg}
\end{figure}

Finally we come to our reinforcement-learning based ranking framework, which is depicted in Diagram~\ref{fig:ddpg}. The bottom half of the network consists of the RNN structure described in the previous subsection. The reinforcement learning part takes the regular RNN output (i.e., non-hidden state related) as the input, and is similar to the actor/critic framework. We closely follow the logic of DDPG network \cite{lillicrap2015continuous}.

The actor network has the same structure as the final MLP layers in RNN that takes the intermediate embedding to per-item logit. The latter is thus used also as the final ranking score for each item within a single query session. 

The critic network (also known as the Q-network) is a separate multi-layer perceptron, $Q: \R^d \to \R$, taking \textbf{a pair of RNN outputs} $\omega_{u, t, a}, \omega_{u, t, b} \in \R^d$ to a single scalar logit. 
\begin{align*}
    q_{u, t} := Q(\omega_{u, t, a}, \omega_{u, t, b}) \in \R.
\end{align*}

$Q$ is introduced here to approximate the following maximal cumulative discounted long term reward:
\begin{align*}
    q_{u, t} \sim \sup_{\eta_{u, t}, \ldots, \eta_{u, T}} \sum_{s = t}^T \gamma^{s - t} r(\eta_{u,  t}, \lambda_{u, t}).
\end{align*}

Here the supremum is taken over all trajectories starting at session $t$, and the reward $r(\eta_{u,t},\lambda_{u, t}) =: r_{u, t}$ is simply given by the opposite of the sigmoid cross entropy loss $\cL(B_{u, t}, \lambda)$ (See \eqref{eq:logloss}):
% reproduced here for convenience:
\begin{align} \label{eq:reward_definition}
    r(\eta_{u, t}, \lambda_{u, t}) = \lambda_{u, t} \log \sigma(\eta_{u, t}) + (1 - \lambda_{u, t}) \log \sigma(1 - \eta_{u, t}).
\end{align}
The time horizon $T$ itself is also random in general.

To summarize, we have introduced three networks and their associated output layers so far
\begin{itemize}
    \item $\omega_{u, t, a}, \omega_{u, t, b} \in \R^d$ are the output vectors of the RNN network for the chosen item pair.
    \item $\eta_{u, t} = P(\omega_{u, t, a}) - P(\omega_{u, t, b})$ is the scalar output of the actor network, which has the interpretation of log-odds of the first item being purchased.
    \item $q_{u, t} = Q(\omega_{u, t, a}, \omega_{u, t, b})$ is the scalar output of the critic (Q) network for the pair.
\end{itemize}


The critic (Q) network differs significantly from the actor network $P$ in that the input consists of pairs of items. Thus unlike $\eta_{u, t}$, it is not anti-symmetric under swapping of the item pair.


It is interesting to note that the original supervised loss function $\cL(\eta, \lambda)$ has been re-purposed as the reward in the Q-network. The actual loss functions are defined next.
% At this stage, we have not defined the loss function of our S3DDPG model yet, which we proceed to do next.

\subsubsection{Loss Functions}
There are two loss functions in the S3DDPG framework. The first of these two, the temporal difference (TD) loss, is well-known since the first DQN paper \cite{mnih2013playing}. It aims to enforce the Bellman's equation on the Q-values:
\begin{align} \label{eq:bellman}
    q_{u, t} = \sup_{\eta_{u, t}} r(\eta_{u, t}, \lambda_{u, t}) + \gamma q_{u, t + 1}.
\end{align}
Here $\gamma$ is a discount factor, which is set to $0.8$ throughout our experiments. The associated TD loss would then be
\begin{align} \label{eq:dqn_td_loss}
    \cL^{\text{DQN}}_{\text{TD}} (B_{u, t}, \lambda_{u, t}) := \sum_{u \in \cU} \sum_{t=1}^{T - 1} (q_{u, t} - \sup_\eta \left\{r_t(\eta, \lambda) - \gamma q_{u, t+1}\right\})^2.
\end{align}
Here $\cU$ stands for all the users in the training data, and $T$ implicitly depends on the choice of $u$. 

% As mentioned in Table~\ref{tab:modeling_differences}, however,
As mentioned in Section~\ref{sec:ddpg}, however, our action space is either combinatorially explosive ($10!$), or continuous $\R^d$. Thus it is unclear how to compute the supremum on the right hand side. Instead we simply drop the supremum operator and consider the following weakened Bellman equation
\begin{align} \label{eq:weak_bellman}
    q_{u, t} = r(\eta_{u, t}, \lambda_{u, t}) + \gamma q_{u, t + 1}, \quad q_{u, T} = 0.
\end{align}
The TD loss thus aims to minimize the sum-of-square error between the two sides of the equation above:
\begin{align} \label{eq:td_loss}
    \cL_{\text{TD}}(B_{u, t}, \lambda_{u, t}) := \sum_{u \in \cU} \sum_{t=1}^{T - 1} (q_{u, t} - r_{u, t} - \gamma q_{u, t + 1})^2.
\end{align}
The problem with the above weakened TD loss \eqref{eq:td_loss} is that by itself, it is under-specified. Indeed, $r_{u, t} = r(\eta_{u, t}, \lambda_{u, t})$ can take on any (negative) value without affecting $\cL_{\text{TD}}$, since the extra degrees of freedom in $q_{u, t}$ can easily compensate for its wild moves. By contrast, the original TD Loss (for DQN) \eqref{eq:dqn_td_loss} eliminates this extra degree of freedom by taking the supremum over all actions $\eta_{u, t}$. 

To make the training loss fully specified, we thus introduce a second loss term, the policy gradient (PG) loss, which seeks to maximize the cumulative Q-value over the RNN and critic network model parameters. 
\begin{align}
    \cL_{\text{PG}}(B_{u, t}, \lambda_{u, t}) := \sum_{u \in \cU} \sum_{t=1}^T q_{u, t}, \quad q_{u, t} = Q(O(B_{u, t})).
\end{align}
where recall $q_{u, t} = Q(O(B_{u, t, a}), O(B_{u, t, b}))$ for the chosen positive / negative item pair. Note that since the actor network also depends on the RNN network parameters, the PG loss also indirectly optimizes over the action space. Furthermore, since $q_{u, t}$ are very closely tied with the supervised reward function $r_t$, by maximizing $q_{u, t}$, we are implicitly also maximizing the original supervised reward.


% This prediction is then fed into a policy gradient loss, which takes into account the logged user feedback label, which in our case is the binary signal of whether the current item has been purchased. The exact policy gradient loss is given by the formula below
% \begin{align}
%     \rm{PGLoss} = \sum_{t=1}^T \gamma^t \mathbb{E} Q_t, \quad Q_t := Q(\omega_{u, t}, P_{u, t}) 
% \end{align}

% The critic network takes the RNN output and actor output (the scalar logit value) as its input, and tries to simulate the Q value associated with the Bellman equation, which is a vector, $\{Q_t: 1 \leq t \leq T\}$, indexed by the session ids. The resulting Q value output is then used to compute a temporal difference loss which measures the deviation from the exact Bellman equation.
% \begin{align}
%     \rm{TDLoss} = \sum_{t=1}^{T-1} (Q_t - \gamma Q_{t+1} - r_t)^2.
% \end{align}

As is standard in DQN and DDPG, we also add the so-called target Q-network \cite{mnih2015human}, denoted by $\tilde{Q}$, that differs from the original Q-network only by one time-step, which is useful for stabilizing its learning. In other words, the exact weight updates are given by,
\begin{align}
    Q &\leftarrow Q + \alpha \nabla_Q (\sum_{t = 1}^{T - 1} Q(\omega_t) - \gamma \tilde{Q}(\omega_{t+1}) - r_t) \\
    \tilde{Q} &\leftarrow Q ,
\end{align}
where $\alpha$ is the effective learning rate that depends on the actual 1st order optimizer used.


% TODOs:
% 1. Put a formula about target Q-network here.
% 2. Put a formula about GRU here.
% 3. Put a formula about 


% One distinguishing feature of DDPG from DQN is that besides Q-learning, there is also policy learning, expressed through the policy gradient loss. The latter tries to maximizes the expected discounted cumulative reward over available actions given the current state. This is naturally done through gradient descent in our ranking problem since the parameter space is continuous, where exact discrete maximization is highly infeasible.
% Due to the instability of optimizing both the target and the source in Q loss, we use two copies of the Q network, and periodically copy the source version (at $t$) to the target (at $t + 1$).

We have also tried two versions of the actor networks, but the difference in evaluation metrics is small (about 0.04\% in session AUC), thus was discarded for simplicity and training efficiency.

Another important way S3DDPG differs from traditional DDPG implmementation is the relation between the two losses and weight updates. In the original proposal \cite{lillicrap2015continuous}, the actor and critic network weights are updated separately by the PG and TD losses:
\begin{align*}
    P \leftarrow P + \alpha \nabla_P \cL_{\text{PG}}, \qquad Q \leftarrow Q + \alpha \nabla_Q \cL_{\text{TD}}.
\end{align*}
However we cannot get the model to converge under this gradient update schedule. Instead we simply take the sum of the two losses $\cL_{\text{PG}} + \cL_{\text{TD}}$, and update all the network weights according to
\begin{align*}
    O, P, Q \leftarrow \alpha \nabla_{O, P, Q} (\cL_{\text{PG}} + \cL_{\text{TD}}).
\end{align*}


\section{Online Incremental Update}
To capitalize on the underlying RNN modeling framework, we perform incremental update when the model is served online, so that the most recent user interactions can be captured by the model to update the user states. The overall architecture and its relation to offline training is summarized in Figure~\ref{fig:incremental_update}. The offline trained model can be divided into two sets of network parameters: 
\begin{itemize}
    \item The user state aggregation network takes the hidden states associated with all the items in the session, along with their corresponding labels, and perform average pooling to obtain a fixed size updated user state. If the session contains no purchase action, we do not update the user state.
    \item The remaining network take in the usual input features, along with the user state, to output predictions for each item.
\end{itemize}
The first of these is sent to an online incremental update component. While the latter goes directly to the neural network scorer.

The online serving component is roughly divided into three modules. At the center is the search engine itself, which is in charge of distributing and receiving features. 

When a user types in a query, the associated user context features, including query text, user's basic profile information, as well as user's historical actions, are all sent to the search engine. The search engine then relays this information to the neural network predictor, which in turn computes the predicted scores as well as the hidden state for each item, all of which are sent back to the search engine. Finally if the user makes any purchase in the current session, the new user state is updated to be the average of hidden states from the purchase items. 

\begin{figure}
    \centering
    \includegraphics[width=\linewidth]{DDPG-IncrementalUpdate.png}
    \caption{Real-Time Incremental Update Pipeline.}
    \label{fig:incremental_update}
\vspace{10pt}
\end{figure}

% TODOs:
% 1. add some diversity time series plot.
% 2. add UV value time series plot.
% 3. show some arena side by side to show diversity.
% 4. 

% \section{Analysis}
We look at users with long historical behavior sequence and found the DDPG model to be 

TODO: 添加类目新用户,老用户
\section{Experiment}
\label{sec:experiment}
\subsection{Evaluation Setup}
% \subsection{Setup}
% \label{sec:setup}
% Our primary evaluation is carried out on an in-house dataset collected from the 1 month of search log. This has the advantage of being 
% % We used one industrial scale in-house dataset and one public Amazon review dataset. The former has the advantage of being 
% directly reflected in online experiments, which is the most important way to validate our idea in an industrial setting. 
% The latter is better suited for reproducibility; furthermore we release our method to generate RL compatibility training and test data from the original public data. 
% We believe this will serve as a useful benchmark for other RL related studies in the e-commerce context.


% talk about details of the two datasets, evaluation metrics, training setup, parameters, implementation in Tensorflow, and so on.

\subsubsection{Training Data Generation}
We collect 30 days of training data from our in-house search log. Table~\ref{tab:in-house-data} summarizes its basic statistics. The total number of examples in DIN-S (pre-RNN) training is 200m, while under the RNN/S3DDPG data format, we have 6m variable length sessions instead. While the majority of users only have a single session, the number of sessions per user can go as high as 100. This makes our knapsack session packing algorithm~\ref{alg:knapsack_rnn} a key step towards efficient training. 
\begin{table}[htbp]
\centering
\caption{In-house data statistics.}
\small
\begin{tabular}{c|c|c|c}
\hline
statistics & mean & minimum & maximum \\
\hline
Number of unique users & 3788232 & - & - \\
\hline
sessions per user &	13.42 & 1 & 113 \\
\hline
items per session & 26.97 & 1 & 499 \\
\hline
Features per (query, item) & 110 & - & - \\
\hline
\end{tabular}
\label{tab:in-house-data}
\end{table}

% The variance of number of items across different query sessions is quite high. This is characteristic of e-commerce search, which resembles recommendation streams more than web search. 
A characteristic of e-commerce search sessions is the huge variance in browsing depth (number of items in a session). In fact, some power user can browse content up to thousands of items within a single session. 
The short sessions (such as the minimum number of 2 items in the table) are due to lack of relevant results. 



% \subsubsection{Training Data Generation}
% For the DIN baseline training, we generate pointwise dataset from $N = 30$ days of search log. 
Each DIN-S training example consists of a single query and a single item under the query session. To leverage users' historic sequence information, the data also includes the item id, category id, shop id, and brand id of the historical sequence of clicked / purchased / carted items by the current user. The sequence is truncated at a maximum length of 500 for online serving efficiency. 

For RNN and S3DDPG, each example consists of a pair of items under the same query. In order to keep the training data compact, i.e., without expanding all possible item pairs, the training data adopts the User Session Input format (Section~\ref{subsubsec:data_format}). To ensure all sessions under a user are contained within each minibatch, and ordered chronologically, the session data is further sorted by session id as primary key and session timestamp as secondary key during the data generation mapreduce job.


% To further ensure each minibatch contains entire history of a user, the data is constructed so that sessions under the same user appear contiguously in the data. This is achieved by relying on sorting by the primary query key and secondary timestamp key in the mapreduce reducer phase. 

During training, a random pair of items is sampled from each session, with one positive label (purchased) and one negative label (viewed/clicked only). Thus each minibatch consists of $\sum_{u=1}^B |S_u|$ item pairs, where $S_u$ stands for the set of all sessions under user $u$ and $B$ is the minibatch size, in terms of number of users. 

\subsubsection{Offline Evaluation}
We evaluate all models on one day of search log data beyond the training period. For RRNN and S3DDPG, however, 
% the evluation set also includes the previous 29 days of data, for a total of $N = 30$ days. This is 
% To evaluate the baseline DIN-S model, we simply take one day of search log data beyond the training period and construct an evaluation set similar to training, that is, each example consists of a query and an item, with binary label being whether or not the item was purchased under that query.
% For the RNN and S3DDPG evaluation, 
% we construct the evaluation set in the User Session Input format similar to training. In particular, all sessions under a single user are ordered chronologically and continguously within the data set. in order to capture the historical information,
we also include $N-1$ days prior to the last day, for a total of $N = 30$ days. The first $29$ days are there to build the user state vector only. Their labels are needed for user state aggregation during RNN forward evolution. Only labels from the last day sessions are used in the evaluation metrics, to prevent any leakage between training and validation.
% Labels from earlier days are needed only for aggregation of the user states during the RNN forward evolution. 

\subsubsection{Offline Evaluation Metrics}
While cross entropy loss \eqref{eq:reward_definition} and square loss \eqref{eq:dqn_td_loss} are used during training of S3DDPG, for hold-out evaluation, we aim to assess the ability of the model to generalize forward in time. Furthermore even though the training is performed on sampled item pairs, in actual online serving, the objective is to optimize ranking for an entire session worth of items, whose number of can reach the hundreds. Thus we mainly look at session-wise metrics such as Session AUC or NDCG. Session AUC in particular is used to decide early stopping of model training: 

\begin{align} \label{eq:session_auc}
    \text{Session AUC}(\eta, \lambda) := \sum_{u = 1}^B \sum_{t = 1}^{|S_u|} \rm{AUC}(\eta_{u, t}, \lambda_{u, t}),
\end{align}
where $\eta_{u, t}$ denotes the list of model predictions for \textbf{all items} within the session $(u, t)$ and $\lambda_{u, t}$ the corresponding binary item purchase labels. This is in contrast with training, where $\eta_{u, t}$, $\lambda_{u, t}$ denote predictions and labels for a randomly chosen positive / negative item pair. 

The following standard definition of ROC AUC is used in \eqref{eq:session_auc} above. For two vectors $\boldsymbol{p}, \boldsymbol{t} \in \R^n$, where $t_i \in \{0, 1\}$:
\begin{align*}
\rm{AUC}(\boldsymbol{p}, \boldsymbol{t}) := \frac{\sum_{1 \leq i < j \leq n} \sign(p_i - p_j) \sign(t_i - t_j)}{n(n-1) / 2},
\end{align*}
where $\sign(x) = x / |x|$ for $x \neq 0$ and $\sign(0) = 0$.

% We systematically extract a model checkpoint that attains the highest session AUC among all saved checkpoints.

NDCG is another popular metric in search ranking, intended to judge full page result relevance \cite{distinguishability2013theoretical}. It again takes the model predictions (which can be converted into ranking positions) as well as corresponding labels for all items, and compute a position-weighted average of the label, normalized by its maximal possible value:
\begin{align*}
    \rm{NDCG}(\boldsymbol{p}, \boldsymbol{t}) = \sum_{i=1}^n \frac{2^{t_i} - 1}{\log_2(i + 1)} / \sum_{i=1}^{\sum_j t_j} \frac{2^{t_i} - 1}{\log_2(i + 1)}.
\end{align*}

% where $\rm{DCG}$ and $\rm{IDCG}$ are in turn defined by
% \begin{align}
% \rm{DCG}(\boldsymbol{p}, \boldsymbol{t}) &:= \sum_{i=1}^n \frac{2^{t_i} - 1}{\log_2(i + 1)} \\
% \rm{IDCG}(\boldsymbol{p}, \boldsymbol{t}) &:= \sum_{i \leq n: t_i = 1} \frac{2^{t_i} - 1}{\log_2(i + 1)}.
% \end{align}

\subsubsection{Online Metrics}
For e-commerce search, there are essentially three types of core online metrics. 
\begin{itemize}
    \item GMV stands for gross merchandise value, which measures the total revenue generated by a platform. Due to the variation of A/B bucket sizes, it is often more instructive to consider GMV per user.
    \item CVR stands for conversion rate and essentially measures the number of purchases per click. Again this is averaged over the number of users.
    \item CTR is simply click-through rate, which measures number of clicks per query request. We do not consider this metric in our online experiments since it is not directly optimized by our models.
\end{itemize}
% \subsubsection{Online Serving}
% The ultimate objective of introducing the sequential search context as well as reinforcement learning architecture is to promote better user experience and personalized result relevance, measured primarily through a set of core metrics such as conversion rate and GMV. Therefore model online serving is a crucial component of the overall evaluation.

% The serving of the baseline DIN model is straightforward. First the model gets loaded into the system. Then upon a query request from a given user, the search engine computes all the required features for all retrieved items under the query, and sends them to the DIN model. The bulk of the feature preparation and transmission lies in the item features (either 1-sided or jointly 2-sided with query and user information), as well as the user behaviorial sequence features, which however are capped at the length of 500. Once the scores are computed for all items, they are sent back to the search engine and sorted to produce a ranking among the retrieved results.

% RNN and S3DDPG models are served essentially identically. Therefore most of the serving infrastructure (Figure~\ref{fig:incremental_update}) has been built when we launched the RNN model. There are two major differences with the DIN or non-RNN types of models:
% \begin{itemize}
%     \item In addition to the regular input features required by the model, RNN models also receive a numeric vector for the previous user state. 
%     \item This user vector is updated in real time after every user query request, so that when the same user comes back next time, he or she gets served a model with an updated state vector.
% \end{itemize}
% The updating of the user vector takes the following two inputs
% \begin{itemize}
%     \item A list of state vector candidates returned by the model, one for each item scored.
%     \item The user's behavioral feedback upon viewing the ranked results in the search page; for our purpose, which items got purchased, if any. 
% \end{itemize}
% If no item was purchased within the previous query session, the user state stays the same. Otherwise it gets updated with the candidate state vector corresponding to the purchased item.

\subsection{Evaluation Results}


\begin{table}[htbp] 
\centering
\caption{Offline Metrics}
\begin{tabular}{c|c|c}
\hline
Model name & Session AUC & NDCG \\
\hline
DNN & 0.6765 & 0.5104 \\
DIN-S & 0.6875 & 0.5200 \\
RNN & 0.6915 & 0.5272 \\
S3DDPG & 0.6968 & 0.5307 \\
\hline
\end{tabular}
\label{tab:offline_metrics}
\end{table}
We present both Session AUC and NDCG for the 4 models listed in Table~\ref{tab:offline_metrics}. The DNN baseline simply aggregates the user sequential features through sum-pooling, all of which are id embeddings. The successive improvements are consistent between the two session-wise metrics: RNN improved upon DIN-S by about $0.4\%$ in Session AUC and $0.7\%$ in NDCG, while S3DDPG further improves upon RNN by another $0.5\%$ in Session AUC and $0.4\%$ in NDCG. The overall gain of S3DDPG is around a full $1\%$ in either metrics from the DIN-S baseline, and $2\%$ from the DNN baseline.


Table~\ref{tab:user_group_evaluation} highlights the gain of S3DDPG over the myopic RNN baseline on a variety of user subsets. For instance, along the dimension of users' past session counts, S3DDPG shows a significantly stronger performance for more seasoned users in both validation metrics. Another interesting dimension is whether the current query belongs to a completely new category of shopping intent compared to the users' past search experience. Users who issue such queries in the evaluation set are labeled ``Category New Users". Along that dimension, we see that S3DDPG clearly benefits more than RNN from similar queries searched in the past.

While there are a number of hyperparameters associated with reinforcement learning models in general, the most important one is arguably the discount factor $\gamma$ parameter. We choose $\gamma = 0.8$ for all our S3DDPG experiments, since we found little improvement when switching to other $\gamma$ value. Another important parameter specific to actor-critic style architecture is the relative weight $\mu$ between PG loss and TD loss. Interestingly, as we shift weight from TD to PG loss (increasing $\mu$), there is a noticeable trend of improvement in both AUC and NDCG, as shown in Figure~\ref{fig:hyperparameter}. This suggests the effectiveness of maximizing the long term cumulative reward directly, even at the expense of less strict enforcement of the Bellman equation through the TD loss. When $\mu$ is set to $1$, however, training degenerates, as the Q-value optimized by PG loss is not bound to the actual reward (cross entropy loss) any more.

\begin{figure}
    \centering
    %\includegraphics[width=0.9\linewidth]{SessionAUC_µ.png}
    \includegraphics[width=\linewidth]{mu.png}
    \caption{influence of hyperparameter $\mu$}
    \label{fig:hyperparameter}
\end{figure}

Finally we conduct 3 sets of online A/B tests, each over a timespan of 2 weeks. The overall metric improvements are reported in Table~\ref{tab:online_metrics}. The massive gain from DNN to DIN-S is expected, since the DNN baseline, with sum-pooling of the sequential features, is highly ineffective at using the rich source of sequential data. Nonetheless we also see modest to large improvements between RNN and DIN-S and between S3DDPG and RNN respectively, in all core business metrics. Figures~\ref{fig:daily_ucvr} present the daily UCVR metric comparison for the last 2 sets of A/B tests.  Aside from a single day of traffic variation, both RNN and S3DDPG show consistent improvement over their respective baselines. 
\begin{figure}
    \centering
%    \includegraphics[width=\linewidth]{UCVR-S3DDPG-RNN.png}
%     \caption{UCVR (conversion rate/unique user) delta in \% from DIN-S and RNN models over 14 days.}
%     \label{fig:rnn_v_DIN-S}
% \end{figure}
% \begin{figure}
%     \centering
    \includegraphics[width=\linewidth]{UCVR_ONLINE.png}
    \caption{Daily UCVR \% improvement for online A/B tests over 14 days.}
    \label{fig:daily_ucvr}
\end{figure}
% The gain from DIN to RNN is relatively easy to understand: by giving the model the full spectrum of input features from previous sessions in the near-term (30 days), RNN is equipped with much more context information about the user than the DIN model, which only gets a few categorical features of the past interacted items. 
% % The gain from RNN to S3DDPG is much more elusive. After all, they both have the same set of input features, and RNN, with its GRU kernel, is already a time-proven method for dealing with sequential data. Furthermore, the vast majority of users in the evaluation set only have a single session on the last day.

% The offline metrics gain from RNN to S3DDPG mainly comes from the removal of a weighting scheme used in both DIN and RNN, intended to promote items with higher prices, since besides user conversion rate, another important metric is GMV (gross merchandise volume), which measure the the total revenue generated by users' shopping activities on the search platform. While the price weighting scheme was very helpful in boosting GMV per user for the baseline DIN model as well as the RNN model, we found in online experiments that removing it did not cause any negative effect on GMV/user. Quite surprisingly, it in fact led to huge boost of both GMV/user as well as the closely related RPM (revenue-per-mil) metrics, both of which increased by nearly $2\%$ in relative term compared to RNN with price weighting, at a significance level of p-value $\leq 1\%$. 

% In fact, the trick lies in a technical detail mentioned earlier in Section~\ref{sec:method}. Namely, for both DIN and RNN, we introduced a weighting scheme to promote higher priced items within the search results. This was effective in boosting GMV and related revenue metrics, without sacrificing too much on click-through and conversion rate. For S3DDPG, however, this was removed. The hope was that reinforcement learning with its emphasis on long term reward is able to bring in more user conversion that would offset the short-term loss in per item revenue.

% Indeed from the online metrics Table~\ref{tab:online_metrics}, we see highly significant gains in GMV / user 


% As mentioned in Section~\ref{subsec:rnn:pairwise}, 

% Experiments to report:
% 1. DIN
% 2. RNN
% 3. DDPG:
% 3a. no target Q
% 3b. target Q
% 4. DQN: not convergent
% 5. 4 reward experiments
% 6. Analysis: long/short validation set
% 7. off-policy TopK
% 8. gamma grid search
% 9. TD + PG no tuning
% 10. target actor 



% \begin{table}[htbp]
% \centering
% \caption{Hyperparameter µ}
% \begin{tabular}{c|c|c}
% \hline
% Parameter & Session AUC & NDCG \\
% \hline
% 0 & -- & -- \\
% \hline
% 0.1 & 0.6887 & 0.5226 \\
% \hline
% 0.2 & 0.6946 & 0.5284 \\
% \hline
% 0.3 & 0.6939 & 0.5281 \\
% \hline
% 0.4 & 0.6941 & 0.5281 \\
% \hline
% 0.5 & 0.6968 & 0.5307 \\
% \hline
% 0.6 & 0.6959 & 0.5307 \\
% \hline
% 0.7 & 0.6970 & 0.5304 \\
% \hline
% 0.8 & 0.6979 & 0.5311 \\
% \hline
% 0.9 & 0.6985 & 0.5329 \\
% \hline
% 1.0 & -- & -- \\
% \hline
% \end{tabular}
% \end{table}

% \begin{table}[htbp] \label{tab:offline_metrics}
% \centering
% \caption{S3DDPG vs RNN for Category New Users}
% \begin{tabular}{c|c|c|c}
% \hline
% Data Type & Model pairs & Session AUC & NDCG \\
% \hline
% New User & S3DDPG vs RNN & +0.3195\% & +0.1412\% \\
% \hline
% Old User & S3DDPG vs RNN & +0.6584\% & +0.6688\%\\
% \hline
% \end{tabular}
% \end{table}

\begin{table}[htbp]
\centering
\caption{S3DDPG vs RNN for different user evaluation groups}
\begin{tabular}{c|c|c}
\hline
User Group  & Session AUC & NDCG \\
\hline
Past Session Count < 5  & +0.6534\% & +0.2418\% \\
Past Session Count >= 5  & +0.9544\% & +0.9092\%\\
Category New Users & +0.3195\% & +0.1412\% \\
Category Old Users & +0.6584\% & +0.6688\% \\
\hline
\end{tabular}
 \label{tab:user_group_evaluation}
\end{table}

\begin{table}[htbp]
\centering
\caption{Online Metrics}
\begin{tabular}{c|c|c|c}
\hline
Model pairs & GMV/user & CVR/user & RPM \\
\hline
DIN-S vs DNN & +4.05\%(1e-3) & +3.51\%(1e-3) & +4.08\%(5e-3) \\
RNN vs DIN-S & +0.60\%(5e-3) & +1.58\%(9e-3) & +0.49\%(8e-3) \\
S3DDPG vs RNN & +1.91\%(8e-3) & +0.78\%(1e-2) & +1.94\%(9e-3) \\
\hline
\end{tabular}
 \label{tab:online_metrics}
\end{table}
\vspace{-10pt}
% \begin{itemize}
% \item 离线实验指标
% \begin{itemize}
% \item RNN vs. DNN
% \begin{itemize}
% \item auc +0.94\%
% \end{itemize}
% \item DQN(gamma=0.8) vs. DNN
% \begin{itemize}
% \item auc +1.51\%
% \end{itemize}
% \end{itemize}
% \begin{itemize}
% \item DQN(gamma=0.8) vs. RNN
% \begin{itemize}
% \item auc +0.56\%
% \item results for other gamma 
% % (see Diagram~\ref{dia:gamma_sweep}
% \end{itemize}
% \end{itemize}
% \end{itemize}
% 在线实验指标
% \begin{itemize}
% \item RNN vs. DNN(线上观察7天)
% \begin{itemize}
% \item UV价值(搜索GMV/搜索用户) : +2.40\%
% \item UCVR(订单行/搜索用户): +3.09\%
% \item GMV: +1.99\%
% \item RPM(1000*搜索GMV/搜索量): +2.54\%
% \end{itemize}
% \item DDPG vs. RNN(线上观察7天)
% \begin{itemize}
% \item UV价值:+1.91\%
% \item UCVR:+0.78\%
% \item GMV:+1.77\%
% \item RPM:+1.94\%
% \item 流动性:+4.24\%
% \item 流动性: 使用rbo距离计算每个query今天的昨天TOP20 SKU排序的diff率,多天取平均。
% \item CID4基尼不纯度:0.95\%
% \item CID4基尼不纯度:TOP60坑位有曝光的CID4的gini不纯度
% \end{itemize}
% \item AB实验指标
% \end{itemize}


\label{sec:industrial}

% \subsection{Public Dataset Experiment}
% \label{sec:public}
% We take the Amazon review dataset as our starting point. Each reviewer is identified through his or her reviewer id. Since every review event contains a timestamp, we can naturally form a temporal sequence. Unlike our in-house dataset, however, there is no analogue of a session (with multiple items) in the dataset. Instead, we simply consider the sequence of review events. To take advantage of full recurrence as proposed, we include expensive features such as item titles, past reviews, etc, in addition to asin\_id (item id) and category names.

% \subsection{Case Study}
\label{sec:case}



\section{Conclusion}
In this study, we take the first step to systematically explore multi-level feature fusion for the isotropic architecture, such as ViT, in masked image modeling. Initially, we recognize that pixel-based MIM approaches tend to excessively rely on low-level features from shallow layers to complete the pixel value reconstruction task by a pilot experiment. We then apply a simple and intuitive multi-level feature fusion to two pixel-based MIM approaches, MAE and PixMIM, and observe significant improvements in both, gradually closing the performance gap with these approaches by using an extra heavy tokenizer. Finally, we conduct an extensive analysis of multi-level feature fusion and find that it can suppress high-frequency information and flatten the loss landscape. We believe that this work can provide the community with a fresh perspective on these pixel-based MIM approaches and continue to rejuvenate this kind of simple and efficient self-supervised learning paradigm.


\bibliographystyle{ACM-Reference-Format}
\balance
\bibliography{references}
\citestyle{acmauthoryear}




\end{document}
\endinput
%%
%% End of file `sample-authordraft.tex'.
