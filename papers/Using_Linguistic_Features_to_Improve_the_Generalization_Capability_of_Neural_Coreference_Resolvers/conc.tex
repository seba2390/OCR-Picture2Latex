\section{Conclusions}
In this paper, we show that employing linguistic features 
in a neural coreference resolver significantly improves generalization.
However, the incorporated features should be informative enough 
to be taken into account in the presence of lexical features, 
which are very strong features in the CoNLL dataset.
We propose an efficient algorithm to determine informative feature-values in large datasets.
As a result of a better generalization, we achieve state-of-the-art results in all examined out-of-domain evaluations.
%The use of EPM for selecting informative feature-values for other tasks is a future direction. 
% The use of linguistic features in state-of-the-art NLP systems 
% does not have a consistent effect on generalization,
% e.g.\ \newcite{wang-EtAl17} show that linguistic features improve the generalization for sentence compression, 
% while \newcite{marcheggiani-titov17} show that incorporating syntax makes their approach less generalizable
% than its syntactic-agnostic counterpart.
% It is worth investigating whether this is because of linguistic features themselves, or the way that we incorporate them.
