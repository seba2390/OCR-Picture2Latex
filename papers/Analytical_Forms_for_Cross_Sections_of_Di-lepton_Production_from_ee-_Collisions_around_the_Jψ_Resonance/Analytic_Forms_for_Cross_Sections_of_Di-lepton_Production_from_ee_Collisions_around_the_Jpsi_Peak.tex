\documentclass[a4paper,10pt,twoside]{cpc-hepnp}
\usepackage{CJK,upgreek,fancyhdr}
\usepackage{multicol}
\usepackage{graphicx}
\usepackage{booktabs}
\usepackage{amssymb,bm,mathrsfs,bbm,amscd}
\usepackage[tbtags]{amsmath}
\usepackage{lastpage}
\usepackage{cases}
\usepackage{subfigure}
\usepackage{epstopdf}
\usepackage{bm}

%\usepackage[pdftex,colorlinks,linkcolor=black,anchorcolor=black,citecolor=black,urlcolor=black]{hyperref}
%Revised for arXiv
\usepackage[colorlinks,linkcolor=black,anchorcolor=black,citecolor=black,urlcolor=black]{hyperref}
%\input{My_commands/My_commands}
%frequently-used particles

\newcommand{\jpsi}{$J/\psi$~}
\newcommand{\jpsiwos}{$J/\psi$}
\newcommand{\psip}{$\psi(3686)$~}
\newcommand{\psipwos}{$\psi(3686)$}

%frequently-used channels
\newcommand{\eetoee}{$e^+e^- \to e^+e^-$~}
\newcommand{\eetomumu}{$e^+e^- \to \mu^+\mu^-$~}
\newcommand{\eetohad}{$e^+e^- \to hadrons$~}
\newcommand{\eetodigam}{$e^+e^- \to \gamma\gamma$~}
\newcommand{\eetotwogam}{$e^+e^- \to e^+e^-X$~}
\newcommand{\eetoll}{$e^+e^- \to l^+l^-$~}
\newcommand{\eetomumugam}{$e^+e^- \to \mu^+\mu^-\gamma$~}
\newcommand{\eetoeewos}{$e^+e^- \to e^+e^-$}
\newcommand{\eetomumuwos}{$e^+e^- \to \mu^+\mu^-$}
\newcommand{\eetohadwos}{$e^+e^- \to hadrons$}
\newcommand{\eetodigamwos}{$e^+e^- \to \gamma\gamma$}
\newcommand{\eetotwogamwos}{$e^+e^- \to e^+e^-X$}
\newcommand{\eetollwos}{$e^+e^- \to l^+l^-$}
\newcommand{\eetomumugamwos}{$e^+e^- \to \mu^+\mu^-\gamma$}

%frequently-used branching ratios
\newcommand{\Bjpsitoll}{B(J/\psi \to l^+l^-)}

%frequently-used parameters
\newcommand{\Tw}{\Gamma_{\rm tot}}
\newcommand{\Ew}{\Gamma_{ee}}
\newcommand{\Mw}{\Gamma_{\mu\mu}}
\newcommand{\Lw}{\Gamma_{ll}}
\newcommand{\EwMMw}{\Gamma_{ee}\Gamma_{\mu\mu}}
\newcommand{\EwDMw}{\Gamma_{ee}/\Gamma_{\mu\mu}}
\newcommand{\LwDTw}{\Gamma_{ll}/\Gamma_{\rm tot}}
\newcommand{\EwMEwDTw}{\Gamma_{ee}\Gamma_{ee}/\Gamma_{\rm tot}}
\newcommand{\EwMMwDTw}{\Gamma_{ee}\Gamma_{\mu\mu}/\Gamma_{\rm tot}}
\newcommand{\LwMLwDTw}{\Gamma_{ll}\Gamma_{ll}/\Gamma_{\rm tot}}
\newcommand{\Es}{\sigma}

%frequently-used math expressions
\newcommand{\csth}{\cos\theta}
\newcommand{\snth}{\sin\theta}


\begin{document}
\begin{CJK*}{UTF8}{gbsn}

%\input{Fancyhead_fancyfoot_footnotetext/Fancyhead_fancyfoot_footnotetext}
%\fancyhead[c]{\small Chinese Physics C~~~Vol. xx, No. x (201x) xxxxxx}
%Revised for arXiv
\fancyhead[c]{\small Accepted by Chinese Physics C}
\fancyfoot[C]{\small xxxxxx-\thepage}
%\footnotetext[0]{Received xx Month xxxx}
%Revised for arXiv
\footnotetext[0]{}

%\input{Title_author_address/Title_author_address}
\title{Analytic Forms for Cross Sections of Di-lepton Production \\ from $e^+e^-$ Collisions around the \jpsi Resonance \thanks{Supported by National Natural Science Foundation of China (11275211) and Istituto Nazionale di Fisica Nucleare, Italy}}

\author{Xing-Yu Zhou (周兴玉)$^{1;1)}$ \email{xyzhou@ihep.ac.cn} \quad Ya-Di Wang (王雅迪)$^{2;2)}$ \email{Y.Wang@him.uni-mainz.de} \quad Li-Gang Xia (夏力钢)$^{3;3)}$ \email{xialigang@tsinghua.edu.cn}}

\maketitle

\address
{
	$^1$ Institute of High Energy Physics, Chinese Academy of Sciences, Beijing 100049, China \\
	$^2$ Helmholtz Institute Mainz, Mainz 55128, Germany \\
	$^3$ Department of Physics, Tsinghua University, Beijing 100084, China
}


%\input{Abstract_keyword_pacs_footnotetext/Abstract_keyword_pacs_footnotetext}
\begin{abstract}
	A detailed theoretical derivation of the cross sections of \eetoee and \eetomumu around the \jpsi resonance is reported. The resonance and interference parts of the cross sections, related to \jpsi resonance parameters, are calculated. Higher-order corrections for vacuum polarization and initial-state radiation are considered. An arbitrary upper limit of radiative correction integration is involved. Full and simplified versions of analytic formulae are given with precision at the level of 0.1\% and 0.2\%, respectively. Moreover, the results obtained in the paper can be applied to the case of the \psip resonance.
\end{abstract}

\begin{keyword}
	initial-state radiation, vacuum polarization, $e^+e^-$ collision, di-lepton production, the \jpsi resonance
\end{keyword}

\begin{pacs}
	13.20.Gd, 13.66.De, 13.66.Jn, 14.40.Pq, 13.40.Hq
\end{pacs}

%The last three lines must not be deleted. They are needed by the CPC template.
%\footnotetext[0]{\hspace*{-3mm}\raisebox{0.3ex}{$\scriptstyle\copyright$}2013 Chinese Physical Society and the Institute of High Energy Physics of the Chinese Academy of Sciences and the Institute of Modern Physics of the Chinese Academy of Sciences and IOP Publishing Ltd}
%Revised out for arXiv
\footnotetext[0]{\ \\}

\begin{multicols}{2}

\section{Introduction}
%\section{Introduction}
\label{sec:Introduction}


The goal in top-$\size$ recommendation is to recommend to each
consumer a small set of $\size$ items from a large collection of
items~\cite{cremonesi2010performance}.  For example, Netflix may want
to recommend $\size$ appealing movies to each consumer.  Collaborative
Filtering (CF)~\cite{herlocker2002empirical,lee2012comparative} is a
common top-$\size$ recommendation method.  CF infers user interests by
analyzing partially observed user-item interaction data, such as user
ratings on movies or historical purchase
logs~\cite{kanagal2012supercharging}. The main assumption in CF is that
users with similar interaction patterns have similar interests.


Standard CF methods for top-$\size$ recommendation focus on making  suggestions  that accurately reflect the user's preference history. However, as  observed in previous work,  CF recommendations are generally biased toward  popular items, leading to a rich get richer effect~\cite{vargas2014improving,steck2011item}.  The major reasons for this are \textit{popularity bias} and \textit{sparsity} of CF interaction data (detailed in Section~\ref{sec:related-work}). In a nutshell, to maintain  accuracy, recommendations are generated from the dense regions of the data,  where the popular items lie.  

However,  accurately suggesting popular items, may not be satisfactory for the consumers. For example, in Netflix, an accuracy-focused movie recommender may recommend ``Star Wars: The Force Awakens'' to users who have seen ``Star Wars: Rogue One''.  But, those users are probably already aware of ``The Force Awakens''. Considering additional factors, such as novelty of recommendations,  can lead to more effective suggestions~\cite{cremonesi2010performance,Castells2015,zhang2008avoiding,ziegler2005improving,zhang2012auralist}. 
%Second, accuracy-focused models typically achieve a   overall item-space coverage across their recommendations,  whereas high item-space coverage helps providers of the items increase revenue
%, users satisfaction since they are  likely already aware of or can find these items on their own.  

Focusing on popular items also adversely affects the satisfaction of  the providers of the items. This is because  accuracy-focused models typically achieve a  low overall item space coverage across their recommendations, whereas   high item space coverage helps providers of the items increase their revenue~\cite{vargas2014improving,Castells2015,adomavicius2011maximizing,anderson2006thelongtail, yin2012challenging,adomavicius2012improving}.
%accuracy-focused models typically achieve a

In contrast to the relatively small number of popular items, there are copious  {\it long-tail\/} items that have fewer observations (e.g., ratings) available. More precisely,  using the Pareto  principle (i.e.,~the $80/20$ rule),  long-tail items can be defined as items that generate the lower $20\%$ of observations~\cite{yin2012challenging}. Experimentally we found that these items correspond to almost $85\%$ of the items in several datasets (Sections~\ref{sec:Notation} and \ref{sec:Experiments}). %Table~\ref{tab:DatasetStatsticsSmall})


As previously shown, one way to improve the novelty of top-$\size$ sets is to recommend interesting long-tail items~\cite{cremonesi2010performance,ge2010beyond}.  The intuition  is that since they have fewer observations available,  they are more likely to be unseen~\cite{Kaminskas:2016:DSN:3028254.2926720}.  
 %For example, in online commerce,  newly added items are long-tail items that are yet to be discovered.  
Moreover, long-tail item promotion also results in higher overall coverage of the item space%, which increases profits for providers of the items
~\cite{vargas2014improving,Castells2015,zhang2008avoiding,zhang2012auralist,adomavicius2011maximizing,anderson2006thelongtail,yin2012challenging,jambor2010optimizing}. Because long-tail promotion reduces accuracy~\cite{steck2011item}, there are trade-offs to be explored.


%original submitted to ICDE
%This work studies three aspects of top-$\size$ recommendation: accuracy, novelty, and item-space coverage, and examines their trade-offs. In most previous work, predictions of a base recommendation system are re-ranked to handle their trade-offs~\cite{adomavicius2012improving,jambor2010optimizing,zhang2013personalize,wang2009portfolio}. Due to performance considerations, however, these techniques are not customized per user. For example,  parameters that balance the trade-off between novelty and accuracy are cross-validated at a global level.  This can be detrimental since users have varying preferences for  objectives such as long-tail novelty. We explore how to  automatically infer  user  preference for long-tail novelty, and how to leverage  it to correct  the popularity bias in standard recommender models. Our work does not rely on any additional contextual data, although such data, if available, can help promote newly-added long-tail items~\cite{agarwal2009regression,Saveski:2014:ICR:2645710.2645751}.

This work studies three aspects of top-$\size$ recommendation: accuracy, novelty, and item space coverage, and examines their trade-offs. In most previous work, predictions of a base recommendation algorithm are \textit{re-ranked} to handle these trade-offs~\cite{adomavicius2012improving,jambor2010optimizing,zhang2013personalize,wang2009portfolio}. The re-ranking models are computationally efficient but suffer from two drawbacks. First, due to performance considerations,  parameters that balance the trade-off between novelty and accuracy  are not customized per user. Instead they are cross-validated at a global level.  This can be detrimental since users have varying preferences for  objectives such as long-tail novelty. Second,  the re-ranking methods are often limited to a specific base recommender  that may be sensitive to dataset density. 
As a result, the datasets are pruned and the problem is studied in dense settings~\cite{adomavicius2012improving,ho2014likes}; but real world  scenarios are often sparse~\cite{kanagal2012supercharging,liu2017experimental}.   
% Because  dataset density can impact the performance of most base recommenders (like R-SVD), which in turn affects the performance of the re-ranking model, 

\iffalse
We address these limitations by directly inferring  user  preference for long-tail novelty  from interaction data.  This  allows us to customize the re-ranking  per user, and design a \textit{generic} framework, which resolves the second problem. In particular, since the long-tail novelty preferences are estimated independently of any base  recommender model, we can  plug-in an appropriate base recommender w.r.t. the dataset sparsity.% including ones that are more suitable for sparse settings.  

Modelling  user  preference for  long-tail novelty using only item popularity statistics, e.g., the average popularity of rated items as in~\cite{jugovac2017efficient}, disregards additional information like whether the user found the item interesting and the long-tail preferences of other users  of the items. \iffalse To incorporate them, we introduce the notion of  \emph{item long-tail importance}. Both  user long-tail preferences and item long-tail importance are dependent:  a user has high preference for discovering long-tail items if she is interested in important long-tail items, and an item that is associated with many of these kinds of users is likely to be more important.  We propose a joint optimization framework to directly learn,  from interaction data, both the users' long-tail preferences and the  items' long-tail importance. \fi
We propose an optimization approach that  incorporates  this information and  directly learns,  from interaction data, the users' long-tail novelty preferences.

Next, we use these learned preferences  to design a  top-$\size$ recommendation framework thats is generic, and provides customized balance between accuracy, novelty, and coverage. We refer to it as framework as GANC.  Using GANC, we design a novel algorithm, {\it Ordered Sampling-based Locally Greedy (OSLG)\/}, that relies on the learned long-tail novelty preferences  to scalably correct for popularity bias. Our work does not rely on any additional contextual data, although such data, if available, can help promote newly-added long-tail items~\cite{agarwal2009regression,Saveski:2014:ICR:2645710.2645751}. In summary:
\fi

We address the first limitation by directly inferring  user  preference for long-tail novelty  from interaction data.   Estimating these  preferences  using only item popularity statistics, e.g., the average popularity of rated items as in~\cite{jugovac2017efficient}, disregards additional information, like whether the user found the item interesting or the long-tail preferences of other users  of the items. We propose an approach that  incorporates  this information and  learns the users' long-tail novelty preferences from interaction data.

This approach allows us to customize the re-ranking  per user, and  design a \textit{generic} re-ranking framework, which resolves the second limitation of prior work. In particular, since the long-tail novelty preferences are estimated independently of any base recommender, we can  plug-in an appropriate one w.r.t. different factors, such as the dataset sparsity.

Our top-$\size$ recommendation framework, \textbf{GANC}, is \textbf{G}eneric, and provides customized balance between \textbf{A}ccuracy, \textbf{N}ovelty, and \textbf{C}overage. % Moreover, based on the learned long-tail novelty preferences, we also design a novel algorithm, {\it Ordered Sampling-based Locally Greedy (OSLG)\/}, that relies on the learned long-tail novelty preferences  to scalably correct for popularity bias. 
Our work does not rely on any additional contextual data, although such data, if available, can help promote newly-added long-tail items~\cite{agarwal2009regression,Saveski:2014:ICR:2645710.2645751}. In summary:

%Consider  the following toy example:
\vspace{-0.2cm}
\begin{table}[htb]
\centering
\scriptsize
%\small
\begin{tabular}{ccccccc} 
%\toprule
%&\multirow{2}{*}{}&\multicolumn{7}{c}{Ratings}\\
& & \cellcolor{blue!35}$w_1$ &\cellcolor{blue!18} $w_2$ & $\dots$ &\cellcolor{blue!8} $w_{89}$  &\cellcolor{blue!8} $w_{99}$   
\\
&   &$i_1$&$i_2$&$\dots$&$i_{89}$&$i_{90}$\\ 
\cmidrule(r){3-7} 	 
%\midrule
\cellcolor{red!35}$\theta_1$  &$u_1 $   &5 &   & $\dots$ &  &   \\
\cellcolor{red!28}$\theta_2$  &$u_2$     &5 &    & $\dots$ &  &  \\
 $\theta_3=?$  &$\bf u_3$  &5 &  &   $\dots$ &  &  \\
\cellcolor{red!10}$\theta_4$ & $u_4$  &  &5   & $\dots$ & &\\ 
\cellcolor{red!10}$\theta_5$ & $u_5$  &  & 5  & $\dots$ & &\\ 
$\theta_6=?$  & $\bf u_6$ & &5  &      $\dots$& &  \\ 
 & & $\hdots$  &$\hdots$   &$\hdots$   &$\hdots$   &$\hdots$  \\
%\midrule 
\cmidrule(r){3-7} 	 
\multicolumn{2}{c}{item pop.}  & 3  & 3  & $\dots$ &50&60\\  
%\bottomrule
%$ f_i$    &3  &3  &1  &3  &1  &2  \\  \hline
\end{tabular}
%#.
\caption{Simplified user-item interaction data. The user long-tail novelty preference ($\theta_u$), item long-tail importance weight ($w_i$) are highlighted. Darker colors indicate larger values. } \label{tab:example}
\end{table} 
\vspace{-0.2cm}
\begin{example}  
In Table~\ref{tab:example}, we are interested in estimating $\theta_3$ and $\theta_6$,  the long-tail preference of users $u_3$ and $u_6$ who have each rated a single movie. Additional ratings for other users  are not included here.  Considering only rating information, we observe $i_1$ and $i_2$ are  equally popular $|\mathcal{U}_{i_1}^{\trainset}| = |\mathcal{U}_{i_2}^{\trainset}|=3$, and $r_{31}=5$ and $r_{62}=5$. Using Eq.~\ref{eq:tfidf-risk}  we have $\theta_3 = \theta_6$. However, if we were given the long-tail preferences of the each item's user set, specifically that $u_1$ and $u_2$ have high long-tail preference (darker red), while $u_4$ and $u_5$ have lower long-tail preference (lighter red), we could conclude $i_1$ is a more important long-tail item compared to $i_2$ (indicated by a darker blue shade for $w_1$), and we expect  $\theta_3 \geq \theta_6$.

% On the other hand, if we knew that $u_4$ and $u_5$ have lower long-tail preference, we could conclude $i_2$ is a  less significant long-tail item. Therefore, However, if we  consider the long-tail preferences of other users, we may reason differently.    We need another variable $w_i$ which captures this information. 
%we would conclude that $u_3$ has higher long-tail preference compared to $u_6$, since the users $i_1$ is a more prominent long-tail item. 

% Relying only  on item popularity information, we would  conclude   $u_3$ and $u_6$ have equal long-tail preference, since $i_1$ and $i_2$ are  equally popular. However, considering  the second column,  long-tail preference of users,  long-tail importance for each item,  which captures the long-tail preference of its users. Since  that  both users of $i_1$ have high long-tail preference while  the users of $i_2$ have lower preference,  we may conclude $i_1$ is a more important long-tail item compared to $i_2$. Therefore, $u_3$'s long-tail preference should be at least as large as $u_6$'s preference. Specifically, consider two  items $i_1$ and $i_2$, with the following rating data: $i_1=\{u_1:5, u_2:5, u_3:5 \}$, $i_2=\{u_4:5, u_5:5, u_6:5\}$.  

%Table~\ref{tab:example} shows  simplified rating data. We want an estimate of the long-tail preference of $u_3$ and $u_6$, who have each  rated a single movie.  Relying only  on movie popularity information, we would  conclude   $u_3$ and $u_6$ have similar long-tail preference, since $m_1$ and $m_2$ are  equally popular. However, considering the long-tail preferences of other users of those movies, we may reason differently: since $u_1$ and $u_2$ have high long-tail preference, and $u_4$ and $u_5$ have low long-tail preference, $m_1$ is a more prominent long-tail item compared to $m_2$. Therefore, it is likely that $u_3$ has higher long-tail preference compared to $u_6$.considering the long-tail preferences of other users of those movies, we may reason differently.  For example, 
\label{ex:running}
\end{example}



%------------------------------

\iffalse
\begin{example}
Table~\ref{tab:example} shows rating data for a simplified system. %Note the user-item interaction matrix is sparse.
For this example, we define popular movies as those that have received  three or more ratings; $\{m_1, m_2, m_4\}$ are popular and  $\{m_3, m_5, m_6\}$ are niche movies. We observe $u_1$ and $u_3$  have rated relatively popular movies (risk-averse) while $u_2$ and $u_4$ have rated niche movies (risk-loving). 
\label{ex:running}
\end{example}

\begin{table}[htb]
\centering
\scriptsize
\begin{tabular}{ccccccc} 
\toprule
			&$m_1$ &$m_2$   &$m_3$    &$m_4$   &$m_5$ &$m_6$  \\ \hline 
$u_1 $ &5  &4  & - &-  &-  &-   \\
$u_2$  &-  &-  &-  &-  &5  &5   \\
$u_3$  &-  &4  &-  &5  &-  &-   \\
$u_4$  &-  &-  &3  &-  &-  &4   \\ 
$u_5$  &5  &-  &-  &3  &-  &-   \\ 
$u_6$  &4  &2  &-  &4  &-  &-   \\ 
\bottomrule
%$ f_i$    &3  &3  &1  &3  &1  &2  \\  \hline
\end{tabular}
\caption{User-Movie rating data} \label{tab:example}
\end{table}

It is essential to consider consumer characteristics in designing recommender systems so that they promote long-tail items to the right group of users and spread demand evenly between hit and niche items.  

\fi





%------------------------------
\iffalse
\begin{table}[htb]
\centering
\scriptsize
\begin{tabular}{ccccccc} 
\toprule
			&$m_1$ &$m_2$   &$m_3$    &$m_4$   &$m_5$ &$m_6$  \\ \hline 
$u_1 $ &\textbf{5}  & \textbf{4}  &\textcolor{gray}{ 1.2} &-  &-  &-   \\
$u_2$  &-  &-  &-  &-  & \textbf{5}  &\textbf{5}   \\
$u_3$  &-  &\textbf{4}  &-  &\textbf{5}  &-  &-   \\
$u_4$  &-  &-  &\textbf{3}  &-  &-  &\textbf{4}   \\ 
$u_5$  &\textbf{5}  &-  &-  &\textbf{3}  &-  &-   \\ 
$u_6$  &\textbf{4}  &\textbf{2}  &-  &\textbf{4}  &-  &-   \\ 
\bottomrule
%$ f_i$    &3  &3  &1  &3  &1  &2  \\  \hline
\end{tabular}
\caption{User-Movie rating data} \label{tab:example}
\end{table}
% $\mathcal{P}^1= \{ \mathcal{P}_1^1 \{i_1,i_2,i_3\}, \mathcal{P}_2^1:\{i_2,i_3,i_5\}  \}$
 %$\mathcal{P}^2= \{ \mathcal{P}_1^2: \{i_1,i_2,i_3\}, \mathcal{P}_2^2:\{i_2,i_5,i_6\}  \}$
 %$\mathcal{P}^3= \{ \mathcal{P}_1^3: \{i_7,i_8,i_9\}, \mathcal{P}_2^3:\{i_{10},i_{11},i_{12}\}  \}$
\begin{table}[htb]
\centering
\tiny
\begin{tabular}{ccc} 
\toprule
		&$u_1$&$u_2$  \\ \hline 
$\mathcal{P}^1 $ & $\{i_1,i_2,i_3\}$ & $\{i_2,i_3,i_5\} $ \\
$\mathcal{P}^2$ & $\{i_1,i_2,i_3\}$ & $\{i_2,i_5,i_6\} $ \\
$\mathcal{P}^3$ & $\{i_7,i_8,i_9\}$ & $\{i_{10},i_{11},i_{12} \}$ \\
\bottomrule
%$ f_i$    &3  &3  &1  &3  &1  &2  \\  \hline
\end{tabular}
\caption{Top-$\size$ allocations to users.} \label{tab:paretoExamples}
\end{table}
\fi


\iffalse
When considering long-tail items, it is important to consider consumers' willingness  to explore niche or unpopular items and their propensity towards similar items. In particular, they can be characterized by their  {\it risk degree\/} and {\it focusing degree\/}, respectively.  We compute these estimates  based on historical rating information. The following example further describes these notions in the context of movie rating data. 

\begin{example}  
Table~\ref{tab:example} shows rating data for a simplified system with $6$ users, $6$ movies, and $3$ genres. $m_i^{j}$ implies that movie $m_i$ belongs to genre $j$. Note the user-item interaction matrix is sparse. 
  For this setting, we define popular movies as those that have received  three or more ratings; $\{m_1, m_2, m_4\}$ are popular and  $\{m_3, m_5, m_6\}$ are niche movies. We now profile the users according to their risk and focusing degree. E.g., $u_1$ has rated relatively popular movies belonging to the same genre (risk-averse, high focusing degree); $u_2$ has rated niches movies in the same genre (risk-loving, high focusing degree); $u_3$ has rated popular movies in two different genres (risk-averse, low focusing degree), and $u_4$ has rated niches movies in two different genres (risk-loving, low focusing degree). 
\label{ex:running}
\end{example}
\begin{table}[htb]
\centering
\tiny
\begin{tabular}{ccccccc} 
\toprule
			&$m_1^{1}$ &$m_2^{1}$   &$m_3^{2}$    &$m_4^{3}$   &$m_5^{3}$ &$m_6^{3}$  \\ \hline 
$u_1 $ &5  &4  &-  &-  &-  &-   \\
$u_2$  &-  &-  &-  &-  &5  &5   \\
$u_3$  &-  &4  &-  &5  &-  &-   \\
$u_4$  &-  &-  &3  &-  &-  &4   \\ 
$u_5$  &5  &-  &-  &3  &-  &-   \\ 
$u_6$  &4  &2  &-  &4  &-  &-   \\ 
\bottomrule
%$ f_i$    &3  &3  &1  &3  &1  &2  \\  \hline
\end{tabular}
\caption{User-Movie rating data} \label{tab:example}
\end{table}
It is essential to consider these consumer characteristics in designing recommender systems so that they promote long-tail items to the right group of users and spread demand evenly between the hit and niche items.  
\fi
\iffalse
\begin{center}
\begin{figure*}[tp]
%\scalebox{0.5}{%
\resizebox{1\textwidth}{!}{%
%\small%\addtolength{\tabcolsep}{5pt}% below sums to 8
\begin{tabularx}{1.5\textwidth}{>{\hsize=2.5\hsize}X>{\hsize=2.5\hsize}X>{\hsize=0.5\hsize}X>{\hsize=0.5\hsize}X>{\hsize=0.5\hsize}X>{\hsize=0.5\hsize}X>{\hsize=0.5\hsize}X>{\hsize=0.5\hsize}X}
    \multirow{12}{*}{\includegraphics[scale=0.3]{codeForExample/popularity-movie.png}} & \multirow{12}{*}{\includegraphics[scale=0.3]{codeForExample/scatterplot.png}} & & & & & & \\
%   & &               &       &       &       &       &       \\
    & &\multicolumn{1}{l|}{}               &$m_1^{g1}$   	&$m_2^{g1}$    	&$m_3^{g2}$    &$m_4^{g2}$      &$m_5^{g3}$    \\ \cline{3-8}%\hline
    & &\multicolumn{1}{l|}{u1}          &5  &5  &-  &-   &-  \\
    & &\multicolumn{1}{l|}{u2}    		&-  &-  &4  &4  &5  \\
    & &\multicolumn{1}{l|}{u3}   			&1  &2  &1  &-  &-   \\
    & &\multicolumn{1}{l|}{u4}     		&1  &-  &-  &-  &-  \\
    & &               &       &       &       &       &       \\
    & &               &       &       &       &       &       \\
    & &               &       &       &       &       &       \\
    & &               &       &       &       &       &	\\
    \\
\end{tabularx}}
\caption{User-Movie interaction data a) Popularity-Movie histogram b)Movie genres/clusters c) User-Movie rating data} \label{fig:example}
\end{figure*}
\end{center}
\fi



%We propose a novel approach that allows us to  promote long-tail items in a targeted manner, thereby improving the novelty of top-$\size$ sets, the overall item-space coverage across recommendations, while maintaining reasonable levels of accuracy.

%Next, we integrate these learned preferences  in a generic  top-$\size$ recommendation framework to provide customized balance between accuracy and coverage.

%sequentially make recommendations, while adjusting its parameters with regard to the set of top-$\size$ recommendations made so far. However, since  sequential parameter updates  cause  scalability issues, we propose a sampling based algorithm. This variant of our framework, called {\it Ordered Sampling-based Locally Greedy (OSLG)\/},  allows us to  correct for the popularity bias in recommendations with regard to individual user long-tail preferences. 

%ICDE submission
%Our framework differs with  prior work in the following aspects:  unlike~\cite{adomavicius2011maximizing,adomavicius2012improving,zhang2013personalize,ho2014likes},  the long-tail preference personalization in our framework is learned rather than optimized using cross-validation or parameter tuning. In other words, our personalization method is independent of the underlying base  recommendation models.  Moreover, our framework is  generic. This enables us to  plug-in several base recommenders, and evaluate their  effectiveness without requiring  extensive tuning for the accuracy and coverage trade-off. 


%\vspace{-2.8pt}
\begin{itemize}

\item  We examine various measures for estimating user long-tail novelty preference in Section~\ref{sec:lt-pref} and formulate an optimization problem  to directly learn users' preferences for long-tail  items from interaction data in Section~\ref{sec:learning-lt-pref}. %In addition, we introduce several heuristics for measuring the user preference for less common items from historical rating data.% 

\item  We integrate the user preference estimates into GANC %, a generic re-ranking framework that provides customized balance between accuracy, novelty, and coverage 
(Section~\ref{sec:RiskbasedReranking}), and  introduce {\it Ordered Sampling-based Locally Greedy (OSLG)\/}, a scalable algorithm that relies  on user long-tail preferences to correct the popularity bias (Section~\ref{sec:optimizationAlgorithm}).
%We introduce OSLG, a scalable algorithm that relies  on user long-tail preferences to  maximize item space coverage \textcolor{red}{while maintaining acceptable levels of accuracy} (Section~\ref{sec:optimizationAlgorithm}).

\item   We conduct an extensive empirical study and evaluate performance from  accuracy, novelty, and coverage perspectives (Section~\ref{sec:Experiments}).  We use five  datasets with varying density and difficulty levels. %:  Netflix, MovieTweetings, and MovieLens (100K, 1M, 10M). 
  In contrast to most related work,  our evaluation considers realistic settings that include a large number of infrequent  items and users. %This enables us to study the impact of  data density on the performance trade-offs of several  state of the art top-$\size$ recommendation algorithms. %   %,  and use the all-items ranking protocol~\cite{steck2013evaluation,vargas2014improving}, where performance is measured using all items with train data. to evaluate the performance of several  state of the art top-$\size$ recommendation algorithms 
 
\item Our empirical results confirm that the performance of re-ranking models is impacted by the underlying   base recommender and the dataset density. Our generic approach enables us to easily incorporate a suitable base recommender to devise an effective solution for both dense and sparse settings. In dense settings, we use the same base recommender as existing re-ranking approaches, and we outperform them in accuracy and coverage metrics. For sparse settings, we plug-in a more suitable base recommender, and devise an effective solution that is competitive with existing top-$\size$ recommendation methods in accuracy and novelty. 

%Directly estimating the long-tail novelty preferences allows us to customize re-ranking per user, and  devise a generic framework.   
 
\end{itemize}

Section~\ref{sec:related-work} describes related work. Section~\ref{sec:conclusion} concludes.

The \jpsi resonance is frequently referred to as a hydrogen atom for QCD, and its resonance parameters (mass $M$, total width $\Tw$, leptonic widths $\Ew$ and $\Mw$, and so on) describe the fundamental properties of the strong and electromagnetic interactions. In theory, the decay widths can be predicted by different potential \\ models \cite{potential models 1,potential models 2} and lattice QCD calculations \cite{lattice calculations}. In experiment, with results from BABAR \cite{BABAR}, CLEO \cite{CLEO} and KEDR \cite{KEDR}, determinations of these decay widths have entered a period of precision measurement.

In 2012, data samples were taken at 15 center-of-mass energy points around the \jpsi resonance with the BESIII detector \cite{BESIII detector} operated at the BEPCII collider \cite{BESIII detector}. In this energy region, BEPCII provides high luminosity and BESIII shows excellent performance, which helps us accurately measure  the cross sections of \eetoee and \eetomumuwos. To measure \jpsi decay widths, accurate theoretical formulae taking into account higher-order corrections are also needed. If one wishes to have a high-efficiency optimization procedure, it is better to have analytic expressions for the theoretical cross sections. Because the continuum parts of these cross sections do not involve \jpsi decay widths and can be evaluated precisely by Monte-Carlo generators such as the Babayaga generator \cite{BABAYAGA}, only the analytic forms for the resonance and interference parts are derived in this paper.

We will start with theoretical fundamentals on the structure function method, its applications to the cases of \eetoee and \eetomumuwos, Born cross sections and the vacuum polarization function in Section \ref{Theoretical fundamentals}. Then, we will give the definitions and resulting formulae for the resonance and interference parts of the cross sections of \eetoee and \eetomumu in Section \ref{Calculations of the resonance and interference parts}. Most of the purely mathematical derivation is given in Appendix A to make the text easier to read.



\section{Theoretical fundamentals}
\label{Theoretical fundamentals}
%\input{Theoretical_fundamentals/Theoretical_fundamentals}

\subsection{Structure function method}

Generally, initial-state radiation (ISR), final-state radiation (FSR) and their interference (ISR-FSR relation) must be considered when one makes higher-order corrections to cross sections. Here, the ISR-FSR relation includes interference of diagrams with emission of real and virtual photons between initial- and final-state particles. The suppression level of the ISR-FSR relation between the production and decay stages of heavy unstable particles is discussed in Ref. \cite{ISR-FSR relation}. According to the conclusion in Ref. \cite{ISR-FSR relation}, there is no need to take into account the ISR-FSR relation in the case of \jpsi, because it is suppressed by $\Gamma_{\rm tot}/M$ (about $3\times10^{-5}$). As for FSR, a universal calculation is impossible if one has no explicit knowledge of selection criteria, so it needs to be handled separately with a numerical method, which is outside the scope of this paper. Thus, in this paper the calculation with ISR only is presented.

The structure function method \cite{STRUCTUREFUNCTIONMETHOD} is adopted here to deal with ISR. Its fundamental formula is
\begin{equation}
	\sigma(s) = \int \int_0^X \frac{d\bar{\sigma}}{d\Omega}(s(1-x),\cos\theta) F(s,x) dx d\Omega.
	\label{Equation: Fundamental formula of structure function method}
\end{equation}
Here, $\sigma$ stands for the cross section after correction, $\frac{d\bar{\sigma}}{d\Omega}$ for the differential cross section before correction, $F$ for the radiator, $s$ for the square of the center-of-mass energy and $\theta$ for the polar angle of the positively charged final particle in the center-of-mass frame. The upper limit $X$ of the integration variable $x$ is usually set as $1-s^{\prime}_{min}/s$, where $s^{\prime}_{min}$ is the minimum of the invariant mass squared of the final-state particle system excluding the emitted photons.

The radiator $F$ adopted in this paper was first derived in Ref. \cite{Radiator 1} and slightly revised in Ref. \cite{Radiator 2}. Both documents are in Chinese, although the former has an English-language preprint (Ref. \cite{Radiator 3}). It is different from but a very good approximation of the classical one in Ref. \cite{STRUCTUREFUNCTIONMETHOD}. Its expression is
\begin{align}
	F(s,x) & = x^{v-1}v(1+\delta) \nonumber \\
	& + x^{v}\left(-v-\frac{v^2}{4}\right) + x^{v+1}\left(\frac{v}{2}-\frac{3}{8}v^2\right),
	\label{Equation: Expression of radiator.}
\end{align}
where
\begin{equation}
	\delta(v) = \frac{\alpha}{\pi}(\frac{\pi^2}{3}-\frac{1}{2})+\frac{3}{4}v+\left(\frac{9}{32}-\frac{\pi^2}{12}\right)v^2
\end{equation}
and
\begin{equation}
	v(s) = \frac{2\alpha}{\pi}\left(\ln\frac{s}{m_e^2}-1\right).
\end{equation}
Here, $\alpha$ stands for the fine structure constant and $m_e$ denotes the electron mass.

\subsection{Applications of the structure fuction meth-\\od to \eetoee and \eetomumu}
\label{Subsection: Applications of the structure fuction method to eetoee and eetomumu}

Applying the structure function method to the cases of \eetoee and \eetomumuwos, one can get
\begin{align}
	&\ \ \  \left(\frac{d\sigma}{d\Omega}\right)_{ee|\mu\mu}(s,\cos\theta) \nonumber \\
	& = \int_0^X \left(\frac{d\bar{\sigma}}{d\Omega}\right)_{ee|\mu\mu}(s(1-x),\cos\theta) F(s,x) dx,
	\label{Equation: Formula of structure function method applied to eetoee and eetomumu.}
\end{align}
%\begin{align}
%&\ \ \  \left(\frac{d\sigma}{d\Omega}\right)_{ee}(s,\cos\theta) \nonumber \\
%& = \int_0^X \left(\frac{d\bar{\sigma}}{d\Omega}\right)_{ee}(s(1-x),\cos\theta) F(s,x) dx
%\label{Equation: Formula of structure function method applied to eetoee.}
%\end{align}
%and
%\begin{align}
%&\ \ \  \left(\frac{d\sigma}{d\Omega}\right)_{\mu\mu}(s,\cos\theta) \nonumber \\
%& = \int_0^X \left(\frac{d\bar{\sigma}}{d\Omega}\right)_{\mu\mu}(s(1-x),\cos\theta) F(s,x) dx,
%\label{Equation: Formula of structure function method applied to eetomumu.}
%\end{align}
where the symbol $|$ stands for ``or",
\begin{align}
	\left(\frac{d\bar{\sigma}}{d\Omega}\right)_{ee} & = \left(\frac{d\sigma_0}{d\Omega}\right)_{ee}^{\rm S}\left|\frac{1}{1-\Pi(s)}\right|^2 \nonumber \\
	& + \left(\frac{d\sigma_0}{d\Omega}\right)_{ee}^{\rm T}\left|\frac{1}{1-\Pi(t)}\right|^2 \nonumber \\
	& + \left(\frac{d\sigma_0}{d\Omega}\right)_{ee}^{\rm STI} Re\left(\frac{1}{1-\Pi(s)}\overline{\frac{1}{1-\Pi(t)}}\right) \label{Equation: Cross sections of eetoee with vacuum polarization considered}
\end{align}
and
\begin{equation}
	\left(\frac{d\bar{\sigma}}{d\Omega}\right)_{\mu\mu} = \left(\frac{d\sigma_0}{d\Omega}\right)_{\mu\mu}^{\rm S}\left|\frac{1}{1-\Pi(s)}\right|^2. \label{Equation: Cross sections of eetomumu with vacuum polarization considered}
\end{equation}
Here, $t$ denotes the square of the 4-momentum transferred in the t channel. As for \eetoeewos, the relation between $t$ and $s$ is
\begin{equation}
	t \approx -\frac{s}{2}(1-\csth).
\end{equation}
In addition, $\left(\frac{d\sigma_0}{d\Omega}\right)_{ee}^{\rm S}$, $\left(\frac{d\sigma_0}{d\Omega}\right)_{ee}^{\rm T}$, $\left(\frac{d\sigma_0}{d\Omega}\right)_{ee}^{\rm STI}$ and $\left(\frac{d\sigma_0}{d\Omega}\right)_{\mu\mu}^{\rm S}$ are Born cross sections, and $\frac{1}{1-\Pi}$ is the vacuum polarization function. They will be discussed in the following two subsections.

\subsection{Born cross sections}

The quantities $\left(\frac{d\sigma_0}{d\Omega}\right)_{ee}^{\rm S}$, $\left(\frac{d\sigma_0}{d\Omega}\right)_{ee}^{\rm T}$ and $\left(\frac{d\sigma_0}{d\Omega}\right)_{ee}^{\rm STI}$ are the s channel part, the t channel part and the s-t interference part of the Born cross section of \eetoee $\left(\left(\frac{d\sigma_0}{d\Omega}\right)_{ee}\right)$, respectively, that is
\begin{align}
	& \ \ \  \left(\frac{d\sigma_0}{d\Omega}\right)_{ee} = \left(\frac{d\sigma_0}{d\Omega}\right)_{ee}^{\rm S} + \left(\frac{d\sigma_0}{d\Omega}\right)_{ee}^{\rm T} + \left(\frac{d\sigma_0}{d\Omega}\right)_{ee}^{\rm STI},
\end{align}
where
\begin{subequations}
	\label{Align: Born cross section of eetoee}
	\begin{align}
		\left(\frac{d\sigma_0}{d\Omega}\right)_{ee}^{\rm S} & = \frac{\alpha^2}{4s} (1+\cos^2\theta), \label{Equation: S channel part of Born cross section of eetoee} \\
		\left(\frac{d\sigma_0}{d\Omega}\right)_{ee}^{\rm T} & = \frac{\alpha^2}{2s} \frac{(1+\cos\theta)^2+4}{(1-\cos\theta)^2}, \label{Equation: T channel part of Born cross section of eetoee} \\
		\left(\frac{d\sigma_0}{d\Omega}\right)_{ee}^{\rm STI} & = - \frac{\alpha^2}{2s} \frac{(1+\cos\theta)^2}{1-\cos\theta}. \label{Equation: S-T interference part of Born cross section of eetoee}
	\end{align}
\end{subequations}
The Born cross section of \eetomumu $\left(\left(\frac{d\sigma_0}{d\Omega}\right)_{\mu\mu}\right)$ has only an s channel part $\left(\frac{d\sigma_0}{d\Omega}\right)_{\mu\mu}^{\rm S}$, which equals exactly $\left(\frac{d\sigma_0}{d\Omega}\right)_{ee}^{\rm S}$ given by Eq. (\ref{Equation: S channel part of Born cross section of eetoee}).

\subsection{Vacuum polarization function}

In Section 4 of Ref. \cite{KEDRpsip}, the distinction and relationship between the ``bare" and ``dressed" parameters of $J^{PC}=1^{--}$ resonances (for example \jpsiwos) are discussed in detail. In the discussion there, the vacuum polarization function is written as
\begin{equation}
	\frac{1}{1-\Pi(q^2)} = \frac{1}{1-\Pi_{\rm 0}(q^2)}+\Pi_{\rm R}(q^2),
	\label{Equation: The vacuum polarization function}
\end{equation}
where $\Pi_{\rm R}$ is expressed with the ``dressed" parameters $M$, $\Gamma_{\rm tot}$ and $\Gamma_{ee}$ as
\begin{equation}
	\Pi_{\rm R}(q^2) = \frac{3\Gamma_{ee}}{\alpha} \frac{q^2}{M} \frac{1}{q^2-M^2+iM\Gamma_{\rm tot}}.
	\label{Equation: Resonance part of vacuum polarization function}
\end{equation}
Here, $\Pi_{\rm R}$ stands for the contribution from the resonance itself (in our case, it is \jpsiwos), while $\Pi_{\rm 0}$ denotes contributions from other sources. Based on the lepton universality assumption, $\Gamma_{ee}$ in Eq. (\ref{Equation: Resonance part of vacuum polarization function}) can be substituted by $\sqrt{\Gamma_{ee}\Gamma_{\mu\mu}}$ in the case of \eetomumuwos.

According to Eq. (\ref{Equation: The vacuum polarization function}), $\frac{1}{1-\Pi(s)}$ and $\frac{1}{1-\Pi(t)}$ in Eq. (\ref{Equation: Cross sections of eetoee with vacuum polarization considered}) and (\ref{Equation: Cross sections of eetomumu with vacuum polarization considered}) can be expressed as
\begin{equation}
	\frac{1}{1-\Pi(s)} = \frac{1}{1-\Pi_0(s)} + \Pi_{\rm R}(s) \label{Equation: Vacuum polarization in the timelike region}
\end{equation}
and
\begin{equation}
	\frac{1}{1-\Pi(t)} = \frac{1}{1-\Pi_0(t)}. \label{Equation: Vacuum polarization in the spacelike region}
\end{equation}
No $\Pi_{\rm R}(t)$ term appears in Eq. (\ref{Equation: Vacuum polarization in the spacelike region}) because it can be safely ignored in the spacelike region. Besides, the imaginary parts of $\frac{1}{1-\Pi_0(s)}$ and $\frac{1}{1-\Pi_0(t)}$ can be safely ignored as well. Consequently, $\frac{1}{1-\Pi_0(s)}$ and $\frac{1}{1-\Pi_0(t)}$ will be regarded as real in the following section.


\section{Calculations of the resonance and interference parts}
\label{Calculations of the resonance and interference parts}
%\input{Calculations_of_the_resonance_and_interference_parts/Calculations_of_the_resonance_and_interference_parts}
\subsection{Definitions}

Considering $\left(\frac{d\bar{\sigma}}{d\Omega}\right)_{ee}$ and $\left(\frac{d\bar{\sigma}}{d\Omega}\right)_{\mu\mu}$ given by Eq. (\ref{Equation: Cross sections of eetoee with vacuum polarization considered}) and (\ref{Equation: Cross sections of eetomumu with vacuum polarization considered}) as well as $\frac{1}{1-\Pi(s)}$ and $\frac{1}{1-\Pi(t)}$ given by Eq. (\ref{Equation: Vacuum polarization in the timelike region}) and (\ref{Equation: Vacuum polarization in the spacelike region}), one can expand $\left(\frac{d\sigma}{d\Omega}\right)_{ee}$ and $\left(\frac{d\sigma}{d\Omega}\right)_{\mu\mu}$ via Eq. (\ref{Equation: Formula of structure function method applied to eetoee and eetomumu.}) into \\ many small terms. With these small terms regrouped, the resonance and interference parts of $\left(\frac{d\sigma}{d\Omega}\right)_{ee}$ and $\left(\frac{d\sigma}{d\Omega}\right)_{\mu\mu}$, namely $\left(\frac{d\sigma}{d\Omega}\right)_{ee}^{\rm R}$, $\left(\frac{d\sigma}{d\Omega}\right)_{ee}^{\rm CRI}$, $\left(\frac{d\sigma}{d\Omega}\right)_{\mu\mu}^{\rm R}$ and $\left(\frac{d\sigma}{d\Omega}\right)_{\mu\mu}^{\rm CRI}$, can be defined as

\end{multicols}

\begin{subequations}
	\label{Align: definitions}
	\begin{align}
		\left(\frac{d\sigma}{d\Omega}\right)_{ee}^{\rm R} & = \int_0^X \left(\frac{d\sigma_0}{d\Omega}\right)_{ee}^{\rm S}(s(1-x),\cos\theta)\left|\Pi_{\rm R}(s(1-x))\right|^2 F(s,x)dx, \label{Equation: Resonance part of cross section of eetoee} \\
		\nonumber \\
		\left(\frac{d\sigma}{d\Omega}\right)_{ee}^{\rm CRI} & = \int_0^X \Bigg(\left(\frac{d\sigma_0}{d\Omega}\right)_{ee}^{\rm S}(s(1-x),\cos\theta) 2Re\left(\frac{1}{1-\Pi_0(s(1-x))}\Pi_{\rm R}(s(1-x))\right) + \nonumber \\
		& \ \ \ \ \ \ \ \ \ \ \  \left(\frac{d\sigma_0}{d\Omega}\right)_{ee}^{\rm STI}(s(1-x),\cos\theta) Re\left(\Pi_{\rm R}(s(1-x))\frac{1}{1-\Pi_0(t(1-x))}\right)\Bigg)F(s,x)dx, \label{Equation: Continuous-Resonance interference part of cross section of eetoee} \\
		\nonumber \\
		\left(\frac{d\sigma}{d\Omega}\right)_{\mu\mu}^{\rm R} & = \int_0^X \left(\frac{d\sigma_0}{d\Omega}\right)_{\mu\mu}^{\rm S}(s(1-x),\cos\theta)\left|\Pi_{\rm R}(s(1-x))\right|^2 F(s,x)dx, \label{Equation: Resonance part of cross section of eetomumu} \\
		\nonumber \\
		\left(\frac{d\sigma}{d\Omega}\right)_{\mu\mu}^{\rm CRI} & = \int_0^X \left(\frac{d\sigma_0}{d\Omega}\right)_{\mu\mu}^{\rm S}(s(1-x),\cos\theta) 2Re\left(\frac{1}{1-\Pi_0(s(1-x))}\Pi_{\rm R}(s(1-x))\right) F(s,x)dx. \label{Equation: Continuous-Resonance interference part of cross section of eetomumu}
	\end{align}
\end{subequations}

\centerline{\rule{80mm}{0.1pt}}

\begin{multicols}{2}
	
	With $\left(\frac{d\sigma_0}{d\Omega}\right)_{ee|\mu\mu}^{\rm S}$ and $\left(\frac{d\sigma_0}{d\Omega}\right)_{ee}^{\rm STI}$ expressed in Eq. (\ref{Equation: S channel part of Born cross section of eetoee}) and (\ref{Equation: S-T interference part of Born cross section of eetoee}) as well as $\Pi_{\rm R}$ expressed in Eq. (\ref{Equation: Resonance part of vacuum polarization function}) further employed, one can rewrite $\left(\frac{d\sigma}{d\Omega}\right)_{ee}^{\rm R}$, $\left(\frac{d\sigma}{d\Omega}\right)_{ee}^{\rm CRI}$, $\left(\frac{d\sigma}{d\Omega}\right)_{\mu\mu}^{\rm R}$ and $\left(\frac{d\sigma}{d\Omega}\right)_{\mu\mu}^{\rm CRI}$ more explicitly as
	
\end{multicols}

\begin{subequations}
	\label{Align: semi-finished results}
	\begin{align}
		\left(\frac{d\sigma}{d\Omega}\right)_{ee}^{\rm R} & = \frac{9\Gamma_{ee}^2}{4M^2} \cdot I^{\rm R} \cdot (1+\cos^2\theta), \label{Equation: Resonance part of cross section of eetoee --- semi-finished results} \\
		\nonumber \\
		\left(\frac{d\sigma}{d\Omega}\right)_{ee}^{\rm CRI} & = \frac{3\Gamma_{ee}\alpha}{2M} \cdot I^{\rm CRI} \cdot \left( (1+\cos^2\theta) \frac{1}{1-\Pi_0(s)} - \frac{(1+\cos\theta)^2}{1-\cos\theta} \frac{1}{1-\Pi_0(t)} \right), \label{Equation: Continuous-Resonance interference part of cross section of eetoee --- semi-finished results} \\
		\nonumber \\
		\left(\frac{d\sigma}{d\Omega}\right)_{\mu\mu}^{\rm R} & = \frac{9\Gamma_{ee}\Gamma_{\mu\mu}}{4M^2} \cdot I^{\rm R} \cdot (1+\cos^2\theta), \label{Equation: Resonance part of cross section of eetomumu --- semi-finished results}
	\end{align}
	\begin{align}
		\left(\frac{d\sigma}{d\Omega}\right)_{\mu\mu}^{\rm CRI} & = \frac{3\sqrt{\Gamma_{ee}\Gamma_{\mu\mu}}\alpha}{2M} \cdot I^{\rm CRI} \cdot (1+\cos^2\theta) \frac{1}{1-\Pi_0(s)} , \label{Equation: Continuous-Resonance interference part of cross section of eetomumu --- semi-finished results}
	\end{align}
\end{subequations}
where
\begin{subequations}
	\label{Align: definitions of the two integrals}
	\begin{align}
		I^{\rm R} & = \int_0^X \frac{s(1-x)}{(s(1-x)-M^2)^2+M^2\Gamma_{\rm tot}^2} F(s,x)dx, \\
		\nonumber \\
		I^{\rm CRI} & = \int_0^X \frac{s(1-x)-M^2}{(s(1-x)-M^2)^2+M^2\Gamma_{\rm tot}^2} F(s,x)dx.
	\end{align}
\end{subequations}

\centerline{\rule{80mm}{0.1pt}}

\begin{multicols}{2}
	Here, in the cases of $\left(\frac{d\sigma}{d\Omega}\right)_{ee}^{\rm CRI}$ and $\left(\frac{d\sigma}{d\Omega}\right)_{\mu\mu}^{\rm CRI}$, $\frac{1}{1-\Pi_0(s)}$ and $\frac{1}{1-\Pi_0(t)}$ are used as very good approximations to the equivalents of $\frac{1}{1-\Pi_0(s(1-x))}$ and $\frac{1}{1-\Pi_0(t(1-x))}$ after integration in Eq. (\ref{Align: definitions}). Numerical calculation indicates that the resulting deviations are less than 0.01\%.
	
	As can be seen from  Eq. (\ref{Align: semi-finished results}), to evaluate further, only $I^{\rm R}$ and $I^{\rm CRI}$ have to be calculated. Detailed calculations of the two integrals are put in Appendix A, which includes three parts: A.1, A.2, A.3. Their analytic formulae are fully derived in part A.1. Due to complexity, simplified versions of the analytic formulae are further obtained in part A.2. Finally, both versions of the analytic formulae are compared with numerical computing results in part A.3.
	
	Based on those of $I^{\rm R}$ and $I^{\rm CRI}$, we will list directly the full and simplified version of analytic results of $\left(\frac{d\sigma}{d\Omega}\right)_{ee}^{\rm R}$, $\left(\frac{d\sigma}{d\Omega}\right)_{ee}^{\rm CRI}$, $\left(\frac{d\sigma}{d\Omega}\right)_{\mu\mu}^{\rm R}$ and $\left(\frac{d\sigma}{d\Omega}\right)_{\mu\mu}^{\rm CRI}$ and discuss briefly their comparisons with numerical computing results in the following three subsections.
	
	\subsection{Full version of analytic results}
	
	With $I^{\rm R}$ and $I^{\rm CRI}$ expressed in Eq. (\ref{Equation: the first key integral}) and (\ref{Equation: the second key integral}) adopted, the full versions of the analytic formulae for $\left(\frac{d\sigma}{d\Omega}\right)_{ee}^{\rm R}$, $\left(\frac{d\sigma}{d\Omega}\right)_{ee}^{\rm CRI}$, $\left(\frac{d\sigma}{d\Omega}\right)_{\mu\mu}^{\rm R}$ and $\left(\frac{d\sigma}{d\Omega}\right)_{\mu\mu}^{\rm CRI}$ can be written as
	
\end{multicols}

\begin{subequations}
	\label{Align: full version of the final results}
	\begin{align}
		\left(\frac{d\sigma}{d\Omega}\right)_{ee}^{\rm R} & = \frac{9\Gamma_{ee}^2}{4M^2} \cdot s(P - Q) \cdot (1+\cos^2\theta), \\
		\nonumber \\
		\left(\frac{d\sigma}{d\Omega}\right)_{ee}^{\rm CRI} & = \frac{3\Gamma_{ee}\alpha}{2M} \cdot ((s-M^2) P  - s Q ) \cdot \left( (1+\cos^2\theta) \frac{1}{1-\Pi_0(s)} - \frac{(1+\cos\theta)^2}{1-\cos\theta} \frac{1}{1-\Pi_0(t)} \right), \\
		\nonumber \\
		\left(\frac{d\sigma}{d\Omega}\right)_{\mu\mu}^{\rm R} & = \frac{9\Gamma_{ee}\Gamma_{\mu\mu}}{4M^2} \cdot s(P - Q) \cdot  (1+\cos^2\theta), \\
		\nonumber \\
		\left(\frac{d\sigma}{d\Omega}\right)_{\mu\mu}^{\rm CRI} & = \frac{3\sqrt{\Gamma_{ee}\Gamma_{\mu\mu}}\alpha}{2M} \cdot ((s-M^2) P - s Q) \cdot (1+\cos^2\theta) \frac{1}{1-\Pi_0(s)},
	\end{align}
\end{subequations}
where
\begin{subequations}
	\begin{align}
		P & = \frac{1}{s^2}(A\ G(a,\beta,v,X) + B\ G(a,\beta,v+1,X) + C\ H(a,\beta,v,X)), \\
		Q & = \frac{1}{s^2}(D\ G(a,\beta,v+1,X) + E\ H(a,\beta,v,X) + C\ H(a,\beta,v+1,X))
	\end{align}
\end{subequations}
\centerline{\rule{80mm}{0.1pt}}
\begin{multicols}{2}
	\noindent with
	\begin{subequations}
		\begin{align}
			a & = \sqrt{\left(\frac{M^2}{s}-1\right)^2+\frac{M^2\Gamma_{\rm tot}^2}{s^2}},
		\end{align}
		\begin{align}
			\beta & = \cos^{-1}\left(\frac{\left(\frac{M^2}{s}-1\right)}{\sqrt{\left(\frac{M^2}{s}-1\right)^2+\frac{M^2\Gamma_{\rm tot}^2}{s^2}}}\right),
		\end{align}
		\begin{align}
			A & = 1+\delta, \\
			B & = \frac{1}{v+1}\left(-v-\frac{v^2}{4}\right), \\
			C & = \frac{v}{2}-\frac{3}{8}v^2, \\
			D & = \frac{Av}{v+1}, \\
			E & = B(v+1)
		\end{align}
	\end{subequations}
	and
	\begin{subequations}
		\begin{align}
			&\ \ \  G(a,\beta,v,X) \nonumber \\
			& = a^{v-2} \left(\frac{\pi v}{\sin\pi v}\right) \left(\frac{\sin[(1-v)\beta]}{\sin\beta}\right) + v X^{v-4} \bigg(\frac{X^2}{v-2} \nonumber \\
			& +\frac{2a(\cos\beta) X}{v-3}-\frac{a^2(4\cos^2\beta-1)}{v-4}\bigg) \ \ \ (0<v<2),
		\end{align}
		\begin{align}
			&\ \ \  H(a,\beta,v,X) \nonumber \\
			& = h(a\sin\beta,a\cos\beta,v+1,X+a\cos\beta) \nonumber \\
			& - h(a\sin\beta,a\cos\beta,v+1,a\cos\beta), \\
			&\ \ \  h(a,b,c,x) = -\frac{i}{2ac} \nonumber \\
			& \cdot \Bigg(\left(\frac{1}{-ia+x}\right)^{-c}{}_2\mathbb{F}_1\left(-c,-c,1-c,\frac{a+ib}{a+ix}\right) \nonumber \\
			& - \left(\frac{1}{ia+x}\right)^{-c}{}_2\mathbb{F}_1\left(-c,-c,1-c,\frac{ia+b}{ia+x}\right)\Bigg).
		\end{align}
	\end{subequations}
	Here, ${}_2\mathbb{F}_1$ is the Gauss hypergeometric function.
	
	\subsection{Simplified version of analytic results}
	
	With $I^{\rm R}$ and $I^{\rm CRI}$ given by Eq. (\ref{Equation: The approximate results 1}) and (\ref{Equation: The approximate results 2}), the simplified versions of the analytic formulae for $\left(\frac{d\sigma}{d\Omega}\right)_{ee}^{\rm R}$, $\left(\frac{d\sigma}{d\Omega}\right)_{ee}^{\rm CRI}$, $\left(\frac{d\sigma}{d\Omega}\right)_{\mu\mu}^{\rm R}$ and $\left(\frac{d\sigma}{d\Omega}\right)_{\mu\mu}^{\rm CRI}$ can be written as
	
\end{multicols}

\begin{subequations}
	\label{Align: Simplified version of the final results}
	\begin{align}
		\left(\frac{d\sigma}{d\Omega}\right)_{ee}^{\rm R} & = \frac{9\Gamma_{ee}^2}{4M^3\Gamma_{\rm tot}} (1+\delta) Im\mathcal{F} \cdot (1+\cos^2\theta), \label{Equation: The approximate results 1} \\
		\nonumber \\
		\left(\frac{d\sigma}{d\Omega}\right)_{ee}^{\rm CRI} & = - \frac{3\Gamma_{ee}\alpha}{2 M s} (1+\delta) Re\mathcal{F} \cdot \left( (1+\cos^2\theta) \frac{1}{1-\Pi_0(s)} - \frac{(1+\cos\theta)^2}{1-\cos\theta} \frac{1}{1-\Pi_0(t)} \right), \label{Equation: The approximate results 2} \\
		\nonumber \\
		\left(\frac{d\sigma}{d\Omega}\right)_{\mu\mu}^{\rm R} & = \frac{9\Gamma_{ee}\Gamma_{\mu\mu}}{4M^3\Gamma_{\rm tot}} (1+\delta) Im\mathcal{F} \cdot (1+\cos^2\theta), \label{Equation: The approximate results 3} \\
		\nonumber \\
		\left(\frac{d\sigma}{d\Omega}\right)_{\mu\mu}^{\rm CRI} & = - \frac{3\sqrt{\Gamma_{ee}\Gamma_{\mu\mu}}\alpha}{2 M s} (1+\delta) Re\mathcal{F} \cdot (1+\cos^2\theta) \frac{1}{1-\Pi_0(s)}, \label{Equation: The approximate results 4}
	\end{align}
\end{subequations}

\centerline{\rule{80mm}{0.1pt}}

\begin{multicols}{2}
	\noindent where
	\begin{equation}
		\mathcal{F} = \left( \frac{\pi v}{\sin\pi v} \right) \left(\frac{s}{M^2 - s - i M\Gamma_{\rm tot}}\right)^{1-v}.
		\label{Equation: The expressions of mathcalF.}
	\end{equation}
	
	\subsection{Comparison of analytic and numerical computing results}
	\label{Subsection: Comparisons with numerical computing results}
	
	As one can see from Eq. (\ref{Align: semi-finished results}) and (\ref{Align: definitions of the two integrals}),
	\begin{align*}
		& \ \ \  \left(\frac{\Delta \sigma}{\sigma}\right)_{ee|\mu\mu}^{\rm R|CRI}(\text{F}|\text{S},\text{N}) = \frac{\sigma_{ee|\mu\mu}^{\rm R|CRI}(\text{F}|\text{S})-\sigma_{ee|\mu\mu}^{\rm R|CRI}(\text{N})}{\sigma_{ee|\mu\mu}^{\rm R|CRI}(\text{N})} \\
		& = \frac{I^{\rm R|CRI}(\text{F}|\text{S})-I^{\rm R|CRI}(\text{N})}{I^{\rm R|CRI}(\text{N})} = \left(\frac{\Delta I}{I}\right)^{\rm R|CRI}(\text{F}|\text{S},\text{N}).
	\end{align*}
	%\end{multicols}
	%
	%\begin{equation*}
	%\left(\frac{\Delta \sigma}{\sigma}\right)_{ee|\mu\mu}^{\rm R|CRI}(\text{F}|\text{S},\text{N}) = \frac{\sigma_{ee|\mu\mu}^{\rm R|CRI}(\text{F}|\text{S})-\sigma_{ee|\mu\mu}^{\rm R|CRI}(\text{N})}{\sigma_{ee|\mu\mu}^{\rm R|CRI}(\text{N})} = \frac{I^{\rm R|CRI}(\text{F}|\text{S})-I^{\rm R|CRI}(\text{N})}{I^{\rm R|CRI}(\text{N})} = \left(\frac{\Delta I}{I}\right)^{\rm R|CRI}(\text{F}|\text{S},\text{N}).
	%\end{equation*}
	%
	%\centerline{\rule{80mm}{0.1pt}}
	%
	%\begin{multicols}{2}
	\noindent Here, the symbols F, S and N stand for the full version of the analytic results, the simplified version of the analytic results and the numerical computing results, respectively.
	
	According to part A.3 (the last part of Appendix A), from $\sqrt{s}=M-10\Tw$ to $\sqrt{s}=M+10\Tw$ with $X$ set at 1 as well as $M$ and $\Tw$ at their PDG values \cite{PDG2016}:
	\begin{equation*}
		\left(\frac{\Delta \sigma}{\sigma}\right)_{ee|\mu\mu}^{\rm R|CRI}(\text{F},\text{N}) = \left(\frac{\Delta I}{I}\right)^{\rm R|CRI}(\text{F},\text{N})<0.01\%
	\end{equation*}
	and
	\begin{equation*}
		\left(\frac{\Delta \sigma}{\sigma}\right)_{ee|\mu\mu}^{\rm R|CRI}(\text{S},\text{N}) = \left(\frac{\Delta I}{I}\right)^{\rm R|CRI}(\text{S},\text{N})<0.1\%.
	\end{equation*}
	Taking into account the precision of the structure function method itself is 0.1\% \cite{STRUCTUREFUNCTIONMETHOD}, we regard 0.1\% and 0.2\% as the precision of the full and simplified versions of the analytic formulae for $\left(\frac{d\sigma}{d\Omega}\right)_{ee}^{\rm R}$, $\left(\frac{d\sigma}{d\Omega}\right)_{ee}^{\rm CRI}$, $\left(\frac{d\sigma}{d\Omega}\right)_{\mu\mu}^{\rm R}$ and $\left(\frac{d\sigma}{d\Omega}\right)_{\mu\mu}^{\rm CRI}$, respectively.



\section{Conclusions}
%\vspace{-1em}
\section{Conclusions}
\vspace{-0.6em}
\label{SEC:CONC}
In this paper, we investigated the impact of workload dependent parameters on the failure ratio of the SSDs under power outage. To this end, we presented a fault injection and failure detection platform which injects the realistic power faults to the under test SSDs. During power failure, SSDs experience the exact voltage drop behavior that occurs during power failures in data centers. The results of our experiments reveal that the failure ratio in SSDs due to power outage is significantly affected by the parameters of the running workloads in the application layer. In addition, we show that failures in SSDs are not only due to volatile DRAM cache but also we observe similar failures in SSDs with disabled internal cache.


We have derived the detailed formulae for the resonance and interference parts of the cross sections of \eetoee and \eetomumu around the \jpsi resonance with higher-order corrections for vacuum polarization and initial-state radiation considered. In the derivation, the arbitrary upper limit of radiative correction integration $X$ has been involved. Two (full and simplified) versions of the analytic formulae are given with precision at the levels of 0.1\% and 0.2\%, which are accurate enough for the measurement of \jpsi decay widths at present.

In our derivation, only a very few steps rely on the values of \jpsi resonance parameters and they can be easily verified to be workable for the case of the \psip resonance. In the coming round of data-taking at BESIII, there is a plan for an energy scan around the \psip resonance for the measurement of the resonance parameters. By that time, the results obtained in this paper will be good references.



%\section{Acknowledgments}

This work was supported by NSFC programs (61976138, 61977047), the National Key Research and Development Program (2018YFB2100 500), STCSM (2015F0203-000-06), SHMEC (2019-01-07-00-01-E00003) and Shanghai YangFan Program (21YF1429500).
\ \\

%\acknowledgments{The authors would like to thank Prof. Ping Wang, Prof. Hai-Ming Hu and Prof. Chang-Zheng Yuan for their kind help and beneficial discussions as well as Prof. Wei-Guo Li for his suggestion on the contributing.}
\acknowledgments{The authors would like to thank Prof. Wei-Guo Li for his suggestion on the contributing as well as Prof. Ping Wang, Prof. Hai-Ming Hu and Prof. Chang-Zheng Yuan for their kind help and beneficial discussions.}


\end{multicols}

%\section{}
\label{sec: error state dynamics}
The $\mathbf{F}$ and $\mathbf{G}$ in Eq.~\eqref{eq: error state dynamics} are,
\begin{equation*}
\mathbf{F} = 
\begin{pmatrix}
-\lfloor\hat{\bm{\omega}}{}_{\times}\rfloor & -\mathbf{I}_3 & 
\mathbf{0}_{3\times 3} & \mathbf{0}_{3\times 3} & \mathbf{0}_{3\times 3} \\
\mathbf{0}_{3\times 3} & \mathbf{0}_{3\times 3} & \mathbf{0}_{3\times 3} & 
\mathbf{0}_{3\times 3} & \mathbf{0}_{3\times 3} \\
-C\left({}^I_G\hat{\mathbf{q}}\right)^\top\lfloor\hat{\mathbf{a}}{}_{\times}\rfloor & 
\mathbf{0}_{3\times 3} & \mathbf{0}_{3\times 3} & 
-C\left({}^I_G\hat{\mathbf{q}}\right)^\top & \mathbf{0}_{3\times 3} \\
\mathbf{0}_{3\times 3} & \mathbf{0}_{3\times 3} & \mathbf{0}_{3\times 3} & 
\mathbf{0}_{3\times 3} & \mathbf{0}_{3\times 3} \\
\mathbf{0}_{3\times 3} & \mathbf{0}_{3\times 3} & \mathbf{I}_3 & 
\mathbf{0}_{3\times 3} & \mathbf{0}_{3\times 3} \\
\mathbf{0}_{3\times 3} & \mathbf{0}_{3\times 3} & \mathbf{0}_{3\times 3} & 
\mathbf{0}_{3\times 3} & \mathbf{0}_{3\times 3} \\
\mathbf{0}_{3\times 3} & \mathbf{0}_{3\times 3} & \mathbf{0}_{3\times 3} & 
\mathbf{0}_{3\times 3} & \mathbf{0}_{3\times 3}
\end{pmatrix}
\end{equation*}
and, 
\begin{equation*}
\mathbf{G} = 
\begin{pmatrix}
-\mathbf{I}_3 & \mathbf{0}_{3\times 3} & 
\mathbf{0}_{3\times 3} & \mathbf{0}_{3\times 3} \\
\mathbf{0}_{3\times 3} & \mathbf{I}_3 & 
\mathbf{0}_{3\times 3} & \mathbf{0}_{3\times 3} \\
\mathbf{0}_{3\times 3} & \mathbf{0}_{3\times 3} & 
-C\left({}^I_G\hat{\mathbf{q}}\right)^\top & \mathbf{0}_{3\times 3} \\
\mathbf{0}_{3\times 3} & \mathbf{0}_{3\times 3} & 
\mathbf{0}_{3\times 3} & \mathbf{0}_{3\times 3} \\
\mathbf{0}_{3\times 3} & \mathbf{0}_{3\times 3} & 
\mathbf{0}_{3\times 3} & \mathbf{I}_3 \\
\mathbf{0}_{3\times 3} & \mathbf{0}_{3\times 3} & 
\mathbf{0}_{3\times 3} & \mathbf{0}_{3\times 3} \\
\mathbf{0}_{3\times 3} & \mathbf{0}_{3\times 3} & 
\mathbf{0}_{3\times 3} & \mathbf{0}_{3\times 3}
\end{pmatrix}
\end{equation*}

\section{}
\label{sec: state augmentation jacobian}
The state augmentation Jacobian, $\mathbf{J}$, given in Eq.~\eqref{eq: state covariance augmentation}, is of the form,
\begin{equation*}
\mathbf{J} = 
\begin{pmatrix}
\mathbf{J}_I & \mathbf{0}_{6\times 6N}
\end{pmatrix}
\end{equation*}
where $\mathbf{J}_I$ is,
\begin{equation*}
\mathbf{J}_I = 
\begin{pmatrix}
C\left({}^I_G\hat{\mathbf{q}}\right) & \mathbf{0}_{3\times 9} & 
\mathbf{0}_{3\times 3} & \mathbf{I}_3 & \mathbf{0}_{3\times 3} \\
-C\left({}^I_G\hat{\mathbf{q}}\right)^\top \lfloor{}^I\hat{\mathbf{p}}_C {}_{\times}\rfloor & 
\mathbf{0}_{3\times 9} & \mathbf{I}_3 & \mathbf{0}_{3\times 3} & 
\mathbf{I}_{3}
\end{pmatrix}
\end{equation*}
Note that $\mathbf{J}_I$ given above corrects the typo in Eq. (16) of~\cite{mourikis2007multi}. 

\section{}
\label{sec: measurement jacobian}
Following the chain rule, $\mathbf{H}_{C_i}^j$ and $\mathbf{H}_{f_i}^j$, in Eq.~\eqref{eq: error measurement model}, can be computed as,
\begin{equation}
\label{eq: measurement jacobian}
\begin{gathered}
\mathbf{H}_{C_i}^j = 
\frac{\partial \mathbf{z}_i^j}{\partial {}^{C_{i,1}}\mathbf{p}_j} \cdot 
\frac{\partial {}^{C_{i,1}}\mathbf{p}_j}{\partial \mathbf{x}_{C_{i,1}}} + 
\frac{\partial \mathbf{z}_i^j}{\partial {}^{C_{i,2}}\mathbf{p}_j} \cdot 
\frac{\partial {}^{C_{i,2}}\mathbf{p}_j}{\partial \mathbf{x}_{C_{i,1}}} \\
\mathbf{H}_{f_i}^j = 
\frac{\partial \mathbf{z}_i^j}{\partial {}^{C_{i,1}}\mathbf{p}_j} \cdot 
\frac{\partial {}^{C_{i,1}}\mathbf{p}_j}{\partial {}^G\mathbf{p}_j} +
\frac{\partial \mathbf{z}_i^j}{\partial {}^{C_{i,2}}\mathbf{p}_j} \cdot 
\frac{\partial {}^{C_{i,2}}\mathbf{p}_j}{\partial {}^G\mathbf{p}_j} 
\end{gathered}
\end{equation}
where,
\begin{equation}
\label{eq: measurment jacobian expression}
\begin{gathered}
\frac{\partial \mathbf{z}_i^j}{\partial {}^{C_{i,1}}\mathbf{p}_j} = 
\frac{1}{{}^{C_{i, 1}}\hat{Z}_j}
\begin{pmatrix}
1 & 0 & -\frac{{}^{C_{i, 1}}\hat{X}_j}{{}^{C_{i, 1}}\hat{Z}_j} \\
0 & 1 & -\frac{{}^{C_{i, 1}}\hat{Y}_j}{{}^{C_{i, 1}}\hat{Z}_j} \\
0 & 0 & 0 \\
0 & 0 & 0 
\end{pmatrix} \\
\frac{\partial \mathbf{z}_i^j}{\partial {}^{C_{i,2}}\mathbf{p}_j} = 
\frac{1}{{}^{C_{i, 2}}\hat{Z}_j}
\begin{pmatrix}
0 & 0 & 0 \\
0 & 0 & 0 \\
1 & 0 & -\frac{{}^{C_{i, 2}}\hat{X}_j}{{}^{C_{i, 1}}\hat{Z}_j} \\
0 & 1 & -\frac{{}^{C_{i, 2}}\hat{Y}_j}{{}^{C_{i, 1}}\hat{Z}_j} 
\end{pmatrix} \\
\frac{\partial {}^{C_{i,1}}\mathbf{p}_j}{\partial \mathbf{x}_{C_{i,1}}} = 
\begin{pmatrix}
\lfloor{}^{C_{i,1}}\hat{\mathbf{p}}_j{}_{\times}\rfloor & 
-C\left({}^{C_{i,1}}_G\hat{\mathbf{q}}\right)
\end{pmatrix} \\
\frac{\partial {}^{C_{i,1}}\mathbf{p}_j}{\partial {}^G\mathbf{p}_j} = 
C\left({}^{C_{i,1}}_G\hat{\mathbf{q}}\right) \\
\frac{\partial {}^{C_{i,2}}\mathbf{p}_j}{\partial \mathbf{x}_{C_{i,1}}} = 
C\left({}^{C_{i,1}}_{C_{i,2}}\mathbf{q}\right)^\top
\begin{pmatrix}
\lfloor{}^{C_{i,1}}\hat{\mathbf{p}}_j{}_{\times}\rfloor & 
-C\left({}^{C_{i,1}}_G\hat{\mathbf{q}}\right)
\end{pmatrix} \\
\frac{\partial {}^{C_{i,2}}\mathbf{p}_j}{\partial {}^G\mathbf{p}_j} = 
C\left({}^{C_{i,1}}_{C_{i,2}}\mathbf{q}\right)^\top
C\left({}^{C_{i,1}}_G\hat{\mathbf{q}}\right)
\end{gathered}
\end{equation}

\section{}
\label{sec: nullify measurement jacobian}
By defining the following short-hand notation from Eq.~\eqref{eq: measurment jacobian expression}
\begin{equation*}
\begin{gathered}
\frac{\partial \mathbf{z}_i^j}{\partial {}^{C_{i,1}}\mathbf{p}_j} = 
\begin{pmatrix}
\mathbf{J}_1 \\ \mathbf{0}
\end{pmatrix}, \quad
\frac{\partial \mathbf{z}_i^j}{\partial {}^{C_{i,2}}\mathbf{p}_j} = 
\begin{pmatrix}
\mathbf{0} \\ \mathbf{J}_2
\end{pmatrix}\\
\frac{\partial {}^{C_{i,1}}\mathbf{p}_j}{\partial \mathbf{x}_{C_{i,1}}} = 
\mathbf{H}_1, \quad 
\frac{\partial {}^{C_{i,1}}\mathbf{p}_j}{\partial {}^G\mathbf{p}_j} = 
\mathbf{H}_2, \quad
C\left({}^{C_{i,1}}_{C_{i,2}}\mathbf{q}\right) = 
\mathbf{R}\ ,
\end{gathered}
\end{equation*}
the measurement Jacobian in Eq.~\eqref{eq: measurement jacobian} can be compactly written as
\begin{equation*}
\mathbf{H}_{C_i}^j = 
\begin{pmatrix}
\mathbf{J}_1 \mathbf{H}_1 \\
\mathbf{J}_2 \mathbf{R}^\top \mathbf{H}_1
\end{pmatrix},\quad
\mathbf{H}_{f_i}^j =
\begin{pmatrix}
\mathbf{J}_1 \mathbf{H}_2 \\
\mathbf{J}_2 \mathbf{R}^\top \mathbf{H}_2
\end{pmatrix}\ .
\end{equation*}
Assuming $\mathbf{v} = \left(\mathbf{v}_1^\top,\ \mathbf{v}_2^\top\right)^\top\in\mathbb{R}^4$ is the left null space of $\mathbf{H}_{f_i}^j$, then,
\begin{equation*}
\mathbf{v}^\top \mathbf{H}_{f_i}^j  = 
\left(\mathbf{v}_1^\top \mathbf{J}_1 + 
\mathbf{v}_2^\top\mathbf{J}_2\mathbf{R}^\top\right) 
\mathbf{H}_2 = \mathbf{0}
\end{equation*}
Since $\mathbf{H}_2 = C\left({}^{C_{i,1}}_G\hat{\mathbf{q}}\right)$ is a rotation matrix, $\text{rank}\left(\mathbf{H}_2\right) = 3$ which implies that $\mathbf{v}_1^\top \mathbf{J}_1 + \mathbf{v}_2^\top\mathbf{J}_2\mathbf{R}^\top = \mathbf{0}$. With such property, it immediately follows that $\mathbf{v}$ is also the left null space of $\mathbf{H}_{C_i}^j$, 
\begin{equation*}
\mathbf{v}^\top \mathbf{H}_{C_i}^j = 
\left(\mathbf{v}_1^\top \mathbf{J}_1 + 
\mathbf{v}_2^\top\mathbf{J}_2\mathbf{R}^\top\right) 
\mathbf{H}_1 = \mathbf{0}
\end{equation*}
Therefore, a singe stereo measurement cannot be directly used for measurement update.

%The first line must not be deleted. It is needed by the CPC template.
\vspace{15mm}
%\input{Appendices/Two_key_integrals/Two_key_integrals}
\begin{small}
	\renewcommand{\theequation}{A\arabic{equation}}
	\setcounter{equation}{0}
	\begin{multicols}{2}
		\subsection*{Appendix A}
		\label{Section: Appendix A}
		\noindent{\bf Calculations of $I^{\rm R}$ and $I^{\rm CRI}$} \\
		
		\noindent \textbf{A.1 Full versions of analytic formulae} \\
		%\subsubsection*{\textbf{A.1 The full version of analytic formulae} \\}
		
		In the appendix, we evaluate the two integrals $I^{\rm R}$ and $I^{\rm CRI}$ required in Section \ref{Calculations of the resonance and interference parts}. For the convenience of further calculations, it is necessary to make some simple transformations by introducing some new variables. The first transformation is
		\begin{equation}
			\frac{1}{(s(1-x)-M^2)^2+M^2\Gamma_{\rm tot}^2} = \frac{1}{s^2} \frac{1}{x^2+2a(\cos\beta) x+a^2},
		\end{equation}
		where
		\begin{subnumcases}{}
			a = \sqrt{\left(\frac{M^2}{s}-1\right)^2+\frac{M^2\Gamma_{\rm tot}^2}{s^2}}, \\
			\beta = \cos^{-1}\left(\frac{\left(\frac{M^2}{s}-1\right)}{\sqrt{\left(\frac{M^2}{s}-1\right)^2+\frac{M^2\Gamma_{\rm tot}^2}{s^2}}}\right).
		\end{subnumcases}
		
		The second transformation is
		\begin{align}
			F(s,x) & = x^{v-1}v(1+\delta) \nonumber \\
			& + x^{v}\left(-v-\frac{v^2}{4}\right) + x^{v+1}\left(\frac{v}{2}-\frac{3}{8}v^2\right) \nonumber \\
			& = Avx^{v-1} + B(v+1)x^{v} + Cx^{v+1},
		\end{align}
		where
		\begin{subnumcases}{}
			A = 1+\delta, \\
			B = \frac{1}{v+1}\left(-v-\frac{v^2}{4}\right), \\
			C = \frac{v}{2}-\frac{3}{8}v^2.
		\end{subnumcases}
		
		The third transformation is
		\begin{align}
			xF(s,x) & = x^{v}v(1+\delta) \nonumber \\
			& + x^{v+1}\left(-v-\frac{v^2}{4}\right) + x^{v+2}\left(\frac{v}{2}-\frac{3}{8}v^2\right) \nonumber \\
			& = D(v+1)x^{v} + Ex^{v+1} + Cx^{v+2},
		\end{align}
		where
		\begin{subnumcases}{}
			D = \frac{Av}{v+1}, \\
			E = B(v+1).
		\end{subnumcases}
		
		In addition, some integral formulae are crucial for further calculations. From the following two integral formulae
		\begin{align}
			&\ \ \  \int_0^{\infty} \frac{vx^{v-1}}{x^2+2a(\cos\beta) x+a^2} dx \nonumber \\
			& = a^{v-2} \left(\frac{\pi v}{\sin\pi v}\right) \left(\frac{\sin[(1-v)\beta]}{\sin\beta}\right) \ \ \ (0<v<2)
		\end{align}
		and
		\begin{align}
			&\ \ \  \int_{X}^{\infty} \frac{vx^{v-1}}{x^2+2a(\cos\beta) x+a^2} dx \simeq v X^{v-4} \Bigg(-\frac{X^2}{v-2} \nonumber \\
			& - \frac{2a(\cos\beta) X}{v-3}+\frac{a^2(4\cos^2\beta-1)}{v-4}\Bigg) \ \ \ (v<2),
		\end{align}
		one obtains for the first integral formula
		\begin{align}
			&\ \ \  G(a,\beta,v,X) = \int_{0}^{X} \frac{vx^{v-1}}{x^2+2a(\cos\beta) x+a^2} dx \nonumber \\
			& \simeq a^{v-2} \left(\frac{\pi v}{\sin\pi v}\right) \left(\frac{\sin[(1-v)\beta]}{\sin\beta}\right) + v X^{v-4} \bigg(\frac{X^2}{v-2} \nonumber \\
			& +\frac{2a(\cos\beta) X}{v-3}-\frac{a^2(4\cos^2\beta-1)}{v-4}\bigg) \ \ \ (0<v<2).
		\end{align}
		
		The second integral formula is
		\begin{align}
			&\ \ \  H(a,\beta,v,X) = \int_{0}^{X} \frac{x^{v+1}}{x^2+2a(\cos\beta) x+a^2} dx \nonumber \\
			& = \int_{0}^{X} \frac{x^{v+1}}{(x+a\cos\beta)^2+(a\sin\beta)^2} dx \nonumber \\
			& = \int_{a\cos\beta}^{X+a\cos\beta} \frac{(y-a\cos\beta)^{v+1}}{y^2+(a\sin\beta)^2} dy \nonumber \\
			%\end{align}
			%\begin{align}
			& = h(a\sin\beta,a\cos\beta,v+1,X+a\cos\beta) \nonumber \\
			& - h(a\sin\beta,a\cos\beta,v+1,a\cos\beta),
		\end{align}
		where
		\begin{align}
			&\ \ \  h(a,b,c,x) = \int_0^x \frac{(y-b)^c}{y^2+a^2} dy = -\frac{i}{2ac} \nonumber \\
			& \cdot \Bigg(\left(\frac{1}{-ia+x}\right)^{-c}{}_2\mathbb{F}_1\left(-c,-c,1-c,\frac{a+ib}{a+ix}\right) \nonumber \\
			%\end{align}
			%\begin{align}
			& - \left(\frac{1}{ia+x}\right)^{-c}{}_2\mathbb{F}_1\left(-c,-c,1-c,\frac{ia+b}{ia+x}\right)\Bigg).
		\end{align}
		Here, ${}_2\mathbb{F}_1$ is the Gauss hypergeometric function.
		
		Using the newly introduced variables and the important integral formulae, we get
	\end{multicols}
	\begin{align}
		P & = \int_0^X \frac{1}{(s(1-x)-M^2)^2+M^2\Gamma_{\rm tot}^2} F(s,x)dx = \frac{1}{s^2} \int_0^X \frac{1}{x^2+2a(\cos\beta) x+a^2} (Avx^{v-1} + B(v+1)x^{v} + Cx^{v+1})dx \nonumber \\
		& = \frac{1}{s^2}\left(A \int_0^X \frac{vx^{v-1}}{x^2+2a(\cos\beta) x+a^2} dx + B \int_0^X \frac{(v+1)x^{v}}{x^2+2a(\cos\beta) x+a^2}dx + C \int_0^X \frac{x^{v+1}}{x^2+2a(\cos\beta) x+a^2} dx\right) \nonumber \\
		& = \frac{1}{s^2}(A\ G(a,\beta,v,X) + B\ G(a,\beta,v+1,X) + C\ H(a,\beta,v,X))
	\end{align}
	and
	\begin{align}
		Q & = \int_0^X \frac{x}{(s(1-x)-M^2)^2+M^2\Gamma_{\rm tot}^2} F(s,x)dx = \frac{1}{s^2} \int_0^X \frac{x}{x^2+2a(\cos\beta) x+a^2} (Avx^{v-1} + B(v+1)x^{v} + Cx^{v+1})dx \nonumber \\
		& = \frac{1}{s^2} \int_0^X \frac{1}{x^2+2a(\cos\beta) x+a^2} (D(v+1)x^{v} + Ex^{v+1} + Cx^{v+2})dx \nonumber \\
		& = \frac{1}{s^2} \left( D \int_0^X \frac{(v+1)x^{v}}{x^2+2a(\cos\beta) x+a^2} dx + E \int_0^X \frac{x^{v+1}}{x^2+2a(\cos\beta) x+a^2} dx + C \int_0^X \frac{x^{v+2}}{x^2+2a(\cos\beta) x+a^2} dx \right) \nonumber \\
		& = \frac{1}{s^2} (D\ G(a,\beta,v+1,X) + E\ H(a,\beta,v,X) + C\ H(a,\beta,v+1,X)),
	\end{align}
	and then get
	\begin{align}
		I^{\rm R} & = \int_0^X \frac{s(1-x)}{(s(1-x)-M^2)^2+M^2\Gamma_{\rm tot}^2} F(s,x)dx = s \int_0^X \frac{1-x}{(s(1-x)-M^2)^2+M^2\Gamma_{\rm tot}^2} F(s,x)dx \nonumber \\
		& = s \left(\int_0^X \frac{1}{(s(1-x)-M^2)^2+M^2\Gamma_{\rm tot}^2} F(s,x)dx - \int_0^X \frac{x}{(s(1-x)-M^2)^2+M^2\Gamma_{\rm tot}^2} F(s,x)dx \right) \nonumber \\
		& = s (P - Q)
		\label{Equation: the first key integral}
	\end{align}
	and
	\begin{align}
		I^{\rm CRI} & = \int_0^X \frac{s(1-x)-M^2}{(s(1-x)-M^2)^2+M^2\Gamma_{\rm tot}^2} F(s,x)dx = \int_0^X \frac{(s-M^2)-s x}{(s(1-x)-M^2)^2+M^2\Gamma_{\rm tot}^2} F(s,x)dx \nonumber \\
		& = (s-M^2) \int_0^X \frac{1}{(s(1-x)-M^2)^2+M^2\Gamma_{\rm tot}^2} F(s,x)dx - s \int_0^X \frac{x}{(s(1-x)-M^2)^2+M^2\Gamma_{\rm tot}^2} F(s,x)dx \nonumber \\
		& =(s-M^2) P - s Q.
		\label{Equation: the second key integral}
	\end{align}
	\centerline{\rule{80mm}{0.1pt}}
	\begin{multicols}{2}
		
		Equations (\ref{Equation: the first key integral}) and (\ref{Equation: the second key integral}) give the analytic formulae for $I^{\rm R}$ and $I^{\rm CRI}$. Since there are no approximations made in the derivation, we refer to the formulae as the full versions of the analytic formulae. Considering all the quantities involved in $P$ and $Q$ ($A$, $B$, $C$ and so on), the results are actually very complicated. For ease of use, simplified versions of the analytic formulae are needed.
		
		\noindent \textbf{A.2 Simplified versions of analytic formulae} \\
		%\subsubsection*{\textbf{A.2 The simplified version of analytic formulae} \\}
		
		In this part, we will make some approximations to obtain simplified versions of the analytic formulae. The first step is to reduce $F(s,x)$ to $x^{v-1}v(1+\delta)$. Since $0 \le x \le 1$ and $v \approx 0.08$ in the \jpsi region, the parts discarded are negligible. This reduction leads to $B=0$, $C=0$, $E=0$.
		
		The second step is to reduce $G(a,\beta,v,X)$ to $a^{v-2} \left(\frac{\pi v}{\sin\pi v}\right) \left(\frac{\sin[(1-v)\beta]}{\sin\beta}\right)$. This reduction means that $X \to +\infty$, which is unreasonable from the physical point of view. However, since $v \approx 0.08$ and $a \in (3\times 10^{-5},\  3\times 10^{-2})$, the reduction itself is a reasonable mathematical approximation when $X$ is large enough. In addition, in the cases of $\left(\frac{d\sigma}{d\Omega}\right)_{ee}^{\rm R}$ and $\left(\frac{d\sigma}{d\Omega}\right)_{\mu\mu}^{\rm R}$, a reasonable reduction of $\sin[(1-v)\beta] - a\sin[(-v)\beta]$ to $\sin[(1-v)\beta]$ is also carried out at this step. With the two steps of approximation applied, one can get
		\begin{equation}
			I^{\rm R} \approx \frac{1}{s a\sin\beta} (1+\delta) a^{v-1} \left(\frac{\pi v}{\sin\pi v}\right) \sin[(1-v)\beta]
		\end{equation}
		and
		\begin{equation}
			I^{\rm CRI} \approx - \frac{1}{s} (1+\delta) a^{v-1} \left(\frac{\pi v}{\sin\pi v}\right) \cos[(1-v)\beta].
		\end{equation}
		
		At this point, if one introduces a complex variable
		\begin{equation}
			\mathcal{F} = \left( \frac{\pi v}{\sin\pi v} \right) (a\cos\beta - i a\sin\beta)^{v-1},
		\end{equation}
		then
		\begin{equation}
			a^{v-1} \left(\frac{\pi v}{\sin\pi v}\right) \sin[(1-v)\beta] = Im\mathcal{F},
			\label{Equation: The Im part of mathcalF}
		\end{equation}
		\begin{equation}
			a^{v-1} \left(\frac{\pi v}{\sin\pi v}\right) \cos[(1-v)\beta] = Re\mathcal{F}.
			\label{Equation: The Re part of mathcalF}
		\end{equation}
		Getting $a$ and $\beta$ back to $\sqrt{\left(\frac{M^2}{s}-1\right)^2+\frac{M^2\Gamma_{\rm tot}^2}{s^2}}$ and $\cos^{-1}\left(\frac{\left(\frac{M^2}{s}-1\right)}{\sqrt{\left(\frac{M^2}{s}-1\right)^2+\frac{M^2\Gamma_{\rm tot}^2}{s^2}}}\right)$, respectively, one has
		\begin{equation}
			a \sin\beta=\frac{M\Gamma_{\rm tot}}{s}
			\label{Equation: ASinBeta}
		\end{equation}
		and
		\begin{equation}
			\mathcal{F} = \left( \frac{\pi v}{\sin\pi v} \right) \left(\frac{s}{M^2 - s - i M\Gamma_{\rm tot}}\right)^{1-v}.
			\label{Equation: The expressions of mathcalF.}
		\end{equation}
		
		With Eq. (\ref{Equation: The Im part of mathcalF}), (\ref{Equation: The Re part of mathcalF}) and (\ref{Equation: ASinBeta}), $I^{\rm R}$ and $I^{\rm CRI}$ can be expressed further as
		\begin{equation}
			I^{\rm R} \approx \frac{1}{M \Gamma_{\rm tot}} (1+\delta) Im\mathcal{F} \label{Equation: The approximate results 1}
		\end{equation}
		and
		\begin{equation}
			I^{\rm CRI} \approx - \frac{1}{s} (1+\delta) Re\mathcal{F}. \label{Equation: The approximate results 2}
		\end{equation}
		These are the simplified versions of the analytic formulae we need. \\
		
		\noindent \textbf{A.3 Comparisons of analytic formulae with numerical computing results} \\
		%\subsubsection*{\textbf{A.3 Comparisons of analytic formulae with numerical computing results} \\}
		%\label{Subsubsection: A.3 Comparisons of analytic formulae with numerical computing results}
		
		To check the validity of these analytic formulae, we compare them with numerical computing results. In the comparisons, the two integrals $I^{\rm R}$ and $I^{\rm CRI}$ are compared from $\sqrt{s}=M-10\Tw$ to $\sqrt{s}=M+10\Tw$ with $X$ set at 1 as well as $M$ and $\Tw$ at their PDG values \cite{PDG2016}. The results are shown in Fig. \ref{Figure: Comparisons with numerical computing results.}.
	\end{multicols}
	\begin{figure}[!h]
		\centering
		\subfigure{\includegraphics[width=0.4\textwidth]{ecm_rdperIresFN_rdperIresSN.eps}}
		\subfigure{\includegraphics[width=0.4\textwidth]{ecm_rdperIinfFN_rdperIinfSN.eps}}
		\caption{Comparisons of analytic formulae with numerical computing results. In the middle of the right-hand plot, the dotted line has a similar structure to the solid one. It does not show clearly in the plot because of its small scale.}
		\label{Figure: Comparisons with numerical computing results.}
	\end{figure}
	\begin{multicols}{2}
		The variables in the legends are defined as
		\begin{equation*}
			\left(\frac{\Delta I}{I}\right)^{\rm R|CRI}(\text{F}|\text{S},\text{N}) = \frac{I^{\rm R|CRI}(\text{F}|\text{S})-I^{\rm R|CRI}(\text{N})}{I^{\rm R|CRI}(\text{N})}.
		\end{equation*}
		Here, the symbols $|$, F, S and N are same as those used at the beginning of Subsections \ref{Subsection: Applications of the structure fuction method to eetoee and eetomumu} and \ref{Subsection: Comparisons with numerical computing results}.
		
		As can be seen from the dotted lines, the full versions of the analytic formulae agree very well with the numerical computing results. In fact, detailed numbers show that their relative differences are less than 0.01\%. Similarly, from the solid lines, one can see that except for $I^{\rm CRI}$ at energies very close to the \jpsi peak, the simplified versions of the analytic formulae agree with the numerical computing results to better than 0.1\%. The upward and downward peaks of $\left(\frac{\Delta I}{I}\right)^{\rm CRI}(\text{S},\text{N})$ at energies near the \jpsi peak is caused by the smallness of the absolute values (very close to 0) of $I^{\rm CRI}$, which makes $\sigma^{\rm CRI}$ values negligible when compared with their corresponding $\sigma^{\rm R}$ values. Because in the end, only the sum of $\sigma^{\rm R}$ and $\sigma^{\rm CRI}$ will be used in our data analysis, the peaks of $\left(\frac{\Delta I}{I}\right)^{\rm CRI}(\text{S},\text{N})$ are not worrying for us.
		
	\end{multicols}
	
\end{small}


%%\chapter*{}
%\addcontentsline{toc}{chapter}{Bibliography}

\bibliographystyle{apacite}
\bibliography{
biblio_ch_5}
%The first three lines must not be deleted. They are needed by the CPC template.
%\vspace{-1mm}
%\centerline{\rule{80mm}{0.1pt}}
%\vspace{2mm}

\begin{multicols}{2}
	\begin{thebibliography}{90}
		\vspace{3mm}
		\bibitem{potential models 1}A.M. Badalian and I.V. Danilkin, Phys. Atom. Nucl., $\bm{72}$: 1206 (2009)
		\bibitem{potential models 2}O. Lakhina and E.S. Swanson, Phys. Rev. D, $\bm{74}$: 014012 (2006)
		\bibitem{lattice calculations}J.J. Dudek, R.G. Edwards, and D.G. Richards, Phys. Rev. D, $\bm{73}$: 074507 (2006)
		\bibitem{BABAR}B. Aubert et al (BABAR Collaboration), Phys. Rev. D, $\bm{69}$: 011103 (2004)
		\bibitem{CLEO}G.S. Adams et al (CLEO Collaboration), Phys. Rev. D, $\bm{73}$: 051103 (2006)
		\bibitem{KEDR}V.V. Anashin et al (KEDR Collaboration), Phys. Lett. B, $\bm{685}$: 134 (2010)
		\bibitem{BESIII detector}M. Ablikim et al (BESIII Collaboration), Nucl. Instrum. Methods A, $\bm{614}$: 345 (2010)
		\bibitem{BABAYAGA}C.M. Carloni Calame, G. Montagna, O. Nicrosini et al, Nucl. Phys. B (Proc. Suppl.), $\bm{131}$: 48 (2004)
		\bibitem{ISR-FSR relation}V.S. Fadin, V.A. Khoze and A.D. Martin, Phys. Lett. B, $\bm{320}$: 141 (1994)
		\bibitem{STRUCTUREFUNCTIONMETHOD}E.A. Kuraev and V.S. Fadin, Sov. J. Nucl. Phys., $\bm{41}$: 466 (1985)
		\bibitem{Radiator 1}F.Z. Chen, P. Wang, J.M. Wu et al, HEP \& NP, $\bm{14}$ (7): 585 (1990) (in Chinese)
		\bibitem{Radiator 2}X.H. Mo, Measurement of $\psi(2S)$ Resonance Parameters, Ph.D. Thesis (Beijing: Institute of High Energy Physics, CAS, 2001) (in Chinese)
		\bibitem{Radiator 3}F.Z. Chen, P. Wang, C.M. Wu et al, BIHEP-EP-90-01
		\bibitem{KEDRpsip}V.V. Anashin et al (KEDR Collaboration), Phys. Lett. B, $\bm{711}$: 280 (2012)
		\bibitem{PDG2016}C. Patrignani et al (Particle Data Group), Chin. Phys. C, $\bm{40}$ (10): 1 (2016)
	\end{thebibliography}
\end{multicols}



\clearpage
\end{CJK*}
\end{document}
