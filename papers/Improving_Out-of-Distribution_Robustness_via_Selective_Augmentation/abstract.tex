\begin{abstract}
Machine learning algorithms typically assume that training and test examples are drawn from the same distribution. However, distribution shift is a common problem in real-world applications and can cause models to perform dramatically worse at test time. In this paper, we specifically consider the problems of subpopulation shifts (e.g., imbalanced data) and domain shifts. While prior works often seek to explicitly regularize internal representations or predictors of the model to be domain invariant, we instead aim to learn invariant predictors without restricting the model's internal representations or predictors. This leads to a simple mixup-based technique which learns invariant predictors via selective augmentation called LISA. LISA selectively interpolates samples either with the same labels but different domains or with the same domain but different labels. Empirically, we study the effectiveness of LISA on nine benchmarks ranging from subpopulation shifts to domain shifts, and we find that LISA consistently outperforms other state-of-the-art methods and leads to more invariant predictors. We further analyze a linear setting and theoretically show how LISA leads to a smaller worst-group error. Code is released in \href{https://github.com/huaxiuyao/LISA}{https://github.com/huaxiuyao/LISA}
\end{abstract}