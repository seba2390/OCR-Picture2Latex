% LaTeX template for creating an MNRAS paper
%
% v3.0 released 14 May 2015
% (version numbers match those of mnras.cls)
%
% Copyright (C) Royal Astronomical Society 2015
% Authors:
% Keith T. Smith (Royal Astronomical Society)

% Change log
%
% v3.0 May 2015
%    Renamed to match the new package name
%    Version number matches mnras.cls
%    A few minor tweaks to wording
% v1.0 September 2013
%    Beta testing only - never publicly released
%    First version: a simple (ish) template for creating an MNRAS paper

%%%%%%%%%%%%%%%%%%%%%%%%%%%%%%%%%%%%%%%%%%%%%%%%%%
% Basic setup. Most papers should leave these options alone.
\documentclass[fleqn,usenatbib]{mnras}
\usepackage{newtxtext,newtxmath}
\usepackage[T1]{fontenc}

\DeclareRobustCommand{\VAN}[3]{#2}
\let\VANthebibliography\thebibliography
\def\thebibliography{\DeclareRobustCommand{\VAN}[3]{##3}\VANthebibliography}


%%%%% AUTHORS - PLACE YOUR OWN PACKAGES HERE %%%%%
\usepackage{graphicx}	% Including figure files
\usepackage{amsmath}	% Advanced maths commands
% \usepackage{amssymb}	% Extra maths symbols
\usepackage[dvipsnames,svgnames]{xcolor}
%%%%%%%%%%%%%%%%%%%%%%%%%%%%%%%%%%%%%%%%%%%%%%%%%%

%%%%% AUTHORS - PLACE YOUR OWN COMMANDS HERE %%%%%
\usepackage{color}
\newcommand{\nianyi}[1]{\textcolor{RoyalBlue}{[{\bf Nianyi}: #1]}}
\newcommand{\yueying}[1]{\textcolor{Bittersweet}{{\bf Yueying}: #1}}
\newcommand{\mjt}[1]{\textcolor{DarkOrchid}{{\bf Michael}: #1}}
\newcommand{\spb}[1]{\textcolor{ForestGreen}{{\bf Simeon}: #1}}
\newcommand{\tiziana}[1]{\textcolor{red}{{\bf Tiziana}: #1}}
\newcommand{\cjd}[1]{\textcolor{brown}{{\bf Colin}: #1}}

% Please keep new commands to a minimum, and use \newcommand not \def to avoid
% overwriting existing commands. Example:
%\newcommand{\pcm}{\,cm$^{-2}$}	% per cm-squared

%%%%%%%%%%%%%%%%%%%%%%%%%%%%%%%%%%%%%%%%%%%%%%%%%%

%%%%%%%%%%%%%%%%%%% TITLE PAGE %%%%%%%%%%%%%%%%%%%

% Title of the paper, and the short title which is used in the headers.
% Keep the title short and informative.
\title[BH Dynamics and Mergers]{Dynamical Friction Modeling of Massive Black Holes in Cosmological Simulations and Effects on Merger Rate Predictions}

% The list of authors, and the short list which is used in the headers.
\author[N.Chen et al.]{
Nianyi Chen,$^{1}$\thanks{E-mail: nianyic@andrew.cmu.edu}
Yueying Ni,$^{1}$
Michael Tremmel,$^{2}$
Tiziana Di Matteo,$^{1,3,4}$ Simeon Bird, $^5$
Colin DeGraf,$^{1}$
\newauthor
 Yu Feng $^6$
\\
% List of institutions
$^{1}$McWilliams Center for Cosmology, Department of Physics, Carnegie Mellon University, Pittsburgh, PA 15213, USA\\
$^{2}$Astronomy Department, Yale University, P.O. Box 208120, New Haven, CT 06520, USA\\
$^3$NSF AI Planning Institute for Physics of the Future, 
Carnegie   Mellon  University, Pittsburgh, PA 15213, USA \\
$^{4}$ OzGrav-Melbourne, Australian Research Council Centre of Excellence for Gravitational Wave Discovery\\
$^{5}$ Department of Physics and Astronomy, University of California Riverside, Riverside, CA 90217, USA\\
$^{6}$Berkeley Center for Cosmological Physics and Department of Physics, University of California, Berkeley, CA 94720, USA
}

% These dates will be filled out by the publisher
\date{Accepted XXX. Received YYY; in original form ZZZ}

% Enter the current year, for the copyright statements etc.
\pubyear{2021}

% Don't change these lines
\begin{document}
\label{firstpage}
\pagerange{\pageref{firstpage}--\pageref{lastpage}}
\maketitle

% Abstract of the paper
\begin{abstract}
In this work we establish and test methods for implementing dynamical friction for massive black hole pairs that form in large volume cosmological hydrodynamical simulations which include galaxy formation and black hole growth. We verify our models and parameters both for individual black hole dynamics and for the black hole population in cosmological volumes. Using our model of dynamical friction (DF) from collisionless particles, black holes can effectively sink close to the galaxy center, provided that the black hole's dynamical mass is at least twice that of the lowest mass resolution particles in the simulation. Gas drag also plays a role in assisting the black holes' orbital decay, but it is typically less effective than that from collisionless particles, especially after the first billion years of the black hole's evolution. DF from gas becomes less than $1\%$ of DF from collisionless particles for BH masses $> 10^{7}$ M$_{\odot}$.
Using our best DF model, we calculate the merger rate down to $z=1.1$ using an $L_{\rm box}=35$ Mpc$/h$ simulation box.  We predict $\sim 2$ mergers per year for $z>1.1$ peaking at $z\sim 2$.
These merger rates are within the range obtained in previous work using similar-resolution hydro-dynamical simulations. We show that the rate is enhanced by factor of $\sim 2$ when DF is taken into account in the simulations compared to the no-DF run. This is due to $>40\%$ more black holes reaching the center of their host halo when DF is added.


\end{abstract}
% Select between one and six entries from the list of approved keywords.
% Don't make up new ones.
\begin{keywords}
gravitational waves -- methods: numerical -- quasars: supermassive black holes.
\end{keywords}

%%%%%%%%%%%%%%%%%%%%%%%%%%%%%%%%%%%%%%%%%%%%%%%%%%

%%%%%%%%%%%%%%%%% BODY OF PAPER %%%%%%%%%%%%%%%%%%


\section{Introduction}  \label{sec:introduction}

\newcommand\inexpIntro[3]{#1?(#2,#3).}
\newcommand\rinexpIntro[3]{*#1?(#2,#3).}
\newcommand\outexpIntro[3]{#1!(#2,#3).}
\newcommand\outatomIntro[3]{#1!(#2,#3)}

We propose a fully automated method for proving termination of \(\pi\)-calculus processes.
Although there have been a lot of studies on termination analysis for the \(\pi\)-calculus
and related calculi~\cite{Deng06IC,Demangeon07,SangiorgiTermination,KobayashiHybrid,Yoshida04IC,DBLP:journals/jlp/DemangeonHS10,Venet98SAS}, most of them have been rather theoretical,
and there have been surprisingly little efforts in developing  fully automated termination
verification methods and tools based on them. To our knowledge,
Kobayashi's \typical{}~\cite{TyPiCal,KobayashiHybrid} is the only exception that
can prove termination of \(\pi\)-calculus processes (extended with natural numbers)
fully automatically, but its termination analysis is quite limited (see Section~\ref{sec:relatedwork}).

Our method is based on a reduction to termination analysis for sequential programs:
we translate a \(\pi\)-calculus process \(P\) to a sequential program \(S_P\), so that
if \(S_P\) is terminating, so is \(P\). The reduction allows us to use
powerful, mature methods and tools
for termination analysis of sequential programs~\cite{heizmann2016ultimate,freqterm,DBLP:conf/lics/PodelskiR04,Kuwahara2014Termination,DBLP:journals/cacm/CookPR11}.

The idea of the translation is to convert a chain of communications on replicated input
channels to a chain of recursive function calls of the target sequential program.
Let us consider the following Fibonacci process:
\begin{align*}
    & \rinexpIntro{\fib}{n}{r}
        \ifexp{n<2}{ \soutatom{r}{1} \\ &\quad}
                   { \nuexp{s_1} \nuexp{s_2} (\outatomIntro{\fib}{n-1}{s_1} \PAR \outatomIntro{\fib}{n-2}{s_2} \PAR \sinexp{s_1}{x}\sinexp{s_2}{y}\soutatom{r}{x+y}) \\}
    & \PAR \outatomIntro{\fib}{m}{r}
\end{align*}
Here, the process
$\rinexpIntro{\fib}{n}{r} \ldots$ is a function server that computes the \(n\)-th Fibonacci number
in parallel and returns the result to \(r\),
and $\outatom{\fib}{m}{r}$ sends a request for computing the \(m\)-th Fibonacci number;
those who are not familiar with the syntax of the \(\pi\)-calculus may wish to consult
Section~\ref{sec:targetlanguage} first.
To prove that the process above is terminating for any integer \(m\),
it suffices to show that there is no infinite chain of communications on $\fib$:
\[
    \fib(m,r) \to \fib(m_1,r_1) \to \fib(m_2,r_2) \to \cdots.
\]
We convert the process above to the following program:\footnote{The actual translation
  given later is a little more complex.}
\begin{verbatim}
 let rec fib(n) = if n<2 then () else (fib(n-1) [] fib(n-2)) in
 fib(m)
\end{verbatim}
Here, \texttt{[]} represents the non-deterministic choice.
Note that, although the calculation of Fibonacci numbers is not preserved,
for each chain of communications on \texttt{fib}, there is a corresponding
sequence of recursive calls:
\[
\mathtt{fib}(m) \to \mathtt{fib}(m_1) \to \mathtt{fib}(m_2) \to \cdots.
\]
Thus, the termination of the sequential program above implies the termination of
the original process.
As shown in the example above, (i) each communication on a replicated input channel
is converted to a function call, (ii) each communication on a non-replicated input
channel is just removed (or, in the actual translation, replaced by a call of
a trivial function defined by \(f(\seq{x})=(\,)\)), and (iii) parallel composition
is replaced by a non-deterministic choice.
We formalize the translation outlined above and prove its correctness.

The basic translation sketched above sometimes loses too much information.
For example, consider the following process:
\begin{align*}
    & \rinexpIntro{\pre}{n}{r} \soutatom{r}{n-1} \\
    & \PAR \rinexpIntro{f}{n}{r} \ifexp{n<0}{ \soutatom{r}{1} }
                                       { \nuexp{s} (\outatomIntro{\pre}{n}{s} \PAR \sinexp{s}{x}\outatomIntro{f}{x}{r}) } \\
    & \PAR \outatomIntro{f}{m}{r}
\end{align*}
The translation sketched above would yield:
\begin{verbatim}
  let pred(n) = n-1 in
  let rec f(n) = if n<0 then () else (pred(n) [] f(*)) in
  f(m)
\end{verbatim}
Here, \texttt{*} represents a non-deterministic integer: since we have removed
the input $\sinatom{s}{x}$, we do not have information about the value of \( x \).
As a result, the sequential program above is non-terminating, although the original
process is terminating.
To remedy this problem, we also refine the basic translation above by using a refinement
type system for the \(\pi\)-calculus. Using the refinement type system,
we can infer that the value of \(x\) in the original process is less than \(n\),
so that we can refine the definition of \texttt{f} to:
\begin{verbatim}
 let rec f(n) = ... else (pred(n) [] let x=* in assume(x<n);f(x))
\end{verbatim}
The target program is now terminating, from which
we can deduce that the original process is also terminating.
We have implemented an automated tool based on the refined translation above.

The contributions of this paper are summarized as follows.
\begin{itemize}
\item The formalization of the basic translation from the \(\pi\)-calculus
  (extended with integers) to sequential programs, and a proof of its correctness.
\item The formalization of a refined translation based on a refinement type system.
\item An implementation of the refined translation, including automated refinement type
  inference based on CHC solving, and experiments to evaluate the effectiveness of
  our method.
\end{itemize}

The rest of this paper is structured as follows.
Section~\ref{sec:targetlanguage} introduces the source and target languages
of our translation.
Section~\ref{sec:approach} 
formalizes the basic translation, and proves its correctness.
Section~\ref{sec:refinement} refines the basic translation by using a refinement type system.
Section~\ref{sec:implementation} reports an implementation and experiments.
Section~\ref{sec:relatedwork} discusses related work,
and Section~\ref{sec:conclusion} concludes the paper.

\section{The Simulations}
\label{sec:simulations}

\subsection{The Numerical Code}
\label{subsec:code}

We use the massively parallel cosmological smoothed particle hydrodynamic (SPH) simulation software, MP-Gadget \citep{Feng2016}, to run all the simulations in this paper. 
The hydrodynamics solver of MP-Gadget adopts the new pressure-entropy formulation of SPH \citep{Hopkins2013}.
We apply a variety of sub-grid models to model the galaxy and black hole formation and associated feedback processes already validated against a number of observables \citep[e.g.][]{Feng2016,Wilkins2017,Waters2016,DiMatteo2017,Tenneti2018,Huang2018,Ni2018,Bhowmick2018,Marshall2020,Marshall2021}. Here we review briefly the main aspects of these.
In the simulations, gas is allowed to cool through radiative processes~\citep{Katz}, including metal cooling. For metal cooling, we follow the method in \cite{Vogelsberger2014}, and scale a solar metallicity template according to the metallicity of gas particles.
Our star formation (SF) is based on a multi-phase SF model ~\citep{SH03} with modifications following~\cite{Vogelsberger2013}.
We model the formation of molecular hydrogen and its effects on SF at low metallicity according to the prescription of \cite{Krumholtz}. 
We self-consistently estimate the fraction of molecular hydrogen gas from the baryon column density, which in turn couples the density gradient to the SF rate.
We include Type II supernova wind feedback ~\citep[the model used in BlueTides][]{Feng2016,Okamoto2010} in our simulations, assuming that the wind speed is proportional to the local one dimensional dark matter velocity dispersion.

BHs are seeded with an initial seed mass of $M_{\mathrm {seed}} = 5 \times 10^5 M_{\odot}/h$ in halos with mass more than $10^{10} M_{\odot}/h$ if the halo does not already contain a BH. We model BH growth and AGN feedback in the same way as in the \textit{MassiveBlack} $I \& II$ simulations, using the BH sub-grid model developed in \cite{SDH2005,DSH2005} with modifications consistent with BlueTides. 
The gas accretion rate onto the BHs is given by Bondi accretion rate,
\begin{equation}
\label{equation:Bondi}
    \dot{M}_B = \alpha \frac{4 \pi G^2 M_{\rm BH}^2 \rho}{(c^2_s+v_{\rm rel}^2)^{3/2}},
\end{equation}
where $c_s$ and $\rho$ are the local sound speed and density of the cold gas, $v_{\rm rel}$ is the relative velocity of the BH to the nearby gas, and $\alpha=100$ is a numerical correction factor introduced by \citep{Springel2005b}. This can also be eliminated (without affecting the values of the accretion rate significantly) in favor of a more detailed modeling of the contributions in the cold and hot phase accretion \citep{Pelupessy2006}.


We allow for super-Eddington accretion in the simulation \citep[e.g.][]{Volonteri2005,Volonteri2015}, but limit the accretion rate to 2 times the Eddington accretion rate:
\begin{equation}
\label{equation:Meddington}
    \dot{M}_{\rm Edd} = \frac{4 \pi G M_{\rm BH} m_p}{\eta \sigma_{T} c},
\end{equation}
where $m_p$ is the proton mass, $\sigma_T$ the Thompson cross section, c is the speed of light, and $\eta=0.1$ is the radiative efficiency of the accretion flow onto the BH.
Therefore, the BH accretion rate is determined by:
\begin{equation}
    \dot{M}_{\rm BH} = {\rm Min} (\dot{M}_{B}, 2\dot{M}_{\rm Edd}).
\end{equation}


The SMBH is assumed to radiate with a bolometric luminosity $L_{\rm Bol}$ proportional to the accretion rate $\dot{M}_{\rm BH}$:
\begin{equation}
    L_{\rm Bol} = \eta \dot{M}_{\rm BH} c^2
\end{equation}
with $\eta = 0.1$ being the mass-to-light conversion efficiency in an accretion disk according to \cite{Shakura1973}.
5\% of the radiated energy is thermally coupled to the surrounding gas that resides within twice the radius of the SPH smoothing kernel of the BH particle. This scale is typically about $\sim 1-3\%$ of the virial radius of the halo.

The cosmological parameters used are from the nine-year Wilkinson Microwave Anisotropy Probe (WMAP) \citep{Hinshaw2013} ($\Omega_0=0.2814$, $\Omega_\Lambda=0.7186$, $\Omega_b=0.0464$, $\sigma_8=0.82$, $h=0.697$, $n_s=0.971$).
For our fiducial resolution simulations, the mass resolution is $M_{\rm DM} = 1.2 \times 10^7 M_\odot/h$ and $M_{\rm gas} = 2.4 \times 10^6 M_\odot/h$ in the initial conditions.
The mass of a star particle is $M_{*} = 1/4 M_{\rm gas} = 6 \times 10^5 M_\odot/h$. The gravitational softening length is $\epsilon_g = 1.5$ ckpc/$h$ in the fiducial resolution for both DM and gas particles. The detailed simulation and model parameters are listed in Tables \ref{tab:cons} and \ref{tab:norm}. 
\subsection{Gaussian Constrained Realization}
\label{subsec:CR}
% \nianyi{Need inputs from Yueying's paper here}

MBHs at high redshift typically reside in rare density peaks, which are absent in the small uniform box ($\sim 10$ Mpc/$h$) simulations. 
In order to test the dynamics for more massive BHs (with $M_{\rm BH} > 10^8 M_{\odot}$) in our small volume simulation, we apply the Constrained Realization (CR) technique \footnote{\url{https://github.com/yueyingn/gaussianCR}} to impose a relatively high density peak in the initial condition (IC), with peak height $\nu = 4 \sigma_0$ on scale of $R_G = 1$ Mpc/$h$.  

The prescription for the CR technique was first introduced by \cite{Hoffman1991} as an optimal way to construct samples of constrained Gaussian random fields.
This formalism was further elaborated and extended by \cite{vandeWeygaert1996} as a more general type of convolution format constraints.
The CR technique imposes constraints on different characteristics of the linear density field. 
It can specify density peaks in the Gaussian random field with any desired height and shape, providing an efficient way to study rare massive objects with a relatively small box and thus lower computational costs \citep[e.g.][]{Ni2020}.
In this study, we specify a $4 \sigma_0$ density peak in the IC of our $10$ Mpc/$h$ box, boosting the early formation of halos and BHs to study the dynamics of massive BHs. Before applying the peak height constraint, the highest density peak has $\nu = 2.4 \sigma_0$ and the largest BH has mass $<6\times 10^7M_\odot$ at $z=3$ in our fiducial model (\texttt{DF\_4DM\_G} in Table \ref{tab:cons}). After applying the $4 \sigma_0$ constraint, the largest BH has mass $3\times 10^8 M_\odot$ at $z=3$ in the same box.



\begin{table*}
\caption{Constrained Simulations}
\label{tab:cons}
\begin{tabular}{lccccccc}
\hline
Name & Lbox & ${\rm N}_{\rm part}$ & ${\rm M}_{\rm DM}$ & ${\rm M}_{\rm Dyn,seed}$ & $\epsilon_{\rm g}$ & BH Dynamics & Merging Criterion \\
& [$h^{-1}$Mpc] & & [$h^{-1} {\rm M}_\odot$] & [${\rm M}_{\rm DM}$] & [$h^{-1}{\rm kpc}$] & &  \\
\hline
NoDF\_4DM & 10 & $176^3$ & $1.2\times 10^7$ & 4 & 1.5 & gravity & distance\\
NoDF\_4DM\_G & 10 & $176^3$ & $1.2\times 10^7$ & 4 & 1.5 & gravity & distance \& grav.bound\\
DF\_4DM & 10 & $176^3$ & $1.2\times 10^7$ & 4 & 1.5 & gravity+DF & distance\\
Drag\_4DM\_G & 10 & $176^3$ & $1.2\times 10^7$ & 4 & 1.5 & gravity+Drag & distance \& grav.bound\\
DF+Drag\_4DM\_G & 10 & $176^3$ & $1.2\times 10^7$ & 4 & 1.5 & gravity+DF+Drag & distance \& grav.bound\\
DF\_4DM\_G & 10 & $176^3$ & $1.2\times 10^7$ & 4 & 1.5 & gravity+DF & distance \& grav.bound\\
DF\_2DM\_G & 10 & $176^3$ & $1.2\times 10^7$ & 2 & 1.5 & gravity+DF & distance \& grav.bound\\
DF\_1DM\_G & 10 & $176^3$ & $1.2\times 10^7$ & 1 & 1.5 & gravity+DF & distance \& grav.bound\\
DF(T15)\_4DM\_G & 10 & $176^3$ & $1.2\times 10^7$ & 4 & 1.5 & gravity+DF(T15) & distance \& grav.bound\\
DF\_HR\_4DM\_G & 10 & $256^3$ & $4\times 10^6$ & 4 & 1.0 & gravity+DF & distance \& grav.bound\\
DF\_HR\_12DM\_G & 10 & $256^3$ & $4\times 10^6$ & 12 & 1.0 & gravity+DF & distance \& grav.bound\\
\hline
\end{tabular}
\end{table*}

\begin{table*}
\caption{Unconstrained Simulations}
\label{tab:norm}
\begin{tabular}{lccccccc}
\hline
Name & Lbox & ${\rm N}_{\rm part}$ & ${\rm M}_{\rm DM}$ & ${\rm M}_{\rm Dyn,seed}$ & $\epsilon_{\rm g}$ & BH Dynamics & Merging Criterion \\
& [$h^{-1}$Mpc] & & [$h^{-1} {\rm M}_\odot$] & [${\rm M}_{\rm DM}$] & [$h^{-1}{\rm kpc}$] & &  \\
\hline
L15\_Repos\_4DM & 15 & $256^3$ & $1.2\times 10^7$ & 4 & 1.5 & reposition & distance\\
L15\_NoDF\_4DM & 15 & $256^3$ & $1.2\times 10^7$ & 4 & 1.5 & gravity & distance\\
L15\_NoDF\_4DM\_G & 15 & $256^3$ & $1.2\times 10^7$ & 4 & 1.5 & gravity & distance \& grav.bound\\
L15\_DF\_4DM & 15 & $256^3$ & $1.2\times 10^7$ & 4 & 1.5 & gravity+DF & distance\\
L15\_DF\_4DM\_G & 15 & $256^3$ & $1.2\times 10^7$ & 4 & 1.5 & gravity+DF & distance \& grav.bound\\
L15\_DF(T15)\_4DM\_G & 15 & $256^3$ & $1.2\times 10^7$ & 4 & 1.5 & gravity+DF(T15) & distance \& grav.bound\\
L15\_DF+drag\_4DM\_G & 15 & $256^3$ & $1.2\times 10^7$ & 4 & 1.5 & gravity+DF+Drag & distance \& grav.bound\\
L35\_NoDF\_4DM\_G & 35 & $600^3$ & $1.2\times 10^7$ & 4 & 1.5 & gravity+DF & distance \& grav.bound\\
L35\_DF+drag\_4DM\_G & 35 & $600^3$ & $1.2\times 10^7$ & 4 & 1.5 & gravity+DF+drag & distance \& grav.bound\\

\hline
\end{tabular}
\end{table*}
\section{BH Dynamics}
\label{sec:bh_model}
\subsection{BH Dynamical Mass}
\label{subsec:mdyn}
In our simulations, the seed mass of the black holes is $5\times 10^5 M_\odot/h$, which is 20 times smaller than the fiducial dark matter particle mass at $1.2\times 10^7 M_\odot/h$. Such a small mass of the BH relative to the dark matter particles will result in very noisy gravitational acceleration on the black holes, and causes instability in the black hole's motion as well as drift from the halo center. Moreover, as shown in previous works \citep[e.g.][]{Tremmel2015,Pfister2019}, under the low $M_{\rm BH}/M_{\rm DM}$ regime, it is challenging to effectively model dynamical friction in a sub-grid fashion.

To alleviate dynamical heating by the noisy potential due to the low $M_{\rm BH}/M_{\rm DM}$ ratio, we introduce a second mass tracer, the dynamical mass $M_{\rm dyn}$, which is set to be comparable to $M_{\rm DM}$ when the black hole is seeded. This mass is used in force calculation for the black holes, including the gravitational force and dynamical friction, while the intrinsic black hole mass $M_{\rm BH}$ is used in the accretion and feedback process. $M_{\rm dyn}$ is kept at its seeding value $M_{\rm dyn,seed}$ until $M_{\rm BH}>M_{\rm dyn,seed}$. After that $M_{\rm dyn}$ grows following the black hole's mass accretion. With the boost in the seed dynamical mass, the sinking time scale will be shortened by a factor of $\sim M_{\rm BH}/M_{\rm dyn}$ compared to the no-boost case. Note that the bare black hole sinking time scale estimated in the no-boost case could over-estimate the true sinking time, as the high-density stellar bulges sinking together with the black hole are not fully resolved \citep[e.g.][]{Antonini2012,Dosopoulou2017,Biernacki2017}.

 The boost we need to prevent dynamical heating depends on the dark matter particle mass $M_{\rm DM}$ (if we have high enough resolution the boost is no longer necessary), so we parametrize the dynamical mass in terms of the dark matter particle mass, $M_{\rm dyn,seed} = k_{\rm dyn} M_{\rm DM}$, instead of setting an absolute seeding dynamical mass for all simulations. We expect that as we go to higher resolutions where $M_{\rm DM}$ is comparable to $M_{\rm BH,seed}$, the dynamical seed mass should converge to the black hole seed mass, if we keep $k_{\rm dyn}$ constant. We study the effect of setting different $k_{\rm dyn}$ by running three simulations with the same resolution and dynamical friction models, but various $k_{\rm dyn}$ ratios. They are listed in Table \ref{tab:cons} as \texttt{DF\_4DM\_G}, \texttt{DF\_2DM\_G}, and \texttt{DF\_1DM\_G}, with $k_{\rm dyn}=4,2,1$, respectively.

To explore the effects of the BH seed dynamical mass on the motion and mergers of the black hole, we test a variety of $M_{\rm dyn,seed}$ values in our simulations. The comparison between different $M_{\rm dyn,seed}$ can be found in Appendix \ref{app:res}. 
\subsection{Modeling of Black Hole Dynamics}
\label{subsec:df}
\subsubsection{Reposition of the Black Hole}

Before introducing our dynamical friction implementations, we first describe a baseline model utilized by many large-volume cosmological simulations: the reposition model. As the name suggests, the reposition model of black hole dynamics places the black hole at the location of a local gas particle with minimum gravitational potential at each time step, in order to avoid the unrealistic motion of the black holes due to limited mass and force resolution. This is particularly preferred for large-volume, low-resolution cosmological simulations \citep[e.g.][]{Springel2005b, Sijacki2007, Booth2009,Schaye2015,Pillepich2018}, where the black hole mass is smaller than a star or gas particle mass and the BH can be inappropriately scattered around by two-body forces as well as the noisy local potential.

This simple fix of repositioning, however, comes with many disadvantages. For example, it may lead to higher accretion and feedback of the black holes, as they sink to the high-density regions too quickly. As was shown in \cite{Wurster2013} and \cite{Tremmel2017}, repositioning also leads to burstier feedback of the BHs, which is more likely to quench star-formation in the host galaxies. Moreover, repositioning leads to ill-defined velocity and non-smooth trajectories of the black hole particles. Because of the ill-defined velocity and extremely short orbital decay time, such methods cannot be reliably used for merger rate predictions without careful post-processing calculations to account for the orbital decays.

In our study, we use the reposition model as a reference for the black hole statistics, as it is still widely adopted in many existing simulations. We want to compare the dynamical friction models with the reposition model and quantify the effect of repositioning on BH mass growth and merger rate compared with the dynamical friction models.


\subsubsection{Dynamical Friction from Collisionless Particles}

When the black hole travels through a continuous medium or a medium consisting of particles with smaller masses than the black hole, it attracts the surrounding mass towards itself, leaving a tail of overdensity behind.  Dynamical friction is the resulting gravitational force exerted onto the black hole by this tail of overdensity \citep[e.g.][]{Chandrasekhar1943,Binney2008}. Dynamical friction causes the orbits of SMBHs to decay towards the center of massive galaxies \citep[e.g.][]{Governato1994,Kazantzidis2005}, and enables the black holes to stay at the high-density regions where they could go through efficient accretion and mergers.

We follow Equation (8.3) in \cite{Binney2008} for the acceleration of the black hole due to dynamical friction:

\begin{equation}
\label{eq:df_full}
    \mathbf{F}_{\rm DF} = -16\pi^2 G^2 M_{\rm BH}^2 m_{a} \;\text{log}(\Lambda) \frac{\mathbf{v}_{\rm BH}}{v_{\rm BH}^3} \int_0^{v_{\rm BH}} dv_a v_a^2 f(v_a),
\end{equation}
where $M_{\rm BH}$ is the black hole mass, $\textbf{v}_{\rm BH}$ is the velocity of the black hole relative to its surrounding medium, $m_a$ and $v_a$ are the masses and velocities of the particles surrounding the black hole, and $\text{log}(\Lambda)=\text{log}(b_{\rm max}/b_{\rm min})$ is the Coulomb logarithm that accounts for the effective range of the friction between $b_{\rm min}$ and $b_{\rm max}$(we will specify how we set these parameters later). $f(v_a)$ is the velocity distribution of the surrounding particles (unless we explicitly state otherwise, all variables involving the black hole's surrounding particles are calculated using stars and dark matter particles). Here we have assumed an isotropic velocity distribution of the particles surrounding the black hole, so that we are left with an 1D integration. 

We test two different numerical implementations of the dynamical friction (DF) in our simulations: one with a more aggressive approach which likely overestimates the effective range of DF, but could be more suitable for large-volume simulations (we refer to it as DF(fid) in places where we carry out explicit comparisons between the two DF models, and drop the 'fid' in all other places); the other with a more conservative method which aims to only account for the DF below the gravitational softening length, and is well-tested for smaller volume, high-resolution simulations \citep{Tremmel2015} (we refer to it as DF(T15)).

We begin by introducing the DF(fid) model. In this model, we further follow the derivation in \cite{Binney2008}, and approximate $f(v_a)$ by the Maxwellian distribution, so that Equation \ref{eq:df_full} reduces to:
\begin{equation}
    \label{eq:H14}
    \mathbf{F}_{\rm DF,fid} = -4\pi \rho_{\rm sph} \left(\frac{GM_{\rm dyn}}{v_{\rm BH}}\right)^2  \;\text{log}(\Lambda_{\rm fid}) \mathcal{F}\left(\frac{v_{\rm BH}}{\sigma_v}\right) \frac{\bf{v}_{\rm BH}}{v_{\rm BH}}.
\end{equation}
Here $\rho_{\rm sph}$ is the density of dark matter and star particles within the SPH kernel (we will sometimes refer to these particles as "surrounding particles") of the black hole. All other definitions follow those of Equation \ref{eq:df_full}, except that we have substituted $M_{\rm BH}$ with $M_{\rm dyn}$ following the discussion in \ref{subsec:mdyn}.
The function $\mathcal{F}$ defined as:
\begin{equation}
    \label{eq:fx}
    \mathcal{F}(x) =  \text{erf}(x)-\frac{2x}{\sqrt{\pi}} e^{-x^2}, \;
    x=\frac{v_{\rm BH}}{\sigma_v}
\end{equation}
is the result of analytically integrating the Maxwellian distribution, where $\sigma_v$ is the velocity dispersion of the surrounding particles.

The subscript "fid" in $\text{log}(\Lambda)$ means that this definition of $\Lambda$ is specific to the DF(fid) model, with
\begin{equation}
    \Lambda_{\rm fid} = \frac{b_{\rm max,fid}}{(GM_{\rm dyn})/v_{\rm BH}^2}, \; b_{\rm max,fid} = 10\text{ ckpc}/h.
\end{equation}
Note that here we have defined $b_{\rm max}$ as a constant roughly equal to 6 times the gravitational softening. As there is no general agreement on the distance above which dynamical friction is fully resolved, we tested several values ranging from $\epsilon_g$ to $20\epsilon_g$. We found that values above $2\epsilon_g$ are effective in sinking the black hole, although a smaller $b_{\rm max}$ tends to result in more drifting black holes at higher redshift. By using this definition, we are likely overestimating the effective range of dynamical friction. However, we find this over-estimation necessary in the early stage of black hole growth to stabilize the black hole motion.

We also implement a more localized version of dynamical friction following  \cite{Tremmel2015} which we call DF(T15). Under the DF(T15) model, the dynamical friction is expressed as:

\begin{equation}
    \label{eq:T15}
    \mathbf{F}_{\rm DF,T15} = -4\pi \rho (v<v_{\rm BH}) \left(\frac{GM_{\rm dyn}}{v_{\rm BH}}\right)^2  \text{log}(\Lambda_{\rm T15}) \frac{\bf{v}_{\rm BH}}{v_{\rm BH}}.
\end{equation}
Here the surrounding density only accounts for the particles moving slower than the BH with respect to the environment. More formally,
\begin{equation}
\label{eq:rho}
    \rho (v<v_{\rm BH}) = \frac{M(<v_{\rm BH})}{M_{\rm total}} \rho_{\rm T15},
\end{equation}
where $M_{\rm total}$ is the total mass of the nearest 100 DM and stars, $M(<v_{\rm BH})$ is the fractional mass counting only DM and star particles with velocities smaller than the BH, and $\rho_{T15}$ is the density calculated from the nearest 100 DM/Star particles (note that in comparison, the SPH kernel contains 113 gas particles but far more collisionless particles (see Figure \ref{fig:k100_case1})). By using $\rho (v<v_{\rm BH})$ in place of $\rho_{\rm sph} \mathcal{F}$, we are approximating the velocity distribution of surrounding particles by the distribution of the nearest 100 collisionless particles. Another major difference from the DFsph model is the Coulomb logarithm, where in this model we define:
\begin{equation}
    \Lambda_{\rm T15} = \frac{b_{\rm max,T15}}{(GM_{\rm dyn})/v_{\rm BH}^2}, \; b_{\rm max,T15} = \epsilon_g.
\end{equation}
The choice of a lower $b_{\rm max}$ is consistent with the localized density and velocity calculations, and by doing so we have assumed that dynamical friction is fully resolved above the gravitational softening.


\subsubsection{Gas Drag}
\label{subsection:drag}
In addition to the dynamical friction from dark matter and stars, the black hole can also lose its orbital energy due to the dynamical friction from gas (to distinguish from dynamical friction from dark matter and stars, we will refer to the gas dynamical friction as "gas drag" hereafter). \cite{Ostriker1999} first came up with the analytical expression for the gas drag term from linear perturbation theory, and showed that in the transonic regime the gas drag can be more effective than the dynamical friction from collisionless particles. Although later studies show that \cite{Ostriker1999} likely overestimates the gas drag for gas with Mach numbers slightly above unity \citep[e.g.][]{Escala2004ApJ,Chapon2013}, simulations with gas drag implemented still demonstrate that this is an effective channel for black hole energy loss during orbital decays \citep[e.g.][]{Chapon2013,Dubois2013,Pfister2019}.

In order to investigate the relative effectiveness of DF and gas drag, we also include gas drag onto black holes in our simulations following the analytical approximation from \cite{Ostriker1999}:
\begin{equation}
\label{eq:drag}
    \mathbf{F}_{\rm drag} = -4 \pi\rho \left( \frac{G M_{\rm dyn}}{c_s^2} \right)^2 \times \mathcal{I(M)}\frac{\bf{v}_{\rm BH}}{v_{\rm BH}},
\end{equation}
where $c_s$ is the sound speed, $\mathcal{M} = \frac{| \mathbf{v}_{\rm BH} - \mathbf{v}_{\rm gas}|}{c_s}$ is the Mach number, and $\mathcal{I(M)}$ is given by:
\begin{align}
    \mathcal{I}_{\rm subsonic} &= \mathcal{M}^{-2} \left[ \frac{1}{2} \text{log}\left(\frac{1+\mathcal{M}}{1-\mathcal{M}}\right) -\mathcal{M}\right] \\
    \mathcal{I}_{\rm supersonic} &= \mathcal{M}^{-2} \left[ \frac{1}{2} \text{log}\left(\frac{\mathcal{M}+1}{\mathcal{M}-1}\right) -\text{log} \Lambda_{\rm fid} \right],
\end{align}
where $\text{log} \Lambda_{\rm fid}$ is the Coulomb logarithm defined similarly to the collisionless dynamical friction.





\begin{figure*}
\includegraphics[width=0.49\textwidth]{RESULTS/plots/dftest_L10_adv0df2b30mg1_dm5e7-010.png}
\includegraphics[width=0.49\textwidth]{RESULTS/plots/dftest_L10_adv0df2b30mg1_dm5e7-011.png}

\caption{Visualization of $4\sigma_0$ density peak of the \texttt{DF\_4\_DM\_G} simulation at $z=4.0$ and $z=3.5$. The brightness corresponds to the gas density, and the warmness of the tone indicates the mass-weighted temperature of the gas. We plot the black holes (\textbf{cross}) with mass $>10^6 M_\odot$, as well as the halos (subhalos) hosting them (\textbf{red circles} correspond to central halos, \textbf{orange circles} correspond to subhalos. The circle radius shows the virial radius of the halo; halos are identified by Amiga's Halo Finder(AHF)). This density peak hosts the two largest black holes in our simulations (\textbf{yellow cross}), and they are going through a merger along with the merger of their host halos between $z=4$ and $z=3$. For the black hole and merger case studies, we will use examples from the circled halos/black holes shown in this figure.} 
\label{fig:halos}
\end{figure*}


\subsection{Merging Criterion}
\label{subsec:merger}
In all of our simulations, we set the merging distance to be $2\epsilon_{\rm g}$, because the BH dynamics below this distance is not well-resolved due to our limited spatial resolution. We conserve the total momentum of the binary during the merger.

Under the baseline repositioning treatment of the BH dynamics, the velocity of the black hole is not a well-defined quantity. Therefore, in cosmological simulations with repositioning, the distance between the two black holes is often the only criterion imposed during the time of mergers (for example BlueTides \citep{Feng2016}, Illustris \citep{Vogelsberger2013} and IllustrisTNG \citep{Pillepich2018}). One problem with using only the distance as a merging criterion is that it can spuriously merge two passing-by black holes with high velocities, when in reality they are not gravitationally bound and should not merge just yet (or may never merge). Although some similar-resolution simulations such as EAGLE \citep{Crain2015,Schaye2015} also check whether two black hole particles are gravitationally bound, the black holes still do not have a well-defined orbit and sinking time due to the discrete positioning.

When we turn off the repositioning of the BHs to the nearby minimum potential, the BHs will have well-defined velocities at each time step (this is true whether or not we add the dynamical friction). This allows us to apply further merging criteria based on the velocities and accelerations of the black hole pair, and thus avoid earlier mergers of the gravitationally unbound pairs. Also, as the BH pairs now have well-defined orbits all the way down to the numerical merger time, we will be able to directly measure binary separation and eccentricity from the numerical merger, and use the measurements as the initial condition for post-processing methods without having to assume a constant initial value \citep[e.g.][]{Kelley2017}.

We follow \cite{Bellovary2011} and \cite{Tremmel2017}, and use the criterion
\begin{equation}
    \label{eq:merge_criterion}
    \frac{1}{2}|\bf{\Delta v}|^2 < \bf{\Delta a} \bf{\Delta r}
\end{equation}
 to check whether two black holes are gravitationally bound. Here $\bf{\Delta a}$,$\bf{\Delta v}$ and $\bf{\Delta r}$ denote the relative acceleration, velocity and position of the black hole pair, respectively. Note that this expression is not strictly the total energy of the black hole pair, but an approximation of the kinetic energy and the work needed to get the black holes to merge. Because in the simulations the black hole is constantly interacting with surrounding particles, on the right-hand side we use the overall gravitational acceleration instead of the acceleration purely from the two-body interaction.

\section{Case Studies of BH Models}
\label{sec:case}

%%%%%%%%%%%%%%%%%%%%%%%%%%%%
\begin{figure*}
\includegraphics[width=0.91\textwidth]{RESULTS/plots/big_plot2.pdf}

\caption{ The evolution of BH2 in Figure \ref{fig:halos} under different BH dynamics prescriptions. We show the distance to halo center (\textbf{top}), black hole mass (\textbf{middel}) and the $x$-component of the black hole velocity (\textbf{bottom}). Mergers are shown in vertical lines (thick dashed lines are major mergers ($q>0.3$), and thin dotted lines are minor mergers) \textbf{(a):} comparison between no-DF and DF models. DF clearly helps the black hole sink to the halo center and stay there. \textbf{(b):} Effects of DF from stars and dark matter compared with gas drag. DF has a stronger effect throughout, except that in the very early stage the drag-only model is comparable to the DF-only model. \textbf{(c)}: Comparison between the DF(fid) and DF(T15) model. In general, the DF(fid) model results in a more stable black hole motion and faster sinking, but the difference is small. \textbf{(d)}: Black hole dynamics with and without the gravitational bound check during mergers. Without the gravitational bound check, the black holes can merge while still moving with large momenta, and thereby get kicked out of the halo by the injected momentum.}
\label{fig:big_plot}
\end{figure*}

%%%%%%%%%%%%%%%%%%%%%%%%%%%%



Given the variety of models we have described so far, we first study the effect of different BH dynamics models by looking at the individual black hole evolution and black hole pairs using the constrained simulations. The details of these simulations and specific dynamical models are shown in Table \ref{tab:cons}. For all the constrained simulations, we use the same initial conditions, which enables us to do a case-by-case comparison between different BH dynamical models.

For the case studies, we choose to study the growth and merger histories of the two largest black holes and a few surrounding black holes within the density peak of our simulations. The halos and black holes at the $4\sigma_0$ density peak in \texttt{DF\_4DM\_G} are shown in Figure \ref{fig:halos}. The halos and subhalos shown in circles are identified with Amiga's Halo Finder \citep[AHF,][]{Knollmann2009}. The halos are centered at the minimum-potential gas particle within the halo, and the sizes of the circles correspond to the virial radius of the halo. Throughout the paper, we will always define the halo centers by the position of the minimum-potential gas particle, and we note that the offset between the minimum-potential gas and the halo center given by AHF (found via density peaks) is always less than 1.5 ckpc$/h$. The cyan crosses are black holes with mass larger than $10^6 M_\odot/h$, and the yellow crosses are the two largest black holes in the simulation. From the plot, we can see that in the \texttt{DF\_4DM\_G} simulation, most of the black holes already reside in the center of their hosting halos at $z=4$, although we also see some cases of wandering BHs outside of the halos.

%%%%%%%%%%%%%%%%%%%%%%%%%%%%%%%%
\subsection{Black Hole Dynamics Modeling}
\label{subsec:models}

To compare different dynamical models, we look at the distance between the black hole and the halo center $\Delta r_{\rm BH}$ (we will sometimes refer to this distance as "drift" hereafter), the black hole mass, and the velocity along the $x$ direction through the entire history of BH2 from Figure \ref{fig:halos}. 
 
 We evaluate the black hole drift with two approaches: at each time-step, we find the minimum potential gas particle within 10 ckpc$/h$ of the black hole and calculate the distance between this gas particle and the black hole. This is a quick evaluation of the drift that allows us to trace the black hole motion at each time step, but it fails to account for orbits larger than 10 ckpc$/h$, and the minimum-potential gas particle may not reside in the same halo as the black hole. Therefore, for each snapshot we saved, we define the drift more carefully by running the halo finder and calculate the distance between the black hole and the center of its host halo. Whenever the black hole is further than 9 ckpc$/h$ from the minimum potential gas particle, we take the distance from the two nearest snapshots and linearly interpolate in time between them. Otherwise we use the 
 distance to the local minimum potential gas particle calculated at each time step.

%%%%%%%%%%%%%%%%%%%%%%%%%%%%%%%%%%%%%%%%%%
\subsubsection{DF and No Correction}

Before calibrating our dynamical friction modeling, we first demonstrate the effectiveness of our fiducial DF model, \texttt{DF\_4DM\_G}, by comparing it with the no-DF run \texttt{NoDF\_4DM\_G} (note that throughout the paper, no-DF means no correction to the BH dynamics of any form besides the resolved gravity). We keep all parameters fixed except for the black hole dynamics modeling. The details of these simulations can be found in Table \ref{tab:cons}.

In Figure \ref{fig:big_plot}(a), we show the evolution of BH2 in Figure \ref{fig:halos} under the no-DF and the fiducial DF models. Without any correction to the black hole dynamics, even the largest black hole in the simulation does not exhibit efficient orbital decay throughout its evolution: the distance from the halo center is always fluctuating above $2\epsilon_g$. This is because the black hole does not experience enough gravity on scales below the softening length, and cannot lose its angular momentum efficiently. Now when we add the additional dynamical friction to compensate for the missing small-scale gravity, the black hole is able to sink to within 1 ckpc$/h$ of the halo centers in <200 Myr and remain there. 

The 90 ckpc$/h$ peak in the drift of the black hole marks the merger between BH1 and BH2 in Figure \ref{fig:halos}, when the host halo of BH2 merges into the host of BH1, and the halo center is redefined near the merger. After the halo merger, dynamical friction is able to sink the black hole to the new halo center and allows it to merge with the black hole in the other halo, whereas in the no-DF case we do not see the clear orbital decay of the black holes after the merger of their host halo until the end of the simulation.

Besides the drift, we also show the x-component of the black hole's velocity relative to its surrounding collisionless particles (lower panel). Here we show one component instead of the magnitude to better visualize the velocity oscillation. With dynamical friction turned on, the velocity of the black hole is more stable, as the black hole's orbit has already become small and is effectively moving together with the host halo. Without dynamical friction, the black hole tends to oscillate with large velocities around the halo center without losing its angular momentum.

The different dynamics of the black hole can also affect accretion due to differences in density and velocities, so we also look at the black holes' mass growth in the two scenarios (middle panel). The mass growths of the two black holes are similar under the two models, although when subjected to dynamical friction, the black holes have more and earlier mergers. Even though the black hole mass is less sensitive to the dynamics modeling, the merger rate predictions can be affected significantly as we will discuss later. 

Note that for our no-DF model, we have also boosted the dynamical mass to $4\times M_{\rm DM}$ at the early stage to prevent scattering by the dark matter and star particles. However, even after the boost, the black holes cannot lose enough angular momentum to be able to stay at the halo center. This means that even though dynamical heating is alleviated through the large dynamical mass, the sub-resolution gravity is still essential in sinking the black hole to the host halo center.

%%%%%%%%%%%%%%%%%%%%%%%%%%%%%%%%%%%%%%%%%%

\subsubsection{Dynamical Friction and Gas Drag}
\label{subsec:drag}
\begin{figure*}
\includegraphics[width=0.49\textwidth]{RESULTS/plots/drag_df2drg3b10mg1_4dm8406903.pdf}
\includegraphics[width=0.49\textwidth]{RESULTS/plots/DF_drag.pdf}

\caption{Comparisons between DF and hydro drag. \textbf{Left:} comparison for a single black hole. In the top panel we show the magnitude of the DF (\textbf{red}) and gas drag (\textbf{blue}) relative to gravity for the same black hole, in the \texttt{DF+Drag\_4DM\_G} run. During the early stage of the black hole evolution, DF and gas drag have comparable effect, while after $z=7.5$ the gas drag becomes less and less important, as the gas density decreases relative to the stellar density (\textbf{middle}), and the black hole velocity goes into the subsonic regime (\textbf{lower}). \textbf{Right:} Ratio between DF and gas drag for all black holes. We plot the ratio both as a function of redshift (\textbf{top}) and as a function of time after a black hole is seeded (\textbf{bottom}). The orange lines represent the logarithmic mean of the scatter. The $F_{\rm DF}/F_{\rm drag}$ ratio depends strongly on the evolution time of the black hole: the longer the black hole evolves, the less important the drag force is. However, there is not a strong correlation between redshift and the $F_{\rm DF}/F_{\rm drag}$ ratio.}
\label{fig:drag}
\end{figure*}


\begin{figure*}
\includegraphics[width=0.33\textwidth]{RESULTS/plots/ratio_Mbh.pdf}
\includegraphics[width=0.66\textwidth]{RESULTS/plots/hexbin.pdf}
\caption{\textbf{Left:} Scattering relation between the $F_{\rm DF}/F_{\rm drag}$ ratio and the black hole mass. For each black hole, we sample its mass at uniformly-distributed time bins throughout its evolution, and we show the scattered density of all samples. DF has significantly larger effects over gas drag on larger BHs. We fit the scatter to a power-law shown in the orange line. \textbf{Right:} Scattering relation between the $F_{\rm DF}/F_{\rm drag}$ ratio and the BHs' distance to the halo center. Comparing with the BH mass, we do not see a clear dependence of the $F_{\rm DF}/F_{\rm drag}$ ratio on the distance to halo center. For BHs at all locations within the halo, DF is in general larger than the gas drag.}
\label{fig:drag_scatter}
\end{figure*}

In the previous subsection, we've only included collisionless particles (DM+Star) when modeling the dynamical friction, now we will look into the effects of dynamical friction of gas (gas drag) in comparison with the collisionless particles in the context of our simulations.

From Equation \ref{eq:H14} and \ref{eq:drag}, the relative magnitudes of DF and drag mainly depend on two components: the relative density of DM+stars versus gas, and the values of $\mathcal{F}(x)$ and $\mathcal{I(M)}$. \cite{Ostriker1999} has shown that when a black hole's velocity relative to the medium falls in the transonic regime (i.e. near the local sound speed), $\mathcal{I}$ is a few times higher than $\mathcal{F}$, while in the subsonic and highly supersonic regimes $\mathcal{I}$ is smaller or equal to $\mathcal{F}$. Therefore, we would expect the gas drag to be larger when the black hole is in the early sinking stage with a relatively high velocity and a high gas fraction. 

In Figure \ref{fig:drag}, the left panel shows the comparison between the magnitude of DF and gas drag through different stages of the black hole evolution, as well as the factors that can alter the effectiveness of the gas drag. In the very early stages ($z>7.5$) of black hole evolution, DF and gas drag have comparable effects, while after $z=7.5$ the gas drag becomes significantly less important and almost negligible compared with DF. The reason follows what we have discussed earlier: the gas density decreases relative to the stellar density (shown in the middle panel), and the black hole's velocity relative to the surrounding medium goes into the subsonic regime as a result of the orbital decay (shown in the lower panel). Around $z=3.5$, there is a boost in the black hole's velocity due to disruption during a major merger with a larger galaxy and black hole. The effect of gas is again raised for a short period of time (although still subdominant compared to the DF).

In Figure \ref{fig:big_plot}(b) we plot the black hole evolution for the DF-only (\texttt{DF\_4DM\_G}), drag-only (\texttt{Drag\_4DM\_G}), and DF+drag (\texttt{DF+Drag\_4DM\_G}) simulations.
Both the drag-only and DF-only models are effective in sinking the black hole at early times ($z>7$). However, at lower redshifts, the gas drag is not able to sink the black hole by itself, whereas DF is far more effective in stabilizing the black hole at the halo center. For this reason, in low-resolution cosmological simulations, dynamical friction from collisionless particles is necessary to prevent the drift of the black holes out of the halo center.

To further illustrate the relative importance between DF and gas drag for the entire BH population, we examine the dependencies of the $F_{\rm DF}/F_{\rm drag}$ on variables related to the BH evolution for all BHs in the \texttt{DF+Drag\_4DM\_G} simulation. First, in the right panel of Figure \ref{fig:drag} we show the time evolution of $F_{\rm DF}/F_{\rm drag}$. The top panel shows the ratio as a function of cosmic time, while the bottom panel shows the ratio as a function of each BH's seeding time. The DF/Drag ratio has a wide range for different BHs, but overall DF is becoming larger relative to the gas drag as the black hole evolves. From the mean value of the DF/drag ratio, we see that when the black holes are first seeded, DF is only a few times larger than the gas drag. After a few Gyrs of evolution, DF becomes 2-3 orders of magnitude larger than the gas drag. However, there is not a strong correlation between redshift and the $F_{\rm DF}/F_{\rm drag}$ ratio. 

In the left panel of Figure \ref{fig:drag_scatter}, we show the scattering relation between the $F_{\rm DF}/F_{\rm drag}$ ratio and the black hole mass $M_{\rm BH}$. We see a strong correlation between the $F_{\rm DF}/F_{\rm drag}$ ratio and the black hole mass: DF has significantly larger effects over gas drag on larger BHs, although the range of the ratio is large ar the low mass end. We fit a power-law to the median of the scatter:
\begin{equation}
    \frac{F_{\rm DF}}{F_{\rm drag}} = 250 \left(\frac{M_{\rm BH}}{10^7 M_\odot}\right)^{1.7},
\end{equation}
which roughly characterize the effect of the two forces on BHs of different masses. From this relation we see that for BHs with masses $>10^7 M_\odot$, gas drag is in general less than $1\%$ of DF. Finally, the right panels show the relation between the $F_{\rm DF}/F_{\rm drag}$ ratio and the BH's distance to the halo center: there is not a strong dependency on the BH's position within the halo.


%%%%%%%%%%%%%%%%%%%%%%%%%%%%%%%%%%%%%%%%%%


\subsubsection{Comparisons with the T15 Model}
\label{subsec:df100}
\begin{figure}
\includegraphics[width=0.5\textwidth]{RESULTS/plots/compare_kernel.pdf}

\caption{ Comparison between different components in the two dynamical friction models, DF(fid) (\textbf{red}) and DF(T15) (\textbf{blue}) (see Section \ref{sec:bh_model} for descriptions). We show the number of stars and dark matter particles included in the DF density and velocity calculation (\textbf{top panel}), the density used for DF calculation (\textbf{second panel}), the Coulomb logarithm used in the two methods (\textbf{third panel}), the velocity of the BH relative to the surrounding particles (\textbf{forth panel}, note that the "surrouding particles" are defined differently for the two models), and the magnitude of DF relative to gravity (\textbf{bottom panel}). The higher DF in the DF(fid) model at $z>8$ is due to the larger Coulomb logarithm. After $z\sim 7$, the higher density of DF(T15) due to more localized density calculation counterbalances its lower $\text{log}(\Lambda)$, resulting in similar DF between $z=8$ and $z=3.5$. During the halo merger at $z=3.5$, the DF(fid) model included particles from the target halo into the density calculation, and therefore yields larger DF during the merger.}
\label{fig:k100_case1}
\end{figure}


For the collisionless particles, we test and study two different implementations for the dynamical friction: DF(fid) and DF(T15) (see Section \ref{sec:bh_model} for detailed descriptions). In Section \ref{sec:bh_model} we pointed out three main differences between them: different kernel sizes (SPH kernel vs. nearest 100 DM+star), different definitions of $b_{\rm max}$ (10 ckpc vs. 1.5 ckpc$/h$), and different approximation of the surrounding velocity distribution (Maxwellian vs. nearest 100-sample distribution). Essentially, these differences mean that DF(fid) is a less-localized implementation than DF(T15). Now we would like to evaluate the effectiveness of these two implementations and show how different factors affect the final dynamical friction calculation.

Figure \ref{fig:k100_case1} shows the relevant quantities in the DF computation for the two methods. The two kernels both contain $\sim 100$ dark matter and star particles at high redshift ($z>8$), but after that the SPH kernel (defined to include the nearest 113 gas particles) begins to include more and more stars and dark matter. The mass fraction of stars in the SPH kernel dominates over that of dark matter by $\sim 10$ times for a BH at the center of the galaxy. The larger kernel of DF(fid) has two effects: first, the DF density will be smoother over time; second, during halo mergers, the DF(fid) kernel can "see" the high-density region of the larger halo, which results in a higher DF near mergers compared to DF(T15). This is confirmed by the second panel, where we show the density for dynamical friction calculation from the two kernels. The densities calculated from the two kernels are similar in magnitude throughout the evolution, although the DF(T15) kernel yields slightly larger density due to its smaller size. Around the BH merger, the density in DF(fid) is larger due to its inclusion of the host halo's central region.

The third panel shows the Coulomb logarithm in the two models. Recall that $\Lambda = \frac{b_{\rm max}}{(GM_{\rm BH})/v_{\rm BH}^2}$, and so the Coulomb logarithm depends on the black hole's mass, its velocity relative to the surrounding particles, and the value of $b_{\rm max}$. From Figure \ref{fig:big_plot}(c), the mass of the DF(T15) black hole is slightly smaller, but the mass difference is small compared with the 6 times difference in $b_{\rm max}$. Given $b_{\rm max}$=10 ckpc$/h$ in DF(fid) and $b_{\rm max}$=1.5 ckpc$/h$ in DF(T15), we would expect the Coulomb logarithm to be larger for the former. However, there is yet another tweak: the $v_{\rm BH}^2$ term turns out to be significantly larger in the DF(T15) model(fourth panel). Note that in the DF(T15) model $v_{\rm BH}^2$ is calculated using only 100 surrounding particles, and for the high-density region we are considering here, the velocity of the nearest 100 particles is very noisy in time.  As we will show in Appendix \ref{app:df100}, for smaller black holes the difference in $v_{\rm BH}^2$ is not as large, and usually DF(fid) has a larger $\text{log}\Lambda$ due to its larger $b_{\rm max}$.

In Figure \ref{fig:big_plot}(c), we show the evolution of the black hole under these two models. At high redshift ($z>8$), due to the large $\text{log}(\Lambda)$, the black hole in the DF(fid) simulation sinks slightly faster to the halo center. Between $z=8$ and $z=3.5$, both models have similar dynamical friction (as discussed in the previous paragraph) and the motion and mass accretion are also similar. Then at $z=3.5$, within the host halo of the black hole major merger, dynamical friction in DF(fid) is again larger because the density kernel includes more particles from the high-density region in the target halo, and this leads to an earlier merger time.

Overall, the performance of the two models is similar. However, as we have seen in the velocity calculation of the black holes relative to the surrounding particles, DF(T15) could be too localized for simulations of our resolution ($\epsilon_g \sim 1$kpc/h) and is sometimes subject to numerical noise. Therefore, in our subsequent statistical runs we pick DF(fid) as our fiducial model, and will drop the 'fid' in its name hereafter.


%%%%%%%%%%%%%%%%%%%%%%%%%%%%%%%%%%%%%%%%%%%
\subsubsection{Gravitationally Bound Merging Criterion}
\label{subsec:bound_check}

The merging criterion can affect not only the merging time, but also the dynamics and evolution of the black holes. Naively, we might expect the distance-only merging to produce more massive black holes, because black holes are merged more easily. However, in many cases this is not true, and we will illustrate here through one example. 

Figure \ref{fig:big_plot}(d) shows the evolution of the same black hole with the same dynamical friction prescription, but different merging criteria. We note a drastic difference in the black hole's trajectories: while the BH in the gravitationally bound merger case is staying at the center of its host halo, the BH in the distance-only merger flies out of its host after a merger. This is because with the distance-only model, it is possible for one black hole to have a very large velocity at the time of the merger, since we do not limit the black hole's velocity. By momentum conservation, the black hole with a larger velocity can transfer the momentum to the other black hole (and the merger remnant) which might have already sunk to the halo center. The sunk black hole then drifts out of the halo center after a merger due to the large momentum injection. This is especially common in simulations where the black hole's dynamical mass is boosted, because the injected momentum is also boosted with mass and a smaller black hole in a satellite galaxy can easily kick a larger black hole out. If we add on the gravitational bound check, there will be more time for the black holes to lose their angular momentum, and so the injected momentum is far less, and in most cases does not kick each other out of the central region.








\subsection{Black Hole Mergers}
\begin{figure*}
\includegraphics[width=0.49\textwidth]{RESULTS/plots/merger_case1.pdf}
\includegraphics[width=0.49\textwidth]{RESULTS/plots/merger_case2.pdf}
\includegraphics[width=0.49\textwidth]{RESULTS/plots/stars2.png}
\includegraphics[width=0.49\textwidth]{RESULTS/plots/stars1.png}


\caption{The comparison between the distance of two merging black holes in the no-correction, DF(fid), DF(T15) and gas drag models in the early stage (\textbf{left}) and later stage (\textbf{right}) of the black hole evolution. For early mergers, the effect of the frictional forces (DF and drag) is not very prominent but still noticeable. The DF and gas drag both allow the black holes to merge faster compare to the no-DF case. For the later merger happening in a denser environment, the effect of dynamical friction is clear. However, the gas drag does not have a big effect on the black hole at this late stage compared with the no-DF case. The lower panels show the merging black holes within their host galaxies as well as their trajectories towards the merger in the \texttt{DF\_4DM\_G} run. The left images show the early phase of the orbital decay, and the right images show the later phase when the orbits get smaller.}
\label{fig:merger_case1}
\end{figure*}
Having seen the effect of different dynamical models on the evolution of individual black holes, next we will discuss how the dynamics, together with different BH merging criteria,  affect the evolution and mergers of the black holes.
 In particular, we want to study their merging time and trajectories before and after the mergers. Similar to the previous subsection, we will draw our examples from the two halos shown in Figure {\ref{fig:halos}}.
 
 %%%%%%%%%%%%%%%%%%%%%%%%%%%%%%%%%
\subsubsection{Effect of Dynamical  Friction Modeling}
\label{subsec:case_merger_calc}

We first look at how different dynamical models affect the time scale of black hole orbital decay and mergers. We pick two cases of mergers:  one is an early merger at $z>5$ when the black holes have not outgrown their dynamical masses; the other is a later merger at $z \sim 3.3$ when both BHs are larger than their seed dynamical masses (the major merger between BH1 and BH2 in Figure \ref{fig:halos}). Following \cite{Tremmel2015}, we also compute the dynamical friction time for the two mergers using Equation (12) - Equation (15) from \cite{Taffoni2003}:
\begin{equation}
\label{eq:tdf}
    t_{\rm DF} = 0.6\times 1.67\text{Gyr} \times \frac{r_c^2 V_h}{G M_s} \text{log}^{-1} \left( 1+\frac{M_{\rm vir}}{M_s} \right) \left(\frac{J}{J_c}\right)^\alpha,
\end{equation}
where $M_s$ is the mass of the smaller black hole (which we treat as the satellite), $M_{\rm vir}$ is the virial mass of the host halo of the larger black hole (found by AHF), $V_{h}$ is the circular velocity at the virial radius of the host, and $r_c$ is the radius of a circular orbit with the same energy as the satellite black hole's initial orbit. The last term $\left(\frac{J}{J_c}\right)^\alpha$ is the correction for orbital eccentricity, where $J$ is the angular momentum of the satellite, $J_c$ is the angular momentum of the circular orbit with the same energy as the satellite, and $\alpha$ is given by:
\begin{equation}
    \alpha \left( \frac{r_c}{R_{\rm vir}}, \frac{M_s}{M_{\rm vir}} \right) = 0.475 \left[ 1-\text{tanh} \left( 10.3 \left(\frac{M_s}{M_{\rm vir}}\right)^{0.33} - 7.5 \left(\frac{r_c}{R_{\rm vir}}\right) \right)  \right].
\end{equation}
In our calculation the virial radius, velocity, and mass are obtained from the AHF outputs, and the circular radius, orbit energy, and angular momentum are calculated by fitting the halo density profile to the NFW profile.

Figure \ref{fig:merger_case1} shows distances between two merging black holes in the no-DF, DF(fid), DF(T15), and gas drag models in the early and later stages of their evolution. For the early merger, the effect of the frictional forces (DF and drag) is not very big but still noticeable. The DF and gas drag have similar effects on the orbital decay at higher redshifts, consistent with our discussion in Section \ref{subsec:drag}. The DF(T15) model sinks the black hole a little slower than the DF(fid) model, but the difference is within $50$ Myrs. All three friction models allow the black holes to merge faster compare to the no-DF case by $\sim 150$ Myrs.

For the later merger, which takes place in a denser environment, the effect of dynamical friction is clearer: the dynamical friction allows the black holes to sink within the gravitational softening of the particles in $<200$ Myrs. Without dynamical friction the black hole's orbit does not have a clear decay below $2$ kpc and does not merge at the end of our simulation. Furthermore, the gas drag does not have a big effect on the black hole at this late stage compared with the no-correction case. This follows from our discussion in section \ref{subsec:drag} that gas drag is much less effective at lower redshift compared to dynamical friction.
 
In both plots, the yellow shaded region is the dynamical friction time from the analytical calculation in Equation \ref{eq:tdf}. Here we draw a band instead of a single line, because the black hole's orbit is not a strict ellipse, and the black hole is continuously losing energy. We calculate $t_{\rm DF}$ at multiple points between the first and second peak in the black hole's orbit (e.g. between $z=5.9$ and $z=5.7$ in the earlier case), and plot the range of those $t_{\rm DF}$. For both mergers, the analytical prediction is less than 150 Myrs later than the merger of the (fid) model. We note that the \cite{Taffoni2003} analytical $t_{\rm DF}$ is a fit to the NFW profiles, and the previous numerical and analytical comparisons on the black hole dynamical friction\citep[e.g.][]{Tremmel2015,Pfister2019} are performed in idealized NFW halos with a fixed initial black hole orbit. In our case, the halo profiles and black hole orbits are not directly controlled, and therefore deviation from the analytical prediction is expected. We will study such deviations statistically later in Section \ref{sec:merger_stats}.




%%%%%%%%%%%%%%%%%%%%%%%%%%%%%%%%%%%%%%%%%%%%%%%%%%%%%%%%%%
 \subsubsection{Effect of Gravitational Bound Check}

 In Section \ref{subsec:merger} we introduced two criteria which we use to perform black hole mergers in our simulations: we can merge two BHs when they are close in distance, and we can also require that the two BHs are gravitationally bounded in addition to the distance check. 
 
 In Figure \ref{fig:merger_case1} we show the difference in black holes' merging time with and without the gravitational bound criterion. The vertical dashed line marks the time that the two black holes in the \texttt{DF\_4DM\_G} simulation would merge if there was not the gravitational bound check. Without the gravitational bound check, the orbit of the black holes is still larger than 1 kpc when they merge, whereas with the gravitational bound check, the orbit size generally decays to less than 300 pc when the black holes merge. The merger without gravitational bound check generally makes the merger happen earlier by a few hundred Myrs (we will study the orbital decay time statistically in the next section). Therefore, for more accurate merger rate predictions as well as the correct accretion and feedback, it is necessary to apply the gravitational bound check during black hole mergers whenever the black hole has a well-defined velocity.
 




\section{Black Hole Statistics}
\label{sec:stats}
After looking at individual cases of black hole evolution, we now turn to the whole SMBH population in the simulations with different modeling of black hole dynamics. For statistics comparison, instead of using the $L_{\rm box} = 10$ Mpc$/h$ constrained realizations, we now use $L_{\rm box} = 15$ Mpc$/h$ unconstrained simulations. The details of our $L_{\rm box} = 15$ Mpc$/h$ simulations are shown in Table \ref{tab:norm}.

\subsection{Sinking of the Black Holes}
\label{subsec:drift}

\begin{figure}
\includegraphics[width=0.49\textwidth]{RESULTS/plots/L15_drift.pdf}
\caption{The effect of different BH dynamics modeling on BH position relative to its host. We include the reposition model (blue), no-DF model (orange),DF(T15) model (green), DF(fid) model (red) and the DF+drag model (purple). \textbf{Top:} The fraction of halos(subhalos) without a black hole for halos with masses above the black hole seeding mass at $M_{\rm halo} = 10^{10} M_\odot/h$. \textbf{Middle:} The fraction of halos without a central black hole ("central" means within $2\epsilon_g$ from the halo center identified by the halo finder), out of all halos with black holes. \textbf{Bottom:} Distribution of black holes' distance to its host halo center.} 
\label{fig:drift}
\end{figure}


\begin{figure}
\includegraphics[width=0.49\textwidth]{RESULTS/plots/L15_BHMF.pdf}
\caption{Mass functions for reposition, DF and no-DF simulations. With reposition (\textbf{blue}), we have the highest mass function and earlier formation of $10^8 M_\odot$ black holes. The no-DF simulations (\textbf{green}) have lower mass functions, which is expected due to low-accretion and merger rates from the black hole drifting. The dynamical friction model (\textbf{red}) yields a mass function in between.} 
\label{fig:bhmf}
\end{figure}


The primary reason for adding dynamical friction onto black holes within the cosmological simulations is to stabilize the black hole at the halo center (defined as the position of the minimum-position gas particle within the halo). Hence, we start by looking at the black holes' position relative to the host halos. Due to the resolution limit of our simulations, we would not expect the black holes to be able to sink to the exact minimum potential. Instead we consider a $<2\epsilon_g = 3$ ckpc$/h$ distance to be "good sinking".

In Figure \ref{fig:drift}, we show the statistics related to black holes' sinking status. We included the comparison between the reposition model (\texttt{L15\_Repos\_4DM}), the no-DF model (\texttt{L15\_NoDF\_4DM}), the two dynamical friction models (\texttt{L15\_DF\_4DM} and \texttt{L15\_DF(T15)\_4DM}) and the DF+drag model(\texttt{L15\_DF+drag\_4DM}). To start with, we simply count the fraction of halos without a black hole when its mass is already above the black hole seeding criterion (i.e. $10^{10}M_{\odot}/h$). The top panel shows the fraction of large halos without a BH for different models at $z=3.5$ and $z=2$. Surprisingly, the no-DF model ends up with the least halos without a black hole. This is because even though the black holes without dynamical corrections cannot sink effectively, the high dynamical mass still prevents sudden momentum injections from surrounding particles, and therefore most BHs still stays within their host galaxies. The dynamical friction models perform equally well, with $<10\%$ no-BH halos at the low-mass end. The reposition model, however, ends up with the most no-BH halos, even though repositioning is meant to pin the black holes to the halo center. This happens because under the repositioning model, the central black holes tend to spuriously merge into a larger halo during fly-by encounters, leaving the smaller sub-halo BH-less.

Next we look at where the black holes are located within their host galaxies. For all the halos with at least one black hole, we examine whether the black hole is located at the center (i.e.$<2\epsilon_g = 3$ckpc/h from the halo center). The middle panel of Figure \ref{fig:drift} shows the fraction of halos without a central BH. The no-DF model has significantly more halos without a central BH compared to the other models, with over half of the halos hosting off-center BHs. Among the three runs with dynamical friction, the DF(T15) and DF(fid) models have a similar fraction of halos ($\sim 20\%$) without a central BH, and we can see this fraction dropping from $z=3.5$ to $z=2$, meaning that many BHs are still in the process of sinking towards the halo center. When we further add the gas drag, $10\%$ more halos host at least one central BH, and the difference between the drag and no-drag central BHs is more prominent at high redshifts. 

Interestingly, the repositioning algorithm is not as efficient at sinking the BHs at $z=2$ as the DF. This is because our repositioning algorithm places the BHs at the minimum potential position within the accretion kernel, instead of within the entire halo. The majority of the offset between the BH positions and the halo center comes from the offset between the minimum-potential position accessible to the BH (i.e. minimum-potential in the accretion kernel) and the minimum-potential position in the halo. Such offset can be especially severe at lower redshift, when the size of the accretion kernel gets smaller and mergers happen more frequently, making it easier for the black holes to get stuck at a local minimum.


In the bottom panels we show the distributions of the black holes' distance to the halo centers under different models. For the no-DF run, again we see that the black holes fail to move towards the halo center at lower redshift, resulting in a much flatter distribution compared to all the other models. In comparison, when we add dynamical friction to the black holes, for both the DF(fid) and the DF(T15) models the distributions are pushed much closer to the halo center, with a peak around the gravitational softening length. When we then add the gas drag in addition to DF, the peak at $\epsilon_g$ becomes slightly higher than those in the DF-only runs. The combination of DF and gas drag, as we would expect from the case studies, is the most effective in sinking the black holes to the halo centers and stabilizing them. Finally, we plot the repositioning model for reference.  It does well in putting the black hole close to the minimum potential, and often the black holes can be located at the exact minimum-potential position (the distributions peak at 0 for $z=3.5$). However, as discussed in the previous paragraph, there are cases where the local minimum potential found by the repositioning algorithm does not coincide with the global minimum potential of the halo, and that is why we also see non-zero probability density for $\Delta r > 3$ ckpc$/h$ at $z=2$.

The statistics we have seen for the models above are consistent with the results from the case studies. This shows that even though for the case studies we have focused mainly on large black holes in one of the biggest halo, a similar trend still applies to other black holes in the cosmological simulations, which are embedded in smaller halos or subhalos.


 

%%%%%%%%%%%%%%%%%%%%%%%%%%%%%%%%%%%%%%%%%%%%%%%%%%%%%%%%%%%%%%%%%%%%%%%%%%%%%%

\subsection{Black Hole Mass Function}
\label{subsec:bhmf}
Next we look at how different dynamics affect the black hole mass function (BHMF). One problem with the repositioning method is that it places the black holes at the galaxy center too quickly, which could result in excess accretion and thus a higher mass function. On the other hand, if we do not add any correction to the black hole motion, many BHs will not go though efficient accretion and mergers, and we will see a lower mass function. We would expect the BHMF in the dynamical friction run to fall between the repositioning case and the no-DF case.

Figure \ref{fig:bhmf} shows the BHMF from the reposition(\texttt{L15\_Repos\_4DM}), dynamical friction (without gravitational bound check:\texttt{L15\_DF\_4DM}; with gravitational bound check:\texttt{L15\_DF\_4DM\_G}), and no-DF (without gravitational bound check: \texttt{L15\_NoDF\_4DM}; with gravitational bound check: \texttt{L15\_NoDF\_4DM\_G}) runs. The reposition model yields the highest mass function, and is the only simulation with more than one $10^8 M_{\sun}/h$ black holes at $z=2$. This is expected from the over-efficient BH mergers and the high-density surroundings in the reposition model. Moreover, it creates increasingly more massive BHs over time, as the increased merger rate produces a stronger effect over time. The no-DF runs produces the lowest mass function due to the off-centering, while the DF mass function falls between the reposition and no-DF case as we expected. 

Naively, we would expect the models without gravitational bound checks to produce a higher mass function, because it allows for easier mass-accretion via mergers. However, as discussed in Section \ref{subsec:bound_check}, this is not the case if we compare the dashed lines and solid lines with the same colors. For example, under the DF model, the \texttt{L15\_DF\_4DM\_G} simulation forms more massive black holes than the \texttt{L15\_DF\_4DM} simulation, especially at lower redshift. The reason can be traced back to what we have seen in Figure \ref{fig:big_plot}(d): when there is no gravitational bound check, the large momentum injection during a merger kicks the black hole out of the halo center, thus preventing the efficient growth of large black holes.

Considering the relatively large uncertainties due to the limited volume, the difference in the mass function is not very significant. We would expect other factors such as the black hole seeding, accretion and feedback to have a larger effect on the mass function compared to the dynamical models we show here \citep[e.g.][]{Booth2009}.
\subsection{Dynamical Friction Time and Mergers}
\label{sec:merger_stats}

\begin{figure}
\includegraphics[width=0.48\textwidth]{RESULTS/plots/tdf.pdf}
\caption{The delay of mergers due to the dynamical friction time. Here we compare the numerical dynamical friction time,$t_{\rm num}$, to the analytically calculated time (following Equation \ref{eq:tdf}) $t_{\rm analy}$. \textbf{Top left:} distribution of the dynamical friction time from numerical merger (blue) and analytical predictions (red). \textbf{Top right:} ratio between the numerical and analytical $t_{\rm df}$. Their difference is less than one order of magnitude in all merger cases. \textbf{Bottom:} dynamical friction time as a function of the virial mass of the host halo for the numerical (blue) merger and analytical predictions (red). The same merger event is linked by a grey line.} 
\label{fig:delay}
\end{figure}

\begin{figure}
\includegraphics[width=0.49\textwidth]{RESULTS/plots/L15_mergers.pdf}
\caption{The cumulative mergers for different BH dynamics and merging models. The reposition model (\textbf{blue solid}) predicts more than two times the total mergers compared with the other models. Without the gravitational bound check, the DF (\textbf{red dashed}) and the no-DF model (\textbf{green dashed}) predicts similar numbers of mergers, indicating that the first encounters of the black hole pairs are similar under the two models. However, if we add the gravitational bound check, the dynamical friction model (\textbf{red solid}) yields $\sim 50\%$ more mergers compared to the no-correction model. Adding the gas drag in addition to dynamical friction (\textbf{purple solid}) raises the mergers by a few. } 
\label{fig:merger_stats}
\end{figure}



Because the reposition method is used in most large-volume cosmological simulations, a post-processing analytical dynamical friction time is calculated in order to make more accurate merger rate predictions. Now that we have accounted for the dynamical friction on-the-fly, we want to study how our numerical mergers with dynamical friction compare against the analytical predictions, and how different dynamical models impact the black hole merger rate.

In Section \ref{subsec:case_merger_calc}, we compared the numerical merging time to the analytical predictions for two merger cases. Now we use the same method to calculate an analytical dynamical friction time for all black hole mergers in our \texttt{L15\_DF\_4DM\_G} simulation. For each pair, we begin the calculation at the time $t_{\rm beg}$ when the black hole pair first comes within 3 ckpc$/h$ of each other, as this mimics the merging time without the gravitational bound check, and is also close to the merging criterion under the reposition model. The numerical dynamical friction time $t_{\rm num}$ is the time between the numerical merger and $t_{\rm beg}$. The analytical dynamical friction time $t_{\rm analy}$ is calculated using the host halo information in the snapshot just before $t_{\rm beg}$ and the black hole information at the exact time-step of $t_{\rm beg}$.

Figure \ref{fig:delay} shows the comparison between the numerical and analytical dynamical friction times. In the top panel we show the distribution of the two times as well as the distribution of their ratio. We note that for all the mergers happening numerically, $t_{\rm analy}$ does not exceed 2 Gyrs, and most have $t_{\rm analy}$ less than 1 Gyr. This means that we do not have many fake mergers that shouldn't merge until much later (or never). Also, the ratio plot shows that the numerical and analytical times are always within an order of magnitude of each other, with most of the numerical mergers earlier than the analytical mergers. The numerical merger time is peaked between 100 Myrs and 1 Gyrs, whereas the analytical calculation yields a flatter distribution. We would expect $t_{\rm analy}$ to be longer than $t_{\rm num}$, both because we have a selection bias on $t_{\rm DF}$ by ending the similation at $z=2$, and because we numerically merge the black holes when their orbit is still larger than 3 ckpc$/h$. However, this does not explain why $t_{\rm analy}$ has a higher probability between 10 Myrs and 100 Myrs. 

To see the individual merger cases in the distribution more clearly, in the lower panel of Figure \ref{fig:delay} we plot all the numerical and analytical dynamical friction times as a function of the host halo's virial mass. From this figure we do not see a clear dependence of either dynamical friction times on the host halo's virial mass. There is also no strong correlation between the $t_{\rm num}/t_{\rm analy}$ ratio and the halo mass. We do not further investigate the discrepancies between the numerical and analytical results, as these results can vary significantly from system to system. 

We note that although the numerical model has free parameters (such as $b_{\rm max}$, $M_{\rm dyn, seed}$) that can impact the merging time (but see Appendix \ref{app:merger_param}), it can account for the immediate environment around black hole and adjust the dynamical friction on-the-fly. More importantly, it also accounts for the interaction between the satellite BH and its own host galaxy, which could reduce the sinking time significantly \citep[e.g.][]{Dosopoulou2017}. 
The analytical model, though verified by N-body simulations, does not react to the environment of the merging galaxies by always assuming an NFW profile. Moreover, it only models the sinking of a single BH without embedding it in its host galaxy. Therefore, we expect the numerical result to be a more realistic modeling of the binary sinking process.

After comparing the DF model against the analytical prediction, next we compare different numerical models in terms of the black hole merger rate. Figure \ref{fig:merger_stats} shows the cumulative mergers from $z=8$ to $z=2$. We have included comparisons between the reposition, dynamical friction and no-DF models, both with and without the gravitational bound check. The reposition model predicts more than twice the total number of mergers compared to the other models. Without the gravitational bound check, the DF and the no-DF models predict similar numbers of mergers, indicating that the first encounters of the black hole pairs are similar under the two models. However, if we add the gravitational bound check, the DF model yields $\sim 50\%$ more mergers compared to the no-DF model, because the addition of DF assists energy loss of the binaries and leads to earlier bound pairs. Finally, the merger rate is not very sensitive to adding the gas drag: the merger rate in the DF-only model is similar to that of the DF+drag model. This can be foreseen in the comparison shown in Figure \ref{fig:drag}, where the gas drag is subdominant in magnitude.


\section{Merger Rates in the 35Mpc/h Simulations}
\label{sec:L35}

\begin{figure*}
\includegraphics[width=0.99\textwidth]{RESULTS/plots/hist.pdf}
\caption{ \textbf{Left:} Distribution of the mass of the smaller black hole ($M_s$), and distribution of the total mass of the binary ($M_{\mathrm{tot}}$). For both simulations, the mergers in which at least one of the black holes is slightly above the seed mass dominate. The most massive binary has a total mass of $3\times 10^8 M_\odot$. \textbf{Middle:} The mass ratio $q$ between the two black holes in the binary. We see a peak at $\text{log(q)}=-0.5$, corresponding to pairs in which one BH is about three times larger than the other. \textbf{Right:} Scatter of the two black hole masses in the binaries, binned by redshift. To separate the scatter in the two simulations, for the DF+drag run we take $M_1$ to be the mass of the larger BH, while for the NoDF run $M_2$ is the larger BH.}
\label{fig:hist}
\end{figure*}

\begin{figure}
\includegraphics[width=0.49\textwidth]{RESULTS/plots/rates.pdf}
\caption{Merger rate per year of observation per unit redshift predicted from our \texttt{L35\_DF+drag\_4DM\_G} (\textbf{purple}) and \texttt{L35\_NoDF\_4DM\_G} (\textbf{blue}) simulations. 
For comparison, we also show the the prediction from recent hydro-dynamical simulations. 
We include three simulations of similar mass-resolution: \citet{Volonteri2020} from the Horizon-AGN simulation (\textbf{gray}), \citet{Katz2020} (\textbf{yellow}) from the Illustris simulation and \citet{Salcido2016} from the EAGLE simulations (\textbf{pink}).
Since we do not apply any post-processing delays after the numerical mergers, we only compare to results without delays.}
\label{fig:rates}
\end{figure}

Based on all the previous test of BH dynamics modeling, we have reached the conclusion that the DF+drag model with $M_{\rm dyn} = 4 M_{\rm DM}$ is most capable of sinking the black hole to the halo center. Hence, we choose to use this model to run our larger-volume simulation \texttt{L35\_DF+drag\_4DM\_G} for the prediction of the BH coalescence rate. Besides this model, we also perform a same-size run without the dynamical friction, \texttt{L35\_NoDF\_4DM\_G}, as a lower limit for the predicted rate.  
Our \texttt{L35} simulations are run down to $z=1.1$. The black hole seed mass is $5\times10^5 M_\odot/h$ and the minimum halo mass for seeding is $10^{10} M_\odot/h$. The details of these two simulations are shown in Table \ref{tab:norm}.

\subsection{The Binary Population}
\label{subsec:L35_catalog}
Because this work mainly focuses on model verification and is not intended for accurate merger-rate predictions, we do not account for the various post-numerical-merger time delays. These delays can be caused by physical processes such as sub-ckpc scale dynamical friction, scattering with stars, gravitational wave driven inspiral and triple MBH systems \citep[e.g.][]{Quinlan1996,Sesana2007b,Vasiliev2015,Dosopoulou2017,Bonetti2018}. We consider all the numerical mergers as true black hole merger events. Without any post-process selection, there are 25224 black holes and 4237 mergers in the \texttt{L35\_DF+drag\_4DM\_G} run, and 27693 black holes and 2349 mergers in the \texttt{L35\_NoDF\_4DM\_G} run down to $z=1.1$.

Figure \ref{fig:hist} shows the distribution of the binary parameters for the mergers in our simulations. For both simulations, there is at least one black hole around the seed mass for most mergers, but the peak does not lie at the exact seed mass. The most massive binary has a total mass of $3\times 10^8 M_\odot$. For the mass ratio $q$ between the two black holes in the binary, we see a peak at $\text{log}(q)=-0.3$, corresponding to pairs in which one BH is about two times larger than the other. Finally, we show the scatter of the two progenitor masses. The low mass end of the population deviates more from $q=1$, while the majority of same-mass mergers come from the $5\times 10^6 M_\odot\sim 5\times 10^7 M_\odot$ mass range.

Comparing with previous simulations such as \cite{Salcido2016,Katz2020}, we do not see as many cases of seed-seed mergers, but our distribution in q is similar to that shown in \cite{Weinberger2017} where the larger progenitor is a few times larger than the small progenitor. This is due to our larger black hole seed mass of $5\times10^5 M_\odot$ ($10^6 M_\odot$ in \cite{Weinberger2017}): the mass accretion in the early stage is proportional to $M_{\rm BH}^2$, and so during the time before the black hole mergers, our black holes accrete more mass compared to the simulations with smaller seeds. This explains why both of our black holes in the binaries are not peaked at the exact seed mass.


\subsection{Merger Rate Predictions}
\label{subsec:L35_rates}
We use the binary population shown in the previous section to predict the merger rate observed per year per unit redshift.
The merger rate per unit redshift per year is calculated as:
\begin{equation}
     \frac{dN}{dz\;dt} =  \frac{N(z)}{\Delta z V_{c,sim}} \frac{dz}{dt} \frac{dV_c(z)}{dz}\frac{1}{1+z},
\end{equation}
where $N(z)$ is the total number of mergers in the redshift bin $z$, $\Delta z$ is the width of the redshift bin, $V_{c,sim}$ is the comoving volume of our simulation box and $dV_c(z)$ is the comoving volume of the spherical shell corresponding to the $z$ bin. 

We compare our results against recent predictions from hydro-dynamical simulations of similar resolution, \cite{Salcido2016}, \cite{Katz2020} and \cite{Volonteri2020}. Here we briefly summarize relevant information about their merger catalogs. The Ref-L100N1504 simulation in the EAGLE suite used in \cite{Salcido2016} has an $2^3$ times larger simulation box and slightly higher resolution than our simulations. They seed $1.4\times 10^5M_\odot$ black holes in $1.4\times 10^{10}M_\odot$ halos. They adopt the reposition algorithm for black hole dynamics, but set a distance and relative speed upper limit on the repositioning to prevent black holes from jumping to satellites during fly-by encounters. We compare with their no-delay rate during the inspiral phase. The \textit{Illustris} simulation used in \cite{Katz2020} has a similar box size, resolution and BH dynamics to the Ref-L100N1504 simulation in EAGLE, except that their halo mass threshold for seeding BHs is $7\times 10^{10} M_\odot$. We compare against their ND model, in which mergers are also taken to occur at the numerical merger time without any delay processes. The Horizon-AGN simulation in \cite{Volonteri2020} is $4^3$ times larger than our simulation box, with $\sim 5$ times coarser mass resolution and a black hole seed mass of $10^5 M_\odot$. Instead of seeding BHs in halos above certain mass threshold, the seeding in \cite{Volonteri2020} is based on the local gas density and velocity dispersion, and seeding is stopped at $z=1.5$. For black hole dynamics, they apply dynamical friction from gas, but not from collisionless particles.

Figure \ref{fig:rates} shows our merger rate prediction in the \texttt{L35\_DF+drag\_4DM\_G} and  \texttt{L35\_NoDF\_4DM\_G} simulations. The \texttt{L35\_DF+drag\_4DM\_G} run predicts $\sim 2$ mergers per year of observation down to $z=1.1$, while the \texttt{L35\_NoDF\_4DM\_G} run predicts $\sim 1$. The merger rates from both simulations peak at $z\sim 2$. This factor-of-two difference between the two simulations is consistent with what we predicted in the $L_{\rm box} = 15$ Mpc$/h$ runs in Figure \ref{fig:merger_stats}. Although we did not run a $L_{\rm box} = 35$ Mpc$/h$ simulation with the repositioning model, we expect such a run to predict $5\sim 6$ mergers per year down to $z=1.1$ according to \ref{fig:merger_stats}.

Generally speaking, our simulations yield similar merger rates as the raw predictions from the previous works of comparable resolution. However, we still note some differences both in the overall rates and in the peak of the rates. We will now elaborate on the reasons for those discrepancies.

First, both of our simulations predict more mergers compared with the \cite{Katz2020} ND model prediction. This is surprising given that in the 15 Mpc$/h$ runs we saw $2\sim 3$ times more mergers when we used the reposition method like \cite{Katz2020} and \cite{Salcido2016} did, comparing to our DF+Drag model. Although \cite{Katz2020} cut out $\sim 25\%$ secondary seed mergers and binaries with extreme density profiles, their rate is still lower after adding the cut-out population. One major reason for the higher rate from our simulation compared to \cite{Katz2020} is the different seeding parameters we use: our minimum halo mass for seeding a black hole is $10^{10} M_\odot/h$, which is 5 times smaller compared with \cite{Katz2020}. Moreover, our seeds are a factor of 5 larger. Hence, we have a denser population of black holes in less-massive galaxies, which is likely to result in a higher merger rate even compared to the reposition model used in \textit{Illustris}.

Second, although the rates from EAGLE, Horozon-AGN and our \texttt{L35\_DF+drag\_4DM\_G} simulation cross over at $z\sim 2$, the slope of our merger rate is very different. \cite{Volonteri2020} predicts most mergers at $z\sim 3$, whereas the \cite{Salcido2016} rate peaks at $z\sim 1$. This difference can also be traced to the different seeding rate in the three simulations: in \cite{Salcido2016}, the seeding rate keeps increasing until $z\sim 0.1$, while we observe a drop in seeding rate at $z=3$ in our simulations. In \cite{Volonteri2020}, due to the different seeding mechanism, BH seeds form significantly earlier, leading to a peak in merger rate at a higher redshift. Hence the peak in the BH merger rate is strongly correlated with the peak in the BH seeding rate.

Finally, besides the effect due to different BH seed models on the merger rate, higher resolution can significantly increase the BH merger rates in the simulations. As was shown in previous work \citep[e.g.][]{Volonteri2020,Barausse2020}, dwarf galaxies in low-mass halos can have large numbers of (small mass) BH mergers, and so resolving such halos and galaxies can increase the BH merger rate significantly. The merger rate differences between high and low resolution and the associate choice for the seed models can lead to large differences in the predictions of merger rates than taking account DF in the BH dynamics. 


\begin{comment}
\begin{figure}
\includegraphics[width=\linewidth]{figs/beyond_tss_lesion.pdf}
\caption[]{End-to-End runtime lesion study of the entire MNIST dataset and the FMA featurized music dataset. Each of DROP's contributions provides a runtime improvement.}
\label{fig:beyond_lesion}
\end{figure}
\end{comment}



\section{Conclusion}
\label{sec:conclusion}

Advanced data analytics techniques must scale to rising data volumes. 
DR techniques offer a powerful toolkit when processing these datasets, with PCA frequently outperforming popular techniques in exchange for high computational cost. 
In response, we propose DROP, a new dimensionality reduction optimizer. 
DROP combines progressive sampling, progress estimation, and online aggregation to identify high quality low dimensional bases via PCA without processing the entire dataset by balancing the runtime of downstream tasks and achieved dimensionality. 
Thus, DROP provides a first step in bridging the gap between quality and efficiency in end-to-end DR for downstream \red{analytics}. 

%We revisit canonical operators for time series dimensionality reduction and the measurement study of~\cite{keogh-study}, and show that PCA is more effective than popular alternatives in the data mining literature often by a margin of over $2\times$ on average on gold-standard time series benchmark data sets with respect to output data dimension. More surprisingly, we empirically demonstrate that a small number of samples are sufficient to accurately characterize directions of maximum variance and obtain a high-quality low-dimensional transformation.



% %\documentclass[11pt, oneside]{article}   	% use "amsart" instead of "article" for AMSLaTeX format
%\documentclass[linenumbers,twocolumn]{aastex62}
\documentclass[twocolumn]{aastex62}
%\documentclass[reprint,nofootinbib]{revtex4-2}
%\usepackage{geometry}                		% See geometry.pdf to learn the layout options. There are lots.
%\geometry{letterpaper}                   		% ... or a4paper or a5paper or ... 
%\geometry{landscape}                		% Activate for rotated page geometry
%\usepackage[parfill]{parskip}    		% Activate to begin paragraphs with an empty line rather than an indent
%\usepackage{graphicx}				% Use pdf, png, jpg, or eps§ with pdflatex; use eps in DVI mode
%\usepackage{mathrsfs} 

\usepackage{amsmath}
\usepackage{amssymb}
\usepackage{empheq}
\usepackage{mathrsfs}
				
								% TeX will automatically convert eps --> pdf in pdflatex		
\usepackage{amssymb}

%SetFonts

%SetFonts

\newcommand{\G}{\mathcal{G}}
\newcommand{\M}{M_{\star}}
\newcommand{\Msun}{M_{\odot}}
\newcommand{\R}{\mathcal{R}}
\newcommand{\Ham}{\mathcal{H}}
\newcommand{\mn}{m_{\rm{N}}}
\newcommand{\an}{a_{\rm{N}}}
\newcommand{\en}{e_{\rm{N}}}
\newcommand{\Mn}{M_{\rm{N}}}
\newcommand{\order}{\mathcal{O}}




\begin{document}

%CTL throughout the article stands for CHECK THIS LATER

\title{The Stability Boundary of the Distant Scattered Disk}

\author{Konstantin Batygin}
\affiliation{Division of Geological and Planetary Sciences California Institute of Technology, Pasadena, CA 91125, USA}

\author{Rosemary A. Mardling}
\affiliation{School of Physics and Astronomy, Monash University, Victoria, 3800, Australia}

\author{David Nesvorn{\'y}}
\affiliation{Department of Space Studies, Southwest Research Institute, 1050 Walnut St., Suite 300, Boulder, CO 80302, USA}

\begin{abstract}
The distant scattered disk is a vast population of trans-Neptunian minor bodies that orbit the sun on highly elongated, long-period orbits. The orbital stability of scattered disk objects is primarily controlled by a single parameter -- their perihelion distance. While the existence of a perihelion boundary that separates chaotic and regular motion of long-period orbits is well established through numerical experiments, its theoretical basis as well as its semi-major axis dependence remain poorly understood. In this work, we outline an analytical model for the dynamics of distant trans-Neptunian objects and show that the orbital architecture of the scattered disk is shaped by an infinite chain of exterior $2:j$ resonances with Neptune. The widths of these resonances increase as the perihelion distance approaches Neptune's semi-major axis, and their overlap drives chaotic motion. Within the context of this theoretical picture, we derive an analytic criterion for instability of long-period orbits, and demonstrate that rapid dynamical chaos ensues when the perihelion drops below a critical value, given by $q_{\rm{crit}}=\an\,\big(\ln((24^2/5)\,(\mn/\Msun)\,(a/\an)^{5/2})\big)^{1/2}$. This expression constitutes an analytic boundary between the ``detached" and actively ``scattering" sub-populations of distant trans-Neptunian minor bodies. Additionally, we find that within the stochastic layer, the Lyapunov time of scattered disk objects approaches the orbital period, and show that the semi-major axis diffusion coefficient is approximated by $\mathcal{D}_a\sim(8/(5\,\pi))\,(\mn/\Msun)\,\sqrt{\G\,\Msun\,\an}\,\exp\big[-(q/\an)^2/2\big]$. We confirm our results with direct $N-$body simulations and highlight the connections between scattered disk dynamics and the Chirikov Standard Map. Implications of our results for the long-term evolution of minor bodies in the distant solar system are discussed.
\end{abstract}

\keywords{Orbital dynamics, Scattered disk objects, Perturbation theory}%Use showkeys class option if keyword
                              
%\maketitle

\section{Introduction} \label{sec:intro}

Among the various sub-populations of the icy debris that comprise the Kuiper belt, the most prominent -- both in terms of mass and radial extent -- is the scattered disk. A remnant of Neptune's early outward migration \citep{Nesvorny2018REV}, the scattered disk is largely made up of eccentric, low-inclination orbits that ``hug" the orbit of Neptune, maintaining a perihelion distance slightly above $q\gtrsim30\,$AU \citep{2020tnss.book...25M}. Interesting in its own right, the orbital architecture of the distant scattered disk is especially distinctive, as it provides an observational handle on the gravitational processes that have sculpted the outermost reaches of the solar system \citep{2004ApJ...617..645B, 2010ARA&A..48...47A, 2019PhR...805....1B, 2021arXiv210501065C}.

The dynamics of scattered disk objects (SDOs) have been studied in considerable detail over the past two and a half decades, and the general characteristics of their long-term evolution are relatively well understood (see \citealt{Saillenfest2020} and the references therein). Crudely speaking, objects with perihelion distance small enough to strongly interact with Neptune experience chaotic diffusion and eventually become Centaurs\footnote{Centaurs are broadly defined as objects with perihelion distance or semi-major axis that fall between the orbits of Jupiter and Neptune.}, or leave the solar system altogether. To be more precise, the survival probability of a chaotic SDO over the age of the sun is about $1\%$ \citep{Gomes2008}. Conversely, objects with large perihelia -- often referred to as the ``detached" population -- are immune to strong Neptune-induced perturbations, and simply orbit the sun on slowly precessing Keplerian orbits.

Today, orbital integration of scattered disk objects does not present a significant practical challenge. Well-tested symplectic integrators, predominantly based on the \citet{1991AJ....102.1528W} mapping, are widely available \citep{1998AJ....116.2067D, 1999MNRAS.304..793C, 2019MNRAS.485.5490R, 2019MNRAS.489.4632R}, bringing precise modeling of the distant solar system's long-term evolution within reach of virtually any modern Ghz-grade machine. Nevertheless, such numerical experiments can only solve for the emergent dynamics, not illuminate their theoretical basis. In other words, accurate realizations of orbital evolution can only expose \textit{what} the SDOs do, not \textit{why} they do it. Understanding the latter question requires a simplified analytic model. Here we develop such a model for the distant scattered disk with an eye towards quantifying its underlying dynamical structure and elucidating the processes that drive orbital diffusion, from analytic grounds. We begin by sketching out the statement of the problem. 

\paragraph{Statement of the Problem} The principal goal of the calculation we aim to carry out is easy to summarize: we wish to develop a simple theory for the long term behavior of highly eccentric, long-period minor bodies, subject to perturbations from Neptune. In other words, our goal is to solve the circular restricted three-body problem in a regime where the test particle possesses an orbital period much larger than that of the perturber, but still experiences material interactions with the planet, owing to the closeness of the perihelion distance to the planet's semi-major axis. The geometric setup of the problem is summarized in Figure \ref{Fig:SETUP}.

The circular restricted three-body problem is by no means a new problem, and the relevant literature spans centuries. Nevertheless, the vast majority of perturbation theory devoted to understanding the relevant dynamics is unsuitable for the problem at hand. Classical expansions of the planetary disturbing function (\citealt{1995CeMDA..62..193L, 2000Icar..147..129E} and the references therein) treat eccentricities and inclinations as small parameters, developing the governing Hamiltonian as a power-series in $e$ and $i$, while placing no constraints on the semi-major axis ratio with the exception of the formal requirement that the orbits do not cross. Conversely, the scattered disk is characterized by large (even near-unity) eccentricities, placing it outside of the domain of applicability of standard models.

As a means to circumvent the limitations of classical methods, various authors have reframed long-term evolution of the scattered disk as mapping problem. That is, rather than attempting to formulate a conventional perturbation theory, \citet{Malyshkin1999, 2004AJ....128.1418P, 2013Icar..222...20F, 2020PASP..132l4401K} envisioned the dynamics as a process wherein the test particle executes unperturbed Keplerian motion, with the exception of the perihelion, where it receives an energy kick of some magnitude that generally depends on the planetary mass and semi-major axis, as well as the particle's perihelion distance. Intuitive in its own right, the essence of this approach lies in the so-called \textit{Kepler Map} (see \citealt{2011NewA...16...94S} for a review). It is worth noting that this mapping was first derived by \citet{1986PhLA..117..328P}, and has since become an important tool for understanding a variety of physical phenomena, including those beyond the realm of dynamical astronomy \citep{Chirikov1989, 1987JPhB...20.5051G, 1988IJQE...24.1420C, 1988JPhB...21L.527J, 1994PhRvA..50..575S, 1998PhLA..241...53S}.

Despite the successes of the mapping approach in modeling chaotic motion at a vastly reduced computational cost, a full understanding of scattered disk dynamics remains incomplete. In particular, the crucial questions of which resonances underly the stochastic layer, and how the scattering process connects to a perturbative description of particle motion, remain to be elucidated. To address this issue, in this work we take an approach that is similar in spirit -- but not in detail -- to classical perturbation theories. More specifically, we adopt a description of the disturbing function as an infinite series developed in terms of a small parameter, which we take to be the semi-major axis ratio, $\alpha=\an/a$, rather than the eccentricity \citep{1962AJ.....67..300K,Laskar2010DISTFUNCT, 2013MNRAS.435.2187M}. As we show below, this Kaula-type expansion attractively lends itself to a simplified description of SDO dynamics and naturally illuminates the relationship between Neptune's exterior mean motion resonances and the scattering process.

The remainder of this paper is organized as follows. In section \ref{sec:anmodel}, we outline the basis of our analytical theory. The results are presented in section \ref{sec:results}. Particularly, in section \ref{sec:intmod}, we derive a simplified Hamiltonian model of particle motion. We formulate the stability boundary of the scattered disk in terms of a critical perihelion distance in section \ref{sec:chirikov}. We validate our analytic results with $N-$body simulations in section \ref{sec:numsim}. In section \ref{sec:standard}, we highlight the connection between our model and the Chirikov Standard Map, thus outlining the equivalence between our perturbative approach and scattering viewpoint. Finally, in section \ref{sec:analytic} we derive analytic estimates of the Lyapunov time and the semi-major axis diffusion coefficient within the scattered disk. We summarize and discuss our results in section \ref{sec:discuss}.

\begin{figure*}[t]
\centering
\includegraphics[width=0.8\textwidth]{SETUP.png}
\caption{Scattered disk dynamics modeled as a circular planar restricted three-body problem. A long-period scattered disk object (SDO) is envisioned to orbit the sun on a highly eccentric orbit with a perihelion distance that exceeds Neptune’s semi-major axis by a small margin. The SDO orbit is shown as a purple ellipse in the digram. Mutual inclination between Neptune and the SDO, as well as perturbations from other planets are neglected. The SDO is modeled as a test-particle. A quadrupole-level spherical harmonic expansion of Neptune’s gravitational potential illuminates that SDO evolution is facilitated by an infinite chain of Neptune's exterior $2:\chi$ resonances. Within the stochastic layer, dynamics of the test particle are primarily driven by the nearest $2:\chi$ resonance (where $\chi$ is an integer approximation to $2\,(a/a_{\rm{p}})^{3/2}$), while its chaotic evolution is facilitated by interactions with neighboring resonances. On the diagram, the nominal locations of $2:\chi\pm1$, $2:\chi\pm2$, $2:\chi\pm3$, and $2:\chi\pm4$ resonances adjacent to the SDO orbit are shown as green ellipses. As we discuss in the text, for the problem at hand, the resonance overlap criterion can be recast as a critical perihelion distance, $q_{\rm{crit}}$, below which chaotic evolution ensues.}
\label{Fig:SETUP}
\end{figure*}

\section{Perturbation Theory} \label{sec:anmodel}
As a starting point in our calculation, let us outline our basic assumptions. First and foremost, we treat the SDO as a massless test particle, and assume that its orbital period exceeds that of Neptune by a large margin (i.e., $\alpha\ll1$). Second, we assume that Neptune's eccentricity, $\en$, is sufficiently small to be negligible for our purposes. Third, we neglect all inclination-node dynamics, reducing the problem to a common plane (Figure \ref{Fig:SETUP}). 

\subsection{The Disturbing Function}
A general spherical harmonic expansion of the planetary disturbing function for the co-planar three-body problem is presented in \citet{2013MNRAS.435.2187M}. Employing the usual notation of celestial mechanics, the disturbing function is expressed as a quadruple infinite series:
\begin{align}
\R&=\frac{\G\,\mn}{a}\, \sum_{\ell=2}^{\infty} \, \sum_{m=m_{\rm{min}},\,2}^{\ell} \, \sum_{j' = -\infty}^{\infty} \, \sum_{j = -\infty}^{\infty} \zeta_m\, c_{\ell\,m}^2\,\mathcal{M}_\ell \nonumber \\
&\times \alpha^\ell\, X_{j'}^{\ell,m} (\en) \,X_{j}^{-(\ell+1),m} (e)\, \nonumber \\
&\times \cos\big(j'\,\Mn -j\,M+m(\omega-\omega_{\rm{N}})\big).
\label{R}
\end{align}
In the above expression, the unmarked variables ($a,M,e,\omega$) refer to the orbital elements of the SDO, while those with the subscript $\rm{N}$ correspond to Neptune\footnote{Technically, in equation (\ref{R}), Neptune's mass $\mn$ should be the reduced mass $\mu=\mn\,\Msun/(\mn+\Msun)\approx\mn$, but our choice to replace $\mu$ with $\mn$ is of no practical consequence.}. Note that for the planar problem, the distinction between the longitude and argument of perihelion vanishes, such that $\omega$ simply corresponds to the azimuthal orientation of the apsidal line (see Figure \ref{Fig:SETUP}). We will remark on the dimensionless constants $\zeta, c, \mathcal{M}$, as well as Hansen coefficients $X$ in greater detail below.

In equation (\ref{R}), the index $\ell$ informs the \textit{degree} of the spherical harmonic expansion. Because we are specifically interested in the $\alpha\ll1$ limit, our needs are sufficed by truncating the expansion at quadrupolar level, corresponding to $\ell_{\rm{max}}=2$. This removes the first sum of the series completely, as well as any dependence on the mass-factor $\mathcal{M}$ because $\mathcal{M}_2=1$ for all mass ratios \citep{2013MNRAS.435.2187M}. On a quantitative level, however, this assumption restricts the applicability of our model to long-period (e.g., $a\gtrsim400\,$AU) orbits.

Beyond the first sum, truncation of the series at $\ell=2$ sets manageable bounds of the \textit{order} of the expansion, $m$, such that the second sum runs from $m_{\rm{min}}=0$ to $m_{\rm{max}}=2$. Carrying on, the third sum can be eliminated fully, thanks to an important property of the Hansen coefficient $X_{j'}^{l,m}$. Specifically, it is possible to demonstrate that to leading order in eccentricity, $X_{j'}^{l,m}(\en)=\order(\en^{|m-j'|})$ \citep{1981CeMec..25..101H}, meaning that within the context of our adopted limit of $\en\rightarrow0$, all terms with $m\ne j'$ vanish. The dependence on the first Hansen coefficient in the series is further trivialized by the fact that $X_0^{2,0}(0) = X_1^{2,1}(0) = X_2^{2,2}(0) = 1$. In addition, because apsidal orientation is ill-defined at null eccentricity, we can set $\omega_{\rm{N}}=0$ without loss of generality.

A final simplification comes from the functional form of the constant $c_{\ell\,m}^2$. Written explicitly in terms of spherical harmonics $Y_{\ell,m}$, we have $c_{2\,m}^2=(8\,\pi/5)\,(Y_{2,m}(\pi/2,0))^2$. Crucially, $Y_{2,1}(\pi/2,0)=0$, meaning that all terms of order unity have zero amplitude, and the expansion only contains harmonics with $m=0$ and $m=2$. Writing out all remaining constants explicitly, we have $c_{2\,0}^2 = 1/2$, $c_{2\,2}^2 = 3/4$, $\zeta_0=1/2$, and $\zeta_2=1$. 

With the approximation scheme outlined above, the full quadrupole-level expression for the disturbing function takes the form:
\begin{align}
\R_{\rm{q}}&=\frac{\G\,\mn}{4\,a}\, \alpha^2  \sum_{j = -\infty}^{\infty} \bigg[\underbrace{X_{j}^{-3,0}\, \cos\big(j\,M\big)}_{m\,=\,0} \nonumber \\
&+ \underbrace{3\,X_{j}^{-3,2}\, \cos\big(j\,M- 2(\Mn-\omega)\big)}_{m\,=\,2} \bigg],
\label{Rq}
\end{align}
where the dependence of $X$ on $e$ is implied. Let us now classify these zeroth and second order (in $m$) terms, according to the dynamics they govern.

\subsection{$m=0$: short-periodic and secular terms}
Ignoring the pre-factor in equation (\ref{Rq}), let us begin by examining the first set of harmonics. Noting the general property of Hansen coefficients $X_{j}^{-(\ell+1),m}=X_{-j}^{-(\ell+1),-m}$, it is convenient to write the leading sum as:
\begin{align}
\sum_{j = -\infty}^{\infty} X_{j}^{-3,0}\, \cos\big(j\,M\big) &= (1-e^2)^{-3/2} \nonumber \\
&+2\,\sum_{j = 1}^{\infty} X_{j}^{-3,0} \cos\big(j\,M\big),
\label{sumexp}
\end{align}
where we have taken advantage of the fact that $X_{0}^{-3,0}$ can be evaluated in closed form \citep{1981CeMec..25..101H}.

The first member of the RHS of equation (\ref{sumexp}) is a pure secular term that governs the apsidal precession of an SDO due to the phase-averaged potential of Neptune. The remainder of the RHS is the epitome of short-periodic terms i.e., harmonics that average out on the orbital timescale and contribute virtually nothing to long-term orbital evolution. As is well known, these rapidly oscillating terms can be removed from the Hamiltonian all-together through a near-identity variable transformation, which essentially corresponds to a change from osculating to orbit-averaged (mean) orbital elements (see e.g., Ch. 2 of \citealt{Morbybook}). Thus, we are justified in dropping them from the expression.

Although marginally illuminating, our discussion of $m=0$ perturbations is neither interesting nor new. That is to say, at zeroth order in $m$, the quadrupole-level disturbing function contains no terms that can explain the underlying chaotic structure of the scattered disk. Therefore, scattering dynamics must arise at $m=2$ order, which we examine next. 

\subsection{$m=2$: resonant terms}
The functional form of the $\ell=2,m=2$ harmonic is easy to interpret: the critical argument
\begin{align}
\varphi=j\,M- 2(\Mn-\omega))=j(\lambda-\varpi)-2(\lambda_{\rm{N}}-\varpi)
\label{varphi}
\end{align}
governs the exterior $2:j$ mean-motion resonance with Neptune. Correspondingly, the functional form of equation (\ref{Rq}) indicates that the underlying dynamical structure of the distant scattered disk is nothing more than an infinite sequence of $2:j$ resonances. In fact, to the extent that the quadrupole-level expansion of $\R$ is an accurate representation of the planetary potential, $2:j$ resonances \textit{must} drive the scattering process, since no other harmonics exist in the expansion.

This notion immediately suggests that the stability boundary of the scattered disk (which separates the chaotic and regular dynamics) can be understood within the context of the \citet{Chirikov1979} resonance overlap criterion. We will examine this suspicion more closely below. For the time being, however, we will limit ourselves to simply writing down the model Hamiltonian for the SDO. Dropping short-periodic terms as described above, we have:
\begin{align}
&\Ham=-\frac{\G\,\Msun}{2\,a} - \frac{1}{4} \frac{\G\,\mn}{a}\, \alpha^2 \, (1-e^2)^{-3/2} +\mathcal{T} \nonumber \\
&- \frac{3}{4} \frac{\G\,\mn}{a}\, \alpha^2  \sum_{j = -\infty}^{\infty} X_{j}^{-3,2} \cos\big(j\,M- 2(n_{\rm{N}}\,t-\omega)\big),
\label{Hammy}
\end{align}
where in anticipation of canonical transformations that will follow, we have replaced $\Mn$ with $n_{\rm{N}}\,t$ and introduced a dummy action, $\mathcal{T}$, conjugate to time, in order to keep $\Ham$ formally autonomous.

\subsection{Computation of Hansen Coefficients $X_{j}^{-3,2}$}
The final piece that is needed to complete the specification of our framework is the evaluation of the integral-defined functions $X_{j}^{-3,2}$. Since their introduction by \citet{Hansen1885}, these coefficients have been earnestly studied in the literature (see e.g., \citealt{1970ceme.book.....H,1981CeMec..25..101H, 2008CeMDA.100..287S}), and the general consensus holds that closed-form expressions for coefficients with $j\neq0$ do not exist. Nevertheless, \citet{Sadov2006} has demonstrated that in the double limit of $e\rightarrow1^{-}$ \textit{and} $j\rightarrow\infty$ the specific coefficient $X_{j}^{-3,2}$ (which \citealt{Sadov2006} calls a Chernousko function with index $j-2$) approaches the asymptotic form:
\begin{align}
X_{j}^{-3,2}\,\xrightarrow[j\rightarrow\infty]{e\rightarrow1^{-}}\, -\frac{4}{9}(j-2).
\label{Sadov}
\end{align}

Even the most eccentric scattered disk objects within the current census of TNOs are insufficiently close to $e$ of unity for equation (\ref{Sadov}) to apply. Similarly, a series approximation of $X_{j}^{-3,2}$ in terms of $\sqrt{1-e^2}$ (see \citealt{2008CeMDA.100..287S}) does not converge rapidly enough to be quantitatively useful. Nevertheless, the quasi-linear scaling of $X_{j}^{-3,2}$ with $j$ is intriguing, and through numerical evaluation we have found that the relationship $X_{j}^{-3,2}\propto j$ holds with a surprising degree of accuracy along contours of constant $q=a\,(1-e)=(j/2)^{2/3}\,a_{\rm{N}} \,(1-e)$. Taking advantage of this, the slope of the linear relationship can be expressed as sole function of $q/\an$, and we have found that a simple Gaussian-like parameterization achieves satisfactory precision:
\begin{align}
X_{j}^{-3,2} \approx \frac{2\,j}{5}\exp\bigg[ -\bigg(\frac{q}{a_{\rm{N}}} \bigg)^2\, \bigg].
\label{param}
\end{align}
In fact, applied specifically to the observationally relevant $q\in(30,50)\,$AU range, equation (\ref{param}) agrees with direct evaluation of Hansen coefficients with $j>10$ down to a few percent. We will elaborate on the calculational advantage of evaluating $X_{j}^{-3,2}$ along a locus of constant perihelion further below.

It is interesting to compare the form of equation (\ref{param}) to expression (B5) of \citet{2013MNRAS.435.2187M}, which is based on an asymptotic expansion for the overlap integral representing the energy exchanged during one outer orbit (see also equations 3.55 and 3.73 of \citealt{Mardling2008Chaos}). In particular, the exponential decay of the low and high-eccentricity tails reflects the fact that an exponentially small amount of (specific) energy is exchanged between the orbits when the angular frequency of the test particle at perihelion is significantly different to the orbital frequency of Neptune. Conversely, significant energy is exchanged when these frequencies are similar. In fact, they are the same when $q/\an= (1+e)^{1/3}$ which for $e\sim 1$ corresponds to $q\approx38\,$AU. Thus, even before examining the onset of instability from the vantage point of the Chirikov criterion, we may intuitively expect the semi-major axis dependence of the perihelion stability boundary to be relatively shallow. %With our theoretical basis delineated, let us now proceed to write down the Chirikov criterion

\section{Results} \label{sec:results}
In order to evaluate the stability boundary of the scattered disk established by $2:j$ resonances, we must estimate the critical value of $q$ as a function of $a$, at which neighboring resonances overlap. Accordingly, we now project the separatrixes of the individual resonances onto the $q-a$ plane. As is usual for calculations of this type, the first step is to write down an integrable pendulum-like Hamiltonian for an isolated $2:\chi$ resonance, where $\chi$ is an integer nearest to $2\,(a/a_{\rm{p}})^{3/2}$. 

\subsection{An Integrable Model for an Isolated $2:\chi$ Resonance} \label{sec:intmod}
Because the resonance width is expected to be small compared to the SDO semi-major axis, a conventional approach to circumventing the inverse semi-major axis dependence of the Keplerian term in equation (\ref{Hammy}) is to Taylor-expand it around the nominal resonance location. Accordingly, in terms of conventional Delaunay variables $L=\sqrt{\G\,\Msun\,a}, l=M, G=L\,\sqrt{1-e^2}, g=\omega$ (see Ch. 2 of \citealt{MD99}), we have:
\begin{align}
&-\frac{1}{2}\bigg(\frac{\G\,\Msun}{L} \bigg)^2 \approx -\frac{1}{2}\bigg(\frac{\G\,\Msun}{[L]} \bigg)^2 + \bigg(\frac{\G\,\Msun}{[L]} \bigg)^2 \bigg( \frac{\delta L}{[L]} \bigg) \nonumber \\
& - \frac{3}{2} \bigg(\frac{\G\,\Msun}{[L]} \bigg)^2 \bigg( \frac{\delta L}{[L]} \bigg)^2 =[n]\,\bigg(\delta L - \frac{3}{2}\frac{\delta L^2}{[L]} \bigg)+\dots ,
\label{Hkeptaylor}
\end{align}
where $[L]=\sqrt{\G\,\M\,(\chi/2)^{2/3}\,\an}$ is the nominal action and $[n]=(2/\chi)\,n_{\rm{N}}$ is the mean motion at the center of the $2:\chi$ resonance. At this stage, it is convenient to adopt $\delta L=L-[L]$ as the action instead of $L$ itself, keeping in mind that translation of an action by a constant is always canonical. 

Let us now define a change of variables through a type-2 generating function:
\begin{align}
\mathcal{F}_2=(\underbrace{\chi\,l/2-(n_{\rm{N}}\,t-g)}_{\phi})\,\Phi+(\underbrace{l}_{\psi})\,\Psi+(\underbrace{t}_{\xi})\,\Xi.
\label{Hkeptaylor}
\end{align}
The actions conjugate to the new angles $\phi$, $\psi$, and $\xi$ are defined by the usual transformation equations:
\begin{align}
&\delta L=\frac{\partial\,\mathcal{F}_2}{\partial\,l}=\frac{\chi}{2}\,\Phi+\Psi \nonumber \\
&G=\frac{\partial\,\mathcal{F}_2}{\partial\,g}=\Phi \nonumber \\
&\mathcal{T}=\frac{\partial\,\mathcal{F}_2}{\partial\,t}=\Xi-n_{\rm{N}}\,\Phi.
\label{transform}
\end{align}

With the preliminaries (\ref{Hkeptaylor}) and (\ref{transform}) delineated, we are now in a position to write down an idealized Hamiltonian, $\Ham_{\chi}$, for each isolated resonance. Neglecting the unimportant $m=0$ secular term in equation (\ref{Hammy}) and retaining only the principal harmonic, we have:
\begin{align}
\Ham_{\chi}&=-\frac{3}{4}\frac{n_{\rm{N}}\,\chi}{[L]} \, \Phi^2-\frac{3\,n_{\rm{N}}}{[L]}\,\Psi\,\Phi \nonumber \\
&- \frac{3}{4} \frac{\G\,\mn}{a}\, \bigg(\frac{a_{\rm{N}}}{a} \bigg)^2\,X_{\chi}^{-3,2}\,\cos\big(2\,\phi \big) \nonumber \\
&+\underbrace{\Xi + \frac{2\,n_{\rm{N}}}{[\chi]}\,\Psi-\frac{3\,n_{\rm{N}}}{\chi\,[L]}\,\Psi^2 -\frac{1}{2}\bigg(\frac{\G\,\Msun}{[L]} \bigg)^2}_{\rm{const.}}
\label{Hchi}
\end{align}
Notice that upon switching to variables (\ref{transform}), the linear term in $\Phi$ arising from the $[n]\,\delta L$ term in equation (\ref{Hkeptaylor}) is exactly cancelled by the $-n_{\rm{N}}\,\Phi$ term that ensues from the dummy action $\mathcal{T}$, owing to the fact that $(\chi/2)\,[n]=n_{\rm{N}}$.

$\Ham_{\chi}$ is now independent of the angles $\psi$ and $\xi$, and the conjugate actions $\Psi$ and $\Xi$ are integrals of motion. Accordingly, all terms on the third line of equation (\ref{Hchi}) are constant and can simply be dropped from the Hamiltonian. Moreover, the linear action term (proportional to $\Phi\,\Psi$) can be absorbed into the leading term by adding $(3\,n_{\rm{N}}\Psi^2)/(\chi\,[L])$ to $\Ham_{\chi}$ and completing the square, such that the nonlinear action term becomes proportional to $\tilde{\Phi}^2=(\Phi-2\Psi/\chi)^2$. Then, adopting $\tilde{\Phi}$ as the new action conjugate to $\phi$ (again, by canonical translation) and substituting parameterization (\ref{param}) for the Hansen coefficient, we obtain the Hamiltonian of a mathematical pendulum:
\begin{align}
\Ham_{\chi}=&-\underbrace{\frac{3}{\an^2}\,\bigg( \frac{\chi}{2} \bigg)^{2/3}}_{\beta}\,\frac{\tilde{\Phi}^2}{2}\nonumber \\
&-\underbrace{\frac{6}{5}\frac{G\,\mn}{\an\,\chi}\exp\bigg[-\bigg(\frac{q}{\an}\bigg)^2\,\bigg]}_{\gamma}\,\cos(2\,\phi).
\label{Hchi2}
\end{align}

\begin{figure*}[t]
\centering
\includegraphics[width=\textwidth]{MEGNO.png}
\caption{Chaos map of the distant scattered disk, modeled within the framework of the circular planar restricted three-body problem. A heat map of the MEGNO chaos indicator, $Y$, is shown on the semi-major axis vs. perihelion plane. Blue regions of the diagram depict initial conditions that lead to regular motion, whereas yellow and red regions correspond to chaotic dynamics. Within the chaotic layer, the Lyapunov time of the SDO approaches the orbital period. The analytic threshold for chaotic motion ($q_{\rm{crit}}$, given by equation \ref{qcrit}) is shown with a thick black line. The nominal locations and widths of individual $2:\chi$ mean motion resonances are shown with thin green and white lines, respectively.}
\label{Fig:MEGNO}
\end{figure*}

It is worth noting that in the well-studied case of low-$e$ mean motion resonance dynamics \citep{MD99,Morbybook} the oscillation period of the resonant angle exceeds the orbital period by a large margin. This is not the case for the SDO scattering problem at hand: in the $q\sim \an$ regime, the ratio of the libration frequency to mean motion is given by $\sqrt{\beta\,\gamma}/n\sim (a/\an)^{5/4}\,\sqrt{\mn/\M}\sim1/4-1/2$. Thus, the orbital frequency exceeds the libration frequency only by a factor of a few.

The expression for the resonance width of a mathematical pendulum is well known: $\Delta\tilde{\Phi}=4\sqrt{\gamma/\beta}$ (Ch. 4 of \citealt{Morbybook}). It is important to understand, however, that this width -- expressed in terms of the canonical action $\tilde{\Phi}$ -- is ultimately related to the SDO's eccentricity. To relate this quantity to the resonance width in terms of the semi-major axis, we use the conservation of $\Psi$. In fact, it is straightforward to demonstrate that the conservation of $\Psi$ is nothing more than a re-statement of the conservation of the Tisserand parameter (Ch. 8 of \citealt{MD99,2013A&A...556A..28B}). Moreover, it is easy to show that for long-period and highly eccentric orbits, the conservation of the Tisserand parameter is equivalent to a conservation of the perihelion distance. A more detailed discussion of the physical meaning of the conservation of $\Psi$, and how it relates to other quasi-integrals of motion of the circular restricted three-body problem is presented in the Appendix.

\subsection{The Chirikov Criterion} \label{sec:chirikov}

From equation (\ref{transform}), it follows that $\Delta\delta L=\chi\,\Delta\tilde{\Phi}/2$. Direct substitution therefore yields: 
\begin{align}
\Delta a = 4\,a_{\rm{N}}\,\sqrt{\frac{2\,\chi\,\mn}{5\,\Msun}} \, \exp\bigg[-\bigg(\frac{q}{2\,\an}\bigg)^2\,\bigg].
\label{deltaa}
\end{align}
We note that while we arrived at this expression from the Hamiltonian formalism, an alternative approach would have been to start with the disturbing function (\ref{Rq}), write down Lagrange's equations of motion, and proceed to derive a pendulum-like equation of motion for the critical argument, $\phi$. Indeed, both approaches yield equivalent results (see e.g., Ch. 8.6. of \citealt{MD99}, section 2.3 of \citealt{2013MNRAS.435.2187M}; see also \citealt{1980AJ.....85.1122W, Mardling2008Chaos}). 

As stipulated by \citet{1959SPhD....4..390C,Chirikov1979}, the width of the resonance, $\Delta a$ should be compared with the distance between adjacent resonances, $\delta a = ([a]_{\chi+1}-[a]_{\chi-1})$. In the limit of large $\chi$, it is straightforward to show that $\delta a \approx (2\,\an/3)\,(2/\chi)^{1/3}$. The degree of resonance overlap is characterized by the ratio of $\Delta a$ and $\delta a/2$ (recall that neighboring resonances also widen in an equivalent way). Expressing this number in terms of SDO semi-major axis instead of $\chi$, we have:
\begin{align}
\frac{\Delta a}{\delta a} = \frac{24}{\sqrt{5}}\,\bigg(\frac{a}{\an}\bigg)^{5/4}\,\sqrt{\frac{\mn}{\Msun} }\,\exp\bigg[-\bigg(\frac{q}{2\,\an}\bigg)^2\,\bigg].
\label{K}
\end{align}
This result demonstrates an intriguing trend: along a locus of constant perihelion distance, the degree of overlap \textit{grows} with increasing particle semi-major axis. As importantly, we can set the overlap number equal to the critical value of unity ($\Delta a/\delta a\rightarrow1$), and invert this relation:
\begin{align}
q_{\rm{crit}}=\an\,\sqrt{\ln\bigg( \frac{24^2}{5}\,\frac{\mn}{\Msun }\,\bigg(\frac{a}{\an}\bigg)^{5/2}\bigg)}. 
\label{qcrit}
\end{align}
This expression yields a critical perihelion distance, $q_{\rm{crit}}$, as a function of semi-major axis, below which chaotic diffusion is expected to ensue. 

\subsection{Numerical Validation}  \label{sec:numsim}
The analytic calculations outlined above yield a compact result for the chaotic threshold of the scattered disk. This result is a key prediction of our theory that can be tested with numerical experiments in a straightforward manner. Here, we carry out this examination as a sequence of two sets of $N-$body simulations employing distinct levels of complexity. More specifically, our first task is to compare our expression with a chaos map generated within the context of an identical physical configuration -- the circular planar restricted three-body problem. This Sun-Neptune-SDO setup provides the closest point of comparison between our analytic theory and numerical calculations, and is equivalent to lifting the assumptions employed in reducing the complexity of the disturbing function (\ref{R}).

%\subsection{Circular Restricted Three-Body Problem}

A customary way to map out the boundaries between chaotic and regular motion is to compute the system's Lyapunov coefficient, $\Lambda$ (or its siblings), on a plane of initial conditions. Here, we follow this conventional approach, substituting the Lyapunov coefficient for the more-rapidly-convergent MEGNO chaos indicator, $Y$ \citep{MEGNO}. We carried out these simulations using the \texttt{REBOUND} gravitational dynamics software package \citep{2019MNRAS.485.5490R,2019MNRAS.489.4632R}, employing the \texttt{whfast} integration algorithm with an initial time-step of $\delta t=63$ code units\footnote{The code uses units where the gravitational constant $\G$ is set to unity, such that in a unit system that employs solar masses and astronomical units, this time-step corresponds to approximately 10 years, or equivalently, $6\%$ of Neptune's orbital period.}. We generated two such maps, with $a\sim500\,$AU and $a\sim1000\,$AU. Each integration spanned $\Delta t = 0.1\,$Myr for $a\sim500\,$AU runs but was increased to $\Delta t = 0.3\,$Myr for $a\sim1000\,$AU runs to accommodate the longer orbital period. The resolution of our grid of initial conditions in SDO perihelion distance and semi-major axis was set to $\delta q = \delta a = 0.1\,$AU. Neptune's eccentricity remained at $e_{\rm{N}}=0$ throughout the integrations. Additionally, all starting orbital angles were set to null values, with the exception of the SDO mean anomaly, which was initialized at $M=\pi$ (aphelion). 

The left and right panels of Figure \ref{Fig:MEGNO} show MEGNO maps centered around a SDO semi-major axes of $a=500$ and $1000\,$AU, respectively. On the same plane, we mark the locations of individual $2:j$ resonances with green lines and project their widths according to equation (\ref{deltaa}) with white curves. The critical perihelion distance, corresponding to marginal overlap given by equation (\ref{qcrit}), is shown with a thick black line. As the color-bar indicates, blue regions of the plot (where $Y\sim2$) correspond to regular motion while initial conditions depicted with red and yellow points (where $Y\sim\Lambda\,\Delta t/2$) indicate chaotic SDO dynamics. As a check on our simulations, we recomputed portions of the shown MEGNO maps with a different choice of integration algorithm (\texttt{IAS15}) and longer timespan ($3\times\Delta t$) and got equivalent results.

Overall, the analytic criterion (\ref{qcrit}) provides a satisfactory approximation for the boundary between regular particle motion and large-scale chaos. Nevertheless, we remark that this threshold is inexact, and fine structure, including that arising from higher-order resonances, causes equation (\ref{qcrit}) to underestimate the critical value of $q$ at some values of $a$ while overestimating it at others. To elaborate on this further, the fact that regular regions exist at $q<q_{\rm{crit}}$ may in part be attributed to the fact that regular islands exist within the chaotic sea even if there is substantial overlap. The existence of chaotic regions for $q>q_{\rm{crit}}$, however, likely illuminates the limitations of our analytic model. To this end, it is likely that a more detailed resonance overlap criterion that also accounts for octupole-level resonances could generate better agreement. Note further that the agreement between $N-$body simulations and our theory is somewhat better for $a=1000\,$AU than for $a=500\,$AU. This is not surprising, given that the assumptions of our model are better satisfied for increasingly long-period orbits.  

%\subsection{Circular Restricted Three-Body Problem}

\begin{figure}[t]
\centering
\includegraphics[width=\columnwidth]{NUMSIM.png}
\caption{Detailed models of the distant scattered disk. Orange points depict a model of an evolved scattered disk that is created exclusively by giant planet scattering, accounting for early migration of Neptune through the solar system. The purple points show a model scattered disk that is affected by giant planets as well as the galactic tide and passing stars. An inclination cut of $i<40\deg$ was applied to both models. The analytic threshold for chaos is shown with a thick black curve, as in Figure \ref{Fig:MEGNO}. While the resonance overlap criterion marks the boundary between regular and stochastic dynamics, it should not be interpreted as the boundary of the scattered disk itself. In an idealized scenario that only includes giant planet scattering, the near-conservation of the Tisserand parameter prevents SDOs from filling the entirety of the chaotic domain. In a more realistic model that also accounts for extrinsic effects, Galactic perturbations can raise and lower SDO perihelia across the chaotic threshold.}
\label{Fig:NUMSIM}
\end{figure}

With our analytic expression for the chaotic boundary verified through numerical experimentation, we now consider how this threshold for orbital stability compares with detailed models of the formation and evolution of the scattered disk. To this end, we reference the published simulation suite of \citet{Nesvorny2017}, where the genesis of the scattered disk was simulated accounting for the early outward migration of Neptune \citep{2005Natur.435..459T,2011ApJ...738...13B, 2016ApJ...825...94N}, and its long-term fate was self-consistently modeled subject to gravitational forcing from the giant planets as well as (optionally) the Galactic tide and passing stars.

The orbital structure of evolved ($t=4.5\,$Gyr) synthetic models of the distant scattered disk with $i<40\,\deg$ are contrasted against our analytic stability boundary in Figure \ref{Fig:NUMSIM}. More specifically, the results of a simulation where extrinsic effects were omitted are shown with orange points, while a scattered disk that is sculpted by Galactic tides and passing stars in concert with the planets is shown with purple dots. Upon examination, an important conclusion can immediately be drawn: the boundary of the scattered disk (meaning the parameter space occupied by the particles) does \textit{not} uniformly trace its chaotic threshold. That is, in absence of Galactic forcing, long-period particles retain relatively low perihelia with $q\lesssim36\,$AU and do not extend to the edge of the chaotic zone. Conversely, when the effects of the Galactic tide and passing stars are included, the resulting eccentricity modulation can lift the perihelia of SDOs well above the critical value for chaos, especially for $a\gtrsim1000\,$AU orbits.

These results can be understood within the framework of our model as follows. While the $q<q_{\rm{crit}}$ orbital domain is largely chaotic, the long-period SDO dynamics nevertheless approximately obey the conservation of the Tisserand parameter. As shown in the Appendix of this work, preservation of the Tisserand parameter (or analogously the resonant integral of motion $\Psi$ defined in equation \ref{transform}) is equivalent to evolution along a constant-perihelion contour for orbits with $a\gg\an$ and $e\sim1$. This near-conservation of the perihelion distance prevents SDOs from exploring the full range of parameter space spanned by the chaotic sea in simulations that only include planetary forcing. 

The opposite situation ensues in numerical experiments that include the Galactic tide. Under the action of the Galactic tide, all symmetry inherent to the circular restricted three-body problem is broken, allowing significant $q$ variation to take place. Accordingly, at sufficiently long orbital periods, SDOs can be carried to large perihelion distances with no regard for the chaotic boundary facilitated by Neptune. The transition between scattering-dominated dynamics and evolution primarily driven by the Galactic tide is relatively sharp, and occurs at a semi-major axis of $a\gtrsim1000\,$AU. Qualitatively, this shift corresponds to a point where the timescale associated with von Zeipel-Lidov-Kozai type perihelion oscillations facilitated by the Galactic tide becomes markedly shorter than the perihelion precession timescale forced by the giant planets.

The dynamical origins of $q\gtrsim36\,$AU $a\lesssim1000\,$AU objects are considerably more subtle. Notice that unlike their more distant counterparts, these objects follow the stability boundary of the scattered disk relatively well. Owing to comparatively rapid perihelion precession, the perihelia of these objects cannot be affected by the Galactic tide directly. Nevertheless, their lowered eccentricities indicate that they have been materially affected by the Galactic tide \textit{at some point}, implying that they must have attained $a\gtrsim1000\,$AU in the past. Correspondingly, these are objects that initially get scattered onto large heliocentric distances, and after significant Galactic perturbation diffuse back to smaller semi-major axes. As inward semi-major axis diffusion gets terminated at the chaos boundary, parameter space traced by $q_{\rm{crit}}$ gets filled in from the outside. Examination of individual time-series of particles in the simulations confirms this interpretation.


%This begs the question of how the chaotic threshold regulates the orbital 

%Cumulatively, these numerical experiments highlight two important aspects of our results: 

%the distant scattered disk is expected to stretch over an extended range of perihelion distances and 

%while our expression for the chaotic threshold, $q_{\rm{crit}}$, constitutes an analytic boundary between the ``detached" and actively ``scattering" sub-populations of distant trans-Neptunian minor bodies, SDOs with semi-major axes in excess of $\a\gtrsim$

\subsection{Linking the Scattered Disk with the Modulated Pendulum and the Standard Map} \label{sec:standard}
Against the backdrop of the perturbative treatment of the dynamics developed in the preceding sections, it is important to not forget that the more rudimentary -- but somewhat more physically intuitive -- picture of scattered disk dynamics is one wherein perturbations are envisioned as ``kicks" to the orbit that ensue when the SDO passes through perihelion and experiences a gravitational interaction with Neptune \citep{2004AJ....128.1418P, 2013Icar..222...20F}. Accordingly, it is useful to briefly examine the connection between our perturbative framework and this ``mapping" viewpoint.%, as well as other paradigmatic results of chaos theory.

To begin making the analogy, note that in the limit of large $\chi$, the Hansen coefficients $X_{\chi}^{-3,2}\approx X_{\chi \pm 1}^{-3,2}$. Thus, let us assume that the Hansen coefficients with neighboring indexes are not simply similar, but are in fact, equal to one-another. Under this approximation, we can factorize the Hansen coefficient in equation (\ref{Hammy}), to obtain a simple non-autonomous Hamiltonian that accounts for interactions between the primary $2:\chi$ resonance and its nearest neighbors:
\begin{align}
\Ham_{\chi\pm}&=\beta\,\frac{\tilde{\Phi}^2}{2}-\gamma\,\big( \cos(2\,\phi-l)+\cos(2\,\phi)+\cos(2\,\phi+l)\big)\nonumber \\
&=\beta\,\frac{\tilde{\Phi}^2}{2}-\gamma\,(1+2\cos(n\,t)) \,\cos(2\,\phi).
\label{modpend}
\end{align}
Note that here we have used a trigonometric identity and set $l=M=n\,t$ to arrive at the second line (recall further that $\beta$ and $\gamma$ are defined in equation \ref{Hchi2}). This expression corresponds to the Hamiltonian of a \textit{modulated pendulum}, where the modulation frequency is equal to the SDO's mean-motion (Ch. 4 of \citealt{Morbybook}). Recalling that the mean motion is faster than the libration frequency by a factor of a few, chaotic dynamics that arise within the context of our problem lie squarely outside of the ``adiabatic" domain.

Let us now push our luck, and extend the aforementioned approximation by assuming that \textit{all} Hansen coefficients in the infinite perturbation series are equal. Although seemingly crude, this approximation in fact holds relatively well in practice because the dynamics of any given resonance is most strongly affected by perturbations that are ``nearby" in action space (or equivalently, in frequency space). Indeed, the amplitudes of faraway resonances do not matter much, since the harmonics vary rapidly and the corresponding terms quickly average out (see e.g., \citealt{Wisdom1982} for a discussion). In this limit, we can imagine that the sum in equation (\ref{Hammy}) runs exclusively over the cosines. Thus, employing a Fourier representation of the periodic $\delta$-function, we can write:
\begin{align}
&\sum_{j=-\infty}^{\infty}\cos(j\,l+2\,\phi)=\cos(2\,\phi)\,\sum_{j=-\infty}^{\infty}\cos(j\,l) \nonumber\\
&= \frac{1}{2\,\pi} \cos(2\,\phi) \, \delta_{2\pi/n},
\label{deltacos}
\end{align}
where $\delta_{2\pi/n}$ represents an impulse comb that is applied with the orbital period of the SDO at $l=0$ (perihelion).

Substituting equation (\ref{deltacos}) back into the expression for $\Ham$, we see that when expanded in the vicinity of a $2:\chi$ resonance, Hamiltonian (\ref{Hammy}) takes on the familiar form of a periodically kicked pendulum: 
\begin{align}
\Ham=\beta\,\frac{\tilde{\Phi}^2}{2}- \frac{\gamma}{2\,\pi} \cos(2\,\phi)\, \delta_{2\pi/n}.
\label{kicked}
\end{align}
As is well known, Hamiltonian (\ref{kicked}) generates the \textit{Chirikov Standard Map} -- an emblematic model of chaotic dynamics (e.g., \citealt{LLbook,Chirikov1979}). In fact, the appearance of the Standard Map within the context of this problem acts as the bridge between our analytic framework and the scattering viewpoint. To this end, it is crucial to note that the Kepler Map discussed in section \ref{sec:intro}, is locally identical to the Standard Map, which is governed by the above Hamiltonian \citep{2011NewA...16...94S, 2009IJMPD..18.1903K}. The connection between the perturbative treatment of SDO evolution and a mapping approach to modeling the orbital motion is thus clear.

\subsection{Chaos in the Scattered Disk: Analytic Estimates} \label{sec:analytic}
An important motivation behind making the connections between our perturbative theory of scattered disk dynamics and archetypal models of chaotic motion described above, is that the latter naturally lend themselves to analytic estimates \citep{LLbook}. In this vein, previous work aimed at quantifying Lyapunov times and the action diffusion constants of main belt Asteroids \citep{1996AJ....112.1278H,1997AJ....114.1246M,1998CeMDA..71..243N} and Mercury \citep{Laskar2008,2011ApJ...739...31L, 2015ApJ...799..120B} played an important role in expanding our overall understanding of chaotic small body evolution within the inner solar system. Here we continue this program, and focus on quantifying the Lyapunov time and semi-major axis diffusion coefficient within the scattered disk, from analytic grounds.  

\paragraph{Lyapunov Time} Our estimate the SDO Lyapunov time, $\tau_{\rm{L}}$, follows directly from the analogy with a modulated pendulum equation (\ref{modpend}) made above. To outline the qualitative picture, recall that the resonance width of a mathematical pendulum scales as the square root of the factor that multiplies the harmonic term of the Hamiltonian. Because this factor is time-dependent in equation (\ref{modpend}), however, the separatrix in our problem is not steady, and instead pulsates at the modulation frequency. In the regime of strong resonance overlap -- which we can crudely assume for orbits with $q< q_{\rm{crit}}$ -- a large fraction of the resonant phase-space area is periodically swept by a homoclinic curve, that instills hyperbolicity upon the SDO trajectory with the same frequency (Ch. 9.4 of \citealt{Morbybook}). Therefore, to an order of magnitude, the SDO's Lyapunov time can be interpreted as the modulation period, which in the case of Hamiltonian (\ref{modpend}) is nothing other than the orbital period:
\begin{align}
\tau_{\rm{L}}\sim\Lambda^{-1}\sim\frac{2\,\pi}{n}=\sqrt{\frac{4\,\pi^2\,a^3}{\G\,\Msun}}.
\label{Lyapana}
\end{align}

The fact that the Lyapunov time in the scattered disk is comparable to the orbital period can be understood from intuitive grounds. While macroscopic divergence of neighboring trajectories may require multiple Lyapunov times to ensue (depending on the initial separation of nearby starting conditions), it is important to keep in mind that $\tau_{\rm{L}}$ itself is a measure of decoherence on a microscopic scale. Accordingly, two initially nearby trajectories within the scattered disk will experience perturbations from Neptune at slightly distinct phases, meaning that their separation in phase-space will be amplified on the orbital timescale.

To test this assertion, let us return to Figure \ref{Fig:MEGNO} and examine the values of the MEGNO chaos indicator that ensue within the stochastic layer. At $a=500\,$AU, where the SDO orbital period is approximately $11{,}000$ years, the chaotic domain is characterized by $Y\sim9$. Recalling that $Y\sim2\,\Delta t/\tau_{\rm{L}}$ with $\Delta t=0.1\,$Myr, we thus obtain $\tau_{\rm{L}}\sim2\times10^{4}\,$years -- a value comparable to the orbital period. We have further checked these results with a few traditional calculations of the Lyapunov times through direct integration of the variational equations (for SDOs randomly initialized with $31<q<36$ and $a=500\,$AU; \citealt{2016MNRAS.459.2275R}) and obtained estimates of $\tau_{\rm{L}}$ that were even closer to the orbital period. The MEGNO map at $a=1000\,$AU tells a similar story: with $\Delta t = 0.3\,$Myr and a characteristic $Y\sim20$, we obtain $\tau_{\rm{L}}\sim3\times10^{4}\,$years -- a value very close to the approximately $31{,}000$ year orbital period. 

\paragraph{Diffusion Coefficient} It is well established that within a stochastic system subject to vigorous mixing, the statistical properties of the actions obey the Fokker-Plank equation \citep{1945RvMP...17..323W}. Moreover, if the system is Hamiltonian, it can be shown that the Fokker-Plank equation reduces to the conventional diffusion equation, such that all of the relevant physics is encapsulated in the diffusion coefficient, $\mathcal{D}$. 

In the quasi-linear approximation, the value of $\mathcal{D}$ can be generally estimated as the product of the Lyapunov coefficient and the square of the resonant half-width \citep{Chirikov1979, LLbook}. The physical interpretation of this relation is that the resonant half-width represents a typical stochastic ``step-size" that a trajectory attains over a single decoherence (Lyapunov) time. For the problem at hand, the resonance half width, $\Delta a/2$, follows from equation (\ref{deltaa}), and we have already shown that $\tau_{\rm{L}}$ is well-approximated by the orbital period\footnote{Similar dynamics can arise in the case of first order resonances of a high degree, where a kick received during conjunction can produce changes in action that are comparable with the resonance width \citep{1994A&A...289..972S}}. The semi-major axis diffusion coefficient thus has the form:
\begin{align}
\mathcal{D}_a\sim\frac{\Delta a^2}{4\,\tau_{\rm{L}}}&=\frac{8}{5\,\pi}\frac{\mn\,\sqrt{\G\,\Msun\,\an}}{\Msun} \nonumber \\
&\times \exp\bigg[-\frac{1}{2}\bigg(\frac{q}{\an}\bigg)^2\,\bigg].
\label{Diffcoeff}
\end{align}
Note that this expression is independent of the particle's semi-major axis, and only depends on its perihelion distance. 

As a numerical check on our assumption that $\Delta a/2$ is truly a suitable approximation for a characteristic semi-major kick experienced by an SDO over a single orbital period, we ran 500 single-orbit Sun-Neptune-SDO simulations with $q=35\,$AU, randomized phases, and semi-major axis sampled uniformly in the $a=500\pm5\,$AU range. We then measured the aphelion-to-aphelion variation in particle semi-major axes, and found a mean value of $2.32\,$AU, in good agreement with the results of \citet{2013Icar..222...20F}. This quantity is close to the theoretically predicted value of $\Delta a/2 = 2.26\,$AU, leading us to conclude that equation (\ref{Diffcoeff}) provides an adequate approximation for the semi-major axis diffusion coefficient of long-period scattered disk objects.

%TO DO: This naturally leads to a connection with the standard map. Some ideas: the Lyapunov time is simply the resonance modulation time. Does this mean that within the chaotic layer, the Lyapunov timescale is just the frequency distance between resonances, meaning that it's just the orbital period? Could be... The diffusion coefficient is trivial to calculate in a similar fashion: just the resonance width squared, divided by the Lyapunov time....

\section{Discussion} \label{sec:discuss}

Owing to the unrelenting observational mapping of the trans-Neptunian solar system that has ensued over the last two decades, the orbital structure of the scattered disk continues to come into an ever-shaper focus. Several attempts have been made to describe the stochastic dynamics of this remarkable population of minor bodies. In this vein, $1:j$ resonances have been broadly discussed in the literature as an attractive theoretical explanation for the emergent behavior of actively scattering TNOs \citep{2004AJ....128.1418P, 2018AJ....155..260V, 2019CeMDA.131...39L}. Nevertheless, a complete understanding of the evolution of long-period orbits has remained incomplete.

In this work, we have approached the problem of scattered disk dynamics from a perturbative viewpoint. In particular, we have derived a simple Hamiltonian model for the orbital motion of long-period TNOs, based upon a quadrupole-level expansion of the planetary disturbing function \citep{1962AJ.....67..300K,Laskar2010DISTFUNCT,2013MNRAS.435.2187M}. Our analysis indicates that the scattered disk's dynamical machinery is comprised of a chain of $2:j$ resonances and that their overlap is responsible for driving chaotic motion. To be clear, $1:j$ harmonics are not entirely absent from the dynamical picture, but are smaller than $2:j$ resonances by a factor of $\en$ at quadrupole order, or a factor of $\alpha$ (i.e., appearing at octupole+ order) in the $\en\rightarrow0$ limit. We further demonstrate how our theoretical model can be reduced to the Chirikov Standard Map \citep{Chirikov1979}, illuminating the physical connection between resonant perturbations and the scattering process itself.

Interpreting the intersection point among nonlinear $2:j$ resonances as the dividing line between regular and stochastic motion, we have derived an analytic stability boundary of the distant scattered disk. In practice, this criterion is given by equation (\ref{qcrit}) and translates to a critical perihelion distance below which chaos ensues. For chaotic orbits that satisfy this criterion, we have obtained analytic estimates of Lyapunov time, $\tau_{\rm{L}}$ (equation \ref{Lyapana}), and the semi-major axis diffusion coefficient, $\mathcal{D}_a$ (equation \ref{Diffcoeff}). Importantly, these calculations indicate that within the strongly chaotic domain of the scattered disk, the Lyapunov time approaches the orbital period, while the semi-major axis diffusion coefficient is on the order of Neptune's angular momentum divided by the solar mass. Our analysis further shows that the semi-major axis diffusion rate (or equivalently, the rate of energy diffusion) is insensitive to the semi-major axis itself. Instead, $\mathcal{D}_a$ only depends on the perihelion distance -- a result that is consistent with previous findings \citep{2004AJ....128.1418P, 2013Icar..222...20F}.

Although compact and easy to implement, we caution that our results only strictly apply to long-period orbits, where quadrupole-level expansion of the planetary disturbing function provides an acceptable description of the long-term dynamics. We further remind the reader of the various approximations that we have employed in our formalism. Specifically, we have neglected Neptune's eccentricity along with perturbations arising from the other planets, and have limited our analysis to a common plane. Of course, the solar system is not a 2D restricted three-body problem, meaning that our analytic estimate of the stability boundary is, by construction, inexact. Still, a comparison of our results with direct $N$-body simulations indicates that our estimates are sufficiently close to their numerically computed counterparts to provide a useful blueprint for the dynamical architecture of the distant scattered disk.

We conclude this work by remarking that the stability boundary of the scattered disk does not correspond to a single value of the perihelion distance, as is often quoted in the literature. Instead, for long-period orbits, the critical perihelion distance slowly increases with semi-major axis. In other words, the gravitational ``reach" of Neptune's exterior resonances grows with $a$, such that chaos facilitated by Neptune covers a broader perihelion range at longer periods. Taken in isolation, however, scattered disk objects still obey the conservation of the Tisserand parameter, which is well approximated by the preservation of the perihelion distance for highly eccentric long-period orbits (see Appendix). This means that objects that stochastically diffuse outward through the scattered disk do so at approximately constant $q$, and Neptune scattering alone cannot readily populate the large-$a$ chaotic parameter space with $q\gtrsim36\,$AU. For this reason, the generation of chaotic high-perihelion TNOs must be interpreted as a dynamical signature of the interplay between Neptune's exterior $2:j$ resonances and extrinsic gravitational effects that sculpt the outermost regions of the solar system.

%Beyond illuminating the underlying physics, our result makes an important prediction regarding the perihelion gap of the distant scattered disk. At present, the census of distant (sometimes called ``extreme", although the boundary between extreme and run-of-the-mill objects is not a sharp one) TNOs is divided into two populations: those with $q<50\,$AU and those with $q>65\,$AU. The dearth of objects with perihelia in the $50-60\,$AU range is routinely referred to as the perihelion gap, and its dynamical origins have been a subject of considerable interest. Our work, however, indicates that the boundary of the scattered disk itself is not limited to a single value of $q$ and SDOs with semi-major axes in excess of $a\gtrsim1500\,$AU are expected to have $q\gtrsim65\,$AU. Indeed, it would appear that separating the Census of TNOs along the stability boundary itself is a more interesting exercise than sorting them according to $q$.



\begin{acknowledgments}
We are indebted to Alessandro Morbidelli, Matt Clement, and Mike Brown for illuminating discussions, as well as to Dan Tamayo for providing a thorough and insightful referee report. We are additionally grateful to Hanno Rein for sharing his expertise in numerical implementation of chaos indicators. K.B. is grateful to Caltech, and the David and Lucile Packard Foundation for their generous support. 
\end{acknowledgments}


\begin{appendix} \label{sec:app}

In the following text, we consider the relationship between the Jacobi constant, the Tisserand parameter, the perihelion distance, and the resonant integral of motion $\Psi$. To start this discussion, let us go back to the full Hamiltonian of the circular planar restricted three-body problem: 
\begin{align}
\Ham=\frac{1}{2}\bigg(P_r^2+\frac{P_{\theta}^2}{r^2} \bigg)-\frac{\G\,\Msun}{r}+V(r,\theta-n_{\rm{p}}\,t)+\mathcal{T},
\end{align}
where $P_r$ is the specific linear momentum conjugate to $r$, $P_{\theta}$ is the specific angular momentum conjugate to the azimuthal angle $\theta$, $\mathcal{T}$ is a dummy action conjugate to $t$, $V$ is the planetary potential, and $n_{\rm{p}}$ is the planetary mean motion. 

\paragraph{The Jacobi Constant} Arguably the most fundamental integral of the restricted three-body problem is the Jacobi constant, which follows directly from the Hamiltonian. Defining a contact transformation through the type-2 generating function $\mathcal{F}_2=(r)\,P_{r}'+(\theta-n_{\rm{p}}\,t)\,P_{\theta}'+(t)\,\Xi$, we have $P_r=P_r'$, $P_{\theta}=P_{\theta}'$, and $\mathcal{T}=\Xi-n_{\rm{p}}\,P_{\theta}$. This canonical change of variables corresponds to a transition into a reference frame that co-rotates with the planet at the orbital frequency $n_{\rm{p}}$, such that the new azimuthal angle is $\theta'=\theta-n_{\rm{p}}\,t$. 

Dropping the new constant dummy action $\Xi$, the Hamiltonian is now expressed as:
\begin{align}
\Ham=\frac{1}{2}\bigg(P_r'^2+\frac{P_{\theta}'^2}{r'^2} \bigg)-\frac{\G\,\Msun}{r'}+V(r',\theta')-n_{\rm{p}}\,P_{\theta}'.
\end{align}
Because $\Ham$ now has no explicit time dependence, it is conserved. This locum of energy in a rotating frame, $\Ham$, \textit{is} the Jacobi constant (technically, the conventional expression of the Jacobi constant differs from $\Ham$ by a factor of $-2$, but this is obviously irrelevant).

\paragraph{The Tisserand Parameter}  The above expression contains one term that is guaranteed to be much smaller than others: by virtue of being proportional to the planet-star mass ratio, $V(r',\theta')$ is assuredly negligible. Accordingly, employing the usual expression for the specific energy of a Keplerian orbit and noting that $P_{\theta}'=\sqrt{\G\,\Msun\,a\,(1-e^2)}$, we obtain:
\begin{align}
\Ham \approx - \frac{\G\,\Msun}{2a}-n_{\rm{p}}\,\sqrt{\G\,\Msun\,a\,(1-e^2)} +\order\bigg(\frac{m_{\rm{p}}}{\Msun} \bigg).
\end{align}
Scaling this expression by the inverse specific energy of the planet, $-2\,a_{\rm{p}}/(\G\,\Msun))$, we obtain the Tisserand parameter:
\begin{align}
T=\alpha+2\sqrt{\frac{1-e^2}{\alpha}}.
\end{align}

\paragraph{The Perihelion Distance} As discussed in the main text of the article, distant scattered disk orbits are characterized by small semi-major axis ratios $\alpha\ll1$ and near-unity eccentricities. Accordingly, expanding the above expression for $T$ to zeroth order in $\alpha$ around $0$ and first order in $e$ around $1$, we obtain
\begin{align}
T\approx2\,\sqrt{2}\,\sqrt{\frac{1-e}{\alpha}}+\order\big(\sqrt{\alpha},(1-e)^{3/2} \big).
\end{align}
Multiplying the square of this approximate expression for the Tisserand parameter by $a_{\rm{p}}/8$, we recover the perihelion distance:
\begin{align}
q=a\,(1-e)\approx\frac{a_{\rm{p}}}{8}\,T^2.
\end{align}

\paragraph{The Resonant Integral} A key consequence of the conservation of the action $\Psi$ (defined in equation \ref{transform}) is that changes in the Delaunay actions $L=\sqrt{\G\,\Msun\,a}$ and $G=\sqrt{\G\,\Msun\,a\,(1-e^2)}$, are related through $\Delta L=\chi\,\Delta G/2$. Let us examine this relationship in further detail. Returning to the ``exact" expression for the Tisserand parameter, let us express it in terms of Delaunay variables: 
\begin{align}
T=\sqrt{\frac{G^2}{\G\,\Msun\,a_{\rm{p}}}}+\frac{\G\,\Msun\,a_{\rm{p}}}{2\,L^2}
\end{align}

Taking the finite difference, we have:
\begin{align}
\sqrt{\frac{1}{\G\,\Msun\,a_{\rm{p}}}}\,\Delta G=\frac{\G\,\Msun\,a_{\rm{p}}}{[L]^3}\,\Delta L.
\end{align}
Rearranging the expression and noting that in the vicinity of a $2:\chi$ resonance $(a/a_{\rm{p}})^{3/2}\approx\chi/2$, we obtain
\begin{align}
\Delta L=\bigg(\frac{a}{a_{\rm{p}}} \bigg)^{3/2}\, \Delta G = \frac{\chi}{2}\,\Delta G.
\end{align}
This is result is identical to the one that ensues from the conservation of $\Psi$, implying that (to within an additive constant) $\Psi$ \textit{is} a near-resonant approximation to the Tisserand parameter.

The above formulae highlight the fact that the approximate maintenance of the perihelion distance by scattered disk objects, the preservation of the resonant action we employed in our analysis, as well as the near-constancy of the Tisserand parameter -- which is itself nothing other than an approximation to the Jacobi constant -- are all re-statements of the same conservation law.

\end{appendix}


\begin{thebibliography}{00}

%A

\bibitem[Adams(2010)]{2010ARA&A..48...47A} Adams, F.~C.\ 2010, \araa, 48, 47. doi:10.1146/annurev-astro-081309-130830


%B

\bibitem[Batygin et al.(2011)]{2011ApJ...738...13B} Batygin, K., Brown, M.~E., \& Fraser, W.~C.\ 2011, \apj, 738, 13. doi:10.1088/0004-637X/738/1/13

\bibitem[Batygin \& Morbidelli(2013)]{2013A&A...556A..28B} Batygin, K. \& Morbidelli, A.\ 2013, \aap, 556, A28. doi:10.1051/0004-6361/201220907

\bibitem[Batygin et al.(2015)]{2015ApJ...799..120B} Batygin, K., Morbidelli, A., \& Holman, M.~J.\ 2015, \apj, 799, 120. doi:10.1088/0004-637X/799/2/120

\bibitem[Batygin et al.(2019)]{2019PhR...805....1B} Batygin, K., Adams, F.~C., Brown, M.~E., et al.\ 2019, \physrep, 805, 1. doi:10.1016/j.physrep.2019.01.009

\bibitem[Brown et al.(2004)]{2004ApJ...617..645B} Brown, M.~E., Trujillo, C., \& Rabinowitz, D.\ 2004, \apj, 617, 645. doi:10.1086/422095

%C

\bibitem[Casati et al.(1988)]{1988IJQE...24.1420C} Casati, G., Guarneri, I., \& Shepelianskii, D.~L.\ 1988, IEEE Journal of Quantum Electronics, 24, 1420. doi:10.1109/3.982


\bibitem[Chambers(1999)]{1999MNRAS.304..793C} Chambers, J.~E.\ 1999, \mnras, 304, 793. doi:10.1046/j.1365-8711.1999.02379.x

\bibitem[Chirikov(1959)]{1959SPhD....4..390C} Chirikov, B.~V.\ 1959, Soviet Physics Doklady, 4, 390

\bibitem[Chirikov(1979)]{Chirikov1979} Chirikov, B.~V.\ 1979, \physrep, 52, 263. doi:10.1016/0370-1573(79)90023-1

\bibitem[Chirikov \& Vecheslavov(1989)]{Chirikov1989} Chirikov, R.~V. \& Vecheslavov, V.~V.\ 1989, \aap, 221, 146

\bibitem[Cincotta \& Sim{\'o}(2000)]{MEGNO} Cincotta, P.~M. \& Sim{\'o}, C.\ 2000, \aaps, 147, 205. doi:10.1051/aas:2000108

\bibitem[Clement \& Sheppard(2021)]{2021arXiv210501065C} Clement, M.~S. \& Sheppard, S.~S.\ 2021, arXiv:2105.01065


%D

\bibitem[Duncan et al.(1998)]{1998AJ....116.2067D} Duncan, M.~J., Levison, H.~F., \& Lee, M.~H.\ 1998, \aj, 116, 2067. doi:10.1086/300541


%E

\bibitem[Ellis \& Murray(2000)]{2000Icar..147..129E} Ellis, K.~M. \& Murray, C.~D.\ 2000, \icarus, 147, 129. doi:10.1006/icar.2000.6399


%F

\bibitem[Fouchard et al.(2013)]{2013Icar..222...20F} Fouchard, M., Rickman, H., Froeschl{\'e}, C., et al.\ 2013, \icarus, 222, 20. doi:10.1016/j.icarus.2012.10.027

%G

\bibitem[Gomes et al.(2008)]{Gomes2008} Gomes, R.~S., Fern{\'a}ndez, J.~A., Gallardo, T., et al.\ 2008, The Solar System Beyond Neptune, 259

\bibitem[Gontis \& Kaulakys(1987)]{1987JPhB...20.5051G} Gontis, V. \& Kaulakys, B.\ 1987, Journal of Physics B Atomic Molecular Physics, 20, 5051. doi:10.1088/0022-3700/20/19/016


%H

\bibitem[Hansen(1885)]{Hansen1885} Hansen, P.~A.,\ 1885, Abhandlungen der Mathematisch-Physischen Class der Königlich Sachsischen Gesselschaft der Wissenschaften, vol. 2, pp. 181–281. Leipzig

\bibitem[Hagihara(1970)]{1970ceme.book.....H} Hagihara, Y.\ 1970, Cambridge, Mass.: Massachusetts Institute of Technology (MIT), 1970

\bibitem[Holman \& Murray(1996)]{1996AJ....112.1278H} Holman, M.~J. \& Murray, N.~W.\ 1996, \aj, 112, 1278. doi:10.1086/118098

\bibitem[Hughes(1981)]{1981CeMec..25..101H} Hughes, S.\ 1981, Celestial Mechanics, 25, 101. doi:10.1007/BF01301812


%I

%J

\bibitem[Jensen et al.(1988)]{1988JPhB...21L.527J} Jensen, R.~V., Leopold, J.~G., \& Richards, D.\ 1988, Journal of Physics B Atomic Molecular Physics, 21, L527. doi:10.1088/0953-4075/21/17/001


%K

\bibitem[Kaula(1962)]{1962AJ.....67..300K} Kaula, W.~M.\ 1962, \aj, 67, 300. doi:10.1086/108729

\bibitem[Khain et al.(2020)]{2020PASP..132l4401K} Khain, T., Becker, J.~C., \& Adams, F.~C.\ 2020, \pasp, 132, 124401. doi:10.1088/1538-3873/abbd8a

\bibitem[Khriplovich \& Shepelyansky(2009)]{2009IJMPD..18.1903K} Khriplovich, I.~B. \& Shepelyansky, D.~L.\ 2009, International Journal of Modern Physics D, 18, 1903. doi:10.1142/S0218271809015758

%L

\bibitem[Lan \& Malhotra(2019)]{2019CeMDA.131...39L} Lan, L. \& Malhotra, R.\ 2019, Celestial Mechanics and Dynamical Astronomy, 131, 39. doi:10.1007/s10569-019-9917-1

\bibitem[Laskar \& Robutel(1995)]{1995CeMDA..62..193L} Laskar, J. \& Robutel, P.\ 1995, Celestial Mechanics and Dynamical Astronomy, 62, 193. doi:10.1007/BF00692088

\bibitem[Laskar(2008)]{Laskar2008} Laskar, J.\ 2008, \icarus, 196, 1. doi:10.1016/j.icarus.2008.02.017

\bibitem[Laskar \& Bou{\'e}(2010)]{Laskar2010DISTFUNCT} Laskar, J. \& Bou{\'e}, G.\ 2010, \aap, 522, A60. doi:10.1051/0004-6361/201014496

\bibitem[Lichtenberg \& Lieberman(1992)]{LLbook} Lichtenberg, A. \& Lieberman, M.\ 1992, Regular and Chaotic Dynamics, Second Edition, by A. Lichtenberg and M. Lieberman. Springer-Verlag, New York, ISBN 0-387-97745-7, 1992

\bibitem[Lithwick \& Wu(2011)]{2011ApJ...739...31L} Lithwick, Y. \& Wu, Y.\ 2011, \apj, 739, 31. doi:10.1088/0004-637X/739/1/31



%M

\bibitem[Malyshkin \& Tremaine(1999)]{Malyshkin1999} Malyshkin, L. \& Tremaine, S.\ 1999, \icarus, 141, 341. doi:10.1006/icar.1999.6174


\bibitem[Mardling(2008)]{Mardling2008Chaos} Mardling, R.~A.\ 2008, The Cambridge N-Body Lectures, 59. doi:10.1007/978-1-4020-8431-7\_3

\bibitem[Mardling(2013)]{2013MNRAS.435.2187M} Mardling, R.~A.\ 2013, \mnras, 435, 2187. doi:10.1093/mnras/stt1438

\bibitem[Morbidelli(2002)]{Morbybook} Morbidelli, A.\ 2002, Modern celestial mechanics : aspects of solar system dynamics. London: Taylor \& Francis, ISBN 0415279399

%\bibitem[Morbidelli \& Levison(2004)]{Morby2004} Morbidelli, A. \& Levison, H.~F.\ 2004, \aj, 128, 2564. doi:10.1086/424617

\bibitem[Morbidelli \& Nesvorn{\'y}(2020)]{2020tnss.book...25M} Morbidelli, A. \& Nesvorn{\'y}, D.\ 2020, The Trans-Neptunian Solar System, 25. doi:10.1016/B978-0-12-816490-7.00002-3

\bibitem[Murray \& Holman(1997)]{1997AJ....114.1246M} Murray, N. \& Holman, M.\ 1997, \aj, 114, 1246. doi:10.1086/118558

\bibitem[Murray \& Dermott(1999)]{MD99} Murray, C.~D. \& Dermott, S.~F.\ 1999, Solar system dynamics. Cambridge, UK: Cambridge University Press, ISBN 0-521-57295-9

%N

\bibitem[Nesvorn{\'y} \& Morbidelli(1998)]{1998CeMDA..71..243N} Nesvorn{\'y}, D. \& Morbidelli, A.\ 1998, Celestial Mechanics and Dynamical Astronomy, 71, 243. doi:10.1023/A:1008347020890

\bibitem[Nesvorn{\'y} \& Vokrouhlick{\'y}(2016)]{2016ApJ...825...94N} Nesvorn{\'y}, D. \& Vokrouhlick{\'y}, D.\ 2016, \apj, 825, 94. doi:10.3847/0004-637X/825/2/94

\bibitem[Nesvorn{\'y} et al.(2017)]{Nesvorny2017} Nesvorn{\'y}, D., Vokrouhlick{\'y}, D., Dones, L., et al.\ 2017, \apj, 845, 27. doi:10.3847/1538-4357/aa7cf6

\bibitem[Nesvorn{\'y}(2018)]{Nesvorny2018REV} Nesvorn{\'y}, D.\ 2018, \araa, 56, 137. doi:10.1146/annurev-astro-081817-052028

%O

%P

\bibitem[Pan \& Sari(2004)]{2004AJ....128.1418P} Pan, M. \& Sari, R.\ 2004, \aj, 128, 1418. doi:10.1086/423214

\bibitem[Petrosky(1986)]{1986PhLA..117..328P} Petrosky, T.~Y.\ 1986, Physics Letters A, 117, 328. doi:10.1016/0375-9601(86)90673-0


%Q

%R

\bibitem[Rein \& Tamayo(2016)]{2016MNRAS.459.2275R} Rein, H. \& Tamayo, D.\ 2016, \mnras, 459, 2275. doi:10.1093/mnras/stw644

\bibitem[Rein et al.(2019a)]{2019MNRAS.485.5490R} Rein, H., Hernandez, D.~M., Tamayo, D., et al.\ 2019, \mnras, 485, 5490. doi:10.1093/mnras/stz769

\bibitem[Rein et al.(2019b)]{2019MNRAS.489.4632R} Rein, H., Tamayo, D., \& Brown, G.\ 2019, \mnras, 489, 4632. doi:10.1093/mnras/stz2503


%S

\bibitem[Sadov(2006)]{Sadov2006} Sadov, S.~Y.\ 2006, Cosmic Research, 44, 160. https://doi.org/10.1134/S0010952506020080

\bibitem[Sadov(2008)]{2008CeMDA.100..287S} Sadov, S.~Y.\ 2008, Celestial Mechanics and Dynamical Astronomy, 100, 287. doi:10.1007/s10569-008-9123-z

\bibitem[Saillenfest(2020)]{Saillenfest2020} Saillenfest, M.\ 2020, Celestial Mechanics and Dynamical Astronomy, 132, 12. doi:10.1007/s10569-020-9954-9

\bibitem[Shepelyansky(1994)]{1994PhRvA..50..575S} Shepelyansky, D.~L.\ 1994, \pra, 50, 575. doi:10.1103/PhysRevA.50.575

\bibitem[Shevchenko(1998)]{1998PhLA..241...53S} Shevchenko, I.~I.\ 1998, Physics Letters A, 241, 53. doi:10.1016/S0375-9601(98)00093-0

\bibitem[Shevchenko(2011)]{2011NewA...16...94S} Shevchenko, I.~I.\ 2011, \na, 16, 94. doi:10.1016/j.newast.2010.06.008

\bibitem[Sidlichovsky \& Nesvorny(1994)]{1994A&A...289..972S} Sidlichovsky, M. \& Nesvorny, D.\ 1994, \aap, 289, 972


%T

\bibitem[Tsiganis et al.(2005)]{2005Natur.435..459T} Tsiganis, K., Gomes, R., Morbidelli, A., et al.\ 2005, \nat, 435, 459. doi:10.1038/nature03539


%U

%V

\bibitem[Volk et al.(2018)]{2018AJ....155..260V} Volk, K., Murray-Clay, R.~A., Gladman, B.~J., et al.\ 2018, \aj, 155, 260. doi:10.3847/1538-3881/aac268

%W

\bibitem[Wang \& Uhlenbeck(1945)]{1945RvMP...17..323W} Wang, M.~C. \& Uhlenbeck, G.~E.\ 1945, Reviews of Modern Physics, 17, 323. doi:10.1103/RevModPhys.17.323

\bibitem[Wisdom(1980)]{1980AJ.....85.1122W} Wisdom, J.\ 1980, \aj, 85, 1122. doi:10.1086/112778

\bibitem[Wisdom(1982)]{Wisdom1982} Wisdom, J.\ 1982, \aj, 87, 577. doi:10.1086/113132

\bibitem[Wisdom \& Holman(1991)]{1991AJ....102.1528W} Wisdom, J. \& Holman, M.\ 1991, \aj, 102, 1528. doi:10.1086/115978


%X

%Y

%Z


\end{thebibliography}

%\bibliography{apssamp}% Produces the bibliography via BibTeX.



\end{document}  

\section*{Acknowledgements}
We thank Marta Volonteri for discussions on the merger rate comparisons. The simulations were performed on the Bridges and Vera clusters at the Pittsburgh Super-computing Center (PSC). TDM acknowledges funding from NSF ACI-1614853, NSF AST-1616168, NASA ATP 19-ATP19-0084 NASA ATP 80NSSC18K101, NASA ATP NNX17AK56G, and 80NSSC20K0519.
SPB was supported by NSF grant AST-1817256.


%%%%%%%%%%%%%%%%%%%% REFERENCES %%%%%%%%%%%%%%%%%%

\bibliographystyle{mnras}
\bibliography{main}

%%%%%%%%%%%%%%%%%%%%%%%%%%%%%%%%%%%%%%%%%%%%%%%%%%

%%%%%%%%%%%%%%%%% APPENDICES %%%%%%%%%%%%%%%%%%%%%

\appendix
\onecolumn


% \tableofcontents{}

% \newpage

\section*{Supplementary Material}
\addcontentsline{toc}{section}{Supplementary Material}


Throughout this discussion, 
we will make frequently use 
of the following standard results
concerning the exponential concentration 
of random variables:

\begin{lemma}[Hoeffding's inequality for independent RVs~\citep{hoeffding1994probability}] Let $Z_1, Z_2, \ldots, Z_n$ be independent bounded random variables with $Z_i \in [a,b]$ for all $i$, then 
    \begin{align*}
        \prob\left( \frac{1}{n} \sum_{i=1}^n (Z_i - \Expo{Z_i}) \ge t \right) \le \exp{\left( -\frac{2nt^2}{(b-a)^2} \right) }
    \end{align*} 
    and 
    \begin{align*}
        \prob\left( \frac{1}{n} \sum_{i=1}^n (Z_i - \Expo{Z_i}) \le -t \right) \le \exp{\left( -\frac{2nt^2}{(b-a)^2} \right) }
    \end{align*} 
    for all $t \ge 0$. 
\end{lemma}

\begin{lemma}[Hoeffding's inequality for sampling with replacement~\citep{hoeffding1994probability}] \label{lem:hoeffding_sampling} Let $\calZ = (Z_1, Z_2, \ldots, Z_N)$ be a finite population of $N$ points with $Z_i \in [a.b]$ for all $i$. Let $X_1, X_2, \ldots X_n$ be a random sample drawn without replacement from $\calZ$. Then for all $t \ge 0$, we have 
    \begin{align*}
        \prob\left( \frac{1}{n} \sum_{i=1}^n (X_i - \mu ) \ge t \right) \le \exp{\left( -\frac{2nt^2}{(b-a)^2} \right) }
    \end{align*} 
    and 
    \begin{align*}
        \prob\left( \frac{1}{n} \sum_{i=1}^n (X_i - \mu ) \le -t \right) \le \exp{\left( -\frac{2nt^2}{(b-a)^2} \right) } \,,
    \end{align*} 
    where $\mu = \frac{1}{N} \sum_{i=1}^{N} Z_i$. 
\end{lemma}

We now discuss one condition that generalizes the exponential concentration to dependent random variables.
\begin{condition}[Bounded difference inequality] \label{cond:BDC} Let $\calZ$ be some set and $\phi: \calZ^n \to \Real$. We say that $\phi$ satisfies the bounded difference assumption if 
there exists $c_1, c_2, \ldots c_n \ge 0$ s.t. for all $i$, we have 
\begin{align*}
    \sup_{Z_1,Z_2, \ldots,Z_n, Z_i^\prime \in \calZ^{n+1} } \abs{\phi (Z_1, \ldots, Z_i, \ldots, Z_n ) - \phi (Z_1, \ldots, Z_i^\prime, \ldots, Z_n ) } \le c_i \,.
\end{align*} 
\end{condition}

\begin{lemma}[McDiarmid’s inequality~\citep{mcdiarmid1989}] \label{lem:McDiarmid} Let $Z_1, Z_2, \ldots, Z_n$ be independent random variables on set $\calZ$ and $\phi : \calZ^n \to \Real$ satisfy bounded difference inequality (\codref{cond:BDC}). Then for all $t>0$, we have 
    \begin{align*}
        \prob\left( \phi(Z_1, Z_2, \ldots, Z_n) - \Expo{\phi(Z_1, Z_2, \ldots, Z_n)} \ge t \right) \le \exp{\left( -\frac{2t^2}{\sum_{i=1}^n c_i^2} \right) } 
    \end{align*} 
    and 
    \begin{align*}
        \prob\left( \phi(Z_1, Z_2, \ldots, Z_n) - \Expo{\phi(Z_1, Z_2, \ldots, Z_n)} \le -t \right) \le \exp{\left( -\frac{2t^2}{\sum_{i=1}^n c_i^2} \right) } \,.
    \end{align*} 
\end{lemma}


\section{Proofs from \secref{sec:ERM_training}}\label{app:proof_erm}

\textbf{Additional notation {} {}} Let $m_1$ be the number of mislabeled points ($\wt S_M$) and $m_2$ be the number of correctly labeled points ($\wt S_C$). Note $m_1 + m_2 = m$. 


\subsection{Proof of \thmref{thm:error_ERM}}


\begin{proof}[Proof of \lemref{lem:fit_mislabeled}] 
    The main idea of our proof is to regard 
    the clean portion of the data 
    ($S \cup \wt S_C$) as fixed.   
    Then, there exists an (unknown) classifier $f^*$ 
    that minimizes the expected risk
    calculated on the (fixed) clean data
    and (random draws of) the mislabeled data $\wt S_M$. 
    % 
    % 
    Formally, 
    \begin{align}
    f^* \defeq \argmin_{f \in \calF} \error_{\widecheck {\calD}} (f) \,, \label{eq:modified_ERM}
    \end{align}
    where $$\widecheck \calD = \frac{n}{m+n} \calS + \frac{m_2}{m+n} \wt \calS_C  + \frac{m_1}{m+n}\calDm \,.$$ 
    Note here that $\widecheck \calD$ is a combination 
    of the \emph{empirical distribution} 
    over correctly labeled data $S \cup \wt S_C$
    and the (population) distribution 
    over mislabeled data $\calDm$.
    Recall that 
    \begin{align}
    \wh f \defeq \argmin_{f \in \calF} \error_{\calS \cup \wt S} (f) \,. \label{eq:orig_ERM}
    \end{align}
    % 
    % 
    Since, $\widehat f$ minimizes 0-1 error 
    on $S \cup \wt S$, using ERM optimality on \eqref{eq:orig_ERM},  
    we have 
    \begin{align}
        \error_{\calS \cup \wt \calS}(\widehat f) \le \error_{
            \calS \cup \wt \calS}(f^*) \,.    \label{eq:step1}
    \end{align}
    Moreover, since $f^*$ is independent of $\wt S_M$, using Hoeffding's bound,
    % \footnote{For a fully rigorous argument,
    % refer to the complete proof in App.~\ref{app:proof_erm}.} 
    we have with probability at least $1-\delta$ that
    \begin{align}
      \error_{\wt \calS_M}(f^*) \le \error_{ \calDm}(f^*) +  \sqrt{\frac{\log(1/\delta)}{2 m_1}} \,. \label{eq:step2} 
    \end{align}
    %$ 
    %for some constant $c_1\le 1/2$. 
    Finally, since $f^*$ is the optimal classifier on $\widecheck \calD$, 
    we have 
    \begin{align}
        \error_{\widecheck \calD}(f^*) \le \error_{\widecheck \calD}(\widehat f) \,. \label{eq:step3}
    \end{align}
    Now to relate \eqref{eq:step1} and \eqref{eq:step3}, we multiply \eqref{eq:step2} by $\frac{m_1}{m+n}$ and add $\frac{n}{m+n} \error_{\calS} (f)  + \frac{m_2}{m+n} \error_{\wt \calS_C} (f)$ both the sides. Hence, 
    we can rewrite \eqref{eq:step2} as follows: 
    \begin{align}
        \error_{\calS \cup \wt\calS}(f^*) \le \error_{ \widecheck \calD}(f^*) +  \frac{m_1}{m+n}\sqrt{\frac{\log(1/\delta)}{2 m_1}} \,. \label{eq:step4} 
    \end{align}
    Now we combine equations \eqref{eq:step1}, \eqref{eq:step4}, and \eqref{eq:step3}, to get 
    \begin{align}
        \error_{\calS \cup \wt \calS}(\wh f) \le \error_{\widecheck \calD}(\wh f) +  \frac{m_1}{m+n}\sqrt{\frac{\log(1/\delta)}{2 m_1}} \,, 
    \end{align}
    which implies 
    \begin{align}
        \error_{ \wt \calS_M}(\wh f) \le \error_{\calDm}(\wh f) + \sqrt{\frac{\log(1/\delta)}{2 m_1}} \,. \label{eq:lemma1_final}
    \end{align}
    Since $\wt S$ is obtained by randomly labeling an unlabeled dataset, we assume $2m_1 \approx m$ \footnote{Formally, with probability at least $1-\delta$, we have  $(m - 2m_1)\le \sqrt{m\log(1/\delta)/2}$.}. Moreover, using $\error_{\calDm} = 1 - \error_{\calD}$ we obtain the desired result.   
    % Combining the above steps and using the fact 
    % that $\error_\calD = 1- \error_{\calDm} $, 
    % we obtain the desired result.
\end{proof}

\begin{proof}[Proof of \lemref{lem:mislabeled_error}]
    Recall $\error_{\wt S} (f) = \frac{m_1}{m} \error_{\wt S_M}(f) + \frac{m_2}{m} \error_{\wt S_C}(f)$. Hence, we have 
    \begin{align}
        2\error_{\wt S}(f) - \error_{\wt S_M}(f) - \error_{\wt S_C}(f) &= \left(\frac{2m_1}{m} \error_{\wt S_M}(f) - \error_{\wt S_M}(f)\right) + \left(\frac{2m_2}{m} \error_{\wt S_C}(f) - \error_{\wt S_C}(f)\right) \\ &= \left(\frac{2m_1}{m} - 1\right) \error_{\wt S_M}(f) + \left(\frac{2m_2}{m} - 1 \right)\error_{\wt S_C} (f) \,.
    \end{align} 
    Since the dataset is labeled uniformly at random, with probability at least $1-\delta$, we have  $\left(\frac{2m_1}{m} - 1\right) \le \sqrt{\frac{\log(1/\delta)}{2m}}$. Similarly, we have with probability at least $1-\delta$, $\left(\frac{2m_2}{m} - 1\right) \le \sqrt{\frac{\log(1/\delta)}{2m}}$. Using union bound, with probability at least $1-\delta$, we have
    % \begin{align}
    %     2\error_{\wt S} - \error_{\wt S_M}(f) - \error_{\wt S_C}(f) \le \sqrt{\frac{\log(2/\delta)}{2m}} \left(\error_{\wt S_M}(f) + \error_{\wt S_C}(f) \right) \le 2\sqrt{\frac{\log(2/\delta)}{2m}} \,. \label{eq:lemma2_final}
    % \end{align}
    \begin{align}
        2\error_{\wt S} - \error_{\wt S_M}(f) - \error_{\wt S_C}(f) \le \sqrt{\frac{\log(2/\delta)}{2m}} \left(\error_{\wt S_M}(f) + \error_{\wt S_C}(f) \right) \,. \label{eq:lemma2_prefinal}
    \end{align}
    With re-arranging $\error_{\wt S_M}(f) + \error_{\wt S_C}(f)$ and using the inequality $ 1- a\le \frac{1}{1+a} $, we have  
    \begin{align}
        2\error_{\wt S} - \error_{\wt S_M}(f) - \error_{\wt S_C}(f) \le 2\error_{\wt \calS} \sqrt{\frac{\log(2/\delta)}{2m}}  \,. \label{eq:lemma2_final}
    \end{align}

    % We obtain the desired result by using 
\end{proof}

\begin{proof}[Proof of \lemref{lem:clear_error}]
% Recall 0-1 error on each point  $(x,y) \in S \cup \wt S$ is given by $\I{ f(x)\ne y}$.
In the set of correctly labeled points $S \cup \wt S_C$, we have $S$ as a random subset of $S \cup \wt S_C$. Hence, using Hoeffding's inequality for sampling without replacement (\lemref{lem:hoeffding_sampling}), we have with probability at least $1-\delta$
\begin{align}
    \error_{\wt \calS_C} (\wh f)- \error_{\calS \cup \wt \calS_C}( \wh f) \le  \sqrt{\frac{\log(1/\delta)}{2m_2}} \,.
\end{align}
Re-writing $\error_{\calS \cup \wt \calS_C}( \wh f)$ as $\frac{m_2}{m_2 + n} \error_{\wt \calS_C }(\wh f) + \frac{n}{m_2 + n} \error_{\calS }(\wh f)$, we have with probability at least $1-\delta$
\begin{align}
   \left(\frac{n}{n+m_2}\right) \left(\error_{\wt \calS_C} (\wh f)- \error_{\calS}( \wh f) \right) \le  \sqrt{\frac{\log(1/\delta)}{2m_2}} \,.
\end{align}
As before, assuming $2m_2 \approx m$, we have with probability at least $1-\delta$ 
\begin{align}
    \error_{\wt \calS_C} (\wh f)- \error_{\calS}( \wh f) \le \left(1+\frac{m_2}{n}\right)  \sqrt{\frac{\log(1/\delta)}{m}} \le \left(1 + \frac{m}{2n}\right) \sqrt{\frac{\log(1/\delta)}{m}} \,. \label{eq:lemma3_final}
\end{align} 
\end{proof}

\begin{proof}[Proof of \thmref{thm:error_ERM}] 
    Having established these core intermediate results, we can now combine above three lemmas to prove the main result. 
    In particular, we bound the population error on clean data ($\error_\calD(\wh f)$) as follows:  
    \begin{enumerate}[(i)]
        \item First, use \eqref{eq:lemma1_final}, to obtain an upper bound on the population error on clean data, i.e., with probability at least $1-\delta/4$, we have
        \begin{align}
            \error_{ \calD} (\wh f) \le 1 - \error_{ \wt \calS_M}(\wh f) + \sqrt{\frac{\log(4/\delta)}{m}} \,. 
        \end{align}
        \item  Second, use \eqref{eq:lemma2_final}, to relate the error on the mislabeled fraction with error on clean portion of randomly labeled data and error on whole randomly labeled dataset, i.e., with probability at least $1-\delta/2$, we have 
        \begin{align}
            - \error_{\wt S_M}(f) \le \error_{\wt S_C}(f) - 2\error_{\wt S}  + 2\error_{\wt S} \sqrt{\frac{\log(4/\delta)}{2m}}  \,. 
        \end{align} 
        \item Finally, use \eqref{eq:lemma3_final} to relate the error on the clean portion of randomly labeled data and error on clean training data, i.e., with probability $1-\delta/4$, we have 
        \begin{align}
            \error_{\wt \calS_C} (\wh f)\le - \error_{\calS}( \wh f) + \left(1 + \frac{m}{2n} \right) \sqrt{\frac{\log(4/\delta)}{m}} \,. 
        \end{align} 
    \end{enumerate}

    Using union bound on the above three steps, we have with probability at least $1-\delta$: 
    \begin{align}
        \error_\calD (\wh f) \le \error_{\calS}(\wh f)   + 1 - 2\error_{\wt \calS}(\wh f)   + \left(\sqrt{2} \error_{\wt S} + 2 + \frac{m}{2n}\right)  \sqrt{\frac{\log(4/\delta)}{m}} \,.
    \end{align}
    % Note that $(1/\sqrt{2} + 2.5)$ is a loose constant. In experiments, we use the ratio $\frac{m}{n}$
    %  the exact error $\error_{\wt \calS}(\wh f)$ 
    % to evaluate R.H.S.    
\end{proof}

\subsection{Proof of \propref{prop:rademacher}}

\begin{proof}[Proof of \propref{prop:rademacher}]
    For a classifier $ f: \calX \to \{-1, 1\}$, we have $1 - 2\,\indict{ f(x) \ne y} = y \cdot f(x)$. Hence, by definition of $\error$, we have 
    \begin{align}
        1 -2\error_{\wt \calS}(f) = \frac{1}{m}\sum_{i=1}^m y_i \cdot f(x_i) \le \sup_{f \in \calF} \, \frac{1}{m} \sum_{i=1}^m y_i \cdot f(x_i)  \,. \label{eq:error_rademacher}
    \end{align}
    Note that for fixed inputs $(x_1, x_2, \ldots, x_m)$ in $\wt S$, $(y_1, y_2, \ldots y_m)$ are random labels. Define $\phi_1 (y_1, y_2, \ldots, y_m) \defeq \sup_{f \in \calF} \, \frac{1}{m} \sum_{i=1}^m y_i \cdot f(x_i)$. We have the following bounded difference condition on $\phi_1$. For all i, 
    \begin{align}
        \sup_{y_1, \ldots y_m, y_i^\prime \in \{-1, 1\}^{m+1} } \abs{ \phi_1 (y_1,\ldots, y_i, \ldots, y_m) - \phi_1 (y_1,\ldots, y_i^\prime, \ldots, y_m)  } \le 1/m \,. \label{cond1_rademacher}
    \end{align} 
    
    Similarly, we define $\phi_2 (x_1, x_2, \ldots, x_m) \defeq \Expt{ y_i \sim_U \{-1, 1\}  }{ \sup_{f \in \calF} \, \frac{1}{m}  \sum_{i=1}^m y_i \cdot f(x_i)}$. We have the following bounded difference condition on $\phi_2$. 
    For all i,
    \begin{align}
        \sup_{x_1, \ldots x_m, x_i^\prime \in \calX^{m+1} } \abs{ \phi_2 (x_1,\ldots, x_i, \ldots, x_m) - \phi_1 (x_1,\ldots, x_i^\prime, \ldots, x_m)  } \le 1/m \,. \label{cond2_rademacher}
    \end{align}
    Using McDiarmid’s inequality (\lemref{lem:McDiarmid}) twice 
    with Condition \eqref{cond1_rademacher} and \eqref{cond2_rademacher}, 
    with probability at least $1-\delta$, we have
    \begin{align}
        \sup_{f \in \calF} \, \frac{1}{m} \sum_{i=1}^m y_i \cdot f(x_i)  - \Expt{x,y}{\sup_{f \in \calF} \, \frac{1}{m} \sum_{i=1}^m y_i \cdot f(x_i) } \le \sqrt{\frac{2\log(2/\delta)}{m}} \,. \label{eq:final_rademacher}
    \end{align} 
    Combining \eqref{eq:error_rademacher} and \eqref{eq:final_rademacher}, we obtain the desired result. 
\end{proof}


\subsection{Proof of \thmref{thm:error_regularized_ERM}}

Proof of \thmref{thm:error_regularized_ERM} follows similar to the proof of \thmref{thm:error_ERM}. Note that the same results in \lemref{lem:fit_mislabeled}, \lemref{lem:mislabeled_error}, and \lemref{lem:clear_error} hold in the regularized ERM case. However, the arguments in the proof of \lemref{lem:fit_mislabeled} change slightly. Hence, we state the lemma for regularized ERM and prove it here for completeness. 

\begin{lemma} \label{lem:lemma1_reg}
    Assume the same setup as \thmref{thm:error_regularized_ERM}. 
    Then for any $\delta >0$, with probability at least  $1-\delta$ 
    over the random draws of mislabeled data $\wt S_M$, we have 
    \begin{align}
        \error_\calD(\widehat f)  \le 1 -\error_{\wt \calS_M}(\widehat f) + \sqrt{\frac{\log(1/\delta)}{m}}\,. 
    \end{align} 
\end{lemma}
\begin{proof}
    The main idea of the proof remains the same, i.e. regard 
    the clean portion of the data 
    ($S \cup \wt S_C$) as fixed.   
    Then, there exists a classifier $f^*$ 
    that is optimal over draws 
    of the mislabeled data $\wt S_M$. 

    
    Formally, 
    \begin{align}
    f^* \defeq \argmin_{f \in \calF} \error_{\widecheck {\calD}} (f)  + \lambda R(f) \,, \label{eq:modified_ERM_reg}
    \end{align}
    where $$\widecheck \calD = \frac{n}{m+n} \calS + \frac{m_1}{m+n} \wt \calS_C  + \frac{m_2}{m+n}\calDm \,.$$ That is, $\widecheck \calD$ a combination of 
    the \emph{empirical distribution} 
    over correctly labeled data $S \cup \wt S_C$
    % in $S\cup \wt S$ 
    and the (population) distribution 
    over mislabeled data $\calDm$.
    Recall that 
    \begin{align}
    \wh f \defeq \argmin_{f \in \calF} \error_{\calS \cup \wt S} (f) + \lambda R(f) \,. \label{eq:orig_ERM_reg}
    \end{align}
    % 
    % 
    Since, $\widehat f$ minimizes 0-1 error 
    on $S \cup \wt S$, using ERM optimality on \eqref{eq:orig_ERM},  
    we have 
    \begin{align}
        \error_{\calS \cup \wt \calS}(\widehat f) + \lambda R(\wh f) \le \error_{
            \calS \cup \wt \calS}(f^*) + \lambda R(f^*) \,.    \label{eq:step1_reg}
    \end{align}
    Moreover, since $f^*$ is independent of $\wt S_M$, using Hoeffding's bound,
    % \footnote{For a fully rigorous argument,
    % refer to the complete proof in App.~\ref{app:proof_erm}.} 
    we have with probability at least $1-\delta$ that
    \begin{align}
      \error_{\wt \calS_M}(f^*) \le \error_{ \calDm}(f^*) +  \sqrt{\frac{\log(1/\delta)}{2 m_1}} \,. \label{eq:step2_reg} 
    \end{align}
    %$ 
    %for some constant $c_1\le 1/2$. 
    Finally, since $f^*$ is the optimal classifier on $\widecheck \calD$, 
    we have 
    \begin{align}
        \error_{\widecheck \calD}(f^*) + \lambda R(f^*) \le \error_{\widecheck \calD}(\widehat f) + \lambda R(\wh f) \,. \label{eq:step3_reg}
    \end{align}
     Now to relate \eqref{eq:step1_reg} and \eqref{eq:step3_reg}, we can re-write the \eqref{eq:step2_reg} as follows: 
    \begin{align}
        \error_{\calS \cup \wt\calS}(f^*) \le \error_{ \widecheck \calD}(f^*) +  \frac{m_1}{m+n}\sqrt{\frac{\log(1/\delta)}{2 m_1}} \,. \label{eq:step4_reg} 
    \end{align}
    After adding $\lambda R(f^*)$ on both sides in \eqref{eq:step4_reg}, we combine equations \eqref{eq:step1_reg}, \eqref{eq:step4_reg}, and \eqref{eq:step3_reg}, to get 
    \begin{align}
        \error_{\calS \cup \wt \calS}(\wh f) \le \error_{\widecheck \calD}(\wh f) +  \frac{m_1}{m+n}\sqrt{\frac{\log(1/\delta)}{2 m_1}} \,, 
    \end{align}
    which implies 
    \begin{align}
        \error_{ \wt \calS_M}(\wh f) \le \error_{\calDm}(\wh f) + \sqrt{\frac{\log(1/\delta)}{2 m_1}} \,. \label{eq:lemma_reg_final}
    \end{align}
    Similar as before, since $\wt S$ is obtained by randomly labeling an unlabeled dataset, we assume 
    $2m_1 \approx m$. Moreover, using $\error_{\calDm} = 1 - \error_{\calD}$ we obtain the desired result. 
\end{proof}
% \begin{proof}[Proof of ]
    
% \end{proof}

\subsection{Proof of \thmref{thm:multiclass_ERM}}

To prove our results in the multiclass case,
we first state and prove lemmas
parallel to those
% We first state and prove lemmas 
% parallel 
% to the three lemmas 
used in the proof of balanced binary case. 
We then combine these results 
% in the three lemmas 
to obtain the result in \thmref{thm:multiclass_ERM}. 

Before stating the result, 
we define mislabeled distribution $\calDm$ for any $\calD$.
While $\calDm$ and $\calD$ share 
the same marginal distribution over inputs $\calX$,
the conditional distribution over labels $y$ 
given an input $x\sim \calD_\calX$ is changed as follows:
For any $x$, the Probability Mass Function (PMF) over $y$ is defined as:  
$p_{\calDm} (\cdot \vert x) \defeq \frac{1 - p_{\calD}(\cdot \vert x)}{k - 1}$, where $ p_{\calD}(\cdot \vert x)$ is the PMF over $y$ for the distribution $\calD$. 

\begin{lemma} \label{lem:fit_mislabeled_multi}
    Assume the same setup as \thmref{thm:multiclass_ERM}. 
    Then for any $\delta >0$, with probability at least  $1-\delta$ 
    over the random draws of mislabeled data $\wt S_M$, we have 
    \begin{align}
        \error_\calD(\widehat f)  \le (k-1)\left(1 -\error_{\wt \calS_M}(\widehat f)\right) + (k-1)\sqrt{\frac{\log(1/\delta)}{m}}\,. \label{eq:lemma1_multi}
    \end{align}   
\end{lemma} 

\begin{proof}
   
    The main idea of the proof remains the same.
    We begin by regarding the clean portion of the data 
    ($S \cup \wt S_C$) as fixed. 
    Then, there exists a classifier $f^*$ 
    that is optimal over draws 
    of the mislabeled data $\wt S_M$. 
    
    However, in the multiclass case,
    we cannot as easily relate the population error on mislabeled data 
    to the population accuracy on clean data.   
    While for binary classification, 
    % we could upper bound $\error_{\wt \calS_M}$ 
    % with $1-\error_\calD$ 
    we could lower bound the population accuracy $1-\error_\calD$
    with the empirical error on mislabeled data $\error_{\wt \calS_M}$ 
    (in the proof of \lemref{lem:fit_mislabeled}), 
    for multiclass classification, 
    error on the mislabeled data 
    and accuracy on the clean data 
    in the population 
    are not so directly related.  
    To establish \eqref{eq:lemma1_multi},
    we break the error on the 
    (unknown) mislabeled data 
    into two parts: one term corresponds 
    to predicting the true label on mislabeled data, 
    and the other corresponds to predicting 
    neither the true label 
    nor the assigned (mis-)label.  
    Finally, we relate these errors to their
    population counterparts to establish \eqref{eq:lemma1_multi}. 
    
    Formally, 
    \begin{align}
    f^* \defeq \argmin_{f \in \calF} \error_{\widecheck {\calD}} (f)  + \lambda R(f) \,, \label{eq:modified_ERM_reg2}
    \end{align}
    where $$\widecheck \calD = \frac{n}{m+n} \calS + \frac{m_1}{m+n} \wt \calS_C  + \frac{m_2}{m+n}\calDm \,.$$ 
    That is, $\widecheck \calD$ is a combination 
    of the \emph{empirical distribution} 
    over correctly labeled data $S \cup \wt S_C$
    % in $S\cup \wt S$ 
    and the (population) distribution 
    over mislabeled data $\calDm$.
    Recall that 
    \begin{align}
    \wh f \defeq \argmin_{f \in \calF} \error_{\calS \cup \wt S} (f) + \lambda R(f) \,. \label{eq:orig_ERM_reg2}
    \end{align}
    % 
    % 
    Following the exact steps from the proof of \lemref{lem:lemma1_reg}, 
    with probability at least $1-\delta$, we have  
    \begin{align}
        \error_{ \wt \calS_M}(\wh f) \le \error_{\calDm}(\wh f) + \sqrt{\frac{\log(1/\delta)}{2 m_1}} \,. \label{eq:lemma1_final_multi_prev}
    \end{align}
    Similar to before, since $\wt S$ is obtained 
    by randomly labeling an unlabeled dataset, 
    we assume 
    $\frac{k}{k-1} m_1 \approx m$. 
    
    Now we will relate $\error_{\calDm} (\wh f)$ with $\error_{\calD}(\wh f)$. 
    Let $y^T$ denote the (unknown) true label 
    for a mislabeled point $(x, y)$ 
    (i.e., label before replacing it with a mislabel). 
    \begin{align*}    
         \Expt{(x, y) \in \sim \calDm}{\indict{ \wh f(x) \ne y }}  &= \underbrace{\Expt{(x, y) \in \sim \calDm}{\indict{ \wh f(x) \ne y \land \wh f(x) \ne y^T}}}_{\RN{1}} \\ &\qquad \qquad + \underbrace{\Expt{(x, y) \in \sim \calDm}{\indict{ \wh f(x) \ne y \land \wh f(x) = y^T}}}_{\RN{2}} \,. \numberthis \label{eq:excess_term}
    \end{align*}
    Clearly, term 2 is one minus the accuracy 
    on the clean unseen data, i.e.,
    \begin{align}
        \RN{2} = 1 - \Expt{{x,y} \sim \calD}{ \indict{ \wh f(x) \ne y}} = 1- \error_{\calD}(\wh f) \,. \label{eq:term1}    
    \end{align}
    Next, we relate term 1 with the error on the unseen clean data. 
    We show that term 1 is equal to the error on the unseen clean data 
    scaled by $\frac{k-2}{k-1}$,
    where $k$ is the number of labels.
    Using the definition of mislabeled distribution $\calDm$,  
    we have 
    \begin{align}
        \RN{1} = \frac{1}{k-1} \left( \Expt{(x, y) \in \sim \calD}{ \sum_{i \in \calY \land i\ne y}  \indict{ \wh f(x) \ne i \land \wh f(x) \ne y}} \right) = \frac{k-2}{k-1} \error_{\calD}(\wh f) \,.\label{eq:term2}
    \end{align}    

    Combining the result in \eqref{eq:term1}, \eqref{eq:term2} and \eqref{eq:excess_term}, we have 
    \begin{align}
        \error_{\calDm}(\wh f) = 1- \frac{1}{k-1} \error_{\calD}(\wh f) \,.\label{eq:combine_terms}
    \end{align}
    Finally, combining the result in \eqref{eq:combine_terms} 
    with equation \eqref{eq:lemma1_final_multi_prev}, 
    we have with probability $1-\delta$, 
    \begin{align}
      \error_{\calD}(\wh f) \le  (k-1) \left( 1- \error_{ \wt \calS_M}(\wh f) \right)  + (k-1) \sqrt{\frac{k \log(1/\delta)}{ 2(k-1)m}} \,. \label{eq:lemma1_final_multi}
    \end{align}
\end{proof}

\begin{lemma} \label{lem:mislabeled_error_multi}
    Assume the same setup as \thmref{thm:multiclass_ERM}. 
    Then for any $\delta >0$, 
    with probability at least $1-\delta$ 
    over the random draws of $\wt S$, we have  
    % \begin{align}
        $$\abs{k\error_{\wt \calS}(\widehat f) - \error_{\wt \calS_C}(\widehat f) -  (k-1)\error_{\wt \calS_M}(\widehat f) } \le  2k\sqrt{\frac{\log(4/\delta)}{2m}}\,. $$ % \label{eq:lemma2}
    % \end{align}   
    %  for some constant $c_3 \le 1.0\,$.
\end{lemma} 


\begin{proof}
    Recall $\error_{\wt S} (f) = \frac{m_1}{m} \error_{\wt S_M}(f) + \frac{m_2}{m} \error_{\wt S_C}(f)$. Hence, we have 
    \begin{align*}
        k\error_{\wt S}(f) - (k-1)\error_{\wt S_M}(f) - \error_{\wt S_C}(f) &= (k-1)\left(\frac{k m_1}{(k-1) m} \error_{\wt S_M}(f) - \error_{\wt S_M}(f)\right) \\ & \qquad \qquad + \left(\frac{km_2}{m} \error_{\wt S_C}(f) - \error_{\wt S_C}(f)\right) \\ &= k \left[ \left(\frac{m_1}{m} - \frac{k-1}{k}\right) \error_{\wt S_M}(f) + \left(\frac{m_2}{m} - \frac{1}{k} \right) \error_{\wt S_C} (f) \right] \,.
    \end{align*} 
    Since the dataset is randomly labeled, 
    we have with probability at least $1-\delta$, 
    $\left(\frac{m_1}{m} - \frac{k-1}{k}\right) \le \sqrt{\frac{\log(1/\delta)}{2m}}$. 
    Similarly, we have with probability at least $1-\delta$, 
    $\left(\frac{m_2}{m} - \frac{1}{k}\right) \le \sqrt{\frac{\log(1/\delta)}{2m}}$. 
    Using union bound, we have with probability at least $1-\delta$
    % \begin{align}
    %     2\error_{\wt S} - \error_{\wt S_M}(f) - \error_{\wt S_C}(f) \le \sqrt{\frac{\log(2/\delta)}{2m}} \left(\error_{\wt S_M}(f) + \error_{\wt S_C}(f) \right) \le 2\sqrt{\frac{\log(2/\delta)}{2m}} \,. \label{eq:lemma2_final}
    % \end{align}
    \begin{align}
        k\error_{\wt S}(f) - (k-1)\error_{\wt S_M}(f) - \error_{\wt S_C}(f)  \le k \sqrt{\frac{\log(2/\delta)}{2m}} \left(\error_{\wt S_M}(f) + \error_{\wt S_C}(f) \right) \,. \label{eq:lemma2_final_multi}
    \end{align}

    % We obtain the desired result by using 
\end{proof}

\begin{lemma} \label{lem:clear_error_multi}
    Assume the same setup as \thmref{thm:multiclass_ERM}. 
    Then for any $\delta >0$, with probability at least $1-\delta$ 
    over the random draws of $\wt S_C$ and $S$, we have 
    % \begin{align}
        $$\abs{\error_{\wt \calS_C}(\widehat f) - \error_{\calS}(\widehat f) } \le 1.5 \sqrt{\frac{k\log(2/\delta)}{2m}}\,.$$ %\label{eq:lemma3}
    % \end{align}   
    % for some constant $c_2 \le 1.2\,$.
\end{lemma} 
\begin{proof}
    % Recall 0-1 error on each point  $(x,y) \in S \cup \wt S$ is given by $\I{ f(x)\ne y}$.
    In the set of correctly labeled points $S \cup \wt S_C$,
    we have $S$ as a random subset of $S \cup \wt S_C$. 
    Hence, using Hoeffding's inequality 
    for sampling without replacement 
    (\lemref{lem:hoeffding_sampling}), 
    we have with probability at least $1-\delta$
    \begin{align}
        \error_{\wt \calS_c} (\wh f)- \error_{\calS \cup \wt \calS_C}( \wh f) \le  \sqrt{\frac{\log(1/\delta)}{2m_2}} \,.
    \end{align}
    Re-writing $\error_{\calS \cup \wt \calS_C}( \wh f)$ 
    as $\frac{m_2}{m_2 + n} \error_{\wt \calS_C }(\wh f) + \frac{n}{m_2 + n} \error_{\calS }(\wh f)$, 
    we have with probability at least $1-\delta$
    \begin{align}
       \left(\frac{n}{n+m_2}\right) \left(\error_{\wt \calS_c} (\wh f)- \error_{\calS}( \wh f) \right) \le  \sqrt{\frac{\log(1/\delta)}{2m_2}} \,.
    \end{align}
    As before, assuming $km_2 \approx m$, 
    we have with probability at least $1-\delta$ 
    \begin{align}
        \error_{\wt \calS_c} (\wh f)- \error_{\calS}( \wh f) \le \left(1+\frac{m_2}{n}\right)  \sqrt{\frac{k\log(1/\delta)}{2m}} \le \left( 1 + \frac{1}{k}\right) \sqrt{\frac{k\log(1/\delta)}{2m}} \,. \label{eq:lemma3_final_multi}
    \end{align} 
\end{proof}

\begin{proof}[Proof of \thmref{thm:multiclass_ERM}] 
    Having established these core intermediate results, 
    we can now combine above three lemmas. 
    In particular, we bound the population error 
    on clean data ($\error_\calD(\wh f)$) as follows:  
    \begin{enumerate}[(i)]
        \item First, use \eqref{eq:lemma1_final_multi}, 
        to obtain an upper bound on the population error on clean data, 
        i.e., with probability at least $1-\delta/4$, we have
        \begin{align}
            \error_{ \calD} (\wh f) \le (k-1)\left(1 - \error_{ \wt \calS_M}(\wh f) \right) + (k-1) \sqrt{\frac{k\log(4/\delta)}{2(k-1)m}} \,. 
        \end{align}
        \item  Second, use \eqref{eq:lemma2_final_multi}
        to relate the error on the mislabeled fraction 
        with error on clean portion of randomly labeled data 
        and error on whole randomly labeled dataset, 
        i.e., with probability at least $1-\delta/2$, we have 
        \begin{align}
            - (k-1)\error_{\wt S_M}(f) \le \error_{\wt S_C}(f) - k\error_{\wt S}  + k\sqrt{\frac{\log(4/\delta)}{2m}}  \,. 
        \end{align} 
        \item Finally, use \eqref{eq:lemma3_final_multi} 
        to relate the error on the clean portion of randomly labeled data 
        and error on clean training data, 
        i.e., with probability $1-\delta/4$, we have 
        \begin{align}
            \error_{\wt \calS_C} (\wh f)\le - \error_{\calS}( \wh f) + \left(1 + \frac{m}{kn} \right) \sqrt{\frac{k\log(4/\delta)}{2m}} \,. 
        \end{align} 
    \end{enumerate}

    Using union bound on the above three steps, 
    we have with probability at least $1-\delta$: 
    \begin{align}
        \error_\calD (\wh f) \le \error_{\calS}(\wh f) + (k-1) - k\error_{\wt \calS}(\wh f)   + (\sqrt{k(k-1)} + k + \sqrt{k} + \frac{m}{n\sqrt{k}})  \sqrt{\frac{\log(4/\delta)}{2m}} \,.\label{eq:multiclass_ERM_final}
    \end{align}
    Simplifying the term in RHS of \eqref{eq:multiclass_ERM_final}, 
    we get the desired result. 
    % Note that since $\frac{m}{n\sqrt{k}}$ 
    % is much smaller than the sum of the other terms
    % the other terms in summation, 
    % we ignore $\frac{m}{n\sqrt{k}}$  
    % Z: ??? --- great
    % that 
    % them
    in the final bound. 
    % we ignore that in the final bound. 
    % Note that $(1/\sqrt{2} + 2.5)$ is a loose constant. In experiments, we use the ratio $\frac{m}{n}$
    %  the exact error $\error_{\wt \calS}(\wh f)$ 
    % to evaluate R.H.S.    
\end{proof}

\newpage
\section{Proofs from \secref{sec:linear_models}}\label{app:proof_gd}
We suppose that the parameters of the linear function 
are obtained via gradient descent on 
the following $L_2$ regularized problem: 
\begin{align}
    % n in denominator is avoided deliberately
    \calL_S(w; \lambda) \defeq \sum_{i=1}^n{(w^Tx_i - y_i)^2} + \lambda \norm{w}{2}^2 \,, \label{eq:l2_MSE_app}   
\end{align}
where $\lambda\ge0$ is a regularization parameter. 
We assume access to a clean dataset 
$S = \{(x_i, y_i)\}_{i=1}^n \sim \calD^n$ 
and randomly labeled dataset 
$\wt S = \{(x_i, y_i)\}_{i=n+1}^{n+m} \sim \wt \calD^m$. 
Let $\bX = [x_1, x_2, \cdots, x_{m+n}]$ 
and $\by = [y_1, y_2, \cdots, y_{m+n}]$. 
Fix a positive learning rate $\eta$ such that 
$\eta \le 1/\left(\norm{\bX^T\bX}{\text{op}} + \lambda^2\right)$ 
and an initialization $w_0 = 0$. 
% \todos{Assumption made for simplicty}. 
Consider the following gradient descent iterates 
to minimize objective \eqref{eq:l2_MSE_app} on $S \cup \wt S$:
\begin{align}
w_t = w_{t-1} - \eta \grad_w \calL_{S \cup \wt S} (w_{t-1}; \lambda) \quad \forall t=1,2,\ldots \label{eq:GD_iterates_app}
\end{align} 
Then we have $\{ w_t\}$ converge to the limiting solution 
$\wh w = \left( \bX^T\bX+\lambda \boldsymbol{I}\right)^{-1}\bX^T\by$. Define $\widehat f (x) \defeq f(x ; \wh w) $.  

% \subsection{\textcolor{red}{Errata}}

% We wish to correct the following error in the body:
% \codref{cond:error_stability} is not enough 
% to guarantee the result in \thmref{thm:linear}. 
% We now present a slightly stronger condition 
% called \emph{hypothesis stability} 
% under which we obtain a result 
% similar to \thmref{thm:linear}. 

% This error doesn't change the main arguments of the proof,
% where we show that the empirical train error 
% is less than or equal to the leave-one-out error.
% We need a stronger condition to relate leave-one-out error 
% with the population error of the original classifier. 
% Specifically, while \codref{cond:error_stability} 
% relates the average population error of leave-one-out classifiers 
% with the population error of the original classifier, 
% we need the new condition to show the concentration 
% of the empirical leave-one-out error 
% and average population error of leave-one-out classifiers. 
% main takeaway 

% Note that the new condition, 
% while being stronger than the previous one, 
% still doesn't imply generalization \citep{bousquet2002stability,elisseeff2003leave,abou2019exponential}. 
% Overall, the main results in \secref{sec:ERM_training} 
% and takeaways of the paper remain unaffected by the error.  

% We now present the new condition 
% and a corrected statement of \thmref{thm:linear}. 
% Recall, for a given training set $S \sim \calD^n $, 
% we use $S_{(i)}$ to denote the training set $S$ 
% with the $i^{\text{th}}$ point removed.

% \begin{condition}[Hypothesis Stability] 
%     \label{cond:hypothesis_stability}
%     We have $\beta$ hypothesis stability 
%     if our training algorithm $\calA$ satisfies the following: 
%     \begin{align*}
%     % ${\sum_{i=1}^n \frac{\error_{\calD}( f(\calA, S_{(i)}))}{n} - \error_\calD(f(\calA, S))} \le \beta\,$.
%     \forall i \in \{1,2,\ldots, n\}, \quad  \Expt{\calS, (x,y) \in \calD}{ \abs{\error\left( f(x) ,y  \right) - \error\left( f_{(i)}(x), y \right) }} \le \frac{\beta}{n} \,,
%     \end{align*}
%     where $f_{(i)} \defeq f(\calA, S_{(i)})$ and $ f \defeq f(\calA, S)$.
% \end{condition}

% \begin{theorem}[Correct statement of \thmref{thm:linear}] \label{thm:new_linear}
%     Assume that this gradient descent algorithm satisfies \codref{cond:hypothesis_stability}
%     with $\beta=\calO(1)$.  
%     Then for any $\delta >0$, with probability at least $1-\delta$ 
%     over the random draws of datasets $\wt S$ and $S$, we have:
%     \begin{align}
%         \error_\calD(\widehat f) \le \error_\calS(\widehat f) + 1 - 2 \error_{\wt\calS}(\widehat f) + \left(\frac{1}{\sqrt{2}} + 1.5 \right) \sqrt{\frac{\log(4/\delta)}{m}} + \sqrt{\frac{4}{\delta}\left(\frac{1}{m} +\frac{3\beta}{m+n} \right)}  \,. \label{eq:gd_error}
%     \end{align} 
%     % for some constant $c\le 3.2$.
% \end{theorem}

\subsection{Proof of \thmref{thm:linear}}
We use a standard result from linear algebra, 
namely the Shermann-Morrison formula 
\citep{sherman1950adjustment} for matrix inversion:  

\begin{lemma}[\citet{sherman1950adjustment}] \label{lem:sherman}
    Suppose $\bA \in \Real^{n \times n}$ 
    is an invertible square matrix 
    and $u,v \in \Real^n$ are column vectors. 
    Then $\bA + uv^T$ is invertible iff $1 + v^T \bA u \ne 0$ 
    and in particular
    \begin{align}
        (\bA + u v^T)^{-1} = \bA^{-1}  - \frac{\bA^{-1} uv^T \bA^{-1} }{ 1 + v^T \bA^{-1} u} \,.
    \end{align}   
\end{lemma}
\newcommand\byy[1]{\by_{\left(#1\right)}}
\newcommand\bXX[1]{\bX_{\left(#1\right)}}
\newcommand\ff[1]{\wh f_{\left(#1\right)}}

For a given training set $S \cup \wt S_C$, 
define leave-one-out error 
on mislabeled points in the training data 
as $$\error_{\text{LOO}(\wt S_M) } = \frac{\sum_{(x_i, y_i) \in \wt S_M} \error( f_{(i)}( x_i), y_i)}{ \abs{\wt S_M }} \,, $$
where $f_{(i)} \defeq f(\calA, (S \cup \wt S)_{(i)})$. 
To relate empirical leave-one-out error and population error 
with hypothesis stability condition, 
we use the following lemma:   

\begin{lemma}[\citet{bousquet2002stability}] \label{lem:stability_error}
    For the leave-one-out error, we have
    \begin{align}
        \Expo{ \left( \error_{\calDm}(\wh f) -\error_{\text{LOO}(\wt S_M) } \right)^2 } \le \frac{1}{2m_1}+  \frac{3\beta}{n + m}\,.
    \end{align}   
    % where $ f \defeq f(\calA, S \cup \wt S) $.
\end{lemma}

Proof of the above lemma is similar 
to the proof of Lemma 9 in \citet{bousquet2002stability} 
and can be found in \appref{app:proof_lem_error}. 
% 
% Before presenting the result, we introduce some notation. 
Before presenting the proof of \thmref{thm:linear}, 
we introduce some more notation. 
Let $\bX_{(i)}$ denote the matrix of covariates 
with the $i^{\text{th}}$ point removed. 
Similarly, let $\by_{(i)}$ be the array of responses 
with the $i^{\text{th}}$ point removed. 
Define the corresponding regularized GD solution 
as $\wh w_{(i)} = \left( \bXX{i}^T\bXX{i}+\lambda \boldsymbol{I}\right)^{-1}\bXX{i}^T\byy{i}$. 
Define $\ff{i}(x) \defeq f(x ; \wh w_{(i)}) $.

\begin{proof}[Proof of \thmref{thm:linear}]
    Because squared loss minimization does not imply 0-1 error minimization, 
    we cannot use arguments from \lemref{lem:fit_mislabeled}. 
    This is the main technical difficulty. 
    To compare the 0-1 error at a train point with an unseen point, 
    we use the closed-form expression for $\widehat{w}$ 
    and Shermann-Morrison formula 
    to upper bound training error 
    with leave-one-out cross validation error. 
    
    The proof is divided into three parts: 
    In part one, we show that 0-1 error 
    on mislabeled points in the training set 
    is lower than the error obtained 
    by leave-one-out error at those points. 
    In part two, we relate this leave-one-out error 
    with the population error on mislabeled distribution
    using \codref{cond:hypothesis_stability}.
    While the empirical leave-one-out error is an unbiased estimator 
    of the average population error of leave-one-out classifiers, 
    we need hypothesis stability 
    to control the variance 
    of empirical leave-one-out error. 
    Finally, in part three, we show 
    that the error on the mislabeled training points 
    can be estimated with just the randomly labeled 
    and clean training data (as in proof of \thmref{thm:error_ERM}).  

    \textbf{Part 1 {} {}} First we relate training error with leave-one-out error.        
    For any training point $(x_i, y_i)$ in $\wt S \cup S$, we have 
    \begin{align}
        \error(\wh f(x_i), y_i ) &= \indict{ y_i \cdot x_i^T \wh w < 0 } = \indict{ y_i \cdot x_i^T \left( \bX^T\bX+\lambda \boldsymbol{I}\right)^{-1}\bX^T\by < 0 } \\
        &= \indict{ y_i \cdot x_i^T \underbrace{\left( \bXX{i}^T\bXX{i} + x_i ^T x_i +\lambda \boldsymbol{I}\right)^{-1}}_{\RN{1}} (\bXX{i}^T\byy{i} + y_i \cdot x_i) < 0 } \,.
    \end{align}
    Letting $\bA = \left(\bXX{i}^T\bXX{i} +\lambda \boldsymbol{I}\right)$ 
    and using \lemref{lem:sherman} on term 1, we have 
    \begin{align}
        \error(\wh f(x_i), y_i ) &= \indict{ y_i \cdot x_i^T \left[\bA^{-1} -  \frac{\bA^{-1} x_i x_i^T \bA^{-1}}{ 1 + x_i ^T \bA^{-1} x_i } \right] (\bXX{i}^T\byy{i} + y_i \cdot x_i) < 0 } \\
        &= \indict{ y_i \cdot\left[ \frac{ x_i^T \bA^{-1} ( 1 + x_i ^T \bA^{-1} x_i ) -  x_i^T \bA^{-1} x_i x_i^T \bA^{-1}}{ 1 + x_i ^T \bA ^{-1}x_i } \right] (\bXX{i}^T\byy{i} + y_i \cdot x_i) < 0 } \\
        &= \indict{ y_i \cdot\left[ \frac{ x_i^T \bA^{-1}}{ 1 + x_i ^T \bA ^{-1}x_i } \right] (\bXX{i}^T\byy{i} + y_i \cdot x_i) < 0 } \,.
    \end{align}

    Since $1 + x_i^T \bA^{-1} x_i > 0$, we have 
    \begin{align}
        \error(\wh f(x_i), y_i ) &= \indict{ y_i \cdot x_i^T \bA^{-1} (\bXX{i}^T\byy{i} + y_i \cdot x_i) < 0 } \\
        &= \indict{ x_i^T \bA^{-1} x_i +  y_i \cdot x_i^T \bA^{-1} (\bXX{i}^T\byy{i}) < 0 } \\
        &\le \indict{ y_i \cdot x_i^T \bA^{-1} (\bXX{i}^T\byy{i}) < 0 } = \error(\ff{i}(x_i), y_i ) \,.\label{eq:LOO_error}
    \end{align}

    Using \eqref{eq:LOO_error}, we have 
    \begin{align}
        \error_{\wt \calS_M } (\wh f) \le \error_{\text{LOO} (\wt S_M)} \defeq \frac{\sum_{(x_i, y_i) \in \wt S_M} \error(\ff{i}(x_i), y_i ) }{\abs{\wt \calS_M}}\label{eq:LOO_error_final} \,.
    \end{align}
    \textbf{Part 2 {}{}} We now relate RHS in \eqref{eq:LOO_error_final} 
    with the population error on mislabeled distribution. 
    To do this, we leverage \codref{cond:hypothesis_stability} 
    and \lemref{lem:stability_error}. 
    In particular, we have 

    \begin{align}
        \Expt{\calS \cup \wt \calS_M }{ \left(\error_{\calDm}(\wh f) - \error_{\text{LOO} (\wt S_M)}\right)^2 } \le \frac{1}{2m_1} + \frac{3\beta}{m+n} \,.
    \end{align}

    Using Chebyshev's inequality, with probability at least $1-\delta$, we have 
    \begin{align}
        \error_{\text{LOO} (\wt S_M)} \le  \error_{\calDm}(\wh f)   + \sqrt{\frac{1}{\delta}\left(\frac{1}{2m_1} +\frac{3\beta}{m+n} \right)} \,. \label{eq:final_mislabeled_linear}
    \end{align}
    

    \textbf{Part 3 {}{}} Combining \eqref{eq:final_mislabeled_linear} and \eqref{eq:LOO_error_final}, we have 

    \begin{align}
        \error_{\wt \calS_M } (\wh f) \le \error_{\calDm}(\wh f)   + \sqrt{\frac{1}{\delta}\left(\frac{1}{2m_1} +\frac{3\beta}{m+n} \right)} \,. \label{eq:linear_parallel_lem1}
    \end{align}

    Compare \eqref{eq:linear_parallel_lem1} with \eqref{eq:lemma1_final} 
    in the proof of \lemref{lem:fit_mislabeled}. 
    We obtain a similar relationship 
    between $\error_{\wt \calS_M }$ and $\error_{\calDm}$ 
    but with a polynomial concentration 
    instead of exponential concentration. 
    In addition, since we just use concentration arguments 
    to relate mislabeled error to the errors
    on the clean and unlabeled portions 
    of the randomly labeled data, 
    we can directly use the results 
    in \lemref{lem:mislabeled_error} and \lemref{lem:clear_error}. 
    Therefore, combining results in \lemref{lem:mislabeled_error}, \lemref{lem:clear_error}, and \eqref{eq:linear_parallel_lem1} with union bound, 
    we have with probability at least $1-\delta$
    \begin{align}
        \error_\calD(\widehat f) \le \error_\calS(\widehat f) + 1 - 2 \error_{\wt\calS}(\widehat f) + \left(\sqrt{2}\error_{\wt\calS}(\widehat f) + 1 + \frac{m}{2n} \right) \sqrt{\frac{\log(4/\delta)}{m}} + \sqrt{\frac{4}{\delta}\left(\frac{1}{m} +\frac{3\beta}{m+n} \right)}  \,.
    \end{align}
    

       
\end{proof}

\subsection{Extension to multiclass classification} \label{app:multiclass_linear}
For multiclass problems with squared loss minimization, as standard practice, we consider one-hot encoding for the underlying label, i.e., a class label $c \in [k]$ is treated as $(0, \cdot, 0,1,0, \cdot, 0) \in \Real^k$ (with $c$-th coordinate being 1).  As before, we suppose that the parameters of the linear function 
are obtained via gradient descent on the following $L_2$ regularized problem: 
\begin{align}
    % n in denominator is avoided deliberately
    \calL_S(w; \lambda) \defeq \sum_{i=1}^n\norm{w^Tx_i - y_i}{2}^2 + \lambda \sum_{j=1}^k \norm{w_j}{2}^2 \,, \label{eq:l2_multiclass_MSE_app}   
\end{align}
where $\lambda\ge0$ is a regularization parameter. 
We assume access to a clean dataset 
$S = \{(x_i, y_i)\}_{i=1}^n \sim \calD^n$ 
and randomly labeled dataset 
$\wt S = \{(x_i, y_i)\}_{i=n+1}^{n+m} \sim \wt \calD^m$. 
Let $\bX = [x_1, x_2, \cdots, x_{m+n}]$ 
and $\by = [e_{y_1}, e_{y_2}, \cdots, e_{y_{m+n}}]$. 
Fix a positive learning rate $\eta$ such that 
$\eta \le 1/\left(\norm{\bX^T\bX}{\text{op}} + \lambda^2\right)$ 
and an initialization $w_0 = 0$. 
% \todos{Assumption made for simplicty}. 
Consider the following gradient descent iterates 
to minimize objective \eqref{eq:l2_MSE_app} on $S \cup \wt S$:
\begin{align}
{w_j}^t = {w_j}^{t-1} - \eta \grad_{w_j} \calL_{S \cup \wt S} (w^{t-1}; \lambda) \quad \forall t=1,2,\ldots \text{ and } j=1,2,\ldots,k  \,. \label{eq:GD_multi_iterates_app}
\end{align} 
Then we have $\{ {w_j}^t\}$ for all $j =1,2,\cdots, k$ converge to the limiting solution 
$\wh w_j = \left( \bX^T\bX+\lambda \boldsymbol{I}\right)^{-1}\bX^T\by_j$. Define $\widehat f (x) \defeq f(x ; \wh w) $.  

\begin{theorem}\label{thm:multi_linear}
    Assume that this gradient descent algorithm satisfies \codref{cond:hypothesis_stability}
    with $\beta=\calO(1)$.  
    Then for a multiclass classification problem wth $k$ classes, for any $\delta >0$, with probability at least $1-\delta$, we have:
    \begin{align*}
        \error_\calD(\widehat f) \le \error_\calS(\widehat f) &+ (k-1)\left(1 - \frac{k}{k-1} \error_{\wt\calS}(\widehat f) \right) \\ &+ \left(k + \sqrt{k} + \frac{m}{n\sqrt{k}} \right) \sqrt{\frac{\log(4/\delta)}{2m}} + \sqrt{k(k-1)} \sqrt{\frac{4}{\delta}\left(\frac{1}{m} +\frac{3\beta}{m+n} \right)}  \,. \numberthis \label{eq:gd_multi_error}
    \end{align*} 
    % for some constant $c\le 3.2$.
\end{theorem}
\begin{proof}
    The proof of this theorem is divided into two parts. In the first part, we relate the error on the mislabeled samples with the population error on the mislabeled data. Similar to the proof of \thmref{thm:linear}, we use Shermann-Morrison formula to upper bound training error with leave-one-out error on each $\wh w^j$. Second part of the proof follows entirely from the proof of \thmref{thm:multiclass_ERM}. In essence, the first part derives an equivalent of \eqref{eq:lemma1_final_multi_prev} for GD training with squared loss and then the second part follows from the proof  of \thmref{thm:multiclass_ERM}. 
    
    \textbf{Part-1:} Consider a training point $(x_i,y_i)$ in $\wt S \cup S $. For simplicity, we use $c_i$ to denote the class of $i$-th point and use $y_i$ as the corresponding one-hot embedding. Recall error in multiclass point is given by $\error(\wh f(x_i), y_i ) = \indict{ c_i \not \in \argmax x_i^T \wh w }$. Thus, there exists a $j \ne c_i \in [k]$, such that we have
     \begin{align}
        \error(\wh f(x_i), y_i ) &= \indict{ c_i \not \in \argmax x_i^T \wh w } = \indict{ x_i^T \wh w_{c_i} < x_i^T \wh w_{j}  } \\ &= \indict{ x_i^T \left( \bX^T\bX+\lambda \boldsymbol{I}\right)^{-1}\bX^T\by_{c_i} < x_i^T \left( \bX^T\bX+\lambda \boldsymbol{I}\right)^{-1}\bX^T\by_{j} } \\
        &= \indict{ x_i^T \underbrace{\left( \bXX{i}^T\bXX{i} + x_i ^T x_i +\lambda \boldsymbol{I}\right)^{-1}}_{\RN{1}} \left(\bXX{i}^T{\by_{c_i}}_{(i)} + x_i - \bXX{i}^T{\by_{j}}_{(i)}\right) < 0 } \,.
    \end{align}
    Letting $\bA = \left(\bXX{i}^T\bXX{i} +\lambda \boldsymbol{I}\right)$ 
    and using \lemref{lem:sherman} on term 1, we have 
    \begin{align}
        \error(\wh f(x_i), y_i ) &= \indict{ x_i^T \left[\bA^{-1} -  \frac{\bA^{-1} x_i x_i^T \bA^{-1}}{ 1 + x_i ^T \bA^{-1} x_i } \right]  \left(\bXX{i}^T{\by_{c_i}}_{(i)} + x_i - \bXX{i}^T{\by_{j}}_{(i)}\right) < 0 } \\
        &= \indict{ \left[ \frac{ x_i^T \bA^{-1} ( 1 + x_i ^T \bA^{-1} x_i ) -  x_i^T \bA^{-1} x_i x_i^T \bA^{-1}}{ 1 + x_i ^T \bA ^{-1}x_i } \right]  \left(\bXX{i}^T{\by_{c_i}}_{(i)} + x_i - \bXX{i}^T{\by_{j}}_{(i)}\right) < 0 } \\
        &= \indict{ \left[ \frac{ x_i^T \bA^{-1}}{ 1 + x_i ^T \bA ^{-1}x_i } \right]  \left(\bXX{i}^T{\by_{c_i}}_{(i)} + x_i - \bXX{i}^T{\by_{j}}_{(i)}\right) < 0} \,.
    \end{align}
    Since $1 + x_i^T \bA^{-1} x_i > 0$, we have 
    \begin{align}
        \error(\wh f(x_i), y_i ) &= \indict{ x_i^T \bA^{-1}  \left(\bXX{i}^T{\by_{c_i}}_{(i)} + x_i - \bXX{i}^T{\by_{j}}_{(i)}\right) < 0 } \\
        &= \indict{ x_i^T \bA^{-1} x_i +  x_i^T \bA^{-1}  \bXX{i}^T{\by_{c_i}}_{(i)}  - x_i^T\bA^{-1}  \bXX{i}^T{\by_{j}}_{(i)} < 0 } \\
        &\le \indict{  x_i^T \bA^{-1}  \bXX{i}^T{\by_{c_i}}_{(i)}  - x_i^T\bA^{-1}  \bXX{i}^T{\by_{j}}_{(i)} < 0  } = \error(\ff{i}(x_i), y_i ) \,.\label{eq:LOO_error_multi}
    \end{align}
    Using \eqref{eq:LOO_error_multi}, we have 
    \begin{align}
        \error_{\wt \calS_M } (\wh f) \le \error_{\text{LOO} (\wt S_M)} \defeq \frac{\sum_{(x_i, y_i) \in \wt S_M} \error(\ff{i}(x_i), y_i ) }{\abs{\wt \calS_M}}\label{eq:LOO_error_multi_final} \,.
    \end{align}
    
    We now relate RHS in \eqref{eq:LOO_error_final} 
    with the population error on mislabeled distribution. 
    Similar as before, to do this, we leverage \codref{cond:hypothesis_stability} 
    and \lemref{lem:stability_error}. Using  \eqref{eq:final_mislabeled_linear} and \eqref{eq:LOO_error_multi_final}, we have 
    \begin{align}
        \error_{\wt \calS_M } (\wh f) \le \error_{\calDm}(\wh f)   + \sqrt{\frac{1}{\delta}\left(\frac{1}{2m_1} +\frac{3\beta}{m+n} \right)} \,. \label{eq:linear_multi_parallel_lem1}
    \end{align}
    
    We have now derived a parallel to \eqref{eq:lemma1_final_multi_prev}. Using the same arguments in the proof of \lemref{lem:fit_mislabeled_multi}, we have 
    \begin{align}
      \error_{\calD}(\wh f) \le  (k-1) \left( 1- \error_{ \wt \calS_M}(\wh f) \right)  + (k-1)\sqrt{\frac{k}{\delta(k-1)}\left(\frac{1}{2m_1} +\frac{3\beta}{m+n} \right)}  \,. \label{eq:lemma1_linear_final_multi}
    \end{align}
    
    \textbf{Part-2:} We now combine the results in \lemref{lem:mislabeled_error_multi} and \lemref{lem:clear_error_multi} to obtain the final inequality in terms of quantities that can be computed from just the randomly labeled and clean data. Similar to the binary case, we obtained a polynomial concentration instead of exponential concentration. Combining \eqref{eq:lemma1_linear_final_multi} with \lemref{lem:mislabeled_error_multi} and \lemref{lem:clear_error_multi}, we have with probability at least $1-\delta$
    \begin{align*}
        \error_\calD(\widehat f) \le \error_\calS(\widehat f) &+ (k-1)\left(1 - \frac{k}{k-1} \error_{\wt\calS}(\widehat f) \right) \\ &+ \left(k + \sqrt{k} + \frac{m}{n\sqrt{k}} \right) \sqrt{\frac{\log(4/\delta)}{2m}} + \sqrt{k(k-1)} \sqrt{\frac{4}{\delta}\left(\frac{1}{m} +\frac{3\beta}{m+n} \right)}  \,. \numberthis \label{eq:gd_multi_error_proof}
    \end{align*} 
\end{proof}

\subsection{Discussion on \codref{cond:hypothesis_stability}} \label{app:discuss_cond1}
The quantity in LHS of \codref{cond:hypothesis_stability} 
measures how much the function learned by the algorithm 
(in terms of error on unseen point) will change 
when one point in the training set is removed. 
% Discussion on exponential concentration and stronger condition. 
% Notice that hypothesis stability implies error stability, i.e., \codref{cond:error_stability} \citep{bousquet2002stability}.  
% In summary, while error stability allowed us 
% to relate the average population error 
% of the leave-one-out classifiers 
% with the population error of the original classifier, 
We need hypothesis stability condition 
to control the variance of the empirical leave-one-out error to show concentration of average leave-one-error with the population error. 

Additionally, we note that while the dominating term in the RHS of \thmref{thm:linear} matches with the dominating term in ERM bound in \thmref{thm:error_ERM}, there is a polynomial concentration term 
(dependence on $1/\delta$ instead of $\log(\sqrt{1/\delta})$) 
in \thmref{thm:linear}. 
Since with hypothesis stability, 
we just bound the variance, 
the polynomial concentration is due 
to the use of Chebyshev's inequality 
instead of an exponential tail inequality
(as in \lemref{lem:fit_mislabeled}).
Recent works have highlighted that 
a slightly stronger condition than hypothesis stability 
can be used to obtain an exponential concentration 
for leave-one-out error \citep{abou2019exponential},
but we leave this for future work for now. 
% We leave 
% However, the constants 

% we also want to highlight  

\subsection{Formal statement and proof of \propref{prop:early_stop}} \label{app:formal_early_stop}

Before formally presenting the result, 
we will introduce some notation.  
By $\calL_{S}(w)$, we denote 
the objective in \eqref{eq:l2_MSE_app} with $\lambda=0$. 
Assume Singular Value Decomposition (SVD) of $\bX$
as $\sqrt{n} \bU \bS^{1/2} \bV^T$. 
Hence $\bX^T \bX = \bV \bS \bV^T$.
Consider the GD iterates defined in \eqref{eq:GD_iterates_app}. 
% 
We now derive closed form expression 
for the $t^\text{th}$ iterate of gradient descent:  
% 
\begin{align}
    w_t = w_{t-1} + \eta \cdot \bX^T (\by - \bX w_{t-1}) = (\bI - \eta \bV \bS \bV^T )w_{k-1} + \eta \bX^T \by \,.
\end{align}
Rotating by $\bV^T$, we get 
\begin{align}
    \wt w_t = (\bI - \eta\bS )\wt w_{k-1} + \eta \wt \by \label{eq:GD_recur},
\end{align}
where $\wt w_t = \bV^T w_t $ and $\wt \by = \bV^T \bX^T \by$. 
Assuming the initial point $w_0 = 0$ 
and applying the recursion in \eqref{eq:GD_recur}, we get
\begin{align}
    \wt w_t = \bS ^{-1} ( \bI - (\bI - \eta \bS)^k ) \wt \by \,, 
\end{align} 
Projecting solution back to the original space, we have 
\begin{align}
     w_t = \bV \bS ^{-1} ( \bI - (\bI - \eta \bS)^k ) \bV^T \bX^T \by \,. 
\end{align} 
% We will work with this GD solution at any iterate $t$ in the next proposition. 
Define $f_t(x) \defeq f(x;w_t)$ 
as the solution at the $t^{\text{th}}$ iterate. 
Let $\wt w_{\lambda} = \argmin_{w} \calL_\calS (w;\lambda) = (\bX^T \bX + \lambda \bI)^{-1} \bX^T \by = \bV (\bS + \lambda \bI )^{-1} \bV^T \bX^T \by $. 
% ) \,,$ for all $t=1,2,\ldots\,.$ 
and define $\wt f_\lambda(x) \defeq f(x;\wt w_\lambda)$ as the regularized solution. 
Assume $\kappa$ be the condition number 
of the population covariance matrix 
and let $s_\text{min}$ be the minimum positive 
singular value of the empirical covariance matrix. 
Our proof idea is inspired from recent work 
on relating gradient flow solution 
and regularized solution 
for regression problems \citep{ali2018continuous}. 
We will use the following lemma in the proof: 
\begin{lemma} \label{lem:ineq_soln}
    For all $x \in [0,1]$ and for all $ k \in \mathbb{N}$, 
    we have (a) $ \frac{kx}{1+kx} \le 1- (1-x)^k$ 
    and (b) $ 1- (1-x)^k \le 2 \cdot \frac{kx}{kx+1} $.
    %  where $g(c)$ is a constant dependent on $c$. For $c = 1$, $g(c) = 2.0$.   
\end{lemma}
\begin{proof}
    % [Proof of \lemref{lem:ineq_soln}]
    % Part (a) is easy. 
    Using $ (1-x)^k \le \frac{1}{1+kx}$, we have part (a). 
    For part (b), we numerically maximize 
    $\frac{ (1+kx ) (1 - (1-x)^k) }{kx}$ 
    for all $k\ge 1$ and for all $x \in [0, 1]$.  
\end{proof}

% 
% Next, 

\begin{prop}[Formal statement of \propref{prop:early_stop}] \label{prop:formal_early_stop}
Let $\lambda = \frac{1}{t\eta}$. 
For a training point $x$, we have 
\begin{align*}
    \Expt{x \sim \calS}{(f_t(x) - \wt f_\lambda(x))^2} &\le c(t,\eta) \cdot \Expt{x \sim \calS}{f_t(x)^2} \,, %\label{eq:early_stop}
\end{align*}
where $c(t, \eta) \defeq \min( 0.25, \frac{1}{s_\text{min}^2 t^2 \eta^2})$. 
Similarly for a test point, we have 
\begin{align*}
    \Expt{x \sim \calD_\calX}{(f_t(x) - \wt f_\lambda(x))^2} &\le \kappa \cdot c(t,\eta) \cdot \Expt{x \sim \calD_\calX}{f_t(x)^2} \,. %\label{eq:early_stop}
\end{align*}
\end{prop} 

\begin{proof}
    %%%%%%%%%%%%% 
    We want to analyze the expected squared difference output 
    of regularized linear regression 
    with regularization constant $\lambda = \frac{1}{\eta t}$ 
    and the gradient descent solution at the $t^\text{th}$ iterate. 
    We separately expand the algebraic expression 
    for squared difference at a training point and a test point. 
    % We start by considering the difference  
    Then the main step is to show that 
    $\left[ \bS ^{-1} ( \bI - (\bI - \eta \bS)^k )  - (\bS + \lambda \bI )^{-1}\right] \preceq c(\eta, t) \cdot \bS ^{-1} ( \bI - (\bI - \eta \bS)^k ) $.

    %%%%%%%%%%%%%
    
   \textbf{Part 1 {} {}} 
    First, we will analyze the squared difference 
    of the output at a training point 
    (for simplicity, we refer to $S \cup \wt S$ as $S$), i.e., 
    \begin{align}
        \Expt{ x \sim \calS }{\left(f_t(x) - \wt f_\lambda (x)\right)^2} &= \norm{\bX w_t - \bX \wt w_\lambda}{2}^2\\ &=   \norm{\bX \bV \bS ^{-1} ( \bI - (\bI - \eta \bS)^t ) \bV^T \bX^T \by - \bX \bV (\bS + \lambda \bI )^{-1} \bV^T \bX^T \by }{2}^2 \\
        &= \norm{\bX \bV \left(\bS ^{-1} ( \bI - (\bI - \eta \bS)^t ) - (\bS + \lambda \bI )^{-1} \right) \bV^T \bX^T \by  }{2} \\
        &=  \by^T \bV \bX \left( \underbrace{\bS ^{-1} ( \bI - (\bI - \eta \bS)^t ) - (\bS + \lambda \bI )^{-1}}_{\RN{1}} \right)^2 \bS \bV^T \bX^T \by \label{eq:train_GD_rel} \,.
        %  (\bX \bV \bS ^{-1} ( \bI - (\bI - \eta \bS)^k ) \bV^T \bX^T \by)^T \bX \bV \bS ^{-1} ( \bI - (\bI - \eta \bS)^k ) \bV^T \bX^T \by
    \end{align}
    We now separately consider term 1. 
    Substituting $\lambda = \frac{1}{t \eta}$, 
    we get
    \begin{align}
        \bS ^{-1} ( \bI - (\bI - \eta \bS)^t ) - (\bS + \lambda \bI )^{-1} &= \bS^{-1} \left( ( \bI - (\bI - \eta \bS)^t ) - (\bI + \bS^{-1} \lambda )^{-1}\right) \\
        &= \underbrace{\bS^{-1} \left( ( \bI - (\bI - \eta \bS)^t ) - (\bI + ( \bS t \eta)^{-1}  )^{-1}\right)}_{\bA} \,.
    \end{align}

    We now separately bound the diagonal entries in matrix $\bA$. 
    With $s_i$, we denote $i^{\text{th}}$ diagonal entry of $\bS$.
    Note that since $ \eta\le 1/\norm{S}{\text{op}}$, 
    for all $i$, $\eta s_i  \le 1$.  
    Consider $i^{\text{th}}$ diagonal term (which is non-zero) 
    of the diagonal matrix $\bA$, we have 
    \begin{align}
        \bA_{ii} = \frac{1}{s_i} \left(  1 - (1 - s_i \eta)^t - \frac{t \eta s_i}{1 + t \eta s_i } \right) &=  \frac{1 - (1 - s_i \eta)^t}{s_i} \left( \underbrace{ 1 - \frac{t \eta s_i}{(1 + t \eta s_i)(1 - (1 - s_i \eta)^t)}}_{\RN{2}} \right) \\ 
         &\le \frac{1}{2}\left[ \frac{1 - (1 - s_i \eta)^t}{ s_i} \right] \tag*{(Using \lemref{lem:ineq_soln} (b))} \,.
    \end{align} 
    Additionally, we can also show the following upper bound on term 2: 
    \begin{align}
         1 - \frac{t \eta s_i}{(1 + t \eta s_i)(1 - (1 - s_i \eta)^t)} &= \frac{(1 + t \eta s_i)(1 - (1 - s_i \eta)^t) - t \eta s_i }{(1 + t \eta s_i)(1 - (1 - s_i \eta)^t)} \\
         & \le  \frac{ 1 -  (1 - s_i \eta)^t - t \eta s_i (1 - s_i \eta)^t}{(1 + t \eta s_i)(1 - (1 - s_i \eta)^t)} \\
         & \le \frac{1}{t\eta s_i} \,. \tag{Using \lemref{lem:ineq_soln} (a)}
        %  &\le \frac{1}{2}\left[ \frac{1 - (1 - s_i \eta)^t}{ s_i} \right] \tag*{(Using \lemref{lem:ineq_soln})} \,.
    \end{align} 

    Combining both the upper bounds 
    on each diagonal entry $\bA_{ii}$, we have 
    \begin{align}
    \bA \preceq c_1(\eta, t) \cdot \bS^{-1} ( \bI - (\bI - \eta \bS)^t ) \,, \label{eq:upperbound_diagonal}
    \end{align}
    where $c_1(\eta, t ) = \min(0.5, \frac{1}{t s_i \eta })$. Plugging this into \eqref{eq:train_GD_rel}, we have 
    \begin{align}
        \Expt{ x \sim \calS }{\left(f_t(x) - \wt f_\lambda (x)\right)^2} &\le c(\eta, t) \cdot \by^T \bV \bX  \left( \bS^{-1} ( \bI - (\bI - \eta \bS)^t ) \right)^2 \bS \bV^T \bX^T \by \\
        &=   c(\eta, t) \cdot \by^T \bV \bX  \left( \bS^{-1} ( \bI - (\bI - \eta \bS)^t ) \right) \bS \left( \bS^{-1} ( \bI - (\bI - \eta \bS)^t ) \right) \bV^T \bX^T \by \\
        & =  c(\eta, t) \cdot \norm{\bX w_t}{2}^2 \\
        &= c(\eta, t) \cdot  \Expt{ x \sim \calS }{\left(f_t(x) \right)^2} \,,
    \end{align}
    where $c(\eta, t ) = \min(0.25, \frac{1}{t^2 s^2_i \eta^2 })$.

    \textbf{Part 2 {} {}} With $\bSigma$, 
    we denote the underlying true covariance matrix. 
    We now consider the squared difference of output at an unseen point: 
    \begin{align}
        \Expt{ x \sim \calD_{\calX} }{\left(f_t(x) - \wt f_\lambda (x)\right)^2} &= \Expt{x \sim \calD_{\calX}}{\norm{x^T w_t - x^T \wt w_\lambda}{2}} \\
        &=   \norm{x^T \bV \bS ^{-1} ( \bI - (\bI - \eta \bS)^t ) \bV^T \bX^T \by - x^T \bV (\bS + \lambda \bI )^{-1} \bV^T \bX^T \by }{2} \\
        &= \norm{x^T \bV \left(\bS ^{-1} ( \bI - (\bI - \eta \bS)^t ) - (\bS + \lambda \bI )^{-1} \right) \bV^T \bX^T \by  }{2} \\
        &= \by^T \bV \bX \left( \bS ^{-1} ( \bI - (\bI - \eta \bS)^t ) - (\bS + \lambda \bI )^{-1} \right) \bV^T \bSigma \bV \\ &\qquad \qquad \qquad \qquad \qquad \left( (\bI - (\bI - \eta \bS)^t ) - (\bS + \lambda \bI )^{-1} \right) \bV^T \bX^T \by \\
        &\le \sigma_{\text{max}} \cdot \by^T \bV \bX \left( \underbrace{\bS ^{-1} ( \bI - (\bI - \eta \bS)^t ) - (\bS + \lambda \bI )^{-1}}_{\RN{1}} \right)^2 \bV^T \bX^T \by \,, \label{eq:test_GD_rel}
        %  (\bX \bV \bS ^{-1} ( \bI - (\bI - \eta \bS)^k ) \bV^T \bX^T \by)^T \bX \bV \bS ^{-1} ( \bI - (\bI - \eta \bS)^k ) \bV^T \bX^T \by
    \end{align}
    where $\sigma_{\text{max}}$ is the maximum eigenvalue 
    of the underlying covariance matrix $\bSigma$. 
    Using the upper bound on term 1 in \eqref{eq:upperbound_diagonal}, 
    we have 
    \begin{align}
        \Expt{ x \sim \calD_{\calX} }{\left(f_t(x) - \wt f_\lambda (x)\right)^2} &\le \sigma_{\text{max}} \cdot c(\eta, t) \cdot \by^T \bV \bX  \left( \bS^{-1} ( \bI - (\bI - \eta \bS)^t ) \right)^2 \bV^T \bX^T \by \\
        &=   \kappa \cdot c(\eta, t) \cdot \sigma_{\text{min}}\cdot \norm{\bV \left( \bS^{-1} ( \bI - (\bI - \eta \bS)^t ) \right) \bV^T \bX^T \by}{2}^2 \\
        &\le \kappa \cdot c(\eta, t) \cdot \left[ \bV \left( \bS^{-1} ( \bI - (\bI - \eta \bS)^t ) \right) \bV^T \bX^T \right]^T \bSigma \\
        &\qquad \qquad \qquad \qquad \qquad \left[ \bV \left( \bS^{-1} ( \bI - (\bI - \eta \bS)^t ) \right) \bV^T \bX^T \right] \by \\
        & = \kappa \cdot c(\eta, t) \cdot \Expt{x \sim \calD_{\calX}}{\norm{x^T w_t}{2}} \,.
    \end{align}
% 
% 
    % Since $ \eta\le 1/\norm{S}{\text{op}}$, invoking \lemref{lem:ineq_soln} to upper bound term 1 with
\end{proof}

\subsection{Extension to deep learning} \label{appsubsec:ext_DL}
Under \asmpref{appsubsec:justifying_assumption1}, we present the formal result parallel to \thmref{thm:multiclass_ERM}. 
\begin{theorem} \label{thm:multiclass_ERM_algoA}
    Consider a multiclass classification problem 
    with $k$ classes. Under \asmpref{asmp:deep_models}, 
    for any $\delta >0$, with probability at least $1-\delta$,
    we have
    \vspace{-10pt}
    \begin{align*}
        \error_\calD(\widehat f)  \le \error_\calS(\widehat f) + (k-1) \left(1 - \tfrac{k}{k-1} \error_{\wt\calS}(\widehat f)\right) + c\sqrt{\frac{\log(\frac{4}{\delta})}{2m}} \,,\numberthis \label{eq:multiclass_ERM_deep}
    % \vspace{-20pt}
    \end{align*}
    for some constant $c \le ((c+1) k+\sqrt{k} + \frac{m}{n\sqrt{k}})$.
\end{theorem}

The proof follows exactly as in step (i) to (iii) in \thmref{thm:multiclass_ERM}.  

\subsection{Justifying~\asmpref{asmp:deep_models}} \label{appsubsec:justifying_assumption1}

Motivated by the analysis on linear models, we now discuss alternate (and weaker) conditions that imply \asmpref{asmp:deep_models}. 
We need hypothesis stability (\codref{cond:hypothesis_stability}) and the following assumption relating training error and leave-one-error: 

\begin{assumption} \label{asmp:loo_error}
Let $\wh f$ be a model obtained by training with algorithm $\calA$ on a mixture of clean $S$ and randomly labeled data $\wt S$. Then we assume we have 
\begin{align*}
    \error_{\wt \calS_M} (\wh f) \le  \error_{\text{LOO} (\wt S_M)} \,, 
\end{align*}
for all $(x_i, y_i) \in  \wt S_M$ where $\wh f_{(i)} \defeq f(\calA, S \cup {{}\wt S_M}_{(i)})$ and  $\error_{\text{LOO} (\wt S_M)} \defeq  \frac{\sum_{(x_i, y_i) \in \wt S_M} \error(\ff{i}(x_i), y_i ) }{\abs{\wt \calS_M}}$.  
\end{assumption}

% we assume this to extend our result (parallel to \thmref{thm:multi_linear}) for deep models. 
Intuitively, this assumption states that the error on a (mislabeled) datum $(x,y)$ included in the training set is less than the error on that datum $(x,y)$ obtained by a model trained on the training set $S - \{(x,y)\}$. We proved this for linear models trained with GD in the proof of \thmref{thm:multi_linear}. 
% 
\codref{cond:hypothesis_stability} with $\beta = \calO(1)$ and \asmpref{asmp:loo_error} together with \lemref{lem:stability_error} implies \asmpref{asmp:deep_models} with a polynomial residual term (instead of logarithmic in $1/\delta$): 
\begin{align}
     \error_{\calS_M} (\wh f) \le  \error_{\calDm}(\wh f)   + \sqrt{\frac{1}{\delta}\left(\frac{1}{m} +\frac{3\beta}{m+n} \right)} \,.
\end{align}
% Note that this  

\newpage 
\section{Additional experiments and details}\label{app:exp}
\newcommand\tab[1][1cm]{\hspace*{#1}}

\subsection{Datasets} \label{sec:app_dataset}

\textbf{Toy Dataset {} {}} Assume fixed constants $\mu$ and $\sigma$. For a given label $y$, we simulate features $x$ in our toy classification setup as follows: 
\begin{align*}
    x \defeq \texttt{concat} \left[ x_1, x_2\right] \quad \text{where} \quad  x_1 \sim  \calN( y \cdot \mu, \sigma^2 I_{d \times d}) \ \  \text{and} \ \  x_1 \sim  \calN( 0, \sigma^2 I_{d \times d}) \,.
\end{align*}  
% where $y$ is the true label and $x$ is the corresponding feature vector. 
In experiements throughout the paper, we fix dimention $d=100$, $\mu = 1.0 $, and $\sigma = \sqrt{d}$. Intuitively, $x_1$ carries the information about the underlying label and $x_2$ is additional noise independent of the underlying label. 

\textbf{CV datasets {} {}} We use MNIST~\citep{lecun1998mnist} and CIFAR10~\cite{krizhevsky2009learning}. 
% For binary tasks, 
We produce a binary variant from the multiclass classification problem by mapping classes $\{0,1,2,3,4\}$ to label $1$ and $\{ 5,6,7,8,9\}$ to label $-1$. For CIFAR dataset, we also use the standard data augementation of random crop and horizontal flip. PyTorch code is as follows: 

\texttt{(transforms.RandomCrop(32, padding=4),\\
\tab transforms.RandomHorizontalFlip())}

\textbf{NLP dataset {} {}} We use IMDb Sentiment analysis~\citep{maas2011learning} corpus.  

\subsection{Architecture Details} 

All experiments were run on NVIDIA GeForce RTX 2080 Ti GPUs. We used PyTorch~\citep{NEURIPS2019a9015} and Keras with Tensorflow~\citep{abadi2016tensorflow} backend for experiments. 
% , ELMo embeddings~\citep{Peters:2018}, and Hugging Face Transformers~\citep{wolf-etal-2020-transformers}. 

\textbf{Linear model {} {}} For the toy dataset, we simulate a linear model with scalar output and the same number of parameters as the number of dimensions.   

\textbf{Wide nets {} {}} To simulate the NTK regime, we experiment with $2-$layered wide nets. The PyTorch code for 2-layer wide MLP is as follows: 


\texttt{ nn.Sequential( \\
\tab     nn.Flatten(),\\
\tab    nn.Linear(input\_dims, 200000, bias=True),\\
\tab    nn.ReLU(),\\
\tab    nn.Linear(200000, 1, bias=True)\\
\tab     )}


We experiment both (i) with the second layer fixed at random initialization; (ii)  and updating both layers' weights.     

\textbf{Deep nets for CV tasks {} {}} We consider a 4-layered MLP. The PyTorch code for 4-layer MLP is as follows: 

\texttt{ nn.Sequential(nn.Flatten(), \\
\tab        nn.Linear(input\_dim, 5000, bias=True),\\
\tab        nn.ReLU(),\\
\tab        nn.Linear(5000, 5000, bias=True),\\
\tab        nn.ReLU(),\\
\tab        nn.Linear(5000, 5000, bias=True),\\
\tab        nn.ReLU(),\\
% \tab        nn.Linear(5000, 5000, bias=True),\\
% \tab        nn.ReLU(),\\
\tab        nn.Linear(1024, num\_label, bias=True)\\
\tab        )}

For MNIST, we use $1000$ nodes instead of $5000$ nodes in the hidden layer. 
% 
We also experiment with convolutional nets. In particular, we use ResNet18 \citep{he2016deep}. Implementation adapted from:  \url{https://github.com/kuangliu/pytorch-cifar.git}. 

\textbf{Deep nets for NLP {} {}} We use a simple LSTM model with embeddings intialized with ELMo embeddings~\citep{Peters:2018}. Code adapted from: \url{https://github.com/kamujun/elmo_experiments/blob/master/elmo_experiment/notebooks/elmo_text_classification_on_imdb.ipynb} 

We also evaluate our bounds with a BERT model. In particular, we fine-tune an off-the-shelf uncased BERT model~\citep{devlin2018bert}. Code adapted from Hugging Face Transformers~\citep{wolf-etal-2020-transformers}: \url{https://huggingface.co/transformers/v3.1.0/custom_datasets.html}. 


\subsection{Additonal experiments}

\textbf{Results with SGD on underparameterized linear models {} {}} 

\begin{figure*}[h]
    \centering 
    % \vspace{-15pt}
    % \includegraphics[width=0.9\linewidth]{example-image-a}
    \includegraphics[width=0.3\linewidth]{figures/lowdim-Gaussian-SGD.pdf}
    % \includegraphics[width=0.9\linewidth]{figures/{CIFAR10_rn=0.1_lr=0.2_wd=0.005}.png}
    \vspace{-5pt}
    \caption{ 
    % Predicted lower bound 
    % on different
    We plot the accuracy and corresponding bound 
    (RHS in \eqref{eq:erm}) at $\delta = 0.1$
    for toy binary classification task. 
    Results aggregated over $3$ seeds. 
    % i.e., $1-\error$ where $\error$ is the term in the RHS of \eqref{eq:erm}
    Accuracy vs fraction of unlabeled data (w.r.t clean data) 
    in the toy setup with a linear model trained with SGD. Results parallel to \figref{fig:error_binary}(a) with SGD.  }
    \label{fig:error_binary_linear}
    \vspace{-5pt}
\end{figure*}

\textbf{Results with wide nets on binary MNIST {} {}}

\begin{figure*}[h]
    \centering 
    % \vspace{-15pt}
    % \includegraphics[width=0.9\linewidth]{example-image-a}
    \subfigure[GD with MSE loss]{\includegraphics[width=0.3\linewidth]{figures/MNIST-GD_MSE.pdf}} \hfil
    \subfigure[SGD with CE loss]{\includegraphics[width=0.3\linewidth]{figures/MNIST-SGD_CE.pdf}}
    \subfigure[SGD with MSE loss]{\includegraphics[width=0.3\linewidth]{figures/MNIST-SGD_MSE-first-layer.pdf}}
    % \includegraphics[width=0.9\linewidth]{figures/{CIFAR10_rn=0.1_lr=0.2_wd=0.005}.png}
    \vspace{-5pt}
    \caption{ 
    % Predicted lower bound 
    % on different
    We plot the accuracy and corresponding bound 
    (RHS in \eqref{eq:erm}) at $\delta = 0.1$ 
    for binary MNIST classification. 
    Results aggregated over $3$ seeds. 
    % i.e., $1-\error$ where $\error$ is the term in the RHS of \eqref{eq:erm}
    Accuracy vs fraction of unlabeled data 
    for a 2-layer wide network on binary MNIST with both the layers training in (a,b) and only first layer training in (c). 
    Results parallel to \figref{fig:error_binary}(b) .  }
    \label{fig:error_binary_MNIST}
    \vspace{-5pt}
\end{figure*}

% \begin{figure*}[h]
%     \centering 
%     % \vspace{-15pt}
%     % \includegraphics[width=0.9\linewidth]{example-image-a}
%     \subfigure[GD with MSE loss]{\includegraphics[width=0.3\linewidth]{figures/MNIST.pdf}} \hfil
    
%     \subfigure[SGD with CE loss]{\includegraphics[width=0.3\linewidth]{figures/MNIST.pdf}}
%     % \includegraphics[width=0.9\linewidth]{figures/{CIFAR10_rn=0.1_lr=0.2_wd=0.005}.png}
%     \vspace{-5pt}
%     \caption{ 
%     % Predicted lower bound 
%     % on different
%     We plot the accuracy and corresponding bound 
%     (RHS in \eqref{eq:erm}) at $\delta = 0.1$
%     for binary MNIST classification. 
%     Results aggregated over $3$ seeds. 
%     % i.e., $1-\error$ where $\error$ is the term in the RHS of \eqref{eq:erm}
%     Accuracy vs fraction of unlabeled data 
%     for a 2-layer wide network on binary MNIST with just the first layer training. 
%     Results parallel to \figref{fig:error_binary}(b) with only the first layer training.  }
%     \label{fig:error_binary_MNIST}
%     \vspace{-5pt}
% \end{figure*}

\textbf{Results on CIFAR 10 and MNIST {} {}} 
% 
We plot epoch wise error curve for results in \tabref{table:multiclass}(\figref{fig:error_epoch_CIFAR10} and \figref{fig:error_epoch_MNIST}). We observe the same trend as in \figref{fig:error_CIFAR10}. Additionally, we plot an \emph{oracle bound} obtained by tracking the error on mislabeled data which nevertheless were predicted as true label. To obtain an exact emprical value of the oracle bound, we need underlying true labels for the randomly labeled data. 
% Note that our bound in \thmref{thm:multiclass_ERM}, lower bounds the accuracy as predicted by the oracle bound. 
While with just access to extra unlabeled data we cannot calculate oracle bound, we note that the oracle bound is very tight and never violated in practice underscoring an importamt aspect of generalization in multiclass problems. This highlight that even a stronger conjecture may hold in multiclass classification, i.e., error on mislabeled data (where nevertheless true label was predicted) lower bounds the population error on the distribution of mislabeled data and hence, the error on (a specific) mislabeled portion predicts the population accuracy on clean data. 
% 
On the other hand, the dominating term of in \thmref{thm:multiclass_ERM} is loose when compared with the oracle bound. The main reason, we believe is the pessimistic upper bound in \eqref{eq:lemma1_final_multi_prev} in the proof of \lemref{lem:fit_mislabeled_multi}. We leave an investigation on this gap for future. 
% of fit 

% However, oracle bound highlights two . One,  



\begin{figure}[h]
    \centering 
    % \vspace{-15pt}
    % \includegraphics[width=0.9\linewidth]{example-image-a}
    \subfigure[MLP]{\includegraphics[width=0.3\linewidth]{figures/CIFAR10-FNN.pdf}} \hfil
    \subfigure[ResNet]{\includegraphics[width=0.3\linewidth]{figures/CIFAR10-Resnet.pdf}}
    % \includegraphics[width=0.9\linewidth]{figures/{CIFAR10_rn=0.1_lr=0.2_wd=0.005}.png}
    % \vspace{-10pt}
    \caption{ Per epoch curves for CIFAR10 corresponding results in \tabref{table:multiclass}. As before, we just plot the dominating term in the RHS of \eqref{eq:multiclass_ERM} as predicted bound. Additionally, we also plot the predicted lower bound by the error on mislabeled data which nevertheless were predicted as true label. We refer to this as ``Oracle bound''. See text for more details. 
    % 
    % except for the stopping point. 
    % The bound predicted by RATT (RHS in \eqref{eq:multiclass_ERM}) is vacuous. 
    }\label{fig:error_epoch_CIFAR10}
    % \vspace{-15pt}
\end{figure}


\begin{figure}[h]
    \centering 
    % \vspace{-15pt}
    % \includegraphics[width=0.9\linewidth]{example-image-a}
    \subfigure[MLP]{\includegraphics[width=0.3\linewidth]{figures/MNIST-FNN.pdf}} \hfil
    \subfigure[ResNet]{\includegraphics[width=0.3\linewidth]{figures/MNIST-Resnet.pdf}}
    % \includegraphics[width=0.9\linewidth]{figures/{CIFAR10_rn=0.1_lr=0.2_wd=0.005}.png}
    % \vspace{-10pt}
    \caption{ Per epoch curves for MNIST corresponding results in \tabref{table:multiclass}. As before, we just plot the dominating term in the RHS of \eqref{eq:multiclass_ERM} as predicted bound. Additionally, we also plot the predicted lower bound by the error on mislabeled data which nevertheless were predicted as true label. We refer to this as ``Oracle bound''. See text for more details. 
    % 
    % except for the stopping point. 
    % The bound predicted by RATT (RHS in \eqref{eq:multiclass_ERM}) is vacuous. 
    }\label{fig:error_epoch_MNIST}
    % \vspace{-15pt}
\end{figure}

\textbf{Results on CIFAR 100 {} {}} 
% 
On CIFAR100, our bound in \eqref{eq:multiclass_ERM} yields vacous bounds. However, the oracle bound as explained above yields tight guarantees in the initial phase of the learning (i.e., when learning rate is less than $0.1$) (\figref{fig:error_CIFAR100}).  

\begin{figure}[h]
    \centering 
    % \vspace{-15pt}
    % \includegraphics[width=0.9\linewidth]{example-image-a}
    \includegraphics[width=0.3\linewidth]{figures/CIFAR100-Resnet.pdf}
    % \includegraphics[width=0.9\linewidth]{figures/{CIFAR10_rn=0.1_lr=0.2_wd=0.005}.png}
    % \vspace{-10pt}
    \caption{ Predicted lower bound by the error on mislabeled data which nevertheless were predicted as true label with ResNet18 on CIFAR100. We refer to this as ``Oracle bound''. See text for more details. 
    % 
    % except for the stopping point. 
    The bound predicted by RATT (RHS in \eqref{eq:multiclass_ERM}) is vacuous. 
    }\label{fig:error_CIFAR100}
    % \vspace{-15pt}
\end{figure}


% \paragraph{Experiments on CIFAR100} 


% \subsection{Model Selection using RATT}


\subsection{Hyperparameter Details}


\textbf{\figref{fig:error_CIFAR10} {} {}} We use clean training dataset of size $40,000$. We fix the amount of unlabeled data at $20\%$ of the clean size, i.e. we include additional $8,000$ points with randomly assigned labels. We use test set of $10,000$ points. For both MLP and ResNet, we use SGD with an initial learning rate of $0.1$ and momentum $0.9$. We fix the weight decay parameter at $5\times 10^{-4}$. After $100$ epochs, we decay the learning rate to $0.01$. We use SGD batch size of $100$. 

\textbf{\figref{fig:error_binary} (a) {} {}} We obtain a toy dataset according to the process described in \secref{sec:app_dataset}. We fix $d=100$ and create a dataset of $50,000$ points with balanced classes. Moreover, we sample additional covariates with the same procedure to create randomly labeled dataset. For both SGD and GD training, we use a fixed learning rate $0.1$.    

\textbf{\figref{fig:error_binary} (b) {} {}} Similar to binary CIFAR, we use clean training dataset of size $40,000$ and fix the amount of unlabeled data at $20\%$ of the clean dataset size. To train wide nets, we use a fixed learning of $0.001$ with GD and SGD. We decide the weight decay parameter and the early stopping point that maximizes our generalization bound (i.e. without peeking at unseen data ).  We use SGD batch size of $100$. 

\textbf{\figref{fig:error_binary} (c) {} {}} With IMDb dataset, we use a clean dataset of size $20,000$ and as before, fix the amount of unlabeled data at $20\%$ of the clean data. To train ELMo model, we use Adam optimizer with a fixed learning rate $0.01$ and weight decay $10^{-6}$ to minimize cross entropy loss. We train with batch size $32$ for 3 epochs. To fine-tune BERT model, we use Adam optimizer with learning rate $5\times 10^{-5}$ to minimize cross entropy loss. We train with a batch size of $16$ for 1 epoch.    

\textbf{\tabref{table:multiclass} {} {}} For multiclass datasets, we train both MLP and ResNet with the same hyperparameters as described before. We sample a clean training dataset of size $40,000$ and fix the amount of unlabeled data at $20\%$ of the clean size. We use SGD with an initial learning rate of $0.1$ and momentum $0.9$. We fix the weight decay parameter at $5\times 10^{-4}$. After $30$ epochs for ResNet and after $50$ epochs for MLP, we decay the learning rate to $0.01$.  We use SGD with batch size $100$. 
For \figref{fig:error_CIFAR100}, we use the same hyperparameters as 
CIFAR10 training, except we now decay learning rate after $100$ epochs. 


In all experiments, to identify the best possible accuracy on just the clean data, we use the exact same set of hyperparamters except the stopping point. We choose a stopping point that maximizes test performance. 

\subsection{Summary of experiments }

\begin{center}
    \begin{table}[H] 
        \centering
        \begin{tabular}{|c|c|c|c|} 
        \hline
        Classification type & Model category & Model & Dataset  \\ [0.5ex] 
        \hline
        \hline
        \multirow{10}{*}{Binary} & Low dimensional & Linear model & Toy Gaussain dataset  \\
                        \cline{2-4}
                         & Overparameterized 
                        %  & Linear model & Toy Gaussain dataset \\
                        %  \cline{3-4}
                        %  & & 2-layer wide net& Toy Gaussain dataset \\
                        %  \cline{3-4}
                         & \multirow{2}{*}{2-layer wide net} & \multirow{2}{*}{Binary MNIST} \\
                         & linear nets & &  
                         \\
                         \cline{2-4}                 
                         & \multirow{6}{*}{Deep nets} & \multirow{2}{*}{MLP} & Binary MNIST \\
                         \cline{4-4}
                         & &  & Binary CIFAR \\
                         \cline{3-4}
                         &  & \multirow{2}{*}{ResNet} & Binary MNIST \\
                         \cline{4-4}
                         & &  & Binary CIFAR \\
                         \cline{3-4}
                         &  & ELMo-LSTM model & IMDb Sentiment Analysis \\
                         \cline{3-4}
                         & & BERT pre-trained model & IMDb Sentiment Analysis \\
        \hline
        \multirow{5}{*}{Multiclass} & \multirow{5}{*}{Deep nets} & \multirow{2}{*}{MLP} & MNIST \\
                        \cline{4-4} 
                        & & & CIFAR10 \\                   
                        \cline{3-4}
                         &   & \multirow{3}{*}{ResNet} & MNIST \\
                         \cline{4-4}
                         &   & & CIFAR10 \\
                         \cline{4-4}
                         &   & & CIFAR100 \\
        \hline
        \end{tabular}
        % \caption{Summary of experiments performed} \label{table:experiments}
    \end{table}    
    % \footnotetext[6]{We use both MSE loss and cross-entropy loss.}
    % \footnotetext[6]{We try 2 variants: one with a fixed first layer and the other with both layers trainable.}
\end{center}

\newpage
\section{Proof of \lemref{lem:stability_error}} \label{app:proof_lem_error}

\begin{proof}[Proof of \lemref{lem:stability_error}]
    Recall, we have a training set $S \cup \wt S_C$. We defined leave-one-out error on mislabeled points as $$\error_{\text{LOO}(\wt S_M) } = \frac{\sum_{(x_i, y_i) \in \wt S_M} \error( f_{(i)}( x_i), y_i)}{ \abs{\wt S_M }} \,, $$
    where $f_{(i)} \defeq f(\calA, (S \cup \wt S)_{(i)})$. Define $S^\prime \defeq S \cup \wt S$. Assume $(x,y)$ and $(x^\prime,y^\prime)$ as i.i.d. samples from ${\calDm}$. 
    Using Lemma 25 in \citet{bousquet2002stability}, we have
    \begin{align*}
        \Expo{ \left( \error_{\calDm}(\wh f) -\error_{\text{LOO}(\wt S_M) } \right)^2 } \le & \Expt{ S^\prime, (x,y), (x^\prime,y^\prime) }{ \error(\wh f(x), y ) \error(\wh f(x^\prime), y^\prime )} - 2 \Expt{ S^\prime, (x,y) }{ \error(\wh f(x), y ) \error(f_{(i)}(x_i), y_i )} \\
        & + \frac{m_1-1}{m_1}\Expt{ S^\prime }{  \error(f_{(i)}(x_i), y_i )  \error(f_{(j)}(x_j), y_j )} + \frac{1}{m_1} \Expt{ S^\prime }{  \error(f_{(i)}(x_i), y_i ) } \,. \numberthis \label{eq:main_reln}
    \end{align*}
    We can rewrite the equation above as : 
    \begin{align*}
        \Expo{ \left( \error_{\calDm}(\wh f) -\error_{\text{LOO}(\wt S_M) } \right)^2 } \le &  \, \underbrace{\Expt{ S^\prime, (x,y), (x^\prime,y^\prime) }{ \error(\wh f(x), y ) \error(\wh f(x^\prime), y^\prime ) - \error(\wh f(x), y ) \error(f_{(i)}(x_i), y_i )}}_{\RN{1}} \\
        & + \underbrace{\Expt{ S^\prime }{  \error(f_{(i)}(x_i), y_i )  \error(f_{(j)}(x_j), y_j ) -  \error(\wh f(x), y ) \error(f_{(i)}(x_i), y_i )}}_{\RN{2}} \\ &+ \underbrace{\frac{1}{m_1} \Expt{ S^\prime }{  \error(f_{(i)}(x_i), y_i ) - \error(f_{(i)}(x_i), y_i )  \error(f_{(j)}(x_j), y_j ) }}_{\RN{3}} \,. \numberthis \label{eq:main_reln2}
    \end{align*}
    
    We will now bound term $\RN{3}$.  Using Cauchy-Schwarz's inequality, we have
    
    \begin{align}
        \Expt{ S^\prime }{  \error(f_{(i)}(x_i), y_i ) - \error(f_{(i)}(x_i), y_i )  \error(f_{(j)}(x_j), y_j ) }^2 &\le  \Expt{ S^\prime }{  \error(f_{(i)}(x_i), y_i ) }^2 \Expt{S^\prime}{1 -   \error(f_{(j)}(x_j), y_j ) }^2 \\
        &\le \frac{1}{4} \,.\label{eq:term1_lem12}
    \end{align}
    
    Note that since $(x_i,y_i)$, $(x_j ,y_j )$, $(x,y)$, and $(x^\prime, y^\prime)$ are all from same distribution $\calDm$, we directly incorporate the bounds on term $\RN{1}$ and $\RN{2}$ from the proof of Lemma 9 in \citet{bousquet2002stability}. Combining that with \eqref{eq:term1_lem12} and our definition of hypothesis stability in \codref{cond:hypothesis_stability}, we have the required claim. 
    
    
    % We now re-write term $\RN{1}$ as
    % \begin{align*}
    %         &\Expt{S^\prime, (x,y), (x^\prime,y^\prime) }{ \error(\wh f(x), y ) \error(\wh f(x^\prime), y^\prime ) - \error(\wh f(x), y ) \error(f_{(i)}(x_i), y_i )} \\ & \qquad = \Expt{ S^\prime, (x,y), (x^\prime,y^\prime) }{ \error(\wh f(x), y ) \error(\wh f  (x^\prime), y^\prime ) - \error(\wh f ^\prime(x), y ) \error(f_{(i)}(x^\prime), y^\prime )} \tag{Exchanging $(x_i, y_i)$ with $(x^\prime, y^\prime)$ in the second term} \\
    %         & \qquad = \Expt{ S^\prime, (x,y), (x^\prime,y^\prime) }{  \left(\error(\wh f(x), y )-  \error(f_{(i)}(x), y ) \right) \error(\wh f  (x^\prime), y^\prime )  } \\
    %         & \qquad  + \Expt{ S^\prime, (x,y), (x^\prime,y^\prime) }{  \left(\error(f_{(i)}(x), y ) -\error(\wh f ^\prime(x), y ) \right) \error(\wh f  (x^\prime), y^\prime )}  \\
    %         & \qquad +\Expt{ S^\prime, (x,y), (x^\prime,y^\prime) }{  \left( \error(\wh f  (x^\prime), y^\prime ) -  \error(f_{(i)}(x^\prime), y^\prime ) \right) \error(\wh f ^\prime(x), y ) }  \,, \numberthis \label{eq:term1_final}
    % \end{align*}
    % where $\wh f^\prime$ is the classifier obtained by training on $ S^\prime_{(i)} \cup \{ (x^\prime, y^\prime) \} $. Similarly we can re-write term $\RN{2}$ as 
    % \begin{align*}
    %     & \Expt{ S^\prime }{  \error(f_{(i)}(x_i), y_i )  \error(f_{(j)}(x_j), y_j ) -  \error(\wh f(x), y ) \error(f_{(i)}(x_i), y_i )} \\
    %     &\quad  = \Expt{ S^\prime, (x,y), (x^\prime,y^\prime)}{  \error(f^{\prime\prime}_{(i)}(x), y )  \error(f_{(j)}^{\prime}(x^\prime), y^\prime ) -  \error(\wh f(x), y ) \error(f_{(i)}(x_i), y_i )} \tag{Exchanging $(x_i, y_i)$ with $(x, y)$ and $(x_j, y_j)$ with $(x^\prime, y^\prime)$ in the first term}\\
    %     &\quad = \Expt{ S^\prime, (x,y), (x^\prime,y^\prime)}{  \error(f^{\prime\prime}_{(j)}(x), y )  \error(f_{(i)}^{\prime}(x^\prime), y^\prime ) -  \error(\wh f^\prime (x), y ) \error(f^\prime_{(j)}(x^\prime), y^\prime )} \tag{Exchanging $(x_i, y_i)$ and $(x_j, y_j)$ and then replacing $(x_j, y_j)$ with $(x^\prime, y^\prime)$ in the second term} \\
    %     & \quad = \Expt{ S^\prime, (x,y), (x^\prime,y^\prime) }{  \left( \error(f_{(i)}^{\prime}(x^\prime), y^\prime )   -  \error(\wh f^{\prime\prime}  (x^\prime), y^\prime ) \right)  \error(f^{\prime\prime}_{(j)}(x), y )   } \\
    %     & \quad  + \Expt{ S^\prime, (x,y), (x^\prime,y^\prime) }{  \left( \error(f^{\prime\prime}_{(j)}(x), y )  -\error(\wh f ^\prime(x), y ) \right) \error(\wh f^{\prime\prime}  (x^\prime), y^\prime )  }  \\
    %     & \quad+ \Expt{ S^\prime, (x,y), (x^\prime,y^\prime) }{  \left( \error(\wh f^{\prime\prime}  (x^\prime), y^\prime )  -  \error(f^\prime_{(j)}(x^\prime), y^\prime ) \right)  \error(\wh f^\prime (x), y ) }   \\
    %     & \quad = \Expt{ S^\prime, (x,y), (x^\prime,y^\prime) }{  \left( \error(f_{(i)}^{\prime}(x^\prime), y^\prime )   -  \error(\wh f (x^\prime), y^\prime ) \right)  \error(f_{(i)}(x_j), y_j )   } \\
    %     & \quad  + \Expt{ S^\prime, (x,y), (x^\prime,y^\prime) }{  \left( \error(f^{\prime\prime}_{(j)}(x), y )  -\error(\wh f (x), y ) \right) \error(\wh f^{\prime\prime}  (x_j), y_j )  }  \\
    %     & \quad+ \Expt{ S^\prime, (x,y), (x^\prime,y^\prime) }{  \left( \error(\wh f^{\prime\prime}  (x^\prime), y^\prime )  -  \error(f^\prime_{(j)}(x^\prime), y^\prime ) \right)  \error(\wh f^\prime (x^\prime), y^\prime ) }  \,, \numberthis \label{eq:term2_final}
    % \end{align*}
    % where $f^{\prime\prime}_{(j)}$ is trained on $S^\prime_{(j,i)} \cup {(x,y)}$, $f^{\prime}_{(i)}$ is trained on $S^\prime_{(j,i)} \cup {(x^\prime,y^\prime)}$, and $\wh f^{\prime\prime} $ is trained on $S^\prime_{(j)} \cup {(x,y)}$. Note in the last line we replaced $(x,y)$ by $(x_j, y_j)$ in the first term, replaced $(x^\prime,y^\prime)$ by $(x_j, y_j)$ in the second term and exchanged $(x_i,y_i)$ with $(x_j,y_j)$ and also $(x,y)$ and $(x^\prime, y^\prime)$
    
    
\end{proof}


% 
% 16th Century Version Control 
% 

% \onecolumn

% \section*{Supplementary Material}
% We will be using the following standard results
% on exponential concentration of random variables 
% all throughout the discussion:

% \begin{lemma}[Hoeffding's inequality for independent RVs~\citep{hoeffding1994probability}] Let $Z_1, Z_2, \ldots, Z_n$ be independent bounded random variables with $Z_i \in [a,b]$ for all $i$, then 
%     \begin{align*}
%         \prob\left( \frac{1}{n} \sum_{i=1}^n (Z_i - \Expo{Z_i}) \ge t \right) \le \exp{\left( -\frac{2nt^2}{(b-a)^2} \right) }
%     \end{align*} 
%     and 
%     \begin{align*}
%         \prob\left( \frac{1}{n} \sum_{i=1}^n (Z_i - \Expo{Z_i}) \le -t \right) \le \exp{\left( -\frac{2nt^2}{(b-a)^2} \right) }
%     \end{align*} 
%     for all $t \ge 0$. 
% \end{lemma}

% \begin{lemma}[Hoeffding's inequality for sampling with replacement~\citep{hoeffding1994probability}] \label{lem:hoeffding_sampling} Let $\calZ = (Z_1, Z_2, \ldots, Z_N)$ be a finite population of $N$ points with $Z_i \in [a.b]$ for all $i$. Let $X_1, X_2, \ldots X_n$ be a random sample drawn without replacement from $\calZ$. Then for all $t \ge 0$, we have 
%     \begin{align*}
%         \prob\left( \frac{1}{n} \sum_{i=1}^n (X_i - \mu ) \ge t \right) \le \exp{\left( -\frac{2nt^2}{(b-a)^2} \right) }
%     \end{align*} 
%     and 
%     \begin{align*}
%         \prob\left( \frac{1}{n} \sum_{i=1}^n (X_i - \mu ) \le -t \right) \le \exp{\left( -\frac{2nt^2}{(b-a)^2} \right) } \,,
%     \end{align*} 
%     where $\mu = \frac{1}{N} \sum_{i=1}^{N} Z_i$. 
% \end{lemma}

% We now discuss one condition that generalizes the exponential concentration to dependent random variables.
% \begin{condition}[Bounded difference inequality] \label{cond:BDC} Let $\calZ$ be some set and $\phi: \calZ^n \to \Real$. We say that $\phi$ satisfies the bounded difference assumption if 
% there exists $c_1, c_2, \ldots c_n \ge 0$ s.t. for all $i$, we have 
% \begin{align*}
%     \sup_{Z_1,Z_2, \ldots,Z_n, Z_i^\prime in \calZ^{n+1} } \abs{\phi (Z_1, \ldots, Z_i, \ldots, Z_n ) - \phi (Z_1, \ldots, Z_i^\prime, \ldots, Z_n ) } \le c_i \,.
% \end{align*} 
% \end{condition}

% \begin{lemma}[McDiarmid’s inequality~\citep{mcdiarmid1989}] \label{lem:McDiarmid} Let $Z_1, Z_2, \ldots, Z_n$ be independent random variables on set $\calZ$ and $\phi : \calZ^n \to \Real$ satisfy bounded difference assumption (\codref{cond:BDC}). Then for all $t>0$, we have 
%     \begin{align*}
%         \prob\left( \phi(Z_1, Z_2, \ldots, Z_n) - \Expo{\phi(Z_1, Z_2, \ldots, Z_n)} \ge t \right) \le \exp{\left( -\frac{2t^2}{\sum_{i=1}^n c_i^2} \right) } 
%     \end{align*} 
%     and 
%     \begin{align*}
%         \prob\left( \phi(Z_1, Z_2, \ldots, Z_n) - \Expo{\phi(Z_1, Z_2, \ldots, Z_n)} \le -t \right) \le \exp{\left( -\frac{2t^2}{\sum_{i=1}^n c_i^2} \right) } \,
%     \end{align*} 
% \end{lemma}


% \section{Proofs from \secref{sec:ERM_training}}\label{app:proof_erm}

% \textbf{Additional notation {} {}} Let $m_1$ be the number of mislabeled points ($\wt S_M$) and $m_2$ be the number of correctly labeled points ($\wt S_C$). Note $m_1 + m_2 = m$. 


% \subsection{Proof of \thmref{thm:error_ERM}}


% \begin{proof}[Proof of \lemref{lem:fit_mislabeled}] 
%     The main idea of our proof is to regard 
%     the clean portion of the data 
%     ($S \cup \wt S_C$) as fixed.   
%     Then, there exists a classifier $f^*$ 
%     that is optimal over draws 
%     of the mislabeled data $\wt S_M$. 
% % 
%     % 
%     Formally, 
%     \begin{align}
%     f^* \defeq \argmin_{f \in \calF} \error_{\widecheck {\calD}} (f) \,, \label{eq:modified_ERM}
%     \end{align}
%     where $$\widecheck \calD = \frac{n}{m+n} \calS + \frac{m_1}{m+n} \wt \calS_C  + \frac{m_2}{m+n}\calDm \,.$$ That is, $\widecheck \calD$ a combination of 
%     the \emph{empirical distribution} 
%     over correctly labeled data $S \cup \wt S_C$
%     % in $S\cup \wt S$ 
%     and the (population) distribution 
%     over mislabeled data $\calDm$.
%     Recall that 
%     \begin{align}
%     \wh f \defeq \argmin_{f \in \calF} \error_{\calS \cup \wt S} (f) \,. \label{eq:orig_ERM}
%     \end{align}
%     % 
%     % 
%     Since, $\widehat f$ minimizes 0-1 error 
%     on $S \cup \wt S$, using ERM optimality on \eqref{eq:orig_ERM},  
%     we have 
%     \begin{align}
%         \error_{\calS \cup \wt \calS}(\widehat f) \le \error_{
%             \calS \cup \wt \calS}(f^*) \,.    \label{eq:step1}
%     \end{align}
%     Moreover, since $f^*$ is independent of $\wt S_M$, using Hoeffding's bound,
%     % \footnote{For a fully rigorous argument,
%     % refer to the complete proof in App.~\ref{app:proof_erm}.} 
%     we have with probability at least $1-\delta$ that
%     \begin{align}
%       \error_{\wt \calS_M}(f^*) \le \error_{ \calDm}(f^*) +  \sqrt{\frac{\log(1/\delta)}{2 m_1}} \,. \label{eq:step2} 
%     \end{align}
%     %$ 
%     %for some constant $c_1\le 1/2$. 
%     Finally, since $f^*$ is the optimal classifier on $\widecheck \calD$, 
%     we have 
%     \begin{align}
%         \error_{\widecheck \calD}(f^*) \le \error_{\widecheck \calD}(\widehat f) \label{eq:step3}
%     \end{align}
%      Now to relate \eqref{eq:step1} and \eqref{eq:step3}, we can re-write the \eqref{eq:step2} as follows: 
%     \begin{align}
%         \error_{\calS \cup \wt\calS}(f^*) \le \error_{ \widecheck \calD}(f^*) +  \frac{m_1}{m+n}\sqrt{\frac{\log(1/\delta)}{2 m_1}} \,. \label{eq:step4} 
%     \end{align}
%     Now we combine equations \eqref{eq:step1}, \eqref{eq:step4}, and \eqref{eq:step3}, to get 
%     \begin{align}
%         \error_{\calS \cup \wt \calS}(\wh f) \le \error_{\widecheck \calD}(\wh f) +  \frac{m_1}{m+n}\sqrt{\frac{\log(1/\delta)}{2 m_1}} \,, 
%     \end{align}
%     which implies 
%     \begin{align}
%         \error_{ \wt \calS_M}(\wh f) \le \error_{\calDm}(\wh f) + \sqrt{\frac{\log(1/\delta)}{2 m_1}} \,. \label{eq:lemma1_final}
%     \end{align}
%     Since $\wt S$ is obtained by randomly labeling an unlabeled dataset, we assume $2m_1 \approx m$ \footnote{Formally, with probability at least $1-\delta$, we have  $(m - 2m_1)\le \sqrt{m\log(1/\delta)/2}$ }. Moreover, using $\error_{\calDm} = 1 - \error_{\calD}$ we obtain the desired result.   
%     % Combining the above steps and using the fact 
%     % that $\error_\calD = 1- \error_{\calDm} $, 
%     % we obtain the desired result.
% \end{proof}

% \begin{proof}[Proof of \lemref{lem:mislabeled_error}]
%     Recall $\error_{\wt S} (f) = \frac{m_1}{m} \error_{\wt S_M}(f) + \frac{m_2}{m} \error_{\wt S_C}(f)$. Hence, we have 
%     \begin{align}
%         2\error_{\wt S}(f) - \error_{\wt S_M}(f) - \error_{\wt S_C}(f) &= \left(\frac{2m_1}{m} \error_{\wt S_M}(f) - \error_{\wt S_M}(f)\right) + \left(\frac{2m_2}{m} \error_{\wt S_C}(f) - \error_{\wt S_C}(f)\right) \\ &= \left(\frac{2m_1}{m} - 1\right) \error_{\wt S_M}(f) + \left(\frac{2m_2}{m} - 1 \right)\error_{\wt S_C} (f) \,.
%     \end{align} 
%     Since the dataset is randomly labeled, with probability at least $1-\delta$, we have  $\left(\frac{2m_1}{m} - 1\right) \le \sqrt{\frac{\log(1/\delta)}{2m}}$. Similarly, we have with probability at least $1-\delta$, $\left(\frac{2m_2}{m} - 1\right) \le \sqrt{\frac{\log(1/\delta)}{2m}}$. Using union bound, we have with probability at least $1-\delta$
%     % \begin{align}
%     %     2\error_{\wt S} - \error_{\wt S_M}(f) - \error_{\wt S_C}(f) \le \sqrt{\frac{\log(2/\delta)}{2m}} \left(\error_{\wt S_M}(f) + \error_{\wt S_C}(f) \right) \le 2\sqrt{\frac{\log(2/\delta)}{2m}} \,. \label{eq:lemma2_final}
%     % \end{align}
%     \begin{align}
%         2\error_{\wt S} - \error_{\wt S_M}(f) - \error_{\wt S_C}(f) \le \sqrt{\frac{\log(2/\delta)}{2m}} \left(\error_{\wt S_M}(f) + \error_{\wt S_C}(f) \right) \,. \label{eq:lemma2_prefinal}
%     \end{align}
%     With re-arranging $\error_{\wt S_M}(f) + \error_{\wt S_C}(f)$ and using the inequality $ 1- a\le \frac{1}{1+a} $, we have  
%     \begin{align}
%         2\error_{\wt S} - \error_{\wt S_M}(f) - \error_{\wt S_C}(f) \le 2\error_{\wt \calS} \sqrt{\frac{\log(2/\delta)}{2m}}  \,. \label{eq:lemma2_final}
%     \end{align}

%     % We obtain the desired result by using 
% \end{proof}

% \begin{proof}[Proof of \lemref{lem:clear_error}]
% % Recall 0-1 error on each point  $(x,y) \in S \cup \wt S$ is given by $\I{ f(x)\ne y}$.
% In the set of correctly labeled points $S \cup \wt S_C$, we have $S$ as a random subset of $S \cup \wt S_C$. Hence, using Hoeffding's inequality for sampling without replacement (\lemref{lem:hoeffding_sampling}), we have with probability at least $1-\delta$
% \begin{align}
%     \error_{\wt \calS_c} (\wh f)- \error_{\calS \cup \wt \calS_C}( \wh f) \le  \sqrt{\frac{\log(1/\delta)}{2m_2}} \,.
% \end{align}
% Re-writing $\error_{\calS \cup \wt \calS_C}( \wh f)$ as $\frac{m_2}{m_2 + n} \error_{\wt \calS_C }(\wh f) + \frac{n}{m_2 + n} \error_{\calS }(\wh f)$, we have with probability at least $1-\delta$
% \begin{align}
%   \left(\frac{n}{n+m_2}\right) \left(\error_{\wt \calS_c} (\wh f)- \error_{\calS}( \wh f) \right) \le  \sqrt{\frac{\log(1/\delta)}{2m_2}} \,.
% \end{align}
% As before, assuming $2m_2 \approx m$, we have with probability at least $1-\delta$ 
% \begin{align}
%     \error_{\wt \calS_c} (\wh f)- \error_{\calS}( \wh f) \le \left(1+\frac{m_2}{n}\right)  \sqrt{\frac{\log(1/\delta)}{m}} \le 1.5 \sqrt{\frac{\log(1/\delta)}{m}} \,. \label{eq:lemma3_final}
% \end{align} 
% \end{proof}

% \begin{proof}[Proof of \thmref{thm:error_ERM}] 
%     Having established these core intermediate results, we can now combine above three lemmas to prove the main result. 
%     In particular, we bound the population error on clean data ($\error_\calD(\wh f)$) as follows:  
%     \begin{enumerate}[(i)]
%         \item First, use \eqref{eq:lemma1_final}, to obtain an upper bound on the population error on clean data, i.e., with probability at least $1-\delta/4$, we have
%         \begin{align}
%             \error_{ \calD} (\wh f) \le 1 - \error_{ \wt \calS_M}(\wh f) + \sqrt{\frac{\log(4/\delta)}{m}} \,. 
%         \end{align}
%         \item  Second, use \eqref{eq:lemma2_final}, to relate the error on the mislabeled fraction with error on clean portion of randomly labeled data and error on whole randomly labeled dataset, i.e., with probability at least $1-\delta/2$, we have 
%         \begin{align}
%             - \error_{\wt S_M}(f) \le \error_{\wt S_C}(f) - 2\error_{\wt S}  + \sqrt{\frac{\log(4/\delta)}{2m}}  \,. 
%         \end{align} 
%         \item Finally, use \eqref{eq:lemma3_final} to relate the error on the clean portion of randomly labeled data and error on clean training data, i.e., with probability $1-\delta/4$, we have 
%         \begin{align}
%             \error_{\wt \calS_C} (\wh f)\le - \error_{\calS}( \wh f) + \left(1 + \frac{m}{2n} \right) \sqrt{\frac{\log(4/\delta)}{m}} \,. 
%         \end{align} 
%     \end{enumerate}

%     Using union bound on the above three steps, we have with probability at least $1-\delta$: 
%     \begin{align}
%         \error_\calD (\wh f) \le \error_{\calS}(\wh f)   + 1 - 2\error_{\wt \calS}(\wh f)   + (1/\sqrt{2} + 2.5)  \sqrt{\frac{\log(4/\delta)}{m}} \,.
%     \end{align}
%     Note that $(1/\sqrt{2} + 2.5)$ is a loose constant. In experiments, we use the ratio $\frac{m}{n}$
%     %  the exact error $\error_{\wt \calS}(\wh f)$ 
%     to evaluate R.H.S.    
% \end{proof}

% \subsection{Proof of \propref{prop:rademacher}}

% \begin{proof}[Proof of \propref{prop:rademacher}]
%     For a classifier $ f: \calX \to \{-1, 1\}$, we have $1 - 2\,\indict{ f(x) \ne y} = y \cdot f(x)$. Hence, by definition of $\error$, we have 
%     \begin{align}
%         1 -2\error_{\wt \calS}(f) = \frac{1}{m}\sum_{i=1}^m y_i \cdot f(x_i) \le \sup_{f \in \calF} \, \frac{1}{m} \sum_{i=1}^m y_i \cdot f(x_i)  \,. \label{eq:error_rademacher}
%     \end{align}
%     Note that for fixed inputs $(x_1, x_2, \ldots, x_m)$ in $\wt S$, $(y_1, y_2, \ldots y_m)$ are random labels. Define $\phi_1 (y_1, y_2, \ldots, y_m) \defeq \sup_{f \in \calF} \, \frac{1}{m} \sum_{i=1}^m y_i \cdot f(x_i)$. We have the following bounded difference condition on $\phi_1$. For all i, 
%     \begin{align}
%         \sup_{y_1, \ldots y_m, y_i^\prime \in \{-1, 1\}^{m+1} } \abs{ \phi_1 (y_1,\ldots, y_i, \ldots, y_m) - \phi_1 (y_1,\ldots, y_i^\prime, \ldots, y_m)  } \le 1/m \,. \label{cond1_rademacher}
%     \end{align} 
    
%     Similarly define $\phi_2 (x_1, x_2, \ldots, x_m) \defeq \Expt{ y_i \sim_U \{-1, 1\}  }{ \sup_{f \in \calF} \, \frac{1}{m}  \sum_{i=1}^m y_i \cdot f(x_i)}$. We have the following bounded difference condition on $\phi_2$. For all i,
%     \begin{align}
%         \sup_{x_1, \ldots x_m, x_i^\prime \in \calX^{m+1} } \abs{ \phi_2 (x_1,\ldots, x_i, \ldots, x_m) - \phi_1 (x_1,\ldots, x_i^\prime, \ldots, x_m)  } \le 1/m \,. \label{cond2_rademacher}
%     \end{align}
%     Using McDiarmid’s inequality (\lemref{lem:McDiarmid}) twice with Condition \eqref{cond1_rademacher} and \eqref{cond2_rademacher}, with probability at least $1-\delta$, we have
%     \begin{align}
%         \sup_{f \in \calF} \, \frac{1}{m} \sum_{i=1}^m y_i \cdot f(x_i)  - \Expt{x,y}{\sup_{f \in \calF} \, \frac{1}{m} \sum_{i=1}^m y_i \cdot f(x_i) } \le \sqrt{\frac{2\log(2/\delta)}{m}} \label{eq:final_rademacher}
%     \end{align} 
%     Combining \eqref{eq:error_rademacher} and \eqref{eq:final_rademacher}, we obtain the desired result. 
% \end{proof}


% \subsection{Proof of \thmref{thm:error_regularized_ERM}}

% Proof of \thmref{thm:error_regularized_ERM} follows similar to the proof of \thmref{thm:error_ERM}. Note that the same results in \lemref{lem:fit_mislabeled}, \lemref{lem:mislabeled_error}, and \lemref{lem:clear_error} hold in the regularized ERM case. However, the arguments in the proof of \lemref{lem:fit_mislabeled} changes slightly. Hence, we state and prove a lemma parallel to \lemref{lem:fit_mislabeled} for completeness. 

% \begin{lemma} \label{lem:lemma1_reg}
%     Assume the same setup as \thmref{thm:error_regularized_ERM}. 
%     Then for any $\delta >0$, with probability at least  $1-\delta$ 
%     over the random draws of mislabeled data $\wt S_M$, we have 
%     \begin{align}
%         \error_\calD(\widehat f)  \le 1 -\error_{\wt \calS_M}(\widehat f) + \sqrt{\frac{\log(1/\delta)}{m}}\,. 
%     \end{align} 
% \end{lemma}
% \begin{proof}
%     The main idea of the proof remains the same, i.e. regard 
%     the clean portion of the data 
%     ($S \cup \wt S_C$) as fixed.   
%     Then, there exists a classifier $f^*$ 
%     that is optimal over draws 
%     of the mislabeled data $\wt S_M$. 

    
%     Formally, 
%     \begin{align}
%     f^* \defeq \argmin_{f \in \calF} \error_{\widecheck {\calD}} (f)  + \lambda R(f) \,, \label{eq:modified_ERM_reg}
%     \end{align}
%     where $$\widecheck \calD = \frac{n}{m+n} \calS + \frac{m_1}{m+n} \wt \calS_C  + \frac{m_2}{m+n}\calDm \,.$$ That is, $\widecheck \calD$ a combination of 
%     the \emph{empirical distribution} 
%     over correctly labeled data $S \cup \wt S_C$
%     % in $S\cup \wt S$ 
%     and the (population) distribution 
%     over mislabeled data $\calDm$.
%     Recall that 
%     \begin{align}
%     \wh f \defeq \argmin_{f \in \calF} \error_{\calS \cup \wt S} (f) + \lambda R(f) \,. \label{eq:orig_ERM_reg}
%     \end{align}
%     % 
%     % 
%     Since, $\widehat f$ minimizes 0-1 error 
%     on $S \cup \wt S$, using ERM optimality on \eqref{eq:orig_ERM},  
%     we have 
%     \begin{align}
%         \error_{\calS \cup \wt \calS}(\widehat f) + \lambda R(\wh f) \le \error_{
%             \calS \cup \wt \calS}(f^*) + \lambda R(f^*) \,.    \label{eq:step1_reg}
%     \end{align}
%     Moreover, since $f^*$ is independent of $\wt S_M$, using Hoeffding's bound,
%     % \footnote{For a fully rigorous argument,
%     % refer to the complete proof in App.~\ref{app:proof_erm}.} 
%     we have with probability at least $1-\delta$ that
%     \begin{align}
%       \error_{\wt \calS_M}(f^*) \le \error_{ \calDm}(f^*) +  \sqrt{\frac{\log(1/\delta)}{2 m_1}} \,. \label{eq:step2_reg} 
%     \end{align}
%     %$ 
%     %for some constant $c_1\le 1/2$. 
%     Finally, since $f^*$ is the optimal classifier on $\widecheck \calD$, 
%     we have 
%     \begin{align}
%         \error_{\widecheck \calD}(f^*) + \lambda R(f^*) \le \error_{\widecheck \calD}(\widehat f) + \lambda R(\wh f) \label{eq:step3_reg}
%     \end{align}
%      Now to relate \eqref{eq:step1_reg} and \eqref{eq:step3_reg}, we can re-write the \eqref{eq:step2_reg} as follows: 
%     \begin{align}
%         \error_{\calS \cup \wt\calS}(f^*) \le \error_{ \widecheck \calD}(f^*) +  \frac{m_1}{m+n}\sqrt{\frac{\log(1/\delta)}{2 m_1}} \,. \label{eq:step4_reg} 
%     \end{align}
%     After adding $\lambda R(f^*)$ on both sides in \eqref{eq:step4_reg}, we combine equations \eqref{eq:step1_reg}, \eqref{eq:step4_reg}, and \eqref{eq:step3_reg}, to get 
%     \begin{align}
%         \error_{\calS \cup \wt \calS}(\wh f) \le \error_{\widecheck \calD}(\wh f) +  \frac{m_1}{m+n}\sqrt{\frac{\log(1/\delta)}{2 m_1}} \,, 
%     \end{align}
%     which implies 
%     \begin{align}
%         \error_{ \wt \calS_M}(\wh f) \le \error_{\calDm}(\wh f) + \sqrt{\frac{\log(1/\delta)}{2 m_1}} \,. \label{eq:lemma_reg_final}
%     \end{align}
%     Similar as before, since $\wt S$ is obtained by randomly labeling an unlabeled dataset, we assume 
%     $2m_1 \approx m$. Moreover, using $\error_{\calDm} = 1 - \error_{\calD}$ we obtain the desired result. 
% \end{proof}
% % \begin{proof}[Proof of ]
    
% % \end{proof}

% \subsection{Proof of \thmref{thm:multiclass_ERM}}

% We first state and prove lemmas parallel to three lemmas used in the proof of balanced binary case. Then we combine the results in the three lemmas to obtain the result in \thmref{thm:multiclass_ERM}. 

% Before stating the result, we define mislabeled distribution $\calDm$ for any $\calD$. While $\calDm$ and $\calD$ share 
% the same marginal distribution over $\calX$, 
% the distribution over labels $y$ 
% given an input $x\sim \calD_\calX$ is changed.
% In particular, for any $x$, the pdf over $y$ is changed to:  
% $p_{\calDm} (\cdot \vert x) \defeq \frac{1 - p_{\calD}(\cdot \vert x)}{k - 1}$.

% \begin{lemma} \label{lem:fit_mislabeled_multi}
%     Assume the same setup as \thmref{thm:multiclass_ERM}. 
%     Then for any $\delta >0$, with probability at least  $1-\delta$ 
%     over the random draws of mislabeled data $\wt S_M$, we have 
%     \begin{align}
%         \error_\calD(\widehat f)  \le (k-1)\left(1 -\error_{\wt \calS_M}(\widehat f)\right) + (k-1)\sqrt{\frac{\log(1/\delta)}{m}}\,. \label{eq:lemma1_multi}
%     \end{align}   
% \end{lemma} 

% \begin{proof}
%     The main idea of the proof remains the same, i.e. regard 
%     the clean portion of the data 
%     ($S \cup \wt S_C$) as fixed. 
%     Then, there exists a classifier $f^*$ 
%     that is optimal over draws 
%     of the mislabeled data $\wt S_M$. 
    
%     However, we need to be careful while relating population error on mislabeled data with population accuracy on clean data.   
%     While for binary classification,  we could upper bound $\error_{\wt \calS_M}$ 
%     with $1-\error_\calD$  (in the proof of \lemref{lem:fit_mislabeled}), 
%     for multiclass classification, 
%     error on the mislabeled data 
%     and accuracy on the clean data 
%     in the population 
%     are not so directly related.  
%     To establish \eqref{eq:lemma1_multi},
%     we break the error on the 
%     (unknown) mislabeled data 
%     into two parts: one term corresponds 
%     to predicting the true label on mislabeled data, 
%     and the other corresponds to predicting 
%     neither the true label 
%     nor the assigned (mis-)label.  
%     Finally, we relate these errors to their
%     population counterparts to establish \eqref{eq:lemma1_multi}. 
    
%     Formally, 
%     \begin{align}
%     f^* \defeq \argmin_{f \in \calF} \error_{\widecheck {\calD}} (f)  + \lambda R(f) \,, \label{eq:modified_ERM_reg2}
%     \end{align}
%     where $$\widecheck \calD = \frac{n}{m+n} \calS + \frac{m_1}{m+n} \wt \calS_C  + \frac{m_2}{m+n}\calDm \,.$$ That is, $\widecheck \calD$ a combination of 
%     the \emph{empirical distribution} 
%     over correctly labeled data $S \cup \wt S_C$
%     % in $S\cup \wt S$ 
%     and the (population) distribution 
%     over mislabeled data $\calDm$.
%     Recall that 
%     \begin{align}
%     \wh f \defeq \argmin_{f \in \calF} \error_{\calS \cup \wt S} (f) + \lambda R(f) \,. \label{eq:orig_ERM_reg2}
%     \end{align}
%     % 
%     % 
%     Following the exact steps from the proof of \lemref{lem:lemma1_reg}, with probability at least $1-\delta$, we have  
%     \begin{align}
%         \error_{ \wt \calS_M}(\wh f) \le \error_{\calDm}(\wh f) + \sqrt{\frac{\log(1/\delta)}{2 m_1}} \,. \label{eq:lemma1_final_multi_prev}
%     \end{align}
%     Similar to before, since $\wt S$ is obtained by randomly labeling an unlabeled dataset, we assume 
%     $\frac{k}{k-1} m_1 \approx m$. 
    
%     Now we will relate $\error_\calDm (\wh f)$ with $\error_{\calD}(\wh f)$. Let $y^T$ denote the (unknown) true label for a mislabeled point $(x, y)$ (i.e., label before replacing it with a mislabel). 
%     \begin{align}    
%          \Expt{(x, y) \in \sim \calDm}{\indict{ \wh f(x) \ne y }}  &= \underbrace{\Expt{(x, y) \in \sim \calDm}{\indict{ \wh f(x) \ne y \land \wh f(x) \ne y^T}}}_{\RN{1}} + \underbrace{\Expt{(x, y) \in \sim \calDm}{\indict{ \wh f(x) \ne y \land \wh f(x) = y^T}}}_{\RN{2}} \,. \label{eq:excess_term}
%     \end{align}
%     Clearly, term 2 is one minus the accuracy on the clean unseen data, i.e. 
%     \begin{align}
%         \RN{2} = 1 - \Expt{{x,y} \sim \calD}{ \indict{ \wh f(x) \ne y}} = 1- \error_{\calD}(\wh f) \,. \label{eq:term1}    
%     \end{align}
%     Next, we  relate term 1 with the error on the unseen clean data. We show that term 1 is equal to the error on the unseen clean data scaled by $\frac{k-2}{k-1}$ where $k$ is the number of labels. Using the definition of mislabeled distribution $\calDm$,  we have 
%     \begin{align}
%         \RN{1} = \frac{1}{k-1} \left( \Expt{(x, y) \in \sim \calD}{ \sum_{i \in \calY \land i\ne y}  \indict{ \wh f(x) \ne i \land \wh f(x) \ne y}} \right) = \frac{k-2}{k-1} \error_{\calD}(\wh f) \,.\label{eq:term2}
%     \end{align}    

%     Combining the result in \eqref{eq:term1}, \eqref{eq:term2} and \eqref{eq:excess_term}, we have 
%     \begin{align}
%         \error_{\calDm}(\wh f) = 1- \frac{1}{k-1} \error_{\calD}(\wh f) \,.\label{eq:combine_terms}
%     \end{align}
%     Finally, combining the result in \eqref{eq:combine_terms} with equation \eqref{eq:lemma1_final_multi_prev}, we have with probability $1-\delta$, 
%     \begin{align}
%       \error_{\calD}(\wh f) \le  (k-1) \left( 1- \error_{ \wt \calS_M}(\wh f) \right)  + (k-1) \sqrt{\frac{k \log(1/\delta)}{ 2(k-1)m}} \,. \label{eq:lemma1_final_multi}
%     \end{align}
% \end{proof}

% \begin{lemma} \label{lem:mislabeled_error_multi}
%     Assume the same setup as \thmref{thm:multiclass_ERM}.  Then for any $\delta >0$, with probability at least $1-\delta$ over the random draws of $\wt S$, we have  
%     % \begin{align}
%         $$\abs{k\error_{\wt \calS}(\widehat f) - \error_{\wt \calS_C}(\widehat f) -  (k-1)\error_{\wt \calS_M}(\widehat f) } \le  2k\sqrt{\frac{\log(4/\delta)}{2m}}\,. $$ % \label{eq:lemma2}
%     % \end{align}   
%     %  for some constant $c_3 \le 1.0\,$.
% \end{lemma} 


% \begin{proof}
%     Recall $\error_{\wt S} (f) = \frac{m_1}{m} \error_{\wt S_M}(f) + \frac{m_2}{m} \error_{\wt S_C}(f)$. Hence, we have 
%     \begin{align}
%         k\error_{\wt S}(f) - (k-1)\error_{\wt S_M}(f) - \error_{\wt S_C}(f) &= (k-1)\left(\frac{k m_1}{(k-1) m} \error_{\wt S_M}(f) - \error_{\wt S_M}(f)\right) + \left(\frac{km_2}{m} \error_{\wt S_C}(f) - \error_{\wt S_C}(f)\right) \\ &= k \left[ \left(\frac{m_1}{m} - \frac{k-1}{k}\right) \error_{\wt S_M}(f) + \left(\frac{m_2}{m} - \frac{1}{k} \right) \error_{\wt S_C} (f) \right] \,.
%     \end{align} 
%     Since the dataset is randomly labeled, we have with probability at least $1-\delta$, $\left(\frac{m_1}{m} - \frac{k-1}{k}\right) \le \sqrt{\frac{\log(1/\delta)}{2m}}$. Similarly, we have with probability at least $1-\delta$, $\left(\frac{m_2}{m} - \frac{1}{k}\right) \le \sqrt{\frac{\log(1/\delta)}{2m}}$. Using union bound, we have with probability at least $1-\delta$
%     % \begin{align}
%     %     2\error_{\wt S} - \error_{\wt S_M}(f) - \error_{\wt S_C}(f) \le \sqrt{\frac{\log(2/\delta)}{2m}} \left(\error_{\wt S_M}(f) + \error_{\wt S_C}(f) \right) \le 2\sqrt{\frac{\log(2/\delta)}{2m}} \,. \label{eq:lemma2_final}
%     % \end{align}
%     \begin{align}
%         k\error_{\wt S}(f) - (k-1)\error_{\wt S_M}(f) - \error_{\wt S_C}(f)  \le k \sqrt{\frac{\log(2/\delta)}{2m}} \left(\error_{\wt S_M}(f) + \error_{\wt S_C}(f) \right) \,. \label{eq:lemma2_final_multi}
%     \end{align}

%     % We obtain the desired result by using 
% \end{proof}

% \begin{lemma} \label{lem:clear_error_multi}
%     Assume the same setup as \thmref{thm:multiclass_ERM}. 
%     Then for any $\delta >0$, with probability at least $1-\delta$ 
%     over the random draws of $\wt S_C$ and $S$, we have 
%     % \begin{align}
%         $$\abs{\error_{\wt \calS_C}(\widehat f) - \error_{\calS}(\widehat f) } \le 1.5 \sqrt{\frac{k\log(2/\delta)}{2m}}\,.$$ %\label{eq:lemma3}
%     % \end{align}   
%     % for some constant $c_2 \le 1.2\,$.
% \end{lemma} 
% \begin{proof}
%     % Recall 0-1 error on each point  $(x,y) \in S \cup \wt S$ is given by $\I{ f(x)\ne y}$.
%     In the set of correctly labeled points $S \cup \wt S_C$, we have $S$ as a random subset of $S \cup \wt S_C$. Hence, using Hoeffding's inequality for sampling without replacement (\lemref{lem:hoeffding_sampling}), we have with probability at least $1-\delta$
%     \begin{align}
%         \error_{\wt \calS_c} (\wh f)- \error_{\calS \cup \wt \calS_C}( \wh f) \le  \sqrt{\frac{\log(1/\delta)}{2m_2}} \,.
%     \end{align}
%     Re-writing $\error_{\calS \cup \wt \calS_C}( \wh f)$ as $\frac{m_2}{m_2 + n} \error_{\wt \calS_C }(\wh f) + \frac{n}{m_2 + n} \error_{\calS }(\wh f)$, we have with probability at least $1-\delta$
%     \begin{align}
%       \left(\frac{n}{n+m_2}\right) \left(\error_{\wt \calS_c} (\wh f)- \error_{\calS}( \wh f) \right) \le  \sqrt{\frac{\log(1/\delta)}{2m_2}} \,.
%     \end{align}
%     As before, assuming $km_2 \approx m$, we have with probability at least $1-\delta$ 
%     \begin{align}
%         \error_{\wt \calS_c} (\wh f)- \error_{\calS}( \wh f) \le \left(1+\frac{m_2}{n}\right)  \sqrt{\frac{k\log(1/\delta)}{2m}} \le \left( 1 + \frac{1}{k}\right) \sqrt{\frac{k\log(1/\delta)}{2m}} \,. \label{eq:lemma3_final_multi}
%     \end{align} 
% \end{proof}

% \begin{proof}[Proof of \thmref{thm:multiclass_ERM}] 
%     Having established these core intermediate results, we can now combine above three lemmas. 
%     In particular, we bound the population error on clean data ($\error_\calD(\wh f)$) as follows:  
%     \begin{enumerate}[(i)]
%         \item First, use \eqref{eq:lemma1_final_multi}, to obtain an upper bound on the population error on clean data, i.e., with probability at least $1-\delta/4$, we have
%         \begin{align}
%             \error_{ \calD} (\wh f) \le (k-1)\left(1 - \error_{ \wt \calS_M}(\wh f) \right) + (k-1) \sqrt{\frac{k\log(4/\delta)}{2(k-1)m}} \,. 
%         \end{align}
%         \item  Second, use \eqref{eq:lemma2_final_multi}, to relate the error on the mislabeled fraction with error on clean portion of randomly labeled data and error on whole randomly labeled dataset, i.e., with probability at least $1-\delta/2$, we have 
%         \begin{align}
%             - (k-1)\error_{\wt S_M}(f) \le \error_{\wt S_C}(f) - k\error_{\wt S}  + k\sqrt{\frac{\log(4/\delta)}{2m}}  \,. 
%         \end{align} 
%         \item Finally, use \eqref{eq:lemma3_final_multi} to relate the error on the clean portion of randomly labeled data and error on clean training data, i.e., with probability $1-\delta/4$, we have 
%         \begin{align}
%             \error_{\wt \calS_C} (\wh f)\le - \error_{\calS}( \wh f) + \left(1 + \frac{m}{kn} \right) \sqrt{\frac{k\log(4/\delta)}{2m}} \,. 
%         \end{align} 
%     \end{enumerate}

%     Using union bound on the above three steps, we have with probability at least $1-\delta$: 
%     \begin{align}
%         \error_\calD (\wh f) \le \error_{\calS}(\wh f) + (k-1) - k\error_{\wt \calS}(\wh f)   + (\sqrt{k(k-1)} + k + \sqrt{k} + \frac{m}{n\sqrt{k}})  \sqrt{\frac{\log(4/\delta)}{2m}} \,.
%     \end{align}
%     % Note that $\frac{m}{n\sqrt{k}}$ is much smaller than the other terms in addition. Hence, we ignore this in the final bound. 
%     % Note that $(1/\sqrt{2} + 2.5)$ is a loose constant. In experiments, we use the ratio $\frac{m}{n}$
%     %  the exact error $\error_{\wt \calS}(\wh f)$ 
%     % to evaluate R.H.S.    
% \end{proof}

% \newpage
% \section{Proofs from \secref{sec:linear_models}}\label{app:proof_gd}

% We suppose that the parameters of the linear function 
% are obtained via gradient descent on 
% the following $L_2$ regularized problem: 
% \begin{align}
%     % n in denominator is avoided deliberately
%     \calL_S(w; \lambda) \defeq \sum_{i=1}^n{(w^Tx_i - y_i)^2} + \lambda \norm{w}{2}^2 \,, \label{eq:l2_MSE_app}   
% \end{align}
% where $\lambda\ge0$ is a regularization parameter. 
% We assume access to a clean dataset 
% $S = \{(x_i, y_i)\}_{i=1}^n \sim \calD^n$ 
% and randomly labeled dataset 
% $\wt S = \{(x_i, y_i)\}_{i=n+1}^{n+m} \sim \wt \calD^m$. 
% Let $\bX = [x_1, x_2, \cdots, x_{m+n}]$ 
% and $\by = [y_1, y_2, \cdots, y_{m+n}]$. 
% Fix a positive learning rate $\eta$ such that 
% $\eta \le 1/\left(\norm{\bX^T\bX}{\text{op}} + \lambda^2\right)$ 
% and an initialization $w_0 = 0$. 
% % \todos{Assumption made for simplicty}. 
% Consider the following gradient descent iterates 
% to minimize objective \eqref{eq:l2_MSE_app} on $S \cup \wt S$:
% \begin{align}
% w_t = w_{t-1} - \eta \grad_w \calL_{S \cup \wt S} (w_{t-1}; \lambda) \quad \forall t=1,2,\ldots \label{eq:GD_iterates_app}
% \end{align} 
% Then we have $\{ w_t\}$ converge to the limiting solution 
% $\wh w = \left( \bX^T\bX+\lambda \boldsymbol{I}\right)^{-1}\bX^T\by$. Define $\widehat f (x) \defeq f(x ; \wh w) $.  

% \subsection{\textcolor{red}{Errata}}

% We wish to correct the following error in the body: \codref{cond:error_stability} is not enough to guarantee the result in \thmref{thm:linear}. We now present a slightly stronger condition called \emph{hypothesis stability} under which we obtain a result similar to \thmref{thm:linear}. 

% This error doesn't change the main arguments of the proof where we show that the empirical train error is less than or equal to the leave-one-out error. We need a stronger condition to relate leave-one-out error with the population error of the original classifier. Specifically, while \codref{cond:error_stability} relates the average population error of leave-one-out classifiers with the population error of the original classifier, we need the new condition to show the concentration of the empirical leave-one-out error and  average population error of leave-one-out classifiers. 
% % main takeaway 

% Note that the new condition, while being stronger than the previous one, still doesn't imply generalization~\cite{bousquet2002stability,elisseeff2003leave,abou2019exponential}. Overall, the main results in \secref{sec:ERM_training} and takeaways of the paper remain unaffected by the error.  

% We now present the new condition and a corrected statement of \thmref{thm:linear}. Recall, for a given training set $S \sim \calD^n $, 
% we use $S_{(i)}$ to denote the training set $S$ 
% with the $i^{\text{th}}$ point removed.

% \begin{condition}[Hypothesis Stability] 
%     \label{cond:hypothesis_stability}
%     We have $\beta$ hypothesis stability 
%     if our training algorithm $\calA$ satisfies the following: 
%     \begin{align*}
%     % ${\sum_{i=1}^n \frac{\error_{\calD}( f(\calA, S_{(i)}))}{n} - \error_\calD(f(\calA, S))} \le \beta\,$.
%     \forall i \in \{1,2,\ldots, n\}, \quad  \Expt{\calS, (x,y) \in \calD}{ \abs{\error\left( f(x) ,y  \right) - \error\left( f_{(i)}(x), y \right) }} \le \frac{\beta}{n} \,,
%     \end{align*}
%     where $f_{(i)} \defeq f(\calA, S_{(i)})$ and $ f \defeq f(\calA, S)$.
% \end{condition}

% \begin{theorem}[Correct statement of \thmref{thm:linear}] \label{thm:new_linear}
%     Assume that this gradient descent algorithm satisfies \codref{cond:hypothesis_stability}
%     with $\beta=\calO(1)$.  
%     Then for any $\delta >0$, with probability at least $1-\delta$ 
%     over the random draws of datasets $\wt S$ and $S$, we have:
%     \begin{align}
%         \error_\calD(\widehat f) \le \error_\calS(\widehat f) + 1 - 2 \error_{\wt\calS}(\widehat f) + \left(\frac{1}{\sqrt{2}} + 1.5 \right) \sqrt{\frac{\log(4/\delta)}{m}} + \sqrt{\frac{4}{\delta}\left(\frac{1}{m} +\frac{3\beta}{m+n} \right)}  \,. \label{eq:gd_error}
%     \end{align} 
%     % for some constant $c\le 3.2$.
% \end{theorem}

% \subsection{Proof of \thmref{thm:new_linear}}
% We use a standard result from linear algebra, namely Shermann-Morrison formula~\citep{sherman1950adjustment} for matrix inversion:  

% \begin{lemma}[\citet{sherman1950adjustment}] \label{lem:sherman}
%     Suppose $\bA \in \Real^{n \times n}$ is an invertible square matrix and $u,v \in \Real^n$ are column vectors. Then $\bA + uv^T$ is invertible iff $1 + v^T \bA u \ne 0$ and in particular
%     \begin{align}
%         (\bA + u v^T)^{-1} = \bA^{-1}  - \frac{\bA^{-1} uv^T \bA^{-1} }{ 1 + v^T \bA^{-1} u} \,.
%     \end{align}   
% \end{lemma}
% \newcommand\byy[1]{\by_{\left(#1\right)}}
% \newcommand\bXX[1]{\bX_{\left(#1\right)}}
% \newcommand\ff[1]{\wh f_{\left(#1\right)}}

% For a given training set $S \cup \wt S_C$, define leave-one-out error on mislabeled points in the training data as $$\error_{\text{LOO}(\wt S_M) } = \frac{\sum_{(x_i, y_i) \in \wt S_M} \error( f_{(i)}( x_i), y_i)}{ \abs{\wt S_M }} \,, $$
% where $f_{(i)} \defeq f(\calA, (S \cup \wt S)_{(i)})$. To relate empirical leave-one-out error and population error with hypothesis stability condition, we use the following lemma:   

% \begin{lemma}[\citet{bousquet2002stability}] \label{lem:stability_error}
%     For the leave-one-out error, we have
%     \begin{align}
%         \Expo{ \left( \error_{\calDm}(\wh f) -\error_{\text{LOO}(\wt S_M) } \right)^2 } \le \frac{1}{2m_1}+  \frac{3\beta}{n + m}\,.
%     \end{align}   
%     % where $ f \defeq f(\calA, S \cup \wt S) $.
% \end{lemma}

% Proof of the above lemma is similar to the proof of  Lemma 9 in \citet{bousquet2002stability} and can be found in \appref{app:proof_lem_error}. 
% % 
% % Before presenting the result, we introduce some notation. 
% Before presenting the proof of \thmref{thm:new_linear}, we introduce some more notation. Let $\bX_{(i)}$ denote the matrix of covariates with $i^{\text{th}}$ point removed. Similarly let $\by_{(i)}$ be the array of responses with $i^{\text{th}}$ point removed. Define the corresponding regularized GD solution as $\wh w_{(i)} = \left( \bXX{i}^T\bXX{i}+\lambda \boldsymbol{I}\right)^{-1}\bXX{i}^T\byy{i}$. Define $\ff{i}(x) \defeq f(x ; \wh w_{(i)}) $.

% \begin{proof}[Proof of \thmref{thm:new_linear}]
%     Because squared loss minimization does not imply 0-1 error minimization, we cannot use arguments from \lemref{lem:fit_mislabeled}. This is the main technical difficulty. To compare the 0-1 error at a train point with an unseen point, 
%     we use the closed-form expression for $\widehat{w}$ and Shermann-Morrison formula to upper bound training error with leave-one-out cross validation error. 
    
%     The proof is divided into three parts: In part one, we show that 0-1 error on mislabeled points in the training set is lower than the error obtained by leave-one-out error at those points. In part two, we relate this leave-one-out error with the population error on mislabeled distribution using \codref{cond:hypothesis_stability}. While the empirical leave-one-out error is unbiased estimator of the average population error of leave-one-out classifiers, we need hypothesis stability to control the variance of empirical leave-one-out error. Finally in part three, we show that the error on the mislabeled training points can be estimated with just the randomly labeled and  clean training data (as in proof of \thmref{thm:error_ERM}).  

%     \textbf{Part 1 {} {}} First we relate training error with leave-one-out error.        
%     For any 
%     training point $(x_i, y_i)$ in $\wt S \cup S$, we have 
%     \begin{align}
%         \error(\wh f(x_i), y_i ) &= \indict{ y_i \cdot x_i^T \wh w < 0 } = \indict{ y_i \cdot x_i^T \left( \bX^T\bX+\lambda \boldsymbol{I}\right)^{-1}\bX^T\by < 0 } \\
%         &= \indict{ y_i \cdot x_i^T \underbrace{\left( \bXX{i}^T\bXX{i} + x_i ^T x_i +\lambda \boldsymbol{I}\right)^{-1}}_{\RN{1}} (\bXX{i}^T\byy{i} + y \cdot x_i) < 0 }
%     \end{align}
%     Letting $\bA = \left(\bXX{i}^T\bXX{i} +\lambda \boldsymbol{I}\right)$ and using \lemref{lem:sherman} on term 1, we have 
%     \begin{align}
%         \error(\wh f(x_i), y_i ) &= \indict{ y_i \cdot x_i^T \left[\bA^{-1} -  \frac{\bA^{-1} x_i x_i^T \bA^{-1}}{ 1 + x_i ^T \bA^{-1} x_i } \right] (\bXX{i}^T\byy{i} + y \cdot x_i) < 0 } \\
%         &= \indict{ y_i \cdot\left[ \frac{ x_i^T \bA^{-1} ( 1 + x_i ^T \bA^{-1} x_i ) -  x_i^T \bA^{-1} x_i x_i^T \bA^{-1}}{ 1 + x_i ^T \bA ^{-1}x_i } \right] (\bXX{i}^T\byy{i} + y \cdot x_i) < 0 } \\
%         &= \indict{ y_i \cdot\left[ \frac{ x_i^T \bA^{-1}}{ 1 + x_i ^T \bA ^{-1}x_i } \right] (\bXX{i}^T\byy{i} + y \cdot x_i) < 0 } \,.
%     \end{align}

%     Since $1 + x_i^T \bA^{-1} x_i > 0$, we have 
%     \begin{align}
%         \error(\wh f(x_i), y_i ) &= \indict{ y_i \cdot x_i^T \bA^{-1} (\bXX{i}^T\byy{i} + y \cdot x_i) < 0 } \\
%         &= \indict{ x_i^T \bA^{-1} x_i +  y_i \cdot x_i^T \bA^{-1} (\bXX{i}^T\byy{i}) < 0 } \\
%         &\le \indict{ y_i \cdot x_i^T \bA^{-1} (\bXX{i}^T\byy{i}) < 0 } = \error(\ff{i}(x_i), y_i ) \,.\label{eq:LOO_error}
%     \end{align}

%     Using \eqref{eq:LOO_error}, we have 
%     \begin{align}
%         \error_{\wt \calS_M } (\wh f) \le \error_{\text{LOO} (S_M)} \defeq \frac{\sum_{(x_i, y_i) \in \wt S_M} \error(\ff{i}(x_i), y_i ) }{\abs{\wt \calS_M}}\label{eq:LOO_error_final}
%     \end{align}
%     \textbf{Part 2 {}{}} We now relate RHS in \eqref{eq:LOO_error_final} with the population error on mislabeled distribution. To do this, we leverage \codref{cond:hypothesis_stability} and \lemref{lem:stability_error}. In particular, we have 

%     \begin{align}
%         \Expt{\calS \cup \wt \calS_M }{ \left(\error_{\calDm}(\wh f) - \error_{\text{LOO} (S_M)}\right)^2 } \le \frac{1}{2m_1} + \frac{3\beta}{m+n} \,.
%     \end{align}

%     Using Chebyshev's inequality, with probability at least $1-\delta$, we have 
%     \begin{align}
%         \error_{\text{LOO} (S_M)} \le  \error_{\calDm}(\wh f)   + \sqrt{\frac{1}{\delta}\left(\frac{1}{2m_1} +\frac{3\beta}{m+n} \right)} \,. \label{eq:final_mislabeled_linear}
%     \end{align}
    

%     \textbf{Part 3 {}{}} Combining \eqref{eq:final_mislabeled_linear} and \eqref{eq:LOO_error_final}, we have 

%     \begin{align}
%         \error_{\wt \calS_M } (\wh f) \le \error_{\calDm}(\wh f)   + \sqrt{\frac{1}{\delta}\left(\frac{1}{2m_1} +\frac{3\beta}{m+n} \right)} \,. \label{eq:linear_parallel_lem1}
%     \end{align}

%     Compare \eqref{eq:linear_parallel_lem1}, with \eqref{eq:lemma1_final} in the proof of \lemref{lem:fit_mislabeled}. We obtain a similar relationship between $\error_{\wt \calS_M }$ and $\error_{\calDm}$ but with a polynomial concentration instead of exponential concentration. 
%     In addition, since we just use concentration arguments to relate mislabeled error with the error on clean portion and unlabeled portion, we can directly use the results in \lemref{lem:mislabeled_error} and \lemref{lem:clear_error}. Therefore, combining results in \lemref{lem:mislabeled_error}, \lemref{lem:clear_error}, and \eqref{eq:linear_parallel_lem1} with union bound, we have with probability at least $1-\delta$

%     \begin{align}
%         \error_\calD(\widehat f) \le \error_\calS(\widehat f) + 1 - 2 \error_{\wt\calS}(\widehat f) + \left(\frac{1}{\sqrt{2}} + 1.5 \right) \sqrt{\frac{\log(4/\delta)}{m}} + \sqrt{\frac{4}{\delta}\left(\frac{1}{m} +\frac{3\beta}{m+n} \right)}  \,.
%     \end{align}
    

       
% \end{proof}

% \subsection{Discussion on \codref{cond:hypothesis_stability}}

% The quantity in LHS of \codref{cond:hypothesis_stability} measures how much the function learned by the algorithm (in terms of error on unseen point) will change when one point in the training set is removed. 
% % Discussion on exponential concentration and stronger condition. 
% Notice that hypothesis stability implies error stability, i.e., \codref{cond:error_stability} ~\cite{bousquet2002stability}.  In summary, while error stability allowed us to relate the average population error of the leave-one-out classifiers with the population error of the original classifier, we need hypothesis stability condition to control the variance of the empirical leave-one-out error. 

% Additionally, we note that while the dominating term in the RHS of \thmref{thm:new_linear} matches with the dominating term in ERM bound in \thmref{thm:error_ERM}, there is a polynomial concentration term (dependence on $1/\delta$ instead of $\log(\sqrt{1/\delta})$) in  \thmref{thm:new_linear}. 
% Since with hypothesis stability, we just bound the variance,  the polynomial concentration is due to the use of Chebyshev's inequality instead of an exponential tail inequality (as in \lemref{lem:fit_mislabeled}).
% Recent works have highlighted that slightly stronger condition than hypothesis stability can be used to obtained an exponential concentration for leave-one-out error~\citep{abou2019exponential}, but we leave this for future work for now. 
% % We leave 
% % However, the constants 

% % we also want to highlight  

% \subsection{Formal statement and proof of  of \propref{prop:early_stop}}

% Before formally presenting the result, we will introduce some notation.  By $\calL_{S}(w)$, we denote 
% the objective in \eqref{eq:l2_MSE_app} with $\lambda=0$. 
% Assume Singular Value Decomposition (SVD) of $\bX$  as $\sqrt{n} \bU \bS^{1/2} \bV^T$. Hence $\bX^T \bX = \bV \bS \bV^T$.
% Consider the GD iterates defined in \eqref{eq:GD_iterates_app}. 
% % 
% We now derive closed form expression for the $t^\text{th}$ iterate of gradient descent:  
% % 
% \begin{align}
%     w_t = w_{t-1} + \eta \cdot \bX^T (\by - \bX w_{t-1}) = (\bI - \eta \bV \bS \bV^T )w_{k-1} + \eta \bX^T \by \,.
% \end{align}
% Rotating by $\bV^T$, we get 
% \begin{align}
%     \wt w_t = (\bI - \eta\bS )\wt w_{k-1} + \eta \wt \by \,, \label{eq:GD_recur}
% \end{align}
% where $\wt w_t = \bV^T w_t $ and $\wt \by = \bV^T \bX^T \by$. Assuming the initial point $w_0 = 0$ and applying the recursion in \eqref{eq:GD_recur}, we get
% \begin{align}
%     \wt w_t = \bS ^{-1} ( \bI - (\bI - \eta \bS)^k ) \wt \by \,, 
% \end{align} 
% Projecting solution back to the original space, we have 
% \begin{align}
%      w_t = \bV \bS ^{-1} ( \bI - (\bI - \eta \bS)^k ) \bV^T \bX^T \by \,, 
% \end{align} 
% % We will work with this GD solution at any iterate $t$ in the next proposition. 
% Define $f_t(x) \defeq f(x;w_t)$ as the solution at the $t^{\text{th}}$ iterate. 
% Let $\wt w_{\lambda} = \argmin_{w} \calL_\calS (w;\lambda) = (\bX^T \bX + \lambda \bI)^{-1} \bX^T \by = \bV (\bS + \lambda \bI )^{-1} \bV^T \bX^T \by $. 
% % ) \,,$ for all $t=1,2,\ldots\,.$ 
% and define $\wt f_\lambda(x) \defeq f(x;\wt w_\lambda)$ as the regularized solution. 
% Assume $\kappa$ be the condition number of the population covariance matrix 
% and 
% let $s_\text{min}$ be the minimum positive singular value of the empirical covariance matrix. Our proof idea is inspired from recent work on relating gradient flow solution and regularized solution for regression problems \citep{ali2018continuous}. We will use the following lemma in the proof: 
% \begin{lemma} \label{lem:ineq_soln}
%     For all $x \in [0,1]$ and for all $ k \in \mathbb{N}$, we have (a) $ \frac{kx}{1+kx} \le 1- (1-x)^k$ and (b) $ 1- (1-x)^k \le 2 \cdot \frac{kx}{kx+1} $.
%     %  where $g(c)$ is a constant dependent on $c$. For $c = 1$, $g(c) = 2.0$.   
% \end{lemma}
% \begin{proof}
%     % [Proof of \lemref{lem:ineq_soln}]
%     % Part (a) is easy. 
%     Using $ (1-x)^k \le \frac{1}{1+kx}$, we have part (a). For part (b), we numerically maximize $\frac{ (1+kx ) (1 - (1-x)^k) }{kx}$ for all $k\ge 1$ and for all $x \in [0, 1]$.  
% \end{proof}

% % 
% % Next, 

% \begin{prop}[Formal statement of \propref{prop:early_stop}] \label{prop:formal_early_stop}
% Let $\lambda = \frac{1}{t\eta}$. For a training point $x$, we have 
% \begin{align*}
%     \Expt{x \sim \calS}{(f_t(x) - \wt f_\lambda(x))^2} &\le c(t,\eta) \cdot \Expt{x \sim \calS}{f_t(x)^2} \,, %\label{eq:early_stop}
% \end{align*}
% where $c(t, \eta) \defeq \min( 0.25, \frac{1}{s_\text{min}^2 t^2 \eta^2})$. Similarly for a test point, we have 
% \begin{align*}
%     \Expt{x \sim \calD_\calX}{(f_t(x) - \wt f_\lambda(x))^2} &\le \kappa \cdot c(t,\eta) \cdot \Expt{x \sim \calD_\calX}{f_t(x)^2} \,. %\label{eq:early_stop}
% \end{align*}
% \end{prop} 

% \begin{proof}
%     %%%%%%%%%%%%% 
%     We want to analyze the expected squared difference output of regularized linear regression with regularization constant $\lambda = \frac{1}{\eta t}$ and gradient descent solution at $t^\text{th}$ iterate. We separately expand the algebraic expression for squared difference at a training point and a test point. 
%     % We start by considering the difference  
%     Then the main step is to show that  $\left[ \bS ^{-1} ( \bI - (\bI - \eta \bS)^k )  - (\bS + \lambda \bI )^{-1}\right] \preceq c(\eta, t) \cdot \bS ^{-1} ( \bI - (\bI - \eta \bS)^k ) $.

%     %%%%%%%%%%%%%
    
%   \textbf{Part 1 {} {}} 
%     First, we will analyze the squared difference of output at a training point (for simplicity, we refer to $S \cup \wt S$ as $S$), i.e. 
%     \begin{align}
%         \Expt{ x \sim \calS }{\left(f_t(x) - \wt f_\lambda (x)\right)^2} &= \norm{\bX w_t - \bX \wt w_\lambda}{2}^2 =   \norm{\bX \bV \bS ^{-1} ( \bI - (\bI - \eta \bS)^t ) \bV^T \bX^T \by - \bX \bV (\bS + \lambda \bI )^{-1} \bV^T \bX^T \by }{2}^2 \\
%         &= \norm{\bX \bV \left(\bS ^{-1} ( \bI - (\bI - \eta \bS)^t ) - (\bS + \lambda \bI )^{-1} \right) \bV^T \bX^T \by  }{2} \\
%         &=  \by^T \bV \bX \left( \underbrace{\bS ^{-1} ( \bI - (\bI - \eta \bS)^t ) - (\bS + \lambda \bI )^{-1}}_{\RN{1}} \right)^2 \bS \bV^T \bX^T \by \label{eq:train_GD_rel}
%         %  (\bX \bV \bS ^{-1} ( \bI - (\bI - \eta \bS)^k ) \bV^T \bX^T \by)^T \bX \bV \bS ^{-1} ( \bI - (\bI - \eta \bS)^k ) \bV^T \bX^T \by
%     \end{align}
%     We now separately consider term 1. Substituting $\lambda = \frac{1}{t \eta}$, we get
%     \begin{align}
%         \bS ^{-1} ( \bI - (\bI - \eta \bS)^t ) - (\bS + \lambda \bI )^{-1} &= \bS^{-1} \left( ( \bI - (\bI - \eta \bS)^t ) - (\bI + \bS^{-1} \lambda )^{-1}\right) \\
%         &= \underbrace{\bS^{-1} \left( ( \bI - (\bI - \eta \bS)^t ) - (\bI + ( \bS t \eta)^{-1}  )^{-1}\right)}_{\bA}
%     \end{align}

%     We now separately bound the diagonal entries in matrix $\bA$. 
%     With $s_i$, we denote $i^{\text{th}}$ diagonal entry of $\bS$. Note that since $ \eta\le 1/\norm{S}{\text{op}}$, for all $i$, $\eta s_i  \le 1$.  Consider $i^{\text{th}}$ diagonal term (which is non-zero) of the diagonal matrix $\bA$, we have 
%     \begin{align}
%         \bA_{ii} = \frac{1}{s_i} \left(  1 - (1 - s_i \eta)^t - \frac{t \eta s_i}{1 + t \eta s_i } \right) &=  \frac{1 - (1 - s_i \eta)^t}{s_i} \left( \underbrace{ 1 - \frac{t \eta s_i}{(1 + t \eta s_i)(1 - (1 - s_i \eta)^t)}}_{\RN{2}} \right) \\ 
%          &\le \frac{1}{2}\left[ \frac{1 - (1 - s_i \eta)^t}{ s_i} \right] \tag*{(Using \lemref{lem:ineq_soln} (b))} \,.
%     \end{align} 
%     Additionally, we can also show the following upper bound on term 2: 
%     \begin{align}
%          1 - \frac{t \eta s_i}{(1 + t \eta s_i)(1 - (1 - s_i \eta)^t)} &= \frac{(1 + t \eta s_i)(1 - (1 - s_i \eta)^t) - t \eta s_i }{(1 + t \eta s_i)(1 - (1 - s_i \eta)^t)} \\
%          & \le  \frac{ 1 -  (1 - s_i \eta)^t - t \eta s_i (1 - s_i \eta)^t}{(1 + t \eta s_i)(1 - (1 - s_i \eta)^t)} \\
%          & \le \frac{1}{t\eta s_i} \,. \tag{Using \lemref{lem:ineq_soln} (a)}
%         %  &\le \frac{1}{2}\left[ \frac{1 - (1 - s_i \eta)^t}{ s_i} \right] \tag*{(Using \lemref{lem:ineq_soln})} \,.
%     \end{align} 

%     Combining both the upper bounds on each diagonal entry $\bA_{ii}$, we have 
%     \begin{align}
%     \bA \preceq c_1(\eta, t) \cdot \bS^{-1} ( \bI - (\bI - \eta \bS)^t ) \,, \label{eq:upperbound_diagonal}
%     \end{align}
%     where $c_1(\eta, t ) = \min(0.5, \frac{1}{t s_i \eta })$. Plugging this into \eqref{eq:train_GD_rel}, we have 
%     \begin{align}
%         \Expt{ x \sim \calS }{\left(f_t(x) - \wt f_\lambda (x)\right)^2} &\le c(\eta, t) \cdot \by^T \bV \bX  \left( \bS^{-1} ( \bI - (\bI - \eta \bS)^t ) \right)^2 \bS \bV^T \bX^T \by \\
%         &=   c(\eta, t) \cdot \by^T \bV \bX  \left( \bS^{-1} ( \bI - (\bI - \eta \bS)^t ) \right) \bS \left( \bS^{-1} ( \bI - (\bI - \eta \bS)^t ) \right) \bV^T \bX^T \by \\
%         & =  c(\eta, t) \cdot \norm{\bX w_t}{2}^2 \\
%         &= c(\eta, t) \cdot  \Expt{ x \sim \calS }{\left(f_t(x) \right)^2} \,,
%     \end{align}
%     where $c(\eta, t ) = \min(0.25, \frac{1}{t^2 s^2_i \eta^2 })$.

%     \textbf{Part 2 {} {}} With $\bSigma$, we denote the underlying true covariance matrix. We now consider the squared difference of output at an unseen point: 
%     \begin{align}
%         \Expt{ x \sim \calD_{\calX} }{\left(f_t(x) - \wt f_\lambda (x)\right)^2} &= \Expt{x \sim \calD_{\calX}}{\norm{x^T w_t - x^T \wt w_\lambda}{2}} \\
%         &=   \norm{x^T \bV \bS ^{-1} ( \bI - (\bI - \eta \bS)^t ) \bV^T \bX^T \by - x^T \bV (\bS + \lambda \bI )^{-1} \bV^T \bX^T \by }{2} \\
%         &= \norm{x^T \bV \left(\bS ^{-1} ( \bI - (\bI - \eta \bS)^t ) - (\bS + \lambda \bI )^{-1} \right) \bV^T \bX^T \by  }{2} \\
%         &= \by^T \bV \bX \left( \bS ^{-1} ( \bI - (\bI - \eta \bS)^t ) - (\bS + \lambda \bI )^{-1} \right) \bV^T \bSigma \bV \\ &\qquad \qquad \qquad \qquad \qquad \left( (\bI - (\bI - \eta \bS)^t ) - (\bS + \lambda \bI )^{-1} \right) \bV^T \bX^T \by \\
%         &\le \sigma_{\text{max}} \cdot \by^T \bV \bX \left( \underbrace{\bS ^{-1} ( \bI - (\bI - \eta \bS)^t ) - (\bS + \lambda \bI )^{-1}}_{\RN{1}} \right)^2 \bV^T \bX^T \by \,, \label{eq:test_GD_rel}
%         %  (\bX \bV \bS ^{-1} ( \bI - (\bI - \eta \bS)^k ) \bV^T \bX^T \by)^T \bX \bV \bS ^{-1} ( \bI - (\bI - \eta \bS)^k ) \bV^T \bX^T \by
%     \end{align}
%     where $\sigma_{\text{max}}$ is the maximum eigenvalue of the underlying covariance matrix $\bSigma$. Using the upper bound on term 1 in \eqref{eq:upperbound_diagonal}, we have 
%     \begin{align}
%         \Expt{ x \sim \calD_{\calX} }{\left(f_t(x) - \wt f_\lambda (x)\right)^2} &\le \sigma_{\text{max}} \cdot c(\eta, t) \cdot \by^T \bV \bX  \left( \bS^{-1} ( \bI - (\bI - \eta \bS)^t ) \right)^2 \bV^T \bX^T \by \\
%         &=   \kappa \cdot c(\eta, t) \cdot \sigma_{\text{min}}\cdot \norm{\bV \left( \bS^{-1} ( \bI - (\bI - \eta \bS)^t ) \right) \bV^T \bX^T \by}{2}^2 \\
%         &\le \kappa \cdot c(\eta, t) \cdot \left[ \bV \left( \bS^{-1} ( \bI - (\bI - \eta \bS)^t ) \right) \bV^T \bX^T \right]^T \bSigma \\
%         &\qquad \qquad \qquad \qquad \qquad \left[ \bV \left( \bS^{-1} ( \bI - (\bI - \eta \bS)^t ) \right) \bV^T \bX^T \right] \by \\
%         & = \kappa \cdot c(\eta, t) \cdot \Expt{x \sim \calD_{\calX}}{\norm{x^T w_t}{2}} \,.
%     \end{align}
% % 
% % 
%     % Since $ \eta\le 1/\norm{S}{\text{op}}$, invoking \lemref{lem:ineq_soln} to upper bound term 1 with
% \end{proof}


% \newpage
% \section{Additional experiments and details}\label{app:exp}
% \newcommand\tab[1][1cm]{\hspace*{#1}}

% \subsection{Datasets} \label{sec:app_dataset}

% \textbf{Toy Dataset {} {}} Assume fixed constants $\mu$ and $\sigma$. For a given label $y$, we simulate features $x$ in our toy classification setup as follows: 
% \begin{align*}
%     x \defeq \texttt{concat} \left[ x_1, x_2\right] \quad \text{where} \quad  x_1 \sim  \calN( y \cdot \mu, \sigma^2 I_{d \times d}) \ \  \text{and} \ \  x_1 \sim  \calN( 0, \sigma^2 I_{d \times d}) \,.
% \end{align*}  
% % where $y$ is the true label and $x$ is the corresponding feature vector. 
% In experiements throughout the paper, we fix dimention $d=100$, $\mu = 1.0 $, and $\sigma = \sqrt{d}$. Intuitively, $x_1$ carries the information about the underlying label and $x_2$ is additional noise independent of the underlying label. 

% \textbf{CV datasets {} {}} We use MNIST~\citep{lecun1998mnist} and CIFAR10~\cite{krizhevsky2009learning}. 
% % For binary tasks, 
% We produce a binary variant from the multiclass classification problem by mapping classes $\{0,1,2,3,4\}$ to label $1$ and $\{ 5,6,7,8,9\}$ to label $-1$. For CIFAR dataset, we also use the standard data augementation of random crop and horizontal flip. PyTorch code is as follows: 

% \texttt{(transforms.RandomCrop(32, padding=4),\\
% \tab transforms.RandomHorizontalFlip())}

% \textbf{NLP dataset {} {}} We use IMDb Sentiment analysis~\citep{maas2011learning} corpus.  

% \subsection{Architecture Details} 

% All experiments were run on NVIDIA GeForce RTX 2080 Ti GPUs. We used PyTorch~\citep{NEURIPS2019a9015} and Keras with Tensorflow~\citep{abadi2016tensorflow} backend for experiments. 
% % , ELMo embeddings~\citep{Peters:2018}, and Hugging Face Transformers~\citep{wolf-etal-2020-transformers}. 

% \textbf{Linear model {} {}} For the toy dataset, we simulate a linear model with scalar output and the same number of parameters as the number of dimensions.   

% \textbf{Wide nets {} {}} To simulate the NTK regime, we experiment with $2-$layered wide nets. The PyTorch code for 2-layer wide MLP is as follows: 


% \texttt{ nn.Sequential( \\
% \tab     nn.Flatten(),\\
% \tab    nn.Linear(input\_dims, 200000, bias=True),\\
% \tab    nn.ReLU(),\\
% \tab    nn.Linear(200000, 1, bias=True)\\
% \tab     )}


% We experiment both (i) with the first layer fixed at random initialization; (ii)  and updating both layers' weights.     

% \textbf{Deep nets for CV tasks {} {}} We consider a 4-layered MLP. The PyTorch code for 4-layer MLP is as follows: 

% \texttt{ nn.Sequential(nn.Flatten(), \\
% \tab        nn.Linear(input\_dim, 5000, bias=True),\\
% \tab        nn.ReLU(),\\
% \tab        nn.Linear(5000, 5000, bias=True),\\
% \tab        nn.ReLU(),\\
% \tab        nn.Linear(5000, 5000, bias=True),\\
% \tab        nn.ReLU(),\\
% % \tab        nn.Linear(5000, 5000, bias=True),\\
% % \tab        nn.ReLU(),\\
% \tab        nn.Linear(1024, num\_label, bias=True)\\
% \tab        )}

% For MNIST, we use $1000$ nodes instead of $5000$ nodes in the hidden layer. 
% % 
% We also experiment with convolutional nets. In particular, we use ResNet18 \citep{he2016deep}. Implementation adapted from:  \url{https://github.com/kuangliu/pytorch-cifar.git}. 

% \textbf{Deep nets for NLP {} {}} We use a simple LSTM model with embeddings intialized with ELMo embeddings~\citep{Peters:2018}. Code adapted from: \url{https://github.com/kamujun/elmo_experiments/blob/master/elmo_experiment/notebooks/elmo_text_classification_on_imdb.ipynb} 

% We also evaluate our bounds with a BERT model. In particular, we fine-tune an off-the-shelf uncased BERT model~\citep{devlin2018bert}. Code adapted from Hugging Face Transformers~\citep{wolf-etal-2020-transformers}: \url{https://huggingface.co/transformers/v3.1.0/custom_datasets.html}. 


% \subsection{Additonal experiments}

% 1. SGD with linear models on cross entropy and MSE loss. 

% 2. CE loss and SGD. GD with MSE loss 

% 3. Binary MNIST with MLP. multiclass MNIST  

% \textbf{Results on CIFAR 10 {} {}} 
% % 
% We plot epoch wise error curve for results in \tabref{table:multiclass}. We observe the same trend as in \figref{fig:error_CIFAR10}. Additionally, we plot an \emph{oracle bound} obtained by tracking the error on mislabeled data which nevertheless were predicted as true label. To obtain an exact emprical value of the oracle bound, we need underlying true labels for the randomly labeled data. 
% % Note that our bound in \thmref{thm:multiclass_ERM}, lower bounds the accuracy as predicted by the oracle bound. 
% While with just access to extra unlabeled data we cannot calculate oracle bound, we note that the oracle bound is very tight and never violated in practice underscoring an importamt aspect of generalization in multiclass problems. This highlight that even a stronger conjecture may hold in multiclass classification, i.e., error on mislabeled data (where nevertheless true label was predicted) lower bounds the population error on the distribution of mislabeled data and hence, the error on (a specific) mislabeled portion predicts the population accuracy on clean data. 
% % 
% On the other hand, the dominating term of in \thmref{thm:multiclass_ERM} is loose when compared with the oracle bound. The main reason, we believe is the pessimistic upper bound in \eqref{eq:lemma1_final_multi_prev} in the proof of \lemref{lem:fit_mislabeled_multi}. We leave an investigation on this gap for future. 
% % of fit 

% % However, oracle bound highlights two . One,  



% \begin{figure}[h]
%     \centering 
%     % \vspace{-15pt}
%     % \includegraphics[width=0.9\linewidth]{example-image-a}
%     \includegraphics[width=0.4\linewidth]{figures/CIFAR10-FNN.pdf} \hfil
%     \includegraphics[width=0.4\linewidth]{figures/CIFAR10-Resnet.pdf}
%     % \includegraphics[width=0.9\linewidth]{figures/{CIFAR10_rn=0.1_lr=0.2_wd=0.005}.png}
%     % \vspace{-10pt}
%     \caption{ Per epoch curves for CIFAR10 corresponding results in \tabref{table:multiclass}. As before, we just plot the dominating term in the RHS of \eqref{eq:multiclass_ERM} as predicted bound. Additionally, we also plot the predicted lower bound by the error on mislabeled data which nevertheless were predicted as true label. We refer to this as ``Oracle bound''. See text for more details. 
%     % 
%     % except for the stopping point. 
%     % The bound predicted by RATT (RHS in \eqref{eq:multiclass_ERM}) is vacuous. 
%     }\label{fig:error_epoch_CIFAR10}
%     % \vspace{-15pt}
% \end{figure}


% \textbf{Results on CIFAR 100 {} {}} 
% % 
% On CIFAR100, our bound in \eqref{eq:multiclass_ERM} yields vacous bounds. However, the oracle bound as explained above yields tight guarantees in the initial phase of the learning (i.e., when learning rate is less than $0.1$). 

% \begin{figure}[h]
%     \centering 
%     % \vspace{-15pt}
%     % \includegraphics[width=0.9\linewidth]{example-image-a}
%     \includegraphics[width=0.4\linewidth]{figures/CIFAR100-Resnet.pdf}
%     % \includegraphics[width=0.9\linewidth]{figures/{CIFAR10_rn=0.1_lr=0.2_wd=0.005}.png}
%     % \vspace{-10pt}
%     \caption{ Predicted lower bound by the error on mislabeled data which nevertheless were predicted as true label with ResNet18 on CIFAR100. We refer to this as ``Oracle bound''. See text for more details. 
%     % 
%     % except for the stopping point. 
%     The bound predicted by RATT (RHS in \eqref{eq:multiclass_ERM}) is vacuous. 
%     }\label{fig:error_CIFAR100}
%     % \vspace{-15pt}
% \end{figure}


% % \paragraph{Experiments on CIFAR100} 



% \subsection{Hyperparameter Details}


% \textbf{\figref{fig:error_CIFAR10} {} {}} We use clean training dataset of size $40,000$. We fix the amount of unlabeled data at $20\%$ of the clean size, i.e. we include additional $8,000$ points with randomly assigned labels. We use test set of $10,000$ points. For both MLP and ResNet, we use SGD with an initial learning rate of $0.1$ and momentum $0.9$. We fix the weight decay parameter at $5\times 10^{-4}$. After $100$ epochs, we decay the learning rate to $0.01$. We use SGD batch size of $100$. 

% \textbf{\figref{fig:error_binary} (a) {} {}} We obtain a toy dataset according to the process described in \secref{sec:app_dataset}. We fix $d=100$ and create a dataset of $50,000$ points with balanced classes. Moreover, we sample additional covariates with the same procedure to create randomly labeled dataset. For both SGD and GD training, we use a fixed learning rate $0.1$.    

% \textbf{\figref{fig:error_binary} (b) {} {}} Similar to binary CIFAR, we use clean training dataset of size $40,000$ and fix the amount of unlabeled data at $20\%$ of the clean dataset size. To train wide nets, we use a fixed learning of $0.001$ with GD and SGD. We decide the weight decay parameter and the early stopping point that maximizes our generalization bound (i.e. without peeking at unseen data ).  We use SGD batch size of $100$. 

% \textbf{\figref{fig:error_binary} (c) {} {}} With IMDb dataset, we use a clean dataset of size $20,000$ and as before, fix the amount of unlabeled data at $20\%$ of the clean data. To train ELMo model, we use Adam optimizer with a fixed learning rate $0.01$ and weight decay $10^{-6}$ to minimize cross entropy loss. We train with batch size $32$ for 3 epochs. To fine-tune BERT model, we use Adam optimizer with learning rate $5\times 10^{-5}$ to minimize cross entropy loss. We train with a batch size of $16$ for 1 epoch.    

% \textbf{\tabref{table:multiclass} {} {}} For multiclass datasets, we train both MLP and ResNet with the same hyperparameters as described before. We sample a clean training dataset of size $40,000$ and fix the amount of unlabeled data at $20\%$ of the clean size. We use SGD with an initial learning rate of $0.1$ and momentum $0.9$. We fix the weight decay parameter at $5\times 10^{-4}$. After $30$ epochs for ResNet and after $50$ epochs for MLP, we decay the learning rate to $0.01$.  We use SGD with batch size $100$. 
% For \figref{fig:error_CIFAR100}, we use the same hyperparameters as 
% CIFAR10 training, except we now decay learning rate after $100$ epochs. 


% In all experiments, to identify the best possible accuracy on just the clean data, we use the exact same set of hyperparamters except the stopping point. We choose a stopping point that maximizes test performance. 

% \subsection{Summary of experiments }

% \begin{center}
%     \begin{table}[H] 
%         \centering
%         \begin{tabular}{|c|c|c|c|} 
%         \hline
%         Classification type & Model category & Model & Dataset  \\ [0.5ex] 
%         \hline
%         \hline
%         \multirow{9}{*}{Binary} & Low dimensional & Linear model & Toy Gaussain dataset  \\
%                         \cline{2-4}
%                          & \multirow{1}{*}{Overparameterized linear nets} 
%                         %  & Linear model & Toy Gaussain dataset \\
%                         %  \cline{3-4}
%                         %  & & 2-layer wide net& Toy Gaussain dataset \\
%                         %  \cline{3-4}
%                          & 2-layer wide net & Binary MNIST \\
%                          \cline{2-4}                 
%                          & \multirow{6}{*}{Deep nets} & \multirow{2}{*}{MLP} & Binary MNIST \\
%                          \cline{4-4}
%                          & &  & Binary CIFAR \\
%                          \cline{3-4}
%                          &  & \multirow{2}{*}{ResNet} & Binary MNIST \\
%                          \cline{4-4}
%                          & &  & Binary CIFAR \\
%                          \cline{3-4}
%                          &  & ELMo-LSTM model & IMDb Sentiment Analysis \\
%                          \cline{3-4}
%                          & & BERT pre-trained model & IMDb Sentiment Analysis \\
%         \hline
%         \multirow{5}{*}{Multiclass} & \multirow{5}{*}{Deep nets} & \multirow{2}{*}{MLP} & MNIST \\
%                         \cline{4-4} 
%                         & & & CIFAR10 \\                   
%                         \cline{3-4}
%                          &   & \multirow{3}{*}{ResNet} & MNIST \\
%                          \cline{4-4}
%                          &   & & CIFAR10 \\
%                          \cline{4-4}
%                          &   & & CIFAR100 \\
%         \hline
%         \end{tabular}
%         % \caption{Summary of experiments performed} \label{table:experiments}
%     \end{table}    
%     % \footnotetext[6]{We use both MSE loss and cross-entropy loss.}
%     % \footnotetext[6]{We try 2 variants: one with a fixed first layer and the other with both layers trainable.}
% \end{center}

% \newpage
% \section{Proof of \lemref{lem:stability_error}} \label{app:proof_lem_error}

% \begin{proof}[Proof of \lemref{lem:stability_error}]
%     Recall, we have a training set $S \cup \wt S_C$. We defined leave-one-out error on mislabeled points as $$\error_{\text{LOO}(\wt S_M) } = \frac{\sum_{(x_i, y_i) \in \wt S_M} \error( f_{(i)}( x_i), y_i)}{ \abs{\wt S_M }} \,, $$
%     where $f_{(i)} \defeq f(\calA, (S \cup \wt S)_{(i)})$. Define $S^\prime \defeq S \cup \wt S$. Assume $(x,y)$ and $(x^\prime,y^\prime)$ as i.i.d. samples from ${\calDm}$. 
%     Using Lemma 25 in \citet{bousquet2002stability}, we have
%     \begin{align*}
%         \Expo{ \left( \error_{\calDm}(\wh f) -\error_{\text{LOO}(\wt S_M) } \right)^2 } \le & \Expt{ S^\prime, (x,y), (x^\prime,y^\prime) }{ \error(\wh f(x), y ) \error(\wh f(x^\prime), y^\prime )} - 2 \Expt{ S^\prime, (x,y) }{ \error(\wh f(x), y ) \error(f_{(i)}(x_i), y_i )} \\
%         & + \frac{m_1-1}{m_1}\Expt{ S^\prime }{  \error(f_{(i)}(x_i), y_i )  \error(f_{(j)}(x_j), y_j )} + \frac{1}{m_1} \Expt{ S^\prime }{  \error(f_{(i)}(x_i), y_i ) } \,. \numberthis \label{eq:main_reln}
%     \end{align*}
%     We can rewrite the equation above as : 
%     \begin{align*}
%         \Expo{ \left( \error_{\calDm}(\wh f) -\error_{\text{LOO}(\wt S_M) } \right)^2 } \le &  \, \underbrace{\Expt{ S^\prime, (x,y), (x^\prime,y^\prime) }{ \error(\wh f(x), y ) \error(\wh f(x^\prime), y^\prime ) - \error(\wh f(x), y ) \error(f_{(i)}(x_i), y_i )}}_{\RN{1}} \\
%         & + \underbrace{\Expt{ S^\prime }{  \error(f_{(i)}(x_i), y_i )  \error(f_{(j)}(x_j), y_j ) -  \error(\wh f(x), y ) \error(f_{(i)}(x_i), y_i )}}_{\RN{2}} \\ &+ \underbrace{\frac{1}{m_1} \Expt{ S^\prime }{  \error(f_{(i)}(x_i), y_i ) - \error(f_{(i)}(x_i), y_i )  \error(f_{(j)}(x_j), y_j ) }}_{\RN{3}} \,. \numberthis \label{eq:main_reln2}
%     \end{align*}
    
%     We will now bound term $\RN{3}$.  Using Schwarz's inequality, we have
    
%     \begin{align}
%         \Expt{ S^\prime }{  \error(f_{(i)}(x_i), y_i ) - \error(f_{(i)}(x_i), y_i )  \error(f_{(j)}(x_j), y_j ) }^2 &\le  \Expt{ S^\prime }{  \error(f_{(i)}(x_i), y_i ) }^2 \Expt{S^\prime}{1 -   \error(f_{(j)}(x_j), y_j ) }^2 \\
%         &\le \frac{1}{4} \label{eq:term1_lem12}
%     \end{align}
    
%     Note that since $(x_i,y_i)$, $(x_j ,y_j )$, $(x,y)$, and $(x^\prime, y^\prime)$ are all from same distribution $\calDm$, we directly incorporate the bounds on term $\RN{1}$ and $\RN{2}$ from proof of Lemma 9 in \citet{bousquet2002stability}. Combining that with \eqref{eq:term1_lem12} and our definition of hypothesis stability in \codref{cond:hypothesis_stability}, we have the required claim. 
    
    
%     % We now re-write term $\RN{1}$ as
%     % \begin{align*}
%     %         &\Expt{S^\prime, (x,y), (x^\prime,y^\prime) }{ \error(\wh f(x), y ) \error(\wh f(x^\prime), y^\prime ) - \error(\wh f(x), y ) \error(f_{(i)}(x_i), y_i )} \\ & \qquad = \Expt{ S^\prime, (x,y), (x^\prime,y^\prime) }{ \error(\wh f(x), y ) \error(\wh f  (x^\prime), y^\prime ) - \error(\wh f ^\prime(x), y ) \error(f_{(i)}(x^\prime), y^\prime )} \tag{Exchanging $(x_i, y_i)$ with $(x^\prime, y^\prime)$ in the second term} \\
%     %         & \qquad = \Expt{ S^\prime, (x,y), (x^\prime,y^\prime) }{  \left(\error(\wh f(x), y )-  \error(f_{(i)}(x), y ) \right) \error(\wh f  (x^\prime), y^\prime )  } \\
%     %         & \qquad  + \Expt{ S^\prime, (x,y), (x^\prime,y^\prime) }{  \left(\error(f_{(i)}(x), y ) -\error(\wh f ^\prime(x), y ) \right) \error(\wh f  (x^\prime), y^\prime )}  \\
%     %         & \qquad +\Expt{ S^\prime, (x,y), (x^\prime,y^\prime) }{  \left( \error(\wh f  (x^\prime), y^\prime ) -  \error(f_{(i)}(x^\prime), y^\prime ) \right) \error(\wh f ^\prime(x), y ) }  \,, \numberthis \label{eq:term1_final}
%     % \end{align*}
%     % where $\wh f^\prime$ is the classifier obtained by training on $ S^\prime_{(i)} \cup \{ (x^\prime, y^\prime) \} $. Similarly we can re-write term $\RN{2}$ as 
%     % \begin{align*}
%     %     & \Expt{ S^\prime }{  \error(f_{(i)}(x_i), y_i )  \error(f_{(j)}(x_j), y_j ) -  \error(\wh f(x), y ) \error(f_{(i)}(x_i), y_i )} \\
%     %     &\quad  = \Expt{ S^\prime, (x,y), (x^\prime,y^\prime)}{  \error(f^{\prime\prime}_{(i)}(x), y )  \error(f_{(j)}^{\prime}(x^\prime), y^\prime ) -  \error(\wh f(x), y ) \error(f_{(i)}(x_i), y_i )} \tag{Exchanging $(x_i, y_i)$ with $(x, y)$ and $(x_j, y_j)$ with $(x^\prime, y^\prime)$ in the first term}\\
%     %     &\quad = \Expt{ S^\prime, (x,y), (x^\prime,y^\prime)}{  \error(f^{\prime\prime}_{(j)}(x), y )  \error(f_{(i)}^{\prime}(x^\prime), y^\prime ) -  \error(\wh f^\prime (x), y ) \error(f^\prime_{(j)}(x^\prime), y^\prime )} \tag{Exchanging $(x_i, y_i)$ and $(x_j, y_j)$ and then replacing $(x_j, y_j)$ with $(x^\prime, y^\prime)$ in the second term} \\
%     %     & \quad = \Expt{ S^\prime, (x,y), (x^\prime,y^\prime) }{  \left( \error(f_{(i)}^{\prime}(x^\prime), y^\prime )   -  \error(\wh f^{\prime\prime}  (x^\prime), y^\prime ) \right)  \error(f^{\prime\prime}_{(j)}(x), y )   } \\
%     %     & \quad  + \Expt{ S^\prime, (x,y), (x^\prime,y^\prime) }{  \left( \error(f^{\prime\prime}_{(j)}(x), y )  -\error(\wh f ^\prime(x), y ) \right) \error(\wh f^{\prime\prime}  (x^\prime), y^\prime )  }  \\
%     %     & \quad+ \Expt{ S^\prime, (x,y), (x^\prime,y^\prime) }{  \left( \error(\wh f^{\prime\prime}  (x^\prime), y^\prime )  -  \error(f^\prime_{(j)}(x^\prime), y^\prime ) \right)  \error(\wh f^\prime (x), y ) }   \\
%     %     & \quad = \Expt{ S^\prime, (x,y), (x^\prime,y^\prime) }{  \left( \error(f_{(i)}^{\prime}(x^\prime), y^\prime )   -  \error(\wh f (x^\prime), y^\prime ) \right)  \error(f_{(i)}(x_j), y_j )   } \\
%     %     & \quad  + \Expt{ S^\prime, (x,y), (x^\prime,y^\prime) }{  \left( \error(f^{\prime\prime}_{(j)}(x), y )  -\error(\wh f (x), y ) \right) \error(\wh f^{\prime\prime}  (x_j), y_j )  }  \\
%     %     & \quad+ \Expt{ S^\prime, (x,y), (x^\prime,y^\prime) }{  \left( \error(\wh f^{\prime\prime}  (x^\prime), y^\prime )  -  \error(f^\prime_{(j)}(x^\prime), y^\prime ) \right)  \error(\wh f^\prime (x^\prime), y^\prime ) }  \,, \numberthis \label{eq:term2_final}
%     % \end{align*}
%     % where $f^{\prime\prime}_{(j)}$ is trained on $S^\prime_{(j,i)} \cup {(x,y)}$, $f^{\prime}_{(i)}$ is trained on $S^\prime_{(j,i)} \cup {(x^\prime,y^\prime)}$, and $\wh f^{\prime\prime} $ is trained on $S^\prime_{(j)} \cup {(x,y)}$. Note in the last line we replaced $(x,y)$ by $(x_j, y_j)$ in the first term, replaced $(x^\prime,y^\prime)$ by $(x_j, y_j)$ in the second term and exchanged $(x_i,y_i)$ with $(x_j,y_j)$ and also $(x,y)$ and $(x^\prime, y^\prime)$
    
    
% \end{proof}


%%%%%%%%%%%%%%%%%%%%%%%%%%%%%%%%%%%%%%%%%%%%%%%%%%


% Don't change these lines
\bsp	% typesetting comment
\label{lastpage}
\end{document}

% End of mnras_template.tex
