% LaTeX template for creating an MNRAS paper
%
% v3.0 released 14 May 2015
% (version numbers match those of mnras.cls)
%
% Copyright (C) Royal Astronomical Society 2015
% Authors:
% Keith T. Smith (Royal Astronomical Society)

% Change log
%
% v3.0 May 2015
%    Renamed to match the new package name
%    Version number matches mnras.cls
%    A few minor tweaks to wording
% v1.0 September 2013
%    Beta testing only - never publicly released
%    First version: a simple (ish) template for creating an MNRAS paper

%%%%%%%%%%%%%%%%%%%%%%%%%%%%%%%%%%%%%%%%%%%%%%%%%%
% Basic setup. Most papers should leave these options alone.
\documentclass[fleqn,usenatbib]{mnras}
\usepackage{newtxtext,newtxmath}
\usepackage[T1]{fontenc}

\DeclareRobustCommand{\VAN}[3]{#2}
\let\VANthebibliography\thebibliography
\def\thebibliography{\DeclareRobustCommand{\VAN}[3]{##3}\VANthebibliography}


%%%%% AUTHORS - PLACE YOUR OWN PACKAGES HERE %%%%%
\usepackage{graphicx}	% Including figure files
\usepackage{amsmath}	% Advanced maths commands
% \usepackage{amssymb}	% Extra maths symbols
\usepackage[dvipsnames,svgnames]{xcolor}
%%%%%%%%%%%%%%%%%%%%%%%%%%%%%%%%%%%%%%%%%%%%%%%%%%

%%%%% AUTHORS - PLACE YOUR OWN COMMANDS HERE %%%%%
\usepackage{color}
\newcommand{\nianyi}[1]{\textcolor{RoyalBlue}{[{\bf Nianyi}: #1]}}
\newcommand{\yueying}[1]{\textcolor{Bittersweet}{{\bf Yueying}: #1}}
\newcommand{\mjt}[1]{\textcolor{DarkOrchid}{{\bf Michael}: #1}}
\newcommand{\spb}[1]{\textcolor{ForestGreen}{{\bf Simeon}: #1}}
\newcommand{\tiziana}[1]{\textcolor{red}{{\bf Tiziana}: #1}}
\newcommand{\cjd}[1]{\textcolor{brown}{{\bf Colin}: #1}}

% Please keep new commands to a minimum, and use \newcommand not \def to avoid
% overwriting existing commands. Example:
%\newcommand{\pcm}{\,cm$^{-2}$}	% per cm-squared

%%%%%%%%%%%%%%%%%%%%%%%%%%%%%%%%%%%%%%%%%%%%%%%%%%

%%%%%%%%%%%%%%%%%%% TITLE PAGE %%%%%%%%%%%%%%%%%%%

% Title of the paper, and the short title which is used in the headers.
% Keep the title short and informative.
\title[BH Dynamics and Mergers]{Dynamical Friction Modeling of Massive Black Holes in Cosmological Simulations and Effects on Merger Rate Predictions}

% The list of authors, and the short list which is used in the headers.
\author[N.Chen et al.]{
Nianyi Chen,$^{1}$\thanks{E-mail: nianyic@andrew.cmu.edu}
Yueying Ni,$^{1}$
Michael Tremmel,$^{2}$
Tiziana Di Matteo,$^{1,3,4}$ Simeon Bird, $^5$
Colin DeGraf,$^{1}$
\newauthor
 Yu Feng $^6$
\\
% List of institutions
$^{1}$McWilliams Center for Cosmology, Department of Physics, Carnegie Mellon University, Pittsburgh, PA 15213, USA\\
$^{2}$Astronomy Department, Yale University, P.O. Box 208120, New Haven, CT 06520, USA\\
$^3$NSF AI Planning Institute for Physics of the Future, 
Carnegie   Mellon  University, Pittsburgh, PA 15213, USA \\
$^{4}$ OzGrav-Melbourne, Australian Research Council Centre of Excellence for Gravitational Wave Discovery\\
$^{5}$ Department of Physics and Astronomy, University of California Riverside, Riverside, CA 90217, USA\\
$^{6}$Berkeley Center for Cosmological Physics and Department of Physics, University of California, Berkeley, CA 94720, USA
}

% These dates will be filled out by the publisher
\date{Accepted XXX. Received YYY; in original form ZZZ}

% Enter the current year, for the copyright statements etc.
\pubyear{2021}

% Don't change these lines
\begin{document}
\label{firstpage}
\pagerange{\pageref{firstpage}--\pageref{lastpage}}
\maketitle

% Abstract of the paper
\begin{abstract}
In this work we establish and test methods for implementing dynamical friction for massive black hole pairs that form in large volume cosmological hydrodynamical simulations which include galaxy formation and black hole growth. We verify our models and parameters both for individual black hole dynamics and for the black hole population in cosmological volumes. Using our model of dynamical friction (DF) from collisionless particles, black holes can effectively sink close to the galaxy center, provided that the black hole's dynamical mass is at least twice that of the lowest mass resolution particles in the simulation. Gas drag also plays a role in assisting the black holes' orbital decay, but it is typically less effective than that from collisionless particles, especially after the first billion years of the black hole's evolution. DF from gas becomes less than $1\%$ of DF from collisionless particles for BH masses $> 10^{7}$ M$_{\odot}$.
Using our best DF model, we calculate the merger rate down to $z=1.1$ using an $L_{\rm box}=35$ Mpc$/h$ simulation box.  We predict $\sim 2$ mergers per year for $z>1.1$ peaking at $z\sim 2$.
These merger rates are within the range obtained in previous work using similar-resolution hydro-dynamical simulations. We show that the rate is enhanced by factor of $\sim 2$ when DF is taken into account in the simulations compared to the no-DF run. This is due to $>40\%$ more black holes reaching the center of their host halo when DF is added.


\end{abstract}
% Select between one and six entries from the list of approved keywords.
% Don't make up new ones.
\begin{keywords}
gravitational waves -- methods: numerical -- quasars: supermassive black holes.
\end{keywords}

%%%%%%%%%%%%%%%%%%%%%%%%%%%%%%%%%%%%%%%%%%%%%%%%%%

%%%%%%%%%%%%%%%%% BODY OF PAPER %%%%%%%%%%%%%%%%%%


% \leavevmode
% \\
% \\
% \\
% \\
% \\
\section{Introduction}
\label{introduction}

AutoML is the process by which machine learning models are built automatically for a new dataset. Given a dataset, AutoML systems perform a search over valid data transformations and learners, along with hyper-parameter optimization for each learner~\cite{VolcanoML}. Choosing the transformations and learners over which to search is our focus.
A significant number of systems mine from prior runs of pipelines over a set of datasets to choose transformers and learners that are effective with different types of datasets (e.g. \cite{NEURIPS2018_b59a51a3}, \cite{10.14778/3415478.3415542}, \cite{autosklearn}). Thus, they build a database by actually running different pipelines with a diverse set of datasets to estimate the accuracy of potential pipelines. Hence, they can be used to effectively reduce the search space. A new dataset, based on a set of features (meta-features) is then matched to this database to find the most plausible candidates for both learner selection and hyper-parameter tuning. This process of choosing starting points in the search space is called meta-learning for the cold start problem.  

Other meta-learning approaches include mining existing data science code and their associated datasets to learn from human expertise. The AL~\cite{al} system mined existing Kaggle notebooks using dynamic analysis, i.e., actually running the scripts, and showed that such a system has promise.  However, this meta-learning approach does not scale because it is onerous to execute a large number of pipeline scripts on datasets, preprocessing datasets is never trivial, and older scripts cease to run at all as software evolves. It is not surprising that AL therefore performed dynamic analysis on just nine datasets.

Our system, {\sysname}, provides a scalable meta-learning approach to leverage human expertise, using static analysis to mine pipelines from large repositories of scripts. Static analysis has the advantage of scaling to thousands or millions of scripts \cite{graph4code} easily, but lacks the performance data gathered by dynamic analysis. The {\sysname} meta-learning approach guides the learning process by a scalable dataset similarity search, based on dataset embeddings, to find the most similar datasets and the semantics of ML pipelines applied on them.  Many existing systems, such as Auto-Sklearn \cite{autosklearn} and AL \cite{al}, compute a set of meta-features for each dataset. We developed a deep neural network model to generate embeddings at the granularity of a dataset, e.g., a table or CSV file, to capture similarity at the level of an entire dataset rather than relying on a set of meta-features.
 
Because we use static analysis to capture the semantics of the meta-learning process, we have no mechanism to choose the \textbf{best} pipeline from many seen pipelines, unlike the dynamic execution case where one can rely on runtime to choose the best performing pipeline.  Observing that pipelines are basically workflow graphs, we use graph generator neural models to succinctly capture the statically-observed pipelines for a single dataset. In {\sysname}, we formulate learner selection as a graph generation problem to predict optimized pipelines based on pipelines seen in actual notebooks.

%. This formulation enables {\sysname} for effective pruning of the AutoML search space to predict optimized pipelines based on pipelines seen in actual notebooks.}
%We note that increasingly, state-of-the-art performance in AutoML systems is being generated by more complex pipelines such as Directed Acyclic Graphs (DAGs) \cite{piper} rather than the linear pipelines used in earlier systems.  
 
{\sysname} does learner and transformation selection, and hence is a component of an AutoML systems. To evaluate this component, we integrated it into two existing AutoML systems, FLAML \cite{flaml} and Auto-Sklearn \cite{autosklearn}.  
% We evaluate each system with and without {\sysname}.  
We chose FLAML because it does not yet have any meta-learning component for the cold start problem and instead allows user selection of learners and transformers. The authors of FLAML explicitly pointed to the fact that FLAML might benefit from a meta-learning component and pointed to it as a possibility for future work. For FLAML, if mining historical pipelines provides an advantage, we should improve its performance. We also picked Auto-Sklearn as it does have a learner selection component based on meta-features, as described earlier~\cite{autosklearn2}. For Auto-Sklearn, we should at least match performance if our static mining of pipelines can match their extensive database. For context, we also compared {\sysname} with the recent VolcanoML~\cite{VolcanoML}, which provides an efficient decomposition and execution strategy for the AutoML search space. In contrast, {\sysname} prunes the search space using our meta-learning model to perform hyperparameter optimization only for the most promising candidates. 

The contributions of this paper are the following:
\begin{itemize}
    \item Section ~\ref{sec:mining} defines a scalable meta-learning approach based on representation learning of mined ML pipeline semantics and datasets for over 100 datasets and ~11K Python scripts.  
    \newline
    \item Sections~\ref{sec:kgpipGen} formulates AutoML pipeline generation as a graph generation problem. {\sysname} predicts efficiently an optimized ML pipeline for an unseen dataset based on our meta-learning model.  To the best of our knowledge, {\sysname} is the first approach to formulate  AutoML pipeline generation in such a way.
    \newline
    \item Section~\ref{sec:eval} presents a comprehensive evaluation using a large collection of 121 datasets from major AutoML benchmarks and Kaggle. Our experimental results show that {\sysname} outperforms all existing AutoML systems and achieves state-of-the-art results on the majority of these datasets. {\sysname} significantly improves the performance of both FLAML and Auto-Sklearn in classification and regression tasks. We also outperformed AL in 75 out of 77 datasets and VolcanoML in 75  out of 121 datasets, including 44 datasets used only by VolcanoML~\cite{VolcanoML}.  On average, {\sysname} achieves scores that are statistically better than the means of all other systems. 
\end{itemize}


%This approach does not need to apply cleaning or transformation methods to handle different variances among datasets. Moreover, we do not need to deal with complex analysis, such as dynamic code analysis. Thus, our approach proved to be scalable, as discussed in Sections~\ref{sec:mining}.
\section{The Simulations}
\label{sec:simulations}

\subsection{The Numerical Code}
\label{subsec:code}

We use the massively parallel cosmological smoothed particle hydrodynamic (SPH) simulation software, MP-Gadget \citep{Feng2016}, to run all the simulations in this paper. 
The hydrodynamics solver of MP-Gadget adopts the new pressure-entropy formulation of SPH \citep{Hopkins2013}.
We apply a variety of sub-grid models to model the galaxy and black hole formation and associated feedback processes already validated against a number of observables \citep[e.g.][]{Feng2016,Wilkins2017,Waters2016,DiMatteo2017,Tenneti2018,Huang2018,Ni2018,Bhowmick2018,Marshall2020,Marshall2021}. Here we review briefly the main aspects of these.
In the simulations, gas is allowed to cool through radiative processes~\citep{Katz}, including metal cooling. For metal cooling, we follow the method in \cite{Vogelsberger2014}, and scale a solar metallicity template according to the metallicity of gas particles.
Our star formation (SF) is based on a multi-phase SF model ~\citep{SH03} with modifications following~\cite{Vogelsberger2013}.
We model the formation of molecular hydrogen and its effects on SF at low metallicity according to the prescription of \cite{Krumholtz}. 
We self-consistently estimate the fraction of molecular hydrogen gas from the baryon column density, which in turn couples the density gradient to the SF rate.
We include Type II supernova wind feedback ~\citep[the model used in BlueTides][]{Feng2016,Okamoto2010} in our simulations, assuming that the wind speed is proportional to the local one dimensional dark matter velocity dispersion.

BHs are seeded with an initial seed mass of $M_{\mathrm {seed}} = 5 \times 10^5 M_{\odot}/h$ in halos with mass more than $10^{10} M_{\odot}/h$ if the halo does not already contain a BH. We model BH growth and AGN feedback in the same way as in the \textit{MassiveBlack} $I \& II$ simulations, using the BH sub-grid model developed in \cite{SDH2005,DSH2005} with modifications consistent with BlueTides. 
The gas accretion rate onto the BHs is given by Bondi accretion rate,
\begin{equation}
\label{equation:Bondi}
    \dot{M}_B = \alpha \frac{4 \pi G^2 M_{\rm BH}^2 \rho}{(c^2_s+v_{\rm rel}^2)^{3/2}},
\end{equation}
where $c_s$ and $\rho$ are the local sound speed and density of the cold gas, $v_{\rm rel}$ is the relative velocity of the BH to the nearby gas, and $\alpha=100$ is a numerical correction factor introduced by \citep{Springel2005b}. This can also be eliminated (without affecting the values of the accretion rate significantly) in favor of a more detailed modeling of the contributions in the cold and hot phase accretion \citep{Pelupessy2006}.


We allow for super-Eddington accretion in the simulation \citep[e.g.][]{Volonteri2005,Volonteri2015}, but limit the accretion rate to 2 times the Eddington accretion rate:
\begin{equation}
\label{equation:Meddington}
    \dot{M}_{\rm Edd} = \frac{4 \pi G M_{\rm BH} m_p}{\eta \sigma_{T} c},
\end{equation}
where $m_p$ is the proton mass, $\sigma_T$ the Thompson cross section, c is the speed of light, and $\eta=0.1$ is the radiative efficiency of the accretion flow onto the BH.
Therefore, the BH accretion rate is determined by:
\begin{equation}
    \dot{M}_{\rm BH} = {\rm Min} (\dot{M}_{B}, 2\dot{M}_{\rm Edd}).
\end{equation}


The SMBH is assumed to radiate with a bolometric luminosity $L_{\rm Bol}$ proportional to the accretion rate $\dot{M}_{\rm BH}$:
\begin{equation}
    L_{\rm Bol} = \eta \dot{M}_{\rm BH} c^2
\end{equation}
with $\eta = 0.1$ being the mass-to-light conversion efficiency in an accretion disk according to \cite{Shakura1973}.
5\% of the radiated energy is thermally coupled to the surrounding gas that resides within twice the radius of the SPH smoothing kernel of the BH particle. This scale is typically about $\sim 1-3\%$ of the virial radius of the halo.

The cosmological parameters used are from the nine-year Wilkinson Microwave Anisotropy Probe (WMAP) \citep{Hinshaw2013} ($\Omega_0=0.2814$, $\Omega_\Lambda=0.7186$, $\Omega_b=0.0464$, $\sigma_8=0.82$, $h=0.697$, $n_s=0.971$).
For our fiducial resolution simulations, the mass resolution is $M_{\rm DM} = 1.2 \times 10^7 M_\odot/h$ and $M_{\rm gas} = 2.4 \times 10^6 M_\odot/h$ in the initial conditions.
The mass of a star particle is $M_{*} = 1/4 M_{\rm gas} = 6 \times 10^5 M_\odot/h$. The gravitational softening length is $\epsilon_g = 1.5$ ckpc/$h$ in the fiducial resolution for both DM and gas particles. The detailed simulation and model parameters are listed in Tables \ref{tab:cons} and \ref{tab:norm}. 
\subsection{Gaussian Constrained Realization}
\label{subsec:CR}
% \nianyi{Need inputs from Yueying's paper here}

MBHs at high redshift typically reside in rare density peaks, which are absent in the small uniform box ($\sim 10$ Mpc/$h$) simulations. 
In order to test the dynamics for more massive BHs (with $M_{\rm BH} > 10^8 M_{\odot}$) in our small volume simulation, we apply the Constrained Realization (CR) technique \footnote{\url{https://github.com/yueyingn/gaussianCR}} to impose a relatively high density peak in the initial condition (IC), with peak height $\nu = 4 \sigma_0$ on scale of $R_G = 1$ Mpc/$h$.  

The prescription for the CR technique was first introduced by \cite{Hoffman1991} as an optimal way to construct samples of constrained Gaussian random fields.
This formalism was further elaborated and extended by \cite{vandeWeygaert1996} as a more general type of convolution format constraints.
The CR technique imposes constraints on different characteristics of the linear density field. 
It can specify density peaks in the Gaussian random field with any desired height and shape, providing an efficient way to study rare massive objects with a relatively small box and thus lower computational costs \citep[e.g.][]{Ni2020}.
In this study, we specify a $4 \sigma_0$ density peak in the IC of our $10$ Mpc/$h$ box, boosting the early formation of halos and BHs to study the dynamics of massive BHs. Before applying the peak height constraint, the highest density peak has $\nu = 2.4 \sigma_0$ and the largest BH has mass $<6\times 10^7M_\odot$ at $z=3$ in our fiducial model (\texttt{DF\_4DM\_G} in Table \ref{tab:cons}). After applying the $4 \sigma_0$ constraint, the largest BH has mass $3\times 10^8 M_\odot$ at $z=3$ in the same box.



\begin{table*}
\caption{Constrained Simulations}
\label{tab:cons}
\begin{tabular}{lccccccc}
\hline
Name & Lbox & ${\rm N}_{\rm part}$ & ${\rm M}_{\rm DM}$ & ${\rm M}_{\rm Dyn,seed}$ & $\epsilon_{\rm g}$ & BH Dynamics & Merging Criterion \\
& [$h^{-1}$Mpc] & & [$h^{-1} {\rm M}_\odot$] & [${\rm M}_{\rm DM}$] & [$h^{-1}{\rm kpc}$] & &  \\
\hline
NoDF\_4DM & 10 & $176^3$ & $1.2\times 10^7$ & 4 & 1.5 & gravity & distance\\
NoDF\_4DM\_G & 10 & $176^3$ & $1.2\times 10^7$ & 4 & 1.5 & gravity & distance \& grav.bound\\
DF\_4DM & 10 & $176^3$ & $1.2\times 10^7$ & 4 & 1.5 & gravity+DF & distance\\
Drag\_4DM\_G & 10 & $176^3$ & $1.2\times 10^7$ & 4 & 1.5 & gravity+Drag & distance \& grav.bound\\
DF+Drag\_4DM\_G & 10 & $176^3$ & $1.2\times 10^7$ & 4 & 1.5 & gravity+DF+Drag & distance \& grav.bound\\
DF\_4DM\_G & 10 & $176^3$ & $1.2\times 10^7$ & 4 & 1.5 & gravity+DF & distance \& grav.bound\\
DF\_2DM\_G & 10 & $176^3$ & $1.2\times 10^7$ & 2 & 1.5 & gravity+DF & distance \& grav.bound\\
DF\_1DM\_G & 10 & $176^3$ & $1.2\times 10^7$ & 1 & 1.5 & gravity+DF & distance \& grav.bound\\
DF(T15)\_4DM\_G & 10 & $176^3$ & $1.2\times 10^7$ & 4 & 1.5 & gravity+DF(T15) & distance \& grav.bound\\
DF\_HR\_4DM\_G & 10 & $256^3$ & $4\times 10^6$ & 4 & 1.0 & gravity+DF & distance \& grav.bound\\
DF\_HR\_12DM\_G & 10 & $256^3$ & $4\times 10^6$ & 12 & 1.0 & gravity+DF & distance \& grav.bound\\
\hline
\end{tabular}
\end{table*}

\begin{table*}
\caption{Unconstrained Simulations}
\label{tab:norm}
\begin{tabular}{lccccccc}
\hline
Name & Lbox & ${\rm N}_{\rm part}$ & ${\rm M}_{\rm DM}$ & ${\rm M}_{\rm Dyn,seed}$ & $\epsilon_{\rm g}$ & BH Dynamics & Merging Criterion \\
& [$h^{-1}$Mpc] & & [$h^{-1} {\rm M}_\odot$] & [${\rm M}_{\rm DM}$] & [$h^{-1}{\rm kpc}$] & &  \\
\hline
L15\_Repos\_4DM & 15 & $256^3$ & $1.2\times 10^7$ & 4 & 1.5 & reposition & distance\\
L15\_NoDF\_4DM & 15 & $256^3$ & $1.2\times 10^7$ & 4 & 1.5 & gravity & distance\\
L15\_NoDF\_4DM\_G & 15 & $256^3$ & $1.2\times 10^7$ & 4 & 1.5 & gravity & distance \& grav.bound\\
L15\_DF\_4DM & 15 & $256^3$ & $1.2\times 10^7$ & 4 & 1.5 & gravity+DF & distance\\
L15\_DF\_4DM\_G & 15 & $256^3$ & $1.2\times 10^7$ & 4 & 1.5 & gravity+DF & distance \& grav.bound\\
L15\_DF(T15)\_4DM\_G & 15 & $256^3$ & $1.2\times 10^7$ & 4 & 1.5 & gravity+DF(T15) & distance \& grav.bound\\
L15\_DF+drag\_4DM\_G & 15 & $256^3$ & $1.2\times 10^7$ & 4 & 1.5 & gravity+DF+Drag & distance \& grav.bound\\
L35\_NoDF\_4DM\_G & 35 & $600^3$ & $1.2\times 10^7$ & 4 & 1.5 & gravity+DF & distance \& grav.bound\\
L35\_DF+drag\_4DM\_G & 35 & $600^3$ & $1.2\times 10^7$ & 4 & 1.5 & gravity+DF+drag & distance \& grav.bound\\

\hline
\end{tabular}
\end{table*}
\section{BH Dynamics}
\label{sec:bh_model}
\subsection{BH Dynamical Mass}
\label{subsec:mdyn}
In our simulations, the seed mass of the black holes is $5\times 10^5 M_\odot/h$, which is 20 times smaller than the fiducial dark matter particle mass at $1.2\times 10^7 M_\odot/h$. Such a small mass of the BH relative to the dark matter particles will result in very noisy gravitational acceleration on the black holes, and causes instability in the black hole's motion as well as drift from the halo center. Moreover, as shown in previous works \citep[e.g.][]{Tremmel2015,Pfister2019}, under the low $M_{\rm BH}/M_{\rm DM}$ regime, it is challenging to effectively model dynamical friction in a sub-grid fashion.

To alleviate dynamical heating by the noisy potential due to the low $M_{\rm BH}/M_{\rm DM}$ ratio, we introduce a second mass tracer, the dynamical mass $M_{\rm dyn}$, which is set to be comparable to $M_{\rm DM}$ when the black hole is seeded. This mass is used in force calculation for the black holes, including the gravitational force and dynamical friction, while the intrinsic black hole mass $M_{\rm BH}$ is used in the accretion and feedback process. $M_{\rm dyn}$ is kept at its seeding value $M_{\rm dyn,seed}$ until $M_{\rm BH}>M_{\rm dyn,seed}$. After that $M_{\rm dyn}$ grows following the black hole's mass accretion. With the boost in the seed dynamical mass, the sinking time scale will be shortened by a factor of $\sim M_{\rm BH}/M_{\rm dyn}$ compared to the no-boost case. Note that the bare black hole sinking time scale estimated in the no-boost case could over-estimate the true sinking time, as the high-density stellar bulges sinking together with the black hole are not fully resolved \citep[e.g.][]{Antonini2012,Dosopoulou2017,Biernacki2017}.

 The boost we need to prevent dynamical heating depends on the dark matter particle mass $M_{\rm DM}$ (if we have high enough resolution the boost is no longer necessary), so we parametrize the dynamical mass in terms of the dark matter particle mass, $M_{\rm dyn,seed} = k_{\rm dyn} M_{\rm DM}$, instead of setting an absolute seeding dynamical mass for all simulations. We expect that as we go to higher resolutions where $M_{\rm DM}$ is comparable to $M_{\rm BH,seed}$, the dynamical seed mass should converge to the black hole seed mass, if we keep $k_{\rm dyn}$ constant. We study the effect of setting different $k_{\rm dyn}$ by running three simulations with the same resolution and dynamical friction models, but various $k_{\rm dyn}$ ratios. They are listed in Table \ref{tab:cons} as \texttt{DF\_4DM\_G}, \texttt{DF\_2DM\_G}, and \texttt{DF\_1DM\_G}, with $k_{\rm dyn}=4,2,1$, respectively.

To explore the effects of the BH seed dynamical mass on the motion and mergers of the black hole, we test a variety of $M_{\rm dyn,seed}$ values in our simulations. The comparison between different $M_{\rm dyn,seed}$ can be found in Appendix \ref{app:res}. 
\subsection{Modeling of Black Hole Dynamics}
\label{subsec:df}
\subsubsection{Reposition of the Black Hole}

Before introducing our dynamical friction implementations, we first describe a baseline model utilized by many large-volume cosmological simulations: the reposition model. As the name suggests, the reposition model of black hole dynamics places the black hole at the location of a local gas particle with minimum gravitational potential at each time step, in order to avoid the unrealistic motion of the black holes due to limited mass and force resolution. This is particularly preferred for large-volume, low-resolution cosmological simulations \citep[e.g.][]{Springel2005b, Sijacki2007, Booth2009,Schaye2015,Pillepich2018}, where the black hole mass is smaller than a star or gas particle mass and the BH can be inappropriately scattered around by two-body forces as well as the noisy local potential.

This simple fix of repositioning, however, comes with many disadvantages. For example, it may lead to higher accretion and feedback of the black holes, as they sink to the high-density regions too quickly. As was shown in \cite{Wurster2013} and \cite{Tremmel2017}, repositioning also leads to burstier feedback of the BHs, which is more likely to quench star-formation in the host galaxies. Moreover, repositioning leads to ill-defined velocity and non-smooth trajectories of the black hole particles. Because of the ill-defined velocity and extremely short orbital decay time, such methods cannot be reliably used for merger rate predictions without careful post-processing calculations to account for the orbital decays.

In our study, we use the reposition model as a reference for the black hole statistics, as it is still widely adopted in many existing simulations. We want to compare the dynamical friction models with the reposition model and quantify the effect of repositioning on BH mass growth and merger rate compared with the dynamical friction models.


\subsubsection{Dynamical Friction from Collisionless Particles}

When the black hole travels through a continuous medium or a medium consisting of particles with smaller masses than the black hole, it attracts the surrounding mass towards itself, leaving a tail of overdensity behind.  Dynamical friction is the resulting gravitational force exerted onto the black hole by this tail of overdensity \citep[e.g.][]{Chandrasekhar1943,Binney2008}. Dynamical friction causes the orbits of SMBHs to decay towards the center of massive galaxies \citep[e.g.][]{Governato1994,Kazantzidis2005}, and enables the black holes to stay at the high-density regions where they could go through efficient accretion and mergers.

We follow Equation (8.3) in \cite{Binney2008} for the acceleration of the black hole due to dynamical friction:

\begin{equation}
\label{eq:df_full}
    \mathbf{F}_{\rm DF} = -16\pi^2 G^2 M_{\rm BH}^2 m_{a} \;\text{log}(\Lambda) \frac{\mathbf{v}_{\rm BH}}{v_{\rm BH}^3} \int_0^{v_{\rm BH}} dv_a v_a^2 f(v_a),
\end{equation}
where $M_{\rm BH}$ is the black hole mass, $\textbf{v}_{\rm BH}$ is the velocity of the black hole relative to its surrounding medium, $m_a$ and $v_a$ are the masses and velocities of the particles surrounding the black hole, and $\text{log}(\Lambda)=\text{log}(b_{\rm max}/b_{\rm min})$ is the Coulomb logarithm that accounts for the effective range of the friction between $b_{\rm min}$ and $b_{\rm max}$(we will specify how we set these parameters later). $f(v_a)$ is the velocity distribution of the surrounding particles (unless we explicitly state otherwise, all variables involving the black hole's surrounding particles are calculated using stars and dark matter particles). Here we have assumed an isotropic velocity distribution of the particles surrounding the black hole, so that we are left with an 1D integration. 

We test two different numerical implementations of the dynamical friction (DF) in our simulations: one with a more aggressive approach which likely overestimates the effective range of DF, but could be more suitable for large-volume simulations (we refer to it as DF(fid) in places where we carry out explicit comparisons between the two DF models, and drop the 'fid' in all other places); the other with a more conservative method which aims to only account for the DF below the gravitational softening length, and is well-tested for smaller volume, high-resolution simulations \citep{Tremmel2015} (we refer to it as DF(T15)).

We begin by introducing the DF(fid) model. In this model, we further follow the derivation in \cite{Binney2008}, and approximate $f(v_a)$ by the Maxwellian distribution, so that Equation \ref{eq:df_full} reduces to:
\begin{equation}
    \label{eq:H14}
    \mathbf{F}_{\rm DF,fid} = -4\pi \rho_{\rm sph} \left(\frac{GM_{\rm dyn}}{v_{\rm BH}}\right)^2  \;\text{log}(\Lambda_{\rm fid}) \mathcal{F}\left(\frac{v_{\rm BH}}{\sigma_v}\right) \frac{\bf{v}_{\rm BH}}{v_{\rm BH}}.
\end{equation}
Here $\rho_{\rm sph}$ is the density of dark matter and star particles within the SPH kernel (we will sometimes refer to these particles as "surrounding particles") of the black hole. All other definitions follow those of Equation \ref{eq:df_full}, except that we have substituted $M_{\rm BH}$ with $M_{\rm dyn}$ following the discussion in \ref{subsec:mdyn}.
The function $\mathcal{F}$ defined as:
\begin{equation}
    \label{eq:fx}
    \mathcal{F}(x) =  \text{erf}(x)-\frac{2x}{\sqrt{\pi}} e^{-x^2}, \;
    x=\frac{v_{\rm BH}}{\sigma_v}
\end{equation}
is the result of analytically integrating the Maxwellian distribution, where $\sigma_v$ is the velocity dispersion of the surrounding particles.

The subscript "fid" in $\text{log}(\Lambda)$ means that this definition of $\Lambda$ is specific to the DF(fid) model, with
\begin{equation}
    \Lambda_{\rm fid} = \frac{b_{\rm max,fid}}{(GM_{\rm dyn})/v_{\rm BH}^2}, \; b_{\rm max,fid} = 10\text{ ckpc}/h.
\end{equation}
Note that here we have defined $b_{\rm max}$ as a constant roughly equal to 6 times the gravitational softening. As there is no general agreement on the distance above which dynamical friction is fully resolved, we tested several values ranging from $\epsilon_g$ to $20\epsilon_g$. We found that values above $2\epsilon_g$ are effective in sinking the black hole, although a smaller $b_{\rm max}$ tends to result in more drifting black holes at higher redshift. By using this definition, we are likely overestimating the effective range of dynamical friction. However, we find this over-estimation necessary in the early stage of black hole growth to stabilize the black hole motion.

We also implement a more localized version of dynamical friction following  \cite{Tremmel2015} which we call DF(T15). Under the DF(T15) model, the dynamical friction is expressed as:

\begin{equation}
    \label{eq:T15}
    \mathbf{F}_{\rm DF,T15} = -4\pi \rho (v<v_{\rm BH}) \left(\frac{GM_{\rm dyn}}{v_{\rm BH}}\right)^2  \text{log}(\Lambda_{\rm T15}) \frac{\bf{v}_{\rm BH}}{v_{\rm BH}}.
\end{equation}
Here the surrounding density only accounts for the particles moving slower than the BH with respect to the environment. More formally,
\begin{equation}
\label{eq:rho}
    \rho (v<v_{\rm BH}) = \frac{M(<v_{\rm BH})}{M_{\rm total}} \rho_{\rm T15},
\end{equation}
where $M_{\rm total}$ is the total mass of the nearest 100 DM and stars, $M(<v_{\rm BH})$ is the fractional mass counting only DM and star particles with velocities smaller than the BH, and $\rho_{T15}$ is the density calculated from the nearest 100 DM/Star particles (note that in comparison, the SPH kernel contains 113 gas particles but far more collisionless particles (see Figure \ref{fig:k100_case1})). By using $\rho (v<v_{\rm BH})$ in place of $\rho_{\rm sph} \mathcal{F}$, we are approximating the velocity distribution of surrounding particles by the distribution of the nearest 100 collisionless particles. Another major difference from the DFsph model is the Coulomb logarithm, where in this model we define:
\begin{equation}
    \Lambda_{\rm T15} = \frac{b_{\rm max,T15}}{(GM_{\rm dyn})/v_{\rm BH}^2}, \; b_{\rm max,T15} = \epsilon_g.
\end{equation}
The choice of a lower $b_{\rm max}$ is consistent with the localized density and velocity calculations, and by doing so we have assumed that dynamical friction is fully resolved above the gravitational softening.


\subsubsection{Gas Drag}
\label{subsection:drag}
In addition to the dynamical friction from dark matter and stars, the black hole can also lose its orbital energy due to the dynamical friction from gas (to distinguish from dynamical friction from dark matter and stars, we will refer to the gas dynamical friction as "gas drag" hereafter). \cite{Ostriker1999} first came up with the analytical expression for the gas drag term from linear perturbation theory, and showed that in the transonic regime the gas drag can be more effective than the dynamical friction from collisionless particles. Although later studies show that \cite{Ostriker1999} likely overestimates the gas drag for gas with Mach numbers slightly above unity \citep[e.g.][]{Escala2004ApJ,Chapon2013}, simulations with gas drag implemented still demonstrate that this is an effective channel for black hole energy loss during orbital decays \citep[e.g.][]{Chapon2013,Dubois2013,Pfister2019}.

In order to investigate the relative effectiveness of DF and gas drag, we also include gas drag onto black holes in our simulations following the analytical approximation from \cite{Ostriker1999}:
\begin{equation}
\label{eq:drag}
    \mathbf{F}_{\rm drag} = -4 \pi\rho \left( \frac{G M_{\rm dyn}}{c_s^2} \right)^2 \times \mathcal{I(M)}\frac{\bf{v}_{\rm BH}}{v_{\rm BH}},
\end{equation}
where $c_s$ is the sound speed, $\mathcal{M} = \frac{| \mathbf{v}_{\rm BH} - \mathbf{v}_{\rm gas}|}{c_s}$ is the Mach number, and $\mathcal{I(M)}$ is given by:
\begin{align}
    \mathcal{I}_{\rm subsonic} &= \mathcal{M}^{-2} \left[ \frac{1}{2} \text{log}\left(\frac{1+\mathcal{M}}{1-\mathcal{M}}\right) -\mathcal{M}\right] \\
    \mathcal{I}_{\rm supersonic} &= \mathcal{M}^{-2} \left[ \frac{1}{2} \text{log}\left(\frac{\mathcal{M}+1}{\mathcal{M}-1}\right) -\text{log} \Lambda_{\rm fid} \right],
\end{align}
where $\text{log} \Lambda_{\rm fid}$ is the Coulomb logarithm defined similarly to the collisionless dynamical friction.





\begin{figure*}
\includegraphics[width=0.49\textwidth]{RESULTS/plots/dftest_L10_adv0df2b30mg1_dm5e7-010.png}
\includegraphics[width=0.49\textwidth]{RESULTS/plots/dftest_L10_adv0df2b30mg1_dm5e7-011.png}

\caption{Visualization of $4\sigma_0$ density peak of the \texttt{DF\_4\_DM\_G} simulation at $z=4.0$ and $z=3.5$. The brightness corresponds to the gas density, and the warmness of the tone indicates the mass-weighted temperature of the gas. We plot the black holes (\textbf{cross}) with mass $>10^6 M_\odot$, as well as the halos (subhalos) hosting them (\textbf{red circles} correspond to central halos, \textbf{orange circles} correspond to subhalos. The circle radius shows the virial radius of the halo; halos are identified by Amiga's Halo Finder(AHF)). This density peak hosts the two largest black holes in our simulations (\textbf{yellow cross}), and they are going through a merger along with the merger of their host halos between $z=4$ and $z=3$. For the black hole and merger case studies, we will use examples from the circled halos/black holes shown in this figure.} 
\label{fig:halos}
\end{figure*}


\subsection{Merging Criterion}
\label{subsec:merger}
In all of our simulations, we set the merging distance to be $2\epsilon_{\rm g}$, because the BH dynamics below this distance is not well-resolved due to our limited spatial resolution. We conserve the total momentum of the binary during the merger.

Under the baseline repositioning treatment of the BH dynamics, the velocity of the black hole is not a well-defined quantity. Therefore, in cosmological simulations with repositioning, the distance between the two black holes is often the only criterion imposed during the time of mergers (for example BlueTides \citep{Feng2016}, Illustris \citep{Vogelsberger2013} and IllustrisTNG \citep{Pillepich2018}). One problem with using only the distance as a merging criterion is that it can spuriously merge two passing-by black holes with high velocities, when in reality they are not gravitationally bound and should not merge just yet (or may never merge). Although some similar-resolution simulations such as EAGLE \citep{Crain2015,Schaye2015} also check whether two black hole particles are gravitationally bound, the black holes still do not have a well-defined orbit and sinking time due to the discrete positioning.

When we turn off the repositioning of the BHs to the nearby minimum potential, the BHs will have well-defined velocities at each time step (this is true whether or not we add the dynamical friction). This allows us to apply further merging criteria based on the velocities and accelerations of the black hole pair, and thus avoid earlier mergers of the gravitationally unbound pairs. Also, as the BH pairs now have well-defined orbits all the way down to the numerical merger time, we will be able to directly measure binary separation and eccentricity from the numerical merger, and use the measurements as the initial condition for post-processing methods without having to assume a constant initial value \citep[e.g.][]{Kelley2017}.

We follow \cite{Bellovary2011} and \cite{Tremmel2017}, and use the criterion
\begin{equation}
    \label{eq:merge_criterion}
    \frac{1}{2}|\bf{\Delta v}|^2 < \bf{\Delta a} \bf{\Delta r}
\end{equation}
 to check whether two black holes are gravitationally bound. Here $\bf{\Delta a}$,$\bf{\Delta v}$ and $\bf{\Delta r}$ denote the relative acceleration, velocity and position of the black hole pair, respectively. Note that this expression is not strictly the total energy of the black hole pair, but an approximation of the kinetic energy and the work needed to get the black holes to merge. Because in the simulations the black hole is constantly interacting with surrounding particles, on the right-hand side we use the overall gravitational acceleration instead of the acceleration purely from the two-body interaction.

\section{Case Studies of BH Models}
\label{sec:case}

%%%%%%%%%%%%%%%%%%%%%%%%%%%%
\begin{figure*}
\includegraphics[width=0.91\textwidth]{RESULTS/plots/big_plot2.pdf}

\caption{ The evolution of BH2 in Figure \ref{fig:halos} under different BH dynamics prescriptions. We show the distance to halo center (\textbf{top}), black hole mass (\textbf{middel}) and the $x$-component of the black hole velocity (\textbf{bottom}). Mergers are shown in vertical lines (thick dashed lines are major mergers ($q>0.3$), and thin dotted lines are minor mergers) \textbf{(a):} comparison between no-DF and DF models. DF clearly helps the black hole sink to the halo center and stay there. \textbf{(b):} Effects of DF from stars and dark matter compared with gas drag. DF has a stronger effect throughout, except that in the very early stage the drag-only model is comparable to the DF-only model. \textbf{(c)}: Comparison between the DF(fid) and DF(T15) model. In general, the DF(fid) model results in a more stable black hole motion and faster sinking, but the difference is small. \textbf{(d)}: Black hole dynamics with and without the gravitational bound check during mergers. Without the gravitational bound check, the black holes can merge while still moving with large momenta, and thereby get kicked out of the halo by the injected momentum.}
\label{fig:big_plot}
\end{figure*}

%%%%%%%%%%%%%%%%%%%%%%%%%%%%



Given the variety of models we have described so far, we first study the effect of different BH dynamics models by looking at the individual black hole evolution and black hole pairs using the constrained simulations. The details of these simulations and specific dynamical models are shown in Table \ref{tab:cons}. For all the constrained simulations, we use the same initial conditions, which enables us to do a case-by-case comparison between different BH dynamical models.

For the case studies, we choose to study the growth and merger histories of the two largest black holes and a few surrounding black holes within the density peak of our simulations. The halos and black holes at the $4\sigma_0$ density peak in \texttt{DF\_4DM\_G} are shown in Figure \ref{fig:halos}. The halos and subhalos shown in circles are identified with Amiga's Halo Finder \citep[AHF,][]{Knollmann2009}. The halos are centered at the minimum-potential gas particle within the halo, and the sizes of the circles correspond to the virial radius of the halo. Throughout the paper, we will always define the halo centers by the position of the minimum-potential gas particle, and we note that the offset between the minimum-potential gas and the halo center given by AHF (found via density peaks) is always less than 1.5 ckpc$/h$. The cyan crosses are black holes with mass larger than $10^6 M_\odot/h$, and the yellow crosses are the two largest black holes in the simulation. From the plot, we can see that in the \texttt{DF\_4DM\_G} simulation, most of the black holes already reside in the center of their hosting halos at $z=4$, although we also see some cases of wandering BHs outside of the halos.

%%%%%%%%%%%%%%%%%%%%%%%%%%%%%%%%
\subsection{Black Hole Dynamics Modeling}
\label{subsec:models}

To compare different dynamical models, we look at the distance between the black hole and the halo center $\Delta r_{\rm BH}$ (we will sometimes refer to this distance as "drift" hereafter), the black hole mass, and the velocity along the $x$ direction through the entire history of BH2 from Figure \ref{fig:halos}. 
 
 We evaluate the black hole drift with two approaches: at each time-step, we find the minimum potential gas particle within 10 ckpc$/h$ of the black hole and calculate the distance between this gas particle and the black hole. This is a quick evaluation of the drift that allows us to trace the black hole motion at each time step, but it fails to account for orbits larger than 10 ckpc$/h$, and the minimum-potential gas particle may not reside in the same halo as the black hole. Therefore, for each snapshot we saved, we define the drift more carefully by running the halo finder and calculate the distance between the black hole and the center of its host halo. Whenever the black hole is further than 9 ckpc$/h$ from the minimum potential gas particle, we take the distance from the two nearest snapshots and linearly interpolate in time between them. Otherwise we use the 
 distance to the local minimum potential gas particle calculated at each time step.

%%%%%%%%%%%%%%%%%%%%%%%%%%%%%%%%%%%%%%%%%%
\subsubsection{DF and No Correction}

Before calibrating our dynamical friction modeling, we first demonstrate the effectiveness of our fiducial DF model, \texttt{DF\_4DM\_G}, by comparing it with the no-DF run \texttt{NoDF\_4DM\_G} (note that throughout the paper, no-DF means no correction to the BH dynamics of any form besides the resolved gravity). We keep all parameters fixed except for the black hole dynamics modeling. The details of these simulations can be found in Table \ref{tab:cons}.

In Figure \ref{fig:big_plot}(a), we show the evolution of BH2 in Figure \ref{fig:halos} under the no-DF and the fiducial DF models. Without any correction to the black hole dynamics, even the largest black hole in the simulation does not exhibit efficient orbital decay throughout its evolution: the distance from the halo center is always fluctuating above $2\epsilon_g$. This is because the black hole does not experience enough gravity on scales below the softening length, and cannot lose its angular momentum efficiently. Now when we add the additional dynamical friction to compensate for the missing small-scale gravity, the black hole is able to sink to within 1 ckpc$/h$ of the halo centers in <200 Myr and remain there. 

The 90 ckpc$/h$ peak in the drift of the black hole marks the merger between BH1 and BH2 in Figure \ref{fig:halos}, when the host halo of BH2 merges into the host of BH1, and the halo center is redefined near the merger. After the halo merger, dynamical friction is able to sink the black hole to the new halo center and allows it to merge with the black hole in the other halo, whereas in the no-DF case we do not see the clear orbital decay of the black holes after the merger of their host halo until the end of the simulation.

Besides the drift, we also show the x-component of the black hole's velocity relative to its surrounding collisionless particles (lower panel). Here we show one component instead of the magnitude to better visualize the velocity oscillation. With dynamical friction turned on, the velocity of the black hole is more stable, as the black hole's orbit has already become small and is effectively moving together with the host halo. Without dynamical friction, the black hole tends to oscillate with large velocities around the halo center without losing its angular momentum.

The different dynamics of the black hole can also affect accretion due to differences in density and velocities, so we also look at the black holes' mass growth in the two scenarios (middle panel). The mass growths of the two black holes are similar under the two models, although when subjected to dynamical friction, the black holes have more and earlier mergers. Even though the black hole mass is less sensitive to the dynamics modeling, the merger rate predictions can be affected significantly as we will discuss later. 

Note that for our no-DF model, we have also boosted the dynamical mass to $4\times M_{\rm DM}$ at the early stage to prevent scattering by the dark matter and star particles. However, even after the boost, the black holes cannot lose enough angular momentum to be able to stay at the halo center. This means that even though dynamical heating is alleviated through the large dynamical mass, the sub-resolution gravity is still essential in sinking the black hole to the host halo center.

%%%%%%%%%%%%%%%%%%%%%%%%%%%%%%%%%%%%%%%%%%

\subsubsection{Dynamical Friction and Gas Drag}
\label{subsec:drag}
\begin{figure*}
\includegraphics[width=0.49\textwidth]{RESULTS/plots/drag_df2drg3b10mg1_4dm8406903.pdf}
\includegraphics[width=0.49\textwidth]{RESULTS/plots/DF_drag.pdf}

\caption{Comparisons between DF and hydro drag. \textbf{Left:} comparison for a single black hole. In the top panel we show the magnitude of the DF (\textbf{red}) and gas drag (\textbf{blue}) relative to gravity for the same black hole, in the \texttt{DF+Drag\_4DM\_G} run. During the early stage of the black hole evolution, DF and gas drag have comparable effect, while after $z=7.5$ the gas drag becomes less and less important, as the gas density decreases relative to the stellar density (\textbf{middle}), and the black hole velocity goes into the subsonic regime (\textbf{lower}). \textbf{Right:} Ratio between DF and gas drag for all black holes. We plot the ratio both as a function of redshift (\textbf{top}) and as a function of time after a black hole is seeded (\textbf{bottom}). The orange lines represent the logarithmic mean of the scatter. The $F_{\rm DF}/F_{\rm drag}$ ratio depends strongly on the evolution time of the black hole: the longer the black hole evolves, the less important the drag force is. However, there is not a strong correlation between redshift and the $F_{\rm DF}/F_{\rm drag}$ ratio.}
\label{fig:drag}
\end{figure*}


\begin{figure*}
\includegraphics[width=0.33\textwidth]{RESULTS/plots/ratio_Mbh.pdf}
\includegraphics[width=0.66\textwidth]{RESULTS/plots/hexbin.pdf}
\caption{\textbf{Left:} Scattering relation between the $F_{\rm DF}/F_{\rm drag}$ ratio and the black hole mass. For each black hole, we sample its mass at uniformly-distributed time bins throughout its evolution, and we show the scattered density of all samples. DF has significantly larger effects over gas drag on larger BHs. We fit the scatter to a power-law shown in the orange line. \textbf{Right:} Scattering relation between the $F_{\rm DF}/F_{\rm drag}$ ratio and the BHs' distance to the halo center. Comparing with the BH mass, we do not see a clear dependence of the $F_{\rm DF}/F_{\rm drag}$ ratio on the distance to halo center. For BHs at all locations within the halo, DF is in general larger than the gas drag.}
\label{fig:drag_scatter}
\end{figure*}

In the previous subsection, we've only included collisionless particles (DM+Star) when modeling the dynamical friction, now we will look into the effects of dynamical friction of gas (gas drag) in comparison with the collisionless particles in the context of our simulations.

From Equation \ref{eq:H14} and \ref{eq:drag}, the relative magnitudes of DF and drag mainly depend on two components: the relative density of DM+stars versus gas, and the values of $\mathcal{F}(x)$ and $\mathcal{I(M)}$. \cite{Ostriker1999} has shown that when a black hole's velocity relative to the medium falls in the transonic regime (i.e. near the local sound speed), $\mathcal{I}$ is a few times higher than $\mathcal{F}$, while in the subsonic and highly supersonic regimes $\mathcal{I}$ is smaller or equal to $\mathcal{F}$. Therefore, we would expect the gas drag to be larger when the black hole is in the early sinking stage with a relatively high velocity and a high gas fraction. 

In Figure \ref{fig:drag}, the left panel shows the comparison between the magnitude of DF and gas drag through different stages of the black hole evolution, as well as the factors that can alter the effectiveness of the gas drag. In the very early stages ($z>7.5$) of black hole evolution, DF and gas drag have comparable effects, while after $z=7.5$ the gas drag becomes significantly less important and almost negligible compared with DF. The reason follows what we have discussed earlier: the gas density decreases relative to the stellar density (shown in the middle panel), and the black hole's velocity relative to the surrounding medium goes into the subsonic regime as a result of the orbital decay (shown in the lower panel). Around $z=3.5$, there is a boost in the black hole's velocity due to disruption during a major merger with a larger galaxy and black hole. The effect of gas is again raised for a short period of time (although still subdominant compared to the DF).

In Figure \ref{fig:big_plot}(b) we plot the black hole evolution for the DF-only (\texttt{DF\_4DM\_G}), drag-only (\texttt{Drag\_4DM\_G}), and DF+drag (\texttt{DF+Drag\_4DM\_G}) simulations.
Both the drag-only and DF-only models are effective in sinking the black hole at early times ($z>7$). However, at lower redshifts, the gas drag is not able to sink the black hole by itself, whereas DF is far more effective in stabilizing the black hole at the halo center. For this reason, in low-resolution cosmological simulations, dynamical friction from collisionless particles is necessary to prevent the drift of the black holes out of the halo center.

To further illustrate the relative importance between DF and gas drag for the entire BH population, we examine the dependencies of the $F_{\rm DF}/F_{\rm drag}$ on variables related to the BH evolution for all BHs in the \texttt{DF+Drag\_4DM\_G} simulation. First, in the right panel of Figure \ref{fig:drag} we show the time evolution of $F_{\rm DF}/F_{\rm drag}$. The top panel shows the ratio as a function of cosmic time, while the bottom panel shows the ratio as a function of each BH's seeding time. The DF/Drag ratio has a wide range for different BHs, but overall DF is becoming larger relative to the gas drag as the black hole evolves. From the mean value of the DF/drag ratio, we see that when the black holes are first seeded, DF is only a few times larger than the gas drag. After a few Gyrs of evolution, DF becomes 2-3 orders of magnitude larger than the gas drag. However, there is not a strong correlation between redshift and the $F_{\rm DF}/F_{\rm drag}$ ratio. 

In the left panel of Figure \ref{fig:drag_scatter}, we show the scattering relation between the $F_{\rm DF}/F_{\rm drag}$ ratio and the black hole mass $M_{\rm BH}$. We see a strong correlation between the $F_{\rm DF}/F_{\rm drag}$ ratio and the black hole mass: DF has significantly larger effects over gas drag on larger BHs, although the range of the ratio is large ar the low mass end. We fit a power-law to the median of the scatter:
\begin{equation}
    \frac{F_{\rm DF}}{F_{\rm drag}} = 250 \left(\frac{M_{\rm BH}}{10^7 M_\odot}\right)^{1.7},
\end{equation}
which roughly characterize the effect of the two forces on BHs of different masses. From this relation we see that for BHs with masses $>10^7 M_\odot$, gas drag is in general less than $1\%$ of DF. Finally, the right panels show the relation between the $F_{\rm DF}/F_{\rm drag}$ ratio and the BH's distance to the halo center: there is not a strong dependency on the BH's position within the halo.


%%%%%%%%%%%%%%%%%%%%%%%%%%%%%%%%%%%%%%%%%%


\subsubsection{Comparisons with the T15 Model}
\label{subsec:df100}
\begin{figure}
\includegraphics[width=0.5\textwidth]{RESULTS/plots/compare_kernel.pdf}

\caption{ Comparison between different components in the two dynamical friction models, DF(fid) (\textbf{red}) and DF(T15) (\textbf{blue}) (see Section \ref{sec:bh_model} for descriptions). We show the number of stars and dark matter particles included in the DF density and velocity calculation (\textbf{top panel}), the density used for DF calculation (\textbf{second panel}), the Coulomb logarithm used in the two methods (\textbf{third panel}), the velocity of the BH relative to the surrounding particles (\textbf{forth panel}, note that the "surrouding particles" are defined differently for the two models), and the magnitude of DF relative to gravity (\textbf{bottom panel}). The higher DF in the DF(fid) model at $z>8$ is due to the larger Coulomb logarithm. After $z\sim 7$, the higher density of DF(T15) due to more localized density calculation counterbalances its lower $\text{log}(\Lambda)$, resulting in similar DF between $z=8$ and $z=3.5$. During the halo merger at $z=3.5$, the DF(fid) model included particles from the target halo into the density calculation, and therefore yields larger DF during the merger.}
\label{fig:k100_case1}
\end{figure}


For the collisionless particles, we test and study two different implementations for the dynamical friction: DF(fid) and DF(T15) (see Section \ref{sec:bh_model} for detailed descriptions). In Section \ref{sec:bh_model} we pointed out three main differences between them: different kernel sizes (SPH kernel vs. nearest 100 DM+star), different definitions of $b_{\rm max}$ (10 ckpc vs. 1.5 ckpc$/h$), and different approximation of the surrounding velocity distribution (Maxwellian vs. nearest 100-sample distribution). Essentially, these differences mean that DF(fid) is a less-localized implementation than DF(T15). Now we would like to evaluate the effectiveness of these two implementations and show how different factors affect the final dynamical friction calculation.

Figure \ref{fig:k100_case1} shows the relevant quantities in the DF computation for the two methods. The two kernels both contain $\sim 100$ dark matter and star particles at high redshift ($z>8$), but after that the SPH kernel (defined to include the nearest 113 gas particles) begins to include more and more stars and dark matter. The mass fraction of stars in the SPH kernel dominates over that of dark matter by $\sim 10$ times for a BH at the center of the galaxy. The larger kernel of DF(fid) has two effects: first, the DF density will be smoother over time; second, during halo mergers, the DF(fid) kernel can "see" the high-density region of the larger halo, which results in a higher DF near mergers compared to DF(T15). This is confirmed by the second panel, where we show the density for dynamical friction calculation from the two kernels. The densities calculated from the two kernels are similar in magnitude throughout the evolution, although the DF(T15) kernel yields slightly larger density due to its smaller size. Around the BH merger, the density in DF(fid) is larger due to its inclusion of the host halo's central region.

The third panel shows the Coulomb logarithm in the two models. Recall that $\Lambda = \frac{b_{\rm max}}{(GM_{\rm BH})/v_{\rm BH}^2}$, and so the Coulomb logarithm depends on the black hole's mass, its velocity relative to the surrounding particles, and the value of $b_{\rm max}$. From Figure \ref{fig:big_plot}(c), the mass of the DF(T15) black hole is slightly smaller, but the mass difference is small compared with the 6 times difference in $b_{\rm max}$. Given $b_{\rm max}$=10 ckpc$/h$ in DF(fid) and $b_{\rm max}$=1.5 ckpc$/h$ in DF(T15), we would expect the Coulomb logarithm to be larger for the former. However, there is yet another tweak: the $v_{\rm BH}^2$ term turns out to be significantly larger in the DF(T15) model(fourth panel). Note that in the DF(T15) model $v_{\rm BH}^2$ is calculated using only 100 surrounding particles, and for the high-density region we are considering here, the velocity of the nearest 100 particles is very noisy in time.  As we will show in Appendix \ref{app:df100}, for smaller black holes the difference in $v_{\rm BH}^2$ is not as large, and usually DF(fid) has a larger $\text{log}\Lambda$ due to its larger $b_{\rm max}$.

In Figure \ref{fig:big_plot}(c), we show the evolution of the black hole under these two models. At high redshift ($z>8$), due to the large $\text{log}(\Lambda)$, the black hole in the DF(fid) simulation sinks slightly faster to the halo center. Between $z=8$ and $z=3.5$, both models have similar dynamical friction (as discussed in the previous paragraph) and the motion and mass accretion are also similar. Then at $z=3.5$, within the host halo of the black hole major merger, dynamical friction in DF(fid) is again larger because the density kernel includes more particles from the high-density region in the target halo, and this leads to an earlier merger time.

Overall, the performance of the two models is similar. However, as we have seen in the velocity calculation of the black holes relative to the surrounding particles, DF(T15) could be too localized for simulations of our resolution ($\epsilon_g \sim 1$kpc/h) and is sometimes subject to numerical noise. Therefore, in our subsequent statistical runs we pick DF(fid) as our fiducial model, and will drop the 'fid' in its name hereafter.


%%%%%%%%%%%%%%%%%%%%%%%%%%%%%%%%%%%%%%%%%%%
\subsubsection{Gravitationally Bound Merging Criterion}
\label{subsec:bound_check}

The merging criterion can affect not only the merging time, but also the dynamics and evolution of the black holes. Naively, we might expect the distance-only merging to produce more massive black holes, because black holes are merged more easily. However, in many cases this is not true, and we will illustrate here through one example. 

Figure \ref{fig:big_plot}(d) shows the evolution of the same black hole with the same dynamical friction prescription, but different merging criteria. We note a drastic difference in the black hole's trajectories: while the BH in the gravitationally bound merger case is staying at the center of its host halo, the BH in the distance-only merger flies out of its host after a merger. This is because with the distance-only model, it is possible for one black hole to have a very large velocity at the time of the merger, since we do not limit the black hole's velocity. By momentum conservation, the black hole with a larger velocity can transfer the momentum to the other black hole (and the merger remnant) which might have already sunk to the halo center. The sunk black hole then drifts out of the halo center after a merger due to the large momentum injection. This is especially common in simulations where the black hole's dynamical mass is boosted, because the injected momentum is also boosted with mass and a smaller black hole in a satellite galaxy can easily kick a larger black hole out. If we add on the gravitational bound check, there will be more time for the black holes to lose their angular momentum, and so the injected momentum is far less, and in most cases does not kick each other out of the central region.








\subsection{Black Hole Mergers}
\begin{figure*}
\includegraphics[width=0.49\textwidth]{RESULTS/plots/merger_case1.pdf}
\includegraphics[width=0.49\textwidth]{RESULTS/plots/merger_case2.pdf}
\includegraphics[width=0.49\textwidth]{RESULTS/plots/stars2.png}
\includegraphics[width=0.49\textwidth]{RESULTS/plots/stars1.png}


\caption{The comparison between the distance of two merging black holes in the no-correction, DF(fid), DF(T15) and gas drag models in the early stage (\textbf{left}) and later stage (\textbf{right}) of the black hole evolution. For early mergers, the effect of the frictional forces (DF and drag) is not very prominent but still noticeable. The DF and gas drag both allow the black holes to merge faster compare to the no-DF case. For the later merger happening in a denser environment, the effect of dynamical friction is clear. However, the gas drag does not have a big effect on the black hole at this late stage compared with the no-DF case. The lower panels show the merging black holes within their host galaxies as well as their trajectories towards the merger in the \texttt{DF\_4DM\_G} run. The left images show the early phase of the orbital decay, and the right images show the later phase when the orbits get smaller.}
\label{fig:merger_case1}
\end{figure*}
Having seen the effect of different dynamical models on the evolution of individual black holes, next we will discuss how the dynamics, together with different BH merging criteria,  affect the evolution and mergers of the black holes.
 In particular, we want to study their merging time and trajectories before and after the mergers. Similar to the previous subsection, we will draw our examples from the two halos shown in Figure {\ref{fig:halos}}.
 
 %%%%%%%%%%%%%%%%%%%%%%%%%%%%%%%%%
\subsubsection{Effect of Dynamical  Friction Modeling}
\label{subsec:case_merger_calc}

We first look at how different dynamical models affect the time scale of black hole orbital decay and mergers. We pick two cases of mergers:  one is an early merger at $z>5$ when the black holes have not outgrown their dynamical masses; the other is a later merger at $z \sim 3.3$ when both BHs are larger than their seed dynamical masses (the major merger between BH1 and BH2 in Figure \ref{fig:halos}). Following \cite{Tremmel2015}, we also compute the dynamical friction time for the two mergers using Equation (12) - Equation (15) from \cite{Taffoni2003}:
\begin{equation}
\label{eq:tdf}
    t_{\rm DF} = 0.6\times 1.67\text{Gyr} \times \frac{r_c^2 V_h}{G M_s} \text{log}^{-1} \left( 1+\frac{M_{\rm vir}}{M_s} \right) \left(\frac{J}{J_c}\right)^\alpha,
\end{equation}
where $M_s$ is the mass of the smaller black hole (which we treat as the satellite), $M_{\rm vir}$ is the virial mass of the host halo of the larger black hole (found by AHF), $V_{h}$ is the circular velocity at the virial radius of the host, and $r_c$ is the radius of a circular orbit with the same energy as the satellite black hole's initial orbit. The last term $\left(\frac{J}{J_c}\right)^\alpha$ is the correction for orbital eccentricity, where $J$ is the angular momentum of the satellite, $J_c$ is the angular momentum of the circular orbit with the same energy as the satellite, and $\alpha$ is given by:
\begin{equation}
    \alpha \left( \frac{r_c}{R_{\rm vir}}, \frac{M_s}{M_{\rm vir}} \right) = 0.475 \left[ 1-\text{tanh} \left( 10.3 \left(\frac{M_s}{M_{\rm vir}}\right)^{0.33} - 7.5 \left(\frac{r_c}{R_{\rm vir}}\right) \right)  \right].
\end{equation}
In our calculation the virial radius, velocity, and mass are obtained from the AHF outputs, and the circular radius, orbit energy, and angular momentum are calculated by fitting the halo density profile to the NFW profile.

Figure \ref{fig:merger_case1} shows distances between two merging black holes in the no-DF, DF(fid), DF(T15), and gas drag models in the early and later stages of their evolution. For the early merger, the effect of the frictional forces (DF and drag) is not very big but still noticeable. The DF and gas drag have similar effects on the orbital decay at higher redshifts, consistent with our discussion in Section \ref{subsec:drag}. The DF(T15) model sinks the black hole a little slower than the DF(fid) model, but the difference is within $50$ Myrs. All three friction models allow the black holes to merge faster compare to the no-DF case by $\sim 150$ Myrs.

For the later merger, which takes place in a denser environment, the effect of dynamical friction is clearer: the dynamical friction allows the black holes to sink within the gravitational softening of the particles in $<200$ Myrs. Without dynamical friction the black hole's orbit does not have a clear decay below $2$ kpc and does not merge at the end of our simulation. Furthermore, the gas drag does not have a big effect on the black hole at this late stage compared with the no-correction case. This follows from our discussion in section \ref{subsec:drag} that gas drag is much less effective at lower redshift compared to dynamical friction.
 
In both plots, the yellow shaded region is the dynamical friction time from the analytical calculation in Equation \ref{eq:tdf}. Here we draw a band instead of a single line, because the black hole's orbit is not a strict ellipse, and the black hole is continuously losing energy. We calculate $t_{\rm DF}$ at multiple points between the first and second peak in the black hole's orbit (e.g. between $z=5.9$ and $z=5.7$ in the earlier case), and plot the range of those $t_{\rm DF}$. For both mergers, the analytical prediction is less than 150 Myrs later than the merger of the (fid) model. We note that the \cite{Taffoni2003} analytical $t_{\rm DF}$ is a fit to the NFW profiles, and the previous numerical and analytical comparisons on the black hole dynamical friction\citep[e.g.][]{Tremmel2015,Pfister2019} are performed in idealized NFW halos with a fixed initial black hole orbit. In our case, the halo profiles and black hole orbits are not directly controlled, and therefore deviation from the analytical prediction is expected. We will study such deviations statistically later in Section \ref{sec:merger_stats}.




%%%%%%%%%%%%%%%%%%%%%%%%%%%%%%%%%%%%%%%%%%%%%%%%%%%%%%%%%%
 \subsubsection{Effect of Gravitational Bound Check}

 In Section \ref{subsec:merger} we introduced two criteria which we use to perform black hole mergers in our simulations: we can merge two BHs when they are close in distance, and we can also require that the two BHs are gravitationally bounded in addition to the distance check. 
 
 In Figure \ref{fig:merger_case1} we show the difference in black holes' merging time with and without the gravitational bound criterion. The vertical dashed line marks the time that the two black holes in the \texttt{DF\_4DM\_G} simulation would merge if there was not the gravitational bound check. Without the gravitational bound check, the orbit of the black holes is still larger than 1 kpc when they merge, whereas with the gravitational bound check, the orbit size generally decays to less than 300 pc when the black holes merge. The merger without gravitational bound check generally makes the merger happen earlier by a few hundred Myrs (we will study the orbital decay time statistically in the next section). Therefore, for more accurate merger rate predictions as well as the correct accretion and feedback, it is necessary to apply the gravitational bound check during black hole mergers whenever the black hole has a well-defined velocity.
 




\section{Black Hole Statistics}
\label{sec:stats}
After looking at individual cases of black hole evolution, we now turn to the whole SMBH population in the simulations with different modeling of black hole dynamics. For statistics comparison, instead of using the $L_{\rm box} = 10$ Mpc$/h$ constrained realizations, we now use $L_{\rm box} = 15$ Mpc$/h$ unconstrained simulations. The details of our $L_{\rm box} = 15$ Mpc$/h$ simulations are shown in Table \ref{tab:norm}.

\subsection{Sinking of the Black Holes}
\label{subsec:drift}

\begin{figure}
\includegraphics[width=0.49\textwidth]{RESULTS/plots/L15_drift.pdf}
\caption{The effect of different BH dynamics modeling on BH position relative to its host. We include the reposition model (blue), no-DF model (orange),DF(T15) model (green), DF(fid) model (red) and the DF+drag model (purple). \textbf{Top:} The fraction of halos(subhalos) without a black hole for halos with masses above the black hole seeding mass at $M_{\rm halo} = 10^{10} M_\odot/h$. \textbf{Middle:} The fraction of halos without a central black hole ("central" means within $2\epsilon_g$ from the halo center identified by the halo finder), out of all halos with black holes. \textbf{Bottom:} Distribution of black holes' distance to its host halo center.} 
\label{fig:drift}
\end{figure}


\begin{figure}
\includegraphics[width=0.49\textwidth]{RESULTS/plots/L15_BHMF.pdf}
\caption{Mass functions for reposition, DF and no-DF simulations. With reposition (\textbf{blue}), we have the highest mass function and earlier formation of $10^8 M_\odot$ black holes. The no-DF simulations (\textbf{green}) have lower mass functions, which is expected due to low-accretion and merger rates from the black hole drifting. The dynamical friction model (\textbf{red}) yields a mass function in between.} 
\label{fig:bhmf}
\end{figure}


The primary reason for adding dynamical friction onto black holes within the cosmological simulations is to stabilize the black hole at the halo center (defined as the position of the minimum-position gas particle within the halo). Hence, we start by looking at the black holes' position relative to the host halos. Due to the resolution limit of our simulations, we would not expect the black holes to be able to sink to the exact minimum potential. Instead we consider a $<2\epsilon_g = 3$ ckpc$/h$ distance to be "good sinking".

In Figure \ref{fig:drift}, we show the statistics related to black holes' sinking status. We included the comparison between the reposition model (\texttt{L15\_Repos\_4DM}), the no-DF model (\texttt{L15\_NoDF\_4DM}), the two dynamical friction models (\texttt{L15\_DF\_4DM} and \texttt{L15\_DF(T15)\_4DM}) and the DF+drag model(\texttt{L15\_DF+drag\_4DM}). To start with, we simply count the fraction of halos without a black hole when its mass is already above the black hole seeding criterion (i.e. $10^{10}M_{\odot}/h$). The top panel shows the fraction of large halos without a BH for different models at $z=3.5$ and $z=2$. Surprisingly, the no-DF model ends up with the least halos without a black hole. This is because even though the black holes without dynamical corrections cannot sink effectively, the high dynamical mass still prevents sudden momentum injections from surrounding particles, and therefore most BHs still stays within their host galaxies. The dynamical friction models perform equally well, with $<10\%$ no-BH halos at the low-mass end. The reposition model, however, ends up with the most no-BH halos, even though repositioning is meant to pin the black holes to the halo center. This happens because under the repositioning model, the central black holes tend to spuriously merge into a larger halo during fly-by encounters, leaving the smaller sub-halo BH-less.

Next we look at where the black holes are located within their host galaxies. For all the halos with at least one black hole, we examine whether the black hole is located at the center (i.e.$<2\epsilon_g = 3$ckpc/h from the halo center). The middle panel of Figure \ref{fig:drift} shows the fraction of halos without a central BH. The no-DF model has significantly more halos without a central BH compared to the other models, with over half of the halos hosting off-center BHs. Among the three runs with dynamical friction, the DF(T15) and DF(fid) models have a similar fraction of halos ($\sim 20\%$) without a central BH, and we can see this fraction dropping from $z=3.5$ to $z=2$, meaning that many BHs are still in the process of sinking towards the halo center. When we further add the gas drag, $10\%$ more halos host at least one central BH, and the difference between the drag and no-drag central BHs is more prominent at high redshifts. 

Interestingly, the repositioning algorithm is not as efficient at sinking the BHs at $z=2$ as the DF. This is because our repositioning algorithm places the BHs at the minimum potential position within the accretion kernel, instead of within the entire halo. The majority of the offset between the BH positions and the halo center comes from the offset between the minimum-potential position accessible to the BH (i.e. minimum-potential in the accretion kernel) and the minimum-potential position in the halo. Such offset can be especially severe at lower redshift, when the size of the accretion kernel gets smaller and mergers happen more frequently, making it easier for the black holes to get stuck at a local minimum.


In the bottom panels we show the distributions of the black holes' distance to the halo centers under different models. For the no-DF run, again we see that the black holes fail to move towards the halo center at lower redshift, resulting in a much flatter distribution compared to all the other models. In comparison, when we add dynamical friction to the black holes, for both the DF(fid) and the DF(T15) models the distributions are pushed much closer to the halo center, with a peak around the gravitational softening length. When we then add the gas drag in addition to DF, the peak at $\epsilon_g$ becomes slightly higher than those in the DF-only runs. The combination of DF and gas drag, as we would expect from the case studies, is the most effective in sinking the black holes to the halo centers and stabilizing them. Finally, we plot the repositioning model for reference.  It does well in putting the black hole close to the minimum potential, and often the black holes can be located at the exact minimum-potential position (the distributions peak at 0 for $z=3.5$). However, as discussed in the previous paragraph, there are cases where the local minimum potential found by the repositioning algorithm does not coincide with the global minimum potential of the halo, and that is why we also see non-zero probability density for $\Delta r > 3$ ckpc$/h$ at $z=2$.

The statistics we have seen for the models above are consistent with the results from the case studies. This shows that even though for the case studies we have focused mainly on large black holes in one of the biggest halo, a similar trend still applies to other black holes in the cosmological simulations, which are embedded in smaller halos or subhalos.


 

%%%%%%%%%%%%%%%%%%%%%%%%%%%%%%%%%%%%%%%%%%%%%%%%%%%%%%%%%%%%%%%%%%%%%%%%%%%%%%

\subsection{Black Hole Mass Function}
\label{subsec:bhmf}
Next we look at how different dynamics affect the black hole mass function (BHMF). One problem with the repositioning method is that it places the black holes at the galaxy center too quickly, which could result in excess accretion and thus a higher mass function. On the other hand, if we do not add any correction to the black hole motion, many BHs will not go though efficient accretion and mergers, and we will see a lower mass function. We would expect the BHMF in the dynamical friction run to fall between the repositioning case and the no-DF case.

Figure \ref{fig:bhmf} shows the BHMF from the reposition(\texttt{L15\_Repos\_4DM}), dynamical friction (without gravitational bound check:\texttt{L15\_DF\_4DM}; with gravitational bound check:\texttt{L15\_DF\_4DM\_G}), and no-DF (without gravitational bound check: \texttt{L15\_NoDF\_4DM}; with gravitational bound check: \texttt{L15\_NoDF\_4DM\_G}) runs. The reposition model yields the highest mass function, and is the only simulation with more than one $10^8 M_{\sun}/h$ black holes at $z=2$. This is expected from the over-efficient BH mergers and the high-density surroundings in the reposition model. Moreover, it creates increasingly more massive BHs over time, as the increased merger rate produces a stronger effect over time. The no-DF runs produces the lowest mass function due to the off-centering, while the DF mass function falls between the reposition and no-DF case as we expected. 

Naively, we would expect the models without gravitational bound checks to produce a higher mass function, because it allows for easier mass-accretion via mergers. However, as discussed in Section \ref{subsec:bound_check}, this is not the case if we compare the dashed lines and solid lines with the same colors. For example, under the DF model, the \texttt{L15\_DF\_4DM\_G} simulation forms more massive black holes than the \texttt{L15\_DF\_4DM} simulation, especially at lower redshift. The reason can be traced back to what we have seen in Figure \ref{fig:big_plot}(d): when there is no gravitational bound check, the large momentum injection during a merger kicks the black hole out of the halo center, thus preventing the efficient growth of large black holes.

Considering the relatively large uncertainties due to the limited volume, the difference in the mass function is not very significant. We would expect other factors such as the black hole seeding, accretion and feedback to have a larger effect on the mass function compared to the dynamical models we show here \citep[e.g.][]{Booth2009}.
\subsection{Dynamical Friction Time and Mergers}
\label{sec:merger_stats}

\begin{figure}
\includegraphics[width=0.48\textwidth]{RESULTS/plots/tdf.pdf}
\caption{The delay of mergers due to the dynamical friction time. Here we compare the numerical dynamical friction time,$t_{\rm num}$, to the analytically calculated time (following Equation \ref{eq:tdf}) $t_{\rm analy}$. \textbf{Top left:} distribution of the dynamical friction time from numerical merger (blue) and analytical predictions (red). \textbf{Top right:} ratio between the numerical and analytical $t_{\rm df}$. Their difference is less than one order of magnitude in all merger cases. \textbf{Bottom:} dynamical friction time as a function of the virial mass of the host halo for the numerical (blue) merger and analytical predictions (red). The same merger event is linked by a grey line.} 
\label{fig:delay}
\end{figure}

\begin{figure}
\includegraphics[width=0.49\textwidth]{RESULTS/plots/L15_mergers.pdf}
\caption{The cumulative mergers for different BH dynamics and merging models. The reposition model (\textbf{blue solid}) predicts more than two times the total mergers compared with the other models. Without the gravitational bound check, the DF (\textbf{red dashed}) and the no-DF model (\textbf{green dashed}) predicts similar numbers of mergers, indicating that the first encounters of the black hole pairs are similar under the two models. However, if we add the gravitational bound check, the dynamical friction model (\textbf{red solid}) yields $\sim 50\%$ more mergers compared to the no-correction model. Adding the gas drag in addition to dynamical friction (\textbf{purple solid}) raises the mergers by a few. } 
\label{fig:merger_stats}
\end{figure}



Because the reposition method is used in most large-volume cosmological simulations, a post-processing analytical dynamical friction time is calculated in order to make more accurate merger rate predictions. Now that we have accounted for the dynamical friction on-the-fly, we want to study how our numerical mergers with dynamical friction compare against the analytical predictions, and how different dynamical models impact the black hole merger rate.

In Section \ref{subsec:case_merger_calc}, we compared the numerical merging time to the analytical predictions for two merger cases. Now we use the same method to calculate an analytical dynamical friction time for all black hole mergers in our \texttt{L15\_DF\_4DM\_G} simulation. For each pair, we begin the calculation at the time $t_{\rm beg}$ when the black hole pair first comes within 3 ckpc$/h$ of each other, as this mimics the merging time without the gravitational bound check, and is also close to the merging criterion under the reposition model. The numerical dynamical friction time $t_{\rm num}$ is the time between the numerical merger and $t_{\rm beg}$. The analytical dynamical friction time $t_{\rm analy}$ is calculated using the host halo information in the snapshot just before $t_{\rm beg}$ and the black hole information at the exact time-step of $t_{\rm beg}$.

Figure \ref{fig:delay} shows the comparison between the numerical and analytical dynamical friction times. In the top panel we show the distribution of the two times as well as the distribution of their ratio. We note that for all the mergers happening numerically, $t_{\rm analy}$ does not exceed 2 Gyrs, and most have $t_{\rm analy}$ less than 1 Gyr. This means that we do not have many fake mergers that shouldn't merge until much later (or never). Also, the ratio plot shows that the numerical and analytical times are always within an order of magnitude of each other, with most of the numerical mergers earlier than the analytical mergers. The numerical merger time is peaked between 100 Myrs and 1 Gyrs, whereas the analytical calculation yields a flatter distribution. We would expect $t_{\rm analy}$ to be longer than $t_{\rm num}$, both because we have a selection bias on $t_{\rm DF}$ by ending the similation at $z=2$, and because we numerically merge the black holes when their orbit is still larger than 3 ckpc$/h$. However, this does not explain why $t_{\rm analy}$ has a higher probability between 10 Myrs and 100 Myrs. 

To see the individual merger cases in the distribution more clearly, in the lower panel of Figure \ref{fig:delay} we plot all the numerical and analytical dynamical friction times as a function of the host halo's virial mass. From this figure we do not see a clear dependence of either dynamical friction times on the host halo's virial mass. There is also no strong correlation between the $t_{\rm num}/t_{\rm analy}$ ratio and the halo mass. We do not further investigate the discrepancies between the numerical and analytical results, as these results can vary significantly from system to system. 

We note that although the numerical model has free parameters (such as $b_{\rm max}$, $M_{\rm dyn, seed}$) that can impact the merging time (but see Appendix \ref{app:merger_param}), it can account for the immediate environment around black hole and adjust the dynamical friction on-the-fly. More importantly, it also accounts for the interaction between the satellite BH and its own host galaxy, which could reduce the sinking time significantly \citep[e.g.][]{Dosopoulou2017}. 
The analytical model, though verified by N-body simulations, does not react to the environment of the merging galaxies by always assuming an NFW profile. Moreover, it only models the sinking of a single BH without embedding it in its host galaxy. Therefore, we expect the numerical result to be a more realistic modeling of the binary sinking process.

After comparing the DF model against the analytical prediction, next we compare different numerical models in terms of the black hole merger rate. Figure \ref{fig:merger_stats} shows the cumulative mergers from $z=8$ to $z=2$. We have included comparisons between the reposition, dynamical friction and no-DF models, both with and without the gravitational bound check. The reposition model predicts more than twice the total number of mergers compared to the other models. Without the gravitational bound check, the DF and the no-DF models predict similar numbers of mergers, indicating that the first encounters of the black hole pairs are similar under the two models. However, if we add the gravitational bound check, the DF model yields $\sim 50\%$ more mergers compared to the no-DF model, because the addition of DF assists energy loss of the binaries and leads to earlier bound pairs. Finally, the merger rate is not very sensitive to adding the gas drag: the merger rate in the DF-only model is similar to that of the DF+drag model. This can be foreseen in the comparison shown in Figure \ref{fig:drag}, where the gas drag is subdominant in magnitude.


\section{Merger Rates in the 35Mpc/h Simulations}
\label{sec:L35}

\begin{figure*}
\includegraphics[width=0.99\textwidth]{RESULTS/plots/hist.pdf}
\caption{ \textbf{Left:} Distribution of the mass of the smaller black hole ($M_s$), and distribution of the total mass of the binary ($M_{\mathrm{tot}}$). For both simulations, the mergers in which at least one of the black holes is slightly above the seed mass dominate. The most massive binary has a total mass of $3\times 10^8 M_\odot$. \textbf{Middle:} The mass ratio $q$ between the two black holes in the binary. We see a peak at $\text{log(q)}=-0.5$, corresponding to pairs in which one BH is about three times larger than the other. \textbf{Right:} Scatter of the two black hole masses in the binaries, binned by redshift. To separate the scatter in the two simulations, for the DF+drag run we take $M_1$ to be the mass of the larger BH, while for the NoDF run $M_2$ is the larger BH.}
\label{fig:hist}
\end{figure*}

\begin{figure}
\includegraphics[width=0.49\textwidth]{RESULTS/plots/rates.pdf}
\caption{Merger rate per year of observation per unit redshift predicted from our \texttt{L35\_DF+drag\_4DM\_G} (\textbf{purple}) and \texttt{L35\_NoDF\_4DM\_G} (\textbf{blue}) simulations. 
For comparison, we also show the the prediction from recent hydro-dynamical simulations. 
We include three simulations of similar mass-resolution: \citet{Volonteri2020} from the Horizon-AGN simulation (\textbf{gray}), \citet{Katz2020} (\textbf{yellow}) from the Illustris simulation and \citet{Salcido2016} from the EAGLE simulations (\textbf{pink}).
Since we do not apply any post-processing delays after the numerical mergers, we only compare to results without delays.}
\label{fig:rates}
\end{figure}

Based on all the previous test of BH dynamics modeling, we have reached the conclusion that the DF+drag model with $M_{\rm dyn} = 4 M_{\rm DM}$ is most capable of sinking the black hole to the halo center. Hence, we choose to use this model to run our larger-volume simulation \texttt{L35\_DF+drag\_4DM\_G} for the prediction of the BH coalescence rate. Besides this model, we also perform a same-size run without the dynamical friction, \texttt{L35\_NoDF\_4DM\_G}, as a lower limit for the predicted rate.  
Our \texttt{L35} simulations are run down to $z=1.1$. The black hole seed mass is $5\times10^5 M_\odot/h$ and the minimum halo mass for seeding is $10^{10} M_\odot/h$. The details of these two simulations are shown in Table \ref{tab:norm}.

\subsection{The Binary Population}
\label{subsec:L35_catalog}
Because this work mainly focuses on model verification and is not intended for accurate merger-rate predictions, we do not account for the various post-numerical-merger time delays. These delays can be caused by physical processes such as sub-ckpc scale dynamical friction, scattering with stars, gravitational wave driven inspiral and triple MBH systems \citep[e.g.][]{Quinlan1996,Sesana2007b,Vasiliev2015,Dosopoulou2017,Bonetti2018}. We consider all the numerical mergers as true black hole merger events. Without any post-process selection, there are 25224 black holes and 4237 mergers in the \texttt{L35\_DF+drag\_4DM\_G} run, and 27693 black holes and 2349 mergers in the \texttt{L35\_NoDF\_4DM\_G} run down to $z=1.1$.

Figure \ref{fig:hist} shows the distribution of the binary parameters for the mergers in our simulations. For both simulations, there is at least one black hole around the seed mass for most mergers, but the peak does not lie at the exact seed mass. The most massive binary has a total mass of $3\times 10^8 M_\odot$. For the mass ratio $q$ between the two black holes in the binary, we see a peak at $\text{log}(q)=-0.3$, corresponding to pairs in which one BH is about two times larger than the other. Finally, we show the scatter of the two progenitor masses. The low mass end of the population deviates more from $q=1$, while the majority of same-mass mergers come from the $5\times 10^6 M_\odot\sim 5\times 10^7 M_\odot$ mass range.

Comparing with previous simulations such as \cite{Salcido2016,Katz2020}, we do not see as many cases of seed-seed mergers, but our distribution in q is similar to that shown in \cite{Weinberger2017} where the larger progenitor is a few times larger than the small progenitor. This is due to our larger black hole seed mass of $5\times10^5 M_\odot$ ($10^6 M_\odot$ in \cite{Weinberger2017}): the mass accretion in the early stage is proportional to $M_{\rm BH}^2$, and so during the time before the black hole mergers, our black holes accrete more mass compared to the simulations with smaller seeds. This explains why both of our black holes in the binaries are not peaked at the exact seed mass.


\subsection{Merger Rate Predictions}
\label{subsec:L35_rates}
We use the binary population shown in the previous section to predict the merger rate observed per year per unit redshift.
The merger rate per unit redshift per year is calculated as:
\begin{equation}
     \frac{dN}{dz\;dt} =  \frac{N(z)}{\Delta z V_{c,sim}} \frac{dz}{dt} \frac{dV_c(z)}{dz}\frac{1}{1+z},
\end{equation}
where $N(z)$ is the total number of mergers in the redshift bin $z$, $\Delta z$ is the width of the redshift bin, $V_{c,sim}$ is the comoving volume of our simulation box and $dV_c(z)$ is the comoving volume of the spherical shell corresponding to the $z$ bin. 

We compare our results against recent predictions from hydro-dynamical simulations of similar resolution, \cite{Salcido2016}, \cite{Katz2020} and \cite{Volonteri2020}. Here we briefly summarize relevant information about their merger catalogs. The Ref-L100N1504 simulation in the EAGLE suite used in \cite{Salcido2016} has an $2^3$ times larger simulation box and slightly higher resolution than our simulations. They seed $1.4\times 10^5M_\odot$ black holes in $1.4\times 10^{10}M_\odot$ halos. They adopt the reposition algorithm for black hole dynamics, but set a distance and relative speed upper limit on the repositioning to prevent black holes from jumping to satellites during fly-by encounters. We compare with their no-delay rate during the inspiral phase. The \textit{Illustris} simulation used in \cite{Katz2020} has a similar box size, resolution and BH dynamics to the Ref-L100N1504 simulation in EAGLE, except that their halo mass threshold for seeding BHs is $7\times 10^{10} M_\odot$. We compare against their ND model, in which mergers are also taken to occur at the numerical merger time without any delay processes. The Horizon-AGN simulation in \cite{Volonteri2020} is $4^3$ times larger than our simulation box, with $\sim 5$ times coarser mass resolution and a black hole seed mass of $10^5 M_\odot$. Instead of seeding BHs in halos above certain mass threshold, the seeding in \cite{Volonteri2020} is based on the local gas density and velocity dispersion, and seeding is stopped at $z=1.5$. For black hole dynamics, they apply dynamical friction from gas, but not from collisionless particles.

Figure \ref{fig:rates} shows our merger rate prediction in the \texttt{L35\_DF+drag\_4DM\_G} and  \texttt{L35\_NoDF\_4DM\_G} simulations. The \texttt{L35\_DF+drag\_4DM\_G} run predicts $\sim 2$ mergers per year of observation down to $z=1.1$, while the \texttt{L35\_NoDF\_4DM\_G} run predicts $\sim 1$. The merger rates from both simulations peak at $z\sim 2$. This factor-of-two difference between the two simulations is consistent with what we predicted in the $L_{\rm box} = 15$ Mpc$/h$ runs in Figure \ref{fig:merger_stats}. Although we did not run a $L_{\rm box} = 35$ Mpc$/h$ simulation with the repositioning model, we expect such a run to predict $5\sim 6$ mergers per year down to $z=1.1$ according to \ref{fig:merger_stats}.

Generally speaking, our simulations yield similar merger rates as the raw predictions from the previous works of comparable resolution. However, we still note some differences both in the overall rates and in the peak of the rates. We will now elaborate on the reasons for those discrepancies.

First, both of our simulations predict more mergers compared with the \cite{Katz2020} ND model prediction. This is surprising given that in the 15 Mpc$/h$ runs we saw $2\sim 3$ times more mergers when we used the reposition method like \cite{Katz2020} and \cite{Salcido2016} did, comparing to our DF+Drag model. Although \cite{Katz2020} cut out $\sim 25\%$ secondary seed mergers and binaries with extreme density profiles, their rate is still lower after adding the cut-out population. One major reason for the higher rate from our simulation compared to \cite{Katz2020} is the different seeding parameters we use: our minimum halo mass for seeding a black hole is $10^{10} M_\odot/h$, which is 5 times smaller compared with \cite{Katz2020}. Moreover, our seeds are a factor of 5 larger. Hence, we have a denser population of black holes in less-massive galaxies, which is likely to result in a higher merger rate even compared to the reposition model used in \textit{Illustris}.

Second, although the rates from EAGLE, Horozon-AGN and our \texttt{L35\_DF+drag\_4DM\_G} simulation cross over at $z\sim 2$, the slope of our merger rate is very different. \cite{Volonteri2020} predicts most mergers at $z\sim 3$, whereas the \cite{Salcido2016} rate peaks at $z\sim 1$. This difference can also be traced to the different seeding rate in the three simulations: in \cite{Salcido2016}, the seeding rate keeps increasing until $z\sim 0.1$, while we observe a drop in seeding rate at $z=3$ in our simulations. In \cite{Volonteri2020}, due to the different seeding mechanism, BH seeds form significantly earlier, leading to a peak in merger rate at a higher redshift. Hence the peak in the BH merger rate is strongly correlated with the peak in the BH seeding rate.

Finally, besides the effect due to different BH seed models on the merger rate, higher resolution can significantly increase the BH merger rates in the simulations. As was shown in previous work \citep[e.g.][]{Volonteri2020,Barausse2020}, dwarf galaxies in low-mass halos can have large numbers of (small mass) BH mergers, and so resolving such halos and galaxies can increase the BH merger rate significantly. The merger rate differences between high and low resolution and the associate choice for the seed models can lead to large differences in the predictions of merger rates than taking account DF in the BH dynamics. 

% \vspace{-0.5em}
\section{Conclusion}
% \vspace{-0.5em}
Recent advances in multimodal single-cell technology have enabled the simultaneous profiling of the transcriptome alongside other cellular modalities, leading to an increase in the availability of multimodal single-cell data. In this paper, we present \method{}, a multimodal transformer model for single-cell surface protein abundance from gene expression measurements. We combined the data with prior biological interaction knowledge from the STRING database into a richly connected heterogeneous graph and leveraged the transformer architectures to learn an accurate mapping between gene expression and surface protein abundance. Remarkably, \method{} achieves superior and more stable performance than other baselines on both 2021 and 2022 NeurIPS single-cell datasets.

\noindent\textbf{Future Work.}
% Our work is an extension of the model we implemented in the NeurIPS 2022 competition. 
Our framework of multimodal transformers with the cross-modality heterogeneous graph goes far beyond the specific downstream task of modality prediction, and there are lots of potentials to be further explored. Our graph contains three types of nodes. While the cell embeddings are used for predictions, the remaining protein embeddings and gene embeddings may be further interpreted for other tasks. The similarities between proteins may show data-specific protein-protein relationships, while the attention matrix of the gene transformer may help to identify marker genes of each cell type. Additionally, we may achieve gene interaction prediction using the attention mechanism.
% under adequate regulations. 
% We expect \method{} to be capable of much more than just modality prediction. Note that currently, we fuse information from different transformers with message-passing GNNs. 
To extend more on transformers, a potential next step is implementing cross-attention cross-modalities. Ideally, all three types of nodes, namely genes, proteins, and cells, would be jointly modeled using a large transformer that includes specific regulations for each modality. 

% insight of protein and gene embedding (diff task)

% all in one transformer

% \noindent\textbf{Limitations and future work}
% Despite the noticeable performance improvement by utilizing transformers with the cross-modality heterogeneous graph, there are still bottlenecks in the current settings. To begin with, we noticed that the performance variations of all methods are consistently higher in the ``CITE'' dataset compared to the ``GEX2ADT'' dataset. We hypothesized that the increased variability in ``CITE'' was due to both less number of training samples (43k vs. 66k cells) and a significantly more number of testing samples used (28k vs. 1k cells). One straightforward solution to alleviate the high variation issue is to include more training samples, which is not always possible given the training data availability. Nevertheless, publicly available single-cell datasets have been accumulated over the past decades and are still being collected on an ever-increasing scale. Taking advantage of these large-scale atlases is the key to a more stable and well-performing model, as some of the intra-cell variations could be common across different datasets. For example, reference-based methods are commonly used to identify the cell identity of a single cell, or cell-type compositions of a mixture of cells. (other examples for pretrained, e.g., scbert)


%\noindent\textbf{Future work.}
% Our work is an extension of the model we implemented in the NeurIPS 2022 competition. Now our framework of multimodal transformers with the cross-modality heterogeneous graph goes far beyond the specific downstream task of modality prediction, and there are lots of potentials to be further explored. Our graph contains three types of nodes. while the cell embeddings are used for predictions, the remaining protein embeddings and gene embeddings may be further interpreted for other tasks. The similarities between proteins may show data-specific protein-protein relationships, while the attention matrix of the gene transformer may help to identify marker genes of each cell type. Additionally, we may achieve gene interaction prediction using the attention mechanism under adequate regulations. We expect \method{} to be capable of much more than just modality prediction. Note that currently, we fuse information from different transformers with message-passing GNNs. To extend more on transformers, a potential next step is implementing cross-attention cross-modalities. Ideally, all three types of nodes, namely genes, proteins, and cells, would be jointly modeled using a large transformer that includes specific regulations for each modality. The self-attention within each modality would reconstruct the prior interaction network, while the cross-attention between modalities would be supervised by the data observations. Then, The attention matrix will provide insights into all the internal interactions and cross-relationships. With the linearized transformer, this idea would be both practical and versatile.

% \begin{acks}
% This research is supported by the National Science Foundation (NSF) and Johnson \& Johnson.
% \end{acks}
% \documentclass[11pt]{article}

%--------- Packages -----------
% \usepackage[margin=1in]{geometry}
\usepackage[margin=1in,footskip=0.25in]{geometry}
\usepackage{fullpage}
\usepackage{amssymb}
\usepackage{amsmath}
\usepackage{amsthm}
\usepackage{color}
\usepackage{enumitem}
\usepackage{url}
\usepackage{listings}
\usepackage{bbm}
\usepackage{matlab-prettifier}
\usepackage{mathtools}
\usepackage{caption}
\usepackage{algorithm}
\usepackage{algpseudocode}
\usepackage{subcaption}
\usepackage{float}
\usepackage{hyperref}

%-------------Notation--------------
%\usepackage{algorithm,algorithmic}
\usepackage{shorthands}
%----------Spacing-------------
\setlength{\topmargin}{-0.6 in}
\setlength{\headsep}{0.75 in}
\setlength{\parindent}{0 in}
\setlength{\parskip}{0.1 in}
\setlength{\hoffset}{0 in}



%----------Header--------------
\title{On a Relation Between the Rate-Distortion Function and \\Optimal Transport}
\author{Eric Lei}


%========= BEGIN DOCUMENT =========

\begin{document}
    \maketitle

    \section{Introduction}
    \subsection{Rate-Distortion}

    Let $X \sim P_X$ be the random variable we wish to compress, supported on $\mathcal{X}$. Let $\mathcal{Y}$ be the reproduction space, and $\dist: \mathcal{X} \times \mathcal{Y} \rightarrow \mathbb{R}_{\geq 0}$ be a distortion measure. The asymptotic limit on the minimum number of bits required to achieve average distortion $D$ is given by the rate-distortion function \cite{shannonRD, CoverThomas},  defined as
 %   \begin{definition} The rate-distortion function $R(D)$ of a source $P_X$ under distortion function $\dist$ is given by 
        \begin{equation}
    	    R(D) := \inf_{\substack{P_{Y|X}:  \EE_{P_{X,Y}}[\dist(X,Y)]  \leq D}} I(X;Y), 
    	    \label{eq:RD}
    	\end{equation} 
   % \end{definition} 
    where $P_{X,Y}=P_X P_{Y|X}$ is a joint distribution. Any rate-distortion pair $(R,D)$ satisfying $R > R(D)$ is achievable by some lossy source code, and no code can achieve a rate-distortion less than $R(D)$. It is important to note that $R(D)$ is achievable only under asymptotic blocklengths.
    

    Let $\DKL(\mu || \nu)$ be the Kullback-Leibler (KL) divergence, defined as $\EE_\mu\bracket{\log \frac{d\mu}{d\nu}}$ when the density $\frac{d\mu}{d\nu}$ exists and $+\infty$ otherwise. $R(D)$ has the following alternate form \cite[Ch.~10]{CoverThomas},
    \begin{equation}
    R(D) = \inf_{Q_Y} \inf_{\substack{P_{Y|X}:  \EE_{P_{X,Y}}[\dist(X,Y)]  \leq D}} \DKL(P_{X,Y}||P_X \otimes Q_Y).
    \label{eq:RD_alt}
    \end{equation}
    Due to the convex and strictly decreasing properties \cite{CoverThomas} of $R(D)$, it suffices to fix some $\lambda > 0$, and solve 
    \begin{equation}
        \inf_{Q_Y} \inf_{P_{Y|X}}\DKL(P_{X,Y}|| P_X \otimes Q_Y) + \lambda \mathop{\EE}_{P_{X,Y}}[\dist(X,Y)].
        \label{eq:RD_alt_regl}
    \end{equation}

    

    % \begin{lemma}[Double-Minimization Form, cf. {\cite[Ch.~10]{CoverThomas}}, {\cite{Yeung2002AFC}}]
    %     The minimizers $P^{(\beta)}_{Y|X}$, $Q_Y^{(\beta)}$ of
    %     \begin{equation}
	   %   \RD(\beta):=\inf_{Q_Y} \inf_{P_{Y|X}   } \DKL(P_{X,Y}|| P_X \otimes Q_Y) + \beta \mathop{\EE}_{P_{X,Y}}[\dist(X,Y)]
	   %   \label{eq:double}
	   %  \end{equation}
	   %  yield a unique point $R_\beta = \DKL(P_XP^{(\beta)}_{Y|X} || P_X \otimes Q_Y^{(\beta)})$ and $D_\beta = \EE_{P_X P^{(\beta)}_{Y|X}} [\dist(X,Y)]$ on the positive-rate regime of the rate-distortion curve, i.e. $R(D_\beta) = R_\beta$, such that $D_\beta < D_{\mathrm{max}}$ where $R(D_{\mathrm{max}}) = 0$.
    %     \begin{proof}
    %     See \cite[Lemma~10.8.1]{CoverThomas}.
    %     \end{proof}
    % \end{lemma}
	
    In discrete settings,  the optimizers $Q_Y, P_{Y|X}$ satisfy
    \begin{align}
        \frac{dP_{Y|X=x}}{dQ_Y}(x,y) &= \frac{e^{-\lambda \dist(x,y)}}{\int_{\mathcal{Y}} e^{-\lambda \dist(x,\tilde{y})}dQ_Y} , \label{eq:BA_1} \\
        Q_Y &= \int_{\mathcal{X}}  dP_{Y|X} dP_X. 
        \label{eq:BA_2}
    \end{align}

    This solution corresponds to a point on $R(D)$ corresponding to $\lambda$. The Blahut-Arimoto (BA) algorithm \cite{blahut, arimoto} solves \eqref{eq:RD_alt} by alternating steps on $P_{Y|X}$ and $Q_Y$ until convergence. Similarly, NERD \cite{NERD} provides a variant for continuous sources when only samples are available. Sweeping over $\lambda$ gives the entire rate-distortion curve.
    

    \subsection{Optimal Transport}
    Optimal transport theory has two formulations; the Monge problem, and the Kantorovich relaxation. Intuitively, suppose we have two sets of points $\mathfrak{X} = \{x_1,\dots,x_n\}$ and $\mathfrak{Y} = \{y_1,\dots,y_n\}$. The Monge problem seeks to find a matching (i.e., a bijection) between $\mathfrak{X}$ and $\mathfrak{Y}$ that minimizes the average pairwise cost. The Kantovorich problem relaxes the bijection requirement by allowing each $x_i$ to be soft-matched to all $y_j$'s, i.e., each $x_i$ can send some mass to each $y_j$ such that the total mass sent sums to 1. In the discrete case, they have the same optimal solution, which is actually given by a hard-matching, i.e., a bijection. In the case for general measures, they are only equivalent under certain conditions. We will only discuss the Kantorovich case since that is most relevant.

    The Kantorovich problem seeks to find the minimum distortion coupling between two probability measures $\mu$ and $\nu$,
    \begin{equation}
        W(\mu, \nu) := \inf_{\substack{\pi \in \Pi(\mu, \nu)}} \EE_{X,Y\sim \pi}[\dist(X,Y)],
    \end{equation}
    where $\pi$ is a joint distribution that marginalizes to $\mu$ and $\nu$. The Kantorovich problem can be regularized with an entropy term, 
    \begin{equation}
        S_\epsilon(\mu, \nu) := \inf_{\substack{\pi \in \Pi(\mu, \nu)}} \EE_{\pi}[\dist(X,Y)] + \epsilon \DKL(\pi||\mu \otimes \nu),
        \label{eq:entropic_OT}
    \end{equation}
    which is known as entropy-regularized optimal transport, with regularization $\epsilon > 0$. For discrete measures $\mu,\nu$, \eqref{eq:entropic_OT} can be solved efficiently using the Sinkhorn-Knopp algorithm \cite{sinkhorn, knopp_sinkhorn}.

    A brief introduction to optimal transport theory can be found in \cite{CourseNotesOT, ComputationalOT}. 

    \section{Optimal Transport and the Rate-Distortion Function}
    \subsection{Equivalence of Extremal Sinkhorn Divergence and Rate Distortion}
    We first show that entropic OT can be used to upper bound $R(D)$. First, observe that the inner minimization problem in \eqref{eq:RD_alt_regl} looks similar to the entropic OT problem. Let us define 
    \begin{align}
        S(D) :=& \inf_{Q_Y} \inf_{\substack{\pi_{XY} \in \Pi(P_X, Q_Y): \\ \EE_{\pi_{X,Y}}[\dist(X,Y)]  \leq D }} \DKL(\pi_{X,Y}||P_X \otimes Q_Y) \label{eq:SD_1}\\ 
        =&\inf_{Q_Y} \inf_{\substack{P_{Y|X}: \\ \EE_{P_{X,Y}}[\dist(X,Y)]  \leq D \\ Q_Y = \int_{\mathcal{X}} dP_{Y|X}P_X}} \DKL(P_{X,Y}||P_X \otimes Q_Y), \label{eq:SD_2}
    \end{align}
    which we call the \emph{Sinkhorn-distortion function}. Similar to $R(D)$, we can trace out $S(D)$ by sweeping over $\lambda > 0$, and solving the inner minimization \eqref{eq:entropic_OT} by setting  $\epsilon = \frac{1}{\lambda}$, and then optimizing over all $Q_Y$ for the outer minimization, which is convex in $Q_Y$ \cite{feydy2019interpolating}. We call this \texttt{SinkhornRD}. A variant of the Sinkhorn-distortion function is often used to solve generative modeling tasks with Sinkhorn divergences \cite{sinkhornGAN, salimans2018improving, shen2020sinkhorn}, where one wishes to find some $Q_Y \approx P_X$ by solving $\min_{Q_Y} S_\epsilon (P_x, Q_Y)$. The neural methods parallel NERD.

	    
	    \begin{proposition}[Sinkhorn Distortion Upper Bound]
	    \label{prop:SD_UB}
	    Given source $P_X$ on $\mathcal{X}$, reconstruction space $\mathcal{Y}$, and distortion measure $\dist$,  
	    \begin{equation}
	        R(D) \leq S(D).
	        \label{eq:RD-Sinkhorn}
	    \end{equation}
	    \end{proposition}
	   \begin{proof}
            The inner minimization problem of $R(D)$ in \eqref{eq:RD_alt} only has a marginal constraint on $P_X$, whereas the inner minimization of $S(D)$ in  \eqref{eq:SD_2} has an additional marginal constraint on $Q_Y$ as well.
    	\end{proof}
        
        

    	
    	% \begin{figure*}[t]
     %         \centering
     %         \begin{subfigure}[b]{0.475\textwidth}
     %             \centering
     %             \includegraphics[width=0.9\textwidth]{figures/RDcurve_discrete_bad.pdf}
     %             \caption{$R(D)$ (via Blahut-Arimoto) and entropic OT upper bound on a discrete source with 6 atoms.}
     %             \label{fig:simple}
     %         \end{subfigure}
     %         \hfill
     %         \begin{subfigure}[b]{0.475\textwidth}
     %             \centering
     %             \includegraphics[width=0.9\textwidth]{figures/RD_minmax_mnist_RD_sinkhorn.pdf}
     %             \caption{$R(D)$ (via NERD) and entropic OT upper bound on MNIST images.}  
     %            \label{fig:mnist_sinkhorn}
     %         \end{subfigure}
     %        \caption{Comparison of the RD-Sinkhorn objective with the true rate-distortion objective. }
     %     \end{figure*}
    	
    	% Under discrete settings, \eqref{eq:entropic_OT} can be solved efficiently via Sinkhorn iterations \cite{lightspeed}. We call the upper bound in \eqref{eq:RD-Sinkhorn} \textit{RD-Sinkhorn}. In order to illustrate the behavior of RD-Sinkhorn, we first examine the rate-distortion curve of a discretized Gaussian source over 6 atoms, shown in Fig.~\ref{fig:simple}. In such a setting, we can compute the true rate-distortion curve and RD-Sinkhorn easily and exactly, via Blahut-Arimoto and Sinkhorn iterations respectively.  Although RD-Sinkhorn traces a curve that is not convex, we see that for large portions of the rate-distortion curve, the gap is small and RD-Sinkhorn empirically provides a good estimate of the true rate-distortion function. 
     
    	% We then apply RD-Sinkhorn to an MNIST source by parametrizing $Q_Y$ with a generator neural network, similar to the work in \cite{sinkhornGAN}. To estimate $R(D)$ in this setting, we use the neural estimator NERD \cite{NERD}, and compare the curves in Fig.~\ref{fig:mnist_sinkhorn}.


        Next, we show that without further assumptions, $R(D)$ and $S(D)$ are equivalent.
        \begin{theorem}[Sinkhorn-Rate-Distortion Equivalence]
            For any source $P_X$ and distortion function $\dist: \mathcal{X} \times \mathcal{Y} \rightarrow \mathbb{R}_{\geq 0}$, it holds that
            \begin{equation}
                R(D) = S(D).
            \end{equation}
        \end{theorem}
        \begin{proof}
            We know that the optimizing distributions for $R(D)$ must satisfy \eqref{eq:BA_1} and \eqref{eq:BA_2} simultaneously. To show that $S(D)$ achieves the same objective as $R(D)$ on the same $P_X$ and distortion measure, it suffices to show that the $R(D)$-optimal $Q_Y^*$ and $P_{Y|X}^*$ are feasible for $S(D)$. From \cite[Ch.~4, Prop.~4.3]{ComputationalOT}, the optimal coupling in entropic OT is unique and has the form 
            \begin{equation}
            \frac{d\pi}{dP_X dQ_Y} (x,y) = u(x) e^{-\lambda \dist(x,y)} v(y)
            \end{equation}
            where $u(x), v(y)$ are dual variables that ensure valid couplings. The $R(D)$-optimal joint distribution $P_XP_{Y|X}^*$, which is guaranteed to be a coupling between $P_X$ and $Q_Y^*$ due to \eqref{eq:BA_2}, indeed has the form
            \begin{equation}
                \frac{dP_XP_{Y|X}^*}{dP_X dQ_Y^*}(x,y) = \frac{1}{\int_{\mathcal{Y}}e^{-\lambda \dist(x,y')}dQ_Y^*} \cdot e^{-\lambda \dist(x,y)} \cdot 1
            \end{equation}
           where the first term only depends on $x$ and the last term only depends on $y$.
            Since from Prop.~\ref{prop:SD_UB}, $R(D)$ is a lower bound of $S(D)$, we are done.
        \end{proof}
        \begin{remark}
            This result implies that (i) Sinkhorn generative models are equivalent to solving $R(D)$, and (ii) the proposed \emph{\texttt{SinkhornRD}} algorithm can also solve $R(D)$ and is an alternative to Blahut-Arimoto. \textcolor{blue}{Perhaps some implications of the rate-distortion optimal codebook from the lens of entropic OT?}
        \end{remark}

        \begin{figure}
            \centering
            \includegraphics[width=0.5\linewidth]{figures/5atoms_comparison.pdf}
            \caption{Equivalence of $S(D)$ and $R(D)$ on a discrete source with $\dist(x,y)=(x-y)^2$. }
            \label{fig:SD-RD}
        \end{figure}

        We numerically verify the equivalence in Fig.~\ref{fig:SD-RD} on a discrete source with 5 atoms under squared-error distortion. For $R(D)$, we use Blahut-Arimoto, and for $S(D)$, we solve the convex problem using sequential quadratic programming \cite{SLSQP, 2020SciPy} with $Q \mapsto S_{\epsilon}(P_X, Q)$ as the objective function, showing that the two different objectives result in the same function.

        \subsection{Entropic OT and the Rate Function}
        \begin{proposition}
            We have that 
            \begin{equation}
                R(D, Q_Y) = S(D, Q_Y)
            \end{equation}
            if and only if 
            \begin{equation}
                \int_{\mathcal{X}} \frac{e^{-\lambda \dist(x,y)}}{\int_{\mathcal{Y}} e^{-\lambda \dist(x,y')} dQ_Y} dP_X = 1, \quad \forall y \in \mathcal{Y}.
                \label{eq:rate_eq_cond}
            \end{equation}
        \end{proposition}
        \begin{proof}
            Note that for a given $Q_Y$, the solution to $R(D,Q_Y)$ is given by \eqref{eq:BA_1}. The optimizing $P_{Y|X}$, however, does not necessarily produce a valid coupling for the joint $P_X P_{Y|X}$ (we only know this holds for the $R(D)$-optimal $Q_Y$). In order for $P_XP_{Y|X}$ to be a valid coupling, the $Q_Y$ marginal constraint must be satisfied, i.e. $\forall y \in \mathcal{Y}$,
            \begin{align}
                \int_{\mathcal{X}} dP_{Y|X}(y) dP_X = dQ_Y(y) &\iff \int_{\mathcal{X}} \frac{e^{-\lambda \dist(x,y)}}{\int_{\mathcal{Y}} e^{-\lambda \dist(x,y')} dQ_Y} dQ_Y(y) dP_X(x) = dQ_Y(y)\\
                & \iff \int_{\mathcal{X}} \frac{e^{-\lambda \dist(x,y)}}{\int_{\mathcal{Y}} e^{-\lambda \dist(x,y')} dQ_Y} dP_X = 1.
            \end{align}
            When a coupling is satisfied, $P_XP_{Y|X}$ is feasible for $S(D, Q_Y)$, and since $R(D, Q_Y) \leq S(D, Q_Y)$, we are done.
        \end{proof}
        \begin{corollary}
            Equivalent when $P_X$ and $Q_Y$ are uniform discrete and the distortion matrix is symmetric (rows and columns are permutations of eachother). Note that \eqref{eq:BA_1} immediately provides the optimal Sinkhorn coupling in this case (matching two sets of points with symmetric cost). Also, note that this induces a weakly symmetric $R(D)$-optimal channel.
        \end{corollary}
        \begin{proof}
            When $P_X = \text{Unif}(\mathcal{X})$, $Q_Y = \text{Unif}(\mathcal{Y})$, \eqref{eq:rate_eq_cond} evaluates to 
            \begin{equation}
                \sum_{x \in \mathcal{X}} 
            \end{equation}
        \end{proof}
        

    \section{Optimal Transport and Channel Capacity}
    Given a channel $p(y|x)$, the channel capacity \cite{shannon48} describes its maximal rate of communication, given by 
    \begin{equation}
        C = \max_{r(x)} I(X;Y).
        \label{eq:capacity}
    \end{equation}
    We show that \eqref{eq:capacity} can also be interpreted as an extremal optimal transport problem. 
    \begin{theorem}
        For any channel $p(y|x)$, we have that 
        \begin{equation}
            C = 2\cdot  \max_{r(x)} S_{\frac{1}{2}}(r(x), r(y)),
        \end{equation}
        where $r(y) := \sum_{x \in \mathcal{X}} p(y|x)r(x)$, and the cost function that induces the Sinkhorn problem in \eqref{eq:entropic_OT} is given by $\dist(x,y) = - \log \frac{p(y|x)}{r(y)}$. 
        \begin{proof}
            Using \cite[Lemma.~10.8.1]{CoverThomas}, we have that 
            \begin{equation}
                C = \max_{r(x)} \min_{q(x|y)} \EE_{r(x)p(y|x)} \bracket*{-\log \frac{q(x|y)}{r(x)}},
                \label{eq:capacity_alt}
            \end{equation}
            with the optimal $q^*(x|y)$ satisfying $q^*(x|y)r(y) = r(x)p(y|x)$. Hence, the inner problem in \eqref{eq:capacity_alt} under $q^*(x|y)$ can be rewritten as 
            \begin{align}
                \EE_{r(x)p(y|x)}\bracket*{-\log \frac{q^*(x|y)}{r(x)}} &= 2\EE_{r(y)q^*(x|y)} \bracket*{-\log \frac{p(y|x)}{r(y)}} + \EE_{r(y)q^*(x|y)} \bracket*{\log \frac{q^*(x|y)}{r(x)}} \\
                &= 2\EE_{r(y)q^*(x|y)} \bracket*{\dist(x,y)} + \DKL(r(y)q^*(x|y)||r(y)r(x)). \label{eq:optimal_sinkhorn}
            \end{align}
            To show that \eqref{eq:optimal_sinkhorn} is equal to $2 \cdot S_{\frac{1}{2}}(r(x), r(y))$, we again use the fact from \cite[Ch.~4]{ComputationalOT} that the optimal Sinkhorn coupling is unique and has the form $\pi(x,y) = u(x) e^{-2 \dist()$
        \end{proof}
    \end{theorem}
    

    

    \section{Applications}
    Potentially:
    \begin{enumerate}
        \item Optimal transport lens of information theory
        \item Operational meaning of codes, coding for optimal transport/matching?
        \item Connecting generalization bounds in learning theory that use optimal transport and rate-distortion theory
    \end{enumerate}


\bibliographystyle{ieeetr} 
\bibliography{ref}



\end{document}




\section*{Acknowledgements}
We thank Marta Volonteri for discussions on the merger rate comparisons. The simulations were performed on the Bridges and Vera clusters at the Pittsburgh Super-computing Center (PSC). TDM acknowledges funding from NSF ACI-1614853, NSF AST-1616168, NASA ATP 19-ATP19-0084 NASA ATP 80NSSC18K101, NASA ATP NNX17AK56G, and 80NSSC20K0519.
SPB was supported by NSF grant AST-1817256.


%%%%%%%%%%%%%%%%%%%% REFERENCES %%%%%%%%%%%%%%%%%%

\bibliographystyle{mnras}
\bibliography{main}

%%%%%%%%%%%%%%%%%%%%%%%%%%%%%%%%%%%%%%%%%%%%%%%%%%

%%%%%%%%%%%%%%%%% APPENDICES %%%%%%%%%%%%%%%%%%%%%

\appendix
\chapter{Supplementary Material}
\label{appendix}

In this appendix, we present supplementary material for the techniques and
experiments presented in the main text.

\section{Baseline Results and Analysis for Informed Sampler}
\label{appendix:chap3}

Here, we give an in-depth
performance analysis of the various samplers and the effect of their
hyperparameters. We choose hyperparameters with the lowest PSRF value
after $10k$ iterations, for each sampler individually. If the
differences between PSRF are not significantly different among
multiple values, we choose the one that has the highest acceptance
rate.

\subsection{Experiment: Estimating Camera Extrinsics}
\label{appendix:chap3:room}

\subsubsection{Parameter Selection}
\paragraph{Metropolis Hastings (\MH)}

Figure~\ref{fig:exp1_MH} shows the median acceptance rates and PSRF
values corresponding to various proposal standard deviations of plain
\MH~sampling. Mixing gets better and the acceptance rate gets worse as
the standard deviation increases. The value $0.3$ is selected standard
deviation for this sampler.

\paragraph{Metropolis Hastings Within Gibbs (\MHWG)}

As mentioned in Section~\ref{sec:room}, the \MHWG~sampler with one-dimensional
updates did not converge for any value of proposal standard deviation.
This problem has high correlation of the camera parameters and is of
multi-modal nature, which this sampler has problems with.

\paragraph{Parallel Tempering (\PT)}

For \PT~sampling, we took the best performing \MH~sampler and used
different temperature chains to improve the mixing of the
sampler. Figure~\ref{fig:exp1_PT} shows the results corresponding to
different combination of temperature levels. The sampler with
temperature levels of $[1,3,27]$ performed best in terms of both
mixing and acceptance rate.

\paragraph{Effect of Mixture Coefficient in Informed Sampling (\MIXLMH)}

Figure~\ref{fig:exp1_alpha} shows the effect of mixture
coefficient ($\alpha$) on the informed sampling
\MIXLMH. Since there is no significant different in PSRF values for
$0 \le \alpha \le 0.7$, we chose $0.7$ due to its high acceptance
rate.


% \end{multicols}

\begin{figure}[h]
\centering
  \subfigure[MH]{%
    \includegraphics[width=.48\textwidth]{figures/supplementary/camPose_MH.pdf} \label{fig:exp1_MH}
  }
  \subfigure[PT]{%
    \includegraphics[width=.48\textwidth]{figures/supplementary/camPose_PT.pdf} \label{fig:exp1_PT}
  }
\\
  \subfigure[INF-MH]{%
    \includegraphics[width=.48\textwidth]{figures/supplementary/camPose_alpha.pdf} \label{fig:exp1_alpha}
  }
  \mycaption{Results of the `Estimating Camera Extrinsics' experiment}{PRSFs and Acceptance rates corresponding to (a) various standard deviations of \MH, (b) various temperature level combinations of \PT sampling and (c) various mixture coefficients of \MIXLMH sampling.}
\end{figure}



\begin{figure}[!t]
\centering
  \subfigure[\MH]{%
    \includegraphics[width=.48\textwidth]{figures/supplementary/occlusionExp_MH.pdf} \label{fig:exp2_MH}
  }
  \subfigure[\BMHWG]{%
    \includegraphics[width=.48\textwidth]{figures/supplementary/occlusionExp_BMHWG.pdf} \label{fig:exp2_BMHWG}
  }
\\
  \subfigure[\MHWG]{%
    \includegraphics[width=.48\textwidth]{figures/supplementary/occlusionExp_MHWG.pdf} \label{fig:exp2_MHWG}
  }
  \subfigure[\PT]{%
    \includegraphics[width=.48\textwidth]{figures/supplementary/occlusionExp_PT.pdf} \label{fig:exp2_PT}
  }
\\
  \subfigure[\INFBMHWG]{%
    \includegraphics[width=.5\textwidth]{figures/supplementary/occlusionExp_alpha.pdf} \label{fig:exp2_alpha}
  }
  \mycaption{Results of the `Occluding Tiles' experiment}{PRSF and
    Acceptance rates corresponding to various standard deviations of
    (a) \MH, (b) \BMHWG, (c) \MHWG, (d) various temperature level
    combinations of \PT~sampling and; (e) various mixture coefficients
    of our informed \INFBMHWG sampling.}
\end{figure}

%\onecolumn\newpage\twocolumn
\subsection{Experiment: Occluding Tiles}
\label{appendix:chap3:tiles}

\subsubsection{Parameter Selection}

\paragraph{Metropolis Hastings (\MH)}

Figure~\ref{fig:exp2_MH} shows the results of
\MH~sampling. Results show the poor convergence for all proposal
standard deviations and rapid decrease of AR with increasing standard
deviation. This is due to the high-dimensional nature of
the problem. We selected a standard deviation of $1.1$.

\paragraph{Blocked Metropolis Hastings Within Gibbs (\BMHWG)}

The results of \BMHWG are shown in Figure~\ref{fig:exp2_BMHWG}. In
this sampler we update only one block of tile variables (of dimension
four) in each sampling step. Results show much better performance
compared to plain \MH. The optimal proposal standard deviation for
this sampler is $0.7$.

\paragraph{Metropolis Hastings Within Gibbs (\MHWG)}

Figure~\ref{fig:exp2_MHWG} shows the result of \MHWG sampling. This
sampler is better than \BMHWG and converges much more quickly. Here
a standard deviation of $0.9$ is found to be best.

\paragraph{Parallel Tempering (\PT)}

Figure~\ref{fig:exp2_PT} shows the results of \PT sampling with various
temperature combinations. Results show no improvement in AR from plain
\MH sampling and again $[1,3,27]$ temperature levels are found to be optimal.

\paragraph{Effect of Mixture Coefficient in Informed Sampling (\INFBMHWG)}

Figure~\ref{fig:exp2_alpha} shows the effect of mixture
coefficient ($\alpha$) on the blocked informed sampling
\INFBMHWG. Since there is no significant different in PSRF values for
$0 \le \alpha \le 0.8$, we chose $0.8$ due to its high acceptance
rate.



\subsection{Experiment: Estimating Body Shape}
\label{appendix:chap3:body}

\subsubsection{Parameter Selection}
\paragraph{Metropolis Hastings (\MH)}

Figure~\ref{fig:exp3_MH} shows the result of \MH~sampling with various
proposal standard deviations. The value of $0.1$ is found to be
best.

\paragraph{Metropolis Hastings Within Gibbs (\MHWG)}

For \MHWG sampling we select $0.3$ proposal standard
deviation. Results are shown in Fig.~\ref{fig:exp3_MHWG}.


\paragraph{Parallel Tempering (\PT)}

As before, results in Fig.~\ref{fig:exp3_PT}, the temperature levels
were selected to be $[1,3,27]$ due its slightly higher AR.

\paragraph{Effect of Mixture Coefficient in Informed Sampling (\MIXLMH)}

Figure~\ref{fig:exp3_alpha} shows the effect of $\alpha$ on PSRF and
AR. Since there is no significant differences in PSRF values for $0 \le
\alpha \le 0.8$, we choose $0.8$.


\begin{figure}[t]
\centering
  \subfigure[\MH]{%
    \includegraphics[width=.48\textwidth]{figures/supplementary/bodyShape_MH.pdf} \label{fig:exp3_MH}
  }
  \subfigure[\MHWG]{%
    \includegraphics[width=.48\textwidth]{figures/supplementary/bodyShape_MHWG.pdf} \label{fig:exp3_MHWG}
  }
\\
  \subfigure[\PT]{%
    \includegraphics[width=.48\textwidth]{figures/supplementary/bodyShape_PT.pdf} \label{fig:exp3_PT}
  }
  \subfigure[\MIXLMH]{%
    \includegraphics[width=.48\textwidth]{figures/supplementary/bodyShape_alpha.pdf} \label{fig:exp3_alpha}
  }
\\
  \mycaption{Results of the `Body Shape Estimation' experiment}{PRSFs and
    Acceptance rates corresponding to various standard deviations of
    (a) \MH, (b) \MHWG; (c) various temperature level combinations
    of \PT sampling and; (d) various mixture coefficients of the
    informed \MIXLMH sampling.}
\end{figure}


\subsection{Results Overview}
Figure~\ref{fig:exp_summary} shows the summary results of the all the three
experimental studies related to informed sampler.
\begin{figure*}[h!]
\centering
  \subfigure[Results for: Estimating Camera Extrinsics]{%
    \includegraphics[width=0.9\textwidth]{figures/supplementary/camPose_ALL.pdf} \label{fig:exp1_all}
  }
  \subfigure[Results for: Occluding Tiles]{%
    \includegraphics[width=0.9\textwidth]{figures/supplementary/occlusionExp_ALL.pdf} \label{fig:exp2_all}
  }
  \subfigure[Results for: Estimating Body Shape]{%
    \includegraphics[width=0.9\textwidth]{figures/supplementary/bodyShape_ALL.pdf} \label{fig:exp3_all}
  }
  \label{fig:exp_summary}
  \mycaption{Summary of the statistics for the three experiments}{Shown are
    for several baseline methods and the informed samplers the
    acceptance rates (left), PSRFs (middle), and RMSE values
    (right). All results are median results over multiple test
    examples.}
\end{figure*}

\subsection{Additional Qualitative Results}

\subsubsection{Occluding Tiles}
In Figure~\ref{fig:exp2_visual_more} more qualitative results of the
occluding tiles experiment are shown. The informed sampling approach
(\INFBMHWG) is better than the best baseline (\MHWG). This still is a
very challenging problem since the parameters for occluded tiles are
flat over a large region. Some of the posterior variance of the
occluded tiles is already captured by the informed sampler.

\begin{figure*}[h!]
\begin{center}
\centerline{\includegraphics[width=0.95\textwidth]{figures/supplementary/occlusionExp_Visual.pdf}}
\mycaption{Additional qualitative results of the occluding tiles experiment}
  {From left to right: (a)
  Given image, (b) Ground truth tiles, (c) OpenCV heuristic and most probable estimates
  from 5000 samples obtained by (d) MHWG sampler (best baseline) and
  (e) our INF-BMHWG sampler. (f) Posterior expectation of the tiles
  boundaries obtained by INF-BMHWG sampling (First 2000 samples are
  discarded as burn-in).}
\label{fig:exp2_visual_more}
\end{center}
\end{figure*}

\subsubsection{Body Shape}
Figure~\ref{fig:exp3_bodyMeshes} shows some more results of 3D mesh
reconstruction using posterior samples obtained by our informed
sampling \MIXLMH.

\begin{figure*}[t]
\begin{center}
\centerline{\includegraphics[width=0.75\textwidth]{figures/supplementary/bodyMeshResults.pdf}}
\mycaption{Qualitative results for the body shape experiment}
  {Shown is the 3D mesh reconstruction results with first 1000 samples obtained
  using the \MIXLMH informed sampling method. (blue indicates small
  values and red indicates high values)}
\label{fig:exp3_bodyMeshes}
\end{center}
\end{figure*}

\clearpage



\section{Additional Results on the Face Problem with CMP}

Figure~\ref{fig:shading-qualitative-multiple-subjects-supp} shows inference results for reflectance maps, normal maps and lights for randomly chosen test images, and Fig.~\ref{fig:shading-qualitative-same-subject-supp} shows reflectance estimation results on multiple images of the same subject produced under different illumination conditions. CMP is able to produce estimates that are closer to the groundtruth across different subjects and illumination conditions.

\begin{figure*}[h]
  \begin{center}
  \centerline{\includegraphics[width=1.0\columnwidth]{figures/face_cmp_visual_results_supp.pdf}}
  \vspace{-1.2cm}
  \end{center}
	\mycaption{A visual comparison of inference results}{(a)~Observed images. (b)~Inferred reflectance maps. \textit{GT} is the photometric stereo groundtruth, \textit{BU} is the Biswas \etal (2009) reflectance estimate and \textit{Forest} is the consensus prediction. (c)~The variance of the inferred reflectance estimate produced by \MTD (normalized across rows).(d)~Visualization of inferred light directions. (e)~Inferred normal maps.}
	\label{fig:shading-qualitative-multiple-subjects-supp}
\end{figure*}


\begin{figure*}[h]
	\centering
	\setlength\fboxsep{0.2mm}
	\setlength\fboxrule{0pt}
	\begin{tikzpicture}

		\matrix at (0, 0) [matrix of nodes, nodes={anchor=east}, column sep=-0.05cm, row sep=-0.2cm]
		{
			\fbox{\includegraphics[width=1cm]{figures/sample_3_4_X.png}} &
			\fbox{\includegraphics[width=1cm]{figures/sample_3_4_GT.png}} &
			\fbox{\includegraphics[width=1cm]{figures/sample_3_4_BISWAS.png}}  &
			\fbox{\includegraphics[width=1cm]{figures/sample_3_4_VMP.png}}  &
			\fbox{\includegraphics[width=1cm]{figures/sample_3_4_FOREST.png}}  &
			\fbox{\includegraphics[width=1cm]{figures/sample_3_4_CMP.png}}  &
			\fbox{\includegraphics[width=1cm]{figures/sample_3_4_CMPVAR.png}}
			 \\

			\fbox{\includegraphics[width=1cm]{figures/sample_3_5_X.png}} &
			\fbox{\includegraphics[width=1cm]{figures/sample_3_5_GT.png}} &
			\fbox{\includegraphics[width=1cm]{figures/sample_3_5_BISWAS.png}}  &
			\fbox{\includegraphics[width=1cm]{figures/sample_3_5_VMP.png}}  &
			\fbox{\includegraphics[width=1cm]{figures/sample_3_5_FOREST.png}}  &
			\fbox{\includegraphics[width=1cm]{figures/sample_3_5_CMP.png}}  &
			\fbox{\includegraphics[width=1cm]{figures/sample_3_5_CMPVAR.png}}
			 \\

			\fbox{\includegraphics[width=1cm]{figures/sample_3_6_X.png}} &
			\fbox{\includegraphics[width=1cm]{figures/sample_3_6_GT.png}} &
			\fbox{\includegraphics[width=1cm]{figures/sample_3_6_BISWAS.png}}  &
			\fbox{\includegraphics[width=1cm]{figures/sample_3_6_VMP.png}}  &
			\fbox{\includegraphics[width=1cm]{figures/sample_3_6_FOREST.png}}  &
			\fbox{\includegraphics[width=1cm]{figures/sample_3_6_CMP.png}}  &
			\fbox{\includegraphics[width=1cm]{figures/sample_3_6_CMPVAR.png}}
			 \\
	     };

       \node at (-3.85, -2.0) {\small Observed};
       \node at (-2.55, -2.0) {\small `GT'};
       \node at (-1.27, -2.0) {\small BU};
       \node at (0.0, -2.0) {\small MP};
       \node at (1.27, -2.0) {\small Forest};
       \node at (2.55, -2.0) {\small \textbf{CMP}};
       \node at (3.85, -2.0) {\small Variance};

	\end{tikzpicture}
	\mycaption{Robustness to varying illumination}{Reflectance estimation on a subject images with varying illumination. Left to right: observed image, photometric stereo estimate (GT)
  which is used as a proxy for groundtruth, bottom-up estimate of \cite{Biswas2009}, VMP result, consensus forest estimate, CMP mean, and CMP variance.}
	\label{fig:shading-qualitative-same-subject-supp}
\end{figure*}

\clearpage

\section{Additional Material for Learning Sparse High Dimensional Filters}
\label{sec:appendix-bnn}

This part of supplementary material contains a more detailed overview of the permutohedral
lattice convolution in Section~\ref{sec:permconv}, more experiments in
Section~\ref{sec:addexps} and additional results with protocols for
the experiments presented in Chapter~\ref{chap:bnn} in Section~\ref{sec:addresults}.

\vspace{-0.2cm}
\subsection{General Permutohedral Convolutions}
\label{sec:permconv}

A core technical contribution of this work is the generalization of the Gaussian permutohedral lattice
convolution proposed in~\cite{adams2010fast} to the full non-separable case with the
ability to perform back-propagation. Although, conceptually, there are minor
differences between Gaussian and general parameterized filters, there are non-trivial practical
differences in terms of the algorithmic implementation. The Gauss filters belong to
the separable class and can thus be decomposed into multiple
sequential one dimensional convolutions. We are interested in the general filter
convolutions, which can not be decomposed. Thus, performing a general permutohedral
convolution at a lattice point requires the computation of the inner product with the
neighboring elements in all the directions in the high-dimensional space.

Here, we give more details of the implementation differences of separable
and non-separable filters. In the following, we will explain the scalar case first.
Recall, that the forward pass of general permutohedral convolution
involves 3 steps: \textit{splatting}, \textit{convolving} and \textit{slicing}.
We follow the same splatting and slicing strategies as in~\cite{adams2010fast}
since these operations do not depend on the filter kernel. The main difference
between our work and the existing implementation of~\cite{adams2010fast} is
the way that the convolution operation is executed. This proceeds by constructing
a \emph{blur neighbor} matrix $K$ that stores for every lattice point all
values of the lattice neighbors that are needed to compute the filter output.

\begin{figure}[t!]
  \centering
    \includegraphics[width=0.6\columnwidth]{figures/supplementary/lattice_construction}
  \mycaption{Illustration of 1D permutohedral lattice construction}
  {A $4\times 4$ $(x,y)$ grid lattice is projected onto the plane defined by the normal
  vector $(1,1)^{\top}$. This grid has $s+1=4$ and $d=2$ $(s+1)^{d}=4^2=16$ elements.
  In the projection, all points of the same color are projected onto the same points in the plane.
  The number of elements of the projected lattice is $t=(s+1)^d-s^d=4^2-3^2=7$, that is
  the $(4\times 4)$ grid minus the size of lattice that is $1$ smaller at each size, in this
  case a $(3\times 3)$ lattice (the upper right $(3\times 3)$ elements).
  }
\label{fig:latticeconstruction}
\end{figure}

The blur neighbor matrix is constructed by traversing through all the populated
lattice points and their neighboring elements.
% For efficiency, we do this matrix construction recursively with shared computations
% since $n^{th}$ neighbourhood elements are $1^{st}$ neighborhood elements of $n-1^{th}$ neighbourhood elements. \pg{do not understand}
This is done recursively to share computations. For any lattice point, the neighbors that are
$n$ hops away are the direct neighbors of the points that are $n-1$ hops away.
The size of a $d$ dimensional spatial filter with width $s+1$ is $(s+1)^{d}$ (\eg, a
$3\times 3$ filter, $s=2$ in $d=2$ has $3^2=9$ elements) and this size grows
exponentially in the number of dimensions $d$. The permutohedral lattice is constructed by
projecting a regular grid onto the plane spanned by the $d$ dimensional normal vector ${(1,\ldots,1)}^{\top}$. See
Fig.~\ref{fig:latticeconstruction} for an illustration of the 1D lattice construction.
Many corners of a grid filter are projected onto the same point, in total $t = {(s+1)}^{d} -
s^{d}$ elements remain in the permutohedral filter with $s$ neighborhood in $d-1$ dimensions.
If the lattice has $m$ populated elements, the
matrix $K$ has size $t\times m$. Note that, since the input signal is typically
sparse, only a few lattice corners are being populated in the \textit{slicing} step.
We use a hash-table to keep track of these points and traverse only through
the populated lattice points for this neighborhood matrix construction.

Once the blur neighbor matrix $K$ is constructed, we can perform the convolution
by the matrix vector multiplication
\begin{equation}
\ell' = BK,
\label{eq:conv}
\end{equation}
where $B$ is the $1 \times t$ filter kernel (whose values we will learn) and $\ell'\in\mathbb{R}^{1\times m}$
is the result of the filtering at the $m$ lattice points. In practice, we found that the
matrix $K$ is sometimes too large to fit into GPU memory and we divided the matrix $K$
into smaller pieces to compute Eq.~\ref{eq:conv} sequentially.

In the general multi-dimensional case, the signal $\ell$ is of $c$ dimensions. Then
the kernel $B$ is of size $c \times t$ and $K$ stores the $c$ dimensional vectors
accordingly. When the input and output points are different, we slice only the
input points and splat only at the output points.


\subsection{Additional Experiments}
\label{sec:addexps}
In this section, we discuss more use-cases for the learned bilateral filters, one
use-case of BNNs and two single filter applications for image and 3D mesh denoising.

\subsubsection{Recognition of subsampled MNIST}\label{sec:app_mnist}

One of the strengths of the proposed filter convolution is that it does not
require the input to lie on a regular grid. The only requirement is to define a distance
between features of the input signal.
We highlight this feature with the following experiment using the
classical MNIST ten class classification problem~\cite{lecun1998mnist}. We sample a
sparse set of $N$ points $(x,y)\in [0,1]\times [0,1]$
uniformly at random in the input image, use their interpolated values
as signal and the \emph{continuous} $(x,y)$ positions as features. This mimics
sub-sampling of a high-dimensional signal. To compare against a spatial convolution,
we interpolate the sparse set of values at the grid positions.

We take a reference implementation of LeNet~\cite{lecun1998gradient} that
is part of the Caffe project~\cite{jia2014caffe} and compare it
against the same architecture but replacing the first convolutional
layer with a bilateral convolution layer (BCL). The filter size
and numbers are adjusted to get a comparable number of parameters
($5\times 5$ for LeNet, $2$-neighborhood for BCL).

The results are shown in Table~\ref{tab:all-results}. We see that training
on the original MNIST data (column Original, LeNet vs. BNN) leads to a slight
decrease in performance of the BNN (99.03\%) compared to LeNet
(99.19\%). The BNN can be trained and evaluated on sparse
signals, and we resample the image as described above for $N=$ 100\%, 60\% and
20\% of the total number of pixels. The methods are also evaluated
on test images that are subsampled in the same way. Note that we can
train and test with different subsampling rates. We introduce an additional
bilinear interpolation layer for the LeNet architecture to train on the same
data. In essence, both models perform a spatial interpolation and thus we
expect them to yield a similar classification accuracy. Once the data is of
higher dimensions, the permutohedral convolution will be faster due to hashing
the sparse input points, as well as less memory demanding in comparison to
naive application of a spatial convolution with interpolated values.

\begin{table}[t]
  \begin{center}
    \footnotesize
    \centering
    \begin{tabular}[t]{lllll}
      \toprule
              &     & \multicolumn{3}{c}{Test Subsampling} \\
       Method  & Original & 100\% & 60\% & 20\%\\
      \midrule
       LeNet &  \textbf{0.9919} & 0.9660 & 0.9348 & \textbf{0.6434} \\
       BNN &  0.9903 & \textbf{0.9844} & \textbf{0.9534} & 0.5767 \\
      \hline
       LeNet 100\% & 0.9856 & 0.9809 & 0.9678 & \textbf{0.7386} \\
       BNN 100\% & \textbf{0.9900} & \textbf{0.9863} & \textbf{0.9699} & 0.6910 \\
      \hline
       LeNet 60\% & 0.9848 & 0.9821 & 0.9740 & 0.8151 \\
       BNN 60\% & \textbf{0.9885} & \textbf{0.9864} & \textbf{0.9771} & \textbf{0.8214}\\
      \hline
       LeNet 20\% & \textbf{0.9763} & \textbf{0.9754} & 0.9695 & 0.8928 \\
       BNN 20\% & 0.9728 & 0.9735 & \textbf{0.9701} & \textbf{0.9042}\\
      \bottomrule
    \end{tabular}
  \end{center}
\vspace{-.2cm}
\caption{Classification accuracy on MNIST. We compare the
    LeNet~\cite{lecun1998gradient} implementation that is part of
    Caffe~\cite{jia2014caffe} to the network with the first layer
    replaced by a bilateral convolution layer (BCL). Both are trained
    on the original image resolution (first two rows). Three more BNN
    and CNN models are trained with randomly subsampled images (100\%,
    60\% and 20\% of the pixels). An additional bilinear interpolation
    layer samples the input signal on a spatial grid for the CNN model.
  }
  \label{tab:all-results}
\vspace{-.5cm}
\end{table}

\subsubsection{Image Denoising}

The main application that inspired the development of the bilateral
filtering operation is image denoising~\cite{aurich1995non}, there
using a single Gaussian kernel. Our development allows to learn this
kernel function from data and we explore how to improve using a \emph{single}
but more general bilateral filter.

We use the Berkeley segmentation dataset
(BSDS500)~\cite{arbelaezi2011bsds500} as a test bed. The color
images in the dataset are converted to gray-scale,
and corrupted with Gaussian noise with a standard deviation of
$\frac {25} {255}$.

We compare the performance of four different filter models on a
denoising task.
The first baseline model (`Spatial' in Table \ref{tab:denoising}, $25$
weights) uses a single spatial filter with a kernel size of
$5$ and predicts the scalar gray-scale value at the center pixel. The next model
(`Gauss Bilateral') applies a bilateral \emph{Gaussian}
filter to the noisy input, using position and intensity features $\f=(x,y,v)^\top$.
The third setup (`Learned Bilateral', $65$ weights)
takes a Gauss kernel as initialization and
fits all filter weights on the train set to minimize the
mean squared error with respect to the clean images.
We run a combination
of spatial and permutohedral convolutions on spatial and bilateral
features (`Spatial + Bilateral (Learned)') to check for a complementary
performance of the two convolutions.

\label{sec:exp:denoising}
\begin{table}[!h]
\begin{center}
  \footnotesize
  \begin{tabular}[t]{lr}
    \toprule
    Method & PSNR \\
    \midrule
    Noisy Input & $20.17$ \\
    Spatial & $26.27$ \\
    Gauss Bilateral & $26.51$ \\
    Learned Bilateral & $26.58$ \\
    Spatial + Bilateral (Learned) & \textbf{$26.65$} \\
    \bottomrule
  \end{tabular}
\end{center}
\vspace{-0.5em}
\caption{PSNR results of a denoising task using the BSDS500
  dataset~\cite{arbelaezi2011bsds500}}
\vspace{-0.5em}
\label{tab:denoising}
\end{table}
\vspace{-0.2em}

The PSNR scores evaluated on full images of the test set are
shown in Table \ref{tab:denoising}. We find that an untrained bilateral
filter already performs better than a trained spatial convolution
($26.27$ to $26.51$). A learned convolution further improve the
performance slightly. We chose this simple one-kernel setup to
validate an advantage of the generalized bilateral filter. A competitive
denoising system would employ RGB color information and also
needs to be properly adjusted in network size. Multi-layer perceptrons
have obtained state-of-the-art denoising results~\cite{burger12cvpr}
and the permutohedral lattice layer can readily be used in such an
architecture, which is intended future work.

\subsection{Additional results}
\label{sec:addresults}

This section contains more qualitative results for the experiments presented in Chapter~\ref{chap:bnn}.

\begin{figure*}[th!]
  \centering
    \includegraphics[width=\columnwidth,trim={5cm 2.5cm 5cm 4.5cm},clip]{figures/supplementary/lattice_viz.pdf}
    \vspace{-0.7cm}
  \mycaption{Visualization of the Permutohedral Lattice}
  {Sample lattice visualizations for different feature spaces. All pixels falling in the same simplex cell are shown with
  the same color. $(x,y)$ features correspond to image pixel positions, and $(r,g,b) \in [0,255]$ correspond
  to the red, green and blue color values.}
\label{fig:latticeviz}
\end{figure*}

\subsubsection{Lattice Visualization}

Figure~\ref{fig:latticeviz} shows sample lattice visualizations for different feature spaces.

\newcolumntype{L}[1]{>{\raggedright\let\newline\\\arraybackslash\hspace{0pt}}b{#1}}
\newcolumntype{C}[1]{>{\centering\let\newline\\\arraybackslash\hspace{0pt}}b{#1}}
\newcolumntype{R}[1]{>{\raggedleft\let\newline\\\arraybackslash\hspace{0pt}}b{#1}}

\subsubsection{Color Upsampling}\label{sec:color_upsampling}
\label{sec:col_upsample_extra}

Some images of the upsampling for the Pascal
VOC12 dataset are shown in Fig.~\ref{fig:Colour_upsample_visuals}. It is
especially the low level image details that are better preserved with
a learned bilateral filter compared to the Gaussian case.

\begin{figure*}[t!]
  \centering
    \subfigure{%
   \raisebox{2.0em}{
    \includegraphics[width=.06\columnwidth]{figures/supplementary/2007_004969.jpg}
   }
  }
  \subfigure{%
    \includegraphics[width=.17\columnwidth]{figures/supplementary/2007_004969_gray.pdf}
  }
  \subfigure{%
    \includegraphics[width=.17\columnwidth]{figures/supplementary/2007_004969_gt.pdf}
  }
  \subfigure{%
    \includegraphics[width=.17\columnwidth]{figures/supplementary/2007_004969_bicubic.pdf}
  }
  \subfigure{%
    \includegraphics[width=.17\columnwidth]{figures/supplementary/2007_004969_gauss.pdf}
  }
  \subfigure{%
    \includegraphics[width=.17\columnwidth]{figures/supplementary/2007_004969_learnt.pdf}
  }\\
    \subfigure{%
   \raisebox{2.0em}{
    \includegraphics[width=.06\columnwidth]{figures/supplementary/2007_003106.jpg}
   }
  }
  \subfigure{%
    \includegraphics[width=.17\columnwidth]{figures/supplementary/2007_003106_gray.pdf}
  }
  \subfigure{%
    \includegraphics[width=.17\columnwidth]{figures/supplementary/2007_003106_gt.pdf}
  }
  \subfigure{%
    \includegraphics[width=.17\columnwidth]{figures/supplementary/2007_003106_bicubic.pdf}
  }
  \subfigure{%
    \includegraphics[width=.17\columnwidth]{figures/supplementary/2007_003106_gauss.pdf}
  }
  \subfigure{%
    \includegraphics[width=.17\columnwidth]{figures/supplementary/2007_003106_learnt.pdf}
  }\\
  \setcounter{subfigure}{0}
  \small{
  \subfigure[Inp.]{%
  \raisebox{2.0em}{
    \includegraphics[width=.06\columnwidth]{figures/supplementary/2007_006837.jpg}
   }
  }
  \subfigure[Guidance]{%
    \includegraphics[width=.17\columnwidth]{figures/supplementary/2007_006837_gray.pdf}
  }
   \subfigure[GT]{%
    \includegraphics[width=.17\columnwidth]{figures/supplementary/2007_006837_gt.pdf}
  }
  \subfigure[Bicubic]{%
    \includegraphics[width=.17\columnwidth]{figures/supplementary/2007_006837_bicubic.pdf}
  }
  \subfigure[Gauss-BF]{%
    \includegraphics[width=.17\columnwidth]{figures/supplementary/2007_006837_gauss.pdf}
  }
  \subfigure[Learned-BF]{%
    \includegraphics[width=.17\columnwidth]{figures/supplementary/2007_006837_learnt.pdf}
  }
  }
  \vspace{-0.5cm}
  \mycaption{Color Upsampling}{Color $8\times$ upsampling results
  using different methods, from left to right, (a)~Low-resolution input color image (Inp.),
  (b)~Gray scale guidance image, (c)~Ground-truth color image; Upsampled color images with
  (d)~Bicubic interpolation, (e) Gauss bilateral upsampling and, (f)~Learned bilateral
  updampgling (best viewed on screen).}

\label{fig:Colour_upsample_visuals}
\end{figure*}

\subsubsection{Depth Upsampling}
\label{sec:depth_upsample_extra}

Figure~\ref{fig:depth_upsample_visuals} presents some more qualitative results comparing bicubic interpolation, Gauss
bilateral and learned bilateral upsampling on NYU depth dataset image~\cite{silberman2012indoor}.

\subsubsection{Character Recognition}\label{sec:app_character}

 Figure~\ref{fig:nnrecognition} shows the schematic of different layers
 of the network architecture for LeNet-7~\cite{lecun1998mnist}
 and DeepCNet(5, 50)~\cite{ciresan2012multi,graham2014spatially}. For the BNN variants, the first layer filters are replaced
 with learned bilateral filters and are learned end-to-end.

\subsubsection{Semantic Segmentation}\label{sec:app_semantic_segmentation}
\label{sec:semantic_bnn_extra}

Some more visual results for semantic segmentation are shown in Figure~\ref{fig:semantic_visuals}.
These include the underlying DeepLab CNN\cite{chen2014semantic} result (DeepLab),
the 2 step mean-field result with Gaussian edge potentials (+2stepMF-GaussCRF)
and also corresponding results with learned edge potentials (+2stepMF-LearnedCRF).
In general, we observe that mean-field in learned CRF leads to slightly dilated
classification regions in comparison to using Gaussian CRF thereby filling-in the
false negative pixels and also correcting some mis-classified regions.

\begin{figure*}[t!]
  \centering
    \subfigure{%
   \raisebox{2.0em}{
    \includegraphics[width=.06\columnwidth]{figures/supplementary/2bicubic}
   }
  }
  \subfigure{%
    \includegraphics[width=.17\columnwidth]{figures/supplementary/2given_image}
  }
  \subfigure{%
    \includegraphics[width=.17\columnwidth]{figures/supplementary/2ground_truth}
  }
  \subfigure{%
    \includegraphics[width=.17\columnwidth]{figures/supplementary/2bicubic}
  }
  \subfigure{%
    \includegraphics[width=.17\columnwidth]{figures/supplementary/2gauss}
  }
  \subfigure{%
    \includegraphics[width=.17\columnwidth]{figures/supplementary/2learnt}
  }\\
    \subfigure{%
   \raisebox{2.0em}{
    \includegraphics[width=.06\columnwidth]{figures/supplementary/32bicubic}
   }
  }
  \subfigure{%
    \includegraphics[width=.17\columnwidth]{figures/supplementary/32given_image}
  }
  \subfigure{%
    \includegraphics[width=.17\columnwidth]{figures/supplementary/32ground_truth}
  }
  \subfigure{%
    \includegraphics[width=.17\columnwidth]{figures/supplementary/32bicubic}
  }
  \subfigure{%
    \includegraphics[width=.17\columnwidth]{figures/supplementary/32gauss}
  }
  \subfigure{%
    \includegraphics[width=.17\columnwidth]{figures/supplementary/32learnt}
  }\\
  \setcounter{subfigure}{0}
  \small{
  \subfigure[Inp.]{%
  \raisebox{2.0em}{
    \includegraphics[width=.06\columnwidth]{figures/supplementary/41bicubic}
   }
  }
  \subfigure[Guidance]{%
    \includegraphics[width=.17\columnwidth]{figures/supplementary/41given_image}
  }
   \subfigure[GT]{%
    \includegraphics[width=.17\columnwidth]{figures/supplementary/41ground_truth}
  }
  \subfigure[Bicubic]{%
    \includegraphics[width=.17\columnwidth]{figures/supplementary/41bicubic}
  }
  \subfigure[Gauss-BF]{%
    \includegraphics[width=.17\columnwidth]{figures/supplementary/41gauss}
  }
  \subfigure[Learned-BF]{%
    \includegraphics[width=.17\columnwidth]{figures/supplementary/41learnt}
  }
  }
  \mycaption{Depth Upsampling}{Depth $8\times$ upsampling results
  using different upsampling strategies, from left to right,
  (a)~Low-resolution input depth image (Inp.),
  (b)~High-resolution guidance image, (c)~Ground-truth depth; Upsampled depth images with
  (d)~Bicubic interpolation, (e) Gauss bilateral upsampling and, (f)~Learned bilateral
  updampgling (best viewed on screen).}

\label{fig:depth_upsample_visuals}
\end{figure*}

\subsubsection{Material Segmentation}\label{sec:app_material_segmentation}
\label{sec:material_bnn_extra}

In Fig.~\ref{fig:material_visuals-app2}, we present visual results comparing 2 step
mean-field inference with Gaussian and learned pairwise CRF potentials. In
general, we observe that the pixels belonging to dominant classes in the
training data are being more accurately classified with learned CRF. This leads to
a significant improvements in overall pixel accuracy. This also results
in a slight decrease of the accuracy from less frequent class pixels thereby
slightly reducing the average class accuracy with learning. We attribute this
to the type of annotation that is available for this dataset, which is not
for the entire image but for some segments in the image. We have very few
images of the infrequent classes to combat this behaviour during training.

\subsubsection{Experiment Protocols}
\label{sec:protocols}

Table~\ref{tbl:parameters} shows experiment protocols of different experiments.

 \begin{figure*}[t!]
  \centering
  \subfigure[LeNet-7]{
    \includegraphics[width=0.7\columnwidth]{figures/supplementary/lenet_cnn_network}
    }\\
    \subfigure[DeepCNet]{
    \includegraphics[width=\columnwidth]{figures/supplementary/deepcnet_cnn_network}
    }
  \mycaption{CNNs for Character Recognition}
  {Schematic of (top) LeNet-7~\cite{lecun1998mnist} and (bottom) DeepCNet(5,50)~\cite{ciresan2012multi,graham2014spatially} architectures used in Assamese
  character recognition experiments.}
\label{fig:nnrecognition}
\end{figure*}

\definecolor{voc_1}{RGB}{0, 0, 0}
\definecolor{voc_2}{RGB}{128, 0, 0}
\definecolor{voc_3}{RGB}{0, 128, 0}
\definecolor{voc_4}{RGB}{128, 128, 0}
\definecolor{voc_5}{RGB}{0, 0, 128}
\definecolor{voc_6}{RGB}{128, 0, 128}
\definecolor{voc_7}{RGB}{0, 128, 128}
\definecolor{voc_8}{RGB}{128, 128, 128}
\definecolor{voc_9}{RGB}{64, 0, 0}
\definecolor{voc_10}{RGB}{192, 0, 0}
\definecolor{voc_11}{RGB}{64, 128, 0}
\definecolor{voc_12}{RGB}{192, 128, 0}
\definecolor{voc_13}{RGB}{64, 0, 128}
\definecolor{voc_14}{RGB}{192, 0, 128}
\definecolor{voc_15}{RGB}{64, 128, 128}
\definecolor{voc_16}{RGB}{192, 128, 128}
\definecolor{voc_17}{RGB}{0, 64, 0}
\definecolor{voc_18}{RGB}{128, 64, 0}
\definecolor{voc_19}{RGB}{0, 192, 0}
\definecolor{voc_20}{RGB}{128, 192, 0}
\definecolor{voc_21}{RGB}{0, 64, 128}
\definecolor{voc_22}{RGB}{128, 64, 128}

\begin{figure*}[t]
  \centering
  \small{
  \fcolorbox{white}{voc_1}{\rule{0pt}{6pt}\rule{6pt}{0pt}} Background~~
  \fcolorbox{white}{voc_2}{\rule{0pt}{6pt}\rule{6pt}{0pt}} Aeroplane~~
  \fcolorbox{white}{voc_3}{\rule{0pt}{6pt}\rule{6pt}{0pt}} Bicycle~~
  \fcolorbox{white}{voc_4}{\rule{0pt}{6pt}\rule{6pt}{0pt}} Bird~~
  \fcolorbox{white}{voc_5}{\rule{0pt}{6pt}\rule{6pt}{0pt}} Boat~~
  \fcolorbox{white}{voc_6}{\rule{0pt}{6pt}\rule{6pt}{0pt}} Bottle~~
  \fcolorbox{white}{voc_7}{\rule{0pt}{6pt}\rule{6pt}{0pt}} Bus~~
  \fcolorbox{white}{voc_8}{\rule{0pt}{6pt}\rule{6pt}{0pt}} Car~~ \\
  \fcolorbox{white}{voc_9}{\rule{0pt}{6pt}\rule{6pt}{0pt}} Cat~~
  \fcolorbox{white}{voc_10}{\rule{0pt}{6pt}\rule{6pt}{0pt}} Chair~~
  \fcolorbox{white}{voc_11}{\rule{0pt}{6pt}\rule{6pt}{0pt}} Cow~~
  \fcolorbox{white}{voc_12}{\rule{0pt}{6pt}\rule{6pt}{0pt}} Dining Table~~
  \fcolorbox{white}{voc_13}{\rule{0pt}{6pt}\rule{6pt}{0pt}} Dog~~
  \fcolorbox{white}{voc_14}{\rule{0pt}{6pt}\rule{6pt}{0pt}} Horse~~
  \fcolorbox{white}{voc_15}{\rule{0pt}{6pt}\rule{6pt}{0pt}} Motorbike~~
  \fcolorbox{white}{voc_16}{\rule{0pt}{6pt}\rule{6pt}{0pt}} Person~~ \\
  \fcolorbox{white}{voc_17}{\rule{0pt}{6pt}\rule{6pt}{0pt}} Potted Plant~~
  \fcolorbox{white}{voc_18}{\rule{0pt}{6pt}\rule{6pt}{0pt}} Sheep~~
  \fcolorbox{white}{voc_19}{\rule{0pt}{6pt}\rule{6pt}{0pt}} Sofa~~
  \fcolorbox{white}{voc_20}{\rule{0pt}{6pt}\rule{6pt}{0pt}} Train~~
  \fcolorbox{white}{voc_21}{\rule{0pt}{6pt}\rule{6pt}{0pt}} TV monitor~~ \\
  }
  \subfigure{%
    \includegraphics[width=.18\columnwidth]{figures/supplementary/2007_001423_given.jpg}
  }
  \subfigure{%
    \includegraphics[width=.18\columnwidth]{figures/supplementary/2007_001423_gt.png}
  }
  \subfigure{%
    \includegraphics[width=.18\columnwidth]{figures/supplementary/2007_001423_cnn.png}
  }
  \subfigure{%
    \includegraphics[width=.18\columnwidth]{figures/supplementary/2007_001423_gauss.png}
  }
  \subfigure{%
    \includegraphics[width=.18\columnwidth]{figures/supplementary/2007_001423_learnt.png}
  }\\
  \subfigure{%
    \includegraphics[width=.18\columnwidth]{figures/supplementary/2007_001430_given.jpg}
  }
  \subfigure{%
    \includegraphics[width=.18\columnwidth]{figures/supplementary/2007_001430_gt.png}
  }
  \subfigure{%
    \includegraphics[width=.18\columnwidth]{figures/supplementary/2007_001430_cnn.png}
  }
  \subfigure{%
    \includegraphics[width=.18\columnwidth]{figures/supplementary/2007_001430_gauss.png}
  }
  \subfigure{%
    \includegraphics[width=.18\columnwidth]{figures/supplementary/2007_001430_learnt.png}
  }\\
    \subfigure{%
    \includegraphics[width=.18\columnwidth]{figures/supplementary/2007_007996_given.jpg}
  }
  \subfigure{%
    \includegraphics[width=.18\columnwidth]{figures/supplementary/2007_007996_gt.png}
  }
  \subfigure{%
    \includegraphics[width=.18\columnwidth]{figures/supplementary/2007_007996_cnn.png}
  }
  \subfigure{%
    \includegraphics[width=.18\columnwidth]{figures/supplementary/2007_007996_gauss.png}
  }
  \subfigure{%
    \includegraphics[width=.18\columnwidth]{figures/supplementary/2007_007996_learnt.png}
  }\\
   \subfigure{%
    \includegraphics[width=.18\columnwidth]{figures/supplementary/2010_002682_given.jpg}
  }
  \subfigure{%
    \includegraphics[width=.18\columnwidth]{figures/supplementary/2010_002682_gt.png}
  }
  \subfigure{%
    \includegraphics[width=.18\columnwidth]{figures/supplementary/2010_002682_cnn.png}
  }
  \subfigure{%
    \includegraphics[width=.18\columnwidth]{figures/supplementary/2010_002682_gauss.png}
  }
  \subfigure{%
    \includegraphics[width=.18\columnwidth]{figures/supplementary/2010_002682_learnt.png}
  }\\
     \subfigure{%
    \includegraphics[width=.18\columnwidth]{figures/supplementary/2010_004789_given.jpg}
  }
  \subfigure{%
    \includegraphics[width=.18\columnwidth]{figures/supplementary/2010_004789_gt.png}
  }
  \subfigure{%
    \includegraphics[width=.18\columnwidth]{figures/supplementary/2010_004789_cnn.png}
  }
  \subfigure{%
    \includegraphics[width=.18\columnwidth]{figures/supplementary/2010_004789_gauss.png}
  }
  \subfigure{%
    \includegraphics[width=.18\columnwidth]{figures/supplementary/2010_004789_learnt.png}
  }\\
       \subfigure{%
    \includegraphics[width=.18\columnwidth]{figures/supplementary/2007_001311_given.jpg}
  }
  \subfigure{%
    \includegraphics[width=.18\columnwidth]{figures/supplementary/2007_001311_gt.png}
  }
  \subfigure{%
    \includegraphics[width=.18\columnwidth]{figures/supplementary/2007_001311_cnn.png}
  }
  \subfigure{%
    \includegraphics[width=.18\columnwidth]{figures/supplementary/2007_001311_gauss.png}
  }
  \subfigure{%
    \includegraphics[width=.18\columnwidth]{figures/supplementary/2007_001311_learnt.png}
  }\\
  \setcounter{subfigure}{0}
  \subfigure[Input]{%
    \includegraphics[width=.18\columnwidth]{figures/supplementary/2010_003531_given.jpg}
  }
  \subfigure[Ground Truth]{%
    \includegraphics[width=.18\columnwidth]{figures/supplementary/2010_003531_gt.png}
  }
  \subfigure[DeepLab]{%
    \includegraphics[width=.18\columnwidth]{figures/supplementary/2010_003531_cnn.png}
  }
  \subfigure[+GaussCRF]{%
    \includegraphics[width=.18\columnwidth]{figures/supplementary/2010_003531_gauss.png}
  }
  \subfigure[+LearnedCRF]{%
    \includegraphics[width=.18\columnwidth]{figures/supplementary/2010_003531_learnt.png}
  }
  \vspace{-0.3cm}
  \mycaption{Semantic Segmentation}{Example results of semantic segmentation.
  (c)~depicts the unary results before application of MF, (d)~after two steps of MF with Gaussian edge CRF potentials, (e)~after
  two steps of MF with learned edge CRF potentials.}
    \label{fig:semantic_visuals}
\end{figure*}


\definecolor{minc_1}{HTML}{771111}
\definecolor{minc_2}{HTML}{CAC690}
\definecolor{minc_3}{HTML}{EEEEEE}
\definecolor{minc_4}{HTML}{7C8FA6}
\definecolor{minc_5}{HTML}{597D31}
\definecolor{minc_6}{HTML}{104410}
\definecolor{minc_7}{HTML}{BB819C}
\definecolor{minc_8}{HTML}{D0CE48}
\definecolor{minc_9}{HTML}{622745}
\definecolor{minc_10}{HTML}{666666}
\definecolor{minc_11}{HTML}{D54A31}
\definecolor{minc_12}{HTML}{101044}
\definecolor{minc_13}{HTML}{444126}
\definecolor{minc_14}{HTML}{75D646}
\definecolor{minc_15}{HTML}{DD4348}
\definecolor{minc_16}{HTML}{5C8577}
\definecolor{minc_17}{HTML}{C78472}
\definecolor{minc_18}{HTML}{75D6D0}
\definecolor{minc_19}{HTML}{5B4586}
\definecolor{minc_20}{HTML}{C04393}
\definecolor{minc_21}{HTML}{D69948}
\definecolor{minc_22}{HTML}{7370D8}
\definecolor{minc_23}{HTML}{7A3622}
\definecolor{minc_24}{HTML}{000000}

\begin{figure*}[t]
  \centering
  \small{
  \fcolorbox{white}{minc_1}{\rule{0pt}{6pt}\rule{6pt}{0pt}} Brick~~
  \fcolorbox{white}{minc_2}{\rule{0pt}{6pt}\rule{6pt}{0pt}} Carpet~~
  \fcolorbox{white}{minc_3}{\rule{0pt}{6pt}\rule{6pt}{0pt}} Ceramic~~
  \fcolorbox{white}{minc_4}{\rule{0pt}{6pt}\rule{6pt}{0pt}} Fabric~~
  \fcolorbox{white}{minc_5}{\rule{0pt}{6pt}\rule{6pt}{0pt}} Foliage~~
  \fcolorbox{white}{minc_6}{\rule{0pt}{6pt}\rule{6pt}{0pt}} Food~~
  \fcolorbox{white}{minc_7}{\rule{0pt}{6pt}\rule{6pt}{0pt}} Glass~~
  \fcolorbox{white}{minc_8}{\rule{0pt}{6pt}\rule{6pt}{0pt}} Hair~~ \\
  \fcolorbox{white}{minc_9}{\rule{0pt}{6pt}\rule{6pt}{0pt}} Leather~~
  \fcolorbox{white}{minc_10}{\rule{0pt}{6pt}\rule{6pt}{0pt}} Metal~~
  \fcolorbox{white}{minc_11}{\rule{0pt}{6pt}\rule{6pt}{0pt}} Mirror~~
  \fcolorbox{white}{minc_12}{\rule{0pt}{6pt}\rule{6pt}{0pt}} Other~~
  \fcolorbox{white}{minc_13}{\rule{0pt}{6pt}\rule{6pt}{0pt}} Painted~~
  \fcolorbox{white}{minc_14}{\rule{0pt}{6pt}\rule{6pt}{0pt}} Paper~~
  \fcolorbox{white}{minc_15}{\rule{0pt}{6pt}\rule{6pt}{0pt}} Plastic~~\\
  \fcolorbox{white}{minc_16}{\rule{0pt}{6pt}\rule{6pt}{0pt}} Polished Stone~~
  \fcolorbox{white}{minc_17}{\rule{0pt}{6pt}\rule{6pt}{0pt}} Skin~~
  \fcolorbox{white}{minc_18}{\rule{0pt}{6pt}\rule{6pt}{0pt}} Sky~~
  \fcolorbox{white}{minc_19}{\rule{0pt}{6pt}\rule{6pt}{0pt}} Stone~~
  \fcolorbox{white}{minc_20}{\rule{0pt}{6pt}\rule{6pt}{0pt}} Tile~~
  \fcolorbox{white}{minc_21}{\rule{0pt}{6pt}\rule{6pt}{0pt}} Wallpaper~~
  \fcolorbox{white}{minc_22}{\rule{0pt}{6pt}\rule{6pt}{0pt}} Water~~
  \fcolorbox{white}{minc_23}{\rule{0pt}{6pt}\rule{6pt}{0pt}} Wood~~ \\
  }
  \subfigure{%
    \includegraphics[width=.18\columnwidth]{figures/supplementary/000010868_given.jpg}
  }
  \subfigure{%
    \includegraphics[width=.18\columnwidth]{figures/supplementary/000010868_gt.png}
  }
  \subfigure{%
    \includegraphics[width=.18\columnwidth]{figures/supplementary/000010868_cnn.png}
  }
  \subfigure{%
    \includegraphics[width=.18\columnwidth]{figures/supplementary/000010868_gauss.png}
  }
  \subfigure{%
    \includegraphics[width=.18\columnwidth]{figures/supplementary/000010868_learnt.png}
  }\\[-2ex]
  \subfigure{%
    \includegraphics[width=.18\columnwidth]{figures/supplementary/000006011_given.jpg}
  }
  \subfigure{%
    \includegraphics[width=.18\columnwidth]{figures/supplementary/000006011_gt.png}
  }
  \subfigure{%
    \includegraphics[width=.18\columnwidth]{figures/supplementary/000006011_cnn.png}
  }
  \subfigure{%
    \includegraphics[width=.18\columnwidth]{figures/supplementary/000006011_gauss.png}
  }
  \subfigure{%
    \includegraphics[width=.18\columnwidth]{figures/supplementary/000006011_learnt.png}
  }\\[-2ex]
    \subfigure{%
    \includegraphics[width=.18\columnwidth]{figures/supplementary/000008553_given.jpg}
  }
  \subfigure{%
    \includegraphics[width=.18\columnwidth]{figures/supplementary/000008553_gt.png}
  }
  \subfigure{%
    \includegraphics[width=.18\columnwidth]{figures/supplementary/000008553_cnn.png}
  }
  \subfigure{%
    \includegraphics[width=.18\columnwidth]{figures/supplementary/000008553_gauss.png}
  }
  \subfigure{%
    \includegraphics[width=.18\columnwidth]{figures/supplementary/000008553_learnt.png}
  }\\[-2ex]
   \subfigure{%
    \includegraphics[width=.18\columnwidth]{figures/supplementary/000009188_given.jpg}
  }
  \subfigure{%
    \includegraphics[width=.18\columnwidth]{figures/supplementary/000009188_gt.png}
  }
  \subfigure{%
    \includegraphics[width=.18\columnwidth]{figures/supplementary/000009188_cnn.png}
  }
  \subfigure{%
    \includegraphics[width=.18\columnwidth]{figures/supplementary/000009188_gauss.png}
  }
  \subfigure{%
    \includegraphics[width=.18\columnwidth]{figures/supplementary/000009188_learnt.png}
  }\\[-2ex]
  \setcounter{subfigure}{0}
  \subfigure[Input]{%
    \includegraphics[width=.18\columnwidth]{figures/supplementary/000023570_given.jpg}
  }
  \subfigure[Ground Truth]{%
    \includegraphics[width=.18\columnwidth]{figures/supplementary/000023570_gt.png}
  }
  \subfigure[DeepLab]{%
    \includegraphics[width=.18\columnwidth]{figures/supplementary/000023570_cnn.png}
  }
  \subfigure[+GaussCRF]{%
    \includegraphics[width=.18\columnwidth]{figures/supplementary/000023570_gauss.png}
  }
  \subfigure[+LearnedCRF]{%
    \includegraphics[width=.18\columnwidth]{figures/supplementary/000023570_learnt.png}
  }
  \mycaption{Material Segmentation}{Example results of material segmentation.
  (c)~depicts the unary results before application of MF, (d)~after two steps of MF with Gaussian edge CRF potentials, (e)~after two steps of MF with learned edge CRF potentials.}
    \label{fig:material_visuals-app2}
\end{figure*}


\begin{table*}[h]
\tiny
  \centering
    \begin{tabular}{L{2.3cm} L{2.25cm} C{1.5cm} C{0.7cm} C{0.6cm} C{0.7cm} C{0.7cm} C{0.7cm} C{1.6cm} C{0.6cm} C{0.6cm} C{0.6cm}}
      \toprule
& & & & & \multicolumn{3}{c}{\textbf{Data Statistics}} & \multicolumn{4}{c}{\textbf{Training Protocol}} \\

\textbf{Experiment} & \textbf{Feature Types} & \textbf{Feature Scales} & \textbf{Filter Size} & \textbf{Filter Nbr.} & \textbf{Train}  & \textbf{Val.} & \textbf{Test} & \textbf{Loss Type} & \textbf{LR} & \textbf{Batch} & \textbf{Epochs} \\
      \midrule
      \multicolumn{2}{c}{\textbf{Single Bilateral Filter Applications}} & & & & & & & & & \\
      \textbf{2$\times$ Color Upsampling} & Position$_{1}$, Intensity (3D) & 0.13, 0.17 & 65 & 2 & 10581 & 1449 & 1456 & MSE & 1e-06 & 200 & 94.5\\
      \textbf{4$\times$ Color Upsampling} & Position$_{1}$, Intensity (3D) & 0.06, 0.17 & 65 & 2 & 10581 & 1449 & 1456 & MSE & 1e-06 & 200 & 94.5\\
      \textbf{8$\times$ Color Upsampling} & Position$_{1}$, Intensity (3D) & 0.03, 0.17 & 65 & 2 & 10581 & 1449 & 1456 & MSE & 1e-06 & 200 & 94.5\\
      \textbf{16$\times$ Color Upsampling} & Position$_{1}$, Intensity (3D) & 0.02, 0.17 & 65 & 2 & 10581 & 1449 & 1456 & MSE & 1e-06 & 200 & 94.5\\
      \textbf{Depth Upsampling} & Position$_{1}$, Color (5D) & 0.05, 0.02 & 665 & 2 & 795 & 100 & 654 & MSE & 1e-07 & 50 & 251.6\\
      \textbf{Mesh Denoising} & Isomap (4D) & 46.00 & 63 & 2 & 1000 & 200 & 500 & MSE & 100 & 10 & 100.0 \\
      \midrule
      \multicolumn{2}{c}{\textbf{DenseCRF Applications}} & & & & & & & & &\\
      \multicolumn{2}{l}{\textbf{Semantic Segmentation}} & & & & & & & & &\\
      \textbf{- 1step MF} & Position$_{1}$, Color (5D); Position$_{1}$ (2D) & 0.01, 0.34; 0.34  & 665; 19  & 2; 2 & 10581 & 1449 & 1456 & Logistic & 0.1 & 5 & 1.4 \\
      \textbf{- 2step MF} & Position$_{1}$, Color (5D); Position$_{1}$ (2D) & 0.01, 0.34; 0.34 & 665; 19 & 2; 2 & 10581 & 1449 & 1456 & Logistic & 0.1 & 5 & 1.4 \\
      \textbf{- \textit{loose} 2step MF} & Position$_{1}$, Color (5D); Position$_{1}$ (2D) & 0.01, 0.34; 0.34 & 665; 19 & 2; 2 &10581 & 1449 & 1456 & Logistic & 0.1 & 5 & +1.9  \\ \\
      \multicolumn{2}{l}{\textbf{Material Segmentation}} & & & & & & & & &\\
      \textbf{- 1step MF} & Position$_{2}$, Lab-Color (5D) & 5.00, 0.05, 0.30  & 665 & 2 & 928 & 150 & 1798 & Weighted Logistic & 1e-04 & 24 & 2.6 \\
      \textbf{- 2step MF} & Position$_{2}$, Lab-Color (5D) & 5.00, 0.05, 0.30 & 665 & 2 & 928 & 150 & 1798 & Weighted Logistic & 1e-04 & 12 & +0.7 \\
      \textbf{- \textit{loose} 2step MF} & Position$_{2}$, Lab-Color (5D) & 5.00, 0.05, 0.30 & 665 & 2 & 928 & 150 & 1798 & Weighted Logistic & 1e-04 & 12 & +0.2\\
      \midrule
      \multicolumn{2}{c}{\textbf{Neural Network Applications}} & & & & & & & & &\\
      \textbf{Tiles: CNN-9$\times$9} & - & - & 81 & 4 & 10000 & 1000 & 1000 & Logistic & 0.01 & 100 & 500.0 \\
      \textbf{Tiles: CNN-13$\times$13} & - & - & 169 & 6 & 10000 & 1000 & 1000 & Logistic & 0.01 & 100 & 500.0 \\
      \textbf{Tiles: CNN-17$\times$17} & - & - & 289 & 8 & 10000 & 1000 & 1000 & Logistic & 0.01 & 100 & 500.0 \\
      \textbf{Tiles: CNN-21$\times$21} & - & - & 441 & 10 & 10000 & 1000 & 1000 & Logistic & 0.01 & 100 & 500.0 \\
      \textbf{Tiles: BNN} & Position$_{1}$, Color (5D) & 0.05, 0.04 & 63 & 1 & 10000 & 1000 & 1000 & Logistic & 0.01 & 100 & 30.0 \\
      \textbf{LeNet} & - & - & 25 & 2 & 5490 & 1098 & 1647 & Logistic & 0.1 & 100 & 182.2 \\
      \textbf{Crop-LeNet} & - & - & 25 & 2 & 5490 & 1098 & 1647 & Logistic & 0.1 & 100 & 182.2 \\
      \textbf{BNN-LeNet} & Position$_{2}$ (2D) & 20.00 & 7 & 1 & 5490 & 1098 & 1647 & Logistic & 0.1 & 100 & 182.2 \\
      \textbf{DeepCNet} & - & - & 9 & 1 & 5490 & 1098 & 1647 & Logistic & 0.1 & 100 & 182.2 \\
      \textbf{Crop-DeepCNet} & - & - & 9 & 1 & 5490 & 1098 & 1647 & Logistic & 0.1 & 100 & 182.2 \\
      \textbf{BNN-DeepCNet} & Position$_{2}$ (2D) & 40.00  & 7 & 1 & 5490 & 1098 & 1647 & Logistic & 0.1 & 100 & 182.2 \\
      \bottomrule
      \\
    \end{tabular}
    \mycaption{Experiment Protocols} {Experiment protocols for the different experiments presented in this work. \textbf{Feature Types}:
    Feature spaces used for the bilateral convolutions. Position$_1$ corresponds to un-normalized pixel positions whereas Position$_2$ corresponds
    to pixel positions normalized to $[0,1]$ with respect to the given image. \textbf{Feature Scales}: Cross-validated scales for the features used.
     \textbf{Filter Size}: Number of elements in the filter that is being learned. \textbf{Filter Nbr.}: Half-width of the filter. \textbf{Train},
     \textbf{Val.} and \textbf{Test} corresponds to the number of train, validation and test images used in the experiment. \textbf{Loss Type}: Type
     of loss used for back-propagation. ``MSE'' corresponds to Euclidean mean squared error loss and ``Logistic'' corresponds to multinomial logistic
     loss. ``Weighted Logistic'' is the class-weighted multinomial logistic loss. We weighted the loss with inverse class probability for material
     segmentation task due to the small availability of training data with class imbalance. \textbf{LR}: Fixed learning rate used in stochastic gradient
     descent. \textbf{Batch}: Number of images used in one parameter update step. \textbf{Epochs}: Number of training epochs. In all the experiments,
     we used fixed momentum of 0.9 and weight decay of 0.0005 for stochastic gradient descent. ```Color Upsampling'' experiments in this Table corresponds
     to those performed on Pascal VOC12 dataset images. For all experiments using Pascal VOC12 images, we use extended
     training segmentation dataset available from~\cite{hariharan2011moredata}, and used standard validation and test splits
     from the main dataset~\cite{voc2012segmentation}.}
  \label{tbl:parameters}
\end{table*}

\clearpage

\section{Parameters and Additional Results for Video Propagation Networks}

In this Section, we present experiment protocols and additional qualitative results for experiments
on video object segmentation, semantic video segmentation and video color
propagation. Table~\ref{tbl:parameters_supp} shows the feature scales and other parameters used in different experiments.
Figures~\ref{fig:video_seg_pos_supp} show some qualitative results on video object segmentation
with some failure cases in Fig.~\ref{fig:video_seg_neg_supp}.
Figure~\ref{fig:semantic_visuals_supp} shows some qualitative results on semantic video segmentation and
Fig.~\ref{fig:color_visuals_supp} shows results on video color propagation.

\newcolumntype{L}[1]{>{\raggedright\let\newline\\\arraybackslash\hspace{0pt}}b{#1}}
\newcolumntype{C}[1]{>{\centering\let\newline\\\arraybackslash\hspace{0pt}}b{#1}}
\newcolumntype{R}[1]{>{\raggedleft\let\newline\\\arraybackslash\hspace{0pt}}b{#1}}

\begin{table*}[h]
\tiny
  \centering
    \begin{tabular}{L{3.0cm} L{2.4cm} L{2.8cm} L{2.8cm} C{0.5cm} C{1.0cm} L{1.2cm}}
      \toprule
\textbf{Experiment} & \textbf{Feature Type} & \textbf{Feature Scale-1, $\Lambda_a$} & \textbf{Feature Scale-2, $\Lambda_b$} & \textbf{$\alpha$} & \textbf{Input Frames} & \textbf{Loss Type} \\
      \midrule
      \textbf{Video Object Segmentation} & ($x,y,Y,Cb,Cr,t$) & (0.02,0.02,0.07,0.4,0.4,0.01) & (0.03,0.03,0.09,0.5,0.5,0.2) & 0.5 & 9 & Logistic\\
      \midrule
      \textbf{Semantic Video Segmentation} & & & & & \\
      \textbf{with CNN1~\cite{yu2015multi}-NoFlow} & ($x,y,R,G,B,t$) & (0.08,0.08,0.2,0.2,0.2,0.04) & (0.11,0.11,0.2,0.2,0.2,0.04) & 0.5 & 3 & Logistic \\
      \textbf{with CNN1~\cite{yu2015multi}-Flow} & ($x+u_x,y+u_y,R,G,B,t$) & (0.11,0.11,0.14,0.14,0.14,0.03) & (0.08,0.08,0.12,0.12,0.12,0.01) & 0.65 & 3 & Logistic\\
      \textbf{with CNN2~\cite{richter2016playing}-Flow} & ($x+u_x,y+u_y,R,G,B,t$) & (0.08,0.08,0.2,0.2,0.2,0.04) & (0.09,0.09,0.25,0.25,0.25,0.03) & 0.5 & 4 & Logistic\\
      \midrule
      \textbf{Video Color Propagation} & ($x,y,I,t$)  & (0.04,0.04,0.2,0.04) & No second kernel & 1 & 4 & MSE\\
      \bottomrule
      \\
    \end{tabular}
    \mycaption{Experiment Protocols} {Experiment protocols for the different experiments presented in this work. \textbf{Feature Types}:
    Feature spaces used for the bilateral convolutions, with position ($x,y$) and color
    ($R,G,B$ or $Y,Cb,Cr$) features $\in [0,255]$. $u_x$, $u_y$ denotes optical flow with respect
    to the present frame and $I$ denotes grayscale intensity.
    \textbf{Feature Scales ($\Lambda_a, \Lambda_b$)}: Cross-validated scales for the features used.
    \textbf{$\alpha$}: Exponential time decay for the input frames.
    \textbf{Input Frames}: Number of input frames for VPN.
    \textbf{Loss Type}: Type
     of loss used for back-propagation. ``MSE'' corresponds to Euclidean mean squared error loss and ``Logistic'' corresponds to multinomial logistic loss.}
  \label{tbl:parameters_supp}
\end{table*}

% \begin{figure}[th!]
% \begin{center}
%   \centerline{\includegraphics[width=\textwidth]{figures/video_seg_visuals_supp_small.pdf}}
%     \mycaption{Video Object Segmentation}
%     {Shown are the different frames in example videos with the corresponding
%     ground truth (GT) masks, predictions from BVS~\cite{marki2016bilateral},
%     OFL~\cite{tsaivideo}, VPN (VPN-Stage2) and VPN-DLab (VPN-DeepLab) models.}
%     \label{fig:video_seg_small_supp}
% \end{center}
% \vspace{-1.0cm}
% \end{figure}

\begin{figure}[th!]
\begin{center}
  \centerline{\includegraphics[width=0.7\textwidth]{figures/video_seg_visuals_supp_positive.pdf}}
    \mycaption{Video Object Segmentation}
    {Shown are the different frames in example videos with the corresponding
    ground truth (GT) masks, predictions from BVS~\cite{marki2016bilateral},
    OFL~\cite{tsaivideo}, VPN (VPN-Stage2) and VPN-DLab (VPN-DeepLab) models.}
    \label{fig:video_seg_pos_supp}
\end{center}
\vspace{-1.0cm}
\end{figure}

\begin{figure}[th!]
\begin{center}
  \centerline{\includegraphics[width=0.7\textwidth]{figures/video_seg_visuals_supp_negative.pdf}}
    \mycaption{Failure Cases for Video Object Segmentation}
    {Shown are the different frames in example videos with the corresponding
    ground truth (GT) masks, predictions from BVS~\cite{marki2016bilateral},
    OFL~\cite{tsaivideo}, VPN (VPN-Stage2) and VPN-DLab (VPN-DeepLab) models.}
    \label{fig:video_seg_neg_supp}
\end{center}
\vspace{-1.0cm}
\end{figure}

\begin{figure}[th!]
\begin{center}
  \centerline{\includegraphics[width=0.9\textwidth]{figures/supp_semantic_visual.pdf}}
    \mycaption{Semantic Video Segmentation}
    {Input video frames and the corresponding ground truth (GT)
    segmentation together with the predictions of CNN~\cite{yu2015multi} and with
    VPN-Flow.}
    \label{fig:semantic_visuals_supp}
\end{center}
\vspace{-0.7cm}
\end{figure}

\begin{figure}[th!]
\begin{center}
  \centerline{\includegraphics[width=\textwidth]{figures/colorization_visuals_supp.pdf}}
  \mycaption{Video Color Propagation}
  {Input grayscale video frames and corresponding ground-truth (GT) color images
  together with color predictions of Levin et al.~\cite{levin2004colorization} and VPN-Stage1 models.}
  \label{fig:color_visuals_supp}
\end{center}
\vspace{-0.7cm}
\end{figure}

\clearpage

\section{Additional Material for Bilateral Inception Networks}
\label{sec:binception-app}

In this section of the Appendix, we first discuss the use of approximate bilateral
filtering in BI modules (Sec.~\ref{sec:lattice}).
Later, we present some qualitative results using different models for the approach presented in
Chapter~\ref{chap:binception} (Sec.~\ref{sec:qualitative-app}).

\subsection{Approximate Bilateral Filtering}
\label{sec:lattice}

The bilateral inception module presented in Chapter~\ref{chap:binception} computes a matrix-vector
product between a Gaussian filter $K$ and a vector of activations $\bz_c$.
Bilateral filtering is an important operation and many algorithmic techniques have been
proposed to speed-up this operation~\cite{paris2006fast,adams2010fast,gastal2011domain}.
In the main paper we opted to implement what can be considered the
brute-force variant of explicitly constructing $K$ and then using BLAS to compute the
matrix-vector product. This resulted in a few millisecond operation.
The explicit way to compute is possible due to the
reduction to super-pixels, e.g., it would not work for DenseCRF variants
that operate on the full image resolution.

Here, we present experiments where we use the fast approximate bilateral filtering
algorithm of~\cite{adams2010fast}, which is also used in Chapter~\ref{chap:bnn}
for learning sparse high dimensional filters. This
choice allows for larger dimensions of matrix-vector multiplication. The reason for choosing
the explicit multiplication in Chapter~\ref{chap:binception} was that it was computationally faster.
For the small sizes of the involved matrices and vectors, the explicit computation is sufficient and we had no
GPU implementation of an approximate technique that matched this runtime. Also it
is conceptually easier and the gradient to the feature transformations ($\Lambda \mathbf{f}$) is
obtained using standard matrix calculus.

\subsubsection{Experiments}

We modified the existing segmentation architectures analogous to those in Chapter~\ref{chap:binception}.
The main difference is that, here, the inception modules use the lattice
approximation~\cite{adams2010fast} to compute the bilateral filtering.
Using the lattice approximation did not allow us to back-propagate through feature transformations ($\Lambda$)
and thus we used hand-specified feature scales as will be explained later.
Specifically, we take CNN architectures from the works
of~\cite{chen2014semantic,zheng2015conditional,bell2015minc} and insert the BI modules between
the spatial FC layers.
We use superpixels from~\cite{DollarICCV13edges}
for all the experiments with the lattice approximation. Experiments are
performed using Caffe neural network framework~\cite{jia2014caffe}.

\begin{table}
  \small
  \centering
  \begin{tabular}{p{5.5cm}>{\raggedright\arraybackslash}p{1.4cm}>{\centering\arraybackslash}p{2.2cm}}
    \toprule
		\textbf{Model} & \emph{IoU} & \emph{Runtime}(ms) \\
    \midrule

    %%%%%%%%%%%% Scores computed by us)%%%%%%%%%%%%
		\deeplablargefov & 68.9 & 145ms\\
    \midrule
    \bi{7}{2}-\bi{8}{10}& \textbf{73.8} & +600 \\
    \midrule
    \deeplablargefovcrf~\cite{chen2014semantic} & 72.7 & +830\\
    \deeplabmsclargefovcrf~\cite{chen2014semantic} & \textbf{73.6} & +880\\
    DeepLab-EdgeNet~\cite{chen2015semantic} & 71.7 & +30\\
    DeepLab-EdgeNet-CRF~\cite{chen2015semantic} & \textbf{73.6} & +860\\
  \bottomrule \\
  \end{tabular}
  \mycaption{Semantic Segmentation using the DeepLab model}
  {IoU scores on the Pascal VOC12 segmentation test dataset
  with different models and our modified inception model.
  Also shown are the corresponding runtimes in milliseconds. Runtimes
  also include superpixel computations (300 ms with Dollar superpixels~\cite{DollarICCV13edges})}
  \label{tab:largefovresults}
\end{table}

\paragraph{Semantic Segmentation}
The experiments in this section use the Pascal VOC12 segmentation dataset~\cite{voc2012segmentation} with 21 object classes and the images have a maximum resolution of 0.25 megapixels.
For all experiments on VOC12, we train using the extended training set of
10581 images collected by~\cite{hariharan2011moredata}.
We modified the \deeplab~network architecture of~\cite{chen2014semantic} and
the CRFasRNN architecture from~\cite{zheng2015conditional} which uses a CNN with
deconvolution layers followed by DenseCRF trained end-to-end.

\paragraph{DeepLab Model}\label{sec:deeplabmodel}
We experimented with the \bi{7}{2}-\bi{8}{10} inception model.
Results using the~\deeplab~model are summarized in Tab.~\ref{tab:largefovresults}.
Although we get similar improvements with inception modules as with the
explicit kernel computation, using lattice approximation is slower.

\begin{table}
  \small
  \centering
  \begin{tabular}{p{6.4cm}>{\raggedright\arraybackslash}p{1.8cm}>{\raggedright\arraybackslash}p{1.8cm}}
    \toprule
    \textbf{Model} & \emph{IoU (Val)} & \emph{IoU (Test)}\\
    \midrule
    %%%%%%%%%%%% Scores computed by us)%%%%%%%%%%%%
    CNN &  67.5 & - \\
    \deconv (CNN+Deconvolutions) & 69.8 & 72.0 \\
    \midrule
    \bi{3}{6}-\bi{4}{6}-\bi{7}{2}-\bi{8}{6}& 71.9 & - \\
    \bi{3}{6}-\bi{4}{6}-\bi{7}{2}-\bi{8}{6}-\gi{6}& 73.6 &  \href{http://host.robots.ox.ac.uk:8080/anonymous/VOTV5E.html}{\textbf{75.2}}\\
    \midrule
    \deconvcrf (CRF-RNN)~\cite{zheng2015conditional} & 73.0 & 74.7\\
    Context-CRF-RNN~\cite{yu2015multi} & ~~ - ~ & \textbf{75.3} \\
    \bottomrule \\
  \end{tabular}
  \mycaption{Semantic Segmentation using the CRFasRNN model}{IoU score corresponding to different models
  on Pascal VOC12 reduced validation / test segmentation dataset. The reduced validation set consists of 346 images
  as used in~\cite{zheng2015conditional} where we adapted the model from.}
  \label{tab:deconvresults-app}
\end{table}

\paragraph{CRFasRNN Model}\label{sec:deepinception}
We add BI modules after score-pool3, score-pool4, \fc{7} and \fc{8} $1\times1$ convolution layers
resulting in the \bi{3}{6}-\bi{4}{6}-\bi{7}{2}-\bi{8}{6}
model and also experimented with another variant where $BI_8$ is followed by another inception
module, G$(6)$, with 6 Gaussian kernels.
Note that here also we discarded both deconvolution and DenseCRF parts of the original model~\cite{zheng2015conditional}
and inserted the BI modules in the base CNN and found similar improvements compared to the inception modules with explicit
kernel computaion. See Tab.~\ref{tab:deconvresults-app} for results on the CRFasRNN model.

\paragraph{Material Segmentation}
Table~\ref{tab:mincresults-app} shows the results on the MINC dataset~\cite{bell2015minc}
obtained by modifying the AlexNet architecture with our inception modules. We observe
similar improvements as with explicit kernel construction.
For this model, we do not provide any learned setup due to very limited segment training
data. The weights to combine outputs in the bilateral inception layer are
found by validation on the validation set.

\begin{table}[t]
  \small
  \centering
  \begin{tabular}{p{3.5cm}>{\centering\arraybackslash}p{4.0cm}}
    \toprule
    \textbf{Model} & Class / Total accuracy\\
    \midrule

    %%%%%%%%%%%% Scores computed by us)%%%%%%%%%%%%
    AlexNet CNN & 55.3 / 58.9 \\
    \midrule
    \bi{7}{2}-\bi{8}{6}& 68.5 / 71.8 \\
    \bi{7}{2}-\bi{8}{6}-G$(6)$& 67.6 / 73.1 \\
    \midrule
    AlexNet-CRF & 65.5 / 71.0 \\
    \bottomrule \\
  \end{tabular}
  \mycaption{Material Segmentation using AlexNet}{Pixel accuracy of different models on
  the MINC material segmentation test dataset~\cite{bell2015minc}.}
  \label{tab:mincresults-app}
\end{table}

\paragraph{Scales of Bilateral Inception Modules}
\label{sec:scales}

Unlike the explicit kernel technique presented in the main text (Chapter~\ref{chap:binception}),
we didn't back-propagate through feature transformation ($\Lambda$)
using the approximate bilateral filter technique.
So, the feature scales are hand-specified and validated, which are as follows.
The optimal scale values for the \bi{7}{2}-\bi{8}{2} model are found by validation for the best performance which are
$\sigma_{xy}$ = (0.1, 0.1) for the spatial (XY) kernel and $\sigma_{rgbxy}$ = (0.1, 0.1, 0.1, 0.01, 0.01) for color and position (RGBXY)  kernel.
Next, as more kernels are added to \bi{8}{2}, we set scales to be $\alpha$*($\sigma_{xy}$, $\sigma_{rgbxy}$).
The value of $\alpha$ is chosen as  1, 0.5, 0.1, 0.05, 0.1, at uniform interval, for the \bi{8}{10} bilateral inception module.


\subsection{Qualitative Results}
\label{sec:qualitative-app}

In this section, we present more qualitative results obtained using the BI module with explicit
kernel computation technique presented in Chapter~\ref{chap:binception}. Results on the Pascal VOC12
dataset~\cite{voc2012segmentation} using the DeepLab-LargeFOV model are shown in Fig.~\ref{fig:semantic_visuals-app},
followed by the results on MINC dataset~\cite{bell2015minc}
in Fig.~\ref{fig:material_visuals-app} and on
Cityscapes dataset~\cite{Cordts2015Cvprw} in Fig.~\ref{fig:street_visuals-app}.


\definecolor{voc_1}{RGB}{0, 0, 0}
\definecolor{voc_2}{RGB}{128, 0, 0}
\definecolor{voc_3}{RGB}{0, 128, 0}
\definecolor{voc_4}{RGB}{128, 128, 0}
\definecolor{voc_5}{RGB}{0, 0, 128}
\definecolor{voc_6}{RGB}{128, 0, 128}
\definecolor{voc_7}{RGB}{0, 128, 128}
\definecolor{voc_8}{RGB}{128, 128, 128}
\definecolor{voc_9}{RGB}{64, 0, 0}
\definecolor{voc_10}{RGB}{192, 0, 0}
\definecolor{voc_11}{RGB}{64, 128, 0}
\definecolor{voc_12}{RGB}{192, 128, 0}
\definecolor{voc_13}{RGB}{64, 0, 128}
\definecolor{voc_14}{RGB}{192, 0, 128}
\definecolor{voc_15}{RGB}{64, 128, 128}
\definecolor{voc_16}{RGB}{192, 128, 128}
\definecolor{voc_17}{RGB}{0, 64, 0}
\definecolor{voc_18}{RGB}{128, 64, 0}
\definecolor{voc_19}{RGB}{0, 192, 0}
\definecolor{voc_20}{RGB}{128, 192, 0}
\definecolor{voc_21}{RGB}{0, 64, 128}
\definecolor{voc_22}{RGB}{128, 64, 128}

\begin{figure*}[!ht]
  \small
  \centering
  \fcolorbox{white}{voc_1}{\rule{0pt}{4pt}\rule{4pt}{0pt}} Background~~
  \fcolorbox{white}{voc_2}{\rule{0pt}{4pt}\rule{4pt}{0pt}} Aeroplane~~
  \fcolorbox{white}{voc_3}{\rule{0pt}{4pt}\rule{4pt}{0pt}} Bicycle~~
  \fcolorbox{white}{voc_4}{\rule{0pt}{4pt}\rule{4pt}{0pt}} Bird~~
  \fcolorbox{white}{voc_5}{\rule{0pt}{4pt}\rule{4pt}{0pt}} Boat~~
  \fcolorbox{white}{voc_6}{\rule{0pt}{4pt}\rule{4pt}{0pt}} Bottle~~
  \fcolorbox{white}{voc_7}{\rule{0pt}{4pt}\rule{4pt}{0pt}} Bus~~
  \fcolorbox{white}{voc_8}{\rule{0pt}{4pt}\rule{4pt}{0pt}} Car~~\\
  \fcolorbox{white}{voc_9}{\rule{0pt}{4pt}\rule{4pt}{0pt}} Cat~~
  \fcolorbox{white}{voc_10}{\rule{0pt}{4pt}\rule{4pt}{0pt}} Chair~~
  \fcolorbox{white}{voc_11}{\rule{0pt}{4pt}\rule{4pt}{0pt}} Cow~~
  \fcolorbox{white}{voc_12}{\rule{0pt}{4pt}\rule{4pt}{0pt}} Dining Table~~
  \fcolorbox{white}{voc_13}{\rule{0pt}{4pt}\rule{4pt}{0pt}} Dog~~
  \fcolorbox{white}{voc_14}{\rule{0pt}{4pt}\rule{4pt}{0pt}} Horse~~
  \fcolorbox{white}{voc_15}{\rule{0pt}{4pt}\rule{4pt}{0pt}} Motorbike~~
  \fcolorbox{white}{voc_16}{\rule{0pt}{4pt}\rule{4pt}{0pt}} Person~~\\
  \fcolorbox{white}{voc_17}{\rule{0pt}{4pt}\rule{4pt}{0pt}} Potted Plant~~
  \fcolorbox{white}{voc_18}{\rule{0pt}{4pt}\rule{4pt}{0pt}} Sheep~~
  \fcolorbox{white}{voc_19}{\rule{0pt}{4pt}\rule{4pt}{0pt}} Sofa~~
  \fcolorbox{white}{voc_20}{\rule{0pt}{4pt}\rule{4pt}{0pt}} Train~~
  \fcolorbox{white}{voc_21}{\rule{0pt}{4pt}\rule{4pt}{0pt}} TV monitor~~\\


  \subfigure{%
    \includegraphics[width=.15\columnwidth]{figures/supplementary/2008_001308_given.png}
  }
  \subfigure{%
    \includegraphics[width=.15\columnwidth]{figures/supplementary/2008_001308_sp.png}
  }
  \subfigure{%
    \includegraphics[width=.15\columnwidth]{figures/supplementary/2008_001308_gt.png}
  }
  \subfigure{%
    \includegraphics[width=.15\columnwidth]{figures/supplementary/2008_001308_cnn.png}
  }
  \subfigure{%
    \includegraphics[width=.15\columnwidth]{figures/supplementary/2008_001308_crf.png}
  }
  \subfigure{%
    \includegraphics[width=.15\columnwidth]{figures/supplementary/2008_001308_ours.png}
  }\\[-2ex]


  \subfigure{%
    \includegraphics[width=.15\columnwidth]{figures/supplementary/2008_001821_given.png}
  }
  \subfigure{%
    \includegraphics[width=.15\columnwidth]{figures/supplementary/2008_001821_sp.png}
  }
  \subfigure{%
    \includegraphics[width=.15\columnwidth]{figures/supplementary/2008_001821_gt.png}
  }
  \subfigure{%
    \includegraphics[width=.15\columnwidth]{figures/supplementary/2008_001821_cnn.png}
  }
  \subfigure{%
    \includegraphics[width=.15\columnwidth]{figures/supplementary/2008_001821_crf.png}
  }
  \subfigure{%
    \includegraphics[width=.15\columnwidth]{figures/supplementary/2008_001821_ours.png}
  }\\[-2ex]



  \subfigure{%
    \includegraphics[width=.15\columnwidth]{figures/supplementary/2008_004612_given.png}
  }
  \subfigure{%
    \includegraphics[width=.15\columnwidth]{figures/supplementary/2008_004612_sp.png}
  }
  \subfigure{%
    \includegraphics[width=.15\columnwidth]{figures/supplementary/2008_004612_gt.png}
  }
  \subfigure{%
    \includegraphics[width=.15\columnwidth]{figures/supplementary/2008_004612_cnn.png}
  }
  \subfigure{%
    \includegraphics[width=.15\columnwidth]{figures/supplementary/2008_004612_crf.png}
  }
  \subfigure{%
    \includegraphics[width=.15\columnwidth]{figures/supplementary/2008_004612_ours.png}
  }\\[-2ex]


  \subfigure{%
    \includegraphics[width=.15\columnwidth]{figures/supplementary/2009_001008_given.png}
  }
  \subfigure{%
    \includegraphics[width=.15\columnwidth]{figures/supplementary/2009_001008_sp.png}
  }
  \subfigure{%
    \includegraphics[width=.15\columnwidth]{figures/supplementary/2009_001008_gt.png}
  }
  \subfigure{%
    \includegraphics[width=.15\columnwidth]{figures/supplementary/2009_001008_cnn.png}
  }
  \subfigure{%
    \includegraphics[width=.15\columnwidth]{figures/supplementary/2009_001008_crf.png}
  }
  \subfigure{%
    \includegraphics[width=.15\columnwidth]{figures/supplementary/2009_001008_ours.png}
  }\\[-2ex]




  \subfigure{%
    \includegraphics[width=.15\columnwidth]{figures/supplementary/2009_004497_given.png}
  }
  \subfigure{%
    \includegraphics[width=.15\columnwidth]{figures/supplementary/2009_004497_sp.png}
  }
  \subfigure{%
    \includegraphics[width=.15\columnwidth]{figures/supplementary/2009_004497_gt.png}
  }
  \subfigure{%
    \includegraphics[width=.15\columnwidth]{figures/supplementary/2009_004497_cnn.png}
  }
  \subfigure{%
    \includegraphics[width=.15\columnwidth]{figures/supplementary/2009_004497_crf.png}
  }
  \subfigure{%
    \includegraphics[width=.15\columnwidth]{figures/supplementary/2009_004497_ours.png}
  }\\[-2ex]



  \setcounter{subfigure}{0}
  \subfigure[\scriptsize Input]{%
    \includegraphics[width=.15\columnwidth]{figures/supplementary/2010_001327_given.png}
  }
  \subfigure[\scriptsize Superpixels]{%
    \includegraphics[width=.15\columnwidth]{figures/supplementary/2010_001327_sp.png}
  }
  \subfigure[\scriptsize GT]{%
    \includegraphics[width=.15\columnwidth]{figures/supplementary/2010_001327_gt.png}
  }
  \subfigure[\scriptsize Deeplab]{%
    \includegraphics[width=.15\columnwidth]{figures/supplementary/2010_001327_cnn.png}
  }
  \subfigure[\scriptsize +DenseCRF]{%
    \includegraphics[width=.15\columnwidth]{figures/supplementary/2010_001327_crf.png}
  }
  \subfigure[\scriptsize Using BI]{%
    \includegraphics[width=.15\columnwidth]{figures/supplementary/2010_001327_ours.png}
  }
  \mycaption{Semantic Segmentation}{Example results of semantic segmentation
  on the Pascal VOC12 dataset.
  (d)~depicts the DeepLab CNN result, (e)~CNN + 10 steps of mean-field inference,
  (f~result obtained with bilateral inception (BI) modules (\bi{6}{2}+\bi{7}{6}) between \fc~layers.}
  \label{fig:semantic_visuals-app}
\end{figure*}


\definecolor{minc_1}{HTML}{771111}
\definecolor{minc_2}{HTML}{CAC690}
\definecolor{minc_3}{HTML}{EEEEEE}
\definecolor{minc_4}{HTML}{7C8FA6}
\definecolor{minc_5}{HTML}{597D31}
\definecolor{minc_6}{HTML}{104410}
\definecolor{minc_7}{HTML}{BB819C}
\definecolor{minc_8}{HTML}{D0CE48}
\definecolor{minc_9}{HTML}{622745}
\definecolor{minc_10}{HTML}{666666}
\definecolor{minc_11}{HTML}{D54A31}
\definecolor{minc_12}{HTML}{101044}
\definecolor{minc_13}{HTML}{444126}
\definecolor{minc_14}{HTML}{75D646}
\definecolor{minc_15}{HTML}{DD4348}
\definecolor{minc_16}{HTML}{5C8577}
\definecolor{minc_17}{HTML}{C78472}
\definecolor{minc_18}{HTML}{75D6D0}
\definecolor{minc_19}{HTML}{5B4586}
\definecolor{minc_20}{HTML}{C04393}
\definecolor{minc_21}{HTML}{D69948}
\definecolor{minc_22}{HTML}{7370D8}
\definecolor{minc_23}{HTML}{7A3622}
\definecolor{minc_24}{HTML}{000000}

\begin{figure*}[!ht]
  \small % scriptsize
  \centering
  \fcolorbox{white}{minc_1}{\rule{0pt}{4pt}\rule{4pt}{0pt}} Brick~~
  \fcolorbox{white}{minc_2}{\rule{0pt}{4pt}\rule{4pt}{0pt}} Carpet~~
  \fcolorbox{white}{minc_3}{\rule{0pt}{4pt}\rule{4pt}{0pt}} Ceramic~~
  \fcolorbox{white}{minc_4}{\rule{0pt}{4pt}\rule{4pt}{0pt}} Fabric~~
  \fcolorbox{white}{minc_5}{\rule{0pt}{4pt}\rule{4pt}{0pt}} Foliage~~
  \fcolorbox{white}{minc_6}{\rule{0pt}{4pt}\rule{4pt}{0pt}} Food~~
  \fcolorbox{white}{minc_7}{\rule{0pt}{4pt}\rule{4pt}{0pt}} Glass~~
  \fcolorbox{white}{minc_8}{\rule{0pt}{4pt}\rule{4pt}{0pt}} Hair~~\\
  \fcolorbox{white}{minc_9}{\rule{0pt}{4pt}\rule{4pt}{0pt}} Leather~~
  \fcolorbox{white}{minc_10}{\rule{0pt}{4pt}\rule{4pt}{0pt}} Metal~~
  \fcolorbox{white}{minc_11}{\rule{0pt}{4pt}\rule{4pt}{0pt}} Mirror~~
  \fcolorbox{white}{minc_12}{\rule{0pt}{4pt}\rule{4pt}{0pt}} Other~~
  \fcolorbox{white}{minc_13}{\rule{0pt}{4pt}\rule{4pt}{0pt}} Painted~~
  \fcolorbox{white}{minc_14}{\rule{0pt}{4pt}\rule{4pt}{0pt}} Paper~~
  \fcolorbox{white}{minc_15}{\rule{0pt}{4pt}\rule{4pt}{0pt}} Plastic~~\\
  \fcolorbox{white}{minc_16}{\rule{0pt}{4pt}\rule{4pt}{0pt}} Polished Stone~~
  \fcolorbox{white}{minc_17}{\rule{0pt}{4pt}\rule{4pt}{0pt}} Skin~~
  \fcolorbox{white}{minc_18}{\rule{0pt}{4pt}\rule{4pt}{0pt}} Sky~~
  \fcolorbox{white}{minc_19}{\rule{0pt}{4pt}\rule{4pt}{0pt}} Stone~~
  \fcolorbox{white}{minc_20}{\rule{0pt}{4pt}\rule{4pt}{0pt}} Tile~~
  \fcolorbox{white}{minc_21}{\rule{0pt}{4pt}\rule{4pt}{0pt}} Wallpaper~~
  \fcolorbox{white}{minc_22}{\rule{0pt}{4pt}\rule{4pt}{0pt}} Water~~
  \fcolorbox{white}{minc_23}{\rule{0pt}{4pt}\rule{4pt}{0pt}} Wood~~\\
  \subfigure{%
    \includegraphics[width=.15\columnwidth]{figures/supplementary/000008468_given.png}
  }
  \subfigure{%
    \includegraphics[width=.15\columnwidth]{figures/supplementary/000008468_sp.png}
  }
  \subfigure{%
    \includegraphics[width=.15\columnwidth]{figures/supplementary/000008468_gt.png}
  }
  \subfigure{%
    \includegraphics[width=.15\columnwidth]{figures/supplementary/000008468_cnn.png}
  }
  \subfigure{%
    \includegraphics[width=.15\columnwidth]{figures/supplementary/000008468_crf.png}
  }
  \subfigure{%
    \includegraphics[width=.15\columnwidth]{figures/supplementary/000008468_ours.png}
  }\\[-2ex]

  \subfigure{%
    \includegraphics[width=.15\columnwidth]{figures/supplementary/000009053_given.png}
  }
  \subfigure{%
    \includegraphics[width=.15\columnwidth]{figures/supplementary/000009053_sp.png}
  }
  \subfigure{%
    \includegraphics[width=.15\columnwidth]{figures/supplementary/000009053_gt.png}
  }
  \subfigure{%
    \includegraphics[width=.15\columnwidth]{figures/supplementary/000009053_cnn.png}
  }
  \subfigure{%
    \includegraphics[width=.15\columnwidth]{figures/supplementary/000009053_crf.png}
  }
  \subfigure{%
    \includegraphics[width=.15\columnwidth]{figures/supplementary/000009053_ours.png}
  }\\[-2ex]




  \subfigure{%
    \includegraphics[width=.15\columnwidth]{figures/supplementary/000014977_given.png}
  }
  \subfigure{%
    \includegraphics[width=.15\columnwidth]{figures/supplementary/000014977_sp.png}
  }
  \subfigure{%
    \includegraphics[width=.15\columnwidth]{figures/supplementary/000014977_gt.png}
  }
  \subfigure{%
    \includegraphics[width=.15\columnwidth]{figures/supplementary/000014977_cnn.png}
  }
  \subfigure{%
    \includegraphics[width=.15\columnwidth]{figures/supplementary/000014977_crf.png}
  }
  \subfigure{%
    \includegraphics[width=.15\columnwidth]{figures/supplementary/000014977_ours.png}
  }\\[-2ex]


  \subfigure{%
    \includegraphics[width=.15\columnwidth]{figures/supplementary/000022922_given.png}
  }
  \subfigure{%
    \includegraphics[width=.15\columnwidth]{figures/supplementary/000022922_sp.png}
  }
  \subfigure{%
    \includegraphics[width=.15\columnwidth]{figures/supplementary/000022922_gt.png}
  }
  \subfigure{%
    \includegraphics[width=.15\columnwidth]{figures/supplementary/000022922_cnn.png}
  }
  \subfigure{%
    \includegraphics[width=.15\columnwidth]{figures/supplementary/000022922_crf.png}
  }
  \subfigure{%
    \includegraphics[width=.15\columnwidth]{figures/supplementary/000022922_ours.png}
  }\\[-2ex]


  \subfigure{%
    \includegraphics[width=.15\columnwidth]{figures/supplementary/000025711_given.png}
  }
  \subfigure{%
    \includegraphics[width=.15\columnwidth]{figures/supplementary/000025711_sp.png}
  }
  \subfigure{%
    \includegraphics[width=.15\columnwidth]{figures/supplementary/000025711_gt.png}
  }
  \subfigure{%
    \includegraphics[width=.15\columnwidth]{figures/supplementary/000025711_cnn.png}
  }
  \subfigure{%
    \includegraphics[width=.15\columnwidth]{figures/supplementary/000025711_crf.png}
  }
  \subfigure{%
    \includegraphics[width=.15\columnwidth]{figures/supplementary/000025711_ours.png}
  }\\[-2ex]


  \subfigure{%
    \includegraphics[width=.15\columnwidth]{figures/supplementary/000034473_given.png}
  }
  \subfigure{%
    \includegraphics[width=.15\columnwidth]{figures/supplementary/000034473_sp.png}
  }
  \subfigure{%
    \includegraphics[width=.15\columnwidth]{figures/supplementary/000034473_gt.png}
  }
  \subfigure{%
    \includegraphics[width=.15\columnwidth]{figures/supplementary/000034473_cnn.png}
  }
  \subfigure{%
    \includegraphics[width=.15\columnwidth]{figures/supplementary/000034473_crf.png}
  }
  \subfigure{%
    \includegraphics[width=.15\columnwidth]{figures/supplementary/000034473_ours.png}
  }\\[-2ex]


  \subfigure{%
    \includegraphics[width=.15\columnwidth]{figures/supplementary/000035463_given.png}
  }
  \subfigure{%
    \includegraphics[width=.15\columnwidth]{figures/supplementary/000035463_sp.png}
  }
  \subfigure{%
    \includegraphics[width=.15\columnwidth]{figures/supplementary/000035463_gt.png}
  }
  \subfigure{%
    \includegraphics[width=.15\columnwidth]{figures/supplementary/000035463_cnn.png}
  }
  \subfigure{%
    \includegraphics[width=.15\columnwidth]{figures/supplementary/000035463_crf.png}
  }
  \subfigure{%
    \includegraphics[width=.15\columnwidth]{figures/supplementary/000035463_ours.png}
  }\\[-2ex]


  \setcounter{subfigure}{0}
  \subfigure[\scriptsize Input]{%
    \includegraphics[width=.15\columnwidth]{figures/supplementary/000035993_given.png}
  }
  \subfigure[\scriptsize Superpixels]{%
    \includegraphics[width=.15\columnwidth]{figures/supplementary/000035993_sp.png}
  }
  \subfigure[\scriptsize GT]{%
    \includegraphics[width=.15\columnwidth]{figures/supplementary/000035993_gt.png}
  }
  \subfigure[\scriptsize AlexNet]{%
    \includegraphics[width=.15\columnwidth]{figures/supplementary/000035993_cnn.png}
  }
  \subfigure[\scriptsize +DenseCRF]{%
    \includegraphics[width=.15\columnwidth]{figures/supplementary/000035993_crf.png}
  }
  \subfigure[\scriptsize Using BI]{%
    \includegraphics[width=.15\columnwidth]{figures/supplementary/000035993_ours.png}
  }
  \mycaption{Material Segmentation}{Example results of material segmentation.
  (d)~depicts the AlexNet CNN result, (e)~CNN + 10 steps of mean-field inference,
  (f)~result obtained with bilateral inception (BI) modules (\bi{7}{2}+\bi{8}{6}) between
  \fc~layers.}
\label{fig:material_visuals-app}
\end{figure*}


\definecolor{city_1}{RGB}{128, 64, 128}
\definecolor{city_2}{RGB}{244, 35, 232}
\definecolor{city_3}{RGB}{70, 70, 70}
\definecolor{city_4}{RGB}{102, 102, 156}
\definecolor{city_5}{RGB}{190, 153, 153}
\definecolor{city_6}{RGB}{153, 153, 153}
\definecolor{city_7}{RGB}{250, 170, 30}
\definecolor{city_8}{RGB}{220, 220, 0}
\definecolor{city_9}{RGB}{107, 142, 35}
\definecolor{city_10}{RGB}{152, 251, 152}
\definecolor{city_11}{RGB}{70, 130, 180}
\definecolor{city_12}{RGB}{220, 20, 60}
\definecolor{city_13}{RGB}{255, 0, 0}
\definecolor{city_14}{RGB}{0, 0, 142}
\definecolor{city_15}{RGB}{0, 0, 70}
\definecolor{city_16}{RGB}{0, 60, 100}
\definecolor{city_17}{RGB}{0, 80, 100}
\definecolor{city_18}{RGB}{0, 0, 230}
\definecolor{city_19}{RGB}{119, 11, 32}
\begin{figure*}[!ht]
  \small % scriptsize
  \centering


  \subfigure{%
    \includegraphics[width=.18\columnwidth]{figures/supplementary/frankfurt00000_016005_given.png}
  }
  \subfigure{%
    \includegraphics[width=.18\columnwidth]{figures/supplementary/frankfurt00000_016005_sp.png}
  }
  \subfigure{%
    \includegraphics[width=.18\columnwidth]{figures/supplementary/frankfurt00000_016005_gt.png}
  }
  \subfigure{%
    \includegraphics[width=.18\columnwidth]{figures/supplementary/frankfurt00000_016005_cnn.png}
  }
  \subfigure{%
    \includegraphics[width=.18\columnwidth]{figures/supplementary/frankfurt00000_016005_ours.png}
  }\\[-2ex]

  \subfigure{%
    \includegraphics[width=.18\columnwidth]{figures/supplementary/frankfurt00000_004617_given.png}
  }
  \subfigure{%
    \includegraphics[width=.18\columnwidth]{figures/supplementary/frankfurt00000_004617_sp.png}
  }
  \subfigure{%
    \includegraphics[width=.18\columnwidth]{figures/supplementary/frankfurt00000_004617_gt.png}
  }
  \subfigure{%
    \includegraphics[width=.18\columnwidth]{figures/supplementary/frankfurt00000_004617_cnn.png}
  }
  \subfigure{%
    \includegraphics[width=.18\columnwidth]{figures/supplementary/frankfurt00000_004617_ours.png}
  }\\[-2ex]

  \subfigure{%
    \includegraphics[width=.18\columnwidth]{figures/supplementary/frankfurt00000_020880_given.png}
  }
  \subfigure{%
    \includegraphics[width=.18\columnwidth]{figures/supplementary/frankfurt00000_020880_sp.png}
  }
  \subfigure{%
    \includegraphics[width=.18\columnwidth]{figures/supplementary/frankfurt00000_020880_gt.png}
  }
  \subfigure{%
    \includegraphics[width=.18\columnwidth]{figures/supplementary/frankfurt00000_020880_cnn.png}
  }
  \subfigure{%
    \includegraphics[width=.18\columnwidth]{figures/supplementary/frankfurt00000_020880_ours.png}
  }\\[-2ex]



  \subfigure{%
    \includegraphics[width=.18\columnwidth]{figures/supplementary/frankfurt00001_007285_given.png}
  }
  \subfigure{%
    \includegraphics[width=.18\columnwidth]{figures/supplementary/frankfurt00001_007285_sp.png}
  }
  \subfigure{%
    \includegraphics[width=.18\columnwidth]{figures/supplementary/frankfurt00001_007285_gt.png}
  }
  \subfigure{%
    \includegraphics[width=.18\columnwidth]{figures/supplementary/frankfurt00001_007285_cnn.png}
  }
  \subfigure{%
    \includegraphics[width=.18\columnwidth]{figures/supplementary/frankfurt00001_007285_ours.png}
  }\\[-2ex]


  \subfigure{%
    \includegraphics[width=.18\columnwidth]{figures/supplementary/frankfurt00001_059789_given.png}
  }
  \subfigure{%
    \includegraphics[width=.18\columnwidth]{figures/supplementary/frankfurt00001_059789_sp.png}
  }
  \subfigure{%
    \includegraphics[width=.18\columnwidth]{figures/supplementary/frankfurt00001_059789_gt.png}
  }
  \subfigure{%
    \includegraphics[width=.18\columnwidth]{figures/supplementary/frankfurt00001_059789_cnn.png}
  }
  \subfigure{%
    \includegraphics[width=.18\columnwidth]{figures/supplementary/frankfurt00001_059789_ours.png}
  }\\[-2ex]


  \subfigure{%
    \includegraphics[width=.18\columnwidth]{figures/supplementary/frankfurt00001_068208_given.png}
  }
  \subfigure{%
    \includegraphics[width=.18\columnwidth]{figures/supplementary/frankfurt00001_068208_sp.png}
  }
  \subfigure{%
    \includegraphics[width=.18\columnwidth]{figures/supplementary/frankfurt00001_068208_gt.png}
  }
  \subfigure{%
    \includegraphics[width=.18\columnwidth]{figures/supplementary/frankfurt00001_068208_cnn.png}
  }
  \subfigure{%
    \includegraphics[width=.18\columnwidth]{figures/supplementary/frankfurt00001_068208_ours.png}
  }\\[-2ex]

  \subfigure{%
    \includegraphics[width=.18\columnwidth]{figures/supplementary/frankfurt00001_082466_given.png}
  }
  \subfigure{%
    \includegraphics[width=.18\columnwidth]{figures/supplementary/frankfurt00001_082466_sp.png}
  }
  \subfigure{%
    \includegraphics[width=.18\columnwidth]{figures/supplementary/frankfurt00001_082466_gt.png}
  }
  \subfigure{%
    \includegraphics[width=.18\columnwidth]{figures/supplementary/frankfurt00001_082466_cnn.png}
  }
  \subfigure{%
    \includegraphics[width=.18\columnwidth]{figures/supplementary/frankfurt00001_082466_ours.png}
  }\\[-2ex]

  \subfigure{%
    \includegraphics[width=.18\columnwidth]{figures/supplementary/lindau00033_000019_given.png}
  }
  \subfigure{%
    \includegraphics[width=.18\columnwidth]{figures/supplementary/lindau00033_000019_sp.png}
  }
  \subfigure{%
    \includegraphics[width=.18\columnwidth]{figures/supplementary/lindau00033_000019_gt.png}
  }
  \subfigure{%
    \includegraphics[width=.18\columnwidth]{figures/supplementary/lindau00033_000019_cnn.png}
  }
  \subfigure{%
    \includegraphics[width=.18\columnwidth]{figures/supplementary/lindau00033_000019_ours.png}
  }\\[-2ex]

  \subfigure{%
    \includegraphics[width=.18\columnwidth]{figures/supplementary/lindau00052_000019_given.png}
  }
  \subfigure{%
    \includegraphics[width=.18\columnwidth]{figures/supplementary/lindau00052_000019_sp.png}
  }
  \subfigure{%
    \includegraphics[width=.18\columnwidth]{figures/supplementary/lindau00052_000019_gt.png}
  }
  \subfigure{%
    \includegraphics[width=.18\columnwidth]{figures/supplementary/lindau00052_000019_cnn.png}
  }
  \subfigure{%
    \includegraphics[width=.18\columnwidth]{figures/supplementary/lindau00052_000019_ours.png}
  }\\[-2ex]




  \subfigure{%
    \includegraphics[width=.18\columnwidth]{figures/supplementary/lindau00027_000019_given.png}
  }
  \subfigure{%
    \includegraphics[width=.18\columnwidth]{figures/supplementary/lindau00027_000019_sp.png}
  }
  \subfigure{%
    \includegraphics[width=.18\columnwidth]{figures/supplementary/lindau00027_000019_gt.png}
  }
  \subfigure{%
    \includegraphics[width=.18\columnwidth]{figures/supplementary/lindau00027_000019_cnn.png}
  }
  \subfigure{%
    \includegraphics[width=.18\columnwidth]{figures/supplementary/lindau00027_000019_ours.png}
  }\\[-2ex]



  \setcounter{subfigure}{0}
  \subfigure[\scriptsize Input]{%
    \includegraphics[width=.18\columnwidth]{figures/supplementary/lindau00029_000019_given.png}
  }
  \subfigure[\scriptsize Superpixels]{%
    \includegraphics[width=.18\columnwidth]{figures/supplementary/lindau00029_000019_sp.png}
  }
  \subfigure[\scriptsize GT]{%
    \includegraphics[width=.18\columnwidth]{figures/supplementary/lindau00029_000019_gt.png}
  }
  \subfigure[\scriptsize Deeplab]{%
    \includegraphics[width=.18\columnwidth]{figures/supplementary/lindau00029_000019_cnn.png}
  }
  \subfigure[\scriptsize Using BI]{%
    \includegraphics[width=.18\columnwidth]{figures/supplementary/lindau00029_000019_ours.png}
  }%\\[-2ex]

  \mycaption{Street Scene Segmentation}{Example results of street scene segmentation.
  (d)~depicts the DeepLab results, (e)~result obtained by adding bilateral inception (BI) modules (\bi{6}{2}+\bi{7}{6}) between \fc~layers.}
\label{fig:street_visuals-app}
\end{figure*}



%%%%%%%%%%%%%%%%%%%%%%%%%%%%%%%%%%%%%%%%%%%%%%%%%%


% Don't change these lines
\bsp	% typesetting comment
\label{lastpage}
\end{document}

% End of mnras_template.tex
