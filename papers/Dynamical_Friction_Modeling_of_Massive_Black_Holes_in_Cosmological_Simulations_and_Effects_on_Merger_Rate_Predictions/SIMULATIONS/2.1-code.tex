\subsection{The Numerical Code}
\label{subsec:code}

We use the massively parallel cosmological smoothed particle hydrodynamic (SPH) simulation software, MP-Gadget \citep{Feng2016}, to run all the simulations in this paper. 
The hydrodynamics solver of MP-Gadget adopts the new pressure-entropy formulation of SPH \citep{Hopkins2013}.
We apply a variety of sub-grid models to model the galaxy and black hole formation and associated feedback processes already validated against a number of observables \citep[e.g.][]{Feng2016,Wilkins2017,Waters2016,DiMatteo2017,Tenneti2018,Huang2018,Ni2018,Bhowmick2018,Marshall2020,Marshall2021}. Here we review briefly the main aspects of these.
In the simulations, gas is allowed to cool through radiative processes~\citep{Katz}, including metal cooling. For metal cooling, we follow the method in \cite{Vogelsberger2014}, and scale a solar metallicity template according to the metallicity of gas particles.
Our star formation (SF) is based on a multi-phase SF model ~\citep{SH03} with modifications following~\cite{Vogelsberger2013}.
We model the formation of molecular hydrogen and its effects on SF at low metallicity according to the prescription of \cite{Krumholtz}. 
We self-consistently estimate the fraction of molecular hydrogen gas from the baryon column density, which in turn couples the density gradient to the SF rate.
We include Type II supernova wind feedback ~\citep[the model used in BlueTides][]{Feng2016,Okamoto2010} in our simulations, assuming that the wind speed is proportional to the local one dimensional dark matter velocity dispersion.

BHs are seeded with an initial seed mass of $M_{\mathrm {seed}} = 5 \times 10^5 M_{\odot}/h$ in halos with mass more than $10^{10} M_{\odot}/h$ if the halo does not already contain a BH. We model BH growth and AGN feedback in the same way as in the \textit{MassiveBlack} $I \& II$ simulations, using the BH sub-grid model developed in \cite{SDH2005,DSH2005} with modifications consistent with BlueTides. 
The gas accretion rate onto the BHs is given by Bondi accretion rate,
\begin{equation}
\label{equation:Bondi}
    \dot{M}_B = \alpha \frac{4 \pi G^2 M_{\rm BH}^2 \rho}{(c^2_s+v_{\rm rel}^2)^{3/2}},
\end{equation}
where $c_s$ and $\rho$ are the local sound speed and density of the cold gas, $v_{\rm rel}$ is the relative velocity of the BH to the nearby gas, and $\alpha=100$ is a numerical correction factor introduced by \citep{Springel2005b}. This can also be eliminated (without affecting the values of the accretion rate significantly) in favor of a more detailed modeling of the contributions in the cold and hot phase accretion \citep{Pelupessy2006}.


We allow for super-Eddington accretion in the simulation \citep[e.g.][]{Volonteri2005,Volonteri2015}, but limit the accretion rate to 2 times the Eddington accretion rate:
\begin{equation}
\label{equation:Meddington}
    \dot{M}_{\rm Edd} = \frac{4 \pi G M_{\rm BH} m_p}{\eta \sigma_{T} c},
\end{equation}
where $m_p$ is the proton mass, $\sigma_T$ the Thompson cross section, c is the speed of light, and $\eta=0.1$ is the radiative efficiency of the accretion flow onto the BH.
Therefore, the BH accretion rate is determined by:
\begin{equation}
    \dot{M}_{\rm BH} = {\rm Min} (\dot{M}_{B}, 2\dot{M}_{\rm Edd}).
\end{equation}


The SMBH is assumed to radiate with a bolometric luminosity $L_{\rm Bol}$ proportional to the accretion rate $\dot{M}_{\rm BH}$:
\begin{equation}
    L_{\rm Bol} = \eta \dot{M}_{\rm BH} c^2
\end{equation}
with $\eta = 0.1$ being the mass-to-light conversion efficiency in an accretion disk according to \cite{Shakura1973}.
5\% of the radiated energy is thermally coupled to the surrounding gas that resides within twice the radius of the SPH smoothing kernel of the BH particle. This scale is typically about $\sim 1-3\%$ of the virial radius of the halo.

The cosmological parameters used are from the nine-year Wilkinson Microwave Anisotropy Probe (WMAP) \citep{Hinshaw2013} ($\Omega_0=0.2814$, $\Omega_\Lambda=0.7186$, $\Omega_b=0.0464$, $\sigma_8=0.82$, $h=0.697$, $n_s=0.971$).
For our fiducial resolution simulations, the mass resolution is $M_{\rm DM} = 1.2 \times 10^7 M_\odot/h$ and $M_{\rm gas} = 2.4 \times 10^6 M_\odot/h$ in the initial conditions.
The mass of a star particle is $M_{*} = 1/4 M_{\rm gas} = 6 \times 10^5 M_\odot/h$. The gravitational softening length is $\epsilon_g = 1.5$ ckpc/$h$ in the fiducial resolution for both DM and gas particles. The detailed simulation and model parameters are listed in Tables \ref{tab:cons} and \ref{tab:norm}. 