\subsection{Gaussian Constrained Realization}
\label{subsec:CR}
% \nianyi{Need inputs from Yueying's paper here}

MBHs at high redshift typically reside in rare density peaks, which are absent in the small uniform box ($\sim 10$ Mpc/$h$) simulations. 
In order to test the dynamics for more massive BHs (with $M_{\rm BH} > 10^8 M_{\odot}$) in our small volume simulation, we apply the Constrained Realization (CR) technique \footnote{\url{https://github.com/yueyingn/gaussianCR}} to impose a relatively high density peak in the initial condition (IC), with peak height $\nu = 4 \sigma_0$ on scale of $R_G = 1$ Mpc/$h$.  

The prescription for the CR technique was first introduced by \cite{Hoffman1991} as an optimal way to construct samples of constrained Gaussian random fields.
This formalism was further elaborated and extended by \cite{vandeWeygaert1996} as a more general type of convolution format constraints.
The CR technique imposes constraints on different characteristics of the linear density field. 
It can specify density peaks in the Gaussian random field with any desired height and shape, providing an efficient way to study rare massive objects with a relatively small box and thus lower computational costs \citep[e.g.][]{Ni2020}.
In this study, we specify a $4 \sigma_0$ density peak in the IC of our $10$ Mpc/$h$ box, boosting the early formation of halos and BHs to study the dynamics of massive BHs. Before applying the peak height constraint, the highest density peak has $\nu = 2.4 \sigma_0$ and the largest BH has mass $<6\times 10^7M_\odot$ at $z=3$ in our fiducial model (\texttt{DF\_4DM\_G} in Table \ref{tab:cons}). After applying the $4 \sigma_0$ constraint, the largest BH has mass $3\times 10^8 M_\odot$ at $z=3$ in the same box.
