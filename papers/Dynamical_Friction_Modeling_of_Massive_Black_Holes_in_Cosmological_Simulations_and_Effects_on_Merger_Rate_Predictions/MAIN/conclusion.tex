\section{Conclusions}
\label{sec:conclusion}
In this work we have tested methods for implementing dynamical friction from collisionless particles (i.e. stars and dark matter) and gas in low-resolution cosmological simulations (with mass resolution $M_{\rm DM} \sim 10^7 M_{\odot}$ and spatial resolution of $\epsilon_g \sim$1kpc/h), both for single black hole evolution/mergers using constrained simulations, and for the black hole population using unconstrained simulations.

We showed that dynamical friction from collisionless particles can effectively assist the black hole orbit to decay to within $2\epsilon_g$ of the galaxy center, representing a marked improvement over models that do not include any dynamical correction. Importantly, we find that for our prescription to work well, the dynamical mass of the black holes must be at least twice the mass of the dark matter particles. This is in agreement with results from \cite{Tremmel2015}. The dynamical friction implementation from \cite{Tremmel2015} (DF(T15)) and our implementation adapted to lower-resolution simulations (DF(fid)) result in dynamical friction of a similar magnitude, and have comparable effects on the black holes' dynamics. However, we find that our fiducial model is marginally more suitable for low-resolution simulations, as the nature of the calculation results in less noisy force corrections.

After applying the dynamical friction and performing the gravitational bound check on the black hole pairs, the dynamical friction time of the black holes is consistent with analytical predictions, although the variances can be large for individual black holes due to their varied environments. We note that checking whether the two black holes are gravitationally bound at the time of the merger is necessary both for preventing sudden momentum injection onto the black holes, and for allowing a more realistic orbital decay time.

By direct comparison of the force magnitudes throughout the simulation, we find that dynamical friction from collisionless particles dominate in the majority of cases. The influence of gas drag is highest at the high redshifts, but even then it is typically similar to or less than the contribution from stars and dark matter. This is in broad agreement with the results from \cite{Pfister2019}, though we stress that our simulations cannot resolve the structure of gas on the smallest scales. It is possible that interactions with gas is still important, such as migration within circumbinary disks \citep[e.g.][]{Haiman2009}.

Using our fiducial DF+drag model, we calculate the cumulative merger rate down to $z=1.1$ using a $L_{\rm box}=35$ Mpc$/h$ simulation. Without considering any post-merger delays, we predict $\sim 2$ mergers per year for $z>1.1$, and we lower bound our prediction by a no-dynamical-friction run which predicts $\sim 1$ merger per year. 
Compared with existing predictions from hydro-dynamical simulations \citep[][]{Salcido2016,Katz2020,Volonteri2020}, our rates are consistent with the raw merger rates (rates before post-processing delays are added) from previous works of similar resolution. 
While the dynamics modeling has significant effects (factor of a few according to our experiments) on the BH merger rate, we also found that the different BH seeding criteria and mechanisms account also play a big role in the merger rate predictions.

 Our work has demonstrated the feasibility of recovering sub-kpc-scale BH dynamics in low-resolution cosmological simulations by adding the unresolved dynamical friction. This is the first step in improving upon the widely-adopted reposition model and in tracking the BH dynamics directly down to the resolution limit. Beyond the resolution limit, we still need to account for several smaller-scale binary processes before we can make realistic merger rate predictions \citep[e.g.][]{Quinlan1996,Sesana2007b,Haiman2009,Vasiliev2015,Dosopoulou2017,Bonetti2018,Kelley2017,Katz2020}. Nevertheless, having access to the full dynamical information of the binary at the time of the numerical merger also helps us to better model these small-scale processes. We will leave the analysis of post-merger delays for future works.

There are still several aspects of the DF model that remain somewhat uncertain. Most importantly, the parameters (e.g. $b_{\rm max}$,$M_{\rm dyn,seed}$) in the current dynamical friction model can induce uncertainties in the sinking timescale and the merger rate predictions. For example, reducing $M_{\rm dyn,seed}$ to a value similar to or below the dark matter particle mass will reduce the merger rate by a factor of two or more. Our current choice is well tested in our simulations, but it is still subject to the limitations of our spatial and mass resolution. The limit in the $M_{\rm BH}/M_{\rm DM}$ ratio also hinders comprehensive studies of BH seeding scenarios in the cosmological context. We would need insights from high-resolution simulations \citep[e.g.][]{Dosopoulou2017,Pfister2019} to better model the dynamics of low-mass BHs within cosmological simulations. 

