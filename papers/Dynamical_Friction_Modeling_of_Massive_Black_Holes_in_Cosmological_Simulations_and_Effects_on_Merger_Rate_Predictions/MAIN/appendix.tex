
\section{Dynamical Mass and Resolution Effect}
\label{app:res}
\begin{figure*}
\includegraphics[width=0.9\textwidth]{RESULTS/plots/app_big_plot.pdf}

\caption{\textbf{(a)}: Comparisons of different black hole seed dynamical mass. The effect of varying $M_{\rm dyn,seed}$ is small in this case. But this is partially due to the large BH we pick. \textbf{(b):} Comparison with higher-resolution run with the same $M_{\rm dyn}/M_{\rm DM}$ ratio.}
\label{fig:app_res}
\end{figure*}

%%%%%%%%%%%%%%%%%%%%%%%%%%%%%%%%%%%%%%%%%%
\subsection{Varying Dynamical Mass}

One major difference between our model and previous modeling of the dynamical friction is that we boost the mass term during the early stage of black hole growth by a factor of $k_{\rm dyn} = M_{\rm dyn,seed}/M_{\rm BH}$. This is to prevent the drifting of the black holes due to dynamical heating when the black hole mass is below the dark matter particle mass in the context of large and low-resolution cosmological simulations.

Here we show the effect of setting different $k_{\rm dyn}$ by running three simulations with the same resolution and dynamical friction models, but various $k_{\rm dyn}$ ratios. They are listed in Table \ref{tab:cons} as \texttt{DF\_4DM\_G}, \texttt{DF\_2DM\_G}, and \texttt{DF\_1DM\_G}, with $k_{\rm dyn}=4,2,1$, respectively.

Figure \ref{fig:app_res}(a) shows the evolution of the same black hole for different $k_{\rm dyn}$. By comparing the three cases, we can see that the black hole's behavior is very similar for all the physical quantities we have plotted. However, we also note that the similar behavior of different $M_{\rm dyn}$ is case-dependent. The case we present here is a black hole within a large density peak where the black hole is subject to a deep potential and can sink more easily, but the sinking of BHs in shallower potentials can be more sensitive to the seed dynamical mass. Nevertheless, $k_{\rm df}=2$ is generally sufficient to assist the sinking of most black holes and produces similar merger rates to $k_{\rm df}=4$ (see Appendix \ref{app:merger_param}). The convergence at $k_{\rm df}=2$ is consistent with the $M_{\rm BH}/M_{\rm DM}=3$ ratio used in \cite{Tremmel2018}, and relaxes the ratio used in previous works \citep[e.g.][]{Tremmel2015,Pfister2019} of $M_{\rm BH}\sim 10M_{\rm DM}$.

%%%%%%%%%%%%%%%%%%%%%%%%%%%%%%%%%%%%%%%%%%
\subsection{Resolution Effect}

Here we show how our model performs under different resolutions. For this experiment we use our fiducial resolution run \texttt{DF\_4DM\_G}, a higher resolution run  \texttt{DF\_HR\_4DM\_G} with the same $k_{\rm df}$, but a factor of three difference in the mass resolution, and a high resolution run \texttt{DF\_HR\_12DM\_G} with the same $M_{\rm dyn,seed}$ as the fiducial resolution run. We would want the black holes to behave similarly independent of resolution if the $M_{\rm dyn,seed}/M_{\rm DM}$ is kept constant.

Figure \ref{fig:app_res}(b) shows the same black hole in the simulations with different resolution. In the high-resolution run \texttt{DF\_HR\_4DM\_G}, even though the seeding dynamical mass is 3 times smaller than the low-resolution run, the sinking time remains the same. Furthermore, if we keep the absolute seeding dynamical mass the same in the low-resolution and high-resolution runs (by comparing \texttt{DF\_HR\_12DM\_G} with \texttt{DF\_4DM\_G}), the black holes still shows similar evolution. This indicates that a constant $k_{\rm df}  = M_{\rm dyn,seed}/M_{\rm DM}$ is robust under different resolutions, and our model of dynamical mass does converges to the true black hole mass if we go to higher resolutions.


%%%%%%%%%%%%%%%%%%%%%%%%%%%%%%%%%%%%%%%%%%



\section{DF(fid) vs. DF(T15): cases of smaller black holes evolution}
\label{app:df100}
\begin{figure*}
\includegraphics[width=0.33\textwidth]{RESULTS/plots/app_kernel1.pdf}
\includegraphics[width=0.33\textwidth]{RESULTS/plots/app_kernel2.pdf}
\includegraphics[width=0.33\textwidth]{RESULTS/plots/app_kernel3.pdf}
\caption{Components of the dynamical friction in the \texttt{DF(fid)\_4DM\_G} (\textbf{red}) and the \texttt{DF(T15)\_4DM\_G} (\textbf{blue}) simulations, for three $M<5\times 10^6 M_\odot$ black holes. In these cases, the number of particles within the SPH kernel is still at least an order of magnitude more than 100 at lower redshift. The value of the Coulomb logarithm is now mainly affected by $b_{\rm max}$, because we do not see as much noise in the velocity of the surrounding particles as in the case of a very large BH. In all three cases shown, the magnitude of the dynamical friction is similar in the two models.}
\label{fig:app_kernel}
\end{figure*}

In Section \ref{subsec:models}, we compared the two DF models by showing the example of an early forming black hole located at the center of the largest halo in the simulation. However, that black hole might not be representative of the entire BH population due to its early seeding and large mass. Now we pick more cases of smaller black holes to demonstrate the differences/similarities between the models. In particular, we will look at how the smaller BHs are affected by the DF(fid)/DF(T15) implementation.

Figure \ref{fig:app_kernel} shows the evolution of three small BHs in the \texttt{DF\_4DM\_G} and the \texttt{DF(T15)\_4DM\_G} simulations. We plot three $M_{\rm BH}<5\times 10^6 M_\odot$ black holes. In these cases, the number of particles within the SPH kernel is still at least an order of magnitude more than 100 at lower redshift, and so the density calculated in DF(T15) still tends to be larger but more noisy. The value of the Coulomb logarithm is now mainly affected by $b_{\rm max}$, because we do not see as much noise in the velocity of the surrounding particles as in the case of a very large BH. The density and the Coulomb logarithm counteract each other, and the magnitude of the dynamical friction is similar in the two models. 

These cases again verifies that the two models are consistent with each other, with DF(T15) a more localized implementation than DF(fid). The choice of DF(fid) as our fiducial model is mainly due to our resolution limit.


\section{Effect of Model Parameters on the Merger Rate}
\label{app:merger_param}
\begin{figure}
\includegraphics[width=0.5\textwidth]{RESULTS/plots/L15_mergers_bmax.pdf}

\caption{The cumulative merger rates for different values of $b_{\rm max}$, in the $L_{\rm  box}$=15 Mpc$/h$ simulations. We tested $b_{\rm max}$ values of 3 ckpc$/h$,10 ckpc$/h$ and 30 kpc, and the difference in the cumulative merger rate is less than 10\%. The difference between the DF(fid) models and the DF(T15) model with $b_{\rm max}$=1.5 ckpc is also very small. Hence, although different choices of $b_{\rm max}$ changes the magnitude of the dynamical friction, it does not affect the merger rate predictions significantly.}
\label{fig:merger_bmax}
\end{figure}

For the merger rate predictions in \ref{sec:L35}, we use the DF+Drag model with $b_{\rm max}$=10 ckpc$/h$ and $M_{\rm dyn,seed} = 4M_{\rm DM}$. In this section, we will show that the merger rate prediction is not sensitive to the choice of these two parameters, and hence our prediction is relatively robust against parameter variations within a reasonable range.

Figure \ref{fig:merger_bmax} shows the cumulative merger rates for different values of $b_{\rm max}$ in the $L_{\rm  box}$=15 Mpc$/h$ simulations. We tested $b_{\rm max}$ values of 3 ckpc$/h$,10 ckpc$/h$ and 30 kpc, and the difference in the cumulative merger rate is less than 10\%. The difference between the DF(T15) models and the DF(fid) model with $b_{\rm max}$=1.5 ckpc is also very small. Hence, although different choices of $b_{\rm max}$ changes the magnitude of the dynamical friction, it does not affect the merger rate predictions significantly.

We also test a lower value of $M_{\rm dyn, seed}=2M_{\rm DM}$ using the $L_{\rm  box}$=15 Mpc$/h$ simulation. The resulting cumulative merger rate prediction is also shown in Figure \ref{fig:merger_bmax}. Compared with the similar run with $M_{\rm dyn, seed}=4M_{\rm DM}$, the earliest merger is slightly postponed, but the cumulative rate at $z\sim 2$ has very little difference. Therefore, even though for the predictions in Section \ref{sec:L35} we have chosen a particular set of parameter values, changing those parameters would not affect the result significantly given the larger effects of other factors such as the resolution and seeding.
 