\section{Introduction}
\label{sec:introduction}
Super Massive Black Holes (SMBHs) are known to exist at the center of the majority of massive galaxies \citep[e.g.][]{Soltan1982,Kormendy1995,Magorrian1998,Kormendy2013}. 
As these galaxies merge \citep[e.g.][]{Lacey1993,Lotz2011,Rodriguez-Gomez2015}, the SMBHs that they host also go through mergers, resulting in the mass growth of the SMBH population \citep[e.g.][]{Begelman1980}.
SMBH mergers following their host galaxy mergers become an increasingly important aspect of SMBH growth for more massive black holes (BHs) in dense environments \citep[e.g.][]{Kulier2015}. 
As a by-product of BH mergers, gravitational waves are emitted, and their detection opens up a new channel for probing the formation and evolution of early BHs in the universe \citep[e.g.][]{Sesana2007a,Barausse2012}. 


The gravitational wave detection by LIGO \citep[][]{LIGO2016PhRvL.116f1102A} proves the experimental feasibility of using gravitational waves for studying BH binaries. 
While LIGO cannot detect gravitational waves from binaries more massive than $\sim 100 M_\odot$ \citep[][]{Mangiagli2019}, long-baseline experiments are being planned for detections of more massive BH binaries. 
Specifically, the upcoming Laser Interferometer Space Antenna (LISA) \citep{LISA2017arXiv170200786A} mission will be sensitive to low-frequency ($10^{-4}-10^{-1}$Hz) gravitational waves from the coalescence of massive black holes (MBHs) with masses $10^4-10^7 M_\odot$ up to $z\sim 20$. 
 At even lower
frequencies Pulsar Timing Arrays (PTAs) are already collecting data and the
Square Kilometer Array (SKA) in the next decade will be a major leap forward in
sensitivity. PTA observations are likely to identify a number of
continuous-wave sources representing the early inspiral phase of MBHBs.
PTAs experiments \citep[e.g.][]{Jenet2004,Jenet2005} may also detect the inspiral of tight MBH binaries with mass $>10^8 M_\odot$. 
While massive BH binaries are the primary sources for PTAs and LISA, these
two experiments probe different stages of massive BH evolution. PTAs are most sensitive to the early inspiral
(orbital periods of years or longer) of nearby ($z <1$) (massive) sources  \citep{Mingarelli2017}. 
In contrast, LISA is sensitive to the inspiral, merger, and ringdown of
MBHBs  at a wide range of redshifts \citep{Amaro-Seoane2012}. 
The two populations of SMBHBs probed by PTAs and LISA are linked
via the growth and evolution of SMBH across cosmic time. 

LISA will provide a unique way of probing the high-redshift universe and understanding the early formation of the SMBHs, especially when combined with the soon-to-come observations of the electromagnetic (EM) counterparts \citep[][]{Natarajan2017,DeGraf2020}.
For instance, they will potentially allow us to distinguish between different BH seeding mechanisms at high-redshift \citep[][]{Ricarte2018}, to obtain information on the dynamical evolution of massive black holes \citep[][]{Bonetti2019}, and to gain information about the gas properties within the accretion disc \citep[][]{Derdzinski2019}. 

To properly analyze the upcoming results from the gravitational wave as well as the EM observations, we need to gain a thorough understanding of the physics of these MBH mergers with theoretical tools and be able to make statistical predictions on the binary population. In particular, it is important that the BH dynamics is modeled accurately, so that we can minimize the degeneracy with other physical properties of the merger, and gain accurate information about when and where BH coalescence is expected.

Hydrodynamical cosmological simulations provide a natural ground for studying the evolution and mergers of MBHs. In particular, large-volume cosmological simulations \citep[e.g.][]{Hirschmann2014,Vogelsberger2014,Schaye2015,Feng2016,Volonteri2016,Pillepich2018,Dave2019} have the statistical power to make merger rate predictions for the upcoming observations. 

In order to accurately predict when black hole mergers occur in these simulations, one must account for the long journey of the central black holes after the merger of their host galaxies: during galaxy mergers, the central SMBHs are usually separated by as much as a few tens of kpc. These SMBHs then gradually lose their orbital energy and sink to the center of the new galaxy due to the dynamical friction exerted by the gas, stars and dark matter around them \citep[e.g.][]{Chandrasekhar1943,Ostriker1999}. When their separation reaches the sub-parsec scale, they form a binary and other energy-loss channels begin to dominate, such as scattering with stars \citep[e.g.][]{Quinlan1996,Sesana2007b,Vasiliev2015}, gas drag from the circumbinary disk \citep[e.g.][]{Haiman2009}, or three-body scattering with a third black hole \citep[e.g.][]{Bonetti2018}.

However, due to limited mass and spatial resolution, large-scale cosmological simulations cannot feasibly include detailed treatment of the black hole binary dynamics. Without any additional correction to the BH dynamics, the smoothed-away small-scale gravity prevents effective orbital decay of the black hole after the orbit approaches the gravitational softening length. Once the binary reaches the innermost region of the remnant galaxy, the gravitational potential (close to the resolution limit) can be noisy. Such a noisy potential can scatter the black hole around within the host galaxy, or in some cases even kick the BH to the outskirts of the galaxy if the black hole mass is small. To avoid unexpected scattering of the BHs around the center of the galaxy, large-volume cosmological simulations usually resort to pinning the black holes at the halo minimum potential (a.k.a. repositioning). This repositioning algorithm has the undesirable effect of making the black holes merge rather efficiently
once they reach the center of the galaxy. Post-processing techniques have been used \citep[e.g.][]{Salcido2016,Kelley2017,Katz2020,Volonteri2020} to account for the additional dynamical friction effects on scales close to the gravitational smoothing scales of the black holes. This allows for an approximate estimation of the expected delay in the BH mergers. The post-processing calculations are mostly based on idealized analytical models, and therefore do not account for the variety of individual black hole environments.

Due to the increased merger efficiency induced by BH repositioning and the limits of post-processing in dynamical friction  calculations, emerging works have been adding sub-grid modeling of dynamical friction  self-consistently in cosmological simulations and removing the artificial repositioning approximation.
\cite{Chapon2013,Dubois2014} are the first large simulations to include the dynamical friction from gas, while \cite{Hirschmann2014} and \cite{Tremmel2017} account for dynamical friction from collisionless particles, and both have shown success in stabilizing the black holes at the halo centers. The dynamical friction modeling and its effect on the BH merger time scale have been well-tested in \cite{Tremmel2015} and \cite{Pfister2019} in the context of their relatively high-resolution simulations in a controlled single-halo environment, but they have also pointed out the failure of their model when the dark matter particle mass exceeds the black hole mass, and so their models might not be directly applicable to lower-resolution cosmological simulations. In the context of low-resolution cosmological simulations, the dynamical friction modeling is less well-tested, and its effects on the BH evolution and merger rate are not fully explored.

In this work, we carefully develop and test the sub-grid modeling of dynamical friction from both gas and collisionless particles in the context of cosmological simulations with resolution similar to the aforementioned large-volume, low-resolution hydrodynamical simulations (i.e. with a spatial resolution of $\sim$1kpc and mass resolution of $M_{\rm DM}\sim 10^7M_\odot$). We evaluate the models both by looking at individual black hole dynamics, growth and mergers, and by statistically comparing the behavior of different models in terms of the mass growth and merger statistics. In particular, we focus on how various models affect the BH merger rate in the cosmological simulations, which is essential for making merger rate predictions for the LISA mission.
    
This paper proceeds as follows: in Section \ref{sec:simulations} we describe the numerical code and the gaussian-constrained technique we use to study large SMBHs within a small volume. In Section \ref{sec:bh_model}, we talk about the different dynamical models for black hole mergers that we study and test in this work. Section \ref{sec:case} is dedicated to investigating the effect of the different models on the evolution of individual black holes, while Section \ref{sec:stats} studies the differences statistically. Finally, in Section \ref{sec:L35}, we show merger rate predictions with a model chosen based on the results of the previous sections, and compare with previous simulations at similar resolution.
