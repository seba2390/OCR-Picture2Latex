\subsection{Black Hole Mergers}
\begin{figure*}
\includegraphics[width=0.49\textwidth]{RESULTS/plots/merger_case1.pdf}
\includegraphics[width=0.49\textwidth]{RESULTS/plots/merger_case2.pdf}
\includegraphics[width=0.49\textwidth]{RESULTS/plots/stars2.png}
\includegraphics[width=0.49\textwidth]{RESULTS/plots/stars1.png}


\caption{The comparison between the distance of two merging black holes in the no-correction, DF(fid), DF(T15) and gas drag models in the early stage (\textbf{left}) and later stage (\textbf{right}) of the black hole evolution. For early mergers, the effect of the frictional forces (DF and drag) is not very prominent but still noticeable. The DF and gas drag both allow the black holes to merge faster compare to the no-DF case. For the later merger happening in a denser environment, the effect of dynamical friction is clear. However, the gas drag does not have a big effect on the black hole at this late stage compared with the no-DF case. The lower panels show the merging black holes within their host galaxies as well as their trajectories towards the merger in the \texttt{DF\_4DM\_G} run. The left images show the early phase of the orbital decay, and the right images show the later phase when the orbits get smaller.}
\label{fig:merger_case1}
\end{figure*}
Having seen the effect of different dynamical models on the evolution of individual black holes, next we will discuss how the dynamics, together with different BH merging criteria,  affect the evolution and mergers of the black holes.
 In particular, we want to study their merging time and trajectories before and after the mergers. Similar to the previous subsection, we will draw our examples from the two halos shown in Figure {\ref{fig:halos}}.
 
 %%%%%%%%%%%%%%%%%%%%%%%%%%%%%%%%%
\subsubsection{Effect of Dynamical  Friction Modeling}
\label{subsec:case_merger_calc}

We first look at how different dynamical models affect the time scale of black hole orbital decay and mergers. We pick two cases of mergers:  one is an early merger at $z>5$ when the black holes have not outgrown their dynamical masses; the other is a later merger at $z \sim 3.3$ when both BHs are larger than their seed dynamical masses (the major merger between BH1 and BH2 in Figure \ref{fig:halos}). Following \cite{Tremmel2015}, we also compute the dynamical friction time for the two mergers using Equation (12) - Equation (15) from \cite{Taffoni2003}:
\begin{equation}
\label{eq:tdf}
    t_{\rm DF} = 0.6\times 1.67\text{Gyr} \times \frac{r_c^2 V_h}{G M_s} \text{log}^{-1} \left( 1+\frac{M_{\rm vir}}{M_s} \right) \left(\frac{J}{J_c}\right)^\alpha,
\end{equation}
where $M_s$ is the mass of the smaller black hole (which we treat as the satellite), $M_{\rm vir}$ is the virial mass of the host halo of the larger black hole (found by AHF), $V_{h}$ is the circular velocity at the virial radius of the host, and $r_c$ is the radius of a circular orbit with the same energy as the satellite black hole's initial orbit. The last term $\left(\frac{J}{J_c}\right)^\alpha$ is the correction for orbital eccentricity, where $J$ is the angular momentum of the satellite, $J_c$ is the angular momentum of the circular orbit with the same energy as the satellite, and $\alpha$ is given by:
\begin{equation}
    \alpha \left( \frac{r_c}{R_{\rm vir}}, \frac{M_s}{M_{\rm vir}} \right) = 0.475 \left[ 1-\text{tanh} \left( 10.3 \left(\frac{M_s}{M_{\rm vir}}\right)^{0.33} - 7.5 \left(\frac{r_c}{R_{\rm vir}}\right) \right)  \right].
\end{equation}
In our calculation the virial radius, velocity, and mass are obtained from the AHF outputs, and the circular radius, orbit energy, and angular momentum are calculated by fitting the halo density profile to the NFW profile.

Figure \ref{fig:merger_case1} shows distances between two merging black holes in the no-DF, DF(fid), DF(T15), and gas drag models in the early and later stages of their evolution. For the early merger, the effect of the frictional forces (DF and drag) is not very big but still noticeable. The DF and gas drag have similar effects on the orbital decay at higher redshifts, consistent with our discussion in Section \ref{subsec:drag}. The DF(T15) model sinks the black hole a little slower than the DF(fid) model, but the difference is within $50$ Myrs. All three friction models allow the black holes to merge faster compare to the no-DF case by $\sim 150$ Myrs.

For the later merger, which takes place in a denser environment, the effect of dynamical friction is clearer: the dynamical friction allows the black holes to sink within the gravitational softening of the particles in $<200$ Myrs. Without dynamical friction the black hole's orbit does not have a clear decay below $2$ kpc and does not merge at the end of our simulation. Furthermore, the gas drag does not have a big effect on the black hole at this late stage compared with the no-correction case. This follows from our discussion in section \ref{subsec:drag} that gas drag is much less effective at lower redshift compared to dynamical friction.
 
In both plots, the yellow shaded region is the dynamical friction time from the analytical calculation in Equation \ref{eq:tdf}. Here we draw a band instead of a single line, because the black hole's orbit is not a strict ellipse, and the black hole is continuously losing energy. We calculate $t_{\rm DF}$ at multiple points between the first and second peak in the black hole's orbit (e.g. between $z=5.9$ and $z=5.7$ in the earlier case), and plot the range of those $t_{\rm DF}$. For both mergers, the analytical prediction is less than 150 Myrs later than the merger of the (fid) model. We note that the \cite{Taffoni2003} analytical $t_{\rm DF}$ is a fit to the NFW profiles, and the previous numerical and analytical comparisons on the black hole dynamical friction\citep[e.g.][]{Tremmel2015,Pfister2019} are performed in idealized NFW halos with a fixed initial black hole orbit. In our case, the halo profiles and black hole orbits are not directly controlled, and therefore deviation from the analytical prediction is expected. We will study such deviations statistically later in Section \ref{sec:merger_stats}.




%%%%%%%%%%%%%%%%%%%%%%%%%%%%%%%%%%%%%%%%%%%%%%%%%%%%%%%%%%
 \subsubsection{Effect of Gravitational Bound Check}

 In Section \ref{subsec:merger} we introduced two criteria which we use to perform black hole mergers in our simulations: we can merge two BHs when they are close in distance, and we can also require that the two BHs are gravitationally bounded in addition to the distance check. 
 
 In Figure \ref{fig:merger_case1} we show the difference in black holes' merging time with and without the gravitational bound criterion. The vertical dashed line marks the time that the two black holes in the \texttt{DF\_4DM\_G} simulation would merge if there was not the gravitational bound check. Without the gravitational bound check, the orbit of the black holes is still larger than 1 kpc when they merge, whereas with the gravitational bound check, the orbit size generally decays to less than 300 pc when the black holes merge. The merger without gravitational bound check generally makes the merger happen earlier by a few hundred Myrs (we will study the orbital decay time statistically in the next section). Therefore, for more accurate merger rate predictions as well as the correct accretion and feedback, it is necessary to apply the gravitational bound check during black hole mergers whenever the black hole has a well-defined velocity.
 


