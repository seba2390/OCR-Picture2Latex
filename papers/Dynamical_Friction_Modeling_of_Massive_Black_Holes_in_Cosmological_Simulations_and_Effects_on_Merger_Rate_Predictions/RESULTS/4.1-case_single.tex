\subsection{Black Hole Dynamics Modeling}
\label{subsec:models}

To compare different dynamical models, we look at the distance between the black hole and the halo center $\Delta r_{\rm BH}$ (we will sometimes refer to this distance as "drift" hereafter), the black hole mass, and the velocity along the $x$ direction through the entire history of BH2 from Figure \ref{fig:halos}. 
 
 We evaluate the black hole drift with two approaches: at each time-step, we find the minimum potential gas particle within 10 ckpc$/h$ of the black hole and calculate the distance between this gas particle and the black hole. This is a quick evaluation of the drift that allows us to trace the black hole motion at each time step, but it fails to account for orbits larger than 10 ckpc$/h$, and the minimum-potential gas particle may not reside in the same halo as the black hole. Therefore, for each snapshot we saved, we define the drift more carefully by running the halo finder and calculate the distance between the black hole and the center of its host halo. Whenever the black hole is further than 9 ckpc$/h$ from the minimum potential gas particle, we take the distance from the two nearest snapshots and linearly interpolate in time between them. Otherwise we use the 
 distance to the local minimum potential gas particle calculated at each time step.

%%%%%%%%%%%%%%%%%%%%%%%%%%%%%%%%%%%%%%%%%%
\subsubsection{DF and No Correction}

Before calibrating our dynamical friction modeling, we first demonstrate the effectiveness of our fiducial DF model, \texttt{DF\_4DM\_G}, by comparing it with the no-DF run \texttt{NoDF\_4DM\_G} (note that throughout the paper, no-DF means no correction to the BH dynamics of any form besides the resolved gravity). We keep all parameters fixed except for the black hole dynamics modeling. The details of these simulations can be found in Table \ref{tab:cons}.

In Figure \ref{fig:big_plot}(a), we show the evolution of BH2 in Figure \ref{fig:halos} under the no-DF and the fiducial DF models. Without any correction to the black hole dynamics, even the largest black hole in the simulation does not exhibit efficient orbital decay throughout its evolution: the distance from the halo center is always fluctuating above $2\epsilon_g$. This is because the black hole does not experience enough gravity on scales below the softening length, and cannot lose its angular momentum efficiently. Now when we add the additional dynamical friction to compensate for the missing small-scale gravity, the black hole is able to sink to within 1 ckpc$/h$ of the halo centers in <200 Myr and remain there. 

The 90 ckpc$/h$ peak in the drift of the black hole marks the merger between BH1 and BH2 in Figure \ref{fig:halos}, when the host halo of BH2 merges into the host of BH1, and the halo center is redefined near the merger. After the halo merger, dynamical friction is able to sink the black hole to the new halo center and allows it to merge with the black hole in the other halo, whereas in the no-DF case we do not see the clear orbital decay of the black holes after the merger of their host halo until the end of the simulation.

Besides the drift, we also show the x-component of the black hole's velocity relative to its surrounding collisionless particles (lower panel). Here we show one component instead of the magnitude to better visualize the velocity oscillation. With dynamical friction turned on, the velocity of the black hole is more stable, as the black hole's orbit has already become small and is effectively moving together with the host halo. Without dynamical friction, the black hole tends to oscillate with large velocities around the halo center without losing its angular momentum.

The different dynamics of the black hole can also affect accretion due to differences in density and velocities, so we also look at the black holes' mass growth in the two scenarios (middle panel). The mass growths of the two black holes are similar under the two models, although when subjected to dynamical friction, the black holes have more and earlier mergers. Even though the black hole mass is less sensitive to the dynamics modeling, the merger rate predictions can be affected significantly as we will discuss later. 

Note that for our no-DF model, we have also boosted the dynamical mass to $4\times M_{\rm DM}$ at the early stage to prevent scattering by the dark matter and star particles. However, even after the boost, the black holes cannot lose enough angular momentum to be able to stay at the halo center. This means that even though dynamical heating is alleviated through the large dynamical mass, the sub-resolution gravity is still essential in sinking the black hole to the host halo center.

%%%%%%%%%%%%%%%%%%%%%%%%%%%%%%%%%%%%%%%%%%

\subsubsection{Dynamical Friction and Gas Drag}
\label{subsec:drag}
\begin{figure*}
\includegraphics[width=0.49\textwidth]{RESULTS/plots/drag_df2drg3b10mg1_4dm8406903.pdf}
\includegraphics[width=0.49\textwidth]{RESULTS/plots/DF_drag.pdf}

\caption{Comparisons between DF and hydro drag. \textbf{Left:} comparison for a single black hole. In the top panel we show the magnitude of the DF (\textbf{red}) and gas drag (\textbf{blue}) relative to gravity for the same black hole, in the \texttt{DF+Drag\_4DM\_G} run. During the early stage of the black hole evolution, DF and gas drag have comparable effect, while after $z=7.5$ the gas drag becomes less and less important, as the gas density decreases relative to the stellar density (\textbf{middle}), and the black hole velocity goes into the subsonic regime (\textbf{lower}). \textbf{Right:} Ratio between DF and gas drag for all black holes. We plot the ratio both as a function of redshift (\textbf{top}) and as a function of time after a black hole is seeded (\textbf{bottom}). The orange lines represent the logarithmic mean of the scatter. The $F_{\rm DF}/F_{\rm drag}$ ratio depends strongly on the evolution time of the black hole: the longer the black hole evolves, the less important the drag force is. However, there is not a strong correlation between redshift and the $F_{\rm DF}/F_{\rm drag}$ ratio.}
\label{fig:drag}
\end{figure*}


\begin{figure*}
\includegraphics[width=0.33\textwidth]{RESULTS/plots/ratio_Mbh.pdf}
\includegraphics[width=0.66\textwidth]{RESULTS/plots/hexbin.pdf}
\caption{\textbf{Left:} Scattering relation between the $F_{\rm DF}/F_{\rm drag}$ ratio and the black hole mass. For each black hole, we sample its mass at uniformly-distributed time bins throughout its evolution, and we show the scattered density of all samples. DF has significantly larger effects over gas drag on larger BHs. We fit the scatter to a power-law shown in the orange line. \textbf{Right:} Scattering relation between the $F_{\rm DF}/F_{\rm drag}$ ratio and the BHs' distance to the halo center. Comparing with the BH mass, we do not see a clear dependence of the $F_{\rm DF}/F_{\rm drag}$ ratio on the distance to halo center. For BHs at all locations within the halo, DF is in general larger than the gas drag.}
\label{fig:drag_scatter}
\end{figure*}

In the previous subsection, we've only included collisionless particles (DM+Star) when modeling the dynamical friction, now we will look into the effects of dynamical friction of gas (gas drag) in comparison with the collisionless particles in the context of our simulations.

From Equation \ref{eq:H14} and \ref{eq:drag}, the relative magnitudes of DF and drag mainly depend on two components: the relative density of DM+stars versus gas, and the values of $\mathcal{F}(x)$ and $\mathcal{I(M)}$. \cite{Ostriker1999} has shown that when a black hole's velocity relative to the medium falls in the transonic regime (i.e. near the local sound speed), $\mathcal{I}$ is a few times higher than $\mathcal{F}$, while in the subsonic and highly supersonic regimes $\mathcal{I}$ is smaller or equal to $\mathcal{F}$. Therefore, we would expect the gas drag to be larger when the black hole is in the early sinking stage with a relatively high velocity and a high gas fraction. 

In Figure \ref{fig:drag}, the left panel shows the comparison between the magnitude of DF and gas drag through different stages of the black hole evolution, as well as the factors that can alter the effectiveness of the gas drag. In the very early stages ($z>7.5$) of black hole evolution, DF and gas drag have comparable effects, while after $z=7.5$ the gas drag becomes significantly less important and almost negligible compared with DF. The reason follows what we have discussed earlier: the gas density decreases relative to the stellar density (shown in the middle panel), and the black hole's velocity relative to the surrounding medium goes into the subsonic regime as a result of the orbital decay (shown in the lower panel). Around $z=3.5$, there is a boost in the black hole's velocity due to disruption during a major merger with a larger galaxy and black hole. The effect of gas is again raised for a short period of time (although still subdominant compared to the DF).

In Figure \ref{fig:big_plot}(b) we plot the black hole evolution for the DF-only (\texttt{DF\_4DM\_G}), drag-only (\texttt{Drag\_4DM\_G}), and DF+drag (\texttt{DF+Drag\_4DM\_G}) simulations.
Both the drag-only and DF-only models are effective in sinking the black hole at early times ($z>7$). However, at lower redshifts, the gas drag is not able to sink the black hole by itself, whereas DF is far more effective in stabilizing the black hole at the halo center. For this reason, in low-resolution cosmological simulations, dynamical friction from collisionless particles is necessary to prevent the drift of the black holes out of the halo center.

To further illustrate the relative importance between DF and gas drag for the entire BH population, we examine the dependencies of the $F_{\rm DF}/F_{\rm drag}$ on variables related to the BH evolution for all BHs in the \texttt{DF+Drag\_4DM\_G} simulation. First, in the right panel of Figure \ref{fig:drag} we show the time evolution of $F_{\rm DF}/F_{\rm drag}$. The top panel shows the ratio as a function of cosmic time, while the bottom panel shows the ratio as a function of each BH's seeding time. The DF/Drag ratio has a wide range for different BHs, but overall DF is becoming larger relative to the gas drag as the black hole evolves. From the mean value of the DF/drag ratio, we see that when the black holes are first seeded, DF is only a few times larger than the gas drag. After a few Gyrs of evolution, DF becomes 2-3 orders of magnitude larger than the gas drag. However, there is not a strong correlation between redshift and the $F_{\rm DF}/F_{\rm drag}$ ratio. 

In the left panel of Figure \ref{fig:drag_scatter}, we show the scattering relation between the $F_{\rm DF}/F_{\rm drag}$ ratio and the black hole mass $M_{\rm BH}$. We see a strong correlation between the $F_{\rm DF}/F_{\rm drag}$ ratio and the black hole mass: DF has significantly larger effects over gas drag on larger BHs, although the range of the ratio is large ar the low mass end. We fit a power-law to the median of the scatter:
\begin{equation}
    \frac{F_{\rm DF}}{F_{\rm drag}} = 250 \left(\frac{M_{\rm BH}}{10^7 M_\odot}\right)^{1.7},
\end{equation}
which roughly characterize the effect of the two forces on BHs of different masses. From this relation we see that for BHs with masses $>10^7 M_\odot$, gas drag is in general less than $1\%$ of DF. Finally, the right panels show the relation between the $F_{\rm DF}/F_{\rm drag}$ ratio and the BH's distance to the halo center: there is not a strong dependency on the BH's position within the halo.


%%%%%%%%%%%%%%%%%%%%%%%%%%%%%%%%%%%%%%%%%%


\subsubsection{Comparisons with the T15 Model}
\label{subsec:df100}
\begin{figure}
\includegraphics[width=0.5\textwidth]{RESULTS/plots/compare_kernel.pdf}

\caption{ Comparison between different components in the two dynamical friction models, DF(fid) (\textbf{red}) and DF(T15) (\textbf{blue}) (see Section \ref{sec:bh_model} for descriptions). We show the number of stars and dark matter particles included in the DF density and velocity calculation (\textbf{top panel}), the density used for DF calculation (\textbf{second panel}), the Coulomb logarithm used in the two methods (\textbf{third panel}), the velocity of the BH relative to the surrounding particles (\textbf{forth panel}, note that the "surrouding particles" are defined differently for the two models), and the magnitude of DF relative to gravity (\textbf{bottom panel}). The higher DF in the DF(fid) model at $z>8$ is due to the larger Coulomb logarithm. After $z\sim 7$, the higher density of DF(T15) due to more localized density calculation counterbalances its lower $\text{log}(\Lambda)$, resulting in similar DF between $z=8$ and $z=3.5$. During the halo merger at $z=3.5$, the DF(fid) model included particles from the target halo into the density calculation, and therefore yields larger DF during the merger.}
\label{fig:k100_case1}
\end{figure}


For the collisionless particles, we test and study two different implementations for the dynamical friction: DF(fid) and DF(T15) (see Section \ref{sec:bh_model} for detailed descriptions). In Section \ref{sec:bh_model} we pointed out three main differences between them: different kernel sizes (SPH kernel vs. nearest 100 DM+star), different definitions of $b_{\rm max}$ (10 ckpc vs. 1.5 ckpc$/h$), and different approximation of the surrounding velocity distribution (Maxwellian vs. nearest 100-sample distribution). Essentially, these differences mean that DF(fid) is a less-localized implementation than DF(T15). Now we would like to evaluate the effectiveness of these two implementations and show how different factors affect the final dynamical friction calculation.

Figure \ref{fig:k100_case1} shows the relevant quantities in the DF computation for the two methods. The two kernels both contain $\sim 100$ dark matter and star particles at high redshift ($z>8$), but after that the SPH kernel (defined to include the nearest 113 gas particles) begins to include more and more stars and dark matter. The mass fraction of stars in the SPH kernel dominates over that of dark matter by $\sim 10$ times for a BH at the center of the galaxy. The larger kernel of DF(fid) has two effects: first, the DF density will be smoother over time; second, during halo mergers, the DF(fid) kernel can "see" the high-density region of the larger halo, which results in a higher DF near mergers compared to DF(T15). This is confirmed by the second panel, where we show the density for dynamical friction calculation from the two kernels. The densities calculated from the two kernels are similar in magnitude throughout the evolution, although the DF(T15) kernel yields slightly larger density due to its smaller size. Around the BH merger, the density in DF(fid) is larger due to its inclusion of the host halo's central region.

The third panel shows the Coulomb logarithm in the two models. Recall that $\Lambda = \frac{b_{\rm max}}{(GM_{\rm BH})/v_{\rm BH}^2}$, and so the Coulomb logarithm depends on the black hole's mass, its velocity relative to the surrounding particles, and the value of $b_{\rm max}$. From Figure \ref{fig:big_plot}(c), the mass of the DF(T15) black hole is slightly smaller, but the mass difference is small compared with the 6 times difference in $b_{\rm max}$. Given $b_{\rm max}$=10 ckpc$/h$ in DF(fid) and $b_{\rm max}$=1.5 ckpc$/h$ in DF(T15), we would expect the Coulomb logarithm to be larger for the former. However, there is yet another tweak: the $v_{\rm BH}^2$ term turns out to be significantly larger in the DF(T15) model(fourth panel). Note that in the DF(T15) model $v_{\rm BH}^2$ is calculated using only 100 surrounding particles, and for the high-density region we are considering here, the velocity of the nearest 100 particles is very noisy in time.  As we will show in Appendix \ref{app:df100}, for smaller black holes the difference in $v_{\rm BH}^2$ is not as large, and usually DF(fid) has a larger $\text{log}\Lambda$ due to its larger $b_{\rm max}$.

In Figure \ref{fig:big_plot}(c), we show the evolution of the black hole under these two models. At high redshift ($z>8$), due to the large $\text{log}(\Lambda)$, the black hole in the DF(fid) simulation sinks slightly faster to the halo center. Between $z=8$ and $z=3.5$, both models have similar dynamical friction (as discussed in the previous paragraph) and the motion and mass accretion are also similar. Then at $z=3.5$, within the host halo of the black hole major merger, dynamical friction in DF(fid) is again larger because the density kernel includes more particles from the high-density region in the target halo, and this leads to an earlier merger time.

Overall, the performance of the two models is similar. However, as we have seen in the velocity calculation of the black holes relative to the surrounding particles, DF(T15) could be too localized for simulations of our resolution ($\epsilon_g \sim 1$kpc/h) and is sometimes subject to numerical noise. Therefore, in our subsequent statistical runs we pick DF(fid) as our fiducial model, and will drop the 'fid' in its name hereafter.


%%%%%%%%%%%%%%%%%%%%%%%%%%%%%%%%%%%%%%%%%%%
\subsubsection{Gravitationally Bound Merging Criterion}
\label{subsec:bound_check}

The merging criterion can affect not only the merging time, but also the dynamics and evolution of the black holes. Naively, we might expect the distance-only merging to produce more massive black holes, because black holes are merged more easily. However, in many cases this is not true, and we will illustrate here through one example. 

Figure \ref{fig:big_plot}(d) shows the evolution of the same black hole with the same dynamical friction prescription, but different merging criteria. We note a drastic difference in the black hole's trajectories: while the BH in the gravitationally bound merger case is staying at the center of its host halo, the BH in the distance-only merger flies out of its host after a merger. This is because with the distance-only model, it is possible for one black hole to have a very large velocity at the time of the merger, since we do not limit the black hole's velocity. By momentum conservation, the black hole with a larger velocity can transfer the momentum to the other black hole (and the merger remnant) which might have already sunk to the halo center. The sunk black hole then drifts out of the halo center after a merger due to the large momentum injection. This is especially common in simulations where the black hole's dynamical mass is boosted, because the injected momentum is also boosted with mass and a smaller black hole in a satellite galaxy can easily kick a larger black hole out. If we add on the gravitational bound check, there will be more time for the black holes to lose their angular momentum, and so the injected momentum is far less, and in most cases does not kick each other out of the central region.






