\subsection{Sinking of the Black Holes}
\label{subsec:drift}

\begin{figure}
\includegraphics[width=0.49\textwidth]{RESULTS/plots/L15_drift.pdf}
\caption{The effect of different BH dynamics modeling on BH position relative to its host. We include the reposition model (blue), no-DF model (orange),DF(T15) model (green), DF(fid) model (red) and the DF+drag model (purple). \textbf{Top:} The fraction of halos(subhalos) without a black hole for halos with masses above the black hole seeding mass at $M_{\rm halo} = 10^{10} M_\odot/h$. \textbf{Middle:} The fraction of halos without a central black hole ("central" means within $2\epsilon_g$ from the halo center identified by the halo finder), out of all halos with black holes. \textbf{Bottom:} Distribution of black holes' distance to its host halo center.} 
\label{fig:drift}
\end{figure}


\begin{figure}
\includegraphics[width=0.49\textwidth]{RESULTS/plots/L15_BHMF.pdf}
\caption{Mass functions for reposition, DF and no-DF simulations. With reposition (\textbf{blue}), we have the highest mass function and earlier formation of $10^8 M_\odot$ black holes. The no-DF simulations (\textbf{green}) have lower mass functions, which is expected due to low-accretion and merger rates from the black hole drifting. The dynamical friction model (\textbf{red}) yields a mass function in between.} 
\label{fig:bhmf}
\end{figure}


The primary reason for adding dynamical friction onto black holes within the cosmological simulations is to stabilize the black hole at the halo center (defined as the position of the minimum-position gas particle within the halo). Hence, we start by looking at the black holes' position relative to the host halos. Due to the resolution limit of our simulations, we would not expect the black holes to be able to sink to the exact minimum potential. Instead we consider a $<2\epsilon_g = 3$ ckpc$/h$ distance to be "good sinking".

In Figure \ref{fig:drift}, we show the statistics related to black holes' sinking status. We included the comparison between the reposition model (\texttt{L15\_Repos\_4DM}), the no-DF model (\texttt{L15\_NoDF\_4DM}), the two dynamical friction models (\texttt{L15\_DF\_4DM} and \texttt{L15\_DF(T15)\_4DM}) and the DF+drag model(\texttt{L15\_DF+drag\_4DM}). To start with, we simply count the fraction of halos without a black hole when its mass is already above the black hole seeding criterion (i.e. $10^{10}M_{\odot}/h$). The top panel shows the fraction of large halos without a BH for different models at $z=3.5$ and $z=2$. Surprisingly, the no-DF model ends up with the least halos without a black hole. This is because even though the black holes without dynamical corrections cannot sink effectively, the high dynamical mass still prevents sudden momentum injections from surrounding particles, and therefore most BHs still stays within their host galaxies. The dynamical friction models perform equally well, with $<10\%$ no-BH halos at the low-mass end. The reposition model, however, ends up with the most no-BH halos, even though repositioning is meant to pin the black holes to the halo center. This happens because under the repositioning model, the central black holes tend to spuriously merge into a larger halo during fly-by encounters, leaving the smaller sub-halo BH-less.

Next we look at where the black holes are located within their host galaxies. For all the halos with at least one black hole, we examine whether the black hole is located at the center (i.e.$<2\epsilon_g = 3$ckpc/h from the halo center). The middle panel of Figure \ref{fig:drift} shows the fraction of halos without a central BH. The no-DF model has significantly more halos without a central BH compared to the other models, with over half of the halos hosting off-center BHs. Among the three runs with dynamical friction, the DF(T15) and DF(fid) models have a similar fraction of halos ($\sim 20\%$) without a central BH, and we can see this fraction dropping from $z=3.5$ to $z=2$, meaning that many BHs are still in the process of sinking towards the halo center. When we further add the gas drag, $10\%$ more halos host at least one central BH, and the difference between the drag and no-drag central BHs is more prominent at high redshifts. 

Interestingly, the repositioning algorithm is not as efficient at sinking the BHs at $z=2$ as the DF. This is because our repositioning algorithm places the BHs at the minimum potential position within the accretion kernel, instead of within the entire halo. The majority of the offset between the BH positions and the halo center comes from the offset between the minimum-potential position accessible to the BH (i.e. minimum-potential in the accretion kernel) and the minimum-potential position in the halo. Such offset can be especially severe at lower redshift, when the size of the accretion kernel gets smaller and mergers happen more frequently, making it easier for the black holes to get stuck at a local minimum.


In the bottom panels we show the distributions of the black holes' distance to the halo centers under different models. For the no-DF run, again we see that the black holes fail to move towards the halo center at lower redshift, resulting in a much flatter distribution compared to all the other models. In comparison, when we add dynamical friction to the black holes, for both the DF(fid) and the DF(T15) models the distributions are pushed much closer to the halo center, with a peak around the gravitational softening length. When we then add the gas drag in addition to DF, the peak at $\epsilon_g$ becomes slightly higher than those in the DF-only runs. The combination of DF and gas drag, as we would expect from the case studies, is the most effective in sinking the black holes to the halo centers and stabilizing them. Finally, we plot the repositioning model for reference.  It does well in putting the black hole close to the minimum potential, and often the black holes can be located at the exact minimum-potential position (the distributions peak at 0 for $z=3.5$). However, as discussed in the previous paragraph, there are cases where the local minimum potential found by the repositioning algorithm does not coincide with the global minimum potential of the halo, and that is why we also see non-zero probability density for $\Delta r > 3$ ckpc$/h$ at $z=2$.

The statistics we have seen for the models above are consistent with the results from the case studies. This shows that even though for the case studies we have focused mainly on large black holes in one of the biggest halo, a similar trend still applies to other black holes in the cosmological simulations, which are embedded in smaller halos or subhalos.


 

%%%%%%%%%%%%%%%%%%%%%%%%%%%%%%%%%%%%%%%%%%%%%%%%%%%%%%%%%%%%%%%%%%%%%%%%%%%%%%

\subsection{Black Hole Mass Function}
\label{subsec:bhmf}
Next we look at how different dynamics affect the black hole mass function (BHMF). One problem with the repositioning method is that it places the black holes at the galaxy center too quickly, which could result in excess accretion and thus a higher mass function. On the other hand, if we do not add any correction to the black hole motion, many BHs will not go though efficient accretion and mergers, and we will see a lower mass function. We would expect the BHMF in the dynamical friction run to fall between the repositioning case and the no-DF case.

Figure \ref{fig:bhmf} shows the BHMF from the reposition(\texttt{L15\_Repos\_4DM}), dynamical friction (without gravitational bound check:\texttt{L15\_DF\_4DM}; with gravitational bound check:\texttt{L15\_DF\_4DM\_G}), and no-DF (without gravitational bound check: \texttt{L15\_NoDF\_4DM}; with gravitational bound check: \texttt{L15\_NoDF\_4DM\_G}) runs. The reposition model yields the highest mass function, and is the only simulation with more than one $10^8 M_{\sun}/h$ black holes at $z=2$. This is expected from the over-efficient BH mergers and the high-density surroundings in the reposition model. Moreover, it creates increasingly more massive BHs over time, as the increased merger rate produces a stronger effect over time. The no-DF runs produces the lowest mass function due to the off-centering, while the DF mass function falls between the reposition and no-DF case as we expected. 

Naively, we would expect the models without gravitational bound checks to produce a higher mass function, because it allows for easier mass-accretion via mergers. However, as discussed in Section \ref{subsec:bound_check}, this is not the case if we compare the dashed lines and solid lines with the same colors. For example, under the DF model, the \texttt{L15\_DF\_4DM\_G} simulation forms more massive black holes than the \texttt{L15\_DF\_4DM} simulation, especially at lower redshift. The reason can be traced back to what we have seen in Figure \ref{fig:big_plot}(d): when there is no gravitational bound check, the large momentum injection during a merger kicks the black hole out of the halo center, thus preventing the efficient growth of large black holes.

Considering the relatively large uncertainties due to the limited volume, the difference in the mass function is not very significant. We would expect other factors such as the black hole seeding, accretion and feedback to have a larger effect on the mass function compared to the dynamical models we show here \citep[e.g.][]{Booth2009}.