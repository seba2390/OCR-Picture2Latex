\subsection{Dynamical Friction Time and Mergers}
\label{sec:merger_stats}

\begin{figure}
\includegraphics[width=0.48\textwidth]{RESULTS/plots/tdf.pdf}
\caption{The delay of mergers due to the dynamical friction time. Here we compare the numerical dynamical friction time,$t_{\rm num}$, to the analytically calculated time (following Equation \ref{eq:tdf}) $t_{\rm analy}$. \textbf{Top left:} distribution of the dynamical friction time from numerical merger (blue) and analytical predictions (red). \textbf{Top right:} ratio between the numerical and analytical $t_{\rm df}$. Their difference is less than one order of magnitude in all merger cases. \textbf{Bottom:} dynamical friction time as a function of the virial mass of the host halo for the numerical (blue) merger and analytical predictions (red). The same merger event is linked by a grey line.} 
\label{fig:delay}
\end{figure}

\begin{figure}
\includegraphics[width=0.49\textwidth]{RESULTS/plots/L15_mergers.pdf}
\caption{The cumulative mergers for different BH dynamics and merging models. The reposition model (\textbf{blue solid}) predicts more than two times the total mergers compared with the other models. Without the gravitational bound check, the DF (\textbf{red dashed}) and the no-DF model (\textbf{green dashed}) predicts similar numbers of mergers, indicating that the first encounters of the black hole pairs are similar under the two models. However, if we add the gravitational bound check, the dynamical friction model (\textbf{red solid}) yields $\sim 50\%$ more mergers compared to the no-correction model. Adding the gas drag in addition to dynamical friction (\textbf{purple solid}) raises the mergers by a few. } 
\label{fig:merger_stats}
\end{figure}



Because the reposition method is used in most large-volume cosmological simulations, a post-processing analytical dynamical friction time is calculated in order to make more accurate merger rate predictions. Now that we have accounted for the dynamical friction on-the-fly, we want to study how our numerical mergers with dynamical friction compare against the analytical predictions, and how different dynamical models impact the black hole merger rate.

In Section \ref{subsec:case_merger_calc}, we compared the numerical merging time to the analytical predictions for two merger cases. Now we use the same method to calculate an analytical dynamical friction time for all black hole mergers in our \texttt{L15\_DF\_4DM\_G} simulation. For each pair, we begin the calculation at the time $t_{\rm beg}$ when the black hole pair first comes within 3 ckpc$/h$ of each other, as this mimics the merging time without the gravitational bound check, and is also close to the merging criterion under the reposition model. The numerical dynamical friction time $t_{\rm num}$ is the time between the numerical merger and $t_{\rm beg}$. The analytical dynamical friction time $t_{\rm analy}$ is calculated using the host halo information in the snapshot just before $t_{\rm beg}$ and the black hole information at the exact time-step of $t_{\rm beg}$.

Figure \ref{fig:delay} shows the comparison between the numerical and analytical dynamical friction times. In the top panel we show the distribution of the two times as well as the distribution of their ratio. We note that for all the mergers happening numerically, $t_{\rm analy}$ does not exceed 2 Gyrs, and most have $t_{\rm analy}$ less than 1 Gyr. This means that we do not have many fake mergers that shouldn't merge until much later (or never). Also, the ratio plot shows that the numerical and analytical times are always within an order of magnitude of each other, with most of the numerical mergers earlier than the analytical mergers. The numerical merger time is peaked between 100 Myrs and 1 Gyrs, whereas the analytical calculation yields a flatter distribution. We would expect $t_{\rm analy}$ to be longer than $t_{\rm num}$, both because we have a selection bias on $t_{\rm DF}$ by ending the similation at $z=2$, and because we numerically merge the black holes when their orbit is still larger than 3 ckpc$/h$. However, this does not explain why $t_{\rm analy}$ has a higher probability between 10 Myrs and 100 Myrs. 

To see the individual merger cases in the distribution more clearly, in the lower panel of Figure \ref{fig:delay} we plot all the numerical and analytical dynamical friction times as a function of the host halo's virial mass. From this figure we do not see a clear dependence of either dynamical friction times on the host halo's virial mass. There is also no strong correlation between the $t_{\rm num}/t_{\rm analy}$ ratio and the halo mass. We do not further investigate the discrepancies between the numerical and analytical results, as these results can vary significantly from system to system. 

We note that although the numerical model has free parameters (such as $b_{\rm max}$, $M_{\rm dyn, seed}$) that can impact the merging time (but see Appendix \ref{app:merger_param}), it can account for the immediate environment around black hole and adjust the dynamical friction on-the-fly. More importantly, it also accounts for the interaction between the satellite BH and its own host galaxy, which could reduce the sinking time significantly \citep[e.g.][]{Dosopoulou2017}. 
The analytical model, though verified by N-body simulations, does not react to the environment of the merging galaxies by always assuming an NFW profile. Moreover, it only models the sinking of a single BH without embedding it in its host galaxy. Therefore, we expect the numerical result to be a more realistic modeling of the binary sinking process.

After comparing the DF model against the analytical prediction, next we compare different numerical models in terms of the black hole merger rate. Figure \ref{fig:merger_stats} shows the cumulative mergers from $z=8$ to $z=2$. We have included comparisons between the reposition, dynamical friction and no-DF models, both with and without the gravitational bound check. The reposition model predicts more than twice the total number of mergers compared to the other models. Without the gravitational bound check, the DF and the no-DF models predict similar numbers of mergers, indicating that the first encounters of the black hole pairs are similar under the two models. However, if we add the gravitational bound check, the DF model yields $\sim 50\%$ more mergers compared to the no-DF model, because the addition of DF assists energy loss of the binaries and leads to earlier bound pairs. Finally, the merger rate is not very sensitive to adding the gas drag: the merger rate in the DF-only model is similar to that of the DF+drag model. This can be foreseen in the comparison shown in Figure \ref{fig:drag}, where the gas drag is subdominant in magnitude.
