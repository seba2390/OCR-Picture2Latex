\section{Merger Rates in the 35Mpc/h Simulations}
\label{sec:L35}

\begin{figure*}
\includegraphics[width=0.99\textwidth]{RESULTS/plots/hist.pdf}
\caption{ \textbf{Left:} Distribution of the mass of the smaller black hole ($M_s$), and distribution of the total mass of the binary ($M_{\mathrm{tot}}$). For both simulations, the mergers in which at least one of the black holes is slightly above the seed mass dominate. The most massive binary has a total mass of $3\times 10^8 M_\odot$. \textbf{Middle:} The mass ratio $q$ between the two black holes in the binary. We see a peak at $\text{log(q)}=-0.5$, corresponding to pairs in which one BH is about three times larger than the other. \textbf{Right:} Scatter of the two black hole masses in the binaries, binned by redshift. To separate the scatter in the two simulations, for the DF+drag run we take $M_1$ to be the mass of the larger BH, while for the NoDF run $M_2$ is the larger BH.}
\label{fig:hist}
\end{figure*}

\begin{figure}
\includegraphics[width=0.49\textwidth]{RESULTS/plots/rates.pdf}
\caption{Merger rate per year of observation per unit redshift predicted from our \texttt{L35\_DF+drag\_4DM\_G} (\textbf{purple}) and \texttt{L35\_NoDF\_4DM\_G} (\textbf{blue}) simulations. 
For comparison, we also show the the prediction from recent hydro-dynamical simulations. 
We include three simulations of similar mass-resolution: \citet{Volonteri2020} from the Horizon-AGN simulation (\textbf{gray}), \citet{Katz2020} (\textbf{yellow}) from the Illustris simulation and \citet{Salcido2016} from the EAGLE simulations (\textbf{pink}).
Since we do not apply any post-processing delays after the numerical mergers, we only compare to results without delays.}
\label{fig:rates}
\end{figure}

Based on all the previous test of BH dynamics modeling, we have reached the conclusion that the DF+drag model with $M_{\rm dyn} = 4 M_{\rm DM}$ is most capable of sinking the black hole to the halo center. Hence, we choose to use this model to run our larger-volume simulation \texttt{L35\_DF+drag\_4DM\_G} for the prediction of the BH coalescence rate. Besides this model, we also perform a same-size run without the dynamical friction, \texttt{L35\_NoDF\_4DM\_G}, as a lower limit for the predicted rate.  
Our \texttt{L35} simulations are run down to $z=1.1$. The black hole seed mass is $5\times10^5 M_\odot/h$ and the minimum halo mass for seeding is $10^{10} M_\odot/h$. The details of these two simulations are shown in Table \ref{tab:norm}.

\subsection{The Binary Population}
\label{subsec:L35_catalog}
Because this work mainly focuses on model verification and is not intended for accurate merger-rate predictions, we do not account for the various post-numerical-merger time delays. These delays can be caused by physical processes such as sub-ckpc scale dynamical friction, scattering with stars, gravitational wave driven inspiral and triple MBH systems \citep[e.g.][]{Quinlan1996,Sesana2007b,Vasiliev2015,Dosopoulou2017,Bonetti2018}. We consider all the numerical mergers as true black hole merger events. Without any post-process selection, there are 25224 black holes and 4237 mergers in the \texttt{L35\_DF+drag\_4DM\_G} run, and 27693 black holes and 2349 mergers in the \texttt{L35\_NoDF\_4DM\_G} run down to $z=1.1$.

Figure \ref{fig:hist} shows the distribution of the binary parameters for the mergers in our simulations. For both simulations, there is at least one black hole around the seed mass for most mergers, but the peak does not lie at the exact seed mass. The most massive binary has a total mass of $3\times 10^8 M_\odot$. For the mass ratio $q$ between the two black holes in the binary, we see a peak at $\text{log}(q)=-0.3$, corresponding to pairs in which one BH is about two times larger than the other. Finally, we show the scatter of the two progenitor masses. The low mass end of the population deviates more from $q=1$, while the majority of same-mass mergers come from the $5\times 10^6 M_\odot\sim 5\times 10^7 M_\odot$ mass range.

Comparing with previous simulations such as \cite{Salcido2016,Katz2020}, we do not see as many cases of seed-seed mergers, but our distribution in q is similar to that shown in \cite{Weinberger2017} where the larger progenitor is a few times larger than the small progenitor. This is due to our larger black hole seed mass of $5\times10^5 M_\odot$ ($10^6 M_\odot$ in \cite{Weinberger2017}): the mass accretion in the early stage is proportional to $M_{\rm BH}^2$, and so during the time before the black hole mergers, our black holes accrete more mass compared to the simulations with smaller seeds. This explains why both of our black holes in the binaries are not peaked at the exact seed mass.


\subsection{Merger Rate Predictions}
\label{subsec:L35_rates}
We use the binary population shown in the previous section to predict the merger rate observed per year per unit redshift.
The merger rate per unit redshift per year is calculated as:
\begin{equation}
     \frac{dN}{dz\;dt} =  \frac{N(z)}{\Delta z V_{c,sim}} \frac{dz}{dt} \frac{dV_c(z)}{dz}\frac{1}{1+z},
\end{equation}
where $N(z)$ is the total number of mergers in the redshift bin $z$, $\Delta z$ is the width of the redshift bin, $V_{c,sim}$ is the comoving volume of our simulation box and $dV_c(z)$ is the comoving volume of the spherical shell corresponding to the $z$ bin. 

We compare our results against recent predictions from hydro-dynamical simulations of similar resolution, \cite{Salcido2016}, \cite{Katz2020} and \cite{Volonteri2020}. Here we briefly summarize relevant information about their merger catalogs. The Ref-L100N1504 simulation in the EAGLE suite used in \cite{Salcido2016} has an $2^3$ times larger simulation box and slightly higher resolution than our simulations. They seed $1.4\times 10^5M_\odot$ black holes in $1.4\times 10^{10}M_\odot$ halos. They adopt the reposition algorithm for black hole dynamics, but set a distance and relative speed upper limit on the repositioning to prevent black holes from jumping to satellites during fly-by encounters. We compare with their no-delay rate during the inspiral phase. The \textit{Illustris} simulation used in \cite{Katz2020} has a similar box size, resolution and BH dynamics to the Ref-L100N1504 simulation in EAGLE, except that their halo mass threshold for seeding BHs is $7\times 10^{10} M_\odot$. We compare against their ND model, in which mergers are also taken to occur at the numerical merger time without any delay processes. The Horizon-AGN simulation in \cite{Volonteri2020} is $4^3$ times larger than our simulation box, with $\sim 5$ times coarser mass resolution and a black hole seed mass of $10^5 M_\odot$. Instead of seeding BHs in halos above certain mass threshold, the seeding in \cite{Volonteri2020} is based on the local gas density and velocity dispersion, and seeding is stopped at $z=1.5$. For black hole dynamics, they apply dynamical friction from gas, but not from collisionless particles.

Figure \ref{fig:rates} shows our merger rate prediction in the \texttt{L35\_DF+drag\_4DM\_G} and  \texttt{L35\_NoDF\_4DM\_G} simulations. The \texttt{L35\_DF+drag\_4DM\_G} run predicts $\sim 2$ mergers per year of observation down to $z=1.1$, while the \texttt{L35\_NoDF\_4DM\_G} run predicts $\sim 1$. The merger rates from both simulations peak at $z\sim 2$. This factor-of-two difference between the two simulations is consistent with what we predicted in the $L_{\rm box} = 15$ Mpc$/h$ runs in Figure \ref{fig:merger_stats}. Although we did not run a $L_{\rm box} = 35$ Mpc$/h$ simulation with the repositioning model, we expect such a run to predict $5\sim 6$ mergers per year down to $z=1.1$ according to \ref{fig:merger_stats}.

Generally speaking, our simulations yield similar merger rates as the raw predictions from the previous works of comparable resolution. However, we still note some differences both in the overall rates and in the peak of the rates. We will now elaborate on the reasons for those discrepancies.

First, both of our simulations predict more mergers compared with the \cite{Katz2020} ND model prediction. This is surprising given that in the 15 Mpc$/h$ runs we saw $2\sim 3$ times more mergers when we used the reposition method like \cite{Katz2020} and \cite{Salcido2016} did, comparing to our DF+Drag model. Although \cite{Katz2020} cut out $\sim 25\%$ secondary seed mergers and binaries with extreme density profiles, their rate is still lower after adding the cut-out population. One major reason for the higher rate from our simulation compared to \cite{Katz2020} is the different seeding parameters we use: our minimum halo mass for seeding a black hole is $10^{10} M_\odot/h$, which is 5 times smaller compared with \cite{Katz2020}. Moreover, our seeds are a factor of 5 larger. Hence, we have a denser population of black holes in less-massive galaxies, which is likely to result in a higher merger rate even compared to the reposition model used in \textit{Illustris}.

Second, although the rates from EAGLE, Horozon-AGN and our \texttt{L35\_DF+drag\_4DM\_G} simulation cross over at $z\sim 2$, the slope of our merger rate is very different. \cite{Volonteri2020} predicts most mergers at $z\sim 3$, whereas the \cite{Salcido2016} rate peaks at $z\sim 1$. This difference can also be traced to the different seeding rate in the three simulations: in \cite{Salcido2016}, the seeding rate keeps increasing until $z\sim 0.1$, while we observe a drop in seeding rate at $z=3$ in our simulations. In \cite{Volonteri2020}, due to the different seeding mechanism, BH seeds form significantly earlier, leading to a peak in merger rate at a higher redshift. Hence the peak in the BH merger rate is strongly correlated with the peak in the BH seeding rate.

Finally, besides the effect due to different BH seed models on the merger rate, higher resolution can significantly increase the BH merger rates in the simulations. As was shown in previous work \citep[e.g.][]{Volonteri2020,Barausse2020}, dwarf galaxies in low-mass halos can have large numbers of (small mass) BH mergers, and so resolving such halos and galaxies can increase the BH merger rate significantly. The merger rate differences between high and low resolution and the associate choice for the seed models can lead to large differences in the predictions of merger rates than taking account DF in the BH dynamics. 
