\subsection{BH Dynamical Mass}
\label{subsec:mdyn}
In our simulations, the seed mass of the black holes is $5\times 10^5 M_\odot/h$, which is 20 times smaller than the fiducial dark matter particle mass at $1.2\times 10^7 M_\odot/h$. Such a small mass of the BH relative to the dark matter particles will result in very noisy gravitational acceleration on the black holes, and causes instability in the black hole's motion as well as drift from the halo center. Moreover, as shown in previous works \citep[e.g.][]{Tremmel2015,Pfister2019}, under the low $M_{\rm BH}/M_{\rm DM}$ regime, it is challenging to effectively model dynamical friction in a sub-grid fashion.

To alleviate dynamical heating by the noisy potential due to the low $M_{\rm BH}/M_{\rm DM}$ ratio, we introduce a second mass tracer, the dynamical mass $M_{\rm dyn}$, which is set to be comparable to $M_{\rm DM}$ when the black hole is seeded. This mass is used in force calculation for the black holes, including the gravitational force and dynamical friction, while the intrinsic black hole mass $M_{\rm BH}$ is used in the accretion and feedback process. $M_{\rm dyn}$ is kept at its seeding value $M_{\rm dyn,seed}$ until $M_{\rm BH}>M_{\rm dyn,seed}$. After that $M_{\rm dyn}$ grows following the black hole's mass accretion. With the boost in the seed dynamical mass, the sinking time scale will be shortened by a factor of $\sim M_{\rm BH}/M_{\rm dyn}$ compared to the no-boost case. Note that the bare black hole sinking time scale estimated in the no-boost case could over-estimate the true sinking time, as the high-density stellar bulges sinking together with the black hole are not fully resolved \citep[e.g.][]{Antonini2012,Dosopoulou2017,Biernacki2017}.

 The boost we need to prevent dynamical heating depends on the dark matter particle mass $M_{\rm DM}$ (if we have high enough resolution the boost is no longer necessary), so we parametrize the dynamical mass in terms of the dark matter particle mass, $M_{\rm dyn,seed} = k_{\rm dyn} M_{\rm DM}$, instead of setting an absolute seeding dynamical mass for all simulations. We expect that as we go to higher resolutions where $M_{\rm DM}$ is comparable to $M_{\rm BH,seed}$, the dynamical seed mass should converge to the black hole seed mass, if we keep $k_{\rm dyn}$ constant. We study the effect of setting different $k_{\rm dyn}$ by running three simulations with the same resolution and dynamical friction models, but various $k_{\rm dyn}$ ratios. They are listed in Table \ref{tab:cons} as \texttt{DF\_4DM\_G}, \texttt{DF\_2DM\_G}, and \texttt{DF\_1DM\_G}, with $k_{\rm dyn}=4,2,1$, respectively.

To explore the effects of the BH seed dynamical mass on the motion and mergers of the black hole, we test a variety of $M_{\rm dyn,seed}$ values in our simulations. The comparison between different $M_{\rm dyn,seed}$ can be found in Appendix \ref{app:res}. 