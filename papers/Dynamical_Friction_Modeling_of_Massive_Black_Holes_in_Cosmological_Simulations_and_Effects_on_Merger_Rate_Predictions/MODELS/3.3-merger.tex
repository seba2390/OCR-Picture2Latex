\subsection{Merging Criterion}
\label{subsec:merger}
In all of our simulations, we set the merging distance to be $2\epsilon_{\rm g}$, because the BH dynamics below this distance is not well-resolved due to our limited spatial resolution. We conserve the total momentum of the binary during the merger.

Under the baseline repositioning treatment of the BH dynamics, the velocity of the black hole is not a well-defined quantity. Therefore, in cosmological simulations with repositioning, the distance between the two black holes is often the only criterion imposed during the time of mergers (for example BlueTides \citep{Feng2016}, Illustris \citep{Vogelsberger2013} and IllustrisTNG \citep{Pillepich2018}). One problem with using only the distance as a merging criterion is that it can spuriously merge two passing-by black holes with high velocities, when in reality they are not gravitationally bound and should not merge just yet (or may never merge). Although some similar-resolution simulations such as EAGLE \citep{Crain2015,Schaye2015} also check whether two black hole particles are gravitationally bound, the black holes still do not have a well-defined orbit and sinking time due to the discrete positioning.

When we turn off the repositioning of the BHs to the nearby minimum potential, the BHs will have well-defined velocities at each time step (this is true whether or not we add the dynamical friction). This allows us to apply further merging criteria based on the velocities and accelerations of the black hole pair, and thus avoid earlier mergers of the gravitationally unbound pairs. Also, as the BH pairs now have well-defined orbits all the way down to the numerical merger time, we will be able to directly measure binary separation and eccentricity from the numerical merger, and use the measurements as the initial condition for post-processing methods without having to assume a constant initial value \citep[e.g.][]{Kelley2017}.

We follow \cite{Bellovary2011} and \cite{Tremmel2017}, and use the criterion
\begin{equation}
    \label{eq:merge_criterion}
    \frac{1}{2}|\bf{\Delta v}|^2 < \bf{\Delta a} \bf{\Delta r}
\end{equation}
 to check whether two black holes are gravitationally bound. Here $\bf{\Delta a}$,$\bf{\Delta v}$ and $\bf{\Delta r}$ denote the relative acceleration, velocity and position of the black hole pair, respectively. Note that this expression is not strictly the total energy of the black hole pair, but an approximation of the kinetic energy and the work needed to get the black holes to merge. Because in the simulations the black hole is constantly interacting with surrounding particles, on the right-hand side we use the overall gravitational acceleration instead of the acceleration purely from the two-body interaction.
