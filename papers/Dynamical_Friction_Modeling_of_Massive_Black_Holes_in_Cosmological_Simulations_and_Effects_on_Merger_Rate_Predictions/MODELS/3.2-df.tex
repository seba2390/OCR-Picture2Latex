\subsection{Modeling of Black Hole Dynamics}
\label{subsec:df}
\subsubsection{Reposition of the Black Hole}

Before introducing our dynamical friction implementations, we first describe a baseline model utilized by many large-volume cosmological simulations: the reposition model. As the name suggests, the reposition model of black hole dynamics places the black hole at the location of a local gas particle with minimum gravitational potential at each time step, in order to avoid the unrealistic motion of the black holes due to limited mass and force resolution. This is particularly preferred for large-volume, low-resolution cosmological simulations \citep[e.g.][]{Springel2005b, Sijacki2007, Booth2009,Schaye2015,Pillepich2018}, where the black hole mass is smaller than a star or gas particle mass and the BH can be inappropriately scattered around by two-body forces as well as the noisy local potential.

This simple fix of repositioning, however, comes with many disadvantages. For example, it may lead to higher accretion and feedback of the black holes, as they sink to the high-density regions too quickly. As was shown in \cite{Wurster2013} and \cite{Tremmel2017}, repositioning also leads to burstier feedback of the BHs, which is more likely to quench star-formation in the host galaxies. Moreover, repositioning leads to ill-defined velocity and non-smooth trajectories of the black hole particles. Because of the ill-defined velocity and extremely short orbital decay time, such methods cannot be reliably used for merger rate predictions without careful post-processing calculations to account for the orbital decays.

In our study, we use the reposition model as a reference for the black hole statistics, as it is still widely adopted in many existing simulations. We want to compare the dynamical friction models with the reposition model and quantify the effect of repositioning on BH mass growth and merger rate compared with the dynamical friction models.


\subsubsection{Dynamical Friction from Collisionless Particles}

When the black hole travels through a continuous medium or a medium consisting of particles with smaller masses than the black hole, it attracts the surrounding mass towards itself, leaving a tail of overdensity behind.  Dynamical friction is the resulting gravitational force exerted onto the black hole by this tail of overdensity \citep[e.g.][]{Chandrasekhar1943,Binney2008}. Dynamical friction causes the orbits of SMBHs to decay towards the center of massive galaxies \citep[e.g.][]{Governato1994,Kazantzidis2005}, and enables the black holes to stay at the high-density regions where they could go through efficient accretion and mergers.

We follow Equation (8.3) in \cite{Binney2008} for the acceleration of the black hole due to dynamical friction:

\begin{equation}
\label{eq:df_full}
    \mathbf{F}_{\rm DF} = -16\pi^2 G^2 M_{\rm BH}^2 m_{a} \;\text{log}(\Lambda) \frac{\mathbf{v}_{\rm BH}}{v_{\rm BH}^3} \int_0^{v_{\rm BH}} dv_a v_a^2 f(v_a),
\end{equation}
where $M_{\rm BH}$ is the black hole mass, $\textbf{v}_{\rm BH}$ is the velocity of the black hole relative to its surrounding medium, $m_a$ and $v_a$ are the masses and velocities of the particles surrounding the black hole, and $\text{log}(\Lambda)=\text{log}(b_{\rm max}/b_{\rm min})$ is the Coulomb logarithm that accounts for the effective range of the friction between $b_{\rm min}$ and $b_{\rm max}$(we will specify how we set these parameters later). $f(v_a)$ is the velocity distribution of the surrounding particles (unless we explicitly state otherwise, all variables involving the black hole's surrounding particles are calculated using stars and dark matter particles). Here we have assumed an isotropic velocity distribution of the particles surrounding the black hole, so that we are left with an 1D integration. 

We test two different numerical implementations of the dynamical friction (DF) in our simulations: one with a more aggressive approach which likely overestimates the effective range of DF, but could be more suitable for large-volume simulations (we refer to it as DF(fid) in places where we carry out explicit comparisons between the two DF models, and drop the 'fid' in all other places); the other with a more conservative method which aims to only account for the DF below the gravitational softening length, and is well-tested for smaller volume, high-resolution simulations \citep{Tremmel2015} (we refer to it as DF(T15)).

We begin by introducing the DF(fid) model. In this model, we further follow the derivation in \cite{Binney2008}, and approximate $f(v_a)$ by the Maxwellian distribution, so that Equation \ref{eq:df_full} reduces to:
\begin{equation}
    \label{eq:H14}
    \mathbf{F}_{\rm DF,fid} = -4\pi \rho_{\rm sph} \left(\frac{GM_{\rm dyn}}{v_{\rm BH}}\right)^2  \;\text{log}(\Lambda_{\rm fid}) \mathcal{F}\left(\frac{v_{\rm BH}}{\sigma_v}\right) \frac{\bf{v}_{\rm BH}}{v_{\rm BH}}.
\end{equation}
Here $\rho_{\rm sph}$ is the density of dark matter and star particles within the SPH kernel (we will sometimes refer to these particles as "surrounding particles") of the black hole. All other definitions follow those of Equation \ref{eq:df_full}, except that we have substituted $M_{\rm BH}$ with $M_{\rm dyn}$ following the discussion in \ref{subsec:mdyn}.
The function $\mathcal{F}$ defined as:
\begin{equation}
    \label{eq:fx}
    \mathcal{F}(x) =  \text{erf}(x)-\frac{2x}{\sqrt{\pi}} e^{-x^2}, \;
    x=\frac{v_{\rm BH}}{\sigma_v}
\end{equation}
is the result of analytically integrating the Maxwellian distribution, where $\sigma_v$ is the velocity dispersion of the surrounding particles.

The subscript "fid" in $\text{log}(\Lambda)$ means that this definition of $\Lambda$ is specific to the DF(fid) model, with
\begin{equation}
    \Lambda_{\rm fid} = \frac{b_{\rm max,fid}}{(GM_{\rm dyn})/v_{\rm BH}^2}, \; b_{\rm max,fid} = 10\text{ ckpc}/h.
\end{equation}
Note that here we have defined $b_{\rm max}$ as a constant roughly equal to 6 times the gravitational softening. As there is no general agreement on the distance above which dynamical friction is fully resolved, we tested several values ranging from $\epsilon_g$ to $20\epsilon_g$. We found that values above $2\epsilon_g$ are effective in sinking the black hole, although a smaller $b_{\rm max}$ tends to result in more drifting black holes at higher redshift. By using this definition, we are likely overestimating the effective range of dynamical friction. However, we find this over-estimation necessary in the early stage of black hole growth to stabilize the black hole motion.

We also implement a more localized version of dynamical friction following  \cite{Tremmel2015} which we call DF(T15). Under the DF(T15) model, the dynamical friction is expressed as:

\begin{equation}
    \label{eq:T15}
    \mathbf{F}_{\rm DF,T15} = -4\pi \rho (v<v_{\rm BH}) \left(\frac{GM_{\rm dyn}}{v_{\rm BH}}\right)^2  \text{log}(\Lambda_{\rm T15}) \frac{\bf{v}_{\rm BH}}{v_{\rm BH}}.
\end{equation}
Here the surrounding density only accounts for the particles moving slower than the BH with respect to the environment. More formally,
\begin{equation}
\label{eq:rho}
    \rho (v<v_{\rm BH}) = \frac{M(<v_{\rm BH})}{M_{\rm total}} \rho_{\rm T15},
\end{equation}
where $M_{\rm total}$ is the total mass of the nearest 100 DM and stars, $M(<v_{\rm BH})$ is the fractional mass counting only DM and star particles with velocities smaller than the BH, and $\rho_{T15}$ is the density calculated from the nearest 100 DM/Star particles (note that in comparison, the SPH kernel contains 113 gas particles but far more collisionless particles (see Figure \ref{fig:k100_case1})). By using $\rho (v<v_{\rm BH})$ in place of $\rho_{\rm sph} \mathcal{F}$, we are approximating the velocity distribution of surrounding particles by the distribution of the nearest 100 collisionless particles. Another major difference from the DFsph model is the Coulomb logarithm, where in this model we define:
\begin{equation}
    \Lambda_{\rm T15} = \frac{b_{\rm max,T15}}{(GM_{\rm dyn})/v_{\rm BH}^2}, \; b_{\rm max,T15} = \epsilon_g.
\end{equation}
The choice of a lower $b_{\rm max}$ is consistent with the localized density and velocity calculations, and by doing so we have assumed that dynamical friction is fully resolved above the gravitational softening.


\subsubsection{Gas Drag}
\label{subsection:drag}
In addition to the dynamical friction from dark matter and stars, the black hole can also lose its orbital energy due to the dynamical friction from gas (to distinguish from dynamical friction from dark matter and stars, we will refer to the gas dynamical friction as "gas drag" hereafter). \cite{Ostriker1999} first came up with the analytical expression for the gas drag term from linear perturbation theory, and showed that in the transonic regime the gas drag can be more effective than the dynamical friction from collisionless particles. Although later studies show that \cite{Ostriker1999} likely overestimates the gas drag for gas with Mach numbers slightly above unity \citep[e.g.][]{Escala2004ApJ,Chapon2013}, simulations with gas drag implemented still demonstrate that this is an effective channel for black hole energy loss during orbital decays \citep[e.g.][]{Chapon2013,Dubois2013,Pfister2019}.

In order to investigate the relative effectiveness of DF and gas drag, we also include gas drag onto black holes in our simulations following the analytical approximation from \cite{Ostriker1999}:
\begin{equation}
\label{eq:drag}
    \mathbf{F}_{\rm drag} = -4 \pi\rho \left( \frac{G M_{\rm dyn}}{c_s^2} \right)^2 \times \mathcal{I(M)}\frac{\bf{v}_{\rm BH}}{v_{\rm BH}},
\end{equation}
where $c_s$ is the sound speed, $\mathcal{M} = \frac{| \mathbf{v}_{\rm BH} - \mathbf{v}_{\rm gas}|}{c_s}$ is the Mach number, and $\mathcal{I(M)}$ is given by:
\begin{align}
    \mathcal{I}_{\rm subsonic} &= \mathcal{M}^{-2} \left[ \frac{1}{2} \text{log}\left(\frac{1+\mathcal{M}}{1-\mathcal{M}}\right) -\mathcal{M}\right] \\
    \mathcal{I}_{\rm supersonic} &= \mathcal{M}^{-2} \left[ \frac{1}{2} \text{log}\left(\frac{\mathcal{M}+1}{\mathcal{M}-1}\right) -\text{log} \Lambda_{\rm fid} \right],
\end{align}
where $\text{log} \Lambda_{\rm fid}$ is the Coulomb logarithm defined similarly to the collisionless dynamical friction.




