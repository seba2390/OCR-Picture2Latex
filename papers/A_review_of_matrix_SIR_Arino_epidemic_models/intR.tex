\section{Introduction}
{\bf Motivation}.
{Mathematical epidemiology  may be said to have started with the SIR ODE model, which saw its birth in the work of
Kermack–McKendrick  \cite{KeMcK}. This was initially applied to model the  Bombay plague of 1905-06, and later to  measles \cite{earn2008light}, smallpox,
chickenpox, mumps, typhoid fever and diphtheria, and recently to the COVID-19 pandemic -- see \fe\ \cite{Schaback,bacaer2020modele,Ketch,Charp,Djidjou,Sofonea,alvarez2020simple,horstmeyer2020balancing,
di2020optimal,Franco,baker2020reactive,caulkins2020long,caulkins2021optimal}, to cite just a few representatives of a huge literature.}

\iffalse
To be able  to fit ODE models to real epidemic data, it is necessary  to have both a good description of the epidemic and sufficient data.
 Note that for the case of the  Bombay plague, it was shown by \cite{bacaer2012model} that the results of the classic SIR may be improved by  adding two more compartments: rats and fleas. \Fr it is suggested by \cite{bacaer2012model} that for fitting their data, \cite{KeMcK} had to make ``creative choices" which were not direct consequences of their model. Let us recall here a warning due to   \cite{earn2008light}: ``mathematical modelling of biological systems tends to proceed in steps. We
begin with the simplest sensible model and try to discover everything we can
about it. If the simplest model cannot explain the phenomenon we’re trying
to understand then we add more biological detail to the model."

The fitting of the real data is the hardest step of this process. We begin therefore with the theoretical part: the study of
compartmental epidemic (and endemic) models, starting with the classic SIR, and emphasizing subsequently ``matrix-SIR/xyz" models.  Of course, after adding compartments, most explicit results available for SIR are lost; however, explicit results are still available for certain scalar characteristics of matrix-SIR models. This is not the end of the story:  compartmental models which replace  the quadratic SI interaction   %with linear ``force of infection"
of SIR  by more  sophisticated non-linear interactions are also of great interest -- see \fe \cite{capasso2008mathematical,Bohner}.

We end this paragraph  with a pessimistic note: it may happen  that the statistical predictions obtained by applying
different models may not  agree -- see \cite{Loisel,massonis2020structural}.

\fi

 Note that during the COVID-19 pandemic, researchers have relied mostly on models with  quadratic interactions (linear force of infection), which belong \frt\ to a particular class \cite{Arino,Andr,Riano,Fre20} of ``$(x,y,z)$" models.  Here $x$ denotes ``entrance/susceptible" classes, $y$ denotes diseased classes, which must converge asymptotically to $0$,  and $z$ denotes output classes. These models are very useful; to make references to them easier, we propose to call them matrix-SIR (SYR)  models, and  also SIR-PH \cite{Riano}, when $x \in \R^1$.





{\bf Contents}. We begin by recalling in Section \ref{s:SIR} several  basic explicit formulas  for the SIR model.  Section \ref{s:Feng} presents the corresponding SIR-PH generalizations, and Section \ref {s:exa} offers some applications: the SEIHRD model
\cite{ivorra2017stability,Palmer,pazos2020control,nave2020theta,ramos2021simple} which adds to the classic SEIR (susceptible+exposed+infectious+recovered)  a  hospitalized (H) class  and a dead class (D), the SEICHRD model \cite{Kantner}
which adds a critically ill class (C),
the SEIARD  \cite{de2020data} and SEIAHR/SEIRAH(D) models \cite{deng2021extended,otoo2021estimating,wang2020evolving,kucharski2020early,hayhoe2020data,khatua2020fuzzy,prague2020population}, which add an asymptomatic class (A), and the S$I^{3}$QR model \cite{shaw2021reproductive}.
This is just a sample chosen from some of our favorite COVID papers. We note in passing that they seem though all unaware of the existence of the Arino and Feng formulas \eqr{R},  \eqr{Y}. Like most  papers nowadays, they do not recognize the matrix-SIR particular case, and    continue to reprove it   (by the Jacobian, next-generation matrix, ot Chavez-Feng-Huang methods for $\mR$ \cite{Mart}, or by direct computations for the Feng formula), which have become  superfluous once the matrix-SIR particular case is recognized. We also note in passing that the concept of epidemic still seems to lack a  mathematical definition. A definition of the most common particular case is offered in \cite{Breda}; this framework is  more general  than matrix SIR by allowing age-dependence, but the Feng invariant is not discussed  there.


Finally, Section \ref {s:het}  reviews briefly the
case of several classes of susceptibles. This topic requires further development; we include it however due to the recognized importance of heterogeneity factors. % (like  the average contact rates between different age or geographical groups).
