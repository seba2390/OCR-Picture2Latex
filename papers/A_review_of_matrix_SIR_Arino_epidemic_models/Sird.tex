\section{The classic Kermack–McKendrick SIR epidemic  model \la{s:SIR}}


The SIR process $(S(t),I(t), R(t), t \geq 0)$ divides a population of size $N$ undergoing an epidemic into three classes called ``susceptibles, infectives and  removed". One may also work  with the \corr fractions $\s(t)=\fr{S(t)}N,\i(t)=\fr{I(t)}N, $
and  $\r(t)=1-\s(t)-\i(t)$.
 It is assumed that only susceptible
individuals can get infected.  After having been infectious for
some time, an
individual recovers and may not become susceptible again. ``Viewed from far away", this yields the  SIR model with demography  \cite{kermack1927contribution,Chavez}
\begin{align}
S'(t)&=-\fr{\b}N S (t) I (t)+ \xi \pr{N  - S(t)},\nonumber\\ %+\rho R(t)
I'(t)&= I(t)\pr{\fr{\b}N  S(t)-\g -\xi},\la{SIR}
\\
R'(t)&= \g  I(t)-\xi R(t),\nonumber %-\rho R(t)
\end{align}
where
\BEN \im N is the total, constant population size. \im $R'(t)$, the number of removed per unit time, is the only quantity which is clearly observable, at least in the easy case when the removed are dead, as was the case of the original study of the Bombay plague \cite{kermack1927contribution}. \im $\xi$ is the population death rate, assumed to equal the birth rate.
\im $\g$ is the  removal rate of the infectious, which equals 1/duration of the infection (under the stochastic model of exponential infection durations, this is the reciprocal of the expected duration).
\im $\b$, the  infection rate, models the probability  that a contact takes place between an infected and a susceptible, and that it results in  infection.
 \EEN


Note that  \BEN
  \im The sum $S+I+R =N$ is conserved and each value is positive, so the values of $S, I, R$ remain in the interval $[0,N]$.
  \im  This system has a unique  solution, since (given the boundedness of $S, I$, and $R$), the RHS above is Lipschitz.
\EEN



\Fno we will assume that $\xi=0$, and work with the fractions $\s,\i,\r$, which \saty
  \begin{align}
\s'(t)&=  -\b \s(t)\i(t) ,\nonumber\\
\i'(t)&= \i(t)\pp{\b \s(t)-\g},\label{sir}
\\
\r'(t)&= \g \i(t).\nonumber
\end{align}

We will call this the classic SIR model.
Note that
  \BEN \im   $\s(t)$ is monotonically decreasing and $ \r(t)$ is monotonically  increasing, to,  say, $s_\I, r_\I$;   therefore  convergence to some fixed stable point $(s_\I, i_\I, r_\I)$ must hold.

  \im
   the equilibrium set of stable points  is
$(s,0,1-s),      s \in [0,1].$

\im solutions starting in the domain
$$\dom:=\left\{(s,i,r): s>0,i>0,r \geq 0, s+i+r \le 1\right\}$$
    cannot leave it.

\im
The second equation of \eqr{SIR} implies  the so-called {\bf threshold phenomenon}:
if
\be{RS} \mR :=\fr{\b}{\g} \leq 1\ee then $\i(t)$ decreases always, without any intervention.

To  avoid trivialities,  we will assume $\mR >1$ \fno.

\im When $\mR >1$, the    epidemic grows iff $\s > 1/\mR$, i.e. until  the susceptibles $\s(t)$ reach the {\bf immunity threshold}
\be{IT} \Th:=\fr 1 {\mR}=\fr{\g}{\b},\ee
after which  infections decline. $\mR$ is called {\bf basic reproduction number}, and it models the number of susceptibles infected by one infectious (expected number, under more sophisticated stochastic, branching models).
\EEN

An advantage of the classic SIR model is that it is essentially solvable explicitly:
\BEN
\im
We can eliminate $\r$ from the system using the invariant $\s + \i + \r = 1$ (this is also possible for various generalizations like SIR with demography, as long as $\r$ does not appear explicitly in the rest of the equations).

\im

It can easily be verified that \be{inv}\mu(s,i):=s+i-\fr 1 {\mR}\ln(s)\ee
is  invariant, so that
 $i$ is explicitly given by
\be{iR} i_{\mR}(s) =-s +\fr 1 {\mR}\ln(s) + \muR(s_0,i_0),\ee
and the full system \eqref{sir} can be reduced to the single ODE
\be{1ode}
    \s'(t) = -\beta \s(s_0 + i_0 - \s) - \gamma \s \ln\left(\frac{\s}{s_0}\right).
\ee

\im

The  maximal value of the infected $\i$, achieved when $\s=1/\mR$,  is
 \be{im}i_{\max}=i_{\max,\mR}(s_0,i_0)= i_0+s_0 -  \fr {1+ \ln(s_0 \mR)} {\mR}  =i_0 + \f{ H(s_0 \mR)}{\mR}, \; H(\mR)= {\mR-1-\ln(\mR)}.
\ee
%where we put \be{h} H(\mR)= {\mR-1-\ln(\mR)}.\ee


\im By differentiating the right-hand side of \eqref{1ode}, one finds that the
maximal value of the ``newly infected" $-\s'=\b \i \s$ is achieved when
\be{ni} \s =\i +\fr \g \b, \s=-\fr 1{2 \mR} L_{-1}\pp{-2 \mR s_0 e^{-1-\mR (s_0+i_0)}},\ee
where  the Lambert function $L_{-1}$ is a real inverse of $L(z)=z e^z$-- see \fe
\cite{pakes2015lambert,kroger2020analytical,berberan2020exact}.
Bounding $ \s \i$ is one interesting possibility for accomodating  ICU constraints \cite[(2.20)]{Mangat}.

\im The infectious class
converges to $0$ and the susceptible and recovered converge monotonically to limits which may be expressed in terms of the ``Lambert-W(right)" function $L_0$.
\EEN




Let us note that accurate numerical solutions of  the evolution of the SIR or other compartmental epidemic may be obtained very quickly.
\figu{sir}{Plot of the states of \eqr{sir}, for $\mR=2.574,\; \g=.1,\; s_0=.99,\; i_0=1-s_0,\; r_0=0 \Lra r_{\I} =s_0+i_0+ \frac{L_0[-\mR s_0 e^{-\mR(s_0+i_0)}]}{\mR}=0.903171$.
    %,  d_{\I}= \frac{\g_d A}{\g_r+\g_d}= 0.37355 $
   %(see Section \ref{s:back} below
   }{.8}%SIRD


