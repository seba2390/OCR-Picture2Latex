\section{SIR-PH epidemics with one susceptible class  (SIR epidemics with \PH ``disease time"))
\la{s:Feng}}



 It has been known for  a long while that $\mR$ and  the final size %,  and approximations of  the highest peak of
 for many compartmental model epidemics may be   explicitly expressed in terms of the matrices which define  the  model, and \cite{ma2006generality,Arino,Feng,Andr} offer  a   quite   general framework of ``xyz" models which ensures this. We believe that these  formulas have not received the attention they deserve (they keep being reproved), and decided therefore to review them  below; we will call them matrix- SIR models.


  A particular but revealing case
is that when there is only one susceptible class, which we will call  SIR-PH, following Riano \cite{Riano}, who emphasized its probabilistic interpretation -- see also \cite{Hurtado}.
\beD  A ``SIR-PH $(\va,V,\bb,W)$ epidemic" contains  a homogeneous  susceptible class, but  vector ``diseased" state $\vi $ (which may contain latent/exposed, infective , asymptomatic, etc) and vector removed states (healthy, dead, vaccinated, etc). It is defined by an ODE  system:
\be{syr}
\begin{aligned}
\bc
& \s'(t)= - \s(t)\; \vi  (t)  \bb \\
&  \vi '(t)=    \s(t)  \; \vi (t)    \bb \; \va +\vi (t) A \\
&\vec \r'(t) =  \vi (t) W
\ec
\end{aligned}
\ee
where %the transmission rate $\b$ is a constant and
\BEN
\im $\vi (t) \in \mathbb{R}^n$ is a row vector whose components are fractions of diseased individuals of various types,
which must \saty $\vi (\I)=0$.
\im $ \bb \in \mathbb{R}^n$ is a column vector whose components represent the relative transmission ability of the various disease classes.  %The sum  of its components will be denoted by $\b$.

\im $\va \in \mathbb{R}^n $ is a {\bf probability row vector} with the components representing the fractions of susceptibles entering into the corresponding disease compartments, when  infection occurs.
\im  $A$ is a $n\times n$ Markovian sub-generator matrix describing  rates of transition between the diseased classes $\vi $ (i.e., a Markovian generator matrix for which the sum of at least one row is strictly negative). Alternatively, $V:=-A$ is a  non-singular M-matrix.\fn[4]{An M-matrix is  a real matrix $V$ with  $ v_{ij} \leq 0, \forall i \neq j,$ and having eigenvalues whose real parts are nonnegative \cite{plemmons1977m}.}
\im $\vec \r(t) \in \mathbb{R}^p $ is a row vector which must \saty $\vr(\I)>0,$ whose components represent (fractions of) various  classes
which survive at the end of an infection.
\im $W\in \mathbb{R}^n \times \mathbb{R}^p$ is a matrix whose components represent the rates at which classes of diseased individuals become recovered. We assume   that the matrix $\T V =\bep A &W \eep \in \mathbb{R}^{m+n} \times \mathbb{R}^n$ has row sums $0$, which implies mass conservation.

\EEN
\eeD

We turn now to a   deceivingly simple particular example  of the SIR-PH model, which explains its name.
 \beR {\bf Probabilistic interpretation of SIR-PH epidemics}. For simplicity, let us group  all the output classes of \eqref{syr} into one $\r=\vr  \bff 1$,  yielding:
 \be{R21} \bc \s'(t)=- \s(t) \vec \i(t) \bff \b\\
\vec \i'(t)=\s(t) \vec \i(t) \bff \b \va + \vec \i(t) A\\ \r'(t)=\vec \i(t) \bff a,\ec \ee
where we put $\bff a:=W \bff 1=(-A) \bff 1$.

\eqr{R21} emphasizes the fact that SIR-PH models are in one to one correspondence with laws of \PH\ $(\va, A)$ \cite[(21)]{Riano}.

Let us recall now, as known essentially since \cite{Kurtz} -- see also \cite[Thm. 2.2.7]{Britton}--
that under proper scaling, the  expected {fractions} $\s(t)$, $\i(t), \r(t)$  of stochastic SIR\fn[4]{One such model  stipulates that each infective $j=1,...,J$ infects a randomly chosen susceptible, at encounter times which belong to independent    Poisson processes $P^j(t), j=1,...,J$, of rate $\b$, and that   infection durations  are i.i.d. r.v.'s  which are
{\bf exponentially distributed} %with parameter $\g$,
at the end of which the individual recovers (or dies).} and more general { compartmental models} obey  a ``law of large numbers/fluid limit"  which recovers the \det\ epidemic.

As an example, the SIR-PH model \eqref{syr}
 may be derived as limit  of a stochastic SIR model in which the exponential infection time  has been replaced by
 a \PH\ $(\va, A)$ ``dwell period" \cite{Hurtado}.
\eeR
 \beP \la{p:Arino}
For processes defined by \eqr{syr}, with $V=-A$ a non-singular M-matrix,
the basic reproduction number is given by \cite[Thm. 2.1]{Arino}\fn[4]{This can be also derived as the expected number of susceptibles infected during a dwell period, for the stochastic model (the so-called ``survival method")--see  \cite{perasso2018introduction} for an excellent review}.
\be{R} \mR=   \va  \;  V^{-1} \bb.\ee

A disease free equilibrium $(s_0,\vec 0, \vec r_0)$ is asymptotically stable iff $s_0 < \fr 1{\mR}.$


\eeP





To illustrate the power of the SIR-PH formalism, consider now the  case with two diseased states,   latent and  infectious,     with \PH\ dwell times, parametrized by $(\va_e,A_e)$ and $(\va_i,A_i)$, \resp.
Using the \wk\ convolution formula -- see \fe \cite[Thm. 3.1.26]{bladt2017matrix} we find
that  formulas like \eqr{R} (see other examples of such formulas below) still apply, with   $(\va, A, \bb)$
given by
 \be{SEIRPH} \va=(\va_e,0), A= \bep A_e& \ba_e \va_i \\ 0 &A_i \eep,
 \bff \b=\bep 0\\0\\ \vdots\\ \b_{i,1}\\ \b_{i,2} \\ \vdots\eep. %W=(0,\va_i),
\ee

The ``epidemic dwell strucure" $(\va, A, \bb)$  of examples with more complicated network structures for the diseased may be constructed using
Kronecker sums of the matrices defining each component.

\iffalse
This example may be also be solved directly. The first
equation yields
$$\ln(\s_0/\s(t))=\int_0^t \vec \i(u) du \; \bff \b;$$
\thr, obtaining the  SIR
final size is equivalent to computing $\int_0^\I \vi'(u) du$.
 Integrating the second equation and using $\vi_{\I}=0$ we find
 $\int_0^\I \vi(u) du= -\pp{(\s_{\I}-\s_0 )   \va-  \vi_0 }\; V^{-1}$ and
 \be{vecI}   \ln(\s_0/\s_\I)= \pp{(\s_0-\s_{\I})   \va+  \vi_0}\; V^{-1} \bff \b,\ee
 recovering \eqr{fs}.
\fi










\iffalse
\beR  The invariance relation \eqr{inv} may be extended for the case when $N$ \sats\ $N'=\lam - \mu N, \lam =0$,  but not \oth. Indeed, consider  the model  defined by the ODE  system:
\be{syrmu}
\begin{aligned}
\bc
& \s'(t)= \lam -\b \s(t)\; \vi  (t)  \bb - \mu \s \\
&  \vi '(t)= \b  \s(t) \; \vi (t)  \bb \; \vec \al - \vi (t)( V + Q)\\
&\vec \r'(t) =  \vi (t) W -\mu \vec \r
\ec
\end{aligned}
\ee
where  $\lam,\mu$ are the birth (into the recruitment state) and death rates, and $Q$ is the diagonal matrix $Diag(\mu)$.

The invariance relation when $\lam =0$ is  $$\fr{d Y}{d \s}=\; -1 +\; \fr{\mu }{\b  \vi (t) \bb + \mu } +\; \fr{\vi (t)\bb +  \vi (t) Q  V^{-1} \bb}{\mR \s(t) \lt( \vi (t) \bb + \fr{\mu}{\b} \rt)};$$ for simplicity, we only consider below the case $\mu=0$.

We also have,  \be{consmu} \bc z(t):= Y(t) +\; \fr{\b \vi (t) \bb}{\b \vi (t) \bb + \mu} \s(t) -\; ln\lt[\s(t)\ri] \fr{\vi (t) \lt(V +Q \rt) V^{-1} \bb}{\mR \lt( \vi (t) \bb + \fr{\mu}{\b} \rt)}\\
e^{- z(t)}= e^{- \fr{\mR \lt( \vi (t) \bb + \fr{\mu}{\b} \rt)}{\vi (t) \lt(V +Q \rt) V^{-1} \bb}\lt(Y(t) +\; \fr{\b \vi (t) \bb}{\b \vi (t) \bb + \mu} \s(t)\rt)} \ec \ee
 are constant along the path of the dynamical system \eqr{syrmu}.


\eeR
\fi



Let us give now an example which {\bf does not in general} belong to the SIR-PH class.

\beXa \iffalse
The SIR model has been  useful for modeling qualitatively the current COVID-19 pandemic. In this context, it is useful to further divide the removed compartment into recovered (R)  and dead  (D).
Under some  objectives
this may not be necessary,  since the final size of the dead is  simply proportional to the final size of the removed; however, this model allows including a   vaccination controlled parameter $\v$ and  recovering individuals becoming  again susceptible, at a  rate $\rho$.
\fi
The  SIRV model (SIR with vaccination --see for example \cite{BBG}) is defined by:

\begin{align}
\s'(t)&=  -\s(t) \pr{\b \i(t)+ \g_s} ,\nonumber\\
\i'(t)&= \i(t)\pp{\b \s(t)-\g_i },\label{sirV}
\\
\r'(t)&=   \g_i \i(t),\nonumber\\
\v'(t)&=\g_s \s (t). \nonumber
\end{align}
\iffalse

\figu{sirv}{Plot of the states of \eqr{sirV}, for $\b=.4,\; \g_s=.05,\;\g_i=.06,\; s_0=.99,\; i_0=1-s_0,\; r_0=v_0=0 $
   }{.8}%SIRV
   \fi

This is of the form \eqr{syr} with $\vi =(\i), \vr=(\r,\v)$ iff $\g_s=0$.

In the opposite case $\g_s \neq 0$, one may still compute an invariant
\be{invv}\mu(s,i):=\b( s+i) -\g \ln(s)+ \v \ln(i), \ee
and for fixed $s$, putting $\T \g=\fr \g {\g_s}, \T \b=\fr \b {\g_s}, \T \mu_0=\fr {\mu_0} {\g_s}$, \ith
 $i$ is explicitly given by
\be{iR} i(s) =\fr 1{\T \b}L_0 \pp{ \T \b s^{\T \g} e^{\T \mu_0-\T \b s}}.\ee



 \im When $\g_s > 0$, the final size is $s_\I=0$.


\eeXa



{We provide now a list of several  formulas,  obtained by replacing  $\i$ in SIR  by a  scalar
linear combination \eqr{Y} \cite{Feng}. They are all easily proved; \how\
 the formula  for the
maximal value of the newly infected involves also a second linear combination $\y $ \eqr{y}. }

\beP \la{p:Feng}
For processes defined by \eqr{syr}, with $V=-A$ a non-singular M-matrix, \ith:
\BEN

\im The following  \textit{weighted sum of the diseased variables}   \cite[(24)]{Feng}
\be{Y}
Y(t)= \fr{\vi (t)  \;  V^{-1} \bb}{\va  \;  V^{-1} \bb} =\fr{1}{\mR} \; \vi (t)  \;  V^{-1} \bb
\ee
has the property that
\be{invd} \frac{dY}{d\s} =  \fr {\fr{1}{\mR} \vi (t) \pr{  \s(t) \;   \bb \; \va - V}V^{-1} \bb}{- \s(t)\; \vi  (t)  \bb}=
\fr { \vi (t) \bb \; \pr{  \s(t) \;    \mR -1}}{-{\mR} \s(t)\; \vi  (t)  \bb}=
 -1 + \frac{1}{\mR \s}, \ee
 \text{ and that }
\be{inv-SYR}\bc  z(t)=\mu(\s(t),Y(t)):=Y(t)+ \s(t)-\fr 1 {\mR} {ln[\s(t)]},\\
 Z(t)=e^{ -{\mR} z(t)} =\s(t) e^{-\mR (\s(t)+Y(t))} \ec\ee
are constant along the paths of the dynamical system \eqr{syr}.

The solution of $Z(s)=Z(0)$  \wrt\ $\s$ \mbe
\be{sY}
  \s(t) =-\fr {1}{\mR}  L_0 \pp{-{ \mR}{Z_0}  e^{\mR Y(t)}},\ee
   where $[-e^{-1}, \I) \ni z \to L_0(z) \in [-1,\I) $ is the principal branch of the  Lambert-W function.

   \im The derivative with respect to time is
\be{y}\frd{Y}{t}=\left(\s-\fr 1{\mR}\right)\vi  \bb:=\left(\s-\fr 1{\mR}\right)\y.\ee
\Thr $\frd{Y}{t}=0=\fr{dY}{ds}$ iff $ s=\mR^{-1}$.

\im
The maximum value of $Y$ occurs for $s =  \min[1/\mR,1]$.
In the case $\mR >1$,
this yields \cite[Sec. 2.1]{Feng}:
\be{mp} Y_0 +\s_0 - Y^*-\fr 1 {\mR} = \fr 1 {\mR} \ln({\s_0 \mR}),\ee
by the conservation of $Y(t)+\s(t)-\fr 1 {\mR} \ln({\s(t)})$ between  the time $0$ and the time $t_{1/\mR}$ of reaching the immunity threshold).



\im The final size  of the susceptibles  \sats \cite[Thm.5.1]{Arino}:
\be{fs}  {ln[\s_0/\s_\I]}={\mR}\pr{\s_0 -\s_\I} +  \vi _0 V^{-1} \bb={\mR}\pr{\s_0 -\s_\I+ Y_0}, \ee
by the conservation of $Y(t)+\s(t)-\fr 1 {\mR} \ln({\s(t)})$ between  the times $0$ and $\I$; explicitly,

\be{si}
  \s_\I =-\fr {1}{\mR}  L_0 \pp{-{ \mR}{Z_0}}=-\fr {1}{\mR}  L_0 \pp{-{ \mR}{s_0 e^{- \mR(s_0 + Y_0)}}}\ee


\im   The integrated infectives  $\vI{a,b}=\int_a^b \vi (s) ds$
\sats\
\be{ini} \bc \vI{a,b} \;  V= \vJ_a-\vJ_b, \vec J_s:= \vi (s)+ s \va\\
 \pr{\vJ_a-\vJ_b} V^{-1} \bb= \log(\fr{\s(a)}{\s(b)}) \ec, \ee
 and the total integrated infectives  $\vI{\I}=\int_0^\I \vi (s) ds$
\sats\ \cite[(6)]{Arino}
\be{ini-total} \vI{\I} \;  V= \vi _0+ (s_0-s_\I)\va . \ee



\beR \la{r:int} In particular, for the SIR model \eqr{sir},
\be{Isi}\log\left(\fr{\s(a)}{\s(b)}\right)=\b I^{(a,b)} =\mR (J_a-J_b), \; J=\s+ \i.\ee
%\itf $I^{(a,b)} $ determines both the final values   $\s(b), \i(b)$.

Note that this has been used to model the total cost of an epidemic \cite{gani1972cost}.
\eeR



\im The final size  of the removed  \sats:
\be{rec} \vr_\I-\vec r_0 =I^{(\I)} \;  W=\pr{\vi _0+ (s_0-s_\I)\va} V^{-1} W , \ee

\im

The  value of the infected combination  $Y$  when $\s=1/\mR$  is
 \be{im-SYR}Y_{\max}=i_{\max,\mR}(s_0,\vi _0)= Y_0+s_0 -  \fr {1+ \ln(s_0 \mR)} {\mR}  =i_0 + \f{ H(s_0 \mR)}{\mR}, \; H(\mR)= {\mR-1-\ln(\mR)}.
\ee

\im The maximum size of the newly infected is achieved when
    \be{sdsmax} s(t) =\frac{\y^2 +\mR Y(t)}{\va \bb \y}. %= \fr{1}{\va \bb} \pr{y(t) + \mR \fr{Y(t)}{y(t)}}
    \ee
%\im \red{The maximum size of the newly infected is achieved when $\vi $ is an eigenvector of $\s \bb \va -V$, corresponding to the dominant eigenvalue which equals $\y$ ?}
\EEN
\eeP

\beR Let us note that for control problems involving optimization objectives which only depend on $Y(t)$, we are  effectively optimizing a SIR model;
this SIR approximation may be used to offer practical solutions for optimizing more complicated compartmental models. %Also,  the original parameters of the model  enter the invariance equations only via the formula of $\mR$.
\eeR

\beXa For SEIR, putting $\vi  =(\e , \i)$, we may write
 \bea %{seirex}
 \bc
 \s'= -\b \s \i \\
 \vi  '=   (\b \s \i  - \g_e \e, \g_e \e  -\g \i)= \b \s \i (1,0)  -(\e,\i) \bep  \g_e & -\g_e \\ 0& \g \eep   \\
 \r'= \g \i
 \ec \eea
so that   $\bc  \va= (1,0) \\  \bff b= \bep
0 \\\b
\eep\\V= \bep  \g_e & -\g_e \\ 0& \g \eep \Lra V^{-1} = \fr 1{\g \g_e} \bep  \g & \g_e \\ 0& \g_e \eep,  Y = \fr{\bep \e,\i \eep \bep  \g & \g_e \\ 0& \g_e \eep \bep
0 \\\b
\eep}{\bep 1,0 \eep \bep  \g & \g_e \\ 0& \g_e \eep \bep
0 \\\b
\eep}=\e +  \i\\ W= \bep 0&\g\eep \ec.$


\iffalse
We add here a point raised by \cite{Post}: ``
the stage of illness each infected person enters the statistics as
a registered infected one is less reflected in the available data. Hence, the division into latent (exposed) and active forms, i.e any
kind of SEIR-models might be over-complexification with respect to
the actual data uncertainty."
\fi

\eeXa




\section{Examples of SIR-PH models used in COVID-19 modelling \la{s:exa}}



{We derive now $\mR$ and $Y$ from \eqr{R},  \eqr{Y}, for some  popular compartmental models.} Note that we will be  reformulating the original results (which, unfortunately,  have already appeared several times with different notations), using  a
unifying notation.

\beXa The SEIHRD model  \cite{ivorra2017stability,Palmer,pazos2020control,nave2020theta,ramos2021simple} has disease states $\vi  =(\e , \i, \h)$. We use here  the version in  \cite{pazos2020control} (we would rather call this S$I^2$HRD model),   defined by $$\va=\bep 1,0,0\eep, \bb=\bep \b_e\\\b_i\\0\eep, V=\left(\begin{array}{ccc}
 \g_e  & -e_i  & 0 \\
 0 & \gamma_i  & -i_h \\
 0 & 0 & \g_h \\
\end{array}
\right), W =\bep e_r&0\\i_r&0\\h_r&h_d \eep,
$$ where we denoted by $\g_e,\g_i, \g_h$ the sum of the constant rates out of $\e, \i, \h$, and by $i _h$ the rate out of $\i$ and reaching $\h$, etc. Then, $\mR=\fr {\b_e}{\g_e}+ \fr {e_i }{\g_e  } \fr { \b_i}{\g_i }$ \cite[2]{pazos2020control}, %\cite[Thm 1]{ivorra2017stability},
and $ Y=\e+\i \frac{ \gamma _e \beta _i}{\beta _e \gamma _i+e_i \beta _i}.$  When $\b_e=0=e_r \Lra \fr{e_i}{\g_e} =1$, we recover $\mR=\fr {\b_i}{\g_i}$ \cite{Palmer} and $ Y=\e+\i.$

\begin{figure}[H]
\centering
\includegraphics[scale=0.8]{SEIHRD1}
\caption{Chart flow of the SEIHRD model. The forces of infection  are $F_{se}= \b_e \e,\; F_{si}= \b_i \i  , F=F_{se}+F_{si}.$
The red edge corresponds to the entrance of susceptibles into the diseased classes; $\va$, the dashed green edges correspond to contacts between  diseased to susceptibles, the brown edges are the rates of the transition matrix $V$, and the remaining yellow dashed flows correspond to the rates of $W$. \label{f:SEIHRD}}
\end{figure}
\eeXa
\beXa The SEIHCRD model of \cite{Kantner} has disease states $\vi  =(\e , \i, \h, \c)$ and is defined by $$\bb=\bep 0\\\b_i\\0\\0\eep, \va=\bep 1,0,0,0\eep, V=
\bep
 \g_e  & -e_i &0  & 0 \\
 0 & \g_i  &-i_h& 0 \\
 0 & 0& \g_h &-h_c \\
 0 & 0 &-c_h& \g_c \\
\eep , \; W =\bep 0&0\\i_r&0\\h_r&0 \\
0 & c_d\eep%\Lra
,$$
then,
$$ \mR=\frac{\b_i}{ \gamma _i}, Y=\e+\i.$$


 \begin{figure}[H]
\centering
\includegraphics[scale=0.8]{SEIHCRD1}
\caption{Chart flow of SEIHCRD model. %The red edge corresponds to the entrance of susceptibles into infectious class; $\va$, the dashed green edge corresponds to the flux of contact between infected and susceptibles , the brown edges are the rates of the transition matrix $V$, and the remaining black flows correspond to the rate of $W$.
The force of infection  is $ F=\b_i \i.$
\label{f:SEIHCRD}}
\end{figure}
\eeXa

\beXa The SEIRAH(SEIAHR) model  \cite{%coelho2020modeling,
deng2021extended,prague2020population} has disease states $\vi  =(\e , \i, \a,\h)$ and
is defined by $$\bb=\bep 0\\\b_i\\\b_a\\0\eep, \va=\bep 1,0,0,0\eep, V=
\bep
 \g_e  & -e_i &-e_a  & 0 \\
 0 & \g_i  &0& -i_h \\
 0 & 0& \g_a &-a_h \\
 0 & 0 &0& \g_h \\
\eep,\;  W =\bep e_r\\i_r\\ a_r \\ \g_h \eep.$$

Then, $$\mR=\frac{ e_a}{\gamma _e } \mR _a +\frac{e_i }{\gamma _e } \mR _i, \quad {\mR _i }=\frac{\b _i }{\g_i}, {\mR _a }=\frac{\b _a }{\g_a},$$ %\cite{otoo2021estimating}
and $$ Y=\e+\i \frac{\mR _i }{\mR}+\a \frac{\mR _a }{\mR}.$$
 %As a check, note that when $\a=0,\b_i=\b_a$ (in the absence of asymptomatics), these formulas reduce to the ones in the SEIHRD example.
   \begin{figure}[H]
\centering
\includegraphics[scale=0.8]{SEIRAH1}
\caption{Chart flow of the SEIAHR model. %The red edge corresponds to the entrance of susceptibles into infectious class; $\va$, the dashed green edges correspond to the flows of contact between infected and susceptibles, the brown edges are the rates of the transition matrix $V$, and the remaining black flows correspond to the rate of $W$.
The forces of infection  are $F_{si}= \b_i \i,\; F_{sa}= \b_a \a  , F=F_{si}+F_{sa}.$
\label{f:SEIRAH}}
\end{figure} \eeXa

