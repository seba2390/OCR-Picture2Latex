\section{Related Work}
\label{sec: related works}


There has been a long history of studies on compositional generalization \citep{lake2018generalization,jia2016data, andreas2019good,lake2018generalization,ouyang2023compositional,keysers2020measuring,chen2020compositional,dziri2023faith,shao2023compositional,saparov2022language,nye2021show,welleck2022naturalprover,dong2019neural,schwarzschild2021can}. Different types of approaches have been developed to solve compositional generalization. One widely studied approach is neuro-symbolic methods \citep{dong2019neural,schwarzschild2021can}, which blend symbolic and distributed representations for modeling the reasoning process. A recent line of work that has gained significant traction is to prompt large language models to unlock its potential compositional generalization capabilities \citep{nye2021show,zhou2022least,khot2022decomposed,dua2022successive,dziri2023faith}. The least-to-most prompting \citep{zhou2022least} boosts the performance of compositional generalization by first decomposing a difficult problem into a sequence of easy-to-hard problems and then solving them sequentially. Meanwhile, the decomposed prompting \citep{khot2022decomposed} breaks the original problem into a set of different subproblems, solves them sequentially, and then aggregates the answers into a final solution. In spite of the significant improvement compared to previous works, the performance of these approaches still degrade quickly over increasingly harder testing problems. Moreover, their applications are limited to a class of problems that can be decomposed into a set of subproblems. For more general complex problems, where the subproblems are highly nested (e.g., the ones shown in \citet{dziri2023faith}), it becomes quite challenging to construct the prompts and the examplars. Unlike these multi-stage prompting methods, which require multiple calls of the LLM inference process, our proposed Skills-in-Context prompting is a simple one-stage strategy that can be used in a plug-and-play manner to replace existing standard or chain-of-thought prompting.

