\section{Modeling Knowledge Markets}
In this section, we introduce economic production models to capture content generation dynamics in real-world knowledge markets. We first draw an analogy between economic production and content generation, and report the content generation factors in knowledge markets (Section 4.1). Then, we concentrate on the knowledge markets in \CQA{StackExchange}---presenting production models for different content types (Section 4.2).

\subsection{Preliminaries} 
Economic production mechanisms well describe content generation in knowledge markets. In economics, \emph{production} is defined as the process by which human labor is applied, usually with the help of tools and other forms of capital, to produce useful goods or services---the \emph{output}~\cite{stanford2008economics}. We assert that participants of a knowledge market function as labor to generate content such as questions and answers. Analogous to economic output, content contributes to participant utility. 

Motivated by the production analogy, we design macroeconomic production models to capture content generation dynamics in knowledge markets. In these models, instead of directly modeling content generation as a dynamic process (function of time), we model it in terms of associated factors which are dynamic. 

There are two key factors that affect content generation in knowledge markets, namely user participation and content dependency. User participation is the most important factor in deciding the quantity of generated content. The participation of more users induce more questions, answers, and other contents in a knowledge market. Content dependency also affects the quantity of generated content for different types. Content dependency refers to the dependency of one type of content (e.g., answers) on other type of content (e.g., questions). In absence of questions, there will be no answers in a knowledge market, even in the presence of many potential participants who are willing to answer. 

\subsection{Modeling \CQA{StackExchange}}
\CQA{StackExchange} is a network of community question answering websites where each site is based on a focused topic. Each user of the \CQA{StackExchange} network participates in one or more of these sites based on their interests. \CQA{StackExchange} sites are free knowledge markets where participants generate content for non-monetary reputation-based incentives. These markets are diverse, varying in theme (subject matter), size (number of users and amount of activity), and age (number of days in existence). 

We design production models for three primary content types in \CQA{StackExchange}: questions (the root content), answers (which nest below questions), and comments (which can nest either beneath questions or answers). Based on the content dependency and user roles in content generation, we propose the following relationships for question, answer and comment generation in \CQA{StackExchange} (see Table~\ref{tab:notations} for notation).

\begin{table}[thb]
	\caption{Notation used in the model}
    \label{tab:notations}
	\begin{center}    
	\begin{tabular}{cl}
	\toprule Symbol & Definition\\ \midrule
	$U_q(t)$ & \# of users who asked questions at time $t$\\ 
	$U_a(t)$ & \# of users who answered questions at time $t$\\
	$U_c(t)$ & \# of users who made comments at time $t$\\
	$N_q(t)$ & \# of active questions at time $t$\\
	$N_a(t)$ & \# of answers to active questions at time $t$\\
	$N_c^q(t)$ & \# of comments to active questions at time $t$\\
	$N_c^a(t)$ & \# of comments to active answers at time $t$\\
    $N_c(t)$ & \# of comments to active questions/answers at time $t$\\
	$f_x$ & The functional relationship for content type $x$\\ \bottomrule
	\end{tabular}
    \end{center}
\end{table}

There is a single factor in generating $N_q$ questions: the number of users $U_q$ who ask questions (askers).
\begin{equation*}
N_q = f_q(U_q)
\end{equation*}

There are two key factors in generating $N_a$ answers: the number of questions $N_q$, and the number of users $U_a$ who answer questions (answerers). 
\begin{equation*}
N_a = f_a(N_q, U_a)
\end{equation*}

There are two types of comments: comments on questions, and comments on answers. Accordingly, there are three key factors in generating $N_c$ comments: the number of questions $N_q$, the number of answers $N_a$, and the number of users $U_c$ who make comments (commenters). 
\begin{equation*}
N_c^q = f_{c^q}(N_q, U_c)
\end{equation*}
\begin{equation*}
N_c^a = f_{c^a}(N_a, U_c)
\end{equation*}
\begin{equation*}
N_c = N_c^q + N_c^a
\end{equation*}

The aforementioned relationships imply that the amount of generated content of each type depends on the function describing its factor dependent growth, and the availability of factor(s). These relationships make three assumptions. First, different content types interact only through their use of factors. Second, the functional relationships depend on the consumption or usage of each factor. Third, the functional relationships depend on the interaction among the factors---how the factors of a particular content type interact. 

Now, we transform the functional relationships into production models by first choosing a basis function to capture how a content type consumes its factor(s), and then choosing an interaction type to capture the interaction among factors.

\textbf{Basis Function.} We use a basis function to capture the effect of a given factor on a particular content type. While there is a variety of basis functions available for regression, we consider three basis functions widely used in economics and growth modeling~\cite{fekedulegn1999}: power--- $g(x) = ax^{\lambda}$; exponential--- $g(x) = ab^x$; and sigmoid--- $g(x) = \frac{L}{1+e^{k(x-x_0)}}$. 

\textbf{Interaction among the Factors.} We use an aggregate function to capture the interaction among multiple factors of a given content type. Specifically, we consider the pairwise interaction functions listed in Table~\ref{tab:interaction}. 

\begin{table}[htb]
  \caption{Pairwise interaction between factors with contour}
  \label{tab:interaction}
  \centering
  \begin{tabular}{m{.28\textwidth}c}
	\vspace{-5pt}
    \uline{Essential:} Essential factors are both required for content generation, with zero marginal return for a single factor. For a pair of essential factors, content generation is determined by the more limiting factor: $z = min(y_1, y_2)$~\cite{tilman1980}. This is known as Liebig's law of the minimum.
    &
    \begin{minipage}{.17\textwidth}
      \includegraphics[width=\textwidth, height=\textwidth]{Figures/Essential.pdf}
    \end{minipage}
    \\ 
    \vspace{-5pt}
    \uline{Interactive Essential:} In interactive essential interaction, we get diminishing return (instead of zero return) for a single factor: $z = y_1y_2$~\cite{tilman1980}. If factors are consumed using power basis function, i.e., $y_i = ax^\lambda_i$, it captures Cobb-Douglas production function.
    &
    \begin{minipage}{.17\textwidth}
      \includegraphics[width=\textwidth, height=.975\textwidth]{Figures/Interactive_Essential.pdf}
    \end{minipage}
    \\
    \vspace{-5pt}
    \uline{Antagonistic:} For antagonistic factors, content generation is determined solely by the availability of the factor which yields the largest return: $z = max(y_1, y_2)$~\cite{tilman1980}. This interaction implies that the production process has maximum possible efficiency. 
    &
    \begin{minipage}{.17\textwidth}
      \includegraphics[width=\textwidth, height=.975\textwidth]{Figures/Antagonistic.pdf}
    \end{minipage}
    \\
    \vspace{-5pt}
    \uline{Substitutable:} Factors that can each support production on their own are substitutable relative to each other: $z = w_1y_1 + w_2y_2$~\cite{tilman1980}. This implies that there exists some equivalence between the two factors. This is analogous to the general additive models.
    &
    \begin{minipage}{.17\textwidth}
      \includegraphics[width=\textwidth, height=.975\textwidth]{Figures/Substitutable.pdf}
    \end{minipage}
    \\
  \end{tabular}
\end{table}

We combine a basis function and an interaction type to design production models for different content types. For example, answer generation can be modeled using power basis and essential interaction as $N_a = min(a_1N_q^{\lambda_1}, a_2U_a^{\lambda_2})$. We consider twelve such possible models (combination of three basis and four interaction type) for answer and comment generation in StackExchange.

\textbf{User Role Distribution.} A fundamental assumption of our model is the awareness of user roles (e.g., asker, answerer, and commenter) and their distribution (e.g., how many users are askers?). We empirically observe that all StackExchange markets have a stable distribution of user roles. In fact, given the number of users, we can accurately predict the number of participants for each role. 

We apply linear regression to determine the number of participants $U_x$ of a particular role $x \in \{q, a, c\}$ from the number of users $U$ in a StackExchange market. For each market, we compute three distinct coefficients of determination, $R^2$, for predicting three roles (asker, answerer, and commenter) using linear regression. In Figure~\ref{fig:roles} we show the distribution of $R^2$ for regressing user roles across 156 StackExchange markets. We use letter value plots\footnote{The letter-value plot display information about the distribution of a variable~\cite{Hofmann2017}. It conveys precise estimates of tail behavior using letter values; boxplots lack such precise estimation.} to present these distributions---showing precise estimates of their tail behavior. We observe that, in most markets, the $R^2$ values are close to 1. Further, the tail capturing low $R^2$ values consists of markets with a relatively small number of monthly users.

\begin{figure}[hbt]
\centering
\includegraphics[scale=0.45]{Figures/User_to_Roles_R_Squared_LV.pdf}
\caption{The distribution of coefficient of determination, $R^2$, for regressing user roles across 156 StackExchange. In most StackExchange, the role distribution is stable---as manifested by the arrangement of $R^2$ in letter-value plots.}
\label{fig:roles}
\end{figure}

\textbf{Number of Users.} The number of users is the only free input to our content generation models; the remaining inputs are functions of the number of users. In all these models, the growth or decline of number of users is exogenous---determined outside the model, by non-economic forces.



