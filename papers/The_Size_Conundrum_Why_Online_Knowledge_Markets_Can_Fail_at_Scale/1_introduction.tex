\section{Introduction}

% What is the problem?
In this paper, we analyze a large group of community question answering (CQA) websites on \CQA{StackExchange} network through the Economic lens of a market. Framing \CQA{StackExchange} sites as knowledge markets has intuitive appeal: in a hypothetical knowledge market, if no one wants to answer questions, but only ask, or conversely, there are individuals who want to only answer but not ask questions, the ``market'' will collapse. What, then, is the required relationship among actions (say between questions and answers) in such a knowledge market for us to deem it healthy? Are larger markets with more participants healthier since there will be more people to ask and answer questions? 

% Why is it important?
Studying CQA websites through an economic lens allows site operators to reason about whether they should grow the user base. Since most of the popular CQA websites (e.g.\ \CQA{Quora}, \CQA{StackExchange}) do not charge participants, but instead depend on site advertisements for revenue, there is a natural temptation for operators of these sites to grow the user base so that there is increase in revenue. As we show in this paper, for most \CQA{StackExchange} sites, growth in the user base is counter-productive in the sense that they turn unhealthy---specifically, more questions remain unanswered.

% Why is it hard?
Explaining the macroscopic behavior of knowledge markets is important, yet challenging. One can regress some variable of interest (say number of questions) on variables including number of users, time spent in the website among others. However, explaining why the regression curve looks the way it does is hard. As we show in this work, using an economic lens of a market allows us to model dependencies between number of participants and the amount of content, and to predict the production of content.

% What did you do?
Our main contribution is to model CQA websites as knowledge markets, and to provide insight on the relationship between size and health of these markets. To this end, we develop models to capture content generation dynamics in knowledge markets. We analyze a set of basis functions (the functional form of how an input contributes to output) and interaction mechanisms (how the inputs interact with each other), and identify the optimal \emph{power basis} function and the \emph{interactive essential} interaction form using a prediction task on the outputs (questions, answers, and comments). This form is the well-known Cobb-Douglas form that connects production inputs with output. Using the best model fits for each \CQA{StackExchange} site, we show that the Cobb-Douglas model predicts the production of content with high accuracy.

The Cobb-Douglas function provides an intuitive explanation for content generation in \CQA{StackExchange} markets. It demonstrates that, in \CQA{StackExchange} markets, \begin{enumerate*}
  \item factors such as user participation and content dependency have \emph{constant elasticity}---percentage increase in any of these inputs will have constant percentage increase in output;
  \item in many markets, factors exhibit \emph{diminishing returns}---decrease in the marginal (incremental) output (e.g., answer production) as an input (e.g. number of people who answer) is incrementally increased, keeping the other inputs constant;
  \item markets vary according to their \emph{returns to scale}---the increase in output resulting from a proportionate increase in all inputs; and
  \item many markets exhibit \emph{diseconomies of scale}---measures of
   health decrease as a function of overall
   system size (number of participants)
 \end{enumerate*}.

There are two reasons why we see diminishing returns in the \CQA{StackExchange} markets. First, the total activity of participants for any \CQA{StackExchange} market unsurprisingly follows a power-law pattern. What is interesting is that the power-law exponent falls with increase in size for most markets, implying that new users do not participate in the same manner as earlier users. Second, we can identify a stable core of users who actively participate for long periods of time, contributing to the market health.

Finally, we show diseconomies of scale through experiments on system size, analysis of health metrics, and user exchangeability. For most \CQA{StackExchange} markets, we see that as system size grows, the ratio of answers to questions falls below a critical point, when some questions go unanswered. Furthermore, using health metrics of the number of questions with an accepted answer, and the number of questions with at least one answer, we observe that most \CQA{StackExchange} markets decline in health with increase in size. Finally, we compare the top contributors with the bottom contributors to see if they are ``exchangeable.'' Most StackExchange markets are not exchangeable in the sense the contributions of the top and the bottom contributors are qualitatively different and differ in absolute terms. These experiments on diseconomies of scale are consistent with the insight from Cobb-Douglas model of production that predicts diminishing returns.

