\section{Failures at Scale}
In this section, we discuss how and why knowledge markets may fail at scale. We first empirically examine diseconomies of scale (Section 8.1), then analyze the effects of scale on market health (Section 8.2), and finally study user exchangeability under scale changes (Section 8.3).

\begin{figure}[b]
\centering
\includegraphics[scale=0.39]{Figures/Size_vs_Econ_Ratio.pdf}
\caption{Diseconomies/economies of scale: the ratio of answers to questions decreasing/increasing with the increase in number of users. Most \CQA{StackExchange} markets exhibit diseconomies of scale. Examples: strong diseconomies---\SE{superuser}; weak economies---\SE{puzzling}; and strong economies---\SE{cstheory}.}
\label{fig:diseconomy}
\end{figure}

\subsection{Diseconomies of Scale}
First, we examine disceconomies of scale---the ratio of answers to questions declining with the increase in number of users. The opposite of diseconomies is economies, when the ratio increases with the increase in number of users. The concept of diseconomies is important because a decrease in the answer to question ratio implies an increase in the gap between market supply (answer) and demand (question). In fact, if the ratio falls below 1.0, the gap becomes critical---guaranteeing there will be some questions with no answers. 

In Figure~\ref{fig:diseconomy} we present the economies and diseconomies of scale in three StackExchange markets: \SE{cstheory}, \SE{puzzling} and \SE{superuser}. We choose these examples to cover three cases: strong diseconomies, strong economies, and weak economies. Among the three markets, \SE{superuser} shows strong diseconomies of scale: if the number of users increases by 1\%, then the answer to question ratio declines by 0.95\%. The other two markets show economies of scale, where \SE{cstheory} shows strong economies: if the number of users increases by 1\%, then the answer to question ratio increases by 0.8\%; and \SE{puzzling} shows weak economies: if the the number of users increases by 1\%, then the answer to question ratio increases by 0.2\%. Note that most markets, especially the ones with more than 500 monthly active participants, exhibit diseconomies of scale similar to \SE{superuser}. Only five markets exhibit strong economies of scale in \CQA{StackExchange}: \SE{cstheory}, \SE{expressionengine}, \SE{puzzling}, \SE{ja\_stackoverflow}, and \SE{softwareengineering}.

The Cobb-Douglas curves well fit the empirical trends of economies and diseconomies (as shown in Figure~\ref{fig:diseconomy}). We derive these curves by dividing the answer models by the corresponding question models, and subsequently developing curves that capture economies and diseconomies ($N_{a/q}$) as a function of number of users (system size). We get similar curves via log regression. Between the two model types, the Cobb-Douglas models provide better explanation.

The Cobb-Douglas models well explain the economies and disceconomies of scale. As per the models, the primary cause of disceconomies is the difference between the diminishing returns of questions and answers for user participation. In other words, in most markets, for user input, the marginal question output is higher compared to the marginal answer output, i.e., an average user is likely to ask more questions and provide few answers. This causes the ratio of answers to questions to decline with an increase in the number of users. 

\subsection{Analyzing Health}
Next, we examine the disadvantage of scale through two health metrics: $H_1$---the fraction of answered questions (questions with at least one answer); and $H_2$---the fraction of questions with an accepted answer (questions for which the asker marked an answer as accepted). $H_1$ and $H_2$ capture the true gap between market supply (answers) and demand (questions). An increase in the number of users may cause a decline in $H_1$ and $H_2$, as both metrics are related to the ratio of answers to questions. In fact, if the ratio falls below 1.0, it guarantees the decline of both metrics. 

In Figure~\ref{fig:health} we present the health advantage and disadvantage of scale (through $H_1$ and $H_2$) for three StackExchange markets: \SE{cstheory}, \SE{puzzling} and \SE{superuser}. We observe that the results are consistent with our analysis of economies and diseconomies---\SE{cstheory} exhibits health advantage at scale, \SE{puzzling} remains stable, whereas \SE{superuser} exhibits disadvantage at scale. These three examples cover the possible health effects of scale in knowledge markets. 

\begin{figure}[t]
\centering
\includegraphics[scale=0.39]{Figures/Size_vs_Health.pdf}
\caption{Health disadvantage/advantage of scale: $H_1$---the fraction of answered questions, and $H_2$---the fraction of questions with accepted answer, decreasing/increasing with the increase in number of users. Most \CQA{StackExchange} markets exhibit health disadvantage at scale. Examples: disadvantage---\SE{superuser}; neutral---\SE{puzzling}; and advantage---\SE{cstheory}.}
\label{fig:health}
\end{figure}

\begin{figure}[b]
\centering
\includegraphics[scale=0.39]{Figures/Size_vs_Exch.pdf}
\caption{User exchangeability under scale: the gap ($E_1$ and $E_2$) between the top contributors and other participants in a knowledge market decreasing or increasing with the increase in number of users. Most markets exhibit a large gap between the top contributors and other participants. Examples: high dissimilarity---\SE{superuser}; moderate dissimilarity---\SE{puzzling}; and low dissimilarity---\SE{cstheory}.}
\label{fig:stability}
\end{figure}

\subsection{Effects on Exchangeability}
Finally, we empirically study the effects of scale on user exchangeability. By exchangeability, we specifically mean the gap between the top contributors and other participants in a knowledge market. Studying this gap is important because it can reveal if a market's success or failure depends on a small group of users.  

To empirically study user exchangeability, we define two metrics that reflect the gap between the top contributors and other participants in a knowledge market. Note that, we only consider the active participants who contributed at least one content. The first metric $E_1$ is defined as the ratio of contribution between the top 5\% and the bottom 5\% of users. For computing $E_1$, we measure the contribution of a user $v$ as the ratio $N^v_{a/q}$ of the number of answers $N^v_{a}$ provided by the user to the number of questions $N^v_{q}$ asked by the user. Notice that $E_1$ is a ratio based metric and we define user contribution to be consistent with this metric. The second metric $E_2$ is defined as the sum of two distances: (i) the distance between the contribution of the top 5\% of users and the median 5\% of users, and (ii) the distance between the contribution of the median 5\% of users and the bottom 5\% of users. For computing $E_2$, we measure the contribution of a user $v$ as a tuple $<N^v_a, N^v_q>$, consisting of the number of answers $N^v_{a}$ provided by the user and the number of questions $N^v_{q}$ asked by the user. Notice that $E_2$ is an interval based metric and we define user contribution to be consistent with this metric. While both metrics have certain limitations, e.g., they are sensitive to outliers, these metrics allow us to comprehend user exchegeability to some extent.

In Figure~\ref{fig:stability} we present the exchangeability of users under scale changes (through $E_1$ and $E_2$) for three \CQA{StackExchange} markets: \SE{cstheory}, \SE{puzzling} and \SE{superuser}. Among the three markets, \SE{superuser} exhibits the highest gap between the top contributors and the other participants. However, as the number of participants increases, this gap decreases, i.e., the users become more exchangeable. In contrast, \SE{cstheory} exhibits the lowest gap between the top contributors and the other participants. However, as the number of participants increase, this gap increases, i.e., the users become less exchangeable.



