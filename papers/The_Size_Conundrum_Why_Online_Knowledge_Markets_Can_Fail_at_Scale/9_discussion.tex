\section{Implications}
Our work promotes two new research directions---size-dependent mechanism design and content dependency in social media---while advancing several others---metrics of market health, power law of participation, and microfoundations of knowledge markets. 

\textbf{Size-Dependent Mechanism Design.} We reveal that the health of a knowledge market depends on the market's size. A natural implication of this dependency is that site operators should adjust mechanisms based on the number of participants. For example, a site operator can decide between retaining existing users (via incentives) and attracting new users (via advertising) based on the number of participants and their activity distribution.

\textbf{Content Dependency in Social Media.} We observe that many social media platforms support several possible user actions with ``complex dependencies''. For example, in Facebook, a post is the root content (primary), comments on the post nest below the post (secondary), and replies to these comments nest beneath the original comments (tertiary). Further, a user can react to any of these content types with several possible reactions. Overall user activity in Facebook is distributed across these possible actions with complex dependencies, which drives the platform's health.

\textbf{Metrics of Market Health.} We demonstrate the presence of diseconomies of scale with several metrics that partially capture the health of a knowledge market. While we concentrate on content-generation based \emph{production metrics}, our concepts can be extended for page-view based \emph{consumption metrics} as well. Also, there is room for developing new health metrics that capture a more detailed picture of a knowledge market's health including \emph{market efficiency}---the degree to which market price (amount of responses and reactions) is an unbiased estimate of the true value of the investment (user effort in content generation)~\cite{damodaran2002}.

\textbf{Power Law of Participation.} In \CQA{StackExchange} markets, a small fraction of the user community participate in high-engagement activities (e.g., linking similar questions), whereas the larger fraction participate in low-threshold activities (e.g., voting). This asymmetry leads to a \emph {Power Law of Participation}~\cite{mayfield2006}. We assert that both low-threshold and high-engagement activities are required for a knowledge market's survival, and should proportionately increase with the increase in number of participants. However, in reality, for most knowledge markets, the size of the user community contributing high-engagement activities does not scale with the system size. This creates a ``gap'' between market supply and demand, and consequently affects market health. 

\textbf{Microfoundations of Knowledge Markets.}
The size-dependent distribution of user contribution implies that users who join a community later in its lifecycle exhibit different behavior than those who were present from the beginning. This very well may imply that the distribution of individual user behaviors (not just their overall production) is ``also'' a function of the system size. We should expect to see a stable user behavior distribution over time for markets that appear to be more scale-insensitive; preliminary results suggest that this may indeed be the case~\cite{geigle2017}.

\section{Limitations}
We discuss several limitations of our work. First, the economic production models do not account for user growth. While there exist several user growth models for two-sided markets~\cite{Kumar2010}, membership based websites~\cite{Ribeiro2014}, and online social networks~\cite{Backstrom2006, kairam2012, zang2016}, it would be useful to introduce economic user growth models that complement our proposed content generation models. Specifically, there is a need to develop resource-based user growth models that account for market health. A potential direction in this research is to extend the Malthusian growth model~\cite{malthus1809}. Second, the proposed production models inherit the fundamental assumptions of macroeconomics: an aggregate is homogeneous (without looking into its internal composition), and aggregates are functionally related etc.~\cite{sims1980}. It would be useful to empirically study these assumptions for real-world knowledge markets.



