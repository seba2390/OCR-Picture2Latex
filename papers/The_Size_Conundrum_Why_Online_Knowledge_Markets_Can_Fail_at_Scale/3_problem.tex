\section{Problem Formulation} 
The goal of this paper is to develop a model for content generation in knowledge markets. Content is integral to the success and failure of a knowledge market. Therefore, we aim to better understand the content generation dynamics.

A model for content dynamics should have the following properties: macro-scale, explanatory, predictive, minimalistic, comprehensive.

\emph{Macro-scale:} The model should capture content generation dynamics via aggregate measures. Aggregate measures help us understand the collective market by summarizing a complex array of information about individuals, which is especially important for policy-making.

\emph{Explanatory:} The model should be insightful about the behavior of a knowledge market. Understanding market behavior is a crucial first step in designing policies to maintain a resilient, sustainable market.

\emph{Predictive:} The model should allow us to make predictions about future content generation and resultant success or failure. These market predictions are integral to the prevention and mitigation of market failures.

\emph{Minimalistic:} The model should have as few parameters as necessary, and still closely reflect the observed reality.

\emph{Comprehensive:} The model should encompass content generation dynamics for different content types (e.g., question, answer, comment) in varieties of knowledge markets. This is important for developing a systematic way to understand the successes and failures of knowledge markets.

In remaining sections we propose models that meet the aforementioned requirements, and show that our best-fit model accurately reflects the content generation dynamics and resultant successes and failures of real-world knowledge markets.
