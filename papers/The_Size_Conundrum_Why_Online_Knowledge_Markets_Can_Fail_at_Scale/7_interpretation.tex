\section{Characterizing Knowledge Markets}
In this section, we characterize the knowledge markets in \CQA{StackExchange}. We explain the best-fit models and their foundations (Section 7.1), reveal two key distributions that control the markets (Section 7.2), and uncover the stable core that maintains market equilibrium (Section 7.3).

\subsection{Model Interpretation} 
First, we explain the best-fit models found in Section 6.1. We observe that content generation in \CQA{StackExchange} markets are best modeled through the combination of power basis and interactive essential interaction. In addition, we found that the best-fit exponents ($\lambda$ parameter in basis $g(x) = ax^\lambda$, where $x$ is a factor) of these models lie between 0 and 1 (inclusive), for all factors of all content types, for all \CQA{StackExchange} markets. 

A model that uses the power basis (where exponents lie between 0 and 1) and interactive essential interaction is known as the Cobb-Douglas production function~\cite{wiki}. In its most standard form for production of a single output $z$ with two inputs $x_1$ and $x_2$, the function is: 
$$z = ax_1^{\lambda_1}x_2^{\lambda_2}.$$
Here, the coefficient $a$ represents the \emph{total factor productivity}---the portion of output not explained by the amount of inputs used in production~\cite{wiki}. As such, its level is determined by how efficiently the inputs are utilized in production. The exponents $\lambda_i$ represent the \emph{output elasticity} of the inputs---the percentage change in output that results from the percentage change in a particular input~\cite{wiki}. 

The Cobb-Douglas function provides intuitive explanation for content generation in \CQA{StackExchange} markets. In particular, the explanation stands on three phenomena or principles: constant elasticity, diminishing returns, and returns to scale.

\textbf{Constant Elasticity.} In \CQA{StackExchange} markets, factors such as user participation and content dependency have \emph{constant elasticity}---percentage increase in any of these inputs will have constant percentage increase in output~\cite{wiki}, as claimed by the corresponding exponents in the model. For example, in \SE{academia} ($N_a = 6.93N_q^{0.18}U_a^{0.65}$), a 1\% increase in number of answerers ($U_a$) leads to a 0.65\% increase in number of answers ($N_a$). 

\textbf{Diminishing Returns.} For a particular factor, when the exponent is less than 1, we observe \emph{diminishing returns}---decrease in the marginal (incremental) output as an input is incrementally increased, while the other inputs are kept constant~\cite{wiki}. This ``law of diminishing returns'' has many interesting implications for the \CQA{StackExchange} markets, including the diminishing benefit of having a new participant in a market. For example, in \SE{academia}, if the number of answerers is 100, then the marginal contribution of a new answerer is $c(101^{0.65} - 100^{0.65}) = 0.129c$, where $c$ is a constant; in contrast, if the number of answerers is 110, then the marginal contribution of a new answerer is $c(111^{0.65} - 110^{0.65}) = 0.125c$. Thus, for answer generation in \SE{academia}, including a participant when the number of participants (system size) is 110 is likely to be less beneficial compared to including a participant when the system size is 100.

\textbf{Returns to scale.} The knowledge markets in \CQA{StackExchange} vary in terms of scale efficiency, as manifested by their \emph{returns to scale}---the increase in output resulting from a proportionate increase in all inputs~\cite{wiki}. If a market has high returns to scale, then greater efficiency is obtained as the market moves from small- to large-scale operations. For example, in \SE{academia}, for answer generation, the returns to scale is $0.18+0.65=0.83<1$. The market becomes less efficient as answer generation is expanded, requiring more questions and answerers to increase the number of answers by same amount. 

\subsection{Two Key Distributions} 
Next, we discuss two key distributions that control content generation in knowledge markets, namely participant activity and subject POV (perspective). These two distributions induce the three phenomena reported in section 7.1.

\textbf{Participant Activity.} The distribution of participant activities implicitly drives a market's return in terms of user participation, as manifested by the corresponding exponent. For example, in a hypothetical knowledge market where each answerer contributes equally, the answer generation model should be $N_a = AN_q^{\lambda_1}U_a^{1.0}$. In reality, the distribution of participant activities is a size dependent distribution controlled by the number of participants (system size). As the system size increases, most participants contribute to the head of the distribution (few activities), whereas very few join the tail (many activities). 

\begin{figure}[b]
\centering
\includegraphics[scale=0.39]{Figures/Size_Dependent_Distribution.pdf}
\caption{The visibility of size dependent distribution: strong---\SE{android}; moderate---\SE{apple}; and weak---\SE{biology}. In most \CQA{StackExchange} markets, the power-law exponent decreases with system size, similar to \SE{android}. In other markets, there exists a non-zero correlation between system size and power-law exponent.}
\label{fig:sdd}
\end{figure}

We systematically reveal the size dependent distribution for participant activities in three steps. First, we empirically a fit power-law distribution to the activities of participants in a month, for each month, for each \CQA{StackExchange} market. We follow the standard procedure to fit a power-law distribution~\cite{adamic2000zipf}. We observe that the power-law well describes the monthly activity distributions. Second, we plot the exponents of the power-law against the number of participants for all observed months in a market, for each market in \CQA{StackExchange}. We observe that for most \CQA{StackExchange} markets, the power-law exponent decreases as the system size increases. Third, we apply linear regression to reveal the relationship between power-law exponent and system size. We observe that in general power-law exponent is negatively correlated with system size. This negative correlation is strongly visible in big knowledge markets that have at least 500 monthly participants in each month.

In Figure~\ref{fig:sdd} we present empirical evidence of the size dependent distribution for answer generation in three \CQA{StackExchange} markets: \SE{android}, \SE{apple}, and \SE{biology}. We choose these examples to cover three possible visibilities of the size dependent distribution, as manifested by the correlation between 
the power-law exponent and system size---strong correlation ($|r^2|\geq 0.5$), moderate correlation ($0.3\leq |r^2|<0.5$), and weak correlation ($|r^2|<0.3$).

\textbf{Subject POV.} The distribution of subject POV implicitly drives a market's return in terms of content dependency, as manifested by the corresponding exponent. Subject POV refers to the number of distinct perspectives on a particular content (e.g., questions) that imposes a conceptual limit to the number of dependent contents (e.g., answers). For example, an open-ended question such as \lq What's your favorite book?\rq\ has many possible answers, whereas a close-ended question such as \lq What's the solution for 3x+5 = 2?\rq\ has a single correct answer. In reality, most questions are neither completely open-ended nor completely closed; however, from an answerer's perspective, there's a diminishing utility in answering a question that already has an answer. This diminishing utility varies from question to question---questions asking for recommendations attract many answers, whereas questions seeking factual information attract few answers. 

\subsection{Uncovering the Stable Core} 
We uncover a stable user community in each \CQA{StackExchange} market, that maintains the \emph{dynamic equilibria}---the increase or decrease in overall user community does not affect the Cobb-Douglas models. We assert that this stable user community generates a large fraction of high-threshold contents that require more effort, e.g, answers and comments, whereas the remaining users are unstable and contribute a small fraction of high-threshold contents.

\begin{figure}[t]
\centering
\includegraphics[scale=0.45]{Figures/Stable_Core.pdf}
\caption{The distribution of monthly answer contribution of users with different \# of active months, for \SE{android}. The users who contribute for many months also contribute a large number of answers.}
\label{fig:stable_core}
\end{figure}

We reveal the presence of the stable core by summarizing the answer contribution of users with different tenure levels (\# of active months). First, for each \CQA{StackExchange} market, we apply equal-width binning to categorize its users into five tenure levels. Then, we plot the distribution of monthly answer contribution by the users of each category using a letter-value plot. We present the letter-value plots for \SE{android} in Figure~\ref{fig:stable_core}. We observe that monthly answer contribution is an increasing function of tenure level---the users who contribute for many months also contribute a large number of answers.


