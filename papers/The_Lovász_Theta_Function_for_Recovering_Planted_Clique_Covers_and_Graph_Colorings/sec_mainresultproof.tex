\section{Exact recovery for graphs with the $\cscc$ proeprty} \label{sec:proof-deterministicrecovery}

In this section, let $G$ be a graph that satisfies the $\cscc$ property.  Our goal is to extend the results in Section \ref{sec:disjointunion}, and show that $(\ac,\ks)$ is also the unique solution to \eqref{eq:lovasz_lambdamax} for graphs satisfying the $\cscc$ condition, provided $c$ is appropriately small.  Following our discussion in Section \ref{sec:disjointunion}, the matrix $\xc$ remains an extreme point of the feasible region of \eqref{eq:lovasz_lambdamax} if $c<1$.  Hence, by Theorem \ref{thm:constraintqualification}, it remains to produce a suitable dual witness $Z$ satisfying the requirements listed in Theorem \ref{thm:constraintqualification}.




\subsection{Incoherence-type result} \label{sec:incoherence}

The certificate we will use  for graphs with the  $\cscc$ property is   $\zc = P_{\tilde{\mathcal{K}} \cap \tilde{\mathcal{L}}} (\zz)$, defined as  the projection of $\zz$ onto $  {\tilde{\mathcal{K}} \cap \tilde{\mathcal{L}}}$ with respect to the Frobenius norm, i.e., 
$$\zc ~~:=~~ \underset{X}{\arg \min} ~\| X - \zz \|_F^2 \quad \mathrm{s.t.} \quad X \in \tilde{\mathcal{K}} \cap \tilde{\mathcal{L}}.$$
Using the first-order optimality conditions we have that 
\begin{equation} 
\zz - \zc = \tilde{L}  + \tilde{K},
\end{equation}
where $\tilde{L}$ lies in the normal cone of $\mathcal{L}$ at $Z'$ and $\tilde{K}$ lies in the normal cone of $\mathcal{K}$ at $Z'$. Recalling that the normal cone of a subspace is its orthogonal complement we have that $\tilde{L}\in \tilde{\mathcal{L}}^\perp$ and $\tilde{K}\in \tilde{\mathcal{K}}^\perp$. 

Our main result  can be viewed as the extension of orthogonality for graphs that are disjoint unions of cliques to graphs with the $\cscc$ property.  The proof is broken up into a sequence of results that follow.


\begin{theorem} 
 \label{thm:incoherence_1}
Let $G$ be a graph satisfying the $\cscc$ property.  Then, 
\begin{itemize}
\item[$(i)$] For any  $\tilde{K} \in \tilde{\mathcal{K}}^\perp$ and  $\tilde{L} \in \tilde{\mathcal{L}}^\perp$ we have   
$$
| \langle \tilde{K}, \tilde{L} \rangle | \leq 2\sqrt{c}\|\tilde{K}\|_F  \|\tilde{L}\|_F .
$$
\item[$(ii)$]  For any  $\tilde{L} \in \tilde{\mathcal{L}}^\perp$ we have 
$$\|P_{\mathcal{K}^\perp}(\tilde{L})\|_F \leq (2 \sqrt{c})^{1/2} \|\tilde{L}\|_F.$$
\end{itemize}
\end{theorem}


\paragraph{A direct sum decomposition for  $\tilde{\mathcal{L}}^\perp$.}  The first  step is to provide a decomposition of the space $\tilde{\mathcal{L}}^\perp$ into simpler subspaces, on which it is easier to prove the near orthogonality property.  We use these results as basic ingredients to build up to our near orthogonality property later.


Define the matrices $\{ F_{x,y,z} : 1 \leq x \neq y \leq k, 1\leq z \leq |\ccs_x | \}$ so that (i) the $(\ccs_x,\ccs_x)$-th block is a diagonal matrix whose $z$-th entry is set equal to $-2(|\ccs_x|-1)$ and all remaining entries equal to $2$, and (ii) the $(\ccs_x,\ccs_y)$-th block ($(\ccs_y,\ccs_x)$-th block) is such that entries in the $z$-th row (column) are equal to $|\ccs_x|-1$ and all other entries equal to $-1$, (iii) all other entries are zero. As an example, in the case where  $k=2$ the matrix  $F_{1,2,1} $ is given by:
 %$$
$$F_{1,2,1} := \left( \begin{array}{cccc|ccc}
-2(|\ccs_1|-1) &&&& |\ccs_1|-1 & \ldots & |\ccs_1|-1  \\
& 2 &&& -1 & \ldots & -1  \\
&& \ddots &&\vdots &&\vdots  \\
&&&2& -1 & \ldots & -1  \\
\hline
|\ccs_1|-1 & -1 & \ldots & -1 &&& \\
\vdots & \vdots & & \vdots &&& \\
|\ccs_1|-1 & -1 & \ldots & -1 &&& \\
\end{array} \right),
$$
where omitted entries are zero. 
%the $(x-1)n+z$-th diagonal entry is set to $-2(n-1)$, (ii) the $(x-1)n+z'$-th diagonal entry, where $1\leq z' \leq n$, $z' \neq z$, is set to $2$, (iii) the $((x-1)n + z, (y-1)n + 1)$ to the $((x-1)n + z, yn)$ row entries as well as the $((y-1)n + 1, (x-1)n + z)$ to the $(yn, (x-1)n + z)$ column entries are set to $n-1$, and (iv) all remaining entries in the $(x,y)$-th and the $(y,x)$-th block are set to $-1$s.
Second, consider the following subspaces:\medskip 


\begin{table}[h]
\centering
\begin{tabular}{|c|c|c|}
\hline
Name & Description & Dimension \\
\hline \hline
$\mathcal{T}_{1}$ & $ \left \{ \left( \begin{array}{c|c|c|c|c} \gamma_1 I & \theta_{1,2} E & \theta_{1,3} E & \ldots & \theta_{1,n} E \\ \hline \theta_{2,1} E & \gamma_2 I & \theta_{2,3} E & \ldots & \theta_{2,n} E \\ \hline \theta_{3,1} E & \theta_{3,2} E & & & \\ \hline \vdots & \vdots & & & \\ \hline \theta_{n,1} E & \theta_{n,2} E & & & \gamma_n I \end{array}\right) : \sum \gamma_i + \sum \theta_{i,j} = 0 \right\}$ & ${k+1 \choose 2}-1$ \\ \hline
%$\mathcal{L}_{N} $ & $ \left\{ \left( \begin{array}{ccccc} &*&* & \ldots \\ * & & * & \ldots \\ *&*&& \\ \vdots & \vdots & &  \end{array}\right) : * \in \mathcal{N}_{C+R} \right\}$ & $k \times (k-2) \times (n-1)$ \\ \hline
$\mathcal{T}_{2} $ & $ \mathrm{Span} \left\{ F_{x,y,z} : 1 \leq x \neq y \leq k, 1\leq z \leq n \right\}$ & $(|V| - k) \times (k-2)$ \\ \hline
\end{tabular}
\end{table}

\begin{proposition} \label{thm:description_of_lperp} We have that 
$$\tilde{\mathcal{L}}^\perp=\mathcal{T}_{1}\oplus \mathcal{T}_{2}.$$
 
\end{proposition}

%Note: Actual rank is $nk(k-1) - (k-1)(k-2)/2$.. verified using matlab

\begin{proof}[Proof of Proposition \ref{thm:description_of_lperp}]
The fact that   $\mathcal{T}_{1}$ and $\mathcal{T}_{2}$ are  orthogonal is easy to check. Define the matrix $L_{x,y,z} \in \mathbb{R}^{|\V|\times|\V|}$, where $1 \leq x \neq y \leq k$, and $1 \leq z \leq |\ccs_x|$ such that (i) the $z$-th entry of the $(\ccs_x,\ccs_x)$-th block is equal to $2$, and (ii) the entries of the $z$-th row (column) of the $(\ccs_x,\ccs_y)$-th block ($(\ccs_y,\ccs_x)$-th block) are all equal to $-1$, and (iii) all other entries are $0$.

As an example, in the case of two cliques (so $k=2$)  the  matrix $L_{1,2,1} $ is given by:
$$
L_{1,2,1} := \left( \begin{array}{cc|ccc} 
2 & & -1 & \ldots & -1  \\
& & & &  \\
\hline
-1 & & & &   \\
\vdots & & & &  \\
-1 & & & &   \\
\end{array}\right),
$$
where all omitted entries are zero. 

%, as the matrix where the $(x-1)n + z$-th diagonal entry is set equal to $2$, while the $((x-1)n + z, (y-1)n + 1)$ to the $((x-1)n + z, yn)$ row entries as well as the $((y-1)n + 1, (x-1)n + z)$ to the $(yn, (x-1)n + z)$ column entries are set to $-1$; all other entries are set to zero.  
First, observe that the matrices $\{ L_{x,y,z} \}$ specify all the linear equalities in the subspace $\tilde{\mathcal{L}}$, and thus, $\tilde{\mathcal{L}}^\perp$ is precisely the span of $\{ L_{x,y,z} \}$.  As such, to show that   $\mathcal{T}_{1}$ and $\mathcal{T}_{2}$  span $\tilde{\mathcal{L}}^\perp$, it suffices to show that every matrix $L_{x,y,z}$ is expressible as the sum of matrices belonging to each of these subspaces.  
As a concrete example, we show this is true for   $L_{1,2,1}$ -- the construction for other matrices are similar.
Indeed, one checks that $L_{1,2,1}$ is the linear sum of these matrices
$$ 
L_{1,2,1}= \underbrace{\frac{1}{|\ccs_1|} \left( \begin{array}{c|c|c|c} 2I & -E & 0 & \ldots \\ \hline -E & 0 & \ldots & \\ \hline 0 & \vdots & \ddots & \\ \hline \vdots & & & \end{array}\right) }_{\mathcal{T}_1}
 -\frac{1}{|\ccs_1|} \underbrace{F_{1,2,1}}_{\mathcal{T}_2}.$$
This completes the proof.
%The subspace $\mathcal{L}^\perp$ is define by matrices of the form
\end{proof}

In view of Proposition \ref{thm:description_of_lperp}, to bound the inner product between vectors  in $\tilde{\mathcal{K}}^\perp$ and $\tilde{\mathcal{L}}^\perp$ respectively, it suffices to bound inner product between $ (i) \ \tilde{\mathcal{K}}^\perp$  and $\mathcal{T}_1$ and $ (ii) \ \tilde{\mathcal{K}}^\perp$  and $\mathcal{T}_2$.


Next, we describe a result that shows how incoherence computation for (a small number of) orthogonal subspaces can be put together to obtain incoherence computations.  %In the following, $\mathcal{S}$ and $\mathcal{T}$ are generic subspaces.

\begin{lemma}\label{thm:orthogonalsubspace_incoherence}  Let  $\{ \mathcal{T}_i \}_{i=1}^{r}$ be orthogonal subspaces of $\R^d$. Consider $s\in \R^d$ such that 
%Then, 
%$\mathcal{T}$ such that $\mathcal{T} = \oplus \mathcal{T}_i$.  Suppose for each $i$ we have 
$$
| \langle s,t_i \rangle | \leq \epsilon \| s \|_2 \| t_i \|_2 \quad \text{ for all }  t_i \in \mathcal{T}_i.
$$
Then, for $t\in \oplus_i \mathcal{T}_i $ we have that 
$$
| \langle s,t \rangle | \leq \epsilon \sqrt{r}  \| s \|_2 \| t \|_2.
$$
\end{lemma}

\begin{proof}[Proof of Lemma \ref{thm:orthogonalsubspace_incoherence}]
For any  $t= \sum t_i\in \oplus \mathcal{T}_i $
%Let $t \in \mathcal{T}$, and write $t_i = P_{\mathcal{T}_i}(t)$.  %Without loss of generality, we may assume that $\|s\|_2=1$.  
%Note that since $\mathcal{T} = \oplus \mathcal{T}_i$ we have $t = \sum t_i$.  
we have 
$$| \langle s, t \rangle | = | \langle s, \sum_{i=1}^r t_i \rangle | \leq \sum_{i=1}^r | \langle s, t_i \rangle | \leq \epsilon \sum_{i=1}^r \| s \|_2 \| t_i \|_2 \leq \epsilon \sqrt{r} \|s\|_2(\sum_{i=1}^r \| t_i \|_2^2)^{1/2} = \epsilon\sqrt{r}\|s\|_2  \| t \|_2,$$
where   the second last inequality follows from Cauchy-Schwarz, while the last equality uses the fact that $\mathcal{T} = \oplus \mathcal{T}_i$.
\end{proof}


%\textbf{Some prepartory results.}  
\paragraph{Bounding inner product between $\tilde{\mathcal{K}}^\perp$ and $\mathcal{T}_2$.}  Define the column vectors $\{\bf_i\}_{i=1}^{n}$ by
$$
\bf_i = n \be_i - \be = (\ldots, \underbrace{-1}_{i-1},\underbrace{n-1}_{i}, \underbrace{-1}_{i+1},\ldots)^T \in \mathbb{R}^{n}.
$$
Note that 
$$ \bf_i \be^T=\begin{pmatrix} f_i & \ldots & f_i\end{pmatrix} \text{ and } \be \bf_i^T=\begin{pmatrix} f_i^T \\ \vdots \\ f_i^T\end{pmatrix}.$$
With a slight abuse of notation, we define the subspace of matrices in $\mathbb{R}^{n_1 \times n_2}$ 
$$
\mathcal{N}_{C} ={\rm span}( \bf_1 \be^T,\ldots,  \bf_n \be^T ) \qquad \text{ and }  \qquad \mathcal{N}_{R}={\rm span}( \be \bf_1^T, \ldots, \be \bf_n^T ).
$$
The abuse of notation arises the vectors $\bf_i$ in the definitions of $\mathcal{N}_{C}$ and $\mathcal{N}_{R}$ are different.  Note that because $\bf_i^T \be = 0$, the subspaces $\mathcal{N}_{C}$ and $\mathcal{N}_{R}$ have dimensions $n_1-1$ and $n_2-1$ respectively and are orthogonal.  
%We define 
%$$
%\mathcal{N}_{C+R} = \mathcal{N}_{C} \oplus \mathcal{N}_{R}.
%$$
%
$\mathcal{N}_{C}$ and $\mathcal{N}_{R}$ are relevant for our problem as   the block off-diagonal entries of any matrix in  $\mathcal{T}_{2}$ belong to %$\mathcal{N}_{C+R}$.
$ \mathcal{N}_{C} \oplus \mathcal{N}_{R}.$

\begin{lemma} \label{thm:ncnr_space_incoherence}
Let $K \in \mathbb{R}^{n_1\times n_2}$  such that each row  has at most $c n_2$ non-zero entries for some  $0 \leq c \leq 1$. Then, for any $L \in \mathcal{N}_{C}$   we have that 
$$| \langle K, L \rangle | \leq \sqrt{c} \| L \|_F \| K \|_F.$$
 The same conclusion holds  if each column  of $K$ has at most $c n_1$ non-zero entries and   $L\in \mathcal{N}_R$. 
\end{lemma}

\begin{proof} [Proof of Lemma \ref{thm:ncnr_space_incoherence}]
Since $L \in \mathcal{N}_{C}$, it has constant columns.  Suppose its column entries are $(\theta_1,\ldots,\theta_{n_1})$, i.e., $L_{x,y}=\theta_x$  for all $y$. We then have 
$$\langle K, L \rangle = \mathrm{tr}(KL^T) = \sum_{x} (\sum_{y} K_{x,y} L_{x,y}) = \sum_{x} \theta_x \sum_{y} K_{x,y} {\leq} \sum_{x} \theta_x \sqrt{cn_2} (\sum_{y} K_{x,y}^2)^{1/2},$$  where the last inequality follows from  Cauchy-Schwarz, and since   there are at most $c n_2$ non-zero entries in each column of $K$.  Finally, we  have 
$$\sqrt{cn_2} \sum_{x} \theta_x  (\sum_{y} K_{x,y}^2)^{1/2} \leq \sqrt{cn_2} (\sum_x \theta_x^2)^{1/2} (\sum_{x,y} K_{x,y}^2)^{1/2} = \sqrt{c} \|L\|_F \|K\|_F,$$
 where in the last equality we use  that $\|L\|_F = \sqrt{n_2}(\sum \theta_i^2)^{1/2}$. % Essentially the same proof gives us the result for $\mathcal{N}_{R}$.
\end{proof}



\begin{corollary} \label{thm:ncnr_combined}
Consider   $K \in \mathbb{R}^{n_1\times n_2}$ where  each column has at most $c n_1$ non-zero entries, and each row has at most $c n_2$ non-zero entries for some  $0 \leq c \leq 1$.  For any  $L \in \mathcal{N}_{C}\oplus \mathcal{N}_{R}$  we have: 
$$| \langle K, L \rangle | \leq \sqrt{2c} \| L \|_F \| K \|_F.$$
\end{corollary}

\begin{proof} [Proof of Corollary \ref{thm:ncnr_combined}]
 By Lemma \ref{thm:ncnr_space_incoherence} for  $ L \in \mathcal{N}_C,$ or $ L\in  \mathcal{N}_R$ we have that
$$| \langle K, L \rangle | \leq \sqrt{c} \| L \|_F \| K \|_F.$$
As $\be^T \bf_j=0$,  $\mathcal{N}_{C}$ and $\mathcal{N}_{R}$ are orthogonal subspaces. 
The result follows by Lemma \ref{thm:orthogonalsubspace_incoherence}.
\end{proof}

\begin{lemma} \label{thm:incoherence_efsum}
Suppose $G$ is a graph satisfying the $\cscc$ property.  %Let $L \in \mathbb{R}^{|\V| \times |\V|}$ be a matrix so that every $(\ccs_x,\ccs_y)$-th block, $x\neq y$, belongs to $\mathcal{N}_{C} \oplus \mathcal{N}_{R}$.  
Then, for any  $K \in \tilde{\mathcal{K}}^\perp$ and $L\in \mathcal{T}_2$ we  have that
$$| \langle K,L \rangle| \leq \sqrt{2c} \|K\|_F \|L\|_F.$$
\end{lemma}

\begin{proof} As $K \in \tilde{\mathcal{K}}^\perp$, its diagonal blocks are zero, and  also, entries corresponding to non-edges of $G$ are zero.  Moreover, as $G$ has the   $c$-SCC property, each  row (and column) of $K$  has at most $cn$ non-zero entries.
Let the block matrices be indexed by $(x,y)$, and let $L_{xy}$ and $K_{xy}$ denote the $xy$-th block.  Recall that  the block off-diagonal entries of any matrix in  $\mathcal{T}_{2}$ belong to %$\mathcal{N}_{C+R}$.
$ \mathcal{N}_{C} \oplus \mathcal{N}_{R}.$ By Corollary~\ref{thm:ncnr_combined} we have $| \langle K_{xy}, L_{xy} \rangle | \leq \sqrt{2c} \| L_{xy} \|_F \| K_{xy} \|_F$.  By summing over the blocks and by applying Cauchy-Schwarz, we have 
$$
 \begin{aligned}
  | \langle K , L \rangle | &  \leq \sum_{x,y} |\langle K_{xy} , L_{xy} \rangle |= \sum_{x\ne  y} |\langle K_{xy} , L_{xy} \rangle | \\
& \leq \sqrt{2c} \sum_{x\ne y} \| K_{xy} \|_F \| L_{xy} \|_F \leq \sqrt{2c} (\sum_{x,y} \| K_{xy} \|_F^2)^{1/2} (\sum_{x,y} \| L_{xy} \|_F^2)^{1/2} = \sqrt{2c} \| K \|_F \| L \|_F.
\end{aligned} $$
\end{proof}

\paragraph{Bounding  the inner product between $\tilde{\mathcal{K}}^\perp$ and $\mathcal{T}_1$.}  This case is easier and the required result is given in the next lemma. 
\begin{lemma}\label{thm:incoherence_Espan}
Suppose $G$ is a graph satisfying the $\cscc$ property.  Then, for any  $K \in \tilde{\mathcal{K}}^\perp$ and $L\in \mathcal{T}_1$ we  have that
$$|\langle K, L \rangle | \leq \sqrt{c} \| K \|_F \| L \|_F .$$
%
%Let $K \in \tilde{\mathcal{K}}^\perp$, and let $L$ be a matrix of the form
%$$L = \left( \begin{array}{c|c|c|c|c} * & \theta_{1,2} E & \theta_{1,3} E & \ldots & \theta_{1,n} E \\ \hline \theta_{2,1} E & * & \theta_{2,3} E & \ldots & \theta_{2,n} E \\ \hline \theta_{3,1} E & \theta_{3,2} E & & & \\ \hline \vdots & \vdots & & & \\ \hline \theta_{n,1} E & \theta_{n,2} E & & & * \end{array}\right).$$
%Then 
%The entries marked by an asterisk $*$ are permitted to be arbitrary.
\end{lemma}

\begin{proof}[Proof of Lemma \ref{thm:incoherence_Espan}]
Note that $K$ has no entries in each block diagonal.  Consider the $(i,j)$-th block matrix where $i \neq j$.  We denote the coordinates in this block by $\mathcal{B}_{i,j}$.  Then
$$
\sum_{x,y \in \mathcal{B}_{i,j}} K_{x,y} L_{x,y} = \theta_{i,j} \left( \sum_{x,y \in \mathcal{B}_{i,j}} K_{x,y} \right) \leq \sqrt{c |\ccs_i| |\ccs_j|} \theta_{i,j} \left( \sum_{x,y \in \mathcal{B}_{i,j}} K_{x,y}^2 \right)^{1/2}.
$$
The last inequality follows from Cauchy-Schwarz, and by noting that $K$ has at most $c |\ccs_i| |\ccs_j|$ non-zero entries within the block $\mathcal{B}_{i,j}$.  Then by summing over the blocks $(i,j)$ we have
$$
|\langle K,L \rangle| \leq \sum_{i,j} \sqrt{c|\ccs_i| |\ccs_j|} \theta_{i,j} ( \sum_{x,y \in \mathcal{B}_{i,j}} K_{x,y}^2 )^{1/2} \leq \sqrt{c} (\sum_{i,j} |\ccs_i| |\ccs_j|\theta_{i,j}^2)^{1/2} ( \sum_{x,y} K_{x,y}^2 )^{1/2}.
$$
The last inequality follows from Cauchy-Schwarz.  Now note that $( \sum_{x,y} K_{x,y}^2 )^{1/2} = \|K\|_F$, and that $(\sum_{i,j} |\ccs_i| |\ccs_j| \theta_{i,j}^2)^{1/2} = \|L\|_F $, from which the result follows.
\end{proof}


%\subsection{Main incoherence result}  This is the main incoherence result we need.

%\begin{proposition} 
% \label{thm:incoherence_1}
%Let $G$ be a graph satisfying the $c$-neighborly property.  For any  $\tilde{K} \in \mathcal{K}^\perp$ and  $\tilde{L} \in \mathcal{L}^\perp$ we have that  
%$$
%| \langle \tilde{K}, \tilde{L} \rangle | \leq 2\sqrt{c}\|\tilde{K}\|_F  \|\tilde{L}\|_F .
%$$
%\end{proposition}



Finally, we are ready to prove  Theorem \ref{thm:incoherence_1}.


\paragraph{Proof of Theorem \ref{thm:incoherence_1}.} Consider   $\tilde{K} \in \tilde{\mathcal{K}}^\perp$ and  $\tilde{L} \in \tilde{\mathcal{L}}^\perp$.   By Proposition \ref{thm:description_of_lperp} we have that 
$\tilde{\mathcal{L}}^\perp=\mathcal{T}_1\oplus \mathcal{T}_2$ and let   $\tilde{L} =\tilde{L}_1+\tilde{L}_2$, where $\tilde{L}_i\in \mathcal{T}_i$. 


\medskip 
    \textbf{Part $(i)$.} 
By % Note that the linear span of matrices in $\mathcal{T}_1$ have the form in
    Lemma \ref{thm:incoherence_Espan} we have that  
$$| \langle \tilde{K}, \tilde{L}_1 \rangle | \leq \sqrt{c}\|\tilde{K}\|_F  \|\tilde{L}_1 \|_F.$$ 

%Let $\tilde{L}_2 \in \mathcal{T}_2$ be arbitrary. 
 Let $\tilde{L}_{2,\mathrm{off}}$ be the block off-diagonal component of $\tilde{L}_2$ and $\tilde{L}_{2,\mathrm{diag}}$ be the block diagonal component of $\tilde{L}_2$ so that $\tilde{L}_2 = \tilde{L}_{2,\mathrm{off}} + \tilde{L}_{2,\mathrm{diag}}$.  
By definition, the matrix $\tilde{L}_{2,\mathrm{off}}$ is zero on the diagonal  blocks  and  each block belongs to $\mathcal{N}_{C} \oplus \mathcal{N}_{R}$.  Hence by Lemma \ref{thm:incoherence_efsum}, we have 
$$| \langle \tilde{K}, \tilde{L}_{2,\mathrm{off}} \rangle | \leq \sqrt{2c} \| \tilde{K} \|_F \| \tilde{L}_{2,\mathrm{off}} \|_F.$$
Moreover, as the diagonal blocks of $\tilde{K}$ are zero we  have that 
$$\langle \tilde{K}, \tilde{L}_{2,\mathrm{diag}} \rangle =0.$$
Putting everything together,
$$| \langle \tilde{K}, \tilde{L}_2 \rangle | = | \langle K, \tilde{L}_{2,\mathrm{off}} \rangle | \leq \sqrt{2c}  \| \tilde{K} \|_F \| \tilde{L}_{\mathrm{2,off}} \|_F \leq \sqrt{2c} \| \tilde{K} \|_F \| \tilde{L} \|_F,$$   
where the last inequality follows by noting that $\| L \|_F^2 = \| L_{\mathrm{off}} \|_F^2 + \| L_{\mathrm{diag}} \|_F^2$.

Finally, by Proposition \ref{thm:description_of_lperp}, the subspaces $\mathcal{T}_1$ and $\mathcal{T}_2$ are orthogonal.  Hence by Lemma \ref{thm:orthogonalsubspace_incoherence} applied to the subspaces $\mathcal{T}_1$ and $\mathcal{T}_2$ we have $| \langle \tilde{K}, \tilde{L} \rangle | \leq \sqrt{2} \times \sqrt{2c} \|\tilde{K}\|_F \|\tilde{L}\|_F$.
%Let $\tilde{L} = \sum \theta_{x,y} L_{x,y}$ where $\{ L_{x,y} \}$ are basis matrices specified in [REF].  Fix an index $(x,y)$, and observe that
%$$\langle \tilde{K}, \tilde{L}_{x,y} \rangle = \mathrm{tr}(\tilde{K} \tilde{L}_{x,y}) = 2 \sum_{t=1}^{n} \tilde{K}_{x,(y-1)n+t} \leq  2 \sqrt{s} (\sum_{t=1}^{n} \tilde{K}_{x,(y-1)n+t}^2 )^{1/2}.$$ 
%We obtain the second equality by noting that the entries of $K$ are permitted to be non-zero only when the corresponding entries of $L$ are $1$.  We obtain the factor $2$ by accounting for the column sum.  Also, we obtain the last inequality by the Cauchy-Schwarz, and by noting that there are at most $s$ instances $\tilde{K}_{x,(y-1)n+t} \neq 0$ in the range $1\leq t \leq n$ by the ABCD-$s$ property.  
%Now, we perform the sum over all indices $(x,y)$.  We have
%\begin{equation*}
%\begin{aligned}
%| \langle \tilde{K}, \sum_{x,y} \theta_{x,y} L_{x,y} \rangle | ~\leq~ & 2 \sqrt{s} \sum_{x,y} |\theta_{x,y}| (\sum_{t=1}^{n} \tilde{K}_{x,(y-1)n+t}^2 )^{1/2} \\
%~\leq~ & 2 \sqrt{s} \big( \sum_{x,y} \theta_{x,y}^2  \big)^{1/2} \big( \sum_{x,y} \sum_{t=1}^{n} \tilde{K}_{x,(y-1)n+t}^2 )^{1/2} \big)^{1/2} = 2 \sqrt{s} \big( \sum_{x,y} \theta_{x,y}^2  \big)^{1/2}  \| \tilde{K} \|_F.
%\end{aligned}
%\end{equation*}
%The last inequality follows by noting that all indices of the matrix $\tilde{K}$ are counted exactly once.  We further note that $\| L_{x,y} \|_F  \geq ??n$.  [THIS IS THE SOURCE OF THE ERROR.  I NEED TO SHOW THAT THE NORM OF THE MATRIX IS BOUNDED BELOW BY $n$]
 % Thus $|\langle \tilde{K}, \tilde{L} \rangle | \leq (2\sqrt{s}/n) \| \tilde{L} \|_F$, as required.
%
%
%\begin{proposition}  \label{thm:incoherence_2}
%Let $G$ be a graph satisfying the $c$-neighborly property.  For any  $\tilde{L} \in \mathcal{L}^\perp$ we have 
% $$\|P_{\mathcal{K}^\perp}(\tilde{L})\|_F \leq (2 \sqrt{c})^{1/2} \|\tilde{L}\|_F.$$
%\end{proposition}
\medskip 

 \textbf{Part $(ii)$.} Note that
$$
\| P_{\mathcal{K}^\perp}(\tilde{L}) \|_F^2 = \langle P_{\mathcal{K}^\perp}(\tilde{L}), P_{\mathcal{K}^\perp}(\tilde{L}) \rangle = \langle P_{\mathcal{K}^\perp} (P_{\mathcal{K}^\perp}(\tilde{L})), \tilde{L} \rangle = \langle P_{\mathcal{K}^\perp}(\tilde{L}), \tilde{L} \rangle,
$$
where the second and third equalities  follow  from the fact that orthogonal projections are  self-adjoint and idempotent. 
 %to thenseand the third equality follows from the fact that $P_{\mathcal{K}^\perp}$ is a projection operator.  
 Since $P_{\mathcal{K}^\perp}(\tilde{L}) \in \mathcal{K}^\perp$, by Proposition \ref{thm:incoherence_1}, we have 
$$|\langle P_{\mathcal{K}^\perp}(\tilde{L}), \tilde{L} \rangle| \leq 2\sqrt{c} \|P_{\mathcal{K}^\perp}(\tilde{L})\|_F \|\tilde{L}\|_F \leq 2\sqrt{c} \|\tilde{L}\|_F^2,$$ from which the result follows.\qed
%Let $\tilde{L} = \sum_{x,y} \theta_{x,y} L_{x,y}$.  Consider each row..  For each fixed $\theta$, we have $\sum \tilde{L}^2 \leq s \theta^2$, because each entry of $L$ is equal to one, and there are at most $s$ non-zero entries.  Then summing across all $\theta$ we have $\|\tilde{L}\|_F^2 \leq s \sum \theta^2 \leq (s/n^2) \| \tilde{L} \|_F^2$.


%}

