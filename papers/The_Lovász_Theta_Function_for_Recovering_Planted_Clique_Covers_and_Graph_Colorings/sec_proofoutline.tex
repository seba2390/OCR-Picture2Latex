\subsection{Proof Outline} \label{sec:outline}

In this section, we give an outline of the proof of our main results.  The key part of our argument is to produce a suitable dual optimal solution that certifies the optimality as well as the uniqueness of $(\ac,\ks)$ in \eqref{eq:lovasz_lambdamax}.  First, as a reminder, the dual program to \eqref{eq:lovasz_lambdamax} is the following semidefinite~program
\beq\label{eq:dual}\tag{D}
\max \Big\{ \, \langle J , Z \rangle \,:\, \mathrm{tr}(Z) = 1, Z_{i,j} = 0 \text{ for all } {(i,j) \in \mathcal{E}}, \ Z \succeq 0  \Big\}.
\eeq

A basic consequence of weak duality is that, in order to show $(\ac,\ks)$ is {\em an} optimal solution to~\eqref{eq:lovasz_lambdamax} (note that this is equivalent to $\vartheta(G) = \ks$), it suffices to produce a dual feasible solution that attains the same objective; i.e., one would need to exhibit $\hat{Z}$ that is a feasible solution to \eqref{eq:dual} and satisfies $\langle J, \hat{Z} \rangle = \ks$. To show that $(\ac,\ks)$ is the {\em unique} optimal solution to \eqref{eq:lovasz_lambdamax},  it is necessary to appeal to a strict complementarity-type of result for SDPs; see, e.g., \cite{overton,LV}.  In what follows, we specialize a set of conditions from \cite{LV} that guarantee unique solutions in SDPs to the Lov\'asz theta function.

\begin{theorem}\label{thm:constraintqualification}
Let $(t,A)$ and ${Z}$ be a pair of strict complementary primal and dual optimal solutions to \eqref{eq:lovasz_lambdamax} and \eqref{eq:dual} respectively, i.e., they satisfy:
 %Suppose these evaluate to the same objective value.  Furthermore, suppose that
$$\la X , Z \ra = 0 \quad \text{ and } \quad \mathrm{rank}(X) + \mathrm{rank}(Z) = |\V|, \qquad \text{ where } \qquad X = tI + A - J. $$  
Then $A$ is the unique primal optimal solution to \eqref{eq:lovasz_lambdamax} if and only if $A$ is an extreme point of the feasible region of \eqref{eq:lovasz_lambdamax}.
\end{theorem}
In view of Theorem \ref{thm:constraintqualification}, to show that $(\ac,\ks)$ is the unique optimal solution of \eqref{eq:lovasz_lambdamax}, it suffices to $(i)$ show that $(\ac,\ks)$ is an extreme point of the feasible region of the feasible region of \eqref{eq:lovasz_lambdamax}, and $(ii) $ produce a dual optimal $\hat{Z} \in \mathbb{R}^{|V| \times |V|}$ that satisfies strict complementarity.

The answer to the first task has simple answer: in Section \ref{sec:extremality}, we show that $(\ac,\ks)$ is an extreme point of the feasible region of \eqref{eq:lovasz_lambdamax} whenever the graph $G$ satisfies the $\cscc$ condition for some $c < 1$.  Our proof relies on results that characterize extreme points of spectrahedra, and the specific version we use is found in Corollary 3 of \cite{RG:95}:
\begin{theorem}\label{thm:extremal_rg}
Let $\mathcal{S}$ be a spectrahedron specified in linear matrix inequality(LMI)  form
\begin{equation} \label{eq:LMI}
\mathcal{S}=\Big \{ (x_1, \ldots, x_n)^T \in \R^n : Q_0 +\sum_{i=1}^n x_i Q_i \succeq 0 \Big \},
\end{equation} 
where $Q_0,Q_1,\ldots, Q_n \in \mathbb{R}^{m \times m}$ are symmetric matrices.  Let $\by = (y_1,\ldots, y_n)^T \in \R^n$, and $U$ be an $m\times k$ matrix whose columns span the kernel of the matrix $Q_0 +\sum_i y_i Q_i$. Then, $\by$ is an extreme point of $\mathcal{S}$ if and only if the vectors ${\rm vec}(Q_1U), \ldots, {\rm vec}(Q_nU)$ are linearly independent. 
\end{theorem}


The answer to the second task is considerably more involved.  As a reminder, a matrix $Z \in~\mathbb{R}^{|V| \times |V|}$ is optimal  for \eqref{eq:dual}  if it satisfies:
\begin{align} 
& Z \succeq 0 \label{cond1}, \\    
& Z \in {\rm span}\{E_{i,j}: \ (i,j)\in \E \}^\perp \label{eq:supportcondition}, \\
& \langle Z, \xc \rangle = 0 ~ (\text{i.e. } Z \in {\rm span}(\xc )^\perp)\label{eq:complementaryslackness}, \\ 
& \mathrm{tr}(Z) = 1. \label{eq:traceone}
\end{align}
Here we use the notation $E_{i,j}={1\over 2}(\be_i \be_j^T+\be_j \be_i^T)$.  We call a matrix $Z$ satisfying these conditions a {\em dual certificate}.  Furthermore, we say that a dual certificate  $Z$ and $\xc$ satisfy  {\em strict complementarity}~if:%the following also holds:
\begin{equation}
{\rm rank}(Z)=|\V| - {\rm rank}(\xc) \label{cond3}.
\end{equation}
%Finally, note that a dual certificate is also dual optimal; indeed,  this is true as 
% & \langle J, Z \rangle = \ks \label{eq:objvalks}.


 %In what follows, we focus on producing dual certificates that satisfy the conditions \eqref{cond1}, \eqref{eq:supportcondition},  \eqref{eq:complementaryslackness} and  \eqref{cond3}. This is sufficient because any matrix $Z$ that satisfies these conditions will be non-zero and PSD.  One is able to apply a non-negative scale so that $\mathrm{tr}(Z) = 1$.  %Later, in Section~\ref{sec:extremality} (see proof of Theorem \ref{thm:disjointcliques_uniquerecovery}), we show that such a scaled $Z$ necessarily satisfies \eqref{eq:objvalks} as a consequence of~\eqref{eq:complementaryslackness}.

%the strict complementarity conditions on $Z$ are:
%  The latter set of conditions are:
%\begin{align} 
%& Z \succeq 0 \label{cond1} \\    
%& Z \in {\rm span}\{E_{i,j}: \ (i,j)\in \E \}^\perp \label{eq:supportcondition} \\
%& Z \in {\rm span}(\xc )^\perp \label{eq:complementaryslackness}\\ 
%& {\rm rank}(Z)=|\V| - {\rm rank}(\xc) \label{cond3},
    % &  \zz\in \label{eq:supportcondition} \tag{SUPP},
%\end{align} 

Note that we can omit condition \eqref{eq:traceone} as we can always scale a matrix that satisfies conditions \eqref{cond1}, \eqref{eq:supportcondition},  \eqref{eq:complementaryslackness} and  \eqref{cond3} to make  it trace-one.  The proof our two main results   is given in three steps.

\paragraph{Step 1: Exact recovery in the case of disjoint cliques.}  In the first step, we consider the simplest setting where $G$ is the disjoint union of cliques $\{\ccs_l\}_{l=1}^{\ks}$.  With a bit of guesswork, we construct the following matrix $\zz$ defined as follows:
\begin{equation}\label{zmatrix}
\zz := \left(\begin{array}{ccc}
\frac{I}{|\ccs_1|} & \frac{1}{|\ccs_1||\ccs_2|} E & \ldots \\
\frac{1}{|\ccs_1||\ccs_2|} E & \frac{I}{|\ccs_2|} & \ldots \\
\vdots & \vdots & \ddots
 \end{array}\right).
\end{equation}
Here, the $(i,j)$-th block has dimension $|\ccs_i| \times |\ccs_j|$.

It is relatively straightforward to verify that the conditions \eqref{eq:supportcondition} and \eqref{eq:complementaryslackness} are satisfied.  In  Proposition \ref{thm:Unequalsizecanonicalmatrix_spectrum} we characterize the spectrum of the matrix $\zz$, and in so doing, show that $\zz$ satisfies the conditions \eqref{cond1} and \eqref{cond3}.  We explain these steps in Sections \ref{sec:extremality} and \ref{sec:disjointunion}.

%In this setting, we construct a dual certificate  $\zz$ that satisfies conditions~{\eqref{cond1}--\eqref{cond3}}. By Theorem \ref{thm:constraintqualification}, this certifies  that $\xc$ is the unique optimal solution of \eqref{eq:lovasz_lambdamax}.
%In the case  where $G$ is the disjoint union  of cliques   we can easily see  that $\tilde{\mathcal{K}}^\perp=\{0\}$   so trivially $\tilde{\mathcal{K}}^\perp$ and $\tilde{\mathcal{L}}^\perp $ are orthogonal. Furthermore,  as $\tilde{\mathcal{K}}\cap \tilde{\mathcal{L}}=\tilde{\mathcal{L}}$  it  suffices  to find a matrix $\zz \in \tilde{\mathcal{L}}$ that also satisfies condition~\eqref{cond3}.  The matrix $\zz$ designed to serve as our dual witness is given by:
%   In Proposition \ref{thm:Unequalsizecanonicalmatrix_spectrum}, we characterize the spectrum of the matrix $\zz$, and in so doing, show that $\zz$ satisfies the conditions \eqref{cond1}-\eqref{cond3}.  



\paragraph{Step 2: Exact recovery  for graphs with the $\cscc$ property.}  The matrix $\zz$ that served as our dual certificate for graphs formed by disjoint cliques does not work in the more general setting as it is non-zero on every edge between distinct cliques; i.e., it violates \eqref{eq:supportcondition}.  The key idea for constructing a suitable dual certificate in this case begins by recognizing that conditions \eqref{eq:supportcondition} and~\eqref{eq:complementaryslackness} define subspaces:
$$ \mathcal{K} :={\rm span}\{E_{i,j}: \ (i,j)\in \E \}^\perp  \ \text{ and } \  \mathcal{L} := {\rm span}(\xc)^\perp.
$$
In view of this, we consider a dual certificate $\zc$ obtained by projecting the matrix $\zz$ with respect to the Frobenious norm onto the intersection of these spaces:
$$
\zc ~~:=~~ \underset{X}{\arg \min} ~\| X - \zz \|_F^2 \quad \mathrm{s.t.} \quad X \in \mathcal{K}\cap \mathcal{L}.
$$
The remainder of proof is a matrix pertubation argument in which we show that $\|\zc - \zz \|$ is quite small, which allows us to conclude the remaining conditions \eqref{cond1} and \eqref{cond3}.
We now provide some geometric intuition why  the  projection onto the intersection $\mathcal{K}\cap \mathcal{L}$ succeeds.  

The  graphs considered in this work  are either disjoint union of cliques or  have the $\cscc$ property.  In both cases, we have that $\mathcal{K}$ is a subspace of  
% Thinking of any symmetric matrix $ X \in \mathbb{R}^{|\V| \times |\V|}$ indexed by the vertices of the graph as a block matrix  whose blocks are indexed by cliques $\ccs_l$, it follows that its  diagonal blocks   are diagonal.  Equivalently, $\mathcal{K}$ is a subspace of the linear space:$$
$$\mathcal{M} := \{ X \in \mathbb{R}^{|\V| \times |\V|}: \ X=X^T, (\ccs_l,\ccs_l) \text{-th block is diagonal for all } 1 \leq l \leq k \},
$$
which we   treat  as the ambient space, rather than the space of all symmetric matrices.
 Setting 
 % $\tilde{\mathcal{K}}$ and $\tilde{\mathcal{L}}$ denote the restrictions of $\mathcal{K}$
 %and  $\mathcal{L}$ on $\mathcal{M}$ respectively, i.e., 
$$   \tilde{\mathcal{K}} :={\rm span}\{E_{i,j}: \ (i,j)\in \E \}^\perp \cap \mathcal{M}  \ \text{ and } \  \tilde{\mathcal{L}}:= {\rm span}(\xc)^\perp\cap \mathcal{M},$$
%In other words,  
%Although  $\tilde{\mathcal{K}}=\mathcal{K}$, however  restricting  on $\mathcal{M}$ affects its orthogonal complement. 
%%
%First, we define the following subspace of symmetric matrices
%$$
%\mathcal{M} := \{ X \in \mathbb{R}^{|\V| \times |\V|}: \ X=X^T, (\ccs_l,\ccs_l) \text{-th block is diagonal for all } 1 \leq l \leq \ks \}. 
%$$
%With that, we define the subspaces 
%$$   \tilde{\mathcal{K}} :={\rm span}\{E_{i,j}:  (i,j)\in \E \}^\perp \cap \mathcal{M}  \ \text{ and } \  \tilde{\mathcal{L}}:= {\rm span}(\xc)^\perp\cap \mathcal{M}.
%$$
 $\zc$ can be equivalently defined as the projection of $\zz$ onto the intersection of $\tilde{\mathcal{K}} \cap \tilde{\mathcal{L}}$.
\begin{equation} \label{eq:euclideanprojection}
\zc ~~:=~~ \underset{X}{\arg \min} ~\| X - \zz \|_F^2 \quad \mathrm{s.t.} \quad X \in \tilde{\mathcal{K}} \cap \tilde{\mathcal{L}}.
\end{equation}
 In Section \ref{sec:proof-deterministicrecovery} we give the main technical result of this work, where we establish  that the subspaces $ \tilde{\mathcal{K}}^\perp$ and $ \tilde{\mathcal{L}}^\perp$ are ``almost orthogonal''. For this, it is now crucial that our ambient space is $\mathcal{M}$.

%Throughout this work we consider graphs that are either disjoint union of cliques or that have the $\cscc$ property.  Thinking of any symmetric matrix $ X \in \mathbb{R}^{|\V| \times |\V|}$ indexed by the vertices of the graph as a block matrix  whose blocks are indexed by cliques $\ccs_l$, it follows that its  diagonal blocks   are diagonal.  Equivalently, $\mathcal{K}$ is a subspace of the linear space:

% We restrict our search for a dual certificate in $\tilde{\mathcal{K}}\cap \tilde{\mathcal{L}},$ where  $\tilde{\mathcal{K}}$ and $\tilde{\mathcal{L}}$ denote the restrictions of $\mathcal{K}$
% and  $\mathcal{L}$ on $\mathcal{M}$ respectively, i.e., 
%$$   \tilde{\mathcal{K}} :={\rm span}\{E_{i,j}: \ (i,j)\in \E \}^\perp \cap \mathcal{M}  \ \text{ and } \  \tilde{\mathcal{L}}:= {\rm span}(\xc)^\perp\cap \mathcal{M}.$$
% It is evident  that $\tilde{\mathcal{K}}=\mathcal{K}$, however  restricting  on $\mathcal{M}$ affects its orthogonal complement.  
 

\paragraph{Step 3: Recovery for planted clique cover instances.}  Our third  step shows that planted clique instances are $\cscc$, with high probability.  The proof is a direction application of Hoeffding's inequality combined with a union bound, and is provided in Section \ref{sec:randomgraphs}.




