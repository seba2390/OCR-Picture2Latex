\section{Numerical comparison to alternative  techniques } \label{subsec:related worlk}
In this section, we assess the effectiveness of the Lov\'asz theta function for recovering planted clique cover instances by comparing it to alternative techniques in the literature.  Specifically, we compare our method with the following approaches:

\begin{itemize} 
\item {\em Community Detection approach:}  We frame the clique cover problem as an instance of community detection, and use the method proposed by Oymak and Hassibi~\cite{OH:11}.

\item {\em $k$-Disjoint-Clique approach: } We consider the clique cover problem as an instance of the $k$-disjoint-clique problem, and apply the method presented by Ames and Vavasis \cite{AV:14}.

\item {\em Subgraph Isomorphism approach:}  We view the problem as an instance of  subgraph isomorphism, and apply the framework proposed by Candogan and Chandrasekaran \cite{CC:18}.
\end{itemize}


%In this section we review some prior work related to ours, and we compare the performance of these methods with ours at recovering clique covers.

\paragraph{Community Detection/Graph clustering.}  Consider a network where the presence of an edge (or possibly the size of its edge weight) models the strength of association between two nodes.  The graph clustering task concerns grouping these nodes into subsets -- often called {\em communities} -- such that the strength of association between pairs of nodes from the same subset is significantly stronger that the strength of association between pairs belonging to different subsets.  The task of community detection finds application in various fields such as sociology, physics, epidemiology, and more \cite{For:10,New:06}. The term {\em community detection} is often used interchangeably with {\em graph clustering}. %  In what follows, we use the term `community detection' because the bulk of prior work that is most relevant to ours tend to make this choice.

Oymak and Hassibi \cite{OH:11}  (see also \cite{CJSX:14}) present a  method for  detecting communities in an undirected  graph  $\mathcal{G} = (\mathcal{V},\mathcal{E})$  by solving the 
following  SDP:
 %be an undirected graph, ax. \cite{OH:11} 
\begin{equation} \label{eq:sparselowrank_comdetect}
\begin{aligned}
\min_{S,L} ~~ \lambda \| S \|_{1} +  \| L \|_{*} \quad \mathrm{s.t.} \quad S+L = A, \quad 0 \leq L \leq J.
\end{aligned}
\end{equation}
Here     $A$ is  the adjacency matrix of $G$,  $\lambda>0$ is a tuning parameter, $\| X \|_1 = \sum_{i,j} |X_{i,j}|$ is the L1-norm over the matrix entries, while $\|X\|_* = \sum_i \sigma_i (X)$ is the matrix nuclear norm. %(or the Schatten-1 norm).  

  Let $(\hat{S},\hat{L})$ be the optimal solution to this SDP.   The intuition behind the % formulation 
  SDP \eqref{eq:sparselowrank_comdetect} % proposed by Oymak and Hassibi \cite{OH:11} 
is that it performs a matrix deconvolution in which the adjacency matrix $A$ is separated as the %linear 
sum of a low-rank component $\hat{L}$ and a sparse matrix $\hat{S}$.  %It is hoped that 
The hope is that (possibly by tuning the parameter~$\lambda$) the low-rank component takes the form $\hat{L} = \sum_{l} \bone_{\ccs_l} \bone_{\ccs_l}^T$, where $\ccs_l$ are the indicator vectors for {\em disjoint} communities, while the sparse component $\hat{S}$ captures noise that is (hopefully)~sparse.  

The specific form in \eqref{eq:sparselowrank_comdetect} builds upon a long sequence of ideas, originating from the observation that the L1-norm is effective at inducing sparse structure, while the matrix nuclear norm is effective at inducing low-rank structure \cite{CanPla:11,RFP:10}.  Subsequent work proposed the formulation \eqref{eq:sparselowrank_comdetect} for deconvolving matrices as the linear sum of a sparse matrix and a low-rank matrix \cite{CSPW:11,WPMGR:09}; in particular, Chandarasekaran et. al. \eqref{eq:sparselowrank_comdetect} show that penalization with the L1-norm plus the nuclear-norm succeeds at recovering both components so long as these components satisfy a {\em mutual incoherence} condition.  The formulation in \eqref{eq:sparselowrank_comdetect} is a reasonable approach for obtaining clique covers because one can view cliques as clusters.  
%
%\paragraph{Planted clique / sub-graph problems.}  A different body of work that closely relates to ours is that of finding certain graphs embedded within larger graphs.   The most prominent problem within this body is perhaps the {\em planted clique problem}, which decides if there exists a clique of size $k$ embedded in a larger graph $\mathcal{G}$.  As we noted earlier (see discussion on $\alpha(G)$ and $\omega(G)$), this problem is NP-complete.  To this end, there is a line of work that develop computationally tractable algorithms and show that these succeed at finding cliques in suitably random instances. 
%

% The adjacency matrix of a clique $\bone_{\mathcal{C}} \bone_{\mathcal{C}}^T$ has rank equals to one.  Hence a reasonable approach to finding a clique is to seek a (non-zero) matrix of small rank and whose entries $(i,j)$-th entry is zero if $(i,j)$ is not an edge.  To this end, Ames and Vavasis propose a method based on relaxing the rank constraint to a nuclear norm objective \cite{AV:11}
%\begin{equation} \label{eq:av_clique}
%\begin{aligned}
%\min ~~ & ~~ \| X \|_{*} \\
%\mathrm{s.t.} ~~ & ~~ \langle X, J \rangle \geq k^2 \\
%& ~~ X_{i,j} = 0 \quad (i,j) \notin \mathcal{E}, i \neq j .
%\end{aligned}
%\end{equation}
%In particular, the formulation \eqref{eq:av_clique} hopes to recover $\hat{X} = \bone_{\mathcal{C}} \bone_{\mathcal{C}}^T$ as the optimal solution.
%
%The ideas in \cite{AV:11} were subsequently generalized to more general graphs \cite{AV:14,CC:18}. 
\paragraph{$k$-Disjoint-Clique Approach.} In the $k$-disjoint-clique problem,  the goal is to find $k$ disjoint cliques (of any size) that cover as many nodes of the graph as possible.  It is worth noting that in this particular setting, the number of cliques $k$ is constant and is determined by the user as a parameter. Ames and Vavasis  \cite{AV:14}  introduce the following SDP for this problem:
\begin{equation} \label{eq:av}
\begin{aligned}
\max ~~ & ~~ \langle X, J \rangle \\
\mathrm{s.t.} ~~ & ~~ X \bone \leq \bone \\
& ~~ X_{i,j} = 0 \quad (i,j) \notin \mathcal{E}, i \neq j \\
& ~~ \langle X, I \rangle = k \\
& ~~ X \succeq 0.
\end{aligned}
\end{equation}



\paragraph{Subgraph isomorphism.}  Candogan and Chandrasekaran \cite{CC:18} investigate the problem of finding {\em any} subgraph embedded within a larger graph, known as the {\em subgraph isomorphism problem}.  A natural approach is to search over matrices obtained by conjugating the adjacency matrix of the subgraph with the permutation group, leading to a quadratic assignment problem instance.  The basic idea in \cite{CC:18} is to relax this set by conjugating with the orthogonal group $O(|V|)$ instead.  The resulting convex hull is known as the Schur-Horn orbitope \cite{SSS:11}, and it admits a tractable description via semidefinite programming \cite{DW:09}.  In our context, the adjacency matrix of the subgraph (technically not a proper subgraph) is $\sum_{l} \bone_{\ccs_l} \bone_{\ccs_l}^T$, and the resulting SDP is
\begin{equation} \label{eq:sh}
\begin{aligned}
\max ~~ & ~~ \langle A, X \rangle \\
\mathrm{s.t.} ~~ & ~~ X_{i,j} = 0 \quad (i,j) \notin \mathcal{E}, i \neq j \\
& ~~ X = n \cdot Z_1 + 0 \cdot Z_0 \\
& ~~ \langle Z_0, I \rangle = nk-k \\
& ~~ \langle Z_1, I \rangle = k \\
& ~~ Z_0 + Z_1 = I \\
& ~~ Z_1, Z_2 \succeq 0.
\end{aligned}
\end{equation}



\subsection{Experimental setup and results}

%Our next numerical experiment compares the performance of the Lov\'asz theta function at recovering a planted clique instance in comparison with the above mentioned clustering algorithms. 


In our experimental setup, we generate  clique cover  instances from the planted clique cover model  with $\ks=10$ cliques, each of size $n=10$.  For each value of   $p \in \{ 0.00, 0.05, 0.10, \ldots, 1.00 \}$, we generate $20$ random clique cover  instances.  We compare the Lov\'asz theta against the three methods described in the previous section. Our result are summarized in  Figure \ref{fig:comparison},  where we illustrate the average success rates of each method in recovering the underlying clique cover instances across the entire spectrum of $p$ values. In  the reported experiments we use two notions of recovery for $\lt$:
\begin{enumerate}
\item We declare {\em strong recovery}  when the solver outputs an optimal solution $\hat{X}$ that is equal to $\xc$ (up to a small tolerance).
\item We declare {\em weak recovery} when $\lt(G) = \ks$ but the output solution $\hat{X}$ is not equal to $\xc$. \end{enumerate}



%First, we start off by noting two limitations of the   Note that Ames and Vavasis' formulation is specifically designed for finding disjoint unions of cliques.  One minor limitation of their work is that one needs to specify the size  of cliques one seeks, though that is easily circumvented with a sweep.
%
%  Second, in order to implement Candogan and Chandrasekaran's ideas to finding a clique cover, we implicitly assume that we know what we are searching for in the first place (to be precise, it is the {\em spectrum} of the target graph we require as an input).  This assumption is somewhat unrealistic, but one should bear in mind that Candogan and Chandrasekaran's work in \cite{CC:18} was intended to solve a completely different problem in the first place.  



First, we  compare against  the SDP   \eqref{eq:sparselowrank_comdetect} for community detection.  We sweep over the choices regularization parameters $\lambda \in \{ 0.0, 0.1, \ldots, 1.0 \}$.    This sweeping process makes their method the most expensive to implement among all methods listed here. For a given choice of  $\lambda$, let $(\hat{L},\hat{S})$ denote the optimal solution.  We say that the method succeeds if $\hat{L} = \sum \bone_{\ccs_l}\bone_{\ccs_l}^T$ for some choice of parameter~$\lambda$. 

Second,  we compare against  the SDP \eqref{eq:av} for the $k$-disjoint clique problem.  We say that the method succeeds if the optimal solution satisfies $\hat{X} = \sum \bone_{\ccs_l}\bone_{\ccs_l}^T$. It's worth noting that when applying this method to uncover clique covers, there is a minor constraint in that we must specify the size of cliques $\ks=10$, or alternatively, sweep over all feasible clique sizes.






Our third basis of comparison is the SDP \eqref{eq:sh} proposed by Candogan and Chandrasekaran for finding subgraphs based on the Schur-Horn orbitope of a graph \cite{CC:18}.  We say that the method succeeds if the optimal solution satisfies $\hat{X} = \sum \bone_{\ccs_l}\bone_{\ccs_l}^T$. A limitation of using this  
approach  for finding a clique cover is that we  implicitly assume that we know what we are searching for in the first place (to be precise, it is the {\em spectrum} of the target graph we require as an input).  This assumption is somewhat unrealistic, but one should bear in mind that Candogan and Chandrasekaran's work in \cite{CC:18} was intended to solve a completely different problem in the first place.


Our experiments are summarized in Figure \ref{fig:comparison}. First, when examining the performance of the Lovász theta function $\lt$, we observe a congruence between our theoretical predictions, which predict  strong recovery for small $p$ values, and the outcomes derived from our numerical experiments. Furthermore, it is noteworthy that this strong recovery pattern persists  until approximately $p\approx 0.4$.

In the context of our comparisons with the other three methods, the Lovász theta function emerges as the most successful,  with only a slight edge over the SDP \eqref{eq:av} proposed by Ames and Vavasis \cite{AV:14} for finding disjoint cliques, as well as the Schur-Horn orbitope-based relaxation \eqref{eq:sh} presented by Candogan and Chandrasekaran. The method that exhibited the weakest performance in our experiments was the matrix deconvolution method by Oymak and Hassibi \cite{OH:11}.
%
%We observe that all the graphs exhibit similar behavior; namely, they succeed for small values of $p$, before failing on the larger values of $p$.    We do point out that both of these methods require some additional information about the underlying clique cover we seek (as elaborated earlier) while the Lov\'asz theta function in \eqref{eq:lovasz_lambdamax} does not.  One difficulty with implementing is that it requires choosing the parameter $\lambda$ appropriately; in some cases, it might not even exist.  

\begin{figure}[H]
\centering
\includegraphics[width=0.5\textwidth]{images/comparison.png}
\caption{Comparison of Lov\'asz theta function with other methods.}
\label{fig:comparison}
\end{figure}





