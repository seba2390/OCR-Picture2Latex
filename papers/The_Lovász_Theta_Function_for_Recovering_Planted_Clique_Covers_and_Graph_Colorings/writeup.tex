\documentclass[11pt,letterpaper]{article}

\newcommand{\figref}[1]{Fig.~\ref{#1}}
\newcommand{\tblref}[1]{Table~\ref{#1}}
\newcommand{\secref}[1]{Section~\ref{#1}}
\renewcommand{\eqref}[1]{Equation~(\ref{#1})}

\def\availableat{\url{url-published-on-acceptance}}

\newcommand{\todo}[1]{{\color{red} TODO: {#1}}}
\newcommand{\newstuff}[1]{{\color{red} CHECK: {#1}}}
%\newcommand{\todo}[1]{{}}

\newcommand{\ckp}[2]{$CK_{#1}P_{#2}$}
\newcommand{\cext}{\ckp{8}{16}$ext$}
\newcommand{\cfin}{$F_{CK_{X}P_{Y}}$}
\newcommand{\cray}{\ckp{8}{8}$ray$}
\newcommand{\csin}{$SK_{8}P_{8}$}
\newcommand{\casin}{$SK_{combined}$}
%\renewcommand{\cext}{$SK_{8}K_{8}P_{8}$}
\newcommand{\ckpnl}[2]{$CK_{#1}P_{#2}nl$}


\makeatletter
\newcommand{\Spvek}[2][r]{%
	\gdef\@VORNE{1}
	\left(\hskip-\arraycolsep%
	\begin{array}{#1}\vekSp@lten{#2}\end{array}%
	\hskip-\arraycolsep\right)}

\def\vekSp@lten#1{\xvekSp@lten#1;vekL@stLine;}
\def\vekL@stLine{vekL@stLine}
\def\xvekSp@lten#1;{\def\temp{#1}%
	\ifx\temp\vekL@stLine
	\else
	\ifnum\@VORNE=1\gdef\@VORNE{0}
	\else\@arraycr\fi%
	#1%
	\expandafter\xvekSp@lten
	\fi}
\makeatother




%\title{Recovering Clique Covers via the Lov\'asz Theta Function}
%\title{Exact recovery of graph colorings via the Lov\'asz theta function}
\title{The Lov\'asz Theta Function for Recovering Planted Clique Covers and Graph Colorings} 
\author{Jiaxin Hou, Yong Sheng Soh, and Antonios Varvitsiotis}

\begin{document}


\maketitle

\begin{abstract}
The problems of computing graph colorings and clique covers are central challenges in combinatorial optimization.  Both of these are known to be NP-hard, and thus computationally intractable in the worst-case instance.  A prominent approach for computing approximate solutions to these problems is the celebrated Lov\'asz theta function $\vartheta(G)$, which is specified as the solution of a semidefinite program (SDP), and hence tractable to compute.  In this work, we move beyond the worst-case analysis and set out to understand whether the Lov\'asz theta function recovers clique covers for random instances  that have  a latent clique cover structure, possibly obscured by noise.  We answer this question in the affirmative, and show that  for graphs generated from the planted clique model  we introduce in this work, the SDP formulation of $\vartheta(G)$ has a unique solution that reveals  the underlying clique-cover structure  with high-probability.  The main technical step is an intermediate  result  where we prove a deterministic condition of recovery based on an appropriate notion of sparsity.  
\end{abstract}

\noindent \emph{Keywords}: clique cover, graph coloring,  clustering, community detection, Lov\'asz theta function, beyond worst-case analysis


\subsection{The privacy policies and challenges in medical intelligence}
The privacy issue, while important in every domain, is enforced vigorously for medical data. Multiple level of regulations such as HIPAA~\cite{annas2003hipaa,centers2003hipaa,mercuri2004hipaa,gostin2009beyond} and the approval process for the Institutional Review Board (IRB)~\cite{bankert2006institutional} protect the patients' sensitive data from malicious copy or even tamper evidence of medical conditions~\cite{mirsky2019ct}. Like a double-edge sword, these regulations objectively cause insufficient collaborations in health records.
For instance, America, European Union and many other countries do not allow patient data leave their country~\cite{kerikmae2017challenges,seddon2013cloud}. As a result, many hospitals and research institutions are wary of cloud platforms and prefer to use their own server. Even if in the same country the medical data collaborate still face a big hurdle.


\subsection{The restriction of the medical data accessibility}
It's widely known that sufficient data volume is necessary for training a successful machine learning algorithm~\cite{domingos2012few} for medical image analysis. 
However, due to the policies and challenges mentioned above, it is hard to acquire enough medical scans for training a machine learning model. In 2016, there were approximately 38 million MRI scans and 79 million CT scans performed in the United States~\cite{papanicolas2018health}. Even so, the available datasets for machine learning research are still very limited: the largest set of medical image data available to public is 32 thousand~\cite{yan2018deeplesion} CT images, only 0.02\% of the annual acquired images in the United States.
In contrast, the ImageNet~\cite{deng2009imagenet} project, which is the large visual dataset designed for use in visual object recognition research, has more than 14 million images that have been annotated in more than 20,000 categories.

\subsection{Learning from synthetic images: a solution}
In this work, we design a framework using centralized generator and distributed discriminators to learn the generative distribution of target dataset. In the health entities learning context, our proposed framework can aggregate datasets from multiple hospitals to obtain a faithful estimation of the overall distribution. The specific task (e.g., segmentation and classification) can be accomplished locally by acquiring data from the generator. Learning from synthetic images has several advantages:

\textbf{Privacy mechanism}:
The central generator has limited information for the raw images in each hospital. When the generator communicates with discriminators in hospitals, only information about the synthetic image is transmitted. Such a mechanism prohibits the central generator's direct access to raw data thus secures privacy.

\textbf{Synthetic data sharing}: The natural of synthetic data allows the generator to share the synthetic images without restriction. Such aggregation and redistribution system can build a public accessible and faithful medical database. The inexhaustible database can benefit researchers, practitioners and boost the development of medical intelligence.
  
\textbf{Adaptivity to architecture updates}: The machine learning architecture evolves rapidly to achieve a better performance by novel loss functions~\cite{sudre2017generalised,hochberg1964depth}, network modules~\cite{hoffman2016fcn, ronneberger2015u,milletari2016v,qu2019improving} or optimizers~\cite{ruder2016overview, zeiler2012adadelta, mason2000boosting,zhang2019taming,zhanglocal}. 
%To accelerate this process, modern machine learning frameworks like Pytorch~\cite{ketkar2017introduction} and Tensorflow~\cite{abadi2016tensorflow} provide more efficient ways to build and update networks and training configurations.
We could reasonably infer that the recently well-trained model may be outdated or underperformed in the future as new architectures invented. Since the private-sensitive data may be not always accessible, even if we trained a model based on these datasets, we couldn't embrace new architectures to achieve higher performance. Instead of training a task-specific model, our proposed method trains a generator that learns from distributed discriminators. Specifically, we learn the distribution of private datasets by a generator to produce synthetic images for future use, without worrying about the lost of the proprietary datasets.

%Our proposed approach
To the best of our knowledge, we are the first to use GAN to address the medical privacy problem. Briefly, our contributions lie in three folds: (1) A distributed asynchronized discriminator GAN (AsynDGAN) is proposed to learn the real images' distribution without sharing patients' raw data from different datasets. (2) AsynDGAN achieves higher performance than models that learn from real images of only one dataset. (3) AsynDGAN achieves almost the same performance as the model that learns from real images of all datasets.



%regulation...
%
%
%Especially when Adversarial Generative Network(GAN) attract everyone's attention, the privacy of medical data face a more serious challenge. Though we could apply GAN to achieve many goals like artifact reduction[adversarial sparse-view CBCT], domain adaption for different disease or modality[task driven...], data augmentation[], the GAN, like a double-edged sword, could also hurt us by tampering the medical images, ie., add or remove critical medical findings.


%Since patient data in European countries is typically not allowed to leave Europe, many hospitals and research institutions are wary of cloud platforms and prefer to use their own servers.


\subsection{Proof Outline} \label{sec:outline}

In this section, we give an outline of the proof of our main results.  The key part of our argument is to produce a suitable dual optimal solution that certifies the optimality as well as the uniqueness of $(\ac,\ks)$ in \eqref{eq:lovasz_lambdamax}.  First, as a reminder, the dual program to \eqref{eq:lovasz_lambdamax} is the following semidefinite~program
\beq\label{eq:dual}\tag{D}
\max \Big\{ \, \langle J , Z \rangle \,:\, \mathrm{tr}(Z) = 1, Z_{i,j} = 0 \text{ for all } {(i,j) \in \mathcal{E}}, \ Z \succeq 0  \Big\}.
\eeq

A basic consequence of weak duality is that, in order to show $(\ac,\ks)$ is {\em an} optimal solution to~\eqref{eq:lovasz_lambdamax} (note that this is equivalent to $\vartheta(G) = \ks$), it suffices to produce a dual feasible solution that attains the same objective; i.e., one would need to exhibit $\hat{Z}$ that is a feasible solution to \eqref{eq:dual} and satisfies $\langle J, \hat{Z} \rangle = \ks$. To show that $(\ac,\ks)$ is the {\em unique} optimal solution to \eqref{eq:lovasz_lambdamax},  it is necessary to appeal to a strict complementarity-type of result for SDPs; see, e.g., \cite{overton,LV}.  In what follows, we specialize a set of conditions from \cite{LV} that guarantee unique solutions in SDPs to the Lov\'asz theta function.

\begin{theorem}\label{thm:constraintqualification}
Let $(t,A)$ and ${Z}$ be a pair of strict complementary primal and dual optimal solutions to \eqref{eq:lovasz_lambdamax} and \eqref{eq:dual} respectively, i.e., they satisfy:
 %Suppose these evaluate to the same objective value.  Furthermore, suppose that
$$\la X , Z \ra = 0 \quad \text{ and } \quad \mathrm{rank}(X) + \mathrm{rank}(Z) = |\V|, \qquad \text{ where } \qquad X = tI + A - J. $$  
Then $A$ is the unique primal optimal solution to \eqref{eq:lovasz_lambdamax} if and only if $A$ is an extreme point of the feasible region of \eqref{eq:lovasz_lambdamax}.
\end{theorem}
In view of Theorem \ref{thm:constraintqualification}, to show that $(\ac,\ks)$ is the unique optimal solution of \eqref{eq:lovasz_lambdamax}, it suffices to $(i)$ show that $(\ac,\ks)$ is an extreme point of the feasible region of the feasible region of \eqref{eq:lovasz_lambdamax}, and $(ii) $ produce a dual optimal $\hat{Z} \in \mathbb{R}^{|V| \times |V|}$ that satisfies strict complementarity.

The answer to the first task has simple answer: in Section \ref{sec:extremality}, we show that $(\ac,\ks)$ is an extreme point of the feasible region of \eqref{eq:lovasz_lambdamax} whenever the graph $G$ satisfies the $\cscc$ condition for some $c < 1$.  Our proof relies on results that characterize extreme points of spectrahedra, and the specific version we use is found in Corollary 3 of \cite{RG:95}:
\begin{theorem}\label{thm:extremal_rg}
Let $\mathcal{S}$ be a spectrahedron specified in linear matrix inequality(LMI)  form
\begin{equation} \label{eq:LMI}
\mathcal{S}=\Big \{ (x_1, \ldots, x_n)^T \in \R^n : Q_0 +\sum_{i=1}^n x_i Q_i \succeq 0 \Big \},
\end{equation} 
where $Q_0,Q_1,\ldots, Q_n \in \mathbb{R}^{m \times m}$ are symmetric matrices.  Let $\by = (y_1,\ldots, y_n)^T \in \R^n$, and $U$ be an $m\times k$ matrix whose columns span the kernel of the matrix $Q_0 +\sum_i y_i Q_i$. Then, $\by$ is an extreme point of $\mathcal{S}$ if and only if the vectors ${\rm vec}(Q_1U), \ldots, {\rm vec}(Q_nU)$ are linearly independent. 
\end{theorem}


The answer to the second task is considerably more involved.  As a reminder, a matrix $Z \in~\mathbb{R}^{|V| \times |V|}$ is optimal  for \eqref{eq:dual}  if it satisfies:
\begin{align} 
& Z \succeq 0 \label{cond1}, \\    
& Z \in {\rm span}\{E_{i,j}: \ (i,j)\in \E \}^\perp \label{eq:supportcondition}, \\
& \langle Z, \xc \rangle = 0 ~ (\text{i.e. } Z \in {\rm span}(\xc )^\perp)\label{eq:complementaryslackness}, \\ 
& \mathrm{tr}(Z) = 1. \label{eq:traceone}
\end{align}
Here we use the notation $E_{i,j}={1\over 2}(\be_i \be_j^T+\be_j \be_i^T)$.  We call a matrix $Z$ satisfying these conditions a {\em dual certificate}.  Furthermore, we say that a dual certificate  $Z$ and $\xc$ satisfy  {\em strict complementarity}~if:%the following also holds:
\begin{equation}
{\rm rank}(Z)=|\V| - {\rm rank}(\xc) \label{cond3}.
\end{equation}
%Finally, note that a dual certificate is also dual optimal; indeed,  this is true as 
% & \langle J, Z \rangle = \ks \label{eq:objvalks}.


 %In what follows, we focus on producing dual certificates that satisfy the conditions \eqref{cond1}, \eqref{eq:supportcondition},  \eqref{eq:complementaryslackness} and  \eqref{cond3}. This is sufficient because any matrix $Z$ that satisfies these conditions will be non-zero and PSD.  One is able to apply a non-negative scale so that $\mathrm{tr}(Z) = 1$.  %Later, in Section~\ref{sec:extremality} (see proof of Theorem \ref{thm:disjointcliques_uniquerecovery}), we show that such a scaled $Z$ necessarily satisfies \eqref{eq:objvalks} as a consequence of~\eqref{eq:complementaryslackness}.

%the strict complementarity conditions on $Z$ are:
%  The latter set of conditions are:
%\begin{align} 
%& Z \succeq 0 \label{cond1} \\    
%& Z \in {\rm span}\{E_{i,j}: \ (i,j)\in \E \}^\perp \label{eq:supportcondition} \\
%& Z \in {\rm span}(\xc )^\perp \label{eq:complementaryslackness}\\ 
%& {\rm rank}(Z)=|\V| - {\rm rank}(\xc) \label{cond3},
    % &  \zz\in \label{eq:supportcondition} \tag{SUPP},
%\end{align} 

Note that we can omit condition \eqref{eq:traceone} as we can always scale a matrix that satisfies conditions \eqref{cond1}, \eqref{eq:supportcondition},  \eqref{eq:complementaryslackness} and  \eqref{cond3} to make  it trace-one.  The proof our two main results   is given in three steps.

\paragraph{Step 1: Exact recovery in the case of disjoint cliques.}  In the first step, we consider the simplest setting where $G$ is the disjoint union of cliques $\{\ccs_l\}_{l=1}^{\ks}$.  With a bit of guesswork, we construct the following matrix $\zz$ defined as follows:
\begin{equation}\label{zmatrix}
\zz := \left(\begin{array}{ccc}
\frac{I}{|\ccs_1|} & \frac{1}{|\ccs_1||\ccs_2|} E & \ldots \\
\frac{1}{|\ccs_1||\ccs_2|} E & \frac{I}{|\ccs_2|} & \ldots \\
\vdots & \vdots & \ddots
 \end{array}\right).
\end{equation}
Here, the $(i,j)$-th block has dimension $|\ccs_i| \times |\ccs_j|$.

It is relatively straightforward to verify that the conditions \eqref{eq:supportcondition} and \eqref{eq:complementaryslackness} are satisfied.  In  Proposition \ref{thm:Unequalsizecanonicalmatrix_spectrum} we characterize the spectrum of the matrix $\zz$, and in so doing, show that $\zz$ satisfies the conditions \eqref{cond1} and \eqref{cond3}.  We explain these steps in Sections \ref{sec:extremality} and \ref{sec:disjointunion}.

%In this setting, we construct a dual certificate  $\zz$ that satisfies conditions~{\eqref{cond1}--\eqref{cond3}}. By Theorem \ref{thm:constraintqualification}, this certifies  that $\xc$ is the unique optimal solution of \eqref{eq:lovasz_lambdamax}.
%In the case  where $G$ is the disjoint union  of cliques   we can easily see  that $\tilde{\mathcal{K}}^\perp=\{0\}$   so trivially $\tilde{\mathcal{K}}^\perp$ and $\tilde{\mathcal{L}}^\perp $ are orthogonal. Furthermore,  as $\tilde{\mathcal{K}}\cap \tilde{\mathcal{L}}=\tilde{\mathcal{L}}$  it  suffices  to find a matrix $\zz \in \tilde{\mathcal{L}}$ that also satisfies condition~\eqref{cond3}.  The matrix $\zz$ designed to serve as our dual witness is given by:
%   In Proposition \ref{thm:Unequalsizecanonicalmatrix_spectrum}, we characterize the spectrum of the matrix $\zz$, and in so doing, show that $\zz$ satisfies the conditions \eqref{cond1}-\eqref{cond3}.  



\paragraph{Step 2: Exact recovery  for graphs with the $\cscc$ property.}  The matrix $\zz$ that served as our dual certificate for graphs formed by disjoint cliques does not work in the more general setting as it is non-zero on every edge between distinct cliques; i.e., it violates \eqref{eq:supportcondition}.  The key idea for constructing a suitable dual certificate in this case begins by recognizing that conditions \eqref{eq:supportcondition} and~\eqref{eq:complementaryslackness} define subspaces:
$$ \mathcal{K} :={\rm span}\{E_{i,j}: \ (i,j)\in \E \}^\perp  \ \text{ and } \  \mathcal{L} := {\rm span}(\xc)^\perp.
$$
In view of this, we consider a dual certificate $\zc$ obtained by projecting the matrix $\zz$ with respect to the Frobenious norm onto the intersection of these spaces:
$$
\zc ~~:=~~ \underset{X}{\arg \min} ~\| X - \zz \|_F^2 \quad \mathrm{s.t.} \quad X \in \mathcal{K}\cap \mathcal{L}.
$$
The remainder of proof is a matrix pertubation argument in which we show that $\|\zc - \zz \|$ is quite small, which allows us to conclude the remaining conditions \eqref{cond1} and \eqref{cond3}.
We now provide some geometric intuition why  the  projection onto the intersection $\mathcal{K}\cap \mathcal{L}$ succeeds.  

The  graphs considered in this work  are either disjoint union of cliques or  have the $\cscc$ property.  In both cases, we have that $\mathcal{K}$ is a subspace of  
% Thinking of any symmetric matrix $ X \in \mathbb{R}^{|\V| \times |\V|}$ indexed by the vertices of the graph as a block matrix  whose blocks are indexed by cliques $\ccs_l$, it follows that its  diagonal blocks   are diagonal.  Equivalently, $\mathcal{K}$ is a subspace of the linear space:$$
$$\mathcal{M} := \{ X \in \mathbb{R}^{|\V| \times |\V|}: \ X=X^T, (\ccs_l,\ccs_l) \text{-th block is diagonal for all } 1 \leq l \leq k \},
$$
which we   treat  as the ambient space, rather than the space of all symmetric matrices.
 Setting 
 % $\tilde{\mathcal{K}}$ and $\tilde{\mathcal{L}}$ denote the restrictions of $\mathcal{K}$
 %and  $\mathcal{L}$ on $\mathcal{M}$ respectively, i.e., 
$$   \tilde{\mathcal{K}} :={\rm span}\{E_{i,j}: \ (i,j)\in \E \}^\perp \cap \mathcal{M}  \ \text{ and } \  \tilde{\mathcal{L}}:= {\rm span}(\xc)^\perp\cap \mathcal{M},$$
%In other words,  
%Although  $\tilde{\mathcal{K}}=\mathcal{K}$, however  restricting  on $\mathcal{M}$ affects its orthogonal complement. 
%%
%First, we define the following subspace of symmetric matrices
%$$
%\mathcal{M} := \{ X \in \mathbb{R}^{|\V| \times |\V|}: \ X=X^T, (\ccs_l,\ccs_l) \text{-th block is diagonal for all } 1 \leq l \leq \ks \}. 
%$$
%With that, we define the subspaces 
%$$   \tilde{\mathcal{K}} :={\rm span}\{E_{i,j}:  (i,j)\in \E \}^\perp \cap \mathcal{M}  \ \text{ and } \  \tilde{\mathcal{L}}:= {\rm span}(\xc)^\perp\cap \mathcal{M}.
%$$
 $\zc$ can be equivalently defined as the projection of $\zz$ onto the intersection of $\tilde{\mathcal{K}} \cap \tilde{\mathcal{L}}$.
\begin{equation} \label{eq:euclideanprojection}
\zc ~~:=~~ \underset{X}{\arg \min} ~\| X - \zz \|_F^2 \quad \mathrm{s.t.} \quad X \in \tilde{\mathcal{K}} \cap \tilde{\mathcal{L}}.
\end{equation}
 In Section \ref{sec:proof-deterministicrecovery} we give the main technical result of this work, where we establish  that the subspaces $ \tilde{\mathcal{K}}^\perp$ and $ \tilde{\mathcal{L}}^\perp$ are ``almost orthogonal''. For this, it is now crucial that our ambient space is $\mathcal{M}$.

%Throughout this work we consider graphs that are either disjoint union of cliques or that have the $\cscc$ property.  Thinking of any symmetric matrix $ X \in \mathbb{R}^{|\V| \times |\V|}$ indexed by the vertices of the graph as a block matrix  whose blocks are indexed by cliques $\ccs_l$, it follows that its  diagonal blocks   are diagonal.  Equivalently, $\mathcal{K}$ is a subspace of the linear space:

% We restrict our search for a dual certificate in $\tilde{\mathcal{K}}\cap \tilde{\mathcal{L}},$ where  $\tilde{\mathcal{K}}$ and $\tilde{\mathcal{L}}$ denote the restrictions of $\mathcal{K}$
% and  $\mathcal{L}$ on $\mathcal{M}$ respectively, i.e., 
%$$   \tilde{\mathcal{K}} :={\rm span}\{E_{i,j}: \ (i,j)\in \E \}^\perp \cap \mathcal{M}  \ \text{ and } \  \tilde{\mathcal{L}}:= {\rm span}(\xc)^\perp\cap \mathcal{M}.$$
% It is evident  that $\tilde{\mathcal{K}}=\mathcal{K}$, however  restricting  on $\mathcal{M}$ affects its orthogonal complement.  
 

\paragraph{Step 3: Recovery for planted clique cover instances.}  Our third  step shows that planted clique instances are $\cscc$, with high probability.  The proof is a direction application of Hoeffding's inequality combined with a union bound, and is provided in Section \ref{sec:randomgraphs}.





\section{Extremality of $(\ac,\ks)$} \label{sec:extremality}


The goal of this section is to provide a simple sufficient condition that guarantees when $(\ac,\ks)$ is an extreme point of the feasible region of \eqref{eq:lovasz_lambdamax}.  %We state the main result of this section in the following.
 As we noted in Section \ref{sec:outline}, we rely  on a  general result that describes extreme points of spectrahedra specified by a linear matrix inequality (LMI), stated in the form of Theorem \ref{thm:extremal_rg}.  We begin by expressing the feasible region of \eqref{eq:lovasz_lambdamax} as the following LMI
$$
\Big \{  (t, a_{i,j})_{(i,j) \in \E} \in \mathbb{R}^{|\E|+1} : - J + t I + \sum_{(i,j) \in \E} a_{i,j} E_{i,j} \succeq 0 \Big \}.
$$
This suggests we should take the matrices $\{I\} \cup \{ E_{i,j} : (i,j) \in \E \}$ to be $\{Q_i\}_{i=1}^{n}$, and to identify the matrix $\xc$ with $Q_0 +\sum_i x_i Q_i$ in Theorem \ref{thm:extremal_rg}.  Our next task is to characterize the kernel of~$\xc$.



\subsection{Computing the kernel of $\xc$}


\begin{proposition}\label{thm:jspace} Setting 
$$
\mathcal{J}:=\big \{ \bx \in \R^{|\V|} : \la \bx, \bone_{\ccs_1} \ra = \ldots = \la \bx,  \bone_{\ccs_{\ks}}\ra \big\},
$$
 we have that
$$\mathcal{J} = \lspan\{\bone_{\ccs_1} - \bone_{\ccs_l}:  2\le l\le \ks \}^\perp = \lspan\{ \ks \bone_{\ccs_l} -\be: l\in [\ks]\}^\perp.$$
\end{proposition}

\begin{proof}  The first equality follows immediately from definition since $\la \bx, \bone_{\ccs_1} \ra = \la \bx, \bone_{\ccs_l} \ra \Leftrightarrow \la \bx, \bone_{\ccs_1} - \bone_{\ccs_l} \ra$.  We focus on the second equality.  To do so, it suffices to show that every vector of the form $\ks \bone_{\ccs_l} -\be$, $l\in [\ks]$, is in the span of $\bone_{\ccs_1} - \bone_{\ccs_l}$, $2\le l\le \ks $, and vice versa.

First, we have $\bone_{\ccs_i} - \bone_{\ccs_j} = (\bone_{\ccs_1} - \bone_{\ccs_j}) - (\bone_{\ccs_1} - \bone_{\ccs_i})$.  Then, note that $\ks \bone_{\ccs_l} -\be = \sum_{i=1}^{\ks} (\bone_{\ccs_l} - \bone_{\ccs_i})$.  Hence the vectors $\ks \bone_{\ccs_l} -\be $ lies in the linear span of vectors of the form $\bone_{\ccs_1} - \bone_{\ccs_l}$, $2\le l\le \ks $.

In the reverse direction, note that $\bone_{\ccs_1} - \bone_{\ccs_l} = \frac{1}{\ks}(\ks \bone_{\ccs_1} -\be) - \frac{1}{\ks}(\ks \bone_{\ccs_l} -\be)$.  In other words, every vector of the form $\bone_{\ccs_1} - \bone_{\ccs_l}$, $2\le l\le \ks$, lies in the linear span of vectors of the form $\ks \bone_{\ccs_l} -\be$, $l\in [\ks]$.  This establishes the second equality.
%%We first focus on the first equality. %For this, 
%Finally, we show that 
%$$\lspan\{ \ks\bone_{\ccs_l} -\be: l\in [\ks]\}^\perp=\mathcal{J}$$ 
%\{\bx\in \R^n : \la \bx,   \bone_{\ccs_1}\ra =\ldots= \la \bx,  \bone_{\ccs_{\ks}} \ra \}.$$
%For this take $\bw\in \lspan\{ \ks \bone_{\ccs_l} -\be: l\in [\ks]\}^\perp$. Then for any $l,l'\in [\ks]$ we have 
%$$\la \bw, \ks\bone_{\ccs_l}-\be\ra =\la \bw, \ks\bone_{\ccs_{l'}}-\be\ra=0.$$
%In particular, we have that
%$$\la \bw, \bone_{\ccs_l}\ra=\la \bw, \bone_{\ccs_{l'}}\ra.$$
%Conversely, consider $\bw\in \{ \bx\in \R^{|\V|}:   \la \bx,   \bone_{\ccs_1}\ra =\ldots= \la \bx,  \bone_{\ccs_k} \ra \}$. Using that 
%$$k\bone_{\ccs_l}-\be=(\ks-1)\bone_{\ccs_l}-\sum_{l'\ne l}\bone_{\ccs_{l'}}=\sum_{l'\ne l}\left(\bone_{\ccs_l}- \bone_{\ccs_{l'}}\right)$$
%we immediately get that 
%$$\la \bw, \ks\bone_{\ccs_l}-\be\ra=\sum_{l'\ne l}\la \bw, \bone_{\ccs_l}- \bone_{\ccs_{l'}}\ra=0.$$
%
%Finally, the second characterization of $\mathcal{J}$ follows directly from  its definition.  
\end{proof} 

As an immediate consequence we have that:
%%In this section, we %discuss extremality of $\xc$ and 
%show
% that $\xc$ is the unique optimal solution to \eqref{eq:lovasz_lambdamax}  when $G$ is a disjoint union of cliques.  Our discussion corresponds to Step 1 and 2 of the proof outline.

\begin{proposition}\label{kernellemma}
The kernel  of $\xc$ is equal to  $ \mathcal{J}.$ 
%We proceed to establish our results concerning extremality.  
\end{proposition}

\begin{proof} By definition of $\xc$ we have that 
${\rm range}(\xc)={\rm span}\{ \ks \bone_{\mathcal{C}_l}-\be: l \in [k]\}.$
Thus,
$${\rm ker}(\xc) ={\rm span}\{ \ks \bone_{\mathcal{C}_l}-\be: l \in [k]\}^\perp.$$

\end{proof} 

%Our first step is to state a more convenient form of the complementary slackness condition.  First, we define the following subspace of $\mathbb{R}^{|\V|}$ corresponding to vectors that have constant inner product with the indicator vectors of cliques
%$$
%\mathcal{J} := 
%$$


\begin{proposition}\label{csexpnaded}
%Let $\xc= {1\over \ks}\sum_{l=1}^{\ks} (\ks \bone_{\mathcal{C}_l}-\be) (\ks \bone_{\mathcal{C}_l}-\be)^T$.
The following  statements are equivalent for a PSD matrix $Z \in \mathbb{R}^{|\V| \times |\V|}$:

\begin{itemize}
\item[(1)]  $  Z\in  {\rm span}( \xc)^\perp$.% $Z$  satisfies the complementary slackness condition \eqref{eq:complementaryslackness}. % i.e., 
\item[(2)] $Z(\ks\bone_{\ccs_l} -\be)=0,$ for all $ l \in [\ks].$

\item[(3)] $\range(Z) \subseteq \mathcal{J}$.
%\Big\{x\in \R^n:   \la x,   \bone_{\mathcal{C}_1^*} \ra=\ldots= \la x,  \bone_{\mathcal{C}_k^*}\ra \Big\}.$ 

\item[(4)] $\bz_i \in \mathcal{J}$ for all $i \in \V$ -- here, $\bz_i$ are rows of $Z$.
%For any $i\in [n]$,  the $i$-th row of $Z$, denoted by  $Z_i$  satisfies $Z_i\in \mathcal{J}$. 
% the following set of equalities
%\begin{equation} \label{eq:complementaryslackness}
%\la Z_i,  \bone_{\ccs_1}\ra = \ldots =\la  Z_i , \bone_{\mathcal{C}_k^\star}\ra  \tag{CS-2}.
%\end{equation}
\end{itemize} 
Moreover, we have that ${\rm rank}(\xc)+{\rm rank}(Z)=|\V|$ if and only if   $\range(Z) = \mathcal{J}$.

\end{proposition}



\begin{proof}$(1)\Longleftrightarrow (2)$ We have that 
$$\la Z, \xc\ra=0 \iff \sum_l(\ks\bone_{\ccs_l}-\be)^T Z (\ks\bone_{\ccs_l}-\be)=0\iff Z(\ks\bone_{\ccs_l} -\be)=0, \  \forall l \in [\ks]$$ 
 where the last equivalence holds as  $Z\succeq 0$. 

$(2)\Longleftrightarrow (3)$ Note that (2) is equivalent to  
$$\ker(Z)\supseteq  \range(\xc)=\lspan\{ \ks \bone_{\ccs_l} -\be: l\in [\ks]\}.$$ 
Taking orthogonal complements, this is in turn equivalent to 
$$\range(Z)\subseteq \lspan\{ \ks\bone_{\ccs_l} -\be: l\in [\ks]\}^\perp=\mathcal{J}.$$
Finally $(3) \iff (4)$ since $\range(Z)=\range(Z^T)$. 
\end{proof}


Our next result describes the spectrum of the matrix $\zz$ 
and  show that the columns of the matrix $\zz$ span the kernel of $\xc$. Recall that
%The dual certificate we will employ in the case of disjoint cliques is the following:
\begin{equation}\label{zfullmatrix}
\zz := \left(\begin{array}{ccc}
\frac{I}{|\ccs_1|} & \frac{1}{|\ccs_1||\ccs_2|} E & \ldots \\
\frac{1}{|\ccs_1||\ccs_2|} E & \frac{I}{|\ccs_2|} & \ldots \\
\vdots & \vdots & \ddots
 \end{array}\right),
\end{equation}
This matrix can be alternatively expressed in a more convenient form:
$$
\zz = \mathrm{diag}(\bg) + \bg\bg^T - \sum_{l} \frac{\bone_{\ccs_l}\bone_{\ccs_l}^T}{|\ccs_l|^2} \quad \text{where} \quad \bg = \Big(\underbrace{\frac{1}{|\ccs_1|}, \ldots }_{|\ccs_1|} , \ldots,  \underbrace{\ldots, \frac{1}{|\ccs_{\ks}|} }_{|\ccs_{\ks}|} \Big)^T.
$$
To be clear, this is precisely the same matrix \eqref{zmatrix} we use as our dual certificate whenever $G$ is a disjoint union of cliques.  However, we will not require any information about $\zz$ being a dual certificate at this juncture.  
\begin{proposition} \label{thm:Unequalsizecanonicalmatrix_spectrum}
The eigenvalues of $\zz$ are $\sum_{l=1}^{\ks} 1/|\ccs_l|$ (with multiplicity one), $0$ (with multiplicity $\ks-1$), and $1/|\ccs_l|$ with multiplicity $|\ccs_l|-1$. Moreover, we have that 
$\range(\zz)=\mathcal{J}$.
%$\alpha$ (with multiplicity $N-k$).
\end{proposition}

\begin{proof}
First, note that the $\bg$ is an eigenvector whose eigenvalue is $\sum_{i=1}^{\ks} 1/|\ccs_i|$.  Second, note that the vector $\bone_{\ccs_1} - \bone_{\ccs_l} $ is an eigenvector with eigenvalue $0$, for all $2 \leq l \leq \ks$.  Third, fix a subset $\ccs_l$.  Let $\bx \in \mathbb{R}^{|V|}$ be a vector with entries in the coordinates corresponding to $\ccs_l$ satisfying $\be^T \bx = 0$.  One can check that $(\bg\bg^T - \sum_{l} (\bone_{\ccs_l}\bone_{\ccs_l}^T)/|\ccs_l|^2) \bx = 0$, and hence $\zz\bx = \mathrm{diag}(\bg) \bx = \bx / |\ccs_l|$.  The dimension of this eigenspace is $|\ccs_l|-1$.  Finally, we have that 
$$\ker(\zz)=\lspan\{\bone_{\ccs_1} - \bone_{\ccs_l}:  2\le l\le \ks \}$$
and using Proposition \ref{thm:jspace} we get that 
$$\range(\zz)=\lspan\{\bone_{\ccs_1} - \bone_{\ccs_l}:  2\le l\le \ks \}^\perp=\mathcal{J}.$$

\end{proof}

\subsection{Establishing extremality}

\begin{lemma} \label{thm:non_degen_Z} Consider  a graph $G$ with a  clique cover $\{\ccs_l\}_{l=1}^{\ks}$.  Let $\mathcal{S} \subset \V$ be a subset of vertices such that, for all cliques except one, $ \mathcal{S}$ leaves out at least one vertex.  Then, the columns of $\zz$ corresponding to $\mathcal{S}$ are linearly independent.
\end{lemma}

\begin{proof} Without loss of generality, we assume that for all cliques except $ \ccs_1$, the set of vertices $\mathcal{S}$ contains at most $ |\ccs_{l}|-1$ vertices from clique $\ccs_{l}$, for all $l\ne 1$.

We index the columns of $\zz$ by $(l,j)$, where the index $1\leq l \leq \ks$ refers to the clique while the index $j\in \ccs_l$ refers to the vertex within the clique.  Let $\{\bv_{l,j}\}$ be the columns of the matrix $\zz$ and consider a linear~combination:
\begin{equation}\label{eq:non_degen_Z_1}
\sum_{(l,j) \in \mathcal{S}} \theta_{l,j} \bv_{l,j} = 0.
\end{equation}
%Without loss of generality, assume that for $l \in \{2,\ldots,\ks\}$ we leave out at least one column from~$\ccs_l$.  
Fix a cluster $\ccs_l$, and let $P_l$ be the projection of $\mathbb{R}^{|\V|}$ onto the coordinates corresponding to $\ccs_l$.  By applying $P_l$ to \eqref{eq:non_degen_Z_1} we get:
\begin{equation}\label{eq:non_degen_Z_2}
\sum_{j : (l,j)\in \mathcal{S}} \theta_{l,j} P_l(\bv_{l,j}) = - \sum_{(l',j) \in \mathcal{S}, 1\leq l' \neq l \leq \ks} \theta_{l',j} P_l(\bv_{l',j}) =  c (1,\ldots,1)^T.
\end{equation}
The rightmost equality follows by noting that $P_l(\bv_{l',j}) = \be \in \mathbb{R}^{|\ccs_l|}$ for all $j$.  On the other hand, $P_l(\bv_{l,j}) = \be_j  \in \mathbb{R}^{|\ccs_l|}$ are standard basis vectors.  Since $\mathcal{S}$ leaves out at least one vector in $\ccs_l$, the span of these vectors cannot include $\be$.  So it follows from \eqref{eq:non_degen_Z_1} that $c=0$, and in particular, $\theta_{l,j} = 0$ for all $j$.  We repeat this argument for all $2 \leq l \leq \ks$ and conclude that $\theta_{l,j} = 0$ for all $l \geq 2$, and all $j$.

Finally, we consider the vectors $\bv_{1,j}$ and project these to the rows in $\ccs_1$.  The vectors $\{P_l(\bv_{1,j})\}$ are standard basis vectors.  Since $\sum_{(1,j) \in \mathcal{S}} \theta_{1,j} P_l(\bv_{1,j}) = 0$, it must be that $\theta_{1,j} = 0$ for all $j$.  Hence all the coefficients are zero.  This proves that the columns in $\mathcal{S}$ are linearly independent.
\end{proof}


\begin{lemma} \label{thm:non_degen_alg} Let $G = (\V,\E)$ be a graph that satisfies the $\cscc$ property for some $c<1$.  Then the collection of matrices 
$
\{ E_{i,j} \zz : (i,j) \in \mathcal{E}\} \cup \{ \zz \}
$ 
are linearly independent.
\end{lemma}

\begin{proof}
Consider a linear combination  
$$
\sum_{(i,j) \in \mathcal{E}} \theta_{i,j} E_{i,j} \zz  + \theta \zz = 0.
$$
Fix a vertex $m \in \ccs_l$, and for all $l' \neq l$, we let $N_{l'}(m) \subset \ccs_{l'}$ denote the neighbors of vertex $m$ in $\ccs_{l'}$.  Then the $m$-th row of the above expression is given by:
$$
%\begin{equation}\label{eq:non_degen_alg_1}
\sum_{\substack{ j\in N_{l'}(m), l'\ne m}} \theta_{m,j} \bz^\star_j + \theta \bz^\star_m=0,
$$
%\end{equation}
where $\bz_j^\star$ denotes the $j$-th row of $\zz$. % Let $N_l(m)$ be the neighbors of vertex $m$ in the cluster $\ccs_l$. 
Now, since the vertex $m$ is not fully connected to clusters $\ccs_{l'}$ for all $l' \neq l$, the collection of vertices $\{N_{l'}(m)\}_{l' \neq m}\cup \{m\}$ satisfy the conditions of the subset $\mathcal{S}$ in Lemma \ref{thm:non_degen_Z}.  By Lemma \ref{thm:non_degen_Z}, the corresponding columns of $\zz$ are linearly independent, which means that $\theta = \theta_{m,j} = 0$ for all $j$.  The result follows by repeating this argument for all $m$.
\end{proof}



\begin{theorem} \label{thm:extremepoint}
Let $G = (\V, \E)$ be a graph, and let $\{\ccs_l\}_{l=1}^{k}$ be a clique cover for $G$.  Suppose $G$ satisfies the $\cscc$ property for some $c<1$.  Then $(\ac,\ks)$ is an extreme point of the feasible region of~\eqref{eq:lovasz_lambdamax}.
\end{theorem}

\begin{proof}[Proof of Theorem \ref{thm:extremepoint}]  Suppose $G$ satisfies the $\cscc$ property for some $c<1$.  By Lemma \ref{thm:non_degen_alg}, the collection of matrices $\{ E_{i,j} \zz : (i,j) \in \E \} \cup \{ \zz \}$ are linearly independent.  Hence the vectors in $\{ {\rm vec}( E_{i,j} \zz) : (i,j) \in \E \} \cup \{ {\rm vec}(\zz) \}$ are also linearly independent. % (note that vectorization is bijective). 
This means that the  following matrix formed by combining all the (vectorized) matrix product $E_{i,j} \zz$, ranging over all edges $(i,j) \in \E$, as well as $\zz$, has full column rank
$$
\left( \begin{array}{c|c|c|c}
\ldots & \mathrm{vec}( E_{i,j} \zz) & \ldots
& \mathrm{vec}(\zz)\end{array} \right)_{(i,j) \in \E}.
$$ 
Hence, by Theorem \ref{thm:constraintqualification}, $(\ac,\ks)$ is an extreme point of the feasible region of \eqref{eq:lovasz_lambdamax}.
\end{proof}


\section{Exact recovery in the case of  disjoint cliques} \label{sec:disjointunion}

In this section, we show that $(\ac,\ks)$ is the unique optimal solution to the Lov\'asz theta formulation~\eqref{eq:lovasz_lambdamax} when $G$ is a disjoint union of cliques.  In this simple case, $G$ satisfies the $\cscc$ condition with $c=0$.  Following the conclusions of Theorem \ref{thm:extremepoint}, $(\ac,\ks)$ is an extreme point of the feasible region of \eqref{eq:lovasz_lambdamax}.  Based on Theorem \ref{thm:constraintqualification}, it remains to produce a suitable dual witness $Z$ satisfying the requirements listed in Theorem \ref{thm:constraintqualification}.

As it turns out, the matrix $\zz$ satisfies all the requirements we need!  In fact, in the process of computing the spectrum of the matrix $\zz$ in Proposition \ref{thm:Unequalsizecanonicalmatrix_spectrum}, we have also verified that $\zz$ does indeed satisfy the requirements of Theorem \ref{thm:constraintqualification}.  We collect these conclusions below.
 
\begin{theorem} \label{thm:disjointcliques_uniquerecovery}
Let $G$ be the graph formed by the disjoint union of cliques $\{\ccs_l\}_{l=1}^{\ks}$.  Then the matrix $(\ac,\ks)$ is the unique solution of \eqref{eq:lovasz_lambdamax}.
\end{theorem} 


\begin{proof}
As we noted above, we simply need to check that the matrix $(1/\ks)\zz$ satisfies conditions \eqref{cond1} to \eqref{cond3}.  First, from Proposition \ref{thm:Unequalsizecanonicalmatrix_spectrum}, the matrix $\zz$ is PSD, which is \eqref{cond1}.  Second, all the edges of $G$ are within cliques.  The matrix $\zz$, when restricted to each clique $\ccs_l$, is the identity matrix, and thus satisfies the support condition \eqref{eq:supportcondition}.  It is easy to verify that all the rows of $\zz$ belong to $\mathcal{J}$.  Hence, by Proposition \ref{csexpnaded}, $\zz\in {\rm span}(\xc)^\perp$, which is \eqref{eq:complementaryslackness}.  It is easy to see that $\mathrm{tr}(\zz) = \ks$ and $\langle J, \zz \rangle = (\ks)^2$, which shows  \eqref{eq:traceone}.  Last, by Proposition \ref{thm:Unequalsizecanonicalmatrix_spectrum}, we have $\range(\zz)=\mathcal{J}$.  Hence, by Proposition \ref{csexpnaded}, we have ${\rm rank}(\xc)+{\rm rank}(\zz)=|\V|$, which is~\eqref{cond3}.
\end{proof}

 
\section{Exact recovery for graphs with the $\cscc$ proeprty} \label{sec:proof-deterministicrecovery}

In this section, let $G$ be a graph that satisfies the $\cscc$ property.  Our goal is to extend the results in Section \ref{sec:disjointunion}, and show that $(\ac,\ks)$ is also the unique solution to \eqref{eq:lovasz_lambdamax} for graphs satisfying the $\cscc$ condition, provided $c$ is appropriately small.  Following our discussion in Section \ref{sec:disjointunion}, the matrix $\xc$ remains an extreme point of the feasible region of \eqref{eq:lovasz_lambdamax} if $c<1$.  Hence, by Theorem \ref{thm:constraintqualification}, it remains to produce a suitable dual witness $Z$ satisfying the requirements listed in Theorem \ref{thm:constraintqualification}.




\subsection{Incoherence-type result} \label{sec:incoherence}

The certificate we will use  for graphs with the  $\cscc$ property is   $\zc = P_{\tilde{\mathcal{K}} \cap \tilde{\mathcal{L}}} (\zz)$, defined as  the projection of $\zz$ onto $  {\tilde{\mathcal{K}} \cap \tilde{\mathcal{L}}}$ with respect to the Frobenius norm, i.e., 
$$\zc ~~:=~~ \underset{X}{\arg \min} ~\| X - \zz \|_F^2 \quad \mathrm{s.t.} \quad X \in \tilde{\mathcal{K}} \cap \tilde{\mathcal{L}}.$$
Using the first-order optimality conditions we have that 
\begin{equation} 
\zz - \zc = \tilde{L}  + \tilde{K},
\end{equation}
where $\tilde{L}$ lies in the normal cone of $\mathcal{L}$ at $Z'$ and $\tilde{K}$ lies in the normal cone of $\mathcal{K}$ at $Z'$. Recalling that the normal cone of a subspace is its orthogonal complement we have that $\tilde{L}\in \tilde{\mathcal{L}}^\perp$ and $\tilde{K}\in \tilde{\mathcal{K}}^\perp$. 

Our main result  can be viewed as the extension of orthogonality for graphs that are disjoint unions of cliques to graphs with the $\cscc$ property.  The proof is broken up into a sequence of results that follow.


\begin{theorem} 
 \label{thm:incoherence_1}
Let $G$ be a graph satisfying the $\cscc$ property.  Then, 
\begin{itemize}
\item[$(i)$] For any  $\tilde{K} \in \tilde{\mathcal{K}}^\perp$ and  $\tilde{L} \in \tilde{\mathcal{L}}^\perp$ we have   
$$
| \langle \tilde{K}, \tilde{L} \rangle | \leq 2\sqrt{c}\|\tilde{K}\|_F  \|\tilde{L}\|_F .
$$
\item[$(ii)$]  For any  $\tilde{L} \in \tilde{\mathcal{L}}^\perp$ we have 
$$\|P_{\mathcal{K}^\perp}(\tilde{L})\|_F \leq (2 \sqrt{c})^{1/2} \|\tilde{L}\|_F.$$
\end{itemize}
\end{theorem}


\paragraph{A direct sum decomposition for  $\tilde{\mathcal{L}}^\perp$.}  The first  step is to provide a decomposition of the space $\tilde{\mathcal{L}}^\perp$ into simpler subspaces, on which it is easier to prove the near orthogonality property.  We use these results as basic ingredients to build up to our near orthogonality property later.


Define the matrices $\{ F_{x,y,z} : 1 \leq x \neq y \leq k, 1\leq z \leq |\ccs_x | \}$ so that (i) the $(\ccs_x,\ccs_x)$-th block is a diagonal matrix whose $z$-th entry is set equal to $-2(|\ccs_x|-1)$ and all remaining entries equal to $2$, and (ii) the $(\ccs_x,\ccs_y)$-th block ($(\ccs_y,\ccs_x)$-th block) is such that entries in the $z$-th row (column) are equal to $|\ccs_x|-1$ and all other entries equal to $-1$, (iii) all other entries are zero. As an example, in the case where  $k=2$ the matrix  $F_{1,2,1} $ is given by:
 %$$
$$F_{1,2,1} := \left( \begin{array}{cccc|ccc}
-2(|\ccs_1|-1) &&&& |\ccs_1|-1 & \ldots & |\ccs_1|-1  \\
& 2 &&& -1 & \ldots & -1  \\
&& \ddots &&\vdots &&\vdots  \\
&&&2& -1 & \ldots & -1  \\
\hline
|\ccs_1|-1 & -1 & \ldots & -1 &&& \\
\vdots & \vdots & & \vdots &&& \\
|\ccs_1|-1 & -1 & \ldots & -1 &&& \\
\end{array} \right),
$$
where omitted entries are zero. 
%the $(x-1)n+z$-th diagonal entry is set to $-2(n-1)$, (ii) the $(x-1)n+z'$-th diagonal entry, where $1\leq z' \leq n$, $z' \neq z$, is set to $2$, (iii) the $((x-1)n + z, (y-1)n + 1)$ to the $((x-1)n + z, yn)$ row entries as well as the $((y-1)n + 1, (x-1)n + z)$ to the $(yn, (x-1)n + z)$ column entries are set to $n-1$, and (iv) all remaining entries in the $(x,y)$-th and the $(y,x)$-th block are set to $-1$s.
Second, consider the following subspaces:\medskip 


\begin{table}[h]
\centering
\begin{tabular}{|c|c|c|}
\hline
Name & Description & Dimension \\
\hline \hline
$\mathcal{T}_{1}$ & $ \left \{ \left( \begin{array}{c|c|c|c|c} \gamma_1 I & \theta_{1,2} E & \theta_{1,3} E & \ldots & \theta_{1,n} E \\ \hline \theta_{2,1} E & \gamma_2 I & \theta_{2,3} E & \ldots & \theta_{2,n} E \\ \hline \theta_{3,1} E & \theta_{3,2} E & & & \\ \hline \vdots & \vdots & & & \\ \hline \theta_{n,1} E & \theta_{n,2} E & & & \gamma_n I \end{array}\right) : \sum \gamma_i + \sum \theta_{i,j} = 0 \right\}$ & ${k+1 \choose 2}-1$ \\ \hline
%$\mathcal{L}_{N} $ & $ \left\{ \left( \begin{array}{ccccc} &*&* & \ldots \\ * & & * & \ldots \\ *&*&& \\ \vdots & \vdots & &  \end{array}\right) : * \in \mathcal{N}_{C+R} \right\}$ & $k \times (k-2) \times (n-1)$ \\ \hline
$\mathcal{T}_{2} $ & $ \mathrm{Span} \left\{ F_{x,y,z} : 1 \leq x \neq y \leq k, 1\leq z \leq n \right\}$ & $(|V| - k) \times (k-2)$ \\ \hline
\end{tabular}
\end{table}

\begin{proposition} \label{thm:description_of_lperp} We have that 
$$\tilde{\mathcal{L}}^\perp=\mathcal{T}_{1}\oplus \mathcal{T}_{2}.$$
 
\end{proposition}

%Note: Actual rank is $nk(k-1) - (k-1)(k-2)/2$.. verified using matlab

\begin{proof}[Proof of Proposition \ref{thm:description_of_lperp}]
The fact that   $\mathcal{T}_{1}$ and $\mathcal{T}_{2}$ are  orthogonal is easy to check. Define the matrix $L_{x,y,z} \in \mathbb{R}^{|\V|\times|\V|}$, where $1 \leq x \neq y \leq k$, and $1 \leq z \leq |\ccs_x|$ such that (i) the $z$-th entry of the $(\ccs_x,\ccs_x)$-th block is equal to $2$, and (ii) the entries of the $z$-th row (column) of the $(\ccs_x,\ccs_y)$-th block ($(\ccs_y,\ccs_x)$-th block) are all equal to $-1$, and (iii) all other entries are $0$.

As an example, in the case of two cliques (so $k=2$)  the  matrix $L_{1,2,1} $ is given by:
$$
L_{1,2,1} := \left( \begin{array}{cc|ccc} 
2 & & -1 & \ldots & -1  \\
& & & &  \\
\hline
-1 & & & &   \\
\vdots & & & &  \\
-1 & & & &   \\
\end{array}\right),
$$
where all omitted entries are zero. 

%, as the matrix where the $(x-1)n + z$-th diagonal entry is set equal to $2$, while the $((x-1)n + z, (y-1)n + 1)$ to the $((x-1)n + z, yn)$ row entries as well as the $((y-1)n + 1, (x-1)n + z)$ to the $(yn, (x-1)n + z)$ column entries are set to $-1$; all other entries are set to zero.  
First, observe that the matrices $\{ L_{x,y,z} \}$ specify all the linear equalities in the subspace $\tilde{\mathcal{L}}$, and thus, $\tilde{\mathcal{L}}^\perp$ is precisely the span of $\{ L_{x,y,z} \}$.  As such, to show that   $\mathcal{T}_{1}$ and $\mathcal{T}_{2}$  span $\tilde{\mathcal{L}}^\perp$, it suffices to show that every matrix $L_{x,y,z}$ is expressible as the sum of matrices belonging to each of these subspaces.  
As a concrete example, we show this is true for   $L_{1,2,1}$ -- the construction for other matrices are similar.
Indeed, one checks that $L_{1,2,1}$ is the linear sum of these matrices
$$ 
L_{1,2,1}= \underbrace{\frac{1}{|\ccs_1|} \left( \begin{array}{c|c|c|c} 2I & -E & 0 & \ldots \\ \hline -E & 0 & \ldots & \\ \hline 0 & \vdots & \ddots & \\ \hline \vdots & & & \end{array}\right) }_{\mathcal{T}_1}
 -\frac{1}{|\ccs_1|} \underbrace{F_{1,2,1}}_{\mathcal{T}_2}.$$
This completes the proof.
%The subspace $\mathcal{L}^\perp$ is define by matrices of the form
\end{proof}

In view of Proposition \ref{thm:description_of_lperp}, to bound the inner product between vectors  in $\tilde{\mathcal{K}}^\perp$ and $\tilde{\mathcal{L}}^\perp$ respectively, it suffices to bound inner product between $ (i) \ \tilde{\mathcal{K}}^\perp$  and $\mathcal{T}_1$ and $ (ii) \ \tilde{\mathcal{K}}^\perp$  and $\mathcal{T}_2$.


Next, we describe a result that shows how incoherence computation for (a small number of) orthogonal subspaces can be put together to obtain incoherence computations.  %In the following, $\mathcal{S}$ and $\mathcal{T}$ are generic subspaces.

\begin{lemma}\label{thm:orthogonalsubspace_incoherence}  Let  $\{ \mathcal{T}_i \}_{i=1}^{r}$ be orthogonal subspaces of $\R^d$. Consider $s\in \R^d$ such that 
%Then, 
%$\mathcal{T}$ such that $\mathcal{T} = \oplus \mathcal{T}_i$.  Suppose for each $i$ we have 
$$
| \langle s,t_i \rangle | \leq \epsilon \| s \|_2 \| t_i \|_2 \quad \text{ for all }  t_i \in \mathcal{T}_i.
$$
Then, for $t\in \oplus_i \mathcal{T}_i $ we have that 
$$
| \langle s,t \rangle | \leq \epsilon \sqrt{r}  \| s \|_2 \| t \|_2.
$$
\end{lemma}

\begin{proof}[Proof of Lemma \ref{thm:orthogonalsubspace_incoherence}]
For any  $t= \sum t_i\in \oplus \mathcal{T}_i $
%Let $t \in \mathcal{T}$, and write $t_i = P_{\mathcal{T}_i}(t)$.  %Without loss of generality, we may assume that $\|s\|_2=1$.  
%Note that since $\mathcal{T} = \oplus \mathcal{T}_i$ we have $t = \sum t_i$.  
we have 
$$| \langle s, t \rangle | = | \langle s, \sum_{i=1}^r t_i \rangle | \leq \sum_{i=1}^r | \langle s, t_i \rangle | \leq \epsilon \sum_{i=1}^r \| s \|_2 \| t_i \|_2 \leq \epsilon \sqrt{r} \|s\|_2(\sum_{i=1}^r \| t_i \|_2^2)^{1/2} = \epsilon\sqrt{r}\|s\|_2  \| t \|_2,$$
where   the second last inequality follows from Cauchy-Schwarz, while the last equality uses the fact that $\mathcal{T} = \oplus \mathcal{T}_i$.
\end{proof}


%\textbf{Some prepartory results.}  
\paragraph{Bounding inner product between $\tilde{\mathcal{K}}^\perp$ and $\mathcal{T}_2$.}  Define the column vectors $\{\bf_i\}_{i=1}^{n}$ by
$$
\bf_i = n \be_i - \be = (\ldots, \underbrace{-1}_{i-1},\underbrace{n-1}_{i}, \underbrace{-1}_{i+1},\ldots)^T \in \mathbb{R}^{n}.
$$
Note that 
$$ \bf_i \be^T=\begin{pmatrix} f_i & \ldots & f_i\end{pmatrix} \text{ and } \be \bf_i^T=\begin{pmatrix} f_i^T \\ \vdots \\ f_i^T\end{pmatrix}.$$
With a slight abuse of notation, we define the subspace of matrices in $\mathbb{R}^{n_1 \times n_2}$ 
$$
\mathcal{N}_{C} ={\rm span}( \bf_1 \be^T,\ldots,  \bf_n \be^T ) \qquad \text{ and }  \qquad \mathcal{N}_{R}={\rm span}( \be \bf_1^T, \ldots, \be \bf_n^T ).
$$
The abuse of notation arises the vectors $\bf_i$ in the definitions of $\mathcal{N}_{C}$ and $\mathcal{N}_{R}$ are different.  Note that because $\bf_i^T \be = 0$, the subspaces $\mathcal{N}_{C}$ and $\mathcal{N}_{R}$ have dimensions $n_1-1$ and $n_2-1$ respectively and are orthogonal.  
%We define 
%$$
%\mathcal{N}_{C+R} = \mathcal{N}_{C} \oplus \mathcal{N}_{R}.
%$$
%
$\mathcal{N}_{C}$ and $\mathcal{N}_{R}$ are relevant for our problem as   the block off-diagonal entries of any matrix in  $\mathcal{T}_{2}$ belong to %$\mathcal{N}_{C+R}$.
$ \mathcal{N}_{C} \oplus \mathcal{N}_{R}.$

\begin{lemma} \label{thm:ncnr_space_incoherence}
Let $K \in \mathbb{R}^{n_1\times n_2}$  such that each row  has at most $c n_2$ non-zero entries for some  $0 \leq c \leq 1$. Then, for any $L \in \mathcal{N}_{C}$   we have that 
$$| \langle K, L \rangle | \leq \sqrt{c} \| L \|_F \| K \|_F.$$
 The same conclusion holds  if each column  of $K$ has at most $c n_1$ non-zero entries and   $L\in \mathcal{N}_R$. 
\end{lemma}

\begin{proof} [Proof of Lemma \ref{thm:ncnr_space_incoherence}]
Since $L \in \mathcal{N}_{C}$, it has constant columns.  Suppose its column entries are $(\theta_1,\ldots,\theta_{n_1})$, i.e., $L_{x,y}=\theta_x$  for all $y$. We then have 
$$\langle K, L \rangle = \mathrm{tr}(KL^T) = \sum_{x} (\sum_{y} K_{x,y} L_{x,y}) = \sum_{x} \theta_x \sum_{y} K_{x,y} {\leq} \sum_{x} \theta_x \sqrt{cn_2} (\sum_{y} K_{x,y}^2)^{1/2},$$  where the last inequality follows from  Cauchy-Schwarz, and since   there are at most $c n_2$ non-zero entries in each column of $K$.  Finally, we  have 
$$\sqrt{cn_2} \sum_{x} \theta_x  (\sum_{y} K_{x,y}^2)^{1/2} \leq \sqrt{cn_2} (\sum_x \theta_x^2)^{1/2} (\sum_{x,y} K_{x,y}^2)^{1/2} = \sqrt{c} \|L\|_F \|K\|_F,$$
 where in the last equality we use  that $\|L\|_F = \sqrt{n_2}(\sum \theta_i^2)^{1/2}$. % Essentially the same proof gives us the result for $\mathcal{N}_{R}$.
\end{proof}



\begin{corollary} \label{thm:ncnr_combined}
Consider   $K \in \mathbb{R}^{n_1\times n_2}$ where  each column has at most $c n_1$ non-zero entries, and each row has at most $c n_2$ non-zero entries for some  $0 \leq c \leq 1$.  For any  $L \in \mathcal{N}_{C}\oplus \mathcal{N}_{R}$  we have: 
$$| \langle K, L \rangle | \leq \sqrt{2c} \| L \|_F \| K \|_F.$$
\end{corollary}

\begin{proof} [Proof of Corollary \ref{thm:ncnr_combined}]
 By Lemma \ref{thm:ncnr_space_incoherence} for  $ L \in \mathcal{N}_C,$ or $ L\in  \mathcal{N}_R$ we have that
$$| \langle K, L \rangle | \leq \sqrt{c} \| L \|_F \| K \|_F.$$
As $\be^T \bf_j=0$,  $\mathcal{N}_{C}$ and $\mathcal{N}_{R}$ are orthogonal subspaces. 
The result follows by Lemma \ref{thm:orthogonalsubspace_incoherence}.
\end{proof}

\begin{lemma} \label{thm:incoherence_efsum}
Suppose $G$ is a graph satisfying the $\cscc$ property.  %Let $L \in \mathbb{R}^{|\V| \times |\V|}$ be a matrix so that every $(\ccs_x,\ccs_y)$-th block, $x\neq y$, belongs to $\mathcal{N}_{C} \oplus \mathcal{N}_{R}$.  
Then, for any  $K \in \tilde{\mathcal{K}}^\perp$ and $L\in \mathcal{T}_2$ we  have that
$$| \langle K,L \rangle| \leq \sqrt{2c} \|K\|_F \|L\|_F.$$
\end{lemma}

\begin{proof} As $K \in \tilde{\mathcal{K}}^\perp$, its diagonal blocks are zero, and  also, entries corresponding to non-edges of $G$ are zero.  Moreover, as $G$ has the   $c$-SCC property, each  row (and column) of $K$  has at most $cn$ non-zero entries.
Let the block matrices be indexed by $(x,y)$, and let $L_{xy}$ and $K_{xy}$ denote the $xy$-th block.  Recall that  the block off-diagonal entries of any matrix in  $\mathcal{T}_{2}$ belong to %$\mathcal{N}_{C+R}$.
$ \mathcal{N}_{C} \oplus \mathcal{N}_{R}.$ By Corollary~\ref{thm:ncnr_combined} we have $| \langle K_{xy}, L_{xy} \rangle | \leq \sqrt{2c} \| L_{xy} \|_F \| K_{xy} \|_F$.  By summing over the blocks and by applying Cauchy-Schwarz, we have 
$$
 \begin{aligned}
  | \langle K , L \rangle | &  \leq \sum_{x,y} |\langle K_{xy} , L_{xy} \rangle |= \sum_{x\ne  y} |\langle K_{xy} , L_{xy} \rangle | \\
& \leq \sqrt{2c} \sum_{x\ne y} \| K_{xy} \|_F \| L_{xy} \|_F \leq \sqrt{2c} (\sum_{x,y} \| K_{xy} \|_F^2)^{1/2} (\sum_{x,y} \| L_{xy} \|_F^2)^{1/2} = \sqrt{2c} \| K \|_F \| L \|_F.
\end{aligned} $$
\end{proof}

\paragraph{Bounding  the inner product between $\tilde{\mathcal{K}}^\perp$ and $\mathcal{T}_1$.}  This case is easier and the required result is given in the next lemma. 
\begin{lemma}\label{thm:incoherence_Espan}
Suppose $G$ is a graph satisfying the $\cscc$ property.  Then, for any  $K \in \tilde{\mathcal{K}}^\perp$ and $L\in \mathcal{T}_1$ we  have that
$$|\langle K, L \rangle | \leq \sqrt{c} \| K \|_F \| L \|_F .$$
%
%Let $K \in \tilde{\mathcal{K}}^\perp$, and let $L$ be a matrix of the form
%$$L = \left( \begin{array}{c|c|c|c|c} * & \theta_{1,2} E & \theta_{1,3} E & \ldots & \theta_{1,n} E \\ \hline \theta_{2,1} E & * & \theta_{2,3} E & \ldots & \theta_{2,n} E \\ \hline \theta_{3,1} E & \theta_{3,2} E & & & \\ \hline \vdots & \vdots & & & \\ \hline \theta_{n,1} E & \theta_{n,2} E & & & * \end{array}\right).$$
%Then 
%The entries marked by an asterisk $*$ are permitted to be arbitrary.
\end{lemma}

\begin{proof}[Proof of Lemma \ref{thm:incoherence_Espan}]
Note that $K$ has no entries in each block diagonal.  Consider the $(i,j)$-th block matrix where $i \neq j$.  We denote the coordinates in this block by $\mathcal{B}_{i,j}$.  Then
$$
\sum_{x,y \in \mathcal{B}_{i,j}} K_{x,y} L_{x,y} = \theta_{i,j} \left( \sum_{x,y \in \mathcal{B}_{i,j}} K_{x,y} \right) \leq \sqrt{c |\ccs_i| |\ccs_j|} \theta_{i,j} \left( \sum_{x,y \in \mathcal{B}_{i,j}} K_{x,y}^2 \right)^{1/2}.
$$
The last inequality follows from Cauchy-Schwarz, and by noting that $K$ has at most $c |\ccs_i| |\ccs_j|$ non-zero entries within the block $\mathcal{B}_{i,j}$.  Then by summing over the blocks $(i,j)$ we have
$$
|\langle K,L \rangle| \leq \sum_{i,j} \sqrt{c|\ccs_i| |\ccs_j|} \theta_{i,j} ( \sum_{x,y \in \mathcal{B}_{i,j}} K_{x,y}^2 )^{1/2} \leq \sqrt{c} (\sum_{i,j} |\ccs_i| |\ccs_j|\theta_{i,j}^2)^{1/2} ( \sum_{x,y} K_{x,y}^2 )^{1/2}.
$$
The last inequality follows from Cauchy-Schwarz.  Now note that $( \sum_{x,y} K_{x,y}^2 )^{1/2} = \|K\|_F$, and that $(\sum_{i,j} |\ccs_i| |\ccs_j| \theta_{i,j}^2)^{1/2} = \|L\|_F $, from which the result follows.
\end{proof}


%\subsection{Main incoherence result}  This is the main incoherence result we need.

%\begin{proposition} 
% \label{thm:incoherence_1}
%Let $G$ be a graph satisfying the $c$-neighborly property.  For any  $\tilde{K} \in \mathcal{K}^\perp$ and  $\tilde{L} \in \mathcal{L}^\perp$ we have that  
%$$
%| \langle \tilde{K}, \tilde{L} \rangle | \leq 2\sqrt{c}\|\tilde{K}\|_F  \|\tilde{L}\|_F .
%$$
%\end{proposition}



Finally, we are ready to prove  Theorem \ref{thm:incoherence_1}.


\paragraph{Proof of Theorem \ref{thm:incoherence_1}.} Consider   $\tilde{K} \in \tilde{\mathcal{K}}^\perp$ and  $\tilde{L} \in \tilde{\mathcal{L}}^\perp$.   By Proposition \ref{thm:description_of_lperp} we have that 
$\tilde{\mathcal{L}}^\perp=\mathcal{T}_1\oplus \mathcal{T}_2$ and let   $\tilde{L} =\tilde{L}_1+\tilde{L}_2$, where $\tilde{L}_i\in \mathcal{T}_i$. 


\medskip 
    \textbf{Part $(i)$.} 
By % Note that the linear span of matrices in $\mathcal{T}_1$ have the form in
    Lemma \ref{thm:incoherence_Espan} we have that  
$$| \langle \tilde{K}, \tilde{L}_1 \rangle | \leq \sqrt{c}\|\tilde{K}\|_F  \|\tilde{L}_1 \|_F.$$ 

%Let $\tilde{L}_2 \in \mathcal{T}_2$ be arbitrary. 
 Let $\tilde{L}_{2,\mathrm{off}}$ be the block off-diagonal component of $\tilde{L}_2$ and $\tilde{L}_{2,\mathrm{diag}}$ be the block diagonal component of $\tilde{L}_2$ so that $\tilde{L}_2 = \tilde{L}_{2,\mathrm{off}} + \tilde{L}_{2,\mathrm{diag}}$.  
By definition, the matrix $\tilde{L}_{2,\mathrm{off}}$ is zero on the diagonal  blocks  and  each block belongs to $\mathcal{N}_{C} \oplus \mathcal{N}_{R}$.  Hence by Lemma \ref{thm:incoherence_efsum}, we have 
$$| \langle \tilde{K}, \tilde{L}_{2,\mathrm{off}} \rangle | \leq \sqrt{2c} \| \tilde{K} \|_F \| \tilde{L}_{2,\mathrm{off}} \|_F.$$
Moreover, as the diagonal blocks of $\tilde{K}$ are zero we  have that 
$$\langle \tilde{K}, \tilde{L}_{2,\mathrm{diag}} \rangle =0.$$
Putting everything together,
$$| \langle \tilde{K}, \tilde{L}_2 \rangle | = | \langle K, \tilde{L}_{2,\mathrm{off}} \rangle | \leq \sqrt{2c}  \| \tilde{K} \|_F \| \tilde{L}_{\mathrm{2,off}} \|_F \leq \sqrt{2c} \| \tilde{K} \|_F \| \tilde{L} \|_F,$$   
where the last inequality follows by noting that $\| L \|_F^2 = \| L_{\mathrm{off}} \|_F^2 + \| L_{\mathrm{diag}} \|_F^2$.

Finally, by Proposition \ref{thm:description_of_lperp}, the subspaces $\mathcal{T}_1$ and $\mathcal{T}_2$ are orthogonal.  Hence by Lemma \ref{thm:orthogonalsubspace_incoherence} applied to the subspaces $\mathcal{T}_1$ and $\mathcal{T}_2$ we have $| \langle \tilde{K}, \tilde{L} \rangle | \leq \sqrt{2} \times \sqrt{2c} \|\tilde{K}\|_F \|\tilde{L}\|_F$.
%Let $\tilde{L} = \sum \theta_{x,y} L_{x,y}$ where $\{ L_{x,y} \}$ are basis matrices specified in [REF].  Fix an index $(x,y)$, and observe that
%$$\langle \tilde{K}, \tilde{L}_{x,y} \rangle = \mathrm{tr}(\tilde{K} \tilde{L}_{x,y}) = 2 \sum_{t=1}^{n} \tilde{K}_{x,(y-1)n+t} \leq  2 \sqrt{s} (\sum_{t=1}^{n} \tilde{K}_{x,(y-1)n+t}^2 )^{1/2}.$$ 
%We obtain the second equality by noting that the entries of $K$ are permitted to be non-zero only when the corresponding entries of $L$ are $1$.  We obtain the factor $2$ by accounting for the column sum.  Also, we obtain the last inequality by the Cauchy-Schwarz, and by noting that there are at most $s$ instances $\tilde{K}_{x,(y-1)n+t} \neq 0$ in the range $1\leq t \leq n$ by the ABCD-$s$ property.  
%Now, we perform the sum over all indices $(x,y)$.  We have
%\begin{equation*}
%\begin{aligned}
%| \langle \tilde{K}, \sum_{x,y} \theta_{x,y} L_{x,y} \rangle | ~\leq~ & 2 \sqrt{s} \sum_{x,y} |\theta_{x,y}| (\sum_{t=1}^{n} \tilde{K}_{x,(y-1)n+t}^2 )^{1/2} \\
%~\leq~ & 2 \sqrt{s} \big( \sum_{x,y} \theta_{x,y}^2  \big)^{1/2} \big( \sum_{x,y} \sum_{t=1}^{n} \tilde{K}_{x,(y-1)n+t}^2 )^{1/2} \big)^{1/2} = 2 \sqrt{s} \big( \sum_{x,y} \theta_{x,y}^2  \big)^{1/2}  \| \tilde{K} \|_F.
%\end{aligned}
%\end{equation*}
%The last inequality follows by noting that all indices of the matrix $\tilde{K}$ are counted exactly once.  We further note that $\| L_{x,y} \|_F  \geq ??n$.  [THIS IS THE SOURCE OF THE ERROR.  I NEED TO SHOW THAT THE NORM OF THE MATRIX IS BOUNDED BELOW BY $n$]
 % Thus $|\langle \tilde{K}, \tilde{L} \rangle | \leq (2\sqrt{s}/n) \| \tilde{L} \|_F$, as required.
%
%
%\begin{proposition}  \label{thm:incoherence_2}
%Let $G$ be a graph satisfying the $c$-neighborly property.  For any  $\tilde{L} \in \mathcal{L}^\perp$ we have 
% $$\|P_{\mathcal{K}^\perp}(\tilde{L})\|_F \leq (2 \sqrt{c})^{1/2} \|\tilde{L}\|_F.$$
%\end{proposition}
\medskip 

 \textbf{Part $(ii)$.} Note that
$$
\| P_{\mathcal{K}^\perp}(\tilde{L}) \|_F^2 = \langle P_{\mathcal{K}^\perp}(\tilde{L}), P_{\mathcal{K}^\perp}(\tilde{L}) \rangle = \langle P_{\mathcal{K}^\perp} (P_{\mathcal{K}^\perp}(\tilde{L})), \tilde{L} \rangle = \langle P_{\mathcal{K}^\perp}(\tilde{L}), \tilde{L} \rangle,
$$
where the second and third equalities  follow  from the fact that orthogonal projections are  self-adjoint and idempotent. 
 %to thenseand the third equality follows from the fact that $P_{\mathcal{K}^\perp}$ is a projection operator.  
 Since $P_{\mathcal{K}^\perp}(\tilde{L}) \in \mathcal{K}^\perp$, by Proposition \ref{thm:incoherence_1}, we have 
$$|\langle P_{\mathcal{K}^\perp}(\tilde{L}), \tilde{L} \rangle| \leq 2\sqrt{c} \|P_{\mathcal{K}^\perp}(\tilde{L})\|_F \|\tilde{L}\|_F \leq 2\sqrt{c} \|\tilde{L}\|_F^2,$$ from which the result follows.\qed
%Let $\tilde{L} = \sum_{x,y} \theta_{x,y} L_{x,y}$.  Consider each row..  For each fixed $\theta$, we have $\sum \tilde{L}^2 \leq s \theta^2$, because each entry of $L$ is equal to one, and there are at most $s$ non-zero entries.  Then summing across all $\theta$ we have $\|\tilde{L}\|_F^2 \leq s \sum \theta^2 \leq (s/n^2) \| \tilde{L} \|_F^2$.


%}



\subsection{Proof of Result \ref{thm:result2}} 

\begin{theorem} \label{thm:abcd-exact-recovery}
Suppose $G$ satisfies the $\cscc$ property for some $c < \min \{ \frac{1}{4} (\frac{\min_l 1/|\ccs_l|}{\sum_l 1/|\ccs_l|})^2,\frac{1}{100} \}$.  Then $\ac$ is the unique optimal solution to \eqref{eq:lovasz_lambdamax}.
\end{theorem}

Following Proposition \ref{thm:Unequalsizecanonicalmatrix_spectrum}, the largest singular value of $\zz$ is $\sigma_{\max}(\zz) = \sum_l 1/|\ccs_l|$, while the smallest non-zero singular value of $\zz$ is $\sigma_{\ks}(\zz) = \min 1/|\ccs_l|$.  Hence an alternative way to express the lower bound in Theorem \ref{thm:abcd-exact-recovery} in terms of a condition number-type of quantity 
$$\frac{ \min_l 1/|\ccs_l| }{ \sum_l 1/|\ccs_l| }= \frac{ \sigma_{\ks}(\zz) }{ \sigma_{\max}(\zz) }.$$

\begin{proof}
As a reminder, our candidate dual certificate is $\zc = P_{\tilde{\mathcal{K}} \cap\tilde{\mathcal{L}}} (\zz)$ as defined earlier in \eqref{eq:euclideanprojection}.  By Theorem \ref{thm:extremepoint}, $\ac$ is an extreme point of the feasible region.  Thus, by  Theorem \ref{thm:constraintqualification}, we need to show that $\zc$ satisfies conditions \eqref{cond1}--\eqref{cond3}.  Conditions \eqref{eq:supportcondition} and \eqref{eq:complementaryslackness} are satisfied by construction.  As we noted in the proof of Theorem \ref{thm:disjointcliques_uniquerecovery}, conditions \eqref{eq:traceone} and   \eqref{eq:objvalks} are taken care of by scaling.  As such, it remains to show that $Z'$ is PSD and that ${\rm range}(Z')={\rm ker}(\xc)$.

%Then $Z_{\mathcal{C}} := P_{\mathcal{H}} (\zz)$ satisfies (i) $Z_{\mathcal{C}} \succeq 0$, (ii) \eqref{eq:complementaryslackness}, (iii) \eqref{eq:supportcondition}, and $\mathrm{rank}(Z_{\mathcal{C}}) = |V| - (k-1)$.

\paragraph{Simplification:}  We begin by showing that it suffices to prove the inequality $\sigma_{\max}(\zc - \zz) < \sigma_{\ks}(\zz)$ (as a reminder, $\sigma_{\ks}(\zz)$ is the smallest non-zero singular value of $\zz$).  To see why this is sufficient, let $\bv \in \mathbb{R}^{|\V|}$ and consider its direct sum decomposition with respect to $\mathcal{J}$; i.e., $\bv = \bv_{\mathcal{J}} + \bv_{\mathcal{J}^\perp}$.  Since $\zc \in \tilde{\mathcal{L}}$, we have by Lemma \ref{csexpnaded}, part (4) that   
$$
\bv^{T} \zc \bv = \bv^{T}_{\mathcal{J}} \zc \bv_{\mathcal{J}} = \bv^{T}_{\mathcal{J}} \zz \bv_{\mathcal{J}} + \bv^{T}_{\mathcal{J}} (\zc - \zz) \bv_{\mathcal{J}}.
$$
By Lemma \ref{thm:Unequalsizecanonicalmatrix_spectrum}, $\zz$ restricted on $\mathcal{J}$ is positive definite with smallest eigenvalue at least $\sigma_{\ks}(\zz)$, and hence 
$$
\bv^{T}_{\mathcal{J}} \zz \bv_{\mathcal{J}} \geq \sigma_{\ks}(\zz) \|\bv_{\mathcal{J}}\|_2^2.
$$
On the other hand, the inequality $\sigma_{\max}(\zc-\zz) < \sigma_{\ks}(\zz)$ implies
$$\bv^{T}_{\mathcal{J}} (\zc - \zz) \bv_{\mathcal{J}} <  \sigma_{\ks}(\zz) \|\bv_{\mathcal{J}}\|_2^2.$$
This means that $\bv^{T} \zc \bv \geq 0$ for all $\bv \in \mathbb{R}^{|\V|}$, which means that $\zc$ is PSD.  
  
Finally, we need to show that  ${\rm range}(Z')={\rm ker}(\xc)$. By definition, we have that $Z'\in \tilde{\mathcal{L}}$, so by Proposition \ref{csexpnaded}, we have that 
  $$ {\rm range}(Z')\subseteq \mathcal{J}={\rm ker}(\xc).$$
  For the converse inequality, we show that ${\rm ker}(Z')\subseteq \mathcal{J}^\perp$. For this, take $\bv\in {\rm ker}(Z')$. Let  $\bv = \bv_{\mathcal{J}} + \bv_{\mathcal{J}^\perp}$ and assume  that $\bv_{\mathcal{J}} \ne 0$. Then,  we would have
  $$0=\bv^{T} \zc \bv =\bv^{T}_{\mathcal{J}} \zc \bv_{\mathcal{J}}>0,$$
  leading to a contradiction. 
  
%  If $v_{\mathcal{J}} \ne 0$ it follows from \eqref{xsdcvdfv} that 
%  
%  Moreover, the inequality is strict whenever $v_{\mathcal{J}}$ is non-zero, and hence 
%$$\dim (\ker(\zc))=\dim \mathcal{J}^\perp=\dim {\rm ran}(X_\mathcal{C}).$$ 
%Finally, as 
%$${\rm ran}(X_\mathcal{C}) \subseteq {\rm ker}(\zc) $$
%it follows that  ${\rm ran}(X_\mathcal{C}) ={\rm  ker}(\zc).$

%the matrix $\zc$ is positive definite when restricted to the subspace $\mathcal{J}^\perp$, and thus has rank $|V|-(k-1)$.


\paragraph{Expressing  $\zz - \zc$ via the KKT conditions.} %First, we express the difference $\zz - \zc$ via the KKT conditions.  
The first-order optimality condition corresponding to \eqref{eq:euclideanprojection} gives 
\begin{equation} \label{eq:foc}
\zz - \zc = \tilde{L}  + \tilde{K},
\end{equation}
where $\tilde{L}$ lies in the normal cone of $\tilde{\mathcal{L}}$ at $Z'$ and $\tilde{K}$ lies in the normal cone of $\mathcal{K}$ at $Z'$. Here 
we used the fact  that the normal cone to an intersection is the sum of the normal cones, e.g. see \cite[Corollary 23.8.1]{rockafellar}.  Moreover, as  the normal cone to a subspace is the orthogonal subspace, we have that  $\tilde{L} \in {\tilde{\mathcal{L}}}^\perp$ and $\tilde{K}\in \tilde{\mathcal{K}}^\perp$. First, by projecting \eqref{eq:foc} onto $\tilde{\mathcal{L}}$ we have
\begin{equation} \label{eq:projection_v1}
\zz - \zc = P_{\tilde{\mathcal{L}}}(\tilde{K}).
\end{equation}
%and by projecting onto $\tilde{\tilde{\mathcal{L}}}^\perp$ we have
%$$
%0 = N_{\tilde{\mathcal{L}}} + P_{\tilde{\tilde{\mathcal{L}}}^\perp} (N_{\mathcal{K}}).
%$$
This follows as  $\zz \in \tilde{\mathcal{L}}$, $\zc \in \tilde{\mathcal{K}} \cap \tilde{\mathcal{L}}$.  Second, by projecting \eqref{eq:foc} onto the subspace $\tilde{\mathcal{K}}^\perp$ we have
%$$
%P_{\mathcal{K}} (Z) - \zc  = P_{\mathcal{K}}(N_{\tilde{\mathcal{L}}}).
%$$
%and
\begin{equation} \label{eq:projection_v2}
P_{\tilde{\mathcal{K}}^\perp} (\zz)  = P_{\tilde{\mathcal{K}}^\perp}(\tilde{L}) + \tilde{K}.
\end{equation}
Combining  \eqref{eq:projection_v1} and \eqref{eq:projection_v2}, we have
\begin{equation} \label{eq:zhatx_firstbound}
\| \zz - \zc \|_F = \| P_{\tilde{\mathcal{L}}}(\tilde{K}) \|_F \leq \| \tilde{K} \|_F =  \| P_{\tilde{\mathcal{K}}^\perp} (\zz) - P_{\tilde{\mathcal{K}}^\perp}(\tilde{L}) \|_F \leq \| P_{\tilde{\mathcal{K}}^\perp} (\zz) \|_F + \| P_{\tilde{\mathcal{K}}^\perp}(\tilde{L}) \|_F,
\end{equation}
where  the first equality follows from \eqref{eq:projection_v1}, and the last  inequality follows from  the triangle inequality.  
%{\color{red} By Lemma \ref{thm:incoherence_2} $\| P_{\tilde{\mathcal{K}}^\perp}(N_{\tilde{\mathcal{L}}}) \|_F \leq (2\sqrt{s/n})^{1/2} \|N_{\tilde{\mathcal{L}}}\|_F$.  }

We next proceed to bound the terms $\|P_{\tilde{\mathcal{K}}^\perp} (\zz) \|_F$ and $  \| P_{\tilde{\mathcal{K}}^\perp}(\tilde{L}) \|_F$.


\paragraph{ Bound  the term $  \| P_{\tilde{\mathcal{K}}^\perp} (\zz) \|_F $.} We show that

%[First bound]:  First, we bound the term $  \| P_{\tilde{\mathcal{K}}^\perp} (\zz) \|_F $.  Specifically, we show that
\begin{equation}\label{csvdgb}
\| P_{\tilde{\mathcal{K}}^\perp}(Z^*) \|_F \leq \sqrt{c} \left(\sum_l 1/|\ccs_l|\right).
\end{equation}
The $(i,j)$-th entry of the matrix $P_{\tilde{\mathcal{K}}^\perp}(Z)$ is equal to $Z_{i,j}$ if $(i,j) \in \mathcal{E}$, and is equal to zero otherwise.  Consider the block corresponding to $(\ccs_x,\ccs_y)$, where $x\neq y$.  Each non-zero entry is $1/(|\ccs_x||\ccs_y|)$ and there are at most $c |\ccs_x||\ccs_y|$ entries.  Hence the sum of squares of the entries in this block is at most $c / (|\ccs_x||\ccs_y|)$.  We sum over the all indices $x$ and $y$ to obtain $\| P_{\tilde{\mathcal{K}}^\perp}(Z) \|_F^2  \leq \sum_{x,y} c / (|\ccs_x||\ccs_y|) \leq c (\sum_l 1/|\ccs_l|)^2$, from which the result follows.  (In the first inequality, recall that the block-diagonal entries of $P_{\tilde{\mathcal{K}}^\perp}(Z)$ are zero and thus do not contribute to the sum.)
%\end{proof}

\paragraph{Bound the term  $ \| P_{\tilde{\mathcal{K}}^\perp}(\tilde{L}) \|_F$.}  Setting   
 $$\epsilon = \langle \tilde{L} / \|\tilde{L}\|_F, \tilde{K} / \|\tilde{K}\|_F \rangle,$$
   it follows  from \eqref{eq:foc} that  
$$\| \zz - \zc \|_F^2 = \| \tilde{L} \|_F^2 + \| \tilde{K} \|_F^2 + 2 \epsilon\| \tilde{L} \|_F  \| \tilde{K} \|_F .$$  By the AM-GM inequality we have that 
$$ 2\| \tilde{L} \|_F   \| \tilde{K} \|_F\ge -\| \tilde{L} \|_F^2 - \| \tilde{K} \|_F^2.$$
Consequently, we get  
$$\| \zz - \zc \|_F^2 \geq (1-|\epsilon|)(\| \tilde{L} \|_F^2 + \| \tilde{K} \|_F^2)$$ and since $|\epsilon|\le 1$ we have 
\begin{equation}\label{csdvdfb}
\| \tilde{L} \|_F^2 \le \| \tilde{L} \|_F^2 + \| \tilde{K} \|_F^2 \leq \left(\frac{1}{1-|\epsilon|}\right) \| \zz - \zc \|_F^2.
\end{equation}
Combining \eqref{csdvdfb} with Theorem \ref{thm:incoherence_1} $ (ii)$   we have
\begin{equation}\label{xsdvdfv}
\| P_{\tilde{\mathcal{K}}^\perp}(\tilde{L}) \|_F \leq (2\sqrt{c})^{1/2} \|\tilde{L}\|_F \leq { (2\sqrt{c})^{1/2} \over \sqrt{1-|\epsilon|}} \| \zz - \zc \|_F.
\end{equation}

\paragraph{Concluding the proof.} %We are now in a position to conclude the result.  
We have established in \eqref{eq:zhatx_firstbound} that 
$$ \| \zz - \zc \|_F  \leq \| P_{\tilde{\mathcal{K}}^\perp} (\zz) \|_F + \| P_{\tilde{\mathcal{K}}^\perp}(\tilde{L}) \|_F,$$
which combined with \eqref{csvdgb} and \eqref{xsdvdfv}  shows that
\begin{equation}\label{cxvdbdf}
\| \zz - \zc \|_F \leq \sqrt{c} (\sum_l 1/|\ccs_l|) + { (2\sqrt{c})^{1/2}\over \sqrt{1-|\epsilon|}} \| \zz - \zc \|_F.
\end{equation}
%Since $c\leq 1/23$, we have $(2\sqrt{c})^{1/2} \leq 2^{1/2} 23^{-1/4}$.  
By Theorem \ref{thm:incoherence_1} $(i)$ we have that 
$$|\epsilon| =|\langle \tilde{L} / \|\tilde{L}\|_F, \tilde{K} / \|\tilde{K}\|_F \rangle| \leq 2 \sqrt{c},$$ so for $c\le 1/100$ we get 
%and since $c \leq 1/23$, we have $1/\sqrt{1-|\epsilon|} \leq 1/(1-2 \sqrt{1/23})^{1/2}$.  Hence, 
$${ (2\sqrt{c})^{1/2}\over \sqrt{1-|\epsilon|}}  \leq { 1\over 2}.$$
 Then,   \eqref{cxvdbdf} implies  that 
$$\| \zz - \zc \|_F \leq 2 \sqrt{c} (\sum_l 1/|\ccs_l|).$$
 Finally, using that  the Frobenius norm is always greater than the spectral norm, we have
$$
\| \zz - \zc \|_2 \leq \| \zz - \zc \|_F \leq 2 \sqrt{c} (\sum_l 1/|\ccs_l|),
$$
which is strictly smaller than $\sigma_{\ks}(\zz)$ whenever $c < \frac{1}{4} ^2 (\sigma_{\ks}(\zz)/\sigma_{\max}(\zz))^{2} $.
\end{proof}


\section{Recovery for planted clique cover instances} \label{sec:randomgraphs}


We conclude Result \ref{thm:result1} by applying Hoeffding's inequality onto the conclusions of Result \ref{thm:result2} directly.


\begin{theorem} \label{thm:uneq_random_cond}
Let $G$ be a random planted clique cover instance defined on the cliques $\{ \ccs_l \}_{l=1}^{\ks}$  where we introduce edges between cliques with probability $p$. 
%Let $G$ be a random graph specified by $\{ \mathcal{C}_l \}_{l=1}^{k}$. 
If $p < c$ then $G$ is $c$-neighborly with probability greater than 
$$
1 - \sum_{i=1}^{\ks} {|\ccs_i| \sum_{j \neq i}\exp(-2|\ccs_j|(c-p)^2}).
$$
\end{theorem}

\begin{proof}
Let $X_{ia,jb}$ be a binary variable equal to 1 if vertices $a \in \ccs_i$ and $b\in \ccs_j$ are connected, and equal to 0 otherwise.  First we have $\mathbb{E}[\sum_{b \in \ccs_j}{X_{ia, jb}}] = \sum_{b \in \ccs_j}{\mathbb{E}[X_{ia, jb}]} = |\ccs_j|p$.
  %  $(\sum_{m=1}^{i-1}{|\mathcal{C}_m|} + a, \sum_{m=1}^{j-1}{|\mathcal{C}_m|} +b) \in \mathcal{E}$, 
    %equals to 0 otherwise. Fix the $a$-th vertex in the $i$-th clique and clique $j \neq i$, consider all edges connecting vertex $\sum_{m=1}^{i-1}{|\mathcal{C}_m|} + a$ to each vertices $b$ in cluster $j$, where $1\leq b \leq \mathcal{C}_j$. Since the edges are generated in an i.i.d. fashion, $X_{ia,jb}\sim Bern (p)$.
By Hoeffding's inequality we have
\begin{equation}\label{ineq:Hoeffding_U}
\mathbb{P} \Big(\sum_{b \in \ccs_j}{X_{ia, jb}}  \geq t_j + |\ccs_j|p \Big)  = \mathbb{P} \Big(\sum_{b \in \ccs_j}{X_{ia, jb}} - \mathbb{E}[\sum_{b \in \ccs_j}{X_{ia, jb}}] \geq t_j \Big) \leq \exp ( -2t_j^2 / |\ccs_j| ).
\end{equation}
Using a union bound we get that 
$$\mathbb{P} \Big( \bigcap _{i, a\in \ccs_i, j \ne i } \Big( \sum_{b}{X_{ia, jb}} <t_j + |\ccs_j|p \Big) \Big) \ge 1- \sum_i|\ccs_i|
\sum_{j\ne i}\exp(-2t_j^2/|\ccs_j|).$$
Finally, given that $p<c$, we set $t_j=|\ccs_j|(c-p)>0$ to get the desired result. 
\end{proof}



\section{Numerical comparison to alternative  techniques } \label{subsec:related worlk}
In this section, we assess the effectiveness of the Lov\'asz theta function for recovering planted clique cover instances by comparing it to alternative techniques in the literature.  Specifically, we compare our method with the following approaches:

\begin{itemize} 
\item {\em Community Detection approach:}  We frame the clique cover problem as an instance of community detection, and use the method proposed by Oymak and Hassibi~\cite{OH:11}.

\item {\em $k$-Disjoint-Clique approach: } We consider the clique cover problem as an instance of the $k$-disjoint-clique problem, and apply the method presented by Ames and Vavasis \cite{AV:14}.

\item {\em Subgraph Isomorphism approach:}  We view the problem as an instance of  subgraph isomorphism, and apply the framework proposed by Candogan and Chandrasekaran \cite{CC:18}.
\end{itemize}


%In this section we review some prior work related to ours, and we compare the performance of these methods with ours at recovering clique covers.

\paragraph{Community Detection/Graph clustering.}  Consider a network where the presence of an edge (or possibly the size of its edge weight) models the strength of association between two nodes.  The graph clustering task concerns grouping these nodes into subsets -- often called {\em communities} -- such that the strength of association between pairs of nodes from the same subset is significantly stronger that the strength of association between pairs belonging to different subsets.  The task of community detection finds application in various fields such as sociology, physics, epidemiology, and more \cite{For:10,New:06}. The term {\em community detection} is often used interchangeably with {\em graph clustering}. %  In what follows, we use the term `community detection' because the bulk of prior work that is most relevant to ours tend to make this choice.

Oymak and Hassibi \cite{OH:11}  (see also \cite{CJSX:14}) present a  method for  detecting communities in an undirected  graph  $\mathcal{G} = (\mathcal{V},\mathcal{E})$  by solving the 
following  SDP:
 %be an undirected graph, ax. \cite{OH:11} 
\begin{equation} \label{eq:sparselowrank_comdetect}
\begin{aligned}
\min_{S,L} ~~ \lambda \| S \|_{1} +  \| L \|_{*} \quad \mathrm{s.t.} \quad S+L = A, \quad 0 \leq L \leq J.
\end{aligned}
\end{equation}
Here     $A$ is  the adjacency matrix of $G$,  $\lambda>0$ is a tuning parameter, $\| X \|_1 = \sum_{i,j} |X_{i,j}|$ is the L1-norm over the matrix entries, while $\|X\|_* = \sum_i \sigma_i (X)$ is the matrix nuclear norm. %(or the Schatten-1 norm).  

  Let $(\hat{S},\hat{L})$ be the optimal solution to this SDP.   The intuition behind the % formulation 
  SDP \eqref{eq:sparselowrank_comdetect} % proposed by Oymak and Hassibi \cite{OH:11} 
is that it performs a matrix deconvolution in which the adjacency matrix $A$ is separated as the %linear 
sum of a low-rank component $\hat{L}$ and a sparse matrix $\hat{S}$.  %It is hoped that 
The hope is that (possibly by tuning the parameter~$\lambda$) the low-rank component takes the form $\hat{L} = \sum_{l} \bone_{\ccs_l} \bone_{\ccs_l}^T$, where $\ccs_l$ are the indicator vectors for {\em disjoint} communities, while the sparse component $\hat{S}$ captures noise that is (hopefully)~sparse.  

The specific form in \eqref{eq:sparselowrank_comdetect} builds upon a long sequence of ideas, originating from the observation that the L1-norm is effective at inducing sparse structure, while the matrix nuclear norm is effective at inducing low-rank structure \cite{CanPla:11,RFP:10}.  Subsequent work proposed the formulation \eqref{eq:sparselowrank_comdetect} for deconvolving matrices as the linear sum of a sparse matrix and a low-rank matrix \cite{CSPW:11,WPMGR:09}; in particular, Chandarasekaran et. al. \eqref{eq:sparselowrank_comdetect} show that penalization with the L1-norm plus the nuclear-norm succeeds at recovering both components so long as these components satisfy a {\em mutual incoherence} condition.  The formulation in \eqref{eq:sparselowrank_comdetect} is a reasonable approach for obtaining clique covers because one can view cliques as clusters.  
%
%\paragraph{Planted clique / sub-graph problems.}  A different body of work that closely relates to ours is that of finding certain graphs embedded within larger graphs.   The most prominent problem within this body is perhaps the {\em planted clique problem}, which decides if there exists a clique of size $k$ embedded in a larger graph $\mathcal{G}$.  As we noted earlier (see discussion on $\alpha(G)$ and $\omega(G)$), this problem is NP-complete.  To this end, there is a line of work that develop computationally tractable algorithms and show that these succeed at finding cliques in suitably random instances. 
%

% The adjacency matrix of a clique $\bone_{\mathcal{C}} \bone_{\mathcal{C}}^T$ has rank equals to one.  Hence a reasonable approach to finding a clique is to seek a (non-zero) matrix of small rank and whose entries $(i,j)$-th entry is zero if $(i,j)$ is not an edge.  To this end, Ames and Vavasis propose a method based on relaxing the rank constraint to a nuclear norm objective \cite{AV:11}
%\begin{equation} \label{eq:av_clique}
%\begin{aligned}
%\min ~~ & ~~ \| X \|_{*} \\
%\mathrm{s.t.} ~~ & ~~ \langle X, J \rangle \geq k^2 \\
%& ~~ X_{i,j} = 0 \quad (i,j) \notin \mathcal{E}, i \neq j .
%\end{aligned}
%\end{equation}
%In particular, the formulation \eqref{eq:av_clique} hopes to recover $\hat{X} = \bone_{\mathcal{C}} \bone_{\mathcal{C}}^T$ as the optimal solution.
%
%The ideas in \cite{AV:11} were subsequently generalized to more general graphs \cite{AV:14,CC:18}. 
\paragraph{$k$-Disjoint-Clique Approach.} In the $k$-disjoint-clique problem,  the goal is to find $k$ disjoint cliques (of any size) that cover as many nodes of the graph as possible.  It is worth noting that in this particular setting, the number of cliques $k$ is constant and is determined by the user as a parameter. Ames and Vavasis  \cite{AV:14}  introduce the following SDP for this problem:
\begin{equation} \label{eq:av}
\begin{aligned}
\max ~~ & ~~ \langle X, J \rangle \\
\mathrm{s.t.} ~~ & ~~ X \bone \leq \bone \\
& ~~ X_{i,j} = 0 \quad (i,j) \notin \mathcal{E}, i \neq j \\
& ~~ \langle X, I \rangle = k \\
& ~~ X \succeq 0.
\end{aligned}
\end{equation}



\paragraph{Subgraph isomorphism.}  Candogan and Chandrasekaran \cite{CC:18} investigate the problem of finding {\em any} subgraph embedded within a larger graph, known as the {\em subgraph isomorphism problem}.  A natural approach is to search over matrices obtained by conjugating the adjacency matrix of the subgraph with the permutation group, leading to a quadratic assignment problem instance.  The basic idea in \cite{CC:18} is to relax this set by conjugating with the orthogonal group $O(|V|)$ instead.  The resulting convex hull is known as the Schur-Horn orbitope \cite{SSS:11}, and it admits a tractable description via semidefinite programming \cite{DW:09}.  In our context, the adjacency matrix of the subgraph (technically not a proper subgraph) is $\sum_{l} \bone_{\ccs_l} \bone_{\ccs_l}^T$, and the resulting SDP is
\begin{equation} \label{eq:sh}
\begin{aligned}
\max ~~ & ~~ \langle A, X \rangle \\
\mathrm{s.t.} ~~ & ~~ X_{i,j} = 0 \quad (i,j) \notin \mathcal{E}, i \neq j \\
& ~~ X = n \cdot Z_1 + 0 \cdot Z_0 \\
& ~~ \langle Z_0, I \rangle = nk-k \\
& ~~ \langle Z_1, I \rangle = k \\
& ~~ Z_0 + Z_1 = I \\
& ~~ Z_1, Z_2 \succeq 0.
\end{aligned}
\end{equation}



\subsection{Experimental setup and results}

%Our next numerical experiment compares the performance of the Lov\'asz theta function at recovering a planted clique instance in comparison with the above mentioned clustering algorithms. 


In our experimental setup, we generate  clique cover  instances from the planted clique cover model  with $\ks=10$ cliques, each of size $n=10$.  For each value of   $p \in \{ 0.00, 0.05, 0.10, \ldots, 1.00 \}$, we generate $20$ random clique cover  instances.  We compare the Lov\'asz theta against the three methods described in the previous section. Our result are summarized in  Figure \ref{fig:comparison},  where we illustrate the average success rates of each method in recovering the underlying clique cover instances across the entire spectrum of $p$ values. In  the reported experiments we use two notions of recovery for $\lt$:
\begin{enumerate}
\item We declare {\em strong recovery}  when the solver outputs an optimal solution $\hat{X}$ that is equal to $\xc$ (up to a small tolerance).
\item We declare {\em weak recovery} when $\lt(G) = \ks$ but the output solution $\hat{X}$ is not equal to $\xc$. \end{enumerate}



%First, we start off by noting two limitations of the   Note that Ames and Vavasis' formulation is specifically designed for finding disjoint unions of cliques.  One minor limitation of their work is that one needs to specify the size  of cliques one seeks, though that is easily circumvented with a sweep.
%
%  Second, in order to implement Candogan and Chandrasekaran's ideas to finding a clique cover, we implicitly assume that we know what we are searching for in the first place (to be precise, it is the {\em spectrum} of the target graph we require as an input).  This assumption is somewhat unrealistic, but one should bear in mind that Candogan and Chandrasekaran's work in \cite{CC:18} was intended to solve a completely different problem in the first place.  



First, we  compare against  the SDP   \eqref{eq:sparselowrank_comdetect} for community detection.  We sweep over the choices regularization parameters $\lambda \in \{ 0.0, 0.1, \ldots, 1.0 \}$.    This sweeping process makes their method the most expensive to implement among all methods listed here. For a given choice of  $\lambda$, let $(\hat{L},\hat{S})$ denote the optimal solution.  We say that the method succeeds if $\hat{L} = \sum \bone_{\ccs_l}\bone_{\ccs_l}^T$ for some choice of parameter~$\lambda$. 

Second,  we compare against  the SDP \eqref{eq:av} for the $k$-disjoint clique problem.  We say that the method succeeds if the optimal solution satisfies $\hat{X} = \sum \bone_{\ccs_l}\bone_{\ccs_l}^T$. It's worth noting that when applying this method to uncover clique covers, there is a minor constraint in that we must specify the size of cliques $\ks=10$, or alternatively, sweep over all feasible clique sizes.






Our third basis of comparison is the SDP \eqref{eq:sh} proposed by Candogan and Chandrasekaran for finding subgraphs based on the Schur-Horn orbitope of a graph \cite{CC:18}.  We say that the method succeeds if the optimal solution satisfies $\hat{X} = \sum \bone_{\ccs_l}\bone_{\ccs_l}^T$. A limitation of using this  
approach  for finding a clique cover is that we  implicitly assume that we know what we are searching for in the first place (to be precise, it is the {\em spectrum} of the target graph we require as an input).  This assumption is somewhat unrealistic, but one should bear in mind that Candogan and Chandrasekaran's work in \cite{CC:18} was intended to solve a completely different problem in the first place.


Our experiments are summarized in Figure \ref{fig:comparison}. First, when examining the performance of the Lovász theta function $\lt$, we observe a congruence between our theoretical predictions, which predict  strong recovery for small $p$ values, and the outcomes derived from our numerical experiments. Furthermore, it is noteworthy that this strong recovery pattern persists  until approximately $p\approx 0.4$.

In the context of our comparisons with the other three methods, the Lovász theta function emerges as the most successful,  with only a slight edge over the SDP \eqref{eq:av} proposed by Ames and Vavasis \cite{AV:14} for finding disjoint cliques, as well as the Schur-Horn orbitope-based relaxation \eqref{eq:sh} presented by Candogan and Chandrasekaran. The method that exhibited the weakest performance in our experiments was the matrix deconvolution method by Oymak and Hassibi \cite{OH:11}.
%
%We observe that all the graphs exhibit similar behavior; namely, they succeed for small values of $p$, before failing on the larger values of $p$.    We do point out that both of these methods require some additional information about the underlying clique cover we seek (as elaborated earlier) while the Lov\'asz theta function in \eqref{eq:lovasz_lambdamax} does not.  One difficulty with implementing is that it requires choosing the parameter $\lambda$ appropriately; in some cases, it might not even exist.  

\begin{figure}[H]
\centering
\includegraphics[width=0.5\textwidth]{images/comparison.png}
\caption{Comparison of Lov\'asz theta function with other methods.}
\label{fig:comparison}
\end{figure}








\section{Future directions} \label{sec:numerical}

In this section, we conduct and discuss additional numerical experiments involving the Lov\'asz theta function $\lt(G)$.  Our objective is to report interesting behavior of the Lov\'asz theta function in the context of recovering clique covers that are not supported or apparent from our theoretical findings, and suggest future directions to investigate.

\subsection{Comparison with ILP}

In our first example, we compare how tight the Lov\'asz theta function is in comparison with the clique cover number.  To investigate this question, we generate a graph from the planted clique cover model for varying choices of $p \in [0,1]$.  We compute the clique cover number by solving the following Integer Linear Program (ILP)
%\begin{equation} \label{eq:cliquecover_ilp}
%\begin{aligned}
%\min ~~ & ~~ \sum_{c} y_c \\
%\mathrm{s.t.} ~~ & ~~ x_{i,c} + x_{j,c} \leq y_c \quad (i,j) \in \mathcal{E} \\
%& ~~ \sum_{c} x_{i,c} = 1 \quad \text{ for all } i \\
%& ~~ x_{i,c}, y_c \in \{0,1\}.
%\end{aligned}
%\end{equation}
%[Or this one?]
\begin{equation} \label{eq:cliquecover_ilp}
\begin{aligned}
\min ~~ & ~~ \sum_{c} y_c \\
\mathrm{s.t.} ~~ & ~~ x_{i,c} \leq y_c \quad \text{ for all } \quad i,c \\
& ~~ x_{i,c} + x_{j,c} \leq 1 \quad (i,j) \in \mathcal{E} \\
& ~~ \sum_{c} x_{i,c} = 1 \quad \text{ for all } \quad i \\
& ~~ x_{i,c}, y_c \in \{0,1\}.
\end{aligned}
\end{equation}
Here, the binary variable $x_{i,c}$ indicates whether vertex $i$ is contained in clique $c$, while the binary variable $y_c$ indicates whether clique $c$ is non-empty.  The constraint $x_{i,c} + x_{j,c} \leq 1$ ensures no adjacent pair of vertices are in the same clique, while the constraint $\sum_{c} x_{i,c} = 1$ ensures that every vertex belongs to a unique clique. % have a covering.

Our motivation for studying this question is as follows:  In our experiments in Section \ref{subsec:related worlk}, we generally take the clique covering number to be equal to $\ks$.  However, short of solving computing the clique covering number outright (say, via \eqref{eq:cliquecover_ilp}), there is no simple way of verifying this is indeed the case.  The concern increases as $p$ increases, as we may inadvertently lower the clique covering number when more edges are added.  As such, one objective of this experiment is to understand if our assumption that the clique covering number is well approximated by $\ks$ is sound.  

%The other objective is to understand how the Lov\'asz theta function differs from the clique covering number for varying levels of $p$.  As we increase $p$, we will generally observe that the Lov\'asz theta function decreases.  We wish to see if that is also because the clique covering number possibly decreases.

In the left sub-plot of Figure \ref{fig:ILP} we compare the clique covering number with the Lov\'asz theta function on a planted clique cover instance with $8$ cliques, each of size $8$, and with varying choices of parameter $p$.  Based on our results we notice that the deviation between these quantities is at most $\approx 2$, and is greatest at $p \approx 0.7$.  

In the right sub-plot of Figure \ref{fig:ILP} we compare the time taken for the ILP solver Gurobi \cite{gurobi} to compute \eqref{eq:cliquecover_ilp} with the SDP solver SDPT3 \cite{sdpt3:1,sdpt3:2} to compute \eqref{eq:lovasz_lambdamax}.  The time taken for SDP solver is approximately equal across all problem instances, as one would expect.  The time taken for the ILP solver is quite small for $p \leq 0.4$, but becomes substantially longer for $p=0.6$.  In the same plot we also track the number of simplex solves required by the ILP solver Gurobi -- the trend of this curve is largely similar to the previous curve.  These observations suggest that the most complex regime for computing the clique covering number for the Erd\H{o}s-R\'enyi noise model occurs around $p \approx 0.6$.  There are a number of explanations: First, our theoretical findings suggest that it is generally easier for the Lov\'asz theta function to discover minimal clique covers for small values of $p$, and these curves do also suggest that small values of $p$ correspond to `easier' instances.  On the other extreme, at $p = 1$, there is only one clique.  For values of $p = 1-\epsilon$, it may be that large cliques are easy to find, and hence ILP solvers are able to certify optimality relatively quickly.  It would be interesting to investigate the behavior of these curves for increasing problem sizes, though one obvious difficulty is that the ILP instances may become impractical to compute. 



\begin{figure}[H]
\centering
\includegraphics[width=0.48\textwidth]{images/ILP_objval.png}
\includegraphics[width=0.48\textwidth]{images/ILP_timetaken.png}
\caption{Comparison of the clique covering number with Lov\'asz theta for random clique cover model (left).  Comparison of time taken for ILP solver to compute clique covering number with time taken for SDP solver to compute Lov\'asz theta function (right). The number of simplex iterations taken by ILP solver is included as a reference.}
\label{fig:ILP}
\end{figure}




\subsection{Phase Transition}



In the second experiment, we investigate the probability that the Lov\'asz theta function correctly recovers the planted cliques as certain parameters are taken to $+\infty$.  Our objective is to see if a phase transition arises.

To investigate this problem, we consider an experimental set-up where all the cliques are of equal size.  We set the number of cliques $\ks$ to be equal to the size of each clique $n$, with the values $n = \ks$ ranging over $\{5,\ldots, 15\}$.  For each value of $n$ and $\ks$, we generate $10$ random graphs from the planted clique cover model with $p$ taking values from $\{ 0.00, 0.05, 0.10, \ldots, 1.00 \}$.  In Figure \ref{fig:phasetransition} we plot the success probabilities corresponding to each value of $p$.  We plot the curve corresponding to strong recovery in black, and the curve corresponding to weak recovery in red.

We make a number of observations.  First, we observe that the transition from success to failure becomes more abrupt as we increase the values of $n$ and $\ks$.  This observation holds for the both types of recovery conditions.  In particular, our results suggest that a phase transition occurs as these values are taken to $+\infty$, and we conjecture that this is indeed the case, as is typical with many phenomena in random graphs.  Second, and quite interestingly, the gap between both types of recovery {\em shrink} as the problem parameters $n$ and $\ks$ increase.  We conjecture that both curves {\em coincide} in the limit.

\begin{figure}[H]
\centering
\includegraphics[width=0.5\textwidth]{images/PhaseTransition.png}
\caption{Probability of correctly recovering a planted clique instance with increasing clique size and increasing number of cliques.  The black curves correspond to strong recovery while the red curves correspond to weak recovery.  The thickness of the lines increase with the problem parameters $n$ and $\ks$.}
\label{fig:phasetransition}
\end{figure}

%In this section, we are interested in the probability that Lov\'asz succeeds when the size of the graph goes to infinity. Here we take 20 equal-step mesh points for $p \in [0,1]$ and for each value of $p$, we iterate all steps starting from cliques generation for 10 times. Based on observation, we conclude that there exist a transition near $p=0.5$ when the size of graph goes to infinity.

% \begin{figure}[H]
% \centering
% \includegraphics[width=0.8\textwidth]{pic/ch-Numerical/equal_k_n10iter5p20.jpg}
% \caption{nk = 100}
% \label{fig:nk = 100}
% \end{figure}


%\begin{figure}[H]
%\centering
%\includegraphics[width=0.5\textwidth]{images/unequal_k10n10_20iter10p20.jpg}
%\caption{k = 10, n $\in$ [10, 20]}
%\label{fig:k =10, n from 10 to 20}
%\end{figure}




\bibliography{bib_kcliques}

%\input{sec_deletedstuff}

\end{document}