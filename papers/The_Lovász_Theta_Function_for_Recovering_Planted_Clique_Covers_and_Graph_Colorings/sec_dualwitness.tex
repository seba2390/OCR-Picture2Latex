\section{Exact recovery in the case of  disjoint cliques} \label{sec:disjointunion}

In this section, we show that $(\ac,\ks)$ is the unique optimal solution to the Lov\'asz theta formulation~\eqref{eq:lovasz_lambdamax} when $G$ is a disjoint union of cliques.  In this simple case, $G$ satisfies the $\cscc$ condition with $c=0$.  Following the conclusions of Theorem \ref{thm:extremepoint}, $(\ac,\ks)$ is an extreme point of the feasible region of \eqref{eq:lovasz_lambdamax}.  Based on Theorem \ref{thm:constraintqualification}, it remains to produce a suitable dual witness $Z$ satisfying the requirements listed in Theorem \ref{thm:constraintqualification}.

As it turns out, the matrix $\zz$ satisfies all the requirements we need!  In fact, in the process of computing the spectrum of the matrix $\zz$ in Proposition \ref{thm:Unequalsizecanonicalmatrix_spectrum}, we have also verified that $\zz$ does indeed satisfy the requirements of Theorem \ref{thm:constraintqualification}.  We collect these conclusions below.
 
\begin{theorem} \label{thm:disjointcliques_uniquerecovery}
Let $G$ be the graph formed by the disjoint union of cliques $\{\ccs_l\}_{l=1}^{\ks}$.  Then the matrix $(\ac,\ks)$ is the unique solution of \eqref{eq:lovasz_lambdamax}.
\end{theorem} 


\begin{proof}
As we noted above, we simply need to check that the matrix $(1/\ks)\zz$ satisfies conditions \eqref{cond1} to \eqref{cond3}.  First, from Proposition \ref{thm:Unequalsizecanonicalmatrix_spectrum}, the matrix $\zz$ is PSD, which is \eqref{cond1}.  Second, all the edges of $G$ are within cliques.  The matrix $\zz$, when restricted to each clique $\ccs_l$, is the identity matrix, and thus satisfies the support condition \eqref{eq:supportcondition}.  It is easy to verify that all the rows of $\zz$ belong to $\mathcal{J}$.  Hence, by Proposition \ref{csexpnaded}, $\zz\in {\rm span}(\xc)^\perp$, which is \eqref{eq:complementaryslackness}.  It is easy to see that $\mathrm{tr}(\zz) = \ks$ and $\langle J, \zz \rangle = (\ks)^2$, which shows  \eqref{eq:traceone}.  Last, by Proposition \ref{thm:Unequalsizecanonicalmatrix_spectrum}, we have $\range(\zz)=\mathcal{J}$.  Hence, by Proposition \ref{csexpnaded}, we have ${\rm rank}(\xc)+{\rm rank}(\zz)=|\V|$, which is~\eqref{cond3}.
\end{proof}

 