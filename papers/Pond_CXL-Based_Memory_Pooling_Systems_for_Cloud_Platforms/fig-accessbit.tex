
\def \hmina {\hspace{0.00in}}
\def \hminb {\hspace*{-0.05in}}

\begin{figure}[t!]
\hmina
\begin{center}

\begin{minipage}[t]{\columnwidth}
\hspace{-1mm}
\includegraphics[width=.9\columnwidth]{F/stranding/access_bit_scan_teams}
\end{minipage}

\vspace*{16pt}

%-----------------------------------------------------------------------
\begin{minipage}[t]{\columnwidth}
%\vspace{-30mm}
%\small
\setlength{\tabcolsep}{4pt}

%\hminb
\vspace{-8mm}
\begin{center}
\small
\begin{tabular}{l|>{\raggedleft\arraybackslash}p{30mm}}
Workloads & Traffic to \cvn \\
\hline
Video & 0.25\%\\
Database & 0.06\% \\
KV store & 0.11\% \\
Analytics & 0.38\%
\end{tabular}
\end{center}
\vspace{-6mm}
\end{minipage}

\vspace*{12pt}

% reconfigures the system and enforced security and robustness
\mycaption{fig-accessbit}{Effectiveness of \cvn (\sec\ref{sec:eval:accessbit})}{Latency sensitive workloads get a local vNUMA node large enough to cover the workload's footprint. \cvn nodes holds the VM's remaining memory on \sys CXL pool.
Access bit scans, \eg, for Video (right), show that this configuration indeed minimizes traffic to the \cvn node.
%
}

\vminten
\end{center}
\end{figure}


