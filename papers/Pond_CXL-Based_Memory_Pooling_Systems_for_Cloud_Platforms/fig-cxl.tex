
\def \hmina {\hspace{0.02in}}
\def \hminb {\hspace{-0.2in}}

\def \fgw {6.5in}
\def \fgh {1.085in}

% \begin{floatingfigure}[r]{2in}  (and \end{..})

\begin{figure*}[ht!]
\centerline{
%\hmina
%\includegraphics[width=\fgw]{F/cxl/eps/L0N-stacked.pdf}
\includegraphics[width=\fgw]{F/cxl/eps/L0N-stacked-hcl.pdf}
}

%\vminfive
%
\mycaption{fig-cxl}{Performance slowdowns when memory latency increases by 182-222\% (\sec\ref{sec-mot-cxl})}{\newtxt{Workloads have different sensitivity to additional memory latency (as in CXL).
% The results motivate various \sys design choices in
% \sec\ref{sec-des}.
X-axis shows \numTotalApps\ representative workloads;
Y is the normalized performance slowdown, \ie, performance under higher (remote) latency relative to all local memory.
``Proprietary'' denotes production workloads at \azure.\vten
%
\if 0
%
{\bf Workloads:} ``Proprietary'' denotes \azure's internal
production workloads, \eg, databases, web search,
machine learning, and analytics. The rest are open-source workloads,
such as YCSB (A--F)~\cite{ycsb.socc10} with in-memory stores
(Redis/VoltDB), in-memory computing Spark workloads in HiBench~\cite{hibench.web21},
graph processing (GAPBS)~\cite{gapbs.corr15}, and high-performance
computing benchmark sets such as SPEC CPU~\cite{speccpu2017.web21},
PARSEC~\cite{parsec.pact08}, and SPLASH2x~\cite{parsec3.can16}.
%
\fi
}}
%\vspace{-0.1in}
%

\end{figure*}


