\section{\sys Design}
\label{sec-des}

% We first present \sys's design goals and overview, and then describe the
% details of its individual components.

Our measurements and observations at \azure (\sec\ref{sec-bg}--\ref{sec:measurements}) lead us to define the following design goals.
\begin{description}
\item[G1] Performance comparable to NUMA-local DRAM
\item[G2] Compatibility with virtualization accelerators
\item[G3] Compatibility with opaque VMs and unchanged guest OSes\slash applications
\item[G4] Low host resource overhead
\end{description}
  %Additionally, VMs are opaque to the platform~\cite{fuerst2022memory,shahrad2021provisioning,chen2020alita}.

To quantify (G1), we define a \emph{performance degradation margin} (\pdm) for a given workload as the allowable slowdown relative to running the workload entirely on NUMA-local DRAM.
\sys seeks to achieve a configurable \pdm, \eg, 1\%, for a configurable tail-percentage (\tp) of VMs, \eg, 98\% (\sec\ref{sec-strand}).
To achieve this high performance, \sys uses a small but fast CXL pool (\sec\ref{sec:design:hw}).
As \sys's memory savings come from pooling instead of oversubscription, \sys must minimize pool fragmentation and wastage in its system software layer (\sec\ref{sec:design:systemsw}).
To achieve (G2), \sys preallocates local and pool memory at VM start.
\sys decides this allocation in its allocation, performance monitoring, and mitigation pipeline (\sec\ref{sec:design:distributedsw}).
This pipeline uses novel prediction models to achieve the \pdm (\sec\ref{sec:design:ml}).
Finally, \sys overcomes VM-opaqueness (G3) and host-overheads (G4) using lightweight hardware counter telemetry.



% cut for length
%\myparagraph{PG2: Resource efficiency} Providers seek low overheads
%within the platform software and hardware.
%For example, CPU cores lost to virtualization
%overhead cannot be rented out to users~\cite{intelvtd.web20,
%  leapio.asplos20, sriov}.
%Another constraint is the number of available memory decoder entries (\sec\ref{sec-bg}).

%\myparagraph{PG3: Small blast radius} Providers seek to offer
%their VMs high availability while minimizing capacity that cannot be
%used due to component failure. We thus need to minimize the impact
%(\emph{blast radius}) of failures for additional components introduced
%for memory disaggregation.





\begin{figure}[t!]
  \centering
%\includegraphics[width=\columnwidth]{F/sketches/cxl_analysis_overheads}
\includegraphics[width=\columnwidth]{F/sketches/cxl_analysis_overheads_hcl}
  \vminten
    \mycaption{fig-analysis-overheads}{Pool size and latency tradeoffs (\sec\ref{sec:design:hw})}{\newtxt{Small \sys pools of 8-16 sockets add only 75-90ns relative to NUMA-local DRAM.
  Latency increases for larger pools that require retimers and a switch.}}
\end{figure}






\subsection{Hardware Layer}\label{sec:design:hw}

%% The hardware for disaggregated memory pools requires a protocol
%% that surfaces pool memory as a load/store device to the CPU. This model
%% facilitates adding pooled memory to existing applications in a way that
%% does not require software modifications, satisfying the inertia
%% requirement (FR1) discussed above. The CXL
%% standard (\sec\ref{sec-bg}) defines an interface that accomplishes this
%% goal, and is gaining wide adoption.

%\vni\myparagraph{\sys memory pool design}
% We propose a novel device and
% system architecture that uses CXL to implement disaggregated memory pooling that meets
% the design goals outlined above. The disaggregated architecture is built

\newtxt{Hosts within a \sys pool have separate cache coherency domains and run separate hypervisors.
\sys uses an ownership model where pool memory is explicitly moved among hosts.
A new external memory controller (EMC) ASIC implements the pool using multiple DDR5 channels accessed through a collection of CXL ports running at PCIe 5 speeds.}

%HDM decoder resources on current-generation hosts limit tracking online\slash offline status to 1GB granularity.
%Thus, CXL poses fragmentation challenges similar to huge pages~\cite{kwon2016coordinated,panwar2018making}.

\myparagraph{\newtxt{EMC memory management}}
\newtxt{The EMC offers multiple CXL ports and appears to each host as a single logical memory device~\cite{cxl2spec.web20,cxl2whitepaper.web21}.
In CXL 3.0~\cite{debendra2022fms,cxl3spec}, this configuration is standardized as multi-headed device (MHD)~\cite[\S2.5]{cxl3spec}.
The EMC exposes its entire capacity on each port (\eg, to hosts) via a Host-managed Device Memory (HDM) decoder.
Hosts program each EMC's address range but treat them initially as offline.
\sys dynamically assigns memory at the granularity of 1GB memory slices.
Each slice is assigned to at most one host at a given time and hosts are explicitly notified about changes (\S\ref{sec:design:systemsw}).
Tracking 1024 slices (1TB) and 64 hosts (6 bits) requires 768B of EMC state.
The EMC implements dynamic slice assignment by checking permission of each memory access, \ie, whether requestor and owner of the cacheline's slice match.
Disallowed accesses result in fatal memory errors.}


\begin{figure}[t!]
  \centering
%\includegraphics[width=0.99\columnwidth]{F/stranding/latency_tradeoff}
\includegraphics[width=\columnwidth]{F/stranding/latency_tradeoff_hcl}
\vminfifteen
    \mycaption{fig-poollatency}{Pool access latency comparison (\sec\ref{sec:design:hw})}{\newtxt{\sys reduces latencies by 1/3 compared to switch-only designs.}}
    \vfive
\end{figure}





{\myparagraph{\newtxt{EMC ASIC design}}
\newtxt{The EMC offers multiple \tms{8}-CXL ports, which communicate with DDR5 memory controllers (MC) via an on-chip network (NOC).
The MCs must offer the same reliability, availability, and serviceability capabilities~\cite{intelras,amdras} as server-grade memory controllers including memory error correction, management, and isolation.
A key design parameter of \sys's EMC is the pool size, which defines the number of CPU sockets able to use pool memory.
We first observe that the EMC's IO, (De)Serializer, and MC requirements resemble AMD Genoa's $397mm^2$ IO-die (IOD)~\cite{amdgenoa,genoa2021leak}.
Figure~\ref{fig-contactcount-revised} shows that EMC requirements for a 16-socket \sys parallel the IOD's requirements, with a small 8-socket \sys paralleling half an IOD.
Thus, up to 16-sockets can directly connect to an EMC.
Pool sizes of 32-64 would combine CXL switches with \sys's multi-headed EMC.
The optimal design point balances the potential pool savings for larger pool sizes (\S\ref{sec-eval}) with the added cost of larger EMCs and switches.}


%We remark that Using switches in the pool architecture adds significant additional cost, power, and latency.
%The EMC uses a 512b data path to process cachelines as is the IP

\myparagraph{\newtxt{EMC Latency}}
\newtxt{
While latency is affected by propagation delays, it is dominated by CXL port latency, and any use of CXL retimers and CXL switches.
Port latencies are discussed in \sec\ref{sec-bg} and~\cite{debendra2022hoti}.
Retimers are devices used to maintain CXL/PCIe signal integrity over longer distances and add about 10ns of latency in each direction~\cite{microchipretimer,asteraretimer}.
In datacenter conditions, signal integrity simulations~\cite{cxlretimers} indicate that CXL could require retimers above 500mm.
Switches add at least 70ns of latency due to ports/arbitration/NOC with estimates above 100ns~\cite{cxlswitchlatency}.}
%As we know from emulation experiments (\sec\ref{sec:measurements}), even moderate increases in memory access latency significantly degrade performance for many workloads, and so switched CXL pools are unable to reach acceptable performance levels.

\newtxt{Figure~\ref{fig-analysis-overheads} breaks down \sys's latency for different pool sizes.
Figure~\ref{fig-poollatency} compares \sys's latency to a design that relies only on switches instead of a multi-headed EMC.
We find that \sys reduces latencies by 1/3 with 8-and 16-socket pools adding only 70-90ns relative to NUMA-local DRAM.
In practice, we expect \sys to be deployed primarily with small 8/16-socket pools, given the latency and cost overheads, and diminishing returns of larger pools (\sec\ref{sec:measurements}).
Modern CPUs can connect to multiple EMCs which allows scaling to meet bandwidth and capacity goals for different clusters.}



\def \hmina {\hspace{0.05in}}
\def \hminb {\hspace{-0.2in}}

\begin{figure}[t!]
\centering
\hmina
%\includegraphics[width=\columnwidth]{F/sketches/pool_management}
\includegraphics[width=\columnwidth]{F/sketches/pool_management_hcl}
%
% reconfigures the system and enforced security and robustness
\vminfive
    \mycaption{fig-poolmng}{Pool management example (\sec\ref{sec:design:systemsw})}{
% For simplicity and
% performance, \sys pools are not cache coherent. Consequently,
\newtxt{\sys assigns pool memory to at most one host at a time.
This example shows \sys's asynchronous memory release strategy which engages when a VM departs ($t$=1 and $t$=2).
During VM scheduling, memory is added to the corresponding host before the VM starts ($t$=3 and $t$=4).}}
\vten
\end{figure}




\subsection{System Software Layer}
%
\label{sec:design:systemsw}

% domains
\sys's system software involves multiple components.

\myparagraph{Pool memory ownership}
Pool management involves assigning \sys's memory slices to hosts and reclaiming them for the pool (Figure~\ref{fig-poolmng}).
It involves 1) implementing the control paths for pool-level memory assignment and 2) preventing pool memory fragmentation.

\newtxt{Hosts discover local and pool capacity through CXL device discovery and map them to their address space. % through. % \ts{ACPI\slash SRAT} tables~\cite{acpi}.
Once mapped, the pool address range is marked hot-pluggable and ``not enabled.''}
\newtxt{Slice assignment is controlled at runtime via a Pool Manager (PM) that is colocated on the same blade as the EMCs (Figure~\ref{fig-analysis-overheads}).
%via the fabric manager protocol~\cite{cxl2whitepaper.web21}.
In \sys's current design, the PM is connected to EMCs and CPU sockets via a low-power management bus
(\eg,~\cite{i3c}).
To allocate pool memory, the Pool Manager triggers two types of interrupts at the EMC and host driver.
\ts{Add\_capacity(host, slice)} interrupts the host driver which reads the address range to be hot-plugged.
The driver then communicates with the OS memory manager to bring the memory online.
The EMC adds the host id to its permission table at the slice offset.
\ts{Release\_capacity(host, slice)} works similarly by offlining the slice on the host and resetting the slice's permission table entry on the EMC.
An alternative to this design would be inband-communication using the Dynamic Capacity Device (DCD) feature in CXL 3.0~\cite[\S9.13]{cxl3spec}.
This change would maintain the same functionality for \sys.}


%\newtxt{Software flow
%Disable Caching
%Initiate WB and Inval
%Device must send CacheFlushed message to the host per CXL protocol
%Poll for completion (timeout derived from cache size)
%Re-enable caching.} \hcl{What are these?}

\sys must avoid fragmenting its online pool memory as the contiguous 1GB address range must be free before it can be offlined for reassignment to another host.
Pool memory is allocated to VMs in 1GB-aligned increments (\sec\ref{sec:design:distributedsw}).
While this prevents fragmentation due to VM starts and completions, our experience has shown that host agents and drivers can allocate pool memory and cause fragmentation.
\sys thus uses a special-purpose memory partition that is only available to the hypervisor.
Host agents and drivers allocate memory in host-local memory partition, which effectively contains fragmentation.

With these optimizations, offlining 1GB slices empirically takes 10-100 milliseconds\slash GB.
Onlining memory is near instantaneous with microseconds\slash GB.
These observations are reflected in \sys's asynchronous release strategy (\sec\ref{sec:design:distributedsw}).
%This means that shifting memory after a 128GB VM is completed would require tens of seconds, which is insufficient for VM-start SLOs at \azure.



%Offlining multiple slices can thus take multiple seconds.


\myparagraph{Failure management}
Hosts only interleave across local memory.
This minimizes the EMCs' blast radius and facilitate memory hot-plugging.
EMC failures affect only VMs with memory on that EMC, while VMs with memory on other EMCs continue normally.
CPU\slash host failures are isolated and associated pool memory is reallocated to other
hosts.
Pool Manager failures prevent reallocating pool memory but do not affect the datapath.

%%LH - I don't think the following is vital for the paper.  Commenting out for now.
%Memory onlining and offlining requires multiple changes to the proprietary hypervisor stack used at \azure.
%First, fragmentation of host memory prevents offlining a page.
%Fragmentation can occur as host memory allocations do not always happen at the 1GB granularity of \sys slices.
%For example, there are multiple small VM types that demand memory in 0.25 and 0.5GB increments.
%Further, the host runs multiple management agents which each use a small amount of memory.
%The host thus implements a memory partition that contains all these small memory allocation pieces.
%This memory partition can span local and pool memory, but is permanently allocated to this specific host.


\def \vmina {\vspace{-0.1in}}
\def \vminb {\vspace{0.05in}}

\begin{figure}[t!]
\centering
\includegraphics[width=0.99\columnwidth]{F/sketches/linux_computeless_vnuma_wide.png}
%\vmina
% that is running on our emulation testbed

%\vspace{-2mm}

    \mycaption{fig-vnuma}{zNUMA (\sec\ref{sec:design:systemsw})}{zNUMA
seen from a Linux VM.}

\vfive

\end{figure}





\myparagraph{Exposing pool memory to VMs}
VMs that use both NUMA-local and pool memory see pool memory as a \cvn node.
The hypervisor creates a \cvn node by adding a memory block (\texttt{node\_memblk}) without
an entry in the \texttt{node\_cpuid} in the
\ts{SLIT\slash SRAT} tables~\cite{acpi}.
We later show the guest-OS preferentially allocates memory from the local NUMA node before going to \cvn (\sec\ref{sec-eval}).
Thus, if \cvn is sized to the amount of untouched memory, it is never going to be used.
Figure~\ref{fig-vnuma} shows the view of a Linux VM which includes
the correct latency in the NUMA distance
matrix (\texttt{numa\_slit}).
This facilitates guest-OS NUMA-aware memory management~\cite{nimblepage.asplos19, autonuma.web19} for the rare case that the \cvn is used (\sec\ref{sec:design:ml}).



\def \hmina {\hspace{-0.1in}}
\def \hminb {\hspace{-0.2in}}

\begin{figure}[t!]
%\hmina
\centering
%\includegraphics[width=0.98\columnwidth]{F/sketches/workflow}
\includegraphics[width=0.98\columnwidth]{F/sketches/workflow-hcl}
\vtri
\mycaption{fig-workflow}{\sys control plane workflow
(\sec\ref{sec:design:distributedsw})}{
{\bf A)}~The VM scheduler uses ML-based
predictions that identify latency-sensitive VMs and their likely
amount of untouched memory to decide on VM placement (see Figure~\ref{fig-premodels}).
{\bf B)}
The monitoring pipeline reconfigures VMs if quality-of-service (QoS) not
met.}
\vten

\end{figure}



\myparagraph{Reconfiguration of memory allocation}
To remain compatible with ({\bf G2}), local and pool memory mapping generally remain static during a VM's lifetime.
There are two exceptions that are implemented today.
When live-migrating a VM or when remapping a page with a memory fault, the hypervisor temporarily disables virtualization acceleration
and the VM falls back to a slower I/O path~\cite{ruprecht2018vm}.
Both events are quick and transient and typically only happen once during a VM's lifetime.
We implement a third variant which allows \sys a one-time correction to a suboptimal memory allocation.
Specifically, if the host has local memory available, \sys disables the accelerator, copies all of the VM's memory to local memory, and enables the accelerator again.
This takes about 50ms for every GB of pool memory that \sys allocated to the VM.


\myparagraph{Telemetry for opaque VMs}
\sys requires two types of telemetry for VMs.
First, we use the core-performance-measurement-unit (PMU) to gather hardware counters related to memory performance.
Specifically, we use the top-down-method for analysis
(TMA)~\cite{topdownanalysis.ispass14,tmam.web21}.
TMA characterizes how the core pipeline slots are used.
For example, we use the ``memory-bound'' metric, which
is defined as pipeline stalls due to memory loads and stores.
Figure~\ref{ml-training1} lists these metrics.
While TMA was developed for Intel, its relevant parts are available on AMD and
ARM as well~\cite{jarus2016top}.
We modify \azure's production hypervisor to associate these metrics with individual
VMs (\sec\ref{sec-impl}) and record VM counter samples in a distributed database.
All our core-PMU-metrics use simple counters and induce negligible overhead (unlike event-based sampling~\cite{akiyama2017quantitative,hemem.sosp21}).

Second, we use hypervisor telemetry to track a VM's untouched pages.
We use an existing hypervisor counter that tracks guest-committed memory, which overestimates used memory.
This counter is available for 98\% of \azure VMs.
We also scan access bits in the hypervisor page table (\sec\ref{sec-impl}).
Since we only seek untouched pages, frequently access bits reset is not required. This minimizes overhead.
% , as used by prior work on working set estimation~\cite{vmwss.cloud13, harby2019more}

\subsection{Distributed Control Plane Layer}
%
\label{sec:design:distributedsw}

Figure~\ref{fig-workflow} shows the two tasks performed by \sys's control plane:
(A) predictions to allocate memory during VM scheduling
and (B) QoS monitoring and resolution.

\myparagraph{Predictions and VM scheduling (A)}
\sys uses ML-based prediction models (\sec\ref{sec:design:ml})
to decide how much pool memory
to allocate for a VM during scheduling.
After a VM request arrives (A1), the scheduler queries the
distributed ML serving system (A2) for a prediction on how much local memory
to allocate for the VM. The scheduler
then informs the Pool Manager about the target host and associated pool memory
needs (A3).
The Pool Manager triggers a memory onlining workflow using the configuration
bus to the EMCs and host (A4).
Memory onlining is fast enough to not block a VM's start time (\sec\ref{sec:design:systemsw}).
The scheduler informs the hypervisor
to start the VM on a \cvn node matching the onlined memory amount.

Memory offlining is slow and cannot happen on the critical path of VM starts (\sec\ref{sec:design:systemsw}).
\sys resolves this by always keeping a buffer of unallocated pool memory.
This buffer is replenished when VMs terminate and hosts asynchronously release associated slices.
% and communicates to that host's virtualization stack.



\begin{figure}[t!]
\centering
%\includegraphics[width=0.9\columnwidth]{F/sketches/training_latency_insensitivity}
\includegraphics[width=0.75\columnwidth]{F/sketches/training_latency_insensitivity_hcl}
%
\vtri
\mycaption{ml-training1}{Metrics and training of the latency
    insensitive model (\sec\ref{sec:design:systemsw})}{This model uses metrics from the
core's performance-measurement-unit (PMU). It is trained with labels
gathered from offline runs and internal workloads.}
\vten
%
\end{figure}




\myparagraph{QoS monitoring (B)} \sys
continuously inspects the performance of all running VMs via its QoS monitor.
The monitor queries hypervisor and hardware performance counters (B1) and
uses an ML model of latency sensitivity (\sec\ref{sec:design:ml})
to decide whether the VM's performance impact exceeds the \pdm.
In this case, the monitor asks its mitigation manager (B2) to
trigger a memory reconfiguration (\sec\ref{sec:design:systemsw}) through the hypervisor (B3).
After this reconfiguration, the VM uses only local memory.

% for Disaggregated Memory
\subsection{Prediction Models}
%
\label{sec:design:ml}

\sys's VM scheduling (A) and QoS monitoring (B) algorithms rely on two
prediction models (in Figure~\ref{fig-premodels}).

\myparagraph{Predictions for VM scheduling (A)} For scheduling, we first
check if we can correlate a workload history with the VM requested.
This works by checking if there have been previous VMs with the same
metadata as the request VM, \eg, the customer-id, VM type, and location.
This is based on the observation that
VMs from the same customer tend to exhibit similar
behavior~\cite{resourcecentral.sosp17}.

If we have prior workload history, we make a prediction on whether this
VM is likely to be memory latency insensitive, \ie, its performance would be
within the \pdm while using only pool memory. (Model details appear below.)
Latency-insensitive VMs are allocated entirely on pool DRAM.


\def \hmina {\hspace{-0.1in}}
\def \hminb {\hspace{-0.2in}}

\begin{figure}[t!]
\centering
%\includegraphics[width=\columnwidth]{F/sketches/predictionmodels}
\includegraphics[width=\columnwidth]{F/sketches/predictionmodels-hcl}
\vminfive
\mycaption{fig-premodels}{\sys prediction models (\sec\ref{sec:design:ml})}{
\sys's two prediction models (dark grey) rely on telemetry (blue boxes) that is available for all VM types, including third-party opaque VMs.}
\vten

\end{figure}







If the VM has no workload history or is predicted to be latency-sensitive,
we predict untouched memory ($\texttt{UM}$) over its
lifetime.  Interestingly, $\texttt{UM}$ predictions with only generic VM metadata such as customer history, VM type, guest OS,
and location are accurate (\sec\ref{sec-eval}).  VMs without untouched memory ($\texttt{UM}=0$) are allocated
entirely with local DRAM.  VMs with a $\texttt{UM}>0$ are allocated with a rounded-down GB-aligned percentage of pool
%$\lfloor\texttt{UM}\rfloor_{GB}$
memory and a corresponding \cvn node;
the remaining memory is allocated on local DRAM.

If we underpredict $\texttt{UM}$, the VM will not touch the slower pool
memory as the guest OS prioritizes allocating local DRAM.
If we overpredict $\texttt{UM}$, we rely on the QoS monitor for mitigation.
Importantly, \sys always keeps a VM's memory mapped in hypervisor page
tables at all times. This means that even if our predictions happen to
be incorrect, performance does not fall off a cliff.


\myparagraph{QoS monitoring (B)}
For \cvn VMs, \sys monitors if it overpredicted
the amount of untouched memory during scheduling.
For pool-backed VMs and \cvn VMs with
less untouched memory than predicted, we use the sensitivity model to
determine whether the VM workload is suffering excessive performance loss.
If not, the QoS monitor initiates a live VM migration to a configuration
allocated entirely on local DRAM.

\myparagraph{Model details}
\sys's two ML prediction models consume telemetry that is available for opaque VMs from \sys's system software layer
(\sec\ref{sec:design:systemsw}).
Figure~\ref{ml-training1} shows features, labels, and the training procedure for the
latency insensitivity model.
The model uses supervised learning (\S\ref{sec-impl}) with core-PMU metrics as features
and the slowdown of pool memory relative to NUMA-local memory as labels.
\sys gets samples of slowdowns from offline test runs and A\slash B tests of internal
workloads which make their performance numbers available.
These feature-label-pairs are used to retrain the model daily.
As the core-PMU is lightweight (\sec\ref{sec-impl}),
\sys continuously measures core-PMU metrics at VM runtime.
This enable the QoS monitor to react quickly and enables retaining
a history of VMs that have been latency sensitive.

\newtxt{Figure~\ref{ml-training2} shows the inputs and training procedure for the
untouched-memory model.
The model uses supervised learning (details in \S\ref{sec-impl}) with VM metadata as features and the minimum untouched memory over each VM's lifetime as labels.
Its most important feature is a range of percentiles (\eg, 80\th--99\th) of the recorded untouched memory by a customer's VMs in the last week.}

%
\setlength{\tabcolsep}{3pt}

\begin{table}[t!]
\begin{center}
%\footnotesize
\small
\begin{tabular}{p{2.8cm}p{2.5cm}p{2.8cm}}
\hline
\textbf{Prediction target}   & \textbf{Purpose}    & \textbf{Inputs (features)} \\
%\hline
Memory footprint             & Allocation decision & Past VM allocations        \\
%\hline
% over future VM lifetime
Lat. sensitivity             & Allocation decision & Past VM allocations        \\
%\hline
% over VM's past lifetime, QoS impact
Lat. sensitivity/QoS & For VM migration    & VM’s runtime history      \\
\hline
\end{tabular}
\end{center}
%
\vminfifteen
%
\mycaption{tbl:predmodels}{Prediction models}{}
%
\vminfifteen
\end{table}
 % prediction model table

\begin{figure}[t!]
  \centering
%\includegraphics[width=1.01\columnwidth]{F/sketches/training_untouched}
\includegraphics[width=1.01\columnwidth]{F/sketches/training_untouched_hcl}
\vminfive
    \mycaption{ml-training2}{Training of the untouched memory model (\sec\ref{sec:design:ml})}{This model uses VM metadata as features and labels of untouched memory gathered from hypervisor telemetry.}
    \vten
\end{figure}




\myparagraph{Parameterization of prediction models}
\sys's latency insensitivity model is parameterized to stay below a \emph{target rate of false positives} (\texttt{FP}), \ie, workloads it incorrectly specifies as latency insensitive but which are actually sensitive to memory latency.
This parameter enforces a tradeoff as the percentage of workloads that are labeled as latency insensitive (\texttt{LI}) is a function of \texttt{FP}.
For example, a rate of 0.1\% \texttt{FP} may force the model to 5\% of \texttt{LI}.

Similarly, \sys's untouched memory model is parameterized to stay below a \emph{target rate of overpredictions} (\texttt{OP}), \ie, workloads that touch more memory than predicted and thus would use memory pages on the \cvn node.
This parameter enforces a tradeoff as the percentage of untouched memory (\texttt{UM}) is a function of \texttt{OP}.
For example, a rate of 0.1\% \texttt{OP} may force the model to 3\% of \texttt{UM}.

With two models and their respective parameters, \sys needs to decide how to balance \texttt{FP} and \texttt{OP} between the two models.
This balance is done by solving an optimization problem based on the given performance degradation margin (\pdm) and the target percentage of VMs that meet this margin (\tp).
Specifically, \sys seeks to maximize the average amount of memory that is allocated on the CXL pool, which is defined by \texttt{LI} and \texttt{UM}, while keeping the percentage of false positives (\texttt{FP}) and untouched overpredictions (\texttt{OP}) below the \tp.
\begin{align}
\nonumber  \text{maximize} & \quad (\texttt{LI}_\pdm) + (\texttt{UM})\\
\label{eq:optimization}  \text{subject to} & \thinspace (\texttt{FP}_\pdm) + (\texttt{OP}) \leq (100 - \tp)
\end{align}
Note that \tp essentially defines how often the QoS monitor has to engage and initiate memory reconfigurations.

Besides \pdm and \tp, \sys has no other parameters as it automatically solves the optimization problem from Eq.\eqref{eq:optimization}.
The models rely on their respective framework's default hyperparameters (\sec\ref{sec-impl}).
