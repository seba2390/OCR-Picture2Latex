
%\newpage

%\setlength{\tabcolsep}{3pt}

\begin{table*}[th!]
\begin{center}
%\footnotesize
%\small
\begin{tabular}{p{8.5cm}|p{9cm}}
\hline
{\bf Key results, findings and observations} & {\bf Implications / Root causes} \\
\hline
Of the \numTotalApps\ workloads we evaluated, 42\% are not sensitive to CXL latency while some are very sensitive with up to 51\% performance impact. & Even a 90ns latency increase will render full CXL memory backing for VMs impractical, and this validates design decision to keep local DRAM on nodes. \\
\hline
Working set size and DRAM bandwidth are independent of CXL performance while CPI almost perfectly captures it. & We need better methods to predict CXL-caused performace slowdown whereas CPI is a posterior metric. \\
\hline
Peak dram bandwidth for a single workload is less than 40 GB/s. & This validates the design decision that we do not need to interleave across CXL ports. \\
\hline
%Workload specific findings: performance of modern high performance I/O devices (\eg, NVMe SSDs) will be significantly affected; VoltDB is experience much less performance impact than Redis; Spark workload (usually with the highest observable DRAM traffic) are not severely impacted as graph processing workloads. \\
%\hline
\cvn works well if workload footprint fits into the VM's local memory, there will be no performance impact. & We can pool DRAM for \cvn and frigid memory for latency sensitive workloads by only locating frigid memory on pool. \\
\hline
Mixing with 80\% local DRAM ensured good perfornamce for XX-YY\% of workloads -- but tail remains. & A static split between local and CXL DRAM is not sufficient for various types of VMs and workloads. \\
\hline
%Interleaving does not work consistently. \\
\hline
TMA works better than ML and 20\% of workloads can safely be placed entirely into CXL pool & \\
\hline
ML works better than static for frigid memory, and we can safely predict 20\% frigid memory on average, 80\% local memory & \\
\hline
Stranding vs pool size, for different models. & Small pool size is enough and our prediction models make a big difference (\eg, vs. static mixing) \\
\hline
\end{tabular}
\end{center}

\vminfifteen
%
\mycaption{tab-findings}{Summary of evaluation findings}{}
%
%\vminfive

\end{table*}

