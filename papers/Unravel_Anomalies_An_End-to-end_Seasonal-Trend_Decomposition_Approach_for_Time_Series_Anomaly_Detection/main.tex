% Template for ICASSP-2024 paper; to be used with:
%          spconf.sty  - ICASSP/ICIP LaTeX style file, and
%          IEEEbib.bst - IEEE bibliography style file.
% --------------------------------------------------------------------------
\documentclass{article}
\usepackage{spconf,amsmath,graphicx}
\usepackage{amsfonts}
\usepackage{multirow}
\usepackage{booktabs}
\usepackage[hidelinks]{hyperref}
\usepackage{color}
\usepackage{balance}

% Example definitions.
% --------------------
\def\x{{\mathbf x}}
\def\L{{\cal L}}

% Title.
% ------
\title{Unravel Anomalies: An End-to-end Seasonal-Trend Decomposition Approach for Time Series Anomaly Detection}
%
% Single address.
% ---------------
% \name{Zhenwei Zhang, Ruiqi Wang, Ran Ding, Yuantao Gu
% % \thanks{Thanks to XYZ agency for funding.}
% }
% \address{Department of Electronic Engineering, Tsinghua University, Beijing, China}

\name{Zhenwei Zhang\textsuperscript{1}, Ruiqi Wang\textsuperscript{1}, Ran Ding\textsuperscript{2}, Yuantao Gu\textsuperscript{1}
\thanks{© 2023 IEEE. Personal use of this material is permitted. Permission from IEEE must be obtained for all other uses, in any current or future media, including reprinting/republishing this material for advertising or promotional purposes, creating new collective works, for resale or redistribution to servers or lists, or reuse of any copyrighted component of this work in other works. Code available at \url{https://github.com/zhangzw16/TADNet}}
}
\address{
\textsuperscript{1}Department of Electronic Engineering, Tsinghua University, Beijing, China \\
\textsuperscript{2}Department of Electronic Engineering, Shanghai Jiao Tong University, Shanghai, China
}


\begin{document}
\topmargin=0mm

\ninept
\maketitle

\begin{abstract}
Traditional Time-series Anomaly Detection (TAD) methods often struggle with the composite nature of complex time-series data and a diverse array of anomalies. We introduce TADNet, an end-to-end TAD model that leverages Seasonal-Trend Decomposition to link various types of anomalies to specific decomposition components, thereby simplifying the analysis of complex time-series and enhancing detection performance. Our training methodology, which includes pre-training on a synthetic dataset followed by fine-tuning, strikes a balance between effective decomposition and precise anomaly detection. Experimental validation on real-world datasets confirms TADNet's state-of-the-art performance across a diverse range of anomalies. 
\end{abstract}

\begin{keywords}  
time-series anomaly detection, seasonal-trend decomposition, time-series analysis, end-to-end
\end{keywords}

\section{Introduction}
\label{sec:intro}
Time-series analysis has emerged as a pivotal area of focus across diverse application domains \cite{gu2020request, 10043819, 10.1145/3583780.3615159}. The ability to understand underlying temporal patterns and identify anomalies in these signals is paramount. Time-series Anomaly Detection (TAD) takes on the intricate task of pinpointing such deviations from expected behavior in time-series. Despite extensive efforts to accurately label data, real-world datasets are still prone to errors \cite{lai2021revisiting, SHANG2023110046}. Recognizing these challenges, our work aligns with a growing trend toward semi-supervised anomaly detection for time-series data \cite{xu2021anomaly, pang2021deep}, which operates on the assumption of a purely normal training dataset.

%  unsupervised TAD 主要有两点挑战。
% 挑战1: 真实数据往往来源多样,是多种复杂模式的叠加体现,分析困难 -> models need to learn informative representations from complex temporal dynamics through unsupervised tasks
% 挑战2: 时序异常的类别多,point, contextual, collective, 不同种类的异常差异非常大 -> 即使是在contextual anomalies中,异常产生的原因也可能差距很大,比如unusual shape, lower seasonality, and decreasing trend等等。因此模型 would be challenging to identify these anomalies due to the ambiguity.
\textbf{Challenges.} 
TAD presents two primary challenges. First, real-world data, with its diverse origins, displays a composition of complex patterns. This necessitates models to effectively learn representations from such intricate temporal dynamics without labeled data. Second, temporal anomalies span various categories, including point- (global and contextual) and pattern-wise (shapelet, seasonal, and trend) anomalies \cite{lai2021revisiting, xu2021anomaly}. Specifically, pattern-wise anomalies can arise from varied causes, such as unusual shapes, reduced seasonality, or a declining trend. Such diversity introduces ambiguity, making accurate anomaly detection more challenging for models.
% These categories have marked differences, intensifying the identification task.

\textbf{Existing solutions.}
In recent years, deep learning models \cite{tuli2022tranad, xu2021anomaly} have surpassed classical techniques \cite{scholkopf2001estimating} in TAD tasks. These models generally fall into two categories: autoregression-based and reconstruction-based approaches. Autoregression-based methods identify anomalies through prediction errors \cite{hundman2018detecting}, while reconstruction-based methods \cite{su2019robust, li2021multivariate} use reconstruction errors. Despite their overall accuracy, many of these models fail to account for the complex compositional nature of patterns in time-series data or distinguish between different types of anomalies. Consequently, these approaches often overlook nuanced deviations and lack interpretability.
% with notable models including VAR and LSTMs \cite{tariq2019detecting}. In contrast,

\begin{figure}[t]
% \vspace{-5pt}
\centering
\includegraphics[width=0.95\linewidth]{imgs/Decomp_TAD.pdf}
% \vspace{-20pt}
\caption{Schematic of the STD and TAD workflow, which begins by applying STD to the anomalous time-series, yielding its decomposed components. Subsequently, the original series is compared with the reconstructed series to compute the reconstruction error. Finally, the error is transformed into anomaly scores and labels.}
\label{fig:decomp_vis}
\vspace{-15pt}
\end{figure}

\begin{figure*}[t]
\centering
\includegraphics[width=\linewidth]{imgs/TADNet.pdf}
% \vspace{-10pt}
\caption{Overview of TADNet. The workflow initiates with data augmentation, wherein trend, seasonal, anomalies, and reaminder are integrated to formulate the synthetic dataset. The training of TADnet unfolds in two phases: first, it masters STD on the synthetic data guided by \(L_{\text{dec}}\), followed by fine-tuning on real-world data, leveraging \(L_{\text{rec}}\) to capture typical patterns. The backbone of TADnet consists of an Encoder-Decoder pair, facilitating mapping between the time-domain and latent space, and a separator tasked with generating masks tailored for STD targets. The parameters are shared between the three decoders.}
\label{fig:model}
\vspace{-10pt}
\end{figure*}

\textbf{New insights.} Time series are inherently composed of multiple overlapping patterns: seasonality, trend, and remainder. This overlapping nature can obscure different types of anomalies. The advantage of using Seasonal-Trend Decomposition (STD) \cite{cleveland1990stl} is vividly demonstrated in Fig.\ref{fig:decomp_vis}, which shows how anomalies can be effectively unraveled by separating them into their respective components. By leveraging the power of STD, our approach can uniquely break down these complex composite patterns. Furthermore, following the taxonomy proposed in \cite{lai2021revisiting}, we find that different types of anomalies can be systematically associated with their respective components: seasonal anomalies with the seasonal component, trend anomalies with the trend component, and point anomalies with the remainder component.

Although existing research has incorporated time-series decomposition into TAD task \cite{qin2022decomposed, gao2020robusttad}, these approaches do not follow an end-to-end training manner. Specifically, they either depend on pre-defined decomposition algorithms, necessitating elaborate parameter tuning \cite{gao2020robusttad}, or employ decomposition only for data preprocessing \cite{qin2022decomposed}.
To overcome the lack of supervised signals for end-to-end training, we introduce a novel two-step training approach. Initially, we generate a synthetic dataset that mimics the decomposed components of real-world data. First, we pre-train our model on this synthetic dataset for decomposition tasks. The model is subsequently fine-tuned on real-world anomalous data, yielding enhanced time-series decomposition and anomaly detection.

\textbf{Contributions.} Our contributions are threefold. First, we present TADNet, a novel end-to-end TAD model that leverages STD, as illustrated in Fig.\ref{fig:model}. Inspired by TasNet \cite{luo2018tasnet}, TADNet is designed to handle time-series data similarly to audio signals, and offers detailed decomposition components, enhancing both interpretability and accuracy. Second, we adopt a targeted training strategy with pre-training and fine-tuning to enable end-to-end decomposition and detection. Third, our model achieves state-of-the-art performance and validates its efficacy through decomposition visualizations.

\section{Preliminaries}

\textbf{Time-series Anomaly Detection.} Consider a time-series \(\mathcal{T} \in \mathbb{R}^{T \times D}\) of length \(T\). The series is designated as univariate when \(D = 1\) and multivariate when \(D > 1\). The primary objective of TAD task is to identify anomalies within \(\mathcal{T}\), consequently generating an output series \(\mathcal{Y}\). Each element in \(\mathcal{Y}\) corresponds to the anomalous status of the respective data points in \(\mathcal{T}\), with $1$ indicating an anomaly. 
To facilitate this, point-wise scoring methods \cite{tuli2022tranad, schmidl2022anomaly} are employed to produce an anomaly score series \(\mathcal{S} = \{ s_1, s_2, ... , s_m \}\), each \( s_i \in \mathbb{R} \). These scores are then converted into binary anomaly labels \(\mathcal{Y}\) through an independent thresholding process.

\textbf{Seasonal-Trend Decomposition.} For a univariate time-series \(x \in \mathbb{R}^T\), its structural composition capturing trend and seasonality is represented as \(x_t = \tau_t + s_t + r_t\), where \(\tau_t, s_t\), and \(r_t\) denote the trend, seasonal, and remainder components at the \(t\)-th timestamp, respectively. 

The primary focus of our method for TAD task lies in the seasonal-trend decomposition of univariate time-series. The current literature underscores the benefits of evaluating each variable individually for increased predictive accuracy \cite{zhang2023sageformer}. Thus, each variable in a multivariate series undergoes independent decomposition, while the overall anomaly detection strategy accounts for its multivariate nature.

\textbf{Time-domain Audio Separation.} The task is formulated in terms of estimating \(C\) sources \(s^{(1)}_t, \ldots, s^{(C)}_t \in \mathbb{R}^T\), given the discrete waveform of the mixture \(x_t \in \mathbb{R}^T\). Mathematically, this is expressed as \(x_t = \sum\nolimits_{i=1}^{C} s^{(i)}_t\). 

The field of monaural audio source separation has seen advancements through various deep learning models. TasNet \cite{luo2018tasnet} introduced the concept of end-to-end learning in this area. Conv-TasNet \cite{luo2019conv} further developed this approach by incorporating convolutional layers. DPRNN \cite{luo2020dual} focused on improving long-term modeling through recurrent neural networks. More recently, architectures like SepFormer \cite{subakan2021attention} have integrated attention mechanisms.

The theoretical framework of Time-domain Audio Separation exhibits striking resemblances to the Seasonal-Trend Decomposition tasks, thus leading us to contemplate the possible transference of methodologies between these domains for improved time-series decomposition and anomaly detection.

\section{Methodology}
\label{sec:method}
\subsection{Overall Framework}

The overall flowchart of TADNet is shown in Fig.\ref{fig:model}. The preprocessing contains data normalization and segmentation. The input is first normalized to the range $[0, 1)$. Segmentation involves a sliding window approach of length \( P \), converting the normalized \( \mathcal{T} \) into non-overlap blocks of length $P$ denoted as \( \mathcal{D}=\left\{\mathcal{X}_1, \mathcal{X}_2, \ldots, \mathcal{X}_N\right\} \). Notably, while segmentation offers a more flexible approach to managing longer sequences, it does not influence results.

Within the TADNet backbone, we leverage the TasNet architecture and its variants \cite{luo2018tasnet, luo2019conv, luo2020dual} from speech separation. Viewing the seasonal and trend components as distinct audio signals, TasNet facilitates effective STD (see Fig.\ref{fig:model}).

Since the training utilizes only normal samples, anomalies typically disrupt the reconstruction process. To detect these anomalies, we compute the reconstruction error, denoted by \(Score(t) = \lVert \mathcal{T}_{t,:} - \hat{\mathcal{T}}_{t,:} \rVert_2\), where \(\lVert \cdot \rVert_2\) represents the L2 norm. 
% Essentially, TasNet and its variants comprise an encoder-decoder pair and a separator. The encoder transforms the input time series into a dynamic 2-D representation, while the decoder restores these representations to the original time series format. Meanwhile, the separator generates masks for the three target channels.

\subsection{TADNet Backbone}
\begin{table*}[t!]
    \caption{Quantitative results for TADNet across five real-world datasets use metrics \emph{P}, \emph{R}, and \emph{F1} for precision, recall, and F1-score (\%). Higher values indicate better performance. Best and second-best results are in bold and underlined, respectively. Dataset are followed by brackets, where \emph{u} indicates univariate and \emph{m} multivariate.}\label{tab:main_results}
    % \vspace{-10pt}
    % \vskip 0.05in
    \centering
    \resizebox{\textwidth}{!}{
    % \begin{small}
    \renewcommand{\multirowsetup}{\centering}
    % \setlength{\tabcolsep}{3pt}
    \begin{tabular}{c|ccc|ccc|ccc|ccc|ccc}
    \toprule
    Dataset & \multicolumn{3}{c|}{\textbf{UCR} (\emph{u})}& \multicolumn{3}{c|}{\textbf{SMD} (\emph{m})}& \multicolumn{3}{c|}{\textbf{SWaT} (\emph{m})}& \multicolumn{3}{c|}{\textbf{PSM} (\emph{m})} & \multicolumn{3}{c}{\textbf{WADI} (\emph{m})}\\
    Metric& P& R& F1& P& R& F1& P& R& F1& P& R& F1& P& R& F1 \\
    \midrule

    OCSVM        & 41.14 & 94.00  & 57.23 & 44.34 & 76.72 & 56.19 & 45.39 & 49.22 & 47.23 & 62.75 & 80.89 & 70.67 & 61.89 & 62.31 & 62.10 \\
    % BeatGAN      & 45.20 & 88.42  & 59.82 & 72.90 & 84.09 & 78.10 & 64.01 & 87.46 & 73.92 & 90.30 & 93.84 & 92.04 & 65.13 & 38.32 & 48.25 \\
    OmniAnomaly  & 64.21 & 86.93  & 73.86 & 83.34 & 94.49 & 88.57 & 86.33 & 76.94 & 81.36 & 91.61 & 71.36 & 80.23 & 31.58 & 65.41 & 42.60 \\
    InterFusion  & 60.74 & 95.20  & 74.16 & 87.02 & 85.43 & 86.22 & 80.59 & 85.58 & 83.01 & 83.61 & 83.45 & 83.52 & 80.26 & 30.38 & 44.08 \\
    AnomalyTran  & 72.80 & 99.60  & 84.12 & 89.40 & 95.45 & \underline{92.33} & 91.55 & 96.73 & \textbf{94.07} & 96.91 & 98.90 & \underline{97.89} & 80.30 & 79.23 & 79.76 \\
    TranAD       & 94.07 & 100.00 & \underline{96.94} & 88.03 & 89.42 & 88.72 & 97.60 & 69.97 & 81.51 & 96.44 & 87.37 & 91.68 & 35.29 & 82.96 & 49.51 \\
    DecompTran   & 71.58 & 96.83  & 82.31 & 89.32 & 93.94 & 91.57 & 95.17 & 80.30 & 87.10 & 97.65 & 87.21 & 92.14 & 79.40 & 81.01 & \underline{80.20} \\

    \midrule
    \textbf{TADNet(Ours)}& 97.51 & 100.00 & \textbf{98.74} & 94.81 & 91.93 & \textbf{93.35} & 92.15 & 88.35 & \underline{90.21} & 98.12 & 99.21 & \textbf{98.66} & 94.03 & 82.96 & \textbf{88.15} \\
    \bottomrule
    \end{tabular}
    % \end{small}
    }
    \vspace{-10pt}
\end{table*}

The encoder accepts a univariate time series \(x_d\in\mathbb{R}^{P}\), where \(d=1,2,\ldots,D\), sourced from multivariate \(\mathcal{X}_i\). It segments this series into multiple overlapping frames. Each frame possesses a length of \( L \) and overlaps with adjacent frames by a stride of \( S \). Sequentially, these frames are collated to constitute \( \mathbf{X}_d \in \mathbb{R}^{L \times K} \). Through a subsequent linear transformation, the encoder projects \( \mathbf{X}_d \) into a latent space, given by:
$\mathbf{E} = \mathbf{U}\mathbf{X}_d.$
The matrix \( \mathbf{U} \in \mathbb{R}^{N \times L} \) contains trainable transformation bases in its rows, while \( \mathbf{E} \in \mathbb{R}^{N \times K} \) represents the feature representation of the input time series in the latent space.

% The separator receives this representation as input and learns to estimate the corresponding mask for each decomposed target (trend, seasonal, and remainder):
% \begin{equation}
% \mathcal{M} = \mathcal{F}(\mathbf{E};{\theta}),
% \end{equation}
% where $\mathcal{M} = \{ \mathbf{M}_{\tau},\mathbf{M}_s,\mathbf{M}_r \}$. From this, we can obtain the embedding \( \mathbf{E} \) for each target by applying the generated mask \( M \) on the global feature \( E \):
% \begin{equation}
% \mathbf{E} = \mathbf{M} \odot \mathbf{E}.
% \end{equation}

The separator receives the encoded representation and is tasked with generating masks for each of the decomposed components. Formally, \(\{ \mathbf{M}_{\tau}, \mathbf{M}_s, \mathbf{M}_r \} = \mathcal{F}_{\text{sep}}(\mathbf{E}; \theta)\), where \(\mathbf{M}_{\tau}\), \(\mathbf{M}_s\), and \(\mathbf{M}_r\) represent the masks for the trend, seasonal, and remainder components, respectively. Here, \(\mathcal{F}_{\text{sep}}\) denotes the separator subnetwork, which can be implemented using various architectures such as CNN \cite{luo2019conv}, RNN \cite{luo2020dual}, or Transformer \cite{subakan2021attention}.
Utilizing these masks, the embeddings for each target from the global feature \( \mathbf{E} \) are:
\begin{equation}
\mathbf{E}_{\tau}= \mathbf{M}_{\tau} \odot \mathbf{E}, \quad
\mathbf{E}_s= \mathbf{M}_s \odot \mathbf{E}, \quad
\mathbf{E}_r= \mathbf{M}_r \odot \mathbf{E},
\end{equation}
achieved via point-wise products of the corresponding masks.

The decoder architecture mirrors the encoder, taking in the masked embeddings generated by the separator. These embeddings are mapped back to the time domain via a linear transformation \( V \):
\begin{equation}
\hat{\mathbf{S}}_{\tau} = \mathbf{E}_{\tau}^T \mathbf{V},\quad
\hat{\mathbf{S}}_s = \mathbf{E}_s^T \mathbf{V},\quad
\hat{\mathbf{S}}_r = \mathbf{E}_r^T \mathbf{V}.
\end{equation}
Here, \( V\in\mathbb{R}^{N\times L}\) has \( N \) decoder bases. The reconstructed trend, seasonal, and remainder, denoted as \( \hat{\mathbf{S}}_{\tau} \), \( \hat{\mathbf{S}}_s \), and \( \hat{\mathbf{S}}_r \), are derived from their respective embeddings. The output time-domain signals, $\hat\tau_d$, $\hat s_d$, and $\hat r_d$, are obtained through an overlap-and-add operation.

\subsection{Synthetic Dataset}

In anomaly detection, real-world data often lack the nuanced trend and seasonal patterns essential for STD. To equip TADnet with robust STD capabilities, we constructed a composite dataset. This dataset is meticulously designed with intricate seasonal and trend shifts, anomalies, and noise to emulate real-world contexts, as illustrated in Fig.\ref{fig:decomp_vis}. Both deterministic and stochastic trends are leveraged to craft the trend and seasonal components, which are subsequently normalized to maintain a zero mean and unit variance.


% The deterministic trend is generated using a quadratic trend function with stochastic coefficients:
% $\tau_t^{(d)} = \beta_0+\beta_1\cdot t + \beta_2 \cdot t^2$, 
% where \( \beta_0 \), \( \beta_1 \), and \( \beta_2 \) stand as tunable parameters. The stochastic trend is conceived using \(\operatorname{ARIMA}(0,2,0)\).
\textbf{Trend.}
The deterministic trend is generated using a linear trend function with fixed coefficients: \( \tau_t^{(d)} = \beta_0 + \beta_1 \cdot t \), where \( \beta_0 \) and \( \beta_1 \) are tunable parameters. The stochastic trend component is modeled using an ARIMA(0,2,0) process, integrated into the trend model as follows:  $\tau_t^{(s)}=\sum_{n=1}^t n X_n$, where \( X_t \) is a normally-distributed white noise term, satisfying \( \Delta^2 \tau^{(s)}_t = X_t \).

\textbf{Seasonal.}
The deterministic seasonal component combines various types of periodic signals. It includes sinusoidal waves with varying amplitudes, frequencies, and phases, as well as square waves with different amplitudes, periods, and phases. 

For the slow-changing stochastic sequence, the seasonal component is composed of repeating cycles of a slow-changing trend series \( \tau^{(s)}_t \). This series is generated using the trend generation algorithm to ensure a smooth transition between cycles. Each cycle is uniquely characterized by a period \( T_0 \) and a phase \( \phi \). The stochastic seasonal component is thus formulated as \( s_t = \tau^{(s)}_{\text{mod}(t + \phi, T_0)} \).

To enrich the dataset, minor adjustments are made to both the cycle's length and amplitude, including resampling individual cycles and scaling values within a cycle, aiming for more diverse and generalizable signal decomposition.

\textbf{Remainder.}
The remainder component is conceived using a white noise process with adjustable variances.
% \begin{equation}
% s_t^{(d)}=\sum_{i=1}^n A_{s_i} \cdot \sin \left(2 \pi f_{s_i} t+\phi_{s_i}\right)+\sum_{j=1}^m A_{q_j} \cdot \operatorname{sgn}\left(\sin \left(2 \pi f_{q_j} t+\phi_{q_j}\right)\right)
% \end{equation}

To enhance the robustness of the decomposition model against anomalies and ensure stable decomposition performance, we injected a portion of anomalous data into the synthetic dataset, following the method outlined in \cite{lai2021revisiting}.

\subsection{Two-Phase Training Strategy}
We propose a two-phase training strategy for TADnet to ensure its efficacy in both time-series decomposition and anomaly detection tasks.

In the first phase, TADnet is pre-trained on a synthetic dataset, with a focus on time-series decomposition. The corresponding loss function, which aggregates the Mean Squared Errors for each decomposed component, is formulated by:
\begin{equation}
    L_{\text{dec}} = \sum\nolimits_{d=1}^{D} \left( \lVert \tau_d-\hat\tau_d \rVert_2^2 + \lVert s_d-\hat s_d \rVert_2^2 + \lVert r_d-\hat r_d \rVert_2^2 \right)
\end{equation}
Here, \( \tau_d \), \( s_d \), and \( r_d \) denote the actual seasonal, trend, and residual components for the \( d \)-th dimension, respectively, while \( \hat{\tau}_d \), \( \hat{s}_d \), and \( \hat{r}_d \) represent their predicted counterparts.

In the second phase, TADnet is fine-tuned using a real-world TAD dataset. This stage emphasizes the accurate reconstruction of the original time series following its decomposition, a key requirement for effective anomaly detection. The loss function for this stage, which focuses on overall reconstruction accuracy, is given by:
\begin{equation}
L_{\text{rec}} = \sum\nolimits_{d=1}^{D} \lVert x_d- (\hat\tau_d + \hat s_d)\rVert_2^2
\end{equation}
Here, \( x_d \) represents the original time series in the \( d \)-th dimension, and \( \hat{\tau}_d + \hat{s}_d \) is its predicted reconstruction.

\section{The Semantic Urban Mesh Dataset}\label{sec:framework}
\subsection{Dataset Specification}

We have used Helsinki's 3D texture meshes as input and annotated them as a benchmark dataset of semantic urban meshes. 
The Helsinki's raw dataset covers about 12 $ km^2 $, and it was generated in 2017 from oblique aerial images that have about a 7.5 $cm$  ground sampling distance (GSD) using an off-the-shelf commercial software namely ContextCapture~\citep{contextcap}.
The source images have three colour channels (i.e., red, green, and blue) and are collected from an airplane with five cameras that have $80\%$ length coverage and $60\%$ side coverage.
To recover the 3D water bodies that do not fulfil the Lambertian hypothesis, 2D vector maps and ortho-photos are used when performing the surface reconstruction.
Furthermore, processing like aerial triangulation, dense image matching, and mesh surface reconstruction were all performed with ContextCapture.
It should be noticed that the entire region of Helsinki is split into tiles, and each of them covers about 250 $ m^2 $~\citep{kalasatamaReport}.
As shown in Figure \ref{fig:overview},  we have selected the central region of Helsinki as the study area, which includes 64 tiles and covers about 4 $km^2$ map area (8 $km^2$ surface area) in total.   

\subsection{Object Classes}
We define the semantic categories for urban meshes by the most common objects in the urban environment with unambiguous geometry and texture appearance.
Moreover, each triangle face is assigned to a label of one of the six semantic classes. 
Ambiguous regions (which account for about 2.6\% of the total mesh surface area), such as shadowed regions or distorted surfaces, are labelled as unclassified (see Figure \ref{fig:ambigious}).
The object classes we consider in the benchmark dataset are: 
\begin{itemize}
	\item \textbf{terrain}: roads, bridges, grass fields, and impervious surfaces;
	\item \textbf{building}: houses,high-rises, monuments, and security booths;
	\item \textbf{high vegetation}: trees, shrubs, and bushes;
	\item \textbf{water}: rivers, sea, and pools;
	\item \textbf{vehicle}: cars, buses, and lorries;  
	\item \textbf{boat}: boats, ships, freighters, and sailboats;
	\item \textbf{unclassified}: incomplete objects like buses and trains, distorted surfaces like tables, tents and facades, construction sites, underground walls.
\end{itemize}

\begin{figure}[!tb]
	\includegraphics[height=0.48\textwidth]{figures/overview_grids/yaxis.png}
	\begin{subfigure}[t]{0.48\textwidth}
		\includegraphics[width=\linewidth]{figures/overview_grids/texture_global_birdsview00.png}
		\includegraphics[width=\linewidth]{figures/overview_grids/xaxis.png}
		\label{fig:textop}
	\end{subfigure}
	\hspace*{\fill}
	\begin{subfigure}[t]{0.48\textwidth}		
		\includegraphics[width=\linewidth]{figures/overview_grids/semantic_global_birdsview00.png}
		\vspace*{-0.78cm}
		\begin{center}
		\includegraphics[width=0.8\linewidth]{figures/semantic_results/semantic_legend2.png}
		\end{center}
		\label{fig:semtop}
	\end{subfigure}
	\vspace*{-0.7cm}
	\caption{Overview of the semantic urban mesh benchmark.
	Left: the texture meshes covering about 4 $km^2$ map area. Right: the ground truth meshes.
	More views of the same scene (with different visualization styles) are shown in Figures \ref{fig:texside} and \ref{fig:semside}.}
	\label{fig:overview}
\end{figure}

\begin{figure}[!tb]
	\centering
	\begin{subfigure}[t]{0.48\textwidth}
		\includegraphics[width=\linewidth]{figures/ambigious/shadow_tex_zoom.png}
		\caption{}
	\end{subfigure}
	\hspace*{\fill}
	\begin{subfigure}[t]{0.48\textwidth}
		\includegraphics[width=\linewidth]{figures/ambigious/shadow_fc_zoom.png}
		\caption{}
	\end{subfigure}
	\begin{subfigure}[t]{0.48\textwidth}
		\includegraphics[width=\linewidth]{figures/ambigious/distort_tex_zoom.png}
		\caption{}
	\end{subfigure}
	\hspace*{\fill}
	\begin{subfigure}[t]{0.48\textwidth}
		\includegraphics[width=\linewidth]{figures/ambigious/distort_fc_zoom.png}
		\caption{}
	\end{subfigure}
	\caption{Ambiguous regions are labelled as unclassified (in black). 
		(a) Shadow region with texture.
		(b) Shadow region with semantic colour.
		(c) Distorted region with texture.
		(d) Distorted region with semantic colour.} 
	\label{fig:ambigious}
\end{figure}


\subsection{Semi-automatic Mesh Annotation}  \label{sec:mesh_annota}
Rather than manually labelling each triangle face of the raw meshes, we design a semi-automatic mesh labelling framework to accelerate the labelling process. Figure~\ref{fig:pipeline} shows the overall pipeline of our labelling workflow.

Given the fact that urban environments consist of a large number of planar regions in the data, we opt to label the data at the segment level instead of individual triangle faces. 
Specifically, we over-segment the input meshes into a set of planar segments. 
These segments can enrich local contextual information for feature extraction and serve as the basic annotation unit to improve annotation efficiency.

\begin{figure}[!tb]
	\centering
	\includegraphics[width=\textwidth]{figures/pipeline/pipeline_L1.png}
	\caption{The pipeline of the labelling workflow.}
	\label{fig:pipeline}
\end{figure}

Instead of randomly choosing a mesh tile as input for annotation and refinement, which is insufficient for manual annotation progress, we favour picking a mesh tile that is more difficult to classify.
Similar to active learning, we first compute the feature diversity (see Equation \ref{eq:fea_div}) to optimally select a mesh tile containing a variety of classes and objects at different scales and complexity.
The feature diversity $F_{m}$ of tile $m$ is computed as
\begin{equation}\label{eq:fea_div}
	F_{m}=\frac{\sum_{i=1}^{N_{f}}\left ( f_i - \bar{f} \right )^{2}}{N_{f}}
\end{equation}
where $f_i$ represents each handcrafted feature which describe in Section \ref{sec:initial_seg}, and $\bar{f}$ is mean value of a $N_{f}$ dimensional feature vector.
To acquire the first ground truth data, we manually annotate the mesh (with segments) that is selected with the highest feature diversity.
Then, we add the first labelled mesh into the training dataset for the supervised classification.
Specifically, we use the segment-based features as input for the classifier, and the output is a pre-labelled mesh dataset.
Next, we use the mesh annotation tool to manually refine the pre-labelled mesh according to the feature diversity.
Finally, the new refined mesh will be added to the training dataset to improve the automatic classification accuracy incrementally.


\subsubsection{Initial Segmentation}\label{sec:initial_seg}

To avoid redundant computations of numerous triangles, we first apply mesh over-segmentation (i.e., linear least-squares fitting of planes) based on region growing on the input data to group triangle faces into homogeneous regions~\citep{lafarge2012creating}.
Such grouped regions are beneficial for computing local contextual features.
We then extract both geometric and radiometric features from those mesh segments as follows: 
\begin{itemize}
	\item[$\bullet$] \textit{Eigen-based features} are computed from the covariance matrix of the triangle vertices with respect to the average centre within each segment, which is beneficial for identifying urban objects with various surface distributions.
	The linearity $= (\lambda_{1} - \lambda_{2}) / \lambda_{1}$, sphericity $= \lambda_{3}/ \lambda_{1}$ and change of curvature $= \lambda_{3} / (\lambda_{1} + \lambda_{2} + \lambda_{3})$ are computed based on the three eigenvalues $\lambda_{1} \geq \lambda_{2} \geq \lambda_{3}\geq 0$.
	The local eigenvectors $\mathbf{n}_{i} $ and the unit normal vector $\mathbf{n}_{z} $ along Z-axis are used to compute the verticality $=1-\left | \mathbf{n}_{i}\cdot \mathbf{n}_{z} \right | $~\citep{hackel2016fast}.
	Note that many eigen-based features have been studied in literature~\citep{hackel2016fast,west2004context,weinmann2013feature}, and some of them were designed for and tested on LiDAR point clouds. 
	\textcolor{ao}{
	These eigen-based features are mostly computed per point based on its spherical neighbourhood, which often contains noise and does not form a surface. 
	Our chosen eigen-based features are defined on a segment representing the surface of a mesh, and thus they can capture non-local geometric properties of an object.
	}
	Additionally, in this work, we have tested all eigen-based features from the literature~\citep{hackel2016fast}, and we only present the ones that are effective for texture meshes.
	\item[$\bullet$] \textit{Elevation} is divided into absolute elevation $z_{a}$, relative elevation $z_{r}$ and multiscale elevations $z_{m}$.
	Where $z_{a}$ is the average elevation of the segment;
	the relative elevation is computed as $z_{r} = z_{a}-z_{r_{min}}$;
	the multiscale elevation~\citep{Verdie2015,Rouhani2017} $z_{m} = \sqrt{\frac{z_{a} - z_{min}}{z_{max} - z_{min}}}$.
	And $z_{r_{min}}$ denotes the lowest elevation of the local largest ground segment computed within a cylindrical neighbourhood with 30 meters radius around the segment centre.
	$z_{min}$ and $z_{max}$ represent the local minimum and maximum elevation values of a cylindrical neighbourhood within the scale of 10 meters, 20 meters, and 40 meters.
	Such large cylindrical neighbourhoods allow to find the local ground considering the resilience to hilly environments, \textcolor{ao}{and the square root ensures that small relative height values (i.e., values smaller than 1 $ m $) get a larger elevation attribute to enlarge elevation differences between small objects and the local ground (e.g., cars against the ground, boats against the water surfaces).}
	More importantly, due to the influence of terrain fluctuations and various scales of urban objects, the elevation of these three categories can complement each other.
	\item[$\bullet$] \textit{Segment area} is computed as $area(S_k) = \sum_{i = 1}^{N} area(f_i) $, where $f_i$ denotes a triangle of the segment $S_k$, and $N$ denotes the total number of triangles in $S_k$.
	\item[$\bullet$] \textit{Triangle density} is defined as $density(S_k) = \frac{N}{area(S_k)} $,  which reveals the object complexity, especially for adaptive urban meshes.
	\item[$\bullet$] \textit{Interior radius of 3D medial axis transform (InMAT)}~\citep{ma20123d,peters2016robust} of a segment $S_k$ is formulated as $r_k = \frac{\sum_{i=1}^{M} r_i}{M}$, where $M$ denotes the total number of triangle vertices of $S_k$, and $r_i$ denotes the interior radius of the shrinking ball that touches the vertex $v_i$ within the segment $S_k$. 
	It is designed to distinguish objects with different scales. 
	\item[$\bullet$] \textit{HSV colour-based features} are derived from the RGB channel of the entire texture map.
	We use the HSV colour space since it can better differentiate different objects than RGB.
	We compute the average colour, the variance of the colour distribution of all pixels within each segment, and we further discretize it into a histogram that consists of 15 bins of the hue channel, five bins of the saturation channel, and five bins of the value channel.
	\item[$\bullet$] \textit{Greenness} $a_{g}$ is used to classify objects that are similar to green vegetation.
	Specifically, it is computed according to the averaged RGB colour of each segment via $a_{g}=G-0.39\cdot R-0.61\cdot B$~\citep{mckinnon2017comparing}. 
\end{itemize}
	All the above features are concatenated into a 44-dimensional feature vector used by our random forest (RF) classifier in the initial segmentation. 

\subsubsection{Annotation Tool for Refinement}

Because of the under-segmentation errors and the imperfect results of the semantic mesh segmentation process, we design a mesh annotation tool (see Figure \ref{fig:annotator}) to manually correct the labelling errors.
Our mesh annotation tool is developed based on the labelling tool of CGAL~\citep{cgal:eb-20b}.

\begin{figure}[!tb]
	\centering
	\includegraphics[width=\textwidth]{figures/annotator/annotator.png}
	\caption{The interface of our annotation tool for 3D texture meshes. }
	\label{fig:annotator}
\end{figure}

As shown in Table \ref{tab:annotation_operation}, it consists of three operation categories: view, selection, and annotation.
The	view operations provide essential functions for the user to manipulate the scene camera, such as translate, rotate, zoom, or set the new pivot for the scene.
In addition, to use textures as a reference for labelling, we map texture and face colour with a certain degree of transparency, and we visualize the segment border to differentiate each segment. 

\begin{table}[!tb]
	\centering
	\noindent\adjustbox{max width=0.8\textwidth}
	{
		\begin{threeparttable}
			\centering
			\begin{tabular}{ccc}
				\toprule
				Categories & Operations & Objects \\
				\midrule
				\multirow{4}[2]{*}{View} & Translate & Camera \\
				& Rotate & Camera \\
				& Zoom in / out & Camera \\
				& Set pivot & Camera \\
				\midrule
				\multirow{6}[2]{*}{Selection} & Multi-selection / Lasso & Triangles / Segments \\
				& Expand / Reduce & Triangles / Segments \\
				& Semantic selection & Segments \\
				& Split region & Segments \\
				& Planar region extraction & Triangles \\
				& Split mesh & Triangles \\
				\midrule
				\multirow{3}[2]{*}{Annotation} & Probability slider & Segments \\
				& Segment area slider & Segments \\
				& Progress bar & Triangles \\
				& Switch semantic view & Triangles \\ 
				& Labelling & Triangles / Segments \\
				\bottomrule
			\end{tabular}%
		\end{threeparttable}
	}
	\caption{Basic operations in our annotation tool.} 
	\label{tab:annotation_operation}%
\end{table}%


The	selection operations allow the user to select or deselect either triangle faces (see Figure \ref{fig:tri_sel}) or segments (see Figure \ref{fig:seg_sel}) freely via a brush or a lasso.
Specifically, the face selection operation is used to fix the under-segmentation errors and generate new segments, and the segment selection operation is to fix incorrect segment labels.

\begin{figure}[!tb]
	\centering
	\begin{subfigure}[t]{0.32\textwidth}
		\includegraphics[width=\linewidth]{figures/pipeline/tri_select_a.png}
		\caption{}
	\end{subfigure}
	\hspace*{\fill}
	\begin{subfigure}[t]{0.32\textwidth}
		\includegraphics[width=\linewidth]{figures/pipeline/tri_select_b.png}
		\caption{}
	\end{subfigure}
	\hspace*{\fill}
	\begin{subfigure}[t]{0.32\textwidth}
		\includegraphics[width=\linewidth]{figures/pipeline/tri_select_c.png}
		\caption{}
	\end{subfigure}
	\caption{An example of labelling by selecting triangles using the lasso tool (blue edges: segment boundaries). 
		(a) Before selection.
		(b) Lasso selection result (in red).
		(c) The correct label has been assigned to the selected region. 
		In this example, the label of the selected region has been changed from `ground' to `vehicle'.
	} 
	\label{fig:tri_sel}
\end{figure}


\begin{figure}[!tb]
	\centering
	\begin{subfigure}[t]{0.32\textwidth}
		\includegraphics[width=\linewidth]{figures/pipeline/seg_select_a.png}
		\caption{}
	\end{subfigure}
	\hspace*{\fill}
	\begin{subfigure}[t]{0.32\textwidth}
		\includegraphics[width=\linewidth]{figures/pipeline/seg_select_b.png}
		\caption{}
	\end{subfigure}
	\hspace*{\fill}
	\begin{subfigure}[t]{0.32\textwidth}
		\includegraphics[width=\linewidth]{figures/pipeline/seg_select_c.png}
		\caption{}
	\end{subfigure}
	\caption{An example of segment labelling. 
		(a) Part of a wall of the building was previously labelled as `high vegetation' (in green).
		(b) Segment selection result (in red).
		(c) The label of the selected segment has been corrected with the new label `building'.
	}
	\label{fig:seg_sel}
\end{figure}

We also allow the user to edit the selection of each individual segment with splitting functions (see Figure \ref{fig:pnp_func}) and automatic extraction of the most planar region (see Figure \ref{fig:seg_func}). 
As for splitting, we first detect the potential planar and non-planar segments marked by user strokes, and then the non-planar one is split according to the vertex-to-plane distance.
It allows generating candidate non-planar regions (with respect to the detected planar segment) for the user to edit, and
it is useful to split a segment that covers large non-planar regions or contains more than one dominant planar area.
To extract the most planar region, we apply the region growing algorithm~\citep{lafarge2012creating} within the selected segment to automatically generate the candidate triangle faces with user-defined thresholds (i.e., the maximum distance to the plane, the maximum accepted angle, and the minimum region size).
Such an operation allows the user to filter out some small bumpy regions of the selected segment.

\begin{figure}[!tb]
	\centering
	\begin{subfigure}[t]{0.48\textwidth}
		\includegraphics[width=\linewidth]{figures/annotator/pnp_pipeline1.png}
		\caption{}
	\end{subfigure}
	\hspace*{\fill}
	\begin{subfigure}[t]{0.48\textwidth}
		\includegraphics[width=\linewidth]{figures/annotator/pnp_pipeline2.png}
		\caption{}
	\end{subfigure}
	\caption{An example splitting planar and non-planar regions. 
		(a) The user draws a stroke (in red) across the border of the non-planar segment and the planar segment. 
		(b) The detected non-planar segment has been split into two parts (i.e., a non-planar region shown in red and a planar segment shown in green).
	} 
	\label{fig:pnp_func}
\end{figure}

\begin{figure}[!tb]
	\centering
	\begin{subfigure}[t]{0.48\textwidth}
		\includegraphics[width=\linewidth]{figures/annotator/planar_split_pipeline1.png}
		\caption{}
	\end{subfigure}
	\hspace*{\fill}
	\begin{subfigure}[t]{0.48\textwidth}
		\includegraphics[width=\linewidth]{figures/annotator/planar_split_pipeline3.png}
		\caption{}
	\end{subfigure}
	\caption{Editing an individual segment. 
		(a) A segment is selected (highlighted in green) for splitting. 
		(b) Automatic extraction of the most planar region (shown in red) within the selected segment according to user-defined thresholds.} 
	\label{fig:seg_func}
\end{figure}

Besides, probability and area-based sliders and a progress bar are provided in the annotation panel to improve annotation efficiency and experience, respectively. 
Specifically, the probability slider is introduced for the user to visually inspect the segments that are most likely misclassified.
Moreover, the user can further use it to inspect a specific class by switching the view to highlight a specific semantic class.
The segment area slider is used to identify isolated tiny segments, which commonly appear as errors.
The progress bar is used to indicate the estimated labelling progress during the annotation.
After performing the selection, the user can easily assign the corresponding label to the selected area.


\section{Experiments}

\begin{figure*}[t]
% \vspace{-5pt}
\centering
\includegraphics[width=0.96\linewidth]{imgs/anomaly_real.pdf}
\vspace{-10pt}
\caption{Visualization of decomposition and detection results in UCR and SMD. The first row shows the raw time series with anomalies, the second and third rows display the seasonal and trend components, respectively, and the final row depicts the reconstruction error. Anomalies are marked with a red background.}
\label{fig:real_results}
% \vspace{-15pt}
\end{figure*}

% \footnote{\url{https://www.cs.ucr.edu/~eamonn/time_series_data_2018/}}
In our experiments, we employ five real-world datasets encompassing both univariate and multivariate time-series, as outlined in Table \ref{tab:dataset}. 
These include multi-univariate dataset UCR \cite{keogh2021multi}, featured in the KDD 2021 Cup; SMD \cite{su2019robust}, which provides five weeks of data from a leading Internet company; and SWaT \cite{mathur2016swat}, offering sensor data from water treatment plant. WADI extends SWaT but contains over twice as many sensors and actuators. Additionally, PSM \cite{abdulaal2021practical} originates from eBay's application server nodes.
NASA's MSL and SMAP \cite{hundman2018detecting} datasets were excluded due to their abundance of binary sequences, which are incompatible with our decomposition methods.
% Wadi ref: \cite{ahmed2017wadi}

We benchmark TADNet against: one classic method, OCSVM \cite{scholkopf2001estimating}, and five deep models including OmniAnomaly \cite{su2019robust}, InterFusion \cite{li2021multivariate}, AnomalyTran \cite{xu2021anomaly}, TranAD \cite{tuli2022tranad}, and DecompTran \cite{qin2022decomposed}. Other traditional methods are excluded, as deep learning models have been proven superior \cite{xu2021anomaly}. Evaluation is based on standard TAD metrics such as precision, recall, and F1 score. We adopt a widely-used adjustment strategy \cite{su2019robust, xu2021anomaly, tuli2022tranad, qin2022decomposed}: if any time point in an abnormal segment is detected, the entire segment is considered correctly identified, aligning with real-world applications.

We partition the dataset into blocks of \(P=8,000\) time points to optimize memory. The backbone utilizes TasNet with DPRNN \cite{luo2020dual}, featuring kernel size \(W=2\), encoding dimension \(E=256\), feature dimension \(F=64\), hidden dimension \(H=128\), and six layers. Training employs ADAM with an initial learning rate of \(1 \times 10^{-3}\) for 200 epochs, followed by fine-tuning at \(5 \times 10^{-4}\) for 10-20 epochs. Experiments run on an NVIDIA GeForce RTX 3090. To ensure fair comparisons with previous work, the Peak Over Threshold method \cite{siffer2017anomaly} is employed to determine the anomaly threshold. A timestamp is labeled as anomalous if its score exceeds this threshold.

\subsection{Results and Analysis}
Our study presents the experimental outcomes in Table \ref{tab:main_results}, where TADNet is rigorously evaluated against six baselines on five real-world datasets. Demonstrating outstanding performance, TADNet notably achieves the highest F1-score in four of the datasets. The application of seasonal trend decomposition in our method effectively discerns complex temporal patterns, broadening the anomaly detection scope. These results empirically substantiate the effectiveness of decomposition techniques in time-series anomaly detection. Additionally, while TADNet's two-stage pre-training extends the training duration to less than an hour, its inference time aligns with that of benchmark algorithms.

\textbf{Visualization.}
To demonstrate the effectiveness of STD in unraveling complex anomalies, we showcase decomposition and reconstruction error results on multiple real-world datasets in Fig.\ref{fig:decomp_vis} (NeurIPS-TS) and Fig.\ref{fig:real_results} (UCR and SMD). It's noteworthy that anomalies become more clearly evident when comparing their respective decomposed components to the original series. As the reconstruction error and anomaly scores are positively correlated, our reconstruction error closely aligns with the anomalous regions, as seen in the last row. This validates that our method adeptly identifies anomalies, thereby improving detection accuracy while reducing false positives.

\begin{table}
  \centering 
  \setlength{\tabcolsep}{5mm}
  \caption{The ablation of different modules in Omni-Relational Network reported with MSE on MNIST given 10$\%$ context points by ablating (1) graph structure (2) attentive pooling (3) positional embedding (P.E.) (4) information bottleneck constrain (I.B.).}
  \begin{tabular}{lcccc}
    \toprule
      Graph       & Attention       & P.E.            & I.B.             & MSE \\
    \midrule
                  &                 &                &                   & $0.073$\\
    $\checkmark$   &              &                 &                    & $0.066$\\
    $\checkmark$  &               &               &    $\checkmark$    & $0.058$\\
    $\checkmark$  &                 &   $\checkmark$     &  $\checkmark$   & $0.053$\\
    $\checkmark$  &  $\checkmark$  &               &     $\checkmark$   & $0.049$\\
    $\checkmark$  &  $\checkmark$  &   $\checkmark$    &  $\checkmark$   & $0.045$\\
    \bottomrule
  \end{tabular}
  \label{tab:3-ablation}
\end{table}

\textbf{Ablation Study.}
As shown in Table \ref{tab:ablation}, we further investigate the effect of each part in TADNet. The removal of key components such as the Separator (\textit{w/o Sep}), Decomposition (\textit{w/o Decomp}), or Augmentation (\textit{w/o Augment}) leads to substantial drops in F1-scores, highlighting their essential roles in TADNet's performance. While the iterative training approach (\textit{Iterative}) shows some improvement in specific datasets (notably WADI, increased to 92.06\%), its computational overhead makes it less practical for broader applications. We therefore opt for the pretrain-finetune paradigm and leave the study of iterative training for future work.

\section{Conclusion}
In this work, we present TADNet, an end-to-end TAD model that employs STD to tackle the challenges of complex patterns and diverse anomalies. Our model distinguishes itself by providing interpretable and accurate decomposition components. The model's effectiveness is enhanced through a dual-phase training approach, initially using synthetic data and subsequently fine-tuning with real-world data, achieving top-tier performance with clear decomposition visualizations. While primarily effective for univariate time series, TADNet extends to multivariate series, though its accuracy improvement in certain complex multivariate scenarios may be limited.
% 


% Below is an example of how to insert images. Delete the ``\vspace'' line,
% uncomment the preceding line ``\centerline...'' and replace ``imageX.ps''
% with a suitable PostScript file name.
% -------------------------------------------------------------------------
% \begin{figure}[htb]

% \begin{minipage}[b]{1.0\linewidth}
%   \centering
%   \centerline{\includegraphics[width=8.5cm]{image1}}
% %  \vspace{2.0cm}
%   \centerline{(a) Result 1}\medskip
% \end{minipage}
% %
% \begin{minipage}[b]{.48\linewidth}
%   \centering
%   \centerline{\includegraphics[width=4.0cm]{image3}}
% %  \vspace{1.5cm}
%   \centerline{(b) Results 3}\medskip
% \end{minipage}
% \hfill
% \begin{minipage}[b]{0.48\linewidth}
%   \centering
%   \centerline{\includegraphics[width=4.0cm]{image4}}
% %  \vspace{1.5cm}
%   \centerline{(c) Result 4}\medskip
% \end{minipage}
% %
% \caption{Example of placing a figure with experimental results.}
% \label{fig:res}
% %
% \end{figure}


% To start a new column (but not a new page) and help balance the last-page
% column length use \vfill\pagebreak.
% -------------------------------------------------------------------------
%\vfill
%\pagebreak
\vfill\pagebreak
\bibliographystyle{IEEEbib}
{
\ninept
% \balance
\bibliography{refs}
}


\end{document}
