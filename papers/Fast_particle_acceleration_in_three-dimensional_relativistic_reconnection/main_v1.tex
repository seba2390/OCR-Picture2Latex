\listfiles

\documentclass[twocolumn,twocolappendix]{aastex63}
\usepackage{amssymb, amsfonts, amsmath,framed}

\usepackage{gensymb}
\usepackage{bm}
\usepackage{multirow}
\usepackage{mathrsfs}

\usepackage{graphicx,verbatim}
\usepackage{stackengine}
\usepackage[caption=false]{subfig}
\usepackage{threeparttable}
\usepackage{tabularx}

%\usepackage[colorlinks=true,linkcolor=blue]{hyperref}
\expandafter\ifx\csname package@font\endcsname\relax\else
 \expandafter\expandafter
 \expandafter\usepackage
 \expandafter\expandafter
 \expandafter{\csname package@font\endcsname}
\fi
\hyphenation{title}\expandafter\ifx\csname package@font\endcsname\relax\else
 \expandafter\expandafter
 \expandafter\usepackage
 \expandafter\expandafter
 \expandafter{\csname package@font\endcsname}
\fi
\hyphenation{title}


%%%%%%%%%%%%%%%%%%%%%%%%%%%%%%%%%%%%
\newcommand{\figg}[1]{Fig.~\ref{fig:#1}}
\def\bi{\begin{itemize}}
\def\ei{\end{itemize}}
\def\be{\begin{eqnarray}}
\def\ee{\end{eqnarray}}
%%%%%%%%%%%%%%%%%%%%%%%%%%%%%%%%%%%%
%\newcommand{\fig}[1]{Fig.~\ref{fig:#1}}
\newcommand{\eq}[1]{Eq.~(\ref{eq:#1})}
\newcommand{\gammacool}{\gamma_{\rm cool}}
\newcommand{\gammaacc}{\gamma_{\rm acc}}
\newcommand{\gammacr}{\gamma_{\rm cr}}
\newcommand{\ls}{\textcolor{teal}}
\newcommand{\lss}{\textcolor{green}}
\newcommand{\hz}{\textcolor{orange}}
\newcommand{\comp}{c/\omega_{\rm p}}
\newcommand{\omp}{\omega_{\rm p}}



 
%%%%%%%%%%%%%%%%%%%%%%%%%%%%%%


\begin{document}
%\title{Ultra-high-energy particles from relativistic reconnection}
\title{Fast particle acceleration in three-dimensional relativistic reconnection}


%\correspondingauthor{Hao Zhang}
\email{zhan2966@purdue.edu \\ lsironi@astro.columbia.edu}

\author{Hao Zhang}
\affiliation{Department of Physics, Purdue University, West Lafayette, IN, 47907, USA}

\author{Lorenzo Sironi}
\affiliation{Department of Astronomy and Columbia Astrophysics Laboratory, Columbia University, New York, NY 10027, USA}

\author{Dimitrios Giannios}
\affiliation{Department of Physics, Purdue University, West Lafayette, IN, 47907, USA}


\begin{abstract}
Magnetic reconnection is invoked as one of the primary mechanisms to produce energetic particles. %In this paper, with large-scale three-dimensional (3D) particle-in-cell simulations of relativistic magnetic reconnection in pair plasmas with magnetization $\sigma=10$ and a weak guide field, we study the acceleration process of high energy particles. 
We employ large-scale three-dimensional (3D) particle-in-cell simulations of reconnection in magnetically-dominated ($\sigma=10$) pair plasmas to study the energization physics of high-energy particles. We identify a novel acceleration mechanism that only operates in 3D. For weak guide fields, 3D plasmoids / flux ropes extend along the $z$ direction of the electric current for a length comparable to their cross-sectional radius. Unlike in 2D simulations, where particles are buried in plasmoids, in 3D we find that a fraction of particles with $\gamma\gtrsim 3\sigma$ can escape from plasmoids by moving along $z$, and so they can experience the large-scale fields in the upstream region. These ``free'' particles preferentially move in $z$ along Speiser-like orbits sampling both sides of the layer, and are accelerated linearly in time --- their Lorentz factor scales as $\gamma\propto t$, in contrast to $\gamma\propto \sqrt{t}$ in 2D. The energy gain rate approaches $\sim eE_{\rm rec}c$, where $E_{\rm rec}\simeq 0.1 B_0$ is the reconnection electric field and $B_0$ the upstream magnetic field. The spectrum of free particles is hard, $dN_{\rm free}/d\gamma\propto \gamma^{-1.5}$,  contains $\sim 20\%$ of the dissipated magnetic energy independently of domain size, and extends up to a cutoff energy scaling linearly with box size. Our results demonstrate that relativistic reconnection in GRB and AGN jets may be a promising mechanism for generating ultra-high-energy cosmic rays.
%We use pair plasmas with magnetization $\sigma=10$ in a weak guide field.
%We find a new mechanism besides accelerating in compressing plasmoids by the increasing local magnetic fields as in 2D. Particles in 3D can also travel outside of the reconnection region. They gain kinetic energy from the large-scale electric field in the unreconnected upstream at a fast constant rate. We find that those particles can escape from plasmoids at $\gamma\sim 3\sigma$ %when they travel along the current sheet disrupted by the drift kink instability.
%due to finite length in the $z$-direction (along electric field in the upstream).
%We show that they follow Speiser orbits and are accelerated to high energy along $z$ \ls{[i would mention that they move along electric field ["along" needs to be made a bit more accurate]]}. They spend most of their time and gain most of their energy outside the reconnection layer. This energization process leads to %a power-law particle spectrum $(\gamma-1){\rm d} N/{\rm d}\gamma \propto (\gamma-1)^{-s}$ with $s\gtrsim 1.0$ and 
%a much larger cutoff in particle spectrum compared to the 2D result from a similar configuration. We find that a larger box can produce particles of {\it proportionally?} higher energy. Although the number of particles (as a fraction of total reconnected particles) that escape from the reconnection region decreases as the box size increases, the energy they carry remains constant at around $20\%$ of dissipated energy. Our results demonstrate that relativistic magnetic reconnection could be the acceleration mechanism for ultra-high-energy cosmic rays produced in GRB and AGN jets.
%or pulsar winds.
%\hz{[is there a word limit on abstract?]}
\end{abstract}

\keywords{magnetic reconnection – radiation mechanisms: non-thermal – gamma-ray burst: general – pulsars: general – galaxies: jets}


\section{Introduction}

High-energy emission from pulsar wind nebulae (PWNe) and the relativistic jets of active galactic nuclei (AGNs) and gamma-ray bursts (GRBs) raises a question about the origin of the emitting particles. 
%It is believed that magnetic reconnection plays an important role in producing non-thermal particles and emissions \citep[][]{liu_02,lyutikov_03} in a rich set of high-energy astrophysical activities, like GRBs \citep[][]{spruit_01,spruit_02,giannios_08,mckinney_12}, AGNs \citep[][]{romanova_92,giannios_09,giannios_13,bottcher_19}, and pulsar wind  \citep[][]{coroniti_90,lyubarsky_03,kirk_sk_03,hoshino_lyubarsky_12,cerutti_17}. These activities usually happen in plasma environments near compact objects (\textit{e.g.} pulsars, magnetars, black holes, etc.), which are associated with extremely strong magnetic fields. 
Outflows from these compact objects are believed to be dominated by Poynting flux, i.e., the magnetic energy density is greater than the plasma rest-mass energy density. In GRB and AGN jets, magnetic field lines can reverse on small scales, as a  result of the nonlinear stages of magnetohydrodynamic (MHD) instabilities \citep[][]{romanova_92,begelman_98,spruit_01,lyutikov_03,giannios_spruit_06,bottcher_19}. Alternatively, the jet can carry current sheets from its base, like in pulsar winds \citep[][]{lyubarsky_kirk_01,drenkhahn_02a,drenkhahn_02b,kirk_sk_03,giannios_uzdensky_19,cerutti_20}. In both cases, field reversals on small scales are prone to magnetic reconnection, driving heating and particle acceleration.

%While several analytical models have been proposed to study reconnection rate and particle acceleration \citep[see][]{lyubarsky_05,guo_20}, 
Magnetic reconnection, and in particular the ``relativistic'' regime where the magnetic energy dominates over the plasma rest mass energy, is now established as an efficient mechanism of particle acceleration. Three-dimensional particle-in-cell (PIC) simulations, which offer a self-consistent description of plasma kinetics, have shown that relativistic reconnection naturally produces power-law spectra of accelerated particles \citep{zenitani_08,kagan_13,guo_14,ss_14,werner_17,guo_20_b}.
%Recent studies using particle-in-cell (PIC) techniques demonstrate that magnetic reconnection can be a fast and efficient dissipative mechanism. PIC simulations offer a self-consistent and complete description of plasma kinetics. The acceleration of particles in the relativistic reconnection of pair plasmas has been investigated in a number of PIC studies, both in two dimensions \citep[2D,][]{daughton_07,guo_15a,sironi_16,kilian_20} and three dimensions \citep[3D,][]{zenitani_08,kagan_13,guo_14,ss_14,dahlin_2017,werner_17,li_2019,guo_20_b}. 
The origin of the power-law particle spectrum in two-dimensional relativistic reconnection has been recently investigated by, e.g., \citet{guo_14, uzdensky_20}.
Yet basic questions, such as how particles are accelerated to high energies, the time scale of  acceleration, and whether these processes proceed up to larger (fluid) scales, remain debated. The answer to these questions is critical when evaluating the potential of relativistic reconnection for explaining high-energy astrophysical phenomena  in relativistic outflows ({e.g.,} the emission of very-high-energy photons  or the acceleration of ultra-high-energy cosmic rays, UHECRs). For instance, \citet{giannios_10} proposed that protons escaping the reconnection layer can undergo first-order Fermi acceleration due to repeated deflections by the converging reconnection upstream flows, and can reach energies up to $E\sim10^{20}$~eV in GRB and powerful AGN jets. 

%Previous large-scale 2D PIC simulations by \citet{sironi_16} studied the properties of plasmoids generated as a self-consistent by-product of the reconnection process. They found that the plasmoids are in rough energy equipartition between particles and magnetic fields. However, due to 2D topology, particles can be artificially confined in large magnetic islands, suppressing the acceleration of high-energy particles \citep[][]{dahlin_2017,li_2019}. 

In this context, it is critical to determine from first principles the acceleration rate of the highest energy particles. PIC simulations of relativistic reconnection showed that the reconnection layer fragments into a chain of plasmoids / flux ropes \citep[e.g.,][]{sironi_16}. Recent large-scale 2D PIC simulations by \citet{petropoulou_18} and \citet{hakobyan_20} suggested that the particles populating the high-energy spectral cutoff reside in a strongly magnetized ring around the plasmoid core. Their acceleration is driven by the increase in the local field strength, coupled with the conservation of the first adiabatic invariant. They also found that the high-energy spectral cutoff grows in time as $\propto \sqrt{t}$, which appears too slow to explain, e.g., UHECR acceleration. 

These conclusions may change in a 3D geometry, which would account for the finite length of plasmoids along the $z$ direction of the electric current.
%, comparable to their cross-sectional radius. 
In 3D, the $z$-invariance postulated by 2D simulations can be broken by the oblique tearing instability \citep[e.g.,][]{daughton_11} and the drift kink instability \citep[e.g.,][]{zenitani_07}, which may modify the 2D picture of particle energization. While in 2D particles are efficiently trapped within plasmoids, 3D simulations of non-relativistic reconnection  \citep[][]{dahlin_2017,li_2019} have shown that 
 self-generated turbulence and chaotic magnetic fields allow high-energy particles to access multiple acceleration sites within the reconnected plasma, resulting in faster acceleration rates than in 2D.

%These may also change the reconnection rate and the energy spectrum of non-thermal particles.
%Recent advances in computational plasma physics allow the exploration of relativistic magnetic reconnection in large-scale 3D PIC simulation. It is most likely that 3D simulation will generate turbulence and instabilities, such as oblique tearing instability \citep[][]{daughton_11} and drift kink instability \citep[][]{zenitani_07, guo_15a}, which will modify the 2D picture of particle energization.  \citet{li_2019} showed that self-generated turbulence and chaotic magnetic field lines allow the high-energy particles to access several main acceleration region \hz{(\textit{e.g.} reconnection exhausts region)} and results in a sustained and nearly constant-rate acceleration. \ls{[need to emphasize that these acceleration regions were all in the post-reconnection plasma (am I right?)]}

%In this work, we perform three-dimensional (3D) PIC simulations of relativistic reconnection in electron-positron highly magnetized plasma to examine the 3D effect on particle acceleration. We set the magnetization parameter $\sigma$ to be 10. We investigate the reconnection in a weak guide field $B_g=0.1B_0$ such that instabilities can develop in the current sheet. The outflow boundary condition and the inflow injection in unprecedentedly large computational domains allow us to study the statistical steady state beyond the initial transient phase. We compare the reconnection rate between 2D and 3D with a similar configuration. We present both energy and momentum spectra of positrons when the reconnection becomes steady.  We identify that particles can escape and travel far away from the reconnection region. We track particles and investigate the acceleration mechanism by studying the relationship between particle displacement and energy gain. We also perform the simulation in different sizes of boxes in order to show how the size of boxes affect particle acceleration and efficiency of energy transfer. We estimate the largest energy particles can be accelerated by magnetic reconnection in several astrophysical objects.

In this work, we perform 3D PIC simulations of relativistic reconnection in a magnetically-dominated electron-positron plasma, with magnetization (i.e., the ratio of magnetic energy density to plasma rest mass energy density) $\sigma=10$. Our inflow/outflow boundary conditions allow to reliably study the statistical steady state of the system, beyond the initial transient. We identify and characterize a novel acceleration mechanism, unique to 3D. 
We find that a fraction of particles with $\gamma\gtrsim 3\sigma$ can escape from plasmoids by moving along $z$ and experience the large-scale fields in the ``upstream'' region.\footnote{We point out that the mechanism discussed by \citet{li_2019} in non-relativistic reconnection relied on particles moving between multiple acceleration sites in the reconnection ``downstream'', i.e., in the post-reconnection plasma.} 
The momentum of these ``free'' particles is preferentially oriented along $z$. They undergo Speiser-like deflections by the converging upstream flows (as envisioned by \citet{giannios_10}; see also \citet{degouveia_05}), and are accelerated linearly in time --- their Lorentz factor scales as $\gamma\propto t$. The energy gain rate approaches $\sim eE_{\rm rec}c$, where $E_{\rm rec}\simeq 0.1 B_0$ is the reconnection electric field and $B_0$ the upstream magnetic field. The spectrum of free particles is hard and can be modeled as a power law $dN_{\rm free}/d\gamma\propto \gamma^{-1.5}$ --- whose slope we justify analytically ---
extending up to a cutoff energy that scales linearly with box size. We find that the free particles account for $\sim 20\%$ of the dissipated magnetic energy, independently of domain size, yet their number (as compared to the particle count in the downstream plasma) decreases with increasing box size.

%electron-positron highly magnetized plasma to examine the 3D effect on particle acceleration. We employ an outflow boundary condition and inflow injection in unprecedentedly large computational domains, which allows us to study the statistical steady state beyond the initial transient phase. We compare the particle acceleration between 2D and 3D and identify a new acceleration mechanism - particles in 3D simulation can escape from plasmoids and travel far away from the reconnection region. Those particles follow Speiser orbits. Compared to particles gaining energy from the increase of local magnetic fields in plasmoids,  the acceleration outside by the large-scale electric field is capable of producing particles of higher energy. These particles are accelerated at a fast rate $\dot{\gamma}\propto t$ and their momentum is dominated by its component along the electric field in the upstream. We also perform simulations with different box sizes. For the larger boxes, the relative number of particles in Speiser orbits (compared to the total reconnected particles) drops. However, the energy they carry remains a constant, around $20\%$ of total converted energy. 

The layout of this paper is as follows. In Section~\ref{2}, we describe the simulation setup  we employ. In Section~\ref{3}, we present our main results, as regard to the particle energy and momentum spectrum, the characterization of particle orbits inside and outside the reconnection layer, and the dependence on the size of the computational domain. 
%the results on the positron energy and momentum spectra, particle orbits inside and outside of plasmoids, and how the box size affects the particle distribution. We show that particles can escape plasmoids and follow Speiser orbits where they are further accelerated by the upstream electric field. 
In Section~\ref{4}, we draw our conclusions and discuss implications for astrophysical systems.
We argue that relativistic reconnection in GRB and AGN jets may be a promising mechanism for generating UHECRs.


%%%%%%%%%%%%%%%%%
\section{Simulation Setup}\label{2}
We employ 3D PIC simulations performed with the TRISTAN-MP code \citep{buneman_93, spitkovsky_05}. The magnetic field is initialized in Harris sheet configuration, with the field along $x$ reversing at $y=0$. We parameterize the field strength $B_0$ by the magnetization $\sigma = B_0^2 / 4\pi  n_0 m c^2 = \left(\omega_{\rm c} / \omega_{\rm p}\right)^2$, where $\omega_{\rm c} = e B_0 / m c$  and $\omega_{\rm p} = \sqrt{4\pi n_0 e^2 / m}$ are respectively the Larmor frequency and the plasma frequency for the cold electron-positron plasma outside the layer, with density $n_0$. The Alfv\'{e}n speed is related to the magnetization as $v_A / c = \sqrt{\sigma/\left(\sigma + 1\right)}$; we take $\sigma = 10$. In addition to the reversing field, we initialize a uniform guide field along $z$ with strength $B_g = 0.1 \,B_0$. We have also explored a case with zero guide field and found similar results (see Tab.~\ref{tab:box}). %but do not see appreciable difference {\bf DG: this sentence clause "but do not see..." is too general and confusing. Let's leave it out from here and discuss more in detail the zero guide field properties later in the text.}. 
We resolve the plasma skin depth $c/\omp$ with $2.5$ cells, and initialize an average of one particle in each cell. We have also tested a larger value of four particles per cell, finding no significant change in  reconnection rate, maximum energy, and particle spectra (for more details, see Tab.~\ref{tab:box}). %Therefore, the number of particles in each cell for the initial state does not change our findings. 
The numerical speed of light is 0.45 cells/timestep. We employ periodic boundary conditions in $z$, outflow boundary conditions in $x$, while along $y$ two injectors continuously introduce fresh plasma and magnetic flux into the domain \citep[for details see][]{sironi_16,sironi_beloborodov_20}. As opposed to the commonly-adopted triple-periodic boundaries, our setup allows to evolve the system to arbitrarily long times, so we can study the statistical steady state for several Alfv\'enic crossing times. 

We trigger reconnection near the center of the simulation domain (i.e., near $x=y=0$, but along the whole $z$ extent), by removing the pressure of the hot particles initialized in the current sheet, as in \citet{sironi_16}. The characteristic $x$-length of this region is defined as $\Delta_{\rm init}$. For our largest 3D simulation (see below), we choose $\Delta_{\rm init} = 500 \comp$. For smaller boxes, we have tested different values of $\Delta_{\rm init}$, finding no difference in our main results (see Tab.~\ref{tab:box} for details). 

For our reference 3D simulation, the box length in  $x$ and $z$ (respectively, $L_x$ and $L_z$) is $\simeq4000\,{\rm cells}\sim 1600\,\comp$, while the box extent along $y$ increases over time as the two injectors recede from the current sheet. %Here, we define $L_x$ and $L_z$ as the box length in the $x$- and $z$-direction. 
We also present results from a set of boxes with fixed $L_x$ but various $L_z$ from $1600\,\comp$ down to $12\,\comp$, and two sets of experiments with a fixed ratio $L_x/L_z$ ($L_x/L_z = 1$ and $L_x/L_z = 2$), but different box sizes. In the following, unless otherwise indicated, we employ our reference box with $L_x=1560\,\comp$ and $L_z= 1613\,\comp$, and we define $L=1560\,\comp$ as our unit of length.

We have also performed a 2D simulation with identical physical and numerical parameters as our reference 3D run (aside from a choice of 16 particles per cell to increase particle statistics), to emphasize 3D effects.
%\ls{is this accurate? I can’t remember…} \hz{Yes, the only difference is ppc. I also use the result from dvstripe=500 for 2D (same as 3D).}

\section{Results}\label{3}
\begin{figure}
    \includegraphics[width=\columnwidth]{dens3d_008_3.png}
    \includegraphics[width=\columnwidth]{dens3d_036_3.png}
    \caption{Two snapshots of density from our reference 3D simulation. We show the density structure at a relatively early time (top, $t = 0.47\, L/c$), when reconnection fronts are moving outwards, and at a later time (bottom, $t=2.13\, L/c$), when the system has achieved a steady state. The upstream plasma flows into the layer along $y$, while reconnection outflows move along $x$. The electric current is along the  $z$ direction, which is invariant in 2D simulations.
    %showing the formation and development of plasmoids after the reconnection becomes steady. 
    %Two injectors continuously introduce fresh plasma and magnetic flux from $\pm y$ direction into the domain. The outflows moves relativistically towards $\pm x$ direction. A uniform guide field along $+z$ with strength $B_g = 0.1B_0$ is initialized.
    }
    \label{fig:dens_3d}
\end{figure}

Fig.\ref{fig:dens_3d} shows two snapshots of the 3D density structure from our reference simulation \footnote{A Movie showing the evolution of the density structure can be found at \url{https://youtu.be/fMictkK1QNU}.}. The top panel refers to $ct/L\simeq 0.47$, and shows the two reconnection fronts (see the two overdense regions at $|x|\sim L/4$) propagating away from the center, at near the Alfv\'{e}n speed. The bottom panel of Fig.\ref{fig:dens_3d} refers to a representative time ($ct/L\simeq 2.13$) when the layer has achieved a statistical steady state. The layer is fragmented into flux ropes of various sizes, with comparable lengths in the $z$ direction as in the $x-y$ plane. The finite extent of plasmoids along the $z$ direction, likely due to the relativistic drift-kink instability  \citep{zenitani_07, zenitani_08}, plays a fundamental role for the physics of high-energy particle acceleration, as we describe below.
%, one just after the reconnection is triggered, the other is when the reconnection has become quasi steady. Two reconnection fronts (see the overdense regions at $\sim \pm L/4$ in the top panel of Fig.\ref{fig:dens_3d},) are propagating away from the center at the Alfv\'{e}n speed. In 3D, the rapid growth of the relativistic drift-kink instability can disrupt the current layer along $z$-direction . \ls{[is zenitani 2007 or 2008?]} \hz{[Both? 2007 discussed RDKI, 2008 discussed the effect of guide fields on RDKI and when RDKI dominates in 3D current sheet.]}

\begin{figure}
    \includegraphics[width=\columnwidth]{vin_t_3.pdf}
    \caption{A comparison of the reconnection rate between 3D (red) and 2D (blue) simulations. The reconnection rate is calculated by averaging the plasma inflow velocity (in units of the speed of light) in the region $0.03L<y<0.08L$.}
    \label{fig:vin}
\end{figure}

3D instabilities can also change the reconnection rate, as compared to  2D. Fig.\ref{fig:vin} illustrates the temporal evolution of the reconnection rate $\eta_{\rm rec}\equiv v_{\rm in}/v_{\rm A}$ for both  2D (blue) and 3D (red) simulations, where $v_{\rm in}$ is the inflow speed and $v_A \simeq c$ for magnetically-dominated plasmas. The initial growth of the box-averaged reconnection rate before $ct/L\sim 0.8$ is just due to the increase of the region where reconnection is active (i.e., between the two reconnection fronts). 
%As the reconnection is triggered, magnetized plasma starts to flow towards the mid-plane ($y=0$) and the rate increases with time. 
When the two reconnection fronts exit the computational domain, the rate becomes quasi-steady. The reconnection rate in 3D, $\eta_{\rm rec}\sim0.075$, is slower than in 2D, $\eta_{\rm rec}\sim0.12$. 
%Compared to similar configurations of the 2D simulation that has a steady inflow speed (or equivalently, the reconnection rate) of $0.12c$, the 3D reconnection is slower with around $\eta_{\rm rec}\sim0.075c$. 
In either case, the rate is in reasonable agreement with analytical expectations \citep{lyubarsky_05}.
%- $v_{\rm in} \sim 0.1c$. %\citep{liu_17,zenitani_08}.

The inflowing particles from the two sides of the layer mix in the reconnection region, which we shall also call ``reconnected plasma'' or ``downstream'' region. In contrast, the pre-reconnection flow shall be called ``upstream''.
To identify the region of reconnected plasma, we define a ``mixing’’ factor $\mathcal{M}$:
\begin{equation}
\mathcal{M} \equiv 1-2\left |\frac{n_{\rm top}}{n} - \frac{1}{2} \right |,
\end{equation}
where $n_{\rm top}$ is the density of particles that started from $y>0$, while $n$ is the total density. %\ls{maybe lets use $n_{\rm top}$, otherwise it may be confused with positron density.} 
It follows that $\mathcal{M} = 1$ represents the  downstream plasma, where particles from the two sides of the layer are well mixed, whereas $\mathcal{M} = 0$ characterizes the upstream, where no mixing has occurred. We will use the mixing factor $\mathcal{M}$  to identify whether a particle is located in the upstream or downstream region.

\begin{figure*}
    \centering
    \subfloat{\stackunder{\includegraphics[width=0.48\linewidth]{prtl_hist_3d_v4.png}}{}}
    \qquad
    \subfloat{\stackunder{\includegraphics[width=0.48\linewidth]{prtl_hist_2d_v4.png}}{}}
    \caption{2D histograms of the particle Lorentz factor $\gamma$ and the mixing factor $\mathcal{M}$ (interpolated to the nearest cell) at time $t=2.37L/c$, for 3D (left) and 2D (right). The red dashed line in the left panel marks the threshold $\mathcal{M}_0=0.3$ that we employ to distinguish upstream ($\mathcal{M}<\mathcal{M}_0$) from downstream ($\mathcal{M}>\mathcal{M}_0$). }
    \label{fig:prtl_hist}
\end{figure*}

%Using $\mathcal{M}$ as a criterion for identifying upstream and downstream plasma, we study where particles of different energies are located. Fig.~\ref{fig:prtl_hist} shows histograms of the particles Lorentz factor $\gamma$ and mix factor $\mathcal{M}$ at time $t=2.37L/c$. For 3D, the population of high energy particles (\textit{i.e.} $\gamma\gtrsim 20$) can be easily divided into two groups - the group with $\mathcal{M} < \mathcal{M}_0$ represents particles at the outside of reconnection regions and the $\mathcal{M} > \mathcal{M}_0$ one represents particles residing inside. In the following discussion, we set $\mathcal{M}_0 = 0.30$ to distinguish the upstream and downstream.

%Both histograms suggest that most of particles with $\gamma\lesssim30$ are located in the reconnection region. However, an appreciable difference between the two cases is that for 3D, there are a large amount of ultra-relativistic particles traveling into the low-$\mathcal{M}$ region, indicating that they are capable of escaping from the reconnection layer. In 2D, as other previous studies have pointed out \citep{sironi_16}, large plasmoids or magnetic islands can trap most particles and particles cannot move into upstream. It is intriguing that this difference may result in new features for 3D reconnections. In contrast, due to one more freedom of motion along $z$-direction, particles in 3D may flee from the strong magnetic confinement in magnetic islands and move into upstream where the local $\mathcal{M}$ is close to 0. Hence, they can be further accelerated to higher energy compared to those in 2D simulations.

Using $\mathcal{M}$ as a criterion for separating upstream and downstream regions, we study where particles of different energies are located. Fig.~\ref{fig:prtl_hist} shows histograms of the particle Lorentz factor $\gamma$ (horizontal axis) and mixing factor $\mathcal{M}$ (vertical axis) at time $t=2.37L/c$, for 3D (left) and 2D (right) simulations. Both histograms suggest that most of the low-energy particles ($\gamma\lesssim30$) are located in the downstream region ({i.e.,} $\mathcal{M}$ near unity). In 2D, all of the high-energy particles are also located in well-mixed regions, i.e., in the downstream. In agreement with earlier studies, high-energy particles in 2D are trapped within plasmoids \citep[][]{sironi_16,petropoulou_18,hakobyan_20}.
In contrast, a significant %number
fraction of high-energy particles ($\gamma\gtrsim30$) in the 3D simulation lie in low-mixing regions, i.e., in the upstream. As we show below, these are particles that have escaped from reconnection plasmoids, and are now being rapidly accelerated by the large-scale upstream fields. In the following, we will take a threshold of $\mathcal{M}_0 = 0.3$ (horizontal red dotted line in the left panel) to separate the downstream region ($\mathcal{M} > \mathcal{M}_0$) from the upstream region ($\mathcal{M} < \mathcal{M}_0$). We expect that our results will not change significantly as long as $\mathcal{M}_0$ is near $0.3$ (e.g., between 0.25 and 0.35).

 %However, an appreciable difference between the two cases is that for 3D, particles with $\gamma\gtrsim30$ can be divided into two groups - the group with $\mathcal{M} < \mathcal{M}_0 = 0.30$ represents particles that have escaped from the reconnection layer and travel in the upstream, while the $\mathcal{M} > \mathcal{M}_0$ one represents particles residing in the downstream. 
 %For 2D, due to lack of freedom of motion along $z-$direction, particles are trapped by loops of magnetic fields in large plasmoids or magnetic islands and cannot move into upstream  As a result, in the right panel of Fig.~\ref{fig:prtl_hist}, particles with $\gamma\gtrsim30$ concentrate around $\mathcal{M}\sim 1$. This difference between 3D and 2D may result in new features for 3D reconnection.

In the following of this section, we first study the particle energy and momentum spectra in the 3D simulation and identify that %the motion along $z$-axis plays a key role in non-thermal particle acceleration
high energy particles preferentially move along the $z$-direction (Section~\ref{3.1}). Then, we track particles and investigate in detail their acceleration mechanism (Section~\ref{3.2}). Finally, we investigate the dependence of our results on the domain size, in order to show that the acceleration physics should operate effectively out to larger scales (Section~\ref{3.3}).

%%%%%%%%%%%%%%%%%%%
\subsection{Particle Spectra}\label{3.1}

\begin{figure}
    \includegraphics[width=\columnwidth]{spectra_v6.pdf}
    \caption{Momentum spectrum $p_z dN/dp_z$ of positrons, where $p_z=\gamma \beta_z$ is the dimensionless 4-velocity along the $z$ direction.  We show spectra of positrons with $p_z>0$ (blue, indicated as $p_{z+}$ in the legend) and $p_z<0$ (green, indicated as $p_{z-}$ in the legend). Spectra from the overall box are shown as solid lines (indicated with subscript ``box'' in the legend), whereas the dashed lines refer only to positrons belonging to the downstream region, as defined by the mixing condition $\mathcal{M}>\mathcal{M}_0$ (indicated with subscript ``rr'' in the legend). The spectrum of high-energy ``free'' positrons residing in the upstream region (with $\mathcal{M}<\mathcal{M}_0$), which preferentially have $p_z>0$, is indicated by the dotted blue line. The dotted black line shows a power-law $p_z^{-1}$. In the inset, we present the box-integrated positron spectra of kinetic energy (grey) and momenta in different directions, as indicated in the legend. All spectra in the main plot and in the inset are time-averaged between $t = 3.34L/c$ and $3.56L/c$ and normalized to the total number of positrons in the box.
%   of the momentum along $+z$ (blue lines) and $-z$ direction (green lines) . All spectra have been normalized to the total number of particles. Spectra in the whole simulation box are marked with solid lines; spectra only in the reconnection region are shown by dashed lines. Particles outside of the reconnection region with $p_z>0$ are shown by a blue dotted line.  In the inset plot, we show both the energy spectrum and momentum spectrum for high energy particles. The gray line shows the energy spectrum.
    }
    \label{fig:spectra}
\end{figure}

A non-thermal power-law spectrum extending to high energies is a well-established outcome of relativistic reconnection \citep[e.g.,][]{ss_14}. Fig.~\ref{fig:spectra} shows the positron momentum spectrum $p_z dN/dp_z$, where $p_z=\gamma \beta_z$ is the dimensionless 4-velocity along $z$ ($\beta_z$ is the particle $z$-velocity in units of the speed of light). The spectrum is obtained by averaging between $t = 3.34L/c$ and $3.56L/c$, when the system is in steady state. The box-integrated spectrum of positrons with $p_z>0$ (blue, indicated as $p_{z+,\rm box}$ in the legend) can be modeled for $p_z\gtrsim 3$  as a power law $p_z {\rm d}N/{\rm d}p_z \propto p_z ^{-1}$.

The figure compares the momentum spectrum between positrons with $p_z>0$ (blue lines, indicated as $p_{z+}$ in the legend) and $p_z<0$ (green lines, indicated as $p_{z-}$ in the legend), and further distinguishes between spectra integrated in the whole box (solid lines) and only extracted from the reconnection downstream ($\mathcal{M}>\mathcal{M}_0$, dashed lines). We find that high-energy positrons with $p_z<0$ are mostly located within the downstream region (compare green solid and dashed lines), i.e., non-thermal positrons with $p_z<0$ are trapped in plasmoids, in analogy to 2D results \citep[see][]{petropoulou_18, hakobyan_20}. 

In contrast, a significant fraction of high-energy positrons with $p_z > 0$ reside outside the reconnection region (compare blue solid and dashed lines), and we shall call them ``free''. The fraction of free positrons is an increasing function of momentum, and for $p_z\gtrsim 100$ they are more numerous than the ones located in the reconnection downstream. The $p_{z+}$ spectrum of free positrons (dotted blue line) can be modeled as a hard power law, ${\rm d}N_{\rm free}/{\rm d}p_z \propto p_z^{-1.5}$. In Appendix~\ref{slope}, we provide an analytical justification of the measured spectral slope. The cutoff in the spectrum of $p_z>0$ positrons is much higher than for $p_z<0$ positrons, suggesting that free positrons can be accelerated to much larger energies than trapped ones, as we indeed demonstrate below.\footnote{The electron spectrum shows the opposite asymmetry: electrons with $p_z>0$ mostly reside in plasmoids, and their spectrum extends to lower momenta than for free electrons with $p_z<0$.}

The asymmetry between positrons with $p_z>0$ vs  $p_z<0$ is a unique feature of our 3D setup. In a corresponding 2D simulation (see Appendix~\ref{appd}), $p_{z+}$ and $p_{z-}$ spectra are nearly identical, and nearly all  high-energy particles reside within the reconnection downstream, as already shown by Fig.~\ref{fig:prtl_hist} (right panel).

%can escape and travel to upstream. As $p_{z}$ increases, so does the difference between those two regions (blue dotted line). When $p_z \gtrsim 100$, the difference of $p_{z+}$ spectra for particles in the whole box and the reconnection region is more than a factor of $2$. This indicates that there are more particles with $p_{z} \gtrsim 100$ in the upstream than in the reconnection layer.
% The spectrum of high-energy ``free'' positrons residing in the upstream region (with $\mathcal{M}<\mathcal{M}_0$), which preferentially have $p_z>0$, is indicated by the dotted blue line. 
 In the inset of Fig.~\ref{fig:spectra}, we present the box-integrated positron spectra of kinetic energy (grey) and momentum in different directions, as indicated in the legend. In contrast to the $p_z$ spectrum, there is no broken symmetry between positive and negative directions in the $p_x$ and $p_y$ spectra.
 The inset shows that the peak of the energy spectrum (grey), at $\gamma-1\sim 3$, is dominated by motions along the $x$ direction of the reconnection outflows (compare with the $p_x$ spectrum, red line). In contrast, the high-energy cutoff of the positron energy spectrum at $\gamma\sim 500$ is dominated by the  $p_{z+}$ spectrum (blue). So, the most energetic positrons move mostly along the $+z$ direction (conversely, the highest energy electrons along $-z$). We also remark that the $p_y$ spectrum (orange) reaches rather high momenta (albeit, not as high as the $p_{z+}$ spectrum). This is consistent with the trajectories of high-energy positrons that we illustrate in Sec.~\ref{3.2}.
  %which are deflected back and forth in the reconnection generated bulk flows.
 
 %$p_x$ (red), which is the bulk motion of the relativistic outflow. For particles with $\gamma>500$, the energy spectrum is almost identical to the $p_{z+}$ spectrum (blue). So, the energy of the most energetic particles is dominated by the motion in $z$-direction.

%In contrast to $p_z$, there is no broken symmetry along positive and negative directions in the $p_x$ and $p_y$ spectra and we are only showing $p_x>0$ and $p_y>0$ parts. 

%the time averaged spectra of the particle (positrons only) momentum along z-direction within the time interval $3.34L/c$ and $3.56L/c$ when the reconnection is steady. The spectra for particles with $p_z < 0$ ($p_{z-}$) and $p_z > 0$ ($p_{z+}$) are given separately. Here, $p_z$ is the four-velocity of a particle in the $z$-direction: $p_z \equiv\gamma\beta_z$ and $\beta_z$ is the particle velocity along $z$-direction in unit of $c$. Spectra for particles in reconnection regions and in the whole box are also given in solid and dashed lines correspondingly. 
%We stress that all spectra shown in Fig.~\ref{fig:spectra} are for positrons. For electrons, the $p_{z-}$ and $p_{z+}$ spectra are reversed.
%{\bf DG: very basic question: do we show the spectra of both electrons and positrons? If, yes, why are there more particles moving in the +z direction? (isn't the preferred motion of the electrons the opposite of that of the positrons for particles leaked in the upstream)? \hz{It's only positrons here.}}

%For non-thermal particles (i.e. $|p_z| \gtrsim 3$), a power-law $p_z {\rm d}N/{\rm d}p_z \propto p_z ^{-s}$ with $s\sim 1.0$ develops %for both $p_{z+}$ and $p_{z-}$
%primarily in $p_{z+}$ spectra. On one hand, $p_{z-}$ spectra\footnote{We emphasis again that all spectra shown in Fig.~\ref{fig:spectra} are for positrons. For electrons, the $p_{z-}$ and $p_{z+}$ spectra will be reversed.} in the reconnection region and in the whole simulation box are almost coincident, indicating that most non-thermal particles with $p_z < 0$ stay in the reconnection region. On the other hand, $p_{z+}$ spectra, which exhibits a large difference between those two regions, demonstrate that particles with $p_z > 0$ escape and travel to upstream. As $p_{z}$ increases, so does the difference. When $p_z \gtrsim 100$, the difference of $p_{z+}$ spectra for particles in the whole box and the reconnection region is more than a factor of $2$. This indicates that there are more particles with $p_{z} \gtrsim 100$ in the upstream than in the reconnection layer.

%Comparing with the $p_{z-}$ spectra with a cutoff at around $2\times10^2$, we find that the $p_{z+}$ ones have a larger cutoff ($\sim 10^3$). Here is the explanation: 
%particles trapped in plasmoids have a more symmetric distribution in the momentum space, because the field lines are chaotic inside \citep{petropoulou_18, hakobyan_20}. 
%For these particles, the largest energy they can get can be derived from the equipartition of magnetic and kinetic energy \citep{sironi_15}: $\Delta\gamma \sim U_{B} / n mc^2 \approx B^2/(4\pi n mc^2) \approx 100$ \hz{check!}, which is consistent with the cutoff of the $p_{z-}$ spectra.
%for $|p_z|$ less than the cutoff of the $p_{z-}$ spectrum (\textit{i.e.} $|p_z| \lesssim 2\times10^2$), the main contribution to both $p_{z+}$ and $p_{z-}$ spectra comes from particles trapped in plasmoids (especially in a strongly magnetized ring around plasmoid cores \citep[see][]{petropoulou_18, hakobyan_20}).
%When particles escape from this region, the large scale field lines (i.e. electric field lines along the $+z$ direction) will dominate the acceleration. This will be further examined by particle trajectories in Sec.\ref{3.2}. As a result, these positrons gain momentum from the electric field along $+z$ direction instead of $-z$, while the electrons gain momentum along $-z$. Therefore, particles which are not trapped can be further accelerated and have a larger cutoff than those staying in plasmoids. On the contrary, particles in 2D are confined in plasmoids. The cutoffs of both $p_{z+}$ and $p_{z-}$ spectra are nearly identical (see Appendix. ~A).

%For non-thermal particles (i.e. $|p_z| \gtrsim 3$), a power-law $p_z {\rm d}N/{\rm d}p_z \propto p_z ^{-s}$ with $s\sim 1.0$ develops primarily in $p_{z+}$ spectra. On one hand, $p_{z-}$ spectra in the reconnection region and in the whole simulation box are almost coincident, indicating that most non-thermal particles with $p_z < 0$ are trapped in plasmoids (especially in a strongly magnetized ring around plasmoid cores \citep[see][]{petropoulou_18, hakobyan_20}). 

%The $p_{z+}$ spectra also have a larger cutoff ($\sim 10^3$) than $p_{z-}$ ones ($2\times10^2$). Positrons traveling to upstream keep gaining momentum from the electric field along $+z$ direction, while the electrons gain momentum along $-z$. Therefore, they can be further accelerated and have a larger cutoff than those trapped in plasmoids. 

%In the inset plot of Fig.~\ref{fig:spectra} are the average energy spectrum and momentum spectra along the $x$-, $y$-, and $z$- axes. In contrast to $p_z$, there is no broken symmetry along positive and negative directions in the $p_x$ and $p_y$ spectra and we are %not showing this here.
%only showing $p_x>0$ and $p_y>0$ parts.

%The energy spectrum (grey) near its peak ($\gamma-1\sim 3$) is dominated by $p_x$ (red), which is the bulk motion of the relativistic outflow. For particles with $\gamma>500$, the energy spectrum is almost identical to the $p_{z+}$ spectrum (blue). So, the energy of the most energetic particles are dominated by the motion in $z$-direction. It is interesting that $p_y$ spectrum also extends to large valus. This is consistent with  particle trajectories (see Sec.~\ref{3.2}) which are deflected back and forth in the reconnection generated bulk flows.

%In the inset plot of Fig.~\ref{fig:spectra} are the average energy spectrum and momentum spectra along the $x$-, $y$-, and $z$-axes. 

%We emphasis again that all spectra shown in Fig.~\ref{fig:spectra} are for positrons. As for electrons, we expect that most of them will moving along $-z$ direction after escaping from reconnection region. Thus, $p_{z+}$ and $p_{z-}$ spectra for electrons are opposite to positrons, while $p_x$ and $p_y$ spectra remain similar to positrons.  

\begin{figure*}
    \centering
    \subfloat{\stackunder{\includegraphics[width=0.48\linewidth]{dgamma_dz_3d_v4.png}}{}}
    \qquad
    \subfloat{\stackunder{\includegraphics[width=0.48\linewidth]{dgamma_dz_2d_v4.png}}{}}
    \caption{2D histograms of the gain in positron Lorentz factor ($\Delta \gamma$) and displacement along the $z$-axis ($\Delta z$), in 3D (left plot) and 2D (right plot). The positrons are selected at the end of the simulations ($ct/L=3.48$) and traced back to the first time they are saved. For the 2D case, the particle displacement along the $z$-axis is calculated by time integration of the $z$ velocity. 
    The relation expected from  Eq.~\ref{eq:e_est} is marked with a dashed white line in the left panel. The red and yellow lines in the left panel represent the tracks in the $\Delta \gamma-\Delta z$ plane of the two high-energy positrons shown in Fig.~\ref{fig:prtl_traj}; in this case, the differences $\Delta \gamma$ and $\Delta z$ are computed at each time with respect to the initial time when the particle Lorentz factor first exceeded $\gamma=3$.
    %\ls{this plot is for all particles at all times, right?} \hz{No, all particles at snapshot 294. For each particle, $\Delta \gamma$ and $\Delta z$ are the difference between time max(t0, 181) (t0: after this time the particle is always saved) and time 294.}
    }
    \label{fig:dgamma_dz}
\end{figure*}

In summary, the momentum spectra in \figg{spectra} show that
most of the highest energy positrons are located in the reconnection upstream, and their momentum is dominated by the $z$ component, which is aligned with the large-scale motional electric field $\vec{E}_{\rm rec}=E_{\rm rec}\hat{z}=\eta_{\rm rec}B_0 \hat{z}$ carried by the upstream converging flows. If $\vec{E}_{\rm rec}$ is the primary agent of acceleration, we expect a linear relation between the gain in Lorentz factor ($\Delta \gamma$) and the displacement along the $z$-axis ($\Delta z$), of the form
\begin{equation}\label{eq:e_est}
    \frac{\Delta\gamma}{\Delta z} \approx \frac{e E_{\rm rec}}{mc^2} = \frac{\eta_{\rm rec}\sqrt{\sigma}\omega_{\rm p}}{c}.% \approx 370 L^{-1} ,
\end{equation}
In Fig.~\ref{fig:dgamma_dz}, we show the relation between $\Delta \gamma$ and $\Delta z$ for a sample of $\sim 2\times 10^6$ positrons selected at the end of the simulations ($t = 3.48L/c$), for 3D (left) and 2D (right). Each particle is traced back to the first time its Lorentz factor exceeded $\gamma=3$, and its overall $\Delta \gamma$ and $\Delta z$ are computed.\footnote{In 3D, $\Delta z$ is directly recorded. In 2D, it is obtained by time integration of the $z$ velocity.} The plot only shows the quadrant with $\Delta \gamma>0$ and $\Delta z>0$, which includes most of positrons and displays the strongest difference between 2D and 3D.

For $\Delta \gamma \lesssim 100$, 2D and 3D results are similar. There appears a trend that particles gaining more energy also display a larger $z$ displacement, but the spread is quite large ($\Delta \gamma$ may vary by two orders of magnitude for the same $\Delta z$). The similarity between 2D and 3D for $\Delta \gamma \lesssim 100$ suggests that most of these particles are accelerated while trapped in plasmoids, as found in 2D simulations \citep[][]{petropoulou_18, hakobyan_20}.

The most striking difference between 2D and 3D results is in the behavior of particles experiencing large energy gains, $\Delta \gamma \gtrsim 100$. In this range ($\Delta \gamma \gtrsim 100$ and $\Delta z > 0.4\,L$), positrons from the 3D simulation follow a linear relation  $\Delta \gamma\propto \Delta z$, indicating that they are all accelerated by the same electric field. Such a branch is absent in the corresponding 2D simulation. For comparison, in the left panel of Fig.~\ref{fig:dgamma_dz} we plot with a dashed white line the expectation of \eq{e_est} for the measured  $\eta_{\rm rec} = 0.075$. The agreement of the high-energy branch in the 3D histogram with \eq{e_est} confirms that particles experiencing the largest energy gains are accelerated in the upstream by the motional electric field $E_{\rm rec}$.

We also point out the excess of positrons lying along the extrapolation of the dashed white line to low $\Delta \gamma$, in the left panel at $1\lesssim \Delta \gamma\lesssim 5$. These positrons are currently being injected into the acceleration process by the reconnection electric field, so they still obey \eq{e_est}.

%However, particles in 3D with large $z$-movement and energy gains  ($\Delta \gamma>100$ and $\Delta z > 0.8L$) form a narrow line: $ \Delta \gamma \propto \Delta z$, indicating that particles are being accelerated by a similar electric field $E_{\rm rec}=\eta_{\rm rec} B_0$, where $B_0$ is the magnetic field in the upstream. To further examine this, we estimate% the electric field in the upstream

%If we take $\eta_{\rm rec} = 0.08$ and $\sigma = 10$ from the simulation, we derive the white dashed line in Fig.~\ref{fig:dgamma_dz} (which corresponds to $\Delta\gamma/\Delta z \approx 370 L^{-1}$). The overlap of this line with the particle energy gains provides further evidence that the accelerating electric field is that in the reconnecton upstream instead of electric field in plasmoids. 


%a particle that gains more energy tends to have a larger displacement, although there is a large spread of $\Delta \gamma$ and it may vary two orders of magnitude for the same $\Delta z$. This similarity between 2D and 3D suggests that partiicles are undergoing a similar acceleration mechanism, most likely due to magnetic moment conservation in internal magnetic fields amplified by compressing plasmoids \citep[][]{petropoulou_18, hakobyan_20}.

%the motion along $z$-axis may play an important role in the non-thermal particle acceleration, we study the relation between the energy that particles obtain ($\Delta \gamma$) and their displacement along the $z$-axis ($\Delta z$), and compare the result between 3D and 2D cases. The results are shown in Fig.~\ref{fig:dgamma_dz}. 
%All particles existing at the end of the 2D simulation ($t = 3.48L/c$) are tracked; we randomly select the same amount of particles in 3D at the same time. Each particle is traced back to the first time it is saved. $\Delta \gamma$ and $\Delta z$ are calculated during this time interval. $\Delta z$ is calculated by integrating particle velocity along $z$- direction with respect to time in 2D, while it is recorded directly in 3D. Here, we only show the ones with both positive $\Delta \gamma$ and $\Delta z$, which includes the most of particles and has the strongest difference between 2D and 3D. %For $\Delta z< 0$ or $\Delta \gamma< 0$ cases, the results for 3D and 2D cases are similar - most particles fall into a rectangular region with $|z|<L$ and $|\gamma|<200$.

%Both 2D and 3D display a weak linear correlation for $\Delta z \lesssim 0.5L$, though $\Delta \gamma$ may vary two orders of magnitude for the same displacement in the z-direction. 
%In the 3D simulation, there is a line starting from $\Delta\gamma\sim 2$ (at the low end of the plotted $\Delta z$) and increasing linearly with $\Delta z$. Those are particles being just injected into the reconnection region. They are also accelerated by the large scale electric field.

%For $\Delta \gamma \lesssim 100$, a particle that gains more energy tends to have a larger displacement, although there is a large spread of $\Delta \gamma$ and it may vary two orders of magnitude for the same $\Delta z$. This similarity between 2D and 3D suggests that partiicles are undergoing a similar acceleration mechanism, most likely due to magnetic moment conservation in internal magnetic fields amplified by compressing plasmoids \citep[][]{petropoulou_18, hakobyan_20}.
%Particles gain energy from the reconnection electric field near X-points or the merging of plasmoids\citep{zenitani_01,ss_14,sironi_16}.
%internal magnetic field amplified by compressing plasmoids is driving particles acceleration as a result of magnetic moment conservation \citep[][]{petropoulou_18, hakobyan_20}.

%However, there is a pronounced tail in the 3D case. 

%Therefore, unlike 2D in which high energy particles are trapped in plasmoids \citep{dahlin_2017,li_2019}, particles in 3D reconnection can escape from plasmoids and be further accelerated to higher energy ($\gamma\gtrsim 200$). This is also supported by Fig.~\ref{fig:prtl_hist}.


%%%%%%%%%%%%%%%%%%%%%
\subsection{Particle Orbits}\label{3.2}

\begin{figure*}
    \centering
    \includegraphics[width=1\linewidth]{particle_3d_3.png}
    \caption{Trajectories of two representative positrons. For each particle, its trajectory in the $x-y$ plane is shown in the left panel, and in the $y-z$ plane in the middle panel. The color of the line represents the particle energy (from red to white as the energy increases). A white filled circle shows the position at a specific time: $t=2.28L/c$  for particle \textit{A}, corresponding to a time $\Delta t = 0.87L/c$ in the particle life; and $t=2.80L/c$ for particle \textit{B}, corresponding to $\Delta t = 1.03L/c$. The background color shows the plasma density at that same time, in the $x-y$ and $y-z$ slices where the particle is located. In the right panel, we show the particle Lorentz factor as a function of its lifetime $\Delta t$ since it first crossed a threshold $\gamma=3$. The maximum expected acceleration rate corresponding to \eq{ratemax} is shown with a red dashed line.
    %\ls{can the right panel go down to the lowest gamma we save?}
    }
    \label{fig:prtl_traj} 
\end{figure*}

To investigate the acceleration mechanism of the highest energy particles, we have studied the trajectory of a large number of high-energy ($\gamma>200$) positrons. We present in Fig.~\ref{fig:prtl_traj} two representative orbits.\footnote{Movies showing the orbits of positron \emph{A} and \emph{B} can be found online at \url{https://youtu.be/pjpYzw2VKe0} and \url{https://youtu.be/kOycphI0WUw}, respectively.} Their $ \Delta \gamma- \Delta z$ tracks are shown in the left panel of Fig.~\ref{fig:dgamma_dz} by the two colored lines, demonstrating that for $\Delta \gamma\gtrsim 10$ they follow a linear relation akin to \eq{e_est}.

Fig.~\ref{fig:prtl_traj} shows the particle orbits projected on the $x-y$ (left) and $y-z$ (middle) planes, as well as the particle Lorentz factor as a function of lifetime $\Delta t$ (right panel), measured since a particle first crosses the threshold $\gamma=3$. The acceleration rate due to the electric field $\vec{E}_{\rm rec}=E_{\rm rec}\hat{z}=\eta_{\rm rec}B_0 \hat{z}$ in the upstream flow can be written
\begin{equation}\label{eq:acc_rate}
    \dot{\gamma} =\frac{\Delta \gamma}{\Delta t} \approx \frac{ e E_{\rm rec}}{mc} \beta_z \approx \beta_z\eta_{\rm rec}\sqrt{\sigma} \omp,
\end{equation}
where $\beta_z$ is some time-averaged $z$ velocity in units of the speed of light. The highest acceleration rate will be achieved when $\beta_z\simeq 1$, leading to a maximal rate 
\begin{equation}\label{eq:ratemax}
\dot{\gamma}_{\rm max}=\eta_{\rm rec}\sqrt{\sigma}\omp,
\end{equation}
indicated by the dashed red line in Fig.~\ref{fig:prtl_traj} (right). We refer to this as $\dot{\gamma}_{\rm max}$, since it is the maximum acceleration rate that can be provided by the large scale electric field. Even stronger electric fields may transiently appear  within the reconnection region, which explains why some particles can temporarily experience an energization rate even larger than this value (e.g., positron \emph{B} between $\Delta t\simeq0.4$-$0.8L/c$).

We find that both positron \emph{A} and \emph{B} are injected into the acceleration process in the vicinity of an X-point in the midplane of the layer ($y=0$). Yet, at later times their histories diverge. 
Positron \emph{A} is energized at nearly the maximal rate $\dot{\gamma}_{\rm max}$ for most of its life (compare blue and dashed red lines in the right panel of Fig.~\ref{fig:prtl_traj}). Its orbit in the $y-z$ plane displays a series of quasi-periodic deflections between the two sides of the reconnection layer, as expected for Speiser motion \citep{speiser_65}. Yet, while Speiser orbits in reconnection with a weak guide field are expected to get focused towards the midplane $y=0$ \citep[e.g.,][]{cerutti_13a}, the trajectory of positron \emph{A} displays a $y$-extent increasing over time. This is caused by interactions with plasmoids, whose effect is  not taken into account in standard Speiser orbits. In fact, at the time corresponding to the white circle in Fig.~\ref{fig:prtl_traj}~(a-2), the positron has just been deflected towards the upstream by the interaction with the plasmoid located at $z\sim 0.45 L$. The positron Lorentz factor at this time is $\gamma\sim 200$ and its Lamor radius is $r_{\rm L} =\gamma m c^2/eB_0\simeq 0.08L$, which is larger than the plasmoid transverse width. It follows that the positron will not be captured by the plasmoid, but rather it is deflected away from the midplane, which allows positron \emph{A} to keep gaining energy at nearly the maximal rate, while executing a Speiser-like motion.

The orbit of particle \emph{B} is different, and more typical of the majority of high-energy positrons. It is trapped in a plasmoid in the interval $0.1L/c\lesssim \Delta t\lesssim0.4L/c$. During this stage, it moves back and forth in both  $x$- and  $z$-directions, while its Lorentz factor stays roughly constant at $\gamma\sim 20$. The positron succeeds in escaping the plasmoid at $\Delta t\sim 0.5 L/c$. After that, it experiences fast acceleration while being deflected in a Speiser-like fashion between the two converting upstream flows, similarly to positron \emph{A}. By studying a sample of $\gamma\sim 30$ particles temporarily trapped in a given plasmoid, we have found that the ones that manage to escape have typically larger $z$ velocities and are preferentially located in the plasmoid outskirts. This is expected, since such particles, by moving along $z$, will be able to successfully travel outside the plasmoid, and thus experience efficient acceleration by the upstream field. Clearly, this cannot happen in 2D, where the $z$ direction is invariant (i.e., plasmoids are infinitely long in $z$).

%It also gains and loses energy several times by interacting within the plasmoid before it %obtains sufficient energy and 
%escapes in the $z$-direction at $\Delta t\sim 0.5 L/c$. We observe a linear acceleration since then and it also moves along a Speiser orbit. By studying a series of particles with Lorentz factor near $30$ in a plasmoid, we notice that those particles which are finally untrapped are located at the outskirt of the plasmoid and more likely to escape in the $z$-direction. This is because in 3D the $z$-direction is no longer invariant (as it is in 2D), the relativistic drift kink instability disrupts the current sheet in the $z$-direction and above a plasmoid there can be a thin current sheet. %When a particle moves into such a region, it may escape the reconnection region along $z$-axis primarily.
%When moving into such a thin layer, particles are likely to escape from reconnection region and the fast acceleration stage starts.

%which is larger than the size of the plasmoid \citep{sironi_16} and it is not captured. However, it gains a large momentum in the $y$-direction ($\Delta p_y\sim 100$) and is ejected into the upstream. The meandering width (the maximum distance from the mid-plain) of the Speiser orbit increases significantly to $\sim 0.1L$. \citet{cerutti_13a} predicted that particles following a Speiser orbit should be slowly focused towards the mid-plane as time passes. In our simulation, we do not see this because particles may undergo several deflections by interacting with plasmoids. Similar to Eq.~\ref{eq:e_est}, we could also estimate the acceleration rate outside the reconnection region: 

%To demonstrate the temporal evolution of the acceleration process, we select two particles which gain $\Delta\gamma>200$ and are typical of high energy particles in our simulation. We mark their $\Delta z - \Delta \gamma$ trace in the left panel of Fig.~\ref{fig:dgamma_dz}. They follow an asymptotic linear relationship: $\Delta \gamma \propto \Delta z$ once their Lorentz factor reaches $\sim 20$. 

%we study a variety of particles of which the maximum Lorentz factor is greater than 200 - the cutoff $p_{z-}$ spectrum as we have discussed in the previous section. 
%We find two types of particles among them - some are always accelerated in a nearly constant rate, others are trapped in plasmoids at some stages but manage to escape. 
%Here, we show characteristic examples of both types. We show their trajectory in . Both particles start at nearby of an X point in the mid-plane ($y=0$), where magnetic field lines are rearranged. 
%Both experience a Fermi-type of acceleration later - deflected back and forth between the two  the reconnection generated bulk flows - and follow a Speiser orbit \citep{speiser_65}.

%Particle \emph{A} exhibits a nearly constant acceleration rate ($\Delta \gamma \propto \Delta t$) since the beginning. The strong electric field $E_z$ accelerates the particle almost linearly along the $z$-direction, while the magnetic field $B_x$ confines the particle toward the mid-plane. It follows a relativistic Speiser orbit. At $t \simeq 2.20L/c$ \hz{($\Delta t = 0.79L/c$)}, it encounters a plasmoid located near $y = 0.05L$ and $z = 0.45L$. 

%where $\beta_z$ is the velocity of the particle in $z$-direction in unit of $c$. The upper limit of acceleration rate can be reached when particles move along $z$-direction. In such cases, $\beta_z\approx 1$ and the maximum acceleration rate is $\dot{\gamma}_{\rm upper} = \eta_{\rm rec}\sqrt{\sigma} \omp$. In the right panel of Fig.~\ref{fig:prtl_traj}, we show this upper limit using a red dashed line.

%Particle \emph{A} and \emph{B} highlight two possible accelerating regions in the 3D magnetic reconnection: inside of plasmoids\footnote{The background of Fig.\ref{fig:prtl_traj} is instantaneous density at the time when the particle is located at the filled white circle.} (\textit{e.g.} Particle \emph{B} at $\Delta t<0.4L/c$), which is similar to 2D cases, and outside of plasmoids (\textit{e.g.} Particle \emph{A} and Particle \emph{B} at $\Delta t>0.4L/c$). In 2D magnetic reconnection, due to magnetic topology, particles are confined in plasmoids\citep{dahlin_2017,li_2019}. However, in 3D, or, more generally, in the post-reconnection plasma, particles can escape from plasmoids. Like particle \emph{A}, they can travel far from the mid-plane and be further accelerated by large-scale electromagnetic fields in upstream. 
%We investigate 218 particles chosen randomly with $\gamma>100$ at the last time step ($t = 3.48L/c$) in our simulation. 35 of them moves along Speiser orbits similar to Particle \emph{A}. Only 37 of them are consistently trapped throughout their life. The rest are trapped in plasmoids for some time and then escape. We also notice that among those 35 particles following the Speiser motions similar to Particle \emph{A}, 33 are accelerated to $\gamma > 200$ at $t = 3.48L/c$, and they are also the most energetic particles. We conclude that the most energetic particles ($\gamma>200$) in 3D reconnection gain their energy mainly at the outside of plasmoids. 

\begin{figure*}
    \centering
    \subfloat[\label{fig:mix_stat}]{%
        \includegraphics[width=0.319\textwidth]{mix_med_stat.pdf}}
    \hspace{\fill}
        \subfloat[\label{fig:acc_rate_stat}]{%
        \includegraphics[width=0.322\textwidth]{acc_rate_med_stat.pdf}}
    \hspace{\fill}
    \subfloat[\label{fig:dgamma_frac_stat}]{%
        \includegraphics[width=0.329\textwidth]{dgamma_frac_med_stat.pdf}}\\
    \caption{%\hz{Any suggestions of one sentence to conclude these three plots?}. \ls{summary is ok: high energy particles are outside the layer, where they accrue most of their energy; acceleration is almost the max expected.} \ls{the vertical range of the second plot can be from 0 to 0.3; of the third from 0 to 1; also, in fig8 change gammamax into gammacut} 
    %High energy particles mainly stay outside the layer, where they accrue most of their energy $\gamma_{\rm end}$; their acceleration is almost the max expected.
    Statistical assessment of the properties of accelerated positrons. We first separate the positrons in six groups, based on the largest  Lorentz factor  $\gamma_{\rm end}$ they attain in their lifetime   (see legend in the left panel). For each group, we then compute (as described in the text) the following quantities, as a function of the particle Lorentz factor $\gamma$: the median mixing factor $\mathcal{M}$ (left); the median acceleration rate $\dot{\gamma}$ (middle), distinguishing between particles in the downstream (solid) and in the upstream (dashed); the fractional energy $\Delta \gamma_{\rm up}/\Delta \gamma_{\rm tot}$ gained while in the upstream (right panel). In the middle panel we also show, as a reference, the maximum  acceleration rate quantified by Eq.~\ref{eq:acc_rate} (horizontal dotted line). 
 %   Particles are divided into six groups based on their maximum energy in the simulation. For each particle and within each energy bin, the median value of $\mathcal{M}$ and acceleration rate are calculated (so each particle contributes only once in the statistics), as well as the ratio of energy gained in the reconnection to the total gained energy while the particle is in the given interval of energy. The median values for all particles in the same group are used to produce: (a) $\mathcal{M}$;  (b) acceleration rate of particles in the reconnection region (solid lines) and in the upstream (dashed lines); (c) the fraction of energy that a particle gains in the reconnection region. We also show the maximum possible acceleration rate (see Eq.~\ref{eq:acc_rate}) in a black dotted line in (b).
    }
    \label{fig:stat} 
\end{figure*}

Motivated by the trajectory of particle \emph{B}, we now employ a statistical approach to further investigate the properties of accelerated particles, and in particular ascertain at which energy they are most likely to escape from plasmoids and start experiencing fast acceleration by the upstream large-scale fields. This is shown in Fig.~\ref{fig:stat}. We first separate the positrons in six groups, based on the largest  Lorentz factor they attain in their lifetime (we shall call it $\gamma_{\rm end}$, given that it is typically attained at the end of the particle life; we only consider $\gamma_{\rm end}>30$). Each group corresponds to a different color in Fig.~\ref{fig:stat}. Each of the colored curves is obtained as follows. For each particle in a given $\gamma_{\rm end}$-group, its history is followed since its birth, dividing it depending on the instantaneous  Lorentz factor (for each of the six $\gamma_{\rm end}$-groups, we employ ten $\gamma$-bins, logarithmically spaced between $\gamma=3$ and $\gamma=300$). Taking all the times when a particle lies in a given $\gamma$-bin, we compute the median mixing factor $\mathcal{M}$, median acceleration rate $\dot{\gamma}$, and the fractional energy $\Delta \gamma_{\rm up}/\Delta \gamma_{\rm tot}$ gained while in the upstream (still, while crossing the selected $\gamma$-bin). The colored lines are then computed by taking the median among particles belonging to the same $\gamma_{\rm end}$-group.

Fig.~\ref{fig:stat} (left panel) shows that at low energies ($\gamma\lesssim 20$) most of the particles reside in the downstream region, regardless of their $\gamma_{\rm end}$. In fact, the mixing fraction is $\mathcal{M}\simeq 0.8$. As particles gain energy, the median $\mathcal{M}$ of the two groups with the largest $\gamma_{\rm end}$ (green and blue lines in Fig.~\ref{fig:stat}) starts to drop, down to $\mathcal{M}\lesssim 0.1$ for the particles reaching the highest energies. As also demonstrated above, particles of high energy ($\gamma\gtrsim 100$) are preferentially located in the upstream. The transition from being trapped to breaking free appears at $\gamma\sim  3\sigma\sim 30$.

%The medians of $\mathcal{M}$ for all groups are around $0.8$ when their Lorentz factor is smaller than 20. This shows that most particles reside in the reconnection region at the beginning. As particles are accelerated, the median values for three groups with larger $\gamma_{\rm end}$ (green, cyan, purple lines in Fig.~\ref{fig:stat}) start to drop. On one hand, it demonstrates that particles that can reach $\gamma>100$ can escape downstream. Particles with higher energy spend more time in the upstream \ls{[i don't think "more time" is correct. remind me whether each particle counts only once? (we should say this) if it counts only once, it is a statement about how many particles, not how much time]} \hz{[Each particle counts only once, and the value for each particle is the median value in each gamma bin. So it is like 50\% particles spend 50\% of their time in the upstream? I mention ``count once'' in the caption of Fig.7]}. On the other hand, as early as reaching the cutoff of particles impulsively accelerated by reconnection $\gamma\sim 3\sigma = 30$ \citep[][]{werner_16}, the particles ending up with highest energies have exhibited different behaviors from other ``regular'' particles which mainly stay and are accelerated in the reconnection region.

The middle panel of Fig.~\ref{fig:stat} presents the acceleration rate, distinguishing between particles in the downstream (solid lines) and in the upstream (dashed lines). The acceleration rate should be compared with the maximum rate $\dot{\gamma}_{\rm max}$ in \eq{ratemax}, which is indicated in the plot by the horizontal dotted line. We find that, regardless of $\gamma_{\rm end}$, downstream particles gain energy at a relatively slow rate, $\dot{\gamma}\lesssim 0.1\,\omp$. Particles residing in the upstream with Lorentz factors $\gamma\gtrsim 30$ --- the same threshold as derived from $\mathcal{M}$ in the left panel ---
gain energy at a faster rate, that asymptotes to  $\dot{\gamma}\sim 0.2\,\omp$ for the highest energy upstream particles.\footnote{In the highest $\gamma$-bin, all the curves bend towards slower acceleration rates. This can be simply understood as a selection bias: for a given $\gamma_{\rm end}$-group, particles in the highest $\gamma$-bin are biased towards having slower acceleration rates, otherwise they would move up in energy, and be classified in the next $\gamma_{\rm end}$-group.} This rate approaches $\simeq 0.8\,\dot{\gamma}_{\rm max}$, which implies that the highest energy particles move with an average $z$ velocity $\beta_z\simeq 0.8$ (see \eq{acc_rate}). This is in agreement with the momentum spectra presented in Sec.~\ref{3.1}, i.e., the highest energy particles preferentially move in the $z$ direction.

%that the acceleration rate is larger outside of the region than inside. When $\gamma>100$, the rate in upstream is about as large as in downstream. Compared with the maximum possible acceleration rate (dotted black line) estimated from Eq.~\ref{eq:acc_rate}, we find that $\beta_z\approx0.8$. This is consistent with the discussion in Section~\ref{3.1} that %the energy of the most energetic particles is dominated by the motion in $z$-direction.
%particles with highest energy mostly move along $z$-direction.

The right panel of Fig.~\ref{fig:stat} shows the fraction $\Delta \gamma_{\rm up}/\Delta \gamma_{\rm tot}$  of energy acquired in the upstream, while traversing a given $\gamma$-bin. Regardless of the $\gamma_{\rm end}$-group, we find that this is an increasing function of $\gamma$, reaching $\sim 80\%$ for the highest energy particles. Again, the transition to the stage when acceleration is dominated by the upstream motional field occurs at $\gamma\gtrsim 30$, the same threshold already derived from the left and middle panels. So, we conclude that most particles ending up with high energies escape from plasmoids at $\gamma\sim 3 \sigma\sim 30$, at which point their energization starts to be dominated by the large-scale upstream field.

%, the fraction of energy that a particle gains in the reconnection region surges with $\gamma$ when $\gamma > 30$. 
%Here, the fraction of energy is calculated when particles cross each energy interval, instead of the cumulative energy fraction. The acceleration in upstream plays a more and more important role as the energy of particles increases. 

%For those particles of which Lorentz factor can reach $300$, about $80\%$ of energy are obtained outside of plasmoids (when their $\gamma>300$). So, we conclude that the highest energy particles reside outside the layer, where they acquire their energy at a fast rate. Particles that will end up to high energy escape from plasmoids at $\gamma\sim30$.

%\citet{cerutti_13a} predicted that particles follow a Speiser orbit, which should be slowly focused towards the mid-plane as time passes. In our simulation, we do not see the orbit to be gradually focused, because particles may undergo small deflections when they cross the current sheet and their direction can be changed.


%%%%%%%%%%%%%%%%%
\subsection{Dependence on the Domain Size}\label{3.3}

\begin{figure*}
    \centering
    \subfloat[\label{fig:g_max}]{%
        \includegraphics[width=0.31\textwidth]{g_max_1.pdf}}
    \hspace{\fill}
    \subfloat[\label{fig:ratio_x} ]{%
        \includegraphics[width=0.33\textwidth]{ratio_y_1.pdf}}
    \hspace{\fill}
    \subfloat[\label{fig:ratio_z}]{%
        \includegraphics[width=0.33\textwidth]{ratio_z_1.pdf}}\\
    \caption{Dependence on the box size. The unit of $L_x$ and $L_z$ is the plasma skin depth ($\comp$).
    (a) The energy cutoff $\gamma_{\rm cut}$ for free particles as a function of time for boxes with $L_x = 2\,L_z$ (see legend). (b) Number and energy efficiency for boxes with different $L_x$, at fixed aspect ratio $L_x/L_z=1$ (blue) and $L_x/L_z=2$ (red). Empty circles show the number efficiency, defined as $N_{\rm free}/N_{\rm rr}$, where $N_{\rm free}$ and $N_{\rm rr}$  are respectively the number of free particles and of particles in the reconnection region; filled circles show the energy efficiency $E_{\rm free}/E_{\rm rr}$.  
    (c) Number (empty grey circles) and energy (filled grey circles) efficiency for boxes with different $L_z$, at fixed $L_x\sim 1600\comp$. Blue and red points are the same as in (b).
    %\ls{in final plot, can we have crosses instead of asterisk?. also, in first two panels, we can say "number and energy efficiency" instead of "ratio".} \hz{ok.} \ls{finally, i want to make sure that the definition of L is consistent between fig.1 and fig.5. is this half box or full box? maybe we can define L to be the FULL box in all cases. also good to label panels (a), (b), (c)}
    }
    \label{fig:scan_box}
\end{figure*}


\begin{table}[] 
\begin{threeparttable}
\begin{tabularx}{\columnwidth}{cccc}
\hline\hline
box size & $\Delta_{\rm init}/[\comp]$ & $N_{\rm free}/N_{\rm rr}$ & $E_{\rm free}/E_{\rm rr}$ \\ 
\hline
$1.6k\times0.8k$ & $500$ & $0.006$($0.008$)\tnote{$\dagger$} & $0.131$($0.173$) \\ 
\hline
\multirow{2}{*}{$0.8k\times0.4k$} & $500$ & $0.012$ & $0.161$ \\
& $250$ & $0.015$ & $0.200$ \\ 
\hline
\multirow{3}{*}{$0.4k\times0.2k$} & $500$ & $0.012$ & $0.149$ \\
& $250$ & $0.016$ & $0.199$ \\
& $125$ & $0.013$ & $0.166$ \\ 
\hline
$1.6k\times1.6k$ & $500$ & $0.008$ & $0.179$ \\ 
\hline
\multirow{2}{*}{$0.8k\times0.8k$} & $500$ & $0.014$ & $0.188$ \\
& $250$ & $0.014$ & $0.197$ \\ 
\hline
\multirow{3}{*}{$0.4k\times0.4k$} & $500$ & $0.022$ & $0.224$ \\
& $250$ & $0.023$ & $0.250$ \\
& $125$ & $0.015$ & $0.201$ \\
\hline
$1.6k\times0.4k$ & $500$ & $0.009$($0.012$)\tnote{$\ddagger$} & $0.153$($0.199$) \\
\hline
\end{tabularx}
\begin{tablenotes}\footnotesize
\item[$\dagger$] The results in  parentheses are from a simulation with the same parameters, but with four particles per cell.
\item[$\ddagger$] The results in  parentheses are from a simulation with the same parameters, but with guide field $B_g = 0$.
\end{tablenotes}
\caption{Number and energy efficiency for the population of high-energy free particles. The number (energy, respectively) efficiency is the ratio of the number (energy) of free particles normalized to the number (energy) of particles in the reconnection region. The unit of length for the box size (leftmost column) is the plasma skin depth ($\comp$). 
%The guide field is $B_g = 0.1B_{0}$ and $n_0 = 1$ for all results except the values in parentheses.
} \label{tab:box}
\end{threeparttable}
\end{table}

In this subsection, we investigate the dependence of the properties of high-energy ``free'' particles on the size of the computational domain, in order to extrapolate our conclusions to larger (fluid) scales. Free particles are defined such that they reside in the upstream, with mixing parameter $\mathcal{M}<0.3$. We also require that they have $\gamma>10$, to exclude the cold upstream particles that have yet to reach the reconnection region. Our results are presented in Fig.~\ref{fig:scan_box} and Tab.~\ref{tab:box}.



%As we have mentioned in the previous section that most of the particles with $\gamma>200$ are not confined in plasmoids, it is interesting to investigate how many of non-thermal particles are not confined and how much energy they carry in the relativistic reconnetion. 
%In Fig.~\ref{fig:scan_box}, we compare the number of ``free'' particles and their energy for different box sizes, where ``free'' particles are defined as particles that (1) $p_z > 10$; and (2) locate outside of reconnection regions (\textit{$\mathcal{M} < 0.3$}). We only consider particles with $p_z > 10$ because { particles with $p_z < 10$ in the upstream are mainly those which have not yet reached the reconnection region and accelerated through magnetic reconnection.}
%the main contribution to the $p_z \lesssim 10$ part of the spectrum comes from the ``unreconnected'' particles that are being injected into the reconnection region from upstream 
% The ``free'' particles defined by these two criteria is shown in a blue dotted line in Fig.~\ref{fig:spectra}.

In the left panel, we show the cutoff Lorentz factor $\gamma_{\rm cut}$ of
free particles, as a function of time (horizontal axis) and box size (different colors, as indicated in the legend). The cutoff Lorentz factor is obtained by calculating the location of the peak of $(\gamma-1)^3{\rm d}N_{\rm free}/{\rm d}\gamma$. As described in \citet{petropoulou_18}, this is generally a good proxy for the location of the exponential cutoff of the spectrum. We find that $\gamma_{\rm cut}\propto L_x$. The proportionality constant is such that the Larmor radius of particles with Lorentz factor $\gamma_{\rm cut}$ is $r_{\rm L}(\gamma_{\rm cut})\sim 0.2 \,L_{x}$, regardless of $L_x$. This is expected, since particles accelerated near the maximal rate in \eq{ratemax} over the typical advection time $\sim L_x/c$ will obtain a Larmor radius $r_{\rm L}\sim \eta_{\rm rec}L_x\sim 0.1 L_x$.
This can be phrased as a ``Hillas criterion'' for relativistic magnetic reconnection. 

%in different boxes as a function of time in Fig~\ref{fig:g_max}. It is obtained by calculating the maximum of $(\gamma-1)^s{\rm d}N/{\rm d}\gamma$ when $\gamma>10$. We choose $s=3$ because the energy spectrum (\textit{i.e.} $(\gamma-1){\rm d}N/{\rm d}\gamma$) of ``free'' particles is nearly flat, and is followed by an exponential cutoff. 
%The cutoff energy that a particle can reach scales as $L_x$ and the Lamor radius of such a particle is $\sim 0.2 L_x$. \ls{can you quote a number? ideally, this should be a ratio of the Larmor radius of such particles to the box length L [Hillas criterion]}. 
%This is because for larger boxes, particles spend more time in the box before being advected to the boundary. 
%This is because the escape time of particles scales as $\sim L_{x}/c$ due to advection out of the boundary.
%As the most energetic particles are accelerated linearly (\textit{i.e.} $\Delta\gamma\propto \Delta t$) as we have discussed, we expect that the cutoff energy of the spectrum scales as $\gamma_{\rm cut}\propto L_x$.

%The energy efficiency (\textit{i.e.} $E_{\rm free}/E_{\rm rr}$, where $E_{\rm free}$ and $E_{\rm rr}$ are the energy of ``free'' particles and particles in the reconnection regions correspondingly) remains constant. On the one hand, in Fig.~\ref{fig:spectra}, we notice that the spectrum of ``free'' particles is harder than $-1$. It indicates that the contribution to the total energy mainly comes from the most energetic particles, so $E_{\rm free}$ scales as $L_x^2L_z$. On the other hand, the particles in the reconnection region have a spectrum softer than $-1$. As a result, $E_{\rm free}$ also scales as $L_x^2L_z$. Therefore, we observe a constant energy efficiency in Fig.~\ref{fig:ratio_x} and \ref{fig:ratio_z}.

%The total number of ``free'' particles ($N_{\rm free}$) scales as $L_xL_z$, whereas the number of particles in reconnection region ($N_{\rm rr}$) scales as $L_x^2L_z$ roughly. Therefore, the number efficiency of converting injected particles to ``free'' particles (\textit{i.e.} $N_{\rm free}/N_{\rm rr}$) is proportional to $L_x^{-1}$. In Fig.~\ref{fig:ratio_x} and \ref{fig:ratio_z}, we find that in the two largest boxes in each series (\textit{i.e.} $1.6{\rm k}\times0.8{\rm k}$ and $0.8{\rm k}\times0.4{\rm k}$; $1.6{\rm k}\times1.6{\rm k}$ and $0.8{\rm k}\times0.8{\rm k}$), the number efficiency drops as the box becomes larger. However, we notice that the number efficiency flattened at lower $L$. This may be the result of the difference of the reconnection rate. The number of particles in the reconnection region in this case does not scale as $L_x^2L_z$.

%\ls{i suggest to start discussing energy fraction and cutoff first, and then number fraction.}
%\hz{when I discussed $E_{\rm free}\propto L_y^2L_z$, I used $N_{\rm free}\propto L_yL_z$ from the number fraction.}
%In contrast to $N_{\rm free}/N_{\rm rr}$, which drops as the size of the box increases, the energy efficiency (\textit{i.e.} $E_{\rm free}/E_{\rm rr}$, where $E_{\rm free}$ and $E_{\rm rr}$ are the energy of ``free'' particles and particles in the reconnection regions correspondingly) remains constant. On the one hand, in Fig.~\ref{fig:spectra}, we notice that the spectrum of ``free'' particles $ (\gamma-1) {\rm d}N /{\rm d}p_z $ is harder than $(\gamma-1)^{-1}$. It indicates that the contribution to the total energy mainly comes from the most energetic particles, so $E_{\rm free}$ scales as $L_x^2L_z$. On the other hand, the particles in the reconnection region have a spectrum softer than $(\gamma-1)^{-1}$. As a result, $E_{\rm free}$ also scales as $L_x^2L_z$. 
%\ls{i don’t fully understand this argument.} 
%\hz{My understanding is that $E_{tot}\sim\int n d\gamma$, the largest contribution in the integral is from the peak of the spectrum, so $E_{tot}\sim n_{peak} \gamma_{peak}$. $\gamma_{peak}$ is independent of the size of the box, $n_{peak}$ scales as $L_y^2L_z$, so $E_{tot}$ scales as $L_y^2L_z$. Does it make sense?}
%Therefore, we observe a constant energy efficiency in Fig.~\ref{fig:ratio_x} and \ref{fig:ratio_z}.

We also calculate the energy efficiency $E_{\rm free}/E_{\rm rr}$ (filled circles), where $E_{\rm free}$ and $E_{\rm rr}$ are respectively the energy content of free particles and of particles in the reconnection region. The number fraction $N_{\rm free}/N_{\rm rr}$ (open circles) is obtained in a similar way. We examine their dependence on the $x$ length of the box in the middle panel (at fixed aspect ratio $L_x/L_z$, blue for $L_x/L_z=1$ and red for $L_x/L_z=2$) and on the $z$ length in the right panel, for fixed $L_x\sim 1600\comp$ (grey points).
%$E_{\rm free}$ is derived by integrating the ``free'' particles with $\gamma>10$ while $E_{\rm tt}$ is the total energy of particles in the reconnection region.%:
%\[E_{\rm free} = \int_{\gamma>10}(\gamma-1)\left(\frac{{\rm d}N_tot}{{\rm d}\gamma}-\frac{{\rm d}N_rr}{{\rm d}\gamma}\right) {\rm d}(\gamma-1)\]
%and
%\[E_{\rm rr} = \int_{\gamma>10}(\gamma-1)\frac{{\rm d}N_{\rm rr}}{{\rm d}\gamma} {\rm d}(\gamma-1)\]
As shown in Fig.~\ref{fig:ratio_x}, $E_{\rm free}/E_{\rm rr}$ is nearly independent of $L_x$. This demonstrates that, regardless of the box size, free particles carry a constant fraction ($\sim 20\%$) of the post-reconnection particle energy. Given that $\gamma_{\rm cut}\propto L_x$ and that the spectrum of free particles is hard, $dN_{\rm free}/d\gamma\propto \gamma^{-1.5}$, this implies that their number fraction needs to decrease with increasing box size, as indeed confirmed by Fig.~\ref{fig:ratio_x} (open circles).

Fig.~\ref{fig:ratio_z} shows that convergent 3D results are obtained only if the box is sufficiently extended in the $z$ direction. For our reference case with $L_x\sim 1600\comp$, convergent 3D results are obtained for $L_z\gtrsim 400\,\comp\sim L_x/4$. This may be due to the requirement that the $z$ extent of the largest plasmoids, $\sim 0.1 L_x$ (assuming spherical plasmoids), be smaller than the box length along $z$, i.e.,  $z$ invariance should be broken even for the largest plasmoids. Fig.~\ref{fig:ratio_z} also shows that the 2D limit is approached for $L_z\lesssim 20\,\comp$, such that even small plasmoids do not fit within the vertical extent of the box. 

%\ls{[we need to be a bit more accurate here (and earlier as well). "reconnected particle energy" in principle may/should include both free particles and particles in rec region. so, "fraction of dissipated energy" should actually be 0.2/(1.2)?]} \hz{[yes. $0.2/1.2= 0.167\approx 0.2$]}

%In contrast to $E_{\rm free}/E_{\rm rr}$, which remains constant as the size of the box increases, the
%In addition, the number fraction of ``free'' particles decreases as the box increases. We notice that the spectrum of ``free'' particles (blue dotted line in Fig.~\ref{fig:spectra} and the green solid line in the inset of Fig.~\ref{fig:spect_2d}) is nearly flat from $\gamma\sim 1$ to $\gamma > 100$. This indicates that the main contribution to $E_{\rm free}$ comes from the most energetic particles and $\gamma_{\rm{max}}$ increase with the box size as we have discussed previously. The total energy carried by ``free'' particles does not change. As a result, the number of ``free'' particles has to decrease.

%This is consistent with the two largest boxes in each series (\textit{i.e.} $1.6{\rm k}\times0.8{\rm k}$ and $0.8{\rm k}\times0.4{\rm k}$; $1.6{\rm k}\times1.6{\rm k}$ and $0.8{\rm k}\times0.8{\rm k}$, unit: $\comp\times\comp$) in Fig.~\ref{fig:ratio_x} and \ref{fig:ratio_z}.  However, we notice that the number efficiency is flattened at lower $L_x$. This may be the result of the difference in the reconnection rate \hz{for the two smallest boxes - the one in the $L_{x}/L_{z}=1$ group has $\eta_{\rm rec} = 0.080$ while the one in the $L_{x}/L_{z}=2$ group has $\eta_{\rm rec} = 0.065$}. \ls{[clarify this statement]} \hz{[Rec rate : Lx=4k box: 0.055; Lx=2k box: 0.080; Lx = 1k box: 0.080(Lz=1k), 0.065(Lz=500). I am not sure this is convincing that the number efficiency is flattened at lower $L_x$.]}

%\ls{purpose: connect 2d and 3d.}
%To study the transition from 2D magnetic reconnections to 3D ones,  we also show the efficiency of boxes with the same $L_x$ and various $L_z$ from $L_z=12\comp\ll\L_x$ to $L_z=2000~\comp\sim L_x$ in Fig.~\ref{fig:ratio_z}. When $L_z < L_x/80$, both number and energy efficiency drop significantly. Due to the short length of $L_z$, there are few structures developed in the $z$-direction. Thus, the reconnection in this configuration is similar to the 2D case, where particles are trapped in plasmoids. When $L_x/80 < L_z < L_x/4$, the size of the box along $z$-direction is comparable to the size of the largest plasmoid ($\sim 0.05L_x$). As $L_z$ increases, more particles can escape from plasmoids, and the efficiency of reconnection generating ``free'' particles also increases.  When $L_z > L_x/4$, the efficiency is saturated. At this stage, $L_z$ is much larger than %the size of plasmoids along the $z$-direction or the typical length of any flux ropes ($\sim 0.05L_x$). So the increase of $L_z$ no longer enhance reconnection efficiency.

%%%%%%%%%%%%%
\section{Summary and Discussion}\label{4}
In this work, we performed large-scale 3D PIC simulations of relativistic reconnection in a $\sigma=10$ electron-positron plasma.
We found that a fraction of particles with $\gamma\gtrsim 3\sigma$ can ``break free'' from plasmoids by moving along $z$ and then experience the large-scale motional electric field in the upstream region. This process cannot be captured by 2D simulations, which are invariant along the $z$ direction.
The free particles preferentially move along $z$ and are accelerated linearly in time ($\gamma\propto t$) while undergoing Speiser-like deflections by the converging upstream flows, as already hypothesized by \citet{giannios_10}.
Their spectrum is  hard and can be modeled as a power law $dN_{\rm free}/d\gamma\propto \gamma^{-1.5}$ --- in Appendix \ref{slope}, we analytically justify the value of the power-law slope. The free particles account for $\sim 20\%$ of the dissipated magnetic energy, independently of domain size.

{The acceleration rate of the particles, in this mechanism,
is closely connected to the reconnection speed. 
Its accurate description, therefore, relies on 
the reconnection system having reached a statistical steady
state. To this end, our adopted boundary conditions    
are of crucial importance. By adopting continuous
injection of plasma (and magnetic flux) in the far upstream and outflow boundaries in the 
reconnection exhaust direction, the system can be followed for many Alfv\'en crossing times
after it has reached a statistical steady state. We find that the most energetic particles take
a few Alfv\'en crossing times to reach their maximum energy. In contrast, the more commonly adopted triple periodic
boundaries can only study reconnection transiently  and may not be able to capture
this mechanism accurately. In addition, periodic boundaries do not allow plasmoids to escape, so the largest plasmoids can grow up to a size comparable to the system length. This would artificially enhance the rate at which high-energy free particles get captured back by plasmoids.}

%we performed several 3D PIC simulations of anti-parallel relativistic reconnection in pair plasmas with $\sigma = 10$ under a guide field $B_g=0.1B_0$ to study the acceleration rate and efficiency. Though 3D reconnection has a lower rate than 2D, it produces particles of higher energy and a harder spectrum.%, indicating that more magnetic energy is dissipated than 2D due to the enhanced acceleration efficiency.

%In our simulation, unlike the 2D case in which particles are trapped in plasmoids because of magnetic topology, we notice that accelerated particles in 3D can escape from the reconnection region. Those particles can carry $\gtrsim20\%$ of total energy of reconnected particles. Once particles ``leak’’ out from the reconnection layer, they can be exposed to the large scale electromagnetic fields of the reconnection upstream and can be further accelerated. 

We find that the particle acceleration rate is
\begin{equation}
    \dot{\gamma} \sim \frac{\eta_{\rm rec} eB}{mc} \beta_z,
\end{equation}
where $B$ is the magnetic field in the upstream and $\eta_{\rm rec}\beta_z\sim 0.06$ as determined by our simulations. As far as we can infer from our simulations, the maximum 
energy achievable by this process is determined by either radiative losses or by the size of the reconnection region. If synchrotron cooling
is the dominant energy loss, then the particles are accelerated until the energy gain rate $\dot{\gamma} m c^2$ is balanced by the synchrotron loss rate
$(4/9)e^4B^2\gamma^2/m^2c^3$, resulting in a maximum $\gamma_{\rm max}$:
\begin{equation}\label{eq:gmax_red}
\gamma_{\rm max}=\sqrt{\frac{9\beta_z\eta_{\rm rec}m^2c^4}{4e^3B}}.
\end{equation}
The corresponding synchrotron emission energy is $E_{\rm syn}=\gamma_{\rm max}^2Be\hbar/mc=160\beta_z\eta_{\rm rec}$~MeV$\sim 10$~MeV, for electrons, i.e., about one order of magnitude below the well-known burnoff limit \citep{dejager_harding_92}. We remark that in this argument we have assumed that the accelerated particles have large pitch angles (i.e., the angle between the particle velocity and the magnetic field), as indeed observed in our simulations. Yet, if this acceleration mechanism were to operate also in the limit of strong guide fields, the accelerated particles would likely have small pitch angles, which would comparatively reduce their synchrotron  losses.\footnote{On the other hand, a strong guide field would also help in enforcing invariance along the $z$ direction (i.e., plasmoids are likely to be very elongated in $z$), so our proposed mechanism may turn out to be less efficient.} 

If radiative losses are negligible, the maximum energy is only limited by the size of the reconnecting system.
%(i.e., $L_x$ and $L_z$ in Fig.~\ref{fig:dens_3d}). 
For a given length of the reconnection layer $L_z$ in the $z$ direction, the maximum Lorentz factor a particle can reach can be estimated as: 
\begin{equation}\label{eq:gmax}
    \gamma_{\rm max} = L_{\rm z}e\eta_{\rm rec}B/mc^2.
\end{equation}
The particle motion along the $x$-direction of reconnection outflows is unlikely to constrain the maximum energy, since we have shown that the accelerated particles mostly move along the $z$ direction, so their escape time along $x$ is likely longer than along $z$.
%because particles' motion in $x$- direction is more stochastic and may be scattered several times if the outflow is not ordered. Their escape time along $x$-axis is longer than that of $z$-axis.

%\hz{[I discuss the implication of strong guide field here. Not sure if this is a good idea to do it here.]} We employ our simulation with a guide field $0.1B_0$. We also exam the reconnection without a guide field and get almost identical results. Guide fields can play an important role in 3D PIC simulation. If there is a weak guide field or no guide field, there might be fewer well-defined flux tubes due to the relativistic drift-kink instability \citep{zenitani_08}. A sufficiently strong guide field, which may be the most generic case of reconnection in astrophysical plasma,  will balance the pressure and inhibit strong turbulence in the $z$ direction. This will produce well-defined flux tubes, where fewer particles can escape and be further accelerated, compared to strong turbulence cases. Therefore, we expected that the maximum energy of particles can gain through reconnection will decrease and the spectrum will be softer and have a lower cutoff energy for a stronger guide field \citep{guo_20_b} \ls{[I am a bit confused as to the main results of these two papers: werner+17, guo+20. are the references in the appropriate place?]} \hz{[I changed the reference to guide fields effect to \citep{zenitani_08}]}. If the guide field is so strong that the instability length scale along the guide field is much larger than the plasmoid size, the picture of particle acceleration will look more like 2D. 

Although in this work we have only focused on electron-positron plasmas, our results may still be applicable to electron-proton cases, since high-$\sigma$ reconnection is virtually identical in pair plasmas and electron-proton plasmas. 
%In astrophysical plasmas that contain a substantial number of protons, the different particle species cool at very different rates. 
Protons are much less affected by cooling as compared to leptons and may escape from plasmoids with higher efficiency. Therefore, given that we have neglected cooling losses, our results may be most applicable to protons in astrophysical systems where radiative lepton losses are severe. Nevertheless, in less extreme environments (e.g., the emission zone of a blazar jet), even leptons should be able to able to escape small plasmoids and participate in this acceleration process. Therefore, one may expect that both leptonic and hadronic signatures will be affected by the acceleration mechanism we have discussed here. 
%by this efficient mechanism.% Those protons may contribute to the cosmic ray we observe.

%DG: Discuss here that the rate of acceleration is linear with time and the slope is fast. Also, the maximum energy is only limited by the system size (the potential available in the z diirection of the layer) and not related to plasma scales. Give some estimates for the maximum energy that can be attained in realistic Astrophysical systems following the Giannios 2010 estimates.

The sources and acceleration mechanism of UHECRs with energies between $\sim 10^{18}$ and $\sim10^{20}$~eV are still under debate. Relativistic jets launched by GRBs \citep{milgrom_1995, waxman_1995} and AGNs \citep{halzen_02} have been proposed as sources of UHECRs. Magnetic reconnection  taking place in the magnetically dominated plasma of these jets may be a promising accelerator of UHECRs \citep{giannios_10}.

The rest-frame magnetic field strength can be estimated by the Poynting luminosity of the jet $\mathscr{L}_{\rm P}$, the bulk Lorentz factor $\Gamma$, and the distance $R$ from the central engine (see, e.g., \citet{giannios_10}):
\begin{equation}
B\simeq \frac{\mathscr{L}_{\rm P}^{1/2}}{c^{1/2}R\Gamma}.
\end{equation}
Combining with Eq.~\ref{eq:gmax}, we can estimate the maximum energy that a proton can be accelerated to:
\begin{equation}
	E_{\rm max} = \eta_{\rm rec}  \mathscr{L}_{\rm P}^{1/2} c^{-1/2} R^{-1} L e,
\end{equation}
where $L$ is the length scale of the reconnection region. We also introduce a factor of $\Gamma$ when we boost from the jet frame to the observer frame.

For long-duration GRBs, the (isotropic equivalent) energy they release in gamma rays is around $10^{53}$~erg in a duration of about $10$~s. This gives a lower limit for their Poynting luminosity  of $\mathscr{L}_{\rm P}\sim 10^{52}~{\rm erg\,s^{-1}}$, since the energy conversion efficiency to gamma rays needs to be below unity. The size of the reconnection region can be estimated as $L\sim R/\Gamma$, and $\Gamma$ usually varies from 100 to 1000 in GRB jets. Therefore, the maximum energy that a particle can be accelerated to is $E_{\rm max} \approx 2\times 10 ^{20} \eta_{\rm rec,-1}\mathscr{L}_{\rm P, 52}^{1/2} \Gamma_2^{-1} {\;\rm eV}$, where $\eta_{\rm rec,-1} = \eta_{\rm rec}/0.1$, $\mathscr{L}_{\rm P, 52} = \mathscr{L}_{\rm P}/10^{52}{\rm erg \,s^{-1}}$, and $\Gamma_2 = \Gamma/10^2$.

A powerful AGN jet can reach a luminosity of $10^{48}~{\rm erg\,s^{-1}}$. They usually have bulk Lorentz factors of  $\Gamma \sim 3-30$. Using these values, we estimate that protons accelerated by reconnection in AGN jets can reach energies $E_{\rm max}\sim 6\times10^{19} \eta_{\rm rec,-1}\mathscr{L}_{\rm P, 48}^{1/2} \Gamma_{0.5}^{-1} {\;\rm eV}$ where $\mathscr{L}_{\rm P, 48} = \mathscr{L}_{\rm P}/10^{48}{\rm erg\,s^{-1}}$ and $\Gamma_{0.5} = \Gamma/\sqrt{10}$.

Both jets are then capable of accelerating protons (or heavier nuclei in AGN jets) to $10^{18}$~eV and even to the highest energies that have been observed so far, $10^{20}$~eV. Though we do not consider constrains imposed by cooling losses (see, e.g., \citet{giannios_10} for further discussion), our analysis demonstrates that relativistic reconnection is, in principle, a promising way to produce UHECRs.

%As for pulsar wind nebula, several studies have been conducted by \citep{lyutikov_18} to explain how particles producing flares are accelerated in highly magnetized regions of the crab nebula. The observation of brief but bright flares of energetic gamma-ray reveals a fast acceleration of particles to $\sim$~PeV$=10^{15}$~eV. From Eq.~\ref{eq:gmax}, we can estimate the largest energy that a particle can get:
%\[E_{\rm max} = \eta_{\rm rec}L Be.\]
%Assuming that the flaring emitting region is causally connected, we can estimate the size of the current sheet as $ct_{\rm flare}\sim10^{16}$~cm where $t_{\rm flare}$ is the time scale of each flare \citep{cerutti_14}. The magnetic field $B\sim 10^{-4}$~Gauss. Therefore, $E_{\rm max}\sim 3\times 10^{14}$~eV?


\acknowledgements
HZ and DG acknowledge support from the NASA ATP NNX17AG21G, the NSF AST-1910451, and the NSF AST-1816136 grants. LS acknowledges support from the Sloan Fellowship, the Cottrell Scholars Award, NASA ATP NNX17AG21G and NSF PHY-1903412. The simulations have been performed at Columbia (Habanero and Terremoto), with NERSC (Cori) and NASA (Pleiades) resources.



%%%%%%%%%%%%%%%%% APPENDICES %%%%%%%%%%%%%%%%%%%%%

\appendix

\section{2D Particle Spectra}\label{appd}



In Fig.~\ref{fig:spectra}, we have shown the energy and momentum spectra extracted from the 3D simulation. For comparison, in Fig.~\ref{fig:spect_2d} we show the $z$-momentum spectrum from the corresponding 2D run. In agreement with earlier 2D studies, the particle spectrum is non-thermal. Unlike the 3D case, where the cutoff in the $p_{z+}$ spectrum is much larger than in the $p_{z-}$ spectrum, here the two are roughly comparable, though particles moving along $+z$ extend to slightly larger energies, and dominate in number at high energies. In addition, as discussed in the main text, at $\gamma>2$ the spectra from the whole box are coincident with corresponding spectra extracted from the reconnection region alone.
%along  spectra for the 2D simulation with similar configurations (see Sec.~/ref{2}). 
%Consistent with the previous 2D study, a power-law momentum (main) and energy (inset, red solid line)  distribution is developed for non-thermal particles. Unlike the 3D case where there is a significant difference between $p_{z+}$ and $p_{z-}$ for high energy particles ($\gamma\gtrsim10$), the  $p_{z+}$ and $p_{z-}$ spectra in 2D are on the same order of magnitude, though particles moving along $+z$ direction are slightly dominant. This supports the claim that particles in 2D are confined in plasmoids and may not undergoing acceleration by the large-scale electric field in upstream. 

\begin{figure}
    \includegraphics[width=1\columnwidth]{spectra_2d_v6.pdf}
    \caption{The main panel is as in Fig.~\ref{fig:spectra}, but for a 2D simulation. We display the positron time-averaged (between $t = 3.34L/c$ and $3.56L/c$) spectra of the momentum along $+z$ (blue lines) and $-z$  (green lines). 
    Solid lines refer to the whole box, while dashed lines to the particles in the downstream.  In the inset, we show the particle energy spectrum $(\gamma-1){\rm d}N/{\rm d}\gamma$ from the whole simulation box in the 2D simulation (red line) and in the 3D simulation (blue), as well as the spectrum of free particles from the 3D simulation (green). The latter can be fit as a power law ${\rm d}N/{\rm d}\gamma\propto (\gamma-1)^{-1.5}$, as indicated by the dotted black line in the inset. %(see Fig.~\ref{fig:spectra}) at between $t = 3.45~L/c$ and $3.55~L/c$. All spectra in the inset plot are normalized to the total number of particles with $\gamma>2$ in each simulation box.
    }
    \label{fig:spect_2d}
\end{figure}

In the inset, we compare energy spectra between 2D and 3D simulations. The 3D spectrum has a higher cutoff than the 2D one and it is dominated at high energies by particles outside the reconnection region. It indicates once again that  acceleration outside of plasmoids plays an important role for the highest energy particles. 

%\begin{figure}
%    \includegraphics[width=0.5\columnwidth]{figures/spect_app.pdf}
%    \caption{Temporal evolution of the particle energy spectrum $(\gamma-1){\rm d}N/{\rm d}\gamma$ from the whole simulation box and the upstream (3D), as well as the whole simulation box for the 2D case with similar configurations (see \ref{fig:spect_2d}). The dotted, dashed, and solid lines are the spectrum at $t=2.20~L/c$, $2.81~L/c$, and $3.45~L/c$ correspondingly. All spectra are normalized to the total number of particles with $\gamma>2$ in each simulation box.}
%    \label{fig:spect}
%\end{figure}

\section{The Power-Law Slope of Free Particles}\label{slope}

\begin{figure}
    \includegraphics[width=1\columnwidth]{n_prtl_r_v03.pdf}
    \caption{The ratio of the number of free particles to the number of trapped particles. The trapped particles at a given time are defined as the ones that are currently trapped, but were free at some point in the previous $\sim 0.25 L_x/c$.
    }
    \label{fig:ratio_n_free_n_trap}
\end{figure}

In the main body of the paper, we have demonstrated that the spectrum of free particles can be modeled as a power law $f={\rm d} N_{\rm free}/{\rm d} \gamma\propto \gamma^{-1.5}$ followed by a cutoff, which scales linearly with the system size. In this Appendix, we aim at providing a theoretical framework to interpret the value of the power-law slope.\footnote{We remark that, as shown in our paper, the acceleration mechanism of free particles is distinct from the one of particles accelerated in the reconnection layer, whose spectral shape has been discussed by, e.g., \citet{guo_14,uzdensky_20}.} In steady state, the distribution of free particles will follow 
\begin{equation}\label{eq:fp}
    \frac{\partial}{\partial \gamma} \left(\frac{\gamma}{t_{\rm acc}}f\right) + \frac{f}{t_{\rm esc}} = Q_0 \delta(\gamma-3 \sigma),
\end{equation}
where both the acceleration time $t_{\rm acc}$ and the escape time $t_{\rm esc}$ generally depend on the particle Lorentz factor. In the equation above, $Q_0$ quantifies the particle injection rate, which we have assumed to happen at a fixed energy of $\gamma=3\sigma$, in agreement with the results obtained in the main body of the paper.

For free particles, the acceleration time can be calculated as $t_{\rm acc} \equiv \gamma/\dot{\gamma} = \gamma/(\eta_{\rm rec}\beta_z\sqrt{\sigma}\omp)$, where our results yield $\eta_{\rm rec}\beta_z\simeq 0.06$. Free particles are accelerated while residing in the upstream. When they get captured and trapped by plasmoids, they effectively escape the accelerator, so the escape time $t_{\rm esc}$ from the acceleration region is, for free particles, the time spent in the upstream before getting trapped.\footnote{For simplicity, we assume that no significant energization occurs while trapped in plasmoids.}

In order to compute the escape time of free particles, we assume a steady state scenario: the rate at which free particles are trapped in plasmoids balances the rate at which trapped particles advect out of the domain, so 
\begin{equation}\label{eq:eq1}
t_{\rm esc}=t_{\rm adv}\frac{dN_{\rm free}/d\gamma}{dN_{\rm trap}/d\gamma} 
\end{equation}
%%%
Fig.~\ref{fig:ratio_n_free_n_trap} shows, at different times, the ratio of the number of free particles to the number of trapped particles. The trapped particles at a given time are defined as the ones that are currently trapped, but were free at some point in the previous $\sim 0.25 L_x/c$ (our results are not appreciably sensitive to this choice). The plot shows that the ratio is in steady state, and in the range $30\lesssim\gamma\lesssim 300$,\footnote{This is the energy range between the injection Lorentz factor at $\gamma=3\sigma=30$ and the spectral cutoff at $\gamma\sim 300$, see the inset of \figg{spect_2d}.} it can be fit as 
\begin{equation}\label{eq:eq2}
\frac{dN_{\rm free}/d\gamma}{dN_{\rm trap}/d\gamma}\simeq 0.005\, \gamma
\end{equation}
We have computed the advection time $t_{\rm adv}$ considering the lifetime of particles that remain always trapped in plasmoids. We find that $t_{\rm adv}\sim 0.4 L_x/c$ independently of the Lorentz factor. In retrospect, this is not surprising. The mass-weighted bulk motions of relativistic reconnection are trans-relativistic, with typical outflow velocities of $\sim 0.6 \,c$ \citep{sironi_beloborodov_20}. On average, a trapped particle travels a distance $\sim 0.25 L_x$ before advecting out of the system, which indeed leads to an advection time $t_{\rm adv}\sim 0.4 L_x/c$.  
%\ls{revise estimate, and possibly the strategy}
%low-energy particles with $\gamma\lesssim 20$, which are typically trapped in plasmoids and advected out of the $x$ boundaries.

It follows that the ratio of acceleration time to escape time is energy-independent (for $30\lesssim\gamma\lesssim 300$) and equal to $t_{\rm acc}/t_{\rm esc}\simeq 1.6$, where we have used that $L_x/\comp\sim 1600$ in our reference simulation. This allows to compute the solution of \eq{fp}. As discussed by e.g., \citet{kirk_98}, the solution of \eq{fp} in the case that both the acceleration time and the escape time scale linearly with $\gamma$ is a power law
\begin{equation}
f=\frac{dN_{\rm free}}{d\gamma}\propto \gamma^{-t_{\rm acc}/t_{\rm esc}}
\end{equation}
which for our case yields $dN_{\rm free}/d\gamma\propto \gamma^{-1.6}$, in good agreement with the spectrum measured directly from our simulation.

Based on this model for the acceleration of free particles, one can address the question of what is the power-law slope expected in the asymptotic (and astrophysically-relevant) regime $L_x\gg \comp$. Let us call $s=-d\log N_{\rm free}/d\log \gamma$ the power-law slope of free particles in the asymptotic limit $L_x\gg \comp$. Given that $s=t_{\rm acc}/t_{\rm esc}$ should be independent of the box size $L_x$, this requires that $t_{\rm esc}$ in \eq{eq1} be independent of $L_x$. In turn, given that $t_{\rm adv}\propto L_x/c$, the ratio in \eq{eq2} should scale as $\propto 1/L_x$. In other words, for $d N_{\rm free}/d\gamma=C_{\rm free} \gamma^{-s}$ and $d N_{\rm trap}/d\gamma=C_{\rm trap} \gamma^{-s-1}$, we require $C_{\rm free}/C_{\rm trap}\propto 1/L_x$. 

Let us now consider the specific case of $s=1$, and assume that the spectrum of free particles extends from $\gamma_{\rm min,free}\sim 3\sigma$ up to $\gamma_{\rm cut}\propto L_x$, while the spectrum of trapped particles extends from  $\gamma_{\rm min,free}\sim \sigma$ up to the same $\gamma_{\rm cut}\propto L_x$. The ratio of number of free particles $N_{\rm free}$ to number of trapped particles $N_{\rm trap}$ (which we called $N_{\rm rr}$ for ``reconnection region'' in the main paper) is, in the limit $\gamma_{\rm cut}\gg\gamma_{\rm min,free}\gtrsim\gamma_{\rm min,trap}$,
\be
\frac{N_{\rm free}}{N_{\rm trap}}=\frac{C_{\rm free}}{C_{\rm trap}} \,{\gamma_{\rm min,trap}} \,{\log\left({\frac{\gamma_{\rm cut}}{\gamma_{\rm min,free}}}\right)}
\ee
which, aside from logarithmic corrections, scales as $\propto C_{\rm free}/C_{\rm trap}\propto 1/L_x$. On the other hand, the energy fraction can be written as
\be
\frac{E_{\rm free}}{E_{\rm trap}}=\frac{C_{\rm free}}{C_{\rm trap}}\frac{\gamma_{\rm cut}}{\log({\gamma_{\rm cut}}/{\gamma_{\rm min,trap}})}~,
\ee
which, aside from logarithmic corrections, scales as $\propto(C_{\rm free}/C_{\rm trap})\,\gamma_{\rm cut}\propto {\rm const}$, in agreement with our results in \figg{scan_box} (there, we called $E_{\rm rr}$ the energy content of trapped particles). Based on our model of acceleration, and requiring that the energy fraction of free particles stays constant with box size, we then expect that the spectrum of free particles in the limit $L_x\gg \comp$ should reach an asymptotic power-law slope $s\simeq 1$.


\begin{comment}
\ls{I would like to raise here a point of discussion, which I already presented last week during our meeting. Let us assume that $t_{adv} \propto L$, whereas $t_{esc}$ is independent of $L$ (this is required, since $t_{acc}$ is independent of L, in order to have a slope that is box-independent). Then the ratio of the two $dN/d\gamma$ should scale as $1/t_{adv}\propto 1/L$. Is this observed? And, more fundamentally, is this consistent with our observation that the ($\gamma$-integrated) $N_{free }\propto L^{-0.5}$? It is possible that everything is consistent, since one is a ratio $\gamma$-dependent, and the other is $\gamma$-integrated, up to a cutoff of the distribution that scales as $L$, but I still find it a bit weird...}

\ls{Another way of looking at it is that there may be just one value of the slope that will make the two arguments self-consistent? Hao, what do you think?}

\hz{From Fig.~\ref{fig:ratio_n_free_n_trap} we have: $\frac{dN_{\rm free}/d\gamma}{dN_{\rm trap}/d\gamma} \simeq \alpha \gamma$. Assume the free particle energy spectrum is $dN_{\rm free}/d\gamma = \beta_{\rm free}\gamma^{-s}$, the trapped particle energy spectrum is $dN_{\rm trap}/d\gamma = \beta_{\rm trap}\gamma^{-s-1}$, where $\beta_{\rm free}/\beta_{\rm trap} = \alpha$. Integrate both $dN_{\rm free}/d\gamma$ and $dN_{\rm trap}/d\gamma$:  $N_{\rm free} = \frac{\beta_{\rm free}}{s-1} \gamma_{\rm min, free}^{1-s}$ and $N_{\rm trap} = \frac{\beta_{\rm trap}}{s} \gamma_{\rm min, trap}^{-s}$ (supposing $\gamma_{\rm min} \gg \gamma_{\rm min}$ and $s > 1$). Therefore, $N_{\rm free}/N_{\rm trap} = \alpha \frac{s}{s-1} \frac{\gamma_{\rm min, free}^{1-s}}{\gamma_{\rm min, trap}^{-s}}$. Here $\alpha$ scales as $1/L$. If we assume that slope ($s$) is box-independent, $\gamma_{\rm min, free}$ and $\gamma_{\rm min, trap}$ have to be different and box-dependent.}
\end{comment}

%$N_{\rm trap}=(N_{\rm free}/t_{\rm esc}) t_{\rm trap}$, where $t_{\rm esc}$ is the escape time of free particles, and $t_{\rm trap}$ is the escape time of trapped particles. We have $t_{\rm trap} \sim0.15L_x/c$ \hz{[I am checking this. Lorenzo, I think this value may be too small?]} for steady reconnection in our simulation. Therefore, we can estimate the escape time of free particles $t_{\rm esc}\approx7.5\times10^{-4}\gamma L_x/c$. 
%The acceleration time can be calculated as $t_{\rm acc} \equiv \gamma/\dot{\gamma} = \gamma/(\eta_{\rm rec}\sqrt{\sigma}\omp)$. Therefore, for our largest simulation box, $t_{\rm esc}/t_{\rm acc} \approx 0.3$. %t_{\rm esc}/t_{\rm acc} = $0.277\beta_z$.


%where $f={\rm d} N_{\rm free}/{\rm d} \gamma$ is the spectrum for free particles. Solving the above equation, we derive $f\propto \gamma^{-3.6}$. \hz{This is too soft?}

%\section{Tentative plots}
%\bi
%\item trajectory of one or two best particles. this could be the x vs z coordinates, and maybe some fluid quantities at a time when the particles are clearly outside the layer. maybe 3d visualization would be best here, if we find a way to plot the particle track on top of the 3d fluid isosurface. please experiment! like fig.1 here https://iopscience.iop.org/article/10.3847/1538-4357/ab4268/pdf
%\begin{figure*}
%    \centering
%    \begin{subfigure}{2\columnwidth}
%       \includegraphics[width=1\linewidth]{figures/particle_1.png}
%       \caption{Particle \textit{A} at $t=4.55L/c$}
%       \label{fig:prtl_1_traj} 
%    \end{subfigure}
%
%    \begin{subfigure}{2\columnwidth}
%       \includegraphics[width=1\linewidth]{figures/particle_2.png}
%       \caption{Particle \textit{B} at $t=5.59L/c$}
%       \label{fig:prtl_2_traj}
%    \end{subfigure}
%    \caption{Trajectories of two particles. For each particle, its trojectory in the x-y plane is shown on the left, and the x-z plane on the right. The color of the line demonstrates the energy of the particle. A white spot shows its position at a given time ($t=4.55L/c$ for Particle \textit{A} and $t=5.59L/c$ for Particle \textit{B}). The background shows the density map of plasma at that time along the x-y or x-z plane.}
%    \label{fig:prtl_traj}
%\end{figure*}
%
%\item \hz{[Hao] Shall we include a figure showing fluid quantities for both 2D and 3D? We found that it is interesting that there are isolated ``spots'' outside of the reconnection layer in  the $\epsilon_{\rm kin}$ plot for 3D but not for 2D. This might indicate that in 3D particles can escape from plasmoids.}
%\item 2d histogram of instantaneous gamma (log) and mixing in interpolated nearest cell, comparison of 2d and 3d sims. this justifies the mixing choice of 0.15. 2D is shown on the left while 3D is on the right.
%
%\begin{figure}
%    \includegraphics[width=\columnwidth]{figures/prtl_hist_v7.png}
%    \caption{Distribution of the instantaneous Lorentz factor of particles $\gamma$ and their mixing factor interpolated to the nearest cell at time $t=4.74L/c$. The solid contour lines demonstrate in the 3D simulation the region where $50\%$, $80\%$, $95\%$, $99\%$ of particles are located, while the dashed lines are the corresponding 2D results. The black points show all the particles of with $p_z<-80m_{\rm i}c$. }
%    \label{fig:prtl_hist_v7}
%\end{figure}
%\begin{figure}
%    \includegraphics[width=\columnwidth]{figures/prtl_hist_v8.png}
%    \caption{Histograms of the instantaneous Lorentz factor of particles $\gamma$ and their mixing factor interpolated to the nearest cell at time $t=4.74L/c$. The solid contour lines demonstrate in the 3D simulation the region where $50\%$, $80\%$, $95\%$, $99\%$ of particles are located, while the dashed lines are the corresponding 2D results. 8 }
%    \label{fig:prtl_hist_v8}
%\end{figure}
%
%\item pz vs -pz spectrum, both rec region and whole box spectrum (this could be the time average, or one representative time). so, 4 lines overall. in inset, we can plot pz and energy spectrum, to show that high particles are mostly moving in +pz.
%
%\item 2d histogram of deltagamma and deltaz, comparison of 2d and 3d. maybe, instead of 2d histogram (or, on top of it), we could just plot the median, for particles starting with different gammas. dashed line for linear accelerator by rec electric field.
%
%\item same, but deltagamma vs deltat. [ choose either this or the previous one; I prefer the previous one, what do you think] \hz{[Hao] I also think deltagamma vs deltz is better, since it indicates high energy particles in 3D can be accelerated by the E-field in z direction.}
%\begin{figure*}
%    \centering
%    \begin{subfigure}{\columnwidth}
%       \includegraphics[width=1\linewidth]{figures/dgamma_dt_2d.png}
%       \caption{2D}
%       \label{fig:dgamma_dt_2d} 
%    \end{subfigure}
%    \begin{subfigure}{\columnwidth}
%       \includegraphics[width=1\linewidth]{figures/dgamma_dt_3d.png}
%       \caption{3D}
%       \label{fig:dgamma_dt_3d}
%    \end{subfigure}
%    \caption{$\Delta t$ versus $\Delta z$. The discrete lines are caused by the time resolution (about $0.02 L/c$).  \hz{[Hao: I prefer Fig 4 (dz-dgamma).]}}
%    \label{fig:dgamma_dt}
%\end{figure*}
%
%\item fraction (energy) of interesting particles as a function of box size, both along z and along y (for two or all three aspect ratios).
%\ei
%
%\section{Diagnostics}
%\ls{we should mention that our measurement of efficiency is really a lower limit. many more partilces are interesting, yet they are trapped at the final time. maybe we can construct a spectrum asking to indicate the particles that have spent at least 30 percent of their life outside the layer, they will be  more.}
%
%
%\ls{IMPORTANT: when showing spectra, we should exclude the region at the y boundary where we impose the boundary conditions. You can see from the fluid files and gauge how many spectral slabs (each one  100 cells thick) we should exclude.}
%\hz{For all the spectra we have showed in the past, we excluded the region.}
%LS: I am adding here ideas of additional diagnostics
%\bi
%\item \ls{it would still be useful to plot gammai vs the mixing at the particle location. This however should be based on interpolating the particle location to the closest cell where we have fluid outputs, and compute the mixing with the fluid-based (flds.tot) quantities.}\hz{Done.}% \hz{Should I make a histogram for all particles, or just a plot for a single particle?} \ls{I am thinking of 2d histogram, mixing on y axis as log scale}
%\item (superseded by better diagnostics, let's discard this one) similarly (but less elegantly), we could think about other quantities that are associated with current sheet, for example (in rough order of preference): Ez (which is the actual field that accelerates), Jz (or the overall J), Bz/B0 (which is compressed), bulk velocity, density (compressed). Possibly even the $v_E \cdot \kappa$ parameter discussed by the paper we talked about.
%\item (nice to show, but not really quantitative) the easiest will be to take one representative time and show 2d and 3d snapshot, and put the highest energy particles. 
%\item plot momentum spectra on top of energy spectra. if we are right, in 3d the highest energy particles should have pz spectrum coincident with energy spectrum (but not in 2d!). true!
%\item \ls{choose different (more restrictive) mixing threshold for spectra and momentum spectra (in progress).}
%\item \ls{PRIORITY: for assessing number of particles experiencing this process: identify gamma such that the + and - pz momentum spectra differ by one order of magnitude (or more), and compute total number of particles beyond this energy; or, same thing for mixing spectrum (see above). also, energy budget of this additional population.}
%\item \ls{the 2d histogram of delta gammai vs delta zi should also be done for 2d simulation, as a comparison.} \hz{See Google Drive. For 2d, I chose 65\% of particles in the last saved output file and calculated $\Delta z$ and $\Delta \gamma$.}
%\item \ls{I am still uncertain about the plots of gammai vs time for particles starting with the same gammai. we know that in 2d the scaling is $\gamma\propto t^{1/2}$ while it is linear in 3d for the highest energy particles, but maybe we can also give this message from delta gammai vs delta zi plot?} %\hz{We do see the linear relation between $\Delta\gamma$ and $\Delta z$ for particles with $\gamma\gtrsim 300$}
%\ei
%
%\section{Further insight}
%
%\ls{we will have to think about focusing by speiser orbits, see uzdensky 2011: main question is: in a speiser orbit, particles should be focused more and more towards the midplane, and so they would be captured by plasmoids. yet, some of our particles go farther and farther from the midplane, why?}
%
%\begin{itemize}
%    \item what allows some particles to enter this acceleration mechanism? maybe best thing here is to isolate a few good particles, follow them back in time when they had smaller energies ($\sim 50$) and comparing their properties with same-energy particles that did not get accelerated much. maybe they all started near center of layer?  maybe they always were kind of directed along z? \ls{probably best strategy here is to pick the smallest 3d box for which we see the effect, and follow a bunch of these particles. by any chance, they all happened to be in the same place at the same time?} \hz{Working on this one}
%    \item \ls{maybe they got a preferential parallel work contribution (wpari) when they got injected? for this, we can just plot 2d histogram of gammai vs wpari} \hz{Done.}
%    \item what is the fraction of such particles? is it a constant with box size? in the movie it looked like their normalization was dropping as their energy was increasing....we also have a 4k.2k box now \ls{we now have a way to quantify their number, based on pz + and - momentum.} \hz{Done. The fraction of those ``interesting'' particles looks different between 4k*4k and 2k*2k boxs. The difference is about a factor of 2.}
%    \item compute spatial diffusion coefficient in the xy plane (if doable). actually, it will be advection plus diffusion, but there are standard ways of distinguishing (see wong, uzdensky 2020) \hz{working on this one}
%    \item for LS to think: check bz0 as well.
%\item topic 1:
%
%- we have yet to understand the origin of the weird scaling of number fraction with Ly, but we suspect it may be due to our choice of defining the vertical line for interesting particles.
%
%- one thing we can try for alternative definition: we should just use +pz from whole box and compare to the one from reconnection region, the vertical line is defined where the latter drops below 0.5 of the former.
%
%- alternative: we use vertical line that is the same at all times. for the interesting particles, we could take the highest cutoff ever achieved by the -pz momentum spectrum. the cutoff can be obtained with the procedure described here then we use this vertical line for the +pz spectrum.
%
%\item topic 2: what is special about these particles?
%
%- the rapid rise in number fraction for some boxes is not very informative since it primarily tracks lack of -pz particles

%- we should probably then use the boxes with the smallest Lz. can you make a movie with pz, -pz (for both whole box and rec region) for the case with mz=3e1, and compare it with the 2d case. by the way, for the 2d run that we have, the threshold is 0.15?
%    \item maybe one more (final?) choice to compute the interesting particles: compute difference of spectrum in whole box and rec region, for pz above 10. these are the interesting particles.
%    \item y tracks vs time could be used to estimate mode of escape (advection vs diffusion). maybe even one figure for the paper.
%\end{itemize}


\bibliographystyle{aasjournal}
\bibliography{blob}

\end{document}