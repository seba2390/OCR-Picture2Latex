% \documentclass[12pt, a4paper, aps, prb, amssymb, amsmath,superscriptaddress]{revtex4}
\documentclass[superscriptaddress, amsmath,amssymb, preprint]{revtex4-2}
%\documentclass[journal = jpcc]{achemso}
%\documentclass[a4paper,10pt]{article}
%\documentclass[prb,twocolumn,showpacs,amsfonts,amssymb,amsmath,superscriptaddress]{revtex4}
\usepackage[utf8]{inputenc}
\usepackage{epsfig}
\usepackage{graphicx}
\usepackage{adjustbox}
\usepackage{indentfirst}
\usepackage{multirow}
\usepackage{latexsym}
\usepackage{color, soul}
\usepackage{ragged2e}
\usepackage{upgreek}
 \usepackage{amsmath}
\usepackage{amsmath}
\usepackage{amssymb}
\usepackage{mathtools}
\renewcommand{\thetable}{\arabic{table}}

\usepackage{hyperref}
\providecommand{\url}[1]{\texttt{#1}}
\usepackage{dcolumn}
\usepackage{graphicx, xcolor}
\usepackage{sansmath}
\usepackage{float}
\renewcommand{\thetable}{\arabic{table}}

	\hypersetup{
	   	bookmarks		= false,		% show bookmarks bar %
		unicode			= false,	% non-Latin characters in Acrobat's bookmarks
		pdfborder			= {0 0 0}	% style of the border around a link; {0 0 0} gives no border %
		pdftoolbar			= false,		% show Acrobat's toolbar %
		pdfmenubar		= true,		% show Acrobat's menu %
		pdffitwindow		= false,     	% window fit to page when opened %
		pdfstartview		= {FitH},    	% fits the width of the page to the window %
		pdfnewwindow	= true,      	% links in new window %
		colorlinks			= true,		% false: boxed links; true: colored links %
		linkcolor			= blue,		% color of internal links %
		citecolor			= blue,		% color of links to bibliography %
		filecolor			= blue,		% color of file links %
		urlcolor			= blue,		% color of external links %
		linktocpage		= true,		% whole entry or only page in the toc is made into a link %
		pdftitle				= {Moir\'{e} superlattice effects and band structure evolution in near-30-degree twisted bilayer graphene},
		pdfauthor			= {M. Mucha-Kruczynski},
		pdfsubject			= {},
		pdfcreator			= {MikTeX},
		pdfproducer		= {Marcin Mucha-Kruczynski},
		pdfkeywords		= {},
		}

\newcounter{bla}
\newenvironment{refnummer}{%
\list{[\arabic{bla}]}%
{\usecounter{bla}%
 \setlength{\itemindent}{0pt}%
 \setlength{\topsep}{0pt}%
 \setlength{\itemsep}{0pt}%
 \setlength{\labelsep}{2pt}%
 \setlength{\listparindent}{0pt}%
 \settowidth{\labelwidth}{[9]}%
 \setlength{\leftmargin}{\labelwidth}%
 \addtolength{\leftmargin}{\labelsep}%
 \setlength{\rightmargin}{0pt}}}
 {\endlist}

\usepackage[normalem]{ulem}
%  \providecommand{\NZ}[1]{{\color{blue}{#1}}}
%  \providecommand{\NZc}[1]{\NZ{[NZ: #1]}}
%  \providecommand{\NZx}[1]{\NZ{\sout{#1}}}
% 
% \providecommand{\NZ}[1]{{\color{red}{#1}}}
% \providecommand{\NZc}[1]{\NZ{\color{blue}[NZ: #1]}}
% \providecommand{\NZx}[1]{\NZ{\sout{#1}}}

% abbreviations for commom commands
% \def\bk{\ensuremath{{\mathbf{k}}}}
% \def\bq{\ensuremath{{\mathbf{q}}}}
% \def\br{\ensuremath{{\mathbf{r}}}}
% \def\sgw{\mbox{\textsc{SternheimerGW}}}
% \def\GW{GW}
% \def\GWnot{G$_0$W$_0$}
% \def\Ex{\ensuremath{{E_{\rm x}}}}
% \def\Ec{\ensuremath{{E_{\rm c}}}}
% \def\me{\ensuremath{{m_{\rm e}}}}
% \def\mh{\ensuremath{{m_{\rm h}}}}
% \def\mr{\ensuremath{{m_{\rm r}}}}
% \def\Sr{\ensuremath{\Sigma}}
\def\gap{HSE gap}
\def\lgap{LDA gap}
\def\SSex{ReS$_{2-x}$Se$_x$}
\def\SSeOne{ReS$_{1.75}$Se$_{0.25}$}
\def\SSeTwo{ReS$_{1.5}$Se$_{0.5}$}
\def\SSeThree{ReS$_{1.25}$Se$_{0.75}$}
\def\SeSx{ReSe$_{2-x}$S$_x$}
\def\SeSOne{ReSe$_{1.75}$S$_{0.25}$}
\def\SeSTwo{ReSe$_{1.5}$S$_{0.5}$}
\def\SeSThree{ReSe$_{1.25}$S$_{0.75}$}
\newcommand\ReXTwoY{ReX$_{2-x}$Y$_x$}
\newcommand\SOne{1}
\newcommand\STwo{2}
\newcommand\SThree{3}
\newcommand\SFour{4}

\newcommand{\vect}[1]{\boldsymbol{#1}}	% I use command \vect{} for vectors

	\newcommand{\beginsupplement}{%
        \setcounter{table}{0}
        \renewcommand{\thetable}{S\arabic{table}}%
        \setcounter{figure}{0}
        \renewcommand{\thefigure}{S\arabic{figure}}%
     }


\def\myscale{0.5}
\makeatletter
\DeclareMathOperator{\myhelper@Re}{Re} % workaround as \Re is already defined
\renewcommand{\Re}{\myhelper@Re}
\DeclareMathOperator{\myhelper@Im}{Im} % workaround as \Im is already defined
\renewcommand{\Im}{\myhelper@Im}
\makeatother

\begin{document}
%opening

\title{Supplemental Material: \\ Using In-Plane Anisotropy to Engineer Janus Monolayers of Rhenium Dichalcogenides}
\author{Nourdine Zibouche}
\email{n.zibouche@bath.ac.uk}
\affiliation{Department of Chemistry, University of Bath, Bath BA2 7AY, United Kingdom\\}
\author{Surani M.~Gunasekera}
\affiliation{Department of Physics, University of Bath, Bath BA2 7AY, United Kingdom\\}
\affiliation{Centre for Nanoscience and Nanotechnology, University of Bath, Bath BA2 7AY, United Kingdom\\}
\author{Daniel Wolverson}
\affiliation{Department of Physics, University of Bath, Bath BA2 7AY, United Kingdom\\}
\affiliation{Centre for Nanoscience and Nanotechnology, University of Bath, Bath BA2 7AY, United Kingdom\\}
\author{Marcin Mucha-Kruczynski}
\email{m.mucha-kruczynski@bath.ac.uk}
\affiliation{Department of Physics, University of Bath, Bath BA2 7AY, United Kingdom\\}
\affiliation{Centre for Nanoscience and Nanotechnology, University of Bath, Bath BA2 7AY, United Kingdom\\}


% \date{\today}
%\begin{abstract}
%The new class of Janus two-dimensional (2D) transition metal dichalcogenides with two different interfaces is currently gaining increasing attention due to the possibility to access properties different from the typical 2D materials. Here, we show that in-plane anisotropy of a 2D atomic crystal, like ReS$_{2}$ or ReSe$_{2}$, allows formation of a large number of inequivalent Janus monolayers. We use first-principles calculations to investigate the structural stability of 29 distinct ReX$_{2-x}$Y$_{x}$ ($\mathrm{X,Y \in  \{S,Se\}}$) structures, which can be obtained by selective exchange of exposed chalcogens in a ReX$_{2}$ monolayer. We also examine the electronic properties and work function of the most stable Janus monolayers and show that the large number of inequivalent structures provides a way to engineer spin-orbit splitting of the electronic bands. We find that the breaking of inversion symmetry leads to sizable spin splittings and spontaneous dipole moments that are larger than those in other Janus dichalcogenides. Moreover, our calculations suggest that the work function of the Janus monolayers can be tuned by varying the content of the substituting chalcogen. Our work demonstrates that in-plane anisotropy provides additional flexibility in sub-layer engineering of 2D atomic crystals.
%\end{abstract}

\maketitle

\newpage
\beginsupplement

%\section{Dynamical Stability}
%
%In the main text, by investigating the formation energies, we show that Janus monolayers ReX$_{2-x}$Y$_{x}$, $\mathrm{X=S,Se}$, are thermodynamically stable. In Fig.~\ref{fig:phonons}, we present the phonon spectra for three selected monolayers: ReS$_{2}$, ReS$_{1.5}$Se$_{0.5}$(1,2) and ReSSe to show that these materials are also dynamically stable. In all cases, all the phonon energies are real for all wave vectors and no indication of lattice instabilities is found. We suggest that this is because the low symmetry of the pure ReX$_{2}$ compounds easily accommodates small distortions of bonds due to S$\leftrightarrow$Se substitutions. As also seen in Fig.~\ref{fig:phonons}, increasing the concentration of Se in ReS$_{2}$ leads to the appearance of new optical phonon modes in the frequency range 250-280 cm$^{-1}$. This because heavier Se atoms lead to the formation of lower frequency phonons as compared to the S atoms \cite{hart_npj2d_2017}. Moreover, addition of Se in place of S leads to the increase of the velocity of the lowest acoustic phonon.
%
%\begin{figure}[b]
%\begin{center}
%\includegraphics[width=1.00\columnwidth]{figS1_phonons}
%\caption{Phonon spectra for ReS$_{2}$ (left), ReS$_{1.5}$Se$_{0.5}$(1,2) (centre) and ReSSe (right) along the high symmetry directions $\Gamma$-$K$-$M$-$\Gamma$ of the Brillouin zone.}
%\label{fig:phonons}
%\end{center}
%\end{figure}


\section{Binary convex hull}

In Fig.~\ref{fig:convex_hull}, we present data from Fig.~2 of the main text after subtracting a linear term so that the formation energies of both pure compounds are scaled to zero. This results in a binary convex hull which allows us to determine whether a particular Janus monolayer is energetically favourable to form as compared to the pristine materials -- this is the case if its formation energy lies below zero (dashed line in the figure). We denote with black dots the most stable compounds for a given S/Se concentration and use other colours for the remaining structures (the energy ordering of the structures for a set concentration is the same as described in the main text). It is clear that a small addition of S to ReSe$_{2}$ or Se to ReS$_{2}$ stabilizes the structure. Interestingly, the full Janus layer, ReSSe, lies the highest above zero which suggests that it is the least stable. However, this simply indicates that the ReSSe Janus monolayer would not form in a $1:1:1$ mixture of Re, S and Se but, rather, some other allotrope structure could result. Note that here, we focus on the stability and properties of materials prepared in a specific way starting from a single layer of a transition metal dichalcogenide. The Janus monolayers of transition metal dichalcogenides are not grown using any typical crystal growth method but rather obtained in a process which strips the top layer of chalcogens to allow their substitution with another \cite{lu_natnano_2017, zhang_acsnano_2017, Sant_2020}. The final material preserves the planar configuration and the process itself puts strict constraints on which lattice sites are available for chalcogen substitution (only on one side of the transition metals) so that Janus layers form even if they are only metastable. 

\begin{figure}[t]
\begin{center}
\includegraphics[width=0.80\columnwidth]{figS1_binary_convex_hull}
\caption{Binary convex hull of ReX$_{2-x}$Y$_{x}$. The black dots denote the most stable structures for a given S/Se concentration. The other colours correspond to the other structures with their overall energetic ordering for a set concentration the same as described in the main text.}
\label{fig:convex_hull}
\end{center}
\end{figure}

\section{Crystal orbital Hamilton populations analysis}

In order to investigate the changes in bonding occuring in Janus ReS$_{2-x}$Se$_{x}$ due to S$\leftrightarrow$Se exchange, we compute the energy-resolved (projected) Crystal Orbital Hamilton Populations (COHP) \cite{dronskowski_jpc_1993} for these materials,
\begin{align}
\mathrm{COHP}_{ij}(\epsilon) = H_{ij}\rho_{ij}(\epsilon),
\end{align}
where $H_{ij}$ is the Hamiltonian matrix element between Bloch states constructed from orbitals $i$ and $j$ and $\rho_{ij}(\epsilon)$ is the corresponding matrix element of the density of states matrix,
\begin{align}
\rho_{ij}(\epsilon)=\sum_{n}f_{n}c_{n,i}^{*}c_{n,j}\delta(\epsilon-\epsilon_{n}).
\end{align}
Above, $f_{n}$ and $\epsilon_{n}$ are the occupation factor and energy of band $n$, $c_{n,j}$ is the expansion coefficient of the wave function for the Bloch state composed from orbital $j$ and $\delta(\epsilon-\epsilon_{n})$ is the Dirac delta function in energy. COHP partitions the band structure energy into orbital-pair interactions and can be interpreted as a ''bond-weighted'' density-of-states between a pair of adjacent atoms. Whereas the electronic density of states shows where the electrons are but adds nothing about their bonding character, COHP indicates bonding, nonbonding and antibonding contributions to the band structure. For a bonding contribution, the corresponding Hamiltonian off-site element (and hence COHP) is negative, indicating lowering of the energy by forming a bond. Consequently, antibonding interactions are indicated by positive off-site Hamiltonian elements and COHP. Finally, while integrating the electronic density of states gives the number of electrons in the system, an energy integral of the COHP shows the contribution of a specific bond to the band energy so that integral of COHP provides information about the bond strength.

\begin{figure}[t]
\begin{center}
\includegraphics[width=1.00\columnwidth]{figS2_cohp}
\caption{Crystal Orbital Hamilton Populations (COHP) for chalcogen-transition metal bonds in (a) ReS$_{2}$ and (b) ReS$_{1.75}$Se$_{0.25}$. For each chalcogen site, 1-4, COHP for bonds with the three nearest Re sites (numbered like in the diagram on top of the figure) are shown.}
\label{fig:cohp}
\end{center}
\end{figure}

In Fig.~\ref{fig:cohp}, we show COHP calculated for the bonds between each of the four chalcogen sites and the nearest three of the four transition metals, both for ReS$_{2}$ (top row) as well as the most stable Janus monolayer with one of the S exchanged to Se, ReS$_{1.75}$Se$_{0.25}$(1) (bottom row). To identify each bond, in addition to the chalcogen sites numbered from 1 to 4 like in the main text, we number the rhenium sites from Re1 to Re4 as shown in the diagram on top of the figure (while two of the rhenium sites, X and Y, are related to the other two by inversion through the centre of the unit cell, for a given chalcogen site bonds with all transition metals are different). In the legend of each panel of Fig.~\ref{fig:cohp}, we also give the length of each bond. 

Overall, COHP plots show little change between a pure compound, ReS$_{2}$, and one of the Janus monolayers, ReS$_{1.75}$Se$_{0.25}$(1). Despite consistent lengthening of all chalcogen-rhenium bonds by at least 0.11 {\AA} after the exchange of S to Se at only one of the chalcogen sites, for both materials equivalent bonds show bonding/antibonding character at similar energies, although in ReS$_{1.75}$Se$_{0.25}$(1) both the bonding and antibonding features are consistently appearing at lower energies than in ReS$_{2}$. This confirms that the preference of the chalcogen sites for S$\leftrightarrow$Se exchange can be interpreted primarily in terms of the space available at a particular site versus the size of the chalcogen atom, as in the main text.

\section{Bader charge analysis}

\begin{center}
\begin{table}[b]
\caption{Bader charge (in the units of electron charge) for the pristine ReX$_{2}$ and energetically most stable rhenium Janus monolayers. Black, blue and red indicates, for each Janus monolayer, sites occupied by sulphur, rhenium and selenium atoms, respectively.} \label{bader_charge}
\scriptsize
%\footnotesize
\makebox[\textwidth]{\begin{tabular}{|c|c|c|c|c|c|c|c|c|c|} 
\hline
Site & ReS$_{2}$ & ReS$_{1.75}$Se$_{0.25}$(1) & ReS$_{1.5}$Se$_{0.5}$(1,2) & ReS$_{1.25}$Se$_{0.75}$(1,2,3) & ReSSe & ReSe$_{1.25}$S$_{0.75}$(2,3,4) & ReSe$_{1.5}$S$_{0.5}$(3,4) & ReSe$_{1.75}$Se$_{0.25}$(4) & ReSe$_{2}$ \\
\hline
1t & 6.544599 & \textcolor{red}{6.336945} & \textcolor{red}{6.345631} & \textcolor{red}{6.362025} & \textcolor{red}{6.371746} & \textcolor{red}{6.334817} & \textcolor{red}{6.348748} & \textcolor{red}{6.362706} & \textcolor{red}{6.374848} \\
2t & 6.474301 & 6.484028 & \textcolor{red}{6.266196} & \textcolor{red}{6.285994} & \textcolor{red}{6.305323} & 6.491471 & \textcolor{red}{6.271878} & \textcolor{red}{6.279836} & \textcolor{red}{6.294228} \\
3t & 6.453022 & 6.482132 & 6.483440 & \textcolor{red}{6.263021} & \textcolor{red}{6.278671} & 6.477553 & 6.489886 & \textcolor{red}{6.270704} & \textcolor{red}{6.286246} \\
4t & 6.429599 & 6.439648 & 6.465184 & 6.483046 & \textcolor{red}{6.264098} & 6.451559 & 6.465634 & 6.477053 & \textcolor{red}{6.260515} \\
Re1 & \textcolor{blue}{14.041145} & \textcolor{blue}{14.095648} & \textcolor{blue}{14.160576} & \textcolor{blue}{14.155069} & \textcolor{blue}{14.220043} & \textcolor{blue}{14.261307} & \textcolor{blue}{14.326605} & \textcolor{blue}{14.328686} & \textcolor{blue}{14.399420} \\
Re2 & \textcolor{blue}{14.057225} & \textcolor{blue}{14.083037} & \textcolor{blue}{14.157820} & \textcolor{blue}{14.230763} & \textcolor{blue}{14.228753} & \textcolor{blue}{14.248626} & \textcolor{blue}{14.313685} & \textcolor{blue}{14.384037} & \textcolor{blue}{14.385648} \\ 
Re3 & \textcolor{blue}{14.041377} & \textcolor{blue}{14.085657} & \textcolor{blue}{14.096086} & \textcolor{blue}{14.147831} & \textcolor{blue}{14.209142} & \textcolor{blue}{14.270193} & \textcolor{blue}{14.269175} & \textcolor{blue}{14.335754} & \textcolor{blue}{14.397757} \\
Re4 & \textcolor{blue}{14.057577} & \textcolor{blue}{14.060922} & \textcolor{blue}{14.077708} & \textcolor{blue}{14.130088} & \textcolor{blue}{14.187400} & \textcolor{blue}{14.227497} & \textcolor{blue}{14.264671} & \textcolor{blue}{14.330714} & \textcolor{blue}{14.385614} \\
1b & 6.544452 & 6.550127 & 6.559783 & 6.562215 & 6.552776 & \textcolor{red}{6.380268} & \textcolor{red}{6.387583} & \textcolor{red}{6.390818} & \textcolor{red}{6.374640} \\
2b & 6.473904 & 6.489662 & 6.494441 & 6.478242 & 6.481587 & \textcolor{red}{6.306781} & \textcolor{red}{6.310824} & \textcolor{red}{6.293594} & \textcolor{red}{6.293976} \\
3b & 6.453137 & 6.460951 & 6.460355 & 6.466402 & 6.465974 & \textcolor{red}{6.283784} & \textcolor{red}{6.283642} & \textcolor{red}{6.283117} & \textcolor{red}{6.286072} \\
4b & 6.429984 & 6.431740 & 6.432906 & 6.435708 & 6.434777 & \textcolor{red}{6.266458} & \textcolor{red}{6.267746} & \textcolor{red}{6.262848} & \textcolor{red}{6.260575} \\
\hline
%Total & 14.057225 & 14.083037 & 14.157820 & 14.230763 & 6 & 7 & 8 & 9 & 10 \\ 
%\hline
\end{tabular}}
\end{table}
\end{center}

In order to investigate the redistribution of the electron density due to S$\leftrightarrow$Se exchange, we perform Bader charge analysis \cite{bader_charge, henkelman_cms_2006}. In this approach, the volume containing the material (here, its unit cell) is divided into atomic (Bader) volumes which contain a single density maximum and are separated from other regions by surfaces on which the charge density is a minimum normal to the surface. Importantly, as Bader partitioning is based upon the charge density, it is insensitive to the basis set used.

We show in Table \ref{bader_charge} the calculated Bader charges (charges contained within each Bader volume which can be identified with a specific atomic site) for the pristine ReX$_{2}$ as well as all the most stable Janus monolayers. To identify the atomic sites, we use the numering 1-4, as in the main text, for the chalcogen sites, with the letter t or b denoting the top (above the transition metals) or bottom (below transition metals) chalcogen layer. For the pure compounds, ReS$_{2}$ and ReSe$_{2}$, inversion symmetry maps a site $i$t, $i=1,2,3,4$, onto $i$b. In turn, we number the rhenium sites as shown on the top of Fig.~\ref{fig:cohp}. For a given Janus monolayer, we indicate with black, blue and red sites occupied by sulphur, rhenium and selenium, respectively. For all compounds, the total Bader charge sums up to the total number of valence electrons in the system, 108 (15 for rheniums and 6 for sulphurs and seleniums), with the accuracy of 0.0005 of electron charge. As seen in the table exchange of S for Se at a site leads to lowering of the electron density around that site to the level close to that found at that location in ReSe$_{2}$. The electron density pushed away from a chalcogen site as a results of such substitution is mainly transferred to the rhenium sites. This means that, on an interatomic scale, charge redistribution is only weakly affected by the detailed composition of a particular monolayer. 
 



%1t & 6.544599 & 6.336945 & 6.345631 & 6.362025 & 6.371746 & 6.334817 & 6.348748 & 6.362706 & 6.374848 \\
%2t & 6.474301 & 6.484028 & 6.266196 & 6.285994 & 6.305323 & 6.491471 & 6.271878 & 6.279836 & 6.294228 \\
%3t & 6.453022 & 6.482132 & 6.483440 & 6.263021 & 6.278671 & 6.477553 & 6.489886 & 6.270704 & 6.286246 \\
%4t & 6.429599 & 6.439648 & 6.465184 & 6.483046 & 6.264098 & 6.451559 & 6.465634 & 6.477053 & 6.260515 \\
%Re1 & 14.041145 & 14.095648 & 14.160576 & 14.155069 & 14.220043 & 14.261307 & 14.326605 & 14.328686 & 14.399420 \\
%Re2 & 14.057225 & 14.083037 & 14.157820 & 14.230763 & 14.228753 & 14.248626 & 14.313685 & 14.384037 & 14.385648 \\ 
%Re3 & 14.041377 & 14.085657 & 14.096086 & 14.147831 & 14.209142 & 14.270193 & 14.269175 & 14.335754 & 14.397757 \\
%Re4 & 14.057577 & 14.060922 & 14.077708 & 14.130088 & 14.187400 & 14.227497 & 14.264671 & 14.330714 & 14.385614 \\
%1b & 6.544452 & 6.550127 & 6.559783 & 6.562215 & 6.552776 & 6.380268 & 6.387583 & 6.390818 & 6.374640 \\
%2b & 6.473904 & 6.489662 & 6.494441 & 6.478242 & 6.481587 & 6.306781 & 6.310824 & 6.293594 & 6.293976 \\
%3b & 6.453137 & 6.460951 & 6.460355 & 6.466402 & 6.465974 & 6.283784 & 6.283642 & 6.283117 & 6.286072 \\
%4b & 6.429984 & 6.431740 & 6.432906 & 6.435708 & 6.434777 & 6.266458 & 6.267746 & 6.262848 & 6.260575 \\
%108.000322 & 108.000497 & 108.000126 & 108.000404 & 108.00029 & 108.000314 & 108.000077 & 107.999867 & 107.999539



\begin{thebibliography}{116}

\bibitem{hart_npj2d_2017} L.~S.~Hart, J.~L.~Webb, S.~Murkin, D.~Wolverson, and D.-Y.~Lin, Identifying light impurities in transition metal dichalcogenides: the local vibrational modes of S and O in ReSe$_{2}$ and MoSe$_{2}$, \href{https://dx.doi.org/10.1038/s41699-017-0043-1}{npj 2D Materials and Applications {\bf 1}, 41 (2017)}.

\bibitem{lu_natnano_2017} A.-Y.~Lu, H.~Zhu, J.~Xiao, C.-P.~Chuu, Y.~Han, M.-H.~Chiu, C.-C.~Cheng, C.-W.~Yang, K.-H.~Wei, Y.~Yang, Y.~Wang, D.~Sokaras, D.~Nordlund, P.~Yang, D.~A.~Muller, M.-Y.~Chou, X.~Zhang, and L.-J.~L.~Lu, Janus monolayers of transition metaldichalcogenides, \href{https://dx.doi.org/10.1038/nnano.2017.100}{Nature Nanotechnology {\bf 12}, 744 (2017)}.

\bibitem{zhang_acsnano_2017} J.~Zhang, S.~Jia, I.~Kholmanov, L.~Dong, D.~Er, W.~Chen, H.~Guo, Z.~Jin, V.~B.~Shenoy, L.~Shi, and J.~Lou, Janus Monolayer Transition-Metal Dichalcogenides, \href{https://dx.doi.org/10.1021/acsnano.7b03186}{ACS Nano {\bf 11}, 8192 (2017)}.

\bibitem{Sant_2020} R.~Sant, M.~Gay, A.~Marty, S.~Lisi, R.~Harrabi, C.~Vergnaud, M.~T.~Dau, X.~Weng, J.~Coraux, N.~Gauthier, O.~Renault, G.~Renaud, and M.~Jamet, Synthesis of epitaxial monolayer Janus SPtSe, \href{https://doi.org/10.1038/s41699-020-00175-z}{npj 2D Materials \& Applications {\bf 4}, 41 (2020)}.

\bibitem{dronskowski_jpc_1993} R.~Dronskowski and P.~E.~Bloechl, Crystal orbital Hamilton populations (COHP): energy-resolved visualization of chemical bonding in solids based on density-functional calculations, \href{https://doi.org/10.1021/j100135a014}{Journal of Physical Chemistry {\bf 97}, 8617 (1993)}.

%\bibitem{deringer_jpca_2011} V.~L.~Deringer, A.~L.~Tchougreeff, R.~Dronskowski, Crystal Orbital Hamilton Population (COHP) Analysis As Projected from Plane-Wave Basis Sets, \href{https://doi.org/10.1021/jp202489s}{Journal of Physical Chemistry A {\bf 115}, 5461 (2011)}.

\bibitem{bader_charge} R.~F.~W.~Bader, Atoms in Molecules: A Quantum Theory, Oxford University Press, New York, 1990.

\bibitem{henkelman_cms_2006} G.~Henkelman, A.~Arnaldsson, and H.~Jonsson, A fast and robust algorithm for Bader decomposition of charge density, \href{https://doi.org/10.1016/j.commatsci.2005.04.010}{Computational Materials Science {\bf 36}, 354 (2006)}.






%\bibitem{li_small_2018} R.~Li, Y.~Cheng, and W.~Huang, Recent Progress of Janus 2D Transition Metal Chalcogenides: From Theory to Experiments, \href{https://dx.doi.org/10.1002/smll.201802091}{Small {\bf 14}, 1802091 (2018)}. % review of Janus 2D materials
%
%\bibitem{yagmurcukardes_apr_2020} M.~Yagmurcukardes, Y.~Qin, S.~Ozen, M.~Sayyad, F.~M.~Peeters, S.~Tongay, and H.~Sahin, Quantum properties and applications of 2D Janus crystals and their superlattices, \href{https://dx.doi.org/10.1063/1.5135306}{Applied Physics Reviews {\bf 7}, 011311 (2020)}. % review of Janus 2D materials
%
%\bibitem{zhang_jmca_2020} L.~Zhang, Z.~Yang, T.~Gong, R.~Pan, H.~Wang, Z.~Guo, H.~Zhang, and X.~Fu, Recent advances in emerging Janus two-dimensional materials: from fundamental physics to device applications, \href{https://dx.doi.org/10.1039/d0ta01999b}{Journal of Materials Chemistry A {\bf 8}, 8813 (2020)}. % review of Janus 2D materials
%
%\bibitem{geim_nature_2013} A.~K.~Geim and I.~V.~Grigorieva, Van der Waals heterostructures, \href{https://dx.doi.org/10.1038/nature12385}{Nature {\bf 499}, 419 (2013)}.
%
%\bibitem{yuan_pccp_2017} X.~Yuan, M.~Yang, L.~Wang, and Y.~Li, Structural stability and intriguing electronic properties of two-dimensional transition metal dichalcogenide alloys, \href{https://dx.doi.org/10.1039/c7cp01727h}{Physical Chemistry Chemical Physics {\bf 19}, 13846 (2017)}.
%
%\bibitem{lamfers_jac_1996} H.-J.~Lamfers, A.~Meetsma, G.~Wiegers, and J.~de Boer, The crystal structure of some rhenium and technetium dichalcogenides, \href{https://dx.doi.org/10.1016/0925-8388(96)02313-4}{Journal of Alloys and Compounds {\bf 241}, 34 (1996)}.
%
%\bibitem{kertesz_jacs_1984} M.~Kertesz and R.~Hoffmann, Octahedral vs trigonal-prismatic coordination and clustering in transition-metal dichalcogenides, \href{https://dx.doi.org/10.1021/ja00324a012}{Journal of the American Chemical Society {\bf 106}, 3453 (1984)}.
%
%\bibitem{riis-jensen_jpcc_2018} A.~C.~Riis-Jensen, M.~Pandey, and K.~S.~Thygesen, Efficient Charge Separation in 2D Janus van der Waals Structures with Built-in Electric Fields and Intrinsic p--n Doping, \href{https://dx.doi.org/10.1021/acs.jpcc.8b05792}{Journal of Physical Chemistry C {\bf 122}, 24520 (2018)}.
%
%\bibitem{Xia_2018} C.~Xia, W.~Xiong, J.~Du, T.~Wang, Y.~Peng, and J.~Li, Universality of electronic characteristics and photocatalyst applications in the two-dimensional Janus transition metal dichalcogenides, \href{https://doi.org/10.1103/PhysRevB.98.165424}{Physical Review B {\bf 98}, 165424 (2018)}.

	% References for the Methods section

%\bibitem{kohn_physrev_1965} W.~Kohn and L.~J.~Sham, Self-Consistent Equations Including Exchange and Correlation Effects, \href{https://dx.doi.org/10.1103/PhysRev.140.A1133}{Physical Review {\bf 140}, A1133 (1965)}.
%
%\bibitem{gianozzi_jpcm_2009} P.~Giannozzi, S.~Baroni, N.~Bonini, M.~Calandra, R.~Car, C.~Cavazzoni, D.~Ceresoli, G.~L.~Chiarotti, M.~Cococcioni, I.~Dabo, A.~Dal Corso, S.~de Gironcoli, S.~Fabris, G.~Fratesi, R.~Gebauer, U.~Gerstmann, C.~Gougoussis, A.~Kokalj, M.~Lazzeri, L.~Martin-Samos, N.~Marzari, F.~Mauri, R.~Mazzarello, S.~Paolini, A.~Pasquarello, L.~Paulatto, C.~Sbraccia, S.~Scandolo, G.~Sclauzero, A.~P.~Seitsonen, A.~Smogunov, P.~Umari, and R.~M.~Wentzcovitch, QUANTUM ESPRESSO: a modular and open-source software project for quantum simulations of materials, \href{https://dx.doi.org/10.1088/0953-8984/21/39/395502}{Journal of Physics: Condensed Matter {\bf 21}, 395502 (2009)}.
%
%\bibitem{gianozzi_jpcm_2017} P.~Giannozzi, O.~Andreussi, T.~Brumme, O.~Bunau, M.~Buongiorno Nardelli, M.~Calandra, R.~Car, C.~Cavazzoni, D.~Ceresoli, M.~Cococcioni, N.~Colonna, I.~Carnimeo, A.~Dal Corso, S.~de Gironcoli, P.~Delugas, R.~A.~DiStasio Jr, A.~Ferretti, A.~Floris, G.~Fratesi, G.~Fugallo, R.~Gebauer, U.~Gerstmann, F.~Giustino, T.~Gorni, J.~Jia, M.~Kawamura, H.-Y.~Ko, A.~Kokalj, E.~Kucukbenli, M.~Lazzeri, M.~Marsili, N.~Marzari, F.~Mauri, N.~L.~Nguyen, H.-V.~Nguyen, A.~Otero-de-la-Roza, L.~Paulatto, S.~Ponce, D.~Rocca, R.~Sabatini, B.~Santra, M.~Schlipf, A.~P.~Seitsonen, A.~Smogunov, I.~Timrov, T.~Thonhauser, P.~Umari, N.~Vast, X.~Wu, and S.~Baroni, Advanced capabilities for materials modelling with Quantum ESPRESSO, \href{https://dx.doi.org/10.1088/1361-648X/aa8f79}{Journal of Physics: Condensed Matter {\bf 29}, 465901 (2017)}.
%
%\bibitem{pz_1981} J.~P.~Perdew and A.~Zunger, Self-interaction correction to density-functional approximations for many-electron systems, \href{https://doi.org/10.1103/PhysRevB.23.5048}{Physical Review B {\bf 23}, 5048 (1981)}.
%
%\bibitem{PAW1994} P.~E.~Bl\"{o}chl, Projector augmented-wave method, \href{https://doi.org/10.1103/PhysRevB.50.17953}{Physical Review B {\bf 50}, 17953 (1994)}.
%
%\bibitem{monkhorst_prb_1976} H.~J.~Monkhorst and J.~D.~Pack, Special points for Brillouin-zone integrations, \href{https://dx.doi.org/10.1103/PhysRevB.13.5188}{Physical Review B {\bf 13}, 5188 (1976)}.
%
%\bibitem{HSE06_2003} J.~Heyd, G.~E.~Scuseria, and M.~Ernzerhof, Hybrid functionals based on a screened Coulomb potential, \href{ https://doi.org/10.1063/1.1564060}{Journal of Chemical Physics {\bf 118}, 8207 (2003)}.
%
%\bibitem{HSE06_2006} J.~Heyd, G.~E.~Scuseria, and M.~Ernzerhof, Erratum: “Hybrid functionals based on a screened Coulomb potential” [J. Chem. Phys. 118, 8207 (2003)], \href{https://doi.org/10.1063/1.2204597}{Journal of Chemical Physics {\bf 124}, 219906 (2006)}.
%
%\bibitem{Hedin_1965} L.~Hedin, New Method for Calculating the One-Particle Green's Function with Application to the Electron-Gas Problem, \href{https://doi.org/10.1103/PhysRev.139.A796}{Physical Review {\bf 139}, A796 (1965)}.
%
%\bibitem{SLouie_2003} M.~S.~Hybertsen and S.~G.~Louie, Electron correlation in semiconductors and insulators: Band gaps and quasiparticle energies, \href{https://doi.org/10.1103/PhysRevB.34.5390}{Physical Review B {\bf 34}, 5390 (1986)}.
%
%\bibitem{SGW2020} M.~Schlipf, H.~Lambert, N.~Zibouche, and F.~Giustino, SternheimerGW: A program for calculating GW quasiparticle band structures and spectral functions without unoccupied states, \href{https://doi.org/10.1016/j.cpc.2019.07.019}{Computer Physics Communications {\bf 247}, 106856 (2020)}.

	%%%%%%%%%%%%%%%%%%%%%%%%


%\bibitem{hong_acsnano_2018} M.~Hong, X.~Zhou, N.~Gao, S.~Jiang, C.~Xie, L.~Zhao, Y.~Gao, Z.~Zhang, P.~Yang, Y.~Shi, Q.~Zhang, Z.~Liu, J.~Zhao, and Y.~Zhang, Identifying the Non-Identical Outermost Selenium Atoms and Invariable Band Gaps across the Grain Boundary of Anisotropic Rhenium Diselenide, \href{https://dx.doi.org/10.1021/acsnano.8b04872}{ACS Nano {\bf 12}, 10095 (2018)}.
%
%\bibitem{choi_acsnano_2020} B.~K.~Choi, S.~Ulstrup, S.~M.~Gunasekera, J.~Kim, S.~Y.~Lim, L.~Moreschini, J.~S.~Oh, S.-H.~Chun, C.~Jozwiak, A.~Bostwick, E.~Rotenberg, H.~Cheong, I.-W.~Lyo, M.~Mucha-Kruczynski, and Y.~J.~Chang, Visualizing orbital content of electronic bands in anisotropic 2D semiconducting ReSe$_{2}$, \href{https://dx.doi.org/10.1021/acsnano.0c01054}{ACS Nano {\bf 14}, 7880 (2020)}.
%
%\bibitem{hart_prb_2021} L.~S.~Hart, S.~M.~Gunasekera, M.~Mucha-Kruczynski, J.~L.~Webb, J.~Avila, M.~C.~Asensio, and D.~Wolverson, Interplay of crystal thickness and in-plane anisotropy and evolution of quasi-one dimensional electronic character in ReSe$_{2}$, \href{https://dx.doi.org/10.1103/PhysRevB.104.035421}{Physical Review B {\bf 104}, 035421 (2021)}.
%
%\bibitem{wen_nanoscale_2017} W.~Wen, J.~Lin, K.~Suenaga, Y.~Guo, Y.~Zhu, H.-P.~Hsud, and L.~Xie, Preferential S/Se occupation in an anisotropic ReS$_{2-x}$Se$_{2x}$ monolayer alloy, \href{https://dx.doi.org/10.1039/c7nr05289h}{Nanoscale {\bf 9}, 18275 (2017)}.
%
%\bibitem{vegard_zphys_1921} L.~Vegard, Die Konstitution der Mischkristalle und die Raumf\:{u}llung der Atome, \href{https://dx.doi.org/10.1007/BF01349680}{Zeitschrift fur Physik {\bf 5}, 17 (1921)}.
%
%\bibitem{zhong_prb_2015} H.-X.~Zhong, S.~Gao, J.-J.~Shi, and L.~Yang, Quasiparticle band gaps, excitonic effects, and anisotropic optical properties of the monolayer distorted 1T diamond-chain structures ReS$_{2}$ and ReSe$_{2}$, \href{https://dx.doi.org/10.1103/PhysRevB.92.115438}{Physical Review B {\bf 92}, 115438 (2015)}.
%
%\bibitem{echeverry_prb_2018} J.~Echeverry, and I.~Gerber, Theoretical investigations of the anisotropic optical properties of distorted 1T-ReS$_{2}$ and ReSe$_{2}$ monolayers, bilayers, and in the bulk limit, \href{https://dx.doi.org/10.1103/PhysRevB.97.075123}{Physical Review B {\bf 97}, 075123 (2018)}.
%
%\bibitem{webb_prb_2017} J.~L.~Webb, L.~S.~Hart, D.~Wolverson, C.~Chen, J.~Avila, and M.~C.~Asensio, Electronic band structure of ReS$_{2}$ by high-resolution angle-resolved photoemission spectroscopy, \href{https://dx.doi.org/10.1103/PhysRevB.96.115205}{Physical Review B {\bf 96}, 115205 (2017)}. % bulk ReS2 - indirect band gap
%
%\bibitem{hart_scirep_2017} L.~S.~Hart, J.~L.~Webb, S.~Dale, S.~J.~Bending, M.~Mucha-Kruczynski, D.~Wolverson, C.~Chen, J.~Avila, and M.~C.~Asensio, Electronic bandstructure and van der Waals coupling of ReSe$_{2}$ revealed by high-resolution angle-resolved photoemission spectroscopy, \href{https://dx.doi.org/10.1038/s41598-017-05361-6}{Scientific Reports {\bf 7}, 5145 (2017)}. % bulk ReSe2 - indirect
%
%\bibitem{eickholt_prb_2018} P.~Eickholt, J.~Noky, E.~F.~Schwier, K.~Shimada, K.~Miyamoto, T.~Okuda, C.~Datzer, M.~Druppel, P.~Kruger, M.~Rohlfing, and M.~Donath, Location of the valence band maximum in the band structure of anisotropic 1T\'{}-ReSe$_{2}$, \href{https://dx.doi.org/10.1103/PhysRevB.97.165130}{Physical Review B {\bf 97}, 165130 (2018)}. % bulk ReSe2 - undecided VBM location
%
%\bibitem{gunasekera_jem_2018} S.~M.~Gunasekera, D.~Wolverson, L.~S.~Hart, and M.~Mucha-Kruczynski, Electronic band structure of rhenium dichalcogenides, \href{https://dx.doi.org/10.1007/s11664-018-6239-0}{Journal of Electronic Materials {\bf 47}, 4314 (2018)}. % electronic bands of bulk ReSe2 and ReS2
%
%\bibitem{ji_jpcc_2018} Y.~Ji, M.~Yang, H.~Lin, T.~Hou, L.~Wang, Y.~Li, and S.-T.~Lee, Janus Structures of Transition Metal Dichalcogenides as the Heterojunction Photocatalysts for Water Splitting, \href{https://dx.doi.org/10.1021/acs.jpcc.7b11584}{Journal of Physical Chemistry C {\bf 122}, 3123 (2018)}.

\end{thebibliography}

\end{document}