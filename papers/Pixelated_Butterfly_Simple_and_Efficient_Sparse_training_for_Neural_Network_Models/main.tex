\documentclass{article} %

\usepackage{etoolbox}
\newtoggle{arxiv}
\toggletrue{arxiv}

\iftoggle{arxiv}{}
{
\usepackage{iclr2022_conference,times}
}



\usepackage{amsmath}
\usepackage{amsthm}
\usepackage{amssymb}
\usepackage{hyperref}
\usepackage{url}
\usepackage{algpseudocode}
\usepackage{algorithm}

\usepackage[inline]{enumitem}
\usepackage{caption}

\usepackage{graphicx}
\usepackage{grffile}
\usepackage{natbib}
\usepackage{wrapfig,epsfig}
\usepackage{epstopdf}
\usepackage{algpseudocode}
\usepackage{multirow}
\usepackage[T1]{fontenc}
\usepackage{bbm}
\usepackage{comment}
\usepackage{dsfont}
\usepackage{makecell}
\usepackage{enumitem}
\usepackage{booktabs}

\usepackage{amsmath}
\usepackage{amsthm}
\usepackage{amssymb}
\usepackage{algorithm}
\usepackage{color}
\usepackage[english]{babel}
\usepackage{graphicx}
\usepackage{grffile}
\usepackage{natbib}
\usepackage{wrapfig,epsfig}
\usepackage{epstopdf}
\usepackage{algpseudocode}
\usepackage{multirow}
\usepackage[T1]{fontenc}
\usepackage{bbm}
\usepackage{comment}
\usepackage{dsfont}
\usepackage{makecell}
\usepackage{enumitem}
\usepackage{booktabs}
\usepackage{afterpage}

\usepackage[capitalize]{cleveref}
\usepackage{capt-of}




\newtheorem{theorem}{Theorem}[section]
\newtheorem{lemma}[theorem]{Lemma}
\newtheorem{definition}[theorem]{Definition}
\newtheorem{notation}[theorem]{Notation}
\newtheorem{proposition}[theorem]{Proposition}
\newtheorem{corollary}[theorem]{Corollary}
\newtheorem{conjecture}[theorem]{Conjecture}
\newtheorem{assumption}[theorem]{Assumption}
\newtheorem{observation}[theorem]{Observation}
\newtheorem{fact}[theorem]{Fact}
\newtheorem{remark}[theorem]{Remark}
\newtheorem{claim}[theorem]{Claim}
\newtheorem{example}[theorem]{Example}
\newtheorem{problem}[theorem]{Problem}
\newtheorem{open}[theorem]{Open Problem}
\newtheorem{property}[theorem]{Property}
\newtheorem{hypothesis}[theorem]{Hypothesis}
\newtheorem{process}{Process}
\algnewcommand\algorithmicforeach{\textbf{for each}}
\algdef{S}[FOR]{ForEach}[1]{\algorithmicforeach\ #1\ \algorithmicdo}

\newcommand{\rank}{\mathrm{rank}}
\newcommand{\wh}{\widehat}
\newcommand{\wt}{\widetilde}
\newcommand{\ov}{\overline}
\newcommand{\eps}{\epsilon}
\newcommand{\N}{\mathcal{N}}
\newcommand{\R}{\mathbb{R}}
\newcommand{\I}{\mathbb{I}}
\newcommand{\RHS}{\mathrm{RHS}}
\newcommand{\LHS}{\mathrm{LHS}}
\renewcommand{\d}{\mathrm{d}}
\renewcommand{\i}{\mathbf{i}}
\renewcommand{\varepsilon}{\epsilon}
\renewcommand{\tilde}{\wt}
\renewcommand{\hat}{\wh}
\newcommand{\ReLU}{{$\mathsf{ReLU}$}}
\newcommand{\new}{\mathrm{new}}
\newcommand{\nnz}{\mathrm{nnz}}
\newcommand{\diag}{\mathrm{diag}}
\newcommand{\poly}{\mathrm{poly}}
\newcommand{\norm}[1]{\left\|{#1}\right\|} %
\newcommand{\abs}[1]{\left\lvert#1\right\rvert}
\newcommand{\defeq}{:=}
\newcommand{\vB}{\mathbf{B}}
\newcommand{\vD}{\mathbf{D}}
\newcommand{\B}{\mathcal{B}}
\newcommand{\BS}{\B^*}
\newcommand{\BBS}{\B\B^*}
\newcommand{\BSB}{\B^*\B}

\DeclareMathOperator*{\E}{{\mathbb{E}}}

\newcommand{\Zhao}[1]{{\color{red} [Zhao: {#1}]}}
\newcommand{\Beidi}[1]{{\color{orange} [Beidi: {#1}]}}
\newcommand{\Jiaming}[1]{{\color{blue} [Jiaming: {#1}]}}
\newcommand{\Tri}[1]{{\color{cyan} [Tri: {#1}]}}


\newcommand*\samethanks[1][\value{footnote}]{\footnotemark[#1]}

\iftoggle{arxiv}{
  \setlength{\textwidth}{6.5in}
  \setlength{\textheight}{9in}
  \setlength{\oddsidemargin}{0in}
  \setlength{\evensidemargin}{0in}
  \setlength{\topmargin}{-0.5in}
  \newlength{\defbaselineskip}
  \setlength{\defbaselineskip}{\baselineskip}
  \setlength{\marginparwidth}{0.8in}
}{
\usepackage[compact]{titlesec}
\titlespacing{\section}{0pt}{*1}{*0}
\titlespacing{\subsection}{0pt}{*0}{*0}

\usepackage[subtle, mathdisplays=tight, charwidths=normal, leading=normal]{savetrees}

\addtolength\textwidth{0.15in}
\addtolength\textheight{0.15in}
\addtolength\textfloatsep{-0.5em}
\addtolength\intextsep{-0.2em}

\def\setstretch#1{\renewcommand{\baselinestretch}{#1}}
\setstretch{0.99}
\addtolength{\parskip}{-0.3pt}
}


\title{Pixelated Butterfly: Simple and Efficient Sparse Training for Neural Network Models}

\iftoggle{arxiv}{
  \usepackage{authblk}
  \author[$\dagger$]{Tri Dao\thanks{Equal contribution. Order determined by coin flip.}}
  \author[$\dagger$]{Beidi Chen\samethanks}
  \author[$\oplus$]{Kaizhao Liang}
  \author[$\diamond$]{Jiaming Yang}
  \author[$\S$]{Zhao Song}
  \author[$\ddagger$]{Atri Rudra}
  \author[$\dagger$]{Christopher R{\'e}}
  \affil[$\dagger$]{Department of Computer Science, Stanford University}
  \affil[$\oplus$]{SambaNova Systems, Inc}
  \affil[$\diamond$]{Department of Probability and Statistics, Peking University}
  \affil[$\S$]{Adobe Research}
  \affil[$\ddagger$]{Department of Computer Science and Engineering, University at Buffalo, SUNY\vspace{4pt}}
  \affil[ ]{\small{\texttt{\{trid,beidic\}@stanford.edu}, \texttt{kaizhao.liang@sambanovasystems.com}, \texttt{edwinyjmpku@gmail.com}, \texttt{zsong@adobe.com}, \texttt{atri@buffalo.edu}, \texttt{chrismre@cs.stanford.edu}}}
}{

\author{%
  Tri Dao\thanks{Equal contribution. Order determined by coin flip.}\, $^1$, Beidi
  Chen\samethanks\, $^1$, Kaizhao Liang $^2$, Jiaming Yang $^3$, Zhao Song $^4$,
  Atri Rudra $^5$, Christopher R\'{e} $^1$ \\
  $^1$ Department of Computer Science, Stanford University \\
  $^2$ SambaNova Systems, Inc \\
  $^3$ Department of Probability and Statistics, Peking University \\
  $^4$ Adobe Research \\
  $^5$ Department of Computer Science and Engineering, University at Buffalo, The State University of New York\\
  \texttt{\{trid,beidic\}@stanford.edu},
  \texttt{kaizhao.liang@sambanovasystems.com}, \\
  \texttt{edwinyjmpku@gmail.com}, \texttt{zsong@adobe.com}, \\ \texttt{atri@buffalo.edu}, \texttt{chrismre@cs.stanford.edu}
}

}

\newcommand{\fix}{\marginpar{FIX}}

\iftoggle{arxiv}{}{
\iclrfinalcopy %
}
\begin{document}

 
\maketitle
\begin{abstract}


Overparameterized neural networks generalize well but are expensive to train. Ideally, one would like to reduce their computational cost while retaining their generalization benefits. Sparse model training is a simple and promising approach to achieve this, but there remain challenges as existing methods struggle with accuracy loss, slow training runtime, or difficulty in sparsifying all model components.
The core problem is that searching for a sparsity mask over a discrete set of sparse matrices is difficult and expensive.
To address this, our main insight is to optimize over a continuous superset of sparse matrices with a fixed structure known as products of butterfly matrices.
As butterfly matrices are not hardware efficient, we propose simple variants of butterfly (block and flat) to take advantage of modern hardware.
Our method (Pixelated Butterfly) uses a simple fixed sparsity pattern based on flat block butterfly and low-rank matrices to sparsify most network layers (e.g., attention, MLP).
We empirically validate that Pixelated Butterfly is $3\times$ faster than butterfly and speeds up training to achieve favorable accuracy--efficiency tradeoffs.
On the ImageNet classification and WikiText-103 language modeling tasks, our sparse models train up to 2.5$\times$ faster than the dense MLP-Mixer, Vision Transformer, and GPT-2 medium with no drop in accuracy.



\end{abstract}


Reinforcement learning has achieved great success in areas such as Game-playing \citep{silver2018general,vinyals2019grandmaster}, robotics \cite{kober2013reinforcement}, large language models \citep{ouyang2022training}, etc.
However, due to safety concerns or physical limitations, in some real-world reinforcement learning problems, we must consider additional constraints that may influence the optimal policy and the learning process \citep{garcia2015comprehensive}.
% For example, a robotic arm must not take actions that may cause harm to itself or the environments.
A standard framework to handle such cases is the constrained Markov Decision Process (CMDP) \citep{altman1999constrained}.
Within the CMDP framework, the agent has to maximize
the expected cumulative reward while
obeying a finite number of constraints, which are usually in the form of expected cumulative cost criteria.

However, we are sometimes concerned with the problem with a continuum of constraints.
For example,
the constraints we meet might be time-evolving or subject to uncertain parameters, which
cannot be formulated as an ordinary CMDP
(see Examples \ref{Example_Time_Evolving} and  \ref{Example_Uncertain}).
In this paper we would study a generalized CMDP  
to address the above problem.  Because the constraints are not only infinite-number but also lie
in a continuous set,
the generalization is not trivial. Fortunately, we find that we can borrow the idea behind semi-infinite programming (SIP) \citep{remez1934determination, hettich1993semi} to deal with the semi-infinite constraints.
Accordingly, we propose \emph{semi-infinitely constrained Markov decision processes} (SICMDPs)
as a novel complement to the ordinary CMDP framework.
%More specifically,  an SICMDP model %, we consider 
%contains a continuum of constraints whereas an ordinary CMDP contains a finite number of constraints. 

%This generalization is natural but not trivial. However, we can brows the idea  
%The idea is quite natural and can be backtracked
%to the practice of extending linear programming to linear semi-infinite programming (LSIP) %\cite{remez1934determination, GobernaLSIO1998}.
%In addition, 
%As a complementary approach to the ordinary CMDP framework, 
%SICMDP can be used to model these problems  which cannot be described by a finite number of constraints
%that are not covered by .
%For example,
%the restrictions we consider can be time-evolving or subject to uncertain parameters
%, thus
%cannot be described by a finite number of constraints but a continuum of constraints 
%(see Examples \ref{Example_Time_Evolving} and  \ref{Example_Uncertain}).

We also present two reinforcement learning algorithms to solve SICMDPs called SI-CRL and SI-CPO, respectively.
SI-CRL is a model-based reinforcement learning algorithm designed for tabular cases, and SI-CPO is a policy optimization algorithm for non-tabular cases.
% and analyze its performance both theoretically and empirically.
The main challenge is that we need to deal with a continuum of constraints, thus reinforcement learning algorithms for ordinary CMDPs do not work anymore.
In SI-CRL, we tackle this difficulty by first transforming the reinforcement learning problem to an equivalent LSIP problem, which can then be solved using methods in the LSIP literature like the dual exchange methods \citep{Hu1990,reemtsen1998numerical}.
In SI-CPO, we resort to the idea of cooperative stochastic approximation developed in \cite{lan2020algorithms, wei2020comirror}.
As far as we know, we are the first to introduce tools from semi-infinitely programming (SIP) into the reinforcement learning community for solving constrained reinforcement learning problems.

% To the best of our knowledge, we are the first to apply tools from semi-infinitely programming (SIP) to solve reinforcement learning problems.
Furthermore, we give theoretical analysis for both SI-CRL and SI-CPO.
We decompose the error of SI-CRL into two parts: the statistical error from approximating the true SICMDP with an offline dataset and the optimization error due to the fact that the solution of the LSIP problem obtained by the dual exchange method is inexact.
On the optimization side, we show that the iteration complexity of SI-CRL is $O\left(\left\{\mathrm{diam}(Y)L\sqrt{|\gS|^2|\gA|m}/\left[(1-\gamma)\epsilon\right]\right\}^m\right)$.
On the statistical side, we show that the sample complexity of SI-CRL is $\widetilde O\left(\frac{|S|^2|A|^2}{\epsilon^2(1-\gamma)^3}\right)$ if the offline dataset is generated by a generative model, and $\widetilde O\left(\frac{|S||A|}{\nu_{\min} \epsilon^2(1-\gamma)^3}\right)$ if the dataset is generated by a probability measure $\nu$ as considered in \cite{chen2019information}.
Here $\widetilde O$ means that all logarithm terms are discarded.
For SI-CPO, things become a little more complicated because other than the statistical error and the optimization error, we also need to consider the function approximation error, which comes from imperfect policy parametrizations.
It is shown if the function approximation error can be controlled to $O(\epsilon)$ order, the iteration complexity of SI-CPO is $\widetilde{O}\left(\frac{1}{\epsilon^2(1-\gamma)^6}\right)$ and the sample complexity of SI-CPO is $\widetilde{O}(\frac{1}{\epsilon^4(1-\gamma)^{10}})$.
Here our iteration complexity bound is equivalent to a typical $\widetilde O(1/\sqrt{T})$ global convergence rate.

We perform a set of numerical experiments to illustrate the SICMDP model and validate our proposed algorithms.
Specifically, we examine two numerical examples, namely the discharge of sewage and ship route planning.
Through the discharge of sewage example, we show the advantage of the SICMDP framework over the CMDP baseline obtained by naive discretization in modeling realistic sequential decision-making problems.
Moreover, we demonstrate the effectiveness of the SI-CRL and SI-CPO algorithms in such tabular environments. 
In the ship route planning example, we illustrate the benefits of the SICMDP framework and the ability of the SI-CPO algorithm to address complex continuous control tasks involving continuous state spaces with modern deep reinforcement learning techniques.

% In summary, our contributions are listed as follows.
% First, we present the SICMDP model, which can be viewed as a generalization of the ordinary CMDP model.
% Second, we propose an algorithm to perform reinforcement learning for SICMDPs, which is called SI-CRL, and we believe that we are the first to apply tools from SIP
% to solve reinforcement learning problems.
% Third, we give a theoretical analysis of SI-CRL and identify both its sample complexity and iteration complexity.
% In addition, we perform numerical experiments to illustrate the SICMDP model and validate the SI-CRL algorithm.
% \{This paragraph can be removed!!! \}






\section{Theory}
In this section, we give guarantees on our grid-based approach. Suppose there is some underlying distribution $\mathcal{P}$ with corresponding density function $p : \mathbb{R}^d \rightarrow \mathbb{R}_{\ge 0}$ from which our data points $X_{[n]} = \{x_1,...,x_n\}$ are drawn i.i.d. We show guarantees on the density estimator based on the grid cell counts.

We need the following regularity assumptions on the density function. The first ensures that the density function has compact support with smooth boundaries and is lower bounded by some positive quantity, and the other ensures that the density function has smoothness. These are standard assumptions in analyses on density estimation e.g. \cite{gine2002rates,jiang2017uniform,chen2017tutorial,singh2009adaptive}.
\begin{assumption}\label{assumption1}
$p$ has compact support $\mathcal{X} \in \mathbb{R}^d$ and there exists $\lambda_0, r_0, C_0 > 0$ such that $p(x) \ge \lambda_0$ for all $x \in \mathcal{X}$ and $\text{Vol}(B(x, r) \cap \mathcal{X}) \ge C_0 \cdot \text{Vol}(B(x, r))$ for all $x \in \mathcal{X}$ and $0 < r \le r_0$, where $B(x, r) := \{x' \in \mathbb{R}^d: |x-x'| \le r\}$.
\end{assumption}
\begin{assumption}\label{assumption2}
$p$ is $\alpha$-Hölder continuous for some $0 < \alpha \le 1$: i.e. there exists $C_\alpha > 0$ such that $|p(x) - p(x')| \le C_\alpha \cdot |x - x'|^\alpha$ for all $x, x' \in \mathbb{R}^d$.
\end{assumption}

We now give the result, which says that for $h$ sufficiently small depending on $p$ (if $h$ is too large, then the grid is too coarse to learn a statistically consistent density estimator), and $n$ sufficiently large, there will be a high probability finite-sample uniform bound on the difference between the density estimator and the true density. The proof can be found in the Appendix.
\begin{theorem}\label{theorem}
Suppose Assumption~\ref{assumption1} and~\ref{assumption2} hold. Then there exists constants $C, C_{1} > 0$ depending on $p$ such that the following holds.
Let $0 < \delta < 1$, $0 < h < \text{min}\{\left(\frac{\lambda_0}{2\cdot C_\alpha}\right)^{1/\alpha}, r_0\}$, $nh^d \ge C_1$. Let $\mathcal{G}_h$ be a partitioning of $\mathbb{R}^d$ into grid cells of edge-length $h$ and for $x \in \mathbb{R}^d$. Let $G(x)$ denote the cell in $\mathcal{G}_h$ that $x$ belongs to.  Then, define the corresponding density estimator $\widehat{p}_h$ as:
\begin{align*}
    \widehat{p}_h(x) := \frac{|X_{[n]} \cap G(x)|}{n\cdot h^d}.
\end{align*}
Then, with probability at least $1 - \delta$:
\begin{align*}
    \sup_{x \in \mathbb{R}^d} |\widehat{p}_h(x)  - p(x)| \le C\cdot \left( h^\alpha + \frac{\sqrt{\log(1/(h\delta)}}{\sqrt{n\cdot h^d}} \right).
\end{align*}
\end{theorem}


\begin{remark}
In the above result, choosing $h \approx n^{-1/(2\alpha+d)}$ optimizes the convergence rate to $\tilde{O}(n^{-\alpha/(2\alpha+d)})$, which is the minimax optimal convergence up to logarithmic factors for the density estimation problem as established by Tsybakov \cite{tsybakov1997nonparametric,tsybakov2008introduction}.
\end{remark}
In other words, the grid-based approach statistically performs at least as well as any estimator of the density function, including the density estimator used by MeanShift. It is worth noting that while our results only provide results for the density estimation portion of MeanShift++ (i.e. the grid-cell binning technique), we prove the near-minimax optimality of this estimation. This implies that the information contained in the density estimation portion serves as an approximately sufficient statistic for the rest of the procedure, which behaves similarly to MeanShift, which operates on another, also nearly-optimal density estimator. Thus, existing analyses of MeanShift e.g. \cite{arias2016estimation,chen2015convergence,xiang2005convergence,li2007note,ghassabeh2015sufficient,ghassabeh2013convergence,subbarao2009nonlinear} can be adapted here; however, it is known that MeanShift is very difficult to analyze \cite{dasgupta2014optimal} and a complete analysis is beyond the scope of this paper.

\vspace{-0.1cm}
\section{Butterfly matrices and Pixelated Butterfly}
\label{sec:butterfly}



Butterfly matrices~\citep{parker1995random, dao2019learning} are expressive
and theoretically efficient.
As they contain the set of sparse matrices, we choose to search for the sparsity
pattern in this larger class due to their fixed sparsity structure.
However, there are three technical challenges.
We highlight them here along with our approaches to address them:
\begin{enumerate}[leftmargin=*,nosep,nolistsep]
  \item Slow speed: butterfly matrices are not friendly to modern hardware as their
  sparsity patterns are not block-aligned, thus are slow.
  We introduce a variant of butterfly matrices, \emph{block butterfly}, which operate at the block level, yielding
  a block-aligned sparsity pattern.
  \item Difficulty of parallelization: the sequential nature of butterfly matrices as products
  of many factors makes it hard to parallelize the multiplication.
  We propose another class of matrices, \emph{flat butterfly} matrices, that are
  the first-order approximation of butterfly with residual connections.
  Flat butterfly turns the product of factors into a sum, facilitating parallelization.
  \item Reduced expressiveness of flat butterfly: even though flat butterfly
  matrices can approximate butterfly matrices with residual connections, they are
  necessarily high-rank and cannot represent low-rank matrices~\citep{udell2019big}.
  We propose to add a low-rank matrix (that is also block-aligned) to flat
  butterfly to increase their expressiveness.
\end{enumerate}
Combining these three approaches (flat \& block butterfly + low-rank), our
proposal (Pixelated Butterfly) is a very simple method to train sparse networks.

\subsection{Block Butterfly Matrices}
\label{sec:block_butterfly}




We propose a block version of butterfly matrices, which is more
hardware-friendly than the regular butterfly.
The regular butterfly matrices~\citet{dao2019learning, dao2020kaleidoscope} will be a special case of block butterfly with
block size $b = 1$.
We omit $b$ in the notation if $b = 1$.
\begin{definition} \label{def:bfactor}
  A \textbf{block butterfly factor} (denoted as $\vB_{k, b}$) of size $kb$ (where $k \ge 2$) and block size $b$ is a matrix of the form
    \(
        \vB_{k, b} = \begin{bmatrix}
            \vD_1 & \vD_2 \\ \vD_3 & \vD_4
        \end{bmatrix}
    \)
    where each $\vD_i$ is a $\frac{k}{2} \times \frac{k}{2}$ block diagonal
    matrix of block size $b$ of the form
    $\mathrm{diag} \left( D_{i, 1}, \dots, D_{i, k/2} \right)$ where
    $D_{i, j} \in \mathbb{R}^{b \times b}$.
    We restrict $k$ to be a power of 2.
\end{definition}
\begin{definition}
  A \textbf{block butterfly factor matrix} (denoted as $\vB_{k}^{(n, b)}$) of size $nb$ with stride $k$ and
     block size $b$ is a block diagonal matrix
     of $\frac{n}{k}$ (possibly different) butterfly factors of size $kb$ and
     block size $b$:
     \[
        \vB_{k}^{(n, b)} = \mathrm{diag} \left( \left[ \vB_{k, b} \right]_1, \left[ \vB_{k, b} \right]_2, \hdots, \left[ \vB_{k, b} \right]_\frac{n}{k} \right)
     \]
\end{definition}

\begin{definition} \label{def:bmatrix}
    A \textbf{block butterfly matrix} of size $nb$ with block size $b$ (denoted as $\vB^{(n, b)}$) is a matrix that can be expressed as a product of butterfly factor matrices:
    \(
        \vB^{(n, b)} = \vB_n^{(n, b)} \vB_{\frac{n}{2}}^{(n, b)} \hdots \vB_2^{(n, b)}.
    \)
    Define $\B_b$ as the set of all matrices that can be expressed in the form $\vB^{(n, b)}$ (for some $n$).
\end{definition}

\begin{figure}
\vspace{-1.2cm}
	\begin{center}
		\begin{tabular}{c}
			\includegraphics[width=0.87\linewidth]{figs/flat_block_butterfly.pdf}
		\end{tabular}
	\end{center}
	\captionsetup{font=small}
		\vspace{-0.8cm}
	\caption{Visualization of Flat, Block, and Flat Block butterfly.}
	\label{fig:tradeoff}
	\vspace{-0.3cm}
\end{figure}

\subsection{Flat butterfly matrices}
\label{sec:flat_butterfly}
In most applications of butterfly matrices to neural networks, one multiplies
the $O(\log n)$ butterfly factors.
However, this operation is hard to be efficiently implemented on parallel hardware (e.g., GPUs) due to
the sequential nature of the operation\footnote{Even with a very specialized
  CUDA kernel, butterfly matrix multiply ($O(n \log n)$ complexity) is only
faster than dense matrix multiply ($O(n^2)$ complexity) for large values of $n$
(around 1024)~\citep{dao2019learning}.}.
We instead propose to use a sum of butterfly factors that can approximate the
products of the factors.
This sum of factors results in one sparse matrix with a fixed sparsity pattern,
which yields up to 3$\times$ faster multiplication on GPUs (\cref{sec:appx_benchmark}).

Residual connections have been proposed to connect the butterfly
factors~\citep{vahid2020butterfly}.
We show that residual products of butterfly matrices have a first-order
approximation as a sparse matrix with a fixed sparsity.
Let $M$ be a matrix in the set of butterfly matrices $\B$.
In residual form, for some $\lambda \in \mathbb{R}$:
\begin{equation}
  \label{eq:residual_butterfly}
  M = (I + \lambda \vB_n^{(n)}) (I + \lambda \vB_{n/2}^{(n)}) \dots (I + \lambda \vB_2^{(n)}).
\end{equation}
Note that this form can represent the same matrices in the class of butterfly
matrices $\vB$, since any $\vB_k^{(n)}$ contains the identity matrix $I$.

Assuming that $\lambda$ is small, we can expand the residual and collect the
terms\footnote{We make the approximation rigorous in \cref{sec:analysis}.}:
\begin{equation*}
  M = I + \lambda (\vB_2^{(n)} + \vB_{4}^{(n)} + \dots + \vB_n^{(n)}) + \tilde{O}(\lambda^2).
\end{equation*}
\begin{definition}
  \label{def:flat_butterfly}
  \emph{Flat butterfly} matrices of maximum stride $k$ (for $k$ a power of 2)
  are those of the form $I + \lambda (\vB_2^{(n)} + \vB_{4}^{(n)} + \dots + \vB_k^{(n)})$.
\end{definition}
Flat butterfly matrices of maximum stride $n$ are the first-order approximation
of butterfly matrices in residual form (\cref{eq:residual_butterfly}).
Notice that flat butterfly of maximum stride $k$ are sparse matrices with $O(n \log k)$ nonzeros with a fixed
sparsity pattern, as illustrated in~\cref{fig:tradeoff}.
We call this sparsity pattern the \emph{flat butterfly} pattern.

\emph{Flat block butterfly} matrices are block versions of flat butterfly in~\cref{sec:flat_butterfly} (shown in~\cref{fig:tradeoff}).
We empirically validate that flat block butterfly matrices are up to 3$\times$
faster than block butterfly or regular butterfly (\cref{sec:appx_benchmark}).

Since flat butterfly matrices approximate the residual form of butterfly
matrices, they have high rank if $\lambda$ is small (\cref{sec:analysis}).
This is one of the motivations for the addition of the low-rank term in our
method.

\subsection{Pixelated Butterfly: Flat Block Butterfly + Low-rank for Efficient Sparse Training}
\label{sec:method}

We present Pixelated Butterfly, an efficient sparse model with a simple and fixed sparsity
pattern based on butterfly and low-rank matrices.
Our method targets GEMM-based neural networks, which are networks whose computation is dominated by general matrix multiplies (GEMM), such as Transformer and MLP-Mixer.
As a result, we can view the network as a series of matrix multiplies.

Given a model schema (layer type, number of layers, matrix dimension) and a
compute budget, Pixelated Butterfly has three steps: compute budget allocation per layer,
sparsity mask selection from the flat butterfly pattern, and model
sparsification.
We describe these steps in more details:
\begin{enumerate}[leftmargin=*,nosep,nolistsep]
  \item \textbf{Compute budget allocation}: based on our cost model
  (\cref{app:problem_formulation}), given the layer type, number of layers, and
  matrix dimension, we can find the density (fraction of nonzero weights) of
  each layer type to minimize the projected compute cost.
  Continuing our goal for a simple method, we propose to use a simple rule of
  thumb: allocate sparsity compute budget proportional to the compute fraction
  of the layer.
  For example, if the MLP layer and attention layers are projected to takes 60\%
  and 40\% the compute time respectively, then allocate 60\% of the sparsity compute budget
  to MLP and 40\% to attention.
  We verify in \cref{sec:appx_method_details} that this simple rule of thumb
  produces similar results to solving for the density from the cost model.

  \item \textbf{Sparsity mask selection}: given a layer and a sparsity compute budget for
  that layer, we use one-quarter to one-third of the budget for the low-rank
  part as a simple rule of thumb.
  We pick the rank as a multiple of the smallest supported block
  size of the device (e.g., 32) so that the low-rank matrices are also block-aligned.
  The remaining compute budget is used to select the sparsity mask from the flat
  block butterfly sparsity pattern: we choose the butterfly block size as the
  smallest supported block size of the device (e.g., 32), and pick the maximum
  stride of the flat block butterfly (\cref{def:flat_butterfly}) to fill up the
  budget.

  \item \textbf{Model sparsification}: The resulting sparse model is simply a
  model whose weights or attention follow the fixed sparsity mask chosen in
  step 2, with the additional low-rank terms (rank also chosen in step 2).
  In particular, we parameterize each weight matrix\footnote{We describe how
  to add sparse and low-rank for attention in \cref{sec:appx_method_details}} as:
  $W = \gamma B + (1 - \gamma) U V^\top$, where $B$ is a flat block butterfly
  matrix (which is sparse), $U V^\top$ is the low-rank component, and $\gamma$
  is a learnable parameter.
  We train the model from scratch as usual.
\end{enumerate}

Our method is very simple, but competitive with more complicated procedures that
search for the sparsity pattern (\cref{sec:appx_ntk_algorithm}).
We expect more sophisticated techniques (dynamic sparsity, a better approximation
of butterfly) to improve the accuracy of the method.


% \vspace{-0.20in}
%\subsection{Analysis:}


\textbf{Use of Multiple Projection Heads:} The use of different projection heads for each view on OpenImages classification gives us a boost of $1.1$ mAP on Obj-Obj+Dilate crop. Pre-training on COCO and finetuning on VOC dataset for object-detection task gives a boost of $0.4$ mAP. Hence using multiple projection heads results in a consistent improvement. 

\textbf{Varying Dilation Parameter:} Table 3 (appendix) shows the effect of varying the dilation parameter. A sweet spot exists at a moderate dilation value of $\delta=0.1$ for COCO object detection. 

% \textbf{Computational Cost:} BING adds negligible time to the pre-training. Generating object proposals takes ~29 mins for the full OHMS dataset (one-time cost) and ~16 mins for COCO. Instead of pre-generating, adding the BING operator to the data loader pipeline has a trivial overhead (+$0.1\%$). %As an example, the wall-clock time taken for 1 epoch of training is 1'46'' for the Dense-CL baseline and 1'45'' for our method.
% \textbf{}



%Between two views, we measure the number of common pixels; and then measure the fraction of these common pixels that overlap with a ground truth bounding box (object). We find that this fraction for COCO is $99\%$ for object-scene crops and $92.1\%$ for the scene-scene crop. In the case of OpenImages-Subset, the numbers are, respectively, $99.1\%$ and $87.3\%$. This is another way of seeing that OpenImages-Subset can benefit more from object-scene crops, borne out by the numbers in Tables \ref{tab:ssl_comparison_classification} and \ref{tab:coco_detection}. 


% \as{Shlok: could you please make this description a little better and clear?}
% We find the overlapping pixels between two crops ($C_{int} = C_1 \cap C_2$). Next we calculate intersection of $C_{int}$  with the most overlapping ground truth object ($O$) and calculate the score $\frac{C_{int} \cap O}{C_{int}}$ for each image and average it. 
% To do this, we calculate the \% intersection of the most overlapping ground truth object with the inter
% Next we try to find the probability of an actual ground truth object co-occuring in between two crops. We find  object-overlap between both Scene-Object crops and Scene-Scene crops. To do this we firstly calculate the overlapping region between two crops. Overlapping region is the area of overlap between two crops before the resize operation. Then for all the ground truth objects present in the original image  we find the object with maximum overlap in the overlapping region. Intuitively for a object to have high overlap, the object should be present in both the crops. 

% Similarly instead of taking an crop with maximum overlap we calculate average of all the crops that are present in the image. We find this average probability to be 65.12 \% for Object-Scene crop and 73.47 \% for Scene-Scene crop. 
% This is consistent with the findings of the InfoMin \cite{tian2020contrastive} that there is a tradeoff between how much information views can share.  

% Similarly in the case of OpenImages we can see from Fig \ref{fig:radius_openimages} that as we increase the radius of the object-object crops the performance firstly increases and then decreases, suggesting there is a sweet point on mutual information on OpenImages dataset as well.
% \\

% \textbf{Performance on 5 classes per image images?}









\newcommand{\twomoons}{{\tt Twomoons}}
\newcommand{\gauss}{{\tt Gauss}}
\newcommand{\sculpture}{{\tt Sculpture}}
\newcommand{\baseline}{{\tt Baseline}}
\newcommand{\MM}{{\tt MsgPassing}}
\newcommand{\blackboard}{{\tt Blackboard}}
\newcommand{\ncut}{\text{ncut}}
\newcommand{\chensays}[2][]{\textcolor{blue} {\textsc{Jiecao #1:} \emph{#2}}}

\section{Experiments}
In this section we present experimental results for  graph clustering in the message passing and blackboard models. We will compare the following three algorithms. (1) \baseline: each site sends all the data to the coordinator directly; (2) \MM: our algorithm in the message passing model (Section~\ref{sec:gcmessage}); (3) 
\blackboard: our algorithm in  the blackboard model (Section~\ref{sec:bb}).


%Since both of our algorithms are crucially based on the use of spectral scarification, our main focus in the experiments is to investigate to what extend the quality of the spectral clustering algorithms will be affected by using spectral sparsification, the saving of communication costs by using spectral sparsificaion, ...
%
%
%The goal of this experiment is not to demonstrate the effectiveness of the spectral clustering algorithm. We mainly want to investigate the following, 
%\begin{itemize}
%\item to what extend the quality of clustered results will be affected by using spectral sparsification.
%\item saving of communication costs by using spectral sparsifier.
%\item the affect of constants in algorithms of the message passing/blackboard model.
%\end{itemize}
%
%
%\subsection{The Setup}
%\paragraph{Reference Algorithms}
%We compare different algorithms in our experiment.

%Note that we can also run \MM~ in the blackboard model.

Besides giving the visualized results of these algorithms on various datasets, we also measure the qualities of the results via the {\em normalized cut}, defined as 
\[
\ncut(A_1, \ldots, A_{k}) = \frac{1}{2}\sum_{i\in[k]}\frac{w(A_i, V\backslash A_i)}{\vol(A_i)},
\]
 which is a standard objective function to be minimized for spectral clustering algorithms. 
%We will compare the communication costs of these algorithms in different settings.

%We also compare the total communication costs of different algorithms/models. As the unit does not matter in our case, we normalize all communication costs by the cost of \baseline.  Whenever possible, we will visualize the clustered results.

We implemented the algorithms using multiple languages, including Matlab, Python and C++. Our experiments were conducted on an IBM NeXtScale nx360 M4 server, which is equipped with 2 Intel Xeon E5-2652 v2 8-core processors, 32GB RAM and 250GB local storage.


\subsection{Datasets.}
We test the algorithms in the following real and synthetic datasets, which is visualized in \figref{visualization}.


\begin{figure}[h]
     \centering
     \subfigure[\twomoons]{\includegraphics[width=0.23\textwidth]{twomoons-14000-original.png}\label{fig:twomoons}}
     ~~
     \subfigure[\gauss]{\includegraphics[width=0.23\textwidth]{gauss-10000-original.png}\label{fig:gauss}}
     ~~
     \subfigure[\sculpture]{\includegraphics[width=0.13\textwidth,height=0.16\textwidth]{sculpture-11680-original.jpg}\label{fig:sculpture}}
     \caption{Visualization of the datasets for our experiments.}
     \label{fig:visualization}
\end{figure}



\vspace{-1mm}
\begin{itemize}
\item \twomoons : this dataset contains $n=14,000$ coordinates in $\mathbb{R}^2$. We consider each point to be a vertex. For any two vertices $u, v$, we add an edge with weight $w(u,v) = \exp\{-\|u-v\|_2^2/\sigma^2\}$ with $\sigma = 0.1$ when one vertex is among the $7000$-nearest points of the other.  This construction results in a graph with about $110,000,000$ edges.

\item  \gauss : this dataset contains $n = 10,000$ points in $\mathbb{R}^2$. There are $4$ clusters in this dataset, each generated using a Gaussian distribution. We construct a complete graph as the similarity graph.  For any two vertices $u, v$, we define the weight $w(u,v) = \exp\{-\|u-v\|_2^2/\sigma^2\}$ with $\sigma = 1$. The resulting graph has about $100,000,000$ edges.

\item \sculpture : a photo of \textit{The Greek Slave}~\footnote{Available in e.g., \url{http://artgallery.yale.edu/collections/objects/14794}}. We use an $80\times 150$ version of this photo where each pixel is viewed as a vertex. To construct a similarity graph, we map each pixel to a point in $\mathbb{R}^5$, i.e., $(x, y, r, g, b)$, where the latter three coordinates are the RGB values. For any two vertices $u, v$, we  put an edge between $u, v$ with weight $w(u,v) = \exp\{-\|u-v\|_2^2/\sigma^2\}$ with $\sigma = 0.5$ if one of $u, v$ is among the $5000$-nearest points of the other. This results in a graph with about $70,000,000$ edges.
\end{itemize}
\vspace{-1mm}
In the distributed model edges are randomly partitioned across $s$ sites. 

%\vspace{-1.5mm}



\subsection{Results on clustering quality}
%{\em Quality.} \
\begin{figure*}[ht]
     \centering
     \subfigure[\baseline]{\includegraphics[width=0.2\textwidth]{twomoons-14000-original-clustered.png}\label{fig:twomoons-clustered-original}}
     \subfigure[\MM]{\includegraphics[width=0.2\textwidth]{twomoons-14000-sparsify-clustered-15.png}\label{fig:twomoons-clustered-sparsify}}
     \subfigure[\blackboard]{\includegraphics[width=0.2\textwidth]{twomoons-14000-chain-clustered.png}\label{fig:twomoons-clustered-chain}}
     \caption*{\twomoons, $k = 2$;}

\subfigure[\baseline]{\includegraphics[width=0.2\textwidth]{gauss-10000-original-clustered.png}\label{fig:gauss-clustered-original}}
     \subfigure[\MM]{\includegraphics[width=0.2\textwidth]{gauss-10000-sparsify-clustered-15.png}\label{fig:gauss-clustered-sparsify}}
     \subfigure[\blackboard]{\includegraphics[width=0.2\textwidth]{gauss-10000-chain-clustered.png}\label{fig:gauss-clustered-chain}}
     \caption*{\gauss, $k = 4$}


     \subfigure[\baseline]{\includegraphics[width=0.2\textwidth,height=0.2\textwidth]{sculpture-11680-original-clustered.png}\label{fig:sculpture-clustered-original}}  
     \subfigure[\MM]{\includegraphics[width=0.2\textwidth,height=0.2\textwidth]{sculpture-11680-sparsify-clustered-15.png}\label{fig:sculpture-clustered-sparsify}}
     \subfigure[\blackboard]{\includegraphics[width=0.2\textwidth,height=0.2\textwidth]{sculpture-11680-chain-clustered.png}\label{fig:sculpture-clustered-chain}}
     \caption*{\sculpture, $k = 3$. }


     
     \caption{Visualization of the results on \twomoons, \gauss\ and \sculpture. In the message passing model each site samples $5 n$ edges; in the blackboard model all sites jointly sample $10n$ edges (in \twomoons~ and \gauss) or $20n$ edges (in \sculpture) and the chain has length $18$. $s = 15$.}
     \label{fig:quality-1}
\end{figure*}

We visualize the clustered results for 
the \twomoons, \gauss\ and \sculpture\ in Figure~\ref{fig:quality-1}.
% and visualize the clustered results for \gauss\ and \sculpture in Figure~\ref{fig:quality-2}.
It can be seen that \baseline, \MM\ and \blackboard\ give results of very similar qualities.  For simplicity, here we only present the visualization for $s=15$. Similar results were observed when we varied the values of $s$.  
%\he{To Qin: Do you plan to have two titles (Results \& Quality)?}


% \begin{figure*}[h]
%      \centering
% \subfigure[\baseline]{\includegraphics[width=0.3\textwidth]{gauss-10000-original-clustered.png}\label{fig:gauss-clustered-original}}
%      \subfigure[\MM]{\includegraphics[width=0.3\textwidth]{gauss-10000-sparsify-clustered-15.png}\label{fig:gauss-clustered-sparsify}}
%      \subfigure[\blackboard]{\includegraphics[width=0.3\textwidth]{gauss-10000-chain-clustered.png}\label{fig:gauss-clustered-chain}}
%      \caption*{\gauss, $k = 4$}


%      \subfigure[\baseline]{\includegraphics[width=0.2\textwidth]{sculpture-11680-original-clustered.png}\label{fig:sculpture-clustered-original}}  
%      \subfigure[\MM]{\includegraphics[width=0.2\textwidth]{sculpture-11680-sparsify-clustered-15.png}\label{fig:sculpture-clustered-sparsify}}
%      \subfigure[\blackboard]{\includegraphics[width=0.2\textwidth]{sculpture-11680-chain-clustered.png}\label{fig:sculpture-clustered-chain}}
%      \caption*{\sculpture, $k = 3$. }

%      \caption{Visualization of results on \gauss\ and \sculpture; in the message passing model each site samples $5 n$ edges; in the blackboard model all sites jointly sample $10n$ (in \gauss) or $20n$ (in \sculpture) edges and the chain has length $18$.}
%      \label{fig:quality-2}
% \end{figure*}


We also compare the normalized cut (ncut) values of the clustering results of different algorithms.  The results are presented in Figure \ref{fig:quality}. In all datasets, the ncut values of different algorithms are very close. The ncut value of \MM\ slightly decreases when we increase the value of $s$, while the ncut value of \blackboard\ is independent of $s$.
%We comment that in general, it is difficult to compare \MM\ and \blackboard\ directly because they are affected by different parameters.


\begin{figure*}[!ht]
  \centering
  \subfigure[\twomoons]{\includegraphics[width=0.33\textwidth]{twomoons-14000-ncut.png}\label{fig:twomoons-quality}}\hspace*{-1.1em}
  \subfigure[\gauss]{\includegraphics[width=0.31\textwidth]{gauss-10000-ncut.png}\label{fig:gauss-quality}}\hspace*{-1.1em}
  \subfigure[\sculpture]{\includegraphics[width=0.31\textwidth]{sculpture-11680-ncut.png}\label{fig:sculpture-quality}}\hspace*{-1.1em}
  \subfigure{\includegraphics[width=0.14\textwidth]{legend.png}}
     \caption{Comparisons on normalized cuts. In the message passing model, each site samples $5n$ edges; in each round of the algorithm in the blackboard model, all sites jointly sample $10n$ edges (in \twomoons~and \gauss) or $20n$ edges (in \sculpture) edges and the chain has length $18$.}
     \label{fig:quality}
\end{figure*}

%\textcolor{red}{To Jiecao: Can you put the color lines indicating baseline, message passing, and blackboard within one row in Pic 2? Withthis we can save some space.}

%\vspace{-1.5mm}

\subsection{Results on communication costs} 
\begin{figure*}[!ht]
     \centering
     \subfigure[\twomoons]{\includegraphics[width=0.3\textwidth]{twomoons-14000-communication.png}\label{fig:twomoons-communication}}
     \subfigure[\gauss]{\includegraphics[width=0.3\textwidth]{gauss-10000-communication.png}\label{fig:gauss-communication}}
     \subfigure[\sculpture]{\includegraphics[width=0.3\textwidth]{sculpture-11680-communication.png}\label{fig:sculpture-communication}}


     \subfigure[\twomoons]{\includegraphics[width=0.32\textwidth]{twomoons-14000-communication-2.png}\label{fig:twomoons-communication-2}}
     \subfigure[\gauss]{\includegraphics[width=0.32\textwidth]{gauss-10000-communication-2.png}\label{fig:gauss-communication-2}}
     \subfigure[\sculpture]{\includegraphics[width=0.32\textwidth]{sculpture-11680-communication-2.png}\label{fig:sculpture-communication-2}}
     \caption{Comparisons on communication costs. In the message passing model, each site samples $5n$ edges; in each round of the algorithm in the blackboard model, all sites jointly sample $10n$ (in \twomoons~and \gauss) or $20n$ (in \sculpture) edges and the chain has length $18$. }
     \label{fig:communication}
\end{figure*}

We compare the communication costs of different algorithms in Figure \ref{fig:communication}. We observe that while achieving similar clustering qualities as \baseline, both \MM\ and \blackboard\ are significantly more communication-efficient (by one or two orders of magnitudes in our experiments). We also notice that the value of $s$ does not affect the communication cost of \blackboard, while the communication cost of \MM\ grows almost linearly with $s$; when $s$ is large, \MM\ uses significantly more communication than \blackboard. These confirm our theory.  %In Figure~\ref{fig:mm-const} and Figure~\ref{fig:blackboard-const}   in Appendix~\ref{sec:parameters} we present how the performance of \MM\ and \blackboard\ are affected by their parameters.

%
%
%\vspace{-1.5mm}
%\paragraph{Summary.}  From our experimental results we conclude that \MM\ and \blackboard\ achieve similar clustering quality as the native algorithm \baseline, while significantly reduce the communication cost.  When the number of sites is large, \blackboard\ is more communication efficient than \MM, as predicted by our theory.



\subsection{Parameters in \MM\ and \blackboard}
\label{sec:parameters}

Figure \ref{fig:mm-const} shows in \MM how the value of ncut is affected by the number of sites and the number of edges sampled in each site. 
Here, each site samples $cn$ edges. 
When $c=3$ and $s=1$, the ncut value diverges in all datasets. This is because with such a small $c$, the algorithm does not generate a valid sparsifier. In general, increasing $c$ or $s$ will slightly decrease the ncut value. But once they are above some thresholds, the ncut values of \MM\ and \baseline\ become very close.

Figure \ref{fig:blackboard-const} shows in \blackboard  how the ncut value is affected by the number of iterations and the number of edges sampled. When the number of iterations is set to be $5$, ncut values diverge in all datasets. This is because we cannot expect to generate a valid sparsifier by using such few iterations. It can be seen from \ref{fig:bb-gauss-constant} that for a fixed $c$, performing more iterations will help to reduce ncut values. From the same figure, one can also conclude that for fixed iterations, increasing $c$ also helps to reduce the ncut values.



\begin{figure*}[h!t]
     \centering
     \subfigure[\twomoons]{\includegraphics[width=0.3\textwidth]{twomoons-c.png}\label{fig:mm-twomoons-constant}}
     \subfigure[\gauss~dataset]{\includegraphics[width=0.3\textwidth]{gauss-c.png}\label{fig:mm-gauss-constant}}
     \subfigure[\sculpture]{\includegraphics[width=0.3\textwidth]{sculpture-c.png}\label{fig:mm-sculpture-constant}}
     \caption{The pictures above show the $\ncut$ values with respect to the values of $c$ and $s$ for the \MM\ algorithm. Here  
 each site samples $c n$ edges.}
     \label{fig:mm-const}
\end{figure*}


\begin{figure*}[h!t]
     \centering
     \subfigure[\twomoons]{\includegraphics[width=0.3\textwidth]{twomoons-iter.png}\label{fig:bb-twomoons-constant}}
     \subfigure[\gauss]{\includegraphics[width=0.3\textwidth]{gauss-iter.png}\label{fig:bb-gauss-constant}}
     \subfigure[\sculpture]{\includegraphics[width=0.3\textwidth]{sculpture-iter.png}\label{fig:bb-sculpture-constant}}
     \caption{The pictures above show how the $\ncut$ values are affected by the number of iterations and the value of $c$ for the \blackboard\ algorithm. Here 
all sites jointly sample $c n$ edges. }
     \label{fig:blackboard-const}
\end{figure*}






The industry standard for pose edition is to create rigs, a collection of pieces of software designed to manipulate a character's skeleton. The rig describes the skeleton's bones, how they relate to each other, are constrained in their possible motion and are deformed. These rules are loosely specified and creating a good rig requires a detailed understanding of physics and anatomy, as well as technical and artistic skills. Rigging is thus a time consuming task even for experienced animators, and even more so in large scale productions which often require a different in-depth rig for each character in the cast.
Previous work has helped alleviate this difficulty by providing efficient tools to speed up/and or ease the rigging process, relying on inverse kinematics or data-driven methods.
\subsection{Character pose design}
\subsubsection{Inverse Kinematics (IK)}
IK solvers are a family of methods commonly used in robotics, engineering and computer graphics, in which the parameterization of a kinematic chain is determined from the position of its end effector.
They are a staple tool in pose design software, ensuring the respect of elementary constraints during pose edition. Their de-facto role is to guarantee the length of the limbs, and in some cases to enforce the orientation angle range of a joint.
Many IK solutions have been studied over the years \cite{aristidou_inverse_2018}; usually revolving around approximated linearizations or heuristics. 

Numerical methods require a set of iterations to achieve a satisfactory solution formulated by a cost function to be minimized.
IK solutions can generally be divided into three sub-categories: Jacobian \cite{Siciliano_Handbook_Robot_2007}, Newtonians \cite{cohen_ik_1996} and Heuristics. Most software implement heuristic methods such as Cyclic Coordinate Descent (CCD) \cite{wang_ccd_1991} or 
Forward-Backward Reaching IK (FABRIK) \cite{aristidou_fabrik:_2011} due to their simplicity and extensibility. 

The main drawback of 
these solvers is that they manipulate kinematic chains without taking into account many morphological aspects that make a pose more or less plausible. They offer a first level of help to users but are not sufficient to guarantee a realistic pose. Many joints constraints are dependent on each other and require subjective, human-made approximations.

\subsubsection{Data-driven pose edition}
Data-driven methods offer promising opportunities to solve these approximations. Using real-life data can help in modelling the complex inter-dependencies of skeletons and providing users with smarter edition tools.
While it is still an early field of research, some solutions have been studied. Wu \etal \cite{wu_posing_2009} propose a method for natural character posing from a large motion database. It employs adaptive KD-clustering to select a representative frame from a database and sparse approximations to accelerate training and posing. 
Huang \etal in \cite{Huang_IK_MGDM_2017} present a method based on the formulation of multi-variate Gaussian distribution models (MGDMs), which learn the joint constraints of a kinematic skeleton from motion capture data. 

Some work has also been dedicated to finding new editing interfaces. \modify{}{Instead of the usual setup manipulating joints directly, Guay \etal \cite{guay_line_2013} articulate a framework based on the conceptual "line of action" which describes the overall pose dynamics. They provide a mathematical definition of the line of action, and a interface in which the software modifies the pose to follow a user-provided line. In the same line of though} Garcia \etal \cite{garcia_sketching_2019} propose \modify{a method transforming doodle of trajectories (position and orientation over time) }{a virtual reality-based interface where the user's hands motion (position and orientation over time) are transformed} into sequences of actions and then into detailed character animations using a dataset of parametrized motion clips automatically fitted to the trajectory. 

% ==> DL et Latent Space. 
\subsection{Neural modelling of human motion}
Neural networks have received a great amount of attention over the last decade and shown impressive result in modelling complex data. Human motion has not been spared and deep learning methods have proven their capability of generating realistic motion in a number of difficult cases. 

The literature in neural-based animation include example in user-controlled character navigation \cite{Holden2017} and interactions with the environment \cite{starke_neural_2019}. 
Holden \etal \cite{Holden2020} also show that neural networks can be used to replace parts of existing data-driven methods, improving their scalability potential.
More recently, some work has also focused on improving smaller parts of the animation pipeline rather than replacing it completely. Berson et al. \cite{berson_intuitive_2020} leverage neural networks to provide an interactive system to edit facial animation. 

% Wrap up
Data-driven IK and pose editing can relieve animators from time-consuming, back-and-forth pose adjustments by applying constraints extracted from real-world data. Recently, neural-network-based approaches have demonstrated their ability to model the intricacies of human motion while scaling to large amount of data and retaining a fast inference time. In this paper we seek to take advantage of these properties to create an efficient posing tool, intuitively usable even by a inexperienced user.

\begin{comment}
\begin{figure}
\includegraphics[width=\linewidth]{figs/beyond_tss_lesion.pdf}
\caption[]{End-to-End runtime lesion study of the entire MNIST dataset and the FMA featurized music dataset. Each of DROP's contributions provides a runtime improvement.}
\label{fig:beyond_lesion}
\end{figure}
\end{comment}



\section{Conclusion}
\label{sec:conclusion}

Advanced data analytics techniques must scale to rising data volumes. 
DR techniques offer a powerful toolkit when processing these datasets, with PCA frequently outperforming popular techniques in exchange for high computational cost. 
In response, we propose DROP, a new dimensionality reduction optimizer. 
DROP combines progressive sampling, progress estimation, and online aggregation to identify high quality low dimensional bases via PCA without processing the entire dataset by balancing the runtime of downstream tasks and achieved dimensionality. 
Thus, DROP provides a first step in bridging the gap between quality and efficiency in end-to-end DR for downstream \red{analytics}. 

%We revisit canonical operators for time series dimensionality reduction and the measurement study of~\cite{keogh-study}, and show that PCA is more effective than popular alternatives in the data mining literature often by a margin of over $2\times$ on average on gold-standard time series benchmark data sets with respect to output data dimension. More surprisingly, we empirically demonstrate that a small number of samples are sufficient to accurately characterize directions of maximum variance and obtain a high-quality low-dimensional transformation.




\subsubsection*{Acknowledgments}

We thank Laurel Orr, Xun Huang, Sarah Hooper, Sen Wu, Megan Leszczynski, and Karan Goel for their helpful discussions and feedback on early drafts of the paper.

We gratefully acknowledge the support of NIH under No.\ U54EB020405 (Mobilize), NSF under Nos.\ CCF1763315 (Beyond Sparsity), CCF1563078 (Volume to Velocity), and 1937301 (RTML); ONR under No.\ N000141712266 (Unifying Weak Supervision); ONR N00014-20-1-2480: Understanding and Applying Non-Euclidean Geometry in Machine Learning; N000142012275 (NEPTUNE); the Moore Foundation, NXP, Xilinx, LETI-CEA, Intel, IBM, Microsoft, NEC, Toshiba, TSMC, ARM, Hitachi, BASF, Accenture, Ericsson, Qualcomm, Analog Devices, the Okawa Foundation, American Family Insurance, Google Cloud, Salesforce, Total, the HAI-AWS Cloud Credits for Research program, the Stanford Data Science Initiative (SDSI), and members of the Stanford DAWN project: Facebook, Google, and VMWare. The Mobilize Center is a Biomedical Technology Resource Center, funded by the NIH National Institute of Biomedical Imaging and Bioengineering through Grant P41EB027060. The U.S.\ Government is authorized to reproduce and distribute reprints for Governmental purposes notwithstanding any copyright notation thereon. Any opinions, findings, and conclusions or recommendations expressed in this material are those of the authors and do not necessarily reflect the views, policies, or endorsements, either expressed or implied, of NIH, ONR, or the U.S.\ Government.
Atri Rudra’s research is supported by NSF grant CCF-1763481.


\bibliography{ref}
\bibliographystyle{plainnat}


\appendix
\newpage
\section{Problem Formulation}
\label{app:problem_formulation}

We formulate the problem of sparse model training as sparse matrix approximation with a simple hardware cost model (\cref{sec:problem_formulation}).





We first describe our simple cost model for sparse matrix multiplication to reflect the fact that parallel hardware such as GPUs are block-oriented~\citep{cook2012cuda,gray2017gpu}: accessing one single element from memory costs the same as accessing one whole block of elements.
We then formulate the sparse matrix approximation in the forward pass and the backward pass.
The cost model necessitates narrowing the sparsity pattern candidates to those that are block-aligned.

\paragraph{Cost model}
We model the time cost of an operation based on the number of floating point operations and memory access.
The main feature is that our cost model takes into account \emph{memory coalescing}, where accessing a memory location costs the same as accessing the whole block of $b$ elements around it (typically $b = 16 \text{ or } 32$ depending on the hardware).

Let $\mathrm{Cost}_\mathrm{mem}$ be the memory access cost (either read or write) for a block of $b$ contiguous elements.
Accessing any individual element within that block also costs $\mathrm{Cost}_\mathrm{mem}$ time.
Let $\mathrm{Cost}_\mathrm{flop}$ be the compute cost of a floating point operation.
Let $N_\mathrm{block mem}$ be the number of block memory access, and $N_\mathrm{flop}$ be the number of floating point operations.
Then the total cost of the operation is
\begin{equation*}
  \mathrm{Total cost} = \mathrm{Cost}_\mathrm{mem} \cdot N_\mathrm{block mem} + \mathrm{Cost}_\mathrm{flop} \cdot N_\mathrm{flop}.
\end{equation*}
This cost model is a first order approximation of the runtime on modern hardware (GPUs), ignoring the effect of caching.

\paragraph{Block-aligned sparsity pattern, Block cover, and Memory access cost}
As the memory access cost depends on the number of block of memory being accessed, we describe how the number of nonzero elements in a sparse matrix relates to the number of blocks being accessed.
We first define a \emph{block cover} of a sparse mask.
\begin{definition}
  A sparse mask $M \in \{ 0, 1 \}^{m \times n}$ is $(b_1, b_2)$-\emph{block-aligned} if for any index $i, j$ where $M_{ij} = 1$, we also have $M_{i'j'} = 1$ where:
  \begin{equation*}
     i' = b_1 \lfloor i / b_1 \rfloor + r_1, j' = b_2 \lfloor j / b_2 \rfloor + r_2 \text{ for all } r_1 = 0, 1, \dots, b_1 - 1 \text{ and } r_2 = 0, 1, \dots, b_2 - 1.
  \end{equation*}

  The $(b_1, b_2)$-\emph{block cover} of a sparse mask $M \in \{ 0, 1 \}^{m \times n}$ is the $(b_1, b_2)$-block-aligned mask $M' \in \{ 0, 1 \}^{m \times n}$ with the least number of nonzeros such that $M_{ij} \leq M'_{ij}$ for all $i, j$.
\end{definition}
We omit the block size $(b_1, b_2)$ if it is clear from context.

A sparse mask $M$ being $(b_1, b_2)$ block-aligned means that if we divide $M$ into blocks of size $b_1 \times b_2$, then each block is either all zeros or all ones.
To get the $(b_1, b_2)$-block cover of a sparse mask $M$, we simply divide $M$ into blocks of size $b_1 \times b_2$ and set each block to all ones if any location in that block is one.


For a sparse matrix with sparse mask $M$ on a device with block size $b$, the number of block memory access $N_\mathrm{block mem}$ is the number of nonzero blocks in its $(1, b)$-block cover $M'$ (assuming row-major storage).
This corresponds to the fact that to access a memory location on modern hardware (GPUs), the device needs to load a whole block of $b$ elements around that location.

\paragraph{Fast sparse matrices means block-aligned sparsity pattern}
For sparsity patterns that are not block-aligned, such as the random sparse pattern where each location is independently zero or nonzero, its $(1, b)$-block cover might increase the density by a factor of close to $b$ times (we show this more rigorously in the Appendix).
As memory access often dominates the computation time, this means that non block-aligned sparsity will often result is $b$ times slower execution than block-aligned ones.
In other words, exploiting hardware locality is crucial to obtain speed up.

Therefore, this cost model indicates that instead of searching over sparsity patterns whose total cost is less than some budget $C$, we can instead search over block-aligned patterns whose number of nonzeros is less than some limit $k$.
For our theoretical analysis, we consider sparsity patterns that are $(1, b)$-block-aligned.
In practice, since we need to access both the matrix and its transpose (in the forward and backward pass), we require the sparsity pattern to be both $(1, b)$-block-aligned and $(b, 1)$-block-aligned.
This is equivalent to the condition that the sparsity pattern is $(b, b)$-block-aligned.

\paragraph{Sparse matrix approximation in the forward pass}
We now formulate the sparse matrix approximation in the forward pass.
That is, we have weight matrix $A$ with input $B$ and we would like to sparsify $A$ while minimizing the difference in the output.
For easier exposition, we focus on the case where number of nonzeros in each row is the same.
 \begin{definition}[Forward regression]\label{def:sparse_mask_factorization_before:informal}
Given four positive integers $m \geq n \geq d \geq k \geq 1$, matrices $A \in \R^{m \times d}$ and $B \in \R^{d\times n}$. The goal is to find a $(1, b)$-block-aligned binary mask matrix $M\in \{0, 1\}^{m \times d}$ that satisfies
\begin{align*}
     \min_{M \in \{0,1\}^{m \times d} } & ~ \| A \cdot B - (A \circ M) \cdot B \|_1 \\
    \mathrm{s.t.} & ~ \| M_{i} \|_0 = k , \forall i \in [d]
\end{align*}
where $M_i$ is the $i$-th row of $M$.
\end{definition}

\paragraph{Sparse matrix approximation in the backward pass}
In the backward pass to compute the gradient wrt to the weight matrix $A$, we would like to sparsify the gradient $C B^\top$ while preserving as much of the gradient magnitude as possible.
\begin{definition}[Backward regression]\label{def:sparse_mask_factorization_after:informal}
Given four positive integers $m \geq n \geq d \geq k \geq 1$, matrices $B \in \R^{d \times n}$ and $C \in \R^{m \times n}$.
The goal is to find a $(1, b)$-block-aligned binary mask matrix $M \in \{0,1\}^{m \times d}$ such that
\begin{align*}
  \min_{M \in \{0,1\}^{m \times d} } & ~ \| C \cdot B^\top  - ( C \cdot B^\top ) \circ M \|_1 \\
  \mathrm{s.t.}& ~ \| M_i \|_0 = k , \forall i \in [d]
\end{align*}
where $M_i$ is the $i$-th row of $M$.
\end{definition}
Without making any assumptions, such problems are in general computationally hard \cite{fkt15,rsw16}.








\newpage
\section{Analysis of Butterfly Variants}
\label{sec:butterfly_proofs}

We present formal versions of theorems in~\cref{sec:analysis} regarding variants
of butterfly matrices.
We provide full proofs of the results here.

\subsection{Block Butterfly Analysis}
\label{subsec:block_butterfly_proofs}

\begin{proof}[Proof of \cref{thm:block_butterfly}]
  Let $M$ be an $n \times n$ block butterfly matrix with block size $b$.
  We want to show that $M$ also has a representation as an $n \times n$ block
  butterfly matrix with block size $2b$.

  By \cref{def:bmatrix}, $M$ has the form:
  \begin{equation*}
    M = \vB_{\frac{n}{b}}^{ \left( \frac{n}{b}, b \right)} \vB_{\frac{n}{2b}}^{ \left( \frac{n}{b}, b \right)} \hdots \vB_4^{ \left( \frac{n}{b}, b \right)} \vB_2^{ \left( \frac{n}{b}, b \right)}.
  \end{equation*}
  Notice that we can combine that last two terms to form a matrix of the form
  $\vB_2^{ \left( \frac{n}{2b}, 2b \right)}$ (see \cref{fig:tradeoff}).
  Moreover, other terms in the product of the form
  $\vB_{\frac{n}{2^ib}}^{ \left( \frac{n}{b}, b \right)}$ can also be written as
  $\vB_{\frac{n}{2^{i-1} 2b}}^{ \left( \frac{n}{2b}, 2b \right)}$ (see \cref{fig:tradeoff}).
  Thus $M$ also has the form:
  \begin{equation*}
    M = \vB_{\frac{n}{2b}}^{ \left( \frac{n}{2b}, 2b \right)} \vB_{\frac{n}{4b}}^{ \left( \frac{n}{2b}, 2b \right)} \hdots \vB_2^{ \left( \frac{n}{2b}, 2b \right)}.
  \end{equation*}
  In other words, $M$ is also an $n \times n$ block butterfly matrix with block
  size $2b$.

\end{proof}

\begin{proof}[Proof of \cref{cor:block_butterfly_contains_sparse}]
  \citet[Theorem 3]{dao2020kaleidoscope} states that any $n \times n$ sparse
  matrix with $s$ nonzeros can be represented as products of butterfly matrices
  and their transposes, with $O(s \log n)$ parameters.

  For a constant block size $b$ that is a power of 2, the set of $n \times n$ block butterfly
  matrices of block size $b$ contains the set of regular butterfly matrices by
  \cref{thm:block_butterfly}.
  Therefore any such $n \times n$ sparse matrix also has a representation has
  products of block butterfly matrices of block size $b$ and their transposes,
  with $O(s \log n)$ parameters.
\end{proof}

\subsection{Flat Butterfly Analysis}
\label{subsec:flat_butterfly_proofs}

We prove \cref{thm:flat_butterfly_approx}, which relates the first-order
approximation in the form of a flat butterfly matrix with the original butterfly
matrix.
\begin{proof}[Proof of \cref{thm:flat_butterfly_approx}]
  Let $n = 2^m$ and let $B_1, \dots, B_m \in \mathbb{R}^{n \times n}$ be the $m$
  butterfly factor matrices (we rename them here for simplicity of notation).

  Let
  \begin{equation*}
    E = \prod_{i=1}^m \left(I + \lambda B_i\right) - \left( I + \sum_{i=1}^m \lambda B_i \right).
  \end{equation*}
  Our goal is to show that $\norm{E}_F \leq \epsilon$.

  We first recall some properties of Frobenius norm.
  For any matrices $A$ and $C$, we have
  $\norm{A C}_F \leq \norm{A}_F \norm{C}_F$ and
  $\norm{A + C}_F \leq \norm{A}_F + \norm{C}_F$.

  Expanding the terms of the product in $E$, we have
  \begin{equation*}
    E = \sum_{i=2}^{m} \lambda^i \sum_{s \in [m], \abs{s} = i} \prod_{j \in s} B_j.
  \end{equation*}
  Using the above properties of Frobenius norm, we can bound $E$:
  \begin{align*}
    \norm{E}_F
    &\leq \sum_{i=2}^{m} \lambda^i \sum_{s \in [m], \abs{s} = i} \prod_{j \in s} \norm{B_j}_F \\
    &\leq \sum_{i=2}^{m} \lambda^i \sum_{s \in [m], \abs{s} = i} \prod_{j \in s} B_\mathrm{max} \\
    &= \sum_{i=2}^{m} \lambda^2 m^i \left( B_\mathrm{max} ^i \right) \\
    &= \sum_{i=2}^{m} (\lambda m B_\mathrm{max})^i \\
    &\leq \sum_{i=}^{m} \left(c \sqrt{\epsilon} \right)^i \\
    &\leq c^2 \epsilon \sum_{i=0}^{\infty} (c \sqrt{\epsilon})^i \\
    &\leq \frac{c^2\epsilon}{1 - c\sqrt{\epsilon}} \\
    &\leq \epsilon,
  \end{align*}
  where in the last step we use the assumption that $c \leq \frac{1}{2}$.
\end{proof}

We now bound the rank of the first-order approximation.
\begin{proof}[Proof of \cref{thm:flat_butterfly_rank}]
  Let $M^* = I + \sum_{i=1}^{m} \lambda B_i$.
  Note that any entry in $\sum_{i=1}^{m} \lambda B_i$ has absolute value at most
  \begin{equation*}
    m \lambda B^\infty_\mathrm{max} \leq \frac{c \sqrt{\epsilon} B^\infty_\mathrm{max}}{B_\mathrm{max}} \leq \frac{1}{4},
  \end{equation*}
  where we use the assumption that $B^\infty_\mathrm{max} \leq B_\mathrm{max}$
  and $c \leq \frac{1}{4}$.

  Thus any diagonal entry in $M^*$ has absolute value at least
  $1 - \frac{1}{4} = \frac{3}{4}$ and the off-diagonal entries are at most
  $\frac{c \sqrt{\epsilon}B^\infty_\mathrm{max}}{bm}$.

  \citet[Theorem 1.1]{alon2009perturbed} states that: there exists some $c > 0$
  such that for any real $M \in \mathbb{R}^{n \times n}$, if the diagonal
  elements have absolute values at least $\frac{1}{2}$ and the off-diagonal
  elements have absolute values at most $\epsilon$ where
  $\frac{1}{2 \sqrt{n}} \leq \epsilon \leq \frac{1}{4}$, then
  $\rank(M) \geq \frac{c \log n}{\epsilon^2 \log 1/\epsilon}$.

  Applying this theorem to our setting, we have that
  \begin{equation*}
    \rank(M^*) \geq \Omega \left( \left( \frac{B_\mathrm{max}}{B^\infty_\mathrm{max}} \right)^2 \cdot \frac{m}{\epsilon \log \left( \frac{B_\mathrm{max}}{B^\infty_\mathrm{max}} \right) }\right).
  \end{equation*}
  We just need to show that
  $\frac{B^\infty_\mathrm{max}}{B_\mathrm{max}} \geq \frac{1}{2 c \sqrt{\epsilon n}}$
  to satisfy the condition of the theorem.

  Indeed, we have that
  $1 \leq \frac{B_\mathrm{max}}{B^\infty_\mathrm{max}} \leq \sqrt{2n}$ as each
  $\norm{B_i}_0 \leq 2n$.
  Combining the two conditions on
  $\frac{B_\mathrm{max}}{B^\infty_\mathrm{max}}$, we have shown that
  $1 \leq \frac{B_\mathrm{max}}{B^\infty_\mathrm{max}} \leq 2 c \sqrt{\epsilon n}$.
  This concludes the proof.
\end{proof}


\subsection{Flat Block Butterfly + Low-rank Analysis}
\label{subsec:flat_butterfly_lr_proofs}

We show that flat butterfly + low-rank (an instance) of sparse + low-rank, is
more expressive than either sparse or low-rank alone.
We adapt the argument from~\citet{scatterbrain} to show a generative process
where the attention matrix can be well approximated by a flat butterfly +
low-rank matrix, but not by a sparse or low-rank alone.

We describe here a generative model of an input sequence to attention, parameterized by the inverse temperature $\beta \in \mathbb{R}$ and the intra-cluster distance $\Delta \in \mathbb{R}$.
\begin{process}
  \label{ex:generative}
  Let $Q \in \mathbb{R}^{n \times d}$, where $d\geq\Omega(\log^{3/2}(n))$, with every row of $Q$ generated randomly as follows:
  \begin{enumerate}[leftmargin=*,nosep,nolistsep]
    \item For $C = \Omega(n)$, sample $C$ number of cluster centers $c_1, \dots, c_C \in \mathbb{R}^{d}$ independently from $\mathcal{N}(0, I_d/\sqrt{d})$.
    \item For each cluster around $c_i$, sample $n_i = b$ number of elements around $c_i$, of the form $z_{ij} = c_i + r_{ij}$ for $j = 1, \dots, n_i$ where $r_{ij} \sim \mathcal{N}(0, I_d \Delta/\sqrt{d})$.
    Assume that the total number of elements is $n = c b$ and $\Delta\leq O(1/\log^{1/4} n)$.
  \end{enumerate}
  Let $Q$ be the matrix whose rows are the vectors $z_{ij}$ where $i = 1, \dots, C$ and $j = 1, \dots, n_i$.
  Let $A = Q Q^\top$ and let the attention matrix be $M_\beta = \exp(\beta \cdot A)$.
\end{process}

\begin{theorem}
  \label{thm:temperature}
  Let $M_\beta$, be the attention matrix in~\cref{ex:generative}. Fix $\epsilon\in (0,1)$. Let $R \in \mathbb{R}^{n \times n}$ be a matrix.
  Consider low-rank, sparse, and sparse + low-rank approximations to $M_\beta$.
    Assume $(1 - \Delta^2) \log n  \leq \beta \leq O(\log n)$.
    \begin{enumerate}
        \item \textbf{Flat butterfly + low-rank}: There exists a flat butterfly + low-rank $R$ with $n^{1+o(1)}$ parameters with $\|M_\beta-R\|_F\leq \eps n$.
        \item \textbf{Low-rank}: If $R$ is such that $n-\rank(R)=\Omega(n)$, then $\|M_\beta-R\|_F\geq \Omega(n)$.
        \item \textbf{Sparse}: If $R$ has sparsity $o(n^2)$, then $\|M_\beta - R\|_F\geq \Omega(n)$.
   \end{enumerate}
\end{theorem}

\begin{proof}[Proof sketch]
  As the argument is very similar to that of \citet[Theorem 1]{scatterbrain}, we
  describe here the modifications needed to adapt their proof.

  The main difference between our generative process and that of
  \citet{scatterbrain} is that each cluster has the same number of elements,
  which is the same as the block size.
  The resulting attention matrix will have a large block diagonal component,
  similar to that \citet{scatterbrain}.
  However, all the blocks in the block diagonal component has the same block
  size, which is $b$.
  Moreover, a flat block butterfly of block size $b$ contains a block diagonal
  component of block size $b$.
  Therefore, this flat block butterfly matrix plays the same role as the sparse
  matrix in the proof of \citet{scatterbrain}.
  The rest of the argument follows that of theirs.
\end{proof}


\newpage
\section{Influence in Completely Bounded Block-multilinear Forms}
\label{sec:proof}
\newcommand{\blocks}{\mathrm{blocks}}
\newcommand{\lt}{\mathrm{left}}
\newcommand{\rt}{\mathrm{right}}



In this section we prove the non-commutative root-influence inequality (\thmref{thm:bh-intro}),  the special case of the Aaronson-Ambainis conjecture given in \thmref{thm:aa}, and also briefly mention how the simulation result in \corref{cor:sim} follows from \thmref{thm:aa} and the results in \cite{AA14}. We first need some preliminaries from free probability theory. 



\subsection{Low-degree Polynomials of Haar Random Unitaries}

As discussed in the proof overview, we require bounds on the operator norm (as well as normalized trace) of low-degree polynomials of random unitaries and these follow from known results in free probability theory. Here we explain these connections and also prove some auxillary lemmas needed for the proof of \thmref{thm:bh-intro} and \thmref{thm:aa}. 



Let $z_{\ui}$ denote the non-commutative monomial $z_{i_1} z_{i_2} \cdots z_{i_d}$ for a $d$-tuple $\ui  = (i_1, \ldots, i_d) \in [t]^d$ and let $p(z_1, \ldots, z_t)$ be a non-commutative polynomial in the variables $z_1, \ldots, z_t$. We are interested in computing the operator norm $\|\cdot\|_{\op}$ and the normalized trace  $\tr_N$ of the polynomial $p(z_1, \ldots, z_t)$ (or its higher moments) when substituting $N \times N$ Haar random unitaries for the variables $z_i$.

As explained previously, the theory of free probability gives us tools that allow us to compute  the above in the limit $N \to \infty$. In particular, Voiculescu \cite{V98} showed that the  (normalized) trace of polynomials in Haar random unitaries and their conjugates converges to the trace of the same polynomial evaluated on certain infinite-dimensional operators called \emph{Haar unitaries} that satisfy a non-commutative notion of independence called \emph{free independence}. This was strengthened by Collins and Male \cite{CM11} who showed that such convergence also holds for the operator norm. A short primer on free probability is given in \appref{sec:free}, but for now one can think of $\CA$ as a self-adjoint algebra of bounded linear operators on a Hilbert space and $\phi$ as a trace functional for such operators in the statement given below.


\begin{theorem}[\cite{V98, CM11}] \label{thm:voiculescu}
    Let $p(z_1, \ldots, z_{2t})$ be a non-commutative polynomial in $\BR\langle z_1, \ldots, z_{2t}\rangle$. If $U_1, \ldots, U_t$ are $N \times N$ Haar random unitaries, then almost surely,
    \begin{align*}
     \ \tr_N[p(U_1, \ldots, U_t, U^*_1, \ldots, U_t^*)] &~\xrightarrow[N \to \infty]{}~ \phi[p(u_1, \ldots, u_t, u^*_1, \ldots, u^*_t)],\\
    \  \|p(U_1, \ldots, U_t, U^*_1, \ldots, U_t^*)\|_{\op} &~\xrightarrow[N \to \infty]{}~ \| p(u_1, \ldots, u_t, u^*_1, \ldots, u^*_t)\|,
    \end{align*}
    where $u_1, \ldots, u_t$ are free Haar unitaries in a $C^*$-probability space $(\CA, \phi)$ and $\|\cdot\|$ is the norm for the underlying $C^*$-algebra.
\end{theorem}




Using the above result it suffices to consider free Haar unitaries in a $C^*$-probability space to compute the operator norm and trace of polynomials of random unitaries. For a non-commutative polynomial $p(z_1, \ldots, z_t) = \sum_{|\ui|\le d} c_{\ui}z_{\ui}$, denoting by $\|p\|_2 =  \left(\sum_{|\ui| \le d} |c_{\ui}|^2\right)^{1/2}$, one can show the following easily using techniques from free probability. 

\begin{lemma} \label{thm:trace}
    Let $p(z_1, \ldots, z_t) = \sum_{|\ui|\le d} c_{\ui}z_{\ui} $ be a non-commutative degree-$d$ polynomial in $\R\langle z_1, \ldots, z_t\rangle$ and $u_1, \ldots, u_t$ be free Haar unitaries in a $C^*$-probability space $(\CA, \phi)$. Then, 
     \[ \phi[p(u_1, \ldots, u_t) (p(u_1, \ldots, u_t))^*] =  \|p\|_2^2.\]
\end{lemma}

The above implies that $\tr_N[p(U_1, \ldots, U_t) (p(U_1, \ldots, U_t))^*]$ converges to $\|p\|_2^2$ almost surely as $N \to \infty$. We shall defer the proof of \lref{thm:trace} to \appref{sec:app}, but to aid our intuition we note here that since the  $U_i$'s are independent $N \times N$ Haar random unitaries, the expected value

\[ \BE\left[\tr_N[p(U_1, \ldots, U_t) (p(U_1, \ldots, U_t))^*\right] = \|p\|_2^2,\] 
{and from concentration of measure, it is natural to expect that it converges to the above value}. 


Similarly, to compute the operator norm of $p(U_1, \ldots, U_t)$ for Haar random unitaries one can instead study the norm of the polynomial evaluated on free Haar unitaries. Such bounds are easier to prove using the trace method since free independence imposes strong restrictions on the non-commutative moments. For instance, if $U_1$ and $U_2$ are independent $N \times N$ Haar random matrices, then $\BE[\tr_N(U_1U_2U^*_1U_2^*)]$ is non-zero (albeit quite small), while the corresponding trace evaluated on free Haar unitaries $u_1$ and $u_2$ is zero, that is $\phi(u_1u_2u^*_1u_2^*) = 0$. Thus, computing the trace $\phi[p(u_1,\ldots, u_t, u^*_1, \ldots, u_t^*)]$ reduces to handling the combinatorics of the patterns of $u_i$'s and $u_i^*$'s. 

In particular, we will rely on the following result that follows from the work of Kemp and Speicher \cite{KS05}  who consider the operator norm of homogeneous polynomials evaluated on free $R$-diagonal operators, a class that includes free Haar unitaries as well. We also remark that a bound where the right-hand side below is worse by a multiplicative $O(d^{1/2})$ factor also follows from the work of Haagerup\footnote{We note that Haagerup considered the more general case of polynomials in both $u_i$'s and $u^*_i$'s.}\cite{H78} who proved it in another context, predating even the introduction of free probability theory. 


\begin{theorem}[\cite{KS05}]
\label{thm:kemp-speicher}
    Let $p(z_1, \ldots, z_t) = \sum_{|\ui| = d} c_{\ui}z_{\ui} $ be a homogeneous non-commutative degree-$d$ polynomial in $\R\langle z_1, \ldots, z_t\rangle$ and $u_1, \ldots, u_t$ be free Haar unitaries in a $C^*$-probability space. Then, 
    \[ 
    \|p(u_1, \ldots, u_t)\| \le \sqrt{e(d+1)} \cdot \|p\|_2,
    \]
    where the left-hand side denotes the norm in the underlying $C^*$-algebra. 
\end{theorem}

For completeness, we  introduce the necessary free probability background and some combinatorial details in \appref{sec:app}, and we present the fairly short proof of \thmref{thm:kemp-speicher} (from \cite{KS05}) there in a self-contained way. We shall need to extend the above bound to non-homogeneous polynomials. Let $p(z_1, \ldots, z_t) = \sum_{|\ui| \le d} c_{\ui}z_{\ui}$ and  let $p_k(z_1, \ldots, z_t) = \sum_{|\ui| = k} c_{\ui}z_{\ui}$ denote the degree-$k$ homogeneous part of $p$. Writing $p_k = p_k(u_1, \ldots, u_t)$ for $0 \le k  \le d$ and $p = p(u_1, \ldots, u_t)$, it follows from the triangle inequality,  \thmref{thm:kemp-speicher}, and Cauchy-Schwarz, that
    \begin{align*}
        \ \|p\| &\le \sum_{k=0}^d \|p_k\| 
        \le 
        \sum_{k=0}^d\sqrt{e(k+1)}\|p_k\|_2
        \le
       \sqrt{e}\left(\sum_{k=0}^d (k+1)\right)^{1/2} \left(\sum_{k=0}^d  \|p_k\|^2_2\right)^{1/2} \leq \sqrt{e}(d+1)  \cdot\|p\|_2.
    \end{align*}
Thus, we essentially get the same bound as in the homogeneous case, at the expense of an additional $O(d^{1/2})$ factor.



Collecting all the above we have the following as a direct consequence:

\begin{theorem} \label{thm:op-norm}
    Let $p(z_1, \ldots, z_t) = \sum_{|\ui|\le d} c_{\ui}z_{\ui} $ be a non-commutative degree-$d$ polynomial in $\R\langle z_1, \ldots, z_t\rangle$ and $U_1, \ldots, U_t$ be independent $N \times N$ Haar random unitaries. Then, as $N \to \infty$, the following holds almost surely, 
    \[ \tr_N[p(U_1, \ldots, U_t) (p(U_1, \ldots, U_t))^*] =  \|p\|_2^2,\]
    and
    \[ \|p(U_1, \ldots, U_t)\|_{\op} \le \sqrt{e}(d+1)  \cdot \|p\|_2,\]
    Moreover, the factor $(d+1)$ in the operator norm bound can be improved to $\sqrt{d+1}$ if the polynomial is homogeneous.
\end{theorem}

Based on the above theorem, we prove the following key lemma which captures the polar decomposition strategy mentioned in the earlier proof overview (\secref{sec:bh}). This will serve as the key ingredient in the proof of \thmref{thm:aa} and \thmref{thm:bh-intro}. 

\begin{lemma}\label{lem:polar}
    Let $p$ be a non-commutative degree-$d$ polynomial in $\R\langle y_1, \ldots, y_m, z_1, \ldots, z_t\rangle$ given by
    \[ p(y_1, \ldots, y_m, z_1, \ldots, z_t) = \sum_{i=1}^m y_i q_i(z_1, \ldots, z_t) + q_0(z_1, \ldots, z_t).\]
    Then, for every $\delta > 0$, there exist an integer $N$ and $N \times N$ unitaries $V_1,\ldots, V_m, W_1, \ldots, W_t$ such that 
    \[ \|p(V_1, \ldots, V_m, W_1, \ldots, W_t)\|_{\op} \ge \frac{1}{\sqrt{e}(d+1)} \sum_{i=1}^m \|q_i\|_2 - \delta.\]
    Moreover, the factor in front can be improved to $(e(d+1))^{-1/2}$ if $p$ is homogeneous. 
\end{lemma}

\begin{proof}[Proof of \lref{lem:polar}]
     For an arbitrary integer $N$, let us pick independent $N \times N$ Haar random unitaries $W_1, \ldots, W_t$ which we substitute for the variables $z_1,\ldots,z_t$, respectively, and let $M_i = q_i(W_1, \ldots, W_t)$ be the corresponding random matrices. Then, for any tuple of matrices $V_1, \ldots, V_m$ that we could substitute for the variables $y_1, \ldots, y_m$, we have that 
    \[ 
    p(V_1, \ldots, V_m, W_1, \ldots, W_t) = \sum_{i=1}^m V_i M_i + M_0.
    \] 
     \thmref{thm:op-norm} and union bound imply that as $N \to \infty$, with probability $1$ all the following events simultaneously hold: 
    \begin{itemize}
        \item $\|M_i\|_{\op} \le \sqrt{e}(d+1) \cdot \|q_i\|_2$ for each $i$,
        \item $\tr_N(M^*_iM_i) = \|q_i\|_2^2$ for each $i$, where $\tr_N(M)$ is the normalized trace.
    \end{itemize}
   To show that the operator norm must be large, let us fix a sufficiently large $N$ and a choice of $N\times N$ unitaries $W_1, \ldots, W_t$ such that $M_i$ satisfies $\|M_i\|_{\op} \le \sqrt{e}(d+1) \cdot \|q_i\|_2 + \epsilon$ and $\tr_N(M^*_iM_i) \ge \|q_i\|_2^2 - \epsilon$ for each $0\le i\le m$, where $\epsilon$ can be made arbitrarily small by increasing $N$. For $0 \leq i \leq m$, let $M_i = U_i P_i$ be the left polar decomposition of $M_i$, where $U_i$ is a unitary matrix and $P_i$ is a positive semidefinite matrix.
   
   We select the tuple of unitary matrices $V_1, \ldots, V_m$ that we substitute for the variables $y_1, \ldots, y_m$ to be $V_i = U_0U^*_i$ for $i \in [m]$. With this we have that $\|p(V_1, \ldots, V_m, W_1, \ldots, W_t)\|_{\op}$ is at least
    \begin{align*}
         \Big\|M_0 + \sum_{i=1}^m V_iM_i\Big\|_{\op} & = \Big\|U_0 P_0 + \sum_{i=1}^m U_0 U_i^* U_iP_i \Big\|_{\op} \\
        \ & =  \Big\|U_0 P_0 + \sum_{i=1}^m U_0 P_i\Big\|_{\op}  = \Big\| P_0 + \sum_{i=1}^m  P_i\Big\|_{\op}\ge \tr_N\Big(P_0 + \sum_{i=1}^m P_i\Big) \ge \tr_N\Big(\sum_{i=1}^m P_i\Big),
    \end{align*}
    where the last equality follows since the operator norm is unitarily invariant and the last two inequalities follow from the positive semidefiniteness of the $P_i$'s.

    For every positive semidefinite matrix $P$, we have that $\tr_N(P) \ge {\tr_N(P^2)}/{\|P\|_{\op}}$. 
  
    Hence,
     \[ \|p(V_1, \ldots, V_m, W_1, \ldots, W_t)\|_{\op} \ge \sum_{i=1}^m \frac{\tr_N(P_i^2)}{\|P_i\|_{\op}}.\]
     By our choice of $M_i$, we have that $\tr_N(P_i^2) = \tr_N(M_i^* M_i) \ge \|q_i\|_2^2 - \eps$ and $\|P_i\|_{\op} = \|M_i\|_{\op} \le \sqrt{e}(d+1)\|q_i\|_2 + \eps$. Since $\eps$ can be made arbitrarily small by increasing $N$, it follows that 
      \[ \|p(V_1, \ldots, V_m, W_1, \ldots, W_t)\|_{\op} \ge \frac1{\sqrt{e}(d+1)} \sum_{i=1}^m \|q_i\|_2 - \delta ,\]
     for large enough $N$. The improved bound for the homogeneous case follows directly by plugging the bound of \thmref{thm:op-norm} into the above proof.
\end{proof}





\subsection{Non-commutative root-influence inequality}
\label{sec:bh-proof}


For clarity in the proofs below, we remind our  convention that all tuples or blocks are denoted with boldface fonts (e.g. $\BU_1$ or $\BA$), while a single element is denoted without boldface (e.g. $U_1(i)$ or $A_i$ or $A$). Before proceeding with the proof, we restate the statement for convenience.

\bh*





\begin{proof}[Proof of \thmref{thm:bh-intro}] 
Since $f$ is homogeneous, we can write
   \begin{align*}
    f(\x_1,\ldots, \x_d) &= \sum_{i_1, \ldots, i_d \in [n]} \hf_{i_1, \ldots, i_d} ~x_1(i_1)x_2(i_2)\cdots x_d({i_d}) \\
    \ & = \sum_{i=1}^n  x_1(i) \underbrace{\left(\sum_{i_2,\ldots, i_d \in [n]} \hf_{i_1, \ldots, i_d} ~x_2(i_2)\cdots x_d({i_d})\right)}_{\textstyle := f_i(\x_2,\ldots, \x_d)}.
\end{align*}
 In this case, it follows from \eqref{eqn:inf-tensor} that for each $i \in [n]$, we have 
 \begin{equation}\label{eqn:var}
     \ \Var[f_i] = \|f_i\|^2_2 = \inf_{1,i}(f) \text{ and }  \Var[f] = \sum_{i=1}^n \inf_{1,i}(f).
 \end{equation}

  Let us denote the corresponding non-commutative block-multilinear polynomials by $f(\BU_1, \ldots, \BU_d)$ and $f_i(\BU_2, \ldots,\BU_d)$ where $\BU_b = (U_b(1), \ldots, U_b(n))$ denotes the $b^\text{th}$ block of non-commutative variables. To show a lower bound on $\cbnorm{f}$ it suffices to exhibit a collection of square matrices $\{U_b(i)\}_{b\in [d], i \in [n]}$ with operator norm at most~1, such that $\|f(\BU_1, \ldots, \BU_d)\|_{\op}$ is large. 
  
%  

Applying \lref{lem:polar} for the homogeneous case (with $p = f$, $q_i=f_i$ for $i \in [n]$, and $q_0=0)$, it follows that for every $\delta > 0$ there exists an integer $N$ and a choice of tuples of $N \times N$ unitaries $\BU_1, \ldots, \BU_d$ such that  
      \[ \cbnorm{f} \ge \|f(\BU_1, \ldots, \BU_d)\|_{\op} \ge \frac1{\sqrt{e(d+1)}} \sum_{i\in [n]} \|f_i\|_2  -\delta \stackrel{\eqref{eqn:var}}{\ge}  \frac{1}{\sqrt{e(d+1)}} \left(\sum_{i=1}^n \sqrt{\Inf_{1,i}(f)} \right) -\delta.\]
Taking $\delta \to 0$, we get the statement of the lemma. The proof for the inequality when $b=d$ is the last block follows similarly by using the right polar decomposition.
\end{proof}

\subsection{Aaronson-Ambainis Conjecture for non-homogeneous forms}

In this section, we prove \thmref{thm:aa}, which requires handling non-homogeneous forms. The proof will be similar to the proof of \thmref{thm:bh-intro} but we will need to be careful about certain details. 

\begin{proof}[Proof of \thmref{thm:aa}]
Any block-multilinear polynomial $f(x_1, \ldots, x_d)$ can be written as 
\begin{align*}
    f(\x_1,\ldots, \x_d) &= \BE f + \sum_{b\in [d]} f_b(\x_b, \x_{b+1}, \ldots, \x_d),
\end{align*}
where $f_b$ consists of all monomials of $f$ that start with a variable in the $b^\text{th}$ block $\x_b$. Note that $f_b$ depends only on the variables in blocks $\x_b, \x_{b+1},\ldots, \x_d$. Moreover, it follows from \eqref{eqn:inf-tensor} that 
 \begin{equation}\label{eqn:var-general}
     \ \Var[f] = \sum_{b \in [d]} \|f_b\|_2^2 = \sum_{b \in [d]} \Var[f_b],
 \end{equation}
so there exists a block $\beta \in [d]$ such that $\Var[f_{\beta}] \ge \frac{1}{d}\Var[f]$. 

Since $f_{\beta}$ contributes a lot to the variance, it is natural to try to find an influential variable in the block $\x_{\beta}$. Towards this end,  we pull out the variables $x_{\beta}(i)$ and write
\begin{align*}
    f_{\beta}(\x_{\beta},\ldots, \x_d) &= \sum_{i\in [n]} x_{\beta}(i) f_{\beta,i}(\x_{\beta+1}, \ldots, \x_d),
\end{align*}
for block-multilinear polynomials $f_{\beta,i}(\x_{\beta+1}, \ldots, \x_d)$. Note that some of the $f_{\beta,i}$'s could be identically zero, so let us define $S$ to be the set of those $i$ such that $f_{\beta,i}$ is non-zero. We note that
\begin{align} \label{eqn:part-inf}
  \|f_{\beta,i}\|_2^2  =  \Inf_{\beta,i}(f_{\beta}) \le \Inf_{\beta,i}(f)  
\end{align}
which implies that
\begin{align}\label{eqn:var-main}
    \frac{1}{d} \Var[f] \le \Var[f_{\beta}] = \sum_{i \in S}\|f_{\beta,i}\|_2^2 = \sum_{i \in S} \Inf_{\beta,i}(f_{\beta}).
\end{align}
\begin{sloppypar}
Denote the corresponding non-commutative block-multilinear polynomials by $f(\BU_1, \ldots, \BU_d)$,  $f_b(\BU_{b}, \ldots,\BU_d)$, and $f_{\beta}(\BU_{\beta+1}, \ldots,\BU_d)$ where $\BU_b = (U_b(1), \ldots, U_b(n))$ denotes the $b^\text{th}$ block of non-commutative variables. To show a lower bound on $\cbnorm{f}$ it suffices to exhibit a collection of square matrices $\{U_b(i)\}_{b\in [d], i \in [n]}$ with operator norm at most~1 such that $\|f(\BU_1, \ldots, \BU_d)\|_{\op}$ is large.
\end{sloppypar}
  
 We set the matrices in blocks $\BU_1, \ldots, \BU_{\beta-1}$ to be zero (that is, the all-zero matrix $\BZ$). Note that with this choice all polynomials $f_b(\U_b, \ldots, \U_d)$ where $b < \beta$ vanish and the non-commutative polynomial becomes 
 \[ f(\BZ, \ldots, \BZ, \BU_{\beta}, \BU_{\beta+1}, \ldots, \BU_d) = \sum_{i\in S} U_{\beta}(i) f_{\beta,i}(\BU_{\beta+1}, \ldots, \BU_d) + \sum_{b=\beta+1}^d f_b(\BU_b, \BU_{b+1}, \ldots, \BU_d) + \Ef,\]
  which is a non-commutative polynomial of the form considered in \lref{lem:polar} (with $m = |S|$, $q_i = f_{\beta,i}$ and $q_0 = \sum_{b=\beta+1}^d f_b + \Ef$). Thus, by \lref{lem:polar} for every small $\delta>0$ there exists an integer $N$ and a choice of $N \times N$ matrices for the blocks $\BU_{\beta},\ldots, \BU_d$ such that 
        \begin{align*}
             \ \cbnorm{f} & \ge \|f(\BZ, \ldots, \BZ, \BU_{\beta}, \BU_{\beta+1}, \ldots, \BU_d)\|_{\op} & \\
             \  & \ge \frac1{\sqrt{e}(d+1)} \sum_{i\in S} \|f_{\beta,i}\|_2 -\delta  \stackrel{\eqref{eqn:part-inf}}{=}  \frac{1}{\sqrt{e}(d+1)} \left(\sum_{i \in S} \sqrt{\Inf_{\beta,i}(f_{\beta})} \right) -\delta & \\
             \ &\stackrel{\eqref{eqn:var-main}}{\ge}  \frac{1}{\sqrt{e}(d+1)} \left( \frac{\sum_{i \in S} \Inf_{\beta,i}(f_{\beta})}{\sqrt{\maxinf(f)}} \right) -\delta  \stackrel{\eqref{eqn:part-inf}}{\ge}  \frac{1}{\sqrt{e}(d+1)^{2}} \left( \frac{\Var[f]}{ \sqrt{\maxinf(f)}} \right) -\delta
        \end{align*}
        Taking $\delta \to 0$ and using the assumption that $\|f\|_{\cb} \le 1$, we obtain the statement of the theorem:
     \[
     1\geq \cbnorm{f} \ge \frac{1}{\sqrt{e}(d+1)^{2}} \cdot \frac{\Var[f]}{\sqrt{\maxinf(f)}} \implies \maxinf(f) \ge  \frac{(\Var[f])^2}{e(d+1)^4}. \qedhere
     \]
\end{proof}
 

     
     

\subsection{Approximating completely bounded forms with decision trees}



In this section, we briefly mention how to obtain \corref{cor:sim}.
Aaronson and Ambainis \cite[Theorem 3.3]{AA14} showed that querying the most influential variable reduces the variance of the function~$f$, and if that influence is lower bounded by a polynomial in $\Var[f]/d$, then after $\poly(d)$ queries (the exact quantitative dependence can be read off from their proof), the variance of the function becomes small enough so that it can be approximated almost-everywhere by its expectation.  Since the family of degree-$d$ block-multilinear forms with completely bounded norm at most one is closed under restrictions, one can apply \thmref{thm:aa} repeatedly. This gives us \corref{cor:sim}.
\newpage
\section{Neural Tangent Kernel, Convergence, and Generalization}
\label{sec:appx_ntk}

Our analysis relies on the neural tangent kernel (NTK)~\citep{jacot2018neural} of the network.
\begin{definition}
  Let $f(\cdot, \theta) \colon \mathbb{R}^{d} \to \mathbb{R}$ be the function specified by a neural network with parameters $\theta \in \mathbb{R}^p$ and input dimension $d$.
  The parameter $\theta$ is initialized randomly from a distribution $P$.
  Then its neural tangent kernel (NTK) \citep{jacot2018neural} is a kernel $K \colon \mathbb{R}^{d} \times \mathbb{R}^{d} \to \mathbb{R}$ defined by:
  \begin{equation*}\label{eq:kernel}
    K(x, y) = \E_{\theta \sim P} \left[ \left\langle \frac{\partial f(x; \theta)}{\partial \theta}, \frac{\partial f(y; \theta) }{\partial \theta} \right\rangle\right].
  \end{equation*}
\end{definition}

We can relate the training and generalization behavior of dense and sparse
models through their NTK.
The standard result~\citep{sy19} implies the following.
\begin{proposition}
  \label{thm:ntk}
  Let $f_\mathrm{dense}$ denote a ReLU neural network with $L$ layers with dense weight matrices $\theta_\mathrm{dense}$ with NTK $K_\mathrm{dense}$, and let $f_\mathrm{sparse}$ be the ReLU neural network with the same architecture and with weight matrices $\theta_\mathrm{sparse}$ whose rows are $k$-sparse, and with NTK $K_\mathrm{sparse}$.
  Let $x_1, \dots, x_N$ be the inputs sampled from some distribution $P_X$.
  Suppose that the empirical NTK matrices $K_d = K_\mathrm{dense}(x_i, x_j)$ and $K_s = K_\mathrm{sparse}(x_i, x_j)$ for $(i, j) \in [N] \times [N]$ satisfy $\| K_d - K_s \| \leq \delta$.

  {\bf Training.}
  We knew the the number of iterations of dense network is $\lambda_{\min}(K_d)^{-2} n^2 \log(1/\epsilon)$ to reach the $\epsilon$ training loss. For sparse network we need $(\lambda_{\min}(K_d) -\delta)^{-2} n^2 \log(1/\epsilon)$.

  {\bf Generalization.}
  We knew the the number of iterations of dense network is $\lambda_{\min}(K_d)^{-2} n^2 \log(1/\epsilon)$ to reach the generalization error $\epsilon$ training loss. For sparse network we need $(\lambda_{\min}(K_d) -\delta)^{-2} n^2 \log(1/\epsilon)$.
\end{proposition}
These results relate the generalization bound of sparse models to that of dense models.
\newpage



\newpage
\section{Dropout Neural Network and KRR}\label{sec:dropout_KRR}
\label{sec:notation1}
We consider a two layer neural network with ReLU activation function, and write 
\begin{align}
\label{eq:pb2}
f(W, x) := \frac{1}{\sqrt{m}}\sum_{r = 1}^{m}a_r \phi(w_{r}^{\top}x) = \frac{1}{\sqrt{m}}\sum_{r = 1}^{m}a_r w_{r}^{\top}x \mathbf{1}_{w_{r}^{\top}x\geq 0}
\end{align}
where $w_r(0) \sim N(0, I_d) \in \R^d$, $a_{r} \sim \mathrm{unif}(\{-1, +1\})$ and all randomnesses are independent. We will fix $a_r$ during the training process and use $\frac{1}{\sqrt{m}}$ normalization factor, both of which are in the literature of \cite{dzps19,sy19,bpsw21}.

Suppose the training data are $(x_1,y_1), \ldots, (x_n, y_n) \in \R^{d}\times \R$, we define the classical objective function $\hat{L}$ as follows:
\begin{align*}
\hat{L}(W) := \frac{1}{2}\sum_{i = 1}^{n}\left(f(W, x_i) - y_i\right)^2.
\end{align*}

The gradient with respect to loss function $\hat{L}$ is %
\begin{align*}
\frac{\partial \hat{L}}{\partial w_r} = \frac{1}{\sqrt{m}}\sum_{i=1}^{n} (f(W, x_i) - y_i)a_r x_i\mathbf{1}_{w_r^{\top}x_i \geq 0}.
\end{align*}


We consider the effect of dropout on network training. For each $r \in [m]$, we introduce the mask by defining random variable $\sigma_{r}$ as follows:
\begin{align*}
\sigma_r = 
\begin{cases}
0, & \mathrm{with~probability~} 1-q ; \\
1/q, & \mathrm{with~probability~} q .
\end{cases}
\end{align*}

It is easy to see that $\E[ \sigma_r ] = 0 \cdot (1-q) + (1/q) \cdot q = 1$ and $\E[ \sigma_r^2 ] = 0^2 \cdot (1-q) + (1/q)^2 \cdot q = 1/q$. %
We assume $\sigma_i$ and $\sigma_j$ are independent for any $i\neq j$, then $\E[\sigma_i\sigma_j] = \E[\sigma_i]\E[\sigma_j]=1$. Let $\sigma = (\sigma_1,\cdots, \sigma_m)$, we define our \textbf{dropout neural net} as
\begin{align}
\label{eq:pb1}
F(W, x, \sigma) := \frac{1}{\sqrt{m}}\sum_{r = 1}^{m}a_r \sigma_r\phi(w_{r}^{\top}x) = \frac{1}{\sqrt{m}}\sum_{r = 1}^{m}a_r\sigma_r w_{r}^{\top}x \mathbf{1}_{w_{r}^{\top}x\geq 0}.
\end{align}
Dropout explicitly change the target function, since we need to minimize the $\ell_2$ distance between $F(W, x, \sigma)$ and $y$, instead of $f(W, x)$ and $y$. Formally, we define the \textbf{dropout loss} as %
\begin{align}
\label{eq:dropout_loss_def}
L(W) := \frac{1}{2}\E_{\sigma}\left[\sum_{i = 1}^{n}\left(F(W, x_i, \sigma) - y_i\right)^2\right].
\end{align}

We first give an explicit formulation of $L$ which also shows the difference between $L$ and $\hat{L}$.%

\begin{lemma}
\label{lem:explicit-regularization}
The dropout loss defined in Eq.~\eqref{eq:dropout_loss_def} can be expressed as the sum of classical loss $\hat{L}$ and a regularization term as
\begin{align}
\label{eq:explicit-regularization}
L(W) = \hat{L}(W) + \frac{1-q}{2mq}\sum_{i=1}^{n} \sum_{r = 1}^{m}\phi(w_r^\top x_i)^{2}.
\end{align}
\end{lemma}

\begin{proof}
Since $\E[\sigma_r] = 1$, we have
\begin{align}\label{eq:E_F}
    \E_{\sigma}[F(W, x_i, \sigma)] = & ~ \frac{1}{\sqrt{m}}\E_{\sigma}[\sum_{r = 1}^{m}a_r \sigma_r\phi(w_{r}^{\top}x)] =  \frac{1}{\sqrt{m}}\sum_{r=1}^{m}a_r\phi(w_r^{\top}x_i) = f(W, x_i)
\end{align}
holds for any $i \in [n]$. Next, we show the difference between $L$ and $\hat{L}$:
\begin{align}
    & ~ 2( L(W) - \hat{L}(W) ) \notag\\
    = & ~ \E_{\sigma}\left[\sum_{i = 1}^{n}\left(F(W, x_i, \sigma) - y_i\right)^2\right] -  \sum_{i = 1}^{n}\left(f(W, x_i) - y_i\right)^2\notag\\
    = & ~ \sum_{i = 1}^{n}\left(\E_{\sigma}\left[\left(F(W, x_i, \sigma) - y_i\right)^2\right] - \left(f(W, x_i) - y_i\right)^2 \right)\notag\\
    = & ~ \sum_{i = 1}^{n}\left(\E_{\sigma}\left[F(W, x_i, \sigma)^2\right] - f(W, x_i)^2  \right)\notag \\
    = & ~ \sum_{i = 1}^{n}\left(\frac{1}{m}\sum_{r_1, r_2 \in [m]}\E[a_{r_1}a_{r_2}\sigma_{r_1}\sigma_{r_2}\phi(w_{r_1}^{\top}x_i)\phi(w_{r_2}^{\top}x_i)]- \frac{1}{m}\sum_{r_1, r_2 \in [m]}a_{r_1}a_{r_2}\phi(w_{r_1}^{\top}x_i)\phi(w_{r_2}^{\top}x_i)
    \right)\notag\\
    = & ~ \frac{1}{m}\cdot \frac{1-q}{q}\sum_{i = 1}^{n}\sum_{r=1}^{m}a_r^{2}\phi(w_{r}^{\top}x_i)^2 \notag\\
    = & ~ \frac{1}{m}\cdot \frac{1-q}{q}\sum_{i = 1}^{n}\sum_{r=1}^{m}\phi(w_{r}^{\top}x_i)^2 \label{eq:pb3}
\end{align}
where the first step follows from definition, the second step follows from the linearity of expectation, the third step follows from Eq.~\eqref{eq:E_F}, the forth step follows from expansion, the fifth step follows from $\E[\sigma_{r_1}\sigma_{r_2}] = 1$ for $r_1 \neq r_2$ and $\E[\sigma_{r_1}^2] = \frac{1}{q}$, and the last step follows from $a_r^2 = 1$. Thus we have
\begin{align*}
    L(W) = \hat{L}(W) + \frac{1-q}{2mq}\sum_{i = 1}^{n}\sum_{r=1}^{m}\phi(w_{r}^{\top}x_i)^2
\end{align*}
and finish the proof.
\end{proof}



Before we move on, we introduce some extra notations and definitions. We denote%
\begin{align*}
    \overline{W} = \mathrm{vec}(W) = \left[\begin{matrix}
    w_1\\
    w_2\\
    \vdots\\
    w_m
    \end{matrix}
    \right] \in \R^{md}, ~~~and~~~
    Y = \left[\begin{matrix}
    y_1\\
    y_2\\
    \vdots\\
    y_n
    \end{matrix}
    \right] \in \R^{n}.
\end{align*}

\begin{definition}
We define matrix $G^{\infty}\in\R^{n\times n}$ which can be viewed as a Gram matrix from a kernel associated with ReLU function as follows:
\begin{align*}
    G^{\infty}_{ij}(X) = \E_{w\sim\N(0,I)}[x_i^\top x_j \mathbf{1}_{w^\top x_i \geq 0, w^\top x_j\geq 0}],~~~ \forall i, j\in [n]\times [n]
\end{align*}
and assume $\lambda_0 = \lambda_{\min}(G^{\infty}) > 0$\footnote{According to Theorem 3.1 in \cite{dzps19}, the assumption holds when $x_i$ is not parallel with $x_j$ for $i\neq j$, which is reasonable in reality.}.
\end{definition}

\begin{definition}\label{def:Phi_first_time}
We define the masked matrix $\Phi_{W}(X, \sigma)\in \R^{n \times md}$ as %
\begin{align*}
    \Phi_{W}(X,\sigma) := & ~ \frac{1}{\sqrt{m}}\left[
    \begin{matrix}
    \Phi(x_1, \sigma)\\
    \Phi(x_2, \sigma)\\
    \vdots\\
    \Phi(x_n, \sigma)
    \end{matrix}
    \right] \\
    = & ~
    \frac{1}{\sqrt{m}}\left[
    \begin{matrix}
    a_1 \sigma_1  \mathbf{1}_{\langle w_{1}, x_1\rangle \geq 0} x_1^{\top} & a_2 \sigma_2 \mathbf{1}_{\langle w_{2}, x_1\rangle \geq 0} x_1^{\top} &\ldots &a_m \sigma_m \mathbf{1}_{\langle w_{m}, x_1\rangle \geq 0} x_1^{\top}\\
    a_1 \sigma_1 \mathbf{1}_{\langle w_{1}, x_2\rangle \geq 0} x_2^{\top} & a_2 \sigma_2 \mathbf{1}_{\langle w_{2}, x_2\rangle \geq 0} x_2^{\top} &\ldots &a_m \sigma_m \mathbf{1}_{\langle w_{m}, x_2\rangle \geq 0} x_2^{\top}\\
    \vdots &\vdots &\vdots &\vdots\\
    a_1 \sigma_1 \mathbf{1}_{\langle w_{1}, x_n\rangle \geq 0} x_n^{\top} & a_2 \sigma_2 \mathbf{1}_{\langle w_{2}, x_n\rangle \geq 0} x_n^{\top} &\ldots &a_m \sigma_m \mathbf{1}_{\langle w_{m}, x_n\rangle \geq 0} x_n^{\top}\\
    \end{matrix}
    \right]
\end{align*}
and also define the unmasked matrix $\hat{\Phi}_W(X)\in\R^{n\times md}$ as
\begin{align*}
    \hat{\Phi}_W(X) := \frac{1}{\sqrt{m}}\left[
    \begin{matrix}
    a_1  \mathbf{1}_{\langle w_{1}, x_1\rangle \geq 0} x_1^{\top} & a_2 \mathbf{1}_{\langle w_{2}, x_1\rangle \geq 0} x_1^{\top} &\ldots &a_m \mathbf{1}_{\langle w_{m}, x_1\rangle \geq 0} x_1^{\top}\\
    a_1 \mathbf{1}_{\langle w_{1}, x_2\rangle \geq 0} x_2^{\top} & a_2 \mathbf{1}_{\langle w_{2}, x_2\rangle \geq 0} x_2^{\top} &\ldots &a_m \mathbf{1}_{\langle w_{m}, x_2\rangle \geq 0} x_2^{\top}\\
    \vdots &\vdots &\vdots &\vdots\\
    a_1 \mathbf{1}_{\langle w_{1}, x_n\rangle \geq 0} x_n^{\top} & a_2 \mathbf{1}_{\langle w_{2}, x_n\rangle \geq 0} x_n^{\top} &\ldots &a_m \mathbf{1}_{\langle w_{m}, x_n\rangle \geq 0} x_n^{\top}\\
    \end{matrix}
    \right].
\end{align*}
\end{definition}

\begin{definition}\label{def:Psi_first_time}
We  define the masked block diagonal matrix $\Psi_W(X, \sigma) \in\R^{md \times md}$ as
\begin{align*}
    \Psi_{W}(X, \sigma) :=
    \frac{1}{m}\diag \Big( \psi_1, \psi_2, \cdots, \psi_m \Big).
\end{align*}
where $\forall r \in [m]$, $\psi_r \in \R^{d \times d}$ is defined as
\begin{align*}
    \psi_r := a_r^2 \sigma_r^2\sum_{i=1}^{n} x_i x_i^{\top}\cdot \mathbf{1}_{\langle w_{r}, x_i \rangle \geq 0}^{2} = \sigma_r^2 \sum_{i=1}^{n} x_i x_i^{\top}\cdot \mathbf{1}_{\langle w_{r}, x_i \rangle \geq 0}.
\end{align*}
We also define the unmasked block diagonal matrix $\hat{\Psi}_W(X) \in\R^{md \times md}$ as
\begin{align*}
    \hat{\Psi}_{W}(X) :=
    \frac{1}{m}\diag \Big( \hat{\psi}_1, \hat{\psi}_2, \cdots, \hat{\psi}_m \Big).
\end{align*}
where $\forall r \in [m]$, $\hat{\psi}_r \in \R^{d \times d}$ is defined as
\begin{align*}
    \hat{\psi}_r := \sum_{i=1}^{n} x_i x_i^{\top}\cdot \mathbf{1}_{\langle w_{r}, x_i \rangle \geq 0}.
\end{align*}
\end{definition}

\begin{lemma}
It is easy to verify that
\begin{align*}
    \Phi_W(X,\sigma) = \hat{\Phi}_W(X) \cdot D_{\sigma} ~~~ and ~~~\Psi_W(X, \sigma) = \hat{\Psi}_W(X) \cdot D_{\sigma}^2
\end{align*}
where
\begin{align*}
    D_{\sigma} := \diag (\underbrace{\sigma_1,\cdots, \sigma_1}_{d}, \cdots, \underbrace{\sigma_m,\cdots, \sigma_m}_{d})\in\R^{md\times md}.
\end{align*}
\end{lemma}

For convenience, we will simply denote $\Phi_W = \Phi_W(X, \sigma)$ and $\Psi_W = \Psi_W(X,\sigma)$. Then by using the above notations, we can express our dropout loss as $L(W)=\frac{1}{2}\E_{\sigma} [\|\Phi_W\overline{W} - Y\|_2^2]$.


\begin{lemma}\label{lem:explicit_regularization_2}
If we denote $\lambda = \frac{1-q}{q}\geq 0$, then we have
\begin{align*}
    L(W) = \frac{1}{2}\|\hat{\Phi}_{W} \overline{W} - Y\|_{2}^{2} + \frac{\lambda}{2}\overline{W}^{\top}\hat{\Psi}_{W}\overline{W}.
\end{align*}
\end{lemma}
\begin{proof}
As for the first term, we have
\begin{align*}
    \|\hat{\Phi}_W \overline{W} - Y\|_2^2 = & ~ \sum_{i=1}^n (\frac{1}{\sqrt{m}}\sum_{r=1}^m a_r \mathbf{1}_{\langle w_{r}, x_i \rangle \geq 0} x_i^\top \cdot w_r - y_i)^2 \\
    = & ~ \sum_{i=1}^n (\frac{1}{\sqrt{m}}\sum_{r=1}^m a_r \phi(w_r^\top x_i) - y_i)^2 \\
    = & ~ \sum_{i=1}^n (f(W,x_i) - y_i)^2 \\
    = & ~ 2\hat{L}(W).
\end{align*}
As for the second term, since $\hat{\Psi}_W$ is a block diagonal matrix, we have
\begin{align*}
    \overline{W}^\top \hat{\Psi}_W \overline{W} = & ~ \frac{1}{m}\sum_{r=1}^m \Big(w_r^\top \cdot \big( a_r^2\sum_{i=1}^n x_i x_i^\top \cdot \mathbf{1}_{\langle w_{r}, x_i \rangle \geq 0}^{2}\big) \cdot w_r\Big) \\
    = & ~ \frac{1}{m}\sum_{r=1}^m\sum_{i=1}^n \big((w_r^\top x_i)\cdot(w_r^\top x_i)^\top\cdot \mathbf{1}_{\langle w_{r}, x_i \rangle \geq 0}^{2}\big) \\
    = & ~ \frac{1}{m}\sum_{i=1}^n \sum_{r=1}^m\phi(w_r^\top x_i)^2.
\end{align*}
Thus by using Lemma~\ref{lem:explicit-regularization}, we have
\begin{align*}
    L(W) = & ~ \hat{L}(W) + \frac{1-q}{2mq}\sum_{i=1}^n\sum_{r=1}^m \phi(w_r^\top x_i)^2 \\
    = & ~ \frac{1}{2}\|\hat{\Phi}_{W}\overline{W} - Y\|_{2}^{2} + \frac{\lambda}{2}\overline{W}^{\top}\hat{\Psi}_{W}\overline{W}
\end{align*}
and finish the proof.
\end{proof}

\begin{remark}
A classical kernel ridge regression problem can be defined as
\begin{align*}
    \min_{W} \frac{1}{2} \|\phi(X)^\top W - Y\|_2^2 + \frac{\lambda}{2} \|W\|_2^2
\end{align*}
where $\phi: \R^d \to \mathcal{F}$ is a feature map. Note that Lemma~\ref{lem:explicit_regularization_2} breaks the dropout loss into two parts: the first part is an error term, and the second part can be seen as a regularization term. 
Thus the task of minimizing the dropout loss $L(W)$ is equivalent to a kernel ridge regression (KRR) problem. 
\end{remark}






\newpage
\section{Dynamics of Kernel Methods (Continuous Gradient Flow)}\label{sec:gradient_flow}

The NTK also allows us to analyze the training convergence of sparse networks.
We show that gradient descent converges globally when training wide sparse networks.
This convergence speed is similar to that of dense models~\citep{dzps19,als19_dnn}.

In this section we will discuss the dynamics of kernel method under the mask $\sigma$, which adds sparsity in the output layer. Our problem will be considered in over-parameterized scheme.
First we introduce some additional definitions and notations. We define symmetric Gram matrix $G(W)$ as $G(W) := \hat{\Phi}_W\cdot\hat{\Phi}_W^\top \in \R^{n\times n}$. For all $i, j\in [n] \times [n]$, we have
\begin{align*}
    G(W)_{ij} = \frac{1}{m}\sum_{r=1}^m a_r^2 \mathbf{1}_{\langle w_r, x_i \rangle\geq 0, \langle w_r, x_j \rangle\geq 0} x_i^\top x_j = \frac{1}{m} x_i^\top x_j \sum_{r=1}^m \mathbf{1}_{\langle w_r, x_i \rangle\geq 0, \langle w_r, x_j \rangle\geq 0}.
\end{align*}
We define block symmetric matrix $H(W)$ as $H(W) = \hat{\Phi}_W^\top \cdot \hat{\Phi}_W\in\R^{md\times md}$. Then for all $i, j\in [m] \times [m]$, the $(i,j)$-th block of $H(W)$ is
\begin{align*}
    H(W)_{ij} = \frac{1}{m} a_i a_j \sum_{k=1}^n x_k x_k^\top \cdot \mathbf{1}_{\langle w_i, x_k \rangle\geq 0, \langle w_j, x_k \rangle\geq 0} \in \R^{d\times d}.
\end{align*}


By using Lemma~\ref{lem:explicit_regularization_2}, we consider the corresponding kernel regression problem: 
\begin{align}
\label{eq:sc1}
    \min_{W}L_{k}(W) = \min_{W}\frac{1}{2}\|\hat{\Phi} \overline{W} - Y\|_2^2 + \frac{\lambda}{2}\overline{W}^{\top}\hat{\Psi} \overline{W}
\end{align}
where $\hat{\Phi} \in \R^{n \times md}$, $\overline{W}\in \R^{md\times 1}$, $Y\in \R^{n\times 1}$ and $\hat{\Psi}\in \R^{md\times md}$. The main difference from neural network is that we assume $\hat{\Phi}$ (related to NTK, e.g., see Definition~\ref{def:Phi_first_time}) and $\hat{\Psi}$ (related to regularization term, e.g., see Definition~\ref{def:Psi_first_time}) do not change during the training process. 






The gradient of $L_k$ can be expressed as
\begin{align}
\label{eq:nabla_W}
\nabla_{\overline{W}} L_k(W) = \hat{\Phi}^{\top}\hat{\Phi} \overline{W} - \hat{\Phi}^{\top} Y + \lambda\hat{\Psi} \overline{W}.
\end{align}
We use $\overline{W^{\star}}$ to denote the optimal solution of Eq.~\eqref{eq:sc1}, and it satisfies 
\begin{align}
\label{eq:st2}
    \nabla_{\overline{W}} L_k(W)\big|_{\overline{W} = \overline{W^{\star}}} = (\hat{\Phi}^{\top}\hat{\Phi} + \lambda \hat{\Psi})\overline{W^{\star}} - \hat{\Phi}^{\top}Y = 0.
\end{align}
Since $\hat{\Psi}$ is a positive diagonal matrix, $\hat{\Phi}^{-\frac{1}{2}}$ exists, thus we have
\begin{align*}
    \overline{W^{\star}} = & ~ (\hat{\Phi}^\top \hat{\Phi} + \lambda \hat{\Psi})^{-1} \hat{\Phi}^\top Y.
\end{align*}
Next, we consider the question from a continuous gradient flow aspect. In time $t$, we denote $\overline{W}(t) = \mathrm{vec}(W(t)), \hat{\Phi}(t) = \hat{\Phi}_{W(t)}, \hat{\Psi}(t) = \hat{\Psi}_{W(t)}$. We also denote $G(t) = G(W(t))$ and $H(t) = H(W(t))$. Following the literature of \cite{dzps19}, we consider the ordinary differential equation defined by
\begin{align}
    \label{eq:ode_flow}
    \frac{\d w_r(t)}{\d t} = - \frac{\partial L_k(W(t))}{\partial w_r(t)}.
\end{align}

\begin{lemma}[Lemma 3.1 in \cite{dzps19}]\label{lem:dzps3.1}
If $m = \Omega(\frac{n^2}{\lambda_0^2}\log(\frac{n}{\delta}))$, we have with probability at least $1-\delta$, $\|G(0) - G^{\infty}\|_2 \leq \frac{\lambda_0}{4}$ and $\lambda_{\min}(G(0))\geq \frac{3}{4}\lambda_0$.
\end{lemma}

\begin{lemma}[Lemma 3.2 in \cite{dzps19}]\label{lem:dzps3.2}
If $w_1,\cdots, w_m$ are i.i.d generated from $\mathcal{N}(0,I_d)$, then with probability at least $1-\delta$, the following holds. For any set of weight vectors $w_1,\cdots,w_m \in\R^d$ that satisfy for any $r\in [m], \|w_r - w_r(0)\|_2\leq \frac{c\delta \lambda_0}{n^2}$ for some small positive constant $c$, then matrix $G\in\R^{d\times d}$ satisfies $\|G - G(0)\|_2 < \frac{\lambda_0}{4}$ and $\lambda_{\min}(G) > \frac{\lambda_0}{2}$.
\end{lemma}
The above lemma shows that for $W$ that is close to $W(0)$, the Gram matrix $G$ also stays close to the initial Gram matrix $G(0)$, and its minimal eigenvalue is lower bounded. 

\begin{lemma}[Gradient Flow]\label{lem:gradient_flow}
If we assume $\lambda_{\min}(\hat{\Psi}) \geq \Lambda_0 > 0$, then with probability at least $1 - \delta$, for $w_1,\cdots,w_m \in\R^d$ that satisfy $\forall r\in [m], \|w_r - w_r(0)\|_2\leq \frac{c\delta \lambda_0}{n^2}$, we have
\begin{align*}
    \frac{\d\|\hat{\Phi} \overline{W} -\hat{\Phi} \overline{W^{\star}} \|_{2}^{2}}{\d t} \leq -\gamma \|\hat{\Phi} \overline{W} -\hat{\Phi} \overline{W^{\star}}\|_{2}^{2}
\end{align*}
holds some constant $\gamma > 0$.
\end{lemma}
\begin{proof}
By using Eq.~\eqref{eq:nabla_W} and Eq.~\eqref{eq:ode_flow}, we can express $\frac{\d \overline{W}}{\d t}$ as
\begin{align}
    \label{eq:d_W_bar}
    \frac{\d \overline{W}}{\d t} = - \nabla_{\overline{W}} L_k(W) = -( \hat{\Phi}^{\top}\hat{\Phi} \overline{W} - \hat{\Phi}^{\top} Y + \lambda\hat{\Psi} \overline{W}).
\end{align}
Then we have 
\begin{align}\label{eq:d_Phi_W}
    & ~ \frac{\d \|\hat{\Phi} \overline{W} -\hat{\Phi} \overline{W^{\star}} \|_{2}^{2}}{\d  t} \notag \\
    = & ~ \frac{\d \|\hat{\Phi} \overline{W} -\hat{\Phi} \overline{W^{\star}} \|_{2}^{2}}{\d  \overline{W}}\cdot\frac{\d \overline{W}}{\d t} \notag \\
    = & ~ 2(\hat{\Phi} \overline{W} - \hat{\Phi} \overline{W^{\star}})^{\top}\hat{\Phi} \cdot (- ( \hat{\Phi}^{\top}\hat{\Phi} \overline{W} - \hat{\Phi}^{\top}Y + \lambda\hat{\Psi} \overline{W})) \notag \\
    = & ~ -2 (\hat{\Phi} \overline{W} - \hat{\Phi} \overline{W^{\star}})^{\top}\hat{\Phi} ( \hat{\Phi}^{\top}\hat{\Phi} \overline{W} - \hat{\Phi}^{\top}Y + \lambda\hat{\Psi} \overline{W}) \notag \\
    = & ~ -2 (\hat{\Phi} \overline{W} - \hat{\Phi} \overline{W^{\star}})^{\top}\hat{\Phi} (\hat{\Phi}^{\top}\hat{\Phi} \overline{W} - \hat{\Phi}^{\top}\hat{\Phi} \overline{W^{\star}} - \lambda\hat{\Psi} \overline{W^{\star}} + \lambda\hat{\Psi} \overline{W}) \notag \\
    = & ~ -2 (\hat{\Phi} \overline{W} - \hat{\Phi} \overline{W^{\star}})^{\top}\hat{\Phi}\hat{\Phi}^{\top}(\hat{\Phi} \overline{W} - \hat{\Phi} \overline{W^{\star}}) - 2\lambda(\hat{\Phi} \overline{W} - \hat{\Phi} \overline{W^{\star}})^{\top}\hat{\Phi} (\hat{\Psi} \overline{W} - \hat{\Psi} \overline{W^{\star}}) \notag \\
    \leq & ~ -2 \lambda_0 \|\hat{\Phi} \overline{W} - \hat{\Phi} \overline{W^{\star}}\|_2^2 - 2\lambda (\overline{W} - \overline{W^{\star}})^{\top}\hat{\Phi}^{\top} \hat{\Phi} \hat{\Psi} (\overline{W} - \overline{W^{\star}})
\end{align}
where the second step follows from Eq.~\eqref{eq:d_W_bar}, the fourth step follows from Eq.~\eqref{eq:st2}, and the last step follows from the definition that $\lambda_0 = \lambda_{\min}(G) = \lambda_{\min}(\hat{\Phi} \hat{\Phi}^\top)$.

As for the second term in the Eq.~\eqref{eq:d_Phi_W}, we have
\begin{align}
    \label{eq:second_term}
    & ~ 2\lambda (\overline{W} - \overline{W^{\star}})^{\top}\hat{\Phi}^{\top} \hat{\Phi} \hat{\Psi} (\overline{W} - \overline{W^{\star}}) \notag \\
    = & ~ 2\lambda (\overline{W}\hat{\Phi}^{\top} \hat{\Phi} - \overline{W^{\star}}\hat{\Phi}^{\top} \hat{\Phi})^{\top} \hat{\Psi} (\overline{W} - \overline{W^{\star}}) \notag \\
    \geq & ~ 2\lambda \Lambda_0 (\overline{W} - \overline{W^{\star}})^{\top}\hat{\Phi}^{\top} \hat{\Phi} (\overline{W} - \overline{W^{\star}}) \notag \\
    = & ~ 2\lambda \Lambda_0 \|\hat{\Phi} \overline{W} - \hat{\Phi}\overline{W^{\star}}\|_2^2
\end{align}
Thus by Eq.~\eqref{eq:d_Phi_W} and Eq.~\eqref{eq:second_term} we have
\begin{align*}
     \frac{\d \|\hat{\Phi} \overline{W} -\hat{\Phi} \overline{W^{\star}} \|_{2}^{2}}{\d t} \leq -(2\lambda_0 + 2\lambda \Lambda_0) \|\hat{\Phi} \overline{W} - \hat{\Phi}\overline{W^{\star}}\|_2^2.
\end{align*}
By letting $\gamma = 2\lambda_0 + 2\lambda \Lambda_0$ we finish the proof.







\end{proof}

For convenience, we denote $u(t) = \hat{\Phi}(t) \cdot \overline{W}(t) \in \R^n$. Then it is easy to verify that
\begin{align*}
    u_i(t) = \frac{1}{\sqrt{m}}\sum_{r=1}^m a_r \phi(w_r^\top x_i) = f(W(t), x_i), ~~~\forall i\in [n],
\end{align*}
showing that $u(t)$ is the prediction in time $t$. %

\begin{lemma}[Convergence rate]\label{lem:gradient_flow_2}
If we assume $\lambda_{\min}(G(s))\geq \frac{\lambda_0}{2}$ holds for $0\leq s\leq t$, then we have
\begin{enumerate}
    \item $\|u(t) - Y\|_2^2 \leq e^{-(\lambda_0+2\lambda/m) t} \|u(0) - Y\|_2^2;$
    \item $\forall r\in [m], \|w_r(t) - w_r(0)\|_2 \leq \frac{\sqrt{n}\|u(0) - Y\|_2}{\lambda_0 \sqrt{m}}.$
\end{enumerate}
\end{lemma}
\begin{proof}
From Eq.~\eqref{eq:nabla_W}, we can express the dynamics by using $u(t)$ as
\begin{align}
    \label{eq:d_u}
    \frac{\d u(t)}{\d t} = & ~ - \hat{\Phi} (\hat{\Phi}^{\top}\hat{\Phi} \overline{W} - \hat{\Phi}^{\top} Y + \lambda\hat{\Psi} \overline{W}) \notag \\
    = & ~ G(t) (Y - u(t)) - \lambda \hat{\Phi} \hat{\Psi} \overline{W}.
\end{align}
Thus we have
\begin{align}
    \label{eq:d_y_u}
    \frac{\d \|u(t) - Y\|_2^2}{\d t} = & ~ 2(u(t) - Y)^\top \big(G(t) (Y - u(t)) - \lambda \hat{\Phi} \hat{\Psi} \overline{W} \big) \notag \\
    = & ~ -2 (u(t) - Y)^\top G(t) (u(t) - Y) - 2\lambda (u(t) - Y)^\top\hat{\Phi} \hat{\Psi} \overline{W} \notag \\
    \leq & ~ - \lambda_0 \|u(t) - Y\|_2^2- 2\lambda (u(t) - Y)^\top\hat{\Phi} \hat{\Psi} \overline{W}.
\end{align}
As for the second term, we have
\begin{align}
    \label{eq:phi_psi_W}
    2\lambda (u(t) - Y)^\top\hat{\Phi} \hat{\Psi} \overline{W} = & ~ \frac{2\lambda}{m} (u(t) - Y)^\top\hat{\Phi}\cdot [\hat{\psi}_1\cdot w_1, \cdots, \hat{\psi}_m\cdot w_m]^\top \notag \\
    = & ~ \frac{2\lambda}{m} (u(t) - Y)^\top\hat{\Phi}\cdot [\sum_{i=1}^n x_i \phi(w_1^\top x_i), \cdots, \sum_{i=1}^n x_i \phi(w_m^\top x_i)]^\top \notag \\
    = & ~ \frac{2\lambda}{m} (u(t) - Y)^\top \cdot [U_1(t),\cdots, U_n(t)]^\top
\end{align}
where for $j\in [n]$, $U_j(t)\in\R$ can be expressed as
\begin{align*}
    U_j(t) = & ~ \frac{1}{\sqrt{m}}\sum_{r=1}^m \big(a_r \mathbf{1}_{\langle w_r, x_j \rangle \geq 0}x_j^\top \cdot \sum_{i=1}^n x_i \phi(w_r^\top x_i)\big) \\
    = & ~ \frac{1}{\sqrt{m}}\sum_{r=1}^m \sum_{i=1}^n a_r x_j^\top (x_i x_i^\top) w_r \cdot \mathbf{1}_{\langle w_r, x_i \rangle \geq 0, \langle w_r, x_j \rangle \geq 0} \\
    = & ~ \frac{1}{\sqrt{m}}\sum_{r=1}^m \Big(a_r x_j^\top w_r \cdot \mathbf{1}_{\langle w_r, x_j \rangle \geq 0}\cdot \sum_{i=1}^n \mathbf{1}_{\langle w_r, x_i \rangle \geq 0}\Big). %
\end{align*}
We denote $U(t) = [U_1(t),\cdots, U_n(t)]^\top\in\R^n$ and have
\begin{align}
    \label{eq:U}
    2\lambda (u(t) - Y)^\top\hat{\Phi} \hat{\Psi} \overline{W} = \frac{2\lambda}{m} (u(t) - Y)^\top \cdot U(t)
\end{align}
and our dynamics becomes
\begin{align}
    \label{eq:new_ode}
    \frac{\d \|u(t) - Y \|_2^2}{\d t} \leq & ~ -\lambda_0 \|u(t) - Y\|_2^2 - \frac{2\lambda}{m} (u(t) - Y)^\top \cdot U(t) \notag \\
    \leq & ~ -(\lambda_0 + \frac{2\lambda}{m}) \|u(t) - Y\|_2^2
\end{align}
showing that $\frac{\d}{\d t}\big(e^{(\lambda_0+2\lambda/m) t}\|u(t) - Y\|_2^2 \big) \leq 0$. Thus $e^{(\lambda_0+2\lambda/m) t}\|u(t) - Y\|_2^2$ is a decreasing function with respect to $t$, and we have
\begin{align*}
    \|u(t) - Y\|_2^2 \leq e^{-(\lambda_0+2\lambda/m) t} \|u(0) - Y\|_2^2.
\end{align*}
As for bounding $\|w_r(t) - w_r(0)\|_2$, we use the same method as in Lemma 3.3 of \cite{dzps19}. Thus we complete the proof.
\end{proof}

Finally, by combining Lemma~\ref{lem:dzps3.1}, \ref{lem:dzps3.2}, \ref{lem:gradient_flow} and \ref{lem:gradient_flow_2}, we have the following convergence result.
\begin{theorem}[Convergence of gradient flow]\label{thm:convergence_1}
Suppose $\lambda_0>0$, $m = \poly(n, 1/ \lambda_0, 1/ \delta)$, then with probability at least $1-\delta$ over the randomness of initialization, we have
\begin{align*}
    \|u(t) - Y\|_2^2 \leq e^{-(\lambda_0 + 2\lambda/m)t} \|u(0) - Y\|_2^2.
\end{align*}
\end{theorem}
The above theorem shows that in the over-parameterized setting (when $m$ is large enough), the training loss of the kernel ridge regression problem define in Eq.~\eqref{eq:sc1} converges to $0$ in a linear rate. By comparing our Theorem~\ref{thm:convergence_1} with Theorem 3.2 in \cite{dzps19}, we can find that the introducing of regularization term makes the convergence speed faster, though the improvement is limited. Further notice that in Section~\ref{sec:dropout_KRR} we prove the equivalence between minimizing the dropout loss and the kernel ridge regression problem. So we conclude our results as:
\begin{center}
    \emph{The introducing of sparsity into neural network makes the convergence speed faster, but the improvement is limited due to the over-parameterized scheme.}
\end{center}


\newpage
\section{Method Details}
\label{sec:appx_method_details}

We describe some details of our method.
\subsection{Compute budget allocation}

We describe here a procedure to compute the budget allocation
based on our cost model.
This procedure is more complicated than our simple rule of thumb in
\cref{sec:method}, and tend to produce the same allocation.
For completeness, we include the procedure here for the interested reader.

Given a parameter budget $B$, we find the density of each layer type that
minimize the models' total cost of matrix multiplication.
For example, in Transformers, let $d_a$ and $d_m$ be the density of the
attention and the MLP layers.
Let $s$ be the sequence length and $d$ be the feature size.
The attention layer with density $d_a$ will cost $d_a (n^2 + nd)$, and the fully
connected layers with density $d_m$ will cost $2 d_m nd$.
We then set $d_a$ and $d_m$ to minimize the total cost while maintaining the
parameter budget:
\begin{equation}\label{eq:budget}
  \text{minimize}_{\delta_a, \delta_m} \delta_a (n^2 + nd) + 2 \delta_m n d \quad
  \text{subject to} \quad \text{$\#$ of trainable parameters} \leq B.
\end{equation}
As this is a problem with two variables, we can solve it in closed form.

\subsection{Low-rank in Attention}

In \cref{sec:method}, we describe how to use the sparsity pattern from flat
block butterfly and the low-rank term for weight matrices.
This applies to the linear layer in MLP and the projection steps in the
attention.

We also use the sparse + low-rank structure in the attention step itself.
\citet{scatterbrain} describes a general method to combine sparse and low-rank
attention, where one uses the sparse component to discount the contribution from
the low-rank component to ensure accurate approximation of the attention matrix.

We follow a simpler procedure, which in practice yields similar performance.
We use a restricted version of low-rank of the form a ``global'' sparsity mask
(as shown in \cref{fig:block_sparse_visualization}).
Indeed, a sparse matrix whose sparsity pattern follows the ``global'' pattern is
a sum of two sparse matrices, one containing the ``horizontal'' global components
and one containing the ``vertical'' components.
Let $w$ be the width of each of those components, then each of them has rank at
most $w$.
Therefore, this sparse matrix has rank at most $2w$, and is low-rank (for small $w$).

We also make the global component block-aligned (i.e., set $w$ to be a multiple
of the smallest supported block size such as 32) for hardware efficiency.

\subsection{Comparison to Other Sparsity Patterns for Attention}

In the context of sparse attention, other sparsity patterns such as BigBird and
Longformer also contain a ``global'' component, analogous to our low-rank
component.
Their ``local'' component is contained in the block diagonal part of the flat
block butterfly sparsity pattern.

The main difference that we do not use the random components (e.g., BigBird),
and the diagonal strides from flat block butterfly are not found in BigBird or
Longformer.
Moreover, we apply the same sparsity pattern (+ low-rank) to the linear layers
in the MLP and the projection step in attention as well, allowing our method to
target most neural network layers, not just the attention layer.

\subsection{Sparsity Mask for Rectangular Matrices}

We have described the sparsity masks from flat block butterfly for square
matrices.
For rectangular weight matrices, we simply ``stretch'' the sparsity mask.
The low-rank component applies to both square and rectangular matrices (as shown in~\cref{fig:rec}).
We have found this to work consistently well across tasks.
\begin{figure}[ht]
  \centering
  \includegraphics[width=0.7\linewidth]{figs/rec_butterfly.pdf}
  \caption{\label{fig:rec} Sparsity Mask for Rectangular Matrices.}
\end{figure}



\newpage
\section{Benchmarking of Butterfly Multiply}
\label{sec:appx_benchmark}

We validate that flat butterfly matrices (sum of factors) can speed up multiplication on GPUs
compared to butterfly matrices (products of factors).

Consider the matrix $M \in \mathbb{R}^{n \times n}$ that can be written as products of butterfly factors of
strides of up $k$ (a power of 2), with residual connection:
\begin{equation*}
  M = (I + \lambda \vB_k^{(n)}) (I + \lambda \vB_{k/2}^{(n)}) \dots (I + \lambda \vB_2^{(n)}).
\end{equation*}
The first-order approximation of $M$ has the form of a flat butterfly matrix
with maximum stride $k$ (\cref{sec:flat_butterfly}):
\begin{equation*}
  M_\mathrm{flat} = I + \lambda (\vB_2^{(n)} + \dots + \vB_{k/2}^{(n)} + \vB_k^{(n)}).
\end{equation*}

Notice that $M$ is a product of $\log_2 k$ factors, each has $2n$ nonzeros, so
multiplying $M$ by a input vector $x$ costs $O(n \log k)$ operations (by
sequentially multiplying $x$ by the factors of $M$).
The flat version $M_\mathrm{flat}$ is a sparse matrix with $O(n \log k)$
nonzeros as well, and the cost of multiplying $M_\mathrm{flat} x$ is also
$O(n \log k)$.
However, in practice, multiplying $M_\mathrm{flat} x$ is much more efficient on
GPUs than multiplying $Mx$ because of the ease of parallelization.

We measure the total time of forward and backward passes of multiplying either
$M_\mathrm{flat} x$ and compare to that of multiplying $Mx$ for different
maximum strides, as shown
in~\cref{fig:flat_butterfly_speed}.
We see that ``flattening'' the products brings up to 3$\times$ speedup.
\begin{figure}[ht]
  \centering
  \includegraphics[width=0.6\linewidth]{figs/flat_butterfly_speed.pdf}
  \caption{\label{fig:flat_butterfly_speed}Speedup of multiplying
    $M_\mathrm{flat}x$ compared to multiplying $Mx$. Flattening the products
    yields up 3$\times$ speedup.}
\end{figure}

We use matrix size $1024 \times 1024$ with block size 32.
The input batch size is 2048.
We use the block sparse matrix multiply library from
\url{https://github.com/huggingface/pytorch_block_sparse}.
The speed measurement is done on a V100 GPU.



\newpage
\begin{figure}[t]
	\begin{center}
	\scriptsize
		\begin{tabular}{c}
			\includegraphics[width=\linewidth]{figs/candidates.pdf}
		\end{tabular}
	\end{center}
	\caption{Sparsity pattern candidate components:  Local corresponds to local interaction of neighboring elements; Global (low-rank) involves the interaction between all elements and a small subset of elements; Butterfly captures the interaction between elements that are some fixed distance apart; Random is common in the pruning literature.}
	\label{fig:block_sparse_visualization} 
\end{figure}

\section{Exhausted Searching Sparsity Patterns for Efficient Sparse Training}
\label{sec:appx_ntk_algorithm}
We describe here our early exploration of searching among different sparsity patterns that has been proposed in the literature.
We use a metric derived from the NTK, which has emerged as one of the standard metric to predict the training and generalization of the model.
We consistently found the butterfly + low-rank pattern to perform among the best.

In~\cref{sec:challenges}, we describe the challenges of selecting sparsity patterns for every model components using the a metric derived from the NTK, followed by our approaches.
Then in , we describe details of empirical NTK computation, which is an important step in our method implementation. 
Last, in~\cref{sec:property}, we highlight important properties of our method -- it rediscovers several classical sparsity patterns, and the sparse models can inherit the training hyperparamters of the dense models, reducing the need for hyperparameters tuning.

\subsection{Challenges and Approaches}
\label{sec:challenges}

\textbf{Challenge 1:} We seek sparsity patterns for each model components that can closely mimic the training dynamics of the dense counterpart. As mentioned in~\cref{thm:mask_regression}, it is NP-hard to find the optimal sparse matrix approximation. Although NTK provides insights and measurement on the ``right'' sparse model, bruteforcely computing NTK for one-layer models with all sparsity patterns is still infeasible.

\textbf{Approach 1: Sparsity Pattern Candidates.}
To address the above challenge, we design our search space to be a limited set of sparsity pattern candidates, each is either a component visualized in \cref{fig:block_sparse_visualization} or the combination of any two of them.
These components encompass the most common types of sparsity pattern used, and can express
We provide the intuition behind these sparsity components:
\begin{itemize}[leftmargin=*,nosep,nolistsep]
  \item Local: this block-diagonal component in the matrix corresponds to local interaction of neighboring elements. This has appeared in classical PDE discretization \citep{collins1971diagonal}, and has been rediscovered in Longformer and BigBird attention patterns.
  \item Global: this component involves interaction between all elements and a small subset of elements (i.e., ``global'' elements).
  This global pattern is low-rank, and this sparse + low-rank structure is common in data science~\citep{udell2019big}, and rediscovered in Longformer and BigBird patterns as well.
  \item Butterfly: this component corresponds to interaction between elements that are some fixed distance apart.
  The many divide-and-conquer algorithms, such as the classical fast Fourier transform~\citep{cooley1965algorithm}, uses this pattern at each step. Butterfly matrices reflects this divide-and-conquer structure, and hence this sparsity component. The sparse transformer~\citep{child2019generating} also found this pattern helpful for attention on image data.
  \item Random: this component is a generalization of sparsity patterns found in one-shot magnitude, gradient, or momentum based pruning~\citep{lee2018snip}. Note that at network initialization, they are equivalent to random sparsity.
\end{itemize}

\textbf{Challenge 2:} Even with a fixed pool of sparsity patterns for each layer, if the model has many layers, the number of possible layer-pattern assignments is exponentially large.

\textbf{Approach 2:} To further reduce the search space,
we constrain each layer type (attention, MLP) to have the same sparsity pattern.
For example, if there are 10 patterns and 2 layer types, the candidate pool is $10^2 = 100$ combinations.

\textbf{Challenge 3:} Computing the empirical NTK on the whole dataset is expensive in time and space, as it scales quadratically in the dataset size.

\textbf{Approach 3:} We compute the empirical NTK on a randomly chosen subset of the data (i.e., a principal submatrix of the empirical NTK matrix).
In our experiments, we verify that increasing the subset size beyond 1000 does not change the choices picked by the NTK heuristic.
The subsampled empirical NTK can be computed within seconds or minutes.

\subsection{Algorithm Description}
\label{sec:algorithm_description}












\begin{algorithm}[t]
{\small
    \begin{algorithmic}[1]
        \State \textbf{Input: model schema $\Omega$, compute budget $B$, dataset subset $X$, sparsity mask candidate set $C$.}
        \State $K_{dense} \leftarrow$ \textsc{NTK}$(f_\theta, X)$. \Comment{\cref{eq:empirical_ntk}}
        \State output sparsity mask assignment $s_\mathrm{out}$, $d_{min} \leftarrow \inf$
        \For {$M_1, \dots, M_{|\Omega|} \in C^{|\Omega|}$} \Comment{Enumerate all sparsity mask candidate combinations}
            \State Let $s$ be the sparsity mask assignment $(t_i, r_i, m_i, n_i) \to M_i$.
            \If {$\text{TotalCompute}(s) < B$} \Comment{\cref{eq:budget}, Check if masks satisfy budget constraint}
                \State Let $M_s$ be the flattened sparse masks
                \State $K_{sparse} \leftarrow \textsc{NTK}(f_{\theta \circ M_s}, X)$
                \State $d_s \leftarrow \textsc{Distance}(K_{dense}, K_{sparse})$ \Comment{\cref{eq:empirical_ntk}}
                \If{$d_{min}>d_s$}
                    \State $d_{min} \leftarrow d_s$, $s_\mathrm{out} \leftarrow s$
                \EndIf
            \EndIf
        \EndFor
        \State\Return $s_\mathrm{out}$ \Comment{Return sparsity mask assignment}
    \end{algorithmic}
    }
    \caption{Model Sparsification}\label{algo:pre}
    \end{algorithm}



Our method targets GEMM-based neural networks, which are networks whose computation is dominated by general matrix multiplies (GEMM), such as Transformer and MLP-Mixer.
As a result, we can view the network as a series of matrix multiplies.
We first define:
\begin{itemize}[leftmargin=*,nosep,nolistsep]
  \item Model schema: a list of layer types $t$ (e.g., attention, linear layers in MLP), number of layers $r$ of that type, and dimension of the matrix multiplies $m \times n$.
  We denote it as $\Omega = \{(t_1, r_1, m_1, n_1), \dots, (t_{|\Omega|}, r_{|\Omega|}, m_{|\Omega|}, n_{|\Omega|})\}$.
  \item A \emph{mask} $M$ of dimension $m \times n$ is a binary matrix $\{0, 1\}^{m \times n}$.
  The compute of a mask is the total number of ones in the matrix: $\mathrm{compute}(M) = \sum_{i, j} M_{ij}$.
  \item A \emph{sparsity pattern} $P_{m \times n}$ for matrix dimension $m \times n$ is a set of masks $\{M_1, ..., M_{|P|}\}$, each of dimension $m \times n$.
  \item A \emph{sparsity mask assignment} is a mapping from a model schema $\Omega$ to masks $M$ belonging to some sparsity pattern $P$: $s \colon (t, r, m, n) \to M$.
  \item Given a set of sparsity patterns $P_1, \dots, P_k$, the set of sparsity mask candidate $C$ is the union of sparsity masks in each of $P_i$: $C = \cup P_i$
  \item A sparsity pattern assignment $s$ satisfies the compute budget $B$ if:
\begin{equation}
\label{eq:budget}
  \mathrm{TotalCompute}(s) := \sum_{\text{layer type } l} \mathrm{compute}(s(t, r, m, n)) \le B.
\end{equation}
  \item Let $\theta$ be the flattened vector containing the model parameters, and let $M_s$ be the flattened vector containing the sparsity mask by the sparsity mask assignment $s$.
  Let $f_\theta(x)$ be the output of the dense network with parameter $\theta$ and input $x$.
  Then the output of the sparse network is $f_{\theta \circ M_s}(x)$.
  \item The empirical NTK of a network $f_\theta$ on a data subset $X = \{x_1, \dots, x_{|X|}\}$ is a matrix of size $|X| \times |X|$:
\begin{equation}
  \label{eq:empirical_ntk}
  \mathrm{NTK}(f_\theta, X)_{i, j} = \left \langle \frac{\partial f_\theta(x_i)}{\partial \theta}, \frac{\partial f_\theta(x_j)}{\partial \theta} \right \rangle.
\end{equation}
\end{itemize}

The formal algorithm to assign the sparsity mask to each layer type is described in \cref{algo:pre}.
The main idea is that, as the set of sparsity mask candidate is finite, we can enumerate all possible sparsity mask assignment satisfying the budget and pick the one with the smallest NTK distance to the dense NTK.
In practice, we can use strategies to avoid explicitly enumerating all possible sparsity mask, e.g. for each sparsity pattern, we can choose the largest sparse mask that fits under the budget.









\subsection{Method Properties: Rediscovering Classical Sparsity Patterns, No Additional Hyperparameter Tuning}
\label{sec:property}
When applied to the Transformer architecture, among the sparsity components described in \cref{sec:challenges}, the NTK-guided heuristic consistently picks the local and global components for \emph{both} the attention and MLP layers.
Moreover, the butterfly component is also consistently picked for image data, reflecting the 2D inductive bias in this component\footnote{Convolution (commonly used in image data) can be written in terms of the fast Fourier transform, which has this same sparse pattern at each step of the algorithm}.
While some of these patterns have been proposed for sparse attention, it is surprising that they are also picked for the MLP layers.
The most popular type of sparsity pattern in MLP layers is top-k (in magnitude or gradient, which at initialization is equivalent to random sparsity).
We have proved that lower NTK difference results in better generalization bound for the sparse model. As expected, we observe that this allows the sparse model to use the same hyperparamters (optimizer, learning rate, scheduler) as the dense model (\cref{sec:experiments}).







\newpage
We provide the details of our experiments on two well-known instruction tuning datasets: NIV2~\cite{Wang2022SuperNaturalInstructionsGV} and Self-Instruct dataset~\cite{wang2022self}. 
\paragraph{NIV2 - Active Instruction Tuning Details}
We utilize the NIV2 English tasks split, comprising 756 training tasks and 119 testing tasks, including classification and generative tasks.
We employ five random seeds without selection in our active instruction tuning experiment. Each seed involves randomly sampling 68 tasks as initial training tasks and 68 tasks as validation tasks. The remaining 620 training tasks form the remaining task pool. 
In each active learning iteration, we maintain a fixed classification and generative task ratio and select 24 classification tasks and 44 generative tasks using different task selection strategies. This fixed ratio allows a more meaningful comparison of our results as we evaluate overall, classification, and generative task scores separately. After the new tasks are sampled, we add them to the previously selected training tasks and form a new training task set. We further train a new instruction tuning model with the updated training task set.
\paragraph{Self-Instruct - Active Instruction Tuning Details}
We utilize the 52K self-instruct dataset as the task pool. For the active instruction tuning experiment, we will randomly sample 500 tasks as the initial training set and further compare model performance at $[1000, 2000, 4000, 8000, 16000]$ training tasks. For task selection, we will first divide all tasks into 13 chunks by output sequence length $[[1,10], [11,20], ..., [121, 130]]$, and then apply the task selection methods on each chunk of tasks, following the ratio of the number of tasks in all chunks. We conduct this extra step to normalize the output sequence length of the selected task for each task selection method. This ensures there is no imbalance in output sequence length during task selection.
\paragraph{Training Details}
For experiments on NIV2 dataset~\cite{Wang2022SuperNaturalInstructionsGV}, we follow the TK-instruct setting, the SOTA model on the NIV2 dataset to train the T5-770M model~\cite{raffel2020exploring} with learning rate 2e-5, batch size 128 and 200 instances per task for eight epochs. We evaluate the model's zero-shot performance on the validation set at each epoch and select the model checkpoint with the best validation score. For evaluation, we follow ~\cite{kung2023models} setting to report the Rouge-L score of \textit{Overall}, \textit{Classification}, and \textit{Generative} tasks on both validation and testing sets. For experiments on Self-Instruct dataset~\cite{wang2022self}, We follow Alpaca's settings to train the LLaMA-7B model with learning rate 2e-5, batch size 128 for four epochs.

\paragraph{Computing Resources}
For the experiment on NIV2 dataset~\cite{Wang2022SuperNaturalInstructionsGV}, we conduct our experiments using 4 to 8 Nvidia 48GB A6000 GPUs. For each uncertainty method, it takes around 1200 GPU hours, a total of 5000 GPU hours(for a single GPU), to run all experiments for \autoref{fig:niv2-results}. For the experiment on Self-Instruct dataset~\cite{wang2022self}, we run with 2 Nvidia 80GB A100 GPUs. Each uncertainty method takes around 40 GPU hours, which sums to 160 GPU hours for all experiments in \autoref{fig:alpaca-results}.

\newpage



\section{Extended Related Work}
\label{app:related}
In this section, we extend the related works referenced in the main paper and discuss them in detail.

\subsection{Neural Pruning} 
Our work is loosely related to neural network pruning. By iteratively eliminating neurons and connections, pruning has seen great success in compressing complex models.\citet{han2015deep,han2015learning} put forth two naive but effective algorithms to compress models up to 49x and maintain comparable accuracy. \citet{li2016pruning} employ filter pruning to reduce the cost of running convolution models up to 38 $\%$, \citet{NIPS2017_a51fb975} prunes the network at runtime, hence retaining the flexibility of the full model. \citet{dong2017learning} prunes the network locally in a layer by layer manner.  \citet{sanh2020movement} prunes with deterministic first-order information, which is more adaptive to pretrained model weights. \citet{lagunas2021block} prunes transformers models with block sparsity pattern during fine-tuning, which leads to real hardware speed up while maintaining the accuracy. \citet{zhu2017prune} finds large pruned sparse network consistently outperform the small dense networks with the same compute and memory footprints. Although both our and all the pruning methods are aiming to produce sparse models, we differ in our emphasis on the overall efficiency, whereas pruning mostly focuses on inference efficiency and disregards the cost in finding the smaller model.\\

\subsection{Lottery Ticket Hypothesis} 
Models proposed in our work can be roughly seen as a class of manually constructed lottery tickets. Lottery tickets \citet{frankle2018lottery} are a set of small sub-networks derived from a larger dense network, which outperforms their parent networks in convergence speed and potentially in generalization. A huge number of studies are carried out to analyze these tickets both empirically and theoretically: \citet{morcos2019one} proposed to use one generalized lottery tickets for all vision benchmarks and got comparable results with the specialized lottery tickets; \citet{frankle2019stabilizing} improves the stability of the lottery tickets by iterative pruning; \citet{frankle2020linear} found that subnetworks reach full accuracy only if they are stable against SGD noise during training; \citet{orseau2020logarithmic} provides a logarithmic upper bound for the number of parameters it takes for the optimal sub-networks to exist; \citet{pensia2020optimal} suggests a way to construct the lottery ticket by solving the subset sum problem and it's a proof by construction for the strong lottery ticket hypothesis. Furthermore, follow-up works \citep{liu2020finding, wang2020picking, tanaka2020pruning} show that we can find tickets without any training labels.\\

\subsection{Neural Tangent Kernel} 

Our work rely heavily on neural tangent kernel in theoretical analysis. Neural Tangent Kernel \citet{jacot2018neural} is first proposed to analyse the training dynamic of infinitely wide and deep networks. The kernel is deterministic with respect to the initialization as the width and depth go to infinity, which provide an unique mathematical to analyze deep overparameterized networks. Couples of theoretical works are built based upon this: \cite{lee2019wide} extend on the previous idea and prove that finite learning rate is enough for the model to follow NTK dynamic. \citet{arora2019exact} points out that there is still a gap between NTK and the real finite NNs. \citet{cao2020generalization} sheds light on  the good generalization behavior of overparameterized deep neural networks. \citet{arora2019fine} is the first one to show generalization bound independent of the network size. Later, some works reveal the training dynamic of models of finite width, pointing out the importance of width in training: \citet{hayou2019training} analyzes stochastic gradient from the stochastic differential equations' point of view; Based on these results, we formulate and derive our theorems on sparse network training.\\

\subsection{Overparameterized Models} 
Our work mainly targets overparameterized models. In \citet{nakkiran2019deep},  the double descendent phenomenon was observed. Not long after that, \cite{d2020triple} discover the triple descendent phenomenon. It's conjectured in both works that the generalization error improves as the parameter count grows. On top of that, \citet{arora2018optimization} speculates that overparameterization helps model optimization,
and without "enough" width, training can be stuck at local optimum. Given these intuitions, it's not surprising that the practitioning community is racing to break the record of the largest parameter counts: The two large language models, GPT-2 and GPT-3 \citep{radford2019language, brown2020language}, are pushing the boundary on text generation and understanding; Their amazing zero-shot ability earn them the title of foundation models \citep{bommasani2021opportunities}. On the computer vision side, \citet{dosovitskiy2020image, tolstikhin2021mlp, zhai2021scaling} push the top-1 accuracy on various vision benchmarks to new highs after scaling up to 50 times the parameters; \citet{naumov2019deep} shows impressive results on recommendation with a 21 billion large embedding; \citet{jumper2021highly} from DeepMind solve a 50 year old grand challenge in protein research with a 46-layer Evoformer. In our work, we show that there is a more efficient way to scale up model training through sparsification and double descent only implies the behavior of the dense networks.



\end{document}
