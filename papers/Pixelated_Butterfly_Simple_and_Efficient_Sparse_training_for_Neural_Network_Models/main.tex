\documentclass{article} %

\usepackage{etoolbox}
\newtoggle{arxiv}
\toggletrue{arxiv}

\iftoggle{arxiv}{}
{
\usepackage{iclr2022_conference,times}
}



\usepackage{amsmath}
\usepackage{amsthm}
\usepackage{amssymb}
\usepackage{hyperref}
\usepackage{url}
\usepackage{algpseudocode}
\usepackage{algorithm}

\usepackage[inline]{enumitem}
\usepackage{caption}

\usepackage{graphicx}
\usepackage{grffile}
\usepackage{natbib}
\usepackage{wrapfig,epsfig}
\usepackage{epstopdf}
\usepackage{algpseudocode}
\usepackage{multirow}
\usepackage[T1]{fontenc}
\usepackage{bbm}
\usepackage{comment}
\usepackage{dsfont}
\usepackage{makecell}
\usepackage{enumitem}
\usepackage{booktabs}

\usepackage{amsmath}
\usepackage{amsthm}
\usepackage{amssymb}
\usepackage{algorithm}
\usepackage{color}
\usepackage[english]{babel}
\usepackage{graphicx}
\usepackage{grffile}
\usepackage{natbib}
\usepackage{wrapfig,epsfig}
\usepackage{epstopdf}
\usepackage{algpseudocode}
\usepackage{multirow}
\usepackage[T1]{fontenc}
\usepackage{bbm}
\usepackage{comment}
\usepackage{dsfont}
\usepackage{makecell}
\usepackage{enumitem}
\usepackage{booktabs}
\usepackage{afterpage}

\usepackage[capitalize]{cleveref}
\usepackage{capt-of}




\newtheorem{theorem}{Theorem}[section]
\newtheorem{lemma}[theorem]{Lemma}
\newtheorem{definition}[theorem]{Definition}
\newtheorem{notation}[theorem]{Notation}
\newtheorem{proposition}[theorem]{Proposition}
\newtheorem{corollary}[theorem]{Corollary}
\newtheorem{conjecture}[theorem]{Conjecture}
\newtheorem{assumption}[theorem]{Assumption}
\newtheorem{observation}[theorem]{Observation}
\newtheorem{fact}[theorem]{Fact}
\newtheorem{remark}[theorem]{Remark}
\newtheorem{claim}[theorem]{Claim}
\newtheorem{example}[theorem]{Example}
\newtheorem{problem}[theorem]{Problem}
\newtheorem{open}[theorem]{Open Problem}
\newtheorem{property}[theorem]{Property}
\newtheorem{hypothesis}[theorem]{Hypothesis}
\newtheorem{process}{Process}
\algnewcommand\algorithmicforeach{\textbf{for each}}
\algdef{S}[FOR]{ForEach}[1]{\algorithmicforeach\ #1\ \algorithmicdo}

\newcommand{\rank}{\mathrm{rank}}
\newcommand{\wh}{\widehat}
\newcommand{\wt}{\widetilde}
\newcommand{\ov}{\overline}
\newcommand{\eps}{\epsilon}
\newcommand{\N}{\mathcal{N}}
\newcommand{\R}{\mathbb{R}}
\newcommand{\I}{\mathbb{I}}
\newcommand{\RHS}{\mathrm{RHS}}
\newcommand{\LHS}{\mathrm{LHS}}
\renewcommand{\d}{\mathrm{d}}
\renewcommand{\i}{\mathbf{i}}
\renewcommand{\varepsilon}{\epsilon}
\renewcommand{\tilde}{\wt}
\renewcommand{\hat}{\wh}
\newcommand{\ReLU}{{$\mathsf{ReLU}$}}
\newcommand{\new}{\mathrm{new}}
\newcommand{\nnz}{\mathrm{nnz}}
\newcommand{\diag}{\mathrm{diag}}
\newcommand{\poly}{\mathrm{poly}}
\newcommand{\norm}[1]{\left\|{#1}\right\|} %
\newcommand{\abs}[1]{\left\lvert#1\right\rvert}
\newcommand{\defeq}{:=}
\newcommand{\vB}{\mathbf{B}}
\newcommand{\vD}{\mathbf{D}}
\newcommand{\B}{\mathcal{B}}
\newcommand{\BS}{\B^*}
\newcommand{\BBS}{\B\B^*}
\newcommand{\BSB}{\B^*\B}

\DeclareMathOperator*{\E}{{\mathbb{E}}}

\newcommand{\Zhao}[1]{{\color{red} [Zhao: {#1}]}}
\newcommand{\Beidi}[1]{{\color{orange} [Beidi: {#1}]}}
\newcommand{\Jiaming}[1]{{\color{blue} [Jiaming: {#1}]}}
\newcommand{\Tri}[1]{{\color{cyan} [Tri: {#1}]}}


\newcommand*\samethanks[1][\value{footnote}]{\footnotemark[#1]}

\iftoggle{arxiv}{
  \setlength{\textwidth}{6.5in}
  \setlength{\textheight}{9in}
  \setlength{\oddsidemargin}{0in}
  \setlength{\evensidemargin}{0in}
  \setlength{\topmargin}{-0.5in}
  \newlength{\defbaselineskip}
  \setlength{\defbaselineskip}{\baselineskip}
  \setlength{\marginparwidth}{0.8in}
}{
\usepackage[compact]{titlesec}
\titlespacing{\section}{0pt}{*1}{*0}
\titlespacing{\subsection}{0pt}{*0}{*0}

\usepackage[subtle, mathdisplays=tight, charwidths=normal, leading=normal]{savetrees}

\addtolength\textwidth{0.15in}
\addtolength\textheight{0.15in}
\addtolength\textfloatsep{-0.5em}
\addtolength\intextsep{-0.2em}

\def\setstretch#1{\renewcommand{\baselinestretch}{#1}}
\setstretch{0.99}
\addtolength{\parskip}{-0.3pt}
}


\title{Pixelated Butterfly: Simple and Efficient Sparse Training for Neural Network Models}

\iftoggle{arxiv}{
  \usepackage{authblk}
  \author[$\dagger$]{Tri Dao\thanks{Equal contribution. Order determined by coin flip.}}
  \author[$\dagger$]{Beidi Chen\samethanks}
  \author[$\oplus$]{Kaizhao Liang}
  \author[$\diamond$]{Jiaming Yang}
  \author[$\S$]{Zhao Song}
  \author[$\ddagger$]{Atri Rudra}
  \author[$\dagger$]{Christopher R{\'e}}
  \affil[$\dagger$]{Department of Computer Science, Stanford University}
  \affil[$\oplus$]{SambaNova Systems, Inc}
  \affil[$\diamond$]{Department of Probability and Statistics, Peking University}
  \affil[$\S$]{Adobe Research}
  \affil[$\ddagger$]{Department of Computer Science and Engineering, University at Buffalo, SUNY\vspace{4pt}}
  \affil[ ]{\small{\texttt{\{trid,beidic\}@stanford.edu}, \texttt{kaizhao.liang@sambanovasystems.com}, \texttt{edwinyjmpku@gmail.com}, \texttt{zsong@adobe.com}, \texttt{atri@buffalo.edu}, \texttt{chrismre@cs.stanford.edu}}}
}{

\author{%
  Tri Dao\thanks{Equal contribution. Order determined by coin flip.}\, $^1$, Beidi
  Chen\samethanks\, $^1$, Kaizhao Liang $^2$, Jiaming Yang $^3$, Zhao Song $^4$,
  Atri Rudra $^5$, Christopher R\'{e} $^1$ \\
  $^1$ Department of Computer Science, Stanford University \\
  $^2$ SambaNova Systems, Inc \\
  $^3$ Department of Probability and Statistics, Peking University \\
  $^4$ Adobe Research \\
  $^5$ Department of Computer Science and Engineering, University at Buffalo, The State University of New York\\
  \texttt{\{trid,beidic\}@stanford.edu},
  \texttt{kaizhao.liang@sambanovasystems.com}, \\
  \texttt{edwinyjmpku@gmail.com}, \texttt{zsong@adobe.com}, \\ \texttt{atri@buffalo.edu}, \texttt{chrismre@cs.stanford.edu}
}

}

\newcommand{\fix}{\marginpar{FIX}}

\iftoggle{arxiv}{}{
\iclrfinalcopy %
}
\begin{document}

 
\maketitle
\begin{abstract}


Overparameterized neural networks generalize well but are expensive to train. Ideally, one would like to reduce their computational cost while retaining their generalization benefits. Sparse model training is a simple and promising approach to achieve this, but there remain challenges as existing methods struggle with accuracy loss, slow training runtime, or difficulty in sparsifying all model components.
The core problem is that searching for a sparsity mask over a discrete set of sparse matrices is difficult and expensive.
To address this, our main insight is to optimize over a continuous superset of sparse matrices with a fixed structure known as products of butterfly matrices.
As butterfly matrices are not hardware efficient, we propose simple variants of butterfly (block and flat) to take advantage of modern hardware.
Our method (Pixelated Butterfly) uses a simple fixed sparsity pattern based on flat block butterfly and low-rank matrices to sparsify most network layers (e.g., attention, MLP).
We empirically validate that Pixelated Butterfly is $3\times$ faster than butterfly and speeds up training to achieve favorable accuracy--efficiency tradeoffs.
On the ImageNet classification and WikiText-103 language modeling tasks, our sparse models train up to 2.5$\times$ faster than the dense MLP-Mixer, Vision Transformer, and GPT-2 medium with no drop in accuracy.



\end{abstract}


% !TEX root = ../arxiv.tex

Unsupervised domain adaptation (UDA) is a variant of semi-supervised learning \cite{blum1998combining}, where the available unlabelled data comes from a different distribution than the annotated dataset \cite{Ben-DavidBCP06}.
A case in point is to exploit synthetic data, where annotation is more accessible compared to the costly labelling of real-world images \cite{RichterVRK16,RosSMVL16}.
Along with some success in addressing UDA for semantic segmentation \cite{TsaiHSS0C18,VuJBCP19,0001S20,ZouYKW18}, the developed methods are growing increasingly sophisticated and often combine style transfer networks, adversarial training or network ensembles \cite{KimB20a,LiYV19,TsaiSSC19,Yang_2020_ECCV}.
This increase in model complexity impedes reproducibility, potentially slowing further progress.

In this work, we propose a UDA framework reaching state-of-the-art segmentation accuracy (measured by the Intersection-over-Union, IoU) without incurring substantial training efforts.
Toward this goal, we adopt a simple semi-supervised approach, \emph{self-training} \cite{ChenWB11,lee2013pseudo,ZouYKW18}, used in recent works only in conjunction with adversarial training or network ensembles \cite{ChoiKK19,KimB20a,Mei_2020_ECCV,Wang_2020_ECCV,0001S20,Zheng_2020_IJCV,ZhengY20}.
By contrast, we use self-training \emph{standalone}.
Compared to previous self-training methods \cite{ChenLCCCZAS20,Li_2020_ECCV,subhani2020learning,ZouYKW18,ZouYLKW19}, our approach also sidesteps the inconvenience of multiple training rounds, as they often require expert intervention between consecutive rounds.
We train our model using co-evolving pseudo labels end-to-end without such need.

\begin{figure}[t]%
    \centering
    \def\svgwidth{\linewidth}
    \input{figures/preview/bars.pdf_tex}
    \caption{\textbf{Results preview.} Unlike much recent work that combines multiple training paradigms, such as adversarial training and style transfer, our approach retains the modest single-round training complexity of self-training, yet improves the state of the art for adapting semantic segmentation by a significant margin.}
    \label{fig:preview}
\end{figure}

Our method leverages the ubiquitous \emph{data augmentation} techniques from fully supervised learning \cite{deeplabv3plus2018,ZhaoSQWJ17}: photometric jitter, flipping and multi-scale cropping.
We enforce \emph{consistency} of the semantic maps produced by the model across these image perturbations.
The following assumption formalises the key premise:

\myparagraph{Assumption 1.}
Let $f: \mathcal{I} \rightarrow \mathcal{M}$ represent a pixelwise mapping from images $\mathcal{I}$ to semantic output $\mathcal{M}$.
Denote $\rho_{\bm{\epsilon}}: \mathcal{I} \rightarrow \mathcal{I}$ a photometric image transform and, similarly, $\tau_{\bm{\epsilon}'}: \mathcal{I} \rightarrow \mathcal{I}$ a spatial similarity transformation, where $\bm{\epsilon},\bm{\epsilon}'\sim p(\cdot)$ are control variables following some pre-defined density (\eg, $p \equiv \mathcal{N}(0, 1)$).
Then, for any image $I \in \mathcal{I}$, $f$ is \emph{invariant} under $\rho_{\bm{\epsilon}}$ and \emph{equivariant} under $\tau_{\bm{\epsilon}'}$, \ie~$f(\rho_{\bm{\epsilon}}(I)) = f(I)$ and $f(\tau_{\bm{\epsilon}'}(I)) = \tau_{\bm{\epsilon}'}(f(I))$.

\smallskip
\noindent Next, we introduce a training framework using a \emph{momentum network} -- a slowly advancing copy of the original model.
The momentum network provides stable, yet recent targets for model updates, as opposed to the fixed supervision in model distillation \cite{Chen0G18,Zheng_2020_IJCV,ZhengY20}.
We also re-visit the problem of long-tail recognition in the context of generating pseudo labels for self-supervision.
In particular, we maintain an \emph{exponentially moving class prior} used to discount the confidence thresholds for those classes with few samples and increase their relative contribution to the training loss.
Our framework is simple to train, adds moderate computational overhead compared to a fully supervised setup, yet sets a new state of the art on established benchmarks (\cf \cref{fig:preview}).


\section{Theory}
In this section, we give guarantees on our grid-based approach. Suppose there is some underlying distribution $\mathcal{P}$ with corresponding density function $p : \mathbb{R}^d \rightarrow \mathbb{R}_{\ge 0}$ from which our data points $X_{[n]} = \{x_1,...,x_n\}$ are drawn i.i.d. We show guarantees on the density estimator based on the grid cell counts.

We need the following regularity assumptions on the density function. The first ensures that the density function has compact support with smooth boundaries and is lower bounded by some positive quantity, and the other ensures that the density function has smoothness. These are standard assumptions in analyses on density estimation e.g. \cite{gine2002rates,jiang2017uniform,chen2017tutorial,singh2009adaptive}.
\begin{assumption}\label{assumption1}
$p$ has compact support $\mathcal{X} \in \mathbb{R}^d$ and there exists $\lambda_0, r_0, C_0 > 0$ such that $p(x) \ge \lambda_0$ for all $x \in \mathcal{X}$ and $\text{Vol}(B(x, r) \cap \mathcal{X}) \ge C_0 \cdot \text{Vol}(B(x, r))$ for all $x \in \mathcal{X}$ and $0 < r \le r_0$, where $B(x, r) := \{x' \in \mathbb{R}^d: |x-x'| \le r\}$.
\end{assumption}
\begin{assumption}\label{assumption2}
$p$ is $\alpha$-Hölder continuous for some $0 < \alpha \le 1$: i.e. there exists $C_\alpha > 0$ such that $|p(x) - p(x')| \le C_\alpha \cdot |x - x'|^\alpha$ for all $x, x' \in \mathbb{R}^d$.
\end{assumption}

We now give the result, which says that for $h$ sufficiently small depending on $p$ (if $h$ is too large, then the grid is too coarse to learn a statistically consistent density estimator), and $n$ sufficiently large, there will be a high probability finite-sample uniform bound on the difference between the density estimator and the true density. The proof can be found in the Appendix.
\begin{theorem}\label{theorem}
Suppose Assumption~\ref{assumption1} and~\ref{assumption2} hold. Then there exists constants $C, C_{1} > 0$ depending on $p$ such that the following holds.
Let $0 < \delta < 1$, $0 < h < \text{min}\{\left(\frac{\lambda_0}{2\cdot C_\alpha}\right)^{1/\alpha}, r_0\}$, $nh^d \ge C_1$. Let $\mathcal{G}_h$ be a partitioning of $\mathbb{R}^d$ into grid cells of edge-length $h$ and for $x \in \mathbb{R}^d$. Let $G(x)$ denote the cell in $\mathcal{G}_h$ that $x$ belongs to.  Then, define the corresponding density estimator $\widehat{p}_h$ as:
\begin{align*}
    \widehat{p}_h(x) := \frac{|X_{[n]} \cap G(x)|}{n\cdot h^d}.
\end{align*}
Then, with probability at least $1 - \delta$:
\begin{align*}
    \sup_{x \in \mathbb{R}^d} |\widehat{p}_h(x)  - p(x)| \le C\cdot \left( h^\alpha + \frac{\sqrt{\log(1/(h\delta)}}{\sqrt{n\cdot h^d}} \right).
\end{align*}
\end{theorem}


\begin{remark}
In the above result, choosing $h \approx n^{-1/(2\alpha+d)}$ optimizes the convergence rate to $\tilde{O}(n^{-\alpha/(2\alpha+d)})$, which is the minimax optimal convergence up to logarithmic factors for the density estimation problem as established by Tsybakov \cite{tsybakov1997nonparametric,tsybakov2008introduction}.
\end{remark}
In other words, the grid-based approach statistically performs at least as well as any estimator of the density function, including the density estimator used by MeanShift. It is worth noting that while our results only provide results for the density estimation portion of MeanShift++ (i.e. the grid-cell binning technique), we prove the near-minimax optimality of this estimation. This implies that the information contained in the density estimation portion serves as an approximately sufficient statistic for the rest of the procedure, which behaves similarly to MeanShift, which operates on another, also nearly-optimal density estimator. Thus, existing analyses of MeanShift e.g. \cite{arias2016estimation,chen2015convergence,xiang2005convergence,li2007note,ghassabeh2015sufficient,ghassabeh2013convergence,subbarao2009nonlinear} can be adapted here; however, it is known that MeanShift is very difficult to analyze \cite{dasgupta2014optimal} and a complete analysis is beyond the scope of this paper.

\vspace{-0.1cm}
\section{Butterfly matrices and Pixelated Butterfly}
\label{sec:butterfly}



Butterfly matrices~\citep{parker1995random, dao2019learning} are expressive
and theoretically efficient.
As they contain the set of sparse matrices, we choose to search for the sparsity
pattern in this larger class due to their fixed sparsity structure.
However, there are three technical challenges.
We highlight them here along with our approaches to address them:
\begin{enumerate}[leftmargin=*,nosep,nolistsep]
  \item Slow speed: butterfly matrices are not friendly to modern hardware as their
  sparsity patterns are not block-aligned, thus are slow.
  We introduce a variant of butterfly matrices, \emph{block butterfly}, which operate at the block level, yielding
  a block-aligned sparsity pattern.
  \item Difficulty of parallelization: the sequential nature of butterfly matrices as products
  of many factors makes it hard to parallelize the multiplication.
  We propose another class of matrices, \emph{flat butterfly} matrices, that are
  the first-order approximation of butterfly with residual connections.
  Flat butterfly turns the product of factors into a sum, facilitating parallelization.
  \item Reduced expressiveness of flat butterfly: even though flat butterfly
  matrices can approximate butterfly matrices with residual connections, they are
  necessarily high-rank and cannot represent low-rank matrices~\citep{udell2019big}.
  We propose to add a low-rank matrix (that is also block-aligned) to flat
  butterfly to increase their expressiveness.
\end{enumerate}
Combining these three approaches (flat \& block butterfly + low-rank), our
proposal (Pixelated Butterfly) is a very simple method to train sparse networks.

\subsection{Block Butterfly Matrices}
\label{sec:block_butterfly}




We propose a block version of butterfly matrices, which is more
hardware-friendly than the regular butterfly.
The regular butterfly matrices~\citet{dao2019learning, dao2020kaleidoscope} will be a special case of block butterfly with
block size $b = 1$.
We omit $b$ in the notation if $b = 1$.
\begin{definition} \label{def:bfactor}
  A \textbf{block butterfly factor} (denoted as $\vB_{k, b}$) of size $kb$ (where $k \ge 2$) and block size $b$ is a matrix of the form
    \(
        \vB_{k, b} = \begin{bmatrix}
            \vD_1 & \vD_2 \\ \vD_3 & \vD_4
        \end{bmatrix}
    \)
    where each $\vD_i$ is a $\frac{k}{2} \times \frac{k}{2}$ block diagonal
    matrix of block size $b$ of the form
    $\mathrm{diag} \left( D_{i, 1}, \dots, D_{i, k/2} \right)$ where
    $D_{i, j} \in \mathbb{R}^{b \times b}$.
    We restrict $k$ to be a power of 2.
\end{definition}
\begin{definition}
  A \textbf{block butterfly factor matrix} (denoted as $\vB_{k}^{(n, b)}$) of size $nb$ with stride $k$ and
     block size $b$ is a block diagonal matrix
     of $\frac{n}{k}$ (possibly different) butterfly factors of size $kb$ and
     block size $b$:
     \[
        \vB_{k}^{(n, b)} = \mathrm{diag} \left( \left[ \vB_{k, b} \right]_1, \left[ \vB_{k, b} \right]_2, \hdots, \left[ \vB_{k, b} \right]_\frac{n}{k} \right)
     \]
\end{definition}

\begin{definition} \label{def:bmatrix}
    A \textbf{block butterfly matrix} of size $nb$ with block size $b$ (denoted as $\vB^{(n, b)}$) is a matrix that can be expressed as a product of butterfly factor matrices:
    \(
        \vB^{(n, b)} = \vB_n^{(n, b)} \vB_{\frac{n}{2}}^{(n, b)} \hdots \vB_2^{(n, b)}.
    \)
    Define $\B_b$ as the set of all matrices that can be expressed in the form $\vB^{(n, b)}$ (for some $n$).
\end{definition}

\begin{figure}
\vspace{-1.2cm}
	\begin{center}
		\begin{tabular}{c}
			\includegraphics[width=0.87\linewidth]{figs/flat_block_butterfly.pdf}
		\end{tabular}
	\end{center}
	\captionsetup{font=small}
		\vspace{-0.8cm}
	\caption{Visualization of Flat, Block, and Flat Block butterfly.}
	\label{fig:tradeoff}
	\vspace{-0.3cm}
\end{figure}

\subsection{Flat butterfly matrices}
\label{sec:flat_butterfly}
In most applications of butterfly matrices to neural networks, one multiplies
the $O(\log n)$ butterfly factors.
However, this operation is hard to be efficiently implemented on parallel hardware (e.g., GPUs) due to
the sequential nature of the operation\footnote{Even with a very specialized
  CUDA kernel, butterfly matrix multiply ($O(n \log n)$ complexity) is only
faster than dense matrix multiply ($O(n^2)$ complexity) for large values of $n$
(around 1024)~\citep{dao2019learning}.}.
We instead propose to use a sum of butterfly factors that can approximate the
products of the factors.
This sum of factors results in one sparse matrix with a fixed sparsity pattern,
which yields up to 3$\times$ faster multiplication on GPUs (\cref{sec:appx_benchmark}).

Residual connections have been proposed to connect the butterfly
factors~\citep{vahid2020butterfly}.
We show that residual products of butterfly matrices have a first-order
approximation as a sparse matrix with a fixed sparsity.
Let $M$ be a matrix in the set of butterfly matrices $\B$.
In residual form, for some $\lambda \in \mathbb{R}$:
\begin{equation}
  \label{eq:residual_butterfly}
  M = (I + \lambda \vB_n^{(n)}) (I + \lambda \vB_{n/2}^{(n)}) \dots (I + \lambda \vB_2^{(n)}).
\end{equation}
Note that this form can represent the same matrices in the class of butterfly
matrices $\vB$, since any $\vB_k^{(n)}$ contains the identity matrix $I$.

Assuming that $\lambda$ is small, we can expand the residual and collect the
terms\footnote{We make the approximation rigorous in \cref{sec:analysis}.}:
\begin{equation*}
  M = I + \lambda (\vB_2^{(n)} + \vB_{4}^{(n)} + \dots + \vB_n^{(n)}) + \tilde{O}(\lambda^2).
\end{equation*}
\begin{definition}
  \label{def:flat_butterfly}
  \emph{Flat butterfly} matrices of maximum stride $k$ (for $k$ a power of 2)
  are those of the form $I + \lambda (\vB_2^{(n)} + \vB_{4}^{(n)} + \dots + \vB_k^{(n)})$.
\end{definition}
Flat butterfly matrices of maximum stride $n$ are the first-order approximation
of butterfly matrices in residual form (\cref{eq:residual_butterfly}).
Notice that flat butterfly of maximum stride $k$ are sparse matrices with $O(n \log k)$ nonzeros with a fixed
sparsity pattern, as illustrated in~\cref{fig:tradeoff}.
We call this sparsity pattern the \emph{flat butterfly} pattern.

\emph{Flat block butterfly} matrices are block versions of flat butterfly in~\cref{sec:flat_butterfly} (shown in~\cref{fig:tradeoff}).
We empirically validate that flat block butterfly matrices are up to 3$\times$
faster than block butterfly or regular butterfly (\cref{sec:appx_benchmark}).

Since flat butterfly matrices approximate the residual form of butterfly
matrices, they have high rank if $\lambda$ is small (\cref{sec:analysis}).
This is one of the motivations for the addition of the low-rank term in our
method.

\subsection{Pixelated Butterfly: Flat Block Butterfly + Low-rank for Efficient Sparse Training}
\label{sec:method}

We present Pixelated Butterfly, an efficient sparse model with a simple and fixed sparsity
pattern based on butterfly and low-rank matrices.
Our method targets GEMM-based neural networks, which are networks whose computation is dominated by general matrix multiplies (GEMM), such as Transformer and MLP-Mixer.
As a result, we can view the network as a series of matrix multiplies.

Given a model schema (layer type, number of layers, matrix dimension) and a
compute budget, Pixelated Butterfly has three steps: compute budget allocation per layer,
sparsity mask selection from the flat butterfly pattern, and model
sparsification.
We describe these steps in more details:
\begin{enumerate}[leftmargin=*,nosep,nolistsep]
  \item \textbf{Compute budget allocation}: based on our cost model
  (\cref{app:problem_formulation}), given the layer type, number of layers, and
  matrix dimension, we can find the density (fraction of nonzero weights) of
  each layer type to minimize the projected compute cost.
  Continuing our goal for a simple method, we propose to use a simple rule of
  thumb: allocate sparsity compute budget proportional to the compute fraction
  of the layer.
  For example, if the MLP layer and attention layers are projected to takes 60\%
  and 40\% the compute time respectively, then allocate 60\% of the sparsity compute budget
  to MLP and 40\% to attention.
  We verify in \cref{sec:appx_method_details} that this simple rule of thumb
  produces similar results to solving for the density from the cost model.

  \item \textbf{Sparsity mask selection}: given a layer and a sparsity compute budget for
  that layer, we use one-quarter to one-third of the budget for the low-rank
  part as a simple rule of thumb.
  We pick the rank as a multiple of the smallest supported block
  size of the device (e.g., 32) so that the low-rank matrices are also block-aligned.
  The remaining compute budget is used to select the sparsity mask from the flat
  block butterfly sparsity pattern: we choose the butterfly block size as the
  smallest supported block size of the device (e.g., 32), and pick the maximum
  stride of the flat block butterfly (\cref{def:flat_butterfly}) to fill up the
  budget.

  \item \textbf{Model sparsification}: The resulting sparse model is simply a
  model whose weights or attention follow the fixed sparsity mask chosen in
  step 2, with the additional low-rank terms (rank also chosen in step 2).
  In particular, we parameterize each weight matrix\footnote{We describe how
  to add sparse and low-rank for attention in \cref{sec:appx_method_details}} as:
  $W = \gamma B + (1 - \gamma) U V^\top$, where $B$ is a flat block butterfly
  matrix (which is sparse), $U V^\top$ is the low-rank component, and $\gamma$
  is a learnable parameter.
  We train the model from scratch as usual.
\end{enumerate}

Our method is very simple, but competitive with more complicated procedures that
search for the sparsity pattern (\cref{sec:appx_ntk_algorithm}).
We expect more sophisticated techniques (dynamic sparsity, a better approximation
of butterfly) to improve the accuracy of the method.



\section{Case Studies}
\label{sec:case_studies}
In this section, we present a case study of Facebook posts from an Australian public page.
The page shifts between early 2020 (\emph{2019-2020 Australian bushfire season}) and late 2020 (\emph{COVID-19 crises}) from being a moderate-right group for discussion around climate change to a far-right extremist group for conspiracy theories.


\begin{figure*}[!tbp]
	\begin{subfigure}{0.21\textwidth}
		\includegraphics[width=\textwidth]{images/facebook1.png}
		\caption{}
		\label{subfig:first-posting}
		\includegraphics[width=0.9\textwidth]{images/facebook3.jpg}
		\caption{}
		\label{subfig:comment-post-1}
	\end{subfigure}
    \begin{subfigure}{0.28\textwidth}
		\includegraphics[width=\textwidth]{images/facebook2.jpg}
		\caption{}
		\label{subfig:second-posting}
	\end{subfigure}
    \begin{subfigure}{0.23\textwidth}
		\includegraphics[width=\textwidth]{images/facebook4.jpg}
		\caption{}
		\label{subfig:comment-post-2a}
	\end{subfigure}
    \begin{subfigure}{0.23\textwidth}
		\includegraphics[width=\textwidth]{images/facebook5.jpg}
		\caption{}
		\label{subfig:comment-post-2b}
	\end{subfigure}
	\caption{
		Examples of postings and comment threads from a public Facebook page from two periods of time early 2020 (a) and late 2020 (b)-(e), which show a shift from climate change debates to extremist and far-right messaging.
	}
	\label{fig:facebook}
\end{figure*}

We focus on a sample of 2 postings and commenting threads from one Australian Facebook page we classified as ``far-right'' based on the content on the page. 
We have anonymized the users in \Cref{fig:facebook} to avoid re-identification.
The first posting and comment thread (see \Cref{subfig:first-posting}) was collected on Jan 10, 2020, and responded to the Australian bushfire crisis that began in late 2019 and was still ongoing in January 2020. It contains an ambivalent text-based provocation that references disputes in the community regarding the validity of climate change and climate science. 

The second posting and comment thread (see \Cref{subfig:second-posting}) was collected from the same page in September 2020, months after the bushfire crisis had abated.
At that time, a new crisis was energizing and connecting the far-right groups in our dataset --- i.e., the COVID-19 pandemic and the government interventions to curb the spread of the virus. 
The post is different in style compared to the first.
It is image-based instead of text-based and highly emotive, with a photo collage bringing together images of prison inmates with iron masks on their faces (top row) juxtaposed to people wearing face masks during COVID-19 (bottom row). 
The image references the public health orders issued during Melbourne's second lockdown and suggests that being ordered to wear masks is an infringement of citizen rights and freedoms, similar to dehumanizing restraints used on prisoners.

To analyze reactions to the posts, two researchers used a deductive analytical approach to separately code and to analyze the commenting threads --- see \Cref{subfig:comment-post-1} for comments of the first posting, and \Cref{subfig:comment-post-2a,subfig:comment-post-2b} for comments on the second posting. 
Conversations were also inductively coded for emerging themes. 
During the analysis, we observed qualitative differences in the types of content users posted, interactions between commenters, tone and language of debate, linked media shared in the commenting thread, and the opinions expressed.
The rest of this section further details these differences.
To ensure this was not a random occurrence, we tested the exemplar threads against field notes collected on the group during the entire study.
We also used Facebook's search function within pages to find a sample of posts from the same period and which dealt with similar topics. 
After this analysis, we can confidently say that key changes occurred in the group between the bushfire crisis and COVID-19, that we detail next.

\subsubsection*{Exemplar 1 --- climate change skepticism.}
To explore this transformation in more depth, we analyzed comments scraped on the first posting --- \cref{sub@subfig:comment-post-1} shows a small sample of these comments.
The language used was similar to comments that we observed on numerous far-right nationalist pages at the time of the bushfires.
These comments are usually text-based, employing emojis to denote emotions, and sometimes being mocking or provocative in tone. 
Noteworthy for this commenting thread is the 50/50 split in the number of members posting in favor of action on climate change (on one side) and those who posted anti-Greens and anti-climate change science posts and memes (on the other side).
The two sides aligned strongly with political partisanship --- either with Liberal/National coalition (climate change deniers) or Labor/Green (climate change believers) parties. 
This is rather unusual for pages classified as far-right. 

We observed trolling practices between the climate change deniers and believers, which often descend into \emph{flame wars} --- i.e., online ``firefights that take place between disembodied combatants on electronic bulletin boards''~\citep{bukatman1994flame}.
The result is a boosted engagement on the post but also the frustration and confusion of community members and lurkers who came to the discussions to become informed or debate rationally on key differences between the two positions.
They often even become targeted, victimized, and baited by trolls on both sides of the partisan divide. 
The opinions expressed by deniers in commenting sections range from skepticism regarding climate change science to plain denial.
Deniers also regard a range of targets as embroiled in a climate change conspiracy to deceive the public, such as The Greens and their environmental policy, in some cases the government, the United Nations, and climate change celebrities like David Attenborough and Greta Thunberg. 
These figures are blamed for either exaggerating risks of climate change or creating a climate change hoax to increase the influence of the UN on domestic governments or to increase domestic governments' social control over citizens. 

Both coders noted that flame wars between these opposing personas contained very few links to external media. 
Where links were added, they often seemed disconnected from the rest of the conversation and were from users whose profiles suggested they believed in more radical conspiracy theories.
One such example is ``geo-engineering'' (see \cref{sub@subfig:comment-post-1}).
Its adherents believe that solar geo-engineering programs designed to combat climate change are secretly used by a global elite to depopulate the world through sterilization or to control and weaponize the weather.

Nonetheless, apart from the random comments that hijack the thread, redirecting users to external ``alternative'' news sites and Twitter, and the trolls who seem to delight in victimizing unsuspecting victims, the discussion was pretty healthy.
There are many questions, rational inquiries, and debates between users of different political persuasion and views on climate change.
This, however, changes in the span of only a couple of months.

\subsubsection*{Exemplar 2 --- posting and commenting thread.}
We observe a shift in the comment section of the post collected during the second wave of the COVID pandemic (\Cref{sub@subfig:second-posting}) --- which coincided with government laws mandating the public to wear masks and stay at home in Victoria, Australia.
There emerges much more extreme far-right content that converges with anti-vaccination opinions and content.
We also note a much higher prevalence of conspiracy theories often implicating racialized targets.
This is exemplified in the comments on the second post (\Cref{sub@subfig:comment-post-2a,sub@subfig:comment-post-2b}) where Islamophobia and antisemitism are confidently asserted alongside anti-mask rhetoric.
These comments consider face masks similar to the religious head coverings worn by some Muslim women, which users describe as ``oppressive'' and ``silencing''. 
In this way, anti-maskers cast women as a distinct, sympathetic marginalized demographic.
However, this is enacted alongside the racialization and demonization of Islam as an oppressive religion. 

Given the extreme racialization of anti-mask rhetoric, some commenters contest these positions, arguing that the page is becoming less an anti-Scott Morrison page (Australia's Prime Minister at the time) and changing into a page that harbors ``far-right dickheads''.
This questioning is actively challenged by far-right commenters and conspiracy theorists on the page, who regarded pro-mask users and the Scott Morrison government as ``puppets'' being manipulated by higher forces (see \Cref{sub@subfig:comment-post-2b}). 

This indicates a significant change on the page's membership towards the extreme-right, who employs more extreme forms of racialized imagery, with more extreme opinion being shared.
Conspiracy theorists become more active and vocal, and they consistently challenge the opinions of both center conservative and left-leaning users. 
This is evident in the final two comments in \Cref{subfig:comment-post-2b}, which reflect QAnon style conspiracy theories and language.
Public health orders to wear masks are being connected to a conspiracy that all of these decisions are directed by a secret network of global Jewish elites, who manipulate the pandemic to increase their power and control. 
This rhetoric intersects with the contemporary ``QAnon'' conspiracy theory, which evolved from the ``Pizzagate'' conspiracy theory.
They also heavily draw on well-established antisemitic blood libel conspiracy theories, which foster beliefs that a powerful global elite is controlling the decisions of organizations such as WHO and are responsible for the vaccine rollout and public health orders related to the pandemic.
The QAnon conspiracy is also influenced by Bill Gates' Microchips conspiracy theory, i.e., the theory that the WHO and the Bill Gates Foundation global vaccine programs are used to inject tracking microchips into people.

These conspiracy theories have, since COVID-19, connected formerly separate communities and discourses, uniting existing anti-vaxxer communities, older demographics who are mistrustful of technology, far-right communities suspicious of global and national left-wing agendas, communities protesting against 5G mobile networks (for fear that they will brainwash, control, or harm people), as well as generating its own followers out of those anxious during the 2020 onset of the COVID-19 pandemic.
We detect and describe some of these opinion dynamics in the next section.

\section{Experimental Evaluation}
\label{sec:experiment}
To demonstrate the viability of our modeling methodology, we show experimentally how through the deliberate combination and configuration of parallel FREEs, full control over 2DOF spacial forces can be achieved by using only the minimum combination of three FREEs.
To this end, we carefully chose the fiber angle $\Gamma$ of each of these actuators to achieve a well-balanced force zonotope (Fig.~\ref{fig:rigDiagram}).
We combined a contracting and counterclockwise twisting FREE with a fiber angle of $\Gamma = 48^\circ$, a contracting and clockwise twisting FREE with $\Gamma = -48^\circ$, and an extending FREE with $\Gamma = -85^\circ$.
All three FREEs were designed with a nominal radius of $R$ = \unit[5]{mm} and a length of $L$ = \unit[100]{mm}.
%
\begin{figure}
    \centering
    \includegraphics[width=0.75\linewidth]{figures/rigDiagram_wlabels10.pdf}
    \caption{In the experimental evaluation, we employed a parallel combination of three FREEs (top) to yield forces along and moments about the $z$-axis of an end effector.
    The FREEs were carefully selected to yield a well-balanced force zonotope (bottom) to gain full control authority over $F^{\hat{z}_e}$ and $M^{\hat{z}_e}$.
    To this end, we used one extending FREE, and two contracting FREEs which generate antagonistic moments about the end effector $z$-axis.}
    \label{fig:rigDiagram}
\end{figure}


\subsection{Experimental Setup}
To measure the forces generated by this actuator combination under a varying state $\vec{x}$ and pressure input $\vec{p}$, we developed a custom built test platform (Fig.~\ref{fig:rig}). 
%
\begin{figure}
    \centering
    \includegraphics[width=0.9\linewidth]{figures/photos/rig_labeled.pdf}
    \caption{\revcomment{1.3}{This experimental platform is used to generate a targeted displacement (extension and twist) of the end effector and to measure the forces and torques created by a parallel combination of three FREEs. A linear actuator and servomotor impose an extension and a twist, respectively, while the net force and moment generated by the FREEs is measured with a force load cell and moment load cell mounted in series.}}
    \label{fig:rig}
\end{figure}
%
In the test platform, a linear actuator (ServoCity HDA 6-50) and a rotational servomotor (Hitec HS-645mg) were used to impose a 2-dimensional displacement on the end effector. 
A force load cell (LoadStar  RAS1-25lb) and a moment load cell (LoadStar RST1-6Nm) measured the end-effector forces $F^{\hat{z_e}}$ and moments $M^{\hat{z_e}}$, respectively.
During the experiments, the pressures inside the FREEs were varied using pneumatic pressure regulators (Enfield TR-010-g10-s). 

The FREE attachment points (measured from the end effector origin) were measured to be:
\begin{align}
    \vec{d}_1 &= \bmx 0.013 & 0 & 0 \emx^T  \text{m}\\
    \vec{d}_2 &= \bmx -0.006 & 0.011 & 0 \emx^T  \text{m}\\
    \vec{d}_3 &= \bmx -0.006 & -0.011 & 0 \emx^T \text{m}
%    \vec{d}_i &= \bmx 0 & 0 & 0 \emx^T , && \text{for } i = 1,2,3
\end{align}
All three FREEs were oriented parallel to the end effector $z$-axis:
\begin{align}
    \hat{a}_i &= \bmx 0 & 0 & 1 \emx^T, \hspace{20pt} \text{for } i = 1,2,3
\end{align}
Based on this geometry, the transformation matrices $\bar{\mathcal{D}}_i$ were given by:
\begin{align}
    \bar{\mathcal{D}}_1 &= \bmx 0 & 0 & 1 & 0 & -0.013 & 0 \\ 0 & 0 & 0 & 0 & 0 & 1 \emx^T  \\
    \bar{\mathcal{D}}_2 &= \bmx 0 & 0 & 1 & 0.011 & 0.006 & 0 \\ 0 & 0 & 0 & 0 & 0 & 1 \emx^T  \\
    \bar{\mathcal{D}}_3 &= \bmx 0 & 0 & 1 & -0.011 & 0.006 & 0 \\ 0 & 0 & 0 & 0 & 0 & 1 \emx^T 
%    \bar{\mathcal{D}}_i &= \bmx 0 & 0 & 1 & 0 & 0 & 0 \\ 0 & 0 & 0 & 0 & 0 & 1 \emx^T , && \text{for } i = 1,2,3
\end{align}
These matrices were used in equation \eqref{eq:zeta} to yield the state-dependent fluid Jacobian $\bar{J}_x$ and to compute the resulting force zontopes.
%while using measured values of $\vec{\zeta}^{\,\text{meas}} (\vec{q}, \vec{P})$ and $\vec{\zeta}^{\,\text{meas}} (\vec{q}, 0)$ in \eqref{eq:fiberIso} yields the empirical measurements of the active force.



\subsection{Isolating the Active Force}
To compare our model force predictions (which focus only on the active forces induced by the fibers)
to those measured empirically on a physical system, we had to remove the elastic force components attributed to the elastomer. 
Under the assumption that the elastomer force is merely a function of the displacement $\vec{x}$ and independent of pressure $\vec{p}$ \cite{bruder2017model}, this force component can be approximated by the measured force at a pressure of $\vec{p}=0$. 
That is: 
\begin{align}
    \vec{f}_{\text{elast}} (\vec{x}) = \vec{f}_{\text{\,meas}} (\vec{x}, 0)
\end{align}
With this, the active generalized forces were measured indirectly by subtracting off the force generated at zero pressure:
\begin{align}
    \vec{f} (\vec{x}, \vec{p})  &= \vec{f}_{\text{meas}} (\vec{x}, \vec{p}) - \vec{f}_{\text{meas}} (\vec{x}, 0)     \label{eq:fiberIso}
\end{align}


%To validate our parallel force model, we compare its force predictions, $\vec{\zeta}_{\text{pred}}$, to those measured empirically on a physical system, $\vec{\zeta}_\text{meas}$. 
%From \eqref{eq:Z} and \eqref{eq:zeta}, the force at the end effector is given by:
%\begin{align}
%    \vec{\zeta}(\vec{q}, \vec{P}) &= \sum_{i=1}^n \bar{\mathcal{D}}_i \left( {\bar{J}_V}_i^T(\vec{q_i}) P_i + \vec{Z}_i^{\text{elast}} (\vec{q_i}) \right) \\
%    &= \underbrace{\sum_{i=1}^n \bar{\mathcal{D}}_i {\bar{J}_V}_i^T(\vec{q_i}) P_i}_{\vec{\zeta}^{\,\text{fiber}} (\vec{q}, \vec{P})} + \underbrace{\sum_{i=1}^n \bar{\mathcal{D}}_i \vec{Z}_i^{\text{elast}} (\vec{q_i})}_{\vec{\zeta}^{\text{elast}} (\vec{q})}   \label{eq:zetaSplit}
%     &= \vec{\zeta}^{\,\text{fiber}} (\vec{q}, \vec{P}) + \vec{\zeta}^{\text{elast}} (\vec{q})
%\end{align}
%\Dan{These will need to reflect changes made to previous section.}
%The model presented in this paper does not specify the elastomer forces, $\vec{\zeta}^{\text{elast}}$, therefore we only validate its predictions %of the fiber forces, $\vec{\zeta}^{\,\text{fiber}}$. 
%We isolate the fiber forces by noting that $\vec{\zeta}^{\text{elast}} (\vec{q}) = \vec{\zeta}(\vec{q}, 0)$ and rearranging \eqref{eq:zetaSplit}
%\begin{align}
%    \vec{\zeta}^{\,\text{fiber}} (\vec{q}, \vec{P})  &= \vec{\zeta}(\vec{q}, \vec{P}) - \vec{\zeta}(\vec{q}, 0)     \label{eq:fiberIso}
%%    \vec{\zeta}^{\,\text{fiber}}_{\text{emp}} (\vec{q}, \vec{P})  &= \vec{\zeta}_{\text{emp}}(\vec{q}, \vec{P}) - %\vec{\zeta}_{\text{emp}}(\vec{q}, 0)
%\end{align}
%Thus we measure the fiber forces indirectly by subtracting off the forces generated at zero pressure.  


\subsection{Experimental Protocol}
The force and moment generated by the parallel combination of FREEs about the end effector $z$-axis  was measured in four different geometric configurations: neutral, extended, twisted, and simultaneously extended and twisted (see Table \ref{table:RMSE} for the exact deformation amounts). 
At each of these configurations, the forces were measured at all pressure combinations in the set
\begin{align}
    \mathcal{P} &= \left\{ \bmx \alpha_1 & \alpha_2 & \alpha_3 \emx^T p^{\text{max}} \, : \, \alpha_i = \left\{ 0, \frac{1}{4}, \frac{1}{2}, \frac{3}{4}, 1 \right\} \right\}
\end{align}
with $p^{\text{max}}$ = \unit[103.4]{kPa}. 
\revcomment{3.2}{The experiment was performed twice using two different sets of FREEs to observe how fabrication variability might affect performance. The results from Trial 1 are displayed in Fig.~\ref{fig:results} and the error for both trials is given in Table \ref{table:RMSE}.}



\subsection{Results}

\begin{figure*}[ht]
\centering

\def\picScale{0.08}    % define variable for scaling all pictures evenly
\def\plotScale{0.2}    % define variable for scaling all plots evenly
\def\colWidth{0.22\linewidth}

\begin{tikzpicture} %[every node/.style={draw=black}]
% \draw[help lines] (0,0) grid (4,2);
\matrix [row sep=0cm, column sep=0cm, style={align=center}] (my matrix) at (0,0) %(2,1)
{
& \node (q1) {(a) $\Delta l = 0, \Delta \phi = 0$}; & \node (q2) {(b) $\Delta l = 5\text{mm}, \Delta \phi = 0$}; & \node (q3) {(c) $\Delta l = 0, \Delta \phi = 20^\circ$}; & \node (q4) {(d) $\Delta l = 5\text{mm}, \Delta \phi = 20^\circ$};

\\

&
\node[style={anchor=center}] {\includegraphics[width=\colWidth]{figures/photos/s0w0pic_colored.pdf}}; %\fill[blue] (0,0) circle (2pt);
&
\node[style={anchor=center}] {\includegraphics[width=\colWidth]{figures/photos/s5w0pic_colored.pdf}}; %\fill[blue] (0,0) circle (2pt);
&
\node[style={anchor=center}] {\includegraphics[width=\colWidth]{figures/photos/s0w20pic_colored.pdf}}; %\fill[blue] (0,0) circle (2pt);
&
\node[style={anchor=center}] {\includegraphics[width=\colWidth]{figures/photos/s5w20pic_colored.pdf}}; %\fill[blue] (0,0) circle (2pt);

\\

\node[rotate=90] (ylabel) {Moment, $M^{\hat{z}_e}$ (N-m)};
&
\node[style={anchor=center}] {\includegraphics[width=\colWidth]{figures/plots3/s0w0.pdf}}; %\fill[blue] (0,0) circle (2pt);
&
\node[style={anchor=center}] {\includegraphics[width=\colWidth]{figures/plots3/s5w0.pdf}}; %\fill[blue] (0,0) circle (2pt);
&
\node[style={anchor=center}] {\includegraphics[width=\colWidth]{figures/plots3/s0w20.pdf}}; %\fill[blue] (0,0) circle (2pt);
&
\node[style={anchor=center}] {\includegraphics[width=\colWidth]{figures/plots3/s5w20.pdf}}; %\fill[blue] (0,0) circle (2pt);

\\

& \node (xlabel1) {Force, $F^{\hat{z}_e}$ (N)}; & \node (xlabel2) {Force, $F^{\hat{z}_e}$ (N)}; & \node (xlabel3) {Force, $F^{\hat{z}_e}$ (N)}; & \node (xlabel4) {Force, $F^{\hat{z}_e}$ (N)};

\\
};
\end{tikzpicture}

\caption{For four different deformed configurations (top row), we compare the predicted and the measured forces for the parallel combination of three FREEs (bottom row). 
\revcomment{2.6}{Data points and predictions corresponding to the same input pressures are connected by a thin line, and the convex hull of the measured data points is outlined in black.}
The Trial 1 data is overlaid on top of the theoretical force zonotopes (grey areas) for each of the four configurations.
Identical colors indicate correspondence between a FREE and its resulting force/torque direction.}
\label{fig:results}
\end{figure*}






% & \node (a) {(a)}; & \node (b) {(b)}; & \node (c) {(c)}; & \node (d) {(d)};


For comparison, the measured forces are superimposed over the force zonotope generated by our model in Fig.~\ref{fig:results}a-~\ref{fig:results}d.
To quantify the accuracy of the model, we defined the error at the $j^{th}$ evaluation point as the difference between the modeled and measured forces
\begin{align}
%    \vec{e}_j &= \left( {\vec{\zeta}_{\,\text{mod}}} - {\vec{\zeta}_{\,\text{emp}}} \right)_j
%    e_j &= \left( F/M_{\,\text{mod}} - F/M_{\,\text{emp}} \right)_j
    e^F_j &= \left( F^{\hat{z}_e}_{\text{pred}, j} - F^{\hat{z}_e}_{\text{meas}, j} \right) \\
    e^M_j &= \left( M^{\hat{z}_e}_{\text{pred}, j} - M^{\hat{z}_e}_{\text{meas}, j} \right)
\end{align}
and evaluated the error across all $N = 125$ trials of a given end effector configuration.
% using the following metrics:
% \begin{align}
%     \text{RMSE} &= \sqrt{ \frac{\sum_{j=1}^{N} e_j^2}{N} } \\
%     \text{Max Error} &= \max \{ \left| e_j \right| : j = 1, ... , N \}
% \end{align}
As shown in Table \ref{table:RMSE}, the root-mean-square error (RMSE) is less than \unit[1.5]{N} (\unit[${8 \times 10^{-3}}$]{Nm}), and the maximum error is less than \unit[3]{N}  (\unit[${19 \times 10^{-3}}$]{Nm}) across all trials and configurations.

\begin{table}[H]
\centering
\caption{Root-mean-square error and maximum error}
\begin{tabular}{| c | c || c | c | c | c|}
    \hline
     & \rule{0pt}{2ex} \textbf{Disp.} & \multicolumn{2}{c |}{\textbf{RMSE}} & \multicolumn{2}{c |}{\textbf{Max Error}} \\ 
     \cline{2-6}
     & \rule{0pt}{2ex} (mm, $^\circ$) & F (N) & M (Nm) & F (N) & M (Nm) \\
     \hline
     \multirow{4}{*}{\rotatebox[origin=c]{90}{\textbf{Trial 1}}}
     & 0, 0 & 1.13 & $3.8 \times 10^{-3}$ & 2.96 & $7.8 \times 10^{-3}$ \\
     & 5, 0 & 0.74 & $3.2 \times 10^{-3}$ & 2.31 & $7.4 \times 10^{-3}$ \\
     & 0, 20 & 1.47 & $6.3 \times 10^{-3}$ & 2.52 & $15.6 \times 10^{-3}$\\
     & 5, 20 & 1.18 & $4.6 \times 10^{-3}$ & 2.85 & $12.4 \times 10^{-3}$ \\  
     \hline
     \multirow{4}{*}{\rotatebox[origin=c]{90}{\textbf{Trial 2}}}
     & 0, 0 & 0.93 & $6.0 \times 10^{-3}$ & 1.90 & $13.3 \times 10^{-3}$ \\
     & 5, 0 & 1.00 & $7.7 \times 10^{-3}$ & 2.97 & $19.0 \times 10^{-3}$ \\
     & 0, 20 & 0.77 & $6.9 \times 10^{-3}$ & 2.89 & $15.7 \times 10^{-3}$\\
     & 5, 20 & 0.95 & $5.3 \times 10^{-3}$ & 2.22 & $13.3 \times 10^{-3}$ \\  
     \hline
\end{tabular}
\label{table:RMSE}
\end{table}

\begin{figure}
    \centering
    \includegraphics[width=\linewidth]{figures/photos/buckling.pdf}
    \caption{At high fluid pressure the FREE with fiber angle of $-85^\circ$ started to buckle.  This effect was less pronounced when the system was extended along the $z$-axis.}
    \label{fig:buckling}
\end{figure}

%Experimental precision was limited by unmodeled material defects in the FREEs, as well as sensor inaccuracy. While the commercial force and moment sensors used have a quoted accuracy of 0.02\% for the force sensor and 0.2\% for the moment sensor (LoadStar Sensors, 2015), a drifting of up to 0.5 N away from zero was noticed on the force sensor during testing.

It should be noted, that throughout the experiments, the FREE with a fiber angle of $-85^\circ$ exhibited noticeable buckling behavior at pressures above $\approx$ \unit[50]{kPa} (Fig.~\ref{fig:buckling}). 
This behavior was more pronounced during testing in the non-extended configurations (Fig.~\ref{fig:results}a~and~\ref{fig:results}c). 
The buckling might explain the noticeable leftward offset of many of the points in Fig.~\ref{fig:results}a and Fig.~\ref{fig:results}c, since it is reasonable to assume that buckling reduces the efficacy of of the FREE to exert force in the direction normal to the force sensor. 

\begin{figure}
    \centering
    \includegraphics[width=\linewidth]{figures/zntp_vs_x4.pdf}
    \caption{A visualization of how the \emph{force zonotope} of the parallel combination of three FREEs (see Fig.~\ref{fig:rig}) changes as a function of the end effector state $x$. One can observe that the change in the zonotope ultimately limits the work-space of such a system.  In particular the zonotope will collapse for compressions of more than \unit[-10]{mm}.  For \revcomment{2.5}{scale and comparison, the convex hulls of the measured points from Fig.~\ref{fig:results}} are superimposed over their corresponding zonotope at the configurations that were evaluated experimentally.}
    % \marginnote{\#2.5}
    \label{fig:zntp_vs_x}
\end{figure}
\section{Related Work}
%\mz{We lack a comparison to this paper: https://arxiv.org/abs/2305.14877}
%\anirudh{refine to be more on-topic?}
\iffalse
\paragraph{In-Context Learning} As language models have scaled, the ability to learn in-context, without any weight updates, has emerged. \cite{brown2020language}. While other families of large language models have emerged, in-context learning remains ubiquitous \cite{llama, bloom, gptneo, opt}. Although such as HELM \cite{helm} have arisen for systematic evaluation of \emph{models}, there is no systematic framework to our knowledge for evaluating \emph{prompting methods}, and validating prompt engineering heuristics. The test-suite we propose will ensure that progress in the field of prompt-engineering is structured and objectively evaluated. 

\paragraph{Prompt Engineering Methods} Researchers are interested in the automatic design of high performing instructions for downstream tasks. Some focus on simple heuristics, such as selecting instructions that have the lowest perplexity \cite{lowperplexityprompts}. Other methods try to use large language models to induce an instruction when provided with a few input-output pairs \cite{ape}. Researchers have also used RL objectives to create discrete token sequences that can serve as instructions \cite{rlprompt}. Since the datasets and models used in these works have very little intersection, it is impossible to compare these methods objectively and glean insights. In our work, we evaluate these three methods on a diverse set of tasks and models, and analyze their relative performance. Additionally, we recognize that there are many other interesting angles of prompting that are not covered by instruction engineering \cite{weichain, react, selfconsistency}, but we leave these to future work.

\paragraph{Analysis of Prompting Methods} While most prompt engineering methods focus on accuracy, there are many other interesting dimensions of performance as well. For instance, researchers have found that for most tasks, the selection of demonstrations plays a large role in few-shot accuracy \cite{whatmakesgoodicexamples, selectionmachinetranslation, knnprompting}. Additionally, many researchers have found that even permuting the ordering of a fixed set of demonstrations has a significant effect on downstream accuracy \cite{fantasticallyorderedprompts}. Prompts that are sensitive to the permutation of demonstrations have been shown to also have lower accuracies \cite{relationsensitivityaccuracy}. Especially in low-resource domains, which includes the large public usage of in-context learning, these large swings in accuracy make prompting less dependable. In our test-suite we include sensitivity metrics that go beyond accuracy and allow us to find methods that are not only performant but reliable.

\paragraph{Existing Benchmarks} We recognize that other holistic in-context learning benchmarks exist. BigBench is a large benchmark of 204 tasks that are beyond the capabilities of current LLMs. BigBench seeks to evaluate the few-shot abilities of state of the art large language models, focusing on performance metrics such as accuracy \cite{bigbench}. Similarly, HELM is another benchmark for language model in-context learning ability. Rather than only focusing on performance, HELM branches out and considers many other metrics such as robustness and bias \cite{helm}. Both BigBench and HELM focus on ranking different language model, while fix a generic instruction and prompt format. We instead choose to evaluate instruction induction / selection methods over a fixed set of models. We are the first ever evaluation script that compares different prompt-engineering methods head to head. 
\fi

\paragraph{In-Context Learning and Existing Benchmarks} As language models have scaled, in-context learning has emerged as a popular paradigm and remains ubiquitous among several autoregressive LLM families \cite{brown2020language, llama, bloom, gptneo, opt}. Benchmarks like BigBench \cite{bigbench} and HELM \cite{helm} have been created for the holistic evaluation of these models. BigBench focuses on few-shot abilities of state-of-the-art large language models, while HELM extends to consider metrics like robustness and bias. However, these benchmarks focus on evaluating and ranking \emph{language models}, and do not address the systematic evaluation of \emph{prompting methods}. Although contemporary work by \citet{yang2023improving} also aims to perform a similar systematic analysis of prompting methods, they focus on simple probability-based prompt selection while we evaluate a broader range of methods including trivial instruction baselines, curated manually selected instructions, and sophisticated automated instruction selection.

\paragraph{Automated Prompt Engineering Methods} There has been interest in performing automated prompt-engineering for target downstream tasks within ICL. This has led to the exploration of various prompting methods, ranging from simple heuristics such as selecting instructions with the lowest perplexity \cite{lowperplexityprompts}, inducing instructions from large language models using a few annotated input-output pairs \cite{ape}, to utilizing RL objectives to create discrete token sequences as prompts \cite{rlprompt}. However, these works restrict their evaluation to small sets of models and tasks with little intersection, hindering their objective comparison. %\mz{For paragraphs that only have one work in the last line, try to shorten the paragraph to squeeze in context.}

\paragraph{Understanding in-context learning} There has been much recent work attempting to understand the mechanisms that drive in-context learning. Studies have found that the selection of demonstrations included in prompts significantly impacts few-shot accuracy across most tasks \cite{whatmakesgoodicexamples, selectionmachinetranslation, knnprompting}. Works like \cite{fantasticallyorderedprompts} also show that altering the ordering of a fixed set of demonstrations can affect downstream accuracy. Prompts sensitive to demonstration permutation often exhibit lower accuracies \cite{relationsensitivityaccuracy}, making them less reliable, particularly in low-resource domains.

Our work aims to bridge these gaps by systematically evaluating the efficacy of popular instruction selection approaches over a diverse set of tasks and models, facilitating objective comparison. We evaluate these methods not only on accuracy metrics, but also on sensitivity metrics to glean additional insights. We recognize that other facets of prompting not covered by instruction engineering exist \cite{weichain, react, selfconsistency}, and defer these explorations to future work. 
% \vspace{-0.5em}
\section{Conclusion}
% \vspace{-0.5em}
Recent advances in multimodal single-cell technology have enabled the simultaneous profiling of the transcriptome alongside other cellular modalities, leading to an increase in the availability of multimodal single-cell data. In this paper, we present \method{}, a multimodal transformer model for single-cell surface protein abundance from gene expression measurements. We combined the data with prior biological interaction knowledge from the STRING database into a richly connected heterogeneous graph and leveraged the transformer architectures to learn an accurate mapping between gene expression and surface protein abundance. Remarkably, \method{} achieves superior and more stable performance than other baselines on both 2021 and 2022 NeurIPS single-cell datasets.

\noindent\textbf{Future Work.}
% Our work is an extension of the model we implemented in the NeurIPS 2022 competition. 
Our framework of multimodal transformers with the cross-modality heterogeneous graph goes far beyond the specific downstream task of modality prediction, and there are lots of potentials to be further explored. Our graph contains three types of nodes. While the cell embeddings are used for predictions, the remaining protein embeddings and gene embeddings may be further interpreted for other tasks. The similarities between proteins may show data-specific protein-protein relationships, while the attention matrix of the gene transformer may help to identify marker genes of each cell type. Additionally, we may achieve gene interaction prediction using the attention mechanism.
% under adequate regulations. 
% We expect \method{} to be capable of much more than just modality prediction. Note that currently, we fuse information from different transformers with message-passing GNNs. 
To extend more on transformers, a potential next step is implementing cross-attention cross-modalities. Ideally, all three types of nodes, namely genes, proteins, and cells, would be jointly modeled using a large transformer that includes specific regulations for each modality. 

% insight of protein and gene embedding (diff task)

% all in one transformer

% \noindent\textbf{Limitations and future work}
% Despite the noticeable performance improvement by utilizing transformers with the cross-modality heterogeneous graph, there are still bottlenecks in the current settings. To begin with, we noticed that the performance variations of all methods are consistently higher in the ``CITE'' dataset compared to the ``GEX2ADT'' dataset. We hypothesized that the increased variability in ``CITE'' was due to both less number of training samples (43k vs. 66k cells) and a significantly more number of testing samples used (28k vs. 1k cells). One straightforward solution to alleviate the high variation issue is to include more training samples, which is not always possible given the training data availability. Nevertheless, publicly available single-cell datasets have been accumulated over the past decades and are still being collected on an ever-increasing scale. Taking advantage of these large-scale atlases is the key to a more stable and well-performing model, as some of the intra-cell variations could be common across different datasets. For example, reference-based methods are commonly used to identify the cell identity of a single cell, or cell-type compositions of a mixture of cells. (other examples for pretrained, e.g., scbert)


%\noindent\textbf{Future work.}
% Our work is an extension of the model we implemented in the NeurIPS 2022 competition. Now our framework of multimodal transformers with the cross-modality heterogeneous graph goes far beyond the specific downstream task of modality prediction, and there are lots of potentials to be further explored. Our graph contains three types of nodes. while the cell embeddings are used for predictions, the remaining protein embeddings and gene embeddings may be further interpreted for other tasks. The similarities between proteins may show data-specific protein-protein relationships, while the attention matrix of the gene transformer may help to identify marker genes of each cell type. Additionally, we may achieve gene interaction prediction using the attention mechanism under adequate regulations. We expect \method{} to be capable of much more than just modality prediction. Note that currently, we fuse information from different transformers with message-passing GNNs. To extend more on transformers, a potential next step is implementing cross-attention cross-modalities. Ideally, all three types of nodes, namely genes, proteins, and cells, would be jointly modeled using a large transformer that includes specific regulations for each modality. The self-attention within each modality would reconstruct the prior interaction network, while the cross-attention between modalities would be supervised by the data observations. Then, The attention matrix will provide insights into all the internal interactions and cross-relationships. With the linearized transformer, this idea would be both practical and versatile.

% \begin{acks}
% This research is supported by the National Science Foundation (NSF) and Johnson \& Johnson.
% \end{acks}

\subsubsection*{Acknowledgments}

We thank Laurel Orr, Xun Huang, Sarah Hooper, Sen Wu, Megan Leszczynski, and Karan Goel for their helpful discussions and feedback on early drafts of the paper.

We gratefully acknowledge the support of NIH under No.\ U54EB020405 (Mobilize), NSF under Nos.\ CCF1763315 (Beyond Sparsity), CCF1563078 (Volume to Velocity), and 1937301 (RTML); ONR under No.\ N000141712266 (Unifying Weak Supervision); ONR N00014-20-1-2480: Understanding and Applying Non-Euclidean Geometry in Machine Learning; N000142012275 (NEPTUNE); the Moore Foundation, NXP, Xilinx, LETI-CEA, Intel, IBM, Microsoft, NEC, Toshiba, TSMC, ARM, Hitachi, BASF, Accenture, Ericsson, Qualcomm, Analog Devices, the Okawa Foundation, American Family Insurance, Google Cloud, Salesforce, Total, the HAI-AWS Cloud Credits for Research program, the Stanford Data Science Initiative (SDSI), and members of the Stanford DAWN project: Facebook, Google, and VMWare. The Mobilize Center is a Biomedical Technology Resource Center, funded by the NIH National Institute of Biomedical Imaging and Bioengineering through Grant P41EB027060. The U.S.\ Government is authorized to reproduce and distribute reprints for Governmental purposes notwithstanding any copyright notation thereon. Any opinions, findings, and conclusions or recommendations expressed in this material are those of the authors and do not necessarily reflect the views, policies, or endorsements, either expressed or implied, of NIH, ONR, or the U.S.\ Government.
Atri Rudra’s research is supported by NSF grant CCF-1763481.


\bibliography{ref}
\bibliographystyle{plainnat}


\appendix
\newpage
\section{Problem Formulation}
\label{app:problem_formulation}

We formulate the problem of sparse model training as sparse matrix approximation with a simple hardware cost model (\cref{sec:problem_formulation}).





We first describe our simple cost model for sparse matrix multiplication to reflect the fact that parallel hardware such as GPUs are block-oriented~\citep{cook2012cuda,gray2017gpu}: accessing one single element from memory costs the same as accessing one whole block of elements.
We then formulate the sparse matrix approximation in the forward pass and the backward pass.
The cost model necessitates narrowing the sparsity pattern candidates to those that are block-aligned.

\paragraph{Cost model}
We model the time cost of an operation based on the number of floating point operations and memory access.
The main feature is that our cost model takes into account \emph{memory coalescing}, where accessing a memory location costs the same as accessing the whole block of $b$ elements around it (typically $b = 16 \text{ or } 32$ depending on the hardware).

Let $\mathrm{Cost}_\mathrm{mem}$ be the memory access cost (either read or write) for a block of $b$ contiguous elements.
Accessing any individual element within that block also costs $\mathrm{Cost}_\mathrm{mem}$ time.
Let $\mathrm{Cost}_\mathrm{flop}$ be the compute cost of a floating point operation.
Let $N_\mathrm{block mem}$ be the number of block memory access, and $N_\mathrm{flop}$ be the number of floating point operations.
Then the total cost of the operation is
\begin{equation*}
  \mathrm{Total cost} = \mathrm{Cost}_\mathrm{mem} \cdot N_\mathrm{block mem} + \mathrm{Cost}_\mathrm{flop} \cdot N_\mathrm{flop}.
\end{equation*}
This cost model is a first order approximation of the runtime on modern hardware (GPUs), ignoring the effect of caching.

\paragraph{Block-aligned sparsity pattern, Block cover, and Memory access cost}
As the memory access cost depends on the number of block of memory being accessed, we describe how the number of nonzero elements in a sparse matrix relates to the number of blocks being accessed.
We first define a \emph{block cover} of a sparse mask.
\begin{definition}
  A sparse mask $M \in \{ 0, 1 \}^{m \times n}$ is $(b_1, b_2)$-\emph{block-aligned} if for any index $i, j$ where $M_{ij} = 1$, we also have $M_{i'j'} = 1$ where:
  \begin{equation*}
     i' = b_1 \lfloor i / b_1 \rfloor + r_1, j' = b_2 \lfloor j / b_2 \rfloor + r_2 \text{ for all } r_1 = 0, 1, \dots, b_1 - 1 \text{ and } r_2 = 0, 1, \dots, b_2 - 1.
  \end{equation*}

  The $(b_1, b_2)$-\emph{block cover} of a sparse mask $M \in \{ 0, 1 \}^{m \times n}$ is the $(b_1, b_2)$-block-aligned mask $M' \in \{ 0, 1 \}^{m \times n}$ with the least number of nonzeros such that $M_{ij} \leq M'_{ij}$ for all $i, j$.
\end{definition}
We omit the block size $(b_1, b_2)$ if it is clear from context.

A sparse mask $M$ being $(b_1, b_2)$ block-aligned means that if we divide $M$ into blocks of size $b_1 \times b_2$, then each block is either all zeros or all ones.
To get the $(b_1, b_2)$-block cover of a sparse mask $M$, we simply divide $M$ into blocks of size $b_1 \times b_2$ and set each block to all ones if any location in that block is one.


For a sparse matrix with sparse mask $M$ on a device with block size $b$, the number of block memory access $N_\mathrm{block mem}$ is the number of nonzero blocks in its $(1, b)$-block cover $M'$ (assuming row-major storage).
This corresponds to the fact that to access a memory location on modern hardware (GPUs), the device needs to load a whole block of $b$ elements around that location.

\paragraph{Fast sparse matrices means block-aligned sparsity pattern}
For sparsity patterns that are not block-aligned, such as the random sparse pattern where each location is independently zero or nonzero, its $(1, b)$-block cover might increase the density by a factor of close to $b$ times (we show this more rigorously in the Appendix).
As memory access often dominates the computation time, this means that non block-aligned sparsity will often result is $b$ times slower execution than block-aligned ones.
In other words, exploiting hardware locality is crucial to obtain speed up.

Therefore, this cost model indicates that instead of searching over sparsity patterns whose total cost is less than some budget $C$, we can instead search over block-aligned patterns whose number of nonzeros is less than some limit $k$.
For our theoretical analysis, we consider sparsity patterns that are $(1, b)$-block-aligned.
In practice, since we need to access both the matrix and its transpose (in the forward and backward pass), we require the sparsity pattern to be both $(1, b)$-block-aligned and $(b, 1)$-block-aligned.
This is equivalent to the condition that the sparsity pattern is $(b, b)$-block-aligned.

\paragraph{Sparse matrix approximation in the forward pass}
We now formulate the sparse matrix approximation in the forward pass.
That is, we have weight matrix $A$ with input $B$ and we would like to sparsify $A$ while minimizing the difference in the output.
For easier exposition, we focus on the case where number of nonzeros in each row is the same.
 \begin{definition}[Forward regression]\label{def:sparse_mask_factorization_before:informal}
Given four positive integers $m \geq n \geq d \geq k \geq 1$, matrices $A \in \R^{m \times d}$ and $B \in \R^{d\times n}$. The goal is to find a $(1, b)$-block-aligned binary mask matrix $M\in \{0, 1\}^{m \times d}$ that satisfies
\begin{align*}
     \min_{M \in \{0,1\}^{m \times d} } & ~ \| A \cdot B - (A \circ M) \cdot B \|_1 \\
    \mathrm{s.t.} & ~ \| M_{i} \|_0 = k , \forall i \in [d]
\end{align*}
where $M_i$ is the $i$-th row of $M$.
\end{definition}

\paragraph{Sparse matrix approximation in the backward pass}
In the backward pass to compute the gradient wrt to the weight matrix $A$, we would like to sparsify the gradient $C B^\top$ while preserving as much of the gradient magnitude as possible.
\begin{definition}[Backward regression]\label{def:sparse_mask_factorization_after:informal}
Given four positive integers $m \geq n \geq d \geq k \geq 1$, matrices $B \in \R^{d \times n}$ and $C \in \R^{m \times n}$.
The goal is to find a $(1, b)$-block-aligned binary mask matrix $M \in \{0,1\}^{m \times d}$ such that
\begin{align*}
  \min_{M \in \{0,1\}^{m \times d} } & ~ \| C \cdot B^\top  - ( C \cdot B^\top ) \circ M \|_1 \\
  \mathrm{s.t.}& ~ \| M_i \|_0 = k , \forall i \in [d]
\end{align*}
where $M_i$ is the $i$-th row of $M$.
\end{definition}
Without making any assumptions, such problems are in general computationally hard \cite{fkt15,rsw16}.








\newpage
\section{Analysis of Butterfly Variants}
\label{sec:butterfly_proofs}

We present formal versions of theorems in~\cref{sec:analysis} regarding variants
of butterfly matrices.
We provide full proofs of the results here.

\subsection{Block Butterfly Analysis}
\label{subsec:block_butterfly_proofs}

\begin{proof}[Proof of \cref{thm:block_butterfly}]
  Let $M$ be an $n \times n$ block butterfly matrix with block size $b$.
  We want to show that $M$ also has a representation as an $n \times n$ block
  butterfly matrix with block size $2b$.

  By \cref{def:bmatrix}, $M$ has the form:
  \begin{equation*}
    M = \vB_{\frac{n}{b}}^{ \left( \frac{n}{b}, b \right)} \vB_{\frac{n}{2b}}^{ \left( \frac{n}{b}, b \right)} \hdots \vB_4^{ \left( \frac{n}{b}, b \right)} \vB_2^{ \left( \frac{n}{b}, b \right)}.
  \end{equation*}
  Notice that we can combine that last two terms to form a matrix of the form
  $\vB_2^{ \left( \frac{n}{2b}, 2b \right)}$ (see \cref{fig:tradeoff}).
  Moreover, other terms in the product of the form
  $\vB_{\frac{n}{2^ib}}^{ \left( \frac{n}{b}, b \right)}$ can also be written as
  $\vB_{\frac{n}{2^{i-1} 2b}}^{ \left( \frac{n}{2b}, 2b \right)}$ (see \cref{fig:tradeoff}).
  Thus $M$ also has the form:
  \begin{equation*}
    M = \vB_{\frac{n}{2b}}^{ \left( \frac{n}{2b}, 2b \right)} \vB_{\frac{n}{4b}}^{ \left( \frac{n}{2b}, 2b \right)} \hdots \vB_2^{ \left( \frac{n}{2b}, 2b \right)}.
  \end{equation*}
  In other words, $M$ is also an $n \times n$ block butterfly matrix with block
  size $2b$.

\end{proof}

\begin{proof}[Proof of \cref{cor:block_butterfly_contains_sparse}]
  \citet[Theorem 3]{dao2020kaleidoscope} states that any $n \times n$ sparse
  matrix with $s$ nonzeros can be represented as products of butterfly matrices
  and their transposes, with $O(s \log n)$ parameters.

  For a constant block size $b$ that is a power of 2, the set of $n \times n$ block butterfly
  matrices of block size $b$ contains the set of regular butterfly matrices by
  \cref{thm:block_butterfly}.
  Therefore any such $n \times n$ sparse matrix also has a representation has
  products of block butterfly matrices of block size $b$ and their transposes,
  with $O(s \log n)$ parameters.
\end{proof}

\subsection{Flat Butterfly Analysis}
\label{subsec:flat_butterfly_proofs}

We prove \cref{thm:flat_butterfly_approx}, which relates the first-order
approximation in the form of a flat butterfly matrix with the original butterfly
matrix.
\begin{proof}[Proof of \cref{thm:flat_butterfly_approx}]
  Let $n = 2^m$ and let $B_1, \dots, B_m \in \mathbb{R}^{n \times n}$ be the $m$
  butterfly factor matrices (we rename them here for simplicity of notation).

  Let
  \begin{equation*}
    E = \prod_{i=1}^m \left(I + \lambda B_i\right) - \left( I + \sum_{i=1}^m \lambda B_i \right).
  \end{equation*}
  Our goal is to show that $\norm{E}_F \leq \epsilon$.

  We first recall some properties of Frobenius norm.
  For any matrices $A$ and $C$, we have
  $\norm{A C}_F \leq \norm{A}_F \norm{C}_F$ and
  $\norm{A + C}_F \leq \norm{A}_F + \norm{C}_F$.

  Expanding the terms of the product in $E$, we have
  \begin{equation*}
    E = \sum_{i=2}^{m} \lambda^i \sum_{s \in [m], \abs{s} = i} \prod_{j \in s} B_j.
  \end{equation*}
  Using the above properties of Frobenius norm, we can bound $E$:
  \begin{align*}
    \norm{E}_F
    &\leq \sum_{i=2}^{m} \lambda^i \sum_{s \in [m], \abs{s} = i} \prod_{j \in s} \norm{B_j}_F \\
    &\leq \sum_{i=2}^{m} \lambda^i \sum_{s \in [m], \abs{s} = i} \prod_{j \in s} B_\mathrm{max} \\
    &= \sum_{i=2}^{m} \lambda^2 m^i \left( B_\mathrm{max} ^i \right) \\
    &= \sum_{i=2}^{m} (\lambda m B_\mathrm{max})^i \\
    &\leq \sum_{i=}^{m} \left(c \sqrt{\epsilon} \right)^i \\
    &\leq c^2 \epsilon \sum_{i=0}^{\infty} (c \sqrt{\epsilon})^i \\
    &\leq \frac{c^2\epsilon}{1 - c\sqrt{\epsilon}} \\
    &\leq \epsilon,
  \end{align*}
  where in the last step we use the assumption that $c \leq \frac{1}{2}$.
\end{proof}

We now bound the rank of the first-order approximation.
\begin{proof}[Proof of \cref{thm:flat_butterfly_rank}]
  Let $M^* = I + \sum_{i=1}^{m} \lambda B_i$.
  Note that any entry in $\sum_{i=1}^{m} \lambda B_i$ has absolute value at most
  \begin{equation*}
    m \lambda B^\infty_\mathrm{max} \leq \frac{c \sqrt{\epsilon} B^\infty_\mathrm{max}}{B_\mathrm{max}} \leq \frac{1}{4},
  \end{equation*}
  where we use the assumption that $B^\infty_\mathrm{max} \leq B_\mathrm{max}$
  and $c \leq \frac{1}{4}$.

  Thus any diagonal entry in $M^*$ has absolute value at least
  $1 - \frac{1}{4} = \frac{3}{4}$ and the off-diagonal entries are at most
  $\frac{c \sqrt{\epsilon}B^\infty_\mathrm{max}}{bm}$.

  \citet[Theorem 1.1]{alon2009perturbed} states that: there exists some $c > 0$
  such that for any real $M \in \mathbb{R}^{n \times n}$, if the diagonal
  elements have absolute values at least $\frac{1}{2}$ and the off-diagonal
  elements have absolute values at most $\epsilon$ where
  $\frac{1}{2 \sqrt{n}} \leq \epsilon \leq \frac{1}{4}$, then
  $\rank(M) \geq \frac{c \log n}{\epsilon^2 \log 1/\epsilon}$.

  Applying this theorem to our setting, we have that
  \begin{equation*}
    \rank(M^*) \geq \Omega \left( \left( \frac{B_\mathrm{max}}{B^\infty_\mathrm{max}} \right)^2 \cdot \frac{m}{\epsilon \log \left( \frac{B_\mathrm{max}}{B^\infty_\mathrm{max}} \right) }\right).
  \end{equation*}
  We just need to show that
  $\frac{B^\infty_\mathrm{max}}{B_\mathrm{max}} \geq \frac{1}{2 c \sqrt{\epsilon n}}$
  to satisfy the condition of the theorem.

  Indeed, we have that
  $1 \leq \frac{B_\mathrm{max}}{B^\infty_\mathrm{max}} \leq \sqrt{2n}$ as each
  $\norm{B_i}_0 \leq 2n$.
  Combining the two conditions on
  $\frac{B_\mathrm{max}}{B^\infty_\mathrm{max}}$, we have shown that
  $1 \leq \frac{B_\mathrm{max}}{B^\infty_\mathrm{max}} \leq 2 c \sqrt{\epsilon n}$.
  This concludes the proof.
\end{proof}


\subsection{Flat Block Butterfly + Low-rank Analysis}
\label{subsec:flat_butterfly_lr_proofs}

We show that flat butterfly + low-rank (an instance) of sparse + low-rank, is
more expressive than either sparse or low-rank alone.
We adapt the argument from~\citet{scatterbrain} to show a generative process
where the attention matrix can be well approximated by a flat butterfly +
low-rank matrix, but not by a sparse or low-rank alone.

We describe here a generative model of an input sequence to attention, parameterized by the inverse temperature $\beta \in \mathbb{R}$ and the intra-cluster distance $\Delta \in \mathbb{R}$.
\begin{process}
  \label{ex:generative}
  Let $Q \in \mathbb{R}^{n \times d}$, where $d\geq\Omega(\log^{3/2}(n))$, with every row of $Q$ generated randomly as follows:
  \begin{enumerate}[leftmargin=*,nosep,nolistsep]
    \item For $C = \Omega(n)$, sample $C$ number of cluster centers $c_1, \dots, c_C \in \mathbb{R}^{d}$ independently from $\mathcal{N}(0, I_d/\sqrt{d})$.
    \item For each cluster around $c_i$, sample $n_i = b$ number of elements around $c_i$, of the form $z_{ij} = c_i + r_{ij}$ for $j = 1, \dots, n_i$ where $r_{ij} \sim \mathcal{N}(0, I_d \Delta/\sqrt{d})$.
    Assume that the total number of elements is $n = c b$ and $\Delta\leq O(1/\log^{1/4} n)$.
  \end{enumerate}
  Let $Q$ be the matrix whose rows are the vectors $z_{ij}$ where $i = 1, \dots, C$ and $j = 1, \dots, n_i$.
  Let $A = Q Q^\top$ and let the attention matrix be $M_\beta = \exp(\beta \cdot A)$.
\end{process}

\begin{theorem}
  \label{thm:temperature}
  Let $M_\beta$, be the attention matrix in~\cref{ex:generative}. Fix $\epsilon\in (0,1)$. Let $R \in \mathbb{R}^{n \times n}$ be a matrix.
  Consider low-rank, sparse, and sparse + low-rank approximations to $M_\beta$.
    Assume $(1 - \Delta^2) \log n  \leq \beta \leq O(\log n)$.
    \begin{enumerate}
        \item \textbf{Flat butterfly + low-rank}: There exists a flat butterfly + low-rank $R$ with $n^{1+o(1)}$ parameters with $\|M_\beta-R\|_F\leq \eps n$.
        \item \textbf{Low-rank}: If $R$ is such that $n-\rank(R)=\Omega(n)$, then $\|M_\beta-R\|_F\geq \Omega(n)$.
        \item \textbf{Sparse}: If $R$ has sparsity $o(n^2)$, then $\|M_\beta - R\|_F\geq \Omega(n)$.
   \end{enumerate}
\end{theorem}

\begin{proof}[Proof sketch]
  As the argument is very similar to that of \citet[Theorem 1]{scatterbrain}, we
  describe here the modifications needed to adapt their proof.

  The main difference between our generative process and that of
  \citet{scatterbrain} is that each cluster has the same number of elements,
  which is the same as the block size.
  The resulting attention matrix will have a large block diagonal component,
  similar to that \citet{scatterbrain}.
  However, all the blocks in the block diagonal component has the same block
  size, which is $b$.
  Moreover, a flat block butterfly of block size $b$ contains a block diagonal
  component of block size $b$.
  Therefore, this flat block butterfly matrix plays the same role as the sparse
  matrix in the proof of \citet{scatterbrain}.
  The rest of the argument follows that of theirs.
\end{proof}


\newpage
\subsection{Proofs of \Cref{sec:ergodicity-hmc}}


% \begin{proof}
%\end{proof}

\subsubsection{Proof of \Cref{theo:irred_harris} }
\label{sec:proof-crefth-harris_0}
We first prove  \eqref{theo:irred_harris_a}.  Under the assumption that $\F$ is twice continuously
  differentiable, it follows by a straightforward induction, that for
  all $h >0$ and $q \in \rset^d$, $p \mapsto
  \Phiverletq[h][k](q,p)$, defined by  \eqref{eq:def_Phiverletq}, and $p \mapsto \gperthmc[k](q,p)$, defined by \eqref{eq:def_gperthmc}, are
  continuously differentiable and for all $(q,p) \in \rset^d \times
  \rset^d$,
\begin{equation}
  \Jac_{p,\gperthmc[T]}(q,p) =  \sum_{i=1}^{T-1}(T-i)\defEns{\nabla^2 \F \circ \Phiverletq[h][i](q,p)} \Jac_{p,\Phiverletq[h][i]}(q,p) \eqsp,
\end{equation}
where for all $q \in \rset^d$, $\Jac_{p,\gperthmc[k]}(q,p)$ ($\Jac_{p,\Phiverletq[i][h]}(q,p)$ respectively) is the Jacobian of the function $\tilde{p} \mapsto
\gperthmc[k](q,\tilde{p})$ ($\tilde{p} \mapsto
\Phiverletq[i][h](q,\tilde{p})$ respectively) at $p \in \rset^{d}$.


%  Under \Cref{assum:regOne}, $\sup_{x \in \rset^d} \normLigne{\nabla^2 \F(x)}
% \leq \constzero$ and by
% \Cref{lem:bound_first_iterate_leapfrog_a},
%  $ \sup_{(q,p) \in \rset^d \times \rset^d} \normLigne{\nabla_p \Phiverletq[h][i](q,p)} \leq (1+h \vartheta_1(h))^i$ for any $i \in \nsets$.
% Therefore for any $h >0$, $T \in \nsets$, setting $S = hT$ and using that $\tilde{h} \mapsto \vartheta_1(\tilde{h})$ is nondecreasing and greater than $1$ on $\rset_+^*$ and for any $u,s \geq 0$, $u \geq 1$, $(1+s/u)^{u-1} \leq \log(s+1) \rme^s$, we get that
Under \Cref{assum:regOne}, $\sup_{x \in \rset^d} \normLigne{\nabla^2 \F(x)}
 \leq \constzero$, therefore by \Cref{lem:inverse_1}, we have that for any $T \in \nsets$ and $h >0$,
\begin{equation}
  \label{eq:inverse_1}
 \sup_{(\q,\p) \in \rset^d \times \rset^d} \norm{\Jac_{p,\gperthmc[T]}(q,p)}
 \leq  T (\{1 + h \constzero^{1/2} \vartheta_1(h \constzero^{1/2})\}^T  -1) /h  \eqsp.
\end{equation}
%$\sup_{p \in \rset^d } \nabla_p \gperthmc[k](q,p) \leq C$.
% \begin{equation}
% \label{lem:inverse_1}
% \sup_{p \in \rset^d } \nabla_p \gperthmc[k](q,p) \leq C \eqsp.
% \end{equation}
% Then for all $q, p_1,p_2 \in \rset^d$,
% \begin{equation}
% \label{lem:inverse_1}
% \norm{\gperthmc[k](q,p_1) -\gperthmc[k](q,p_2)} \leq C \norm{p_1 - p_2} \eqsp.
% \end{equation}
For any $q \in \rset^d$, $T\in \nsets$ and $h >0$, define $\phia_{q,T,h}(p)$  for all  $p \in \rset^d$ by
\begin{equation}
  \phia_{q,T,h} (p) = p-(h/T) \gperthmc[T](q,p) \eqsp.
\end{equation}
It is a well known fact (see for example
\cite[Exercise 3.26]{duistermaat:kolk:2004}) that if
\begin{equation}
  \label{eq:inverse_1_2}
  \sup_{(q,p) \in \rset^d \times \rset^d} (h/T)\norm{ \Jac_{p,\gperthmc[T]}(q,p)} < 1 \eqsp,
\end{equation}
then for any $q \in \rset^d$, $\phia_{q,T,h}$ is a
diffeomorphism and  therefore by \eqref{eq:qk}, the same conclusion holds
for $p \mapsto \Phiverletq[h][T](q,p)$. Using \eqref{eq:inverse_1}, if $T \in \nsets$ and $h > 0$  satisfies \eqref{eq:condition-h,T-harris},
then the condition \eqref{eq:inverse_1_2} is verified and as a result \eqref{theo:irred_harris_a}.

Denoting for any $q \in \rset^d$ by $\Phiverletqi[h][T](q,\cdot) : \rset^d \to \rset$ the
continuously differentiable inverse of $p \mapsto
\Phiverletq[h][T](q,p)$ and using a change of variable with $\Phiverletqi[h][T](q,\cdot)$ in \eqref{eq:def_kernel_hmc} concludes the proof of \eqref{eq:def_kernel_hmc_false_density}.

We now show that $\Tker_{h,T}$ satisfies the condition which implies that $\Pkerhmc[h][T]$ is a \Tkernel. We first establish some estimates on the function $(q,p) \mapsto \Phiverletqi[h][T](q,p)$. By
\eqref{eq:inverse_1_2} and \eqref{eq:qk}, for any $q,p,v \in \rset^d$, there exists $\varepsilon \in \ooint{0,1}$ such that $  \normLigne{\Phiverletq[h][T](q,p)-\Phiverletq[h][T](q,v)} \geq (hT) \normLigne{\phi_{q,T,h}(p)-\phi_{q,T,h}(v)} \geq (hT) (1-\varepsilon)\norm{p-v}$ which implies that that there exists $C \geq 0$ satisfying
\begin{equation}
  \label{eq:regularity_phinverse1}
  \begin{aligned}
    \norm{\Phiverletqi[h][T](q,p)-\Phiverletqi[h][T](q,v)} &\leq (1-\varepsilon)^{-1} \norm{v-p}\eqsp, \\
    \norm{  \Phiverletqi[h][T](q,p)} &\leq C\defEns{\norm{\p} + \norm{\Phiverletq[h][T](q,0)}} \eqsp.
  \end{aligned}
\end{equation}
In addition, for $q,x,p \in \rset^d$, we have setting $\tilde{q} = \Phiverletqi[h][T](q,p)$ that
\begin{align}
  \nonumber
  \normLigne{\Phiverletqi[h][T](q,p) - \Phiverletqi[h][T](x,p)} &= \normLigne{\tilde{q} - \Phiverletqi[h][T](x, \Phiverletq[h][T](q,\tilde{q}))} \\
  \nonumber
                                                                &= \normLigne{\Phiverletqi[h][T](x, \Phiverletq[h][T](x,\tilde{q})) - \Phiverletqi[h][T](x, \Phiverletq[h][T](q,\tilde{q}))} \eqsp,
\end{align}
which implies by \eqref{eq:regularity_phinverse1} and \Cref{lem:bound_first_iterate_leapfrog_a}
that there exists $C \geq 0$ satisfying
\begin{equation}
  \label{eq:regularity_phinverse2}
  \norm{\Phiverletqi[h][T](q,p) - \Phiverletqi[h][T](x,p)} \leq C \norm{q-x} \eqsp.
\end{equation}


We now can prove that $\Tker_{h,T}$ is the continuous component of $\Pkerhmc[h][T]$. First by \eqref{eq:def_tker}, for all $\eventB \in \borelSet(\rset^d)$,
\begin{equation}
\label{eq:minoration_pseudo_density_P}
    \Tker_{h,T}(q, \eventB) \geq (2 \uppi)^{-d/2} \Leb(\eventB)
 \times \inf_{\bar{q} \in \eventB} \defEns{ \balphaacc(q,\bar{q}) \rme^{-\norm{\Phiverletqi_q(\bar{q})}^2/2}\detj_{\Phiverletqi[h][T](q,\cdot)}(\bar{q})} \eqsp,
\end{equation}
with the convention $0 \times \plusinfty = 0$ and
\begin{equation}
%  \label{eq:7}
  \balphaacc(q,\bar{q}) =  \alphaacc\defEns{(q,\Phiverletqi[h][T](q,\bar{q})),\Phiverlet[h][T](q,\Phiverletqi[h][T](q,\bar{q}))}\eqsp. 
\end{equation}
Since the function $  (q,p) \mapsto (\Phiverletq[h][T](q,p),\Phiverletqi[h][T](q,p), \detj_{\Phiverletqi[h][T](q,\cdot)}(p)) $
is continuous on $\rset^d\times \rset^d$ by \Cref{lem:bound_first_iterate_leapfrog_a}, \eqref{eq:regularity_phinverse1} and \eqref{eq:regularity_phinverse2}, and for any $q,p \in \rset^d$, $\Jac_{\Phiverletq[h][T](\q,\cdot)}(\Phiverletqi[h][T](q,p))
\Jac_{\Phiverletqi[h][t](q,\cdot)}(\p) = \operatorname{I}_n$, we get that  $\Tker_{h,T}(q,\eventB) >0$ for all $q \in \rset^d$ and all compact set $\eventB$ satisfying $\Leb(\eventB) > 0$. Therefore, using that the Lebesgue measure is regular which implies that for any $\msa \in \mcb(\rset^d)$ with $\Leb(\msa) >0$, there exists a compact set $\msb \subset\msa$, $\Leb(\msb)>0$, we can conclude that $\Pkerhmc[h][T]$ is irreducible with respect to the Lebesgue measure. In addition, we get  $\Tker_{h,T}(q,\rset^d) >0$, and therefore we obtain that $\Pkerhmc[h][T]$ is aperiodic.  Similarly we get that any compact set is $(1,\Leb)$-small.

It remains to show that for any $\eventB \in\mcb(\rset^d)$, $q \mapsto \Tker_{h,T}(q,\eventB)$ is lower semi-continuous which is a straightforward consequence of Fatou's Lemma and that for any $p \in \rset^d$, $q \mapsto (\Phiverlet[h][T](q,p), \Phiverletqi[h][T](q,p),\detj_{\Phiverletqi[h][T](q,\cdot)}(p))$ is continuous.

% We now show that all the compact sets are $(1,\Leb)$-small. Let $\eventB \subset \rset^d$ be compact.  Using
% \eqref{eq:inverse_1_2} there exists $C \geq 0$ such that for all
% $q,p,v \in \rset^d$, $ \norm{p-v} \leq C \normLigne{
%   \Phiverletq[h][T](q,p)- \Phiverletq[h][T](q,v)}$. It follows
% that for all $p \in \rset^d$, $\sup_{q \in \eventB} \normLigne{
%   \Phiverletqi[h][T](q,p)} \leq C\defEnsLigne{\norm{\p} + \sup_{q \in
%     \eventB} \normLigne{\Phiverletq[h][T](q,0)}}$. Using this upper
% bound and $\Jac_{\Phiverletq(\q,\cdot)}(\Phiverletqi_q(\p))
% \Jac_{\Phiverletqi_q}(\tilde{\p}) = \operatorname{I}_n$ in
% \eqref{eq:minoration_pseudo_density_P}, where $\operatorname{I}_n$ is
% the identity matrix, we deduce that there exists $\varepsilon >0$ such
% that for all $\eventA\in \borelSet(\rset^d)$, $\eventA \subset \eventB$,
% \begin{equation}
%   \inf_{q \in \eventB} \Pkerhmc[h][T](q, \eventA)  \geq \varepsilon \Leb(\eventA) \eqsp,
% \end{equation}
% and therefore $\eventB$ is small for $\Pkerhmc[h][T]$.
% \begin{equation}
% \norm{  \Phiverletqi_q(p_1)-  \Phiverletqi_q(p_2)} \leq C \norm{p_1-p_2} \eqsp,
% \end{equation}
% This result, \eqref{eq:def_acc_ratio}, \Cref{lem:bound_first_iterate_leapfrog} and \eqref{eq:minoration_pseudo_density_P} imply that $
% \Pkerhmc[h][T]$ is irreducible with respect to the Lebesgue measure
% and aperiodic.
% and any ball on $\rset^d$ is small.

% A straightforward adaptation of the proof of \cite[Corollary
% 2]{tierney:1994} shows that $ \Pkerhmc[h][T]$ is Harris recurrent, see \Cref{propo:harris_rec} in \Cref{sec:harr-recurr-metr}. The desired conclusion then follows from \cite[Theorem 13.0.1]{meyn:tweedie:2009}.
 % \Cref{theo:irred_harris} implies
% that for all $T \geq 0$, there exists $\hirr>0$ such that for all $h \in \ocintLigne{0,\hirr}$ and all $\q \in \rset^d$
%   \begin{equation}
% \lim_{n \to \plusinfty}    \tvnorm{\delta_\q \Pkerhmc[h][T]^n - \pi} = 0 \eqsp.
%   \end{equation}


% By \cite[Theorem 17.1.4, Theorem
% 17.1.7]{meyn:tweedie:2009}, it suffices the to prove that for all
% bounded harmonic function $\harmonic : \rset^d \to \rset$ satisfying
% \begin{equation}
%   \label{eq:def_harm}
%   \Pkerhmc[h][T]\harmonic = \harmonic \eqsp,
% \end{equation}
% %$\Pkerhmc[h][T]\harmonic = \harmonic$,
% are constant. First since $\Pkerhmc[h][T]$ is irreducible with respect
% to the Lebesgue measure and aperiodic, by \cite[Theorem
% 14.0.1]{meyn:tweedie:2009} for $\Leb$-almost all $q$ we get $\lim_{n
%   \to \plusinfty} \Pkerhmc[h][T]^n \harmonic(q) = \pi(\harmonic)$ and therefore by
% \eqref{eq:def_harm} $\harmonic(q) = \pi(\harmonic)$. Therefore we get that for all $q \in \rset^d$ by \eqref{eq:def_kernel_hmc_false_density},
% \begin{multline}
%    \Pkerhmc[h][T]^n \harmonic(q) = \pi(\harmonic)  \int_{\rset^d}  \alphaacc\defEns{(q,\tilde{p}),\Phiverlet[h][T](q,\tilde{p})} \rme^{-\norm{\tilde{p}}^2/2} \rmd \tilde{p} \\
% +   \harmonic(x) \int_{\rset^d}  \parentheseDeux{1-\alphaacc\defEns{(q,\tilde{p}),\Phiverlet[h][T](q,\tilde{p})}} \rme^{-\norm{\tilde{p}}^2/2} \rmd \tilde{p} \eqsp.
% \end{multline}
% Combining this result with \eqref{eq:def_harm}, we get for all $q \in \rset^d$
% \begin{equation}
% (\harmonic(q)-\pi(\harmonic)) \int_{\rset^d} \alphaacc\defEns{(q,\tilde{p}),\Phiverlet[h][T](q,\tilde{p})} \rme^{-\norm{\tilde{p}}^2/2} \rmd \tilde{p} = 0\eqsp.
% \end{equation}
% It follows from \Cref{lem:bound_first_iterate_leapfrog} and \eqref{eq:def_acc_ratio} that for all $q \in \rset^d$, $\harmonic(q) = \pi(\harmonic)$
% which concludes the proof.
% =======
% The proof of \ref{theo:irred_harris_b} using a change of variable with $\Phiverletqi[h][T](q,\cdot)$.

% We now show that $\Tker_{h,T}$ satisfies the condition which implies that $\Pkerhmc[h][T]$ is a \Tkernel.
% First, for all $\eventB \in \borelSet(\rset^d)$,
% \begin{equation}
% \label{eq:minoration_pseudo_density_P}
% \Tker_{h,T}(q, \eventB) \geq (2 \uppi)^{-d/2} \Leb(\eventB)
%  \times \inf_{\bar{q} \in \eventB} \defEns{\alphaacc\defEns{(q,\Phiverletqi[h][T](q,\bar{q})),\Phiverlet[h][T](q,\Phiverletqi[h][T](q,\bar{q}))} \rme^{-\norm{\Phiverletqi[h][T](q,\bar{q})}^2/2} \detj_{\Phiverletqi[h][T]}(q,\bar{q})} \eqsp,
% \end{equation}
% with the convention $0 \times \plusinfty = 0$. Since $\Phiverletqi[h][T](q,\cdot)$
% is a diffeomorphism on $\rset^d$, we get that  $
% \Tker_{h,T}(q,\eventB) >0$ for all $q \in \rset^d$ and all compact set $\eventB$ satisfying $\Leb(\eventB) > 0$. Since the Lebesgue measure is regular, this implies that $\Pkerhmc[h][T]$ is irreducible with respect to the Lebesgue measure and aperiodic.

% By Fatou's Lemma, for any $\eventB \in\mcb(\rset^d)$, $q \mapsto \Tker_{h,T}(q,\eventB)$ is lower semi-continuous.
% We now show that all the compact sets are small. Let $\eventB \subset \rset^d$ be compact.  Using
% \eqref{eq:inverse_1_2} there exists $C \geq 0$ such that for all
% $q,p,v \in \rset^d$, $ \norm{p-v} \leq C \normLigne{
%   \Phiverletq[h][T](q,p)- \Phiverletq[h][T](q,v)}$. It follows
% that for all $p \in \rset^d$, $\sup_{q \in \eventB} \normLigne{
%   \Phiverletqi[h][T](q,\p)} \leq C\defEnsLigne{\norm{\p} + \sup_{q \in
%     \eventB} \normLigne{\Phiverletq[h][T](q,0)}}$. Using this upper
% bound and $\Jac_{\Phiverletq(\q,\cdot)}(\Phiverletqi[h][T](q,\p))
% \Jac_{\Phiverletqi[h][T]}(q,\tilde{\p}) = \operatorname{I}_n$ in
% \eqref{eq:minoration_pseudo_density_P}, where $\operatorname{I}_n$ is
% the identity matrix, we deduce that there exists $\varepsilon >0$ such
% that for all $\eventA\in \borelSet(\rset^d)$, $\eventA \subset \eventB$,
% \begin{equation}
%   \inf_{q \in \eventB} \Pkerhmc[h][T](q, \eventA)  \geq \varepsilon \Leb(\eventA) \eqsp,
% \end{equation}
% and therefore $\eventB$ is small for $\Pkerhmc[h][T]$.
% >>>>>>> f8207bad5c0353bdfe37210ffc64a715e92e53ed

Finally, the last statements of \ref{theo:irred_harris_c} follows from \Cref{propo:harris_rec} in \Cref{sec:harr-recurr-metr} which implies that  $ \Pkerhmc[h][T]$ is Harris recurrent and  \cite[Theorem 13.0.1]{meyn:tweedie:2009} which implies  \eqref{eq:harris-theorem}.

\subsubsection{Proof of \Cref{theo:irred_D}}
\label{sec:proof-crefth_irred_D}
We use \Cref{coro:irred}. Indeed $\Pkerhmc[h][T]$ is
of form \eqref{eq:def_pkerb} and it is straightforward to check that it
satisfies \Cref{assumG:phi} (note that \Cref{lem:bound_first_iterate_leapfrog_a}
shows that $\Phiverlet[h][T]$ is a Lipshitz function on $\rset^{2d}$).

We now check that $\Pkerhmc[h][T]$ satisfies \Cref{assumG:irred_b}($\rassG,0,\MassG$) for all $\rassG,\MassG \in
\rset_+^*$ using \Cref{le:degree_application}.  By \eqref{eq:qk}, for all $T \in \nsets$, $h >0$, $q,p \in \rset^d$,
\begin{equation}
  \label{eq:phiverlet_gqth}
  \Phiverletq[h][T](q,p) = T
h p + g_{q,T,h}(p)
\end{equation}
where $g_{q,T,h}(p) = q - (Th^2/2) \nabla \F(q) -
h^2 \gperthmc[T](q,p)$ where $\gperthmc[T]$ is defined by \eqref{eq:def_gperthmc}. \Cref{lem:inverse_1} shows that for any $T \in \nsets$ and $h >0$, it holds that
\begin{equation}
    \label{eq:2:theo:irred_D}
\sup_{p,v,q \in  \rset^d} \frac{\norm{g_{q,T,h}(p)-g_{q,T,h}(v)}}{\norm{p - v}} \leq T h [\{1 + h \constzero^{1/2} \vartheta_1(h \constzero^{1/2} )\}^T-1] \eqsp,
\end{equation}
which implies that the condition
\Cref{le:degree_application}-\ref{propo:irred_b_item_i} is satisfied. To check that
condition  \Cref{le:degree_application}-\ref{propo:irred_b_item_ii} holds, we consider separately the two cases: $\beta <1$ and $\beta =1$.

\begin{enumerate}[label=$\bullet$, wide, labelwidth=!, labelindent=0pt]
\item Consider first the case $\beta <1$. By \Cref{assum:regOne}-\ref{assum:regOne_b},
for any $T \in \nsets$ and $h >0$, we get
\begin{equation}
\norm{\gperthmc[T](\q,\p)} \leq  T \sum_{i=1}^{T-1} \norm{\nabla \F \circ \Phiverletq[h][i](\q,\p)} \leq
\constzeroT T \sum_{i=1}^{T-1} \defEns{ 1 + \norm{\Phiverletq[h][i](\q,\p)}^{\expozero}}
 \eqsp.
\end{equation}
Hence, by \Cref{lem:bound_first_iterate_leapfrog_b}-\ref{lem:bound_first_iterate_leapfrog_1}
there exists $C \geq 0$ such that for all $R\in \rset_+^*$ and
$q,p \in \rset^d$, $\norm{q} \leq R$,
\begin{equation}
\label{eq:3:theo:irred_D}
\norm{g_{q,T,h}(p)} \leq C \defEns{1+R^{\beta} +\norm{p}^{\expozero}} \eqsp,
\end{equation}
which implies that condition \ref{propo:irred_b_item_ii} of \Cref{le:degree_application} holds for any $T \in \nsets$ and $h >0$.

\item Consider now the case $\beta =1$.  For any $T \in \nsets$, $h >0$,  $q,p \in \rset^d$ we get using \Cref{assum:regOne}-\ref{assum:regOne_a}
\begin{align}
  \norm{g_{q,T,h}(p)} &\leq \norm{q} + Th^2 \constzero  \norm{q} /2 + Th^2 \norm{\nabla U(0)} /2\\
  & \qquad \qquad +h^2 \norm{\gperthmc[T](q,p) - \gperthmc[T](q,0)} + h^2 \norm{ \gperthmc[T](q,0)} \eqsp.
\end{align}
Therefore using \Cref{lem:inverse_1}, for any $q,p \in \rset^d$, $\norm{q} \leq R$ for $R \geq 0$, for any $T \in \nsets$ and $h >0$ satisfying \eqref{eq:condition-h,T-harris}, there exists $C \geq 0$ such that
\begin{equation}
  \norm{g_{q,T,h}(p)} \leq C + h T  [ \{1+ h \constzero^{1/2} \vartheta_1(h\constzero^{1/2})\}^T-1]  \norm{p} \eqsp,
\end{equation}
showing that condition \ref{propo:irred_b_item_ii} of \Cref{le:degree_application} is satisfied.
\end{enumerate}

Therefore,  \Cref{le:degree_application} can be applied and for any $T \in \nsets$ and $h >0$ if $\beta <1$ and for any $h > 0$ and $T \in \nsets$ satisfying \eqref{eq:condition-h,T-harris} if $\beta =1$, $\Pkerhmc[h][T]$ satisfies \Cref{assumG:irred_b}($\rassG,0,\MassG$) for all $\rassG,\MassG \in
\rset_+^*$.  \Cref{coro:irred} concludes the proof of \ref{theo:irred_D_a} and \ref{theo:irred_D_b}.
The last statement then follows from   \cite[Theorem 14.0.1]{meyn:tweedie:2009}.

% Using this result and \Cref{theo:irred}, we get that for all $\rassG,\MassG  \in \rset_+^*$ there exists $\varepsilon >0$ such that
% for all $\q \in \ball{0}{\rassG}$ and $\eventA \in \borelSet(\rset^d)$,
% \begin{equation}
%   \Pkerhmc[h][T](q, \eventA) \geq \varepsilon \Leb(\eventA \cap \ball{0}{M}) \eqsp.
% \end{equation}
% \Cref{coro:irred} Combining this result and \eqref{eq:1:theo:irred_D} concludes the proof of \ref{theo:irred_D_a} and \ref{theo:irred_D_b}.

% The proof is a consequence of \Cref{lem:bound_first_iterate_leapfrog},
% \Cref{le:degree_application} and \Cref{theo:irred}.  \alain{give some
%   details}

%%% Local Variables:
%%% mode: latex
%%% TeX-master: "main"
%%% End:

\newpage
\section{Neural Tangent Kernel, Convergence, and Generalization}
\label{sec:appx_ntk}

Our analysis relies on the neural tangent kernel (NTK)~\citep{jacot2018neural} of the network.
\begin{definition}
  Let $f(\cdot, \theta) \colon \mathbb{R}^{d} \to \mathbb{R}$ be the function specified by a neural network with parameters $\theta \in \mathbb{R}^p$ and input dimension $d$.
  The parameter $\theta$ is initialized randomly from a distribution $P$.
  Then its neural tangent kernel (NTK) \citep{jacot2018neural} is a kernel $K \colon \mathbb{R}^{d} \times \mathbb{R}^{d} \to \mathbb{R}$ defined by:
  \begin{equation*}\label{eq:kernel}
    K(x, y) = \E_{\theta \sim P} \left[ \left\langle \frac{\partial f(x; \theta)}{\partial \theta}, \frac{\partial f(y; \theta) }{\partial \theta} \right\rangle\right].
  \end{equation*}
\end{definition}

We can relate the training and generalization behavior of dense and sparse
models through their NTK.
The standard result~\citep{sy19} implies the following.
\begin{proposition}
  \label{thm:ntk}
  Let $f_\mathrm{dense}$ denote a ReLU neural network with $L$ layers with dense weight matrices $\theta_\mathrm{dense}$ with NTK $K_\mathrm{dense}$, and let $f_\mathrm{sparse}$ be the ReLU neural network with the same architecture and with weight matrices $\theta_\mathrm{sparse}$ whose rows are $k$-sparse, and with NTK $K_\mathrm{sparse}$.
  Let $x_1, \dots, x_N$ be the inputs sampled from some distribution $P_X$.
  Suppose that the empirical NTK matrices $K_d = K_\mathrm{dense}(x_i, x_j)$ and $K_s = K_\mathrm{sparse}(x_i, x_j)$ for $(i, j) \in [N] \times [N]$ satisfy $\| K_d - K_s \| \leq \delta$.

  {\bf Training.}
  We knew the the number of iterations of dense network is $\lambda_{\min}(K_d)^{-2} n^2 \log(1/\epsilon)$ to reach the $\epsilon$ training loss. For sparse network we need $(\lambda_{\min}(K_d) -\delta)^{-2} n^2 \log(1/\epsilon)$.

  {\bf Generalization.}
  We knew the the number of iterations of dense network is $\lambda_{\min}(K_d)^{-2} n^2 \log(1/\epsilon)$ to reach the generalization error $\epsilon$ training loss. For sparse network we need $(\lambda_{\min}(K_d) -\delta)^{-2} n^2 \log(1/\epsilon)$.
\end{proposition}
These results relate the generalization bound of sparse models to that of dense models.
\newpage
The Predictive Modeling framework with HyperBO can be proven to converge under the Theorem below.

\begin{theorem}
	Let $\delta \in (0,1)$ and $\eta \in (0,1)$. Assuming the kernel functions used in both the BO in the function space and the HyperBO, $k(.,.)$ provides a high probability guarantee on the sample paths of GP derivative to be L'-Lipschitz continuous, and the function $f(A^{t}_{p}(\boldsymbol{\theta}))$ is L-Lipschitz continuous. There exists a $t_O = T_{S} \leq T_{O}$ beyond which $\left\|\boldsymbol{\theta}^{*}-\boldsymbol{\theta}_{t_{O}}\right\| < \epsilon$ is satisfied with probability $1-\delta$. Furthermore, the average cumulative regret of the Predictive Modeling framework will converge to $\textcolor{red}{\lim_{T_{I} \rightarrow \infty}} R_{T}/T = \epsilon L$.
\end{theorem}

\begin{proof}
If BO with Thompson Sampling is used \cite{Basu2017}, then we know that
\begin{align*}
Prob(\left\|\boldsymbol{\theta}^{*}-\boldsymbol{\theta}_{t}\right\|>\epsilon) \leq C^{0}_{\epsilon}\text{exp}(-C^{1}_{\epsilon}t)
\end{align*}
where $C^{0}_{\epsilon}$, $C^{1}_{\epsilon}$ are $\epsilon$ dependent constants.

This implies we can set any arbitrary $\epsilon \ll 1$ and an arbitrary low probability $\delta \ll 1$. Then there exists a $t_{O} = T_{S}$ beyond which $\left\|\boldsymbol{\theta}^{*}-\boldsymbol{\theta}_{t_{O}}\right\| < \epsilon$ happens with high probability ($1 - \delta$).
\begin{align*}
\delta = & C^{0}_{\epsilon}\text{exp}(-C^{1}_{\epsilon}T_{S})\\
\implies T_{S} = & C^{1}_{\epsilon}\text{log}(\frac{C^{0}_{\epsilon}}{\delta})
\end{align*}

Using regret as defined in \cite{Srinivas:2010:GPO:3104322.3104451}, and recalling that $A_{p}^{T}(\boldsymbol{\theta}_{t})=\boldsymbol{x}_{T}$, we can write the cumulative regret as:

\begin{align*}
R_{T} = & \sum_{t=1}^{T_{O}}\sum_{t'=1}^{K}|f(\boldsymbol{x}^{*})-f(A_{p}^{T}(\boldsymbol{\theta}_{t}))|
\end{align*}
where $T=(t-1)\times K+t'$ is the actual iterations that the BO in the function space has gone through. Next, we break down $R_{T}$ by introducing $f(A_{p}^{T}(\boldsymbol{\theta}^{*}))$, where $\boldsymbol{\theta}^{*}=\text{argmax }g(\boldsymbol{\theta})$.
\begin{align*}
R_T= & \sum_{t=1}^{T_{O}}\sum_{t'=1}^{K} |f(\boldsymbol{x}^{*})-f(A_{p}^{T}(\boldsymbol{\theta}^{*})) +f(A_{p}^{T}(\boldsymbol{\theta}^{*}))\\
&-f(A_{p}^{T}(\boldsymbol{\theta}_t))|\\ 
\leq & \underbrace{\sum_{t=1}^{T}|f(\boldsymbol{x}^{*})-f(A_{p}^{T}(\boldsymbol{\theta}^{*}))|}_{O(\sqrt{T\text{log}T})}\\
&+\sum_{t=1}^{T_{O}}\sum_{t'=1}^{K} |f(A_{p}^{T}(\boldsymbol{\theta}^{*}))-f(A_{p}^{T}(\boldsymbol{\theta}_t))|\\ 
\leq & O(\sqrt{T\text{log}T})+\\
&T_{0} \operatorname*{max}_{t= [1,T_{0}]}(K \operatorname*{max}_{t'= [1,K]}\underbrace{|f(A_{p}^{T}(\boldsymbol{\theta}^{*}))-f(A_{p}^{T}(\boldsymbol{\theta}_t))|}_{L\left\|\boldsymbol{\theta}^{*}-\boldsymbol{\theta}_{t}\right\|})
\end{align*}
Because the $GP$ predictive posterior is smooth w.r.t $\boldsymbol{\theta}$ it makes the acquisition function to be smooth as well w.r.t. $\boldsymbol{\theta}$. So we can assume that $f(A_{p}^{T}(\boldsymbol{\theta}))$ is L-Lipschitz. Hence,
\begin{align*}
R_T\leq & O(\sqrt{T\text{log}T})+\underbrace{T_{0} K}_{T} L\underbrace{\left\|\boldsymbol{\theta}^{*}-\boldsymbol{\theta}_{t}\right\|}_{\epsilon} \\ 
\leq & O(\sqrt{T\text{log}T})+T \epsilon L \\ 
& \text{Taking the limit }\lim_{T \rightarrow \infty} \frac{R_{T}}{T} = \epsilon L
\end{align*}
\end{proof}
Although the regret does not vanish in our case, it can be made arbitrary small by setting $\epsilon$ very small. We must also note that the existing convergence analysis assumes that the best model is being used throughout the BO, ignoring the effect of estimation of the model on the convergence. In fact in some analysis it is shown that running model selection would fail the convergence \cite{bull2011convergence}. In contrast, we provide the convergence guarantee of our whole approach, including the model selection part, thus making it more useful to look at.
\newpage
\section{Method Details}
\label{sec:appx_method_details}

We describe some details of our method.
\subsection{Compute budget allocation}

We describe here a procedure to compute the budget allocation
based on our cost model.
This procedure is more complicated than our simple rule of thumb in
\cref{sec:method}, and tend to produce the same allocation.
For completeness, we include the procedure here for the interested reader.

Given a parameter budget $B$, we find the density of each layer type that
minimize the models' total cost of matrix multiplication.
For example, in Transformers, let $d_a$ and $d_m$ be the density of the
attention and the MLP layers.
Let $s$ be the sequence length and $d$ be the feature size.
The attention layer with density $d_a$ will cost $d_a (n^2 + nd)$, and the fully
connected layers with density $d_m$ will cost $2 d_m nd$.
We then set $d_a$ and $d_m$ to minimize the total cost while maintaining the
parameter budget:
\begin{equation}\label{eq:budget}
  \text{minimize}_{\delta_a, \delta_m} \delta_a (n^2 + nd) + 2 \delta_m n d \quad
  \text{subject to} \quad \text{$\#$ of trainable parameters} \leq B.
\end{equation}
As this is a problem with two variables, we can solve it in closed form.

\subsection{Low-rank in Attention}

In \cref{sec:method}, we describe how to use the sparsity pattern from flat
block butterfly and the low-rank term for weight matrices.
This applies to the linear layer in MLP and the projection steps in the
attention.

We also use the sparse + low-rank structure in the attention step itself.
\citet{scatterbrain} describes a general method to combine sparse and low-rank
attention, where one uses the sparse component to discount the contribution from
the low-rank component to ensure accurate approximation of the attention matrix.

We follow a simpler procedure, which in practice yields similar performance.
We use a restricted version of low-rank of the form a ``global'' sparsity mask
(as shown in \cref{fig:block_sparse_visualization}).
Indeed, a sparse matrix whose sparsity pattern follows the ``global'' pattern is
a sum of two sparse matrices, one containing the ``horizontal'' global components
and one containing the ``vertical'' components.
Let $w$ be the width of each of those components, then each of them has rank at
most $w$.
Therefore, this sparse matrix has rank at most $2w$, and is low-rank (for small $w$).

We also make the global component block-aligned (i.e., set $w$ to be a multiple
of the smallest supported block size such as 32) for hardware efficiency.

\subsection{Comparison to Other Sparsity Patterns for Attention}

In the context of sparse attention, other sparsity patterns such as BigBird and
Longformer also contain a ``global'' component, analogous to our low-rank
component.
Their ``local'' component is contained in the block diagonal part of the flat
block butterfly sparsity pattern.

The main difference that we do not use the random components (e.g., BigBird),
and the diagonal strides from flat block butterfly are not found in BigBird or
Longformer.
Moreover, we apply the same sparsity pattern (+ low-rank) to the linear layers
in the MLP and the projection step in attention as well, allowing our method to
target most neural network layers, not just the attention layer.

\subsection{Sparsity Mask for Rectangular Matrices}

We have described the sparsity masks from flat block butterfly for square
matrices.
For rectangular weight matrices, we simply ``stretch'' the sparsity mask.
The low-rank component applies to both square and rectangular matrices (as shown in~\cref{fig:rec}).
We have found this to work consistently well across tasks.
\begin{figure}[ht]
  \centering
  \includegraphics[width=0.7\linewidth]{figs/rec_butterfly.pdf}
  \caption{\label{fig:rec} Sparsity Mask for Rectangular Matrices.}
\end{figure}



\newpage
\section{Verification benchmark}
\label{sec:benchmark}

% Relative to the tokamak core, the characteristic time and spatial scales
% are compressed.  However, type-I ELMs still have the instability time
% scale associated with the fast crash is an order of magnitude faster
% than the transport-time scale associated with the processes that govern
% the build up of the pedestal structure.   This separation of time scales
% still allows the standard decomposition of studying the linear
% instabilities about an equilibrium that is used in core modes as well.
% 
% Like the core modes, these long-wavelength instabilities are dominated
% by the stiffness in the ideal MHD terms, even for the cases when they
% may be strictly ideal stable.  Multiple numerical methods have been
% developed to handle this stiffness for both linear and nonlinear codes.
% For the nonlinear codes, one numerical advantage is to 
% separate the fields into steady-state (e.g. the reconstructed fields)
% and time-dependent parts.  The pure steady-state terms are analytically
% eliminated resulting in the largest terms in the system to be removed
% from the numerical computations.
% 
% Although typically only MHD-force balance (a
% Grad-Shafranov solution) is strictly enforced for the steady state, in practice
% all fields associated are time independent. This effectively assumes the
% presence of implicit (in the sense that they are calculable but not calculated)
% sources, fluxes and sinks.  With these assumptions, if the code is run on a
% MHD-stable case, the fields do not change.  Alternatively, the initial fields
% are self-consistently modified by the presence of unstable modes. 
% {\bf SEK: OK -- I think this is a better place to put the discussion, in
%   the end this is confusing unless there is an appendix to explain
%   things in detail.  We need to discuss whether we want to add it.  I
% think not as this is really an EHO discussion.}
% 
% There is no technical reason to make this time-scale decomposition - the NIMROD
% code has the capability to compute the extended-MHD evolution of the
% reconstructed fields. However, it is well-known that physical mechanisms
% outside the scope of the extended-MHD model mediate tokamak transport such as
% neoclassical bootstrap current, toroidal viscosity, and poloidal flow damping;
% neutral beam and RF drive; turbulence; and coupling to the scrape-off layer
% (SOL), neutrals, impurities and the material boundary. Including these effects
% requires explicit calculation of the sources, fluxes and sinks. These
% transport-type calculations are possible and are becoming practical (e.g.
% \cite{held15}), but this sort of integrated modeling remains in the future.

We begin with a study of a high resolution, lower-single-null, JT-60U-like
equilibrium (`Meudas-1'), which was originally employed in a benchmark of the
MARG2D and ELITE codes \cite{Aiba07}, including a
close approach to the X-point \cite{Snyder09}.
This extends previous benchmarks~\cite{Burke10} of ELITE and NIMROD as it
includes diverted magnetic topology and a higher edge safety factor
($q_{95}=6.74$, the safety factor at 95\% of the normalized poloidal flux) that
leads to increased resolution requirements. An ideal-MHD limit is achieved in
NIMROD by using flat density and resistivity profiles inside the last closed
flux surface (LCFS) with small resistivity, $S=10^8$ where $S$ is the Lundquist
number  ($S=\tau_R/\tau_A$), $\tau_A$ is the Alfvén time ($\tau_A=R_o/v_A$),
$v_A$ is the Alfvén velocity ($B/\sqrt{m_i n_i \mu_0}$), $\tau_R$ is the
resistive diffusion time ($\tau_R=R_o^2 \mu_0/\eta$), $R_o=2.936 m$ is the
radius of the magnetic axis, $\eta$ is the electrical resistivity, $\mu_0$ is
the permeability of free space, $m_\alpha$ is a species mass (the $\alpha$
subscript denotes ions or electrons in this work), and $n_\alpha$ is a species
density. The deuteron mass ($m_i = 3.34\times 10^{-27} kg$) is used. In order to
reproduce the vacuum response model outside the LCFS that is used by ELITE, a
low density ($0.01$ of the core density) and high resistivity ($10^7$ times the
core resistivity) is prescribed beyond the LCFS (more details on these
approximations are in Ref.~\cite{Burke10}).  

\begin{figure}
  \centering
  \includegraphics[width=8cm]{ELITEComparison}
  \vspace{-4mm}
  \caption{[Color online]
  Growth rates for the `Meudas-1' benchmark. ELITE with $\Gamma=5/3$ and
  $\Gamma=0$ are compared against results from NIMROD with $\Gamma=5/3$).
  Associated NIMROD data available in Ref.~\cite{king16Z}.}
  \label{fig:ELITEComp}
\end{figure}

\begin{figure}
  \centering
  \includegraphics[width=8cm]{idealConv}
  \vspace{-4mm}
  \caption{[Color online]
  Spectral convergence of the NIMROD code for the ideal-like parameters. 
  The maximum polynomial degree (P) of the basis functions composing the 
  spectral elements in increased in each subsequent line plotted.
  Associated NIMROD data available in Ref.~\cite{king16Z}.}
  \label{idealConv}
\end{figure}

The normalized growth rates ($\gamma \tau_A$ where the linearized mode grows as
$\text{exp}[\gamma t]$) vs.~toroidal mode number ($n_\phi$) from NIMROD and ELITE are
plotted in Fig.~\ref{fig:ELITEComp}.  There is good agreement between the
codes except for $n_\phi$=4 where there is a 27\% relative difference. All
other cases have a relative difference of less than 8\% with typical
differences of 5\%. The NIMROD convergence in terms of the maximum polynomial
order of the spectral elements is shown in Fig.~\ref{idealConv}. Convergence is
most challenging at high wavenumbers where the resolution requirements are most
stringent (the poloidal mesh is composed of $72\times512$ spectral elements).
%SEK: Great point, but total troll bait for reviewers
These cases converge from the unstable side where the growth rate decreases with
enhanced resolution. Thus the excellent agreement between NIMROD and ELITE at
high $n_\phi$ in Fig.~\ref{fig:ELITEComp} may be spurious and indicate that
slightly more resolution is required for $n_\phi$>25, however, the 
shown growth rates are likely within 5\% of their converged values.
Studying nearly ideal cases with extended MHD codes such as NIMROD is challenging 
given the vanishingly small dissipation operators, and convergence is achieved
more quickly with the additional non-ideal terms in the extended-MHD equations,
as in the cases in Sec.~\ref{sec:xMHD}.

% Relative to modeling with extended MHD, ideal-MHD convergence is more challenging 
% given the vanishingly small dissipation operators and convergence is 
% achieved more quickly with all other model equations shown in this work.

\begin{figure}
  \includegraphics[width=8cm]{meudas_n11_BR}
  \caption{[Color online]
  Poloidal cross section of the radial magnetic field component of the
  $n_\phi=11$ peeling-ballooning mode from the `Meudas-1' benchmark case. }
  \vspace{-4mm}
  \label{meudas_n11_BR}
\end{figure}

Figure \ref{meudas_n11_BR} shows a poloidal cross section of the magnetic
($B_R$) eigenmode.  The mode develops an `interference-pattern' structure near
the X-point when inboard and outboard finger-like structures overlap. The
finite-element-mesh nodes are superimposed atop the smallest-scale sub-figure.
As established by Fig.~\ref{idealConv}, this simulation is spatially
and temporally converged. The high resolution required to resolve these
high-$q_{95}$, diverted cases
motivated development of memory-scaling improvements in the NIMROD code.

\newpage
\begin{figure}[t]
	\begin{center}
	\scriptsize
		\begin{tabular}{c}
			\includegraphics[width=\linewidth]{figs/candidates.pdf}
		\end{tabular}
	\end{center}
	\caption{Sparsity pattern candidate components:  Local corresponds to local interaction of neighboring elements; Global (low-rank) involves the interaction between all elements and a small subset of elements; Butterfly captures the interaction between elements that are some fixed distance apart; Random is common in the pruning literature.}
	\label{fig:block_sparse_visualization} 
\end{figure}

\section{Exhausted Searching Sparsity Patterns for Efficient Sparse Training}
\label{sec:appx_ntk_algorithm}
We describe here our early exploration of searching among different sparsity patterns that has been proposed in the literature.
We use a metric derived from the NTK, which has emerged as one of the standard metric to predict the training and generalization of the model.
We consistently found the butterfly + low-rank pattern to perform among the best.

In~\cref{sec:challenges}, we describe the challenges of selecting sparsity patterns for every model components using the a metric derived from the NTK, followed by our approaches.
Then in , we describe details of empirical NTK computation, which is an important step in our method implementation. 
Last, in~\cref{sec:property}, we highlight important properties of our method -- it rediscovers several classical sparsity patterns, and the sparse models can inherit the training hyperparamters of the dense models, reducing the need for hyperparameters tuning.

\subsection{Challenges and Approaches}
\label{sec:challenges}

\textbf{Challenge 1:} We seek sparsity patterns for each model components that can closely mimic the training dynamics of the dense counterpart. As mentioned in~\cref{thm:mask_regression}, it is NP-hard to find the optimal sparse matrix approximation. Although NTK provides insights and measurement on the ``right'' sparse model, bruteforcely computing NTK for one-layer models with all sparsity patterns is still infeasible.

\textbf{Approach 1: Sparsity Pattern Candidates.}
To address the above challenge, we design our search space to be a limited set of sparsity pattern candidates, each is either a component visualized in \cref{fig:block_sparse_visualization} or the combination of any two of them.
These components encompass the most common types of sparsity pattern used, and can express
We provide the intuition behind these sparsity components:
\begin{itemize}[leftmargin=*,nosep,nolistsep]
  \item Local: this block-diagonal component in the matrix corresponds to local interaction of neighboring elements. This has appeared in classical PDE discretization \citep{collins1971diagonal}, and has been rediscovered in Longformer and BigBird attention patterns.
  \item Global: this component involves interaction between all elements and a small subset of elements (i.e., ``global'' elements).
  This global pattern is low-rank, and this sparse + low-rank structure is common in data science~\citep{udell2019big}, and rediscovered in Longformer and BigBird patterns as well.
  \item Butterfly: this component corresponds to interaction between elements that are some fixed distance apart.
  The many divide-and-conquer algorithms, such as the classical fast Fourier transform~\citep{cooley1965algorithm}, uses this pattern at each step. Butterfly matrices reflects this divide-and-conquer structure, and hence this sparsity component. The sparse transformer~\citep{child2019generating} also found this pattern helpful for attention on image data.
  \item Random: this component is a generalization of sparsity patterns found in one-shot magnitude, gradient, or momentum based pruning~\citep{lee2018snip}. Note that at network initialization, they are equivalent to random sparsity.
\end{itemize}

\textbf{Challenge 2:} Even with a fixed pool of sparsity patterns for each layer, if the model has many layers, the number of possible layer-pattern assignments is exponentially large.

\textbf{Approach 2:} To further reduce the search space,
we constrain each layer type (attention, MLP) to have the same sparsity pattern.
For example, if there are 10 patterns and 2 layer types, the candidate pool is $10^2 = 100$ combinations.

\textbf{Challenge 3:} Computing the empirical NTK on the whole dataset is expensive in time and space, as it scales quadratically in the dataset size.

\textbf{Approach 3:} We compute the empirical NTK on a randomly chosen subset of the data (i.e., a principal submatrix of the empirical NTK matrix).
In our experiments, we verify that increasing the subset size beyond 1000 does not change the choices picked by the NTK heuristic.
The subsampled empirical NTK can be computed within seconds or minutes.

\subsection{Algorithm Description}
\label{sec:algorithm_description}












\begin{algorithm}[t]
{\small
    \begin{algorithmic}[1]
        \State \textbf{Input: model schema $\Omega$, compute budget $B$, dataset subset $X$, sparsity mask candidate set $C$.}
        \State $K_{dense} \leftarrow$ \textsc{NTK}$(f_\theta, X)$. \Comment{\cref{eq:empirical_ntk}}
        \State output sparsity mask assignment $s_\mathrm{out}$, $d_{min} \leftarrow \inf$
        \For {$M_1, \dots, M_{|\Omega|} \in C^{|\Omega|}$} \Comment{Enumerate all sparsity mask candidate combinations}
            \State Let $s$ be the sparsity mask assignment $(t_i, r_i, m_i, n_i) \to M_i$.
            \If {$\text{TotalCompute}(s) < B$} \Comment{\cref{eq:budget}, Check if masks satisfy budget constraint}
                \State Let $M_s$ be the flattened sparse masks
                \State $K_{sparse} \leftarrow \textsc{NTK}(f_{\theta \circ M_s}, X)$
                \State $d_s \leftarrow \textsc{Distance}(K_{dense}, K_{sparse})$ \Comment{\cref{eq:empirical_ntk}}
                \If{$d_{min}>d_s$}
                    \State $d_{min} \leftarrow d_s$, $s_\mathrm{out} \leftarrow s$
                \EndIf
            \EndIf
        \EndFor
        \State\Return $s_\mathrm{out}$ \Comment{Return sparsity mask assignment}
    \end{algorithmic}
    }
    \caption{Model Sparsification}\label{algo:pre}
    \end{algorithm}



Our method targets GEMM-based neural networks, which are networks whose computation is dominated by general matrix multiplies (GEMM), such as Transformer and MLP-Mixer.
As a result, we can view the network as a series of matrix multiplies.
We first define:
\begin{itemize}[leftmargin=*,nosep,nolistsep]
  \item Model schema: a list of layer types $t$ (e.g., attention, linear layers in MLP), number of layers $r$ of that type, and dimension of the matrix multiplies $m \times n$.
  We denote it as $\Omega = \{(t_1, r_1, m_1, n_1), \dots, (t_{|\Omega|}, r_{|\Omega|}, m_{|\Omega|}, n_{|\Omega|})\}$.
  \item A \emph{mask} $M$ of dimension $m \times n$ is a binary matrix $\{0, 1\}^{m \times n}$.
  The compute of a mask is the total number of ones in the matrix: $\mathrm{compute}(M) = \sum_{i, j} M_{ij}$.
  \item A \emph{sparsity pattern} $P_{m \times n}$ for matrix dimension $m \times n$ is a set of masks $\{M_1, ..., M_{|P|}\}$, each of dimension $m \times n$.
  \item A \emph{sparsity mask assignment} is a mapping from a model schema $\Omega$ to masks $M$ belonging to some sparsity pattern $P$: $s \colon (t, r, m, n) \to M$.
  \item Given a set of sparsity patterns $P_1, \dots, P_k$, the set of sparsity mask candidate $C$ is the union of sparsity masks in each of $P_i$: $C = \cup P_i$
  \item A sparsity pattern assignment $s$ satisfies the compute budget $B$ if:
\begin{equation}
\label{eq:budget}
  \mathrm{TotalCompute}(s) := \sum_{\text{layer type } l} \mathrm{compute}(s(t, r, m, n)) \le B.
\end{equation}
  \item Let $\theta$ be the flattened vector containing the model parameters, and let $M_s$ be the flattened vector containing the sparsity mask by the sparsity mask assignment $s$.
  Let $f_\theta(x)$ be the output of the dense network with parameter $\theta$ and input $x$.
  Then the output of the sparse network is $f_{\theta \circ M_s}(x)$.
  \item The empirical NTK of a network $f_\theta$ on a data subset $X = \{x_1, \dots, x_{|X|}\}$ is a matrix of size $|X| \times |X|$:
\begin{equation}
  \label{eq:empirical_ntk}
  \mathrm{NTK}(f_\theta, X)_{i, j} = \left \langle \frac{\partial f_\theta(x_i)}{\partial \theta}, \frac{\partial f_\theta(x_j)}{\partial \theta} \right \rangle.
\end{equation}
\end{itemize}

The formal algorithm to assign the sparsity mask to each layer type is described in \cref{algo:pre}.
The main idea is that, as the set of sparsity mask candidate is finite, we can enumerate all possible sparsity mask assignment satisfying the budget and pick the one with the smallest NTK distance to the dense NTK.
In practice, we can use strategies to avoid explicitly enumerating all possible sparsity mask, e.g. for each sparsity pattern, we can choose the largest sparse mask that fits under the budget.









\subsection{Method Properties: Rediscovering Classical Sparsity Patterns, No Additional Hyperparameter Tuning}
\label{sec:property}
When applied to the Transformer architecture, among the sparsity components described in \cref{sec:challenges}, the NTK-guided heuristic consistently picks the local and global components for \emph{both} the attention and MLP layers.
Moreover, the butterfly component is also consistently picked for image data, reflecting the 2D inductive bias in this component\footnote{Convolution (commonly used in image data) can be written in terms of the fast Fourier transform, which has this same sparse pattern at each step of the algorithm}.
While some of these patterns have been proposed for sparse attention, it is surprising that they are also picked for the MLP layers.
The most popular type of sparsity pattern in MLP layers is top-k (in magnitude or gradient, which at initialization is equivalent to random sparsity).
We have proved that lower NTK difference results in better generalization bound for the sparse model. As expected, we observe that this allows the sparse model to use the same hyperparamters (optimizer, learning rate, scheduler) as the dense model (\cref{sec:experiments}).







\newpage
We provide the details of our experiments on two well-known instruction tuning datasets: NIV2~\cite{Wang2022SuperNaturalInstructionsGV} and Self-Instruct dataset~\cite{wang2022self}. 
\paragraph{NIV2 - Active Instruction Tuning Details}
We utilize the NIV2 English tasks split, comprising 756 training tasks and 119 testing tasks, including classification and generative tasks.
We employ five random seeds without selection in our active instruction tuning experiment. Each seed involves randomly sampling 68 tasks as initial training tasks and 68 tasks as validation tasks. The remaining 620 training tasks form the remaining task pool. 
In each active learning iteration, we maintain a fixed classification and generative task ratio and select 24 classification tasks and 44 generative tasks using different task selection strategies. This fixed ratio allows a more meaningful comparison of our results as we evaluate overall, classification, and generative task scores separately. After the new tasks are sampled, we add them to the previously selected training tasks and form a new training task set. We further train a new instruction tuning model with the updated training task set.
\paragraph{Self-Instruct - Active Instruction Tuning Details}
We utilize the 52K self-instruct dataset as the task pool. For the active instruction tuning experiment, we will randomly sample 500 tasks as the initial training set and further compare model performance at $[1000, 2000, 4000, 8000, 16000]$ training tasks. For task selection, we will first divide all tasks into 13 chunks by output sequence length $[[1,10], [11,20], ..., [121, 130]]$, and then apply the task selection methods on each chunk of tasks, following the ratio of the number of tasks in all chunks. We conduct this extra step to normalize the output sequence length of the selected task for each task selection method. This ensures there is no imbalance in output sequence length during task selection.
\paragraph{Training Details}
For experiments on NIV2 dataset~\cite{Wang2022SuperNaturalInstructionsGV}, we follow the TK-instruct setting, the SOTA model on the NIV2 dataset to train the T5-770M model~\cite{raffel2020exploring} with learning rate 2e-5, batch size 128 and 200 instances per task for eight epochs. We evaluate the model's zero-shot performance on the validation set at each epoch and select the model checkpoint with the best validation score. For evaluation, we follow ~\cite{kung2023models} setting to report the Rouge-L score of \textit{Overall}, \textit{Classification}, and \textit{Generative} tasks on both validation and testing sets. For experiments on Self-Instruct dataset~\cite{wang2022self}, We follow Alpaca's settings to train the LLaMA-7B model with learning rate 2e-5, batch size 128 for four epochs.

\paragraph{Computing Resources}
For the experiment on NIV2 dataset~\cite{Wang2022SuperNaturalInstructionsGV}, we conduct our experiments using 4 to 8 Nvidia 48GB A6000 GPUs. For each uncertainty method, it takes around 1200 GPU hours, a total of 5000 GPU hours(for a single GPU), to run all experiments for \autoref{fig:niv2-results}. For the experiment on Self-Instruct dataset~\cite{wang2022self}, we run with 2 Nvidia 80GB A100 GPUs. Each uncertainty method takes around 40 GPU hours, which sums to 160 GPU hours for all experiments in \autoref{fig:alpaca-results}.

\newpage



\section{Extended Related Work}
\label{app:related}
In this section, we extend the related works referenced in the main paper and discuss them in detail.

\subsection{Neural Pruning} 
Our work is loosely related to neural network pruning. By iteratively eliminating neurons and connections, pruning has seen great success in compressing complex models.\citet{han2015deep,han2015learning} put forth two naive but effective algorithms to compress models up to 49x and maintain comparable accuracy. \citet{li2016pruning} employ filter pruning to reduce the cost of running convolution models up to 38 $\%$, \citet{NIPS2017_a51fb975} prunes the network at runtime, hence retaining the flexibility of the full model. \citet{dong2017learning} prunes the network locally in a layer by layer manner.  \citet{sanh2020movement} prunes with deterministic first-order information, which is more adaptive to pretrained model weights. \citet{lagunas2021block} prunes transformers models with block sparsity pattern during fine-tuning, which leads to real hardware speed up while maintaining the accuracy. \citet{zhu2017prune} finds large pruned sparse network consistently outperform the small dense networks with the same compute and memory footprints. Although both our and all the pruning methods are aiming to produce sparse models, we differ in our emphasis on the overall efficiency, whereas pruning mostly focuses on inference efficiency and disregards the cost in finding the smaller model.\\

\subsection{Lottery Ticket Hypothesis} 
Models proposed in our work can be roughly seen as a class of manually constructed lottery tickets. Lottery tickets \citet{frankle2018lottery} are a set of small sub-networks derived from a larger dense network, which outperforms their parent networks in convergence speed and potentially in generalization. A huge number of studies are carried out to analyze these tickets both empirically and theoretically: \citet{morcos2019one} proposed to use one generalized lottery tickets for all vision benchmarks and got comparable results with the specialized lottery tickets; \citet{frankle2019stabilizing} improves the stability of the lottery tickets by iterative pruning; \citet{frankle2020linear} found that subnetworks reach full accuracy only if they are stable against SGD noise during training; \citet{orseau2020logarithmic} provides a logarithmic upper bound for the number of parameters it takes for the optimal sub-networks to exist; \citet{pensia2020optimal} suggests a way to construct the lottery ticket by solving the subset sum problem and it's a proof by construction for the strong lottery ticket hypothesis. Furthermore, follow-up works \citep{liu2020finding, wang2020picking, tanaka2020pruning} show that we can find tickets without any training labels.\\

\subsection{Neural Tangent Kernel} 

Our work rely heavily on neural tangent kernel in theoretical analysis. Neural Tangent Kernel \citet{jacot2018neural} is first proposed to analyse the training dynamic of infinitely wide and deep networks. The kernel is deterministic with respect to the initialization as the width and depth go to infinity, which provide an unique mathematical to analyze deep overparameterized networks. Couples of theoretical works are built based upon this: \cite{lee2019wide} extend on the previous idea and prove that finite learning rate is enough for the model to follow NTK dynamic. \citet{arora2019exact} points out that there is still a gap between NTK and the real finite NNs. \citet{cao2020generalization} sheds light on  the good generalization behavior of overparameterized deep neural networks. \citet{arora2019fine} is the first one to show generalization bound independent of the network size. Later, some works reveal the training dynamic of models of finite width, pointing out the importance of width in training: \citet{hayou2019training} analyzes stochastic gradient from the stochastic differential equations' point of view; Based on these results, we formulate and derive our theorems on sparse network training.\\

\subsection{Overparameterized Models} 
Our work mainly targets overparameterized models. In \citet{nakkiran2019deep},  the double descendent phenomenon was observed. Not long after that, \cite{d2020triple} discover the triple descendent phenomenon. It's conjectured in both works that the generalization error improves as the parameter count grows. On top of that, \citet{arora2018optimization} speculates that overparameterization helps model optimization,
and without "enough" width, training can be stuck at local optimum. Given these intuitions, it's not surprising that the practitioning community is racing to break the record of the largest parameter counts: The two large language models, GPT-2 and GPT-3 \citep{radford2019language, brown2020language}, are pushing the boundary on text generation and understanding; Their amazing zero-shot ability earn them the title of foundation models \citep{bommasani2021opportunities}. On the computer vision side, \citet{dosovitskiy2020image, tolstikhin2021mlp, zhai2021scaling} push the top-1 accuracy on various vision benchmarks to new highs after scaling up to 50 times the parameters; \citet{naumov2019deep} shows impressive results on recommendation with a 21 billion large embedding; \citet{jumper2021highly} from DeepMind solve a 50 year old grand challenge in protein research with a 46-layer Evoformer. In our work, we show that there is a more efficient way to scale up model training through sparsification and double descent only implies the behavior of the dense networks.



\end{document}
