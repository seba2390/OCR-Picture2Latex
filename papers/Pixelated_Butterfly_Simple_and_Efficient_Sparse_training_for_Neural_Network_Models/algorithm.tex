\begin{figure}[t]
	\vspace{-0.5cm}
	\begin{center}
	\scriptsize
		\begin{tabular}{c}
			\includegraphics[width=0.9\linewidth]{figs/candidates.pdf}
		\end{tabular}
	\end{center}
	\vspace{-0.5cm}
	\caption{Sparsity pattern candidate components:  Local corresponds to local interaction of neighboring elements; Global (low-rank) involves the interaction between all elements and a small subset of elements; Butterfly captures the interaction between elements that are some fixed distance apart; Random is common in the pruning literature.}
	\label{fig:block_sparse_visualization} 
\end{figure}

\section{Pixelated Butterfly: Efficient Sparse Training}
\label{sec:algorithm}
We present Pixelated Butterfly, an efficient sparse model with a simple and fixed sparsity pattern based on butterfly matrices.
In~\cref{sec:challenges}, we describe the challenges of searching for sparsity patterns in the set of butterfly matrices, followed by our approaches.
Then in~\cref{sec:algorithm}, we demonstrate Pixelfly workflow.

\subsection{Challenges and Approaches}
\label{sec:challenges}

Outline:

\textbf{Challenge 1:} butterfly is a product of many factors, making it slow.
\textbf{Approach 1:} Use the 1st order approximation. We empirically validate that this works as well as the original butterfly.

\textbf{Challenge 2:} butterfly is not hardware friendly.
\textbf{Approach 2:} Use block butterfly.

\textbf{Challenge 3:} the first-order approximation loses some expressivity.
\textbf{Approach 3:} We add a low-rank matrix~\citep{udell2019big} to the first-order block butterfly.
Low-rank exploits efficient dense matrix multiply.


































