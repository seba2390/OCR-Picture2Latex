\section{Introduction}

Monolithic infrared (IR) solid-state light sources grown on silicon have attracted significant interest owing to their relevance to scalable photonic integrated circuits (PICs) and compatibility with complementary metal-oxide-semiconductor (CMOS) technologies \cite{powell2022integrated,kim2023short,hu2017silicon}. Among a plethora of applications, this monolithic integration is increasingly crucial to implementing compact and cost-effective IR sensing and imaging technologies. The former requires emitters operating at longer wavelengths in the IR range to overlap with the molecular fingerprint. For instance, emitters in the  $2.3$ $-$ $2.7$ $\mu$m range are useful for sensing both temperature and gas molecules such as  CO, CO$_{2}$, NH$_{3}$, and N$_{2}$O \cite{zia2019high,mathews2021high,stritzke2015tdlas,ji2022mid12,lamoureux2021situ,bader2020progress,ryczko2021interband,olafsen2020optically,dolores2017waveguide,aziz2017multispectral,shterengas2016cascade,li2018dual}.   
These applications call, however,  for broadband emitters such as light emitting diodes (LEDs),  particularly for spectroscopic detection of these critical gases generated in a chemical plant or for their atmospheric monitoring \cite{mikhailova2007optoelectronic,stoyanov2012middle}.
The current emitters serving this wavelength range consist predominantly of III-V compound semiconductors including GaSb- and InP-based device structures \cite{dai2022growth,alexandrov2002portable,sprengel2015continuous,meyer2020interband}. For instance, GaSb LEDs provide wavelength emission peaks between $2.37$ $\mu$m and $2.70$ $\mu$m \cite{danilova2005light,tournie2019mid}. Although GaSb-based lasers and super-luminescent LEDs possess higher tunability of the emitted wavelength \cite{coldren2000monolithic,lee2007widely,sprengel2015continuous,karioja2017multi}, their narrow band emission remains less favorable for spectroscopic gas detection. Notwithstanding the progress in developing III-V LEDs, these materials remain costly and their direct growth on silicon is typically associated with a degradation of device performance \cite{Moutanabbir2010}, thus hindering their integration in large-scale applications. 

 As an alternative, group IV GeSn semiconductors have been explored to circumvent these challenges. These semiconductors are epitaxially grown on silicon and thus can benefit from established scalable manufacturing. Moreover, the Sn content and strain can be controlled to tune the bandgap over the entire range of the IR wavelengths \cite{atalla2023extended,Moutanabbir2021,VondenDriesch2015,Buca2022,wu2023ge}. Up-to-date, the very few reported GeSn LEDs are shown to function in the alternating current (AC) operation mode\cite{Huang2019,Peng2020,Stange2017,bertrand2019mid,Gallagher2015}. While AC-driven LEDs provide quasi-continuous-wave or pulse emission and mitigate the device heating by reducing the thermal power dissipation, direct current- (DC-) driven IR LEDs provide a continuous-wave emission, which is ideal for imaging and portable infrared systems that utilize DC batteries  \cite{thirumalai2018light,lasance2014thermal}. Herein, a GeSn vertical PIN LED with an emission peak in the range of $2.45$ $-$ $2.58$ $\mu$m is demonstrated. The achieved GeSn LEDs are shown to operate under DC bias at a relatively low injection current. Before delving into the device performance, the material growth and characterization are first presented to elucidate the basic properties of the PIN heterostructure including lattice strain and Sn content. Additionally, an eight-band $k.p$ model is implemented to estimate the band alignment of the grown GeSn stack to evaluate the electronic structure of the obtained double heterostructure. Afterward, the electrical and optical measurements are presented to investigate the LED operation under DC- and AC-driven operation modes, the emission wavelength, the emission power profile, and the bandwidth. 



\bigskip

