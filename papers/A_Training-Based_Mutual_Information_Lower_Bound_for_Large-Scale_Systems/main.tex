%\documentclass[conference]{IEEEtran}
\documentclass[12pt, draftclsnofoot,journal,onecolumn]{IEEEtran}
%\documentclass[journal,onecolumn]{IEEEtran}
%,onecolumn
%,onecolumn
\input src/header.tex

\begin{document}

\title{A Training-Based Mutual Information Lower Bound for Large-Scale Systems}
%\title{On the Computation of Large-Scale Tensor Entropy}
%\author{\IEEEauthorblockN{Kang Gao, Bertrand Hochwald}\\
%\IEEEauthorblockA{Department of Electrical Engineering, %University of Notre Dame, Notre Dame, IN, 46556\\
%Email: \texttt{\{kgao,bhochwald\}@nd.edu}}}
\author{Xiangbo Meng,~\IEEEmembership{Student Member,~IEEE,} Kang~Gao,~\IEEEmembership{Student Member,~IEEE,} and~Bertrand~M.~Hochwald,~\IEEEmembership{Fellow,~IEEE}%
\thanks{This work was generously supported by NSF Grant \#1731056 and Futurewei Technologies, Inc.}
\thanks{Xiangbo Meng, Kang Gao, and Bertrand M. Hochwald are with the Department of Electrical Engineering, University of Notre Dame, Notre Dame, IN, 46556 USA (email:xmeng@nd.edu; kgao@nd.edu; bhochwald@nd.edu).}}
\maketitle
\thispagestyle{plain}
\pagestyle{plain}
%\thispagestyle{empty} %--to make title page number less
%\pagestyle{empty}    % -- to make other pages number less.
\begin{abstract}
We provide a mutual information lower bound that can be used to analyze the effect of training in models with unknown parameters. For large-scale systems, we show that this bound can be calculated using the difference between two derivatives of a conditional entropy function.  The bound does not require explicit estimation of the unknown parameters.  We provide a step-by-step process for computing the bound, and provide an example application.  A comparison with known classical mutual information bounds is provided.
%We show that the size of the discontinuity is of particular interest in the analysis of the mutual information of systems that use training to learn unknown the parameters.  
\end{abstract}
\begin{IEEEkeywords}
%phase transition,
information rates, training, entropy, large-scale systems
\end{IEEEkeywords}
\IEEEpeerreviewmaketitle

\section{Introduction}

\comm{
Consider a linear system and get a bound for a system with training. (still require the reference of Taiwanese paper for Entropy computation, but focus on linear systems)
}



%It is shown in \cite{telatar1999capacity} that very high data rate can be achieved for reliable communication in linear MIMO systems, when the channel state information (CSI) is known. Since then, much work has been done to analyze the achievable rate of systems when the CSI is unknown, and two strategies are considered depending on whether training is used to learn the channel. 

%Without any training, the closed form expression can only be obtained in some special cases, including the channel capacity in high SNR limit \cite{zheng2002communication} or a channel capacity lower bound by considering a specific input distribution (unitary space-time modulation) with a small number of antennas at both the transmitter and receiver \cite{hochwald2000unitary}. Therefore, such analysis has very limited applications, and cannot be generalized to nonlinear systems in general.

Many systems have unknown parameters that are estimated during a training-phase with the help of known prescribed training signals.  This phase is followed by a data phase, where knowledge of the estimated parameters is used to process the data.  It is generally assumed that the parameters are constant during these two phases, the total duration of which is called the coherence time.  It is often of great interest to optimize the training time for a given coherence time, since time in the training phase, while useful for parameter estimation, generally takes away from time in the data phase.

In a communication system, the parameters of interest often include the channel, which is typically unknown and learned at the receiver with the help of pilot signals sent by the transmitter.   For example, \cite{hassibi2003much} analyzes a multi-antenna model and a capacity lower-bound is obtained by using the minimum mean-square estimate (MMSE) of the channel, and the residual channel error is treated as Gaussian noise.  This lower bound is maximized over various parameters, including the fraction of the coherence time that should be dedicated to training.  A similar optimization is considered in \cite{muharar2020optimal}, where the power allocation and training duration are chosen to achieve the maximum sum-rate in a multiuser system.  Such ``one-shot learning", where the parameters are estimated only during the training phase, can be augmented by further refinement during the data phase \cite{takeuchi2012large,takeuchi2013achievable}. However, this refinement can suffer from error propagation \cite{miridakis2016joint}, and we do not consider this herein. 
%In \cite{sheng2017optimal}, when the total training power is constrained in a large-scale multi-antenna orthogonal frequency division multiplexing (OFDM) system,  it is shown that setting the training time equal to the number of unknowns is sufficient to minimize the mean squared error (MSE) of the estimator for the unknown parameters.
%Nonlinear multi-antenna models are considered in \cite{li2016much,li2017channel}, where the receivers include one-bit analog-to-digital converters (ADC's), and with one-shot learning the optimum training time is derived by employing the Bussgang decomposition \cite{bussgang1952crosscorrelation} and invoking a worst-case noise analysis.

Many of these previous efforts to analyze training assume that the unknown parameters appear linearly in the system model \cite{hassibi2003much,muharar2020optimal,takeuchi2012large,takeuchi2013achievable,sheng2017optimal}, or appear in a linearized version of the model \cite{li2016much,li2017channel}, often by employing the Bussgang decomposition \cite{bussgang1952crosscorrelation}.  We develop a framework to analyze one-shot training that does not require the parameters to appear linearly in the model, nor does it require additive Gaussian noise; rather, it requires the system to be time-invariant and memoryless, 
%and the inputs during the data phase to be \iid (Assumption A1 shown in  \eqref{eq:fixed_channel_in_data}--\eqref{eq:iid_data_input_distribution}), 
and a certain entropy to be computed in the large-scale system limit.
Herein, large-scale refers to long block lengths (time durations) or large dimensional inputs and outputs, or both.
%(Assumption A2 shown in \eqref{eq:training_phase_continue}--\eqref{eq:continue_in_data}).  
The fact that the large-scale system entropy can sometimes be computed even when the small-scale system entropy cannot is exploited for our training analysis.  

\subsection{Problem setup and statement}

Consider a system model that has input and output processes as $\txA=(\txv_1,\txv_2,\ldots)$ and $\rxA=(\rxv_{1},\rxv_{2},\ldots)$, which comprise vectors $\txv_t$ and $\rxv_{t}$ whose dimensions are $\tn$ and $\rn$ respectively. The input and output are connected through a conditional distribution parameterized by $\chv_{\Tb}$, whose value is unknown.  We assume that $\chv_{\Tb}$ is constant during a coherence time block $\Tb$, and then changes independently in the next block (same length $\Tb$), and so on. 
The system is supplied with known inputs during a ``training phase" to learn the parameters, after which the system is used during its ``data phase".   The unknown parameters are assumed to have a known distribution, and the number of unknown parameters is allowed to be a function of $\Tb$.  {\em Problem statement:} We wish to determine the optimum amount of training.

To analyze the effects of training, a lower bound on the mutual information between the input and output is often used
\begin{align}
    \frac{1}{\Tb}\MuI(\txm_\Tb; \rxm_\Tb) \ge \frac{\Tb - \Tt}{\Tb}\MuI(\txv_{\Tt+1};\rxv_{\Tt+1}|\txm_{\Tt},\rxm_{\Tt}),
    \numberthis
    \label{eq:general_lower_bound}
\end{align}
where $\txv_{t}$ and $\rxv_{t}$ are the $t$th vector input and output of the system,  $\txm_{t}=[\txv_{1},\cdots,\txv_{t}]$, $\rxm_{t}=[\rxv_{1},\cdots,\rxv_{t}]$, and $\Tt$ is the number of training symbols in one coherence block.  We assume $0<\tau<1$ is the fraction of the blocklength devoted to training, and $\Tt$ is integer for convenience.  (We choose this in favor of using $\ceil{\Tt}$ throughout.)  The optimal training fraction, in the sense of maximizing this lower bound, is then
\begin{align}
    \tauopt = \argmax_{\tau}\;(1-\tau)\MuI(\txv_{\Tt+1};\rxv_{\Tt+1}|\txm_{\Tt},\rxm_{\Tt}).
    \numberthis 
    \label{eq:optimal_training_time_finite}
\end{align}
Such an analysis appears, for example, in \cite{hassibi2003much,muharar2020optimal,li2016much,li2017channel}, but the right-hand side of \eqref{eq:optimal_training_time_finite} can be difficult to compute and is itself often approximated or lower bounded.  For example, in \cite{hassibi2003much}, a wireless communication system with Rayleigh block-fading channel and additive Gaussian noise is considered, and the mutual information in \eqref{eq:optimal_training_time_finite} is lower bounded by treating the estimation error of the MMSE estimate of the channel as independent additive Gaussian noise.  However, this form of analysis is often intractable when the parameters appear nonlinearly, or the additive noise is non-Gaussian, since explicit estimates of the unknown parameters are unavailable.


% \begin{align}
%     \left(1-\tau\right)\MuI(\txv_{\Tt+1};\rxv_{\Tt+1}|\txm_{\Tt},\rxm_{\Tt}),
%     \label{eq:general_lower_bound}
% \end{align}
% where $\tau=\frac{\Tt}{\Tb}$ is the fraction of a blocklength used for training, $\txv_{t}$ and $\rxv_{t}$ are the $t$th input and output of the system, $\txm_{t}=[\txv_{1},\cdots,\txv_{t}]$ and $\rxm_{t}=[\rxv_{1},\cdots,\rxv_{t}]$.  

By considering a large-scale limit of the conditional mutual information in \eqref{eq:optimal_training_time_finite}, we provide a method to revisit this computation.  Let $\Tb\to\infty$, and define the ratios
\begin{align}
    \ratio=\frac{\rn}{\tn},\quad \beta=\frac{\Tb}{\tn}.
    \label{eq:ratios}
\end{align}
It is possible, although not required, that $\tn$ and $\rn$ also grow to infinity with $\Tb$, so that $\beta$ is finite.  The large-scale limit of the conditional mutual information $\MuI(\txv_{\Tt+1};\rxv_{\Tt+1}|\txm_{\Tt},\rxm_{\Tt})$ in \eqref{eq:optimal_training_time_finite} is
\begin{align}
    \cIInn{\txA}{\rxA}{}=\lim_{\Tlim\to\infty}\frac{1}{\rn}\MuI(\txv_{\Tt+1};\rxv_{\Tt+1}|\txm_{\Tt},\rxm_{\Tt}).
    \label{eq:MuI_def_high}
\end{align}
The normalization by $1/\rn$ is needed to keep this quantity finite if $\rn\to\infty$, and this limit (assuming that it exists) typically depends on $\ratio$, $\beta$, and $\tau$.  The optimal training time in \eqref{eq:optimal_training_time_finite} then becomes 
\begin{align}
    \tauopt=\argmax_{\tau}\ (1-\tau) \cII{\txA}{\rxA}{},
    \label{eq:opt_training_tau}
\end{align}
and the corresponding optimal rate becomes
\begin{equation*}
    \Ropt = (1-\tauopt)\cII{\txA}{\rxA}{}\bigr|_{\tau=\tauopt}.
\end{equation*}
%A lower-bound result is stated formally in Theorem \ref{thm:MI_equal_gap&computation_of_MuI_from_derivative}.
The value of this analysis depends on our ability to compute $\cII{\txA}{\rxA}{}$, and we show that this quantity can be computed as the derivative of a certain entropy.

\section{Main Results}

\subsection{Assumptions and definitions of useful quantities}
Before we introduce the main results, we first make some assumptions and definitions. The bound in \eqref{eq:general_lower_bound} is fully determined by the distribution of the triple $(\txm_{\Tb},\rxm_{\Tb},\chv_{\Tb})$,  and we make the following assumption:
\begin{align*}
    \text{A1:}\;\;& p(\rxm_{\Tb}|\txm_{\Tb};\chv_{{\Tb}}) =\prod_{t=1}^{\Tb}p(\rxv_t|\txv_t;\chv_{{\Tb}}),
    \numberthis
    \label{eq:fixed_channel_in_data} \\
   & p(\txm_\Tb)=p(\txm_{\Tt})\prod_{t=\Tt+1}^\Tb p(\txv_t),
   \numberthis
    \label{eq:iid_data_input_distribution}
\end{align*}
where $p(\rxv_t|\txv_t;\chv_{\Tb})$ is a fixed conditional distribution for all $t=1,2,\ldots,\Tb$ and $p(\txv_t)$ is a fixed distribution for all $t=\Tt+1,\Tt+2,\ldots,\Tb$.


Equation \eqref{eq:fixed_channel_in_data} says that the system is memoryless and time-invariant (given the input and parameters) and \eqref{eq:iid_data_input_distribution} says that the input $\txv_t$ is \iid and independent of $\txm_\Tt$ for all $t>\Tt$.  
%The distributions of $\txv_t$ during training and afterward can therefore differ.  
We use the common convention of writing $p(\txm_{\Tt})$ and $p(\txv_t)$ when we mean $p_{\txm_{\Tt}}(\cdot)$ and $p_{\txv_t}(\cdot)$, even though these functions can differ. 
Under A1, the distributions of $(\txm_{\Tb},\rxm_{\Tb},\chv_{{\Tb}})$ are described by the set of known distributions
\begin{equation}
\cP(\Tb,\tau)=\{p(\rxv|\txv;\chv_{{\Tb}})),p(\chv_{{\Tb}})),p(\txm_\Tt),p(\txv_{\Tt+1})\}.
\label{eq:P_set_high}
\end{equation}
These distributions are used to calculate all of the entropies and mutual informations throughout.  The entropies and mutual informations are ``ergodic" in the sense that they are averaged over independent realizations of $\chv_{{\Tb}}$.

Define:
\begin{align*}
\EntInnC{\txA}{\rxA}{\dx}{\dy} =\lim_{\Tlim\to\infty}\frac{1}{\rn}\Ent(\rxv_{{\dy\Tt}+1}|\txm_{{\dx\Tt}},\rxm_{{\dy\Tt}}),
    \numberthis
    \label{eq:EntInn_cond_def_high}
\end{align*}
\begin{align*}
\EntInnC{\txA}{\rxA}{\dx^+}{\dy}
=\lim_{\Tlim\to\infty}&\frac{1}{\rn}\Ent(\rxv_{{\dy\Tt}+1}|\txm_{{\dx\Tt}+1},\rxm_{{\dy\Tt}}),
    \numberthis
    \label{eq:EntInn_cond_def_one_extra_high}
\end{align*}
with $\dx,\dy\in[0,\frac{1}{\tau})$, again assuming these limits exist. Notice that here we treat $\dx\Tt, \dy\Tt$ again as integers to avoid excessive use of the ceiling or floor notation.  We drop the subscripts $\dy$ and $\dx$ in $\EntInnC{\txA}{\rxA}{\dx}{\dy}$ and $\EntInnC{\txA}{\rxA}{\dx^+}{\dy}$ when $\dy=1$ or $\dx=1$.  For example, $\EntInnC{\txA}{\rxA}{}{}$ = $\EntInnC{\txA}{\rxA}{\dx}{\dy}|_{\dx=1, \dy =1}$ and $\EntInnC{\txA}{\rxA}{^+}{}$ = $\EntInnC{\txA}{\rxA}{\dx^+}{\dy}|_{\dx=1, \dy =1}$. 
With \eqref{eq:EntInn_cond_def_high} and \eqref{eq:EntInn_cond_def_one_extra_high}, we further make the following assumptions:
\begin{align*}
\text{A2:}&\qquad\quad \EntInnC{\txA}{\rxA}{^+}{}  = \lim_{\dy\downto 1} \EntInnC{\txA}{\rxA}{\dy^+}{\dy},
    \numberthis
    \label{eq:training_phase_continue}\\
&\qquad\quad \EntInnC{\txA}{\rxA}{}{} =\lim_{\dy\downto 1} \EntInnC{\txA}{\rxA}{}{\dy}.
    \numberthis
    \label{eq:continue_in_data}
\end{align*}



%We drop the subscripts $\dy$ and $\dx$ in $\cII{\txA_\dy}{\rxA_\dy}{}$, $\cIRA{\txA_\dy}{\rxA_\dy}{}$, $\EntInnC{\txA}{\rxA}{\dx}{\dy}$, $\EntInnC{\txA}{\rxA}{\dx^+}{\dy}$ and $\EntR(\rxA_\dy|\txA_\dx)$ when $\dy=1$ or $\dx=1$.  For example, $\cII{\txA}{\rxA}{}=\cII{\txA_\dx}{\rxA_\dy}{}|_{\dx=1, \dy =1}$, $ \EntR(\rxA|\txA)= \EntR(\rxA_\dy|\txA_\dx)|_{\dy=1,\dx=1}$. 

Assumptions A1--A2 in \eqref{eq:fixed_channel_in_data}, \eqref{eq:iid_data_input_distribution},  \eqref{eq:training_phase_continue}, and  \eqref{eq:continue_in_data},  are important for the main result (Theorem \ref{thm:MI_equal_gap&computation_of_MuI_from_derivative}). 
A1 is often met in practice for a memoryless and time-invariant system with \iid input in the data phase, independent of the input and output during training. However, we do not have a complete characterization of the processes $\txA$ and $\rxA$ that meet Assumptions A2.  Nevertheless, A2 may be verified on a case-by-case basis by examining expressions of  $\EntInnC{\txA}{\rxA}{\dy^+}{\dy}$ and $\EntInnC{\txA}{\rxA}{}{\dy}$ with $\dy\geq 1$ using Corollary \ref{cor:A1_derivative_relation}(c) in Appendix \ref{app:proof_MI_equal_gap&computation_of_MuI_from_derivative}; see the Example.



% \begin{lem}
% \label{lem:bound_for_A2}
% If Assumption A1 is met and
% \begin{align}
%     \lim_{\dy\downto 0}\EntInnC{\txA}{\rxA}{}{\dy}=\lim_{\dx\upto 0}\EntInnC{\txA}{\rxA}{\dx}{},
%     \label{eq:two_bound_met}
% \end{align}
% then Assumption A2 is met.
% \end{lem}
% \begin{proof}
% Under Assumption A1, for $t\geq\Tt+1$, we have
% \begin{align*}
%   \Ent(\rxv_{t+1}|\txm_{\Tt},\rxm_{t}) \leq\Ent(\rxv_t|\txm_{\Tt},\rxm_{t-1})
%     \numberthis
%     \label{eq:monotonic_condition_2}
% \end{align*}
% which is also shown in \eqref{eq:monotonic_A1}.
% Therefore,
% \begin{equation*}
%   \lim_{\dy\downto 0}\EntInnC{\txA}{\rxA}{}{\dy}\leq \EntInnC{\txA}{\rxA}{}{}.
% \end{equation*}
% Conditioning to reduce entropy again yields
% \begin{align*}
%   \Ent(\rxv_{\Tt+1}|\txm_{\Tt},\rxm_{\Tt}) \leq&\; \Ent(\rxv_{\Tt+1}|\txm_{\ceil{(1-\dx)\Tt}},\rxm_{\Tt}),
% \end{align*}
% for any $\dx>0$ and therefore,
% \begin{equation*}
%     \EntInnC{\txA}{\rxA}{}{}\leq \lim_{\dx\upto 0}\EntInnC{\txA}{\rxA}{\dx}{}.
% \end{equation*}
% Equation \eqref{eq:two_bound_met} then implies A2.
% \end{proof}
% This lemma is helpful because it replaces computing $\EntInnC{\txA}{\rxA}{}{}$ with computing
% $\EntInnC{\txA}{\rxA}{}{\dy}$ and $\EntInnC{\txA}{\rxA}{\dx}{}$, which can be done using Corollary \ref{cor:A1_derivative_relation}. 

Define
\begin{equation}
    \EntR(\rxA_\dy|\txA_\dx)=\lim_{\Tlim\to\infty}\frac{1}{\rn\Tt}\Ent(\rxm_{{\dy\Tt}}|\txm_{{\dx\Tt}}),
    \label{eq:EntR_cond_def_high}
\end{equation}
\begin{align*}
  \cII{\txA_\dy}{\rxA_\dy}{}= \lim_{\Tlim\to\infty}&\frac{1}{\rn}\MuI(\txv_{{\dy\Tt}+1};\rxv_{{\dy\Tt}+1}|\txm_{{\dy\Tt}},\rxm_{{\dy\Tt}}),
  \numberthis
  \label{eq:cI_def_with_P_tau_high}
\end{align*}
\begin{align*}
    \cIRA{\txA_\dy}{\rxA_\dy}{}=\lim_{\Tlim\to\infty}&\frac{1}{\rn\Tb}\MuI(\txm^{{\dy\Tt}+1};\rxm^{{\dy\Tt}+1}|\txm_{{\dy\Tt}},\rxm_{{\dy\Tt}}),
    \numberthis
    \label{eq:joint_MuI_thm_high}
\end{align*}
where $\txm^t = [\txv_t,\txv_{t+1},\cdots,\txv_{\Tb}]$, and $\rxm^t=[\rxv_t,\rxv_{t+1},\cdots,\rxv_{\Tb}]$.  Similarly, we drop the subscripts $\dy$ and $\dx$ in $\cII{\txA_\dy}{\rxA_\dy}{}$, $\cIRA{\txA_\dy}{\rxA_\dy}{}$,  and $\EntR(\rxA_\dy|\txA_\dx)$ when $\dy=1$ or $\dx=1$.

\subsection{Main result}
\begin{thm}
\label{thm:MI_equal_gap&computation_of_MuI_from_derivative}
Under Assumption A1, 
\begin{align*}
    \cIRA{\txA_{0}}{\rxA_{0}}{}&\geq\;(1-\tau)\cII{\txA}{\rxA}{}
    \numberthis
    \label{eq:MuI_begin_lower_bound}\\
    &\geq\; (1-\tau)\lim_{\Tlim\to\infty}\MuI(\txv_{\Tt+1};\rxv_{ \Tt+1}|\hat{\chv}_{\Tb}),
    \numberthis
    \label{eq:MuI_estimate_lower_bound}
\end{align*}
where $\hat{\chv}_{\Tb}$ is any estimate of $\chv_{{\Tb}}$ that is a function of $(\txm_{ \Tt},\rxm_{ \Tt})$.  When A2 is also met,
%\eqref{eq:MuI_begin_lower_bound} can be calculated as
\begin{equation}
    \cIInn{\txA}{\rxA}{} = \lim_{\dy\downto 1}\frac{\partial\EntR(\rxA_\dy|\txA)}{\partial\dy} - \lim_{\dy\downto 1} \frac{\partial\EntR(\rxA_\dy|\txA_\dy)}{\partial\dy}.
    \label{eq:MuI_from_derivative}
\end{equation}


\end{thm}

\begin{proof}
Please see Appendix \ref{app:proof_MI_equal_gap&computation_of_MuI_from_derivative}.
\end{proof}

Hence, the mutual information limit with one-shot learning $(1-\tau)\cIInn{\txA}{\rxA}{}$ is a lower bound of the mutual information without any training, and is an upper bound of the mutual information with any estimate of the unknown parameters.  The expression in \eqref{eq:MuI_begin_lower_bound} can be calculated as a derivative using \eqref{eq:MuI_from_derivative} as long as $\EntR(\rxA_\dy|\txA_\dx)$ is available.
% Based on Theorem \ref{thm:MI_equal_gap}, we further show in the following theorem that under certain conditions, $\cIInn{\txA}{\rxA}{}$ can be easily computed for some large-scale systems without any assumptions of linearity of the system or Gaussianity in any of the processes.
The next section shows how $\EntR(\rxA_\dy|\txA_\dx)$ may be computed, and the section that follows provides operational significance to \eqref{eq:MuI_begin_lower_bound} in the form of a channel coding theorem.  An example application of the theorem appears in Section~\ref{sec:step_by_step_tauopt}.

%\section{An example and channel coding theorem}
\subsection{Computation of $\EntR(\rxA_\dy|\txA_\dx)$}

%\subsection{Availability of $\EntR(\rxA_\dy|\txA)$}
%\label{subsec:compute_EntR}
%Theorem \ref{thm:computation_of_MuI_from_derivative} shows that derivatives of $\EntR(\rxA_\dy|\txA_\dx)$ can be used to compute mutual information in the large $\Tt$ limit.  
An expression for $\EntR(\rxA_\dy|\txA_\dx)$ may be derived from $\EntR(\rxA_\dy|\txA)$ when this latter quantity is available.  In some cases $\EntR(\rxA_\dy|\txA)$ can be obtained through methods employed in statistical mechanics by treating the conditional entropy as free energy in a large-scale system.  Free energy is a fundamental quantity \cite{castellani2005spin,mezard2009information} that has been analyzed through the powerful ``replica method", and this, in turn, has been applied to entropy calculations in 
%compressed sensing \cite{sakata2013sample,schulke2016phase,kabashima2016phase},  
machine learning \cite{engel2001statistical,opper1996statistical,shinzato2008learning,ha1993generalization} and wireless communications \cite{wen2015performance,wen2015joint,wen2016bayes}, in both linear and nonlinear systems.  
%These references generally consider vector inputs and outputs, and our results generalize to these cases as shown in Section \ref{sec:high_dimen_generalize}.

The entropy $\EntR(\rxA|\txA)$ (equivalent to $\EntR(\rxA_\dy|\txA)$ where $\dy=1$) is considered in \cite{engel2001statistical,opper1996statistical,ha1993generalization,shinzato2008learning}, where the input is multiplied by an unknown vector as an inner product and then passes through a nonlinearity to generate a scalar output. In \cite{engel2001statistical,opper1996statistical,ha1993generalization}, the inputs are \iid, while orthogonal inputs are considered in \cite{shinzato2008learning}. 
%In \cite{engel2001statistical,opper1996statistical,shinzato2008learning}, the output is the sign of the input, while a sigmoid nonlinearity with a continuous output is considered in \cite{ha1993generalization}. 
The entropy $\EntR(\rxA_{\dy}|\txA)$ for MIMO systems is considered in \cite{wen2015performance,wen2015joint,wen2016bayes}, where the inputs are \iid in the training phase and are \iid in the data phase, but the distributions in the two phases can differ. In \cite{wen2015performance}, a linear system is considered where the output is the result of the input multiplied by an unknown matrix, plus additive noise, while in \cite{wen2015joint,wen2016bayes} uniform quantization is added at the output.


As we now show, the expression for $\EntR(\rxA_{\dy}|\txA)$ for $\dy\geq 1$ can be leveraged to compute $\EntR(\rxA_\dy|\txA_\dx)$ for all $\dy,\dx>0$.  We consider the case when the input $\txv_t$ are \iid for all $t$, and the distribution set $\cP(\Tb,\tau)$ defined in \eqref{eq:P_set_high} can therefore be simplified as
\begin{align}
    \cP(\Tb,\tau)=\{p(\rxv|\txv;\chva{\Tb}),p(\chva{\Tb}),p(\txv)\}.
    \label{eq:P_set_high_iid_input}
\end{align}
The following theorem assumes that we have $\EntR(\rxA_\dy|\txA)$ available as a function of $(\tau,\dy)$ for all $\dy\geq 1$.

\begin{thm}
\label{thm:scale_joint_entropy}
Assume that Assumption A1 is met, $\txv_t$ are \iid for all $t$, $\EntR(\rxA_\dy|\txA)$ exists and is continuous in $\tau$ and $\dy$ for $\tau\in(0,1)$ and $\dy\in(0,\frac{1}{\tau}]$. Define
\begin{align}
    F(\tau,\dy)=\EntR(\rxA_\dy|\txA),
    \label{eq:entropy_generator}
\end{align}
where $\dy\geq 1$ and $\EntR(\rxA_\dy|\txA)$ is defined in \eqref{eq:EntR_cond_def_high}.
Then
\begin{align*}
    \EntR(\rxA_\dy|\txA_{\dx})=
      u\cdot F\left(u\tau,\frac{\dy-u}{\dx}+1\right), 
    \numberthis
    \label{eq:EntR_rx_gamma_tx_lambda}
\end{align*}
for all $\dy,\dx\in(0,\frac{1}{\tau}]$, where $u=\min(\dy,\dx)$.
\end{thm}
\begin{proof}
According to  \eqref{eq:EntR_cond_def_high} and \eqref{eq:entropy_generator}, we have
\begin{align*}
    F(\tau,\dy)=\lim_{\Tlim\to\infty}\frac{1}{\rn\Tt}\Ent(\rxm_{{\dy\Tt}}|\txm_{\Tt}),
\end{align*}
%where the unknown parameter vector is $\chva$.
which is computed using $\cP(\Tb,\tau)$ defined in \eqref{eq:P_set_high_iid_input}.  When $\dx\geq\dy>0$, we have
\begin{align}
    &\EntR(\rxA_\dy|\txA_\dx)=\lim_{\Tlim\to\infty}\frac{1}{\rn\Tt}\Ent(\rxm_{{\dy\Tt}}|\txm_{{\dx\Tt}}) \label{eq:Hdef}\\
    =&\lim_{\Tlim\to\infty}\frac{1}{\rn\Tt}\Ent(\rxm_{{\dy\Tt}}|\txm_{{\dy\Tt}})=\lim_{{\Tlim}\to\infty}\frac{\dy}{\rn\tilde{\tau}{\Tb}}\Ent(\rxm_{\tilde{\tau}{\Tb}}|\txm_{\tilde{\tau}{\Tb}}).
    \label{eq:change_of_variable}
%    \nonumber
\end{align}
where $\tilde{\tau}={\dy\tau}$.
%\begin{align*}
%    \EntR(\rxA_\dy|\txA_\dx)=
%    \numberthis
%    \label{eq:change_of_variable}
%\end{align*}
% Since $\cP(\Tb,\tau)$ only depends on $\Tb$, we can rewrite $\cP(\Tb,\tau)$ as
% \begin{align*}
%     \cP(\Tb,\tau)=\cP(\Tb,\frac{\tilde{\tau}}{\dy}),
% \end{align*}
% and t
Therefore, \eqref{eq:entropy_generator} and \eqref{eq:change_of_variable} yield
\begin{align*}
   \EntR(\rxA_\dy|\txA_\dx) = \dy\cdot F(\tilde{\tau},1)=\dy\cdot F(\dy\tau,1).
   \numberthis
   \label{eq:expand_1}
\end{align*}
When $\dy>\dx> 0$, let $\tilde{\tau}={\dx\tau}$, and then \eqref{eq:Hdef} yields
\begin{align*}
   \EntR(\rxA_\dy|\txA_\dx) &=\lim_{{\Tlim}\to\infty}\frac{\dx}{\rn\tilde{\tau}{\Tb}}\Ent(\rxm_{\dy\tilde{\tau}{\Tb}/\dx}|\txm_{\tilde{\tau}{\Tb}})\\
    &=\dx\cdot F\left(\tilde{\tau},\frac{\dy}{\dx}\right) = \dx\cdot F\left(\dx{\tau},\frac{\dy}{\dx}\right) .
    \numberthis
   \label{eq:expand_2}
\end{align*}
By combining \eqref{eq:expand_1} and \eqref{eq:expand_2}, we obtain \eqref{eq:EntR_rx_gamma_tx_lambda}.
\end{proof}


\subsection{Channel coding theorem}
\label{subsec:channel_coding_thm_SISO}
We now provide an operational description of the mutual information inequality \eqref{eq:MuI_begin_lower_bound}. 
%We consider a communication system and develop a channel coding theorem that shows the limit of the rate that can be achieved for reliable communication. For simplicity, we consider systems with scalar input and output.
We consider a communication system where the channel is constant for blocklength $\Tb$, and then changes independently and stays constant for another blocklength, and so on. The first $\Tt$ symbols of each block are used for training with known input and output.
Under Assumption A1, the communication system is memoryless, is time-invariant within each block, and the input is \iid independent of $\txm_{\Tt}$ after training.  The system is retrained with every block, and the message to be transmitted is encoded over the data phase of multiple blocks. 

A $(2^{nR\rn\Tb },n,\Tb)$-code for a block-constant channel with blocklength $\Tb$ is defined as an encoder that maps a message $S\in\{1,2,\ldots,2^{nR\rn\Tb}\}$ to the input in the data phase $\txm^{\Tt+1}$ among $n$ blocks,
and a decoder that maps $\txm_{\Tt}$, and the entire output $\rxm_{\Tb}$ for $n$ blocks to $\hat{S}\in\{1,2,\ldots,2^{nR\rn\Tb}\}$, where $\rn = \frac{\alpha\Tb}{\beta}$. The code rate $R$ has units ``bits per transmission per receiver", and the maximum probability of error of the code is defined as
\begin{equation}
    \Pe(n,\Tb)=\max_{S}\Pr(\hat{S}\neq S).
    \label{eq:max_prob_err}
\end{equation}
The channel coding theorem is shown below.
\begin{thm}
\label{thm:channel_coding_thm_inf_Tb}
Assume A1 is met, with a channel that is constant with blocklength $\Tb$, whose conditional distribution is parameterized by $\chva{\Tb}$ and is independent of the input.  If $\cII{\txA}{\rxA}{}$ exists, then for every $R$ that satisfies
\begin{equation*}
    R<(1-\tau)\cII{\txA}{\rxA}{},
%    \label{eq:rate_lower_bound}
\end{equation*}
there exists $\Tb_0>0$, so that for all $\Tb>\Tb_0$, we can find a code $(2^{nR\rn\Tb },n,\Tb)$ with maximum probability of error $\Pe(n,\Tb)\to 0$ as $n\to\infty$.
\end{thm}
\begin{proof}
Define 
\begin{equation*}
    \cR_{\Tb}=\frac{1}{\Tb\rn}\MuI(\txm^{\Tt+1};\rxm_{\Tb}|\txm_{\Tt}).
\end{equation*}
For any finite $\Tb$, according to the classical channel coding theorem \cite{cover2012elements,yeung2008information,effros2010generalizing}, for every $R<\cR_{\Tb}$, there exists a code $(2^{n R \rn \Tb},n,\Tb)$ with maximum probability of error $\Pe(n,\Tb)\to 0$ as $n\tendsto\infty$. 

It is clear that $\txm^{\Tt+1}$ is independent of $(\txm_{\Tt},\rxm_{\Tt})$. Therefore, we have
\begin{equation*}
    \cR_{\Tb}=\frac{1}{\Tb\rn}\MuI(\txm^{\Tt+1};\rxm^{\Tt+1}|\txm_{\Tt},\rxm_{\Tt}).
\end{equation*}

Since $\txv_{\Tt+1},\txv_{\Tt+2},\ldots,\txv_{\Tb}$ are \iid, and $p(\rxv_t|\txv_t;\chva{\Tb})$ is a fixed conditional distribution for all $t=1,2,\ldots,\Tb$, we have 
\begin{equation}
    \cR_{\Tb}\geq \frac{(1-\tau)}{\rn}\MuI(\txv_{\Tt+1};\rxv_{\Tt+1}|\txm_{\Tt},\rxm_{\Tt}).
    \label{eq:bound_of_R}
\end{equation}
According to the definition in \eqref{eq:cI_def_with_P_tau_high},
\begin{align*}
     \cII{\txA}{\rxA}{}=\lim_{\Tlim\to\infty} \frac{1}{\rn} \MuI(\txv_{\Tt+1};\rxv_{\Tt+1}|\txm_{\Tt},\rxm_{\Tt}).
\end{align*}
Therefore, for any $\kappa>0$, there exists a number $\Tb_0>0$ so that when $\Tb>\Tb_0$, we have
\begin{align*}
    \frac{1}{\rn}\MuI(\txv_{\Tt+1};\rxv_{\Tt+1}|\txm_{\Tt},\rxm_{\Tt})>\cII{\txA}{\rxA}{} - \kappa,
\end{align*}
and \eqref{eq:bound_of_R} yields
\begin{align*}
    \cR_{\Tb}>(1-\tau)(\cII{\txA}{\rxA}{} - \kappa),
\end{align*}
which means any rate $R\leq(1-\tau)(\cII{\txA}{\rxA}{} - \kappa)$ is achievable. 

By taking the limit $\kappa\downto 0$, we finish the proof.
\end{proof} 

This theorem shows that rates below $(1-\tau)\cII{\txA}{\rxA}{}$ are achievable when $\Tb$ is chosen large enough.  Only an achievability statement is given here since $(1-\tau)\cII{\txA}{\rxA}{}$ is a lower bound on $\cR_{\Tb}$ for large $\Tb$. 
%In general, both an achievability and converse apply to $\cR_{\Tb}(\tau)$ for large $\Tb$, which converges to $\cIR{\txA}{\rxA}{\tau}$ defined in \eqref{eq:cIR_def} in the limit, but we omit this discussion.


\section{Steps for Computing Optimal Training Time and an Example}
\label{sec:step_by_step_tauopt}
We summarize the process to compute the optimal training time $\tauopt$ for a memoryless, time-invariant system with unknown parameters.  We assume that the input dimension $\tn$, the output dimension $\rn$, and the coherence time (block of symbols) $\Tb$ have the ratios defined in \eqref{eq:ratios}. The unknown parameters of the system are constant within the block, and change independently in the next block.  The first $\Tt$ symbols of each block are used for training and the remaining $\Tb-\Tt$ are for data.  We assume $\Tlim\to \infty$, and solve \eqref{eq:opt_training_tau} as an approximation of \eqref{eq:optimal_training_time_finite}.  The input $\txv_t$ are \iid for all $t=1,\dots,\Tb$. 

The process includes the following seven steps:
\begin{itemize}
    \item [1)] Verify Assumption A1 \eqref{eq:fixed_channel_in_data}--\eqref{eq:iid_data_input_distribution} based on the set of distributions in \eqref{eq:P_set_high}.
    
    \item [2)] Compute $\EntR(\rxA_\dy|\txA)$ defined in \eqref{eq:EntR_cond_def_high} for $\dy \ge 1$ based on \eqref{eq:P_set_high}, and express it as a function of $\tau$ and $\dy$ as $F(\tau,\dy)$ \eqref{eq:entropy_generator}.
    
    \item [3)] Compute $\EntR(\rxA_\dy|\txA_\dx)$ defined in \eqref{eq:EntR_cond_def_high}, for all $\dy, \dx \in (0, \frac{1}{\tau}]$ by using Theorem~\ref{thm:scale_joint_entropy} and $F(\tau,\dy)$.
    
    \item [4)] Compute $\EntInnC{\txA}{\rxA}{\dx}{\dy}$ and $\EntInnC{\txA}{\rxA}{\dy^+}{\dy}$ defined in \eqref{eq:EntInn_cond_def_high}-\eqref{eq:EntInn_cond_def_one_extra_high} by taking the derivative of $\EntR(\rxA_\dy|\txA_\dx)$ and $\EntR(\rxA_\dy|\txA_\dy)$ (Corollary \ref{cor:A1_derivative_relation}(a)--(b) in Appendix).
    
    \item [5)] Verify Assumption A2 \eqref{eq:training_phase_continue} by examining the expressions of $\EntInnC{\txA}{\rxA}{\dy^+}{\dy}$ and verify \eqref{eq:continue_in_data} with Corollary \ref{cor:A1_derivative_relation}(c).
    
    \item [6)] Compute $\cIInn{\txA}{\rxA}{}$ by using \eqref{eq:MuI_from_derivative}.

    \item [7)] Solve $\tauopt$ using \eqref{eq:opt_training_tau}.
    
\end{itemize}
% In this section, we provide a step-by-step process to compute the bound, which includes the following x steps.
% \begin{itemize}
%     \item [1)] Verify Assumption A1 \eqref{eq:fixed_channel_in_data}--\eqref{eq:iid_data_input_distribution} based on the set of distributions in \eqref{eq:P_set_high}.
    
%     \item [2)] Compute $\EntR(\rxA_\dy|\txA), \dy \ge 1$ defined in \eqref{eq:EntR_cond_def_high} based on \eqref{eq:P_set_high}.
    
    
%     \item [3)] Compute $\EntR(\rxA_\dy|\txA_\dy), \dy \ge 1$ defined in \eqref{eq:EntR_cond_def_high}. If $\txv_t$ are \iid for all $t$, Theorem~\ref{thm:scale_joint_entropy} can be used to simplify the calculation by leveraging the expression of $\EntR(\rxA_\dy|\txA)$.
    
%     \item [4)] Compute $\EntInnC{\txA}{\rxA}{}{\dy}$ and $\EntInnC{\txA}{\rxA}{\dy^+}{\dy}, \dy > 1$ defined in \eqref{eq:EntInn_cond_def_high}--\eqref{eq:EntInn_cond_def_one_extra_high} by taking the derivative of $\EntR(\rxA_\dy|\txA)$ and $\EntR(\rxA_\dy|\txA_\dy)$ (Corollary \label{cor:A1_derivative_relation} (a)).
    
%     \item [5)] Verify Assumption A2 \eqref{eq:training_phase_continue}--\eqref{eq:continue_in_data}.
    
%     \item [6)] Compute $\cIInn{\txA}{\rxA}{}$ by using \eqref{eq:MuI_from_derivative}

% \end{itemize}

The following simple example applies these steps.
\subsection*{Example: Bit flipping through random channels}
\noindent
Let 
%$\tx_{\Tb,t}$ and $\rx_{\Tb,t}$ be the input and the output of a binary system with blocklength $\Tb$ :
\begin{equation}
    \rx_{t}=\tx_{t}\oplus \chs_{k_t},\quad t=1,\ldots,\Tb,
    \label{eq:binary_XOR_channel}
\end{equation}
where, since $\tn=\rn=1$, the binary input $\tx_t$ and output $\rx_t$ are scalars, and $\tx_{t}$ is XOR'ed with a random bit $\chs_{k_t}$ intended to model the unknown ``state" of the channel $k_t$.  Thus, each channel either lets the input bit directly through, or inverts it.  The $\tx_{t}$ are \iid equally likely to be zero or one, Bernoulli($\frac{1}{2}$) random variables.  Let $a>0$ be a parameter, where $a\cdot\Tb$ is the (integer) number of unique channels whose states are stored in the vector $\chv_{\Tb}=[\chs_{1},\chs_{2},\cdots,\chs_{a\cdot\Tb}]^\Tp$ comprising \iid Bernoulli($\frac{1}{2}$) random variables that are independent of the input. The channel selections $\bk_{\Tb}=[k_1,\ldots,k_{\Tb}]$ are chosen as an \iid uniform sample from $\{1,2,\cdots,a\cdot\Tb\}$ (with possible repetitions), and the choices are known to the receiver.  We wish to send training signals through these channels to learn $\chv_{\Tb}$; the more entries of this vector that we learn, the more channels become useful for sending data, but the less time we have to send data before the blocklength $\Tb$ runs out and $\chv_{\Tb}$ changes.
We want to determine the optimum $\tau$ as $\Tlim\to\infty$ using \eqref{eq:opt_training_tau}. We therefore follow the steps above.

\begin{itemize}

\item [1)] From \eqref{eq:binary_XOR_channel}, we have
\begin{align*}
    p(\rxv_{\Tb}|\txv_{\Tb};\chva{\Tb}) =\prod_{t=1}^{\Tb}p(\rx_t|\tx_t;\chva{\Tb}),
\end{align*}
where $p(\rx_t|\tx_t;\chva{\Tb})=\mathbbm{1}_{(\rx_t=\tx_{t}\oplus \chs_{k_t})}$ for all $t$.  Here the notation is slightly abused, since now $\tn=\rn=1$, we use $\rxv_{t}$ and $\txv_{t}$ to denote $[\rx_1,\dots,\rx_{t}]^\Tp$ and $[\tx_1,\dots,\tx_{t}]^\Tp$.
It is clear that Assumption A1 is met and $\tx_t$ are \iid for all $t$ independent of $\chv_{\Tb}$.

\item [2)]
By definition, $\EntR(\rxA|\txA)=\lim\limits_{\Tlim\to\infty}\frac{1}{\Tt}\Ent(\rxv_{\Tt}|\txv_{\Tt})$. 
The model \eqref{eq:binary_XOR_channel} yields
\begin{align*}
    \Ent(\rxv_{\Tt}|\txv_{\Tt}) &\stackrel{(\rm a)}{=}\Ent(\{\chs_{k_1},\ldots,\chs_{k_\Tt}\}|\txv_{\Tt})
    \stackrel{(\rm b)}{=}\Ent(\{\chs_{k_1},\ldots,\chs_{k_\Tt}\})
    \stackrel{(\rm c)}{=}\E_{\bk_{\Tt}}|A_\Tt| \\
    &\stackrel{(\rm d)}{=}\sum_{i=1}^{a\Tb}\E(\mathbbm{1}_{(i\in A_{\Tt})})   ={a\Tb}(1-(1-\frac{1}{a\Tb})^{\Tt}).
    %\numberthis
\end{align*}
where $A_{\Tt} = \{k_1,\ldots,k_\Tt\}$, $^{(a)}$ uses $\chs_{k_t}=\rx_t\oplus\tx_t$, $^{(b)}$ uses the independence between $\txv_{\Tt}$ and $\chs_t$, $^{(c)}$ uses the independence between $\chs_t$, $\chs_k$ when $t\neq k$ and $^{(d)}$ uses $|A_{\Tt}|=\sum_{i=1}^{a\Tb} \mathbbm{1}_{(i\in A_{\Tt})}$, where $\mathbbm{1}_{(\cdot)}$ is the indicator function.
Therefore,
\begin{align*}
    &\EntR(\rxA|\txA)=\lim_{\Tlim\to\infty}\frac{a}{\tau}(1-(1-\frac{1}{a\Tb})^{\Tt})=\frac{a}{\tau}(1-e^{-\frac{\tau}{a}}).
    \numberthis
    \label{eq:XOR_tau_1}
\end{align*}
By the chain rule for entropy, for $\dy > 1$, we have
\begin{align*}
    \EntR&(\rxA_\dy|\txA)=\EntR(\rxA|\txA) + \lim_{\Tlim\to\infty}\frac{1}{\Tt}\Ent(\rx_{\Tt+1},\rx_{\Tt+2},\ldots,\rx_{{\dy\Tt}}|\txv_{\Tt},\rxv_{\Tt}).
\end{align*}
Since
\begin{align*}
   {\dy\Tt} - \Tt\geq&\Ent(\rx_{\Tt+1},\rx_{\Tt+2},\ldots,\rx_{{\dy\Tt}}|\txv_{\Tt},\rxv_{\Tt})\\
    \ge& \Ent(\tx_{\Tt+1},\tx_{\Tt+2},\ldots,\tx_{{\dy\Tt}}|\txv_{\Tt},\rxv_{\Tt},\chva{\Tb})\\
    =&\Ent(\tx_{\Tt+1},\tx_{\Tt+2},\ldots,\tx_{{\dy\Tt}})={\dy\Tt} - \Tt,
\end{align*}
we conclude that 
\begin{align*}
    F(\tau,\dy)=\EntR&(\rxA_\dy|\txA)=\frac{a}{\tau}(1-e^{-\frac{\tau}{a}})+\dy-1.
    \numberthis
    \label{eq:tau_XOR_EntR}
\end{align*}
%For $\tau\in(0,1]$ and $\dy>0$, \eqref{eq:XOR_tau_1} and \eqref{eq:tau_XOR_EntR} yield
%Therefore
%\begin{align*}
%    F(\tau,\dy)=\frac{a}{\tau}(1-e^{-\frac{\tau}{a}})+\dy.
%    \numberthis
%    \label{eq:F_expression_XOR}
%\end{align*}
%where $F(\tau,\dy)$ is defined as in \eqref{eq:entropy_generator}.  

\item [3)]
Theorem \ref{thm:scale_joint_entropy} yields
\begin{align*}
    \EntR(\rxA_\dy|\txA_\dx) =  \begin{cases} 
      \frac{a}{\tau}(1-e^{-\frac{\tau}{a}\dy}), & \dy\leq\dx; \\
      \frac{a}{\tau}(1-e^{-\frac{\tau}{a}\dx})+(\dy-\dx), & \dx<\dy,
      \end{cases}
\end{align*}
for $\dy,\dx\in(0,\frac{1}{\tau})$.

\item[4)]
Then, Corollary \ref{cor:A1_derivative_relation}(a)-(b) yields
\begin{align*}
    \EntInnC{\txA}{\rxA}{\dy^+}{\dy} =\frac{\partial\EntR(\rxA_\dy|\txA_\dy)}{\partial\dy}=e^{-\frac{\tau}{a}\dy},
\end{align*}
\begin{align*}
    \EntInnC{\txA}{\rxA}{\dx}{\dy} =\frac{\partial\EntR(\rxA_\dy|\txA_\dx)}{\partial\dy}=  \begin{cases}
      e^{-\frac{\tau}{a}\dy}, & \dy<\dx; \\
      1, & \dy>\dx.
      \end{cases}
\end{align*}
%Since $\tx_t$ are \iid, 

\item[5)]
$\EntInnC{\txA}{\rxA}{\dy^+}{\dy}$ and Corollary \ref{cor:A1_derivative_relation}(c) allow us to conclude that Assumption A2 also holds.

\item[6)]
From Theorem \ref{thm:MI_equal_gap&computation_of_MuI_from_derivative}, we obtain
\begin{align*}
    \cII{\txA}{\rxA}{}
%    =& \lim_{\gamma\downto\tau}\frac{\partial\EntR(\rxA_\gamma|\txA_{\tau})}{\partial\gamma}-\lim_{\gamma\downto\tau}\frac{\partial\EntR(\rxA_\gamma|\txA)}{\partial\gamma}\\
    =&\;1-e^{-\frac{\tau}{a}},
%    \numberthis
%    \label{eq:MuI_example_exp}
%    \cIRA{\txA}{\rxA}{}\geq &\;(1-\tau)(1-e^{-\frac{\tau}{a}}).
\end{align*}
\item [7)]
Finally, \eqref{eq:opt_training_tau} yields \begin{equation}
    \tauopt=\argmax\limits_{\tau}(1-\tau)(1-e^{-\frac{\tau}{a}}),
    \label{eq:Ex5tauopt_def}
\end{equation}
or
\begin{equation*}
    \tauopt=\begin{cases} 
      -a\ln a, & a\to 0; \\
      \frac{1}{2}, & a\to\infty;\\
      \frac{1}{e}, & a=\frac{1}{e}.
 %     0.443, & a=1.
      \end{cases}
      %\label{eq:Ex5tauopt}
\end{equation*}
\end{itemize}
When $a$ is small, $\tauopt$ is larger than $a$; when $a$ is large, $\tauopt$ saturates at $\frac{1}{2}$; and $a=\frac{1}{e}$ is the dividing line between $\tauopt>a$ and $\tauopt<a$.  The corresponding rates are
\begin{equation*}
    \Ropt=\begin{cases} 
      (1+a\ln a)(1-a), & a\to 0; \\
      \frac{1}{2}(1-e^{-\frac{1}{2a}}), & a\to\infty;\\
      (1-\frac{1}{e})^2, & a=\frac{1}{e}.
 %     0.443, & a=1.
      \end{cases}
      %\label{eq:Ex5Ropt}
\end{equation*}
The optimum fraction of the blocklength $\Tb$ that should be devoted to training varies as a function of the number of possible unique channels.  When $a=1$, the number of unique channels equals $\Tb$, and the $\tauopt\approx 0.44$.  For a large number of unique channels relative to the blocklength ($a\tendsto\infty$), the fraction of the training time saturates at $1/2$.  When $a$ is small, the optimum fraction of the blocklength devoted to training decreases to zero, but more slowly than $a$.

In this example, a traditional finite-system information-theoretic analysis and simulation is possible (these calculations are omitted). Figure \ref{fig:example_compare_triditional_and_derivative} shows the results as a function of $\Tb$, where we can see that as $\Tb$ grows, the resulting $\tauopt$ quickly approaches the large-system results.  The fact that we can use a large-system limit to approximate a finite-system limit is important when applying the Theorems in realistic scenarios.

\begin{figure}
\includegraphics[width=3.8in]{figs/Example5_traditional_vs_derivative_figure.jpg}
\centering
    \caption{Optimal training time $\tauopt$ vs blocklength $\Tb$, derived from finite-$\Tb$ analysis (solid curves) and the proposed large-scale method (dashed curves).  The large-scale analysis \eqref{eq:Ex5tauopt_def} shows excellent agreement with the finite-dimensional analysis (which is omitted) for even small values of $\Tb$.}
    \label{fig:example_compare_triditional_and_derivative}
\end{figure}

%The next section develops a channel coding theorem to provide operational significance to $(1-\tau)\cII{\txA}{\rxA}{}$.

\section{Discussion and Conclusion}
\label{sec:discussion}

\subsection{Number of unknowns and bilinear model}
In general, a finite number of unknowns in the model leads to uninteresting results as $\Tb\tendsto\infty$.  For example, consider a system modeled as
\begin{equation*}
    \rx_{t} = \chs\tx_t +v_t,\quad t=1,2,\ldots
%    \label{eq:training_1D}
\end{equation*}
where $\chs$ is the unknown gain of the system, $\tx_t$, $\rx_t$ are the input and corresponding output, $v_t$ is the additive noise, $\tau$ is the fraction of time used for training.  This system is bilinear in the gain and the input.  We assume that $\ns_t$ is modeled as \iid Gaussian $\cN(0,1)$, independent of the input. The training signals are $\tx_t=1$ for all $t=1,2,\ldots, \Tt$, and the data signals $\tx_t$ are modeled as \iid Gaussian $\cN(0,1)$ for all $t=\Tt+1,\Tt+2,\ldots$ An analysis similar to the Example produces
\begin{align*}
    \cIRA{\txA_0}{\rxA_0}{} \geq\frac{1-\tau}{2}\E_{\chs}\log(1+\chs^2),
\end{align*}
and therefore $\tauopt=0$ maximizes this bound. This result reflects the fact that $\chs$ is learned perfectly for any $\tau>0$ because there is only one unknown parameter for $\Tt$ training symbols as $\Tt\tendsto\infty$.  Hence, trivially, it is advantageous to make $\tau$ as small as possible.

More interesting is the ``large-scale" model
\begin{align*}
    \rxv_t=\bff(\chm\txv_t+\nv_t),\quad t=1,2,\ldots,
    \numberthis
    \label{eq:nonlinear_system_model}
\end{align*}
where $\txv_t$ and $\rxv_t$ are the $t$th input and output vectors with dimension $\tn$ and $\rn$, $\chm$ is an $\rn\times\tn$ unknown random matrix that is not a function of $t$, $\nv_1,\nv_2,\ldots$ are \iid unknown vectors with dimension $\rn$ and known distribution (not necessarily Gaussian), and $\bff(\cdot)$ applies a possibly nonlinear function $f(\cdot)$ to each element of its input. The training interval $\Tt$ is used to learn $\chm$.  

Let $\tn$ and $\rn$ increase proportionally to the blocklength $\Tb$ with the ratios defined in \eqref{eq:ratios};
%\begin{equation}
%    \ratio=\frac{\rn}{\tn},\quad\beta=\frac{\Tb}{\tn}.
%    \label{eq:ratios_scale_up}
%\end{equation}
such a model can be used in large-scale wireless communication, signal processing, and machine learning applications. In wireless communication and signal processing \cite{takeuchi2010achievable,takeuchi2012large,hassibi2003much,takeuchi2013achievable,li2016much,li2017channel,wen2015performance,wen2015joint,wen2016bayes}, $\txv_t$ and $\rxv_t$ are the transmitted signal and the received signal at time $t$ in a multiple-input-multiple-output (MIMO) system with $\tn$ transmitters and $\rn$ receivers, $\chm$ models the channel coefficients between the transmitters and receivers, $\Tb$ is the coherence time during which the channel $\chm$ is constant, $\nv_t$ is the additive noise at time $t$, $f(\cdot)$ models receiver effects such as quantization in analog-to-digital converters and nonlinearities in amplifiers. A linear receiver, $f(x)=x$, is considered in \cite{takeuchi2010achievable,takeuchi2012large,takeuchi2013achievable,wen2015performance}. Single-bit ADC's with $f(x)=\sign(x)$ are considered in \cite{li2016much,li2017channel}, and low-resolution ADC's with $f(x)$ modeled as a uniform quantizer are considered in \cite{wen2015joint,wen2016bayes}.  The training and data signals can be chosen from different distributions, as in \cite{hassibi2003much,li2016much,li2017channel}.  Conversely, the training and data signals can both be \iid, as in \cite{takeuchi2013achievable,wen2015performance,wen2015joint,wen2016bayes}.
%and some information-theoretic lower bounds when $f(\cdot)$ is a single-bit quantizer include \cite{kang2019training,gao2019channel}.  [!!!Put this reference in Part II] 

%, $\chm$ is the wireless channel coefficients between the sources and the receive nodes, $\nv_t$ is the additive noise, and $f(\cdot)$ again models each receive node.
%Various receiver models are considered in \cite{shirazinia2015massive,shirazinia2016massive,serra2017distributed,choi2014channel,choi2015quantized}.  In \cite{shirazinia2015massive,serra2017distributed,choi2015quantized}, $\chm$ is assumed known perfectly at the fusion center. In general, $\chm$ needs to be estimated before data transmission, and we consider that the first $\Tt$ inputs are known signals for channel estimation, and the rest time $\Td$ are used for transmission of data generated at the sources. 

Let $\rn=1$.  In machine learning, \eqref{eq:nonlinear_system_model} is a model of a single layer neural network (perceptron) \cite{opper1996statistical,engel2001statistical,shinzato2008learning} and $\txv_t$ is the input to the perceptron with dimension $\tn$, $\rxv_t$ is the scalar decision variable at time $t$, $\chm$ holds the unknown weights of the perceptron, and $f(\cdot)$ is the nonlinear activation function. A perceptron is often used as a classifier, where the output of the perceptron is the class label of the corresponding input. In \cite{opper1996statistical,engel2001statistical}, \iid inputs are used to learn the weights, and orthogonal inputs are used in \cite{shinzato2008learning}. Binary class classifiers are considered in \cite{opper1996statistical,engel2001statistical,shinzato2008learning}.
Training employs $\Tt$ labeled input-output pairs $(\txv_t,\rxv_t)$, and the trained perceptron then classifies new inputs before it is retrained on a new dataset.  Generally, both the training and data are modeled as having the same distribution.

%In many cases, we may approximate a large-blocklength system by letting $\Tb\to\infty$ with $\tn,\rn,\Tt,\Td$ scaled proportionally as in \eqref{eq:ratios_scale_up}.  
To obtain optimal training results for \eqref{eq:nonlinear_system_model}, Theorems \ref{thm:MI_equal_gap&computation_of_MuI_from_derivative}--\ref{thm:scale_joint_entropy} show that a starting point for computing
%$\EntR(\rxA_\gamma|\txA_{\tau}),\EntR(\rxA_\gamma|\txA),$
$\cII{\txA}{\rxA}{}$ is $\EntR(\rxA_\dy|\txA)$ for $\dy\geq 1$.  Fortunately, $\EntR(\rxA_\dy|\txA)$ results can sometimes be found in the existing literature; for example, in \cite{wen2015performance,wen2015joint,wen2016bayes}, $\EntR(\rxA_\dy|\txA)$ is used to calculate the mean-square error of the estimated input signal, conditioned on the training.  Our analysis shows how to leverage these same $\EntR(\rxA_\dy|\txA)$ results to derive the training-based mutual information.  
%We believe that $\EntR(\rxA|\txA_{\tau})$ calculations for other models can be leveraged using the framework provided herein to provide insight into training these mode

\subsection{Models for which assumptions are superfluous}

Assumption A2 is likely to be superfluous for certain common system models, such as when the distribution on $\txv_t$ is \iid through the training and data phases, and the transition probabilities can be written as a product as in Assumption A1. However, we have not yet characterized for which models A2 is automatically satisfied without additional assumptions on ${\cal H}'$, and think that this would be an interesting research topic for further work.



\appendices


\section{Proof of Theorem \ref{thm:MI_equal_gap&computation_of_MuI_from_derivative}}
\label{app:proof_MI_equal_gap&computation_of_MuI_from_derivative}

\subsection{Proof of the inequalities \eqref{eq:MuI_begin_lower_bound}--\eqref{eq:MuI_estimate_lower_bound}}
Under Assumption A1, we have
\begin{align*}
    &\MuI(\txm^{\Tt+1};\rxm_{\Tb}|\txm_\Tt)\\
    =&\MuI(\txm^{\Tt+1};\rxm_{\Tt}|\txm_\Tt)+\MuI(\txm^{\Tt+1};\rxm^{\Tt+1}|\txm_\Tt,\rxm_{\Tt})\\
    =&\MuI(\txm^{\Tt+1};\rxm^{\Tt+1}|\txm_\Tt,\rxm_{\Tt}),
\end{align*}
where the first equality uses the chain rule and the second uses that $\txm^{\Tt+1}$ is independent of $(\txm_\Tt,\rxm_\Tt)$. 
%Therefore,
% \begin{align*}
%     &\MuI(\txm^{\Tt+1};\rxm^{\Tt+1}|\txm_\Tt,\rxm_{\Tt}) = \Ent(\rxm_{\Tb}|\txm_{\Tt})-\Ent(\rxm_{\Tb}|\txm_{\Tb})\\
%     &\leq\Ent(\rxm_{\Tb})-\Ent(\rxm_{\Tb}|\txm_{\Tb})=\MuI(\txm_{\Tb};\rxm_{\Tb}),
% \end{align*}
% which, with \eqref{eq:joint_MuI_thm_high}, yields 
% \begin{align*}
%     &\cIRA{\txA_{0}}{\rxA_{0}}{}\geq\;\cIRA{\txA}{\rxA}{}.
%     \numberthis
%     \label{eq:joint_MuI_expression}
% \end{align*}
Moreover, 
\begin{align*}
    &\MuI(\txm^{\Tt+1};\rxm^{\Tt+1}|\txm_{\Tt},\rxm_{\Tt})\\
    \stackrel{}{=}&\;\Ent(\txm^{\Tt+1}|\txm_{\Tt},\rxm_{\Tt}) - \Ent(\txm^{\Tt+1}|\txm_{\Tt},\rxm_{\Tb})\\
    \stackrel{\rm{( a)}}{=}&\sum_{t={\Tt+1}}^{\Tb}(\Ent(\txv_t|\txm_{t-1},\rxm_{\Tt}) - \Ent(\txv_t|\txm_{t-1},\rxm_{\Tb}))\\
    \stackrel{\rm{( b)}}{\geq}&\sum_{t={\Tt+1}}^{\Tb}(\Ent(\txv_t|\txm_{t-1},\rxm_{t-1}) - \Ent(\txv_t|\txm_{t-1},\rxm_{t}))\\
    =& \sum_{t=\Tt+1}^{\Tb}\MuI(\txv_t;\rxv_t|\txm_{t-1},\rxm_{t-1}).
%    \numberthis
%    \label{eq:chain_rule_MuI_bound}
\end{align*}
Here, $^{(a)}$ uses the chain rule, $^{(b)}$ uses conditioning to reduce entropy. Equality in $^{(b)}$ can be achieved when $\chva{\Tb}$ is estimated perfectly from $(\txm_{\Tt},\rxm_{\Tt})$. 
%Then, by normalizing \eqref{eq:chain_rule_MuI_bound} by $\frac{\tau}{\Tt}$ and letting $\Tt\to\infty$, we see that the sum converges to the integral, which proves \eqref{eq:MuI_int_expression}. 
Since Assumption A1 implies that $\txv_t$ is independent of $(\txv_k,\rxv_k)$ when $k\neq t$ and $t\geq \Tt+1$, for all $t\geq \Tt+1$, we have
\begin{align*}
    &\MuI(\txv_{t+1};\rxv_{t+1}|\txm_{t},\rxm_{t}) - \MuI(\txv_{t};\rxv_{t}|\txm_{t-1},\rxm_{t-1})\\
    \stackrel{\rm{( a)}}{=}&(\Ent(\txv_{t+1}) - \Ent(\txv_{t+1}|\txm_t,\rxm_{t+1})) \\
    &\qquad - (\Ent(\txv_{t}) - \Ent(\txv_{t}|\txm_{t-1},\rxm_{t}))\\
    =&\Ent(\txv_{t}|\txm_{t-1},\rxm_{t}) - \Ent(\txv_{t+1}|\txm_t,\rxm_{t+1})\\
    \stackrel{\rm{( b)}}{=}&\Ent(\txv_{t+1}|\txm_{t-1},\rxm_{t-1},\rxv_{t+1}) - \Ent(\txv_{t+1}|\txm_t,\rxm_{t+1})
    \stackrel{\rm{( c)}}{\geq} 0.
%    \numberthis
%    \label{eq:monotonic_MuI_T}
\end{align*}
Here, $^{(a)}$ uses the independence between $\txv_t$ and $(\txm_{t-1},\rxm_{t-1})$, $^{(b)}$ uses Assumption A1, $^{(c)}$ uses conditioning to reduce entropy. Thus, $\MuI(\txv_t;\rxv_t|\txm_{t-1},\rxm_{t-1})$ is monotonically increasing with $t$ for all $t>\Tt$. 
%In the limit when $\Tt\to\infty$, \eqref{eq:monotonic_MuI_T} yields \eqref{eq:monotonic_MuI}.  A sufficient condition for achieving equality in $^{(c)}$ is when $\chva{\Tb}$ can be estimated perfectly from $(\txm_{\Tt},\rxm_{\Tt})$, and both entropies on the right hand side of $^{(b)}$ equal $\Ent(\txv_{t+1}|\chva{\Tb},\rxv_{t+1})$. 
%All the mutual informations and entropies are computed using the distribution set $\cP(\Tt,\tau)$, which is not expressed explicitly for simplicity in notation.
Then, in the limit when $\Tlim\to\infty$, we get  \eqref{eq:MuI_begin_lower_bound}. 

 
Also, 
 \begin{align*}
     &\MuI(\txv_{\Tt+1};\rxv_{\Tt+1}|\txm_{\Tt},\rxm_{\Tt})\\
     =&\Ent(\txv_{\Tt+1})-\Ent(\txv_{\Tt+1}|\txm_{\Tt},\rxm_{\Tt},\rxv_{\Tt+1})\\
     \stackrel{\rm{( a)}}{=}&\Ent(\txv_{\Tt+1})-\Ent(\txv_{\Tt+1}|\txm_{\Tt},\rxm_{\Tt},\hat{\chv}_{\Tb},\rxv_{\Tt+1})\\
     \stackrel{\rm{( b)}}{\geq}&\Ent(\txv_{\Tt+1})-\Ent(\txv_{\Tt+1}|\hat{\chv}_{\Tb},\rxv_{\Tt+1})\\
     =&\MuI(\txv_{\Tt+1};\rxv_{\Tt+1}|\hat{\chv}_{\Tb}),
 \end{align*}
 where $^{(a)}$ uses that $\hat{\chv}_{\Tb}$ is a function of $(\txm_{\Tt},\rxm_{\Tt})$, and $^{(b)}$ uses conditioning to reduce entropy. By taking the limit $\Tlim\to \infty$, we have \eqref{eq:MuI_estimate_lower_bound}.

\subsection{Proof of  \eqref{eq:MuI_from_derivative}}
We first show the derivative relationship between $\EntInn(\rxA_\dy)$ and $\EntR(\rxA_\dy)$ defined below, and then generalize to the conditional entropies which directly lead to the conclusion \eqref{eq:MuI_from_derivative}.
Define
\begin{align*}
\EntInn(\rxA_\dy) =\lim_{\Tlim\to\infty}\Ent(\rxv_{\ceil{\dy\Tt}+1}|\rxm_{\ceil{\dy\Tt}}),
    \numberthis
    \label{eq:EntInn_def}
\end{align*}
\begin{equation}
    \EntR(\rxA_\dy)=\lim_{\Tlim\to\infty}\frac{1}{\Tt}\Ent(\rxm_{\ceil{\dy\Tt}}),
    \label{eq:EntR_def}
\end{equation}
which can be considered as  $\EntInnC{\txA}{\rxA}{\dx}{\dy}$ and $\EntR(\rxA_\dy|\txA_{\dx})$ with $\dx=0$. For mathematical rigorousness, we keep the $\ceil{\cdot}$ notation here.
%We omit the conditioning on $\txA$ for the moment.
%\subsection{Derivative relationship between $\EntInn$ and $\EntR$}
We show that, under some conditions, 
 $\EntInn(\rxA_\dy)$ is the derivative of $\EntR(\rxA_\dy)$.
\begin{thm}
\label{thm:derivative_monotonic}
Suppose there exists a $\kappa>0$ so that $\Ent(\rxv_{t+1}|\rxm_{t})$ is monotonic in $t$ when $t\in[\floor{(\dy-\kappa)\Tt},\ceil{(\dy+\kappa)\Tt}]$ as $\Tlim\to\infty$. 

If $\EntR(\rxA_\dy)$ and its derivative with respect to $\dy$ exist, we have
\begin{equation}
    \EntInn(\rxA_\dy) =\frac{\partial\EntR(\rxA_\dy)}{\partial\dy}.
    \numberthis
    \label{eq:derivative_2}
\end{equation}
If both $\EntR(\rxA_\dy)$ and $\EntInn(\rxA_\dy)$ exist, and there exists a $c>0$ independent of $t$ and $\Tt$ so that $|\Ent(\rxv_{t+1}|\rxm_{t})|<c$,  we have
\begin{equation}
    \EntR(\rxA_\dy) = \int_{0}^{\dy}\EntInn(\rxA_u) d u .
    \label{eq:theorem_integral_monotonic_thm}
\end{equation}
\end{thm}

\begin{proof}
Equation \eqref{eq:theorem_integral_monotonic_thm} is an integral equivalent of \eqref{eq:derivative_2} and
we only prove \eqref{eq:derivative_2} for simplicity.  Without loss of generality, we assume that $\Ent(\rxv_{t+1}|\rxm_{t})$ is monotonically decreasing. Using the definition of $\EntR(\rxA_\dy)$ in \eqref{eq:EntR_def}, we have
\begin{align*}
    &\frac{1}{\kappa}(\EntR(\rxA_{\dy+\kappa})-\EntR(\rxA_\dy))\\
    =&\lim_{\Tlim\to\infty}\frac{\Ent(\rxm_{\ceil{(\dy+\kappa)\Tt}})-\Ent(\rxm_{\ceil{\dy\Tt}})}{\kappa\Tt}\\
    =&\lim_{\Tlim\to\infty}\frac{\sum_{t=\ceil{\dy\Tt}+1}^{\ceil{(\dy+\kappa)\Tt}}\Ent(\rxv_{t}|\rxm_{t-1})}{\kappa\Tt}\\
    \leq &\lim_{\Tlim\to\infty}\frac{(\kappa\Tt+1)\cdot\Ent(\rxv_{\ceil{\dy\Tt}+1}|\rxm_{\ceil{\dy\Tt}})}{\kappa\Tt}\\
    =&\lim_{\Tlim\to\infty}\Ent(\rxv_{\ceil{\dy\Tt}+1}|\rxm_{\ceil{\dy\Tt}}).
    \numberthis
    \label{eq:limit_upper_bound}
\end{align*}

Similarly to \eqref{eq:limit_upper_bound}, we also have
\begin{align*}
    &\lim_{\Tlim\to\infty}\Ent(\rxv_{\ceil{\dy\Tt}+1}|\rxm_{\ceil{\dy\Tt}})
    \leq\frac{1}{\kappa}(\EntR(\rxA_\dy)-\EntR(\rxA_{\dy-\kappa})).
    \numberthis
    \label{eq:limit_lower_bound}
\end{align*}
%\begin{align*}
%    &\lim_{\Tt\to\infty}\Ent(\rx_{(1+\dy)\Tt+1}|\rxv_{(1+\dy)\Tt})\\
%    \leq&\frac{1}{\kappa}(\EntR(\rxA_\dy)-\EntR(\rxA_{\dy-\kappa})).
%    \numberthis
%    \label{eq:limit_lower_bound}
%\end{align*}
Let $\kappa\downto 0$ in both \eqref{eq:limit_upper_bound} and \eqref{eq:limit_lower_bound}; because we assume that the derivative of $\EntR(\rxA_\dy)$ exists, these limits both equal this derivative.  Then, the definition of $\EntInn(\rxA_\dy)$ in \eqref{eq:EntInn_def} yields \eqref{eq:derivative_2}.
\end{proof}

Theorem \ref{thm:derivative_monotonic} is a consequence of the entropy chain rule and letting an infinite sum converge to an integral (standard Riemann sum approximation).  Such an analysis has also been used in the context of computing mutual information; for example  \cite{shamai2001impact,guo2005randomly,guo2005mutual,guo2008multiuser,honig2009advances}. %Several examples are given in Appendix \ref{app:examples_for_derivative_monotonic}.
Theorem \ref{thm:derivative_monotonic}  can be generalized to include conditioning on $\txA$, thus leading to the following corollary, provided that $\EntR(\rxA_\dy|\txA_\dx)$ and its derivative with respect to $\dy$ exist.
\begin{cor}
\label{cor:A1_derivative_relation}
Assume A1 holds.  (a) For $\dy>1$,
\begin{align*}
    \EntInnC{\txA}{\rxA}{}{\dy} =\frac{\partial\EntR(\rxA_\dy|\txA)}{\partial\dy},
    \numberthis
    \label{eq:derivative_data}
\end{align*}
\begin{align*}
    \EntInnC{\txA}{\rxA}{\dy^+}{\dy} =\frac{\partial\EntR(\rxA_\dy|\txA_\dy)}{\partial\dy}.
    \numberthis
    \label{eq:derivative_training}
\end{align*}

(b) If $\txv_t$ are \iid for all $t$, then for all $\dy,\dx>0$ and $\dy\neq\dx$,
\begin{align*}
    \EntInnC{\txA}{\rxA}{\dx}{\dy} &=\frac{\partial\EntR(\rxA_\dy|\txA_\dx)}{\partial\dy},
    \numberthis
    \label{eq:derivative_more_iid_training} \\
    \EntInnC{\txA}{\rxA}{^+}{} &=\left.\frac{\partial\EntR(\rxA_\dy|\txA_\dy)}{\partial\dy}\right|_{\dy=1}.
    \numberthis
    \label{eq:derivative_training_0}
\end{align*}


(c) If
\begin{align}
    \lim_{\dy\downto 1}\EntInnC{\txA}{\rxA}{}{\dy}=\lim_{\dx\upto 1}\EntInnC{\txA}{\rxA}{\dx}{},
    \label{eq:two_bound_met}
\end{align}
then Assumption A2 \eqref{eq:continue_in_data} is met.
\end{cor}

\begin{proof} (a) Under A1, for all $\dx\geq 1$, we have
\begin{align*}
    \Ent(\rxv_{t+1}|\txm_{\ceil{\dx\Tt}},&\rxm_{t})\leq\Ent(\rxv_{t+1}|\txm_{\ceil{\dx\Tt}},\rxm_{t-1})\\
    =&\;\Ent(\rxv_{t}|\txm_{\ceil{\dx\Tt}},\rxm_{t-1}),
    \numberthis
    \label{eq:monotonic_A1}
\end{align*}
when $\ceil{\Tt}+1\leq t\leq \ceil{\dx\Tt}-1$ or $t\geq \ceil{\dx\Tt}+1$. Here, we use that the input is \iid and the system is memoryless and time invariant; the inequality follows from the fact that conditioning reduces entropy.  
Therefore, $\forall \kappa\in(0,\dy-1)$, $\Ent(\rxv_{t+1}|\txm_{\ceil{\Tt}},\rxm_{t})$ is monotonically decreasing in $t$ for $t\in[\floor{(\dy-\kappa)\Tt},\ceil{(\dy+\kappa)\Tt}]$ when $\Tt>\frac{2}{\dy-1-\kappa}$. Then, Theorem \ref{thm:derivative_monotonic} yields \eqref{eq:derivative_data}.

Also, $\forall \kappa\in(0,\dy-1),\dx>2\dy-1$, $\Ent(\rxm_{t+1}|\txm_{\ceil{\dx\Tt}},\rxm_{t})$ is monotonically decreasing in $t$ for $t\in[\floor{(\dy-\kappa)\Tt},\ceil{(\dy+\kappa)\Tt}]$ when $\Tt>\max(\frac{2}{\dy-1-\kappa},\frac{2}{\dx-\dy-\kappa})$. Then, Theorem \ref{thm:derivative_monotonic} yields
\begin{equation}
    \EntInnC{\txA}{\rxA}{\dx}{\dy} =\frac{\partial\EntR(\rxA_\dy|\txA_{\dx})}{\partial\dy}.
    \label{eq:derivative_1}
\end{equation}
Assumption A1 yields
\begin{align}
    \EntInnC{\txA}{\rxA}{\dx}{\dy}=\EntInnC{\txA}{\rxA}{\dy^+}{\dy},
    \label{eq:H_prime_equ}
\end{align}
where $\EntInnC{\txA}{\rxA}{\dy^+}{\dy}$ is defined in \eqref{eq:EntInn_cond_def_one_extra_high}, and 
\begin{align}
    \EntR(\rxA_\dy|\txA_{\dx})=\EntR(\rxA_\dy|\txA_{\dy}).
    \label{eq:H_equ}
\end{align}
Therefore, \eqref{eq:derivative_1} becomes \eqref{eq:derivative_training}.

(b) If $\txv_t$ are \iid for all $t$, then \eqref{eq:monotonic_A1} is valid for all $t\leq \ceil{\dx\Tt}-1$ or $t\geq \ceil{\dx\Tt}+1$. Therefore, Theorem \ref{thm:derivative_monotonic} yields \eqref{eq:derivative_more_iid_training}. By taking $\dy=1$ and $\dx>1$, \eqref{eq:derivative_more_iid_training}, \eqref{eq:H_prime_equ}, and \eqref{eq:H_equ} then yield \eqref{eq:derivative_training_0}.

(c) For $t\geq\ceil{\Tt}+1$, we have
\begin{align*}
  \Ent(\rxv_{t+1}|\txm_{\ceil{\Tt}},\rxm_{t}) \leq\Ent(\rxv_t|\txm_{\ceil{\Tt}},\rxm_{t-1}).
%    \numberthis
%    \label{eq:monotonic_condition_2}
\end{align*}
Therefore,
\begin{equation*}
   \lim_{\dy\downto 1}\EntInnC{\txA}{\rxA}{}{\dy}\leq \EntInnC{\txA}{\rxA}{}{}.
\end{equation*}
Conditioning to reduce entropy again yields
\begin{align*}
  \Ent(\rxv_{\ceil{\Tt}+1}|\txm_{\ceil{\Tt}},\rxm_{\ceil{\Tt}}) \leq&\; \Ent(\rxv_{\ceil{\Tt}+1}|\txm_{\ceil{\dx\Tt}},\rxm_{\ceil{\Tt}}),
\end{align*}
for any $\dx<1$ and therefore,
\begin{equation*}
    \EntInnC{\txA}{\rxA}{}{}\leq \lim_{\dx\upto 1}\EntInnC{\txA}{\rxA}{\dx}{}.
\end{equation*}
Equation \eqref{eq:two_bound_met} then implies A2 \eqref{eq:continue_in_data}.
\end{proof}

Corollary \ref{cor:A1_derivative_relation} can now be used to finish the proof of Theorem \ref{thm:MI_equal_gap&computation_of_MuI_from_derivative}. By \eqref{eq:cI_def_with_P_tau_high}, and Assumptions A1--2
\begin{equation*}
    \cIInn{\txA}{\rxA}{} =\EntInnC{\txA}{\rxA}{}{} - \EntInnC{\txA}{\rxA}{^+}{} = \lim_{\dy\downto 1} \EntInnC{\txA}{\rxA}{}{\dy} - \lim_{\dy\downto 1} \EntInnC{\txA}{\rxA}{\dy^+}{\dy},
%    \label{eq:Ixy_final}
\end{equation*}
and together with Corollary \ref{cor:A1_derivative_relation}(a), we have \eqref{eq:MuI_from_derivative}.

\comm{
\section{Examples for applying Theorem \ref{thm:derivative_monotonic}}
\label{app:examples_for_derivative_monotonic}
In this section we first show, through two examples, how to apply Theorem \ref{thm:derivative_monotonic} where the conditions are met; and another two examples where the conditions are not met but still can be accommodated by proper extension of the definitions.  In this section, $\rxA=(\rx_{1},\rx_{2},\ldots)$ where $\rx_t$ is a scalar and $\rxv_{t}=[\rx_{1},\cdots,\rx_{t}]^\Tp$.

\subsection{Processes where $\EntInn(\rxA_\dy)$ is the derivative of $\EntR(\rxA_\dy)$ }
Two examples are given.

\noindent
{\em Example 1:} Let $\rxA$ be a stationary process where the joint distribution of any subset of the sequence of random variables is invariant with respect to shifts in the time index \cite{cover2012elements}, and where
\begin{align*}
    \lim_{\Tlim\to\infty}\Ent(\rx_{\Tt+1}|\rxv_{\Tt})=\EntR(\rxA)=\lim_{\Tlim\to\infty}\frac{1}{\Tt}\Ent(\rxv_T),
\end{align*}
where $\EntR(\rxA)$ is called the ``entropy rate" of $\rxA$. For all $\dy>0$, we have
\begin{align*}
\EntR(\rxA_\dy)=\lim_{\Tlim\to\infty}\frac{1}{\Tt}\Ent(\rxv_{{\dy \Tt}})=\dy\EntR(\rxA),
\end{align*}
and
\begin{equation}
\EntInn(\rxA_\dy)  =  \lim_{\Tlim\to\infty}\Ent(\rx_{{\dy\Tt}+1}|\rxv_{{\dy\Tt}})=\EntR(\rxA).
\end{equation}
  Because of stationarity, $\Ent(\rx_{t+1}|\rxv_t)$ is monotonically decreasing in $t$, and we see that $\EntInn(\rxA_\dy)$ is the derivative of $\EntR(\rxA_\dy)$, as expected.
  
However, we are generally interested in non-stationary processes, and the next example is a simple example containing a ``phase change" at $t=\Tt$.

\noindent
{\em Example 2:} Let
\begin{equation}
    \rx_{t} = \begin{cases} 
      b_t, & t=1,\ldots,\Tt; \\
      b_{(t-1\bmod\Tt)+1}, & t=\Tt+1,\Tt+2\ldots,
      \end{cases}
      \label{eq:repetition_eg}
\end{equation}
where $b_{1},b_{2},\ldots$ are \iid with entropy 1. There are two phases in $\rxA$: the first phase contains \iid elements, while the second phase contains repetitions of the first.  Clearly
\begin{equation}
    \EntInnY{\rxA}{\dy}  = \begin{cases} 
      1, & \dy \in[0,1); \\
      0, & \dy \geq 1,
      \end{cases}
\end{equation}
and $\Ent(\rx_{t+1}|\rxv_{t})$ is bounded by 1 and is monotonic for all $t$.
Theorem \ref{thm:derivative_monotonic} (or inspection) yields
\begin{align*}
    \EntR(\rxA_\dy)= \int_{0}^{\dy}\EntInnY{\rxA}{u} du=\begin{cases} 
      \dy, & \dy<1; \\
      1, & \dy\geq 1.
      \end{cases}
\end{align*}
Note that $\EntR(\rxA_\dy)$ is differentiable everywhere but $\dy=1$.  This point will reappear later.

%Both Theorems \ref{thm:derivative_monotonic} and \ref{thm:integral_thm_monotonic} can be generalized to include conditioning on another process $\txA$. 

%\subsection{Processes where $\EntInnC{\txA}{\rxA}{}{\dy}$ is not the derivative of $\EntR(\rxA_\dy|\txA)$}
\subsection{Processes where $\EntInn(\rxA_\dy)$ is not the derivative of $\EntR(\rxA_\dy)$}
\label{sec:general_scalar_process}
%The previous section establish sufficient conditions where the derivative relationship between $\EntInnY{\rxA}{\gamma}$ and $\EntR(\rxA_\gamma)$ holds, and we focus on triangular processes whose conditional entropy $\Ent(\rx_{t+1}|\rxv_{t})$ is monotonic and bounded; see conditions in Theorems \ref{thm:derivative_monotonic} and \ref{thm:integral_thm_monotonic}. 
We examine, through two examples, what can go wrong when the conditions of Theorem \ref{thm:derivative_monotonic}  are not met.

\subsection*{Example 3: $\Ent(\rx_{t+1}|\rxv_{t})$ oscillates as $t$ increases}
Consider the process
\begin{equation}
    \rxA= (b_1,b_1,b_2,b_2,\ldots,b_{k},b_{k},\ldots),
    \label{eq:oscillation_eg_def}
\end{equation}
where $b_1,b_2,\ldots$ are \iid unit-entropy random variables. Then, for all $t\geq \Tt$,
\begin{equation*}
    \Ent(\rx_{t+1}|\rxv_{t}) = \begin{cases}
      1, \quad  t \quad\text{even} \\
      0, \quad t \quad\text{odd }
   \end{cases}
\end{equation*}
and $\EntInn(\rxA_\dy)$ does not exist for any $\dy$. However, $\EntR(\rxA_\dy)$ exists for all $\dy>0$ with
\begin{align}
    \EntR(\rxA_\dy)=\lim_{\Tlim\to\infty}\frac{1}{\Tt}\frac{{\dy \Tt}}{2}=\frac{\dy}{2},
    \label{eq:osc_eg_EntR}
\end{align}
which is differentiable for all $\dy>0$. The conditions for Theorem~\ref{thm:derivative_monotonic} are not met and the derivative relationship \eqref{eq:derivative_2} does not hold.

\subsection*{Example 4: $\Ent(\rx_{t+1}|\rxv_{t})$ is unbounded}
Consider a process $\rxA$ with independent elements whose entropies are
\begin{equation}
    \Ent(\rx_t) = \begin{cases}
      \Tt, & t=\frac{1}{2}\Tt-3\\
      1, & \text{otherwise}
   \end{cases}
   \label{eq:unbounded_eg}
\end{equation}
It is clear that $\Ent(\rx_t|\rxv_{t-1})$ is unbounded at $t=\frac{1}{2}\Tt-3$. Both $\EntInn(\rxA_\dy)$ and $\EntR(\rxA_\dy)$ exist with
\begin{align*}
    \EntInn(\rxA_\dy) = 1,\quad\dy\geq 0,
\end{align*}
\begin{equation}
    \EntR(\rxA_\dy) = \begin{cases}
      \dy, & \dy\in(0,\frac{1}{2}); \\
      1+\dy, & \dy\geq \frac{1}{2},
   \end{cases}
\end{equation}
but the conditions for Theorem \ref{thm:derivative_monotonic} are not met and the integral relationship \eqref{eq:theorem_integral_monotonic_thm} does not hold everywhere.

Nonetheless, these examples can still be accommodated by expanding the definition of $\EntInn(\rxA_\dy)$.
In Example 3, $\EntInn(\rxA_\dy)$ is not a good representative of $\Ent(\rx_{t+1}|\rxv_{t})$ when $t={\dy\Tt}$ because of its oscillatory behavior. A better representative of $\Ent(\rx_{t+1}|\rxv_{t})$ when $t={\dy\Tt}$ can be found by averaging:
\begin{equation*}
    \AEntInnY{\rxA}{\dy} =\lim_{\kappa\downto 0}\lim_{\Tlim\to\infty}\frac{\sum_{t={(\dy-\frac{\kappa}{2})\Tt}}^{{(\dy+\frac{\kappa}{2})\Tt}-1}\Ent(\rx_{t+1}|\rxv_{t})}{\kappa\Tt}.
\end{equation*}
With the new definition, for Example 3, 
\begin{align*}
    \AEntInnY{\rxA}{\dy} =\lim_{\kappa\downto 0}\lim_{\Tlim\to\infty}\frac{\kappa\Tt/2}{\kappa\Tt}=\frac{1}{2},
\end{align*}
which is the derivative of $\EntR(\rxA_\dy|\txA)$ shown in \eqref{eq:osc_eg_EntR}. Thus, averaging smooths out the oscillation and expands the class of processes for which Theorem \ref{thm:derivative_monotonic} holds.

In Example 4, $\AEntInnY{\rxA}{\dy}$ is unbounded at $\dy=\frac{1}{2}$.  By allowing an impulse function in $\AEntInnY{\rxA}{\dy}$ at $\frac{1}{2}$, we may then consider $\EntR(\rxA_\dy)$ as the integral of $\AEntInnY{\rxA}{\dy}$, thereby expanding the class of processes for which Theorem \ref{thm:derivative_monotonic} holds.  We do not pursue these issues any further.
}



\comm{
\begin{IEEEbiography}{Kang Gao}
(S'14) received the B.S. degree in electrical engineering from Huazhong University of Science and Technology, Wuhan, China in 2014, the M.S. and the Ph.D. degree in electrical engineering from the University of Notre Dame in 2017 and 2021, respectively.
\end{IEEEbiography}

\begin{IEEEbiography}{Bertrand M. Hochwald}
(S'90-M'95-SM'06-F'08) was born in New York, NY, USA. He received the bachelor’s degree from Swarthmore College, Swarthmore, PA, USA, the M.S. degree in electrical engineering from Duke University, Durham, NC, USA, and the M.A. degree in statistics, and the Ph.D. degree in electrical engineering from Yale University, New Haven, CT, USA.

From 1986 to 1989, he was with the Department of Defense, Fort Meade, MD, USA. He was a Research Associate and a Visiting Assistant Professor at the Coordinated Science Laboratory, University of Illinois at Urbana–Champaign, Urbana, IL, USA. In 1996, he joined the Mathematics of Communications Research Department, Bell Laboratories, Lucent Technologies, Murray Hill, NJ, USA, where he was a Distinguished Member of the Technical Staff.
In 2005, he joined Beceem Communications, Santa Clara, CA, USA, as the Chief Scientist and Vice-President of Systems Engineering. He served as a Consulting Professor of Electrical Engineering at Stanford University, Palo Alto, CA, USA. In 2011, he joined the University of Notre Dame, Notre Dame, IN, USA, as a Freimann Professor of Electrical Engineering.

Dr. Hochwald received several achievement awards while employed at the Department of Defense
and the Prize Teaching Fellowship at Yale University. He has served as an Editor of several
IEEE journals and has given plenary and invited talks on various aspects of signal processing
and communications. He has forty-six patents and has co-invented several well-known
multiple-antenna techniques, including a differential method, linear dispersion codes, and
multi-user vector perturbation methods. He received the 2006 Stephen O.\ Rice Prize for the
best paper published in the IEEE Transactions on Communications. He co-authored a paper that
won the 2016 Best Paper Award by a young author in the IEEE Transactions on Circuits and
Systems.  He also won the 2018 H.\ A.\ Wheeler Prize Paper Award from the IEEE Transactions on
Antennas and Propagation.  His PhD students have won various honors for their PhD research,
including the 2018 Paul Baran Young Scholar Award from the Marconi Society.  He is listed as a
Thomson Reuters Most Influential Scientific Mind in multiple years.  He is a Fellow of the
National Academy of Inventors.
\end{IEEEbiography}
}


\bibliographystyle{IEEEtran}
\bibliography{bib/IEEEarbv,bib/refs.bib}

\end{document}
