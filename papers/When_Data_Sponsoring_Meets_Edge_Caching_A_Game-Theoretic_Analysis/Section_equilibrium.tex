\section{Game Equilibrium Analysis}
We now analyze the game equilibrium by using backward induction in this Section.
\subsection{User Selection Game in Stage II}
	
\subsubsection{Analysis of the NE}
 
Based on the user payoff formulated in (\ref{equa:userpayoff}), we analyze user selection distribution to get insight into the NE. We assume that $\phi_1 = \phi_2 = \phi$, as the energy spent to keep the communication with the CP is the same. Furthermore, we define $\delta_1=(v-c_1)P$ and $\delta_2=(v-c_2)\rho$ as the instant payoff of delivery networks. Hence, the user payoffs under different memberships can be reformulated as:
\begin{equation}
\begin{array}{l}
V_{(f,r)}(m)=
\left\{\begin{array}{ll}
0, & m=N,\\
\delta_1f-\phi, &m= C,\\
\delta_2fr-\phi, & m=E.\\
(\delta_2-\delta_1\rho)fr+\delta_1f-2\phi, &m=H
\end{array}\right.
\end{array}
\end{equation}

We can derive the user's selection policy as below. The mobile user will choose:

\begin{itemize}
\item m=N, if and only if $f<\frac{\phi}{\delta_1}$, $rf<\frac{\phi}{\delta_2}$.\footnote{The third condition that $(\delta_2-\delta_1\rho)rf+\delta_1f-2\phi<0$ is covered by the former two conditions, since we have $(\delta_2-\delta_1\rho)rf+\delta_1f-2\phi<(\delta_1f-\phi)+(\delta_2fr-\phi)<0$ }
\item m=C, if and only if $f>\frac{\phi}{\delta_1}$, $\delta_1-\delta_2r>0$ and $\phi-(\delta_2-\delta_1\rho)fr>0$.
\item m=E, if and only if $fr>\frac{\phi}{\delta_2}$, $\delta_2r-\delta_1>0$ and $\phi+\delta_1\rho fr-\delta_1f>0$.
\item m=H, if and only if $\phi-(\delta_2-\delta_1\rho)fr<0$, $\phi+\delta_1\rho fr-\delta_1f<0$ and $(\delta_2-\delta_1\rho)fr+\delta_1f-2\phi>0$.
\end{itemize}

We note that only when $\delta_i \ge \phi,\quad i=\{1,2\}$ exists, the sponsor via different networks benefit mobile users. Once $\delta_1, \delta_2$ are determined under the NE, we can derive the payoffs of a type-$(f, r)$ user under different memberships.

To facilitate the later analysis, we introduce the concept of \emph{indifferent point} (on the user type). 

\begin{myDef} An indifferent point is such a type-$(f, r)$ on which users will achieve the same payoff when selecting different memberships.\end{myDef}

%The number of indifferent point will not exceed 2. Specifically, we can calculate these two indifferent points. According to the principles below:

As shown in Fig.~\ref{fig:udn}, there are two kinds of indifferent points: (i) the type-$(f, r)$ where users will achieve the same payoff under \textbf{NoSp}, \textbf{CellSp}, and \textbf{EdgeSp}, and (2) the type-$(f, r)$ where users will achieve the same payoff under \textbf{HybridSp}, \textbf{CellSp}, and \textbf{EdgeSp}. Specifically,

\begin{enumerate}
\item \textbf{NoSp}, \textbf{CellSp}, and \textbf{EdgeSp} are the optimal membership selection for type-$(f, r)$ users where
\begin{equation}
\left\{\begin{array}{ll}
\delta_1f-\phi=0\\
\delta_2fr-\phi=0
\end{array}\right.
\end{equation}

\item \textbf{CellSp}, \textbf{EdgeSp}, and \textbf{HybridSp} are the optimal membership selection for type-$(f, r)$ users where
\begin{equation}
\left\{\begin{array}{ll}
\delta_1f-\phi=\delta_2fr-\phi\\
\phi+\delta_1\rho fr-\delta_1f=0
\end{array}\right.
\end{equation}
\end{enumerate}

We denote these two points as $N_1$ and $N_2$, respectively, where $N_1=(f^*,r^*)=(\frac{\phi}{\delta_1},\frac{\delta_1}{\delta_2})$ and $N_2=(\frac{f^*}{1-\rho r^*}, r^*)$. Next we discuss the existences of $N_1$ and $N_2$. 
\begin{enumerate}
\item Users will select \textbf{CellSp} and \textbf{EdgeSp}, if and only if $N_1$ exists and $N_2$ does not exist.
\item Users will select \textbf{CellSp}, \textbf{EdgeSp} and \textbf{HybridSp}, if and only if when both $N_1$ and $N_2$ exist.

\end{enumerate}

The user divisions in these two types of NE are illustrated in Fig.~\ref{fig:udn}. We can see that users with large $f$ and $r$ prefer edge cache sponsoring, and users with large $f$ and small $r$ prefer cellular sponsoring.


 \begin{figure}[t]
  \centering
%  \begin{minipage}[t]{.45\linewidth}
     \includegraphics[width=.23\textwidth]{case2}
   \includegraphics[width=.23\textwidth]{case1}
%	 \caption{User distribution with Hybrid.}\label{fig:udn}	
%  \end{minipage}
%  \begin{minipage}[t]{.45\linewidth}

	 \caption{User Divisions in NE with ``Hybrid'' and without ``Hybrid''.}\label{fig:udn}	
%  \end{minipage}
\vspace{-4mm}
\end{figure}
 

\subsubsection{Membership Distribution Dynamics}

Obviously, under the user best membership selection, the newly derived membership distributions may be different from the initial one. We denote the newly derived membership distribution by $\{\Theta_N^\prime, \Theta_C^\prime, \Theta_E^\prime, \Theta_H^\prime\}$, and the associated membership percentages as $\{\mu_N^\prime, \mu_C^\prime, \mu_E^\prime, \mu_H^\prime\}$.

The membership distribution continues evolving over time, until it reaches a stable distribution (called \emph{membership selection equilibrium}), where no user has the incentive to change his choice. Now we study the membership distribution dynamics, and characterize the membership selection equilibrium in Stage II.

Suppose that each user selects membership once in each time slot. Without loss of generality,  we consider the membership distribution change in a generic time slot $t$. 
 Let  $\{\Theta_N^t, \Theta_C^t, \Theta_E^t, \Theta_H^t\}$ denote the initial membership distribution at the beginning of slot $t$, and  $\{\Theta_N^{t+1}, \Theta_C^{t+1}, \Theta_E^{t+1}, \Theta_H^{t+1}\}$ denote the  newly derived membership distribution in time slot $t$ (after one round best response update). An \emph{equilibrium} is characterized by the following proposition.

  \begin{myPro}
  A membership distribution $\{\Theta_N^t, \Theta_C^t, \Theta_E^t$, $\Theta_H^t\}$ is a membership selection equilibrium
  if and only if $$\Theta_N^{t+1}=\Theta_N^t, \Theta_C^{t+1}=\Theta_C^t, \Theta_E^{t+1}=\Theta_E^t, \Theta_H^{t+1}=\Theta_H^t.$$
  \end{myPro}

This implies that if $\{\Theta_N^t, \Theta_C^t, \Theta_E^t, \Theta_H^t\}$ is a membership selection equilibrium, then we will have: $\Theta_m^\tau=\Theta_m^t, \forall m\in \{N,C,E,H\}$, for all $\tau>t$.

\begin{myTheo}\label{theo:s2ue}
There exists unique pure-strategy Nash Equilibrium (NE) in Stage II.
\end{myTheo}

\iffalse
\subsubsection{Algorithm to achieve NE in Stage II}

We notice that the indifferent point is a key feature of the equilibrium in Stage II. We propose a method using the indifferent point. Without loss of generality, we assume that $f^t$ and $r^t$ are uniform distributions in $[0,1]$. We define the initial indifferent point as $N_1^0$, and the indifferent point in $t$-th iteration as $(f^t,r^t)$. We can compute $N_C^\prime$ by the distribution of user membership selection, \emph{i.e.,} \\
\begin{equation}
\begin{aligned}
N_C^\prime=U\big\{-\frac{f^2}{2}[1-(\frac{r}{1-r})^2]ln(\frac{1-r}{f})+\\\frac{1}{4}[1-(\frac{f}{1-r})^2]+\frac{rf^2}{2}[(\frac{1}{1-r})^2-1]\big\}. \end{aligned}
\end{equation}
Hence, we can derive the cellular sponsor probability $P^\prime$ by
\begin{equation}
P^\prime=\frac{\alpha_1}{N_C^\prime} \end{equation}
\iffalse
\begin{equation}
P_2^\prime=\frac{\alpha_2}{\frac{1}{4}-\frac{1}{4}f^2\frac{r}{1-r}^2+\frac{1}{2}(fr)^2ln(r)-\frac{1}{2}(\frac{fr}{1-r})^2ln(\frac{1-r}{f})}.
\end{equation}
\fi
Based on the above analysis, we can compute the payoff of users at the indifferent point. $$V_{(f^t,r^t)}(N)=0,$$ $$V_{(f^t,r^t)}(C)=f^tP^\prime(v-c_1)-\phi,$$ $$V_{(f^t,r^t)}(E)=r^tf^t\rho(v-c_2)-\phi.$$

With the definition of the indifferent point, we design algorithm \ref{algo:1} to find the indifferent point, hence derive the NE in Stage II. Specifically, we use an iterative method to find the point satisfying the unique property of the indifferent point, i.e., $V(E)=0, V(C)=0$.
    \begin{algorithm} \label{algo:1}
        \renewcommand{\algorithmicrequire}{\textbf{Initialization:}}
        \caption{User Selection Equilibrium}
        \begin{algorithmic}[1]
            \Require $f^0\in(0,1)$, $r^0\in(0,1)$, $\Delta$, $\delta$.

              \For{$t = 0 \to \infty $}{
                     Update Rule:

                       $\;$ if $V(E)>0, V(C)>0$, $f^{t+1} = f^t-\Delta$.

                       $\;$ if $V(E)<0, V(C)<0$, $f^{t+1} = f^t+\Delta$.

                       $\;$ if $V(E)<0, V(C)>0$, $r^{t+1} = r^t+\Delta$.

                       $\;$ if $V(E)>0, V(C)<0$, $r^{t+1} = r^t-\Delta$.


                     Exit Rule:

                       $\;$ if $|V(E)|+|V(C)|<\Delta$, $\Delta = \frac{\Delta}{2}$.

                       $\;$ if $|V(E)|+|V(C)|<\delta$, exit.}
        \end{algorithmic}
    \end{algorithm}
\fi

\subsection{CP Budget Decision in Stage I}
Now we analyze the optimal decision of the CP in Stage I, given the NE of Stage II. Such analysis will lead to the SPE of the entire two-stage game.

We design a numerical method to solve the problem as follows.
We can check that the CP's optimization problem is non-convex. Hence, it is difficult to obtain the closed form solution of the optimal budgets $(\alpha_1^*, \alpha_2^*)$. Fortunately, the problem is a two variable optimization problem with box constraint sets, and can be solved using numerical methods. 
In this work, we solve the optimal budgets for the CP in a sequential manner: 
First, solve the optimal cellular budget 
$\alpha_1^*(\alpha_2)$ under any $\alpha_2$ and the optimal
$\alpha_2^*(\alpha_1)$ under any $\alpha_1$ through one-dimensional search; 
Then,
\begin{myPro}
The CP's best strategy $(\alpha_1^*, \alpha_2^*)$ must occur at an intersection point of $\alpha_1^*(\alpha_2)$ and $\alpha_2^*(\alpha_1)$.
\end{myPro}

\iffalse
\subsubsection{Low-Complexity Algorithm}
We denote the CP's best cellular budget selection as a function of the budget $\alpha_2$, \emph{i.e.,} $\mathcal{A}^1(\alpha_2)$. We have
\begin{equation}
\mathcal{A}^1(\alpha_2)\in \arg \max_{\alpha_1 \in [\alpha_{\min},\alpha_{\max}]}U(\alpha_1,\alpha_2).
\end{equation}
Similarly, we have $\mathcal{A}^2(\alpha_1)$ as a function of the budget $\alpha_1$. We have
\begin{equation}
\mathcal{A}^2(\alpha_1)\in \arg \max_{\alpha_2 \in [\alpha_{\min},\alpha_{\max}]}U(\alpha_1,\alpha_2).
\end{equation}

\begin{myPro}\label{Pro:alg1}
Let $\Gamma(\alpha_1)=\mathcal{A}^1(\mathcal{A}^2(\alpha_1))-\alpha_1$, For any budget, we have $\Gamma(\alpha_1)<0$ if $ \alpha_1 > \alpha_1^*$, and $\Gamma(\alpha_1)>0$ if $\alpha_1 < \alpha_1^*$.
\end{myPro}

With Proposition 3, we propose an iterative search algorithm, that searches $\alpha_1^*$ in the region. In the $t$-th iteration, the algorithm updates $\alpha_1(t)$ by
$$\alpha_1(t+1)=[\alpha_1(t)+d(t)\delta(t)]_{\alpha_{\min}}^{\alpha_{\max}}$$ or

$$\alpha_1(t+1)=\alpha_1(t)+\Gamma(\alpha_1(t))$$

\begin{myTheo}\label{theo:s1p}
The algorithm converges to an SPE of our Stackelberg game Q-linearly.
\end{myTheo}
\fi
