\section{Introduction}
\label{sec:intro}

\subsection{Background and Motivations}

Nowadays, we are witnessing the explosive growth of global mobile data traffic.
According to Cisco \cite{cisco}, mobile video traffic accounts for a majority of the total mobile traffic (e.g., $60\%$ in 2016).
Due to the fast increase of video traffic, the increased data cost is becoming one of the major concerns for mobile users to watch videos \cite{add-1}.
This brings additional challenge (and also huge opportunity) for video content providers (CPs), as they need to consider not only the quality improvement of their offered video services as before, but also the cost reduction for the users who request their services (to attract more video users).
Due to the intensive competition of CPs, the second issue (i.e., reducing user cost) is becoming increasingly important today \cite{add-2,add-4}.

One effective way to reduce user cost is the so-called \emph{data sponsoring} \cite{sdp1,sdp2,sdp3}, which has been employed by many CPs worldwide.
The key idea is to allow CPs to \emph{subsidize} the users' cost of mobile video data, hence attract more mobile video users and traffic.
Data sponsoring creates a win-win situation for mobile users and CPs, that is,  mobile users benefit from the free access of video contents, and CPs benefit from the increased video users and traffic (through, for example, selling more built-in advertisements).
%In this case, the video data is delivered via cellular networks, and hence the data cost mainly contains the cellular data cost.
Thus, we refer to such a data sponsoring scheme as ``\emph{cellular data sponsoring}''.
As a real-world example, AT\&T announced its sponsored data program in January 2014 \cite{att}.

In the forthcoming 5G cellular networks, \emph{edge caching} is emerging as a promising technique to deliver videos with lower cost and higher quality \cite{sam,add-5}.
The key idea is to cache the popular video contents on edge networks (e.g., femtocell base stations and WiFi access points) in advance and deliver the cached contents to local video users directly via device-to-device connections (e.g., WiFi direct).
Obviously, with edge caching, mobile users can obtain video contents without incurring the cellular data cost.
In this sense, edge caching can be viewed as a new sponsorship scheme for mobile users.
We refer to such a new sponsoring scheme as ``\emph{edge cache sponsoring}''.
%Edge caching has the potential of alleviating backbone network burden, providing high-quality video, and reducing content delivery cost \cite{edgecaching}. Hence, it has been employed by some commercial companies.
As an example, Xunlei \cite{thunder}, one of the largest content delivery networks (CDNs) in China, has deployed WiFi APs with large storage capacity to deliver video contents for mobile users.

%Many existing works have studied the cellular data sponsoring \cite{sdp1,sdp2,sdp3,pricing} and the edge caching \cite{thunder, ec2} comprehensively.
%However, the existing works considered these two sponsoring schemes separately, without considering their mutual interactions.
In this work, we will study the mobile video data market with both sponsoring schemes (as in \cite{joint}).
We aim to understand how the newly introduced edge caching will affect the traditional cellular data sponsoring, and how the co-existence of edge caching and cellular data sponsoring will change the user behavior and the whole data market.

\iffalse
\begin{figure}[t]
	\centering
	 \includegraphics[width=0.3\textwidth]{beijing_map.eps}
	 \caption{System Model.}\label{fig:architecture}	
\end{figure}
\fi


\subsection{Solution and Contributions}

To concentrate on the mutual interaction of edge caching and cellular data sponsoring, we consider a simple model with a \emph{single CP}, who offers both the edge cache sponsoring and the cellular data sponsoring to mobile users.
As in the existing literature \cite{thunder,joint, ec2}, we assume that the cellular network is available in the whole area, while the edge network is only available in part of the area (e.g., hotspots) due to the limited distance of device-to-device transmission.

By covering the data cost for users with either edge cache sponsoring or cellular data sponsoring, the CP can attract more video users and traffic, and hence achieve certain  \emph{revenue gain} (e.g., via build-in advertisements).
When providing sponsoring for users, the CP needs some \emph{budget} for covering the cellular data cost (in cellular data sponsoring) or caching the video contents on edge network (in edge cache sponsoring), hence lead to certain \emph{revenue loss}.
Note that a higher budget for cellular data sponsoring implies that the CP will sponsor more video contents (for those users subscribing to the cellular data sponsoring), and a higher CP budget for edge cache sponsoring implies that the CP will cache more video contents on edge network (for those users subscribing to the edge cache sponsoring).
Clearly, a higher budget (effort) for a particular sponsoring scheme can attract more users to subscribe to it (hence bring more revenue gain for the CP), but will also introduce more loss to the CP to offer the sponsoring.
Thus, the CP needs to determine the budget for each sponsoring scheme carefully to balance the revenue gain and loss.

Given the budgets that the CP offers for both sponsoring schemes, mobile users will decide whether and which sponsoring scheme(s) they are going to choose, which lead to the following four different memberships:
\begin{itemize}
\item \textbf{NoSp}: Choosing neither sponsoring scheme;
%The user chooses to join neither sponsorings and watches videos with his own data quota.
\item \textbf{CellSp}: Choosing cellular data sponsoring;
%The user chooses to join Cellular Sponsor, and compete with other users with the same choice, due to the CP's budget constraint.
\item \textbf{EdgeSp}: Choosing edge cache sponsoring;
%The user chooses to join Edge Cache Sponsor, and get sponsored when requesting cached contents due to the budget constraint and his probability to hit the cached content.
\item \textbf{HybridSp}: Choosing both sponsoring scheme.
%The user chooses to join both Edge Cache and Cellular Sponsoring, and get Edge Cache Sponsoring when he hits the cache and compete with other users with Cellular sponsoring otherwise.
\end{itemize}

More specifically, when a user chooses CellSp, his request will be served by the cellular network, and the cellular data cost will be covered by the CP with a given probability (depending on the CP's budget for   cellular data sponsoring).
When a user chooses EdgeSp, his request will be served by the edge network if the edge cache is available (i.e., the user is within the edge network and meanwhile the requested contents have been cached on the edge network).
When a user chooses HybridSp, his video request will be served by the edge network if the edge cache is available, or otherwise, by the cellular network and the cellular data cost will be covered by the CP with a given probability.

Game theory \cite{gametheory} has been widely used in wireless networks (e.g., \cite{gao-1,gao-2,gao-3,gao-4,gao-5,gao-6}) for modeling and analyzing the competitive and cooperative interactions among different network entities.
We formulate the interactions of the CP and the users as a two-stage Stackelberg game \cite{gametheory}, where the CP acts as game \emph{leader} determining the budgets for both sponsoring schemes in Stage I, and the users act as game \emph{followers} deciding which membership they would like to choose. 
We analyze the sub-game perfect equilibrium (SPE) of the proposed game systematically.
In summary, the key contributions of this works are given below.
\begin{itemize}
\item \emph{Novel Model}:
    This work analyzes the scenario where both cellular data sponsoring and edge caching are provided to users simultaneously.
    Our model captures many important features of practical systems, such as the content popularity and user heterogeneity.

\item \emph{Game-Theoretic Analysis}:
    We formulate the problem as a Stackelberg game and provide a comprehensive game theoretic analysis.
    We prove the uniqueness and existence of equilibrium in Stage II, and employ numerical method to obtain equilibrium in Stage I.

\item \emph{Experiments and Insights}:
    We conduct extensive experiments to evaluate the system performance.
    The experiment results show that by introducing the edge caching, the CP can increase his revenue by $105\%$. Users with high probability to meet an edge device prefer the edge cache sponsoring. %More users will choose both the sponsor schemes when the user subscription cost is low.
\end{itemize}

The rest of this paper is organized as follows.
In Section II, we present the system model.
In Section III, we provide the game formulation.
In Section IV, we provide the game equilibrium analysis.
We provide simulation results in Section V, and conclude in Section VI.

