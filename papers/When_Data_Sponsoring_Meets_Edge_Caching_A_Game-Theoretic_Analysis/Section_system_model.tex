
\section{System Model}

\subsection{Network Model}
We consider a single CP providing video service to a set $\mathcal{U} = \{1,2,...,U\}$ of mobile users. The CP can deliver video contents to mobile users in two different ways:
\begin{itemize}
  \item \emph{Cellular Direct Delivery}: The video content (located in the remote server) will be delivered to the user through the cellular link directly;
  \item \emph{Edge Cache Delivery}: The video content is cached in the edge network in advance, and will be delivered to the (nearby) users through the local link.
\end{itemize}
Each user moves and requests video contents randomly according to his own pattern. Let $\mathcal{S} = \{1,2,...,S\}$ denote the set of all video contents.

To enable the edge cache delivery, the CP needs to cache the video contents in the edge network devices (at the corresponding locations) in advance. Thus, a video request can be delivered via the edge network only when the user is within the edge network and the requested content is cached.

\subsection{CP Model}\label{sec:cpmodel}
The CP's goal is to maximize his total sponsoring revenue, which depends on the revenue loss induced by the budget and the obtained revenue gain from the sponsoring. The CP optimizes his revenue via managing the budgets in sponsoring schemes. When a user's request is sponsored through cellular delivery, the CP will cover the cellular data cost for the user. We denote the cellular data cost of one request for the CP as $h_1$. The total cost spent on cellular sponsoring is proportional to the total data volume of cellular sponsored requests. We denote the cellular sponsor effort as $\alpha_1$, hence the total cost in cellular sponsoring is $h_1 \alpha_1$. The cost in edge cache delivery is the storage cost of cached contents. We denote $h_2$ as the edge cache cost for one content per time period. We denote the edge cache sponsor effort as $\alpha_2$, hence the total cost in edge cache sponsoring is $h_2 \alpha_2$.

Moreover, if a content is sponsored (either via cellular sponsoring or edge cache sponsoring), it will be delivered to the user with certain attachment (e.g., build-in advertisements) called \emph{value-added content}, hence can bring revenue gain for the CP. We define the average revenue gain of a request for the CP as $u$.

\subsection{User Model}
Considering the heterogeneity of users in mobility pattern and service request pattern, we define the probability that a user is covered by the edge network as $r_u \in [0,1]$, and the probability to request a content as $f_u \in [0,1]$. In reality, to allow the user to estimate his mobility probability and request probability, we assume that the CP will announce the related information at the beginning. Specifically, the CP will announce the locations to place the edge caching devices, and the content caching rule, e.g., the CP will cache the latest popular TV series and talk shows. After the announcement made by the CP, the users can estimate the above probabilities based on their mobility and request patterns.

In this work, we focus on the \emph{symmetric equilibrium} where users with the same type $(f, r)$ will always make the same membership choice. Moreover, we focus on the user \emph{pure-strategy behavior} where each user will choose a specific membership under a given network situation. Namely, a user makes his membership choice among \textbf{NoSp}, \textbf{CellSp}, \textbf{EdgeSp}, and \textbf{HybridSp}. For notation convenience, we denote the membership as $m \in \{N, C, E, H\}$, correspondingly.

We denote $\Theta_N$, $\Theta_C$, $\Theta_E$, and $\Theta_H$ as the sets of users choosing \textbf{NoSp}, \textbf{CellSp}, \textbf{EdgeSp}, and \textbf{HybridSp}. Hence, the percentages of $C$, $E$, and $H$ users are: $\mu_C=\frac{|\Theta_C|}{U}$, $\mu_E=\frac{|\Theta_E|}{U}$, $\mu_H=\frac{|\Theta_H|}{U}$.

 The \emph{payoff} of each user is the achieved benefit minus the incurred cost. For convenience, we denote the payoff of a type-$(f, r)$ user under membership $m$ as $V_{(f, r)}(m)$. The objective of the user is to choose the proper membership to maximize his payoff.

 \subsubsection{\textbf{NoSp}}
 A user choosing neither sponsoring scheme cannot connect to any sponsoring network. As a user choosing \textbf{NoSp} has no cost or revenue in the sponsoring network, we set the user payoff as zero: \begin{equation}V_{(f, r)}(N)=0.\end{equation}
	
 \subsubsection{\textbf{CellSp}}
 A user choosing the cellular data sponsoring can connect to the cellular sponsoring network at any locations for any contents. Due to the cellular budget constraint of the CP, the probability that a content request is sponsored via cellular network is $P$ (derived in Section \ref{sec:variables}). Then, the expected payoff (per time slot) of a type-$(f, r)$ user is
\begin{equation}V_{(f, r)}(C)=f\cdot P\cdot (v-c_1)-\phi_1,\end{equation}
where $v$ is the average benefit of a user request, $c_1$ is the average energy cost of a request, and $\phi_1$ is the cost of joining the \textbf{CellSp} (e.g., time-average energy cost). Note that we assume that the user benefit and the user cost are of the same unit.
	
 \subsubsection{\textbf{EdgeSp}}
A cache sponsored user can only connect to the sponsor network only when he is within the edge network and requests the cached video content \cite{mobilitypred}. If a user is sponsored from the edge network, the video is delivered from the edge device to the user via local network connection. Let $\rho$ denote the probability of a user requesting a cached content in a time slot (derived in Section \ref{sec:variables}). Then, the expected payoff (per time slot) of a type-$(f, r)$ user is:
 \begin{equation}V_{(f, r)}(E)=r\cdot f\cdot\rho \cdot (v-c_2)-\phi_2,\end{equation}
where $r$ is the user mobility factor, $v$ is the average value of a request for the user, $c_2$ is the average energy cost of a request via the edge network, and $\phi_2$ is the cost of joining the \textbf{EdgeSp}.

 \subsubsection{\textbf{HybridSp}}
A user can choose to join both the two sponsoring schemes. When he is within the cache region and hits the cached content, he can employ the edge cache delivery. Otherwise, he will compete for the cellular sponsoring. Hence, the expected payoff (per time slot) of a type-$(f, r)$ user is:
\begin{equation}V_{(f, r)}(H)=(f-r\cdot f\cdot\rho)\cdot P\cdot (v-c_1)+r\cdot f\cdot\rho \cdot (v-c_2)-\phi_1-\phi_2,\end{equation}

\iffalse
To avoid trivial scenarios, we further assume:
\begin{itemize}
\item $v-c_1>\phi$: that is, users can always benefit from per sponsored request. Also, the payoff a user achieves from per sponsored video is larger than the time average energy cost. This is important for encouraging users to join sponsoring plans. If this is violated, then there will be no data sponsor.
\end{itemize}
\fi

\section{Game Formulation}

Based on the system model above, we formulate the interaction between the CP and mobile users as a two-stage Stackelberg game.

In Stage I, the CP decides the sponsor budgets (efforts) of cellular and edge cache sponsoring: $\alpha_1$, $\alpha_2$, respectively, to maximize the expected profit. We employ the widely-adopted contracted sponsoring for cellular sponsoring like in \cite{pricing}. The CP will pay a fixed fee $\alpha_1 h_1$ at first. Thus the expected revenues (per time slot) of the cellular and edge cache sponsoring are
\begin{equation}U_C(\alpha_1, \alpha_2)=u\cdot P\cdot N_C  - \alpha_1 h_1,\end{equation}
\begin{equation}U_E(\alpha_1, \alpha_2)= u\cdot N_E - \alpha_2 h_2,\end{equation}
where $u$ is the CP revenue defined in Section \ref{sec:cpmodel}, $N_C$ and $N_E$ denote the expected requests in cellular sponsoring and edge cache sponsoring, respectively, $P$  denotes the probability for users to get sponsoring in the cellular network. Hence we can define the sum CP revenue as
\begin{equation}U(\alpha_1, \alpha_2)=U_C(\alpha_1, \alpha_2)+U_E(\alpha_1, \alpha_2),\end{equation}
Keep in mind that $N_C$, $N_E$, and $P$ are all functions of $(\alpha_1, \alpha_2)$ derived in Stage II.

In Stage II, users determine their memberships (i.e., \textbf{NoSp}, \textbf{CellSp}, \textbf{EdgeSp}, or \textbf{HybridSp}), given the budgets $\alpha_1$ and $\alpha_2$ announced by the CP in Stage I. Note that users' decisions are coupled. For example, with more users choosing \textbf{CellSp}, the user payoff of cellular sponsor will decrease (due to a smaller sponsor probability $P$), resulting in the payoff gain in edge cache sponsor. Similarly, with more users choosing \textbf{EdgeSp}, the payoff of the cellular sponsoring will increase (due to a larger sponsor probability $P$).

Next, we will study the Subgame Perfect Equilibrium (SPE) of the proposed two-stage Stackelberg game. We show the definition of the SPE in Definition 1.

\begin{myDef}
	A strategy profile $\{(\alpha_1^*, \alpha_2^*),(m^*(f, r)$, $\forall r\in [0,1], f\in [0,1]\}$, where $(\alpha_1^*, \alpha_2^*)$ is the CP's strategy in Stage I and $m^*(f, r)$ is a type-$(f, r)$ user's strategy in Stage II, is an SPE if and only if:
\begin{equation}
\left\{
\begin{aligned}
Stage~II: & V_{(f, r)}(m^*(f, r)) \ge V_{(f, r)}(m), \\
& \forall r\in [0,1], f\in [0,1], m \in\{ C, E, H, N\};  \\
Stage~I: & U(\alpha_1^*,\alpha_2^*) \ge U(\alpha_1^\prime,\alpha_2^\prime),\\
& \forall \alpha_1 \in [\alpha_{\min},\alpha_{\max}], \alpha_2 \in [\alpha_{\min},\alpha_{\max}], \\
\end{aligned}
\right.
\end{equation}
 where $\alpha_1^\prime$, $\alpha_2^\prime$ are all the other budget selections, $\alpha_{\min}$, $\alpha_{\max}$ are the lower bound and upper bound of the sponsor budget, respectively. For fairness, we set the budget bounds of cellular sponsoring and edge cache sponsoring as the same.
\end{myDef}

We will derive the SPE by backward induction. Namely, we first study the user membership selection game in Stage II, and derive the users' equilibrium decisions. Then we characterize the optimal budgets of the CP to maximize his profit in Stage I.

\subsection{Stage II: User Membership Selection}
As discussed previously, the sets of users choosing \textbf{NoSp}, \textbf{CellSp}, \textbf{EdgeSp}, and \textbf{HybridSp} are already in the system (i.e., $\Phi_N$, $\Phi_C$, $\Phi_E$, $\Phi_H$) and their corresponding percentages (i.e., $\mu_N$, $\mu_C$, $\mu_E$ and $\mu_H$) will affect the value of $P$, and further affect the user payoff and membership selection. Hence, we will first study what is the user's best membership decision under a particular membership distribution $\{\Phi_N, \Phi_C, \Phi_E, \Phi_H\}$. Then we will study how the user membership decision dynamically evolves over time, and what is the stable membership distribution (called \emph{membership selection equilibrium}).

Given the CP's strategy $(\alpha_1, \alpha_2)$, and under a particular initial membership distribution $\{\Phi_N, \Phi_C, \Phi_E, \Phi_H\}$, the payoff of a type-$(f, r)$ user is:
\begin{equation}\label{equa:userpayoff}
\begin{array}{l}
V_{(f, r)}(m)=
\left\{\begin{array}{ll}
0, & m=N,\\
(v-c_1)Pf-\phi_1, &m= C,\\
(v-c_2)\rho fr-\phi_2, & m=E,\\
 (v-c_2)\rho fr+(v-c_1)Pf-\\
 ~~~~(v-c_1)P\rho fr-\phi_1-\phi_2, & m=H.
\end{array}\right.
\end{array}
\end{equation}

%A type-$(f, r)$ user will choose the membership with the maximal payoff.

A type-$(f, r)$ user will  choose (i)  \textbf{CellSp}, if and only if
\begin{equation}V_{(f, r)}(m=C)\ge \max\{0, V_{(f, r)}(E), V_{(f, r)}(H)\}, \end{equation}
(ii)   \textbf{EdgeSp}, if and only if
\begin{equation}V_{(f, r)}(m=E)\ge \max\{0, V_{(f, r)}(C), V_{(f, r)}(H)\}, \end{equation}
(iii)   \textbf{HybridSp}, if and only if
\begin{equation}V_{(f, r)}(m=H)\ge \max\{0, V_{(f, r)}(C), V_{(f, r)}(E)\}, \end{equation} 
and (iv) \textbf{HybridSp} if  the above conditions are not satisfied. 
% The users apart from the above conditions will choose .
	
\subsection{Stage I: CP Budget Selection}
	Given any CP's strategy $(\alpha_1, \alpha_2)$, the cellular sponsor probability $P^*(\alpha_1, \alpha_2)$, the number of user requests for cellular sponsor and edge cache sponsor $N^*_C(\alpha_1, \alpha_2)$ and $N^*_E(\alpha_1, \alpha_2)$ are achieved under the NE. Hence, we can formulate the CP payoff maximization problem as
	\begin{equation}
	\begin{aligned}
	\max_{\alpha_1, \alpha_2}X^*_C(\alpha_1, \alpha_2)+X^*_L(\alpha_1, \alpha_2)-\alpha_1 h_1-\alpha_2 h_2, \\
	%\max_{\alpha_2}X^*_L(\alpha_1, \alpha_2)-\alpha_2,
	\end{aligned}
	\end{equation}
	where $X^*_C(\alpha_1, \alpha_2)=u N^*_C(\alpha_1, \alpha_2)P^*(\alpha_1, \alpha_2)$
	and \\$X^*_L(\alpha_1, \alpha_2)= u N^*_E(\alpha_1, \alpha_2)$.

\iffalse
\begin{table}[!h] %\label{tab:notation1}
\tabcolsep 0pt \caption{Notations of Important Variables}
\vspace*{-12pt}\label{tab:notation1}
\begin{center}
\def\temptablewidth{0.47\textwidth}
{\rule{\temptablewidth}{0.8pt}}
\begin{tabular*}{\temptablewidth}{@{\extracolsep{\fill}}c | l}
        Notation & Meaning \\   \hline
        $U$ & The number of mobile users  \\
        $S$ & The number of contents \\
		$h_1$ & The average data cost per sponsor \\
		$h_2$ & The average caching cost of a content \\
        $\alpha_i\quad i=\{1,2\}$ & CP's budget in cellular and edge cache sponsor \\
		$u$ & The average CP revenue of a sponsor  \\
        $r_u$ & The probability a user is in the edge cache region\\
        $f_u$ & The probability a user request video contents \\
		$\Theta_j\quad j\in\{N,C,E,H\}$ & User set of membership\\
		$V_{(f, r)}(m)$ & User payoff under selection $m$ \\
		$v$ & The average user revenue of a sponsor\\
       % $F$ & The probability a user sends a request \\
		$\rho$ & The probability a user requests a cached content\\
		$P$ & The probability a user gets the cellular sponsor \\
		$c_i\quad i\in\{1,2\}$ & The per sponsor cost in network $i$\\
		$\phi_i\quad i\in\{1,2\}$ & time-average cost of a user in sponsor $i$\\
		$N_C$ & The expected requests in cellular sponsor \\
		$N_E$ & The expected requests in edge cache sponsor \\
     %   $s_l$ & The number of cached content \\
     %   $n_c$ & The maximum requests that cellular sponsor can support

\end{tabular*}
{\rule{\temptablewidth}{0.8pt}}
\end{center}
\end{table}
\fi

\subsection{Derivation of Important Variables}\label{sec:variables}
	Before studying the game equilibrium, we first derive the values of $\rho$, $N_C$, $N_E$, and $P$ analytically.
	
\subsubsection{\textbf{Calculation of $\rho$} (Cached Video Content Probability)}
 	We assume that the distribution of video requests follows Zipf distribution: $g(s) ,\;s\in \{1,2,...,S\}$. Given the budget $\alpha_2$ of edge cache sponsoring, we can derive the probability that a user requests a cached video content by:
	\begin{equation}\rho= \sum_1^{\alpha_2}g(s)\in [0,1].\end{equation}
Note that $\rho$ relies on the content number the CP decides to cache on each edge device.

\subsubsection{\textbf{Calculation of $N_C$ and $N_E$} (Expected Sponsoring Requests)}
	Given the users who choose \textbf{CellSp}, \textbf{EdgeSp}, and \textbf{HybridSp}, i.e.,  $\Theta_C$, $\Theta_E$, and $\Theta_H$, we can calculate the expected sponsoring request number:
	\begin{equation}
N_C=\sum_{u \in \Theta_C}f_u+\sum_{u \in \Theta_H}(f_u-r_u\cdot f_u\cdot\rho),\end{equation}
	\begin{equation}
N_E=\sum_{u \in \Theta_E\cup \Theta_H}r_u\cdot f_u\cdot\rho.~~~~~~~~~~~~~~~
\end{equation}

\subsubsection{\textbf{Calculation of $P$} (Sponsor Probabilities)}
	Given the budget $\alpha_1$ of cellular data sponsoring, the probability of users choosing \textbf{CellSp} or \textbf{HybridSp} can be computed as \begin{equation}P= \left\lceil \frac{\alpha_1}{N_C} \right\rceil^1,\end{equation}
where $\lceil x\rceil^1=\max(x,1)$.

