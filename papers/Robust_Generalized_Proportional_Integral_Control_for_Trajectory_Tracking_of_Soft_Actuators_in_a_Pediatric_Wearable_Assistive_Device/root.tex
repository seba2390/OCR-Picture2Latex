%%%%%%%%%%%%%%%%%%%%%%%%%%%%%%%%%%%%%%%%%%%%%%%%%%%%%%%%%%%%%%%%%%%%%%%%%%%%%%%%
%2345678901234567890123456789012345678901234567890123456789012345678901234567890
%        1         2         3         4         5         6         7         8

%\documentclass[letterpaper, 10 pt, conference]{ieeeconf}  % Comment this line out if you need a4paper

%\documentclass[letterpaper, 10 pt, journal, twoside]{ieeetran}



\documentclass[letterpaper, 10pt, conference]{ieeeconf}      % Use this line for a4 paper

\IEEEoverridecommandlockouts                              % This command is only needed if 
                                                          % you want to use the \thanks command

\overrideIEEEmargins                                      % Needed to meet printer requirements.

% See the \addtolength command later in the file to balance the column lengths
% on the last page of the document

% The following packages can be found on http:\\www.ctan.org
\usepackage{graphicx} % for pdf, bitmapped graphics files

\usepackage{epsfig} % for postscript graphics files
\usepackage{epstopdf}
\usepackage{tikz}
\usepackage{mathptmx} % assumes new font selection scheme installed
\usepackage{times} % assumes new font selection scheme installed
\usepackage{amsmath} % assumes amsmath package installed
\usepackage{amssymb}  % assumes amsmath package installed
\newcommand{\comment}[1]{{\color{green}{#1}}}
\newcommand{\tofix}[1]{{\color{blue}{#1}}}
\newcommand{\changed}[1]{{\color{red}{#1}}}
\usepackage{algorithm}
\usepackage{algorithmic}
\newtheorem{lemma}{Lemma}
\usepackage{tablefootnote}
\usepackage{color}
\usepackage{amsfonts}
\usepackage[noadjust]{cite}
\usepackage{subfigure}
\usepackage{threeparttable}
\usepackage[framed,numbered,autolinebreaks,useliterate]{mcode}
\usepackage{soul}
%\usepackage{tasks}
\usepackage[normalem]{ulem}
%USE FOR STATE MACHINE
\usepackage{pgf}
\usepackage{tikz}
\usetikzlibrary{arrows,automata}

\usepackage{booktabs}

\usepackage[a-1b]{pdfx}

\usepackage{hyperref} %for hyperlinks to videos: \href{link}{link label}

\usepackage{todonotes}

\def\kkb{\textcolor{blue}}
\def\kkr{\textcolor{red}}
\def\kkg{\textcolor{green}}
\def\kkm{\textcolor{magenta}}

\title{\LARGE \bf Robust Generalized Proportional Integral Control for Trajectory Tracking of Soft Actuators in a Pediatric Wearable Assistive Device} 
% Robust Generalized Proportional Integral Control for \\Trajectory Tracking of Soft Actuators toward Application to \\a Pediatric Wearable Robotic Assistive Device}
%(??)% %Supporting Two Degrees of Freedom Shoulder Motion}% for Infants with Upper Extremity Mobility Impairments}

\author{Caio Mucchiani,$^{1}$ Zhichao Liu,$^{1}$ Ipsita Sahin,$^{2}$ Elena Kokkoni,$^{2}$ Konstantinos Karydis$^{1}$% <-this % stops a space
\thanks{$^{1}$ Dept. of Electrical and Computer Eng. and $^{2}$ Dept. of Bioengineering; Univ. of California, Riverside, 900 University Avenue, Riverside, CA 92521, USA. Email: {\tt\footnotesize\{caiocesr, zliu157, isahi001, elenak, karydis\}@ucr.edu}. 
We gratefully acknowledge the support of NSF \# CMMI-2133084 and ARL \# W911NF-18-1-0266. 
Any opinions, findings, and conclusions or recommendations expressed in this material are those of the authors and do not necessarily reflect the views of the funding agencies.
}}


\begin{document}
\maketitle
%\thispagestyle{empty}
%\pagestyle{empty}


%%%%%%%%%%%%%%%%%%%%%%%%%%%%%%%%%%%%%%%%%%%%%%%%%%%%%%%%%%%%%%%%%%%%%%%%%%%%%%%%
\begin{abstract}
Soft robotics hold promise in the development of safe yet powered assistive wearable devices for infants. Key to this is the development of closed-loop controllers that can help regulate pneumatic pressure in the device's actuators in an effort to induce controlled motion at the user's limbs and be able to track different types of trajectories. This work develops a controller for soft pneumatic actuators aimed to power a pediatric soft wearable robotic device prototype for upper extremity motion assistance. The controller tracks desired trajectories for a system of soft pneumatic actuators supporting two-degree-of-freedom shoulder joint motion on an infant-sized engineered mannequin. The degrees of freedom assisted by the actuators are equivalent to shoulder motion (abduction/adduction and flexion/extension). Embedded inertial measurement unit sensors provide real-time joint feedback. Experimental data from performing reaching tasks using the engineered mannequin are obtained and compared against ground truth to evaluate the performance of the developed controller. Results reveal the proposed controller leads to accurate trajectory tracking performance across a variety of shoulder joint motions. 
\end{abstract}

%, which are intended to support infant motion during reaching tasks.

%when actuated both exclusively and simultaneously in assisting flexion/extension and abduction/adduction, respectively. 

%This work presents the design and control of a new soft pneumatically-actuated wearable device for infants, which supports two degrees of freedom equivalent shoulder motion (abduction/adduction and flexion/extension).
%This work presents the design and control of a new soft pneumatically-actuated wearable device for infants arm motion, which supports two degrees of freedom equivalent shoulder motion (abduction/adduction and flexion/extension)
%which is geared toward future application to promote movement in infancy. 
%
%Our goal is to assist the infant population affected by neurological disorders, such as cerebral palsy, although the same approach is applicable to children and older population. 

%Results demonstrate the proposed controller can help guide the mannequin's arm to the desired points.






%%%%%%%%%%%%%%%%%%%%%%%%%%%%%%%%%%%%%%%%%%%%%%%%%%%%%%%%%%%%%%%%%%%%%%%%%%%%%%%%

\section{Introduction}

Soft robotics technology has been increasingly integrated into wearable assistive devices (e.g.,~\cite{yumbla2021human}) owing to the former's inherent flexibility and adaptability, softness, and lower profile compared to rigid-only devices~\cite{o2017soft}. 
A major portion of previous work in this area include soft wearable assistive devices for upper extremity assistance and rehabilitation of adult populations (notable examples are~\cite{proietti2021sensing, o2020inflatable, park2017development}).
Despite a range of assistive devices proposed for these populations, upper extremity assistive devices for very young pediatric populations (i.e. $<$2 years of age) are limited~\cite{arnold2020exploring, christy2016technology} and mainly passive~\cite{lobo2016playskin, babik2016feasibility}.
The present work specifically focuses on the development of a soft robotic wearable device for upper extremity assistance in infants.

%Examples of previous works \changed{in adults} include soft wearable assistive devices for rehabilitation and assistance of the shoulder and elbow~\cite{proietti2021sensing}, muscle fatigue reduction of the shoulder \cite{park2017development} and upper extremity (UE) stroke rehabilitation~\cite{o2020inflatable}. 
%, all focusing on grown adults.%\kkb{Various assistive applications to humans can benefit from the development of these devices, such as prosthesis for rehabilitation, artificial organs, surgical devices and even drug delivery \cite{cianchetti2018biomedical}.}
%\kkr{need 1-2 more sentences here, some examples of assistive tasks achieved via soft robotics would fit the best.}

Several considerations should be taken into account when developing an assistive device for a young pediatric population; such as their anthropometric, movement, and learning characteristics~\cite{wininger2017geek}. 
For example, infants' kinematic parameters of motion (such as velocity profile of the hand) differ from those of older populations~\cite{konczak1997development,morange2019visual}. 
%Recent work takes these characteristics into consideration for the development of UE assistive devices for infants~\cite{trujillo2017development,dechemi2021babynet, kokkoni2020development}.
Soft wearable assistive devices, in particular, can afford a wide variety of methods for actuation, sensing and control, which in many cases are intertwined with each other and can be adjusted to meet the characteristics of the infant population. %and also facilitate motion while providing comfort and safety and not constraining the upper limbs. 
Actuation methods can utilize 3D-printing~\cite{yap2016high,schaffner20183d,hoang2021pneumatic}, casting~\cite{li2020high,kokkoni2020development,liu2022safely} or fabric~\cite{yap2017fully,kim2021compact,nassour2020high,fu2022textiles,sahin2022bidirectional} to improve strength~\cite{simpson2020upper,o2017soft} and minimize fatigue~\cite{nassour2020high,o2020inflatable}. 
Several of the current actuator designs employed in wearable devices are pneumatic-based (as in this work too), in an effort to facilitate motion while providing comfort and safety with minimal constraints on the arms~\cite{diteesawat2022soft}. 
%The present paper specifically targets UE assistance for supporting reaching tasks in very young pediatric populations.
%Several types of motor delays at infancy can cause significant adverse impact on lifelong mobility~\cite{adolph2019ecological,campos2000travel}, including of the upper extremities, such as congenital physical disabilities and neuromotor pathologies (such as brain injuries~\cite{christy2016technology}, cerebral palsy~\cite{redd2019development} or generalized hypotonia~\cite{jones1990emg}). 
%Since in infants (instrumental) activities of daily living are usually done by the parent, the reaching task is exploratory and provides learning experiences~\cite{muentener2018efficiency}. 
%Specifically, upper extremity (UE) motor skills (which are related to fine motor skills) involve reaching and manipulation tasks, and are typically acquired from birth to six months of age, with cognitive changes largely occurring the first two years of life~\cite{lobo2014not}. 
%Early intervention promoting and stimulating active exploration of the environment~\cite{ angulo2001exploration,blauw2005systematic} is directly linked to one's cognitive development~\cite{diamond2000close,herskind2015early,novak2017early}. 
%As such, extending the range of arm motion in reaching tasks for infants with upper extremity mobility impairments is desirable for better quality of life and essential for their development into adulthood. 
%Despite a range of assistive devices proposed in the literature, designing and controlling devices to promote motor exploration in infants is not analogous to scaling down devices engineered for older children or adults, as infants' kinematic parameters of motion \changed{(such as velocity profile of the hand)} differ from those of older populations~\cite{konczak1997development,trujillo2017development,dechemi2021babynet,addyman2017embodiment,morange2019visual}. 
%Designing devices for infants is not analogous to scaling down devices engineered for older children or adults, as infants' kinematic parameters of motion differ from older populations (such as velocity profile of the hand) and so studies focusing on characterizing infant motions~\cite{konczak1997development,trujillo2017development} including through action recognition~\cite{dechemi2021babynet}, visual exploration~\cite{morange2019visual} and interval timing~\cite{addyman2017embodiment} are needed. In infants, since Activities of Daily Living (ADLs) and Instrumental Activities of Daily Living (IADls) are usually done by the parent, the reaching task for them becomes more an exploratory and learning experience rather than direct functionality. 

Given a higher-level controller (e.g., admittance force control) %determines desired force profiles that can translate into 
yielding desired trajectories %to assist with arm motion, 
(e.g., as seen in other soft wearable assistive devices~\cite{zhou2021human}), 
%
a low-level controller operating at the pneumatic-system level needs to ensure the desired trajectory is achieved. 
%(or in some cases setpoints, e.g., to counter gravity at the shoulder joint~\cite{proietti2021sensing} is achieved. 
%
In our past work we have developed feedforward~\cite{kokkoni2020development,liu2021position} as well as setpoint~\cite{mucchiani2022closed} and learning-based feedback~\cite{shi2022online} controllers for soft pneumatic actuators used in various iterations of our wearable device prototypes~\cite{kokkoni2020development,sahin2022bidirectional}. Feedback controllers can help make the device more responsive, while tracking desired setpoints is a key function on its own, e.g., to counter gravity at the shoulder joint~\cite{proietti2021sensing}. However, to make soft wearable assistive devices for infant reaching more capable, it is important to be able to track complete trajectories too. %The latter is the focus of this present paper.
%

\begin{figure}[!t]
   \vspace{6pt}
   \centering
     \includegraphics[width=0.99\columnwidth]{MEDIA/intro.png}
     \vspace{-18pt}
      \caption{Snapshots of a combined joint angle trajectory motion of the assisted shoulder. The accompanying video offers more visual information and contains footage regarding the experiments conducted herein.}
      \label{fig:intro}
      \vspace{-6pt}
\end{figure}


In this work, we develop a low-level feedback robust generalized proportional integral (GPI) controller to track desired trajectories for a system of soft pneumatic actuators supporting shoulder joint motion on an infant-sized engineered mannequin (Fig.~\ref{fig:intro}). A GPI low-level controller is suitable in the context of this work as it can track trajectories while requiring less control effort compared to standard-of-practice PID controllers~\cite{blanco2022robust}. Further, it only requires knowledge of the mannequin's (and eventually user's) arm joint angles which can in principle reduce the computational load. GPI controllers have been used in conceptually-related works on exoskeletons for rehabilitation purposes of both upper~\cite{blanco2022robust} and lower~\cite{azcaray2018robust} extremities. The former considered the same controllable degrees of freedom (DoFs) as we do herein for the shoulder joint; however, rigid, rather than soft actuators were used, and the study did not specifically aim at pediatric populations. 
%
%The shoulder is the most movable joint in the human body \cite{biasca1996assessment}. As such, it demands higher DoF assistance in case of mobility impairments. 
%
Our approach critically focuses on a system of soft pneumatic actuators tailored for pediatric populations. Our engineered mannequin is designed and fabricated in-house based on average 12-month-old infant anthropometrics~\cite{Fryar2021,edmond2020normal}. 
Experimental testing and evaluation of the performance of the proposed low-level controller off-body using a mannequin is a critical and necessary step prior to being able to test with human subjects. 



%Understanding design and sensing methods for soft actuators~\cite{tapia2020makesense} and design considerations for wearables in infants~\cite{trujillo2017development} is also essential.



%\changed{The shoulder is the most movable joint in the human body \cite{biasca1996assessment}. As such, it demands higher degree-of-freedom (DoF) assistance in case of mobility impairment. Previous work related to shoulder motion assistance using wearable devices include soft~\cite{natividad20202,o2017soft} and rigid~\cite{scheuner2016patient,guzman2018improving,lopez2020development} wearable actuators; see~\cite{majidi2021review} for a comprehensive review of both approaches. Despite numerous works on adult rehabilitation considering shoulder motion, only a few have considered assisting UE motion in infants \cite{kokkoni2020development,sahin2022bidirectional,mucchiani2022closed,dechemi2021babynet,babik2016feasibility,lobo2016playskin}.
%}



%Low-level control techniques which can translate higher-level control generated trajectories are an important component in making soft wearable devices assistive.
%GPI controllers have been used in exoskeletons for rehabilitation purposes of both lower~\cite{azcaray2018robust} and upper~\cite{blanco2022robust} extremities. The latter considered the same controllable DoFs as we do herein for the shoulder joint; however, rigid, rather than soft actuators were used, and the study did not specifically aim at infant populations. As investigated in \cite{blanco2022robust}, our choice for a GPI controller, rather than a traditional PID, considers the ability of the GPI controller to track trajectories while comparatively requiring less control effort, highly desirable on a wearable device. Additionally, the controller only requires knowledge of angular position, and therefore more affordable and less computationally intensive and useful.

Our overall system contributes to on-board sensor-based pneumatic feedback control for soft robotics with application to wearable assistive devices. We consider two types of fabric-based soft pneumatic actuators for supporting up to two-DoF-equivalent shoulder motion (abduction/adduction--AB/AD and flexion/extension--F/E). 
Joint angle variations of the mannequin in response to actuator inflation/deflation are estimated in real time via proprioceptive feedback from inertial measurement units (IMUs). Then, given a desired trajectory in the joint space (relevant to the mannequin), our developed GPI controller regulates the pneumatic actuation Pulse-Width-Modulation (PWM) values to track the desired trajectory. 
%
Our proposed controller is shown to track correctly interpolated trajectories for both AB/AD and F/E shoulder motion separately, as well as simultaneously. In addition, to evaluate system and controller robustness under expected motion primitives common to infants, periodic trajectories in the form of harmonic signals and teach-and-repeat trajectories are also tested and validated experimentally. 

   
%\kkr{Our overall system comprises the mannequin and two types of fabric-based soft pneumatic actuators ...
%
%support of up two DoFs equivalent shoulder motion (abduction/adduction--AB/AD and flexion/extension--F/E) for pediatric shoulder assistance in reaching motions. ...
%
%The paper develops a closed-loop robust generalized proportional integral (GPI) controller based on proprioceptive feedback (Fig. \ref{fig:intro}). %%Compared to our previous related works \cite{kokkoni2020development,sahin2022bidirectional,mucchiani2022closed} this paper innovates both on design and control. Specifically, 
%We actively actuate a second DoF in the shoulder joint via a novel bellow type soft actuator design presented herein, and propose a closed-loop control method.  We estimate the relationship between pneumatic actuation Pulse-Width-Modulation (PWM) values and joint angle variation and the controller acts in the sense of tracking trajectories in the joint space. Through hardware implementation and experimentation utilizing real time proprioceptive feedback with Inertial Measurement Units (IMUs) and Motion Capture (Mocap) ground truth data, the proposed controller is shown to track correctly interpolated trajectories for both AB/AD and F/E shoulder motion separately, as well as simultaneously. In addition, to evaluate the device and controller robustness under expected motion primitives common to infants, periodic trajectories by sine waves and teach and repeat trajectories were also evaluated. %Therefore, considering the importance of assisting shoulder motion in reaching tasks, this work's main contributions can be summarized as follows.}

Succinctly, the paper's contributions are: 
\begin{itemize}
    \item Development of a GPI controller for joint angle trajectory tracking of soft pneumatic actuators with embedded proprioceptive sensing.
    \item Experimentation with a system of two soft actuators supporting shoulder mobility, performing trajectories aimed at motion assistance in infant reaching tasks.
\end{itemize}

To the best of our knowledge, no previous work in infant reaching has considered two DoF shoulder assistance with general motion of the limbs, neither using soft actuators (specifically of the bellow type) nor applying GPI control methods to track desired arm trajectories. 

\section{System Description}
\label{sec:problem}

\subsection{Hardware Design and Integration}
The overall hardware system we have designed in this work comprises the actuation subsystem (soft actuators and pneumatic control board), the mannequin where the actuators are mounted on, and the sensing and computing units required for control design in practice. The different components are highlighted in Fig.~\ref{fig:comp}. 

\begin{figure}[!t]
\vspace{6pt}
\centering
\includegraphics[width=0.99\linewidth]{MEDIA/aa.png}
\vspace{-18pt}
\caption{Hardware with embedded IMUs (left) and experimental setup (right). Different key components are highlighted. The two IMUs are placed so that the supported shoulder abduction/adduction (AB/AD) and flexion/extension (F/E) joint angles can be readily computed.}
\label{fig:comp}
\vspace{0pt}
\end{figure}

\begin{figure}[!t]
\vspace{-3pt}
\includegraphics[width=\linewidth]{MEDIA/acr.png}
\vspace{-21pt}
\caption{We employ two types of fabric-based soft pneumatic actuators designed in-house, with the aim to support (a) shoulder abduction/adduction and (b) shoulder flexion/extension.} \label{fig:act}
\vspace{0pt}
\label{fig:sac}
\end{figure}

\begin{figure}[!t]
\vspace{-3pt}
\centering 
\includegraphics[width=\columnwidth]{MEDIA/hg1.png}
\vspace{-25pt}
\caption{Left: The pneumatic control board designed in-house to operate the system's soft actuators.
Right: Pumps and solenoid valves diagram.}
\label{fig:pneu}
\vspace{-12pt}
\end{figure}

The system features two soft pneumatic actuators made of flexible thermoplastic polyurethane (TPU) coated nylon fabric (Fig.~\ref{fig:sac}) that provides support for two DoF motion assistance of the shoulder (AB/AD and F/E).\footnote{~Here we consider analysis of single-arm motion. The exact same actuation system can be readily replicated for the other arm support.} 
The actuators are independently-controlled and operated via a custom-made pneumatic control board (Fig.~\ref{fig:pneu}) placed at a distance from the mannequin.\footnote{~The selected distance here was selected so as to match the one the board will be placed at during future testing with human subjects (e.g., at the back of a chair the infant will be seated at) for safety purposes.} 
%\footnote{~Recall that our end goal in future research is to test the system with infants. For safety purposes, the pneumatic control board needs to be placed away from them (e.g., at the back of a chair they will be seated at during reaching tasks experimentation) to reduce risk of bodily injury in case of a fault. However, the length of the pneumatic line between the source (the pneumatic board) and the target (the soft actuators) directly affects control performance (most notably it can cause time-delayed signals). Here we have placed the control board at a distance similar to the one in future human subjects experiments for consistency.} 
The actuators are anchored on the mannequin via velcro straps at places that minimize motion restrictions as per our previous works~\cite{kokkoni2020development,mucchiani2022closed,sahin2022bidirectional}.\footnote{~Ongoing research aims to integrate the actuators onto a full wearable suit and study the effect of anchoring and placement on arm motion.} 
%The current prototype consists on pneumatically-powered textile actuators for a single arm, whereas 

The mannequin is designed and fabricated in-house based on average 12-month-old infant anthropometrics~\cite{Fryar2021,edmond2020normal}; the upper arm has a length of $l_a=0.14$\;m. The joints and upper-arm and forearm are 3D-printed. The upper-arm and forearm, in particular, are easily replaceable and hollow (left panel of Fig.~\ref{fig:support}).
%its articulated arm weighs $m_a=0.18$\;kg and 
We use a mix of different-sized beads to fill up the volume to achieve a desired weight so that density remains uniform and inertia/center of mass does not vary during operation. This design consideration allows us to rapidly vary the length and weight of the arm for testing different actuator designs and control parameters. In this work, the forearm of the mannequin is fixed at full extension so as to isolate and study shoulder joint motion alone.\footnote{~We refer the interested reader to~\cite{kokkoni2020development,mucchiani2022closed} for a study of the combined shoulder adduction/abduction and shoulder flexion/extension motion.}

Shoulder AB/AD motion (angle $\theta_1$) employs a recent design that has been extensively tested and validated in our previous work (Fig.~\ref{fig:act} a)~\cite{sahin2022bidirectional}, but a new actuator for shoulder F/E motion (angle $\theta_2$) was developed herein (Fig.~\ref{fig:act} b). 
Specifically, we consider a fabric-based bellow-type design actuator 
fabricated in a modular fashion, with interconnected square segments (pouches). Airflow through the inlet channel causes all pouches to simultaneously inflate; their total number determine the range of the actuator. 
Although proper characterization of the actuator is left as future work, based on empirical testing considering inflation time and range of extension, a total of $n=5$ pouches were selected herein. 
The soft pneumatic actuators employed herein can exert sufficient forces while maintaining a paper-thin profile. 
These are desirable characteristics when designing wearable devices that help minimize weight and impose less motion constraints. 
However, at the current design, F/E motion support via the bellow-type design requires a supporting element at one of its ends to allow the motion to occur (see the red arrow at the middle panel in Fig.~\ref{fig:support}). 

%In the context of infants and how wearables are designed, one can consider such supporting element as part of a costume to make the device more aesthetically-pleasing, such as wings for an angel figure, or a cape for a superhero type of design. 


   \begin{figure}[!t]
   \vspace{6pt}
   \centering
     \includegraphics[width=\columnwidth]{MEDIA/weas.png}
      %\includegraphics[scale=1.0]{figurefile}
      \vspace{-18pt}
      \caption{(Left) Our engineered mannequin built in-house integrates 3D-printed joints and links (upper-arm and forearm) designed based on anthropometric data. The links are adjustable (easily swappable and hollow to be filled in with additive weight) to make the mannequin easily reconfigurable. 
      (Center) Embedded supporting structure to mount one end of the shoulder flexion/extension actuator. (Right) The various coordinate systems employed in kinematics analysis and control of the overall system.}
      \label{fig:support}
      \vspace{-12pt}
   \end{figure}



%\subsection{Hardware}
%Various physical components used in experimental implementation are depicted in Figs.~\ref{fig:comp} and~\ref{fig:pneu}. 
Two \textit{witmotion\footnote{~https://www.wit-motion.com/}} IMUs are attached at the upper-arm and forearm to estimate shoulder joint angles $\theta_1$ and $\theta_2$. A Nvidia Jetson Nano serves as the main computer for the overall system, running our developed controller and sending commands to the pneumatic board through two microcontrollers via serial communication. All the system runs on a Robot Operating System (ROS) framework we developed to combine the controller and sensor (IMU) readings along with actuation from the pneumatic control board.



% \subsection{Experiments}
%\subsection{Software Integration}

%The overall software integration is shown in Fig.~\ref{fig:overall}. First, experimental data from separate joint motions were collected, and using MATLAB, the transfer function ($G_i(s)$) between the input (PWM), which regulates air flow to the actuators, and angular motion of each joint $\theta_i$ was estimated. We also developed a ROS framework to combine the controller and sensor (IMU) and Mocap readings, along with actuation from the pneumatic board.

\subsection{Desired Motion Characteristics}

%We address the problem of tracking a desired joint trajectory $\theta_d(t)$, with angle $\theta_1$ assisting AB/AD motion and $\theta_2$ with F/E, and both assisting in general motion simultaneously. 
The range of motion and associated kinematic constraints for the overall system are shown in Fig.~\ref{fig:bellow} and Table~\ref{table:constr}), respectively. At its current 5-pouch design, the bellow-type actuator assisting shoulder F/E motion can achieve a maximum angle variation of $\Delta\theta_2 = 0.3840 \; rad \;(22^o)$. %Despite lack of literature on mobility characteristics such as range of motion for infants with upper extremity impairments, 
While this is of relatively small F/E motion range, it does not affect testing the validity of our developed controller within the context of this work. In fact, the currently limited F/E motion range can be readily rectified by fabricating and employing actuators that integrate a large number of pouches, and is part of ongoing work. 
Fabricating different actuator designs of distinct length and number of pouches to extend or reduce the range of motion can be done quickly and in a cost-effective manner~\cite{sahin2022bidirectional}.
%
%it still lies within the range of infants with mobility impairments~\cite{lynch2009power}, and it was deemed appropriate to test the developed controller.} T
As far as the AB/AD range of motion is concerned, this was deemed satisfactory. 
%, francisco2021upper

\begin{table}[!ht]
\centering
\caption{Kinematic Constraints and Joint Velocity and Acceleration Limits at Initial ($t=0$) and Final ($t=T$) Time}
\label{table:constr}
\vspace{-6pt}
\begin{tabular}{ccccc}
\toprule
$\theta_i$ &  $\theta_{i,MIN}$ & $\theta_{i,MAX}$ &  
$\dot{\theta}_i$ \& $\ddot{\theta}_i$ ($t=0$) &
$\dot{\theta}_i$ \& $\ddot{\theta}_i$ ($t=T$) \\%& \textbf{$\ddot{\theta}_i(t=0)$} & \textbf{$\ddot{\theta}_i(t=T)$}\\ 
\midrule
$1\;(F/E)$ & $0.1745$    & $1.3963$ & $0$  & $0$\\
$2\;(AB/AD)$ & $0.1745$    & $0.5585$ & $0$  & $0$\\
\bottomrule
\end{tabular}
\vspace{-8pt}
\end{table}

%
%We make the following assumptions. 
%
%\begin{itemize}
    
    %\item The wearable device is assumed to contribute with the mobility of two degrees of freedom to the infant's arm: shoulder adduction and abduction and elbow flexion and extension. 
    %\item No dynamic effects or disturbances were included when modeling the controller.
    %\item We assume a Proportional-Derivative (PD) controller for the shoulder and elbow pneumatic actuators, considering a single arm motion.
    %\item Every motion starts and finishes at rest, and motion constraints are shown in Table \ref{table:constr}, where $\theta_s$, $\theta_e$ and $\ddot{\theta_s}$, $\ddot{\theta_e}$ are the shoulder and elbow joints position and acceleration, respectively.
    %\item In order to minimize jerk in motion and respect constraints on acceleration, we have adopted a quintic polynomial approach to trajectory generation. 
    %\item All trajectories are assumed to have the infant sitting on a chair, which also powers the wearable device.
%\end{itemize}


\begin{figure}[!t]
\vspace{6pt}
\includegraphics[width=\linewidth]{MEDIA/aqaq.png}
\vspace{-18pt}
\caption{Range of motion afforded by the two actuators. (Left two panels) Shoulder adduction/abduction: $\theta_1\in[0.1745,1.3963]$\;rad; (Right two panels) Shoulder flexion/extension: $\theta_2\in[0.1745,0.5585]$\;rad.}
\label{fig:bellow}
\vspace{-6pt}
\end{figure}




%\begin{figure}[htb]
%\centering
%\vspace{-6pt}
%\includegraphics[width=0.9\columnwidth]{MEDIA/otliut.jpg}
%\vspace{-6pt}
%\label{fig:expl}
%\vspace{-12pt}
%\end{figure}





\subsection{Combined System Kinematics}

%\subsection{Kinematics Model}
The last step prior to developing the controller is determination of the combined system (actuators and mannequin) kinematic model. We utilize the frames shown in the right panel of Fig.~\ref{fig:support}. Given that the elbow joint is taken to be fixed at full extension herein, we can consider a simplified model shown in Fig.~\ref{fig:coord} and readily derive its Denavit-Hartenberg (DH) parameters as given in Table~\ref{table:dh}.

\begin{figure}[!t]
   \vspace{2pt}
   \centering
   \includegraphics[width=0.99\columnwidth]{MEDIA/dsq.png}
      %\includegraphics[scale=1.0]{figurefile}
      \vspace{-9pt}
      \caption{Coordinate frames for kinematic calculation of the abduction/adduction (AB/AD) and flexion/extension (F/E) of the shoulder joint.}
      \label{fig:coord}
      \vspace{-4pt}
\end{figure}


\begin{table}[!ht]
\vspace{0pt}
\centering
\caption{Denavit-Hartenberg Parameters}
\label{table:dh}
\vspace{-6pt}
\begin{tabular}{cccc}
\toprule
\textbf{$\theta_j$} & \textbf{$d_j$} & \textbf{$r_j$}  & \textbf{$\alpha_j$} \\ \midrule
$\theta_1$ & $0$    & $0$ & $\pi/2$                     \\
$\theta_2$ & $0$    & $l_a$ & $\pi/2$ \\
\bottomrule
\end{tabular}
\vspace{-12pt}
\end{table}
 
The transformation matrix between the wrist ($w$) and the shoulder origin ($o$) is 

\begin{gather}
  \scalebox{1}{%
  $T =$
        $  \begin{bmatrix}
   c\theta_1c\theta_2 &   -c\theta_1 s \theta_2    & -s\theta_1 & l_ac\theta_1c\theta_2 \\
   c\theta_2s\theta_1 &  -s\theta_1 s \theta_2 &   c\theta_1   & l_ac\theta_2s\theta_1 \\
   -s\theta_2  & - c\theta_2  & 0  & -l_a s\theta_2  \\
      0 & 0 & 0 &  1\\
  \end{bmatrix},
  %
$
    \label{eq:forward}}
  \end{gather}
%
where the Cartesian position of the wrist $w=[x,y,z]^T$ is
%
 \begin{gather}
  \scalebox{1}{%
  $\begin{bmatrix}
   x \\
    y\\
    z\\
  \end{bmatrix}=$
   $\begin{bmatrix}
   l_ac\theta_1c\theta_2 \\
    l_ac\theta_2s\theta_1\\
   - l_a s\theta_2\\
  \end{bmatrix}\enspace.
  %
  $
   \label{eq:forw}}
 \end{gather}
%
For brevity, we use shorthands $c\cdot \equiv cos(\cdot)$ and $s\cdot \equiv sin(\cdot)$ in~\eqref{eq:forward}--\eqref{eq:forw}. 
Then, the respective shoulder angles for AB/AD ($\theta_1$) and F/E ($\theta_2$) can be calculated as 
%
%\begin{equation}
$\theta_1 = atan2\left(\frac{y}{x}\right)$, and %\quad
$\theta_2 = asin\left( -\frac{z}{l_a} \right)$. %\enspace.
%\end{equation}
% \noindent with $l_a$ the combined length of the arm and forearm. 



%Considering the system's range of motion (Fig.~\ref{fig:bellow}), we initially propose a series of desired end-points and respective Cartesian coordinates for reference trajectory generation (Table \ref{table:trajec}---the last column refers to the type of shoulder assistance). %Let $l_a=140\;mm$ represent the average length of the combined arm and forearm of a seven-month-old infant~\cite{sivan1983upper}.

%We can derive the rigid body dynamical model of the 3DOF wearable device. One remark is in relation to the rigid body assumption. Even though the actuators are soft, the use of it by the infant would justify the assumption (or in the present case, the use of the mannequin rather). Using the Euler-Lagrange equation:

%\begin{equation}
%    M(\theta)\ddot{\theta}(t) + C(\theta(t),\dot{\theta}(t))\dot{\theta}(t)+G(\theta(t)) \; = \; \tau(t)
%    \label{eq:dyn}
%\end{equation}

%\noindent with the positive definite inertia matrix $M(\theta)$, the matrix of Coriolis terms $C(\theta(t)$,$\dot{\theta}(t))$ and gravitational vector $G(\theta(t))$. The torque $\tau(t)$ is the vector of applied torques to each of the wearable joints. Eq. \ref{eq:dyn} can be rewritten as:

%\begin{equation}
%    \ddot{\theta}(t) = M(\theta)^{-1}\tau(t)-
%    M(\theta)^{-1}[C(\theta(t),\dot{\theta}(t))\dot{\theta}(t)+G(\theta(t))]
%    \label{eq:rea}
%\end{equation}

%\noindent such that, if 
%\[\begin{aligned}
%    \rho(t) = M(\theta)^{-1}[C(\theta(t),\dot{\theta}(t))\dot{\theta}(t)+G(\theta(t))] \\
%    u(t) = M(\theta)^{-1}\tau(t)
%\end{aligned}\]












\section{Low-level Trajectory Tracking Control}
\label{sec:methods}

\subsection{Overview}
Given a desired reference trajectory in joint space position, velocity and acceleration  $\theta_d(t),\dot{\theta}_d(t)$ and $\ddot{\theta}_d(t)$, let the tracking and input errors be $e = \theta - \theta_d(t)$ and $e_u = u - u_d(t)$, respectively. Our closed-loop control system then is
%
\begin{equation}
\begin{aligned}
        u(t) = u_d(t) - K_{GPI}(\theta-\theta_d(t))\enspace, \\
        \ddot{\theta}(t) = u(t) + \rho(t)\enspace,
\end{aligned}
    \label{eq:ccon1}
\end{equation}
%
\noindent where $K_{GPI}$ is the proposed controller, $\rho(t)$ a polynomial of degree $r$ describing disturbances to the system, and the nominal control input $u_d(t)$ can be determined according to the unperturbed system ($\rho(t)=0$). The proposed GPI controller can be described as a lead compensator, i.e.

\begin{equation}
\begin{aligned}
K_{GPI}(s) &= \left(\frac{k_{r+2,i}s^{r+2}+k_{r+1,i}s^{r+1}+...+k_{1,i}s + k_{0,i}}{s^{r+1}(s+k_{r+3,i})} \right)\enspace,\\
    \end{aligned}
    \label{eq:each}
\end{equation}
%
with controller gains $k_{j,i}$; indices $j=0,\ldots,r+2$ and $i=1,2$ denote the controller constant and actuator, respectively. 
%
Recall that each actuator is controlled independently, hence~\eqref{eq:each} applies independently for $i=1,2$.\footnote{~To improve clarity, indices denoting actuator number are dropped when presenting functional expressions that apply to both actuators (e.g., $K_{GPI}$).} 




The control loop is depicted in Fig.~\ref{fig:cl}. For each DoF, estimated transfer function parameters $[\gamma_{0i},\gamma_{1i},\gamma_{2i}]$ and parameters $\xi_i$ and $\omega_{ni}$ need to be set. The controller acts on the error between the desired joint values and actual ones provided via IMU readings. 
%
In the following, we present the controller derivation and how its parameters are tuned. 

\begin{figure}[!t]
    \vspace{3pt}
   \centering
     \includegraphics[trim={0.5cm 0 0.5cm 0},clip,width=\columnwidth]{MEDIA/cscheme.jpg}
      %\includegraphics[scale=1.0]{figurefile}
      \vspace{-16pt}
      \caption{Schematic of the proposed GPI controller for each actuator, $i=1,2$. Controller parameters are tuned experimentally. The mannequin's joint angle information is provided in real-time via proprioceptive feedback.}
      \label{fig:cl}
      \vspace{-6pt}
\end{figure}

\subsection{Controller Derivation}
In this work we assume $r=0$ (i.e. a constant disturbance).  Then~\eqref{eq:each} reduces to
%
\begin{equation}
    K_{GPI}(s) = \frac{k_{2,i}s^2+k_{1,i}s+k_{0,i}}{s(s+k_{3,i})}\enspace.
\end{equation}
%
Treating each DoF of the combined mannequin and actuators system as a spring-mass-damper system, the transfer function which relates the PWM \% of the pneumatic actuator and angle $\theta_i$ %of the respective supported segment of the arm, 
attains the form
%
\begin{equation}
    G_i(s) = \frac{\gamma_{0,i}}{s^2 + \gamma_{1,i} s + \gamma_{2,i}}\enspace.
    \label{eq:mass}
\end{equation}
 
\noindent Rewriting~\eqref{eq:ccon1}, and dividing with $\gamma_{0_i}$ yields %the controller constant values by $1/\gamma_{0_i}$ yields 
%
\begin{equation}
        u(t) = u_d(t) - \left(\frac{1}{\gamma_{0,i}}\frac{k_{2,i}s^2+k_{1,i}s+k_{0,i}}{s(s+k_{3,i})}\right)e
        \label{eq:imple}
\end{equation}
%
since $e(s)=G_i(s)e_u(s)$ relates the input and output errors. Combining with~\eqref{eq:mass}, we get
%
\begin{equation}
    \left[1+\left( \frac{1}{s^2 + \gamma_{1,i} s + \gamma_{2,i}} \right) \left( \frac{k_{2,i}s^2+k_{1,i}s+k_{0,i}}{s(s+k_{3,i})}\right)\right]e=0
    \label{eq:both}
\end{equation}
%
leading to the characteristic equation 
\begin{equation}
\begin{aligned}
    s^4 + (k_{3,i}+\gamma_{1,i})s^3 + (k_{2,i} + k_{3,i}\gamma_{1,i}+\gamma_{2,i})s^2 +\\ (k_{3,i}\gamma_{2,i}+k_{1,i})s + k_{0,i} = 0\enspace.
    \label{eq:1}
    \end{aligned}
\end{equation}

The controller gains $k_{j,i}$ can be determined by factoring the Hurwitz polynomial %(therefore having poles located in the half left complex plane), 
%i.e.
%
%\begin{equation}
    $(s^2 + 2\xi w_n^2+w_n^2)^2 = 0
$, %\enspace,
    %\label{eq:2}
%\end{equation}
%
which yields
\begin{equation}
    \begin{aligned}
        &k_{0,i} = w_n^4\enspace, \\
        &k_{1,i} = 4w_n^3\xi - \gamma_{2,i}(4\xi w_n-\gamma_{1,i})\enspace,\\
        &k_{2,i} = 2w_n^2 + 4\xi^2w_n^2 - \gamma_{1,i}(4\xi w_n - \gamma_{1,i}) - \gamma_{2,i}\enspace,\\
        &k_{3,i} = 4\xi w_n - \gamma_{1,i}\enspace.
    \end{aligned}
    \label{eq:third}
\end{equation}



Therefore, our controller can be determined with parameters from~\eqref{eq:third}. The controller design parameters are $\xi_i$ and $w_{n_i}$. From~\eqref{eq:imple} we can implement the controller as 
%
\begin{equation}
\begin{aligned}
    u = u_d - k_{3,i}(\dot{\theta}_{int}-\dot{\theta}_d) + \\ \frac{1}{\gamma_{0,i}}\left[ -k_{2,i}(e-e_0) - k_{1,i}\int_t e dt  - k_{0,i}\iint_t e dt \right] \enspace,
    \end{aligned}
    \label{eq:controlad}
\end{equation}
%
%where $u$ is the output, 
where $\theta_{int}$ is the integral reconstruction \cite{romero2014algebraic,blanco2022robust} of input $u$

\begin{equation}
%\begin{aligned}
        \theta_{int} = \int_{0}^{t} u dt~,~~~~
        \dot{\theta} = \theta_{int} -\dot{\theta}_0
            \label{eq:ust}
%\end{aligned}
\end{equation}
%
and the value $u_d$ is given by~\cite{sira2007fast}, 
%
\begin{equation}
    u_d = \frac{1}{\gamma_{0,i}}[\ddot{\theta}_d + \gamma_{1,i}\dot{\theta}_d+\gamma_{2,i}\theta_d]\enspace.
\end{equation}
%
For both~\eqref{eq:controlad} and~\eqref{eq:ust}, the integral values are numerically estimated by the trapezoidal rule. 

\subsection{Parameter Tuning}

Each $G_i(s)$ is estimated from offline experimental data and is inputted to Matlab ($tfest$ function with sampling time of $T_s=0.065s$, 700 data-points) to compute variations of both DoF of the shoulder and PWM percentages of the pneumatic pumps (shown in Fig.~\ref{fig:datap}) leading to
%
   \begin{equation}
   \begin{aligned}
     G_1 &= \frac{0.0005725}{s^2+0.05725s+0.044}\enspace, \\
    G_2 &= \frac{0.0003665}{s^2+0.213s+0.04079}\enspace.
   \label{eq:tf}
   \end{aligned}
 \end{equation}

$G_1$ and $G_2$ relate to shoulder AB/AD (estimation fit of $89\%$) and F/E (estimation fit of $91.36\%$) motion, respectively. We set $\xi_1$, $\xi_2 = 0.9$, $\omega_{n,1}=6.1$ \;rad/s and $\omega_{n,2}=10.25$ \;rad/s. These parameters are essential to compute controller gains as per~\eqref{eq:third}, and were chosen so that the input $u(t)$ is identically set at a maximum value of $100$, i.e. $100\%$ PWM.

\begin{figure}[!t]
   \vspace{5pt}
   \centering
     \includegraphics[width=0.9\columnwidth]{MEDIA/qwd.jpg}
      %\includegraphics[scale=1.0]{figurefile}
      \vspace{-9pt}
      \caption{Empirical correlation of PWM \% and joint angle variations. %for both shoulder $\theta_1$ and $\theta_2$ angles.
      }
      \label{fig:datap}
      \vspace{-9pt}
\end{figure}
   
%The control loop diagram is depicted in Fig.~\ref{fig:cl}. For each DoF, the estimated transfer function parameters $[\gamma_{0i},\gamma_{1i},\gamma_{2i}]$ in \eqref{eq:tf} as well as parameters $\xi_i$ and $\omega_{ni}$ are inputs to the controller that acts on the error produced by the difference between the desired interpolated values and IMU readings. 


\section{Experimental Results}
\label{sec:experim}


%\subsection{Experimental Settings} 
The overall system flow diagram is shown in Fig.~\ref{fig:overall}. %First, experimental data from separate joint motions were collected, and using MATLAB, the transfer function ($G_i(s)$) between the input (PWM), which regulates air flow to the actuators, and angular motion of each joint $\theta_i$ was estimated. 
We test three types of trajectories. The first set includes interpolated trajectories (via quintic polynomials) given desired end-points for both AB/AD and F/E shoulder motion separately, as well as simultaneously. The second set considers periodic trajectories in the form of harmonic signals while the third set focuses on teach-and-repeat trajectories. 

\subsection{End-Point Trajectories}

To support motion smoothness, intermediary desired joint angle values are computed based on quintic polynomial time scaling, $\theta_d(t) = a_0t^5+a_1t^4+a_2t^3+a_3t^2+a_4t+a_5,~t\in[0,T]$, for each joint separately. The polynomial coefficients are calculated based on the kinematic constraints shown in Table~\ref{table:constr} for full-range motion. For distinct end-point trajectories considered during this part of experimentation, the initial ($t=0$) and terminal ($t=T$) time joint velocities and accelerations are all set to 0, while the initial and final joint angles may vary (Table~\ref{table:trajec}). 

\begin{figure}[!t]
\vspace{3pt}
\includegraphics[width=\linewidth]{MEDIA/mkl.jpg}
\vspace{-21pt}
\caption{Overall structure of the system implementation.}
\label{fig:overall}
\vspace{-3pt}
\end{figure}


\begin{table}[!t]
\vspace{0pt}
\centering
\caption{Desired Trajectory End-Points}
\label{table:trajec}
\vspace{-6pt}
\begin{tabular}{ccccc}
\toprule
Case & $\theta_1 [rad]$ & $\theta_2 [rad]$ & $(x,y,z) [m]$ & Type\\ \midrule
$q_1$ & $0.6981$ & $-$    & $[0.107,0.107,0]$     & $AB/AD$              \\

$q_2$ & $1.0472$ & $-$    & $[0.070,0.070,0]$    & $AB/AD$                \\

$q_3$ & $-$ & $ 0.3491$   & $[0.132,0.132, -0.048]$   & $F/E$                \\

$q_4$ & $-$ & $0.5585$    & $[0.119,0.119, -0.074]$             & $F/E$       \\

$q_5$ & $0.6981$ & $0.3491$    & $[0.066,0.066, -0.048]$      & $AB/AD$ \& $F/E$             \\

$q_6$ & $0.6981$ & $0.5585$    & $[0.059, 0.059, -0.074]$      & $AB/AD$ \& $F/E$             \\

$q_7$ & $ 1.3963$ & $0.3491$    & $[0.023, 0.023,-0.048]$      & $AB/AD$ \& $F/E$             \\

$q_8$ & $1.3963$ & $0.5585$    & $[0.021,0.021,-0.074]$         & $AB/AD$ \& $F/E$        \\
\bottomrule
\end{tabular}
\vspace{-15pt}
\end{table}  


% & $0.6981$ & $-$    & $[0.1072,0.1072,0]$     & $AB/AD$              \\

% & $1.0472$ & $-$    & $[0.0700,0.0700,0]$    & $AB/AD$                \\

% & $-$ & $ 0.3491$    & $[0.1316,0.1316, -0.0479]$   & $F/E$                \\

% & $-$ & $0.5585$    & $[0.1187
% ,0.1187,-0.0742]$             & $F/E$       \\

% & $0.6981$ & $0.3491$    & $[0.0658,0.0658, -0.0479]$      & $AB/AD/F/E$             \\

% & $0.6981$ & $0.5585$    & $[0.0594, 0.0594, -0.0742]$      & $AB/AD/F/E$             \\

% & $ 1.3963$ & $0.3491$    & $[0.0228, 0.0228,-0.0479]$      & $AB/AD/F/E$             \\

% & $1.3963$ & $0.5585$    & $[0.0206,0.0206,-0.0742]$         & $AB/AD/F/E$        \\



  
%\subsection{System Description}








 

 



%\subsection{Implemented Trajectories}





     \begin{figure*}[!t]
     \vspace{6pt}
   \centering
     \includegraphics[width=0.80\textwidth]{MEDIA/grapw1.pdf}
      %\includegraphics[scale=1.0]{figurefile}
      \vspace{-12pt}
      \caption{Experimental results for \emph{single actuator operation} cases $q_1$ through $q_4$ (from top to bottom). (Left) Averaged evolution of angle (desired, actual via IMU, and ground truth via motion capture). (Center) Averaged control input evolution. (Right) Averaged error evolution. Individual trials in each case were very close to each other, and hence for clarity of presentation only the mean values are presented. Time is shown in centiseconds ($10^{-2}s$).}
      \label{fig:gf1}
      \vspace{-3pt}
   \end{figure*}
   
   
       \begin{figure*}[!ht]
       \vspace{-6pt}
   \centering
     \includegraphics[width=0.80\textwidth]{MEDIA/grapw2.pdf}
      %\includegraphics[scale=1.0]{figurefile}
      \vspace{-18pt}
      \caption{Experimental results for \emph{simultaneous actuator operation} cases $q_5$ through $q_8$ (from top to bottom). %All quantities are as in~\ref{fig:gf1} and not repeated. 
      (Left) Averaged evolution of angle (desired, actual via IMU, and ground truth via motion capture). (Center) Averaged control input evolution. (Right) Averaged error evolution. Individual trials in each case were very close to each other, and hence for clarity of presentation only the mean values are presented. Time is shown in centiseconds ($10^{-2}s$).
      }
      \label{fig:gf2}
      \vspace{-6pt}
   \end{figure*}


We performed a total of 60 experimental trials considering the various interpolated trajectories from the end-points listed in Table~\ref{table:trajec} (seven trials for cases $q_1$ through $q_4$ each, and eight trials for each of the remainder cases). The starting angle was set at $\theta_i=0.2$\;rad as per kinematic constraints for all cases.  %
Figures~\ref{fig:gf1} and~\ref{fig:gf2} summarize the obtained results (average values depicted). 
%
Actual (integrated) IMU feedback was compared against ground truth measurements provided by a 12-camera Optitrack motion capture system. 
%
%For the separate motions of AB/AD and F/E (Fig. \ref{fig:gf1}) IMU feedback was integrated, and the desired (interpolated), actual (IMU) and motion capture data are shown in sequence. 
Steady-state errors (mean squared errors [MSE] and standard deviation errors [SDE]) are shown in Table~
\ref{table:mse_1} for the single actuator operation cases $\{q_1,q_2,q_3,q_4\}$, and in Table~\ref{table:mse_2} for the simultaneous actuator operation cases $\{q_5,q_6,q_7,q_8\}$.

\begin{figure*}[!t]
\vspace{6pt}
\centering
\includegraphics[width=\textwidth]{MEDIA/SINEd1.pdf}
\vspace{-21pt}
\caption{Top: Experiments with various harmonic trajectories ($h=300$) and: a ($A=1,\;f=1.6\;e^{-3}$), b ($A=1,\;f=2.6 \; e^{-3}$), c ($A=1,\;f=3.6\;e^{-3}$), d ($A=1,\;f=4.3\;e^{-3}$), e ($A=3,\;f=1.3\;e^{-3}$) and f ($A=3,\;f=2.6\;e^{-3}$). Bottom: Custom teach-and-repeat trajectories. All curves denote averaged quantities. Repeatability was high and hence for clarity standard deviations funnels are not shown.}
\label{fig:siteach}
\vspace{-9pt}
\end{figure*}

Overall, it can be readily verified that our proposed controller can successfully track the desired trajectories smoothly and with minimal steady-state error. In single-actuator operation (Fig.~\ref{fig:gf1}) all cases but one ($q_4$) perform as desired. In this case, the desired end-point is at the limit $\theta_2=0.5585$\;rad, and the control input signal $u_i(t)$ (depicted in the middle columns) saturates (i.e. 100\% PWM signal) when attempting to inflate the actuator. This means the controller could not overcome the physical limitation of the actuator while acting against the total weight of the mannequin's arm, and so it was not able to achieve the desired end-point. Additionally, relative motion of the back support with respect to the arm itself was observed, leading to a higher steady state error. However, no limitations were observed for trajectories within the actuator's physical boundary, as shown in the tracking performance for end-point $ \theta_2=0.3491$\;rad. In the combined trajectories (cases $q_5$ through $q_8$) we did not observe any control saturation even when $\theta_2$ was commanded to the limit. This can be associated to implicit actuator synergies whereby one may aid another at different levels of pressurization. The tracking errors are still small (cf MSE in Tables~\ref{table:mse_1} and~\ref{table:mse_2}) but relatively larger compared to those in single-actuator operation. Comparably increased errors can be associated with higher motion variability attained due to no explicit coupling of the two actuators. 

%The combined AB/AD and F/E trajectories were also evaluated experimentally (Fig.~\ref{fig:gf2}), with MSE values also shown in Table \ref{table:mse}. The steady state errors in this case were smaller than $0.1$ rad, as by having both actuators supporting the arm, each actuator will require less pressure to achieve a desired position and therefore will be less constrained in this case. 

% \begin{table}[!t]
% \vspace{4pt}
% \caption{\changed{MSE (SDE) for Joint Angles $\theta_1$ and $\theta_2$}}
% \vspace{-6pt}
% \resizebox{\columnwidth}{!}{%
% \begin{tabular}{@{}cllllllll@{}}
% \toprule
%  &\multicolumn{1}{c}{$q_1$} & \multicolumn{1}{c}{$q_2$} & \multicolumn{1}{c}{$q_3$} & \multicolumn{1}{c}{$q_4$} & \multicolumn{1}{c}{$q_5$} & \multicolumn{1}{c}{$q_6$} & \multicolumn{1}{c}{$q_7$} & \multicolumn{1}{c}{$q_8$} \\ \midrule
% $e_{\theta_1}10^{-3}$ rad & 0.04 (0.20) & 0.08 (0.28) &  &  & 1.00 (1.00) & 1.00 (1.00) & 1.00 (1.00) & 1.00 (1.00) \\
% $e_{\theta_2}10^{-3}$ rad &  &  & 0.06 (0.24) & 1.00 (1.00) & 1.00 (1.00) & 4.00 (2.00) & 0.70 (0.83) & 2.00 (1.41) \\ \bottomrule
% \end{tabular}%
% }
% \label{table:mse}
% \vspace{-9pt}
% \end{table}

\begin{table}[!t]
\vspace{6pt}
\caption{MSE (SDE) for Joint Angles $\theta_1$ and $\theta_2$ during Single Actuator Operation}
\vspace{-6pt}
\resizebox{\columnwidth}{!}{%
\begin{tabular}{@{}cllll@{}}
\toprule
 & $q_1$ & $q_2$ & $q_3$ & $q_4$  \\ \midrule
$e_{\theta_1}10^{-3}$ rad & 0.04 (0.20) & 0.08 (0.28) &  & \\
$e_{\theta_2}10^{-3}$ rad &  &  & 0.06 (0.24) & 1.00 (1.00) \\ 
\bottomrule
\end{tabular}%
}
\label{table:mse_1}
\vspace{-6pt}
\end{table}


\begin{table}[!t]
\vspace{4pt}
\caption{MSE (SDE) for Joint Angles $\theta_1$ and $\theta_2$ during Simultaneous Actuator Operation}
\vspace{-6pt}
\resizebox{\columnwidth}{!}{%
\begin{tabular}{@{}cllll@{}}
\toprule
 &$q_5$ & $q_6$ & $q_7$ & $q_8$ \\ 
 \midrule
$e_{\theta_1}10^{-3}$ rad & 1.00 (1.00) & 1.00 (1.00) & 1.00 (1.00) & 1.00 (1.00) \\
$e_{\theta_2}10^{-3}$ rad & 1.00 (1.00) & 4.00 (2.00) & 0.70 (0.83) & 2.00 (1.41) \\ \bottomrule
\end{tabular}%
}
\label{table:mse_2}
\vspace{-9pt}
\end{table}




\subsection{Harmonic Trajectories}
To test our method's response to periodic motion, harmonic trajectories of various frequency and amplitudes were evaluated with the AB/AD actuator (given its wider range of motion). Desired trajectories were as  
%
%\begin{equation}
%\begin{aligned}
       %&
       $\theta_1(t) = A/2sin(ft+h) + A/2$ %\; \; \dot{\theta}_{1} %, \; \;\ddot{\theta}_{1} 
%\end{aligned}
%\end{equation}
%
($A$: amplitude, $f$: frequency, $h$: constant). 

Results for various trajectories are shown in Fig.~\ref{fig:siteach} (a to f), with a total of 25 experimental trials. It can be seen that the device correctly follows the desired trajectories, while small deviations occurring on valleys (seen as straight lines). This can be associated to physical constraints on the minimum achievable AB/AD joint angle (Fig.~\ref{fig:bellow}), thus not allowing the actuator to reach a closer to $0$\;rad value. % despite vacuum force applied. 

\subsection{Teach-and-Repeat Trajectories}
To investigate more versatile trajectories, particularly aiming at evaluating the controller within the achievable range of the AB/AD actuator motion, we implemented a teach-and-repeat mode on the wearable. Desired trajectories were given for five seconds by a researcher physically manipulating the mannequin, and the controller was task to repeat it afterwards. To record the demonstrated trajectories, input from the IMU angular displacement and velocity were directly recorded, while acceleration was estimated by differentiation.


Results are shown in Fig.~\ref{fig:siteach} (g to n). We performed 25 experimental trials in this mode. The controller is seen to closely follow all taught trajectories, with minor deviations attributed to vibrational noise on the IMU, introduced by the higher frequency segments (as seen in Fig. \ref{fig:siteach} cases g, h and j). The accompanying video demonstrates these experiments.

%\noindent which indicates the right choice of coefficients $k_i$ can lead the error to a global exponential asymptotic equilibrium at $e=0$.

   



\section{Conclusion}
\label{sec:conclusion}

This work introduced a closed-loop control method based on proprioceptive feedback for pressure regulation of soft pneumatic actuators embedded on an infant-scale engineered mannequin so as to follow desired trajectories in support of shoulder motion. The developed low-level controller can help track higher-level force-control-based trajectories in future pediatric wearable robotic assistive devices. 
%
%System characterization in terms of motion constraints and transfer function estimation were presented, as well as the controller proposed to act in following joint trajectories for each DoF independently or combined. Using ROS integration between onboard sensors (IMUs), pneumatic actuation and motion capture system, 
Extensive experimentation confirmed the ability of our proposed controller to track a range of diverse trajectories (desired end-points, harmonic, and teach-and-repeat) smoothly and accurately.
%as well as follow harmonic motion and teach-and-repeat trajectories.

This work conducted herein lays the basis for several future research directions. Notably, we aim to integrate shoulder and arm support. Further we seek to embed the actuators onto a complete wearable device and further test it with the engineered mannequin. Once the efficacy and safety of the wearable is demonstrated, we plan on moving with human subjects testing. %Further, we envision integrating

%Despite the overall tracking performance observed in our experiments, some improvements can be suggested. The bellow-type actuator, although capable to support shoulder F/E motion, has a limited range of motion. It also needs a back support, which in turn can be unsettling for infants when wearing the device. These will inform the next iteration design which intends to improve both actuator support and reachability. Development and ergonomic study towards a full upper extremity wearable suit, integration with high level controllers utilizing additional sensing modalities, and human subject testing with infants will be considered as future work.

%\section*{ACKNOWLEDGMENT}
%The authors would like to thank Zhicao Liu for his invaluable contribution with the pneumatic board design and its software communication, and Ipsita Sahin, Jared Dube and Linh Vu for their commitment in making sure numerous actuators were available for testing during experiments. 
%\balance
\bibliographystyle{ieeetr}
%\newpage
\bibliography{root}

\end{document}