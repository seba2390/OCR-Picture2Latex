\usepackage{tikz}
\usepackage{amssymb}
\usepackage{verbatim} 
\usetikzlibrary{matrix}
\usepackage{color} 
\usepackage{stmaryrd}
\usepackage{amsfonts} 
\usepackage{amsmath} 
\usepackage{amsthm}
\usepackage[all,cmtip]{xy}
\usepackage{catex} 			
\usepackage{fancybox}
\usepackage{relsize}
\usepackage{cmll} 
\usepackage{bussproofs}
%\usepackage{proof}
%\usepackage{hyphenat}
%\usepackage{subcaption}
%\usepackage[pdftex]{graphicx}
\usepackage{standalone}
\usepackage[normalem]{ulem}
%\usepackage{hyperref}
\usepackage{multicol}

\usepackage{mathdots}
%\usepackage{MnSymbol} %http://mirror.jmu.edu/pub/CTAN/fonts/mnsymbol/MnSymbol.pdf

%%%%%%%%%%%%%
\newcommand{\corr}[2]{{\color{lightgray}{#1}} {\color{blue}{#2}}}  %Per i referee

%\newcommand{\corr}[2]{#2} %Per la versione finale

\newenvironment{scprooftree}[1]%
  {\gdef\scalefactor{#1}\begin{center}\proofSkipAmount \leavevmode}%
  {\scalebox{\scalefactor}{\DisplayProof}\proofSkipAmount \end{center} }


%%%%%%%%%%%%%%%

%% intestazioni inglese
\newtheorem{theorem}{Theorem}[section]
\newtheorem{corollary}[theorem]{Corollary}
%\newtheorem{lemma}[teor]{Lemma}
\newtheorem{oss}{Remark} 
\newtheorem{proposition}[theorem]{Proposition}
\newtheorem{conj}[theorem]{Conjecture}

\theoremstyle{definition}
\newtheorem{definition}{Definition}
\newtheorem{exe}{Example} 
\newtheorem{Q}{Question}


%\hyphenation{sequentializable se-quen-tia-li-za-ble}

\newcommand{\nota}{\noindent\textbf{Notation }}

\newcommand{\citamio}[2]{\begin{center}\textit{``#1''}\end{center}\begin{flushright}[#2]\end{flushright}}						%parentesi per le presentazioni sx

%my shortcuts


% rewriting string symbols
\newcommand{\ls}{\mathcal{h}}						%parentesi per le presentazioni sx
\newcommand{\rs}{\mathcal{i}}						%parentesi per le presentazioni gruppi dx
\newcommand{\fr}{\rightarrow}						%freccia
\newcommand{\frr}{\Rightarrow}						%2-freccia
\newcommand{\mongen}[2]{\ls #1 , #2 \rs^+}			% monoide generato da pres
\newcommand{\grgen}[2]{\ls #1 , #2 \rs}				%gruppo generato da press
\newcommand{\red}{\mathfrak{r}}					%parentesi per le presentazioni sx
\newcommand{\semiexc}{ \sim_{exc}}					%excange rule + equivalence
\newcommand{\exc}{{\simeq_{exc}}}					%excange rule
\newcommand{\form}{Form}						%Formal diagram set
\newcommand{\fd}{F}							%formal diagram                  \tilde \phi
\newcommand{\fdd}{G}						%formal diagram                  \tilde \psi
\newcommand{\fds}{\tilde \phi}						%lifting of a formal diagram                  \tilde \phi
\newcommand{\fde}{\fd_{exp}}					%formal diagram (expanded)
\newcommand{\cont}{Cont}						%Formal diagram context
\newcommand{\weq}{\eqcirc}						%ugualianza simbolica stringhe
\newcommand{\req}{{\leftrightarrow^*}}				%relazione di equivalenza indotta da ->
\newcommand{\sem}[1]{\llbracket #1 \rrbracket}


\newcommand{\pmon}[2]{( #1, \R )}

%letters
\newcommand{\Z}{\mathbb Z}
\newcommand{\C}{\mathcal C}
\newcommand{\R}{\mathcal R}
\newcommand{\M}{\mathfrak M}
\newcommand{\Sim}{\mathfrak F}
\newcommand{\N}{\mathbb N}
\newcommand{\s}{\Sigma}
\newcommand{\I}{\mathcal{I}}
\newcommand{\pol}{\textbf{Pol}}
\newcommand{\dia}{\mathfrak{Dia}}

%math symbols
\newcommand{\sig}{\mathcal S} 						%signature
\newcommand{\pro}{\; \square \;}					%prodotto quadrato
\newcommand{\hor}{\mathfrak h }					%horizontal diagram
\newcommand{\starratory}[1]{\underset {#1} {\mathlarger{\mathlarger{\mathlarger *}}}} 						%signature
\newcommand{\circatory}[1]{\underset {#1} {\mathlarger{\mathlarger{\mathlarger \circ}}}} 						%signature
\newcommand{\sigcat}{\mathfrak {Sig}} 						%category of signatures
\newcommand{\sigx}{\Xi} 						%signature for the category of diagrams
\newcommand{\sigext}[1]{\Xi_{#1}} 						%signature for the category of diagrams

\newcommand{\inp}{\mathrm{in}} 						%in
\newcommand{\outp}{\mathrm{out}} 						%out
\newcommand{\hgt}{\mathrm{hgt}} 						%hgt
\newcommand{\ord}{\mathrm{ord}} 						%ord

\newcommand{\catgen}[1]{\ls{#1}\rs} 						%n-cat generated by (n+1)-poly
\newcommand{\Sem}[1]{\mathcal {S}_{{#1}}} 					%pol of Semantics of #1 fragment


%nell'articolo Coherence 
\newcommand{\D}{\phi}
\newcommand{\sym}{\tau}
\newcommand{\e}{\mathit e}
\newcommand{\id}{\textbf{id}}
\newcommand{\comm}{\bigodot}



%articolo Proof diagrams
\newcommand{\frrr}{\xymatrix@C-1pc{{} \ar@3[r]& {}}}

\newcommand{\cutelim}{\xymatrix@C-1pc{{} \ar@3[r]& {}}}
\newcommand{\vcutelim}{\xymatrix@R-0.5pc{~\ar@3[d] \\~ }}
\newcommand{\swap}{\twocell{s}}
\newcommand{\swapl}[2]{T_{#1 ,#2}}
\newcommand{\permpoly}{\mathfrak S} 		   %poly delle permutazioni
\newcommand{\erase}[2]{Er_{#2}(#1)} 		   %cancella filo i-esimo
\newcommand{\Erase}[2]{Er_{\{#2\}}( #1 )}                %cancella set di fili
\newcommand{\cutp}[3]{cut_{ {#1} }^{ {#2} }[ {#3} ]}                %icut permutation


\newcommand{\Mllt}{\tilde{\mathfrak{M}}}				%poly mult withouth megatwist
\newcommand{\Mellt}{\tilde{\mathfrak{E}}}				%poly mult withouth megatwist
\newcommand{\Mlltc}{{\tilde{\mathfrak{U}}}}			%poly mult + c withouth megatwist

\newcommand{\Mll}{\mathfrak{M}}                %poly mult
\newcommand{\Mllc}{\mathfrak{U}}                %icut permutation

\newcommand{\fmll}{{\mathfrak{F}_{M\ell \ell}}}                %formule MLL
\newcommand{\smll}{{{\mathfrak{F}}^*_{M\ell \ell }}}                %sequenti MLL
\newcommand{\fmllc}{{\mathfrak{F}_{M\ell \ell _u}}}                %formule MLLc
\newcommand{\smllc}{{{\mathfrak{F}}^*_{M\ell \ell _u}}}                %sequenti MLLc
\newcommand{\fmell}{{\mathfrak{F}_{Me \ell \ell}}}                %formule MELL
\newcommand{\fmellc}{{\mathfrak{F}_{Me \ell \ell _u}}}                %formule MLLc
\newcommand{\smellc}{{{\mathfrak{F}^*}_{Me \ell \ell }}}                %formule MLLc
\newcommand{\fll}{{\mathfrak{F}_{\ell \ell}}}                %formule LL

\newcommand{\Mell}{\mathfrak{E}}                %poly expo

\newcommand{\MLL}{\mathit{MLL}}				%MLL
\newcommand{\MLLc}{\mathit{MLL_u}}				%MLL con unita
\newcommand{\MELL}{\mathit{MELL}}				%MELL

				%%%%%%%%%%%%%%%%%%%
				%%%%%%CELLISMI%%%%%%%%%
				%%%%%%%%%%%%%%%%%%%


\deftwocell[crossing]{s : 2 -> 2}
\deftwocell[dots]{d : 2 -> 2}
\deftwocell[circle, black]{de : 0 -> 0}
\deftwocell[cap]{cap : 0 -> 2}
\deftwocell[cup]{cup : 2 -> 0}
\deftwocell[circle, black]{e1 : 1 -> 1}

\deftwocell[text=X,white]{g11 : 1 -> 1}				
\deftwocell[rectangle]{g : 2 -> 2} 				%generic gate	
\deftwocell[rectangle]{g35 :3 -> 5}				%generic gate 3 -> 5
\deftwocell[polygon]{g12:1 -> 2}				%generic gate 1-2
\deftwocell[polygon]{g21 :2 -> 1}				%generic gate 2-1
\deftwocell[text=\otimes,yellow]{net : 1 -> 2}			%tensore testa in giu

\deftwocell[text=\alpha,white]{net1 : 2 -> 1}			%interaction net
\deftwocell[text=\beta,white]{net2 : 2 -> 1}			%interaction net
\deftwocell[text=\nu,white]{net3 : 4 -> 1}			%interaction net
				%%%%%%%%%%%%%%%%%%%
				
\deftwocell[black]{m : 2 -> 1}
\deftwocell[circle,black]{e : 0 -> 1}
\deftwocell[white]{penta : 4 -> 1}
\deftwocell[white]{tria : 2 -> 1}
\deftwocell[white,rectangle]{exa : 3 -> 1}
\deftwocell[white,circle]{exa2 : 3 -> 1}
\deftwocell[white, rectangle]{inv : 2 -> 1}
\deftwocell[white]{g : 3 -> 1}
\deftwocell[gray]{alpha : 3 -> 1}
\deftwocell[gray, rectangle]{gamma : 3 -> 1}
\deftwocell[gray, rectangle]{tau : 2 -> 1}
\deftwocell[gray,lefthalfcircle]{l : 1 -> 1}
\deftwocell[gray,righthalfcircle]{r : 1 -> 1}


\deftwocell[text=\phi, white]{phi : 2 -> 2} 			%generic gate	phi
\deftwocell[text=\phi', white]{phi1 : 2 -> 2} 			%generic gate	phi'
\deftwocell[text=\psi, white]{psi : 2 -> 2} 			%generic gate	psi
\deftwocell[text=\psi', white]{psi1 : 2 -> 2} 			%generic gate	psi'

\deftwocell[text=\phi, white]{phi3 : 3 -> 3} 			%generic gate	phi 3->3
\deftwocell[text=\phi, white]{phi23: 2 -> 3} 			%generic gate	phi 2->3
\deftwocell[text=\phi, white]{phi24: 2 -> 4} 			%generic gate	phi 2->4
\deftwocell[text=\psi, white]{psi24: 2 -> 4} 			%generic gate	psi 2->4
\deftwocell[text=\phi', white]{phi32 : 3 -> 2} 			%generic gate	phi 3->2
\deftwocell[text=\phi', white]{phi42 : 4 -> 2} 			%generic gate	phi 4->2
\deftwocell[text=\psi, white]{psi4 : 4 -> 4} 			%generic gate	psi 4->4
\deftwocell[text=\psi', white]{psip4 : 4 -> 4} 			%generic gate	psi' 4->4
\deftwocell[text=\psi_1, white]{psi4b1 : 4 -> 4} 			%generic gate	psi_1 4->4
\deftwocell[text=\phi, white]{phi4 : 4 -> 4} 			%generic gate	phi 4->4
\deftwocell[text=\phi_1, white]{phi4b1 : 4 -> 4} 			%generic gate	phi_1 4->4
\deftwocell[text=\phi', white]{phip4 : 4 -> 4} 			%generic gate	phi' 4->4

\deftwocell[text=\phi, white]{phi04 : 0 ->4} 			%generic gate phi 0->4
\deftwocell[text=\phi, white]{phi05 : 0 -> 5} 			%generic gate phi 0->5
\deftwocell[text=\phi, white]{phi06 : 0 -> 6} 			%generic gate phi 0->6
\deftwocell[text=\phi, white]{phi07 : 0 -> 7} 			%generic gate phi 0->7
\deftwocell[text=\phi, white]{phi08 : 0 -> 8} 			%generic gate phi 0->8
\deftwocell[text = \phi' , white]{phi09 : 0 -> 9} 
\deftwocell[text = \phi' , white]{phi10 : 0 -> 10} 
\deftwocell[text = \psi , white]{psi07: 0 -> 7} 
\deftwocell[text = \psi' , white]{psi09 : 0 -> 9} 
\deftwocell[text = \psi' , white]{psi10 : 0 -> 10} 

\deftwocell[text=\phi_1, white]{phi104 : 0 -> 4} 			%generic gate phi_1 0->4
\deftwocell[text=\phi_1, white]{phi105 : 0 -> 5} 			%generic gate phi_1 0->5
\deftwocell[text=\phi_2, white]{phi205 : 0 -> 5} 			%generic gate phi_2 0->5
\deftwocell[text=\phi_2, white]{phi204 : 0 -> 4} 			%generic gate phi_2 0->4
\deftwocell[text=\phi_3, white]{phi303 : 0 -> 3} 			%generic gate phi_3 0->3
\deftwocell[text=\psi_1, white]{psi105 : 0 -> 5} 			%generic gate psi_1 0->5
\deftwocell[text=\psi_2, white]{psi205 : 0 -> 5} 			%generic gate psi_2 0->5


\deftwocell[text=\phi']{G8: 0 ->8}
\deftwocell[text=\phi']{G6: 0 ->6}
\deftwocell[text=\phi'_l]{G51: 0 -> 5}
\deftwocell[text=\phi'_r]{G52: 0 -> 5}



                                       %%%%%%%%%%%%%%%%%%%% diagrammi orizontali
                                       
\deftwocell[left=\dots]{midD : 1 ->1}	 			%stringa con dots a sinistra
\deftwocell[text=\mathfrak {h} , white]{hor :2  ->2} 			%generic hor diagrams
\deftwocell[text=\mathfrak {h}' , white]{hor1 :2  ->2} 			%generic hor diagrams
\deftwocell[text=\mathfrak {h}'' , white]{hori :2  ->2} 			%generic hor diagrams
\deftwocell[text=\mathfrak {h}''' , white]{hori1 :2  ->2} 			%generic hor diagrams
\deftwocell[text=\alpha , white]{blind1 : 0  ->2} 			%generic hor diagrams


				%%%%%%%%%%%%%%%%%%%
				
				
\deftwocell[text=x, lightgray]{gx1 : 1 -> 1} 			%generic gate	x 1->1
\deftwocell[text=x, lightgray]{gx : 2 -> 2} 			%generic gate	x 2-> 2
\deftwocell[text=g, lightgray]{g1 : 2 -> 2} 			%generic gate	d
\deftwocell[text=g', lightgray]{g2 : 2 -> 2} 			%generic gate	d
\deftwocell[text=N, white]{N : 2 ->1} 			%generic diag	N
\deftwocell[text=N', white]{N1 : 2 ->1} 			%generic diag	N
\deftwocell[text=N', white]{N31 : 3 ->1} 			%generic gate	N
\deftwocell[text=\alpha, white]{split : 4 ->1} 			%generic gate	Split
\deftwocell[text=\beta, white]{split2 : 4 ->1} 			%generic gate	Split

\deftwocell[text=\alpha, white]{a : 2 -> 2} 			%generic gate	alpha
\deftwocell[text=\sigma, white]{sigma : 2 -> 2} 		%generic gate	sigma
\deftwocell[text=\mu, white]{sigma4 : 4 -> 4} 		%generic gate	sigma
\deftwocell[text=\sigma, white]{sigma46 : 4 -> 6} 		%generic gate	sigma
\deftwocell[text=\sigma, white]{sigma57 : 5 -> 7} 		%generic gate	sigma
\deftwocell[text=\sigma, white]{sigma34 : 3 -> 4} 		%generic gate	sigma
\deftwocell[text=\sigma, white]{sigma35 : 3 -> 5} 		%generic gate	sigma
\deftwocell[text=\sigma, white]{sigma36 : 3 -> 6} 		%generic gate	sigma
\deftwocell[text=\sigma, white]{sigma6 : 6 -> 6} 		%generic gate	sigma
\deftwocell[text=\sigma, white]{sigma06 : 0 -> 6} 		%generic gate	sigma
\deftwocell[text=\sigma', white]{sigma1 : 2 -> 2} 		%generic gate	sigma


\deftwocell[text=\phi_1, white]{prem1 : 0 ->5} 			%generic gate phi 0->5
\deftwocell[text=\phi_2, white]{prem2 : 0 -> 5} 			%generic gate phi 0->5
\deftwocell[text=\phi_1', white]{phip : 5 -> 5} 			%generic gate	phi
\deftwocell[text=\phi_2', white]{phis : 5 -> 5} 			%generic gate	phi



\deftwocell[text=\chi_u, white]{up : 2 -> 6} 				%upcontex	
\deftwocell[text=\chi_d, white]{do : 6 -> 2} 				%downcontex	
\deftwocell[text=\chi_u, white]{up4 : 2 -> 4} 				%upcontex	
\deftwocell[text=\chi_d, white]{do4 : 4 -> 2} 				%downcontex	


\deftwocell[rectangle, green]{bigt2 : 2 -> 2} 			%bigtwist W W'
\deftwocell[rectangle, green]{bigt : 8 -> 8} 			%bigtwist W W'
\deftwocell[rectangle, green]{bigt10 : 10 -> 10} 			%bigtwist W W'
\deftwocell[rectangle, green]{bigt12 : 12 -> 12} 				%bigtwist 12
\deftwocell[text=BT{(}W W' {)}, white]{btW : 8 -> 8} 			%bigtwist W W'
             
             
\deftwocell[crossing2]{ln : 3 -> 3}				%Left n
\deftwocell[crossing1]{rn : 3 -> 3}				%Right n

\deftwocell[text=1, white]{dia1: 0 -> 2}
\deftwocell[text=2, white]{dia2: 0 -> 2}
\deftwocell[text=3 , white]{dia3: 0 -> 2}

\deftwocell[text=\phi, white]{diaphi02: 0 -> 2}
\deftwocell[text=\phi, white]{diaphi03: 0 -> 3}
\deftwocell[text=\phi, white]{diaphi04: 0 -> 4}
\deftwocell[text=\psi, white]{diapsi03: 0 -> 3}
\deftwocell[text=\psi, white]{diapsi04: 0 -> 4}
\deftwocell[text=\phi, white]{diaphi05: 0 -> 5}

\deftwocell[text=\chi_{ij}, white]{chigen : 2 -> 2}
%%%%%%%%%% string/background  labels %%%%%%%%%%%%%%%%%%%%%

\deftwocell[mid = \mathcal{C}]{backmC : 0 -> 0}				%background mathcal C
\deftwocell[mid = \mathcal{D}]{backmD : 0 -> 0}				%background mathcal C
\deftwocell[mid = \mathcal{E}]{backmE : 0 -> 0}				%background mathcal C

\deftwocell[text=\circ, white]{circ : 2 -> 1} 			%generic gate	composizione 2->1



\deftwocell[mid = A]{topA : 0 -> 1}						%top filo A
\deftwocell[mid = {?}A]{topAw : 0 -> 1}					%top filo \wnA
\deftwocell[mid = {?}A^\bot]{topAbw : 0 -> 1}					%top filo \wnA^\bot
\deftwocell[mid = A^\bot]{topAb : 0 -> 1}					%top filo A^\bot
\deftwocell[mid = B^\bot]{topBb : 0 -> 1}					%top filo B^\bot
\deftwocell[mid = B]{topB : 0 -> 1}						%top filo B
\deftwocell[mid = {?}B]{topBw : 0 -> 1}					%top filo ?B
\deftwocell[mid = C]{topC : 0 -> 1}						%top filo C
\deftwocell[mid = D]{topD : 0 -> 1}						%top filo D

\deftwocell[mid = \Gamma]{topGam : 0 -> 2}					%fili seq Gamma
\deftwocell[mid = \Gamma']{topGam1 : 0 -> 2}				%fili seq Gamma'
\deftwocell[mid = {?}\Gamma]{topGamw : 0 -> 2}				%fili seq ? Gamma
\deftwocell[mid = {?}\Gamma']{topGam1w : 0 -> 2}				%fili seq ? Gamma'
\deftwocell[mid = {?}\Delta]{topDelw : 0 -> 2}				%fili seq ? Delta
\deftwocell[mid = {?}\Delta']{topDel1w : 0 -> 2}				%fili seq ? Delta'
\deftwocell[mid = \sigma{(} {\Gamma } {)}]{topsiG : 0 -> 2}		%fili seq Gamma
\deftwocell[mid = \Sigma]{topSig : 0 -> 2}					%fili seq Sigma
\deftwocell[mid = \Delta]{topDel : 0 -> 2}					%fili seq Delta
\deftwocell[mid = L]{topL : 0 -> 1}						%top filo L
\deftwocell[mid = R]{topR : 0 -> 1}						%top filo R
\deftwocell[mid = W]{topW : 0 -> 2}						%fili seq W
\deftwocell[mid = W_1]{topWb1 : 0 -> 2}						%fili seq W_1
\deftwocell[mid = W_2]{topWb2 : 0 -> 2}						%fili seq W_2
\deftwocell[mid = W']{topW1 : 0 -> 2}						%fili seq W'
\deftwocell[mid = W'_1]{topW1b1 : 0 -> 2}						%fili seq W'_1
\deftwocell[mid = W'_2]{topW1b2 : 0 -> 2}						%fili seq W'_2
\deftwocell[mid = W'']{topW2 : 0 -> 2}						%fili seq W'
\deftwocell[mid = out {(}\alpha{)} ]{outa : 2 -> 2}				%fili output alpha
\deftwocell[mid = in {(}\alpha{)} ]{ina : 2 -> 2}				%fili input alpha
\deftwocell[mid = out {(}x{)} ]{outx : 2 -> 0}				%fili output x
\deftwocell[mid = in {(}x{)} ]{inx : 0 -> 2}				%fili input x



\deftwocell[mid = A]{midA : 1 -> 1}						% filo A
\deftwocell[mid = B]{midB : 1 -> 1}						% filo B
\deftwocell[mid = \Gamma]{midGam : 2 -> 2}					% filo Gamma
\deftwocell[mid = \Gamma']{midGam1 : 2 -> 2}				% filo Gamma'
\deftwocell[mid = \Delta]{midDel : 2 -> 2}					% filo Delta
\deftwocell[mid = \Delta']{midDel1 : 2 -> 2}					% filo Delta'
\deftwocell[mid = \Sigma]{midSig : 2 -> 2}					% filo Sigma
\deftwocell[mid = \Sigma']{midSig1 : 2 -> 2}					% filo Sigma'

\deftwocell[left = \Delta]{leftDel : 1 -> 1}					% left filo Delta'
\deftwocell[left = \Sigma]{leftSig : 1 -> 1}					% left filo Sigma
\deftwocell[left =\wn \Gamma]{leftGamw : 1 -> 1}				% left filo ? Gamma

\deftwocell[mid = ~]{topV : 0 -> 1}						%top vuoto
\deftwocell[mid = ~]{pitV : 1 -> 0}						%pit vuoto
\deftwocell[mid = A]{pitA : 1 -> 0}						%pit filo A
\deftwocell[mid = {?}A]{pitAw : 1 -> 0}						%pit filo ?A
\deftwocell[mid = {!}A]{pitAo : 1 -> 0}						%pit filo !A
\deftwocell[mid = A^\bot]{pitAb : 1 -> 0}					%pit filo A^\bot
\deftwocell[mid = A^{\bot\bot}]{pitAbb : 1 -> 0}					%pit filo A^\bot\bot
\deftwocell[mid = B]{pitB : 1 -> 0}						%pit filo B
\deftwocell[mid = {?}B]{pitBw : 1 -> 0}						%pit filo ?B
\deftwocell[mid = {!}B]{pitBo : 1 -> 0}						%pit filo !B
\deftwocell[mid = C]{pitC : 1 -> 0}						%pit filo C
\deftwocell[mid = D]{pitD : 1 -> 0}						%pit filo D
\deftwocell[mid = {!}C]{pitCo : 1 -> 0}						%pit filo !C
\deftwocell[mid = \bot]{pitbot : 1 -> 0}						%pit filo bot
\deftwocell[mid = 1]{pit1 : 1 -> 0}						%pit filo bot
\deftwocell[mid = L]{pitL : 1 -> 0}						%pit filo L
\deftwocell[mid = R]{pitR : 1 -> 0}						%pit filo R
\deftwocell[mid = \alpha]{pitAl : 1 -> 0}						%pit alpha
\deftwocell[mid = \beta]{pitBe : 1 -> 0}						%pit beta

\deftwocell[mid = A \otimes {B}]{pitAtenB : 1 -> 0}				%pit filo A \otimes B
\deftwocell[mid = {(}A^\bot \otimes {B^\bot}{)}^\bot]{pitAbtenBbb : 1 -> 0}				%pit filo (Ab \otimes Bb)\bot

\deftwocell[mid = A \parr {B}]{pitAparB : 1 -> 0}				%pit filo A \parr B
\deftwocell[mid = B \otimes {C}]{pitBtenC : 1 -> 0}				%pit filo B \otimes C
\deftwocell[mid = B \parr {C}]{pitBparC : 1 -> 0}				%pit filo B \otimes C
\deftwocell[mid = \Gamma]{pitGam : 2 -> 0}					%fili seq Gamma
\deftwocell[mid = \Gamma']{pitGam1 : 2 -> 0}					%fili seq Gamma'
\deftwocell[mid = \Gamma'']{pitGam2 : 2 -> 0}					%fili seq Gamma''
\deftwocell[mid = \Gamma''']{pitGam3 : 2 -> 0}					%fili seq Gamma'''
\deftwocell[mid = {?}\Gamma]{pitGamw : 2 -> 0}				%fili seq \wn Gamma
\deftwocell[mid = {?}\Gamma']{pitGam1w : 2 -> 0}				%fili seq \wn Gamma
\deftwocell[mid = {?}\Gamma]{pitGamw0 : 0 -> 0}				%fili seq \wn Gamma
\deftwocell[mid = {?}\Delta]{pitDelw : 2 -> 0}					%fili seq \wn Delta
\deftwocell[mid = {?}\Delta']{pitDel1w : 2 -> 0}				%fili seq \wn Delta'
\deftwocell[mid = \Delta]{pitDel  : 2 -> 0}					%fili seq Delta
\deftwocell[mid = \Delta']{pitDel1  : 2 -> 0}					%fili seq Delta'
\deftwocell[mid = \Sigma]{pitSig : 2 -> 0}					%fili seq Sigma
\deftwocell[mid = \Sigma']{pitSig1 : 2 -> 0}					%fili seq Sigma'
\deftwocell[mid = \sigma{(} {\Gamma } {)}]{pitsiG : 2 -> 0}			%fili seq sigma(Gamma)
\deftwocell[mid = W]{pitW : 2 -> 0}						%fili seq W
\deftwocell[mid = W_1]{pitWb1 : 2 -> 0}						%fili seq W_1
\deftwocell[mid = W_2]{pitWb2 : 2 -> 0}						%fili seq W_2
\deftwocell[mid = W']{pitW1 : 2 -> 0}						%fili seq W'
\deftwocell[mid = W'_1]{pitW1b1 : 2 -> 0}						%fili seq W'_1
\deftwocell[mid = W'_2]{pitW1b2 : 2 -> 0}						%fili seq W'_2
\deftwocell[mid = W'']{pitW2 : 2 -> 0}						%fili seq W'

\deftwocell[text=\phi, white]{phi11 : 1 -> 1} 					%generic gate	phi 1 -> 1
\deftwocell[mid = F]{topF : 0 -> 1}						%top filo G
\deftwocell[mid = G]{topG : 0 -> 1}						%top filo F
\deftwocell[mid = G]{pitG : 1 -> 0}						%pit filo F
\deftwocell[mid = G\circ {F}]{pitGcircF : 1 -> 0}				%pit filo F






              %%%%%%%%M LL %%%%%%%
\deftwocell[text=\otimes,yellow]{ten : 2 -> 1}		%tensore
\deftwocell[rectangle]{ax : 0 -> 2}				%assioma
\deftwocell[rectangle]{cut : 2 -> 0}				%cut
\deftwocell[text=\parr, orange]{par : 2 -> 1}			%par
\deftwocell[circle, white]{v : 0 -> 1}				%1
\deftwocell[circle, black]{bot : 0 -> 1}					%bot


\deftwocell[text=A,lightgray]{axA : 0 -> 2}				%assioma A
\deftwocell[text=A^\bot,lightgray]{axAb : 0 -> 2}			%assioma A\bot

\deftwocell[text=A,lightgray]{cutA : 2 -> 0}				%cut  A
\deftwocell[text=A^\bot,lightgray]{cutAb : 2 -> 0}			%cut  A\bot




\deftwocell[text=?,white]{D : 1 -> 1}				%dereliction
\deftwocell[text=?,white]{C : 2 -> 1}				%contraction
\deftwocell[text=?,white]{W : 0 -> 1}				%weakeneng
\deftwocell[text=!,white]{P : 3 -> 3}				%box
\deftwocell[text=!,white]{P4 : 4  -> 4}				%box
\deftwocell[text=!,white]{P5 :5 -> 5}				%box
\deftwocell[text=!,white]{P1 :1 -> 1}				%box
\deftwocell[text=!,white]{P6 : 6 -> 6}				%box
\deftwocell[text=!,white]{P7 : 7 -> 7}				%box
\deftwocell[text=!,white]{P8 : 8 -> 8}				%box
\deftwocell[text=!,white]{P9 : 9 -> 9}				%box


\deftwocell[rectangle]{axt : 2 -> 4}			%assioma twist
\deftwocell[rectangle]{cutt : 4 -> 2}			%cut twist


%\deftwocell[rectangle,purple]{varl : 3 -> 3}			%variabile per il cutcontrol pre
%\deftwocell[rectangle]{varlt : 4 -> 3}			%variabile per il cutcontrol

                            %%%%%%%Ladders %%%%%%%

              %%%%%%%%MLLcont %%%%%%%
\deftwocell[text=\otimes,yellow]{tenc : 4 -> 1}		%tensore
\deftwocell[rectangle]{axc : 0 -> 4}			%assioma
\deftwocell[rectangle]{cutc : 4 -> 0}			%cut
\deftwocell[text=A {\parr}{B},lightgray]{cutcAparB : 4 -> 0}			%cut
\deftwocell[rectangle,white]{vc : 0 -> 3}		%1

\deftwocell[text=A,lightgray]{axcA : 0 -> 4}					%assioma A
\deftwocell[text=A^\bot,lightgray]{axcAb : 0 -> 4}				%assioma A\bot
\deftwocell[text=A^{\bot\bot},lightgray]{axcAbb : 0 -> 4}				%assioma A\bot
\deftwocell[text=A {\otimes} {B},lightgray]{axcAtenB : 0 -> 4}					%assioma Atensor B
\deftwocell[text=B^\bot {\otimes} {A^\bot},lightgray]{axcBbtenAb : 0 -> 4}					%assioma Atensor B

%\deftwocell[text=\otimes,yellow]{tenct : 5 -> 2}			%assioma twist
%\deftwocell[rectangle]{cutct : 6 -> 2}			%cut twist 



\deftwocell[lefthalfcircle,red]{L : 1 -> 1}			%cut
\deftwocell[righthalfcircle,blue]{R : 1 -> 1}			%cut



               %%%%% Diagrammi elementari%%%%%%
\deftwocell[text=AX, lightgray]{elax : 2 -> 2} 			%ax	
\deftwocell[text=Cut, lightgray]{elcut : 2 -> 2} 			%cut
\deftwocell[text=c,brown]{tenpar : 2 -> 2} 				%tenpar
%
%
%
%\deftwocell[rectangle, blue]{bluecut : 2 -> 0} 	%tenpar
%%%%%%%%%%%%%%%



%%%%%%%%%%%%%%%%%%%%%%%%%

\newcommand{\matteo}[1]{\begin{center} COMPLETARE: {#1 } \end{center}}

