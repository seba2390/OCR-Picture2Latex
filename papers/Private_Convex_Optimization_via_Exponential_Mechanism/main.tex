\documentclass[11pt]{article}
%\usepackage{natbib,alife13}
\usepackage{amsmath,stackrel}
\usepackage{amsthm}
\allowdisplaybreaks
\usepackage{amssymb}
%\usepackage{algorithmic}
\usepackage{algorithm}
\usepackage[lined,boxed,ruled,norelsize,algo2e,linesnumbered]{algorithm2e}
\usepackage{subfig}

\usepackage{array}
\usepackage{color}
\usepackage[dvipsnames]{xcolor}
%\usepackage[english]{babel}
\usepackage{graphicx}
%\usepackage{natbib}
%\usepackage[pdf]{pstricks}
\usepackage{wrapfig,epsfig}
%\usepackage{psfrag}
\usepackage{epstopdf}
\usepackage{url}
\usepackage{graphicx}
\usepackage{color}
\usepackage{epstopdf}
\usepackage{algpseudocode}
\usepackage{scrextend}
%\usepackage{subfigure}
%\usepackage[hidelinks,pdfencoding=auto,psdextra]{hyperref} %%% commet for arxiv
\usepackage[T1]{fontenc}
\usepackage{bbm}
\usepackage{comment}
\usepackage{authblk}
\usepackage{multicol}
\usepackage{dsfont}
\usepackage{mathtools}
\usepackage{enumitem}
\usepackage{mathrsfs}
%\usepackage{mathabx}

 %%% print refs in table of contents
\let\C\relax
\usepackage{tikz}
\usepackage{hyperref}  
\hypersetup{colorlinks=true,citecolor=blue,linkcolor=red} 



\usetikzlibrary{arrows}
%\usepackage[lmargin=1in,rmargin=1in,tmargin=0.8in,bmargin=0.8in]{geometry}
\usepackage[margin=1in]{geometry}
%\linespread{1}

\graphicspath{{./figs/}}

\definecolor{b2}{RGB}{51,153,255}
\definecolor{mygreen}{RGB}{80,180,0}

\newcommand{\Yintat}[1]{\textcolor{b2}{[Yintat: #1]}}
\newcommand{\Gopi}[1]{\textcolor{red}{[Gopi: #1]}}
\newcommand{\Daogao}[1]{\textcolor{mygreen}{[Daogao: #1]}}


\newcommand{\tabincell}[2]{\begin{tabular}{@{}#1@{}}#2\end{tabular}}



\theoremstyle{plain}
\newtheorem{theorem}{Theorem}[section]
\newtheorem{lemma}[theorem]{Lemma}
\newtheorem{lem}[theorem]{Lemma}
\newtheorem{thm}{Theorem}[section]
\newtheorem{proposition}[theorem]{Proposition}
\newtheorem{fact}[theorem]{Fact}
\newtheorem{claim}[theorem]{Claim}
\newtheorem{corollary}[theorem]{Corollary}
\newtheorem{conjecture}[theorem]{Conjecture}
\newtheorem{assumption}[theorem]{Assumption}
\newtheorem{condition}[theorem]{Condition}
\newtheorem{hypothesis}[theorem]{Hypothesis}
\newtheorem{question}[]{Question}
\newtheorem{notation}[theorem]{Notation}



\theoremstyle{definition}
\newtheorem{definition}[theorem]{Definition}
\newtheorem{example}[theorem]{Example}
\newtheorem{problem}[theorem]{Problem}
\newtheorem{open}[theorem]{Open Problem}
\newtheorem{observation}[theorem]{Observation}



\theoremstyle{remark}
\newtheorem{remark}[theorem]{Remark}

%\newtheorem{proof}[theorem]{Proof}
%\newtheorem{claimproof}{Proof of Claim}[claim]


\newcommand{\wh}{\widehat}
\newcommand{\wt}{\widetilde}
\newcommand{\ov}{\overline}
\newcommand{\eps}{\varepsilon}
\renewcommand{\epsilon}{\varepsilon}
\renewcommand{\phi}{\varphi}

% \renewcommand{\O}{\mathcal{O}}
%\newcommand{\cA}{\mathcal{A}}
\newcommand{\ALG}{\mathrm{ALG}}
% \newcommand{\B}{\mathcal{B}}
% \newcommand{\D}{\mathrm{D}}
% \newcommand{\I}{\mathbb{I}}
% \newcommand{\N}{\mathcal{N}}
% \newcommand{\R}{\mathbb{R}}
% \newcommand{\F}{\mathcal{F}}
\newcommand{\HF}{\hat{F}}
\newcommand{\Fnr}{\hat{F}_{n_r}}
% \newcommand{\K}{\mathcal{K}}
% \newcommand{\Z}{\mathbb{Z}}
\newcommand{\volume}{\mathrm{volume}}
\newcommand{\RHS}{\mathrm{RHS}}
\newcommand{\LHS}{\mathrm{LHS}}
\newcommand{\prox}{\mathrm{prox}}
\newcommand{\calP}{\mathcal{P}}

\renewcommand{\i}{\mathbf{i}}
\renewcommand{\hat}{\wh}
\renewcommand{\bar}{\ov}
\renewcommand{\d}{\mathrm{d}}
\renewcommand{\P}{\mathbf{Pr}} 


\newcommand{\ACSA}{\ensuremath{\mathsf{AC-}}\ensuremath{\mathsf{SA}}\xspace}
\newcommand{\ISGD}{\ensuremath{\mathrm{ISGD}}\xspace}
\newcommand{\META}{\ensuremath{\mathsf{META}}\xspace}
\newcommand{\METADP}{\ensuremath{\mathsf{META_{DP}}}\xspace}
\newcommand{\aux}{\ensuremath{\mathbf{aux}}}
\newcommand{\homega}{\hat{\omega}}
\newcommand{\oomega}{\overline{\omega}}
\newcommand{\TV}{\mathrm{TV}}
\newcommand{\Ind}{\mathbf{1}}
\newcommand{\n}{\mathbf{n}}
\newcommand{\erf}{\mathrm{erf}}
\newcommand{\conv}{\mathrm{conv}}
\newcommand{\G}{\mathcal{G}}
\newcommand{\Proc}{\mathrm{Proc}}

\newcommand{\defeq}{\stackrel{{\text{def}}}{=}}

\DeclareMathOperator*{\E}{\mathbb{E}}
\DeclareMathOperator*{\M}{\mathcal{M}}
\DeclareMathOperator*{\Var}{\mathbf{Var}}
%\DeclareMathOperator*{\var}{\mathrm{Var}}
%\DeclareMathOperator*{\Z}{\mathbb{Z}}
\newcommand{\NP}{\mathbb{N}_{\geq 1}}
%\DeclareMathOperator*{\C}{\mathbb{C}}
\DeclareMathOperator*{\AND}{\mathrm{AND}}
\DeclareMathOperator*{\OR}{\mathrm{OR}}
\DeclareMathOperator*{\argmin}{argmin}
\DeclareMathOperator*{\argmax}{argmax}
\DeclareMathOperator*{\median}{median}
\DeclareMathOperator*{\mean}{mean}
\DeclareMathOperator{\OPT}{OPT}
\DeclareMathOperator{\supp}{supp}
%\DeclareMathOperator{\head}{head}
%\DeclareMathOperator{\tail}{tail}
\DeclareMathOperator{\sparse}{sparse}
\DeclareMathOperator{\poly}{poly}
\DeclareMathOperator{\nnz}{nnz}
\DeclareMathOperator{\loc}{loc}
\DeclareMathOperator{\ary}{ary}
\DeclareMathOperator{\repeats}{repeat}
\DeclareMathOperator{\heavy}{heavy}
\DeclareMathOperator{\emp}{emp}
\DeclareMathOperator{\est}{est}
\DeclareMathOperator{\sparsity}{sparsity}
\DeclareMathOperator{\rank}{rank}
\DeclareMathOperator{\dist}{dist}
\DeclareMathOperator{\rect}{rect}
\DeclareMathOperator{\sinc}{sinc}
\DeclareMathOperator{\Gram}{Gram}
\DeclareMathOperator{\Gaussian}{Gaussian}
\DeclareMathOperator{\dis}{dis}
\DeclareMathOperator{\Sym}{Sym}
\DeclareMathOperator{\Comb}{Comb}
\DeclareMathOperator{\Dirac}{Delta}
\DeclareMathOperator{\signal}{signal}
\DeclareMathOperator{\cost}{cost}
\DeclareMathOperator{\vect}{vec}
\DeclareMathOperator{\tr}{tr}
\DeclareMathOperator{\RAM}{RAM}
\DeclareMathOperator{\diag}{diag}
\DeclareMathOperator{\Err}{Err}

\DeclareMathOperator{\round}{round}
\DeclareMathOperator{\coll}{coll}
\DeclareMathOperator{\off}{off}
\DeclareMathOperator{\unif}{Unif}

\DeclareMathAlphabet{\mathpzc}{OT1}{pzc}{m}{it}


\DeclarePairedDelimiterX{\xdivergence}[2]{(}{)}{%
  #1\;\delimsize\|\;#2%
}
\newcommand{\deltacurve}{\delta\xdivergence*}
\newcommand{\tradeoff}{T\xdivergence*}
\newcommand{\Ent}{\mathrm{Ent}}

%!TEX root=./main.tex


\newcommand{\lp}{\left(}
\newcommand{\lb}{\left[}
\newcommand{\lc}{\left\{}

\newcommand{\rp}{\right)}
\newcommand{\rb}{\right]}
\newcommand{\rc}{\right\}}


%%%  BOLD SYMBOLS

\def\ba{{\mathbf a}}
\def\bb{{\mathbf b}}
\def\bc{{\mathbf c}}
\def\bd{{\mathbf d}}
\def\be{{\mathbf e}}
\def\bh{{\mathbf h}}
\def\bi{{\mathbf i}}
\def\bq{{\mathbf q}}
\def\br{{\mathbf r}}
\def\bu{{\mathbf u}}
\def\bv{{\mathbf v}}
\def\bw{{\mathbf w}}
\def\bx{{\mathbf x}}
\def\by{{\mathbf y}}
\def\bz{{\mathbf z}}

%%%  BOLD GREEK SYMBOLS

\def\bgamma{{\boldsymbol\gamma}}
\def\blambda{{\boldsymbol\lambda}}

%%% HAT SYMBOLS
\def\hf{{\hat{f}}}
\def\hg{{\hat{g}}}
\def\hh{{\hat{h}}}
\def\hcD{{\widehat{\cD}}}
\def\hH{{\widehat{H}}}
\def\hpi{\hat{\pi}}
\def\hx{\hat{x}}
\def\hv{\hat{v}}


%%% TILDE SYMBOLS

\def\tf{{\tilde{f}}}
\def\tg{{\tilde{g}}}
\def\TF{{\Tilde{F}}}
\def\tpi{\Tilde{\pi}}
%\def\tcD{{\tilde{\cD}}}

%%% BAR SYMBOLS
\def\barf {\bar{f}}
\def\barg{\bar{g}}
\def\barS{\bar{S}}

%%% REALS, NATURALS, INTEGERS... %%%

\newcommand{\R}{\mathbb{R}} % REALS
\newcommand{\Q}{\mathbb{Q}} % RATIONALS
\newcommand{\C}{\mathbb{C}} % COMPLEX NUMBERS
\newcommand{\Z}{\mathbb{Z}} % INTEGERS
\newcommand{\nnZ}{\mathbb{Z}_{\ge 0}} % INTEGERS
\newcommand{\N}{\mathbb{N}} % NATURAL NUMBERS
\newcommand{\T}{\mathbb{T}} % TORUS
\newcommand{\boundedC}{\C_{\le 1}} %COMPLEX UNIT DISK
\def\simplex{{\blacktriangle}} % SIMPLEX


%%% FINITE FIELDS
\newcommand{\F}{\mathbb{F}}
\newcommand{\K}{\mathbb{K}}
\newcommand{\PF}{\mathbb{P}\mathbb{F}} %PROJECTIVE F
\def\PK{{\mathbb{P}\mathbb{K}}} % PROJECTIVE K
\def\PL{{\mathbb{P}\mathbb{L}}} % PROJECTIVE L
\def\bK{{\overline{\mathbb{K}}}} % ALGEBRAIC CLOSURE OF K (BAR K)
\def\PbK{{\mathbb{P}\overline{\mathbb{K}}}} % PROJECTIVE ALGEBRAIC CLOSURE OF K (BAR K)


%%% MATHBB ALPHABET
\newcommand{\bbA}{\mathbb A}
\newcommand{\bbB}{\mathbb B}
\newcommand{\bbS}{\mathbb S}
\newcommand{\bbI}{\mathbb I}
\newcommand{\bbP}{\mathbb P}
\newcommand{\bbU}{\mathbb U}
\newcommand{\bbV}{\mathbb V}
\newcommand{\bbL}{\mathbb L}


%%% MATHCAL ALPHABET
\newcommand{\cA}{\mathcal A}
\newcommand{\cB}{\mathcal B}
\newcommand{\cC}{\mathcal C}
\newcommand{\cD}{\mathcal D}
\newcommand{\cE}{\mathcal E}
\newcommand{\cF}{\mathcal F}
\newcommand{\cG}{\mathcal G}
\newcommand{\cH}{\mathcal H}
\newcommand{\cI}{\mathcal I}
\newcommand{\cJ}{\mathcal J}
\newcommand{\cK}{\mathcal K}
\newcommand{\cL}{\mathcal L}
\newcommand{\cM}{\mathcal M}
\newcommand{\cN}{\mathcal N}
\newcommand{\cO}{\mathcal O}
\newcommand{\cP}{\mathcal P}
\newcommand{\cQ}{\mathcal Q}
\newcommand{\cR}{\mathcal R}
\newcommand{\cS}{\mathcal S}
\newcommand{\cT}{\mathcal T}
\newcommand{\cU}{\mathcal U}
\newcommand{\cV}{\mathcal V}
\newcommand{\cW}{\mathcal W}
\newcommand{\cX}{\mathcal X}
\newcommand{\cY}{\mathcal Y}
\newcommand{\cZ}{\mathcal Z}

%%% MATHSF ALPHABET

\newcommand{\sA}{\mathsf A}
\newcommand{\sB}{\mathsf B}
\newcommand{\sC}{\mathsf C}
\newcommand{\sD}{\mathsf D}
\newcommand{\sE}{\mathsf E}
\newcommand{\sF}{\mathsf F}
\newcommand{\sG}{\mathsf G} 
\newcommand{\sH}{\mathsf H} 
\newcommand{\sI}{\mathsf I} 
\newcommand{\sJ}{\mathsf J} 
\newcommand{\sK}{\mathsf K}
\newcommand{\sL}{\mathsf L}
\newcommand{\sM}{\mathsf M}
\newcommand{\sN}{\mathsf N}
\newcommand{\sO}{\mathsf O}
\newcommand{\sP}{\mathsf P}
\newcommand{\sQ}{\mathsf Q}
\newcommand{\sR}{\mathsf R}
\newcommand{\sS}{\mathsf S}
\newcommand{\sT}{\mathsf T}
\newcommand{\sU}{\mathsf U}
\newcommand{\sV}{\mathsf V}
\newcommand{\sW}{\mathsf W}
\newcommand{\sX}{\mathsf X}
\newcommand{\sY}{\mathsf Y}
\newcommand{\sZ}{\mathsf Z}







%%% OPERATORS
%\DeclarePairedDelimiter\ceil{\lceil}{\rceil} % CEIL
%\DeclarePairedDelimiter\floor{\lfloor}{\rfloor} % FLOOR
\newcommand{\bigexp}[1]{\exp\left(#1\right)} % BIG EXPONENTIAL
\def\vc{{\mathrm{vc}}} % 	VC DIMENSION
\newcommand{\sgn}{\mathrm{sgn}} % SIGN
\def\Li{{\mathrm{Li}}} % LOGARITHMIC INTEGRAL
%\newcommand{\supp}{\mathrm{supp}} % SUPPORT
\newcommand{\indicator}{\mathbbm{1}} % INDICATOR
%\newcommand{\argmin}{\mathrm{argmin}} % ARG MIN
%\newcommand{\dist}{\mathrm{dist}_H} % HAMMING DISTANCE
%\newcommand{\wt}{\mathrm{wt}} % WEIGHT
\newcommand{\abs}[1]{\left|#1\right|}
\newcommand{\sign}{\mathrm{sign}}

%%% MATRIX OPERATORS
\newcommand{\adj}{\mathrm{adj}} % ADJOINT


%%% LINEAR ALGEBRA/ALGEBRA
\newcommand{\inpro}[2]{\left\langle #1,#2 \right\rangle} % INNER PRODUCT
\newcommand{\ideal}[1]{\langle #1 \rangle} % IDEAL GENERATED BY
\newcommand{\kernel}{\mathrm{ker}} % KERNEL
\newcommand{\linearspan}[1]{\mathrm{span}\{#1\}} % LINEAR SPAN
\newcommand{\img}{\mathrm{Im}} % IMAGE 
\newcommand{\rk}{\mathrm{rank}} % RANK
\newcommand{\Tr}{{\rm Tr}}



%%% NORMS
\newcommand{\smallnorm}[1]{\lVert #1\rVert} % ||X|| with non adjusting norms
\newcommand{\norm}[1]{\left\lVert #1\right\rVert} % ||X||
%\newcommand{\spnorm}[1]{\left \lVert #1 \right \rVert}  % SPECTRAL NORM
\newcommand{\specnorm}[1]{\left\lVert #1 \right\rVert_{S_{\infty}}} % SPECTRAL NORM
\newcommand{\elltwo}[1]{\norm{#1}_{\ell_2}} % ELL_2 NORM
\newcommand{\ellnorm}[2]{\norm{#1}_{\ell_{#2}}} % ELL_p NORM
%\DeclareMathOperator{\spec}{sp}
\newcommand{\Anorm}[1]{\norm{#1}_{A}} % ALGEBRA NORM
\newcommand{\Unorm}[2]{\|#1\|_{U^{#2}}}






%%% ASYMPTOTIC NOTATION
\newcommand{\bigo}[1]{O\left(#1\right)}

%%% POLY, POLYLOG...
\newcommand{\polylog}{\mathrm{polylog}}
%\newcommand{\poly}{\mathrm{poly}}



%%% SETS AND SPECIAL SETS
% \newcommand{\set}[1]{\{#1\}}
\newcommand{\Set}[1]{\left\{#1\right\}}
\newcommand{\bits}{\set{0,1}}
\newcommand{\sbits}{\set{-1,1}}
\newcommand{\st}{:\,} % "such that" to define sets

%%% VECTORS AND MATRICES

%\newcommand{\vect}[2]{#1_1,\cdots,#1_#2} % X_1,..., X_Y

\newcommand{\mattwoone}[2]{
\left[
\begin{matrix} 
#1\\
#2\\
\end{matrix}\right]
} % 2 x 1 MATRIX

\newcommand{\matthreeone}[3]{
\left[
\begin{matrix}
#1\\
#2\\
#3\\
\end{matrix}
\right]
} % 3 x 1 MATRIX

\newcommand{\mattwotwo}[4]{
\left[
\begin{matrix}
#1& #2\\
#3& #4\\
\end{matrix}
\right]
} % 2 x2 MATRIX





%%% REDEFINE EPSILON TO BE VAREPSILON
\renewcommand{\epsilon}{\varepsilon}
%\newcommand{\eps}{\epsilon}

%% ENVIRONMENTS
  \newcommand{\beq}{\begin{equation}}
  \newcommand{\eeq}{\end{equation}}
  \newcommand{\beqn}{\begin{equation*}}
  \newcommand{\eeqn}{\end{equation*}}
  \newcommand{\beqr}{\begin{eqnarray}}
  \newcommand{\eeqr}{\end{eqnarray}}
  \newcommand{\beqrn}{\begin{eqnarray*}}
  \newcommand{\eeqrn}{\end{eqnarray*}}
  \newcommand{\bmline}{\begin{multline}}
  \newcommand{\emline}{\end{multline}}
  \newcommand{\bmlinen}{\begin{multline*}}
  \newcommand{\emlinen}{\end{multline*}}

%%% MISCELLANEOUS

\newcommand{\eqdef}{\stackrel{\rm{def}}{=}} % EQUALITY BY DEFINITION
\newcommand{\ds}{\displaystyle} % DISPLAY MATH IN TEXT AS IN EQUATION SYTLE
\newcommand{\Emph}[1]{\colorbox{Yellow}{\textbf{#1}}}





%%% PROBABILITY
%\DeclareMathOperator{\E}{\mathbb{E}} % EXPECTATION
\def\corr{{\mathrm{Corr}}} % CORRELATION
\newcommand{\var}{\mathrm{Var}} % VARIANCE
% \Pr = Pr is already defined for Probability



% \usepackage{todonotes}

% \newcounter{mynotes}
% \setcounter{mynotes}{0}
% \newcommand{\mnote}[1]{\addtocounter{mynotes}{1}{{}}%
% % \todo[color=blue!20!white]{[\arabic{mynotes}] \scriptsize  {{\sf {#1}}}}} 

% \def\notes{1}
% \newcommand{\gnote}[1]{\ifnum\notes=1{{\sf\color{blue} [Gopi: #1]}}\fi}
% \newcommand{\ynote}[1]{\ifnum\notes=1{{\sf\color{red} [YinTat: #1]}}\fi}


\title{Private Convex Optimization via Exponential Mechanism}
%\title{Gaussian Differential Privacy of Exponential Mechanism}
%\title{Regularized Exponential Mechanism Yields Approximate DP}
\author{
Sivakanth Gopi\thanks{Microsoft Research. Email: \texttt{sigopi@microsoft.com}}
\quad
Yin Tat Lee \thanks{University of Washington and Microsoft Research. Email: \texttt{yintat@uw.edu}}
\quad
Daogao Liu \thanks{University of Washington. Email: \texttt{dgliu@uw.edu}}
}
%\date{August 2021}
\date{}
\begin{document}



\begin{titlepage}
\maketitle
\begin{abstract}
    \begin{abstract}
\label{sec:abstract}

%% 1. what is the problem 
Scientific applications that run on leadership computing facilities often face the challenge 
of being unable to fit leading science cases onto accelerator devices due to memory constraints 
(memory-bound applications).
%
% 2. what is your solution 
In this work, the authors studied one such US Department of Energy mission-critical condensed matter 
physics application, Dynamical Cluster Approximation (DCA++), and this paper discusses how device memory-bound challenges were successfully reduced  by proposing an effective 
``all-to-all'' communication method---a ring communication algorithm. 
%
This implementation takes advantage of acceleration on GPUs and remote direct memory access (RDMA) for fast data exchange between GPUs. 
%
\\Additionally, the ring algorithm was optimized with sub-ring communicators
and multi-threaded support to further reduce communication overhead and 
expose more concurrency, respectively.
%
% 3. What's the cherry-picked evaluation result you want to mention
The computation and communication were also analyzed 
by using the Autonomic Performance Environment for Exascale 
(APEX) profiling tool,  and this paper further discusses the 
performance trade-off for the ring algorithm implementation. 
%
The memory analysis on the ring algorithm shows that the allocation size for the authors' most 
memory-intensive data structure per GPU is now reduced to $1/p$ of the original size, where $p$ is the number of GPUs in the ring communicator.
%
The communication analysis suggests that 
the distributed Quantum Monte Carlo execution time grows linearly as sub-ring size increases, and the cost of messages passing through the network interface connector could be a limiting factor.


%
% \todoRed{Ronnie: Next sentence needs rewrite, too much information about Green's function that no one knows in the abstract; recommend generalizing.} \emph {However, DCA++ is currently facing memory-bound challenge as 
% a larger device array $G_t$ is limited by device memory size, where
% $G_t$ is a two-particle Green's function that allows condensed matter
% scientists to explore larger and more complex (higher fidelity)
% physics cases.}

\end{abstract}

\keywords{DCA++, Quantum Monte Carlo, GPU Remote Direct Memory Access, memory-bound issue, exascale machines}

\end{abstract}
  \thispagestyle{empty}
\end{titlepage}


{\hypersetup{linkcolor=black}
\tableofcontents
}
\newpage
Reinforcement learning has achieved great success in areas such as Game-playing \citep{silver2018general,vinyals2019grandmaster}, robotics \cite{kober2013reinforcement}, large language models \citep{ouyang2022training}, etc.
However, due to safety concerns or physical limitations, in some real-world reinforcement learning problems, we must consider additional constraints that may influence the optimal policy and the learning process \citep{garcia2015comprehensive}.
% For example, a robotic arm must not take actions that may cause harm to itself or the environments.
A standard framework to handle such cases is the constrained Markov Decision Process (CMDP) \citep{altman1999constrained}.
Within the CMDP framework, the agent has to maximize
the expected cumulative reward while
obeying a finite number of constraints, which are usually in the form of expected cumulative cost criteria.

However, we are sometimes concerned with the problem with a continuum of constraints.
For example,
the constraints we meet might be time-evolving or subject to uncertain parameters, which
cannot be formulated as an ordinary CMDP
(see Examples \ref{Example_Time_Evolving} and  \ref{Example_Uncertain}).
In this paper we would study a generalized CMDP  
to address the above problem.  Because the constraints are not only infinite-number but also lie
in a continuous set,
the generalization is not trivial. Fortunately, we find that we can borrow the idea behind semi-infinite programming (SIP) \citep{remez1934determination, hettich1993semi} to deal with the semi-infinite constraints.
Accordingly, we propose \emph{semi-infinitely constrained Markov decision processes} (SICMDPs)
as a novel complement to the ordinary CMDP framework.
%More specifically,  an SICMDP model %, we consider 
%contains a continuum of constraints whereas an ordinary CMDP contains a finite number of constraints. 

%This generalization is natural but not trivial. However, we can brows the idea  
%The idea is quite natural and can be backtracked
%to the practice of extending linear programming to linear semi-infinite programming (LSIP) %\cite{remez1934determination, GobernaLSIO1998}.
%In addition, 
%As a complementary approach to the ordinary CMDP framework, 
%SICMDP can be used to model these problems  which cannot be described by a finite number of constraints
%that are not covered by .
%For example,
%the restrictions we consider can be time-evolving or subject to uncertain parameters
%, thus
%cannot be described by a finite number of constraints but a continuum of constraints 
%(see Examples \ref{Example_Time_Evolving} and  \ref{Example_Uncertain}).

We also present two reinforcement learning algorithms to solve SICMDPs called SI-CRL and SI-CPO, respectively.
SI-CRL is a model-based reinforcement learning algorithm designed for tabular cases, and SI-CPO is a policy optimization algorithm for non-tabular cases.
% and analyze its performance both theoretically and empirically.
The main challenge is that we need to deal with a continuum of constraints, thus reinforcement learning algorithms for ordinary CMDPs do not work anymore.
In SI-CRL, we tackle this difficulty by first transforming the reinforcement learning problem to an equivalent LSIP problem, which can then be solved using methods in the LSIP literature like the dual exchange methods \citep{Hu1990,reemtsen1998numerical}.
In SI-CPO, we resort to the idea of cooperative stochastic approximation developed in \cite{lan2020algorithms, wei2020comirror}.
As far as we know, we are the first to introduce tools from semi-infinitely programming (SIP) into the reinforcement learning community for solving constrained reinforcement learning problems.

% To the best of our knowledge, we are the first to apply tools from semi-infinitely programming (SIP) to solve reinforcement learning problems.
Furthermore, we give theoretical analysis for both SI-CRL and SI-CPO.
We decompose the error of SI-CRL into two parts: the statistical error from approximating the true SICMDP with an offline dataset and the optimization error due to the fact that the solution of the LSIP problem obtained by the dual exchange method is inexact.
On the optimization side, we show that the iteration complexity of SI-CRL is $O\left(\left\{\mathrm{diam}(Y)L\sqrt{|\gS|^2|\gA|m}/\left[(1-\gamma)\epsilon\right]\right\}^m\right)$.
On the statistical side, we show that the sample complexity of SI-CRL is $\widetilde O\left(\frac{|S|^2|A|^2}{\epsilon^2(1-\gamma)^3}\right)$ if the offline dataset is generated by a generative model, and $\widetilde O\left(\frac{|S||A|}{\nu_{\min} \epsilon^2(1-\gamma)^3}\right)$ if the dataset is generated by a probability measure $\nu$ as considered in \cite{chen2019information}.
Here $\widetilde O$ means that all logarithm terms are discarded.
For SI-CPO, things become a little more complicated because other than the statistical error and the optimization error, we also need to consider the function approximation error, which comes from imperfect policy parametrizations.
It is shown if the function approximation error can be controlled to $O(\epsilon)$ order, the iteration complexity of SI-CPO is $\widetilde{O}\left(\frac{1}{\epsilon^2(1-\gamma)^6}\right)$ and the sample complexity of SI-CPO is $\widetilde{O}(\frac{1}{\epsilon^4(1-\gamma)^{10}})$.
Here our iteration complexity bound is equivalent to a typical $\widetilde O(1/\sqrt{T})$ global convergence rate.

We perform a set of numerical experiments to illustrate the SICMDP model and validate our proposed algorithms.
Specifically, we examine two numerical examples, namely the discharge of sewage and ship route planning.
Through the discharge of sewage example, we show the advantage of the SICMDP framework over the CMDP baseline obtained by naive discretization in modeling realistic sequential decision-making problems.
Moreover, we demonstrate the effectiveness of the SI-CRL and SI-CPO algorithms in such tabular environments. 
In the ship route planning example, we illustrate the benefits of the SICMDP framework and the ability of the SI-CPO algorithm to address complex continuous control tasks involving continuous state spaces with modern deep reinforcement learning techniques.

% In summary, our contributions are listed as follows.
% First, we present the SICMDP model, which can be viewed as a generalization of the ordinary CMDP model.
% Second, we propose an algorithm to perform reinforcement learning for SICMDPs, which is called SI-CRL, and we believe that we are the first to apply tools from SIP
% to solve reinforcement learning problems.
% Third, we give a theoretical analysis of SI-CRL and identify both its sample complexity and iteration complexity.
% In addition, we perform numerical experiments to illustrate the SICMDP model and validate the SI-CRL algorithm.
% \{This paragraph can be removed!!! \}





\section{Techniques}

The main contribution of this paper is the discovery that adding regularization terms in exponential mechanism leads to optimal algorithms for DP-ERM and DP-SCO. For this, we develop some important tools that could be of independent interest. We now briefly discuss each of the main tools.

% Our proofs mainly rely on some less well-known techniques and involves surprising twists.\Gopi{This isn't a movie, lol. Need to use more formal language in papers.}
% In this section, we discuss some techniques that maybe of independent interest. 

\subsection{Gaussian Differential Privacy (GDP) of Regularized Exponential Mechanism}

To analyze the privacy of the regularized exponential mechanism, we need to bound the privacy curve between a strongly log-concave distribution and its Lipschitz perturbation in the exponent. \cite{MASN16} gave a nearly tight (up to constants) privacy guarantee of exponential mechanism if the distribution $\exp(-k F(x;\cD))$ satisfies Logarithmic Sobolev inequality (LSI). Since strongly log-concave distributions satisfy LSI, their result immediately gives the $(\epsilon,\delta)$-DP guarantee of our algorithm. However, this gives a sub-optimal privacy bound because it does not fully take advantage of the strongly log-concave property.

Instead, we show directly that the privacy curve between a strongly log-concave distribution and its Lipschitz perturbation in the exponent is upper bounded by the privacy curve of an appropriate Gaussian mechanism. This new proof uses the notion of tradeoff function introduced in~\cite{dong2019gaussian} and the isoperimetric inequality for strongly log-concave distribution.

\begin{theorem}
\label{thm:privacy_technical}
Given convex set $\cK\subseteq \R^d$ and $\mu$-strongly convex functions $F,\Tilde{F}$ over $\cK$. Let $P,Q$ be distributions over $\cK$ such that $P(x)\propto e^{-F(x)}$ and $Q(x)\propto e^{-\Tilde{F}(x)}$.
If $\Tilde{F}-F$ is $G$-Lipschitz over $\cK$, then for all $\eps>0$,
\begin{align*}
    \deltacurve{P}{Q}(\epsilon) 
    \leq \deltacurve{\cN\lp 0,1\rp}{\cN\lp\frac{G}{\sqrt{\mu}},1\rp}(\epsilon).
\end{align*}
\end{theorem}
This proves that the privacy curve for distinguishing between $P,Q$ is upper bounded the privacy curve of a Gaussian mechanism with sensitivity $G/\sqrt{\mu}$ and noise scale 1.

\paragraph{Tightness:} Note that Theorem~\ref{thm:privacy_technical} is completely tight because it contains the privacy of Gaussian mechanism as a special case. If $F(x)=\norm{x}_2^2/2$ and $\Tilde{F}(x)=\norm{x-a}_2^2/2$ for some $a\in \R^d$, then $\Tilde{F}(x)-F(x)=-\inpro{x}{a}+\norm{a}_2^2/2$ is $G$-Lipschitz with $G=\norm{a}_2$ and $F,\Tilde{F}$ are $1$-strongly convex. And $P=\cN(0,I_d)$ and $Q=\cN(a,I_d)$. Therefore:
$$\deltacurve{P}{Q}=\deltacurve{\cN(0,I_d)}{\cN(a,I_d)}=\deltacurve{\cN(0,1)}{\cN\lp\norm{a}_2,1\rp}$$ which is precisely the upper bound guaranteed by the theorem.

%\noindent Theorem~\ref{thm:privacy_technical} implies the following corollaries for DP-ERM.

% Yin Tat: There are too many theorem
%\begin{corollary}
%Suppose $F(x;\cD)$ is convex over some convex domain $\cK\subset \R^d$ with diameter $D$ and $F(x;\cD)-F(x;\cD')$ is $G$-Lipschitz for all neighboring databases $\cD,\cD'.$ Then the mechanism (\ref{eqn:our_mechanism}) satisfies $(\eps,\delta(\eps))$-DP for all $\eps$ where
%\begin{align*}
% \delta(\eps)\leq \deltacurve{\cN(0,1)}{\cN\lp\frac{G\sqrt{k}}{\sqrt{\mu}},1\rp}(\epsilon).
%\end{align*}
%Moreover, the expected excess empirical loss is bounded by %$\frac{d}{k}+\mu D^2$.
%\end{corollary}

% Yin Tat: There are too many theorem
%\begin{corollary}
%Suppose $F(x;\cD)$ is $\mu$-strongly convex over some convex domain $\cK\subseteq \R^d$ with diameter $D$ and $F(x;\cD)-F(x;\cD')$ is $G$-Lipschitz for all neighboring databases $\cD,\cD'.$ Then the exponential mechanism (\ref{eqn:exponential_mechanism}) satisfies $(\eps,\delta(\eps))$-DP for all $\eps$ where
%\begin{align*}
% \delta(\eps)\leq \deltacurve{\cN(0,1)}{\cN\lp\frac{G\sqrt{k}}{\sqrt{\mu}},1\rp}(\epsilon).
%\end{align*}
%Moreover, the expected excess empirical loss is bounded by $\frac{d}{k}$.
%\end{corollary}


%\begin{corollary}
%Suppose that $F(x;\cD)$ is convex and for $F(x;\cD)-F(x;\cD')$ is $G$-Lipschitz for any two neighboring databases $\cD,\cD'$. Then we have following:
%\begin{enumerate}
%    \item Sampling $x^{priv}$ from the density in (\ref{eqn:our_mechanism}) with $$k= \frac{c\eps^2\mu}{G^2\log(1/\delta)} \text{ and } \mu=\frac{G\sqrt{d\log(1/\delta)}}{\eps D},$$ for some sufficiently small constant $c>0$, satisfies $(\eps,\delta)$-DP and achieves the optimal excess risk in (\ref{eqn:optimal_empirical_risk}). 
%    \item If $F(x;\cD)$ is already $\mu$-strongly convex, then sampling from the exponential mechanism in (\ref{eqn:exponential_mechanism}) with $k=\frac{c\eps^2\mu}{G^2\log(1/\delta)}$, for some sufficiently small constant $c>0$, satisfies $(\eps,\delta)$-DP and achieves the optimal excess risk in (\ref{eqn:optimal_empirical_risk_stronglyconvex}).
%\end{enumerate}
% Moreover, in both cases, we achieve $\gamma$-Gaussian Differential Privacy (GDP) with $\gamma=\frac{G\sqrt{k}}{\sqrt{\mu}}$, i.e., $$\deltacurve{\pi_\cD}{\pi_\cD'} \ge \deltacurve{N(0,1)}{N(\gamma,1)}.$$ Or in other words, we get as much privacy as a Gaussian mechanism with sensitivity $\gamma$ and noise scale $1.$
%\end{corollary}



% ================================================================

% Since proposed by \cite{MT07}, the simple exponential mechanism has been a fundamental building block to construct a randomized estimator which satisfies pure differential privacy ($(\eps,0)$-differential privacy).
% For some utility score $F:\cK\times\Xi^{n}\rightarrow \R$ with bounded sensitivity $\Delta F\defeq \max_{x\in\cK}\max_{S,S'}|F(x;S)-F(x;S')|$ where $S$ and $S'$ are neighboring, the exponential mechanism selects and outputs an element $r\in\cR$ with probability  proportional to $\exp(\frac{\eps F(x;S)}{2\Delta F})$ for given data-set $S$.
% With the simple structure and easy to analyze, exponential mechanism is widely used, such as mechanism design \cite{HK12}, convex optimization \cite{BST14}, statistic \cite{WZ10,WM10,AKR+19}, artificial intelligence \cite{ZP19} and so on.

% Over the past decade, the research on exponential mechanism has never stopped, and many exciting developments have been made.
% Exponential mechanism can be implemented efficiently over a finite data-set of appropriate size, but how to implement it over super polynomial size or even infinite and continuous domain will incur new challenge, which are discussed by \cite{HT10,CSS13,KT13,BV19,CKS20}.
% The utility of the exponential mechanism depends heavily on the worst case sensitivity, but it is very common that there are very few base cases where the sensitivity is very large, and other instances have low sensitivity. There are many interesting results \cite{TS13,BNS13,RS16,LT19} on how to generalize the exponential mechanism and make use of the different sensitivity to get better utility under weaker privacy condition ($(\eps,\delta)$-differential privacy for $\delta>0$). Read a more detail discussion therein \cite{LT19}.

% The simple mechanism suggests a very good and general framework for pure differential privacy. A very natural and important question is whether there is some simple and similar mechanism which can be used to present approximate differential privacy?
% Though pure DP provides stronger privacy protection from definition, the approximate DP also has its advantages.
% The typical polynomial small setting of $\delta=o(1/n)$ is usually good enough to protect privacy in practice, and there are many problems which can have much better utility guarantee under the requirement of approximate DP than pure DP, for example, the private learning and sanitization \cite{BNS13}, the empirical risk minimization \cite{BST14}, and  statistical queries \cite{SU15}.

% Motivated by this, we study the substitution of exponential mechanism for approximate DP, and find there is actually a simple mechanism which can yields approximate DP.
% Roughly speaking, if $F$ is $\mu$-strongly convex and $F(\cdot,S)-F(\cdot,S')$ is $G$-Lipschitz over $x\in\K$, then sampling $x\in \cK$ with the probability proportional to distribution $\exp(-F(x;S))$ is $(\eps,\delta)$-DP where $\delta\leq \deltacurve{\cN\lp 0,1\rp}{\cN\lp\frac{G}{\sqrt{\mu}},1\rp}(\epsilon)$, which is the privacy curve of Gaussian mechanism.
% We think this can be a quite standard result, and the bound in our conclusion is tight.
% To deal with the general case when the objective function is not strongly convex, we can calibrate it with some carefully chosen regularization terms, and that's the reason we name it by Regularization Exponential Mechanism.
% \begin{theorem}
% Maybe give a statement here about our main result.
% \end{theorem}

%Our proof is pretty simple, as well. \Daogao{Say something, Please.}

%Exponential Mechanism (Generalized Exponential mechanism) \cite{RS16,LT19}
%We show that the sample complexity of these tasks under approximate differential privacy can be significantly lower than that under pure differential privacy

\subsection{Generalization Error of Sampling}
Many important and fundamental problems in machine learning, optimization and operations research are special cases of
SCO, and ERM is a classic and widely-used approach to solve it, though their relationships are not well-understood.
If one can solve the ERM problem optimally and get the exact optimal solution $x^*$ to minimizing $F(\cdot;\cD)$ (see Equation~\ref{eq:DPERM}), then \cite{SSSSS09} showed $x^*$ will also be a good solution to the SCO for strongly convex functions.
But in most situations, solving ERM optimally costs too much or even impossible. 
Can we find a approximately good solution to ERM and hope that it is also a good solution for SCO?
\cite{Fel16} provides a negative answer and shows there is no good uniform convergence between $F(\cdot;\cD)$ and $\HF$, that is there always exists $x\in\cK$ such that $|F(x;\cD)-\HF(x)|$ is large.
This fact forces us to find approximate solution to ERM with very high accuracy, which makes the algorithms inefficient.

Prior works proposed a few interesting ways to overcome this difficulty, such as the uniform stability in \cite{HRS16} and the iterative localization technique in \cite{AFKT21}.
Roughly speaking, uniform stability means that if running algorithms on neighboring datasets lead to similar output distributions, then the generalization error of the ERM algorithm is bounded.
Thus a good solution to ERM obtained by a stable algorithm is also a good solution for SCO.
\cite{bftt19} makes use of the stability of running SGD on smooth functions to get a tight bound on the population loss for DP-SCO.


Recall $F(x;\cD)$ and $\HF(x)$ are defined in Equation~\eqref{eq:DPERM} and \eqref{eq:DPSCO} respectively. 
Our result enriches the toolbox of bounding the generalization error and provides new insights for this problem.
\begin{theorem}
Suppose $\{f_i\}$ is a family of $\mu$-strongly convex functions over $\cK$ and $f_i-f_{i'}$ is $G$-Lipschitz for any two functions $f_i,f_{i'}$ in the family.
For any $k>0$ and suppose the $n$ samples in data set $\cD$ are drawn i.i.d from the underlying distribution, then by sampling $x^{(sol)}$ from density $\propto e^{-kF(x^{(sol)};\cD)}$, the population loss satisfies
\begin{align*}
    \E[\HF(x^{(sol)})]-\min_{x\in\cK}\HF(x)\leq \frac{G^2}{\mu n}+ \frac{d}{k}.
\end{align*}
\end{theorem}

Considering two neighboring datasets $\cD$ and $\cD'$, our result is based on bounding the Wasserstein distance between the distributions proportional to $e^{-kF(x;\cD)}$ and $e^{-kF(x;\cD')}$, which means the sampling scheme is stable and leads to the $\frac{G^2}{\mu n}$ term in generalization error.
% The proof makes use of the fact that any divergence measure that decreases under post-processing such as KL divergence.
The other term $\frac{d}{k}$ is excess empirical loss of the sampling mechanism.
One advantage of our result is that it works for both smooth and non-smooth functions. 
Moreover, we may choose the value $k$ carefully and get a solution with both optimal empirical loss and optimal population loss.

% Write the statement of generalization error here (Theorem 6.10, make it as a restatable, change the statement such that it is just about generalization, but not about DP so that other people can use).
% Also, explain the proof Theorem 6.10.


\subsection{Non-smooth Sampling and DP Convex Optimization }
Implementing the exponential mechanism involves sampling from a log-concave distribution.
When the negative log-density function $F$ is smooth, i.e. the gradient of $F$ is Lipschitz,
there are many efficient algorithms for this sampling tasks
such as \cite{D17,LSV18,MMW+19,CV19,DMM19,shen2019randomized,CDW+20,LST20}.
For example, if $F=\frac{1}{n}\sum_{i=1}^{n}f_{i}$
and each $f_{i}$ is $1$-strongly convex with $\kappa$-Lipschitz
gradient,\footnote{For convenience, we used $f_{i}$ to denote the function $f(\cdot;s_i)$ in this and Section \ref{sec:sampling}.} we can sample $x\sim\exp(-F(x))$ in $\widetilde{O}(n+\kappa\max(d,\sqrt{nd})\log(1/\delta))$
iterations with $\delta$ error in total variation distance and each iteration involves computing one $\nabla f_{i}(x)$ \cite{LST21}.
Note that this is nearly linear time when $n\gg\kappa^2d$ and the $\delta$ error in
total variation distance can be translated to an extra $\delta$ error in the $(\epsilon,\delta)$-DP
guarantee.

\begin{center}
\begin{figure}[ht]
\begin{centering}
\begin{tabular}{|c|c|c|c|}
\hline 
 & Complexity & Oracle & Guarantee\tabularnewline
\hline 
\hline 
\cite{BST14} & $d^{O(1)}$ & $F(x)$ & $\mathrm{D}_{\infty}\leq\epsilon$\tabularnewline
\hline 
\cite{CDJB20} & $G^{O(1)}d^{5/2}/\epsilon^{4}$ & $\nabla F(x)$ & $\mathrm{W}_{2}\leq\delta$\tabularnewline
\hline 
\cite{JLLV21} + \cite{C21} & $d^{3}$ & $F(x)$ & $\mathrm{TV}\leq\delta$\tabularnewline
\hline 
\cite{GT20}& $\frac{\alpha^{2}G^{4}d}{\epsilon^{2}}$ & $\nabla F(x)$ & $\mathrm{D}_{\alpha}\leq\epsilon$\tabularnewline
\hline 
\cite{LC21} & $\frac{G^{2}}{\delta}$ & $\nabla F(x)$ & $\mathrm{TV}\leq\delta$\tabularnewline
\hline 
This & $G^{2}$ & $f_{i}(x)$ & $\mathrm{TV}\leq\delta$\tabularnewline
\hline 
\end{tabular}
\par\end{centering}
\caption{The complexity of sampling from $\exp(-F(x))$ where $F=\frac{1}{n}\sum_{i}f_{i}$
is $1$-strongly convex and $f_{i}$ are $G$-Lipschitz and convex.
For applications in differential privacy, $\epsilon$ is a constant
and $\delta=n^{-\Theta(1)}$. Polylogarithmic terms are omitted. Only the last result uses the summation structure and queries only one $f_{i}$ each step. \label{fig:sample_runtime}}
\end{figure}
\vspace{-7mm}
\par\end{center}

Unfortunately, when the functions $f_i$ are only Lipschitz but not smooth, this problem is more difficult.
In Table \ref{fig:sample_runtime}, we summarize some existing results on this topic. They use different guarantees such as Renyi divergence $\mathrm{D}_{\alpha}$ of order $\alpha$, Wasserstein distance $\mathrm{W}_{2}$ and total variation distance $\mathrm{TV}$ (defined in subsection~\ref{sec:dis_measure}). For applications in differential privacy, we need either polynomially small $\mathrm{W}_{2}$ or $\mathrm{TV}$ distance, or $\epsilon$ small $\mathrm{D}_{\alpha}$ distance. 


All previous results for non-smooth function use oracle access to $F$ or $\nabla F$ (instead of $f_i$) and have iterative complexity at least $d$ iterations for $\mathrm{W}_{2}$ or TV distance smaller than $1/d$. Because of this, our algorithm is significantly faster than
the previous algorithms and can handle the case when $F$ is expectation of (infinitely many)  $f_i$ directly.
For example, to get the optimal private empirical loss with typical settings where $\epsilon=\Theta(1)$
and $\delta=1/n^{\Theta(1)}$, the previous best samplers use $\widetilde{O}(n^{4}d)$ many queries to $\nabla f_{i}(x)$ by \cite{GT20} or $\widetilde{O}(nd^{3})$
many queries to $f_{i}(x)$ by combining \cite{JLLV21} and \cite{C21}. 
% The extra $n$ factors are because of the need to choose a large scaling factor $k$ in the distribution $\exp(-kF)$ to make it private. 
In comparison, our algorithm only takes $\widetilde{O}(n^{2})$
many $f_{i}(x)$. 

Our result is based on the alternating sampler proposed in \cite{LST21} and a new rejection sampling scheme.

%{[}Copy the main statement here{]}
%To sample from $\exp(-F(x))$ where $F\defeq\E_{i\in I}f_i+\psi(x)$ where $\psi(x)$ is $\mu$-strongly convex and each $f_i$ is $G$-Lipschitz and convex, our algorithm needs $\Tilde{O}(\frac{G^2}{\mu}\log^2(1/\delta))$ (ignore other logarithmic terms) iterations, and in each iteration it queries $O(1)$ many values of $f_i$ in expectation and finally outputs a sample, total variation distance between whose distribution and the $\exp(-F)$ is bounded by $\delta$.
%{[}restatable of the sampling result{]}
\begin{theorem}
%\label{thm:sampler}
Given a $\mu$-strongly convex function $\psi(x)$ defined on a convex set $\cK \subseteq \R^{d}$ and $+\infty$ outside. Given a family of $G$-Lipschitz convex functions $\{f_{i}(x)\}_{i\in I}$ defined on $\cK$ and an initial point $x_0\in \cK$.
Define the function $\widehat{F}(x)=\E_{i\in I}f_{i}(x)+\psi(x)$ and 
the distance $D=\|x_{0}-x^{*}\|_{2}$ for some $x^{*}=\arg\min_{x\in\cK}\HF(x)$.
For any $\delta\in(0,1/2)$, we can generate a random point $x$ that
has $\delta$ total variation distance to the distribution proportional to $\exp(-\widehat{F}(x))$ in
\[
T:=\Theta\lp\frac{G^{2}}{\mu}\log^{2}\lp\frac{G^{2}(d/\mu+D^{2})}{\delta}\rp\rp\text{ steps}.
\]
Furthermore, each steps accesses only $O(1)$ many $f_{i}(x)$ and samples from $\exp(-\psi(x) - \frac{1}{2\eta} \|x-y\|^2_2)$ for $O(1)$ many $y$
in expectation with $\eta = \Theta(G^{-2}/\log(T/\delta))$.
\end{theorem}


% Given a family of convex functions $\{f(\cdot,s)\}_{s\in\Xi}$ over a convex set $\cK\subseteq \R^d$.
% In the Empirical Risk Minimization (ERM) problem, we are given a data set $S=\{s_1,s_2,\cdots,s_n\}$ over the universe $\Xi$ with the objective to
% \begin{align}\label{eq:DPERM}
%     \text{minimize } F(x;S)\defeq \frac{1}{n}\sum_{s_i\in S}f(x,s_i) \text{   over } x\in \cK
% \end{align}

% And we define the excess empirical loss with respect to a solution $x$ by $F(x;S)-F^*$ where $F^*=\min_{x\in\cK}F(x;S)$.
\begin{comment}
The Stochastic Convex Optimization (SCO) problem is closely related to ERM.
In the SCO problem, we suppose the $n$ samples are drawn i.i.d. from some underlying distribution $\mathcal{P}$ and the objective is to
\begin{align}\label{eq:DPSCO}
    \text{minimize } \hat{F}(x)\defeq \E_{s\sim \calP}f(x,s) \text{   over } x\in \cK.
\end{align}
We refer $\hat{F}(x)-\hat{F}^*$ as the population loss where $\hat{F}^*=\min_{x\in\cK}\hat{F}(x)$.
In the private setting, we want to design $(\epsilon,\delta)$ algorithms for both problems.

As ERM and SCO are two fundamental and important problems in optimization, machine learning and many other related fields,
DP-ERM and DP-SCO have been studied by the DP community and there are many exciting results \cite{CM08,rbht09,cms11,jt14,BST14,kj16,fts17,zzmw17,ins+19,bftt19,FKT20,KLL21,bgn21,AFKT21,sstt21}.

Our main result can lead to the following result about DP-ERM directly. A main improvement of our result are the explicit small constants in the statement.

\begin{theorem}[Informal, DP-ERM]
For $1>\eps>0$ and $1>\delta>0$, and let convex set $s\cK\subset\R^d$ and $\{f(\cdot,s)\}_{s\in\Xi}$ be a family of $G$-Lipschitz functions and $\mu$-strong convex over $\cK$. Setting $k=\frac{\eps^2n^2\mu}{8G^2\log(1.25/\delta)}$, and sampling with probability proportional to $\exp(-kF(x;S))$ is $(\eps,\delta)$-differentially private with expected excess empirical loss $\frac{8G^2d\log(1.25/\delta)}{\eps^2n^2\mu}$.

Moreover, if $\cK$ has diameter $D$ and $\{f(\cdot,s)\}_{s\in\Xi}$ are $G$-Lipschitz and convex, setting $k=\frac{\eps^2n^2\mu}{8G^2\log(1.25/\delta)}$ where $\mu=\frac{4G\sqrt{d\log(1.25/\delta)}}{\eps n D}$, sampling with probability proportional to $\exp(-kF(x;S)+\mu\|x\|_2^2/2)$ is $(\eps,\delta)$-differentially private with expected excess empirical loss $\frac{4GD\sqrt{d\log(1.25/\delta)}}{\eps n}$.
\end{theorem}


% With this sampling scheme, we can have the following efficient algorithm for DP-ERM:

% \begin{theorem}[Implementation for DP-ERM]
% For any constants $1\geq \eps> 0$ and $1>\delta>0$, there is an efficient sampler to solve DP-ERM which has the following guarantees:
% \begin{itemize}
%     \item The scheme is $(\eps,\delta)$-differentially private;
%     \item The expected excess empirical loss is bounded by
%     $
%         \frac{4GD\sqrt{d\log(1.26/\delta)}}{\eps n}
%     $
%     \item The scheme takes 
%     $
%         \Theta(\frac{\eps^2n^2}{\log(1/\delta)}\log^2(\frac{\eps^2n^3d}{\delta\log(n/\delta)}))
%     $
%     queries of the values on $f_i(x)$ in expectation.
% \end{itemize}
% \end{theorem}

%{[}Write down our statement for the empirical loss case{]}
\cite{FKT20} suggested a iterative localization technique to reduce the DP-SCO to DP-ERM, which is formalized a little to a framework by \cite{KLL21}. With this framework, we can have the following result about DP-SCO.

\begin{theorem}[DP-SCO]
There is an efficient algorithm to solve DP-SCO which has the following guarantees:
\begin{itemize}
    \item The algorithm is $(\eps,\delta)$-differentially private;
    \item The expected population loss for general convex case is bounded by
    $
        O(GD(\frac{1}{\sqrt{n}}+\frac{\sqrt{d\log(1/\delta)}}{\eps n})).
    $
    The expected population loss for strongly-convex case is bounded by
    $
        O(\frac{G^2}{\mu}(\frac{1}{N}+\frac{d\log(1/\delta)}{\eps^2n^2})).
    $
    \item The algorithm takes 
    $
        O(\frac{\min\{\eps^2n^2,\eps nd\}}{\log(1/\delta)}\log^2(\frac{nd\eps\}}{\delta}))
    $
    queries of the values of $f_i(\cdot)$ in expectation.
\end{itemize}
\end{theorem}



Note that it is still open how to get a private algorithm with optimal loss using $\widetilde{O}(n)$ many $\nabla f_{i}(x)$. The previous best uses $\widetilde{O}(\min(n^{5/4}d^{1/8},n^{3/2}/d^{1/8}))$ many queries of
$\nabla f_{i}(x)$ \cite{KLL21} or $\widetilde{O}(\min(n^{3/2},n^{2}/\sqrt{d}))$
many $\nabla f_{i}(x)$ \cite{AFKT21}. However, our algorithm uses
less information from the functions, $\widetilde{O}(n^{2})$ bits
instead of $\widetilde{O}(n^{2.375})$ bits (when $n=d$). We are
not sure if there are super linear information lower bound for function
value oracle. We left the problem as an open problem. 

We draw a picture to compare the current results on query complexity and our results for DP-SCO:
\begin{wrapfigure}{h}{.5\textwidth}
    \centering
    \includegraphics[width = .5\textwidth]{figs/running_time.pdf}
    %\hfill
    %\caption{Reductions between ERM and SCO for general convex and strongly convex cases.}
   % As the lower bound of excess population loss is $x(GD(\frac{1}{\sqrt{N}}+\frac{\sqrt{d}}{N\epsilon}))$ while the lower bound of empirical risk is $x(\frac{GD\sqrt{d}}{N\epsilon})$, we do not know how to reduce from ERM to SCO. }
   \caption{Comparison among our results and the recent results in \cite{AFKT21,KLL21} for the non-trivial regime ($d\leq N^2)$. Suppose $\epsilon,\delta$ are small constants. Note that we are querying the functions values $f_i(x)$ while the previous two works query gradient $\nabla f_i(x)$.
   }
    \label{fig:compare_result}
\end{wrapfigure}

In practice, if we can compute gradient oracle, then we can usually compute the function value. But the opposite may not be true (for
example when $f_{i}(x)$ involves human interaction). Therefore, we suspect our weaker oracle requirement may lead to new applications.

%{[}Write down our statement for the Excess Population Loss{]}
%{[}Draw the plot to highlight we are faster{]}

\subsection{Other related work}
There are some related results in \cite{MASN16}, where they also study about the distribution $\exp(-kF(x;S))$ gives $(\eps,\delta)$-differential privacy for the DP-ERM problem. 
Their proof is based on bounding KL-divergence by Logarithmic Sobolev inequality, and the result is not tight with unnecessary restrictions between $\eps$ and $\delta$ as well.
%\Daogao{Seems their bound is only off by some constant to us}
%Specifically, they should set $k=O(\frac{n\mu\eps^2}{G^2\log(1/\delta)})$ in the strongly convex DP-ERM, while the value we require can be as large as $\frac{\eps^2n^2\mu}{8G^2\log(1.25/\delta)}$.


\end{comment}
\section{Preliminaries}

\subsection{Differential Privacy}

% \textbf{\begin{definition}[Total variation distance]
% The total variation distance between two random variables $Y$ and $Z$ is defined by:
% \begin{align*}
%     \TV(Y, Z) \stackrel{\text { def }}{=} \max _{S}|\operatorname{Pr}[Y \in S]-\operatorname{Pr}[Z \in S]|
% \end{align*}
% \end{definition}}

% \iffalse
% \begin{definition}[Kullback-Leibler (KL) divergence]
% Let $\rho,\cNu$ be probability distributions on $\cK\subseteq \R^d$. Assume $\rho,\cNu$ have full support and smooth densities. 
% The Kullback-Leibler (KL) divergence of $\rho$ with respect to $\cNu$ is defined by
% \begin{align*}
%     H_\cNu(\rho)=\int_{\cK}\rho(\omega)\log \frac{\rho(\omega)}{\cNu(\omega)}\d \omega.
% \end{align*}
% \end{definition}

% \begin{definition}[$\alpha$-Rényi Divergence]
% For two probability distributions $P$ and $Q$ over $\cK\subseteq\R^d$ and $\alpha$, the Rényi divergence of order $\alpha$ ($\alpha>0$ and $\alpha\cNeq 1$) is
% \begin{align*}
%     R_{\alpha}(P\| Q)=\frac{1}{\alpha-1}\log \E_{\omega\sim Q}[(\frac{P(\omega)}{Q(\omega)})^\alpha].
% \end{align*}
% For $\alpha=1$, we let $R_{1}(P\| Q):=\sup_{0<\alpha<1}R_{\alpha}(P\mid Q)$.
% We use $E_{\alpha}(P\|Q)$ to denote the $\alpha^{th}$ moment of the likelihood ratio between $P$ and $Q$:
% \begin{align*}
%     E_{\alpha}(P\|Q)=\E_{\omega\sim Q}[(\frac{P(\omega)}{Q(\omega)})^\alpha].
% \end{align*}
% \end{definition}
% By Theorem 5 in \cite{EH14}, we know $R_1(P\|Q)=H_Q(P)$.
% \fi

% \iffalse
% \begin{definition}[Rényi differential privacy \cite{Mir17}]
% We say a randomized mechanism $\M:\cK\rightarrow \mathrm{Range}(\M)$ has $\epsilon$-Rényi differential privacy of order $\alpha$, or $(\alpha,\epsilon)$-RDP for short, if for any adjacent databases $S,S'$, one has
% \begin{align*}
%     R_\alpha(\M(S)\|\M(S'))\leq \epsilon.
% \end{align*}
% \end{definition}
% \fi

% \begin{definition}[Neighbouring databases]
% We say two databases $S,S'$ are neighbouring if they only differ by one element and the difference between $F(\omega;S)$ and $F(\omega;S')$ is $2G/|S|$-Lipschitz over $\cK\subseteq\R^d$, where
% \begin{align*}
%     F(x;S)=\frac{1}{|S|}\sum_{s_i\in S}f(\omega;s_i).
% \end{align*}
% \end{definition}


A DP algorithm $\M$ usually satisfies a collection of $(\eps,\delta)$-DP guarantees for each $\epsilon$, i.e., for each $\epsilon$ there exists some smallest $\delta$ for which $\M$ is $(\eps,\delta)$-DP. By collecting all of them together, we can form the privacy curve or privacy profile which fully characterizes the privacy of a DP algorithm.
\begin{definition}[Privacy Curve]
Given two random variables $X,Y$ supported on some set $\Omega$, define the privacy curve $\delta(X\|Y):\R_{\geq 0}\rightarrow [0,1]$ as:
\begin{align*}
    \delta(X\|Y)(\epsilon)=\sup_{{S}\subset \Omega} \Pr[Y\in {S}]-e^{\epsilon}\Pr[X\in {S}].
\end{align*}
\end{definition}

One can explicitly calculate the privacy curve of a Gaussian mechanism as
\begin{equation}
\label{eqn:Gaussian_privacycurve}
   \deltacurve{\cN(0,1)}{\cN(s,1)}(\eps)= \Phi\lp -\frac{\eps}{s}+\frac{s}{2} \rp - e^\eps \Phi\lp -\frac{\eps}{s}-\frac{s}{2} \rp 
\end{equation}
 where $\Phi(\cdot)$ is the Gaussian cumulative distribution function (CDF)~\cite{BalleW18}. 
%  One can also prove the following upper bound on the privacy curve by using the upper bound $\Phi(-t)\leq \exp(-t^2/2)/2$ for $t\ge 0,$ % Double check in a computer, the tight bound has a extra 1/2. (I find it in wiki)
% \begin{equation}
% \label{eqn:Gaussian_privacycurve_approx}
%   \deltacurve{\cN(0,1)}{\cN(s,1)}(\eps)  \le \frac{1}{2} \exp\lp-\frac{1}{2}\lp\frac{\eps}{s}-\frac{s}{2}\rp^2\rp  \text{ if } \eps \geq \frac{s^2}{2}.
% \end{equation}


We say a differentially private mechanism $\M$ has privacy curve $\delta:\R_{\ge0}\rightarrow[0,1]$ if for every $\epsilon\geq0$, $\M$ is $(\epsilon,\delta(\epsilon))$-differentially private, i.e., $\delta(\M(\cD)\| \M(\cD'))(\epsilon)\leq \delta(\eps)$ for all neighbouring databases $\cD,\cD'$.
We will also need the notion of tradeoff function introduced in~\cite{dong2019gaussian} which is an equivalent way to describe the privacy curve $\delta(P\|Q)$.
\begin{definition}[Tradeoff function]
Given two (continuous) distributions $P,Q$, we define the trade-off function\footnote{Tradeoff curves in~\cite{dong2019gaussian} are defined using type I and type II errors. The definition given here is equivalent to their definition for continuous distributions.} $T(P\|Q):[0,1]\to [0,1]$ as $$T(P\|Q)(z)= \inf_{S:P(S)=1-z}Q(S).$$

It is easy to compute explicitly the tradeoff function for Gaussian mechanism~\cite{dong2019gaussian},
\begin{equation}
    \label{eqn:gaussian_tradeoff}
    T(\cN(0,1)\|\cN(s,1))(z)=\Phi(\Phi^{-1}(1-z)-s).
\end{equation}
Note that perfect privacy is equivalent to the tradeoff function $\mathrm{Id}(z)=1-z$ and the closer a tradeoff function is to $\mathrm{Id}$, better the privacy. The tradeoff function $T(P\|Q)$ and the privacy curve $\delta(P\|Q)$ are related via convex duality. Therefore to compare privacy curves, it is enough to compare tradeoff curves.

\begin{proposition}[\cite{dong2019gaussian}]
\label{prop:delta_tradeoff}
$\delta(P\|Q)\le \delta(P'\|Q')$ iff $T(P\|Q)\ge T(P'\|Q')$
\end{proposition}


\end{definition}

\subsection{Optimization}
Here we collect some properties of functions which are useful for optimization and sampling.
\begin{definition}[$L$-Lipschitz Continuity]
A function $f:{\cal K}\rightarrow \R$ is $L$-Lipschitz continuous over the domain ${\cal K}\subset \R^{d}$ if the following holds for all $\omega,\omega'\in {\cal K}:|f(\omega)-f(\omega')|\leq L\|\omega-\omega'\|_2$. 
\end{definition}

% never used
%\begin{definition}[$\beta$-Smoothness]
%A function $f:{\cal K}\rightarrow \R$ is $\beta$-smooth over the domain ${\cal K}\subset \R^{d}$ if for all $\omega,\omega'\in{\cal K}$, $\|\nabla f(\omega)-\nabla f(\omega')\|_2\leq \beta \|\omega-\omega'\|_2$.
%\end{definition}

\begin{definition}[$\mu$-Strongly convex]
A differentiable function $f:\cK\rightarrow \R$ is called strongly convex with parameter $\mu>0$ if ${\cal K}\subset \R^{d}$ is convex and the following inequality holds for all points $\omega,\omega'\in {\cal K}$,
% one def is enough here
%\[
%\langle \nabla f(\omega)-\nabla f(\omega'), \omega-\omega' \rangle\geq \mu \|\omega-\omega'\|_2^2.
%\]
%Equivalently,
\[
f(\omega')\geq f(\omega) +\inpro{\nabla f(\omega)}{\omega'-\omega}+\frac{\mu}{2}\|\omega'-\omega\|_2^2.
\]
\end{definition}

\begin{definition}[Log-concave measure and density]
A density function $f:\cK \rightarrow \R_{\geq 0}$ is log-concave if $\int_{\cK} f(x) dx = 1$ and $f(x) = \exp(-F(x))$ for some convex function $F$. We call $f$ is $\mu$-strongly log-concave if $F$ is $\mu$-strongly convex. Similarly, we call $\pi$ a log-concave measure if its density function is log-concave, and we call $\pi$ is a $\mu$-strongly log-concave measure if its density function is $\mu$-strongly log-concave.
\end{definition}


\subsection{Distribution Distance and Divergence}
We present some distribution distances or divergences mentioned or used in this work.
\label{sec:dis_measure}
\begin{definition}{\cite[R{\'e}nyi Divergence]{Ren61}}
Suppose $1<\alpha<\infty$ and $\pi,\nu$ are measures with $\pi\ll\nu$.
The R{\'e}nyi divergence of order $\alpha$ between $\pi$ and $\nu$ is defined as
\begin{align*}
    \mathrm{D}_{\alpha}(\pi\|\nu)=\frac{1}{\alpha}\log\int\lp\frac{\pi(x)}{\nu(x)}\rp^{\alpha}\nu(x)\d x.
\end{align*}
We follow the convention that $\frac{0}{0}=0$. R{\'e}nyi Divergence of orders $\alpha=1,\infty$ are defined by continuity.
For $\alpha=1$, the limit in R{\'e}nyi Divergence equals to the Kullback-Leibler divergence of $\pi$ from $\nu$, which is defined as following:
\end{definition}

\begin{definition}[Kullback–Leibler divergence]
The Kullback–Leibler divergence between probability measures $\pi$ and $\nu$ is defined by 
\begin{align*}
    \mathrm{D}_{KL}(\pi\|\nu)=\int  \log\lp\frac{\pi}{\nu}\rp\d \pi.
\end{align*}
\end{definition}

\begin{definition}[Wasserstein distance]
Let $\pi,\nu$ be two probability distributions on $\R^d$.
The second Wasserstein distance $\mathrm{W}_{2}$ between $\pi$ and $\nu$ is defined by
\begin{align*}
    \mathrm{W}_2(\pi,\nu)=\big( \inf_{\gamma\in\Gamma(\pi,\nu)}\int_{\R^d\times\R^d}\|x-y\|_2^2\d \gamma(x,y) \big)^{1/2},
\end{align*}
where $\Gamma(\pi,\nu)$ is the set of all couplings of $\pi$ and $\nu$.
\end{definition}

\begin{definition}[Total variation distance]
The total variation distance between two probability measures $\pi$ and $\nu$ on a sigma-algebra $\mathcal{F}$ of subsets of the sample space $\Omega$ is defined via
\begin{align*}
    \mathrm{TV}(\pi,\nu)=\sup_{S\in \mathcal{F}}|\pi(S)-\nu(S)|.
\end{align*}
\end{definition}


\subsection{Isoperimetric Inequality for Strongly Log-concave Distributions}
The cumulative distribution function (CDF) of one-dimensional standard Gaussian distribution will be denoted by $\Phi(x)=\Pr_{y\sim\cN(0,1)}[y\le x]$. The following Lemma relates the expanding property of log-concave measures with $\Phi$.
%Note that by the symmetry of the Gaussian distribution, $\Phi(x)+\Phi(-x)=1$ and $\Phi^{-1}(1-z)=-\Phi^{-1}(z).$
\begin{proposition}[Theorem 1.1. in \cite{Led99}]
\label{prop:isoperimetry}
Let $\pi$ be a $\mu$-strongly log-concave measure supported on a convex set $\cK \subseteq \R^d$. Let $A\subset\cK$ by any subset such that $\pi(A)=z$. For any point $x\in\R^d$, define $d(x,A)=\inf_{y\in A}\|x-y\|_2$. Let $A_r = \lc x: d(x,A)\le r\rc$. Then if $A_r\subseteq\cK$, for every $r\ge 0$, $$\pi(A_r) \ge \Phi(\Phi^{-1}(z)+r\sqrt{\mu}).$$
\end{proposition}
The property above implies the concentration of Lipschitz functions over log-concave measures.
\begin{corollary}
\label{cor:isoperimetry_lip}
Let $\pi$ be a $\mu$-strongly log-concave measure supported on a convex set $\cK\subseteq\R^d$. Suppose $\alpha:\cK\to \R$ is $G$-Lipschitz. For $z\in [0,1]$, define $m(z)\in \R$ such that $\Pr_{x\sim \pi}[\alpha(x)\le m(z)]=z$. Then for every $r\ge 0$,
$$\Pr_{x\sim \pi}[\alpha(x) \ge m(z)+r] \le \Phi\lp \Phi^{-1}(1-z)-\frac{r\sqrt{\mu}}{G}\rp,$$
$$\Pr_{x\sim \pi}[\alpha(x) \le m(z)-r] \le \Phi\lp \Phi^{-1}(z)-\frac{r\sqrt{\mu}}{G}\rp.$$
\end{corollary}
\begin{proof}
Fix some $z\in [0,1]$. Let $A=\{x\in \cK:\alpha(x)\le m(z)\}$, so $\pi(A)=z$. Let $A_r=\{x:d(x,A)\le r\}.$ Since $\alpha$ is $G$-Lipschitz, $\alpha(x)\ge m(z)+r$ implies that $d(x,A)\ge r/G.$ Therefore $\lc x: \alpha(x) \ge m(z)+r\rc \subset \lc x: d(x,A) \ge r/G\rc = \overline{A_{r/G}}$ and so 
\begin{align*}
\Pr_{x\sim \pi}[\alpha(x) \ge m(z)+r] &\le \pi(\overline{A_{r/G}}) \\
&= 1- \pi(A_{r/G})\\
&\le 1 - \Phi\lp\Phi^{-1}(z)+\frac{r\sqrt{\mu}}{G}\rp\\
&= \Phi\lp-\Phi^{-1}(z)-\frac{r\sqrt{\mu}}{G}\rp.
\end{align*}
We obtain the other inequality by applying the above inequality to $-\alpha(x).$
\end{proof}



% \begin{theorem}[Sampling]

% \end{theorem}


%\input{exponential}
%\input{continuous}
% \input{log-sobolev}



\section{GDP of Regularized Exponential Mechanism}
In this section, we prove our DP result (Theorem~\ref{thm:privacy_technical}). 
The proof uses the isoperimetric inequality for strongly log-concave measures~\cite{Led99}. Intuitively, the privacy loss random variable will be $G$-Lipschitz under the hypothesis and isoperimetric inequality implies that any Lipschitz function will be as concentrated as a Gaussian with appropriate standard deviation. This allows us compare the privacy curve $\deltacurve{P}{Q}$ to that of a Gaussian mechanism. In our proof, it is actually more convenient to compare tradeoff curves ($\tradeoff{P}{Q}$) which are equivalent to privacy curves via convex duality (Proposition~\ref{prop:delta_tradeoff} and Theorem~\ref{thm:privacy_technical}).

%By Proposition~\ref{prop:delta_tradeoff}, Theorem~\ref{thm:privacy_technical} is equivalent to the following theorem.
%The proof is based on the Isoperemetric inequality, which presents the concentration property of strongly log-concave measure and the worst case is exactly the Gaussian measure.
%Thus the privacy curve can be bounded by the one of Gaussian mechanism.

\begin{theorem}
\label{thm:privacy_technical_tradeoff}
Given convex set $\cK\subseteq \R^d$ and $\mu$-strongly convex functions $F,\Tilde{F}$ over $\cK$. Let $P,Q$ be distributions over $\cK$ such that $P(x)\propto e^{-F(x)}$ and $Q(x)\propto e^{-\Tilde{F}(x)}$.
If $\Tilde{F}-F$ is $G$-Lipschitz over $\cK$, then for all $z\in[0,1]$,
\begin{align*}
    \tradeoff{P}{Q}(z) 
    \ge \tradeoff{\cN\lp 0,1\rp}{\cN\lp\frac{G}{\sqrt{\mu}},1\rp}(z).
\end{align*}
\end{theorem}

% \begin{proof}
% Let $\alpha(x)=\Tilde{F}(x)-F(x)+C$ for some constant $C\in \R$ so that $Q(x)=e^{-\alpha(x)} P(x)$. We have $T(P\|Q)(z)=\inf_{S:P(S)=z} Q(S)$. Note that the infimum is achieved when we choose 
% $$S=\lc x:\log\lp \frac{Q(x)}{P(x)}\rp = -\alpha(x) \le -m(z)\rc$$ for some $m(z)$ chosen such that 
% $$P(S)=\Pr_{x\sim P}[\alpha(x)\ge m(z)]=z.$$ Therefore:
% \begin{align*}
% T(P\|Q)(z)&= \int_{x: \alpha(x)\ge m(z)} Q(x) dx\\
% &= \int_{x: \alpha(x)\ge m(z)} e^{-\alpha(x)} P(x) dx\\
% &= \int_{t=m(z)}^\infty e^{-t} \lp -\frac{d\Pr_{x\sim P}\lb\alpha(x)\ge t\rb}{dt}\rp dx\\
% &=  \left.-e^{-t}\Pr_{x\sim P}\lb\alpha(x)\ge t\rb\right\vert_{m(z)}^\infty - \int_{t=m(z)}^\infty e^{-t} \Pr_{x\sim P}\lb\alpha(x)\ge t\rb dx\\
% &=  ze^{-m(z)} - e^{-m(z)}\int_{t=0}^\infty e^{-t} \Pr_{x\sim P}\lb\alpha(x)\ge t+m(z)\rb dx\\
% \end{align*}
% By isoperimetric inequality\Gopi{Refer to the theorem}, we have
% \begin{align*}
% \Pr_{x\sim P}\lb\alpha(x)\ge t+m(z)\rb &\le \Pr[N(0,1)\ge t+\Phi^{-1}(1-z)]\\
% &=1-\Phi\lp t+ \Phi^{-1}(1-z)\rp\\
% &=\Phi(\Phi^{-1}(z)-t) \tag{$\Phi^{-1}(1-z)=-\Phi^{-1}(z)$ and $\Phi(-x)+\Phi(x)=1$}.
% \end{align*}
% Plugging this into the expression for $T(P\|Q)$, we get:
% \begin{align*}
%     T(P\|Q)&\ge ze^{-m(z)} - e^{-m(z)}\int_{t=0}^\infty e^{-t} \Phi(\Phi^{-1}(z)-t) dx\\
%     &= ze^{-m(z)} - e^{-m(z)}\lp z - \exp\lp\frac{1}{2}-\Phi^{-1}(z)\rp \Phi(\Phi^{-1}(z)-1) \rp \tag{\Gopi{Integrate by parts}}\\
%     &= \exp\lp\frac{1}{2}-\Phi^{-1}(z)-m(z)\rp \Phi(\Phi^{-1}(z)-1)\\
% \end{align*}

% \end{proof}

\begin{proof}
% WLOG, we can assume $G=\mu=1$.
% One can easily bring in specific parameters with the same arguments.
% \Gopi{Explain why}
Let $\gamma=G/\sqrt{\mu}.$
Let $\alpha(x)=\Tilde{F}(x)-F(x)$ so that $Q(x)\propto e^{-\alpha(x)} P(x)$. Recall that we have $T(P\|Q)(z)=\inf_{S:P(S)=1-z} Q(S)$. Note that the infimum is achieved when we choose 
$S=\lc x\in\cK: \alpha(x) \ge m(z)\rc$ for some $m(z)$ chosen such that 
$P(S)=\Pr_{x\sim P}[\alpha(x)\ge m(z)]=1-z$ (Neyman-Pearson lemma). Therefore:
\begin{align*}
T(P\|Q)(z)&= \int_{x\in S} Q(x) \d x\\
&= \frac{\int_{x\in S} e^{-\alpha(x)} P(x) \d x}{\int_{x\in\cK} e^{-\alpha(x)} P(x) \d x}\\
%&= \frac{\E_P[e^{-\alpha}\Ind_S]}{\E_P[e^{-\alpha}\Ind_S] + \E_P[e^{-\alpha}\Ind_{\barS}]}\\
&= \lp 1+\frac{\E_P[e^{-\alpha}\Ind_{\barS}]}{\E_P[e^{-\alpha}\Ind_S]}\rp^{-1}
\end{align*}
% \begin{align*}
%     T(P\|Q)(z)&=\lp 1+\frac{\E_P[e^{-\alpha}\Ind_{\barS}]}{\E_P[e^{-\alpha}\Ind_S]}\rp^{-1}.
% \end{align*}
We will now lower bound $\E_P[e^{-\alpha}\Ind_S]$. Let the random variable $Y=\alpha(x)$ where $x\sim P.$ Let $f_Y(\cdot)$ be the PDF of $Y$.
\begin{align*}
\E_P[e^{-\alpha(x)}\Ind_S]&=\int_{x: \alpha(x)\ge m(z)}e^{-\alpha(x)}P(x) \d x=\E[e^{-Y}\Ind(Y\ge m(z))]=\int_{m(z)}^\infty e^{-t} f_Y(t) dt\\
&= \int_{t=0}^\infty e^{-t-m(z)} \lp -\frac{\d\Pr_{x\sim P}\lb\alpha(x)\ge t+m(z)\rb}{\d t}\rp \d t\\
&=  e^{-m(z)}\lp \left.-e^{-t}\Pr_{x\sim P}\lb\alpha(x)\ge t+m(z)\rb\right\vert_{0}^\infty - \int_{t=0}^\infty e^{-t} \Pr_{x\sim P}\lb\alpha(x)\ge t+m(z)\rb \d t\rp\\
&=  (1-z)e^{-m(z)} - e^{-m(z)}\int_{t=0}^\infty e^{-t} \Pr_{x\sim P}\lb\alpha(x)\ge t+m(z)\rb \d t\\
&\ge (1-z)e^{-m(z)} - e^{-m(z)}\int_{t=0}^\infty e^{-t} \Phi(\Phi^{-1}(1-z)-t/\gamma) \d t \tag{Corollary~\ref{cor:isoperimetry_lip}}\\
&= (1-z)e^{-m(z)} - e^{-m(z)}\lp (1-z) - \exp\lp\frac{\gamma^2}{2}-\Phi^{-1}(1-z)\gamma\rp \Phi(\Phi^{-1}(1-z)-\gamma) \rp \tag{Claim \ref{claim:some_gaussian_integrals}}\\
&= \exp\lp\frac{\gamma^2}{2}+\Phi^{-1}(z)\gamma-m(z)\rp \Phi(-\Phi^{-1}(z)-\gamma)
\end{align*}


%%%%%%%%%%%%%%%%%%%%%%%%%%%%%%%%%%%%%%%%%%%%%



We will now upper bound $\E_P[e^{-\alpha}\Ind_{\barS}]$ in a similar way.
\begin{align*}
\E_P[e^{-\alpha(x)}\Ind_{\barS}]&=\int_{x: \alpha(x)\le m(z)}e^{-\alpha(x)}P(x) \d x\\
&= \int_{t=0}^{\infty} e^{-m(z)+t} \lp -\frac{d\Pr_{x\sim P}\lb\alpha(x)\le m(z)-t\rb}{dt}\rp \d t\\
&=  e^{-m(z)}\lp\left.-e^{t}\Pr_{x\sim P}\lb\alpha(x)\le m(z)-t\rb\right\vert_{0}^\infty + \int_{t=0}^\infty e^{t} \Pr_{x\sim P}\lb\alpha(x)\le m(z)-t\rb \d t \rp\\
&=  ze^{-m(z)} + e^{-m(z)}\int_{t=0}^\infty e^{t} \Pr_{x\sim P}\lb\alpha(x)\le m(z)-t\rb \d t\\
&\le ze^{-m(z)} + e^{-m(z)}\int_{t=0}^\infty e^{t} \Phi(\Phi^{-1}(z)-t/\gamma) \d t \tag{Corollary~\ref{cor:isoperimetry_lip}}\\
    &= ze^{-m(z)} + e^{-m(z)}\lp -z + \exp\lp\frac{\gamma^2}{2}+\Phi^{-1}(z)\gamma\rp \Phi(\Phi^{-1}(z)+\gamma) \rp \tag{Claim~\ref{claim:some_gaussian_integrals}}\\
    &= \exp\lp\frac{\gamma^2}{2}+\Phi^{-1}(z)\gamma-m(z)\rp \Phi(\Phi^{-1}(z)+\gamma)
\end{align*}


Combining the two bounds, we get:
\begin{align*}
T(P\|Q)(z)&= \lp 1+\frac{\E_P[e^{-\alpha}\Ind_{\barS}]}{\E_P[e^{-\alpha}\Ind_S]}\rp^{-1}\\
&\ge \lp 1 + \frac{\Phi(\Phi^{-1}(z)+\gamma)}{\Phi(-\Phi^{-1}(z)-\gamma)} \rp^{-1}\\
&= \Phi(-\Phi^{-1}(z)-\gamma) \tag{Using $\Phi(x)+\Phi(-x)=1$}\\
&= \tradeoff{N(0,1)}{N(\gamma,1)} \tag{Eqn (\ref{eqn:gaussian_tradeoff})}.
\end{align*}
\end{proof}

We finish by calculating the integrals that showed up in the proof.
\begin{claim}
\label{claim:some_gaussian_integrals}
$$\int_0^\infty e^{-t} \Phi\lp a-\frac{t}{\gamma}\rp \d t = \Phi(a)-e^{\frac{\gamma^2}{2}-a\gamma}\Phi(a-\gamma)$$
$$\int_0^\infty e^{t} \Phi\lp a-\frac{t}{\gamma}\rp  \d t = -\Phi(a)+e^{\frac{\gamma^2}{2}+a\gamma}\Phi(a+\gamma)$$
\end{claim}
\begin{proof}
\begin{align*}
    \int_0^\infty e^{-t} \Phi(a-t/\gamma) \d t &= \left.-e^{-t}\Phi(a-t/\gamma)\right\vert_0^\infty - \int_0^\infty e^{-t} \frac{e^{-(a-t/\gamma)^2/2}}{\gamma\sqrt{2\pi}} \d t\\
     &= \Phi(a) - \int_0^\infty e^{\gamma^2/2-a\gamma} \frac{e^{-(t-(\gamma a-\gamma^2))^2/2}}{\gamma\sqrt{2\pi}} \d t\\
     %&= \Phi(a) -  e^{\gamma^2/2-a\gamma} \lp 1- \Phi(-(a-\gamma))\rp\\
     &= \Phi(a) -  e^{\gamma^2/2-a\gamma} \Phi(a-\gamma).
\end{align*}
\begin{align*}
    \int_0^\infty e^{t} \Phi(a-t/\gamma) \d t &= \left.e^{t}\Phi(a-t/\gamma)\right\vert_0^\infty + \int_0^\infty e^{t} \frac{e^{-(a-t/\gamma)^2/2}}{\gamma\sqrt{2\pi}} \d t\\
     &= -\Phi(a) + \int_0^\infty e^{\gamma^2/2+a\gamma} \frac{e^{-(t-(a\gamma+\gamma^2))^2/2\gamma^2}}{\gamma\sqrt{2\pi}} \d t\\
     %&= -\Phi(a) +  e^{\gamma^2/2+a\gamma} \lp 1- \Phi(-a-\gamma)\rp\\
     &= -\Phi(a)  +  e^{\gamma^2/2+a\gamma} \Phi(a+\gamma).
\end{align*}
\end{proof}

As a corollary to Theorem~\ref{thm:privacy_technical_tradeoff}, we can bound any divergence measure that decreases under post-processing such as Renyi divergence or KL divergence. In particular, this also implies Renyi Differential Privacy~\cite{Mir17} of our algorithm.


\begin{corollary}
\label{cor:divergence_privacy}
Suppose $F,\TF$ are two $\mu$-strongly convex functions over $\cK\subseteq \R^d$, and $F-\TF$ is $G$-Lipschitz over $\cK$.
For any $k>0$, if we let $P\propto e^{-kF}$ and $Q\propto e^{-k\TF}$ be two probability distributions on $\cK$, then we have
\begin{align*}
    \mathrm{D}(P\|Q)\leq \mathrm{D}\lp\cN(0,1)\|\cN\lp \frac{G\sqrt{k}}{\sqrt{\mu}},1\rp\rp
\end{align*}
for any divergence measure $\mathrm{D}$ which decreases under post-processing. In particular, $$\mathrm{D}_{\alpha}(P\|Q)\le \frac{\alpha k G^2}{2\mu} \text{ and }\mathrm{D}_{KL}(P\|Q)\le \frac{kG^2}{2\mu}.$$
\end{corollary}

\begin{proof}
% Recall in Theorem~\ref{thm:privacy_technical_tradeoff}, we proved for all $z\in[0,1]$, one has
% \begin{align*}
%     \tradeoff{\pi}{\nu}(z)\ge \tradeoff{\cN\lp 0,1\rp}{\cN\lp\frac{G\sqrt{k}}{\sqrt{\mu}},1\rp}(z).
% \end{align*}

By Theorem 2.10 in \cite{dong2019gaussian}, if $T(P\|Q) \ge T(X\|Y)$, then there exists a randomized algorithm $M$ such that $M(X)=P$ and $M(Y)=Q$. Therefore for any divergence measure which decreases under post-processing we have,
$$\mathrm{D}(P\|Q)= \mathrm{D}(M(X)\|M(Y)) \le \mathrm{D}(X\|Y).$$ The rest follows from Theorem~\ref{thm:privacy_technical_tradeoff}. It is well-known that Renyi divergence and KL divergence decrease with post-processing (see \cite{EH14}, for example). We can also compute $\mathrm{D}_\alpha(\cN(0,1),\cN(s,1))=\alpha s^2/2$ and $\mathrm{D}_{KL}(\cN(0,1),\cN(s,1))=s^2/2$~\cite{Mir17}.
\end{proof}

% Actually the bounds we get on Renyi divergence and Wasserstein distance are tight, where the tightness can be shown by Gaussian case.

\section{Sample-Based PCA}
\label{sec:sampling}

To tackle workload-aware DR, we demonstrate how sample-based PCA can bridge the performance , but that the number of samples required varies per dataset.
Finally, we show how dynamically increasing the sampling rate can help identify how much to sample a given dataset, providing a foundation for workload-aware DR.

\begin{figure}
\includegraphics[width=\linewidth]{figs/progressive.pdf}
\caption[]{ Improvement in representation size for  $TLB = 0.80$ across three datasets. Higher sampling rates improve quality until reaching a state equivalent to running PCA over the full dataset ("convergence")}
\label{fig:progressive}
\end{figure}

\begin{comment}
\subsection{PCA Speed vs. Quality}

While improved quality provides faster repeated query execution, the cost of DR via PCA dominates this speedup, encouraging the use of faster, lower-quality alternatives~\cite{keogh-study}. 

To briefly quantify this trade-off, we augment a widely-cited time series similarity search DR study from VLDB 2008~\cite{keogh-study} by evaluating PCA---which the authors did not benchmark due to it being ``untenable for large data sets." 
We compare PCA via SVD to baseline techniques based on runtime and DR performance with respect to $TLB$ over the largest datasets from~\cite{keogh-study}. 
We use their two fastest methods as our baselines as they show the remainder exhibited ``very little difference'': Fast Fourier Transform (FFT) and Piecewise Aggregate Approximation (PAA).

\minihead{TLB Performance Comparison}
We compute the minimum dimensionality achieved by each technique subject to a $TLB$ constraint. 
On average, PCA provides bases that are $2.3\times$ (up to $3.9\times$) and $3.7\times$ (up to $26\times$)  smaller than PAA and FFT for $TLB = 0.75$, and $2.9\times$ (up to $8.3\times$) and $1.8\times$ (up to $5.1\times$) smaller for $TLB = 0.99$.
While the margin between PCA and alternatives is dataset-dependent, PCA almost always preserves $TLB$ with a lower dimensional representation.

%\section{Additional End-to-End Plots}
%\input{endendplots}

\minihead{Runtime Performance Comparison} 
PCA implemented via out-of-the-box SVD is on average over \red{$26\times$ (up to $56\times$)} slower than PAA and over \red{$4.6\times$ (up to $9.7\times$)} times slower than FFT when computing the smallest $TLB$-preserving basis.
%This substantiates the observation that classic PCA is incredibly slow compared to alternatives~\cite{keogh-study}. 

\end{comment}

\subsection{Incremental, Progressive Sampling}
To bridge this performance-runtime gap, we turn to data sampling. 
Many real-world \red{datasets} are intrinsically low-dimensional, as evidenced by their rapid falloff in their eigenvalue spectrum.
A data sample thus captures much of the dataset's ``interesting'' behavior, so fitting models over data samples generalize well. 
We verify this by varying the target $TLB$ and examining the minimum number of uniformly selected samples required to obtain a $TLB$-preserving transform with output dimension $k$ equal to input dimension $\dvar$.

On average, a sample of under $0.64\%$ $(\text{up to } 5.5\%)$ of the input is sufficient for $TLB = 0.75$, and under $4.2\%$ $(\text{up to } 38.6\%)$ is sufficient for $TLB=0.99$.  
If this sample rate is known, we obtain up to \red{$91\times$ speedup} compared to a na\"ive implementation of PCA via SVD---with no algorithmic improvement. 

However, this benefit is dataset-dependent, and unknown a priori.
We thus turn to progressive sampling (gradually increasing the sample size) to identify how large a sample suffices.
Figure~\ref{fig:progressive} shows how the dimensionality required to attain a given $TLB$ changes when we vary dataset and proportion of data sampled.
Increasing the number of samples provides lower dimensional transformations for the same quality.
However, this decrease in dimension plateaus as the number of samples increases.
Thus, while progressive sampling would allow us to tune the amount of time spent on DR, we must determine when the downstream value of decreased dimension is overpowered by the cost of additional DR---that is, whether to sample to convergence (evaluated in \S\ref{subsec:lesion}) or terminate early (e.g., at $0.3$ proportion of data sampled for SmallKitchenAppliances). 







\section{DP Convex Optimization}
In this section we present our results about DP-ERM and DP-SCO.
%We show our results about DP-ERM first, and discuss how to bound the generalization error and get the optimal result about DP-SCO afterwards. 

%See Cor 1 in \url{https://pubsonline.informs.org/doi/pdf/10.1287/moor.2017.0906}
% \begin{proof}
% %\Daogao{Maybe put the proof into the appendix}
% %We adopt the standard ``peeling'' arguments which have been used a lot in previous works, e.g. \cite{KV06,BST14}.


% Let $x^*=\arg\min_{x\in\cK}F(x)$.
% Without loss of generality, let $F(x^*)=0$
% and it suffices to consider a differential cone $\Omega$ on direction $v$ and centered at $x^*$.
% We will bound the expected loss conditioned on $x\in\Omega\cap\cK$.


% Suppose $\E_{\nu}[F(x)\mid x\in \Omega\cap \cK]=a$.
% We want to prove that $a\leq dT$ and hence we can assume that $a>0$.
% % Let $t^*$ be a point such that $F(x^*+tv)=a$.
% % On this cone we define a new function
% % \begin{align*}
% %     \Tilde{F}(x^*+tv)=\frac{t}{t^*}a.
% % \end{align*}

% % Note that $a=\E_{x\sim e^{-F/T}}[F(x)\mid x\in \Omega\cap \cK]\leq \E_{x\sim e^{-\Tilde{F}/T}}[\Tilde{F}(x)\mid x\in \Omega\in \cK]$ due to the convexity of the $F$ and $\cK$.
% % For $0\leq t\leq t^*$, we know that $\Tilde{F}(tv+x^*)\geq F(tv+x^*)$ while $t\geq t^*$, $\Tilde{F}(tv+x^*)\leq F(tv+x^*)$.
% % Compared to the original distribution $\nu$, the distribution proportional to $e^{-\Tilde{F}/T}$ will have more mass in points with $\Tilde{F}(x)\leq a$ and less mass in points with $\Tilde{F}(x)\geq a$.

% % Note that $\Tilde{F}$ is linear and by \cite{KV06}, we have
% % \begin{align*}
% %     \E_{x\sim e^{-\Tilde{F}/T}}[\Tilde{F}(x)\mid x\in \Omega\in \cK]\leq dT.
% % \end{align*}


% Define the set $H(y)=\{x\in \cK\cap \Omega\mid F(x)=y\}$.
% It suffices to consider $\E_{\nu\mid_{ \Omega\cap\cK}}[F(x)]$.
% We have
% \begin{align*}
%     \E_{\nu\mid_{ \Omega\cap\cK}}[F(x)]=&\frac{\int_{0}^{\infty}ye^{-y/T}\volume_{d-1}(H(y))\d y}{\int_{0}^{\infty}e^{-y/T}\volume_{d-1}(H(y))\d y}
% \end{align*}
% \Gopi{Elaborate a little more. By convexity of what? Are you using some theorem like Brunn-Minkowski implicitly here.}
% By the convexity, it is easy to observe that
% $\volume_{d-1}(H(y))\geq (\frac{y}{a})^{n-1}\volume_{d-1}(H(a))$ for $y\leq a$ and $\volume_{d-1}(H(y))\leq (\frac{y}{a})^{n-1}\volume_{d-1}(H(a))$ for $y\geq a$.
% Removing mass in points with $F(x)\leq a$ and adding mass with $F(x)\geq a$ can only increase the expectation.
% Hence we have
% \begin{align*}
%     \E_{\nu\mid_{\Omega\cap\cK}}[F(x)]\leq& \frac{\int_{0}^{\infty}y e^{-y/T}\volume_{d-1}(H(a))(\frac{y}{a})^{d-1}\d y}{\int_{0}^{\infty} e^{-y/T}\volume_{d-1}(H(a))(\frac{y}{a})^{d-1}\d y}
%     = \frac{\int_{0}^{\infty}y^de^{-y/T}\d y}{\int_{0}^{\infty}y^{d-1}e^{-y/T}\d y}
%     =\frac{d!T^{d+1}}{(d-1)!T^d}
%     = dT.
% \end{align*}
% Hence we complete the proof.
% \end{proof}

\subsection{DP-ERM}
In this subsection, we state our result for the DP-ERM problem \eqref{eq:DPERM}.
Briefly speaking, our main result (Theorem~\ref{thm:privacy_technical}) shows that sampling from $\exp(-kF(x;\cD))$ for some appropriately chosen $k$ is $(\eps,\delta)$-DP and achieves the optimal empirical risk in (\ref{eqn:optimal_empirical_risk}).
Our sampling scheme in Section~\ref{sec:sampling} provides an efficient implementation. We start with the following lemma which shows the utility guarantee for the sampling mechanism.
%Here, we give a proof for the convex setting with the tight constant. The proof is similar to \cite{KV06,BST14}.

\begin{lemma}[Utility Guarantee, {\cite[Corollary 1]{DKL18}}]
\label{lm:utility_tech}
Suppose $k>0$ and $F$ is a convex function over the convex set $\cK\subseteq \R^d$. If we sample $x$ according to distribution $\nu$ whose density is proportional to $\exp(-k F(x))$, then we have
\begin{align*}
    \E_{\nu}[F(x)]\leq \min_{x\in\cK}F(x)+\frac{d}{k}.
\end{align*}
\end{lemma}

This is first shown by \cite{KV06} for any linear function $F$, and \cite{BST14} extends it to any convex function $F$ with a slightly worse constant. 

\begin{theorem}[DP-ERM]\label{thm:DPERM}
Let $\epsilon>0$, $\cK\subseteq \R^d$ be a convex set of diameter $D$ and $\{f(\cdot;s)\}_{s\in\mathcal{D}}$ be a family of convex functions over $\cK$ such that $f(x;s)-f(x;s')$ is $G$-Lipschitz for all $s,s'$.
For any data-set $\cD$ and $k>0$, sampling $x^{(priv)}$ with probability proportional to $\exp\left(-k(F(x;\cD)+\mu\|x\|_2^2/2)\right)$ is $(\epsilon,\delta(\eps))$-differentially private, where
\begin{align*}
 \delta(\eps)\leq \deltacurve{\cN(0,1)}{\cN\lp\frac{G\sqrt{k}}{n\sqrt{\mu}},1\rp}(\epsilon).
\end{align*}
The excess empirical risk is bounded by {$\frac{d}{k}+\frac{\mu D^2}{2}$}.
Moreover, if $\{f(\cdot,s)\}_{s\in\mathcal{D}}$ are already $\mu$-strongly convex, then sampling 
$x^{(priv)}$ with probability proportional to $\exp(-kF(x;\cD))$ is $(\eps,\delta(\eps))$-differentially private where 
\begin{align*}
 \delta(\eps)\leq \deltacurve{\cN(0,1)}{\cN\lp\frac{G\sqrt{k}}{n\sqrt{\mu}},1\rp}(\epsilon).
 \end{align*}
 The excess empirical risk is bounded by $\frac{d}{k}$.
\end{theorem}
% \Gopi{Do we have the factor 2 in the privacy bound?}
\begin{proof}
The privacy guarantee follows directly from our main result Theorem~\ref{thm:privacy_technical}, and the bound on excess empirical loss can be proved by Lemma~\ref{lm:utility_tech}.
\end{proof}
%\begin{remark}
%If we use the result about Gaussian mechanism, {\cite[Theorem 3.22]{DR14}}, we can set $k=\frac{\eps^2n^2\mu}{8G^2\log(1.25/\delta)}$ in the strongly convex case with expected excess empirical loss $\frac{8G^2d\log(1.25/\delta)}{\eps^2n^2\mu} $. In the general convex case, we can set $k=\frac{\eps^2n^2\mu}{8G^2\log(1.25/\delta)}$ where $\mu=\frac{4G\sqrt{d\log(1.25/\delta)}}{\eps n D}$ and get expected excess empirical loss $\frac{4GD\sqrt{d\log(1.25/\delta)}}{\eps n}$.
%\end{remark}

%\Gopi{The bound from {\cite[Theorem 3.22]{DR14}} is only stated in the regime $\eps\in (0,1).$ Find an other reference or use the bounds from Section 3.}
%Tat  I think just stating for eps between 0,1 is okay just for statement simplicity

Before we state the implementation results on DP-ERM, we need the following technical lemma:
\begin{lemma}
\label{lm:privacy_curve_bound}
For any constants $1/2>\delta>0$ and $\eps>0$, if $|s|\leq \sqrt{2\log(1/(2\delta))+2\eps}-\sqrt{2\log(1/(2\delta))}$,
one has
\begin{align*}
  \deltacurve{\cN(0,1)}{\cN(s,1)}\leq \delta.  
\end{align*}
\end{lemma}
\begin{proof}
By Equation~(\ref{eqn:Gaussian_privacycurve}), we know that
\begin{align*}
    \deltacurve{\cN(0,1)}{\cN(s,1)}(\eps)\leq \Phi\lp -\frac{\eps}{s}+\frac{s}{2} \rp.
\end{align*}
Without loss of generality, we assume $s\geq0$ and
want to find an appropriate value of $s$ such that $\Phi\lp -\frac{\eps}{s}+\frac{s}{2} \rp\leq \delta$. 
Denote $t\defeq \Phi^{-1}(1-\delta)$ and since $1-\Phi(t)\le \frac{1}{2}\exp(-t^2/2)$ for $t>0$, we know that $t\leq \sqrt{2\log(1/(2\delta))}$.
It is equivalent to solve the equation $\frac{\eps}{s}-\frac{s}{2}\geq t$, which is equivalent to $0\leq s\leq \sqrt{t^2+2\eps}-t$.
Note that $\sqrt{t^2+2\eps}-t$ decreases as $t$ increases, which implies that we can set $s\leq \sqrt{2\log(1/(2\delta))+2\eps}-\sqrt{2\log(1/(2\delta))}$.
\end{proof}

Combining the sampling scheme (Theorem~\ref{thm:sampler}) and our analysis on DP-ERM, we can get the efficient implementation results on DP-ERM directly.

 

\begin{theorem}[DP-ERM Implementation]
\label{thm:DPERM_impl}
With same assumptions in Theorem~\ref{thm:DPERM}, and assume $f(\cdot;s)$ is $G$-Lipschitz over $\cK$ for all $s$.
For any constants $1/10> \delta>0$ and $ \eps> 0$, there is an efficient sampler to solve DP-ERM which has the following guarantees:
\begin{itemize}
    \item The scheme is $(\eps,\delta)$-differentially private;
    \item The expected excess empirical loss is bounded by $\frac{GD\sqrt{d}}{n(\sqrt{\log(1/\delta)+\eps}-\sqrt{\log(1/\delta)})}$.
    In particular, if $\eps< 1/10$, the expected excess empirical loss is bounded by
    $
        \frac{2GD\sqrt{d\log(1/\delta)}}{\eps n}.
    $
    If $\eps \geq \log(1/\delta)$, the expected excess empirical loss is bounded by $ O(\frac{GD\sqrt{d}}{n\sqrt{\eps}})$.
    \item The scheme takes 
    \begin{align*}
        \Theta\lp\frac{\eps^2n^2}{\log(1/\delta)}\log^2(\frac{nd\eps}{\delta})\rp
    \end{align*}
    queries to the values on $f(x;s)$ in expectation and takes the same number of samples from some Gaussian restricted to the convex set $\mathcal{K}$.
\end{itemize}
\end{theorem}

\begin{proof}
% {\cite[Theorem 3.22]{DR14}} shows that $\delta(N(0,1)||N(\Delta,1))(\epsilon)\leq\delta$
% if 
% \[
% \frac{\sqrt{2\ln(1.25/\delta)}\Delta}{\epsilon}\leq1.
% \]

By Lemma~\ref{lm:privacy_curve_bound}, we can set $s=\sqrt{2\log(3/(4\delta))+2\eps}-\sqrt{2\log(3/(4\delta))}$ to make $\deltacurve{\cN(0,1)}{\cN(s,1)}\leq2\delta/3$.
For our setting, Theorem \ref{thm:DPERM} shows that we have $s=\frac{G\sqrt{k}}{n\sqrt{\mu}}$
and hence we can take
\begin{align*}
    k=\frac{2\mu n^{2}\lp\sqrt{\log(3/(4\delta))+\eps}-\sqrt{\log(3/(4\delta))}\rp^2}{G^{2}}.
\end{align*}
Putting it into the excess empirical loss bound of $\frac{d}{k}+\frac{\mu D^{2}}{2}$
and setting $\mu=\frac{G\sqrt{d}}{ n D\lp\sqrt{\log(3/(4\delta))+\eps}-\sqrt{\log(3/(4\delta))}\rp}$,
we get the result on the empirical loss.

Particularly, consider the case when $\eps<1/10$.
We know the excess empirical loss is bounded by $\frac{GD\sqrt{d}}{n(\sqrt{\log(3/(4\delta))+\eps}-\sqrt{\log(3/(4\delta))})}$.
Note that $1+\frac{x}{2}-\frac{x^2}{8}\leq \sqrt{1+x}\leq 1+\frac{x}{2}$ for $x\geq 0$.
Under the assumption that $\delta,\eps\in(0,\frac{1}{10})$, we know $\frac{GD\sqrt{d}}{n(\sqrt{\log(3/(4\delta))+\eps}-\sqrt{\log(3/(4\delta))})}\leq \frac{2GD\sqrt{d\log(4/(5\delta))}}{n\eps}$.
The case when $\eps\geq \log(1/\delta)$ also follows similarly.
% Particularly, if $\eps\leq \log(3/(4\delta))$, we know $0\leq s\leq \frac{\eps}{\sqrt{2\log(1/\delta)}}\leq \sqrt{2\log(3/(4\delta))+2\eps}-\sqrt{2\log(3/(4\delta))}\leq \sqrt{t^2+2\eps}-t$.
% Hence we can set $k$.
% \gnote{Can remove the text from here}
% Particularly, consider the case when $\eps<1/2$.
% By Equation~\eqref{eqn:Gaussian_privacycurve_approx}, we know that 
% \begin{align*}
%     \deltacurve{\cN(0,1)}{\cN(s,1)}(\eps)  \le \frac{1}{2} \exp\lp-\frac{1}{2}\lp\frac{\eps}{s}-\frac{s}{2}\rp^2\rp  \text{ if } \eps \geq \frac{s^2}{2}.
% \end{align*}
% Hence we know $\deltacurve{\cN(0,1)}{\cN(s,1)}(\eps)\leq 4\delta/5$ if
% $
%     \frac{s\sqrt{2\log(3/(4\delta))}}{\eps}\leq 1
% $
% under the assumptions that $\eps< 1/2$ and $\delta< 1/2$.
% For our setting, Theorem \ref{thm:DPERM} shows that we have $s=\frac{2G\sqrt{k}}{n\sqrt{\mu}}$
% and hence we can take
% \[
% k=\frac{\epsilon^{2}n^{2}\mu}{8G^{2}\log(3/(4\delta))}.
% \]
% Putting it into the excess empirical loss bound of $\frac{d}{k}+\frac{\mu D^{2}}{2}$
% and setting $\mu=\frac{4G\sqrt{d\log(3/(4\delta))}}{\epsilon n D}$,
% we get the result on the empirical loss.
% The case when $\eps\geq \log(1/\delta)$ follows directly by the calculation.
% \gnote{to here. Instead just use upper bounds on the general bound just derived to get the special cases.}

To make it algorithmic, we apply Theorem~\ref{thm:sampler} with the accuracy on the total variation distance to be $\min\{\delta/3,\frac{1}{cn^c\eps}\}$ for some large enough constant $c$. This leads to $(\epsilon,\delta)$-DP and an extra empirical loss and hence we use $\log(1/\delta)$ rather than $\log(3/(4\delta))$ or $\log(4/(5\delta))$ in the final loss term. 

%Note that $1+\frac{x}{2}-\frac{x^2}{8}\leq \sqrt{1+x}\leq 1+\frac{x}{2}$ for $x\geq 0$. \gnote{Why is this fact mentioned here?}
The running time follows from Theorem~\ref{thm:sampler}.
\end{proof}


\subsection{DP-SCO and Generalization Error}
As mentioned before, one can reduce the DP-SCO \eqref{eq:DPSCO} to DP-ERM \eqref{eq:DPERM} by the iterative localization technique proposed by \cite{FKT20}.
But this method forces us to design different algorithms for DP-ERM and DP-SCO, and may lead to a large constant in the final loss.
In this section, we show that the exponential mechanism can achieve both the optimal empirical risk for DP-ERM and the optimal population loss for DP-SCO by simply changing the parameters.
The bound on the generalization error works beyond differential privacy and can be useful for other (non-private) optimization settings.


% We define the log-Sobolev inequality first.
% \begin{definition}[Log-Sobolev Inequality]
% We say a distribution $\nu$ satisfies a logarithmic Sobolev inequality with a constant $C$ if for all smooth
% function $g:\R^n\rightarrow \R$ with $\E_{\nu}[g^2]<\infty$, one has
% \begin{equation}
%     \label{eq:log-sobolev}
%      \Ent_\nu[g^2] \leq 2C \mathbb{E}_{\nu}\left[\|\nabla g\|_2^{2}\right]
% \end{equation}
% where $\Ent_\nu[f]=\mathbb{E}_{\nu}\left[f \log \lp \frac{f}{\E_\nu f}\rp\right]$.
% \end{definition}

% We have the following useful lemma for Log-Sobolev Inequality (LSI):
% \begin{lemma}[\cite{BE85}]
% \label{lm:strongly_convexity_LSI}
% If $\nu$ is $\mu$-strongly log-concave, then $\nu$ satisfies LSI with constant $C=1/\mu$.
%If $\nu$ is obtained by restricting a standard Gaussian distribution with variance $\sigma^2$ to some subset $\cK\subset \R^d$, then $\nu$ satisfies the log-Sobolev inequality (\ref{eqn:log-sobolev}) with $C=\sigma^2.$
% \end{lemma}

% The Talagrand transportation inequality is closely related to LSI. 
The proof will make use of one famous inequality: \emph{Talagrand transportation inequality}.
Recall for two probability distributions $\nu_1,\nu_2$, the Wasserstein distance is equivalently defined as $$W_2(\nu_1,\nu_2)=\inf_\Gamma \lp\E_{(x_1,x_2)\sim \Gamma} \norm{x_1-x_2}_2^2\rp^{1/2},$$ where the infimum is over all couplings $\Gamma$ of $\nu_1,\nu_2.$
\begin{theorem}[Talagrand transportation inequality]{\cite[Theorem 1]{OV00}}
\label{thm:TTI}
Let $\d\pi\propto e^{-F(x)}\d x$ be a $\mu$-strongly log-concave probability measure on $\cK\subseteq\R^d$ with finite moments of order 2. 
For all probability measure $\nu$ absolutely continuous w.r.t. $\pi$ and with finite moments of order 2,
we have
\begin{align*}
\mathrm{W}_2(\nu,\pi)\leq\sqrt{\frac{2}{\mu}\mathrm{D}_{KL}(\nu,\pi)}.   
\end{align*}
\end{theorem}

% To bound the KL divergence of the objective distributions, we have the following lemma.


% \begin{lemma}
% \label{lm:LSI_KL}
% Suppose $F,\TF$ are two $\mu$-strongly convex functions over $\cK\subseteq \R^d$, and $F-\TF$ is $G$-Lipschitz over $\cK$.
% For any $k>0$, if we let $\pi\propto e^{-kF}$ and $\nu\propto e^{-k\TF}$ be two probability distributions on $\cK$, then we have
% \begin{align*}
%     \mathrm{D}_{\alpha}(\pi,\nu)\leq \mathrm{D}_{\alpha}\lp\cN(0,1),\cN(\frac{G\sqrt{k}}{\sqrt{\mu}},1)\rp,\forall \alpha\in[0,\infty].
% \end{align*}
% \end{lemma}

% \begin{proof}
% Recall in Theorem~\ref{thm:privacy_technical_tradeoff}, we proved for all $z\in[0,1]$, one has
% \begin{align*}
%     \tradeoff{\pi}{\nu}(z)\ge \tradeoff{\cN\lp 0,1\rp}{\cN\lp\frac{G\sqrt{k}}{\sqrt{\mu}},1\rp}(z).
% \end{align*}

% By Theorem 2.10 in \cite{dong2019gaussian}, there exists a randomized algorithm $\Proc$ such that $\Proc\lp\cN\lp 0,1\rp\rp=\pi$ and $\Proc\lp\cN\lp\frac{G\sqrt{k}}{\sqrt{\mu}},1\rp\rp=\nu$.
% Due to the data processing inequality (See \cite{EH14}, for example), the statement holds.
% \end{proof}

% \begin{proof}
% Let $g\defeq \frac{\d \pi}{\d \nu}$ and one has
% \begin{align*}
%     \mathrm{D}_{KL}(\pi,\nu)=&\E_{\pi}[\log g]\\
%     =& \E_{\nu}[g \log g] \\
%     \leq & \E_{\nu}[g \log g]-\E_{\nu}[g]\log\E_{\nu}[ g],
% \end{align*}
% where the last inequality follows by that $\E_{\nu}[g]=1$ and $ \log\E_{\nu}[g]= 0$.
% Note that both $F$ and $\TF$ are $\mu$-strongly convex. 
% By Lemma~\ref{lm:strongly_convexity_LSI} and log-Sobolev inequality (\ref{eq:log-sobolev}), we have
% % \Gopi{Don't we have  $$\E_{\nu}[g \log g]-\E_{\nu}[g]\log\E_{\nu}[g]
%     % \leq \frac{2}{k\mu}\E_{\nu}[\|\nabla \sqrt{g}\|_2^2]$$ by LSI?}
% \begin{align*}
%     &~\E_{\nu}[g \log g]-\E_{\nu}[g]\log\E_{\nu}[ g]\\
%     \leq&~ \frac{2}{k\mu}\E_{\nu}[\|\nabla \sqrt{g}\|_2^2]\\
%     = & ~\frac{2}{k\mu}\E_{\nu}[\|\frac{\sqrt{g}}{2}\nabla \log g\|_2^2]\\
%     \leq & ~\frac{1}{2k\mu}(Gk)^2\E_{\nu}[g]\\
%     =& ~\frac{G^2k}{2\mu},
% \end{align*}
% where the fourth line follows by $\|\nabla \log g\|_2^2\leq (Gk)^2$ due to the Lipschitz assumption.
% Hence we complete the proof.
% \end{proof}

To prove our main result on bounding the generalization error of sampling mechanism, we need the following lemma.


\begin{lemma}[Lemma 7 in \cite{BE02}]
\label{lm:generalization_error_erm}
For any learning algorithm $\cA$ and dataset $\cD=\{s_1,\cdots,s_n\}$ drawn i.i.d from the underlying distribution $\cP$, let $\cD'$ be a neighboring dataset formed by replacing a random element of $\cD$ with a freshly sampled $s'\sim \cP$. If $\cA(\cD)$ is the output of $\cA$ with $\cD$, then
\begin{align*}
    \E_{\cD}[\HF(\cA(\cD))-F(\cA(\cD);\cD)]=\E_{\cD,s'\sim\cP,\cA}\Big[f(\cA(\cD);s')-f(\cA(\cD');s') \Big].
\end{align*}
\end{lemma}

Now we begin to state and prove our main result on the generalization error.
\begin{theorem}
\label{thm:generalization_error}
Suppose $\{f(\cdot,s)\}$ is a family $\mu$-strongly convex functions over $\cK$ such that $f(x;s)-f(x;s')$ is $G$-Lipschitz for all $s,s'$.
For any $k>0$ and dataset $\cD=\{s_1,s_2,\cdots,s_n\}$ drawn i.i.d from the underlying distribution $\cP$, let $\cD'$ be a neighboring dataset formed by replacing a random element of $\cD$ with a freshly sampled $s'\sim \cP$,
\begin{align*}
    \mathrm{W}_{2}(\pi_\cD,\pi_{\cD'})\leq \frac{G}{n\mu}.
\end{align*}
If we sample our solution from density $\pi_\cD(x) \propto e^{-kF(x;\cD)}$, we can bound the excess population loss as:
\begin{align*}
    \E_{\cD,x\sim \pi_\cD}[\HF(x)]-\min_{x\in\cK}\HF(x)\leq \frac{G^2}{\mu n}+ \frac{d}{k}.
\end{align*}
\end{theorem}

\begin{proof}
Recall that
\begin{align*}
    F(x;\cD)=\frac{1}{n}\sum_{s_i\in\cD}f(x;s_i).
\end{align*}
We form a neighboring data set $\cD'$ by replacing a random element of $\cD$ by a freshly sampled $s'\sim \cP.$
Let $\pi_\cD\propto e^{-kF(x;\cD)}$ and $\pi_{\cD'}\propto e^{-kF(x;\cD')}$. By Corollary~\ref{cor:divergence_privacy}, we have
\begin{align*}
    \mathrm{D}_{KL}(\pi_\cD,\pi_{\cD'})\leq \frac{G^2k}{2n^2\mu}.
\end{align*}
By the assumptions, we know both $F(x;\cD)$ and $F(x;\cD')$ are $\mu$-strongly convex and by Theorem~\ref{thm:TTI}, we have
\begin{align*}
    \mathrm{W}_2(\pi_\cD,\pi_{\cD'})\leq \sqrt{\frac{2}{k\mu}\mathrm{D}_{KL}(\pi_\cD,\pi_{\cD'})} \leq \frac{G}{n\mu}.
\end{align*}
By Lemma~\ref{lm:generalization_error_erm} and properties of Wasserstein distance, we have
\begin{align*}
    \E_{\cD,x\sim \pi_{\cD}}[\HF(x)-F(x;\cD)]=&~\E_{\cD,s'\sim \cP}\lb\E_{x\sim \pi_\cD}f(x;s')-\E_{x'\sim \pi_{\cD'}}f(x';s')\rb\\
    =&~\E_{\cD,s'\sim \cP}\lb\E_{x\sim \pi_\cD}\lb f(x;s')-f(x;s'')\rb -\E_{x'\sim \pi_{\cD'}}\lb f(x';s')-f(x';s'')\rb \rb \tag{where $s''$ is chosen arbitrarily, note that $\E_{\cD,x\sim \pi_\cD}[f(x;s'')]=\E_{\cD',x'\sim \pi_{\cD'}}[f(x';s'')]$}\\
    \leq &~G\cdot \mathrm{W}_{2}(\pi_\cD,\pi_{\cD'}) \tag{$f(x;s')-f(x;s'')$ is $G$-Lipschitz}\\
    \leq &~ \frac{G^2}{n\mu}.
\end{align*}
Hence, we know that
\begin{align*}
    \E_{\cD,x\sim\pi_{\cD}}[\hat{F}(x)]-\min_{x\in\cK}\HF(x)\leq &~ \E_{\cD,x\sim\pi_{\cD}}[\hat{F}(x)]-\E_{\cD}[\min_{x\in\cK}F(x;\cD)]\\
    \leq &~ \E_{\cD,x\sim\pi_{\cD}}[\hat{F}(x)-F(x;\cD)]+\E_{\cD,x\sim\pi_{\cD}}[F(x;\cD)-\min_{x\in\cK}F(x;\cD)]\\
    \leq &~ \frac{G^2}{n\mu}+\E_{\cD,x\sim\pi_{\cD}}[F(x;\cD)-\min_{x\in\cK}F(x;\cD)]\\
    \leq &~ \frac{G^2}{n\mu}+\frac{d}{k},
\end{align*}
where the last inequality follows from Lemma~\ref{lm:utility_tech}.
\end{proof}

With the bounds on generalization error, we can get our first result on DP-SCO.
\begin{theorem}[DP-SCO]
\label{thm:dpsco}
Let $\epsilon>0$, $\cK\subseteq \R^d$ be a convex set of diameter $D$ and $\{f(\cdot;s)\}_{s\in\mathcal{D}}$ be a family of convex functions over $\cK$ such that $f(x;s)-f(x;s')$ is $G$-Lipschitz for all $s,s'$.
For any data-set $\cD$ and $k>0$, sampling $x^{(priv)}$ with probability proportional to $\exp\left(-k(F(x;\cD)+\mu\|x\|_2^2/2)\right)$ is $(\epsilon,\delta(\eps))$-differentially private, where
\begin{align*}
 \delta(\eps)\leq \deltacurve{\cN(0,1)}{\cN\lp\frac{G\sqrt{k}}{n\sqrt{\mu}},1\rp}(\epsilon).
\end{align*}
If users in the data-set $\cD$ are drawn i.i.d. from the underlying distribution $\cP$, the excess population loss is bounded by $\frac{G}{n\mu}+\frac{d}{k}+\frac{\mu D^2}{2}$. 
Moreover, if $\{f(\cdot;s)\}_{s\in\mathcal{D}}$ are already $\mu$-strongly convex, then sampling 
$x^{(priv)}$ with probability proportional to $\exp(-kF(x;\cD))$ is $(\eps,\delta(\eps))$-differentially private where 
\begin{align*}
 \delta(\eps)\leq \deltacurve{\cN(0,1)}{\cN\lp\frac{G\sqrt{k}}{n\sqrt{\mu}},1\rp}(\epsilon).
 \end{align*}
 The excess population loss is bounded by $\frac{G}{n\mu}+\frac{d}{k}$.
% \Gopi{Add what happens if $f(\cdot;s)$ are already $\mu$-strongly convex.}
\end{theorem}
% Combining our result on DP-ERM (Theorem~\ref{thm:DPERM_impl}) and the bound on generalization error (Theorem~\ref{thm:generalization_error}), we can get our result about DP-SCO.

\begin{proof}
The first part about privacy is a restatement of our result on DP-ERM (Theorem~\ref{thm:DPERM_impl}).
The excess population loss (See Equation~\eqref{eqn:population_loss}) follows from the bound on generalization error (Theorem~\ref{thm:generalization_error}) and utility guarantee (Lemma~\ref{lm:utility_tech}).
\end{proof}
%%%%%%%%%%%
%%%%%%%%%%%%
\begin{comment}

===================\textcolor{red}{OLD TEXT BEGIN}===========================================
\begin{lemma}[Lemma 7 in \cite{BE02}]
\label{lm:generalization_error_erm}
For any learning algorithm $\cA$ and neighboring data-sets $\cD=\{s_1,\cdots,s_n\}$, $\cD^i=\{s_1,\cdots,s_{i-1},s_i',s_{i+1},\cdots,s_n\}$ drawn i.i.d from the underlying distribution $\cP$, let $\cA(\cD)$ be the output of $\cA$ with $\cD$. 
One has
\begin{align*}
    \E_{\cD}[\HF(\cA(\cD))-F(\cA(\cD);\cD)]=\E_{\cD,s_i'\sim\cP}\Big[\E[f(\cA(\cD);s_i')-f(\cA(\cD^{i});s_i')] \Big].
\end{align*}
\end{lemma}

Now we begin to state and prove our main result on the generalization error.
\begin{theorem}
\label{thm:generalization_error}
Suppose $\{f(\cdot,s)\}$ is a family of $G$-Lipschitz and $\mu$-strongly convex functions over $\cK$.
For any $k>0$ and suppose the $n$ samples $\cD=\{s_1,s_2,\cdots,s_n\}$ are drawn i.i.d from the underlying distribution $\cP$, then if we draw a sample $x^{(sol)}$ from density $\propto e^{-kF(x;\cD)}$, we have
\begin{align*}
    \E[\HF(x^{(sol)})]-\min_{x\in\cK}\HF(x)\leq \frac{2G^2}{\mu n}+ \frac{d}{k}.
\end{align*}
\end{theorem}

\begin{proof}
Recall that $\cD=\{s_1,\cdots,s_n\}$.
\begin{align*}
    F(x;\cD)=\frac{1}{n}\sum_{s_i\in\cD}f(x;s_i).
\end{align*}
We consider any neighboring data set $\cD^i=\{s_1,\cdots,s_{i-1},s_i',s_{i+1},\cdots,s_n\}$.

Let $\pi\propto e^{-kF(x;\cD)}$ and $\hat{\pi}\propto \exp^{-kF(x;\cD^i)}$.
By the assumptions, we know both $F(x;\cD)$ and $F(x;\cD^i)$ are $\mu$-strongly convex and by Theorem~\ref{thm:TTI}, we have
\begin{align*}
    \mathrm{W}_2(\pi,\hpi)\leq \sqrt{\frac{2}{k\mu}\mathrm{D}_{KL}(\pi,\hpi)}.
\end{align*}

As $kF(x;\cD)-kF(x;\cD^i)$ is $2Gk/n$ Lipschitz,
%\Yintat{Is this a 2Gk/n?}, 
by Lemma~\ref{lm:LSI_KL} we have
\begin{align*}
    \mathrm{D}_{KL}(\pi,\hpi)\leq \frac{2G^2k}{n^2\mu}.
\end{align*}
Thus we know that 
\begin{align*}
    \mathrm{W}_{2}(\pi,\hpi)\leq \frac{2G}{n\mu}.
\end{align*}

By Lemma~\ref{lm:generalization_error_erm} , the assumption on Lipschitz continuity and definition of Wasserstein distance, we have
\begin{align*}
    \E_{\cD,x\sim \pi}[\HF(x)-F(x;\cD)]=&~\E_{\cD,s_i'}[\E_{x\sim \pi}f(x;s_i')-\E_{x'\sim \hpi}f(x';s_i')]\\
    =&~\E_{\cD,s_i',s''\sim \cP}[\E_{x\sim \pi}\lp f(x;s_i')-f(x;s'')\rp -\E_{x'\sim \hpi}\lp f(x';s_i')-f(x';s'')\rp]\\
    \leq &~G\cdot \mathrm{W}_{2}(\pi,\hpi) \tag{$f(x;s_i')-f(x;s'')$ is $G$-Lipschitz}\\
    \leq &~ \frac{2G^2}{n\mu}.
\end{align*}
Hence, we know that
\begin{align*}
    \E_{\cD,x\sim\pi}[\hat{F}(x)]-\min_{x\in\cK}\HF(x)\leq &~ \E_{\cD,x\sim\pi}[\hat{F}(x)]-\E_{\cD}[\min_{x\in\cK}F(x;\cD)]\\
    \leq &~ \E_{\cD,x\sim\pi}[\hat{F}(x)-F(x;\cD)]+\E_{\cD,x\sim\pi}[F(x;\cD)-\min_{x\in\cK}F(x;\cD)]\\
    \leq &~ \frac{2G^2}{n\mu}+\E_{\cD,x\sim\pi}[F(x;\cD)-\min_{x\in\cK}F(x;\cD)]\\
    \leq &~ \frac{2G^2}{n\mu}+\frac{d}{k},
\end{align*}
where the last inequality follows from Lemma~\ref{lm:utility_tech}.
\end{proof}
======================= \textcolor{red}{OLD TEXT END} ===================================================
\end{comment}



% As suggested in \cite{FKT20,KLL21,AFKT21}, we can reduce the DP-SCO to DP-ERM by the localization technique. 
% As mentioned before, 
% Based on the technique, \cite{KLL21} gave a framework (Algorithm~\ref{alg:local}) below:
% \begin{algorithm2e}
% \caption{Iterative Localized Algorithm Framework $\cA'$}
% \label{alg:local}
% {\bf Input:} A family of $G$-Lipschitz and $\mu$-strongly convex function $f:\cK\times\Xi\rightarrow \R$, initial point $x_0\in \K$ and privacy parameter $\epsilon,\delta$.

% {\bf Process:}
% Set $k=\lceil \log n \rceil$\;
% \For{$i=1,\cdots,k$}
% {
% Set $\epsilon_i=\epsilon/2^i,n_i=n/2^{i},\eta_i=\eta/2^{5i},\delta_i=\delta/2^i$\;

% Apply $(\epsilon_i,\delta_i)$-DP ERM algorithm $\cA_{\epsilon_i,\delta_i}$ over $\cK_{i}=\left\{x \in \cK:\left\|x-x_{i-1}\right\|_{2} \leq 2G\eta_in_i\right\}$ \label{ln:sub-procedure_DP_ERM}\\ with the function
% $\HF_i(x)=\frac{1}{n_i}\sum_{j\in S_i}f(x,x_j)+\frac{1}{\eta_in_i}\|x-x_{i-1}\|^2$ where $S_i$ consists of $n_i$ samples drawn i.i.d. from $\mathcal{P}$\;

% Let $x_i$ be the output of the ERM algorithm\;

% }
% {\bf Return:} The final iterate $x_k$\;
% \end{algorithm2e}

% Adopting our scheme for DP-ERM in the Line~\ref{ln:sub-procedure_DP_ERM} in Algorithm~\ref{alg:local} and use the Theorem~5.1 in \cite{KLL21}, we have the following results directly:
We give an implementation result of our DP-SCO result.
\begin{theorem}[DP-SCO Implementation]
\label{thm:dpsco_impl}
With same assumptions in Theorem~\ref{thm:dpsco}, and assume $f(\cdot;s)$ is $G$-Lipschitz over $\cK$ for all $s$.
For $0<\delta<\frac{1}{10}$ and $0 < \eps<\frac{1}{10}$, there is an efficient algorithm to solve DP-SCO which has the following guarantees:
\begin{itemize}
    \item The algorithm is $(\eps,\delta)$-differentially private;
    \item The expected population loss is bounded by
    \begin{align*}
        GD\lp\frac{2\sqrt{\log(1/\delta)d}}{\eps n}+\frac{2}{\sqrt{n}}\rp,
    \end{align*}
    where $c>0$ is an arbitrary constant to be chosen. 
    %The expected population loss for strongly-convex case is bounded by
    %\begin{align*}
    %    O(\frac{G^2}{\mu}(\frac{1}{n}+\frac{d\log(1/\delta)}{\eps^2n^2})).
    %\end{align*}
    \item The algorithm takes 
    \begin{align*}
        O\lp\min\lc\frac{\eps^2n^2}{\log(1/\delta)},nd\rc\log^2\lp\frac{\eps nd}{\delta}\rp\rp
    \end{align*}
    queries of the values of $f(\cdot,s_i)$ in expectation and takes the same number of samples from some Gaussian restricted to the convex set $\mathcal{K}$.
\end{itemize}
\end{theorem}
\begin{remark}
As for the non-typical case when $\eps\geq 1/10$, one can use the bound in Theorem~\ref{thm:DPERM_impl} and the bound on generalization error (Theorem~\ref{thm:generalization_error}) .
Particularly, one can achieve expected population loss $O\lp GD\lp\frac{\sqrt{d}/n}{\sqrt{\log(1/\delta)+\eps}-\sqrt{\log(1/\delta)}}+\frac{1}{\sqrt{n}}\rp\rp$.
\end{remark}

\begin{proof}
By Theorem~\ref{thm:dpsco}, sampling from $\exp(-k(F(x;\cD)+\mu \|x\|_2^2/2))$ when $k\leq \frac{\eps^2n^2\mu}{2G^2\log(3/(4\delta))}$ is $(\eps,2\delta/3)$-DP.
Besides, 
we can set  $k=\frac{\mu}{G^2}\min\{\frac{\epsilon^{2}n^{2}}{2\log(3/(4\delta))},2nd\}$ for arbitrarily large constant $c>0$ to make the mechanism $(\eps,2\delta/3)$-differentially private, achieving tight population loss and decrease the running time.
Then the population loss is upper bounded by
\begin{align*}
    \frac{d}{k}+\frac{\mu D^2}{2}+\frac{G^2}{\mu n}=& \frac{G^2}{\mu}\max\lc\frac{2\log(3/(4\delta))d}{\eps^2n^2},\frac{1}{2n}\rc+\frac{\mu D^2}{2}+\frac{G^2}{\mu n}.
\end{align*}
% where the generalization error term $\frac{2G^2}{\mu n}$ follows from bounding Wasserstein distance by Theorem~\ref{thm:generalization_error} and making use of Lemma~\ref{lm:generalization_error_erm} the assumption that $f(;s')$ is $G$-Lipschitz for any $s'$.

By setting $\mu=\frac{G}{D}\sqrt{2(\frac{2\log(3/(4\delta))d}{\eps^2 n^2}+\frac{1}{2n})}$, the population loss is upper bounded by
\begin{align*}
    GD\sqrt{\frac{4\log(3/(4\delta))d}{\eps^2 n^2}+\frac{1}{n}}+GD\sqrt{\frac{1}{n}}\leq GD\lp\frac{2\sqrt{\log(3/(4\delta))d}}{\eps n}+\frac{2}{\sqrt{n}}\rp.
\end{align*}

To make it algorithmic, we also apply Theorem~\ref{thm:sampler} with the accuracy on the total variation distance to be $\min\{\delta/3,\frac{1}{cn^c}\}$ for some large enough constant $c$. This leads to an extra empirical loss and hence we use $\log(1/\delta)$ rather than $\log(3/(4\delta))$ in the final loss term. The runtime follows from Theorem~\ref{thm:sampler}.
\end{proof}

\begin{comment}
\begin{remark}
An alternative way to achieve $O(nd)$ first order oracle (query sub-gradient) complexity is to run proximal point algorithm on non-smooth functions, which can be shown with similar contractility as running stochastic gradient descent on smooth functions. Thus replacing the SGD in \cite{FKT20} by proximal point algorithm, we can achieve $\Tilde{O}(nd)$ gradient queries for DP-SCO. But this method takes gradient queries while we query values, and it requires more memory space $\Omega(d^2)$ compared to our algorithm.
\end{remark}

\end{comment}


\section{Information-theoretic Lower Bound for DP-SCO}
\label{sec:infolower}

 In this section, we prove an information-theoretic lower bound for the query complexity required for DP-SCO (with value queries), which matches (up to some logarithmic terms) the query complexity achieved by our algorithm (in Theorem \ref{thm:dpsco_impl}).
Our proof is similar to the previous works like \cite{ACCD12,DJWW15} with some modifications.


% In fact, the query complexity of our (non-private) sampling scheme and algorithm for solving DP-SCO matches the  information-theoretic minimax lower bounds (up to some logarithmic terms).
% Our proof is similar to the previous works like \cite{ACCD12,DJWW15} with some modifications.
% For completeness, we present the proof here.
%We will demonstrate the result of DP-SCO first.

Before stating the lower bound, we define some notations.
Recall that we are given a set $\cD$ of $n$ samples (users) $\{s_1,\cdots,s_n\}$.
Let $\mathbb{A}_k$ be the collection of all algorithms that observe a sequence of $k$ data points $(Y^1,\cdots,Y^k)$ with $Y^t=f(X^t;S^t)$ where %$s^t$ is an independent sample randomly drawn from the underlying distribution $\mathcal{P}$, 
$S^t\in \cD$ and $X^t\in \cK$ are chosen arbitrarily and adaptively by the algorithm (and possibly using some randomness).

For the lower bound, we only consider linear functions, that is we define $f(x;s)\defeq\langle x,s\rangle$. And let $\cP_G$ be the collection of all distributions such that if $\cP\in\cP_G$, then $\E_{s\sim\cP}\|s\|_2^2\leq G^2$.

% \begin{align*}
%     \cF_G\defeq\{(f,\mathcal{P}):f(x;s)=\langle x,s\rangle,\E_{s\sim \cP}[\|s\|^2_2]\leq G^2\}
% \end{align*}
% be a class of linear functionals.
And we define the optimality gap
\begin{align*}
    \eps_k(\cA,\cP,\cK)\defeq&~ \E_{\cD\sim\cP^n,\cA}[\HF(\hat{x}(\cD))]-\inf_{x\in \cK}\hat{F}(x),
\end{align*}
where $\HF(x) = \E_{s\sim\cP} f(x; s)$, $\hat{x}$ is the output the algorithm $\cA$ given the input dataset $\cD$ and the expectation is over the dataset $\cD\sim \cP^n$ and the randomness of the algorithm $\cA.$ Note that we can rewrite the optimality gap as:
\begin{align*}
    \eps_k(\cA,\cP,\cK)=&~ \E_{\cD\sim\cP^n,\cA}[\HF(\hat{x}(\cD))]-\inf_{x\in \cK}\hat{F}(x)\\
    =& \E_{s\sim\cP}\Big[\E_{\cD\sim\cP^n,\cA}f(\hat{x}(\cD);s)]\Big]-\inf_{x\in\cK}\E_{s\sim\cP}[f(x;s)]\\
    =& \E_{s\sim\cP,\cD\sim\cP^n,\cA}[\hat{x}(\cD)^\top s]-\inf_{x\in\cK}\E_{s\sim \cP}[x^\top s].
\end{align*}
The minimax error is defined by
\begin{align*}
    \eps_k^*(\cP_G,\cK)\defeq \inf_{\cA\in \mathbb{A}_k}\sup_{\mathcal{P}\in \cP_G}\eps_k(\cA,\cP,\cK).
\end{align*}
%where the expectation is taken over the observations $(Y^1,\cdots,Y^k)$ and any additional randomness in $\cA$.

\begin{theorem}
%\Yintat{Make this statement easier to understand. Maybe add something like: In particular, there is no algorithm that use something something and get error something something}
\label{thm:info_bound}
Let $\cK$ be the $\ell_2$ ball of diameter $D$ in $\R^d$, then
\begin{align*}
    \eps_k^*(\cP_G,\cK)\geq \frac{GD}{16}\min\left\{1,\sqrt{\frac{d}{4k}}\right\}.
\end{align*}
In particular, for any (randomized) algorithm $\cA$ which can observe a sequence of data points $(Y^1,\cdots,Y^k)$ with $Y^t=f(X^t;S^t)$ where $S^t\in\cD=\{s_1,s_2,\dots,s_n\}$ and $X^t\in\cK$ are chosen arbitrarily and adaptively by $\cA$,
there exists a distribution $\cP$ over convex functions such that $\E_{s\sim \mathcal{P}}[\|\nabla f(x,s)\|_2^2]\leq G^2$ for all $x\in \cK$, such that the output $\hx$ of the algorithm satisfies
\begin{align*}
    \E_{s\sim\cP}\Big[ \E_{\cD\sim\cP^n,\cA}f(\hx;s)]\Big]-\min_{x\in\cK}\E_{s\sim\cP}[f(x;s)]\geq \frac{GD}{16}\min\left\{1,\sqrt{\frac{d}{4k}}\right\}.
\end{align*}
\end{theorem}


% There is one difference between the condition in Theorem~\ref{thm:info_bound_technical} and our desired lower bound.
% We are under the assumption that the convex functions are $G$-Lipschitz, i.e.,  $ \|\nabla_x f(x;s)\|\leq G$ for all $x,s$, while the lower bound in Theorem~\ref{thm:info_bound_technical} gives a distribution $\cP$ over convex functions such that $\E_{s\sim \cP}[\|\nabla_x f(x;s) \|_2^2]\leq G^2$ which is a weaker condition. In the lower bound the functions $f(x;s)=\inpro{x}{s}$ are linear, and so $\nabla_x f(x;s)=s.$ Since the distribution $\cP$  
% To deal with this difference, we can lose a logarithmic term $\log(dn)$ to get a new information lower bound, i.e., we can truncate the Gaussian distribution used in the proof and conditional on the event that $\|\nabla f(x;s) \|_2^2\leq  O(G^2 \log (nd))$ for each $s$ in the dataset, which occurs with probability at least $1-\frac{1}{\mathrm{poly}(nd)}$ and the truncation does not influence the remaining proofs.
% \gnote{Can we remove the above paragraph?}
% we can assume the algorithm can recover the underlying $v$ exactly which leads to a zero error.
%  Note that the probability of a bad event that at least one $s_i\in\cD$ sa can be made polynomially small, and 
% Thus it serves a new bound for our case.

%At the beginning, let me introduce the lower bound in \cite{DJWW15} at first.Their main statement is about getting two values each query, and can be extended to query $d$ values.


%Without loss of generality, we consider the case when $d\leq n$ first.
% \begin{theorem}
% \label{thm:info_bound}
% For any (randomized) algorithm $\cA$ which can observe a sequence of $k$ data points $(Y^1,\cdots,Y^k)$ with $Y^t=f(X^t;S^t)$ where $S^t\in\cD=\{s_1,s_2,\dots,s_n\}$ and $X^t\in\cK$ are chosen arbitrarily and adaptively by $\cA$,
% there exists a distribution $\cP$ over convex functions such that $\E_{s\sim \mathcal{P}}[\|\nabla f(x,s)\|_2^2]\leq G^2$ for all $x\in \cK$, such that the output $\hx$ of the algorithm satisfies
% \begin{align*}
%     \E_{s\sim\cP}\Big[ \E_{\cD\sim\cP^n,\cA}f(\hx;s)]\Big]-\min_{x\in\cK}\E_{s\sim\cP}[f(x;s)]\geq \frac{1}{40}\frac{GD}{\sqrt{k}}\min\{\sqrt{d},\sqrt{k}\}.
% \end{align*}
% \end{theorem}

\subsection{Proof of Theorem \ref{thm:info_bound}}
We reduce the optimization problem into a series of binary hypothesis tests.
Recall we are considering linear functions $f(x;s)\defeq\langle x,s\rangle$.
Let $\cV=\{-1,1\}^d$ be a Boolean hyper-cube and for each $v\in \cV$, let $\cN_{v}=\cN(\delta v,\sigma^2I_{d})$ be a Gaussian distribution for some parameters to be chosen such that $\HF_v(x)\defeq \E_{s\sim\cN_v}[f(x;s)]=\delta\langle x,v\rangle$. Note that $$\E_{s\sim \cN_v}[\|\nabla f(x,s)\|_2^2]=\E_{s\sim \cN_v}[\norm{s}_2^2]=(\delta^2+\sigma^2)d.$$ Therefore $G=\sqrt{d(\delta^2+\sigma^2)}.$
%Then by taking $nd$ queries of function like $f(e_i;s_j)$, we can determine the exact values of the $n$ samples $\{s_1,\cdots,s_n\}$.

Clearly the lower bound should scale linearly with $D$. Therefore without loss of generality, we can assume that the diameter $D=2$ and define $\cK=\{x\in\R^d:\|x\|_2\leq 1\}$ to be the unit ball.
As in \cite{ACCD12}, we suppose that $v$ is uniformly sampled from $\cV=\{-1,1\}^d$.
Note that if we can find a good solution to $\HF_v(x)$, we need to determine the signs of vector $v$ well. Particularly, we have the following claim:
\begin{claim}[\cite{DJWW15}]
\label{clm:error_determine_sign}
For each $v\in \cV$, let $x^v$ minimize $\HF_v$ over $\cK$ and obviously we know that $x^v=-v/\sqrt{d}$. 
For any solution $\hat{x}\in \R^d$, we have 
\begin{align*}
    \HF_v(\hat{x})-\HF_v(x^v)\geq\frac{\delta}{2\sqrt{d}}
    \sum_{j=1}^{d}\indicator\{\sign(\hat{x}_j)\neq \sign(x^v_j) 
    \},
\end{align*}
where the function $\sign(\cdot)$ is defined as:
\begin{align*}
    \sign(\hx_j)=\left\{\begin{array}{cc}
       + &  \text{ if } \hx_j>0 \\
         0 & \text{ if } \hx_j=0\\
         - & \text{ otherwise}
         \end{array}
         \right.
\end{align*}
\end{claim}

Claim~\ref{clm:error_determine_sign} provides a method to lower bound the minimax error.
Specifically, 
we define the hamming distance between any two vectors $x,y\in\R^d$ as $d_H(x,y)=\sum_{j=1}\indicator\{\sign(x_j)\neq \sign(y_j)\}$, and
we have
\begin{align}
\label{eq:minimax_to_testing}
    \eps_k^*(\cP_G,\cK)\geq \frac{\delta}{2\sqrt{d}}\{\inf_{\hat{v}}\E[d_H(\hv,v)]\},
\end{align}
where $\hv$ denotes the output of any algorithm mapping from the observation $(Y^1,\cdots,Y^k)$ to $\{-1,1\}^d$, and the probability is taken over the distribution of the underlying $v$, the observation $(Y^1,\cdots,Y^k) $ and any additional randomness in the algorithm.

% Let $S=\{j:v_j=1\}$, and we define the testing error of a solution $\hat{S}$ for estimating $S$ as
% \begin{align*}
%     \E[\hat{S}\Delta S]\defeq\sum_{j=1}^{d}\Pr[\hat{S}_j\neq S_j].
% \end{align*}

By Equation~\eqref{eq:minimax_to_testing}, it suffices to lower bound the value of the testing error $\E[d_H(\hv,v)]$. 
As discussed in \cite{ACCD12,DJWW15}, the randomness in the algorithm can not help, and we can assume the algorithm is deterministic, i.e. $(X^t,S^t)$ is a deterministic function of $Y^{[t-1]}$.\footnote{We use $Y^{[t]}$ to denote the first $t$ observations, i.e. $(Y^1,\cdots,Y^t)$}
The argument is basically based on the easy direction of Yao's principle.
% Indeed, recall $\cV=\{-1,+1\}^d$ is a finite set indexing a subset $\{\HF_v,\cN_v\}$.
% Then
% \begin{align*}
%     \sup_{\mathcal{P}\in \cP_G}\E[\eps_k(\cA,\cP,\cK)]\geq \frac{1}{|\cV|}\sum_{v\in \cV}\E_{\cN_v}[\HF_v(\hx)-\inf_{x\in\cK}\HF_v(x)]].
% \end{align*}
% At each iteration of any algorithm $\cA$, we can write $(x^t,S^t)=H_t(Y^{[t-1]},U^t)$ where we use $U^t$ is a random variable independent of $Y^{[t-1]}$ and denotes the randomness in $\cA$, and $H_t$ is a deterministic function.
% By the properties of expectations, we have
% \begin{align*}
%     &~\frac{1}{|\cV|}\sum_{v\in\cV}\E_{\cN_v}[\HF(\hx)-\inf_{x\in\cK}\HF_v(x)]\\
%     =&~\frac{1}{|\cV|}\sum_{v\in\cV}\E[\E_{\cN_v}[\HF(\hx)-\inf_{x\in\cK}\HF(x)\mid U^{[k]}]]\\
%     \ge & \inf_{u^{[k]}}\frac{1}{|\cV|}\sum_{v\in\cV}\E[\E_{\cN_v}[\HF(\hx)-\inf_{x\in\cK}\HF(x)\mid U^{[k]}=u^{[k]}]],
% \end{align*}
% which implies we can incorporate the best randomness into the algorithm $\cA$ and assume it is deterministic.

Now we continue our proof of the lower bound. We will make use of the property of the Bayes risk.

\begin{comment}
\begin{definition}[Bayes Risk, \cite{LR05}]
Suppose parameter $\theta$ is assumed
known only that it lies in a certain set $\Theta$ and consider a set of probability distributions $P=\{P_\theta:\theta\in\Theta\}$.
Given $\theta$, we can get observations $Y$ and can output an estimator $\hat{\theta}(Y)$ %\Yintat{what is X?} 
by a testing function $\hat{\theta}$.
We use a loss function $l(\theta,\hat{\theta})$ to quantifies the quality of the estimator.
For a prior distribution $\pi$ on $\Theta$, the average risk is defined as
\begin{align*}
    B_{\pi}(\hat{\theta})=\E_{\theta\sim\pi,Y\sim P_\theta}l(\theta,\hat{\theta}).
\end{align*}
The Bayes risk is the minimum that the average risk can achieve, i.e.
\begin{align*}
    B_{\pi}^*=\inf_{\hat{\theta}}B_{\pi}(\hat{\theta}).
\end{align*}
\end{definition}

\begin{lemma}[{\cite[Lemma 1]{ACCD12}}]
\label{lm:Bayes_risk}
Consider the problem of testing hypothesis $H_{-1}:v \sim \mathbb{P}_{-1}$ and $H_1:v\sim \mathbb{P}_1$, where $H_{-1}$ and $H_1$ are occurred with prior probability $\pi_{-1}$ and $\pi_1$ respectively prior to the experiment. Under the 0-1 loss, the Bayes risk $B$ satisfies
\begin{align*}
    B\geq \min(\pi_{-1},\pi_1)(1-\|\mathbb{P}_1-\mathbb{P}_{-1}\|_{\mathrm{TV}}).
\end{align*}
\end{lemma}
\end{comment}

\begin{lemma}[{\cite[Lemma 1]{ACCD12}}]
\label{lm:Bayes_risk}
Consider the problem of testing hypothesis $H_{-1}:v \sim \mathbb{P}_{-1}$ and $H_1:v\sim \mathbb{P}_1$, where $H_{-1}$ and $H_1$ occur with prior probability $\pi_{-1}$ and $\pi_1\defeq1-\pi_{-1}$ respectively prior to the experiment. 
For any algorithm that takes one sample $v$ and outputs $\hat{i}:v\rightarrow\{-1,1\}$, we define the Bayes risk $B$ be the minimum average probability that algorithm fails ($v$ is not sampled from $H_{\hat{i}(v)}$).
That is $B=\inf_{\hat{i}}\pi_{-1}\Pr[\hat{i}(v)=1\mid v\sim \mathbb{P}_{-1}]+\pi_1\Pr[\hat{i}(v)=0\mid v\sim \mathbb{P}_1]$.
Then, we have
\begin{align*}
    B\geq \min(\pi_{-1},\pi_1)(1-\|\mathbb{P}_1-\mathbb{P}_{-1}\|_{\mathrm{TV}}).
\end{align*}
\end{lemma}

\begin{lemma}
\label{lm:error_binary_test}
Suppose that $v$ is uniformly sampled from $\cV=\{-1,1\}^d$, then any estimate $\hat{v}$
%outputted by algorithm which satisfies the Orthogonal Query assumption
obeys
\begin{align*}
    \E[d_H(\hv,v)]\geq
    \frac{d}{2}\lp 1-\frac{\delta\sqrt{k}}{\sigma\sqrt{d}}\rp.
\end{align*}
\end{lemma}

\begin{proof}
Let $\pi_{-1}=\pi_1=1/2$.
For each $j$, define $\mathbb{P}_{-1,j}=\bbP(Y^{[k]}\mid v_j=-1)$ and $\bbP_{1,j}=\bbP(Y^{[k]}\mid v_j=1)$ to be distributions over the observations $(Y^1,\cdots,Y^k)$ conditional on $v_j\neq 1$ and $v_j=1$ respectively.
Let $B_j$ be the Bayes risk of the decision problem for $j$-th coordinate of $v$ between $H_{-1,j}: v_j=-1$ and $H_{1,j}: v_j=1$.
We have that
\begin{align*}
    \E[d_H(\hv,v)]
    \geq& \sum_{j=1}^{d}B_j\\
    \geq& \pi_1\sum_{j=1}^{d}(1-\|\bbP_{1,j}-\bbP_{-1,j}\|_{\mathrm{TV}})\\
    \geq& \frac{d}{2}\lp 1-\frac{1}{\sqrt{d}}\sqrt{\sum_{j=1}^{d}\|\bbP_{1,j}-\bbP_{-1,j}\|^2_{\mathrm{TV}}}\rp,
\end{align*}
where the first inequality follows from the definition of Bayes risk $B_j$, the second inequality follows by Lemma~\ref{lm:Bayes_risk} and the last inequality follows by the Cauchy-Schwartz inequality.

To complete the proof, it suffices to show that
\begin{align}
\label{eq:bounded_TVD}
    \sum_{j=1}^{d}\|\bbP_{1,j}-\bbP_{-1,j}\|_{\mathrm{TV}}^2\leq \frac{\delta^2}{\sigma^2}k.
\end{align}

Assuming Equation~\eqref{eq:bounded_TVD} first, which will be established later.
Then we know that
\begin{align*}
    \E[d_H(\hv,v)]\geq \frac{d}{2}(1-\frac{\delta\sqrt{k}}{\sigma\sqrt{d}}).
\end{align*}
\end{proof}

We will complete the proof of Lemma~\ref{lm:error_binary_test} by showing the following bounded total variation distance.
\begin{claim}
\begin{align*}
    \sum_{j=1}^{d}\|\bbP_{1,j}-\bbP_{-1,j}\|_{\mathrm{TV}}^2\leq \frac{\delta^2}{\sigma^2}k.
\end{align*}
\end{claim}

\begin{proof}
Applying Pinsker's inequality, we know $\|\bbP_{1,j}-\bbP_{-1,j}\|_{\mathrm{TV}}^2\leq \frac{1}{2}\mathrm{D}_{KL}(\bbP_{-1,j}\|\bbP_{1,j})$.
To bound the KL divergence between $\bbP_{-1,j}$ and $\bbP_{1,j}$ over all possible $Y^{[k]}$, consider $v'=(v_1,\cdots,v_{j-1},v_{j+1},\cdots,v_d)$, and define $\bbP_{-1,j,v'}(Y^{[k]})\defeq \bbP(Y^{[k]}\mid v_j=-1,v')$ to be the distribution conditional on $v_j=-1$ and $v'$.
%\gnote{Use $\bbP_{-1,j,v'}$ notation, since it also depends on $j.$}
We have
\begin{align*}
    \bbP_{-1,j}(Y^{[k]})= \sum_{v'}\Pr[v']\bbP_{-1,j,v'}(Y^{[k]}).
\end{align*}

The convexity of the KL divergence suggests that
\begin{align*}
    \mathrm{D}_{KL}(\bbP_{-1,j}\|\bbP_{1,j})\leq \sum_{v'}\Pr[v']\mathrm{D}_{KL}(\bbP_{-1,j,v'}\|\bbP_{1,j,v'}).
\end{align*}
Fixing any possible $v'$, we want to bound the KL divergence $\mathrm{D}_{KL}(\bbP_{-1,j,v'}\|\bbP_{1,j,v'})$.

Recall we are considering deterministic algorithms and
$(X^t,S^t)$ is a deterministic function of $Y^{[t-1]}$.
Let $Q_i\in \R^{d\times k}$ be a (random) matrix, which records the set of points the algorithm queries for the user $s_i$.
Specifically, for $t$-th step, if the algorithm queries $(X^t,S^t)$, then $Q_{i}^{t}=X^t$ if $S^t=s_i$, otherwise $Q_{i}^{t}=0$, where $Q_i^t$ is the $t$-th column of $Q_i$.

As we are considering linear functions, without loss of generality we can assume  $\langle Q_{i}^j,Q_{i}^{j'}\rangle=0$ for each $i$ and any $j\neq j'$, and $\|Q_{i}^t\|_2\in\{0,1\}$ for any $i$ and $t$.
We name this assumption \textsc{Orthogonal Query}.
Roughly speaking, for any algorithm, we can modify it to satisfy the Orthogonal Query.
Whenever the algorithm wants to query some point, we can use Gram–Schmidt process to query another point and satisfy Orthogonal Query, and recover the function value at the original point queried by the algorithm.
% To establish this, observe that for any algorithm $\cA$, we can find another algorithm $\cA'$ which satisfies the Orthogonal Query assumption, and the distributions of the outputs of $\cA$ and $\cA'$ are the same almost everywhere.
% Give an example, suppose $j$ is the minimum integer that $\cA$ has re-queried information of some user, say $s_i$.
% That is $s^{t_1}\neq s^{t_2}$ for $t_1,t_2\leq j-1$ and $S^j=S^t=s_i$ for some $t\leq j-1$.
% Then we can construct $\cA'$ by taking $\cA$ as an black box: for the first $j-1$-th step, $\cA'$ queries $(X^t/\|X^t\|_2,S^t)$ for the first $j-1$ steps and inputs the values of $f(X^t/\|X^t\|_2,S^t)\cdot \|X^t\|_2$ to $\cA$.
% For $j$th step, $\cA'$ queries $(z';S^t)$ where $z'=\frac{x^j-\frac{\langle x^j,x^i\rangle }{\|x^i\|^2_2}\cdot x^i}{\|z-\frac{\langle x^j,x^i\rangle }{\|x^i\|^2_2}\cdot x^i\|_2}$, then recovers the value of $f(x^j;s^j)$ and inputs it to $\cA$ to observe what's the next query.
% $\cA'$ can repeat this procedure, and output whatever $\cA$ outputs finally.

By the chain-rule of KL-divergence, if we define $P_{-1,j,v'}(Y^t\mid Y^{[t-1]})$ to be the distribution of $t$th observation $Y^t$ conditional on $v'$, $v_j=-1$ and $Y^{[t-1]}$, then we have
\begin{align*}
    \mathrm{D}_{KL}(\bbP_{-1,j,v'}\|\bbP_{1,j,v'})=\sum_{t=1}^{k}\int_{\cY^{t-1}} \mathrm{D}_{KL}(P_{-1,j,v'}(Y^t\mid Y^{[t-1]}=y)\|P_{1,j,v'}(Y^t\mid Y^{[t-1]}=y)\d P_{-1,j,v'}(y).
\end{align*}

Fix $Y^{[t-1]}$ such that $Y^{[t-1]}=y$.
Since the algorithm is deterministic and $(X^t,S^t)$ is fixed given $Y^{[t-1]}$.
Let $S^t=s_i$ so $X^t=Q_i^t$.
%Denote the choice of the algorithm is $(Q_{i}^t,s_i)$ for $t$-th step conditional on $Y^{[t-1]}=y$.

Note that the $n$ users in $\cD$ are i.i.d. sampled.
Then $\mathrm{D}_{KL}(P_{-1,j,v'}(Y^t\mid Y^{[t-1]}=y)\|P_{1,j,v'}(Y^t\mid Y^{[t-1]}=y)$ only depends on the randomness of $s_i$ and the first $t$ columns of $Q_{i}$, which is denoted by $Q_{i}^{[t]}$.
We use $Y^{t}_{j}$ to denote the observation corresponding to user $s_j$ for the $t$th query (if $S^t\neq s_j$, we have $Y^{t}_j=0$).
Note that the observation $Y^{[t]}_i=Q_i^{[t]\top} s_i$ where $s_i\sim \cN(\delta v,\sigma^2 I_d)$. Then we know $Y^{[t]}_i$ is normally distributed with mean $\delta Q_{i}^{[t]\top} v$ and co-variance $\sigma^2 Q_{i}^{[t]\top} Q_{i}^{[t]}$.

Recall that the KL divergence between two normal distributions is $\mathrm{D}_{KL}(\cN(\mu_1,\Sigma)\|\cN(\mu_2,\Sigma))=\frac{1}{2}(\mu_1-\mu_2)^{\top}\Sigma^{-1}(\mu_1-\mu_2)$. Recall that we have the Orthogonal Query assumption and thus $Q_{i}^{[t]\top} Q_{i}^{[t]}\in\{0,1\}^{t\times t}$ is a diagonal matrix.
By the conditional distributions of Gaussian, we know $Y^t_i$ only depends on the $Q_{i}^t$ and it is independent of $Q_{i}^{[t-1]}$.

Hence we have
\begin{align*}
    &\mathrm{D}_{KL}(P_{-1,j,v'}(Y^t\mid Y^{[t-1]}=y)\|P_{1,j,v'}(Y^t\mid Y^{[t-1]}=y))\\
    =&\mathrm{D}_{KL}(P_{-1,j,v'}(Y^t_i\mid Y^{[t-1]}=y)\|P_{1,j,v'}(Y^t_i\mid Y^{[t-1]}=y))\\
    =&\frac{1}{2} (2\delta Q_{i}^t(j))^2/\sigma^2,
\end{align*}
where $Q_{i}^t(j)$ is the $j$-th coordinate of $Q_{i}^t$.
Summing over the terms, one has
\begin{align*}
\sum_{j=1}^{d}\|\bbP_{1,j}-\bbP_{-1,j}\|_{\mathrm{TV}}^2\leq &
    \frac{1}{2}\mathrm{D}_{KL}(\bbP_{-1,j}\|\bbP_{1,j})\\
    \leq&\frac{1}{2} \sum_{t=1}^{k}\sum_{j=1}^{d}\sum_{i=1}^{n}\E[\frac{1}{2} (2\delta Q_{i}^t(j))^2/\sigma^2] \\
    \leq& \frac{\delta^2}{\sigma^2}k,
\end{align*}
where the last line follows from the fact that for each $t$,$ \sum_{i=1}^{n}\|Q_{i}^t\|_2^2=\sum_{i=1}^{n}\sum_{j=1}^{d}(Q_{i}^t(j))^2=1$ as we only query one user for $t$-th step.

This completes the proof.
\end{proof}

Having Lemma~\ref{lm:error_binary_test}, we can complete the proof of Theorem~\ref{thm:info_bound}.

\begin{proof}{of Theorem~\ref{thm:info_bound}.}
As discussed before, we know
\begin{align*}
    \HF_v(\hat{x})-\HF_v(x^v)\geq\frac{\delta}{2\sqrt{d}}\sum_{j=1}^{d}\indicator\{\sign(\hat{x}_j)\neq \sign(x^v_j) \},
\end{align*}
and hence we know that
\begin{align*}
     \eps_k^*(\cP_G,\cK)\geq & \frac{\delta}{2\sqrt{d}}\inf_{\hat{v}}\E[d_H(\hat{v},v)]\\
    \geq & \frac{\delta \sqrt{d}}{4}\lp 1-\frac{\delta\sqrt{k}}{\sigma\sqrt{d}}\rp,
\end{align*}
where the last line follows from Lemma~\ref{lm:error_binary_test}.
%As we assume $\cK$ is the unit ball and thus $D=2$.
We now set $\delta=\frac{\sigma \sqrt{d}}{2\sqrt{k}}$ and $\sigma=\frac{G}{\sqrt{d+d^2/4k}}$, so that $d(\sigma^2+\delta^2)=G^2$.
Hence one has
\begin{align*}
 \eps_k^*(\cP_G,\cK)
\geq \frac{\delta\sqrt{d}}{8} 
= \frac{D\delta\sqrt{d}}{16}
= \frac{GD}{16\sqrt{1+\frac{4k}{d}}}
\ge \frac{GD}{16}\min\left\{1,\sqrt{\frac{d}{4k}}\right\}.
%\ge GD\frac{\min\{\sqrt{d},\sqrt{k}\}}{40\sqrt{k}}.
\end{align*}
%\gnote{Check the last line or modify the lower bound to $ \frac{GD}{16}\min\{1,\sqrt{\frac{d}{4k}}\}$ everywhere.}
Thus we complete the proof.
\end{proof}

\begin{corollary}[Lower bound for DP-SCO]
\label{cor:DPSCOlower}
For any (non-private) algorithm which makes less than $O\lp\min\{\frac{\eps^2n^2}{\log(1/\delta)},nd\}\rp$ function value queries, there exist a convex domain $\cK\subset \R^d$ of diameter $D$, a distribution $\cP$ supported on $G$-Lipschitz linear functions $f(x;s)\defeq\langle x,s\rangle$, such that the output $\hx$ of the algorithm satisfies that
\begin{align*}
    \E_{s\sim\cP}[\langle \hx,s\rangle]-\min_{x\in\cK}\E_{s\sim\cP}[\langle x,s\rangle]\geq \Omega\lp \frac{G D}{\sqrt{1+\log(n)/d}} \cdot \min\lc\frac{\sqrt{\log(1/\delta)d}}{\eps n}+\frac{1}{\sqrt{n}},1\rc \rp.
\end{align*}
\end{corollary}
\begin{proof}
% WLOG, we can assume that $\frac{\sqrt{\log(1/\delta)d}}{\eps n}+\frac{1}{\sqrt{n}}=O(1)$. This is a reasonable assumption as if $\frac{\sqrt{\log(1/\delta)d}}{\eps n}+\frac{1}{\sqrt{n}}=\Omega(1)$, uniformly randomly output a point in $\cK$ is a good DP solution.

Note that Theorem~\ref{thm:info_bound} almost gives us what we want, except that the Lipschitz constant of the functions in the hard distribution is bounded only on average by $G$. To get distributions over $G$-Lipschitz functions, we just condition on the bad event not happening.

Recall that we are considering the set of distributions $\cN_v=\cN(\delta v,\sigma^2 I_d)$ for which $\E_{s\sim\cN_v}\|s\|_2^2\le G^2=d(\delta^2+\sigma^2)$.
And we proved that 
$\inf_{\cA\in \mathbb{A}_k}\sup_{v\in \cV}\E_{s\sim \cN_v,\cA}[\HF_v(\hat{x}_k)-\HF_v^*]\ge \frac{GD}{16}\min\left\{1,\sqrt{\frac{d}{4k}}\right\}$ in Theorem~\ref{thm:info_bound}, where $\hat{x}_k$ is the output of $\cA$ with $k$ observations $Y^{[k]}$.
%\gnote{What is $f,\cF_G,\hat{x}_k$ in the above equation?}
To prove Corollary~\ref{cor:DPSCOlower}, we need to modify the distribution of $s$ to satisfy the Lipschitz continuity.

In particularly, for some constant $c$, we know 
\begin{align*}
    &\E[\HF_v(\hat{x}_k)-\HF_v^*]\\
    =&\E\Big[ \HF_v(\hat{x}_k)-\HF_v^*\mid \max_{s_i\in \cD}\|s_i\|_2\leq cG\sqrt{1+\log(nd)/d}\Big]\Pr\Big[\max_{s_i\in \cD}\|s_i\|_2\leq cG \sqrt{1+\log(nd)/d}\Big]+\\
    &~~\E\Big[ \HF_v(\hat{x}_k)-\HF_v^*\mid \max_{s_i\in \cD}\|s_i\|_2> cG\sqrt{1+\log(nd)/d}\Big]\Pr\Big[\max_{s_i\in \cD}\|s_i\|_2> cG\sqrt{1+\log(nd)/d}\Big].
\end{align*}
By the concentration of spherical Gaussians, we know if $s\sim\cN(\delta v,\sigma^2 I_d)$, then 
\begin{align*}
    \Pr\Big[\|s-\delta v\|_2^2\leq \sigma^2 d(1+2\sqrt{\ln(1/\eta)/d}+2\ln(1/\eta)/d)\Big]\geq 1-\eta.
\end{align*}
% \Gopi{Do we really need to lose $\log(nd)$ factor here? looks like a constant is enough...}
% \Daogao{We want $\eta\leq 1/\poly(nd)$. I use log in case that $n\gg d$. maybe I just modify it.}
We can choose the constant $c$ large enough, such that $\Pr[\max_{s_i\in \cD}\|s_i\|_2\leq cG\sqrt{1+\log(nd)/d}]\geq 1-1/\poly(nd)$, which implies
\begin{align*}
    \inf_{\cA\in \mathbb{A}_k}\sup_{v\in \cV}\E_{\cD \sim \cN_v^n,\cA}\Big[\HF_v(\hat{x}_k)-\HF_v^*\mid \max_{s_i\in \cD}\|s_i\|_2\leq cG\sqrt{1+\log(nd)/d}\Big]\geq \Omega(GD\frac{\min\{\sqrt{d},\sqrt{k}\}}{\sqrt{k}}).
\end{align*}
If we use the distributions conditioned on $\max_{s_i\in \cD}\|s_i\|_2\leq cG\sqrt{1+\log(nd)/d}$ rather than the Gaussians, and scale the constant to satisfy the assumption on Lipschitz continuity, we can prove the statement.
Particularly, let $G'=cG(\sqrt{1+\log(nd)/d})$. 
If the algorithm can only make $k=O\lp\min\{\frac{\eps^2n^2}{\log(1/\delta)},nd\}\rp$ observations, we know 
\begin{align*}
    &\inf_{\cA\in \mathbb{A}_k}\sup_{v\in \cV}\E_{\cD \sim \cN_v^n,\cA}\Big[\HF_v(\hat{x}_k)-\HF_v^*\mid \max_{s_i\in \cD}\|s_i\|_2\leq G'\Big]\\
    \geq&
    \Omega\lp GD\cdot \min\lc(\frac{\sqrt{\log(1/\delta)d}}{\eps n}+\frac{1}{\sqrt{n}}),1\rc \rp  \\
    =&\Omega\lp \frac{G' D}{\sqrt{1+\log(nd)/d}}\cdot \min\lc\frac{\sqrt{\log(1/\delta)d}}{\eps n}+\frac{1}{\sqrt{n}},1\rc  \rp,
\end{align*}
% \Gopi{There is a gap in the proof. We don't get the required result exactly.}
which proves the lower bound claimed in the Corollary statement.
\end{proof}

\begin{corollary}[Lower bound for sampling scheme]
\label{cor:Samplinglower}
Given any $G > 0$ and $\mu > 0$. For any algorithm which takes function values queries less than $O\lp\frac{G^2}{\mu}/(1+\log(G^2/\mu)/d)\rp$ times, there is a family of $G$-Lipschitz linear functions $\{f_i(x)\}_{i\in I}$ defined on some $\ell_2$ ball $\cK\subset\R^d$, such that the total variation distance between the distribution of the output of the algorithm and the distribution proportional to $\exp(-\E_{i\in I}f_i(x)-\mu \|x\|^2 / 2)$ is at least $\min(1/2, \sqrt{d \mu / G^2})$.
\end{corollary}
\begin{proof}
By a similar argument in the proof of Corollary~\ref{cor:DPSCOlower}, for any algorithm which can only make $k$ observations, there are a family of $G$-Lipschitz linear functions restricted on an $\ell_2$ ball $\cK$ of diameter $D$ centered at $\mathbf{0}$ such that  
\begin{align}
\label{eq:lower_sampling_SCO}
    \E\Big[\HF_v(\hat{x}_k)-\HF_v^*\Big]
    \ge&\Omega\lp \frac{G D}{\sqrt{1+\log(k)/d}} \cdot \min\lc \sqrt{\frac{d}{k}},1\rc \rp,
\end{align}
where $\HF_v^*=\min_{x\in\cK}\HF_v(x)$ and $\hx_k\in\cK$ is the output of $\cA$.

%Without loss of generality, we can shift the functions and assume the zero point $\mathbf{0}$ is the center of $\cK$.
Suppose we have a sampling algorithm that takes $k$ queries. We use it to sample from $x^{(sol)}$ proportional to $p(x):=\exp(-\HF_v(x)-\frac{\mu}{2} \|x\|^2)$ on $\cK$ with total variation distance $\eta\leq \min(1/2, \sqrt{d \mu / G^2})$. 
% Note that $\HF_v(x)$ is some $G$-Lipschitz linear function, and by Lemma~\ref{lm:utility_tech}, we know $\E[\|x^{(sol)}\|]\leq O(\sqrt{d/\mu})$.
% To ensure $x^{(sol)}$ is bounded, we can define the bounded variant $\overline{x^{(sol)}} = x^{(sol)}$ if $\|x^{(sol)}\|^2 \leq O(d \log(1/\eta) / \mu)$ and $\mathbf{0}$ otherwise. Note that $p$ is $\mu$-strongly convex, we have $\|x\|^2 \leq O(d \log(1/\eta)/ \mu)$ with probability $1-\eta$ by Lemma~\ref{lem:gaussian_concentration} and hence $\overline{x^{(sol)}}$ has total variation distance $2 \eta$ from $p$.

Lemma \ref{lm:utility_tech} shows that
\begin{align*}
    \E[\HF_{v}(x^{(sol)})+\frac{\mu}{2}\|x^{(sol)}\|^{2}]\leq\min_{x\in \cK}\lp\HF_{v}(x)+\frac{\mu}{2}\|x\|^{2}\rp+O(d) + O(\eta) \cdot (GD+\mu D^2),
\end{align*}
where the last term involving $\eta$ is due to the total variation distance between $x^{(sol)}$ and $p$. Setting $D=\sqrt{d/\mu}$ and using the diameter of $\cK$ is $D$ and $\eta \leq \min(1/2, \sqrt{d \mu / G^2})$, we have
\begin{align*}
\E[\HF_{v}(x^{(sol)})] & \leq\min_{x\in\cK}\HF_{v}(x)+\frac{\mu}{2}D^{2}+O(d+\eta\cdot (GD+\mu D^2))\\
& \leq \min_{x\in\cK}\HF_{v}(x)+O(d).
\end{align*}
Note that we set $D = \sqrt{d/\mu}$. Comparing with \eqref{eq:lower_sampling_SCO}, we have
\[
\frac{G\sqrt{d/\mu}}{\sqrt{1+\log(k)/d}}\min\left\{ \sqrt{\frac{d}{k}},1\right\} \leq O(d).
\]

If $d\leq G^{2}/\mu\leq \exp(d)$, we have
\[
G\sqrt{d/\mu}\sqrt{\frac{d}{k}}\leq O(d)
\]
and hence $k=\Omega(G^{2}/\mu)$.
If $G^{2}/\mu\geq\exp(d)$,
we have
\[
\frac{G\sqrt{d/\mu}}{\sqrt{\log(k)/d}}\sqrt{\frac{d}{k}}\leq O(d)
\]
and hence $k=\Omega(\frac{G^{2}d/\mu}{\log(G^{2}/\mu)})$.
If $G^{2}/\mu\leq d$, we can construct our function only on the first
$O(G^{2}/\mu)$ dimensions to get a lower bound $k=\Omega(G^{2}/\mu).$
Combining all cases gives the result.
\end{proof}


% \begin{corollary}[Lower bound for sampling scheme]
% \label{cor:Samplinglower}
% Given any $G > 0$ and $\mu > 0$. For any algorithm which takes function values queries less than $O\lp\frac{G^2}{\mu}/(1+\log(G^2/\mu)/d)\rp$ times, there is a family of $G$-Lipschitz linear functions $\{f_i(x)\}_{i\in I}$, such that the total variation distance between the distribution of the output of the algorithm and the distribution proportional to $\exp(-\E_{i\in I}f_i(x)-\mu \|x\|^2 / 2)$ is at least $\mu/G^2$.
% \end{corollary}
% \begin{proof}
% By a similar argument in the proof of Corollary~\ref{cor:DPSCOlower}, for any algorithm which can only make $k$ observations, there are a family of $G$-Lipschitz linear functions restricted on an $\ell_2$ ball $\cK$ of diameter $D$ such that  
% \begin{align}
% \label{eq:lower_sampling_SCO}
%     \E\Big[\HF_v(\hat{x}_k)-\HF_v^*\Big]
%     \ge&\Omega\lp \frac{G D}{\sqrt{1+\log(k)/d}} \cdot \min\lc \sqrt{\frac{d}{k}},1\rc \rp,
% \end{align}
% where $\HF_v^*=\min_{x\in\cK}\HF_v(x)$ and $\hx_k\in\cK$ is the output of $\cA$.

% Suppose we have a sampling algorithm that takes $k$ queries. We use it to sample from $x^{(sol)}$ proportional to $p(x):=\exp(-\HF_v(x)-\frac{\mu}{2} \|x\|^2)$ with total variation distance $\eta$. 
% Without loss of generality, we can shift the functions and assume the zero point $\mathbf{0}\in\cK$ and minimizes $\HF_v(x)+\frac{\mu}{2} \|x\|^2$.
% Note that $\HF_v(x)$ is some $G$-Lipschitz linear function, and by Lemma~\ref{lm:utility_tech}, we know $\E[\|x^{(sol)}\|]\leq O(\sqrt{d/\mu})$.
% To ensure $x^{(sol)}$ is bounded, we can define the bounded variant $\overline{x^{(sol)}} = x^{(sol)}$ if $\|x^{(sol)}\|^2 \leq O(d \log(1/\eta) / \mu)$ and $\mathbf{0}$ otherwise. Note that $p$ is $\mu$-strongly convex, we have $\|x\|^2 \leq O(d \log(1/\eta)/ \mu)$ with probability $1-\eta$ by Lemma~\ref{lem:gaussian_concentration}
% and hence $\overline{x^{(sol)}}$ has total variation distance $2 \eta$ from $p$.

% Lemma \ref{lm:utility_tech} shows that
% \begin{align*}
%     \E[\HF_{v}(\overline{x^{(sol)}})+\frac{\mu}{2}\|\overline{x^{(sol)}}\|^{2}]\leq\min_{x\in \R^d}\lp\HF_{v}(x)+\frac{\mu}{2}\|x\|^{2}\rp+O(d) + O(\eta) \cdot (G \sqrt{d \log(1/\eta) / \mu} + d \log(1/\eta))
% \end{align*}
% where the last term involving $\eta$ is due to the total variation distance between $\overline{x^{(sol)}}$ and $p$. Optimizing on the right hand side on $\cK$ gets $\overline{x^{(sol)}_{\cK}}$ and using the diameter of $\cK$ is $D$, we have \Daogao{Seems problematic, as we do not know how to get a solution in $\cK$?}
% \begin{align*}
% \E[\HF_{v}(\overline{x^{(sol)}_{\cK}})] & \leq\min_{x\in\cK}\HF_{v}(x)+\frac{\mu}{2}D^{2}+O(d+\eta\cdot G\sqrt{d\log(1/\eta)/\mu})\\
%  & \leq\min_{x\in\cK}\HF_{v}(x)+\frac{\mu}{2}D^{2}+O(d+G\sqrt{d\eta/\mu})\\
%  & \leq\min_{x\in\cK}\HF_{v}(x)+\frac{\mu}{2}D^{2}+O(d)
% \end{align*}
% \Gopi{Isn't it enough to have $\eta G \sqrt{d\log(1/\eta)/\mu}\le d$ i.e. $\eta \le \Tilde{O}( \frac{\sqrt{\mu d}}{G}).$}
% where we used $\eta \leq \mu / G^2$ at the end. Putting $D = \sqrt{d/\mu}$ and comparing with \eqref{eq:lower_sampling_SCO}, we have
% \[
% \frac{G\sqrt{d/\mu}}{\sqrt{1+\log(k)/d}}\min\left\{ \sqrt{\frac{d}{k}},1\right\} \leq O(d).
% \]

% If $d\leq G^{2}/\mu\leq\exp(d)$, we have
% \[
% G\sqrt{d/\mu}\sqrt{\frac{d}{k}}\leq O(d)
% \]
% and hence $k=\Omega(G^{2}/\mu).$ If $G^{2}/\mu\geq\exp(d)$,
% we have
% \[
% \frac{G\sqrt{d/\mu}}{\sqrt{\log(k)/d}}\sqrt{\frac{d}{k}}\leq O(d)
% \]
% and hence $k=\Omega(\frac{G^{2}d/\mu}{\log(G^{2}/\mu)})$.
% If $G^{2}/\mu\leq d$, we can construct our function only on the first
% $O(\log(G^{2}/\mu))$ dimensions to get a lower bound $k=\Omega(G^{2}/\mu).$
% Combining all cases gives the result.
% \end{proof}




%Picking $\mu = c^2 d / D^2$ for some $c \geq 1$ to be determined, we have
%\begin{align*}
%\E\HF_{v}(x^{(sol)})	\leq \HF_v^* +O(c^{2}d+\frac{\eta}{\sqrt{c}}GD).
%\end{align*}

%we can design a regularized mechanism by sampling from $x^{(sol)}$ proportional to $\exp(-c(F(\cdot;\cD)+\psi(\cdot)))$ where  $\psi(x)=\frac{\mu}{2}\|x\|_2^2$ is a regularization term and $c=O(\mu nd/G^2)$.
%By Theorem~\ref{thm:generalization_error} and our guarantee on the total variation distance $\gamma<\eta$ between our final output and the distribution proportional to $\exp(-c(F(\cdot;\cD)+\psi(\cdot)))$, we know that for any $v\in \cV$ one has
%\begin{align*}
%    \E[\HF_v(x^{(sol)})-\min_{x\in\cK}\HF_v(x)]\leq O\lp GD(\frac{1}{\sqrt{n}}+\eta)\rp.
%\end{align*}
%Then if we choose $n=\Theta(1/\eta^2)$, we can get expected excess population loss $O(GD/\sqrt{n})$ with $O(nd/\log(nd))$ queries. If we set the constants hidden carefully, we can break Equation~\eqref{eq:lower_sampling_SCO}, which is contradiction.
%So the corollary holds and we complete the proof.

% Corollary~\ref{cor:Samplinglower} can be deduced directly by combining the Corollary~\ref{cor:DPSCOlower} and our results (Theorem~\ref{thm:generalization_error}) about generalization error of exponential mechanism.

% Particularly, 
% If we have such a scheme that works for all $G$-Lipschitz linear functions,
% we can try to sample $x^{(sol)}$ proportional to $\exp(-k(F(\cdot;\cD)+\psi(\cdot)))$ where $k=O(\frac{\mu}{G^2}\min\{\frac{\eps^2n^2}{\log(1/\delta)},nd \})$ with $O(\min\{\frac{\eps^2n^2}{\log(1/\delta)},nd \}/\log(nd))$ queries.
% By Theorem~\ref{thm:generalization_error} and our guarantee on the total variation distance $\gamma<\eta$ between our final output and the distribution proportional to $\exp(-k(F(\cdot;\cD)+\psi(\cdot)))$, we know that
% \begin{align*}
%     \E[\HF(x^{(sol)})-\min_{x\in\cK}\HF(x)]\leq O\lp GD(\frac{\sqrt{\log(1/\delta)d}}{\eps n}+\frac{1}{\sqrt{n}}+\eta)\rp.
% \end{align*}
% Then if we choose $n=\Theta(1/\eta^2)$, we can get expected excess population loss $O(GD(\frac{\sqrt{\log(1/\delta)d}}{\eps n}+\frac{1}{\sqrt{n}}))$ with $O(\min\{\frac{\eps^2n^2}{\log(1/\delta)},nd \}/\log(nd))$ queries..
% If we set the constants hidden carefully and can break corollary~\ref{cor:DPSCOlower}, which is contradiction.
% \gnote{Why are $\epsilon,\delta$ appearing here? Write the proof without using $\epsilon,\delta$} 

%Corollary~\ref{cor:DPSCOlower} can be deduced from the same proof of Theorem~\ref{thm:info_bound} and take care of the Lipschitz condition of the linear functions, and


%To address the first difference, for each step $t$, we can consider the $\cA$ which can choose $d$ points $(x_1^t,\cdots,x_d^t)$ to be the $(e_1,\cdots,e_d)$ where $e_j$ is the vector of size $d$ containing all zeros except for a 1 in the $j$-th position, and thus can recover the $S^t$.
%Thus our algorithm does not have any advantage over $\cA$.The second difference can be addressed by using an alternative argument with a truncated Gaussian.Thus we know the lower bound in \cite{DJWW15} can be extended to our case.With queries $\min\{n^2,nd\}$, the lower bound should be $\Omega(GD(\frac{1}{\sqrt{n}}+\frac{\sqrt{d}}{n}))$ by choosing $m=\min\{n,d\}$ and $k=n$.

% \input{localization}
% \input{localization_draft}
% \input{log-sobolev}


%\onecolumn


% \tableofcontents{}

% \newpage

\section*{Supplementary Material}
\addcontentsline{toc}{section}{Supplementary Material}


Throughout this discussion, 
we will make frequently use 
of the following standard results
concerning the exponential concentration 
of random variables:

\begin{lemma}[Hoeffding's inequality for independent RVs~\citep{hoeffding1994probability}] Let $Z_1, Z_2, \ldots, Z_n$ be independent bounded random variables with $Z_i \in [a,b]$ for all $i$, then 
    \begin{align*}
        \prob\left( \frac{1}{n} \sum_{i=1}^n (Z_i - \Expo{Z_i}) \ge t \right) \le \exp{\left( -\frac{2nt^2}{(b-a)^2} \right) }
    \end{align*} 
    and 
    \begin{align*}
        \prob\left( \frac{1}{n} \sum_{i=1}^n (Z_i - \Expo{Z_i}) \le -t \right) \le \exp{\left( -\frac{2nt^2}{(b-a)^2} \right) }
    \end{align*} 
    for all $t \ge 0$. 
\end{lemma}

\begin{lemma}[Hoeffding's inequality for sampling with replacement~\citep{hoeffding1994probability}] \label{lem:hoeffding_sampling} Let $\calZ = (Z_1, Z_2, \ldots, Z_N)$ be a finite population of $N$ points with $Z_i \in [a.b]$ for all $i$. Let $X_1, X_2, \ldots X_n$ be a random sample drawn without replacement from $\calZ$. Then for all $t \ge 0$, we have 
    \begin{align*}
        \prob\left( \frac{1}{n} \sum_{i=1}^n (X_i - \mu ) \ge t \right) \le \exp{\left( -\frac{2nt^2}{(b-a)^2} \right) }
    \end{align*} 
    and 
    \begin{align*}
        \prob\left( \frac{1}{n} \sum_{i=1}^n (X_i - \mu ) \le -t \right) \le \exp{\left( -\frac{2nt^2}{(b-a)^2} \right) } \,,
    \end{align*} 
    where $\mu = \frac{1}{N} \sum_{i=1}^{N} Z_i$. 
\end{lemma}

We now discuss one condition that generalizes the exponential concentration to dependent random variables.
\begin{condition}[Bounded difference inequality] \label{cond:BDC} Let $\calZ$ be some set and $\phi: \calZ^n \to \Real$. We say that $\phi$ satisfies the bounded difference assumption if 
there exists $c_1, c_2, \ldots c_n \ge 0$ s.t. for all $i$, we have 
\begin{align*}
    \sup_{Z_1,Z_2, \ldots,Z_n, Z_i^\prime \in \calZ^{n+1} } \abs{\phi (Z_1, \ldots, Z_i, \ldots, Z_n ) - \phi (Z_1, \ldots, Z_i^\prime, \ldots, Z_n ) } \le c_i \,.
\end{align*} 
\end{condition}

\begin{lemma}[McDiarmid’s inequality~\citep{mcdiarmid1989}] \label{lem:McDiarmid} Let $Z_1, Z_2, \ldots, Z_n$ be independent random variables on set $\calZ$ and $\phi : \calZ^n \to \Real$ satisfy bounded difference inequality (\codref{cond:BDC}). Then for all $t>0$, we have 
    \begin{align*}
        \prob\left( \phi(Z_1, Z_2, \ldots, Z_n) - \Expo{\phi(Z_1, Z_2, \ldots, Z_n)} \ge t \right) \le \exp{\left( -\frac{2t^2}{\sum_{i=1}^n c_i^2} \right) } 
    \end{align*} 
    and 
    \begin{align*}
        \prob\left( \phi(Z_1, Z_2, \ldots, Z_n) - \Expo{\phi(Z_1, Z_2, \ldots, Z_n)} \le -t \right) \le \exp{\left( -\frac{2t^2}{\sum_{i=1}^n c_i^2} \right) } \,.
    \end{align*} 
\end{lemma}


\section{Proofs from \secref{sec:ERM_training}}\label{app:proof_erm}

\textbf{Additional notation {} {}} Let $m_1$ be the number of mislabeled points ($\wt S_M$) and $m_2$ be the number of correctly labeled points ($\wt S_C$). Note $m_1 + m_2 = m$. 


\subsection{Proof of \thmref{thm:error_ERM}}


\begin{proof}[Proof of \lemref{lem:fit_mislabeled}] 
    The main idea of our proof is to regard 
    the clean portion of the data 
    ($S \cup \wt S_C$) as fixed.   
    Then, there exists an (unknown) classifier $f^*$ 
    that minimizes the expected risk
    calculated on the (fixed) clean data
    and (random draws of) the mislabeled data $\wt S_M$. 
    % 
    % 
    Formally, 
    \begin{align}
    f^* \defeq \argmin_{f \in \calF} \error_{\widecheck {\calD}} (f) \,, \label{eq:modified_ERM}
    \end{align}
    where $$\widecheck \calD = \frac{n}{m+n} \calS + \frac{m_2}{m+n} \wt \calS_C  + \frac{m_1}{m+n}\calDm \,.$$ 
    Note here that $\widecheck \calD$ is a combination 
    of the \emph{empirical distribution} 
    over correctly labeled data $S \cup \wt S_C$
    and the (population) distribution 
    over mislabeled data $\calDm$.
    Recall that 
    \begin{align}
    \wh f \defeq \argmin_{f \in \calF} \error_{\calS \cup \wt S} (f) \,. \label{eq:orig_ERM}
    \end{align}
    % 
    % 
    Since, $\widehat f$ minimizes 0-1 error 
    on $S \cup \wt S$, using ERM optimality on \eqref{eq:orig_ERM},  
    we have 
    \begin{align}
        \error_{\calS \cup \wt \calS}(\widehat f) \le \error_{
            \calS \cup \wt \calS}(f^*) \,.    \label{eq:step1}
    \end{align}
    Moreover, since $f^*$ is independent of $\wt S_M$, using Hoeffding's bound,
    % \footnote{For a fully rigorous argument,
    % refer to the complete proof in App.~\ref{app:proof_erm}.} 
    we have with probability at least $1-\delta$ that
    \begin{align}
      \error_{\wt \calS_M}(f^*) \le \error_{ \calDm}(f^*) +  \sqrt{\frac{\log(1/\delta)}{2 m_1}} \,. \label{eq:step2} 
    \end{align}
    %$ 
    %for some constant $c_1\le 1/2$. 
    Finally, since $f^*$ is the optimal classifier on $\widecheck \calD$, 
    we have 
    \begin{align}
        \error_{\widecheck \calD}(f^*) \le \error_{\widecheck \calD}(\widehat f) \,. \label{eq:step3}
    \end{align}
    Now to relate \eqref{eq:step1} and \eqref{eq:step3}, we multiply \eqref{eq:step2} by $\frac{m_1}{m+n}$ and add $\frac{n}{m+n} \error_{\calS} (f)  + \frac{m_2}{m+n} \error_{\wt \calS_C} (f)$ both the sides. Hence, 
    we can rewrite \eqref{eq:step2} as follows: 
    \begin{align}
        \error_{\calS \cup \wt\calS}(f^*) \le \error_{ \widecheck \calD}(f^*) +  \frac{m_1}{m+n}\sqrt{\frac{\log(1/\delta)}{2 m_1}} \,. \label{eq:step4} 
    \end{align}
    Now we combine equations \eqref{eq:step1}, \eqref{eq:step4}, and \eqref{eq:step3}, to get 
    \begin{align}
        \error_{\calS \cup \wt \calS}(\wh f) \le \error_{\widecheck \calD}(\wh f) +  \frac{m_1}{m+n}\sqrt{\frac{\log(1/\delta)}{2 m_1}} \,, 
    \end{align}
    which implies 
    \begin{align}
        \error_{ \wt \calS_M}(\wh f) \le \error_{\calDm}(\wh f) + \sqrt{\frac{\log(1/\delta)}{2 m_1}} \,. \label{eq:lemma1_final}
    \end{align}
    Since $\wt S$ is obtained by randomly labeling an unlabeled dataset, we assume $2m_1 \approx m$ \footnote{Formally, with probability at least $1-\delta$, we have  $(m - 2m_1)\le \sqrt{m\log(1/\delta)/2}$.}. Moreover, using $\error_{\calDm} = 1 - \error_{\calD}$ we obtain the desired result.   
    % Combining the above steps and using the fact 
    % that $\error_\calD = 1- \error_{\calDm} $, 
    % we obtain the desired result.
\end{proof}

\begin{proof}[Proof of \lemref{lem:mislabeled_error}]
    Recall $\error_{\wt S} (f) = \frac{m_1}{m} \error_{\wt S_M}(f) + \frac{m_2}{m} \error_{\wt S_C}(f)$. Hence, we have 
    \begin{align}
        2\error_{\wt S}(f) - \error_{\wt S_M}(f) - \error_{\wt S_C}(f) &= \left(\frac{2m_1}{m} \error_{\wt S_M}(f) - \error_{\wt S_M}(f)\right) + \left(\frac{2m_2}{m} \error_{\wt S_C}(f) - \error_{\wt S_C}(f)\right) \\ &= \left(\frac{2m_1}{m} - 1\right) \error_{\wt S_M}(f) + \left(\frac{2m_2}{m} - 1 \right)\error_{\wt S_C} (f) \,.
    \end{align} 
    Since the dataset is labeled uniformly at random, with probability at least $1-\delta$, we have  $\left(\frac{2m_1}{m} - 1\right) \le \sqrt{\frac{\log(1/\delta)}{2m}}$. Similarly, we have with probability at least $1-\delta$, $\left(\frac{2m_2}{m} - 1\right) \le \sqrt{\frac{\log(1/\delta)}{2m}}$. Using union bound, with probability at least $1-\delta$, we have
    % \begin{align}
    %     2\error_{\wt S} - \error_{\wt S_M}(f) - \error_{\wt S_C}(f) \le \sqrt{\frac{\log(2/\delta)}{2m}} \left(\error_{\wt S_M}(f) + \error_{\wt S_C}(f) \right) \le 2\sqrt{\frac{\log(2/\delta)}{2m}} \,. \label{eq:lemma2_final}
    % \end{align}
    \begin{align}
        2\error_{\wt S} - \error_{\wt S_M}(f) - \error_{\wt S_C}(f) \le \sqrt{\frac{\log(2/\delta)}{2m}} \left(\error_{\wt S_M}(f) + \error_{\wt S_C}(f) \right) \,. \label{eq:lemma2_prefinal}
    \end{align}
    With re-arranging $\error_{\wt S_M}(f) + \error_{\wt S_C}(f)$ and using the inequality $ 1- a\le \frac{1}{1+a} $, we have  
    \begin{align}
        2\error_{\wt S} - \error_{\wt S_M}(f) - \error_{\wt S_C}(f) \le 2\error_{\wt \calS} \sqrt{\frac{\log(2/\delta)}{2m}}  \,. \label{eq:lemma2_final}
    \end{align}

    % We obtain the desired result by using 
\end{proof}

\begin{proof}[Proof of \lemref{lem:clear_error}]
% Recall 0-1 error on each point  $(x,y) \in S \cup \wt S$ is given by $\I{ f(x)\ne y}$.
In the set of correctly labeled points $S \cup \wt S_C$, we have $S$ as a random subset of $S \cup \wt S_C$. Hence, using Hoeffding's inequality for sampling without replacement (\lemref{lem:hoeffding_sampling}), we have with probability at least $1-\delta$
\begin{align}
    \error_{\wt \calS_C} (\wh f)- \error_{\calS \cup \wt \calS_C}( \wh f) \le  \sqrt{\frac{\log(1/\delta)}{2m_2}} \,.
\end{align}
Re-writing $\error_{\calS \cup \wt \calS_C}( \wh f)$ as $\frac{m_2}{m_2 + n} \error_{\wt \calS_C }(\wh f) + \frac{n}{m_2 + n} \error_{\calS }(\wh f)$, we have with probability at least $1-\delta$
\begin{align}
   \left(\frac{n}{n+m_2}\right) \left(\error_{\wt \calS_C} (\wh f)- \error_{\calS}( \wh f) \right) \le  \sqrt{\frac{\log(1/\delta)}{2m_2}} \,.
\end{align}
As before, assuming $2m_2 \approx m$, we have with probability at least $1-\delta$ 
\begin{align}
    \error_{\wt \calS_C} (\wh f)- \error_{\calS}( \wh f) \le \left(1+\frac{m_2}{n}\right)  \sqrt{\frac{\log(1/\delta)}{m}} \le \left(1 + \frac{m}{2n}\right) \sqrt{\frac{\log(1/\delta)}{m}} \,. \label{eq:lemma3_final}
\end{align} 
\end{proof}

\begin{proof}[Proof of \thmref{thm:error_ERM}] 
    Having established these core intermediate results, we can now combine above three lemmas to prove the main result. 
    In particular, we bound the population error on clean data ($\error_\calD(\wh f)$) as follows:  
    \begin{enumerate}[(i)]
        \item First, use \eqref{eq:lemma1_final}, to obtain an upper bound on the population error on clean data, i.e., with probability at least $1-\delta/4$, we have
        \begin{align}
            \error_{ \calD} (\wh f) \le 1 - \error_{ \wt \calS_M}(\wh f) + \sqrt{\frac{\log(4/\delta)}{m}} \,. 
        \end{align}
        \item  Second, use \eqref{eq:lemma2_final}, to relate the error on the mislabeled fraction with error on clean portion of randomly labeled data and error on whole randomly labeled dataset, i.e., with probability at least $1-\delta/2$, we have 
        \begin{align}
            - \error_{\wt S_M}(f) \le \error_{\wt S_C}(f) - 2\error_{\wt S}  + 2\error_{\wt S} \sqrt{\frac{\log(4/\delta)}{2m}}  \,. 
        \end{align} 
        \item Finally, use \eqref{eq:lemma3_final} to relate the error on the clean portion of randomly labeled data and error on clean training data, i.e., with probability $1-\delta/4$, we have 
        \begin{align}
            \error_{\wt \calS_C} (\wh f)\le - \error_{\calS}( \wh f) + \left(1 + \frac{m}{2n} \right) \sqrt{\frac{\log(4/\delta)}{m}} \,. 
        \end{align} 
    \end{enumerate}

    Using union bound on the above three steps, we have with probability at least $1-\delta$: 
    \begin{align}
        \error_\calD (\wh f) \le \error_{\calS}(\wh f)   + 1 - 2\error_{\wt \calS}(\wh f)   + \left(\sqrt{2} \error_{\wt S} + 2 + \frac{m}{2n}\right)  \sqrt{\frac{\log(4/\delta)}{m}} \,.
    \end{align}
    % Note that $(1/\sqrt{2} + 2.5)$ is a loose constant. In experiments, we use the ratio $\frac{m}{n}$
    %  the exact error $\error_{\wt \calS}(\wh f)$ 
    % to evaluate R.H.S.    
\end{proof}

\subsection{Proof of \propref{prop:rademacher}}

\begin{proof}[Proof of \propref{prop:rademacher}]
    For a classifier $ f: \calX \to \{-1, 1\}$, we have $1 - 2\,\indict{ f(x) \ne y} = y \cdot f(x)$. Hence, by definition of $\error$, we have 
    \begin{align}
        1 -2\error_{\wt \calS}(f) = \frac{1}{m}\sum_{i=1}^m y_i \cdot f(x_i) \le \sup_{f \in \calF} \, \frac{1}{m} \sum_{i=1}^m y_i \cdot f(x_i)  \,. \label{eq:error_rademacher}
    \end{align}
    Note that for fixed inputs $(x_1, x_2, \ldots, x_m)$ in $\wt S$, $(y_1, y_2, \ldots y_m)$ are random labels. Define $\phi_1 (y_1, y_2, \ldots, y_m) \defeq \sup_{f \in \calF} \, \frac{1}{m} \sum_{i=1}^m y_i \cdot f(x_i)$. We have the following bounded difference condition on $\phi_1$. For all i, 
    \begin{align}
        \sup_{y_1, \ldots y_m, y_i^\prime \in \{-1, 1\}^{m+1} } \abs{ \phi_1 (y_1,\ldots, y_i, \ldots, y_m) - \phi_1 (y_1,\ldots, y_i^\prime, \ldots, y_m)  } \le 1/m \,. \label{cond1_rademacher}
    \end{align} 
    
    Similarly, we define $\phi_2 (x_1, x_2, \ldots, x_m) \defeq \Expt{ y_i \sim_U \{-1, 1\}  }{ \sup_{f \in \calF} \, \frac{1}{m}  \sum_{i=1}^m y_i \cdot f(x_i)}$. We have the following bounded difference condition on $\phi_2$. 
    For all i,
    \begin{align}
        \sup_{x_1, \ldots x_m, x_i^\prime \in \calX^{m+1} } \abs{ \phi_2 (x_1,\ldots, x_i, \ldots, x_m) - \phi_1 (x_1,\ldots, x_i^\prime, \ldots, x_m)  } \le 1/m \,. \label{cond2_rademacher}
    \end{align}
    Using McDiarmid’s inequality (\lemref{lem:McDiarmid}) twice 
    with Condition \eqref{cond1_rademacher} and \eqref{cond2_rademacher}, 
    with probability at least $1-\delta$, we have
    \begin{align}
        \sup_{f \in \calF} \, \frac{1}{m} \sum_{i=1}^m y_i \cdot f(x_i)  - \Expt{x,y}{\sup_{f \in \calF} \, \frac{1}{m} \sum_{i=1}^m y_i \cdot f(x_i) } \le \sqrt{\frac{2\log(2/\delta)}{m}} \,. \label{eq:final_rademacher}
    \end{align} 
    Combining \eqref{eq:error_rademacher} and \eqref{eq:final_rademacher}, we obtain the desired result. 
\end{proof}


\subsection{Proof of \thmref{thm:error_regularized_ERM}}

Proof of \thmref{thm:error_regularized_ERM} follows similar to the proof of \thmref{thm:error_ERM}. Note that the same results in \lemref{lem:fit_mislabeled}, \lemref{lem:mislabeled_error}, and \lemref{lem:clear_error} hold in the regularized ERM case. However, the arguments in the proof of \lemref{lem:fit_mislabeled} change slightly. Hence, we state the lemma for regularized ERM and prove it here for completeness. 

\begin{lemma} \label{lem:lemma1_reg}
    Assume the same setup as \thmref{thm:error_regularized_ERM}. 
    Then for any $\delta >0$, with probability at least  $1-\delta$ 
    over the random draws of mislabeled data $\wt S_M$, we have 
    \begin{align}
        \error_\calD(\widehat f)  \le 1 -\error_{\wt \calS_M}(\widehat f) + \sqrt{\frac{\log(1/\delta)}{m}}\,. 
    \end{align} 
\end{lemma}
\begin{proof}
    The main idea of the proof remains the same, i.e. regard 
    the clean portion of the data 
    ($S \cup \wt S_C$) as fixed.   
    Then, there exists a classifier $f^*$ 
    that is optimal over draws 
    of the mislabeled data $\wt S_M$. 

    
    Formally, 
    \begin{align}
    f^* \defeq \argmin_{f \in \calF} \error_{\widecheck {\calD}} (f)  + \lambda R(f) \,, \label{eq:modified_ERM_reg}
    \end{align}
    where $$\widecheck \calD = \frac{n}{m+n} \calS + \frac{m_1}{m+n} \wt \calS_C  + \frac{m_2}{m+n}\calDm \,.$$ That is, $\widecheck \calD$ a combination of 
    the \emph{empirical distribution} 
    over correctly labeled data $S \cup \wt S_C$
    % in $S\cup \wt S$ 
    and the (population) distribution 
    over mislabeled data $\calDm$.
    Recall that 
    \begin{align}
    \wh f \defeq \argmin_{f \in \calF} \error_{\calS \cup \wt S} (f) + \lambda R(f) \,. \label{eq:orig_ERM_reg}
    \end{align}
    % 
    % 
    Since, $\widehat f$ minimizes 0-1 error 
    on $S \cup \wt S$, using ERM optimality on \eqref{eq:orig_ERM},  
    we have 
    \begin{align}
        \error_{\calS \cup \wt \calS}(\widehat f) + \lambda R(\wh f) \le \error_{
            \calS \cup \wt \calS}(f^*) + \lambda R(f^*) \,.    \label{eq:step1_reg}
    \end{align}
    Moreover, since $f^*$ is independent of $\wt S_M$, using Hoeffding's bound,
    % \footnote{For a fully rigorous argument,
    % refer to the complete proof in App.~\ref{app:proof_erm}.} 
    we have with probability at least $1-\delta$ that
    \begin{align}
      \error_{\wt \calS_M}(f^*) \le \error_{ \calDm}(f^*) +  \sqrt{\frac{\log(1/\delta)}{2 m_1}} \,. \label{eq:step2_reg} 
    \end{align}
    %$ 
    %for some constant $c_1\le 1/2$. 
    Finally, since $f^*$ is the optimal classifier on $\widecheck \calD$, 
    we have 
    \begin{align}
        \error_{\widecheck \calD}(f^*) + \lambda R(f^*) \le \error_{\widecheck \calD}(\widehat f) + \lambda R(\wh f) \,. \label{eq:step3_reg}
    \end{align}
     Now to relate \eqref{eq:step1_reg} and \eqref{eq:step3_reg}, we can re-write the \eqref{eq:step2_reg} as follows: 
    \begin{align}
        \error_{\calS \cup \wt\calS}(f^*) \le \error_{ \widecheck \calD}(f^*) +  \frac{m_1}{m+n}\sqrt{\frac{\log(1/\delta)}{2 m_1}} \,. \label{eq:step4_reg} 
    \end{align}
    After adding $\lambda R(f^*)$ on both sides in \eqref{eq:step4_reg}, we combine equations \eqref{eq:step1_reg}, \eqref{eq:step4_reg}, and \eqref{eq:step3_reg}, to get 
    \begin{align}
        \error_{\calS \cup \wt \calS}(\wh f) \le \error_{\widecheck \calD}(\wh f) +  \frac{m_1}{m+n}\sqrt{\frac{\log(1/\delta)}{2 m_1}} \,, 
    \end{align}
    which implies 
    \begin{align}
        \error_{ \wt \calS_M}(\wh f) \le \error_{\calDm}(\wh f) + \sqrt{\frac{\log(1/\delta)}{2 m_1}} \,. \label{eq:lemma_reg_final}
    \end{align}
    Similar as before, since $\wt S$ is obtained by randomly labeling an unlabeled dataset, we assume 
    $2m_1 \approx m$. Moreover, using $\error_{\calDm} = 1 - \error_{\calD}$ we obtain the desired result. 
\end{proof}
% \begin{proof}[Proof of ]
    
% \end{proof}

\subsection{Proof of \thmref{thm:multiclass_ERM}}

To prove our results in the multiclass case,
we first state and prove lemmas
parallel to those
% We first state and prove lemmas 
% parallel 
% to the three lemmas 
used in the proof of balanced binary case. 
We then combine these results 
% in the three lemmas 
to obtain the result in \thmref{thm:multiclass_ERM}. 

Before stating the result, 
we define mislabeled distribution $\calDm$ for any $\calD$.
While $\calDm$ and $\calD$ share 
the same marginal distribution over inputs $\calX$,
the conditional distribution over labels $y$ 
given an input $x\sim \calD_\calX$ is changed as follows:
For any $x$, the Probability Mass Function (PMF) over $y$ is defined as:  
$p_{\calDm} (\cdot \vert x) \defeq \frac{1 - p_{\calD}(\cdot \vert x)}{k - 1}$, where $ p_{\calD}(\cdot \vert x)$ is the PMF over $y$ for the distribution $\calD$. 

\begin{lemma} \label{lem:fit_mislabeled_multi}
    Assume the same setup as \thmref{thm:multiclass_ERM}. 
    Then for any $\delta >0$, with probability at least  $1-\delta$ 
    over the random draws of mislabeled data $\wt S_M$, we have 
    \begin{align}
        \error_\calD(\widehat f)  \le (k-1)\left(1 -\error_{\wt \calS_M}(\widehat f)\right) + (k-1)\sqrt{\frac{\log(1/\delta)}{m}}\,. \label{eq:lemma1_multi}
    \end{align}   
\end{lemma} 

\begin{proof}
   
    The main idea of the proof remains the same.
    We begin by regarding the clean portion of the data 
    ($S \cup \wt S_C$) as fixed. 
    Then, there exists a classifier $f^*$ 
    that is optimal over draws 
    of the mislabeled data $\wt S_M$. 
    
    However, in the multiclass case,
    we cannot as easily relate the population error on mislabeled data 
    to the population accuracy on clean data.   
    While for binary classification, 
    % we could upper bound $\error_{\wt \calS_M}$ 
    % with $1-\error_\calD$ 
    we could lower bound the population accuracy $1-\error_\calD$
    with the empirical error on mislabeled data $\error_{\wt \calS_M}$ 
    (in the proof of \lemref{lem:fit_mislabeled}), 
    for multiclass classification, 
    error on the mislabeled data 
    and accuracy on the clean data 
    in the population 
    are not so directly related.  
    To establish \eqref{eq:lemma1_multi},
    we break the error on the 
    (unknown) mislabeled data 
    into two parts: one term corresponds 
    to predicting the true label on mislabeled data, 
    and the other corresponds to predicting 
    neither the true label 
    nor the assigned (mis-)label.  
    Finally, we relate these errors to their
    population counterparts to establish \eqref{eq:lemma1_multi}. 
    
    Formally, 
    \begin{align}
    f^* \defeq \argmin_{f \in \calF} \error_{\widecheck {\calD}} (f)  + \lambda R(f) \,, \label{eq:modified_ERM_reg2}
    \end{align}
    where $$\widecheck \calD = \frac{n}{m+n} \calS + \frac{m_1}{m+n} \wt \calS_C  + \frac{m_2}{m+n}\calDm \,.$$ 
    That is, $\widecheck \calD$ is a combination 
    of the \emph{empirical distribution} 
    over correctly labeled data $S \cup \wt S_C$
    % in $S\cup \wt S$ 
    and the (population) distribution 
    over mislabeled data $\calDm$.
    Recall that 
    \begin{align}
    \wh f \defeq \argmin_{f \in \calF} \error_{\calS \cup \wt S} (f) + \lambda R(f) \,. \label{eq:orig_ERM_reg2}
    \end{align}
    % 
    % 
    Following the exact steps from the proof of \lemref{lem:lemma1_reg}, 
    with probability at least $1-\delta$, we have  
    \begin{align}
        \error_{ \wt \calS_M}(\wh f) \le \error_{\calDm}(\wh f) + \sqrt{\frac{\log(1/\delta)}{2 m_1}} \,. \label{eq:lemma1_final_multi_prev}
    \end{align}
    Similar to before, since $\wt S$ is obtained 
    by randomly labeling an unlabeled dataset, 
    we assume 
    $\frac{k}{k-1} m_1 \approx m$. 
    
    Now we will relate $\error_{\calDm} (\wh f)$ with $\error_{\calD}(\wh f)$. 
    Let $y^T$ denote the (unknown) true label 
    for a mislabeled point $(x, y)$ 
    (i.e., label before replacing it with a mislabel). 
    \begin{align*}    
         \Expt{(x, y) \in \sim \calDm}{\indict{ \wh f(x) \ne y }}  &= \underbrace{\Expt{(x, y) \in \sim \calDm}{\indict{ \wh f(x) \ne y \land \wh f(x) \ne y^T}}}_{\RN{1}} \\ &\qquad \qquad + \underbrace{\Expt{(x, y) \in \sim \calDm}{\indict{ \wh f(x) \ne y \land \wh f(x) = y^T}}}_{\RN{2}} \,. \numberthis \label{eq:excess_term}
    \end{align*}
    Clearly, term 2 is one minus the accuracy 
    on the clean unseen data, i.e.,
    \begin{align}
        \RN{2} = 1 - \Expt{{x,y} \sim \calD}{ \indict{ \wh f(x) \ne y}} = 1- \error_{\calD}(\wh f) \,. \label{eq:term1}    
    \end{align}
    Next, we relate term 1 with the error on the unseen clean data. 
    We show that term 1 is equal to the error on the unseen clean data 
    scaled by $\frac{k-2}{k-1}$,
    where $k$ is the number of labels.
    Using the definition of mislabeled distribution $\calDm$,  
    we have 
    \begin{align}
        \RN{1} = \frac{1}{k-1} \left( \Expt{(x, y) \in \sim \calD}{ \sum_{i \in \calY \land i\ne y}  \indict{ \wh f(x) \ne i \land \wh f(x) \ne y}} \right) = \frac{k-2}{k-1} \error_{\calD}(\wh f) \,.\label{eq:term2}
    \end{align}    

    Combining the result in \eqref{eq:term1}, \eqref{eq:term2} and \eqref{eq:excess_term}, we have 
    \begin{align}
        \error_{\calDm}(\wh f) = 1- \frac{1}{k-1} \error_{\calD}(\wh f) \,.\label{eq:combine_terms}
    \end{align}
    Finally, combining the result in \eqref{eq:combine_terms} 
    with equation \eqref{eq:lemma1_final_multi_prev}, 
    we have with probability $1-\delta$, 
    \begin{align}
      \error_{\calD}(\wh f) \le  (k-1) \left( 1- \error_{ \wt \calS_M}(\wh f) \right)  + (k-1) \sqrt{\frac{k \log(1/\delta)}{ 2(k-1)m}} \,. \label{eq:lemma1_final_multi}
    \end{align}
\end{proof}

\begin{lemma} \label{lem:mislabeled_error_multi}
    Assume the same setup as \thmref{thm:multiclass_ERM}. 
    Then for any $\delta >0$, 
    with probability at least $1-\delta$ 
    over the random draws of $\wt S$, we have  
    % \begin{align}
        $$\abs{k\error_{\wt \calS}(\widehat f) - \error_{\wt \calS_C}(\widehat f) -  (k-1)\error_{\wt \calS_M}(\widehat f) } \le  2k\sqrt{\frac{\log(4/\delta)}{2m}}\,. $$ % \label{eq:lemma2}
    % \end{align}   
    %  for some constant $c_3 \le 1.0\,$.
\end{lemma} 


\begin{proof}
    Recall $\error_{\wt S} (f) = \frac{m_1}{m} \error_{\wt S_M}(f) + \frac{m_2}{m} \error_{\wt S_C}(f)$. Hence, we have 
    \begin{align*}
        k\error_{\wt S}(f) - (k-1)\error_{\wt S_M}(f) - \error_{\wt S_C}(f) &= (k-1)\left(\frac{k m_1}{(k-1) m} \error_{\wt S_M}(f) - \error_{\wt S_M}(f)\right) \\ & \qquad \qquad + \left(\frac{km_2}{m} \error_{\wt S_C}(f) - \error_{\wt S_C}(f)\right) \\ &= k \left[ \left(\frac{m_1}{m} - \frac{k-1}{k}\right) \error_{\wt S_M}(f) + \left(\frac{m_2}{m} - \frac{1}{k} \right) \error_{\wt S_C} (f) \right] \,.
    \end{align*} 
    Since the dataset is randomly labeled, 
    we have with probability at least $1-\delta$, 
    $\left(\frac{m_1}{m} - \frac{k-1}{k}\right) \le \sqrt{\frac{\log(1/\delta)}{2m}}$. 
    Similarly, we have with probability at least $1-\delta$, 
    $\left(\frac{m_2}{m} - \frac{1}{k}\right) \le \sqrt{\frac{\log(1/\delta)}{2m}}$. 
    Using union bound, we have with probability at least $1-\delta$
    % \begin{align}
    %     2\error_{\wt S} - \error_{\wt S_M}(f) - \error_{\wt S_C}(f) \le \sqrt{\frac{\log(2/\delta)}{2m}} \left(\error_{\wt S_M}(f) + \error_{\wt S_C}(f) \right) \le 2\sqrt{\frac{\log(2/\delta)}{2m}} \,. \label{eq:lemma2_final}
    % \end{align}
    \begin{align}
        k\error_{\wt S}(f) - (k-1)\error_{\wt S_M}(f) - \error_{\wt S_C}(f)  \le k \sqrt{\frac{\log(2/\delta)}{2m}} \left(\error_{\wt S_M}(f) + \error_{\wt S_C}(f) \right) \,. \label{eq:lemma2_final_multi}
    \end{align}

    % We obtain the desired result by using 
\end{proof}

\begin{lemma} \label{lem:clear_error_multi}
    Assume the same setup as \thmref{thm:multiclass_ERM}. 
    Then for any $\delta >0$, with probability at least $1-\delta$ 
    over the random draws of $\wt S_C$ and $S$, we have 
    % \begin{align}
        $$\abs{\error_{\wt \calS_C}(\widehat f) - \error_{\calS}(\widehat f) } \le 1.5 \sqrt{\frac{k\log(2/\delta)}{2m}}\,.$$ %\label{eq:lemma3}
    % \end{align}   
    % for some constant $c_2 \le 1.2\,$.
\end{lemma} 
\begin{proof}
    % Recall 0-1 error on each point  $(x,y) \in S \cup \wt S$ is given by $\I{ f(x)\ne y}$.
    In the set of correctly labeled points $S \cup \wt S_C$,
    we have $S$ as a random subset of $S \cup \wt S_C$. 
    Hence, using Hoeffding's inequality 
    for sampling without replacement 
    (\lemref{lem:hoeffding_sampling}), 
    we have with probability at least $1-\delta$
    \begin{align}
        \error_{\wt \calS_c} (\wh f)- \error_{\calS \cup \wt \calS_C}( \wh f) \le  \sqrt{\frac{\log(1/\delta)}{2m_2}} \,.
    \end{align}
    Re-writing $\error_{\calS \cup \wt \calS_C}( \wh f)$ 
    as $\frac{m_2}{m_2 + n} \error_{\wt \calS_C }(\wh f) + \frac{n}{m_2 + n} \error_{\calS }(\wh f)$, 
    we have with probability at least $1-\delta$
    \begin{align}
       \left(\frac{n}{n+m_2}\right) \left(\error_{\wt \calS_c} (\wh f)- \error_{\calS}( \wh f) \right) \le  \sqrt{\frac{\log(1/\delta)}{2m_2}} \,.
    \end{align}
    As before, assuming $km_2 \approx m$, 
    we have with probability at least $1-\delta$ 
    \begin{align}
        \error_{\wt \calS_c} (\wh f)- \error_{\calS}( \wh f) \le \left(1+\frac{m_2}{n}\right)  \sqrt{\frac{k\log(1/\delta)}{2m}} \le \left( 1 + \frac{1}{k}\right) \sqrt{\frac{k\log(1/\delta)}{2m}} \,. \label{eq:lemma3_final_multi}
    \end{align} 
\end{proof}

\begin{proof}[Proof of \thmref{thm:multiclass_ERM}] 
    Having established these core intermediate results, 
    we can now combine above three lemmas. 
    In particular, we bound the population error 
    on clean data ($\error_\calD(\wh f)$) as follows:  
    \begin{enumerate}[(i)]
        \item First, use \eqref{eq:lemma1_final_multi}, 
        to obtain an upper bound on the population error on clean data, 
        i.e., with probability at least $1-\delta/4$, we have
        \begin{align}
            \error_{ \calD} (\wh f) \le (k-1)\left(1 - \error_{ \wt \calS_M}(\wh f) \right) + (k-1) \sqrt{\frac{k\log(4/\delta)}{2(k-1)m}} \,. 
        \end{align}
        \item  Second, use \eqref{eq:lemma2_final_multi}
        to relate the error on the mislabeled fraction 
        with error on clean portion of randomly labeled data 
        and error on whole randomly labeled dataset, 
        i.e., with probability at least $1-\delta/2$, we have 
        \begin{align}
            - (k-1)\error_{\wt S_M}(f) \le \error_{\wt S_C}(f) - k\error_{\wt S}  + k\sqrt{\frac{\log(4/\delta)}{2m}}  \,. 
        \end{align} 
        \item Finally, use \eqref{eq:lemma3_final_multi} 
        to relate the error on the clean portion of randomly labeled data 
        and error on clean training data, 
        i.e., with probability $1-\delta/4$, we have 
        \begin{align}
            \error_{\wt \calS_C} (\wh f)\le - \error_{\calS}( \wh f) + \left(1 + \frac{m}{kn} \right) \sqrt{\frac{k\log(4/\delta)}{2m}} \,. 
        \end{align} 
    \end{enumerate}

    Using union bound on the above three steps, 
    we have with probability at least $1-\delta$: 
    \begin{align}
        \error_\calD (\wh f) \le \error_{\calS}(\wh f) + (k-1) - k\error_{\wt \calS}(\wh f)   + (\sqrt{k(k-1)} + k + \sqrt{k} + \frac{m}{n\sqrt{k}})  \sqrt{\frac{\log(4/\delta)}{2m}} \,.\label{eq:multiclass_ERM_final}
    \end{align}
    Simplifying the term in RHS of \eqref{eq:multiclass_ERM_final}, 
    we get the desired result. 
    % Note that since $\frac{m}{n\sqrt{k}}$ 
    % is much smaller than the sum of the other terms
    % the other terms in summation, 
    % we ignore $\frac{m}{n\sqrt{k}}$  
    % Z: ??? --- great
    % that 
    % them
    in the final bound. 
    % we ignore that in the final bound. 
    % Note that $(1/\sqrt{2} + 2.5)$ is a loose constant. In experiments, we use the ratio $\frac{m}{n}$
    %  the exact error $\error_{\wt \calS}(\wh f)$ 
    % to evaluate R.H.S.    
\end{proof}

\newpage
\section{Proofs from \secref{sec:linear_models}}\label{app:proof_gd}
We suppose that the parameters of the linear function 
are obtained via gradient descent on 
the following $L_2$ regularized problem: 
\begin{align}
    % n in denominator is avoided deliberately
    \calL_S(w; \lambda) \defeq \sum_{i=1}^n{(w^Tx_i - y_i)^2} + \lambda \norm{w}{2}^2 \,, \label{eq:l2_MSE_app}   
\end{align}
where $\lambda\ge0$ is a regularization parameter. 
We assume access to a clean dataset 
$S = \{(x_i, y_i)\}_{i=1}^n \sim \calD^n$ 
and randomly labeled dataset 
$\wt S = \{(x_i, y_i)\}_{i=n+1}^{n+m} \sim \wt \calD^m$. 
Let $\bX = [x_1, x_2, \cdots, x_{m+n}]$ 
and $\by = [y_1, y_2, \cdots, y_{m+n}]$. 
Fix a positive learning rate $\eta$ such that 
$\eta \le 1/\left(\norm{\bX^T\bX}{\text{op}} + \lambda^2\right)$ 
and an initialization $w_0 = 0$. 
% \todos{Assumption made for simplicty}. 
Consider the following gradient descent iterates 
to minimize objective \eqref{eq:l2_MSE_app} on $S \cup \wt S$:
\begin{align}
w_t = w_{t-1} - \eta \grad_w \calL_{S \cup \wt S} (w_{t-1}; \lambda) \quad \forall t=1,2,\ldots \label{eq:GD_iterates_app}
\end{align} 
Then we have $\{ w_t\}$ converge to the limiting solution 
$\wh w = \left( \bX^T\bX+\lambda \boldsymbol{I}\right)^{-1}\bX^T\by$. Define $\widehat f (x) \defeq f(x ; \wh w) $.  

% \subsection{\textcolor{red}{Errata}}

% We wish to correct the following error in the body:
% \codref{cond:error_stability} is not enough 
% to guarantee the result in \thmref{thm:linear}. 
% We now present a slightly stronger condition 
% called \emph{hypothesis stability} 
% under which we obtain a result 
% similar to \thmref{thm:linear}. 

% This error doesn't change the main arguments of the proof,
% where we show that the empirical train error 
% is less than or equal to the leave-one-out error.
% We need a stronger condition to relate leave-one-out error 
% with the population error of the original classifier. 
% Specifically, while \codref{cond:error_stability} 
% relates the average population error of leave-one-out classifiers 
% with the population error of the original classifier, 
% we need the new condition to show the concentration 
% of the empirical leave-one-out error 
% and average population error of leave-one-out classifiers. 
% main takeaway 

% Note that the new condition, 
% while being stronger than the previous one, 
% still doesn't imply generalization \citep{bousquet2002stability,elisseeff2003leave,abou2019exponential}. 
% Overall, the main results in \secref{sec:ERM_training} 
% and takeaways of the paper remain unaffected by the error.  

% We now present the new condition 
% and a corrected statement of \thmref{thm:linear}. 
% Recall, for a given training set $S \sim \calD^n $, 
% we use $S_{(i)}$ to denote the training set $S$ 
% with the $i^{\text{th}}$ point removed.

% \begin{condition}[Hypothesis Stability] 
%     \label{cond:hypothesis_stability}
%     We have $\beta$ hypothesis stability 
%     if our training algorithm $\calA$ satisfies the following: 
%     \begin{align*}
%     % ${\sum_{i=1}^n \frac{\error_{\calD}( f(\calA, S_{(i)}))}{n} - \error_\calD(f(\calA, S))} \le \beta\,$.
%     \forall i \in \{1,2,\ldots, n\}, \quad  \Expt{\calS, (x,y) \in \calD}{ \abs{\error\left( f(x) ,y  \right) - \error\left( f_{(i)}(x), y \right) }} \le \frac{\beta}{n} \,,
%     \end{align*}
%     where $f_{(i)} \defeq f(\calA, S_{(i)})$ and $ f \defeq f(\calA, S)$.
% \end{condition}

% \begin{theorem}[Correct statement of \thmref{thm:linear}] \label{thm:new_linear}
%     Assume that this gradient descent algorithm satisfies \codref{cond:hypothesis_stability}
%     with $\beta=\calO(1)$.  
%     Then for any $\delta >0$, with probability at least $1-\delta$ 
%     over the random draws of datasets $\wt S$ and $S$, we have:
%     \begin{align}
%         \error_\calD(\widehat f) \le \error_\calS(\widehat f) + 1 - 2 \error_{\wt\calS}(\widehat f) + \left(\frac{1}{\sqrt{2}} + 1.5 \right) \sqrt{\frac{\log(4/\delta)}{m}} + \sqrt{\frac{4}{\delta}\left(\frac{1}{m} +\frac{3\beta}{m+n} \right)}  \,. \label{eq:gd_error}
%     \end{align} 
%     % for some constant $c\le 3.2$.
% \end{theorem}

\subsection{Proof of \thmref{thm:linear}}
We use a standard result from linear algebra, 
namely the Shermann-Morrison formula 
\citep{sherman1950adjustment} for matrix inversion:  

\begin{lemma}[\citet{sherman1950adjustment}] \label{lem:sherman}
    Suppose $\bA \in \Real^{n \times n}$ 
    is an invertible square matrix 
    and $u,v \in \Real^n$ are column vectors. 
    Then $\bA + uv^T$ is invertible iff $1 + v^T \bA u \ne 0$ 
    and in particular
    \begin{align}
        (\bA + u v^T)^{-1} = \bA^{-1}  - \frac{\bA^{-1} uv^T \bA^{-1} }{ 1 + v^T \bA^{-1} u} \,.
    \end{align}   
\end{lemma}
\newcommand\byy[1]{\by_{\left(#1\right)}}
\newcommand\bXX[1]{\bX_{\left(#1\right)}}
\newcommand\ff[1]{\wh f_{\left(#1\right)}}

For a given training set $S \cup \wt S_C$, 
define leave-one-out error 
on mislabeled points in the training data 
as $$\error_{\text{LOO}(\wt S_M) } = \frac{\sum_{(x_i, y_i) \in \wt S_M} \error( f_{(i)}( x_i), y_i)}{ \abs{\wt S_M }} \,, $$
where $f_{(i)} \defeq f(\calA, (S \cup \wt S)_{(i)})$. 
To relate empirical leave-one-out error and population error 
with hypothesis stability condition, 
we use the following lemma:   

\begin{lemma}[\citet{bousquet2002stability}] \label{lem:stability_error}
    For the leave-one-out error, we have
    \begin{align}
        \Expo{ \left( \error_{\calDm}(\wh f) -\error_{\text{LOO}(\wt S_M) } \right)^2 } \le \frac{1}{2m_1}+  \frac{3\beta}{n + m}\,.
    \end{align}   
    % where $ f \defeq f(\calA, S \cup \wt S) $.
\end{lemma}

Proof of the above lemma is similar 
to the proof of Lemma 9 in \citet{bousquet2002stability} 
and can be found in \appref{app:proof_lem_error}. 
% 
% Before presenting the result, we introduce some notation. 
Before presenting the proof of \thmref{thm:linear}, 
we introduce some more notation. 
Let $\bX_{(i)}$ denote the matrix of covariates 
with the $i^{\text{th}}$ point removed. 
Similarly, let $\by_{(i)}$ be the array of responses 
with the $i^{\text{th}}$ point removed. 
Define the corresponding regularized GD solution 
as $\wh w_{(i)} = \left( \bXX{i}^T\bXX{i}+\lambda \boldsymbol{I}\right)^{-1}\bXX{i}^T\byy{i}$. 
Define $\ff{i}(x) \defeq f(x ; \wh w_{(i)}) $.

\begin{proof}[Proof of \thmref{thm:linear}]
    Because squared loss minimization does not imply 0-1 error minimization, 
    we cannot use arguments from \lemref{lem:fit_mislabeled}. 
    This is the main technical difficulty. 
    To compare the 0-1 error at a train point with an unseen point, 
    we use the closed-form expression for $\widehat{w}$ 
    and Shermann-Morrison formula 
    to upper bound training error 
    with leave-one-out cross validation error. 
    
    The proof is divided into three parts: 
    In part one, we show that 0-1 error 
    on mislabeled points in the training set 
    is lower than the error obtained 
    by leave-one-out error at those points. 
    In part two, we relate this leave-one-out error 
    with the population error on mislabeled distribution
    using \codref{cond:hypothesis_stability}.
    While the empirical leave-one-out error is an unbiased estimator 
    of the average population error of leave-one-out classifiers, 
    we need hypothesis stability 
    to control the variance 
    of empirical leave-one-out error. 
    Finally, in part three, we show 
    that the error on the mislabeled training points 
    can be estimated with just the randomly labeled 
    and clean training data (as in proof of \thmref{thm:error_ERM}).  

    \textbf{Part 1 {} {}} First we relate training error with leave-one-out error.        
    For any training point $(x_i, y_i)$ in $\wt S \cup S$, we have 
    \begin{align}
        \error(\wh f(x_i), y_i ) &= \indict{ y_i \cdot x_i^T \wh w < 0 } = \indict{ y_i \cdot x_i^T \left( \bX^T\bX+\lambda \boldsymbol{I}\right)^{-1}\bX^T\by < 0 } \\
        &= \indict{ y_i \cdot x_i^T \underbrace{\left( \bXX{i}^T\bXX{i} + x_i ^T x_i +\lambda \boldsymbol{I}\right)^{-1}}_{\RN{1}} (\bXX{i}^T\byy{i} + y_i \cdot x_i) < 0 } \,.
    \end{align}
    Letting $\bA = \left(\bXX{i}^T\bXX{i} +\lambda \boldsymbol{I}\right)$ 
    and using \lemref{lem:sherman} on term 1, we have 
    \begin{align}
        \error(\wh f(x_i), y_i ) &= \indict{ y_i \cdot x_i^T \left[\bA^{-1} -  \frac{\bA^{-1} x_i x_i^T \bA^{-1}}{ 1 + x_i ^T \bA^{-1} x_i } \right] (\bXX{i}^T\byy{i} + y_i \cdot x_i) < 0 } \\
        &= \indict{ y_i \cdot\left[ \frac{ x_i^T \bA^{-1} ( 1 + x_i ^T \bA^{-1} x_i ) -  x_i^T \bA^{-1} x_i x_i^T \bA^{-1}}{ 1 + x_i ^T \bA ^{-1}x_i } \right] (\bXX{i}^T\byy{i} + y_i \cdot x_i) < 0 } \\
        &= \indict{ y_i \cdot\left[ \frac{ x_i^T \bA^{-1}}{ 1 + x_i ^T \bA ^{-1}x_i } \right] (\bXX{i}^T\byy{i} + y_i \cdot x_i) < 0 } \,.
    \end{align}

    Since $1 + x_i^T \bA^{-1} x_i > 0$, we have 
    \begin{align}
        \error(\wh f(x_i), y_i ) &= \indict{ y_i \cdot x_i^T \bA^{-1} (\bXX{i}^T\byy{i} + y_i \cdot x_i) < 0 } \\
        &= \indict{ x_i^T \bA^{-1} x_i +  y_i \cdot x_i^T \bA^{-1} (\bXX{i}^T\byy{i}) < 0 } \\
        &\le \indict{ y_i \cdot x_i^T \bA^{-1} (\bXX{i}^T\byy{i}) < 0 } = \error(\ff{i}(x_i), y_i ) \,.\label{eq:LOO_error}
    \end{align}

    Using \eqref{eq:LOO_error}, we have 
    \begin{align}
        \error_{\wt \calS_M } (\wh f) \le \error_{\text{LOO} (\wt S_M)} \defeq \frac{\sum_{(x_i, y_i) \in \wt S_M} \error(\ff{i}(x_i), y_i ) }{\abs{\wt \calS_M}}\label{eq:LOO_error_final} \,.
    \end{align}
    \textbf{Part 2 {}{}} We now relate RHS in \eqref{eq:LOO_error_final} 
    with the population error on mislabeled distribution. 
    To do this, we leverage \codref{cond:hypothesis_stability} 
    and \lemref{lem:stability_error}. 
    In particular, we have 

    \begin{align}
        \Expt{\calS \cup \wt \calS_M }{ \left(\error_{\calDm}(\wh f) - \error_{\text{LOO} (\wt S_M)}\right)^2 } \le \frac{1}{2m_1} + \frac{3\beta}{m+n} \,.
    \end{align}

    Using Chebyshev's inequality, with probability at least $1-\delta$, we have 
    \begin{align}
        \error_{\text{LOO} (\wt S_M)} \le  \error_{\calDm}(\wh f)   + \sqrt{\frac{1}{\delta}\left(\frac{1}{2m_1} +\frac{3\beta}{m+n} \right)} \,. \label{eq:final_mislabeled_linear}
    \end{align}
    

    \textbf{Part 3 {}{}} Combining \eqref{eq:final_mislabeled_linear} and \eqref{eq:LOO_error_final}, we have 

    \begin{align}
        \error_{\wt \calS_M } (\wh f) \le \error_{\calDm}(\wh f)   + \sqrt{\frac{1}{\delta}\left(\frac{1}{2m_1} +\frac{3\beta}{m+n} \right)} \,. \label{eq:linear_parallel_lem1}
    \end{align}

    Compare \eqref{eq:linear_parallel_lem1} with \eqref{eq:lemma1_final} 
    in the proof of \lemref{lem:fit_mislabeled}. 
    We obtain a similar relationship 
    between $\error_{\wt \calS_M }$ and $\error_{\calDm}$ 
    but with a polynomial concentration 
    instead of exponential concentration. 
    In addition, since we just use concentration arguments 
    to relate mislabeled error to the errors
    on the clean and unlabeled portions 
    of the randomly labeled data, 
    we can directly use the results 
    in \lemref{lem:mislabeled_error} and \lemref{lem:clear_error}. 
    Therefore, combining results in \lemref{lem:mislabeled_error}, \lemref{lem:clear_error}, and \eqref{eq:linear_parallel_lem1} with union bound, 
    we have with probability at least $1-\delta$
    \begin{align}
        \error_\calD(\widehat f) \le \error_\calS(\widehat f) + 1 - 2 \error_{\wt\calS}(\widehat f) + \left(\sqrt{2}\error_{\wt\calS}(\widehat f) + 1 + \frac{m}{2n} \right) \sqrt{\frac{\log(4/\delta)}{m}} + \sqrt{\frac{4}{\delta}\left(\frac{1}{m} +\frac{3\beta}{m+n} \right)}  \,.
    \end{align}
    

       
\end{proof}

\subsection{Extension to multiclass classification} \label{app:multiclass_linear}
For multiclass problems with squared loss minimization, as standard practice, we consider one-hot encoding for the underlying label, i.e., a class label $c \in [k]$ is treated as $(0, \cdot, 0,1,0, \cdot, 0) \in \Real^k$ (with $c$-th coordinate being 1).  As before, we suppose that the parameters of the linear function 
are obtained via gradient descent on the following $L_2$ regularized problem: 
\begin{align}
    % n in denominator is avoided deliberately
    \calL_S(w; \lambda) \defeq \sum_{i=1}^n\norm{w^Tx_i - y_i}{2}^2 + \lambda \sum_{j=1}^k \norm{w_j}{2}^2 \,, \label{eq:l2_multiclass_MSE_app}   
\end{align}
where $\lambda\ge0$ is a regularization parameter. 
We assume access to a clean dataset 
$S = \{(x_i, y_i)\}_{i=1}^n \sim \calD^n$ 
and randomly labeled dataset 
$\wt S = \{(x_i, y_i)\}_{i=n+1}^{n+m} \sim \wt \calD^m$. 
Let $\bX = [x_1, x_2, \cdots, x_{m+n}]$ 
and $\by = [e_{y_1}, e_{y_2}, \cdots, e_{y_{m+n}}]$. 
Fix a positive learning rate $\eta$ such that 
$\eta \le 1/\left(\norm{\bX^T\bX}{\text{op}} + \lambda^2\right)$ 
and an initialization $w_0 = 0$. 
% \todos{Assumption made for simplicty}. 
Consider the following gradient descent iterates 
to minimize objective \eqref{eq:l2_MSE_app} on $S \cup \wt S$:
\begin{align}
{w_j}^t = {w_j}^{t-1} - \eta \grad_{w_j} \calL_{S \cup \wt S} (w^{t-1}; \lambda) \quad \forall t=1,2,\ldots \text{ and } j=1,2,\ldots,k  \,. \label{eq:GD_multi_iterates_app}
\end{align} 
Then we have $\{ {w_j}^t\}$ for all $j =1,2,\cdots, k$ converge to the limiting solution 
$\wh w_j = \left( \bX^T\bX+\lambda \boldsymbol{I}\right)^{-1}\bX^T\by_j$. Define $\widehat f (x) \defeq f(x ; \wh w) $.  

\begin{theorem}\label{thm:multi_linear}
    Assume that this gradient descent algorithm satisfies \codref{cond:hypothesis_stability}
    with $\beta=\calO(1)$.  
    Then for a multiclass classification problem wth $k$ classes, for any $\delta >0$, with probability at least $1-\delta$, we have:
    \begin{align*}
        \error_\calD(\widehat f) \le \error_\calS(\widehat f) &+ (k-1)\left(1 - \frac{k}{k-1} \error_{\wt\calS}(\widehat f) \right) \\ &+ \left(k + \sqrt{k} + \frac{m}{n\sqrt{k}} \right) \sqrt{\frac{\log(4/\delta)}{2m}} + \sqrt{k(k-1)} \sqrt{\frac{4}{\delta}\left(\frac{1}{m} +\frac{3\beta}{m+n} \right)}  \,. \numberthis \label{eq:gd_multi_error}
    \end{align*} 
    % for some constant $c\le 3.2$.
\end{theorem}
\begin{proof}
    The proof of this theorem is divided into two parts. In the first part, we relate the error on the mislabeled samples with the population error on the mislabeled data. Similar to the proof of \thmref{thm:linear}, we use Shermann-Morrison formula to upper bound training error with leave-one-out error on each $\wh w^j$. Second part of the proof follows entirely from the proof of \thmref{thm:multiclass_ERM}. In essence, the first part derives an equivalent of \eqref{eq:lemma1_final_multi_prev} for GD training with squared loss and then the second part follows from the proof  of \thmref{thm:multiclass_ERM}. 
    
    \textbf{Part-1:} Consider a training point $(x_i,y_i)$ in $\wt S \cup S $. For simplicity, we use $c_i$ to denote the class of $i$-th point and use $y_i$ as the corresponding one-hot embedding. Recall error in multiclass point is given by $\error(\wh f(x_i), y_i ) = \indict{ c_i \not \in \argmax x_i^T \wh w }$. Thus, there exists a $j \ne c_i \in [k]$, such that we have
     \begin{align}
        \error(\wh f(x_i), y_i ) &= \indict{ c_i \not \in \argmax x_i^T \wh w } = \indict{ x_i^T \wh w_{c_i} < x_i^T \wh w_{j}  } \\ &= \indict{ x_i^T \left( \bX^T\bX+\lambda \boldsymbol{I}\right)^{-1}\bX^T\by_{c_i} < x_i^T \left( \bX^T\bX+\lambda \boldsymbol{I}\right)^{-1}\bX^T\by_{j} } \\
        &= \indict{ x_i^T \underbrace{\left( \bXX{i}^T\bXX{i} + x_i ^T x_i +\lambda \boldsymbol{I}\right)^{-1}}_{\RN{1}} \left(\bXX{i}^T{\by_{c_i}}_{(i)} + x_i - \bXX{i}^T{\by_{j}}_{(i)}\right) < 0 } \,.
    \end{align}
    Letting $\bA = \left(\bXX{i}^T\bXX{i} +\lambda \boldsymbol{I}\right)$ 
    and using \lemref{lem:sherman} on term 1, we have 
    \begin{align}
        \error(\wh f(x_i), y_i ) &= \indict{ x_i^T \left[\bA^{-1} -  \frac{\bA^{-1} x_i x_i^T \bA^{-1}}{ 1 + x_i ^T \bA^{-1} x_i } \right]  \left(\bXX{i}^T{\by_{c_i}}_{(i)} + x_i - \bXX{i}^T{\by_{j}}_{(i)}\right) < 0 } \\
        &= \indict{ \left[ \frac{ x_i^T \bA^{-1} ( 1 + x_i ^T \bA^{-1} x_i ) -  x_i^T \bA^{-1} x_i x_i^T \bA^{-1}}{ 1 + x_i ^T \bA ^{-1}x_i } \right]  \left(\bXX{i}^T{\by_{c_i}}_{(i)} + x_i - \bXX{i}^T{\by_{j}}_{(i)}\right) < 0 } \\
        &= \indict{ \left[ \frac{ x_i^T \bA^{-1}}{ 1 + x_i ^T \bA ^{-1}x_i } \right]  \left(\bXX{i}^T{\by_{c_i}}_{(i)} + x_i - \bXX{i}^T{\by_{j}}_{(i)}\right) < 0} \,.
    \end{align}
    Since $1 + x_i^T \bA^{-1} x_i > 0$, we have 
    \begin{align}
        \error(\wh f(x_i), y_i ) &= \indict{ x_i^T \bA^{-1}  \left(\bXX{i}^T{\by_{c_i}}_{(i)} + x_i - \bXX{i}^T{\by_{j}}_{(i)}\right) < 0 } \\
        &= \indict{ x_i^T \bA^{-1} x_i +  x_i^T \bA^{-1}  \bXX{i}^T{\by_{c_i}}_{(i)}  - x_i^T\bA^{-1}  \bXX{i}^T{\by_{j}}_{(i)} < 0 } \\
        &\le \indict{  x_i^T \bA^{-1}  \bXX{i}^T{\by_{c_i}}_{(i)}  - x_i^T\bA^{-1}  \bXX{i}^T{\by_{j}}_{(i)} < 0  } = \error(\ff{i}(x_i), y_i ) \,.\label{eq:LOO_error_multi}
    \end{align}
    Using \eqref{eq:LOO_error_multi}, we have 
    \begin{align}
        \error_{\wt \calS_M } (\wh f) \le \error_{\text{LOO} (\wt S_M)} \defeq \frac{\sum_{(x_i, y_i) \in \wt S_M} \error(\ff{i}(x_i), y_i ) }{\abs{\wt \calS_M}}\label{eq:LOO_error_multi_final} \,.
    \end{align}
    
    We now relate RHS in \eqref{eq:LOO_error_final} 
    with the population error on mislabeled distribution. 
    Similar as before, to do this, we leverage \codref{cond:hypothesis_stability} 
    and \lemref{lem:stability_error}. Using  \eqref{eq:final_mislabeled_linear} and \eqref{eq:LOO_error_multi_final}, we have 
    \begin{align}
        \error_{\wt \calS_M } (\wh f) \le \error_{\calDm}(\wh f)   + \sqrt{\frac{1}{\delta}\left(\frac{1}{2m_1} +\frac{3\beta}{m+n} \right)} \,. \label{eq:linear_multi_parallel_lem1}
    \end{align}
    
    We have now derived a parallel to \eqref{eq:lemma1_final_multi_prev}. Using the same arguments in the proof of \lemref{lem:fit_mislabeled_multi}, we have 
    \begin{align}
      \error_{\calD}(\wh f) \le  (k-1) \left( 1- \error_{ \wt \calS_M}(\wh f) \right)  + (k-1)\sqrt{\frac{k}{\delta(k-1)}\left(\frac{1}{2m_1} +\frac{3\beta}{m+n} \right)}  \,. \label{eq:lemma1_linear_final_multi}
    \end{align}
    
    \textbf{Part-2:} We now combine the results in \lemref{lem:mislabeled_error_multi} and \lemref{lem:clear_error_multi} to obtain the final inequality in terms of quantities that can be computed from just the randomly labeled and clean data. Similar to the binary case, we obtained a polynomial concentration instead of exponential concentration. Combining \eqref{eq:lemma1_linear_final_multi} with \lemref{lem:mislabeled_error_multi} and \lemref{lem:clear_error_multi}, we have with probability at least $1-\delta$
    \begin{align*}
        \error_\calD(\widehat f) \le \error_\calS(\widehat f) &+ (k-1)\left(1 - \frac{k}{k-1} \error_{\wt\calS}(\widehat f) \right) \\ &+ \left(k + \sqrt{k} + \frac{m}{n\sqrt{k}} \right) \sqrt{\frac{\log(4/\delta)}{2m}} + \sqrt{k(k-1)} \sqrt{\frac{4}{\delta}\left(\frac{1}{m} +\frac{3\beta}{m+n} \right)}  \,. \numberthis \label{eq:gd_multi_error_proof}
    \end{align*} 
\end{proof}

\subsection{Discussion on \codref{cond:hypothesis_stability}} \label{app:discuss_cond1}
The quantity in LHS of \codref{cond:hypothesis_stability} 
measures how much the function learned by the algorithm 
(in terms of error on unseen point) will change 
when one point in the training set is removed. 
% Discussion on exponential concentration and stronger condition. 
% Notice that hypothesis stability implies error stability, i.e., \codref{cond:error_stability} \citep{bousquet2002stability}.  
% In summary, while error stability allowed us 
% to relate the average population error 
% of the leave-one-out classifiers 
% with the population error of the original classifier, 
We need hypothesis stability condition 
to control the variance of the empirical leave-one-out error to show concentration of average leave-one-error with the population error. 

Additionally, we note that while the dominating term in the RHS of \thmref{thm:linear} matches with the dominating term in ERM bound in \thmref{thm:error_ERM}, there is a polynomial concentration term 
(dependence on $1/\delta$ instead of $\log(\sqrt{1/\delta})$) 
in \thmref{thm:linear}. 
Since with hypothesis stability, 
we just bound the variance, 
the polynomial concentration is due 
to the use of Chebyshev's inequality 
instead of an exponential tail inequality
(as in \lemref{lem:fit_mislabeled}).
Recent works have highlighted that 
a slightly stronger condition than hypothesis stability 
can be used to obtain an exponential concentration 
for leave-one-out error \citep{abou2019exponential},
but we leave this for future work for now. 
% We leave 
% However, the constants 

% we also want to highlight  

\subsection{Formal statement and proof of \propref{prop:early_stop}} \label{app:formal_early_stop}

Before formally presenting the result, 
we will introduce some notation.  
By $\calL_{S}(w)$, we denote 
the objective in \eqref{eq:l2_MSE_app} with $\lambda=0$. 
Assume Singular Value Decomposition (SVD) of $\bX$
as $\sqrt{n} \bU \bS^{1/2} \bV^T$. 
Hence $\bX^T \bX = \bV \bS \bV^T$.
Consider the GD iterates defined in \eqref{eq:GD_iterates_app}. 
% 
We now derive closed form expression 
for the $t^\text{th}$ iterate of gradient descent:  
% 
\begin{align}
    w_t = w_{t-1} + \eta \cdot \bX^T (\by - \bX w_{t-1}) = (\bI - \eta \bV \bS \bV^T )w_{k-1} + \eta \bX^T \by \,.
\end{align}
Rotating by $\bV^T$, we get 
\begin{align}
    \wt w_t = (\bI - \eta\bS )\wt w_{k-1} + \eta \wt \by \label{eq:GD_recur},
\end{align}
where $\wt w_t = \bV^T w_t $ and $\wt \by = \bV^T \bX^T \by$. 
Assuming the initial point $w_0 = 0$ 
and applying the recursion in \eqref{eq:GD_recur}, we get
\begin{align}
    \wt w_t = \bS ^{-1} ( \bI - (\bI - \eta \bS)^k ) \wt \by \,, 
\end{align} 
Projecting solution back to the original space, we have 
\begin{align}
     w_t = \bV \bS ^{-1} ( \bI - (\bI - \eta \bS)^k ) \bV^T \bX^T \by \,. 
\end{align} 
% We will work with this GD solution at any iterate $t$ in the next proposition. 
Define $f_t(x) \defeq f(x;w_t)$ 
as the solution at the $t^{\text{th}}$ iterate. 
Let $\wt w_{\lambda} = \argmin_{w} \calL_\calS (w;\lambda) = (\bX^T \bX + \lambda \bI)^{-1} \bX^T \by = \bV (\bS + \lambda \bI )^{-1} \bV^T \bX^T \by $. 
% ) \,,$ for all $t=1,2,\ldots\,.$ 
and define $\wt f_\lambda(x) \defeq f(x;\wt w_\lambda)$ as the regularized solution. 
Assume $\kappa$ be the condition number 
of the population covariance matrix 
and let $s_\text{min}$ be the minimum positive 
singular value of the empirical covariance matrix. 
Our proof idea is inspired from recent work 
on relating gradient flow solution 
and regularized solution 
for regression problems \citep{ali2018continuous}. 
We will use the following lemma in the proof: 
\begin{lemma} \label{lem:ineq_soln}
    For all $x \in [0,1]$ and for all $ k \in \mathbb{N}$, 
    we have (a) $ \frac{kx}{1+kx} \le 1- (1-x)^k$ 
    and (b) $ 1- (1-x)^k \le 2 \cdot \frac{kx}{kx+1} $.
    %  where $g(c)$ is a constant dependent on $c$. For $c = 1$, $g(c) = 2.0$.   
\end{lemma}
\begin{proof}
    % [Proof of \lemref{lem:ineq_soln}]
    % Part (a) is easy. 
    Using $ (1-x)^k \le \frac{1}{1+kx}$, we have part (a). 
    For part (b), we numerically maximize 
    $\frac{ (1+kx ) (1 - (1-x)^k) }{kx}$ 
    for all $k\ge 1$ and for all $x \in [0, 1]$.  
\end{proof}

% 
% Next, 

\begin{prop}[Formal statement of \propref{prop:early_stop}] \label{prop:formal_early_stop}
Let $\lambda = \frac{1}{t\eta}$. 
For a training point $x$, we have 
\begin{align*}
    \Expt{x \sim \calS}{(f_t(x) - \wt f_\lambda(x))^2} &\le c(t,\eta) \cdot \Expt{x \sim \calS}{f_t(x)^2} \,, %\label{eq:early_stop}
\end{align*}
where $c(t, \eta) \defeq \min( 0.25, \frac{1}{s_\text{min}^2 t^2 \eta^2})$. 
Similarly for a test point, we have 
\begin{align*}
    \Expt{x \sim \calD_\calX}{(f_t(x) - \wt f_\lambda(x))^2} &\le \kappa \cdot c(t,\eta) \cdot \Expt{x \sim \calD_\calX}{f_t(x)^2} \,. %\label{eq:early_stop}
\end{align*}
\end{prop} 

\begin{proof}
    %%%%%%%%%%%%% 
    We want to analyze the expected squared difference output 
    of regularized linear regression 
    with regularization constant $\lambda = \frac{1}{\eta t}$ 
    and the gradient descent solution at the $t^\text{th}$ iterate. 
    We separately expand the algebraic expression 
    for squared difference at a training point and a test point. 
    % We start by considering the difference  
    Then the main step is to show that 
    $\left[ \bS ^{-1} ( \bI - (\bI - \eta \bS)^k )  - (\bS + \lambda \bI )^{-1}\right] \preceq c(\eta, t) \cdot \bS ^{-1} ( \bI - (\bI - \eta \bS)^k ) $.

    %%%%%%%%%%%%%
    
   \textbf{Part 1 {} {}} 
    First, we will analyze the squared difference 
    of the output at a training point 
    (for simplicity, we refer to $S \cup \wt S$ as $S$), i.e., 
    \begin{align}
        \Expt{ x \sim \calS }{\left(f_t(x) - \wt f_\lambda (x)\right)^2} &= \norm{\bX w_t - \bX \wt w_\lambda}{2}^2\\ &=   \norm{\bX \bV \bS ^{-1} ( \bI - (\bI - \eta \bS)^t ) \bV^T \bX^T \by - \bX \bV (\bS + \lambda \bI )^{-1} \bV^T \bX^T \by }{2}^2 \\
        &= \norm{\bX \bV \left(\bS ^{-1} ( \bI - (\bI - \eta \bS)^t ) - (\bS + \lambda \bI )^{-1} \right) \bV^T \bX^T \by  }{2} \\
        &=  \by^T \bV \bX \left( \underbrace{\bS ^{-1} ( \bI - (\bI - \eta \bS)^t ) - (\bS + \lambda \bI )^{-1}}_{\RN{1}} \right)^2 \bS \bV^T \bX^T \by \label{eq:train_GD_rel} \,.
        %  (\bX \bV \bS ^{-1} ( \bI - (\bI - \eta \bS)^k ) \bV^T \bX^T \by)^T \bX \bV \bS ^{-1} ( \bI - (\bI - \eta \bS)^k ) \bV^T \bX^T \by
    \end{align}
    We now separately consider term 1. 
    Substituting $\lambda = \frac{1}{t \eta}$, 
    we get
    \begin{align}
        \bS ^{-1} ( \bI - (\bI - \eta \bS)^t ) - (\bS + \lambda \bI )^{-1} &= \bS^{-1} \left( ( \bI - (\bI - \eta \bS)^t ) - (\bI + \bS^{-1} \lambda )^{-1}\right) \\
        &= \underbrace{\bS^{-1} \left( ( \bI - (\bI - \eta \bS)^t ) - (\bI + ( \bS t \eta)^{-1}  )^{-1}\right)}_{\bA} \,.
    \end{align}

    We now separately bound the diagonal entries in matrix $\bA$. 
    With $s_i$, we denote $i^{\text{th}}$ diagonal entry of $\bS$.
    Note that since $ \eta\le 1/\norm{S}{\text{op}}$, 
    for all $i$, $\eta s_i  \le 1$.  
    Consider $i^{\text{th}}$ diagonal term (which is non-zero) 
    of the diagonal matrix $\bA$, we have 
    \begin{align}
        \bA_{ii} = \frac{1}{s_i} \left(  1 - (1 - s_i \eta)^t - \frac{t \eta s_i}{1 + t \eta s_i } \right) &=  \frac{1 - (1 - s_i \eta)^t}{s_i} \left( \underbrace{ 1 - \frac{t \eta s_i}{(1 + t \eta s_i)(1 - (1 - s_i \eta)^t)}}_{\RN{2}} \right) \\ 
         &\le \frac{1}{2}\left[ \frac{1 - (1 - s_i \eta)^t}{ s_i} \right] \tag*{(Using \lemref{lem:ineq_soln} (b))} \,.
    \end{align} 
    Additionally, we can also show the following upper bound on term 2: 
    \begin{align}
         1 - \frac{t \eta s_i}{(1 + t \eta s_i)(1 - (1 - s_i \eta)^t)} &= \frac{(1 + t \eta s_i)(1 - (1 - s_i \eta)^t) - t \eta s_i }{(1 + t \eta s_i)(1 - (1 - s_i \eta)^t)} \\
         & \le  \frac{ 1 -  (1 - s_i \eta)^t - t \eta s_i (1 - s_i \eta)^t}{(1 + t \eta s_i)(1 - (1 - s_i \eta)^t)} \\
         & \le \frac{1}{t\eta s_i} \,. \tag{Using \lemref{lem:ineq_soln} (a)}
        %  &\le \frac{1}{2}\left[ \frac{1 - (1 - s_i \eta)^t}{ s_i} \right] \tag*{(Using \lemref{lem:ineq_soln})} \,.
    \end{align} 

    Combining both the upper bounds 
    on each diagonal entry $\bA_{ii}$, we have 
    \begin{align}
    \bA \preceq c_1(\eta, t) \cdot \bS^{-1} ( \bI - (\bI - \eta \bS)^t ) \,, \label{eq:upperbound_diagonal}
    \end{align}
    where $c_1(\eta, t ) = \min(0.5, \frac{1}{t s_i \eta })$. Plugging this into \eqref{eq:train_GD_rel}, we have 
    \begin{align}
        \Expt{ x \sim \calS }{\left(f_t(x) - \wt f_\lambda (x)\right)^2} &\le c(\eta, t) \cdot \by^T \bV \bX  \left( \bS^{-1} ( \bI - (\bI - \eta \bS)^t ) \right)^2 \bS \bV^T \bX^T \by \\
        &=   c(\eta, t) \cdot \by^T \bV \bX  \left( \bS^{-1} ( \bI - (\bI - \eta \bS)^t ) \right) \bS \left( \bS^{-1} ( \bI - (\bI - \eta \bS)^t ) \right) \bV^T \bX^T \by \\
        & =  c(\eta, t) \cdot \norm{\bX w_t}{2}^2 \\
        &= c(\eta, t) \cdot  \Expt{ x \sim \calS }{\left(f_t(x) \right)^2} \,,
    \end{align}
    where $c(\eta, t ) = \min(0.25, \frac{1}{t^2 s^2_i \eta^2 })$.

    \textbf{Part 2 {} {}} With $\bSigma$, 
    we denote the underlying true covariance matrix. 
    We now consider the squared difference of output at an unseen point: 
    \begin{align}
        \Expt{ x \sim \calD_{\calX} }{\left(f_t(x) - \wt f_\lambda (x)\right)^2} &= \Expt{x \sim \calD_{\calX}}{\norm{x^T w_t - x^T \wt w_\lambda}{2}} \\
        &=   \norm{x^T \bV \bS ^{-1} ( \bI - (\bI - \eta \bS)^t ) \bV^T \bX^T \by - x^T \bV (\bS + \lambda \bI )^{-1} \bV^T \bX^T \by }{2} \\
        &= \norm{x^T \bV \left(\bS ^{-1} ( \bI - (\bI - \eta \bS)^t ) - (\bS + \lambda \bI )^{-1} \right) \bV^T \bX^T \by  }{2} \\
        &= \by^T \bV \bX \left( \bS ^{-1} ( \bI - (\bI - \eta \bS)^t ) - (\bS + \lambda \bI )^{-1} \right) \bV^T \bSigma \bV \\ &\qquad \qquad \qquad \qquad \qquad \left( (\bI - (\bI - \eta \bS)^t ) - (\bS + \lambda \bI )^{-1} \right) \bV^T \bX^T \by \\
        &\le \sigma_{\text{max}} \cdot \by^T \bV \bX \left( \underbrace{\bS ^{-1} ( \bI - (\bI - \eta \bS)^t ) - (\bS + \lambda \bI )^{-1}}_{\RN{1}} \right)^2 \bV^T \bX^T \by \,, \label{eq:test_GD_rel}
        %  (\bX \bV \bS ^{-1} ( \bI - (\bI - \eta \bS)^k ) \bV^T \bX^T \by)^T \bX \bV \bS ^{-1} ( \bI - (\bI - \eta \bS)^k ) \bV^T \bX^T \by
    \end{align}
    where $\sigma_{\text{max}}$ is the maximum eigenvalue 
    of the underlying covariance matrix $\bSigma$. 
    Using the upper bound on term 1 in \eqref{eq:upperbound_diagonal}, 
    we have 
    \begin{align}
        \Expt{ x \sim \calD_{\calX} }{\left(f_t(x) - \wt f_\lambda (x)\right)^2} &\le \sigma_{\text{max}} \cdot c(\eta, t) \cdot \by^T \bV \bX  \left( \bS^{-1} ( \bI - (\bI - \eta \bS)^t ) \right)^2 \bV^T \bX^T \by \\
        &=   \kappa \cdot c(\eta, t) \cdot \sigma_{\text{min}}\cdot \norm{\bV \left( \bS^{-1} ( \bI - (\bI - \eta \bS)^t ) \right) \bV^T \bX^T \by}{2}^2 \\
        &\le \kappa \cdot c(\eta, t) \cdot \left[ \bV \left( \bS^{-1} ( \bI - (\bI - \eta \bS)^t ) \right) \bV^T \bX^T \right]^T \bSigma \\
        &\qquad \qquad \qquad \qquad \qquad \left[ \bV \left( \bS^{-1} ( \bI - (\bI - \eta \bS)^t ) \right) \bV^T \bX^T \right] \by \\
        & = \kappa \cdot c(\eta, t) \cdot \Expt{x \sim \calD_{\calX}}{\norm{x^T w_t}{2}} \,.
    \end{align}
% 
% 
    % Since $ \eta\le 1/\norm{S}{\text{op}}$, invoking \lemref{lem:ineq_soln} to upper bound term 1 with
\end{proof}

\subsection{Extension to deep learning} \label{appsubsec:ext_DL}
Under \asmpref{appsubsec:justifying_assumption1}, we present the formal result parallel to \thmref{thm:multiclass_ERM}. 
\begin{theorem} \label{thm:multiclass_ERM_algoA}
    Consider a multiclass classification problem 
    with $k$ classes. Under \asmpref{asmp:deep_models}, 
    for any $\delta >0$, with probability at least $1-\delta$,
    we have
    \vspace{-10pt}
    \begin{align*}
        \error_\calD(\widehat f)  \le \error_\calS(\widehat f) + (k-1) \left(1 - \tfrac{k}{k-1} \error_{\wt\calS}(\widehat f)\right) + c\sqrt{\frac{\log(\frac{4}{\delta})}{2m}} \,,\numberthis \label{eq:multiclass_ERM_deep}
    % \vspace{-20pt}
    \end{align*}
    for some constant $c \le ((c+1) k+\sqrt{k} + \frac{m}{n\sqrt{k}})$.
\end{theorem}

The proof follows exactly as in step (i) to (iii) in \thmref{thm:multiclass_ERM}.  

\subsection{Justifying~\asmpref{asmp:deep_models}} \label{appsubsec:justifying_assumption1}

Motivated by the analysis on linear models, we now discuss alternate (and weaker) conditions that imply \asmpref{asmp:deep_models}. 
We need hypothesis stability (\codref{cond:hypothesis_stability}) and the following assumption relating training error and leave-one-error: 

\begin{assumption} \label{asmp:loo_error}
Let $\wh f$ be a model obtained by training with algorithm $\calA$ on a mixture of clean $S$ and randomly labeled data $\wt S$. Then we assume we have 
\begin{align*}
    \error_{\wt \calS_M} (\wh f) \le  \error_{\text{LOO} (\wt S_M)} \,, 
\end{align*}
for all $(x_i, y_i) \in  \wt S_M$ where $\wh f_{(i)} \defeq f(\calA, S \cup {{}\wt S_M}_{(i)})$ and  $\error_{\text{LOO} (\wt S_M)} \defeq  \frac{\sum_{(x_i, y_i) \in \wt S_M} \error(\ff{i}(x_i), y_i ) }{\abs{\wt \calS_M}}$.  
\end{assumption}

% we assume this to extend our result (parallel to \thmref{thm:multi_linear}) for deep models. 
Intuitively, this assumption states that the error on a (mislabeled) datum $(x,y)$ included in the training set is less than the error on that datum $(x,y)$ obtained by a model trained on the training set $S - \{(x,y)\}$. We proved this for linear models trained with GD in the proof of \thmref{thm:multi_linear}. 
% 
\codref{cond:hypothesis_stability} with $\beta = \calO(1)$ and \asmpref{asmp:loo_error} together with \lemref{lem:stability_error} implies \asmpref{asmp:deep_models} with a polynomial residual term (instead of logarithmic in $1/\delta$): 
\begin{align}
     \error_{\calS_M} (\wh f) \le  \error_{\calDm}(\wh f)   + \sqrt{\frac{1}{\delta}\left(\frac{1}{m} +\frac{3\beta}{m+n} \right)} \,.
\end{align}
% Note that this  

\newpage 
\section{Additional experiments and details}\label{app:exp}
\newcommand\tab[1][1cm]{\hspace*{#1}}

\subsection{Datasets} \label{sec:app_dataset}

\textbf{Toy Dataset {} {}} Assume fixed constants $\mu$ and $\sigma$. For a given label $y$, we simulate features $x$ in our toy classification setup as follows: 
\begin{align*}
    x \defeq \texttt{concat} \left[ x_1, x_2\right] \quad \text{where} \quad  x_1 \sim  \calN( y \cdot \mu, \sigma^2 I_{d \times d}) \ \  \text{and} \ \  x_1 \sim  \calN( 0, \sigma^2 I_{d \times d}) \,.
\end{align*}  
% where $y$ is the true label and $x$ is the corresponding feature vector. 
In experiements throughout the paper, we fix dimention $d=100$, $\mu = 1.0 $, and $\sigma = \sqrt{d}$. Intuitively, $x_1$ carries the information about the underlying label and $x_2$ is additional noise independent of the underlying label. 

\textbf{CV datasets {} {}} We use MNIST~\citep{lecun1998mnist} and CIFAR10~\cite{krizhevsky2009learning}. 
% For binary tasks, 
We produce a binary variant from the multiclass classification problem by mapping classes $\{0,1,2,3,4\}$ to label $1$ and $\{ 5,6,7,8,9\}$ to label $-1$. For CIFAR dataset, we also use the standard data augementation of random crop and horizontal flip. PyTorch code is as follows: 

\texttt{(transforms.RandomCrop(32, padding=4),\\
\tab transforms.RandomHorizontalFlip())}

\textbf{NLP dataset {} {}} We use IMDb Sentiment analysis~\citep{maas2011learning} corpus.  

\subsection{Architecture Details} 

All experiments were run on NVIDIA GeForce RTX 2080 Ti GPUs. We used PyTorch~\citep{NEURIPS2019a9015} and Keras with Tensorflow~\citep{abadi2016tensorflow} backend for experiments. 
% , ELMo embeddings~\citep{Peters:2018}, and Hugging Face Transformers~\citep{wolf-etal-2020-transformers}. 

\textbf{Linear model {} {}} For the toy dataset, we simulate a linear model with scalar output and the same number of parameters as the number of dimensions.   

\textbf{Wide nets {} {}} To simulate the NTK regime, we experiment with $2-$layered wide nets. The PyTorch code for 2-layer wide MLP is as follows: 


\texttt{ nn.Sequential( \\
\tab     nn.Flatten(),\\
\tab    nn.Linear(input\_dims, 200000, bias=True),\\
\tab    nn.ReLU(),\\
\tab    nn.Linear(200000, 1, bias=True)\\
\tab     )}


We experiment both (i) with the second layer fixed at random initialization; (ii)  and updating both layers' weights.     

\textbf{Deep nets for CV tasks {} {}} We consider a 4-layered MLP. The PyTorch code for 4-layer MLP is as follows: 

\texttt{ nn.Sequential(nn.Flatten(), \\
\tab        nn.Linear(input\_dim, 5000, bias=True),\\
\tab        nn.ReLU(),\\
\tab        nn.Linear(5000, 5000, bias=True),\\
\tab        nn.ReLU(),\\
\tab        nn.Linear(5000, 5000, bias=True),\\
\tab        nn.ReLU(),\\
% \tab        nn.Linear(5000, 5000, bias=True),\\
% \tab        nn.ReLU(),\\
\tab        nn.Linear(1024, num\_label, bias=True)\\
\tab        )}

For MNIST, we use $1000$ nodes instead of $5000$ nodes in the hidden layer. 
% 
We also experiment with convolutional nets. In particular, we use ResNet18 \citep{he2016deep}. Implementation adapted from:  \url{https://github.com/kuangliu/pytorch-cifar.git}. 

\textbf{Deep nets for NLP {} {}} We use a simple LSTM model with embeddings intialized with ELMo embeddings~\citep{Peters:2018}. Code adapted from: \url{https://github.com/kamujun/elmo_experiments/blob/master/elmo_experiment/notebooks/elmo_text_classification_on_imdb.ipynb} 

We also evaluate our bounds with a BERT model. In particular, we fine-tune an off-the-shelf uncased BERT model~\citep{devlin2018bert}. Code adapted from Hugging Face Transformers~\citep{wolf-etal-2020-transformers}: \url{https://huggingface.co/transformers/v3.1.0/custom_datasets.html}. 


\subsection{Additonal experiments}

\textbf{Results with SGD on underparameterized linear models {} {}} 

\begin{figure*}[h]
    \centering 
    % \vspace{-15pt}
    % \includegraphics[width=0.9\linewidth]{example-image-a}
    \includegraphics[width=0.3\linewidth]{figures/lowdim-Gaussian-SGD.pdf}
    % \includegraphics[width=0.9\linewidth]{figures/{CIFAR10_rn=0.1_lr=0.2_wd=0.005}.png}
    \vspace{-5pt}
    \caption{ 
    % Predicted lower bound 
    % on different
    We plot the accuracy and corresponding bound 
    (RHS in \eqref{eq:erm}) at $\delta = 0.1$
    for toy binary classification task. 
    Results aggregated over $3$ seeds. 
    % i.e., $1-\error$ where $\error$ is the term in the RHS of \eqref{eq:erm}
    Accuracy vs fraction of unlabeled data (w.r.t clean data) 
    in the toy setup with a linear model trained with SGD. Results parallel to \figref{fig:error_binary}(a) with SGD.  }
    \label{fig:error_binary_linear}
    \vspace{-5pt}
\end{figure*}

\textbf{Results with wide nets on binary MNIST {} {}}

\begin{figure*}[h]
    \centering 
    % \vspace{-15pt}
    % \includegraphics[width=0.9\linewidth]{example-image-a}
    \subfigure[GD with MSE loss]{\includegraphics[width=0.3\linewidth]{figures/MNIST-GD_MSE.pdf}} \hfil
    \subfigure[SGD with CE loss]{\includegraphics[width=0.3\linewidth]{figures/MNIST-SGD_CE.pdf}}
    \subfigure[SGD with MSE loss]{\includegraphics[width=0.3\linewidth]{figures/MNIST-SGD_MSE-first-layer.pdf}}
    % \includegraphics[width=0.9\linewidth]{figures/{CIFAR10_rn=0.1_lr=0.2_wd=0.005}.png}
    \vspace{-5pt}
    \caption{ 
    % Predicted lower bound 
    % on different
    We plot the accuracy and corresponding bound 
    (RHS in \eqref{eq:erm}) at $\delta = 0.1$ 
    for binary MNIST classification. 
    Results aggregated over $3$ seeds. 
    % i.e., $1-\error$ where $\error$ is the term in the RHS of \eqref{eq:erm}
    Accuracy vs fraction of unlabeled data 
    for a 2-layer wide network on binary MNIST with both the layers training in (a,b) and only first layer training in (c). 
    Results parallel to \figref{fig:error_binary}(b) .  }
    \label{fig:error_binary_MNIST}
    \vspace{-5pt}
\end{figure*}

% \begin{figure*}[h]
%     \centering 
%     % \vspace{-15pt}
%     % \includegraphics[width=0.9\linewidth]{example-image-a}
%     \subfigure[GD with MSE loss]{\includegraphics[width=0.3\linewidth]{figures/MNIST.pdf}} \hfil
    
%     \subfigure[SGD with CE loss]{\includegraphics[width=0.3\linewidth]{figures/MNIST.pdf}}
%     % \includegraphics[width=0.9\linewidth]{figures/{CIFAR10_rn=0.1_lr=0.2_wd=0.005}.png}
%     \vspace{-5pt}
%     \caption{ 
%     % Predicted lower bound 
%     % on different
%     We plot the accuracy and corresponding bound 
%     (RHS in \eqref{eq:erm}) at $\delta = 0.1$
%     for binary MNIST classification. 
%     Results aggregated over $3$ seeds. 
%     % i.e., $1-\error$ where $\error$ is the term in the RHS of \eqref{eq:erm}
%     Accuracy vs fraction of unlabeled data 
%     for a 2-layer wide network on binary MNIST with just the first layer training. 
%     Results parallel to \figref{fig:error_binary}(b) with only the first layer training.  }
%     \label{fig:error_binary_MNIST}
%     \vspace{-5pt}
% \end{figure*}

\textbf{Results on CIFAR 10 and MNIST {} {}} 
% 
We plot epoch wise error curve for results in \tabref{table:multiclass}(\figref{fig:error_epoch_CIFAR10} and \figref{fig:error_epoch_MNIST}). We observe the same trend as in \figref{fig:error_CIFAR10}. Additionally, we plot an \emph{oracle bound} obtained by tracking the error on mislabeled data which nevertheless were predicted as true label. To obtain an exact emprical value of the oracle bound, we need underlying true labels for the randomly labeled data. 
% Note that our bound in \thmref{thm:multiclass_ERM}, lower bounds the accuracy as predicted by the oracle bound. 
While with just access to extra unlabeled data we cannot calculate oracle bound, we note that the oracle bound is very tight and never violated in practice underscoring an importamt aspect of generalization in multiclass problems. This highlight that even a stronger conjecture may hold in multiclass classification, i.e., error on mislabeled data (where nevertheless true label was predicted) lower bounds the population error on the distribution of mislabeled data and hence, the error on (a specific) mislabeled portion predicts the population accuracy on clean data. 
% 
On the other hand, the dominating term of in \thmref{thm:multiclass_ERM} is loose when compared with the oracle bound. The main reason, we believe is the pessimistic upper bound in \eqref{eq:lemma1_final_multi_prev} in the proof of \lemref{lem:fit_mislabeled_multi}. We leave an investigation on this gap for future. 
% of fit 

% However, oracle bound highlights two . One,  



\begin{figure}[h]
    \centering 
    % \vspace{-15pt}
    % \includegraphics[width=0.9\linewidth]{example-image-a}
    \subfigure[MLP]{\includegraphics[width=0.3\linewidth]{figures/CIFAR10-FNN.pdf}} \hfil
    \subfigure[ResNet]{\includegraphics[width=0.3\linewidth]{figures/CIFAR10-Resnet.pdf}}
    % \includegraphics[width=0.9\linewidth]{figures/{CIFAR10_rn=0.1_lr=0.2_wd=0.005}.png}
    % \vspace{-10pt}
    \caption{ Per epoch curves for CIFAR10 corresponding results in \tabref{table:multiclass}. As before, we just plot the dominating term in the RHS of \eqref{eq:multiclass_ERM} as predicted bound. Additionally, we also plot the predicted lower bound by the error on mislabeled data which nevertheless were predicted as true label. We refer to this as ``Oracle bound''. See text for more details. 
    % 
    % except for the stopping point. 
    % The bound predicted by RATT (RHS in \eqref{eq:multiclass_ERM}) is vacuous. 
    }\label{fig:error_epoch_CIFAR10}
    % \vspace{-15pt}
\end{figure}


\begin{figure}[h]
    \centering 
    % \vspace{-15pt}
    % \includegraphics[width=0.9\linewidth]{example-image-a}
    \subfigure[MLP]{\includegraphics[width=0.3\linewidth]{figures/MNIST-FNN.pdf}} \hfil
    \subfigure[ResNet]{\includegraphics[width=0.3\linewidth]{figures/MNIST-Resnet.pdf}}
    % \includegraphics[width=0.9\linewidth]{figures/{CIFAR10_rn=0.1_lr=0.2_wd=0.005}.png}
    % \vspace{-10pt}
    \caption{ Per epoch curves for MNIST corresponding results in \tabref{table:multiclass}. As before, we just plot the dominating term in the RHS of \eqref{eq:multiclass_ERM} as predicted bound. Additionally, we also plot the predicted lower bound by the error on mislabeled data which nevertheless were predicted as true label. We refer to this as ``Oracle bound''. See text for more details. 
    % 
    % except for the stopping point. 
    % The bound predicted by RATT (RHS in \eqref{eq:multiclass_ERM}) is vacuous. 
    }\label{fig:error_epoch_MNIST}
    % \vspace{-15pt}
\end{figure}

\textbf{Results on CIFAR 100 {} {}} 
% 
On CIFAR100, our bound in \eqref{eq:multiclass_ERM} yields vacous bounds. However, the oracle bound as explained above yields tight guarantees in the initial phase of the learning (i.e., when learning rate is less than $0.1$) (\figref{fig:error_CIFAR100}).  

\begin{figure}[h]
    \centering 
    % \vspace{-15pt}
    % \includegraphics[width=0.9\linewidth]{example-image-a}
    \includegraphics[width=0.3\linewidth]{figures/CIFAR100-Resnet.pdf}
    % \includegraphics[width=0.9\linewidth]{figures/{CIFAR10_rn=0.1_lr=0.2_wd=0.005}.png}
    % \vspace{-10pt}
    \caption{ Predicted lower bound by the error on mislabeled data which nevertheless were predicted as true label with ResNet18 on CIFAR100. We refer to this as ``Oracle bound''. See text for more details. 
    % 
    % except for the stopping point. 
    The bound predicted by RATT (RHS in \eqref{eq:multiclass_ERM}) is vacuous. 
    }\label{fig:error_CIFAR100}
    % \vspace{-15pt}
\end{figure}


% \paragraph{Experiments on CIFAR100} 


% \subsection{Model Selection using RATT}


\subsection{Hyperparameter Details}


\textbf{\figref{fig:error_CIFAR10} {} {}} We use clean training dataset of size $40,000$. We fix the amount of unlabeled data at $20\%$ of the clean size, i.e. we include additional $8,000$ points with randomly assigned labels. We use test set of $10,000$ points. For both MLP and ResNet, we use SGD with an initial learning rate of $0.1$ and momentum $0.9$. We fix the weight decay parameter at $5\times 10^{-4}$. After $100$ epochs, we decay the learning rate to $0.01$. We use SGD batch size of $100$. 

\textbf{\figref{fig:error_binary} (a) {} {}} We obtain a toy dataset according to the process described in \secref{sec:app_dataset}. We fix $d=100$ and create a dataset of $50,000$ points with balanced classes. Moreover, we sample additional covariates with the same procedure to create randomly labeled dataset. For both SGD and GD training, we use a fixed learning rate $0.1$.    

\textbf{\figref{fig:error_binary} (b) {} {}} Similar to binary CIFAR, we use clean training dataset of size $40,000$ and fix the amount of unlabeled data at $20\%$ of the clean dataset size. To train wide nets, we use a fixed learning of $0.001$ with GD and SGD. We decide the weight decay parameter and the early stopping point that maximizes our generalization bound (i.e. without peeking at unseen data ).  We use SGD batch size of $100$. 

\textbf{\figref{fig:error_binary} (c) {} {}} With IMDb dataset, we use a clean dataset of size $20,000$ and as before, fix the amount of unlabeled data at $20\%$ of the clean data. To train ELMo model, we use Adam optimizer with a fixed learning rate $0.01$ and weight decay $10^{-6}$ to minimize cross entropy loss. We train with batch size $32$ for 3 epochs. To fine-tune BERT model, we use Adam optimizer with learning rate $5\times 10^{-5}$ to minimize cross entropy loss. We train with a batch size of $16$ for 1 epoch.    

\textbf{\tabref{table:multiclass} {} {}} For multiclass datasets, we train both MLP and ResNet with the same hyperparameters as described before. We sample a clean training dataset of size $40,000$ and fix the amount of unlabeled data at $20\%$ of the clean size. We use SGD with an initial learning rate of $0.1$ and momentum $0.9$. We fix the weight decay parameter at $5\times 10^{-4}$. After $30$ epochs for ResNet and after $50$ epochs for MLP, we decay the learning rate to $0.01$.  We use SGD with batch size $100$. 
For \figref{fig:error_CIFAR100}, we use the same hyperparameters as 
CIFAR10 training, except we now decay learning rate after $100$ epochs. 


In all experiments, to identify the best possible accuracy on just the clean data, we use the exact same set of hyperparamters except the stopping point. We choose a stopping point that maximizes test performance. 

\subsection{Summary of experiments }

\begin{center}
    \begin{table}[H] 
        \centering
        \begin{tabular}{|c|c|c|c|} 
        \hline
        Classification type & Model category & Model & Dataset  \\ [0.5ex] 
        \hline
        \hline
        \multirow{10}{*}{Binary} & Low dimensional & Linear model & Toy Gaussain dataset  \\
                        \cline{2-4}
                         & Overparameterized 
                        %  & Linear model & Toy Gaussain dataset \\
                        %  \cline{3-4}
                        %  & & 2-layer wide net& Toy Gaussain dataset \\
                        %  \cline{3-4}
                         & \multirow{2}{*}{2-layer wide net} & \multirow{2}{*}{Binary MNIST} \\
                         & linear nets & &  
                         \\
                         \cline{2-4}                 
                         & \multirow{6}{*}{Deep nets} & \multirow{2}{*}{MLP} & Binary MNIST \\
                         \cline{4-4}
                         & &  & Binary CIFAR \\
                         \cline{3-4}
                         &  & \multirow{2}{*}{ResNet} & Binary MNIST \\
                         \cline{4-4}
                         & &  & Binary CIFAR \\
                         \cline{3-4}
                         &  & ELMo-LSTM model & IMDb Sentiment Analysis \\
                         \cline{3-4}
                         & & BERT pre-trained model & IMDb Sentiment Analysis \\
        \hline
        \multirow{5}{*}{Multiclass} & \multirow{5}{*}{Deep nets} & \multirow{2}{*}{MLP} & MNIST \\
                        \cline{4-4} 
                        & & & CIFAR10 \\                   
                        \cline{3-4}
                         &   & \multirow{3}{*}{ResNet} & MNIST \\
                         \cline{4-4}
                         &   & & CIFAR10 \\
                         \cline{4-4}
                         &   & & CIFAR100 \\
        \hline
        \end{tabular}
        % \caption{Summary of experiments performed} \label{table:experiments}
    \end{table}    
    % \footnotetext[6]{We use both MSE loss and cross-entropy loss.}
    % \footnotetext[6]{We try 2 variants: one with a fixed first layer and the other with both layers trainable.}
\end{center}

\newpage
\section{Proof of \lemref{lem:stability_error}} \label{app:proof_lem_error}

\begin{proof}[Proof of \lemref{lem:stability_error}]
    Recall, we have a training set $S \cup \wt S_C$. We defined leave-one-out error on mislabeled points as $$\error_{\text{LOO}(\wt S_M) } = \frac{\sum_{(x_i, y_i) \in \wt S_M} \error( f_{(i)}( x_i), y_i)}{ \abs{\wt S_M }} \,, $$
    where $f_{(i)} \defeq f(\calA, (S \cup \wt S)_{(i)})$. Define $S^\prime \defeq S \cup \wt S$. Assume $(x,y)$ and $(x^\prime,y^\prime)$ as i.i.d. samples from ${\calDm}$. 
    Using Lemma 25 in \citet{bousquet2002stability}, we have
    \begin{align*}
        \Expo{ \left( \error_{\calDm}(\wh f) -\error_{\text{LOO}(\wt S_M) } \right)^2 } \le & \Expt{ S^\prime, (x,y), (x^\prime,y^\prime) }{ \error(\wh f(x), y ) \error(\wh f(x^\prime), y^\prime )} - 2 \Expt{ S^\prime, (x,y) }{ \error(\wh f(x), y ) \error(f_{(i)}(x_i), y_i )} \\
        & + \frac{m_1-1}{m_1}\Expt{ S^\prime }{  \error(f_{(i)}(x_i), y_i )  \error(f_{(j)}(x_j), y_j )} + \frac{1}{m_1} \Expt{ S^\prime }{  \error(f_{(i)}(x_i), y_i ) } \,. \numberthis \label{eq:main_reln}
    \end{align*}
    We can rewrite the equation above as : 
    \begin{align*}
        \Expo{ \left( \error_{\calDm}(\wh f) -\error_{\text{LOO}(\wt S_M) } \right)^2 } \le &  \, \underbrace{\Expt{ S^\prime, (x,y), (x^\prime,y^\prime) }{ \error(\wh f(x), y ) \error(\wh f(x^\prime), y^\prime ) - \error(\wh f(x), y ) \error(f_{(i)}(x_i), y_i )}}_{\RN{1}} \\
        & + \underbrace{\Expt{ S^\prime }{  \error(f_{(i)}(x_i), y_i )  \error(f_{(j)}(x_j), y_j ) -  \error(\wh f(x), y ) \error(f_{(i)}(x_i), y_i )}}_{\RN{2}} \\ &+ \underbrace{\frac{1}{m_1} \Expt{ S^\prime }{  \error(f_{(i)}(x_i), y_i ) - \error(f_{(i)}(x_i), y_i )  \error(f_{(j)}(x_j), y_j ) }}_{\RN{3}} \,. \numberthis \label{eq:main_reln2}
    \end{align*}
    
    We will now bound term $\RN{3}$.  Using Cauchy-Schwarz's inequality, we have
    
    \begin{align}
        \Expt{ S^\prime }{  \error(f_{(i)}(x_i), y_i ) - \error(f_{(i)}(x_i), y_i )  \error(f_{(j)}(x_j), y_j ) }^2 &\le  \Expt{ S^\prime }{  \error(f_{(i)}(x_i), y_i ) }^2 \Expt{S^\prime}{1 -   \error(f_{(j)}(x_j), y_j ) }^2 \\
        &\le \frac{1}{4} \,.\label{eq:term1_lem12}
    \end{align}
    
    Note that since $(x_i,y_i)$, $(x_j ,y_j )$, $(x,y)$, and $(x^\prime, y^\prime)$ are all from same distribution $\calDm$, we directly incorporate the bounds on term $\RN{1}$ and $\RN{2}$ from the proof of Lemma 9 in \citet{bousquet2002stability}. Combining that with \eqref{eq:term1_lem12} and our definition of hypothesis stability in \codref{cond:hypothesis_stability}, we have the required claim. 
    
    
    % We now re-write term $\RN{1}$ as
    % \begin{align*}
    %         &\Expt{S^\prime, (x,y), (x^\prime,y^\prime) }{ \error(\wh f(x), y ) \error(\wh f(x^\prime), y^\prime ) - \error(\wh f(x), y ) \error(f_{(i)}(x_i), y_i )} \\ & \qquad = \Expt{ S^\prime, (x,y), (x^\prime,y^\prime) }{ \error(\wh f(x), y ) \error(\wh f  (x^\prime), y^\prime ) - \error(\wh f ^\prime(x), y ) \error(f_{(i)}(x^\prime), y^\prime )} \tag{Exchanging $(x_i, y_i)$ with $(x^\prime, y^\prime)$ in the second term} \\
    %         & \qquad = \Expt{ S^\prime, (x,y), (x^\prime,y^\prime) }{  \left(\error(\wh f(x), y )-  \error(f_{(i)}(x), y ) \right) \error(\wh f  (x^\prime), y^\prime )  } \\
    %         & \qquad  + \Expt{ S^\prime, (x,y), (x^\prime,y^\prime) }{  \left(\error(f_{(i)}(x), y ) -\error(\wh f ^\prime(x), y ) \right) \error(\wh f  (x^\prime), y^\prime )}  \\
    %         & \qquad +\Expt{ S^\prime, (x,y), (x^\prime,y^\prime) }{  \left( \error(\wh f  (x^\prime), y^\prime ) -  \error(f_{(i)}(x^\prime), y^\prime ) \right) \error(\wh f ^\prime(x), y ) }  \,, \numberthis \label{eq:term1_final}
    % \end{align*}
    % where $\wh f^\prime$ is the classifier obtained by training on $ S^\prime_{(i)} \cup \{ (x^\prime, y^\prime) \} $. Similarly we can re-write term $\RN{2}$ as 
    % \begin{align*}
    %     & \Expt{ S^\prime }{  \error(f_{(i)}(x_i), y_i )  \error(f_{(j)}(x_j), y_j ) -  \error(\wh f(x), y ) \error(f_{(i)}(x_i), y_i )} \\
    %     &\quad  = \Expt{ S^\prime, (x,y), (x^\prime,y^\prime)}{  \error(f^{\prime\prime}_{(i)}(x), y )  \error(f_{(j)}^{\prime}(x^\prime), y^\prime ) -  \error(\wh f(x), y ) \error(f_{(i)}(x_i), y_i )} \tag{Exchanging $(x_i, y_i)$ with $(x, y)$ and $(x_j, y_j)$ with $(x^\prime, y^\prime)$ in the first term}\\
    %     &\quad = \Expt{ S^\prime, (x,y), (x^\prime,y^\prime)}{  \error(f^{\prime\prime}_{(j)}(x), y )  \error(f_{(i)}^{\prime}(x^\prime), y^\prime ) -  \error(\wh f^\prime (x), y ) \error(f^\prime_{(j)}(x^\prime), y^\prime )} \tag{Exchanging $(x_i, y_i)$ and $(x_j, y_j)$ and then replacing $(x_j, y_j)$ with $(x^\prime, y^\prime)$ in the second term} \\
    %     & \quad = \Expt{ S^\prime, (x,y), (x^\prime,y^\prime) }{  \left( \error(f_{(i)}^{\prime}(x^\prime), y^\prime )   -  \error(\wh f^{\prime\prime}  (x^\prime), y^\prime ) \right)  \error(f^{\prime\prime}_{(j)}(x), y )   } \\
    %     & \quad  + \Expt{ S^\prime, (x,y), (x^\prime,y^\prime) }{  \left( \error(f^{\prime\prime}_{(j)}(x), y )  -\error(\wh f ^\prime(x), y ) \right) \error(\wh f^{\prime\prime}  (x^\prime), y^\prime )  }  \\
    %     & \quad+ \Expt{ S^\prime, (x,y), (x^\prime,y^\prime) }{  \left( \error(\wh f^{\prime\prime}  (x^\prime), y^\prime )  -  \error(f^\prime_{(j)}(x^\prime), y^\prime ) \right)  \error(\wh f^\prime (x), y ) }   \\
    %     & \quad = \Expt{ S^\prime, (x,y), (x^\prime,y^\prime) }{  \left( \error(f_{(i)}^{\prime}(x^\prime), y^\prime )   -  \error(\wh f (x^\prime), y^\prime ) \right)  \error(f_{(i)}(x_j), y_j )   } \\
    %     & \quad  + \Expt{ S^\prime, (x,y), (x^\prime,y^\prime) }{  \left( \error(f^{\prime\prime}_{(j)}(x), y )  -\error(\wh f (x), y ) \right) \error(\wh f^{\prime\prime}  (x_j), y_j )  }  \\
    %     & \quad+ \Expt{ S^\prime, (x,y), (x^\prime,y^\prime) }{  \left( \error(\wh f^{\prime\prime}  (x^\prime), y^\prime )  -  \error(f^\prime_{(j)}(x^\prime), y^\prime ) \right)  \error(\wh f^\prime (x^\prime), y^\prime ) }  \,, \numberthis \label{eq:term2_final}
    % \end{align*}
    % where $f^{\prime\prime}_{(j)}$ is trained on $S^\prime_{(j,i)} \cup {(x,y)}$, $f^{\prime}_{(i)}$ is trained on $S^\prime_{(j,i)} \cup {(x^\prime,y^\prime)}$, and $\wh f^{\prime\prime} $ is trained on $S^\prime_{(j)} \cup {(x,y)}$. Note in the last line we replaced $(x,y)$ by $(x_j, y_j)$ in the first term, replaced $(x^\prime,y^\prime)$ by $(x_j, y_j)$ in the second term and exchanged $(x_i,y_i)$ with $(x_j,y_j)$ and also $(x,y)$ and $(x^\prime, y^\prime)$
    
    
\end{proof}


% 
% 16th Century Version Control 
% 

% \onecolumn

% \section*{Supplementary Material}
% We will be using the following standard results
% on exponential concentration of random variables 
% all throughout the discussion:

% \begin{lemma}[Hoeffding's inequality for independent RVs~\citep{hoeffding1994probability}] Let $Z_1, Z_2, \ldots, Z_n$ be independent bounded random variables with $Z_i \in [a,b]$ for all $i$, then 
%     \begin{align*}
%         \prob\left( \frac{1}{n} \sum_{i=1}^n (Z_i - \Expo{Z_i}) \ge t \right) \le \exp{\left( -\frac{2nt^2}{(b-a)^2} \right) }
%     \end{align*} 
%     and 
%     \begin{align*}
%         \prob\left( \frac{1}{n} \sum_{i=1}^n (Z_i - \Expo{Z_i}) \le -t \right) \le \exp{\left( -\frac{2nt^2}{(b-a)^2} \right) }
%     \end{align*} 
%     for all $t \ge 0$. 
% \end{lemma}

% \begin{lemma}[Hoeffding's inequality for sampling with replacement~\citep{hoeffding1994probability}] \label{lem:hoeffding_sampling} Let $\calZ = (Z_1, Z_2, \ldots, Z_N)$ be a finite population of $N$ points with $Z_i \in [a.b]$ for all $i$. Let $X_1, X_2, \ldots X_n$ be a random sample drawn without replacement from $\calZ$. Then for all $t \ge 0$, we have 
%     \begin{align*}
%         \prob\left( \frac{1}{n} \sum_{i=1}^n (X_i - \mu ) \ge t \right) \le \exp{\left( -\frac{2nt^2}{(b-a)^2} \right) }
%     \end{align*} 
%     and 
%     \begin{align*}
%         \prob\left( \frac{1}{n} \sum_{i=1}^n (X_i - \mu ) \le -t \right) \le \exp{\left( -\frac{2nt^2}{(b-a)^2} \right) } \,,
%     \end{align*} 
%     where $\mu = \frac{1}{N} \sum_{i=1}^{N} Z_i$. 
% \end{lemma}

% We now discuss one condition that generalizes the exponential concentration to dependent random variables.
% \begin{condition}[Bounded difference inequality] \label{cond:BDC} Let $\calZ$ be some set and $\phi: \calZ^n \to \Real$. We say that $\phi$ satisfies the bounded difference assumption if 
% there exists $c_1, c_2, \ldots c_n \ge 0$ s.t. for all $i$, we have 
% \begin{align*}
%     \sup_{Z_1,Z_2, \ldots,Z_n, Z_i^\prime in \calZ^{n+1} } \abs{\phi (Z_1, \ldots, Z_i, \ldots, Z_n ) - \phi (Z_1, \ldots, Z_i^\prime, \ldots, Z_n ) } \le c_i \,.
% \end{align*} 
% \end{condition}

% \begin{lemma}[McDiarmid’s inequality~\citep{mcdiarmid1989}] \label{lem:McDiarmid} Let $Z_1, Z_2, \ldots, Z_n$ be independent random variables on set $\calZ$ and $\phi : \calZ^n \to \Real$ satisfy bounded difference assumption (\codref{cond:BDC}). Then for all $t>0$, we have 
%     \begin{align*}
%         \prob\left( \phi(Z_1, Z_2, \ldots, Z_n) - \Expo{\phi(Z_1, Z_2, \ldots, Z_n)} \ge t \right) \le \exp{\left( -\frac{2t^2}{\sum_{i=1}^n c_i^2} \right) } 
%     \end{align*} 
%     and 
%     \begin{align*}
%         \prob\left( \phi(Z_1, Z_2, \ldots, Z_n) - \Expo{\phi(Z_1, Z_2, \ldots, Z_n)} \le -t \right) \le \exp{\left( -\frac{2t^2}{\sum_{i=1}^n c_i^2} \right) } \,
%     \end{align*} 
% \end{lemma}


% \section{Proofs from \secref{sec:ERM_training}}\label{app:proof_erm}

% \textbf{Additional notation {} {}} Let $m_1$ be the number of mislabeled points ($\wt S_M$) and $m_2$ be the number of correctly labeled points ($\wt S_C$). Note $m_1 + m_2 = m$. 


% \subsection{Proof of \thmref{thm:error_ERM}}


% \begin{proof}[Proof of \lemref{lem:fit_mislabeled}] 
%     The main idea of our proof is to regard 
%     the clean portion of the data 
%     ($S \cup \wt S_C$) as fixed.   
%     Then, there exists a classifier $f^*$ 
%     that is optimal over draws 
%     of the mislabeled data $\wt S_M$. 
% % 
%     % 
%     Formally, 
%     \begin{align}
%     f^* \defeq \argmin_{f \in \calF} \error_{\widecheck {\calD}} (f) \,, \label{eq:modified_ERM}
%     \end{align}
%     where $$\widecheck \calD = \frac{n}{m+n} \calS + \frac{m_1}{m+n} \wt \calS_C  + \frac{m_2}{m+n}\calDm \,.$$ That is, $\widecheck \calD$ a combination of 
%     the \emph{empirical distribution} 
%     over correctly labeled data $S \cup \wt S_C$
%     % in $S\cup \wt S$ 
%     and the (population) distribution 
%     over mislabeled data $\calDm$.
%     Recall that 
%     \begin{align}
%     \wh f \defeq \argmin_{f \in \calF} \error_{\calS \cup \wt S} (f) \,. \label{eq:orig_ERM}
%     \end{align}
%     % 
%     % 
%     Since, $\widehat f$ minimizes 0-1 error 
%     on $S \cup \wt S$, using ERM optimality on \eqref{eq:orig_ERM},  
%     we have 
%     \begin{align}
%         \error_{\calS \cup \wt \calS}(\widehat f) \le \error_{
%             \calS \cup \wt \calS}(f^*) \,.    \label{eq:step1}
%     \end{align}
%     Moreover, since $f^*$ is independent of $\wt S_M$, using Hoeffding's bound,
%     % \footnote{For a fully rigorous argument,
%     % refer to the complete proof in App.~\ref{app:proof_erm}.} 
%     we have with probability at least $1-\delta$ that
%     \begin{align}
%       \error_{\wt \calS_M}(f^*) \le \error_{ \calDm}(f^*) +  \sqrt{\frac{\log(1/\delta)}{2 m_1}} \,. \label{eq:step2} 
%     \end{align}
%     %$ 
%     %for some constant $c_1\le 1/2$. 
%     Finally, since $f^*$ is the optimal classifier on $\widecheck \calD$, 
%     we have 
%     \begin{align}
%         \error_{\widecheck \calD}(f^*) \le \error_{\widecheck \calD}(\widehat f) \label{eq:step3}
%     \end{align}
%      Now to relate \eqref{eq:step1} and \eqref{eq:step3}, we can re-write the \eqref{eq:step2} as follows: 
%     \begin{align}
%         \error_{\calS \cup \wt\calS}(f^*) \le \error_{ \widecheck \calD}(f^*) +  \frac{m_1}{m+n}\sqrt{\frac{\log(1/\delta)}{2 m_1}} \,. \label{eq:step4} 
%     \end{align}
%     Now we combine equations \eqref{eq:step1}, \eqref{eq:step4}, and \eqref{eq:step3}, to get 
%     \begin{align}
%         \error_{\calS \cup \wt \calS}(\wh f) \le \error_{\widecheck \calD}(\wh f) +  \frac{m_1}{m+n}\sqrt{\frac{\log(1/\delta)}{2 m_1}} \,, 
%     \end{align}
%     which implies 
%     \begin{align}
%         \error_{ \wt \calS_M}(\wh f) \le \error_{\calDm}(\wh f) + \sqrt{\frac{\log(1/\delta)}{2 m_1}} \,. \label{eq:lemma1_final}
%     \end{align}
%     Since $\wt S$ is obtained by randomly labeling an unlabeled dataset, we assume $2m_1 \approx m$ \footnote{Formally, with probability at least $1-\delta$, we have  $(m - 2m_1)\le \sqrt{m\log(1/\delta)/2}$ }. Moreover, using $\error_{\calDm} = 1 - \error_{\calD}$ we obtain the desired result.   
%     % Combining the above steps and using the fact 
%     % that $\error_\calD = 1- \error_{\calDm} $, 
%     % we obtain the desired result.
% \end{proof}

% \begin{proof}[Proof of \lemref{lem:mislabeled_error}]
%     Recall $\error_{\wt S} (f) = \frac{m_1}{m} \error_{\wt S_M}(f) + \frac{m_2}{m} \error_{\wt S_C}(f)$. Hence, we have 
%     \begin{align}
%         2\error_{\wt S}(f) - \error_{\wt S_M}(f) - \error_{\wt S_C}(f) &= \left(\frac{2m_1}{m} \error_{\wt S_M}(f) - \error_{\wt S_M}(f)\right) + \left(\frac{2m_2}{m} \error_{\wt S_C}(f) - \error_{\wt S_C}(f)\right) \\ &= \left(\frac{2m_1}{m} - 1\right) \error_{\wt S_M}(f) + \left(\frac{2m_2}{m} - 1 \right)\error_{\wt S_C} (f) \,.
%     \end{align} 
%     Since the dataset is randomly labeled, with probability at least $1-\delta$, we have  $\left(\frac{2m_1}{m} - 1\right) \le \sqrt{\frac{\log(1/\delta)}{2m}}$. Similarly, we have with probability at least $1-\delta$, $\left(\frac{2m_2}{m} - 1\right) \le \sqrt{\frac{\log(1/\delta)}{2m}}$. Using union bound, we have with probability at least $1-\delta$
%     % \begin{align}
%     %     2\error_{\wt S} - \error_{\wt S_M}(f) - \error_{\wt S_C}(f) \le \sqrt{\frac{\log(2/\delta)}{2m}} \left(\error_{\wt S_M}(f) + \error_{\wt S_C}(f) \right) \le 2\sqrt{\frac{\log(2/\delta)}{2m}} \,. \label{eq:lemma2_final}
%     % \end{align}
%     \begin{align}
%         2\error_{\wt S} - \error_{\wt S_M}(f) - \error_{\wt S_C}(f) \le \sqrt{\frac{\log(2/\delta)}{2m}} \left(\error_{\wt S_M}(f) + \error_{\wt S_C}(f) \right) \,. \label{eq:lemma2_prefinal}
%     \end{align}
%     With re-arranging $\error_{\wt S_M}(f) + \error_{\wt S_C}(f)$ and using the inequality $ 1- a\le \frac{1}{1+a} $, we have  
%     \begin{align}
%         2\error_{\wt S} - \error_{\wt S_M}(f) - \error_{\wt S_C}(f) \le 2\error_{\wt \calS} \sqrt{\frac{\log(2/\delta)}{2m}}  \,. \label{eq:lemma2_final}
%     \end{align}

%     % We obtain the desired result by using 
% \end{proof}

% \begin{proof}[Proof of \lemref{lem:clear_error}]
% % Recall 0-1 error on each point  $(x,y) \in S \cup \wt S$ is given by $\I{ f(x)\ne y}$.
% In the set of correctly labeled points $S \cup \wt S_C$, we have $S$ as a random subset of $S \cup \wt S_C$. Hence, using Hoeffding's inequality for sampling without replacement (\lemref{lem:hoeffding_sampling}), we have with probability at least $1-\delta$
% \begin{align}
%     \error_{\wt \calS_c} (\wh f)- \error_{\calS \cup \wt \calS_C}( \wh f) \le  \sqrt{\frac{\log(1/\delta)}{2m_2}} \,.
% \end{align}
% Re-writing $\error_{\calS \cup \wt \calS_C}( \wh f)$ as $\frac{m_2}{m_2 + n} \error_{\wt \calS_C }(\wh f) + \frac{n}{m_2 + n} \error_{\calS }(\wh f)$, we have with probability at least $1-\delta$
% \begin{align}
%   \left(\frac{n}{n+m_2}\right) \left(\error_{\wt \calS_c} (\wh f)- \error_{\calS}( \wh f) \right) \le  \sqrt{\frac{\log(1/\delta)}{2m_2}} \,.
% \end{align}
% As before, assuming $2m_2 \approx m$, we have with probability at least $1-\delta$ 
% \begin{align}
%     \error_{\wt \calS_c} (\wh f)- \error_{\calS}( \wh f) \le \left(1+\frac{m_2}{n}\right)  \sqrt{\frac{\log(1/\delta)}{m}} \le 1.5 \sqrt{\frac{\log(1/\delta)}{m}} \,. \label{eq:lemma3_final}
% \end{align} 
% \end{proof}

% \begin{proof}[Proof of \thmref{thm:error_ERM}] 
%     Having established these core intermediate results, we can now combine above three lemmas to prove the main result. 
%     In particular, we bound the population error on clean data ($\error_\calD(\wh f)$) as follows:  
%     \begin{enumerate}[(i)]
%         \item First, use \eqref{eq:lemma1_final}, to obtain an upper bound on the population error on clean data, i.e., with probability at least $1-\delta/4$, we have
%         \begin{align}
%             \error_{ \calD} (\wh f) \le 1 - \error_{ \wt \calS_M}(\wh f) + \sqrt{\frac{\log(4/\delta)}{m}} \,. 
%         \end{align}
%         \item  Second, use \eqref{eq:lemma2_final}, to relate the error on the mislabeled fraction with error on clean portion of randomly labeled data and error on whole randomly labeled dataset, i.e., with probability at least $1-\delta/2$, we have 
%         \begin{align}
%             - \error_{\wt S_M}(f) \le \error_{\wt S_C}(f) - 2\error_{\wt S}  + \sqrt{\frac{\log(4/\delta)}{2m}}  \,. 
%         \end{align} 
%         \item Finally, use \eqref{eq:lemma3_final} to relate the error on the clean portion of randomly labeled data and error on clean training data, i.e., with probability $1-\delta/4$, we have 
%         \begin{align}
%             \error_{\wt \calS_C} (\wh f)\le - \error_{\calS}( \wh f) + \left(1 + \frac{m}{2n} \right) \sqrt{\frac{\log(4/\delta)}{m}} \,. 
%         \end{align} 
%     \end{enumerate}

%     Using union bound on the above three steps, we have with probability at least $1-\delta$: 
%     \begin{align}
%         \error_\calD (\wh f) \le \error_{\calS}(\wh f)   + 1 - 2\error_{\wt \calS}(\wh f)   + (1/\sqrt{2} + 2.5)  \sqrt{\frac{\log(4/\delta)}{m}} \,.
%     \end{align}
%     Note that $(1/\sqrt{2} + 2.5)$ is a loose constant. In experiments, we use the ratio $\frac{m}{n}$
%     %  the exact error $\error_{\wt \calS}(\wh f)$ 
%     to evaluate R.H.S.    
% \end{proof}

% \subsection{Proof of \propref{prop:rademacher}}

% \begin{proof}[Proof of \propref{prop:rademacher}]
%     For a classifier $ f: \calX \to \{-1, 1\}$, we have $1 - 2\,\indict{ f(x) \ne y} = y \cdot f(x)$. Hence, by definition of $\error$, we have 
%     \begin{align}
%         1 -2\error_{\wt \calS}(f) = \frac{1}{m}\sum_{i=1}^m y_i \cdot f(x_i) \le \sup_{f \in \calF} \, \frac{1}{m} \sum_{i=1}^m y_i \cdot f(x_i)  \,. \label{eq:error_rademacher}
%     \end{align}
%     Note that for fixed inputs $(x_1, x_2, \ldots, x_m)$ in $\wt S$, $(y_1, y_2, \ldots y_m)$ are random labels. Define $\phi_1 (y_1, y_2, \ldots, y_m) \defeq \sup_{f \in \calF} \, \frac{1}{m} \sum_{i=1}^m y_i \cdot f(x_i)$. We have the following bounded difference condition on $\phi_1$. For all i, 
%     \begin{align}
%         \sup_{y_1, \ldots y_m, y_i^\prime \in \{-1, 1\}^{m+1} } \abs{ \phi_1 (y_1,\ldots, y_i, \ldots, y_m) - \phi_1 (y_1,\ldots, y_i^\prime, \ldots, y_m)  } \le 1/m \,. \label{cond1_rademacher}
%     \end{align} 
    
%     Similarly define $\phi_2 (x_1, x_2, \ldots, x_m) \defeq \Expt{ y_i \sim_U \{-1, 1\}  }{ \sup_{f \in \calF} \, \frac{1}{m}  \sum_{i=1}^m y_i \cdot f(x_i)}$. We have the following bounded difference condition on $\phi_2$. For all i,
%     \begin{align}
%         \sup_{x_1, \ldots x_m, x_i^\prime \in \calX^{m+1} } \abs{ \phi_2 (x_1,\ldots, x_i, \ldots, x_m) - \phi_1 (x_1,\ldots, x_i^\prime, \ldots, x_m)  } \le 1/m \,. \label{cond2_rademacher}
%     \end{align}
%     Using McDiarmid’s inequality (\lemref{lem:McDiarmid}) twice with Condition \eqref{cond1_rademacher} and \eqref{cond2_rademacher}, with probability at least $1-\delta$, we have
%     \begin{align}
%         \sup_{f \in \calF} \, \frac{1}{m} \sum_{i=1}^m y_i \cdot f(x_i)  - \Expt{x,y}{\sup_{f \in \calF} \, \frac{1}{m} \sum_{i=1}^m y_i \cdot f(x_i) } \le \sqrt{\frac{2\log(2/\delta)}{m}} \label{eq:final_rademacher}
%     \end{align} 
%     Combining \eqref{eq:error_rademacher} and \eqref{eq:final_rademacher}, we obtain the desired result. 
% \end{proof}


% \subsection{Proof of \thmref{thm:error_regularized_ERM}}

% Proof of \thmref{thm:error_regularized_ERM} follows similar to the proof of \thmref{thm:error_ERM}. Note that the same results in \lemref{lem:fit_mislabeled}, \lemref{lem:mislabeled_error}, and \lemref{lem:clear_error} hold in the regularized ERM case. However, the arguments in the proof of \lemref{lem:fit_mislabeled} changes slightly. Hence, we state and prove a lemma parallel to \lemref{lem:fit_mislabeled} for completeness. 

% \begin{lemma} \label{lem:lemma1_reg}
%     Assume the same setup as \thmref{thm:error_regularized_ERM}. 
%     Then for any $\delta >0$, with probability at least  $1-\delta$ 
%     over the random draws of mislabeled data $\wt S_M$, we have 
%     \begin{align}
%         \error_\calD(\widehat f)  \le 1 -\error_{\wt \calS_M}(\widehat f) + \sqrt{\frac{\log(1/\delta)}{m}}\,. 
%     \end{align} 
% \end{lemma}
% \begin{proof}
%     The main idea of the proof remains the same, i.e. regard 
%     the clean portion of the data 
%     ($S \cup \wt S_C$) as fixed.   
%     Then, there exists a classifier $f^*$ 
%     that is optimal over draws 
%     of the mislabeled data $\wt S_M$. 

    
%     Formally, 
%     \begin{align}
%     f^* \defeq \argmin_{f \in \calF} \error_{\widecheck {\calD}} (f)  + \lambda R(f) \,, \label{eq:modified_ERM_reg}
%     \end{align}
%     where $$\widecheck \calD = \frac{n}{m+n} \calS + \frac{m_1}{m+n} \wt \calS_C  + \frac{m_2}{m+n}\calDm \,.$$ That is, $\widecheck \calD$ a combination of 
%     the \emph{empirical distribution} 
%     over correctly labeled data $S \cup \wt S_C$
%     % in $S\cup \wt S$ 
%     and the (population) distribution 
%     over mislabeled data $\calDm$.
%     Recall that 
%     \begin{align}
%     \wh f \defeq \argmin_{f \in \calF} \error_{\calS \cup \wt S} (f) + \lambda R(f) \,. \label{eq:orig_ERM_reg}
%     \end{align}
%     % 
%     % 
%     Since, $\widehat f$ minimizes 0-1 error 
%     on $S \cup \wt S$, using ERM optimality on \eqref{eq:orig_ERM},  
%     we have 
%     \begin{align}
%         \error_{\calS \cup \wt \calS}(\widehat f) + \lambda R(\wh f) \le \error_{
%             \calS \cup \wt \calS}(f^*) + \lambda R(f^*) \,.    \label{eq:step1_reg}
%     \end{align}
%     Moreover, since $f^*$ is independent of $\wt S_M$, using Hoeffding's bound,
%     % \footnote{For a fully rigorous argument,
%     % refer to the complete proof in App.~\ref{app:proof_erm}.} 
%     we have with probability at least $1-\delta$ that
%     \begin{align}
%       \error_{\wt \calS_M}(f^*) \le \error_{ \calDm}(f^*) +  \sqrt{\frac{\log(1/\delta)}{2 m_1}} \,. \label{eq:step2_reg} 
%     \end{align}
%     %$ 
%     %for some constant $c_1\le 1/2$. 
%     Finally, since $f^*$ is the optimal classifier on $\widecheck \calD$, 
%     we have 
%     \begin{align}
%         \error_{\widecheck \calD}(f^*) + \lambda R(f^*) \le \error_{\widecheck \calD}(\widehat f) + \lambda R(\wh f) \label{eq:step3_reg}
%     \end{align}
%      Now to relate \eqref{eq:step1_reg} and \eqref{eq:step3_reg}, we can re-write the \eqref{eq:step2_reg} as follows: 
%     \begin{align}
%         \error_{\calS \cup \wt\calS}(f^*) \le \error_{ \widecheck \calD}(f^*) +  \frac{m_1}{m+n}\sqrt{\frac{\log(1/\delta)}{2 m_1}} \,. \label{eq:step4_reg} 
%     \end{align}
%     After adding $\lambda R(f^*)$ on both sides in \eqref{eq:step4_reg}, we combine equations \eqref{eq:step1_reg}, \eqref{eq:step4_reg}, and \eqref{eq:step3_reg}, to get 
%     \begin{align}
%         \error_{\calS \cup \wt \calS}(\wh f) \le \error_{\widecheck \calD}(\wh f) +  \frac{m_1}{m+n}\sqrt{\frac{\log(1/\delta)}{2 m_1}} \,, 
%     \end{align}
%     which implies 
%     \begin{align}
%         \error_{ \wt \calS_M}(\wh f) \le \error_{\calDm}(\wh f) + \sqrt{\frac{\log(1/\delta)}{2 m_1}} \,. \label{eq:lemma_reg_final}
%     \end{align}
%     Similar as before, since $\wt S$ is obtained by randomly labeling an unlabeled dataset, we assume 
%     $2m_1 \approx m$. Moreover, using $\error_{\calDm} = 1 - \error_{\calD}$ we obtain the desired result. 
% \end{proof}
% % \begin{proof}[Proof of ]
    
% % \end{proof}

% \subsection{Proof of \thmref{thm:multiclass_ERM}}

% We first state and prove lemmas parallel to three lemmas used in the proof of balanced binary case. Then we combine the results in the three lemmas to obtain the result in \thmref{thm:multiclass_ERM}. 

% Before stating the result, we define mislabeled distribution $\calDm$ for any $\calD$. While $\calDm$ and $\calD$ share 
% the same marginal distribution over $\calX$, 
% the distribution over labels $y$ 
% given an input $x\sim \calD_\calX$ is changed.
% In particular, for any $x$, the pdf over $y$ is changed to:  
% $p_{\calDm} (\cdot \vert x) \defeq \frac{1 - p_{\calD}(\cdot \vert x)}{k - 1}$.

% \begin{lemma} \label{lem:fit_mislabeled_multi}
%     Assume the same setup as \thmref{thm:multiclass_ERM}. 
%     Then for any $\delta >0$, with probability at least  $1-\delta$ 
%     over the random draws of mislabeled data $\wt S_M$, we have 
%     \begin{align}
%         \error_\calD(\widehat f)  \le (k-1)\left(1 -\error_{\wt \calS_M}(\widehat f)\right) + (k-1)\sqrt{\frac{\log(1/\delta)}{m}}\,. \label{eq:lemma1_multi}
%     \end{align}   
% \end{lemma} 

% \begin{proof}
%     The main idea of the proof remains the same, i.e. regard 
%     the clean portion of the data 
%     ($S \cup \wt S_C$) as fixed. 
%     Then, there exists a classifier $f^*$ 
%     that is optimal over draws 
%     of the mislabeled data $\wt S_M$. 
    
%     However, we need to be careful while relating population error on mislabeled data with population accuracy on clean data.   
%     While for binary classification,  we could upper bound $\error_{\wt \calS_M}$ 
%     with $1-\error_\calD$  (in the proof of \lemref{lem:fit_mislabeled}), 
%     for multiclass classification, 
%     error on the mislabeled data 
%     and accuracy on the clean data 
%     in the population 
%     are not so directly related.  
%     To establish \eqref{eq:lemma1_multi},
%     we break the error on the 
%     (unknown) mislabeled data 
%     into two parts: one term corresponds 
%     to predicting the true label on mislabeled data, 
%     and the other corresponds to predicting 
%     neither the true label 
%     nor the assigned (mis-)label.  
%     Finally, we relate these errors to their
%     population counterparts to establish \eqref{eq:lemma1_multi}. 
    
%     Formally, 
%     \begin{align}
%     f^* \defeq \argmin_{f \in \calF} \error_{\widecheck {\calD}} (f)  + \lambda R(f) \,, \label{eq:modified_ERM_reg2}
%     \end{align}
%     where $$\widecheck \calD = \frac{n}{m+n} \calS + \frac{m_1}{m+n} \wt \calS_C  + \frac{m_2}{m+n}\calDm \,.$$ That is, $\widecheck \calD$ a combination of 
%     the \emph{empirical distribution} 
%     over correctly labeled data $S \cup \wt S_C$
%     % in $S\cup \wt S$ 
%     and the (population) distribution 
%     over mislabeled data $\calDm$.
%     Recall that 
%     \begin{align}
%     \wh f \defeq \argmin_{f \in \calF} \error_{\calS \cup \wt S} (f) + \lambda R(f) \,. \label{eq:orig_ERM_reg2}
%     \end{align}
%     % 
%     % 
%     Following the exact steps from the proof of \lemref{lem:lemma1_reg}, with probability at least $1-\delta$, we have  
%     \begin{align}
%         \error_{ \wt \calS_M}(\wh f) \le \error_{\calDm}(\wh f) + \sqrt{\frac{\log(1/\delta)}{2 m_1}} \,. \label{eq:lemma1_final_multi_prev}
%     \end{align}
%     Similar to before, since $\wt S$ is obtained by randomly labeling an unlabeled dataset, we assume 
%     $\frac{k}{k-1} m_1 \approx m$. 
    
%     Now we will relate $\error_\calDm (\wh f)$ with $\error_{\calD}(\wh f)$. Let $y^T$ denote the (unknown) true label for a mislabeled point $(x, y)$ (i.e., label before replacing it with a mislabel). 
%     \begin{align}    
%          \Expt{(x, y) \in \sim \calDm}{\indict{ \wh f(x) \ne y }}  &= \underbrace{\Expt{(x, y) \in \sim \calDm}{\indict{ \wh f(x) \ne y \land \wh f(x) \ne y^T}}}_{\RN{1}} + \underbrace{\Expt{(x, y) \in \sim \calDm}{\indict{ \wh f(x) \ne y \land \wh f(x) = y^T}}}_{\RN{2}} \,. \label{eq:excess_term}
%     \end{align}
%     Clearly, term 2 is one minus the accuracy on the clean unseen data, i.e. 
%     \begin{align}
%         \RN{2} = 1 - \Expt{{x,y} \sim \calD}{ \indict{ \wh f(x) \ne y}} = 1- \error_{\calD}(\wh f) \,. \label{eq:term1}    
%     \end{align}
%     Next, we  relate term 1 with the error on the unseen clean data. We show that term 1 is equal to the error on the unseen clean data scaled by $\frac{k-2}{k-1}$ where $k$ is the number of labels. Using the definition of mislabeled distribution $\calDm$,  we have 
%     \begin{align}
%         \RN{1} = \frac{1}{k-1} \left( \Expt{(x, y) \in \sim \calD}{ \sum_{i \in \calY \land i\ne y}  \indict{ \wh f(x) \ne i \land \wh f(x) \ne y}} \right) = \frac{k-2}{k-1} \error_{\calD}(\wh f) \,.\label{eq:term2}
%     \end{align}    

%     Combining the result in \eqref{eq:term1}, \eqref{eq:term2} and \eqref{eq:excess_term}, we have 
%     \begin{align}
%         \error_{\calDm}(\wh f) = 1- \frac{1}{k-1} \error_{\calD}(\wh f) \,.\label{eq:combine_terms}
%     \end{align}
%     Finally, combining the result in \eqref{eq:combine_terms} with equation \eqref{eq:lemma1_final_multi_prev}, we have with probability $1-\delta$, 
%     \begin{align}
%       \error_{\calD}(\wh f) \le  (k-1) \left( 1- \error_{ \wt \calS_M}(\wh f) \right)  + (k-1) \sqrt{\frac{k \log(1/\delta)}{ 2(k-1)m}} \,. \label{eq:lemma1_final_multi}
%     \end{align}
% \end{proof}

% \begin{lemma} \label{lem:mislabeled_error_multi}
%     Assume the same setup as \thmref{thm:multiclass_ERM}.  Then for any $\delta >0$, with probability at least $1-\delta$ over the random draws of $\wt S$, we have  
%     % \begin{align}
%         $$\abs{k\error_{\wt \calS}(\widehat f) - \error_{\wt \calS_C}(\widehat f) -  (k-1)\error_{\wt \calS_M}(\widehat f) } \le  2k\sqrt{\frac{\log(4/\delta)}{2m}}\,. $$ % \label{eq:lemma2}
%     % \end{align}   
%     %  for some constant $c_3 \le 1.0\,$.
% \end{lemma} 


% \begin{proof}
%     Recall $\error_{\wt S} (f) = \frac{m_1}{m} \error_{\wt S_M}(f) + \frac{m_2}{m} \error_{\wt S_C}(f)$. Hence, we have 
%     \begin{align}
%         k\error_{\wt S}(f) - (k-1)\error_{\wt S_M}(f) - \error_{\wt S_C}(f) &= (k-1)\left(\frac{k m_1}{(k-1) m} \error_{\wt S_M}(f) - \error_{\wt S_M}(f)\right) + \left(\frac{km_2}{m} \error_{\wt S_C}(f) - \error_{\wt S_C}(f)\right) \\ &= k \left[ \left(\frac{m_1}{m} - \frac{k-1}{k}\right) \error_{\wt S_M}(f) + \left(\frac{m_2}{m} - \frac{1}{k} \right) \error_{\wt S_C} (f) \right] \,.
%     \end{align} 
%     Since the dataset is randomly labeled, we have with probability at least $1-\delta$, $\left(\frac{m_1}{m} - \frac{k-1}{k}\right) \le \sqrt{\frac{\log(1/\delta)}{2m}}$. Similarly, we have with probability at least $1-\delta$, $\left(\frac{m_2}{m} - \frac{1}{k}\right) \le \sqrt{\frac{\log(1/\delta)}{2m}}$. Using union bound, we have with probability at least $1-\delta$
%     % \begin{align}
%     %     2\error_{\wt S} - \error_{\wt S_M}(f) - \error_{\wt S_C}(f) \le \sqrt{\frac{\log(2/\delta)}{2m}} \left(\error_{\wt S_M}(f) + \error_{\wt S_C}(f) \right) \le 2\sqrt{\frac{\log(2/\delta)}{2m}} \,. \label{eq:lemma2_final}
%     % \end{align}
%     \begin{align}
%         k\error_{\wt S}(f) - (k-1)\error_{\wt S_M}(f) - \error_{\wt S_C}(f)  \le k \sqrt{\frac{\log(2/\delta)}{2m}} \left(\error_{\wt S_M}(f) + \error_{\wt S_C}(f) \right) \,. \label{eq:lemma2_final_multi}
%     \end{align}

%     % We obtain the desired result by using 
% \end{proof}

% \begin{lemma} \label{lem:clear_error_multi}
%     Assume the same setup as \thmref{thm:multiclass_ERM}. 
%     Then for any $\delta >0$, with probability at least $1-\delta$ 
%     over the random draws of $\wt S_C$ and $S$, we have 
%     % \begin{align}
%         $$\abs{\error_{\wt \calS_C}(\widehat f) - \error_{\calS}(\widehat f) } \le 1.5 \sqrt{\frac{k\log(2/\delta)}{2m}}\,.$$ %\label{eq:lemma3}
%     % \end{align}   
%     % for some constant $c_2 \le 1.2\,$.
% \end{lemma} 
% \begin{proof}
%     % Recall 0-1 error on each point  $(x,y) \in S \cup \wt S$ is given by $\I{ f(x)\ne y}$.
%     In the set of correctly labeled points $S \cup \wt S_C$, we have $S$ as a random subset of $S \cup \wt S_C$. Hence, using Hoeffding's inequality for sampling without replacement (\lemref{lem:hoeffding_sampling}), we have with probability at least $1-\delta$
%     \begin{align}
%         \error_{\wt \calS_c} (\wh f)- \error_{\calS \cup \wt \calS_C}( \wh f) \le  \sqrt{\frac{\log(1/\delta)}{2m_2}} \,.
%     \end{align}
%     Re-writing $\error_{\calS \cup \wt \calS_C}( \wh f)$ as $\frac{m_2}{m_2 + n} \error_{\wt \calS_C }(\wh f) + \frac{n}{m_2 + n} \error_{\calS }(\wh f)$, we have with probability at least $1-\delta$
%     \begin{align}
%       \left(\frac{n}{n+m_2}\right) \left(\error_{\wt \calS_c} (\wh f)- \error_{\calS}( \wh f) \right) \le  \sqrt{\frac{\log(1/\delta)}{2m_2}} \,.
%     \end{align}
%     As before, assuming $km_2 \approx m$, we have with probability at least $1-\delta$ 
%     \begin{align}
%         \error_{\wt \calS_c} (\wh f)- \error_{\calS}( \wh f) \le \left(1+\frac{m_2}{n}\right)  \sqrt{\frac{k\log(1/\delta)}{2m}} \le \left( 1 + \frac{1}{k}\right) \sqrt{\frac{k\log(1/\delta)}{2m}} \,. \label{eq:lemma3_final_multi}
%     \end{align} 
% \end{proof}

% \begin{proof}[Proof of \thmref{thm:multiclass_ERM}] 
%     Having established these core intermediate results, we can now combine above three lemmas. 
%     In particular, we bound the population error on clean data ($\error_\calD(\wh f)$) as follows:  
%     \begin{enumerate}[(i)]
%         \item First, use \eqref{eq:lemma1_final_multi}, to obtain an upper bound on the population error on clean data, i.e., with probability at least $1-\delta/4$, we have
%         \begin{align}
%             \error_{ \calD} (\wh f) \le (k-1)\left(1 - \error_{ \wt \calS_M}(\wh f) \right) + (k-1) \sqrt{\frac{k\log(4/\delta)}{2(k-1)m}} \,. 
%         \end{align}
%         \item  Second, use \eqref{eq:lemma2_final_multi}, to relate the error on the mislabeled fraction with error on clean portion of randomly labeled data and error on whole randomly labeled dataset, i.e., with probability at least $1-\delta/2$, we have 
%         \begin{align}
%             - (k-1)\error_{\wt S_M}(f) \le \error_{\wt S_C}(f) - k\error_{\wt S}  + k\sqrt{\frac{\log(4/\delta)}{2m}}  \,. 
%         \end{align} 
%         \item Finally, use \eqref{eq:lemma3_final_multi} to relate the error on the clean portion of randomly labeled data and error on clean training data, i.e., with probability $1-\delta/4$, we have 
%         \begin{align}
%             \error_{\wt \calS_C} (\wh f)\le - \error_{\calS}( \wh f) + \left(1 + \frac{m}{kn} \right) \sqrt{\frac{k\log(4/\delta)}{2m}} \,. 
%         \end{align} 
%     \end{enumerate}

%     Using union bound on the above three steps, we have with probability at least $1-\delta$: 
%     \begin{align}
%         \error_\calD (\wh f) \le \error_{\calS}(\wh f) + (k-1) - k\error_{\wt \calS}(\wh f)   + (\sqrt{k(k-1)} + k + \sqrt{k} + \frac{m}{n\sqrt{k}})  \sqrt{\frac{\log(4/\delta)}{2m}} \,.
%     \end{align}
%     % Note that $\frac{m}{n\sqrt{k}}$ is much smaller than the other terms in addition. Hence, we ignore this in the final bound. 
%     % Note that $(1/\sqrt{2} + 2.5)$ is a loose constant. In experiments, we use the ratio $\frac{m}{n}$
%     %  the exact error $\error_{\wt \calS}(\wh f)$ 
%     % to evaluate R.H.S.    
% \end{proof}

% \newpage
% \section{Proofs from \secref{sec:linear_models}}\label{app:proof_gd}

% We suppose that the parameters of the linear function 
% are obtained via gradient descent on 
% the following $L_2$ regularized problem: 
% \begin{align}
%     % n in denominator is avoided deliberately
%     \calL_S(w; \lambda) \defeq \sum_{i=1}^n{(w^Tx_i - y_i)^2} + \lambda \norm{w}{2}^2 \,, \label{eq:l2_MSE_app}   
% \end{align}
% where $\lambda\ge0$ is a regularization parameter. 
% We assume access to a clean dataset 
% $S = \{(x_i, y_i)\}_{i=1}^n \sim \calD^n$ 
% and randomly labeled dataset 
% $\wt S = \{(x_i, y_i)\}_{i=n+1}^{n+m} \sim \wt \calD^m$. 
% Let $\bX = [x_1, x_2, \cdots, x_{m+n}]$ 
% and $\by = [y_1, y_2, \cdots, y_{m+n}]$. 
% Fix a positive learning rate $\eta$ such that 
% $\eta \le 1/\left(\norm{\bX^T\bX}{\text{op}} + \lambda^2\right)$ 
% and an initialization $w_0 = 0$. 
% % \todos{Assumption made for simplicty}. 
% Consider the following gradient descent iterates 
% to minimize objective \eqref{eq:l2_MSE_app} on $S \cup \wt S$:
% \begin{align}
% w_t = w_{t-1} - \eta \grad_w \calL_{S \cup \wt S} (w_{t-1}; \lambda) \quad \forall t=1,2,\ldots \label{eq:GD_iterates_app}
% \end{align} 
% Then we have $\{ w_t\}$ converge to the limiting solution 
% $\wh w = \left( \bX^T\bX+\lambda \boldsymbol{I}\right)^{-1}\bX^T\by$. Define $\widehat f (x) \defeq f(x ; \wh w) $.  

% \subsection{\textcolor{red}{Errata}}

% We wish to correct the following error in the body: \codref{cond:error_stability} is not enough to guarantee the result in \thmref{thm:linear}. We now present a slightly stronger condition called \emph{hypothesis stability} under which we obtain a result similar to \thmref{thm:linear}. 

% This error doesn't change the main arguments of the proof where we show that the empirical train error is less than or equal to the leave-one-out error. We need a stronger condition to relate leave-one-out error with the population error of the original classifier. Specifically, while \codref{cond:error_stability} relates the average population error of leave-one-out classifiers with the population error of the original classifier, we need the new condition to show the concentration of the empirical leave-one-out error and  average population error of leave-one-out classifiers. 
% % main takeaway 

% Note that the new condition, while being stronger than the previous one, still doesn't imply generalization~\cite{bousquet2002stability,elisseeff2003leave,abou2019exponential}. Overall, the main results in \secref{sec:ERM_training} and takeaways of the paper remain unaffected by the error.  

% We now present the new condition and a corrected statement of \thmref{thm:linear}. Recall, for a given training set $S \sim \calD^n $, 
% we use $S_{(i)}$ to denote the training set $S$ 
% with the $i^{\text{th}}$ point removed.

% \begin{condition}[Hypothesis Stability] 
%     \label{cond:hypothesis_stability}
%     We have $\beta$ hypothesis stability 
%     if our training algorithm $\calA$ satisfies the following: 
%     \begin{align*}
%     % ${\sum_{i=1}^n \frac{\error_{\calD}( f(\calA, S_{(i)}))}{n} - \error_\calD(f(\calA, S))} \le \beta\,$.
%     \forall i \in \{1,2,\ldots, n\}, \quad  \Expt{\calS, (x,y) \in \calD}{ \abs{\error\left( f(x) ,y  \right) - \error\left( f_{(i)}(x), y \right) }} \le \frac{\beta}{n} \,,
%     \end{align*}
%     where $f_{(i)} \defeq f(\calA, S_{(i)})$ and $ f \defeq f(\calA, S)$.
% \end{condition}

% \begin{theorem}[Correct statement of \thmref{thm:linear}] \label{thm:new_linear}
%     Assume that this gradient descent algorithm satisfies \codref{cond:hypothesis_stability}
%     with $\beta=\calO(1)$.  
%     Then for any $\delta >0$, with probability at least $1-\delta$ 
%     over the random draws of datasets $\wt S$ and $S$, we have:
%     \begin{align}
%         \error_\calD(\widehat f) \le \error_\calS(\widehat f) + 1 - 2 \error_{\wt\calS}(\widehat f) + \left(\frac{1}{\sqrt{2}} + 1.5 \right) \sqrt{\frac{\log(4/\delta)}{m}} + \sqrt{\frac{4}{\delta}\left(\frac{1}{m} +\frac{3\beta}{m+n} \right)}  \,. \label{eq:gd_error}
%     \end{align} 
%     % for some constant $c\le 3.2$.
% \end{theorem}

% \subsection{Proof of \thmref{thm:new_linear}}
% We use a standard result from linear algebra, namely Shermann-Morrison formula~\citep{sherman1950adjustment} for matrix inversion:  

% \begin{lemma}[\citet{sherman1950adjustment}] \label{lem:sherman}
%     Suppose $\bA \in \Real^{n \times n}$ is an invertible square matrix and $u,v \in \Real^n$ are column vectors. Then $\bA + uv^T$ is invertible iff $1 + v^T \bA u \ne 0$ and in particular
%     \begin{align}
%         (\bA + u v^T)^{-1} = \bA^{-1}  - \frac{\bA^{-1} uv^T \bA^{-1} }{ 1 + v^T \bA^{-1} u} \,.
%     \end{align}   
% \end{lemma}
% \newcommand\byy[1]{\by_{\left(#1\right)}}
% \newcommand\bXX[1]{\bX_{\left(#1\right)}}
% \newcommand\ff[1]{\wh f_{\left(#1\right)}}

% For a given training set $S \cup \wt S_C$, define leave-one-out error on mislabeled points in the training data as $$\error_{\text{LOO}(\wt S_M) } = \frac{\sum_{(x_i, y_i) \in \wt S_M} \error( f_{(i)}( x_i), y_i)}{ \abs{\wt S_M }} \,, $$
% where $f_{(i)} \defeq f(\calA, (S \cup \wt S)_{(i)})$. To relate empirical leave-one-out error and population error with hypothesis stability condition, we use the following lemma:   

% \begin{lemma}[\citet{bousquet2002stability}] \label{lem:stability_error}
%     For the leave-one-out error, we have
%     \begin{align}
%         \Expo{ \left( \error_{\calDm}(\wh f) -\error_{\text{LOO}(\wt S_M) } \right)^2 } \le \frac{1}{2m_1}+  \frac{3\beta}{n + m}\,.
%     \end{align}   
%     % where $ f \defeq f(\calA, S \cup \wt S) $.
% \end{lemma}

% Proof of the above lemma is similar to the proof of  Lemma 9 in \citet{bousquet2002stability} and can be found in \appref{app:proof_lem_error}. 
% % 
% % Before presenting the result, we introduce some notation. 
% Before presenting the proof of \thmref{thm:new_linear}, we introduce some more notation. Let $\bX_{(i)}$ denote the matrix of covariates with $i^{\text{th}}$ point removed. Similarly let $\by_{(i)}$ be the array of responses with $i^{\text{th}}$ point removed. Define the corresponding regularized GD solution as $\wh w_{(i)} = \left( \bXX{i}^T\bXX{i}+\lambda \boldsymbol{I}\right)^{-1}\bXX{i}^T\byy{i}$. Define $\ff{i}(x) \defeq f(x ; \wh w_{(i)}) $.

% \begin{proof}[Proof of \thmref{thm:new_linear}]
%     Because squared loss minimization does not imply 0-1 error minimization, we cannot use arguments from \lemref{lem:fit_mislabeled}. This is the main technical difficulty. To compare the 0-1 error at a train point with an unseen point, 
%     we use the closed-form expression for $\widehat{w}$ and Shermann-Morrison formula to upper bound training error with leave-one-out cross validation error. 
    
%     The proof is divided into three parts: In part one, we show that 0-1 error on mislabeled points in the training set is lower than the error obtained by leave-one-out error at those points. In part two, we relate this leave-one-out error with the population error on mislabeled distribution using \codref{cond:hypothesis_stability}. While the empirical leave-one-out error is unbiased estimator of the average population error of leave-one-out classifiers, we need hypothesis stability to control the variance of empirical leave-one-out error. Finally in part three, we show that the error on the mislabeled training points can be estimated with just the randomly labeled and  clean training data (as in proof of \thmref{thm:error_ERM}).  

%     \textbf{Part 1 {} {}} First we relate training error with leave-one-out error.        
%     For any 
%     training point $(x_i, y_i)$ in $\wt S \cup S$, we have 
%     \begin{align}
%         \error(\wh f(x_i), y_i ) &= \indict{ y_i \cdot x_i^T \wh w < 0 } = \indict{ y_i \cdot x_i^T \left( \bX^T\bX+\lambda \boldsymbol{I}\right)^{-1}\bX^T\by < 0 } \\
%         &= \indict{ y_i \cdot x_i^T \underbrace{\left( \bXX{i}^T\bXX{i} + x_i ^T x_i +\lambda \boldsymbol{I}\right)^{-1}}_{\RN{1}} (\bXX{i}^T\byy{i} + y \cdot x_i) < 0 }
%     \end{align}
%     Letting $\bA = \left(\bXX{i}^T\bXX{i} +\lambda \boldsymbol{I}\right)$ and using \lemref{lem:sherman} on term 1, we have 
%     \begin{align}
%         \error(\wh f(x_i), y_i ) &= \indict{ y_i \cdot x_i^T \left[\bA^{-1} -  \frac{\bA^{-1} x_i x_i^T \bA^{-1}}{ 1 + x_i ^T \bA^{-1} x_i } \right] (\bXX{i}^T\byy{i} + y \cdot x_i) < 0 } \\
%         &= \indict{ y_i \cdot\left[ \frac{ x_i^T \bA^{-1} ( 1 + x_i ^T \bA^{-1} x_i ) -  x_i^T \bA^{-1} x_i x_i^T \bA^{-1}}{ 1 + x_i ^T \bA ^{-1}x_i } \right] (\bXX{i}^T\byy{i} + y \cdot x_i) < 0 } \\
%         &= \indict{ y_i \cdot\left[ \frac{ x_i^T \bA^{-1}}{ 1 + x_i ^T \bA ^{-1}x_i } \right] (\bXX{i}^T\byy{i} + y \cdot x_i) < 0 } \,.
%     \end{align}

%     Since $1 + x_i^T \bA^{-1} x_i > 0$, we have 
%     \begin{align}
%         \error(\wh f(x_i), y_i ) &= \indict{ y_i \cdot x_i^T \bA^{-1} (\bXX{i}^T\byy{i} + y \cdot x_i) < 0 } \\
%         &= \indict{ x_i^T \bA^{-1} x_i +  y_i \cdot x_i^T \bA^{-1} (\bXX{i}^T\byy{i}) < 0 } \\
%         &\le \indict{ y_i \cdot x_i^T \bA^{-1} (\bXX{i}^T\byy{i}) < 0 } = \error(\ff{i}(x_i), y_i ) \,.\label{eq:LOO_error}
%     \end{align}

%     Using \eqref{eq:LOO_error}, we have 
%     \begin{align}
%         \error_{\wt \calS_M } (\wh f) \le \error_{\text{LOO} (S_M)} \defeq \frac{\sum_{(x_i, y_i) \in \wt S_M} \error(\ff{i}(x_i), y_i ) }{\abs{\wt \calS_M}}\label{eq:LOO_error_final}
%     \end{align}
%     \textbf{Part 2 {}{}} We now relate RHS in \eqref{eq:LOO_error_final} with the population error on mislabeled distribution. To do this, we leverage \codref{cond:hypothesis_stability} and \lemref{lem:stability_error}. In particular, we have 

%     \begin{align}
%         \Expt{\calS \cup \wt \calS_M }{ \left(\error_{\calDm}(\wh f) - \error_{\text{LOO} (S_M)}\right)^2 } \le \frac{1}{2m_1} + \frac{3\beta}{m+n} \,.
%     \end{align}

%     Using Chebyshev's inequality, with probability at least $1-\delta$, we have 
%     \begin{align}
%         \error_{\text{LOO} (S_M)} \le  \error_{\calDm}(\wh f)   + \sqrt{\frac{1}{\delta}\left(\frac{1}{2m_1} +\frac{3\beta}{m+n} \right)} \,. \label{eq:final_mislabeled_linear}
%     \end{align}
    

%     \textbf{Part 3 {}{}} Combining \eqref{eq:final_mislabeled_linear} and \eqref{eq:LOO_error_final}, we have 

%     \begin{align}
%         \error_{\wt \calS_M } (\wh f) \le \error_{\calDm}(\wh f)   + \sqrt{\frac{1}{\delta}\left(\frac{1}{2m_1} +\frac{3\beta}{m+n} \right)} \,. \label{eq:linear_parallel_lem1}
%     \end{align}

%     Compare \eqref{eq:linear_parallel_lem1}, with \eqref{eq:lemma1_final} in the proof of \lemref{lem:fit_mislabeled}. We obtain a similar relationship between $\error_{\wt \calS_M }$ and $\error_{\calDm}$ but with a polynomial concentration instead of exponential concentration. 
%     In addition, since we just use concentration arguments to relate mislabeled error with the error on clean portion and unlabeled portion, we can directly use the results in \lemref{lem:mislabeled_error} and \lemref{lem:clear_error}. Therefore, combining results in \lemref{lem:mislabeled_error}, \lemref{lem:clear_error}, and \eqref{eq:linear_parallel_lem1} with union bound, we have with probability at least $1-\delta$

%     \begin{align}
%         \error_\calD(\widehat f) \le \error_\calS(\widehat f) + 1 - 2 \error_{\wt\calS}(\widehat f) + \left(\frac{1}{\sqrt{2}} + 1.5 \right) \sqrt{\frac{\log(4/\delta)}{m}} + \sqrt{\frac{4}{\delta}\left(\frac{1}{m} +\frac{3\beta}{m+n} \right)}  \,.
%     \end{align}
    

       
% \end{proof}

% \subsection{Discussion on \codref{cond:hypothesis_stability}}

% The quantity in LHS of \codref{cond:hypothesis_stability} measures how much the function learned by the algorithm (in terms of error on unseen point) will change when one point in the training set is removed. 
% % Discussion on exponential concentration and stronger condition. 
% Notice that hypothesis stability implies error stability, i.e., \codref{cond:error_stability} ~\cite{bousquet2002stability}.  In summary, while error stability allowed us to relate the average population error of the leave-one-out classifiers with the population error of the original classifier, we need hypothesis stability condition to control the variance of the empirical leave-one-out error. 

% Additionally, we note that while the dominating term in the RHS of \thmref{thm:new_linear} matches with the dominating term in ERM bound in \thmref{thm:error_ERM}, there is a polynomial concentration term (dependence on $1/\delta$ instead of $\log(\sqrt{1/\delta})$) in  \thmref{thm:new_linear}. 
% Since with hypothesis stability, we just bound the variance,  the polynomial concentration is due to the use of Chebyshev's inequality instead of an exponential tail inequality (as in \lemref{lem:fit_mislabeled}).
% Recent works have highlighted that slightly stronger condition than hypothesis stability can be used to obtained an exponential concentration for leave-one-out error~\citep{abou2019exponential}, but we leave this for future work for now. 
% % We leave 
% % However, the constants 

% % we also want to highlight  

% \subsection{Formal statement and proof of  of \propref{prop:early_stop}}

% Before formally presenting the result, we will introduce some notation.  By $\calL_{S}(w)$, we denote 
% the objective in \eqref{eq:l2_MSE_app} with $\lambda=0$. 
% Assume Singular Value Decomposition (SVD) of $\bX$  as $\sqrt{n} \bU \bS^{1/2} \bV^T$. Hence $\bX^T \bX = \bV \bS \bV^T$.
% Consider the GD iterates defined in \eqref{eq:GD_iterates_app}. 
% % 
% We now derive closed form expression for the $t^\text{th}$ iterate of gradient descent:  
% % 
% \begin{align}
%     w_t = w_{t-1} + \eta \cdot \bX^T (\by - \bX w_{t-1}) = (\bI - \eta \bV \bS \bV^T )w_{k-1} + \eta \bX^T \by \,.
% \end{align}
% Rotating by $\bV^T$, we get 
% \begin{align}
%     \wt w_t = (\bI - \eta\bS )\wt w_{k-1} + \eta \wt \by \,, \label{eq:GD_recur}
% \end{align}
% where $\wt w_t = \bV^T w_t $ and $\wt \by = \bV^T \bX^T \by$. Assuming the initial point $w_0 = 0$ and applying the recursion in \eqref{eq:GD_recur}, we get
% \begin{align}
%     \wt w_t = \bS ^{-1} ( \bI - (\bI - \eta \bS)^k ) \wt \by \,, 
% \end{align} 
% Projecting solution back to the original space, we have 
% \begin{align}
%      w_t = \bV \bS ^{-1} ( \bI - (\bI - \eta \bS)^k ) \bV^T \bX^T \by \,, 
% \end{align} 
% % We will work with this GD solution at any iterate $t$ in the next proposition. 
% Define $f_t(x) \defeq f(x;w_t)$ as the solution at the $t^{\text{th}}$ iterate. 
% Let $\wt w_{\lambda} = \argmin_{w} \calL_\calS (w;\lambda) = (\bX^T \bX + \lambda \bI)^{-1} \bX^T \by = \bV (\bS + \lambda \bI )^{-1} \bV^T \bX^T \by $. 
% % ) \,,$ for all $t=1,2,\ldots\,.$ 
% and define $\wt f_\lambda(x) \defeq f(x;\wt w_\lambda)$ as the regularized solution. 
% Assume $\kappa$ be the condition number of the population covariance matrix 
% and 
% let $s_\text{min}$ be the minimum positive singular value of the empirical covariance matrix. Our proof idea is inspired from recent work on relating gradient flow solution and regularized solution for regression problems \citep{ali2018continuous}. We will use the following lemma in the proof: 
% \begin{lemma} \label{lem:ineq_soln}
%     For all $x \in [0,1]$ and for all $ k \in \mathbb{N}$, we have (a) $ \frac{kx}{1+kx} \le 1- (1-x)^k$ and (b) $ 1- (1-x)^k \le 2 \cdot \frac{kx}{kx+1} $.
%     %  where $g(c)$ is a constant dependent on $c$. For $c = 1$, $g(c) = 2.0$.   
% \end{lemma}
% \begin{proof}
%     % [Proof of \lemref{lem:ineq_soln}]
%     % Part (a) is easy. 
%     Using $ (1-x)^k \le \frac{1}{1+kx}$, we have part (a). For part (b), we numerically maximize $\frac{ (1+kx ) (1 - (1-x)^k) }{kx}$ for all $k\ge 1$ and for all $x \in [0, 1]$.  
% \end{proof}

% % 
% % Next, 

% \begin{prop}[Formal statement of \propref{prop:early_stop}] \label{prop:formal_early_stop}
% Let $\lambda = \frac{1}{t\eta}$. For a training point $x$, we have 
% \begin{align*}
%     \Expt{x \sim \calS}{(f_t(x) - \wt f_\lambda(x))^2} &\le c(t,\eta) \cdot \Expt{x \sim \calS}{f_t(x)^2} \,, %\label{eq:early_stop}
% \end{align*}
% where $c(t, \eta) \defeq \min( 0.25, \frac{1}{s_\text{min}^2 t^2 \eta^2})$. Similarly for a test point, we have 
% \begin{align*}
%     \Expt{x \sim \calD_\calX}{(f_t(x) - \wt f_\lambda(x))^2} &\le \kappa \cdot c(t,\eta) \cdot \Expt{x \sim \calD_\calX}{f_t(x)^2} \,. %\label{eq:early_stop}
% \end{align*}
% \end{prop} 

% \begin{proof}
%     %%%%%%%%%%%%% 
%     We want to analyze the expected squared difference output of regularized linear regression with regularization constant $\lambda = \frac{1}{\eta t}$ and gradient descent solution at $t^\text{th}$ iterate. We separately expand the algebraic expression for squared difference at a training point and a test point. 
%     % We start by considering the difference  
%     Then the main step is to show that  $\left[ \bS ^{-1} ( \bI - (\bI - \eta \bS)^k )  - (\bS + \lambda \bI )^{-1}\right] \preceq c(\eta, t) \cdot \bS ^{-1} ( \bI - (\bI - \eta \bS)^k ) $.

%     %%%%%%%%%%%%%
    
%   \textbf{Part 1 {} {}} 
%     First, we will analyze the squared difference of output at a training point (for simplicity, we refer to $S \cup \wt S$ as $S$), i.e. 
%     \begin{align}
%         \Expt{ x \sim \calS }{\left(f_t(x) - \wt f_\lambda (x)\right)^2} &= \norm{\bX w_t - \bX \wt w_\lambda}{2}^2 =   \norm{\bX \bV \bS ^{-1} ( \bI - (\bI - \eta \bS)^t ) \bV^T \bX^T \by - \bX \bV (\bS + \lambda \bI )^{-1} \bV^T \bX^T \by }{2}^2 \\
%         &= \norm{\bX \bV \left(\bS ^{-1} ( \bI - (\bI - \eta \bS)^t ) - (\bS + \lambda \bI )^{-1} \right) \bV^T \bX^T \by  }{2} \\
%         &=  \by^T \bV \bX \left( \underbrace{\bS ^{-1} ( \bI - (\bI - \eta \bS)^t ) - (\bS + \lambda \bI )^{-1}}_{\RN{1}} \right)^2 \bS \bV^T \bX^T \by \label{eq:train_GD_rel}
%         %  (\bX \bV \bS ^{-1} ( \bI - (\bI - \eta \bS)^k ) \bV^T \bX^T \by)^T \bX \bV \bS ^{-1} ( \bI - (\bI - \eta \bS)^k ) \bV^T \bX^T \by
%     \end{align}
%     We now separately consider term 1. Substituting $\lambda = \frac{1}{t \eta}$, we get
%     \begin{align}
%         \bS ^{-1} ( \bI - (\bI - \eta \bS)^t ) - (\bS + \lambda \bI )^{-1} &= \bS^{-1} \left( ( \bI - (\bI - \eta \bS)^t ) - (\bI + \bS^{-1} \lambda )^{-1}\right) \\
%         &= \underbrace{\bS^{-1} \left( ( \bI - (\bI - \eta \bS)^t ) - (\bI + ( \bS t \eta)^{-1}  )^{-1}\right)}_{\bA}
%     \end{align}

%     We now separately bound the diagonal entries in matrix $\bA$. 
%     With $s_i$, we denote $i^{\text{th}}$ diagonal entry of $\bS$. Note that since $ \eta\le 1/\norm{S}{\text{op}}$, for all $i$, $\eta s_i  \le 1$.  Consider $i^{\text{th}}$ diagonal term (which is non-zero) of the diagonal matrix $\bA$, we have 
%     \begin{align}
%         \bA_{ii} = \frac{1}{s_i} \left(  1 - (1 - s_i \eta)^t - \frac{t \eta s_i}{1 + t \eta s_i } \right) &=  \frac{1 - (1 - s_i \eta)^t}{s_i} \left( \underbrace{ 1 - \frac{t \eta s_i}{(1 + t \eta s_i)(1 - (1 - s_i \eta)^t)}}_{\RN{2}} \right) \\ 
%          &\le \frac{1}{2}\left[ \frac{1 - (1 - s_i \eta)^t}{ s_i} \right] \tag*{(Using \lemref{lem:ineq_soln} (b))} \,.
%     \end{align} 
%     Additionally, we can also show the following upper bound on term 2: 
%     \begin{align}
%          1 - \frac{t \eta s_i}{(1 + t \eta s_i)(1 - (1 - s_i \eta)^t)} &= \frac{(1 + t \eta s_i)(1 - (1 - s_i \eta)^t) - t \eta s_i }{(1 + t \eta s_i)(1 - (1 - s_i \eta)^t)} \\
%          & \le  \frac{ 1 -  (1 - s_i \eta)^t - t \eta s_i (1 - s_i \eta)^t}{(1 + t \eta s_i)(1 - (1 - s_i \eta)^t)} \\
%          & \le \frac{1}{t\eta s_i} \,. \tag{Using \lemref{lem:ineq_soln} (a)}
%         %  &\le \frac{1}{2}\left[ \frac{1 - (1 - s_i \eta)^t}{ s_i} \right] \tag*{(Using \lemref{lem:ineq_soln})} \,.
%     \end{align} 

%     Combining both the upper bounds on each diagonal entry $\bA_{ii}$, we have 
%     \begin{align}
%     \bA \preceq c_1(\eta, t) \cdot \bS^{-1} ( \bI - (\bI - \eta \bS)^t ) \,, \label{eq:upperbound_diagonal}
%     \end{align}
%     where $c_1(\eta, t ) = \min(0.5, \frac{1}{t s_i \eta })$. Plugging this into \eqref{eq:train_GD_rel}, we have 
%     \begin{align}
%         \Expt{ x \sim \calS }{\left(f_t(x) - \wt f_\lambda (x)\right)^2} &\le c(\eta, t) \cdot \by^T \bV \bX  \left( \bS^{-1} ( \bI - (\bI - \eta \bS)^t ) \right)^2 \bS \bV^T \bX^T \by \\
%         &=   c(\eta, t) \cdot \by^T \bV \bX  \left( \bS^{-1} ( \bI - (\bI - \eta \bS)^t ) \right) \bS \left( \bS^{-1} ( \bI - (\bI - \eta \bS)^t ) \right) \bV^T \bX^T \by \\
%         & =  c(\eta, t) \cdot \norm{\bX w_t}{2}^2 \\
%         &= c(\eta, t) \cdot  \Expt{ x \sim \calS }{\left(f_t(x) \right)^2} \,,
%     \end{align}
%     where $c(\eta, t ) = \min(0.25, \frac{1}{t^2 s^2_i \eta^2 })$.

%     \textbf{Part 2 {} {}} With $\bSigma$, we denote the underlying true covariance matrix. We now consider the squared difference of output at an unseen point: 
%     \begin{align}
%         \Expt{ x \sim \calD_{\calX} }{\left(f_t(x) - \wt f_\lambda (x)\right)^2} &= \Expt{x \sim \calD_{\calX}}{\norm{x^T w_t - x^T \wt w_\lambda}{2}} \\
%         &=   \norm{x^T \bV \bS ^{-1} ( \bI - (\bI - \eta \bS)^t ) \bV^T \bX^T \by - x^T \bV (\bS + \lambda \bI )^{-1} \bV^T \bX^T \by }{2} \\
%         &= \norm{x^T \bV \left(\bS ^{-1} ( \bI - (\bI - \eta \bS)^t ) - (\bS + \lambda \bI )^{-1} \right) \bV^T \bX^T \by  }{2} \\
%         &= \by^T \bV \bX \left( \bS ^{-1} ( \bI - (\bI - \eta \bS)^t ) - (\bS + \lambda \bI )^{-1} \right) \bV^T \bSigma \bV \\ &\qquad \qquad \qquad \qquad \qquad \left( (\bI - (\bI - \eta \bS)^t ) - (\bS + \lambda \bI )^{-1} \right) \bV^T \bX^T \by \\
%         &\le \sigma_{\text{max}} \cdot \by^T \bV \bX \left( \underbrace{\bS ^{-1} ( \bI - (\bI - \eta \bS)^t ) - (\bS + \lambda \bI )^{-1}}_{\RN{1}} \right)^2 \bV^T \bX^T \by \,, \label{eq:test_GD_rel}
%         %  (\bX \bV \bS ^{-1} ( \bI - (\bI - \eta \bS)^k ) \bV^T \bX^T \by)^T \bX \bV \bS ^{-1} ( \bI - (\bI - \eta \bS)^k ) \bV^T \bX^T \by
%     \end{align}
%     where $\sigma_{\text{max}}$ is the maximum eigenvalue of the underlying covariance matrix $\bSigma$. Using the upper bound on term 1 in \eqref{eq:upperbound_diagonal}, we have 
%     \begin{align}
%         \Expt{ x \sim \calD_{\calX} }{\left(f_t(x) - \wt f_\lambda (x)\right)^2} &\le \sigma_{\text{max}} \cdot c(\eta, t) \cdot \by^T \bV \bX  \left( \bS^{-1} ( \bI - (\bI - \eta \bS)^t ) \right)^2 \bV^T \bX^T \by \\
%         &=   \kappa \cdot c(\eta, t) \cdot \sigma_{\text{min}}\cdot \norm{\bV \left( \bS^{-1} ( \bI - (\bI - \eta \bS)^t ) \right) \bV^T \bX^T \by}{2}^2 \\
%         &\le \kappa \cdot c(\eta, t) \cdot \left[ \bV \left( \bS^{-1} ( \bI - (\bI - \eta \bS)^t ) \right) \bV^T \bX^T \right]^T \bSigma \\
%         &\qquad \qquad \qquad \qquad \qquad \left[ \bV \left( \bS^{-1} ( \bI - (\bI - \eta \bS)^t ) \right) \bV^T \bX^T \right] \by \\
%         & = \kappa \cdot c(\eta, t) \cdot \Expt{x \sim \calD_{\calX}}{\norm{x^T w_t}{2}} \,.
%     \end{align}
% % 
% % 
%     % Since $ \eta\le 1/\norm{S}{\text{op}}$, invoking \lemref{lem:ineq_soln} to upper bound term 1 with
% \end{proof}


% \newpage
% \section{Additional experiments and details}\label{app:exp}
% \newcommand\tab[1][1cm]{\hspace*{#1}}

% \subsection{Datasets} \label{sec:app_dataset}

% \textbf{Toy Dataset {} {}} Assume fixed constants $\mu$ and $\sigma$. For a given label $y$, we simulate features $x$ in our toy classification setup as follows: 
% \begin{align*}
%     x \defeq \texttt{concat} \left[ x_1, x_2\right] \quad \text{where} \quad  x_1 \sim  \calN( y \cdot \mu, \sigma^2 I_{d \times d}) \ \  \text{and} \ \  x_1 \sim  \calN( 0, \sigma^2 I_{d \times d}) \,.
% \end{align*}  
% % where $y$ is the true label and $x$ is the corresponding feature vector. 
% In experiements throughout the paper, we fix dimention $d=100$, $\mu = 1.0 $, and $\sigma = \sqrt{d}$. Intuitively, $x_1$ carries the information about the underlying label and $x_2$ is additional noise independent of the underlying label. 

% \textbf{CV datasets {} {}} We use MNIST~\citep{lecun1998mnist} and CIFAR10~\cite{krizhevsky2009learning}. 
% % For binary tasks, 
% We produce a binary variant from the multiclass classification problem by mapping classes $\{0,1,2,3,4\}$ to label $1$ and $\{ 5,6,7,8,9\}$ to label $-1$. For CIFAR dataset, we also use the standard data augementation of random crop and horizontal flip. PyTorch code is as follows: 

% \texttt{(transforms.RandomCrop(32, padding=4),\\
% \tab transforms.RandomHorizontalFlip())}

% \textbf{NLP dataset {} {}} We use IMDb Sentiment analysis~\citep{maas2011learning} corpus.  

% \subsection{Architecture Details} 

% All experiments were run on NVIDIA GeForce RTX 2080 Ti GPUs. We used PyTorch~\citep{NEURIPS2019a9015} and Keras with Tensorflow~\citep{abadi2016tensorflow} backend for experiments. 
% % , ELMo embeddings~\citep{Peters:2018}, and Hugging Face Transformers~\citep{wolf-etal-2020-transformers}. 

% \textbf{Linear model {} {}} For the toy dataset, we simulate a linear model with scalar output and the same number of parameters as the number of dimensions.   

% \textbf{Wide nets {} {}} To simulate the NTK regime, we experiment with $2-$layered wide nets. The PyTorch code for 2-layer wide MLP is as follows: 


% \texttt{ nn.Sequential( \\
% \tab     nn.Flatten(),\\
% \tab    nn.Linear(input\_dims, 200000, bias=True),\\
% \tab    nn.ReLU(),\\
% \tab    nn.Linear(200000, 1, bias=True)\\
% \tab     )}


% We experiment both (i) with the first layer fixed at random initialization; (ii)  and updating both layers' weights.     

% \textbf{Deep nets for CV tasks {} {}} We consider a 4-layered MLP. The PyTorch code for 4-layer MLP is as follows: 

% \texttt{ nn.Sequential(nn.Flatten(), \\
% \tab        nn.Linear(input\_dim, 5000, bias=True),\\
% \tab        nn.ReLU(),\\
% \tab        nn.Linear(5000, 5000, bias=True),\\
% \tab        nn.ReLU(),\\
% \tab        nn.Linear(5000, 5000, bias=True),\\
% \tab        nn.ReLU(),\\
% % \tab        nn.Linear(5000, 5000, bias=True),\\
% % \tab        nn.ReLU(),\\
% \tab        nn.Linear(1024, num\_label, bias=True)\\
% \tab        )}

% For MNIST, we use $1000$ nodes instead of $5000$ nodes in the hidden layer. 
% % 
% We also experiment with convolutional nets. In particular, we use ResNet18 \citep{he2016deep}. Implementation adapted from:  \url{https://github.com/kuangliu/pytorch-cifar.git}. 

% \textbf{Deep nets for NLP {} {}} We use a simple LSTM model with embeddings intialized with ELMo embeddings~\citep{Peters:2018}. Code adapted from: \url{https://github.com/kamujun/elmo_experiments/blob/master/elmo_experiment/notebooks/elmo_text_classification_on_imdb.ipynb} 

% We also evaluate our bounds with a BERT model. In particular, we fine-tune an off-the-shelf uncased BERT model~\citep{devlin2018bert}. Code adapted from Hugging Face Transformers~\citep{wolf-etal-2020-transformers}: \url{https://huggingface.co/transformers/v3.1.0/custom_datasets.html}. 


% \subsection{Additonal experiments}

% 1. SGD with linear models on cross entropy and MSE loss. 

% 2. CE loss and SGD. GD with MSE loss 

% 3. Binary MNIST with MLP. multiclass MNIST  

% \textbf{Results on CIFAR 10 {} {}} 
% % 
% We plot epoch wise error curve for results in \tabref{table:multiclass}. We observe the same trend as in \figref{fig:error_CIFAR10}. Additionally, we plot an \emph{oracle bound} obtained by tracking the error on mislabeled data which nevertheless were predicted as true label. To obtain an exact emprical value of the oracle bound, we need underlying true labels for the randomly labeled data. 
% % Note that our bound in \thmref{thm:multiclass_ERM}, lower bounds the accuracy as predicted by the oracle bound. 
% While with just access to extra unlabeled data we cannot calculate oracle bound, we note that the oracle bound is very tight and never violated in practice underscoring an importamt aspect of generalization in multiclass problems. This highlight that even a stronger conjecture may hold in multiclass classification, i.e., error on mislabeled data (where nevertheless true label was predicted) lower bounds the population error on the distribution of mislabeled data and hence, the error on (a specific) mislabeled portion predicts the population accuracy on clean data. 
% % 
% On the other hand, the dominating term of in \thmref{thm:multiclass_ERM} is loose when compared with the oracle bound. The main reason, we believe is the pessimistic upper bound in \eqref{eq:lemma1_final_multi_prev} in the proof of \lemref{lem:fit_mislabeled_multi}. We leave an investigation on this gap for future. 
% % of fit 

% % However, oracle bound highlights two . One,  



% \begin{figure}[h]
%     \centering 
%     % \vspace{-15pt}
%     % \includegraphics[width=0.9\linewidth]{example-image-a}
%     \includegraphics[width=0.4\linewidth]{figures/CIFAR10-FNN.pdf} \hfil
%     \includegraphics[width=0.4\linewidth]{figures/CIFAR10-Resnet.pdf}
%     % \includegraphics[width=0.9\linewidth]{figures/{CIFAR10_rn=0.1_lr=0.2_wd=0.005}.png}
%     % \vspace{-10pt}
%     \caption{ Per epoch curves for CIFAR10 corresponding results in \tabref{table:multiclass}. As before, we just plot the dominating term in the RHS of \eqref{eq:multiclass_ERM} as predicted bound. Additionally, we also plot the predicted lower bound by the error on mislabeled data which nevertheless were predicted as true label. We refer to this as ``Oracle bound''. See text for more details. 
%     % 
%     % except for the stopping point. 
%     % The bound predicted by RATT (RHS in \eqref{eq:multiclass_ERM}) is vacuous. 
%     }\label{fig:error_epoch_CIFAR10}
%     % \vspace{-15pt}
% \end{figure}


% \textbf{Results on CIFAR 100 {} {}} 
% % 
% On CIFAR100, our bound in \eqref{eq:multiclass_ERM} yields vacous bounds. However, the oracle bound as explained above yields tight guarantees in the initial phase of the learning (i.e., when learning rate is less than $0.1$). 

% \begin{figure}[h]
%     \centering 
%     % \vspace{-15pt}
%     % \includegraphics[width=0.9\linewidth]{example-image-a}
%     \includegraphics[width=0.4\linewidth]{figures/CIFAR100-Resnet.pdf}
%     % \includegraphics[width=0.9\linewidth]{figures/{CIFAR10_rn=0.1_lr=0.2_wd=0.005}.png}
%     % \vspace{-10pt}
%     \caption{ Predicted lower bound by the error on mislabeled data which nevertheless were predicted as true label with ResNet18 on CIFAR100. We refer to this as ``Oracle bound''. See text for more details. 
%     % 
%     % except for the stopping point. 
%     The bound predicted by RATT (RHS in \eqref{eq:multiclass_ERM}) is vacuous. 
%     }\label{fig:error_CIFAR100}
%     % \vspace{-15pt}
% \end{figure}


% % \paragraph{Experiments on CIFAR100} 



% \subsection{Hyperparameter Details}


% \textbf{\figref{fig:error_CIFAR10} {} {}} We use clean training dataset of size $40,000$. We fix the amount of unlabeled data at $20\%$ of the clean size, i.e. we include additional $8,000$ points with randomly assigned labels. We use test set of $10,000$ points. For both MLP and ResNet, we use SGD with an initial learning rate of $0.1$ and momentum $0.9$. We fix the weight decay parameter at $5\times 10^{-4}$. After $100$ epochs, we decay the learning rate to $0.01$. We use SGD batch size of $100$. 

% \textbf{\figref{fig:error_binary} (a) {} {}} We obtain a toy dataset according to the process described in \secref{sec:app_dataset}. We fix $d=100$ and create a dataset of $50,000$ points with balanced classes. Moreover, we sample additional covariates with the same procedure to create randomly labeled dataset. For both SGD and GD training, we use a fixed learning rate $0.1$.    

% \textbf{\figref{fig:error_binary} (b) {} {}} Similar to binary CIFAR, we use clean training dataset of size $40,000$ and fix the amount of unlabeled data at $20\%$ of the clean dataset size. To train wide nets, we use a fixed learning of $0.001$ with GD and SGD. We decide the weight decay parameter and the early stopping point that maximizes our generalization bound (i.e. without peeking at unseen data ).  We use SGD batch size of $100$. 

% \textbf{\figref{fig:error_binary} (c) {} {}} With IMDb dataset, we use a clean dataset of size $20,000$ and as before, fix the amount of unlabeled data at $20\%$ of the clean data. To train ELMo model, we use Adam optimizer with a fixed learning rate $0.01$ and weight decay $10^{-6}$ to minimize cross entropy loss. We train with batch size $32$ for 3 epochs. To fine-tune BERT model, we use Adam optimizer with learning rate $5\times 10^{-5}$ to minimize cross entropy loss. We train with a batch size of $16$ for 1 epoch.    

% \textbf{\tabref{table:multiclass} {} {}} For multiclass datasets, we train both MLP and ResNet with the same hyperparameters as described before. We sample a clean training dataset of size $40,000$ and fix the amount of unlabeled data at $20\%$ of the clean size. We use SGD with an initial learning rate of $0.1$ and momentum $0.9$. We fix the weight decay parameter at $5\times 10^{-4}$. After $30$ epochs for ResNet and after $50$ epochs for MLP, we decay the learning rate to $0.01$.  We use SGD with batch size $100$. 
% For \figref{fig:error_CIFAR100}, we use the same hyperparameters as 
% CIFAR10 training, except we now decay learning rate after $100$ epochs. 


% In all experiments, to identify the best possible accuracy on just the clean data, we use the exact same set of hyperparamters except the stopping point. We choose a stopping point that maximizes test performance. 

% \subsection{Summary of experiments }

% \begin{center}
%     \begin{table}[H] 
%         \centering
%         \begin{tabular}{|c|c|c|c|} 
%         \hline
%         Classification type & Model category & Model & Dataset  \\ [0.5ex] 
%         \hline
%         \hline
%         \multirow{9}{*}{Binary} & Low dimensional & Linear model & Toy Gaussain dataset  \\
%                         \cline{2-4}
%                          & \multirow{1}{*}{Overparameterized linear nets} 
%                         %  & Linear model & Toy Gaussain dataset \\
%                         %  \cline{3-4}
%                         %  & & 2-layer wide net& Toy Gaussain dataset \\
%                         %  \cline{3-4}
%                          & 2-layer wide net & Binary MNIST \\
%                          \cline{2-4}                 
%                          & \multirow{6}{*}{Deep nets} & \multirow{2}{*}{MLP} & Binary MNIST \\
%                          \cline{4-4}
%                          & &  & Binary CIFAR \\
%                          \cline{3-4}
%                          &  & \multirow{2}{*}{ResNet} & Binary MNIST \\
%                          \cline{4-4}
%                          & &  & Binary CIFAR \\
%                          \cline{3-4}
%                          &  & ELMo-LSTM model & IMDb Sentiment Analysis \\
%                          \cline{3-4}
%                          & & BERT pre-trained model & IMDb Sentiment Analysis \\
%         \hline
%         \multirow{5}{*}{Multiclass} & \multirow{5}{*}{Deep nets} & \multirow{2}{*}{MLP} & MNIST \\
%                         \cline{4-4} 
%                         & & & CIFAR10 \\                   
%                         \cline{3-4}
%                          &   & \multirow{3}{*}{ResNet} & MNIST \\
%                          \cline{4-4}
%                          &   & & CIFAR10 \\
%                          \cline{4-4}
%                          &   & & CIFAR100 \\
%         \hline
%         \end{tabular}
%         % \caption{Summary of experiments performed} \label{table:experiments}
%     \end{table}    
%     % \footnotetext[6]{We use both MSE loss and cross-entropy loss.}
%     % \footnotetext[6]{We try 2 variants: one with a fixed first layer and the other with both layers trainable.}
% \end{center}

% \newpage
% \section{Proof of \lemref{lem:stability_error}} \label{app:proof_lem_error}

% \begin{proof}[Proof of \lemref{lem:stability_error}]
%     Recall, we have a training set $S \cup \wt S_C$. We defined leave-one-out error on mislabeled points as $$\error_{\text{LOO}(\wt S_M) } = \frac{\sum_{(x_i, y_i) \in \wt S_M} \error( f_{(i)}( x_i), y_i)}{ \abs{\wt S_M }} \,, $$
%     where $f_{(i)} \defeq f(\calA, (S \cup \wt S)_{(i)})$. Define $S^\prime \defeq S \cup \wt S$. Assume $(x,y)$ and $(x^\prime,y^\prime)$ as i.i.d. samples from ${\calDm}$. 
%     Using Lemma 25 in \citet{bousquet2002stability}, we have
%     \begin{align*}
%         \Expo{ \left( \error_{\calDm}(\wh f) -\error_{\text{LOO}(\wt S_M) } \right)^2 } \le & \Expt{ S^\prime, (x,y), (x^\prime,y^\prime) }{ \error(\wh f(x), y ) \error(\wh f(x^\prime), y^\prime )} - 2 \Expt{ S^\prime, (x,y) }{ \error(\wh f(x), y ) \error(f_{(i)}(x_i), y_i )} \\
%         & + \frac{m_1-1}{m_1}\Expt{ S^\prime }{  \error(f_{(i)}(x_i), y_i )  \error(f_{(j)}(x_j), y_j )} + \frac{1}{m_1} \Expt{ S^\prime }{  \error(f_{(i)}(x_i), y_i ) } \,. \numberthis \label{eq:main_reln}
%     \end{align*}
%     We can rewrite the equation above as : 
%     \begin{align*}
%         \Expo{ \left( \error_{\calDm}(\wh f) -\error_{\text{LOO}(\wt S_M) } \right)^2 } \le &  \, \underbrace{\Expt{ S^\prime, (x,y), (x^\prime,y^\prime) }{ \error(\wh f(x), y ) \error(\wh f(x^\prime), y^\prime ) - \error(\wh f(x), y ) \error(f_{(i)}(x_i), y_i )}}_{\RN{1}} \\
%         & + \underbrace{\Expt{ S^\prime }{  \error(f_{(i)}(x_i), y_i )  \error(f_{(j)}(x_j), y_j ) -  \error(\wh f(x), y ) \error(f_{(i)}(x_i), y_i )}}_{\RN{2}} \\ &+ \underbrace{\frac{1}{m_1} \Expt{ S^\prime }{  \error(f_{(i)}(x_i), y_i ) - \error(f_{(i)}(x_i), y_i )  \error(f_{(j)}(x_j), y_j ) }}_{\RN{3}} \,. \numberthis \label{eq:main_reln2}
%     \end{align*}
    
%     We will now bound term $\RN{3}$.  Using Schwarz's inequality, we have
    
%     \begin{align}
%         \Expt{ S^\prime }{  \error(f_{(i)}(x_i), y_i ) - \error(f_{(i)}(x_i), y_i )  \error(f_{(j)}(x_j), y_j ) }^2 &\le  \Expt{ S^\prime }{  \error(f_{(i)}(x_i), y_i ) }^2 \Expt{S^\prime}{1 -   \error(f_{(j)}(x_j), y_j ) }^2 \\
%         &\le \frac{1}{4} \label{eq:term1_lem12}
%     \end{align}
    
%     Note that since $(x_i,y_i)$, $(x_j ,y_j )$, $(x,y)$, and $(x^\prime, y^\prime)$ are all from same distribution $\calDm$, we directly incorporate the bounds on term $\RN{1}$ and $\RN{2}$ from proof of Lemma 9 in \citet{bousquet2002stability}. Combining that with \eqref{eq:term1_lem12} and our definition of hypothesis stability in \codref{cond:hypothesis_stability}, we have the required claim. 
    
    
%     % We now re-write term $\RN{1}$ as
%     % \begin{align*}
%     %         &\Expt{S^\prime, (x,y), (x^\prime,y^\prime) }{ \error(\wh f(x), y ) \error(\wh f(x^\prime), y^\prime ) - \error(\wh f(x), y ) \error(f_{(i)}(x_i), y_i )} \\ & \qquad = \Expt{ S^\prime, (x,y), (x^\prime,y^\prime) }{ \error(\wh f(x), y ) \error(\wh f  (x^\prime), y^\prime ) - \error(\wh f ^\prime(x), y ) \error(f_{(i)}(x^\prime), y^\prime )} \tag{Exchanging $(x_i, y_i)$ with $(x^\prime, y^\prime)$ in the second term} \\
%     %         & \qquad = \Expt{ S^\prime, (x,y), (x^\prime,y^\prime) }{  \left(\error(\wh f(x), y )-  \error(f_{(i)}(x), y ) \right) \error(\wh f  (x^\prime), y^\prime )  } \\
%     %         & \qquad  + \Expt{ S^\prime, (x,y), (x^\prime,y^\prime) }{  \left(\error(f_{(i)}(x), y ) -\error(\wh f ^\prime(x), y ) \right) \error(\wh f  (x^\prime), y^\prime )}  \\
%     %         & \qquad +\Expt{ S^\prime, (x,y), (x^\prime,y^\prime) }{  \left( \error(\wh f  (x^\prime), y^\prime ) -  \error(f_{(i)}(x^\prime), y^\prime ) \right) \error(\wh f ^\prime(x), y ) }  \,, \numberthis \label{eq:term1_final}
%     % \end{align*}
%     % where $\wh f^\prime$ is the classifier obtained by training on $ S^\prime_{(i)} \cup \{ (x^\prime, y^\prime) \} $. Similarly we can re-write term $\RN{2}$ as 
%     % \begin{align*}
%     %     & \Expt{ S^\prime }{  \error(f_{(i)}(x_i), y_i )  \error(f_{(j)}(x_j), y_j ) -  \error(\wh f(x), y ) \error(f_{(i)}(x_i), y_i )} \\
%     %     &\quad  = \Expt{ S^\prime, (x,y), (x^\prime,y^\prime)}{  \error(f^{\prime\prime}_{(i)}(x), y )  \error(f_{(j)}^{\prime}(x^\prime), y^\prime ) -  \error(\wh f(x), y ) \error(f_{(i)}(x_i), y_i )} \tag{Exchanging $(x_i, y_i)$ with $(x, y)$ and $(x_j, y_j)$ with $(x^\prime, y^\prime)$ in the first term}\\
%     %     &\quad = \Expt{ S^\prime, (x,y), (x^\prime,y^\prime)}{  \error(f^{\prime\prime}_{(j)}(x), y )  \error(f_{(i)}^{\prime}(x^\prime), y^\prime ) -  \error(\wh f^\prime (x), y ) \error(f^\prime_{(j)}(x^\prime), y^\prime )} \tag{Exchanging $(x_i, y_i)$ and $(x_j, y_j)$ and then replacing $(x_j, y_j)$ with $(x^\prime, y^\prime)$ in the second term} \\
%     %     & \quad = \Expt{ S^\prime, (x,y), (x^\prime,y^\prime) }{  \left( \error(f_{(i)}^{\prime}(x^\prime), y^\prime )   -  \error(\wh f^{\prime\prime}  (x^\prime), y^\prime ) \right)  \error(f^{\prime\prime}_{(j)}(x), y )   } \\
%     %     & \quad  + \Expt{ S^\prime, (x,y), (x^\prime,y^\prime) }{  \left( \error(f^{\prime\prime}_{(j)}(x), y )  -\error(\wh f ^\prime(x), y ) \right) \error(\wh f^{\prime\prime}  (x^\prime), y^\prime )  }  \\
%     %     & \quad+ \Expt{ S^\prime, (x,y), (x^\prime,y^\prime) }{  \left( \error(\wh f^{\prime\prime}  (x^\prime), y^\prime )  -  \error(f^\prime_{(j)}(x^\prime), y^\prime ) \right)  \error(\wh f^\prime (x), y ) }   \\
%     %     & \quad = \Expt{ S^\prime, (x,y), (x^\prime,y^\prime) }{  \left( \error(f_{(i)}^{\prime}(x^\prime), y^\prime )   -  \error(\wh f (x^\prime), y^\prime ) \right)  \error(f_{(i)}(x_j), y_j )   } \\
%     %     & \quad  + \Expt{ S^\prime, (x,y), (x^\prime,y^\prime) }{  \left( \error(f^{\prime\prime}_{(j)}(x), y )  -\error(\wh f (x), y ) \right) \error(\wh f^{\prime\prime}  (x_j), y_j )  }  \\
%     %     & \quad+ \Expt{ S^\prime, (x,y), (x^\prime,y^\prime) }{  \left( \error(\wh f^{\prime\prime}  (x^\prime), y^\prime )  -  \error(f^\prime_{(j)}(x^\prime), y^\prime ) \right)  \error(\wh f^\prime (x^\prime), y^\prime ) }  \,, \numberthis \label{eq:term2_final}
%     % \end{align*}
%     % where $f^{\prime\prime}_{(j)}$ is trained on $S^\prime_{(j,i)} \cup {(x,y)}$, $f^{\prime}_{(i)}$ is trained on $S^\prime_{(j,i)} \cup {(x^\prime,y^\prime)}$, and $\wh f^{\prime\prime} $ is trained on $S^\prime_{(j)} \cup {(x,y)}$. Note in the last line we replaced $(x,y)$ by $(x_j, y_j)$ in the first term, replaced $(x^\prime,y^\prime)$ by $(x_j, y_j)$ in the second term and exchanged $(x_i,y_i)$ with $(x_j,y_j)$ and also $(x,y)$ and $(x^\prime, y^\prime)$
    
    
% \end{proof}

\addcontentsline{toc}{section}{References}
\bibliographystyle{alpha}
\bibliography{ref}


% \newpage
% \appendix
% \section{Appendix}
% \input{other_norms}
% \input{localization}
% \newpage
% \input{log-sobolev}


\end{document}
