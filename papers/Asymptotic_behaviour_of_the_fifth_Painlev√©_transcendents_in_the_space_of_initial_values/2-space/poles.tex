Here, we consider neighbourhoods of exceptional lines where the Painlev\'e vector field \eqref{eq:PVlog-system} becomes unbounded.
The construction given in Appendix \ref{sec:resolution} shows that these are given by the lines $\mathcal{L}_5$, $\mathcal{L}_6$, $\mathcal{L}_7$ and $\mathcal{L}_{10}$.

\subsubsection*{Movable pole with residue $-\,\theta_{\infty}$} The set $\mathcal{L}_5\setminus\mathcal{I}$ is given by $y_{62}=0$ in the $(y_{62},z_{62})$ chart, see Section \ref{a5-blow}.
Suppose $y_{62}(\tau)=0$, $z_{62}(\tau)=B$, for arbitrary complex numbers $\tau$, $B$.
Solving the system of differential equations for $y_{62}$, $z_{62}$ in Section \ref{a5-blow} in a neighbourhood of $t=\tau$, we obtain
$$
y_{62}=-\theta_{\infty}(t-\tau)+\frac{\theta_{\infty}}{2}\big(2 B-2\theta_{\infty}-\eta-\theta_1e^{\tau}\big)(t-\tau)^2+O((t-\tau)^3).
$$
Since the transformation from $(y_{62}, z_{62})$ to $(y, z)$ (also given in Section \ref{a5-blow}) is
$$
y=\frac1{y_{62}}
\quad\text{and}\quad
z=y_{62}(y_{62}z_{62}+1),
$$ 
we obtain series expansions for $(y, z)$ given by
$$
\begin{aligned}
y= & -\frac{\theta_{\infty}}{t-\tau}-\frac{2 B-2\theta_{\infty}-\eta-\theta_1e^{\tau}}{2\theta_{\infty}}+O(t-\tau),
\\
z= & -\theta_{\infty}(t-\tau)
 +
\frac{\theta_{\infty}}{2}\left(
2B(1+\theta_{\infty})-\eta-2\theta_{\infty}-\theta_1 e^{\tau}
\right)(t-\tau)^2
+O((t-\tau)^3).
\end{aligned}
$$
Clearly, $y$ has a simple pole with residue $-\theta_{\infty}$,  while $z$ has a simple zero at $t=\tau$.

\subsubsection*{Movable pole with residue $\theta_{\infty}$} At the intersection with $\mathcal{L}_6\setminus\mathcal{I}$, $y$ has a simple pole with the residue $\theta_{\infty}$, while $z$ has a simple zero.
This case is analogous to the previous one, see Sections \ref{a5-blow} and \ref{a6-blow}.


\subsubsection*{Movable zero with coefficient $\theta_{0}$} The set $\mathcal{L}_7\setminus\mathcal{I}$ is given by $z_{81}=0$ in the $(y_{81},z_{81})$ chart, see Section \ref{a7-blow}.
Suppose $y_{81}(\tau)=B$ and $z_{81}(\tau)=0$.
Then integration of the vector field gives
\[
%\begin{split}
z_{81}(t)=(t-\tau)
+\frac{\eta-2\theta_0-\theta_1e^{\tau}}{2}(t-\tau)^2%\\&\qquad 
+\frac{F-2\eta\theta_0+\theta_0^2-\theta_1e^{\tau}-4B}{3}(t-\tau)^3
 +O((t-\tau)^4),
%\end{split}
\]
with $F=\dfrac12\epsilon(\theta_0+\eta-\theta_{\infty})$.
Since
$$
y=z_{81}(y_{81}z_{81}+\theta_0),
\quad
z=\frac1{z_{81}},
$$
we obtain
\[
\begin{split}
y= &\ \theta_0(t-\tau)+\frac{2B+\theta_0(\eta-2\theta_0-\theta_1e^{\tau})}{2}(t-\tau)^2+O((t-\tau)^3),
\\
z=&\ \frac{1}{t-\tau}-\frac{\eta-2\theta_0-\theta_1e^{\tau}}{2}%\\&\qquad
+\left(\frac{(\eta-2\theta_0-\theta_1e^{\tau})^2}{4}-\frac{F-2\eta\theta_0+\theta_0^2-\theta_1e^{\tau}-4B}{3}\right)(t-\tau)
\\&\qquad\quad
+O((t-\tau)^2).
\end{split}
\]
Thus, $y$ has a simple zero and $z$ a simple pole at $t=\tau$.

\subsubsection*{Movable points where $y$ becomes unity} The set $\mathcal{L}_{10}\setminus\mathcal{I}$ is given by $z_{111}=0$ in the $(y_{111},z_{111})$ chart, see Section \ref{a10-blow}.
Suppose $y_{111}(\tau)=B$ and $z_{111}(\tau)=0$.
Then
\[
\begin{split}
z_{111}=&\ (t-\tau)+\frac{\theta_1 e^{\tau}-\eta-1}{2}(t-\tau)^2
\\&
+\frac13\left(
1+\frac52\eta+\eta^2-\frac{B}{\theta_1 e^{\tau}}+\theta_1e^{\tau}\left(1-\theta_0-\frac{\eta}2\right)+\frac12\theta_1^2e^{2\tau}
\right)(t-\tau)^3%\\&\qquad
 +O((t-\tau)^4).
\end{split}
\]
Since we have
$$
\begin{aligned}
y =&\ 1 + \theta_1 e^t z_{111} + (1+\eta)\theta_1 e^t z_{111}^2 + y_{111} z_{111}^3,
\\
z =&\ \frac{1}{ z_{111}^2 (\theta_1 e^t + (1+\eta)\theta_1 e^t z_{111} + y_{111} z_{111}^2)},
\end{aligned}
$$
it follows that $y(\tau)=1$ and $z$ has a double pole at $t=\tau$.
Their expansions around this point are given by
\[
\begin{split}
y=&\ 1+\theta_1e^{\tau}(t-\tau)+\frac{\theta_1e^{\tau}(\theta_1 e^{\tau}+\eta+3)}{2}(t-\tau)^2
\\
&+\frac{\theta_1e^{\tau}}{6}\left(
2 - 4 \eta - 4 \eta^2 - \frac{2 B}{\theta_1 e^{\tau}} + (11 +  5 \eta- 2 \theta_0)\theta_1 e^{\tau}
 + \theta_1^2e^{2\tau}
\right)(t-\tau)^3%\\&\qquad 
+O((t-\tau)^4),
\\
z=&\ \frac{(\theta_1 e^{\tau})^{-1}}{(t-\tau)^2}+\frac{1+(\theta_1e^{\tau})^{-1}}{t-\tau}
+ \left(\frac{2 B}{\theta_1 e^{\tau}} + \frac{\eta \theta_1 e^{\tau}}{2}+ \theta_0 \theta_1 e^{\tau} 
+ \frac{\theta_1^2 e^{2\tau}}{4} - \frac{1+3 \eta^2}{4} \right)%\\&\qquad 
+O(1).
\end{split}
\]



















