Since the fifth Painlev\'e equation is a second-order ordinary differential equation, solutions are locally defined by two initial values. Therefore, the space of initial values is two complex-dimensional. However, standard existence and uniqueness theorems only cover values of $y$ that are not arbitrarily close to $0$, $1$ or infinity (where the second derivative given by $\PV$ becomes ill-defined). In this section, we explain how to construct a regularized, compactified space of all possible initial values that overcomes these issues.

We start by formulating $\PV$ as an equivalent system of equations in Section \ref{sec:system} and describing its autonomous limiting form obtained as $x\to0$ in Section \ref{sec:auto}. The mathematical construction of the space of initial values is then given in Section \ref{sec:okamoto}. Where $y$ is arbitrarily close to the singular values $0$, $1$, $\infty$, the solutions have singular power series expansions, which become regularized Taylor expansions in corresponding domains of the initial value space. These regular expansions are provided in Section \ref{sec:poles}.

\subsection{A system equivalent to $\PV$}\label{sec:system}

With the change of the independent variable $t=\log x$, Equation (\ref{eq:PV}) becomes:
\begin{equation}\label{eq:PVlog}
\begin{split}
\frac{d^2y}{dt^2}=%&
\left(\frac{1}{2y}+\frac{1}{y-1}\right) \left(\frac{dy}{dt}\right)^2+(y-1)^2\left(\alpha y+\frac{\beta}{y}\right)
%\\&\qquad 
+e^t\gamma y +e^{2t}\frac{\delta y(y+1)}{y-1}.
\end{split}
\end{equation}
We rewrite Equation (\ref{eq:PVlog}) in the following way:
\begin{equation}\label{eq:PVlog-system}
\begin{split}
\frac{dy}{dt}=&
 2 y (y-1)^2z-(\theta_0+\eta)y^2+(2\theta_0+\eta-\theta_1 e^t)y-\theta_0,
\\
\frac{dz}{dt}=&-
(y-1)(3y-1)z^2+\big(2(\theta_0+\eta)y-2\theta_0-\eta+\theta_1 e^t\big)z
%\\&\qquad 
-\frac12\epsilon(\theta_0+\eta-\theta_{\infty}),
\end{split}
\end{equation}
where
$\theta_{\infty}^2=2\alpha$,
$\theta_0^2=-2\beta$,
$\theta_1^2=-2\delta$ $(\theta_1\neq0)$,
$\eta=-\frac{\gamma}{\theta_1}-1$, and
$\epsilon=\frac12(\theta_0+\theta_{\infty}+\eta)$.

\begin{remark}
Here, we assumed that $\delta\neq0$, which is a generic case of the fifth Painlev\'e equation.
When $\delta=0$, the $\PV$ is equivalent to the third Painlev\'e equation \cite{OO2006}.
\end{remark}

The system (\ref{eq:PVlog-system}) is Hamiltonian:
$$
\frac{dy}{dt}=\frac{\partial H}{\partial z},
\qquad
\frac{dz}{dt}=-\frac{\partial H}{\partial y},
$$
with Hamiltonian function:
\begin{equation}\label{eq:H}
H=y(y-1)^2z^2-(\theta_0+\eta)y^2z+(2\theta_0+\eta-\theta_1 e^t)yz-\theta_0z+\frac12\epsilon(\theta_0+\eta-\theta_{\infty})y.
\end{equation}


\subsection{The autonomous equation}
\label{sec:auto}

The autonomous equation corresponding to (\ref{eq:PVlog}) is:
\begin{equation}\label{eq:PVlog-auto}
\frac{d^2y}{dt^2}=\left(\frac{1}{2y}+\frac{1}{y-1}\right) \left(\frac{dy}{dt}\right)^2+(y-1)^2\left(\alpha y+\frac{\beta}{y}\right),
\end{equation}
which is equivalent to the autonomous version of (\ref{eq:PVlog-system}):
\begin{equation}\label{eq:PVlog-system-auto}
\begin{aligned}
&\frac{dy}{dt}=
(y-1)^2(2yz-\theta_0)+\eta y(y-1),
\\
&\frac{dz}{dt}=
(y-1)z(2\eta+2\theta_0+z-3yz)+\eta z-\frac12\epsilon(\theta_0+\eta-\theta_{\infty}).
\end{aligned}
\end{equation}
System (\ref{eq:PVlog-system-auto}) is also Hamiltonian:
$$
\frac{dy}{dt}=\frac{\partial E}{\partial z},
\qquad
\frac{dz}{dt}=-\frac{\partial E}{\partial y},
$$
with Hamiltonian:
\begin{equation}\label{eq:E}
E=y(y-1)^2z^2-(\theta_0+\eta)y^2z+(2\theta_0+\eta)yz-\theta_0z+\frac12\epsilon(\theta_0+\eta-\theta_{\infty})y.
\end{equation}

Using the first equation of (\ref{eq:PVlog-system-auto}) to express $z$, and using the fact that $E$ is constant along solutions, we get:
$$
\left(\frac{dy}{dt}\right)^2=(y-1)^2(4 C y + \theta_0^2 - 2 \theta_0 (\eta + \theta_0) y  + \theta_{\infty}^2  y^2 ),
\qquad
C=\const.
$$

It is worth observing that the constant function $y\equiv1$ is the only solution of this equation taking the value $1$.
From (\ref{eq:PVlog-system-auto}), the corresponding function $z$ is the solution of
$$
\frac{dz}{dt}=
\eta z-\frac12\epsilon(\theta_0+\eta-\theta_{\infty}).
$$
That is, we have
$$
z=c_1 e^{\eta t}+\frac{\epsilon(\theta_0+\eta-\theta_{\infty})}{2\eta},
$$
where $c_1$ is a constant.

The flow (\ref{eq:PVlog-system-auto}) has four fixed points: 
\begin{equation*}
(y, z)=
  \begin{cases}
   \left(1, \dfrac{\epsilon(\theta_0+\eta-\theta_{\infty})}{2\eta}\right)\\
   \\
   \left(Y_i, \dfrac{\theta_0}{2Y_i}+\dfrac{\eta}{2(1-Y_i)}\right), i\in\{1,2,3\},
  \end{cases}
\end{equation*}
where $Y_1$, $Y_2$, $Y_3$ are the roots of the following cubic polynomial in $Y$:
\[
  \left(-\theta_{\infty}^2 -   6 \eta^2 + 8 \eta \theta_0 + 2\theta_0^2\right) Y^3
 %\\&\qquad 
 +\left( 2 \eta^2 - 12 \eta \theta_0 - 2\theta_0^2+\theta_{\infty}^2 \right) Y^2 
+\theta_0 (4 \eta - \theta_0) Y
+\theta_0^2.
\]

%\subsection{Fibration associated to a differential equation}\label{sec:fibration}
%\input{2-space/fibration}


\subsection{Resolution of singularities}\label{sec:okamoto}
In this section, we explain how to construct the space of initial values for the system (\ref{eq:PVlog-system}).
First, we motivate the reason for this construction before introducing the notion of initial value spaces in Definition \ref{def:initial-values-space}, which is based on foliation theory.
We then explain how to construct such a space by carrying out resolutions or blow-ups, which are described in Definition \ref{def:blow-up}.

The system (\ref{eq:PVlog-system}) is a system of two first-order ordinary differential equations for $(y(t), z(t))$. 
Given initial values $(y_0, z_0)$ at $t_0$, local existence and uniqueness theorems provide a solution that is defined on a local polydisk $U\times V$ in $\mathbf C\times \mathbf C^2$, where $t_0\in U\subset \mathbf C$ and $(y_0, z_0)\in V\subset \mathbf C^2$.
Our interest lies in global extensions of these local solutions.

However, the occurence of movable poles in the Painlev\'e transcendents acts as a barrier to the extension of $U\times V$ to the whole of $\mathbf C\times \mathbf C^2$.
The first step to overcome this obstruction is to compactify the space $\mathbf C^2$, in order to include the poles.
We carry this out by embedding $\mathbf C^2$ into $\mathbf C\mathbf P^2$ and explicitly represent the system (\ref{eq:PVlog-system}) in the three affine coordinate charts of $\mathbf C\mathbf P^2$, which are given in Sections \ref{chart01}--\ref{chart03}.
The second step in this process results from the occurence of singularities in the Painlev\'e vector field \eqref{eq:PVlog-system}.
These occur in the $(y_{02},z_{02})$ and $(y_{03},z_{03})$ charts. 
The appearence of these singularities is related to irreducibility of the solutions of Painlev\'e equations as indicated by the following theorem, due to Painlev\'e. 

\begin{theorem}[\cite{Painleve1897}]
If the space of initial values for a differential equation is a compact rational variety, then the equation can be reduced either to a linear differential equation of higher order or to an equation for elliptic functions.
\end{theorem}
It is well known that the solutions of Painlev\'e equations are irreducible (in the sense of the theorem). Since $\mathbf{CP}^2$ is a compact rational variety, the theorem implies $\mathbf{CP}^2$ cannot be the space of initial values for (\ref{eq:PVlog-system}).

By the term {\em singularity} we mean points where $(dy/dt, dz/dt)$ becomes either unbounded or undefined because at least one component approaches the undefined limit $0/0$.
We are led therefore to construct a space in which the points where the singularities are regularised.
The process of regularisation is called "blowing up" or \emph{resolving a singularity}.

We now define the notion of initial value space.

\begin{definition}\label{def:initial-values-space}[\cite{Gerard1975}, \cite{GerardSec1972,Gerard1983,Okamoto1979}]
Let $(\mathcal{E},\pi,\mathcal{B})$ be a complex analytic fibration, $\Phi$ its foliation, and $\Delta$ a a holomorphic differential system on $\mathcal{E}$, such that:
\begin{itemize}
\item the leaves of $\Phi$ correspond to the solutions of $\Delta$;
\item the leaves of $\Phi$ are transversal to the fibres of $\mathcal{E}$;
\item for each path $p$ in the base $\mathcal{B}$ and each point $X\in \mathcal{E}$, such that $\pi(X)\in p$, the path $p$ can be lifted into the leaf of $\Phi$ containing point $X$.
\end{itemize}
Then each fibre of the fibration is called \emph{a space of initial values} for the system $\Delta$.
\end{definition}

The properties listed in Definition \ref{def:initial-values-space} imply that each leaf of the foliation is isomorphic to the base $\mathcal{B}$.
Since the fifth Painlev\'e transcendents (in the $t$ variable) can be globally extended as meromorphic functions on $\mathbf{C}$ \cite{JK1994,HL2001}, we are searching for the fibration with the base equal to $\mathbf{C}$. 

%In the extension of the system (\ref{eq:PVlog-system}) to $\mathbf{CP}^2$, two types of singular points appear \cite{Okamoto1979}.


%In a neighbourhood of any other singular point, the vector field is defined by expressions which approch the undefined limit of the form $0/0$.

%In order to separate the infinitely many leaves at a singular point which is not of the first class, we will apply the blow-up construction \cite{HartshorneAG,GrifHarPRINC,DuistermaatBOOK}.

In order to construct the fibration, we apply the blow-up procedure, defined below, \cite{HartshorneAG,GrifHarPRINC,DuistermaatBOOK} to the singularities of the system (\ref{eq:PVlog-system}) that occur where at least one component becomes undefined of the form $0/0$.
\ocite{Okamoto1979} showed that such singular points are contained in the closure of infinitely many leaves.
Moreover, these leaves are holomorphically extended at such a point.

\begin{definition}\label{def:blow-up}
\emph{The blow-up} of the plane $\mathbf{C}^2$ at point $(0,0)$ is the closed subset $X$ of $\mathbf{C}^2\times\mathbf{CP}^1$ defined by the equation $u_1t_2=u_2t_1$, where $(u_1,u_2)\in\mathbf{C}^2$ and $[t_1:t_2]\in\mathbf{CP}^1$, see Figure \ref{fig:blow-up}.
There is a natural morphism $\varphi: X\to\mathbf{C}^2$, which is the restriction of the projection from $\mathbf{C}^2\times\mathbf{CP}^1$ to the first factor.
$\varphi^{-1}(0,0)$ is the projective line $\{(0,0)\}\times\mathbf{CP}^1$, called \emph{the exceptional line}.
\end{definition}
\begin{figure}[h]
\centering
\begin{pspicture*}(-10.5,-4)(4,4)

\psset{unit=0.6}

\psset{viewpoint=4 10 0,Decran=40,lightsrc=20 20 20}

 \psline[linecolor=black,fillcolor=black,incolor=black,linewidth=1pt,arrows=->,arrowsize=2pt 4](-3,0)(0,0)
\rput(-1.7,0.3){$\varphi$}

\rput(-15,-2){$X$}

\psSolid[object=plan, linecolor=gray!50, definition=equation, args={[1 0 0 1]}, base=-1.5 1.5 -1.5 1.5]


\defFunction[algebraic]{krug}(t)
{0}{cos(t)}{sin(t)}
\psSolid[object=courbe,linecolor=black,r=0.01,range=0 pi 2 mul,
linewidth=0.06,resolution=1000,
function=krug](-1,0,0)



\psSolid[object=line, linecolor=gray, args=-1 1 0 -1 -1 0]
\psSolid[object=line, linecolor=gray, args=-1 0 1 -1  0 -1]
\psSolid[object=line, linecolor=gray, args=-1 0.707 0.707 -1  -0.707 -0.707]
\psSolid[object=line, linecolor=gray, args=-1 -0.707 0.707 -1  0.707 -0.707]
\psSolid[object=line, linecolor=gray, args=-1 -0.383 0.924 -1  0.383 -0.924]
\psSolid[object=line, linecolor=gray, args=-1 0.383 0.924 -1  -0.383 -0.924]
\psSolid[object=line, linecolor=gray, args=-1  0.924 -0.383 -1   -0.924 0.383]
\psSolid[object=line, linecolor=gray, args=-1  -0.924 -0.383 -1   0.924 0.383]

\psSolid[object=point, args=-1 0 0]


\defFunction[algebraic]{helix1}(h)
{h+1}{cos(h+pi/4)}{sin(h+pi/4)}
\psSolid[object=courbe,r=0,
range=0 pi, linecolor=black, linewidth=0.1, resolution=360, function=helix1]

\defFunction[algebraic]{helix2}(h)
{h+1}{-cos(h+pi/4)}{-sin(h+pi/4)}
\psSolid[object=courbe,r=0,
range=0 pi, linecolor=black, linewidth=0.1, resolution=360, function=helix2]%

\psSolid[object=line, args=1 0 0 1 pi add 0 0]



\psSolid[object=line, linecolor=black, args=1 0.707 0.707 1 -0.707 -0.707 ]
\psSolid[object=line, linecolor=black, args=2.571 -0.707 0.707 2.571 0.707 -0.707 ]
\psSolid[object=line, linecolor=black, args=2.178 -0.383 0.924 2.178  0.383 -0.924 ]
\psSolid[object=line, linecolor=black, args=1.393 -0.383 -0.924 1.393  0.383 0.924 ]
\psSolid[object=line, linecolor=black, args=2.963  0.924 -0.383 2.963   -0.924 0.383]
\psSolid[object=line, linecolor=black, args=3.75  -0.924 -0.383 3.75  0.924 0.383]
\psSolid[object=line, linecolor=black, args=1.8   0 1 1.8  0 -1 ]
\psSolid[object=line, linecolor=black, args=4.14 0.707 0.707 4.14 -0.707 -0.707 ]

\defFunction[algebraic]{helix}(t,h)
{h+1}{t*cos(h+pi/4)}{t*sin(h+pi/4)}
\psSolid[object=surfaceparametree,linewidth=1sp,linecolor=gray!50,
     base=-1 1 0 pi,fillcolor=gray!50,incolor=gray!50,opacity=0.2,
     function=helix,
   ngrid=10 72]

  
 

\end{pspicture*}


\caption{The blow-up of the plane at a point.}\label{fig:blow-up}
\end{figure}

\begin{remark}
Notice that the points of the exceptional line $\varphi^{-1}(0,0)$ are in bijective correspondence with the lines containing $(0,0)$.
On the other hand,
$\varphi$ is an isomorphism between $X\setminus\varphi^{-1}(0,0)$ and $\mathbf{C}^2\setminus\{(0,0)\}$.
More generally, any complex two-dimensional surface can be blown up at a point \cite{HartshorneAG,GrifHarPRINC,DuistermaatBOOK}.
In a local chart around that point, the construction will look the same as described for the case of the plane.
\end{remark}

Notice that the blow-up construction separates the lines containing the point $(0,0)$ in Definition \ref{def:blow-up}, as shown in Figure \ref{fig:blow-up}.
In this way, the solutions of \eqref{eq:PVlog-system} containing the same point can be separated.
Additional blow-ups may be required if the solutions have a commont tangent line or a tangency of higher order at such a point.

The explicit resolution of the vector field \eqref{eq:PVlog-system} is carried out in Appendix \ref{sec:resolution}. As mentioned above, the process requires 11 resolutions of singularities, or, blow-ups.
%After these 11 blow-ups, only the singular points of the first class remain, and the space of the initial values is obtained by removing them.

\begin{figure}[h]
\centering


\begin{pspicture}(-1.5,-5)(12,0)



\psset{linecolor=black,linewidth=0.02,fillstyle=solid}


%beskonacna prava
\psline(0,-0.5)(10,-0.5)
\rput(9.5,-0.2){$\mathcal{L}_{\infty}^*$}
\rput(9.5,-0.8){\small\textbf{-1}}


%L0
\psline(3,-0.5)(3,-4)
\rput(2.7,-3.8){$\mathcal{L}_{0}^*$}
\rput(3.4,-3.8){$\small\textbf{-2}$}



%L2 

\psline(3,-1.5)(-1.5,-1.5)
\rput(-1,-1.2){$\mathcal{L}_{2}^*$}
\rput(-1,-1.8){\small\textbf{-3}}


%L5 

\psline[linestyle=dashed](1.5,-1.5)(1.5,-4)
\rput(1.2,-3.3){$\mathcal{L}_{5}$}
\rput(1.9,-3.3){\small\textbf{-1}}



%L6 

\psline[linestyle=dashed](0,-1.5)(0,-4)
\rput(-0.3,-3.8){$\mathcal{L}_{6}$}
\rput(0.4,-3.8){\small\textbf{-1}}


%L1 
\psline(7,-0.5)(7,-5)
\rput(6.7,-4.8){$\mathcal{L}_{1}^*$}
\rput(7.4,-4.8){\small\textbf{-4}}




%L3 

\psline(7,-1.5)(4,-1.5)
\rput(4.5,-1.2){$\mathcal{L}_{3}^*$}
\rput(4.5,-1.8){\small\textbf{-2}}



%L7 
\psline[linestyle=dashed](5.5,-1.5)(5.5,-4)
\rput(5.2,-3.8){$\mathcal{L}_{7}$}
\rput(5.9,-3.8){\small\textbf{-1}}

%L8 
\psline(7,-2)(11.5,-2)
\rput(11,-1.7){$\mathcal{L}_{8}^*$}
\rput(11,-2.3){\small\textbf{-2}}

%L4
\psline(8.5,-2)(8.5,-5)
\rput(8.2,-3.8){$\mathcal{L}_{4}^{*}$}
\rput(8.9,-3.8){\small\textbf{-2}}

%L9
\psline(10,-2)(10,-5)
\rput(9.4,-4.8){$\mathcal{L}_{9}(t)^{*}$}
\rput(10.4,-4.8){\small\textbf{-2}}

%L10 
\psline[linestyle=dashed](10,-3.5)(12,-3.5)
\rput(11.5,-3.2){$\mathcal{L}_{10}(t)$}
\rput(11.5,-3.8){\small\textbf{-1}}



\end{pspicture}



\caption{The fibre $\mathcal{D}(t)$ is obtained from $\mathbf{CP}^2$ by $11$ blow-ups.}\label{fig:okamoto}
\end{figure}

The resulting surface $\mathcal{D}(t)$ is shown in Figure \ref{fig:okamoto}.
Note that in this figure, $\mathcal{L}_{\infty}^*$ is the proper preimage of the line at the infinity, while
$\mathcal{L}_{0}^*$, $\mathcal{L}_{1}^*$, $\mathcal{L}_{2}^*$, $\mathcal{L}_{3}^*$, $\mathcal{L}_{4}^*$, $\mathcal{L}_{8}^*$, $\mathcal{L}_{9}(t)^*$  are proper preimages of exceptional lines obtained by blow ups at points $a_0$, $a_1$, $a_2$, $a_3$, $a_4$, $a_8$, $a_9$ respectively
and $\mathcal{L}_{5}$, $\mathcal{L}_{6}$, $\mathcal{L}_{7}$, $\mathcal{L}_{10}(t)$ are exceptional lines obtained by blowing up points $a_5$, $a_6$, $a_7$, $a_{10}$ respectively. 
The self-intersection number of each line (after all blow-ups are completed) is indicated in the figure.

Okamoto described so called \emph{singular points of the first class} that are not contained in the closure of any leaf of the foliation given by the system of differential equations.
At such points, the corresponding vector field is infinite.
(For example, in chart $(y_{02},z_{02})$ from Section \ref{chart02} such a singular point is given by $y_{02}=0$ with non-zero $z_{02}$.)
In the surface $\mathcal{D}(t)$, all remaining singular points are of the first class, and the fibre of the initial value space is obtained by removing them:
$$
\mathcal{E}(t)=\mathcal{D}(t)\setminus\left(\bigcup_{j=0}^4\mathcal{L}_j^*\cup\mathcal{L}_8^*\cup\mathcal{L}_9^*\cup\mathcal{L}_{\infty}\right).
$$







In $\mathcal{D}(t)$, each line with self-intersection number $-1$ can be blown down again. 
Blowing down $\mathcal{L}_{\infty}^*$ and then the projection of $\mathcal{L}_{0}^*$, we get the surface $\mathcal{F}(t)$, which is shown in Figure \ref{fig:okamoto-blow-down}.
The projection of each remaining line from $\mathcal{D}(t)$ is denoted by the same index but now with superscript $p$.
Notice that the self-intersection numbers of $\mathcal{L}_{1}^p$ and $\mathcal{L}_{2}^p$ are no longer the same as of the corresponding pre-images $\mathcal{L}_{1}^*$ and $\mathcal{L}_{2}^*$.
In this space,  we denote by $\mathcal{I}$ the set of all singular points of the first class in $\mathcal{F}(t)$:
$$
\mathcal{I}=\bigcup_{j=1}^4\mathcal{L}_j^p\cup\mathcal{L}_8^p\cup\mathcal{L}_9^p.
$$
The fibre $\mathcal{E}(t)$ of the initial value space can be identified with $\mathcal{F}(t)\setminus\mathcal{I}$.

\begin{figure}[h]
\centering


\begin{pspicture}(-1,-5)(12,0)



\psset{linecolor=black,linewidth=0.02,fillstyle=solid}


%L2
\psline(-1,-0.5)(10,-0.5)
\rput(-.5,-0.2){$\mathcal{L}_{2}^p$}
\rput(-.5,-0.8){\small\textbf{-2}}


%L5
\psline[linestyle=dashed](3,-0.5)(3,-4)
\rput(2.7,-3.8){$\mathcal{L}_{5}^p$}
\rput(3.4,-3.8){\small\textbf{-1}}



%L6

\psline[linestyle=dashed](1.5,-0.5)(1.5,-4)
\rput(1.2,-3.3){$\mathcal{L}_{6}^p$}
\rput(1.9,-3.3){\small\textbf{-1}}



%L1 
\psline(7,-0.5)(7,-5)
\rput(6.7,-4.8){$\mathcal{L}_{1}^p$}
\rput(7.4,-4.8){\small\textbf{-2}}




%L3 

\psline(7,-1.5)(4,-1.5)
\rput(4.5,-1.2){$\mathcal{L}_{3}^p$}
\rput(4.5,-1.8){\small\textbf{-2}}



%L7 
\psline[linestyle=dashed](5.5,-1.5)(5.5,-4)
\rput(5.2,-3.8){$\mathcal{L}_{7}^p$}
\rput(5.9,-3.8){\small\textbf{-1}}

%L8 
\psline(7,-2)(11.5,-2)
\rput(11,-1.7){$\mathcal{L}_{8}^p$}
\rput(11,-2.3){\small\textbf{-2}}

%L4
\psline(8.5,-2)(8.5,-5)
\rput(8.2,-3.8){$\mathcal{L}_{4}^p$}
\rput(8.9,-3.8){\small\textbf{-2}}

%L9
\psline(10,-2)(10,-5)
\rput(9.4,-4.8){$\mathcal{L}_{9}(t)^p$}
\rput(10.4,-4.8){\small\textbf{-2}}

%L10 
\psline[linestyle=dashed](10,-3.5)(12,-3.5)
\rput(11.5,-3.2){$\mathcal{L}_{10}(t)^p$}
\rput(11.5,-3.8){\small\textbf{-1}}



\end{pspicture}



\caption{The fibre $\mathcal{F}(t)$ is obtained from $\mathbf{CP}^2$ by $11$ blow-ups and two blow-downs.}\label{fig:okamoto-blow-down}
\end{figure}


If $\mathcal{S}$ is a surface obtained from the projective plane by a several successive blow-ups of points, then the group of all automorphisms of the Picard group $\Pic(\mathcal{S})$ preserving the canonical divisor $K$ is generated by the reflections
$X\mapsto X+(X.\omega)\omega$,
with $\omega\in\Pic(\mathcal{S})$, $K.\omega=0$, $\omega.\omega=-2$.
That group is an affine Weyl group and the lines of self-intersection $-2$ are its simple roots.
Representing each such line by a node and connecting a pair of nodes by a line only if they intersect in the fibre, we obtain the Dynkin diagram shown in Figure \ref{fig:dynkin}, which is of type $D_5^{(1)}$.
For detailed expositions on the topic of surfaces and root systems, see \cite{Demazure1980, Harbourne1985} and references therein.

\begin{figure}[h]
\centering
\begin{pspicture}(-2,-2)(4,2)

\psset{linecolor=black}
\rput(0,0){\circlenode{1}{$\mathcal{L}_1^p$}}
\rput(2,0){\circlenode{2}{$\mathcal{L}_8^p$}}
\rput(-1.414,1.414){\circlenode{3}{$\mathcal{L}_2^p$}}
\rput(-1.414,-1.414){\circlenode{4}{$\mathcal{L}_3^p$}}
\rput(3.414,1.414){\circlenode{5}{$\mathcal{L}_4^p$}}
\rput(3.414,-1.414){\circlenode{6}{$\mathcal{L}_9^p$}}

\ncline{-}{1}{2}
\ncline{-}{1}{3}
\ncline{-}{1}{4}
\ncline{-}{2}{5}
\ncline{-}{2}{6}
    
\end{pspicture}

\caption{The Dynkin diagram of $D_5^{(1)}$.}\label{fig:dynkin}
\end{figure}


In the limit $\Real t\to-\infty$, the resulting Okamoto space is compactified by the fibre $\mathcal{F}_{\infty}$, corresponding to the autonomous system (\ref{eq:PVlog-system-auto}), see Figure \ref{fig:okamoto-blow-down-limit}.
Its infinity set is given by
$$
\mathcal{I}_{\infty}=\mathcal{L}_1^a\cup\mathcal{L}_2^p\cup\mathcal{L}_3^a.
$$
\begin{figure}[h]
\centering

\begin{pspicture}(-1,-5)(8,0)



\psset{linecolor=black,linewidth=0.02,fillstyle=solid}


%L2
\psline(-1,-0.5)(8,-0.5)
\rput(-.5,-0.2){$\mathcal{L}_{2}^a$}
\rput(-.5,-0.8){\small\textbf{-2}}


%L5
\psline[linestyle=dashed](3,-0.5)(3,-4)
\rput(2.7,-3.8){$\mathcal{L}_{5}^a$}
\rput(3.4,-3.8){\small\textbf{-1}}



%L6

\psline[linestyle=dashed](1.5,-0.5)(1.5,-4)
\rput(1.2,-3.3){$\mathcal{L}_{6}^a$}
\rput(1.9,-3.3){\small\textbf{-1}}



%L1 
\psline(7,-0.5)(7,-5)
\rput(6.7,-4.8){$\mathcal{L}_{1}^a$}
\rput(7.4,-4.8){\small\textbf{-1}}




%L3 

\psline(7,-1.5)(4,-1.5)
\rput(4.5,-1.2){$\mathcal{L}_{3}^a$}
\rput(4.5,-1.8){\small\textbf{-2}}



%L7 
\psline[linestyle=dashed](5.5,-1.5)(5.5,-4)
\rput(5.2,-3.8){$\mathcal{L}_{7}^a$}
\rput(5.9,-3.8){\small\textbf{-1}}




\end{pspicture}



\caption{The fibre $\mathcal{F}_{\infty}$ corresponding to the autonomous system (\ref{eq:PVlog-system-auto}).}\label{fig:okamoto-blow-down-limit}
\end{figure}
$\mathcal{F}_{\infty}$ is constructed by a sequence of blow-ups, carried out along the lines of the construction of the space of the initial values for the non-autonomous system (\ref{eq:PVlog-system}).
For those readers who may wish to carry this out separately, we note that the resolution at point $a_4$ (found in Section \ref{a1-blow}) needs to be done only for the non-autonomous system.





\subsection{Movable singularities in Okamoto's space}\label{sec:poles}
Here, we consider neighbourhoods of exceptional lines where the Painlev\'e vector field \eqref{eq:PVlog-system} becomes unbounded.
The construction given in Appendix \ref{sec:resolution} shows that these are given by the lines $\mathcal{L}_5$, $\mathcal{L}_6$, $\mathcal{L}_7$ and $\mathcal{L}_{10}$.

\subsubsection*{Movable pole with residue $-\,\theta_{\infty}$} The set $\mathcal{L}_5\setminus\mathcal{I}$ is given by $y_{62}=0$ in the $(y_{62},z_{62})$ chart, see Section \ref{a5-blow}.
Suppose $y_{62}(\tau)=0$, $z_{62}(\tau)=B$, for arbitrary complex numbers $\tau$, $B$.
Solving the system of differential equations for $y_{62}$, $z_{62}$ in Section \ref{a5-blow} in a neighbourhood of $t=\tau$, we obtain
$$
y_{62}=-\theta_{\infty}(t-\tau)+\frac{\theta_{\infty}}{2}\big(2 B-2\theta_{\infty}-\eta-\theta_1e^{\tau}\big)(t-\tau)^2+O((t-\tau)^3).
$$
Since the transformation from $(y_{62}, z_{62})$ to $(y, z)$ (also given in Section \ref{a5-blow}) is
$$
y=\frac1{y_{62}}
\quad\text{and}\quad
z=y_{62}(y_{62}z_{62}+1),
$$ 
we obtain series expansions for $(y, z)$ given by
$$
\begin{aligned}
y= & -\frac{\theta_{\infty}}{t-\tau}-\frac{2 B-2\theta_{\infty}-\eta-\theta_1e^{\tau}}{2\theta_{\infty}}+O(t-\tau),
\\
z= & -\theta_{\infty}(t-\tau)
 +
\frac{\theta_{\infty}}{2}\left(
2B(1+\theta_{\infty})-\eta-2\theta_{\infty}-\theta_1 e^{\tau}
\right)(t-\tau)^2
+O((t-\tau)^3).
\end{aligned}
$$
Clearly, $y$ has a simple pole with residue $-\theta_{\infty}$,  while $z$ has a simple zero at $t=\tau$.

\subsubsection*{Movable pole with residue $\theta_{\infty}$} At the intersection with $\mathcal{L}_6\setminus\mathcal{I}$, $y$ has a simple pole with the residue $\theta_{\infty}$, while $z$ has a simple zero.
This case is analogous to the previous one, see Sections \ref{a5-blow} and \ref{a6-blow}.


\subsubsection*{Movable zero with coefficient $\theta_{0}$} The set $\mathcal{L}_7\setminus\mathcal{I}$ is given by $z_{81}=0$ in the $(y_{81},z_{81})$ chart, see Section \ref{a7-blow}.
Suppose $y_{81}(\tau)=B$ and $z_{81}(\tau)=0$.
Then integration of the vector field gives
\[
%\begin{split}
z_{81}(t)=(t-\tau)
+\frac{\eta-2\theta_0-\theta_1e^{\tau}}{2}(t-\tau)^2%\\&\qquad 
+\frac{F-2\eta\theta_0+\theta_0^2-\theta_1e^{\tau}-4B}{3}(t-\tau)^3
 +O((t-\tau)^4),
%\end{split}
\]
with $F=\dfrac12\epsilon(\theta_0+\eta-\theta_{\infty})$.
Since
$$
y=z_{81}(y_{81}z_{81}+\theta_0),
\quad
z=\frac1{z_{81}},
$$
we obtain
\[
\begin{split}
y= &\ \theta_0(t-\tau)+\frac{2B+\theta_0(\eta-2\theta_0-\theta_1e^{\tau})}{2}(t-\tau)^2+O((t-\tau)^3),
\\
z=&\ \frac{1}{t-\tau}-\frac{\eta-2\theta_0-\theta_1e^{\tau}}{2}%\\&\qquad
+\left(\frac{(\eta-2\theta_0-\theta_1e^{\tau})^2}{4}-\frac{F-2\eta\theta_0+\theta_0^2-\theta_1e^{\tau}-4B}{3}\right)(t-\tau)
\\&\qquad\quad
+O((t-\tau)^2).
\end{split}
\]
Thus, $y$ has a simple zero and $z$ a simple pole at $t=\tau$.

\subsubsection*{Movable points where $y$ becomes unity} The set $\mathcal{L}_{10}\setminus\mathcal{I}$ is given by $z_{111}=0$ in the $(y_{111},z_{111})$ chart, see Section \ref{a10-blow}.
Suppose $y_{111}(\tau)=B$ and $z_{111}(\tau)=0$.
Then
\[
\begin{split}
z_{111}=&\ (t-\tau)+\frac{\theta_1 e^{\tau}-\eta-1}{2}(t-\tau)^2
\\&
+\frac13\left(
1+\frac52\eta+\eta^2-\frac{B}{\theta_1 e^{\tau}}+\theta_1e^{\tau}\left(1-\theta_0-\frac{\eta}2\right)+\frac12\theta_1^2e^{2\tau}
\right)(t-\tau)^3%\\&\qquad
 +O((t-\tau)^4).
\end{split}
\]
Since we have
$$
\begin{aligned}
y =&\ 1 + \theta_1 e^t z_{111} + (1+\eta)\theta_1 e^t z_{111}^2 + y_{111} z_{111}^3,
\\
z =&\ \frac{1}{ z_{111}^2 (\theta_1 e^t + (1+\eta)\theta_1 e^t z_{111} + y_{111} z_{111}^2)},
\end{aligned}
$$
it follows that $y(\tau)=1$ and $z$ has a double pole at $t=\tau$.
Their expansions around this point are given by
\[
\begin{split}
y=&\ 1+\theta_1e^{\tau}(t-\tau)+\frac{\theta_1e^{\tau}(\theta_1 e^{\tau}+\eta+3)}{2}(t-\tau)^2
\\
&+\frac{\theta_1e^{\tau}}{6}\left(
2 - 4 \eta - 4 \eta^2 - \frac{2 B}{\theta_1 e^{\tau}} + (11 +  5 \eta- 2 \theta_0)\theta_1 e^{\tau}
 + \theta_1^2e^{2\tau}
\right)(t-\tau)^3%\\&\qquad 
+O((t-\tau)^4),
\\
z=&\ \frac{(\theta_1 e^{\tau})^{-1}}{(t-\tau)^2}+\frac{1+(\theta_1e^{\tau})^{-1}}{t-\tau}
+ \left(\frac{2 B}{\theta_1 e^{\tau}} + \frac{\eta \theta_1 e^{\tau}}{2}+ \theta_0 \theta_1 e^{\tau} 
+ \frac{\theta_1^2 e^{2\tau}}{4} - \frac{1+3 \eta^2}{4} \right)%\\&\qquad 
+O(1).
\end{split}
\]






















