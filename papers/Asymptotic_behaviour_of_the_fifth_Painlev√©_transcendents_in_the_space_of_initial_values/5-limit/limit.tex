In this section we consider properties of the limit set of the solutions, when $\Real t\to-\infty$, i.e.~$x\to0$.
First, we define the limit set, generalising the concept of limit sets in dynamical systems.

\begin{definition}
Let $(y(t),z(t))$ be a solution of (\ref{eq:PVlog-system}). \emph{The limit set} $\Omega_{(y,z)}$ of $(y(t),z(t))$ is the set of all $S\in\mathcal{F}_{\infty}\setminus\mathcal{I}_{\infty}$ such that there exists a sequence $t_n\in\mathbf{C}$ satisfying:
$$
\lim_{n\to\infty}\Real t_n=-\infty
\quad\text{and}\quad
\lim_{n\to\infty}(y(t_n),z(t_n))=S.
$$
\end{definition}



\begin{theorem}\label{th:limit}
There exists a compact subset $K$ of $\mathcal{F}_{\infty}\setminus\mathcal{I}_{\infty}$, such that the limit set $\Omega_{(y,z)}$ of any solution $(y,z)$ is contained in $K$.
Moreover, $\Omega_{(y,z)}$ is a non-empty, compact and connected set, which is invariant under the flow of the autonomous system (\ref{eq:PVlog-system-auto}).
\end{theorem}

\begin{proof}
For any positive numbers $\delta_1$, $r$, let $K_{\delta_1,r}$ denote the set of all $s\in\mathcal{F}(t)$ such that $|e^t|\le r$ and $d(s)\ge\delta_1$.
Since $\mathcal{F}(t)$ is a complex analytic family over $\mathbf{C}$ of compact surfaces, $K_{\delta_1,r}$ is also compact.
Furthermore $K_{\delta_1,r}$ is disjoint from the union of the infinity sets $\mathcal{I}(t)$, $t\in\mathbf{C}$, and therefore $K_{\delta_1,r}$ is a compact susbset of the Okamoto space $\mathcal{O}\setminus\mathcal{F}_{0}$.
When $r$ approches zero, the sets $K_{\delta_1,r}$ shrink to the set
$$
K_{\delta_1,0}=\{ s\in\mathcal{F}(0) \mid |d(s)|\ge\delta_1\}\subset\mathcal{F}_{\infty}\setminus\mathcal{I}_{\infty},
$$
which is compact.

It follows from Theorem \ref{th:estimates} that there exists $\delta_1\ge0$ such that for every solution $(y,z)$ there exists $r_0>0$ with the following property:
$$
(y(t),z(t))\in K_{\delta,r_0}
\quad
\text{for every}\ t\ \text{such that}\ |e^t|\le r_0.
$$
In the sequel, we take $r\le r_0$, when it follows that $(y(t),z(t))\in K_{\delta_1,r}$ whenever $|e^t|\le r$.
Let 
$T_r=\{t\in\mathbf{C}\mid |e^t|\le r\}$
and let $\Omega_{(y,z),r}$ denote the closure of $(y,z)(T_r)$ in $\mathcal{O}$.
Since $T_r$ is connected and $(y,z)$ continuous, $\Omega_{(y,z),r}$ is also connected.
Since $(y,z)(T_r)$ is contained in the compact subset $K_{\delta_1,r}$, its closure $\Omega_{(y,z),r}$ is also contained there and therefore $\Omega_{(y,z),r}$ is a non-empty compact and connected subset of $\mathcal{O}\setminus\mathcal{F}(0)$.
The intersection of a decreasing sequence of non-empty compact and connected sets is non-empty, compact and connected: therefore, as $\Omega_{(y,z),r}$ decrease to $\Omega_{(y,z)}$ when $r$ tends to zero, it follows that $\Omega_{(y,z)}$ is a non-empty, compact and connected set of $\mathcal{O}$.
Since $\Omega_{(y,z),r}\subset K_{\delta_1,r}$ for all $r\le r_0$, and the sets $K_{\delta_1,r}$ shrink to the compact subset $K_{\delta_1},0$ of $\mathcal{F}_{\infty}\setminus\mathcal{I}_{\infty}$ as $r$ tends to zero, it follows that $\Omega_{(y,z)}\subset K_{\delta_1,0}$.
This proves the first statement of the theorem with $K=K_{\delta_1,0}$.

Since $\Omega_{(y,z)}$ is the intersection of the decreasing family of compact sets $\Omega_{(y,z),r}$, there exists for every neighbourhood $A$ of $\Omega_{(y,z)}$ in $\mathcal{O}$ and $r>0$ such that $\Omega_{(y,z),r}\subset A$, hence $(y(t),z(t))\in A$ for every $t\in\mathbf{C}$ such that $|e^t|\le r$.
If $t_j$ is any sequence in $\mathbf{C}$ such that $\Real t_j\to-\infty$, then the compactness of $K_{\delta_1,r}$, in combination with $(y,z)T_r\subset K_{\delta_1,r}$, implies that there is a subsequence $j=j(k)\to\infty$ as $k\to\infty$, such that:
$$
(y(t_{j(k)}),z(t_{j(k)}))\to s
\ \ \text{as}\ \ k\to\infty.
$$
Then it follows that $s\in\Omega_{(y,z)}$.

Next we prove that $\Omega_{(y,z)}$ is invariant under the flow $\Phi^{\tau}$ of the autonomous Hamiltonian system (\ref{eq:PVlog-system-auto}).
Let $s\in\Omega_{(y,z)}$ and $t_j$ be a sequence in $\mathbf{C}$ such that $\Real t_j\to-\infty$ and $(y(t_j),z(t_j))\to s$.
Since the $t$-dependent vector field of the system (\ref{eq:PVlog-system}) converges in $C^1$ to the vector field of the autonomous system (\ref{eq:PVlog-system-auto}) as $\Real t\to-\infty$, it follows from the continuous dependence on initial data and parameters, that the distance between $(y(t_j+\tau),z(t_j+\tau))$ and $\Phi^{\tau}(y(t_j),z(t_j))$ converges to zero as $j\to\infty$.
Since $\Phi^{\tau}(y(t_j),z(t_j))\to\Phi^{\tau}(s)$ and $\Real t_j\to-\infty$ as $j\to\infty$, 
it follows that $(y(t_j+\tau),z(t_j+\tau))\to\Phi^{\tau}(s)$ and $t_j+t\to\infty$ as $j\to\infty$, hence $\Phi^{\tau}(s)\in\Omega_{(y,z)}$.
\end{proof}

\begin{proposition}\label{prop:intersections}
If $y$ is a solution of (\ref{eq:PV}) with essential singularity at $x=0$, than the flow $(y,z)$ of the vectory field (\ref{eq:PVlog-system})  meets each of the pole lines $\mathcal{L}_5$, $\mathcal{L}_6$, $\mathcal{L}_7$ infinitely many times.
\end{proposition}

\begin{proof}
First, suppose that a solution $(y(t),z(t))$ intersects the union $\mathcal{L}_5\cup\mathcal{L}_6\cup\mathcal{L}_7$ only finitely many times.

According to Theorem \ref{th:limit}, the limit set $\Omega_{(y,z)}$ is a compact set in $\mathcal{F}_{\infty}\setminus\mathcal{I}_{\infty}$.
If $\Omega_{(y,z)}$ intersects one of the pole lines $\mathcal{L}_5$, $\mathcal{L}_6$, $\mathcal{L}_7$ at a point $p$, then there exists $t$ with arbitrarily large negative real part such that $(u(t),v(t))$ is arbitrarily close to $p$, when the transversality of the vector field to the pole line implies thet $(y(\tau),z(\tau))\in\mathcal{L}_5\cup\mathcal{L}_6\cup\mathcal{L}_7$ for a unique $\tau$ near $t$.
As this would imply that $(y(t),z(t))$ intersects $\mathcal{L}_5\cup\mathcal{L}_6\cup\mathcal{L}_7$ infinitely many times, it follows that $\Omega_{(y,z)}$ is a compact subset of $\mathcal{F}_{\infty}\setminus(\mathcal{I}_{\infty}\cup\mathcal{L}_5\cup\mathcal{L}_6\cup\mathcal{L}_7)$.
It follows that $\Omega_{(y,z)}$ is a compact subset contained in the first affine chart, which implies that $y$ and $z$ remain bounded for large negative $\Real t$.
Thus $y$, $z$ are holomorphic functions of $x=e^t$ in a neighbourhood of $x=0$, which implies that there are complex numbers $y(\infty)$, $z(\infty)$ which are the limit points of $y(t)$, $z(t)$ as $\Real t\to-\infty$.
That means that $y$ is analytic at $x=0$, which contradicts the assumption that it has there an essential singularity.

Since the limit set $\Omega_{(y,z)}$ is invariant under the autonomous flow, it means that it will contain the whole irreducible component of a curve from the pencil $h_c(y,z)=0$ given by (\ref{eq:pencil}), for some constant $c$.
It is shown in Section \ref{sec:conics} that this pencil of curves is birationally equivalent to a pencil of conics.
We identified in Section \ref{sec:conics} the three singular conics in the pencil and found the special solutions corresponding to them.

In all other cases, all three base points $b_5$, $b_6$, $b_7$ will be contained in the limit set, which are projections of the pole lines $\mathcal{L}_5(\infty)$, $\mathcal{L}_6(\infty)$, $\mathcal{L}_7(\infty)$ respectively.
For a general solution $(y,z)$, the base point $b_4$ will not be contained in the limit set, because that point is not a base point of the autonomous system (\ref{eq:PVlog-auto}).
\end{proof}

\begin{remark}
If the limit set $\Omega_{(y,z)}$ contains only one point, that point must be a fixed point of the autonomous system (\ref{eq:PVlog-system-auto}).
As we obtained in Section \ref{sec:auto}, there are four such points.
One of the points has $y$-coordinate equal to unity and it corresponds to the rational solutions of the form $\dfrac{\kappa}{x+\kappa}$ and $\dfrac{\kappa+x}{\kappa-x}$.
\end{remark}

\begin{theorem}\label{th:zeroespoles}
Every solution of (\ref{eq:PV}) with essential singularity at $x=0$ has infinitely many poles and infinitely many zeroes in each neighbourhood of that singular point.
\end{theorem}

\begin{proof}
Applying results from Section \ref{sec:poles} and Proposition \ref{prop:intersections}, we get that each solution has a simple pole at the intersections with $\mathcal{L}_5$ and $\mathcal{L}_6$ and a simple zero at the intersection with $\mathcal{L}_7$.
\end{proof}
