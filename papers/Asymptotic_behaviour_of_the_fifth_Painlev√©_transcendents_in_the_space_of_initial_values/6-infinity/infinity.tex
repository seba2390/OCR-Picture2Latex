For studying the limit $x\to\infty$, it is convenient to represent the $\PV$ as the following system:
\begin{equation}\label{eq:PVsystem}
\begin{split}
&y'=\frac1x\big(
2y(y-1)^2z-(\theta_0+\eta)y^2+(2\theta_0+\eta-\theta_1x)y-\theta_0
    \big),
\\
&z'=-\frac1x\bigg(
(y-1)(3y-1)z^2-\big(2(\theta_0+\eta)y-2\theta_0-\eta+\theta_1x\big)z
+\frac12\epsilon(\theta_0+\eta-\theta_{\infty})
\bigg),
\end{split}
\end{equation}
where
$\theta_{\infty}^2=2\alpha$,
$\theta_0^2=-2\beta$,
$\theta_1^2=-2\delta$ $(\theta_1\neq0)$,
$\eta=-\frac{\gamma}{\theta_1}-1$,
$\epsilon=\frac12(\theta_0+\theta_{\infty}+\eta)$.

\begin{remark}
Using the change of the independent variable $t=\log x$, equation (\ref{eq:PVsystem}) will give $\PV$ in the form (\ref{eq:PVlog}). 
\end{remark}


The resolution of the singularities of (\ref{eq:PVsystem}) will lead to the same space of the initial values as described in Section \ref{sec:space}, and shown in Figure \ref{fig:okamoto-blow-down}.

In the limit $x\to\infty$, the fifth Painlev\'e equation (\ref{eq:PV}) becomes:
\begin{equation}\label{eq:PV-auto}
y''=\left(\frac{1}{2y}+\frac{1}{y-1}\right) y'^2+\frac{\delta y(y+1)}{y-1},
\end{equation}
which has a first integral:
$$
\frac{y'^2}{2y(y-1)}+\frac{\delta y}{(y-1)^2}.
$$
The solutions of (\ref{eq:PV-auto}) are elliptic functions satisfying:
$$
y'^2=2y\left(C(y-1)^2 - \delta y\right),
$$
where $C$ is an arbitrary constant.
Notice that $y\equiv0$ is the only solution of that equation taking the value $0$.
Thus, in contrast to the case when $x\to0$, $a_3$ is not a base point for the autonomous equation, while $a_4$ will be a base point for that equation.

Now, analysing the system (\ref{eq:PVsystem}) in the similar way as shown in Sections \ref{sec:infinity} and \ref{sec:limit}, 
it follows in the limit $x\to\infty$, that a general solution of that system has a compact limit set which is invariant with the respect to the autonomous flow.
This implies, as in the proof of Proposition \ref{prop:intersections}, that the flow $(y,z)$ of the vector field (\ref{eq:PVsystem}) meets each of the pole lines $\mathcal{L}_5$, $\mathcal{L}_6$, $\mathcal{L}_{10}$ infinitely many times.
Thus every solution of (\ref{eq:PV}) with essential singularity at $x=\infty$ has infinitely many poles and take the value $1$ infinitely many times in each neighbourhood of that singular point.



%We expect that the asymptotic analysis there will show that a generic solution of the $\PV$ will have infinitely many poles and infinitely many times will take value $1$ as the independent variable $x$ approaches infinity.