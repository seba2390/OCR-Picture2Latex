%!TEX root = paper.tex
%


More and more people these days use online services to consume media.
Whether they are consuming videos or music, users start an interaction
with a service, consume a number of items, and after some time end their interaction.
We refer to this complete interaction as a \textit{user session}, and the time it takes from
start to finish as the \textit{length} of the user session. 

Being able to predict the length of a session is important, because it allows the service
provider to optimize the user experience along with its business goals.
In terms of user experience, session length can be a useful signal for a
recommendation engine. For a music streaming service, where users
typically consume multiple items in a single session, the recommendation system
can be tuned to be more exploratory or exploitative, based on the expected length
of the session. On the business side, services often need to present ads to users
to generate revenue, but there is a limit on how many ads a user will tolerate
within one session \cite{goldstein2013ads}.
The length of a session provides a vital data point for the trade-off between user satisfaction and
generated revenue.
By having an estimate of the length of a session early on, ads can be rescheduled
so that the revenue target (i.e. number of ads presented) is maintained while
minimizing the annoyance to users.

The length of user sessions can be hard to predict, because of
two main factors: First, there exist a number of extraneous
parameters that can affect session length, that are difficult to model
based only on data that are available to an online service.
Sessions can start and end for any number of reasons: users entering/exiting the subway, 
arriving at home or work etc.
Second, user interactions commonly exhibit long-tail distributions; see for example
dwell time studies \cite{liu2010weibull, Kim2014satisfaction},
phone call duration \cite{seshadri2008mobile} and Section \ref{sec:weibull} of this study.
This, in combination with
the lack of predictive features makes it harder
to correctly place the probability mass of predictive models.
In this work we mitigate this issue by using an appropriate objective function for
our model.

Most of the related research has focused on website visits, modeling the time spent on a clicked result (\textit{dwell time}) after a 
search \cite{Kim2014satisfaction, borisov2016catm} or an ad click \cite{Barbieri2016RSFclick, lalmas2015gemini}. This type
of interaction is very different from the way users consume items on a media streaming service.
In a web search or ad click scenario, users enter a query or click on an ad, check the result
and leave the website, in a ``screen-and-glean'' behavior \cite{liu2010weibull}.
In a media streaming service, users may interact very little with the 
platform, but have very long sessions (``lean-back'' behavior),
or have exploratory sessions, constantly revising their selection until settling down
or abandoning the session.
We show in Section \ref{subsec:weibull-analysis}
that 44\% of the users exhibit ``negative-aging'' length distributions, i.e. sessions
that become less likely to end as they grow longer.

To summarize, the key contributions of this study are the following:

\begin{itemize}
    \item We provide an analysis of user session length in an online media streaming service, using the Weibull distribution in Section \ref{sec:weibull}.
    \item We develop a predictive model for session length using contextual and user-based features
    with appropriate objective functions in Section \ref{sec:prediction} and present experimental
    results in Section \ref{sec:experiments}.
\end{itemize}
