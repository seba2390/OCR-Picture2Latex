%!TEX root = paper.tex
%

Survival analysis and prediction of dwell time has come into focus recently,
with many studies using it as a proxy of user satisfaction in
search and ad click scenarios. One of the first studies using
the Weibull distribution analyzed the
dwell time of users
visiting web sites after performing a search \cite{liu2010weibull}. The study indicated a strong
``negative aging'' effect for websites visited after a search
, i.e. the probability of abandoning a page decreased with increasing dwell time (see Sec. \ref{subsec:weibull-review}).

The distribution of dwell time can also be used to measure and improve the
satisfaction of users with search results \cite{Kim2014satisfaction}.
In this study, the authors segment the results according
to attributes like ``readability level'' and model the distribution
of dwell time for each segment. Their findings include that the
dwell time distributions for satisfied and dissatisfied clicks differ,
and that it is possible to use characteristics of these distributions
to better predict satisfied clicks.

Finally, predicted dwell time has also been used to improve the ranking of
ads \cite{Barbieri2016RSFclick, lalmas2015gemini}.
The predicted dwell time is incorporated
into an ``ad-quality'' model, which may include other aspects
of an ad and a user, summarized as the probability of a user clicking on
an ad. The authors develop predictive models and use features about the
user and ad landing page to estimate the dwell time and bounce rate. They perform an evaluation
on historical data and online experiments to measure the effect
that using the ad quality model has on user engagement.
