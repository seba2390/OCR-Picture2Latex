%!TEX root = paper.tex
%

In this section we perform an analysis of the session length distribution
for users in our sample. We provide a brief introduction into Weibull analysis
then move on to the results and discussion.

\subsection{Weibull Distribution Review}

\label{subsec:weibull-review}

The Weibull distribution is attractive for
survival analysis because it allows us to model different kinds of failure rates,
when the probability of failure changes over time. The probability
density function (PDF) of the distribution is:

\begin{equation}
    \label{eq:weibull-pdf}
    f(t) = \frac{k}{\lambda}\left( \frac{t}{\lambda}\right)^{k-1}e^{-(t/\lambda)^k}, t \geq 0
\end{equation}


\begin{figure}
    \centering
    \includegraphics[width=0.47\textwidth]{weibull_hazard.pdf}
    \caption{The failure rate of the Weibull distribution for different values of the shape parameter, $k$. We set $\lambda = 1$.}
    \label{fig:weibull-failure-rate}
\end{figure}

The distribution has two parameters, $k$ and $\lambda$, which correspond to the \textit{shape}
and \textit{scale} of the distribution. The shape, $k$, determines how the elapsed
time affects the rate of failure. The scale, $\lambda$, affects the spread of
the distribution: the larger it is, the more spread out the distribution becomes.

The effect of $k$ can be better illustrated by the \textit{hazard rate} (or hazard function) which gives us the
failure rate of an item that has survived up to time $t$.
For the Weibull distribution it is given by:

\begin{equation}
    \label{eq:weibull-failure-rate}
    h(t) = \frac{k}{\lambda}\left( \frac{t}{\lambda}\right)^{(k - 1)}
\end{equation}

The effect of $k$ is illustrated in Figure \ref{fig:weibull-failure-rate}.
For values $0 < k < 1$ the hazard rate decreases as time increases. This behavior is
often described as ``negative aging'' or ``infant mortality'' failures, where defective units might 
fail early on, but as time goes on and defective units get weeded out, the probability
of a unit failing decreases. 
For $k > 1$ the probability of failure increases with time. This type of failures are 
called ``wear-out'' failures, when units become more likely
to fail with time. For $k = 1$ the failure rate is constant and the distribution
is equivalent to the exponential distribution.


\subsection{Data}

The dataset we use comes from user interaction data from a major ad supported
music streaming service.
We define a user session as a period of continuous listening, demarcated
by breaks or pauses of 30 minutes or longer, i.e. a new session is started 
if a user stops or pauses the music for 30 minutes or more.
We gathered data from a random subset of users for a period of 3 months (February-April 2016),
resulting in 4,030,755 sessions.

In Figure \ref{fig:duration-hist} we can see a histogram for the session length data. For
confidentiality reasons the x-axis has been normalized to 1000 bins.
The distribution is highly skewed to the right, with a very small number of sessions
going all the way up to the cutoff.


\begin{figure}
    \centering 
    \includegraphics[width=0.47\textwidth]{duration_hist.pdf}
    \caption{Histogram plot of session length. The x-axis has been normalized to the 1-1000 range.}
    \label{fig:duration-hist}
\end{figure}

\subsection{Analysis of user session length distribution}

\label{subsec:weibull-analysis}

For our analysis, we fit a Weibull distribution on the data of each user using Maximum Likelihood
Estimation with the \texttt{fitdistrplus} R package \cite{delignette2015fitdistrplus}.

In Figure \ref{fig:shape-ecdf} we can see the empirical cumulative distribution for the
shape parameter. We observe that the users in our sample are split approximately
down the middle, with 44\% of the users exhibiting Weibull distributions with $k <= 1$ and the rest $k > 1$.
Although not directly comparable, we note that for the dwell time on web sites after a search, 98.5\%
of the web sites visited have dwell time distributions with $k < 1$, exhibiting
almost exclusively the ``negative aging'' effect \cite{liu2010weibull}.

One consideration we should note here is that the variability in $k$ could also be caused by sampling
variability between users. We aim to investigate this through hypothesis testing in
an extended version of this work.

\begin{figure}
	\centering
	\includegraphics[width=0.47\textwidth]{weibull_shape_ecdf.pdf}
	\caption{The empirical cumulative distribution for the shape parameter per user.
		The x axis has been truncated at $x=4$ for readability (~99.5 \% of data points shown).}
	\label{fig:shape-ecdf}
\end{figure}
