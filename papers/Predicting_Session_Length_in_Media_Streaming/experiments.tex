\hl{THEO}
In this section we aim to evaluate our session length prediction algorithm. ...

\subsection{Dataset}
\hl{explain the problem of sessionization in online radio.}
The dataset that we used comes from user interaction data from Pandora \hl{define a macro}.
We define a user sessions in online radio
\hl{citation} as a period of continuous listening, demarcated
by breaks or pauses of 30 minutes or longer, i.e. a new session is started 
if a user is inactive (doesn't interact with the application actively) for more than 30 minutes.
%a user resumes play after paused or left the service for 30 minutes or more before.


We collected data of a uniform random subset of the users during one month, resulting
in a dataset of ??? users, with ??? sessions in total. A description of the dataset can be seen
on Table \ref{table:dataset}. % TODO The table could include things like: num_session/user, mean duration etc.

\subsubsection{Data exploration}

%TODO The graphs wll prolly need to be normalized/remove labels for confidentiality.

In order to investigate the predictive power or different features in our dataset we
performed some exploratory analysis focusing on different dimensions of the data.

We start by showing the empirical cumulative distribution of the session duration
data in Figure \ref{fig:eCDF_duration}.

\begin{figure}
	\includegraphics[width=0.5\textwidth]{length-hist-combined.pdf}
	\caption{eCDF of the session duration in seconds.}
	\label{fig:eCDF_duration}
\end{figure}

Demographic features like age and sex are available for most users, so we can make use of those
in order to make better predictions. \hl{explain better what is in the table, some cool thing for the reviewer from the table, and why is necessary to have it.} For example can see how age affects the statistics of session duration
in Table \ref{table:age}.

\begin{table}
	\begin{center}
	\begin{tabular}{|c|c|c|c|}
		\hline
		Age group & \textbf{Mean} & \textbf{SD} & \textbf{Median} \\
		\hline 
		[0, 18] & 2658.76 & 3528.17 & 1536 \\ 
		\hline 
		(18, 25] & 2924.44 & 4108.25 & 1576 \\ 
		\hline 
		(25, 35] & 3729.84 & 5529.86 & 1873 \\ 
		\hline 
		(35, 45] & 4450.44 & 6556.03 & 2139 \\ 
		\hline 
		(45, 65] & 5279.48 & 7452.42 & 2501 \\ 
		\hline 
		(65, $\infty$) & 4997.25 & 7212.07 & 2437.5 \\ 
		\hline 
	\end{tabular} 
	\caption{Statistics on the duration of sessions according to age group.}
	\label{table:age}
	\end{center}
\end{table}

\subsection{Goodness of fit analysis} %TODO: Could also be called parametric survival analysis?
\hl{introduce better why we do goodness of fit, why distribution of time session length matter, how we do it, cite some paper to justify the choice of distributions}

By fitting a distribution to each user we are able to obtain a parametric model that allows
us to answer questions like: ``What is the probability that a session will last longer
than X seconds?'' or ``What is the estimated amount of time a session will last, given its current
duration X'' on a per-user basis.
Being able to answer such question allows for optimization of the recommendation strategies,
in terms of exploration vs. exploitation, and improvement of the user experience for example in
terms of scheduling advertisements.
Having access to a parametric distribution makes answering these types
of questions in real-time trivial.

%In this section we will describe how we performed the fitting of the parametric distributions.

For our investigation we chose to use 3 of the most popular distributions for parametric
survival modeling: The Weibull, log-logistic and gamma distributions \cite{kleinbaum2005}.
These distributions have been used in the past to model the behavior of users in online
platforms for tasks such as modeling the dwell time for website visits \cite{liu2010weibull}, or
the absence time of users returning to websites \cite{dupret2013absence}. %TODO One more ref?

The fitting method we used is the method of moments estimation. %TODO: Cite and expand fitting method. More importantly: Why doesn't MLE work?
We then 
investigate the goodness of fit of the distributions according to two goodness of fit measures:
The Anderson-Darling and Cram\`{e}r von Mises statistics \cite{citation, explain a little bit for the reviewer why they are important}. In Tables \ref{table:cvm} and 
\ref{table:ad} we can see some descriptive statistics about the measures. For both
measures the p-value of the paired Wilcoxon signed rank test is close to 0, so we can reject the
null hypothesis that the shift in between the two distributions is 0.
% TODO: Does the MWW test say anything about direction?

%TODO: No too happy with this data reporting. The figures don't really work either. Should I report that the MWW test is significant, or is it irrelevant?


%TODO: IMPORTANT: Because the Weibull is more flexible, it could be the case that we are just overfitting to each user, hence the better fit. One way to investigate this is to boostrap each user, and check the variance of the fitted parameters. fitdistrplus has support for doing this.
\begin{table}
	\begin{center}
		\begin{tabular}{|c|c|c|c|}
			\hline 
			Distribution & Median & Mean & Max \\ 
			\hline 
			Gamma & 0.0984 & 0.157 & 8.644 \\ 
			\hline 
			Weibull & 0.0833 & 0.123 & 4.371 \\ 
			\hline 
		\end{tabular} 
	\end{center}
	\caption{Statistics for the Cram\`{e}r von Mises goodness of fit measure (lower indicates better fit)}
	\label{table:cvm}
\end{table}


\begin{table}
	\begin{center}
		\begin{tabular}{|c|c|c|c|}
			\hline 
			Distribution & Median & Mean & Max \\ 
			\hline 
			Gamma & 0.7406 & 1.136 & 38.61 \\ 
			\hline 
			Weibull & 0.5588 & 0.7754 & 21.87 \\ 
			\hline 
		\end{tabular} 
	\end{center}
	\caption{Statistics for the Anderson-Darling goodness of fit measure (lower indicates better fit)}
	\label{table:ad}
\end{table}

\subsection{Distribution of model parameters}

%TODO: This is tentative, looks like the best distribution for now is Weibull, but will need to test further.

In the previous section we showed that the Weibull distribution presents the best fit for most
of the users in our sample. In this section we investigate the distribution of the parameters of the
fitted Weibull distributions. What we want to investigate is whether the parameters exhibit some
structure that would give us an idea of a ``typical'' user behavior. These parameters can then for
example be used as starting points for new users for which we have no data.

The Weibull distribution is characterized by two parameters: \textit{shape}, and \textit{scale} ($k$  and $\lambda$ respectively). The
scale parameter gives us an indication about how the ``time-to-failure'' which in our case is the
time until the end of a session evolves over time. A value $k < 1$ suggests an ``infant mortality'' phenomenon: Items
can be defective from the start and fail quickly, but as time goes on, the failure rate decreases as
faulty items get weeded out.This phenomenon is called ``negative aging''. A value of $k = 1$ is equivalent to the
exponential distribution, where the failure rate remains constant. When $k > 1$ the failure rate
increases with time, in our case sessions that last longer are more likely to end sooner, also
called ``positive aging''.

In Figure \ref{fig:shape_ecdf} we can see the empirical cumulative distribution function (eCDF) for the
shape parameter. The $x$ axis has been limited to the $99$th percentile ($k <= 3.5$) to facilitate plotting.

\begin{figure}
	\includegraphics[width=0.5\textwidth]{weibull_shapes_ecdf.pdf}
	\caption{eCDF for the shape parameter of the fitted Weibull distributions (x axis limited at the $99$th percentile, $k \le 3.55$).}
	\label{fig:shape_ecdf}
\end{figure}

What we can see from Figure \ref{fig:shape_ecdf} is that about $25\%$ of the users exhibit
``negative aging'' behavior, i.e. they tend to have sessions that end quickly at first, but after
a certain point, their sessions tend to last longer. %TODO: Not in love with this explanation. What happens is that they have the quick dying sessions, but when they have long sessions, they tend to last longer (if that makes any sense).
The majority of the users have have values $k > 1$, i.e. sessions that exhibit ``positive aging'': As their sessions get longer
they become more likely to end. In Figure \ref{fig:params_2d} we can see how the parameters are concentrated around
specific values, indicating that there is some ``typical'' behavior that users follow in terms of their session length, with a peak
around a $k$ value of $1.2$ and a $\lambda$ value of 2000.

\begin{figure}
	\includegraphics[width=0.5\textwidth]{weibull_params_2d.pdf}
	\caption{Density plot of the parameters for the fitted Weibull distributions.}
	\label{fit:params_2d}
\end{figure}

%TODO: Provide explanation here?
%TODO: Plot the failure rate distribution?
%TODO: Should we include some text about the scale parameter as well, or is an eCDF plot enough?

\subsection{Predictive modeling}

Apart from investigating the fit of the duration to various distributions, the end would be to
be able to make a prediction about the length of a session given features about the user,
the context in which the session was started, and the interactions of the user with the service.

With that in mind we try to model the problem as a regression problem where the dependent
variable is the duration of the session, and we can use variables like the station the user started,
the time of day, and the user's skipping behavior as predictive features. Table \ref{table:features}
provides summaries for the features we included in our investigation.

\begin{tabular}{|p{1cm}|p{1.5cm}|p{1.6cm}|p{3cm}|}
	\hline 
	\textbf{Feature} & \textbf{Type} & \textbf{Category} & \textbf{Description }\\ 
	\hline 
	Age & Numeric & Static & The age of the user. \\ 
	\hline 
	Gender & Categorical & Static & The gender of the user. \\ 
	\hline 
	Start time difference & Numeric & Contextual & The time elapsed sine the user last started a session \\ 
	\hline 
	Previous session duration & Numeric & Contextual & The duration of the previous session. \\ 
	\hline 
	Station & Categorical & Contextual & The station with which the user started the session. \\ 
	\hline 
	Listener subscription state & Categorical & Static & i.e. registered, subscribed, guest... \\ 
	\hline 
	Device category & Categorical & Contextual & The type of device the user is accessing the service with (smartphone, web) \\ 
	\hline 
	Skips per minute & Numeric & Contextual  & The number of songs the user has skipped during the session  \\ 
	\hline 
	App visibility & Numeric  & Contextual  & The ratio of time the app has been in the foreground vs. the background. \\ 
	\hline 
\end{tabular}

%QUANTILE REGRESSION COMPARISON