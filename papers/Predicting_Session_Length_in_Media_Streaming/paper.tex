\documentclass[sigconf]{acmart}

\usepackage{booktabs} % For formal tables
\usepackage[colorinlistoftodos]{todonotes}

\usepackage{csquotes}

\usepackage{epsfig}
\usepackage{color}
\newcommand{\hl}[1]{\colorbox{yellow}{#1}}
\usepackage{balance}  % for \balance command ON LAST PAGE  (only there!)
\usepackage{algorithmic}
\usepackage{algorithm}
\usepackage{multirow}
\usepackage{graphicx}
\usepackage{amsmath}
\usepackage{amssymb}
\usepackage{amsfonts}
\usepackage{xspace}
\usepackage{url}
%\usepackage[hide]{./style/chato-notes}
%\usepackage[T1]{fontenc}

% Paragraphs
\newcommand{\spara}[1]{\smallskip\noindent{\bf #1}}
\newcommand{\mpara}[1]{\medskip\noindent{\bf #1}}
\newcommand{\para}[1]{\noindent{\bf #1}}
\newcommand{\IGNORE}[1]{}

\newcommand{\pandora}{{\sf Pandora}}


%\renewcommand{\enumhook}{\setlength{\topsep}{0.2pt}
%\setlength{\itemsep}{0pt}}

\newcommand{\alg}[1]{\bigbreak\noindent{\bf #1}}
\newcommand{\algskip}{\itemsep=-8pt\baselineskip=0pt}
\newcommand{\commentedtext}[1]{}
\newcommand{\myalgostyle}[1]  {{#1}\xspace}
%% my algorithms
\newcommand{\optselect}{\myalgostyle{OptSelect}}
%\newcommand{\optselect}{OptSelect\xspace}


% C++ typesetting
\def\CC{{C\nolinebreak[4]\hspace{-.05em}\raisebox{.4ex}{\tiny\bf ++}}}

\graphicspath{{./figures/}}

% Remove headers
\fancyhead{}

\begin{document}

% ACM Copyright info
\copyrightyear{2017}
\acmYear{2017}
\setcopyright{acmcopyright}
\acmConference{SIGIR '17}{August 07-11, 2017}{Shinjuku, Tokyo, Japan}\acmPrice{15.00}\acmDOI{10.1145/3077136.3080695}
\acmISBN{978-1-4503-5022-8/17/08}


% ****************** TITLE ****************************************
\title{Predicting Session Length in Media Streaming}

% ****************** AUTHORS **************************************

\author{Theodore Vasiloudis}
\authornote{Part of this work was performed during an internship at Pandora Media Inc.}
\affiliation{
	\institution{RISE SICS}
	\streetaddress{Stockholm, Sweden}}
\email{tvas@sics.se}

\author{Hossein Vahabi}
\affiliation{
	\institution{Pandora Media Inc.}
	\streetaddress{Oakland, USA}}
\email{puya@pandora.com}

\author{Ross Kravitz}
\affiliation{
	\institution{Pandora Media Inc.}
	\streetaddress{Oakland, USA}}
\email{rkravitz@pandora.com}

\author{Valery Rashkov}
\affiliation{
	\institution{Pandora Media Inc.}
	\streetaddress{Oakland, USA}}
\email{vrashkov@pandora.com}

% The default list of authors is too long for headers
\renewcommand{\shortauthors}{T. Vasiloudis et al.}

\begin{abstract}
	Session length is a very important aspect
	in determining a user's satisfaction with a media streaming service. Being able
	to predict how long a session will last can be of great use
	for various downstream tasks, such as recommendations and ad scheduling.
	Most of the related literature on user interaction duration has focused on dwell time for websites,
	usually in the context of approximating post-click satisfaction
	either in search results, or display ads.
	
	In this work we present the first analysis of session length in a
	mobile-focused online service, using a real world data-set from a major music streaming service.
	We use survival analysis techniques to show that the characteristics of the length distributions
	can differ significantly between users, and use gradient boosted trees with appropriate objectives
	to predict the length of a session using only information available at its
	beginning.
	Our evaluation on real world data illustrates that our proposed technique outperforms the considered baseline.

\end{abstract}


\keywords{User Behavior; Survival Analysis; Dwell Time; Session Length}
\maketitle


\section{Introduction}
\label{sec:introduction}
\section{Introduction}  \label{sec:introduction}

\newcommand\inexpIntro[3]{#1?(#2,#3).}
\newcommand\rinexpIntro[3]{*#1?(#2,#3).}
\newcommand\outexpIntro[3]{#1!(#2,#3).}
\newcommand\outatomIntro[3]{#1!(#2,#3)}

We propose a fully automated method for proving termination of \(\pi\)-calculus processes.
Although there have been a lot of studies on termination analysis for the \(\pi\)-calculus
and related calculi~\cite{Deng06IC,Demangeon07,SangiorgiTermination,KobayashiHybrid,Yoshida04IC,DBLP:journals/jlp/DemangeonHS10,Venet98SAS}, most of them have been rather theoretical,
and there have been surprisingly little efforts in developing  fully automated termination
verification methods and tools based on them. To our knowledge,
Kobayashi's \typical{}~\cite{TyPiCal,KobayashiHybrid} is the only exception that
can prove termination of \(\pi\)-calculus processes (extended with natural numbers)
fully automatically, but its termination analysis is quite limited (see Section~\ref{sec:relatedwork}).

Our method is based on a reduction to termination analysis for sequential programs:
we translate a \(\pi\)-calculus process \(P\) to a sequential program \(S_P\), so that
if \(S_P\) is terminating, so is \(P\). The reduction allows us to use
powerful, mature methods and tools
for termination analysis of sequential programs~\cite{heizmann2016ultimate,freqterm,DBLP:conf/lics/PodelskiR04,Kuwahara2014Termination,DBLP:journals/cacm/CookPR11}.

The idea of the translation is to convert a chain of communications on replicated input
channels to a chain of recursive function calls of the target sequential program.
Let us consider the following Fibonacci process:
\begin{align*}
    & \rinexpIntro{\fib}{n}{r}
        \ifexp{n<2}{ \soutatom{r}{1} \\ &\quad}
                   { \nuexp{s_1} \nuexp{s_2} (\outatomIntro{\fib}{n-1}{s_1} \PAR \outatomIntro{\fib}{n-2}{s_2} \PAR \sinexp{s_1}{x}\sinexp{s_2}{y}\soutatom{r}{x+y}) \\}
    & \PAR \outatomIntro{\fib}{m}{r}
\end{align*}
Here, the process
$\rinexpIntro{\fib}{n}{r} \ldots$ is a function server that computes the \(n\)-th Fibonacci number
in parallel and returns the result to \(r\),
and $\outatom{\fib}{m}{r}$ sends a request for computing the \(m\)-th Fibonacci number;
those who are not familiar with the syntax of the \(\pi\)-calculus may wish to consult
Section~\ref{sec:targetlanguage} first.
To prove that the process above is terminating for any integer \(m\),
it suffices to show that there is no infinite chain of communications on $\fib$:
\[
    \fib(m,r) \to \fib(m_1,r_1) \to \fib(m_2,r_2) \to \cdots.
\]
We convert the process above to the following program:\footnote{The actual translation
  given later is a little more complex.}
\begin{verbatim}
 let rec fib(n) = if n<2 then () else (fib(n-1) [] fib(n-2)) in
 fib(m)
\end{verbatim}
Here, \texttt{[]} represents the non-deterministic choice.
Note that, although the calculation of Fibonacci numbers is not preserved,
for each chain of communications on \texttt{fib}, there is a corresponding
sequence of recursive calls:
\[
\mathtt{fib}(m) \to \mathtt{fib}(m_1) \to \mathtt{fib}(m_2) \to \cdots.
\]
Thus, the termination of the sequential program above implies the termination of
the original process.
As shown in the example above, (i) each communication on a replicated input channel
is converted to a function call, (ii) each communication on a non-replicated input
channel is just removed (or, in the actual translation, replaced by a call of
a trivial function defined by \(f(\seq{x})=(\,)\)), and (iii) parallel composition
is replaced by a non-deterministic choice.
We formalize the translation outlined above and prove its correctness.

The basic translation sketched above sometimes loses too much information.
For example, consider the following process:
\begin{align*}
    & \rinexpIntro{\pre}{n}{r} \soutatom{r}{n-1} \\
    & \PAR \rinexpIntro{f}{n}{r} \ifexp{n<0}{ \soutatom{r}{1} }
                                       { \nuexp{s} (\outatomIntro{\pre}{n}{s} \PAR \sinexp{s}{x}\outatomIntro{f}{x}{r}) } \\
    & \PAR \outatomIntro{f}{m}{r}
\end{align*}
The translation sketched above would yield:
\begin{verbatim}
  let pred(n) = n-1 in
  let rec f(n) = if n<0 then () else (pred(n) [] f(*)) in
  f(m)
\end{verbatim}
Here, \texttt{*} represents a non-deterministic integer: since we have removed
the input $\sinatom{s}{x}$, we do not have information about the value of \( x \).
As a result, the sequential program above is non-terminating, although the original
process is terminating.
To remedy this problem, we also refine the basic translation above by using a refinement
type system for the \(\pi\)-calculus. Using the refinement type system,
we can infer that the value of \(x\) in the original process is less than \(n\),
so that we can refine the definition of \texttt{f} to:
\begin{verbatim}
 let rec f(n) = ... else (pred(n) [] let x=* in assume(x<n);f(x))
\end{verbatim}
The target program is now terminating, from which
we can deduce that the original process is also terminating.
We have implemented an automated tool based on the refined translation above.

The contributions of this paper are summarized as follows.
\begin{itemize}
\item The formalization of the basic translation from the \(\pi\)-calculus
  (extended with integers) to sequential programs, and a proof of its correctness.
\item The formalization of a refined translation based on a refinement type system.
\item An implementation of the refined translation, including automated refinement type
  inference based on CHC solving, and experiments to evaluate the effectiveness of
  our method.
\end{itemize}

The rest of this paper is structured as follows.
Section~\ref{sec:targetlanguage} introduces the source and target languages
of our translation.
Section~\ref{sec:approach} 
formalizes the basic translation, and proves its correctness.
Section~\ref{sec:refinement} refines the basic translation by using a refinement type system.
Section~\ref{sec:implementation} reports an implementation and experiments.
Section~\ref{sec:relatedwork} discusses related work,
and Section~\ref{sec:conclusion} concludes the paper.



\section{Related Work}
\label{sec:related}
\textbf{Related work}:
% Object detection related datasets/algo in non-medical domain
% Locally labeled CXR dataset
A few CXR datasets have localized abnormality annotations \cite{shih2019augmenting,filice2020crowdsourcing,jaeger2014two} that are curated manually. These are high quality gold standard ground truth datasets but tend to be smaller in scale (< 30,000 images) and have a narrow coverage, with typically only 1-2 labels. In addition, since most labeling efforts only have abnormality semantics attached, no direct relationships with the affected anatomical locations are available. 

%MEHDI: repeated concepts from above. I am removing the following: 

%The lack of anatomic semantics in the annotation is a limitation for complex multi-modal clinical reasoning work, e.g., differential diagnosis, since clinicians often integrate information along anatomical lines, and for downstream report generation tasks, which often requires describing not only the abnormality but also correctly communicate the location of the abnormalities (and medical devices) to the receiving clinicians. 

Two recent CXR datasets have labels for anatomies described in the reports. In \cite{datta2020dataset}, a small manually annotated dataset (2000 reports) included 10 abnormalities that are individually associated with 29 unique spatial locations (anatomies) at the report level. Another CXR dataset has automatically extracted abnormality and anatomy labels as disconnected concepts that are only correlated at the study level from  160,000 reports using a supervised NLP algorithm \cite{bustos2020padchest}. This was trained on a smaller set of manually annotated data. Neither datasets contain localized annotations for the associated CXR images, nor any comparison relation annotations between sequential exams, both of which are available in the Chest ImaGenome dataset. In Table \ref{tab:related}, we present a comparison of our Chest ImagGenome dataset with other datasets available in the literature.

% Table -- Kashyap

% MEdical imaging datasets to go here: Discussed that we will only focus on cxr datasets that are available for this paper. 
% \caption{\color{red} Kashyap, feel free to continue with the table. We should remove the questionmarks and add a line for our dataset (since all others are not graph). For longer text, using abbreviations and explaining them in the caption often works better. If fill in the values is not possible, it is better to remove the table altogether.}


\begin{table}[t!]
\caption{Summary of existing chest X-ray datasets}
\resizebox{\textwidth}{!}{%
\begin{tabular}{@{}lllllllll@{}}
\toprule
\textbf{Dataset} & \textbf{Annotation Level} & \textbf{Annotation Method} & \textbf{Num Labels} & \textbf{Anatomy Labeled} & \textbf{Graph} & \textbf{Dataset Size} & \textbf{Temporal Labels} & \textbf{Reports} \\ \midrule
SIIM-ACR Pneumothorax Segmentation \cite{filice2020crowdsourcing} & Segmentation & Manual + augmented & 1 & No & No & 12,047 & No & No \\
RSNA Pneumonia Detection Challenge   \cite{shih2019augmenting} & Bounding Boxes & Manual & 1 & No & No & 30,000 & No & No \\
Indiana University Chest X-ray collection \cite{demner2016preparing} & Global & Automated & 10 & No & No & 3,813 & No & Yes \\
NIH CXR dataset \cite{wang2017chestx} & Global & Automated & 14 & No & No & 112,120 & No & No \\
PLCO \cite{team2000prostate} & Global & Automated & 24 & Yes & No & 236,000 & Yes & No \\
Stanford CheXpert \cite{irvin2019chexpert} & Global & Automated & 14 & No & No & 224,316 & No & No \\
MIMIC-CXR \cite{johnson2019mimic} & Global & Automated & 14 & No & No & 377,110 & No & Yes \\
Dutta \cite{datta2020dataset} & Global & Manual & 10 & Yes & Yes & 2,000 & No & Yes \\
PadChest \cite{bustos2020padchest} & Global & Manual + automated & 297 & Yes & No & 160,868 & No & Yes \\
Montgomery County Chest X-ray   \cite{jaeger2014two} & Segmentation & Manual & 1 & Yes & No & 138 & No & No \\
Shenzen Hospital Chest X-ray   \cite{jaeger2014two} & Segmentation & Manual & 1 & Yes & No & 662 & No & No \\  \hline \hline
\textbf{Chest ImaGenome} & Bounding Boxes & Automated & 131 & Yes & Yes & 242,072 & Yes & Yes \\
\bottomrule
\end{tabular}%
}
\label{tab:related}
\vspace{-0.4cm}
\end{table}
% removed (Derived from MIMIC-CXR \cite{johnson2019mimic}) % makes table really small


\section{Weibull analysis of session length}
\label{sec:weibull}
%!TEX root = paper.tex
%

In this section we perform an analysis of the session length distribution
for users in our sample. We provide a brief introduction into Weibull analysis
then move on to the results and discussion.

\subsection{Weibull Distribution Review}

\label{subsec:weibull-review}

The Weibull distribution is attractive for
survival analysis because it allows us to model different kinds of failure rates,
when the probability of failure changes over time. The probability
density function (PDF) of the distribution is:

\begin{equation}
    \label{eq:weibull-pdf}
    f(t) = \frac{k}{\lambda}\left( \frac{t}{\lambda}\right)^{k-1}e^{-(t/\lambda)^k}, t \geq 0
\end{equation}


\begin{figure}
    \centering
    \includegraphics[width=0.47\textwidth]{weibull_hazard.pdf}
    \caption{The failure rate of the Weibull distribution for different values of the shape parameter, $k$. We set $\lambda = 1$.}
    \label{fig:weibull-failure-rate}
\end{figure}

The distribution has two parameters, $k$ and $\lambda$, which correspond to the \textit{shape}
and \textit{scale} of the distribution. The shape, $k$, determines how the elapsed
time affects the rate of failure. The scale, $\lambda$, affects the spread of
the distribution: the larger it is, the more spread out the distribution becomes.

The effect of $k$ can be better illustrated by the \textit{hazard rate} (or hazard function) which gives us the
failure rate of an item that has survived up to time $t$.
For the Weibull distribution it is given by:

\begin{equation}
    \label{eq:weibull-failure-rate}
    h(t) = \frac{k}{\lambda}\left( \frac{t}{\lambda}\right)^{(k - 1)}
\end{equation}

The effect of $k$ is illustrated in Figure \ref{fig:weibull-failure-rate}.
For values $0 < k < 1$ the hazard rate decreases as time increases. This behavior is
often described as ``negative aging'' or ``infant mortality'' failures, where defective units might 
fail early on, but as time goes on and defective units get weeded out, the probability
of a unit failing decreases. 
For $k > 1$ the probability of failure increases with time. This type of failures are 
called ``wear-out'' failures, when units become more likely
to fail with time. For $k = 1$ the failure rate is constant and the distribution
is equivalent to the exponential distribution.


\subsection{Data}

The dataset we use comes from user interaction data from a major ad supported
music streaming service.
We define a user session as a period of continuous listening, demarcated
by breaks or pauses of 30 minutes or longer, i.e. a new session is started 
if a user stops or pauses the music for 30 minutes or more.
We gathered data from a random subset of users for a period of 3 months (February-April 2016),
resulting in 4,030,755 sessions.

In Figure \ref{fig:duration-hist} we can see a histogram for the session length data. For
confidentiality reasons the x-axis has been normalized to 1000 bins.
The distribution is highly skewed to the right, with a very small number of sessions
going all the way up to the cutoff.


\begin{figure}
    \centering 
    \includegraphics[width=0.47\textwidth]{duration_hist.pdf}
    \caption{Histogram plot of session length. The x-axis has been normalized to the 1-1000 range.}
    \label{fig:duration-hist}
\end{figure}

\subsection{Analysis of user session length distribution}

\label{subsec:weibull-analysis}

For our analysis, we fit a Weibull distribution on the data of each user using Maximum Likelihood
Estimation with the \texttt{fitdistrplus} R package \cite{delignette2015fitdistrplus}.

In Figure \ref{fig:shape-ecdf} we can see the empirical cumulative distribution for the
shape parameter. We observe that the users in our sample are split approximately
down the middle, with 44\% of the users exhibiting Weibull distributions with $k <= 1$ and the rest $k > 1$.
Although not directly comparable, we note that for the dwell time on web sites after a search, 98.5\%
of the web sites visited have dwell time distributions with $k < 1$, exhibiting
almost exclusively the ``negative aging'' effect \cite{liu2010weibull}.

One consideration we should note here is that the variability in $k$ could also be caused by sampling
variability between users. We aim to investigate this through hypothesis testing in
an extended version of this work.

\begin{figure}
	\centering
	\includegraphics[width=0.47\textwidth]{weibull_shape_ecdf.pdf}
	\caption{The empirical cumulative distribution for the shape parameter per user.
		The x axis has been truncated at $x=4$ for readability (~99.5 \% of data points shown).}
	\label{fig:shape-ecdf}
\end{figure}


\section{Prediction}
\label{sec:prediction}
%!TEX root = paper.tex

% Intro
Apart from investigating the distribution of session length, ultimately we would like to
be able to predict the length of a session. To that end we gathered
features about the users and sessions, and treated the problem of predicting the
length of a session as a regression problem.


\subsection{Features}

For each of the users and sessions available in our sample we collected
a number of features.
Some features, which we call ``user-based''
are features that we assume do not change between sessions, for example
the gender of a user. Other features which we call ``contextual'' can change
every time a user starts a new session, for example the type of network or device
that a user was using when they started the session, or the length of their last session.
We provide a summary of some of these
features in Table \ref{tab:prediction-features}, separated into user-based and contextual
features.

% Choice of method

\subsection{Model}

\label{subsec:model}


We selected gradient boosted trees (GBTs) \cite{friedman2001gbt} as our model
for a number of reasons: First, because our dataset contains missing data, the algorithm we chose had to be able to handle them explicitly, which decision trees do.

Second, the method should allow for proper modeling of non-negative data.
For such data it is possible to log-transform the dependent and use a squared error objective,
but using an objective function that is better fitted to the distribution of the dependent is
often desirable, which is a common use-case for Generalized Linear Models.
GBTs provide a flexible
optimization framework that allowed us to do just that.
To test both approaches,
we first log-transformed the dependent
and used a root mean squared error objective,
then ran the same experiments again, this time selecting the log-likelihood objective of a Gamma distribution with a log link function,
which allows for explicit modeling of right-skewed, non-negative data. In Section \ref{sec:experiments} we refer to these models
as \textit{linear} and \textit{Gamma} respectively.

Finally, we tested two versions of each model. One aggregated where a single model was created
using the data of all the users, and one per-user, where we separately trained one model
per user, using only the data originating from that user for the training and testing.
This meant that only contextual and not user-level features could be used
to train the per-user models.
This way we tested the trade-off between
the statistical power that a large dataset provides versus having personalized models.

\begin{table}
	\caption{Example user-based and contextual features used in the models.}
	\label{tab:prediction-features}
	\begin{tabular}{ll}
		\toprule
		Feature & Description\\
		\midrule
		Gender & The gender of the user \\
		Age & The age of the user \\
		Subscription Status & Whether the account is ad-supported \\
		\midrule
		Device & The device used for the session \\
		Network & The type of network used for the session \\
		Previous duration & The duration of the user's last session \\
		Absence time & Time elapsed since the last session \\
		\bottomrule
	\end{tabular}
\end{table}

% Experimental setup

\section{Experiments}

\label{sec:experiments}

The baseline model we tested against is the per-user mean session length;
that is, we calculated the mean session length in the training set for each user, and used that value
to make all predictions for each session that user had in the test set. This gave us a baseline
that is simple, but personalized to account for the differences in listening habits between
users.

Because we are focusing on direct length prediction rather than survival probability
\cite{Barbieri2016RSFclick}, thresholded duration classification \cite{lalmas2015gemini}, or
distribution parameter estimation \cite{Kim2014satisfaction, liu2010weibull}, most of the
related work models used in search and ad click scenarios are not directly applicable. Therefore
we don't include them in our comparison.

We used 10-fold cross validation, and stratified our sample per user to ensure that every user
had data points both in the train and test set of each split.

To ensure that we have enough data points per user, we only retained users
that had at least 20 sessions recorded.
The resulting dataset had 3,563,544 sessions.
Due to the size of the dataset we chose to use
the \textit{xgboost} \cite{Chen2016xgboost} variant of GBTs which is
implemented with scalability in mind, utilizing parallel, cache-aware,
and out-of-core computation to handle massive data sets. The parameters
for xgboost were selected through cross-validation on a separate validation set.

% Choice of metric

\subsection{Metrics}

We chose two evaluation metrics to measure the performance of our algorithms.
The first was the Root Mean Square Error (RMSE), which is a common choice for regression problems.
In particular we used the normalized variant of the measure (nRMSE), which is simply the RMSE scaled
by the mean value of the dependent, $\bar{y}$.


Large errors can be observed more often when the distribution
of the dependent variable is highly skewed as it is in our case (see Figure \ref{fig:duration-hist}).
Therefore, we include the Median Absolute Error (MAE) in our analysis
due to its robustness to outliers. This way
a few very large errors will not affect the metric disproportionately, compared to taking the mean.
For confidentiality we normalize all the MAE measurements by the baseline so that it has has an
error of 1, and lower measurements are better.

\subsection{Results}

We report the performance of the various approaches in Table \ref{tab:prediction-metrics}.  As mentioned before, we refer to the models using the RMSE objective as \textit{linear} to avoid confusion
with the nRMSE metric. The linear aggregated model outperforms all models in terms of Median Absolute Error, but cannot beat the baseline on nRMSE.
The aggregated model using Gamma regression has the best performance in terms of nRMSE, but has worse
MAE than its linear counterpart. We note that it's the only model that beats the baseline in
nRMSE.

The per-user linear models outperform the baseline for MAE but not for nRMSE, similarly to the aggregated linear model.
The per-user models using Gamma regression perform similarly to the linear per-user models,
indicating that the change in objective function becomes less important in small data domains.


\begin{table}
	\caption{Performance metrics for length prediction task. We report the
		mean value across the 10 CV folds, and the standard deviation in parentheses.}
	\label{tab:prediction-metrics}
	\begin{tabular}{lll}
		\toprule
		Method (\textit{Objective}) & Normalized MAE & nRMSE \\
		\midrule
		Baseline & 1 \textit{(0.001)} & 1.16 \textit{(0.005)} \\
		Aggregated (\textit{Linear}) & \textbf{0.71} \textit{(0.008)} & 1.23 \textit{(0.008)} \\
		Aggregated (\textit{Gamma}) & 0.93 \textit{(0.007)} & \textbf{1.10} \textit{(0.005)} \\
		Per-user (\textit{Linear}) & 0.83 \textit{(0.002)} & 1.29 \textit{(0.004)} \\
		Per-user (\textit{Gamma}) & 0.86 \textit{(0.001)} & 1.31 \textit{(0.003)} \\
		\bottomrule
	\end{tabular}
\end{table}

What these results indicate is that
the aggregated model using the Gamma regression objective is able to place the mean of the distribution more accurately because it
places more probability mass in the right tail of the distribution. The linear models place more of their probability mass closer
to the origin, allowing them to better capture the shorter sessions that are over-represented in the data, but as
a result miss many of the longer (outlier) sessions. This causes their mean-based metrics to suffer, while median metrics
benefit.

We also see that the per-user models mostly perform worse than their aggregated counterparts. This can be explained by the
fact that per-user models are mostly trained on few data points, and for a flexible model like GBTs, they are likely to
overfit. In this case the trade-off between having a single model trained with all the data versus
having personalized models trained on each user's data favors the aggregated model.

\section{Conclusions}
\label{sec:conclusions}

\section{Discussion and Conclusions}



Our method based on stabilizing forward and backward pass, resulted in improved accuracy over the baseline and it was able to predict optimal dampening, sharpness and tail-fatness before training. 
Our findings are coherent with the line of research that has established that stabilizing gradients and representations at initialization results in better performance \cite{glorot2010understanding, orthogonal_initialization, he2015delving, roberts2022principles, defazio2022scaling, bengio1994learning, hochreiter1997long, hochreiter2001gradient, arjovsky2016unitary, pascanu2013difficulty}. Moreover it gives an initial reply to the question raised by
\cite{surrogate2019, zenke2021remarkable}, which asked  for a theoretical justification of initialization and SG choice for Spiking Neural Networks. With a similar intention, \cite{rossbroich2022fluctuation} proposed an approach that guarantees sparsity of activity at initialization to pick the weights distribution at initialization, resulting in improved accuracy. Our method differs from theirs in that it starts from a principle of stability to derive constraints, instead of a principle of sparsity. It differs also in that we use it to define the SG shape at initialization, not only the weights distribution, and we can show mathematically how weights initialization is intertwined to the SG shape choice. Our results suggest that a tedious hyper-parameter grid-search can be often avoided by making use of sound and established principles of learning.

One of the conditions was designed to hit the most sensitive part of an SG, its center, which resulted in a low sparsity requirement at initialization. This is very uncommon in the Neuromorphic literature, since sparsity brings large energy gains \cite{henderson2020towards,blouw2019benchmarking, 9395703,taulsnn, rossbroich2022fluctuation}.
However, the energy gains of SNNs also come from their binary activity. A matrix-vector multiplication, with a $\mathbb{R}^{m\times n}$ matrix, has an energy cost of $mnE_{MAC}$ for a real vector, and of $mn\rho E_{AC}$ for a binary vector, where $\rho$ is the Bernouilli probability of the binary vector, and in our case the neuron firing rate, and $E_{AC}, E_{MAC}$ are the energies of an accumulate and a multiply-accumulate operation \cite{yin2021accurate, hunger2005floating}. Since MAC are more costly than AC, 31 times on a $45$nm complementary metal–oxide–semiconductor \cite{yin2021accurate, horowitz20141}, we have energy savings with any $\rho$, e.g., when all neurons fire ($\rho=1$) and when they fire half of the time steps ($\rho=1/2$). This gain does not depend on the simulation speed, since it compares a spiking and an analogue computation, at the same computation speed.
Typically requiring more sparsity through a sparsity encouraging loss term, leads to a measurable decrease in performance \cite{zenke2021remarkable, rossbroich2022fluctuation}. However we observed that it is actually possible to achieve higher performance with higher sparsity, by starting with a strong firing rate at initialization, since their synergy acts as a regularization mechanism. This was possible also because the sparsity encouraging loss term was introduced gradually, and because its contribution was kept comparable to the task loss towards the end of training.

We observed that the more complex the task is and the more complex the network to train is, the more drastic is the difference in performance of different SG shapes. It is known that learning is possible with a wide variety of SG shapes \cite{zenke2021remarkable} and the community has not yet settled for one shape or one method to reliably choose which SG to use in each case \cite{surrogate2019}. We showed how to apply a well known stability principle to the forward and backward pass of the simplest Spiking Neural Network, the LIF, as a starting point, but we think that the principles of good Neuromorphic initialization can be further elaborated, in order to tackle more complex tasks and networks.




%\section*{Acknowledgement}
%The authors thank ... for her help with part of the experimental evaluation.

%
% The following two commands are all you need in the
% initial runs of your .tex file to
% produce the bibliography for the citations in your paper.
\bibliographystyle{ACM-Reference-Format}
\bibliography{paper}

\end{document}
