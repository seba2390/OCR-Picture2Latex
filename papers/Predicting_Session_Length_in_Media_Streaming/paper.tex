\documentclass[sigconf]{acmart}

\usepackage{booktabs} % For formal tables
\usepackage[colorinlistoftodos]{todonotes}

\usepackage{csquotes}

\usepackage{epsfig}
\usepackage{color}
\newcommand{\hl}[1]{\colorbox{yellow}{#1}}
\usepackage{balance}  % for \balance command ON LAST PAGE  (only there!)
\usepackage{algorithmic}
\usepackage{algorithm}
\usepackage{multirow}
\usepackage{graphicx}
\usepackage{amsmath}
\usepackage{amssymb}
\usepackage{amsfonts}
\usepackage{xspace}
\usepackage{url}
%\usepackage[hide]{./style/chato-notes}
%\usepackage[T1]{fontenc}

% Paragraphs
\newcommand{\spara}[1]{\smallskip\noindent{\bf #1}}
\newcommand{\mpara}[1]{\medskip\noindent{\bf #1}}
\newcommand{\para}[1]{\noindent{\bf #1}}
\newcommand{\IGNORE}[1]{}

\newcommand{\pandora}{{\sf Pandora}}


%\renewcommand{\enumhook}{\setlength{\topsep}{0.2pt}
%\setlength{\itemsep}{0pt}}

\newcommand{\alg}[1]{\bigbreak\noindent{\bf #1}}
\newcommand{\algskip}{\itemsep=-8pt\baselineskip=0pt}
\newcommand{\commentedtext}[1]{}
\newcommand{\myalgostyle}[1]  {{#1}\xspace}
%% my algorithms
\newcommand{\optselect}{\myalgostyle{OptSelect}}
%\newcommand{\optselect}{OptSelect\xspace}


% C++ typesetting
\def\CC{{C\nolinebreak[4]\hspace{-.05em}\raisebox{.4ex}{\tiny\bf ++}}}

\graphicspath{{./figures/}}

% Remove headers
\fancyhead{}

\begin{document}

% ACM Copyright info
\copyrightyear{2017}
\acmYear{2017}
\setcopyright{acmcopyright}
\acmConference{SIGIR '17}{August 07-11, 2017}{Shinjuku, Tokyo, Japan}\acmPrice{15.00}\acmDOI{10.1145/3077136.3080695}
\acmISBN{978-1-4503-5022-8/17/08}


% ****************** TITLE ****************************************
\title{Predicting Session Length in Media Streaming}

% ****************** AUTHORS **************************************

\author{Theodore Vasiloudis}
\authornote{Part of this work was performed during an internship at Pandora Media Inc.}
\affiliation{
	\institution{RISE SICS}
	\streetaddress{Stockholm, Sweden}}
\email{tvas@sics.se}

\author{Hossein Vahabi}
\affiliation{
	\institution{Pandora Media Inc.}
	\streetaddress{Oakland, USA}}
\email{puya@pandora.com}

\author{Ross Kravitz}
\affiliation{
	\institution{Pandora Media Inc.}
	\streetaddress{Oakland, USA}}
\email{rkravitz@pandora.com}

\author{Valery Rashkov}
\affiliation{
	\institution{Pandora Media Inc.}
	\streetaddress{Oakland, USA}}
\email{vrashkov@pandora.com}

% The default list of authors is too long for headers
\renewcommand{\shortauthors}{T. Vasiloudis et al.}

\begin{abstract}
	Session length is a very important aspect
	in determining a user's satisfaction with a media streaming service. Being able
	to predict how long a session will last can be of great use
	for various downstream tasks, such as recommendations and ad scheduling.
	Most of the related literature on user interaction duration has focused on dwell time for websites,
	usually in the context of approximating post-click satisfaction
	either in search results, or display ads.
	
	In this work we present the first analysis of session length in a
	mobile-focused online service, using a real world data-set from a major music streaming service.
	We use survival analysis techniques to show that the characteristics of the length distributions
	can differ significantly between users, and use gradient boosted trees with appropriate objectives
	to predict the length of a session using only information available at its
	beginning.
	Our evaluation on real world data illustrates that our proposed technique outperforms the considered baseline.

\end{abstract}


\keywords{User Behavior; Survival Analysis; Dwell Time; Session Length}
\maketitle


\section{Introduction}
\label{sec:introduction}
% \leavevmode
% \\
% \\
% \\
% \\
% \\
\section{Introduction}
\label{introduction}

AutoML is the process by which machine learning models are built automatically for a new dataset. Given a dataset, AutoML systems perform a search over valid data transformations and learners, along with hyper-parameter optimization for each learner~\cite{VolcanoML}. Choosing the transformations and learners over which to search is our focus.
A significant number of systems mine from prior runs of pipelines over a set of datasets to choose transformers and learners that are effective with different types of datasets (e.g. \cite{NEURIPS2018_b59a51a3}, \cite{10.14778/3415478.3415542}, \cite{autosklearn}). Thus, they build a database by actually running different pipelines with a diverse set of datasets to estimate the accuracy of potential pipelines. Hence, they can be used to effectively reduce the search space. A new dataset, based on a set of features (meta-features) is then matched to this database to find the most plausible candidates for both learner selection and hyper-parameter tuning. This process of choosing starting points in the search space is called meta-learning for the cold start problem.  

Other meta-learning approaches include mining existing data science code and their associated datasets to learn from human expertise. The AL~\cite{al} system mined existing Kaggle notebooks using dynamic analysis, i.e., actually running the scripts, and showed that such a system has promise.  However, this meta-learning approach does not scale because it is onerous to execute a large number of pipeline scripts on datasets, preprocessing datasets is never trivial, and older scripts cease to run at all as software evolves. It is not surprising that AL therefore performed dynamic analysis on just nine datasets.

Our system, {\sysname}, provides a scalable meta-learning approach to leverage human expertise, using static analysis to mine pipelines from large repositories of scripts. Static analysis has the advantage of scaling to thousands or millions of scripts \cite{graph4code} easily, but lacks the performance data gathered by dynamic analysis. The {\sysname} meta-learning approach guides the learning process by a scalable dataset similarity search, based on dataset embeddings, to find the most similar datasets and the semantics of ML pipelines applied on them.  Many existing systems, such as Auto-Sklearn \cite{autosklearn} and AL \cite{al}, compute a set of meta-features for each dataset. We developed a deep neural network model to generate embeddings at the granularity of a dataset, e.g., a table or CSV file, to capture similarity at the level of an entire dataset rather than relying on a set of meta-features.
 
Because we use static analysis to capture the semantics of the meta-learning process, we have no mechanism to choose the \textbf{best} pipeline from many seen pipelines, unlike the dynamic execution case where one can rely on runtime to choose the best performing pipeline.  Observing that pipelines are basically workflow graphs, we use graph generator neural models to succinctly capture the statically-observed pipelines for a single dataset. In {\sysname}, we formulate learner selection as a graph generation problem to predict optimized pipelines based on pipelines seen in actual notebooks.

%. This formulation enables {\sysname} for effective pruning of the AutoML search space to predict optimized pipelines based on pipelines seen in actual notebooks.}
%We note that increasingly, state-of-the-art performance in AutoML systems is being generated by more complex pipelines such as Directed Acyclic Graphs (DAGs) \cite{piper} rather than the linear pipelines used in earlier systems.  
 
{\sysname} does learner and transformation selection, and hence is a component of an AutoML systems. To evaluate this component, we integrated it into two existing AutoML systems, FLAML \cite{flaml} and Auto-Sklearn \cite{autosklearn}.  
% We evaluate each system with and without {\sysname}.  
We chose FLAML because it does not yet have any meta-learning component for the cold start problem and instead allows user selection of learners and transformers. The authors of FLAML explicitly pointed to the fact that FLAML might benefit from a meta-learning component and pointed to it as a possibility for future work. For FLAML, if mining historical pipelines provides an advantage, we should improve its performance. We also picked Auto-Sklearn as it does have a learner selection component based on meta-features, as described earlier~\cite{autosklearn2}. For Auto-Sklearn, we should at least match performance if our static mining of pipelines can match their extensive database. For context, we also compared {\sysname} with the recent VolcanoML~\cite{VolcanoML}, which provides an efficient decomposition and execution strategy for the AutoML search space. In contrast, {\sysname} prunes the search space using our meta-learning model to perform hyperparameter optimization only for the most promising candidates. 

The contributions of this paper are the following:
\begin{itemize}
    \item Section ~\ref{sec:mining} defines a scalable meta-learning approach based on representation learning of mined ML pipeline semantics and datasets for over 100 datasets and ~11K Python scripts.  
    \newline
    \item Sections~\ref{sec:kgpipGen} formulates AutoML pipeline generation as a graph generation problem. {\sysname} predicts efficiently an optimized ML pipeline for an unseen dataset based on our meta-learning model.  To the best of our knowledge, {\sysname} is the first approach to formulate  AutoML pipeline generation in such a way.
    \newline
    \item Section~\ref{sec:eval} presents a comprehensive evaluation using a large collection of 121 datasets from major AutoML benchmarks and Kaggle. Our experimental results show that {\sysname} outperforms all existing AutoML systems and achieves state-of-the-art results on the majority of these datasets. {\sysname} significantly improves the performance of both FLAML and Auto-Sklearn in classification and regression tasks. We also outperformed AL in 75 out of 77 datasets and VolcanoML in 75  out of 121 datasets, including 44 datasets used only by VolcanoML~\cite{VolcanoML}.  On average, {\sysname} achieves scores that are statistically better than the means of all other systems. 
\end{itemize}


%This approach does not need to apply cleaning or transformation methods to handle different variances among datasets. Moreover, we do not need to deal with complex analysis, such as dynamic code analysis. Thus, our approach proved to be scalable, as discussed in Sections~\ref{sec:mining}.


\section{Related Work}
\label{sec:related}
\section{Related Work}\label{sec:related}
 
The authors in \cite{humphreys2007noncontact} showed that it is possible to extract the PPG signal from the video using a complementary metal-oxide semiconductor camera by illuminating a region of tissue using through external light-emitting diodes at dual-wavelength (760nm and 880nm).  Further, the authors of  \cite{verkruysse2008remote} demonstrated that the PPG signal can be estimated by just using ambient light as a source of illumination along with a simple digital camera.  Further in \cite{poh2011advancements}, the PPG waveform was estimated from the videos recorded using a low-cost webcam. The red, green, and blue channels of the images were decomposed into independent sources using independent component analysis. One of the independent sources was selected to estimate PPG and further calculate HR, and HRV. All these works showed the possibility of extracting PPG signals from the videos and proved the similarity of this signal with the one obtained using a contact device. Further, the authors in \cite{10.1109/CVPR.2013.440} showed that heart rate can be extracted from features from the head as well by capturing the subtle head movements that happen due to blood flow.

%
The authors of \cite{kumar2015distanceppg} proposed a methodology that overcomes a challenge in extracting PPG for people with darker skin tones. The challenge due to slight movement and low lighting conditions during recording a video was also addressed. They implemented the method where PPG signal is extracted from different regions of the face and signal from each region is combined using their weighted average making weights different for different people depending on their skin color. 
%

There are other attempts where authors of \cite{6523142,6909939, 7410772, 7412627} have introduced different methodologies to make algorithms for estimating pulse rate robust to illumination variation and motion of the subjects. The paper \cite{6523142} introduces a chrominance-based method to reduce the effect of motion in estimating pulse rate. The authors of \cite{6909939} used a technique in which face tracking and normalized least square adaptive filtering is used to counter the effects of variations due to illumination and subject movement. 
The paper \cite{7410772} resolves the issue of subject movement by choosing the rectangular ROI's on the face relative to the facial landmarks and facial landmarks are tracked in the video using pose-free facial landmark fitting tracker discussed in \cite{yu2016face} followed by the removal of noise due to illumination to extract noise-free PPG signal for estimating pulse rate. 

Recently, the use of machine learning in the prediction of health parameters have gained attention. The paper \cite{osman2015supervised} used a supervised learning methodology to predict the pulse rate from the videos taken from any off-the-shelf camera. Their model showed the possibility of using machine learning methods to estimate the pulse rate. However, our method outperforms their results when the root mean squared error of the predicted pulse rate is compared. The authors in \cite{hsu2017deep} proposed a deep learning methodology to predict the pulse rate from the facial videos. The researchers trained a convolutional neural network (CNN) on the images generated using Short-Time Fourier Transform (STFT) applied on the R, G, \& B channels from the facial region of interests.
The authors of \cite{osman2015supervised, hsu2017deep} only predicted pulse rate, and we extended our work in predicting variance in the pulse rate measurements as well.

All the related work discussed above utilizes filtering and digital signal processing to extract PPG signals from the video which is further used to estimate the PR and PRV.  %
The method proposed in \cite{kumar2015distanceppg} is person dependent since the weights will be different for people with different skin tone. In contrast, we propose a deep learning model to predict the PR which is independent of the person who is being trained. Thus, the model would work even if there is no prior training model built for that individual and hence, making our model robust. 

%

\section{Weibull analysis of session length}
\label{sec:weibull}
%!TEX root = paper.tex
%

In this section we perform an analysis of the session length distribution
for users in our sample. We provide a brief introduction into Weibull analysis
then move on to the results and discussion.

\subsection{Weibull Distribution Review}

\label{subsec:weibull-review}

The Weibull distribution is attractive for
survival analysis because it allows us to model different kinds of failure rates,
when the probability of failure changes over time. The probability
density function (PDF) of the distribution is:

\begin{equation}
    \label{eq:weibull-pdf}
    f(t) = \frac{k}{\lambda}\left( \frac{t}{\lambda}\right)^{k-1}e^{-(t/\lambda)^k}, t \geq 0
\end{equation}


\begin{figure}
    \centering
    \includegraphics[width=0.47\textwidth]{weibull_hazard.pdf}
    \caption{The failure rate of the Weibull distribution for different values of the shape parameter, $k$. We set $\lambda = 1$.}
    \label{fig:weibull-failure-rate}
\end{figure}

The distribution has two parameters, $k$ and $\lambda$, which correspond to the \textit{shape}
and \textit{scale} of the distribution. The shape, $k$, determines how the elapsed
time affects the rate of failure. The scale, $\lambda$, affects the spread of
the distribution: the larger it is, the more spread out the distribution becomes.

The effect of $k$ can be better illustrated by the \textit{hazard rate} (or hazard function) which gives us the
failure rate of an item that has survived up to time $t$.
For the Weibull distribution it is given by:

\begin{equation}
    \label{eq:weibull-failure-rate}
    h(t) = \frac{k}{\lambda}\left( \frac{t}{\lambda}\right)^{(k - 1)}
\end{equation}

The effect of $k$ is illustrated in Figure \ref{fig:weibull-failure-rate}.
For values $0 < k < 1$ the hazard rate decreases as time increases. This behavior is
often described as ``negative aging'' or ``infant mortality'' failures, where defective units might 
fail early on, but as time goes on and defective units get weeded out, the probability
of a unit failing decreases. 
For $k > 1$ the probability of failure increases with time. This type of failures are 
called ``wear-out'' failures, when units become more likely
to fail with time. For $k = 1$ the failure rate is constant and the distribution
is equivalent to the exponential distribution.


\subsection{Data}

The dataset we use comes from user interaction data from a major ad supported
music streaming service.
We define a user session as a period of continuous listening, demarcated
by breaks or pauses of 30 minutes or longer, i.e. a new session is started 
if a user stops or pauses the music for 30 minutes or more.
We gathered data from a random subset of users for a period of 3 months (February-April 2016),
resulting in 4,030,755 sessions.

In Figure \ref{fig:duration-hist} we can see a histogram for the session length data. For
confidentiality reasons the x-axis has been normalized to 1000 bins.
The distribution is highly skewed to the right, with a very small number of sessions
going all the way up to the cutoff.


\begin{figure}
    \centering 
    \includegraphics[width=0.47\textwidth]{duration_hist.pdf}
    \caption{Histogram plot of session length. The x-axis has been normalized to the 1-1000 range.}
    \label{fig:duration-hist}
\end{figure}

\subsection{Analysis of user session length distribution}

\label{subsec:weibull-analysis}

For our analysis, we fit a Weibull distribution on the data of each user using Maximum Likelihood
Estimation with the \texttt{fitdistrplus} R package \cite{delignette2015fitdistrplus}.

In Figure \ref{fig:shape-ecdf} we can see the empirical cumulative distribution for the
shape parameter. We observe that the users in our sample are split approximately
down the middle, with 44\% of the users exhibiting Weibull distributions with $k <= 1$ and the rest $k > 1$.
Although not directly comparable, we note that for the dwell time on web sites after a search, 98.5\%
of the web sites visited have dwell time distributions with $k < 1$, exhibiting
almost exclusively the ``negative aging'' effect \cite{liu2010weibull}.

One consideration we should note here is that the variability in $k$ could also be caused by sampling
variability between users. We aim to investigate this through hypothesis testing in
an extended version of this work.

\begin{figure}
	\centering
	\includegraphics[width=0.47\textwidth]{weibull_shape_ecdf.pdf}
	\caption{The empirical cumulative distribution for the shape parameter per user.
		The x axis has been truncated at $x=4$ for readability (~99.5 \% of data points shown).}
	\label{fig:shape-ecdf}
\end{figure}


\section{Prediction}
\label{sec:prediction}
%!TEX root = paper.tex

% Intro
Apart from investigating the distribution of session length, ultimately we would like to
be able to predict the length of a session. To that end we gathered
features about the users and sessions, and treated the problem of predicting the
length of a session as a regression problem.


\subsection{Features}

For each of the users and sessions available in our sample we collected
a number of features.
Some features, which we call ``user-based''
are features that we assume do not change between sessions, for example
the gender of a user. Other features which we call ``contextual'' can change
every time a user starts a new session, for example the type of network or device
that a user was using when they started the session, or the length of their last session.
We provide a summary of some of these
features in Table \ref{tab:prediction-features}, separated into user-based and contextual
features.

% Choice of method

\subsection{Model}

\label{subsec:model}


We selected gradient boosted trees (GBTs) \cite{friedman2001gbt} as our model
for a number of reasons: First, because our dataset contains missing data, the algorithm we chose had to be able to handle them explicitly, which decision trees do.

Second, the method should allow for proper modeling of non-negative data.
For such data it is possible to log-transform the dependent and use a squared error objective,
but using an objective function that is better fitted to the distribution of the dependent is
often desirable, which is a common use-case for Generalized Linear Models.
GBTs provide a flexible
optimization framework that allowed us to do just that.
To test both approaches,
we first log-transformed the dependent
and used a root mean squared error objective,
then ran the same experiments again, this time selecting the log-likelihood objective of a Gamma distribution with a log link function,
which allows for explicit modeling of right-skewed, non-negative data. In Section \ref{sec:experiments} we refer to these models
as \textit{linear} and \textit{Gamma} respectively.

Finally, we tested two versions of each model. One aggregated where a single model was created
using the data of all the users, and one per-user, where we separately trained one model
per user, using only the data originating from that user for the training and testing.
This meant that only contextual and not user-level features could be used
to train the per-user models.
This way we tested the trade-off between
the statistical power that a large dataset provides versus having personalized models.

\begin{table}
	\caption{Example user-based and contextual features used in the models.}
	\label{tab:prediction-features}
	\begin{tabular}{ll}
		\toprule
		Feature & Description\\
		\midrule
		Gender & The gender of the user \\
		Age & The age of the user \\
		Subscription Status & Whether the account is ad-supported \\
		\midrule
		Device & The device used for the session \\
		Network & The type of network used for the session \\
		Previous duration & The duration of the user's last session \\
		Absence time & Time elapsed since the last session \\
		\bottomrule
	\end{tabular}
\end{table}

% Experimental setup

\section{Experiments}

\label{sec:experiments}

The baseline model we tested against is the per-user mean session length;
that is, we calculated the mean session length in the training set for each user, and used that value
to make all predictions for each session that user had in the test set. This gave us a baseline
that is simple, but personalized to account for the differences in listening habits between
users.

Because we are focusing on direct length prediction rather than survival probability
\cite{Barbieri2016RSFclick}, thresholded duration classification \cite{lalmas2015gemini}, or
distribution parameter estimation \cite{Kim2014satisfaction, liu2010weibull}, most of the
related work models used in search and ad click scenarios are not directly applicable. Therefore
we don't include them in our comparison.

We used 10-fold cross validation, and stratified our sample per user to ensure that every user
had data points both in the train and test set of each split.

To ensure that we have enough data points per user, we only retained users
that had at least 20 sessions recorded.
The resulting dataset had 3,563,544 sessions.
Due to the size of the dataset we chose to use
the \textit{xgboost} \cite{Chen2016xgboost} variant of GBTs which is
implemented with scalability in mind, utilizing parallel, cache-aware,
and out-of-core computation to handle massive data sets. The parameters
for xgboost were selected through cross-validation on a separate validation set.

% Choice of metric

\subsection{Metrics}

We chose two evaluation metrics to measure the performance of our algorithms.
The first was the Root Mean Square Error (RMSE), which is a common choice for regression problems.
In particular we used the normalized variant of the measure (nRMSE), which is simply the RMSE scaled
by the mean value of the dependent, $\bar{y}$.


Large errors can be observed more often when the distribution
of the dependent variable is highly skewed as it is in our case (see Figure \ref{fig:duration-hist}).
Therefore, we include the Median Absolute Error (MAE) in our analysis
due to its robustness to outliers. This way
a few very large errors will not affect the metric disproportionately, compared to taking the mean.
For confidentiality we normalize all the MAE measurements by the baseline so that it has has an
error of 1, and lower measurements are better.

\subsection{Results}

We report the performance of the various approaches in Table \ref{tab:prediction-metrics}.  As mentioned before, we refer to the models using the RMSE objective as \textit{linear} to avoid confusion
with the nRMSE metric. The linear aggregated model outperforms all models in terms of Median Absolute Error, but cannot beat the baseline on nRMSE.
The aggregated model using Gamma regression has the best performance in terms of nRMSE, but has worse
MAE than its linear counterpart. We note that it's the only model that beats the baseline in
nRMSE.

The per-user linear models outperform the baseline for MAE but not for nRMSE, similarly to the aggregated linear model.
The per-user models using Gamma regression perform similarly to the linear per-user models,
indicating that the change in objective function becomes less important in small data domains.


\begin{table}
	\caption{Performance metrics for length prediction task. We report the
		mean value across the 10 CV folds, and the standard deviation in parentheses.}
	\label{tab:prediction-metrics}
	\begin{tabular}{lll}
		\toprule
		Method (\textit{Objective}) & Normalized MAE & nRMSE \\
		\midrule
		Baseline & 1 \textit{(0.001)} & 1.16 \textit{(0.005)} \\
		Aggregated (\textit{Linear}) & \textbf{0.71} \textit{(0.008)} & 1.23 \textit{(0.008)} \\
		Aggregated (\textit{Gamma}) & 0.93 \textit{(0.007)} & \textbf{1.10} \textit{(0.005)} \\
		Per-user (\textit{Linear}) & 0.83 \textit{(0.002)} & 1.29 \textit{(0.004)} \\
		Per-user (\textit{Gamma}) & 0.86 \textit{(0.001)} & 1.31 \textit{(0.003)} \\
		\bottomrule
	\end{tabular}
\end{table}

What these results indicate is that
the aggregated model using the Gamma regression objective is able to place the mean of the distribution more accurately because it
places more probability mass in the right tail of the distribution. The linear models place more of their probability mass closer
to the origin, allowing them to better capture the shorter sessions that are over-represented in the data, but as
a result miss many of the longer (outlier) sessions. This causes their mean-based metrics to suffer, while median metrics
benefit.

We also see that the per-user models mostly perform worse than their aggregated counterparts. This can be explained by the
fact that per-user models are mostly trained on few data points, and for a flexible model like GBTs, they are likely to
overfit. In this case the trade-off between having a single model trained with all the data versus
having personalized models trained on each user's data favors the aggregated model.

\section{Conclusions}
\label{sec:conclusions}
\section{Conclusions}
\label{sec:conclusions}

In this paper, we apply shared-workload techniques at the \sql level for
improving the throughput of \qaasl systems without incurring in additional
query execution costs. Our approach is based on query rewriting for grouping
multiple queries together into a single query to be executed in one go. This
results in a significant reduction of the aggregated data access done by the
shared execution compared to executing queries independently.

%execution times and costs of the shared scan operator when
%varying query selectivity and predicate evaluation. We observed that for
%\athena, although the cost only depends on the amount of data read, it is
%conditioned to its ability to use its statistics about the data. In some cases
%a wrong query execution plan leads to higher query execution costs, which the
%end-user has to pay. 

%\bigquery's minimum query execution cost is determined by
%the input size of a query.  However, the query cost can increase depending not
%just in the amount of computation it requires, but in the mix of resources the
%query requires.  

We presented a cost and runtime evaluation of the shared operator driving data access costs. 
Our experimental study using the TPC-H benchmark confirmed the benefits of our
query rewrite approach. Using a shared execution approach reduces significantly
the execution costs. For \athena, we are able to make it 107x cheaper and for
\bigquery, 16x cheaper taking into account Query 10 which we cannot execute,
but 128x if it is not taken into account. Moreover, when having queries that do
not share their entire execution plan, i.e., using a single global plan, we
demonstrated that it is possible to improve throughput and obtain a 10x cost
reduction in \bigquery.

%We followed the TPC systems pricing guideline for
%computing how expensive is to have a TPC-H workload working on the evaluated
%\qaasl systems. The result is that even though we are able to reduce overall
%costs a TPC-H workload in 15x for \bigquery (128x excluding query 10 which we
%could not optimize) and in 107x for \athena, the overall price is at least 10x
%more expensive than the cheapest system price published by the TPC.

There are multiple ways to extend our work. The first is
to implement a full \sql-to-\sql translation layer to encapsulate the proposed
per-operator rewrites.  Another one is to incorporate the initial work on
building a cost-based optimizer for shared execution
\cite{Giannikis:2014:SWO:2732279.2732280} as an external component for \qaasl
systems.  Moreover, incorporating different lines of work (e.g., adding
provenance computation \cite{GA09} capabilities) also based on query
rewriting is part of our future work to enhance our system.


%\section*{Acknowledgement}
%The authors thank ... for her help with part of the experimental evaluation.

%
% The following two commands are all you need in the
% initial runs of your .tex file to
% produce the bibliography for the citations in your paper.
\bibliographystyle{ACM-Reference-Format}
\bibliography{paper}

\end{document}
