%!TEX root = AlshehriLiuChenBasar.tex

\begin{figure*}
\centering
\includegraphics[width=.9\linewidth]{sketch_DR2.pdf}
\caption{The interaction between companies and their consumers, along with power allocation. First, companies play a Nash power allocation game. Once power availabilities are allocated across all periods, companies and consumers play the Stackelberg game which dictates optimal prices and demand selection.}
\label{sketch_DR2}
\end{figure*}
In this section, we exploit the closed-form solutions for consumer demands and companies' prices to formulate and solve a power allocation game for companies. We note that while we use the closed-form solutions to define the power allocation game, it is to be played {\em{before}} the Stackelberg game, and its outcomes define the fixed power availabilities in the constraints of the companies in the Stackelberg game. Given the power availabilities from other companies, ${\bf{G_{-k}}}$, and since the equality in (\ref{prob2}) is satisfied at equilibrium, the revenue function of company $k$ can be represented as
\begin{equation} \pi_k({\bf{G}}_k,{\bf{G_{-k}}})=\sum_{t\in \scr{T}}p^*_k(t)G_k(t). \end{equation}  The optimal prices (\ref{p}) are functions of ${\bf{G}}_k$ and ${\bf{G_{-k}}}$, leading to the revenue function being equal to
\begin{equation} B\sum_{t\in \scr{T}}\frac{G_k(t)}{(G_k(t)+N)(KT-\sum_{j\in \scr{K}}\sum_{h\in \scr{T}}\frac{N}{G_j(h)+N})}, \label{Uk} \end{equation}
where $B=\sum_{n\in \scr{N}}B_n$. {Note that company $k$ receives a {\it fraction} of the total budgets. This fraction depends on what company $k$ offers in the multi-period-multi-company demand response framework, and what other companies also offer. Thus, when company $k$ can change what it offers, it can potentially increase the fraction it receives, and the power allocation game becomes natural, since the revenue function depends on other players' decisions. } For this game, which can be played before the Stackelberg game, which we have already solved, companies allocate their powers across all periods, and the outcome dictates the fixed power availabilities for the Stackelberg game. Figure \ref{sketch_DR2} provides an illustration. 

%However, our numerical results show that although multi-period demand response provides incentives for consumer participation, some companies can gain in terms of revenues while others can lose. But the sum of revenues of all companies is a constant that is equal to the sum of budgets (consumers use up all their budgets for demand selection and these budgets eventually go to companies). This conflict of objectives motivates us to formulate a power allocation game and analytically answer the following question: How can company $k$ allocate its power so that it maximizes its revenue, given its total amount of power available $G^{{\rm total}}_k$ for the entire time-horizon?

%By the above optimal prices and demands, and given the power availability from other companies by ${\bf{G_{-k}}}$, the revenue function of company $k$ can be represented as
%\begin{equation*} U_k := U_{k}(G_k,{\bf{G_{-k}}})=\sum_{t\in \scr{T}}p^*_k(t)G_k(t) \end{equation*} By representing the optimal prices (\ref{p}) as functions of $G_k(t)$'s,
%\begin{equation} U_k=B\sum_{t\in \scr{T}}\frac{G_k(t)}{(G_k(t)+N)(KT-\sum_{k\in \scr{K}}\sum_{t\in \scr{T}}\frac{N}{G_k(t)+N})} \label{Uk} \end{equation}
%where $B=\sum_{n\in \scr{N}}B_n$.

Let the total capacity for company $k$ for the entire time horizon be $G^{{\rm total}}_k$. Denote the action set of company $k$ at time $t$ by $\scr{P}_{k,t}:=[0,G^{{\rm total}}_k]$. Thus, given ${\bf{G_{-k}}}$,  the company $k$ solves the following problem:
\begin{eqnarray}
\underset{\mathbf{G}_k }{\hbox{maximize}} && \pi_k({\bf{G}}_k,{\bf{G_{-k}}})
\nonumber \\
\hbox{subject to} && \sum_{t\in \scr{T}}G_k(t) \leq G^{{\rm total}}_k, \\ \label{prob3}
&& G_k(t)\geq0, \ \forall t\in\scr{T}. \nonumber\end{eqnarray}

{\color{black}The above problem is only applicable for the case when generation is fully controllable. For the smart grid, because of the availability of various generation sources, full-controllability does not always hold, and in fact, for renewable resources it could be completely gone. We demonstrate the possibility of relaxing this assumption later in Section \ref{generalizations}.}

\subsection{Existence and Uniqueness of Nash Equilibrium} 
%Note that (\ref{Uk}) is equivalent to
%\begin{equation}  U_{k}(G_k,{\bf{G_{-k}}})= \sum_{t\in \scr{T}}\frac{BG_k(t)}{(G_k(t)+N)(\alpha_{-k}-\sum_{t\in \scr{T}}\frac{N}{G_k(t)+N})} \label{Uk2} \end{equation}
%where \begin{eqnarray*}  \alpha_{-k} & := & KT-\sum_{j\in \scr{K},j\neq k\,\,} \sum_{t\in \scr{T}}\frac{N}{G_j(t)+N} \\
%& > & KT-(K-1)T=T.\end{eqnarray*}
%
%Note that $\alpha_{-k}$ depends on the strategies of other companies and it is fixed for company $k$. 
The following theorem, whose proof can be found in the Appendix, states the existence and uniqueness of Nash equilibrium in the power allocation game, and provides an expression for it.
\begin{theorem}
Under Assumptions \ref{assumption1}-\ref{assumption2}, if ${\bf G}_k$ is fully controllable, there exists a unique pure-strategy Nash equilibrium for the power allocation game,
and it is given by \begin{equation} G^*_k(t)=\frac{G^{{\rm total}}_k}{T} \,\,\,,\,\, \forall \,t\in \scr{T}, \forall \, k\in \scr{K}.\label{NE}\end{equation}\label{allocation}\end{theorem}

Interestingly, the optimal strategy for each company is to equally allocate its power across all time periods. The proof of Theorem \ref{allocation} reveals that (\ref{Uk}) is strictly concave and increasing in each $G_k(t)$. This is an important property that allows accommodating further company-specific operational constraints and relaxing the full-controllability assumption. To illustrate, suppose that company $k$ has a mix of generation sources for which generation is controllable for some periods and only partially controllable for others. Then, it can add linear constraints to problem (\ref{prob3}) reflecting inter-temporal considerations at the generation-side (such as ramping limits). Existence and uniqueness of a pure-strategy Nash equilibrium are still guaranteed due to the strict concavity of the objective \cite{basar}. Since generation costs are typically assumed to be convex \cite{bosebook} (denote it by $c_k$ for each company $k$), company $k$ can also allocate its generation to maximize its profit, by subtracting the cost from (\ref{Uk}). One can alter the objective function of the power allocation game to 
%\begin{equation*} B\sum_{t\in \scr{T}}\frac{G_k(t)}{(G_k(t)+N)(KT-\sum_{j\in \scr{K}}\sum_{h\in \scr{T}}\frac{N}{G_j(h)+N})}\end{equation*}
%   to 
   \begin{align}&B\sum_{t\in \scr{T}}\frac{G_k(t)}{(G_k(t)+N)(KT-\sum_{j\in \scr{K}}\sum_{h\in \scr{T}}\frac{N}{G_j(h)+N})}\nonumber \\
   & \qquad\qquad\qquad \qquad-\sum_{t\in \scr{T}} c_k(G_k(t)),\label{cvx}\end{align}
   
 \noindent and the problem reflects profit-maximization in this case. Using (\ref{cvx}) and following our analysis, {\color{blue} we conclude that each company maximizes a strictly concave function}, and one can easily conclude the existence of a pure-strategy Nash equilibrium in this case. 
%   Furthermore, if $c_k$ is strictly convex, then, it is a unique Nash equilibrium \cite{basar}. 