%!TEX root = main.tex
\subsection{Proof of Theorem 1} Note that \begin{equation*}
    B_n \geq \frac{E_n^{\min}+\zeta_nKT}{\sum_{k\in \scr{K}}\sum_{t\in \scr{T}}\frac{1}{KTp_k(t)}}-\zeta_n \sum_{k\in \scr{K}}\sum_{t\in \scr{T}}p_k(t)
\end{equation*}
is the same as \begin{equation*}
\sum_{k\in \scr{K}}\sum_{t\in \scr{T}}\frac{B_n+\zeta_n \sum_{k\in \scr{K}}\sum_{t\in \scr{T}}p_k(t)}{KTp_k(t)}- \sum_{k\in \scr{K}}\sum_{t\in \scr{T}}\zeta_n \geq E_n^{\min}.\end{equation*}
By (\ref{xx}), this is equivalent to 
$\sum_{k\in \scr{K}}\sum_{t\in \scr{T}}d^*_{n,k}(t)\geq \,E^{\min}_n.$


\subsection{Proof of Lemma 1} The matrix $A$ can be represented as 
{\footnotesize{\begin{equation*}\begin{split} \begin{pmatrix}
KT(G_1(1)+Z)& 0&\dots& 0\\
0 & KT(G_1(2)+Z)& \dots& 0\\
\vdots & \ddots\\
0 &\dots&0 & KT(G_K(T)+Z)
\end{pmatrix}\\+\begin{pmatrix}
-Z\\
-Z\\
\vdots \\
-Z
\end{pmatrix}\begin{pmatrix}
1 \dots 1
\end{pmatrix}:=\hat{A}+uv^T\end{split}\end{equation*}}}
Note that $\hat{A}$ is invertible. Furthermore, 
%\begin{footnotesize}
\begin{equation*}
1+v^T\hat{A}^{-1}u=1-\frac{1}{KT}\sum_{k\in \scr{K}}\sum_{t\in \scr{T}}\frac{Z}{G_k(t)+Z}.
\end{equation*}
%\end{footnotesize}
Since $G_k(t)>0$ and $Z>0$, each element in the summation is less than $1$ and overall value of the summation is less than $KT$, and this clearly leads to $1+v^T\hat{A}^{-1}u\neq0$. By Sherman-Morrison Formula \cite{SM},  if $1+v^T\hat{A}^{-1}u\neq0$, then \begin{equation}A^{-1}=(\hat{A}+uv^T)^{-1}=\hat{A}^{-1}-\frac{\hat{A}^{-1}uv^T\hat{A}^{-1}}{1+v^T\hat{A}^{-1}u}. \label{sm} \end{equation}
Thus, $A$ is invertible and we can apply (\ref{sm}).
% \hfill\qed


\subsection{Proof of Lemma 2} By Lemma 1, the prices are uniquely given by $P=A^{-1}Y$, and by using (\ref{sm}), the price selection for each $k$ at $t$ is 
$$p^*_k(t)=\frac{B}{G_k(t)+Z}\left(\frac{1}{KT-\sum_{j\in \scr{K}}\sum_{h\in \scr{T}}\frac{Z}{G_j(h)+Z}}\right).$$
 Strict positivity follows from $$\frac{B}{G_k(t)+Z}>0  \  \ \text{and} \ \ \sum_{j\in \scr{K}}\sum_{h\in \scr{T}}\frac{Z}{G_j(h)+Z}<KT.$$
%\hfill\qed

\subsection{Proof of Theorem 2}
%Denote the strategy space of utility company $k$ (a leader in the game) at $t$ by $\scr{L}_{k,t}:=[p^{{\rm min}}_k(t),p^{{\rm max}}_k(t)]$. The strategy space of $k$ for the entire time horizon is $\scr{L}_{k}=\scr{L}_{k,1}\times\dots\times\scr{L}_{k,T}$.
% The strategy space of all companies is $\scr{L}=\scr{L}_{1}\times\dots\times\scr{L}_{K}$. For given price selections  $(p_1,\dots,p_K) \in \scr{L}_1 \times \dots \times \scr{L}_K$, the optimal response from all consumers is
%$${\bf{d^*(p)}}=\{d_1^*({\bf{p}}),d_2^*({\bf{p}}),\dots,d_N^*({\bf{p}})\}$$
%where for each $n \in \scr{N}$, $d_n^*({\bf{p}})$ is the unique maximizer for $U_{{\rm consumer},n}(d_n,{\bf{p}})$ and is given by (\ref{xx}).
\begin{itemize}
\item[(i)] By plugging-in the demands given by (\ref{xx}) in the revenue function (\ref{UC}) for $k$, we have 
$$\pi_k=B/K+(Z/K)\sum_{k\in \scr{K}}\sum_{t\in \scr{T}}p_k(t)-Z\sum_{t\in \scr{T}}p_k(t),$$

\noindent
which is concave (linear) in each $p_k(t)$. Thus, by the compactness of $\scr{L}_{k,t}$,  there exists a pure-strategy Nash Equilibrium (NE) \cite{basar}. Next, suppose that a company $k$ deviates from (\ref{p}) and announces a price of $\hat{p}_k(t)=p^*_k(t)+\epsilon$ at a fixed time $t$. If $\epsilon>0$, then
    $$\hat{\pi}-\pi_k=\epsilon  \frac{Z-ZK}{K} \leq 0,$$
    where the inequality follows from $ZK\geq K$. Thus, $k$ has no incentive to increase the prices from (\ref{p}). Furthermore, since the prices given by (\ref{p}) are attained the equality of the capacity constraint in (\ref{prob2}), company $k$ has no incentive to choose $\epsilon <0$ because it will not result in selling more energy. Therefore, for every period $t$, company $k$ does not benefit from deviating from (\ref{p}). Hence, the prices given by (\ref{p}) maximize the revenues and constitute a NE.

%%    $$\frac{\partial^2 U_{gen,k}(p_k(t),{\bf{p_{-k}}})}{\partial p^2_k(t)}=0 \,\,\,\, \forall \,t \in \scr{T}, k \in \scr{K}$$
%%    since the revenue function is convex and concave in $p_k(t)$.
%    Hence, there exists a Nash equilibrium for the noncooperative game at the leaders level \cite{basar}. 
%By Lemma 2, $p_k(t)$ that constitutes the best response of company $k \in \scr {K}$ to prices set by other companies is uniquely given by (\ref{p}). Thus, the Nash equilibrium is unique.

%(ii) From (i), a unique Nash equilibrium exists at which the maximizing prices ${\bf{p^*}}$ are announced to the consumers. 
\item[(ii)] By the uniqueness of the demands given by (\ref{xx}) and using (i), it follows that there exists a unique Stackelberg equilibrium and it is given by the pair ${\bf{d^*(p)}}$ and (\ref{p}).%\item[(ii)] Immediately follows from Lemma 2 and parts (i)-(ii)\end{enumerate}
%\hfill$\qed$
\end{itemize}
\subsection{Proof of Theorem 3} \label{game2proof}
Note that the revenue $\pi_k({\bf{G}}_k,{\bf{G_{-k}}})$ is equivalent  to
\begin{equation} \sum_{t\in \scr{T}}\frac{BG_k(t)}{(G_k(t)+N)(\alpha_{-k}-\sum_{h\in \scr{T}}\frac{N}{G_k(h)+N})} ,\label{Uk2} \end{equation}
where   $$\alpha_{-k}:=  KT-\sum_{j\in \scr{K},j\neq k\,\,} \sum_{h\in \scr{T}}\frac{N}{G_j(h)+N}> T.$$ 
Note that $\alpha_{-k}$ depends on the strategies of other companies and it is fixed for company $k$. 
A pure-strategy Nash equilibrium exists if $\pi_k$ is concave in each $G_k(t)\in\scr{P}_{k,t}$ for each company $k$ and if  $\scr{P}_{k,t}$ is a compact subset of $\mathbb{R}$ \cite{basar}. 
Since it is clear that $\scr{P}_{k,t}$ is compact, it is enough to show concavity of ${ \rm company,k}$. From (\ref{Uk2}), via a sequence of mathematical tricks, 
\begin{align}\pi_k&=\frac{B\sum_{t\in \scr{T}}G_k(t)\prod_{h\neq t}(G_k(h)+N)\frac{G_k(t)+N}{G_k(t)+N}}{\prod_{h\in\scr{T}}(G_k(h)+N)(\alpha_{-k}-\sum_{h\in \scr{T}}\frac{N}{G_k(h)+N})} \nonumber \\
&=\frac{B\prod_{h\in\scr{T}}(G_k(h)+N)\sum_{t\in \scr{T}}\frac{G_k(t)}{G_k(t)+N}}{\prod_{h\in\scr{T}}(G_k(h)+N)(\alpha_{-k}-\sum_{h\in \scr{T}}\frac{N}{G_k(h)+N})} \nonumber \\
&=\frac{B\sum_{t\in \scr{T}}\frac{G_k(t)}{G_k(t)+N}}{\alpha_{-k}-\sum_{h\in \scr{T}}\frac{G_k(h)+N}{G_k(h)+N}+\sum_{t\in \scr{T}}\frac{G_k(t)}{G_k(t)+N}} \nonumber\\
&=B\frac{\sum_{t\in \scr{T}}\frac{G_k(t)}{G_k(t)+N}}{(\alpha_{-k}-T)+\sum_{t\in \scr{T}}\frac{G_k(t)}{G_k(t)+N}} \nonumber \\ 
&=:\frac{f}{\gamma_{-k}+f}.\end{align}
Note that $f_{G_k(t)}=\frac{\partial f}{\partial G_k(t)}=\frac{N}{(G_k(t)+N)^2} > 0$ and
\begin{equation*}\frac{\partial \pi_k}{\partial G_k(t)}=\frac{f_{G_k(t)}\gamma_{-k}}{(\gamma_{-k}+f)^2}=\frac{N\gamma_{-k}}{(\gamma_{-k}+f)^2(G_k(t)+N)^2} >0. \label{der1} \end{equation*}
This leads to
\begin{equation*}\frac{\partial^2 \pi_k}{\partial G_k(t)^2}=\frac{[-2N\gamma_{-k}][(\gamma_{-k}+f)+f_{G_k(t)}(G_k(t)+N)]}{[(\gamma_{-k}+f)(G_k(t)+N)]^2}, \label{der2} \end{equation*}
which is strictly negative since $f, f_{G_k(t)} ,N,\gamma_{-k} >0$.
% \begin{equation*}\frac{\partial^2 U_k}{\partial G_k(t)^2}=\frac{[-2N\gamma_{-k}][(\gamma_{-k}+f)+f_{G_k(t)}(G_k(t)+N)]}{[(\gamma_{-k}+f)(G_k(t)+N)]^2} <0. \label{der2} \end{equation*}
Hence, strict concavity holds. We can relax the non-negativity constraint as the solution will be positive by the properties of the objective function. The Lagrange function for company $k$ is then given by
\begin{equation}
L_{k}({\bf{G}}_k,{\bf{G_{-k}}},\lambda_k)= \pi_k + \lambda_k \left(\sum_{t\in \scr{T}}G_k(t)-G^{{\rm total}}_k\right),
\end{equation}
and by the first-order necessary condition $\nabla L=0$,
\begin{eqnarray}\lambda_k &=&-\frac{N\gamma_{-k}}{(\gamma_{-k}+f)^2(G_k(t)+N)^2},\,\,\,  \forall \,t\in \scr{T}\\
\frac{\partial L_k}{\partial \lambda_k} &=& 0 \implies \sum_{t\in \scr{T}}G_k(t)=G^{{\rm total}}_k.
\end{eqnarray}
Thus, for company $k$, elements of ${\bf G}_k$ must be identical, and must add up to $G^{{\rm total}}_k$.
\subsection{Proof of Theorem 4}
 To find an appropriate $\epsilon^{(i)}_{k,t}$ that leads to the convergence, recall that the prices must be positive. The algorithm diverges whenever any $p^{(i)}_k(t)$ is negative, which might happen when  $\sum_{n\in \scr{N}}d^{(i)}_{n,k}(t)<G_k(t)$, for any company $k\in\scr{K}$ at any time $t\in\scr{T}$ in iteration $i$. To avoid this, it suffices to require 
$ p^{(i)}_k(t)\epsilon^{(i)}_{k,t} > \left |  \sum_{n\in \scr{N}}d^{(i)}_{n,k}(t)-G_k(t)  \right |$  
whenever we have $\sum_{n\in \scr{N}}d^{(i)}_{n,k}(t)<G_k(t)$. This translates into requiring $$p^{(i)}_k(t)\epsilon^{(i)}_{k,t}> G_k(t)-\sum_{n\in \scr{N}}d^{(i)}_{n,k}(t) $$ for any $k\in\scr{K}$, $t\in\scr{T}$, and~$i$. By (\ref{xx}), it follows that we need
\begin{equation}
\epsilon^{(i)}_{k,t}
>\frac{G_k(t)-\sum_{n\in \scr{N}}\left(\frac{B_n+\sum_{j\in \scr{K}}\sum_{h\in \scr{T}}p^{(i)}_j(h)}{KTp^{(i)}_k(t)}-1\right)}{p^{(i)}_k(t)}.
\label{bound}\end{equation}
The bound (\ref{bound}) is the tightest one, but using it to find $\epsilon^{(i)}_{k,t}$ is not implementable. By choosing 
 \begin{equation}\epsilon^{(i)}_{k,t} \geq \frac{G_k(t)+N}{p^{(i)}_k(t)},\label{eps}\end{equation}
 convergence is guaranteed since $$\frac{B_n+\sum_{j\in \scr{K}}\sum_{h\in \scr{T}}p^{(i)}_j(h)}{KTp^{(i)}_k(t)}\geq0.$$
%  \hfill \qed
% By (\ref{eps}) and (\ref{update}), the algorithm converges if the update rule satisfies
% \begin{equation}p^{(i+1)}_k(t)= p^{(i)}_k(t)(1+\frac{\sum_{n\in \scr{N}}d^{(i)}_{n,k}(t)-G_k(t)}{G_k(t)+N}), \label{eps3}\end{equation}\hfill \qed
% which is equivalent to the statement of the theorem.\hfill \qed

\subsection{Proof of Theorem 5} Suppose that $$\frac{K}{N}<\frac{B_n}{p^{{\rm max}}TG^*_k(t)}.$$
By (\ref{pppp}), this implies that $$p^{{\rm max}}<\frac{NB_n}{KTG^*_k(t)}=p^*_k(t),\,\,\, \forall \,t\in \scr{T},\,\, \forall \,k\in \scr{K},$$
and companies will charge $p^{{\rm max}}$, which implies  $$\sum_{k\in \scr{K}}\pi_k=p^{{\rm max}}KTG^*_k(t)<NB_n=\sum_{n\in \scr{N}}B_n,$$
which means that the sum of the revenues is strictly less than the sum of the budgets and hence companies incur losses, compared to the  equilibrium prices. On the other hand, $$\frac{K}{N} \geq \frac{B_n}{p^{{\rm max}}TG^*_k(t)} $$ is equivalent to  $$p^{{\rm max}}\geq \frac{NB_n}{KTG^*_k(t)}=p^*_k(t),\,\,\, \forall \,t\in \scr{T},\,\, \forall \,k\in \scr{K}.$$ Furthermore, we have $$\sum_{k\in \scr{K}}\pi_k=p^*_k(t)KTG^*_k(t)=NB_n=\sum_{n\in \scr{N}}B_n.$$

{\color{black} 
\subsection{Proof of Theorem 6}
From Section \ref{game2proof}, the revenue function $$\pi_k({\bf{G}}_k,{\bf{G_{-k}}})$$ is strictly concave in $G_k(t)$ for each company $k$ at period $t$. Furthermore, since the constraint set is convex and compact in $G_k(t)$, existence of a pure-strategy Nash equilibrium is guaranteed \cite{basar}. The rest of the proof readily follows from Theorem 2.
}


