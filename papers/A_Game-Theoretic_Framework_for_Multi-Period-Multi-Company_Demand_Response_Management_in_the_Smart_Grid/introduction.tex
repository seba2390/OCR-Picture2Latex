%!TEX root = AlshehriLiuChenBasar.tex

{\color{black}
One critical aspect of demand-side management (DSM) in the smart grid is demand response, which is defined as the response of consumers' demands to price signals from the utility companies (see \cite{DR,DSM,DSview} for tutorial discussions). Demand response allows companies to manage the consumers' demands, either directly (through direct load control) or indirectly (through pricing mechanisms). Demand response comes with great benefits, including -but not limited to- improving the electricity market efficiency \cite{DReff}. It also comes with challenges, particularly in its deployment \cite{challengesDR}. For an overview of the methodologies and the challenges of load/price forecasting and managing demand response in the smart grid, see \cite{DRprice}. A comprehensive survey on the pricing methods and optimization algorithms for demand response programs can be found in \cite{surveyDRM}. An overview of integrated demand response, where consumers participate in multiple energy systems is provided in \cite{reviewAE}.

%As the smart grid evolves, consumers are increasingly having more options in terms of where to buy their energies, which makes investigating load adaptive pricing mechanisms in energy systems important.

%. With the use of game theory, advances in local energy trading considering such possible conflicts have been made \cite{pilz}. Depending on the demand response application, it could also be useful to use cooperative game theory, as in \cite{coopDR}. 

 Using the framework of game theory, load adaptive pricing has been introduced decades ago  \cite{loadadapt}. {\color{blue}In this paper, we use game theory to design a multi-period-multi-company demand response management program at which companies and their consumers reach a unique equilibrium. At the equilibrium point, prices and demands are optimally chosen such that companies maximize their  revenues and consumers maximize their utility functions.} {\color{black} For the purpose of this paper, one can think of ``company" as a utility company serving households, businesses, and industrial consumers.} It is of critical interest to capture competition between companies, and hence we utilize the framework of, and tools from noncooperative game theory. We remark that such tools can be useful for the smart grid in various contexts \cite{survey}. 


%While many energy consumers around the world have access to only one company, alternative structures are now becoming a reality \cite{NYISO}. For example, a company called LO3 Energy has begun setting up a small-scale grid operated by consumers that allows peer-to-peer transactions between distributed energy resource owners and demanders in the neighborhood \cite{lo3}. 

A considerable number of contributions have used game theory to analyze what happens in a smart grid where there are multiple sellers/utilities/retailers \cite{walidPHEV,trading,walidPHEV2,gao,twolevel,tansu,yaagoubi,tushar2,sabita,sabita2,sabita3,han} serving the same set of consumers. For example, analysis of how plug-in hybrid electric vehicles can sell back to the grid has been explored in \cite{walidPHEV,trading,walidPHEV2}. A similar analysis has also been carried out for electric bicycles \cite{gao}. A two-level game has been proposed in \cite{twolevel}. The authors in \cite{tansu} introduce a Stackelberg game to capture the interactions between electricity generator owners and a demand response aggregator. In \cite{yaagoubi}, a distributed game between energy consumers of different types has been designed while emphasizing individual preferences. Furthermore, in \cite{tushar2}, analysis of three-party energy management scheme between residential users, a shared facility controller, and the main power grid, has been conducted via a Stackelberg game. Among the contributions in the literature the ones most relevant to this paper are \cite{sabita} and \cite{sabita2}. A single-period Stackelberg game for demand response management with multiple utility companies has been proposed in \cite{sabita}, where consumers choose their optimal demands in response to prices announced by different utility companies. In \cite{sabita2}, an extension to the large population regime was carried out. Variations of \cite{sabita} to user-centric approaches were discussed in \cite{sabita3,han}. These works \cite{walidPHEV,trading,walidPHEV2,gao,twolevel,tansu,yaagoubi,tushar2,sabita,sabita2,sabita3,han} have demonstrated the usefulness and the power of game theory in capturing the interplay between buyers and sellers in the smart grid, but they are limited to single period setups.




In the smart grid, temporal variations play a critical role on both the supply side and the demand side. There are several papers in the literature that have addressed inter-temporal considerations in DSM and demand response \cite{amir,hazem,roh,zhudiff,PAR,collins,repeated,fourstage,dayahead,wei}, such as scheduling of appliances and/or storage \cite{amir,hazem,roh,zhudiff},  peak-to-average ratio reduction \cite{PAR,collins,repeated}, procurement issues \cite{fourstage}, and wholesale market price fluctuations \cite{dayahead,wei}. While the contributions in \cite{amir,hazem,roh,zhudiff,PAR,collins,repeated,fourstage,dayahead,wei} are important and reveal the importance of game theory for multi-period considerations in demand-side management, they are all limited to a single seller/utility/retailer case. 

}

{  {\color{black}The vast majority of demand response contributions are either limited to a single seller case, or a single period one. Furthermore, they primarily focus on either the utility-side or the consumer-side. Our goal here is to alleviate these limitations by developing a multi-period-multi-company demand response framework in which we address the interests and incentives for both utilities and their consumers in the smart grid. We achieve our goal by formulating and solving a Stackelberg game, which is a hierarchical game consisting of two kinds of players, leaders who act first, and these are utility companies in our framework, and followers who respond  to leaders' decisions, and these are price-responsive consumers. We prove that the proposed game admits a unique equilibrium at which companies find their revenue-maximizing prices and consumers choose their optimal demands that maximize their utility functions while taking into account their budget limitations and energy needs across the time horizon. We further propose a distributed algorithm to compute the equilibrium using only local information. The unique equilibrium is computed for the case in which the power available to sell for each company at each period is fixed. Nevertheless, by exploiting the closed-form solutions we derive, we are able to formulate a new power allocation game at which companies solve for allocations that further maximize their revenues, and also prove that it admits a unique equilibrium, and find its analytical expression. The equilibrium of the power allocation game reveals that companies find it optimal to sell the same amount of power at each period. This affirms that our game-theoretic framework aligns with the incentives of utility companies that prefer to minimize the Peak-to-Average ratio. Furthermore, we study what happens in the large population regime where the number of demand-responsive consumers becomes very large, and reveal that the number of companies needs to change appropriately, leading to an appropriate company-to-consumer ratio. {\color{blue} We also study what happens as the number of periods (subdivisions of the time horizon) grows, and show both theoretically and numerically, that consumers' utility increase as the number of periods increases, making multi-period demand response desirable for them.} Since we also address revenue-maximization for companies, this leads to a win-win situation. Furthermore, we provide a theoretical benchmark to measure whether or not consumers are spending more than what is necessary. We validate the applicability of our game to real life data. Numerical studies show that our benchmark leads to billing savings in excess of $10-30\%$, demonstrate the fast convergence of our distributed algorithm, and quantify the effect of the number of periods.  Our work captures the competition between companies, budget limitations at the consumer-level, and energy need for the entire time-horizon. \footnote{{\color{black}Some of the results in this paper were presented earlier in the conference paper \cite{mywork}, but this paper provides a much more comprehensive treatment of the work, such as the inclusion of the power allocation, asymptotic analysis, distributed algorithm, generalizations, and proofs.}} We stress that we make some simplifying assumptions to keep our analysis tractable, which makes it possible to reveal the main insights and gain deep understanding into the interplay between companies and their consumers. We also demonstrate that our framework has  desirable mathematical properties that make generalizations at both the consumers-level and companies-level possible, which we discuss in Section \ref{generalizations}. }


The remainder of the paper is organized as follows. Preliminaries from game theory are provided in Section \ref{prelim}. The problem is formulated in Section \ref{formulation}, and optimal prices and demands are obtained via a Stackelberg game in Section \ref{game1}. In Section \ref{game2}, a power allocation game at the companies side is formulated based on the closed-form solutions of the Stackelberg game. Next, we provide a distributed algorithm for the computation of all optimal strategies using local information in Section \ref{algorithm}. The asymptotic regimes are studied, in which the number of periods or the number of consumers grows in Section \ref{asymp}. Next, we present results on case studies using real demand response data in Section \ref{numerical}. Generalizations are discussed in Section \ref{generalizations}. Finally, we conclude the paper in Section \ref{conclusion} with a recap of its main points and identification of future directions. An appendix at the end provides details of proofs of the five theorems and some auxiliary results. 