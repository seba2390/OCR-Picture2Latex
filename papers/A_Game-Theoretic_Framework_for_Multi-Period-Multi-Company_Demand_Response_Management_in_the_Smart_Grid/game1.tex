%!TEX root = AlshehriLiuChenBasar.tex
In this section, we solve the above optimization problems in closed form and show that the solutions are unique.

%\subsection{Consumer- and Company-side Analyses}

\subsection{Consumer-Side Analysis}
%Note that the consumer-side utility function  is strictly concave and the constraints are linear.
%Refer to \cite{boyd,NL} for details about analyzing and solving such problems.
We start by relaxing the minimum energy constraint (\ref{cc}).
  For each consumer $n\in \scr{N}$, the associated Lagrange function is given as follows:
\begin{eqnarray*}
L_n &=& \gamma_n\sum_{k\in \scr{K}}\sum_{t\in \scr{T}}\ln(\zeta_n+d_{n,k}(t))
 \\
&&- \lambda_{n,1}\left(\sum_{k\in \scr{K}}\sum_{t\in \scr{T}}p_k(t)d_{n,k}(t)-B_n\right) \\
&&+\sum_{k\in\scr{K}} \sum_{t\in\scr{T}} \lambda_{n,2}(k,t)d_{n,k}(t)
\end{eqnarray*}
%\begin{align}\nonumber
%&L_n=\gamma_n\sum_{k\in \scr{K}}\sum_{t\in \scr{T}}\ln(\zeta_n+d_{n,k}(t))
%\cr
%&-\lambda_{n,1}(\sum_{k\in \scr{K}}\sum_{t\in \scr{T}}p(t)_kd_{n,k}(t)-B_n)+ \lambda_{n,2}(1,1)d_{n,1}(1)
%\cr
%&+\lambda_{n,2}(1,2)d_{n,1}(2)+\dots+\lambda_{n,2}(K,T)d_{n,K}(T)\end{align}
  where ${\bf \lambda_{n}}$ are the Lagrange multipliers. The KKT conditions of optimality in this case are sufficient because the objective function is strictly concave and the constraints are linear \cite{NL}, and solving for them leads to \begin{equation}
d^*_{n,k}(t)= \frac{B_n+\sum_{j\in \scr{K}}\sum_{h\in \scr{T}}p_j(h)\zeta_n}{KTp_k(t)}-\zeta_n,  \; \forall \;t\in \scr{T}, \; k\in \scr{K},  \label{xx}
\end{equation}
which is a generalization of the single-period case in \cite{sabita}. A detailed derivation of (\ref{xx}) can be found in \cite{mywork}. {\color{black} We remark that $d^*_{n,k}(t) \geq 0$  {\color{blue} because the objective function is strictly increasing.}}

The following theorem, whose proof can be found in the Appendix, states the necessary and sufficient condition for $B_n$ so that the above demands meet the minimum energy constraint (\ref{cc}).


\begin{theorem}
For each consumer $n \in \scr{N}$, the demands $d^*_{n,k}(t)$ given by (\ref{xx}) satisfy (\ref{cc}) if, and only if, 
\begin{equation} B_n \geq \frac{E_n^{{\rm min}}+\zeta_nKT}{\sum_{k\in \scr{K}}\sum_{t\in \scr{T}}\frac{1}{KTp_k(t)}}-\zeta_n \sum_{k\in \scr{K}}\sum_{t\in \scr{T}}p_k(t). \label{budget} \end{equation}
\end{theorem}

{\color{black}
\begin{remark} The above theorem can be interpreted as billing costs minimization. At the equality of (\ref{budget}), $B_n$ corresponds to the minimum budget needed for consumer $n$ to satisfy his energy need constraint, given the set of prices chosen by utility companies. Such a minimum $B_n$ can serve as a theoretical benchmark in which one can measure whether or not consumers are paying more than what is necessary.  We later demonstrate that with real data from demand response experiments, using the equality in (\ref{budget}) leads to savings in the range of $10\%-30\%$. \hfill $\Box$
\end{remark}


\begin{assumption}
For each consumer $n$, the budget $B_n$ satisfies the condition (\ref{budget}).
\label{assumption1}
\end{assumption}
}
%Now suppose that
%$$\frac{B_n+\zeta_n\sum_{k\in \scr{K}}\sum_{t\in \scr{T}}p_k(t)}{KTp_k(t)} -\zeta_n \geq 0 \,\,\,\, \forall k \in \scr{K},\,\, t \in \scr{T}$$
\subsection{Company-Side Analysis} 
%Given the prices set by the other companies subject to the power availability constraint (\ref{prob2}), each UC (leader) aims to determine its most profitable prices. At the leaders level, there is a noncooperative game in which each UC chooses its optimal prices in response to the prices set by the other UCs. 
We apply the demands derived in the consumers-side analysis (which were functions of the prices) and show that optimality is achieved at the equality of the constraint (\ref{prob2}). We start by solving for prices that satisfy the equality at (\ref{prob2}) and then prove that they are revenue-maximizing, strictly positive, and unique. 
Consider the equality in (\ref{prob2}), and by the optimal demands (\ref{xx}), there holds
$$
\frac{\sum_{n\in \scr{N}}B_n+\sum_{n\in \scr{N}}\zeta_n\sum_{j\in \scr{K}}\sum_{h\in \scr{T}}p_j(h)}{KTp_k(t)} =\sum_{n\in \scr{N}}\zeta_n + G_k(t),
$$
for all $t \in \scr{T}$. 
%for all $t \in \scr{T}$.
Let $Z=\sum_{n\in \scr{N}}\zeta_n$ and $B=\sum_{n\in \scr{N}}B_n$. Then, for each company $k \in \scr{K}$,
\begin{equation} B+Z\sum_{j\in \scr{K}}\sum_{h\in \scr{T}}p_j(h) = KTp_k(t)(G_k(t)+Z),\;\; \forall \;t \in \scr{T}.  \label{AP} \end{equation}
%Note that the double summation includes $p_k(t)$ and all the other prices.
%Thus, \begin{equation}\begin{split}B+Z\sum_{e\in \scr{K}}\sum_{h\in \scr{T}}p_e(h)= KTp_k(t)(G_k(t)+Z)-p_k(t)Z, \\ \,\,\,\, \forall \;t \in \scr{T}, \;\forall \;k \in \scr{K}, \; (e,h)\neq (k,t) \label{AP}\end{split}\end{equation}
The above equation (\ref{AP})  can be presented as the following system of linear equations
\begin{equation}AP=Y,\label{AP2}\end{equation}
%\begin{equation}\underbrace{\begin{pmatrix}
%KT(G_1(1)+Z)-Z& -Z &\dots& -Z\\
%-Z & KT(G_1(2)+Z)-Z& \dots& -Z\\
%\vdots & \ddots\\
%-Z &\dots&-Z & KT(G_K(T)+Z)-Z
%\end{pmatrix}}_{A}
%\underbrace{\begin{pmatrix}
%p_1(1) \\
%\vdots \\
%p_1(T)\\
%p_2(1)\\
%\vdots\\
%p_2(T)\\
%\vdots\\
%p_K(T) \end{pmatrix}}_P=\underbrace{\begin{pmatrix}
%B\\
%B\\
%\vdots \\
%B\end{pmatrix}}_Y \label{AP2}\end{equation}
%
where $A$ is a $KT\times KT$ matrix 
whose diagonal entries are $KT(G_k(t)+Z)-Z$, $k\in\scr{K}$, $t\in\scr{T}$,
and off-diagonal entries all equal to $-Z$, 
%\footnotesize{\begin{align}\nonumber
%&A=\begin{pmatrix}
%KT(G_1(1)+Z)-Z& -Z &\dots& -Z\\
%-Z & KT(G_1(2)+Z)-Z& \dots& -Z\\
%\vdots & \ddots\\
%-Z &\dots&-Z & KT(G_K(T)+Z)-Z
%\end{pmatrix}\end{align}}
$P$ is a vector in $\R^{KT}$ stacking $p_k(t)$, $k\in\scr{K}$, $t\in\scr{T}$,
and $Y$ a vector in $\R^{KT}$ whose entries all equal to $B$.
%\begin{eqnarray*}
%P &=& \begin{pmatrix}
%p_1(1) &
%\cdots &
%p_1(T) &
%p_2(1) &
%\cdots &
%p_2(T) &
%\cdots &
%p_K(T) \end{pmatrix}^T\\
%Y &=&\begin{pmatrix}
%B &
%B &
%\cdots &
%B\end{pmatrix}^T
%\end{eqnarray*}

%The following results say that matrix $A$ is invertible and the revenue-maximizing prices are positive and unique. 
%We also prove that the Nash equilibrium is at these prices.
We have the following results (proofs are in the Appendix).
\begin{lemma}
The matrix $A$ is invertible.
\end{lemma}


\begin{lemma}
The prices that solve (\ref{AP2}) are strictly positive and are unique. For each $t\in\mathcal{T}$, $k\in\mathcal{K}$, the price is given by
    \begin{equation} p^*_k(t)=\frac{B}{G_k(t)+Z}\left(\frac{1}{KT-\sum_{j\in \scr{K}}\sum_{h\in \scr{T}}\frac{Z}{G_j(h)+Z}}\right),\label{p}\end{equation}
where $B=\sum_{n\in \scr{N}}B_n$ and $Z=\sum_{n\in \scr{N}}\zeta_n$.
\end{lemma}


\begin{remark} 
Letting $\zeta_n=1$ for each consumer, the value of $Z$ coincides with $N$. In this case, by (\ref{p}), we observe that for any given ${\bf G_k}$, the price $p^*_k(t)(G_k(t)+N)$ is a constant for all $t \in \scr{T}$ and $k \in \scr{K}$.
Thus, the power availability is inversely proportional to the prices. \hfill$\Box$
%Whenever any of the $G_k(t)$'s changes, the constant on the right side changes, by (\ref{p}).
\end{remark}

\begin{remark} 
Lemma 2 provides a computationally cheap expression for the prices. Since $p^*_k(t)$ can be directly computed using (\ref{p}), there is no need to numerically compute $A^{-1}$ or $|A|$ to solve (\ref{AP2}). This enables us to deal with a large number of periods or utility companies, without worrying about computational complexity.\hfill$\Box$
\end{remark}

{\color{black}Due to production costs and market regulations, $p^*_k(t)$ cannot be outside the range of some lower and upper bounds $[p^{{\rm min}}_k(t),p^{{\rm max}}_k(t)]$  for all $t \in \scr{T}$ and $k \in \scr{K}$, as in \cite{sabita}. If $p^*_k(t)<p^{{\rm min}}_k(t)$, then $p^*_k(t)$ is set to $p^{{\rm min}}_k(t)$, and similarly for the upper-bound, if  $p^*_k(t)>p^{{\rm max}}_k(t)$, then we set  $p^*_k(t)=p^{{\rm max}}_k(t)$. Accordingly, denote the strategy space of utility company $k$ (a leader in the game) at $t$ by $\scr{L}_{k,t}:=[p^{{\rm min}}_k(t),p^{{\rm max}}_k(t)]$. The strategy space of $k$ for the entire time horizon is $\scr{L}_{k}=\scr{L}_{k,1}\times\dots\times\scr{L}_{k,T}$.
 The strategy space of all companies is $\scr{L}=\scr{L}_{1}\times\dots\times\scr{L}_{K}$. {\color{blue} For given price selections  ${\bf{p}}:=({\bf{p}}_1,\dots,{\bf{p}}_K) \in \scr{L}$}, the optimal response from all consumers is
$${\bf{d^*(p)}}=\{{\bf{d}}_1^*({\bf{p}}),{\bf{d}}_2^*({\bf{p}}),\dots,{\bf{d}}_N^*({\bf{p}})\}$$
where for each $n \in \scr{N}$, ${\bf{d}}^{*}_{n}({\bf{p}})$ is the unique maximizer for $u_n({\bf{d}}_n,{\bf{p}})$ and is given by (\ref{xx}).
}
We now have the following theorem, whose proof can be found in the Appendix.


\begin{theorem}[Existence and Uniqueness of the Stackelberg Equilibrium]Under Assumption \ref{assumption1}, the following statements hold: 
\begin{itemize}
\item[(i)] There exists a unique (open-loop) Nash equilibrium for the price-selection game and it is given by (\ref{p}).\\
\item[(ii)] There exists a unique (open-loop) Stackelberg equilibrium, and it is given by the demands in (\ref{xx}) and the prices in (\ref{p}).
\end{itemize}
%\item The maximizing demands given by (\ref{xx}) and the revenue-maximizing prices given in Lemma 2 constitute the (open-loop) Stackelberg equilibrium for the demand response management game.
\label{mainTHM}
\end{theorem}

At the Stackelberg equilibrium, it can easily be verified that  
\begin{equation}\sum_{k\in \scr{K}}\pi_k({\bf{p}}^*_k,{\bf{p}^*_{-k}})=\sum_{n\in \scr{N}}B_n.\label{eq1}\end{equation}
One observation is that when a company gains in terms of revenue, the same amount must be lost by other companies because the sum of revenues is a constant, which demonstrates a conflict of objectives between utility companies. However, by the definition of the equilibrium strategy, this is the best each company can do, for fixed power availabilities ${\bf G_k}$. But, given a total amount of available power, $G^{{\rm total}}_k$, a company has across the time horizon, it is possible that it gains in terms of revenue by an efficient power allocation. This motivates us to formulate a power allocation game and analytically answer the following question: How can company $k$ allocate its power so that it maximizes its revenue? {\color{black} Furthermore, for now, for ease of exposition, we neglect network and other company-specific constraints. Such considerations are later discussed in Section \ref{generalizations}.  For the remaining part of this paper, unless otherwise stated, we also have the following simplifying assumption.

\begin{assumption}
For each consumer $n$, we have $$\gamma_n=\zeta_n=1.$$ \label{assumption2}
\end{assumption}

The above assumption implies that $Z$ is equal to the number of consumers $N$.}


