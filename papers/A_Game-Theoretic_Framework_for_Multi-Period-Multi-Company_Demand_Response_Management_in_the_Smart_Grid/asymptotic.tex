%!TEX root = main.tex
{\color{black}In this section, we study the asymptotic (limiting) behavior as $T\rightarrow \infty$ or $N\rightarrow \infty$. While neither $T$ or $N$ can be arbitrarily large in practice, analyzing the asymptotic behavior brings in deep insights. For example, it reveals that consumers benefit as $T$ grows. As $N$ grows, our asymptotic analysis allows us to compute an appropriate company-to-consumer ratio $\frac{K}{N}$. We show these insights by studying how the utility functions, revenues, prices, and demands are affected as $T$ or $N$ grows. {For the rest of this section, in addition to Assumptions \ref{assumption1}-\ref{assumption2}, we assume the following. 
\begin{assumption} 
The total power available for the entire time horizon $G_k^{{\rm total}}$ is the same for each company $k\in\scr{K}$.
\label{assumption3}
\end{assumption}
}}
\subsection{When the Number of Periods Grows} Under Assumptions \ref{assumption1}-\ref{assumption3}, at equilibrium, it follows that 
%we have
%\begin{equation*}\sum_{j\in \scr{K}}\sum_{h\in \scr{T}}\frac{N}{G^*_j(h)+N}=KT\frac{N}{G^*_k(t)+N} \label{GT}\end{equation*}
%\begin{equation*}\sum_{j\in \scr{K}}\sum_{h\in \scr{T}}p^*_j(h)=KTp^*_k(t) \label{GT2}\end{equation*}
%Furthermore, 
optimal prices and demands are given by 
\begin{equation} p^*_k(t)=\frac{\sum_{m\in \scr{N}}B_m}{KTG^*_k(t)}=\frac{\sum_{m\in \scr{N}}B_m}{KG^{{\rm total}}_k}, \label{ppp} \end{equation}
\begin{equation}d^*_{n,k}(t)=\frac{B_n+KTp^*_k(t)}{KTp^*_k(t)}-1=\frac{G^{{\rm total}}_kB_n}{T\sum_{m\in \scr{N}}B_m},\,\, \label{t1} \end{equation}
%By (\ref{t1}), the payoff of consumer $n$ becomes
and the utility of consumer $n$ becomes\begin{equation} u_n=KT\ln\left(1+\frac{G^{{\rm total}}_kB_n/\sum_{m\in \scr{N}}B_m}{T}\right),\end{equation}
in which $G^{{\rm total}}_kB_n/\sum_{m\in \scr{N}}B_m$ is positive. Thus, as $T$ increases, the multiplicative term $ KT$ of the logarithmic function increases at a faster rate than the decrease of 
$\ln\left(1+{B_nG^{{\rm total}}_k/B}/{T}\right)$. 
Hence, as $T$ increases, the utility of each consumer $n \in \scr{N}$ monotonically increases.
Taking the limit, it can be verified that 
\begin{equation}\lim_{T\rightarrow\infty} u_n(T)=\frac{KG^{{\rm total}}_kB_n}{\sum_{m\in \scr{N}}B_m}.\end{equation}
Furthermore, note that the demand $d^*_{n,k}(t)$ from consumer $n\in \scr{N}$ from company $k\in \scr{K}$ at time $t\in \scr{T}$ converges to zero as $T \rightarrow \infty$. We claim that the revenues are constants. To see this, recall that
\begin{align*}
\pi_k({\bf{p}}^*_k,{\bf{p}^*_{-k}}) &= p^*_k(t)G^{{\rm total}}_k = \frac{\sum_{m\in \scr{N}}B_m}{K},
\end{align*}
which is a constant since both the number of companies and the budgets of the consumers are fixed.

\begin{remark} At the equilibrium, the monotonicity of the utilities of the consumers shows that increasing the number of periods leads to more incentives for consumers' participation in demand response. However, it might not be very beneficial to increase the number of periods to a very high value. First, the rate of increase in terms of consumers' utilities gets progressively smaller. Second, having a high number of periods leads to smaller demands for each period and that might violate some minimum energy need for particular periods at the consumers' level. So, it is beneficial to increase the number of periods up to a certain point (compared to having $T=1$), but it might not be beneficial to let $T$ become arbitrarily large. 
%t would be interesting to study what would be the appropriate number of periods that keeps consumers motivated to participate in demand response  while being practical.
\hfill$\Box$
\end{remark}


\begin{remark} Note that the limit point of the utility function of consumer $n$ is the proportion of his budget to the total budgets times the total power availability. So if a particular consumer has $1\%$ of the sum of all the budgets, he gets $1\%$ of the available power. Furthermore, the revenue for each company is the proportion of the sum of the budgets to the number of companies. In addition, the demand by consumer $n$ from company $k$ at time $t$ is the proportion of his budget to the total budgets times the total power availability at $t$ from $k$.
\hfill$\Box$
\end{remark}



\subsection{When the Number of Consumers Grows} When the number of consumers increases, each additional consumer has some budget $B_n$. With the total power availability from companies being fixed, they will increase their prices. {\color{black}  We have the following simplifying assumption. 

\begin{assumption} 
The budget for each consumer $n \in \scr{N}$ is the same.
\label{assumption4}
\end{assumption}



Under Assumptions \ref{assumption1}-\ref{assumption4}, we increase the number of consumers $N$ and see what happens as $N \rightarrow \infty$.} In this case, the optimal prices and demands become
%\begin{eqnarray}p^*_k(t)&=&\frac{NB_n}{KTG^*_k(t)} \in \scr{L}_{k,t}:=[p^{{\rm min}}_k(t),p^{{\rm max}}_k(t)]\label{pppp}\\
\begin{eqnarray}p^*_k(t)&=&\frac{NB_n}{KTG^*_k(t)} \label{pppp}\\
d^*_{n,k}(t) &=& \frac{G^{{\rm total}}_k}{TN} \label{dd} \end{eqnarray}
Clearly, $p^*_k(t)\rightarrow\infty$ as $N\rightarrow\infty$ and $d^*_{n,k}(t)\rightarrow0$ as $N\rightarrow\infty$. When the population is large and the power availability is fixed, it is not surprising that $d^*_{n,k}(t)\rightarrow0$ because the portion each consumer can get from the available power gets smaller and smaller as $N$ increases. Furthermore, it can be easily verified that $\lim_{N\rightarrow \infty}\pi_k(N)=\infty$ and $\lim_{N\rightarrow \infty}u_n(N)=0$. Thus, with the limit points resulting in unrealistic outcomes, a balance between the supply and demand needs to be achieved, which we do by finding an appropriate company-to-consumer ratio. 

Now, the question we ask is: For a given maximum allowable market price $p^{{\rm max}}_k(t)$, call it $p^{{\rm max}}$, what is the appropriate company-to-consumer ratio $\frac{K}{N}$? If there are more companies than necessary in the market, there will be losses in terms of revenues. On the other hand, if there are fewer companies than necessary, the prices can exceed $p^{{\rm max}}$, leading to undesirable outcomes. The following theorem, whose proof can be found in the Appendix, provides an optimal ratio at which prices do not exceed $p^{{\rm max}}$ and the revenues being maximized while satisfying the equality in (\ref{eq1}). 
{\color{black}
\begin{theorem}
Under Assumptions \ref{assumption1}-\ref{assumption4}, at the NE of the power allocation game, and at the Stackelberg equilibrium of the price and demand selection game, the optimal prices given by (\ref{p}) satisfy  
\begin{eqnarray*}
 p^*_k(t) &\leq &p^{{\rm max}},\\
 \sum_{k\in \scr{K}}\pi_k({\bf{p}}^*_k,{\bf{p^*_{-k}}})&=&\sum_{n\in \scr{N}}B_n,
\end{eqnarray*}
if, and only if, \begin{equation*} \frac{K}{N} \geq \frac{B_n}{p^{{\rm max}}TG^*_k(t)}, \end{equation*}
for each $t \in \mathcal{T}$ and $k\in\mathcal{K}$.
\end{theorem}}

%\begin{theorem}
%Suppose that the total power availabilities $G^{{\rm total}}_k$ for all companies are the same. Then, at the NE of the power allocation game, and at the Stackelberg equilibrium of the price and demand selection game, the following hold:\\
%(i) If  $N\leq {p^{{\rm max}}KG^{{\rm total}}_k}/{B_n}$, then the optimal prices $p^*_k(t)$'s given by (\ref{pppp}) are feasible and the supply-demand balance (\ref{prob2}) is satisfied.\\
%(ii) If  $N>{p^{{\rm max}}KG^{{\rm total}}_k}/{B_n}$, then the optimal company-to-consumer ratio that maximizes the revenues without exceeding $p^{{\rm max}}$ is 
%$K/N=B_n/(p^{{\rm max}}TG^*_k(t))$.
%\end{theorem}

