%!TEX root = AlshehriLiuChenBasar.tex


In this section, we present results on some case studies on representative days from a Dutch smart grid pilot \cite{dutch} and the EcoGrid EU project \cite{ecogrid}. We numerically study optimal prices and demands, and their corresponding payments and utility functions. We show how our approach results in monetary savings for consumers. Furthermore, we show that increasing the number of periods provides more incentives for consumers' participation in demand response management. Additionally, we demonstrate the fast convergence of our distributed algorithm to optimal prices. We also release an open-source interactive tool containing the simulations in \cite{tool2}. For both the Dutch smart pilot and the EcoGrid EU projects, the data are unavailable in raw format. Thus, whenever it is needed, we estimate some data points from figures available in the corresponding references \cite{dutch,ecogrid}. 

Recall that at the Stackelberg equilibrium, the total power availabilities ${\bf G}$ match the aggregate demands. That is, 
$$ \sum_{n \in \mathcal{N}} d^*_{n,k}(t)=G_k(t), \qquad \forall t\in\mathcal{T}, k\in\mathcal{K}.$$ Here, we use the experimental hourly variation of the total demands to choose values for ${\bf G}$ and the minimum energy need ${\bf E}^{\min}$. This allows us to establish a common aspect between our results and the experimental results, so that we can appropriately explore how our framework compares to real-life experiments. We also use the lower-bound on the minimum budget condition (\ref{budget}), so that we can also quantify potential savings.  
{\color{blue}  From the consumers' perspective, the prices are given parameters in both our model and the experimental setups. The optimal demands are functions of the prices, and the optimal prices naturally depend on the parameters of the consumers and companies. To bring deep insights, we make the differentiating aspect between our model and the experimental results an economic one. And hence, we pick the parameters such that the equilibrium demands and experimental ones are similar, but the prices, and essentially what consumers pay, are different. Utilizing Theorem 1, we conclude that the equilibrium prices bring savings to consumers, and by definition, they automatically consider the incentives of companies as they are revenue-maximizing. A main conclusion of this paper is that this quantifies the economic gap, in terms of consumer savings, between our game-theoretic benchmark and existing experimental results. On the other hand, in our analysis, we have relaxed some constraints for tractability, such as power flow and demand inelasticity considerations, and it remains open to explore the underlying tradeoffs, since adding these considerations might reduce the potential monetary savings for consumers. Such considerations were not directly included in the models studied in \cite{ecogrid,dutch}, as their focus was to experimentally explore the consumers' behavior in response to changing prices. It is worth mentioning that the results in \cite{dutch} revealed that consumers are mainly flexible about adjusting the consumption of white goods (washing machine, dishwasher, etc). It was also concluded in \cite{ecogrid} that demand response did not result in distribution feeder congestion relief, and consumers with automatic equipment were the most responsive ones. Nevertheless, later in Section IX, we demonstrate how our framework can be utilized to include additional network and consumer-specific and/or company-specific constraints, which make it possible to add constraints for congestion relief. Including such constraints will likely make it necessary to compute the equilibrium prices and demands algorithmically, which we relegate to future endeavors, as we emphasize here more on revealing deep insights via having tractable analysis.}
 \begin{figure*}
\centering
\includegraphics[width=\linewidth, height=2in]{EcoGridEU_Price_Power.pdf}
\caption{Total power offered by company (left), Stackelberg game and EcoGrid EU experimental prices (middle), and the cumulative payments and billing savings for all consumers (right). }
\label{EcoGridEU_Price_Power}
\end{figure*}


\subsection{EcoGrid EU Project} This demand response project was conducted from March 2011 to August 2015 in Bornholm, Denmark. The number of consumers in this experiment was approximately $2000$. For a representative day (December 5th, 2014), we apply our method to hourly prices and shiftable demand consumption from this experiment. The experimental prices are in  \text{DKK}/MWh and we scale them to  \text{DKK}/kWh. We start by assuming that there is only one company ($K=1$) and letting the consumers to be homogeneous (they have the same budgets and energy need) with $N=2000$, and then generalize the results to $K>1$ and heterogeneous consumers. Since we are taking hourly prices for a day, we have $T=24$. 

\subsubsection{Finding the necessary parameters}In our model, for each period $t$, we have a fixed power availability $G_1(t)$ on the supply-side. Also, for each consumer $n$, his minimum demand $E^{\min}_n$ and budget $B_n$ are fixed for the entire horizon. These are necessary parameters that need to be known to solve for optimal demands and prices. We let the power availabilities ${\bf G}_1$ match the experimental hourly variation of the total demand. For the entire time-horizon, we have  $$\sum^{2000}_{n=1} E^{\min}_n=\sum^{24}_{t=1}G_1(t) \approx 54 \ \text{{\color{blue}MWh}}.$$ For homogenous consumers, it follows that $$E^{\min}_n= \frac{\sum^{24}_{t=1}G_1(t)}{2000} \approx 27 \ \text{{\color{blue}kWh}}.$$ Next, using Theorem 1, we plug-in $E^{\min}_n$ and the experimental hourly prices in (\ref{budget}) to find the minimum budget need, which is $B_n\approx 7.6 \ \text{ \text{DKK}}$, for each $n$. 
   
 \subsubsection{Numerical Results} Now, using the parameters found above, we can compute the optimal demands and prices for the Stackelberg game using (\ref{xx}) and (\ref{p}), and study their effects. 
 
In Figure \ref{EcoGridEU_Price_Power}, we plot the total power availabilities ${\bf G}_1$, the prices found experimentally and using the Stackelberg game, and the corresponding total payments by all consumers for their demands. Our approach leads to prices that have a slightly smaller mean than in the experiment and a significantly smaller variance, which is a desirable property \cite{IFAC}. {\color{black} At the equilibrium point, as stated in Remark 2, we observe that $$p^*_k(t)(G_k(t)+N)=p^*_k(t)\Bigg(\sum_{n\in\mathcal{N}}  d^*_{n,k}(t)+N\Bigg)$$  is a constant for each period $t$ and each company $k$. Hence, whenever company $k$ at time $t$ has a large amount of power available to sell $G_k(t)$, it would lower its price, and vice versa. Here, consumers are attracted to buy more whenever the price is low, and will buy less whenever the price is high, which is intuitive.} One advantage our approach has is that it results in billing savings for consumers, as we show in Figure \ref{EcoGridEU_Price_Power} (this demonstrates the importance of Theorem 1, which we use to find the minimum budget need for the consumers). {\color{blue} Here, the equilibrium demands are similar to the experimental values, but since the prices differ,  consumers receive the same amount of energy at smaller costs}. This would lead to more monetary incentives for active consumer participation in demand response management, while being consistent with the company's objectives, since the Stackelberg game prices found using (\ref{p}) are revenue-maximizing as shown in the proof of Theorem 2.
\begin{figure*}
\centering
\includegraphics[width=\linewidth, height=4in]{Influence_of_T.pdf}
\caption{The effects of varying the number of periods for companies (with different market shares and at Nash equilibrium of the power allocation game) and heterogeneous consumers (with different budgets) using the EcoGrid EU experimental data.}
\label{Influence_of_T}
\end{figure*}
\begin{figure*}
\centering
\includegraphics[width=\linewidth, height=2in]{algorithm11.pdf}
\caption{ Distributed algorithm's performance (Theorem 4 requires $\delta\geq0$) using the EcoGrid EU experimental data.}
\label{algorithm1}
\end{figure*}

Next, we make consumers heterogeneous and increase the number of companies. We differentiate between consumers by varying their budgets, and take 5 classes of consumers, as in the EcoGrid EU experiment. We let consumers' budgets be $B_{1-400}=4  \ \text{DKK}$, $B_{401-800}=5  \ \text{DKK}$, $B_{801-1200}=6 \  \text{DKK}$, $B_{1201-1600}=7  \ \text{DKK}$, and $B_{1601-2000}=8 \ \text{DKK}$. We also let the number of companies be $K=4$, which is consistent with the actual energy sources used in the experiment. Precisely, the system is powered by 61\% wind energy ($k=1$), 27\% biomass ($k=2$), 9\% solar energy ($k=3$), and 3\% biogas ($k=4$). We split the {\color{black}total need ($54 \ \text{MWh}$)} among the energy sources according to experimental proportions, assuming that each energy source is owned by a single company that acts as a company in our game. 
 
With the above setup, we study the effect of varying the number of periods $T$ from $1$ to $50$. To do this, we need to find a way for companies to allocate their total power across the time horizon for each fixed $T$, which can be done by using Theorem 3, which states that equally splitting the total power across the time horizon for each company $k$ constitutes a unique Nash equilibrium for the power allocation game (it is also shown to be the global maximizer in the proof). 
%Thus, at the Nash equilibrium, for each $k$ with total power $G^{total}_k$, in each period $t$, we have $G^*_k(t)=G^{total}_k/T$.
 \begin{figure*}
\centering
\includegraphics[width=\linewidth, height=2in]{Netherlands_Price_Power.pdf}
\caption{Average consumer demand (left), Stackelberg game and Dutch pilot prices (middle), and the cumulative payments and billing savings for average consumer (right).}
\label{Netherlands_Price_Power}
\end{figure*}
\begin{figure*}
\centering
\includegraphics[width=\linewidth, height=2in]{algorithm2.pdf}
\caption{Distributed algorithm's performance (Theorem 4 requires $\delta\geq0$) using the Dutch pilot data.}
\label{algorithm2}
\end{figure*}
Figure \ref{Influence_of_T} shows the influence of varying the number of periods on prices, power allocated, revenues, and consumer utilities. We observe that as $T$ increases, the power allocated at each period gets progressively smaller. On the other hand, prices can increase or decrease, depending on the company, and they converge to positive constants. Furthermore, revenues might also increase or decrease, depending on the company (note that the company that achieves the highest revenue is the one that offers the lowest prices, and vice-versa). In view of (\ref{eq1}), the sum of revenues at equilibrium is a constant that matches the sum of all consumer budgets. And hence, whenever the revenue increases (decreases) for a company $k$, at least one other company will incur a loss (gain) in terms of revenue. None of the companies can do better by altering its power availabilities across the time horizon, nor by changing its prices. This follows from the definition of Nash equilibrium. Furthermore, we note that the revenues are proportional to the total capacity, and the company with the highest (lowest) portion of the market is the one that incurs the largest increase (decrease) in revenue. 

Interestingly, in Figure \ref{Influence_of_T} we observe that as $T$ increases, the utilities for consumers also increase, and hence they will be more attracted to demand response programs, which is desirable \cite{DOECOM}. In comparison with the single-period setup \cite{sabita,sabita2}, this shows that the multi-period demand response provides improvements on the consumers' end. This increase, however, does not change significantly beyond a certain number of periods.
To demonstrate the performance of our algorithm, we take the case when $T=1$ and study the algorithm's performance for different values of $\delta$ in Figure \ref{algorithm1}. When $\delta=1000$, we observe that the algorithm converges very fast to the optimal prices and takes about less than $5$ iterations to reach equilibrium. The values are consistent with the values in Figure  \ref{Influence_of_T} when $T=1$, where we used the analytical expressions of the prices. Next, we increase $\delta$ to $10000$ and observe that the algorithm converges at a lower rate, but still fast. Thus, the rate of convergence is inversely proportional to the value of $\delta$. However, when $\delta$ decreases to a negative value, there are no guarantees on convergence. Theorem 4 only guarantees the convergence of the algorithm when $\delta\geq0$. We have verified that our distributed algorithm converges very fast for various values of $\delta$ and alternative values of $T$ and $K$, and the reader might experiment with varying them using our open-source code in \cite{tool2}.

\subsection{Dutch Smart Grid Pilot} {\color{black} To further validate our multi-period-multi-commpany framework, we use data from the Dutch Smart Grid Pilot \cite{dutch}, which was conducted in Zwolle, the Netherlands, for about one year (May, 2014 to May, 2015). Tariffs were announced to consumers a day ahead, and the average consumer behavior was reported}. For a group of $77$ homogeneous consumers, we study the average consumer's demand and payments using experimental prices and the prices derived using our method. Here, we take $K=1$, which is consistent with the Dutch pilot. Also, the experimental prices are in EUR/kWh. 
 
  \subsubsection{Finding the necessary parameters}{\color{black} We find the fixed parameters similarly to the EcoGrid EU experiment. For each consumer $n$, we have {\color{black}$E_n^{\min} \approx 8.8 \ \text{kWh}$.} Then, by (\ref{budget}), we find the minimum necessary daily budget, which is $B_n \approx 1.1\ \text{EUR}$ for each consumer}.


   \subsubsection{Numerical Results} Using the above parameters, we again use (\ref{xx}) and (\ref{p}) to find optimal demands and prices. In Figure \ref{Netherlands_Price_Power}, we plot the average consumer's hourly demand, the prices found experimentally and using the Stackelberg game, and the corresponding total payments by the average consumer. We again observe that our approach leads to smaller prices with a significantly smaller variance. For the average consumer, we observe that significant savings can be achieved using our approach (more than $30\%$). Next, we study the performance of our distributed algorithm in Figure \ref{algorithm2}. As in the case of the EcoGrid EU experimental data, our algorithm achieves fast convergence to optimal prices using only local information. 
   
 