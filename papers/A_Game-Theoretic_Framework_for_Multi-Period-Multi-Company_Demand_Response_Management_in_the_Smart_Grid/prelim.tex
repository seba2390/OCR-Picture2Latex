%!TEX root = AlshehriLiuChenBasar.tex
%{\bf Preliminaries from Game Theory.} 
A static $N$-person noncooperative game is comprised of players set, action sets, and utility functions. Let the players set be denoted by $\scr{N} := \{1,\dots,N\}$, where $N$ is the number of players. Each player has an action set $\scr{A}_i$, and the decision of player $i$ is denoted by ${\bf{a}}_i\in\scr{A}_i$. The vector of decisions taken by other players is denoted by $\mathbf{{\bf{a}}_{-i}}:=({\bf{a}}_1,\dots,{\bf{a}}_{i-1},{\bf{a}}_{i+1},\dots,{\bf{a}}_N)$. Each player $i$ aims to maximize his/her utility function $u_i({\bf{a}}_i,\mathbf{{\bf{a}}_{-i}})$. One key point is that the utility function of player $i$ depends not only on his/her actions, but also on the decisions made by other players. An equilibrium concept that is suitable for  such games is the Nash Equilibrium (NE), which is defined below. 

\begin{definition} The action vector $\mathbf{a^*} \in \scr{A}_1\times\dots\times\scr{A}_N$ constitutes a Nash equilibrium for the $N$-person static noncooperative game in pure-strategies if 
\begin{equation}u_i({\bf{a}}^*_i,\mathbf{a^*_{-i}}) \geq u_i({\bf{a}}_i,\mathbf{a^*_{-i}}) \hspace{0.2in} \forall {\bf{a}}_i\in\scr{A}_i, \ i \in {\cal N}. \label{NE}\end{equation} \end{definition}
%Note that given the decisions made by other players, player $i$ cannot benefit by deviating from his action. Moreover, the Nash equilibrium does not necessarily always exist, and one may have to introduce some conditions on the utility function and/or action sets, or expand the strategy spaces to include probability distributions \cite{basar}.  
Sometimes it would be beneficial to allow for hierarchy in the decision process. In such a case, there are two types of players, leaders and followers. The leaders' decisions are dominant, and the followers respond to the decisions taken by the leaders. This kind of {\color{blue}hierarchical} games is called Stackelberg games, and the corresponding solution concept is called the Stackelberg equilibrium. %For a Stackelberg equilibrium to exist in the standard sense and not lead to ambiguity, each follower's optimal response to the actions taken by the leaders (within the equilibrium solution concept among followers, particularly Nash equilibrium) has to be unique\footnote{Otherwise one has to extend the notion of Stackelberg equilibrium to ``robust" equilibrium where non-unique responses of followers are also accommodated \cite{basar}.}.
 The leaders have the privilege of choosing how to take their actions at the beginning of the game. However, they have to take into account how the followers would respond to these actions and how each leader's decision is influenced by the decisions of the other leaders. To be more precise, suppose that we have $K$ leaders and $N$ followers. Denote the followers set by $\scr{N} := \{1,\dots,N\}$, and the leaders set by $\scr{K} := \{1,\dots,K\}$, with action sets $(\scr{F}_i)_{i\in \scr{N}}$ and $(\scr{L}_j)_{j\in \scr{K}}$, respectively. Denote a generic action of leader $j$ by ${\bf{a}}_{j}\in\scr{L}_j$, and that of follower $i$ by  ${\bf{b}}_i\in\scr{F}_i$. The vector of actions taken by all leaders is denoted by $\mathbf{a}:=({\bf{a}}_1,\dots,{\bf{a}}_K)$. The utility of leader $j$ is denoted by $u_j({\bf{a}}_j,\mathbf{{\bf{a}}_{-j}},\mathbf{b(a)})$, where $\mathbf{{\bf{a}}_{-j}}$ denotes the decisions of the other leaders, and $\mathbf{b(a)}=({\bf{b}}_1(\mathbf{a}),\dots,{\bf{b}}_N(\mathbf{a}))\in \scr{F}_1 \times \dots \times \scr{F}_N$. 


\begin{definition}The action vector $\mathbf{a^*} \in \scr{L}_1\times\dots\times\scr{L}_K$ is a Stackelberg Equilibrium strategy for all the $K$ leaders in pure-strategies if, for each $ j \in {\cal K}$,
\begin{eqnarray}u_j({\bf{a}}^*_j,\mathbf{a^*_{-j}},\mathbf{b^*(a^*)}) \geq u_j({\bf{a}}_j,\mathbf{a^*_{-j}},\mathbf{b^*}({\bf{a}}_j;{\bf{a^*_{-j}}}))\ \forall {\bf{a}}_j\in\scr{L}_j \label{SE}\end{eqnarray}
\end{definition}
where $\mathbf{b^*(a)} \in \scr{F}$ is the optimal response by all followers to the leaders' decisions, under the adopted equilibrium solution concept at the followers level. {\color{black} This solution concept is generally the Nash equilibrium, where followers play a Nash game. When there is no direct coupling between different followers, that is, other followers' decisions do not directly appear in the problem follower $i$ solves, they become independent, individual utility maximizers, which is the case we have in this work.} For a Stackelberg game, the pair ($\mathbf{a^*,b^*(a^*)}$) constitutes the equilibrium strategy. 
