%!TEX root = AlshehriLiuChenBasar.tex
\begin{figure}
\centering
\includegraphics[width=.9 \linewidth]{sketch.pdf}
\caption{The interaction between companies and their consumers. Companies play a price-selection Nash game. Then, consumers respond by choosing their demands independently of each other (the entire two-level interaction is a Stackelberg game).}
\label{sketch}
\end{figure}
Let $\scr{K}=\{1,2,\dots,K\}$ be the set of companies, $\scr{N}=\{1,2,\dots,N\}$ be the set of consumers, and $\scr{T}=\{1,2,\dots,T\}$ be the finite set of time slots \footnote{\color{blue}{In this paper, we interchangeably use ``time slot" and ``period" to refer to a subdivision of the time horizon.}}.
We formulate a static Stackelberg game between utility companies (the leaders) and their consumers (the followers) to find revenue maximizing prices and optimal demands. In Stackelberg games, the leader(s) first announce their decisions to the follower(s), and then the followers respond. In our game, the leaders send price signals to the consumers, who respond optimally by choosing their demands. To capture the  market competition between the utility companies, we let them play a price-selection Nash game. The equilibrium point of the price-selection game is what utility companies announce to their consumers. The consumers, on the other hand, do not face a game among themselves as they are individual utility maximizers. Figure 1 illustrates the hierarchical interaction between companies and consumers. 
In the parlance of dynamic game theory \cite{basar}, we are dealing here with open-loop information structures, with the corresponding equilibrium at the companies level being open-loop Nash equilibrium. Therefore, this is a one-shot game at which all the prices for all the periods are announced at the beginning of the game, and the followers respond to these prices by solving their local optimization problems. 
\subsection{Consumer-Side} 
Because of energy scheduling and storage, consumers may have some flexibility on when to receive a certain amount of energy. We are concerned about the total amount of shiftable energy. {\color{blue}Period-specific constraints can be added to include non-shiftable energy demand in the problem formulation, as discussed later in Section \ref{generalizations}}. Each energy consumer $n\in \scr{N}$ receives all price signals from each company $k\in \scr{K}$ at each time slot $t\in \scr{T}$ and aims to select his corresponding utility-maximizing demand $d_{n,k}(t)\geq0$ for each time slot from each company, subject to budget and energy need constraints. Denote the price of company $k$ at time $t$ by $p_k(t)$. Let $B_n \geq 0$  and $E^{{\rm min}}_n \geq 0$ denote, respectively, the budget of consumer $n$ and minimum energy need for the entire time-horizon. The utility of consumer $n$ is defined as 
\begin{equation}{ u_n(\mathbf{d}_{n})=\gamma_n\sum_{k\in \scr{K}}\sum_{t\in \scr{T}}\ln(\zeta_n+d_{n,k}(t))} \label{consumer}\end{equation} 
where $\gamma_n>0$ and $\zeta_n\geq1$ are preference parameters.  Note that if $0\leq\zeta_n<1$ or $\gamma_n<0$, the utility of the consumer becomes negative, which is not realistic for demand response applications, and hence we take $\gamma_n>0$ and $\zeta_n\geq1$. A typical value for $\zeta_n$ is $1$, but we still solve the problem for arbitrary $\zeta_n\geq1$ to keep it general. {\color{black}The logarithmic function (\ref{consumer}) is known to provide proportional fairness and is widely used to model consumer behavior in economics \cite{srikant,shadow,gao,basarDR,basarDR2}, {and it has been validated for demand response applications \cite{gao,fan2,han,DRadaptation,sabita}}. Our analysis in this paper is quite general and can be used in any market arrangement with multiple sellers and buyers under budget limitations and capacity constraints.}
Consumer $n$ aims to achieve the highest payoff while meeting the threshold of minimum amount of energy and not exceeding a certain budget. To be more precise,
given $B_n \geq 0$, $E^{{\rm min}}_n \geq 0$, and $p_k(t)>0$, the consumer-side optimization problem is formulated as follows:
\begin{eqnarray}
\underset{\mathbf{d}_{n}}{\hbox{maximize}} && u_n(\mathbf{d}_{n}) \nonumber \\
\hbox{subject to} && \sum_{k\in \scr{K}}\sum_{t\in \scr{T}}p_k(t)d_{n,k}(t)\leq B_n \label{cc1}\\
&& \sum_{k\in \scr{K}}\sum_{t\in \scr{T}}d_{n,k}(t)\geq \,E^{{\rm min}}_n \label{cc}\\
&&d_{n,k}(t)\geq 0,\;\;   \forall k \in \scr{K},\;\;  \forall t \in \scr{T} \label{cc3}
\end{eqnarray}
Note that, as indicated earlier, there is no game played among the consumers. Each consumer responds to the price signals using only her local information.We indirectly handle consumers' cost minimization via our analysis in later sections.
% These price signals depend on all the demands selected by the consumers and hence consumers indirectly affect each other's decisions, that is, they are coupled through the prices picked by the companies.
\subsection{Company-Side}
Let the prices chosen by other companies be ${\bf{p_{-k}}}$. The revenue for company $k$ is then given by
\begin{equation}{\pi_k}({\bf{p}}_{k},{\bf{p_{-k}}}):=\sum_{t\in \scr{T}}p_k(t)\sum_{n\in \scr{N}}d_{n,k}({\bf{p}}_k,{\bf{p_{-k}}},t).\label{UC}\end{equation}
Given the power availability of company $k$ at period $t$, denoted by $G_k(t)$, and for a fixed ${\bf{p_{-k}}}$, company $k$ solves the following problem:
\begin{eqnarray}
\underset{\mathbf{p}_k}{\hbox{maximize}} && \pi_k ({\bf{p}}_k,{\bf{p}_{-k}})
\nonumber \\
\hbox{subject to} && \sum_{n\in \scr{N}} d_{n,k}({\bf{p}}_k,{\bf{p}_{-k}},t) \leq G_k(t),\;\; \forall \; t \in \scr{T} \label{prob2} \\
&& p_k(t)> 0, \;\; \forall \; t \in \scr{T} \label{end}
\end{eqnarray}
  The goal of each company is to maximize its revenue\footnote{In later sections we show how companies can alter their problems to profit-maximization instead}. Additionally, because of the market competition, the prices announced by other companies also affect the determination of the price at company $k$. Thus, company $k$ selects its price in response to what other competitors in the market have announced; this response is also constrained by the availability of power. Thus, what we have is a Nash game among the companies. We emphasize that while each company's problem is affected by what its competitors decide, we can still achieve the equilibrium strategies using only local information, via our distributed algorithm discussed later in Section \ref{algorithm}. Finally, while at this point we have ${\bf G}_k$ fixed for each company $k$, we will later formulate a power allocation game to optimally choose them. 


