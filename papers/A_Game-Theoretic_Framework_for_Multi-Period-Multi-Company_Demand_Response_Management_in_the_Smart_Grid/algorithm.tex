%!TEX root = AlshehriLiuChenBasar.tex

%{\bf Distributed Algorithm.} 
\section{{\color{black}Distributed Algorithm} }\label{algorithm}
The Nash equilibrium (NE) for the power allocation game given by (\ref{NE}) can easily be computed by each company $k$ using its local information. Moreover, for consumers, it can be seen from (\ref{xx}) that in the computation of optimal demand selection for consumer $n$, no information from other consumers is needed, and consumer $n$ only needs local information for optimal response. However, the closed-form solution for optimal prices given by (\ref{p}) requires each company $k$ to know consumers' budgets and the power availability of all the other companies. Companies might not want to share such information with each other. To circumvent such a privacy concern, we propose a distributed algorithm that allows companies to compute their optimal prices using only local information, and show that this algorithm converges to the optimal prices given by  (\ref{p}). The algorithm, combined with utility-maximizing demands given by (\ref{xx}) and the NE given by (\ref{NE}), leads to the computation of all the optimal strategies with only local information at both the company level and the consumer level. 
\begin{algorithm}
\begin{algorithmic}[1]
\State Arbitrarily choose $p^{(0)}_k(t) ,\,\,\forall t\in\scr{T}, \,\,\,\forall k\in\scr{K}$ 
\State Repeat for $i=1, 2,3,\dots$ \label{next}
\State For each consumer $n \in \scr{N}$, compute $d^{(i)}_{n,k}(t)$ from  $k\in\scr{K}$ at  $t\in\scr{T}$ by (\ref{xx}), then update utility companies with demand signals\label{user}
\State {\color{black} Pick a company $k\in\mathcal{K}$ at time $t\in\mathcal{T}$ such that $p^{(i+1)}_k(t)$ is not yet computed, and compute it using (\ref{update}) \label{UCupdate}}
\State If $p^{(i+1)}_k(t)\neq p^{(i)}_k(t)$, update consumers and go to \ref{user}
\State Else, send a no-change signal to consumers and go to \ref{UCupdate}
\State If $p^{(i+1)}_k(t) = p^{(i)}_k(t)\,
\,\forall t\in\scr{T}, \,\,\,\forall k\in\scr{K}$, stop 
\State Else, go to \ref{next}
\end{algorithmic}
\caption{Distributed algorithm for computing the prices with local information} 
\end{algorithm}

For each iteration $i\in\{0,1,2,\ldots\}$, denote the demand from consumer $n$ at time $t$ from company $k$ by $d^{(i)}_{n,k}(t)$, and the price announced by company $k$ and time $t$ by $p^{(i)}_k(t)$. In our algorithm,  $p^{(0)}_k(t)$ is chosen arbitrarily for each company $k\in \scr{K}$ and time $t\in\scr{T}$. Based on the initial price selection, $d^{(0)}_{n,k}$ is computed using (\ref{xx}).  Then, the prices are sequentially updated using the following update rule: 
\begin{equation}p^{(i+1)}_k(t)=p^{(i)}_k(t)+\frac{\sum_{n\in \scr{N}}d^{(i)}_{n,k}(t)-G_k(t)}{\epsilon^{(i)}_{k,t}},\label{update}\end{equation}
where $\epsilon^{(i)}_{k,t}>0$ is appropriately selected for company $k$ at time $t$ in iteration $i$, and we present an expression for it as a function of $p^{(i)}_k(t)$ in Theorem \ref{d_algorithm}. 
Whenever a company $k$ updates its price at time $t$, it transmits the price to each consumer $n\in\scr{N}$, and they modify their demands accordingly. Once prices converge to their optimal values, consumers optimally respond by (\ref{xx}) and the algorithm terminates. We have the following theorem for the convergence of the algorithm; its proof can be found in the Appendix.


 \begin{theorem}
Under Assumptions \ref{assumption1}-\ref{assumption2}, for each company $k\in\scr{K}$ at time $t\in\scr{T}$ in iteration $i\in\{0,1,2,\ldots\}$, if the prices are sequentially updated using (\ref{update}) such that
$$ \epsilon^{(i)}_{k,t} =  \frac{G_k(t)+N}{p^{(i)}_k(t)} + \delta,$$
where $\delta\geq0$, then Algorithm 1 converges to optimal prices.
\label{d_algorithm}
 \end{theorem}


