%!TEX root = main.tex
In this paper, we model and solve a novel multi-period-multi-company demand response framework. We formulate a Stackelberg game to capture the interactions between companies and energy consumers, and within the framework of this model, we have obtained optimal prices and demands. Using the closed-form expressions, a power allocation game for companies has been formulated and solved. Furthermore, a distributed algorithm has been proposed to compute all equilibrium strategies using only local information. In the large population regime, an appropriate company-to-user ratio has been derived to maximize the revenue of each individual company.  The paper has shown theoretically and numerically that the multi-period scheme provides more incentives for the participation of energy consumers in demand response management, which is of critical importance \cite{DOECOM}. {\color{black} We have derived a minimum budget condition for consumers that can be used to measure whether or not they are spending more than what is necessary, and case studies using real data reveals potential savings for consumers that can exceed $30\%$}. Numerical studies also demonstrate fast convergence of the proposed distributed algorithm. 


While the proposed method focuses on the interplay between competing companies and their consumers, its useful mathematical properties make it generalizable to more consumer-specific and/or company-specific considerations.  For example, it is possible to include period-specific constraints for consumers. The game studied in this paper is multi-period but static. {\color{black} Therefore, it is a one-shot game and all the information are given at the the beginning of the game. Extending it to dynamic information structures, and using tools from dynamic game theory, such as feedback Stackelberg games \cite{basar}, where companies at each period change their prices for the next periods based on the information available at that particular period in which they are making the decisions, is another possible direction. Finally, for the distributed algorithm, it would be interesting to study privacy aspects other than convergence using only local information, such as, the ability of companies to approximate private parameters.}

