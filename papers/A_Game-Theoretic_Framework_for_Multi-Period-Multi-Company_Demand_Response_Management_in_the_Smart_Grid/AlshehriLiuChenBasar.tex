%%%%%%%%%%%%%%%%%%%%%%%%%%%%%%%%%%%%%%%%%%%%%%%%%%%%%%%%%%%%%%%%%%%%%%%%%%%%%%%%
%2345678901234567890123456789012345678901234567890123456789012345678901234567890
%        1         2         3         4         5         6         7         8

\documentclass[10 pt, journal]{IEEEtran}
                                                          % if you need a4paper
%\documentclass[uslettersize, 10pt, conference]{ieeeconf}      % Use this line for a4
                                                          % paper

\IEEEoverridecommandlockouts                              % This command is only
                                                          % needed if you want to
                                                          % use the \thanks command
%\overrideIEEEmargins
% See the \addtolength command later in the file to balance the column lengths
% on the last page of the document

\usepackage{graphics} % for pdf, bitmapped graphics files
\usepackage{epsfig} % for postscript graphics files
%\usepackage{mathptmx} % assumes new font selection scheme installed
\usepackage{times} % assumes new font selection scheme installed
\usepackage{amsmath} % assumes amsmath package installed
\usepackage{amssymb}  % assumes amsmath package installed
\usepackage{amscd}
\usepackage{ifthen}
\usepackage{cite}
\usepackage{lscape}  % Useful for wide tables or figures.
\usepackage[justification=raggedright]{caption}	% makes captions ragged right - thanks to Bryce Lobdell
\usepackage{algorithm}
\usepackage{algpseudocode}
\usepackage{cite}
\newcommand{\matt}[1]{\left[ \matrix{#1} \right]}
\def\scr#1{{\cal #1}}
\newcommand{\R}{\mathbb{R}}
\def\eq#1{\begin{equation}#1\end{equation}}
\def\rep#1{(\ref{#1})}
\newcommand{\bbb}{\mathbb}
\newtheorem{theorem}{Theorem}
\newtheorem{definition}{Definition}
\newtheorem{lemma}{Lemma}
\newtheorem{remark}{Remark}
\newtheorem{proposition}{Proposition}
\newtheorem{assumption}{Assumption}
\newtheorem{corollary}{Corollary}
\newcommand{\dfb}{\stackrel{\Delta}{=}}
\def\qed{ \rule{.08in}{.08in}}
\usepackage{color}
\usepackage{xcolor}
%\usepackage[small,it]{caption}
\definecolor{purple}{rgb}{0,0,0}
\definecolor{blue}{rgb}{0, 0, 0}

%%\linespread{0.99}
\begin{document}

\title{{\color{black}A Game-Theoretic Framework for Multi-Period-Multi-Company Demand Response
 Management in the Smart Grid
%: Revenue Maximization, Power Allocation, and Asymptotic Behavior
} \thanks{K. Alshehri is with the Systems Engineering Department, King Fahd University of Petroleum and Minerals (\texttt{kalshehri@kfupm.edu.sa}). J. Liu is with Department of Electrical and Computer Engineering, Stony Brook University
(\texttt{ji.liu@stonybrook.edu}). X. Chen is with the Department of Electrical, Computer, and Energy Engineering, University of Colorado Boulder  (\texttt{xudong.chen@colorado.edu}). T. Ba\c{s}ar is with the Department of Electrical and Computer Engineering and the Coordinated Science Laboratory, University of Illinois at Urbana-Champaign (\texttt{basar1@illinois.edu}). }
}


%\author{ \parbox{3 in}{\centering Huibert Kwakernaak*
%         \thanks{*Use the $\backslash$thanks command to put information here}\\
%         Faculty of Electrical Engineering, Mathematics and Computer Science\\
%         University of Twente\\
%         7500 AE Enschede, The Netherlands\\
%         {\tt\small h.kwakernaak@autsubmit.com}}
%         \hspace*{ 0.5 in}
%         \parbox{3 in}{ \centering Pradeep Misra**
%         \thanks{**The footnote marks may be inserted manually}\\
%        Department of Electrical Engineering \\
%         Wright State University\\
%         Dayton, OH 45435, USA\\
%         {\tt\small pmisra@cs.wright.edu}}
%}


\author{Khaled Alshehri, ~Ji Liu, ~Xudong Chen, ~Tamer Ba\c sar\\
{\it{\color{red}To appear in IEEE Transactions on Control Systems Technology}}
}
%\thanks{*Proofs of the three main theorems are available from the first author
%upon request and will appear in an expanded version of this paper.
%}

%\linespread{0.9}


\maketitle
\thispagestyle{empty}
\pagestyle{empty}


\begin{abstract}

By utilizing tools from game theory, {\color{black} we develop a novel multi-period-multi-company demand response framework considering the interactions between companies (sellers of energy) and their consumers (buyers of energy).} We model the interactions in terms of a Stackelberg game, where companies set their prices and consumers respond by choosing their demands. We show that the underlying game has a unique equilibrium at which the companies  maximize their revenues while the consumers maximize their utilities subject to their local constraints. Closed-form expressions are provided for the optimal strategies of all players. Based on these solutions, a power allocation game has been formulated, which is shown to admit a unique pure-strategy Nash equilibrium, for which closed-form expressions are also provided. {\color{black} This equilibrium is found under the assumption that companies can freely allocate their power across the time horizon, but we also demonstrate that it is possible to relax this assumption.} {\color{blue} We further provide a fast distributed algorithm for the computation of all optimal strategies using only local information.} {\color{black} We also study the effect of variations in the number of periods (subdivisions of the time horizon) and the number of consumers.} As a consequence, we are able to find an appropriate company-to-consumer ratio when the number of consumers participating in demand response {\color{black} exceeds some threshold}. Furthermore, we show, both analytically and numerically, that the multi-period scheme provides incentives for energy consumers to participate in demand response, {\color{black} compared to the single-period framework studied in the literature \cite{sabita}}. {\color{black}In our framework, we provide a condition for the minimum budgets consumers need, and carry out case studies using real life data to demonstrate the benefits of the approach, which show potential savings of up to $30\%$ and {\color{black} equilibrium prices that have low volatility.}}
 
\end{abstract}


\section{Introduction}
\section{Introduction}  \label{sec:introduction}

\newcommand\inexpIntro[3]{#1?(#2,#3).}
\newcommand\rinexpIntro[3]{*#1?(#2,#3).}
\newcommand\outexpIntro[3]{#1!(#2,#3).}
\newcommand\outatomIntro[3]{#1!(#2,#3)}

We propose a fully automated method for proving termination of \(\pi\)-calculus processes.
Although there have been a lot of studies on termination analysis for the \(\pi\)-calculus
and related calculi~\cite{Deng06IC,Demangeon07,SangiorgiTermination,KobayashiHybrid,Yoshida04IC,DBLP:journals/jlp/DemangeonHS10,Venet98SAS}, most of them have been rather theoretical,
and there have been surprisingly little efforts in developing  fully automated termination
verification methods and tools based on them. To our knowledge,
Kobayashi's \typical{}~\cite{TyPiCal,KobayashiHybrid} is the only exception that
can prove termination of \(\pi\)-calculus processes (extended with natural numbers)
fully automatically, but its termination analysis is quite limited (see Section~\ref{sec:relatedwork}).

Our method is based on a reduction to termination analysis for sequential programs:
we translate a \(\pi\)-calculus process \(P\) to a sequential program \(S_P\), so that
if \(S_P\) is terminating, so is \(P\). The reduction allows us to use
powerful, mature methods and tools
for termination analysis of sequential programs~\cite{heizmann2016ultimate,freqterm,DBLP:conf/lics/PodelskiR04,Kuwahara2014Termination,DBLP:journals/cacm/CookPR11}.

The idea of the translation is to convert a chain of communications on replicated input
channels to a chain of recursive function calls of the target sequential program.
Let us consider the following Fibonacci process:
\begin{align*}
    & \rinexpIntro{\fib}{n}{r}
        \ifexp{n<2}{ \soutatom{r}{1} \\ &\quad}
                   { \nuexp{s_1} \nuexp{s_2} (\outatomIntro{\fib}{n-1}{s_1} \PAR \outatomIntro{\fib}{n-2}{s_2} \PAR \sinexp{s_1}{x}\sinexp{s_2}{y}\soutatom{r}{x+y}) \\}
    & \PAR \outatomIntro{\fib}{m}{r}
\end{align*}
Here, the process
$\rinexpIntro{\fib}{n}{r} \ldots$ is a function server that computes the \(n\)-th Fibonacci number
in parallel and returns the result to \(r\),
and $\outatom{\fib}{m}{r}$ sends a request for computing the \(m\)-th Fibonacci number;
those who are not familiar with the syntax of the \(\pi\)-calculus may wish to consult
Section~\ref{sec:targetlanguage} first.
To prove that the process above is terminating for any integer \(m\),
it suffices to show that there is no infinite chain of communications on $\fib$:
\[
    \fib(m,r) \to \fib(m_1,r_1) \to \fib(m_2,r_2) \to \cdots.
\]
We convert the process above to the following program:\footnote{The actual translation
  given later is a little more complex.}
\begin{verbatim}
 let rec fib(n) = if n<2 then () else (fib(n-1) [] fib(n-2)) in
 fib(m)
\end{verbatim}
Here, \texttt{[]} represents the non-deterministic choice.
Note that, although the calculation of Fibonacci numbers is not preserved,
for each chain of communications on \texttt{fib}, there is a corresponding
sequence of recursive calls:
\[
\mathtt{fib}(m) \to \mathtt{fib}(m_1) \to \mathtt{fib}(m_2) \to \cdots.
\]
Thus, the termination of the sequential program above implies the termination of
the original process.
As shown in the example above, (i) each communication on a replicated input channel
is converted to a function call, (ii) each communication on a non-replicated input
channel is just removed (or, in the actual translation, replaced by a call of
a trivial function defined by \(f(\seq{x})=(\,)\)), and (iii) parallel composition
is replaced by a non-deterministic choice.
We formalize the translation outlined above and prove its correctness.

The basic translation sketched above sometimes loses too much information.
For example, consider the following process:
\begin{align*}
    & \rinexpIntro{\pre}{n}{r} \soutatom{r}{n-1} \\
    & \PAR \rinexpIntro{f}{n}{r} \ifexp{n<0}{ \soutatom{r}{1} }
                                       { \nuexp{s} (\outatomIntro{\pre}{n}{s} \PAR \sinexp{s}{x}\outatomIntro{f}{x}{r}) } \\
    & \PAR \outatomIntro{f}{m}{r}
\end{align*}
The translation sketched above would yield:
\begin{verbatim}
  let pred(n) = n-1 in
  let rec f(n) = if n<0 then () else (pred(n) [] f(*)) in
  f(m)
\end{verbatim}
Here, \texttt{*} represents a non-deterministic integer: since we have removed
the input $\sinatom{s}{x}$, we do not have information about the value of \( x \).
As a result, the sequential program above is non-terminating, although the original
process is terminating.
To remedy this problem, we also refine the basic translation above by using a refinement
type system for the \(\pi\)-calculus. Using the refinement type system,
we can infer that the value of \(x\) in the original process is less than \(n\),
so that we can refine the definition of \texttt{f} to:
\begin{verbatim}
 let rec f(n) = ... else (pred(n) [] let x=* in assume(x<n);f(x))
\end{verbatim}
The target program is now terminating, from which
we can deduce that the original process is also terminating.
We have implemented an automated tool based on the refined translation above.

The contributions of this paper are summarized as follows.
\begin{itemize}
\item The formalization of the basic translation from the \(\pi\)-calculus
  (extended with integers) to sequential programs, and a proof of its correctness.
\item The formalization of a refined translation based on a refinement type system.
\item An implementation of the refined translation, including automated refinement type
  inference based on CHC solving, and experiments to evaluate the effectiveness of
  our method.
\end{itemize}

The rest of this paper is structured as follows.
Section~\ref{sec:targetlanguage} introduces the source and target languages
of our translation.
Section~\ref{sec:approach} 
formalizes the basic translation, and proves its correctness.
Section~\ref{sec:refinement} refines the basic translation by using a refinement type system.
Section~\ref{sec:implementation} reports an implementation and experiments.
Section~\ref{sec:relatedwork} discusses related work,
and Section~\ref{sec:conclusion} concludes the paper.

\section{Preliminaries from Game Theory}\label{prelim}

We study the problem of selling $m$ items to $n$ buyers. We denote a bundle of items as a quantity vector $\vec{q} \in \Z_{\geq 0}^m$. The number of units of item $i$ in the bundle is $q[i]$. The bundle consisting of only one copy of the $i^{th}$ item is denoted by the standard basis vector $\vec{e}_i$, where $e_i[i] = 1$ and $e_i[j] = 0$ for all $j \not= i$. Each buyer $j \in [n]$ has a valuation function $v_j$ over bundles of items. We denote an allocation as $Q = \left(\vec{q}_1, \dots, \vec{q}_n\right)$ where $\vec{q}_j$ is the bundle that buyer $j$ receives. The cost to produce $\vec{q}$ is $c\left(\vec{q}\right)$ and the cost to produce the allocation $Q$ is $c\left(Q\right)$.
Suppose there are $\kappa_i$ units available of item $i$. Let $K = \prod_{i = 1}^m \left(\kappa_i+1\right)$. We use $\vec{v}_j = \left(v_j\left(\vec{q}_1\right), \dots, v_j\left(\vec{q}_K\right)\right)$ to denote buyer $j$'s values for all of the $K$ bundles and we use $\vec{v} = \left(\vec{v}_1, \dots, \vec{v}_n\right)$ to denote a vector of buyer values. We use the notation $\cX$ to denote the set of all valuation vectors $\vec{v}$. Additive buyers have values $v_j\left(\vec{q}\right) = \sum_{i = 1}^m q[i] v_j\left(\vec{e}_i\right)$ and unit-demand buyers have values $v_j\left(\vec{q}\right) = \max_{i : q[i] \geq 1} v_j\left(\vec{e}_i\right)$. The mechanisms  we study are dominant strategy incentive compatible, so we assume that the bids equal the buyers' valuations.

There is an unknown distribution $\pazocal{D}$ over buyers' values. 
The notation $\profit_M\left(\vec{v}\right)$ denotes the profit of a mechanism $M$ on the valuation vector $\vec{v}$. We use the notation $\profit_{\dist}\left(M\right) = \E_{\vec{v} \sim \dist}\left[\profit_M\left(\vec{v}\right)\right]$ and for a set of samples $\sample$, we use the notation \[\profit_{\sample}\left(M\right) = \frac{1}{|\sample|}\sum_{\vec{v} \in \sample}\profit_M\left(\vec{v}\right).\]

We study real-valued functions parameterized by vectors $\vec{p}$ in $\R^d$, denoted as $f_{\vec{p}}:\domain \to \R.$ For a fixed $\vec{v} \in \domain$, we often consider $f_{\vec{p}}\left(\vec{v}\right)$ as a function of its parameters, which we denote as $f_{\vec{v}}\left(\vec{p}\right)$.
\section{Formulation of a Mathematical Model}
 \label{formulation}
%!TEX root = main.tex
\section{Problem Definition and Notations}
\label{sec:problem}







% In this section, we will first describe key concepts and notations used in this paper, and formally define our problem. Then we will use a case study to make our idea of story tree more concrete.

% \subsection{Problem Definition and Notations}
% \label{subsec:problem-define}

We first present some definitions of key concepts in the top-down hierarchy: \textit{topic} $\rightarrow$ \textit{story} $\rightarrow$ \textit{event} to be used in this paper.

\begin{definition}
  \textit{Event}: an event $\mathcal{E}$ is a set of one or several documents that contain highly similar information.
\end{definition}

\begin{definition}
  \textit{Story}: a story $\mathcal{S}$ is a tree of events that revolve around a group of specific persons and happen at certain places during specific times. A directed edge from event $\mathcal{E}_1$ to $\mathcal{E}_2$ indicates a temporal evolution or a logical connection from $\mathcal{E}_1$ to $\mathcal{E}_2$.
\end{definition}

\begin{definition}
  \textit{Topic}: a topic consists of a set of stories that are highly correlated or similar to each other.
  \vspace{-1mm}
\end{definition}


Each topic may contain multiple story trees, and each story tree consists of multiple logically connected events.
In our work, events (instead of news documents) are the smallest atomic units. Each event is also assumed to belong to a single story and contains partial information about that story.
For instance, considering the topic \textit{American presidential election}, \textit{2016 U.S. presidential election} is a story within this topic, and  \textit{Trump and Hilary's first television debate} is an event within this story.


We now introduce some notations and describe our problem formally. Given a news document stream $D = \{ \mathcal{D}_1, \mathcal{D}_2, \ldots, \mathcal{D}_t,\ldots \}$, where $\mathcal{D}_t$ is the set of news documents collected on time period $t$, our objective is to: a) cluster all news documents $D$ into a set of events $E = \{ \mathcal{E}_1, \ldots, \mathcal{E}_{|E|} \}$, and b) connect the extracted events to form a set of stories $S = \{ \mathcal{S}_1, ..., \mathcal{S}_{|S|} \}$. Each story $\mathcal{S} = (E, L)$ contains a set of events $E$ and a set of links $L$, where $L_{i,j} := <\mathcal{E}_i, \mathcal{E}_j>$ denotes a directed link from event $\mathcal{E}_i$ to $\mathcal{E}_j$, which indicates a temporal evolution or logical connection relationship.

%We now illustrate our problem with an example. (A example Fig) Fig... shows ...
Furthermore, we require the events and story trees to be extracted in an online or incremental manner. That is, we extract events from each $\mathcal D_t$ individually when the news corpus $\mathcal D_t$ arrives in time period $t$, and \emph{merge} the discovered events into the existing story trees that were found at time $t-1$. This is a unique strength of our scheme as compared to prior work, since we do not need to repeatedly process older documents and can deliver  a set of evolving yet logically consistent story trees to users.  

% \subsection{Case Study}
% \label{subsec:case-study}

\begin{figure}
\includegraphics[width=3.4in]{figure/StoryStructures}
\caption{Different structures to characterize a story.}
\vspace{-2mm}
\label{fig:storyStructures}
\vspace{-2mm}
\end{figure}

For example, Fig.~\ref{fig:CaseStudy} illustrates the story tree of ``2016 U.S. presidential election''. The story contains $20$ nodes, where each node indicates an event in 2016 U.S. election, and each link indicates a temporal evolution or a logical connection between two events. %For example, event $19$ says America votes to elect new president, and event $20$ says Donald Trump is elected president. 
The index number on each node represents the event sequence over the timeline. There are $6$ paths within this story tree, where the path $1 \rightarrow 20$ indicates the whole presidential election process, branch $3 \rightarrow 6$ is about Hilary's health conditions, branch $7 \rightarrow 13$ talks about television debates, $14 \rightarrow 18$ depicts the investigation into Hilary's ``mail door'', etc. As we can see, by modeling the evolutionary and logical structure of a story into a story tree, users can easily grasp the logic of news stories and learn the main information quickly. 


Let us represent each story by an empty root node $s$ from which the story is originated, and denote each event by an event node $e$. The events in a story can be organized in one of the following four structures shown in Fig. \ref{fig:storyStructures}: a) a flat structure that does not include dependencies between events; b) a timeline structure that organizes events by their timestamps; c) a graph structure that checks the connection between all pairs of events and maintains a subset of most strong connections; d) a tree structure, which represents a story's evolving structure by a tree.  

Compared with a tree structure, sorting events by timestamps omits the logical connection between events, while using directed acyclic graphs to model event dependencies without considering the evolving consistency of the whole story can leads to unnecessary connections between events.
Through extensive user experience studies in Sec.~\ref{sec:eval}, we show that tree structures are the most effective way to represent breaking news stories as compared to other structures, including the more complex graph structures. 

\section{Demand Selection and Revenue Maximization (Stackelberg Game)} \label{game1}
%!TEX root = AlshehriLiuChenBasar.tex
In this section, we solve the above optimization problems in closed form and show that the solutions are unique.

%\subsection{Consumer- and Company-side Analyses}

\subsection{Consumer-Side Analysis}
%Note that the consumer-side utility function  is strictly concave and the constraints are linear.
%Refer to \cite{boyd,NL} for details about analyzing and solving such problems.
We start by relaxing the minimum energy constraint (\ref{cc}).
  For each consumer $n\in \scr{N}$, the associated Lagrange function is given as follows:
\begin{eqnarray*}
L_n &=& \gamma_n\sum_{k\in \scr{K}}\sum_{t\in \scr{T}}\ln(\zeta_n+d_{n,k}(t))
 \\
&&- \lambda_{n,1}\left(\sum_{k\in \scr{K}}\sum_{t\in \scr{T}}p_k(t)d_{n,k}(t)-B_n\right) \\
&&+\sum_{k\in\scr{K}} \sum_{t\in\scr{T}} \lambda_{n,2}(k,t)d_{n,k}(t)
\end{eqnarray*}
%\begin{align}\nonumber
%&L_n=\gamma_n\sum_{k\in \scr{K}}\sum_{t\in \scr{T}}\ln(\zeta_n+d_{n,k}(t))
%\cr
%&-\lambda_{n,1}(\sum_{k\in \scr{K}}\sum_{t\in \scr{T}}p(t)_kd_{n,k}(t)-B_n)+ \lambda_{n,2}(1,1)d_{n,1}(1)
%\cr
%&+\lambda_{n,2}(1,2)d_{n,1}(2)+\dots+\lambda_{n,2}(K,T)d_{n,K}(T)\end{align}
  where ${\bf \lambda_{n}}$ are the Lagrange multipliers. The KKT conditions of optimality in this case are sufficient because the objective function is strictly concave and the constraints are linear \cite{NL}, and solving for them leads to \begin{equation}
d^*_{n,k}(t)= \frac{B_n+\sum_{j\in \scr{K}}\sum_{h\in \scr{T}}p_j(h)\zeta_n}{KTp_k(t)}-\zeta_n,  \; \forall \;t\in \scr{T}, \; k\in \scr{K},  \label{xx}
\end{equation}
which is a generalization of the single-period case in \cite{sabita}. A detailed derivation of (\ref{xx}) can be found in \cite{mywork}. {\color{black} We remark that $d^*_{n,k}(t) \geq 0$  {\color{blue} because the objective function is strictly increasing.}}

The following theorem, whose proof can be found in the Appendix, states the necessary and sufficient condition for $B_n$ so that the above demands meet the minimum energy constraint (\ref{cc}).


\begin{theorem}
For each consumer $n \in \scr{N}$, the demands $d^*_{n,k}(t)$ given by (\ref{xx}) satisfy (\ref{cc}) if, and only if, 
\begin{equation} B_n \geq \frac{E_n^{{\rm min}}+\zeta_nKT}{\sum_{k\in \scr{K}}\sum_{t\in \scr{T}}\frac{1}{KTp_k(t)}}-\zeta_n \sum_{k\in \scr{K}}\sum_{t\in \scr{T}}p_k(t). \label{budget} \end{equation}
\end{theorem}

{\color{black}
\begin{remark} The above theorem can be interpreted as billing costs minimization. At the equality of (\ref{budget}), $B_n$ corresponds to the minimum budget needed for consumer $n$ to satisfy his energy need constraint, given the set of prices chosen by utility companies. Such a minimum $B_n$ can serve as a theoretical benchmark in which one can measure whether or not consumers are paying more than what is necessary.  We later demonstrate that with real data from demand response experiments, using the equality in (\ref{budget}) leads to savings in the range of $10\%-30\%$. \hfill $\Box$
\end{remark}


\begin{assumption}
For each consumer $n$, the budget $B_n$ satisfies the condition (\ref{budget}).
\label{assumption1}
\end{assumption}
}
%Now suppose that
%$$\frac{B_n+\zeta_n\sum_{k\in \scr{K}}\sum_{t\in \scr{T}}p_k(t)}{KTp_k(t)} -\zeta_n \geq 0 \,\,\,\, \forall k \in \scr{K},\,\, t \in \scr{T}$$
\subsection{Company-Side Analysis} 
%Given the prices set by the other companies subject to the power availability constraint (\ref{prob2}), each UC (leader) aims to determine its most profitable prices. At the leaders level, there is a noncooperative game in which each UC chooses its optimal prices in response to the prices set by the other UCs. 
We apply the demands derived in the consumers-side analysis (which were functions of the prices) and show that optimality is achieved at the equality of the constraint (\ref{prob2}). We start by solving for prices that satisfy the equality at (\ref{prob2}) and then prove that they are revenue-maximizing, strictly positive, and unique. 
Consider the equality in (\ref{prob2}), and by the optimal demands (\ref{xx}), there holds
$$
\frac{\sum_{n\in \scr{N}}B_n+\sum_{n\in \scr{N}}\zeta_n\sum_{j\in \scr{K}}\sum_{h\in \scr{T}}p_j(h)}{KTp_k(t)} =\sum_{n\in \scr{N}}\zeta_n + G_k(t),
$$
for all $t \in \scr{T}$. 
%for all $t \in \scr{T}$.
Let $Z=\sum_{n\in \scr{N}}\zeta_n$ and $B=\sum_{n\in \scr{N}}B_n$. Then, for each company $k \in \scr{K}$,
\begin{equation} B+Z\sum_{j\in \scr{K}}\sum_{h\in \scr{T}}p_j(h) = KTp_k(t)(G_k(t)+Z),\;\; \forall \;t \in \scr{T}.  \label{AP} \end{equation}
%Note that the double summation includes $p_k(t)$ and all the other prices.
%Thus, \begin{equation}\begin{split}B+Z\sum_{e\in \scr{K}}\sum_{h\in \scr{T}}p_e(h)= KTp_k(t)(G_k(t)+Z)-p_k(t)Z, \\ \,\,\,\, \forall \;t \in \scr{T}, \;\forall \;k \in \scr{K}, \; (e,h)\neq (k,t) \label{AP}\end{split}\end{equation}
The above equation (\ref{AP})  can be presented as the following system of linear equations
\begin{equation}AP=Y,\label{AP2}\end{equation}
%\begin{equation}\underbrace{\begin{pmatrix}
%KT(G_1(1)+Z)-Z& -Z &\dots& -Z\\
%-Z & KT(G_1(2)+Z)-Z& \dots& -Z\\
%\vdots & \ddots\\
%-Z &\dots&-Z & KT(G_K(T)+Z)-Z
%\end{pmatrix}}_{A}
%\underbrace{\begin{pmatrix}
%p_1(1) \\
%\vdots \\
%p_1(T)\\
%p_2(1)\\
%\vdots\\
%p_2(T)\\
%\vdots\\
%p_K(T) \end{pmatrix}}_P=\underbrace{\begin{pmatrix}
%B\\
%B\\
%\vdots \\
%B\end{pmatrix}}_Y \label{AP2}\end{equation}
%
where $A$ is a $KT\times KT$ matrix 
whose diagonal entries are $KT(G_k(t)+Z)-Z$, $k\in\scr{K}$, $t\in\scr{T}$,
and off-diagonal entries all equal to $-Z$, 
%\footnotesize{\begin{align}\nonumber
%&A=\begin{pmatrix}
%KT(G_1(1)+Z)-Z& -Z &\dots& -Z\\
%-Z & KT(G_1(2)+Z)-Z& \dots& -Z\\
%\vdots & \ddots\\
%-Z &\dots&-Z & KT(G_K(T)+Z)-Z
%\end{pmatrix}\end{align}}
$P$ is a vector in $\R^{KT}$ stacking $p_k(t)$, $k\in\scr{K}$, $t\in\scr{T}$,
and $Y$ a vector in $\R^{KT}$ whose entries all equal to $B$.
%\begin{eqnarray*}
%P &=& \begin{pmatrix}
%p_1(1) &
%\cdots &
%p_1(T) &
%p_2(1) &
%\cdots &
%p_2(T) &
%\cdots &
%p_K(T) \end{pmatrix}^T\\
%Y &=&\begin{pmatrix}
%B &
%B &
%\cdots &
%B\end{pmatrix}^T
%\end{eqnarray*}

%The following results say that matrix $A$ is invertible and the revenue-maximizing prices are positive and unique. 
%We also prove that the Nash equilibrium is at these prices.
We have the following results (proofs are in the Appendix).
\begin{lemma}
The matrix $A$ is invertible.
\end{lemma}


\begin{lemma}
The prices that solve (\ref{AP2}) are strictly positive and are unique. For each $t\in\mathcal{T}$, $k\in\mathcal{K}$, the price is given by
    \begin{equation} p^*_k(t)=\frac{B}{G_k(t)+Z}\left(\frac{1}{KT-\sum_{j\in \scr{K}}\sum_{h\in \scr{T}}\frac{Z}{G_j(h)+Z}}\right),\label{p}\end{equation}
where $B=\sum_{n\in \scr{N}}B_n$ and $Z=\sum_{n\in \scr{N}}\zeta_n$.
\end{lemma}


\begin{remark} 
Letting $\zeta_n=1$ for each consumer, the value of $Z$ coincides with $N$. In this case, by (\ref{p}), we observe that for any given ${\bf G_k}$, the price $p^*_k(t)(G_k(t)+N)$ is a constant for all $t \in \scr{T}$ and $k \in \scr{K}$.
Thus, the power availability is inversely proportional to the prices. \hfill$\Box$
%Whenever any of the $G_k(t)$'s changes, the constant on the right side changes, by (\ref{p}).
\end{remark}

\begin{remark} 
Lemma 2 provides a computationally cheap expression for the prices. Since $p^*_k(t)$ can be directly computed using (\ref{p}), there is no need to numerically compute $A^{-1}$ or $|A|$ to solve (\ref{AP2}). This enables us to deal with a large number of periods or utility companies, without worrying about computational complexity.\hfill$\Box$
\end{remark}

{\color{black}Due to production costs and market regulations, $p^*_k(t)$ cannot be outside the range of some lower and upper bounds $[p^{{\rm min}}_k(t),p^{{\rm max}}_k(t)]$  for all $t \in \scr{T}$ and $k \in \scr{K}$, as in \cite{sabita}. If $p^*_k(t)<p^{{\rm min}}_k(t)$, then $p^*_k(t)$ is set to $p^{{\rm min}}_k(t)$, and similarly for the upper-bound, if  $p^*_k(t)>p^{{\rm max}}_k(t)$, then we set  $p^*_k(t)=p^{{\rm max}}_k(t)$. Accordingly, denote the strategy space of utility company $k$ (a leader in the game) at $t$ by $\scr{L}_{k,t}:=[p^{{\rm min}}_k(t),p^{{\rm max}}_k(t)]$. The strategy space of $k$ for the entire time horizon is $\scr{L}_{k}=\scr{L}_{k,1}\times\dots\times\scr{L}_{k,T}$.
 The strategy space of all companies is $\scr{L}=\scr{L}_{1}\times\dots\times\scr{L}_{K}$. {\color{blue} For given price selections  ${\bf{p}}:=({\bf{p}}_1,\dots,{\bf{p}}_K) \in \scr{L}$}, the optimal response from all consumers is
$${\bf{d^*(p)}}=\{{\bf{d}}_1^*({\bf{p}}),{\bf{d}}_2^*({\bf{p}}),\dots,{\bf{d}}_N^*({\bf{p}})\}$$
where for each $n \in \scr{N}$, ${\bf{d}}^{*}_{n}({\bf{p}})$ is the unique maximizer for $u_n({\bf{d}}_n,{\bf{p}})$ and is given by (\ref{xx}).
}
We now have the following theorem, whose proof can be found in the Appendix.


\begin{theorem}[Existence and Uniqueness of the Stackelberg Equilibrium]Under Assumption \ref{assumption1}, the following statements hold: 
\begin{itemize}
\item[(i)] There exists a unique (open-loop) Nash equilibrium for the price-selection game and it is given by (\ref{p}).\\
\item[(ii)] There exists a unique (open-loop) Stackelberg equilibrium, and it is given by the demands in (\ref{xx}) and the prices in (\ref{p}).
\end{itemize}
%\item The maximizing demands given by (\ref{xx}) and the revenue-maximizing prices given in Lemma 2 constitute the (open-loop) Stackelberg equilibrium for the demand response management game.
\label{mainTHM}
\end{theorem}

At the Stackelberg equilibrium, it can easily be verified that  
\begin{equation}\sum_{k\in \scr{K}}\pi_k({\bf{p}}^*_k,{\bf{p}^*_{-k}})=\sum_{n\in \scr{N}}B_n.\label{eq1}\end{equation}
One observation is that when a company gains in terms of revenue, the same amount must be lost by other companies because the sum of revenues is a constant, which demonstrates a conflict of objectives between utility companies. However, by the definition of the equilibrium strategy, this is the best each company can do, for fixed power availabilities ${\bf G_k}$. But, given a total amount of available power, $G^{{\rm total}}_k$, a company has across the time horizon, it is possible that it gains in terms of revenue by an efficient power allocation. This motivates us to formulate a power allocation game and analytically answer the following question: How can company $k$ allocate its power so that it maximizes its revenue? {\color{black} Furthermore, for now, for ease of exposition, we neglect network and other company-specific constraints. Such considerations are later discussed in Section \ref{generalizations}.  For the remaining part of this paper, unless otherwise stated, we also have the following simplifying assumption.

\begin{assumption}
For each consumer $n$, we have $$\gamma_n=\zeta_n=1.$$ \label{assumption2}
\end{assumption}

The above assumption implies that $Z$ is equal to the number of consumers $N$.}



\section{Power Allocation (Nash Game)} \label{game2}
%!TEX root = AlshehriLiuChenBasar.tex

\begin{figure*}
\centering
\includegraphics[width=.9\linewidth]{sketch_DR2.pdf}
\caption{The interaction between companies and their consumers, along with power allocation. First, companies play a Nash power allocation game. Once power availabilities are allocated across all periods, companies and consumers play the Stackelberg game which dictates optimal prices and demand selection.}
\label{sketch_DR2}
\end{figure*}
In this section, we exploit the closed-form solutions for consumer demands and companies' prices to formulate and solve a power allocation game for companies. We note that while we use the closed-form solutions to define the power allocation game, it is to be played {\em{before}} the Stackelberg game, and its outcomes define the fixed power availabilities in the constraints of the companies in the Stackelberg game. Given the power availabilities from other companies, ${\bf{G_{-k}}}$, and since the equality in (\ref{prob2}) is satisfied at equilibrium, the revenue function of company $k$ can be represented as
\begin{equation} \pi_k({\bf{G}}_k,{\bf{G_{-k}}})=\sum_{t\in \scr{T}}p^*_k(t)G_k(t). \end{equation}  The optimal prices (\ref{p}) are functions of ${\bf{G}}_k$ and ${\bf{G_{-k}}}$, leading to the revenue function being equal to
\begin{equation} B\sum_{t\in \scr{T}}\frac{G_k(t)}{(G_k(t)+N)(KT-\sum_{j\in \scr{K}}\sum_{h\in \scr{T}}\frac{N}{G_j(h)+N})}, \label{Uk} \end{equation}
where $B=\sum_{n\in \scr{N}}B_n$. {Note that company $k$ receives a {\it fraction} of the total budgets. This fraction depends on what company $k$ offers in the multi-period-multi-company demand response framework, and what other companies also offer. Thus, when company $k$ can change what it offers, it can potentially increase the fraction it receives, and the power allocation game becomes natural, since the revenue function depends on other players' decisions. } For this game, which can be played before the Stackelberg game, which we have already solved, companies allocate their powers across all periods, and the outcome dictates the fixed power availabilities for the Stackelberg game. Figure \ref{sketch_DR2} provides an illustration. 

%However, our numerical results show that although multi-period demand response provides incentives for consumer participation, some companies can gain in terms of revenues while others can lose. But the sum of revenues of all companies is a constant that is equal to the sum of budgets (consumers use up all their budgets for demand selection and these budgets eventually go to companies). This conflict of objectives motivates us to formulate a power allocation game and analytically answer the following question: How can company $k$ allocate its power so that it maximizes its revenue, given its total amount of power available $G^{{\rm total}}_k$ for the entire time-horizon?

%By the above optimal prices and demands, and given the power availability from other companies by ${\bf{G_{-k}}}$, the revenue function of company $k$ can be represented as
%\begin{equation*} U_k := U_{k}(G_k,{\bf{G_{-k}}})=\sum_{t\in \scr{T}}p^*_k(t)G_k(t) \end{equation*} By representing the optimal prices (\ref{p}) as functions of $G_k(t)$'s,
%\begin{equation} U_k=B\sum_{t\in \scr{T}}\frac{G_k(t)}{(G_k(t)+N)(KT-\sum_{k\in \scr{K}}\sum_{t\in \scr{T}}\frac{N}{G_k(t)+N})} \label{Uk} \end{equation}
%where $B=\sum_{n\in \scr{N}}B_n$.

Let the total capacity for company $k$ for the entire time horizon be $G^{{\rm total}}_k$. Denote the action set of company $k$ at time $t$ by $\scr{P}_{k,t}:=[0,G^{{\rm total}}_k]$. Thus, given ${\bf{G_{-k}}}$,  the company $k$ solves the following problem:
\begin{eqnarray}
\underset{\mathbf{G}_k }{\hbox{maximize}} && \pi_k({\bf{G}}_k,{\bf{G_{-k}}})
\nonumber \\
\hbox{subject to} && \sum_{t\in \scr{T}}G_k(t) \leq G^{{\rm total}}_k, \\ \label{prob3}
&& G_k(t)\geq0, \ \forall t\in\scr{T}. \nonumber\end{eqnarray}

{\color{black}The above problem is only applicable for the case when generation is fully controllable. For the smart grid, because of the availability of various generation sources, full-controllability does not always hold, and in fact, for renewable resources it could be completely gone. We demonstrate the possibility of relaxing this assumption later in Section \ref{generalizations}.}

\subsection{Existence and Uniqueness of Nash Equilibrium} 
%Note that (\ref{Uk}) is equivalent to
%\begin{equation}  U_{k}(G_k,{\bf{G_{-k}}})= \sum_{t\in \scr{T}}\frac{BG_k(t)}{(G_k(t)+N)(\alpha_{-k}-\sum_{t\in \scr{T}}\frac{N}{G_k(t)+N})} \label{Uk2} \end{equation}
%where \begin{eqnarray*}  \alpha_{-k} & := & KT-\sum_{j\in \scr{K},j\neq k\,\,} \sum_{t\in \scr{T}}\frac{N}{G_j(t)+N} \\
%& > & KT-(K-1)T=T.\end{eqnarray*}
%
%Note that $\alpha_{-k}$ depends on the strategies of other companies and it is fixed for company $k$. 
The following theorem, whose proof can be found in the Appendix, states the existence and uniqueness of Nash equilibrium in the power allocation game, and provides an expression for it.
\begin{theorem}
Under Assumptions \ref{assumption1}-\ref{assumption2}, if ${\bf G}_k$ is fully controllable, there exists a unique pure-strategy Nash equilibrium for the power allocation game,
and it is given by \begin{equation} G^*_k(t)=\frac{G^{{\rm total}}_k}{T} \,\,\,,\,\, \forall \,t\in \scr{T}, \forall \, k\in \scr{K}.\label{NE}\end{equation}\label{allocation}\end{theorem}

Interestingly, the optimal strategy for each company is to equally allocate its power across all time periods. The proof of Theorem \ref{allocation} reveals that (\ref{Uk}) is strictly concave and increasing in each $G_k(t)$. This is an important property that allows accommodating further company-specific operational constraints and relaxing the full-controllability assumption. To illustrate, suppose that company $k$ has a mix of generation sources for which generation is controllable for some periods and only partially controllable for others. Then, it can add linear constraints to problem (\ref{prob3}) reflecting inter-temporal considerations at the generation-side (such as ramping limits). Existence and uniqueness of a pure-strategy Nash equilibrium are still guaranteed due to the strict concavity of the objective \cite{basar}. Since generation costs are typically assumed to be convex \cite{bosebook} (denote it by $c_k$ for each company $k$), company $k$ can also allocate its generation to maximize its profit, by subtracting the cost from (\ref{Uk}). One can alter the objective function of the power allocation game to 
%\begin{equation*} B\sum_{t\in \scr{T}}\frac{G_k(t)}{(G_k(t)+N)(KT-\sum_{j\in \scr{K}}\sum_{h\in \scr{T}}\frac{N}{G_j(h)+N})}\end{equation*}
%   to 
   \begin{align}&B\sum_{t\in \scr{T}}\frac{G_k(t)}{(G_k(t)+N)(KT-\sum_{j\in \scr{K}}\sum_{h\in \scr{T}}\frac{N}{G_j(h)+N})}\nonumber \\
   & \qquad\qquad\qquad \qquad-\sum_{t\in \scr{T}} c_k(G_k(t)),\label{cvx}\end{align}
   
 \noindent and the problem reflects profit-maximization in this case. Using (\ref{cvx}) and following our analysis, {\color{blue} we conclude that each company maximizes a strictly concave function}, and one can easily conclude the existence of a pure-strategy Nash equilibrium in this case. 
%   Furthermore, if $c_k$ is strictly convex, then, it is a unique Nash equilibrium \cite{basar}. 
\begin{algorithm}[!ht]
\begin{algorithmic}[1]
\Require Query workload $Q$, event stream $I$, \app\ graph $G$, hash table of snapshots $S$
\Ensure Hash table of results $R$ 
\State $G \leftarrow \emptyset$, $S, R \leftarrow$ empty hash tables
\ForAll {event $e \in I$ with $e.type=E$} 
    \State $//$ \textbf{\app\ graph construction}
    \ForAll {$q \in Q$ \text{ with event types }T}
        \ForAll {$E' \in T,\ E' \neq E$}
            \State $G_{E'} \leftarrow \mathit{getGraphlet}(G,E')$,
            $G_{E'}.\mathit{active} \leftarrow \mathit{false}$
        \EndFor
    \EndFor
    \If {\textbf{not} $G_E.\mathit{active}$}
        \State $G_E \leftarrow \mathit{createGraphlet()}$, $G_{E}.\mathit{active} \leftarrow \mathit{true}$,
        $G \leftarrow G \cup G_E$
        \If {$G_E.\mathit{shared}$ by $Q_E \subseteq Q$}
            $x \leftarrow \mathit{createSnapshot()}$ 
            \ForAll {$q \in Q_E$}
                \ForAll{$E' \in \mathit{pt}(E,q), E' \neq E$}
                    \State $G_{E'} \leftarrow \mathit{getGraphlet}(G,E')$
                    \State $S(x,q) \leftarrow S(x,q) + sum(G_{E'},q)$ \hspace{0.5cm}$//$ Eq.~5
                \EndFor
            \EndFor
        \EndIf    
    \EndIf
    \State insert $e$ into $G_E$
    \State $//$ \textbf{Trend count computation}
    \If {$G_E.\mathit{shared}$ by $Q_E \subseteq Q$}
        \If {$\forall q \in Q_E\ pe(e,q)$ are identical}
            \State $count(e,Q_E) \leftarrow count(e,q)$ \hspace{2.3cm}$//$ Eq.~2
        \Else\ $y \leftarrow \mathit{createSnapshot()}$, $count(e,Q_E) = y$
            \ForAll {$q \in Q_E$}
                $S(y,q) \leftarrow count(e,q)$ \hspace{0.2cm}$//$ Eq.~2
            \EndFor
          \EndIf
    \Else\ $count(e,q)$ \hspace{5.2cm}$//$ Eq.~2
    \EndIf
    \ForAll{$q \in Q$}
  	    \If {$E \in \mathit{end}(q)$} 
  		    $R(q) \leftarrow R(q) + count(e,q)$ $//$ Eq.~3
        \EndIf
    \EndFor
\EndFor
\State \Return $R$
\end{algorithmic}
\caption{\app\ shared online trend aggregation}
\label{algo:snapshot-propagation}
\end{algorithm}

\section{Asymptotic Regimes} \label{asymp}
%!TEX root = main.tex
{\color{black}In this section, we study the asymptotic (limiting) behavior as $T\rightarrow \infty$ or $N\rightarrow \infty$. While neither $T$ or $N$ can be arbitrarily large in practice, analyzing the asymptotic behavior brings in deep insights. For example, it reveals that consumers benefit as $T$ grows. As $N$ grows, our asymptotic analysis allows us to compute an appropriate company-to-consumer ratio $\frac{K}{N}$. We show these insights by studying how the utility functions, revenues, prices, and demands are affected as $T$ or $N$ grows. {For the rest of this section, in addition to Assumptions \ref{assumption1}-\ref{assumption2}, we assume the following. 
\begin{assumption} 
The total power available for the entire time horizon $G_k^{{\rm total}}$ is the same for each company $k\in\scr{K}$.
\label{assumption3}
\end{assumption}
}}
\subsection{When the Number of Periods Grows} Under Assumptions \ref{assumption1}-\ref{assumption3}, at equilibrium, it follows that 
%we have
%\begin{equation*}\sum_{j\in \scr{K}}\sum_{h\in \scr{T}}\frac{N}{G^*_j(h)+N}=KT\frac{N}{G^*_k(t)+N} \label{GT}\end{equation*}
%\begin{equation*}\sum_{j\in \scr{K}}\sum_{h\in \scr{T}}p^*_j(h)=KTp^*_k(t) \label{GT2}\end{equation*}
%Furthermore, 
optimal prices and demands are given by 
\begin{equation} p^*_k(t)=\frac{\sum_{m\in \scr{N}}B_m}{KTG^*_k(t)}=\frac{\sum_{m\in \scr{N}}B_m}{KG^{{\rm total}}_k}, \label{ppp} \end{equation}
\begin{equation}d^*_{n,k}(t)=\frac{B_n+KTp^*_k(t)}{KTp^*_k(t)}-1=\frac{G^{{\rm total}}_kB_n}{T\sum_{m\in \scr{N}}B_m},\,\, \label{t1} \end{equation}
%By (\ref{t1}), the payoff of consumer $n$ becomes
and the utility of consumer $n$ becomes\begin{equation} u_n=KT\ln\left(1+\frac{G^{{\rm total}}_kB_n/\sum_{m\in \scr{N}}B_m}{T}\right),\end{equation}
in which $G^{{\rm total}}_kB_n/\sum_{m\in \scr{N}}B_m$ is positive. Thus, as $T$ increases, the multiplicative term $ KT$ of the logarithmic function increases at a faster rate than the decrease of 
$\ln\left(1+{B_nG^{{\rm total}}_k/B}/{T}\right)$. 
Hence, as $T$ increases, the utility of each consumer $n \in \scr{N}$ monotonically increases.
Taking the limit, it can be verified that 
\begin{equation}\lim_{T\rightarrow\infty} u_n(T)=\frac{KG^{{\rm total}}_kB_n}{\sum_{m\in \scr{N}}B_m}.\end{equation}
Furthermore, note that the demand $d^*_{n,k}(t)$ from consumer $n\in \scr{N}$ from company $k\in \scr{K}$ at time $t\in \scr{T}$ converges to zero as $T \rightarrow \infty$. We claim that the revenues are constants. To see this, recall that
\begin{align*}
\pi_k({\bf{p}}^*_k,{\bf{p}^*_{-k}}) &= p^*_k(t)G^{{\rm total}}_k = \frac{\sum_{m\in \scr{N}}B_m}{K},
\end{align*}
which is a constant since both the number of companies and the budgets of the consumers are fixed.

\begin{remark} At the equilibrium, the monotonicity of the utilities of the consumers shows that increasing the number of periods leads to more incentives for consumers' participation in demand response. However, it might not be very beneficial to increase the number of periods to a very high value. First, the rate of increase in terms of consumers' utilities gets progressively smaller. Second, having a high number of periods leads to smaller demands for each period and that might violate some minimum energy need for particular periods at the consumers' level. So, it is beneficial to increase the number of periods up to a certain point (compared to having $T=1$), but it might not be beneficial to let $T$ become arbitrarily large. 
%t would be interesting to study what would be the appropriate number of periods that keeps consumers motivated to participate in demand response  while being practical.
\hfill$\Box$
\end{remark}


\begin{remark} Note that the limit point of the utility function of consumer $n$ is the proportion of his budget to the total budgets times the total power availability. So if a particular consumer has $1\%$ of the sum of all the budgets, he gets $1\%$ of the available power. Furthermore, the revenue for each company is the proportion of the sum of the budgets to the number of companies. In addition, the demand by consumer $n$ from company $k$ at time $t$ is the proportion of his budget to the total budgets times the total power availability at $t$ from $k$.
\hfill$\Box$
\end{remark}



\subsection{When the Number of Consumers Grows} When the number of consumers increases, each additional consumer has some budget $B_n$. With the total power availability from companies being fixed, they will increase their prices. {\color{black}  We have the following simplifying assumption. 

\begin{assumption} 
The budget for each consumer $n \in \scr{N}$ is the same.
\label{assumption4}
\end{assumption}



Under Assumptions \ref{assumption1}-\ref{assumption4}, we increase the number of consumers $N$ and see what happens as $N \rightarrow \infty$.} In this case, the optimal prices and demands become
%\begin{eqnarray}p^*_k(t)&=&\frac{NB_n}{KTG^*_k(t)} \in \scr{L}_{k,t}:=[p^{{\rm min}}_k(t),p^{{\rm max}}_k(t)]\label{pppp}\\
\begin{eqnarray}p^*_k(t)&=&\frac{NB_n}{KTG^*_k(t)} \label{pppp}\\
d^*_{n,k}(t) &=& \frac{G^{{\rm total}}_k}{TN} \label{dd} \end{eqnarray}
Clearly, $p^*_k(t)\rightarrow\infty$ as $N\rightarrow\infty$ and $d^*_{n,k}(t)\rightarrow0$ as $N\rightarrow\infty$. When the population is large and the power availability is fixed, it is not surprising that $d^*_{n,k}(t)\rightarrow0$ because the portion each consumer can get from the available power gets smaller and smaller as $N$ increases. Furthermore, it can be easily verified that $\lim_{N\rightarrow \infty}\pi_k(N)=\infty$ and $\lim_{N\rightarrow \infty}u_n(N)=0$. Thus, with the limit points resulting in unrealistic outcomes, a balance between the supply and demand needs to be achieved, which we do by finding an appropriate company-to-consumer ratio. 

Now, the question we ask is: For a given maximum allowable market price $p^{{\rm max}}_k(t)$, call it $p^{{\rm max}}$, what is the appropriate company-to-consumer ratio $\frac{K}{N}$? If there are more companies than necessary in the market, there will be losses in terms of revenues. On the other hand, if there are fewer companies than necessary, the prices can exceed $p^{{\rm max}}$, leading to undesirable outcomes. The following theorem, whose proof can be found in the Appendix, provides an optimal ratio at which prices do not exceed $p^{{\rm max}}$ and the revenues being maximized while satisfying the equality in (\ref{eq1}). 
{\color{black}
\begin{theorem}
Under Assumptions \ref{assumption1}-\ref{assumption4}, at the NE of the power allocation game, and at the Stackelberg equilibrium of the price and demand selection game, the optimal prices given by (\ref{p}) satisfy  
\begin{eqnarray*}
 p^*_k(t) &\leq &p^{{\rm max}},\\
 \sum_{k\in \scr{K}}\pi_k({\bf{p}}^*_k,{\bf{p^*_{-k}}})&=&\sum_{n\in \scr{N}}B_n,
\end{eqnarray*}
if, and only if, \begin{equation*} \frac{K}{N} \geq \frac{B_n}{p^{{\rm max}}TG^*_k(t)}, \end{equation*}
for each $t \in \mathcal{T}$ and $k\in\mathcal{K}$.
\end{theorem}}

%\begin{theorem}
%Suppose that the total power availabilities $G^{{\rm total}}_k$ for all companies are the same. Then, at the NE of the power allocation game, and at the Stackelberg equilibrium of the price and demand selection game, the following hold:\\
%(i) If  $N\leq {p^{{\rm max}}KG^{{\rm total}}_k}/{B_n}$, then the optimal prices $p^*_k(t)$'s given by (\ref{pppp}) are feasible and the supply-demand balance (\ref{prob2}) is satisfied.\\
%(ii) If  $N>{p^{{\rm max}}KG^{{\rm total}}_k}/{B_n}$, then the optimal company-to-consumer ratio that maximizes the revenues without exceeding $p^{{\rm max}}$ is 
%$K/N=B_n/(p^{{\rm max}}TG^*_k(t))$.
%\end{theorem}


\section{Case Studies} \label{numerical}
%!TEX root = AlshehriLiuChenBasar.tex


In this section, we present results on some case studies on representative days from a Dutch smart grid pilot \cite{dutch} and the EcoGrid EU project \cite{ecogrid}. We numerically study optimal prices and demands, and their corresponding payments and utility functions. We show how our approach results in monetary savings for consumers. Furthermore, we show that increasing the number of periods provides more incentives for consumers' participation in demand response management. Additionally, we demonstrate the fast convergence of our distributed algorithm to optimal prices. We also release an open-source interactive tool containing the simulations in \cite{tool2}. For both the Dutch smart pilot and the EcoGrid EU projects, the data are unavailable in raw format. Thus, whenever it is needed, we estimate some data points from figures available in the corresponding references \cite{dutch,ecogrid}. 

Recall that at the Stackelberg equilibrium, the total power availabilities ${\bf G}$ match the aggregate demands. That is, 
$$ \sum_{n \in \mathcal{N}} d^*_{n,k}(t)=G_k(t), \qquad \forall t\in\mathcal{T}, k\in\mathcal{K}.$$ Here, we use the experimental hourly variation of the total demands to choose values for ${\bf G}$ and the minimum energy need ${\bf E}^{\min}$. This allows us to establish a common aspect between our results and the experimental results, so that we can appropriately explore how our framework compares to real-life experiments. We also use the lower-bound on the minimum budget condition (\ref{budget}), so that we can also quantify potential savings.  
{\color{blue}  From the consumers' perspective, the prices are given parameters in both our model and the experimental setups. The optimal demands are functions of the prices, and the optimal prices naturally depend on the parameters of the consumers and companies. To bring deep insights, we make the differentiating aspect between our model and the experimental results an economic one. And hence, we pick the parameters such that the equilibrium demands and experimental ones are similar, but the prices, and essentially what consumers pay, are different. Utilizing Theorem 1, we conclude that the equilibrium prices bring savings to consumers, and by definition, they automatically consider the incentives of companies as they are revenue-maximizing. A main conclusion of this paper is that this quantifies the economic gap, in terms of consumer savings, between our game-theoretic benchmark and existing experimental results. On the other hand, in our analysis, we have relaxed some constraints for tractability, such as power flow and demand inelasticity considerations, and it remains open to explore the underlying tradeoffs, since adding these considerations might reduce the potential monetary savings for consumers. Such considerations were not directly included in the models studied in \cite{ecogrid,dutch}, as their focus was to experimentally explore the consumers' behavior in response to changing prices. It is worth mentioning that the results in \cite{dutch} revealed that consumers are mainly flexible about adjusting the consumption of white goods (washing machine, dishwasher, etc). It was also concluded in \cite{ecogrid} that demand response did not result in distribution feeder congestion relief, and consumers with automatic equipment were the most responsive ones. Nevertheless, later in Section IX, we demonstrate how our framework can be utilized to include additional network and consumer-specific and/or company-specific constraints, which make it possible to add constraints for congestion relief. Including such constraints will likely make it necessary to compute the equilibrium prices and demands algorithmically, which we relegate to future endeavors, as we emphasize here more on revealing deep insights via having tractable analysis.}
 \begin{figure*}
\centering
\includegraphics[width=\linewidth, height=2in]{EcoGridEU_Price_Power.pdf}
\caption{Total power offered by company (left), Stackelberg game and EcoGrid EU experimental prices (middle), and the cumulative payments and billing savings for all consumers (right). }
\label{EcoGridEU_Price_Power}
\end{figure*}


\subsection{EcoGrid EU Project} This demand response project was conducted from March 2011 to August 2015 in Bornholm, Denmark. The number of consumers in this experiment was approximately $2000$. For a representative day (December 5th, 2014), we apply our method to hourly prices and shiftable demand consumption from this experiment. The experimental prices are in  \text{DKK}/MWh and we scale them to  \text{DKK}/kWh. We start by assuming that there is only one company ($K=1$) and letting the consumers to be homogeneous (they have the same budgets and energy need) with $N=2000$, and then generalize the results to $K>1$ and heterogeneous consumers. Since we are taking hourly prices for a day, we have $T=24$. 

\subsubsection{Finding the necessary parameters}In our model, for each period $t$, we have a fixed power availability $G_1(t)$ on the supply-side. Also, for each consumer $n$, his minimum demand $E^{\min}_n$ and budget $B_n$ are fixed for the entire horizon. These are necessary parameters that need to be known to solve for optimal demands and prices. We let the power availabilities ${\bf G}_1$ match the experimental hourly variation of the total demand. For the entire time-horizon, we have  $$\sum^{2000}_{n=1} E^{\min}_n=\sum^{24}_{t=1}G_1(t) \approx 54 \ \text{{\color{blue}MWh}}.$$ For homogenous consumers, it follows that $$E^{\min}_n= \frac{\sum^{24}_{t=1}G_1(t)}{2000} \approx 27 \ \text{{\color{blue}kWh}}.$$ Next, using Theorem 1, we plug-in $E^{\min}_n$ and the experimental hourly prices in (\ref{budget}) to find the minimum budget need, which is $B_n\approx 7.6 \ \text{ \text{DKK}}$, for each $n$. 
   
 \subsubsection{Numerical Results} Now, using the parameters found above, we can compute the optimal demands and prices for the Stackelberg game using (\ref{xx}) and (\ref{p}), and study their effects. 
 
In Figure \ref{EcoGridEU_Price_Power}, we plot the total power availabilities ${\bf G}_1$, the prices found experimentally and using the Stackelberg game, and the corresponding total payments by all consumers for their demands. Our approach leads to prices that have a slightly smaller mean than in the experiment and a significantly smaller variance, which is a desirable property \cite{IFAC}. {\color{black} At the equilibrium point, as stated in Remark 2, we observe that $$p^*_k(t)(G_k(t)+N)=p^*_k(t)\Bigg(\sum_{n\in\mathcal{N}}  d^*_{n,k}(t)+N\Bigg)$$  is a constant for each period $t$ and each company $k$. Hence, whenever company $k$ at time $t$ has a large amount of power available to sell $G_k(t)$, it would lower its price, and vice versa. Here, consumers are attracted to buy more whenever the price is low, and will buy less whenever the price is high, which is intuitive.} One advantage our approach has is that it results in billing savings for consumers, as we show in Figure \ref{EcoGridEU_Price_Power} (this demonstrates the importance of Theorem 1, which we use to find the minimum budget need for the consumers). {\color{blue} Here, the equilibrium demands are similar to the experimental values, but since the prices differ,  consumers receive the same amount of energy at smaller costs}. This would lead to more monetary incentives for active consumer participation in demand response management, while being consistent with the company's objectives, since the Stackelberg game prices found using (\ref{p}) are revenue-maximizing as shown in the proof of Theorem 2.
\begin{figure*}
\centering
\includegraphics[width=\linewidth, height=4in]{Influence_of_T.pdf}
\caption{The effects of varying the number of periods for companies (with different market shares and at Nash equilibrium of the power allocation game) and heterogeneous consumers (with different budgets) using the EcoGrid EU experimental data.}
\label{Influence_of_T}
\end{figure*}
\begin{figure*}
\centering
\includegraphics[width=\linewidth, height=2in]{algorithm11.pdf}
\caption{ Distributed algorithm's performance (Theorem 4 requires $\delta\geq0$) using the EcoGrid EU experimental data.}
\label{algorithm1}
\end{figure*}

Next, we make consumers heterogeneous and increase the number of companies. We differentiate between consumers by varying their budgets, and take 5 classes of consumers, as in the EcoGrid EU experiment. We let consumers' budgets be $B_{1-400}=4  \ \text{DKK}$, $B_{401-800}=5  \ \text{DKK}$, $B_{801-1200}=6 \  \text{DKK}$, $B_{1201-1600}=7  \ \text{DKK}$, and $B_{1601-2000}=8 \ \text{DKK}$. We also let the number of companies be $K=4$, which is consistent with the actual energy sources used in the experiment. Precisely, the system is powered by 61\% wind energy ($k=1$), 27\% biomass ($k=2$), 9\% solar energy ($k=3$), and 3\% biogas ($k=4$). We split the {\color{black}total need ($54 \ \text{MWh}$)} among the energy sources according to experimental proportions, assuming that each energy source is owned by a single company that acts as a company in our game. 
 
With the above setup, we study the effect of varying the number of periods $T$ from $1$ to $50$. To do this, we need to find a way for companies to allocate their total power across the time horizon for each fixed $T$, which can be done by using Theorem 3, which states that equally splitting the total power across the time horizon for each company $k$ constitutes a unique Nash equilibrium for the power allocation game (it is also shown to be the global maximizer in the proof). 
%Thus, at the Nash equilibrium, for each $k$ with total power $G^{total}_k$, in each period $t$, we have $G^*_k(t)=G^{total}_k/T$.
 \begin{figure*}
\centering
\includegraphics[width=\linewidth, height=2in]{Netherlands_Price_Power.pdf}
\caption{Average consumer demand (left), Stackelberg game and Dutch pilot prices (middle), and the cumulative payments and billing savings for average consumer (right).}
\label{Netherlands_Price_Power}
\end{figure*}
\begin{figure*}
\centering
\includegraphics[width=\linewidth, height=2in]{algorithm2.pdf}
\caption{Distributed algorithm's performance (Theorem 4 requires $\delta\geq0$) using the Dutch pilot data.}
\label{algorithm2}
\end{figure*}
Figure \ref{Influence_of_T} shows the influence of varying the number of periods on prices, power allocated, revenues, and consumer utilities. We observe that as $T$ increases, the power allocated at each period gets progressively smaller. On the other hand, prices can increase or decrease, depending on the company, and they converge to positive constants. Furthermore, revenues might also increase or decrease, depending on the company (note that the company that achieves the highest revenue is the one that offers the lowest prices, and vice-versa). In view of (\ref{eq1}), the sum of revenues at equilibrium is a constant that matches the sum of all consumer budgets. And hence, whenever the revenue increases (decreases) for a company $k$, at least one other company will incur a loss (gain) in terms of revenue. None of the companies can do better by altering its power availabilities across the time horizon, nor by changing its prices. This follows from the definition of Nash equilibrium. Furthermore, we note that the revenues are proportional to the total capacity, and the company with the highest (lowest) portion of the market is the one that incurs the largest increase (decrease) in revenue. 

Interestingly, in Figure \ref{Influence_of_T} we observe that as $T$ increases, the utilities for consumers also increase, and hence they will be more attracted to demand response programs, which is desirable \cite{DOECOM}. In comparison with the single-period setup \cite{sabita,sabita2}, this shows that the multi-period demand response provides improvements on the consumers' end. This increase, however, does not change significantly beyond a certain number of periods.
To demonstrate the performance of our algorithm, we take the case when $T=1$ and study the algorithm's performance for different values of $\delta$ in Figure \ref{algorithm1}. When $\delta=1000$, we observe that the algorithm converges very fast to the optimal prices and takes about less than $5$ iterations to reach equilibrium. The values are consistent with the values in Figure  \ref{Influence_of_T} when $T=1$, where we used the analytical expressions of the prices. Next, we increase $\delta$ to $10000$ and observe that the algorithm converges at a lower rate, but still fast. Thus, the rate of convergence is inversely proportional to the value of $\delta$. However, when $\delta$ decreases to a negative value, there are no guarantees on convergence. Theorem 4 only guarantees the convergence of the algorithm when $\delta\geq0$. We have verified that our distributed algorithm converges very fast for various values of $\delta$ and alternative values of $T$ and $K$, and the reader might experiment with varying them using our open-source code in \cite{tool2}.

\subsection{Dutch Smart Grid Pilot} {\color{black} To further validate our multi-period-multi-commpany framework, we use data from the Dutch Smart Grid Pilot \cite{dutch}, which was conducted in Zwolle, the Netherlands, for about one year (May, 2014 to May, 2015). Tariffs were announced to consumers a day ahead, and the average consumer behavior was reported}. For a group of $77$ homogeneous consumers, we study the average consumer's demand and payments using experimental prices and the prices derived using our method. Here, we take $K=1$, which is consistent with the Dutch pilot. Also, the experimental prices are in EUR/kWh. 
 
  \subsubsection{Finding the necessary parameters}{\color{black} We find the fixed parameters similarly to the EcoGrid EU experiment. For each consumer $n$, we have {\color{black}$E_n^{\min} \approx 8.8 \ \text{kWh}$.} Then, by (\ref{budget}), we find the minimum necessary daily budget, which is $B_n \approx 1.1\ \text{EUR}$ for each consumer}.


   \subsubsection{Numerical Results} Using the above parameters, we again use (\ref{xx}) and (\ref{p}) to find optimal demands and prices. In Figure \ref{Netherlands_Price_Power}, we plot the average consumer's hourly demand, the prices found experimentally and using the Stackelberg game, and the corresponding total payments by the average consumer. We again observe that our approach leads to smaller prices with a significantly smaller variance. For the average consumer, we observe that significant savings can be achieved using our approach (more than $30\%$). Next, we study the performance of our distributed algorithm in Figure \ref{algorithm2}. As in the case of the EcoGrid EU experimental data, our algorithm achieves fast convergence to optimal prices using only local information. 
   
 

\section{Generalizations} \label{generalizations}
%!TEX root = main.tex
 
 In the previous sections, we have analyzed our multi-period-multi-company framework under some assumptions to keep the analysis tractable and to reveal various insights on what happens at the equilibrium strategies. Due to the desirable mathematical properties of our framework, it is possible to extend our model at both the consumer-level and company-level. Here, we discuss some such possible extensions. 
 
\subsection{Consumer-Side}
In the utility function (\ref{consumer}), the parameters $\gamma_n$ and $\zeta_n$ for consumer $n$ are time and company independent. However, it is possible, to consider both time-specific and company-specific preferences $\gamma_{n,k,t}$ and $\zeta_{n,k,t}$, which allows consumers to have further flexibilities without violating existence and uniqueness of optimal strategies. In this case, the utility of consumer $n$ would be defined as 
\begin{equation}{u_n(\mathbf{d}_{n})=\sum_{k\in \scr{K}}\sum_{t\in \scr{T}}\gamma_{n,k,t}\ln(\zeta_{n,k,t}+d_{n,k}(t))}. \label{consumer2}\end{equation}
By an analogous analysis to the derivation of (\ref{xx}), it follows that optimal demands, under Assumption \ref{assumption1}, are given by
 \begin{align}
d^*_{n,k}(t)&= \frac{B_n+\sum_{j\in \scr{K}}\sum_{h\in \scr{T}}p_j(h)\zeta_{n,j,h}}{p_k(t)} \Gamma_{n,k,t} \nonumber\\
 &\qquad \qquad\qquad-\zeta_{n,k,t},  \; \forall \;t\in \scr{T}, \; k\in \scr{K},  \label{xx2}
\end{align}
where $$\Gamma_{n,k,t}=\frac{\gamma_{n,k,t}}{\sum_{j\in \scr{K}}\sum_{h\in \scr{T}}\gamma_{n,j,h}}.$$ 
We remark that $\sum_{k\in\mathcal{K}}\sum_{t\in\mathcal{T}}\Gamma_{n,k,t}=1$. Thus, if consumer $n$ prefers a higher demand from company $k$ at time $t$, choosing a higher weight $\gamma_{n,k,t}$ can achieve this. We also note that in (\ref{xx}), where consumer $n$ has identical time-specific/company-specific parameters,  $$\Gamma_{n,k,t}=\frac{1}{KT}.$$ Another alternative, is to expand the constraint set of the optimization problem for consumers to include additional time-specific or company-specific constraints. In general, companies, as leaders of the Stackelberg game, would need to anticipate how consumers would respond to their prices, and given that anticipation, they choose their prices accordingly. Furthermore,  if non-logarithmic utility functions are used by consumers, it might be more difficult to compute a Nash equilibrium for the price-selection game for companies, but the existence of a pure-strategy equilibrium is guaranteed as long as the function 
$$\sum_{t\in \scr{T}}p_k(t)\sum_{n\in \scr{N}}d_{n,k}({\bf{p}}_k,{\bf{p_{-k}}},t)$$
is concave in each $p_k(t)$ over a compact and convex set \cite{basar}, for each company. In case (\ref{consumer2}) is used, by (\ref{xx2}), this condition is satisfied.  

\subsection{Company-Side}

{\color{black} In the current formulation, consumers' demands are coupled through the companies' problems and the power availability constraint (\ref{prob2}). The upper-bound in (\ref{prob2}) is taken to be fixed in the Stackelberg game, but they can be strategically chosen by the power allocation game discussed in Section \ref{game2}. However, this game was solved under restrictive assumptions, such as the absence of network constraints, the full-controllability of generation sources, and the absence of ramping considerations. It is of interest to generalize the power allocation game to alleviate these limitations. Specifically, suppose that power availabilities ${\bf {G}} \in \mathcal{M} \subset \mathbb{R}^{KT}$, where $\mathcal{M}$ represents the transmission and distribution network constraints. One possibility is to assume that $\mathcal{M}$ is a system of linear equations  that approximate power flow equations \cite{eugene,eugene2,eugene3,baran,bolognani}. Furthermore, for simplicity, suppose that company $k$ has a ramping limit $l_{k,t}$ at period $t$. Also, to encode controllability, suppose that $$G^{\min}_{k,t} \leq G_k(t) \leq G^{\max}_{k,t},$$
where $G^{\min}_{k,t}$ ($G^{\max}_{k,t}$) is the minimum (maximum) possible generation  for company $k$ at period $t$.  Thus, company $k$ solves the following optimization problem:
 \begin{eqnarray}
\underset{\mathbf{G}_k }{\hbox{maximize}} && \pi_k({\bf{G}}_k,{\bf{G_{-k}}})
\nonumber \\
\hbox{subject to} && \sum_{t\in \scr{T}}G_k(t) \leq G^{{\rm total}}_k,  \nonumber\\
&& \vert G_k(t)-G_k(t-1)\vert \leq l_{k,t}, \forall t, t-1 \in\scr{T}, \nonumber \\
&& ({\bf{G}}_k,{\bf{G_{-k}}}) \in \mathcal{M}, \label{game3} \\
&& G^{\min}_{k,t} \leq G_k(t) \leq G^{\max}_{k,t},  \ \forall t\in\scr{T}, \nonumber\\
&& G_k(t)\geq0, \ \forall t\in\scr{T}. \nonumber\end{eqnarray}

We have the following result, whose proof is given in the Appendix. 
\begin{theorem}
If the power allocation game (\ref{game3}) is feasible, then, it admits a pure-strategy Nash equilibrium $({\bf{G}}^*_k,{\bf{G^*_{-k}}})$. Furthermore, if $({\bf{G}}^*_k,{\bf{G^*_{-k}}})$ is used for the Stackelberg equilibrium demands and prices given by Theorem \ref{mainTHM}, then, $$\sum_{n\in \scr{N}} d^*_{n,k}(t) = G^*_k(t),\;\; \forall \; t \in \scr{T},\;\; \forall \; k \in \scr{K}.$$ \label{ThmGame3}
\end{theorem}

The above theorem follows from the strict concavity of $\pi_k({\bf{G}}_k,{\bf{G_{-k}}})$ and the compactness and convexity of the constraint set, in addition to the results in Section \ref{game1}. Furthermore, it also demonstrates that it is possible to incentivize consumers to further shift their consumption in a way that is consistent with network considerations and company requirements. Finally, we remark that the control of consumers' demands here is indirect, that is, it is done via the unique equilibrium prices (\ref{p}), which are also affected by consumers' preferences and choices. Hence, at equilibrium, optimal supply provided by companies, ${\bf G^*}$, is equal to aggregate optimal demand, while taking into account consumer budgets and energy needs, in addition to network considerations and company-specific constraints and revenues. 
 
 



\section{Conclusion and Research Directions} \label{conclusion}

\begin{comment}
\begin{figure}
\includegraphics[width=\linewidth]{figs/beyond_tss_lesion.pdf}
\caption[]{End-to-End runtime lesion study of the entire MNIST dataset and the FMA featurized music dataset. Each of DROP's contributions provides a runtime improvement.}
\label{fig:beyond_lesion}
\end{figure}
\end{comment}



\section{Conclusion}
\label{sec:conclusion}

Advanced data analytics techniques must scale to rising data volumes. 
DR techniques offer a powerful toolkit when processing these datasets, with PCA frequently outperforming popular techniques in exchange for high computational cost. 
In response, we propose DROP, a new dimensionality reduction optimizer. 
DROP combines progressive sampling, progress estimation, and online aggregation to identify high quality low dimensional bases via PCA without processing the entire dataset by balancing the runtime of downstream tasks and achieved dimensionality. 
Thus, DROP provides a first step in bridging the gap between quality and efficiency in end-to-end DR for downstream \red{analytics}. 

%We revisit canonical operators for time series dimensionality reduction and the measurement study of~\cite{keogh-study}, and show that PCA is more effective than popular alternatives in the data mining literature often by a margin of over $2\times$ on average on gold-standard time series benchmark data sets with respect to output data dimension. More surprisingly, we empirically demonstrate that a small number of samples are sufficient to accurately characterize directions of maximum variance and obtain a high-quality low-dimensional transformation.




\section{Acknowledgments}
K. Alshehri thanks King Fahd University of Petroleum and Minerals (KFUPM) for the financial support. Research supported in part by the U.S. Air Force Office of Scientific Research (AFOSR) MURI Grant FA9550-10-1-0573, and in part by the AFOSR Grant FA9550-19-1-0353.

\section{Appendix}
\section{Proofs}
\subsection{Proof of Theorem \ref{th:inexactLS1}}
We start from a similar argument as in \cite[proof of Therorem~2]{Blumen}. 
%proof of \cite[Theorem~2]{Blumen}. 
Set $g := 2\nabla f(x^{k-1})=2A^T(Ax^{k-1}-y)$ and $\g:=2\nablaa^{\nug} f(x^{k-1})= g+2\eg^k$ for some vector $\eg^k$ which by definition~\eqref{eq:grad} is bounded $\norm{\eg^k}\leq \nug^k$. It follows that
\ifCLASSOPTIONtwocolumn
\begin{align*} 
&\norm{y-Ax^k}^2-\norm{y-Ax^{k-1}}^2	\\
&= \langle x^k-x^{k-1},g \rangle +\norm{A(x^k-x^{k-1})}^2 \\
&\leq \langle x^k-x^{k-1},g \rangle + \MM \norm{x^k-x^{k-1}}^2, 
\end{align*}
\else
\begin{align*} 
\norm{y-Ax^k}^2-\norm{y-Ax^{k-1}}^2	&= \langle x^k-x^{k-1},g \rangle +\norm{A(x^k-x^{k-1})}^2 \\
&\leq \langle x^k-x^{k-1},g \rangle + \MM \norm{x^k-x^{k-1}}^2, 
\end{align*}
\fi
where the last inequality follows from the ULE property in Definition \ref{def:Lip}. Assuming $\MM \leq 1/\mu$, we have
\ifCLASSOPTIONtwocolumn
\begin{align*}
&\langle x^k-x^{k-1},g \rangle + \MM \norm{x^k-x^{k-1}}^2 \\
& \leq \langle x^k-x^{k-1},g \rangle + \frac{1}{\mu} \norm{x^k-x^{k-1}}^2\\
&= \langle x^k-x^{k-1},\g \rangle + \frac{1}{\mu} \norm{x^k-x^{k-1}}^2 - \langle x^k-x^{k-1},2\eg^k \rangle\\
& = \frac{1}{\mu} \norm{x^k-x^{k-1}+\frac{\mu}{2} \g }^2 - \frac{\mu}{4} \norm{\g}^2 - \langle x^k-x^{k-1},2\eg^k \rangle.
\end{align*}
\else
\begin{align*}
\langle x^k-x^{k-1},g \rangle + \MM \norm{x^k-x^{k-1}}^2 
& \leq \langle x^k-x^{k-1},g \rangle + \frac{1}{\mu} \norm{x^k-x^{k-1}}^2\\
&= \langle x^k-x^{k-1},\g \rangle + \frac{1}{\mu} \norm{x^k-x^{k-1}}^2 - \langle x^k-x^{k-1},2\eg^k \rangle\\
& = \frac{1}{\mu} \norm{x^k-x^{k-1}+\frac{\mu}{2} \g }^2 - \frac{\mu}{4} \norm{\g}^2 - \langle x^k-x^{k-1},2\eg^k \rangle.
\end{align*}
\fi
Due to the update rule of Algorithm \eqref{eq:inIP} and the inexact (fixed-precision) projection step, we have
\ifCLASSOPTIONtwocolumn
\begin{align*}
	&\norm{x^k-x^{k-1}+\frac{\mu}{2} \g }^2 \\
	&\leq  \norm{\pp_{\Cc}(x^{k-1}-\frac{\mu}{2} \g)-x^{k-1}+\frac{\mu}{2} \g }^2 +(\nup^k)^2\\
	&\leq \norm{x^\gt-x^{k-1}+\frac{\mu}{2} \g }^2 +(\nup^k)^2.
\end{align*}
\else
\begin{align*}
\norm{x^k-x^{k-1}+\frac{\mu}{2} \g }^2 
&\leq  \norm{\pp_{\Cc}(x^{k-1}-\frac{\mu}{2} \g)-x^{k-1}+\frac{\mu}{2} \g }^2 +(\nup^k)^2\\
&\leq \norm{x^\gt-x^{k-1}+\frac{\mu}{2} \g }^2 +(\nup^k)^2.
\end{align*}
\fi
The last inequality holds for any member of $\Cc$ and thus here for $x^\gt$. Therefore we can write
\ifCLASSOPTIONtwocolumn
\begin{align}
&\norm{y-Ax^k}^2-\norm{y-Ax^{k-1}}^2 \nonumber	\\
%&=\langle x^{t+1}-x^t,g \rangle + \frac{1}{\mu} \norm{x^{t+1}-x^t}^2 \nonumber\\
&\leq \frac{1}{\mu} \norm{x^\gt-x^{k-1}+\frac{\mu}{2} \g }^2 - \frac{\mu}{4} \norm{\g}^2 \nonumber\\
&\qquad - \langle x^k-x^{k-1},2\eg^k \rangle +(\frac{\nup^k}{\sqrt\mu})^2 \nonumber \\
&= \langle x^\gt-x^{k-1},\g \rangle + \frac{1}{\mu} \norm{x^\gt-x^{k-1}}^2 \nonumber\\
&\qquad - \langle x^k-x^{k-1},2\eg^k \rangle 
 +(\frac{\nup^k}{\sqrt\mu})^2\nonumber\\
%&= \langle x^\gt-x^{k-1},g \rangle + \frac{1}{\mu} \norm{x^\gt-x^{k-1}}^2 - \langle x^k-x^*,2\eg^k \rangle +\frac{\nup^k}{\mu}\nonumber\\
&\leq \langle x^\gt-x^{k-1},g \rangle + \frac{1}{\mu} \norm{x^\gt-x^{k-1}}^2 \nonumber\\
&\qquad+2\nug^k\norm{x^k-x^*} +(\frac{\nup^k}{\sqrt\mu})^2. \label{eq:p1b2}
\end{align}
\else
\begin{align}
\norm{y-Ax^k}^2-\norm{y-Ax^{k-1}}^2 \nonumber	
%&=\langle x^{t+1}-x^t,g \rangle + \frac{1}{\mu} \norm{x^{t+1}-x^t}^2 \nonumber\\
&\leq \frac{1}{\mu} \norm{x^\gt-x^{k-1}+\frac{\mu}{2} \g }^2 - \frac{\mu}{4} \norm{\g}^2 
 - \langle x^k-x^{k-1},2\eg^k \rangle +(\frac{\nup^k}{\sqrt\mu})^2 \nonumber \\
&= \langle x^\gt-x^{k-1},\g \rangle + \frac{1}{\mu} \norm{x^\gt-x^{k-1}}^2 
 - \langle x^k-x^{k-1},2\eg^k \rangle 
+(\frac{\nup^k}{\sqrt\mu})^2\nonumber\\
%&= \langle x^\gt-x^{k-1},g \rangle + \frac{1}{\mu} \norm{x^\gt-x^{k-1}}^2 - \langle x^k-x^*,2\eg^k \rangle +\frac{\nup^k}{\mu}\nonumber\\
&\leq \langle x^\gt-x^{k-1},g \rangle + \frac{1}{\mu} \norm{x^\gt-x^{k-1}}^2 
+2\nug^k\norm{x^k-x^*} +(\frac{\nup^k}{\sqrt\mu})^2. \label{eq:p1b2}
\end{align}
\fi
The last line replaces $\g= g+2\eg^k$ and uses the Cauchy-Schwartz inequality. 


Similarly we use the LLE property in Definition \ref{def:Lip} to obtain an upper bound on $ \langle x^\gt-x^{k-1},g \rangle$:
\begin{align*} 
\langle x^\gt-x^{k-1},g \rangle 	&= w^2-\norm{y-Ax^{k-1}}^2 +\norm{A(x_0-x^{k-1})}^2 \\
&\leq w^2 -\norm{y-Ax^{k-1}}^2 +\mmx\norm{x^\gt-x^{k-1}}^2,
\end{align*}
where $w=\norm{ y-Ax^\gt}$. Replacing this bound in \eqref{eq:p1b2} and cancelling $-\norm{y-Ax^{k-1}}^2$ from both sides of the inequality yields
\ifCLASSOPTIONtwocolumn
\begin{align}
&\norm{y-Ax^k}^2- 2\nug^k\norm{x^k-x^\gt}\nonumber \\ 
&\leq \left(\frac{1}{\mu}-\mmx \right)\norm{x^{k-1}-x^\gt}^2 + (\frac{\nup^k}{\sqrt\mu})^2+w^2. \label{eq:p1b3}
\end{align}
\else
\begin{align}
\norm{y-Ax^k}^2- 2\nug^k\norm{x^k-x^\gt}
\leq \left(\frac{1}{\mu}-\mmx \right)\norm{x^{k-1}-x^\gt}^2 + (\frac{\nup^k}{\sqrt\mu})^2+w^2. \label{eq:p1b3}
\end{align}
\fi
We continue to lower bound the left-hand side of this inequality:
\ifCLASSOPTIONtwocolumn
\begin{align*}
&\norm{y-Ax^k}^2- 2\nug^k\norm{x^k-x^\gt}\\
&= \norm{A(x^k-x^\gt)}^2+w^2-2\langle y-Ax^\gt, A(x^k-x^\gt)\rangle\\
&- 2\nug^k\norm{x^k-x^\gt} \\
&\geq \norm{A(x^k-x^\gt)}^2+w^2-2w \norm{A(x^k-x^\gt)}\\
&- 2\nug^k\norm{x^k-x^\gt} \\
& \geq \mmx\norm{x^k-x^\gt}^2+w^2-2(w \sqrt{\MM}+\nug^k)\norm{x^k-x^\gt}\\
&= \left(\sqrt{\mmx}\norm{x^k-x^\gt}-\frac{\nug^k}{\sqrt{\mmx}}- \sqrt{\frac{\MM}{\mmx}}w\right)^2 \\
&- (\frac{\nug^k}{\sqrt{\mmx}})^2 -(\frac{\MM}{\mmx}-1)w^2.
\end{align*}
\else
\begin{align*}
\norm{y-Ax^k}^2- 2\nug^k\norm{x^k-x^\gt}
&= \norm{A(x^k-x^\gt)}^2+w^2-2\langle y-Ax^\gt, A(x^k-x^\gt)\rangle- 2\nug^k\norm{x^k-x^\gt} \\
&\geq \norm{A(x^k-x^\gt)}^2+w^2-2w \norm{A(x^k-x^\gt)}
- 2\nug^k\norm{x^k-x^\gt} \\
& \geq \mmx\norm{x^k-x^\gt}^2+w^2-2(w \sqrt{\MM}+\nug^k)\norm{x^k-x^\gt}\\
&= \left(\sqrt{\mmx}\norm{x^k-x^\gt}-\frac{\nug^k}{\sqrt{\mmx}}- \sqrt{\frac{\MM}{\mmx}}w\right)^2 - (\frac{\nug^k}{\sqrt{\mmx}})^2 -(\frac{\MM}{\mmx}-1)w^2.
\end{align*}
\fi
The first inequality uses the Cauchy-Schwartz's and the second inequality follows from the ULE and LLE properties. Using this bound together with \eqref{eq:p1b3} we get
\ifCLASSOPTIONtwocolumn
\begin{align*}
&\left(\sqrt{\mmx}\norm{x^k-x^\gt}-\frac{\nug^k}{\sqrt{\mmx}}- \sqrt{\frac{\MM}{\mmx}}w\right)^2\\
&\leq \left(\frac{1}{\mu}-\mmx \right)\norm{x^{k-1}-x^\gt}^2 + (\frac{\nug^k}{\sqrt{\mmx}})^2+ (\frac{\nup^k}{\sqrt\mu})^2+\frac{\MM}{\mmx}w^2 \\
&\leq \left(\sqrt{\frac{1}{\mu}-\mmx} \norm{x^{k-1}-x^\gt} + \frac{\nug^k}{\sqrt{\mmx}}+ \frac{\nup^k}{\sqrt\mu}+\sqrt{\frac{\MM}{\mmx}}w \right)^2.
\end{align*}
\else
\begin{align*}
\left(\sqrt{\mmx}\norm{x^k-x^\gt}-\frac{\nug^k}{\sqrt{\mmx}}- \sqrt{\frac{\MM}{\mmx}}w\right)^2
&\leq \left(\frac{1}{\mu}-\mmx \right)\norm{x^{k-1}-x^\gt}^2 + (\frac{\nug^k}{\sqrt{\mmx}})^2+ (\frac{\nup^k}{\sqrt\mu})^2+\frac{\MM}{\mmx}w^2 \\
&\leq \left(\sqrt{\frac{1}{\mu}-\mmx} \norm{x^{k-1}-x^\gt} + \frac{\nug^k}{\sqrt{\mmx}}+ \frac{\nup^k}{\sqrt\mu}+\sqrt{\frac{\MM}{\mmx}}w \right)^2.
\end{align*}
\fi
The last inequality assumes $\mu\leq \mmx^{-1}$ which holds since we previously assumed $\mu\leq \MM^{-1}$. As a result we deduce that
\begin{align}
\norm{x^k-x^\gt}\leq \rho \norm{x^{k-1}-x^\gt} + \nut^k + 2\frac{\sqrt{\MM}}{\mmx}w \label{eq:p1b4}
\end{align}
for $\rho$ and $\nut^k$ defined in Theorem \ref{th:inexactLS1}. Applying this bound recursively (and setting $x^0=0$) completes the proof:
\begin{align*}
\norm{x^k-x^\gt}\leq \rho^k \norm{x^\gt} + \sum_{i=1}^k \rho^{k-i} \nut^i + \frac{2\sqrt{\MM}}{\mmx(1-\rho)}w.
\end{align*} 
Note that for convergence we require $\rho<1$ and therefore, a lower bound on the step size which is $\mu> (2\mmx)^{-1}$. 

\subsection{Proof of Corollary~\ref{cor:decay}}
Following the error bound \eqref{eq:errbound} derived in   Theorem~\ref{th:inexactLS1} and by setting $\nut^k\leq C r^k$ we obtain:
		\eq{
			\norm{x^{k}-x^\gt}\leq  \rho^k \left(\norm{x^\gt}+C\sum_{i=1}^k (r/\rho)^{i}  \right)+ \frac{2\sqrt{\MM}}{\mmx(1-\rho)}w,			
		}
which for $r<\rho$ it implies 		
		\eq{
			\norm{x^{k}-x^\gt}\leq 
			 \rho^k \left(\norm{x^\gt}+\frac{C}{1-r/\rho}\right)+ \frac{2\sqrt{\MM}}{\mmx(1-\rho)}w,
		}
and for $r>\rho$ implies %and following \eqref{eq:errbound} we get		
\begin{align*}
\norm{x^{k}-x^\gt}&\leq  \rho^k \norm{x^\gt}+C r^k \sum_{i=1}^k (\rho/r)^{k-i}  + \frac{2\sqrt{\MM}}{\mmx(1-\rho)}w\\
&\leq r^k \left(\norm{x^\gt}+\frac{C}{1-\rho/r}\right)+ \frac{2\sqrt{\MM}}{\mmx(1-\rho)}w,	
\end{align*}
and for $r=\rho$ we immediately get
\eq{
\norm{x^{k}-x^\gt}\leq  \rho^k \norm{x^\gt}+C k \rho^k + \frac{2\sqrt{\MM}}{\mmx(1-\rho)}w.	
}
Note that there exists a constant $c$ such that for an arbitrary small $\xi>0$ it holds $k\rho^k\leq c(\rho+\xi)^k$. Therefore we also achieve a linear convergence for the case $r=\rho$.
\subsection{Proof of Theorem \ref{th:inexactLS2}}
As before set $g= 2A^T(Ax^{k-1}-y)$ and $\g= g+2\eg^k$ for some bounded gradient error vector $\eg^k$ i.e. $\norm{\eg^k}\leq \nug^k$. Note that 
here the update rule of Algorithm \eqref{eq:inIP2} uses the  $(1+\epsilon)$-approximate projection  which by definition \eqref{eq:eproj} implies
\ifCLASSOPTIONtwocolumn
\begin{align*}
&\norm{x^k-x^{k-1}+\frac{\mu}{2} \g }^2 =  \norm{\pp^{\epsilon}_{\Cc}(x^{k-1}-\frac{\mu}{2} \g)-x^{k-1}+\frac{\mu}{2} \g }^2\\
&\leq  (1+\epsilon)^2\norm{\pp_{\Cc}(x^{k-1}-\frac{\mu}{2} \g)-x^{k-1}+\frac{\mu}{2} \g }^2\\
&\leq \norm{x^\gt-x^{k-1}+\frac{\mu}{2} \g }^2 + \phi(\epsilon)^2\frac{\mu^2}{4}\norm{\g}^2
\end{align*}
\else
\begin{align*}
\norm{x^k-x^{k-1}+\frac{\mu}{2} \g }^2 &=  \norm{\pp^{\epsilon}_{\Cc}(x^{k-1}-\frac{\mu}{2} \g)-x^{k-1}+\frac{\mu}{2} \g }^2\\
&\leq  (1+\epsilon)^2\norm{\pp_{\Cc}(x^{k-1}-\frac{\mu}{2} \g)-x^{k-1}+\frac{\mu}{2} \g }^2\\
&\leq \norm{x^\gt-x^{k-1}+\frac{\mu}{2} \g }^2 + \phi(\epsilon)^2\frac{\mu^2}{4}\norm{\g}^2
\end{align*}
\fi
where $\phi(\epsilon):=\sqrt{2\epsilon+\epsilon^2}$. For the last inequality we replace $\pp_{\Cc}(x^{k-1}-\frac{\mu}{2} \g)$ with two feasible points $x^\gt,x^{k-1}\in \Cc$. 

As a result by only replacing $\nug^k$ with $\mu\phi(\epsilon)\norm{\g}/2$, we can follow identical steps as for the proof of Theorem \ref{th:inexactLS1} up to \eqref{eq:p1b4}, revise the definition of $\nut^k:={2\nug^k}/{\mmx} + {\sqrt{\mu}\phi(\epsilon)\norm{\g}}/(2\sqrt{{\mmx}})$ and write
\ifCLASSOPTIONtwocolumn
\begin{align*}
\norm{x^k-x^\gt}\leq& \sqrt{\frac{1}{\mu\mmx}-1} \norm{x^{k-1}-x^\gt} \\
&+ \frac{2\nug^k}{\mmx} +\frac{\phi(\epsilon)}{2} \sqrt{\frac{\mu}{\mmx}}\norm{\g} + 2\frac{\sqrt{\MM}}{\mmx}w. 
\end{align*}
\else
\begin{align*}
\norm{x^k-x^\gt}\leq \sqrt{\frac{1}{\mu\mmx}-1} \norm{x^{k-1}-x^\gt} 
+ \frac{2\nug^k}{\mmx} +\frac{\phi(\epsilon)}{2} \sqrt{\frac{\mu}{\mmx}}\norm{\g} + 2\frac{\sqrt{\MM}}{\mmx}w. 
\end{align*}
\fi
Note that so far we only assumed $\mu\leq \MM^{-1}$. 

On the other hand by triangle inequality we have
\begin{align*}
	\norm{\g}&\leq \norm{g}+2\nug^k\\
	&\leq 2\norm{A^TA(x^{k-1}-x^\gt)}+2\norm{A^T(y-Ax^\gt)}+2\nug^k \\
	&\leq 2\sqrt \MM\vertiii{A}\norm{(x^{k-1}-x^\gt)}+2\vertiii{A}w+2\nug^k\\
	&\leq 2\sqrt{ 1/\mu}\vertiii{A}\norm{(x^{k-1}-x^\gt)}+2\vertiii{A}w+2\nug^k.
\end{align*}
The third inequality uses the ULE property and the last one holds since $\mu\leq \MM^{-1}$.
Therefore, we get
\ifCLASSOPTIONtwocolumn
\begin{align*}
&\norm{x^k-x^\gt}\leq
\left(\sqrt{\frac{1}{\mu\mmx}-1}+ \phi(\epsilon)\frac{\vertiii{A}}{\sqrt{\mmx}}\right) \norm{x^{k-1}-x^\gt} \\
&+ \left( \frac{2}{\mmx} +\phi(\epsilon){\sqrt{\frac{\mu}{\mmx}}}\right) \nug^k 
+ \left( 2\frac{\sqrt{\MM}}{\mmx}+ \phi(\epsilon)\sqrt{\frac{\mu}{\mmx}} \vertiii{A} \right)w. 
\end{align*}
\else
\begin{align*}
\norm{x^k-x^\gt}\leq&
\left(\sqrt{\frac{1}{\mu\mmx}-1}+ \phi(\epsilon)\frac{\vertiii{A}}{\sqrt{\mmx}}\right) \norm{x^{k-1}-x^\gt} \\
&+ \left( \frac{2}{\mmx} +\phi(\epsilon){\sqrt{\frac{\mu}{\mmx}}}\right) \nug^k 
+ \left( 2\frac{\sqrt{\MM}}{\mmx}+ \phi(\epsilon)\sqrt{\frac{\mu}{\mmx}} \vertiii{A} \right)w. 
\end{align*}
\fi
Based on assumption $\phi(\epsilon)\frac{\vertiii{A}}{\sqrt{\mmx}}\leq \delta$ of the theorem  we can deduce
\ifCLASSOPTIONtwocolumn
\begin{align*}
\norm{x^k-x^\gt}\leq&
\rho \norm{x^{k-1}-x^\gt} + \left( \frac{2}{\mmx} +\frac{\sqrt \mu}{\vertiii{A}} \delta\right) \nug^k \\
&
+ \left( 2\frac{\sqrt{\MM}}{\mmx}+\sqrt{\mu}\delta \right)w 
\end{align*}
\else
\begin{align*}
\norm{x^k-x^\gt}\leq
\rho \norm{x^{k-1}-x^\gt} + \left( \frac{2}{\mmx} +\frac{\sqrt \mu}{\vertiii{A}} \delta\right) \nug^k 
+ \left( 2\frac{\sqrt{\MM}}{\mmx}+\sqrt{\mu}\delta \right)w 
\end{align*}
\fi
where $\rho=\sqrt{\frac{1}{\mu\mmx}-1}+\delta$.

Applying this bound recursively (and setting $x^0=0$) completes the proof:
\eq{
\norm{x^{k}-x^\gt}\leq  \rho^k \norm{x^\gt}+\kappa_g \sum_{i=1}^k \rho^{k-i} \nug^i+ \frac{\kappa_w}{1-\rho}w
}
for $\kappa_g, \kappa_w$ defined in Theorem \ref{th:inexactLS2}. The condition for convergence is $\rho<1$ which implies $\delta<1$ and a lower bound on the step size which is $\mu> (\mmx+(1-\delta)^2\mmx)^{-1}$. 
		

\bibliographystyle{IEEEtran}
\bibliography{references}

\end{document}
%% The Appendices part is started with the command \appendix;
%% appendix sections are then done as normal sections

%% \section{}
%% \label{}