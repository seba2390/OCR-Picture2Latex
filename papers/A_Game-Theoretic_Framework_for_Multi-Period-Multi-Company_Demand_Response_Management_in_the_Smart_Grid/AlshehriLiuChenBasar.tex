%%%%%%%%%%%%%%%%%%%%%%%%%%%%%%%%%%%%%%%%%%%%%%%%%%%%%%%%%%%%%%%%%%%%%%%%%%%%%%%%
%2345678901234567890123456789012345678901234567890123456789012345678901234567890
%        1         2         3         4         5         6         7         8

\documentclass[10 pt, journal]{IEEEtran}
                                                          % if you need a4paper
%\documentclass[uslettersize, 10pt, conference]{ieeeconf}      % Use this line for a4
                                                          % paper

\IEEEoverridecommandlockouts                              % This command is only
                                                          % needed if you want to
                                                          % use the \thanks command
%\overrideIEEEmargins
% See the \addtolength command later in the file to balance the column lengths
% on the last page of the document

\usepackage{graphics} % for pdf, bitmapped graphics files
\usepackage{epsfig} % for postscript graphics files
%\usepackage{mathptmx} % assumes new font selection scheme installed
\usepackage{times} % assumes new font selection scheme installed
\usepackage{amsmath} % assumes amsmath package installed
\usepackage{amssymb}  % assumes amsmath package installed
\usepackage{amscd}
\usepackage{ifthen}
\usepackage{cite}
\usepackage{lscape}  % Useful for wide tables or figures.
\usepackage[justification=raggedright]{caption}	% makes captions ragged right - thanks to Bryce Lobdell
\usepackage{algorithm}
\usepackage{algpseudocode}
\usepackage{cite}
\newcommand{\matt}[1]{\left[ \matrix{#1} \right]}
\def\scr#1{{\cal #1}}
\newcommand{\R}{\mathbb{R}}
\def\eq#1{\begin{equation}#1\end{equation}}
\def\rep#1{(\ref{#1})}
\newcommand{\bbb}{\mathbb}
\newtheorem{theorem}{Theorem}
\newtheorem{definition}{Definition}
\newtheorem{lemma}{Lemma}
\newtheorem{remark}{Remark}
\newtheorem{proposition}{Proposition}
\newtheorem{assumption}{Assumption}
\newtheorem{corollary}{Corollary}
\newcommand{\dfb}{\stackrel{\Delta}{=}}
\def\qed{ \rule{.08in}{.08in}}
\usepackage{color}
\usepackage{xcolor}
%\usepackage[small,it]{caption}
\definecolor{purple}{rgb}{0,0,0}
\definecolor{blue}{rgb}{0, 0, 0}

%%\linespread{0.99}
\begin{document}

\title{{\color{black}A Game-Theoretic Framework for Multi-Period-Multi-Company Demand Response
 Management in the Smart Grid
%: Revenue Maximization, Power Allocation, and Asymptotic Behavior
} \thanks{K. Alshehri is with the Systems Engineering Department, King Fahd University of Petroleum and Minerals (\texttt{kalshehri@kfupm.edu.sa}). J. Liu is with Department of Electrical and Computer Engineering, Stony Brook University
(\texttt{ji.liu@stonybrook.edu}). X. Chen is with the Department of Electrical, Computer, and Energy Engineering, University of Colorado Boulder  (\texttt{xudong.chen@colorado.edu}). T. Ba\c{s}ar is with the Department of Electrical and Computer Engineering and the Coordinated Science Laboratory, University of Illinois at Urbana-Champaign (\texttt{basar1@illinois.edu}). }
}


%\author{ \parbox{3 in}{\centering Huibert Kwakernaak*
%         \thanks{*Use the $\backslash$thanks command to put information here}\\
%         Faculty of Electrical Engineering, Mathematics and Computer Science\\
%         University of Twente\\
%         7500 AE Enschede, The Netherlands\\
%         {\tt\small h.kwakernaak@autsubmit.com}}
%         \hspace*{ 0.5 in}
%         \parbox{3 in}{ \centering Pradeep Misra**
%         \thanks{**The footnote marks may be inserted manually}\\
%        Department of Electrical Engineering \\
%         Wright State University\\
%         Dayton, OH 45435, USA\\
%         {\tt\small pmisra@cs.wright.edu}}
%}


\author{Khaled Alshehri, ~Ji Liu, ~Xudong Chen, ~Tamer Ba\c sar\\
{\it{\color{red}To appear in IEEE Transactions on Control Systems Technology}}
}
%\thanks{*Proofs of the three main theorems are available from the first author
%upon request and will appear in an expanded version of this paper.
%}

%\linespread{0.9}


\maketitle
\thispagestyle{empty}
\pagestyle{empty}


\begin{abstract}

By utilizing tools from game theory, {\color{black} we develop a novel multi-period-multi-company demand response framework considering the interactions between companies (sellers of energy) and their consumers (buyers of energy).} We model the interactions in terms of a Stackelberg game, where companies set their prices and consumers respond by choosing their demands. We show that the underlying game has a unique equilibrium at which the companies  maximize their revenues while the consumers maximize their utilities subject to their local constraints. Closed-form expressions are provided for the optimal strategies of all players. Based on these solutions, a power allocation game has been formulated, which is shown to admit a unique pure-strategy Nash equilibrium, for which closed-form expressions are also provided. {\color{black} This equilibrium is found under the assumption that companies can freely allocate their power across the time horizon, but we also demonstrate that it is possible to relax this assumption.} {\color{blue} We further provide a fast distributed algorithm for the computation of all optimal strategies using only local information.} {\color{black} We also study the effect of variations in the number of periods (subdivisions of the time horizon) and the number of consumers.} As a consequence, we are able to find an appropriate company-to-consumer ratio when the number of consumers participating in demand response {\color{black} exceeds some threshold}. Furthermore, we show, both analytically and numerically, that the multi-period scheme provides incentives for energy consumers to participate in demand response, {\color{black} compared to the single-period framework studied in the literature \cite{sabita}}. {\color{black}In our framework, we provide a condition for the minimum budgets consumers need, and carry out case studies using real life data to demonstrate the benefits of the approach, which show potential savings of up to $30\%$ and {\color{black} equilibrium prices that have low volatility.}}
 
\end{abstract}


\section{Introduction}
% \leavevmode
% \\
% \\
% \\
% \\
% \\
\section{Introduction}
\label{introduction}

AutoML is the process by which machine learning models are built automatically for a new dataset. Given a dataset, AutoML systems perform a search over valid data transformations and learners, along with hyper-parameter optimization for each learner~\cite{VolcanoML}. Choosing the transformations and learners over which to search is our focus.
A significant number of systems mine from prior runs of pipelines over a set of datasets to choose transformers and learners that are effective with different types of datasets (e.g. \cite{NEURIPS2018_b59a51a3}, \cite{10.14778/3415478.3415542}, \cite{autosklearn}). Thus, they build a database by actually running different pipelines with a diverse set of datasets to estimate the accuracy of potential pipelines. Hence, they can be used to effectively reduce the search space. A new dataset, based on a set of features (meta-features) is then matched to this database to find the most plausible candidates for both learner selection and hyper-parameter tuning. This process of choosing starting points in the search space is called meta-learning for the cold start problem.  

Other meta-learning approaches include mining existing data science code and their associated datasets to learn from human expertise. The AL~\cite{al} system mined existing Kaggle notebooks using dynamic analysis, i.e., actually running the scripts, and showed that such a system has promise.  However, this meta-learning approach does not scale because it is onerous to execute a large number of pipeline scripts on datasets, preprocessing datasets is never trivial, and older scripts cease to run at all as software evolves. It is not surprising that AL therefore performed dynamic analysis on just nine datasets.

Our system, {\sysname}, provides a scalable meta-learning approach to leverage human expertise, using static analysis to mine pipelines from large repositories of scripts. Static analysis has the advantage of scaling to thousands or millions of scripts \cite{graph4code} easily, but lacks the performance data gathered by dynamic analysis. The {\sysname} meta-learning approach guides the learning process by a scalable dataset similarity search, based on dataset embeddings, to find the most similar datasets and the semantics of ML pipelines applied on them.  Many existing systems, such as Auto-Sklearn \cite{autosklearn} and AL \cite{al}, compute a set of meta-features for each dataset. We developed a deep neural network model to generate embeddings at the granularity of a dataset, e.g., a table or CSV file, to capture similarity at the level of an entire dataset rather than relying on a set of meta-features.
 
Because we use static analysis to capture the semantics of the meta-learning process, we have no mechanism to choose the \textbf{best} pipeline from many seen pipelines, unlike the dynamic execution case where one can rely on runtime to choose the best performing pipeline.  Observing that pipelines are basically workflow graphs, we use graph generator neural models to succinctly capture the statically-observed pipelines for a single dataset. In {\sysname}, we formulate learner selection as a graph generation problem to predict optimized pipelines based on pipelines seen in actual notebooks.

%. This formulation enables {\sysname} for effective pruning of the AutoML search space to predict optimized pipelines based on pipelines seen in actual notebooks.}
%We note that increasingly, state-of-the-art performance in AutoML systems is being generated by more complex pipelines such as Directed Acyclic Graphs (DAGs) \cite{piper} rather than the linear pipelines used in earlier systems.  
 
{\sysname} does learner and transformation selection, and hence is a component of an AutoML systems. To evaluate this component, we integrated it into two existing AutoML systems, FLAML \cite{flaml} and Auto-Sklearn \cite{autosklearn}.  
% We evaluate each system with and without {\sysname}.  
We chose FLAML because it does not yet have any meta-learning component for the cold start problem and instead allows user selection of learners and transformers. The authors of FLAML explicitly pointed to the fact that FLAML might benefit from a meta-learning component and pointed to it as a possibility for future work. For FLAML, if mining historical pipelines provides an advantage, we should improve its performance. We also picked Auto-Sklearn as it does have a learner selection component based on meta-features, as described earlier~\cite{autosklearn2}. For Auto-Sklearn, we should at least match performance if our static mining of pipelines can match their extensive database. For context, we also compared {\sysname} with the recent VolcanoML~\cite{VolcanoML}, which provides an efficient decomposition and execution strategy for the AutoML search space. In contrast, {\sysname} prunes the search space using our meta-learning model to perform hyperparameter optimization only for the most promising candidates. 

The contributions of this paper are the following:
\begin{itemize}
    \item Section ~\ref{sec:mining} defines a scalable meta-learning approach based on representation learning of mined ML pipeline semantics and datasets for over 100 datasets and ~11K Python scripts.  
    \newline
    \item Sections~\ref{sec:kgpipGen} formulates AutoML pipeline generation as a graph generation problem. {\sysname} predicts efficiently an optimized ML pipeline for an unseen dataset based on our meta-learning model.  To the best of our knowledge, {\sysname} is the first approach to formulate  AutoML pipeline generation in such a way.
    \newline
    \item Section~\ref{sec:eval} presents a comprehensive evaluation using a large collection of 121 datasets from major AutoML benchmarks and Kaggle. Our experimental results show that {\sysname} outperforms all existing AutoML systems and achieves state-of-the-art results on the majority of these datasets. {\sysname} significantly improves the performance of both FLAML and Auto-Sklearn in classification and regression tasks. We also outperformed AL in 75 out of 77 datasets and VolcanoML in 75  out of 121 datasets, including 44 datasets used only by VolcanoML~\cite{VolcanoML}.  On average, {\sysname} achieves scores that are statistically better than the means of all other systems. 
\end{itemize}


%This approach does not need to apply cleaning or transformation methods to handle different variances among datasets. Moreover, we do not need to deal with complex analysis, such as dynamic code analysis. Thus, our approach proved to be scalable, as discussed in Sections~\ref{sec:mining}.
\section{Preliminaries from Game Theory}\label{prelim}
\section{Preliminaries}
Given a graph $G=(V,E)$, and vertex $u \in V$, let $\deg(u,G)$ be the degree of $u$ in $G$. 
Given a tree $T$ and $u, v \in T$, denote the $u$-$v$ path in $T$ by $\pi(u,v,T)$. When the tree $T$ is clear from the context, we may omit it and write $\pi(u,v)$. For a (possibly weighted) subgraph $G' \subseteq G$ and a vertex pair $s,t \in V$, let $\dist_{G'}(s,t)$ denote the length of the $s$-$t$ shortest path in $G'$. 

\paragraph{Fault-Tolerant Labeling Schemes.}
For a given graph $G$, let $\Pi: V\times V \times \mathcal{G} \to \mathbb{R}_{\geq 0}$ %\mtodo{a reviewer pointed out that the last element in the domain should probably be subgraphs of $G$ and not $G$, see which notation we want here.} \mertodo{I am not sure that it is needed as we consider $G$ in the fault-free setting and $G \setminus F$ in the FT setting. We can of course write $\Pi: V\times V \times \mathcal{G} \to \mathbb{R}_{\geq 0}$ where $\mathcal{G}$ is the family of all $G$-subgraphs, but I cannot see why we need it.} 
be a function defined on pairs of vertices and a subgraph $G' \subset G$, where $\mathcal{G}$ is the family of all subgraphs of $G$. For an integer parameter $f\geq 1$, an $f$-\emph{fault-tolerant labeling scheme} for a function $\Pi$ and a graph family $\mathcal{F}$ is a pair of functions $(L_{\Pi},D_{\Pi})$. The function $L_{\Pi}$ is called the \emph{labeling function}, and $D_{\Pi}$ is called the \emph{decoding function}. For every graph $G$ in the family $\mathcal{F}$, the labeling function $L_{\Pi}$ associates with each vertex $u \in V(G)$ and every edge $e \in E(G)$, a label $L_{\Pi}(u,G)$ (resp., $L_{\Pi}(e,G)$). It is then required that given the labels of any triplets $s,t, F \in V \times V \times E^f$, the decoding function $D_{\Pi}$ computes $\Pi(s,t, G \setminus F)$.  The primary complexity measure of a labeling scheme is the \emph{label length}, measured by the length (in bits) of the largest label it assigns to some vertices (or edges) in $G$ over all graphs $G \in \mathcal{F}$. An $f$-FT connectivity labeling scheme is required to output YES iff $s$ and $t$ are connected in $G \setminus F$.  In $f$-FT \emph{approximate distance labeling scheme} it is required to output an estimate for the $s$-$t$ distance in the graph $G \setminus F$. Formally, an $f$-FT labeling scheme is $q$\emph{-approximate} if the value $\delta(s,t,F)$ returned by the decoder algorithm satisfies that $\dist_{G \setminus F}(s,t)\leq \delta(s,t,F) \leq q \cdot \dist_{G \setminus F}(s,t)$.  Throughout the paper we provide randomized labeling schemes which provide a high probability guarantee of correctness for any fixed triplet $\langle s,t, F \rangle$. 


\paragraph{Fault-Tolerant Routing Schemes.} In the setting of FT routing scheme, one is given a pair of source $s$ and destination $t$ as well as $F$ edge faults, which are initially unknown to $s$. The routing scheme consists of \emph{preprocessing} and \emph{routing} algorithms. The preprocessing algorithm defines labels $L(u)$ to each of the vertices $u$, and a header $H(M)$ to the designated message $M$. In addition, it defines for every vertex $u$ a routing table $R(u)$. The labels and headers are usually required to be short, i.e., of poly-logarithmic bits. 
The routing procedure determines at each vertex $u$ the port-number on which $u$ should send the messages it receives. The computation of the next-hop is done by considering the header of the message $H(M)$, the label of the source and destination $L(s)$ and $L(t)$ and the routing table $R(u)$. The routing procedure at vertex $u$ might also edit the header of the message $H(M)$. The failing edges are not known in advance and can only be revealed by reaching (throughout the message routing) one of their endpoints. The \emph{space} of the scheme is determined based on maximal length of message headers, labels and the individual routing tables. The stretch of the scheme is measured by the ratio between the length of the path traversed until the message arrived its destination and the length of the shortest $s$-$t$ path in $G \setminus F$. In the more relaxed setting of \emph{forbidden-set routing schemes} the failing edges are given as input to the routing algorithm.

%\paragraph{Forbidden-Set Routing Schemes.} One of the key applications of labeling schemes is routing.
%In the setting of forbidden-set routing schemes, given the labels of $s$, $t$ and a set of forbidden edges $
%F$, it is required to determine the next-hop neighbor of $s$ on some short $s$-$t$ path in $G \setminus F$. The main two complexity measures are the stretch induced by the $s$-$t$ path encoded by the labels.  \mtodo{Maybe the definition should be more similar to the next one? (consider labels, tables, headers)}
%That is, given the labels of $u$, $v$ and the faults $F$, the decoder function returns the port-number of $u$'s neighbor lying on a $u$-$v$ path $P$ in $G \setminus F$ such that $P \subseteq G$ and $len(P)\leq s \cdot \dist(u,v, G \setminus F)$ for some approximation factor $s$. 

\section{Formulation of a Mathematical Model}
 \label{formulation}
We briefly recall the framework of statistical inference via empirical risk minimization.
Let $(\bbZ, \calZ)$ be a measurable space.
Let $Z \in \bbZ$ be a random element following some unknown distribution $\Prob$.
Consider a parametric family of distributions $\calP_\Theta := \{P_\theta: \theta \in \Theta \subset \reals^d\}$ which may or may not contain $\Prob$.
We are interested in finding the parameter $\theta_\star$ so that the model $P_{\theta_\star}$ best approximates the underlying distribution $\Prob$.
For this purpose, we choose a \emph{loss function} $\score$ and minimize the \emph{population risk} $\risk(\theta) := \Expect_{Z \sim \Prob}[\score(\theta; Z)]$.
Throughout this paper, we assume that
\begin{align*}
     \theta_\star = \argmin_{\theta \in \Theta} L(\theta)
\end{align*}
uniquely exists and satisfies $\theta_\star \in \text{int}(\Theta)$, $\nabla_\theta L(\theta_\star) = 0$, and $\nabla_\theta^2 L(\theta_\star) \succ 0$.

\myparagraph{Consistent loss function}
We focus on loss functions that are consistent in the following sense.

\begin{customasmp}{0}\label{asmp:proper_loss}
    When the model is \emph{well-specified}, i.e., there exists $\theta_0 \in \Theta$ such that $\Prob = P_{\theta_0}$, it holds that $\theta_0 = \theta_\star$.
    We say such a loss function is \emph{consistent}.
\end{customasmp}

In the statistics literature, such loss functions are known as proper scoring rules \citep{dawid2016scoring}.
We give below two popular choices of consistent loss functions.

\begin{example}[Maximum likelihood estimation]
    A widely used loss function in statistical machine learning is the negative log-likelihood $\score(\theta; z) := -\log{p_\theta(z)}$ where $p_\theta$ is the probability mass/density function for the discrete/continuous case.
    When $\Prob = P_{\theta_0}$ for some $\theta_0 \in \Theta$,
    we have $L(\theta) = \Expect[-\log{p_\theta(Z)}] = \kl(p_{\theta_0} \Vert p_\theta) - \Expect[\log{p_{\theta_0}(Z)}]$ where $\kl$ is the Kullback-Leibler divergence.
    As a result, $\theta_0 \in \argmin_{\theta \in \Theta} \kl(p_{\theta_0} \Vert p_\theta) = \argmin_{\theta \in \Theta} L(\theta)$.
    Moreover, if there is no $\theta$ such that $p_\theta \txtover{a.s.}{=} p_{\theta_0}$, then $\theta_0$ is the unique minimizer of $L$.
    We give in \Cref{tab:glms} a few examples from the class of generalized linear models (GLMs) proposed by \citet{nelder1972generalized}.
\end{example}

\begin{example}[Score matching estimation]
    Another important example appears in \emph{score matching} \citep{hyvarinen2005estimation}.
    Let $\bbZ = \reals^\tau$.
    Assume that $\Prob$ and $P_\theta$ have densities $p$ and $p_\theta$ w.r.t the Lebesgue measure, respectively.
    Let $p_\theta(z) = q_\theta(z) / \Lambda(\theta)$ where $\Lambda(\theta)$ is an unknown normalizing constant. We can choose the loss
    \begin{align*}
        \score(\theta; z) := \Delta_z \log{q_\theta(z)} + \frac12 \norm{\nabla_z \log{q_\theta(z)}}^2 + \text{const}.
    \end{align*}
    Here $\Delta_z := \sum_{k=1}^p \partial^2/\partial z_k^2$ is the Laplace operator.
    Since \cite[Thm.~1]{hyvarinen2005estimation}
    \begin{align*}
        L(\theta) = \frac12 \Expect\left[ \norm{\nabla_z q_\theta(z) - \nabla_z p(z)}^2 \right],
    \end{align*}
    we have, when $p = p_{\theta_0}$, that $\theta_0 \in \argmin_{\theta \in \Theta} L(\theta)$.
    In fact, when $q_\theta > 0$ and there is no $\theta$ such that $p_\theta \txtover{a.s.}{=} p_{\theta_0}$, the true parameter $\theta_0$ is the unique minimizer of $L$ \cite[Thm.~2]{hyvarinen2005estimation}.
\end{example}

\myparagraph{Empirical risk minimization}
Assume now that we have an i.i.d.~sample $\{Z_i\}_{i=1}^n$ from $\Prob$.
To learn the parameter $\theta_\star$ from the data, we minimize the empirical risk to obtain the \emph{empirical risk minimizer}
\begin{align*}
    \theta_n \in \argmin_{\theta \in \Theta} \left[ L_n(\theta) := \frac1n \sum_{i=1}^n \score(\theta; Z_i) \right].
\end{align*}
This applies to both maximum likelihood estimation and score matching estimation. 
In \Cref{sec:main_results}, we will prove that, with high probability, the estimator $\theta_n$ exists and is unique under a generalized self-concordance assumption.

\begin{figure}
    \centering
    \includegraphics[width=0.45\textwidth]{graphs/logistic-dikin} %0.4
    \caption{Dikin ellipsoid and Euclidean ball.}
    \label{fig:logistic_dikin}
\end{figure}

\myparagraph{Confidence set}
In statistical inference, it is of great interest to quantify the uncertainty in the estimator $\theta_n$.
In classical asymptotic theory, this is achieved by constructing an asymptotic confidence set.
We review here two commonly used ones, assuming the model is well-specified.
We start with the \emph{Wald confidence set}.
It holds that $n(\theta_n - \theta_\star)^\top H_n(\theta_n) (\theta_n - \theta_\star) \rightarrow_d \chi_d^2$, where $H_n(\theta) := \nabla^2 L_n(\theta)$.
Hence, one may consider a confidence set $\{\theta: n(\theta_n - \theta)^\top H_n(\theta_n) (\theta_n - \theta) \le q_{\chi_d^2}(\delta) \}$ where $q_{\chi_d^2}(\delta)$ is the upper $\delta$-quantile of $\chi_d^2$.
The other is the \emph{likelihood-ratio (LR) confidence set} constructed from the limit $2n [L_n(\theta_\star) - L_n(\theta_n)] \rightarrow_d \chi_d^2$, which is known as the Wilks' theorem \citep{wilks1938large}.
These confidence sets enjoy two merits: 1) their shapes are an ellipsoid (known as the \emph{Dikin ellipsoid}) which is adapted to the optimization landscape induced by the population risk; 2) they are asymptotically valid, i.e., their coverages are exactly $1 - \delta$ as $n \rightarrow \infty$.
However, due to their asymptotic nature, it is unclear how large $n$ should be in order for it to be valid.

Non-asymptotic theory usually focuses on developing finite-sample bounds for the \emph{excess risk}, i.e., $\Prob(L(\theta_n) - L(\theta_\star) \le C_n(\delta)) \ge 1 - \delta$.
To obtain a confidence set, one may assume that the population risk is twice continuously differentiable and $\lambda$-strongly convex.
Consequently, we have $\lambda \norm{\theta_n - \theta_\star}_2^2 / 2 \le L(\theta_n) - L(\theta_\star)$ and thus we can consider the confidence set $\calC_{\text{finite}, n}(\delta) := \{\theta: \norm{\theta_n - \theta}_2^2 \le 2C_n(\delta)/\lambda\}$.
Since it originates from a finite-sample bound, it is valid for fixed $n$, i.e., $\Prob(\theta_\star \in \calC_{\text{finite}, n}(\delta)) \ge 1 - \delta$ for all $n$; however, it is usually conservative, meaning that the coverage is strictly larger than $1 - \delta$.
Another drawback is that its shape is a Euclidean ball which remains the same no matter which loss function is chosen.
We illustrate this phenomenon in \Cref{fig:logistic_dikin}.
Note that a similar observation has also been made in the bandit literature \citep{faury2020improved}.

We are interested in developing finite-sample confidence sets.
However, instead of using excess risk bounds and strong convexity, we construct in \Cref{sec:main_results} the Wald and LR confidence sets in a non-asymptotic fashion, under a generalized self-concordance condition.
These confidence sets have the same shape as their asymptotic counterparts while maintaining validity for fixed $n$.
These new results are achieved by characterizing the critical sample size enough to enter the asymptotic regime.

\section{Demand Selection and Revenue Maximization (Stackelberg Game)} \label{game1}
%!TEX root = AlshehriLiuChenBasar.tex
In this section, we solve the above optimization problems in closed form and show that the solutions are unique.

%\subsection{Consumer- and Company-side Analyses}

\subsection{Consumer-Side Analysis}
%Note that the consumer-side utility function  is strictly concave and the constraints are linear.
%Refer to \cite{boyd,NL} for details about analyzing and solving such problems.
We start by relaxing the minimum energy constraint (\ref{cc}).
  For each consumer $n\in \scr{N}$, the associated Lagrange function is given as follows:
\begin{eqnarray*}
L_n &=& \gamma_n\sum_{k\in \scr{K}}\sum_{t\in \scr{T}}\ln(\zeta_n+d_{n,k}(t))
 \\
&&- \lambda_{n,1}\left(\sum_{k\in \scr{K}}\sum_{t\in \scr{T}}p_k(t)d_{n,k}(t)-B_n\right) \\
&&+\sum_{k\in\scr{K}} \sum_{t\in\scr{T}} \lambda_{n,2}(k,t)d_{n,k}(t)
\end{eqnarray*}
%\begin{align}\nonumber
%&L_n=\gamma_n\sum_{k\in \scr{K}}\sum_{t\in \scr{T}}\ln(\zeta_n+d_{n,k}(t))
%\cr
%&-\lambda_{n,1}(\sum_{k\in \scr{K}}\sum_{t\in \scr{T}}p(t)_kd_{n,k}(t)-B_n)+ \lambda_{n,2}(1,1)d_{n,1}(1)
%\cr
%&+\lambda_{n,2}(1,2)d_{n,1}(2)+\dots+\lambda_{n,2}(K,T)d_{n,K}(T)\end{align}
  where ${\bf \lambda_{n}}$ are the Lagrange multipliers. The KKT conditions of optimality in this case are sufficient because the objective function is strictly concave and the constraints are linear \cite{NL}, and solving for them leads to \begin{equation}
d^*_{n,k}(t)= \frac{B_n+\sum_{j\in \scr{K}}\sum_{h\in \scr{T}}p_j(h)\zeta_n}{KTp_k(t)}-\zeta_n,  \; \forall \;t\in \scr{T}, \; k\in \scr{K},  \label{xx}
\end{equation}
which is a generalization of the single-period case in \cite{sabita}. A detailed derivation of (\ref{xx}) can be found in \cite{mywork}. {\color{black} We remark that $d^*_{n,k}(t) \geq 0$  {\color{blue} because the objective function is strictly increasing.}}

The following theorem, whose proof can be found in the Appendix, states the necessary and sufficient condition for $B_n$ so that the above demands meet the minimum energy constraint (\ref{cc}).


\begin{theorem}
For each consumer $n \in \scr{N}$, the demands $d^*_{n,k}(t)$ given by (\ref{xx}) satisfy (\ref{cc}) if, and only if, 
\begin{equation} B_n \geq \frac{E_n^{{\rm min}}+\zeta_nKT}{\sum_{k\in \scr{K}}\sum_{t\in \scr{T}}\frac{1}{KTp_k(t)}}-\zeta_n \sum_{k\in \scr{K}}\sum_{t\in \scr{T}}p_k(t). \label{budget} \end{equation}
\end{theorem}

{\color{black}
\begin{remark} The above theorem can be interpreted as billing costs minimization. At the equality of (\ref{budget}), $B_n$ corresponds to the minimum budget needed for consumer $n$ to satisfy his energy need constraint, given the set of prices chosen by utility companies. Such a minimum $B_n$ can serve as a theoretical benchmark in which one can measure whether or not consumers are paying more than what is necessary.  We later demonstrate that with real data from demand response experiments, using the equality in (\ref{budget}) leads to savings in the range of $10\%-30\%$. \hfill $\Box$
\end{remark}


\begin{assumption}
For each consumer $n$, the budget $B_n$ satisfies the condition (\ref{budget}).
\label{assumption1}
\end{assumption}
}
%Now suppose that
%$$\frac{B_n+\zeta_n\sum_{k\in \scr{K}}\sum_{t\in \scr{T}}p_k(t)}{KTp_k(t)} -\zeta_n \geq 0 \,\,\,\, \forall k \in \scr{K},\,\, t \in \scr{T}$$
\subsection{Company-Side Analysis} 
%Given the prices set by the other companies subject to the power availability constraint (\ref{prob2}), each UC (leader) aims to determine its most profitable prices. At the leaders level, there is a noncooperative game in which each UC chooses its optimal prices in response to the prices set by the other UCs. 
We apply the demands derived in the consumers-side analysis (which were functions of the prices) and show that optimality is achieved at the equality of the constraint (\ref{prob2}). We start by solving for prices that satisfy the equality at (\ref{prob2}) and then prove that they are revenue-maximizing, strictly positive, and unique. 
Consider the equality in (\ref{prob2}), and by the optimal demands (\ref{xx}), there holds
$$
\frac{\sum_{n\in \scr{N}}B_n+\sum_{n\in \scr{N}}\zeta_n\sum_{j\in \scr{K}}\sum_{h\in \scr{T}}p_j(h)}{KTp_k(t)} =\sum_{n\in \scr{N}}\zeta_n + G_k(t),
$$
for all $t \in \scr{T}$. 
%for all $t \in \scr{T}$.
Let $Z=\sum_{n\in \scr{N}}\zeta_n$ and $B=\sum_{n\in \scr{N}}B_n$. Then, for each company $k \in \scr{K}$,
\begin{equation} B+Z\sum_{j\in \scr{K}}\sum_{h\in \scr{T}}p_j(h) = KTp_k(t)(G_k(t)+Z),\;\; \forall \;t \in \scr{T}.  \label{AP} \end{equation}
%Note that the double summation includes $p_k(t)$ and all the other prices.
%Thus, \begin{equation}\begin{split}B+Z\sum_{e\in \scr{K}}\sum_{h\in \scr{T}}p_e(h)= KTp_k(t)(G_k(t)+Z)-p_k(t)Z, \\ \,\,\,\, \forall \;t \in \scr{T}, \;\forall \;k \in \scr{K}, \; (e,h)\neq (k,t) \label{AP}\end{split}\end{equation}
The above equation (\ref{AP})  can be presented as the following system of linear equations
\begin{equation}AP=Y,\label{AP2}\end{equation}
%\begin{equation}\underbrace{\begin{pmatrix}
%KT(G_1(1)+Z)-Z& -Z &\dots& -Z\\
%-Z & KT(G_1(2)+Z)-Z& \dots& -Z\\
%\vdots & \ddots\\
%-Z &\dots&-Z & KT(G_K(T)+Z)-Z
%\end{pmatrix}}_{A}
%\underbrace{\begin{pmatrix}
%p_1(1) \\
%\vdots \\
%p_1(T)\\
%p_2(1)\\
%\vdots\\
%p_2(T)\\
%\vdots\\
%p_K(T) \end{pmatrix}}_P=\underbrace{\begin{pmatrix}
%B\\
%B\\
%\vdots \\
%B\end{pmatrix}}_Y \label{AP2}\end{equation}
%
where $A$ is a $KT\times KT$ matrix 
whose diagonal entries are $KT(G_k(t)+Z)-Z$, $k\in\scr{K}$, $t\in\scr{T}$,
and off-diagonal entries all equal to $-Z$, 
%\footnotesize{\begin{align}\nonumber
%&A=\begin{pmatrix}
%KT(G_1(1)+Z)-Z& -Z &\dots& -Z\\
%-Z & KT(G_1(2)+Z)-Z& \dots& -Z\\
%\vdots & \ddots\\
%-Z &\dots&-Z & KT(G_K(T)+Z)-Z
%\end{pmatrix}\end{align}}
$P$ is a vector in $\R^{KT}$ stacking $p_k(t)$, $k\in\scr{K}$, $t\in\scr{T}$,
and $Y$ a vector in $\R^{KT}$ whose entries all equal to $B$.
%\begin{eqnarray*}
%P &=& \begin{pmatrix}
%p_1(1) &
%\cdots &
%p_1(T) &
%p_2(1) &
%\cdots &
%p_2(T) &
%\cdots &
%p_K(T) \end{pmatrix}^T\\
%Y &=&\begin{pmatrix}
%B &
%B &
%\cdots &
%B\end{pmatrix}^T
%\end{eqnarray*}

%The following results say that matrix $A$ is invertible and the revenue-maximizing prices are positive and unique. 
%We also prove that the Nash equilibrium is at these prices.
We have the following results (proofs are in the Appendix).
\begin{lemma}
The matrix $A$ is invertible.
\end{lemma}


\begin{lemma}
The prices that solve (\ref{AP2}) are strictly positive and are unique. For each $t\in\mathcal{T}$, $k\in\mathcal{K}$, the price is given by
    \begin{equation} p^*_k(t)=\frac{B}{G_k(t)+Z}\left(\frac{1}{KT-\sum_{j\in \scr{K}}\sum_{h\in \scr{T}}\frac{Z}{G_j(h)+Z}}\right),\label{p}\end{equation}
where $B=\sum_{n\in \scr{N}}B_n$ and $Z=\sum_{n\in \scr{N}}\zeta_n$.
\end{lemma}


\begin{remark} 
Letting $\zeta_n=1$ for each consumer, the value of $Z$ coincides with $N$. In this case, by (\ref{p}), we observe that for any given ${\bf G_k}$, the price $p^*_k(t)(G_k(t)+N)$ is a constant for all $t \in \scr{T}$ and $k \in \scr{K}$.
Thus, the power availability is inversely proportional to the prices. \hfill$\Box$
%Whenever any of the $G_k(t)$'s changes, the constant on the right side changes, by (\ref{p}).
\end{remark}

\begin{remark} 
Lemma 2 provides a computationally cheap expression for the prices. Since $p^*_k(t)$ can be directly computed using (\ref{p}), there is no need to numerically compute $A^{-1}$ or $|A|$ to solve (\ref{AP2}). This enables us to deal with a large number of periods or utility companies, without worrying about computational complexity.\hfill$\Box$
\end{remark}

{\color{black}Due to production costs and market regulations, $p^*_k(t)$ cannot be outside the range of some lower and upper bounds $[p^{{\rm min}}_k(t),p^{{\rm max}}_k(t)]$  for all $t \in \scr{T}$ and $k \in \scr{K}$, as in \cite{sabita}. If $p^*_k(t)<p^{{\rm min}}_k(t)$, then $p^*_k(t)$ is set to $p^{{\rm min}}_k(t)$, and similarly for the upper-bound, if  $p^*_k(t)>p^{{\rm max}}_k(t)$, then we set  $p^*_k(t)=p^{{\rm max}}_k(t)$. Accordingly, denote the strategy space of utility company $k$ (a leader in the game) at $t$ by $\scr{L}_{k,t}:=[p^{{\rm min}}_k(t),p^{{\rm max}}_k(t)]$. The strategy space of $k$ for the entire time horizon is $\scr{L}_{k}=\scr{L}_{k,1}\times\dots\times\scr{L}_{k,T}$.
 The strategy space of all companies is $\scr{L}=\scr{L}_{1}\times\dots\times\scr{L}_{K}$. {\color{blue} For given price selections  ${\bf{p}}:=({\bf{p}}_1,\dots,{\bf{p}}_K) \in \scr{L}$}, the optimal response from all consumers is
$${\bf{d^*(p)}}=\{{\bf{d}}_1^*({\bf{p}}),{\bf{d}}_2^*({\bf{p}}),\dots,{\bf{d}}_N^*({\bf{p}})\}$$
where for each $n \in \scr{N}$, ${\bf{d}}^{*}_{n}({\bf{p}})$ is the unique maximizer for $u_n({\bf{d}}_n,{\bf{p}})$ and is given by (\ref{xx}).
}
We now have the following theorem, whose proof can be found in the Appendix.


\begin{theorem}[Existence and Uniqueness of the Stackelberg Equilibrium]Under Assumption \ref{assumption1}, the following statements hold: 
\begin{itemize}
\item[(i)] There exists a unique (open-loop) Nash equilibrium for the price-selection game and it is given by (\ref{p}).\\
\item[(ii)] There exists a unique (open-loop) Stackelberg equilibrium, and it is given by the demands in (\ref{xx}) and the prices in (\ref{p}).
\end{itemize}
%\item The maximizing demands given by (\ref{xx}) and the revenue-maximizing prices given in Lemma 2 constitute the (open-loop) Stackelberg equilibrium for the demand response management game.
\label{mainTHM}
\end{theorem}

At the Stackelberg equilibrium, it can easily be verified that  
\begin{equation}\sum_{k\in \scr{K}}\pi_k({\bf{p}}^*_k,{\bf{p}^*_{-k}})=\sum_{n\in \scr{N}}B_n.\label{eq1}\end{equation}
One observation is that when a company gains in terms of revenue, the same amount must be lost by other companies because the sum of revenues is a constant, which demonstrates a conflict of objectives between utility companies. However, by the definition of the equilibrium strategy, this is the best each company can do, for fixed power availabilities ${\bf G_k}$. But, given a total amount of available power, $G^{{\rm total}}_k$, a company has across the time horizon, it is possible that it gains in terms of revenue by an efficient power allocation. This motivates us to formulate a power allocation game and analytically answer the following question: How can company $k$ allocate its power so that it maximizes its revenue? {\color{black} Furthermore, for now, for ease of exposition, we neglect network and other company-specific constraints. Such considerations are later discussed in Section \ref{generalizations}.  For the remaining part of this paper, unless otherwise stated, we also have the following simplifying assumption.

\begin{assumption}
For each consumer $n$, we have $$\gamma_n=\zeta_n=1.$$ \label{assumption2}
\end{assumption}

The above assumption implies that $Z$ is equal to the number of consumers $N$.}



\section{Power Allocation (Nash Game)} \label{game2}
%!TEX root = AlshehriLiuChenBasar.tex

\begin{figure*}
\centering
\includegraphics[width=.9\linewidth]{sketch_DR2.pdf}
\caption{The interaction between companies and their consumers, along with power allocation. First, companies play a Nash power allocation game. Once power availabilities are allocated across all periods, companies and consumers play the Stackelberg game which dictates optimal prices and demand selection.}
\label{sketch_DR2}
\end{figure*}
In this section, we exploit the closed-form solutions for consumer demands and companies' prices to formulate and solve a power allocation game for companies. We note that while we use the closed-form solutions to define the power allocation game, it is to be played {\em{before}} the Stackelberg game, and its outcomes define the fixed power availabilities in the constraints of the companies in the Stackelberg game. Given the power availabilities from other companies, ${\bf{G_{-k}}}$, and since the equality in (\ref{prob2}) is satisfied at equilibrium, the revenue function of company $k$ can be represented as
\begin{equation} \pi_k({\bf{G}}_k,{\bf{G_{-k}}})=\sum_{t\in \scr{T}}p^*_k(t)G_k(t). \end{equation}  The optimal prices (\ref{p}) are functions of ${\bf{G}}_k$ and ${\bf{G_{-k}}}$, leading to the revenue function being equal to
\begin{equation} B\sum_{t\in \scr{T}}\frac{G_k(t)}{(G_k(t)+N)(KT-\sum_{j\in \scr{K}}\sum_{h\in \scr{T}}\frac{N}{G_j(h)+N})}, \label{Uk} \end{equation}
where $B=\sum_{n\in \scr{N}}B_n$. {Note that company $k$ receives a {\it fraction} of the total budgets. This fraction depends on what company $k$ offers in the multi-period-multi-company demand response framework, and what other companies also offer. Thus, when company $k$ can change what it offers, it can potentially increase the fraction it receives, and the power allocation game becomes natural, since the revenue function depends on other players' decisions. } For this game, which can be played before the Stackelberg game, which we have already solved, companies allocate their powers across all periods, and the outcome dictates the fixed power availabilities for the Stackelberg game. Figure \ref{sketch_DR2} provides an illustration. 

%However, our numerical results show that although multi-period demand response provides incentives for consumer participation, some companies can gain in terms of revenues while others can lose. But the sum of revenues of all companies is a constant that is equal to the sum of budgets (consumers use up all their budgets for demand selection and these budgets eventually go to companies). This conflict of objectives motivates us to formulate a power allocation game and analytically answer the following question: How can company $k$ allocate its power so that it maximizes its revenue, given its total amount of power available $G^{{\rm total}}_k$ for the entire time-horizon?

%By the above optimal prices and demands, and given the power availability from other companies by ${\bf{G_{-k}}}$, the revenue function of company $k$ can be represented as
%\begin{equation*} U_k := U_{k}(G_k,{\bf{G_{-k}}})=\sum_{t\in \scr{T}}p^*_k(t)G_k(t) \end{equation*} By representing the optimal prices (\ref{p}) as functions of $G_k(t)$'s,
%\begin{equation} U_k=B\sum_{t\in \scr{T}}\frac{G_k(t)}{(G_k(t)+N)(KT-\sum_{k\in \scr{K}}\sum_{t\in \scr{T}}\frac{N}{G_k(t)+N})} \label{Uk} \end{equation}
%where $B=\sum_{n\in \scr{N}}B_n$.

Let the total capacity for company $k$ for the entire time horizon be $G^{{\rm total}}_k$. Denote the action set of company $k$ at time $t$ by $\scr{P}_{k,t}:=[0,G^{{\rm total}}_k]$. Thus, given ${\bf{G_{-k}}}$,  the company $k$ solves the following problem:
\begin{eqnarray}
\underset{\mathbf{G}_k }{\hbox{maximize}} && \pi_k({\bf{G}}_k,{\bf{G_{-k}}})
\nonumber \\
\hbox{subject to} && \sum_{t\in \scr{T}}G_k(t) \leq G^{{\rm total}}_k, \\ \label{prob3}
&& G_k(t)\geq0, \ \forall t\in\scr{T}. \nonumber\end{eqnarray}

{\color{black}The above problem is only applicable for the case when generation is fully controllable. For the smart grid, because of the availability of various generation sources, full-controllability does not always hold, and in fact, for renewable resources it could be completely gone. We demonstrate the possibility of relaxing this assumption later in Section \ref{generalizations}.}

\subsection{Existence and Uniqueness of Nash Equilibrium} 
%Note that (\ref{Uk}) is equivalent to
%\begin{equation}  U_{k}(G_k,{\bf{G_{-k}}})= \sum_{t\in \scr{T}}\frac{BG_k(t)}{(G_k(t)+N)(\alpha_{-k}-\sum_{t\in \scr{T}}\frac{N}{G_k(t)+N})} \label{Uk2} \end{equation}
%where \begin{eqnarray*}  \alpha_{-k} & := & KT-\sum_{j\in \scr{K},j\neq k\,\,} \sum_{t\in \scr{T}}\frac{N}{G_j(t)+N} \\
%& > & KT-(K-1)T=T.\end{eqnarray*}
%
%Note that $\alpha_{-k}$ depends on the strategies of other companies and it is fixed for company $k$. 
The following theorem, whose proof can be found in the Appendix, states the existence and uniqueness of Nash equilibrium in the power allocation game, and provides an expression for it.
\begin{theorem}
Under Assumptions \ref{assumption1}-\ref{assumption2}, if ${\bf G}_k$ is fully controllable, there exists a unique pure-strategy Nash equilibrium for the power allocation game,
and it is given by \begin{equation} G^*_k(t)=\frac{G^{{\rm total}}_k}{T} \,\,\,,\,\, \forall \,t\in \scr{T}, \forall \, k\in \scr{K}.\label{NE}\end{equation}\label{allocation}\end{theorem}

Interestingly, the optimal strategy for each company is to equally allocate its power across all time periods. The proof of Theorem \ref{allocation} reveals that (\ref{Uk}) is strictly concave and increasing in each $G_k(t)$. This is an important property that allows accommodating further company-specific operational constraints and relaxing the full-controllability assumption. To illustrate, suppose that company $k$ has a mix of generation sources for which generation is controllable for some periods and only partially controllable for others. Then, it can add linear constraints to problem (\ref{prob3}) reflecting inter-temporal considerations at the generation-side (such as ramping limits). Existence and uniqueness of a pure-strategy Nash equilibrium are still guaranteed due to the strict concavity of the objective \cite{basar}. Since generation costs are typically assumed to be convex \cite{bosebook} (denote it by $c_k$ for each company $k$), company $k$ can also allocate its generation to maximize its profit, by subtracting the cost from (\ref{Uk}). One can alter the objective function of the power allocation game to 
%\begin{equation*} B\sum_{t\in \scr{T}}\frac{G_k(t)}{(G_k(t)+N)(KT-\sum_{j\in \scr{K}}\sum_{h\in \scr{T}}\frac{N}{G_j(h)+N})}\end{equation*}
%   to 
   \begin{align}&B\sum_{t\in \scr{T}}\frac{G_k(t)}{(G_k(t)+N)(KT-\sum_{j\in \scr{K}}\sum_{h\in \scr{T}}\frac{N}{G_j(h)+N})}\nonumber \\
   & \qquad\qquad\qquad \qquad-\sum_{t\in \scr{T}} c_k(G_k(t)),\label{cvx}\end{align}
   
 \noindent and the problem reflects profit-maximization in this case. Using (\ref{cvx}) and following our analysis, {\color{blue} we conclude that each company maximizes a strictly concave function}, and one can easily conclude the existence of a pure-strategy Nash equilibrium in this case. 
%   Furthermore, if $c_k$ is strictly convex, then, it is a unique Nash equilibrium \cite{basar}. 
%% This declares a command \Comment
%% The argument will be surrounded by /* ... */
\SetKwComment{Comment}{/* }{ */}

\begin{algorithm}[t]
\caption{Training Scheduler}\label{alg:TS}
% \KwData{$n \geq 0$}
% \KwResult{$y = x^n$}
\LinesNumbered
\KwIn{Training data $\mathcal{D}_{train}=\{(q_i, a_i, p_i^+)\}_{i=1}^m$, \\
\qquad \quad Iteration number $L$.}
\KwOut{A set of optimal model parameters.}

\For{$l=1,\cdots, L$}{
    Sample a batch of questions $Q^{(l)}$\\
    \For{$q_i\in Q^{(l)}$}{
        $\mathcal{P}_{i}^{(l)} \gets \mathrm{arg\,max}_{p_{i,j}}(\mathrm{sim}(q_i^{en},p_{i,j}),K)$\\
        $\mathcal{P}_{Gi}^{(l)} \gets \mathcal{P}_{i}^{(l)}\cup\{p^+_i\}$\\
        Compute $\mathcal{L}^i_{retriever}$, $\mathcal{L}^i_{postranker}$, $\mathcal{L}^i_{reader}$\\ according to Eq.\ref{eq:retriever}, Eq.\ref{eq:rerank}, Eq.\ref{eq:reader}\\
    }
    % $\mathcal{L}^{(l)}_{retriever} \gets \frac{1}{|Q^{(l)}|}\sum_i\mathcal{L}^i_{retriever}$\\
    % $\mathcal{L}^{(l)}_{retriever} \gets \mathrm{Avg}(\mathcal{L}^i_{retriever})$,
    % $\mathcal{L}^{(l)}_{rerank} \gets \mathrm{Avg}(\mathcal{L}^i_{rerank})$,
    % $\mathcal{L}^{(l)}_{reader} \gets \mathrm{Avg}(\mathcal{L}^i_{reader})$\\
    % Compute $\mathcal{L}^{(l)}_{retriever}$, $\mathcal{L}^{(l)}_{rerank}$, and $\mathcal{L}^{(l)}_{reader}$ by averaging over $Q^{(l)}$\\
    $\mathcal{L}^{(l)} \gets \frac{1}{|Q^{(l)}|}\sum_i(\mathcal{L}^{i}_{retriever} + \mathcal{L}^{i}_{postranker}+ \mathcal{L}^{i}_{reader})$\\
    $\mathcal{P}^{(l)}_K\gets\{\mathcal{P}^{(l)}_i|q_i\in Q^{(l)}\}$,\quad $\mathcal{P}^{(l)}_{KG}\gets\{\mathcal{P}^{(l)}_{Gi}|q_i\in Q^{(l)}\}$\\
    Compute the coefficient $v^{(l)}$ according to Eq.~\ref{eq:v}\\
  \eIf{$ v^{(l)}=1$}{
    $\mathcal{L}^{(l)}_{final} \gets \mathcal{L}^{(l)}(\mathcal{P}_{KG}^{(l)})$\\
  }{
      $\mathcal{L}^{(l)}_{final} \gets \mathcal{L}^{(l)}(\mathcal{P}^{(l)}_{K}),$\\
    }
    Optimize $\mathcal{L}^{(l)}_{final}$
}
\end{algorithm}


%  \eIf{$ \mathcal{L}^{(l-1)}_{retriever}<\lambda$}{
%     $\mathcal{L}^{(l)}_{final} \gets \mathcal{L}^{(l)}(\mathcal{P}_K^{(l)})$\\
%   }{
%       $\mathcal{L}^{(l)}_{final} \gets \mathcal{L}^{(l)}(\mathcal{P}^{(l)}_{KG}),$\\
%     }
\section{Asymptotic Regimes} \label{asymp}
%!TEX root = main.tex
{\color{black}In this section, we study the asymptotic (limiting) behavior as $T\rightarrow \infty$ or $N\rightarrow \infty$. While neither $T$ or $N$ can be arbitrarily large in practice, analyzing the asymptotic behavior brings in deep insights. For example, it reveals that consumers benefit as $T$ grows. As $N$ grows, our asymptotic analysis allows us to compute an appropriate company-to-consumer ratio $\frac{K}{N}$. We show these insights by studying how the utility functions, revenues, prices, and demands are affected as $T$ or $N$ grows. {For the rest of this section, in addition to Assumptions \ref{assumption1}-\ref{assumption2}, we assume the following. 
\begin{assumption} 
The total power available for the entire time horizon $G_k^{{\rm total}}$ is the same for each company $k\in\scr{K}$.
\label{assumption3}
\end{assumption}
}}
\subsection{When the Number of Periods Grows} Under Assumptions \ref{assumption1}-\ref{assumption3}, at equilibrium, it follows that 
%we have
%\begin{equation*}\sum_{j\in \scr{K}}\sum_{h\in \scr{T}}\frac{N}{G^*_j(h)+N}=KT\frac{N}{G^*_k(t)+N} \label{GT}\end{equation*}
%\begin{equation*}\sum_{j\in \scr{K}}\sum_{h\in \scr{T}}p^*_j(h)=KTp^*_k(t) \label{GT2}\end{equation*}
%Furthermore, 
optimal prices and demands are given by 
\begin{equation} p^*_k(t)=\frac{\sum_{m\in \scr{N}}B_m}{KTG^*_k(t)}=\frac{\sum_{m\in \scr{N}}B_m}{KG^{{\rm total}}_k}, \label{ppp} \end{equation}
\begin{equation}d^*_{n,k}(t)=\frac{B_n+KTp^*_k(t)}{KTp^*_k(t)}-1=\frac{G^{{\rm total}}_kB_n}{T\sum_{m\in \scr{N}}B_m},\,\, \label{t1} \end{equation}
%By (\ref{t1}), the payoff of consumer $n$ becomes
and the utility of consumer $n$ becomes\begin{equation} u_n=KT\ln\left(1+\frac{G^{{\rm total}}_kB_n/\sum_{m\in \scr{N}}B_m}{T}\right),\end{equation}
in which $G^{{\rm total}}_kB_n/\sum_{m\in \scr{N}}B_m$ is positive. Thus, as $T$ increases, the multiplicative term $ KT$ of the logarithmic function increases at a faster rate than the decrease of 
$\ln\left(1+{B_nG^{{\rm total}}_k/B}/{T}\right)$. 
Hence, as $T$ increases, the utility of each consumer $n \in \scr{N}$ monotonically increases.
Taking the limit, it can be verified that 
\begin{equation}\lim_{T\rightarrow\infty} u_n(T)=\frac{KG^{{\rm total}}_kB_n}{\sum_{m\in \scr{N}}B_m}.\end{equation}
Furthermore, note that the demand $d^*_{n,k}(t)$ from consumer $n\in \scr{N}$ from company $k\in \scr{K}$ at time $t\in \scr{T}$ converges to zero as $T \rightarrow \infty$. We claim that the revenues are constants. To see this, recall that
\begin{align*}
\pi_k({\bf{p}}^*_k,{\bf{p}^*_{-k}}) &= p^*_k(t)G^{{\rm total}}_k = \frac{\sum_{m\in \scr{N}}B_m}{K},
\end{align*}
which is a constant since both the number of companies and the budgets of the consumers are fixed.

\begin{remark} At the equilibrium, the monotonicity of the utilities of the consumers shows that increasing the number of periods leads to more incentives for consumers' participation in demand response. However, it might not be very beneficial to increase the number of periods to a very high value. First, the rate of increase in terms of consumers' utilities gets progressively smaller. Second, having a high number of periods leads to smaller demands for each period and that might violate some minimum energy need for particular periods at the consumers' level. So, it is beneficial to increase the number of periods up to a certain point (compared to having $T=1$), but it might not be beneficial to let $T$ become arbitrarily large. 
%t would be interesting to study what would be the appropriate number of periods that keeps consumers motivated to participate in demand response  while being practical.
\hfill$\Box$
\end{remark}


\begin{remark} Note that the limit point of the utility function of consumer $n$ is the proportion of his budget to the total budgets times the total power availability. So if a particular consumer has $1\%$ of the sum of all the budgets, he gets $1\%$ of the available power. Furthermore, the revenue for each company is the proportion of the sum of the budgets to the number of companies. In addition, the demand by consumer $n$ from company $k$ at time $t$ is the proportion of his budget to the total budgets times the total power availability at $t$ from $k$.
\hfill$\Box$
\end{remark}



\subsection{When the Number of Consumers Grows} When the number of consumers increases, each additional consumer has some budget $B_n$. With the total power availability from companies being fixed, they will increase their prices. {\color{black}  We have the following simplifying assumption. 

\begin{assumption} 
The budget for each consumer $n \in \scr{N}$ is the same.
\label{assumption4}
\end{assumption}



Under Assumptions \ref{assumption1}-\ref{assumption4}, we increase the number of consumers $N$ and see what happens as $N \rightarrow \infty$.} In this case, the optimal prices and demands become
%\begin{eqnarray}p^*_k(t)&=&\frac{NB_n}{KTG^*_k(t)} \in \scr{L}_{k,t}:=[p^{{\rm min}}_k(t),p^{{\rm max}}_k(t)]\label{pppp}\\
\begin{eqnarray}p^*_k(t)&=&\frac{NB_n}{KTG^*_k(t)} \label{pppp}\\
d^*_{n,k}(t) &=& \frac{G^{{\rm total}}_k}{TN} \label{dd} \end{eqnarray}
Clearly, $p^*_k(t)\rightarrow\infty$ as $N\rightarrow\infty$ and $d^*_{n,k}(t)\rightarrow0$ as $N\rightarrow\infty$. When the population is large and the power availability is fixed, it is not surprising that $d^*_{n,k}(t)\rightarrow0$ because the portion each consumer can get from the available power gets smaller and smaller as $N$ increases. Furthermore, it can be easily verified that $\lim_{N\rightarrow \infty}\pi_k(N)=\infty$ and $\lim_{N\rightarrow \infty}u_n(N)=0$. Thus, with the limit points resulting in unrealistic outcomes, a balance between the supply and demand needs to be achieved, which we do by finding an appropriate company-to-consumer ratio. 

Now, the question we ask is: For a given maximum allowable market price $p^{{\rm max}}_k(t)$, call it $p^{{\rm max}}$, what is the appropriate company-to-consumer ratio $\frac{K}{N}$? If there are more companies than necessary in the market, there will be losses in terms of revenues. On the other hand, if there are fewer companies than necessary, the prices can exceed $p^{{\rm max}}$, leading to undesirable outcomes. The following theorem, whose proof can be found in the Appendix, provides an optimal ratio at which prices do not exceed $p^{{\rm max}}$ and the revenues being maximized while satisfying the equality in (\ref{eq1}). 
{\color{black}
\begin{theorem}
Under Assumptions \ref{assumption1}-\ref{assumption4}, at the NE of the power allocation game, and at the Stackelberg equilibrium of the price and demand selection game, the optimal prices given by (\ref{p}) satisfy  
\begin{eqnarray*}
 p^*_k(t) &\leq &p^{{\rm max}},\\
 \sum_{k\in \scr{K}}\pi_k({\bf{p}}^*_k,{\bf{p^*_{-k}}})&=&\sum_{n\in \scr{N}}B_n,
\end{eqnarray*}
if, and only if, \begin{equation*} \frac{K}{N} \geq \frac{B_n}{p^{{\rm max}}TG^*_k(t)}, \end{equation*}
for each $t \in \mathcal{T}$ and $k\in\mathcal{K}$.
\end{theorem}}

%\begin{theorem}
%Suppose that the total power availabilities $G^{{\rm total}}_k$ for all companies are the same. Then, at the NE of the power allocation game, and at the Stackelberg equilibrium of the price and demand selection game, the following hold:\\
%(i) If  $N\leq {p^{{\rm max}}KG^{{\rm total}}_k}/{B_n}$, then the optimal prices $p^*_k(t)$'s given by (\ref{pppp}) are feasible and the supply-demand balance (\ref{prob2}) is satisfied.\\
%(ii) If  $N>{p^{{\rm max}}KG^{{\rm total}}_k}/{B_n}$, then the optimal company-to-consumer ratio that maximizes the revenues without exceeding $p^{{\rm max}}$ is 
%$K/N=B_n/(p^{{\rm max}}TG^*_k(t))$.
%\end{theorem}


\section{Case Studies} \label{numerical}
%!TEX root = AlshehriLiuChenBasar.tex


In this section, we present results on some case studies on representative days from a Dutch smart grid pilot \cite{dutch} and the EcoGrid EU project \cite{ecogrid}. We numerically study optimal prices and demands, and their corresponding payments and utility functions. We show how our approach results in monetary savings for consumers. Furthermore, we show that increasing the number of periods provides more incentives for consumers' participation in demand response management. Additionally, we demonstrate the fast convergence of our distributed algorithm to optimal prices. We also release an open-source interactive tool containing the simulations in \cite{tool2}. For both the Dutch smart pilot and the EcoGrid EU projects, the data are unavailable in raw format. Thus, whenever it is needed, we estimate some data points from figures available in the corresponding references \cite{dutch,ecogrid}. 

Recall that at the Stackelberg equilibrium, the total power availabilities ${\bf G}$ match the aggregate demands. That is, 
$$ \sum_{n \in \mathcal{N}} d^*_{n,k}(t)=G_k(t), \qquad \forall t\in\mathcal{T}, k\in\mathcal{K}.$$ Here, we use the experimental hourly variation of the total demands to choose values for ${\bf G}$ and the minimum energy need ${\bf E}^{\min}$. This allows us to establish a common aspect between our results and the experimental results, so that we can appropriately explore how our framework compares to real-life experiments. We also use the lower-bound on the minimum budget condition (\ref{budget}), so that we can also quantify potential savings.  
{\color{blue}  From the consumers' perspective, the prices are given parameters in both our model and the experimental setups. The optimal demands are functions of the prices, and the optimal prices naturally depend on the parameters of the consumers and companies. To bring deep insights, we make the differentiating aspect between our model and the experimental results an economic one. And hence, we pick the parameters such that the equilibrium demands and experimental ones are similar, but the prices, and essentially what consumers pay, are different. Utilizing Theorem 1, we conclude that the equilibrium prices bring savings to consumers, and by definition, they automatically consider the incentives of companies as they are revenue-maximizing. A main conclusion of this paper is that this quantifies the economic gap, in terms of consumer savings, between our game-theoretic benchmark and existing experimental results. On the other hand, in our analysis, we have relaxed some constraints for tractability, such as power flow and demand inelasticity considerations, and it remains open to explore the underlying tradeoffs, since adding these considerations might reduce the potential monetary savings for consumers. Such considerations were not directly included in the models studied in \cite{ecogrid,dutch}, as their focus was to experimentally explore the consumers' behavior in response to changing prices. It is worth mentioning that the results in \cite{dutch} revealed that consumers are mainly flexible about adjusting the consumption of white goods (washing machine, dishwasher, etc). It was also concluded in \cite{ecogrid} that demand response did not result in distribution feeder congestion relief, and consumers with automatic equipment were the most responsive ones. Nevertheless, later in Section IX, we demonstrate how our framework can be utilized to include additional network and consumer-specific and/or company-specific constraints, which make it possible to add constraints for congestion relief. Including such constraints will likely make it necessary to compute the equilibrium prices and demands algorithmically, which we relegate to future endeavors, as we emphasize here more on revealing deep insights via having tractable analysis.}
 \begin{figure*}
\centering
\includegraphics[width=\linewidth, height=2in]{EcoGridEU_Price_Power.pdf}
\caption{Total power offered by company (left), Stackelberg game and EcoGrid EU experimental prices (middle), and the cumulative payments and billing savings for all consumers (right). }
\label{EcoGridEU_Price_Power}
\end{figure*}


\subsection{EcoGrid EU Project} This demand response project was conducted from March 2011 to August 2015 in Bornholm, Denmark. The number of consumers in this experiment was approximately $2000$. For a representative day (December 5th, 2014), we apply our method to hourly prices and shiftable demand consumption from this experiment. The experimental prices are in  \text{DKK}/MWh and we scale them to  \text{DKK}/kWh. We start by assuming that there is only one company ($K=1$) and letting the consumers to be homogeneous (they have the same budgets and energy need) with $N=2000$, and then generalize the results to $K>1$ and heterogeneous consumers. Since we are taking hourly prices for a day, we have $T=24$. 

\subsubsection{Finding the necessary parameters}In our model, for each period $t$, we have a fixed power availability $G_1(t)$ on the supply-side. Also, for each consumer $n$, his minimum demand $E^{\min}_n$ and budget $B_n$ are fixed for the entire horizon. These are necessary parameters that need to be known to solve for optimal demands and prices. We let the power availabilities ${\bf G}_1$ match the experimental hourly variation of the total demand. For the entire time-horizon, we have  $$\sum^{2000}_{n=1} E^{\min}_n=\sum^{24}_{t=1}G_1(t) \approx 54 \ \text{{\color{blue}MWh}}.$$ For homogenous consumers, it follows that $$E^{\min}_n= \frac{\sum^{24}_{t=1}G_1(t)}{2000} \approx 27 \ \text{{\color{blue}kWh}}.$$ Next, using Theorem 1, we plug-in $E^{\min}_n$ and the experimental hourly prices in (\ref{budget}) to find the minimum budget need, which is $B_n\approx 7.6 \ \text{ \text{DKK}}$, for each $n$. 
   
 \subsubsection{Numerical Results} Now, using the parameters found above, we can compute the optimal demands and prices for the Stackelberg game using (\ref{xx}) and (\ref{p}), and study their effects. 
 
In Figure \ref{EcoGridEU_Price_Power}, we plot the total power availabilities ${\bf G}_1$, the prices found experimentally and using the Stackelberg game, and the corresponding total payments by all consumers for their demands. Our approach leads to prices that have a slightly smaller mean than in the experiment and a significantly smaller variance, which is a desirable property \cite{IFAC}. {\color{black} At the equilibrium point, as stated in Remark 2, we observe that $$p^*_k(t)(G_k(t)+N)=p^*_k(t)\Bigg(\sum_{n\in\mathcal{N}}  d^*_{n,k}(t)+N\Bigg)$$  is a constant for each period $t$ and each company $k$. Hence, whenever company $k$ at time $t$ has a large amount of power available to sell $G_k(t)$, it would lower its price, and vice versa. Here, consumers are attracted to buy more whenever the price is low, and will buy less whenever the price is high, which is intuitive.} One advantage our approach has is that it results in billing savings for consumers, as we show in Figure \ref{EcoGridEU_Price_Power} (this demonstrates the importance of Theorem 1, which we use to find the minimum budget need for the consumers). {\color{blue} Here, the equilibrium demands are similar to the experimental values, but since the prices differ,  consumers receive the same amount of energy at smaller costs}. This would lead to more monetary incentives for active consumer participation in demand response management, while being consistent with the company's objectives, since the Stackelberg game prices found using (\ref{p}) are revenue-maximizing as shown in the proof of Theorem 2.
\begin{figure*}
\centering
\includegraphics[width=\linewidth, height=4in]{Influence_of_T.pdf}
\caption{The effects of varying the number of periods for companies (with different market shares and at Nash equilibrium of the power allocation game) and heterogeneous consumers (with different budgets) using the EcoGrid EU experimental data.}
\label{Influence_of_T}
\end{figure*}
\begin{figure*}
\centering
\includegraphics[width=\linewidth, height=2in]{algorithm11.pdf}
\caption{ Distributed algorithm's performance (Theorem 4 requires $\delta\geq0$) using the EcoGrid EU experimental data.}
\label{algorithm1}
\end{figure*}

Next, we make consumers heterogeneous and increase the number of companies. We differentiate between consumers by varying their budgets, and take 5 classes of consumers, as in the EcoGrid EU experiment. We let consumers' budgets be $B_{1-400}=4  \ \text{DKK}$, $B_{401-800}=5  \ \text{DKK}$, $B_{801-1200}=6 \  \text{DKK}$, $B_{1201-1600}=7  \ \text{DKK}$, and $B_{1601-2000}=8 \ \text{DKK}$. We also let the number of companies be $K=4$, which is consistent with the actual energy sources used in the experiment. Precisely, the system is powered by 61\% wind energy ($k=1$), 27\% biomass ($k=2$), 9\% solar energy ($k=3$), and 3\% biogas ($k=4$). We split the {\color{black}total need ($54 \ \text{MWh}$)} among the energy sources according to experimental proportions, assuming that each energy source is owned by a single company that acts as a company in our game. 
 
With the above setup, we study the effect of varying the number of periods $T$ from $1$ to $50$. To do this, we need to find a way for companies to allocate their total power across the time horizon for each fixed $T$, which can be done by using Theorem 3, which states that equally splitting the total power across the time horizon for each company $k$ constitutes a unique Nash equilibrium for the power allocation game (it is also shown to be the global maximizer in the proof). 
%Thus, at the Nash equilibrium, for each $k$ with total power $G^{total}_k$, in each period $t$, we have $G^*_k(t)=G^{total}_k/T$.
 \begin{figure*}
\centering
\includegraphics[width=\linewidth, height=2in]{Netherlands_Price_Power.pdf}
\caption{Average consumer demand (left), Stackelberg game and Dutch pilot prices (middle), and the cumulative payments and billing savings for average consumer (right).}
\label{Netherlands_Price_Power}
\end{figure*}
\begin{figure*}
\centering
\includegraphics[width=\linewidth, height=2in]{algorithm2.pdf}
\caption{Distributed algorithm's performance (Theorem 4 requires $\delta\geq0$) using the Dutch pilot data.}
\label{algorithm2}
\end{figure*}
Figure \ref{Influence_of_T} shows the influence of varying the number of periods on prices, power allocated, revenues, and consumer utilities. We observe that as $T$ increases, the power allocated at each period gets progressively smaller. On the other hand, prices can increase or decrease, depending on the company, and they converge to positive constants. Furthermore, revenues might also increase or decrease, depending on the company (note that the company that achieves the highest revenue is the one that offers the lowest prices, and vice-versa). In view of (\ref{eq1}), the sum of revenues at equilibrium is a constant that matches the sum of all consumer budgets. And hence, whenever the revenue increases (decreases) for a company $k$, at least one other company will incur a loss (gain) in terms of revenue. None of the companies can do better by altering its power availabilities across the time horizon, nor by changing its prices. This follows from the definition of Nash equilibrium. Furthermore, we note that the revenues are proportional to the total capacity, and the company with the highest (lowest) portion of the market is the one that incurs the largest increase (decrease) in revenue. 

Interestingly, in Figure \ref{Influence_of_T} we observe that as $T$ increases, the utilities for consumers also increase, and hence they will be more attracted to demand response programs, which is desirable \cite{DOECOM}. In comparison with the single-period setup \cite{sabita,sabita2}, this shows that the multi-period demand response provides improvements on the consumers' end. This increase, however, does not change significantly beyond a certain number of periods.
To demonstrate the performance of our algorithm, we take the case when $T=1$ and study the algorithm's performance for different values of $\delta$ in Figure \ref{algorithm1}. When $\delta=1000$, we observe that the algorithm converges very fast to the optimal prices and takes about less than $5$ iterations to reach equilibrium. The values are consistent with the values in Figure  \ref{Influence_of_T} when $T=1$, where we used the analytical expressions of the prices. Next, we increase $\delta$ to $10000$ and observe that the algorithm converges at a lower rate, but still fast. Thus, the rate of convergence is inversely proportional to the value of $\delta$. However, when $\delta$ decreases to a negative value, there are no guarantees on convergence. Theorem 4 only guarantees the convergence of the algorithm when $\delta\geq0$. We have verified that our distributed algorithm converges very fast for various values of $\delta$ and alternative values of $T$ and $K$, and the reader might experiment with varying them using our open-source code in \cite{tool2}.

\subsection{Dutch Smart Grid Pilot} {\color{black} To further validate our multi-period-multi-commpany framework, we use data from the Dutch Smart Grid Pilot \cite{dutch}, which was conducted in Zwolle, the Netherlands, for about one year (May, 2014 to May, 2015). Tariffs were announced to consumers a day ahead, and the average consumer behavior was reported}. For a group of $77$ homogeneous consumers, we study the average consumer's demand and payments using experimental prices and the prices derived using our method. Here, we take $K=1$, which is consistent with the Dutch pilot. Also, the experimental prices are in EUR/kWh. 
 
  \subsubsection{Finding the necessary parameters}{\color{black} We find the fixed parameters similarly to the EcoGrid EU experiment. For each consumer $n$, we have {\color{black}$E_n^{\min} \approx 8.8 \ \text{kWh}$.} Then, by (\ref{budget}), we find the minimum necessary daily budget, which is $B_n \approx 1.1\ \text{EUR}$ for each consumer}.


   \subsubsection{Numerical Results} Using the above parameters, we again use (\ref{xx}) and (\ref{p}) to find optimal demands and prices. In Figure \ref{Netherlands_Price_Power}, we plot the average consumer's hourly demand, the prices found experimentally and using the Stackelberg game, and the corresponding total payments by the average consumer. We again observe that our approach leads to smaller prices with a significantly smaller variance. For the average consumer, we observe that significant savings can be achieved using our approach (more than $30\%$). Next, we study the performance of our distributed algorithm in Figure \ref{algorithm2}. As in the case of the EcoGrid EU experimental data, our algorithm achieves fast convergence to optimal prices using only local information. 
   
 

\section{Generalizations} \label{generalizations}
%!TEX root = main.tex
 
 In the previous sections, we have analyzed our multi-period-multi-company framework under some assumptions to keep the analysis tractable and to reveal various insights on what happens at the equilibrium strategies. Due to the desirable mathematical properties of our framework, it is possible to extend our model at both the consumer-level and company-level. Here, we discuss some such possible extensions. 
 
\subsection{Consumer-Side}
In the utility function (\ref{consumer}), the parameters $\gamma_n$ and $\zeta_n$ for consumer $n$ are time and company independent. However, it is possible, to consider both time-specific and company-specific preferences $\gamma_{n,k,t}$ and $\zeta_{n,k,t}$, which allows consumers to have further flexibilities without violating existence and uniqueness of optimal strategies. In this case, the utility of consumer $n$ would be defined as 
\begin{equation}{u_n(\mathbf{d}_{n})=\sum_{k\in \scr{K}}\sum_{t\in \scr{T}}\gamma_{n,k,t}\ln(\zeta_{n,k,t}+d_{n,k}(t))}. \label{consumer2}\end{equation}
By an analogous analysis to the derivation of (\ref{xx}), it follows that optimal demands, under Assumption \ref{assumption1}, are given by
 \begin{align}
d^*_{n,k}(t)&= \frac{B_n+\sum_{j\in \scr{K}}\sum_{h\in \scr{T}}p_j(h)\zeta_{n,j,h}}{p_k(t)} \Gamma_{n,k,t} \nonumber\\
 &\qquad \qquad\qquad-\zeta_{n,k,t},  \; \forall \;t\in \scr{T}, \; k\in \scr{K},  \label{xx2}
\end{align}
where $$\Gamma_{n,k,t}=\frac{\gamma_{n,k,t}}{\sum_{j\in \scr{K}}\sum_{h\in \scr{T}}\gamma_{n,j,h}}.$$ 
We remark that $\sum_{k\in\mathcal{K}}\sum_{t\in\mathcal{T}}\Gamma_{n,k,t}=1$. Thus, if consumer $n$ prefers a higher demand from company $k$ at time $t$, choosing a higher weight $\gamma_{n,k,t}$ can achieve this. We also note that in (\ref{xx}), where consumer $n$ has identical time-specific/company-specific parameters,  $$\Gamma_{n,k,t}=\frac{1}{KT}.$$ Another alternative, is to expand the constraint set of the optimization problem for consumers to include additional time-specific or company-specific constraints. In general, companies, as leaders of the Stackelberg game, would need to anticipate how consumers would respond to their prices, and given that anticipation, they choose their prices accordingly. Furthermore,  if non-logarithmic utility functions are used by consumers, it might be more difficult to compute a Nash equilibrium for the price-selection game for companies, but the existence of a pure-strategy equilibrium is guaranteed as long as the function 
$$\sum_{t\in \scr{T}}p_k(t)\sum_{n\in \scr{N}}d_{n,k}({\bf{p}}_k,{\bf{p_{-k}}},t)$$
is concave in each $p_k(t)$ over a compact and convex set \cite{basar}, for each company. In case (\ref{consumer2}) is used, by (\ref{xx2}), this condition is satisfied.  

\subsection{Company-Side}

{\color{black} In the current formulation, consumers' demands are coupled through the companies' problems and the power availability constraint (\ref{prob2}). The upper-bound in (\ref{prob2}) is taken to be fixed in the Stackelberg game, but they can be strategically chosen by the power allocation game discussed in Section \ref{game2}. However, this game was solved under restrictive assumptions, such as the absence of network constraints, the full-controllability of generation sources, and the absence of ramping considerations. It is of interest to generalize the power allocation game to alleviate these limitations. Specifically, suppose that power availabilities ${\bf {G}} \in \mathcal{M} \subset \mathbb{R}^{KT}$, where $\mathcal{M}$ represents the transmission and distribution network constraints. One possibility is to assume that $\mathcal{M}$ is a system of linear equations  that approximate power flow equations \cite{eugene,eugene2,eugene3,baran,bolognani}. Furthermore, for simplicity, suppose that company $k$ has a ramping limit $l_{k,t}$ at period $t$. Also, to encode controllability, suppose that $$G^{\min}_{k,t} \leq G_k(t) \leq G^{\max}_{k,t},$$
where $G^{\min}_{k,t}$ ($G^{\max}_{k,t}$) is the minimum (maximum) possible generation  for company $k$ at period $t$.  Thus, company $k$ solves the following optimization problem:
 \begin{eqnarray}
\underset{\mathbf{G}_k }{\hbox{maximize}} && \pi_k({\bf{G}}_k,{\bf{G_{-k}}})
\nonumber \\
\hbox{subject to} && \sum_{t\in \scr{T}}G_k(t) \leq G^{{\rm total}}_k,  \nonumber\\
&& \vert G_k(t)-G_k(t-1)\vert \leq l_{k,t}, \forall t, t-1 \in\scr{T}, \nonumber \\
&& ({\bf{G}}_k,{\bf{G_{-k}}}) \in \mathcal{M}, \label{game3} \\
&& G^{\min}_{k,t} \leq G_k(t) \leq G^{\max}_{k,t},  \ \forall t\in\scr{T}, \nonumber\\
&& G_k(t)\geq0, \ \forall t\in\scr{T}. \nonumber\end{eqnarray}

We have the following result, whose proof is given in the Appendix. 
\begin{theorem}
If the power allocation game (\ref{game3}) is feasible, then, it admits a pure-strategy Nash equilibrium $({\bf{G}}^*_k,{\bf{G^*_{-k}}})$. Furthermore, if $({\bf{G}}^*_k,{\bf{G^*_{-k}}})$ is used for the Stackelberg equilibrium demands and prices given by Theorem \ref{mainTHM}, then, $$\sum_{n\in \scr{N}} d^*_{n,k}(t) = G^*_k(t),\;\; \forall \; t \in \scr{T},\;\; \forall \; k \in \scr{K}.$$ \label{ThmGame3}
\end{theorem}

The above theorem follows from the strict concavity of $\pi_k({\bf{G}}_k,{\bf{G_{-k}}})$ and the compactness and convexity of the constraint set, in addition to the results in Section \ref{game1}. Furthermore, it also demonstrates that it is possible to incentivize consumers to further shift their consumption in a way that is consistent with network considerations and company requirements. Finally, we remark that the control of consumers' demands here is indirect, that is, it is done via the unique equilibrium prices (\ref{p}), which are also affected by consumers' preferences and choices. Hence, at equilibrium, optimal supply provided by companies, ${\bf G^*}$, is equal to aggregate optimal demand, while taking into account consumer budgets and energy needs, in addition to network considerations and company-specific constraints and revenues. 
 
 



\section{Conclusion and Research Directions} \label{conclusion}
% \vspace{-0.5em}
\section{Conclusion}
% \vspace{-0.5em}
Recent advances in multimodal single-cell technology have enabled the simultaneous profiling of the transcriptome alongside other cellular modalities, leading to an increase in the availability of multimodal single-cell data. In this paper, we present \method{}, a multimodal transformer model for single-cell surface protein abundance from gene expression measurements. We combined the data with prior biological interaction knowledge from the STRING database into a richly connected heterogeneous graph and leveraged the transformer architectures to learn an accurate mapping between gene expression and surface protein abundance. Remarkably, \method{} achieves superior and more stable performance than other baselines on both 2021 and 2022 NeurIPS single-cell datasets.

\noindent\textbf{Future Work.}
% Our work is an extension of the model we implemented in the NeurIPS 2022 competition. 
Our framework of multimodal transformers with the cross-modality heterogeneous graph goes far beyond the specific downstream task of modality prediction, and there are lots of potentials to be further explored. Our graph contains three types of nodes. While the cell embeddings are used for predictions, the remaining protein embeddings and gene embeddings may be further interpreted for other tasks. The similarities between proteins may show data-specific protein-protein relationships, while the attention matrix of the gene transformer may help to identify marker genes of each cell type. Additionally, we may achieve gene interaction prediction using the attention mechanism.
% under adequate regulations. 
% We expect \method{} to be capable of much more than just modality prediction. Note that currently, we fuse information from different transformers with message-passing GNNs. 
To extend more on transformers, a potential next step is implementing cross-attention cross-modalities. Ideally, all three types of nodes, namely genes, proteins, and cells, would be jointly modeled using a large transformer that includes specific regulations for each modality. 

% insight of protein and gene embedding (diff task)

% all in one transformer

% \noindent\textbf{Limitations and future work}
% Despite the noticeable performance improvement by utilizing transformers with the cross-modality heterogeneous graph, there are still bottlenecks in the current settings. To begin with, we noticed that the performance variations of all methods are consistently higher in the ``CITE'' dataset compared to the ``GEX2ADT'' dataset. We hypothesized that the increased variability in ``CITE'' was due to both less number of training samples (43k vs. 66k cells) and a significantly more number of testing samples used (28k vs. 1k cells). One straightforward solution to alleviate the high variation issue is to include more training samples, which is not always possible given the training data availability. Nevertheless, publicly available single-cell datasets have been accumulated over the past decades and are still being collected on an ever-increasing scale. Taking advantage of these large-scale atlases is the key to a more stable and well-performing model, as some of the intra-cell variations could be common across different datasets. For example, reference-based methods are commonly used to identify the cell identity of a single cell, or cell-type compositions of a mixture of cells. (other examples for pretrained, e.g., scbert)


%\noindent\textbf{Future work.}
% Our work is an extension of the model we implemented in the NeurIPS 2022 competition. Now our framework of multimodal transformers with the cross-modality heterogeneous graph goes far beyond the specific downstream task of modality prediction, and there are lots of potentials to be further explored. Our graph contains three types of nodes. while the cell embeddings are used for predictions, the remaining protein embeddings and gene embeddings may be further interpreted for other tasks. The similarities between proteins may show data-specific protein-protein relationships, while the attention matrix of the gene transformer may help to identify marker genes of each cell type. Additionally, we may achieve gene interaction prediction using the attention mechanism under adequate regulations. We expect \method{} to be capable of much more than just modality prediction. Note that currently, we fuse information from different transformers with message-passing GNNs. To extend more on transformers, a potential next step is implementing cross-attention cross-modalities. Ideally, all three types of nodes, namely genes, proteins, and cells, would be jointly modeled using a large transformer that includes specific regulations for each modality. The self-attention within each modality would reconstruct the prior interaction network, while the cross-attention between modalities would be supervised by the data observations. Then, The attention matrix will provide insights into all the internal interactions and cross-relationships. With the linearized transformer, this idea would be both practical and versatile.

% \begin{acks}
% This research is supported by the National Science Foundation (NSF) and Johnson \& Johnson.
% \end{acks}

\section{Acknowledgments}
K. Alshehri thanks King Fahd University of Petroleum and Minerals (KFUPM) for the financial support. Research supported in part by the U.S. Air Force Office of Scientific Research (AFOSR) MURI Grant FA9550-10-1-0573, and in part by the AFOSR Grant FA9550-19-1-0353.

\section{Appendix}
\section{Proofs from \secref{sec:qkd}}
\label{app:proofs}

In \secref{sec:qkd} we show how to define the security of QKD in a
composable framework and relate this to the trace distance security
criterion introduced in \textcite{Ren05}. This composable treatment of
the security of QKD follows the literature \cite{BHLMO05,MR09}, and
the results presented in \secref{sec:qkd} may be found in
\textcite{BHLMO05,MR09} as well. The formulation of the statements
differs however from those works, since we use here the Abstract
Cryptography framework of \textcite{MR11}. So for completeness, we
provide here proofs of the main results from \secref{sec:qkd}.

\begin{proof}[Proof of \thmref{thm:qkd}]
  Recall that in \secref{sec:security.simulator} we fixed the
  simulator and show that to satisfy \eqnref{eq:qkd.security} it is
  sufficient for \eqnref{eq:qkd.security.2} to hold. Here, we will
  break \eqnref{eq:qkd.security.2} into security [\eqnref{eq:qkd.cor}]
  and correctness [\eqnref{eq:qkd.sec}], thus proving the theorem.

  Let us define $\gamma_{ABE}$ to be a state obtained from
  $\rho^{\top}_{ABE}$ [\eqnref{eq:qkd.security.tmp}] by throwing away
  the $B$ system and replacing it with a copy of $A$, i.e., \[
  \gamma_{ABE} = \frac{1}{1-p^\bot} \sum_{k_A,k_B \in \cK} p_{k_A,k_B}
  \proj{k_A,k_A} \otimes \rho^{k_A,k_B}_E.\] From the triangle
  inequality we get \begin{multline*} D(\rho^\top_{ABE},\tau_{AB} \otimes
  \rho^\top_{E}) \leq \\ D(\rho^\top_{ABE},\gamma_{ABE}) +
  D(\gamma_{ABE},\tau_{AB} \otimes \rho^\top_{E}) .\end{multline*}

Since in the states $\gamma_{ABE}$ and
$\tau_{AB} \otimes \rho^\top_{E}$ the $B$ system is a copy of the $A$
system, it does not modify the distance. Furthermore,
$\trace[B]{\gamma_{ABE}} =
\trace[B]{\rho^{\top}_{ABE}}$. Hence
\[D(\gamma_{ABE},\tau_{AB} \otimes \rho^\top_{E}) =
  D(\gamma_{AE},\tau_{A} \otimes \rho^\top_{E}) =
  D(\rho^\top_{AE},\tau_{A} \otimes \rho^\top_{E}).\]

For the other term note that
\begin{align*}
  & D(\rho^\top_{ABE},\gamma_{ABE}) \\
  & \qquad \leq \sum_{k_A,k_B} \frac{p_{k_A,k_B}}{1-p^{\bot}}
    D\left(\proj{k_A,k_B} \otimes \rho^{k_A,k_B}_E,\right. \\
  & \qquad \qquad \qquad \qquad \qquad \qquad \left.\proj{k_A,k_A} \otimes \rho^{k_A,k_B}_E \right)\\
  & \qquad = \sum_{k_A \neq k_B} \frac{p_{k_A,k_B}}{1-p^{\bot}} = \frac{1}{1-p^{\bot}}\Pr
  \left[ K_A \neq K_B \right].
\end{align*}
Putting the above together with \eqnref{eq:qkd.security.2}, we get
\begin{align*} & D(\rho_{ABE},\tilde{\rho}_{ABE}) \\
  & \qquad = (1-p^\bot)
  D(\rho^\top_{ABE},\tau_{AB} \otimes \rho^\bot_{E}) \\ & \qquad \leq \Pr
  \left[ K_A \neq K_B \right] + (1-p^\bot) D(\rho^\top_{AE},\tau_{A}
  \otimes \rho^\top_{E}). \qedhere \end{align*}
\end{proof}

\begin{proof}[Proof of \lemref{lem:robustness}]
  By construction, $\aK_\delta$ aborts with exactly the same
  probability as the real system. And because $\sigma^{\qkd}_E$
  simulates the real protocols, if we plug a converter $\pi_E$ in
  $\aK\sigma^{\qkd}_E$ which emulates the noisy channel $\aQ_q$ and
  blogs the output of the simulated authentic channel, then
  $\aK_\delta = \aK\sigma^{\qkd}_E\pi_E$. Also note that by
  construction we have
  $\aQ_q \| \aA' = \left(\aQ \| \aA\right) \pi_E$. Thus
  \begin{multline*} d\left( \pi_A^{\qkd}\pi_B^{\qkd}(\aQ_q \| \aA')
      ,\aK_\delta\right) \\ = d\left( \pi_A^{\qkd}\pi_B^{\qkd}\left(\aQ
        \| \aA\right) \pi_E , \aK\sigma^{\qkd}_E\pi_E\right). \end{multline*}

  Finally, because the converter $\pi_E$ on both the real and ideal
  systems can only decrease their distance (see
  \secref{sec:ac.systems}), the result follows.
\end{proof}


%%% Local Variables:
%%% TeX-master: "main.tex"
%%% End:


\bibliographystyle{IEEEtran}
\bibliography{references}

\end{document}
%% The Appendices part is started with the command \appendix;
%% appendix sections are then done as normal sections

%% \section{}
%% \label{}