%!TEX root = main.tex
 
 In the previous sections, we have analyzed our multi-period-multi-company framework under some assumptions to keep the analysis tractable and to reveal various insights on what happens at the equilibrium strategies. Due to the desirable mathematical properties of our framework, it is possible to extend our model at both the consumer-level and company-level. Here, we discuss some such possible extensions. 
 
\subsection{Consumer-Side}
In the utility function (\ref{consumer}), the parameters $\gamma_n$ and $\zeta_n$ for consumer $n$ are time and company independent. However, it is possible, to consider both time-specific and company-specific preferences $\gamma_{n,k,t}$ and $\zeta_{n,k,t}$, which allows consumers to have further flexibilities without violating existence and uniqueness of optimal strategies. In this case, the utility of consumer $n$ would be defined as 
\begin{equation}{u_n(\mathbf{d}_{n})=\sum_{k\in \scr{K}}\sum_{t\in \scr{T}}\gamma_{n,k,t}\ln(\zeta_{n,k,t}+d_{n,k}(t))}. \label{consumer2}\end{equation}
By an analogous analysis to the derivation of (\ref{xx}), it follows that optimal demands, under Assumption \ref{assumption1}, are given by
 \begin{align}
d^*_{n,k}(t)&= \frac{B_n+\sum_{j\in \scr{K}}\sum_{h\in \scr{T}}p_j(h)\zeta_{n,j,h}}{p_k(t)} \Gamma_{n,k,t} \nonumber\\
 &\qquad \qquad\qquad-\zeta_{n,k,t},  \; \forall \;t\in \scr{T}, \; k\in \scr{K},  \label{xx2}
\end{align}
where $$\Gamma_{n,k,t}=\frac{\gamma_{n,k,t}}{\sum_{j\in \scr{K}}\sum_{h\in \scr{T}}\gamma_{n,j,h}}.$$ 
We remark that $\sum_{k\in\mathcal{K}}\sum_{t\in\mathcal{T}}\Gamma_{n,k,t}=1$. Thus, if consumer $n$ prefers a higher demand from company $k$ at time $t$, choosing a higher weight $\gamma_{n,k,t}$ can achieve this. We also note that in (\ref{xx}), where consumer $n$ has identical time-specific/company-specific parameters,  $$\Gamma_{n,k,t}=\frac{1}{KT}.$$ Another alternative, is to expand the constraint set of the optimization problem for consumers to include additional time-specific or company-specific constraints. In general, companies, as leaders of the Stackelberg game, would need to anticipate how consumers would respond to their prices, and given that anticipation, they choose their prices accordingly. Furthermore,  if non-logarithmic utility functions are used by consumers, it might be more difficult to compute a Nash equilibrium for the price-selection game for companies, but the existence of a pure-strategy equilibrium is guaranteed as long as the function 
$$\sum_{t\in \scr{T}}p_k(t)\sum_{n\in \scr{N}}d_{n,k}({\bf{p}}_k,{\bf{p_{-k}}},t)$$
is concave in each $p_k(t)$ over a compact and convex set \cite{basar}, for each company. In case (\ref{consumer2}) is used, by (\ref{xx2}), this condition is satisfied.  

\subsection{Company-Side}

{\color{black} In the current formulation, consumers' demands are coupled through the companies' problems and the power availability constraint (\ref{prob2}). The upper-bound in (\ref{prob2}) is taken to be fixed in the Stackelberg game, but they can be strategically chosen by the power allocation game discussed in Section \ref{game2}. However, this game was solved under restrictive assumptions, such as the absence of network constraints, the full-controllability of generation sources, and the absence of ramping considerations. It is of interest to generalize the power allocation game to alleviate these limitations. Specifically, suppose that power availabilities ${\bf {G}} \in \mathcal{M} \subset \mathbb{R}^{KT}$, where $\mathcal{M}$ represents the transmission and distribution network constraints. One possibility is to assume that $\mathcal{M}$ is a system of linear equations  that approximate power flow equations \cite{eugene,eugene2,eugene3,baran,bolognani}. Furthermore, for simplicity, suppose that company $k$ has a ramping limit $l_{k,t}$ at period $t$. Also, to encode controllability, suppose that $$G^{\min}_{k,t} \leq G_k(t) \leq G^{\max}_{k,t},$$
where $G^{\min}_{k,t}$ ($G^{\max}_{k,t}$) is the minimum (maximum) possible generation  for company $k$ at period $t$.  Thus, company $k$ solves the following optimization problem:
 \begin{eqnarray}
\underset{\mathbf{G}_k }{\hbox{maximize}} && \pi_k({\bf{G}}_k,{\bf{G_{-k}}})
\nonumber \\
\hbox{subject to} && \sum_{t\in \scr{T}}G_k(t) \leq G^{{\rm total}}_k,  \nonumber\\
&& \vert G_k(t)-G_k(t-1)\vert \leq l_{k,t}, \forall t, t-1 \in\scr{T}, \nonumber \\
&& ({\bf{G}}_k,{\bf{G_{-k}}}) \in \mathcal{M}, \label{game3} \\
&& G^{\min}_{k,t} \leq G_k(t) \leq G^{\max}_{k,t},  \ \forall t\in\scr{T}, \nonumber\\
&& G_k(t)\geq0, \ \forall t\in\scr{T}. \nonumber\end{eqnarray}

We have the following result, whose proof is given in the Appendix. 
\begin{theorem}
If the power allocation game (\ref{game3}) is feasible, then, it admits a pure-strategy Nash equilibrium $({\bf{G}}^*_k,{\bf{G^*_{-k}}})$. Furthermore, if $({\bf{G}}^*_k,{\bf{G^*_{-k}}})$ is used for the Stackelberg equilibrium demands and prices given by Theorem \ref{mainTHM}, then, $$\sum_{n\in \scr{N}} d^*_{n,k}(t) = G^*_k(t),\;\; \forall \; t \in \scr{T},\;\; \forall \; k \in \scr{K}.$$ \label{ThmGame3}
\end{theorem}

The above theorem follows from the strict concavity of $\pi_k({\bf{G}}_k,{\bf{G_{-k}}})$ and the compactness and convexity of the constraint set, in addition to the results in Section \ref{game1}. Furthermore, it also demonstrates that it is possible to incentivize consumers to further shift their consumption in a way that is consistent with network considerations and company requirements. Finally, we remark that the control of consumers' demands here is indirect, that is, it is done via the unique equilibrium prices (\ref{p}), which are also affected by consumers' preferences and choices. Hence, at equilibrium, optimal supply provided by companies, ${\bf G^*}$, is equal to aggregate optimal demand, while taking into account consumer budgets and energy needs, in addition to network considerations and company-specific constraints and revenues. 
 
 
