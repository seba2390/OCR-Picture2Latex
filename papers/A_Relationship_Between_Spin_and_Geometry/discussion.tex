\documentclass{article}

\usepackage{amsmath}
\usepackage{csquotes}
\usepackage{tensor}
\usepackage{vector-spaces}
\usepackage{xcolor}

\begin{document}

\section{Discussion}\label{sec:discussion}

Interpreting the meaning of the spin-$s$ weak Clifford algebras depends heavily on our interpretation of the Euclidean three-space $\threespace$. The simplest and most relevant view of $\threespace$, is as the non-relativistic configuration space for a point-like particle. We may then interpret each vector as a position in three-space, or as the underlying algebraic object for a position operator in a quantum mechanical system. In this setting, we see that each spin-$s$ weak Clifford algebra describes a different algebra of position variables according to its spin structure: the spin-$0$ weak Clifford algebra is commutative and corresponds to the usual position operator algebra in quantum mechanics; and the higher spin algebras are all non-commutative.

In the sense of non-commuting position operators, we see a correspondence between the spin-$s$ weak Clifford algebras, and non-commutative geometries whose structures are determined by their spin. Though the meaning of the spin dependence of area and volume is unclear, especially when identically zero, these phenomena further indicate that the spin structure affects the geometry of, or perhaps experienced by, the system. These non-commutative geometries are, in general, much weaker than those common to the literature\cite{szabo,aschieri,frob}, which typically place the position operators into a Heisenberg-like\cite{schempp,wallach} algebra.

From these observations, we see the $\spinscliffordalgebra{s}$ as a new way to incorporate spin into quantum mechanical theories: directly as certain non-commutative algebras of position (and perhaps, by symmetry, momentum) operators. It is viable to extend the $\spinscliffordalgebra{s}$ to such a more phenomenologically complete model, since they contain the totally symmetric tensors, which are essential to algebraically perform dynamical (symplectic) transformations\cite{fulton-harris,crumeyrolle}. Aside from these considerations, $\spinscliffordalgebra{1/2}$ and $\spinscliffordalgebra{1}$ are also weak enough that the Euclidean Clifford and Duffin-Kemmer-Petiau\cite{fischbach,helmstetter,micali} algebras respectively may be derived from them. Thus, the $\spinscliffordalgebra{s}$ may form the basis for a generalised spin-$s$ theory of such algebras. Furthermore, relativistic versions of this formalism may prove useful in the construction of theories of quantum gravity which incorporate both non-commutative geometry and spin. 

With our interpretation of the $\spinscliffordalgebra{s}$ laid out, it is instructive to compare it against other arbitrary spin models. The most relevant such comparison is with the standard tensor product of center of mass and \enquote{internal} spin degrees of freedom\cite{weyl}. An immediate similarity is that both models include the spin algebra $\spinalgebra{s}$ as a subalgebra, originating from their spin structures. An immediate difference is that the traditional model incorporates the Heisenberg algebra\cite{schempp,wallach}, and therefore the notion of momentum, whereas the spin-$s$ weak Clifford algebras do not. However, the most significant difference is that in the traditional model the position and spin degrees of freedom are commuting, and thus independent of each other; they are not in $\spinscliffordalgebra{s}$ by construction. This implies that there is phenomenology between position and spin in a dynamical model containing $\spinscliffordalgebra{s}$ which the tensor product model cannot describe. The tensor product model should however be recoverable within this richer formalism as an approximation in some suitable setting.

Another standard approach to higher spin in non-relativistic physics is to consider a subspace of $\tensorpower{\threecliffordalgebra}{k}$ with the appropriate spin structure\cite{sommen}. However, due to the strength of the algebra, it lacks totally symmetric tensors, and so cannot easily form part of a model which algebraically encodes symplectic transformations. Such algebras also have interpretational issues regarding the underlying substructure of $\tensorpower{\threecliffordalgebra}{k}$ when applying it to fundamental particles; $\spinscliffordalgebra{s}$ does not suffer from this. Beyond the realm of non-relativistic physics are the Bargmann-Wigner\cite{bargmann} and Joos-Weinberg\cite{jefferey,weinberg-spin} equations. Since the former equations do not have definite spin in general\cite{jaroszewicz}, we shall focus on the latter. The $\gamma^{\mu_{1}\dots\mu_{2s}}$ for a particle of spin-$s$ in the Joos-Weinberg equation are comprised of objects which bear close, but not exact, resemblance to the multipole tensors of order $s$\cite{bradshaw}, revealing a link to the spin structure of the theory. Much like the tensor product model however, the spin sectors of the Joos-Weinberg equations and their center of mass sectors commute, as do the position operators. In this way the comparisons made between $\spinscliffordalgebra{s}$ and the tensor product model are valid for the Joos-Weinberg equations also.

\end{document}