\documentclass{article}

\usepackage{amsmath}
\usepackage{mathtools}
\usepackage{csquotes}
\usepackage{hyperref}
\usepackage{maths-common}
\usepackage{lie-algebras}
\usepackage{groups}
\usepackage{so3}
\usepackage{vector-spaces}
\usepackage{operators}
\usepackage{multipoles}

\usepackage{xcolor}

\begin{document}

\section{Introduction}

\subsection{The Spin Algebras \texorpdfstring{$\spinalgebra{s}$}{A(s)}}\label{sec:spin-algebras}

A recent paper\cite{bradshaw} derived real associative algebras that completely describe the spin structure for non-relativistic systems of arbitrary spin. These \enquote{spin algebras} $\spinalgebra{s}$, are derived from the universal enveloping algebra\cite{humphreys} $\usothree$ of the Lie algebra $\sothree$,
\begin{subequations}
\begin{gather}
    \definition{\sothree}{\textup{span}_{\reals}(\set{\generator[1],\generator[2],\generator[3]})}\\
    \sothreelieproductdefinition{a}{b}{c}\label{eqn:so3-lie-product},
\end{gather}
\end{subequations}
\noindent by quotient,
\begin{equation}\label{eqn:spin-algebra-definition}
    \definition{\spinalgebra{s}}{\frac{\usothree}{\ideal{\image{\multipole{2s+1}}}}},
\end{equation}
\noindent where $\ideal{\image{\multipole{k}}}$ is the two-sided ideal generated by the totally-symmetric and contractionless \enquote{multipoles},
\begin{subequations}
\begin{gather}
    \mapdeclaration{\multipole{k}}{\tensorpower{\tsothree}{k}}{\usothree}\\
    \composition{\sothreead[\sothreecasimirelement+k(k+1)]}{\multipole{k}}=0\label{eqn:multipole-eigenobject}\\
    \forall\tau\in S_k,\; \composition{\multipole{k}}{\tau}=\multipole{k}\\
    \forall m\neq n\in \set{1,...,k},\;\sum_{a_m,a_n=1}^{3}\delta_{a_m a_n}\multipole{k}\Big(\bigotimes_{j=1}^{k}\generator[a_j]\Big)=0,
\end{gather}
\end{subequations}
\noindent with $\tsothree$ the tensor algebra of $\sothree$\cite{bourbaki}.

The multipoles are defined recursively in terms of the adjoint action $\sothreead$,
\begin{equation}\label{eqn:adjoint-action}
    \definition{\sothreead[u]}{\mapdefinition{v}{
        \begin{cases}
            uv & u\in\reals\\
            \tensor{u;v}-\tensor{v;u} & u\in\sothree\\
            \sothreead[a]\!\circ\!\sothreead[b](v) & u=\tensor{a;b},
        \end{cases}
    }}
\end{equation}
\noindent the left multiplication $\forall A\in\usothree$,
\begin{equation}\label{eqn:left-multiplication}
    \definition{\sothreeleft[A]}{\mapdefinition{B}{\tensor{A;B}}},
\end{equation}
\noindent and the Casimir element of $\usothree$,
\begin{equation}
    \sothreecasimirelementdefinition{a},
\end{equation}
\noindent as $\forall k\in\integers^{+}$, $\alpha\in\reals$, $v\in\sothree$, $B_{k}\in\tensorpower{\sothree}{k}$,
\begin{equation}
\begin{gathered}\label{eqn:multipole-chain-identity}
    \multipole{0}(\alpha)=\alpha\\
    \multipole{1}(v)=v\\
    \multipole{k+1}(\tensor{v;B_{k}})=\composition{\frac{\composition{\sothreead[\sothreecasimirelement+k(k-1)]}{\sothreead[\sothreecasimirelement+k(k+1)]}}{4(k+1)(2k+1)}}{\sothreeleft[v]}{\multipole{k}(B_{k})}.
\end{gathered}
\end{equation}
\noindent The multipoles are important to the structure of $\usothree$, since $\forall A_{k}\in\usothree$, $\forall k\in\naturals$, for which $\sothreead[\sothreecasimirelement+k(k+1)](A_{k})=0$ may be written as an $\reals[\sothreecasimirelement]$-linear combination of objects from $\image{\multipole{k}}$, and all elements of $\usothree$ are linear combinations of such $A_{k}$. For compactness, let us define $\forall k\in\integers^{+}$,
\begin{equation}
\begin{gathered}
    \definition{\multipoletensor{}}{\multipole{0}(1)}\\
    \definition{\multipoletensor{a_{1}a_{2}...a_{k}}}{\multipole{k}(\tensor{\generator[a_{1}];\generator[a_{2}];...;\generator[a_{k}]})}.
\end{gathered}
\end{equation}
\noindent The spin algebras $\spinalgebra{s}$ are real unital associative algebra of multipoles, $\forall k\in\integers^{+}$,
\begin{equation}
\begin{gathered}
    \spinalgebra{0}\cong\textup{span}_{\reals}(\set{\multipoletensor{}})\\
    \spinalgebra{k}\cong\textup{span}_{\reals}(\set{\multipoletensor{},\multipoletensor{a_{1}},...,\multipoletensor{a_{1}...a_{2k}}}),
\end{gathered}
\end{equation}
\noindent within which,
\begin{equation}
    \sothreecasimirelement=-s(s+1).
\end{equation}

Since all multipoles $\multipoletensor{a_{1}...a_{2k}}$ are algebraic combinations of the $\set{\generator[a]}$, the $\spinalgebra{s}$ encode the spin structures for arbitrary spins $s$ entirely in terms of $\sothree$. The Lie algebra $\sothree$ generates the Lie group $\specialorthogonalgroup{3}$, which is the connected symmetry group of Euclidean three-space. In this way, the $\spinalgebra{s}$ are connected to the geometry of Euclidean three-space, however the extent and consequences of this connection is unclear.

\subsection{A Geometric Realisation of \texorpdfstring{$\sothree$}{so(3)} through Clifford Algebra}\label{sec:clifford-introduction}

To understand the extent of the relationship between the $\spinalgebra{s}$ and the geometry of Euclidean three-space, let us first attempt to understand the underlying geometric content of $\sothree$, with which any geometric account of $\spinalgebra{s}$ must be compatible. Towards this, we will explore the geometric structure of the more general $\sopq$, the Lie algebra of the connected symmetry group $\textup{SO}^{+}(p,q,\reals)$, which preserves the geometry of a $(p{+}q)$-dimensional space with indefinite signature.

Let $\genericmetricspace$ denote such a non-trivial finite-dimensional vector space $\genericspace$ over $\reals$ equipped with a symmetric, non-degenerate, bilinear map $\genericmetricdeclaration$, which we shall follow relativity by referring to as a \enquote{metric}. We may identify the Lie algebra $\sopq$ with the set of all linear maps $A\in\genericspaceendomorphisms$ satisfying, $\forall v,w\in\genericspace$,
\begin{equation}
    \genericmetric[A(v)][B]+\genericmetric[v][A(w)]=0.
\end{equation}
\noindent Such maps are closed under commutators, which serves as the Lie product. It has long been known that $\sopq$ is in bijection\cite{fulton-harris} with $\genericantisymmetrictensors{2}\subset\generictensoralgebra$, the space of second-order antisymmetric tensors on $\genericspace$\cite{bourbaki},
\begin{equation}\label{eqn:antisymmetric-tensors}
    \definition{\genericantisymmetrictensors{2}}{\textup{span}_{\reals}(\set{\tensorwedge{a}{b}\,\vert\,a,b\in\genericspace})},
\end{equation}
\noindent where $\wedge$ is the multilinear, totally antisymmetric, associative \enquote{wedge product}, $\forall k\in\integers^{+}$,$\forall v_{j}\in\genericspace : j\in\set{1,\dots,k}$,
\begin{equation}\label{eqn:n-blades}
    \tensorwedge{v_{1}}{v_{2}}{\dots}{v_{k}}=\frac{1}{k!}\smashoperator{\sum_{\sigma\in S_{k}}}\textup{sgn}(\sigma)\bigotimes_{j=1}^{k}v_{\sigma(j)},
\end{equation}
\noindent with $S_{k}$ is the set of all permutations of $k$ objects, and $\textup{sgn}(\sigma)$ is the sign of the permutation $\sigma$. Explicitly, this bijection may be given, up to a scalar, as $\forall v,w,x\in\genericspace$,
\begin{equation}\label{eqn:sothree-action-generic}
    \mapdefinition{\tensorwedge{v}{w}}{\big(\mapdefinition{x}{\genericmetric[v][x]w-\genericmetric[w][x]v}\big)}.
\end{equation}

This bijection grants us an immediate geometric interpretation for the objects of $\sopq$: linear combinations of planar elements. More generally, the \enquote{$k$-blade}\cite{doran-lasenby} \eqref{eqn:n-blades} can be interpreted as a hypervolume element of dimension $k$. For $\sothree$, a $2$-blade encodes both the plane and angle of the rotation it generates. We refer to an arbitrary element of $\genericantisymmetrictensors{k}$ as a \enquote{$k$-vector}\cite{doran-lasenby} or \enquote{\textit{prefix-}vector} e.g. $2$-vector and bivector are identical. For completeness, we consider $0$-vectors and $0$-blades to be the scalars of $\genericspace$. 

With the objects of $\sopq$ algebraically identified as bivectors $\genericantisymmetrictensors{2}$, we may find their Lie product by constructing the Clifford algebra\cite{crumeyrolle} $\genericcliffordalgebra$:
\begin{equation}\label{eqn:clifford-quotient}
    \genericcliffordalgebra\cong\frac{\generictensoralgebra}{\ideal{\tensor{v;w}+\tensor{w;v}-2\genericmetric[v][w]}}.
\end{equation}
\noindent This quotient reduces all tensors of $\generictensoralgebra$ to linear combinations of $k$-blades. The survival of the $k$-blades in $\generictensoralgebra$ mark them as objects of geometric significance. $\genericcliffordalgebra$ is finite-dimensional, as all $k$-blades with $k>\dim(\genericspace)$ are $0$ by antisymmetry. Since the field of scalars of $\genericspace$ is not of characteristic $2$, this is equivalent to the construction of $\genericcliffordalgebra$ using a quadratic form\cite{crumeyrolle}.

The structure of the Clifford algebra reveals the Lie product between bivectors,
\begin{equation}\label{eqn:clifford-lie-bracket}
\begin{aligned}
    \tensor{(\tensorwedge{a}{b});(\tensorwedge{c}{d})}-\tensor{(\tensorwedge{c}{d});(\tensorwedge{a}{b})}=2\genericmetric[b][c](\tensorwedge{a}{d})-2\genericmetric[b][d](\tensorwedge{a}{c})-2\genericmetric[a][c](\tensorwedge{b}{d})+2\genericmetric[a][d](\tensorwedge{b}{c}),
\end{aligned}
\end{equation}
\noindent turning $\genericantisymmetrictensors{2}$ into a Lie algebra. This Lie product is related to the usual one\cite{weinberg-qft} by a scaling. The Clifford algebra also naturally defines an $\sopq$-action on vectors $\forall a,b,c\in\genericspace$,
\begin{equation}\label{eqn:clifford-bivector-action}
    m(\tensorwedge{a}{b})(c)=\half\big(\tensor{c;(\tensorwedge{a}{b})}-\tensor{(\tensorwedge{a}{b});c}\big)=\genericmetric[a][c]b-\genericmetric[a][b]c,
\end{equation}
\noindent which is identical to \eqref{eqn:sothree-action-generic}. This enables a natural action of the symmetry group $\textup{SO}^{+}(p,q,\reals)$ to be defined algebraically on $\genericcliffordalgebra$\cite{doran-lasenby}.

Restricting our attention to three-dimensional Euclidean space $\threemetricspace$, we may introduce the transformation,
\begin{equation}\label{eqn:bivector-generator-transform-clifford}
    \definition{\generator[p]'}{-\frac{1}{4}\sothreesum*[r]{a,b}\sothreestructureconstants{a}{b}{p}\,\tensorwedge{e_{a}}{e_{b}}},
\end{equation}
\noindent where $\set{e_{1},e_{2},e_{3}}$ are a basis for $\threespace$ satisfying $\threemetric[e_{a}][e_{b}]=\delta_{ab}$. Then, on basis bivectors the Lie product \eqref{eqn:clifford-lie-bracket} becomes,
\begin{equation}
    \commutator{\generator[p]'}{\generator[q]'}=\sothreesum*[r]{r}\sothreestructureconstants{p}{q}{r}\generator[r]',
\end{equation}
\noindent consistent with \eqref{eqn:so3-lie-product}. Thus, we see that $\threecliffordalgebra$ algebraically realises $\sothree$ in a geometrically meaningful way using bivectors.

\subsection{Limitations of the Clifford Algebra Approach}\label{sec:clifford-limitations}

Despite this natural emergence of $\sothree$ within $\threecliffordalgebra$, this realisation is severely limited. To see this, we note that in $\threecliffordalgebra$, we have $\forall a,b,c,d\in\threespace$,
\begin{equation}
    \half\big(\tensor{(\tensorwedge{a}{b});(\tensorwedge{c}{d})}+\tensor{(\tensorwedge{c}{d});(\tensorwedge{a}{b})}\big)=\genericmetric[a][d]\genericmetric[b][c]-\genericmetric[a][c]\genericmetric[b][d].
\end{equation}
\noindent Applying \eqref{eqn:bivector-generator-transform-clifford}, we find in $\threecliffordalgebra$,
\begin{equation}\label{eqn:clifford-symmetric-bivectors}
    \half(\tensor{\generator[p]';\generator[q]'}+\tensor{\generator[q]';\generator[p]'}) = -\frac{1}{4}\delta_{pq},
\end{equation}
\noindent and the Casimir element $\sothreecasimirelement$ of $\sothree$ is,
\begin{equation}\label{eqn:clifford-casimir}
    \definition{\generator'^2}{\sum_{p=1}^{3}\tensor{\generator[p]';\generator[p]'}}=-\frac{3}{4}.
\end{equation}
\noindent Together, \eqref{eqn:clifford-symmetric-bivectors} and \eqref{eqn:clifford-casimir} imply that the spin quadrupole $\multipoletensor{pq}'=0$ in $\threecliffordalgebra$. By the multipole recurrence relationship \eqref{eqn:multipole-chain-identity}, we also conclude that all spin multipoles $\multipoletensor{p_{1}\dots,p_{k}}=0$  for $k>2$. This shows that, unsurprisingly\cite{doran-lasenby}, the unital subalgebra $\set{\reals,\threeantisymmetrictensors{2}}\subset\threecliffordalgebra$ has spin-$\half$ structure, and is algebra isomorphic to $\spinalgebra{\half}$. This is a direct result of the defining algebraic structure \eqref{eqn:clifford-quotient} of $\threecliffordalgebra$. Thus, $\threecliffordalgebra$ cannot support an arbitrary spin structure within it, and cannot be used to explore the geometric content of $\spinalgebra{s}$ for $s\neq\half$.

Finding an algebra which can will be the focus of this paper. In section \ref{sec:method}, we will define a \enquote{Spinless Weak Clifford Algebra} compatible with the structure of an arbitrary $\spinalgebra{s}$. We will present the \enquote{Spin-$s$ Weak Clifford Algebras} derived from these in section \ref{sec:results}, and show they may naturally entail spin-dependence in the measured sizes of hypervolumes. Finally, in section \ref{sec:discussion}, we will discuss the connection between the spin-$s$ Clifford algebras and non-commutative geometries, and contrast these new algebras with other higher-spin models. We will also consider the implications of these algebras for quantum mechanics.

\end{document}