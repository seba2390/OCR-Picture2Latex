\documentclass{article}

\usepackage[subpreambles=true]{standalone}
\usepackage{amsmath}
\usepackage{csquotes}
\usepackage{hyperref}
\usepackage{maths-common}
\usepackage{map}
\usepackage{tensor}
\usepackage{so3}
\usepackage{groups}
\usepackage{vector-spaces}
\usepackage{multipoles}
\usepackage{operators}
\usepackage{terminology}

\usepackage{xcolor}

\begin{document}

\section{Method}\label{sec:method}

\subsection{Towards a weaker Clifford Algebra}

As we saw in section \ref{sec:clifford-limitations}, the incompatibility of the Clifford algebra with arbitrary spin algebras $\spinalgebra{s}$ is inherent to its algebraic structure. This is unfortunate, since this algebraic structure enabled us to find both a natural Lie algebra action \eqref{eqn:clifford-bivector-action} and the geometric structure of $\sothree$ \eqref{eqn:clifford-lie-bracket}. To study the $\spinalgebra{s}$, we will construct a weaker algebra with both of these features by elementary means. Such an algebra will have no spin structure at all, enabling any $\spinalgebra{s}$ to be embedded within. We expect the Lie product of $\sopq$ to follow from the $\sopq$-action on our algebra, as \eqref{eqn:clifford-lie-bracket} follows from \eqref{eqn:clifford-bivector-action}. Thus, to proceed we require only: an elementary derivation of an $\sopq$-action on vectors; and a determination of how this action must be implemented within an associative algebra.

\subsubsection{Elementary Derivation of the \texorpdfstring{$\sopq$}{so(p,q)}-action on Vectors}\label{sec:derivation-lie-action}

Recall that $\genericspace$ is a finite-dimensional real vector space and that $\genericmetric$ is a non-degenerate, symmetric, bilinear map on $\genericspace$. For any non-null element $a\in\genericspace$, i.e. $\genericmetric[a][a]\neq0$, we may always find a unique direct sum decomposition of $\genericspace$,
\begin{equation}\label{eqn:orthogonal-decomposition}
    \genericspace\cong\textup{span}_{\reals}(\set{a})\oplus W_{a},
\end{equation}
\noindent where $\forall w\in W_{a},\,\genericmetric[a][w]=0$. Specifically, we can write $\forall v\in\genericspace$ uniquely as,
\begin{equation}\label{eqn:orthogonal-decomposition-vector}
    v=\frac{\genericmetric[a][v]}{\genericmetric[a][a]}a+b
\end{equation}
\noindent where $\genericmetric[a][b]=0$. We notice from \eqref{eqn:orthogonal-decomposition-vector} that the decomposition \eqref{eqn:orthogonal-decomposition} is scale-invariant in two ways: $\forall\alpha\in\reals/\set{0}$, $a$ and $a'=\alpha a$ give the same decomposition, as do $\genericmetric$ and $b=\alpha\genericmetric$. This suggests that some of the structure imparted on $\genericspace$ by $\genericmetric$ is independent of scale. Thus, let us explore this structure more easily by accepting scaling of the metric $\genericmetric$ in our arguments. Doing so will enable us to establish results using more mathematically convenient objects.

From \eqref{eqn:orthogonal-decomposition}, we may use $\genericmetric$ to alter the scale of the component of $v\in\genericspace$ parallel to a non-null vector $a\in\genericspace$, $\forall k\in\reals$,
\begin{equation}\label{eqn:component-scale-map}
    \definition{S(k,a)}{\mapdefinition{v}{\genericmetric[a][a]v+(k-1)\genericmetric[a][v]a}},
\end{equation} 
\noindent accepting the overall scaling by $\genericmetric[a][a]$ that occurs. Note that $\forall v,w\in\genericspace$,
\begin{equation}
    \genericmetric[S(k,a)(v)][\,S(k,a)(w)]=\genericmetric^{2}(a,a)\genericmetric[v][w],
\end{equation} 
\noindent precisely when $k_{\pm}=\pm1$. Recognising the $k_{+}$ case as simply an overall scaling, we use the $k_{-}$ solution to define the \enquote{conformal reflection},
\begin{equation}
    \definition{R(a)}{S(-1,a)=\genericmetric[a][a]v-2\genericmetric[a][v]a}.
\end{equation}
\noindent The conformal reflections are a superset of the traditional reflections, but are similarly not closed under composition.

To see this, let us first define a \enquote{\gadjoint} of an endomorphism $A\in\genericspaceendomorphisms$ as an endomorphism $B\in\genericspaceendomorphisms$ such that $\forall v,w\in\genericspace$, 
\begin{equation}
    \genericmetric[A(v)][w]=\genericmetric[v][B(w)].
\end{equation}
\noindent Since $\genericmetric$ is symmetric and non-degenerate, $B$ is unique and its \gadjoint\space is $A$; accordingly, we shall denote the \gadjoint\space of $A$ as $\bar{A}$ with $\bar{\bar{A}}=A$. We also define \enquote{self-\gadjoint} to mean $\bar{A}=A$, \enquote{anti-self-\gadjoint} to mean $\bar{A}=-A$, and two maps $a_{+}$ and $a_{-}$ that respectively yield the self-\gadjoint\space and anti-self-\gadjoint\space parts of an endomorphism $A$,
\begin{equation}
    \definition{a_{\pm}}{\mapdefinition{A}{\half(A\pm\bar{A})}}.
\end{equation}
\noindent
We find all conformal reflections are self-\gadjoint, but the \gadjoint\space of $\composition{R(a)}{R(b)}$ is $\composition{R(b)}{R(a)}$ which is different in general. This indicates that there is structure in the product of two conformal reflections that is not itself a conformal reflection.

To identify this additional structure, we decompose $\composition{R(a)}{R(b)}(v)$, $\forall v\in\genericspace$ into self-\gadjoint\space and anti-self-\gadjoint\space parts $\forall v\in\genericspace$,
\begin{subequations}
\begin{gather}
    a_{-}\big(\composition{R(a)}{R(b)}\big)(v)=-2\genericmetric[a][b]\big(\genericmetric[a][v]b-\genericmetric[b][v]a\big)\label{eqn:conformal-reflection-commutator-proto}\\
    a_{+}\big(\composition{R(a)}{R(b)}\big)(v)=\genericmetric[a][a]\genericmetric[b][b]v+2\genericmetric[a][v]\big(\genericmetric[a][b]b-\genericmetric[b][b]a\big)-2\genericmetric[b][v]\big(\genericmetric[a][a]b-\genericmetric[b][a]a\big).\label{eqn:conformal-reflection-anticommutator-proto}
\end{gather}
\end{subequations}
\noindent Noticing that there is a repeated pattern within \eqref{eqn:conformal-reflection-commutator-proto} and \eqref{eqn:conformal-reflection-anticommutator-proto}, we define $\forall a,b\in\genericspace$,
\begin{equation}
    \definition{\twovectorsaction[a][b]}{\mapdefinition{v}{\genericmetric[a][v]b-\genericmetric[b][v]a}},
\end{equation}
\noindent enabling us to write,
\begin{subequations}
\begin{gather}
    a_{-}\big(\composition{R(a)}{R(b)}\big)=-2\genericmetric[a][b]\twovectorsaction[a][b]\label{eqn:conformal-reflection-commutator}\\
    a_{+}\big(\composition{R(a)}{R(b)}\big)=\genericmetric[a][a]\genericmetric[b][b]\textup{id}+2\composition{\twovectorsaction[a][b]}{\twovectorsaction[a][b]}.\label{eqn:conformal-reflection-anticommutator}
\end{gather}
\end{subequations}
\noindent Since $a_{+}(A)+a_{-}(A)=A$, we find that the map $\twovectorsaction[a][b]$ controls binary products of conformal reflections. It is also antisymmetric, $\twovectorsaction[b][a]=-\twovectorsaction[a][b]$, and anti-self-\gadjoint. Furthermore, given \eqref{eqn:conformal-reflection-commutator}, and $\forall v\in\genericspace$,
\begin{subequations}
\begin{gather}
    a_{-}\big(\composition{R(a)}{\twovectorsaction[b][c]}\big)=\half\big(\twovectorsaction[R(a)(b)][c]+\twovectorsaction[b][R(a)(c)]\big)\\
    \genericmetric[b][c]\,a_{+}\big(\composition{R(a)}{\twovectorsaction[b][c]}\big)=\half\big(\genericmetric^{2}(a{,}c)R(b)-\genericmetric^{2}(a{,}b)R(c)-R(\twovectorsaction[a][b](c))+R(\twovectorsaction[a][c](b))\big),
\end{gather}
\end{subequations}
\noindent we see the $\set{R(a),t(b,c)}$ forms a generating set for the algebra of conformal reflections.

Therefore, we have derived a map with central importance to the conformal reflections, and whose image agrees with the image of \eqref{eqn:sothree-action-generic}. So far our derivation has only accounted for non-null vectors $a,b\in\genericspace$ with $\genericmetric[a][b]\neq0$, so,
\begin{equation}
    \twovectorsaction[a][b]=\frac{\composition{R(a)}{R(b)}-\composition{R(b)}{R(a)}}{-4\genericmetric[a][b]}.
\end{equation}
\noindent Let us extend this definition to the whole of $\genericspace$. Since $\twovectorsaction[a][b+\epsilon a]=\twovectorsaction[a][b]$, $\forall\epsilon\in\reals$, $\twovectorsaction[a][b]$ for non-null vectors is defined without the need for limits. To define $\twovectorsaction[a][b]$ when $b$ is null and $\genericmetric[a][b]=0$, by the non-degeneracy of $\genericmetric$ we may find a null $c\in\genericspace$ such that $\genericmetric[b][c]\neq0$, and use this pair to construct a pair of vectors $\set{p,n}$ such that,
\begin{subequations}
\begin{gather}
    \genericmetric[p][p]=-\genericmetric[n][n]>0\\
    \genericmetric[p][n]=0\\
    b=p+n\\
    c=p-n.
\end{gather}
\end{subequations}
\noindent Thus, we may define,
\begin{equation}
    t(a,b)=t(a,p)+t(a,n),
\end{equation}
\noindent by the bilinearity of $\twovectorsaction$. Similar arguments yield $\twovectorsaction[a][b]$ for $a$ and $b$ both null, regardless of the value of $\genericmetric[a][b]$. Thus, we have defined $\twovectorsaction[a][b]$ $\forall a,b\in\genericspace$ and completed our derivation of the $\sopq$-action on $\genericspace$; this was achieved in a coordinate-free, elementary way, without appealing to differential geometry or the theory of Lie groups\cite{varadarajan}.

\subsubsection{The Algebraic Form of \texorpdfstring{$\twovectorsaction[a][b]$}{t(a,b)}}

Having acquired $\twovectorsaction[a][b]$, the $\sopq$-action on $\genericspace$, $\forall a,b\in\genericspace$, we must consider how to implement it algebraically within an associative algebra. More precisely, we seek a third-order tensor $f(a,b,c)$ whose properties match those of $\twovectorsaction[a][b](c)$, so that a quotient of $\generictensoralgebra$ by the two-sided ideal $\forall a,b,c\in\genericspace$,
\begin{equation}
    \ideal{f(a,b,c)-\twovectorsaction[a][b](c)}
\end{equation}
\noindent yields the most general non-trivial algebra possible.

We first note that $\twovectorsaction[a][b](c)$ is antisymmetric in its first two arguments. The most general third-order tensor sharing this property is,
\begin{equation}\label{eqn:f-tensor-definition}
    f(a,b,c)=k_{1}\big(\tensor{(\tensor{a;b}-\tensor{b;a});c}\big)+k_{2}\big(\tensor{c;(\tensor{a;b}-\tensor{b;a})}\big)+k_{3}\big(\tensor{a;c;b}-\tensor{b;c;a}\big).
\end{equation}
\noindent We may constrain \eqref{eqn:f-tensor-definition} even further by considering  the commutator between two $\twovectorsaction$ maps, which is closed,
\begin{equation}\label{eqn:action-commutator}
    \composition{\twovectorsaction[a][b]}{\twovectorsaction[c][d]}-\composition{\twovectorsaction[c][d]}{\twovectorsaction[a][b]}=\twovectorsaction[\twovectorsaction[a][b](c)][d]+\twovectorsaction[c][\twovectorsaction[a][b](d)].
\end{equation}
\noindent From \eqref{eqn:action-commutator}, we see $\forall a,b,c,d,e\in\genericspace$,
\begin{equation}
    \twovectorsaction[\twovectorsaction[a][b](c)][d](e)+\twovectorsaction[c][\twovectorsaction[a][b](d)](e)=-\twovectorsaction[\twovectorsaction[c][d](a)][b](e)-\twovectorsaction[a][\twovectorsaction[c][d](b)](e),
\end{equation}
\noindent from which we require,
\begin{equation}\label{eqn:commutator-constraint}
    f(f(a,b,c),d,e)+f(c,f(a,b,d),e)+f(f(c,d,a),b,e)+f(a,f(c,d,b),e)=0.
\end{equation}
\noindent The only non-trivial solution to \eqref{eqn:commutator-constraint} is,
\begin{equation*}\label{eqn:f-tensor-constrained}
    f(a,b,c)=k_{1}\big(\tensor{(\tensor{a;b}-\tensor{b;a});c}-\tensor{c;(\tensor{a;b}-\tensor{b;a})}\big),
\end{equation*}
\noindent defining $f(a,b,c)$ up to an arbitrary scaling, which was expected.

\subsection{The Spinless Weak Clifford Algebra}\label{sec:spinless-clifford-algebra}

We now have everything we need to construct the \enquote{Spinless Weak Clifford Algebra} $\spinlesscliffordalgebra$: choosing $k_{1}=\half$ in \eqref{eqn:f-tensor-constrained}, we define $\forall a,b,c\in\genericspace$,
\begin{equation}\label{eqn:spinless-clifford-algebra}
    \spinlesscliffordalgebra\cong\frac{\generictensoralgebra}{\ideal{\tensor{(\tensorwedge{a}{b});c}-\tensor{c;(\tensorwedge{a}{b})}-\genericmetric[a][c]b+\genericmetric[b][c]a}},
\end{equation}
\noindent whose defining identity,
\begin{equation}\label{eqn:lie-algebra-action}
    \tensor{(\tensorwedge{a}{b});c}-\tensor{c;(\tensorwedge{a}{b})}=\genericmetric[a][c]b-\genericmetric[b][c]a,
\end{equation}
\noindent agrees with the Clifford algebra's $\sopq$-action \eqref{eqn:clifford-bivector-action} up to a scaling. We use the term \enquote{weak} Clifford algebra to contrast the \enquote{strong} Clifford algebra in the sense used in logic: the defining relationship of $\genericcliffordalgebra$ is stronger than the defining relationship of $\spinlesscliffordalgebra$.

The bivector $\tensorwedge{a}{b}$ necessarily appearing whole in \eqref{eqn:lie-algebra-action} demonstrates its significance to the properties of $\genericmetric$, and that $\twovectorsaction[a][b]$ is truly a bivector-action on a vector. We may capture this by defining,
\begin{equation}\label{eqn:bivector-action}
    \definition{\bivectoraction[\tensorwedge{a}{b}]}{\mapdefinition{c}{\tensor{(\tensorwedge{a}{b});c}-\tensor{c;(\tensorwedge{a}{b})}}}=\twovectorsaction[a][b].
\end{equation}
\noindent As the unique embedding of $\twovectorsaction[a][b]$ in $\spinlesscliffordalgebra$, we may consider the properties of $\bivectoraction[\tensorwedge{a}{b}]$ to be the natural extension of $\twovectorsaction[a][b]$ to $\spinlesscliffordalgebra$. In particular, we find that $\bivectoraction[\tensorwedge{a}{b}]$ is naturally a derivation\cite{bourbaki} on $\spinlesscliffordalgebra$, $\forall A,B\in\generictensoralgebra$,
\begin{equation}\label{eqn:bivector-action-derivation-property}
    \bivectoraction[\tensorwedge{a}{b}](\tensor{A;B})=\tensor{\big(\bivectoraction[\tensorwedge{a}{b}](A)\big);B}+\tensor{A;\big(\bivectoraction[\tensorwedge{a}{b}](B)\big)},
\end{equation} since,
\begin{equation}
    \tensor{(\tensorwedge{a}{b});\big(\tensor{A;B}\big)}-\tensor{\big(\tensor{A;B}\big);(\tensorwedge{a}{b})}=\tensor{\big(\tensor{(\tensorwedge{a}{b});A}-\tensor{A;(\tensorwedge{a}{b})}\big);B}+\tensor{A;\big(\tensor{(\tensorwedge{a}{b});B}-\tensor{B;(\tensorwedge{a}{b})}\big)}.
\end{equation}
\noindent This agrees with the standard prescription for a Lie algebra-action on a tensor product of representations, so that exponentiation yields a representation of a Lie group\cite{hall}. To ensure consistency with the structure of the $\spinalgebra{s}$, we shall extend $\bivectoraction$ to $T(\genericantisymmetrictensors{2})$ as an associative algebra-action, $\forall\alpha\in\reals,\,A,B\in T(\genericantisymmetrictensors{2})$,%
\begin{subequations}
\begin{gather}
    \definition{\bivectoraction[\alpha]}{\mapdefinition{A}{\alpha A}}\\
    \definition{\bivectoraction[\tensor{A;B}]}{\composition{\bivectoraction[A]}{\bivectoraction[B]}}\label{eqn:bivector-action-associative-property}.
\end{gather}
\end{subequations}

As with \eqref{eqn:clifford-bivector-action}, \eqref{eqn:lie-algebra-action} determines the Lie product of $\sopq$, which in $\spinlesscliffordalgebra$ is, $\forall a,b,c,d\in\genericspace$,
\begin{equation}\label{eqn:lie-product}
\begin{aligned}
    \tensor{(\tensorwedge{a}{b});(\tensorwedge{c}{d})}-\tensor{(\tensorwedge{c}{d});(\tensorwedge{a}{b})}=\genericmetric[a][c](\tensorwedge{b}{d})-\genericmetric[b][c](\tensorwedge{a}{d})-\genericmetric[a][d](\tensorwedge{b}{c})+\genericmetric[b][d](\tensorwedge{a}{c}),
\end{aligned}
\end{equation}
\noindent consistent with the standard Lie product\cite{weinberg-qft} with $\hat{J}^{\mu\nu}=\unitimaginary J^{\mu\nu}=\unitimaginary(\tensorwedge{e^{\mu}}{e^{\nu}})$, and $\hat{J}^{\mu\nu}$ the generators in the physics convention. This is also consistent with the spin generator convention $\hat{S}_{a}=\unitimaginary\generator[a]$ in \cite{bradshaw}. By properties \eqref{eqn:bivector-action-associative-property} and \eqref{eqn:bivector-action-derivation-property}, we may use \eqref{eqn:lie-product} to write the commutator of $\bivectoraction$ compactly,
\begin{equation}\label{eqn:bivector-action-lie-product}
    \composition{\bivectoraction[\tensorwedge{a}{b}]}{\bivectoraction[\tensorwedge{c}{d}]}-\composition{\bivectoraction[\tensorwedge{c}{d}]}{\bivectoraction[\tensorwedge{a}{b}]}=\bivectoraction[\bivectoraction[\tensorwedge{a}{b}](\tensorwedge{c}{d})],
\end{equation}
\noindent showing that $\bivectoraction$ is indeed an $\sopq$-action.

\subsection{The Spinless Weak Clifford Algebra for \texorpdfstring{$\threemetricspace$}{(E,delta)}}

Restricting our attention to the present problem, we consider the three-dimensional Euclidean space $\threemetricspace$ from earlier, and its spinless weak Clifford algebra $\spinlessthreecliffordalgebra$. Using an orthonormal basis $\set{e_{a}}$, we define,
\begin{equation}\label{eqn:spin-generator-bivector-map}
    \definition{\generator[p]}{\half\sothreesum*[r]{a,b}\sothreestructureconstants{a}{b}{p}\,\tensorwedge{e_{a}}{e_{b}}},
\end{equation}
\noindent which in $\spinlessthreecliffordalgebra$ turns \eqref{eqn:lie-product} into the standard Lie product of $\sothree$ \eqref{eqn:so3-lie-product}. Therefore, $\usothree\subset\spinlesscliffordalgebra$, and we identify,
\begin{equation}
    \domainrestriction{\bivectoraction}{T(\genericantisymmetrictensors{2})}=\sothreead,
\end{equation}
\noindent where $\sothreead$ was defined in \eqref{eqn:adjoint-action}. We note the inverse transformation of \eqref{eqn:spin-generator-bivector-map},
\begin{equation}\label{eqn:bivector-spin-generator-map}
    \tensorwedge{e_{a}}{e_{b}}=\sothreesum*[r]{p}\sothreestructureconstants{a}{b}{p}\generator[p],
\end{equation}
\noindent and recognise that it enables us to write the multipoles of the $\spinalgebra{s}$ in the language of bivectors, $\forall k\in\integers^{+}$,
\begin{equation}
    \multipole{k}\Big(\bigotimes_{j=1}^{k}\tensorwedge{e_{a_{j}}}{e_{b_{j}}}\Big)=\sothreesum*[l]{p_{1},\dots,p_{k}}\prod_{j=1}^{k}\sothreestructureconstants{a_{j}}{b_{j}}{p_{j}}\multipole{k}\Big(\bigotimes_{m=1}^{k}\generator[p_{m}]\Big),
\end{equation}
\noindent with,
\begin{equation}
    \sothreecasimirelement=\half\sothreesum*[r]{a,b}\tensor{(\tensorwedge{e_{a}}{e_{b}});(\tensorwedge{e_{a}}{e_{b}})}.
\end{equation}
\noindent Significantly, though $\usothree\subset\spinlessthreecliffordalgebra$, $\spinlessthreecliffordalgebra$ has no spin structure whatsoever. This makes it the ideal basic structure with which to realise the $\spinalgebra{s}$ and explore their geometric content.

\subsection{Measures of \texorpdfstring{$k$}{k}-Volumes}

\subsubsection{\texorpdfstring{$k$}{k}-Volumes in \texorpdfstring{$\spinlesscliffordalgebra$}{Cl-s-w(V,g)}}

In $\genericcliffordalgebra$, geometric measurements about its objects are conveyed by the scalar part\cite{doran-lasenby} $\langle\cdot\rangle$, for example: $|\langle(\tensorwedge{v_{1}}{\dots}{v_{k}})^{2}\rangle|$ is the square of a $k$-blade's $k$-volume; and $\langle(\tensorwedge{v_{1}}{\dots}{v_{k}})(\tensorwedge{w_{1}}{\dots}{w_{k}})\rangle$ describes the projected overlap of two $k$-blades. In $\usothree\subset\spinlessthreecliffordalgebra$, there is a similar notion.

Recall from section \ref{sec:spin-algebras}, that all elements $A_{0}\in\usothree$ for which $\sothreead[\sothreecasimirelement](A_{0})=0$ are $\reals[\sothreecasimirelement]$-linear combinations of the monopole $\multipoletensor{}$. Thus, given an element $A\in\usothree$ on which $\sothreead[\sothreecasimirelement]$ has minimal polynomial $m(x)$, we define its \enquote{Monopole Part} $\mon[A]$,
\begin{equation}
    \definition{\mon[A]}{\begin{cases}
        \frac{n(\sothreead[\sothreecasimirelement])}{n(0)}(A) & m(x)=xn(x)\\
        0 & \textup{otherwise}.
    \end{cases}}
\end{equation}
\noindent We may use $\bivectoraction$ to identify geometrically meaningful objects within $\spinlesscliffordalgebra$ with objects forming simple $\bivectoraction[\usothree]$-modules; this is consistent with how the multipole tensors were identified\cite{bradshaw}. The $k$-blades are such modules, demonstrating their significance even in $\spinlesscliffordalgebra$.

\subsubsection{\texorpdfstring{$k$}{k}-Volumes in \texorpdfstring{$\spinlessthreecliffordalgebra$}{Cl-s-w(E,delta)}}

In $\spinlessthreecliffordalgebra$, we wish to capture the sizes of bivectors and trivectors in some natural way. Accordingly, the invariant tensors at second-order,
\begin{equation}
    \mon[\tensor{\generator[p];\generator[q]}]=\frac{1}{3}\delta_{pq}\sothreecasimirelement
\end{equation}
\noindent and third-order,
\begin{equation}
    \mon[\tensor{\generator[p];\generator[q];\generator[r]}]=\frac{1}{6}\sothreestructureconstants{p}{q}{r}\sothreecasimirelement
\end{equation}
\noindent in $\usothree$ are of particular interest, where we have used that $\multipoletensor{}=1$. In bivector form, these are respectively,
\begin{equation}
    \mon[\tensor{(\tensorwedge{a}{b});(\tensorwedge{c}{d})}]=\frac{1}{3}\big(\threemetric[a][c]\threemetric[b][d]-\threemetric[a][d]\threemetric[b][c]\big)\sothreecasimirelement,
\end{equation}
\noindent and,
\begin{equation}
    \mon[\tensor{(\tensorwedge{a}{b});(\tensorwedge{c}{d});(\tensorwedge{e}{f})}]=\frac{1}{6}\begin{aligned}[t]
        \Big(&\threemetric[a][c]\big(\threemetric[b][e]\threemetric[d][f]-\threemetric[b][f]\threemetric[d][e]\big)\\
        -&\threemetric[a][d]\big(\threemetric[b][e]\threemetric[c][f]-\threemetric[b][f]\threemetric[c][e]\big)\\
        -&\threemetric[a][e]\big(\threemetric[b][c]\threemetric[d][f]-\threemetric[b][d]\threemetric[c][f]\big)\\
        +&\threemetric[a][f]\big(\threemetric[b][c]\threemetric[d][e]-\threemetric[b][d]\threemetric[c][e]\big)\Big)\sothreecasimirelement
    \end{aligned}
\end{equation}
\noindent Since these objects are invariant under $\bivectoraction$, they are invariant under the action of $\specialorthogonalgroup{3}$, and we may use them to extend the metric $\threemetric$ of $\threemetricspace$ to a metric $\antimetric$ on the space of all antisymmetric tensors\cite{bourbaki} $\threeantisymmetrictensors{}$, $\forall\alpha,\beta\in\reals$, $\forall a,b,c,d,e,f\in\threespace$,
\begin{subequations}
\begin{gather}
    \definition{\antimetric[\alpha][\beta]}{\alpha\beta}\\
    \definition{\antimetric[a][b]}{\threemetric[a][b]}\\
    \begin{aligned}
        \definition{\antimetric[\tensorwedge{a}{b}][\,\tensorwedge{c}{d}]&}{\mon[\tensor{(\tensorwedge{a}{b});(\tensorwedge{c}{d})}]}\\
        &=\frac{1}{3}\cbmetric[\tensorwedge{a}{b}][\tensorwedge{c}{d}]\sothreecasimirelement
    \end{aligned}\\
    \begin{aligned}
        \definition{\antimetric[\tensorwedge{a}{b}{c}][\,\tensorwedge{d}{e}{f}]&}{\begin{aligned}[t]
            &\mon[\tensor{(\tensorwedge{a}{b});(\tensorwedge{c}{d});(\tensorwedge{e}{f})}]\\
            &-\mon[\tensor{(\tensorwedge{a}{b});(\tensorwedge{c}{e});(\tensorwedge{d}{f})}]\\
            &+\mon[\tensor{(\tensorwedge{a}{b});(\tensorwedge{c}{f});(\tensorwedge{d}{e})}]
        \end{aligned}}\\
        &=-\frac{1}{3}\cbmetric[\tensorwedge{a}{b}{c}][\tensorwedge{d}{e}{f}]\sothreecasimirelement,
    \end{aligned}
\end{gather}
\end{subequations}
\noindent with all other combinations zero, and $\cbmetric$ is the usual Cauchy-Binet metric\cite{bourbaki} for $\threemetricspace$, $\forall k\in\integers^{+}$,
\begin{equation}
    \definition{\cbmetric[\bigwedge_{j=1}^{n}a_{j}][\bigwedge_{k=1}^{n}b_{k}]}{\det(\threemetric[a_{j}][b_{k}])}.
\end{equation}
\noindent Note that $\antimetric$ is not (yet) scalar-valued,
\begin{equation}
    \mapdeclaration{\antimetric}{\threeantisymmetrictensors{}}{\reals[\sothreecasimirelement]}.
\end{equation}

\end{document}