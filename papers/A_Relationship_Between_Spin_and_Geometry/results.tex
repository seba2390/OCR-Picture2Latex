\documentclass{article}

\usepackage{csquotes}
\usepackage{hyperref}
\usepackage{so3}
\usepackage{tensor}
\usepackage{vector-spaces}
\usepackage{multipoles}
\usepackage{operators}

\usepackage{xcolor}

\begin{document}

\section{Results}\label{sec:results}

\subsection{The Spin-\texorpdfstring{$s\neq0$}{s=/=0} Weak Clifford algebras}

Using the spinless weak Clifford algebra for Euclidean three-space $\spinlessthreecliffordalgebra$ of \eqref{eqn:spinless-clifford-algebra}, we may define the \enquote{Spin-$s$ Weak Clifford Algebra} $\spinscliffordalgebra{s}$ for spin-$s\neq0$,
\begin{equation}\label{eqn:spin-s-weak-clifford-definition}
    \spinscliffordalgebra{s}\cong\frac{\spinlessthreecliffordalgebra}{\ideal{\image{\multipole{2s+1}}}}.
\end{equation}
\noindent Since $\usothree\subset\spinlessthreecliffordalgebra$, this is equivalent to embedding the structure of the $\spinalgebra{s}$ within $\spinlessthreecliffordalgebra$. Within this algebra, we may positively identify the $\spinalgebra{s}$ as algebras of bivectors in general, whose action on $\threespace$ respects its Euclidean geometry $\threemetric$. Thus, we have finally explicated the structure of the $\spinalgebra{s}$ in both geometric and algebraic terms. However, the embedding of their spin structures also has an impact on the geometry itself. 

As before, the quotient \eqref{eqn:spin-s-weak-clifford-definition} entails,
\begin{equation}\label{eqn:casimir-element-weak-clifford}
    \sothreecasimirelement=-s(s+1),
\end{equation}
\noindent within $\spinscliffordalgebra{s}$. This ensures that the metric $\antimetric$ becomes scalar valued for bivectors and trivectors,
\begin{subequations}
\begin{gather}
    \antimetric[\tensorwedge{a}{b}][\,\tensorwedge{a}{b}]=-\frac{s(s+1)}{3}\cbmetric[\tensorwedge{a}{b}][\tensorwedge{a}{b}]\\
    \antimetric[\tensorwedge{a}{b}{c}][\,\tensorwedge{a}{b}{c}]=\frac{s(s+1)}{3}\cbmetric[\tensorwedge{a}{b}{c}][\tensorwedge{a}{b}{c}],
\end{gather}
\end{subequations}
\noindent and naturally \textit{spin dependent}. The metric on vectors and on scalars remains spin independent. Besides this feature, the values of the metric $\antimetric$ are quite different from those of the usual Clifford algebra on $\threemetricspace$. It has more consistency with $\threeconsistentcliffordalgebra$, for example the bivectors have the same size and sign, and the trivector has the same sign. However, the vectors and trivector are too large by a factor of $2$, and the vectors are also positive-definite. There is freedom in scaling $\antimetric$ in these sectors for consistency, but the author can see no mathematical reason for doing so at time of writing.

Despite this change to square norms, the algebraic structure imparted on $\spinscliffordalgebra{s}$ does not affect the action of $\bivectoraction[\tensorwedge{a}{b}]$ nor the rotational behaviour of $\threespace\subset\spinscliffordalgebra{s}$, since \eqref{eqn:spin-s-weak-clifford-definition} constrains only totally symmetric combinations of bivectors. However, the structure of $\spinscliffordalgebra{s}$ is significantly affected by the spin structure from $\spinalgebra{s}$. Recall that $\spinalgebra{s}$ is defined completely from $\usothree$ by $\image{\multipole{2s+1}}=\set{0}$. Taking spin-$\half$ as an example, this is equivalent to a series of tensor identities, $\forall a,b\in\set{1,2,3}$,
\begin{equation}
    \half(\tensor{\generator[a];\generator[b]}+\tensor{\generator[b];\generator[a]})+\frac{1}{4}\delta_{ab}=0.
\end{equation}
\noindent In the language of bivectors this condition is equivalent to $\forall a,b,c,d\in\threespace$,
\begin{equation}\label{eqn:quadrupole-bivector-identity}
    \half\big(\tensor{(\tensorwedge{a}{b});(\tensorwedge{c}{d})}+\tensor{(\tensorwedge{c}{d});(\tensorwedge{a}{b})}\big)=\antimetric[\tensorwedge{a}{b}][\,\tensorwedge{c}{d}].
\end{equation}
\noindent Recognising $\antimetric[\tensorwedge{a}{b}][\,\tensorwedge{c}{d}]$ as a scalar, we may break up each bivector on the left-hand side according to \eqref{eqn:antisymmetric-tensors}, revealing \eqref{eqn:quadrupole-bivector-identity} to be an constraint on fourth-order tensors in $\spinscliffordalgebra{1/2}$. For spin-$s$, these identities constrain order $2(2s+1)$ tensors. Interpreting $\threespace$ as physical Euclidean space, these embeddings of the spin structures of $\spinalgebra{s}$ within $\spinlessthreecliffordalgebra$ constitute non-commutative geometries for $\threespace$, in the sense of non-commuting position observables\cite{szabo,aschieri,frob}.

\subsection{The Spin-\texorpdfstring{$0$}{0} Weak Clifford algebra}

The case of the spin-$0$ algebra is an edge-case requiring separate treatment. $\spinalgebra{0}$ contains only the monopole $\multipoletensor{}=1$, and is defined by $\image{\multipole{1}}=\set{0}$, which entails $\sothreecasimirelement=0$. In the bivector language, this means that $\forall a,b\in\threespace$,
\begin{equation}
    \tensorwedge{a}{b}=0.
\end{equation}
\noindent Trying to apply this identity to $\spinlessthreecliffordalgebra$ as in \eqref{eqn:spin-s-weak-clifford-definition} results in the trivial algebra $\reals$. This is because in \eqref{eqn:spinless-clifford-algebra} we associate $\big[\threeantisymmetrictensors{2},\threespace\big]$ with the whole of $\threespace$, so quotienting all bivectors in $\spinlessthreecliffordalgebra$ sets all vectors in the algebra to $0$. Really, the identity \eqref{eqn:lie-algebra-action} is only reasonable when the bivectors are non-zero in the algebra, otherwise we seek an action of zero mapping any vector to any other. To avoid this, we must impose the structure of $\spinalgebra{0}$ directly on $\threetensoralgebra$,
\begin{equation}
    \spinscliffordalgebra{0}\cong\frac{\threetensoralgebra}{\image{\multipole{1}}}.
\end{equation}
\noindent This algebra is unique amongst the spin-$s$ weak Clifford algebras: $\image{\multipole{1}}=\set{0}$ implies that $\forall a,b\in\threespace$,
\begin{equation}
    \tensor{a;b}=\tensor{b;a},
\end{equation}
\noindent so $\spinscliffordalgebra{0}\cong\textup{Sym}(\threespace)$ is commutative. In fact, a spin-$s$ weak Clifford algebra is commutative iff it has a spin-$0$ structure. Additionally, $\sothreecasimirelement=0$ implies that,
\begin{subequations}
\begin{gather}
    \antimetric[\tensorwedge{a}{b}][\,\tensorwedge{a}{b}]=0\\
    \antimetric[\tensorwedge{a}{b}{c}][\,\tensorwedge{a}{b}{c}]=0,
\end{gather}
\end{subequations}
\noindent which is consistent with our expectations from the other spin-$s$ weak Clifford algebras, and the commutative nature of $\spinscliffordalgebra{0}$.

\end{document}