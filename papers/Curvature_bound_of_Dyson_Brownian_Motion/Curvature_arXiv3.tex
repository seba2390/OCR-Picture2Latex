%\documentclass[a4paper, leqno, 12pt]{amsart}
\documentclass[11pt,letterpaper]{amsart}



%PACKAGES
\usepackage[T1]{fontenc}	
\usepackage{graphicx}
\usepackage{amssymb,centernot}
\usepackage{upgreek}
\usepackage{mathrsfs}
%\usepackage{soul}
%\usepackage{pdfsync}
\usepackage[usenames,dvipsnames]{xcolor}
%\usepackage{enumerate}							%opzioni di enumerate
\usepackage[inline,shortlabels]{enumitem}
\usepackage[sort,numbers]{natbib}						%bibliografia latex
%\usepackage[hyphens]{url}							%per gli url in bliografia
\usepackage{verbatim}								%per commentare cose lunghe
\usepackage{dsfont}									%doublestroke \mathds
%\usepackage{microtype}

%To rotate tables
\usepackage{pdflscape}
\usepackage{afterpage}

\usepackage{anyfontsize}
%HYPERREF E LINKING
%\definecolor{DarkGray}{gray}{0.35}
\usepackage[
						%hidelinks,					%nasconde i link: antagonista di colorlinks (s“ 1 E no 2)
						colorlinks,				%link a colori invece che riquadrati
						breaklinks,unicode,
						hypertexnames=false,
						citecolor=OliveGreen,
						linkcolor=Maroon
						]{hyperref} %hyperref deve stare per ultimo

%\usepackage[ 
%colorlinks=true,
%linkcolor=black,
%filecolor=black,
%urlcolor=black,
%citecolor=black
%]{hyperref}

%BIBLIOGRAFIA
%\usepackage[fixlanguage]{babelbib}		%confligge con KOMA (uspckg: scrextend) quindi deve stare DOPO

%PACKAGES for tabular
\usepackage{multirow}

\usepackage{xypic}
\usepackage{mathtools}
\usepackage{xspace}

%\usepackage[a4paper,top=3.2cm,bottom=2.7cm,left=3cm,right=3cm, bindingoffset=0mm]{geometry}
%\linespread{1.2}
%\usepackage[color]{showkeys}		%mostra i riferimenti. Deve stare dopo AMSmath e DOPO hyperref
\usepackage{booktabs}

%TIKZ
\usepackage{tikz}




\usetikzlibrary{calc, positioning, shapes.geometric, patterns, cd}

%TIKZ FILLING PATTERNS
% defining the new dimensions and parameters
\newlength{\hatchspread}
\newlength{\hatchthickness}
\newlength{\hatchshift}
\newcommand{\hatchcolor}{}
% declaring the keys in tikz
\tikzset{hatchspread/.code={\setlength{\hatchspread}{#1}},
         hatchthickness/.code={\setlength{\hatchthickness}{#1}},
         hatchshift/.code={\setlength{\hatchshift}{#1}},% must be >= 0
         hatchcolor/.code={\renewcommand{\hatchcolor}{#1}}}
% setting the default values
\tikzset{hatchspread=3pt,
         hatchthickness=0.4pt,
         hatchshift=0pt,% must be >= 0
         hatchcolor=black}
% declaring the pattern
\pgfdeclarepatternformonly[\hatchspread,\hatchthickness,\hatchshift,\hatchcolor]% variables
   {custom north west lines}% name
   {\pgfqpoint{\dimexpr-2\hatchthickness}{\dimexpr-2\hatchthickness}}% lower left corner
   {\pgfqpoint{\dimexpr\hatchspread+2\hatchthickness}{\dimexpr\hatchspread+2\hatchthickness}}% upper right corner
   {\pgfqpoint{\dimexpr\hatchspread}{\dimexpr\hatchspread}}% tile size
   {% shape description
    \pgfsetlinewidth{\hatchthickness}
    \pgfpathmoveto{\pgfqpoint{0pt}{\dimexpr\hatchspread+\hatchshift}}
    \pgfpathlineto{\pgfqpoint{\dimexpr\hatchspread+0.15pt+\hatchshift}{-0.15pt}}
    \ifdim \hatchshift > 0pt
      \pgfpathmoveto{\pgfqpoint{0pt}{\hatchshift}}
      \pgfpathlineto{\pgfqpoint{\dimexpr0.15pt+\hatchshift}{-0.15pt}}
    \fi
    \pgfsetstrokecolor{\hatchcolor}
%    \pgfsetdash{{1pt}{1pt}}{0pt}% dashing cannot work correctly in all situation this way
    \pgfusepath{stroke}
   }

\pgfdeclarepatternformonly[\hatchspread,\hatchthickness,\hatchshift,\hatchcolor]% variables
   {custom north east lines}% name
   {\pgfqpoint{\dimexpr-2\hatchthickness}{\dimexpr-2\hatchthickness}}% lower left corner
   {\pgfqpoint{\dimexpr\hatchspread+2\hatchthickness}{\dimexpr\hatchspread+2\hatchthickness}}% upper right corner
   {\pgfqpoint{\dimexpr\hatchspread}{\dimexpr\hatchspread}}% tile size
   {% shape description
    \pgfsetlinewidth{\hatchthickness}
    \pgfpathmoveto{\pgfqpoint{\dimexpr\hatchshift-0.15pt}{-0.15pt}}
    \pgfpathlineto{\pgfqpoint{\dimexpr\hatchspread+0.15pt}{\dimexpr\hatchspread-\hatchshift+0.15pt}}
    \ifdim \hatchshift > 0pt
      \pgfpathmoveto{\pgfqpoint{-0.15pt}{\dimexpr\hatchspread-\hatchshift-0.15pt}}
      \pgfpathlineto{\pgfqpoint{\dimexpr\hatchshift+0.15pt}{\dimexpr\hatchspread+0.15pt}}
    \fi
    \pgfsetstrokecolor{\hatchcolor}
%    \pgfsetdash{{1pt}{1pt}}{0pt}% dashing cannot work correctly in all situation this way
    \pgfusepath{stroke}
   }

%OVERFLOWING TABLES/FIGURES
\makeatletter
\newcommand*{\centerfloat}{%
  \parindent \z@
  \leftskip \z@ \@plus 1fil \@minus \textwidth
  \rightskip\leftskip
  \parfillskip \z@skip}
\makeatother


%REMOVE INTRODUCTION SUBSECTIONS FROM TOC%
\newcommand{\hiddensubsection}[1]{
    \stepcounter{subsection}
    \subsection*{\arabic{section}.\arabic{subsection}.\hspace{.5em}{#1}}
}

\newcommand{\hiddensubsubsection}[1]{
    \stepcounter{subsubsection}
    \subsubsection*{{\rm\arabic{section}.\arabic{subsection}.\arabic{subsubsection}.}\hspace{.5em}{#1}}
}


\newcommand{\SCF}{$(\mathsf{SCF})_{\ref{d:wFe}}$\xspace}
\newcommand{\qSCF}{$(\mathsf{qSCF})_{\ref{d:qSCF}}$\xspace}

%%%%%%%%%%%%%%%%%%%%OPERATORS AND COMMANDS%%%%%%%%%%%%%BOLDSYMBOL GREEK CAPITAL
\newcommand{\boldBeta}{\boldsymbol{\mathsf{B}}}
\newcommand{\boldGamma}{{\boldsymbol\Gamma}}
\newcommand{\boldDelta}{{\boldsymbol\Delta}}
\newcommand{\boldLambda}{{\boldsymbol\Lambda}}
\newcommand{\boldSigma}{{\boldsymbol\A}}
\newcommand{\boldTau}{\mbfT}
\newcommand{\boldPhi}{{\boldsymbol\Phi}}
\newcommand{\boldXi}{{\boldsymbol\Xi}}
\newcommand{\boldUpsilon}{{\boldsymbol\Upsilon}}
\newcommand{\boldOmega}{{\boldsymbol\Omega}}
%\renewcommand{\\}{\\ \smallskip }

%LATIN GREEK CAPITAL
\newcommand{\Alpha}{\mathrm{A}}
\newcommand{\Beta}{\mathrm{B}}
\newcommand{\Eta}{\mathrm{H}}
\newcommand{\Kappa}{\mathrm{K}}
\newcommand{\Mu}{\mathrm{M}}
\newcommand{\Tau}{\mathrm{T}}

\newcommand{\grey}{\textcolor{gray}}
%BOLDSYMBOL GREEK
\newcommand{\boldalpha}{{\boldsymbol \alpha}}
\newcommand{\boldbeta}{{\boldsymbol \beta}}
\newcommand{\boldgamma}{{\boldsymbol \gamma}}
\newcommand{\boldeta}{{\boldsymbol \eta}}
\newcommand{\boldiota}{{\boldsymbol \iota}}
\newcommand{\boldlambda}{{\boldsymbol \lambda}}
\newcommand{\boldkappa}{{\boldsymbol \kappa}}
\newcommand{\boldmu}{{\boldsymbol \mu}}
\newcommand{\boldnu}{{\boldsymbol \nu}}
\newcommand{\boldpi}{{\boldsymbol \pi}}
\newcommand{\boldphi}{{\boldsymbol \varphi}}
\newcommand{\boldvarrho}{{\boldsymbol \varrho}}
%
\newcommand{\rep}[1]{\hat #1}	

%BOLDSYMBOL SANS-SERIF
\newcommand{\bmssd}{\boldsymbol\mssd}

\usepackage{xparse}

\ExplSyntaxOn
%abbreviations
\NewDocumentCommand{\makeabbrev}{mmm}
 {
  \yoruk_makeabbrev:nnn { #1 } { #2 } { #3 }
 }

\cs_new_protected:Npn \yoruk_makeabbrev:nnn #1 #2 #3
 {
  \clist_map_inline:nn { #3 }
   {
    \cs_new_protected:cpn { #2 } { #1 { ##1 } }
   }
 }
 \ExplSyntaxOff
 

\makeabbrev{\textbf}{tbf#1}{a,b,c,d,e,f,g,h,i,j,k,l,m,n,o,p,q,r,s,t,u,v,w,x,y,z,A,B,C,D,E,F,G,H,I,J,K,L,M,N,O,P,Q,R,S,T,U,V,W,X,Y,Z}

\makeabbrev{\textbf}{bf#1}{a,b,c,d,e,f,g,h,i,j,k,l,m,n,o,p,q,r,s,t,u,v,w,x,y,z,A,B,C,D,E,F,G,H,I,J,K,L,M,N,O,P,Q,R,S,T,U,V,W,X,Y,Z}

\makeabbrev{\textsf}{tsf#1}{a,b,c,d,e,f,g,h,i,j,k,l,m,n,o,p,q,r,s,t,u,v,w,x,y,z,A,B,C,D,E,F,G,H,I,J,K,L,M,N,O,P,Q,R,S,T,U,V,W,X,Y,Z}

\makeabbrev{\mathsf}{mss#1}{a,b,c,d,e,f,g,h,i,j,k,l,m,n,o,p,q,r,s,t,u,v,w,x,y,z,A,B,C,D,E,F,G,H,I,J,K,L,M,N,O,P,Q,R,S,T,U,V,W,X,Y,Z}

\makeabbrev{\mathfrak}{mf#1}{a,b,c,d,e,f,g,h,i,j,k,l,m,n,o,p,q,r,s,t,u,v,w,x,y,z,A,B,C,D,E,F,G,H,I,J,K,L,M,N,O,P,Q,R,S,T,U,V,W,X,Y,Z}

\makeabbrev{\mathrm}{mrm#1}{a,b,c,d,e,f,g,h,i,j,k,l,m,n,o,p,q,r,s,t,u,v,w,x,y,z,A,B,C,D,E,F,G,H,I,J,K,L,M,N,O,P,Q,R,S,T,U,V,W,X,Y,Z}

\makeabbrev{\mathbf}{mbf#1}{a,b,c,d,e,f,g,h,i,j,k,l,m,n,o,p,q,r,s,t,u,v,w,x,y,z,A,B,C,D,E,F,G,H,I,J,K,L,M,N,O,P,Q,R,S,T,U,V,W,X,Y,Z}

\makeabbrev{\mathcal}{mc#1}{A,B,C,D,E,F,G,H,I,J,K,L,M,N,O,P,Q,R,S,T,U,V,W,X,Y,Z}

\makeabbrev{\mathbb}{mbb#1}{A,B,C,D,E,F,G,H,I,J,K,L,M,N,O,P,Q,R,S,T,U,V,W,X,Y,Z}

\makeabbrev{\mathscr}{ms#1}{A,B,C,D,E,F,G,H,I,J,K,L,M,N,O,P,Q,R,S,T,U,V,W,X,Y,Z}

\makeabbrev{\mathrm}{#1}{
%Algebraic Ops
Id,id,ran,rk,diag,stab,ann,conv,pr,ev,tr,End,Hom,sgn,im,op,can,fin,ext,red,tot,
%
%Analytic Ops
rot,usc,lsc,Lip,LocLip,lip,bSymLip,osc,AC,loc,uloc,spec,coz,z,ul,
%
%Measure Theory
supp,Opt,Adm,Cpl,Geo,GeoSel,GeoOpt,GeoAdm,GeoCpl,reg,
%
%Topology/Geometry
bd,co,Ric,Exp,dExp,dist,seg,Seg,cut,fcut,Cut,SDiff,Iso,Isom,diam,cl,Homeo,Diff,Der,vol,dvol,inj,relint, Graph, sub,codim,
%
%Probability
var,law,Var,Poi,Gam,pa,so,iso,fs,inv,pqi,mix,
%
TestF,
%Miscellaneous
}

\makeabbrev{\mathsf}{#1}{DP,CD,BE,MCP,Ent,wMTW,MTW,RCD,ncRCD,QCD,EVI,Irr,IH,SC,wFe,VA,UP,Curv,Alex,CAT}


\newcommand{\bLip}{\mathrm{Lip}_b}
\newcommand{\bsLip}{\mathrm{Lip}_{bs}}

\newcommand{\KSD}{D} %dense set in the KS convergence

\newcommand{\T}{\tau} %TOPOLOGY
\newcommand{\A}{\Sigma} %SIGMA-ALGEBRA
\newcommand{\Bo}[1]{\msB_{#1}} %BOREL SIGMA-ALGEBRA
\newcommand{\Bob}[1]{\msB_b(#1)} %BOREL SIGMA-ALGEBRA
\newcommand{\Ko}[1]{\msK_{#1}}
\newcommand{\rKo}[1]{{}_r\!\msK_{#1}}
\newcommand{\Bdd}[1]{\msO_{#1}}
\newcommand{\Ed}{{\msE_\mssd}}  

%%%%%%%%%%%%%%%%%%%%%%%%VARI%%%%%%%%%%%%%%%%%%%%
\newcommand{\eps}{\varepsilon}
\newcommand{\kai}{\varkappa}
\renewcommand{\div}{\mathrm{div}}
\newcommand{\defeq}{\eqqcolon}
\renewcommand{\complement}{\mathrm{c}}

\newcommand{\acts}{\,\raisebox{\depth}{\scalebox{1}[-1]{$\circlearrowleft$}}\,}

\newcommand{\mathsc}[1]{\text{\textsc{#1}}}
\newcommand{\emparg}{{\,\cdot\,}}
\DeclareMathOperator*{\argmin}{argmin}
\DeclareMathOperator{\Span}{span}
\newcommand{\tail}{\mathrm{tail}}

\newcommand{\slo}[2][]{\abs{\mathrm{D}#2}_{#1}}
\newcommand{\wslo}[2][]{\abs{\mathrm{D}#2}_{w,\, #1}}
\newcommand{\slog}[1]{\abs{\nabla#1}}
\newcommand{\Sb}{\A_b}
\newcommand{\pos}[1]{\tbraket{#1}_+}

\newcommand{\dint}[2][]{\sideset{^{#1}\!\!\!}{_{#2}^{\scriptstyle\oplus}}\int}

\newcommand{\Proj}[2]{{{\rm{pr}}^{#1}_{#2}}}
\newcommand{\Ch}{\mathsf{Ch}}
\newcommand{\CE}[1][]{\mathsf{CE}_{#1}}
\newcommand{\qe}[1]{\;{#1}\text{-q.e.}}
\newcommand{\forallae}[1]{{\textrm{\,for ${#1}$-a.e.\,}}}
\newcommand{\as}[1]{\quad #1\text{-a.e.}}
\newcommand{\Leb}{{\mathrm{Leb}}}

\newcommand{\LT}[1]{\mcL\quadre{#1}}		%Laplace transform

\renewcommand{\Cap}{\mathrm{cap}}
\newcommand{\hCap}{\mathrm{\hat{C}ap}}
\newcommand{\ca}{\mathrm{cap}}
\newcommand{\dom}[1]{\mathcal D(#1)}
\newcommand{\sem}[1]{\{#1\}_{t \ge 0}}
\newcommand{\domain}[1]{\msD({#1})}
\DeclareMathOperator{\Dom}{dom}

\newcommand{\domloc}[1]{\mathcal D_{\loc}(#1)}
\newcommand{\domext}[1]{\msD_{e}(#1)}
\newcommand{\DzLoc}[1]{\mbbL^{#1}_{\loc}}
\newcommand{\DzLocB}[1]{\mbbL^{#1}_{\loc,b}}
\newcommand{\DzLocBprime}[1]{\mbbL^{\prime\, #1}_{\loc,b}}
\newcommand{\DzB}[1]{\mbbL^{#1}_b}
\newcommand{\DzBprime}[1]{\mbbL^{\prime\, #1}_b}
\newcommand{\Lipu}{\mathrm{Lip}^1}
\newcommand{\Lipua}{\mathrm{Lip}^\alpha}
\newcommand{\bLipu}{\mathrm{Lip}^1_b}
\newcommand{\dotloc}[1]{{#1}^\bullet_\loc}
\newcommand{\dotlocprime}[1]{{{#1}^{\bullet\prime}_{\loc}}}
\newcommand{\Rad}[2]{\mathsf{Rad}_{#1,#2}}
\newcommand{\kWC}[1]{\mathsf{WC}^{\mathrm{ker}}_{#1}}
\newcommand{\sgWC}[1]{\mathsf{WC}^{\mathrm{sg}}_{#1}}
\newcommand{\dRad}[2]{{#1}\textrm{-}\mathsf{Rad}_{#2}}
\newcommand{\ScL}[3]{\mathsf{ScL}_{#1,#2,#3}}
\newcommand{\cSL}[3]{\mathsf{cSL}_{#1,#2,#3}}
\newcommand{\SL}[2]{\mathsf{SL}_{#1,#2}}
\newcommand{\dcSL}[3]{{#1}\textrm{-}\mathsf{cSL}_{#2,#3}}
\newcommand{\dSL}[2]{{#1}\textrm{-}\mathsf{SL}_{#2}}

\newcommand{\comm}{\,\,\mathrm{,}\;\,}

\DeclareMathOperator{\eqdef}{\coloneqq}
\DeclareMathOperator{\iffdef}{\overset{\cdot}{\iff}}

\renewcommand\qedsymbol{$\blacksquare$}

\let\epsilon\varepsilon
\newcommand{\stigma}{\varsigma}
\let\temp\phi
\let\phi\varphi
\let\varphi\temp

\newcommand{\longrar}{\longrightarrow}
\newcommand{\rar}{\rightarrow}
\newcommand{\Lar}{\,\Longleftarrow\,}
\newcommand{\Rar}{\,\Longrightarrow\,}

\newcommand{\nlim}{\lim_{n}}								%omesso \rightarrow\infty
\newcommand{\mlim}{\lim_{m}}
\newcommand{\klim}{\lim_{k }}
\newcommand{\hlim}{\lim_{h }}
\newcommand{\ilim}{\lim_{i }}
\newcommand{\rlim}{\lim_{r  \rightarrow\infty}}
\newcommand{\xlim}{\lim_{x\rightarrow x_0}}
\newcommand{\nliminf}{\liminf_{n }}
\newcommand{\mliminf}{\liminf_{m }}
\newcommand{\kliminf}{\liminf_{k }}
\newcommand{\iliminf}{\liminf_{i }}
\newcommand{\nlimsup}{\limsup_{n }\,}
\newcommand{\klimsup}{\limsup_{k }\,}
\newcommand{\hlimsup}{\limsup_{h }\,}
\newcommand{\hklimsup}{\limsup_{h,k }\,}
\newcommand{\mlimsup}{\limsup_{m }\,}
\DeclareMathOperator*{\dirlim}{\underrightarrow{\lim}}							%limite diretto (induttivo)

\newcommand{\diff}{\mathop{}\!\mathrm{d}}
%\renewcommand{\diff}{d}						%Differenziale esatto

\newcommand{\ttabs}[1]{\lvert#1\rvert}	
\newcommand{\tabs}[1]{\big\lvert#1\big\rvert}	
\newcommand{\abs}[1]{\left\lvert#1\right\rvert}						%Modulo
\newcommand{\tnorm}[1]{\big\lVert#1\big\rVert}						%Norma
\newcommand{\ttnorm}[1]{\lVert#1\rVert}
\newcommand{\norm}[1]{\left\lVert#1\right\rVert}					%Norma
\newcommand{\set}[1]{\left\{#1\right\}}							%Insieme, graffe
\newcommand{\tset}[1]{\big\{#1\big\}}							%Insieme, graffe
\newcommand{\ttset}[1]{\{#1\}}									%Insieme, graffe
\newcommand{\ceiling}[1]{\left\lceil#1\right\rceil}					%Ceiling
\newcommand{\floor}[1]{\left\lfloor#1\right\rfloor}					%Floor
\newcommand{\paren}[1]{\left(#1\right)}							%Tonde
\newcommand{\tparen}[1]{\big({#1}\big)}
\newcommand{\ttparen}[1]{({#1})}
\newcommand{\tonde}[1]{\left(#1\right)}	
\newcommand{\ttonde}[1]{\big({#1}\big)}
\newcommand{\etonde}[1]{\mathrm{(}#1\mathrm{)}}					%tonde per enumerazione in $$
\newcommand{\quadre}[1]{\left[#1\right]}							%Quadre
\newcommand{\class}[2][]{\left[#2\right]_{#1}}						%Measure classes
\newcommand{\tclass}[2][]{\big [#2\big]_{#1}}						%Measure classes
\newcommand{\ttclass}[2][]{[#2]_{#1}}							%Measure classes
\newcommand{\spclass}[2][]{#2_{#1}}

\newcommand{\Li}[2][]{\mathrm{L}_{#1}(#2)}						%Lipschitz
\newcommand{\bigLi}[2][]{\mathrm{L}_{#1}\!\!\paren{#2}}						%Lipschitz
\newcommand{\bLi}[2][]{\mathrm{L}_{#1}\Big({#2}\Big)}						%Lipschitz
\newcommand{\Lia}[2][]{\mathrm{Lip}^a_{#1}[#2]}						%Lipschitz
\newcommand{\tLi}[2][]{\big [#2\big]_{#1}}							%Lipschitz

%\newcommand{}[1]{\hat #1}									%Rappresentante
\newcommand{\widerep}[1]{\widehat{#1}}									%Rappresentante
%\newcommand{two}[1]{\tilde{#1}}							%Rappresentante
\newcommand{\tbraket}[1]{\big[#1\big]}							%Quadre
\newcommand{\tscalar}[2]{\big\langle #1 \, \big |\, #2\big\rangle}			%Prodotto scalare
\newcommand{\scalar}[2]{\left\langle #1 \,\middle |\, #2\right\rangle}		%Prodotto scalare
\newcommand{\qvar}[1]{\quadre{#1}}								%Quadratic variation
\newcommand{\hotimes}{\widehat{\otimes}}

\newcommand{\hodot}{\widehat{\odot}}
\newcommand{\hoplus}{\widehat{\oplus}}

\newcommand{\asym}[1]{{\scriptscriptstyle{[#1]}}}
\newcommand{\sym}[1]{{\scriptscriptstyle{(#1)}}}
\newcommand{\tym}[1]{{\scriptscriptstyle{\times #1}}}
\newcommand{\otym}[1]{{\scriptscriptstyle{\otimes #1}}}
\newcommand{\osym}[1]{{\scriptscriptstyle{\odot #1}}}
\newcommand{\hotym}[1]{{\scriptscriptstyle{\widehat{\otimes}{#1}}}}

%TO AVOID STMARYARD WARNINGS
\DeclareSymbolFont{symbolsC}{U}{pxsyc}{m}{n}
\SetSymbolFont{symbolsC}{bold}{U}{pxsyc}{bx}{n}
\DeclareFontSubstitution{U}{pxsyc}{m}{n}
\DeclareMathSymbol{\medcirc}{\mathbin}{symbolsC}{7}
\DeclareSymbolFont{symbolsZ}{OMS}{pxsy}{m}{n}
\SetSymbolFont{symbolsZ}{bold}{OMS}{pxsy}{bx}{n}
\DeclareFontSubstitution{OMS}{pxsy}{m}{n}

\DeclareMathOperator{\interior}{int}								%Interno
\newcommand{\open}[1]{#1^\circ}							
\newcommand{\seq}[1]{\paren{#1}}								%Successione
\newcommand{\tseq}[1]{{\big(#1\big)}}
\newcommand{\ttseq}[1]{(#1)}
\newcommand{\net}[1]{\paren{#1}}								%Rete
\newcommand{\Cb}{\mcC_b}									%funzioni continue e limitate 
\newcommand{\Cc}{\mcC_c}									%funzioni continue e limitate 
\newcommand{\Cz}{\mcC_0}									%funzioni continue e evanescenti
\newcommand{\Ccompl}{\mcC_\infty}									%funzioni continue e evanescenti
\newcommand{\Cbs}{{\mcC_{bs}}}							%funzioni limitate e a supporto limitato
\newcommand{\Cinfty}{{\mathrm C^{\infty}}}						%funzioni C^infinito
\newcommand{\Cbinfty}{{\mcC_{b}^{\infty}}}
\newcommand{\Czinfty}{{\mcC_0^{\infty}}}

\newcommand{\Meas}{\mathscr M}
\newcommand{\Mp}{\mathscr M^+}
\newcommand{\Mb}{\mathscr M_b}
\newcommand{\Mbp}{\mathscr M_b^+}
\newcommand{\MbR}{\mathscr M_{\textrm{bR}}}
\newcommand{\pfwd}{\sharp}
\DeclareMathOperator*{\esssup}{esssup}
\DeclareMathOperator*{\essinf}{essinf}

\DeclareMathOperator{\car}{\mathbf 1}

\DeclareMathOperator{\emp}{\varnothing} %Known Sets, Fields and so on...
\newcommand{\N}{{\mathbb N}}
\newcommand{\EN}{\overline{\N}}
\newcommand{\R}{{\mathbb R}}
\DeclareMathOperator{\Rex}{{\overline{\mathbb R}}}
\DeclareMathOperator{\Q}{{\mathbb Q}}
%\newcommand{\C}{{\mathbb C}} 	%ATTENZIONE: si usa renew perchŽ \C  un comando nell'estensione "unicode" del pacchetto hyperref: togliendo l'estensione (o il pacchetto) si pu˜ usare newcommand semplice
\DeclareMathOperator{\Z}{{\mathbb Z}}

\newcommand{\LDS}{\textsc{lds}\xspace}
\newcommand{\TLDS}{\textsc{tlds}\xspace}
\newcommand{\MLDS}{\textsc{mlds}\xspace}
\newcommand{\EMLDS}{\textsc{mlds}\xspace}
\newcommand{\parEMLDS}{\textsc{mlds}\xspace}

\newcommand{\lb}{\mfl}
\newcommand{\llb}{\scriptstyle\lb}
\newcommand{\Lb}{\mathfrak{L}}

\newcommand{\restr}{\big\lvert}
\newcommand{\res}{\!\!\upharpoonleft\!\!}
%\newcommand{\mrestr}[1]{\,\raisebox{-.127ex}{\reflectbox{\rotatebox[origin=br]{-90}{$\lnot$}}}\, #1}
\newcommand{\mrestr}[1]{\!\downharpoonright_{#1}}
\newcommand{\trid}{{\star}}

\allowdisplaybreaks


\usetikzlibrary{shapes.misc}
\tikzset{cross/.style={cross out, draw=black, minimum size=2*(#1-\pgflinewidth), inner sep=0pt, outer sep=0pt},
%default radius will be 1pt. 
cross/.default={4pt}}

\newcommand{\iref}[1]{\ref{#1}}

\newcommand{\comma}{\,\,\mathrm{,}\;\,}
\newcommand{\semicolon}{\,\,\mathrm{;}\;\,}
\newcommand{\fstop}{\,\,\mathrm{.}}

\newcommand{\cdc}{\Gamma}
\newcommand{\LL}[2]{\mcL^{#1, #2}}
\newcommand{\TT}[2]{\mcT^{#1, #2}}
\newcommand{\hh}[2]{\mssh^{#1, #2}}
\newcommand{\TTt}{\mcT}

\newcommand{\GG}[2]{\mcG^{#1, #2}}
\newcommand{\EE}[2]{\mcE^{#1, #2}}
\newcommand{\FF}[2]{\mcI^{#1, #2}}
%\newcommand{SF}[2]{\cdc^{#1, #2}}
%\newcommand{LL}[2]{\mcL^{#1, #2}}
%\newcommand{hh}[2]{\mssh^{#1, #2}}
\newcommand{\SF}[2]{\cdc^{#1, #2}}
\newcommand{\EEe}{\mcE}

\renewcommand{\iint}{\int\!\!\!\!\int}
\newcommand{\bint}{-\!\!\!\!\!\!\int}

\DeclareMathOperator{\zero}{{\mathbf 0}}
\DeclareMathOperator{\uno}{{\mathbf 1}}
\newcommand{\imu}{\mathrm{i}}


\usepackage{scrextend}						%KOMA: confligge quindi deve stare qui
\newcommand{\note}[1]{\begin{addmargin}[8em]{1em}
{\color{blue} \footnotesize{#1}}
\end{addmargin}}

\newcommand{\purple}[1]{{\color{purple}#1}}
\newcommand{\blue}[1]{{\color{blue}#1}}
\newcommand{\green}[1]{{\color{OliveGreen}#1}}

\newcommand{\pnote}[1]{\begin{addmargin}[8em]{1em}
{\color{red} \footnotesize{#1}}
\end{addmargin}}

\newcommand{\Lin}{L} 						%Linear operators

\newcommand{\Radon}{\Mp_R^+}
\newcommand{\dRadon}{\Mp_{dR}^+}
\newcommand{\fsRadon}{\Mp_{fsR}^+}

\DeclareMathOperator{\inter}{int}
\newcommand{\Cyl}[1]{\mcF^\dUpsilon\mcC^\infty_b(#1)}
\newcommand{\CylQP}[2]{\mcF^\dUpsilon\mcC^{\infty}_b(#2)_{#1}}
%\newcommand{\ExtCyl}[1]{\mathrm{Cyl}^{\infty,\ext}_b\paren{#1}}
\newcommand{\CylCar}{\mcI}
\newcommand{\CylSet}[1]{\mcE^{\!\times\!}\mcC(#1)}
\newcommand{\CylSetUps}[1]{\mcE^{\!\dUpsilon\!}\mcC(#1)}

\newcommand{\Dz}{\mcD} %domain
\newcommand{\D}{\mcD} 
\newcommand{\Du}{\mcD_1} %extended domain by constant functions
\newcommand{\De}{\mcD_e} %AKR functions
\newcommand{\Daux}{\mcD_a} %AKR functions
\newcommand{\preW}{\msW^{1,2}_{\mathrm{pre}}} %pre-Sobolev

\newcommand{\Ex}[1]{\mathrm{Exp}^{\!\dUpsilon}(#1)}

\newcommand{\deq}{\overset{\mrmd}{=}}

\newcommand{\locfin}{\mathrm{lcf}}

\newcommand{\PP}{{\pi}}
\newcommand{\CP}{{\mu}}
\newcommand{\cpl}{q}
\newcommand{\QP}{{\mu}}

\newcommand{\hr}[1]{\bar\mssd_{#1}} 									%Hino-Ramirez d_A

\newcommand{\coK}[2]{\co_{#2}({#1})}
\newcommand{\cok}{\kappa}
\newcommand{\cokb}{\bar\kappa}
\newcommand{\coi}{\iota}

\newcommand{\e}{\varepsilon}
\newcommand{\dUpsilon}{{\mathbf \Upsilon}}

\newcommand{\empargop}{{\cdot}}



%---------Suzuki -------------------

\newcommand{\U}{\dUpsilon}
\newcommand{\sine}{\mathsf{sine}}
\newcommand{\Hess}{\mathsf{Hess}}
\newcommand{\E}{\mathcal E}
\newcommand{\F}{\mathcal F}
\renewcommand{\1}{\mathbf 1}
\newcommand{\p}{\pi}
\newcommand{\XX}{X}
\newcommand{\CylF}{{\rm Cyl}}
\renewcommand{\Leb}{\mathcal L}
\newcommand{\lab}{\mathfrak l}

\newcommand{\domm}{\dom{\EE{X}{\mssm}}}

\renewcommand{\msE}{\mathscr K}
\numberwithin{equation}{section}
\theoremstyle{plain}
\newtheorem{thm}{Theorem}[section]
\newtheorem*{thm*}{Theorem}
\newtheorem{thmA}{Theorem}
\newtheorem{corA}{Corollary}
\newtheorem*{mthm*}{Main Theorem}
\newtheorem*{thmUno}{Theorem \ref{t:Uno}}
\newtheorem*{thmDue}{Theorem \ref{t:Due}}
\newtheorem{prop}[thm]{Proposition}%[section]
\newtheorem{lem}[thm]{Lemma}%[section]
\newtheorem{cor}[thm]{Corollary}%[section]
\newtheorem*{cor*}{Corollary}

\theoremstyle{definition}
\newtheorem{defs}[thm]{Definition}%[section]
%[section]
\newtheorem*{defs*}{Definition}%[section]
\newtheorem{conj}[thm]{Conjecture}%[section]

\theoremstyle{remark}
\newtheorem{notat}[thm]{Notation}
\newtheorem{rem}[thm]{\bf Remark}%[section]
\newtheorem{ese}[thm]{\bf Example}%[section]
\newtheorem{ass}[thm]{\bf Assumption}%[section]
\newtheorem*{assA}{\bf Assumption}

\newtheorem{tassx}{Assumption}
\newenvironment{tass}[1]
 {\renewcommand\thetassx{#1}\tassx}
 {\endtassx}

\newcommand{\LS}{{\sf LS}}
\newcommand{\PI}{{\sf PI}}
\newcommand{\LH}{{\sf LH}}
\newcommand{\DFH}{{\sf H}_\infty}
\newcommand{\LP}{${\sf P}_{\loc}$}
\newcommand{\LLS}{{\sf LLS}}
\newcommand{\HWI}{{\sf HWI}}
\newcommand{\ENT}{{\rm Ent}}
\newcommand{\Fish}[2]{\ensuremath{{\rm I}}_{#1} (#2)}

\newcommand{\proj}{{\sf proj}}
\renewcommand{\paragraph}[1]{\medskip\emph{#1}.\quad}

\newcommand{\quot}{{\sf P}}
\newcommand{\CBE}{\mathrm{C}\beta\mathrm{E}}
%\newcommand{\ER}{\overset{\R}}
%\renewcommand{\mssm}{}

\makeatletter
\@namedef{subjclassname@2020}{%
  \textup{2020} Mathematics Subject Classification}
\makeatother

%----- TITLE and AUTHORS %-----
%\title{On the Ergodicity of Interacting Particle Systems}
%\author[]{}
%\author[E.~ Bru\'e]{Elia Bru\'e\thankssymb{1}}
%\author[K.~Suzuki]{Kohei Suzuki\thankssymb{2}}
%\address{Scuola Normale Superiore, Piazza dei Cavalieri 7, 56126 Pisa PI, Italy}
%\email{kohei.suzuki@sns.it}
%\thanks{\thankssymb{1}
 %\\ \indent\phantom{\thankssymb{1}}}
%\thanks{\thankssymb{4} \\ \indent\phantom{\thankssymb{4}} }
%\thanks{\thankssymb{2} Fakult\"at f\"ur Mathematik, Universit\"at Bielefeld, D-33501, Bielefeld, Germany \\ \indent\phantom{\thankssymb{2}} kohei.suzuki@sns.it}

%Metric Measure Spaces Under Riemannian Curvature--Dimension Conditions
  %\author{Kohei Suzuki \thanks{The author is supported by JSPS Overseas Research Fellowships Grant number 290142 and was partly supported by World Premier International Research Center Initiative (WPI), MEXT, Japan, and JSPS Grant-in-Aid for Scientific Research on Innovative Areas ``Discrete Geometric Analysis for Materials Design": Grant Number 17H06465..}}
  % \vspace{2mm} \\ {\it University of Bonn
%} \vspace{2mm}\\ {\it \small  Institute for Applied Mathematics} 
   %\\ {\it \small Endenicher Allee 60 D-53115 Bonn}
      %      \\ {\small E-mail: suzuki@iam.uni-bonn.de} \vspace{3mm} 
         %   }



%\subjclass{Primary 60F17; Secondary 53C23.}
%\keywords{Riemannian Curvature-Dimension Condition, Measured Gromov--Hausdorff Convergence, Non-symmetric Diffusions, Weak Convergence}

%\date{today}
\begin{document}
\title[Curvature bound of Dyson Brownian Motion]{Curvature Bound of Dyson Brownian Motion}

%\date{31/12/2022}
\author[K.~Suzuki]{Kohei Suzuki}
\address{Department of Mathematical Science, Durham University, Science Laboratories, South Road, DH1 3LE, United Kingdom}
%\address{Fakult\"at f\"ur Mathematik, Universit\"at Bielefeld, D-33501, Bielefeld, Germany}
%\email{ksuzuki@math.uni-bielefeld.de}
\thanks{\hspace{-5.5mm} Department of Mathematical Science, Durham University 
\\
\hspace{2.0mm} E-mail: kohei.suzuki@durham.ac.uk
}

\keywords{\vspace{2mm} Dyson Brownian motion, log-gas, Ricci curvature bound}

\subjclass[2020]{Primary 60K35, Secondary 31C25}



%\renewcommand{\abstractname}{\normalsize Abstract}
\begin{abstract} 
%{\fontsize{10.5}{12.0}\selectfont
We show that a differential structure associated with the infinite particle Dyson Brownian motion satisfies the Bakry--\'Emery nonnegative lower Ricci curvature bound $\BE(0, \infty)$. 
As a consequence, various functional inequalities of the transition semigroup follow such as a local spectral gap inequality,  a local log-Sobolev inequality, a local hyper-contractivity, the Lipschitz Feller property and the dimension-free Harnack inequality. 
%This provides applications to the particle dynamics such as a local spectral gap inequality,  a local log-Sobolev inequality, a local hyper-contractivity, the Lipschitz Feller property and the dimension-free Harnack inequality.  
%of a Dirichlet form on the configuration space whose invariant measure is $\mathsf{sine}_\beta$  ensemble for any $\beta>0$. As a particular case of $\beta=2$, our result proves $\BE(0, \infty)$ of a Dirichlet form corresponding to the  infinite particle Dyson Brownian motion. 
%As an application, we obtain the local Poincar\'e and the local log-Sobolev inequality, which induce a local spectral gap inequality for the infinite Dyson Brownian motion, and also obtain the local hyper-contractivity of the semigroup.  
%We prove, furthermore,  the coincidence of the aforementioned Dirichlet form with the Cheeger energy associated with the $L^2$-transportation distance. 
%We further explore interplays with the optimal transport theory, where we prove the Rademacher-type property,  the dimension-free Harnack inequalities and the $L^\infty$-to-Lipschitz Feller property of the semigroup with respect to~a transportation distance~$\bar{\mssd}_\U$.
%As a consequence,  the $1$-Wasserstein contraction property of the semigroup is obtained. 
At the end,  we provide a sufficient condition for $\BE(K, \infty)$ beyond the Dyson model and apply it to the infinite particle model of the $\beta$-Riesz gas. 
% as well as the evolution variational inequality $\EVI(0,\infty)$. As a consequence, we prove that the dual heat flow associated with the Dirichelt form coincides with the Wasserstein gradient flow of the entropy. We finally provide a sufficient condition for the synthetic lower Ricci curvature bound in the case of general invariant measures beyond $\sine_\beta$. 
%In the process, we introduce a concept, called {\it $\tau_\mssd$-upper regularity} for the topology $\tau_\mssd$ induced by an extended distance $\mssd$, which is a weaker concept than the $\tau$-upper regularity introduced in \cite{AmbGigSav15, AmbErbSav16}. In terms of $\tau_{\mssd}$-upper regularity, we characterise the coincidence of Cheeger energies and Dirichlet forms in the setting of general extended metric measure spaces, which could be an independent interest for analysis of Dirichlet forms in general infinite-dimensional spaces. 
%
%\par }
\end{abstract}

%\date{\today}
\maketitle

%\purple{{\bf List of Modification}
%\begin{enumerate}[(i)]
%\item P.9 {\sf Note}: There is a typo for the support of $\QP_r^\eta$
%\end{enumerate}
%}

%\tableofcontents

%\section{Curvature bound for the finite-particle case}
%Let $\mu$ be $\sine_\beta$. For $r<R$, $k \in \N$ and $\eta \in \U(B_r^c)$, define the following finite measure on $\U(B_r)$:
%\begin{align}
%&\diff \mu_{r, R}^{k, \eta}(\gamma) := \exp\Bigl(-  \Psi^{k, \eta}_{r, R}(\gamma) \Bigr) \diff \mssm^{\odot k} \comma
%\\
%& \Psi_{r, R}^{k, \eta}(\gamma) := - \log \Biggl(  \prod_{i \neq j}^k |x_i-x_j|^\beta \prod_{i=1}^k\prod_{y \in \eta_{B_r^c}, |y| \le R} |1-\frac{x_i}{y}|^\beta\Biggr) \fstop
%\end{align}
%We show that $\Psi_{r, R}^{k, \eta}$ is convex in $(\U(B_r), \mssd_{\U})$. 
%\begin{prop} \label{prop: conv}
%$\Psi_{r, R}^{k, \eta}$ is convex in $(\U^{k}(B_r), \mssd_{\U})$ for any $r<R$, $k \in \N$ and $\eta \in \U(B_r^c)$, 
%\end{prop}
%\proof
%Noting that any finite sum of convex functions is convex and that $\log (|x_i-x_j|)$ and $ \log|1-\frac{x_i}{y}|$ are convex functions in $\R^{\times k}$, the following expression concludes the convexity of $\Psi_{r, R}^{k, \eta}$ as a function on $\R^{\times k}$:
%\begin{align} \label{eq: conv}
%\Psi_{r, R}^{k, \eta}(\gamma) = -\beta\sum_{i \neq j}^k \log (|x_i-x_j|) - \beta\sum_{i=1}^k \sum_{y \in \eta_{B_r^c}, |y| \le R} \log|1-\frac{x_i}{y}| \comma
%\end{align}
%Since the convexity is preserved by the quotient map $\pr^A\colon \R^{\times k} \to \U^{k}(B_r)$ equipped with $\mssd_{\U}$, we conclude the statement. 
%%and , we conclude that the sum \eqref{eq: conv} is also convex 
%%We show the positivity of the Hessian of the potential $\Psi_{r, R}^{k, \eta}: B_r^{\times k} \to \R$. The partial derivatives are computed in the following:
%%\begin{align}
%%&\partial_{ii} \Psi_{r, R}^{k, \eta}= 2\beta \sum_{j \neq i}^k \frac{1}{|x_i-x_j|^2} + \beta\sum_{y \in \eta_{B_r^c}, |y| \le R} \frac{1}{|y-x_i|^2} \comma
%%\\
%%&\partial_{ij} \Psi_{r, R}^{k, \eta}= -2\beta \sum_{j \neq i}^k \frac{1}{|x_i-x_j|^2}  \fstop
%%\end{align}
%%For simplicity, we write $A_{ij}:=\frac{1}{|x_i-x_j|^2}$ and $B_i = \sum_{y \in \eta_{B_r^c}, |y| \le R} \frac{1}{|y-x_i|^2}$. 
%%\qed
\setcounter{tocdepth}{1}
\makeatletter
\def\l@subsection{\@tocline{2}{0pt}{2.5pc}{5pc}{}}
\def\l@subsubsection{\@tocline{3}{0pt}{4.75pc}{5pc}{}}
\makeatother

\tableofcontents
%\purple{ List of Modifications
%\begin{itemize}
%\item p2, p.4 Modifications suggested by the purple letters
%\item Application of Thm 6.2 to Riesz$_\beta$ and Airy$_\beta$. See the paper "Number-Rigidity and β-Circular Riesz gas" by David Dereudre and Thibaut Vasseur about the DLR equation for the Riesz gas and ask Maida about the recent progress of DLR equation for Airy$_\beta$.
%\end{itemize}
%}
%\purple{\underline{List of modifications}
%\\
%Complete Remark 4.19 and add the corresponding explanation in Introduction.}
\section{Introduction}
%\paragraph{Dyson Model}
The objective of this article is to reveal {\it a curvature bound} of a differential structure behind infinite particle systems with long-range interactions. 

\paragraph{Infinite Dyson Brownian motion}The interacting particle systems studied in this article can be formally described as the following infinitely  stochastic differential equation (see \cite{Osa12} for $\beta=1, 2, 4$ and \cite{Tsa16} for $\beta \ge 1$):
%There have been a large number of studies on rigorous mathematical constructions of interacting diffusion processes as diffusion processes in the infinite-dimensional space $\dUpsilon$, in particular, for the infinitely many interacting stochastic differential equations on $\R^n$, written `formally' as
\begin{align}  \label{d:DBMS}
\diff X_t^k=    \frac{\beta}{2} \lim_{r \to \infty}\sum_{i \neq k: |X_t^k- X_t^i|<r} \frac{1}{X_t^k- X_t^i} \diff t + \diff B^k_t, \quad k \in \N \comma 
\end{align}
 whereby $\{B_\cdot^k\}_{k\in \N}$ are infinitely many independent Brownian motions on~$\R$. When $\beta=2$, the solution to \eqref{d:DBMS} is called {\it infinite Dyson Brownian motion}, which has a particular importance in relation to random matrix theory (see \cite{Dys62, Spo87, NagFor98, KatTan10, Osa96, Osa13}). The infinite interacting diffusions~\eqref{d:DBMS} can be thought of as {\it a single} diffusion process on {\it the configuration spcae~$\U=\U(\R)$} over $\R$ (i.e., the space of locally finite point measures on~$\R$). This diffusion process on~$\U$ has an invariant measure~$\QP$, called $\sine_\beta$ ensemble (see \S~\ref{subsec:SB}), that is known as the universal limit of the eigenvalue distributions of Gaussian random matrices.
 
\paragraph{Differential structure of interacting particles}The diffusion process on~$\U$ corresponding to~the infinite Dyson SDE~\eqref{d:DBMS} induces a differential structure on~$\U$, called {\it a Dirichlet form}, that is,  a closed symmetric bilinear form~$(\E^{\U, \QP}, \dom{\E^{\U, \QP}})$ on~$L^2(\U, \QP)$ satisfying a maximum principle (called Markovian property). The form~$(\E^{\U, \QP}, \dom{\E^{\U, \QP}})$ is described as follows:
\begin{align*}%\label{e:DDFS}
\E^{\U, \QP}(u):=\int_{\U}\cdc^{\U}(u) \diff \QP \comma \quad  u \in \dom{\E^{\U, \QP}} \comma
\end{align*}
where $\cdc^\U$ corresponds to the (infinite-dimensional) squared gradient operator on $\U$ called {\it square field} (see Dfn.~\ref{d:DFF}).
% and~$\QP$ is the~$\sine_\beta$~ensemble. 
The transition probability of~the diffusion process is then associated with the $L^2$-semigroup~$\{T_t^{\U, \QP}\}_{t \ge 0}$ induced by the Dirichlet form~$(\E^{\U, \QP}, \dom{\E^{\U, \QP}})$.
% The main result reveals that the differential structure corresponding to what is called {\it infinite Dyson Brownian motion} has a uniform curvature lower bound in the sense of Bakry--\'Emery. This result brings many functional inequalities as byproducts and provide new quantitative estimates of the corresponding Dyson Brownian motion. At end of the article, we generalise this result to the case beyond the $\sine_\beta$ ensemble and we apply it to the $\beta$-Riesz ensemble. 
%In the following, we provide more detailed explanations and the precise statements of the main results. 
%\smallskip
%Interacting diffusions Dirichlet form is corresponding to the interacting diffusion in the sense that the $L^2$-semigroup of the Dirichlet form gives the transition probability of the interacting diffusions in a suitable sense. 
%Such an interacting particle system is realised as a continuous-time strong Markov process having continuous paths (called {\it a diffusion process}) taking values in {\it the configuration space}~$\U=\U(\R)$ over~$\R$ (i.e., the space of locally finite point measures on~$\R$). We study a differentiable structure corresponding to such an interacting particle system, called {\it a Dirichlet form}, and discuss its curvature bound in the sense of Bakry--\'Emery. 
%and having the~$\sine_\beta$ $(\beta>0)$ ensemble~$\QP$ as an invariant measure. 
% appears as an universal limit of eigenvalue distributions of Gaussian random matrices. 
%This Dirichlet form on~$\U$, denoted by $(\E^{\U, \QP}, \dom{\E^{\U, \QP}})$,  has the following form:
%\begin{align}\label{e:DDFS}
%\E^{\U, \QP}(u):=\int_{\U}\cdc^{\U}(u) \diff \QP \comma \quad  u \in \dom{\E^{\U, \QP}} \comma
%\end{align}
%where $\cdc^\U$ corresponds to the (infinite-dimensional) squared gradient operator on $\U$ called {\it square field} (see Dfn.~\ref{d:DFF}), and~$\QP$ is the~$\sine_\beta$~ensemble. 
%When $\beta=2$, the corresponding diffusion process on~$\U$ is called {\it infinite Dyson Brownian motion}, which has a particular importance in relation to random matrix theory (see \cite{Dys62, Spo87, NagFor98, KatTan10, Osa96, Osa13}).
%is called {\it infinite Dyson models} associated with~\eqref{e:DDFS} in the sense that the transition probability is given by the~$L^2(\U, \QP)$-semigroup corresponding to~\eqref{e:DDFS}. The case of $\beta=2$ particularly corresponds to what is called {\it infinite Dyson Brownian motion} that has a particular importance in relation to random matrix theory (see \cite{Dys62, Spo87, NagFor98, KatTan10, Osa96, Osa13}),   %, where the $L^2(\U, \QP)$-semigroup~$\sem{T_t^{\U, \QP}}$  associated with $(\E^{\U, \QP}, \dom{\E^{\U, \QP}})$ provides the transition semigroup of the diffusion process uniquely up to {\it quasi-every} starting points in the sense of the corresponding capacity.  
%The case of $\beta=2$ particularly corresponds to the diffusion process called {\it (unlabelled) Dyson Brownian motion} (cf.~\cite{Spo87, KatTan10, Osa13} noting that the core of the Dirichlet form chosen in the third reference was different from the one chosen in the current article). 
%The diffusion processes associated with~\eqref{e:DDFS} can be formally described as the following infinitely many stochastic differential equation with logarithmic interaction (see \cite{Osa12} for $\beta=1, 2, 4$ and \cite{Tsa16} for $\beta \ge 1$):
%%There have been a large number of studies on rigorous mathematical constructions of interacting diffusion processes as diffusion processes in the infinite-dimensional space $\dUpsilon$, in particular, for the infinitely many interacting stochastic differential equations on $\R^n$, written `formally' as
%\begin{align}  \label{d:DBMS}
%\diff X_t^k=    \frac{\beta}{2} \lim_{r \to \infty}\sum_{i \neq k: |X_t^k- X_t^i|<r} \frac{1}{X_t^k- X_t^i} \diff t + \diff B^k_t, \quad k \in \N \comma 
%\end{align}
% whereby $\{B_\cdot^k\}_{k\in \N}$ are infinitely many independent Brownian motions on~$\R$.  
 
 \smallskip
\paragraph{Bakry--\'Emery curvature bound}%Manifolds having a uniform lower Ricci curvature bound~${\rm Ric} \ge K$ have been a long-standing central object of Riemannian geometry as it has rich geometric and analytic quantitative controls such as diameter bounds, volume growth controls, heat kernel estimates and spectral estimates,  which also bring rich quantitative estimates of the corresponding diffusion process such as the recurrence and transience criteria, hitting time estimates, rates of the convergence to the equilibrium. 
In the seminal paper~Bakry--\'Emery~\cite{BakEme84}, they observed that a complete Riemannian manifold~$(M, g)$ has a Ricci curvature lower bound ${\rm Ric} \ge K$ for some $K \in \R$ if and only if the gradient estimate 
\begin{align}\label{D:BE}|\nabla T_tu|^2 \le e^{-Kt}T_t|\nabla u|^2 \tag{$\BE(K,\infty)$} \comma \quad u \in W^{1,2}(M)
\end{align}
 holds in terms of the heat semigroup $\{T_t\}_{t \ge 0}$, the gradient operator $\nabla$ and the $(1,2)$-Sobolev space $W^{1,2}(M)$~on~$M$. This observation opened a way to generalise the concept of lower Ricci curvature bound to singular spaces beyond manifolds such as metric measure spaces and infinite-dimensional spaces as the latter formulation~$\BE(K,\infty)$ requires only a weak (Sobolev) differentiable structure while the former formulation ${\rm Ric} \ge K$ requires {\it Ricci curvature tensors}, which is a $C^2$-structure. This generalised concept of the lower Ricci curvature bound turned out to be powerful and many functional inequalities and quantitative controls of geometry have been proven as a consequence of $\BE(0,\infty)$.
 %such as the Poinca\'e inequality (i.e., the spectral gap inequality), the log-Sobolev inequaliy, heat kernel estimates as well as quantitative controls of geometry such as volume growth estimates, diameter bounds and many others. 
 We refer the readers to e.g., \cite{BakGenLed14} and \cite{Vil09} for  comprehensive references. 

\paragraph{Main results}The main result of this article is to show that the differentiable structure $(\E^{\U, \QP}, \dom{\E^{\U, \QP}})$ induced by the infinite Dyson SDE~\eqref{d:DBMS} has the non-negative Ricci curvature bound~$``{\rm Ric} \ge 0"$ in the sense of Bakry--\'Emery.
 %which provides {\it a differential geometric understanding} of the infinite Dyson SDE~\eqref{d:DBMS}. 
% Let $\{T_t^{\U, \QP}\}_{t \ge 0}$ be the $L^2(\U, \QP)$-semigroup associated with $(\E^{\U, \QP}, \dom{\E^{\U, \QP}})$, which can be thought of as the transition semigroup associated with the infinite Dyson SDE~\eqref{d:DBMS}. 
%and it is identical to a geometric energy functional (called {\it Cheeger energy}) with respect to the $L^2$-transportation distance on the configuration space. 
% As an application, we obtain the $1$-Wasserstein contraction property of the dual heat semigroup~$\sem{\mathcal T_t^{\U, \QP}}$ on the space $\mathcal P_\QP(\U)$ of Borel probabilities absolutely continuous with respect to~$\QP$ (see Section~\ref{sec:LH}). 
 %The main results are summarised in the following.
\begin{thm}[Thm.~\ref{t: main}] \label{t:intromain}
Let $\beta>0$ and $\QP$ be the $\sine_\beta$ ensemble.  The form~$(\E^{\U, \QP}, \dom{\E^{\U, \QP}})$ satisfies the Bakry--\'Emery estimate $\BE(0,\infty)$. Namely, %~for $u \in \dom{\E^{\U, \QP}}$ and $t >0$,
\begin{align*}
%\vspace{-1mm}
\cdc^{\U}\bigl(T_t^{\U, \mu} u\bigr) \le T_t^{\U, \mu} \cdc^{\U}(u) \comma \quad u \in \dom{\E^{\U, \QP}}\quad t >0 \fstop
%\tag{$\BE_1(0,\infty)$}
 %\quad \forall t>0 \quad \forall u \in  \dom{\E^{\U, \mu}} \ ; %\tag{$\BE_1(K,\infty)$}
\end{align*} 
\end{thm}
%\noindent Consequently we obtain the integral Bochner inequality 
%$$\mathbf \Gamma_2(u, \phi) \ge 0\comma$$
%with respect to the integrated $\mathbf \Gamma_2$ operator with domain~$(u, \phi) \in \dom{\mathbf \cdc^{\U, \QP}_2}$ (see Cor.~\ref{t:LPS}). 


\paragraph{Applications to the Dyson SDE~\eqref{d:DBMS}}The Bakry--\'Emery lower curvature bound~$\BE(0,\infty)$ brings various functional inequalities as byproducts, which provide applications to the corresponding infinite Dyson SDE~\eqref{d:DBMS}.  We obtain the local Poincar\'e and the local log-Sobolev inequalities, which bring a {\it local} spectral gap estimate of the corresponding particle dynamics~\eqref{d:DBMS} (see Rem.~\ref{r:LSG} for more details). We also obtain a local hyper-contractivity of the semigroup, which gives a regularising effect of the semigroup regarding the integrability. 
% that decays sufficiently fast at the tail to make every (not necessarily bounded) $1$-Lipschitz function with respect to $\bar{\mssd}_\U$ exponentially integrable. 
 \begin{cor}[Cor.~\ref{t:LPS}] \label{c:1}
Let $\beta>0$ and $\QP$ be the $\sine_\beta$ ensemble.  Then, the form~$(\E^{\U, \QP}, \dom{\E^{\U, \QP}})$ satisfies the following:
\begin{itemize}%[$(a)$]
%\item{\rm (Thm.~\ref{t: main})} Bakry--\'Emery estimate $\BE(0,\infty)$:~for $u \in \dom{\E^{\U, \QP}}$ and $t >0$,
%\begin{align*}
%\vspace{-1mm}
%\cdc^{\U}\bigl(T_t^{\U, \mu} u\bigr) \le T_t^{\U, \mu} \cdc^{\U}(u) \ ;
%\tag{$\BE_1(0,\infty)$}
 %\quad \forall t>0 \quad \forall u \in  \dom{\E^{\U, \mu}} \ ; %\tag{$\BE_1(K,\infty)$}
%\end{align*} 
%\vspace{1mm}
%\item Integral Bochner inequality: for every $(u, \phi) \in \dom{\mathbf \cdc^{\U, \QP}_2}$
%\begin{align*}
%\mathbf \cdc^{\U, \QP}_2(u, \phi) \ge 0 \ ;
%\end{align*}
\item Local Poincar\'e inequality: for $u \in \dom{\E^{\U, \QP}}$, $t >0$,
\begin{align*}
&T^{\U, \QP}_tu^2- (T^{\U, \QP}_tu)^2 \le 2tT^{\U, \QP}_t\cdc^{\U}(u)  \comma %\tag{\LP(0)}
\\
&T^{\U, \QP}_tu^2- (T^{\U, \QP}_tu)^2 \ge 2t\cdc^{\U} (T^{\U, \QP}_tu) \ ;
\end{align*}
\item Local log-Sobolev inequality: for non-negative $u \in \dom{\E^{\U, \QP}}$, $t>0$,
\begin{align*}
&T^{\U, \QP}_tu\log u- T^{\U, \QP}_tu\log T^{\U, \QP}_t u \le t T^{\U, \QP}_t\biggl( \frac{\cdc^{\U}(u)}{u} \biggr) \comma
\\
&T^{\U, \QP}_tu\log u- T^{\U, \QP}_tu\log T^{\U, \QP}_t u \ge t \frac{\cdc^{\U}(T^{\U, \QP}_t u)}{T^{\U, \QP}_t u}  \fstop
\end{align*}
\item Local hyper-contractivity: for all $t>0$, $0<s\le t$ and $1<p<q<\infty$ with $\frac{q-1}{p-1}=\frac{t}{s}$, the following holds:
$$\Bigl( T_s^{\U, \QP}(T^{\U, \QP}_{t-s}u)^{q}\Bigr)^{1/q} \le \Bigl( T_t^{\U, \QP}u^p\Bigr)^{1/p} \comma$$
for all non-negative Borel functions $u$ on $\U$. 
\end{itemize}
\end{cor}
\smallskip

%\vspace{0.5mm}
%\end{itemize}
%In the case of $\beta=2$, the form~$(\E^{\U, \QP}, \dom{\E^{\U, \QP}})$ furthermore satisfies the following:
%\begin{itemize}%[$(a)$] \setcounter{enumi}{3}
\paragraph{Optimal transport and the Dyson Brownian motion}The configuration space~$\U$ has a metric structure lifted from the base space~$\R$, called the $L^2$-transportation distance~$\mssd_\U$ (or the Monge--Kantrovich--Rubinstein--Wasserstein distance), where the cost function is~the squared Euclidean distance~$\mssd_\R^2$ (see~\eqref{eq:d:W2Upsilon}). We use a variant $\bar\mssd_\U$ of $\mssd_\U$ called {\it the $L^2$-transportation-type distance} (see \eqref{eq:dW2L}). We prove that the metric~$\bar\mssd_\U$ has a consistency with the differential structure~$(\E^{\U, \QP}, \dom{\E^{\U, \QP}})$ induced by the infinite Dyson~SDE~\eqref{d:DBMS}. 

\begin{thm}[Prop.~\ref{p:DF}] \label{t:1.5}
Let $\beta>0$ and $\QP$ be the $\sine_\beta$ ensemble.  Then the Rademacher-type property holds:
$$\Lip_b(\bar\mssd_\U, \QP) \subset \dom{\E^{\U, \QP}} \comma \quad \cdc^{\U, \QP}(u) \le \Lip_{\bar\mssd_\U}(u)^2 \comma$$
where $\Lip_b(\bar\mssd_\U, \QP)$ denotes the space of bounded $\QP$-measurable $\bar\mssd_\U$-Lipschitz functions on~$\U$ and $\Lip_{\bar\mssd_\U}(u)$ denotes the $\bar\mssd_\U$--Lipschitz constant of $u$.
 \end{thm}

Combined with the local Poincar\'e inequality~in Cor.~\ref{c:1}, we have the following exponential decay estimate of the heat kernel measure~$P_t^{\U, \QP}(\gamma, \diff \eta)$ (i.e., the transition probability of~\eqref{d:DBMS}) in terms of~$\bar\mssd_\U$.
% which gives a non-trivial geometric quanatitative estimate of the transition probability of the infinite-Dyson SDE~\eqref{d:DBMS}.
\begin{cor}[Cor.~\ref{c:TES}] \label{c:3}
Let $\beta>0$ and $\QP$ be the $\sine_\beta$ ensemble.
If $u$ is a $\bar{\mssd}_\U$-Lipschitz $\QP$-measurable function with $\Lip_{\bar{\mssd}_\U}(u) \le 1$ and $|u(\gamma)|<\infty$ $\QP$-a.e.~$\gamma$, then for every $s<\sqrt{2/t}$
$$\int_{\U} e^{s u(\eta)} P_t^{\U, \QP}(\gamma, \diff \eta)<\infty \fstop$$
\end{cor}

\paragraph{Curvature bound in terms of the metric $\bar\mssd_\U$}In the case of Riemannian manifolds~$(M,g)$, the Ricci curvature lower bound ${\rm Ric} \ge K$ is known to be equivalent to the Wang's dimension-free Harnack inequality (\cite[Thm.~2.3.3]{Wan14}): for $\alpha>1$ and every bounded Borel function $u \ge 0$ on $M$
\begin{align*}%\label{D:DFHA}
(T_tu)^\alpha(x)\le T_tu^\alpha(y) \exp\Bigl\{ \frac{\alpha K}{2(\alpha-1)(1-e^{-2Kt})}\mssd_g(x, y)^2\Bigr\} \comma %\quad \text{$\forall x, y \in M$} \comma
\end{align*}
where $\mssd_g$ is the geodesic distance induced by $g$. This provides a characterisation of ${\rm Ric} \ge K$ in terms of the metric~$\mssd_g$. 
In the following theorem, we prove that the Wang's dimension-free Harnack inequality with $K=0$ holds true for the infinite Dyson SDE~\eqref{d:DBMS} with respect to the $L^2$-optimal transport-type distance~$\bar\mssd_\U$. We furthermore prove the log-Harnack inequality, and the Lipschitz contraction estimate by~$T_t^{\U, \QP}$, the latter of which can be understood as  a metric counter-part of~$\BE(0,\infty)$. As a byproduct, we obtain the Lipschitz regularisation property (Lipschitz-Feller property) of the semigroup~$T_t^{\U, \QP}$. 
%A natural question is whether the lower Ricci curvature bound~$\BE(0,\infty)$ has a metric 
%Recents studies~\cite{LzDSSuz21, LzDSSuz22, Suz22} revealed that the metric~$\mssd_\U$ is the canonical metric 
%\purple{Put the Rademacher in the theorem 1.3. Do not forget the notation before the statement policy when statements are moved}
%We furthermore prove the Rademacher-type theorem with respect to the $L^2$-transportantion type distance~$\bar{\mssd}_\U$  (see~\eqref{eq:dW2L}) in Prop.~\ref{p:DF}, which provides a consistency of the metric structure~$\bar\mssd_\U$ and the differentiable structure~$(\E^{\U, \QP}, \dom{\E^{\U, \QP}})$ in the sense that 
%$$\Lip_b(\U, \bar\mssd_\U, \QP) \subset \dom{\E^{\U, \QP}} \comma \quad \cdc^{\U, \QP}(u) \le \Lip_{\bar\mssd_\U}(u)^2 \comma$$
%where $\Lip_b(\U, \bar\mssd_\U, \QP)$ denotes the space of bounded $\QP$-measurable $\bar\mssd_\U$-Lipschitz functions and $\Lip_{\bar\mssd_\U}(u)$ denotes the $\bar\mssd_\U$--Lipschitz constant of $u$.
%Combined with the local Poincar\'e inequality, we obtain an exponential decay estimate of~the heat kernel measure $P_t^{\U, \QP}(\gamma, \diff \eta)$, i.e., the transition probability of the Dyson SDE~\eqref{d:DBMS}. 
%\purple{Explain the optimal transport distance more}We explore further relations between the constructed diffusion structure~$(\E^{\U, \QP}, \dom{\E^{\U, \QP}})$ and the metric geometry on the infinite-dimensional space~$\U$ arisen from the $L^2$-transportation-type distance~$\bar{\mssd}_\U$. We prove geometric analytic interplays between the semigroup $\{T_t^{\U, \QP}\}_{t \ge 0}$ and the distance~$\bar{\mssd}_\U$.
%: the dimension-free Harnack inequality; the log-Harnack inequality, the Lipschitz contraction; the $L^\infty$-to-Lipschitz regularisation property with respect to~$\bar{\mssd}_\U$. 
\begin{thm}[Thm.~\ref{t:DFH}] \label{t:3}
Let $\beta>0$ and $\QP$ be the $\sine_\beta$ ensemble.  Then, the following hold:
\begin{itemize}
\item Dimension-free Harnack inequality: for every non-negative $u \in L^\infty(\U, \QP)$, $t>0$ and $\alpha>1$ there exists $\Omega \subset \U$ so that $\QP(\Omega)=1$ and 
$$(T^{\U, \QP}_tu)^\alpha(\gamma)\le T^{\U, \QP}_tu^\alpha(\eta) \exp\Bigl\{ \frac{\alpha }{2(\alpha-1)}\bar{\mssd}_\U(\gamma, \eta)^2\Bigr\} \comma \quad \text{$\forall \gamma, \eta \in \Omega$} \ ;$$

\item Log-Harnack inequality: for any non-negative $u \in L^\infty(\U, \QP)$, $\e \in (0, 1]$, $t>0$, there exists $\Omega \subset \U$ so that $\QP(\Omega)=1$ and 
$$T^{\U, \QP}_t\log (u+\e)(\gamma) \le \log (T^{\U, \QP}_tu(\eta) +\e)+ \bar{\mssd}_\U(\gamma, \eta)^2\comma \quad \text{$\forall \gamma, \eta \in \Omega$} \ ;$$
\item  Lipschitz contraction: for every $u \in \Lip(\bar{\mssd}_\U, \QP)$ and $t>0$
\begin{align*}
\text{$T_t^{\U, \QP}u$ has a $\bar{\mssd}_\U$-Lipschitz $\QP$-modification~$\tilde{T}_t^{\U, \QP}u$ }
\end{align*}
and the following Lipschitz contraction holds:
$$\Lip_{\bar{\mssd}_\U}(\tilde{T}_t^{\U, \QP}u) \le \Lip_{\bar{\mssd}_\U}(u) \ ;$$
\item  {$L^\infty(\QP)$-to-$\Lip(\bar{\mssd}_\U, \QP)$ Feller property}: 
For $u \in L^\infty(\QP)$ and $t>0$, 
\begin{align*}
\text{$T_t^{\U, \QP}u$ has a $\bar{\mssd}_\U$-Lipschitz $\QP$-modification~$\tilde{T}_t^{\U, \QP}u$ }
\end{align*}
and the following estimate holds:
\begin{align*}% \label{m:LL}
%&T_t^{\U, \QP} L^\infty(\U, \QP) \subset \Lip_b(\bar{\mssd}_\U, \QP) \quad \forall t>0\comma
%\\
&\Lip_{\bar{\mssd}_\U}(\tilde{T}_t^{\U, \QP} u) \le \frac{1}{\sqrt{2} t} \|u\|_{L^\infty(\QP)}  \fstop
\end{align*}
%\item{\rm (Cor.~\ref{c:MU})} The form $(\E^{\U, \QP}, \Lip_b(\mssd_\U, \QP))$ is Markov unique$;$\footnote{This is wrong}
%, i.e., there exists at most one Dirichlet form extending~$(\E^{\U, \QP}, \Lip(\mssd_\U, \QP))$$;$
%\vspace{2mm}

%\item{\rm (Cor.~\ref{c:1WC})} $1$-Wasserstein contraction: for any $\nu, \sigma \in \mathcal P_\QP(\U)$ and any $t>0$, 
%\begin{align*} %\label{e:1WC}
%W_1(\mathcal T_t^{\U, \QP}\nu, \mathcal T_t^{\U, \QP}\sigma) \le W_1(\nu, \sigma) \fstop
%\end{align*} 
\end{itemize}
 \end{thm}
% We prove, furthermore,  several related functional inequalities including the integral Bochner inequality with respect to the integrated $\mathbf \Gamma_2$ operator $(\mathbf \cdc^{\U, \QP}_2, \dom{\mathbf \cdc^{\U, \QP}_2})$ (see \eqref{d:IG}), the local Poincar\'e, the local log-Sobolev inequalities as well as the dimension-free Harnack inequality, the log-Harnack inequality, the Lipschitz contraction and the $L^\infty$-to-Lipschitz regularisation property with respect to the $L^2$-transportation-type extended distance~$\bar{\mssd}_\U$ on~$\U$ (see \eqref{eq:dW2L}). 
%as well as the evolutional variation inequality $\EVI(0,\infty)$. As a consequence, we prove that the dual flow on the space of Borel probabilities associated with the Dirichlet form coincides with the Wasserstein gradient flow of the entropy. 
\paragraph{Generalisation beyond $\sine_\beta$}At the end of this article, our results will be extended to general point processes beyond the $\sine_\beta$ ensemble, see Thm.~\ref{t:GT}. We apply Thm.~\ref{t:GT} to prove $\BE(0,\infty)$ of the Dirichlet form~$(\E^{\U, \QP}, \dom{\E^{\U, \QP}})$ with $\beta$-Riesz emsemble~$\QP=\QP_\beta$ for $\beta>0$. %\purple{formal SDE for Riesz ensemble}
\begin{cor}[Cor.~\ref{c:BRE}]
The Dirichlet form~$(\E^{\U, \QP}, \dom{\E^{\U, \QP}})$ with the $\beta$-Riesz ensemble $\QP$ satisfies $\BE(0,\infty)$ for $\beta>0$. Furthermore, all the statements in Thm.~\ref{t:intromain}, Cors.~\ref{c:1}, \ref{c:3}, Thm.s~\ref{t:1.5}, ~\ref{t:3} hold true in this case. 
\end{cor}
%To our best knowledge, this is the first article to show the synthetic lower Ricci curvature bound in the case of {\it interacting} infinite particle systems (see, \cite{ErbHue15, LzDSSuz22} in the case of {\it free} particle systems).
% In the process of proving the aforementioned results, we introduce the concept of $\tau_\mssd$-upper regularity for general extended metric measure spaces, which is a weaker notion than $\tau$-upper regularity introduced in \cite{AmbGigSav15, AmbErbSav16}. This concept plays a significant role to prove the identification of given Dirichlet forms with the Cheeger energy and could be an independent interest for analysis of Dirichlet forms in general infinite-dimensional settings. 
%In the remainder of this introduction, we provide more detailed results regarding the aforementioned statements. 
%The ergodicity of interacting particle systems is one of the most significant hypothesis supporting the foundation of statistical physics. In this paper, we provide a new characterisation of the ergodicity of infinite interacting particle systems in terms of the optimal transportation theory and of the theory of point processes. As an application, we prove the ergodicity of particle systems with long-range interaction potentials including $\mathrm{sine}_{2}$, $\mathrm{Airy}_{2}$, $\mathrm{Bessel}_{\alpha, 2}$ ($\alpha \ge 1$), and $\mathrm{Ginibre}$ point processes.

%\smallskip
%To keep the presentation in this introduction simple, we state our results here only in the case of~$X=\R^n$ equipped with the standard Euclidean topology $\T$ and the distance~$\mssd$, the square field operator $\Gamma(\cdot)=|\nabla \cdot|^2$ associated with the standard gradient operator $\nabla$, the Borel $\sigma$-algebra $\Sigma=\mathscr B_\T$ generated by $\T$, the Lebesgue measure $\mssm$, the family $\msE$ of all relatively compact sets in $\R^n$, and we always choose the exhaustion~$E_h=\bar{B}_{r_h}^{d}(\mathbf 0)$ of closed balls centred at $\mathbf 0$ with radii~$r_n \uparrow \infty$ as localising sets in the following arguments. We will present, however, our result in a far more general setting in \S \ref{sec: Pre}, in which we will not assume the local compactness nor the Polish property for $X$. 

%\purple{Define, $\U$, $\sine_\beta$, $\mssd_\U$, $\BE$, functional inequalities in Introduction}
%\subsection*{Configuration spaces}%For the sake of simplicity, throughout this introduction,  let~$(X,\mssd)$ be a proper complete and separable metric space. We will present, however,  a more general setting in \S \ref{sec: Pre}, in which we will not assume the local compactness nor the Polishness. 
%The configuration space $\dUpsilon$ over~a locally compact Polish space $X$ is the set of all locally finite point measures on $X$:
%\begin{equation*} %\label{config}
%\U(X):=\dUpsilon\eqdef \set{\gamma=\sum_{i=1}^N \delta_{x_i}: x_i\in X\comma N \in \N\cup \set{+\infty}\comma \gamma K<\infty \quad K \Subset X }\fstop
%\end{equation*}
%The space~$\dUpsilon$ is endowed with the \emph{vague topology}~$\T_\mrmv$, induced by the duality with continuous compactly supported functions on~$X$. Throughout this article except the last section, we work with the base space $X$ being the one-dimensional Euclidean space $X=\R$, and merely in the last section, we discuss the $n$-dimensional Euclidean space $X=\R^n$. We, therefore, simply write $\U$ without the symbol of the base space $X$ unless the base space is necessary to be specified. 
% %and  and with a reference Borel probability measure~$\QP$, occasionally understood as the law of a proper point process on~$X$. 
%%Throughout this paper, we always assume that the intensity measure~$\mssm_\QP$ of $\QP$ (see \eqref{eq: int}) is finite on any compact sets (Assumption~\ref{ass:Mmu}). 
%%By means of dropping labels of particles, 
%%A system of infinitely interacting diffusion processes on the base space $X$ can be thought of as {\it a single diffusion process} on $\dUpsilon$, i.e., a continuous-time Markov process having $\T_\mrmv$-continuous trajectories on $\dUpsilon$, provided the system of evolving particles does not condense too much by itself in the sense that any compact set in $X$ contains only finitely many particles throughout the time evolutions. 
%
%
%
%\subsection*{$\sine_\beta$ ensemble} 
%In the case of $\beta=2$, the $\sine_2$ ensemble is a probability measure on $\U(\R)$ whose $k$-point correlation function at $x_1, x_2, \ldots, x_k$ is identical to the determinant $\det(\mathsf s(x_i, x_j)_{1 \le i, j \le k})$ of the sine kernel $\mathsf s(x, y)$:
%$$\mathsf s(x, y):=\frac{\sin(\pi(x-y))}{\pi(x-y)} \fstop$$
%It has been extensively studied in the context of random matrix theory and the $\sine_2$ ensemble appears  as the scaling limit of the point process associated with the eigenangles of a Haar-distributed unitary matrix, called the circular unitary ensemble. It is also known that the $\sine_2$ ensemble universally appears as scaling limit of eigenvalue distributions of various random matrices including the Gaussian unitary ensemble. In the case of $\beta=1, 2, 4$, $\sine_\beta$ ensembles are well understood within the framework of determinantal/Pfaffian structures, see e.g.,~\cite{DeiGio09} and references therein. For general $\beta>0$, it has been constructed in~\cite{KilSto09} through a scaling limit of circular~$\beta$ ensemble as well as in~\cite{ValVir09} by a  scaling limit of Gaussian $\beta$-ensemble.  By a recent progress in \cite[Thm.1.1]{DerHarLebMai20}, it turned out that the $\sine_\beta$ ensemble $\mu$ satisfies the following DLR (Dobrushin--Lanford--Ruelle) equation:
%\begin{align} \label{1}
%\diff\mu_r^\eta&=\frac{1}{Z_{r}^{\eta}} \sum_{k=0}^\infty \mu_{r, \eta}^k=\frac{1}{Z_{r}^{\eta}} \sum_{k=0}^\infty \Psi_{r}^{k, \eta} \diff \mssm^{\odot k} \comma \tag{$\star$}
%\\ 
% \Psi_{r}^{k, \eta}(\gamma) &:= - \log \Biggl(  \prod_{i \neq j}^k |x_i-x_j|^\beta \prod_{i=1}^k\lim_{R \to \infty}\prod_{y \in \eta_{B_r^c}, |y| \le R} |1-\frac{x_i}{y}| \Biggr) \comma \notag
%\end{align}
%where $\mu_{r}^\eta$ is the regular conditional probability of $\QP$ conditioned at $\eta \in \U$ in $B_r^c$, $\mssm_r^{\odot k}$ is the $k$-symmetric product measure of the Lebesgue measure $\mssm_r$ restricted on the metric ball $B_r \subset \R$ with radius $r>0$ centred at $0 \in \R$ (or it is equivalent up to constant multiplication to the Poisson measure restricted on the $k$-particle configuration space $\U^k(B_r)$) and $Z_{r}^{\eta}$ is the normalising constant.
%%A probability measure $\mu$ on $\dUpsilon$ is said to be {\it tail trivial} if $\mu(A) \in \{0, 1\}$ whenever $A$ is a set in the tail $\sigma$-algebra (see Definition~\ref{defn: TT}). 
%%The tail-triviality has been originally discussed in relation to {\it phase transition} of Gibbs states (i.e., non-uniqueness of Gibbs measures with a given potential) and  it is equivalent to the {\it extremality} in the convex set of Gibbs measures with a given potential (see \cite[Cor.\ 7.4]{Geo11}). The tail-triviality has been extended also for determinantal/permanental point processes by \cite{Lyo03} and \cite{ShiTak03b} independently. Since then, it has been further developed for a wider class of determinantal/permantental processes both in the continuous and discrete settings by various studies, see Example~\ref{exa: TT}. 
%%
%%A probability measure $\QP$ on $\dUpsilon$ is said to be {\it rigid in number} (Assumption~\ref{ass:Rig}) if, for any $E \in \msE$, the configuration outside $E$ (denoted by $\gamma_{E^c}$) determines the number of particles inside $E$ (denoted by $\gamma E$) almost surely. More rigorously,  for any $E \in \msE$, there exists a $\QP$-measurable set $\Omega \subset \dUpsilon$ so that $\QP(\Omega)=1$ and, for any $\gamma, \eta \in \Omega$, 
%%\begin{align*}
%%\text{$\gamma|_{E^c} = \eta|_{E^c}$ implies $\gamma E = \eta E$} \fstop
%%\end{align*}
%%This strong spatial correlation phenomenon has been studied by \cite{Gho12,  Gho15, GhoPer17} in the case of $\mathrm{sine}_2$, $\mathrm{Gnibre}$ and $\mathrm{GAF}$ point processes, and applied to solve the spanning problem that the closed linear span of a certain class of exponential functions generated by point processes coincides with the $L^2(X, \mssm)$ space in \cite{Gho15}. Since these pioneering works, the rigidity in number has been verified for a variety of point processes, see Example~\ref{exa: R}. 
%
%
%\subsection*{Dirichlet forms and particle systems}A closed Markovian symmetric bilinear form on $L^2(\U, \QP)$ is called {\it symmetric Dirichlet form}. When a symmetric Dirichlet form possesses {\it locality}  and {\it $\tau_\mrmv$-quasi-regularity}, the general theory of Dirichlet form (\cite[e.g., Thm. IV.3.5 \& Thm.~V.1.5]{MaRoe90}]) provides the corresponding diffusion process taking values in the configuration space $\U$ with invariant measure $\mu$ in such a way that the $L^2(\U, \QP)$-semigroup associated with the Dirichlet form provides the transition semigroup of the diffusion process uniquely up to negligible set (with respect to $(1,2)$-capacity) regarding starting points. 
%
%A strongly local $\tau_\mrmv$-quasi-regular symmetric Dirichlet form~$\E^{\U, \QP}$ on~$L^2(\U, \QP)$ with invariant measure $\QP$ can be constructed by lifting Dirichlet forms on $\U(B_r)$ onto Dirichlet forms on $\U$. To be more precise (see \S\ref{sec:CI} for details): 
%\begin{enumerate*}[{\rm (i)}]
%\item we first construct the Dirichlet form $\E^{\U(B_r), \mu_{r}^\eta}$ whose invariant measure is the regular conditional probability $\mu_{r}^\eta$; 
%\item we second construct the Dirichlet form $\E_r^{\U, \mu}$ on $\U$ by {\it superposition} of $\E^{\U(B_r), \mu_{r}^\eta}$ with respect the conditioning $\eta$; 
%\item we thirdly show the monotonicity of the Dirichlet forms $\E_r^{\U, \mu}$ as $r\to \infty$, and construct the Dirichlet form $\E^{\U, \mu}$ on $\U$ as the monotone limit form. 
%\end{enumerate*}
%%\begin{enumerate}[{\rm (i)}]
%%\item we first construct the Dirichlet form 
%%$$\E^{\U(B_r), \mu_{r, \eta}}(u)=\int_{\U(B_r)} \cdc^{\U(B_r),  \mu_{r, \eta}}(u) \diff \mu_{r, \eta} \comma$$
%%  on the configuration space $\U(B_r)$, where $ \cdc^{\U(B_r),  \mu_{r, \eta}}$ is the square field operator on $\U(B_r)$ induced by the square field $|\nabla \cdot|^2$ on $\R$ through symmetric tensorisation; 
%%\item we second construct the Dirichlet form $\E_r^{\U, \mu}$ on $\U$ by {\it superposition} of $\E^{\U(B_r), \mu_{r, \eta}}$ with respect the conditioning $\eta$;
%%\item  we thirdly show the monotonicity of the Dirichlet forms $\E_r^{\U, \mu}$ as $r\to \infty$, and construct the Dirichlet form $\E^{\U, \mu}$ on $\U$ as the monotone limit form. 
%%\end{enumerate}
%The limit Dirichlet form $\E^{\U, \QP}$ admits the square field operator $\cdc^\U$ satisfying 
%$$\E^{\U, \QP}(u)=\int_{\U} \cdc^{\U}(u) \diff \QP \fstop$$
%The square field operator $\cdc^{\U}$ plays a role of differentiable structure on the configuration space $\U$. We call the triplet $(\U, \cdc^{\U}, \QP)$ {\it diffusion space} associated with the square field $\cdc^{\U}$ and the reference measure $\QP$.
%
%According to the DLR equation \eqref{1}, the  diffusion process on $\U$ corresponding to $(\E^{\U, \QP}, \dom{\E^{\U, \QP}})$ can be understood {\it heuristically} as an unlabelled solution to the following infinitely many stochastic differential equation with logarithmic interaction (see,~\cite{Tsa16} for the rigorous construction):
%%There have been a large number of studies on rigorous mathematical constructions of interacting diffusion processes as diffusion processes in the infinite-dimensional space $\dUpsilon$, in particular, for the infinitely many interacting stochastic differential equations on $\R^n$, written `formally' as
%\begin{align*} 
%\diff X_t^k=  -  \frac{\beta}{2} \lim_{r \to \infty}\sum_{i \neq k: |X_t^k- X_t^i|<r} \frac{1}{X_t^k- X_t^i} \diff t + \diff B^k_t, \quad k \in \N \comma 
%\end{align*}
% whereby $\{B^k\}_{k\in \N}$ are independent Brownian motions on~$\R$. This correspondence will be, however, never used in the present article. 
%
%\subsection*{Bakry--\'Emery curvature dimension condition}
%Let $\{T_t^{\U, \QP}\}_{ t \ge 0}$ be the semigroup corresponding to $(\E^{\U, \QP}, \dom{\E^{\U,\QP}})$. The first main result of this article is the Bakry--\'Emery curvature dimension condition for the diffusion space $(\U, \cdc^{\U}, \QP)$.
%%\begin{thmA} \label{t: main}
%%Let $\beta>0$ and $\mu$ be the $\sine_\beta$ ensemble. The diffusion space $(\U, \cdc^{\U}, \QP)$ satisfies the non-negative $1$-Bakry--\'Emery curvature dimension condition $\BE_1(0,\infty)$:
%%\begin{align} \label{m:BE}
%%\cdc^{\U}\bigl(T_t^{\U, \mu} u\bigr)^{\frac{1}{2}} \le T_t^{\U, \mu} \bigl(\cdc^{\U}(u)^{\frac{1}{2}}\bigr) \quad \forall u \in  \dom{\E^{\U, \mu}} \quad \forall t>0\fstop \tag{$\BE_1(0,\infty)$}
%%\end{align}
%%\end{thmA}
%By a simple application of H\"older inequality, $\BE_1(0,\infty)$ can be extended to the following $\BE_p(0,\infty)$ for every $p \in [1, \infty)$:
%\begin{align} \label{m:BEp}
%\cdc^{\U}\bigl(T_t^{\U, \mu} u\bigr)^{\frac{p}{2}} \le T_t^{\U, \mu} \bigl(\cdc^{\U}(u)^{\frac{p}{2}}\bigr) \quad \forall u \in  \dom{\E^{\U, \mu}} \quad \forall t>0\fstop \tag{$\BE_p(0,\infty)$}
%\end{align}
%
%The strategy of the proof of Thm.~\ref{t: main} is to lift $\BE_1(0,\infty)$ condition from the space of finitely many configurations to that of infinitely-many configurations. For so doing, we use the stability of the curvature lower bound under some convergences of reference measures and Dirichlet forms as well as the stability under the superposition of Dirichlet forms.  A key step for the former stability is the approximation of the regular conditional probability $\mu_r^\eta$  in \eqref{1} by the intermediate measures $\mu_{r, R}^\eta$ having {\it convex} densities as is shown in \cite{DerHarLebMai20}.
%%\begin{align} \label{2}
%%\diff\mu_{r, R}^\eta&= \sum_{k=0}^\infty \mu_{r, R}^{k, \eta}= \sum_{k=0}^\infty \Psi_{r, R}^{k, \eta} \diff \mssm^{\odot k} \comma \tag{$\star\star$}
%%\\ 
%% \Psi_{r, R}^{k, \eta}(\gamma) &:= - \log \Biggl(  \prod_{i \neq j}^k |x_i-x_j|^\beta \prod_{i=1}^k \prod_{y \in \eta_{B_r^c}, |y| \le R} |1-\frac{x_i}{y}| \Biggr) \fstop \notag
%%\end{align}
%A key step for the latter stability is to establish {\it the Markov uniqueness} of the superposed Dirichlet forms on a certain core (see, Sections \ref{sec: Pre},~\ref{sec:CI} for more details).
%
%%the following:
%%\begin{enumerate}[{\rm (i)}]
%%\item 
%%We prove $\BE(0,\infty)$ for the intermediate form $\E^{\U(B_r), \mu^{R}_{r, \eta}}$ with the invariant measure $\mu_{r, R}^\eta$ defined as \purple{Use a consistent notation for super-/sub-scripts}
%%\begin{align} \label{1}
%%\diff\mu_{r, R}^\eta&= \sum_{k=0}^\infty \mu_{r, R}^{k, \eta}= \sum_{k=0}^\infty \Psi_{r, R}^{k, \eta} \diff \mssm^{\odot k} \comma \tag{$\star\star$}
%%\\ 
%% \Psi_{r, R}^{k, \eta}(\gamma) &:= - \log \Biggl(  \prod_{i \neq j}^k |x_i-x_j|^\beta \prod_{i=1}^k \prod_{y \in \eta_{B_r^c}, |y| \le R} |1-\frac{x_i}{y}| \Biggr) \fstop \notag
%%\end{align}
%%In this case, the curvature bound $\BE(0,\infty)$ is a consequence of the convexity of the density $\Psi_{r, R}^{k, \eta}$ on $\U^k(B_r)$;
%%\item As the density $\Psi_{r, R}^{k, \eta}$ converges uniformly to $\Psi_{r}^{k, \eta}$ as $R \to \infty$ up to multiple normalisation constants as proved in \cite{DerHarLebMai20}, $\BE(0,\infty)$ is inherited to the limit form $\E^{\U(B_r), \mu_{r, \eta}}$; 
%%\item the superposed form $\E^{\U, \mu}_r$ satisfies $\BE(0,\infty)$, for which we prove {\it the Markov uniqueness} of the superposed form $\E^{\U, \mu}_r$ on a certain core;
%%\item the form $\E^{\U, \mu}$ satisfies $\BE(0,\infty)$ as it is the monotone limit of $\E^{\U, \mu}_r$ as $r \to \infty$.
%%\end{enumerate}
%% A key observation for the latter works is that the closability of the pre-Dirichlet form~$\EE{\dUpsilon}{\QP}$ on a certain class of test functions can be obtained by the closability of the {\it conditioned form}~$\EE{\dUpsilon(E_h)}{\QP^{\eta}_{E_h}}$ with an exhaustion of bounded open sets $E_h \uparrow X$, where the conditioned form is constructed from the square field $\cdc^{\dUpsilon(E_h), \QP^{\eta}_{E_h}}$ lifted from the one on $E_h$ and the reference measure  $\QP^{\eta}_{E_h}$, which is the probability measure on $\dUpsilon(E_h)$ constructued by conditioning $\QP$ at a given configuration $\eta$ outside $E_h$ (see \eqref{eq:VariousFormsB}). To be more precise,  provided (a)  $\QP^{\eta}_{E_h}$ is equivalent to the Poisson measure $\PP_{E_h}$ on $E_h$ for $\QP^{k, E}$-a.e.\ $\eta$ ({\it Conditional Equivalence}~\ref{ass:CE}), where $\QP^{k, E}$ is the restriction of $\QP$ on the set $\{\gamma \in \dUpsilon: \gamma E=k\}$;  (b) the conditioned form 
%% $$\EE{\dUpsilon(E_h)}{\QP^{\eta}_{E_h}}(u) = \int_{\dUpsilon(E_h)} \cdc^{\dUpsilon, \QP^{\eta}_{E_h}} (u) \diff\QP^{\eta}_{E_h}$$ is closable on {\it cylinder functions} (see \eqref{defn: cyl}) on $\dUpsilon(E_h)$ for $\QP$-a.e.\ $\eta$ ({\it Conditional Closability}~\ref{ass:ConditionalClos}), we can construct the square field operator $\cdc^{\dUpsilon}$ on $\dUpsilon$ lifted from the square field $\cdc$ on the base space $X$ (see \eqref{eq:d:LiftCdCRep}) and the corresponding form 
%% $$\EE{\dUpsilon}{\QP}(u)=\int_{\dUpsilon} \cdc^{\dUpsilon, \QP} (u) \diff \QP$$ is closable on the space of cylinder functions. The closure is denoted by $\ttonde{\EE{\dUpsilon}{\QP},\dom{\EE{\dUpsilon}{\QP}}}$, which will be recalled in Theorem~\ref{t:ClosabilitySecond}. 
%% %Under these assumptions, there exists a quasi-regular Dirichlet forms $\ttonde{\EE{\dUpsilon}{\QP},\dom{\EE{\dUpsilon}{\QP}}}$ on $\dUpsilon$ by \cite[Thm.~3.48]{LzDSSuz21}, which will be recalled in Theorem~\ref{t:ClosabilitySecond}. 
%% This observation makes the problems of infinite particle systems come down to {\it essentially finite-dimensional analysis} as the conditioned form~$\EE{\dUpsilon(E_h)}{\QP^{\eta}_{E_h}}$ corresponds to finite particle systems on the bounded set $E_h$ conditioned at $\eta_{E_h^c}$ outside $E_h$.
% %, in which one can use well-developed tools.
%
%
%%the $L^2(\QP)$-propagator $\{T^{\dUpsilon, \QP}_t\}$ of the unlabelled weak solutions to \eqref{1}, i.e.,
%
%
%% (with a slightly different definition of the ergodicity from the one defined above).
%
%\subsection*{Optimal transport distance on $\dUpsilon$}The configuration space~$\dUpsilon$ is equipped with the \emph{$L^2$-transportation} (also: \emph{$L^2$-Wasserstein}, or $L^2$-{\it Monge--Kantrovich--Rubinstein}) {distance}
%\begin{align*}%\label{eq:Intro:WassersteinD}
%\mssd_\dUpsilon(\gamma, \eta) \eqdef \inf \tonde{\int_{X^\tym{2}} |x-y|^2\diff\cpl(x,y)}^{1/2}\comma 
%\end{align*}
%where the infimum is taken over all measures~$\cpl$ on~$X^\tym{2}$ with marginals~$\gamma$ and~$\eta$ and $|x-y|$ be the standard Euclidean distance on $X=\R^n$.
%As opposed to the case of the space of probability measures having finite second moment (i.e., the $L^2$-Wasserstein space), the function $\mssd_{\dUpsilon}$ cannot be a distance function because $\mssd_\dUpsilon$ may attain the value~$+\infty$ very `often', in the sense that this occurs on sets of positive measures for any reasonable choice of reference measures on $\dUpsilon$. It is, therefore, called \emph{extended} distance. This lack of finiteness causes various pathological phenomena (see \cite{LzDSSuz21}): 
%%the extended distance $\mssd_\U$ does not induce the vague topology $\tau_\mrmv$; $\mssd_\U$-metric ball is typically $\QP$-negligible; $\mssd_\dUpsilon$-Lipschitz functions are generally neither $\T_\mrmv$-continuous, nor $\QP$-measurable.
%\begin{itemize}
%%\item $(\dUpsilon,\mssd_\dUpsilon)$ has uncountably many components at infinite distance from each other;
%\item $\mssd_\dUpsilon$ is $\T_\mrmv$-lower semicontinuous, yet not $\T_\mrmv$-continuous;
%\item typically every~$\mssd_\dUpsilon$-ball is $\QP$-negligible;
%\item $\mssd_\dUpsilon$-Lipschitz functions are generally neither $\T_\mrmv$-continuous, nor $\QP$-measurable.
%\end{itemize}
%Nevertheless, recent studies have revealed that the $L^2$-transportation distance $\mssd_{\dUpsilon}$ is the right object to describe geometric analytic and probabilistic structures such as the curvature bounds on $\dUpsilon$~(\cite{ErbHue15, LzDSSuz22}) in the case of the Poisson measure, the consistency with the intrinsic metric of Dirichlet forms~(\cite{RoeSch99, LzDSSuz21}, characterisations of BV functions and sets of finite perimeters on $\dUpsilon$ (\cite{BruSuz21}), and the integral Varadhan short-time asymptotic~(\cite{Zha01, Suz22})
%
%%As a corollary of Thm.~\ref{t: main}, we obtain that $\mssd_{\U}$ is the right object to describe the uniqueness of the differentiable structure induced by $(\E^{\U, \QP}, \dom{\E^{\U, \QP}})$. 
%%%The quadruplet $(\U, \tau_\mrmv, \mssd_\U, \QP)$ is known to be equipped with a certain {\it geometric} energy functional $\Ch_{\mssd_{\U}, \QP}$ with domain $W^{1,2}(\U, \mssd_{\U}, \QP)$ called {\it Cheeger energy} (see Notation and Preliminaries). 
%%We equip the diffusion space~$(\U, \cdc^{\U}, \QP)$ with $\mssd_\U$ and we call it {\it the metric diffusion space}. %The following theorem identifies $\Ch_{\mssd_{\U}, \QP}$ with $\E^{\U}$.
%%\begin{corA}[Cor.s~\ref{t:LL},~\ref{c:MU}] \label{t: main2}
%%Let $\mu$ be the $\sine_2$ ensemble. The metric diffusion space~$(\U, \cdc^{\U}, \mssd_\U, \QP)$ satisfies the following:
%%\begin{enumerate}[$(a)$]
%%\vspace{1mm}
%%\item the semigroup $T_t^{\U, \QP}$ satisfies $L^\infty(\U, \QP)$-to-$\Lip(\U, \mssd_\U)$ reglarisation, i.e., 
%%\begin{align*}% \label{m:LL}
%%T_t^{\U, \QP} L^\infty(\U, \QP) \subset \Lip(\U, \mssd_\U) \quad \forall t>0 \ ;
%%\end{align*}
%%
%%\item 
%% the form $(\E^{\U, \QP}, \Lip(\U, \mssd_\U))$ is Markov unique, i.e., $(\E^{\U, \QP}, \dom{\E^{\U, \QP}})$ is the unique Markovian extension of $(\E^{\U, \QP}, \Lip(\U, \mssd_\U))$.
%% \end{enumerate}
%%\end{corA}
%% To establish Thm.\ \ref{t: main2}, we introduce a concept of $\tau_\mssd$-upper regularity for the setting of general extended metric measure spaces, which is a weaker concept than $\tau$-upper regularity introduced in  \cite{AmbGigSav15} and \cite{AmbErbSav16}. We make use of the $\tau_\mssd$-upper regularity in combination with the $\BE$ property and with the property called {\it Sobolev-to-Lipschitz property} in order to prove the one inequality $\E^{\U, \QP}(u)\ge \Ch_{\mssd_\U, \mu}(u)$.  This general result is particularly applicable to the configuration space as  the {\it Sobolev-to-Lipschitz property} has been proven in \cite{Suz22} in the case of $\beta=2$ and the $\tau_\mssd$-upper regularity is obtained by the {\it Sobolev-to-Lipschitz property} and the $\BE(0,\infty)$ proven in Thm.~\ref{t: main}. See further details in Section~\ref{sec:C=E}. 
%
%%\subsection*{Functional inequalities}
%%We further investigate functional inequalities for the diffusion space $(\U, \cdc^{\U}, \QP)$ related to the non-negative Ricci curvature bound.
%%\begin{thmA}[Cor.~\ref{t:LPS}, Thm.~\ref{t:DFH}] \label{t: main3}
%%Let $\beta>0$ and $\mu$ be the $\sine_\beta$ ensemble. The diffusion space $(\U, \cdc^{\U}, \QP)$ satisfies the following:
%%\begin{enumerate}[{\rm (a)}]
%%\item $(${\bf local Poincar\'e inequality}$)$  for $u \in \dom{\E^{\U, \QP}}$ and $t >0$,
%%\begin{align*}
%%&T^{\U, \QP}_tu^2- (T^{\U, \QP}_tu)^2 \le 2tT^{\U, \QP}_t\cdc^{\U}(u) 
%%\\
%%&T^{\U, \QP}_tu^2- (T^{\U, \QP}_tu)^2 \ge 2t\cdc^{\U} (T^{\U, \QP}_tu) \quad \ ;
%%\end{align*}
%%
%%\item $(${\bf  local logarithmic Sobolev inequality}$)$ for non-negative $u \in \dom{\E^{\U, \QP}}$ and $t>0$,
%%\begin{align*}
%%&T^{\U, \QP}_tu\log u- T^{\U, \QP}_tu\log T^{\U, \QP}_t u \le tT^{\U, \QP}_t\biggl( \frac{\cdc^{\U}(u)}{u} \biggr)
%%\\
%%&T^{\U, \QP}_tu\log u- T^{\U, \QP}_tu\log T^{\U, \QP}_t u \ge t \frac{\cdc^{\U}(T^{\U, \QP}_t u)}{T^{\U, \QP}_t u}  \fstop
%%\end{align*}
%%\noindent In the case of $\beta=2$, the metric diffusion space $(\U, \cdc^{\U}, \mssd_\U, \QP)$ furthermore satisfies the following inequalities:
%%\end{enumerate}
%%\begin{enumerate}[{\rm (a)}] \setcounter{enumi}{2}
%%\item $(${\bf  log-Harnack inequality}$)$ for every non-negative $u \in L^\infty(\U, \QP)$, $t>0$, there exists $\Omega \subset \U$ so that $\QP(\Omega)=1$ and 
%%$$T^{\U, \QP}_t(\log u)(\gamma) \le \log (T^{\U, \QP}_tu)(\eta) + \mssd_\U(\gamma, \eta)^2\comma \quad \text{every $\gamma, \eta \in \Omega$} \ ;$$
%%\item $(${\bf  dimension-free Harnack inequality}$)$ for every non-negative $u \in L^\infty(\U, \QP)$, $t>0$ and $\alpha>1$ there exists $\Omega \subset \U$ so that $\QP(\Omega)=1$ and 
%%$$(T^{\U, \QP}_tu)^\alpha(\eta)\le T^{\U, \QP}_tu^\alpha(\gamma) \exp\Bigl\{ \frac{\alpha}{2(\alpha-1)}d_\U(\gamma, \eta)^2\Bigr\} \comma \quad \text{for every $\gamma, \eta \in \Omega$} \fstop$$
%%\end{enumerate}
%%\end{thmA}
%The statement (a) and (b) in Thm.~\ref{t: main3} are standard consequences of $\BE(0,\infty)$ in Thm.~\ref{t: main} after e.g.,~\cite[Thm.s 4.7.2, 5.5.2.]{BakGenLed14}. For the statements (c) and (d), due to the fact that $\mssd_\U$ is {\it extended} distance taking $+\infty$ on sets of positive measures,  the standard proof in the framework of metric measure spaces (e.g., \cite{Wan14, KopStu21}) does not directly apply. Our proof strategy is to lift the corresponding functional inequalities from the space of finitely many configurations, for which {\it the rigidity} and {\it the tail-triviality} of the $\sine_2$ ensemble play a significant role. 
%
%%We furthermore address {\it evolutional variation inequality} on the space of Borel probabilities on $(\U, \mssd_\U)$ equipped with the $L^2$-transportation extended distance~$W_{\mssd_\U}$. Let~$\mathcal P_\QP(\U)$ be the space of all Borel probabilities on~$\U$ absolutely continuous with respect to~$\mu$. For~$\nu \in \mathcal P_\mu(\U)$, define {\it the relative entropy} as 
%%$$\Ent_\QP(v):=\int_{\U} \rho \log \rho \diff \QP \quad\ \rho:=\frac{\diff \nu}{\diff \QP} \comma \quad \dom{\Ent_\mu}:=\{\nu \in \mathcal P_\mu(\U): \Ent_\QP(\nu)<\infty\} \fstop$$ 
%%For Borel probabilities~$\nu, \sigma$ on~$\U$, {\it the $L^2$-transportation}  (also: \emph{$L^2$-Wasserstein}, or $L^2$-{\it Monge--Kantrovich--Rubinstein}) extended {distance} is defined as
%%\begin{align*}%\label{eq:Intro:WassersteinD}
%%W_\dUpsilon(\nu, \sigma) \eqdef \inf \tonde{\int_{\U^\tym{2}} \mssd_{\U}(x, y)^2\diff\cpl(x,y)}^{1/2}\comma 
%%\end{align*}
%%where the infimum is taken over all Borel probability measures~$\cpl$ on~$\U^\tym{2}$ with marginals~$\nu$ and~$\sigma$.
%%We say that $(\U, \cdc^{\U}, \mssd_\U, \QP)$ satisfies {\it evolutional variation inequality} $\EVI(0,\infty)$ if for any $\sigma=\rho\cdot \mu \in \mathcal P_\mu(\U)$, $\nu \in \dom{\Ent_{\mu}}$ with $W_{d_\U}(\sigma, \nu)<\infty$, 
%%\begin{align} \label{EVI}
%%&\Ent_{\mu}\bigl( {\mathcal T^{\U, \QP}_t \sigma}\bigr) <\infty  \comma \quad W_{\U}(\mathcal T^{\U, \QP}_t \sigma, \nu)<\infty \quad \forall t>0 \comma \tag{$\EVI(0,\infty)$}
%%\\
%%& \frac{1}{2}\frac{d^+}{dt}W_{\U}\bigl({\mathcal T^{\U, \QP}_t \sigma}, \nu \bigr)^2  \le \Ent_{\mu}({\nu}) - \Ent_{\mu}({\mathcal T^{\U, \QP}_t \sigma})\comma \notag
%%\end{align}
%%where $\frac{d^+}{dt}$ stands for the upper right derivative in $t$ and ${\mathcal T^{\U, \QP}_t}$ is the dual semigroup of $T^{\U, \QP}_t$ acting on $\mathcal P_\QP(\U)$ defined as $\diff {\mathcal T^{\U, \QP}_t \sigma} = \bigl( {T^{\U, \QP}_t \rho}\bigr) \cdot \diff \QP$. We call $\bigl({\mathcal T^{\U, \QP}_t}\bigr)_{t \ge 0}$ {\it the dual flow}. 
%%\begin{thmA}\label{t: main4}
%%Let $\mu$ be the $\sine_2$ ensemble. The metric diffusion space $(\U, \cdc^{\U}, \mssd_\U, \QP)$ satisfies $\EVI(0,\infty)$. As a consequence, the dual flow $\bigl({\mathcal T^{\U, \QP}_t}\bigr)_{t \ge 0}$ coincides with the $W_{\U}$-gradient flow of the relative entropy $\Ent_{\QP}$.
%%\end{thmA}
%%As we have shown $\BE(0,\infty)$ in Thm.~\ref{t: main}, $\E^{\U, \QP}=\Ch_{\mssd_\U, \QP}$ in Thm.~\ref{t: main2} and the log-Harnack inequality in Thm.~\ref{t: main3}, the proof of $\EVI(0,\infty)$ goes in the same way as \cite[Thm.~5.10]{ErbHue15} (see \cite[Thm.~5.26]{LzDSSuz22} as well). The coincidence of the dual flow and the gradient flow is a consequence of $\EVI(0,\infty)$ (see \cite[Thm.~9.3]{AmbGigSav14}). 
%
%\subsection*{Generalisation}
%We have been working so far in the case of $\sine_\beta$ ensembles. These arguments can be generalised in more general settings.
%\begin{assA}\label{a:GT}
%Let $K \in \R$ and $\mu$ be a fully supported Borel probability measure on~$\U(\R^n)$ satisfying the following:
%\begin{enumerate}[$(a)$]
%\item for $\QP$-a.e.~$\eta$ and any $r>0$ the regular conditional probability $\mu_r^\eta$ is absolutely continuous with respect to the Poisson measure $\pi_{\mssm_r}$ on $\U(B_r)$ with intensity~$\mssm_r$ and its density $\Psi_r^\eta$ is $\tau_\mrmv$-lower semi-continuous on $\U(B_r)$ and $K$-convex with respect to $\mssd_{\U}$;\footnote{Check if the number-rigidity is not needed for (a)}
%\item $\mu$ possesses the tail-triviality~\ref{ass:TT} and the number-rigidity~\ref{ass:Rig}.
%\end{enumerate}
%\end{assA}
%Under (a) in Assumption, the form~$(\E^{\U, \QP}, \dom{\E^{\U, \QP}})$ is constructed in the say proof as in the case of $\sine_\beta$ ensemble. We further show the synthetic curvature bound for the form~$(\E^{\U, \QP}, \dom{\E^{\U, \QP}})$ and related functional inequalities.
%\begin{thmA}\label{t:GT}
%Suppose that $\QP$ satisfies (a) in Assumption.  The form~$(\E^{\U, \QP}, \dom{\E^{\U, \QP}})$ satisfies
%\begin{enumerate}[$(a)$]
%\item {\bf $($$K$-Bakry--\'Emery inequality$)$}
%\begin{align*}
%\cdc^{\U}\bigl(T_t^{\U, \mu} u\bigr)^{\frac{1}{2}} \le e^{-2Kt}T_t^{\U, \mu} \bigl(\cdc^{\U}(u)^{\frac{1}{2}}\bigr) \quad \forall u \in  \dom{\E^{\U, \mu}} \ ; %\tag{$\BE_1(K,\infty)$}
%\end{align*}
%\item $(${\bf $K$-local Poincar\'e inequality}$)$  for $u \in \dom{\E^{\U, \QP}}$ and $t >0$,
%\begin{align*}
%&T^{\U, \QP}_tu^2- (T^{\U, \QP}_tu)^2 \le \frac{1-e^{-2Kt}}{K}T^{\U, \QP}_t\cdc^{\U}(u)  \comma
%\\
%&T^{\U, \QP}_tu^2- (T^{\U, \QP}_tu)^2 \ge \frac{e^{-2Kt}-1}{K}\cdc^{\U} (T^{\U, \QP}_tu) \ ;
%\end{align*}
%\item $(${\bf  $K$-local logarithmic Sobolev inequality}$)$ for non-negative $u \in \dom{\E^{\U, \QP}}$ and $t>0$,
%\begin{align*}
%&T^{\U, \QP}_tu\log u- T^{\U, \QP}_tu\log T^{\U, \QP}_t u \le \frac{1-e^{-2Kt}}{2K}T^{\U, \QP}_t\biggl( \frac{\cdc^{\U}(u)}{u} \biggr) \comma
%\\
%&T^{\U, \QP}_tu\log u- T^{\U, \QP}_tu\log T^{\U, \QP}_t u \ge \frac{e^{-2Kt}-1}{2K} \frac{\cdc^{\U}(T^{\U, \QP}_t u)}{T^{\U, \QP}_t u}  \fstop
%\end{align*}
%\end{enumerate}
%If, furthermore, $\QP$ satisfies (b) in Assumption, then 
%\begin{enumerate}[$(a)$] \setcounter{enumi}{3}
%\item the semigroup $T_t^{\U, \QP}$ satisfies $L^\infty(\U, \QP)$-to-$\Lip(\U, \mssd_\U)$ regularisation, i.e., 
%\begin{align*}% \label{m:LL}
%T_t^{\U, \QP} L^\infty(\U, \QP) \subset \Lip(\U, \mssd_\U) \quad \forall t>0 \ ;
%\end{align*}
%\item  the form $(\E^{\U, \QP}, \Lip(\mssd_\U))$ is Markov unique;
%\item $(${\bf  $K$-log Harnack inequality}$)$ for every non-negative $u \in L^\infty(\U, \QP)$, $t>0$, there exists $\Omega \subset \U$ so that $\QP(\Omega)=1$ and 
%$$T^{\U, \QP}_t(\log u)(\gamma) \le \log (T^{\U, \QP}_tu)(\eta) + \frac{K}{2(1-e^{-2Kt})}\mssd_\U(\gamma, \eta)^2\comma \quad \text{$\forall \gamma, \eta \in \Omega$} \ ;$$
%\item $(${\bf  $K$-dimension-free Harnack inequality}$)$ for every non-negative $u \in L^\infty(\U, \QP)$, $t>0$ and $\alpha>1$ there exists $\Omega \subset \U$ so that $\QP(\Omega)=1$ and 
%$$(T^{\U, \QP}_tu)^\alpha(\gamma)\le T^{\U, \QP}_tu^\alpha(\eta) \exp\Bigl\{ \frac{\alpha K}{2(\alpha-1)(1-e^{-2Kt})}d_\U(\gamma, \eta)^2\Bigr\} \comma \quad \text{$\forall \gamma, \eta \in \Omega$} \fstop$$
%
%\end{enumerate}
%\end{thmA}

\subsection*{Comparison with Literature}
To the author's best knowledge, this is the first article addressing the lower Ricci curvature bound on~$\U$ under the presence of interactions. In fact, in the setting of {\it interacting and infinite} particle systems of diffusion processes, no result regarding the curvature bound  has been known so far even with a simpler interaction potential like compactly supported smooth pair potential with Ruelle condition. In the non-interacting case where the invariant measure is the Poisson measure, it has been studied in~\cite{ErbHue15} in the case where the base space is Riemannian manifolds and in~\cite{LzDSSuz22} in the case where the base space is general diffusion spaces. In the case of finite particle systems, a variable Ricci curvature bound has been addressed in~\cite{GyuVon20} for Coulomb-type potentials. 

Up until now, little is understood about the transition probability of interacting {\it infinite} particle diffusions.  In particular so far, almost nothing is known about {\it quantitative estimates} of the transition semigroup corresponding to the infinite particle Dyson models as well as the $\beta$-Riesz gas model. The functional inequalities in  Cor.~\ref{c:1}, Thm.~\ref{t:3}  and the exponential decay estimate of the transition semigroup in Cor.~\ref{c:3} make a considerable contribution to this direction. 
%provide non-trivial quantitative estimates of the transition semigroup. 
%The local spectral gap estimate, the exponential decay of the heat kernel measure and the local hyper-contractivity in Cor.~\ref{c:1} 

Furthermore, the dimension-free Harnack inequality in Thm.~\ref{t:3} provides quantitative estimates of the transition semigroup of the Dyson SDE~\eqref{d:DBMS} in term of the metric structure~$\bar\mssd_\U$, which could give a new approach to study the Dyson SDEs in a geometric manner.  We note that \cite{KopStu21} provided an equivalence between a synthetic lower Ricci curvature bound (what is called $\RCD$ condition) and the Wang's dimension Harnack inequality in a framework of metric measure spaces. We cannot however apply their result to our setting as $(\U, \bar\mssd_\U, \QP)$ is not a metric measure space due to the fact that $\bar\mssd_\U$ does not generate the given topology (the vague topology) on~$\U$ and $\bar\mssd_\U$ takes $+\infty$ on sets of positive measure with respect to~$\QP$. We therefore prove the dimension-free Harnack inequality through a finite-particle approximation.
%\purple{Explain the reverse Talagrand and the log-Sobolev, local Hyper contractivity (BGL)}
%Furthermore, Thm.~\ref{t:DFH} provides a lower Ricci curvature bound in terms of the metric structure~$\bar\mssd_\U$, whereas Thm.~\ref{t:intromain} provides a %lower Ricci curvature bound in terms of the diffusion structure~$(\E^{\U, \QP}, \dom{\E^{\U, \QP}})$ . 

As a final remark, we provide the construction of Dirichlet forms with $\sine_\beta$ for arbitrary $\beta>0$ while it was previously known only for $\beta=1, 2, 4$ by~\cite{Osa96, Osa13}. It should be noted that our construction of the Dirichlet form $(\E^{\U, \QP}, \dom{\E^{\U, \QP}})$ is based on the Lipschitz core~$\Lip_b(\bar\mssd_\U, \QP)$ with respect to the $L^2$-transportation distance~$\bar\mssd_\U$, which is used to prove~$\BE(0,\infty)$, whereas the construction in~\cite{Osa96, Osa13} was based on the core of local functions. As $\Lip_b(\bar\mssd_\U, \QP)$ does not consist of local functions, the domain~$\dom{\E^{\U, \QP}}$~of the Dirichlet form in this paper could be possibly different from the one in~\cite{Osa96, Osa13} even for $\beta=1,2,4$. The identification issue of these domains could be resolved by showing the uniqueness of extensions of Dirichlet forms such as the Markov uniqueness or the essential self-adjointness with a core contained in~$\Lip_b(\bar\mssd_\U, \QP)$. 
%of the transition semigroup corresponding to the dynamics of the {\it infinite} Dyson models. \purple{to be continued} \purple{Write the application side and the construction side beyond $\beta\ge 1$}




\subsection*{Key Ideas}The main issue is to find a {\it good reduction} to finite particle systems  having the lower curvature bound.
In contrast to the case of non-interacting particles as in \cite{ErbHue15} and \cite{LzDSSuz22}, due to the presence of the interaction, the transition semigroup $T_t^{\U, \QP}$ cannot be described as the infinite tensor product~$T_t^{\R, \otimes \infty}$ of  the one particle semigroup~$T_t^{\R}$ of the Brownian motion on $\R$. We also note that the $\sine_\beta$ ensemble with $\beta>0$ is singular to the Poisson ensemble as a probability measure on~$\U$, so that we cannot deduce our case to the case discussed in \cite{ErbHue15} and \cite{LzDSSuz22} in any straightforward way in terms of the Radon--Nikod\'ym density. We therefore need a different approach from the one in \cite{ErbHue15} and \cite{LzDSSuz22} to prove the curvature bound.

%\ \purple{{\it Summarise the idea in the case of Poisson measure briefly and explain that the same idea cannot work in the interacting case}: in the non-interacting case, where the invariant measure is the Poisson measure~$\pi$, as the transition semigroup~$T_t^{\U, \pi}$ of the infinite particles splits as the (infinite) tensor of the one particle semigroup $T^{X}_t$ on the base space~$X$,  the proof comes down to the tensorisation property of the Bakry--\'Emery lower Ricci curvature bound, which is a core idea of \cite{ErbHue15, LzDSSuz22}. Under the presence of interactions, such a simple tensorisation formula cannot be expected, therefore, we need an essentially new strategy for the finite-particle approximation.}

Our reduction strategy relies on {\it the Gibbs specification} of~$\sine_\beta$ to a bounded region and {\it a superposition technique of Dirichlet forms}.
By the Gibbs specification,  infinitely many particles outside the region are frozen and have interactions with finitely many particles moving inside the region. 
For this approach, the main issues to be addressed are 
\begin{enumerate}[(a)]
\item\label{e:ICL1} to compute the curvature bound of finite particles based on {\it the Gibbs specification}; %outside which infinitely many particles are frozen and having interactions with particles inside the region; 
\item\label{e:ICL2} to construct a Dirichlet form on~the space of infinite particles by a superposition technique and lift the curvature bound from the space of finite particles onto that of infinite particles.
\end{enumerate}
In general, computing the Gibbs specification in the case of long-range interactions like the logarithmic potential is quite hard. We can however accomplish~\ref{e:ICL1} thanks to the recent achievement~\cite{DerHarLebMai20} proving that $\sine_\beta$ satisfies a Dobrushin--Lanford--Ruelle (DLR) type equation with a logarithmic interaction potential.  In order to tackle~\ref{e:ICL2}, we construct a Dirichlet form on the space of infinite particles by making use of {\it a superposition Dirichlet form}.  In this process, we need to identify two possibly different domains of (truncated) Dirichlet forms on $\U$, one is obtained from the superposition of Dirichlet forms from the space of finite particles; the other is obtained as the smallest closed extension of a Dirichlet form defined a certain core $\mathcal C$. We can identify these two Dirichlet forms on $\U$ by choosing an appropriate core~$\mathcal C$, which is based on interplays among the $L^2$-transportation distance $\mssd_\U$, the (Cheeger's) energy functional and the corresponding heat semigroup, e.g., the Rademacher-type property and a Lipschitz regularisation of the semigroup.  Many of these arguments are inspired by recent developments of {\it synthetic lower Ricci curvature bound} on metric measure spaces. 

%To lift the lower curvature bound, the choice of the domain of Dirichlet forms are essential as there could exist (infinitely) many different the $L^2$-transportation distance on the configuration space and construct a Dirichlet form based on the Lipschitz algebra with respect to~$\bar{\mssd}_\U$. Making use of interplays between the metric structure~$\bar{\mssd}_\U$ and the diffusion structure such as the Rademacher property and the Lipschitz smoothing property of the semigroup, we superpose the curvature bound from the space of finite particles to that of infinite particles. 
%In the process of the construction of the Dirichlet form, we establish the Rademacher-type property as well as the Lipschitz smoothing property of the heat semigroup. 
%Many of these arguments are inspired by recent developments on {\it synthetic lower Ricci curvature bound} on metric measure spaces.
% such as the Rademacher-type theorem and the Lipschitz smoothing property of the heat semigroup. 
% where \purple{the curvature bound depends on the number of particles}\footnote{Check this point carefully or rephrase it}. 
%In addition to the curvature bound, the coincidence of the Dirichlet form $\E^{\U, \QP}$ and the Cheeger enegy $\Ch_{\mssd_\U, \QP}$ as well as the~$L^\infty$-to-$\Lip(\mssd_\U)$ regularisation obtained in Thm.~\ref{t: main2} have been so far proved only in the case of the Poisson measure in~\cite{LzDSSuz21, LzDSSuz22}. 

%\footnote{This paragraph is wrong}We stress the significance of the Markov uniqueness. The Markov uniqueness implying the uniqueness of the corresponding diffusion process, it has been extensively studied in the context of stochastic analysis and Dirichlet forms in a number of different settings (see, e.g.,~\cite{Ebe99}). In the case of configuration spaces, however, it has been studied in rather limited settings: the Poisson measure (\cite{AlbKonRoe98, LzDSSuz22}); Gibbs measures having exponentially decaying Ruelle class potential (\cite{ChoParYoo98}), where the Markov uniqueness is a consequence of the essentially self-adjointness in these cases; hard-core potential~\cite[Thm.~1]{Tan97}. Our result regarding the Markov uniqueness in Theorem in the case of $\sine_2$ as well as in Thm.~\ref{t:GT} in more general settings reveals that the Lipschitz algebra~$\Lip(\mssd_\U)$ with respect to the $L^2$-transportation distance~$\mssd_\U$ is a right core to deal with the Markov uniqueness for long-range interaction potentials never falling into the class of Ruelle potentials.
% \footnote{Usually the Markov uniqueness is stated for a core in the domain of {\it the generator}.  Is it so interesting to have the Markov uniqueness with $\Lip(\mssd_\U)$, or $\Lip(\mssd_\U) \cap \dom{A}$?} 
%\subsection*{Key }

\subsection*{Discussion}The curvature lower bound $K=0$~in Thm.~\ref{t:intromain} does not depend on the value~$\beta$. One might wonder if there is a positive constant~$K_\beta>0$ depending on~$\beta$ so that the sharper curvature bound $\BE(K_\beta, \infty)$ holds. We however do not believe the existence of such a positive constant and we believe that $K=0$ is the best constant for the lower bound. Indeed, the Hessian matrix of the Hamiltonian in the DLR equation is computed in~\eqref{e:HCP}, which suggests that the Hessian cannot be bounded from below by a fixed positive constant as the radii $r, R$ go to infinity. 

A possible improvement of the lower bound $K=0$ should be rather by a {\it strictly positive function} $k_\beta: \U \to \R_{>0}$ depending on $\beta$. The generalisation of the concept of the lower Ricci curvature bound $\Ric \ge k$ by a function~$k$, called $\BE(k, \infty)$, has been recently developed by~\cite{Stu15} and~\cite{BraHabStu21} in a framework of metric measure spaces. This direction would be an interesting topic for further study of the geometry of the Dyson models. 

\subsection*{Outlook for further study}
In Thm.~\ref{t:GT}, we provide a sufficient condition for the Bakry--\'Emery lower Ricci curvature bound $\BE(K,\infty)$ in the case of general point processes.
The $\sine_\beta$ ensemble, the $\beta$-Riesz ensemble and the Poisson ensemble are currently only the examples for which one can verify this sufficient condition. It would be interesting as a further study to explore the curvature bound for other $1$-dimensional point processes such as $\beta$-Airy or Bessel ensembles. In this direction, one of the important points is to know the exact information of the Gibbs specification (i.e., DLR equation).

\subsection*{Outline of the article}
In Section~\ref{sec:pre}, the notation and the preliminary materials are presented. In Section~\ref{sec: Pre}, we discuss the lower Ricci curvature bound of finite particle systems, i.e., we discuss the Dirichlet forms 
\begin{align}\label{e:TDFI}
(\E^{\U(B_r), \QP_r^\eta}, \dom{\E^{\U(B_r), \QP_r^\eta}})
\end{align} on the configuration space~$\U(B_r)$ over the closed metric ball $B_r$ with radius $r>0$ centred at~$0$, whose invariant measure is the projected regular conditional probability $\QP_r^\eta$ on~$\U(B_r)$ conditioned at $\eta$ on the compliment~$B^c_r \subset \R$. The key point for the lower Ricci curvature bound of~\eqref{e:TDFI} is {\it the geodesical convexity} of the corresponding Hamiltonian on $(\U(B_r), \bar{\mssd}_\U)$, i.e., the logarithm of the Radon--Nikod\'ym density $\Psi_{r}^{\eta}:=-\log(\diff \mu_r^\eta/\diff \pi_{\mssm_r})$, where  $\pi_{\mssm_r}$ denotes the Poisson measure  on $\U(B_r)$ with the intensity measure~$\mssm_r$ being the Lebesgue measure restricted on~$B_r$.  This convexity is due to the following DLR (Dobrushin--Lanford--Ruelle) equation proven in~\cite[Thm.1.1]{DerHarLebMai20}: for $\mu$-a.e.~$\eta$, there exists a unique $k=k(\eta) \in \N_0$ so that  
%for $$
\begin{align*} %\label{1}
\diff\mu_r^\eta&%=\frac{1}{Z_{r}^{\eta}} \sum_{k=0}^\infty \mu_{r, \eta}^k
=\frac{1}{Z_{r}^{\eta}} e^{-\Psi_{r}^{k, \eta}} \diff \mssm_r^{\odot k} \comma \quad \gamma=\sum_{i=1}^k \delta_{x_i} \in \U(B_r)
\\ 
 \Psi_{r}^{k, \eta}(\gamma) &:= - \log \Biggl(  \prod_{i<j}^k |x_i-x_j|^\beta \prod_{i=1}^k\lim_{R \to \infty}\prod_{y \in \eta_{B_r^c}, |y| \le R} \Bigl|1-\frac{x_i}{y}\Bigr|^\beta \Biggr) \comma \notag
\end{align*}
where $\mssm_r^{\odot k}$ is the $k$-symmetric product measure of the Lebesgue measure $\mssm_r$ restricted on~$B_r \subset \R$
%(or it is equivalent up to constant multiplication to the Poisson measure restricted on the $k$-particle configuration space $\U^k(B_r)$) 
and $Z_{r}^{\eta}$ is the normalising constant (note that the roles of the notation~$\gamma$ and $\eta$ in \cite{DerHarLebMai20} are opposite to this article). 

In Section~\ref{sec:CI}, we prove $\BE(0,\infty)$ for $(\E^{\U, \QP}, \dom{\E^{\U, \QP}})$ in the following steps: we first construct {\it the truncated form}~$(\E_r^{\U, \QP}, \dom{\E_r^{\U, \QP}})$ on $\U$ whose gradient operator is truncated up to configurations on $B_r$ (Prop.~\ref{t:ClosabilitySecond}). We then identify it with the {\it superposition} Dirichlet form $(\bar{\E}_r^{\U, \QP}, \dom{\bar{\E}_r^{\U, \QP}})$ lifted from $(\E^{\U(B_r), \QP_r^\eta}, \dom{\E^{\U(B_r), \QP_r^\eta}})$ with respect to the conditioning $\eta$ (Thm.~\ref{t:S=M}). By this identification, we can lift $\BE(0,\infty)$ from $(\E^{\U(B_r), \QP_r^\eta}, \dom{\E^{\U(B_r), \QP_r^\eta}})$ onto the truncated form~$(\E_r^{\U, \QP}, \dom{\E_r^{\U, \QP}})$. Showing the monotonicity of the form $(\E_r^{\U, \QP}, \dom{\E_r^{\U, \QP}})$ with respect to $r$ and passing to the limit $r \to \infty$, we prove $\BE(0,\infty)$ for the limit form~$(\E^{\U, \QP}, \dom{\E^{\U, \QP}})$ (Thm.~\ref{t: main}). As a consequence of $\BE(0,\infty)$, we obtain the local Poincar\'e inequality, the local log-Sobolev inequality and the local hyper-contractivity (Cor.~\ref{t:LPS}) and an exponential decay of the transition semigroup (Cor.~\ref{c:TES}). 
\begin{figure}[htb!]
\begin{equation*} 
\hspace{-10mm}{ \xymatrix @C=1pc { 
&  (\bar{\E}_r^{\U, \QP}, \dom{\bar{\E}_r^{\U, \QP}})  \ar@{=}[r] \ar@{<-}[d]^{superposition} & (\E_r^{\U, \QP}, \dom{\E_r^{\U, \QP}}) \ar@{->}[r]^{monotone}  & (\E^{\U, \QP}, \dom{\E^{\U, \QP}}) &\hspace{-14.0mm}\BE(0,\infty)&
\\
&(\E^{\U(B_r), \QP_r^\eta}, \dom{\E^{\U(B_r), \QP_r^\eta}}) & \hspace{-20mm}\BE(0,\infty)&   \hspace{45mm} &
%\ar@{~>}[r] & ({\sf Ch}_{{\sf d}_2}, \mathscr D({\sf Ch}_{\sf d_2})) \ar@{->}[r] &  (\Upsilon, {\sf d}_{{\sf Ch}_{{\sf d}_2}},  \pi_{\sf m}) \ar@{=}[r]&(\Upsilon, {\sf d}_{2},  \pi_{\sf m})
}
}
\end{equation*}
\caption{{\rm $\BE(0,\infty)$ is transferred to $\E^{\U, \QP}$ via the vertical arrow $\uparrow$, the equality~$=$,  and the right arrow $\rightarrow$.}}
\end{figure} 
%Combining it with {\it the Sobolev-to-Lipschitz property} proven in \cite{Suz22}, we obtain the $L^\infty(\QP)$-to-$\Lip(\mssd_\U)$ property (Cor.~\ref{t:LL}).
%\purple{as well as the Markov uniqueness of $(\E^{\U, \QP}, \Lip_b(\mssd_\U))$ (Cor.s~\ref{t:LL},~\ref{c:MU}).}

In Section~\ref{sec:LH}, we prove the dimension-free Harnack inequality, the log-Harnack inequality, the Lipschitz contraction and $L^\infty(\QP)$-to-$\Lip(\bar{\mssd}_\U)$ Feller properties (Thm.~\ref{t:DFH}). 
%Due to the fact that $\mssd_\U$ is {\it extended} distance taking $+\infty$ on sets of positive measures,  the standard proof in the framework of metric measure spaces (e.g., \cite{Wan14, KopStu21, AmbGigSav14b}) does not directly apply. 
%Our proof strategy is to lift the corresponding functional inequalities from the space of finitely many configurations.
%, for which {\it the number-rigidity} and {\it the tail-triviality} of the $\sine_2$ ensemble play a significant role. 
%As a corollary, we prove the $1$-Wasserstein contraction property of the dual semigroup (Cor.~\ref{c:1WC}).

In Section~\ref{sec:GL}, we extend Theorem to the case of general point processes beyond $\sine_\beta$ (Thm.~\ref{t:GT}) and discuss $\beta$-Riesz ensembles.  

%The strategy of the proof of Thm.~\ref{t: main} is to lift $\BE_1(0,\infty)$ condition from the space of finitely many configurations to that of infinitely-many configurations. For so doing, we use the stability of the curvature lower bound under some convergences of reference measures and Dirichlet forms as well as the stability under the superposition of Dirichlet forms.  A key step for the former stability is the approximation of the regular conditional probability $\mu_r^\eta$  in \eqref{1} by the intermediate measures $\mu_{r, R}^\eta$ having {\it convex} densities as is shown in \cite{DerHarLebMai20}.
%%\begin{align} \label{2}
%%\diff\mu_{r, R}^\eta&= \sum_{k=0}^\infty \mu_{r, R}^{k, \eta}= \sum_{k=0}^\infty \Psi_{r, R}^{k, \eta} \diff \mssm^{\odot k} \comma \tag{$\star\star$}
%%\\ 
%% \Psi_{r, R}^{k, \eta}(\gamma) &:= - \log \Biggl(  \prod_{i \neq j}^k |x_i-x_j|^\beta \prod_{i=1}^k \prod_{y \in \eta_{B_r^c}, |y| \le R} |1-\frac{x_i}{y}| \Biggr) \fstop \notag
%%\end{align}
%A key step for the latter stability is to establish {\it the Markov uniqueness} of the superposed Dirichlet forms on a certain core (see, Sections \ref{sec: Pre},~\ref{sec:CI} for more details).
%By a recent progress in \cite[Thm.1.1]{DerHarLebMai20}, it turned out that the $\sine_\beta$ ensemble $\mu$ satisfies the following DLR (Dobrushin--Lanford--Ruelle) equation:
%\begin{align} \label{1}
%\diff\mu_r^\eta&=\frac{1}{Z_{r}^{\eta}} \sum_{k=0}^\infty \mu_{r, \eta}^k=\frac{1}{Z_{r}^{\eta}} \sum_{k=0}^\infty \Psi_{r}^{k, \eta} \diff \mssm^{\odot k} \comma \tag{$\star$}
%\\ 
% \Psi_{r}^{k, \eta}(\gamma) &:= - \log \Biggl(  \prod_{i \neq j}^k |x_i-x_j|^\beta \prod_{i=1}^k\lim_{R \to \infty}\prod_{y \in \eta_{B_r^c}, |y| \le R} |1-\frac{x_i}{y}| \Biggr) \comma \notag
%\end{align}
%where $\mu_{r}^\eta$ is the regular conditional probability of $\QP$ conditioned at $\eta \in \U$ in $B_r^c$, $\mssm_r^{\odot k}$ is the $k$-symmetric product measure of the Lebesgue measure $\mssm_r$ restricted on the metric ball $B_r \subset \R$ with radius $r>0$ centred at $0 \in \R$ (or it is equivalent up to constant multiplication to the Poisson measure restricted on the $k$-particle configuration space $\U^k(B_r)$) and $Z_{r}^{\eta}$ is the normalising constant.

%A closed Markovian symmetric bilinear form on $L^2(\U, \QP)$ is called {\it symmetric Dirichlet form}. When a symmetric Dirichlet form possesses {\it locality}  and {\it $\tau_\mrmv$-quasi-regularity}, the general theory of Dirichlet form (\cite[e.g., Thm. IV.3.5 \& Thm.~V.1.5]{MaRoe90}]) provides the corresponding diffusion process taking values in the configuration space $\U$ with invariant measure $\mu$ in such a way that the $L^2(\U, \QP)$-semigroup associated with the Dirichlet form provides the transition semigroup of the diffusion process uniquely up to negligible set (with respect to $(1,2)$-capacity) regarding starting points. 
%
%A strongly local $\tau_\mrmv$-quasi-regular symmetric Dirichlet form~$\E^{\U, \QP}$ on~$L^2(\U, \QP)$ with invariant measure $\QP$ can be constructed by lifting Dirichlet forms on $\U(B_r)$ onto Dirichlet forms on $\U$. To be more precise (see \S\ref{sec:CI} for details): 
%\begin{enumerate*}[{\rm (i)}]
%\item we first construct the Dirichlet form $\E^{\U(B_r), \mu_{r}^\eta}$ whose invariant measure is the regular conditional probability $\mu_{r}^\eta$; 
%\item we second construct the Dirichlet form $\E_r^{\U, \mu}$ on $\U$ by {\it superposition} of $\E^{\U(B_r), \mu_{r}^\eta}$ with respect the conditioning $\eta$; 
%\item we thirdly show the monotonicity of the Dirichlet forms $\E_r^{\U, \mu}$ as $r\to \infty$, and construct the Dirichlet form $\E^{\U, \mu}$ on $\U$ as the monotone limit form. 
%\end{enumerate*}
%%\begin{enumerate}[{\rm (i)}]
%%\item we first construct the Dirichlet form 
%%$$\E^{\U(B_r), \mu_{r, \eta}}(u)=\int_{\U(B_r)} \cdc^{\U(B_r),  \mu_{r, \eta}}(u) \diff \mu_{r, \eta} \comma$$
%%  on the configuration space $\U(B_r)$, where $ \cdc^{\U(B_r),  \mu_{r, \eta}}$ is the square field operator on $\U(B_r)$ induced by the square field $|\nabla \cdot|^2$ on $\R$ through symmetric tensorisation; 
%%\item we second construct the Dirichlet form $\E_r^{\U, \mu}$ on $\U$ by {\it superposition} of $\E^{\U(B_r), \mu_{r, \eta}}$ with respect the conditioning $\eta$;
%%\item  we thirdly show the monotonicity of the Dirichlet forms $\E_r^{\U, \mu}$ as $r\to \infty$, and construct the Dirichlet form $\E^{\U, \mu}$ on $\U$ as the monotone limit form. 
%%\end{enumerate}
%The limit Dirichlet form $\E^{\U, \QP}$ admits the square field operator $\cdc^\U$ satisfying 
%$$\E^{\U, \QP}(u)=\int_{\U} \cdc^{\U}(u) \diff \QP \fstop$$
%The square field operator $\cdc^{\U}$ plays a role of differentiable structure on the configuration space $\U$. We call the triplet $(\U, \cdc^{\U}, \QP)$ {\it diffusion space} associated with the square field $\cdc^{\U}$ and the reference measure $\QP$.
%
%According to the DLR equation \eqref{1}, the  diffusion process on $\U$ corresponding to $(\E^{\U, \QP}, \dom{\E^{\U, \QP}})$ can be understood {\it heuristically} as an unlabelled solution to the following infinitely many stochastic differential equation with logarithmic interaction (see,~\cite{Tsa16} for the rigorous construction):
%%There have been a large number of studies on rigorous mathematical constructions of interacting diffusion processes as diffusion processes in the infinite-dimensional space $\dUpsilon$, in particular, for the infinitely many interacting stochastic differential equations on $\R^n$, written `formally' as
%\begin{align*} 
%\diff X_t^k=  -  \frac{\beta}{2} \lim_{r \to \infty}\sum_{i \neq k: |X_t^k- X_t^i|<r} \frac{1}{X_t^k- X_t^i} \diff t + \diff B^k_t, \quad k \in \N \comma 
%\end{align*}
% whereby $\{B^k\}_{k\in \N}$ are independent Brownian motions on~$\R$. This correspondence will be, however, never used in the present article. 



\subsection*{Acknowledgement}
The author expresses his great appreciation to Professor Thomas Lebl\'e for making him aware of the $\beta$-Riesz ensemble as an example of Assumption~\ref{a:GT}.
A large part of the current work has been completed while he was at Bielefeld University.  He gratefully acknowledges funding by the Alexander von Humboldt Stiftung to support his stay.

\subsection*{Data Availability Statement}
No datasets were generated or analysed during the current study.

%Finally we make a remark regarding $\tau_{\mssd}$-upper regularity introduced in this article to prove $\Ch_{\mssd_\U, \QP} \le \E^{\U, \QP}$.  In \cite[Def.\ 3.13]{AmbGigSav15} and in \cite[Def.\ 12.4]{AmbErbSav16}, the notion of $\tau$-upper regularity has been introduced  in the setting of (extended) metric measure spaces with respect to a give topology $\tau$, which is a stronger concept than $\tau_\mssd$-upper regularity as the topology $\tau$ is in general weaker than the topology $\tau_\mssd$ generated by extended distance $\mssd$. 
%The $\tau$-upper regularity is in general not simple to check in the setting of infinite-dimensional spaces because it rarely happens that there is a good mollifier with respect to the topology $\tau$ and we cannot expect much the topological regularisation effect by actions of semigroups in these settings. In contrast, the $\tau_\mssd$-upper regularity is an accessible condition  since the topological regularisation with respect to $\tau_\mssd$ can be expected in these settings. Indeed, we are able to show $L^\infty$-to-$\Lip(\mssd_\U)$ regularisation by the heat semigroups in several cases of the configuration space: \cite{LzDSSuz22} for the Poisson measure; Thm.~\ref{t: main2} in this article for $\sine_2$,  as well as the Wiener space. We expect that the $\tau_{\mssd}$-upper regularity can provide an effective approach to tackle the problem of the identification of give Dirichlet forms with the Cheeger energy for general extended metric measure spaces, which could be an independent interest for analysis of Dirichlet forms. 


%Under the hypothesis of the existence of an invariant measure $\QP$, `a weak solution' to the formal equation \eqref{1} has been investigated so far for various classes of Gibbs measures whose potentials satisfying the Ruelle's super-stable and lower-regular condition (e.g., \cite{AlbKonRoe98b, Yos96}), as well as long-range interaction potentials in relation to determinantal/permanental point processes (e.g., \cite{Spo87, Osa96, Yoo05, Osa13, OsaTan14, HonOsa15, LzDSSuz21}). `A weak solution' is here meant by the construction of the quasi-regular Dirichlet form $\ttonde{\EE{\dUpsilon}{\QP},\dom{\EE{\dUpsilon}{\QP}}}$ on $L^2(\dUpsilon, \QP)$, where the reference measure $\QP$ is constructed by the potentials $(\Phi, \Psi)$ and the form $\EE{\dUpsilon}{\QP}$ can be defined as the integral against $\QP$ of an appropriate squared gradient operator $\cdc^{\dUpsilon, \QP}$ (called {\it square field}) on $\dUpsilon$ lifted from the base space $X$.  By the theory of Dirichlet form (e.g., \cite[Thm.\ 3.5 in Chap.\ IV]{MaRoe90}), there exists the system of diffusion processes on $\dUpsilon$ corresponding to $\ttonde{\EE{\dUpsilon}{\QP},\dom{\EE{\dUpsilon}{\QP}}}$,  which can be thought of as a weak solution to \eqref{1} with dropping labelling of particles. We refer the readers to \cite{Roe09, Osa19} and also \cite[\S 1.6]{LzDSSuz21} for more complete accounts and references. 
%The objective of the present work is to bring a new connection among {\it ergodicity}, {\it optimal transportation}, and {\it tail-triviality}. 
\smallskip

%We introduce a property regarding a consistency between the $L^2$-transportation distance $\mssd_{\dUpsilon(E_h)}$ on $E_h \subset X$, and the conditioned form~$\ttonde{\EE{\dUpsilon(E_h)}{\QP^{\eta}_{E_h}}, \dom{\EE{\dUpsilon(E_h)}{\QP^\eta_{E_h}}}}$ on $\dUpsilon(E_h)$, called {\it Sobolev-to-Lipschitz property for the conditioned form} (Assumption~\ref{ass:ConditionalSL}): {\it for every $h\in \N$, $k \in \N_0:=\N\cup\{0\}$ and $\mu$-a.e.\ $\eta \in \dUpsilon$, if $u \in \dom{\EE{\dUpsilon(E_h)}{\QP^\eta_{E_h}}}$ with $\Gamma^{\dUpsilon(E_h), \QP^{\eta}_{E_h}}(u) \le \alpha^2$, then there exists $\tilde{u}_{E_h, \eta}^{(k)} \in \Lipua(\mssd_{\dUpsilon^{(k)}(E_h)})$ so that  }  
%\begin{align*}
%u = \tilde{u}_{E_h, \eta}^{(k)} \quad \text{$\QP_{E_h}^\eta\mrestr{{\dUpsilon^{(k)}(E_h)}}$-a.e.} \qquad \forall \alpha \ge 0 \fstop
%%\text{any $u \in \dom{\EE{\dUpsilon(E_h)}{\QP^\eta_{E_h}}}$ with $\Gamma^{\dUpsilon(E_h)}(u) \le 1$ has a $\QP^\eta_{E_h}$-modification $\tilde{u} \in \Lipu(\mssd_{\dUpsilon(E_h)})$.} 
%\end{align*}
%Here $\dUpsilon^{(k)}(E_h):=\{\gamma \in \dUpsilon^{(k)}(E_h): \gamma E_h =k\}$ is the set of configurations on $E_h$ whose number of particles is $k$ , $\mssd_{\dUpsilon^{(k)}(E_h)}$ is the $L^2$-transportation distance on $\dUpsilon^{(k)}(E_h)$, 
%%$\cdc^{\dUpsilon(E_h)}$ is the canonical square-field operator on $\dUpsilon(E_h)$ lifted from the base space~$X$ (see \eqref{eq:d:LiftCdCRep}), 
%and $\Lipua(\mssd_{\dUpsilon^{(k)}(E_h)})$ denotes the space of $\mssd_{\dUpsilon^{(k)}(E_h)}$-Lipschitz functions whose Lipschitz constant is less than $\alpha$.
% Assumption~\ref{ass:ConditionalSL} can be verified for a wide class of probability measures on $\dUpsilon$, see Proposition~\ref{prop: ESL}, and \S \ref{sec: Exa}. 
% %constant of $u$ with respect to the distance~$\mssd_{\dUpsilon(E_h)}$. 
%%for $\QP$-a.e.\ $\eta$, 
% %\begin{align*}
%%\text{if}\ \cdc^{\dUpsilon(E_h)}(u) \le 1\ \text{$\QP_{E_h}^{\eta}$-a.e.}, \quad \text{then}\  \text{$u$ has a Lipschitz $\QP_{E_h}^{\eta}$-modification with ${\rm L}_{\mssd_{\dUpsilon(E_h)}}(u) \le 1$,}
% %\end{align*}
% %whereby  $\cdc^{\dUpsilon(E_h)}$ is the canonical square-field operator on $\dUpsilon(E_h)$ lifted from the base space~$X$ (see \eqref{eq:d:LiftCdCRep}), and ${\rm L}_{\mssd_{\dUpsilon(E_h)}}(u)$ denotes the Lipschitz constant of $u$ with respect to the distance~$\mssd_{\dUpsilon(E_h)}$. 
%% Assumption~\ref{ass:ConditionalSL} can be verified for a wide class of probability measures on $\dUpsilon$, see Proposition~\ref{prop: ESL}, and \S \ref{sec: Exa}. 
%We set the following set-to-set function generated by the $L^2$-transportation distance $\mssd_{\dUpsilon}$:
%$$\mssd_\dUpsilon(A, B):=\QP\text{-}\essinf_{\gamma \in A}\inf_{\eta \in B}\mssd_\dUpsilon(\gamma, \eta) \quad A, B \subset \dUpsilon \fstop$$
%We now state the main theorem.
%\begin{thm*}[Cor.\ \ref{cor: Erg}] 
%Let $\QP$ be a Borel probability measure on~$\dUpsilon$ satisfying~\ref{ass:Mmu}, and suppose Assumptions~\ref{ass:CE}, \ref{ass:ConditionalClos}, \ref{ass:ConditionalSL}, and \ref{ass:Rig}.    
%%Suppose further that $(\E^{m, X}, \mathcal D(\E^{m, X}))$ is irreducible on $X$\footnote{We probably only need the irreducibility of $\E^{m, E}$ on any localising sets $E$ for the assumption. See the proof.}. 
%%Suppose that $(\E^\mu, \F^\mu)$ is quasi-regular
% %$(Irr)_{B_r, \otimes n}$ $(\forall r>0, n \in \N)$\footnote{I believe that the irreducibility on $X$ tensorises and we only need to assume (Irr) on $X$. Then this is purely measure theoretic assumption and we never use the distance information.}. 
%Then, the following  are equivalent:
%\begin{enumerate}[$(i)$]
%\item $\mu$ is tail trivial;
% \item $\mssd_{\dUpsilon}(A, B)<\infty$ whenever $A$ is $\QP$-measurable,  $B$ is Borel and $\mu(A) \mu(B)>0$;
%%\item $\ttonde{\EE{\dUpsilon}{\QP},\dom{\EE{\dUpsilon}{\QP}}}$ is irreducible;
%\item $\{T_t^{\dUpsilon, \QP}\}$ is ergodic.
%%$\{T^{\dUpsilon, \QP}_t\}$ is ergodic 
%%\item $\Delta^{\dUpsilon, \QP}$-harmonic functions are trivial, i.e., 
%%$$\text{If}\ u \in \dom{\Delta^{\dUpsilon, \QP}}\ \text{and}\  \Delta^{\dUpsilon, \QP}u =0, \quad \text{then}\  \ u=\text{const.}\,.$$%\item[(iii)] ${\sf d}_{\U, 0}(A, B)<\infty$ for any $A, B$ with $\mu(A)\mu(B)>0$.
%%Furthermore, if (Rad)$_{{\sf d}_{\U}}$ holds, then 
%\end{enumerate}
%\end{thm*}
%
%According to \eqref{eq:intro:Erg}, Theorem describes the ergodicity from the operator theoretic viewpoint, which gives us the information about the long-time behaviour of the corresponding particle systems {\it in average}.  In the following, we further obtain the ergodicity from the viewpoint of each realisation, i.e., in the {\it pathwise sense}.  
%
%As we restricted the presentation here to the case of $X=\R^n$, $\ttonde{\EE{\dUpsilon}{\QP},\dom{\EE{\dUpsilon}{\QP}}}$ is quasi-regular (\cite[Cor.\ 6.3]{LzDSSuz21}) under the same assumptions in Theorem. Therefore, there exists a diffusion process $(\mathbf X_t, \mathbf P_\gamma)$ uniquely constructed for {\it quasi-every} starting point $\gamma$ on $\dUpsilon$ associated with the quasi-regular Dirichlet form $\ttonde{\EE{\dUpsilon}{\QP},\dom{\EE{\dUpsilon}{\QP}}}$ (see \cite[Thm.\ 3.5 in Chap.\ IV]{MaRoe90}). In the context of \eqref{1}, $(\mathbf X_t, \mathbf P_\gamma)$ is considered to be an unlabelled solution to \eqref{1} at the starting point $\gamma$ when $\QP$ corresponds to the potential $(\Phi, \Psi)$ in the sense of {\it quasi-Gibbs} measure (see, e.g., Def.\ \ref{d:QuasiGibbs}). We write $\mathbf P_{\nu}$ for $\int_{\dUpsilon} \mathbf P_\gamma(\cdot) d\nu(\gamma)$ for a bounded Borel measure $\nu$ on $\dUpsilon$, which corresponds to taking the initial distribution $\nu$ for the Markov processes $(\mathbf X_t, \mathbf P_\gamma)$.
%\smallskip
%
%We have {\it the convergence to the equilibrium} of the infinite particle systems $(\mathbf X_t, \mathbf P_\gamma)$.
%\begin{cor*}[Cor.\ \ref{cor: TE}]
%Let $\QP$ be a Borel probability measure on~$\dUpsilon$ satisfying~\ref{ass:Mmu}. Suppose Assumptions~\ref{ass:CE}, ~\ref{ass:ConditionalClos}, \ref{ass:ConditionalSL}, and~\ref{ass:Rig}.  
%%Suppose further that $(\E^{m, X}, \mathcal D(\E^{m, X}))$ is irreducible on $X$\footnote{We probably only need the irreducibility of $\E^{m, E}$ on any localising sets $E$ for the assumption. See the proof.}. 
%%Suppose that $(\E^\mu, \F^\mu)$ is quasi-regular
% %$(Irr)_{B_r, \otimes n}$ $(\forall r>0, n \in \N)$\footnote{I believe that the irreducibility on $X$ tensorises and we only need to assume (Irr) on $X$. Then this is purely measure theoretic assumption and we never use the distance information.}. 
%If either one of (i)--(iii) in Theorem holds, then the following are true: 
%\begin{enumerate}[$(i)$]
%\item  for any Borel measurable $\QP$-integrable function $u$, it holds in $\mathbf P_\QP$-a.e.\ that
%\begin{align} \label{eq: TE0}
%\lim_{t \to \infty} \frac{1}{t} \int_0^t u(\mathbf X_s)\diff s = \int_{\dUpsilon} u  \diff\QP \, ;
%\end{align}
%\item for any non-negative bounded function $h$, \eqref{eq: TE0} holds in $L^1(\mathbf P_{h\cdot\QP})$; 
%\item The convergence \eqref{eq: TE0} holds $\mathbf P_\gamma$-a.s.\ for $\EE{\dUpsilon}{\QP}$-quasi-every\ $\gamma$.
%\end{enumerate}
%\end{cor*}

% \paragraph{Applications}The first application of Theorem is to considerably enlarge the list of (long-range) interactions with which one can prove the ergodicity of particle systems. Up to now, there are only few classes of invariant measures with which one can show the ergodicity of $\{T^{\dUpsilon, \QP}_t\}$: one is a class of Gibbs measures with potentials satisfying the Ruelle-type conditions and having {\it sufficiently small activity constant $z$}, with which the corresponding class of Gibbs measures is uniquely determined, i.e., no phase transition occurs (\cite[Cor.\ 6.2]{AlbKonRoe98b}); the other is the $\mathrm{sine}_2$ process, which has been recently addressed in \cite{OsaTub21} by relying upon the arguments of strong solutions to~\eqref{1}  developed  in~\cite{OsaTan20}.
% 
%Theorem provides us a new {\it systematic} approach to tackle the problem of the ergodicity for a class of invariant measures satisfying \ref{ass:Rig}. As an illustration, we will prove in \S \ref{sec: Exa} that $\{T^{\dUpsilon, \QP}_t\}$ is ergodic for all the measures~$\QP$ belonging to~$\mathrm{sine}_{2}$, $\mathrm{Airy}_{2}$, $\mathrm{Bessel}_{\alpha, 2}$ ($\alpha \ge 1$), and $\mathrm{Ginibre}$ point processes. In particular, the semigroup $\{T^{\dUpsilon, \QP}_t\}$ associated with {\it Dyson Brownian motion} is covered. Our proof relies upon fundamental relations between Dirichlet forms and extended metric measure structures on $\dUpsilon$ recently developed in \cite{LzDSSuz21}. We  never access the arguments of strong solutions in \cite{OsaTan20}. 
%
%%As explained above, there are only few examples known to be ergodic:
%%Gibbs measures with Ruelle-type interactions and with small activity constants (\cite[Cor.\ 6.2]{AlbKonRoe98b}). We note that the recent paper~\cite{OsaTub21} has addressed the case of the~$\mathrm{Sine}_2$ process under a slightly different definition of the ergodicity  from the one used here, and it is not certain if the one used in~\cite{OsaTub21} implies the one used in this paper. Their proof relies on the arguments of strong solutions to~\eqref{1} and on labelling paths of the solutions discussed in \cite{OsaTan20}. In contrast, our proof relies only on Dirichlet form theory and we never access the arguments of strong solutions discussed in \cite{OsaTan20}. 
%\smallskip
%
%The second application is to show the finiteness of the $L^2$-transportation distance $\mssd_{\dUpsilon}(A, B)$ between sets~$A, B \subset \dUpsilon$. As $\mssd_{\dUpsilon}$ takes value $+\infty$ on sets of positive measure, it is not straightforward to answer the following geometric question: 
%\begin{align*}\tag*{{\sf (Q)}}
%\text{when does $\mssd_{\dUpsilon}(A, B)$ return a finite value?}
%\end{align*}
%To answer this, one is required to know {\it the shape of the tail structures} of $A, B \subset \dUpsilon$ since $\mssd_{\dUpsilon}(\gamma, \eta)$ is finite only when the two configurations~$\gamma$ and $\eta$ are sufficiently close to each other at the tail. Theorem tells us the finiteness of~$\mssd_{\dUpsilon}(A, B)$ only by checking the positivity of measures~$\mu(A)\mu(B)>0$, {\it essentially} due to the tail-triviality of $\QP$, but {\it technically} the rigidity in number \ref{ass:Rig} as well is used in the proof to control the number of particles inside bounded domains.  
%
%%The author does not know any simple proof of this statement even in the case of $\QP=\PP$ the Poisson measure.
%
%\paragraph{Comparisons and Discussions}In the case of probability measures $\QP$ on $\dUpsilon$ satisfying a certain integration-by-parts formula (denoted by {\rm (IbP1), (IbP2)} in \cite{AlbKonRoe98b}) and under the assumption of a certain uniqueness of extensions of {\it the smallest Dirichlet form},  S. Albeverio, Y. Kondratiev, and M. R\"ockner in \cite[Thm.\ 6.2]{AlbKonRoe98b} proved the equivalence of the irreducibility of the smallest form and {\it extremality} of~$\QP$ in the convex set of measures satisfying {\rm (IbP1) and (IbP2)}. Applying this general statement to the case of Gibbs measures and noting that the extremality of~$\QP$ is equivalent to the tail-triviality for Gibbs measures (see~\cite[Cor.\ 7.4]{Geo11}), the statements~\cite[Thm.\ 6.2, 6.5]{AlbKonRoe98b} confirm the equivalence between the irreducibility of the smallest Dirichlet form and the tail-triviality in the case of Gibbs measures satisfying {\rm (IbP1)}, {\rm (IbP2)} and the uniqueness of extensions of the smallest form. 
%
%Assumptions {\rm (IbP1)} and {\rm (IbP2)} can be typically verified for Ruelle-type potentials, which is sufficient for the implication from {\it extremality of $\QP$} to {\it irreducibility of the smallest form}. To show the opposite implication, the uniqueness of extensions of the smallest form (which is phrased as $H_0^{1,2}(\Gamma, \QP)= W^{1,2}(\Gamma, \QP)$ in \cite[\S6]{AlbKonRoe98b}) is needed in \cite{AlbKonRoe98b}, which is more demanding to be verified: indeed, the condition $H_0^{1,2}(\Gamma, \QP)= W^{1,2}(\Gamma, \QP)$  is stronger than {\it Markov uniqueness} since $W^{1,2}(\Gamma, \QP)$ defined there is not necessarily a {\it Dirichlet} form as noted in \cite[p.285]{AlbKonRoe98b}. 
%%his condition has been verified only in the case of  mixed Poisson measures \cite[Rem\ 6.2]{AlbKonRoe98b}.  
%Furthermore, Assumptions~{\rm (IbP1), (IbP2)} are also not simple to be verified if $\QP$ is a more general Gibbs measure beyond Ruelle-type or determinantal/permanental point process. For these reasons, the same line of strategies does not seem to work in the setting of the current paper.  
%%as their proof uses the extremality of $\QP$ --- there is no such concept beyond Gibbs measures --- rather than the tali-triviality, and 
%
%The novelty of the current paper in this context is 
%\begin{enumerate*}[(a)]
%\item to replace the concept of {\it extremality} by {\it tail-triviality}, the latter of which is well-defined and verifiable beyond Gibbs measures; \item instead of {\rm (IbP1), (IbP2)}, we rely upon the rigidity in number~\ref{ass:Rig} and  {\it the localisation argument by conditioned forms} developed in the recent paper \cite{LzDSSuz21}, which is encoded in Assumptions~\ref{ass:CE}, ~\ref{ass:ConditionalClos}, \ref{ass:ConditionalSL} in Theorem; \item to remove any assumption of the uniqueness of extensions of the smallest form.
%\end{enumerate*}
%As a consequence, our result can be applied to the case of long-range interactions such as $\mathrm{sine}_{2}$, $\mathrm{Airy}_{2}$, $\mathrm{Bessel}_{\alpha, 2}$ ($\alpha \ge 1$), and $\mathrm{Ginibre}$ point processes. 
%% which have not been covered in the framework of  \cite{AlbKonRoe98b}. 
%%(Note that, as mentioned above,  the case of $\mathrm{Sine}_{2}$ has been addressed in \cite{OsaTub21} by an approach of strong solutions.) 
%%Furthermore, by utilising the recent results in \cite{LzDSSuz21} regarding the equivalence between the Dirichlet form structure and the extended metric measure structure, we relate {\it the finiteness of the $L^2$-transportation distance $\mssd_{\dUpsilon}$} to {\it the tail-triviality of $\QP$} and {\it the ergodicity of $\{T_T^{\dUpsilon, \QP}\}$}, which  brings a new insight into this field.  
%%We stress that Theorem never assumes the uniqueness of extensions of $\ttonde{\EE{\dUpsilon}{\QP},\dom{\EE{\dUpsilon}{\QP}}}$ because our proofs rely only upon the smallest form, but never upon some extension of it. 
%
%Furthermore, our setting does not require any smooth structures of the base space ~$X$ and, therefore,  include a number of singular spaces as listed in Example~\ref{subsec: ES} while the aforementioned literature assumes $X$ to be a Riemannian manifold. We refer the readers to \cite[\S1.5.4, \S1.6.4, \S 7.1]{LzDSSuz21} for a more complete account of the motivation for working in this generality in relation to applications, e.g., to statistical physics, random graphs, hyperplane arrangements.  
%
%
%The equivalence (ii) $\iff$ (iii) in Theorem provides a new characterisation of the ergodicity of interacting diffusion processes by optimal transport. 
%The ergodicity of $\{T_t^{\dUpsilon, \QP}\}$ is a {\it statistical physical concept}, while the finiteness of the $L^2$-transportation distance between $\QP$-positive sets  is a {\it purely geometric concept} of the  extended metric measure space $(\dUpsilon, \mssd_{\dUpsilon}, \QP)$. The proof relies upon (a) the recent results \cite{LzDSSuz21} concerning foundations between the extended metric measure structure and the Dirichlet forms on $\dUpsilon$; (b) showing {\it the Sobolev-to-Lipschitz property} (Theorem~\ref{thm: SL}), which was conjectured originally by R\"ockner--Schied~\cite[Remark after Thm.~1.5]{RoeSch99}. 

%\smallskip
%We close this introduction by providing an outlook on further improvements. Our assumption of the rigidity in number~\ref{ass:Rig} requires a strong spatial correlation to $\QP$, which is, however,  not a necessary condition for the ergodicity. Indeed, \cite[Thm.\ 4.3]{AlbKonRoe98} proved the ergodicity for the Poisson measures, which obviously do not posses the rigidity in number~\ref{ass:Rig} since the laws of the Poisson point processes inside and outside any bounded sets are independent. A challenging question is whether we can prove the ergodicity of the Dirichlet form $\ttonde{\EE{\dUpsilon}{\QP},\dom{\EE{\dUpsilon}{\QP}}}$ for general tail-trivial invariant measures without the rigidity in number~\ref{ass:Rig}. 
%
%\newpage
%\begin{ese}\normalfont
%As we explained in Example 1.5., the tail triviality holds for a wide class of determinantal Point processes and extremal Gibbs measures. 
%Our result is a drastic generalisation of the irreducibility result of AKRII in the framework of canonical Gibbs measures. 
%In AKR II, the irreducibility is characterised by the extremal property of Gibbs measures in the case of smooth manifolds. Their proofs heavily relies both on the smooth structure of manifolds and the structure of canonical Gibbs measures. Our result extends their result to (i) a large class of non-smooth $X$; (ii) a large class of invariant measures covering not only both of derminantal point processes and canonical Gibbs measures, but also all tail-trivial measures. Our proof does not use any structural property of $\mu$ (such as the interaction potential in the Gibbs framework  or determinatal $n$-point correlation functions in the derminatal PP framework), our proof  only relies on the tail triviality. 
%\end{ese}
%\paragraph{Organisation of the paper}In \S\ref{sec: Pre}, we introduce necessary concepts and recall results used for the arguments in later sections. In \S\ref{sec: Irr}, we prove the equivalence between the ergodicity and the tail-triviality, for which we do not assume any metric structure of the base space $\mcX$. In \S\ref{sec: SL}, we prove the Sobolev-to-Lipschitz property. In \S\ref{sec: Equiv}, we provide the proof of the main theorem. In \S\ref{sec: Ver}, we give sufficient conditions to verify the main assumptions of Theorem.  In~\S\ref{sec: Exa}, we confirm that Theorem can be applied to $\mathrm{sine}_{2}$, $\mathrm{Airy}_{2}$, $\mathrm{Bessel}_{\alpha, 2}$ ($\alpha \ge 1$), and $\mathrm{Ginibre}$ point processes. 




\section{Notation and Preliminaries} \label{sec:pre}

\subsection{Numbers, Tensors, Function Spaces}
We write $\N:=\{1, 2, 3, \ldots\}$, $\N_0=\{0, 1, 2, \ldots \}$, $\EN:=\N \cup \{+\infty\}$ and $\EN_0:=\N_0 \cup\{+\infty\}$. 
%The analogous notation $\ER$ is set also for $\R$. 
The uppercase letter $N$ is used for  $N \in \EN_0$, while the lowercase letter $n$ is used for $n \in \N_0$. 
%\subsection*{Product, tensor and symmetric tensor}
We shall adhere to the following conventions:
\begin{itemize}
\item the superscript~${\square}^\tym{N}$ (the subscript~$\square_\tym{N}$) denotes ($N$-fold) \emph{product objects};

\item the superscript~${\square}^\otym{N}$ (the subscript~$\square_\otym{N}$) denotes ($N$-fold) \emph{tensor objects};

\item the superscript~${\square}^\osym{N}$ (the subscript~$\square_\osym{N}$) denotes ($N$-fold) \emph{symmetric tensor objects};

%\item the superscript~${\square}^\asym{N}$ denotes objects \emph{relative to products}, not necessarily in product form;

%\item the superscript~${\square}^\sym{N}$ denotes objects \emph{relative to symmetric products}, not necessarily in symmetric product form;

%\item for a permutation~$\sigma$ the subscript~${\square}_\sigma$ denotes that the objects in~$\square$ are permuted according to~$\sigma$, e.g., for~$\mbfx\eqdef\seq{x_i}_{i\leq N}$, we write~$\mbfx_\sigma$ in place of~$\seq{x_{\sigma(i)}}_{i\leq N}$;
%\item upper case \textbf{boldface} letters always denote (subsets of) infinite-product spaces.
\end{itemize}
%Following the convention, the $n$-dimensional Euclidean space $\R^{\times n}$ is exceptionally written as $\R^{n}$ without product symbol $\times$. 
%\subsection*{Function spaces}
Let~$(X, \tau)$ be a topological space with $\sigma$-finite Borel measure~$\nu$. We use the following symbols:
\begin{enumerate}[$(a)$]
%\item $\Sb(X)$ for the space of real-valued \emph{bounded} $\A$-measurable functions; 
\item $L^p(\nu)$ $(1 \le p \le \infty)$ for the space of $\nu$-equivalence classes of functions $u$ with~$|u|^p$ $\nu$-integrable when $1 \le p<\infty$, and with $u$ $\nu$-essentially bounded when $p=\infty$. The $L^p(\nu)$-norm is denoted by $\|u\|^p_{L^p(\nu)}:=\|u\|^p_p:=\int_X |u|^p \diff \nu$ for $1 \le p <\infty$, and $\|u\|_{L^\infty(\nu)}:=\|u\|_\infty=\esssup_X u$.  In the case of $p=2$, the inner-product is denoted by $(u, v)_{L^2(\nu)}:=\int_X uv \diff \nu$; 
\item  $L^p_s(\nu^{\otimes n}):=\{u \in L^p(\nu^{\otimes n}): u\ \text{is symmetric}\}$ where $u$ is said to be {\it symmetric} if and only if $u(x_1, \ldots, x_k)=u(x_{\sigma(1)}, \ldots, x_{\sigma(k)})$ for any element $\sigma \in \mathfrak S(k)$ in the $k$-symmetric group.
%and
%\begin{align*}
%\mcL^\infty(\mssm)\longrar L^\infty(\mssm)\colon f\longmapsto\class[\mssm]{f}
%\end{align*}
%for the quotient map to the corresponding quotient Banach algebra~$L^\infty(\mssm)$.
%Here and elsewhere we indicate by~$\class[\mssm]{f}$ the class of a function~$f$ up to $\mssm$-a.e.\ equivalence.
%
%For~$f_i\in L^\infty(\mssm)$, resp.~$ f_i\in \mcL^\infty(\mssm)$,~$i\leq k$, set
%\begin{align*}
%\mbff\eqdef \seq{f_1,\dotsc, f_k}\in L^\infty(\mssm;\R^k)\comma \qquad \text{resp.}\qquad \mbff \eqdef \tseq{ f_1,\dotsc,  f_k}\in\mcL^\infty(\mssm;\R^k) \fstop
%\end{align*}
%\item $\Sb(\msE)$ for the space of bounded $\A$-measurable $\msE$-\emph{eventually vanishing} functions on~$X$, viz.
%\begin{align}\label{eq:SbExhaustion}
%$$\Sb(\msE)\eqdef \set{f\in\Sb(X) : f\equiv 0 \text{~~on } E^\complement \text{~~for some }E\in\msE};$$
%\end{align}
\item $C_b(X)$ for the space of $\T$-continuous bounded functions on~$X$; if $X$ is locally compact, $C_0(X)$ denotes the space of $\tau$-continuous and compactly supported  functions on $X$; $C_0^\infty(\R)$ for the space of compactly supported smooth functions on $\R$;
%and~$\Cz(\msE)\eqdef \Cb(X)\cap \Sb(\msE)$ for the space of $\T$-continuous bounded $\msE$-\emph{eventually vanishing} functions on~$X$; 
%$\Ccompl(\msE)$ for the space of $\T$-continuous bounded functions on~$X$ \emph{vanishing at $\msE$-infinity}, viz.
%\begin{equation*}
%\Ccompl(\msE)\eqdef \set{f\in\Cb(\T): \forall \eps>0 \quad \exists E_\eps\in\msE : \abs{f(x)}<\eps \quad x\in E_\eps^\complement};
%\end{equation*}
%\subsubsection{Topological local diffusion spaces}
%\item $\Cbinfty(\R^k)$ for the space of real-valued bounded smooth functions on~$\R^k$ with bounded derivatives of all orders; 
\item We write $\1_{A}$ for the indicator function on $A$, i.e., $\1_{A}(x)=1$ if and only if $x \in A$, and $\1_A(x)=0$ otherwise; $\delta_x$ for the Dirac measure at $x$, i.e., $\delta_x(A)=1$ if and only if $x \in A$, and $\delta_{x}(A)=0$ otherwise;
\item For a space of real-valued functions, the subscript $+$ is used for the subspace of non-negative functions. For instance, $C_{b,+}(X):=\{u \in C_b(X): u \ge 0\}$. 
\end{enumerate}


\subsection{Dirichlet forms}
We refer the reader to \cite{MaRoe90, BouHir91} for this subsection. Throughout this paper, a Hilbert space always means a separable Hilbert space with inner product~$(\cdot, \cdot)_H$ taking value in $\R$. 

\paragraph{Dirichlet forms}Given a bilinear form~$(Q,\dom{Q})$ on a Hilbert space~$H$, we write
\begin{align*}
Q(u)\eqdef Q(u,u) \comm \qquad Q_\alpha(u,v)\eqdef Q(u,v)+\alpha(u, v)_H\comm \alpha>0\fstop
\end{align*}
Let $(X, \Sigma, \nu)$ be a $\sigma$-finite measure space. A \emph{symmetric Dirichlet form on~$L^2(\nu)$} is a non-negative definite densely defined closed symmetric bilinear form~$(Q,\dom{Q})$ on~$L^2(\nu)$ satisfying the Markov property
\begin{align*}
u_0\eqdef 0\vee u \wedge 1\in \dom{Q} \quad \text{and} \quad Q(u_0)\leq Q(u)\comm \quad u\in\dom{Q}\fstop
\end{align*}
Throughout this article, {\it Dirichlet form} always means {\it symmetric} Dirichlet form.
If not otherwise stated,~$\dom{Q}$ is always regarded as a Hilbert space with norm
$$\norm{\emparg}_{\dom{Q}}\eqdef Q_1(\emparg)^{1/2}:=\sqrt{Q(\emparg)+\norm{\emparg}_{L^2(\nu)}^2}\fstop $$
%A \emph{Dirichlet space} is a pair~$(\mbbX,\mcE)$, where~$\mbbX$ satisfies~\ref{ass:Hausdorff} and~$(\mcE,\dom)$ is a Dirichlet form on~$L^2(\mssm)$.
%A \emph{pseudo-core} is any $\dom$-dense linear subspace of~$\dom$.
In order to distinguish Dirichlet forms defined in different base spaces with different reference measures, we often use the notation $Q^{X, \nu}$ to specify the base space $X$ and the reference measure $\nu$. 

\paragraph{Square field}A Dirichlet form $(Q, \dom{Q})$ {\it admits square field $\cdc$} if there exists a dense subspace $H \subset \dom{Q} \cap L^\infty(\nu)$ having the following property: for any $u \in H$, there exists $v \in L^1(\nu)$ so that 
$$2Q(uh, u) -Q(h, u^2) = \int_X h v \diff \nu \quad \forall h \in \dom{Q} \cap L^\infty(\nu) \fstop$$
Such $v$ is denoted by $\Gamma(u)$. The square field $\Gamma$ can be uniquely extended as an operator on $\dom{Q} \times \dom{Q} \to L^1(\nu)$ (\cite[Thm.\ I.4.1.3]{BouHir91}).

\paragraph{Semigroups and generators}We refer the reader to \cite[Chap.~I, Sec.~2]{MaRoe90} for the following contents. Let $(Q, \dom{Q})$ be a symmetric closed form on a Hilbert space~$H$.  {\it The infinitesimal generator $(A, \dom{A})$} corresponding to $(Q, \dom{Q})$ is the unique densely defined closed operator on $H$ satisfying the following integration-by-parts formula:
$$-(u, Av)_{H}=Q(u, v) \quad \forall u \in \dom{Q},\ v\in \dom{A} \fstop$$
{\it The resolvent operator} $\{R_\alpha\}_{\alpha \ge 0}$ is the unique bounded linear operator on $H$ satisfying 
$$Q_\alpha(R_\alpha u, v) = (u, v)_{H} \quad \forall u \in H \quad v \in \dom{Q} \fstop$$
{\it The semigroup} $\{T_t\}_{t \ge 0}$ is the unique bounded linear operator on $H$ satisfying 
$$G_\alpha u = \int_{0}^\infty e^{-\alpha t} T_{t}u \diff t \quad u \in H\fstop$$

\paragraph{Locality}Let $(Q, \dom{Q})$ be a Dirihclet form on $L^2(\nu)$. It is called {\it local} (\cite[Def.~5.1.2 in Chap.~I]{BouHir91}) if for any $F, G \in C_c^\infty(\R)$ and any $u \in \dom{Q}$, 
$${\rm supp}[F] \cap {\rm supp}[G] = \emptyset \implies Q(F_0\circ u, G_0\circ u)=0 \comma$$
where $F_0(x):=F(x)-F(0)$ and $G_0(x):=G(x)-G(0)$.

%\purple{Add definition of quasi-regular}
% We say that $(Q, \dom{Q})$ is called {\it strongly local} if it is local and the corresponding semigroup $\sem{T_t}$ satisfies 
%$$T_t\1 =\1 \quad \forall t>0 \fstop$$

%\paragraph{Local domain}Let $(X, \tau)$ be a locally compact topological space and~$\nu$ be a Radon measure. We say that~$u \in L^2(\nu)$ belongs to the \emph{local domain}~$\domloc{Q}$ if, for any relatively compact open set~$U \subset X$, there exists~$v \in \dom{Q}$ so that~$u = v$ on~$U$. 



\subsection{Metric space} 
Let~$X$ be any non-empty set. A function $\mssd\colon X^\tym{2}\rar [0,\infty]$ is an \emph{extended distance} if it is symmetric and satisfying the triangle inequality, and it does not vanish outside the diagonal in~$X^{\tym{2}}$, i.e.~$\mssd(x,y)=0$ iff~$x=y$; a \emph{distance} if it is finite. 
%
%A quadruplet $(X, \tau, \mssd, \nu)$ is said to be {\it a Polish extended metric finite-measure space} in the sense of \cite[Def.~2.3]{AmbGigSav14} if 
%\begin{enumerate}[$(a)$]
%\item $(X, \tau)$ is a Polish space;
%\item $\mssd$ is a complete extended distance;
%\item the topology $\tau_\mssd$ generated by $\mssd$ is finer than $\tau$ and $\mssd$ is $\tau^{\otimes 2}$-lower semi-continuous;
%\item $\nu$ is a non-negative finite Borel measure; 
%\end{enumerate}
%%\footnote{Decide which one we mainly use. It seems that the Polish one is not used.}
%A quadruplet $(X, \tau, \mssd, \nu)$ is said to be {\it an extended metric finite-measure space} in the sense of \cite[Def.~4.7]{AmbErbSav16} if 
%\begin{enumerate}[$(a)$]
%\item $(X, \tau, \mssd)$ is a Hausdorff extended metric space; 
%\item there exists a family of $\tau^{\times 2}$-continuous bounded semi-distances $\{\mssd_i\}_{i \in I}$ with 
%$$\sup_{i \in I} \mssd_{i} = \mssd \ ;$$
%\item the topology $\tau$ is generated by the following algebra:
%$$\Lip_b(X, \tau, \mssd):=\{u: X\to \R: u\ \text{is bounded,\ $\mssd$-Lipschitz,\ $\tau$-continuous}\} \fstop$$
%\item $\nu$ is a finite Radon measure on $(X, \mathcal B(\tau))$.
%\end{enumerate}
Let~$x_0\in X$ and~$r\in [0,\infty)$. We write~$B_r(x_0)\eqdef \set{\mssd_{x_0} \le r}$, where $\mssd_{x_0}:=\mssd(x_0, \cdot)$. 
%We call~$B_\infty(x_0)$ the \emph{$\mssd$-accessible component} of~$x_0$ in~$X$.
%Note that, if~$\mssd$ is an extended pseudo-metric, then both of the inclusions~$\set{x_0}\subset \cap_{r>0} B_r(x_0)$ and~$B_\infty(x_0)\subset X$ may be strict ones. 
%
%We say that an extended metric space is \emph{complete} if~$B_\infty(x)$ is complete for each~$x\in X$.
%
%Finally set
%\begin{align*}
%\mssd(\emparg, A)\eqdef& \inf_{x\in A} \mssd(\emparg,x) \colon X\longrar [0,\infty] \comm \qquad A\subset X\fstop
%\end{align*}
%For an extended distance~$\mssd$ on~$X$, let~$\T_\mssd$ denote the topology on~$X$ induced by the distance~$\mssd\wedge 1$.
%The topology~$\T_\mssd$ is Hausdorff if and only if~$\mssd$ is an extended distance.
%
%The topology~$\T_\mssd$ is separable if and only if there exists a countable family of points~$\seq{x_n}_n\subset X$ so that~$X=\cup_n B^\mssd_\infty(x_n)$ and~$(B^\mssd_\infty(x_n),\mssd)$ is a separable pseudo-metric space for every~$n\in \N$.

%If~$\mssd$ is an extended pseudo-distance, then~$\mssd(\emparg, A)=\mssd(\emparg, \cl_{\mssd} A)$ for every~$A\subset X$, where~$\cl_{\mssd}=\cl_{\T_\mssd}$ always denotes the closure of~$A$ in the topology~$\T_\mssd$.

\paragraph{Lipschitz algebras}A function~$f\colon X\rar \R$ is {\it $\mssd$-Lipschitz} if there exists a constant~$L>0$ so that
\begin{align}\label{eq:Lipschitz}
\tabs{ u(x)-u(y)}\leq L\, \mssd(x,y) \comm \qquad x,y\in X \fstop
\end{align}
The smallest constant~$L$ so that~\eqref{eq:Lipschitz} holds is the (global) \emph{Lipschitz constant of $u$}, denoted by~$\Lip_\mssd{(u)}$.
For any non-empty~$A\subset X$, we write~$\Lip(A,\mssd)$, resp.~$\bLip(A,\mssd)$ for the family of all $\mssd$-Lipschitz functions, resp.\ bounded $\mssd$-Lipschitz functions on~$A$. 
%we write~$\Lip(A,\mssd)$, resp.~$\bLip(A,\mssd)$ for the family of all finite, resp.\ bounded, $\mssd$-Lipschitz functions on~$A$. 
For simplicity of notation, further let 
$$\Lip(\mssd)\eqdef \Lip(X,\mssd)\comma\quad \ \bLip(\mssd)\eqdef \bLip(X,\mssd)\fstop$$ 
Set also $\Lip^1(\mssd):=\{u \in \Lip(\mssd): \Lip_{\mssd}(u) \le 1\}$ and $\Lip^1_b(\mssd):=\Lip^1(\mssd) \cap \bLip(\mssd).$ For a given measure $\nu$, we set 
$$\Lip(\mssd, \nu):=\{u \in \Lip(\mssd): \text{$u$ is $\nu$-measurable}\} \comma$$
as well as~$\Lip_b(\mssd, \nu)$ and $\Lip^1_b(\mssd, \nu)$ denoting the corresponding subspaces of $\nu$-measurable functions respectively.
%Note that $\mssd_\U$ being merely an extended distance, $\mssd_\U$-Lipschitz functions are not necessarily measurable with respect to Borel measures associated with the vague topology~$\tau_\mrmv$, for which the aforementioned notation is needed.

%\paragraph{Geodesic space}Let $(X, \mssd)$ be a complete separable metric space. A continuous map~$\omega: [0,1] \to X$ is called {\it absolutely continuous curve} if 
%\begin{align} \label{e:AC}
%\mssd(\omega_t, \omega_s) \le \int_t^s g(r) \diff r\quad \forall t, s \in [0,1], \quad t \le s \comma
%\end{align}
%for some $g \in L^1([0,1])$. For an absolutely continuous curve~$\omega$, {\it the metric speed} is defined as
%$$|\dot{\omega}|:=\lim_{h \to 0}\frac{\mssd(\omega_{t+h}, \omega_t)}{|h|} \comma$$
%where the limit $|\dot{\omega}|$ exists for almost every~$t$ and $|\dot{\omega}|$ is  the minimal $L^1$-function (up to negligible sets with respect to the Lebesgue measure) among functions $g$ in~\eqref{e:AC}.   
%
%A continuous map~$\omega: [0,1] \to X$ is called {\it constant speed geodesic} if $\mssd(\omega_t, \omega_s)=|t-s|\mssd(\omega_0, \omega_1)$ for any~$t, s\in [0,1]$. The space~$(X, \mssd)$ is called {\it geodesic space} if for any $x_0, x_1 \in X$ there exists an absolutely continuous curve $\omega$ with $\omega_i=x_i$ $(i=0, 1)$ so that 
%$$\int_0^1 |\dot\omega_t| \diff t \le \mssd(x_0, x_1) \fstop$$
%The curve $\omega$ is in particular a constant speed geodesic up to a time-reparametrisation. 

\paragraph{Geodesical convexity}A metric space~$(X, \mssd)$ is called {\it a geodesic space} if for any $x_0, x_1 \in X$ there exists {\it a constant speed geodesic}~$\omega:[0,1] \to X$ connecting $x_0$ and~$x_1$:
$$\omega_0=x_0 \comma \quad \omega_1=x_1\comma \quad \mssd(\omega_t, \omega_s)=|t-s|\mssd(\omega_0, \omega_1) \quad \forall t, s \in [0,1]\fstop$$
For a function $U: X \to \R\cup\{+\infty\}$, define $\dom{U}:=\{x \in X: U(x)<\infty\}$. We say that $U$ is {\it $K$-geodesically convex} for $K \in \R$ if for any $x_0, x_1 \in \dom{U}$ there exists a constant speed geodesic $\omega: [0,1] \to X$ with $\omega_0=x_0$ and $\omega_1=x_1$ and 
$$U(\omega_t)\le (1-t)U(\omega_0)+tU(\omega_1)-\frac{K}{2}t(1-t)\mssd^2(\omega_0, \omega_1) \quad \forall t \in [0,1] \fstop$$
When $K=0$, we say that $U$ is {\it geodesically convex}. 


\subsection{Cheeger energies} \label{sec:Ch}
A complete separable geodesic metric space~$(X, \mssd)$ equipped with fully supported Radon measure~$\nu$ with finite total mass~$\nu(X)<\infty$ is called {\it a metric measure space} in this article. 
%\footnote{Take the definition in \cite[Def. 6.1]{AmbErbSav16}}%We follow the presentation as in \cite[\S 4]{AmbGigSav14}.
Let $(X, \mssd, \nu)$ be a metric measure space. For $u \in \Lip(\mssd)$, {\it the slope $|\mathsf D_{\mssd}u|(x)$} is defined as 
\begin{align*}
|\mathsf D_{\mssd}u|(x):=
\begin{cases} \displaystyle
\limsup_{y \to x}\frac{|u(x)-u(y)|}{\mssd(x, y)} &\text{if $x$ is not isolated}; 
\\
0 &\text{otherwise} \fstop
\end{cases}
\end{align*}
The slope is universally measurable, see \cite[Lem.~2.6]{AmbGigSav14}. 
%In the case that $(X, \tau, \mssd, \nu)$ is an extended metric topological finite-measure space, we alternatively work with the following slope:
%{\it The asymptotic slope} $|\mathsf D^a_{\mssd}u|(x)$ is defined as 
%$$|\mathsf D^a_{\mssd}u|(x):=\lim_{r\downarrow 0}\Lip_{\mssd}^a(u, x, r), \quad \Lip_{\mssd}^a(u, x, r):=\sup_{\mssd(y,x)\vee\mssd(z,x)<r,\ \mssd(y,z)>0}\frac{|f(y)-f(z)|}{\mssd(y,z)} \comma$$
%with $\Lip_{\mssd}^a(u, x, r)=0$ if $x$ is an isolated point. By construction, $|\mathsf D^a_{\mssd}u|$ is $\mssd$-upper semi-continuous. 
The Cheeger energy $\Ch_{\mssd, \nu}: L^2(\nu) \to \R \cup\{+\infty\}$
 %and $\Ch^a_{\mssd, \nu}: L^2(\nu) \to \R \cup\{+\infty\}$ 
 is defined as the $L^2(\nu)$-lower semi-continuous envelope of $\int_{X} |\mathsf D_\mssd u|^2 \diff \nu$: 
 %and $\int_{X} |\mathsf D_\mssd^a u|^2 \diff \nu$ respectively, i.e.,
 \begin{align*}% \label{d:CH}
 \Ch_{\mssd, \nu}(u)&:=\inf \biggl\{ \liminf_{n \to \infty} \int_{X} |\mathsf D_{\mssd} u_n|^2 \diff \nu:\ u_n \in \Lip(\mssd)\cap L^2(\nu) \xrightarrow{L^2} u \biggr\} \fstop
% \\
%  \Ch^a_{\mssd, \nu}(u)&:=\inf \biggl\{ \liminf_{n \to \infty} \int_{X}g_n^2 \diff \nu:\ u_n \in \Lip(\mssd)\cap L^2(\nu) \xrightarrow{L^2} u,\  |\mathsf D^a_{\mssd} u_n| \le g_n \biggr\} \fstop \notag
\end{align*}
% Namely, $\Ch_{\mssd, \nu}$ is  the $L^2(\nu)$-lower semi-continuous envelop of the convex functional $\int_{X} |\mathsf D_{\mssd} u|^2 \diff \nu$ and $\int_{X} |\mathsf D^a_{\mssd} u|^2 \diff \nu$ respectively. 
The domain is denoted by $W^{1,2}(X, \mssd, \nu):=\{u \in L^2(\nu): \Ch_{\mssd, \nu}(u)<\infty\}$.
%and $W^{1,2}_a(X, \mssd, \nu):=\{u \in L^2(\nu): \Ch^a_{\mssd, \nu}(u)<\infty\}$. 
The Cheeger energy $\Ch_{\mssd, \nu}$ 
%and $\Ch^a_{\mssd, \nu}$ 
can be expressed by the following integration, see~\cite[Thm.~4.5]{AmbGigSav14} 
%and \cite{AmbErbSav16}
: there exists a measurable function $|\nabla u|_* \in L^2(\nu)$ so that $|\nabla u|_* \le  |\mathsf D_{\mssd} u|$
% (resp.~$|\nabla u|_w \le  |\mathsf D^a_{\mssd} u|$) 
$\nu$-a.e. for every $u \in \Lip(\mssd)$ and 
 \begin{align*}
 \Ch_{\mssd, \nu}(u)&=\int_X |\nabla u|_*^2 \diff\nu \quad \forall  u \in W^{1,2}(X, \mssd, \nu) \comma
% \\
%  \Ch^a_{\mssd, \nu}(u)&=\int_X |\nabla u|_w^2 \diff\nu \quad \forall  u \in W^{1,2}_a(X, \mssd, \nu) \comma
 \end{align*}
 where $|\nabla u|_*$ is called {\it minimal relaxed slope}.% \footnote{It seems more natural to use the relaxed slope instead. }
 
%In the setting of Polish extended metric finite-measure spaces, these two energies coincide (see, e.g., \cite[Prop.~4.21]{LzDSSuz21}), therefore, we simply write $\Ch_{\mssd, \nu}$. In the setting of extended metric finite-measure spaces, the slope $|\mathsf D_{\mssd} u|$ is unknown to be measurable, therefore,  $\Ch_{\mssd, \nu}$ is not necessarily well-defined. The latter energy~$\Ch^a_{\mssd, \nu}$ is, however,  well-defined for an extended metric finite-measure space~$(X, \tau, \mssd, \nu)$, see~\cite[Sec.~6]{AmbErbSav16}.
%Let~$(X,\T)$ be a Hausdorff topological space. A family of pseudo-distances~$\UP$ is a \emph{uniformity} (\emph{of pseudo-distances}) if:
%\begin{enumerate*}[$(a)$]
%\item it is directed, i.e., $\mssd_1\vee \mssd_2\in\UP$ for every~$\mssd_1,\mssd_2\in \UP$;
%and
%\item it is order-closed, i.e., $\mssd_2\in \UP$ and~$\mssd_1\leq \mssd_2$ implies~$\mssd_1\in \UP$ for every pseudo-distance~$\mssd_1$ on~$X$.
%\end{enumerate*}
%%
%A uniformity is %\emph{bounded} if every~$\mssd\in \UP$ is bounded;
%\emph{Hausdorff} if it separates points.
%
%The next definition is a reformulation of~\cite[Dfn.~4.1]{AmbErbSav16}.
%
%\begin{defs}[Extended metric-topological space]\label{d:AES}
%Let~$(X,\T)$ be a Hausdorff topological space. An extended pseudo-distance~$\mssd\colon X^{\tym{2}}\rar [0,\infty]$ is $\T$-\emph{admissible} if there exists a uniformity~$\UP$ of \emph{$\T^\tym{2}$-continuous} pseudo-distances~$\mssd'\colon X^\tym{2}\rar [0,\infty)$, so that
%\begin{align}\label{eq:d=supUP}
%\mssd=\sup\set{\mssd':\mssd'\in\UP}\fstop
%\end{align}
%The triple $(X,\T,\mssd)$ is an \emph{extended metric-topological space} if~$\mssd$ is $\T$-admissible, and there exists a uniformity~$\UP$ witnessing the $\T$-admissibility of~$\mssd$, and additionally Hausdorff and generating~$\T$.
%\end{defs}
%
%Let~$\mssd\colon X^\tym{2}\to[0,\infty]$ be an extended pseudo-distance on~$X$.
%We denote by~$\T_\mssd$ the topology induced by~$\mssd$ and we note that, even in the case when~$\mssd$ is $\T$-admissible, $\T_\mssd$ is in general strictly finer than~$\T$.
%If~$\mssd$ is a distance however, then~$\T_\mssd=\T$.

%\subsection*{Entropy, Wasserstein space}
\subsection{Riemannian Curvature-dimension condition}
%The following definition is an equivalent characterisation of $\RCD(K,\infty)$ by \cite[Cor. 4.18]{AmbGigSav15}.
%\begin{defs}[$\RCD(K,\infty)$] \label{d:RCD}

Let $(X, \mssd, \nu)$ be a metric measure space. 
%In this setting, the Cheeger energies $\Ch_{\mssd, \nu}$ and $\Ch^a_{\mssd, \nu}$ coincide (\cite{AmbColDiM15}) and we simply write $\Ch_{\mssd, \nu}$ in this subsection. 
 %and $(\E, \F)$ be a strongly local Dirichlet form $(\E, \F)$ having a square field $\Gamma$. 
 The following definition is an equivalent characterisation of $\RCD(K,\infty)$ by \cite[Cor.\ 4.18]{AmbGigSav15}.
We say that $(X, \mssd, \nu)$ satisfies the {\it Riemannian Curvature-Dimension Condition} $\RCD(K,\infty)$ for $K \in \R$ if 
\begin{enumerate}[{\rm (i)}]
\item $\Ch_{\mssd, \nu}$ is quadratic, i.e., $\Ch_{\mssd, \nu}(u+v)+\Ch_{\mssd, \nu}(u-v)= 2\Ch_{\mssd, \nu}(u)+2\Ch_{\mssd, \nu}(v)$; 
\item Sobolev-to-Lipschitz property holds, i.e., every $u \in W^{1,2}(X, \mssd, \nu)$ with $|\nabla u|_* \le 1$ has a $\mssd$-Lipschitz $\nu$-representative $\tilde{u}$ with $\Lip(\tilde{u}) \le 1$;
\item $\Ch_{\mssd, \nu}$ satisfies $\BE_2(K,\infty)$, i.e., $|\nabla T_t u|^2_* \le e^{-2Kt} T_t|\nabla u|^2_*$ for every $u \in W^{1,2}(X, \mssd, \nu)$ and $t>0$.  
\end{enumerate}
In this case, the Cheeger energy $\Ch_{\mssd, \nu}$ is a local Dirichlet form (\cite[\S 4.3]{AmbGigSav14b}). We note that, while \cite[Cor.\ 4.18]{AmbGigSav15} is stated in terms of {\it the minimal weak upper gradient} denoted by $|\nabla\cdot|_w$, it is identical to the minimal relaxed slope $|\nabla \cdot|_*$ due to \cite[Thm.~6.2]{AmbGigSav14}.
 %and the corresponding $L^2$-semigroup is denoted by $\{H_t\}_{t \ge 0}$. 
% \purple{Put the definition of CD as well}



%\end{enumerate}
%\end{defs}

\subsection{Configuration spaces}%Let~$\mcX$ be a topological local structure.
%Let~$\mcX$ be a topological local structure. 
%Let $\R^n$ be the $n$-dimensional Euclidean space, $\mssd$ be the standard Euclidean metric, $\tau$ be the topology generated by open $\mssd$-balls, $\A$ be the Borel $\sigma$-algebra, $\mssm$ be the $n$-dimenisonal Lebesgue measure, $\msE$ be the family of all compact sets in $\R^n$. We write $\mcX:=(\R^n, \tau, \A, \mssd, \mssm, \msE)$. 
A \emph{configuration} on a locally compact Polish space~$X$ is any $\overline\N_0$-valued Radon measure~$\gamma$ on~$X$, which can be expressed by $\gamma = \sum_{i=1}^N \delta_{x_i}$ for $N \in \overline{\N}_0$, where $x_i \in X$ for every $i$ and  $\gamma \equiv 0$ if $N=0$.
%$\delta_{\emptyset} \equiv 0$ denotes the empty configuration.  
%By assumption on~$\mcX$, cf.\ e.g.~\cite[Cor.~6.5]{LasPen18},
%\begin{align*}
%\gamma= \sum_{i=1}^N \delta_{x_i} \comma \qquad N\in \overline\N_0\comma \qquad \seq{x_i}_{i\leq N}\subset X \fstop
%\end{align*}
%In particular, we allow for~$x_i=x_j$ when~$i\neq j$.
%
%~$\gamma_x\eqdef\gamma\!\set{x}$\footnote{Is it used in this paper? $\rightarrow$ Comment somewhere that $\QP_r^\eta$ does not have multiple configuration},~$x\in \gamma$ whenever~$\gamma_x>0$, and
%Write the {\it evaluation} $\gamma(A)$ for a subset $A \subset X$. 
%\begin{defs}[Configuration spaces] 
The \emph{configuration space}~$\U=\dUpsilon(X)$ is the space of all configurations over~$X$. The space~$\dUpsilon$ is equipped with the vague topology, i.e., the topology generated by the duality of the space $C_0(X)$ of continuous functions with compact support. We write the {\it restriction}~$\gamma_A\eqdef \gamma\mrestr{A}$ for a Polish subspace~$A \subset X$ and the corresponding restriction map is denoted by 
\begin{align}\label{eq:ProjUpsilon}
\pr_A\colon \dUpsilon\longrar \dUpsilon(A)\colon \gamma\longmapsto \gamma_{A}\fstop
\end{align}
%
The $N$-particle configuration space is denoted by
\begin{equation*}
\begin{aligned}
\dUpsilon^N\ \eqdef&\ \set{\gamma\in \dUpsilon: \gamma(X)=N}\comma
\end{aligned}
\quad N\in\overline\N_0 \fstop
\end{equation*}
Let $\mathfrak S_k$ be the $k$-symmetric group. It can be readily seen that the $k$-particle configuration space~$\U^k$ is isomorphic to the quotient space~$X^{\times k}/\mathfrak S_k$:
\begin{align} \label{e:STS}
\dUpsilon^k\cong X^{\odot k}:=X^{\times k}/\mathfrak S_k \comma \quad k \in \N_0\fstop
\end{align}
The associated projection map from $X^{\times k}$ to the quotient space~$X^{\times k}/\mathfrak S_k$ is denoted by~$\quot_k$. 
%Let the analogous definitions of~$\dUpsilon^\sym{\geq N}$ be given.
For $\eta \in \U$ and $r>0$, we set 
\begin{align} \label{e:CES}
\U_r^\eta:=\{\gamma \in \U: \gamma_{B_r^c}=\eta_{B_r^c}\} \fstop
\end{align}
%Finally set~$\dUpsilon(E)\eqdef \dUpsilon(\msE_E)=\dUpsilon(\A_E)$ for all~$E\in\msE$, and analogously for~$\Upsilon(E)$.
%We endow~$\dUpsilon$ with the $\sigma$-algebra $\A_\mrmv(\msE)$ generated by the functions $\gamma\mapsto \gamma E$ with~$E\in\msE$, and endow~$\dUpsilon$ with the vague topology $\T_\mrmv$, i.e., the convergence in the duality of $\Cz(\msE)$. 
%\end{defs}

%This coincides with the $\sigma$-algebra on~$\dUpsilon$ given in~\cite[Rmk.~1.5]{MaRoe00} because of Definition~\ref{d:LS}\iref{i:d:LS:2}.

%\begin{defs}[Concentration set]%
%For~$E\in\msE$ we define the $n$-concentration sets of~$E$ as
%\begin{equation}\label{eq:ConcentrationSet}
%\Xi_{=n}(E)\eqdef\ \set{\gamma\in\dUpsilon: \gamma E= n} \comma
%\end{equation}
%and similarly for~`$\geq$' or~`$\leq$' in place of~`$=$'.
%Analogously to the notation established for configuration spaces, we write~$\Xi^\sym{\infty}_{=n}(E)=\Xi_{=n}(E)\cap \dUpsilon^\sym{\infty}$.
%\end{defs}



%\paragraph{Poisson random measures}
%%Contrary to~\cite{LzDSSuz21}, in this work we restrict our attention to configuration spaces endowed with Poisson measures.
%Let $\mssm$ denote the Lebesgue measure on $\R$. For~a compact set $E$, set~$\mssm_E\eqdef \mssm\mrestr{E}$.
%%
%Let~$\mfS_n$ be the $n^\text{th}$ symmetric group, and denote by
%%\begin{equation}\label{eq:ProjSymmetricG}
%$\pr^\sym{n}\colon E^\tym{n}\rar E^\sym{n}\eqdef E^\tym{n}/\mfS_n$
%%\end{equation}
%the quotient projection, and by~$\mssm_E^\sym{n}$ the quotient measure~$\mssm_E^\sym{n}\eqdef \pr^\sym{n}_\pfwd \mssm_E^{\hotym{n}}$.
%The space~$E^\sym{n}$ is naturally isomorphic to~$\dUpsilon^\sym{n}(E)$.
%Under this isomorphism, define the \emph{Poisson--Lebesgue measure of~$\PP_E$} on~$\dUpsilon^{\sym{<\infty}}(E)$, cf.\ e.g.~\cite[Eqn.~(2.5),~(2.6)]{AlbKonRoe98}, by
%\begin{align}\label{eq:PoissonLebesgue}
%\PP_E\eqdef e^{-\mssm E}\sum_{n=0}^\infty \frac{\mssm_E^\sym{n}}{n!} \fstop
%\end{align}
%%\begin{defs}[Poisson measures]\label{d:Poisson}
%The \emph{Poisson} (\emph{random}) \emph{measure~$\PP$} is the unique probability measure on~$\dUpsilon$ satisfying% either of the following equivalent conditions:
%\begin{align}\label{eq:PoissonRestriction}
%\pr^{E}_\pfwd \PP=\PP_{E}\comma \qquad E\in\msE \fstop
%\end{align}
%\begin{align}\label{eq:LaplacePoisson}
%\int_{\dUpsilon} e^{f^\trid\gamma} \diff\PP_\mssm(\gamma)=\exp\paren{\int_X (e^{f}-1) \diff\mssm}\comma \qquad f\in \Sb(X)^+\fstop
%\end{align}
%
%\begin{itemize}
%\item the \emph{projective-limit characterization} 
%\begin{align}\label{eq:PoissonRestriction}
%\pr^{E}_\pfwd \PP_\mssm=\PP_{\mssm_E}\comma \qquad E\in\msE \fstop
%\end{align}
%\item the \emph{Mecke identity}~\cite[Satz~3.1]{Mec67}, viz.
%\begin{align}\label{eq:Mecke}
%\iint_{\dUpsilon\times X} u(\gamma, x) \diff\gamma(x) \diff\PP_\mssm(\gamma)= \iint_{\dUpsilon\times X} u(\gamma+\delta_x,x) \diff\mssm(x) \diff\PP_\mssm(\gamma)
%\end{align}
%for every bounded $\A_\mrmv(\msE)\hotimes \A$-measurable~$u\colon \dUpsilon\times X\rar \R$.
%\item the \emph{Laplace transform characterization}, cf.~\cite[Thm.~3.9]{LasPen18},
%\begin{align}\label{eq:LaplacePoisson}
%\int_{\dUpsilon} e^{f^\trid\gamma} \diff\PP_\mssm(\gamma)=\exp\paren{\int_X (e^{f}-1) \diff\mssm}\comma \qquad f\in \Sb(X)^+\fstop
%\end{align}
%\end{itemize}
%\end{defs}
%We note that 
%The Poisson measures has a different concentration set according to the finiteness of the total measure of $\mssm$:
%\begin{itemize}
%\item 
%if~$\mssm X=\infty$, then~$\PP$ is concentrated in~$\Upsilon^\sym{\infty}$, viz.\ $\PP \Upsilon^\sym{\infty}=1$;
%if~$\mssm X<\infty$, then~$\PP$ is concentrated in~$\Upsilon^\sym{<\infty}$, viz.\ $\PP \Upsilon^\sym{<\infty}=1$, and~$\PP$ coincides with the right-hand side of~\eqref{eq:PoissonLebesgue} with~$X$ in place of~$E$.
%\end{itemize}
%Everywhere in the following we omit the specification of the intensity measure whenever apparent from context, thus writing~$\PP$ in place of~$\PP_\mssm$, and~$\PP_E$ in place of~$\PP_{\mssm_E}$. 

%\paragraph{Intensity} For a probability measure $\QP$ on the space $\ttonde{\dUpsilon,\A_\mrmv(\msE)}$, we denote by~$\mssm_\QP$ its  (\emph{mean}) \emph{intensity} (\emph{measure}), defined by
%\begin{align} \label{eq: int}
%\mssm_\QP A \eqdef \int_\dUpsilon \gamma A \diff\QP(\gamma) \comma \qquad A\in\A_\mrmv(\msE)\fstop
%\end{align}
%For many results here and later on, we shall make the following assumption.
%\begin{ass}\label{d:ass:Mmu}
%We say that~$\QP$ satisfies \ref{ass:Mmu} if it has $\msE$-locally finite intensity, viz.\
%\begin{equation}\tag*{$(\mssm_\QP)_{\ref{d:ass:Mmu}}$}\label{ass:Mmu}
%\mssm_\QP E<\infty \comma \quad E\in\msE\fstop
%\end{equation}
%\end{ass}
%%\begin{ese} \label{exa: FI}
%Assumption~\ref{ass:Mmu} is a natural condition from the following viewpoints:
%from a mathematical point of view, it implies that we have sufficiently many functions in~$L^1(\QP)$;
%from a physical point of view, it implies that any system of randomly $\QP$-distributed particles is $\msE$-locally finite in average. 
%\purple{Example}
%\end{ese}
%\subsubsection{Dirichlet forms}
%Let us briefly recall the construction and main analytical properties of the Dirichlet form~\eqref{eq:DirichletForm} constructed in~\cite{LzDSSuz21}.
%We specialize all the st

%\paragraph{Ensemble $\sine_\beta$ }

%\paragraph{Metric structures on $\dUpsilon$}
%In this section, we recall the metric structure on $\dUpsilon$. 

\paragraph{Conditional probability}Let $\mu$ be a Borel probability measure on $\U$. 
Let 
$$\mu(\cdot \ | \ \pr_{B_r^c}(\cdot)=\eta_{B_r^c})$$ denote the regular conditional probability of $\mu$ conditioned at $\eta \in \U$ with respect to the $\sigma$-field generated by the projection map $\gamma \in \U \mapsto \pr_{B_r}(\gamma):=\pr_r(\gamma):=\gamma_{B_r} \in \U(B_r)$ (see e.g., \cite[Def.~3.32]{LzDSSuz21} for the precise definition).  Let~$\mu_{r}^\eta$ be the probability measure on~$\U(B_r)$ defined as 
\begin{align} \label{d:CPB}
\mu_{r}^\eta:=(\pr_{r})_\#\mu(\cdot \ | \ \pr_{B_r^c}(\cdot)=\eta_{B_r^c}) \comma
\end{align}
and its restriction on~$\U^k(B_r)$ is denoted by~$\mu_{r}^{k, \eta}:=\mu_{r}^\eta|_{\U^k(B_r)}$.
\vspace{2mm}\\
{\sf Note}: The conditional probability~$\mu(\cdot \ | \ \pr_{B_r^c}(\cdot)=\eta_{B_r^c})$ is a probability measure on the whole space~$\U$ whose support is contained in~$\U_r^\eta=\{\gamma \in \U: \gamma_{B_r^c}=\eta_{B_r^c}\}$. 
We may project the conditional probability to the probability measure~$\mu_r^\eta$ on $\U(B_r)$ as in~\eqref{d:CPB} without loss of information in the sense that  
%Although the conditional probability~$\mu_{r}^\eta$ is a probability measure on the whole space~$\U$ whose support is $\U_r^\eta=\{\gamma \in \U: \gamma_{B_r^c}=\eta_{B_r^c}\}$, without loss of information we may think of $\mu_{r}^\eta$  to be a probability measure on~$\U(B_r)$ indexed by $\eta$, which is, to be precise,  the push-forwarded measure~$(\pr_r)_\#\mu_r^\eta$. 
%This identification is justified because 
\begin{align} \label{r:BMP}
\pr_r: \U_r^\eta \to \U(B_r) \ \text{is  a bi-measure-preserving bijection} \fstop
\end{align}
Namely, the projection map~$\pr_r$ is bijective with the inverse map $\pr_r^{-1}$ defined as~$\pr^{-1}_r(\gamma):=\gamma+\eta$, and both $\pr_r$ and~$\pr_r^{-1}$ are measure-preserving between the two measures~$\mu(\cdot \ | \ \pr_{B_r^c}(\cdot)=\eta_{B_r^c})$ and $\mu_{r}^\eta$.  
%\end{rem}
%does not lose information of~$\mu_r^\eta$ since the conditional probability~$\mu_{r}^\eta$ is supported only on the set~$\U_{r}^\eta:=\{\gamma \in \U: \gamma_{B_r^c}=\eta_{B_r^c}\}$. 
%Hereinafter, we will not distinguish these two measures for the sake of the notational simplicity and we will consider $\mu_{r}^\eta$ to be a probability measure on $\U$ as well as on $\U(B_r)$, while not changing the symbol. 
%For the sake of later use, we display the following proposition, which is a straightforward consequence of the property of the conditional probability. 
%which is well-defined also for $\mu$-equivalent classes of $\A_\mrmv(\msE)$-measurable functions for $\mu^\eta_E$-a.e.\ $\gamma$ and $\mu$-a.e.\ $\eta$. 
%\begin{prop}[E.g., {\cite[Prop.\ 3.44]{LzDSSuz21}}]
\smallskip

For a measurable function $ u\colon \dUpsilon\to \R$, $r>0$ and for $\eta \in \dUpsilon$, we set 
%\begin{equation*}
 \begin{align} \label{e:SEF}
u_{r}^\eta(\gamma)\eqdef  u(\gamma+\eta_{B_r^\complement})  \qquad \gamma\in \dUpsilon(B_r) \fstop
 \end{align}
%\end{equation*}
By  the property of the conditional probability, it is straightforward to see that for any $u \in L^1(\mu)$, 
\begin{align} \label{p:ConditionalIntegration}
\int_{\dUpsilon} u \diff\QP = \int_{\dUpsilon} \quadre{\int_{\dUpsilon(B_r)} u_{r}^\eta \diff \QP^\eta_r }\diff\QP(\eta) \fstop
\end{align}
See, e.g., \cite[Prop.\ 3.44]{LzDSSuz21}. For a measurable set~$\Omega \subset \U$, define a {\it section}~$\Omega_r^\eta \subset \U(B_r)$ at~$\eta \in \U$ on~$B_r^c$ by
 \begin{align} \label{e:SEF2}
 \Omega_r^\eta:=\{\gamma \in \U(B_r): \gamma+\eta_{B_r^c} \in \Omega\} \fstop
 \end{align}
%The section~$\Omega_r^\eta$ is $\QP_r^\eta$-measurable (see, e.g.,~\cite[Lem.\ 3.1]{BruSuz21}) and
 By applying the disintegration formula~\eqref{p:ConditionalIntegration} to $u=\1_{\Omega}$, we obtain
\begin{align} \label{p:ConditionalIntegration2}
\QP(\Omega)=\int_{\U} \QP_r^\eta(\Omega_r^\eta) \diff \QP(\eta) \fstop
\end{align}
%\end{prop}

\paragraph{Poisson measure}Let $(X, \tau, \nu)$ be a locally compact Polish space with Radon measure~$\nu$ satisfying~$\nu(X)<\infty$.  {\it The Poisson measure} $\pi_{\nu}$ on~$\U(X)$ with intensity~$\nu$ is defined in terms of the symmetric tensor measure $\nu^{\odot}$ as follows:
\begin{align} \label{d:PS}
\pi_{\nu}(\cdot):=e^{-\nu(X)}\sum_{k=1}^\infty \nu^{\odot k}\bigl(\cdot \cap {\U^k(X)}\bigr)=e^{-\nu(X)}\sum_{k=1}^\infty \frac{1}{k!}(\quot_k)_\#\nu^{\otimes k}\bigl(\cdot \cap {\U^k(X)} \bigr) \fstop
\end{align}
%When $\nu(X)=\infty$ and $\nu(B_r)<\infty$ on every metric ball $B_r=B_r(o)$ with radius $r>0$ centred at some fixed point $o$, {\it the Poisson measure $\pi_{\nu}$ on~$\U(X)$ with intensity~$\nu$} is defined as the projective limit 
%$$\pi_{\nu}:={\rm projlim}_{r \to \infty}\pi_{\nu|_{B_r}}\comma$$ with respect to the projection maps~$\pr_{r, s}: \U(B_r) \to \U(B_s)$ defined as $\gamma \mapsto \pr_{r, s}(\gamma):=\gamma_{B_s}$ for every $s \le r$. 

\paragraph{$L^2$-transportation distance}Let $(X, \mssd)$ be a locally compact complete separable metric space. For~$i=1,2$ let~$\proj_i\colon X^{\times 2}\rar X$ denote the projection to the~$i^\text{th}$ coordinate for $i=1,2$. 
For~$\gamma,\eta\in \dUpsilon$, let~$\Cpl(\gamma,\eta)$ be the set of all couplings of~$\gamma$ and~$\eta$, i.e., 
\begin{align*}
\Cpl(\gamma,\eta)\eqdef \set{\cpl\in \Meas(X^{\tym{2}}) \colon (\proj_1)_\pfwd \cpl =\gamma \comma (\proj_2)_\pfwd \cpl=\eta} \fstop
\end{align*}
%Let $E \subset X$ be a subset. For~$i=1,2$ let~$\pr^i\colon E^{\times 2}\rar E$ denote the projection to the~$i^\text{th}$ coordinate.
Here $\Meas(X^{\tym{2}})$ denotes the space of all Radon measures on $X^{\tym{2}}$. 
%For~$\gamma,\eta\in \dUpsilon$, let~$\Cpl(\gamma,\eta)\subset \Meas(X^{\tym{2}},\A^{\hotimes 2}|_{X^{\tym 2}})$ be the set of all couplings of~$\gamma$. %and~$\eta$, viz.
%\begin{align*}
%\Cpl_E(\gamma,\eta)\eqdef \set{\cpl\in \Meas(E^{\tym{2}},\A^{\hotimes 2}|_{E^{\tym 2}}) \colon \pr^1_\pfwd \cpl =\gamma \comma \pr^2_\pfwd \cpl=\eta} \fstop
%\end{align*}
%\begin{defs}[$L^2$-transportation distance]
The \emph{$L^2$-transportation}  \emph{extended distance} on~$\dUpsilon(X)$ is
\begin{align}\label{eq:d:W2Upsilon}
\mssd_{\dUpsilon}(\gamma,\eta)\eqdef \inf_{\cpl\in\Cpl(\gamma,\eta)} \paren{\int_{X^{\times 2}} \mssd^2(x,y) \diff\cpl(x,y)}^{1/2}\comma \qquad \inf{\emp}=+\infty \fstop
\end{align}
%When $E=\R^n$, we simply write $\mssd_{\dUpsilon}$ instead of $\mssd_{\dUpsilon(\R^n)}$. 
%\end{defs}
We refer the readers to e.g., \cite[Prop.~4.27, 4.29, Thm.~4.37, Prop.~5.12]{LzDSSuz21} and \cite[Lem.~4.1, 4.2]{RoeSch99} for details regarding the $L^2$-transportation extended distance $\mssd_{\dUpsilon}$ and examples of $\mssd_{\dUpsilon}$-Lipschitz functions. 
It is important to note that $\mssd_\dUpsilon$ is an {\it extended} distance, attaining the value~$+\infty$ and $\mssd_\U$ is lower semi-continuous with respect to the product vague topology $\tau_\mrmv^{\times 2}$ but never $\tau_\mrmv^{\times 2}$-continuous. 

We introduce a variant of the $L^2$-transportation extended distance, called \emph{$L^2$-transportation-type}  \emph{extended distance}~$\bar{\mssd}_\U$ defined as 
\begin{align} \label{eq:dW2L}
\bar{\mssd}_\U(\gamma, \eta):=
\begin{cases}
\mssd_\U(\gamma, \eta) \quad &\text{if $\gamma_{B_r^c}=\eta_{B_r^c}$ for some $r>0$\ ,}
\\
+\infty \quad  &\text{otherwise} \fstop
\end{cases}
\end{align}
By definition, $\mssd_\U \le \bar{\mssd}_\U$ on $\U$,  and  $\mssd_\U = \bar{\mssd}_\U$ on $\U(B_r)$ for any $r>0$. In particular, we have 
\begin{align} \label{e:LLR}
\Lip(\U, \mssd_\U) \subset \Lip(\U, \bar{\mssd}_\U) \comma \quad \Lip_{\bar{\mssd}_\U}(u) \le  \Lip_{\mssd_\U}(u)\comma  \quad u \in \Lip(\U, \mssd_\U)\fstop
\end{align}
It can be readily seen that 
\begin{align} \label{e:LLR2}
\bar{\mssd}_\U(\gamma, \eta) <\infty \quad \iff \quad \gamma_{B_r^c}=\eta_{B_r^c} \comma \gamma(B_r)=\eta(B_r) \quad \text{for some $r>0$}\fstop
\end{align}

When we work with the configuration space~$\U(\R^n)$ over~the $n$-dimensional Euclidean space~$\R^n$ or over any Polish subset in $\R^n$, we always choose the Euclidean distance~$\mssd(x, y)=|x-y|$ and the $L^2$-transportation distance~$\mssd_{\U}$ and $\bar{\mssd}_\U$ associated with~$\mssd$.  


%on a set of positive~$\QP^\otym{2}$-measure in~$\dUpsilon^\tym{2}$ for any reasonable probability measure $\QP$. 

\subsection{$\sine_\beta$ ensemble} \label{subsec:SB}
Let $\beta>0$ and $\CBE_k$ be {\it the circular $\beta$ ensemble} on the $k$-particle configuration space, i.e., it is the probability measure $\mathbb P_{k, \beta}$ on the space~$\U^k(\mathbb S^1)$ over the unit circle $\mathbb S^1 \subset \mathbb C$ defined as
$$\diff \mathbb P_{k, \beta} := \frac{1}{Z_{k, \beta}}\prod_{1 \le j<l \le k}\bigl| e^{i\theta_j}- e^{i\theta_l}\bigr|^\beta\frac{\diff \theta_1}{2\pi} \cdots\frac{\diff \theta_k}{2\pi} \comma$$
where the normalisation constant~$Z_{k, \beta}$ is given in terms of Gamma function~$\Gamma$:
$$Z_{k, \beta}:=\frac{\Gamma(\frac{1}{2}\beta k +1)}{\Gamma(\frac{1}{2}\beta k +1)^k} \fstop$$
According to \cite[Def.\ 1.6]{KilSto09}, the {\it circular $\beta$ ensemble} $\CBE$  is defined as the limit probability measure $\mathbb P_{\beta}$ whose Laplace transform is determined as 
$$\int \exp\Bigl(-\sum_{x \in \gamma} f(x) \Bigr) \diff \mathbb P_{\beta}(\gamma) = \lim_{k \to \infty} \int \exp\Bigl( - \sum_{i=1}^kf(k \theta_i)\Bigr) \diff \mathbb P_{k ,\beta}(\theta_1, \ldots, \theta_k) \comma$$
for all $f \in C_0(\R)$.
%\begin{defs}[Rigidity in number: Ghosh--Peres {\cite[Thm.\ 1]{GhoPer17}}]\label{ass:Rigidity}
In \cite{ValVir09}, a Borel probability measure $\mu_\beta$ on $\U(\R)$ called {\it sine $\beta$ ensemble} has been constructed by a limit of Gaussian $\beta$-ensemble. These two measures $\mathbb P_\beta$ and $\mu_\beta$ turned out to be identical each other by the work of~\cite{Nak14}. Throughout the rest of the article, we use the symbol~$\QP=\mu_\beta$ to denote $\sine_\beta$~ensemble (equivalently, circular $\beta$~ensemble) and we do not specify the inverse temperature $\beta$ as there is no particular role played by a special $\beta$. 



\smallskip
\paragraph{Number-rigidity}A Borel probability~$\mu$ on~$\U=\U(\R^n)$ is said to be {\it number rigid} (in short: \ref{ass:Rig}) if for any bounded domain $E \subset \R^n$, there exists $\Omega \subset \dUpsilon$ so that $\QP(\Omega)=1$  and, for any $\gamma, \eta \in \Omega$
\begin{align*}\tag*{$(\mathsf{R})$}\label{ass:Rig}
\text{$\gamma_{E^c} = \eta_{E^c}$ implies $\gamma E = \eta E$}  \fstop
\end{align*}
Namely, the configuration outside $E$ determines the number of particle inside $E$. The number-rigidity has been proven in~\cite{Gho15} for the $\sine_2$ ensemble and in \cite{NajRed18}, \cite{DerHarLebMai20} for the $\sine_\beta$ ensemble for general $\beta>0$.
%\end{defs}
%%\begin{ese} \label{exa: R}
%
%The rigidity in number has been verified for a variety of point processes: Ginibre and GAF (\cite{GhoPer17}), $\mathrm{sine}_\beta$ (\cite[Thm.\ 4.2]{Gho15}, \cite{NajRed18}, \cite{DerHarLebMai20}), $\mathrm{Airy}$, ~$\mathrm{Bessel}$, and $\mathrm{Gamma}$ (\cite{Buf16}), and Pfaffian (\cite{BufNikQiu19}) point processes. We refer the readers also to the survey \cite{GhoLeb17}. 
%%\end{ese}

%\paragraph{Tail-triviality}Let $\{B_r\}_{r \in \N}$ be the exhaustion of the metric balls $B_r \subset \R^n$ with radius $r$ centred at $o$, and let $\sigma(\pr_{B_r^c})$ denote the $\sigma$-field generated by the projection map $\U \ni \gamma \mapsto \pr_{B_r^c}(\gamma)=\gamma_{B^c_r} \in \U(B_r^c)$. 
%We set $\mathscr T(\U):=\cap_{r \in \N}\sigma(\pr_{B_r^c})$ and call it {\it tail $\sigma$-algebra}. For a set $A \subset \U$,  define $\mathcal T_{B_r}({A}):=(\pr_{B_r^c})^{-1}\circ \pr_{B_r^c}(A)$. By definition, $A \subset \mathcal T_{B_r}(A)$, and $\mathcal T_{B_r}({A}) \subset \mathcal T_{B_{r'}}({A})$ whenever $r \le r'$. Define {\it the tail set of $A$ with respect to $\{B_r\}_{r \in \N}$} by 
%\begin{align} \label{eq: ts}
%\mathcal T(A):=\cup_{r \in \N} \mathcal T_{B_r}({A}) \fstop
%\end{align}
%%\limsup_{n \to \infty} \bar{A}_n:=\cap_{n \ge 1} \cup_{j \ge n} \bar{A}_j.$$ 
%%It is simple to check that the tail set $\mathcal T(A)$ of $A$ does not depend on the choice of countable exhaustions $\seq{E_h}$. 
%It is straightforward to see that $\mathcal T(A) \in \mathscr T(\U)$ and $A \subset \mathcal T({A})$. 
%A Borel probability measure $\mu$ on $\U(X)$ is called {\it tail trivial} (in short:~\ref{ass:TT}) if 
%\begin{align*}\tag*{$(\mathsf{T})$}\label{ass:TT}
%\mu(A)\in \{0, 1\} \qquad \forall A \in \mathscr T(\U) \fstop
%\end{align*}
%All determinantal point processes whose kernel are locally trace-class positive contraction satisfy the tail triviality (see \cite[Theorem 2.1]{Lyo18} and \cite{BufQiuSha21,  OsaOsa18, ShiTak03b}). In particular, $\sine_2$ is tail-trivial. 

\section{Curvature bound for finite-particle systems} \label{sec: Pre}
In this section, we study Dirichlet forms on the configuration space~$\U(B_r)$ over metric balls~$B_r \subset \R$. We denoted by $\mssm$ and $\mssm_r$ the Lebesgue measure on $\R$ and its restriction on the metric ball $B_r:=[-r, r]$ respectively, and take the Euclidean distance $\mssd(x, y):=|x-y|$ for $x, y \in B_r$. 
%In this section, we construct a Dirichlet form whose invariant measure is $\sine_\beta$ by finite-particle approximation. 
%Let $\mu_\beta$ be the $\sine_\beta$ ensemble. Throughout this section, we fix $\beta \ge 1$ and simply write $\mu=\mu_\beta$.
 %For $0<r<R<\infty$, $k \in \N$ and $\eta \in \U(B_r^c)$, define the following finite measure on $\U^k(B_r)$:
%\begin{align}
%&\diff \mu_{r, R}^{k, \eta}(\gamma) := \exp\Bigl(-  \Psi^{k, \eta}_{r, R}(\gamma) \Bigr) \diff \mssm_r^{\odot k}(\gamma) \comma
%\\
%& \Psi_{r, R}^{k, \eta}(\gamma) := - \log \Biggl(  \prod_{i \neq j}^k |x_i-x_j|^\beta \prod_{i=1}^k\prod_{y \in \eta_{B_r^c}, |y| \le R} |1-\frac{x_i}{y}|^\beta\Biggr) \fstop
%\end{align}


\subsection{Construction of Dirichlet forms on $\U^k(B_r)$}
Let $W^{1,2}_s(\mssm^{\otimes k}_r)$ be the space of $\mssm^{\otimes k}_r$-classes of~$(1,2)$-Sobolev and {\it symmetric} functions on the product space $B_r^{\times k}$, i.e., 
$$W^{1,2}_s(\mssm^{\otimes k}_r):=\biggl\{u \in L^2_s(\mssm^{\otimes k}_{r}): \int_{B_r^{\times k}} |\nabla^{\otimes k} u|^2 \diff \mssm^{\otimes k}_{r} <\infty \biggr\} \comma$$
where $\nabla^{\otimes k}$ denotes the weak derivative on $\R^{\times k}$: $\nabla^{\otimes k}u:=(\partial_1 u, \ldots, \partial_ku)$.   
The space $W^{1,2}_s(\mssm^{\otimes k}_r)$ consisting of symmetric functions, the projection $\quot_k: B_r^{\times k} \to \U^k(B_r) \cong B_r^{\times k} /\mathfrak S_k$ naturally acts on $W^{1,2}_s(\mssm^{\otimes k}_r)$ and the resulting quotient space is denoted by $W^{1,2}(\mssm_r^{\odot k})$, which is the $(1,2)$-Sobolev space on $\U^k(B_r)$:
$$W^{1,2}(\mssm^{\odot k}_r):=\biggl\{u \in L^2(\mssm^{\odot k}_{r}): \int_{\U^k(B_r)} |\nabla^{\odot k} u|^2 \diff \mssm^{\odot k}_{r} <\infty \biggr\} \comma$$
% be the quotient space of $W^{1,2}_s(\mssm_r^{\otimes k})$ with respect to the equivalence relation $f \sim g \iff f\circ \lab = g \circ \lab$.  
where $\nabla^{\odot k}$ is the quotient operator of the weak gradient operator $\nabla^{\otimes k}$ through the projection $\quot_k$ and $\mssm_r^{\odot k}$ is the symmetric product measure defined as 
$$\mssm_r^{\odot k}:=\frac{1}{k!} (\quot_k)_\#\mssm_r^{\otimes k} \fstop$$
%also descends to the quotient space $\U^k(B_r)$, which is denoted by $\nabla^{\odot k}$.  %The corresponding quotient Dirichlet form is denoted by 

 For $0<r<R<\infty$, $k \in \N_0$ and $\eta \in \U(B_r^c)$, we introduce the following finite Borel measure on $\U^k(B_r)$: for $\gamma=\sum_{i=1}^{k}\delta_{x_i}$
\begin{align} \label{d:TDLR}
&\diff \mu_{r, R}^{k, \eta}(\gamma) := e^{-\Psi^{k, \eta}_{r, R}(\gamma)}\diff \mssm_r^{\odot k}(\gamma) \comma
\\
& \Psi_{r, R}^{k, \eta}(\gamma) := - \log \Biggl(  \prod_{i < j}^k |x_i-x_j|^\beta \prod_{i=1}^k\prod_{y \in \eta_{B_r^c}, |y| \le R} \Bigl|1-\frac{x_i}{y}\Bigr|^\beta\Biggr) \fstop \notag
\end{align}
%where $Z_{r, R}^{\eta}$ is the normalising constant for the measure $\sum_{k=1}^\infty \mu_{r, R}^{k, \eta}$.
The corresponding weighted Sobolev norm is denoted by 
\begin{align} \label{eq:form1}
\E^{\U(B_r), \mu_{r, R}^{k, \eta}}(u) := \int_{\U^k(B_r)} |\nabla^{\odot k} u|^2 \diff \mu_{r, R}^{k, \eta} \comma \quad u \in \Lip_b(\U^k(B_r), \mssd_{\U}) \comma
%W^{1,2}(\mssm_r^{\odot k}) \fstop 
\end{align}
where we note that as $\Lip(\U^k, \mssd_{\U}) \subset W^{1,2}(\mssm_r^{\odot k})$ and $|\nabla^{\odot k} u| \le \Lip_{\mssd_\U}(u)$ due to the Rademacher theorem descendent from the one in the product Sobolev space~$W^{1,2}(\mssm_r^{\otimes k})$ through the quotient, the expression~$|\nabla^{\odot k} u|$ and its integral against the probability measure~$\mu_{r, R}^{k, \eta}$ make sense for $u \in \Lip_b(\U^k(B_r), \mssd_{\U})$.
\begin{prop} \label{p:form1}
The  form \eqref{eq:form1} is well-defined and closable. The closure is a local Dirichlet form on~$L^2(\mu_{r, R}^{k, \eta})$ and its domain is denoted by $\mathcal D\bigl(\E^{\U(B_r), \mu_{r, R}^{k, \eta}})$. %It holds that $W^{1,2}(\mssm_r^{\odot k}) \subset \mathcal D\bigl(\E_{r, R}^{k, \eta})$.
\end{prop}
\begin{proof}
%See Prop 4.2 in the tensor paper.
The well-definedness follows from the following inequality:
\begin{align} \label{eq:form2}
 \int_{\U^k(B_r)} |\nabla^{\odot k} u|^2 \diff \mu_{r, R}^{k, \eta} \le\Bigl\|e^{-\Psi_{r, R}^{k, \eta}}\Bigr\|_{L^\infty(\U^k(B_r), \mu_{r, R}^{k, \eta})}\int_{\U^k(B_r)} |\nabla^{\odot k} u|^2 \diff \mssm^{\odot k}<\infty \fstop
\end{align}
The closability of $\E^{\U(B_r), \mu_{r, R}^{k, \eta}}$ descends from the closability of the corresponding Dirichlet form on the product space $B_r^{\times k}$ defined on the space of symmetric $\mssd^{\times k}$-Lipschitz functions:  
$$\E^{B_r^{\times k}, \mu_{r, R}^{k, \eta}}:=\int_{B_r^{\times k}} |\nabla^{\otimes k} u|^2 e^{-\Psi_{r, R}^{k, \eta}} \diff \mssm^{\otimes k}_{r}\comma$$
where the closability of~$\E^{B_r^{\times k}, \mu_{r, R}^{k, \eta}}$ is a consequence of the continuity of the density~$e^{-\Psi_{r, R}^{k, \eta}}$ on $B_r^{\times k}$ and the standard Hamza-type argument by \cite{RoeWie85, Fuk97}, see for an accessible reference, e.g., \cite[pp.\ 44-45]{MaRoe90}. The locality of the form is an immediate consequence of the locality of the gradient operator~$\nabla^{\odot k}$.
%Noting that $C^1(\U^k(B_r)) \subset W^{1,2}(\mssm_r^{\odot k})$ and \eqref{eq:form1}, the closability on $W^{1,2}(\mssm_r^{\odot k})$ follows as well. In particular, we have shown that $W^{1,2}(\mssm_r^{\odot k}) \subset  \mathcal D\bigl(\E_{r, R}^{k, \eta})$. 
\end{proof}

Let $\QP$ be the $\sine_\beta$ ensemble. Due to \cite[Thm. 1.1]{DerHarLebMai20}, the following limit exists for $\QP$-a.e.~$\eta$, all $x \in B_r$ and $r>0$:
$$\lim_{R \to \infty}\prod_{y \in \eta_{B_r^c}, |y| \le R} \Bigl|1-\frac{x}{y}\Bigr|^\beta  \fstop$$
%Let $\mu(\cdot \ | \ \pr_{B_r^c}(\cdot)=\eta_{B_r^c})$ denote the regular conditional probability of $\mu$ conditioned at $\eta \in \U$ with respect to the $\sigma$-field generated by the projection map $\gamma \in \U \mapsto \pr_r(\gamma)=\gamma_{B_r} \in \U(B_r)$ (see e.g., \cite[Def.~3.32]{LzDSSuz21} for the precise definition).  Let~$\mu_{r}^\eta$ be the probability measure on~$\U(B_r)$ defined as 
%$$\mu_{r}^\eta:=(\pr_{r})_\#\mu_{r}^\eta \fstop$$
Recall that $\QP_r^\eta$ has been defined in~\eqref{d:CPB}.  By~\cite[Thm. 1.1]{DerHarLebMai20} and the number-rigidity~\ref{ass:Rig} of~$\QP$, 
 %and the number-rigidity~\ref{ass:Rig}, the following measure has been constructed as the weak limit of $\{\mu_{r, R}^{k, \eta}\}_R$ as $R \to \infty$: for $\mu$-a.e.~$\eta$, 
for $\mu$-a.e.~$\eta$ there exists $k=k(\eta)$  so that 
\begin{align} \label{e:R1}
\text{$\mu_{r}^{\eta}(\U^l(B_r))>0$ if and only if $l=k(\eta)$} \comma
\end{align}% which is the weak limit of $\{\mu_{r, R}^{k, \eta}\}_R$ as $R \to \infty$: 
and for $\gamma=\sum_{i=1}^k \delta_{x_i}$, 
\begin{align} \label{d:cp}
&\diff \mu_{r}^{\eta} = \diff \mu_{r}^{k, \eta}=\frac{e^{-\Psi^{k, \eta}_{r}}}{Z_{r}^\eta} \diff \mssm_r^{\odot k}  \comma
\\
& \Psi_{r}^{k, \eta}(\gamma) := - \log \Biggl(  \prod_{i<j}^k |x_i-x_j|^\beta \prod_{i=1}^k\lim_{R \to \infty}\prod_{y \in \eta_{B_r^c}, |y| \le R} \Bigl|1-\frac{x_i}{y}\Bigr|^\beta\Biggr) \comma \notag
\end{align}
where $Z_r^\eta$ is the normalising constant. Note that the roles of the notation~$\gamma$ and $\eta$ in~\cite{DerHarLebMai20} are opposite to this article.  
The corresponding weighted Sobolev norm is defined as 
\begin{align} \label{eq:form2}
\E^{\U(B_r), \mu_{r}^{k, \eta}}(u) := \int_{\U^k(B_r)} |\nabla^{\odot k} u|^2 \diff \mu_{r}^{k, \eta} \comma \quad u \in \Lip_b(\U^k(B_r), \mssd_{\U}) \fstop 
%\in W^{1,2}(\mssm_r^{\odot k}) \fstop 
\end{align}

%\paragraph{Caveat}Strictly speaking, the measure constructed in~\cite[Thm. 1.1]{DerHarLebMai20} is a probability measure on~$\U(B_r)$, which is equivalent to $\sum_{k=0}^\infty \mu_{r}^{k, \eta}$ up to multiplicative constant (normalisation constant). As the constant multiplication does not affect neither the construction of Dirichlet forms nor the Ricci curvature lower bound discussed in the following section (Section~\ref{sec:CBFPS}), we drop the constant multiplication for the purpose of the simplicity here. In Section~\ref{sec:CI}, we shall discuss the conditional probability measure $\mu_r^\eta$ with the normalising constant that is exactly the same probability measure constructed in~\cite[Thm. 1.1]{DerHarLebMai20}.

\begin{prop}\label{p:form2}
Let $\QP$ be the $\sine_\beta$ ensemble for $\beta>0$. The form \eqref{eq:form2} is well-defined and closable for $\mu$-a.e.~$\eta$. The closure is a local Dirichlet form on~$L^2(\mu_{r}^{k, \eta})$ and its domain is denoted by $\dom{\E^{\U(B_r), \mu_{r}^{k, \eta}}}$. 
%It holds that $W^{1,2}(\mssm_r^{\odot k}) \subset \mathcal D\bigl(\E_{r, R}^{k, \eta})$.
\end{prop}
\begin{proof}
As $e^{-\Psi_{r, R}^{k, \eta}} \xrightarrow{R \to \infty} e^{-\Psi_{r}^{k, \eta}}$ uniformly on $\U^k(B_r)$ for $\mu$-a.e.~$\eta$ by \cite[Lem.~2.3 and Proof of Thm.\ 2.1 in~p.~183]{DerHarLebMai20}, the density $e^{-\Psi_{r}^{k, \eta}}$ is continuous on $B_r^{\odot k}$, hence the same proof as Prop.~\ref{p:form1} applies to conclude the statement. 
\end{proof}

%\begin{rem}[number rigidity]
%\end{rem}

%\subsection{Convergence of forms}
%In this subsection, we discuss that the sequence of the forms $\{(\E_{r, R}^{k, \eta}, \mathcal D\bigl(\E_{r, R}^{k, \eta}))\}_R$ converges to $(\E_{r}^{k, \eta}, \mathcal D\bigl(\E_{r}^{k, \eta}))$ as $R \to \infty$ in the sense of Kuwae-Mosco-Shioya \cite{Mos94, KuwShi03}. 
%We recall the necessary notions  below:
%\begin{defs}[Convergence of Hilbert space {\cite[p.\ 611]{KuwShi03}}] \label{d:KS}
%A sequence of a separable Hilbert space $H_n$ is said to converge {\it in the Kuwae-Shioya sense}  to a separable Hilbert space $H_\infty$ if there exists a dense subspace $\mathcal C \subset H_\infty$ and linear maps $\Phi_n: \mathcal C \to H_n$ $(n \in \overline{\N}) $ so that $\Phi_\infty(u)=u$ for any $u \in \mathcal C$ and 
%\begin{align} \label{e:CH}
%\lim_{n \to \infty}\|\Phi_n u\|_{H_n} = \|u\|_{H} \comma \forall u \in \mathcal C\fstop
%\end{align}
%\end{defs}
%
%\begin{defs}[{Strong/weak topology \cite[Defs 2.4, 2.5]{KuwShi03}}] \label{d:SWKS}
%Assume that $H_n$ converges to $H$ in the Kuwae-Shioya sense. 
%\begin{enumerate}[$(i)$]
%\item We say that $u_n \in H_n$ converges {\it strongly} to $u \in H$ if there exists $\{v_m\} \subset \mathcal C$ converging to $u$ strongly in $H$ so that 
%\begin{align}\label{e:SC}
%\lim_{m\to \infty} \limsup_{n \to \infty} \|\Phi_n v_m -u_n\|_{H_n} =0 \fstop
%\end{align}
%\item We say that $u_n \in H_n$ converges {\it weakly} to $u \in H$ if for any strongly converging sequence $v_n \in H_n$ to $v \in H$, 
%\begin{align}\label{e:wC}
%\lim_{n\to \infty}(u_n, v_n)_{H_n} = (u, v)_H \fstop
%\end{align}
%\end{enumerate}
%\end{defs}
%
%\begin{defs}[Convergence of forms {\cite[Def.\ 2.11]{KuwShi03}}] \label{d:KSM}
%Assume that $H_n$ converges to $H$ in the Kuwae-Shioya sense. 
%A sequence of closed forms $(\E_n, \mathcal D(\E_n))$ defined in a Hilbert space $H_n$ is said to converge {\it in the Kuwae-Mosco-Shioya sense}  to a closed form $(\E, \mathcal D(\E))$ defined in $H$ if the following two conditions hold:
%\begin{enumerate}[$\mathrm{(KMS1)}$]
%\item for any  $u_n \in H_n$ weakly converging to $u \in H$, 
%$$\E(u) \le \liminf_{n \to \infty} \E_n(u_n) \,;$$
%\item for any $u \in H$, there exists $u_n \subset H_n$ converging strongly to $u$ so that 
%$$\limsup_{n \to \infty} \E_n(u_n) \le \E(u) \fstop$$
%\end{enumerate}
%\end{defs}
%\begin{prop}[Convergence of semigroups {\cite[Thm.\ 2.4]{KuwShi03}}]\label{p:EMS}
%Assume that $H_n$ converges to $H$ in the Kuwae-Shioya sense. 
%Let $(\E_n, \mathcal D(\E_n))$ and $(\E, \mathcal D(\E))$ be closed forms defined in Hilbert spaces $H_n$ and $H$ respectively. Let $\{T_t^n\}_{t \ge 0}$ and $\{T_t\}_{t \ge 0}$ be the corresponding semigroups on $H_n$ and $H$ respectively. Then, the following are equivalent:
%\begin{enumerate}[$(a)$]
%\item $(\E_n, \mathcal D(\E_n))$ converges  to $(\E, \mathcal D(\E))$ in the Kuwae-Shioya-Mosco sense;
%\item for  any $u_n \in H_n$ converging strongly to $u \in H$ and any $t>0$, 
%$$\text{$T_t^n u_n$ converges strongly to $T_tu$} \fstop$$
%\end{enumerate}
%\end{prop}
%
%We now verify the Kuwae-Mosco-Shioya convergence of $\{(\E_{r, R}^{k, \eta}, \mathcal D\bigl(\E_{r, R}^{k, \eta}))\}_R$  to $(\E_{r}^{k, \eta}, \mathcal D\bigl(\E_{r}^{k, \eta}))$.
%\begin{thm} \label{p:Mosco2}
%The form $(\E_{r, R}^{k, \eta}, \mathcal D\bigl(\E_{r, R}^{k, \eta}))$ converges to $(\E_{r}^{k, \eta}, \mathcal D\bigl(\E_{r}^{k, \eta}))$ as $R \to \infty$  in the Kuwae-Mosco-Shioya sense for $\mu$-a.e.\ $\eta$, any $r>0$ and any $k \in \N_0$.
%\end{thm}
%\begin{proof}
%\paragraph{Step 1: Convergence of Hilbert spaces $L^2(\QP_{r, R}^{k, \eta})$ to $L^2(\QP_{r}^{k, \eta})$} \\
%Set 
%$$w_{r, R}^{k, \eta}(\gamma):=\prod_{i=1}^k\prod_{y \in \eta_{B_r^c}, |y| \le R} |1-\frac{x_i}{y}|^\beta \quad w_{r}^{k, \eta}(\gamma):=\lim_{R \to \infty}\prod_{i=1}^k\prod_{y \in \eta_{B_r^c}, |y| \le R} |1-\frac{x_i}{y}|^\beta \fstop$$ 
%By \cite[Thm.\ 2.1]{DerHarLebMai20}, both $w_{r, R}^{k, \eta}, w_{r}^{k, \eta}$ are strictly positive, i.e., for $r>0$, $k \in \N_0$ and $\mu$-a.e.\ $\eta$, there exists a constant $c_r^{k, \eta}>0$ independent of $R$ so that 
%\begin{align} \label{e:Mosco2-1}w_{r, R}^{k, \eta} \wedge w_{r}^{k, \eta}>c_r^{k, \eta}>0 \fstop 
%\end{align}
%Noting also that $w_{r, R}^{k, \eta}, w_{r}^{k, \eta}$ are bounded functions on $\U^k(B_r)$, we obtain the equivalence $L^2(\QP_{r, R}^{k, \eta}) \cong L^2(\QP_{r}^{k, \eta})$ as Banach space. 
%%(caveat: not necessarily equivalent as Hilbert space). 
%We now verify \eqref{e:CH} by taking $\mathcal C=L^2(\QP_{r}^{k, \eta})$ and $\Phi_R: \mathcal C \to L^2(\QP_{r, R}^{k, \eta})$ being $\Phi_R(u)=u$. For simplicity, set $\|\cdot\|_R:=\|\cdot\|_{L^2(\QP_{r, R}^{k, \eta})}$ and $\|\cdot\|_\infty:=\|\cdot\|_{L^2(\QP_{r}^{k, \eta})}$. By \cite[Thm.\ 2.1]{DerHarLebMai20}, the densities $\Psi_{r, R}^{k, \eta}$ converge to $\Psi_{r}^{k, \eta}$ as $R \to \infty$ uniformly on $\U^k(B_r)$ for $\mu$-a.e.\ $\eta$, any $r>0$ and any $k \in \N_0$. Therefore,  
%$$\lim_{R \to \infty}\|\Phi_R u\|_{R} = \lim_{R \to \infty}\|u\|_{R} = \lim_{R \to \infty} \int_{\U^k(B_r)} u^2 \Phi_{r, R}^{k, \eta} \diff \mssm^{\odot k} =  \int_{\U^k(B_r)} u^2 \Phi_{r}^{k, \eta} \diff \mssm^{\odot k} = \|u\|_\infty \comma$$
%which concludes Step 1. 
% 
%%Since the densities $\Psi_{r, R}^{k, \eta}$ converge to $\Psi_{r}^{k, \eta}$ as $R \to \infty$ uniformly on $\U^k(B_r)$ for $\mu$-a.e.\ $\eta$, any $r>0$ and any $k \in \N$ by \cite[p.183, Proof of Thm.\ 2.1]{DerHarLebMai20}, the statement follows from \cite[\S.\ 5]{Mos94} (see also \cite[Example 4.3]{Hin98} for a readily accessible proof). 
%
%\paragraph{Step 2: Convergence of forms: $\mathrm{(KMS2)}$}We show $\mathrm{(KMS2)}$ in Def.\ \ref{d:KSM}. For the same reason as $L^2(\QP_{r, R}^{k, \eta}) \cong L^2(\QP_{r}^{k, \eta})$, we have the equivalence of the domains $\mathcal D\bigl(\E_{r, R}^{k, \eta}) \cong \mathcal D\bigl(\E_{r}^{k, \eta})$. Thus, for $u \in \mathcal D\bigl(\E_{r}^{k, \eta})$, we can take $u_R:=u \in \mathcal D\bigl(\E_{r, R}^{k, \eta})$, for which we can readily verify that $u_R=u \in \mathcal D\bigl(\E_{r, R}^{k, \eta})$ converges to $u \in \mathcal D\bigl(\E_{r}^{k, \eta})$ strongly  in the senes of Def.\ \ref{d:SWKS}, and noting again the uniform convergence of the densities $\Psi_{r, R}^{k, \eta}$  to $\Psi_{r}^{k, \eta}$, we have 
%$$\lim_{R \to \infty}\E_{r, R}^{k, \eta}(u_R)= \lim_{R \to \infty} \int_{\U^k(B_r)}|\nabla^{\odot k} u|^{2} \Psi_{r, R}^{k, \eta} \diff \mssm^{\odot k} = \int_{\U^k(B_r)}|\nabla^{\odot k} u|^{2}  \Psi_{r}^{k, \eta} \diff \mssm^{\odot k} = \E_{r}^{k, \eta}(u) \comma$$
%which concludes Step 2. 
%
%\paragraph{Step 3: Convergence of forms: $\mathrm{(KMS1)}$}We show $\mathrm{(KMS1)}$ in Def.\ \ref{d:KSM}. Take $u_R \in L^2(\QP_{r, R}^{k, \eta})$ converging weakly to $u \in L^2(\QP_{r}^{k, \eta})$. We first show that $u_R$ converges weakly to $u$ as a sequence in the single space $L^2(\QP_{r}^{k, \eta})$.
%Let $v \in L^2(\QP_{r}^{k, \eta})$. Then, we have that $\Phi_R(v)=v \in L^2(\QP_{r, R}^{k, \eta})$ and $\Phi_R(v)$ converges strongly to $v$ in the senes of Def.\ \ref{d:SWKS} as noted above. Noting again the uniform lower-bound \eqref{e:Mosco2-1} away from zero and the uniform convergence of  $w_{r, R}^{k, \eta}$ to $w_{r}^{k, \eta}$ on $\U^k(B_R)$,  we obtain 
%\begin{align} \label{e:Mosco2-00}
%\Bigl\|\frac{w_{r, R}^{k, \eta} }{w_{r}^{k, \eta} }\Bigr\|_{\infty, \U^k(B_r)} \xrightarrow{R \to \infty} 1 \comma
%\\
%\Bigl\|\frac{w_{r}^{k, \eta} }{w_{r, R}^{k, \eta} }\Bigr\|_{\infty, \U^k(B_r)} \xrightarrow{R \to \infty} 1 \fstop \notag
%\end{align}
%Then, $v \cdot \frac{w_{r}^{k, \eta} }{w_{r, R}^{k, \eta}} \in  L^2(\QP_{r, R}^{k, \eta})$ converges strongly to $v \in  L^2(\QP_{r}^{k, \eta})$. Therefore, by \cite[Lem.\ 2.4]{KuwShi03}, we obtain
%\begin{align} \label{e:Mosco2-0}
%(u_R, v)_{\infty} = \biggl( u_R, v \cdot \frac{w_{r}^{k, \eta} }{w_{r, R}^{k, \eta} }\biggr)_R \xrightarrow{R \to \infty} ( u, v )_\infty \comma
%\end{align}
%which concludes that $u_R$ converges weakly to $u$ as a sequence in the single space $L^2(\QP_{r}^{k, \eta})$.
%
%Now we verify $\mathrm{(KMS1)}$ in Def.\ \ref{d:KSM}. Let $M:=\liminf_{R \to \infty}\E_{r, R}^{k, \eta}(u_R)$. We may assume $M<\infty$, otherwise there is nothing to be proven in $\mathrm{(KMS2)}$. Without loss of generality, we may take a (non-relabelled) subsequence $u_R$ so that $\lim_{R \to \infty}\E_{r, R}^{k, \eta}(u_R)=M$.
%Noting again the uniform convergence of the densities $\Psi_{r, R}^{k, \eta}$  to $\Psi_{r}^{k, \eta}$ on $\U^k(B_r)$, we have that 
%$$\lim_{R \to \infty}\E_{r, R}^{k, \eta}(u) = \lim_{R \to \infty}\int_{\U^k(B_r)}|\nabla^{\odot k} u|^{2} \ \Psi_{r, R}^{k, \eta} \diff \mssm^{\odot k} = \int_{\U^k(B_r)}|\nabla^{\odot k} u|^{2} \ \Psi_{r}^{k, \eta} \diff \mssm^{\odot k} = \E_{r}^{k, \eta}(u)\fstop$$
%Thus, it suffices to show 
%\begin{align} \label{e:Mosco2-2}
%\lim_{R \to \infty}\E_{r, R}^{k, \eta}(u) \le \liminf_{R \to \infty}\E_{r, R}^{k, \eta}(u_R). 
%\end{align}
%We see that 
%\begin{align} \label{e:Mosco2-3}
%\E_{r, R}^{k, \eta}(u) &= \int_{\U^k(B_r)}|\nabla^{\odot k} u|^{2} \ \frac{w_{r, R}^{k, \eta} }{w_{r}^{k, \eta} } \Psi_{r}^{k, \eta}  \diff \mssm^{\odot k} \le \Bigl\|\frac{w_{r, R}^{k, \eta} }{w_{r}^{k, \eta} }\Bigr\|_{\infty, \U^k(B_r)} \E_{r}^{k, \eta}(u)  
%\\
%&= \Bigl\|\frac{w_{r, R}^{k, \eta} }{w_{r}^{k, \eta} }\Bigr\|_{\infty, \U^k(B_r)} \E_{r}^{k, \eta}(u) \fstop \notag
%\end{align}
%By noting that $u_R$ converges weakly to $u$ as a sequence in the single space $L^2(\QP_{r}^{k, \eta})$, we have 
%\begin{align} \label{e:Mosco2-000}\E_{r}^{k, \eta}(u)\le \liminf_{R \to \infty}\E_{r}^{k, \eta}(u_R).
%\end{align} 
%Thus, by applying \eqref{e:Mosco2-00} and \eqref{e:Mosco2-000} to \eqref{e:Mosco2-3},  we obtain
%\begin{align} \label{e:Mosco2-4}
%\lim_{R \to \infty}\E_{r, R}^{k, \eta}(u) \le  \liminf_{R \to \infty}\E_{r}^{k, \eta}(u_R) \comma
%\end{align}
%%Noting again the uniform lower-bound \eqref{e:Mosco2-1} away from zero and the uniform convergence of  $w_{r, R}^{k, \eta}$ to $w_{r}^{k, \eta}$ on $\U^k(B_R)$,  
%%By \eqref{e:Mosco2-00}, we obtain 
%%$$ \Bigl\|\frac{w_{r, R}^{k, \eta} }{w_{r}^{k, \eta} }\Bigr\|_{\infty, \U^k(B_r)} \xrightarrow{R \to \infty} 1 \comma$$
% which confirms \eqref{e:Mosco2-2} and concludes Step 3.
%\end{proof}
%
%\smallskip
%Let $\{T_{r, R, t}^{k, \eta}\}_{t>0}$ and $\{T_{r, t}^{k, \eta}\}_{t>0}$ be the $L^2$-semigroups corresponding to $(\E_{r, R}^{k, \eta}, \mathcal D\bigl(\E_{r, R}^{k, \eta}))$ and $(\E_{r}^{k, \eta}, \mathcal D\bigl(\E_{r}^{k, \eta}))$ respectively. 
%By Prop.\ \ref{p:EMS} and Thm.\ \ref{p:Mosco2}, we have the convergence of the corresponding semigroups.
%\begin{cor}\label{c:CS}
%$T_{r, R, t}^{k, \eta}u_R$ converges strongly to $T_{r, t}^{k, \eta}u$ as $R \to \infty$ for  any $u_R \in L^2(\QP_{r, R}^{k, \eta})$ converging strongly to $u \in L^2(\QP_{r}^{k, \eta})$ and $t>0$.
%\end{cor}
%%In this section, we summarise the necessary notions regarding diffusion- and metric topological-properties both for base spaces and configuration spaces. 
%%Throughout this paper, let $X=\R^n$ be the $n$-dimensional Euclidean space, $\mssd$ be the standard Euclidean metric, $\tau$ be the topology generated by open $\mssd$-balls, $\A$ be the Borel $\sigma$-algebra associated with $\tau$, $\mssm$ be the $n$-dimenisonal Lebesgue measure, $\msE$ be the family of all compact sets in $\R^n$. We write $\mcX:=(X, \tau, \A, \mssd, \mssm, \msE)$. 

\subsection{Curvature bound for finite-particle systems} \label{sec:CBFPS}
%Let $\mu$ be $\sine_\beta$. For $r<R$, $k \in \N$ and $\eta \in \U(B_r^c)$, define the following finite measure on $\U(B_r)$:
%\begin{align}
%&\diff \mu_{r, R}^{k, \eta}(\gamma) := \exp\Bigl(-  \Psi^{k, \eta}_{r, R}(\gamma) \Bigr) \diff \mssm^{\odot k} \comma
%\\
%& \Psi_{r, R}^{k, \eta}(\gamma) := - \log \Biggl(  \prod_{i \neq j}^k |x_i-x_j|^\beta \prod_{i=1}^k\prod_{y \in \eta_{B_r^c}, |y| \le R} |1-\frac{x_i}{y}|^\beta\Biggr) \fstop
%\end{align}
We show that the potential~$\Psi_{r, R}^{k, \eta}$ defined in~\eqref{d:TDLR} is geodesically convex in $(\U(B_r), \mssd_{\U})$. 
\begin{prop} \label{p:conv}
$\Psi_{r, R}^{k, \eta}$ is geodesically convex in $(\U^{k}(B_r), \mssd_{\U})$ for any $0<r<R<\infty$, $k \in \N$ and $\eta \in \U(B_r^c)$, 
\end{prop}
\begin{proof}
Note that if $u_1, \ldots, u_k$ are convex and $\alpha_1,\ldots, \alpha_k \ge 0$, then $\sum_{i=1}^k \alpha_iu_i$ is again convex.  
Let $H_{ij}, H_i^y$ be the Hessian matrices of the functions $(x_1, \ldots, x_k) \mapsto -\log |x_i-x_j|$ and~$(x_1, \ldots, x_k) \mapsto -\log|1-\frac{x_i}{y}|$ respectively. For any vector $\mathbf v=(v_1, \ldots, v_k) \in \R^{k}$, 
\begin{align} \label{e:HCP}
\mathbf v H_{ij} \mathbf v^t = \frac{(v_i-v_j)^2}{|x_i-x_j|^2}, \quad  \mathbf v H^y_{i} \mathbf v^t = \frac{v_i^2}{|y-x_i|^2} \fstop
\end{align}
Both $H_{ij}$ and $H_i^y$ are, therefore,  positive semi-definite. Thus, for any $0<r<R$, any $y \in [-R, -r] \cup [r, R]$ and any $i, j \in  \{1, 2, \ldots, k\}$ with $i \neq j$, the functions~$(x_1, \ldots, x_k) \mapsto -\log |x_i-x_j|$ and~$(x_1, \ldots, x_k) \mapsto -\log|1-\frac{x_i}{y}|$ are convex  in the following areas for any~$\sigma \in \mathfrak S_k$: 
$$\bigl\{(x_1, \ldots, x_k) \in B_r^{\times k}: x_{\sigma(1)} < x_{\sigma(2)}<\cdots <x_{\sigma(k)}\bigr\} \fstop$$ 
%$$-\log |x_i-x_j| \comma \quad -\log\Bigl|1-\frac{x_i}{y}\Bigr| \comma \quad \forall y \in [-R, -r] \cup [r, R] \quad i \in \{1, 2, \ldots, k\}\fstop$$ 
In view of \eqref{e:STS}, the following expression, therefore, concludes that $\Psi_{r, R}^{k, \eta}$ is geodesically convex as a function on $\U^k(B_r)$: for any $\gamma =\sum_{i=1}^k\delta_{x_i}$
\begin{align} \label{eq: conv}
\Psi_{r, R}^{k, \eta}(\gamma) = -\beta\sum_{i<j}^k \log (|x_i-x_j|) - \beta\sum_{i=1}^k \sum_{y \in \eta_{B_r^c}, |y| \le R} \log\Bigl|1-\frac{x_i}{y}\Bigr| \fstop
\end{align}
The proof is complete.
%Since the convexity is preserved by the quotient map $\pr_r\colon B_r^{\times k} \to \U^{k}(B_r)$ equipped with $\mssd_{\U}$, we conclude the statement. 
\end{proof}

Thanks to the geodesical convexity of the potential~$\Psi_{r, R}^{k, \eta}$ shown in Prop.~\ref{p:conv}, the Dirichlet form $(\E^{\U(B_r), \mu_{r, R}^{k, \eta}}, \dom{\E^{\U(B_r), \mu_{r, R}^{k, \eta}}})$ satisfies the Riemannian Curvature Dimension condition $\RCD(0,\infty)$. 
%We recall the notion of {\it Riemannian energy measure space}.
%\begin{defs}[Riemannian energy measure space] \label{d:EMS}
%Let $(X, \tau)$ be a Polish space with a fully supported Borel measure $\nu$ with finite total mass $\nu(X)<\infty$, $\mathcal B^{\nu}$ be the completion of the Borel $\sigma$-field. Let $(\E, \mathcal D(\mathcal E))$ be a strongly local symmetric Dirichet form in $L^2(\nu)$ having a square-field operator $\Gamma: \mathcal D(\E)^{\times 2} \to \R$. 
%We say that $(X, \tau, \E, \mathcal D(\mathcal E))$ is {\it a Riemannian energy measure space}  if 
%\begin{enumerate}[{\rm (i)}]
%\item the intrinsic distance $\mssd_{\E}(x, y):=\sup\{f(x)-f(y): \Gamma(f) \le 1,\ f \in \mathcal D(\E) \cap C_b(\tau)\}$ induces the topology $\tau$;
%\item
%\end{enumerate}
% %and $(\E, \F)$ be a strongly local Dirichlet form $(\E, \F)$ having a square field $\Gamma$. 
% Let $\Ch_{\mssd, \nu}$ be the Cheeger energy defined as the $L^2(\nu)$-lower semi-continuous envelope of $\int_{X} |\mathsf D u|^2 \diff \nu$, i.e., 
% $$\Ch_{\mssd, \nu}(u):=\inf \biggl\{ \liminf_{n \to \infty} \int_{X} |\mathsf D u_n|^2 \diff \nu:\ u_n \in \Lip(d)\cap L^2(\nu) \xrightarrow{L^2} u \biggr\}\comma$$
% where $|\mathsf Du|(x):=\limsup_{y \to x}\frac{|f(x)-f(y)|}{\mssd(x, y)}$ if $x$ is not isolated, and $|\mathsf Du|(x)=0$ otherwise. The domain is denoted by $W^{1,2}(X, \mssd, \nu):=\{u \in L^2(\nu): \Ch_{\mssd, \nu}(u)<\infty\}$. The Cheeger energy $\Ch_{\mssd, \nu}$ can be expressed by the following integration by \cite{AmbGigSav14b, AmbGigMonRaj12}
% $$\Ch_{\mssd, \nu}(u)=\int_X |\nabla u|_w^2 \diff\nu \comma$$
% where $|\nabla u|_w$ is called {\it minimal weak upper gradient}. 
% We say that $(X, \mssd, \nu)$ satisfies the Riemannian Curvature-Dimension Condition $\RCD(K,\infty)$ for $K \in \R$ if 
%\begin{enumerate}[{\rm (i)}]
%\item $\Ch_{\mssd, \nu}$ is quadratic, i.e., $\Ch_{\mssd, \nu}(u+v)+\Ch_{\mssd, \nu}(u-v)= 2\Ch_{\mssd, \nu}(u)+2\Ch_{\mssd, \nu}(v)$; 
%\item Sobolev-to-Lipschitz property holds, i.e., every $u \in W^{1,2}(X, \mssd, \nu)$ with $|\nabla u|_w \le 1$ has a $\mssd$-Lipschitz representative $\tilde{u}$ with $\Lip(\tilde{u}) \le 1$;
%\item $\Ch_{\mssd, \nu}$ satisfies $\BE(K,\infty)$, i.e., $|\nabla T_t u| \le e^{-Kt} T_t|\nabla u|$ for every $u \in W^{1,2}(X, \mssd, \nu)$ and $t>0$.  
%\end{enumerate}
%\end{defs}
 
\begin{prop} \label{p:BE1}
The space $(\U^k(B_r), \mssd_\U, \mu_{r, R}^{k, \eta})$ satisfies $\RCD(0, \infty)$ for every $k \in \N_0$, $0<r<R<\infty$ and $\eta \in \U$, and it holds that 
$$\bigl(\E^{\U(B_r), \mu_{r, R}^{k, \eta}}, \mathcal D(\E^{\U(B_r), \mu_{r, R}^{k, \eta}}) \bigr)=\bigl(\Ch_{\mssd_\U, \mu_{r, R}^{k, \eta}}, W^{1,2}(\U^k(B_r), \mssd_\U, \mu_{r, R}^{k, \eta})\bigr) \fstop$$ 
\end{prop}
\begin{proof}
%We first prove $\E_{r, R}^{k, \eta}=\Ch_{\mssd_\U, \mu_{r, R}^{k, \eta}}$. 
%\footnote{\purple{Note that the density can take zero, in which case, can we really apply these arguments?}}
Noting that $B_r^{\times k}$ is a convex subset in $\R^{k}$, the space $(B_r^{\times k}, \mssd^{\times k}, \mssm_r^{\otimes k})$ is a geodesic subspace of~$\R^{k}$ and, therefore, satisfies $\RCD(0, \infty)$ by the Global-to-Local property of $\RCD(0,\infty)$, see~\cite[Thm.~6.20]{AmbGigSav14b}. 
Noting that the $k$-particle configuration space~$(\U^k(B_r), \mssd_\U, \mu_{r, R}^{k, \eta})$ is the quotient space of~$(B_r^{\times k}, \mssd^{\times k}, \mssm^{\otimes k})$  with respect to the symmetric group~$\mathfrak S_k$ and that the property~$\RCD(0,\infty)$ is preserved under the quotient operation with respect to~$\mathfrak S_k$ thanks to~\cite{GalKelMonSos18}, we obtain that $(\U^k(B_r), \mssd_\U, \mssm^{\odot k})$ satisfies $\RCD(0, \infty)$ as well. By the geodesical convexity of the potential~$\Psi_{r, R}^{k, \eta}$ shown in Prop.~\ref{p:conv} and the continuity of the density~$e^{-\Psi_{r, R}^{k, \eta}}$, the weighted space $(\U^k(B_r), \mssd_\U, \mu_{r, R}^{k, \eta})$ satisfies $\RCD(0,\infty)$ by \cite[Prop.~6.21]{AmbGigSav14b}.
% and \cite[Lem.4.11]{AmbGigSav14}.  

To conclude the statement, it suffices to check the identity 
$$\E^{\U(B_r), \mu_{r, R}^{k, \eta}}=\Ch_{\mssd_\U, \mu_{r, R}^{k, \eta}} \fstop$$ 
By the Rademacher theorem on~ $\U^k(B_r)$ descendent from the Rademacher theorem on~$B_r^{\times k}$,  the slope $|\mathsf D_{\mssd_\U} u|$ coincides with the gradient $|\nabla^{\odot k} u|$ for any $u \in \Lip(\U^k(B_r), \mssd_\U)$.  
Thus, 
\begin{align} \label{eq:BE1}
\E^{\U(B_r), \mu_{r, R}^{k, \eta}}(u) = \int_{\U^k(B_r)} |\mathsf D_{\mssd_\U} u|^2 \diff \QP_{r, R}^{k ,\eta} \quad  u \in \Lip_b(\U^k(B_r), \mssd_\U) \fstop
\end{align}
Since $\Ch_{\mssd_\U, \mu_{r, R}^{k, \eta}}$ is  the $L^2$-lower semi-continuous envelope by definition, the functional $\Ch_{\mssd_\U, \mu_{r, R}^{k, \eta}}$ is the maximal $L^2$-lower semi-continuous functional satisfying 
$$\Ch_{\mssd_\U, \mu_{r, R}^{k, \eta}}(u) \le \int_{\U^k(B_r)} |\mathsf D_{\mssd_\U} u|^2 \diff \QP_{r, R}^{k ,\eta} \fstop$$
As $\E^{\U(B_r), \mu_{r, R}^{k, \eta}}$ is closed by Prop.~\ref{p:form1}, in particular, $\E^{\U(B_r), \mu_{r, R}^{k, \eta}}$ is $L^2$-lower semi-continuous. Therefore, combining the maximality of $\Ch_{\mssd_\U, \mu_{r, R}^{k, \eta}}$ with \eqref{eq:BE1}, it holds that 
$$\text{$\E^{\U(B_r), \mu_{r, R}^{k, \eta}}\le \Ch_{\mssd_\U, \mu_{r, R}^{k, \eta}}$  \ and \ $W^{1,2}(\U^k(B_r), \mssd_\U, \mu_{r, R}^{k, \eta}) \subset \dom{\E^{\U(B_r), \mu_{r, R}^{k, \eta}}}$}$$ 
and 
$$\Ch_{\mssd_\U, \mu_{r, R}^{k, \eta}}(u) = \E^{\U(B_r), \mu_{r, R}^{k, \eta}}(u) \quad u \in \Lip_b(\U^k(B_r), \mssd_\U) \fstop$$
As $\Lip_b(\U^k(B_r), \mssd_\U)$ is dense both in $\dom{\E^{\U(B_r), \mu_{r, R}^{k, \eta}}}$ and $W^{1,2}(\U^k(B_r), \mssd_\U, \mu_{r, R}^{k, \eta})$ by construction, the proof is completed.
% is the smallest closed extension on $\Lip(\U^k(B_r), \mssd_\U)$ by Prop.~\ref{p:form1}, the opposite inclusion holds as well:
%$$W^{1,2}(\U^k(B_r), \mssd_\U, \mu_{r, R}^{k, \eta}) \supset \dom{\E^{\U(B_r), \mu_{r, R}^{k, \eta}}} \comma$$
%which concludes the statement.  
%To prove the opposite inequality \footnote{Markov uniquenss is wrong. reconsider the other method, we probably are able to simply the proof by using Lem 4.11 in AKR Inventiones to say the slope coincides with the m.w.u.g and use (3.7).}
%$$\text{$\E^{\U(B_r), \mu_{r, R}^{k, \eta}}\ge \Ch_{\mssd_\U, \mu_{r, R}^{k, \eta}}$  \ and \ $W^{1,2}(\U^k(B_r), \mssd_\U, \mu_{r, R}^{k, \eta}) \supset \dom{\E^{\U(B_r), \mu_{r, R}^{k, \eta}}}$} \comma$$ 
%%$\E_{r, R}^{k, \eta} \ge \Ch_{\mssd_\U, \mu_{r, R}^{k, \eta}}$ on $L^2(\mu_{r, R}^{k, \eta})$ or equivalently, the opposite inclusion $W^{1,2}(\U^k(B_r), \mssd_\U, \mu_{r, R}^{k, \eta}) \supset \dom{\E_{r, R}^{k, \eta}}$, 
%it suffices to show that~$\bigl(\Ch_{\mssd_\U, \mu_{r, R}^{k, \eta}}, \Lip(\mssd_\U) \bigr)$ is Markov unique, namely, there exists at most one Dirichlet form extending 
%$$\bigl(\Ch_{\mssd_\U, \mu_{r, R}^{k, \eta}},\Lip(\mssd_\U)\bigr) \fstop$$ 
%As $(\U^k(B_r), \mssd_\U, \mu_{r, R}^{k, \eta})$  is $\RCD(0,\infty)$ concluded in the first paragraph, by \cite[(iv) Thm.~6.1]{AmbGigSav14b}, the semigroup $T^{\U(B_r), \mu_{r, R}^{k, \eta}}_t$ corresponding to $\Ch_{\mssd_\U, \mu_{r, R}^{k, \eta}}$ preserves $\Lip(\mssd_\U)$ algebra, i.e., 
%$$T^{\U(B_r), \mu_{r, R}^{k, \eta}}_t  \Lip(\mssd_\U) \subset \Lip(\mssd_\U) \quad \forall t>0\fstop$$
%Therefore, by Lem.~\ref{l:MU}, we conclude the statement. 
%In particular, taking the corresponding the infinitesimal generator $(A^{\mu_{r, R}^{k, \eta}}, \dom{A^{\mu_{r, R}^{k, \eta}}})$, this concludes 
%$$T^{\mu_{r, R}^{k, \eta}}_t  \Bigl(\Lip(\mssd_\U) \cap \dom{A^{\mu_{r, R}^{k, \eta}}} \Bigr) \subset \Lip(\mssd_\U) \cap \dom{A^{\mu_{r, R}^{k, \eta}}}\fstop$$
%Therefore, by \cite[Thm.~X.49]{ReeSim75}, we conclude that 
%$$\text{$(A^{\mu_{r, R}^{k, \eta}}, \dom{A^{\mu_{r, R}^{k, \eta}}})$ is the unique self-adjoint extension on $\Lip(\mssd_\U) \cap \dom{A^{\mu_{r, R}^{k, \eta}}}$} \fstop$$
%As $\Lip(\mssd_\U) \subset W^{1,2}(\U^k(B_r), \mssd_\U, \mu_{r, R}^{k, \eta})$ is dense by construction of Cheeger energy, taking $u_n \in \Lip(\mssd_\U)$ converging to $u \in W^{1,2}(\U^k(B_r), \mssd_\U, \mu_{r, R}^{k, \eta})$ and noting the lower semi-continuity of $\E_{r, R}^{k, \eta}$ with respect to $L^2$-strong convergence, we see that
%$$\E_{r, R}^{k, \eta}(u) \le \liminf_{n \to \infty} \E_{r, R}^{k, \eta}(u_n) = \lim_{n \to \infty}\Ch_{\mssd_\U, \mu_{r, R}^{k, \eta}}(u_n) =\Ch_{\mssd_\U, \mu_{r, R}^{k, \eta}}(u) \fstop$$
%and that $\Ch_{\mssd_\U, \mu_{r, R}^{k, \eta}}$ is the lower semi-continuous envelope for $\int_{\U^k(B_r)} |\mathsf D_{\mssd_\U} u|^2 \diff \mu_{r, R}^{k, \eta}$ (\purple{make the last argument more rigorous}).  %\purple{Add accessible references, Galz-Galcia, Sturm}
\end{proof}

%Due to the number rigidity \ref{ass:Rig} of $\sine_\beta$ ensemble~$\mu$, for $\mu$-a.e.~$\eta$, there exists a unique integer $k(\eta) \in \N_0$ so that 
%\begin{align*} %\label{e:R1}
%\text{$\mu_{r}^{l, \eta}(\U^k(B_r))>0$ if and only if $l=k(\eta)$} \fstop
%\end{align*}
%%\footnote{\purple{In the case of $l \neq k(\eta)$, $L^2(\mu_{r, \eta}^l)=\{0\}$ and every function space is trivial. As it is not fully supported, it is not an RCD, but we have nothing to prove in this case.}}
%We, therefore, only need to discuss the case $l=k(\eta)$.
 In view of Prop.~\ref{p:BE1} and the approximation $\Psi_{r, R}^{k, \eta}$ to  $\Psi_{r}^{k, \eta}$ as $R \to \infty$, we prove that  $(\U^k(B_r), \mssd_\U, \mu_{r}^{k, \eta})$ satisfies $\RCD(0,\infty)$ as well.
\begin{prop} \label{p:BE2} 
Let $\QP$ be the $\sine_\beta$ ensemble for $\beta>0$. For any $0<r<\infty$ and $\mu$-a.e.\ $\eta \in \U$, 
the space $(\U^k(B_r), \mssd_\U, \mu_{r}^{k, \eta})$ satisfies $\RCD(0, \infty)$, where $k=k(\eta)$ as in \eqref{e:R1}. Furthermore,  
$$\bigl(\E^{\U(B_r), \mu_{r}^{k, \eta}}, \dom{\E^{\U(B_r), \mu_{r}^{k, \eta}}} \bigr)=\bigl(\Ch_{\mssd_\U, \mu_{r}^{k, \eta}}, W^{1,2}(\U^k(B_r), \mssd_\U, \mu_{r}^{k, \eta})\bigr) \fstop$$
\end{prop}
\begin{proof} 
Since the potential~$\Psi_{r, R}^{k, \eta}$ is geodesically convex for any $R$ and it converges pointwise to $\Psi_{r}^{k, \eta}$ as $R \to \infty$ for $\mu$-a.e.~$\eta$ by \cite[Lem.~2.3 and Proof of Thm.\ 2.1 in~p.~183]{DerHarLebMai20},  the potential~$\Psi_{r}^{k, \eta}$ is again geodesically convex on~$(\U^k(B_r), \mssd_\U)$. Furthermore, as the density~$e^{-\Psi_{r, R}^{k, \eta}}$ converges uniformly to~$e^{-\Psi_{r}^{k, \eta}}$ on~$\U^k(B_r)$ as $R \to \infty$ for $\mu$-a.e.~$\eta$ by \cite[Lem.~2.3 and Proof of Thm.\ 2.1 in~p.~183]{DerHarLebMai20}, the density~$e^{-\Psi_{r}^{k, \eta}}$ is continuous on~$\U(B_r)$. Noting the fact that the constant multiplication (by the normalisation constant~$Z_r^\eta$) does not change the lower Ricci curvature bound (see e.g., \cite[Prop.~4.13]{Stu06a}), the same proof as~Prop.\ \ref{p:BE1} applies to conclude the statement.
%the spaces $(\U^k(B_r), \mssd_\U, \mu_{r, R}^{k, \eta})$ converge to $(\U^k(B_r), \mssd_\U, \mu_{r}^{k, \eta})$ in the measured Gromov sense \cite[Def.\ 3.16]{GigMonSav15}  for $\mu$-a.e.~$\eta$.  Since $\RCD(0, \infty)$ is preserved under the measured Gromov convergence by \cite[Thm.\ IV]{GigMonSav15}, the limit $(\U^k(B_r), \mssd_\U, \mu_{r}^{k, \eta})$ satisfies $\RCD(0, \infty)$  as well. The proof of the latter statement is identical to that in the proof of Prop.\ \ref{p:BE1}. 
%\purple{\bf We have a short proof by using the stability of RCD by the measured Gromov convergence. In this case, we do not need even Theorem \ref{p:Mosco2}. And we can remove all the definitions of Kuwae-Shioya things}
%
%By noting that $\mu_{r, R}^{k, \eta}$
%Applying \cite[Thm. 3.14, 3.17, 4.17]{AmbGigSav15} to the current setting, it suffices to show that 
%\begin{enumerate}[{\rm (i)}]
%\item the intrinsic distance $\mssd_{\E_{r, R}^{k, \eta}}(x, y):=\sup\{u(x)-u(y): |\nabla^{\odot k}u| \le 1,\ u \in \mathcal D(\E_{r, R}^{k, \eta}) \cap \mathcal C_b\}$ coincides with $\mssd_{\U}$;
%\item every $u \in \mathcal L:=\{u \in \mathcal D(\E_{r, R}^{k, \eta}): |\nabla^{\odot k}u| \le 1\}$ has a continuous representative;
%\item $\BE(0,\infty)$.
%\end{enumerate}
%
%\paragraph{Proof of (i)}The space $(\E_{r, R}^{k, \eta}, \mathcal D(\E_{r, R}^{k, \eta}))$ satisfies {\it Rademacher-type property}, i.e., 
%$$ \text{$\Lip(\mssd_\U) \subset \mathcal D(\E_{r, R}^{k, \eta})$ \ \  and \ \ $|\nabla^{\odot k}u| \le \Lip(u)$} \comma$$
% which holds true by the construction of $(\E_{r, R}^{k, \eta}, \mathcal D(\E_{r, R}^{k, \eta}))$. Thus, by \cite[Lem.\ 3.6]{LzDSSuz20}, we obtain 
% $$\mssd_{\U} \le \mssd_{\E_{r, R}^{k, \eta}} \fstop$$
%  The space $(\E_{r, R}^{k, \eta}, \mathcal D(\E_{r, R}^{k, \eta}))$ satisfies {\it Sobolev-to-Lipschitz property}, i.e., every $u \in \mathcal L$ has a $\mssd_{\U}$-Lipschitz modification $\tilde{u}$ with $\Lip(\tilde{u}) \le 1$. This can be verified in the following steps: The unweighted Sobolev space $W^{1,2}(\mssm^{\otimes k}_r)$ satisfies Sobolev-to-Lipschitz property with respect to $\mssd^{\times k}$, which is well-known in convex domains in Euclidean spaces. As the quotient map $\pr_r: B_r^{\times k} \to \U^k(B_r)$ preserves this property, the quotient unweighted Sobolev space $W^{1,2}(\mssm^{\odot k}_r)$ satisfies Sobolev-to-Lipschitz property with respect to $\mssd_\U$. To transfer it to the weighted space $\mathcal D(\E_{r, R}^{k, \eta})$, we apply \cite[Prop. 4.2]{LzDSSuz22b}: We need to verify three conditions $(a_1), (a_2), (b)$ there for the weight $\Psi_{r, R}^{k, \eta}$: Condition $(a_1)$ is readily verified for $\Psi_{r, R}^{k, \eta}$ by taking a  relatively open exhaustion $\{G_k\}$ with $\U(B_r) \setminus \mathsf{diag} \subset \cup_{k \in \N} G_k$, where $\mathsf{diag}$ is the diagonal set in $\U(B_r)$; Condition $(a_2)$ is verified by taking a cut-off function on $G_k$ by the distance $\mssd_{\U}$; Condition $(b)$ is verified by taking the Lipschitz algebra $\Lip(\mssd_\U)$ as a core. Therefore, by applying \cite[Prop. 4.2]{LzDSSuz22b}, we obtain that 
% $$\mathcal L  \subset \{u \in W^{1,2}(\mssm^{\odot k}_r): |\nabla^{\odot k} u| \le 1\} \comma$$
% with which and Sobolev-to-Lipschitz property for $W^{1,2}(\mssm^{\odot k}_r)$ concludes Sobolev-to-Lipschitz property for $\mathcal D(\E_{r, R}^{k, \eta})$. 
% By \cite[Prop.\ 4.2]{LzDSSuz20}, we obtain 
% $$\mssd_{\U} \ge \mssd_{\E_{r, R}^{k, \eta}} \fstop$$
% 
%\paragraph{Proof of $\mathrm{(ii)}$}Condition $\mathrm{(ii)}$ has been already verified in the proof of $\mathrm{(i)}$ by Sobolev-to-Lispchitz property for $\mathcal D(\E_{r, R}^{k, \eta})$.
%
%\paragraph{Proof of $\mathrm{(iii)}$}
%
%According to Def.\ \ref{d:RCD}, it suffices to show $\E_r^{k, \eta}=\Ch_{\mssd_\U, \mu_{r}^{k, \eta}}$, $\BE(0, \infty)$ and the Sobolev-to-Lipschitz property. 
%By \cite[(iv) in Thm. 6.1]{AmbGigSav14b}, 
%$$T_t^{\QP_{r}^{k, \eta}}u, T_t^{\QP_{r, R}^{k, \eta}}u \in \Lip(\U^k, \mssd_\U)\comma \quad \text{whenever $u \in \Lip(\U^k, \mssd_\U)$}.$$
% As $\Lip(\U^k, \mssd^k)$ is a core for both $(\E_{r, R}^{k, \eta}, \mathcal D\bigl(\E_{r, R}^{k, \eta}))$ and $(\E_{r}^{k, \eta}, \mathcal D\bigl(\E_{r}^{k, \eta}))$,   the square field operators $\cdc^{\dUpsilon^k, \QP_{r}^{k, \eta}}$ and $\cdc^{\dUpsilon^k, \QP_{r, R}^{k, \eta}}$ coincide with $\cdc^{\dUpsilon^k}$ on $\Lip(\U^k, \mssd^k)$, in particular, it holds that 
%\begin{align} \label{eq:Mosco2-1}
%\cdc^{\dUpsilon^k, \QP_{r}^{k, \eta}}(T_t^{\QP_{r}^{k, \eta}}u) 
%= \cdc^{\dUpsilon^k}(T_t^{\QP_{r}^{k, \eta}}u), \quad  \cdc^{\dUpsilon^k, \QP_{r, R}^{k, \eta}}(T_t^{\QP_{r, R}^{k, \eta}}u) 
%= \cdc^{\dUpsilon^k}(T_t^{\QP_{r, R}^{k, \eta}}u) \fstop 
%\end{align}
%As a consequence of the Mosco convergence in Prop.\ \ref{p:Mosco2} (\purple{reference of Mosco convergence}),  it holds that,  for any $u \in \Lip(\U^k, \mssd^k)$, 
%\begin{align} \label{eq:Mosco2-2}
%\cdc^{\dUpsilon^k}(T_t^{\QP_{r}^{k, \eta}}u)  \le \liminf_{R \to \infty} \cdc^{\dUpsilon^k}(T_t^{\QP_{r, R}^{k, \eta}}u) \le \liminf_{R \to \infty}T_t^{\QP_{r, R}^{k, \eta}}\cdc^{\dUpsilon^k} (u) = T_t^{\QP_{r}^{k, \eta}}\cdc^{\dUpsilon^k} (u) \fstop
%\end{align}
%The proof of $\BE(0, \infty)$is concluded by \eqref{eq:Mosco2-1}, \eqref{eq:Mosco2-2}. \purple{The Sobolev-to-Lipschitz property is a consequence of Tensorisation Paper.}
\end{proof}
%\begin{rem}
%We can improve the statements in Props.\ \ref{p:BE1}, \ref{p:BE2} to $\RCD(0, k)$ 
%\end{rem}
%\begin{cor}[functional inequalities]
%\end{cor}


\section{Curvature bound for infinite-particle systems} \label{sec:CI}
%\purple{Change the symbol in this section like $\E^{\U(B_r), \mu_{r, R}^{k, \eta}}$.}
In this section, we construct a local Dirichlet form on $\U=\U(\R)$ associated with $\sine_\beta$ ensemble~$\QP$ and show~the~$\BE(0,\infty)$ property by the following steps: we first construct {\it truncated Dirichlet forms} on $\U$ whose gradient operators are truncated up to configurations inside~$B_r$. We then identify them with the superposition Dirichlet forms lifted from~$\U(B_r)$, thanks to which we can show~$\BE(0,\infty)$ for the truncated forms.   We take the monotone limit of the truncated forms to construct a Dirichlet form with invariant measure $\sine_\beta$ ensembles~$\QP$ and $\BE(0,\infty)$ extends to the limit form.
 In the end of this section, we discuss several applications of the~$\BE(0,\infty)$ property.
%In the end of this section, we prove that our construction of  Dirichlet forms is {\it the unique} Dirichlet form corresponding to $\sine_\beta$ invariant measures in the sense of the Markov uniqueness, i.e., there is only one Markov extension of the form on the core we work with.  
%\smallskip
%Let $\mu(\cdot \ | \ \pr_{B_r^c}(\cdot)=\eta_{B_r^c})$ (simply written by $\mu_{r}^\eta$) denote the regular conditional probability of $\mu$ conditioned at $\eta \in \U$ with respect to the $\sigma$-field generated by the projection map $\gamma \in \U \mapsto \pr_r(\gamma)=\gamma_{B_r} \in \U(B_r)$ (see e.g., \cite[Def.~3.32]{LzDSSuz21} for the precise definition).  
%\paragraph{Caveat}Although the conditional probability~$\mu_{r}^\eta$ is a probability measure on the whole space~$\U$ whose support is $\U_r^\eta=\{\gamma \in \U: \gamma_{B_r^c}=\eta_{B_r^c}\}$, without loss of information we may think of $\mu_{r}^\eta$  to be a probability measure on~$\U(B_r)$ indexed by $\eta$, which is, to be precise,  the push-forwarded measure~$(\pr_r)_\#\mu_r^\eta$. This identification is justified because the projection map
%$$\pr_r: \bigl(\U\cap \U_r^\eta, \mu_{r}^\eta\bigr) \to \bigl(\U(B_r), (\pr_r)_\#\mu_r^\eta\bigr)$$ is bijective with the inverse map $\pr_r^{-1}$ defined as~$\pr^{-1}_r(\gamma):=\gamma+\eta$, and because both $\pr_r$ and~$\pr_r^{-1}$ are measure-preserving.  
%%does not lose information of~$\mu_r^\eta$ since the conditional probability~$\mu_{r}^\eta$ is supported only on the set~$\U_{r}^\eta:=\{\gamma \in \U: \gamma_{B_r^c}=\eta_{B_r^c}\}$. 
%Hereinafter, we will not distinguish these two measures for the sake of the notational simplicity and we will consider $\mu_{r}^\eta$ to be a probability measure on $\U$ as well as on $\U(B_r)$, while not changing the symbol. 
%\smallskip
%The following proposition shows the explicit expression  of the density of the conditional probability $\QP_{r}^\eta$ with respect to the Poisson measure~$\pi_{\mssm_r}$ defined in~\eqref{d:PS}.
%\begin{prop}[{\cite[Thm.\ 1.1]{DerHarLebMai20}}]  The regular conditional probability~$\mu_{r}^\eta$ of~$\sine_\beta$ ensemble $\mu$ has the following form: 
%\begin{align} \label{d:CP}
%\diff\mu_r^\eta=\frac{1}{Z_{r}^{\eta}} \sum_{k=0}^\infty \mu_{r}^{k, \eta}=\frac{1}{Z_{r}^{\eta}} \sum_{k=0}^\infty e^{-\Psi_{r}^{k, \eta}} \diff \mssm_r^{\odot k} \comma
%\end{align}
%where $\Psi_{r}^{k, \eta}$ has been defined in~\eqref{d:cp} and~$Z_{r}^{\eta}$ is the normalising constant for the measure $\sum_{k=0}^\infty \mu_{r}^{k, \eta}$ defined in \eqref{d:cp}.
%\end{prop}



%%For many results here and later on, we shall make the following assumption.
%\begin{ass}\label{d:ass:Mmu}
%We say that~$\QP$ satisfies \ref{ass:Mmu} if it has $\msE$-locally finite intensity, viz.\
%\begin{equation}\tag*{$(\mssm_\QP)_{\ref{d:ass:Mmu}}$}\label{ass:Mmu}
%\mssm_\QP E<\infty \comma \quad E\in\msE\fstop
%\end{equation}
%\end{ass}
%%\begin{ese} \label{exa: FI}
%Assumption~\ref{ass:Mmu} is a natural condition from the following viewpoints:
%from a mathematical point of view, it implies that we have sufficiently many functions in~$L^1(\QP)$;
%from a physical point of view, it implies that any system of randomly $\QP$-distributed particles is $\msE$-locally finite in average. 
%\purple{Example}
%\end{ese}
%\subsubsection{Dirichlet forms}
%Let us briefly recall the construction and main analytical properties of the Dirichlet form~\eqref{eq:DirichletForm} constructed in~\cite{LzDSSuz21}.
%We specialize all the statements in~\cite{LzDSSuz21} to the case of our interest, namely that of Poisson measures.

%For a probability measure~$\QP$ on~$\ttonde{\dUpsilon,\A_{\mrmv}(\msE)}$ and $E \in \msE$, set {\it the restricted measure on $\{\gamma \in \dUpsilon: \gamma E = k\}$}:
%\begin{align}\label{defn: RP}
%\QP^{k, E}(\cdot):=\QP\bigl(\cdot \cap \{\gamma \in \dUpsilon: \gamma E = k\} \bigr) \fstop
%\end{align}
%For each fixed~$\eta\in\dUpsilon$, and for each fixed~$E\in\msE$, we let~$\QP^{\eta_{E^\complement}}$ be the regular conditional probability strongly consistent with~$\pr^{E^\complement}$ (i.e., $(\pr^{E^\complement})^{-1}(\eta)$ is co-negligible with respect to $\QP^{\eta_{E^\complement}}$)\footnote{Write a reference for this definition, e.g., the first paper.}. By the definition of the conditional probability, it holds 
%\begin{equation}\label{eq:ConditionalQP}
%\QP\tonde{\Lambda\cap (\pr^{E^\complement})^{-1}(\Xi)}=\int_\Xi \QP^{\eta_{E^\complement}} \Lambda \, \diff\QP(\eta) \comma
%\end{equation}
%for any $\Lambda, \Xi \in \A_{\mrmv}(\msE)$ where, with slight abuse of notation, we regard~$\dUpsilon(E)$ as a subset of~$\dUpsilon$, and thus~$\pr^{E^\complement}$ as a map~$\pr^{E^\complement}\colon\dUpsilon\to\dUpsilon$.
%In probabilistic notation,
%\begin{equation*}
%\QP^{\eta_{E^\complement}}=\QP\ttonde{\emparg \big |\, \pr^{E^\complement}(\emparg)=\eta_{E^\complement}} \fstop
%\end{equation*}
%The \emph{projected conditional probabilities of~$\QP$} are the system of measures on $\dUpsilon(E)$:
%\begin{equation}\label{eq:ProjectedConditionalQP}
%\QP^\eta_E\eqdef \pr^E_\pfwd \QP^{\eta_{E^\complement}}\comma \qquad \eta\in\dUpsilon\comma \qquad E\in\msE \fstop
%\end{equation}
%%\end{defs}
%The restriction of $\QP^\eta_E$ on $\dUpsilon^{(k)}(E_h)$ is denoted by
%\begin{equation}\label{eq:ProjectedConditionalQP1}
%\QP^{\eta, k}_E\eqdef \QP^\eta_E\mrestr{\dUpsilon^{(k)}(E_h)}\comma \qquad \eta\in\dUpsilon\comma \qquad E\in\msE \fstop
%\end{equation}


\subsection{Truncated Dirichlet forms}
%Recall that we defined in \eqref{eq:ConditionalFunction}: 
In this subsection, we construct the truncated Dirichlet forms on~$\U$. 
We first construct square field operators on~$\U$ and~$\U(B_r)$ respectively. For so doing, we introduce a map~$\mathcal U_{\gamma, x}$ transferring functions on the configuration space~$\U$  to functions on~the base space ~$\R$.  For~$u: \U \to \R$, define~$\mathcal U_{\gamma, x}(u): \R \to \R$ by 
\begin{align} \label{d:UO}
\mathcal U_{\gamma, x}(u)(y):=u\ttonde{\car_{X\setminus\set{x}}\cdot\gamma + \delta_y}-u\ttonde{\car_{X\setminus\set{x}}\cdot\gamma} \comma \quad \gamma \in \U,\quad x \in \gamma \fstop
\end{align}
In the context of configuration spaces, the operation $\mathcal U_{\gamma, x}$ has been firstly discussed in~\cite[Lem.~1.2]{MaRoe00}, see also \cite[Lem.~2.16]{LzDSSuz21}.  
 We introduce the localisation of the operator~$\mathcal U_{\gamma, x}$ on $B_r$.  Recall that for a measurable function $ u\colon \dUpsilon\to \R$, $r>0$ and for $\eta \in \dUpsilon$, we set in~\eqref{e:SEF} 
$$\text{$u_{r}^\eta(\gamma)\eqdef  u(\gamma+\eta_{B_r^\complement})$ for $\gamma\in \dUpsilon(B_r)$.}$$
%The second lemma states the operation~$(\cdot)_r^\eta$ maps $\Lip(\U, \mssd_\U) \to \Lip(\U(B_r), \mssd_\U)$ and contracts Lipschitz constants. 
\begin{lem} \label{l:LSO}
For $u: \U(B_r) \to \R$, define  $\mathcal U_{\gamma, x}^r(u): B_r \to \R$ by 
$$\mathcal U_{\gamma, x}^r(u)(y):=u(\1_{X \setminus \{x\}} \cdot \gamma + \delta_y) - u(\1_{X \setminus \{x\}} \cdot \gamma) \quad \gamma \in \U(B_r), \ x \in \gamma \fstop$$
The operation $\mathcal U^r_{\gamma, x}$ maps from $\Lip(\U(B_r), \mssd_\U)$ to $\Lip(B_r)$ and Lipschitz constants are contracted by $\mathcal U^r_{\gamma, x}$ for any $r>0$:
 $$\Lip(\mathcal U^r_{\gamma, x}(u)) \le \Lip_{\mssd_\U}(u) \quad \forall \gamma \in \U(B_r) \quad \forall x \in \gamma \fstop$$
Furthermore, for any $u:  \U \to \R$, 
$$\mathcal U_{\gamma_{B_r}, x}^r(u_{r}^\gamma)(y) = \mathcal U_{\gamma, x}(u)(y) \quad \text{for every $\gamma \in \U$, $x \in \gamma_{B_r}$ and $y \in B_r$} \fstop$$
\end{lem}
\begin{proof}
 Let $u \in \Lip(\U(B_r), \mssd_\U)$. Then
\begin{align*} %\label{e:ULIP}
|\mathcal U^r_{\gamma, x}(u)(y)- \mathcal U^r_{\gamma, x}(u)(z)| &= |u(\car_{X\setminus\set{x}}\cdot\gamma + \delta_y)-u(\car_{X\setminus\set{x}}\cdot\gamma + \delta_z)|
\\
& \le \Lip_{\mssd_{\U}}(u)\mssd_\U(\car_{X\setminus\set{x}}\cdot\gamma + \delta_y, \car_{X\setminus\set{x}}\cdot\gamma + \delta_z) \notag
\\
&= \Lip_{\mssd_{\U}}(u) |y-z| \comma \notag
\end{align*}
which concludes the first assertion. 

We verify the second assertion.
For every $x \in \gamma_{B_r}$ and $y \in B_r$, 
\begin{align*}
 \mathcal U_{\gamma, x}(u)(y)& = u(\1_{X \setminus \{x\}} \cdot \gamma + \delta_y)- u(\1_{X \setminus \{x\}} \cdot \gamma)
 \\
 &  = u(\1_{X \setminus \{x\}} \cdot \gamma_{B_r} + \gamma_{B_r^c}+ \delta_y)- u(\1_{X \setminus \{x\}} \cdot \gamma_{B_r} + \gamma_{B_r^c})
 \\
 &= u_{r, \gamma}(\1_{X \setminus \{x\}} \cdot \gamma_{B_r}+ \delta_y)- u_{r, \gamma}(\1_{X \setminus \{x\}} \cdot \gamma_{B_r})
 \\
 &=\mathcal U^r_{\gamma_{B_r}, x}(u_{r}^\gamma)(y) \fstop
\end{align*}
%The proof of the latter statement is identical to that of Lem.~\ref{p:ULP}. 
The proof is complete.
\end{proof}

%The operation $\mathcal U_{\gamma, x}$ maps from $\Lip(\U, \mssd_\U)$ to $\Lip(\R)$ for any $r>0$ and contracts Lipschitz constants.
%
%\begin{lem} \label{p:ULP}
%The operation $\mathcal U_{\gamma, x}$ maps from $\Lip(\U, \mssd_\U)$ to $\Lip(\R)$ and Lipschitz constants are contracted by $\mathcal U_{\gamma, x}$:
% $$\Lip(\mathcal U_{\gamma, x}(u)) \le \Lip_{\mssd_\U}(u) \quad \forall \gamma \in \U \quad \forall x \in \gamma \fstop$$
%\end{lem}
%\begin{proof}
%Let $u \in \Lip(\U, \mssd_\U)$. Then
%\begin{align*} %\label{e:ULIP}
%|\mathcal U_{\gamma, x}(u)(y)- \mathcal U_{\gamma, x}(u)(z)| &= |u(\car_{X\setminus\set{x}}\cdot\gamma + \delta_y)-u(\car_{X\setminus\set{x}}\cdot\gamma + \delta_z)|
%\\
%& \le \Lip_{\mssd_{\U}}(u)\mssd_\U(\car_{X\setminus\set{x}}\cdot\gamma + \delta_y, \car_{X\setminus\set{x}}\cdot\gamma + \delta_z) \notag
%\\
%&= \Lip_{\mssd_{\U}}(u) |y-z| \fstop \notag
%\end{align*}
%The proof is complete.
%\end{proof}


We now define a square field operator on $\U$ truncated up to particles inside $B_r$.
\begin{defs}[Truncated square field on $\U$]\label{d:DT}
Let $u:\U \to \R$ be a measurable function so that $\mathcal U_{\gamma, x}(u)|_{B_r} \in W^{1,2}(\mssm_r)$ for $\QP$-a.e.~$\gamma$ and every $x \in \gamma_{B_r}$. 
 The following operator is called {\it the truncated square field} $\Gamma^\U_r$, 
\begin{equation}\label{eq:d:LiftCdCRep}
\begin{gathered}
\cdc^{\dUpsilon}_r(u)(\gamma):= \sum_{x\in \gamma_{B_r}} |\nabla \mathcal U_{\gamma, x}(u)|^2(x)\fstop
%\sum_{i,j=1}^{k,m} (\partial_i F)(\mbff^\trid\gamma) \cdot (\partial_j G)(\mbfg^\trid\gamma) \cdot \cdc( f_i,  g_j)^\trid \gamma \quad \text{$\QP$-a.e.\ $\gamma$}\comma
% u, v\in \Cyl{\Dz} \fstop
\end{gathered}
\end{equation}
Thanks to~Lem.~\ref{l:WDG}, Formula~\eqref{eq:d:LiftCdCRep}  is well-defined for $\QP$-a.e.~$\gamma$. Indeed, as $\mathcal U_{\gamma, x}(u)|_{B_r}\in W^{1,2}_{loc}(\mssm_r)$, the weak gradient~$\nabla \mathcal U_{\gamma, x}(u)$ is well-defined pointwise on a measurable set $\Sigma \subset B_r$ with $\mssm_r(\Sigma^c)=0$. By applying  Lem.~\ref{l:WDG}, Formula~\eqref{eq:d:LiftCdCRep} is well-defined on the set~$\Omega(r)$ of $\QP$-full measure. 
\end{defs} 

%\begin{lem} \label{l:RSP}
%Let $u \in \Lip_b(\U(B_r), \mssd_\U)$.  Then, 
%\begin{align} \label{e:RSP}
%\Bigl|\nabla^{\odot k} \bigl(u_r^\eta \bigr)\Bigr|^2(\gamma) = 
%%\\
%%&=  \sum_{i=1}^k\Bigl|\nabla \bigl(u_r^\eta \bigr)\Bigr|^2(x_i)
%&= \sum_{x\in\gamma} \bigl|\nabla u_{r}^{\eta}\tparen{\car_{B_r\setminus \set{x}} \cdot\gamma+\delta_\bullet}\bigr|^2(x) \notag
%\\
%&= \sum_{x\in\gamma}  \Bigl|\nabla \Bigl( u_{r}^{\eta}\tparen{\car_{B_r\setminus \set{x}} \cdot\gamma+\delta_\bullet}-u^\eta_{r}\tparen{\car_{B_r\setminus\set{x}}\cdot\gamma} \Bigr)\Bigr|^2(x)
%\end{align}
%\end{lem}
%\begin{proof}
%\begin{align} \label{e:RSP}
%\Bigl|\nabla^{\odot k} \bigl(u_r^\eta \bigr)\Bigr|^2(\gamma)
%%\\
%%&=  \sum_{i=1}^k\Bigl|\nabla \bigl(u_r^\eta \bigr)\Bigr|^2(x_i)
%&= \sum_{x\in\gamma} \bigl|\nabla u_{r}^{\eta}\tparen{\car_{B_r\setminus \set{x}} \cdot\gamma+\delta_\bullet}\bigr|^2(x) \notag
%\\
%&= \sum_{x\in\gamma}  \Bigl|\nabla \Bigl( u_{r}^{\eta}\tparen{\car_{B_r\setminus \set{x}} \cdot\gamma+\delta_\bullet}-u^\eta_{r}\tparen{\car_{B_r\setminus\set{x}}\cdot\gamma} \Bigr)\Bigr|^2(x)
%\end{align}
%\end{proof}
Based on the truncated square field $\cdc^\U_r$, we introduce the truncated form on $\U$ defined on a certain core. 
\begin{defs}[Core]\label{d:core}For $r>0$, let $\mathcal C_r$ be defined as the space of $\mu$-classes of measurable functions $u$ so that
\begin{enumerate}[$(a)$]
\item\label{i:d:core1} $u\in L^\infty(\mu)$; 
\item\label{i:d:core2} $u_r^\eta \in \Lip_b(\U(B_r), \mssd_\U)$ for $\QP$-a.e.~$\eta$;

%$\mathcal U_{\gamma, x}(u)|_{B_r} \in \Lip(B_r, \mssd)$ for $\gamma \in \U$, $x \in \gamma_{B_r}$ and $r>0$, and\footnote{Change this definition simply as $u_r^\eta \in \Lip(\U(B_r), \mssd_\U)$ for $\QP$-a.e.~$\eta$, and prepare the lemma stating that this implies $\mathcal U_{\gamma, x}(u) \in \Lip(B_r, \mssd)$ by Lem.~\ref{l:LSO}, and delete Lem.~\ref{l:SLFP}}
%\begin{align}\label{e:d:core2}  L_r(u):=\sup_{\gamma\in \U}\sup_{x \in \gamma} \Lip_\mssd(\mathcal U_{\gamma, x}(u)|_{B_r})<\infty \ ;
%\end{align}

\item\label{i:d:core3}  The following integral is finite:
\begin{align} \label{eq:VariousFormsA} 
\E^{\U, \QP}_{r} (u) :=\int_{\U} \cdc^{\dUpsilon}_r(u) \diff\QP <\infty \fstop
\end{align} 
%$\EE{\dUpsilon}{\QP}_{r}(u) := \int_{\U} \cdc^{\dUpsilon, \QP}_r(u,  u) d\QP < \infty$.

%\item\label{i:d:TLS:3} $\Bo{\T}\subset \A\subset \Bo{\T}^\mssm$ and $\A$ is \emph{$\mssm$-essentially countably generated}, i.e.\ there exists a countably generated $\sigma$-sub\-al\-ge\-bra~$\A_0$ of~$\A$ so that for every~$A\in \A$ there exists $A_0\in \A_0$ with~$\mssm(A\triangle A_0)=0$.
%
%\item\label{i:d:TLS:4} $\msE \subset \A_\mssm$ is a \emph{localizing ring}, i.e.\ it is a ring ideal of~$\A$, and there exists a \emph{localizing sequence}~$\seq{E_h}_h\subset \msE$ so that $\msE=\cup_{h\geq 0} (\A\cap E_h)$.
%
%\item\label{i:d:TLS:5} for every~$x\in X$ there exists a $\T$-neighborhood~$U_x$ of~$x$ so that~$U_x\in\msE$.
\end{enumerate}
Note that, thanks to Lem.~\ref{l:LSO}, if a measurable function $u:\U \to \R$ satisfies~\ref{i:d:core2}, then $\mathcal U_{\gamma, x}(u)|_{B_r} \in \Lip(B_r, \mssd) \subset W^{1,2}(\mssm_r)$. Thus, the expression~$\cdc^{\U}_r(u)$ in \eqref{eq:VariousFormsA} is well-posed. It will be proved in Prop.~\ref{t:ClosabilitySecond} that
$\mathcal C_r$ is non-trivial in the sense that every $\QP$-measurable bounded $\bar{\mssd}_{\U}$-Lipschitz functions on $\U$ belongs to $\mathcal C_r$. 
\end{defs}
%\begin{rem} \label{r:SLRAD}
%and the Rademacher-type property on $B_r$, Condition~\ref{i:d:core2} in Def.~\ref{d:core} is equivalent to say that $\mathcal U_{\gamma, x}(u)|_{B_r}$ has a $\mssd$-Lipschitz $\mssm_r$-modification on $B_r$ for $\mu$-a.e.\ $\gamma \in \U$, any $x \in \gamma_{B_r}$ and any $r>0$.
%\end{rem}

%\begin{lem} \label{l:SLFP}
%Suppose that $u$ satisfies Condition~\ref{i:d:core2} in Def.~\ref{d:core}. Then, for any $k \in \N_0$, $\eta \in \U$ and and $r>0$
%\begin{align*}
%u_r^\eta|_{\U^k(B_r)} \in \Lip(\U^k(B_r), \mssd_\U) \quad  \text{with}  \quad \Lip_\mssd(u_r^\eta|_{\U^k(B_r)}) \le \sqrt{k}L_r(u)  \comma 
%\end{align*}
%where $L_r(u)$ has been defined in~\eqref{e:d:core2}.
%\end{lem}
%\begin{proof} 
%%Let $\Omega \subset \U$ with $\mu(\Omega)=1$ so that for every $\gamma \in \Omega$ and $x \in \gamma_{B_r}$, $\mathcal U_{\gamma, x}(u)|_{B_r} \in W^{1,\infty}(\mssm_r)$. Thanks to Lem.~\ref{l:FL}, the section~$\Omega_r^\eta \subset \U(B_r)$ satisfies~$\QP_r^\eta(\Omega_r^\eta)=1$ for $\QP$-a.e.~$\eta$.
%% By Rem.~\ref{r:SLRAD}, there exists a measurable set $\Sigma \subset B_r$ with $\mssm_r(\Sigma^c)=0$ on which $\mathcal U_{\gamma, x}(u)|_{B_r}$ is $\mssd$-Lipschitz. Let $k=k(\eta)$ as in~\eqref{e:R1} and set $\Lambda=\Omega_r^\eta \cap \Sigma^{\odot k}$. By the DLR equation~\eqref{d:cp}, $\QP_r^\eta(\Lambda)=1$. 
%% As $\mu_r^\eta$ is absolutely continuous with respect to the Poisson measure~$\pi_{\mssm_r}$, we may assume that every $\gamma \in \Omega_r^\eta$ does not have multiple points, i.e., $\gamma(\{x\}) \in \{0, 1\}$ for every $x \in B_r$.  
% Take $\gamma, \zeta \in \U^k(B_r)$ with $\gamma=\sum_{i=1}^k \delta_{x_i}$ and $\zeta=\sum_{i=1}^k \delta_{y_i}$. 
%% As $\mu_r^\eta$ is absolutely continuous with respect to the Poisson measure~$\pi_{\mssm_r}$, we may assume that every $\gamma, \eta$ do not have multiple points, i.e., $\gamma(\{x\}) \in \{0, 1\}$ for every $x \in B_r$. 
% Up to parametrisation of the index $\{i\}_{i=1}^k$ for $\{x_i\}$ and $\{y_i\}$, we may assume without loss of generality that the optimal coupling of~$\mssd_\U$ is attained as
% $$\mssd_\U(\gamma, \zeta)^2=\sum_{i=1}^k \mssd(x_i, y_i)^2 \fstop$$
% Set $\eta_0=\gamma+\eta_{B_r^c}$ and $\eta_l=\1_{B_r \setminus \{x_1, x_2, \ldots, x_{l}\}}\cdot\gamma + \sum_{i=1}^{l-1}\delta_{y_i} +\eta_{B_r^c}$.
% We have that   
% \begin{align*}
% |u_r^\eta(\gamma)-u_r^\eta(\zeta)|  &=  |u(\gamma+\eta_{B_r^c})-u(\zeta+\eta_{B_r^c})| 
% \\
% &\le \sum_{l=1}^k\bigl|u(\eta_l +\delta_{x_l}) - u(\eta_l +\delta_{y_l}) \bigr|
%% \\
%% &\le  \bigl|u(\1_{B_r \setminus \{x_1\}}\cdot\gamma +\delta_{x_1}+\eta_{B_r^c}) - u(\1_{B_r \setminus \{x_1\}}\cdot\gamma +\delta_{y_1}+\eta_{B_r^c}) \bigr|
%% \\
%% & \quad +  \bigl|u(\1_{B_r \setminus \{x_1, x_2\}}\cdot\gamma +\delta_{y_1}+\delta_{x_2}+\eta_{B_r^c}) - u(\1_{B_r \setminus \{x_1, x_2\}}\cdot\gamma +\delta_{y_1}+\delta_{y_2}+\eta_{B_r^c}) \bigr|
%% \\
%% &\quad +\cdots + \biggl|u\Bigl(\1_{B_r \setminus \{x_1, x_2, \ldots, x_k\}}\cdot\gamma +\sum_{i=1}^{k-1}\delta_{y_i}+\delta_{x_k}+\eta_{B_r^c}\Bigr) 
%% \\
%% & \qquad - u\Bigl(\1_{B_r \setminus \{x_1, x_2, \ldots, x_k\}}\cdot\gamma +\sum_{i=1}^{k-1}\delta_{y_i}+\delta_{y_k}+\eta_{B_r^c}\Bigr) \biggr| 
% \\
% &\le \sum_{l=1}^k \Lip_{\mssd}\bigl(\mathcal U_{\eta_{l-1}, x_l}(u)\bigr)\mssd(x_l, y_l) 
% \\
% & \le \sqrt{k}L_r(u)\mssd_\U(\gamma, \zeta) \fstop
% \end{align*}
%As this argument works for all $k \in \N_0$, the proof is completed.
%\end{proof}

%%Define
%%We first confirm that the space $\mathcal C_r$ contains sufficiently many functions. 
%%\begin{lem} \label{p:CL}
%%$\Lip_b(\mssd_\U, \QP) \subset \mathcal C_r$  for any $r>0$. In particular, $\mathcal C_r$ is dense in~$L^2(\mu)$.
%%\end{lem}
%%\begin{proof}
%%The verification of (a) in~Def.~\ref{d:core} is obvious. By Prop.~\ref{p:ULP}, it holds that $\mathcal U_{\gamma, x}(u) \in \Lip(B_r)$ whenever $u \in \Lip(\U(B_r), \mssd_\U)$.
%%Combining it with the Rademacher theorem on any convex domain in the Euclidean space implying $\Lip(B_r) \subset W^{1,2}_{loc}(\mssm_r)$ and $|\nabla u| \le \Lip(u)$ for $u \in \Lip(B_r)$, we verified $(b)$ and $(c)$. Since $\Lip_b(\mssd_\U, \QP) \subset L^2(\mu)$ is dense (e.g., \cite[Prop.\ 4.1]{AmbGigSav14}), the latter statement has been confirmed.  
%%\end{proof}
%%Throughout this section,~$\QP$ will denote a probability measure on the space $\ttonde{\dUpsilon,\A_\mrmv(\msE)}$.
%
%%\begin{defs}[Square field on $\U(B_r)$] \label{d:sff}
%%For a $\mu$-measurable function $u: \U(B_r) \to \R$ satisfying $u|_{\U^k(B_r)} \in \mathcal D(\E_r^{k, \eta})$ for any $k \in \N_0$, define
%%$$\Gamma^{\U(B_r)}(u) := \sum_{k=0}^\infty \bigl|\nabla^{\odot k} u |_{\U^k(B_r)}\bigr|^2 \comma$$
%%and 
%%$$\E_{r}^{\eta}(u):=\int_{\U(B_r)} \Gamma^{\U(B_r)}(u) \diff\mu_{r}^{\eta} \comma \quad \mathcal D(\E_r^{\eta}):=\{u: \U(B_r) \to \R,\ \E_r^\eta(u)<\infty \} \fstop$$ 
%%\end{defs}
%In order to show that the form $\E^{\U, \QP}_{r} $ defined in \eqref{eq:VariousFormsA} is a Dirichlet form, we compare it with the following square field defined on $\U(B_r)$.
%Recall that the form $\E^{\U(B_r), \mu_r^{k, \eta}}$ has been defined in \eqref{eq:form2}.
\begin{defs}[Square field on $\U(B_r)$]\label{d:DT2} 
Fix $r>0$ and $\eta \in \U$. For a $\mu$-measurable function $u: \U(B_r) \to \R$ satisfying $u|_{\U^k(B_r)} \in \mathcal D(\E^{\U(B_r), \mu_r^{k, \eta}})$ for any $k \in \N_0$, we define the following square field operator on $\U(B_r)$:
\begin{align} \label{d:GSF}
\Gamma^{\U(B_r)}(u) := \sum_{k=0}^\infty \Bigl|\nabla^{\odot k} \bigl(u |_{\U^k(B_r)}\bigr)\Bigr|^2 \comma
\end{align}
and define the following form:
\begin{align*}
\E^{\U(B_r), \mu_r^\eta}(u)&:=\int_{\U(B_r)} \Gamma^{\U(B_r)}(u) \diff\mu_{r}^{\eta} \comma
\\
\dom{\E^{\U(B_r), \mu_r^\eta}}&:=\{u: \U(B_r) \to \R,\ \E^{\U(B_r), \mu_r^\eta}(u)<\infty \} \fstop
\end{align*}
Due to the number-rigidity~\ref{ass:Rig}, the Dirichlet form~$\E^{\U(B_r), \mu_r^\eta}$ is equal to $\E^{\U(B_r), \mu_r^{k, \eta}}$ for some $k=k(\eta)$ up to the normalising constant multiplication, therefore, it is a Dirichlet form as well. 
The corresponding semigroup operator is denoted by $\sem{T_t^{\U(B_r), \QP_r^\eta}}$.

%\begin{enumerate}[$(a)$]
%\item {\bf (truncated square field on $\U(B_r)$)} For $u: \U \to \R$, define $\mathcal U_{\gamma, x}(u): \R \to \R$ by 
%$$\mathcal U_{\gamma, x}(u)(y):=u\ttonde{\car_{X\setminus\set{x}}\cdot\gamma + \delta_y}-u\ttonde{\car_{X\setminus\set{x}}\cdot\gamma} \comma \quad \gamma \in \U,\quad x \in X \fstop$$
%If $\mathcal U_{\gamma, x}(u) \in W^{1,2}_{loc}(\mssm_r)$ for $\mu$-a.e.\ $\gamma \in \U$ and every $x \in B_r$, define 
%\begin{equation}\label{eq:d:LiftCdCRep}
%\begin{gathered}
%\cdc^{\dUpsilon}_r(u)(\gamma):=\cdc^{\dUpsilon}_r(u,  u)(\gamma) := \sum_{x\in \gamma_{B_r}} |\nabla \tonde{\mathcal U_{\gamma, x}(u)}|^2(x)\quad \text{$\mu$-a.e.\ $\gamma$} \fstop
%%\sum_{i,j=1}^{k,m} (\partial_i F)(\mbff^\trid\gamma) \cdot (\partial_j G)(\mbfg^\trid\gamma) \cdot \cdc( f_i,  g_j)^\trid \gamma \quad \text{$\QP$-a.e.\ $\gamma$}\comma
%% u, v\in \Cyl{\Dz} \fstop
%\end{gathered}
%\end{equation}
%\item {\bf (square field on $\U(B_r)$)} Fix $r>0$ and $\eta \in \U$. For a $\mu$-measurable function $u: \U(B_r) \to \R$ satisfying $u|_{\U^k(B_r)} \in \mathcal D(\E_r^{k, \eta})$ for any $k \in \N_0$, define
%$$\Gamma^{\U(B_r)}(u) := \sum_{k=0}^\infty \Bigl|\nabla^{\odot k} \bigl(u |_{\U^k(B_r)}\bigr)\Bigr|^2 \comma$$
%and define the following form
%$$\E_{r}^{\eta}(u):=\int_{\U(B_r)} \Gamma^{\U(B_r)}(u) \diff\mu_{r}^{\eta} \comma \quad \mathcal D(\E_r^{\eta}):=\{u: \U(B_r) \to \R,\ \E_r^\eta(u)<\infty \} \fstop$$
%The form $(\E_{r}^{\eta}, \mathcal D(\E_r^{\eta}))$ is closed (see \cite[Exercise 3.9]{MaRoe90}).
%\end{enumerate}
\end{defs} 
\begin{rem}\label{r:NRI}
The number-rigidity~\ref{ass:Rig} is not necessary to conclude that $\E^{\U(B_r), \mu_r^\eta}$ is a Dirichlet form since
any countable sum of Dirichlet forms is a Dirichlet form (see e.g., \cite[Exercise~3.9 in p.31]{MaRoe90}) and we know that $(\E^{\U(B_r), \mu_r^{k, \eta}}, \dom{\E^{\U(B_r), \mu_r^{k, \eta}}})$ is a Dirichlet form for every $k \in \N$ by construction. 
\end{rem}

Before discussing properties of truncated forms, we prepare a lemma, which states that the operation $(\cdot)_r^\eta$ defined in~\eqref{e:SEF} maps from $\Lip(\U, \bar{\mssd}_\U)$ to $\Lip(\U(B_r), \mssd_\U)$ and contracts Lipschitz constants. 
\begin{lem}\label{l:SEF3}
Let $u \in \Lip(\U, \bar{\mssd}_\U)$. Then, $u_r^\eta \in \Lip(\U(B_r), \mssd_\U)$ and 
\begin{align} \label{e:SEF3}
\Lip_{\mssd_\U}(u_r^\eta) \le \Lip_{\bar{\mssd}_\U}(u) \comma \quad \eta \in \U\comma \quad r>0\fstop
\end{align}
\end{lem}
\begin{proof}
Let $\gamma, \zeta \in \U(B_r)$ and $\eta \in \U$. Then, 
\begin{align*}
|u_{r}^\eta(\gamma)-u_{r}^\eta(\zeta)|=|u(\gamma+\eta_{B_r^c})-u(\zeta+\eta_{B_r^c})| &\le \Lip_{\bar{\mssd}_\U}(u) \bar{\mssd}_\U(\gamma+\eta_{B_r^c}, \zeta+\eta_{B_r^c}) 
\\
&= \Lip_{\bar{\mssd}_\U}(u) \mssd_\U(\gamma, \zeta) \fstop
\end{align*}
The proof is completed. 
\end{proof}

The following proposition relates the two square fields $\cdc^{\U}_r$ and $\cdc^{\U(B_r)}$. %\begin{equation*}
% \begin{align} \label{e:SEF}
 %\text{$u_{r}^\eta(\gamma)\eqdef  u(\gamma+\eta_{B_r^\complement})$ for $\gamma\in \dUpsilon(B_r)$.}
 %\end{align}
%Recall that the projection map $\pr_r: \U \to \U(B_r)$ is defined by the restriction $\gamma \mapsto \gamma_{B_r}$.
\begin{prop}[Truncated form]%[{\cite[Lem.~3.40, Prop.~3.50]{LzDSSuz21}}]
\label{t:ClosabilitySecond} 
%Let~$(\mcX,\cdc)$ be a \TLDS,~$\QP$ be a probability measure on $\ttonde{\dUpsilon,\A_\mrmv(\msE)}$ satisfying Assumptions~\ref{ass:CE} and~\ref{ass:ConditionalClos}. Then, the following hold:
The following relations hold on~$\mathcal C_r$: 
\begin{align}\label{eq:p:MarginalWP:0}
\cdc^{\dUpsilon}_r(u)(\gamma+\eta_{B_r^c}) &= \cdc^{\dUpsilon(B_r)}(u_{r}^\eta)(\gamma) \comma \quad \text{$\QP$-a.e.~$\eta$, \ $\mu_{r}^\eta$-a.e.~$\gamma \in \U(B_r)$}  \comma
\\
\E_{r}^{\U, \mu}(u) &=\int_\dUpsilon  \E^{\U(B_r), \mu_r^\eta}(u_{r}^\eta) \diff\QP(\eta)\comma \quad u \in \mathcal C_r\fstop \notag
%:=\{\gamma \in \U: \gamma_{B_r^c}=\eta_{B_r^c}\} \comma 
\end{align}
Furthermore, the Rademacher-type property holds: $\Lip_b(\bar{\mssd}_\U, \QP)\subset \mathcal C_r$ and 
%\purple{Use the large distance $\bar{\mssd}_\U$ instead of $\mssd_\U$ taking finite values only when $\gamma_{B_r^c} =\eta_{B_r^c}$ for sufficiently large $r>0$. }
\begin{align}\label{p:Rad}
 \cdc^\U_r(u) \le \Lip_{\bar{\mssd}_\U}(u)^2 \qquad u \in \Lip_b(\bar{\mssd}_\U, \QP) \fstop
\end{align}
As a consequence, the form $(\E^{\U, \QP}_{r}, \mathcal C_r)$ in~\eqref{eq:VariousFormsA}  is a densely defined closable Markovian form and 
%\begin{align} \label{eq:VariousFormsA} 
%\E^{\U, \QP}_{r} (u) :=\int_{\U} \cdc^{\dUpsilon}_r(u) \diff\QP \comma
%\end{align} 
%and the following relation holds:
%\begin{align} \label{eq:VariousFormsAA} 
%\E_{r}^{\U, \mu}(u) &=\int_\dUpsilon  \E^{\U(B_r), \mu_r^\eta}(u_{r}^\eta) \diff\QP(\eta) \comma\quad r>0\comma\quad u\in \Lip_b(\mssd_\U, \QP) \fstop 
%\end{align} 
% Furthermore, $\cdc^\U_r$ and $\cdc^{\U(B_r)}$ have the following relation: for $\mu$-a.e.\ $\eta$, 
%\begin{align}\label{eq:p:MarginalWP:0}
%\cdc^{\dUpsilon}_r(u)(\gamma+\eta_{B_r^c}) &= \cdc^{\dUpsilon(B_r)}(u_{r}^\eta)(\gamma) \quad \text{$\mu_{r}^\eta$-a.e.~$\gamma \in \U(B_r)$}  \comma
%%:=\{\gamma \in \U: \gamma_{B_r^c}=\eta_{B_r^c}\} \comma 
%\\
%\E_{r}^{\U, \mu}(u) &=\int_\dUpsilon  \E^{\U(B_r), \mu_r^\eta}(u_{r}^\eta) \diff\QP(\eta) \comma\quad r>0\comma\quad u\in \Lip_b(\mssd_\U, \QP) \fstop \notag
%\end{align}
%The closure of $(\E_{r}^{\U, \mu}, \mathcal C_r)$ is denoted by $(\E_{r}^{\U, \mu}, \dom{\E_{r}^{\U, \mu}})$.  
the closure~$(\E_r^{\U, \QP}, \dom{\E_r^{\U, \QP}})$ is a local Dirichlet form on~$L^2(\mu)$. The $L^2$-semigroups corresponding to $(\E_{r}^{\U, \mu}, \dom{\E_{r}^{\U, \mu}})$ is denoted by $\sem{T_{r, t}^{\U, \QP}}$. 
\end{prop}
%\begin{lem}
%%Fix~$E\in\msE$, and let~$\Lambda_{\eta,E^\complement}$ be defined as in~\eqref{eq:RoeSch99Set}.
%%Then, for every~$u,v\in\mathcal C_r$,
%%\begin{equation}\label{eq:p:WPSquareField:2}
%%\rep\cdc^{\dUpsilon(E)}(u_{E,\eta}, v_{E,\eta})\circ \pr_E\equiv \cdc^\dUpsilon_E(u,v) \qquad \text{on} \quad \Lambda_{\eta, E^\complement} \comma\qquad \eta\in\dUpsilon \fstop
%%\end{equation}
%
%\begin{proof}
%
%\end{proof}
%\end{lem}

\begin{proof}
We first prove \eqref{eq:p:MarginalWP:0}. Let $u \in \mathcal C_r$. Thanks to~\ref{i:d:core2} in~Def.~\ref{d:core} and Lem.~\ref{l:LSO},    
% that
$$\mathcal U_{\gamma, x}(u) \in \Lip(B_r, \mssd) \comma \quad   \QP\text{-a.e.}\ \gamma \quad r>0 \fstop$$
Thus, noting $\Lip(B_r, \mssd) \subset W^{1,2}(\mssm_r)$, the LHS of~\eqref{eq:p:MarginalWP:0} is well-defined. Thus, there exists $\Omega \subset \U$ with $\QP(\Omega)=1$ so that the LHS of~\eqref{eq:p:MarginalWP:0} is well-defined everywhere on $\Omega$. 
The RHS of~\eqref{eq:p:MarginalWP:0} is also well-defined by~\ref{i:d:core2} in~Def.~\ref{d:core} and by $\Lip_b(\U(B_r), \mssd_\U) \subset \dom{\E^{\U(B_r), \mu_r^\eta}}$ by construction. 
Let $\Omega_r^\eta$ be a section as defined in~\eqref{e:SEF2}. 
%It holds that  $\QP_r^\eta(\Omega_r^\eta)=1$ for $\QP$-a.e.~$\eta$ by \eqref{p:ConditionalIntegration2}.
 %we can take  a measurable set~$\Omega =\Omega(\eta)\subset \U(B_r)$ with $\mu_r^\eta(\Omega)=1$ so that \eqref{eq:p:MarginalWP:0} is well-defined for every $\gamma \in \Omega$. 
 As $\mu_r^\eta$ is absolutely continuous with respect to the Poisson measure~$\pi_{\mssm_r}$ and the Poisson measure does not have multiple points almost everywhere, we may assume that every $\gamma \in \Omega_r^\eta$ does not have multiple points, i.e., $\gamma(\{x\}) \in \{0, 1\}$ for every $x \in B_r$. 
Let $\gamma \in \Omega_r^\eta \cap \U^k(B_r)$. Then, according to~\eqref{d:GSF}, 
%Then, according to~\eqref{d:GSF}, 
\begin{align*}
&\cdc^\dUpsilon_r(u) (\gamma+\eta_{B_r^\complement})
\\
&= \sum_{x\in\gamma} \Bigl|\nabla \Bigl( u\tparen{\car_{X\setminus \set{x}} \cdot(\gamma+\eta_{B_r^c})+\delta_\bullet}-u\tparen{\car_{X\setminus\set{x}}\cdot(\gamma+\eta_{B_r^c})} \Bigr)\Bigl|^2(x) 
%\\
%&=  \sum_{i=1}^k\Bigl|\nabla \bigl(u_r^\eta \bigr)\Bigr|^2(x_i)
\\
&=\sum_{x\in\gamma}  \Bigl|\nabla \Bigl( u_{r}^{\eta}\tparen{\car_{X\setminus \set{x}} \cdot\gamma+\delta_\bullet}-u^\eta_{r}\tparen{\car_{X\setminus\set{x}}\cdot\gamma} \Bigr)\Bigr|^2(x) 
\\
&=  \sum_{x\in\gamma} \bigl|\nabla u_{r}^{\eta}\tparen{\car_{X\setminus \set{x}} \cdot\gamma+\delta_\bullet}\bigr|^2(x)
\\
&=  \Bigl|\nabla^{\odot k} \bigl(u_r^\eta \bigr)\Bigr|^2(\gamma)
\\
%\\
%=&\ \sum_{x\in\gamma_{B_r}}  \cdc^{X}\paren{u_{r}^\eta\tparen{\car_{X\setminus\set{x}}\cdot\gamma_{B_r}+\delta_\bullet}-u_{r}^\eta\tparen{\car_{X\setminus\set{x}}\cdot\gamma_{B_r}}}(x)
%\\
%=&\ \sum_{x\in\gamma_E} \tparen{(\gamma_E+\eta_{E^\complement})_x}^{-1}\cdot 
%\\
%&\quad \cdot SF{X}{\mssm}\paren{\rep u\tparen{\car_{X\setminus\set{x}}\cdot\gamma_E+(\gamma_E)_x\delta_\bullet+\eta_{E^\complement}}- u\tparen{\car_{X\setminus\set{x}}\cdot\gamma_E+\eta_{E^\complement}}}(x)
%\\
%=&\ \sum_{x\in\gamma_E} \tparen{(\gamma_E+\eta_{E^\complement})_x}^{-1}\cdot
%\\
%&\quad \cdot SF{X}{\mssm}\paren{u\tparen{\car_{X\setminus\set{x}}\cdot(\gamma_E+\eta_{E^\complement})+(\gamma_E+\eta_{E^\complement})_x\delta_\bullet}-\rep u\tparen{\car_{X\setminus\set{x}}\cdot(\gamma_E+\eta_{E^\complement})}}(x)
%\\
&=  \cdc^{\dUpsilon(B_r)}(u_{r}^\eta)(\gamma)
%\\
%&=\cdc^\dUpsilon_r(u) (\gamma) \comma
\end{align*}
where the first equality is the definition of the square field~$\cdc^\dUpsilon_r$;  the third equality holds as $u^\eta_{r}\tparen{\car_{X\setminus\set{x}}\cdot\gamma}$ does not depend on the variable denoted as~$\bullet$ on which the weak gradient~$\nabla$ operates; the fourth equality followed from the definition of the symmetric gradient operator $\nabla^{\odot k}$, for which we used the fact that $\gamma \in \Omega_r^\eta$ does not have multiple points. 
%
%\begin{align*}
%&\cdc^{\dUpsilon(B_r)}(u_{r}^\eta)(\gamma)
%\\
%&=  \Bigl|\nabla^{\odot k} \bigl(u_r^\eta \bigr)\Bigr|^2(\gamma)
%%\\
%%&=  \sum_{i=1}^k\Bigl|\nabla \bigl(u_r^\eta \bigr)\Bigr|^2(x_i)
%\\
%&= \sum_{x\in\gamma} \bigl|\nabla u_{r}^{\eta}\tparen{\car_{X\setminus \set{x}} \cdot\gamma+\delta_\bullet}\bigr|^2(x)
%\\
%&= \sum_{x\in\gamma}  \Bigl|\nabla \Bigl( u_{r}^{\eta}\tparen{\car_{X\setminus \set{x}} \cdot\gamma+\delta_\bullet}-u^\eta_{r}\tparen{\car_{X\setminus\set{x}}\cdot\gamma} \Bigr)\Bigr|^2(x)
%\\
%&= \sum_{x\in\gamma} \Bigl|\nabla \Bigl( u\tparen{\car_{X\setminus \set{x}} \cdot(\gamma+\eta_{B_r^c})+\delta_\bullet}-u\tparen{\car_{X\setminus\set{x}}\cdot(\gamma+\eta_{B_r^c})} \Bigr)\Bigl|^2(x)
%\\
%%\\
%%=&\ \sum_{x\in\gamma_{B_r}}  \cdc^{X}\paren{u_{r}^\eta\tparen{\car_{X\setminus\set{x}}\cdot\gamma_{B_r}+\delta_\bullet}-u_{r}^\eta\tparen{\car_{X\setminus\set{x}}\cdot\gamma_{B_r}}}(x)
%%\\
%%=&\ \sum_{x\in\gamma_E} \tparen{(\gamma_E+\eta_{E^\complement})_x}^{-1}\cdot 
%%\\
%%&\quad \cdot SF{X}{\mssm}\paren{\rep u\tparen{\car_{X\setminus\set{x}}\cdot\gamma_E+(\gamma_E)_x\delta_\bullet+\eta_{E^\complement}}- u\tparen{\car_{X\setminus\set{x}}\cdot\gamma_E+\eta_{E^\complement}}}(x)
%%\\
%%=&\ \sum_{x\in\gamma_E} \tparen{(\gamma_E+\eta_{E^\complement})_x}^{-1}\cdot
%%\\
%%&\quad \cdot SF{X}{\mssm}\paren{u\tparen{\car_{X\setminus\set{x}}\cdot(\gamma_E+\eta_{E^\complement})+(\gamma_E+\eta_{E^\complement})_x\delta_\bullet}-\rep u\tparen{\car_{X\setminus\set{x}}\cdot(\gamma_E+\eta_{E^\complement})}}(x)
%%\\
%&= \cdc^\dUpsilon_r(u) (\gamma+\eta_{B_r^\complement})
%%\\
%%&=\cdc^\dUpsilon_r(u) (\gamma) \comma
%\end{align*}
%where the second equality followed from the definition of the symmetric gradient operator $\nabla^{\odot k}$, for which we used the fact that $\gamma \in \Omega$ does not have multiple points;  the third equality follows simply as $u^\eta_{r}\tparen{\car_{B_r\setminus\set{x}}\cdot\gamma}$ does not depend on the variable denoted as~$\bullet$, on which the weak gradient~$\nabla$ operates; the fifth equality followed from the definition of the square field~$\cdc^\dUpsilon_r$.  
As this argument holds for arbitrary $k \in \N_0$, \eqref{eq:p:MarginalWP:0} has been shown. The locality and the Markov property of~$\E_{r}^{\U, \mu}$ follow from~\eqref{eq:p:MarginalWP:0} and the fact that $\E^{\U(B_r), \mu_r^\eta}$ possesses the corresponding properties by construction. 
% The Markov property of~$\E_{r}^{\U, \mu}$ also follows by the same argument and 
%\begin{equation*}
%\cdc^{\dUpsilon(E)}(u_{E,\eta})\circ \pr_E\equiv \cdc^\dUpsilon_E(u) \qquad \text{on} \quad \Lambda_{\eta, E^\complement} \comma \qquad \eta\in\dUpsilon\fstop
%\end{equation*}
%The conclusion follows by polarization.

 %The formula~\eqref{eq:p:MarginalWP:0} follows from  \cite[Lem.~3.40]{LzDSSuz21}. 
 We now show the Rademacher-type property: $\Lip_b(\bar{\mssd}_\U, \QP) \subset \mathcal C_r$ and
\begin{align} \label{e:RD}
\cdc^\U_r(u) \le \Lip_{\bar{\mssd}_\U}(u)^2  \quad \forall u \in \Lip_b(\bar{\mssd}_\U, \QP) \quad \forall r>0\fstop
\end{align}
We first show $\Lip_b(\bar{\mssd}_\U, \QP) \subset \mathcal C_r$. The verification of~\ref{i:d:core1} in Def.~\ref{d:core} is obvious. The verification of~\ref{i:d:core2} in Def.~\ref{d:core} follows from the Lipschitz contraction~\eqref{e:SEF3} of the operator~$(\cdot)_r^\eta$. The verification of~\ref{i:d:core3} in Def.~\ref{d:core} follows by showing~\eqref{e:RD} as~$\QP$ is a probability measure.  

We now prove~\eqref{e:RD}.
%We verify $\Lip_b(\mssd_\U, \QP) \subset \mathcal C_r$, namely that any element in~$\Lip(\mssd_\U, \QP)$ satisfies Conditions~(a)--(c) in~Def.~\ref{d:core}. The verification of (a) in~Def.~\ref{d:core} is obvious. By Prop.~\ref{p:ULP}, it holds that $\mathcal U_{\gamma, x}(u) \in \Lip(B_r)$ whenever $u \in \Lip(\mssd_\U)$.
%Combining it with the Rademacher theorem on $B_r$ implying $\Lip(B_r) \subset W^{1,2}_{loc}(\mssm_r)$, we verified~$(b)$.  
As the Cheeger energy~$\Ch_{\mssd_\U, \QP_r^{k, \eta}}$ coincided with the form~$\E^{\U(B_r), \QP_r^{k, \eta}}$ by Prop.~\ref{p:BE2}, the Rademacher-type property for~$\E^{\U(B_r), \QP_r^{k, \eta}}$ follows from that for~$\Ch_{\mssd_\U, \QP_r^{k, \eta}}$, the latter of which is an immediate consequence by the definition of the Cheeger energy. Therefore, we have that
\begin{align} \label{e:RD2}
\cdc^{\U(B_r)}(u) \le \Lip_{\mssd_\U}(u)^2  \quad \forall u \in \Lip(\U(B_r), \mssd_\U) \quad \forall r>0\fstop
\end{align}
In view of the relation between~$\cdc^\U_r$ and~$\cdc^{\U(B_r)}$ in~\eqref{eq:p:MarginalWP:0} and the Lipschitz contraction~\eqref{e:SEF3} of the operator~$(\cdot)_r^\eta$, we concluded~\eqref{e:RD}. 
%As a consequence,  the integration in the RHS of~\eqref{eq:VariousFormsA} is well-defined on~$\Lip_b(\mssd_\U, \QP) $ as~$\QP$ is a probability measure. By these arguments, we have also concluded that $\Lip_b(\mssd_\U, \QP) \subset \mathcal C_r$. 
%Therefore,  holds by~\eqref{eq:p:MarginalWP:0} and~\eqref{eq:VariousFormsA}. 

Noting that~$\Lip_b(\mssd_\U, \QP) \subset L^2(\mu)$ is dense (e.g., \cite[Prop.\ 4.1]{AmbGigSav14}) and the fact that $\Lip_b(\mssd_\U, \QP) \subset 
\Lip_b(\bar{\mssd}_\U, \QP) \subset \mathcal C_r$ by \eqref{e:LLR} and \eqref{e:RD}, we obtain that the form~$(\E_{r}^{\U, \mu}, \mathcal C_r)$ is densely defined.  

We now show the closability. Noting that $\E^{\U(B_r), \QP_r^\eta}$ is closable for $\mu$-a.e.~$\eta$ by Prop.~\ref{p:BE2},  the superposition form~$(\bar{\E}^{\U, \QP}_r,\dom{\bar{\E}^{\U, \QP}_r})$ (defined below in Def.~\ref{d:SPF}) is closable (indeed it is closed) by \cite[Prop.~V.3.1.1]{BouHir91}. As the two forms~$(\E_{r}^{\U, \mu}, \mathcal C_r)$ and~$(\bar{\E}^{\U, \QP}_r,\dom{\bar{\E}^{\U, \QP}_r})$ coincide on $\mathcal C_r$ by definition and $\mathcal C_r\subset \dom{\bar{\E}^{\U, \QP}_r}$ by construction, the closability of~$(\E_{r}^{\U, \mu}, \mathcal C_r)$ is inherited from the closedness of the superposition form~$(\bar{\E}^{\U, \QP}_r,\dom{\bar{\E}^{\U, \QP}_r})$. 
 The proof is complete.
%The closability of $\E_r^{\U, \QP}$ on~$\mathcal C_r$ now follows from 
%The more detailed proof can be found in~\cite[Prop.~3.50]{LzDSSuz21}. 
%\purple{See the rest of the argument in the first paper Section 3.3.2. Our definition of the core $\mathcal C_r$ is slightly deviated from there, but the same proof works.}
\end{proof}
%\begin{rem}
%The setting of Prop.~\ref{t:ClosabilitySecond} slightly deviates from the one in \cite[Lem.~3.40, Prop.~3.50]{LzDSSuz21}\footnote{Check ref.~no. This remark can be deleted.}, where we proved the corresponding statement on a certain core $\mathcal C$ possibly smaller than $\mathcal C_r$. The proof is, however, essentially the same.
%%The closability of~a similar form to $\E_{r}^{\U, \mu}$ has been proved in~\cite[Prop.~3.50]{LzDSSuz21}. The difference of our form $\E_{r}^{\U, \mu}$ is only the choice of the core $\mathcal C_r$, which is in general larger than the core taken in~\cite[Prop.~3.50]{LzDSSuz21}. The proof of the closability is, however, identical to the proof there. 
%\end{rem}
%Define the superposition Dirichlet form:
\subsection{Superposition form}
The superposition of the Dirichlet form $\E^{\U(B_r), \QP_r^\eta}$ onto $\U$ is now defined below. 
\begin{defs}[Superposition Dirichlet form,~e.g., {\cite[Prop.\ V.3.1.1]{BouHir91}}] \label{d:SPF}
\begin{align} \label{eq:SP} 
\mathcal D(\bar{\E}^{\U, \mu}_r) &:= \biggl\{u \in L^2(\QP): \ \int_\dUpsilon  \E^{\U(B_r), \QP_r^\eta}(u_{r}^\eta) \diff\QP(\eta)<\infty  \biggr\} \comma
\\
\bar{\E}^{\U, \QP}_r(u) &:=\int_\dUpsilon  \E^{\U(B_r), \QP_r^\eta}(u_{r}^\eta) \diff\QP(\eta) \fstop \notag
\end{align}
It is known that $(\bar{\E}^{\U, \QP}_r,\dom{\bar{\E}^{\U, \QP}_r})$ is a Dirichlet form on~$L^2(\mu)$ \cite[Prop.\ V.3.1.1]{BouHir91}.  The $L^2$-semigroup and the infinitesimal generator corresponding to $(\bar{\E}^{\U, \QP}_r,\dom{\bar{\E}^{\U, \QP}_r})$  are denoted by $\sem{\bar{T}_{r, t}^{\U,\QP}}$ and $(\bar{A}_r^{\U, \QP}, \dom{\bar{A}_r^{\U, \QP}})$ respectively. 
\end{defs}
%We denote the closures of the {\it localised form}~\eqref{eq:VariousFormsA} and \eqref{eq:Temptation} respectively by 
%$$\ttonde{\EE{\dUpsilon}{\QP}_{E_h},\dom{\EE{\dUpsilon}{\QP}_{E_h}}}, \quad \ttonde{\EE{\dUpsilon}{\QP},\dom{\EE{\dUpsilon}{\QP}}} \fstop$$
%The square field~$\cdc^{\dUpsilon}$ naturally extends to the domain $\dom{\EE{\dUpsilon}{\QP}}$, which is denoted by the same symbol~$\cdc^{\dUpsilon}$.
% is a Dirichlet form on~$L^2(\QP)$;and its closure denoted by 
%$$$$ 
%is a Dirichlet form on~$L^2(\QP)$. 
%The $L^2$-resolvent operators and the $L^2$-semigroups corresponding to the form~\eqref{eq:VariousFormsA} and the form~\eqref{eq:Temptation} are denoted respectively by 
%$$\bigl\{G_{E_h, \alpha}^{\dUpsilon, \QP}\bigr\}_{\alpha>0},\ \bigl\{T^{\dUpsilon, \QP}_{E_h, t}\bigr\}_{t >0} \quad \text{and} \quad \bigl\{G_{\alpha}^{\dUpsilon, \QP}\bigr\}_{\alpha>0}, \ \bigl\{T^{\dUpsilon, \QP}_t\bigr\}_{t >0} \fstop$$ 

%Let $\Delta^{\dUpsilon, \QP}$ denote the $L^2$-generator corresponding to $\ttonde{\EE{\dUpsilon}{\QP},\dom{\EE{\dUpsilon}{\QP}}}$. Let ${\rm Cap}_{\EE{\dUpsilon}{\QP}}$ denote the capacity associated with $\ttonde{\EE{\dUpsilon}{\QP},\dom{\EE{\dUpsilon}{\QP}}}$, see \cite[Def.\ 2.4 with $h=g=1$, or Exe. 2.10 in Chap.\ III]{MaRoe90}. 


%\begin{prop}[{\cite[Prop.\ 3.44]{LzDSSuz21}}]\label{p:ConditionalIntegration}
%Let~$(\mcX,\cdc)$ be a \TLDS,~$\QP$ be a probability measure on $\ttonde{\dUpsilon,\A_\mrmv(\msE)}$ satisfying~\ref{ass:Mmu}, and~$u\in L^1(\QP)$. Then, for any $E \in \msE$, 
%\begin{align*}
%\int_{\dUpsilon} u \diff\QP = \int_{\dUpsilon} \quadre{\int_{\dUpsilon(E)} u_{E,\eta} \diff \QP^\eta_E }\diff\QP(\eta) \fstop
%\end{align*}
%\end{prop}

%The next proposition is taken from \cite[(4.1), Prop.~4.1]{Osa96} or \cite[Proposition~3.45]{DS21}.
%\begin{prop}\label{prop: 0}
%For any $F \in \Cyl$, $\eta \in \U(B_r^c)$ and $r>0$, it holds
%\begin{align}
%\E_r(F) &= \int_{\U(B_r^c)} \EE_{\U(B_r)}(F_{\eta, r}) d\p_{B_r^c}(\eta) \, , \label{eqP: 2} 
%\end{align}%
%where $\E_{\U(B_r)}$ is the energy on $\U(B_r)$ defined in Definition~\ref{defn: WS}, and $F_{\eta, r}=F_{\eta, B_r^c}$ was introduced in Definition~\ref{def: CF}.
%\end{prop}
%Note that $(\EE_{\U(B_r)}, \CylF(\U(B_r)))$ is closable and the closure is denoted by $(\EE_{\U(B_r)}, H^{1,2}(\U(B_r), \p))$,  see Definition~\ref{defn: WS}. Recall that $\{G_{\alpha}^{\U(B_r)}\}_{\alpha}$ and $\{T^{\U(B_r)}_t\}$ denote the $L^2$-resolvent operator and the semigroup corresponding to $(\EE_{\U(B_r)}, H^{1,2}(\U(B_r), \p))$.

%\begin{proof}
%Let $u, v \in \Cyl{\Dz}$.  
%For simplicity of the notation, set $G_\alpha u(\cdot) := G_{\alpha}^{\dUpsilon(E), \QP^\cdot_E}u_{E, \cdot}(\cdot_{E})$. Then, by the definition \eqref{eq:ConditionalFunction},  
%\begin{align} \label{eq: 1-0}
%\bigl(G_\alpha u \bigr)_{E, \eta}(\cdot)= G_{\alpha}^{\dUpsilon(E), \QP^\eta_E}u_{E, \eta}(\cdot), \quad \QP^\eta_{E}\text{-a.e.\ on $\dUpsilon(E)$ for $\QP$-a.e.\ $\eta$} \fstop
%\end{align}
%Then, we have that\footnote{\purple{We need to assume $\mathcal D(\E_E^\U)=\{u \in L^2(\U, \mu): \int_{\U} \|u_\eta\|^2_{\mathcal D(\E^{\U(E), \mu_E^\eta})} d\mu<\infty\}$. This follows from the Markov uniqueness of $\mathcal D(\E_E^\U)$ since the r.h.s is a Dirichlet form by the superposition theorem in Bouleau--Hirsch p.213 }}
%%By combining \eqref{eq:p:MarginalWP:0}, \eqref{eq: 1-0} and the standard relation between the form and the resolvent operator, we obtain
%\begin{align} \label{eq: 1-1}
%	\EE{\dUpsilon}{\QP}_{E}(G_\alpha u, v) 
%	&=  \int_{\dUpsilon} \EE{\dUpsilon(E)}{\QP^\eta_{E}}((G_\alpha u)_{E, \eta}, v_{E, \eta}) \diff \QP(\eta) \notag
%	&
%	\\
%	&=  \int_{\dUpsilon} \EE{\dUpsilon(E)}{\QP^\eta_{E}}(G_{\alpha}^{\dUpsilon(E), \QP^\eta_E}u_{E, \eta}, v_{E, \eta}) \diff \QP(\eta) \notag
%	& 
%	\\
%	&=  \int_{\dUpsilon} \biggl( \int_{\dUpsilon(E)} \Bigl(u_{E, \eta} - \alpha  G_{\alpha}^{\dUpsilon(E), \QP^\eta_E}u_{E,\eta} \Bigr)  v_{E,\eta}    \diff\QP^\eta_{E}\biggr) \diff \QP(\eta) \notag
%%	&\because \ (\text{the general fact $\EE_\alpha(\alpha G_\alpha u, v) = \langle u, v\rangle_2$}) 
%	&\\
%	&=  \int_{\dUpsilon}  (u - \alpha G_\alpha  u) v   \diff\QP \, , \ 
%%	&\because \ (\text{Prop. \ref{prop: 0}}). 
%\end{align}
%where the first line follows from \eqref{eq:p:MarginalWP:0}, the second line follows from \eqref{eq: 1-0}, the third line follows from the standard relation between the form and the resolvent operator, and the fourth line follows from Prop.~\ref{p:ConditionalIntegration} and \eqref{eq: 1-0}. 
%%the fundamental equality $-\Delta G_\alpha^{\U(B_r)}F_{\eta,r} + \alpha G_\alpha^{\U(B_r)}F_{\eta,r} = F_{\eta,r}$. 
%Since the resolvent operator $G_{E, \alpha}^{\dUpsilon, \QP}$ is characterised as the unique $L^2$-bounded operator satisfying  \eqref{eq: 1-1} with $G_\alpha$ being replaced by $G_{E, \alpha}^{\dUpsilon, \QP}$, 
%%is the characterisation of 
%%$$\EE{\dUpsilon}{\QP}_{E}(G_\alpha u, v)  = \int_{\U(\R^n)}  (F - \alpha G^{r}_\alpha F) H \ d\p \, ,$$
%we conclude that $G_{E, \alpha}^{\dUpsilon, \QP}u(\gamma) = G_\alpha u(\gamma) = G_{\alpha}^{\dUpsilon(E), \QP^\gamma_E}u_{E, \gamma}(\gamma_{E})$ $\QP$-a.e.\ $\gamma$. The proof of \eqref{eq: R-1} is complete. 
%
%
%
%
%The second equality \eqref{eq: R-2} follows from the identity
%\begin{equation*}
%	\int_0^\infty e^{-\alpha t} T^{\dUpsilon, \QP}_{E, t}u(\gamma) \diff t
%	= G_{E, \alpha}^{\dUpsilon, \QP}u(\gamma) 
%	= G_{\alpha}^{\dUpsilon(E), \QP^\gamma_E}u_{E, \gamma}(\gamma_{E})
%	=\int_0^\infty e^{-\alpha t} T^{\dUpsilon(E), \QP^\gamma_E}_tu_{E, \gamma}(\gamma_{E}) \diff t \comma
%\end{equation*}
%and the injectivity of the Laplace transform on continuous functions in $t$.
%\end{proof}
The semigroup $\sem{\bar{T}_{r, t}^{\U, \QP}}$ corresponding to the superposition form $\bar{\E}_r^{\U, \QP}$ can be obtained as the superposition of  the semigroup $\sem{T^{\U(B_r), \QP_r^\eta}_{t}}$ associated with the form~$\E^{\U(B_r), \QP_r^\eta}$. For the following proposition, we refer the reader to %\cite[Prop.~3.5]{Suz22}, or the essentially same proof has been given in terms of direct integral in a more general setting by 
\cite[(iii) Prop.~2.13]{LzDS20}.
\begin{prop}[{\cite[(iii) Prop.~2.13]{LzDS20}}]\label{prop: 1-1}
	%Let $\alpha>0$, $t>0$, and $r>0$ be fixed. 
The following holds: 
	\begin{align} \label{eq: R-1}
%		G_{E, \alpha}^{\dUpsilon, \QP} u(\gamma) & = G_{\alpha}^{\dUpsilon(E), \QP^\gamma_{E}}u_{E, \gamma}(\gamma_{E}) \, ,
%		\\
		%\label{eq: R-2}
		\bar{T}_{r,t}^{\U, \QP}u(\gamma) & = T^{\U(B_r), \QP_r^\gamma}_{t} u_{r}^{\gamma}(\gamma_{B_r}) \, ,
	\end{align}
	for $\QP$-a.e.\ $\gamma\in \dUpsilon$, any $t>0$.
\end{prop}
\begin{rem}
The proof of \cite[(iii) Prop.~2.13]{LzDS20} has been given in terms of direct integral in a general setting. As the measure $\QP_r^\eta$ can be identified to the conditional probability~$\QP(\cdot\ |\ \cdot_{B_r^c}=\eta_{B_r^c})$ by a bi-measure-preserving isomorphism as remarked in~\eqref{r:BMP}, our setting is a particular case of direct integrals discussed in~\cite{LzDS20}.
\end{rem}
%\begin{proof}
%The proof ha
%\purple{See \cite[\S.3]{Suz22}. See also \cite{LzDS20}}.
%\end{proof}

We now discuss the relation between $\E_r^{\U, \QP}$ and $\bar{\E}_r^{\U, \QP}$. As the former form is constructed as the smallest closed extension of $(\E_r^{\U, \QP}, \mathcal C_r)$, it is clear by definition that 
$$\E_r^{\U, \QP}=\bar{\E}_r^{\U, \QP} \quad \text{on} \quad \mathcal C_r \comma \quad \dom{\E_r^{\U, \QP}} \subset \dom{\bar{\E}_r^{\U, \QP}} \fstop$$ 
The following theorem proves that the opposite inclusion holds as well. 
%In the following theorem, we prove that the core $\mathcal C_r$ is sufficiently large in the sense that the closure $(\E_r, \dom{\E_r})$ coincides with the superposed form $(\overline{\E}_r, \mathcal D(\overline{\E}_r))$. We further observe that $(\E_r, \mathcal C_r)$ is {\it Markov unique}, i.e., there exists at most one Markov extension on $(\E_r, \mathcal C_r)$. See the definition of Markov uniqueness e.g., in \cite[Def.\ 1.2]{Ebe99}.
\begin{thm} \label{t:S=M}
$(\E_r^{\U, \QP}, \dom{\E_r^{\U, \QP}}) = (\bar{\E}_r^{\U, \QP}, \dom{\bar{\E}_r^{\U, \QP}})$. 
%Furthermore, $(\E_r^{\U, \QP}, \mathcal C_r)$ is Markov unique. Namely, there is at most one Dirichlet form extending $(\E_r^{\U, \QP}, \mathcal C_r)$.} \footnote{The Markov uniqueness is inappropriate. We only need the density of the core in the maximal domain, which still works by Appeldix lemma}
%, which is $(\E_r^{\U, \QP}, \dom{\E_r^{\U, \QP}})$. 
\end{thm}
%Before providing the  proof of Thm.\ \ref{t:S=M}, we need a lemma, in which the operator~$\mathcal U_{\gamma, x}$ defined in~\eqref{d:UO} is localised on $B_r$.  
%\begin{lem} \label{l:LSO}
%For $u: \U(B_r) \to \R$, define  $\mathcal U_{\gamma, x}^r(u): B_r \to \R$ by 
%$$\mathcal U_{\gamma, x}^r(u)(y):=u(\1_{X \setminus \{x\}} \cdot \gamma + \delta_y) - u(\1_{X \setminus \{x\}} \cdot \gamma) \quad \gamma \in \U(B_r), \ x \in \gamma \fstop$$
%Then, for any $u:  \U \to \R$, 
%$$\mathcal U_{\gamma_{B_r}, x}^r(u_{r}^\gamma)(y) = \mathcal U_{\gamma, x}(u)(y) \quad \text{for every $\gamma \in \U$, $x \in \gamma_{B_r}$ and $y \in B_r$} \fstop$$
%Furthermore, the operation $\mathcal U^r_{\gamma, x}$ maps from $\Lip(\U(B_r), \mssd_\U)$ to $\Lip(B_r)$ and Lipschitz constants are contracted by $\mathcal U^r_{\gamma, x}$ for any $r>0$:
% $$\Lip(\mathcal U_{\gamma, x}(u)) \le \Lip_{\mssd_\U}(u) \quad \forall \gamma \in \U(B_r) \quad \forall x \in \gamma \fstop$$
%\end{lem}
%\begin{proof}
%By the straightforward argument by definition, for every $x \in \gamma_{B_r}$ and $y \in B_r$, 
%\begin{align*}
% \mathcal U_{\gamma, x}(u)(y)& = u(\1_{X \setminus \{x\}} \cdot \gamma + \delta_y)- u(\1_{X \setminus \{x\}} \cdot \gamma)
% \\
% &  = u(\1_{X \setminus \{x\}} \cdot \gamma_{B_r} + \gamma_{B_r^c}+ \delta_y)- u(\1_{X \setminus \{x\}} \cdot \gamma_{B_r} + \gamma_{B_r^c})
% \\
% &= u_{r, \gamma}(\1_{X \setminus \{x\}} \cdot \gamma_{B_r}+ \delta_y)- u_{r, \gamma}(\1_{X \setminus \{x\}} \cdot \gamma_{B_r})
% \\
% &=\mathcal U^r_{\gamma_{B_r}, x}(u_{r, \gamma})(y) \fstop
%\end{align*}
%The proof of the latter statement is identical to that of Lem.~\ref{p:ULP}. The proof is complete.
%\end{proof}

%\grey{ The following proof is the same as in \eqref{e:ULIP}, so we might cut it
%\begin{lem}\label{l:DT}
%If $u \in \Lip(\U(B_r), \mssd_{\U})$, then $\mathcal U^r_{\gamma, x}(u) \in \Lip(B_r)$  and 
%\begin{align} \label{e:DT-1}
%\Lip_{\R}(\mathcal U^r_{\gamma, x}(u)) \le \Lip_{\mssd_{\U}}(u)
%\end{align}
%%the Lipschitz constant of $\mathcal U_{\gamma, x}(u)$ is bounded by $\Lip_{\mssd_{\U}}(\mathcal U_{\gamma, x}(u)$ 
%for every $\gamma$ and every $x \in B_r$.  In particular, $\mathcal U^r_{\gamma, x}(u) \in W_{loc}^{1,2}(\mssm_r)$.
%\end{lem}
%\begin{proof}
%It can be checked $\mathcal U_{\gamma, x}(u) \in \Lip(B_r)$ whenever $u \in \Lip(\U(B_r), \mssd_\U)$ as follows:
%\begin{align*}
%|\mathcal U_{\gamma, x}(u)(y)- \mathcal U_{\gamma, x}(u)(z)| &= |u(\car_{X\setminus\set{x}}\cdot\gamma + \delta_y)-u(\car_{X\setminus\set{x}}\cdot\gamma + \delta_z)|
%\\
%& \le \Lip_{\mssd_{\U}}(u)\mssd_\U(\car_{X\setminus\set{x}}\cdot\gamma + \delta_y, \car_{X\setminus\set{x}}\cdot\gamma + \delta_z)
%\\
%&= \Lip_{\mssd_{\U}}(u) |y-z| \fstop
%\end{align*}
%As $\Lip(B_r) \subset W^{1,2}_{loc}(\mssm_r)$, the proof is completed.
%%By the definition of $\mathcal D(\E_r^\eta)$, we have that $u|_{\U^k(B_r)} \in \mathcal D(\E_r^{k, \eta})$ for any $k \in \N_0$.  Noting that the density $\Psi_r^{k, \eta}=0$ only at the diagonal set on $\U^k(B_r)$, we see that  $\mathcal D_{loc}(\E_r^{k, \eta}) = W^{1,2}_{loc}(\mssm^{\odot k})$ for any $k \in \N$. In particular, $u|_{\U^k(B_r)} \in \mathcal D_{loc}(\E_r^{k, \eta}) = W^{1,2}_{loc}(\mssm^{\odot k})$. Thus,  we obtain $\mathcal U_{\gamma, x}(u|_{\U^k(B_r)} ) \in W_{loc}^{1,2}(\mssm_r)$ for any $k \in \N$. \purple{make it more rigorous.}
%\end{proof}
%}
%\grey{
%\begin{lem} \label{l:RPL}
%Fix $\eta \in \U$ and $r>0$ and let $\Omega \subset \U(B_r)$ so that $\mu_{r, \eta}(\Omega)=1$. Define  
%$$\Omega_+:=\{\1_{\R \setminus \{x\}}\cdot\gamma+\delta_y: \gamma \in \Omega,\ y \in B_r,\ x \in \gamma\} \fstop$$
%Then $\mu_{r, \eta}(\Omega_+)=1$. In this case, furthermore, there exists $\Omega' \subset \U(B_r)$ and $A \subset B_r$ so that $\mu_{r, \eta}(\Omega')=1$, $\mssm_r(A^c)=0$ and 
%$$\{\1_{\R \setminus \{x\}}\cdot\gamma+\delta_y: \gamma \in \Omega',\ x \in \gamma\} \subset \Omega\comma \quad \forall y \in A\fstop $$ 
%\end{lem}
%\begin{proof}
%\purple{Make the following rigorous!} 
%%\purple{As $\mu^k_{r, \eta}$ is equivalent to $\mssm_r^{\odot k}$, it suffices to show the corresponding statement for $\mssm_r^{\odot k}$.} \purple{Explain the rigidity!}
%%\purple{It suffices to show the case of $\mssm_r^{\odot k}$ due to the equivalence of $\mu_{r, \eta}$ and $\mssm_r^{\odot k(\eta)}$.}
%According to the expression \eqref{d:CP}, it suffices to show that $\mu_{r, \eta}^k(\Omega_+)=\mu_{r, \eta}^k(\U^k(B_r))$ for every $k \in \N_0$.
%Due to \purple{the number rigidity}, the conditioning $\eta$ selects a unique $k(\eta)$ so that $\mu_{r, \eta}^{k(\eta)}(\U^k(B_r))=1$. We fix such $k=k(\eta)$ in the following argument. Let $\Omega_{+, y}:=\{\1_{\R \setminus \{x\}}\cdot\gamma+\delta_y: \gamma \in \Omega,,\ x \in \gamma\}$, and define the sections $\Omega_+^y:=\{\gamma \in \Omega_+: y \in \gamma\}$ and $\Omega^y:=\{\gamma \in \Omega: y \in \gamma\}$.
%By definition, we observe that $\Omega^y \subset \Omega_{+, y} \subset \Omega_+^y$. A simple disintegration argument leads to 
%\begin{align} \label{e:RPL-1}
%\mssm^{\odot k}_r(\U^k(B_r))= \mssm_r^{\odot k}(\Omega) &=\frac{1}{k} \int_{B_r} \mssm_r^{\odot k-1}(\Omega^y) \diff \mssm_r(y) 
%\\
%&\le  \frac{1}{k}\int_{B_r} \mssm_r^{\odot k-1}(\Omega_{+, y})  \diff \mssm_r(y)\notag
%\\
%&\le \frac{1}{k}\int_{B_r} \mssm_r^{\odot k-1}(\Omega_{+}^y) \diff \mssm_r(y) \notag
%\\
%&= \mssm_r^{\odot k}(\Omega_+)\fstop \notag
%\end{align}
%\purple{reference: By noting that the measure $\mu_{r, \eta}^k$ is equivalent to $\mssm_r^{\odot k}$}, the proof of the first assertion is complete. 
%By \eqref{e:RPL-1}, there exists $A \subset B_r$ with $\mssm_r(A^c)=0$ so that for every $y \in A$ it holds that $\mssm_r^{\odot k-1}(\Omega^y) =\mssm_r^{\odot k-1}(\U^{k-1}(B_r))$. 
%%Let $\Omega_y':=\Omega^y \cap \Omega_{+, y}$. Then, $\mssm_r^{\odot k-1}(\Omega_y')=\mssm^{\odot k-1}_r(\U^{k-1}(B_r))$ for every $y \in A$.
%Let $\Omega'\subset \U^k(B_r)$ be a $\mssm_r^{\odot k}$-measurable set whose section at $y \in A$ is $\Omega^y$, i.e., $\Omega'=\cup_{y \in A} \Omega_y' \times \{y\}$ \purple{(Justify it!)}. Then, 
%$$\mssm_r^{\odot k}(\Omega')=\frac{1}{k}\int_{\R} \mssm_r^{\odot k-1}(\Omega_y') \mssm(\diff y) = \frac{1}{k}\int_{A} \mssm_r^{\odot k-1}(\Omega_y') \mssm(\diff y) = \mssm_r^{\odot k}(\U^k(B_r)) \fstop$$
%The pair $(A, \Omega')$, therefore,  possesses the claimed properties in the statement, which completes the proof. 
%%Note that the replacement of particles by $\delta_y$ is the projection onto the line horizontal via $\{y\}$ (imagine the case of the interval times interval for two particle case). Therefore, the projection contains the section at $y$. As the section at $y$ satisfies the desired property, we can conclude the projection as well satisfies the conclusion. 
%\end{proof}
%}
%\grey{
%\begin{lem} \label{l:RPL}
%Fix $\eta \in \U$ and $r>0$ and let $\Omega \subset \U(B_r)$ so that $\mu_{r, \eta}(\Omega)=1$. Define  
%$$\Omega_+:=\{\1_{\R \setminus \{x\}}\cdot\gamma+\delta_y: \gamma \in \Omega,\ y \in B_r,\ x \in \gamma\} \fstop$$
%Then $\mu_{r, \eta}(\Omega_+)=1$. In this case, furthermore, there exists $\Omega' \subset \U(B_r)$ and $A \subset B_r$ so that $\mu_{r, \eta}(\Omega')=1$, $\mssm_r(A^c)=0$ and 
%$$\{\1_{\R \setminus \{x\}}\cdot\gamma+\delta_y: \gamma \in \Omega',\ x \in \gamma\} \subset \Omega\comma \quad \forall y \in A\fstop $$ 
%\end{lem}
%\begin{proof}
%\purple{Make the following rigorous!} 
%%\purple{As $\mu^k_{r, \eta}$ is equivalent to $\mssm_r^{\odot k}$, it suffices to show the corresponding statement for $\mssm_r^{\odot k}$.} \purple{Explain the rigidity!}
%%\purple{It suffices to show the case of $\mssm_r^{\odot k}$ due to the equivalence of $\mu_{r, \eta}$ and $\mssm_r^{\odot k(\eta)}$.}
%%According to the expression \eqref{d:CP}, it suffices to show that $\mu_{r, \eta}^k(\Omega_+)=\mu_{r, \eta}^k(\U^k(B_r))$ for every $k \in \N_0$.
%Let $\Omega_{+, y}:=\{\1_{\R \setminus \{x\}}\cdot\gamma+\delta_y: \gamma \in \Omega,,\ x \in \gamma\}$, and define the sections $\Omega_+^y:=\{\gamma \in \Omega_+: y \in \gamma\}$ and $\Omega^y:=\{\gamma \in \Omega: y \in \gamma\}$. \purple{Make the rigorous construction of the following measure: We denote by $\mu_{r, \eta}^y$ and $\mu_{r, \eta}^{+}$ the disintegration measure and the disintegrated measure at section $y \in B_r$, i.e., for any measurable set $A  \subset \U(B_r)$ and its section $A^y$ at $y \in B_r$, 
%$$\mu_{r, \eta}(A)=\int_{B_r} \mu_{r, \eta}^y(A^y) \diff\mu_{r, \eta}^+(y) \fstop$$
%}
%By definition, we observe that $\Omega^y \subset \Omega_{+, y} \subset \Omega_+^y$. A simple disintegration argument leads to 
%\begin{align} \label{e:RPL-2}
%1= \mu_{r, \eta}(\Omega) &= \int_{B_r} \mu_{r, \eta}^y(\Omega^y) \diff \mu_{r, \eta}^+(y) 
%\\
%&\le  \int_{B_r} \mu_{r, \eta}^y(\Omega_{+, y})  \diff \mu_{r, \eta}^+(y)\notag
%\\
%&\le \int_{B_r} \mu_{r, \eta}^y(\Omega_{+}^y) \diff \mu_{r, \eta}^+(y) \notag
%\\
%&= \mu_{r, \eta}(\Omega_+)\fstop \notag
%\end{align}
%%\begin{align} \label{e:RPL-1}
%%\mssm^{\odot k}_r(\U^k(B_r))= \mssm_r^{\odot k}(\Omega) &= \int_{\R} \mssm_r^{\odot k-1}(\Omega^y) \diff \mssm_r(y) 
%%\\
%%&\le  \int_{\R} \mssm_r^{\odot k-1}(\Omega_{+, y})  \diff \mssm_r(y)\notag
%%\\
%%&\le \int_{\R} \mssm_r^{\odot k-1}(\Omega_{+}^y) \diff \mssm_r(y) \notag
%%\\
%%&= \mssm_r^{\odot k}(\Omega_+)\fstop \notag
%%\end{align}
%The proof of the first assertion is complete. 
%By \eqref{e:RPL-2}, there exists $A \subset B_r$ with $\mssm_r(A^c)=0$ so that for every $y \in A$ it holds that $\mssm_r^{\odot k-1}(\Omega^y)= \mssm_r^{\odot k-1}(\Omega_{+, y}) =\mssm_r^{\odot k-1}(\U^{k-1}(B_r))$. Let $\Omega_y':=\Omega^y \cap \Omega_{+, y}$. Then, $\mssm_r(\Omega_y')=\mssm_r(B_r)$.
%Let $\Omega'\subset \U^k(B_r)$ be a $\mssm_r^{\odot k}$-measurable set whose section at $y \in A$ is $\Omega_y'$, i.e., $\Omega'=\cup_{y \in A} \Omega_y'$ \purple{(Justify it!)}. Then, 
%$$\mssm_r^{\odot k}(\Omega')=\int_{\R} \mssm_r^{\odot k-1}(\Omega_y') \mssm(\diff y) = \int_{A} \mssm_r^{\odot k-1}(\Omega_y') \mssm(\diff y) = \mssm_r^{\odot k}(\U^k(B_r)) \fstop$$
%The pair $(A, \Omega')$ possesses the desired properties, which completes the proof. 
%%Note that the replacement of particles by $\delta_y$ is the projection onto the line horizontal via $\{y\}$ (imagine the case of the interval times interval for two particle case). Therefore, the projection contains the section at $y$. As the section at $y$ satisfies the desired property, we can conclude the projection as well satisfies the conclusion. 
%\end{proof}
%}
%
%\newpage
%\textcolor{gray}{ The following is perhaps not needed
%\begin{lem}
%Fix $\eta \in \U$ and $r>0$ and define  
%$$\Omega_+^x:=\{\1_{\R \setminus \{x\}}\cdot\gamma+\delta_y: \gamma \in \Omega,\ y \in \R\} \fstop$$
%If $\mu_{r, \eta}(\Omega)=1$, then $\mu_{r, \eta}(\Omega^x_+)=1$ for every $x \in B_r$. 
%\end{lem}
%\begin{proof}
%As $\mu_{r, \eta}^k \cong \mssm_r^{\odot k}$ as a measure on $\U^k(B_r)$, it suffices to show that 
%$$\mssm_r^{\odot k}(\U^k(B_r) \setminus (\U^k(B_r) \cap \Omega_+^x))=0 \quad \forall k \in \N \fstop$$
%By assumption  $\mu_{r, \eta}(\Omega)=1$ and the equivalence $\mu_{r, \eta}^k \cong \mssm_r^{\odot k}$, we have that 
%$$\mssm_r^{\odot k}(\U^k(B_r) \setminus (\U^k(B_r) \cap \Omega))=0 \quad \forall k \in \N \fstop$$
%We have that 
%$$\U^{k+1}(B_r) \cap \Omega_+^x = \Omega_+^x(k) \subset $$
%\begin{align*}
%\mssm_r^{\odot k+1}(\U^{k+1}(B_r) \cap \Omega_+^x)) 
%&= \mssm_r^{\odot k+1}(\{ \1_{\R \setminus \{x\}}\cdot\gamma+\delta_y: \gamma \in \Omega^k,\ y \in \R\}) 
%\\
%&=\int_{B_r} \mssm_{r, y}^{\odot k}(\{ \1_{\R \setminus \{x\}}\cdot\gamma+\delta_y: \gamma \in \Omega^k\}) \mssm_r(dy)
%\\
%&=\int_{B_r} \mssm_{r}^{\odot k}( \Omega^k) \mssm_r(dy)
%\\
%&=\mssm_r^{\odot k+1}(\U^{k+1}(B_r)) \fstop
%\end{align*}
%\end{proof}
%\begin{lem}
%Define  
%$$\Omega_+^x:=\{\1_{\R \setminus \{x\}}\cdot\gamma+\delta_y: \gamma \in \Omega,\ y \in \R\} \fstop$$
%If $\mu(\Omega)=1$, then $\mu(\Omega^x_+)=1$ for every $x \in \R$. 
%\end{lem}
%\begin{proof}
%Take $r>0$. 
%\begin{align*}
%\mu(\Omega_+^x)&=\int_{\U} \mu_{r, \eta}((\Omega_+^x)_{r, \eta}) \diff \mu(\eta) 
%\\
%&= \int_{\U} \mu_{r, \eta}((\{ \1_{\R \setminus \{x\}}\cdot\gamma+\delta_y: \gamma \in \Omega,\ y \in \R \})_{r, \eta}) \diff \mu(\eta)
%\\
%&= \int_{\U} \mu_{r, \eta}((\{ \1_{\R \setminus \{x\}}\cdot\gamma+\delta_y: \gamma \in \Omega,\ y \in B_r \})_{r, \eta}) 
%\\
%&\qquad+ \mu_{r, \eta}((\{ \1_{\R \setminus \{x\}}\cdot\gamma+\delta_y: \gamma \in \Omega,\ y \in B_r^c \})_{r, \eta}) \diff \mu(\eta)
%\\
%&= \int_{\U} \mu_{r, \eta}((\{ \1_{\R \setminus \{x\}}\cdot\gamma+\delta_y: \gamma \in \Omega_{r, \eta},\ y \in B_r \})_{r, \eta}) 
%\\
%&\qquad+ \mu_{r, \eta}(\Omega_{r, \eta}) \diff \mu(\eta)
%\end{align*}
%\end{proof}
%}
\begin{proof}
The inclusion $\dom{\E^{\U, \QP}_r}\subset\dom{\bar{\E}^{\U, \QP}_r}$ with the inequality $\bar{\E}^{\U, \QP}_r \le \E^{\U, \QP}_r$ is straightforward by definition. 
Noting $\bar{\E}^{\U, \QP}_r = \E^{\U, \QP}_r$ on $\mathcal C_r$ and $\dom{\E^{\U, \QP}_r}$ is the closure of~$\mathcal C_r$, it suffices to show that $\mathcal C_r\subset \dom{\bar{\E}^{\U, \QP}_r}$ is dense. 
%\smallskip
Thanks to Lem.~\ref{l:MU}, we only need to show that $\bar{T}_{r, t}^{\U, \QP}\mathcal C_r \subset \mathcal C_r$. 

As $\bar{T}_{r, t}^{\U, \QP}$ is an $L^\infty$-contraction semigroup by the sub-Markovian property of the semigroup (see, e.g.,~\cite[Def.~I.4.1]{MaRoe90}), we obtain $\bar{T}_{r, t}^{\U, \QP} \mathcal C_r \subset L^\infty(\mu)$, which verifies (a) in Def.~\ref{d:core}
%In the following, we verify (b) and (c) in~Def.~\ref{d:core} for~$\bar{T}_{r, t}^{\U, \QP}u$ with~$u \in \mathcal C_r$. 
%We now show that the Lipschitz property (b) in Def.~\ref{d:core} is preserved under the action by~$\bar{T}_{r, t}^{\U, \QP}$.

\paragraph{Verification of (b) in Def.~\ref{d:core}}Let $u \in \mathcal C_r$ and we show that $\bar{T}_{r, t}^{\U, \QP}u$ satisfies   (b) in Def.~\ref{d:core}. 
% We first show that  for $\mu$-a.e.\ $\gamma$ and $x \in \gamma_{B_r}$, 
%\begin{align*}
%\mathcal U_{\gamma, x}(\bar{T}_{r, t}^{\U, \QP}u) \in \Lip_b(B_r)\fstop
%\end{align*}
By Prop.~\ref{prop: 1-1}, we can identify the following two operators:
%and $\Sigma_{r, \eta} \subset \U(B_r)$ for $\eta \in \Omega$ with $\mu_{r, \eta}(\Sigma)=1$ so that  
\begin{align*} 
\bar{T}_{r, t}^{\U, \QP} u=T_{t}^{\U(B_r), \QP_r^\cdot}u_{r}^{\cdot}(\cdot_{B_r})  \fstop
\end{align*}
This implies that 
\begin{align*} 
\Bigl( \bar{T}_{r, t}^{\U, \QP} u \Bigr)_r^\eta(\cdot)=\bar{T}_{r, t}^{\U, \QP} u (\cdot + \eta_{B_r^c})=T_{t}^{\U(B_r), \QP_r^\eta}u_{r}^{\eta}(\cdot)   \fstop
\end{align*}
%where $\bullet$ is the placeholder for a variable in $\U$. 
%By Lem.~\ref{l:LSO}, for any $\gamma \in \U $ and $x \in \gamma_{B_r}$
%\begin{align*} %\label{e:St1}
%\mathcal U_{\gamma, x}(T_{t}^{\U(B_r), \QP_r^\cdot}u_{r}^{\cdot}(\cdot_{B_r}))=\mathcal U^r_{\gamma_{B_r}, x}\Bigl(\Bigl(T_{t}^{\U(B_r), \QP_r^\gamma}u_{r}^{\gamma}\Bigr)_{r}^{\gamma}(\cdot_{B_r})\Bigr) \quad \text{on} \quad B_r \fstop
%\end{align*}
%Therefore, it suffices to show that for any $\gamma$ and $x \in \gamma_{B_r}$,  
%\begin{align*} %\label{e:St1}
%\mathcal U^r_{\gamma_{B_r}, x}\Bigl(\Bigl(T_{t}^{\U(B_r), \QP_r^\gamma}u_{r}^{\gamma}\Bigr)_{r}^{\gamma}(\cdot_{B_r})\Bigr) \in \Lip_b(B_r)\fstop
%\end{align*}
%Take $k=k(\gamma)$ as in \eqref{e:R1}. As the conditional probability $\QP_{r}^{\gamma}$ is supported only on $\U^k(B_r)$, we only need to show 
%\begin{align} \label{e:St1}
%\mathcal U^r_{\gamma_{B_r}, x}\Bigl(\Bigl(T_{t}^{\U(B_r), \QP_r^{k, \gamma}}u_{r}^{\gamma}\Bigr)_{r}^{\gamma}(\cdot_{B_r})\Bigr) \in \Lip_b(B_r)\fstop
%\end{align}
%
%$$T_{t}^{\U(B_r), \QP_r^{k\eta}}=T_{t}^{\U(B_r), \QP_r^\eta}|_{\U^k(B_r)}\comma$$
Take $k=k(\eta)$ as in \eqref{e:R1}. As the conditional probability $\QP_{r}^{\eta}$ is supported only on $\U^k(B_r)$, we only need to show 
\begin{align}\label{e:St1}
T_{t}^{\U(B_r), \QP_r^{k, \eta}}u_{r}^{\eta} \in \Lip_b(\U^k(B_r), \mssd_\U) \fstop
\end{align}
As $(\U^k(B_r), \mssd_\U, \mu_r^{k, \gamma})$ is $\RCD(0,\infty)$ for $k=k(\eta)$ for $\QP$-a.e.~$\eta$ by~Prop.~\ref{p:BE2}, the corresponding semigroup satisfies $L^\infty(\QP_r^{k, \eta}$)-to-$\Lip_b(\U^k(B_r), \mssd_\U)$-regularisation property (\cite[Thm.\ 6.5]{AmbGigSav14}), which shows that for $\QP$-a.e.~$\eta$
\begin{align*} %\label{e:LIT}
T_{t}^{\U(B_r), \QP_r^{k, \eta}}v \in \Lip_b(\U^k(B_r), \mssd_\U) \quad \forall v \in L^\infty(\QP_r^{k, \eta}) \comma
\end{align*}
and its Lipchitz constant is bounded as 
$$\Lip_{\mssd_\U}(T_{t}^{\U(B_r), \QP_r^{k, \eta}}v) \le c(t, K)\|v\|_{L^\infty(\QP_r^{k, \eta})}  \comma$$
 with constant $c(t, K)$ depending only on $t$ and the curvature bound $K=0$ (to be more precise, $c(t, 0)=\frac{1}{\sqrt{2t}}$). This proves \eqref{e:St1}, which completes the verification of~\ref{i:d:core2}. 
%As the operation~$\mathcal U^r_{\gamma_{B_r}, x}$ contracts Lipchitz constants by the latter statement of Lem.~\ref{l:LSO}, we conclude that for $\QP$-a.e.~$\gamma$
%\begin{align*} %\label{e:St1}
%\mathcal U^r_{\gamma_{B_r}, x}\Bigl(\Bigl(T_{t}^{\U(B_r), \QP_r^{k, \gamma}}u_{r}^{\gamma}\Bigr)_{r}^{\gamma}(\cdot_{B_r})\Bigr)\in \Lip(B_r) \comma
%\end{align*}
%which concludes \eqref{e:St1}.
% \begin{align*} %\label{e:ULIP}
%&|\mathcal U_{\gamma, x}(T_{t}^{\U(B_r), \QP_r^{k, \cdot}}u_{r}^{\cdot}(\cdot_{B_r}))(y)- \mathcal U_{\gamma, x}(T_{t}^{\U(B_r), \QP_r^{k, \cdot}}u_{r}^{\cdot}(\cdot_{B_r}))(y)(z)| 
%\\
%&= |u(\car_{X\setminus\set{x}}\cdot\gamma + \delta_y)-u(\car_{X\setminus\set{x}}\cdot\gamma + \delta_z)|
%\\
%& \le \Lip_{\mssd_{\U}}(u)\mssd_\U(\car_{X\setminus\set{x}}\cdot\gamma + \delta_y, \car_{X\setminus\set{x}}\cdot\gamma + \delta_z) \notag
%\\
%&= \Lip_{\mssd_{\U}}(u) |y-z| \fstop \notag
%\end{align*}
%
%\begin{align} \label{e:St1}
%\Lip\Bigl(\mathcal U_{\gamma, x}(T_{t}^{\U(B_r), \QP_r^{k, \eta}}u_{r}^{\eta}(\cdot_{B_r})) \Bigr)\in W_{loc}^{1,2}(\mssm_r)\fstop
%\end{align}
%Thu,s 
%In view of the Rademacher theorem on $B_r$, every Lipschitz function on $B_r$ belongs to $W_{loc}^{1,2}(\mssm_r)$, therefore, it suffices to show that $\mathcal U_{\gamma, x}(\bar{T}_{r, t}^{\U, \QP}u)$ is Lipschitz on~$B_r$. Thanks to Lem.~\ref{p:ULP}, we only need to show that $\bar{T}_{r, t}^{\U, \QP}u$ is $\mssd_\U$-Lipschitz on $\U$. 
%By Prop.~\ref{prop: 1-1}, the following holds: 
%%and $\Sigma_{r, \eta} \subset \U(B_r)$ for $\eta \in \Omega$ with $\mu_{r, \eta}(\Sigma)=1$ so that  
%\begin{align} \label{e:SLE}
%(\bar{T}_{r, t}^{\U, \QP} u)_{r}^{\gamma}(\gamma_{B_r})=T_{t}^{\U(B_r), \QP_r^\gamma}u_{r}^{\gamma}(\gamma_{B_r}) \quad \text{$\mu$-a.e.\ $\gamma$} \fstop
%\end{align} 
%%By Lem.\ \ref{l:LSO}, the following equality holds true for every $\gamma \in \U$ with $\gamma(B_r)=l$, every $x \in \gamma_{B_r}$ and every $y \in B_r$ 
%%\begin{align} \label{e:UE}
%% \mathcal U_{\gamma, x}(\overline{T}_t^{r} u)(y) =\mathcal U^r_{\gamma_{B_r}, x}(((\overline{T}_t^{r} u)_{r, \gamma})|_{\U^l(B_r)})(y) \fstop
%% \end{align}
%%Therefore, it suffices to show $\mathcal U^r_{\gamma_{B_r}, x}(((\overline{T}_t^{r} u)_{r, \gamma})|_{\U^l(B_r)}) \in W_{loc}^{1,2}(\mssm_r)$ for every $l \in \N_0$. 
%
%%As $\mu(\Omega)=1$, by a simple disintegration argument, we can take $\tilde{\Omega} \subset \Omega$ with $\mu(\tilde{\Omega})=1$ and $\mu_{r, \eta}(\Omega_{r, \eta})=1$ for {\it every} $\eta \in \tilde{\Omega}$. For the notational simplicity, we do not relabel this set and use the symbol $\Omega$ in place of such $\tilde{\Omega}$. %We fix $\eta \in \Omega$ in the following argument. 
%%%\purple{(otherwise, replace $\Omega$ with $\Omega' \subset \Omega$ satisfying the same property with $\mu(\Omega')=1$). }
%%%for every $\eta \in \Omega$ and $\gamma \in \Sigma_{r, \eta}$. 
%%By applying Lem.\ \ref{l:RPL} in such a way that we take $\Omega_{r, \eta}$ to be $\Omega$\footnote{Use a different symbol, otherwise it is confusing.} in Lem.\ \ref{l:RPL}, there exists $\Omega' \subset \U(B_r)$ with $\mu_{r, \eta}(\Omega')=1$ and $A \subset B_r$ with $\mssm_r(A^c)=0$ so that for every $\gamma \in \Omega'$, $x \in \gamma$ and $y \in A$, 
%%$$\{\1_{\R \setminus \{x\}}\cdot\gamma+\delta_y: \gamma \in \Omega',\ x \in \gamma\} \subset \Omega_{r, \eta} \comma \quad \forall y \in A\comma$$
%%which implies that for every $\gamma \in \Omega'$, $x \in \gamma_{B_r}$, $y \in A$ and $\eta \in \Omega$, the following holds: 
%%\begin{align*}% \label{eq:S=T-2}
%%\mathcal U^r_{\gamma, x}((\overline{T}_t^{r} u)_{r, \eta})(y)=\mathcal U^r_{\gamma, x}(T_{t}^{r, \eta}u_{r, \eta}(\cdot_{B_r}))(y)\fstop
%%\end{align*}
%%In particular, we have that 
%%for every $\gamma \in \Omega'$, $x \in \gamma_{B_r}$ and $\eta \in \Omega$,
%%\begin{align} \label{eq:S=T-2}
%%\mathcal U^r_{\gamma, x}((\overline{T}_t^{r} u)_{r, \eta})=\mathcal U^r_{\gamma, x}(T_{t}^{r, \eta}u_{r, \eta}(\cdot_{B_r})) \quad \text{$\mssm_r$-a.e.}\fstop
%%\end{align}
%%We next discuss the regularity of $T_t^{r, \eta} v$ for $v \in  \mathcal D(\E_r^\eta)$. 
%For $\mu$-a.e.\ $\eta$, it holds that $T_{t}^{\QP_r^\eta} v \in \dom{\E^{\U(B_r), \QP_r^\eta}}$ for every $v \in \dom{\E^{\U(B_r), \QP_r^\eta}}$ by a general property of Dirichlet form (see, e.g., \cite[p.23, Lem.\ 1.3.3]{FukOshTak11}). 
%
%In the following, we show $\mathcal U_{\gamma, x}((\overline{T}_t^ru)|_{\U^l(B_r)}) \in W_{loc}^{1,2}(\mssm_r)$ separately in the case of $l = k(\eta)$ where $k(\eta)$ was defined in \eqref{e:R1},  and the other cases. We start from the case of $l=k(\eta)$. As $(\U^k(B_r), \mssd_\U, \mu_r^{k, \eta})$ is $\RCD(0,\infty)$ for $k=k(\eta)$, the corresponding semigroup $T_{t}^{r, \eta}|_{\U^k(B_r)}$ restricted on $\U^k(B_r)$ satisfies $L^\infty(\nu_r^{k, \eta})$-to-$\Lip(\U^k(B_r), \mssd_\U)$ property (\cite[Thm.\ 6.5]{AmbGigSav14}), which implies  
%\begin{align} \label{e:LIT}
%T_{t}^{r, \eta}|_{\U^k(B_r)}v \in \Lip(\U^k(B_r), \mssd_\U)
%\end{align} for any bounded function $v$ on $\U^k(B_r)$, and its Lipchitz constant $\Lip_{\mssd_\U}(T_{t}^{r, \eta}|_{\U^k(B_r)}v)$ is bounded by $c(t)\|T_{t}^{r, \eta}|_{\U^k(B_r)}v\|_\infty \le c(t)\|v\|_\infty$ with constant $c(t)$ depending only on $t$.  Thus, by \eqref{e:SLE} and \eqref{e:LIT}, we obtain that, for $\mu$-a.e.\ $\gamma$ (\purple{non-rigorous: the following statement is only up to $\mu_{r, \gamma}$-negligible set})
%\begin{align} \label{e:TLIP}
%(\overline{T}_t^{r} u)_{r, \gamma}|_{\U^k(B_r)} \in  \Lip(\U^k(B_r), \mssd_\U)\fstop
%\end{align} 
%%By Lem.\ \ref{l:LSO}, the following equality holds true for every $\gamma \in \U$ with $\gamma(B_r)=k$, every $x \in \gamma_{B_r}$ and every $y \in B_r$ 
%%\begin{align} \label{e:UE}
%% \mathcal U_{\gamma, x}(\overline{T}_t^{r} u)(y) =\mathcal U^r_{\gamma_{B_r}, x}(((\overline{T}_t^{r} u)_{r, \gamma})|_{\U^k(B_r)})(y) \fstop
%% \end{align}
%By \eqref{e:UE}, \eqref{e:TLIP} and Lem.\ \ref{l:DT} combined with the fact that $\Lip(\R) \subset W_{loc}^{1,2}(\mssm_r)$, we conclude $\mathcal U_{\gamma, x}((\overline{T}_t^{r} u)|_{\U^k(B_r)}) \in W_{loc}^{1,2}(\mssm_r)$ for $\mu$-a.e.\ $\gamma$ and every $x\in \gamma_{B_r}$. 
%We next discuss the case $l \neq k(\eta)$. \purple{In this case, $L^2(\mu_{r, \eta}^l)=\{0\}$ and every function space is trivialised. So we have nothing to prove} 
%By the definition \eqref{e:R1} of $k(\eta)$, if $l\neq k(\eta)$, the density $\Psi_{r, \eta}^l \equiv 0$, which, therefore implies that the corresponding semigroup is the trivial operator, i.e., $T_t^{r, \eta}|_{\U^l(B_r)}v=v$ for $v \in L^2(\mu_{r, \eta}^l)$. This implies
%\begin{align} \label{e:KNL}
%(\overline{T}_t^{r} u)_{r, \gamma}|_{\U^l(B_r)} = (u_{r, \gamma})|_{\U^l(B_r)}  \quad \text{whenever $l\neq k(\eta)$} \fstop
%\end{align}
%By the assumption $\mathcal U_{\gamma, x}(u) \in W_{loc}^{1,2}(\mssm_r)$,  for every $\gamma \in \U$ with $\gamma(B_r)=l$ and every $x \in \gamma_{B_r}$
%\begin{align} \label{e:UE-2}
% \mathcal U_{\gamma, x}((\overline{T}_t^{r} u)|_{\U^l(B_r)})
% &=\mathcal U^r_{\gamma_{B_r}, x}(((\overline{T}_t^{r} u)_{r, \gamma})|_{\U^l(B_r)})
% \\
% &= \mathcal U^r_{\gamma_{B_r}, x}(((u)_{r, \gamma})|_{\U^l(B_r)}) \in W_{loc}^{1,2}(\mssm_r) \fstop \notag
% \end{align} 
%The Step 1 has been terminated. 

\paragraph{Verificaiton of (c) in Def.~\ref{d:core}}Let $u \in \mathcal C_r$. Thanks to the verification of (b), the square field~$\cdc^{\dUpsilon}_r(\bar{T}_{r, t}^{\U, \QP} u)$ is well-defined, and by~\eqref{eq:p:MarginalWP:0} it holds that for $\QP$-a.e.~$\eta$
\begin{align} \label{e:RLSP}
\cdc^{\dUpsilon}_r(\bar{T}_{r, t}^{\U, \QP} u)(\gamma+\eta_{B_r^c}) &= \cdc^{\dUpsilon(B_r)}\bigl((\bar{T}_{r, t}^{\U, \QP} u)_{r}^\eta\bigr)(\gamma) \quad \text{$\mu_{r}^\eta$-a.e.~$\gamma \in \U(B_r)$} \fstop 
\end{align}
In view of the contraction property of the semigroup with respect to the form by general theory (see, e.g.,~\cite[p.23, Lem.~1.3.3]{FukOshTak11}), viz.
$$\E^{\U(B_r), \QP_r^{\eta}}(T_{t}^{\U(B_r), \QP_r^{\eta}} u_{r}^\eta) \le \E^{\U(B_r), \QP_r^{\eta}}(u_{r}^\eta)$$
as well as Prop.~\ref{prop: 1-1} and \eqref{e:RLSP}, we obtain
\begin{align*}
\int_{\U} \cdc^{\dUpsilon}_r(\bar{T}_{r, t}^{\U, \QP} u) \diff \QP &= \int_{\U} \E^{\U(B_r), \QP_r^{\eta}}\bigl((\bar{T}_{r, t}^{\U, \QP} u)_{r}^\eta \bigr)  \diff \mu(\eta)
%\int_{\U}  \cdc^{\dUpsilon(B_r)}\bigl((\bar{T}_{r, t}^{\U, \QP} u)_{r}^\eta\bigr) \circ \pr_r \diff\mu(\eta) 
\\
&= \int_{\U} \E^{\U(B_r), \QP_r^{\eta}}(T_{t}^{\U(B_r), \QP_r^{\eta}} u_{r}^\eta)  \diff \mu(\eta)
\\
&\le   \int_{\U} \E^{\U(B_r), \QP_r^{\eta}}(u_{r}^\eta) \diff\mu(\eta) 
\\
&=   \E_r^{\U, \QP}(u) <\infty \fstop
\end{align*}
The verification of (c) is completed. Therefore, we confirmed $\bar{T}_{r, t}^{\U, \QP} \mathcal C_r \subset \mathcal C_r$, which concludes the statement.  %Thanks to Lem.~\ref{l:MU}
%\smallskip
%
%Now, thanks to Lem.~\ref{l:MU}, we conclude that $\mathcal C_r \subset \dom{\bar{\E}_r^{\U, \QP}}$ is dense, which concludes the statement.  
%and $(\bar{\E}_r^{\U, \QP}, \mathcal C_r)$ is Markov unique. 
%As $\bar{\E}_r^{\U, \QP}$ coincides with ${\E}_r^{\U, \QP}$ on $\mathcal C_r$, we conclude that $({\E}_r^{\U, \QP}, \mathcal C_r)$ is Markov unique as well, and 
%$$(\E_r^{\U, \QP}, \dom{\E_r^{\U, \QP}}) = (\bar{\E}_r^{\U, \QP}, \dom{\bar{\E}_r^{\U, \QP}}) \fstop$$
%The proof is complete.
%We second show that $\mathcal C_r$ is dense in $\dom{\E_r}$.  by aid of $\overline{T}_t^{r} \mathcal C_r \subset \mathcal C_r$. \purple{Use Lem.~\ref{l:MU} and shorten the proof.}
%Recall that $(\overline{A}_r, \mathcal D(\overline{A}_r))$ is the $L^2$-infinitesimal generator associated with $(\E_r, \mathcal D(\overline{\E}_r))$. Then, noting that $\overline{T}^r_{t}\mathcal D(\overline{A}_r) \subset \mathcal D(\overline{A}_r)$ by general theory of Dirichlet forms, combining it with $\overline{T}^r_{t} \mathcal C_r \subset \mathcal C_r$, we obtain that 
%$$\overline{T}^r_{t}\bigl( \mathcal C_r \cap \mathcal D(\overline{A}_r) \bigr) \subset  \mathcal C_r \cap \mathcal D(\overline{A}_r) \fstop$$
% Thus, by \cite[Thm.~X.49]{ReeSim75}, we have that $\mathcal C_r \cap \mathcal D(\overline{A}_r)$ is dense in the graph norm in the space $(\overline{A}_r, \mathcal D(\overline{A}_r))$. Namely, we obtained
% \begin{align} \label{e:ESA}
% \text{$(\overline{A}_r, \mathcal C_r \cap \mathcal D(\overline{A}_r))$ is essentially self-adjoint} \comma
% \end{align}
% i.e., there is at most one self-adjoint extension of the operator $(\overline{A}_r, \mathcal C_r \cap \mathcal D(\overline{A}_r))$. 
% By a simple integration-by-parts argument
% $$-\E_r(u,u) = (\overline{A}_r u, u)_{L^2(\QP)} \le \|\overline{A}_r u\|_{L^2(\QP)}\|u\|_{L^2(\QP)} \comma$$
% and by the density of $\mathcal C_r \subset L^2(\QP)$, we conclude that $\mathcal C_r \cap \mathcal D(\overline{A}_r)$ is dense in $\mathcal D(\bar{\E}_r)$. In particular, $\mathcal C_r$ is dense in $\mathcal D(\bar{\E}_r)$, which concludes the first statement. The latter statement regarding the Markov uniqueness is now a consequence of the essential self-adjointness \eqref{e:ESA}, see e.g., \cite[p.28]{Ebe99}.
\end{proof}


%\begin{proof}[Proof of Theorem \ref{t:S=M}]
%The inclusion $\mathcal D(\E_r) \subset \mathcal D(\overline{\E}_r)$ with the inequality $\overline{\E}_r \le \E_r$ is straightforward by definition. 
%Noting $\overline{\E}_r = \E_r$ on $\mathcal C_r$, it suffices to show that $\mathcal C_r \subset \mathcal D(\bar{\E}_r)$ is dense. 
%\smallskip
%
%We first show that $\overline{T}_t^{r} \mathcal C_r \subset \mathcal C_r$. As $\overline{T}_t^{r}$ is an $L^\infty$-contraction semigroup by a general result of Dirichlet forms, we obtain $\overline{T}_t^{r} \mathcal C_r \subset L^\infty(\mu)$, which verifies (a) in Def.\ \ref{d:core}. In the following, we verify (b) and (c) in Def.\ \ref{d:core} for $\overline{T}^r_tu$ with $u \in \mathcal C_r$. 
%
%\paragraph{Step 1: Verification of (b)}Let $u \in \mathcal C_r$ and we verify (b) in Def.\ \ref{d:core}, i.e., we show $\mathcal U_{\gamma, x}(\overline{T}_t^ru) \in W_{loc}^{1,2}(\mssm_r)$ for $\mu$-a.e.\ $\gamma$ and $x \in \gamma_{B_r}$. 
%By Prop.\ \ref{prop: 1-1}, there exists $\Omega \subset \U$ with $\mu(\Omega)=1$ so that  
%%and $\Sigma_{r, \eta} \subset \U(B_r)$ for $\eta \in \Omega$ with $\mu_{r, \eta}(\Sigma)=1$ so that  
%\begin{align} \label{e:SLE}
%(\overline{T}_t^{r} u)_{r, \gamma}(\gamma_{B_r})=T_{t}^{r, \gamma}u_{r, \gamma}(\gamma_{B_r}) \quad \forall \gamma \in \Omega \fstop
%\end{align} 
%As $\mu(\Omega)=1$, by a simple disintegration argument, we can take $\tilde{\Omega} \subset \Omega$ with $\mu(\tilde{\Omega})=1$ and $\mu_{r, \eta}(\Omega_{r, \eta})=1$ for {\it every} $\eta \in \tilde{\Omega}$. For the notational simplicity, we do not relabel this set and use the symbol $\Omega$ in place of such $\tilde{\Omega}$. %We fix $\eta \in \Omega$ in the following argument. 
%%\purple{(otherwise, replace $\Omega$ with $\Omega' \subset \Omega$ satisfying the same property with $\mu(\Omega')=1$). }
%%for every $\eta \in \Omega$ and $\gamma \in \Sigma_{r, \eta}$. 
%By applying Lem.\ \ref{l:RPL} in such a way that we take $\Omega_{r, \eta}$ to be $\Omega$\footnote{Use a different symbol, otherwise it is confusing.} in Lem.\ \ref{l:RPL}, there exists $\Omega' \subset \U(B_r)$ with $\mu_{r, \eta}(\Omega')=1$ and $A \subset B_r$ with $\mssm_r(A^c)=0$ so that for every $\gamma \in \Omega'$, $x \in \gamma$ and $y \in A$, 
%$$\{\1_{\R \setminus \{x\}}\cdot\gamma+\delta_y: \gamma \in \Omega',\ x \in \gamma\} \subset \Omega_{r, \eta} \comma \quad \forall y \in A\comma$$
%which implies that for every $\gamma \in \Omega'$, $x \in \gamma_{B_r}$, $y \in A$ and $\eta \in \Omega$, the following holds: 
%\begin{align*}% \label{eq:S=T-2}
%\mathcal U^r_{\gamma, x}((\overline{T}_t^{r} u)_{r, \eta})(y)=\mathcal U^r_{\gamma, x}(T_{t}^{r, \eta}u_{r, \eta}(\cdot_{B_r}))(y)\fstop
%\end{align*}
%In particular, we have that 
%for every $\gamma \in \Omega'$, $x \in \gamma_{B_r}$ and $\eta \in \Omega$,
%\begin{align} \label{eq:S=T-2}
%\mathcal U^r_{\gamma, x}((\overline{T}_t^{r} u)_{r, \eta})=\mathcal U^r_{\gamma, x}(T_{t}^{r, \eta}u_{r, \eta}(\cdot_{B_r})) \quad \text{$\mssm_r$-a.e.}\fstop
%\end{align}
%
%We next discuss the regularity of $T_t^{r, \eta} v$ for $v \in  \mathcal D(\E_r^\eta)$. 
%It holds that $T_{t}^{r, \eta} v \in \mathcal D(\E_r^\eta)$ for every $v \in \mathcal D(\E_r^\eta)$ by a general property of Dirichlet form (see, e.g., \cite[p.23, Lem.\ 1.3.3]{FukOshTak11}). 
%As $(\U^k(B_r), \mssd_\U, \mu_r^{k, \eta})$ is $\RCD(0,\infty)$ for every $k \in \N_0$, the corresponding semigroup $T_{t}^{r, \eta}$ restricted on each $\U^k(B_r)$ satisfies $L^\infty(\nu_r^{k, \eta})$-to-$\Lip(\U^k(B_r), \mssd_\U)$ property (\cite[Thm.\ 6.5]{AmbGigSav14}) \purple{Take care that $\Psi_{r, \eta}^k$ might be zero due to the rigidity, in which case $L^\infty$-to-Lip property is void but in this case $\overline{T}_t^r \mathcal C_r = \mathcal C_r$ and there is nothing to be proved}, which implies  $T_{t}^{r, \eta}v \in \Lip(\U(B_r), \mssd_\U)$ for any bounded function $v$, and its Lipchitz constant $\Lip_{\mssd_\U}(T_{t}^{r, \eta}v)$ is bounded by $c(t)\|T_{t}^{r, \eta}v\|_\infty \le c(t)\|v\|_\infty$ with constant $c(t)$ depending only on $t$.  Thus, by \eqref{e:SLE} we obtain that, for $\mu$-a.e.\ $\eta$ 
%\begin{align} \label{e:TLIP}
%T_t^{r, \eta} u_{r, \eta} \in  \Lip(\U(B_r), \mssd_\U)\fstop
%\end{align} 
%By Lem.\ \ref{l:LSO}, the following equality holds true for every $\gamma \in \U$, every $x \in \gamma_{B_r}$ and every $y \in B_r$ 
%\begin{align} \label{e:UE}
% \mathcal U_{\gamma, x}(\overline{T}_t^{r} u)(y) =\mathcal U^r_{\gamma_{B_r}, x}((\overline{T}_t^{r} u)_{r, \gamma})(y) \fstop
% \end{align}
%By \eqref{eq:S=T-2}, \eqref{e:TLIP},  \eqref{e:UE} and Lem.\ \ref{l:DT} combined with the fact that $\Lip(\R) \subset W_{loc}^{1,2}(\mssm_r)$, we conclude $\mathcal U_{\gamma, x}(\overline{T}_t^{r} u) \in W_{loc}^{1,2}(\mssm_r)$ for $\mu$-a.e.\ $\gamma$ and every $x\in \gamma_{B_r}$. 
%
%\paragraph{Step 2: Verificaiton of (c)}By the contraction property of the semigroup with respect to the form (see, e.g., \cite[p.23, Lem.\ 1.3.3]{FukOshTak11}), 
%$$\overline{\E}_r(\overline{T}^r_t u) = \int_{\U} \E_r^\eta((\overline{T}^r_t u)_{r, \eta}) \diff\mu(\eta) = \int_{\U} \E_r^\eta(T^{r, \eta}_{t} u_{r, \eta})  \diff \mu(\eta)\le   \int_{\U} \E_r^\eta(u_{r, \eta}) \diff\mu(\eta) <\infty \fstop$$
%Thus, $\overline{T}_t^{r} \mathcal C_r \subset \mathcal C_r$. 
%\smallskip
%
%We second show that $\mathcal C_r$ is dense in $\dom{\E_r}$ by aid of $\overline{T}_t^{r} \mathcal C_r \subset \mathcal C_r$. 
%Recall that $(\overline{A}_r, \mathcal D(\overline{A}_r))$ is the $L^2$-infinitesimal generator associated with $(\E_r, \mathcal D(\overline{\E}_r))$. Then, noting that $\overline{T}^r_{t}\mathcal D(\overline{A}_r) \subset \mathcal D(\overline{A}_r)$ by general theory of Dirichlet forms, combining it with $\overline{T}^r_{t} \mathcal C_r \subset \mathcal C_r$, we obtain that 
%$$\overline{T}^r_{t}\bigl( \mathcal C_r \cap \mathcal D(\overline{A}_r) \bigr) \subset  \mathcal C_r \cap \mathcal D(\overline{A}_r) \fstop$$
% Thus, by \cite[Thm.~X.49]{ReeSim75}, we have that $\mathcal C_r \cap \mathcal D(\overline{A}_r)$ is dense in the graph norm in the space $(\overline{A}_r, \mathcal D(\overline{A}_r))$. Namely, we obtained
% \begin{align} \label{e:ESA}
% \text{$(\overline{A}_r, \mathcal C_r \cap \mathcal D(\overline{A}_r))$ is essentially self-adjoint} \comma
% \end{align}
% i.e., there is at most one self-adjoint extension of the operator $(\overline{A}_r, \mathcal C_r \cap \mathcal D(\overline{A}_r))$. 
% By a simple integration-by-parts argument
% $$-\E_r(u,u) = (\overline{A}_r u, u)_{L^2(\QP)} \le \|\overline{A}_r u\|_{L^2(\QP)}\|u\|_{L^2(\QP)} \comma$$
% and by the density of $\mathcal C_r \subset L^2(\QP)$, we conclude that $\mathcal C_r \cap \mathcal D(\overline{A}_r)$ is dense in $\mathcal D(\bar{\E}_r)$. In particular, $\mathcal C_r$ is dense in $\mathcal D(\bar{\E}_r)$, which concludes the first statement. The latter statement regarding the Markov uniqueness is now a consequence of the essential self-adjointness \eqref{e:ESA}, see e.g., \cite[p.28]{Ebe99}.
%\end{proof}

As a consequence of Thm.~\ref{t:S=M} and Prop.~\ref{prop: 1-1}, we obtain the superposition formula for  the semigroup $\sem{T_{r, t}^{\U, \QP}}$ in terms of the semigroup $\sem{T_t^{\U(B_r), \QP_r^\eta}}$. %We now provide the relation between $\{T^{\eta, r}_{t}\}_{t>0}$ and $\{\overline{T}^r_{t}\}_{t>0}$. 
\begin{cor}[Coincidence of semigroups]\label{prop: 1}
	%Let $\alpha>0$, $t>0$, and $r>0$ be fixed. 
The following three operators coincide: 
	\begin{align} \label{eq: R-1}
%		G_{E, \alpha}^{\dUpsilon, \QP} u(\gamma) & = G_{\alpha}^{\dUpsilon(E), \QP^\gamma_{E}}u_{E, \gamma}(\gamma_{E}) \, ,
%		\\
		%\label{eq: R-2}
		T_{r, t}^{\U, \QP}u(\gamma) & = \bar{T}_{r, t}^{\U, \QP}u(\gamma) =T^{\U(B_r), \QP_r^\gamma}_{t} u_{r}^\gamma(\gamma_{B_r}) \, ,
	\end{align}
	for $\QP$-a.e.\ $\gamma\in \dUpsilon$, any $t>0$.
\end{cor}
%\begin{proof}
%This is an immediate consequence of Prop.\ \ref{prop: 1-1} and Thm.\ \ref{t:S=M}. 
%\end{proof}


%\subsection{Monotone limit of Dirichlet forms $\E_r$}
\subsection{Monotone limit form}
We now construct a Dirichlet form on $\U$ with $\sine_\beta$-invariant measure~$\QP$ as the monotone limit of $(\E^{\U, \QP}_r, \dom{\E^{\U, \QP}_r}$ as $r \to \infty$.
The following proposition follows immediately from the definitions of the square field~$\Gamma^{\U}_r$ and the core~$\mathcal C_r$.
\begin{prop}[Monotonicity]\label{p:mono}
The form $(\E^{\U, \QP}_r, \dom{\E^{\U, \QP}_r}$ and the square field $\Gamma^\U_r$ are monotone increasing as $r \uparrow \infty$, viz., 
$$\cdc^\U_r(u) \le \cdc^\U_s(u)\comma \quad \E^{\U, \QP}_r(u) \le \E^{\U, \QP}_s(u) \comma \quad \dom{\E^{\U, \QP}_s} \subset \dom{\E^{\U, \QP}_r} \quad r\le s \fstop$$
\end{prop}
\begin{proof}
As $\mathcal C_r$ is a core of the form~$(\E^{\U, \QP}_r, \dom{\E^{\U, \QP}_r})$, it suffices to check~$\mathcal C_s \subset \mathcal C_r$ and~$\cdc^\U_r(u) \le \cdc^\U_s(u)$ on~$\mathcal C_s$. Let $u \in \mathcal C_s$ and we show $u \in \mathcal C_r$.  
%It can be readily seen that $\mathcal C_s \subset \mathcal C_r$ for $r \le s$:  
%Let $u \in \mathcal C_s$ and we show $u \in \mathcal C_r$.  By Lem~\ref{l:LSO} and the assumption~$u \in \mathcal C_s$, we have that for any $\gamma \in \U$ and $x \in \gamma_{B_s}$
%\begin{align} \label{e: LOS1}
%\mathcal U^s_{\gamma_{B_s}, x}(u_s^\gamma) =\mathcal U_{\gamma, x}(u) \in W_{loc}^{1,2}(\mssm_s)\fstop
%\end{align} 
 By a simple reasoning similar to the proof of~Lem.~\ref{l:SEF3}, we can see %for any $f \in W_{loc}^{1,2}(\mssm_s)$,
\begin{align*}% \label{e: LOS}
 u_r^\eta \in \Lip_b(\U(B_r), \mssd_\U) \quad \text{$\QP$-a.e.~$\eta$ \ \ if} \ \    u_s^\eta \in \Lip_b(\U(B_s), \mssd_\U) \quad \text{$\QP$-a.e.~$\eta$} \fstop
\end{align*} 
%By Lem~\ref{l:LSO} and the definition of the operator $\mathcal U^r_{\gamma, x}$, we have that for any~$\gamma \in \U$ and $x \in \gamma_{B_r}$, 
%\begin{align*}% \label{e: LOS2}
%\mathcal U_{\gamma, x}(u)=\mathcal U^r_{\gamma_{B_r}, x}(u_r^\gamma)= \mathcal U^s_{\gamma_{B_s}, x}(u_s^\gamma) \cdot\1_{B_r} \quad \text{on $B_r$}\fstop
%\end{align*}
%By~\eqref{e: LOS1}, ~\eqref{e: LOS} and~\eqref{e: LOS2}, we conclude~$\mathcal U_{\gamma, x}(u)\in  W_{loc}^{1,2}(\mssm_r)$. 
%\begin{align*}
%\mathcal U_{\gamma, x}(u)=\mathcal U^r_{\gamma_{B_r}, x}(u_r^\gamma)= \mathcal U^s_{\gamma_{B_s}, x}(u_s^\gamma) \cdot\1_{B_r}   \in W_{loc}^{1,2}(\mssm_r)
%\end{align*}
 %we conclude \purple{$\mathcal U^r_{\gamma, x}(u) \in  W_{loc}^{1,2}(\mssm_r)$. 
%\purple{ (Introduce a convenient notation for $\mathcal U_{\gamma, x}$ to distinguish the space on which this function is defined.)}
By Def.~\ref{d:DT}, it is straightforward to see $\cdc^\U_r(u) \le \cdc^\U_s(u)$. Thus, 
$$\E^{\U, \QP}_{r}(u) =\int_{\U}\cdc^\U_r(u) \diff \mu \le \int_{\U}\cdc^\U_s(u) \diff \mu=\int_{\U}\cdc^\U_s(u) \diff \mu =\E^{\U, \QP}_s(u) <\infty \fstop$$
Therefore, we conclude $u \in \mathcal C_r$.
%\purple{Explain this more.}
 %By definition, it is clear that $\mathcal C_s \subset \mathcal C_{r}$ whenever $r \le s$ and 
%$$\Gamma_r^\U(u) \le \Gamma_s^{\U}(u) \quad  u \in \mathcal C_r \fstop$$
The proof is completed. 
\end{proof}


We now define a Dirichlet form on $\U$ whose invariant measure is the~$\sine_\beta$ measure~$\QP$ by the monotone limit of~$(\E^{\U, \QP}_r, \dom{\E^{\U, \QP}_r})$. 
\begin{defs}[Monotone limit form] \label{d:DFF}
The form $(\E^{\U, \QP}, \dom{\E^{\U, \QP}})$ is defined as the monotone limit:
\begin{align} \label{eq:Temptation} 
\dom{\E^{\U, \QP}}&:=\{u \in \cap_{r>0} \dom{\E^{\U, \QP}_r}: \E^{\U, \QP}(u) = \lim_{r \to \infty}\E^{\U, \QP}_r(u) <\infty\} \comma
\\
\E^{\U, \QP}(u)&:=\lim_{r \to \infty}\E^{\U, \QP}_r(u) \fstop \notag
\end{align} 
The form $(\E^{\U, \QP}, \dom{\E^{\U, \QP}})$ is a Dirichlet form on~$L^2(\mu)$ as it is the monotone limit of Dirichlet forms (e.g., by \cite[Exercise 3.9]{MaRoe90}).
%Let $\mathcal C \subset \cap_{r \in \N} \mathcal C_r$ be defined as the space of $\mu$-classes of measurable functions so that
%\begin{align} \label{eq:Temptation} 
%\E(u) = \lim_{r \to \infty}\E_r(u) <\infty \fstop
%\end{align} 
The square field~$\cdc^\U$ is defined as the monotone limit of~$\cdc^\U_r$ as well:
\begin{align} \label{d:SF}
\cdc^\U(u):=\lim_{r \to \infty}\cdc^\U_r(u) \quad u \in \dom{\E^{\U, \QP}}\fstop
\end{align}
The corresponding $L^2(\mu)$-semigroup is denoted by $\sem{T^{\U, \QP}_t}$.
%\begin{enumerate}[$(a)$]
%%\item\label{i:d:core1} $ u \in L^\infty(\mu)$;  
%%\item\label{i:d:core2} \purple{Ignore the following: $U_{\gamma, x} \in W^{1,2}_{loc}(\mssm)$ for every $x \in X$ and $\gamma \in \U$; \purple{Change the definition like $U_{\gamma, x}(u_{r, \eta}) \in W^{1,2}_{loc}(\mssm_r)$ for any $r>0$}}
%%\item\label{i:d:core3}  The following monotone limit is finite:
%\begin{align} \label{eq:Temptation} 
%\E(u) = \lim_{r \to \infty}\E_r(u) <\infty \fstop
%\end{align} 

%\item\label{i:d:TLS:3} $\Bo{\T}\subset \A\subset \Bo{\T}^\mssm$ and $\A$ is \emph{$\mssm$-essentially countably generated}, i.e.\ there exists a countably generated $\sigma$-sub\-al\-ge\-bra~$\A_0$ of~$\A$ so that for every~$A\in \A$ there exists $A_0\in \A_0$ with~$\mssm(A\triangle A_0)=0$.
%
%\item\label{i:d:TLS:4} $\msE \subset \A_\mssm$ is a \emph{localizing ring}, i.e.\ it is a ring ideal of~$\A$, and there exists a \emph{localizing sequence}~$\seq{E_h}_h\subset \msE$ so that $\msE=\cup_{h\geq 0} (\A\cap E_h)$.
%
%\item\label{i:d:TLS:5} for every~$x\in X$ there exists a $\T$-neighborhood~$U_x$ of~$x$ so that~$U_x\in\msE$.
%\end{enumerate}

\end{defs}
%\begin{rem} \label{r:LP}
%By Lem.~\ref{p:CL} and the construction of~$\E^{\U, \QP}$ in~\eqref{eq:Temptation}, the domain~$\dom{\E^{\U, \QP}}$ contains the Lipschitz algebra~$\Lip(\mssd_{\U})$. 
%\end{rem}

We now show that the form~$(\E^{\U, \QP}, \dom{\E^{\U, \QP}})$ is a local Dirichlet form on~$L^2(\QP)$ and satisfies the Rademacher-type property  with respect to the $L^2$-transportation-type distance~$\bar{\mssd}_\U$.
%We introduce the following notation:
%$$\Lip(\mssd_\U, \QP):=\{u \in \Lip(\mssd_\U): \text{$u$ is $\mu$-measurable}\} \fstop$$
%Note that $\mssd_\U$ being merely an extended distance, $\mssd_\U$-Lipschitz functions are not necessarily measurable with respect to Borel measures associated with the vague topology~$\tau_\mrmv$, for which the aforementioned notation is needed.
\begin{prop}\label{p:DF}
The form $(\E^{\U, \QP}, \dom{\E^{\U, \QP}})$ is a
%\footnote{As we do not know if there exists a core consisting of continuous functions, we cannot check (QR2) in an obvious way. It would be better not to state the quasi-regularity. } 
local Dirichlet form on~$L^2(\QP)$.
Furthermore,  $(\E^{\U, \QP}, \dom{\E^{\U, \QP}})$ satisfies Rademacher-type property:
\begin{align}\label{p:Rad}
\Lip(\bar{\mssd}_\U, \QP) \subset \dom{\E^{\U, \QP}}\comma \quad \cdc^\U(u) \le \Lip_{\bar{\mssd}_\U}(u)^2 \qquad \forall u \in \Lip(\bar{\mssd}_\U, \QP) \fstop
\end{align}
%\purple{Use the large distance $\bar{\mssd}_\U$ instead of $\mssd_\U$. }
%Furthermore, $\cdc^\U$ is the square field for $(\E^{\U, \QP}, \dom{\E^{\U, \QP}})$.
%Furthermore, the following holds:
%\begin{enumerate}[$(a)$]
%\item $(\E, \mathcal C)$ is Markov unique, i.e., the closure $(\E, \dom{\E})$ is the unique Markov extension of $(\E, \mathcal C)$;
%\item $(\E, \dom{\E})$ is quasi-regular. 
%\end{enumerate}
\end{prop}
\proof
%In view of the monotonicity in Prop.\ \ref{p:mono}, the form $(\E^{\U, \QP}, \dom{\E^{\U, \QP}})$  is a Dirichlet form by \cite[Prop.~3.7]{MaRoe90}.
%The proof of Rademacher-type property is identical to that of \cite[Thm.~5.3]{LzDSSuz21}, which is, however, stated under slightly different assumption from the current statement. 
%We, therefore, sketch the proof below. As noted in Rem.~\ref{r:LP}, it holds that $\Lip(\mssd_\U) \subset \dom{\E^{\U, \QP}}$. We only need to sketch $\cdc^\U(u) \le \Lip_{\mssd_\U}(u)^2$. 
The local property of $(\E^{\U, \QP}, \dom{\E^{\U, \QP}})$ follows from~\eqref{d:SF} and the locality of $\cdc^{\U}_r$ for every $r>0$.
% the fact that $(\E^{\U, \QP}, \dom{\E^{\U, \QP}})$ is the monotone limit of the local Dirichlet form~$(\E^{\U, \QP}_r, \dom{\E^{\U, \QP}_r})$. 
We show the Rademacher-type property. 
Since~$\cdc^\U$ is the limit square field of~$\cdc^{\U}_r$ as in~\eqref{d:SF}, it suffices to show 
\begin{align*} %\label{e:RD2}
\Lip(\bar{\mssd}_\U, \QP)  \subset \mathcal C_r \quad \text{and} \quad \cdc^\U_r(u) \le \Lip_{\bar{\mssd}_\U}(u)^2  \quad \forall u \in \Lip(\bar{\mssd}_\U, \QP) \quad \forall r>0\comma
\end{align*}
which has been already proven in Prop.~\ref{t:ClosabilitySecond}. We verified~\eqref{p:Rad}. 
%As~$(\U^k(B_r), \mssd_\U, \mu_r^{k, \eta})$ is $\RCD(0,\infty)$ with $k=k(\eta)$ for $\mu$-a.e.~$\eta$ and the Cheeger energy~$\Ch_{\mssd_\U, \QP_r^{k, \eta}}$ coincided with~$\E^{\U(B_r), \QP_r^{k, \eta}}$ by Prop.~\ref{p:BE2}, the Rademacher-type property for~$\E^{\U(B_r), \QP_r^{k, \eta}}$ follows from the one for~$\Ch_{\mssd_\U, \QP_r^{k, \eta}}$, the latter of which is an immediate consequence by the definition of the Cheeger energy. Thus, this implies  
%\begin{align} \label{e:RD2}
%\cdc^{\U(B_r)}(u) \le \Lip_{\mssd_\U}(u)^2  \quad \forall u \in \Lip(\U(B_r), \mssd_\U) \quad \forall r>0\fstop
%\end{align}
%In view of the relation between~$\cdc^\U_r$ and~$\cdc^{\U(B_r)}$ in~\eqref{eq:p:MarginalWP:0} and the fact that the operation~$(\cdot)_r^\eta$ transfers $u \in \Lip(\U, \mssd_\U)$ to~$u_r^\eta \in \Lip(\U(B_r), \mssd_\U)$ and contracts Lipschitz constants~$\Lip_{\mssd_\U}(u_r^\eta) \le \Lip_{\mssd_\U}(u)$, we conclude~\eqref{e:RD}.
%The $\tau_\mrmv$-quasi-regularity is a consequence of the Rademacher-type property~\eqref{p:Rad} of~$(\E^{\U, \QP}, \dom{\E^{\U, \QP}})$, see~the proof of~\cite[Prop.~6.1]{LzDSSuz21}.
%\footnote{Use the reference no.~in the uploaded version.} 
%For the sake of the completeness, we prove a short proof below. 
%Since~$\QP$ is a probability measure,~$L^2(\QP)$ contains constant functions.
%By definition of~$\SF{\dUpsilon}{\QP}$ we have~$\SF{\dUpsilon}{\QP}(\car)\equiv 0$, and conservativeness follows.
%
%In order to show the quasi-regularity, it suffices to show that the $\EE{\dUpsilon}{\QP}_1$-capacity is tight.
%Since~$\QP$ is Radon by Remark~\ref{r:QPRadon}, there exists a sequence~$\seq{\Kappa_n}_n$ of $\T_\mrmv(\Ed)$-compact sets so that~$\nlim \QP \Kappa_n=1$.
%For~$n\in \N$ let~$\rho_{\Kappa_n}$ be defined as in~\eqref{eq:W2Upsilon:0} and note that the sub-level sets~$\set{\rho_{\Kappa_n}\leq 1/2}$ are $\T_\mrmv(\Ed)$-compact by Proposition~\ref{p:W2Upsilon}\iref{i:p:W2Upsilon:8}.
%By~$(\Rad{\mssd_\dUpsilon}{\QP})$ we have that~$\sup_n \EE{\dUpsilon}{\QP}(\rho_{\Kappa_n}\wedge 1)<\infty$, hence, by e.g.~\cite[Lem.~I.2.12]{MaRoe92}, there exists a subsequence~$\seq{k_n}_n$ so that~$u_n\eqdef \tfrac{1}{n}\sum_j^n \rho_{\Kappa_{k_j}}\wedge 1$ $\tparen{\EE{\dUpsilon}{\QP}_1}^{1/2}$-converges to~$0$.
%Furthermore,~$u_n\geq 1/2$ on~$\tset{\rho_{\Kappa_{k_j}}\geq 1/2}$ for all~$j\leq n$.
%As a consequence,
%\begin{align*}
%\Cap_{\EE{\dUpsilon}{\QP}_1}\set{\rho_{\Kappa_{k_n}}> 1/2}\leq \Cap_{\EE{\dUpsilon}{\QP}_1}\set{u_n> 1/2} \leq \EE{\dUpsilon}{\QP}_1(u_n) \xrightarrow{n\rar\infty} 0\comma
%\end{align*}
%which shows that~$\Cap_{\EE{\dUpsilon}{\QP}_1}$ is tight.
The proof is complete.
\qed

%\begin{rem}
%The construction of the form~$(\E^{\U, \QP}, \dom{\E^{\U, \QP}})$ slightly deviates from the one defined in \cite[Thm.~3.48]{LzDSSuz21}, where the form was constructed as the smallest extension of a certain core $\mathcal C \subset \cap_{r>0} \mathcal C_r$. 
%%As the core $\mathcal C$ used there is contained in $\dom{\E^{\U, \QP}}$ by construction, our form~$(\E^{\U, \QP}, \dom{\E^{\U, \QP}})$ is in general larger than the form defined there. 
%%As the domains of the both forms coincide on the Lipschitz algebra~$\Lip(\mssd_\U, \QP)$, these two forms will be, {\it a posteriori}, identified each other due to the Markov uniqueness of $(\E^{\U, \QP}, \Lip(\mssd_\U, \QP))$ shown later in~Cor.~\ref{c:MU}. 
%\end{rem}

\begin{prop}\label{prop: MGS}
%Suppose the same assumptions in Theorem~\ref{thm: Erg1} with $\seq{E_h}_{h}$. 
The semigroup $\{T^{\U, \QP}_{t}\}_{t \ge 0}$ is the $L^2(\QP)$-strong operator limit of the semigroups $\{T^{\U, \QP}_{r, t}\}_{t \ge 0}$, viz., 
%are monotone non-increasing in $h$ on non-negative functions, i.e.,
%\begin{align} \label{ineq: RM}
%G_{E_h, \alpha}^{\dUpsilon, \QP} u \le G^{\dUpsilon, \QP}_{E_{h'}, \alpha} u,  \quad T^{\dUpsilon, \QP}_{E_h, t} u \le T^{\dUpsilon, \QP}_{E_{h'}, t} u,   \quad \text{for any nonnegative}\, \, u \in L^2(\dUpsilon, \QP), \quad h \le h'.
%\end{align}
$$\text{{\small $L^2(\mu)$--}}\lim_{r \to \infty} T^{\U, \QP}_{r, t} u= T^{\U, \QP}_t u  \quad \forall u \in  L^2(\QP)\comma \quad t>0 \fstop$$
Furthermore, $\{T^{\U, \QP}_{t}\}_{t \ge 0}$ is also the $L^1(\QP)$-strong operator limit of the semigroups $\{T^{\U, \QP}_{r, t}\}_{t \ge 0}$, viz., 
$$\text{{\small $L^1(\mu)$--}}\lim_{r \to \infty} T^{\U, \QP}_{r, t} u= T^{\U, \QP}_t u  \quad \forall u \in  L^1(\QP)\comma \quad t>0 \fstop$$
\end{prop}
\begin{proof}
%Thanks to the identity 
%\begin{equation*}
%	G_{E_h, \alpha}^{\dUpsilon, \QP}  = \int_0^\infty e^{-\alpha t} T^{\dUpsilon, \QP}_{E_h, t} \diff t\, ,
%\end{equation*}
%it suffices to show \eqref{ineq: RM} only for $T^{\dUpsilon, \QP}_{E_h, t}$. By a direct application of~\cite[Theorem 3.3]{Ouh96} and the monotonicity of the Dirichlet form by~Theorem \ref{t:ClosabilitySecond}, we obtain the monotonicity of the semigroup $T^{\dUpsilon, \QP}_{E_h, t}$. 
The first statement follows from the monotonicity of~$(\E^{\U, \QP}_r, \dom{\E^{\U, \QP}_r}$ as $r\uparrow\infty$ proven in~Prop.~\ref{p:mono}
and~\cite[S.14, p.373]{ReeSim80}. The latter statement is a standard consequence of the first statement for the strongly continuous Markovian contraction semigroup. We give a proof for the sake of the completeness. We note that the $L^2$-operators $T^{\U, \QP}_{r, t}$ and $T^{\U, \QP}_{t}$ are uniquely extended to $L^1$-strongly continuous Markovian contraction semigroups (see, e.g, \cite[(1.5.2) in \S 1.5]{FukOshTak11}). As $L^1(\QP) \cap L^2(\QP)$ is dense in $L^1(\QP)$, for any $u \in L^1(\QP)$ and $\e>0$, there exists $u_\e \in L^1(\QP) \cap L^2(\QP)$ so that $\|u-u_\e\|_{L^1(\QP)}<\e$ and   
\begin{align*}
&\|T^{\U, \QP}_{r, t} u - T^{\U, \QP}_{t}u\|_{L^1(\QP)} 
\\
&\le \|T^{\U, \QP}_{r, t} u - T^{\U, \QP}_{r, t}u_\e\|_{L^1(\QP)} +\|T^{\U, \QP}_{r, t} u_\e - T^{\U, \QP}_{t}u_\e\|_{L^1(\QP)} +\|T^{\U, \QP}_{t} u_\e - T^{\U, \QP}_{t}u\|_{L^1(\QP)} 
\\
&\le \|u - u_\e\|_{L^1(\QP)} +\|T^{\U, \QP}_{r, t} u_\e - T^{\U, \QP}_{t}u_\e\|_{L^2(\QP)} +\| u_\e - u\|_{L^1(\QP)} 
\\
&\xrightarrow{r\to\infty} \e + 0 +\e \fstop
\end{align*}
As $\e>0$ is arbitrarily small, the proof is completed. 
\end{proof}
\begin{cor} \label{c:LSG}For any fixed $r>0$ and $u \in \dom{\E^{\U, \QP}}$, 
\begin{align} \label{e:LSG0}
T_{r', t}^{\U, \QP}u \xrightarrow{r' \to \infty} T_t^{\U, \QP} u \quad \text{weakly in $\dom{\E^{\U, \QP}_{r}}$} \fstop
\end{align}
In particular, 
\begin{align} \label{e:LSG}
\int_{\U} \cdc^{\U}_r(T^{\U, \QP}_tu)h \diff \QP 
 \le  \liminf_{r' \to \infty} \int_{\U} \cdc^{\U}_{r}(T_{r', t}^{\U, \QP}u)   h \diff \QP  \comma
\end{align}
for all non-negative $h \in \mathcal D(\E^{\U, \QP}_r) \cap L^\infty(\QP)$.
\end{cor}
\begin{proof}
First of all, \eqref{e:LSG0} is well-posed as $T_{r', t}^{\U, \QP}u, T_t^{\U, \QP} u \in \dom{\E^{\U, \QP}_{r}}$ thanks to the inclusion $\dom{\E^{\U, \QP}}\subset \cap_{r>0} \dom{\E^{\U, \QP}_r}$ and the monotonicity $\dom{\E^{\U, \QP}_{r'}} \subset  \dom{\E^{\U, \QP}_{r}}$ for $r \le r'$. 
%
%We further take a subsequence $(n') \subset (n)$ so that 
%$$\lim_{n' \to \infty} \cdc^{\U}_r(T_{r'_{n'}, t}^{\U, \QP}u)^{1/2} = \liminf_{n \to \infty}\cdc^{\U}_r(T_{r'_n, t}^{\U, \QP}u)^{1/2} \fstop$$
By the monotonicity proven in~Prop.~\ref{p:mono} and the general contraction property of the semigroup with respect to the Dirichlet form, it holds that for $r \le r'$
\begin{align*}
\E^{\U, \QP}_{r}(T_{r', t}^{\U, \QP}u) \le \E^{\U, \QP}_{r'}(T_{r', t}^{\U, \QP}u) \le \E^{\U, \QP}_{r'}(u) \le  \E^{\U, \QP}(u)<\infty \fstop
\end{align*}
Noting also that the semigroup $T_{r', t}^{\U, \QP}$ contracts the $L^2(\QP)$-norm by a general property of semigroups, $\{T_{r', t}^{\U, \QP}u\}_{r'}$ is bounded in $\dom{\E^{\U, \QP}_{r}}$. Thanks to \eqref{prop: MGS},  $\{T_{r', t}^{\U, \QP}u\}_{r'}$ converges to $T_{t}^{\U, \QP}u$ weakly in $\dom{\E^{\U, \QP}_{r}}$. 
The latter statement is a consequence of the first statement, see, e.g., \cite[Lem.~2.4]{HinRam03}. 
%By Mazur's lemma, we may take some convex combination of $\{T_{r'_n, t}^{\U, \QP}u\}_{r'_n}$, denoted by $w_n:=\mathsf{Conv}(T_{r'_i, t}^{\U, \QP}u: {1 \le i \le n})$, so that $w_n$ converges to $T_{t}^{\U, \QP}u$ strongly in $\dom{\E^{\U, \QP}_{r}}$. Thus, up to extracting a (non-relabelled) subsequence, we may assume that $\cdc^{\U}_r(w_n)^{1/2}$ converges to $\cdc^{\U}_r(T_{t}^{\U, \QP}u)^{1/2}$ $\QP$-a.e.. Noting $\cdc^{\U}_r$ is non-negative and quadratic, we have 
%\begin{align*}
%\cdc^{\U}_r(T_{t}^{\U, \QP}u)^{1/2}  = \lim_{n \to \infty} \cdc^{\U}_r(w_n)^{1/2} \le \mathsf{Conv}(\cdc^{\U}_r(T_{r'_i, t}^{\U, \QP}u)^{1/2}: {1 \le i \le n})
%\end{align*}
\end{proof}
% Furthermore, the space $\mathcal C$ is dense in $(\bar{\E}^{\U, \mu}, \mathcal D(\bar{\E}^{\U, \mu}))$. 






\subsection{Bakry--\'Emery Curvature bound for $(\E^{\U, \QP}, \dom{\E^{\U, \QP}})$}
In this subsection, we prove the Bakry--\'Emery curvature bound for the local Dirichlet form~$(\E^{\U, \QP}, \dom{\E^{\U, \QP}})$.
\begin{thm} \label{t: main}
Let $\beta>0$ and $\mu$ be the $\sine_\beta$ ensemble. The local Dirichlet form~$(\E^{\U, \QP}, \dom{\E^{\U, \QP}})$ satisfies the Bakry--\'Emery curvature dimension condition $\BE(0,\infty)$:
\begin{align} \label{m:BE}
\cdc^{\U}\bigl(T_t^{\U, \mu} u\bigr) \le T_t^{\U, \mu} \bigl(\cdc^{\U}(u)\bigr) \quad \forall u \in  \dom{\E^{\U, \mu}} \quad \forall t>0\fstop \tag{$\BE(0,\infty)$}
\end{align}
\end{thm}


%We start from a technical lemma regarding measurability of disintegration. We recall that for $\eta \in \U$, we set $\U_r^\eta:=\{\gamma \in \U: \gamma_{B_r^c}=\eta_{B_r^c}\}$.
%\begin{lem}[disintegration lemma] \label{l:sp}
%Assume that there exists a measurable set $\Xi \subset \U$ with $\QP(\Xi)=1$ so that for every $\eta \in \Xi$, there exists a family of measurable sets $\Omega^\eta \subset \U(B_r)$ so that $\mu_{r}^\eta(\Omega^\eta)=1$ for every $\eta \in \Xi$. Let $\Omega \subset \U$ be the (not necessarily measurable) subset defined by
%$$\Omega:=\bigcup_{\eta \in \Xi} \pr_{r}^{-1}(\Omega^\eta) \cap \U_{r}^{\eta} \fstop$$ 
%Assume further that there exists a measurable set $\Theta \subset \U$ so that $\Omega \subset \Theta$. Then, $\QP(\Theta)=1$. 
%\end{lem}
%\paragraph{Caveat}As the set $\Omega$ is defined as {\it uncountable union} of measurable sets, the measurability of~$\Omega$ is not necessarily true in general. The disintegration formula~\eqref{p:ConditionalIntegration2} is, therefore, not necessarily applicable directly to $\Omega$,  which motivates the aforementioned lemma. 
%\begin{proof}[Proof of Lem.\ \ref{l:sp}]
%Let $\Theta_{r}^{\eta}=\{\gamma \in \U(B_r): \gamma+\eta_{B_r^c} \in \Theta\}$ be a section of~$\Theta$ at~$\eta_{B_r^c}$ as in \eqref{e:SEF2}. Then, $\Omega^{\eta} \subset \Theta_{r}^{\eta}$ by assumption. Thus, $\mu_r^\eta(\Theta_{r}^{\eta}) \ge \mu_r^\eta(\Omega^{\eta})\ge 1$. By the disintegration formula in~\eqref{p:ConditionalIntegration2}, we have that 
%\begin{align*}
%\mu(\Theta)= \int_{\U} \mu_r^\eta(\Theta_{r}^{\eta}) \diff \mu(\eta)  \ge 1\fstop
%\end{align*}
%The proof is completed. 
%\end{proof}

\begin{proof}
We first prove $\BE(0,\infty)$ for the form $(\E^{\U, \QP}_{r}, \mathcal D(\E^{\U, \QP}_r))$.
Let $u \in \mathcal D(\E^{\U, \QP}_r)$. By Prop.~\ref{p:BE2},  by the expression~\eqref{e:R1} of $\QP_r^\eta$ in terms of~$\QP_r^{k, \eta}$ and by the definition~\eqref{d:GSF} of~$\cdc^{\U(B_r)}$, there exists $\Xi_r^1 \subset \U$ with $\QP(\Xi_r^1)=1$ so that for every~$\eta \in \Xi_r^1$ there exists a measurable set $\Omega_{r}^{1, \eta} \subset \U(B_r)$ with  $\QP_{r}^{\eta}(\Omega_{r}^{1, \eta} )=1$ satisfying that for every $\gamma \in \Omega_{r}^{1, \eta}$, the following $1$-Bakry--\'Emery gradient estimate holds:
\begin{align} \label{e:BE2}
\Gamma^{\U(B_r)}(T_t^{\U(B_r), \QP_r^\eta}u_{r}^\eta)(\gamma) \le T_t^{\U(B_r), \QP_r^\eta}\bigl(\Gamma^{\U(B_r)}(u_{r}^\eta)\bigr)(\gamma) \fstop
\end{align}
By~Prop.~\ref{t:ClosabilitySecond}, there exists $\Xi_r^2 \subset \U$ with $\QP(\Xi_r^2)=1$ so that for every~$\eta \in \Xi_r^2$ there exists a measurable set $\Omega_{r}^{2, \eta} \subset \U(B_r)$ with  $\QP_{r}^{\eta}(\Omega_{r}^{2, \eta} )=1$ satisfying that for every $\gamma \in \Omega_{r}^{2, \eta}$ 
\begin{align} \label{e:BE3}
&\cdc^{\U}_r(T_{r, t}^{\U, \QP} u)(\gamma+\eta_{B_r^c})=\Gamma^{\U(B_r)}\Bigl( \bigl(T_{r, t}^{\U, \QP} u\bigr)_r^\eta\Bigr)(\gamma)  \ ;
\\
&\cdc^{\U}_r(u)(\gamma+\eta_{B_r^c})=\Gamma^{\U(B_r)}( u_r^\eta)(\gamma) \fstop \notag
\end{align}
By~Cor.~\ref{prop: 1}, there exists $\Lambda^3_r \subset \U$ with $\QP(\Lambda^3_r)=1$ so that for every $\gamma \in \Lambda^3_r$
\begin{align} \label{e:BE4}
T_{r, t}^{\U, \QP} u(\gamma)=T_{t}^{\U(B_r), \QP_r^\gamma} u_r^\gamma(\gamma)  \fstop
\end{align}
By the standard disintegration argument, we can write 
$$\Lambda^3_r=\bigcup_{\eta \in \Xi_r^3}\pr_{r}^{-1}(\Omega_{r}^{3, \eta})\cap \U_{r}^{\eta} \comma$$ where $ \Omega_{r}^{3, \eta}=(\Lambda^3_r)_r^\eta:=\{\gamma \in \U(B_r): \gamma+\eta_{B_r^c} \in \Lambda^3_r\}$ and $\Xi_r^3={\rm pr}_{B_r^c}(\Lambda^3_r)$, and $ \U_{r}^{\eta}$ has been defined in~\eqref{e:CES}. 
By the disintegration formula~\eqref{p:ConditionalIntegration2}, $\QP(\Xi_r^3)=1$ and $\QP_r^\eta(\Omega_{r}^{3, \eta})=1$ for every~$\eta \in \Xi_r^3$. 

Let  $\Xi_r:=\Xi_{r}^{1} \cap \Xi_{r}^{2}  \cap\Xi_{r}^{3}$ and $\Omega^\eta_r:=\Omega_{r}^{1, \eta} \cap \Omega_{r}^{2, \eta} \cap \Omega_{r}^{3, \eta}$ for $\eta \in \Xi_r$.
Set 
$$\Kappa_r:=\bigcup_{\eta \in \Xi_r} \pr_{r}^{-1}(\Omega_{r}^{\eta})\cap \U_{r}^{\eta} \fstop$$
By construction, $\QP(\Xi_r)=1$ and $\QP_r^\eta(\Omega^\eta_r)=1$ for every~$\eta \in \Xi_r$. By \eqref{e:BE2}, \eqref{e:BE3} and \eqref{e:BE4}, the following inequalities hold for every $\gamma \in \Kappa_r$:
\begin{align} \label{e:BER1-1}
\Gamma^\U_r(T_{r, t}^{\U, \QP} u)(\gamma) 
&= \Gamma^\U_r(T_{r, t}^{\U, \QP} u)(\gamma_{B_r}+\gamma_{B_r^c}) 
\\
&= \Gamma^{\U(B_r)}((T^{\U, \QP}_{r, t} u)_{r}^{\gamma})(\gamma_{B_r})  \notag
\\
&\le T_{t}^{\U(B_r), \QP_r^\gamma}\Gamma^{\U(B_r)}(u_{r}^{\gamma})(\gamma_{B_r})  \notag
\\
&= T_{t}^{\U(B_r), \QP_r^\gamma}\bigl(\Gamma^\U_r(u)_{r}^{\gamma}\bigr)(\gamma_{B_r})  \notag
\\
&= T_{r, t}^{\U, \QP} \Gamma^\U_r(u)(\gamma) \fstop \notag
\end{align}
Let $\Theta:=\{\gamma \in \U: \Gamma^\U_r(T_{r, t}^{\U, \QP} u)(\gamma) \le T_{r, t}^{\U, \QP} \Gamma^\U_r(u)(\gamma)\}$.  Then $\Theta$ is $\QP$-measurable by construction, and thanks to~\eqref{e:BER1-1}, it holds that $\Kappa_r  \subset \Theta$. By applying Lem.~\ref{l:sp}, we obtain $\QP(\Theta)=1$, which concludes $\BE(0,\infty)$ for the truncated form $(\E^{\U, \QP}_{r}, \mathcal D(\E^{\U, \QP}_r))$ for any $r>0$.
%\smallskip

We now prove $\BE(0,\infty)$ of the form $(\E^{\U, \QP}, \dom{\E^{\U, \QP}})$. It suffices to prove 
\begin{align} \label{e:BEG}
\int_{\U}\cdc^{\U}(T^{\U, \QP}_tu) h \diff \QP \le \int_{\U}  T^{\U, \QP}_t\cdc^{\U}(u) h \diff \QP  \comma
\end{align}
for all non-negative $h \in \mathcal D(\E^{\U, \QP}_r) \cap L^\infty(\QP)$. Indeed, thanks to the Rademacher-type property proven in Prop.~\ref{t:ClosabilitySecond},
we have 
$$\Lip_{b, +}(\mssd_\U)\subset  \mathcal D(\E^{\U, \QP}_r) \cap L^\infty_+(\QP).$$
As $\Lip_{b,+}(\mssd_\U)\cap C(\tau_{\mrmv})$ is point separating (see \cite[(a) in Rem.~5.13]{LzDSSuz21}), it is separating (i.e., measure determining) by e.g., \cite[p.113, (a) in Thm.~4.5 in Chap.~3]{EthKur86}. Thus, the inequality \eqref{e:BEG} implies $\cdc^{\U}(T^{\U, \QP}_tu) \le T^{\U, \QP}_t\cdc^{\U}(u)$. 
We now prove \eqref{e:BEG}. Let $u \in \dom{\E^{\U, \QP}} \subset \cap_{r>0} \dom{\E^{\U, \QP}_r}$. By making use of the monotonicity $\cdc^{\U}_{r} \le \cdc^{\U}_{r'}$ for $r \le r'$ (we will use it in the following displayed formulas in the first equality and in the second inequality), Cor.~\ref{c:LSG} (used in the first inequality below),  $\BE(0,\infty)$ for the truncated form~$(\E^{\U, \QP}_{r}, \mathcal D(\E^{\U, \QP}_r))$ for any $r>0$ (used in the third inequality below), and the convergence of $T_{r', t}^{\U, \QP}$ to $T^{\U, \QP}_t$ as $r' \to \infty$ in the $L^1$-strong operator sense by Prop.~\ref{prop: MGS} (used in the last equality),  the following inequalities hold true: %for any non-negative $h \in  \mathcal D(\E^{\U, \QP}_r) \cap L^\infty(\QP)$, 
\begin{align*}
\int_{\U}\cdc^{\U}(T^{\U, \QP}_tu) h \diff \QP
&= \int_{\U} \lim_{r \to \infty} \cdc^{\U}_r(T^{\U, \QP}_tu) h \diff \QP 
\\
&=  \lim_{r \to \infty} \int_{\U} \cdc^{\U}_r(T^{\U, \QP}_tu) h \diff \QP 
\\
& \le   \limsup_{r \to \infty} \liminf_{r' \to \infty} \int_{\U} \cdc^{\U}_{r}(T_{r', t}^{\U, \QP}u)   h \diff \QP 
\\
& \le  \limsup_{r' \to \infty} \int_{\U} \cdc^{\U}_{r'}(T_{r', t}^{\U, \QP}u)   h \diff \QP 
\\
& \le  \limsup_{r' \to \infty} \int_{\U} T_{r', t}^{\U, \QP} \cdc^{\U}_{r'}(u)   h \diff \QP 
%\\
%&\le \liminf_{r' \to \infty} \cdc^{\U}_{r'}(T_{r', t}^{\U, \QP}u)^{1/2}  
%\\
%&\le \liminf_{r' \to \infty} T_{r', t}^{\U, \QP} \cdc^{\U}_{r'}(u)^{1/2}  
\\
&= \int_{\U}  T^{\U, \QP}_t\cdc^{\U}(u) h \diff \QP \fstop
\end{align*}
%\begin{align*}
%\cdc^{\U}(T^{\U, \QP}_tu)^{1/2} = \lim_{r \to \infty} \cdc^{\U}_r(T^{\U, \QP}_tu)^{1/2} 
%& \le   \lim_{r \to \infty}\liminf_{r' \to \infty} \cdc^{\U}_{r}(T_{r', t}^{\U, \QP}u)^{1/2}  
%\\
%&\le \liminf_{r' \to \infty} \cdc^{\U}_{r'}(T_{r', t}^{\U, \QP}u)^{1/2}  
%\\
%&\le \liminf_{r' \to \infty} T_{r', t}^{\U, \QP} \cdc^{\U}_{r'}(u)^{1/2}  
%\\
%&= T^{\U, \QP}_t\cdc^{\U}(u)^{1/2} \fstop
%\end{align*}
The last equality in the above displayed formulas followed by the $L^1$-contraction property of $T_{r', t}^{\U, \QP}$ and the second statement of Prop.\ \ref{prop: MGS}:
\begin{align*}
&\bigl\|T_{r', t}^{\U, \QP} \cdc^{\U}_{r'}(u) - T^{\U, \QP}_t \cdc^{\U}(u) \bigr\|_{L^1(\QP)} 
\\
&= \bigl\|T_{r', t}^{\U, \QP} \cdc^{\U}_{r'}(u) - T^{\U, \QP}_{r', t} \cdc^{\U}(u) \bigr\|_{L^1(\QP)} + \bigl\|T^{\U, \QP}_{r', t} \cdc^{\U}(u) - T^{\U, \QP}_t \cdc^{\U}(u)\bigr\|_{L^1(\QP)} 
\\
& \le \bigl\| \cdc^{\U}_{r'}(u) -  \cdc^{\U}(u)\bigr\|_{L^1(\QP)} + \bigl\|T^{\U, \QP}_{r', t} \cdc^{\U}(u) - T^{\U, \QP}_t \Gamma^{\U}(u)\bigr\|_{L^1(\QP)} 
\xrightarrow{r' \to \infty} 0 \fstop
\end{align*}
We verified \eqref{e:BEG}, the proof is completed.
\end{proof}

\subsection{Integral Bochner, local Poicar\'e and local log-Sobolev inequalities} \label{sec:AP1}
As an application of $\BE(0,\infty)$ proven in Thm.~\ref{t: main}, we show several functional inequalities.
We define {\it the integral $\mathbf \Gamma_2$-operator} as follows:
\begin{align} \label{d:IG}
&\mathbf \Gamma_2^{\U, \QP}(u, \phi):=\int_{\U}\biggl( \frac{1}{2}\cdc^{\U}(u)A^{\U, \QP} \phi - \cdc^{\U}(u, A^{\U, \QP}u) \phi\biggr) \diff \QP \comma
\\
&\dom{\mathbf \Gamma_2^{\U, \QP}}:=\bigl\{(u, \phi): \dom{A^{\U, \QP}}^{\times 2}: A^{\U, \QP}u \in \dom{\E^{\U, \QP}},\ \phi, A^{\U, \QP}u  \in L^\infty(\QP) \bigr\} \comma \notag
\end{align}
where $A^{\U, \QP}$ denotes the $L^2(\QP)$-infinitesimal generator associated with the form~$(\E^{\U, \QP}, \dom{\E^{\U, \QP}})$.
\begin{cor}\label{t:LPS}
Let $\mu$ be the $\sine_\beta$ ensemble with $\beta>0$.  The following  hold:
\begin{enumerate}[{\rm (a)}]
\item$(${\bf lntegral Bochner inequality}$)$ for every $(u, \phi) \in \dom{\mathbf \cdc^{\U, \QP}_2}$
\begin{align*}
\mathbf \cdc^{\U, \QP}_2(u, \phi) \ge 0 \ ;
\end{align*}
\item $(${\bf local Poincar\'e inequality}$)$  for $u \in \dom{\E^{\U, \QP}}$ and $t >0$,
\begin{align*}
&T^{\U, \QP}_tu^2- (T^{\U, \QP}_tu)^2 \le 2tT^{\U, \QP}_t\cdc^{\U}(u)  \comma
\\
&T^{\U, \QP}_tu^2- (T^{\U, \QP}_tu)^2 \ge 2t\cdc^{\U} (T^{\U, \QP}_tu) \ ;
\end{align*}

\item $(${\bf  local logarithmic Sobolev inequality}$)$ for non-negative $u \in \dom{\E^{\U, \QP}}$ and $t>0$,
\begin{align*}
&T^{\U, \QP}_tu\log u- T^{\U, \QP}_tu\log T^{\U, \QP}_t u \le tT^{\U, \QP}_t\biggl( \frac{\cdc^{\U}(u)}{u} \biggr) \comma
\\
&T^{\U, \QP}_tu\log u- T^{\U, \QP}_tu\log T^{\U, \QP}_t u \ge t \frac{\cdc^{\U}(T^{\U, \QP}_t u)}{T^{\U, \QP}_t u}  \fstop
\end{align*}
\item $(${\bf Local hyper-contractivity}$)$ for all $t>0$, $0<s\le t$ and $1<p<q<\infty$ with $\frac{q-1}{p-1}=\frac{t}{s}$, the following holds:
$$\Bigl( T_s^{\U, \QP}(T^{\U, \QP}_{t-s}u)^{q}\Bigr)^{1/q} \le \Bigl( T_t^{\U, \QP}u^p\Bigr)^{1/p} \comma\quad f \ge 0 \fstop$$
\end{enumerate}
\end{cor}
\begin{proof}
The statement~(a) follows from~$\BE(0,\infty)$ proven in Thm.~\ref{t: main} and \cite[Cor.~2.3]{AmbGigSav15}. The statement (b), (c) and (d)  are consequences of $\BE(0,\infty)$, see e.g.,~\cite[Thm.s 4.7.2, 5.5.2, 5.5.5]{BakGenLed14}. 
%Alternatively, we can prove these statements also in the following approximation and superposition steps: (i) the corresponding inequalities hold true for $\E^{k, \eta}_{r, R}$ on $\U^k(B_r)$ for any $k$ and $\QP$-a.e.~$\eta$ by Prop.~\ref{p:BE2}; (ii) we can show the corresponding inequalities for the superposition form $\E_r$ by making use of \eqref{eq:p:MarginalWP:0}, Prop.~\ref{prop: 1-1}, Cor.~\ref{prop: 1}; (iii) we approximate $\E$ and $T_t$ by $\E_r$ and $T_t^r$ respectively by the monotonicity in view of Prop.~\ref{p:mono} and Prop.~\ref{prop: MGS}, which concludes (a) and (b) in Thm.~\ref{t: main3}. 
\end{proof}

\begin{rem}[Spectral gap inequality]\label{r:LSG}
Let $P^{\U, \QP}_t(\gamma, \diff\eta)$ denote the heat kernel measure, which is defined as (see, e.g., \cite[(1.2.4) in p.12]{BakGenLed14})
$$T_t^{\U, \QP}u(\gamma)=\int_{\U} u(\eta) P^{\U, \QP}_t(\gamma, \diff\eta) \qquad \text{$\QP$-a.e.~$\gamma$} \fstop$$
The local Poincar\'e inequality in Cor.~\ref{t:LPS} is the spectral gap inequality {\it with respect to the heat kernel measure $P^{\U, \QP}_t(\gamma, \diff\eta)$}:
$$
\int_{\U} |u(\eta)-\QP(u)|^2P^{\U, \QP}_t(\gamma, \diff\eta) \le  2t\int_{\U} \cdc^{\U, \QP}(u)(\eta) P^{\U, \QP}_t(\gamma, \diff\eta)  \fstop 
$$
The name ``local'' stems from the observation that $P_t^{\U, \QP}(\gamma, \diff \eta)$ is expected to be {\it concentrated around $\gamma$} when $t$ is small (see \cite[\S 4.7 in p.~206]{BakGenLed14}). 
\end{rem}

The following corollary provides a non-trivial tail estimate of the heat kernel measure~$P_t^{\U, \QP}(\gamma, \diff \eta)$, which decays sufficiently fast at the tail to make every (not necessarily bounded) $1$-Lipschitz function exponentially integrable. 
\begin{cor}[Exponential integrability of $1$-Lipschitz functions] \label{c:TES}
Let $\mu$ be the $\sine_\beta$ ensemble with $\beta>0$. 
 If $u$ is a $\bar{\mssd}_\U$-Lipschitz function with $\Lip_{\bar{\mssd}_\U}(u) \le 1$ and $|u(\gamma)|<\infty$ $\QP$-a.e.~$\gamma$, then for every $s<\sqrt{2/t}$
$$\int_{\U} e^{s u(\eta)} P_t^{\U, \QP}(\gamma, \diff \eta)<\infty \fstop$$
\end{cor}
\begin{proof}
By the Rademacher-type theorem with respect to $\bar{\mssd}_\U$ proven in Prop.~\ref{p:DF} and the local Poincar\'e inequality in Cor.~\ref{t:LPS}, the same proof works as in \cite[Prop.~4.4.2]{BakGenLed14}.
\end{proof}
%\subsection{Application II: $L^\infty$-to-$\Lip(\mssd_\U)$ regularisation}
%\purple{As the second application, we show a Lipschitz regularisation property by the action of the semigroup~$\sem{T_t^{\U, \QP}}$.\footnote{SL relies on the quasi-regularity in the current arXiv version. So the following corollary is incomplete as we do not know if the quasi-regularity holds in our form.}\footnote{We can prove this statement in the same way in Thm 5.1 without using SL}
%%, which leads to the Markov uniqueness on the Lipschitz algebra with respect to the $L^2$-transportation distance~$\mssd_\U$. 
%\begin{cor} \label{t:LL}
%Let $\QP$ be the $\sine_2$ ensemble.
%The semigroup $T_t^{\U, \QP}$ satisfies $L^\infty(\U, \QP)$-to-$\Lip(\U, \mssd_\U)$ regularisation, i.e., for any~$u \in \Lip(\mssd_\U, \QP)$ and any~$t>0$, 
%\begin{align*}
%\text{$T_t^{\U, \QP}u$ has a $\mssd_\U$-Lipschitz $\QP$-modification~$\tilde{T}_t^{\U, \QP}u$ }
%\end{align*}
%and the following estimate holds:
%\begin{align*}% \label{m:LL}
%%&T_t^{\U, \QP} L^\infty(\U, \QP) \subset \Lip(\U, \mssd_\U) \quad \forall t>0 \comma
%%\\
%\Lip_{\mssd_\U}(\tilde{T}_t^{\U, \QP} u) \le \frac{1}{\sqrt{2} t} \|u\|_{L^\infty(\QP)}  \fstop
%\end{align*}
%\begin{proof}
%As the form~$(\E^{\U, \QP}, \mathcal D(\E^{\U, \QP}))$ satisfies~$\BE_1(0,\infty)$ by~Thm.~\ref{t: main} and the Sobolev-to-Lipschitz property by~\cite[Thm.~4.1]{Suz22}, the same proof as in~\cite[Thm.~6.5]{AmbGigSav14b} applies and there exists~$\Omega \subset \U$ with~$\QP(\Omega)=1$ so that
%\begin{align} \label{e:LILP}
%|T_t^{\U, \QP}(\gamma)- T_t^{\U, \QP}(\eta)| \le \frac{1}{\sqrt{2} t} \|u\|_{L^\infty(\QP)}\mssd_\U(\gamma, \eta) \quad \forall \gamma, \eta \in \Omega \fstop
%\end{align}
%The extension of~\eqref{e:LILP} to the whole~$\gamma, \eta \in \U$ follows by the McShane extension Theorem for extended metric spaces, see~\cite[Lem.~2.1]{LzDSSuz20}. The proof is complete.
%\end{proof}
%\end{cor}
%}
%\begin{cor}\label{c:MU}
%\purple{Let $\QP$ be the $\sine_2$ ensemble. 
%The form $\bigl(\E^{\U, \QP}, \Lip_b(\mssd_\U, \mu)\bigr)$ is Markov unique.}
%\end{cor}
%\begin{proof}
%Thanks to $L^\infty(\QP)$-to-$\Lip(\mssd_\U)$ regularisation verified in Cor.~\ref{t:LL}, the action by~$T_t^{\U, \QP}$ preserves $\QP$-classes $\Lip(\mssd_\U, \mu)_{\QP}$ of the Lipschitz algebra~$\Lip(\mssd_\U, \mu)$:
%$$T^{\U, \QP}_t \Lip_b(\mssd_\U, \QP)_{\QP} \subset \Lip_b(\mssd_\U,  \QP)_{\QP} \fstop$$
%We thus conclude the statement by~Lem.~\ref{l:MU}.
%%\purple{Just use $L^\infty$-to-$\Lip$ and explain the density of $\Lip_b \subset \dom{Q}$ by $T_t \Lip_b \subset \Lip_b$.}
%%\purple{Use Lem.~\ref{l:MU}.}
%\end{proof}

%\section{Coincidence of $\E^{\U, \QP}$ with $\Ch_{\mssd_\U, \mu}$} \label{sec:C=E}
%In this section, we prove that the Dirichlet form $(\E^{\U, \QP}, \dom{\E^{\U, \QP}})$ constructed in Prop.\ \ref{p:DF} coincides with the Cheeger energy $(\Ch_{\mssd_\U, \mu}, W^{1,2}(\U, \mssd_\U, \mu))$ associated with the $L^2$-transport distance $\mssd_\U$ and the $\sine_2$ measure $\mu$. 
%%The one direction $\E^{\U, \QP} \le \Ch_{\mssd_\U, \mu}$ has been proven \purple{by \cite{LzDSSuz21}} in a more general setting. We only need to address the other direction $\E^{\U, \QP} \ge \Ch_{\mssd_\U, \mu}$. 
%
%The key properties are threehold: (i) $\BE_1(0,\infty)$ proven in Thm.\ \ref{t: main}; (ii) Sobolev-to-Lipschitz property proven in \cite[Thm.\ 4.1]{Suz22}; (iii) the notion of $\tau_{\mssd_{\U}}$-upper regularity of $(\E^{\U, \QP}, \dom{\E^{\U, \QP}})$, which is a weaker notion than the $\tau$-upper regularity introduced in \cite{AmbGigSav15} and \cite{AmbErbSav16}. 
%
%In the following section, we first discuss a general characterisation of the equivalence of energies in terms of the $\tau_{\mssd_{\U}}$-upper regularity in the generality of extended metric finite-measure spaces. Secondly we apply the general result to the configuration space to obtain $\E^{\U, \QP}=\Ch_{\mssd_\U, \mu}$. 
%
%\subsection{$\tau_{\mssd}$-upper regularity in extended metric measure spaces}
%We start from the definition of $\tau_{\mssd}$-upper regularity for a Dirichlet form $(Q, \dom{Q})$ with square field operator $\Gamma^Q$. 
%\begin{defs}[Energy measure space {\cite{AmbErbSav16}}]
%We say that $(X,  \nu, Q, \dom{Q}, \mathcal L)$ is {an energy finite-measure space} if the following conditions hold:
%\begin{enumerate}[$(a)$]
%\item $(X, \nu)$ is a finite measure space;
%\item $(Q, \dom{Q})$ is strongly local Markovian symmetric form in $L^2(X, \nu)$;
%\item $(Q, \dom{Q})$ admits a square field operator $\cdc^Q$;
%\item there is an algebra $\mathcal L$ of pointwisely defined functions satisfying that 
%\begin{align} \label{d:GS}
%&\mathcal L \subset \{u: X \to \R:\ \text{$\nu$-measurable and bounded, $\cdc^Q(u) \le 1$}\}
%\\
%&\text{$\mathcal L$ separates points in $X$} \fstop \notag
%\end{align}
% The coarsest topology generated by $\mathcal L$ is denoted by $\tau_\mathcal L$. The topology $\tau_\mathcal L$ is completely regular and Hausdorff by construction; 
% \item the measure $\nu$ restricted on the Borel $\sigma$-algebra $\mathcal B(\tau_{\mathcal L})$ is Radon and 
% $${\rm supp}[\nu]=X \fstop$$
%\end{enumerate}
%\end{defs}
%\begin{defs}[$\tau_{\mssd}$-upper regularity] \label{d:TUR}
%Let  $(X, \nu, Q, \dom{Q}, \mathcal L)$ be an energy finite-measure space and $\mssd: X^{\times 2} \to \R_+$ be a $\nu^{\otimes 2}$-measurable extended distance. 
%We say that $(Q, \dom{Q})$ possesses {\it $\tau_{\mssd}$-upper regularity} if for every $u$ in a dense subset in $\dom{Q}$ there exists $u_n \in\dom{Q} \cap C_b(\U, \tau_{\mssd})$ converging strongly to $u$ in $L^2(\mu)$ and $v_n: \U \to \R$ bounded and $\tau_{\mssd}$-upper semi-continuous so that 
%\begin{align} \label{e:UR}
%\cdc^{Q}(u_n)^{\frac{1}{2}} \le v_n\comma \quad \limsup_{n \to \infty} \int_{X} v_n^2 \diff\nu \le Q(u) \fstop
%\end{align}
%\end{defs}
%\begin{rem}[Comparison with $\tau$-upper regularity] \label{r:com}
%In \cite[Def.\ 3.13]{AmbGigSav15} and also in \cite[Def.\ 12.4]{AmbErbSav16}, the notion of $\tau$-upper regularity has been introduced, which replaces $\tau_\mssd$ in Def.\ \ref{d:TUR} with a pre-existing topology $\tau$ in the setting of (extended) metric measure spaces $(X, \tau, \mssd, \nu)$. As the topology $\tau$ can be coarser than the topology $\tau_\mssd$ generated by $\mssd$ in the setting of extended metric measure spaces, the corresponding spaces of continuous functions (as well as upper semi-continuous functions) have the inclusion $C(\tau) \subset C(\tau_{\mssd})$. Thus, the notion of $\tau_\mssd$-upper regularity is a weaker notion than $\tau$-upper regularity. Indeed, in the setting of the configuration space with $\tau_{\mssd_\U}$ and $\tau$ being the vague topology, the inclusion $C(\tau) \subset C(\tau_{\mssd})$ is strict. 
%\end{rem}
%We introduce several extended distances associated with three different algebras. 
%\begin{defs}[Extended distances associated with algebras] \label{d:EDA}
%Let $(X, \nu, Q, \dom{Q}, \mathcal L)$ be an energy finite-measure space equipped with $\nu^{\otimes}$-measurable extended distance~$\mssd$.
%%$(X, \tau, \mssd, \nu)$ be an extended topoloPolish extended metric finite-measure space having a  Dirichlet form $(Q, \dom{Q})$ with square field operator $\Gamma^Q$. 
%Set $L:=\{u \in \dom{Q}:  \cdc^Q(u) \le 1\}$, and define 
%$$L_{\tau_\mathcal L}:=L \cap C_b(\tau_\mathcal L)\comma L_{\tau_\mssd}:=L \cap C_b(\tau_\mssd) \comma$$
%and
%$$\Lip_1(\mssd):=\{u \in \Lip_b(\mssd): u\ \text{is $\nu$-measurable},\ \Lip_{\mssd}(u) \le 1\} \fstop$$
%For each algebra, define 
%$$\mssd_{\bullet}(x, y):=\sup\{|u(x)-u(y)|: u \in \bullet\} \comma \quad \bullet \in \{L_{\tau_\mathcal L}, L_{\tau_\mssd}, \Lip_1(\mssd)\} \fstop$$
%The extended distance $\mssd_{L_{\tau}}$ has been conventionally called {\it the intrinsic distance associated with $(Q, \dom{Q})$} and, following the convention in~\cite{AmbErbSav16}, we use the notation~$\mssd_{Q}$ instead of~$\mssd_{L_{\tau}}$ below. The space~$(X, \tau_\mathcal L, \mssd_Q, \nu)$ is an extended metric-topological space in the sense of \cite[Def.~4.1]{AmbErbSav16}. In particular, the topology $\tau_{\mssd_Q}$ is finer than $\tau_\mathcal L$, see~\cite[(4.3), (4.4)]{AmbErbSav16}. 
%% $\mssd_{L_{\tau_\mssd}}(x, y):=\sup\{|u(x)-u(y)|: u \in L_{\tau_\mssd}\}$ and $\mssd_{\overline{L}}(x, y):=\sup\{|u(x)-u(y)|: u \in \overline{L}\}$.
%\end{defs}
%%Let $\mssd_Q:=\sup\{u(x)-u(y): \cdc^Q(u) \le 1,\ u \in C_b(\tau)\}$ be the {\it intrinsic distance} associated with $(Q, \dom{Q})$.
%The following  property plays a key role for connecting Dirichlet forms and metric measure spaces.
%\begin{defs}[Sobolev-to-Lipschitz]
%Let  $(X, \nu, Q, \dom{Q}, \mathcal L)$ be an energy finite-measure space and $\mssd$ be a $\nu^{\otimes}$-measurable extended distance. 
%%Let $(X, \tau, \mssd, \mu)$ be a Polish extended metric finite-measure space having a strongly local symmetric Dirichlet form $(Q, \dom{Q})$ with square field operator $\Gamma^Q$.
%We say that  $(Q, \dom{Q})$ has {\it Sobolev-to-Lipschitz property with respect to $\mssd$} if every $u \in \dom{Q}$ with $\cdc^Q(u) \le 1$ has a $\mssd$-Lipschitz representative $\tilde{u}$ with $\Lip_{\mssd}(\tilde{u}) \le 1$. 
%\end{defs}
%As noted in Def.~\ref{d:EDA}, the space~$(X, \tau_\mathcal L, \mssd_Q, \nu)$ is an extended metric finite-measure space, therefore, the Cheeger energy $\Ch_{\mssd_{Q}, \nu}^a$ based on asymptotic slope (see Sec.~\ref{sec:Ch}) is well-defined 
%\begin{thm} \label{t:EEC}
%Let  $(X, \nu, Q, \dom{Q}, \mathcal L)$ be an energy finite-measure space and $\mssd$ be a $\nu^{\otimes}$-measurable extended distance. 
%%Let $(X, \tau, \mssd, \mu)$ be a Polish extended metric finite-measure space having a strongly local symmetric Dirichlet form $(Q, \dom{Q})$ with square field operator $\Gamma^Q$. 
%Assume that $(Q, \dom{Q})$ has the Sobolev-to-Lipschitz property with respect to $\mssd_{Q}$. Then, 
%\begin{align} \label{e:EEC}
%\text{$(Q, \mathcal D(Q))$ is $\tau_{\mssd_Q}$-upper regular} \ \iff \ (Q, \mathcal D(Q)) = (\Ch^a_{\mssd_Q, \nu}, W_a^{1,2}(\U, \mssd_Q, \nu)) \fstop
%\end{align}
%\end{thm}
%\begin{proof}
%The one inequality $Q \le \Ch^a_{\mssd_Q, \nu}$ has been proven in \cite[Thm.\ 12.5]{AmbErbSav16} without $\tau_{\mssd_Q}$-upper regularity nor the Sobolev-to-Lipschitz assumption.
%%\purple{Show that the Polish emmspace is emmspace in the sense of Savar\'e }. 
%In particular, this implies the following Rademacher-type property for $(Q, \dom{Q})$:
%\begin{align} \label{e:EEC-1}
%\Lip(\mssd_Q) \subset W_a^{1,2}(\U, \mssd_Q, \nu) \subset \dom{Q}\comma \quad \Gamma^Q(u)^{1/2} \le |\nabla u|_{w, \mssd_Q, \nu} \le \Lip_{\mssd_Q}(u)\fstop
%\end{align}
% What we need to prove is, therefore,  {\bf (i)} the implication from the LHS in \eqref{e:EEC} to the inequality $Q \ge \Ch^a_{\mssd_Q, \mu}$; {\bf (ii)} the implication from the RHS to the LHS in \eqref{e:EEC}. 
%% The proof strategy is essentially the same as \cite[Thm.\ 3.14]{AmbGigSav15}, which is, however, formulated in the setting of metric measure spaces (as opposed to {\it extended} metric measure spaces). For the sake of clarity, we explain below that their proof can be adapted to the setting of Polish extended metric measure spaces. 
% The proof of {\bf (ii)} is straightforward since the RHS of \eqref{e:EEC} implies the $\tau_\mathcal L$-upper regularity by \cite[Thm.\ 12.5]{AmbErbSav16} and the $\tau_{\mathcal L}$-upper regularity implies $\tau_{\mssd_Q}$-upper regularity since $\tau_{\mssd_Q}$ is finer than $\tau_\mathcal L$ as noted in Def.~\ref{d:EDA}.
% %noted in Rem.\ \ref{r:com}.
%
%We verify {\bf (i)}. We prepare two lemmas. %\purple{Present below more systematically.}
%%Let $L:=\{u \in \dom{Q} \cap L^\infty(\nu): \cdc^Q(u) \le 1\}$, $L_{\tau_Q}:=\{u \in \dom{Q} \cap C_b(\tau_{\mssd_Q}): \cdc^Q(u) \le 1\}$ and $\overline{L}:=\{u \in \Lip_b(\mssd_Q): u\ \text{$\nu$-measurable},\ \Lip_{\mssd_Q}(u) \le 1\}.$
%%Define $$\mssd_{L_{\tau_Q}}(x, y):=\sup\{u(x)-u(y): u \in L_{\mssd_Q}\}\comma  \quad {\mssd}_{\Lip}(x, y):=\sup\{u(x)-u(y): u \in \overline{L}\} \fstop$$
%\begin{lem} \label{l: CID}
%${\mssd}_{Q}=\mssd_{L_{\tau_{\mssd_Q}}}=\mssd_{\Lip_1(\mssd_Q)}$.
%\end{lem}
%\begin{proof}
%By the Sobolev-to-Lipschitz property and \eqref{e:EEC-1}, we have that the following spaces are isomorphic 
%\begin{align} \label{q:ILL}
%L \cong L_{\tau_{\mssd_Q}} \cong \Lip_1(\mssd_Q)\comma
%\end{align}
%where the isomorphism is in the sense of preserving complete lattice structure induced by the pointwise maximum~$\vee$ and minimum~$\wedge$ as well as the Banach norms equipped respectively with 
%$$\bigl(L, \|\cdot\|_\infty+\|\cdc^Q(\cdot)\|_\infty \bigr) \comma \quad \bigl( L_{\tau_{\mssd_Q}}, \|\cdot\|_\infty+\|\cdc^Q(\cdot)\|_\infty \bigr) \comma \quad \bigl(\Lip_1(\mssd_Q), \|\cdot\|_\infty+\Lip_{\mssd_Q}(\cdot) \bigr) \fstop $$
% On the one hand, by a priori inclusion $L_{\tau}:=L \cap C_b(\tau) \subset L \cong L_{\tau_{\mssd_Q}} \cong \Lip_1(\mssd_Q)$, and recalling $\mssd_{Q}:=\mssd_{L_{\tau}}$ as noted in Def.\ \ref{d:EDA}, we have that 
%$$\mssd_{Q} \le \mssd_{\Lip_1(\mssd_Q)}\comma \quad  \mssd_{Q} \le \mssd_{L_{\mssd_{\tau_Q}}} \fstop$$
%On the other hand, for any fixed $x, y \in X$, take a sequence $\{u_n\} \subset\Lip_1(\mssd_Q)$ so that ${\mssd}_{\Lip_1(\mssd_Q)}(x, y)=\lim_{n \to \infty} |u_n(x)-u_n(y)|$. By \eqref{q:ILL} and the definition of $\Lip_1(\mssd_Q)$, it holds that $|u_n(x)-u_n(y)| \le \mssd_{Q}(x, y)$, which concludes 
%$${\mssd}_{\Lip_1(\mssd_Q)}(x, y) \le  \mssd_{Q}(x, y) \fstop$$
%Therefore, ${\mssd}_{\Lip_1(\mssd_Q)} = \mssd_{Q}$. Since $L_{\tau_{\mssd_Q}} \cong \Lip_1(\mssd_Q)$, we also conclude $\mssd_{L_{\mssd_{\tau_Q}}} = \mssd_{Q}$.
%%By the same argument applied to $\mssd_{L_{\tau_{\mssd_{Q}}}}$, we conclude the same inequality $\mssd_{L_{\tau_{\mssd_{Q}}}}(x, y) \le  \mssd_{Q}(x, y) \fstop$
%%Thus, we obtained the conclusion.
%\end{proof}
% The following lemma is a modification of \cite[Thm.\ 12.5]{AmbErbSav16} to the case of $\tau_{\mssd_Q}$-upper regularity.
%\begin{lem}\label{l:LSL}
%Let $ u\in \dom{Q} \cap \Lip_b(\mssd_Q)$, and $v: X \to [0,\infty)$ be a bounded $\tau_{\mssd_Q}$-upper semi-continuous function so that $\Gamma^Q(u) \le v^2$ $\nu$-a.e.. Then, $|\mathsf{D}_{\mssd_Q} u| \le v$. 
%\end{lem}
%\begin{proof}
%Thanks to the $\tau_{\mssd_Q}$-upper semi-continuity of $v$, it suffices to show $|\mathsf{D}_{\mssd_Q} \tilde{u}| \le c$ on a $\tau_{\mssd_Q}$-open set $U=\{v <c\}$ for any $c \in \R$. 
%We may assume $c=1$ without loss of generality by the homogeneity. Fix $x_0 \in U$. As $\tau_{\mssd_Q}$ is generated by $ \Lip_1(\mssd_Q)$, there exists a finite family of functions $\{u_i\}_{i=1}^{n} \subset  \Lip_1(\mssd_Q)$ and $r>0$ so that 
%$$V:=\Bigl\{x \in X: \max_{1 \le i \le n}\{|f_i(x)-f(x_0)| \le r\} \Bigr\} \subset U \fstop$$
%Define 
%$$\delta(x):=\max_{1 \le i \le n}\{|f_i(x)-f(x_0)|\comma \quad l(x):=r \wedge |u(x)-u(x_0)| \comma \quad h(x):=\delta(x) \vee l(x) \fstop$$
%Noting $\{h=l\} = \{l \ge \delta\} \subset F \subset U$ and $\cdc^Q(l) \le 1$ by \eqref{e:EEC-1}, thanks to the locality of $\cdc^Q$ we obtain $\cdc^Q(h) \le 1$. As $h$ is $\tau_{\mssd_Q}$-continuous, we have $h \in L_{\tau_{\mssd_Q}}$. By Lem.\ \ref{l: CID}, we have 
%$$h(x)=h(x)-h(x_0) \le\mssd_{L_{\tau_Q}}(x, x_0) = {\mssd}_{Q}(x, x_0) \fstop$$ 
%As $u$ is $\tau_{\mssd_Q}$-continuous, $h(x) \ge |u(x)-u(x_0)|$ when $\mssd_Q(x, x_0)$ is sufficiently small, which concludes $|\mathsf{D}_{\mssd_Q}u|(x_0) \le 1$.
%%\end{proof}
%%As $v$ is bounded, $\Gamma^Q(u)$ is bounded as well. Thus, by the Sobolev-to-Lipschitz property, there exists a $\mssd_Q$-Lipschitz representative $\tilde{u}$ of $u$. Fix $x \in X$ and for $\e>0$, define $V_{x, \e}:=\sup_{B_{3\e}(x)} v$, where $B_r$ is the metric ball with respect to $\mssd_{Q}$. Then, the following function is $\mssd_{Q}$-Lipschitz:
%%$$\psi_{\e, x}(\cdot):=\Bigl(|\tilde{u}(x)-\tilde{u}(\cdot)| \vee V_{\e,x} \mssd_{Q}(x,\cdot)\Bigr) \wedge V_{\e,x} S_\e(\mssd_Q(x, \cdot))\fstop$$  
%%%\purple{Show the same statement without using the square field, but only using the slope and the chain rule for the slope (AGS 14 Inv, p327 the proof of (4.15)), which is needed for the case of extended mmspace since the $\tau_Q$ continuity does not trivialise the modification.}
%%
%%{\Large Show $\mssd_Q=\sup\{f(x)-f(y): \Gamma^Q(f) \le 1,\ f \in C_b(\purple{\tau_{\mssd_{Q}}})\}$ by Sobolev-to-Lipschitz and use the same proof as in AES. Note that the RHS is a priori large than $\mssd_Q$ since we take $C_b(\tau_{\mssd_Q})$ in place of $C_b(\tau)$}
%%
%%Thus, by \eqref{e:EEC-1}, $\psi_{\e, x}(\cdot) \in \dom{Q} \cap L^\infty(\nu)$ and $\cdc^Q(\psi_{\e,x})^{1/2} \le \max\{v, V_{\e,x}\}$ as $|S'_\e|\le 1$. 
%%Furthermore, $\psi_{\e,x}(y)=0$ whenever $\mssd_Q(x, y) \ge 3\e$. Thus, $\cdc^Q(\psi_\e)^{1/2} \le V_{\e,x}$ $\nu$-a.e.. Thus, by applying the Sobolev-to-Lipschitz property again, there exists a measurable set $\Omega_{\e, x} \subset X$ with $\nu(\Omega_{\e, x})=1$ so that $\psi_{\e,x}$ is $V_\e$-Lipschitz on $\Omega_{\e, x}$. 
%\end{proof}
% 
% We now complete the proof of {\bf (i)}. Let $u \in \dom{Q}$, and $u_n \in \dom{Q} \cap C_b(\tau_{\mssd_Q})$ and $v_n$ be sequences as in Def.\ \ref{d:TUR}. 
% Then, by Lem.\ \ref{l:LSL}, 
% it holds $|\mathsf{D}_{\mssd_{Q}} u_n| \le v_n$. Thus, noting the $L^2$-lower semi-continuity of $ \Ch^a_{\mssd_Q, \nu}$, 
% $$ \Ch^a_{\mssd_Q, \nu}(u) \le  \liminf_{n \to \infty}\Ch^a_{\mssd_Q, \nu}(u_n) \le  \liminf_{n \to \infty}\int_X |\mathsf{D}_{\mssd_{Q}} u_n|^2 \diff \nu \le  \limsup_{n \to \infty}\int_X v_n^2 \diff \nu \le Q(u) \comma$$
% which concludes {\bf (i)}. The proof of Theorem \ref{t:EEC} has been completed. 
%\end{proof}
%
%As a corollary of Thm.~\ref{t:EEC}, we obtain the equivalence of $\tau_{\mssd_Q}$-upper regularity and $\tau$-upper regularity under the Sobolev-to-Lipschitz property with respect to $\mssd_Q$:
%\begin{cor}
%Let  $(X, \nu, Q, \dom{Q}, \mathcal L)$ be an energy finite-measure space and $\mssd$ be a $\nu^{\otimes}$-measurable extended distance. 
%%Let $(X, \tau, \nu)$ be a Polish space with a finite Borel measure $\nu$ having a Dirichlet form $(Q, \dom{Q})$ with square field $\Gamma^Q$ and assume that $(X, \tau, \mssd_{Q}, \nu)$ is a Polish extended metric finite-measure space. 
%%Let $(X, \tau, \mssd, \mu)$ be a Polish extended metric finite-measure space having a strongly local symmetric Dirichlet form $(Q, \dom{Q})$ with square field operator $\Gamma^Q$. 
%Under the assumption that $(Q, \dom{Q})$ has the Sobolev-to-Lipschitz property with respect to $\mssd_{Q}$, 
%\begin{align} \label{e:EEC2}
%\text{$(Q, \mathcal D(Q))$ is $\tau_\mathcal L$-upper regular} \ \iff \ \text{$(Q, \mathcal D(Q))$ is $\tau_{\mssd_Q}$-upper regular} \fstop
%\end{align}
%\end{cor}
%\begin{proof}
%The implication from the LHS to the RHS follows as $\tau_{Q}$ is finer than $\tau$. The opposite implication is proved in the following steps: By Thm.~\ref{t:EEC}, the RHS implies the identity~\eqref{e:EEC-BE}, which further implies the $\tau$-upper regularity by \cite[Thm.\ 12.5]{AmbErbSav16}. The proof is complete. 
%\end{proof}
%The $\tau_{\mssd_Q}$-upper regularity can be verified under $\BE(K,\infty)$ and the Sobolev-to-Lipschitz property. 
%\begin{cor} \label{t:EEC-BE}
%Let  $(X, \nu, Q, \dom{Q}, \mathcal L)$ be an energy finite-measure space and $\mssd$ be a $\nu^{\otimes}$-measurable extended distance. 
%%Let $(X, \tau, \nu)$ be a Polish space with a finite Borel measure $\nu$ having a Dirichlet form $(Q, \dom{Q})$ with square field $\Gamma^Q$ and assume that $(X, \tau, \mssd_{Q}, \nu)$ is a Polish extended metric finite-measure space. 
%%Let $(X, \tau, \mssd, \mu)$ be a Polish extended metric finite-measure space having a strongly local symmetric Dirichlet form $(Q, \dom{Q})$ with square field operator $\Gamma^Q$. 
%Assume that $(Q, \dom{Q})$ satisfies $\BE(K,\infty)$ and has the Sobolev-to-Lipschitz property with respect to $\mssd_{Q}$. Then, $(Q, \dom{Q})$ is $\tau_{\mssd_Q}$-upper regular. In particular, 
%\begin{align} \label{e:EEC-BE}
% (Q, \mathcal D(Q)) = (\Ch^a_{\mssd_Q, \nu}, W_a^{1,2}(\U, \mssd_Q, \nu)) \fstop
%\end{align}
%\end{cor}
%\begin{proof}
%We take the subspace $L_\infty:=\{u \in \dom{Q}: \cdc^Q(u) \in L^\infty(\nu)\}$ as a dense subset in the statement in Def.~\ref{d:TUR}. We first show that $L_\infty$ is dense in $\dom{Q}$.
%Under the hypothesis,  the same proof as in \cite[Thm.~6.5]{AmbGigSav14b} can be applied to obtain the $L^\infty$-to-$\Lip(\mssd_Q)$ regularisation property of the semigroup~$\sem{T_t}$:
%$$T_t L^\infty(\nu) \subset \Lip(\mssd_Q) \fstop$$
%In particular,  $L_\infty$ is dense in $L^2(\nu)$ thanks to the following arguments: (i) $T_t L^\infty(\nu) \subset  \Lip(\mssd_Q) \cong L_\infty$, where the latter equivalence is the Sobolev-to-Lipschitz property; (ii) $\sem{T_t}$ is strongly continuous at $t=0$, i.e., $T_tu$ converges strongly to $u \in L^2(\nu)$ as $t \to 0$; (iii) $L^\infty(\nu)$ is dense in $L^2(\nu)$ by general theory. 
%
% Furthermore, by $\BE_1(K,\infty)$,  the subspace $L_\infty$ is dense in $\dom{Q}$ since $T_t L_\infty \subset L_\infty$ by the $L^\infty$-contraction property of $T_t$ and $\BE(K,\infty)$ gradient estimate:
%\begin{align} \label{e:BE22}
%\Gamma^Q(T_tu)^{1/2} \le e^{-K/2t}T_t\cdc^Q(u)^{1/2} \in L^\infty(\nu) \fstop
%\end{align}
%By Lem.~\ref{l:MU}, therefore, we conclude that $L_\infty$ is dense in $\dom{Q}$.
%
%We second show the $\tau_{\mssd_Q}$-upper regularity of $(Q, \dom{Q})$. By the $L^\infty(\nu)$-to-$\Lip(\mssd)$ regularisation of $\sem{T_t}$, it holds that 
%\begin{align} \label{e:BE222}
%\text{$T_t\cdc^Q(u)^{1/2}$ is $\tau_{\mssd_Q}$-continuous (in particular, $\tau_{\mssd_Q}$-upper semi-continuous). }
%\end{align}
%%As $\sem{T_t}$ is strongly continuous at $t=0$, we see that $T_tu$ converges to $u$ in $L^2(\nu)$ as $t \to 0$. 
%Take a sequence $\{t_n\}$ with $t_n \to 0$ as $n \to \infty$. Noting again the strong continuity of $\sem{T_t}$ at $t=0$, \eqref{e:BE22} and \eqref{e:BE222},  the sequence $u_n:=T_{t_n} u$ and $v_n:=\cdc^Q(T_{t_n} u)$ witness the $\tau_{\mssd_Q}$-upper regularity. The latter statement~\eqref{e:EEC-BE} now follows from Thm.~\ref{t:EEC}. The proof is completed. 
%\end{proof}
%\subsection{Application to the configuration space}
%In this subsection, we apply a general result proven in Thm.\ \ref{t:EEC-BE} to the configuration space $(\U, \tau_\mrmv, \mssd_\U, \mu)$ equipped with the Dirichlet form~$(\E^{\U, \QP}, \dom{\E^{\U, \QP}})$ constructed in Prop.~\ref{p:DF}. 
%\begin{thm}\label{t:E=C}
%Let $\QP$ be the $\sine_2$ ensemble. Then
%$$(\E^{\U, \QP}, \mathcal D(\E^{\U, \QP})) = (\Ch_{\mssd_\U, \mu}, W^{1,2}(\U, \mssd_\U, \mu)) \fstop$$
%\end{thm}
%\begin{proof}
%The quadruplet~$(\U, \tau_v, \mssd_\U, \mu)$ is a Polish extended metric measure space.\footnote{We probably need to mention Savar\'e's definition as well} As $(\E^{\U, \QP}, \dom{(\E^{\U, \QP}})$ satisfies the Rademacher-type property with respect to~$\mssd_\U$ by~Prop.~\ref{p:DF} and the Sobolev-to-Lipschitz property with respect to~$\mssd_\U$ by \cite[Thm.~4.1]{Suz22}, the intrinsic distance~$\mssd_{\E^{\U, \QP}}$ coincides with the $L^2$-transportation distance~$\mssd_\U$, see \cite[Lem.~3.6, Prop.~4.2]{LzDSSuz20}:
%$$\mssd_{\E^{\U, \QP}}=\mssd_\U\fstop$$
%Combining it with $\BE_1(0,\infty)$ proven in Thm.~\ref{t: main}, we can apply Cor.~\ref{t:EEC-BE} to conclude the statement. 
%%\purple{Mention that $(\U, \tau_v, \mssd_\U, \mu)$ is a Polish extended mmspace by the first paper. Use the previous corollary and (SL) from the number rigid papre and $\BE(0,\infty)$ from the previous section and $\mssd_{\E}=\mssd_\U$ by (dSL) + (Rad) from the first paper.}
%\end{proof}



%\purple{Show also the Markov uniqueness on Lipschitz algebra as a corollary.}
%\begin{defs}[$\tau_{\mssd_\U}$-upper regularity]
%We say that $(\E, \dom{\E})$ possesses {\it $\tau_{\mssd_\U}$-upper regularity} if for every $u$ in a dense subset in $\dom{\E}$ there exists $u_n \in\dom{\E} \cap C_b(\U, \tau_{\mssd_\U})$ converging strongly to $u$ in $L^2(\mu)$ and $v_n: \U \to \R$ bounded and $\tau_{\mssd_\U}$-upper semi-continuous so that 
%\begin{align} \label{e:UR}
%\cdc(u_n)^{1/2} \le v_n\comma \quad \limsup_{n \to \infty} \int_{X} v_n^2 \diff\mssm \le \E(u) \fstop
%\end{align}
%\end{defs}
% \newpage
% \begin{thm} \label{t:MUW}
%$(\E, \mathcal C)$ is Markov unique.
%\end{thm}
%\purple{We must use $\BE(0,\infty)$ for $\E_r$, therefore, this theorem must be located here.}
%\begin{proof}
%By the same argument as in the last paragraph of the proof of Thm.\ \ref{t:S=M}, we only need to show that $T_t \mathcal C \subset \mathcal C$. Let $u \in \mathcal C$ and we verify (a)--(c) in Def.\ \ref{d:core}.  By the $L^\infty$-contraction property $\|T_t u\|_\infty \le \|u\|_\infty $, we verified (a). 
%
%We verify (b). 
%\begin{lem} \label{l:CU}
%If $u_n$ converges to $u$ $\mu$-a.e.\, then $\mathcal U_{\gamma, x}(u_n)$ converges  to $\mathcal U_{\gamma, x}(u)$ pointwise for $\mu$-a.e.\ $\gamma$ and every $x \in X$. 
%\end{lem}
%\begin{proof}
%Let $\Omega \subset \U$ so that $\mu(\Omega)=1$ and $u_n(\gamma)$ converges to $u(\gamma)$ for every $\gamma \in \Omega$. Then, for any $\gamma \in \Omega$
%\begin{align*}
%\mathcal U_{\gamma, x}(u_n)(y):=u_n(\1_{\R \setminus \{x\}}\cdot\gamma +\delta_y)-u_n(\gamma \cdot\1_{\R \setminus \{x\}})
%\end{align*}
%\end{proof}
%
%By Cor.\ \ref{prop: 1} and the second paragraph in the proof of Thm.\ \ref{t:S=M}, it holds that $\mathcal U_{\gamma, x}(T^r_tu) \in \Lip(\R) \subset W_{loc}^{1,2}(\mssm_r)$. Furthermore, combining Lem.\ \ref{l:DT}, we obtain
%$$\Lip_{\R}(\mathcal U_{\gamma, x}(T^r_tu)) \le \Lip_{\mssd_{\U}}(T^r_tu) \le c(t)\|u\|_\infty$$
% for $\mu$-a.e.\ $\gamma, \eta$ and $r>0$ with constant $c(t)$ depending only on $t$. In particular, we have that the Sobolev norm is bounded: Fix $s$ and for any $r>s$, 
% $$\|\mathcal U_{\gamma, x}(T^r_tu)\|_{W^{1,2}(\mssm_s)} \le \|u\|_\infty+c(t)\|u\|_\infty \fstop$$ Thus, by Prop.\ \ref{prop: MGS} and Lem.\ \ref{l:CU}, we obtain that $\mathcal U_{\gamma, x}(T_tu) \in W^{1,2}(\mssm_s)$ for any $s>0$ for $\mu$-a.e.\ $\gamma$ and $x \in B_s$ and that $|\nabla \mathcal U_{\gamma, x}(T_tu) | \le c(t)\|u\|_\infty$. By the Sobolev-to-Lipschitz property for $W^{1,2}(\mssm_s)$, it holds that $\mathcal U_{\gamma, x}(T_tu)  \in \Lip(B_s)$ with Lipschitz constant bounded by $c(t)\|u\|_\infty$ for arbitrary $s>0$. Therefore, it implies $\mathcal U_{\gamma, x}(T_tu) \in \Lip(X)$ with Lipschitz constant bounded by $c(t)\|u\|_\infty$ and it concludes $\mathcal U_{\gamma, x}(T_tu)  \in W_{loc}^{1,2}(\mssm)$. The verification of (b) is completed. 
% 
% The verification of (c). By the energy contraction property $\E_r(T^r_t u)  \le \E_r(u)$ by  general theory and the monotonicity of $\E_r \uparrow$, we have 
% $$\E(T_t u) \le  \lim_{r' \to \infty}\liminf_{r \to \infty}\E_{r'}(T^r_t u) \le \liminf_{r \to \infty}\E_r(T^r_t u)  \le \liminf_{r \to \infty}\E_r(u) = \E(u)<\infty \fstop$$
% The proof has been completed.
%
%\end{proof}

\section{Dimension-free and log Harnack inequalities} \label{sec:LH}
%In this section, we prove Thm.\ \ref{t: main3}. For the sake of readers' convenience, we recall the statements of Thm.~\ref{t: main3} in the following.
In this section, we prove functional inequalities involving the Bakry--\'Emery curvature bound~$\BE(0,\infty)$ and the $L^2$-transportation-type extended distance~$\bar{\mssd}_\U$.
% as well as the number-rigidity~\ref{ass:Rig} and the tail-triviality~\ref{ass:TT} of the~$\sine_2$ ensemble. 
%\purple{Use the large distance $\bar{\mssd}_\U$ instead of $\mssd_\U$. }
%For the following statement, we call the quadruplet~$(\U, \cdc^{\U}, \mssd_\U, \QP)$ {\it metric diffusion space}. 
\begin{thm}\label{t:DFH}
Let $\QP$ be the $\sine_\beta$ ensemble with $\beta>0$. Then the following inequalities hold:
\begin{enumerate}[{\rm (a)}]
\item $(${\bf  log-Harnack inequality}$)$ for every non-negative $u \in L^\infty(\U, \QP)$ $\e \in (0,1]$, $t>0$, there exists $\Omega \subset \U$ so that $\QP(\Omega)=1$ and 
$$T^{\U, \QP}_t\log (u+\e)(\gamma) \le \log (T^{\U, \QP}_tu(\eta)+\e) + \bar{\mssd}_\U(\gamma, \eta)^2\comma \quad \text{every $\gamma, \eta \in \Omega$} \ ;$$
\item $(${\bf  dimension-free Harnack inequality}$)$ for every non-negative $u \in L^\infty(\U, \QP)$, $t>0$ and $\alpha>1$ there exists $\Omega \subset \U$ so that $\QP(\Omega)=1$ and 
$$(T^{\U, \QP}_tu)^\alpha(\gamma)\le T^{\U, \QP}_tu^\alpha(\eta) \exp\Bigl\{ \frac{\alpha}{2(\alpha-1)}\bar{\mssd}_\U(\gamma, \eta)^2\Bigr\} \comma \quad \text{for every $\gamma, \eta \in \Omega$} \ ;$$
\item $(${\bf  Lipschitz contraction}$)$ For $u \in \Lip_b(\bar{\mssd}_\U,  \QP)$ and $t>0$, 
\begin{align*}
\text{$T_t^{\U, \QP}u$ has a $\bar{\mssd}_\U$-Lipschitz $\QP$-modification~$\tilde{T}_t^{\U, \QP}u$ }
\end{align*}
and the following estimate holds:
$$\Lip_{\bar{\mssd}_\U}(\tilde{T}_t^{\U, \QP} u) \le \Lip_{\bar{\mssd}_\U}(u) \ ;$$
\item $(${\bf $L^\infty$-to-$\Lip$ regularisation}$)$ For ~$u \in L^\infty(\QP)$ and any~$t>0$, 
\begin{align*}
\text{$T_t^{\U, \QP}u$ has a $\bar{\mssd}_\U$-Lipschitz $\QP$-modification~$\tilde{T}_t^{\U, \QP}u$ }
\end{align*}
and the following estimate holds:
\begin{align*}% \label{m:LL}
%&T_t^{\U, \QP} L^\infty(\U, \QP) \subset \Lip(\U, \mssd_\U) \quad \forall t>0 \comma
%\\
\Lip_{\bar{\mssd}_\U}(\tilde{T}_t^{\U, \QP} u) \le \frac{1}{\sqrt{2 t}} \|u\|_{L^\infty(\QP)}  \fstop
\end{align*}
\end{enumerate}
\end{thm}
%Before giving the proofs, we provide a prototype lemma.
%\begin{lem} \label{l:DRT}
%Let $F, G: \U \to \R$ be $\QP$-measurable functions 
%\end{lem}
\begin{proof}
%various functional inequalities related to $\BE(0,\infty)$ curvature bound proven in Thm.\ \ref{t: main}. 
%\begin{thm} The following holds:
%\begin{enumerate}[{\rm (i)}]
%\item $X$ support {\it the log-Harnack inequality $\LH(K)$} if for every non-negative $u \in L^\infty(\nu)$, $t>0$, 
%$$H_t(\log u)(x) \le \log (H_tu)(y) + \frac{K \mssd(x, y)^2}{2(1-e^{-2Kt})}\comma \quad \text{$\nu^{\otimes 2}$-a.e.\ $x, y \in X$}\fstop$$
%
%\item  $X$ support {\it the dimension-free Harnack inequality $\DFH(K)$} if 
%$$(H_tu)^\alpha(y)\le H_tu^\alpha(x) \exp\Bigl\{ \frac{\alpha K d_X(x, y)^2}{2(\alpha-1)(1-e^{-2Kt})}\Bigr\} \comma \quad \text{$\nu^{\otimes 2}$-a.e.\ $x, y \in X$}\fstop$$
%\item We say that $X$ support {\it the local Poincar\'e inequality $\LP(K)$} if either one of the equivalent inequalities: for $u \in W^{1,2}(X, \mssd, \nu)$,
%\begin{align*}
%H_tu^2(x)- (H_tu)^2(x) \le \frac{1-e^{-2Kt}}{K}H_t|\nabla u|^2(x) \quad  \quad \text{$\nu$-a.e.\ $x\in X$} \comma
%\\
%H_tu^2(x)- (H_tu)^2(x) \ge \frac{e^{-2Kt}-1}{K}|\nabla H_t u|^2(x) \quad  \quad \text{$\nu$-a.e.\ $x \in X$} \fstop
%\end{align*}
%
%\item $X$ support {\it the local logarithmic Sobolev inequality $\LLS(K)$} if either one of the equivalent inequalities: $u \in W^{1,2}_+(X, \mssd, \nu)$,
%\begin{align*}
%H_tu\log u(x)- H_tu\log H_t u(x) \le \frac{1-e^{-2Kt}}{2K}H_t\frac{|\nabla u|^2}{u}(x) \quad  \quad \text{$\nu$-a.e.\ $x\in X$} \comma
%\\
%H_tu\log u(x)- H_tu\log H_t u(x) \ge \frac{e^{-2Kt}-1}{2K}\frac{|\nabla H_t u|^2}{H_tu}(x) \quad  \quad \text{$\nu$-a.e.\ $x\in X$}\comma
%\end{align*}
%\item we say that $X$ support {\it the Evolution Variational Inequality $\EVI(K,\infty)$} if for any $\sigma=\rho\cdot \mu_X \in \mathcal P_*(X)$, $\nu \in D(\ENT_{\mu_X})$ with $W_{2, d_X}(\sigma, \nu)<\infty$, it holds that $\Ent{\mu_X}{H_t \rho \cdot \mu_X} \le \Ent{\mu_X}{\rho \cdot \mu_X}<\infty$ for all $t>0$ and the following inequality holds:
%$$\frac{1}{2}\frac{d^+}{dt}W_{2, d_X}\bigl((H_t \rho) \cdot \mu_X, \nu \bigr)^2 + \frac{K}{2} W_{2, d_X}\bigl((H_t \rho) \cdot \mu_X, \nu\bigr)^2 \le \Ent{\mu_X}{\nu} - \Ent{\mu_X}{(H_t \rho) \cdot \mu_X}\comma$$
%where $\frac{d^+}{dt}$ stands for the upper right derivative in $t$. 
%\end{enumerate}
%\end{thm}

%Under the assumption that $(X, \mssd, \nu)$ is $\RCD(K',\infty)$ for some $K'<K$, then the following are equivalent (\cite{KopStu21}):
%Another equivalent characterisation of $\RCD(K, \infty)$ has been given by \cite[Thm.\ 5.1]{AmbGigSav14b}: 
%\end{defs}
%\subsection*{Functional inequalities}
%\begin{defs} \label{d:CFI}

%\item 
%We say that $X$ support {\it the HWI inequality $\HWI(K)$} if every probability density $\rho: X \to \R_+$ satisfies the following inequality:
%$$\Ent{\mu_X}{\rho\cdot \mu_X} \le W_{2, d_X}(\rho\cdot \mu_X, \mu_X)\Fish{\mu_X}{\rho}^{1/2}- \frac{K}{2}W_{2, d_X}(\rho \cdot \mu_X, \mu_X)^2 \comma $$
%where $\Fish{\mu_X}{\rho}$ is the Fisher information functional $\Fish{\mu_X}{\rho}:=\Ch(\sqrt{\rho})=\int_{X} \frac{|\nabla \rho|^2}{\rho} \diff \mu_X$.
%\section{Log-Harnack inquality}
%\section{Evolutional variation inequality}
%\begin{proof}[Proof of (c) and (d)]
%As the proofs of (a) and (b) are similar, we provide the full proof only for~(a). 
We prove (a).
By the relation between $T^{\U, \QP}_{r, t}$ and~$T^{\U(B_r), \QP_r^{\cdot}}_t(\cdot_{B_r})$ in~Prop.~\ref{prop: 1}, there exists a measurable set~$\Omega^r_{\mathsf{sem}} \subset \U$ with $\QP(\Omega^r_{\mathsf{sem}})=1$ so that for every $\eta \in \Omega^r_{\mathsf{sem}}$
\begin{align}\label{e:semSO}
T^{\U, \QP}_{r, t}(\eta)=T^{\U(B_r), \QP_r^{\eta}}_t(\eta_{B_r}) \fstop
\end{align}

%By making use of the number-rigidity~\ref{ass:Rig} of the~$\sine_2$~ensemble~$\QP$, for each $r>0$ there exists a measurable set $\Omega_{\sf rig}^r \subset \U$ with $\mu(\Omega_{\sf rig}^r)=1$ satisfying the following: if $\gamma, \eta \in \Omega_{\sf rig}^r$ with $\gamma_{B_r^c} = \eta_{B_r^c}$, then $\gamma B_r = \eta B_r$. 
%Let $\Omega_{\sf rig} = \cap_{r \in \N} \Omega_{\sf rig}^r$, which is of $\QP$-full measure as well.  

Let $u \in L^\infty(\QP)$. Thanks to~Lem.~\ref{l:FL},
%By a proposition in our paper, 
%$$\cdc^\dUpsilon_{E_h}(u) \le \cdc^\dUpsilon(u)\le 1, \quad \text{$\QP$-a.e., \ $\forall h \in \N$}\,.$$
%$\Gamma_{n}:=\Gamma_{B_n}(u) \le \Gamma(u) =0$ for any $n \in \N$. 
%By \cite[Prop.\ 5.14 (iii)]{LzDSSuz21}, 
there exists $\Omega^r_\infty \subset \dUpsilon$ so that $\QP(\Omega^r_\infty)=1$ and 
$$u_r^\eta \in L^\infty(\QP_r^\eta), \quad \forall \eta \in \Omega^r_\infty,\quad \forall r \in \N \fstop$$

By~Prop.~\ref{p:BE2}, there exists a measurable set~$\Omega^r_{\mathsf{rcd}} \subset \U$ so that $\QP(\Omega^r_{\mathsf{rcd}})=1$ and $(\U^k, \mssd_\U, \mu_r^\eta)$ is $\RCD(0,\infty)$ with $k=k(\eta)$ as in \eqref{e:R1} for every $\eta \in \Omega^r_{\mathsf{rcd}}$. 

Let $\Omega^r:=\Omega^r_{\mathsf{sem}}\cap\Omega_\infty^r \cap \Omega_{\mathsf{rcd}}^r$. As the log-Harnack inequality holds in RCD~spaces (see, \cite[Lem.~4.6]{AmbGigSav15}), the following holds for every $\eta \in \Omega^r$ and $k=k(\eta)$ and $\e\in (0,1]$
\begin{align} \label{e:LHR}
T^{\U^k(B_r), \QP_r^{k, \eta}}_t\log (u_r^\eta+\e)(\gamma) \le \log \Bigl(T^{\U^k(B_r), \QP_r^{k, \eta}}_tu_r^\eta(\zeta)+\e\Bigr) + \mssd_\U(\gamma, \zeta)^2\comma 
\end{align}
for every $\gamma, \zeta \in \U^k(B_r)$. 
%By the Sobolev-to-Lipschitz \ref{ass:ConditionalSL} for the conditioned forms, 
%there exists a measurable set $\Omega^h_{\sf rig, slc}\subset \Omega_0^h \cap \Omega_{\sf rig}$ of full $\mu$-measure so that, for all $\eta \in \Omega^h_{\sf rig, slc}$, there exists  $k=k(\eta) \in \N_0$ and  $u^{(k)}_{E_h, \eta} \in \Lipu(\mssd_{\dUpsilon(E_h)})$ so that 
%%$\tilde{u}$ is continuous on $\Omega^{m}_2$.  Therefore, $\tilde{u}_{B_n, \eta_{B_n^c}}$ is continuous on $(\Omega^{m}_2)_{B_n, \eta_{B_n^c}} \subset \U(B_n)$ for every $\eta \in \Omega^m_2$. Since $\E^{m}$ is irreducible, and the irreducibility tensorises on $X^{\times k}$, the form $\E^{m, \otimes k}$ is irreducible on $B_n^{\times k}$. Thus, we see that 
%\begin{align} \label{eq: m: sl}
%\text{$u_{E_{h}, \eta} = u^{(k)}_{E_h, \eta}$ $\quad \QP_{E_h}^\eta$-a.e.}\, .
%\end{align}  
%$\tilde{u}$ is continuous on $\Omega^{m}_2$.  Therefore, $\tilde{u}_{B_n, \eta_{B_n^c}}$ is continuous on $(\Omega^{m}_2)_{B_n, \eta_{B_n^c}} \subset \U(B_n)$ for every $\eta \in \Omega^m_2$. Since $\E^{m}$ is irreducible, and the irreducibility tensorises on $X^{\times k}$, the form $\E^{m, \otimes k}$ is irreducible on $B_n^{\times k}$. Thus, we see that 
%\begin{align} \label{eq: E: erg0}
%\text{$u_{E_{h}, \eta}$ is equal to $C^k_{E_h, \eta}$  in $\dUpsilon(E_h)$ $\quad \QP_{E_h}^\eta$-a.e.}\, .
%\end{align} 
%Note that the measure $\QP_{r}^\eta$ is fully supported on $\dUpsilon^{k}(B_r)$ by the rigidity in number \ref{ass:Rig} and \ref{ass:CE}.

%By the assumption of the quasi-regularity, there exists a quasi-continuous $\QP$-modification~$\tilde{u}$ of $u$ (see \cite[Prop.\ 3.3]{MaRoe90}). 
%Therefore, we can take a closed monotone increasing exhaustion $\{K_m\} \uparrow \dUpsilon$ so that $\tilde{u}$ is $\T_\mrmv$-continuous on $K_m$ for any $m \in \N$, and ${\rm Cap}_{\EE{\dUpsilon}{\QP}}(K^c_m)\le 1/m$. 
%Set $\Omega_{\sf qc}:=\cup_{m \in \N} K_m$, which is of $\mu$-full measure since $\Omega_{\sf qc}$ is co-negligible with respect to the capacity~${\rm Cap}_{\EE{\dUpsilon}{\QP}}$. Thus, we may assume that $\QP(K_m)>1-\epsilon$ with $\epsilon<1/2$ for all $m \ge m_\epsilon$ whereby $m_\epsilon$ is a sufficiently large integer depending on $\epsilon$.
%Set $\Omega:=\Omega_{\sf rig} \cap \Omega_{\sf ic} \cap \Omega_{\sf qc}$, which is of $\mu$-full measure by construction. 

%Let $\Omega_m:=\Omega_{{\sf rig, slc}} \cap K_m$ for $m \ge m_\epsilon$. %Therefore, ${\rm pr}^{E_h}(\Omega)$ 
%Since $\tilde{u}$ is $\T_\mrmv$-continuous on $\Omega_m$, the function $\tilde{u}_{E_h, \eta}$ is continuous on $\Omega_{E_h, \eta, m}$ for any $\eta \in \Omega_m$ whereby $\Omega_{E_h, \eta, m}:=\{\gamma \in \dUpsilon(E_h): \gamma + \eta_{E_h} \in \Omega_m\}$. %We now show that there exists $\Omega_{m, 0} \subset \Omega_m$ so that $\QP(\Omega_{m, 0}) \ge \QP(\Omega_{m})$ and 
%By Proposition~\ref{p:ConditionalIntegration}, we have that
%\begin{align} \label{eq: CPSL}
%\QP(\Omega_m)= \int_{\dUpsilon}\QP_{E_h}^\eta \bigl( \Omega_{E_h, \eta, m}\bigr) \diff \QP(\eta) \fstop 
%\end{align}
%Thus, by noting that $\QP(\Omega_m) \ge 1-\epsilon$ and the integrand of the r.h.s.\ of \eqref{eq: CPSL} is non-negative and bounded from above by $1$,
%%by noting that the non-negative function $\eta \mapsto \QP_{E_h}^\eta \bigl( \Omega^{m}_{E_h, \eta}\bigr)$ is bounded from above by $1$, 
%there exists $\Omega_{m, h} \subset \dUpsilon$ so that $\QP(\Omega_{m, h}) \ge \QP(\Omega_m) \ge 1-\epsilon$ for any $h \in \N$, and for any $\eta \in \Omega_{m, h}$,
%\begin{align} \label{eq: sl: 11}
%\QP_{E_h}^\eta \bigl( \Omega_{E_h, \eta, m}\bigr)>0 \fstop
%\end{align}
%Furthermore, set $\Omega_{m}^h:= \Omega_m \cap \Omega_{m, h}$. 
%Noting that $\epsilon <1/2$ and  $\QP(\Omega_{m, h}) \ge \QP(\Omega_m) \ge 1-\epsilon$, by a simple application of Inclusion-Exclusion formula, it holds that, for any $m \ge m_\epsilon$, 
%\begin{align} \label{eq: sl 111}
%\QP(\Omega_{m}^h) \ge 1-2\epsilon \qquad \forall h \in \N \fstop
%\end{align}
%%Noting that $\epsilon <1/2$ and $\QP(\Omega_{m, h}) \ge \QP(\Omega_m) \ge 1-\epsilon$, it holds that $\QP(\Omega_{m, 1})>0$. 
%%Note that, by construction, $\Omega_{m, 1} \uparrow \dUpsilon$ $\QP$-a.e.\ as $m \uparrow \infty$. 
%Combining~\eqref{eq: m: sl}, \eqref{eq: sl: 11} with the fact that  $\QP_{E_h}^\eta\mrestr{{\dUpsilon^{(k)}(E_h)}}$ is fully supported in $\dUpsilon^{(k)}(E_h)$ and $\tilde{u}$ is $\T_\mrmv$-continuous on $ \Omega_{E_h, \eta, m}$,   we obtain that 
%%we may take $\Omega_m \subset \Omega_{m, 0}$ since ${\rm pr}_{E_h^c}\Omega_m={\rm pr}_{E_h^c}\Omega_{m, 0}$ and \eqref{eq: Ir: 11} only depends on ${\rm pr}_{E_h^c}\Omega_{m, 0}$. 
%% \quad \forall \eta \in \Omega_{m, 0}.$$
%\begin{align} \label{eq: m: erg1}
%\text{$\tilde{u}_{E_{h}, \eta}$ is equal to $u^{(k)}_{E_h, \eta}$ {\it everywhere} in $\Omega_{E_h, \eta, m} \subset \dUpsilon^{(k)}(E_h)$ for all $\eta \in \Omega_{m}^h$}\, \fstop
%\end{align}  
%%whereby $\Omega^{m}_{E_h, \eta}:=\{\gamma \in \dUpsilon(E_h): \gamma + \eta_{E_h} \in \Omega_{m, 1}\}$.
%%Let $\Omega^h_{{\sf ec}, \eta} := ({\rm pr}^{E_h})^{-1}(\Omega^h_{{\sf ec}, E_h, \eta})$ which satisfies $\mu(\Omega^h_{{\sf ec}, \eta})=1$ as well. 
%%Recall by definition that 
%%\begin{align} \label{eq: m: 1}
%%\tilde{u}(\gamma)=\tilde{u}(\gamma_{E_h}+\gamma_{E_h^c})=\tilde{u}_{E_h, \gamma}(\gamma_{E_h}) \,.
%%\end{align}  
%%Instead of replacing $\Omega$ by another measurable set in $\Omega$ of $\mu$-full measure on which \eqref{eq: m: 1} holds for all $h \in \N$, we may use the same $\Omega$ without relabelling for the sake of the simpler notation. 
%%Without loss of generality, we may assume that \eqref{eq: m: 1} holds for all $\gamma \in \Omega$ (otherwise we can replace $\Omega$ by another measurable set $\Omega' \subset \Omega$).  
%%Let $\Omega= K_m \cap \Omega_0\cap_{n \in \N}\Omega^n_1$. We may assume that $\mu(\Omega^{m}_2)>0$ for sufficiently large $m \in \N$. Note that $
%%Recall by definition that $\tilde{u}(\gamma)=\tilde{u}(\gamma_{B_n}+\gamma_{B_n^c})=\tilde{u}_{B_n, \gamma_{B_n^c}}(\gamma_{B_n})$ for $\gamma \in \U(X)$. 
%%Note that this definition implicitly depends on $n$ since we took the modification $\tilde{u}_{K_n}$ dependently of $K_n$, therefore, we keep the subscript $n$.
%%Define $\tilde{u}_\infty(\gamma):=\liminf_{n \to \infty} \tilde{u}_n(\gamma)$. 
%By Lemma \ref{lem: IEF} in Appendix applied to $\Omega_m^h$ in \eqref{eq: sl 111},  we can take $n\mapsto m_n \in \N$ so that, by setting 
%$\Omega=\limsup_{n \to \infty}\cap_{h=1}^n\Omega_{m_n}^h$,
%it holds that 
%%\begin{align} \label{ineq: e:Ir}
%$$\QP(\Omega) = 1 \fstop$$ 
Noting  the convergence of the semigroups~$\sem{T_{r, t}^{\U, \QP}}$ to~$\sem{T_t^{\U, \QP}}$ in the $L^2(\QP)$-operator sense by~Prop.~\ref{prop: MGS}, there exist $\Omega_{\mathsf{con}} \subset \U$ with $\QP(\Omega_{\mathsf{con}})=1$ and a (non-relabelled) subsequence of $\{r\}$ so that for every $\gamma \in \Omega_{\mathsf{con}}$ 
\begin{align} \label{e:semO}
&T^{\U, \QP}_{r, t}\log (u+\e)(\gamma) \xrightarrow{r \to \infty} T^{\U, \QP}_{t}\log (u+\e)(\gamma) \comma 
\\
&\log (T^{\U, \QP}_{r, t}u(\gamma)+\e) \xrightarrow{r \to \infty}  \log (T^{\U, \QP}_{t}u(\gamma)+\e) \fstop \notag
\end{align}

Let $\Omega'=\Omega_{\mathsf{con}}\cap_{r\in \N}\Omega^r$, which by construction satisfies $\mu(\Omega')=1$. Our goal is now to prove that there exists $\Omega \subset \Omega'$ with $\QP(\Omega)=1$ so that 
\begin{align} \label{eq: SL: 2}
T^{\U, \QP}_{t}\log (u+\e)(\gamma) \le \log (T^{\U, \QP}_{t}u(\eta)+\e) + \bar{\mssd}_\U(\gamma, \eta)^2\comma \quad \text{every $\gamma, \eta \in \Omega$} \fstop
\end{align}
Thanks to~\eqref{e:semO}, Formula~\eqref{eq: SL: 2} comes down to the corresponding inequality for the semigroup~$\sem{T^{\U, \QP}_{r, t}}$ for any $r>0$:
\begin{align} \label{eq: SL: 3}
T^{\U, \QP}_{r, t}\log (u+\e)(\gamma) \le \log (T^{\U, \QP}_{r, t}u(\eta)+\e) + \bar{\mssd}_\U(\gamma, \eta)^2\comma \quad \text{every $\gamma, \eta \in \Omega$} \fstop
\end{align}
%We now prove that 
%\begin{align} \label{eq: m: 2}
%\text{$\tilde{u}$ is constant on $\Omega_{m, 1}$}.
%\end{align}

We prove \eqref{eq: SL: 3} by contradiction. Suppose that for any $\Omega \subset \Omega'$ with $\QP(\Omega)=1$, there exists $\gamma, \eta \in \Omega$ so that 
\begin{align} \label{eq: LH: chyp}
T^{\U, \QP}_{r, t}\log (u+\e)(\gamma) \ge \log (T^{\U, \QP}_{r, t}u(\eta)+\e) + \bar{\mssd}_\U(\gamma, \eta)^2 \fstop
\end{align}
We may assume that $\bar{\mssd}_\U(\gamma, \eta)<\infty$ without loss of generality.
Thus, by~\eqref{e:LLR2}, there exists $r>0$ so that  %\footnote{Take care that the second equality does not follow only by the finiteness of distance.}
 \begin{align} \label{e:NRS}
 \gamma_{B_r^c}=\eta_{B_r^c} \comma \quad \gamma(B_r)=\eta(B_r)\fstop
 \end{align}  
%It suffices to show that for any for any $\Xi_1, \Xi_2 \subset \Omega'$ with $\mu(\Xi_1)\mu(\Xi_2)>0$, there exists $\gamma^1 \in \Xi_1$ and $\gamma^2 \in \Xi_2$ so that 
%   \begin{align} \label{eq: LH: chyp}
%   T^{\U, \QP}_{r, t}(\log u)(\gamma^1) \le \log (T^{\U, \QP}_{r, t}u)(\gamma^2) + \mssd_\U(\gamma^1, \gamma^2)^2\fstop  
%%   \quad  \text{for every $\gamma^1 \in \Xi_1$ and $\gamma^2 \in \Xi_2$ with ${\sf d}_{\U}(\gamma^1, \gamma^2)<\infty$}.
%   \end{align}
% %Indeed, take $\Xi_1=\{\tilde{u}>a\}$ and $\Xi_2=\{\tilde{u} \le a\}$ for $a \in \R$. In the case of $\QP(\Xi_1)\QP(\Xi_2)=0$ for any $a \in \R$, $\tilde{u}$ must be constant. In the case that  there exists $a \in \R$ so that $\mu(\Xi_1)\mu(\Xi_2)>0$, then this contradicts \eqref{eq: chyp}. 
% %and
% %\begin{align} \label{eq: chyp}
% %\tilde{u}(\gamma^1) \neq \tilde{u}(\gamma^2)\quad  \text{for every $\gamma^1 \in \Xi_1$ and $\gamma^2 \in \Xi_2$}.
%% \end{align}
%%We now prove \eqref{eq: SL: chyp}. 
%Since $\mu$ possesses the tail-triviality~\ref{ass:TT} and $\mu(\Xi_1)>0$, we obtain that $\mu(\mathcal T(\Xi_1))=1$, where $\mathcal T(\Xi_1)$ is the tail set of $\Xi_1$ as defined in~\eqref{eq: ts}. Thus, $\mu(\mathcal T(\Xi_1) \cap \Xi_2)>0$, by which $\mathcal T(\Xi_1) \cap \Xi_2$ is non-empty and, therefore, we can take an element $\gamma^2 \in \mathcal T(\Xi_1) \cap \Xi_2$. 
%Recalling the definition of~$\Omega'$
%%\begin{align} \label{eq: slLSP}
%%\Omega=\limsup_{n \to \infty}\cap_{h=1}^n\Omega_{m_n}^h := \bigcap_{n \ge1} \bigcup_{j \ge n} \bigcap_{h=1}^j\Omega_{m_j}^h \comma
%%\end{align}
%and the definition~\eqref{eq: ts} of the tail operation~$\mathcal T$ as well as the number-rigidity~\ref{ass:Rig}, 
%%By the definitions of the tail operation and the rigidity in number \ref{ass:Rig}, 
%there exists $r \in \N$, $k=k(\gamma^2) \in \N_0$ and $\gamma^1 \in \Xi_1$  so that
% \begin{align} \label{e:NRS}
% \gamma^1_{B_r^c} = \gamma^2_{B_r^c}, \quad \gamma^1{B_r}=\gamma^2{B_r}=k \fstop
% \end{align}
%which implies $\gamma^1_{B_r}\in \Omega_{B_r, \gamma^2}$ for some $j \in \N$. Furthermore, $\QP_{r}^{\gamma^2}(\Omega_{B_r, \gamma^2, m_j})>0$ by \eqref{eq: sl: 11} and the definition \eqref{eq: slLSP} of $\Omega$. Therefore, by making use of \eqref{eq: m: erg1}, 
%we conclude 
%\begin{align} \label{eq: SL: 0}
%\tilde{u}(\gamma^i) =  \tilde{u}(\gamma^i_{E_h}+\gamma^i_{E_h^c}) = \tilde{u}_{E_h, \gamma^2}(\gamma^i_{E_h})= u^{(k)}_{E_h, \gamma^2}(\gamma^i_{E_h}), \quad (i=1,2)\,.
%%\\
% %\tilde{u}(\gamma^2)=\tilde{u}(\gamma^2_{E_h}+\gamma^2_{E_h^c}) =\tilde{u}_{E_h, \gamma^2}(\gamma^2_{E_h})=\tilde{u}(\gamma^2_{E_h}+\gamma^2_{E_h^c}) =\tilde{u}(\gamma^2),
%\end{align}
By making use of \eqref{e:semSO}, \eqref{e:LHR}, \eqref{e:NRS}, we obtain %and the relation between $T^{\U, \QP}_{r, t}$ and~$T^{\U(B_r), \QP_r^{\cdot}}_t(\cdot_{B_r})$ in~Prop.~\ref{prop: 1}, we, therefore, obtain
\begin{align} \label{eq: LH:3}
 T^{\U, \QP}_{r, t}\log (u+\e)(\gamma)  
 & =  T^{\U, \QP}_{r, t}\log (u+\e)(\gamma_{B_r} + \gamma_{B_r^c} )  
 \\
 &= T^{\U(B_r), \QP_r^{\gamma}}_t\log (u_r^{\gamma}+\e)(\gamma_{B_r})  \notag
 \\
 & \le \log (T^{\U(B_r), \QP_r^{\gamma}}_tu_r^\gamma(\eta_{B_r})+\e) + \mssd_\U(\gamma_{B_r}, \eta_{B_r})^2 \notag
 \\
  &=\log (T^{\U, \QP}_{r, t}u(\eta) +\e)+ \bar{\mssd}_\U(\gamma, \eta)^2 \notag \comma
\end{align}
which contradicts \eqref{eq: LH: chyp}, therefore, the proof of (a) is completed. 

The proof of (b) follows precisely in the same strategy as above by replacing $T^{\U, \QP}_{t}\log (u+\e)$,  $\log (T^{\U, \QP}_{t}u+\e)$ and $\bar{\mssd}_\U(\gamma, \eta)^2$ by~$(T^{\U, \QP}_tu)^\alpha$, $T^{\U, \QP}_tu^\alpha$ and $\frac{\alpha}{2(\alpha-1)}\bar{\mssd}_\U(\gamma, \eta)^2$ respectively, and noting that the dimension-free Harnack inequality holds on~$\RCD(K,\infty)$ spaces (\cite[Thm.~3.1]{Li15}).

The proof of (c): Note that $u_r^\eta \in \Lip(\U(B_r), \mssd_\U)$ whenever $u \in \Lip(\U, \bar{\mssd}_\U)$ and $\Lip_{\mssd_\U}(u_r^\eta) \le \Lip_{\bar{\mssd}_\U}(u)$ by Lem.~\ref{l:SEF3}.
Note also that the sought conclusion of (c) can be rephrased as 
$$\tilde{T}^{\U, \QP}_tu(\gamma)-\tilde{T}^{\U, \QP}_tu(\eta) \le \Lip_{\bar{\mssd}_\U}(u) \bar{\mssd}_\U(\gamma, \eta) \quad \forall \gamma, \eta \in \U\fstop$$
Thus, by the same proof strategy as in (a) replacing~$T^{\U, \QP}_{t}\log (u+\e)(\gamma)$ and~$\log (T^{\U, \QP}_{t}u(\eta)+\e)$ with~$T^{\U, \QP}_{t}u(\gamma)$ and~$T^{\U, \QP}_{t}u(\eta)$, and noting that the Lipschitz contraction property holds on RCD spaces (\cite[(iv) in Thm.~6.1]{AmbGigSav14b}), we conclude that there exists $\Omega \subset \U$ with $\QP(\Omega)=1$ so that 
$$T^{\U, \QP}_t(\gamma)-T^{\U, \QP}_t(\eta) \le \Lip_{\bar{\mssd}_\U}(u) \bar{\mssd}_\U(\gamma, \eta) \quad \forall \gamma, \eta \in \Omega\fstop$$
The conclusion now follows from the McShane extension Theorem (for extended metric spaces, see~\cite[Lem.~2.1]{LzDSSuz20}).

The proof of (d) is the same as that of (c) but using the $L^\infty$-to-$\Lip$ property (\cite[Thm.~6.5]{AmbGigSav14b}) in $\RCD(K,\infty)$ spaces instead of \cite[(iv) in Thm.~6.1]{AmbGigSav14b}).
The proof is complete.
%where the last equality follows from \eqref{eq: SL: 0}, which concludes \eqref{eq: SL: 2}. 
%We finally upgrade \eqref{eq: SL: 2} to the statement:
%\begin{align} \label{eq: SL: 3}
%\text{$|\tilde{u}(\gamma)-\tilde{u}(\eta)| \le \mssd_{\dUpsilon}(\gamma, \eta)$ for $\QP^{\otimes 2}\text{-a.e.}\ (\gamma, \eta) \in \Omega^{\times 2}$} \comma
%\end{align}
%whereby $\Omega:=\cup_{m} \Omega_{m, 1}$ and $\QP(\Omega)=1$ by construction. We utilise the same strategy as \eqref{eq: SL: chyp}. Namely, we see that for any $\Xi_1, \Xi_2 \subset \Omega$ with $\mu(\Xi_1)\mu(\Xi_2)>0$, there exists $\gamma^1 \in \Xi_1$ and $\gamma^2 \in \Xi_2$ so that 
   %\begin{align} \label{eq: SL: chyp2}
  %|\tilde{u}(\gamma^1) -\tilde{u}(\gamma^2)| \le  \mssd_{\dUpsilon}(\gamma^1, \gamma^2)<\infty \fstop
%   \quad  \text{for every $\gamma^1 \in \Xi_1$ and $\gamma^2 \in \Xi_2$ with ${\sf d}_{\U}(\gamma^1, \gamma^2)<\infty$}.
   %\end{align}
%Since $\QP(\Omega)=1$,  we may take a sufficiently large $m \ge m_\epsilon$ so that both $\QP(\Omega_{m, 1} \cap \Xi_1)>0$ and $\QP(\Omega_{m, 1} \cap \Xi_2)>0$ hold. 
%Then, by applying the conclusion \eqref{eq: SL: chyp}  to $\Omega_{m, 1} \cap \Xi_1 \subset \Omega_{m, 1}$ and $\Omega_{m, 1} \cap \Xi_2  \subset \Omega_{m, 1}$,  we conclude \eqref{eq: SL: chyp2}. %The proof is complete.
\end{proof}

%In the case of $\beta=2$, due to the tail-triviality and the number-rigidity, we have contraction properties with respect to $\mssd_\U$ in place of~$\bar{\mssd}_\U$. For $A, B \subset \U$ with $\QP(A) \QP(B)>0$, set
%$$\mssd_\U(A, B):=\essinf_{\gamma \in A} \inf_{\eta \in B} \mssd_\U(\gamma, \eta) \fstop$$
%\begin{thm}\label{t:DFHSW}
%Let $\QP$ be the $\sine_2$ ensemble. Then the following inequalities hold:
%\begin{enumerate}[{\rm (a)}]
%\item $(${\bf  set-wise log-Harnack inequality}$)$ for every non-negative $u \in L^\infty(\U, \QP)$, $t>0$, there exists $\Omega \subset \U$ with $\QP(\Omega)=1$ so that for any $A, B \subset \Omega$ with $\QP(A) \QP(B)>0$, 
%$$\essinf_{\gamma \in A}T^{\U, \QP}_t(\log u)(\gamma) \le \sup_{\eta \in B}\log (T^{\U, \QP}_tu)(\eta) + \mssd_\U(A, B)^2 \ ;$$
%\item $(${\bf  set-wise Lipschitz contraction}$)$ For $u \in \Lip_b(\bar{\mssd}_\U,  \QP)$ and $t>0$, $T_t^{\U, \QP}u$ has a $\QP$-modification~$\tilde{T}_t^{\U, \QP}u$ so that for any $A, B \subset \U$ with $\QP(A) \QP(B)>0$, 
%\begin{align*}
%\essinf_{\gamma \in A}\inf_{\eta \in B}|\tilde{T}_t^{\U, \QP}u(\gamma)- \tilde{T}_t^{\U, \QP}u(\eta)| \le \Lip_{\bar{\mssd}_\U}(u) \mssd_\U(A, B) \ ;
%\end{align*}
%\item $(${\bf set-wise $L^\infty$-to-$\Lip$ regularisation}$)$ For ~$u \in L^\infty(\QP)$ and any~$t>0$, $T_t^{\U, \QP}u$ has a $\QP$-modification~$\tilde{T}_t^{\U, \QP}u$ so that for any $A, B \subset \U$ with $\QP(A) \QP(B)>0$,  
%\begin{align*}
%\text{$T_t^{\U, \QP}u$ has a $\bar{\mssd}_\U$-Lipschitz $\QP$-modification~$\tilde{T}_t^{\U, \QP}u$ }
%\end{align*}
%and the following estimate holds:
%\begin{align*}% \label{m:LL}
%%&T_t^{\U, \QP} L^\infty(\U, \QP) \subset \Lip(\U, \mssd_\U) \quad \forall t>0 \comma
%%\\
%\essinf_{\gamma \in A}\inf_{\eta \in B}|\tilde{T}_t^{\U, \QP}u(\gamma)- \tilde{T}_t^{\U, \QP}u(\eta)| \le  \frac{1}{\sqrt{2 t}} \|u\|_{L^\infty(\QP)}   \mssd_\U(A, B) \fstop
%\end{align*}
%\end{enumerate}
%\end{thm}
%\begin{rem}
%Due to the fact that $\mssd_\U$ is {\it extended} distance taking $+\infty$ on sets of positive measures,  the standard proof in the framework of metric measure spaces (e.g., \cite{Wan14, KopStu21, AmbGigSav14b}) does not seem to apply directly, for which we provided a different proof based on the approximation.
%\end{rem}
%As a consequence of the Lipschitz contraction~(c) in Thm.~\ref{t:DFH}, we obtain the $1$-Wasserstein contraction property of the dual semigroup $\sem{\mathcal T_t^{\U, \QP}}$. Let $\mathcal P_\QP(\U):=\{\nu \in \mathcal P(\U): \nu \ll \QP\}$. For a probability measure~$\nu \in \mathcal P_\QP(\U)$with~$\diff \nu=\rho\cdot\diff \QP$, define {\it the dual semigroup}~$\sem{\mathcal T_t^{\U, \QP}}$ on~$\mathcal P_{\QP}(\U)$ as
%$$\diff \mathcal T_t^{\U, \QP}\nu:=(T_t^{\U, \QP}\rho) \cdot \diff \QP \fstop$$
%
%The $1$-Wasserstein distance is defined as follows: for $\nu, \sigma \in \mathcal P(\U)$, 
%$$W_1(\nu, \sigma):=\inf\biggl\{\int_{\U \times \U} \bar{\mssd}_{\U}(\gamma, \eta) \diff \mathsf c(\gamma, \eta): \mathsf c \in \mathsf{Cpl}(\nu, \sigma)\biggr\} \comma$$
%where $\mathsf{Cpl}(\nu, \sigma)$ is the set of all Borel probabilities on $\U^{\times 2}$ whose marginals coincide with $\nu$ and $\sigma$ respectively.
%\begin{cor}[$1$-Wasserstein contraction] \label{c:1WC}
%Let $\QP$ be the $\sine_\beta$ ensemble with $\beta>0$.
%For any $\nu, \sigma \in \mathcal P_\QP(\U)$ and any $t>0$, 
%\begin{align*} %\label{e:1WC}
%W_1(\mathcal T_t^{\U, \QP}\nu, \mathcal T_t^{\U, \QP}\sigma) \le W_1(\nu, \sigma) \fstop
%\end{align*} 
%\end{cor}
%\begin{proof}
%By making use of the Lipschitz contraction~(c) of Thm.~\ref{t:DFH}, the proof is standard (see, e.g.,~\cite[Prop.~3.2]{AmbGigSav15}).
%For the sake of the completeness, however, we provide a proof. 
%
%Let $\diff\nu=\rho_\nu \diff \nu$ and $\diff\sigma=\rho_\sigma \diff \sigma$. Noting that \purple{$\bar{\mssd}_\U$ is $\tau^{\times 2}_\mrmv$-lower semi-continuous}\footnote{This is not true}, the Kantrovich--Rubinstein duality holds for $W_1$ (see, e.g.,~\cite[Thm.~5.10, Particular Case 5.16]{Vil09}). Thus, 
%\begin{align} %\label{e:KD}
%W_1(\mathcal T_t^{\U, \QP}\nu, \mathcal T_t^{\U, \QP}\sigma) &=\sup\biggl\{ \int_{\U}\phi(T_t^{\U, \QP}\rho_\nu -T_t^{\U, \QP}\rho_\sigma ) \diff \QP: \phi \in \Lip^{1}_b(\bar{\mssd}_\U, \QP)\biggr\} \notag
%\\
%&= \sup\biggl\{ \int_{\U}\bigl( T_t^{\U, \QP}\phi \bigr)\cdot (\rho_\nu -\rho_\sigma ) \diff \QP: \phi \in \Lip^{1}_b(\bar{\mssd}_\U, \QP)\biggr\} \notag
%\\
%&\le \sup\biggl\{ \int_{\U}\phi \cdot (\rho_\nu -\rho_\sigma ) \diff \QP: \phi \in \Lip^{1}_b(\bar{\mssd}_\U, \QP)\biggr\} \notag
%\\
%&= W_1(\nu, \sigma) \comma\notag
%\end{align}
%where the second equality followed from the $L^2$-symmetry of the semigroup~$\sem{T_t^{\U, \QP}}$, the third equality followed from the fact that $T_t^{\U, \QP}\phi$ is $1$-Lipschitz with respect to~$\bar{\mssd}_\U$ up to $\QP$-modification by~(c) of Thm.~\ref{t:DFH}. The proof has been completed. 
%\end{proof}
%\end{proof}
\begin{cor} \label{cor:DLA}
Let $\QP$ be the $\sine_\beta$ ensemble with $\beta>0$. Then
$$\text{$\Lip_b(\bar{\mssd}_\U, \QP)$ is dense in $\dom{\E^{\U, \QP}}$. }$$
\end{cor}
\begin{proof}
$\Lip_b(\bar{\mssd}_\U, \QP)$ is dense in $L^2(\QP)$ as noted in the fourth paragraph of the proof of Prop.~\ref{t:ClosabilitySecond}. Thus, the statement follows from (c) of Thm.~\ref{t:DFH} and Lem.~\ref{l:MU}.
\end{proof}
\section{Generalisation} \label{sec:GL}
%\purple{Application to Riesz$_\beta$ and Airy$_\beta$}\footnote{Number-Rigidity and β-Circular Riesz gas David Dereudre and Thibaut Vasseur}
We have been so far working in the case of $\sine_\beta$ ensemble. In this section, we seek to generalise the aforementioned statements to general probability measures on~$\U=\U(\R^n)$ for $n \in \N$. As an application, we prove $\BE(0,\infty)$ in the case of $\beta$-Riesz ensemble. 
\smallskip

In this section, we denote by $\mssm$ and $\mssm_r$ the Lebesgue measure on~$\R^n$ and its restriction on~$B_r(0)$ respectively, and we take the Euclidean distance~$\mssd(x, y):=|x-y|$ for $x, y \in \R^n$. 
Let $\mu$ be a Borel probability on~$\U$. Let $\mathcal K(\mu^\eta_r) \subset \N_0$ be defined as
$$\mathcal K(\mu^\eta_r):=\{k \in \N_0: \mu_r^{k, \eta}(\U^k(B_r))>0\} \fstop$$  
\begin{ass}\label{a:GT}
Let $K \in \R$ and $\mu$ be a Borel probability. Assume the following conditions:
\begin{enumerate}[$(a)$]
\item the measure~$\mu_r^\eta$ is absolutely continuous with respect to the Poisson measure~$\pi_{\mssm_r}$, and $\mu_r^{k, \eta}$ is equivalent to $\pi_{\mssm_r}|_{\U^k(B_r)}$ for any $k \in \mathcal K(\mu^\eta_r)$, $\QP$-a.e.~$\eta$ and any $r>0$;
\item the density~$$\frac{\diff\mu_r^{k, \eta}}{\diff \pi_{\mssm_r}|_{\U^k(B_r)}}$$ is $\tau_\mrmv$-continuous on $\U^k(B_r)$, and  the logarithmic density 
$$\Psi_r^{k, \eta}=-\log\Bigl(\frac{\diff\mu_r^{k, \eta}}{\diff \pi_{\mssm_r} |_{\U^k(B_r)}}\Bigr)$$ is $K$-geodesically convex with respect to~$\mssd_{\U}$ on $\U^k(B_r)$ for any $k \in \mathcal K(\mu^\eta_r)$, $\QP$-a.e.~$\eta$ and any $r>0$.%\footnote{Check if the number-rigidity is not needed for (a)}
%\item $\mu$ possesses the tail-triviality~\ref{ass:TT} and the number-rigidity~\ref{ass:Rig}.
\end{enumerate}
\end{ass}
Under (a) in Assumption, the local Dirichlet form~$(\E^{\U, \QP}, \dom{\E^{\U, \QP}})$ is constructed in the same proof as in the case of $\sine_\beta$ ensemble as we have not use any particular property of $K=0$. We further show the synthetic curvature bound for the form~$(\E^{\U, \QP}, \dom{\E^{\U, \QP}})$ and related functional inequalities.
\begin{thm}\label{t:GT}
Suppose that $\QP$ satisfies Assumption~\ref{a:GT}.  Then the local Dirichlet form~$(\E^{\U, \QP}, \dom{\E^{\U, \QP}})$ satisfies
\begin{enumerate}[$(a)$]
\item {\bf $($Bakry--\'Emery inequality $\BE(K,\infty)$$)$}
\begin{align*}
\cdc^{\U}\bigl(T_t^{\U, \mu} u\bigr) \le e^{-Kt}T_t^{\U, \mu} \cdc^{\U}(u) \quad \forall u \in  \dom{\E^{\U, \mu}} \ ; %\tag{$\BE_1(K,\infty)$}
\end{align*}
\item$(${\bf lntegral Bochner inequality}$)$ for every $(u, \phi) \in \dom{\mathbf \cdc^{\U, \QP}_2}$
\begin{align*}
\mathbf \cdc^{\U, \QP}_2(u, \phi) \ge K \int_{\U} \cdc^{\U}(u) \phi \diff \QP \ ;
\end{align*}
\item $(${\bf local Poincar\'e inequality}$)$  for $u \in \dom{\E^{\U, \QP}}$ and $t >0$,
\begin{align*}
&T^{\U, \QP}_tu^2- (T^{\U, \QP}_tu)^2 \le \frac{1-e^{-2Kt}}{K}T^{\U, \QP}_t\cdc^{\U}(u)  \comma
\\
&T^{\U, \QP}_tu^2- (T^{\U, \QP}_tu)^2 \ge \frac{e^{-2Kt}-1}{K}\cdc^{\U} (T^{\U, \QP}_tu) \ ;
\end{align*}
\item $(${\bf  local logarithmic Sobolev inequality}$)$ for non-negative $u \in \dom{\E^{\U, \QP}}$ and $t>0$,
\begin{align*}
&T^{\U, \QP}_tu\log u- T^{\U, \QP}_tu\log T^{\U, \QP}_t u \le \frac{1-e^{-2Kt}}{2K}T^{\U, \QP}_t\biggl( \frac{\cdc^{\U}(u)}{u} \biggr) \comma
\\
&T^{\U, \QP}_tu\log u- T^{\U, \QP}_tu\log T^{\U, \QP}_t u \ge \frac{e^{-2Kt}-1}{2K} \frac{\cdc^{\U}(T^{\U, \QP}_t u)}{T^{\U, \QP}_t u}  \fstop
\end{align*}
\item$(${\bf Exponential integrability of $1$-Lipschitz functions}$)$
 If $u$ is a $\bar{\mssd}_\U$-Lipschitz function with $\Lip_{\bar{\mssd}_\U}(u) \le 1$ and $|u(\gamma)|<\infty$ $\QP$-a.e.~$\gamma$, then for every $s<\sqrt{\frac{8K}{1-e^{-2Kt}}}$
$$\int_{\U} e^{s u(\eta)} P_t^{\U, \QP}(\gamma, \diff \eta)<\infty \fstop$$
%\end{enumerate}
%%If, furthermore, $\QP$ satisfies (c) in Assumption~\ref{a:GT}, then 
%\begin{enumerate}[$(a)$] \setcounter{enumi}{3}
%\item \purple{ the form $(\E^{\U, \QP}, \Lip(\mssd_\U, \QP))$ is Markov unique;}
\item $(${\bf Local hyper-contractivity}$)$ for all $t>0$, $0<s\le t$ and $1<p<q<\infty$ with $\frac{q-1}{p-1}=\frac{e^{2Kt}-1}{e^{2Ks}-1}$, the following holds:
$$\Bigl( T_s^{\U, \QP}(T^{\U, \QP}_{t-s}u)^{q}\Bigr)^{1/q} \le \Bigl( T_t^{\U, \QP}u^p\Bigr)^{1/p} \comma\quad f \ge 0 \ ;$$
\item $(${\bf  log Harnack inequality}$)$ for every non-negative $u \in L^\infty(\U, \QP)$, $\e \in (0, 1]$, $t>0$, there exists $\Omega \subset \U$ so that $\QP(\Omega)=1$ and 
$$T^{\U, \QP}_t\log (u+\e)(\gamma) \le \log (T^{\U, \QP}_tu(\eta)+\e) + \frac{K}{2(1-e^{-2Kt})}\bar{\mssd}_\U(\gamma, \eta)^2\comma \quad \text{$\forall \gamma, \eta \in \Omega$} \ ;$$
\item $(${\bf  dimension-free Harnack inequality}$)$ for every non-negative $u \in L^\infty(\U, \QP)$, $t>0$ and $\alpha>1$ there exists $\Omega \subset \U$ so that $\QP(\Omega)=1$ and 
$$(T^{\U, \QP}_tu)^\alpha(\gamma)\le T^{\U, \QP}_tu^\alpha(\eta) \exp\Bigl\{ \frac{\alpha K}{2(\alpha-1)(1-e^{-2Kt})}\bar{\mssd}_\U(\gamma, \eta)^2\Bigr\} \comma \quad \text{$\forall \gamma, \eta \in \Omega$} \ ;$$
\item $(${\bf  Lipschitz contraction}$)$ For $u \in \Lip(\bar{\mssd}_\U, \QP)$ and $t>0$, 
\begin{align*}
\text{$T_t^{\U, \QP}u$ has a $\bar{\mssd}_\U$-Lipschitz $\QP$-modification~$\tilde{T}_t^{\U, \QP}u$ }
\end{align*}
and the following estimate holds:
$$\Lip_{\bar{\mssd}_\U}(\tilde{T}_t^{\U, \QP} u) \le e^{-Kt}\Lip_{\bar{\mssd}_\U}(u) \ ;$$
\item$(${\bf $L^\infty$-to-$\Lip$ regularisation}$)$ 
For $u \in L^\infty(\QP)$ and $t>0$, 
\begin{align*}
\text{$T_t^{\U, \QP}u$ has a $\bar{\mssd}_\U$-Lipschitz $\QP$-modification~$\tilde{T}_t^{\U, \QP}u$ }
\end{align*}
and the following estimate holds:
\begin{align*}% \label{m:LL}
\Lip_{\bar{\mssd}_\U}(\tilde{T}_t^{\U, \QP} u) &\le \frac{1}{\sqrt{2I_{2K}(t)} } \|u\|_{L^\infty(\QP)} \quad \forall t>0 \comma
\end{align*}
where $I_K(t):=\int_0^t e^{Kr} \diff r$;
\item$(${\bf The density of Lipschitz algebra}$)$ 
$$\text{$\Lip_b(\bar{\mssd}_\U, \QP)$ is dense in $\dom{\E^{\U, \QP}}$.}$$
%\item $(${\bf  $1$-Wasserstein contraction}$)$ For any $\nu, \sigma \in \mathcal P_\QP(\U)$ and any $t>0$, 
%\begin{align*} %\label{e:1WC}
%W_1(\mathcal T_t^{\U, \QP}\nu, \mathcal T_t^{\U, \QP}\sigma) \le e^{-Kt}W_1(\nu, \sigma) \fstop
%\end{align*} 


\end{enumerate}
\end{thm}

\begin{proof}
Thanks to Assumption~\ref{a:GT}, the space~$(\U^k(B_r), \mssd_\U, \QP_r^{k, \eta})$ satisfies $\RCD(K,\infty)$ for every $k \in \mathcal K(\mu_r^\eta)$ as in the same proof of Prop.~\ref{p:BE1}. 
%Note that under (a), (b) and (c) of Assumption~\ref{a:GT}, the measure $\QP$ satisfies all the conditions in~\cite[Thm.~4.1]{Suz22}
%%\footnote{Take care of (CE), which is not necessarily implied by Assumption $\rightarrow$ \purple{Why not directly assuming $(SL)$ on $\U$?}}, 
%so that the form~$(\E^{\U, \QP}, \dom{\E^{\U, \QP}})$ satisfies the Sobolev-to-Lipschitz property with respect to~$\mssd_\U$.
%\footnote{Take care again of this statement in particular, of the support of $\QP_r^\eta$.}  
After Prop.~\ref{p:BE1}, we have not used any particular properties of $K=0$ nor $n=1$ in the proofs in Sections \ref{sec:CI} and \ref{sec:LH}. Hence, the completely same proofs work (with the multiplicative constant $e^{-Kt}$ instead of $1$) under Assumption~\ref{a:GT} for the $\BE(K,\infty)$ inequality as well as all the other statements in Thm.~\ref{t:GT}.% and Cor.~\ref{cor:DLA}.
\end{proof}

\begin{rem}[Non-necessity of the number rigidity~\ref{ass:Rig}]
Under the number rigidity~\ref{ass:Rig}, we have $\#\mathcal K(\mu^\eta_r) =1$. This however has not been essentially used for the proofs in the case of~$\sine_\beta$. We therefore do not need to assume~\ref{ass:Rig} in Thm.~\ref{t:GT}. See Remark~\ref{r:NRI} for the construction of Dirichlet forms. For the arguments in Section~\ref{sec:CI} and~\ref{sec:LH}  involving the semigroup $T_t^{\U(B_r), \QP_r^\eta}$, we just need to observe that each $k$-particle space~$\U^k(B_r)$ is an invariant set of the semigroup~$T_t^{\U(B_r), \QP_r^\eta}$, i.e., $$T_t^{\U(B_r), \QP_r^\eta}u\1_{\U^k(B_r)} = \1_{\U^k(B_r)}T_t^{\U(B_r), \QP_r^\eta}u \comma \quad u \in L^2(\U(B_r), \QP_r^\eta)\fstop$$
Thus, we may think of $\U(B_r)$ as the disjoint union $\sqcup_{k \in \mathcal K(\mu^\eta_r)}\U^{k}(B_r)$ for $\QP$-a.e.~$\eta$ regarding the semigroup action. %Therefore, the arguments in~Section~\ref{sec:CI} and \ref{sec:LH} for each fixed $k$ suffices to conclude those in the case of $\#\mathcal K(\mu^\eta_r) >1$. 
Hence by applying the same proofs as in~Section~\ref{sec:CI} and \ref{sec:LH} to each fixed $k \in \mathcal K(\mu^\eta_r)$ (instead of using $k=k(\eta)$ selected by the number rigidity~\ref{ass:Rig}), we concluded Thm.~\ref{t:GT} without number rigidity~\ref{ass:Rig}. 
\end{rem}

\subsection{$\beta$-Riesz ensemble}In this section, we apply Thm.~\ref{t:GT} to prove $\BE(0,\infty)$ in the case of $\beta$-Riesz ensemble~$\QP=\QP_\beta$ for every~$\beta>0$ on~$\U(\R)$. We drop the subscript~$\beta$ as it does not play any particular role in the following argument. Let $g(x)=|x|^{-s}$ with $s \in (0, 1)$ for $x \in \R$. Define 
\begin{align*}
&H^k_r(\gamma):=\sum_{i<j}^kg(x_i-x_j) \comma \quad M_{r, R}^{k, \eta}(\gamma, \eta):=\sum_{i=1}^k\sum_{y \in \eta_{B_r^c},\ |y| \le R}\bigl(g(x_i-y)-g(y)\bigr) \comma
\\
&\Psi_{r, R}^{k, \eta}(\gamma):= \beta \Bigl(H^k_r(\gamma)+M_{r, R}^{k, \eta}(\gamma, \eta)\Bigr)  \quad \text{for $\gamma=\sum_{i=1}^k \delta_{x_i} \in \U^k(B_r)$ and $\eta \in \U(\R)$} \fstop
\end{align*}

\begin{prop} \label{p:conv2}
$\Psi_{r, R}^{k, \eta}$ is geodesically convex in $(\U^{k}(B_r), \mssd_{\U})$ for any $0<r<R<\infty$, $k \in \N$, $\eta \in \U(B_r^c)$ and $\beta>0$.
\end{prop}
\begin{proof}
Let $H_{ij}, H_i^y$ be the Hessian matrices of the functions $(x_1, \ldots, x_k) \mapsto g(x_i-x_j)$ and~$(x_1, \ldots, x_k) \mapsto g(x_i-y)-g(y)$ respectively. 
By observing
\begin{align} \label{e:HCP}
&\mathbf v H_{ij} \mathbf v^t = \frac{\beta s(s+1)(v_i-v_j)^2}{|x_i-x_j|^{s+2}}, \quad  \mathbf v H^y_{i} \mathbf v^t = \frac{\beta s(s+1)v_i^2}{|y-x_i|^{s+2}} \comma
\\
& \mathbf v=(v_1, \ldots, v_k) \in \R^{k} \comma \notag
\end{align}
the same proof works as in Prop.~\ref{p:conv}. 
\end{proof}
\begin{thm}[{\cite[Thm.~1.8]{DerVas21}}]\label{t:BRG}
There exists a Borel probability measure $\QP=\QP_\beta$ so that the pointwise limit $\Phi_{r}^{k, \eta}:=\lim_{R\to \infty}\Phi_{r, R}^{k, \eta}$ exists $\mu$-a.e.~$\eta$ and satisfying the following DLR equation:
\begin{align} \label{d:cp2}
 \diff \mu_{r}^{k, \eta}=\frac{e^{-\Psi^{k, \eta}_{r}}}{Z_{r}^\eta} \diff \mssm_r^{\odot k}  \comma \quad k \in \mathcal K(\QP_r^\eta)
\end{align}
where $\mu_{r}^{k, \eta}$ was defined after~\eqref{d:CPB} and $Z_{r}^\eta$ is the normalisation constant. 
\end{thm}
\begin{rem}
The measure $\QP=\QP_\beta$ was constructed as a subsequencial limit of certain finite-volume Gibbs measures. The uniqueness of the limit points seems still open, and any limit point is currently called {\it $\beta$-circular Riesz gas (or ensemble)}, see e.g., \cite[Prop.~1.5]{DerVas21} for more details. 
\end{rem}
\begin{cor}
Any $\beta$-circular Riesz ensemble $\QP$ satisfies Assumption~\ref{a:GT} for $\beta>0$.
\end{cor}
\begin{proof}
The condition~(a) and the geodesical convexity in (b) of Assumption~\ref{a:GT} follow by Thm.~\ref{t:BRG} and Prop.~\ref{p:conv2}. 
We only need to verify the continuity of the map
\begin{align}\label{g:GM}
\U^k(B_r) \ni \gamma \mapsto e^{-\Psi^{k, \eta}_{r}(\gamma)}
\end{align}
 for every~$k \in \mathcal K(\mu^\eta_r)$ and $\QP$-a.e.~$\eta$. 
Thanks to \cite[Lem.~1.7]{DerVas21} (note that the roles of $\gamma$ and $\eta$ there are opposite to this article), the following pointwise limit exists for $\QP$-a.e.~$\eta$
$$M_{r}^{k, \eta}(\gamma, \eta):=\lim_{R \to \infty}M_{r, R}^{k, \eta}(\gamma, \eta) <\infty \comma \quad k \in \mathcal K(\QP_r^\eta)\comma $$
and $\Psi^{k, \eta}_{r}(\gamma, \eta)$ can be written as
$$\Psi^{k, \eta}_{r}(\gamma, \eta)=\lim_{R \to \infty}\Psi^{k, \eta}_{r, R}(\gamma, \eta)=\beta \Bigl(H^k_r(\gamma)+M_{r}^{k, \eta}(\gamma, \eta)\Bigr) \fstop$$
Thus, it suffices for the continuity of ~\eqref{g:GM} to show the continuity of the map $\gamma \mapsto M_{r}^{k, \eta}(\gamma, \eta)$ on~$\U^k(B_r)$ for~$\QP$-a.e.~$\eta$ and each $k \in  \mathcal K(\QP_r^\eta)$. 
%$\gamma=\sum_{i=1}^k\delta_{x_i}$ with $x_i \neq x_j$ whenever $i \neq j$. Here, the pointwise limit~$M_{r}^{k, \eta}(\gamma, \eta)=\lim_{R \to \infty}M_{r, R}^{k, \eta}(\gamma, \eta)$ is well-defined for $\QP_\beta$-a.e.~$\eta$ by \cite[Lem.~1.7]{DerVas21}. 
%Let $\eta_{B_R\setminus B_r} = \sum_{i=1}^l \delta_{y_i}$ and let $\e_{R}:=\min\{|r-y_i| \wedge |-r-y_i|: 1 \le i \le l\}$. As the configuration~$\eta$ does not have an accumulation point in~$\R$ and $B_r=[-r, r]$ is a closed interval, we have $\e:=\inf_{R >r} \e_R>0$.  
Let $\gamma_n=\sum_{i=1}^{k}\delta_{x_i^{(n)}} \in \U^k(B_r)$ converge to $\gamma$ vaguely in $\U^k(B_r)$. Then, the continuity follows by observing 
%Let $L_\e:=\Lip(g: [\e, \infty))$ be the Lipschitz constant of $g$ on $[\e, \infty)$. 
\begin{align*} %\label{e:EMES}
|M_{r}^{k, \eta}(\gamma, \eta)-M_{r}^{k, \eta}(\gamma_n, \eta)| 
&= \lim_{R \to \infty}\biggl|\sum_{i=1}^k\sum_{y \in \eta_{B_r^c},\ |y| \le R}\bigl(g(x_i-y)-g(x_i^{(n)}-y)\bigr) \biggr| 
\\
&= \lim_{R \to \infty}\biggl|\sum_{i=1}^k\sum_{y \in \eta_{B_r^c},\ |y| \le R}g'(c_y^{(n)})(x_i-x_i^{(n)}) \biggr|  
\\
&=\biggl|\sum_{i=1}^k(x_i-x_i^{(n)})  \lim_{R \to \infty}\sum_{y \in \eta_{B_r^c},\ |y| \le R}g'(c_y^{(n)}) \biggr| \comma 
\end{align*}
for some $c^{(n)}_y \in [x_i-y, x_i^{(n)}-y]$ when $x_i-y <x_i^{(n)}-y$ and $c^{(n)}_y \in [x_i^{(n)}-y, x_i-y]$ when $x_i-y > x_i^{(n)}-y$ by Mean Value Theorem. Note that the case~of $x_i-y =x_i^{(n)}-y$ trivialises the argument as $g(x_i-y)-g(x_i^{(n)}-y)=0$, which can be therefore ignored. Passing to the limit~$n \to \infty$, the proof is completed. 
%Let $M:=\sup\{|x_i-x_i^{(n)}|: 1\le i \le k,\ n \in \N\}<\infty$ and $G_y=\sup_{n \in \N}g'(c_y^{(n)})$. 
%No
%Then, the RHS of~\eqref{e:EMES} is bounded uniformly in $n$ from above by 
%\begin{align*}
%\lim_{R \to \infty}\biggl|\sum_{i=1}^k\sum_{y \in \eta_{B_r^c},\ |y| \le R}MG_y \biggr| \purple{<\infty} \comma
%\end{align*}
%\purple{where the finiteness of the summation is easy to be verified by the definition of $G_y$. }
%Thus, by Dominated Convergence with dominating function $MG_y$, we can take the limit $n \to \infty$ in~\eqref{e:EMES} and obtain the continuity.
\end{proof}
\begin{cor}\label{c:BRE}
The Dirichlet form~$(\E^{\U, \QP}, \dom{\E^{\U, \QP}})$  in~\eqref{eq:Temptation} with the $\beta$-Riesz ensemble $\QP$ satisfies $\BE(0,\infty)$. Furthermore, all the statements in Thm.~\ref{t:GT} hold true with $K=0$. 
\end{cor}
%\begin{align} \label{e:HCP}
%\mathbf v H_{ij} \mathbf v^t = \frac{(v_i-v_j)^2}{|x_i-x_j|^2}, \quad  \mathbf v H^y_{i} \mathbf v^t = \frac{v_i^2}{|y-x_i|^2} \fstop
%\end{align}
%Both $H_{ij}$ and $H_i^y$ are, therefore,  positive semi-definite. Thus, for any $0<r<R$, any $y \in [-R, -r] \cup [r, R]$ and any $i, j \in  \{1, 2, \ldots, k\}$ with $i \neq j$, the functions~$(x_1, \ldots, x_k) \mapsto -\log |x_i-x_j|$ and~$(x_1, \ldots, x_k) \mapsto -\log|1-\frac{x_i}{y}|$ are convex  in the following areas for any~$\sigma \in \mathfrak S_k$: 
%$$\bigl\{(x_1, \ldots, x_k) \in B_r^{\times k}: x_{\sigma(1)} < x_{\sigma(2)}<\cdots <x_{\sigma(k)}\bigr\} \fstop$$ 
%%$$-\log |x_i-x_j| \comma \quad -\log\Bigl|1-\frac{x_i}{y}\Bigr| \comma \quad \forall y \in [-R, -r] \cup [r, R] \quad i \in \{1, 2, \ldots, k\}\fstop$$ 
%In view of \eqref{e:STS}, the following expression, therefore, concludes that $\Psi_{r, R}^{k, \eta}$ is geodesically convex as a function on $\U^k(B_r)$: for any $\gamma =\sum_{i=1}^k\delta_{x_i}$
%\begin{align} \label{eq: conv}
%\Psi_{r, R}^{k, \eta}(\gamma) = -\beta\sum_{i<j}^k \log (|x_i-x_j|) - \beta\sum_{i=1}^k \sum_{y \in \eta_{B_r^c}, |y| \le R} \log\Bigl|1-\frac{x_i}{y}\Bigr| \fstop
%\end{align}
%The proof is complete.
%Since the convexity is preserved by the quotient map $\pr_r\colon B_r^{\times k} \to \U^{k}(B_r)$ equipped with $\mssd_{\U}$, we conclude the statement. 


%\begin{rem} 
%The case where $\mu$ is the Poisson measure $\pi_{\mssm}$ with the Lebesgue intensity measure $\mssm$ on $\R^n$, i.e., the case of no interaction potentials,  is a particular case of Theorem~\ref{t:GT}. This case has been covered by \cite{ErbHue15} and also by \cite{LzDSSuz22}. Our proof strategy is, however,  different from theirs, where they constructed the infinite-product semigroup operator on the infinite product space $(\R^n)^{\times \infty}$ and relate it to the semigroup on the configuration space~$\U(\R^n)$ by using exponential cylinder functions. In their approach, the condition of {\it no interaction potentials} is essential, by which the semigroup on $\U(\R^n)$ can be expressed by the product semigroup on the base space~$\R^n$ and  {\it the tensorisation property} of $\BE(0,\infty)$ under taking product spaces of the base space~$\R^n$ can be used effectively.  
%\end{rem}
%\begin{ass}\label{a:GT}
%Let $K \in \R$ and $\mu$ be a Borel probability measure on $\U(\R^n)$ satisfying the following:
%\begin{enumerate}[$(a)$]
%\item the regular conditional probability $\mu_r^\eta$ is absolutely continuous with respect to the Poisson measure $\pi_{\mssm_r}$ and its density $\Psi_r^\eta$ is $\tau_\mrmv$-lower semi-continuous on $\U(B_r)$ and $K$-convex with respect to $\mssd_{\U}$;
%\item $\mu$ possesses the tail-triviality~\ref{ass:TT} and the number-rigidity~\ref{ass:Rig}.
%\end{enumerate}
%\end{ass}
%\begin{thm}\label{t:GT}
%Suppose that $\QP$ satisfies (a) in Assumption~\ref{a:GT}. Then, $(\E^{\U, \QP}, \dom{\E^{\U, \QP}})$ is well-defined and satisfies all the statements in Prop.~\ref{p:DF}. Furthermore, the form~$(\E^{\U, \QP}, \dom{\E^{\U, \QP}})$ satisfies
%\begin{enumerate}[$(a)$]
%\item $\BE(K, \infty)$;
%\item $(${\bf $K$-local Poincar\'e inequality}$)$  for $u \in \dom{\E^{\U, \QP}}$ and $t >0$,
%\begin{align*}
%&T^{\U, \QP}_tu^2- (T^{\U, \QP}_tu)^2 \le \frac{1-e^{-2Kt}}{K}T^{\U, \QP}_t\cdc^{\U}(u)  \comma
%\\
%&T^{\U, \QP}_tu^2- (T^{\U, \QP}_tu)^2 \ge \frac{e^{-2Kt}-1}{K}\cdc^{\U} (T^{\U, \QP}_tu) \ ;
%\end{align*}
%\item $(${\bf  $K$-local logarithmic Sobolev inequality}$)$ for non-negative $u \in \dom{\E^{\U, \QP}}$ and $t>0$,
%\begin{align*}
%&T^{\U, \QP}_tu\log u- T^{\U, \QP}_tu\log T^{\U, \QP}_t u \le \frac{1-e^{-2Kt}}{2K}T^{\U, \QP}_t\biggl( \frac{\cdc^{\U}(u)}{u} \biggr) \comma
%\\
%&T^{\U, \QP}_tu\log u- T^{\U, \QP}_tu\log T^{\U, \QP}_t u \ge \frac{e^{-2Kt}-1}{2K} \frac{\cdc^{\U}(T^{\U, \QP}_t u)}{T^{\U, \QP}_t u}  \fstop
%\end{align*}
%\end{enumerate}
%If, furthermore, $\QP$ satisfies (b) in Assumption~\ref{a:GT}, then 
%\begin{enumerate}[$(a)$] \setcounter{enumi}{3}
%\item the semigroup $T_t^{\U, \QP}$ satisfies $L^\infty(\U, \QP)$-to-$\Lip(\U, \mssd_\U)$ regularisation, i.e., 
%\begin{align*}% \label{m:LL}
%T_t^{\U, \QP} L^\infty(\U, \QP) \subset \Lip(\U, \mssd_\U) \quad \forall t>0 \ ;
%\end{align*}
%\item  the form $(\E^{\U, \QP}, \Lip(\mssd_\U))$ is Markov unique;
%\item $(${\bf  $K$-log Harnack inequality}$)$ for every non-negative $u \in L^\infty(\U, \QP)$, $t>0$, there exists $\Omega \subset \U$ so that $\QP(\Omega)=1$ and 
%$$T^{\U, \QP}_t(\log u)(\gamma) \le \log (T^{\U, \QP}_tu)(\eta) + \frac{K}{2(1-e^{-2Kt})}\mssd_\U(\gamma, \eta)^2\comma \quad \text{$\forall \gamma, \eta \in \Omega$} \ ;$$
%\item $(${\bf  $K$-dimension-free Harnack inequality}$)$ for every non-negative $u \in L^\infty(\U, \QP)$, $t>0$ and $\alpha>1$ there exists $\Omega \subset \U$ so that $\QP(\Omega)=1$ and 
%$$(T^{\U, \QP}_tu)^\alpha(\gamma)\le T^{\U, \QP}_tu^\alpha(\eta) \exp\Bigl\{ \frac{\alpha K}{2(\alpha-1)(1-e^{-2Kt})}d_\U(\gamma, \eta)^2\Bigr\} \comma \quad \text{$\forall \gamma, \eta \in \Omega$} \fstop$$
%
%\end{enumerate}
%\end{thm}
\begin{appendix}
\section{}
Let~$\mssm$ and $\mssm_r$ be the Lebesgue measure on~$\R^n$ and its restriction on $B_r$ respectively. Set $\U=\U(\R^n)$. 
\begin{lem} \label{l:WDG}
Let~$\QP$ be a Borel probability on~$\U$ satisfying that $\QP_r^\eta$ is absolutely continuous with respect to the Poisson measure~$\pi_{\mssm_r}$ for any $r>0$ and $\QP$-a.e.~$\eta$. Let $\Sigma \subset B_r$ so that $\mssm_r(\Sigma^c)=0$. Let $\Omega(r):=\{\gamma \in \U: \gamma_\Sigma=\gamma_{B_r}\}$. Then, 
$$\QP\bigl(\Omega(r)\bigr)=1 \qquad \forall r>0 \fstop$$
\end{lem}
\begin{proof}
We fix $r>0$ and write simply $\Omega=\Omega(r)$.
By the disintegration formula~\eqref{p:ConditionalIntegration2}, 
$$\QP(\Omega)=\int_{\U} \QP_r^\eta(\Omega_r^\eta) \diff \QP(\eta) \fstop$$
Thus, it suffices to show $\QP_r^\eta(\Omega_r^\eta)=1$ for $\QP$-a.e.~$\eta$. This is equivalent to show 
\begin{align} \label{e:NMP}
\QP_r^\eta(\Omega_r^\eta)=\sum_{k \in \N_0}\QP_r^{k, \eta}(\Omega_r^\eta)=1 \fstop
\end{align}
As $\QP_r^{k, \eta}$ is absolutely continuous with respect to~$\pi_{\mssm_r}|_{\U^k(B_r)}$, 
it suffices to prove 
$$\pi_{\mssm_r}|_{\U^k(B_r)}((\Omega_r^\eta)^c)=0$$ for every~$k \in \N_0$ and~$\eta \in \U$.

%Recall $\Omega_r^\eta=\{\gamma \in \U(B_r): \gamma+\eta_{B_r^c} \in \Omega\}$ for $\eta \in \U$ defined in~\eqref{e:SEF2}. 
We show that (recall the definition of symmetric product~$\Sigma^{\odot k}$ in~\eqref{e:STS})
\begin{align} \label{e:TMO}
\Sigma^{\odot k} \subset \Omega_r^\eta \cap \U^k(B_r)  \qquad \forall \eta \in \U \fstop
\end{align}
Let $\gamma \in \Sigma^{\odot k}$. Then by the definition of~$\Omega$, it holds that $\gamma+\eta_{B_r^c} \in \Omega$ for any $\eta \in \U$. Thus, by recalling the definition~\eqref{e:SEF2} of~$\Omega_r^\eta$, we obtain $\gamma \in \Omega_r^\eta \cap \U^k(B_r)$. Thus, \eqref{e:TMO} holds true. 

By using~\eqref{e:TMO}, $\pi_{\mssm_r}|_{\U^k(B_r)}=e^{-\mssm_r(B_r)}\mssm_r^{\odot k}$ by~\eqref{d:PS} and $\mssm_r^{\odot k}\bigl((\Sigma^{\odot k})^c\bigr)=0$ by hypothesis, we conclude that for every $\eta \in \U$
$$\pi_{\mssm_r}|_{\U^k(B_r)}((\Omega_r^\eta)^c) = e^{-\mssm_r(B_r)}\mssm_r^{\odot k}\Bigl(\bigl(\Omega_r^\eta\cap \U^k(B_r)\bigl)^c\Bigr) \le  e^{-\mssm_r(B_r)}\mssm_r^{\odot k}\bigl((\Sigma^{\odot k})^c\bigr)=0 \fstop$$
The proof is complete.
% Let $\gamma \in \Omega_r^\eta \cap \U^k(B_r)$.   
%By the definition of~$\Omega$, 
\end{proof}

We recall that for $\eta \in \U$, we set $\U_r^\eta:=\{\gamma \in \U: \gamma_{B_r^c}=\eta_{B_r^c}\}$.
\begin{lem}[disintegration lemma] \label{l:sp}
Assume that there exists a measurable set $\Xi \subset \U$ with $\QP(\Xi)=1$ so that for every $\eta \in \Xi$, there exists a family of measurable sets $\Omega^\eta \subset \U(B_r)$ so that $\mu_{r}^\eta(\Omega^\eta)=1$ for every $\eta \in \Xi$. Let $\Omega \subset \U$ be the (not necessarily measurable) subset defined by
$$\Omega:=\bigcup_{\eta \in \Xi} \pr_{r}^{-1}(\Omega^\eta) \cap \U_{r}^{\eta} \fstop$$ 
Assume further that there exists a measurable set $\Theta \subset \U$ so that $\Omega \subset \Theta$. Then, $\QP(\Theta)=1$. 
\end{lem}
\paragraph{Caveat}As the set $\Omega$ is defined as {\it uncountable union} of measurable sets, the measurability of~$\Omega$ is not necessarily true in general. The disintegration formula~\eqref{p:ConditionalIntegration2} is, therefore, not necessarily applicable directly to $\Omega$,  which motivates the aforementioned lemma. 
\begin{proof}[Proof of Lem.\ \ref{l:sp}]
Let $\Theta_{r}^{\eta}=\{\gamma \in \U(B_r): \gamma+\eta_{B_r^c} \in \Theta\}$ be a section of~$\Theta$ at~$\eta_{B_r^c}$ as in \eqref{e:SEF2}. Then, $\Omega^{\eta} \subset \Theta_{r}^{\eta}$ by assumption. Thus, $\mu_r^\eta(\Theta_{r}^{\eta}) \ge \mu_r^\eta(\Omega^{\eta})\ge 1$. By the disintegration formula in~\eqref{p:ConditionalIntegration2}, we have that 
\begin{align*}
\mu(\Theta)= \int_{\U} \mu_r^\eta(\Theta_{r}^{\eta}) \diff \mu(\eta)  \ge 1\fstop
\end{align*}
The proof is completed. 
\end{proof}



  
  \begin{lem}\label{l:FL}
  Let~$\QP$ be a Borel probability on~$\U$. 
  Let~$\Omega \subset \U$ satisfy $\QP(\Omega)=1$. Then, there exists $\Omega'\subset \Omega$ with $\QP(\Omega')=1$ and 
  \begin{align}\label{e:FL}
  \QP_r^\eta(\Omega_r^\eta)=1\comma  \quad \forall\eta \in \Omega' \fstop
  \end{align}
  \end{lem}
  \begin{proof}
  By the disintegration formula~\eqref{p:ConditionalIntegration2}, 
  $$1=\QP(\Omega)=\int_{\U} \QP_r^\eta(\Omega_r^\eta) \diff \QP(\eta) =\int_{\Omega} \QP_r^\eta(\Omega_r^\eta) \diff \QP(\eta) \comma$$
  by which the statement is readily concluded.
  \end{proof}
  
  \begin{lem} \label{l:MU}
Let $(Q, \dom{Q})$ be a closed form on a separable Hilbert space~$H$. Let $\{T_t\}$ and $(A, \dom{A})$ be the corresponding semigroup and infinitesimal generator respectively. Suppose that there exists an algebra $\mathcal C \subset \dom{Q}$ so that $\mathcal C \subset H$ is dense and $T_t \mathcal C \subset \mathcal C$ for any $t>0$.
% \purple{Then, $(Q, \mathcal C)$ is Markov unique, i.e., there exists at most one Markovian extension of $(Q, \mathcal C)$.} 
Then $\mathcal C$ is dense in~$\dom{Q}$.
\end{lem}
\begin{proof}
It holds that $T_{t}\dom{A} \subset \dom{A}$ by the general property of semigroups associated with closed forms. Thus,  combining it with the hypothesis $T_t \mathcal C \subset \mathcal C$, 
$$ T_{t}(\mathcal C \cap \dom{A}) \subset \mathcal C \cap \dom{A}\fstop$$
 Thus, by \cite[Thm.~X.49]{ReeSim75}, $\mathcal C \cap \dom{A}$ is dense in the graph norm in the space $(A, \dom{A})$. Namely, we obtained
 \begin{align*} 
 \text{$(A, \mathcal C \cap \dom{A})$ is essentially self-adjoint} \fstop
 \end{align*}
% The Markov uniqueness of $(Q, \mathcal C)$ is now a consequence of the essential self-adjointness, see e.g., \cite[p.28]{Ebe99}. 
The density $\mathcal C \subset \dom{Q}$ now follows by the density of $\mathcal C \cap \dom{A}$ in the graph norm, by the density of $\dom{A} \subset \dom{Q}$ due to the general property of closed forms, by the density of $\mathcal C \subset H$ and by a simple integration-by-parts argument
 $$Q(u,u) = (-A u, u)_{H} \le \|A u\|_{H}\|u\|_{H} \fstop$$
 The proof is complete. 
  \end{proof}
 %, we conclude that $\mathcal C_r \cap \mathcal D(\overline{A}_r)$ is dense in $\mathcal D(\bar{\E}_r)$. In particular, $\mathcal C_r$ is dense in $\mathcal D(\bar{\E}_r)$, which concludes the first statement. 
%We now show that $\mathcal C$ is dense in $\dom{Q}$.  
%By general theory of Dirichlet form,  $T_t\dom{A} \subset \dom{A}$ for any $t>0$. Combining it with the hypothesis~$T_t \mathcal C \subset \mathcal C$, we obtain 
%$$T_t( \mathcal C \cap\dom{A} ) \subset  \mathcal C \cap \dom{A} \fstop$$
% Thus, by \cite[Thm.~X.49]{ReeSim75}, we have that $\mathcal C \cap \dom{A}$ is dense in the graph norm in the space $(A, \dom{A})$. Namely, we obtained
% 
% By a simple integration-by-parts argument
% $$-Q(u,u) = (A u, u)_{L^2(\nu)} \le \|A u\|_{L^2(\nu)}\|u\|_{L^2(\nu)} \comma$$
% and by the density of $\mathcal C_r \subset L^2(\QP)$, we conclude that $\mathcal C_r \cap \mathcal D(\overline{A}_r)$ is dense in $\mathcal D(\bar{\E}_r)$. In particular, $\mathcal C_r$ is dense in $\mathcal D(\bar{\E}_r)$, which concludes the first statement. The latter statement regarding the Markov uniqueness is now a consequence of the essential self-adjointness \eqref{e:ESA}, see e.g., \cite[p.28]{Ebe99}.
% By a simple integration-by-parts argument
% $$-Q(u,u) = (A u, u)_{L^2(\nu)} \le \|A u\|_{L^2(\nu)}\|u\|_{L^2(\nu)} \comma$$
% and by the density of $\mathcal C \subset L^2(\nu)$, we conclude that $\mathcal C \cap \dom{A}$ is dense in $\dom{Q}$. In particular, $\mathcal C$ is dense in $\dom{Q}$. Thus, $(Q, \mathcal C)$

\end{appendix}
\bibliographystyle{alpha}
\bibliography{Curvature_submission.bib}
\end{document}












%\paragraph{Irreducibility for conditioned forms}Let $\mcX$ be a \TLDS. Let~$\QP$ be a probability measure on $\ttonde{\dUpsilon,\A_{\mrmv}(\msE)}$ satisfying Assumptions~\ref{ass:Mmu} and~\ref{ass:CE} for some localising exhaustion~$\seq{E_h}_h$ with $E_h \uparrow X$, and  the \emph{conditional closability} assumption~\ref{ass:ConditionalClos}. 
%%Let 
%%\begin{align} \label{eq: condF1}
%%\ttonde{\EE{\dUpsilon(E_h)}{\QP^\eta_{E_h}},\dom{\EE{\dUpsilon(E_h)}{\QP^\eta_{E_h}}}}
%%\end{align}
%%be the corresponding conditioned Dirichlet form on $E_h$ and $\eta \in \dUpsilon$ as defined in \eqref{eq: condF}. 
%\begin{defs}[Irreducibility for the conditioned form]\normalfont\label{ass:ConditionalErgodicity}
%We say that {\it the irreducibility for the conditioned form }\eqref{eq: condF} (in short: \ref{ass:ConditionalErg}) holds if,
%%, for every $h$ and $\QP$-a.e.\ $\eta \in \dUpsilon$,  the conditioned form \eqref{eq: condF} with exhaustion $\seq{E_h}$ is irreducible with respect to $\QP^\eta_{\dUpsilon^{(k)}(E_h)}$ for each $k$. Namely, 
%for every $h \in \N$, $k \in \N_0$ and $\QP$-a.e.\ $\eta \in \dUpsilon$, 
%\begin{align*}\tag*{$(\mathsf{IC})_{\ref{ass:ConditionalErgodicity}}$}\label{ass:ConditionalErg}
%\text{if $u \in \dom{\EE{\dUpsilon(E_h)}{\QP^\eta_{E_h}}}=0$, then $u\mrestr{\dUpsilon^{(k)}(E_h)}=C_{E_h, \eta}^k$ \ $\QP^{\eta, k}_{E_h}$-a.e.} \comma 
%\end{align*}
%where $C_{E_h, \eta}^k$ is a constant depending only on $E_h, \eta$ and $k$.
%%We denote this condition by ${\sf (SL)}_{\dUpsilon(E_h), \mu^\eta_{\dUpsilon(E_h)}}$.
%\end{defs}
%\begin{rem} We give several remarks:
%\begin{enumerate}[$(a)$]
%\item In terms of the corresponding diffusion process,  Assumption~\ref{ass:ConditionalErg} is decoded as the ergodicity of the interacting {\it finite} particles in $\dUpsilon(E_h)$ conditioned at $\eta_{E_h^c}$ outside $E_h$. 
%\item Assumption \ref{ass:ConditionalErg} can be verified for a wide class of invariant measures $\QP$ such as Gibbs measures including Ruelle measures,  and determinantal/permanental point processes including $\mathrm{sine}_\beta$, $\mathrm{Airy}_\beta$, $\mathrm{Bessel}_{\alpha, \beta}$, Ginibre, which will be discussed in \S \ref{sec: Exa}.
%\end{enumerate}
%%to \purple{Write an intuitive understanding of \ref{ass:ConditionalErg}, and the fact that we can verify it for a wide class, which will be discussed in Example section}
%\end{rem}


%making~$\ttonde{\dUpsilon,\mssd_\dUpsilon, \QP}$ into an extended metric measure space.
%\begin{rem}
%\begin{enumerate*}[$(a)$]
%\item We stress that Theorem~\ref{t:ClosabilitySecond} can be proved in many different ways:
%\begin{enumerate*}[$({a}_1)$] 
%\item as an application of the theory of superpositions of Dirichlet forms in~\cite{BouHir91};

%\item as an application of the theory of direct integrals of Dirichlet forms developed by the first named author in~\cite{LzDS20};

%\item by the very same arguments as in the proof of~\cite[Thm.~4]{Osa96}, noting that~\cite[Prop.~4.1]{Osa96} there is replaced by our Assumption~\ref{ass:ConditionalClosability}.
%\end{enumerate*}


%\item It is possible to show that the projected conditional systems~$\set{\QP^\eta_E}_{\eta\in\dUpsilon,E\in\msE}$ are consistent similarly to \emph{specifications} in the sense of Preston~\cite[\S6]{Pre76}.
%\item If the measure~$\QP$ has full $\T_\mrmv(\msE)$-support, the well-posedness of the form~$\ttonde{\EE{\dUpsilon}{\QP},\CylQP{\QP}{\Dz}}$ is immediate, since every function in~$\CylQP{\QP}{\Dz}$ has a unique continuous representative~$ u\in \Cyl{\Dz}$.
%\end{enumerate*}
%\end{rem}





\paragraph{Sobolev-to-Lipschitz for conditioned forms}%In this subsection, we will introduce Sobolev-to-Lipschitz property for conditional probability, which corresponds to finite-particle systems on compact sets. 
Let~$\mcX$ be an \parEMLDS. Let~$\QP$ be a probability measure on~$\ttonde{\dUpsilon,\A_{\mrmv}(\msE)}$ satisfying Assumptions~\ref{ass:Mmu} and~\ref{ass:CE} for some localising exhaustion~$\seq{E_h}_h$ with $E_h \uparrow X$, and  the \emph{conditional closability} assumption~\ref{ass:ConditionalClos}. %Let 
%\begin{align} \label{eq: condF1}
%\ttonde{\EE{\dUpsilon(E_h)}{\QP^\eta_{E_h}},\dom{\EE{\dUpsilon(E_h)}{\QP^\eta_{E_h}}}}
%\end{align}
 %be the corresponding Dirichlet form on $E_h$ and $\eta \in \dUpsilon$ as defined in \eqref{eq: condF}. 
\begin{defs}[Sobolev-to-Lipschitz for the conditioned form]\normalfont \label{ass:ConditionalSobLip}
We say that the Sobolev-to-Lipschitz property for the conditioned form \eqref{eq: condF} (in short: \ref{ass:ConditionalSL}) holds if, for every $h \in \N$, $k \in \N_0$ and $\mu$-a.e.\ $\eta \in \dUpsilon$, if $u \in \dom{\EE{\dUpsilon(E_h)}{\QP^\eta_{E_h}}}$ with $\Gamma^{\dUpsilon(E_h), \QP^\eta_{E_h}}(u) \le \alpha^2$, then there exists $\tilde{u}_{E_h, \eta}^{(k)} \in \Lipua(\mssd_{\dUpsilon^{(k)}(E_h)})$ so that    
\begin{align*}\tag*{$(\mathsf{SLC})_{\ref{ass:ConditionalSobLip}}$}\label{ass:ConditionalSL}
u = \tilde{u}_{E_h, \eta}^{(k)} \quad \text{$\QP_{E_h}^{\eta, k}$-a.e.} \qquad \forall \alpha \ge 0 \fstop
%\text{any $u \in \dom{\EE{\dUpsilon(E_h)}{\QP^\eta_{E_h}}}$ with $\Gamma^{\dUpsilon(E_h)}(u) \le 1$ has a $\QP^\eta_{E_h}$-modification $\tilde{u} \in \Lipu(\mssd_{\dUpsilon(E_h)})$.} 
\end{align*}
%We denote this condition by ${\sf (SL)}_{\dUpsilon(E_h), \mu^\eta_{\dUpsilon(E_h)}}$.
\end{defs}

\begin{rem} \label{rem: SL} We give several remarks:
\begin{enumerate}[$(a)$]
\item 
%By the linearity of $\cdc^{\dUpsilon(E_h), \QP^\eta_{E_h}}$, it is easy to see \ref{ass:ConditionalSL} implies the corresponding statement with the replacements $\Gamma^{\dUpsilon(E_h), \QP^\eta_{E_h}}(u) \le \alpha$ and $\mathrm{Lip}^\alpha(\mssd_{\dUpsilon^{(k)}(E_h)})$ for any $\alpha \ge 0$. 
In particular with $\alpha=0$, we recover Assumption~\ref{ass:ConditionalErg} as $0$-Lipschitz functions with respect to $\mssd_{\dUpsilon^{(k)}(E_h)}$ are constants on $\dUpsilon^{(k)}(E_h)$.  
\item Assumption~\ref{ass:ConditionalSL} can be verified for a wide class of invariant measures $\QP$ such as Gibbs measures including Ruelle measures,  and determinantal/permanental point processes including $\mathrm{sine}_2$, $\mathrm{2}_\beta$, $\mathrm{Bessel}_{\alpha, 2}$, Ginibre, which will be discussed in \S \ref{sec: Exa}.
\end{enumerate}
\end{rem}


\newpage
\subsection{Base spaces}
%Throughout this paper, $(X, \T)$ is a Luzin space, i.e., $(X, \T)$ is a Hausdorff topological space that is a continuous bijective image of a Polish space.
%\subsubsection{Topological local structures}\label{ss:TLS}
We recall that a \emph{Hausdorff} topological space~$(X,\T)$ is
\begin{enumerate*}[$(a)$]
\item \emph{Polish} if there exists a distance~$\mssd$ on~$X$ inducing the topology~$\T$ and additionally so that~$(X,\mssd)$ is a separable and complete metric space;
\item \emph{Luzin} if it is a continuous bijective image of a Polish space.%;
%\item \emph{Suslin} if it is a continuous image of a Polish space.
\end{enumerate*}
%\note{Explain that Luzin is necessary, cf. first paper Rmk 3.8}
Let~$(X,\A,\mssm)$ be a measure space. We denote by~$\A_\mssm\subset \A$ the algebra of sets of finite $\mssm$-measure, and by~$(X,\A^\mssm, \mssm)$ the Carath\'eodory completion of~$(X,\A,\mssm)$ w.r.t.~$\mssm$.
A sub-ring $\msE\subset \A$ is a \emph{ring ideal} of~$\A$, if it is closed under finite unions and intersections and~$A\cap E\in \msE$ for every~$A\in \A$ and~$E\in \msE$.

\paragraph{Topological local structures}Let~$X$ be a non-empty set. A \emph{topological local structure} is a tuple $\mcX\eqdef (X,\T,\A,\mssm,\msE)$ so that
\begin{enumerate}[$(a)$]
\item\label{i:d:TLS:1} $(X,\T)$ is a separable metrizable Luzin space, with Borel $\sigma$-algebra~$\Bo{\T}$;

\item\label{i:d:TLS:2} $\A$ is a $\sigma$-algebra on~$X$, and $\mssm$ is a $\sigma$-finite atomless Radon measure on~$(X,\T,\A)$ with full $\T$-support;

\item\label{i:d:TLS:3} $\Bo{\T}\subset \A\subset \Bo{\T}^\mssm$ and $\A$ is \emph{$\mssm$-essentially countably generated}, i.e.\ there exists a countably generated $\sigma$-sub\-al\-ge\-bra~$\A_0$ of~$\A$ so that for every~$A\in \A$ there exists $A_0\in \A_0$ with~$\mssm(A\triangle A_0)=0$.

\item\label{i:d:TLS:4} $\msE \subset \A_\mssm$ is a \emph{localizing ring}, i.e.\ it is a ring ideal of~$\A$, and there exists a \emph{localizing sequence}~$\seq{E_h}_h\subset \msE$ so that $\msE=\cup_{h\geq 0} (\A\cap E_h)$.

\item\label{i:d:TLS:5} for every~$x\in X$ there exists a $\T$-neighborhood~$U_x$ of~$x$ so that~$U_x\in\msE$.
\end{enumerate}

We refer the readers to \cite[\S2.2]{LzDSSuz21} for details. Our definition of localising ring is a modification of~\cite{Kal17}: in comparison with~\cite[p.~19]{Kal17}, we additionally require~$\mssm E<\infty$ for each~$E$ in~$\msE$.
As noted in~\cite[p.~15]{Kal17}, the datum of a localizing ring is equivalent to that of a localizing sequence~$\seq{E_h}_h$, just by taking as a definition of~$\msE$ the ring ideal generated by the sequence in (d).  
\begin{rem} We give several remarks below: 
\begin{enumerate*}[$(a)$]
\item When~$(X,\T)$ is locally compact,  we can take $\msE$ as the family~$\rKo{\T,\A}$ of all $\A$-measurable relatively $\T$-compact subset of~$X$, noting that all such sets have finite $\mssm$-measure since~$\mssm$ is Radon; 
\item\label{rem:LR2} When~$(X,\T)$ is induced by a metric~$\mssd$ and $\mssm$ takes finite values on every $\mssd$-bounded set, 
%In the concrete cases discussed in \S7, e.g., $X=\R^n$ or $X$ is a complete Riemannian manifold,  as $(X, \T)$ is locally compact, we can take the family of all compact sets in $(X, \T)$ as $\msE$. 
we can choose $\msE$ as the family~$\Ed$ of $\mssd$-bounded sets in~$\A_\mssm$;
%
\item When $\msE=\rKo{\T,\A}$, resp.\ when~$\msE=\Ed$, then~$\Cz(\msE)=\Cc(\T)$, the space of continuous compactly supported functions on~$X$, resp.~$\Cz(\msE)=\Cbs(\mssd)$, the space of continuous bounded functions on~$X$ with $\mssd$-bounded support.
\end{enumerate*}
\end{rem}




%
%\paragraph{Square field operators}Let~$\mcX$ be a topological local structure. For every~$\phi\in \Cb^\infty(\R^k)$ set $\phi_0\eqdef \phi-\phi(\zero)$. A \emph{square field operator on $\mcX$} is a pair~$(\cdc, \Dz)$ so that, for every~$\mbff\in\Dz^\otym{k}$, every~$\phi\in \Cb^\infty(\R^k)$ and every~$k\in \N_0$,
%\begin{enumerate}[$(a)$]
%\item\label{i:d:SF:1} $\Dz$ is a subalgebra of~$L^\infty(\mssm)$ with~$\phi_0\circ\mbff\in \Dz$;
%
%\item\label{i:d:SF:2} $\cdc\colon \Dz^\otym{2}\longrar L^\infty(\mssm)$ is a symmetric non-negative definite bilinear form;
%
%\item\label{i:d:SF:3} $(\cdc,\Dz)$ satisfies the following \emph{diffusion property}
%\begin{align}\label{eq:i:d:SF:3}
%\cdc\tparen{\phi_0\circ \mbff, \psi_0\circ \mbfg}=
%\sum_{i,j}^k (\partial_i \phi)\circ \mbff \cdot (\partial_j\psi)\circ\mbfg \cdot \cdc(f_i, g_j) \as{\mssm}\fstop
%\end{align}
%\end{enumerate}
%
%%Let the analogous definition of a \emph{pointwise defined square field operator}~$(\cdc, \Dz)$ be given, with~$\Dz\subset \mcL^\infty(\mssm)$ in place of~$\Dz\subset L^\infty(\mssm)$, and with~\eqref{eq:i:d:SF:3} to hold pointwise (as opposed to: $\mssm$-a.e.).
%
%
%%A pointwise defined square field operator~$(\cdc, \Dz)$ defines a square field operator~$(\cdc,\Dz)$ on $\mssm$-classes as soon as, cf.~\cite[p.~282]{MaRoe00},
%%\begin{align}\label{eq:Ss}
%%\cdc( f, g)=0\comma \qquad  f, g\in \Dz\comma  f \equiv 0 \as{\mssm}\fstop
%%\end{align}
%
%\begin{defs}[Topological local diffusion spaces,~{\cite[Dfn.~3.7]{LzDSSuz21}}]  \label{d:TLDS}
%A \emph{topological local diffusion space} (in short: \TLDS) is a pair $(\mcX,\cdc)$ so that
%\begin{enumerate}[$(a)$]
%\item\label{i:d:TLDS:1} $\mcX$ is a topological local structure;
%\item\label{i:d:TLDS:2} $\Dz\subset \Cz(\msE)$ is a subalgebra of~$\Cz(\msE)$ generating the topology~$\T$ of~$\mcX$;
%\item\label{i:d:TLDS:3} $\cdc\colon \Dz^\otym{2}\rar L^\infty(\mssm)$ is a square field operator;
%\item\label{i:d:TLDS:4} the bilinear form~$(\EE{X}{\mssm},\Dz)$ defined by
%\begin{align}\label{eq:i:d:DS:3}
%\EE{X}{\mssm}(f,g)\eqdef \int \cdc(f,g) \diff\mssm \comma \qquad f, g\in \Dz\comma
%\end{align}
%is closable and densely defined in~$L^2(\mssm)$;
%\item\label{i:d:TLDS:5} its closure~$\tparen{\EE{X}{\mssm},\dom{\EE{X}{\mssm}}}$ is a quasi-regular Dirichlet form in the sense of e.g. \cite{CheMaRoe94}.
%\end{enumerate}
%\end{defs}
%%We stress that the square field operator~$\cdc$ on a \TLDS takes values in a space of $\mssm$-classes, \emph{not} representatives.
%%
%%Let us introduce some further notation relative to Definition~\ref{d:TLDS}\ref{i:d:TLDS:5}.
%%\begin{notat}\label{n:Form} 
%\paragraph{Quasi notions and broad local domains}We recall necessary notions from potential analysis associated with Dirichlet forms. We refer the readers to e.g., \cite[\S2.4]{LzDSSuz20} for a more complete account of the following contents. 
%Let~$(\mcX,\cdc)$ be a~\TLDS.
%For any~$A\in\Bo{\T}$ set
%$$\dom{\EE{X}{\mssm}}_A\eqdef \set{u\in \domm: u= 0 \ \text{$\mssm$-a.e. on}\ X\setminus A}.$$
%%
%A sequence $\seq{A_n}_n\subset \Bo{\T}$ is a \emph{Borel $\mcE$-nest} if $\cup_n \domm_{A_n}$ is dense in~$\domm$.
%%For any~$A\in\Bo{\T}$, let~$(p)$ be a proposition defined with respect to~$A$. 
%%We say that~`$(p_A)$ holds' if~$A$ satisfies~$(p)$.
%%A \emph{$(p)$-$\mcE$-nest} is a Borel nest~$\seq{A_n}$ so that~$(p_{A_n})$ holds for every~$n$. In particular, 
%A \emph{closed $\mcE$-nest}, henceforth simply referred to as an \emph{$\mcE$-nest}, is a Borel $\mcE$-nest consisting of closed sets.
%A set~$N\subset X$ is \emph{$\mcE$-polar} if there exists an $\mcE$-nest~$\seq{F_n}_n$ so that~$N\subset X\setminus \cup_n F_n$.
%%
%A set~$G\subset X$ is \emph{$\mcE$-quasi-open} if there exists an $\mcE$-nest~$\seq{F_n}_n$ so that~$G\cap F_n$ is relatively open in~$F_n$ for every~$n\in \N$. 
%A set~$F$ is \emph{$\mcE$-quasi-closed} if~$X\setminus F$ is $\mcE$-quasi-open.
%%Any countable union or finite intersection of $\mcE$-quasi-open sets is $\mcE$-quasi-open; analogously, any countable intersection or finite union of $\mcE$-quasi-closed sets is $\mcE$-quasi-closed;~\cite[Lem.~2.3]{Fug71}.
%
%A property~$(p_x)$ depending on~$x\in X$ holds $\mcE$-\emph{quasi-everywhere} (in short:~$\mcE$-q.e.) if there exists an $\mcE$-polar set~$N$ so that~$(p_x)$ holds for every~$x\in X\setminus N$.
%Set
%\begin{align}\label{eq:Xi0}
%\msG\eqdef\set{ G_\bullet\eqdef\seq{G_n}_n :  G_n \text{~$\mcE$-quasi-open,~} G_n\subset G_{n+1} \text{~$\mcE$-q.e.,~} \cup_n G_n=X \qe{\mcE}} \fstop
%\end{align}
%%
%%When~$E=X$ we simply write~$\msG$ in place of~$\msG(X)$.
%%
%For~$G_\bullet\in\msG$,  we say that~$f\in L^0(\mssm)$ is in the \emph{broad local domain}~$\domm(G_\bullet)$ if for every~$n$ there exists~$f_n\in \domm$ so that~$f_n=f$ $\mssm$-a.e.\ on~$G_n$.
%The \emph{broad local space}~$\dotloc{\domm}$ is the space defined as %~\cite[\S4, p.~696]{Kuw98},
%\begin{align*}%\label{eq:Xi}
%\dotloc{\domm}\eqdef \bigcup_{G_\bullet\in \msG} \dotloc{\domm}(G_\bullet) \fstop
%\end{align*}
%%
%%The set~$\dotloc{\msA}(E,G_\bullet)$ depends on~$G_\bullet$. We shall comment extensively on this fact in Remark~\ref{r:AriyoshiHino} below. We omit the specification of~$E$ whenever~$E=X$.
%%\item\label{i:n:Form:1} It is readily verified that~$\tparen{\EE{X}{\mssm},\dom{\EE{X}{\mssm}}}$ admits square field operator~$\tparen{\SF{X}{\mssm},\dom{\SF{X}{\mssm}}}$ with $\dom{\SF{X}{\mssm}}=\dom{\EE{X}{\mssm}}\cap L^\infty(\mssm)$, and extending~$\tparen{\cdc,\Dz}$.
%%
%%Further let:
%%(1) \label{i:n:Form:3} $\tparen{\LL{X}{\mssm}, \dom{\LL{X}{\mssm}}}$ be the generator of~$\tparen{\EE{X}{\mssm},\dom{\EE{X}{\mssm}}}$;
%%
%%(2)\label{i:n:Form:4} $\TT{X}{\mssm}_\bullet\eqdef \tseq{\TT{X}{\mssm}_t}_{t\geq 0}$ be the semigroup of~$\tparen{\LL{X}{\mssm}, \dom{\LL{X}{\mssm}}}$, defined on~$L^p(\mssm)$ for every~$p\in [1,\infty)$.
%%
%%%\item by~$\hh{X}{\mssm}_\bullet\eqdef \tseq{\hh{X}{\mssm}_t(\emparg,\diff\emparg)}_{t\geq 0}$ the corresponding Markov kernel of measures, satisfying
%The square field $\cdc$ can be extended to $\domloc{\EE{X}{\mssm}}$, which is denoted by the same symbol $\cdc$; see e.g.~\cite[\S2.4]{LzDSSuz20}.
%%The \emph{broad local space of functions with $\mssm$-uniformly bounded $\EE{X}{\mssm}$-energy} is the space
%%\begin{align*}
%%\DzLoc{\mssm}\eqdef \set{f\in \domloc{\EE{X}{\mssm}}: \cdc(f)\leq 1 \as{\mssm}}\fstop
%%\end{align*}
%%
%The \emph{broad local space of functions with $\mssm$-uniformly bounded $\cdc$} is the space
%\begin{align*}
%\DzLoc{\mssm}(\alpha)\eqdef \set{f\in \domloc{\EE{X}{\mssm}}: \cdc(f)\leq \alpha \as{\mssm}}\fstop
%\end{align*}
%When $\alpha=1$, then we shortly write $\DzLoc{\mssm}$ instead of $\DzLoc{\mssm}(1)$. Let us additionally set
%%\begin{align*}
%$\DzLocB{\mssm,\T}(\alpha)\eqdef \DzLoc{\mssm}(\alpha)\cap \Cb(\T)$ and $\DzB{\mssm,\T}(\alpha)\eqdef \DzLocB{\mssm,\T}(\alpha)\cap \dom{\EE{X}{\mssm}}.$
%%\end{align*}
%%\begin{align}\label{eq:n:Form:1}
%%\tparen{\TT{X}{\mssm}_t f}(x)=\int f(y)\, \hh{X}{\mssm}_t(x,\diff y)\comma \quad \forallae{\mssm} x\in X\comma \qquad f\in L^2(\mssm)\comma t\geq 0 \semicolon
%%\end{align}
%
%%Finally, let
%%\begin{enumerate*}[$(a)$]\setcounter{enumi}{5}
%%
%%\item\label{i:n:Form:6}$\domext{\EE{X}{\mssm}}$ be the extended Dirichlet space of~$\tparen{\EE{X}{\mssm},\dom{\EE{X}{\mssm}}}$, i.e.\ the space of $\mssm$-classes of functions~$f\colon X\rar \R$ so that there exists an $(\EE{X}{\mssm})^{1/2}$-fun\-da\-men\-tal sequence $\seq{f_n}_n\subset \dom{\EE{X}{\mssm}}$ with $\nlim f_n=f$ $\mssm$-a.e..
%%The form~$\EE{X}{\mssm}$ naturally extends to a quadratic form on~$\domext{\EE{X}{\mssm}}$, denoted by the same symbol~$\EE{X}{\mssm}$, and we always consider~$\domext{\EE{X}{\mssm}}$ as endowed with this extension;
%%
%%\item\label{i:n:Form:7}$\domext{\SF{X}{\mssm}}\eqdef\domext{\EE{X}{\mssm}}\cap L^\infty(\mssm)$ be the extended space of~$\tparen{\SF{X}{\mssm},\dom{\SF{X}{\mssm}}}$, endowed with the non-relabeled extension of~$\SF{X}{\mssm}$.
%%\end{enumerate*}
%%\end{notat}
%
%
%
%
%
%\begin{comment}
%\subsubsection{Liftings}\label{ss:Liftings}
%Let~$\mcX$ be a topological local structure.
%
%\begin{defs}[Liftings]\label{d:Liftings}
%A \emph{lifting} is a map~$\ell\colon \mcL^\infty(\mssm)\rar \mcL^\infty(\mssm)$
%satisfying
%\begin{enumerate*}[$(a)$]
%\item\label{i:d:Liftings:1} $\tclass{\ell(f)}=\class{f}$; 
%\item\label{i:d:Liftings:2} if $\class{f}=\class{g}$, then~$\ell(f)=\ell(g)$;
%\item\label{i:d:Liftings:3} $\ell(\car)=\car$;
%\item\label{i:d:Liftings:4} if~$f\geq 0$, then~$\ell(f)\geq 0$;
%\item\label{i:d:Liftings:5} $\ell(a f+b g)=a\ell(f)+b\ell(g)$ for~$a,b\in \R$;
%\item\label{i:d:Liftings:6} $\ell(fg)=\ell(f)\ell(g)$.
%\end{enumerate*}
%%
%A lifting is \emph{strong} if, additionally,
%\begin{enumerate*}[$(a)$]\setcounter{enumi}{6}
%\item\label{i:d:Liftings:7} $\ell(\phi)=\phi$ for~$\phi\in\Cb(X)$.
%\end{enumerate*}
%%
%A \emph{Borel} \emph{lifting} is a lifting with~$\A=\Bo{\T}$.
%%
%As a consequence of~\iref{i:d:Liftings:1}, a lifting~$\ell\colon \mcL^\infty(\mssm)\rar \mcL^\infty(\mssm)$ descends to a (linear multiplicative order-preserving) right inverse~$\ell\colon L^\infty(\mssm)\rar \mcL^\infty(\mssm)$ of the quotient map~$\class[\mssm]{\emparg}\colon\mcL^\infty(\mssm)\rar L^\infty(\mssm)$.
%As customary, by a (\emph{strong}/\emph{Borel}) \emph{lifting} we shall mean without distinction either~$\ell$ or~$\ell$ as above.
%\end{defs}
%
%The existence of a lifting is a non-trivial consequence of Definition~\ref{d:TLS}\iref{i:d:TLS:3}.
%
%\begin{thm}[{\cite[Thm.~4.12]{StrMacMus02}}]
%Every topological local structure admits a strong (Borel) lifting.
%\end{thm}
%
%Let~$\mcX$ be a topological local structure, and~$(\Gamma,\Dz)$ be a square field operator on~$\mcX$ with~$\Dz\subset \Cb(X)$. 
%We write~$\Dz\subset \Cb(X)$ to indicate that every $\mssm$-class in~$\Dz$ has a continuous representative. For any strong lifting~$\ell\colon L^\infty(\mssm)\rar \mcL^\infty(\mssm)$, we set
%\begin{align}\label{eq:LiftCdC}
%\cdc_\ell(f)\eqdef \ell(\cdc(f)) \comma \qquad f\in\Dz\comma
%\end{align}
%and denote again by~$\cdc_\ell\colon \ell(\Dz)^{\otym 2}\rar \mcL^\infty(\mssm)$ the bilinear form induced by~$\cdc_\ell$ via polarization.
%By the strong lifting property for~$\ell$ we have that~$\ell(\Dz)=\Dz\subset\Cb(X)$.
%
%\begin{prop}\label{p:StrongLiftingCdC} $(\cdc_\ell,\Dz)$ is a pointwise defined square field operator satisfying
%\begin{align*}
%\tclass[\mssm]{\cdc_\ell(\emparg)}=\cdc(\emparg) \qquad \text{on} \qquad \Dz\fstop
%\end{align*}
%\end{prop}
%\end{comment}
%
%%\subsection{Metric structure}
%\begin{comment}
%\begin{defs}[Metric local structure]\label{d:EMLS}
%Let~$\mcX$ be a topological local structure as in Definition~\ref{d:TLS}, and~$\mssd\colon X^\tym{2}\rar[0,\infty]$ be an extended distance. We say that~$(\mcX,\mssd)$ is a \emph{metric local structure}, if
%\begin{enumerate}[$(a)$]
%\item\label{i:d:EMLS:0} $\mcX$ is a topological local structure in the sense of Definition~\ref{d:TLS};
%
%\item\label{i:d:EMLS:1} the space~$(X,\mssd)$ is a \emph{complete}  metric space;% in the sense of Definition~\ref{d:AES};
%
%\item\label{i:d:EMLS:2} $\msE=\msE_\mssd\eqdef \Bdd{\mssd}\cap \A$ is the localizing ring of $\A$-measurable $\mssd$-bounded sets (in particular~$\mssm$ is finite on~$\Ed$).
%\end{enumerate}
%
%If~$\mssd$ is finite, then it metrizes~$\T$,~\iref{i:d:EMLS:1} reduces to the requirement that~$(X,\mssd)$ be a complete metric space, and we say that~$(\mcX,\mssd)$ is a \emph{metric local structure}.
%\end{defs}
%
%The Definition~\ref{d:TLDS} of \TLDS can be now combined with that of a metric local structure above.
%
%\paragraph{Metric local diffusion spaces}
%An \emph{metric local diffusion space} (in \linebreak short: \parEMLDS) is a triple $(\mcX,\cdc,\mssd)$ so that
%\begin{enumerate}[$(a)$]
%\item $(\mcX,\mssd)$ is an (extended) metric local structure;
%\item $(\mcX,\cdc)$ is a \TLDS.
%\end{enumerate}
%\end{comment}
%
%
%\paragraph{Metric local structure}Let~$\mcX$ be a topological local structure, and $\mssd\colon X^\tym{2}\rar[0,\infty]$ be a distance. We say that~$(\mcX,\mssd)$ is a \emph{metric local structure}, if
%\begin{enumerate}[$(a)$]
%\item\label{i:d:EMLS:0} $\mcX$ is a topological local structure; %in the sense of Definition~\ref{d:TLS};
%
%\item\label{i:d:EMLS:1} the distance $\mssd$ is \emph{complete} and induces the topology $\T$;% in the sense of Definition~\ref{d:AES};
%
%\item\label{i:d:EMLS:2} $\msE=\msE_\mssd$ is the localising ring of $\A$-measurable $\mssd$-bounded sets (in particular~$\mssm$ is finite on any~$E \in \Ed$).
%% (see, \ref{rem:LR2} for the definition of $\msE_\mssd$).
%\end{enumerate}
%%
%%If~$\mssd$ is finite, then it metrizes~$\T$,~\iref{i:d:EMLS:1} reduces to the requirement that~$(X,\mssd)$ be a complete metric space, and we say that~$(\mcX,\mssd)$ is a \emph{metric local structure}.
%%Definition \ref{d:TLDS} of \TLDS can be now combined with that of a metric local structure above.
%\begin{defs}[Metric local diffusion spaces]\label{d:EMLDS}
%A \emph{metric local diffusion space} (in short: \parEMLDS) is a triple $(\mcX,\cdc,\mssd)$ so that
%\begin{enumerate}[$(a)$]
%\item $(\mcX,\mssd)$ is a metric local structure;
%\item $(\mcX,\cdc)$ is a \TLDS.
%\end{enumerate}
%\end{defs}
%
%\paragraph{Rademacher property}%In this section, we focus on the interplay between the diffusion-space structure and the metric structure of an \parEMLDS.
%%We refer the reader to~\cite{LzDSSuz20} for a detailed discussion on the subject.
%A function~$f\colon X\rar \R$ is {\it $\mssd$-Lipschitz} if there is a constant~$L>0$ s.t.\
%\begin{align}\label{eq:Lipschitz}
%\abs{f(x)-f(y)}\leq L\, \mssd(x,y) \comma \qquad x,y\in X \fstop
%\end{align}
%The infimal constant~$L$ such that~\eqref{eq:Lipschitz} holds is the (global) \emph{Lipschitz constant of~$f$}, denoted by~$\Li[d]{f}$. We write~$\Lip(\mssd)$, resp.~$\bLip(\mssd)$ for the family of all finite, resp.\ bounded, $\mssd$-Lipschitz functions on~$(X,\mssd)$.
% Finally, set
%$\Lipua(\mssd)\eqdef \set{f\in \Lip(\mssd): \Li[\mssd]{f}\leq \alpha}.$
%%In the following, we shall be concerned with triples~$(X,\T,\mssd)$, where~$(X,\T)$ is a (completely regular Hausdorff) topological space, and~$\mssd$ is a (possibly extended) metric on~$X$ not necessarily compatible with~$\T$.
%%We caution that, on any such space, the standard intuition of $\mssd$-Lipschitz function as an amenable class of functions fails.
%%In particular, let~$f\colon (X,\T,\mssd)\rar \Rex$ be $\mssd$-Lipschitz with~$\Dom(f)\neq \emp$.
%%In general, $f$ is \emph{neither} finite, nor $\T$-continuous, nor $\Bo{\T}$-measurable.
%%For a given $\sigma$-algebra~$\A$, resp.\ a topology~$\T$, on~$X$, this motivates to set
%%\begin{align*}
%%\bLip(\mssd,\A)\eqdef \bLip(\mssd)\cap \Sb(X)\qquad \textrm{and}\qquad \bLip(\mssd,\T)\eqdef \bLip(\mssd)\cap \Cb(\T)\fstop
%%\end{align*}
%%
%%Let the analogous definitions for unbounded Lipschitz functions be given. Finally, set
%%\begin{align*}
%%\Lipu(\mssd)\eqdef \set{f\in \Lip(\mssd): \Li[\mssd]{f}\leq 1} \comma
%%\end{align*}
%%and let the analogous definitions for continuous, measurable, or bounded Lipschitz functions be given.
%\begin{defs}[Rademacher-type property]\label{d:RadSL}
%We say that an \EMLDS  $(\mcX,\cdc,\mssd)$ has the \emph{Rademacher-type property} (in short: \ref{d:RS}) if
%\begin{align*}\tag*{$(\mathsf{Rad}_{\mssd, \mssm})_{\ref{d:RadSL}}$}\label{d:RS}
%\text{$f\in \DzLoc{\mssm}(\alpha^2)$ whenever~$ f\in \Lipua(\mssd)$} \qquad \forall \alpha>0\fstop
%\end{align*}
%%whereby $f\in \DzLoc{\mssm}$ means that  
%%\begin{align}\label{eq:Rad}
%%\tag{$\Rad{\mssd}{\mu}$}  f\in \Lipu(\mssd,\A) \qquad \implies \qquad f\in \DzLoc{\mssm}\semicolon
%%\end{align}
%%
%%\item[($\dRad{\mssd}{\mssm}$)] the \emph{distance-Rademacher property} if~$\mssd\leq \mssd_\mssm$;
%%\begin{align}\label{eq:dRad}
%%\tag{$\dRad{\mssd}{\mu}$} \mssd\leq \mssd_\mssm\fstop
%%\end{align}
%%\item[($\ScL{\mssm}{\T}{\mssd}$)] the \emph{Sobolev--to--continuous-Lipschitz property} if each~$f\in\DzLoc{\mssm}$ has an $\mssm$-representative $ f\in\Lip^1(\mssd,\T)$;
%
%%\note{Maybe remove ScL}
%
%%\item[($\SL{\mssm}{\mssd}$)] the \emph{Sobolev--to--Lipschitz property} if each~$f\in\DzLoc{\mssm}$ has an $\mssm$-representat\-ive $ f\in\Lip^1(\mssd,\A)$;
%
%%\item[($\dcSL{\mssd}{\mssm}{\mssd}$)] the \emph{$\mssd$-continuous-Sobolev--to--Lipschitz property} if each~$f\in \DzLoc{\mssm}$ having a $\mssd$-continuous $\A$-measurable representative~$ f$ also has a representative $two f\in\Lip^1(\mssd,\A^\mssm)$ (possibly,~$two f\neq  f$);
%
%%\item[($\cSL{\T}{\mssm}{\mssd}$)] the \emph{continuous-Sobolev--to--Lipschitz property} if each~$f\in \DzLoc{\mssm,\T}$ satisfies $f\in\Lip^1(\mssd,\T)$;
%
%%\item[($\dSL{\mssd}{\mssm}$)] the \emph{distance Sobolev-to-Lipschitz property} if~$\mssd\geq \mssd_\mssm$.
%%\end{itemize}
%\end{defs}
%
%%Say that:
%%\begin{enumerate}[$(d_1)$]
%%\item a function~$f\colon (X,\A)\rar \R$ is \emph{virtually measurable} if there exists an $\mssm$-negligible set~$N$ such that~$f\colon N^\complement \rar \R$ is $\A_{N^\complement}$-measurable. In this case,~$f$ may be left undefined on~$N$;
%
%%\item a measurable function~$f\colon (X,\A)\rar \R$ is \emph{virtually} $\msE$-\emph{eventually vanishing} if there exists an $\mssm$-negligible set~$N$ such that~$f$ is identically vanishing on~$E^\complement\cap N^\complement$ for some~$E\in\msE$.
%%\end{enumerate}
%%
%%Let \emph{virtually measurable virtually $\msE$-eventually vanishing functions} be defined in the obvious way. We denote the space of all \emph{bounded} such functions by~$\Sb(\msE,\mssm)$.
%
%\subsection{Configuration spaces}
%%Let~$\mcX$ be a topological local structure.
%%Let~$\mcX$ be a topological local structure. 
%%Let $\R^n$ be the $n$-dimensional Euclidean space, $\mssd$ be the standard Euclidean metric, $\tau$ be the topology generated by open $\mssd$-balls, $\A$ be the Borel $\sigma$-algebra, $\mssm$ be the $n$-dimenisonal Lebesgue measure, $\msE$ be the family of all compact sets in $\R^n$. We write $\mcX:=(\R^n, \tau, \A, \mssd, \mssm, \msE)$. 
%A (\emph{multiple}) \emph{configuration} on~$\mcX$ is any $\overline\N_0$-valued measure~$\gamma$ on~$(X,\A)$, finite on~$E$ for every~$E\in\msE$. Let $\delta_{x}$ denote the Dirac measure at $x$, i.e., $\delta_x(A)=1$ if and only if $x \in A$.  By assumption on~$\mcX$, cf.\ e.g.~\cite[Cor.~6.5]{LasPen18},
%\begin{align*}
%\gamma= \sum_{i=1}^N \delta_{x_i} \comma \qquad N\in \overline\N_0\comma \qquad \seq{x_i}_{i\leq N}\subset X \fstop
%\end{align*}
%In particular, we allow for~$x_i=x_j$ when~$i\neq j$.
%%
%Write~$\gamma_x\eqdef\gamma\!\set{x}$,~$x\in \gamma$ whenever~$\gamma_x>0$, and the {\it restriction}~$\gamma_A\eqdef \gamma\mrestr{A}$ for every~$A\in\A$. We write the {\it evaluation} $\gamma A := \gamma(A)$ for a subset $A \subset \purple{\dUpsilon}$\footnote{$A \subset X$}.
%
%\begin{defs}[Configuration spaces] The \emph{multiple configuration space}~$\dUpsilon=\dUpsilon(X,\msE)$ is the space of all (multiple) configurations over~$\mcX$. The \emph{configuration space} is the space
%\begin{align*}
%\Upsilon(\msE)=&\Upsilon(X,\msE)\eqdef\set{\gamma\in\dUpsilon: \gamma_x\in \set{0,1} \text{~~for all } x\in X} \fstop
%\end{align*}
%%
%The $N$-particles multiple configuration spaces, resp.~configuration spaces, are
%\begin{equation*}
%\begin{aligned}
%\dUpsilon^\sym{N}\ \eqdef&\ \set{\gamma\in \dUpsilon : \gamma X=N}\comma 
%\\
%\text{resp.} \qquad \Upsilon^\sym{N}\ \eqdef&\ \dUpsilon^\sym{N}\cap \Upsilon\comma 
%\end{aligned}
%\qquad N\in\overline\N_0 \fstop
%\end{equation*}
%Let the analogous definitions of~$\dUpsilon^\sym{\geq N}(\msE)$, resp.~$\Upsilon^\sym{\geq N}(\msE)$, be given.
%For fixed~$A\in\A$ further set~$\msE_A\eqdef \set{E\cap A: E\in\msE}$ and
%\begin{align}\label{eq:ProjUpsilon}
%\pr^A\colon \dUpsilon\longrar \dUpsilon(\msE_A)\colon \gamma\longmapsto \gamma_A\comma \qquad A\in\A \fstop
%\end{align}
%Finally set~$\dUpsilon(E)\eqdef \dUpsilon(\msE_E)=\dUpsilon(\A_E)$ for all~$E\in\msE$, and analogously for~$\Upsilon(E)$.
%We endow~$\dUpsilon$ with the $\sigma$-algebra $\A_\mrmv(\msE)$ generated by the functions $\gamma\mapsto \gamma E$ with~$E\in\msE$, and endow~$\dUpsilon$ with the vague topology $\T_\mrmv$, i.e., the convergence in the duality of $\Cz(\msE)$. 
%\end{defs}
%
%%This coincides with the $\sigma$-algebra on~$\dUpsilon$ given in~\cite[Rmk.~1.5]{MaRoe00} because of Definition~\ref{d:LS}\iref{i:d:LS:2}.
%
%%\begin{defs}[Concentration set]%
%%For~$E\in\msE$ we define the $n$-concentration sets of~$E$ as
%%\begin{equation}\label{eq:ConcentrationSet}
%%\Xi_{=n}(E)\eqdef\ \set{\gamma\in\dUpsilon: \gamma E= n} \comma
%%\end{equation}
%%and similarly for~`$\geq$' or~`$\leq$' in place of~`$=$'.
%%Analogously to the notation established for configuration spaces, we write~$\Xi^\sym{\infty}_{=n}(E)=\Xi_{=n}(E)\cap \dUpsilon^\sym{\infty}$.
%%\end{defs}
%
%
%
%\paragraph{Poisson random measures}
%%Contrary to~\cite{LzDSSuz21}, in this work we restrict our attention to configuration spaces endowed with Poisson measures.
%For~$E\in\msE$ set~$\mssm_E\eqdef \mssm\mrestr{E}$.
%%
%Let~$\mfS_n$ be the $n^\text{th}$ symmetric group, and denote by
%%\begin{equation}\label{eq:ProjSymmetricG}
%$\pr^\sym{n}\colon E^\tym{n}\rar E^\sym{n}\eqdef E^\tym{n}/\mfS_n$
%%\end{equation}
%the quotient projection, and by~$\mssm_E^\sym{n}$ the quotient measure~$\mssm_E^\sym{n}\eqdef \pr^\sym{n}_\pfwd \mssm_E^{\purple{\hotym{n}}}$.
%The space~$E^\sym{n}$ is naturally isomorphic to~$\dUpsilon^\sym{n}(E)$.
%Under this isomorphism, define the \emph{Poisson--Lebesgue measure of~$\PP_\mssm$} on~$\dUpsilon^{\sym{<\infty}}(E)$, cf.\ e.g.~\cite[Eqn.~(2.5),~(2.6)]{AlbKonRoe98}, by
%\begin{align}\label{eq:PoissonLebesgue}
%\PP_{\mssm_E}\eqdef e^{-\mssm E}\sum_{n=0}^\infty \frac{\mssm_E^\sym{n}}{n!} \fstop
%\end{align}
%%\begin{defs}[Poisson measures]\label{d:Poisson}
%The \emph{Poisson} (\emph{random}) \emph{measure~$\PP_\mssm$} with intensity~$\mssm$ is the unique probability measure on~$\tparen{\dUpsilon, \A_\mrmv(\msE)}$ satisfying% either of the following equivalent conditions:
%\begin{align}\label{eq:PoissonRestriction}
%\pr^{E}_\pfwd \PP_\mssm=\PP_{\mssm_E}\comma \qquad E\in\msE \fstop
%\end{align}
%%\begin{align}\label{eq:LaplacePoisson}
%%\int_{\dUpsilon} e^{f^\trid\gamma} \diff\PP_\mssm(\gamma)=\exp\paren{\int_X (e^{f}-1) \diff\mssm}\comma \qquad f\in \Sb(X)^+\fstop
%%\end{align}
%%
%%\begin{itemize}
%%\item the \emph{projective-limit characterization} 
%%\begin{align}\label{eq:PoissonRestriction}
%%\pr^{E}_\pfwd \PP_\mssm=\PP_{\mssm_E}\comma \qquad E\in\msE \fstop
%%\end{align}
%%\item the \emph{Mecke identity}~\cite[Satz~3.1]{Mec67}, viz.
%%\begin{align}\label{eq:Mecke}
%%\iint_{\dUpsilon\times X} u(\gamma, x) \diff\gamma(x) \diff\PP_\mssm(\gamma)= \iint_{\dUpsilon\times X} u(\gamma+\delta_x,x) \diff\mssm(x) \diff\PP_\mssm(\gamma)
%%\end{align}
%%for every bounded $\A_\mrmv(\msE)\hotimes \A$-measurable~$u\colon \dUpsilon\times X\rar \R$.
%%\item the \emph{Laplace transform characterization}, cf.~\cite[Thm.~3.9]{LasPen18},
%%\begin{align}\label{eq:LaplacePoisson}
%%\int_{\dUpsilon} e^{f^\trid\gamma} \diff\PP_\mssm(\gamma)=\exp\paren{\int_X (e^{f}-1) \diff\mssm}\comma \qquad f\in \Sb(X)^+\fstop
%%\end{align}
%%\end{itemize}
%%\end{defs}
%We note that 
%%The Poisson measures has a different concentration set according to the finiteness of the total measure of $\mssm$:
%%\begin{itemize}
%%\item 
%if~$\mssm X=\infty$, then~$\PP_\mssm$ is concentrated in~$\Upsilon^\sym{\infty}(\msE)$, viz.\ $\PP_\mssm \Upsilon^\sym{\infty}(\msE)=1$;
%if~$\mssm X<\infty$, then~$\PP_\mssm$ is concentrated in~$\Upsilon^\sym{<\infty}(\msE)$, viz.\ $\PP_\mssm \Upsilon^\sym{<\infty}(\msE)=1$, and~$\PP_\mssm$ coincides with the right-hand side of~\eqref{eq:PoissonLebesgue} with~$X$ in place of~$E$.
%%\end{itemize}
%Everywhere in the following we omit the specification of the intensity measure whenever apparent from context, thus writing~$\PP$ in place of~$\PP_\mssm$, and~$\PP_E$ in place of~$\PP_{\mssm_E}$. 
%
%\paragraph{Intensity} For a probability measure $\QP$ on the space $\ttonde{\dUpsilon,\A_\mrmv(\msE)}$, we denote by~$\mssm_\QP$ its  (\emph{mean}) \emph{intensity} (\emph{measure}), defined by
%\begin{align} \label{eq: int}
%\mssm_\QP A \eqdef \int_\dUpsilon \gamma A \diff\QP(\gamma) \comma \qquad A\in\A_\mrmv(\msE)\fstop
%\end{align}
%%For many results here and later on, we shall make the following assumption.
%\begin{ass}\label{d:ass:Mmu}
%We say that~$\QP$ satisfies \ref{ass:Mmu} if it has $\msE$-locally finite intensity, viz.\
%\begin{equation}\tag*{$(\mssm_\QP)_{\ref{d:ass:Mmu}}$}\label{ass:Mmu}
%\mssm_\QP E<\infty \comma \quad E\in\msE\fstop
%\end{equation}
%\end{ass}
%%\begin{ese} \label{exa: FI}
%Assumption~\ref{ass:Mmu} is a natural condition from the following viewpoints:
%from a mathematical point of view, it implies that we have sufficiently many functions in~$L^1(\QP)$;
%from a physical point of view, it implies that any system of randomly $\QP$-distributed particles is $\msE$-locally finite in average. 
%\purple{Example}
%\end{ese}
%\subsubsection{Dirichlet forms}
%Let us briefly recall the construction and main analytical properties of the Dirichlet form~\eqref{eq:DirichletForm} constructed in~\cite{LzDSSuz21}.
%We specialize all the statements in~\cite{LzDSSuz21} to the case of our interest, namely that of Poisson measures.


\subsection{Dirichlet forms on $\dUpsilon$}
We recall the construction of Dirichlet forms on $\dUpsilon$. Let~$(\mcX,\cdc)$ be a~\TLDS. 
%Throughout this section,~$\QP$ will denote a probability measure on the space $\ttonde{\dUpsilon,\A_\mrmv(\msE)}$.



\paragraph{Cylinder functions}%We shall start by defining a core of \emph{cylinder functions} for the form \eqref{eq:DirichletForm}.
For~$\gamma\in \dUpsilon$ and~$ f\in\Sb(\msE)$ let~$ f^\trid\colon \dUpsilon\rar \R$ be defined as 
\begin{equation}\label{eq:Trid}
 f^\trid\colon \gamma\longmapsto \int_X  f(x)\diff\gamma(x) 
\end{equation}
and set further
\begin{align*}
\mbff^\trid\colon \gamma\longmapsto\tparen{ f_1^\trid\gamma,\dotsc,  f_k^\trid\gamma}\in \R^k\comma \qquad  f_1,\dotsc,  f_k\in \Sb(\msE)\fstop
\end{align*}
%\begin{defs}[Cylinder functions on~$\dUpsilon$] 
%Let~$\mcX$ be a topological local structure, and~$ D$ be a linear subspace of~$\Sb(\msE)$. 
We define the space of \emph{cylinder functions}
\begin{align} \label{defn: cyl}
\Cyl{ D}\eqdef \set{\begin{matrix}  u\colon \dUpsilon\rar \R :  u=F\circ\mbff^\trid \comma  F\in \mcC^\infty_b(\R^k)\comma \\  f_1,\dotsc,  f_k\in  \mathcal D\comma\quad k\in \N_0 \end{matrix}}\fstop
\end{align}
%\end{defs}
It is readily seen that cylinder functions of the form~$\Cyl{\Sb(\msE)}$ are $\A_\mrmv(\msE)$-measurable.
If~$\mathcal D$ generates the $\sigma$-algebra~$\A$ on~$X$, then~$\Cyl{\mathcal D}$ generates the $\sigma$-algebra~$\A_\mrmv(\msE)$ on~$\dUpsilon$. We stress that the representation of~$ u$ by~$ u=F\circ\mbff$ is \emph{not} unique.

\paragraph{Lifted square field operators}By the result~\cite[Lem.\ 2.16, Thm.\ 3.48]{LzDSSuz21}, there exists the square field operator~$\cdc^{\dUpsilon, \purple{\QP}}$\footnote{$\QP$ has not been introduced in the previous sections. So this paragraph should be placed after the definition of the conditional equivalence.} on~$\dUpsilon$ lifted from the square field~$\cdc$ on the base space~$\mcX$, i.e., $\cdc^{\dUpsilon, \QP}$ is described by $\cdc$ in the following form on $\Cyl{\Dz}$:
%, and it satisfies the Leibniz rule on $\Cyl{\Dz}$ \ $\QP$-a.e.:
\begin{equation}\label{eq:d:LiftCdCRep}
\begin{gathered}
\cdc^{\dUpsilon, \QP}(u,  u)(\gamma) = \sum_{x\in \gamma} \gamma_x^{-1}\cdot \cdc\tonde{u\ttonde{\car_{X\setminus\set{x}}\cdot\gamma + \gamma_x\delta_\bullet}-u\ttonde{\car_{X\setminus\set{x}}\cdot\gamma}}(x), \quad \text{$\QP$-a.e.\ $\gamma$} \comma
%\sum_{i,j=1}^{k,m} (\partial_i F)(\mbff^\trid\gamma) \cdot (\partial_j G)(\mbfg^\trid\gamma) \cdot \cdc( f_i,  g_j)^\trid \gamma \quad \text{$\QP$-a.e.\ $\gamma$}\comma
% u, v\in \Cyl{\Dz} \fstop
\end{gathered}
\end{equation}
and the square field $\cdc^{\dUpsilon, \QP}$ satisfies the following {\it diffusion property} (in other words, chain rule):
\begin{equation}\label{eq:d:LiftCdCRep1}
\begin{gathered}
\cdc^{\dUpsilon, \QP}(u,  v)(\gamma) =\sum_{i,j=1}^{k,m} (\partial_i F)(\mbff^\trid\gamma) \cdot (\partial_j G)(\mbfg^\trid\gamma) \cdot \cdc( f_i,  g_j)^\trid \gamma \quad \text{$\QP$-a.e.\ $\gamma$}\comma 
\end{gathered}
\end{equation}
for  $u=F\circ \mbff^\trid$ and $v=G\circ \mbfg^\trid\in \Cyl{\Dz}$.
\begin{rem} \label{rem: lift}
If the square field operator~$\cdc$ on the base space~$\mcX$ maps $\cdc: \Dz^{\otimes 2} \to \A_b(X)$ (as opposed to $\cdc: \Dz^{\otimes 2} \to L^\infty(\mssm)$), then the r.h.s.~of~\eqref{eq:d:LiftCdCRep} or of~\eqref{eq:d:LiftCdCRep1} per se can work as the definition of~$\cdc^{\dUpsilon}(u, v)(\gamma)$ for {\it every} $\gamma$ (as opposed to {\it a.e.}\ $\gamma$), see \cite[Lem.\ 1.2]{MaRoe00}. In particular, it is enough for the readers who are interested only in the case where $\mcX$ is a smooth Riemannian manifold~$(M, g)$ with the standard squire field $\cdc(\cdot)=|\nabla \cdot|_g^2$ induced by $g$, see \cite{AlbKonRoe98, AlbKonRoe98b}. 

However, if $\mcX$ is a singular space such as a metric measure space equipped with the minimal weak upper gradient (e.g., in the sense of \cite[Def. 5.11]{AmbGigSav14}), the square field~$\cdc$ in the base space $\mcX$ is defined only $\mssm$-a.e.\ even on the core $\Dz$, i.e., it holds only $\cdc: \Dz^{\otimes 2} \to L^\infty(\mssm)$, in which case the pointwise definition of $\cdc( f_i,  g_j)^\trid \gamma$ in the r.h.s.~of~\eqref{eq:d:LiftCdCRep} loses its meaning since countably many points, on which configurations are supported, are negligible sets for any non-atomic measures $\mssm$. In this case, we need further arguments to justify the formula \eqref{eq:d:LiftCdCRep}. For so doing in \cite{LzDSSuz21}, we rely upon {\it lifting map} $\ell: L^\infty(\mssm) \to \mathcal L^\infty$, which selects representative of $\cdc( f_i,  g_j)$ in a systematic way in the sense that $\ell$ is an order-preserving homomorphism.  As these arguments are quite technical,  we refer the readers to \cite[\S2.3.1, \S3.4.2]{LzDSSuz21} for details.
\end{rem}
%By~\cite[Lem.~1.2]{MaRoe00} the bilinear form~$\cdc^\dUpsilon$ is well-defined on~$\Cyl{\Dz}^\tym{2}$, in the sense that~$\cdc^\dUpsilon( u,  v)$ does not depend on the choice of representatives $ u=F\circ\mbff$ and $ v=G\circ\mbfg$ for~$ u$ and~$ v$.
%We refer the reader to~\cite[\S.1]{MaRoe00} and~\cite[\S2.3.1]{LzDSSuz21} for details on the issue of well-posedness on cylinder functions.
%For the pointwise defined square field~$\cdc_\ell$ corresponding to a strong lifting~$\ell$ on~$(\mcX,\cdc)$ let us set
%Since~$\cdc^{\dUpsilon}_\ell( u,  v)(\gamma)$ is everywhere well-defined in the sense above,~$\EE{\dUpsilon}{\PP}_\ell$ is a well-defined bilinear form on the space of representatives~$\mcL^2(\PP)$.
%It is however not obvious that the $\PP$-class~$\cdc^{\dUpsilon}_\ell(u, v)$ is defined independently of the chosen representatives~$ u$,~$ v$ of the corresponding $\PP$-classes functions~$u$,~$v$.
%
%It is shown in~\cite{LzDSSuz21} that~$\cdc^\dUpsilon_\ell$ descends to a bilinear symmetric functional~$\cdc^\dUpsilon_\ell$ on the space of $\PP$-classes $\CylQP{\PP}{\Dz}$ of the cylinder functions $\Cyl{\Dz}$, which proves that~$\EE{\dUpsilon}{\PP}_\ell$ descends to a non-relabeled well-defined pre-Dirichlet form on~$L^2(\PP)$.
%
%The closure of this form will be the main object of our study throughout this work.


\begin{comment}
\begin{prop}[Closability,~{\cite[Prop.~3.9]{LzDSSuz21}}]\label{p:MRLifting}
Let~$(\mcX,\cdc)$ be a \TLDS, and~$\ell$ be a strong lifting.
Then, the form
\begin{align*}
\tparen{\EE{\dUpsilon}{\PP},\CylQP{\PP}{\Dz}}=\tparen{\EE{\dUpsilon}{\PP}_\ell,\CylQP{\PP}{\Dz}}
\end{align*}
is well-defined, densely defined and closable on~$L^2(\PP)$, and independent of~$\ell$.
Its closure $\tparen{\EE{\dUpsilon}{\PP},\dom{\EE{\dUpsilon}{\PP}}}$ is a Dirichlet form with carr\'e du champ operator~$\tparen{\SF{\dUpsilon}{\PP},\dom{\SF{\dUpsilon}{\PP}}}$ satisfying
\begin{equation*}
\SF{\dUpsilon}{\PP}(u,v)=\cdc^\dUpsilon(u,v) \as{\PP}\comma\qquad u,v\in \CylQP{\PP}{\Dz}\fstop
\end{equation*}
\end{prop}

We denote by
\begin{align*}
\tparen{\LL{\dUpsilon}{\PP},\dom{\LL{\dUpsilon}{\PP}}}\comma \qquad \text{resp.} \qquad \TT{\dUpsilon}{\PP}_\bullet\eqdef\tseq{\TT{\dUpsilon}{\PP}_t}_{t\geq 0}\comma
\end{align*}
the $L^2(\PP)$-generator, resp.~$L^2(\PP)$-semigroup, corresponding to~$\tparen{\EE{\dUpsilon}{\PP},\dom{\EE{\dUpsilon}{\PP}}}$.

The domain~$\dom{\EE{\dUpsilon}{\PP}}$ is in fact much larger than the class of cylinder functions~$\CylQP{\PP}{\Dz}$, as recalled below.
%
Let~$\coK{\Dz}{1,\mssm}$ be the abstract linear completion of~$\Dz$ w.r.t.~the norm (cf.~\cite[p.~301]{MaRoe00})
\begin{align*}
\norm{\emparg}_{1,\mssm}\eqdef \EE{X}{\mssm}(\emparg)^{1/2} + \norm{\emparg}_{L^1(\mssm)} 
\comma \end{align*}
endowed with the unique (non-relabeled) continuous extension of~$\norm{\emparg}_{1,\mssm}$ to the completion $\coK{\Dz}{1,\mssm}$.

%\begin{defs}[{\cite[\S4.2, p.~300]{MaRoe00}}] A function~$ u\colon \dUpsilon\rar \R\cup\set{\pm\infty}$ is called \emph{extended cylinder} if there exist~$k\in \N_0$, functions~$ f_1,\dotsc,  f_k$ with~$f_1,\dotsc,f_k\in \coK{\Dz}{1,\mssm}$, and a function~$F\in \Cb^\infty(\R^k)$, so that~$ u=F\circ\mbff$.
%We denote by~$\Cyl{\coK{\Dz}{1,\mssm}}$ the space of all extended cylinder functions, and by~$\CylQP{\PP}{\coK{\Dz}{1,\mssm}}$ the space of their $\PP$-representa\-tives.
%\end{defs}
\end{comment}

\begin{comment}

\begin{prop}[{\cite[Prop.~4.6]{MaRoe00}}, {\cite[Prop.~3.52]{LzDSSuz21}}]\label{p:ExtDom}
Let~$(\mcX,\cdc)$ be a \TLDS.
Then,
\begin{enumerate}[$(i)$]
\item\label{i:p:ExtDom:1}~$u=\tclass[\PP]{F\circ \mbff^\trid}\in \CylQP{\PP}{\coK{\Dz}{1,\mssm}}$ is defined in~$\dom{\EE{\dUpsilon}{\PP}}$ and independent of the $\mssm$-rep\-re\-sen\-ta\-tives~$ f_i$ of~$f_i$;

\item\label{i:p:ExtDom:2} for~$ u=F\circ \mbff^\trid$ and $ v=G\circ \mbfg^\trid\in \Cyl{\coK{\Dz}{1,\mssm}}$,
\begin{align}\label{eq:d:LiftCdC}
\SF{\dUpsilon}{\PP}(u,v)(\gamma)=& \sum_{i,j=1}^{k,m} (\partial_i F)(\mbff^\trid\gamma) \cdot (\partial_j G)(\mbfg^\trid\gamma) \cdot \cdc( f_i,  g_j)^\trid \gamma \as{\PP} \semicolon
\end{align}

\item\label{i:p:ExtDom:3} for every~$ f$ with~$f \in \coK{\Dz}{1,\mssm}$ one has~$\ttclass[\PP]{ f^\trid}\in\CylQP{\PP}{\coK{\Dz}{1,\mssm}}$, and
\begin{align*}
\SF{\dUpsilon}{\PP}\tparen{\ttclass[\PP]{ f^\trid}}=\class[\PP]{\cdc( f)^\trid}
\end{align*}
is independent of the chosen $\mssm$- (i.e.,~$\mssm$-)representative~$ f$ of~$f$.
\end{enumerate}
\end{prop}
\end{comment}


\paragraph{Conditional Closability and Closability}%In order to state the conditional closability, we introduce the projected conditional measures.
%Let~$(\mcX,\cdc)$ be a \TLDS, and recall that~$\cdc$ is assumed to be defined on~$\Dz\subset \Cz(\msE)$.
%In particular, we shall always assume that~$f\in\Dz$ is identified with its continuous representative in~$\Dz=\ell(\Dz)$, where $\ell\colon L^\infty(\mssm)\to\mcL^\infty(\mssm)$ is any strong lifting.
%\begin{defs}[Projected conditional measures]\label{d:ConditionalQP}
For a probability measure~$\QP$ on~$\ttonde{\dUpsilon,\A_{\mrmv}(\msE)}$ and $E \in \msE$, set {\it the restricted measure on $\{\gamma \in \dUpsilon: \gamma E = k\}$}:
\begin{align}\label{defn: RP}
\QP^{k, E}(\cdot):=\QP\bigl(\cdot \cap \{\gamma \in \dUpsilon: \gamma E = k\} \bigr) \fstop
\end{align}
For each fixed~$\eta\in\dUpsilon$, and for each fixed~$E\in\msE$, we let~$\QP^{\eta_{E^\complement}}$ be the regular conditional probability strongly consistent with~$\pr^{E^\complement}$ (i.e., $(\pr^{E^\complement})^{-1}(\eta)$ is co-negligible with respect to $\QP^{\eta_{E^\complement}}$)\footnote{Write a reference for this definition, e.g., the first paper.}. By the definition of the conditional probability, it holds 
\begin{equation}\label{eq:ConditionalQP}
\QP\tonde{\Lambda\cap (\pr^{E^\complement})^{-1}(\Xi)}=\int_\Xi \QP^{\eta_{E^\complement}} \Lambda \, \diff\QP(\eta) \comma
\end{equation}
for any $\Lambda, \Xi \in \A_{\mrmv}(\msE)$ where, with slight abuse of notation, we regard~$\dUpsilon(E)$ as a subset of~$\dUpsilon$, and thus~$\pr^{E^\complement}$ as a map~$\pr^{E^\complement}\colon\dUpsilon\to\dUpsilon$.
In probabilistic notation,
\begin{equation*}
\QP^{\eta_{E^\complement}}=\QP\ttonde{\emparg \big |\, \pr^{E^\complement}(\emparg)=\eta_{E^\complement}} \fstop
\end{equation*}
The \emph{projected conditional probabilities of~$\QP$} are the system of measures on $\dUpsilon(E)$:
\begin{equation}\label{eq:ProjectedConditionalQP}
\QP^\eta_E\eqdef \pr^E_\pfwd \QP^{\eta_{E^\complement}}\comma \qquad \eta\in\dUpsilon\comma \qquad E\in\msE \fstop
\end{equation}
%\end{defs}
The restriction of $\QP^\eta_E$ on $\dUpsilon^{(k)}(E_h)$ is denoted by
\begin{equation}\label{eq:ProjectedConditionalQP1}
\QP^{\eta, k}_E\eqdef \QP^\eta_E\mrestr{\dUpsilon^{(k)}(E_h)}\comma \qquad \eta\in\dUpsilon\comma \qquad E\in\msE \fstop
\end{equation}
%\purple{Set\footnote{Definition of quasi-Gibbs should be more rigorous in relation to rigidity} 
%\begin{equation*}
%\QP^{\eta_{E^\complement}, k}(d\gamma)=\QP\ttonde{d\gamma \in \dUpsilon \big |\, \pr^{E^\complement}(\gamma)=\eta_{E^\complement}\,, \, \gamma E=k} \comma
%\end{equation*}
%and 
%\begin{equation*}
%\QP^{\eta, k}_{E}(d\gamma)= \pr^E_\pfwd \QP\ttonde{d\gamma \in \dUpsilon(E)\big |\, \pr^{E^\complement}(\gamma)=\eta_{E^\complement}\,, \, \gamma E=k} \fstop
%\end{equation*}
%}
\begin{defs}[Conditional equivalence]\label{d:ConditionalAC}
A probability measure~$\QP$ on $\ttonde{\dUpsilon,\A_{\mrmv}(\msE)}$ 
%\emph{conditionally absolutely continuous} (\emph{w.r.t.~$\PP_\mssm$}) if there exists a localizing sequence~$\seq{E_h}_h$ such that the projected conditional probabilities satisfy
%\begin{equation}\tag*{$(\mathsf{CAC})_{\ref{d:ConditionalAC}}$}
%\label{ass:CAC}
%\forallae{\QP} \eta\in\dUpsilon \qquad \QP^\eta_{E_h} \ll \PP_{\mssm_{E_h}} \comma \qquad h\in\N \fstop
%\end{equation}
is \emph{conditionally equivalent} (\emph{to~$\PP_\mssm$}) if there exists a localising exhaustion $\{E_h\}_{h \in \N} \subset \msE$ with $E_h \uparrow X$ so that, for any $E \in \{E_h\}_{h \in \N}$, $k \in \N_0$ and for $\QP^{k, E}$-a.e.\ $\eta$,
\begin{equation}\tag*{$(\mathsf{CE})_{\ref{d:ConditionalAC}}$}
\label{ass:CE}
 \QP^{\eta, k}_{E} \sim \PP_{\mssm_{E}}\mrestr{\dUpsilon^{(k)}(E)}  \fstop
\end{equation}
\end{defs}
\begin{rem}[{Comparison with \cite[Def. 3.41]{LzDSSuz21}}] \label{rem: CP1} We compare  \ref{ass:CE} with \cite[Def. 3.41]{LzDSSuz21}.
\begin{enumerate}[$(a)$]
\item Assumption \ref{ass:CE} is slightly weaker than Assumption (CE) given in \cite[Def. 3.41]{LzDSSuz21}: Assumption \ref{ass:CE}  requires the equivalence of the measures for $\QP^{k, E}$-a.e.\ $\eta$ while \cite[Def. 3.41]{LzDSSuz21} requires it for $\QP$-a.e.\ $\eta$. 
\item However, Assumption \ref{ass:CE} implies Assumption (CAC) in \cite[Def. 3.41]{LzDSSuz21}, i.e., for any $E \in \msE$, $k \in \N_0$ and for $\QP$-a.e.\ $\eta$ ({\it not only $\QP^{k, E}$-a.e.\ $\eta$}), 
\begin{equation*}
 \QP^{\eta, k}_{E} \ll \PP_{\mssm_{E}}\mrestr{\dUpsilon^{(k)}(E)}  \fstop
\end{equation*}
This can be  immediately seen by noting that $ \QP^{\eta, k}_{E}$ is the zero measure on $\dUpsilon^{(k)}(E)$ whenever $\eta$ does not belong to the set 
$${\rm pr}_{E^c}^{-1}\circ {\rm pr}_{E^c}\bigl(\{\gamma \in \dUpsilon: \gamma E=k\}\bigr).$$
\end{enumerate}
\end{rem}
%\begin{equation}\tag*{$(\mathsf{CE})_{\ref{d:ConditionalAC}}$}
%\label{ass:CE}
%\forallae{\QP} \eta\in\dUpsilon \qquad \QP^\eta_{E}\mrestr{\dUpsilon^{(k)}(E)} \sim \PP_{\mssm_{E}}\mrestr{\dUpsilon^{(k)}(E)}  \fstop
%\end{equation}
%\end{defs}
\smallskip
\begin{comment}
\begin{lem}\label{l:Isometry}
The map~$\emparg^\trid\colon  f\mapsto  f^\trid$ descends to an isometric order-preserving embedding
\begin{align*}
\emparg^\trid\colon L^1(\mssm)\longrar L^1(\PP)\fstop
\end{align*}
\begin{proof}
Let~$ f\in\mcL^1(\mssm)^+$. By a simple application of~\eqref{eq:Mecke}, we have that~$\ttnorm{ f^\trid}_{\mcL^1(\PP)}= \ttnorm{ f}_{\mcL^1(\mssm)}$, i.e.~$\emparg^\trid\colon  f\mapsto  f^\trid$ is an isometry~$\mcL^1(\mssm)\to\mcL^1(\PP)$.
By standard arguments with the positive and negative parts of~$ f$, the above isometry extends to the whole of~$\mcL^1(\mssm)$.
In order to show that it descends to~$L^1(\mssm)$ it is enough to compute~$\ttnorm{ f^\trid-two f^\trid}_{\mcL^1(\PP)}=\ttnorm{ f-two f}_{\mcL^1(\mssm)}$ on different $\mssm$-representatives~$ f$, and~$two f$ of the same $\mssm$-class~$f\in L^1(\mssm)$.
The order-preserving property is straightforward.
\end{proof}
\end{lem}
%\note{COLLECT HERE ALL PROPERTIES OF POISSON MEASURES}

\begin{lem}[{\cite[Lem.~3.25, Rmk.~4.26]{LzDSSuz21}}]
Let~$\mcX$ be a topological local structure. Then,~$\PP$ has full $\T_\mrmv(\msE)$-support.
If~$\mcX$ is, additionally, Polish, then~$\PP$ is Radon.
\end{lem}
\end{comment}


%Now, let~$(\mcX,\cdc)$ be a \TLDS.
%For simplicity, we shall consider a strong lifting~$\ell\colon L^\infty(\mssm)\rar \mcL^\infty(\mssm)$ fixed throughout this section, and we shall write~$\cdc$ in place of $\cdc_\ell$.
%We now turn to discuss the conditional closability.  

%For a probability measure~$\QP$ on~$\ttonde{\dUpsilon,\A_\mrmv(\msE)}$, and for the corresponding system of projected conditional probabilities~\eqref{eq:ProjectedConditionalQP}, we have the following standard result.
%
\begin{comment}
\begin{prop}\label{p:ConditionalIntegration}
Let~$(\mcX,\cdc)$ be a \TLDS,~$\QP$ be a probability measure on~$\ttonde{\dUpsilon,\A_\mrmv(\msE)}$, and~$ u\in \mcL^1(\QP)$. Then,
\begin{align*}
\int_{\dUpsilon} u \diff\QP = \int_{\dUpsilon} \quadre{\int_{\dUpsilon(E)}  u_{E,\eta} \diff \QP^\eta_E }\diff\QP(\eta) \fstop
\end{align*}
\begin{proof}
By definition of conditional probability
\begin{align*}
\int_{\dUpsilon}  u\diff\QP = \int_{\dUpsilon} \quadre{\int_{\dUpsilon}  u \diff\QP^{\eta_{E^\complement}} }\diff\QP(\eta) \fstop
\end{align*}
%
By regularity of the conditional system~$\seq{\QP^{\eta_{E^\complement}}}_{\eta\in\dUpsilon}$, the measure~$\QP^{\eta_{E^\complement}}$ is concentrated on the set
\begin{equation}\label{eq:RoeSch99Set}
\Lambda_{\eta, E^\complement} \eqdef \set{\gamma \in \dUpsilon : \gamma_{E^\complement}=\eta_{E^\complement} } 
\end{equation}
and we have that
\begin{equation}\label{eq:p:ConditionalIntegration:1}
 u\equiv  u_{E,\eta}\circ \pr_E \quad \text{everywhere on } \Lambda_{\eta, E^\complement}\fstop
\end{equation}
As a consequence,
\begin{align*}
\int_{\dUpsilon}  u\diff\QP = \int_{\dUpsilon} \quadre{\int_{\dUpsilon}  u_{E,\eta} \circ \pr_E \diff\QP^{\eta_{E^\complement}} }\diff\QP(\eta) \comma
\end{align*}
and the conclusion follows by definition of the projected conditional system.
\end{proof}
\end{prop}
\end{comment}
By the same process as in~\eqref{eq:d:LiftCdCRep} (see also Remark~\ref{rem: lift}), we can define the restricted square field operator~$\cdc^{\dUpsilon,  \QP}_E$ on~$\dUpsilon$ associated to~$E \in \msE$ and it has the following form on $\Cyl{\Dz}$:
\begin{equation}\label{eq:RestrictedCdCUpsilon}
\cdc^{\dUpsilon, \QP}_E(u)(\gamma) = \sum_{x\in \gamma_E} \gamma_x^{-1}\cdot \cdc\tonde{ u\ttonde{\car_{X\setminus\set{x}}\cdot\gamma + \gamma_x\delta_\bullet}- u\ttonde{\car_{X\setminus\set{x}}\cdot\gamma}}(x), \quad \text{$\QP$-a.e.\ $\gamma$.}
\end{equation}
We further introduce the following objects on $\dUpsilon(E)$: the symbol $\Cyl{\Dz}_{\QP^\eta_E}$ denotes the $\QP^\eta_E$-equivalence classes of $\Cyl{\Dz}$ (note that $\QP^\eta_E$ is supported on $\dUpsilon(E)$), and $\cdc^{\dUpsilon(E), \QP^\eta_E}$ is the square field on $\dUpsilon(E)$, which is described by \eqref{eq:d:LiftCdCRep} (see also Remark~\ref{rem: lift}) with $\dUpsilon(E)$ and $\QP^\eta_E$ in place of $\dUpsilon$ and $\QP$. 

The associated quadratic functionals with respect to $\QP$ and $\QP^\eta_E$ are defined respectively as:
\begin{subequations}\label{eq:VariousForms}
\begin{align}\label{eq:VariousFormsA}
\EE{\dUpsilon}{\QP}_E( u)\eqdef& \int_{\dUpsilon} \cdc^{\dUpsilon, \QP}_E( u)\diff\QP\comma && E\in\msE\comma\quad  u\in\Cyl{\Dz} \comma
\\
\label{eq:VariousFormsB}
\EE{\dUpsilon(E)}{\QP^\eta_E}( u) \eqdef& \int_{\dUpsilon(E)} \cdc^{\dUpsilon(E), \QP^\eta_E}(u)\diff\QP^\eta_E\comma &&\begin{gathered} E\in\msE\comma\quad\eta\in\dUpsilon\comma\\  u\in\Cyl{\Dz}_{\QP^\eta_E}\fstop \end{gathered}
\end{align}

\end{subequations}

%Before proving the next results, let us comment on the meaning of the forms above.
%For simplicity, let us assume we have already shown that both the forms in~\eqref{eq:VariousForms} are quasi-regular and strongly local, so that their properties can be recast in terms of the properly associated Markov diffusions.
%Let~$\mbfM_E$, resp.~$\mbfM^{\eta,E}$, be the diffusion associated to~\eqref{eq:VariousFormsA}, resp.~\eqref{eq:VariousFormsB}.
%The corresponding sample paths~$\gamma^E_t$ and~$\gamma^{\eta,E}_t$ coincide almost surely when restricted to~$E$.
%Restricting on~$E^\complement$ we have instead that~$t\mapsto\gamma^E_t\restr_{E^\complement}$ is almost surely a constant configuration on~$E^\complement$ randomly distributed according to~$\pr^{E^\complement}_\pfwd\QP$, whereas~$t\mapsto\gamma^{\eta,E}_t\restr_{E^\complement}=\eta_{E^\complement}$ almost surely.

\begin{comment}
\begin{prop}\label{p:MarginalWP}
Let~$(\mcX,\cdc)$ be a \TLDS, and~$\QP$ be a probability measure on $\ttonde{\dUpsilon,\A_\mrmv(\msE)}$ satisfying Assumption~\ref{ass:CAC} for some localizing sequence~$\seq{E_h}_h$.
Then,
\begin{align}\label{eq:p:MarginalWP:0}
\EE{\dUpsilon}{\QP}_{E_h}( u)=\int_\dUpsilon  \EE{\dUpsilon(E_h)}{\QP^\eta_{E_h}}( u_{E_h,\eta}) \diff\QP(\eta) \comma\qquad h\in\N\comma\qquad  u\in\Cyl{\Dz}\fstop
\end{align}
Furthermore,~$\EE{\dUpsilon}{\QP}_{E_h}$ is well-defined on~$\CylQP{\QP}{\Dz}$, in the sense that:
\begin{enumerate}[$(i)$]
\item\label{i:p:MarginalWP:1} it does not depend on the choice of the $\QP$-representative~$ u$ of~$u\in \CylQP{\QP}{\Dz}$;
\item\label{i:p:MarginalWP:2} it does not depend on the choice of the strong lifting~$\ell$.
\end{enumerate}

\begin{proof}
Let~$h$ and~$\eta$ be fixed and set~$E\eqdef E_h$ for simplicity of notation.
By Assumption~\ref{ass:CAC} and Remark~\ref{r:ConditionalAC}, we may apply Proposition~\ref{p:NewWP} to~$\QP^\eta_E$ on~$\dUpsilon(E)$ to obtain that the form~$\EE{\dUpsilon(E)}{\QP^\eta_E}$ is well-posed, in the sense that it satisfies~\iref{i:p:MarginalWP:1}.
Furthermore, by Lemma~\ref{l:MaRoeckner} applied to~$\dUpsilon(E)$,
\begin{align*}
&\tonde{\cdc^{\dUpsilon(E)}( u_{E,\eta})\circ \pr_E}(\gamma)=
\\
=&\ \sum_{x\in\gamma_E} \ttonde{(\gamma_E)_x}^{-1}\cdot \cdc\tonde{ u_{E,\eta}\ttonde{\car_{E\setminus \set{x}} \cdot\gamma_E+(\gamma_E)_x\delta_\bullet}- u_{E,\eta}\ttonde{\car_{E\setminus\set{x}}\cdot\gamma_E}}
\\
=&\ \sum_{x\in\gamma_E} \ttonde{(\gamma_E)_x}^{-1}\cdot \cdc\tonde{ u_{E,\eta}\ttonde{\car_{X\setminus\set{x}}\cdot\gamma_E+(\gamma_E)_x\delta_\bullet}- u_{E,\eta}\ttonde{\car_{X\setminus\set{x}}\cdot\gamma_E}}
\\
=&\ \sum_{x\in\gamma_E} \ttonde{(\gamma_E+\eta_{E^\complement})_x}^{-1}\cdot \cdc\tonde{ u\ttonde{\car_{X\setminus\set{x}}\cdot\gamma_E+(\gamma_E)_x\delta_\bullet+\eta_{E^\complement}}- u\ttonde{\car_{X\setminus\set{x}}\cdot\gamma_E+\eta_{E^\complement}}}
\\
=&\ \sum_{x\in\gamma_E} \ttonde{(\gamma_E+\eta_{E^\complement})_x}^{-1}\cdot \cdc\tonde{ u\ttonde{\car_{X\setminus\set{x}}\cdot(\gamma_E+\eta_{E^\complement})+(\gamma_E+\eta_{E^\complement})_x\delta_\bullet}- u\ttonde{\car_{X\setminus\set{x}}\cdot(\gamma_E+\eta_{E^\complement})}}
\\
=&\ \cdc^\dUpsilon_E( u) (\gamma_E+\eta_{E^\complement})\comma
\end{align*}
where the last equality holds by definition~\eqref{eq:RestrictedCdCUpsilon} of~$\cdc^\dUpsilon_E$. This shows that
\begin{equation}\label{eq:CdCRestrConditionalFormCylinderF}
\cdc^{\dUpsilon(E)}( u_{E,\eta})\circ \pr_E\equiv \cdc^\dUpsilon_E( u) \qquad \text{on} \quad \Lambda_{\eta, E^\complement} \comma
\end{equation}
where~$\Lambda_{\eta,E^\complement}$ is defined as in~\eqref{eq:RoeSch99Set}.
Since~$\QP^{\eta_{E^\complement}}$ is concentrated on~$\Lambda_{\eta, E^\complement}$, we thus have
\begin{equation*}
\int_{\dUpsilon} \cdc^\dUpsilon_E( u) \diff \QP^{\eta_{E^\complement}}
=\int_{\dUpsilon} \cdc^{\dUpsilon(E)}( u_{E,\eta})\circ \pr_E \diff\QP^{\eta_{E^\complement}}= \int_{\dUpsilon(E)} \cdc^{\dUpsilon(E)}( u_{E,\eta}) \diff\QP^\eta_E
\end{equation*}
for every~$ u\in\Cyl{\Dz}$, which concludes the proof of~\eqref{eq:p:MarginalWP:0} by integration w.r.t.~$\QP$ and Proposition~\ref{p:ConditionalIntegration}.

Assertion~\iref{i:p:MarginalWP:1} immediately follows from the well-posedness of the right-hand side in~\eqref{eq:p:MarginalWP:0}.
%
Assertion~\iref{i:p:MarginalWP:2} follows similarly, as soon as we show that, for~$\QP$-a.e.~$\eta$, the form~$\EE{\dUpsilon(E)}{\QP^\eta_E}$ is independent of~$\ell$.

To this end, note that, for different strong liftings~$\ell_1, \ell_2$,
\begin{align*}
\int_{\dUpsilon(E)} &\abs{ \cdc^{\dUpsilon(E)}_{\ell_1}( u_{E,\eta})- \cdc^{\dUpsilon(E)}_{\ell_2}( u_{E,\eta}) }\diff\QP^\eta_E
\\
=&\int_{\dUpsilon} \tonde{\abs{\cdc^{\dUpsilon(E)}_{\ell_1}( u_{E,\eta})-\cdc^{\dUpsilon(E)}_{\ell_2}( u_{E,\eta})} \cdot \frac{\diff \QP^\eta_E}{\diff (\pr_E)_\pfwd \PP_\mssm}} \circ \pr_E \diff\PP_\mssm
\\
=&\int_\dUpsilon \int_X \abs{\cdc^{\dUpsilon(E)}_{\ell_1}( u_{E,\eta})-\cdc^{\dUpsilon(E)}_{\ell_2}( u_{E,\eta})}(\gamma_E+\car_E\delta_x) \, \cdot 
\\
&\qquad \cdot \frac{\diff \QP^\eta_E}{\diff (\pr_E)_\pfwd \PP_\mssm}(\gamma_E+\car_E\delta_x) \diff\mssm(x)\, \diff\PP_\mssm(\gamma) \comma
\end{align*}
where the second equality follows from~\eqref{eq:Mecke}.
%
By definition of lifting, and computing~$\cdc^{\dUpsilon(E)}_{\ell_i}( u_{E,\eta})$ by~\eqref{eq:d:LiftCdCRep} for $i=1,2$, we have that
\begin{align*}
x\longmapsto \abs{\cdc^{\dUpsilon(E)}_{\ell_1}( u_{E,\eta})-\cdc^{\dUpsilon(E)}_{\ell_2}( u_{E,\eta})}(\gamma_E+\car_E\delta_x) \equiv 0 \as{\mssm}\comma
\end{align*}
and the conclusion follows.
\end{proof}
\end{prop}
\end{comment}
%In light of Proposition~\ref{p:MarginalWP}, the following assumption is well-posed.

\begin{defs}[Conditional closability]\label{ass:ConditionalClosability}
Let~$\QP$ be a probability on~$\ttonde{\dUpsilon,\A_{\mrmv}(\msE)}$ satisfying Assumptions~\ref{ass:Mmu} and~\ref{ass:CE} for some localising exhaustion~$\seq{E_h}_h$ with $E_h \uparrow X$.
%
We say that~$\QP$ satisfies the \emph{conditional closability} assumption~\ref{ass:ConditionalClos} if the forms
\begin{equation}\tag*{$(\mathsf{CC})_{\ref{ass:ConditionalClosability}}$}\label{ass:ConditionalClos}
\EE{\dUpsilon(E_h)}{\QP^\eta_{E_h}}(u,v)=\int_{\dUpsilon(E_h)} \cdc^{\dUpsilon(E_h), \QP^\eta_{E_h}}(u,v) \diff\QP^\eta_{E_h}\comma\qquad
\begin{aligned}
u,v\in&\ \CylQP{\QP^\eta_{E_h}}{\Dz}\comma
\\ 
h\in\N&\comma \quad \eta\in\dUpsilon\comma
\end{aligned}
\end{equation}
are closable on~$L^2\ttonde{\dUpsilon(E_h),\QP^\eta_{E_h}}$ for $\QP$-a.e.~$\eta\in\dUpsilon$, and for every~$h\in\N$. 
\end{defs}
 We write its closure, called {\it the conditioned form}, by
\begin{equation} \label{eq: condF}
\ttonde{\EE{\dUpsilon(E_h)}{\QP^\eta_{E_h}},\dom{\EE{\dUpsilon(E_h)}{\QP^\eta_{E_h}}}} \fstop
\end{equation}
The corresponding $L^2$-resolvent operator and the $L^2$-semigroup are denoted respectively by 
$$\bigl\{G_{\alpha}^{\U(E_h), \QP^\eta_E}\bigr\}_{\alpha>0} \quad \text{and} \quad \bigl\{T^{\U(E_h), \QP^\eta_E}_t\bigr\}_{t >0}\fstop$$ 
The square field~$\cdc^{\dUpsilon(E_h)}$ naturally extends to the domain $\dom{\EE{\dUpsilon(E_h)}{\QP^\eta_{E_h}}}$, which is denoted by the same symbol~$\cdc^{\dUpsilon(E_h)}$.
\smallskip

%Under the conditional closability, we obtained in \cite{LzDSSuz21} the closability of \eqref{eq:VariousFormsA} and 
For every $\A_\mrmv(\msE)$-measurable~$ u\colon \dUpsilon\to \R$ define
\begin{equation}\label{eq:ConditionalFunction}
 u_{E,\eta}(\gamma)\eqdef  u(\gamma+\eta_{E^\complement})\comma \qquad \gamma\in \dUpsilon(E) \fstop
\end{equation}
We now recall the result on the closability of the the following pre-Dirichlet form:
\begin{align}\label{eq:Temptation}
\EE{\dUpsilon}{\QP}(u,v)\eqdef \int_{\dUpsilon} \cdc^{\dUpsilon, \QP}(u,v) \diff\QP\comma \qquad u,v\in \Cyl{\Dz} \fstop
\end{align}
The following theorem is \cite[Thm.\ 3.48]{LzDSSuz21}, see also (b) in Rem.\ \ref{rem: CP1}. 
%For~$E\eqdef E_h$ in a suitable localizing sequence, combining Lemma~\ref{l:MmuL1} with the disintegration result in Proposition~\ref{p:ConditionalIntegration} shows that~$\EE{\dUpsilon(E)}{\QP^\eta_E}$ is finite on~$\CylQP{\QP^\eta_E}{\Dz}$ for every~$E$, for $\QP$-a.e.~$\eta\in\dUpsilon$.
%
%Since each of the forms~\ref{ass:ConditionalClos} is densely defined by Remark~\ref{r:DensityQP}\iref{i:r:DensityQP:1}, its closure
%\begin{equation} \label{eq: condF}
%\ttonde{\EE{\dUpsilon(E)}{\QP^\eta_E},\dom{\EE{\dUpsilon(E)}{\QP^\eta_E}}}
%\end{equation}
%is a Dirichlet form.

%\begin{ese}
%When~$X=\R^n$ is a standard Euclidean space, then all canonical Gibbs measures  and the laws of some determinantal/permanental point processes (e.g., the Ginibre,~$\mathrm{sine}_\beta$, $\mathrm{Airy}_\beta$, $\mathrm{Bessel}_{\alpha,\beta}$) satisfy~\ref{ass:ConditionalClos}.
%See~\S\ref{sss:ExamplesAC} for references and further examples.
%\end{ese}
%\purple{Note that $(\EE_{\U(B_r)}, \CylF(\U(B_r)))$ is closable and the closure is denoted by $(\EE_{\U(B_r)}, H^{1,2}(\U(B_r), \p))$,  see Definition~\ref{defn: WS}. Recall that $\{G_{\alpha}^{\U(B_r)}\}_{\alpha}$ and $\{T^{\U(B_r)}_t\}$ denote the $L^2$-resolvent operator and the semigroup corresponding to $(\EE_{\U(B_r)}, H^{1,2}(\U(B_r), \p))$.
%Define $\{G_{E_h, \alpha}^{\dUpsilon, \QP}\}_{\alpha>0}$, $\{T^{\dUpsilon, \QP}_{E_h, t}\}_{t>0}$ and $\{G_{\alpha}^{\U(B_r)}\}_{\alpha}$, $\{T^{\U(B_r)}_t\}$.}
\begin{thm}[{Closability, \cite[Thm.\ 3.48]{LzDSSuz21}}]\label{t:ClosabilitySecond}
Let~$(\mcX,\cdc)$ be a \TLDS,~$\QP$ be a probability measure on $\ttonde{\dUpsilon,\A_\mrmv(\msE)}$ satisfying Assumptions~\ref{ass:CE} and~\ref{ass:ConditionalClos}. Then, the following hold:
\begin{itemize}
\item[(i)]  the form~\eqref{eq:VariousFormsA} is well-defined, densely defined, and closable, and the following holds: 
\begin{align}\label{eq:p:MarginalWP:0}
\EE{\dUpsilon}{\QP}_{E_h}(u)=\int_\dUpsilon  \EE{\dUpsilon(E_h)}{\QP^\eta_{E_h}}(u_{E_h,\eta}) \diff\QP(\eta) \comma\quad h\in\N\comma\quad u\in\Cyl{\Dz}\, ;
\end{align}
\item[(ii)] the form~\eqref{eq:Temptation} is well-defined, densely defined, and closable. Furthermore, the form~\eqref{eq:Temptation} is the monotone increasing limit of the forms~\eqref{eq:VariousFormsA} with respect to $\{E_h\}\uparrow~X$. 
\end{itemize}

\end{thm}
We denote the closures of the {\it localised form}~\eqref{eq:VariousFormsA} and \eqref{eq:Temptation} respectively by 
$$\ttonde{\EE{\dUpsilon}{\QP}_{E_h},\dom{\EE{\dUpsilon}{\QP}_{E_h}}}, \quad \ttonde{\EE{\dUpsilon}{\QP},\dom{\EE{\dUpsilon}{\QP}}} \fstop$$
The square field~$\cdc^{\dUpsilon}$ naturally extends to the domain $\dom{\EE{\dUpsilon}{\QP}}$, which is denoted by the same symbol~$\cdc^{\dUpsilon}$.
% is a Dirichlet form on~$L^2(\QP)$;and its closure denoted by 
%$$$$ 
%is a Dirichlet form on~$L^2(\QP)$. 
The $L^2$-resolvent operators and the $L^2$-semigroups corresponding to the form~\eqref{eq:VariousFormsA} and the form~\eqref{eq:Temptation} are denoted respectively by 
$$\bigl\{G_{E_h, \alpha}^{\dUpsilon, \QP}\bigr\}_{\alpha>0},\ \bigl\{T^{\dUpsilon, \QP}_{E_h, t}\bigr\}_{t >0} \quad \text{and} \quad \bigl\{G_{\alpha}^{\dUpsilon, \QP}\bigr\}_{\alpha>0}, \ \bigl\{T^{\dUpsilon, \QP}_t\bigr\}_{t >0} \fstop$$ 
Let $\Delta^{\dUpsilon, \QP}$ denote the $L^2$-generator corresponding to $\ttonde{\EE{\dUpsilon}{\QP},\dom{\EE{\dUpsilon}{\QP}}}$. Let ${\rm Cap}_{\EE{\dUpsilon}{\QP}}$ denote the capacity associated with $\ttonde{\EE{\dUpsilon}{\QP},\dom{\EE{\dUpsilon}{\QP}}}$, see \cite[Def.\ 2.4 with $h=g=1$, or Exe. 2.10 in Chap.\ III]{MaRoe90}. 


\paragraph{Irreducibility for conditioned forms}Let $\mcX$ be a \TLDS. Let~$\QP$ be a probability measure on $\ttonde{\dUpsilon,\A_{\mrmv}(\msE)}$ satisfying Assumptions~\ref{ass:Mmu} and~\ref{ass:CE} for some localising exhaustion~$\seq{E_h}_h$ with $E_h \uparrow X$, and  the \emph{conditional closability} assumption~\ref{ass:ConditionalClos}. 
%Let 
%\begin{align} \label{eq: condF1}
%\ttonde{\EE{\dUpsilon(E_h)}{\QP^\eta_{E_h}},\dom{\EE{\dUpsilon(E_h)}{\QP^\eta_{E_h}}}}
%\end{align}
 %be the corresponding conditioned Dirichlet form on $E_h$ and $\eta \in \dUpsilon$ as defined in \eqref{eq: condF}. 
\begin{defs}[Irreducibility for the conditioned form]\normalfont\label{ass:ConditionalErgodicity}
We say that {\it the irreducibility for the conditioned form }\eqref{eq: condF} (in short: \ref{ass:ConditionalErg}) holds if,
%, for every $h$ and $\QP$-a.e.\ $\eta \in \dUpsilon$,  the conditioned form \eqref{eq: condF} with exhaustion $\seq{E_h}$ is irreducible with respect to $\QP^\eta_{\dUpsilon^{(k)}(E_h)}$ for each $k$. Namely, 
for every $h \in \N$, $k \in \N_0$ and $\QP$-a.e.\ $\eta \in \dUpsilon$, 
\begin{align*}\tag*{$(\mathsf{IC})_{\ref{ass:ConditionalErgodicity}}$}\label{ass:ConditionalErg}
\text{if $u \in \dom{\EE{\dUpsilon(E_h)}{\QP^\eta_{E_h}}}=0$, then $u\mrestr{\dUpsilon^{(k)}(E_h)}=C_{E_h, \eta}^k$ \ $\QP^{\eta, k}_{E_h}$-a.e.} \comma 
\end{align*}
where $C_{E_h, \eta}^k$ is a constant depending only on $E_h, \eta$ and $k$.
%We denote this condition by ${\sf (SL)}_{\dUpsilon(E_h), \mu^\eta_{\dUpsilon(E_h)}}$.
\end{defs}
\begin{rem} We give several remarks:
\begin{enumerate}[$(a)$]
\item In terms of the corresponding diffusion process,  Assumption~\ref{ass:ConditionalErg} is decoded as the ergodicity of the interacting {\it finite} particles in $\dUpsilon(E_h)$ conditioned at $\eta_{E_h^c}$ outside $E_h$. 
\item Assumption \ref{ass:ConditionalErg} can be verified for a wide class of invariant measures $\QP$ such as Gibbs measures including Ruelle measures,  and determinantal/permanental point processes including $\mathrm{sine}_\beta$, $\mathrm{Airy}_\beta$, $\mathrm{Bessel}_{\alpha, \beta}$, Ginibre, which will be discussed in \S \ref{sec: Exa}.
\end{enumerate}
%to \purple{Write an intuitive understanding of \ref{ass:ConditionalErg}, and the fact that we can verify it for a wide class, which will be discussed in Example section}
\end{rem}

\subsection{Metric structures on $\dUpsilon$}
%In this section, we recall the metric structure on $\dUpsilon$. 
\paragraph{$L^2$-transportation distance}Let~$\mcX$ be an \parEMLDS.
%
For~$i=1,2$ let~$\pr^i\colon X^{\times 2}\rar X$ denote the projection to the~$i^\text{th}$ coordinate. 
For~$\gamma,\eta\in \dUpsilon$, let~$\Cpl(\gamma,\eta)\subset \Meas(X^{\tym{2}},\A^{\hotimes 2})$ be the set of all couplings of~$\gamma$ and~$\eta$, viz.
\begin{align*}
\Cpl(\gamma,\eta)\eqdef \set{\cpl\in \Meas(X^{\tym{2}},\A^{\hotimes 2}) \colon \pr^1_\pfwd \cpl =\gamma \comma \pr^2_\pfwd \cpl=\eta} \fstop
\end{align*}
%Let $E \subset X$ be a subset. For~$i=1,2$ let~$\pr^i\colon E^{\times 2}\rar E$ denote the projection to the~$i^\text{th}$ coordinate.
Here $\Meas(X^{\tym{2}},\A^{\hotimes 2})$ denotes the space of all $\msE_{\mssd}$-locally finite measures on $(X^{\tym{2}},\A^{\hotimes 2})$. 
%For~$\gamma,\eta\in \dUpsilon$, let~$\Cpl(\gamma,\eta)\subset \Meas(X^{\tym{2}},\A^{\hotimes 2}|_{X^{\tym 2}})$ be the set of all couplings of~$\gamma$. %and~$\eta$, viz.
%\begin{align*}
%\Cpl_E(\gamma,\eta)\eqdef \set{\cpl\in \Meas(E^{\tym{2}},\A^{\hotimes 2}|_{E^{\tym 2}}) \colon \pr^1_\pfwd \cpl =\gamma \comma \pr^2_\pfwd \cpl=\eta} \fstop
%\end{align*}
\begin{defs}[$L^2$-transportation distance]
The \emph{$L^2$-transportation}  \emph{distance} on~$\dUpsilon$ is
\begin{align}\label{eq:d:W2Upsilon}
\mssd_{\dUpsilon}(\gamma,\eta)\eqdef \inf_{\cpl\in\Cpl(\gamma,\eta)} \paren{\int_{X^{\times 2}} \mssd^2(x,y) \diff\cpl(x,y)}^{1/2}\comma \qquad \inf{\emp}=+\infty \fstop
\end{align}
%When $E=\R^n$, we simply write $\mssd_{\dUpsilon}$ instead of $\mssd_{\dUpsilon(\R^n)}$. 
\end{defs}
We refer the readers to \cite[\S4.2, p.52]{LzDSSuz21} for details regarding the $L^2$-transportation distance $\mssd_{\dUpsilon}$. 
It is important to note that $\mssd_\dUpsilon$ is an extended distance, attaining the value~$+\infty$ on a set of positive~$\QP^\otym{2}$-measure in~$\dUpsilon^\tym{2}$ for any reasonable probability measure $\QP$. 
%making~$\ttonde{\dUpsilon,\mssd_\dUpsilon, \QP}$ into an extended metric measure space.
%\begin{rem}
%\begin{enumerate*}[$(a)$]
%\item We stress that Theorem~\ref{t:ClosabilitySecond} can be proved in many different ways:
%\begin{enumerate*}[$({a}_1)$] 
%\item as an application of the theory of superpositions of Dirichlet forms in~\cite{BouHir91};

%\item as an application of the theory of direct integrals of Dirichlet forms developed by the first named author in~\cite{LzDS20};

%\item by the very same arguments as in the proof of~\cite[Thm.~4]{Osa96}, noting that~\cite[Prop.~4.1]{Osa96} there is replaced by our Assumption~\ref{ass:ConditionalClosability}.
%\end{enumerate*}


%\item It is possible to show that the projected conditional systems~$\set{\QP^\eta_E}_{\eta\in\dUpsilon,E\in\msE}$ are consistent similarly to \emph{specifications} in the sense of Preston~\cite[\S6]{Pre76}.
%\item If the measure~$\QP$ has full $\T_\mrmv(\msE)$-support, the well-posedness of the form~$\ttonde{\EE{\dUpsilon}{\QP},\CylQP{\QP}{\Dz}}$ is immediate, since every function in~$\CylQP{\QP}{\Dz}$ has a unique continuous representative~$ u\in \Cyl{\Dz}$.
%\end{enumerate*}
%\end{rem}





\paragraph{Sobolev-to-Lipschitz for conditioned forms}%In this subsection, we will introduce Sobolev-to-Lipschitz property for conditional probability, which corresponds to finite-particle systems on compact sets. 
Let~$\mcX$ be an \parEMLDS. Let~$\QP$ be a probability measure on~$\ttonde{\dUpsilon,\A_{\mrmv}(\msE)}$ satisfying Assumptions~\ref{ass:Mmu} and~\ref{ass:CE} for some localising exhaustion~$\seq{E_h}_h$ with $E_h \uparrow X$, and  the \emph{conditional closability} assumption~\ref{ass:ConditionalClos}. %Let 
%\begin{align} \label{eq: condF1}
%\ttonde{\EE{\dUpsilon(E_h)}{\QP^\eta_{E_h}},\dom{\EE{\dUpsilon(E_h)}{\QP^\eta_{E_h}}}}
%\end{align}
 %be the corresponding Dirichlet form on $E_h$ and $\eta \in \dUpsilon$ as defined in \eqref{eq: condF}. 
\begin{defs}[Sobolev-to-Lipschitz for the conditioned form]\normalfont \label{ass:ConditionalSobLip}
We say that the Sobolev-to-Lipschitz property for the conditioned form \eqref{eq: condF} (in short: \ref{ass:ConditionalSL}) holds if, for every $h \in \N$, $k \in \N_0$ and $\mu$-a.e.\ $\eta \in \dUpsilon$, if $u \in \dom{\EE{\dUpsilon(E_h)}{\QP^\eta_{E_h}}}$ with $\Gamma^{\dUpsilon(E_h), \QP^\eta_{E_h}}(u) \le \alpha^2$, then there exists $\tilde{u}_{E_h, \eta}^{(k)} \in \Lipua(\mssd_{\dUpsilon^{(k)}(E_h)})$ so that    
\begin{align*}\tag*{$(\mathsf{SLC})_{\ref{ass:ConditionalSobLip}}$}\label{ass:ConditionalSL}
u = \tilde{u}_{E_h, \eta}^{(k)} \quad \text{$\QP_{E_h}^{\eta, k}$-a.e.} \qquad \forall \alpha \ge 0 \fstop
%\text{any $u \in \dom{\EE{\dUpsilon(E_h)}{\QP^\eta_{E_h}}}$ with $\Gamma^{\dUpsilon(E_h)}(u) \le 1$ has a $\QP^\eta_{E_h}$-modification $\tilde{u} \in \Lipu(\mssd_{\dUpsilon(E_h)})$.} 
\end{align*}
%We denote this condition by ${\sf (SL)}_{\dUpsilon(E_h), \mu^\eta_{\dUpsilon(E_h)}}$.
\end{defs}

\begin{rem} \label{rem: SL} We give several remarks:
\begin{enumerate}[$(a)$]
\item 
%By the linearity of $\cdc^{\dUpsilon(E_h), \QP^\eta_{E_h}}$, it is easy to see \ref{ass:ConditionalSL} implies the corresponding statement with the replacements $\Gamma^{\dUpsilon(E_h), \QP^\eta_{E_h}}(u) \le \alpha$ and $\mathrm{Lip}^\alpha(\mssd_{\dUpsilon^{(k)}(E_h)})$ for any $\alpha \ge 0$. 
In particular with $\alpha=0$, we recover Assumption~\ref{ass:ConditionalErg} as $0$-Lipschitz functions with respect to $\mssd_{\dUpsilon^{(k)}(E_h)}$ are constants on $\dUpsilon^{(k)}(E_h)$.  
\item Assumption~\ref{ass:ConditionalSL} can be verified for a wide class of invariant measures $\QP$ such as Gibbs measures including Ruelle measures,  and determinantal/permanental point processes including $\mathrm{sine}_2$, $\mathrm{2}_\beta$, $\mathrm{Bessel}_{\alpha, 2}$, Ginibre, which will be discussed in \S \ref{sec: Exa}.
\end{enumerate}
\end{rem}

\subsection{Tail triviality} \label{subsec: TT}
%In this subsection, we recall the notion of the tail-triviality for probability measures on the configuration space $\U$. 
%show that there are sufficiently many admissible sets under the tail triviality of $\mu$. 
Let $\mcX$ be a topological local structure. For $E \in \msE$, let $\sigma({\rm pr}_{E^c})$ denote the $\sigma$-field generated by the projection map $\U(X) \ni \gamma \mapsto {\rm pr}^{E^c}(\gamma)=\gamma_{E^c} \in \U(E^c)$. 
We set $\mathscr T(\U):=\cap_{E \in \msE}\sigma({\rm pr}^{E^c})$ and call it {\it tail $\sigma$-algebra}. For a set $A \subset \U$,  define $\mathcal T_E({A}):=({\rm pr}^{E^c})^{-1}\circ {\rm pr}^{E^c}(A)$. By definition, $A \subset \mathcal T_E(A)$, and $\mathcal T_E({A}) \subset \mathcal T_{E'}({A})$ whenever $E \subset E'$. For a given countable exhaustion $E_h \in \msE$ with $E_h \uparrow X$, define {\it the tail set of $A$ with respect to $\seq{E_h}$} by 
\begin{align} \label{eq: ts}
\mathcal T(A):=\cup_{h \in \N} \mathcal T_{E_h}({A}) \fstop
\end{align}
%\limsup_{n \to \infty} \bar{A}_n:=\cap_{n \ge 1} \cup_{j \ge n} \bar{A}_j.$$ 
It is simple to check that the tail set $\mathcal T(A)$ of $A$ does not depend on the choice of countable exhaustions $\seq{E_h}$. It is straightforward to see that $\mathcal T(A) \in \mathscr T(\U)$ and $A \subset \mathcal T({A})$. 

\begin{defs}[Tail triviality] \normalfont \label{defn: TT}
A Borel probability measure $\mu$ on $\U(X)$ is called {\it tail trivial} if $\mu(A)\in \{0, 1\}$ whenever $A \in \mathscr T(\U)$.
\end{defs}
\begin{ese}\normalfont  \label{exa: TT}
The tail triviality has been verified for a wide class of point processes. 
\begin{itemize}
\item[(i)] (Determinantal point processes) Let $X$ be a locally compact Polish space. 
Then, all determinantal point processes whose kernel are locally trace-class positive contraction satisfy the tail triviality (see \cite[Theorem 2.1]{Lyo18} and \cite{BufQiuSha21,  OsaOsa18, ShiTak03b}). In particular, $\mathrm{sine}_2$,~$\mathrm{Bessel}_{\alpha,2}$, $\mathrm{Airy}_2$ and Ginibre point processes are tail trivial 
\item[(ii)] (Extremal Gibbs measure)  A canonical Gibbs measure $\mu$ is tail  trivial iff $\mu$ is extremal (see \cite[Cor.\ 7.4]{Geo11}). In particular, Gibbs measures of the Ruelle type with sufficiently small activity constants are extremal (see \cite[Thm.\ 5.7]{Rue70}).
\end{itemize}
\end{ese}


\subsection{Rigidity} \label{subsec: Rig}
\begin{defs}[Rigidity in number: Ghosh--Peres {\cite[Thm.\ 1]{GhoPer17}}]\label{ass:Rigidity}
Let $\mcX$ be a topological local space. 
A probability measure $\QP$ on $\ttonde{\dUpsilon,\A_{\mrmv}(\msE)}$ has {\it the rigidity in number} (in short: \ref{ass:Rig}) if, for any $E \in \msE$, there exists $\Omega \subset \dUpsilon$ so that $\QP(\Omega)=1$  and, for any $\gamma, \eta \in \Omega$
\begin{align*}\tag*{$(\mathsf{R})_{\ref{ass:Rigidity}}$}\label{ass:Rig}
\text{$\gamma_{E^c_h} = \eta_{E^c_h}$ implies $\gamma E = \eta E$} \quad \forall h \in \N \fstop
\end{align*}
\end{defs}
\begin{ese} \label{exa: R}
The rigidity in number has been verified for a variety of point processes: Ginibre and GAF (\cite{GhoPer17}), $\mathrm{sine}_\beta$ (\cite[Thm.\ 4.2]{Gho15}, \cite{NajRed18}, \cite{DerHarLebMai20}), $\mathrm{Airy}$, ~$\mathrm{Bessel}$, and $\mathrm{Gamma}$ (\cite{Buf16}), and Pfaffian (\cite{BufNikQiu19}) point processes. We refer the readers also to the survey \cite{GhoLeb17}. 
\end{ese}


\section{Irreducibility} \label{sec: Irr}
In this section, we prove the equivalence between the irreducibility and the tail-triviality under the rigidity in number of invariant measures. We do not assume any metric structure for the base space $\mcX$ in this section. We recall that $\ttonde{\EE{\dUpsilon}{\QP},\dom{\EE{\dUpsilon}{\QP}}}$ is {\it irreducible} if $\EE{\dUpsilon}{\QP}(u)=0$ implies that $u$ is constant $\QP$-almost everywhere.
 
\smallskip
The main theorem of this section is the following:
\begin{thm}\label{thm: Erg} Let $\mcX$ be a \TLDS and $\QP$ be a probability measure on~$\ttonde{\dUpsilon,\A_{\mrmv}(\msE)}$ satisfying~\ref{ass:Mmu}. Suppose that  Assumptions~\ref{ass:CE}, \ref{ass:ConditionalClos}, \ref{ass:ConditionalErg}, and~\ref{ass:Rig} hold and that $\ttonde{\EE{\dUpsilon}{\QP},\dom{\EE{\dUpsilon}{\QP}}}$ is quasi-regular.  
%Suppose further that $(\E^{m, X}, \mathcal D(\E^{m, X}))$ is irreducible on $X$\footnote{We probably only need the irreducibility of $\E^{m, E}$ on any localising sets $E$ for the assumption. See the proof.}. 
%Suppose that $(\E^\mu, \F^\mu)$ is quasi-regular
 %$(Irr)_{B_r, \otimes n}$ $(\forall r>0, n \in \N)$\footnote{I believe that the irreducibility on $X$ tensorises and we only need to assume (Irr) on $X$. Then this is purely measure theoretic assumption and we never use the distance information.}. 
Then, the following  are equivalent:
\begin{enumerate}[$(i)$]
\item $\mu$ is tail trivial;
\item $\ttonde{\EE{\dUpsilon}{\QP},\dom{\EE{\dUpsilon}{\QP}}}$ is irreducible.

%\item[(iii)] ${\sf d}_{\U, 0}(A, B)<\infty$ for any $A, B$ with $\mu(A)\mu(B)>0$.
%Furthermore, if (Rad)$_{{\sf d}_{\U}}$ holds, then 
\end{enumerate}
%If, furthermore, suppose that (Rad)$_{X}$, and $(SL)_{B_{r}, \otimes n}$ holds for any $r>0$ and $n \in \N$, then the following statement is equivalent to (i) (therefore, also to (ii))
%\begin{itemize}
%\item[(iii)] for any $A, B \subset \U(X)$ with $\mu(A)\mu(B)>0$, 
%$${\rm essinf}_{B}{\sf d}_{\U}(A, \cdot)<\infty.$$ 
%\item[(iii)] ${\sf d}_{\U, 0}(A, B)<\infty$ for any $A, B$ with $\mu(A)\mu(B)>0$.
%Furthermore, if (Rad)$_{{\sf d}_{\U}}$ holds, then 
%\end{itemize}
\end{thm}


In the following subsections, we give the proof of Theorem \ref{thm: Erg}.


\subsection{The proof of (ii) $\implies$ (i) of Theorem \ref{thm: Erg}}
We first prove the implication (ii) $\implies$ (i), for which one does not need the quasi-regularity of $\ttonde{\EE{\dUpsilon}{\QP},\dom{\EE{\dUpsilon}{\QP}}}$, ~\ref{ass:ConditionalErg}, nor \ref{ass:Rig}. 
\begin{thm} \label{thm: Erg1} Let $\mcX$ be an \TLDS, ~$\QP$ be a probability measure on~$\ttonde{\dUpsilon,\A_{\mrmv}(\msE)}$ satisfying~\ref{ass:Mmu}, and suppose Assumptions~\ref{ass:CE}, ~\ref{ass:ConditionalClos}.
%Suppose further that $(\E^{m, X}, \mathcal D(\E^{m, X}))$ is irreducible on $X$\footnote{We probably only need the irreducibility of $\E^{m, E}$ on any localising sets $E$ for the assumption. See the proof.}. 
%Suppose that $(\E^\mu, \F^\mu)$ is quasi-regular
 %$(Irr)_{B_r, \otimes n}$ $(\forall r>0, n \in \N)$\footnote{I believe that the irreducibility on $X$ tensorises and we only need to assume (Irr) on $X$. Then this is purely measure theoretic assumption and we never use the distance information.}. 
Then, (ii) implies (i) in Theorem \ref{thm: Erg}.
%\begin{itemize}
%\item[(i)] $\ttonde{\EE{\dUpsilon}{\QP},\dom{\EE{\dUpsilon}{\QP}}}$ is ergodic;
%\item[(ii)] $\mu$ is tail trivial.
%\item[(iii)] ${\sf d}_{\U, 0}(A, B)<\infty$ for any $A, B$ with $\mu(A)\mu(B)>0$.
%Furthermore, if (Rad)$_{{\sf d}_{\U}}$ holds, then 
%\end{itemize}
%If, furthermore, suppose that (Rad)$_{X}$, and $(SL)_{B_{r}, \otimes n}$ holds for any $r>0$ and $n \in \N$, then the following statement is equivalent to (i) (therefore, also to (ii))
%\begin{itemize}
%\item[(iii)] for any $A, B \subset \U(X)$ with $\mu(A)\mu(B)>0$, 
%$${\rm essinf}_{B}{\sf d}_{\U}(A, \cdot)<\infty.$$ 
%\item[(iii)] ${\sf d}_{\U, 0}(A, B)<\infty$ for any $A, B$ with $\mu(A)\mu(B)>0$.
%Furthermore, if (Rad)$_{{\sf d}_{\U}}$ holds, then 
%\end{itemize}
\end{thm}

Before giving the proof of Theorem \ref{thm: Erg1}, we give several preparatory statements. Hereinafter, we fix an exhaustion $\seq{E_h}_h \uparrow X$ witnessing \ref{ass:CE}.
Recall that we defined in \eqref{eq:ConditionalFunction}: for every $\A_\mrmv(\msE)$-measurable~$ u\colon \dUpsilon\to \R$, and for $\eta \in \dUpsilon$, 
\begin{equation*}
 u_{E,\eta}(\gamma)\eqdef  u(\gamma+\eta_{E^\complement})\comma \qquad \gamma\in \dUpsilon(E) \,,
\end{equation*}
which is well-defined also for $\mu$-equivalent classes of $\A_\mrmv(\msE)$-measurable functions for $\mu^\eta_E$-a.e.\ $\gamma$ and $\mu$-a.e.\ $\eta$. The following proposition is a direct application of the property \eqref{eq:ConditionalQP} of the conditional probability.
%\begin{prop}\label{prop: MD}
%The form $(\E_r, \mathcal D(\E_r))$ is monotone non-decreasing in $r$, i.e. for any $s \le r$, 
%$$\mathcal D(\E_r) \subset \mathcal D(\E_s), \quad \E_s(F) \le \E_r(F), \quad F \in \mathcal D(\E_r). $$
%Furthermore, $\lim_{r \to \infty}\E_r(F) = \E(F)$ for $F \in H^{1,2}(\U(\R^n),\p)$.
%\end{prop}
%The next proposition shows the monotonicity property for
%the resolvent operator $G_\alpha^r$ and the semigroup $T^r_t$.
\begin{prop}[{\cite[Prop.\ 3.44]{LzDSSuz21}}]\label{p:ConditionalIntegration}
Let~$(\mcX,\cdc)$ be a \TLDS,~$\QP$ be a probability measure on $\ttonde{\dUpsilon,\A_\mrmv(\msE)}$ satisfying~\ref{ass:Mmu}, and~$u\in L^1(\QP)$. Then, for any $E \in \msE$, 
\begin{align*}
\int_{\dUpsilon} u \diff\QP = \int_{\dUpsilon} \quadre{\int_{\dUpsilon(E)} u_{E,\eta} \diff \QP^\eta_E }\diff\QP(\eta) \fstop
\end{align*}
\end{prop}

\begin{prop}\label{prop: MGS}
Suppose the same assumptions in Theorem~\ref{thm: Erg1} with $\seq{E_h}_{h}$. The resolvent operator $\{G_{E_h, \alpha}^{\dUpsilon, \QP}\}_{\alpha}$ and the semigroup $\{T^{\dUpsilon, \QP}_{E_h, t}\}_t$ satisfy
%are monotone non-increasing in $h$ on non-negative functions, i.e.,
%\begin{align} \label{ineq: RM}
%G_{E_h, \alpha}^{\dUpsilon, \QP} u \le G^{\dUpsilon, \QP}_{E_{h'}, \alpha} u,  \quad T^{\dUpsilon, \QP}_{E_h, t} u \le T^{\dUpsilon, \QP}_{E_{h'}, t} u,   \quad \text{for any nonnegative}\, \, u \in L^2(\dUpsilon, \QP), \quad h \le h'.
%\end{align}
$$\text{{\small $L^2(\mu)$ -- }}\lim_{h \to \infty} G_{E_h, \alpha}^{\dUpsilon, \QP} u = G^{\dUpsilon, \QP}_\alpha u \quad \text{and} \quad \text{{\small $L^2(\mu)$ -- }}\lim_{h \to \infty} T^{\dUpsilon, \QP}_{E_h, t} u= T^{\dUpsilon, \QP}_t u$$ for any $u \in  L^2(\dUpsilon, \QP)$ and $\alpha, t>0$.
\end{prop}
\begin{proof}
%Thanks to the identity 
%\begin{equation*}
%	G_{E_h, \alpha}^{\dUpsilon, \QP}  = \int_0^\infty e^{-\alpha t} T^{\dUpsilon, \QP}_{E_h, t} \diff t\, ,
%\end{equation*}
%it suffices to show \eqref{ineq: RM} only for $T^{\dUpsilon, \QP}_{E_h, t}$. By a direct application of~\cite[Theorem 3.3]{Ouh96} and the monotonicity of the Dirichlet form by~Theorem \ref{t:ClosabilitySecond}, we obtain the monotonicity of the semigroup $T^{\dUpsilon, \QP}_{E_h, t}$. 
The statement follows from the monotonicity of the forms by~Theorem~\ref{t:ClosabilitySecond}  combined with \cite[S.14, p.373]{ReeSim75}. 
\end{proof}
%The next proposition is taken from \cite[(4.1), Prop.~4.1]{Osa96} or \cite[Proposition~3.45]{DS21}.
%\begin{prop}\label{prop: 0}
%For any $F \in \Cyl$, $\eta \in \U(B_r^c)$ and $r>0$, it holds
%\begin{align}
%\E_r(F) &= \int_{\U(B_r^c)} \EE_{\U(B_r)}(F_{\eta, r}) d\p_{B_r^c}(\eta) \, , \label{eqP: 2} 
%\end{align}%
%where $\E_{\U(B_r)}$ is the energy on $\U(B_r)$ defined in Definition~\ref{defn: WS}, and $F_{\eta, r}=F_{\eta, B_r^c}$ was introduced in Definition~\ref{def: CF}.
%\end{prop}
%Note that $(\EE_{\U(B_r)}, \CylF(\U(B_r)))$ is closable and the closure is denoted by $(\EE_{\U(B_r)}, H^{1,2}(\U(B_r), \p))$,  see Definition~\ref{defn: WS}. Recall that $\{G_{\alpha}^{\U(B_r)}\}_{\alpha}$ and $\{T^{\U(B_r)}_t\}$ denote the $L^2$-resolvent operator and the semigroup corresponding to $(\EE_{\U(B_r)}, H^{1,2}(\U(B_r), \p))$.
We now provide the relation of  $\{G_{E, \alpha}^{\dUpsilon, \QP}\}_{\alpha>0}$, $\{T^{\dUpsilon, \QP}_{E, t}\}_{t>0}$ and $\{G_{\alpha}^{\dUpsilon(E), \QP^\eta_E}\}_{\alpha>0}$, $\{T^{\dUpsilon(E), \QP^\eta_E}_t\}_{t>0}$ for $E \in \{E_h\}_{h \in \N}$. 


\begin{prop}[Localised-/conditioned-operators] \label{prop: 1}
Suppose the same assumptions in Theorem~\ref{thm: Erg1}. 
	%Let $\alpha>0$, $t>0$, and $r>0$ be fixed. 
	Then,  the following hold: 
	\begin{align} \label{eq: R-1}
		G_{E, \alpha}^{\dUpsilon, \QP} u(\gamma) & = G_{\alpha}^{\dUpsilon(E), \QP^\gamma_{E}}u_{E, \gamma}(\gamma_{E}) \, ,
		\\
		\label{eq: R-2}
		T^{\dUpsilon, \QP}_{E, t}u(\gamma) & = T^{\dUpsilon(E), \QP^\gamma_E}_t u_{E, \gamma}(\gamma_{E}) \, ,
	\end{align}
	for $\QP$-a.e.\ $\gamma\in \dUpsilon$, any $\alpha>0$, $t>0$ and $E \in \{E_h\}_{h \in \N}$.
\end{prop}

\begin{proof}
Let $u, v \in \Cyl{\Dz}$.  
For simplicity of the notation, set $G_\alpha u(\cdot) := G_{\alpha}^{\dUpsilon(E), \QP^\cdot_E}u_{E, \cdot}(\cdot_{E})$. Then, by the definition \eqref{eq:ConditionalFunction},  
\begin{align} \label{eq: 1-0}
\bigl(G_\alpha u \bigr)_{E, \eta}(\cdot)= G_{\alpha}^{\dUpsilon(E), \QP^\eta_E}u_{E, \eta}(\cdot), \quad \QP^\eta_{E}\text{-a.e.\ on $\dUpsilon(E)$ for $\QP$-a.e.\ $\eta$} \fstop
\end{align}
Then, we have that\footnote{\purple{We need to assume $\mathcal D(\E_E^\U)=\{u \in L^2(\U, \mu): \int_{\U} \|u_\eta\|^2_{\mathcal D(\E^{\U(E), \mu_E^\eta})} d\mu<\infty\}$. This follows from the Markov uniqueness of $\mathcal D(\E_E^\U)$ since the r.h.s is a Dirichlet form by the superposition theorem in Bouleau--Hirsch p.213 }}
%By combining \eqref{eq:p:MarginalWP:0}, \eqref{eq: 1-0} and the standard relation between the form and the resolvent operator, we obtain
\begin{align} \label{eq: 1-1}
	\EE{\dUpsilon}{\QP}_{E}(G_\alpha u, v) 
	&=  \int_{\dUpsilon} \EE{\dUpsilon(E)}{\QP^\eta_{E}}((G_\alpha u)_{E, \eta}, v_{E, \eta}) \diff \QP(\eta) \notag
	&
	\\
	&=  \int_{\dUpsilon} \EE{\dUpsilon(E)}{\QP^\eta_{E}}(G_{\alpha}^{\dUpsilon(E), \QP^\eta_E}u_{E, \eta}, v_{E, \eta}) \diff \QP(\eta) \notag
	& 
	\\
	&=  \int_{\dUpsilon} \biggl( \int_{\dUpsilon(E)} \Bigl(u_{E, \eta} - \alpha  G_{\alpha}^{\dUpsilon(E), \QP^\eta_E}u_{E,\eta} \Bigr)  v_{E,\eta}    \diff\QP^\eta_{E}\biggr) \diff \QP(\eta) \notag
%	&\because \ (\text{the general fact $\EE_\alpha(\alpha G_\alpha u, v) = \langle u, v\rangle_2$}) 
	&\\
	&=  \int_{\dUpsilon}  (u - \alpha G_\alpha  u) v   \diff\QP \, , \ 
%	&\because \ (\text{Prop. \ref{prop: 0}}). 
\end{align}
where the first line follows from \eqref{eq:p:MarginalWP:0}, the second line follows from \eqref{eq: 1-0}, the third line follows from the standard relation between the form and the resolvent operator, and the fourth line follows from Prop.~\ref{p:ConditionalIntegration} and \eqref{eq: 1-0}. 
%the fundamental equality $-\Delta G_\alpha^{\U(B_r)}F_{\eta,r} + \alpha G_\alpha^{\U(B_r)}F_{\eta,r} = F_{\eta,r}$. 
Since the resolvent operator $G_{E, \alpha}^{\dUpsilon, \QP}$ is characterised as the unique $L^2$-bounded operator satisfying  \eqref{eq: 1-1} with $G_\alpha$ being replaced by $G_{E, \alpha}^{\dUpsilon, \QP}$, 
%is the characterisation of 
%$$\EE{\dUpsilon}{\QP}_{E}(G_\alpha u, v)  = \int_{\U(\R^n)}  (F - \alpha G^{r}_\alpha F) H \ d\p \, ,$$
we conclude that $G_{E, \alpha}^{\dUpsilon, \QP}u(\gamma) = G_\alpha u(\gamma) = G_{\alpha}^{\dUpsilon(E), \QP^\gamma_E}u_{E, \gamma}(\gamma_{E})$ $\QP$-a.e.\ $\gamma$. The proof of \eqref{eq: R-1} is complete. 




The second equality \eqref{eq: R-2} follows from the identity
\begin{equation*}
	\int_0^\infty e^{-\alpha t} T^{\dUpsilon, \QP}_{E, t}u(\gamma) \diff t
	= G_{E, \alpha}^{\dUpsilon, \QP}u(\gamma) 
	= G_{\alpha}^{\dUpsilon(E), \QP^\gamma_E}u_{E, \gamma}(\gamma_{E})
	=\int_0^\infty e^{-\alpha t} T^{\dUpsilon(E), \QP^\gamma_E}_tu_{E, \gamma}(\gamma_{E}) \diff t \comma
\end{equation*}
and the injectivity of the Laplace transform on continuous functions in $t$.
\end{proof}


{\it Proof of Theorem \ref{thm: Erg1}.}
Recall that $\ttonde{\EE{\dUpsilon}{\QP},\dom{\EE{\dUpsilon}{\QP}}}$ is irreducible if and only if $\{T^{\dUpsilon, \QP}_t\}_{t>0}$-invariant sets are trivial (see, e.g., \cite[Prop.\ 2.3 and Appendix]{AlbKonRoe97}), i.e., any $A\subset \dUpsilon$ satisfying
 $$T^{\dUpsilon, \QP}_t (\1_{A}u) = \1_{A}T^{\dUpsilon, \QP}_t u,\quad  \forall u \in L^2(\QP) $$
 is either $\QP(A)=1$ or $\QP(A)=0$. Therefore,  it suffices to show that every set $A \in \mathscr T(\dUpsilon)$ is $\{T^{\dUpsilon, \QP}_t\}_{t>0}$-invariant. 
By Proposition \ref{prop: MGS}, we obtain $T^{\dUpsilon, \QP}_{E_h, t}u \to T^{\dUpsilon, \QP}_{ t}u$ in $L^2(\mu)$ as $h \to \infty$  for any $u \in L^2(\mu)$. Thus, it suffices to show that
\begin{align} \label{eq: TS}
\text{every tail set $A \in \mathscr T(\dUpsilon)$ is $\{T^{\dUpsilon, \QP}_{E_h, t}\}_{t>0}$-invariant for any $h \in \N$} \fstop
\end{align}
 Indeed, if it is true, then 
\begin{align*}
T^{\dUpsilon, \QP}_t\1_{A}u = \text{{\small $L^2(\mu)$ --}}\lim_{h\to \infty} T^{\dUpsilon, \QP}_{E_h, t}\1_{A}u = \text{{\small $L^2(\mu)$ --}}\lim_{h\to \infty} \1_{A}T^{\dUpsilon, \QP}_{E_h, t}u = \1_{A}T^{\dUpsilon, \QP}_tu \fstop
\end{align*}
%Let $T^{\dUpsilon, \QP}_{E_h, t}$ be the semigroup corresponding to the localised form $\E^{r}$ on $\U(X)$. By the monotonicity of $\E^r \uparrow \E$, we obtain 

We now show \eqref{eq: TS}. By the definition of the tail $\sigma$-algebra $\mathscr T(\dUpsilon)$, any set $A \in \mathscr T(\dUpsilon)$ has the following expression:
\begin{align} \label{eq: EXP}
A=\U(E_h) \times {\rm pr}_{E_h^c}(A), \quad \forall h \in \N \fstop
\end{align}
%Let $T^{\dUpsilon(E_h), \eta}$ denote the semigroup associated with $\E^{\U(E_h), \mu_{\eta, E_h}}$. 
By Proposition \ref{prop: 1}, for $u \in L^2(\QP)$, 
$$T^{\dUpsilon, \QP}_{E_h, t}u(\gamma) = T^{\U(E_h), \QP^\gamma_{E_h}}_t u_{E_h, \gamma}(\gamma_{E_h}), \quad \text{$\mu$-a.e.\ $\gamma$} \fstop$$
By \eqref{eq: EXP}, the function $(\1_{A})_{E_h, \gamma}$ is equal to the constant $1$ on $\dUpsilon(E_h)$ whenever $\gamma \in A$ and  $(\1_{A})_{E_h, \gamma}=0$ whenever $\gamma \in A^c$, which leads to  
$$T^{\U(E_h), \QP^\gamma_{E_h}}_t (u\1_{A})_{E_h, \gamma}(\gamma_{E_h}) 
= \1_{A}(\gamma)T^{\U(E_h), \QP^\gamma_{E_h}}_t u_{E_h, \gamma}(\gamma_{E_h}) \fstop$$
Therefore, 
 \begin{align*}
 T^{\dUpsilon, \QP}_{E_h, t}\1_{A} u(\gamma) &=  T^{\U(E_h), \QP^\gamma_{E_h}}_t(\1_{A}u)_{E_h, \gamma}(\gamma_{B_r}) = \1_AT^{\U(E_h), \QP^\gamma_{E_h}}_tu_{E_h, \gamma}(\gamma_{E_h}) = \1_{A}T^{\dUpsilon, \QP}_{E_h, t} u(\gamma) \comma
 %\\
 %&= \1_{A}(\gamma_{B_r})T^{\U(B_r), \gamma_{B_r^c}}f_{\gamma_{B^c_r}, B_r^c}(\gamma_{B_r}) = \1_{A}T_t^r f(\gamma), \quad \text{$\mu$-a.e.\ $\gamma$.}
 \end{align*}
for $\mu$-a.e.\ $\gamma$.
The proof of (ii) $\implies$ (i) is complete.
\qed


\subsection{The proof of (i) $\implies$ (ii) of Theorem \ref{thm: Erg}}
In this subsection, we present the proof of (i) $\implies$ (ii) of Theorem \ref{thm: Erg}.
\\
{\it Proof of  (i) $\implies$ (ii) in Theorem \ref{thm: Erg}.}
Let $\seq{E_h}_h$ with $E_h \uparrow X$ be a localising exhaustion witnessing Assumptions~\ref{ass:CE}.
 %Let $\tilde{u}$ be a quasi-continuous modification with increasing compact nests $K_m \subset \U(X)$ so that ${\sf cap}_{\E^\mu}(K_m)\le 1/m$.
%Let $\tilde{K}_m$ be a modification of $K_m$ on which $\tilde{u}|_{K_m}$ is $\tau_v$-continuous. 
%Let $B_{n}\subset X$ be an open increasing exhaustion on $X$. 

By the rigidity in number \ref{ass:Rig}, we can take a measurable set $\Omega_{\sf rig}^h \subset \U$ so that $\mu(\Omega_{\sf rig}^h)=1$ and the rigidity in number on $E_h$ holds, i.e., if $\gamma, \eta \in \Omega_{\sf rig}^h$ with $\gamma_{E_h^c} = \eta_{E_h^c}$, then $\gamma E_h = \eta E_h$. Let $\Omega_{\sf rig} = \cap_{h \in \N} \Omega_{\sf rig}^h$, which is of $\QP$-full measure as well.  

Let $u \in \dom{\EE{\dUpsilon}{\QP}}$ so that $ \cdc^{\dUpsilon, \QP}(u) =0$. 
%By \cite[Prop.\ 5.14 (iii)]{LzDSSuz21},  
%$$\cdc^\dUpsilon_{E_h}(u) \le \cdc^\dUpsilon(u)=0, \quad \text{$\QP$-a.e., \ $\forall h \in \N$}\,.$$
%$\Gamma_{n}:=\Gamma_{B_n}(u) \le \Gamma(u) =0$ for any $n \in \N$. 
%\purple{By a proposition}, 
By \cite[Prop.\ 5.14]{LzDSSuz21},  for any $h \in \N$, there exists $\Omega^h_0 \subset \dUpsilon$ so that $\QP(\Omega^h_0)=1$ and 
$$\cdc^{\dUpsilon(E_h), \QP^\eta_{E_h}}(u_{E_{h}, \eta}) =0 \quad \text{$\QP_{E_h}^\eta$-a.e.} \quad \forall \eta \in \Omega^h_0 \fstop$$
By Assumption~\ref{ass:ConditionalErg} of the irreducibility for the conditioned form, for $h \in \N$, there exists a measurable set $\Omega^h_{\sf rig, ic}\subset \Omega_0^h \cap \Omega_{\sf rig}$ of full $\mu$-measure so that, for all $\eta \in \Omega^h_{\sf rig, ic}$, there exists  $k=k(\eta) \in \N_0$ and a constant $C^k_{E_h, \eta}$ satisfying that 
%$\tilde{u}$ is continuous on $\Omega^{m}_2$.  Therefore, $\tilde{u}_{B_n, \eta_{B_n^c}}$ is continuous on $(\Omega^{m}_2)_{B_n, \eta_{B_n^c}} \subset \U(B_n)$ for every $\eta \in \Omega^m_2$. Since $\E^{m}$ is irreducible, and the irreducibility tensorises on $X^{\times k}$, the form $\E^{m, \otimes k}$ is irreducible on $B_n^{\times k}$. Thus, we see that 
\begin{align} \label{eq: E: erg0}
\text{$u_{E_{h}, \eta}$ is equal to $C^k_{E_h, \eta}$  in $\dUpsilon(E_h)$ $\quad \QP_{E_h}^\eta$-a.e.}\, .
\end{align} 
Note that the measure $\QP_{E_h}^\eta$ is fully supported on $\dUpsilon^{(k(\eta))}(E_h)$ by the rigidity in number \ref{ass:Rig} and \ref{ass:CE}.
Let $\Omega_{\sf rig, ic}:= \cap_{h \in \N}\Omega^h_{\sf rig, ic}$, which satisfies $\mu(\Omega_{\sf rig, ic})=1$. 

By the assumption of the quasi-regularity, there exists a quasi-continuous $\QP$-modification $\tilde{u}$ of $u$ (see \cite[Prop.\ 3.3]{MaRoe90}). 
Therefore, we can take a closed monotone increasing exhaustion $\{K_m\} \uparrow \dUpsilon$ so that $\tilde{u}$ is $\T_\mrmv$-continuous on $K_m$ for any $m \in \N$, and ${\rm Cap}_{\EE{\dUpsilon}{\QP}}(K^c_m)\le 1/m$. 
Set $\Omega_{\sf qc}:=\cup_{m \in \N} K_m$, which is of $\mu$-full measure since $\Omega_{\sf qc}$ is co-negligible with respect to the capacity~${\rm Cap}_{\EE{\dUpsilon}{\QP}}$. Thus, we may assume that $\QP(K_m)>1-\epsilon$ with $\epsilon<1/2$ for all $m \ge m_\epsilon$ whereby $m_\epsilon$ is a sufficiently large integer depending on $\epsilon$.
%Set $\Omega:=\Omega_{\sf rig} \cap \Omega_{\sf ic} \cap \Omega_{\sf qc}$, which is of $\mu$-full measure by construction. 

Let $\Omega_m:=\Omega_{{\sf rig, ic}} \cap K_m$ for $m \ge m_\epsilon$. %Therefore, ${\rm pr}^{E_h}(\Omega)$ 
Since $\tilde{u}$ is $\T_\mrmv$-continuous on $\Omega_m$, the function $\tilde{u}_{E_h, \eta}$ is continuous on $\Omega_{E_h, \eta, m}$ for any $\eta \in \Omega_m$ and any $h \in \N$ whereby $\Omega_{E_h, \eta, m}:=\{\gamma \in \dUpsilon(E_h): \gamma + \eta_{E_h} \in \Omega_m\}$. %We now show that there exists $\Omega_{m, 0} \subset \Omega_m$ so that $\QP(\Omega_{m, 0}) \ge \QP(\Omega_{m})$ and 
By Proposition \ref{p:ConditionalIntegration}, we have that
\begin{align} \label{eq: CEIR}
\QP(\Omega_m)= \int_{\dUpsilon}\QP_{E_h}^\eta \bigl( \Omega_{E_h, \eta, m}\bigr) \diff \QP(\eta) \fstop 
\end{align}
Thus, by noting that $\QP(\Omega_m) \ge 1-\epsilon$ and the integrand of the r.h.s.\ of \eqref{eq: CEIR}  is non-negative and bounded from above by $1$, there exists $\Omega_{m, h} \subset \dUpsilon$ so that $\QP(\Omega_{m, h}) \ge \QP(\Omega_m) \ge 1-\epsilon$ for any $h \in \N$, and for any $\eta \in \Omega_{m, h}$,
\begin{align} \label{eq: Ir: 11}
\QP_{E_h}^\eta \bigl( \Omega_{E_h, \eta, m}\bigr)>0 \fstop
\end{align}
Furthermore, set $\Omega_{m}^h:= \Omega_m \cap \Omega_{m, h}$. Noting that $\epsilon <1/2$ and  $\QP(\Omega_{m, h}) \ge \QP(\Omega_m) \ge 1-\epsilon$, by a simple application of Inclusion-Exclusion formula, it holds that, for any $m \ge m_\epsilon$, 
\begin{align} \label{eq: Ir 111}
\QP(\Omega_{m}^h) \ge 1-2\epsilon \qquad \forall h \in \N \fstop
\end{align}
% Note that, by construction, $\Omega_{m}^h \uparrow \dUpsilon$ $\QP$-a.e.\ as $m \uparrow \infty$ for any $h \in \N$. 
Combining~\eqref{eq: E: erg0}, \eqref{eq: Ir: 11} with the fact that  $\QP_{E_h}^\eta\mrestr{{\dUpsilon^{(k)}(E_h)}}$ is fully supported in $\dUpsilon^{(k)}(E_h)$ and $\tilde{u}$ is $\T_\mrmv$-continuous on $ \Omega_{E_h, \eta, m}$,  we obtain that 
%we may take $\Omega_m \subset \Omega_{m, 0}$ since ${\rm pr}_{E_h^c}\Omega_m={\rm pr}_{E_h^c}\Omega_{m, 0}$ and \eqref{eq: Ir: 11} only depends on ${\rm pr}_{E_h^c}\Omega_{m, 0}$. 
% \quad \forall \eta \in \Omega_{m, 0}.$$
\begin{align} \label{eq: E: erg1}
\text{$\tilde{u}_{E_{h}, \eta}$ is equal to $C^k_{E_h, \eta}$ {\it everywhere} in $\Omega_{E_h, \eta, m}$ for all $\eta \in \Omega_{m}^h$}\, \fstop
\end{align}  
%whereby $\Omega^{m}_{E_h, \eta}:=\{\gamma \in \dUpsilon(E_h): \gamma + \eta_{E_h} \in \Omega_{m, 1}\}$.
%Let $\Omega^h_{{\sf ec}, \eta} := ({\rm pr}^{E_h})^{-1}(\Omega^h_{{\sf ec}, E_h, \eta})$ which satisfies $\mu(\Omega^h_{{\sf ec}, \eta})=1$ as well. 
%Recall by definition that 
%\begin{align} \label{eq: m: 1}
%\tilde{u}(\gamma)=\tilde{u}(\gamma_{E_h}+\gamma_{E_h^c})=\tilde{u}_{E_h, \gamma}(\gamma_{E_h}) \,.
%\end{align}  
%Instead of replacing $\Omega$ by another measurable set in $\Omega$ of $\mu$-full measure on which \eqref{eq: m: 1} holds for all $h \in \N$, we may use the same $\Omega$ without relabelling for the sake of the simpler notation. 
%Without loss of generality, we may assume that \eqref{eq: m: 1} holds for all $\gamma \in \Omega$ (otherwise we can replace $\Omega$ by another measurable set $\Omega' \subset \Omega$).  
%Let $\Omega= K_m \cap \Omega_0\cap_{n \in \N}\Omega^n_1$. We may assume that $\mu(\Omega^{m}_2)>0$ for sufficiently large $m \in \N$. Note that $
%Recall by definition that $\tilde{u}(\gamma)=\tilde{u}(\gamma_{B_n}+\gamma_{B_n^c})=\tilde{u}_{B_n, \gamma_{B_n^c}}(\gamma_{B_n})$ for $\gamma \in \U(X)$. 
%Note that this definition implicitly depends on $n$ since we took the modification $\tilde{u}_{K_n}$ dependently of $K_n$, therefore, we keep the subscript $n$.
%Define $\tilde{u}_\infty(\gamma):=\liminf_{n \to \infty} \tilde{u}_n(\gamma)$. 
By Lemma \ref{lem: IEF} in Appendix applied to $\Omega_m^h$ in \eqref{eq: Ir 111},  we can take $n\mapsto m_n \in \N$ so that, by setting 
$\Omega=\limsup_{n \to \infty}\cap_{h=1}^n\Omega_{m_n}^h$,
it holds that 
%\begin{align} \label{ineq: e:Ir}
$$\QP(\Omega) = 1 \fstop$$ 
%\fstop
%\end{align}

We now prove that 
\begin{align} \label{eq: m: 2}
\text{$\tilde{u}$ is constant $\QP\text{-a.e.}$ on $\Omega$} \fstop
\end{align}
It suffices to show that for any $\Xi_1, \Xi_2 \subset \Omega$ with $\mu(\Xi_1)\mu(\Xi_2)>0$, there exists $\gamma^1 \in \Xi_1$ and $\gamma^2 \in \Xi_2$ so that 
 \begin{align} \label{eq: chyp}
 \tilde{u}(\gamma^1) = \tilde{u}(\gamma^2)\fstop
 \end{align}
 Indeed, take $\Xi_1=\{\tilde{u}>a\}$ and $\Xi_2=\{\tilde{u} \le a\}$ for $a \in \R$. If there exists $a \in \R$ so that $\mu(\Xi_1)\mu(\Xi_2)>0$, then this contradicts \eqref{eq: chyp}. 
 Thus, $\QP(\Xi_1)\QP(\Xi_2)=0$ for any $a \in \R$, which means that $\tilde{u}$ is constant $\QP$-a.e.\ on $\Omega$.
  %and
 %\begin{align} \label{eq: chyp}
 %\tilde{u}(\gamma^1) \neq \tilde{u}(\gamma^2)\quad  \text{for every $\gamma^1 \in \Xi_1$ and $\gamma^2 \in \Xi_2$}.
% \end{align}

We thus prove \eqref{eq: chyp}. Since $\mu$ is tail trivial and $\mu(\Xi_1)>0$, we obtain that $\mu(\mathcal T(\Xi_1))=1$, where $\mathcal T(\Xi_1)$ is the tail set of $\Xi_1$ as defined in~\eqref{eq: ts}. Thus, $\mu(\mathcal T(\Xi_1) \cap \Xi_2)>0$, by which $\mathcal T(\Xi_1) \cap \Xi_2$ is non-empty and, therefore, we can take an element $\gamma^2 \in \mathcal T(\Xi_1) \cap \Xi_2$. Recalling 
\begin{align} \label{eq: LSP}
\Omega=\limsup_{n \to \infty}\cap_{h=1}^n\Omega_{m_n}^h := \bigcap_{n \ge1} \bigcup_{j \ge n} \bigcap_{h=1}^j\Omega_{m_j}^h \comma
\end{align}
and the definition \eqref{eq: ts} of the tail operation and the rigidity in number \ref{ass:Rig}, there exists $h \in \N$, $k \in \N_0$ and $\gamma^1 \in \Xi_1$  so that
 $$\gamma^1_{E_h^c} = \gamma^2_{E_h^c}, \quad \gamma^1{E_h}=\gamma^2{E_h}=k \comma$$
which implies $\gamma^1_{E_h}\in \Omega_{E_h, \gamma^2, m_j}$ for some $j \in \N$. Furthermore, $\QP_{E_h}^{\gamma^2}(\Omega_{E_h, \gamma^2, m_j})>0$ by \eqref{eq: Ir: 11} and the definition \eqref{eq: LSP} of $\Omega$. Therefore, by making use of \eqref{eq: E: erg1}, we conclude 
\begin{align*}
\tilde{u}(\gamma^1) =  \tilde{u}(\gamma^1_{E_h}+\gamma^1_{E_h^c}) = \tilde{u}_{E_h, \gamma^1}(\gamma^1_{E_h})= C^k_{E_h, \gamma^2} =\tilde{u}_{E_h, \gamma^2}(\gamma^2_{E_h})=\tilde{u}(\gamma^2_{E_h}+\gamma^2_{E_h^c}) =\tilde{u}(\gamma^2) \comma
\end{align*}
which proves \eqref{eq: chyp}. 
\qed

\begin{comment}
We now prove that 
\begin{align} \label{eq: m: 2}
\text{$\tilde{u}$ is constant $\QP\text{-a.e.}$ on $\Omega_{m, 1}$}.
\end{align}
It suffices to show that for any $\Xi_1, \Xi_2 \subset \Omega_{m, 1}$ with $\mu(\Xi_1)\mu(\Xi_2)>0$, there exists $\gamma^1 \in \Xi_1$ and $\gamma^2 \in \Xi_2$ so that 
 \begin{align} \label{eq: chyp}
 \tilde{u}(\gamma^1) = \tilde{u}(\gamma^2)\fstop
 \end{align}
 Indeed, take $\Xi_1=\{\tilde{u}>a\}$ and $\Xi_2=\{\tilde{u} \le a\}$ for $a \in \R$. If there exists $a \in \R$ so that $\mu(\Xi_1)\mu(\Xi_2)>0$, then this contradicts \eqref{eq: chyp}. 
 Thus, $\QP(\Xi_1)\QP(\Xi_2)=0$ for any $a \in \R$, which means that $\tilde{u}$ is constant $\QP$-a.e.\ on $\Omega_{m, 1}$.
  %and
 %\begin{align} \label{eq: chyp}
 %\tilde{u}(\gamma^1) \neq \tilde{u}(\gamma^2)\quad  \text{for every $\gamma^1 \in \Xi_1$ and $\gamma^2 \in \Xi_2$}.
% \end{align}

We now prove \eqref{eq: chyp}. Since $\mu$ is tail trivial and $\mu(\Xi_1)>0$, we obtain that $\mu(\mathcal T(\Xi_1))=1$, where $\mathcal T(\Xi_1)$ is the tail set of $\Xi_1$ as defined in~\eqref{eq: ts}. Thus, $\mu(\mathcal T(\Xi_1) \cap \Xi_2)>0$, by which $\mathcal T(\Xi_1) \cap \Xi_2$ is non-empty and, therefore, we can take an element $\gamma^2 \in \mathcal T(\Xi_1) \cap \Xi_2$. 
By the definitions of the tail operation and the rigidity in number \ref{ass:Rig}, there exists $h \in \N$, $k \in \N_0$ and $\gamma^1 \in \Xi_1$  so that
 $$\gamma^1_{E_h^c} = \gamma^2_{E_h^c}, \quad \gamma^1{E_h}=\gamma^2{E_h}=k,$$
which implies $\gamma^1_{E_h}\in \Omega_{E_h, \gamma^2, m}$. Therefore, by making use of \eqref{eq: E: erg1}, we conclude 
\begin{align*}
\tilde{u}(\gamma^1) =  \tilde{u}(\gamma^1_{E_h}+\gamma^1_{E_h^c}) = \tilde{u}_{E_h, \gamma^1}(\gamma^1_{E_h})= C^k_{E_h, \gamma^2} =\tilde{u}_{E_h, \gamma^2}(\gamma^2_{E_h})=\tilde{u}(\gamma^2_{E_h}+\gamma^2_{E_h^c}) =\tilde{u}(\gamma^2),
\end{align*}
which proves \eqref{eq: chyp}. 

We finally upgrade \eqref{eq: m: 2} to the statement:
\begin{align} \label{eq: m: 3}
\text{$\tilde{u}$ is constant $\QP\text{-a.e.}$ on $\Omega$},
\end{align}
whereby $\Omega:=\cup_{m} \Omega_{m, 1}$ and $\QP(\Omega)=1$ by construction. We adopt the same strategy as \eqref{eq: chyp}. Namely, we see that for any $\Xi_1, \Xi_2 \subset \Omega$ with $\mu(\Xi_1)\mu(\Xi_2)>0$, there exists $\gamma^1 \in \Xi_1$ and $\gamma^2 \in \Xi_2$ so that 
 \begin{align} \label{eq: chyp2}
 \tilde{u}(\gamma^1) = \tilde{u}(\gamma^2)\fstop
 \end{align}
Since $\QP(\Omega)=1$ and $\Omega_{m, 1} \uparrow \dUpsilon$ $\QP$-a.e.,  we may take a sufficiently large $m \ge m_\epsilon$ so that both $\QP(\Omega_{m, 1} \cap \Xi_1)>0$ and $\QP(\Omega_{m, 1} \cap \Xi_2)>0$ hold. 
Then, by applying the conclusion \eqref{eq: chyp}  to $\Omega_{m, 1} \cap \Xi_1 \subset \Omega_{m, 1}$ and $\Omega_{m, 1} \cap \Xi_2 \subset \Omega_{m, 1}$ in place of $\Xi_1$ and $\Xi_2$,  we conclude \eqref{eq: chyp2}. The proof is complete. 
%The proof is complete. 
%Here the second equality is the definition, and the third equality follows from \eqref{eq: m: erg1}.
%and by recalling the definition
%\begin{align} \label{eq: m: 1}
%\tilde{u}(\gamma)=\tilde{u}(\gamma_{E_h}+\gamma_{E_h^c})=:\tilde{u}_{E_h, \gamma}(\gamma_{E_h}) \,,
%\end{align}
%and by using \eqref{eq: m: erg1}, 
% Recall that $\overline{\Xi}_1:=\cup_{m} \overline{\Xi}_{1, m}$ where $\overline{\Xi}_{1,m}:={\rm pr}_{B_m^c}^{-1}({\rm pr}_{B_m^c}(\Xi))$. Define 
 %$$\overline{\Xi}^k_{1,m}:=\overline{\Xi}_{1,m} \cap \{\gamma \in \U(X): \gamma B_n =k\}.$$
 %Then, 
 %$$\overline{\Xi}_{1}= \bigcup_{m}\overline{\Xi}_{1, m}=\bigcup_{m} \bigsqcup_{k \in \N} \overline{\Xi}^k_{1,m}: =  \bigsqcup_{k \in \N}\bigcup_{m}\overline{\Xi}^k_{1,m}.$$
 %Since $\mu(\overline{\Xi}_1)=1$ and $\mu(\Xi_2)>0$, there exists $k$ so that 
 %$$\mu\Bigl(\Xi_2 \cap \bigcup_{m}\overline{\Xi}^k_{1,m}\Bigr)>0.$$
 %Thus, $\Xi_2 \cap \bigcup_{m}\overline{\Xi}^k_{1,m} \neq \emptyset$ and take $\gamma^2 \in \Xi_2 \cap \bigcup_{m}\overline{\Xi}^k_{1,m}$. By the definition of the tail operation and the property of the rigid set $\Omega_0$, there exists $B_n$ and $\gamma^1 \in \Xi_1$ so that
 %$$\gamma^1_{B_n^c} = \gamma^2_{B_n^c}, \quad \gamma^1{B_n}=\gamma^2{B_n}=k.$$
%By the argument in the first paragraph, we conclude that 
%$$\tilde{u}(\gamma^1) = \tilde{u}(\gamma^1_{B_n}+\gamma^1_{B_n^c})=\tilde{u}_{B_n, \gamma^1_{B_n^c}}(\gamma^1_{B_n}) = C_{\eta, B_n}^k=\tilde{u}_{B_n, \gamma^1_{B_n^c}}(\gamma^2_{B_n})= \tilde{u}_{B_n, \gamma^2_{B_n^c}}(\gamma^2_{B_n})=\tilde{u}(\gamma^2).$$
%This contradicts the hypothesis. 
%Noting that $\{\Omega^m_2\}$ is monotone increasing, take $\Omega= \uparrow_{m} \Omega^{m}_2:=\cup_m \Omega^m_2$. Then, $\mu(\Omega)=1$ by definition of $\Omega_2^m$ and the nest $K_m$.  Since $\tilde{u}$ is constant on $\Omega_2^m$ for any $m$ by the argument in the previous paragraph, $\tilde{u}$ is constant everywhere on $\Omega$. The proof is complete.
%Therefore, it is invariant under actions of $\phi_v \in X_0(\R^n)$ with $v \in V_0(\R^n)$. Namely, $\phi_v(A) = A$ for any $v \in V_0(\R^n)$. 
%
%We now prove that for every $\phi_v$-invariant set $B$  it holds either $\mu(B)=0$ or $\mu(B)=1$. By the hypothesis of the ergodicity of $(\E^\mu, \F^\mu)$, it suffices to show that $\1_B \in \F^\mu$. Since $B$ is $\phi_v$-invariant for any $v \in V_0$, we have that 
%$$\frac{\1_B(\phi_v(\gamma, t))-\1_B(\phi(\gamma))}{t} = 0, \quad \forall \gamma \in \U.$$
%Thus $\nabla_v \1_B=0$, which concludes $\1_B \in \F^\mu$. 
\qed
\end{comment}

\begin{cor}\label{cor: Erg0}
Let $\mcX$ be a \TLDS and $\QP$ be a probability measure on~$\ttonde{\dUpsilon,\A_{\mrmv}(\msE)}$ satisfying~\ref{ass:Mmu}. Suppose that  Assumptions~\ref{ass:CE}, \ref{ass:ConditionalClos}, \ref{ass:ConditionalErg}, and~\ref{ass:Rig} hold and that $\ttonde{\EE{\dUpsilon}{\QP},\dom{\EE{\dUpsilon}{\QP}}}$ is quasi-regular. 
%Suppose further that $(\E^{m, X}, \mathcal D(\E^{m, X}))$ is irreducible on $X$\footnote{We probably only need the irreducibility of $\E^{m, E}$ on any localising sets $E$ for the assumption. See the proof.}. 
%Suppose that $(\E^\mu, \F^\mu)$ is quasi-regular
 %$(Irr)_{B_r, \otimes n}$ $(\forall r>0, n \in \N)$\footnote{I believe that the irreducibility on $X$ tensorises and we only need to assume (Irr) on $X$. Then this is purely measure theoretic assumption and we never use the distance information.}. 
Then, the following  are equivalent:
\begin{enumerate}[$(i)$]
\item  $\mu$ is tail trivial.
\item  $\ttonde{\EE{\dUpsilon}{\QP},\dom{\EE{\dUpsilon}{\QP}}}$ is irreducible;
\item $\{T^{\dUpsilon, \QP}_t\}$ is irreducible, i.e., any $A\in \A_\mrmv(\msE)^\QP$ with
 $$T^{\dUpsilon, \QP}_t (\1_{A}f) = \1_{A}T^{\dUpsilon, \QP}_t f,\quad  \forall f \in L^2(\QP) $$
 satisfies either $\QP(A)=1$ or $\QP(A)=0$;
\item $\{T^{\dUpsilon, \QP}_t\}$ is ergodic, i.e., 
\begin{align*}
\int_{\dUpsilon} \biggl( T^{\dUpsilon, \QP}_t u - \int_{\dUpsilon} u \diff \QP \biggr)^2 \diff \QP \xrightarrow{t \to \infty} 0, \quad \forall u \in L^2(\QP);
\end{align*}
\item $\Delta^{\dUpsilon, \QP}$-harmonic functions are trivial, i.e., 
$$\text{If}\ u \in \dom{\Delta^{\dUpsilon, \QP}}\ \text{and}\  \Delta^{\dUpsilon, \QP}u =0, \quad \text{then}\  \ u=\text{const.}\,.$$
%\item[(iii)] ${\sf d}_{\U, 0}(A, B)<\infty$ for any $A, B$ with $\mu(A)\mu(B)>0$.
%Furthermore, if (Rad)$_{{\sf d}_{\U}}$ holds, then 
\end{enumerate}
\end{cor}
\proof
The equivalence (i)$\iff$(ii) is the consequence of Theorem~\ref{thm: Erg}. The equivalences (ii)$\iff$(iii)$\iff$(iv)$\iff$(v) are standard in functional analysis, we refer the readers to, e.g., \cite[Prop.\ 2.3 and Appendix]{AlbKonRoe97}. 
\qed
\smallskip

%Corollary \ref{cor: Erg} describes the ergodicity from the operator theoretic viewpoint, which gives us the information about the behaviour of the corresponding particle systems in average,  In the following, we further obtain the ergodicity from the viewpoint of each realisation. In other words, we obtain the ergodicity in the pathwise sense.  
Let $(\mathbf X_t, \mathbf P_\gamma)$ be the system of Markov processes associated with the quasi-regular Dirichlet form $\ttonde{\EE{\dUpsilon}{\QP},\dom{\EE{\dUpsilon}{\QP}}}$ (see \cite[Thm.\ 3.5 in Chap.\ IV]{MaRoe90}). We write $\mathbf P_{\nu}$ for $\int_{\dUpsilon} \mathbf P_\gamma(\cdot) d\nu(\gamma)$ for a bounded Borel measure $\nu$ on $\dUpsilon$.
\begin{cor}\label{cor: Erg1}
Let $\mcX$ be a \TLDS and $\QP$ be a probability measure on~$\ttonde{\dUpsilon,\A_{\mrmv}(\msE)}$ satisfying~\ref{ass:Mmu}.  Suppose that  Assumptions~\ref{ass:CE}, \ref{ass:ConditionalClos}, \ref{ass:ConditionalErg}, and~\ref{ass:Rig} hold and that $\ttonde{\EE{\dUpsilon}{\QP},\dom{\EE{\dUpsilon}{\QP}}}$ is quasi-regular. 
%Suppose further that $(\E^{m, X}, \mathcal D(\E^{m, X}))$ is irreducible on $X$\footnote{We probably only need the irreducibility of $\E^{m, E}$ on any localising sets $E$ for the assumption. See the proof.}. 
%Suppose that $(\E^\mu, \F^\mu)$ is quasi-regular
 %$(Irr)_{B_r, \otimes n}$ $(\forall r>0, n \in \N)$\footnote{I believe that the irreducibility on $X$ tensorises and we only need to assume (Irr) on $X$. Then this is purely measure theoretic assumption and we never use the distance information.}. 
If either one of (i)--(v) in Corollary~\ref{cor: Erg0} holds, then the following are true: 
\begin{enumerate}[$(i)$]
\item  for any Borel measurable $\QP$-integrable function $u$, it holds in $\mathbf P_\QP$-a.e.\ that
\begin{align} \label{eq: TE3}
\lim_{t \to \infty} \frac{1}{t} \int_0^t u(\mathbf X_s)ds = \int_{\dUpsilon} u \diff \QP;
\end{align}
\item for any non-negative bounded function $h$, \eqref{eq: TE3} holds in $L^1(\mathbf P_{h\cdot\QP})$; 
\item The convergence \eqref{eq: TE3} holds $\mathbf P_\gamma$-a.s.\ for $\EE{\dUpsilon}{\QP}$-q.e.\ $\gamma$.
\end{enumerate}
\end{cor}
\proof
The form $\ttonde{\EE{\dUpsilon}{\QP},\dom{\EE{\dUpsilon}{\QP}}}$ is irreducible by Theorem~\ref{thm: Erg}. Furthermore, it is recurrent by $\1 \in \dom{\EE{\dUpsilon}{\QP}}$ and $\EE^{\dUpsilon, \QP}(\mathbf 1)=0$. (See e.g., \cite[Thm.\ 1.6.3]{FukOshTak11} for the characterisation of the recurrence).  Therefore, by \cite[Thm.\ 4.7.3]{FukOshTak11}, the proof is complete (although \cite[Thm.\ 4.7.3]{FukOshTak11} assumes the local compactness of the state space, the same proof follows verbatim). 
\qed

\begin{comment}
We now prove that 
$$\text{$\tilde{u}(\gamma)$ is constant $\mu$-a.e.\ $\gamma$ in $\Omega^{m}_2$}.$$
We prove it by contradiction. Suppose that there exist $\Xi_1, \Xi_2 \subset \Omega^m_2$ so that $\mu(\Xi_i)>0$ for $i=1,2$ and
 $$\tilde{u}(\gamma^1) \neq \tilde{u}(\gamma^2)\quad  \text{for every $\gamma^1 \in \Xi_1$ and $\gamma^2 \in \Xi_2$}.$$
 Since $\mu$ is tail trivial, we obtain that $\mu(\overline{\Xi}_1)=1$ where $\overline{\Xi}_1$ is the tail set of $\Xi_1$. 
 Recall that $\overline{\Xi}_1:=\cup_{m} \overline{\Xi}_{1, m}$ where $\overline{\Xi}_{1,m}:={\rm pr}_{B_m^c}^{-1}({\rm pr}_{B_m^c}(\Xi))$. Define 
 $$\overline{\Xi}^k_{1,m}:=\overline{\Xi}_{1,m} \cap \{\gamma \in \U(X): \gamma B_n =k\}.$$
 Then, 
 $$\overline{\Xi}_{1}= \bigcup_{m}\overline{\Xi}_{1, m}=\bigcup_{m} \bigsqcup_{k \in \N} \overline{\Xi}^k_{1,m}: =  \bigsqcup_{k \in \N}\bigcup_{m}\overline{\Xi}^k_{1,m}.$$
 Since $\mu(\overline{\Xi}_1)=1$ and $\mu(\Xi_2)>0$, there exists $k$ so that 
 $$\mu\Bigl(\Xi_2 \cap \bigcup_{m}\overline{\Xi}^k_{1,m}\Bigr)>0.$$
 Thus, $\Xi_2 \cap \bigcup_{m}\overline{\Xi}^k_{1,m} \neq \emptyset$ and take $\gamma^2 \in \Xi_2 \cap \bigcup_{m}\overline{\Xi}^k_{1,m}$. By the definition of the tail operation and the property of the rigid set $\Omega_0$, there exists $B_n$ and $\gamma^1 \in \Xi_1$ so that
 $$\gamma^1_{B_n^c} = \gamma^2_{B_n^c}, \quad \gamma^1{B_n}=\gamma^2{B_n}=k.$$
By the argument in the first paragraph, we conclude that 
$$\tilde{u}(\gamma^1) = \tilde{u}(\gamma^1_{B_n}+\gamma^1_{B_n^c})=\tilde{u}_{B_n, \gamma^1_{B_n^c}}(\gamma^1_{B_n}) = C_{\eta, B_n}^k=\tilde{u}_{B_n, \gamma^1_{B_n^c}}(\gamma^2_{B_n})= \tilde{u}_{B_n, \gamma^2_{B_n^c}}(\gamma^2_{B_n})=\tilde{u}(\gamma^2).$$
This contradicts the hypothesis. 

Noting that $\{\Omega^m_2\}$ is monotone increasing, take $\Omega= \uparrow_{m} \Omega^{m}_2:=\cup_m \Omega^m_2$. Then, $\mu(\Omega)=1$ by definition of $\Omega_2^m$ and the nest $K_m$.  Since $\tilde{u}$ is constant on $\Omega_2^m$ for any $m$ by the argument in the previous paragraph, $\tilde{u}$ is constant everywhere on $\Omega$. The proof is complete.
\end{comment}

\bibliographystyle{alpha}
\bibliography{MasterBib.bib}
\end{document}



























%\paragraph{Cylinder functions}%We shall start by defining a core of \emph{cylinder functions} for the form \eqref{eq:DirichletForm}.
%For~$\gamma\in \dUpsilon$ and~$ f\in\Sb(\msE)$ let~$ f^\trid\colon \dUpsilon\rar \R$ be defined as 
%\begin{equation}\label{eq:Trid}
% f^\trid\colon \gamma\longmapsto \int_X  f(x)\diff\gamma(x) 
%\end{equation}
%and set further
%\begin{align*}
%\mbff^\trid\colon \gamma\longmapsto\tparen{ f_1^\trid\gamma,\dotsc,  f_k^\trid\gamma}\in \R^k\comma \qquad  f_1,\dotsc,  f_k\in \Sb(\msE)\fstop
%\end{align*}
%%\begin{defs}[Cylinder functions on~$\dUpsilon$] 
%%Let~$\mcX$ be a topological local structure, and~$ D$ be a linear subspace of~$\Sb(\msE)$. 
%We define the space of \emph{cylinder functions}
%\begin{align} \label{defn: cyl}
%\Cyl{ D}\eqdef \set{\begin{matrix}  u\colon \dUpsilon\rar \R :  u=F\circ\mbff^\trid \comma  F\in \mcC^\infty_b(\R^k)\comma \\  f_1,\dotsc,  f_k\in  \mathcal D\comma\quad k\in \N_0 \end{matrix}}\fstop
%\end{align}
%%\end{defs}
%It is readily seen that cylinder functions of the form~$\Cyl{\Sb(\msE)}$ are $\A_\mrmv(\msE)$-measurable.
%If~$\mathcal D$ generates the $\sigma$-algebra~$\A$ on~$X$, then~$\Cyl{\mathcal D}$ generates the $\sigma$-algebra~$\A_\mrmv(\msE)$ on~$\dUpsilon$. We stress that the representation of~$ u$ by~$ u=F\circ\mbff$ is \emph{not} unique.

%\paragraph{Lifted square field operators}By the result~\cite[Lem.\ 2.16, Thm.\ 3.48]{LzDSSuz21}, there exists the square field operator~$\cdc^{\dUpsilon, \purple{\QP}}$\footnote{$\QP$ has not been introduced in the previous sections. So this paragraph should be placed after the definition of the conditional equivalence.} on~$\dUpsilon$ lifted from the square field~$\cdc$ on the base space~$\mcX$, i.e., $\cdc^{\dUpsilon, \QP}$ is described by $\cdc$ in the following form on $\Cyl{\Dz}$:
%, and it satisfies the Leibniz rule on $\Cyl{\Dz}$ \ $\QP$-a.e.:
%\begin{equation}\label{eq:d:LiftCdCRep}
%\begin{gathered}
%\cdc^{\dUpsilon, \QP}(u,  u)(\gamma) = \sum_{x\in \gamma} \gamma_x^{-1}\cdot \cdc\tonde{u\ttonde{\car_{X\setminus\set{x}}\cdot\gamma + \gamma_x\delta_\bullet}-u\ttonde{\car_{X\setminus\set{x}}\cdot\gamma}}(x), \quad \text{$\QP$-a.e.\ $\gamma$} \comma
%\sum_{i,j=1}^{k,m} (\partial_i F)(\mbff^\trid\gamma) \cdot (\partial_j G)(\mbfg^\trid\gamma) \cdot \cdc( f_i,  g_j)^\trid \gamma \quad \text{$\QP$-a.e.\ $\gamma$}\comma
% u, v\in \Cyl{\Dz} \fstop
%\end{gathered}
%\end{equation}
%and the square field $\cdc^{\dUpsilon, \QP}$ satisfies the following {\it diffusion property} (in other words, chain rule):
%\begin{equation}\label{eq:d:LiftCdCRep1}
%\begin{gathered}
%\cdc^{\dUpsilon, \QP}(u,  v)(\gamma) =\sum_{i,j=1}^{k,m} (\partial_i F)(\mbff^\trid\gamma) \cdot (\partial_j G)(\mbfg^\trid\gamma) \cdot \cdc( f_i,  g_j)^\trid \gamma \quad \text{$\QP$-a.e.\ $\gamma$}\comma 
%\end{gathered}
%\end{equation}
%for  $u=F\circ \mbff^\trid$ and $v=G\circ \mbfg^\trid\in \Cyl{\Dz}$.
%\begin{rem} \label{rem: lift}
%If the square field operator~$\cdc$ on the base space~$\mcX$ maps $\cdc: \Dz^{\otimes 2} \to \A_b(X)$ (as opposed to $\cdc: \Dz^{\otimes 2} \to L^\infty(\mssm)$), then the r.h.s.~of~\eqref{eq:d:LiftCdCRep} or of~\eqref{eq:d:LiftCdCRep1} per se can work as the definition of~$\cdc^{\dUpsilon}(u, v)(\gamma)$ for {\it every} $\gamma$ (as opposed to {\it a.e.}\ $\gamma$), see \cite[Lem.\ 1.2]{MaRoe00}. In particular, it is enough for the readers who are interested only in the case where $\mcX$ is a smooth Riemannian manifold~$(M, g)$ with the standard squire field $\cdc(\cdot)=|\nabla \cdot|_g^2$ induced by $g$, see \cite{AlbKonRoe98, AlbKonRoe98b}. 
%
%However, if $\mcX$ is a singular space such as a metric measure space equipped with the minimal weak upper gradient (e.g., in the sense of \cite[Def. 5.11]{AmbGigSav14}), the square field~$\cdc$ in the base space $\mcX$ is defined only $\mssm$-a.e.\ even on the core $\Dz$, i.e., it holds only $\cdc: \Dz^{\otimes 2} \to L^\infty(\mssm)$, in which case the pointwise definition of $\cdc( f_i,  g_j)^\trid \gamma$ in the r.h.s.~of~\eqref{eq:d:LiftCdCRep} loses its meaning since countably many points, on which configurations are supported, are negligible sets for any non-atomic measures $\mssm$. In this case, we need further arguments to justify the formula \eqref{eq:d:LiftCdCRep}. For so doing in \cite{LzDSSuz21}, we rely upon {\it lifting map} $\ell: L^\infty(\mssm) \to \mathcal L^\infty$, which selects representative of $\cdc( f_i,  g_j)$ in a systematic way in the sense that $\ell$ is an order-preserving homomorphism.  As these arguments are quite technical,  we refer the readers to \cite[\S2.3.1, \S3.4.2]{LzDSSuz21} for details.
%\end{rem}
%%By~\cite[Lem.~1.2]{MaRoe00} the bilinear form~$\cdc^\dUpsilon$ is well-defined on~$\Cyl{\Dz}^\tym{2}$, in the sense that~$\cdc^\dUpsilon( u,  v)$ does not depend on the choice of representatives $ u=F\circ\mbff$ and $ v=G\circ\mbfg$ for~$ u$ and~$ v$.
%%We refer the reader to~\cite[\S.1]{MaRoe00} and~\cite[\S2.3.1]{LzDSSuz21} for details on the issue of well-posedness on cylinder functions.
%%For the pointwise defined square field~$\cdc_\ell$ corresponding to a strong lifting~$\ell$ on~$(\mcX,\cdc)$ let us set
%%Since~$\cdc^{\dUpsilon}_\ell( u,  v)(\gamma)$ is everywhere well-defined in the sense above,~$\EE{\dUpsilon}{\PP}_\ell$ is a well-defined bilinear form on the space of representatives~$\mcL^2(\PP)$.
%%It is however not obvious that the $\PP$-class~$\cdc^{\dUpsilon}_\ell(u, v)$ is defined independently of the chosen representatives~$ u$,~$ v$ of the corresponding $\PP$-classes functions~$u$,~$v$.
%%
%%It is shown in~\cite{LzDSSuz21} that~$\cdc^\dUpsilon_\ell$ descends to a bilinear symmetric functional~$\cdc^\dUpsilon_\ell$ on the space of $\PP$-classes $\CylQP{\PP}{\Dz}$ of the cylinder functions $\Cyl{\Dz}$, which proves that~$\EE{\dUpsilon}{\PP}_\ell$ descends to a non-relabeled well-defined pre-Dirichlet form on~$L^2(\PP)$.
%%
%%The closure of this form will be the main object of our study throughout this work.
%
%
%\begin{comment}
%\begin{prop}[Closability,~{\cite[Prop.~3.9]{LzDSSuz21}}]\label{p:MRLifting}
%Let~$(\mcX,\cdc)$ be a \TLDS, and~$\ell$ be a strong lifting.
%Then, the form
%\begin{align*}
%\tparen{\EE{\dUpsilon}{\PP},\CylQP{\PP}{\Dz}}=\tparen{\EE{\dUpsilon}{\PP}_\ell,\CylQP{\PP}{\Dz}}
%\end{align*}
%is well-defined, densely defined and closable on~$L^2(\PP)$, and independent of~$\ell$.
%Its closure $\tparen{\EE{\dUpsilon}{\PP},\dom{\EE{\dUpsilon}{\PP}}}$ is a Dirichlet form with carr\'e du champ operator~$\tparen{\SF{\dUpsilon}{\PP},\dom{\SF{\dUpsilon}{\PP}}}$ satisfying
%\begin{equation*}
%\SF{\dUpsilon}{\PP}(u,v)=\cdc^\dUpsilon(u,v) \as{\PP}\comma\qquad u,v\in \CylQP{\PP}{\Dz}\fstop
%\end{equation*}
%\end{prop}
%
%We denote by
%\begin{align*}
%\tparen{\LL{\dUpsilon}{\PP},\dom{\LL{\dUpsilon}{\PP}}}\comma \qquad \text{resp.} \qquad \TT{\dUpsilon}{\PP}_\bullet\eqdef\tseq{\TT{\dUpsilon}{\PP}_t}_{t\geq 0}\comma
%\end{align*}
%the $L^2(\PP)$-generator, resp.~$L^2(\PP)$-semigroup, corresponding to~$\tparen{\EE{\dUpsilon}{\PP},\dom{\EE{\dUpsilon}{\PP}}}$.
%
%The domain~$\dom{\EE{\dUpsilon}{\PP}}$ is in fact much larger than the class of cylinder functions~$\CylQP{\PP}{\Dz}$, as recalled below.
%%
%Let~$\coK{\Dz}{1,\mssm}$ be the abstract linear completion of~$\Dz$ w.r.t.~the norm (cf.~\cite[p.~301]{MaRoe00})
%\begin{align*}
%\norm{\emparg}_{1,\mssm}\eqdef \EE{X}{\mssm}(\emparg)^{1/2} + \norm{\emparg}_{L^1(\mssm)} 
%\comma \end{align*}
%endowed with the unique (non-relabeled) continuous extension of~$\norm{\emparg}_{1,\mssm}$ to the completion $\coK{\Dz}{1,\mssm}$.
%
%%\begin{defs}[{\cite[\S4.2, p.~300]{MaRoe00}}] A function~$ u\colon \dUpsilon\rar \R\cup\set{\pm\infty}$ is called \emph{extended cylinder} if there exist~$k\in \N_0$, functions~$ f_1,\dotsc,  f_k$ with~$f_1,\dotsc,f_k\in \coK{\Dz}{1,\mssm}$, and a function~$F\in \Cb^\infty(\R^k)$, so that~$ u=F\circ\mbff$.
%%We denote by~$\Cyl{\coK{\Dz}{1,\mssm}}$ the space of all extended cylinder functions, and by~$\CylQP{\PP}{\coK{\Dz}{1,\mssm}}$ the space of their $\PP$-representa\-tives.
%%\end{defs}
%\end{comment}
%
%\begin{comment}
%
%\begin{prop}[{\cite[Prop.~4.6]{MaRoe00}}, {\cite[Prop.~3.52]{LzDSSuz21}}]\label{p:ExtDom}
%Let~$(\mcX,\cdc)$ be a \TLDS.
%Then,
%\begin{enumerate}[$(i)$]
%\item\label{i:p:ExtDom:1}~$u=\tclass[\PP]{F\circ \mbff^\trid}\in \CylQP{\PP}{\coK{\Dz}{1,\mssm}}$ is defined in~$\dom{\EE{\dUpsilon}{\PP}}$ and independent of the $\mssm$-rep\-re\-sen\-ta\-tives~$ f_i$ of~$f_i$;
%
%\item\label{i:p:ExtDom:2} for~$ u=F\circ \mbff^\trid$ and $ v=G\circ \mbfg^\trid\in \Cyl{\coK{\Dz}{1,\mssm}}$,
%\begin{align}\label{eq:d:LiftCdC}
%\SF{\dUpsilon}{\PP}(u,v)(\gamma)=& \sum_{i,j=1}^{k,m} (\partial_i F)(\mbff^\trid\gamma) \cdot (\partial_j G)(\mbfg^\trid\gamma) \cdot \cdc( f_i,  g_j)^\trid \gamma \as{\PP} \semicolon
%\end{align}
%
%\item\label{i:p:ExtDom:3} for every~$ f$ with~$f \in \coK{\Dz}{1,\mssm}$ one has~$\ttclass[\PP]{ f^\trid}\in\CylQP{\PP}{\coK{\Dz}{1,\mssm}}$, and
%\begin{align*}
%\SF{\dUpsilon}{\PP}\tparen{\ttclass[\PP]{ f^\trid}}=\class[\PP]{\cdc( f)^\trid}
%\end{align*}
%is independent of the chosen $\mssm$- (i.e.,~$\mssm$-)representative~$ f$ of~$f$.
%\end{enumerate}
%\end{prop}
%\end{comment}
%
%
%\paragraph{Conditional Closability and Closability}%In order to state the conditional closability, we introduce the projected conditional measures.
%%Let~$(\mcX,\cdc)$ be a \TLDS, and recall that~$\cdc$ is assumed to be defined on~$\Dz\subset \Cz(\msE)$.
%%In particular, we shall always assume that~$f\in\Dz$ is identified with its continuous representative in~$\Dz=\ell(\Dz)$, where $\ell\colon L^\infty(\mssm)\to\mcL^\infty(\mssm)$ is any strong lifting.
%%\begin{defs}[Projected conditional measures]\label{d:ConditionalQP}
%For a probability measure~$\QP$ on~$\ttonde{\dUpsilon,\A_{\mrmv}(\msE)}$ and $E \in \msE$, set {\it the restricted measure on $\{\gamma \in \dUpsilon: \gamma E = k\}$}:
%\begin{align}\label{defn: RP}
%\QP^{k, E}(\cdot):=\QP\bigl(\cdot \cap \{\gamma \in \dUpsilon: \gamma E = k\} \bigr) \fstop
%\end{align}
%For each fixed~$\eta\in\dUpsilon$, and for each fixed~$E\in\msE$, we let~$\QP^{\eta_{E^\complement}}$ be the regular conditional probability strongly consistent with~$\pr^{E^\complement}$ (i.e., $(\pr^{E^\complement})^{-1}(\eta)$ is co-negligible with respect to $\QP^{\eta_{E^\complement}}$)\footnote{Write a reference for this definition, e.g., the first paper.}. By the definition of the conditional probability, it holds 
%\begin{equation}\label{eq:ConditionalQP}
%\QP\tonde{\Lambda\cap (\pr^{E^\complement})^{-1}(\Xi)}=\int_\Xi \QP^{\eta_{E^\complement}} \Lambda \, \diff\QP(\eta) \comma
%\end{equation}
%for any $\Lambda, \Xi \in \A_{\mrmv}(\msE)$ where, with slight abuse of notation, we regard~$\dUpsilon(E)$ as a subset of~$\dUpsilon$, and thus~$\pr^{E^\complement}$ as a map~$\pr^{E^\complement}\colon\dUpsilon\to\dUpsilon$.
%In probabilistic notation,
%\begin{equation*}
%\QP^{\eta_{E^\complement}}=\QP\ttonde{\emparg \big |\, \pr^{E^\complement}(\emparg)=\eta_{E^\complement}} \fstop
%\end{equation*}
%The \emph{projected conditional probabilities of~$\QP$} are the system of measures on $\dUpsilon(E)$:
%\begin{equation}\label{eq:ProjectedConditionalQP}
%\QP^\eta_E\eqdef \pr^E_\pfwd \QP^{\eta_{E^\complement}}\comma \qquad \eta\in\dUpsilon\comma \qquad E\in\msE \fstop
%\end{equation}
%%\end{defs}
%The restriction of $\QP^\eta_E$ on $\dUpsilon^{(k)}(E_h)$ is denoted by
%\begin{equation}\label{eq:ProjectedConditionalQP1}
%\QP^{\eta, k}_E\eqdef \QP^\eta_E\mrestr{\dUpsilon^{(k)}(E_h)}\comma \qquad \eta\in\dUpsilon\comma \qquad E\in\msE \fstop
%\end{equation}
%%\purple{Set\footnote{Definition of quasi-Gibbs should be more rigorous in relation to rigidity} 
%%\begin{equation*}
%%\QP^{\eta_{E^\complement}, k}(d\gamma)=\QP\ttonde{d\gamma \in \dUpsilon \big |\, \pr^{E^\complement}(\gamma)=\eta_{E^\complement}\,, \, \gamma E=k} \comma
%%\end{equation*}
%%and 
%%\begin{equation*}
%%\QP^{\eta, k}_{E}(d\gamma)= \pr^E_\pfwd \QP\ttonde{d\gamma \in \dUpsilon(E)\big |\, \pr^{E^\complement}(\gamma)=\eta_{E^\complement}\,, \, \gamma E=k} \fstop
%%\end{equation*}
%%}
%\begin{defs}[Conditional equivalence]\label{d:ConditionalAC}
%A probability measure~$\QP$ on $\ttonde{\dUpsilon,\A_{\mrmv}(\msE)}$ 
%%\emph{conditionally absolutely continuous} (\emph{w.r.t.~$\PP_\mssm$}) if there exists a localizing sequence~$\seq{E_h}_h$ such that the projected conditional probabilities satisfy
%%\begin{equation}\tag*{$(\mathsf{CAC})_{\ref{d:ConditionalAC}}$}
%%\label{ass:CAC}
%%\forallae{\QP} \eta\in\dUpsilon \qquad \QP^\eta_{E_h} \ll \PP_{\mssm_{E_h}} \comma \qquad h\in\N \fstop
%%\end{equation}
%is \emph{conditionally equivalent} (\emph{to~$\PP_\mssm$}) if there exists a localising exhaustion $\{E_h\}_{h \in \N} \subset \msE$ with $E_h \uparrow X$ so that, for any $E \in \{E_h\}_{h \in \N}$, $k \in \N_0$ and for $\QP^{k, E}$-a.e.\ $\eta$,
%\begin{equation}\tag*{$(\mathsf{CE})_{\ref{d:ConditionalAC}}$}
%\label{ass:CE}
% \QP^{\eta, k}_{E} \sim \PP_{\mssm_{E}}\mrestr{\dUpsilon^{(k)}(E)}  \fstop
%\end{equation}
%\end{defs}
%\begin{rem}[{Comparison with \cite[Def. 3.41]{LzDSSuz21}}] \label{rem: CP1} We compare  \ref{ass:CE} with \cite[Def. 3.41]{LzDSSuz21}.
%\begin{enumerate}[$(a)$]
%\item Assumption \ref{ass:CE} is slightly weaker than Assumption (CE) given in \cite[Def. 3.41]{LzDSSuz21}: Assumption \ref{ass:CE}  requires the equivalence of the measures for $\QP^{k, E}$-a.e.\ $\eta$ while \cite[Def. 3.41]{LzDSSuz21} requires it for $\QP$-a.e.\ $\eta$. 
%\item However, Assumption \ref{ass:CE} implies Assumption (CAC) in \cite[Def. 3.41]{LzDSSuz21}, i.e., for any $E \in \msE$, $k \in \N_0$ and for $\QP$-a.e.\ $\eta$ ({\it not only $\QP^{k, E}$-a.e.\ $\eta$}), 
%\begin{equation*}
% \QP^{\eta, k}_{E} \ll \PP_{\mssm_{E}}\mrestr{\dUpsilon^{(k)}(E)}  \fstop
%\end{equation*}
%This can be  immediately seen by noting that $ \QP^{\eta, k}_{E}$ is the zero measure on $\dUpsilon^{(k)}(E)$ whenever $\eta$ does not belong to the set 
%$${\rm pr}_{E^c}^{-1}\circ {\rm pr}_{E^c}\bigl(\{\gamma \in \dUpsilon: \gamma E=k\}\bigr).$$
%\end{enumerate}
%\end{rem}
%%\begin{equation}\tag*{$(\mathsf{CE})_{\ref{d:ConditionalAC}}$}
%%\label{ass:CE}
%%\forallae{\QP} \eta\in\dUpsilon \qquad \QP^\eta_{E}\mrestr{\dUpsilon^{(k)}(E)} \sim \PP_{\mssm_{E}}\mrestr{\dUpsilon^{(k)}(E)}  \fstop
%%\end{equation}
%%\end{defs}
%\smallskip
%\begin{comment}
%\begin{lem}\label{l:Isometry}
%The map~$\emparg^\trid\colon  f\mapsto  f^\trid$ descends to an isometric order-preserving embedding
%\begin{align*}
%\emparg^\trid\colon L^1(\mssm)\longrar L^1(\PP)\fstop
%\end{align*}
%\begin{proof}
%Let~$ f\in\mcL^1(\mssm)^+$. By a simple application of~\eqref{eq:Mecke}, we have that~$\ttnorm{ f^\trid}_{\mcL^1(\PP)}= \ttnorm{ f}_{\mcL^1(\mssm)}$, i.e.~$\emparg^\trid\colon  f\mapsto  f^\trid$ is an isometry~$\mcL^1(\mssm)\to\mcL^1(\PP)$.
%By standard arguments with the positive and negative parts of~$ f$, the above isometry extends to the whole of~$\mcL^1(\mssm)$.
%In order to show that it descends to~$L^1(\mssm)$ it is enough to compute~$\ttnorm{ f^\trid-two f^\trid}_{\mcL^1(\PP)}=\ttnorm{ f-two f}_{\mcL^1(\mssm)}$ on different $\mssm$-representatives~$ f$, and~$two f$ of the same $\mssm$-class~$f\in L^1(\mssm)$.
%The order-preserving property is straightforward.
%\end{proof}
%\end{lem}
%%\note{COLLECT HERE ALL PROPERTIES OF POISSON MEASURES}
%
%\begin{lem}[{\cite[Lem.~3.25, Rmk.~4.26]{LzDSSuz21}}]
%Let~$\mcX$ be a topological local structure. Then,~$\PP$ has full $\T_\mrmv(\msE)$-support.
%If~$\mcX$ is, additionally, Polish, then~$\PP$ is Radon.
%\end{lem}
%\end{comment}
%
%
%%Now, let~$(\mcX,\cdc)$ be a \TLDS.
%%For simplicity, we shall consider a strong lifting~$\ell\colon L^\infty(\mssm)\rar \mcL^\infty(\mssm)$ fixed throughout this section, and we shall write~$\cdc$ in place of $\cdc_\ell$.
%%We now turn to discuss the conditional closability.  
%
%%For a probability measure~$\QP$ on~$\ttonde{\dUpsilon,\A_\mrmv(\msE)}$, and for the corresponding system of projected conditional probabilities~\eqref{eq:ProjectedConditionalQP}, we have the following standard result.
%%
%\begin{comment}
%\begin{prop}\label{p:ConditionalIntegration}
%Let~$(\mcX,\cdc)$ be a \TLDS,~$\QP$ be a probability measure on~$\ttonde{\dUpsilon,\A_\mrmv(\msE)}$, and~$ u\in \mcL^1(\QP)$. Then,
%\begin{align*}
%\int_{\dUpsilon} u \diff\QP = \int_{\dUpsilon} \quadre{\int_{\dUpsilon(E)}  u_{E,\eta} \diff \QP^\eta_E }\diff\QP(\eta) \fstop
%\end{align*}
%\begin{proof}
%By definition of conditional probability
%\begin{align*}
%\int_{\dUpsilon}  u\diff\QP = \int_{\dUpsilon} \quadre{\int_{\dUpsilon}  u \diff\QP^{\eta_{E^\complement}} }\diff\QP(\eta) \fstop
%\end{align*}
%%
%By regularity of the conditional system~$\seq{\QP^{\eta_{E^\complement}}}_{\eta\in\dUpsilon}$, the measure~$\QP^{\eta_{E^\complement}}$ is concentrated on the set
%\begin{equation}\label{eq:RoeSch99Set}
%\Lambda_{\eta, E^\complement} \eqdef \set{\gamma \in \dUpsilon : \gamma_{E^\complement}=\eta_{E^\complement} } 
%\end{equation}
%and we have that
%\begin{equation}\label{eq:p:ConditionalIntegration:1}
% u\equiv  u_{E,\eta}\circ \pr_E \quad \text{everywhere on } \Lambda_{\eta, E^\complement}\fstop
%\end{equation}
%As a consequence,
%\begin{align*}
%\int_{\dUpsilon}  u\diff\QP = \int_{\dUpsilon} \quadre{\int_{\dUpsilon}  u_{E,\eta} \circ \pr_E \diff\QP^{\eta_{E^\complement}} }\diff\QP(\eta) \comma
%\end{align*}
%and the conclusion follows by definition of the projected conditional system.
%\end{proof}
%\end{prop}
%\end{comment}
%By the same process as in~\eqref{eq:d:LiftCdCRep} (see also Remark~\ref{rem: lift}), we can define the restricted square field operator~$\cdc^{\dUpsilon,  \QP}_E$ on~$\dUpsilon$ associated to~$E \in \msE$ and it has the following form on $\Cyl{\Dz}$:
%\begin{equation}\label{eq:RestrictedCdCUpsilon}
%\cdc^{\dUpsilon, \QP}_E(u)(\gamma) = \sum_{x\in \gamma_E} \gamma_x^{-1}\cdot \cdc\tonde{ u\ttonde{\car_{X\setminus\set{x}}\cdot\gamma + \gamma_x\delta_\bullet}- u\ttonde{\car_{X\setminus\set{x}}\cdot\gamma}}(x), \quad \text{$\QP$-a.e.\ $\gamma$.}
%\end{equation}
%We further introduce the following objects on $\dUpsilon(E)$: the symbol $\Cyl{\Dz}_{\QP^\eta_E}$ denotes the $\QP^\eta_E$-equivalence classes of $\Cyl{\Dz}$ (note that $\QP^\eta_E$ is supported on $\dUpsilon(E)$), and $\cdc^{\dUpsilon(E), \QP^\eta_E}$ is the square field on $\dUpsilon(E)$, which is described by \eqref{eq:d:LiftCdCRep} (see also Remark~\ref{rem: lift}) with $\dUpsilon(E)$ and $\QP^\eta_E$ in place of $\dUpsilon$ and $\QP$. 
%
%The associated quadratic functionals with respect to $\QP$ and $\QP^\eta_E$ are defined respectively as:
%\begin{subequations}\label{eq:VariousForms}
%\begin{align}\label{eq:VariousFormsA}
%\EE{\dUpsilon}{\QP}_E( u)\eqdef& \int_{\dUpsilon} \cdc^{\dUpsilon, \QP}_E( u)\diff\QP\comma && E\in\msE\comma\quad  u\in\Cyl{\Dz} \comma
%\\
%\label{eq:VariousFormsB}
%\EE{\dUpsilon(E)}{\QP^\eta_E}( u) \eqdef& \int_{\dUpsilon(E)} \cdc^{\dUpsilon(E), \QP^\eta_E}(u)\diff\QP^\eta_E\comma &&\begin{gathered} E\in\msE\comma\quad\eta\in\dUpsilon\comma\\  u\in\Cyl{\Dz}_{\QP^\eta_E}\fstop \end{gathered}
%\end{align}
%
%\end{subequations}
%
%%Before proving the next results, let us comment on the meaning of the forms above.
%%For simplicity, let us assume we have already shown that both the forms in~\eqref{eq:VariousForms} are quasi-regular and strongly local, so that their properties can be recast in terms of the properly associated Markov diffusions.
%%Let~$\mbfM_E$, resp.~$\mbfM^{\eta,E}$, be the diffusion associated to~\eqref{eq:VariousFormsA}, resp.~\eqref{eq:VariousFormsB}.
%%The corresponding sample paths~$\gamma^E_t$ and~$\gamma^{\eta,E}_t$ coincide almost surely when restricted to~$E$.
%%Restricting on~$E^\complement$ we have instead that~$t\mapsto\gamma^E_t\restr_{E^\complement}$ is almost surely a constant configuration on~$E^\complement$ randomly distributed according to~$\pr^{E^\complement}_\pfwd\QP$, whereas~$t\mapsto\gamma^{\eta,E}_t\restr_{E^\complement}=\eta_{E^\complement}$ almost surely.
%
%\begin{comment}
%\begin{prop}\label{p:MarginalWP}
%Let~$(\mcX,\cdc)$ be a \TLDS, and~$\QP$ be a probability measure on $\ttonde{\dUpsilon,\A_\mrmv(\msE)}$ satisfying Assumption~\ref{ass:CAC} for some localizing sequence~$\seq{E_h}_h$.
%Then,
%\begin{align}\label{eq:p:MarginalWP:0}
%\EE{\dUpsilon}{\QP}_{E_h}( u)=\int_\dUpsilon  \EE{\dUpsilon(E_h)}{\QP^\eta_{E_h}}( u_{E_h,\eta}) \diff\QP(\eta) \comma\qquad h\in\N\comma\qquad  u\in\Cyl{\Dz}\fstop
%\end{align}
%Furthermore,~$\EE{\dUpsilon}{\QP}_{E_h}$ is well-defined on~$\CylQP{\QP}{\Dz}$, in the sense that:
%\begin{enumerate}[$(i)$]
%\item\label{i:p:MarginalWP:1} it does not depend on the choice of the $\QP$-representative~$ u$ of~$u\in \CylQP{\QP}{\Dz}$;
%\item\label{i:p:MarginalWP:2} it does not depend on the choice of the strong lifting~$\ell$.
%\end{enumerate}
%
%\begin{proof}
%Let~$h$ and~$\eta$ be fixed and set~$E\eqdef E_h$ for simplicity of notation.
%By Assumption~\ref{ass:CAC} and Remark~\ref{r:ConditionalAC}, we may apply Proposition~\ref{p:NewWP} to~$\QP^\eta_E$ on~$\dUpsilon(E)$ to obtain that the form~$\EE{\dUpsilon(E)}{\QP^\eta_E}$ is well-posed, in the sense that it satisfies~\iref{i:p:MarginalWP:1}.
%Furthermore, by Lemma~\ref{l:MaRoeckner} applied to~$\dUpsilon(E)$,
%\begin{align*}
%&\tonde{\cdc^{\dUpsilon(E)}( u_{E,\eta})\circ \pr_E}(\gamma)=
%\\
%=&\ \sum_{x\in\gamma_E} \ttonde{(\gamma_E)_x}^{-1}\cdot \cdc\tonde{ u_{E,\eta}\ttonde{\car_{E\setminus \set{x}} \cdot\gamma_E+(\gamma_E)_x\delta_\bullet}- u_{E,\eta}\ttonde{\car_{E\setminus\set{x}}\cdot\gamma_E}}
%\\
%=&\ \sum_{x\in\gamma_E} \ttonde{(\gamma_E)_x}^{-1}\cdot \cdc\tonde{ u_{E,\eta}\ttonde{\car_{X\setminus\set{x}}\cdot\gamma_E+(\gamma_E)_x\delta_\bullet}- u_{E,\eta}\ttonde{\car_{X\setminus\set{x}}\cdot\gamma_E}}
%\\
%=&\ \sum_{x\in\gamma_E} \ttonde{(\gamma_E+\eta_{E^\complement})_x}^{-1}\cdot \cdc\tonde{ u\ttonde{\car_{X\setminus\set{x}}\cdot\gamma_E+(\gamma_E)_x\delta_\bullet+\eta_{E^\complement}}- u\ttonde{\car_{X\setminus\set{x}}\cdot\gamma_E+\eta_{E^\complement}}}
%\\
%=&\ \sum_{x\in\gamma_E} \ttonde{(\gamma_E+\eta_{E^\complement})_x}^{-1}\cdot \cdc\tonde{ u\ttonde{\car_{X\setminus\set{x}}\cdot(\gamma_E+\eta_{E^\complement})+(\gamma_E+\eta_{E^\complement})_x\delta_\bullet}- u\ttonde{\car_{X\setminus\set{x}}\cdot(\gamma_E+\eta_{E^\complement})}}
%\\
%=&\ \cdc^\dUpsilon_E( u) (\gamma_E+\eta_{E^\complement})\comma
%\end{align*}
%where the last equality holds by definition~\eqref{eq:RestrictedCdCUpsilon} of~$\cdc^\dUpsilon_E$. This shows that
%\begin{equation}\label{eq:CdCRestrConditionalFormCylinderF}
%\cdc^{\dUpsilon(E)}( u_{E,\eta})\circ \pr_E\equiv \cdc^\dUpsilon_E( u) \qquad \text{on} \quad \Lambda_{\eta, E^\complement} \comma
%\end{equation}
%where~$\Lambda_{\eta,E^\complement}$ is defined as in~\eqref{eq:RoeSch99Set}.
%Since~$\QP^{\eta_{E^\complement}}$ is concentrated on~$\Lambda_{\eta, E^\complement}$, we thus have
%\begin{equation*}
%\int_{\dUpsilon} \cdc^\dUpsilon_E( u) \diff \QP^{\eta_{E^\complement}}
%=\int_{\dUpsilon} \cdc^{\dUpsilon(E)}( u_{E,\eta})\circ \pr_E \diff\QP^{\eta_{E^\complement}}= \int_{\dUpsilon(E)} \cdc^{\dUpsilon(E)}( u_{E,\eta}) \diff\QP^\eta_E
%\end{equation*}
%for every~$ u\in\Cyl{\Dz}$, which concludes the proof of~\eqref{eq:p:MarginalWP:0} by integration w.r.t.~$\QP$ and Proposition~\ref{p:ConditionalIntegration}.
%
%Assertion~\iref{i:p:MarginalWP:1} immediately follows from the well-posedness of the right-hand side in~\eqref{eq:p:MarginalWP:0}.
%%
%Assertion~\iref{i:p:MarginalWP:2} follows similarly, as soon as we show that, for~$\QP$-a.e.~$\eta$, the form~$\EE{\dUpsilon(E)}{\QP^\eta_E}$ is independent of~$\ell$.
%
%To this end, note that, for different strong liftings~$\ell_1, \ell_2$,
%\begin{align*}
%\int_{\dUpsilon(E)} &\abs{ \cdc^{\dUpsilon(E)}_{\ell_1}( u_{E,\eta})- \cdc^{\dUpsilon(E)}_{\ell_2}( u_{E,\eta}) }\diff\QP^\eta_E
%\\
%=&\int_{\dUpsilon} \tonde{\abs{\cdc^{\dUpsilon(E)}_{\ell_1}( u_{E,\eta})-\cdc^{\dUpsilon(E)}_{\ell_2}( u_{E,\eta})} \cdot \frac{\diff \QP^\eta_E}{\diff (\pr_E)_\pfwd \PP_\mssm}} \circ \pr_E \diff\PP_\mssm
%\\
%=&\int_\dUpsilon \int_X \abs{\cdc^{\dUpsilon(E)}_{\ell_1}( u_{E,\eta})-\cdc^{\dUpsilon(E)}_{\ell_2}( u_{E,\eta})}(\gamma_E+\car_E\delta_x) \, \cdot 
%\\
%&\qquad \cdot \frac{\diff \QP^\eta_E}{\diff (\pr_E)_\pfwd \PP_\mssm}(\gamma_E+\car_E\delta_x) \diff\mssm(x)\, \diff\PP_\mssm(\gamma) \comma
%\end{align*}
%where the second equality follows from~\eqref{eq:Mecke}.
%%
%By definition of lifting, and computing~$\cdc^{\dUpsilon(E)}_{\ell_i}( u_{E,\eta})$ by~\eqref{eq:d:LiftCdCRep} for $i=1,2$, we have that
%\begin{align*}
%x\longmapsto \abs{\cdc^{\dUpsilon(E)}_{\ell_1}( u_{E,\eta})-\cdc^{\dUpsilon(E)}_{\ell_2}( u_{E,\eta})}(\gamma_E+\car_E\delta_x) \equiv 0 \as{\mssm}\comma
%\end{align*}
%and the conclusion follows.
%\end{proof}
%\end{prop}
%\end{comment}
%%In light of Proposition~\ref{p:MarginalWP}, the following assumption is well-posed.
%
%\begin{defs}[Conditional closability]\label{ass:ConditionalClosability}
%Let~$\QP$ be a probability on~$\ttonde{\dUpsilon,\A_{\mrmv}(\msE)}$ satisfying Assumptions~\ref{ass:Mmu} and~\ref{ass:CE} for some localising exhaustion~$\seq{E_h}_h$ with $E_h \uparrow X$.
%%
%We say that~$\QP$ satisfies the \emph{conditional closability} assumption~\ref{ass:ConditionalClos} if the forms
%\begin{equation}\tag*{$(\mathsf{CC})_{\ref{ass:ConditionalClosability}}$}\label{ass:ConditionalClos}
%\EE{\dUpsilon(E_h)}{\QP^\eta_{E_h}}(u,v)=\int_{\dUpsilon(E_h)} \cdc^{\dUpsilon(E_h), \QP^\eta_{E_h}}(u,v) \diff\QP^\eta_{E_h}\comma\qquad
%\begin{aligned}
%u,v\in&\ \CylQP{\QP^\eta_{E_h}}{\Dz}\comma
%\\ 
%h\in\N&\comma \quad \eta\in\dUpsilon\comma
%\end{aligned}
%\end{equation}
%are closable on~$L^2\ttonde{\dUpsilon(E_h),\QP^\eta_{E_h}}$ for $\QP$-a.e.~$\eta\in\dUpsilon$, and for every~$h\in\N$. 
%\end{defs}
% We write its closure, called {\it the conditioned form}, by
%\begin{equation} \label{eq: condF}
%\ttonde{\EE{\dUpsilon(E_h)}{\QP^\eta_{E_h}},\dom{\EE{\dUpsilon(E_h)}{\QP^\eta_{E_h}}}} \fstop
%\end{equation}
%The corresponding $L^2$-resolvent operator and the $L^2$-semigroup are denoted respectively by 
%$$\bigl\{G_{\alpha}^{\U(E_h), \QP^\eta_E}\bigr\}_{\alpha>0} \quad \text{and} \quad \bigl\{T^{\U(E_h), \QP^\eta_E}_t\bigr\}_{t >0}\fstop$$ 
%The square field~$\cdc^{\dUpsilon(E_h)}$ naturally extends to the domain $\dom{\EE{\dUpsilon(E_h)}{\QP^\eta_{E_h}}}$, which is denoted by the same symbol~$\cdc^{\dUpsilon(E_h)}$.
%\smallskip
%
%%Under the conditional closability, we obtained in \cite{LzDSSuz21} the closability of \eqref{eq:VariousFormsA} and 
%For every $\A_\mrmv(\msE)$-measurable~$ u\colon \dUpsilon\to \R$ define
%\begin{equation}\label{eq:ConditionalFunction}
% u_{E,\eta}(\gamma)\eqdef  u(\gamma+\eta_{E^\complement})\comma \qquad \gamma\in \dUpsilon(E) \fstop
%\end{equation}
%We now recall the result on the closability of the the following pre-Dirichlet form:
%\begin{align}\label{eq:Temptation}
%\EE{\dUpsilon}{\QP}(u,v)\eqdef \int_{\dUpsilon} \cdc^{\dUpsilon, \QP}(u,v) \diff\QP\comma \qquad u,v\in \Cyl{\Dz} \fstop
%\end{align}
%The following theorem is \cite[Thm.\ 3.48]{LzDSSuz21}, see also (b) in Rem.\ \ref{rem: CP1}. 
%%For~$E\eqdef E_h$ in a suitable localizing sequence, combining Lemma~\ref{l:MmuL1} with the disintegration result in Proposition~\ref{p:ConditionalIntegration} shows that~$\EE{\dUpsilon(E)}{\QP^\eta_E}$ is finite on~$\CylQP{\QP^\eta_E}{\Dz}$ for every~$E$, for $\QP$-a.e.~$\eta\in\dUpsilon$.
%%
%%Since each of the forms~\ref{ass:ConditionalClos} is densely defined by Remark~\ref{r:DensityQP}\iref{i:r:DensityQP:1}, its closure
%%\begin{equation} \label{eq: condF}
%%\ttonde{\EE{\dUpsilon(E)}{\QP^\eta_E},\dom{\EE{\dUpsilon(E)}{\QP^\eta_E}}}
%%\end{equation}
%%is a Dirichlet form.
%
%%\begin{ese}
%%When~$X=\R^n$ is a standard Euclidean space, then all canonical Gibbs measures  and the laws of some determinantal/permanental point processes (e.g., the Ginibre,~$\mathrm{sine}_\beta$, $\mathrm{Airy}_\beta$, $\mathrm{Bessel}_{\alpha,\beta}$) satisfy~\ref{ass:ConditionalClos}.
%%See~\S\ref{sss:ExamplesAC} for references and further examples.
%%\end{ese}
%%\purple{Note that $(\EE_{\U(B_r)}, \CylF(\U(B_r)))$ is closable and the closure is denoted by $(\EE_{\U(B_r)}, H^{1,2}(\U(B_r), \p))$,  see Definition~\ref{defn: WS}. Recall that $\{G_{\alpha}^{\U(B_r)}\}_{\alpha}$ and $\{T^{\U(B_r)}_t\}$ denote the $L^2$-resolvent operator and the semigroup corresponding to $(\EE_{\U(B_r)}, H^{1,2}(\U(B_r), \p))$.
%%Define $\{G_{E_h, \alpha}^{\dUpsilon, \QP}\}_{\alpha>0}$, $\{T^{\dUpsilon, \QP}_{E_h, t}\}_{t>0}$ and $\{G_{\alpha}^{\U(B_r)}\}_{\alpha}$, $\{T^{\U(B_r)}_t\}$.}

