%!TEX root = main.tex
\section{Introduction}
\label{sec:intro}

\begin{figure}[h!]
\centering
\includegraphics[width=0.48\textwidth]{figures/motivation.png}
\caption{\textbf{A motivating example of our proposed framework.} After browsing one white dress, different users want to purchase a dress with some degree of differences in a certain attribute to the white one. }
\label{fig:toy}
\end{figure}

Controllable recommendations are essential for enhancing the customer experience in real-world recommendation scenarios. An example is shown in Figure \ref{fig:toy}: customers are inspired by one white dress and want to purchase a dress with some degree of differences in certain attributes to the white one. In offline shopping, it is easy for the customer to make the salesperson promote a series of products that only differ in certain attributes indicated by the customer in a gradient manner. Then the customer can select the most favorite product from the series of products conveniently. However, it is hard for current recommendation systems to present a sequence of products in a gradient form on a certain attribute based on a reference product. The controllable recommendation as a new type of interaction paradigm can solve the problem. In our work, we define controllable recommendation as a two-stage process. In the first stage, a product will be promoted by the recommendation system along with several modification options for each customer. In the second stage, based on the product and the customer-selected modification, a sequence of products with gradient change on a certain attribute will be retrieved. As this is a new type of interaction with a lot of uncertainty, we need to verify in prototype whether the gradient retrieval is feasible. To make it simple and clean, we keep the discussion of the impact of the customers and the performance of the overall controllable recommendation in the future works. As a first step to approach the controllable recommendation, in this work, we only study the problem of gradient item retrieval with a reference item and a modification as query.
%, i.e., the second stage.\he{Please double check the previous sentence.}

Current methods usually formulate the second stage as a retrieval problem with a text as a query~\cite{Zhen19DSCMR, Vo19TIRG}. Those methods mainly care about whether the target items are retrieved at the top of the retrieved item sequence. Thus, the items in a retrieved item sequence are ranked by the similarities between the input query and items. The demand of retrieving a list of items with gradual change on a certain attribute is largely ignored. The key limitation of these methods is that they only try to model the similarity between the query and target item in their common representation space. In contrast, our method regards a modification text as a "walk" starting from a certain item in the hidden space. By gradually increasing the "step size", a sequence of items can be retrieved in a gradient manner. 

Furthermore, we aim to retrieve a sequence of items with gradual change on a certain attribute with weak supervision.
% \he{It seems that the rest of this paragraph has nothing to do with weak supervision. Maybe add a sentence or two to explain what you mean by weak supervision?} 
Specifically, the goal is to retrieve a sequence of items, where the relevance of a certain attribute is in increasing/decreasing order and other attributes keep the same level. Note the desired attributes (e.g. "floral", "formal") and modification actions ("more" or "less") are indicated by a modification text. To solve the problem, we propose a novel \textbf{C}ontrollable \textbf{G}radient \textbf{I}tem \textbf{R}etrieval framework, called \CGIR~, which learns disentangled item representations with semantic meanings. In the training stage, we only need to know whether a certain product has this attribute or not in order to ground the semantic meanings of each attribute to dimensions of the factorized representation space. This type of weak supervision alleviates the burden of obtaining hand-labeled item sequences with gradual change for an attribute. Thanks to the disentanglement property of learned item representations, we can modify the value on dimensions associated with an indicated attribute to form queries without affecting irrelevant attributes. In the inference stage, by using the queries with different modification strength, a sequence of items can be retrieved in a gradient manner.

Unlike previous unsupervised disentanglement methods which have been demonstrated to rely heavily on model inductive bias and require careful supervision-based hyper-parameter tuning~\cite{Locatello19challengedisentangle}, in this work, we propose a weakly supervised setting to learn disentangled item representations. Specifically, to achieve disentanglement, our method grounds the semantic meanings of attributes to different dimensions of the factorized representation. Following the previous discussion about disentanglement~\cite{shu20disentangleguarantee}, we decompose disentanglement into two distinct concepts: \textit{consistency} and \textit{restrictiveness}. Specifically, \textit{consistency} means only when the hidden factor of one attribute changes, the attribute will change accordingly; and \textit{restrictiveness} means when one hidden factor changes, irrelevant attributes will keep the same~\cite{shu20disentangleguarantee}. By enforcing the disentangled factors to match the oracle hidden factors and encoding them into separate dimensions of representation, our proposed method can satisfy the two properties, which allow us to retrieve items with gradual changes along a certain attribute by tuning the value of relevant dimensions. 
% Besides, in our case, the hidden factors are semantically meaningful attributes. When we disentangled hidden factors to different dimensions, semantic meanings of attributes can be grounded to corresponding dimensions at the same time, which is good for interpretation.\\


To summarize, the main contributions of this paper are:
\begin{itemize}[leftmargin=15pt]
    \item We identify and define the task of gradient item retrieval.
    \item For the first time, we propose a weakly-supervised disentanglement framework that can ground semantic meanings to dimensions of a disentangled representation space.
    \item We demonstrate that our weakly-supervised method can achieve the desired representation disentanglement with semantic meanings, and
    empirically show that our method can achieve gradient retrieval on both public and industrial datasets.
\end{itemize}


The rest of this paper is organized as follows. The proposed \CGIR~ is introduced in Section 2. Qualitative and quantitative experiments are given in Section 3. Section 4 reviews the related work.  Finally, we conclude this work in Section 5.