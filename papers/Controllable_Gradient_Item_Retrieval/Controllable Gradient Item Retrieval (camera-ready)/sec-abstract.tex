%!TEX root = main.tex
\begin{abstract}
In this paper, we identify and study an important problem of gradient item retrieval. We define the problem as retrieving a sequence of items with a gradual change on a certain attribute, given a reference item and a modification text. For example, after a customer saw a white dress, she/he wants to buy a similar one but more floral on it. The extent of "more floral" is subjective, thus prompting one floral dress is hard to satisfy the customer's needs. A better way is to present a sequence of products with increasingly floral attributes based on the white dress, and allow the customer to select the most satisfactory one from the sequence. Existing item retrieval methods mainly focus on whether the target items appear at the top of the retrieved sequence, but ignore the demand for retrieving a sequence of products with gradual change on a certain attribute. To deal with this problem, we propose a weakly-supervised method that can learn a disentangled item representation from user-item interaction data and ground the semantic meaning of attributes to dimensions of the item representation. Our method takes a reference item and a modification as a query. During inference, we start from the reference item and "walk" along the direction of the modification in the item representation space to retrieve a sequence of items in a gradient manner. We demonstrate our proposed method can achieve disentanglement through weak supervision. Besides, we empirically show that an item sequence retrieved by our method is gradually changed on an indicated attribute and, in the item retrieval task, our method outperforms existing approaches on three different datasets.

\end{abstract}
\keywords{information retrieval; recommendation system; weakly-supervised learning; disentangled representation learning; variational autoencoder}