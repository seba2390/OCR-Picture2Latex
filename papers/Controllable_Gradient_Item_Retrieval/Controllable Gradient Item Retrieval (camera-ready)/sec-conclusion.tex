%!TEX root = main.tex
\section{conclusion}
\label{sec:conclusion}
In this paper, we identify and study a new problem -- gradient item retrieval. It is defined as retrieving a sequence of items with gradual change with respect to a certain attribute indicated by a modification text. To solve this problem, we proposed a novel method Controllable Gradient Item Retrieval \CGIR. Our method takes a product and a modification text, which indicates what attributes to change and how to change, as a query and retrieves a sequence of items with gradual change on the relevance between the indicated tag and items in the sequence. To achieve the gradient effect, our method learns a disentangled item representation with weak supervision and grounds semantic meanings to dimensions of the representation. We show that our method can achieve consistency and restrictiveness under a previously proposed theoretical framework. Empirically, we demonstrate that our method can retrieve items in a gradient manner; and in item retrieval tasks, our method outperforms existing approaches on three different datasets.