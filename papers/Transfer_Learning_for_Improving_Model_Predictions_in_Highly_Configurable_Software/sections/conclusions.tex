%!TEX root = ../paper.tex
\section{Conclusions}
\label{sec:conclusions}

Today most software systems are configurable and performance reasoning is typically used to adapt the configuration in order to respond to environmental changes. Machine learning and sampling techniques have been previously proposed to build models in order to predict the performance of unseen configurations. However, the models are either expensive to learn, or they become extremely unreliable if trained on sparse samples. Our cost-aware transfer learning method is orthogonal to the both previously proposed directions promoting to learn from other cheaper sources. Our approach requires only very few samples from the target response function and can learn an accurate and reliable model based on sampling from other relevant sources. We have done extensive experiments with 5 highly configurable systems demonstrating that our approach (i) improves the model accuracy up to several orders of magnitude, (ii) is able to trade-off between different number of samples from source and target, and (iii) imposes an acceptable model building and evaluation cost making appropriate for application in the self-adaptive community.  

{\noindent \em Future directions.}
Beside performance reasoning, our cost-aware transfer learning can contribute to reason about other non-functional properties and quality attributes, \emph{e.g.}, energy consumption. Also, our approach could be extended to support configuration optimization. For example, in our previous work, we have used GP models with Bayesian optimization to focus on interesting zones of the response functions to find optimum configuration quickly for big data systems \cite{jamshidi2016bo4co}. %We found that, if we learn from an unrelated source, it leads to a negative transfer and the predictions become worse comparing with no transfer learning. Hence, it is important to choose a relevant source for model learning.
%In general, our notion of cost-aware transfer learning is independent of the concrete learning technique. In this work, we have used GP models, because our experimental results indicated that GP models outperform other regression models for the systems under consideration \cite{jamshidi2016bo4co}. 
In general, our notion of cost-aware transfer learning is independent of a particular black-box model and complementary to white-box approaches in performance modeling such as queuing theory. A more intelligent way (\emph{e.g.}, active learning) of sampling the source and the target environment to gain more information is also another fruitful future avenue.