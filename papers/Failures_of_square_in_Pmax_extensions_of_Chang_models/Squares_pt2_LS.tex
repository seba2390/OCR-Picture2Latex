\documentclass[12pt]{article}
%\usepackage{fancyhdr, array,calc,graphicx,url,tabularx}
%\usepackage{geometry}
\usepackage{latexsym}
\usepackage{amssymb, amsthm}
\usepackage{amsmath}
\usepackage{makeidx}
\usepackage{undertilde}
\usepackage{enumerate}
\usepackage{verbatim}
%\usepackage{tikz}
%\usetikzlibrary{calc,arrows.meta,positioning}
%\usepackage[colorlinks=true,linkcolor=blue]{hyperref}
%\usepackage{changepage}
%\usepackage{enumitem}

\newtheorem{theorem}{Theorem}[section]
\newtheorem{proposition}[theorem]{Proposition}
\newtheorem{definition}[theorem]{Definition}
\newtheorem{rem}[theorem]{Remark}
\newtheorem{fact}[theorem]{Fact}
\newtheorem{lemma}[theorem]{Lemma}
\newtheorem{corollary}[theorem]{Corollary}
\newtheorem{claim}[theorem]{Claim}
\newtheorem{subclaim}[theorem]{Subclaim}
\newtheorem{conjecture}[theorem]{Conjecture}
\newtheorem{question}[theorem]{Question}
\newtheorem{hypothesis}[theorem]{Hypothesis}
\newtheorem{sublemma}[theorem]{Sublemma}
\newtheorem{exercise}[theorem]{Exercise}
%Nam added remark environment
\newtheorem{remark}[theorem]{Remark}
\newtheorem{notation}[theorem]{Notation}
\newtheorem{terminology}[theorem]{Terminology}
%\numberwithin{figure}{section}

\def\H{{\rm{HOD}}}

\def\c{{\mathbf{C}}}
\def\bbC{{\mathbf{C}}}

\def\mH{{\mathcal{H}}}
\def\cH{{\mathcal{H}}}
\def\K{{\mathcal{ K}}}
\def\J{{\mathcal{ J}}}

\newcommand{\bbP}{\mathbb{P}}

\def\A{{\mathcal{A}}}
\def\R{{\mathcal R}}

\def\X{{\mathbb X}}

\def\M{{\mathcal{M}}}
\def\N{{\mathcal{N}}}
\def\F{{\mathcal{F}}}
\def\T {{\mathcal{T}}}
\def\U{{\mathcal{U}}}
\def\S{{\mathcal{S}}}
\def\V{{\mathcal{V}}}
\def\X{{\mathcal{X}}}
\def\Y{{\mathcal{Y}}}
\def\F{{\mathcal{F}}}
\def\D{{\mathcal{D}}}
\def\B{{\mathcal{B}}}
\def\P{{\mathcal{P} }}

\def\bP{{\mathbb{P} }}

\def\W{{\mathcal{W} }}
\def\Q{{\mathcal{ Q}}}

\def\bQ{{\mathbb{ Q}}}

\def\rng{{\mathrm{rng}}}

\def\VT{{\vec{\mathcal{T}}}}
\def\VU{{\vec{\mathcal{U}}}}
\def\VS{{\vec{\mathcal{S}}}}
\def\VW{{\vec{\mathcal{W}}}}
\def\Qdot{\dot\Q}

\def\bf{{\mathrm{F}}}

\def\Pforces{\forces_{\P}}
\def\of{{\subseteq}}

%
%  Hebrew letters
%

\newcommand{\ha}{\aleph}
\newcommand{\hb}{\beth}
\newcommand{\hg}{\gimel}
\newcommand{\hd}{\daleth}

\newcommand{\dom}{{\rm dom}}
\newcommand{\gen}{{\rm gen}}
\newcommand{\rge}{{\rm rge}}
\newcommand{\cp}{{\rm crit }}
\newcommand{\ind}{{\rm ind}}
\newcommand{\ext}{{\rm ext}}
\renewcommand{\top}{{\rm top}}
\newcommand{\supp}{{\rm supp}}
\newcommand{\support}{{\rm support}}
\newcommand{\rank}{\mathrm{rank}}

\newcommand{\lh}{{\rm lh}}
\newcommand{\lp}{{\rm lp}}
\newcommand{\up}{{\rm up}}

\newcommand{\FF}{{\mathbb F}}
\newcommand{\FP}{{\mathbb P}}
\newcommand{\FQ}{{\mathbb Q}}
\newcommand{\FR}{{\mathbb R}}
\newcommand{\FS}{{\mathbb S}}
\newcommand{\FT}{{\mathbb T}}

\newcommand{\ZFC}{\mathsf{ZFC}}


\def\inseg{\trianglelefteq}
\def\tlt{\triangleleft}

\def\lted{{{\leq}\d}}
\def\ltk{{{<}\k}}

\def\inseg{\triangleleft}
\def\insegeq{\trianglelefteq}


\newcommand{\setof}[2]{{\{\; #1 \; \vert \; #2 \; \} } }
\newcommand{\seq}[1]{{\langle #1 \rangle} }
\newcommand{\card}[1]{{\vert #1 \vert} }
\newcommand{\ot}[1]{\hbox{o.t.($#1$)}}
\newcommand{\forces}{\Vdash}
\newcommand{\decides}{\parallel}
\newcommand{\ndecides}{\nparallel}
\renewcommand{\models}{\vDash}

\newcommand{\bool}{{\bf b} }%

\newcommand{\bA}{{\bf A}}
\newcommand{\bB}{{\bf B}}
\newcommand{\bG}{\boldGamma}
\newcommand{\bD}{\boldDelta}
\newcommand{\bEp}{{\bf E}}
\newcommand{\bZ}{{\bf Z}}
\newcommand{\bEe}{{\bf H}}
\newcommand{\bTh}{\boldTheta}
\newcommand{\bI}{{\bf I}}
\newcommand{\bK}{{\bf K}}
\newcommand{\bL}{{\bf L}}
\newcommand{\bM}{{\bf M}}
\newcommand{\bN}{{\bf N}}
\newcommand{\bX}{\boldXi}
\newcommand{\bOm}{{\bf O}}
\newcommand{\bPi}{\boldPi}

\newcommand{\bR}{{\mathbb{R}}}

\newcommand{\bS}{\boldSigma}
\newcommand{\bT}{{\bf T}}
\newcommand{\bU}{\boldUpsilon}
\newcommand{\bPh}{\boldPhi}
\newcommand{\bCh}{{\bf X}}
\newcommand{\bPs}{\boldPsi}
\newcommand{\bO}{\boldOmega}
\newcommand{\rest}{\restriction}
\newcommand{\la}{\langle}
\newcommand{\ra}{\rangle}
\newcommand{\ov}{\overline}
\newcommand{\add}{{\rm Add}}

\def\L{{\rm{L}}}


\newcommand{\powerset}{\mathcal{P}}


\newcommand{\cf}{{\rm cf}}

\def\k{\kappa}
\def\a{\alpha}
\def\b{\beta}
\def\d{\delta}
\def\s{\sigma}
\def\t{\theta}
\def\l{\lambda}
\def\m{\mu}
\def\n{\nu}


\newcommand{\rfact}[1]{Fact~\ref{#1}}
\newcommand{\rslem}[1]{Sublemma~\ref{#1}}
\newcommand{\rcase}[1]{Case~\ref{#1}}
\newcommand{\rcon}[1]{Conjecture~\ref{#1}}
\newcommand{\rcl}[1]{Claim~\ref{#1}}
\newcommand{\rscl}[1]{Subclaim~\ref{#1}}
\newcommand{\rprop}[1]{Proposition~\ref{#1}}
\newcommand{\rthm}[1]{Theorem~\ref{#1}}
\newcommand{\rlem}[1]{Lemma~\ref{#1}}
\newcommand{\rsublem}[1]{Sublemma~\ref{#1}}
\newcommand{\rprob}[1]{Problem~\ref{#1}}
\newcommand{\rcor}[1]{Corollary~\ref{#1}}
\newcommand{\rdef}[1]{Definition~\ref{#1}}
\newcommand{\rfig}[1]{Figure~\ref{#1}}
\newcommand{\rexe}[1]{Exercise~\ref{#1}}
\newcommand{\rsec}[1]{Section~\ref{#1}}
\newcommand{\rchap}[1]{Chapter~\ref{#1}}
\newcommand{\rsubsec}[1]{Section~\ref{#1}}
\newcommand{\rrem}[1]{Remark~\ref{#1}}
\newcommand{\rnot}[1]{Notation~\ref{#1}}
\newcommand{\rter}[1]{Terminology~\ref{#1}}
\newcommand{\req}[1]{(\ref{#1})}


\newcommand{\AD}{\mathsf{AD}}
\newcommand{\ADR}{\mathsf{AD}_{\mathbb{R}}}
\newcommand{\DCR}{\mathsf{DC}_{\mathbb{R}}}
\newcommand{\rmw}{\mathrm{w}}
\newcommand{\Add}{\mathrm{Add}}
\newcommand{\pmax}{\mathbb{P}_{\mathrm{max}}}
\newcommand{\OD}{\mathrm{OD}}

\newcommand{\bbQ}{\mathbb{Q}}
\newcommand{\bbR}{\mathbb{R}}

\newcommand{\breals}{\omega^{\omega}}
\newcommand{\HOD}{\mathrm{HOD}}
\newcommand{\DC}{\mathsf{DC}}

\newcommand{\cP}{\mathcal{P}}

\newcommand{\MM}{\mathsf{MM}}



\newcommand{\Pmax}{\pmax}
\newcommand{\ZF}{\mathsf{ZF}}

\newcommand{\less}{\mathord{<}}
\newcommand{\cof}{\mathrm{cof}}
\newcommand{\Col}{\mathrm{Col}}
\newcommand{\restrict}{\mathord{\upharpoonright}}

\newcommand{\bls}{\vspace{\baselineskip}}
\newcommand{\PFA}{\mathsf{PFA}}

%\newcommand{\max}{\mathrm{max}}

%\newcounter{nameOfYourChoice}

\title{Failures of square in $\pmax$ extensions of Chang models\thanks{2000 Mathematics Subject Classifications:
03E15, 03E45, 03E60.}
\thanks{Keywords: Square, $\pmax$, Chang models.}}
%\thanks{The author's research was partially supported by the NSF Career Award DMS-1352034.}}


\author{Paul B. Larson and Grigor Sargsyan}

%\author{Grigor Sargsyan\\
%Department of Mathematics\\
%     Rutgers University}
 %    http://math.rutgers.edu/$\sim$gs481\\
%     grigor@math.rutgers.edu\\\\}


\date{\today}


\pagenumbering{arabic}

%\makeindex



\makeindex


\begin{document}

\maketitle

\begin{abstract}
We show that the statements $\square(\omega_{3})$ and $\square(\omega_{4})$ both fail in the $\pmax$ extension of a variation of the Chang model introduced by Sargsyan.
\end{abstract}

%\tableofcontents

%\clearpage


\section{Introduction}

The Chang model \cite{Chang} is the smallest inner model of Zermelo-Fraenkel set theory ($\ZF$) containing every countable sequence of ordinals. Variations of the Chang model can be produced by adding parameters, restricting to the countable sequences from some fixed ordinal or by closing under ordinal definability. In this paper we show that (consistently, assuming the consistency of certain large cardinals) Jensen's square principles $\square(\omega_{3})$ and $\square(\omega_{4})$ fail in extensions of certain Chang models by Woodin's $\pmax$ forcing. The existence of Chang models with the required properties is proved in \cite{Changmodels_1} by the second author, from the existence of a Woodin cardinal which is a limit of Woodin cardinals.

In combination with the results of \cite{Changmodels_1}, the results in this paper have consequences for the inner model theory program.
In particular, we produce forcing extensions of Chang models satisfying the assumptions of the first sentence of the following theorem (see Theorems \ref{weakmain} and \ref{mainthrm}).

\begin{theorem}[Jensen-Schimmerling-Schindler-Steel \cite{JSSS}]\label{JSSSthrm} Assume that $\aleph_2^\omega=\aleph_2$ and that the principles $\square(\omega_3)$ and $\square_{\omega_3}$ both fail to hold. Let $g\subseteq \Col(\omega_3, \omega_3)$ be a $V$-generic filter. If $V[g]\models ``\sf{K}^c_{jsss}$ converges" then $(\sf{K}^c_{jsss})^{V[g]}\models$ ``there is a subcompact cardinal".
\end{theorem}

Since subcompact cardinals have greater consistency strength than Woodin limits of Woodin cardinals, this gives the following theorem, where the transitive model is the model $V[g]$ from Theorem \ref{JSSSthrm}.

\begin{theorem}\label{Kc corollary} It is consistent relative to the existence of a Woodin cardinal that is a limit of Woodin cardinals that there is a transitive model of $\sf{ZFC}$ in which the $\sf{K}^c_{jsss}$ construction does not converge.
\end{theorem}

This paper does not involve any inner model theory. We refer the reader to \cite{Changmodels_1} for a discussion of $\sf{K}^c_{jsss}$ and Theorem \ref{Kc corollary}.

\subsection{Square principles}

The square principles we consider in this paper were introduced by Ronald Jensen \cite{Jensen}. We briefly review their definitions.

\begin{definition}
Given a cardinal $\kappa$, the principle $\square_{\kappa}$ says that there exists a sequence $\langle C_{\alpha} : \alpha < \kappa^{+} \rangle$ such that for each $\alpha < \kappa^{+}$,
\begin{itemize}
\item each $C_{\alpha}$ is a closed cofinal subset of $\alpha$;
\item for each limit point $\beta$ of $C_{\alpha}$, $C_{\beta} = C_{\alpha} \cap \beta$;
\item the ordertype of each $C_\alpha$ is at most $\kappa$.
\end{itemize}
\end{definition}

For any cardinal $\kappa$, $\square_{\kappa}$ implies the statement $\square(\kappa^{+})$ as defined below.



\begin{definition}
Given an ordinal $\gamma$, the principle $\square(\gamma)$ says that there exists a sequence $\langle C_{\alpha} : \alpha < \gamma \rangle$ such that
\begin{itemize}
\item for each $\alpha < \gamma$,
\begin{itemize}
\item each $C_{\alpha}$ is a closed cofinal subset of $\alpha$;
\item for each limit point $\beta$ of $C_{\alpha}$, $C_{\beta} = C_{\alpha} \cap \beta$;
\end{itemize}
\item there is no thread through the sequence, i.e., there is no closed unbounded $E \subseteq \gamma$ such that $C_{\alpha} = E \cap \alpha$ for every limit point $\alpha$ of $E$.
\end{itemize}
\end{definition}

A $\square(\gamma)$-\emph{sequence} is a sequence $\langle C_{\alpha} : \alpha < \gamma \rangle$ as in the definition of $\square(\gamma)$.
A \emph{potential} $\square(\gamma)$-sequence is a sequence $\langle C_{\alpha} : \alpha < \gamma \rangle$ satisfying all but the last condition in the definition.  An elementary argument gives the important fact that if $\gamma$ has uncountable cofinality, then each potential $\square(\gamma)$-sequence has at most one thread.

We will in fact obtain the negation of a weaker version of square, also due to Jensen.


\begin{definition} Given an ordinal $\gamma$ and a cardinal $\delta$, the principle $\square(\gamma, \delta)$ asserts
the existence of a sequence \[\langle \mathcal{C}_{\alpha} \mid \alpha < \gamma \rangle\] satisfying the following conditions.
\begin{itemize}
\item
For each $\alpha < \gamma$,
\begin{itemize}
\item
$0 < |\mathcal{C}_{\alpha}| \leq \delta$;
\item
each element of $\mathcal{C}_{\alpha}$ is club in $\alpha$;
\item
for each member $C$ of $\mathcal{C}_{\alpha}$, and each limit point $\beta$ of $C$, \[C \cap
\beta \in\mathcal{C}_{\beta}.\]
\end{itemize}
\item
There is no thread through the sequence, that is, there is no club $E \subseteq
\gamma$ such that $E \cap \alpha \in \mathcal{C}_{\alpha}$ for every limit point $\alpha$ of $E$.
\end{itemize}
\end{definition}

As above, a $\square(\gamma,\delta)$-sequence is a sequence $\langle C_{\alpha} : \alpha < \gamma \rangle$ as in the definition of $\square(\gamma, \delta)$. A \emph{potential} $\square(\gamma,\delta)$-sequence is a sequence $\langle C_{\alpha} : \alpha < \gamma \rangle$ satisfying all but the last condition in the definition. Again, an elementary argument shows that if the cofinality of $\gamma$ is greater than $|\delta|^{+}$, then each potential $\square(\gamma, \delta)$-sequence has at most $|\delta|$ many threads.
Note that $\square(\gamma)$ is $\square(\gamma, 1)$ and if $\delta < \eta$ then $\square(\kappa, \delta)$ implies $\square(\kappa, \eta)$.

We use Todorcevic's theorem \cite{Stevo_square} that if $\gamma$ has cofinality at least $\omega_{2}$ then the restriction of the Proper Forcing Axiom ($\PFA$) to partial orders of cardinality $\gamma^{\omega}$ implies the failure of $\square(\gamma, \omega_{1})$. For $\gamma < \omega_3$ this fragment of $\PFA$ follows from $\MM^{++}(\mathfrak{c})$ (a technical strengthening of the restriction of Martin's Maximum to partial orders of cardinality at most the continuum), since $\MM^{++}(\mathfrak{c})$ implies that $\mathfrak{c} = \aleph_{2}$ by the results of \cite{FMSI}. We will not need the definition of $\MM^{++}(\mathfrak{c})$ in this paper, as our only use of it will be to apply Todocevic's theorem, and Woodin's theorems on obtaining $\MM^{++}(\mathfrak{c})$ in $\pmax$ extensions (see Subsection \ref{pmaxssec}).

Theorem \ref{weakmain} is one version of the main theorem of this paper. A more explicit version is given in Theorem \ref{mainthrm} below. In light of Todorcevic's theorem it should be possible to replace $\neg\square(\omega_{3}, \omega)$  and $\neg\square(\omega_{4}, \omega)$ below with $\neg\square(\omega_{3}, \omega_1)$ and $\neg\square(\omega_{4}, \omega_1)$, but this remains open.

\begin{theorem}\label{weakmain} The consistency of $\ZFC$ plus the existence of a Woodin limit of Woodin cardinals implies the consistency of \[\ZFC + \aleph_2^\omega=\aleph_2 + \neg\square(\omega_{3}, \omega)  + \neg\square(\omega_{4}, \omega).\]
\end{theorem}


\subsection{Chang models and $\Join_{\lambda}$}\label{cmjoinssec}

We let $\cH$ represent the class of pairs of ordinals $(\alpha, \beta)$ such that the $\alpha$th element of the standard definability order of $\HOD$ is an element of the $\beta$th. This is just a technical convenience that allows us to give a concise statement of results from \cite{Changmodels_1}; the only property of $\cH$ we use in this paper is that it is a definable class of pairs of ordinals. Given an ordinal $\gamma$, we write $\cH\restrict \gamma$ for $\cH \cap (\gamma \times \gamma)$. Given an ordinal $\gamma$,
%and a set $B \subseteq (\gamma^{\omega})^{\less\omega}$,
we write $\c_\gamma^{-}$ for the structure $L_\gamma(\cH, \gamma^{\omega})$, which is constructed relative to the predicate $\cH$, adding (for each ordinal $\alpha < \gamma$) all $\omega$-sequences from $\alpha$ at stage $\alpha + 1$. Note that $\gamma$ is the ordinal height of this structure. We also write $\c_\gamma$ for $L(\cH\restrict \gamma, \gamma^{\omega})$ and $\c^+_\gamma$ for $\HOD_{\gamma^{\omega}}$.

%We omit mention of $B$ when $B = \emptyset$, and, given an ordinal $\gamma$, write $B \restrict \gamma$ for $B \cap (\gamma^{\omega})^{<\omega}$.




We let $\rmw(A)$ denote the Wadge rank of a set $A \subseteq \breals$, and for any ordinal $\alpha$ let $\Delta_{\alpha}$ denote the set of subsets of $\breals$ of Wadge rank less than $\alpha$.
We will work with models of $\AD^{+}$ (an extension of the Axiom of Determinacy due to Hugh Woodin; see for instance \cite{LarsonAD}) in which some ordinal satisfies the following statement (we refer the reader to \cite{Solovay, Woodin, LarsonAD} for the definition of the Solovay sequence).

\begin{definition}
For an ordinal $\lambda$, $\Join_\lambda$ is the statement that, letting $\kappa$ be $\Theta^{\c_\lambda}$,
\begin{itemize}
\item $\kappa$ is a regular member of the Solovay sequence below $\Theta$,
\item $\c^+_{\lambda}\models \lambda=\kappa^{+} +\cf(\lambda)=\lambda$,
\item $\c_\lambda^{-} \cap \powerset(\bR)
%= \c_\a\cap \powerset(\bR)
=\c_{\lambda}^+\cap \powerset(\bR)=\Delta_\kappa$,
\item $\powerset(\kappa^\omega)\cap \c_{\lambda}=\powerset(\kappa^\omega)\cap \c_{\lambda}^+$,
\item $\kappa\leq \cf(\lambda)$.
%\footnote{I need to figure out how to generalize the rest of this to arbitrary $n$.}
\end{itemize}
\end{definition}

%\begin{remark}

Since $\AD^{+}$ implies that successor members of the Solovay sequence below $\Theta$ have cofinality $\omega$ (see \cite{LarsonAD}), $\AD^{+}$ + $\Join_{\lambda}$ implies that $\kappa = \Theta^{\c_\lambda}$ is a limit member of the Solovay sequence. Woodin has shown that $\AD^{+}$ implies each of the following (see \cite{LarsonAD}):
\begin{itemize}
\item $\AD^{+}$ holds in every inner model of $\ZF$ containing $\bbR$;
\item $\ADR$ holds if and only if the Solovay sequence has limit length.
\end{itemize}
It follows that, assuming $\Join_{\kappa^{+}}$, $L(\Delta_{\kappa}) \models \ADR$.

%\end{remark}

We let $\ddagger$ stand for the theory $\sf{ZF}$ + $V=L(\powerset(\bR))$ + $\sf{AD}_{\mathbb{R}}$ + ``$\Theta$ is regular".
%\begin{enumerate}
%\item $\sf{ZF}$ + $V=L(\powerset(\bR))$;
%\item $\sf{AD}_{\mathbb{R}}$ + ``$\Theta$ is regular".
%\item\label{Sigma21reflectionitem} For each limit member $\theta$ of the Solovay sequence $L(\Delta_{\theta})$ is $\Sigma^{2}_{1}$ elementary in $V$.\footnote{This may in fact just follow from $\ZF + \AD + \DCR$. And we may in fact not use it.}
%\item $\sf{AD}^+$.
%\end{enumerate}
%We let $\ddagger$ be the same as $\dagger$ except we change clause 3 to $\sf{AD}_{\mathbb{R}}$ + ``$\Theta$ is regular".
Results of Solovay from \cite{Solovay} say that $\ddagger$ implies $\DC$ (the statement that every tree of height $\omega$ without terminal nodes has a cofinal branch) and also the statement that the sharp of each set of reals exists. By
results of Becker and Woodin (see \cite{LarsonAD}), $\ADR + \DC$ implies that all subsets of $\breals$ are Suslin, and thus that $\AD^{+}$ holds.

%\begin{remark}\label{Sigma21remark}
%  Standard arguments (using part (\ref{Sigma21reflectionitem}) of the definition) give the following consequence of $\ddagger$ : if $\theta$ is a %limit member of the Solovay sequence then $\HOD$ and $\HOD^{L(\Delta_{\theta})}$ have the same bounded subsets of $\theta$.
%\end{remark}



%\begin{remark}

%We will use the following consequence of work of Kunen, Martin and Woodin : assuming $\ADR + \DC$, every set of reals is ordinal definable from a %bounded countable subset of $\Theta$.\footnote{Citation. For a proof see the AD+ book? Actually, I'm not sure we do use this. }

%\end{remark}

Models of $\exists \lambda \Join_{\lambda}$ are given by the following theorem from \cite{Changmodels_1}.

\begin{theorem}\label{inputthrm}
  Suppose that there exists a Woodin cardial which is a limit of Woodin cardinals. Then in a forcing extension there is an inner model satisfying $\ddagger$  + $\exists \lambda \Join_{\lambda}$.
\end{theorem}




%\section{Square principles and $\pmax$}
%The second step is the consistency of $\ddagger$ with $\exists \a \Join_\a$.


\subsection{Variants of $\DC$}

The principle of Dependent Choice ($\DC$) can be varied by restricting the nodes of the tree to some set, or by considering trees of uncountable height.

Given a binary relation $R$ on a set $X$ and an ordinal $\delta$ we say that $f\colon\delta\rightarrow X$ is an $R$-\emph{chain} if $f(\alpha)R f(\beta)$ holds for all ordinals $\alpha < \beta$ below $\delta$. Given an ordinal $\eta$ we say that $R$ is $\eta$-\emph{closed} if for every $\delta<\eta$ and for every $R$-chain $f\colon \delta \rightarrow X$ there is an $r\in X$ such that for every $\alpha < \delta$, $f(\alpha) R r$. We then say that $\sf{DC}_{\gamma}$ holds for a cardinal $\gamma$ if for every  cardinal $\eta\leq \gamma$ and every $\eta$-closed binary relation $R$ there is an $R$-chain $f:\gamma \rightarrow X$. We write $\DC$ for $\DC_{\omega}$.

Given a set $X$ and a cardinal $\gamma$, we write $\DC_{\gamma}\restrict X$ for the restriction of $\DC_{\gamma}$ to binary relations on $X$, which we also call $\DC_{\gamma}$ \emph{for relations on} $X$.


\subsection{$\pmax$}\label{pmaxssec}

The partial order $\pmax$ was introduced by Woodin in \cite{Woodin}. We list here the facts about $\pmax$ (all from \cite{Woodin}) that we will need.
\begin{itemize}
\item $\pmax$ conditions are elements of $H(\aleph_{1})$ and the corresponding order is definable in $H(\aleph_{1})$.

\item $\pmax$ is $\sigma$-closed.

\item  Forcing with $\pmax$ over a model of $\AD^{+} + \DC$ preserves the property of having cofinality at least $\omega_{2}$ (this follows from a combination of Theorems 3.45 and 9.32 of \cite{Woodin}, as outlined in Section \ref{threadsec} below).

\item If $M$ is a model of $\ZF + \AD^{+}$ and $G\subseteq \pmax^{M}$ is an $M$-generic filter, then the following hold in $M[G]$:
\begin{itemize}
\item $2^{\aleph_{0}} = \aleph_{2}$; \item $\Theta^{M} = \omega_{3}$; \item $\cP(\omega_{1}) \subseteq L(\bbR)[G]$.
\end{itemize}

%\item Forcing with $\pmax$ over a model of $\AD^{+}$ wellorders $\bbR$ in ordertype $\omega_{2}$ and forces that $\cP(\omega_{1}) \subs.


%\item $\Theta$ becomes $\omega_{3}$.

\item Forcing with $\pmax$ over a model of $\ADR$ +  $V = L(\cP(\bbR))$ + ``$\Theta$ is regular" produces a model of $\ZF$ + $\DC_{\aleph_{2}}$ +  $\MM^{++}(\mathfrak{c})$.
\end{itemize}

Forcing with $\pmax$ over a model of $\ADR$ cannot wellorder $\cP(\bbR)$ (since a name for such a wellorder would induce a failure of Uniformization), but $\DC_{\aleph_{2}} + 2^{\aleph_{0}} = \aleph_{2}$ implies that $\cP(\bbR)$ may be wellordered  by forcing with $\Add(\omega_{3}, 1)$ (where, for any ordinal $\gamma$, $\Add(\gamma, 1)$ is the partial order adding a generic subset of $\gamma$ by initial segments). Since (by $\DC_{\aleph_{2}}$) $\Add(\omega_{3}, 1)$ does not add subsets of $\omega_{2}$, $\MM^{++}(\mathfrak{c})$ is preserved. This gives the following theorem, which is essentially Theorem 9.39 of \cite{Woodin}.



\begin{theorem}[Woodin]\label{wsthrm} Forcing with $\pmax * \Add(\omega_{3}, 1)$ over a model of $\ADR$ +  $V = L(\cP(\bbR))$ + ``$\Theta$ is regular" produces a model of $\ZFC$ + $\MM^{++}(\mathfrak{c})$.
\end{theorem}

Again, it follows from Todorcevic's theorem that $\square(\omega_{2}, \omega_{1})$ fails in such an extension.

By the results mentioned at the end of Section \ref{cmjoinssec}, the following hold in the context of $\ddagger$ + $\Join_{\lambda}$:
\begin{itemize}
\item $\ADR$ +  $V = L(\cP(\bbR))$ + ``$\Theta$ is regular";
\item the sharp of each subset of $\bbR$ exists;
\item $\bbC^{+}_{\lambda}$ $\models$ $\ADR$ + ``$\Theta$ is regular".
\end{itemize} 
However, $\bbC^{+}_{\lambda}$ is not a model of ``$V =L(\cP(\bbR))$", since, being closed under ordinal definability, it contains the sharp of its version of $\cP(\bbR)$ (i.e., $\Delta_{\kappa}$).
%\footnote{Check : one can see from Lemma \ref{strongreglem} that it contains $\omega$-sequences of Silver indiscernibles for $L(\Delta_{\kappa})$} 
So we cannot just cite Theorem \ref{wsthrm} for our main result. As we shall see, it suffices, however, to wellorder $\lambda^{\omega}$, which can be done by forcing with $\Add(\omega_{4}, 1)$.



%\begin{theorem}[Larson-Sargsyan]
%Suppose that the conclusions of Sargsyan's theorem hold, and that $\lambda$ is an ordinal for which $\Join_{\lambda}$ holds. Let $\kappa = %\Theta^{\bbC_{\lambda}}$ and let $(G, H, K)$ be a $V$-generic filter for
%\[(\pmax*\Add(\kappa, 1)*\Add(\lambda, 1))^{\mathbb{C}^{+}_{\lambda}}.\]

The following then is our main theorem.
%\footnote{For the moment we have two versions. We have to choose. Maybe it doesn't matter.}
The theorem builds upon \cite{SquarePaper} and, of course, \cite{Woodin}. As we shall see in Section \ref{threadsec}, the proof uses an argument from the proof of \cite[Theorem 7.3]{SquarePaper}.

\begin{theorem}\label{mainthrm}
Suppose $V\models \ddagger$ and that $\lambda$ is an ordinal for which $\Join_{\lambda}$ holds. Let $\kappa = \Theta^{\c_{\lambda}}$.
Let $(G, H, K)$ be a $V$-generic filter for the forcing iteration
\[(\pmax*\Add(\k, 1)*\Add(\l, 1))^{\c^{+}_{\l}}.\]
% and let $(G, H, K) \subseteq \mathbb{P}$ be a $V$-generic filter.
%Fix $\k, \a$ such that $\k$ is a regular member of the Solovay sequence and $\Join_{\k, \a}$ holds. Let $M=\c^+_{\a}$, and set $\mathbb{P}=(\mathbb{P}_{\max}*\Add(1, \k)*\Add(1, \a))^M$. Let $G\subseteq \mathbb{P}$ be generic. Then $M[G]\models \aleph_2^\omega=\aleph_2+\neg \square(\omega_3)+\neg\square_{\omega_3}$.
Then
\[\c^+_{\l}[G, H, K]\models \ZFC + \sf{MM}^{++}(c)+\neg\square(\omega_3, \omega)+\neg\square(\omega_{4}, \omega).\]
\end{theorem}


For the rest of the paper we fix $\kappa$, $\lambda$, $G$, $H$ and $K$ as in the statement of Theorem \ref{mainthrm}.
Since $\pmax \subseteq H(\aleph_{1})$, and $\kappa$ is both regular and equal to $\Theta^{\c^{+}_{\lambda}}$, \[(\pmax * \Add(\kappa, 1))^{L(\Delta_{\k})}\] is the same as
$(\pmax * \Add(\kappa, 1))^{\c^{+}_{\lambda}}$.
In addition the partial orders \[(\pmax*\Add(\k, 1)*\Add(\l, 1))^{\c_{\l}}\] and $(\pmax*\Add(\k, 1)*\Add(\l, 1))^{\c^{+}_{\l}}$ are the same, from which it follows that the theorem implies the corresponding version with $\bbC_{\lambda}$ in place of $\bbC^{+}_{\lambda}$.

%\section{Consequences of $\Join_{\lambda}$}

%\begin{theorem}\label{main theorem} Assume that $\ADR$ holds and that $\l$ is an ordinal for which $\Join_{\l}$ holds, and let $\kappa = %\Theta^{\c_\l}$. Let
%\[\mathbb{P}=(\pmax*\Add(\k, 1)*\Add(\l, 1))^{\mathbb{C}_{\l}}\] and let $(G, H, K) \subseteq \mathbb{P}$ be a $V$-generic filter.\footnote{The %other note has the partial order above defined in $\c^{+}_{\l}$. Maybe it doesn't matter?}
%Then
%\[\mathbb{C}^+_{\l}[G, H, K]\models \sf{MM}^{++}(c)+\neg\square(\omega_3)+\neg\square_{\omega_3}.\]
%\end{theorem}

%\section{The proof of $\neg\square{\omega_2}+\neg\square_{\omega_3}$}




\section{Threading coherent sequences}\label{threadsec}

The material in this section is adapted from \cite{SquarePaper}, and reduces (via Theorem \ref{theorem:vanilla}) the proof of Theorem \ref{mainthrm} to showing the following:
\begin{itemize}
  \item $\Add(\kappa, 1)*\Add(\lambda, 1)$ is $\omega_{2}$-closed in $\c^{+}_{\lambda}[G]$;
  \item $V[G] \models ((\Add(\kappa, 1)*\Add(\lambda, 1))^{\c^{+}_{\lambda}[G]})^{\omega_{1}} \subseteq \c^{+}_{\lambda}[G]$;
  \item $\c^{+}_{\lambda}[G, H] \models \DC_{\aleph_{3}}$.
\end{itemize}
%that suitable forms of $\DC$ hold in $\c^{+}_{\lambda}[G]$.
The first of these follows from the fact that $\c^{+}_{\lambda}[G] \models \DC_{\aleph_{2}}$, which is shown in Lemma \ref{cpgdc2lem}.
The second is Lemma \ref{kaplamcllem}.
The third is Lemma \ref{dcomega_3lem}.
The first two facts show that the partial order $(\Add(\kappa, 1)*\Add(\lambda, 1))^{\c^{+}_{\lambda}[G]}$ satisfies, in $V[G]$, the conditions on the partial order $\bbQ$ from the statement of Theorem \ref{theorem:vanilla}.
This gives the failures of $\square(\omega_{3}, \omega)$ and $\square(\omega_{4}, \omega)$ in $\c^{+}_{\lambda}[G,H,K]$. The third
is used only to show that $\c^{+}_{\lambda}[G,H,K]$ is a model of $\ZFC$.



%Notice that $\square_{\kappa, \lambda}$ implies $\square(\kappa^{+},\lambda)$.
%Arguments of Todorcevic \cite{Tod84, Tod02} show that $\MM(\mathfrak{c})$ implies the failure of
%$\square(\gamma, \omega_{1})$ for any ordinal $\gamma < \mathfrak{c}^{+}$ of cofinality at least $\omega_{2}$.

In order to apply Todorcevic's theorem to show that $\square(\omega_{3}, \omega)$ and $\square(\omega_{4}, \omega)$ fail, we need to show that $\kappa$ and $\lambda$ (from Theorem \ref{mainthrm}) have cofinality $\omega_{2}$ in
$V[G]$ (recall that they are less than $\Theta^{V}$, which is $\omega_{3}^{V[G]}$). To do this, we use the following covering theorem of Woodin from Section 3.1 of \cite{Woodin}.
The notion of {\em $A$-iterability} in the following theorem is introduced in
Woodin \cite[Definition 3.30]{Woodin}. Given $X\prec H(\omega_2)$, $M_X$ denotes its transitive
collapse.

\begin{theorem}[Woodin {\cite[Theorem 3.45]{Woodin}}] \label{thm:woodin}
Suppose that $M$ is a proper class inner model that contains all the reals and satisfies $\AD+\DC$.
Suppose that for any $A\in{\mathcal P}(\bbR)\cap M$, the set
$$ \{X\prec H(\omega_2)\mid \mbox{\rm$X$ is countable, and $M_X$ is $A$-iterable}\} $$
is stationary. Let $X$ in $V$ be a bounded subset of $\Theta^M$ of size $\omega_1$. Then there is a set
$Y\in M$, of size $\aleph_1$ in $M$, such that $X\subseteq Y$.
\end{theorem}

We apply Theorem \ref{thm:woodin} in the proof of Lemma \ref{lemma:coflemma} with $M$ as a model of the form $L(A,\bbR)$ for some $A \subseteq \breals$, and the $V$ of Theorem \ref{thm:woodin} as a $\pmax$ extension of $M$.

%this result as follows :  start with $M=L({\mathcal P}(\bbR))$, a model of $\AD^++
%\DC$, and force with $\Pmax$ to produce an extension $M[G_0]$. The technical stationarity assumption\footnote{is there an easier way to see %this?} from Theorem  \ref{thm:woodin} is
%then true in $M[G_0]$ by virtue of Woodin \cite[Theorem 9.32]{Woodin}, and we can then apply Theorem
%\ref{thm:woodin} in this setting.


%The following (which is is proved using Theorem 3.45 of \cite{Woodin}.

\begin{lemma} \label{lemma:coflemma}
Suppose that $M$ is a model of $\ZF + \AD^{+}$ and $\gamma$ is an ordinal of cofinality at least
$\omega_{2}$ in $M$.
Let $G_0\subset \pmax$ be an $M$-generic filter. Then $\gamma$ has cofinality at least $\omega_{2}$ in
$M[G_0]$.
\end{lemma}

\begin{proof}
Suppose first that $\gamma < \Theta^{M}$. Let $X$ be a subset of $\gamma$ of cardinality $\aleph_{1}$ in $M[G_0]$,
and let $A \in \cP(\breals) \cap M$ have Wadge rank at least $\gamma$. Since $|\gamma| \leq 2^{\aleph_{0}}$ in $M[G_{0}]$ and $\mathcal{P}(\omega_{1})^{M[G_0]}$ is contained in $L(A, \mathbb{R})[G]$ by Theorem 9.23 of \cite{Woodin},
$X$ is in $L(A, \mathbb{R})[G]$. By Theorem 9.32 of Woodin, the hypotheses of Theorem \ref{thm:woodin} are satisfied with $L(A, \bbR)$ as $M$ and $L(A, \bbR)[G]$ as $V$.
Applying Theorem \ref{thm:woodin} we have that $X$ is a subset of an element
of $L(A, \mathbb{R})$ of cardinality $\aleph_{1}$ in $L(A, \mathbb{R})$.

The lemma follows immediately from the previous paragraph for $\gamma$ of cofinality less than $\Theta^{M}$ in $M$.
If $\gamma \geq \Theta$ is regular in $M$ there is no cofinal function from $\breals$ to $\gamma$ in $M$, so there is no such
function in $M[G_0]$, either.
%, so $\kappa$ remains regular in $M[G]$.
The theorem then follows for arbitrary $\gamma$.
%If $\kappa = \Theta^{M}$, then the theorem follows from the previous paragraph if $\Theta$ is singular.
\end{proof}



In conjunction with the facts mentioned at the beginning of this section, the following theorem (with $M_{1}$ as $V$, $M_{0}$ as $\bbC^{+}_{\lambda}$, $\gamma$ as either $\kappa$ or $\lambda$ and $\bbQ$ as $(\Add(\kappa,1) * \Add(\lambda,1))^{\bbC^{+}_{\lambda}[G]}$) completes the proof of Theorem \ref{mainthrm}.
The theorem and its proof are taken from \cite{SquarePaper}, except that the specific partial order used in \cite{SquarePaper} has been replaced with a more general class of partial orders.
%\footnote{Cut?: By Theorem 9.10 of Woodin \cite{Woodin}, the hypotheses of the following theorem imply that
%$\AD^{+}$ holds in $M_{0}$.}

\begin{theorem} \label{theorem:vanilla}
Suppose that $M_{1}$ is a model of $\ddagger$, and that for some set $X \in M_{1}$ containing
$\breals \cap M_{1}$, $M_{0} = \HOD^{M_{1}}_{X}$. Suppose also that  $\Theta^{M_{0}} < \Theta^{M_{1}}$ and that $\gamma \in [\Theta^{M_{0}}, \Theta^{M_{1}})$ has cofinality at least $\omega_{2}$ in $M_{1}$.
Let $G_0 \subset \pmax$ be $M_{1}$-generic, and let $I \subset \bbQ$ be
$M_{1}[G_0]$-generic, for some partial order $\bbQ \in M_{0}[G_0]$ which, in $M_{1}[G_0]$, is
$\less\omega_{2}$-directed closed and of cardinality at most $\mathfrak{c}$.
Then $\square(\gamma, \omega)$ fails in $M_{0}[G_0][I]$.
\end{theorem}

\begin{proof}
Suppose that $\tau$ is a $\pmax * \dot{\bbQ}$-name in $M_{0}$ for a
$\square(\gamma,\omega)$-sequence. We may assume that the realization of $\tau$ comes with
an indexing of each member of the sequence in order type at most $\omega$. In $M_{0}$, $\tau$ is
ordinal definable from some $S \in X$.
%Since $\Theta^{M_{0}}$ is on the Solovay sequence of $M_{1}$, $\HOD^{M_{1}}_{\Gamma_{0}}$
%has the same sets of reals as $M_{0}$.

By Theorems 9.35 and 9.39 of \cite{Woodin}, $\DC_{\aleph_{2}}$ and $\MM^{++}(\mathfrak{c})$ hold in $M_{1}[G_0]$.
By Lemma \ref{lemma:coflemma},
$\gamma$ has cofinality $\omega_{2}$ in $M_{1}[G_0]$.
Forcing with $\less\omega_{2}$-directed closed partial orders
of size at most $\mathfrak{c}$ preserves $\MM^{++}(\mathfrak{c})$ (see\cite{Larson:separating}). It follows then
that $\DC_{\aleph_{1}}$ and $\MM^{++} (\mathfrak{c})$ hold in the $\dot{\bbQ}_{G_0}$-extension of $M_{1}[G_0]$,
and thus that in this extension every potential $\square(\gamma, \omega)$-sequence is
threaded.

Let $\mathcal{C} = \langle \mathcal{C}_{\alpha} : \alpha < \gamma \rangle$ be the realization of
$\tau$ in the $\dot{\bbQ}_{G}$-extension of $M_{1}[G_0]$. Since $\gamma$ has cofinality at least $\omega_{2}$ in this
extension, which satisfies $\DC_{\aleph_{1}}$, $\mathcal{C}$ has at most $\omega$ many
threads, since otherwise one could find a $\mathcal{C}_{\alpha}$ in the sequence with uncountably
many members. Therefore, some member of some $\mathcal{C}_{\alpha}$ in the realization of $\tau$
will be extended by a unique thread through the sequence, and since the realization of $\tau$ indexes
each $\mathcal{C}_{\alpha}$ in order type at most $\omega$, there is in $M_{1}$ a name, ordinal
definable from $S$, for a thread through the realization of $\tau$. This name is then a member of
$M_{0} = \HOD^{M_{1}}_{X}$.
\end{proof}

\section{Proving $\DC_{\aleph_{m}}$}

As stated at the beginning of Section \ref{threadsec}, two of our three remaining tasks are showing that $\c^{+}_{\lambda}[G] \models \DC_{\aleph_{2}}$ and $\c^{+}_{\lambda}[G,H] \models \DC_{\aleph_{3}}$. Section \ref{dcreducesec} reduces each of these to the case of relations on $\lambda^{\omega}$. In Section \ref{dcproveoutlinessec} we outline our strategy for proving that $\DC_{\aleph_{2}}$ holds in $\bbC^{+}_{\lambda}[G]$ for relations on $\lambda^{\omega}$. A proof of $\c^{+}_{\lambda}[G,H] \models \DC_{\aleph_{3}}$ (using essentially the same strategy) is given in Section \ref{dc3sec}.

\subsection{Reducing to $\DC_{\aleph_{m}}\restrict \lambda^{\omega}$}\label{dcreducesec}

Lemma \ref{dcreducelem} is applied in this paper in the cases $m=2$ and $m=3$ (recall that, as we have defined it, $\DC_{\aleph_{m}}$ implies $\DC_{\aleph_{k}}$ for all $k \leq m$). Since (by the theorem of Solovay cited in Subsection \ref{cmjoinssec}), $\ddagger$ implies $\DC$, the lemma also shows (in the case $m=0$) that $\DC$ holds in $\bbC^{+}_{\lambda}$.

%\begin{frame}{Reducing to $\DC_{\aleph_{m}}| \lambda^{\omega}$}

\begin{lemma}\label{dcreducelem} Let $\bbP$ be a partial order in $\bbC^{+}_{\lambda}$, and let $I \subseteq \bbP$ be a $\bbC^{+}_{\lambda}$-generic filter. Let $m$ be an element of $\omega$ such that $\DC_{\aleph_{k}}$ holds in $\bbC^{+}_{\lambda}[I]$ for all $k < m$.
Suppose also that, in $\bbC^{+}_{\lambda}[I]$,  every $\less\omega_{m}$-closed tree  on $\lambda^{\omega}$ of height $\omega_{m}$ has a cofinal branch.
Then $\DC_{\aleph_{m}}$ holds in $\bbC^{+}_{\lambda}[I]$.
\end{lemma}

\begin{proof}
Fix a $\less\omega_{m}$-closed tree $T$ in $\bbC^{+}_{\lambda}[I]$. Fix an ordinal $\gamma$ such that every node of $T$ is the realization of a $\bbP$-name which is ordinal definable in $V_{\gamma}$ from some element of $\lambda^{\omega}$.
Given $(n,\delta, x) \in \omega \times \gamma \times \lambda^{\omega}$, let \[t_{n,\delta, x}\] be the set defined in $V_{\gamma}$ from $\delta$ and $x$ by the formula with G\"{o}del number $n$.

Let $T'$ be the tree of sequences $\langle x_{\alpha} : \alpha < \beta\rangle$ (for some $\beta < \omega_{m}$) for which there exists a sequence \[\langle y_{\alpha} : \alpha < \beta \rangle\] such that, for each $\eta < \beta$, \[y_{\eta} = t_{n,\delta, x_{\eta}, I},\] where $(n, \delta) \in (\omega, \gamma)$ is minimal such that  $t_{n,\delta,x_{\eta}}$ is a $\bbP$-name and
\[\langle y_{\alpha} : \alpha  < \eta \rangle^{\frown} \langle t_{n,\delta,x_{\eta},I}\rangle \in T.\]

Then $T'$ is also $\less\omega_{m}$-closed, and an $\omega_{m}$-chain through $T'$ induces one through $T$.
\end{proof}

\subsection{Proving $\DC_{\aleph_{2}}| \lambda^{\omega}$}\label{dcproveoutlinessec}

To show that $\DC_{\aleph_{2}} \restrict \lambda^{\omega}$ holds in $\bbC^{+}_{\lambda}[G]$, we show that the following statements hold in $\bbC^{+}_{\lambda}[G]$:

\begin{itemize}
\item there is no cofinal map from $\omega_{2}$ to $\lambda$;

\item there is no cofinal map from $(\gamma^{\omega})^{\beta}$ to $\lambda$, for any $\gamma < \lambda$ and $\beta < \omega_{2}$ (it suffices to show this for $\gamma = \kappa$ and $\beta = \omega_{1}$).
\end{itemize}

The first of these follows from Lemma \ref{strongreglem} with $b$ as $\omega_{2} \times \pmax$. The second is shown in the proof of Lemma \ref{cpgdc2lem}, whose statement is just the desired statement that $\bbC^{+}_{\lambda}[G] \models \DC_{\aleph_{2}}$. These two facts imply that every cardinal $\delta \leq \aleph_{2}$ and each $\delta$-closed relation $R$ on $\lambda^{\omega}$ in $\bbC^{+}_{\lambda}[G]$, there exists a $\gamma < \lambda$ such that $R \cap \gamma^{\omega}$ is also $\delta$-closed. Since $\lambda = \kappa^{+}$, it suffices then (once we have established the two facts above) to consider trees on $\kappa^{\omega}$.
To show that, in $\bbC^{+}_{\lambda}[G]$, every $\omega_{2}$-closed tree on $\kappa^{\omega}$ has a cofinal branch, we use the fact (which follows from standard $\pmax$ arguments) that the following statements hold in $\bbC^{+}_{\lambda}[G]$:

\begin{itemize}
\item there is no cofinal map from $\omega_{2}$ to $\kappa$ (because $\kappa = \Theta^{\bbC^{+}_{\lambda}}$ is regular in $\bbC^{+}_{\lambda}$ and $\pmax \subseteq H(\aleph_{1})$);

\item there is no cofinal map from $(\gamma^{\omega})^{\beta}$ to $\kappa$, for any $\gamma < \kappa$ and $\beta < \omega_{2}$ (because $\kappa = \omega_{3}^{\bbC^{+}_{\lambda}[G]}$ and $\aleph_{2}^{\aleph_{1}} = \aleph_{2}$).
\end{itemize}
These facts imply that it suffices to consider $\omega_{2}$-closed trees on $\gamma^{\omega}$ for any $\gamma < \kappa$.
Since each such $\gamma^{\omega}$ is a surjective image of the wellordered set $\breals$ in $\bbC^{+}_{\lambda}[G]$,
$\bbC^{+}_{\lambda}[G]$ satisfies the statement that each such tree has a cofinal branch.







%\footnote{Probably to be cut: We write $\DC^{*}_{\gamma} \restrict X$ for the restriction of $\DC_{\gamma}$ to relations $R$ on $X^{\less\gamma}$ %with the property that $f R g$ implies that $f$ is an initial segment of $g$. The reason for making this definition is given in the following %proposition, whose proof is essentially the same as the proof of Proposition \ref{dc}.}


\section{Strong regularity of $\lambda$}

In this section we prove a regularity property of $\lambda$ in $\bbC^{+}_{\lambda}$ and derive several consequences, including the fact that $\DC_{\aleph_{2}}$ holds in $\bbC^{+}_{\lambda}[G]$. We will refer to the property of $\lambda$ established in Lemma \ref{strongreglem} as \emph{strong regularity}.

\begin{lemma}\label{strongreglem} Whenever
%\[A \in \bigcup_{n \in \omega}\cP((\lambda^{\omega})^{n}) \cap \bbC^{+}_{\lambda},\]
$b\in \bbC^-_{\lambda}$ and $f \colon b \to \lambda$ is in $\bbC_{\lambda}^+$,  there exists a $\gamma<\lambda$ such that $f[b]\subseteq \gamma$.
\end{lemma}



\begin{proof} Let $\beta<\lambda$ be such that $b\in \bbC^{-}_{\beta}$.
Since
$\lambda = \kappa^{+}$ in $\bbC_{\lambda}$ there exists a surjection $h\colon \kappa\to \beta$ in $\bbC_{\lambda}$ (recall that $\bbC^{+}_{\lambda}$ and $\bbC_{\lambda}$ have the same subsets of $\kappa$).
Let $B$ be the set of $y \in \kappa^\omega$ such that $b$ has a member definable in $\bbC^{-}_{\beta}$ from
$h \circ y$ and $\cH \restrict \beta$.
Then $B$ induces a surjection $g \colon \kappa^{\omega} \to b$ in $\bbC_{\lambda}$.

Since $\kappa = \Theta^{\bbC^{+}_{\lambda}}$ is regular, the ordertype of $f[g[\alpha^{\omega}]]$ less than $\kappa$ for each $\alpha < \kappa$.
Since $\lambda$ is regular in $\bbC^{+}_{\lambda}$, $f[g[\alpha^{\omega}]]$ is a bounded subset of $\lambda$, for each $\alpha < \kappa$.
Again applying the regularity of $\lambda$ in $\bbC^{+}_{\lambda}$, $f[b]$ is bounded in $\lambda$.

\end{proof}

It follows immediately from Lemma \ref{strongreglem} that there is no cofinal map from $\omega_{2}$ to $\lambda$ in $\bbC^{+}_{\lambda}[G]$.
As noted in Section \ref{dcproveoutlinessec}, this reduces our proof that $\bbC^{+}_{\lambda}[G] \models \DC_{\aleph_{2}}$ to showing that there is no
cofinal map from $\kappa^{\omega_{1}}$ to $\lambda$ in $\bbC^{+}_{\lambda}[G]$. We will in fact show that $\kappa^{\omega_{1}} \cap \bbC^{+}_{\lambda}[G] \in \bbC^{-}_{\lambda}[G]$, which will suffice, by Lemma \ref{strongregghlem} below, which shows that the strong regularity of $\lambda$ persists to $\bbC^{+}_{\lambda}[G, H]$.
Lemmas \ref{cmpscllem} and \ref{dkcontlem} use the strong regularity of $\lambda$ to prove closure properties of $\bbC^{-}_{\lambda}$.
The proof of Lemma \ref{cmpscllem} is similar to the proof of Lemma \ref{strongreglem}.

\begin{lemma}\label{cmpscllem}
For all $b\in \bbC^-_{\lambda}$, $\cP(b) \cap \bbC^{+}_{\lambda} \subseteq \bbC^{-}_{\lambda}$.
\end{lemma}



\begin{proof}
Fix $b \in \bbC^{-}_{\lambda}$ and a $\beta<\lambda$ such that
$b\in \bbC^{-}_{\beta}$. Let
$h \colon \kappa \to \beta $ be a surjection in $\bbC_{\lambda}$.
Fix $a\in \powerset(b)\cap \bbC^+_{\lambda}$  and let $B_{a}$ be the set of $(x,n) \in \kappa^\omega \times \omega$ such that some member of $a$ is definable over $\bbC^{-}_{\beta}$ from $h \circ x$ and $\cH \restrict \beta$ via the formula with G\"{o}del number $n$.
Then $B_{a} \in \bbC^{+}_{\lambda}$. Since $\cP(\kappa^{\omega}) \cap \bbC^{+}_{\lambda} \subseteq \bbC_{\lambda}$, $B_{a}\in \bbC_{\lambda}$, so $a\in \bbC_{\lambda}$. A reflection argument using the strong regularity of $\lambda$ shows that $a \in \bbC^{-}_{\lambda}$.
\end{proof}

The following lemma implies that $(\pmax * \Add(\kappa, 1))^{\bbC^{+}_{\lambda}} \in \bbC^{-}_{\lambda}$ and
$(\pmax * \Add(\kappa, 1) * \Add(\lambda, 1))^{\bbC^{+}_{\lambda}} \subseteq \bbC^{-}_{\lambda}$.


\begin{lemma}\label{dkcontlem} $\Delta_{\kappa} \in \bbC^{-}_{\lambda}$
\end{lemma}

\begin{proof} For each $\alpha < \kappa$ let $f(\alpha)$ be the least $\beta < \lambda$ such that there is a set of reals of Wadge rank $\alpha$ in $\bbC^{-}_{\beta}$. By Lemma \ref{cmpscllem}, $f$ is well-defined.
By the strong regularity of $\lambda$ in $\bbC^{+}_{\lambda}$, the range of $f$ is bounded below $\lambda$.
It follows then that $\Delta_{\kappa} \in \bbC^{-}_{\eta}$ for any $\eta < \lambda$ containing the range of $f$.
\end{proof}




Since $(\pmax * \Add(\kappa, 1))^{\bbC^{+}_{\lambda}} \in \bbC^{-}_{\lambda}$, we get the following lemma, which implies that $\lambda$ is regular in $\bbC^{+}_{\lambda}[G,H]$.

\begin{lemma}\label{strongregghlem} Whenever
%\[B \in \cP((\lambda^{\omega})^{\less\omega}) \cap \bbC^{+}_{\lambda},\]
$b\in \bbC^-_{\lambda}[G,H]$ and
\[f \colon b \to \lambda\] is in $\bbC_{\lambda}^{+}[G,H]$,  there exists a $\gamma<\lambda$ such that $f[b]\subseteq \gamma$.
\end{lemma}

%Finally, we note the models $\bbC^{-}_{\lambda}(B)$ that we consider all happen to be the same as $\bbC^{-}_{\lambda}$.


%\begin{lemma}
%For each $B \in \cP(\lambda^\omega)^{\less\omega} \cap \bbC^{+}_{\lambda}$, $\bbC^{-}_{\lambda}(B) = \bbC^{-}_{\lambda}$.
%\end{lemma}


%\begin{proof}
%Fix  $B \in \cP(\lambda^\omega)^{\less\omega} \cap \bbC^{+}_{\lambda}$.
%Since $\bbC^{+}_{\lambda} \models \lambda = \kappa^{+}$, and
%\[\cP(\kappa^\omega)\cap \bbC_{\lambda}= \cP(\kappa^\omega)\cap \bbC_{\lambda}^+\]
%$B \restrict \beta \in \bbC_{\lambda}$, for all $\beta < \lambda$. Moreover, the strong regularity of $\lambda$ implies that each such $B %\restrict \beta$ is in $\bbC_{\gamma}$, for some $\gamma < \lambda$. Every element $b$ of $\bbC^{-}_{\lambda}(B)$ is in $\bbC^{-}_{\beta}(B %\restrict \beta)$ for some $\beta < \lambda$, which in turn is in $\bbC_{\gamma}$, for some $\gamma < \lambda$.
%\end{proof}





\begin{lemma}\label{cpgdc2lem} $\bbC^{+}_{\lambda} \models \DC_{\aleph_{2}}$
\end{lemma}

\begin{proof} By the remarks in Section \ref{dcproveoutlinessec}, and Lemma \ref{strongreglem}, it suffices to show that there is no cofinal map
from $\kappa^{\omega_{1}}$ to $\lambda$ in $\bbC^{+}_{\lambda}[G]$. Since $\kappa$ is regular in $V$, $\cof(\kappa) > \omega_{1}$ in $\bbC^{+}_{\lambda}[G]$, so every element of $\kappa^{\omega_{1}} \cap \bbC^{+}_{\lambda}[G]$ is the realization of a name coded by a set of reals.
Since \[\Delta_{\kappa} = \cP(\bbR) \cap \bbC^{+}_{\lambda} \in \bbC^{-}_{\lambda}\] by Lemma \ref{dkcontlem}, $(\kappa^{\omega_{1}})^{\bbC^{+}_{\lambda}[G]} \in \bbC^{-}_{\lambda}[G]$. A reference to Lemma \ref{strongregghlem} then completes the proof.
%so  $\bbC^{+}_{\lambda}[G] \models \DC_{\aleph_{2}}$.
\end{proof}





%\begin{frame}

%\begin{lemma}
%Suppose that $\lambda$ is an ordinal for which $\Join_{\lambda}$ holds.
%For every $b\in\bbC_{\lambda}^-(A)$,
%\[\powerset(b)\cap \mathbb{C}_{\lambda}(A)=\bigcup_{\xi<\lambda}(\powerset(b)\cap \bbC^-_{\xi}(A \restrict \xi)).\]%\end{lemma}

%\end{frame}



%\begin{frame}

%\begin{lemma}\label{bounded powersets2}
%Suppose that $\lambda$ is an ordinal for which $\Join_{\lambda}$ holds.
%For every $b\in\bbC_{\lambda}^-(A)$,
%\[\powerset(b)\cap \mathbb{C}_{\lambda}^+=\powerset(b)\cap \bbC_{\lambda}(A) = \bigcup_{\xi<\lambda}(\powerset(b)%\cap \bbC^-_{\xi}(A \restrict \xi))=\powerset(b)\cap \bbC^{-}_{\lambda}(A).\]
%\end{lemma}

%\end{frame}

\section{$\omega_{1}$-closure in $V[G]$}

%For the rest of the talk, $U$ represents a subset of $\breals$ of Wadge rank at least the least member of the Solovay sequence above $\lambda$.

%Lemma \ref{omega1_functions} gives the same fact with $V[G]$ in place of $L(U, \bbR)[G]$, since $V[G]$ thinks each of the following:\bls
%\begin{itemize}
%\item $b^{\omega_{1}} \subset \bigcup_{U \in \cP(\bbR)}L(U, \bbR)[G]$.
%\item $\cof(\Theta) > \lambda$.
%\end{itemize}

%The previous lemma shows that $\kappa^{\omega_{1}} \cap \bbC^{+}_{\lambda}[G] \subseteq
%L(U, \bbR)[G]$. \bls

%The following lemma then completes the proof that $\bbC^{+}_{\lambda}[G] \models
%%\DC_{\aleph_2}$.\bls

In this section we show that, in $V[G]$, $\bbC^{+}_{\lambda}[G]$ is closed under $\lambda$-sequences from
$(\Add(\kappa, 1) * \Add(\lambda, 1))^{\bbC^{+}_{\lambda}[G]}$, which is the second statement from the beginning of Section \ref{threadsec}. This Lemma \ref{kaplamcllem} below, which follows from Lemma \ref{omega1 functions}.


\begin{lemma}\label{omega1 functions} In $V[G]$, for each $b\in \bbC^-_{\lambda}[G]$,
%there is a
%$(\l, A)$-reflection point
%$\xi < \lambda$ such that
$b^{\omega_1}\in \bbC^{-}_{\lambda}[G]$.
\end{lemma}

\begin{proof}
Since $\lambda < \Theta$, $\bbC^{-}_{\lambda}$ is a surjective image of $\breals$ in $V$.
Let $U \subseteq \breals$ be such that $\bbC^{-}_{\lambda}$ is a surjective image of $\breals$ in $L(U, \bbR)$.
Since $\breals$ is wellordered in $L(U, \bbR)[G]$, there exists in $L(U, \bbR)[G]$ a function picking for each
$x \in \bbC^{-}_{\lambda}[G]$ a $\pmax$-name $\tau_{x} \in \bbC^{-}_{\lambda}$ such that $\tau_{x,G} = x$.
Since $\cP(\omega_{1}) \cap V[G] \subseteq L(\bbR)[G]$, \[(\bbC^{-}_{\lambda}[G])^{\omega_{1}} \cap V[G]
\subseteq L(U, \bbR)[G].\]

%$L(U, \bbR)[G]$, since $V[G]$ thinks each of the following:
%\begin{itemize}
%\item $b^{\omega_{1}} \subset \bigcup_{U \in \cP(\bbR)}L(U, \bbR)[G]$.
%\item $\cof(\Theta) > \lambda$.
%\end{itemize}

We work in $L(U, \bbR)[G]$, which satisfies Choice.
Fix $b\in \bbC^-_{\lambda}[G]$, and let $\beta < \lambda$ be
such that $\Delta_{\kappa}, \tau_{b} \in \bbC^{-}_{\beta}$.
% and $b$ is $\dot{b}_{G}$ for some $\dot{b}$ in $\bbC^{-}_{\beta}$.
It follows that every member of $b$ is the realization of a name in $\bbC^{-}_{\beta}$.
We first show that $b^{\omega_1}\subseteq \bbC^{-}_{\lambda}[G]$.

Fix $f \in b^{\omega_{1}}$. Since Choice holds, there is an \[h_f \in (\bbC^{-}_{\beta})^{\omega_{1}}\] such that, for every $\alpha<\omega_1$,
$h_f(\alpha)$ is a $\pmax$ name in $\bbC^{-}_{\beta}$ such that $h_{f}(\alpha)_{G} = f(\alpha)$.
Fix  a function $c_{f} \colon \omega_{1} \to \omega$  and a sequence $\langle B_\alpha:\alpha<\omega_1\rangle$ such that each
$B_\alpha$ is a nonempty subset of $\beta^{\omega}$ and each $h_f(\alpha)$ is definable in
$\bbC^{-}_{\beta}$ from $\cH \restrict \beta$, and each member of the corresponding $B_\alpha$,
via the formula with G\"{o}del number $c_{f}(\alpha)$.
Since $\cP(\omega_{1}) \subseteq L_{\lambda}(\bbR)[G]$, $c_{f} \in \bbC_{\lambda}[G]$.

Let $h \colon \kappa \to \beta$ be a surjection in $\bbC^{-}_\lambda$ (which exists by Lemma \ref{cmpscllem}), and, for each $\alpha < \omega_1$ let
\[B'_\alpha = \{ x \in \kappa^{\omega} : h \circ x \in B_\alpha\}.\]
As there is no cofinal function from $\omega_{1}$ to $\kappa$ in $\bbC^{+}_{\lambda}[G]$, there is a $\gamma < \kappa$ such that $B'_\alpha \cap \gamma^{\omega}$ is nonempty
 for each $\alpha < \omega_{1}$.
Let $r \colon \breals \to \gamma^{\omega}$ be a surjection in $L(\Delta_{\kappa})$, and for each $\alpha < \omega_{1}$ let
\[C_\alpha = r^{-1}[B'_a \cap \gamma^{\omega}].\]
Then each $C_\alpha$ is a set of reals in $L(\Delta_{\kappa})$.

In $L(\Delta_{\kappa})$ there is a set of reals of Wadge rank greater than each $C_{\alpha}$, so, in $L(U, \bbR)[G]$,  there is a subset $T$ of $\omega_{1}$ such that
\[\langle C_\alpha : \alpha < \omega_{1} \rangle \in L(\Delta_\kappa)[T].\]
Since $\powerset(\omega_{1}) \subseteq L_{\lambda}(\bbR)[G]$,
it follows that $\langle B'_{\alpha} \cap \gamma^{\omega} : \alpha \in \omega_{1} \rangle$ is in $L(\Delta_{\kappa})[G]$, and that
\[\langle \{ h \circ x : x \in B'_{\alpha} \cap \gamma^{\omega}\} : \alpha \in \omega_{1} \rangle,\]
$\langle h_{f}(\alpha) : \alpha < \omega_{1} \rangle$ and $f$  are in $\bbC^{-}_{\lambda}[G]$.

Suppose now that $b^{\omega_1}\not \subseteq \bbC_{\alpha}[G]$ for any $\alpha < \lambda$.
We then have a function $g\colon b^{\omega_1}\to \lambda$ that is unbounded in $\lambda$ with
$g\in \bbC_{\lambda}[G]$.
Using the above coding, $g$ induces a cofinal function
\[h \colon \powerset(\omega_{1}) \times \Delta_{\kappa} \to \lambda\] in $\bbC_{\lambda}[G]$, with the first argument playing the role of $T$ above and the second coding both a wellordering of $\breals$ in ordertype $\gamma$ and set of reals of Wadge rank above each $C_{\alpha}$.
%Since $2^{\aleph_{0}} = 2^{\aleph_{1}}$ in $\bbC_{\lambda}[G]$, this induces a cofinal function with domain
%$\breals \times \Delta_{\kappa}$.
This contradicts Lemma \ref{strongregghlem}.
%As $\mathbb{P}_{\max}\subseteq H(\aleph_{1})$, we get a function $h'\colon \breals \times \Delta_{\kappa} \to \lambda$
%that is unbounded in $\lambda$ with $h'\in \bbC_{\lambda}$.
%This contradicts the strong regularity of $\lambda$.
\end{proof}


\begin{lemma}\label{kaplamcllem} $V[G] \models ((\Add(\kappa, 1)*\Add(\lambda, 1))^{\bbC^{+}_{\lambda}[G]})^{\omega_{1}} \subseteq \bbC^{+}_{\lambda}[G]$
\end{lemma}

\begin{proof}
As noted before Lemma \ref{dkcontlem}, each element of $((\Add(\kappa, 1)*\Add(\lambda, 1))^{\bbC^{+}_{\lambda}[G]}$
%is coded using $G$ by a pair consisting of an element of $\Delta_{\kappa}$ and a function from $\Delta_{\kappa}$ to $\cP(\kappa)$, and is therefore
an element of $\bbC^{-}_{\lambda}[G]$.
%To see that
%\[V[G] \models ((\Add(\kappa, 1)*\Add(\lambda, 1))^{\bbC^{+}_{\lambda}[G]})^{\omega_{1}} \subseteq \bbC^{+}_{\lambda}[G]\]
%note that,
Since $\cof(\lambda) = \omega_{2}$ in $V[G]$, every element of
\[((\Add(\kappa, 1)*\Add(\lambda, 1))^{\bbC^{+}_{\lambda}[G]})^{\omega_{1}}\] in $V[G]$ has range contained in some element of $\bbC^{-}_{\lambda}[G]$. The lemma then follows from Lemma \ref{omega1 functions}.
\end{proof}












\section{$\sf{DC}_{\aleph_3}$ in $\bbC^+_{\lambda}[G,H]$}\label{dc3sec}

%\subsection{$\DC_{\aleph_{3}}$}

%Suppose $A\in \powerset((\lambda^\omega)^n)\cap \bbC^+_{\lambda, \kappa}$. We say $\eta$ is a \textit{strong} $(\kappa, A)$-reflection point if $\eta$ is a $(\kappa, A)$-reflection point with $\cf(\eta)=\kappa$.

%We let $A_\eta=A\cap (\eta^\omega)^{n(A)}$.\footnote{We should have been doing this the whole time.}

Lemma \ref{dcomega_3lem} is the third item from the beginning of Section \ref{threadsec}, and completes the proof of Theorem \ref{mainthrm}.
The proof is a reflection argument as in Subsection \ref{dcproveoutlinessec}.

\begin{lemma}\label{omega2 closure}
% If $\eta$ is a strong $(\kappa, A)$-reflection point then
%Setting $W=\bbC^-_{\eta}(A_\eta)$,
There are stationarily many $\eta < \lambda$ such that, in $\bbC^+_{\lambda}[G, H]$,
\[\bbC^{-}_{\eta}[G, H]^{\omega_2}\subseteq \bbC^{-}_{\eta}[G, H].\]
\end{lemma}

\begin{proof}

Since $\lambda$ is regular in $\bbC^{+}[G, H]$ by Lemma \ref{strongregghlem}, it suffices to show that for all $\alpha < \beta < \lambda$, if there exists a surjection $s \colon \kappa \to \alpha$ in $\bbC^{-}_{\beta}$ then \[\bbC^{-}_{\alpha}[G, H]^{\omega_2}\subseteq \bbC^{-}_{\beta}[G, H].\]
Fix such $\alpha < \beta$, and let $s \colon \kappa \to \alpha$ be a surjection in $\bbC^{-}_{\beta}$.
Fix a function $f \colon \omega_{2} \to \bbC^{-}_{\alpha}[G, H]$ in $\bbC^{+}_{\lambda}[G,H]$.


For each $\gamma < \omega_{2}$, let $B_{\gamma}$ be the set of $x \in \alpha^{\omega}$ such that
$f(\gamma)$ is definable in $\bbC^{-}_{\alpha}[G, H]$ from $\cH \restrict \alpha$, $G$, $H$ and $x$ via the formula with G\"{o}del code $x(0)$.
Since $\cof(\kappa) > \omega_{2}$ in $\bbC^{+}_{\lambda}[G, H]$, there is a $\delta < \kappa$ such that for all $\gamma < \omega_{2}$, \[B'_{\gamma} = \{ y \in \delta^{\omega} : s \circ y \in B_{\gamma}\}\] is nonempty.
The sequence $\langle B'_{\gamma} : \gamma < \omega_{2} \rangle$ is coded by a set of reals in $\bbC^{+}_{\lambda}[G, H]$, so it is in $L(\Delta_{\kappa})[G]$. It follows that $f \in \bbC^{-}_{\beta}[G,H]$.
\end{proof}


\begin{lemma}\label{dcomega_3lem} $\bbC^+_{\lambda}[G, H]\models \sf{DC}_{\aleph_3}$.
\end{lemma}



\begin{proof}
By Lemma \ref{dcreducelem}, it suffices to show that, in $\bbC^{+}_{\lambda}[G, H]$, every $\less\omega_{3}$-closed tree of height $\omega_{3}$ on $\lambda^{\omega}$ has a cofinal branch.

In $\bbC^{+}_{\lambda}[G,H]$, $\cof(\lambda) > \omega_{3}$ and $\lambda$ is strongly regular, by Lemma \ref{strongregghlem}.
By Lemma \ref{omega2 closure}, then, it suffices to consider trees on $\kappa^{\omega}$.
Since $\kappa^{\omega}$ is wellordered in $\bbC^{+}_{\lambda}[G, H]$ the lemma follows.
\end{proof}

\section{Further work}

The arguments in this paper naturally adapt to produce models of $\ZFC$ in which $\square(\aleph_{n}, \omega)$ fails for all $n \in \omega$, from models of the appropriate generalizations of $\Join_{\lambda}$. Since these generalizations are not yet known to be consistent, we save these arguments for a later paper. In addition, there is much more that can be said about the types of Chang models that we consider in this paper. Some observations that were not needed for the proof of Theorem \ref{mainthrm} have been collected in \cite{LSmoreChang}.

\bibliographystyle{plain}
\bibliography{KC}
\end{document}




%%%%%%%%%%%%%%%%%%
%%%%%%%%%%%%%%%%%%
%%%%%%%%%%%%%%%%%%


























\subsection{$((\Add(\kappa, 1)*\Add(\lambda, 1))^{\bbC^{+}_{\lambda}[G]}$}

To show that
\[V[G] \models ((\Add(\kappa, 1)*\Add(\lambda, 1))^{\bbC^{+}_{\lambda}[G]})^{\omega_{1}} \subseteq \bbC^{+}_{\lambda}[G]\]
we show the following.

\begin{itemize}
\item In $L(U, \bbR)[G]$ there is a surjection from $\Delta_{\lambda}$ to
\[(\Add(\kappa, 1)*\Add(\lambda, 1))^{\bbC^{+}_{\lambda}[G]}\]
\item Since $\Theta^{L(U, \bbR)} > \lambda$, \[V[G] \models \Delta_{\lambda}^{\omega_{1}} \subseteq L(U, \bbR)[G].\]
\item Since $\cof(\lambda) = \omega_{2}$ in $L(U, \bbR)[G]$, every element of
\[(((\Add(\kappa, 1)*\Add(\lambda, 1))^{\bbC^{+}_{\lambda}[G]})^{\omega_{1}}\] in $L(U, \bbR)[G]$ has range contained in some element of $\bbC^{-}_{\lambda}[G]$.
\end{itemize}


\section{Closure of $\bbC^{-}_{\lambda}$ under longer sequences}

We refer the reader to\footnote{somewhere} for a statement of the Moschovakis Coding Lemma, which is a consequence of $\AD$.
An application of the Coding Lemma gives the following, where $(\bbC^-_{\lambda})^\eta$ is as defined in $V$.

\begin{lemma}\label{closure under long sequences}
%Suppose $(\l, \k)$ is a type 3 pair. Then
For all $\eta<\kappa$, $(\bbC^-_{\lambda})^\eta \subseteq \bbC^{-}_{\lambda}$.
\end{lemma}



\begin{proof}
Fix $\eta<\kappa$ and $f:\eta\rightarrow \bbC^-_{\lambda}$ in $\bbC^{+}_{\lambda}$. By Lemma \ref{strongreglem}, we may fix a $\beta < \lambda$ such that $f[\eta] \subseteq \bbC_{\beta}^-$.
Let $h : \kappa\rightarrow \beta$ be a surjection in $\bbC^-_{\lambda}$.
For each $\alpha<\eta$, let $A_\alpha$ be the set of $x \in \kappa^\omega$ such that $f(\alpha)$ is  definable from $h \circ x$ and $\cH\restrict \beta$ over $\bbC^-_{\beta}$ by the formula with G\"{o}del code $x(0)$.
Since $\kappa$ is regular, there is a $\gamma<\kappa$ such that, for all $\alpha < \eta$,
$A_\alpha\cap \gamma^\omega\not =\emptyset$.
For each $\alpha<\eta$ let $B_\alpha=A_\alpha\cap \gamma^\omega$.
Let now $\leq^*$ be a prewellordering of $\breals$ of length $\gamma$ in $L(\Delta_{\kappa})$.
Fix a recursive bijection $s \colon \breals \to (\breals)^{\omega}$.
For each $i < \omega$ let $s_{i}$ denote the $i$th component function of $s$.


%Given $x\in \bR$, let $\langle x_i: i<\omega\rangle$ be the sequence $x$ codes according to the coding we fixed above.



For each $\alpha< \eta$ let \[D_\alpha=\{y\in \breals: \langle \rank_{\leq^{*}}(s_{i}(y)): i<\omega \rangle\in B_\alpha\}.\]
Let $D$ be the set of $(x,y) \in (\breals)^{2}$ for which $y \in D_{\rank_{\leq^{*}}(x)}$.
By the Coding Lemma applied to $D$, there exists an $E\subseteq D$ (projective in $\leq^*$) such that
%\begin{itemize}
%\item for every $x\in \breals$, $E_{x}\subseteq D_{\rank_{\leq^{*}(x)}}$;
%\item
for each $\alpha < \eta$ there is an $x \in \breals$ with $\rank_{\leq^{*}}(x) = \alpha$ and $E_{x}\not =\emptyset$.
%\end{itemize}
For each $\alpha<\eta$, let $D_\alpha'=\bigcup\{ E_x  : \rank(x)_{\leq^*}=\alpha\}$.

Then $f \in \bbC^{-}_{\lambda}$, since for each $\alpha < \eta$,  $f(\alpha)$ is the set defined over $\bbC^{-}_{\beta}$ via \[h \circ  \langle \rank(s_{i}(x))_{\leq^*}: i<\omega \rangle,\] and $\cH \restrict \beta$, for each $x \in D_\alpha'$.
\end{proof}



The strong regularity of $\lambda$ in $\bbC^{+}_{\lambda}$, then gives the following.

\begin{lemma}\label{bounding powersets}
%Suppose $(\l, \k)$ is a type 3 pair. Then
For all $\eta<\kappa$ and any $b\in \bbC^-_{\lambda}$, $b^\eta\in \bbC^{-}_{\lambda}$.
\end{lemma}


\begin{proof}
Fixing $\eta$ and $b$, we have from Lemma \ref{closure under long sequences} that $b^\eta\subseteq \bbC^{-}_{\lambda}$.  Define the
function
$F\colon b^\eta\rightarrow \lambda$ in $\bbC_{\lambda}$ by setting $F(g)$ to be the least $\beta$ with $g \in \bbC^{-}_{\beta}$.
Supposing toward a contradiction that $b^\eta\not \in \bbC^{-}_{\lambda}$, we have that the range of $f$ is unbounded.

Let $\beta<\lambda$ be such that $b\in \bbC_{\beta}$ and let
$h\colon \kappa\rightarrow \beta$ be a surjection in $\bbC^{-}_{\lambda}$.
The proof of Lemma \ref{closure under long sequences} shows how to code any $g\colon\eta\rightarrow b$ via $h$,
$\cH \restrict \beta$, and some triple \[(\gamma, \leq_{*}, (B_\alpha: \alpha<\eta))\in \bbC^{-}_\lambda,\] where $\gamma <\kappa$, $\leq_{*}$ is a prewellordering of length $\gamma$ and each $B_\alpha\subseteq \gamma^\omega$.
Working in $\bbC^{-}_\lambda$ we can code each such triple \[(\gamma, \leq_{*}, (B_\alpha: \alpha<\eta))\] via a set in $\Delta_{\kappa}$.
% and as every set of reals is definable in $L(\Delta_{\kappa})$ from an element of $\kappa^{\omega}$, we can %code the triple $(\gamma, \leq_{*},(B_\alpha: \alpha<\eta))$ by an element of $\kappa^{\omega}$.\bls
It follows that $f$ induces an unbounded function from $\Delta_{\kappa}$ to $\lambda$, contradicting the strong regularity of $\lambda$.
\end{proof}





\section{$\sf{DC}_{\aleph_2}$ in $\c_{\l}(A)[G]$ and in $\c^+_{\l}[G]$}

\footnote{This last part isn't done}



Recall that $G$ is assumed to be a $\c^{+}_{\lambda}$-generic filter for $\pmax$.

\begin{proposition}\label{omega1 function} In $L(U, \bbR)[G]$, for each $b\in \c^-_{\l}(A)[G]$, there is a
%$(\l, A)$-reflection point
$\xi\in (\k, \l)$ such that  $b^{\omega_1}\in \c_{\xi}(A\restrict \xi)[G]$.
\end{proposition}



\begin{proof} We work in $L(U, \bbR)[G]$, which satisfies Choice.
Fix $b\in \c^-_{\l}(A)[G]$, and let $\b < \l$ be
%a $(\l, A)$-reflection point
such that $b$ is $\dot{b}_{G}$ for some $\dot{b}$ in
$\c^{-}_{\b}(A\restrict \beta)$. It follows that every member of $b$ is the realization of a name in $\c^{-}_{\b}(A \restrict \beta)$.
We first show that $b^{\omega_1}\subseteq \c_{\l}(A)[G]$.
Fix $f \in b^{\omega_{1}}$.
%Because $\sf{DC}_{\aleph_2}$
Since Choice holds,
%for each $f:\omega_1\rightarrow b$ in $L(\bbR, U)[g]$
there is an $h_f \in \c^{-}_{\b}(A \restrict \beta)^{\omega_{1}}$ such that for every $\a<\omega_1$,
$h_f(\a)$ is a $\pmax$ name such that $h_{f}(\a)_{G} = f(\a)$.
%Then $h_f$ can be coded by
Fix  a function $c_{f} \colon \omega_{1} \to \omega$  and a sequence $\langle B_\a:\a<\omega_1\rangle$ such that each
$B_\alpha$ is a nonempty subset of $\beta^{\omega}$ and each $h_f(\a)$ is definable in
$\c^{-}_{\b}(A \restrict \beta)$ from $\cH \restrict \beta$, $A \restrict \beta$ and every member of the corresponding $B_\a$, via the formula with G\"{o}del code $c_{f}(\alpha)$. Since $\cP(\omega_{1}) \subseteq L_{\lambda}(\bbR)[G]$, $c_{f} \in \c_{\l}(A)[G]$.
Let $h \colon \kappa \to \beta$ be a surjection in $\c_\l$, and, for each $\a < \omega_1$ let
$B'_\a = \{ x \in \kappa^{\omega} : h \circ x \in B_\a\}$.
As $\k=\omega_3^{\c^{+}_{\lambda}[G]}$, there is a $\gamma < \kappa$ such that $B'_{\alpha}\cap \gamma^{\omega}$ is nonempty
 for each $\a < \omega_{1}$.
Let $r \colon \breals \to \gamma^{\omega}$ be a surjection in $L(\Delta_{\k})$, and for each $\a < \omega_{1}$ let
$C_\a = r^{-1}[B'_a \cap \gamma^{\omega}]$. Then each $C_\a$ is a set of reals in $L(\Delta_{\k})$.
In $L(\Delta_{\k})$ there is a set of reals of Wadge rank greater than each $C_{\a}$, so, in $L(U, \bbR)[G]$,  there is a subset $T$ of $\omega_{1}$ such that $\langle C_\a : \a < \omega_{1} \rangle \in L(\Delta_\k)[T]$. Since $\powerset(\omega_{1})^{L(U, \bbR)[G]} \subseteq L(\Delta_{\k})[G]$,
it follows that $\langle B'_{\a} \cap \gamma^{\omega} : \a \in \omega_{1} \rangle$ is also in $L(\Delta_{\k})[G]$, and that
$\langle \{ h \circ x : x \in B'_{\alpha} \cap \gamma^{\omega}\} : \a \in \omega_{1} \rangle$, $\langle h_{f}(\a) : \a < \omega_{1} \rangle$ and $f$  are in $\c_{\l}[G]$.


%To see this, we use the fact that $\lambda = \kappa^{+}$ in $\c_{\l}(A)$ and the fact that
%$\c_{\l}(A)$ has the same subsets of $\kappa^{\omega}$ as $L(U, \bbR)$.
% It follows that
%$\c_{\l}(A)$ has the same subsets of $(\mH\rest \beta)^{\omega}$ as $L(U, \bbR)$.
%Each $\pmax$ name in $L(U, \bbR)$ for an element of $b^{\omega_{1}}$ is coded by such a subset.

%Then to reflect to a specific $\xi$, use the fact that, in $\c_{\l,\l}$,  $\kappa$ maps onto $\beta$ and $\kappa$ is regular.
%So each $f \in b^{\omega_{1}}$ will have a representative from some $\gamma < \gamma$, and if we can't reflect to some
%$\xi$ then we get a cofinal map from $\kappa$ to $\lambda$.


%We show that  $b^{\omega_1}\in \c_{\xi, \k}[g]$.



%As $\cof(\beta) > \omega_{1}$,  there is a $\gamma<\k$ such that $B_\a\cap  (\mH|\gamma)^{\omega} \neq \emptyset$ for every $\a < \omega_{1}$.
%Since $L(\bbR, U)[h] \models \sf{DC}_{\aleph_2}$, there is a
%Fix  $\vec{Y} \in \prod_{\alpha < \omega_{1}}B_\a\cap  (\mH|\gamma)^{\omega}$.

%Since $\c_\k(A)[G]^{\omega_1}\subseteq \c_\k(A)[G]$, we have that $\vec{Y}\in \c_\k[G]$. Hence, $h_f$ and consequently $f$ are in $\c_{\l}(A)[G]$.

Suppose now that $b^{\omega_1}\not \subseteq \c_{\b}(A \restrict \beta)[G]$ for any $\beta < \lambda$.
%$(\l, A)$-reflection point $\b$.
We then have a function $g: b^{\omega_1}\rightarrow \l$ that is unbounded in $\l$ with $g\in \c_{\l}(A)[G]$. Using the above coding, $g$ induces a cofinal function $h \colon \powerset(\omega_{1}) \to \lambda$ in $\c_{\l}(A)[G]$.
Since $2^{\aleph_{0}} = 2^{\aleph_{1}}$ in $\c_{\l}(A)[G]$, this induces a cofinal function with domain $\breals$.
% $h: \cup_{\gamma<\k}(\gamma^\omega)^{\omega_1}\rightarrow \l$ that is unbounded in $\l$ with $h\in \c_{\l}(A)[G]$.
%As $\cf(\k)>\omega_1$, $h$ induces an unbounded
%$h':\k\rightarrow \l$. Hence, $\cof(\lambda)^{L(U, \bbR)[g]}=\k$.
As $\mathbb{P}_{\max}\subseteq H(\aleph_{1})$, we have a function $h''\colon \breals \to \l$ that is unbounded in $\l$ with $h''\in \c_{\l}(A)$. This contradicts \rprop{strongly regular}.
\end{proof}

\begin{lemma}\label{subsets of omega2 1} In $\c^+_{\l}[G]$, for every $\gamma<\k$, $\powerset(\gamma)\subseteq \c^-_{\l}[G]$.
\end{lemma}

\begin{proof} Since $\k = \omega_{3}$ in  $\c^+_{\l}[G]$, it is enough to show that the claim is true for $\gamma=\omega_2$. Let $B\in \c^+_{\l}[G]$ be a subset of $\omega_2$ and let $\sigma$ be a $\mathbb{P}_{\max}$-name such that $\sigma_G=B$. Let $\tau=\{(p, \check{\a}): p\in \pmax, \a<\omega_2, p\forces \check{\a}\in \sigma\}$. Then $\tau_G=B$ and $\tau$ is coded by a set of reals in $\bbC^{+}_{\lambda}$. Therefore $\tau\in \c^-_{\l}$. Hence, $B\in \c^-_{\l}[G]$.
\end{proof}

\begin{lemma}\label{subsets of omega2 1} In $\c^+_{\l}[G]$, for every $\gamma<\l$, $\powerset(\gamma)\subseteq \c^{-}_{\l}[G]$.
\end{lemma}

\begin{proof} Since $\c^{-}_{\l}$ contains surjections from $\kappa$ to each element of $\lambda$, it is enough to show that the claim is true for $\gamma=\k$. Let then $B\in \c^+_{\l}[G]$ be a subset of $\k$. Let $\sigma$ be any $\pmax$-name such that $\sigma_G=B$. Let $\tau=\{(p, \check{\a}): p\in \mathbb{P}_{\max}, \a<\k, p\forces \check{\a}\in \sigma\}$. Clearly $\tau_g=B$ and $\tau\subseteq H(\aleph_{1}) \times \k$.  Therefore $\tau\in \c^-_{\l}$ (see \rprop{bounded powersets2}). Hence, $B\in \c^-_{\l}[G]$.
\end{proof}

\begin{lemma}\label{kapparegular_1} In $\c^{+}_{\lambda}[G]$, $\kappa$ is regular.
\end{lemma}

\begin{proof} This follows from the regularity of $\kappa$ in $V$, along with the fact that $\kappa = \Theta^{\c^{+}_{\lambda}}$ and $\pmax \subseteq H(\aleph_{1})$.
\end{proof}


\begin{lemma}\label{aleph2DClem} $\c_{\l}(A)[G]\models \sf{DC}_{\aleph_2}$.
\end{lemma}

\begin{proof}
By Proposition \ref{DCordlem} it suffices to prove the lemma for relations on $\lambda^{\omega}$.  This in turn follows from the following observations.
\begin{itemize}
\item For each $B \subseteq \bbR$, $L(B, \bbR)[G]$ is a model of Choice, so it satisfies $\DC_{\aleph_{2}}$.
\item $\c_{\l}(A)[G]$ satisfies $\DC_{\aleph_{2}}$ for relations on $\k^{\omega}$.
To see this, note first of all that $\kappa$ is regular in $\c_{\lambda}(A)[G]$, by Lemma \ref{kapparegular_1}.
%$\kappa$ is regular in $V$ and is the $\Theta$ of $\c_{\l}(A)$, and is therefore still regular in $\c_{\l}(A)[G]$, since $\pmax \subseteq %H(\aleph_{1})$.
For each $\gamma < \kappa$, $(\gamma^{\omega})^{<\omega_{2}}$ has cardinality
$\aleph_{2}$ in $\c_{\l}(A)[G]$, so there does not exist in $\c_{\l}(A)[G]$ a cofinal map from $(\gamma^{\omega})^{<\omega_{2}}$ to $\kappa$. Then for any $\omega_{1}$-closed $R$ on $\kappa^{\omega}$ in $\c_{\l}(A)[G]$ there is a
$\gamma < \kappa$ such that
the restriction of $R$ to $\gamma^{\omega}$ is $\omega_{1}$-closed. Since $\gamma < \Theta^{\c_{\l}}$, this restriction exists in $L(B, \bbR)[G]$ for some $B \subseteq \bbR$ in $\c_{\l}$.
\item Since $\lambda = \kappa^{+}$ in $\c_{\l}(A)$, $\DC_{\aleph_{2}}$ holds in $\c_{\l}(A)[G]$ for relations on $\beta^{\omega}$ for any $\beta < \lambda$.
\item By Proposition \ref{bounding powersets}, $\beta^{\omega} \in \c^{-}_{\lambda}(A)$. By Proposition \ref{bounded powersets1}, $\c^{-}_{\lambda}$ contains surjections from $\kappa$ to each $\beta < \lambda$.
\end{itemize}
Now, since $\lambda$ is strongly regular in $\c^{+}_{\l}$ and $\pmax \subseteq H(\aleph_{1})$, there is no cofinal map from $( \beta^{\omega})^{<\omega_{2}}$ to $\lambda$ in $\c_{\l}(A)[G]$. It follows from this that
$\DC_{\aleph_{2}}$ holds in $\c_{\l}(A)[G]$ for relations on $\lambda^{\omega}$.
\end{proof}

Essentially the same arguments prove the corresponding fact for $\c^{+}_{\lambda}[G]$.\footnote{Say more, probably.}

Before proving the following corollary, we observe that conditions in $\pmax * \Add(\kappa, 1)$ can be coded by sets of reals, and therefore by elements of $\k^{\omega}$. To see this, note first that if $\sigma$ is a $\pmax$ condition for a bounded subset of $\k$ then there is a $\gamma < \k$ such that $\sigma$ is forced by all conditions in $\pmax$ to be a subset of $\gamma$. Using a set of reals of Wadge rank $\gamma$, then, $\sigma$ can be coded by a name for a set of reals, and therefore by a set of reals.\footnote{We've said this already, but this explanation might be better.}

\begin{corollary}\label{cghkappacor} $\c^{+}_{\l}[G, H] \models \cof(\l) >\omega_2$
\end{corollary}

\begin{proof}
This follows from the fact that $\kappa^{\omega}$ can be mapped onto $\pmax * \Add(\kappa, 1)$ in $\c_\l$, and $\lambda$ is
strongly regular as shown in \rprop{strongly regular}.
\end{proof}

%\begin{proof} Let $\tau$ be a $\pmax*\bQ_0$-name for a function $f:\omega_2\rightarrow \l$. Then $\tau$ can be naturally coded as a subset of $(\mH|\l)^\omega\times \l$. Because $\tau_G\in \c_{\l}(\tau)[G]$ and because $\c_{\l}(\tau)[G]\models \sf{DC}_{\omega_2}$, $\tau_G$ must be bounded. To see this, assume otherwise. Then $\card{\l}^{\c_{\l}(\tau)[G]}=\k$. Notice that $\pmax*\bQ_0$ can be coded as a subset of $\k^\omega$ (in $\c_{\l}$).\footnote{Only $\pmax$ need be considered, right? So you just need that there is no cofinal map from $\kappa \times \bbR$ to $\lambda$ in
%$\c^{+}_{\l}$?} It follows that $\card{\l}^{\c_{\l}(\tau)}=\card{\k^\omega}^{\c_{\l}(\tau)}$. This contradicts %\rprop{strongly regular}. \end{proof}

In fact we can prove a stronger result but we leave it to the reader.\footnote{Doesn't this just follow immediately from the fact that
$\Add(1,\kappa)$ is closed under sequences of length $\omega_{2}$?}

\begin{corollary} $\c^+_{\l}[G, H]\models \sf{DC}_{\aleph_2}$.
\end{corollary}

%\clearpage

\section{$\sf{DC}_{\aleph_3}$ in $\c_{\l}[G, H]$ and in $\c^+_{\l}[G,H]$}

Suppose $A\in \powerset((\l^\omega)^n)\cap \c^+_{\l, \k}$. We say $\eta$ is a \textit{strong} $(\k, A)$-reflection point if $\eta$ is a $(\k, A)$-reflection point with $\cf(\eta)=\k$.

%We let $A_\eta=A\cap (\eta^\omega)^{n(A)}$.\footnote{We should have been doing this the whole time.}


\begin{lemma}\label{omega2 closure} If $\eta$ is a strong $(\k, A)$-reflection point then
%Setting $W=\c^-_{\eta}(A_\eta)$,
in $\c^+_{\l}[G, H]$,
\[\c^{-}_{\eta}(A \restrict \eta)[G, H]^{\omega_2}\subseteq \c^{-}_{\eta}(A\restrict \eta)[G, H].\]
\end{lemma}

\begin{proof} We work in $\c^+_{\l}[G, H]$. Fix $f:\omega_2\rightarrow \c^{-}_{\eta}(A\restrict \eta)[G,H]$. We will show that  $f\in \c^{-}_{\eta}(A\restrict \eta)[G, H]$.

Since $\cf(\eta)>\omega_2$, there is a $(\l, A)$-reflection point $\b<\eta$ such that for each $\a<\omega_2$, $f(\alpha) = \tau_{G, H}$ for some $\pmax * \Add(\kappa, 1)$-name $\tau$ definable over $\c^{-}_{\eta}(A\restrict \eta)$ from $\cH \restrict \beta$, $A \restrict \beta$ and some element of $\beta^{\omega}$.

Let $S\colon\kappa\to \beta$ be a surjection in $\c^{-}_{\eta}(A\restrict \eta)$. For each $\a<\omega_2$, let $B_\a\subseteq \k^\omega$ be the set of $x\in \k^\omega$  such that $f(\a) = \tau_{G, H}$ for some name $\tau$ which is definable over $\c_{\b}$ from $\c^{-}_{\eta}(A\restrict \eta)$ from $\cH \restrict \beta$, $A \restrict \beta$  and $S \circ x$.
Then each $B_\a$ is in $\c^{-}_{\eta}(A\restrict \eta)[G, H]$, and moreover can be defined from $(f(\a), S)$.

Because $\cf(\k)>\omega_2$ by Corollary \ref{cghkappacor}, there is a $\gamma<\k$ such that each set $B_\a=A_\a\cap (\gamma)^\omega$ is nonempty. Then $\langle B_\a:\a<\omega_2\rangle \in L(\Delta_{\k})[G]$, since it is coded by a set of reals in $\c^{+}_{\l}[G]$. Now, $G$ induces a well-ordering of each $B_\a$, so there exists a $\langle x_\a: \a<\omega_2\rangle \in L(\Delta_{\k})[G]$ such that each $x_\a\in B_\a$. It follows that there exists a sequence $\langle \tau_\a: \a<\omega_2\rangle \in \c^{-}_{\eta}(A\restrict \eta)[G, H]$ such that for each $\a<\omega_2$, $f(\a)=(\tau_\a)_{(G, H)}$.
\end{proof}

\begin{lemma}\label{dcomega_3lem} $\c^+_{\l}[G, H]\models \sf{DC}_{\aleph_3}$.
\end{lemma}

\begin{proof} By a Skolem hull argument it is enough to show that $\sf{DC}_{\aleph_3}$ holds for each $\aleph_3$-closed  binary relation $R\in \c^+_{\l}[G, H]$ such that $\card{R}=\card{\l^\omega}$. Fix then such an $R$. We can assume that $R$ is a relation on $\l^\omega$. It follows that $R = \tau_{G, H}$ for some $\pmax * \Add(\kappa, 1)$-name $\tau$, which can in turn be coded by a subset $A$ of $\l^\omega$.
%Let then $A$ be a subset of $\l^\omega$ that codes a name for $R$.
Let $\eta$ be a strong $(\k, A)$-reflection point. Letting $\tau_{\eta}$ be the restriction of $\tau$ to $\c^{-}_{\eta}$, it follows from \rlem{omega2 closure} that $(\tau_\eta)_{G, H}=R\cap ( \eta^\omega)^2$ is $\aleph_3$-closed.\footnote{It's ok to think of the relation as on pairs of elements of $\lambda^{\omega}$ instead of $\less\kappa$-sequences from $\lambda^{\omega}$, but we still need to get the exposition right.}

Let $S: \kappa\rightarrow \eta$ be a surjection in $\c_{\l}$. Set $T= \{ (x,y) \in \k^{\omega} : S(x) R S(y)\}$.
Since $\k^\omega$ is well-ordered in $\c_{\l}[G, H]$, there exists a $T$-chain $\langle x_\a:\a<\k\rangle$ in $\c_{\l}[G, H]$. Then $\langle H(x_\a): \a<\k \rangle$ is an $R$-chain.
\end{proof}


\begin{corollary} In $\c^+[G, H]$, if $\vec{E}=
\{E_\a: \a<\k\}$ is a collection of dense open subsets of $\Add(\lambda, 1)$ then $\bigcap_{\a<\k}E_\a$ is a dense open subset of $\Add(\lambda, 1)$.
\end{corollary}






