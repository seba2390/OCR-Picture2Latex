\documentclass[camera,letterpaper,nomarginnotes,nonarrowgutter]{jpaper} 

\usepackage{cite}
\usepackage{amsmath,amssymb,amsfonts}
\usepackage{algorithmic}
\usepackage{graphicx}
\usepackage{textcomp}
%\usepackage[hyphens]{url}
\usepackage[bookmarks=true,breaklinks=true,letterpaper=true,colorlinks,linkcolor=black,citecolor=black,urlcolor=black]{hyperref}
\usepackage{balance} % for balance
\usepackage{fancyhdr}

\def\BibTeX{{\rm B\kern-.05em{\sc i\kern-.025em b}\kern-.08em
    T\kern-.1667em\lower.7ex\hbox{E}\kern-.125emX}}
    
\usepackage[linesnumbered,ruled]{algorithm2e}
\usepackage{algorithmic}
\usepackage{graphicx}
\usepackage{textcomp}
\usepackage{xcolor}
\usepackage{fancyhdr}
\usepackage{xspace}
\usepackage{datetime} 
\usepackage[normalem]{ulem}
\usepackage{booktabs}
\usepackage{multirow}
\usepackage{setspace}
\usepackage{siunitx}
\usepackage{marginnote}
\usepackage{pifont}
\usepackage{enumitem}
\newcommand{\boldone}{\ding{202}}
\newcommand{\boldtwo}{\ding{203}}
\newcommand{\boldthree}{\ding{204}}
\newcommand{\boldfour}{\ding{205}}
\newcommand{\boldfive}{\ding{206}}
\newcommand{\boldsix}{\ding{207}}
\newcommand{\boldseven}{\ding{208}}
\newcommand{\boldeight}{\ding{209}}

\usepackage[colorinlistoftodos,prependcaption,textsize=small]{todonotes}
\usepackage{xargs}                    
\usepackage{titlesec}
\usepackage{titletoc}
\usepackage{clipboard}
\usepackage[all]{nowidow}
\usepackage[framemethod=tikz]{mdframed}
\usepackage[most]{tcolorbox}
\tcbuselibrary{breakable}
\tcbuselibrary{hooks}
\usepackage[font={small,bf}]{caption}
\usepackage{subcaption}
\usepackage{titlesec}
\usepackage{dblfloatfix}

\iffalse
\makeatletter
\g@addto@macro{\normalsize}{%
  \setlength{\abovedisplayskip}{3pt plus 0.5pt minus 1pt}
  \setlength{\belowdisplayskip}{3pt plus 0.5pt minus 1pt}
  \setlength{\abovedisplayshortskip}{0pt}
  \setlength{\belowdisplayshortskip}{0pt}
  \setlength{\intextsep}{4pt plus 1pt minus 1pt}
  \setlength{\textfloatsep}{4pt plus 1pt minus 1pt}
  \setlength{\skip\footins}{5pt plus 1pt minus 1pt}}
  \setlength{\abovecaptionskip}{3pt plus 1pt minus 1pt}
\makeatother

% requires \usepackage{titlesec}
\titlespacing\section{0pt}{2pt plus 1pt minus 1pt}{3pt plus 1pt minus 2pt}
\titlespacing\subsection{0pt}{2pt plus 1pt minus 1pt}{3pt plus 1pt minus 2pt}
\titlespacing\subsubsection{0pt}{2pt plus 1pt minus 1pt}{3pt plus 1pt minus 2pt}
\fi

\newcommand{\onur}[1]{{#1}}

\newcommand\X[0]{PiDRAM\xspace}

\newcommand{\cmark}{\ding{51}}%
\newcommand{\xmark}{\ding{55}}%

\newcommand\outline[1]{\textcolor{orange}{\textbf{Outline:} #1}}

\newif\ifrebuttal
\rebuttalfalse
\newif\iftacorev
\tacorevfalse

\ifrebuttal
\newcommand{\reva}[1]{\textcolor{red}{#1}}
\definecolor{dark-green}{rgb}{0.00, 0.45, 0.00}
\newcommand{\revb}[1]{\textcolor{dark-green}{#1}}
\definecolor{goldbutdark}{rgb}{0.85, 0.65, 0.12}
\newcommand{\revd}[1]{\textcolor{goldbutdark}{#1}}
\newcommand{\revf}[1]{\textcolor{blue}{#1}}
\newcommand{\revcommon}[1]{\textcolor{purple}{#1}}
\definecolor{pink-hot}{rgb}{0.98, 0.40, 0.78}
\newcommand{\reve}[1]{\textcolor{pink-hot}{#1}}
\else
\newcommand{\reva}[1]{#1}
\newcommand{\revb}[1]{#1}
\newcommand{\revd}[1]{#1}
\newcommand{\revf}[1]{#1}
\newcommand{\revcommon}[1]{#1}
\newcommand{\reve}[1]{#1}
\fi

\iftacorev
\newcommand{\tacoreva}[1]{\textcolor{red}{#1}}
\definecolor{dark-green}{rgb}{0.00, 0.45, 0.00}
\newcommand{\tacorevb}[1]{\textcolor{dark-green}{#1}}
\definecolor{goldbutdark}{rgb}{0.85, 0.65, 0.12}
\newcommand{\tacorevc}[1]{\textcolor{goldbutdark}{#1}}
\newcommand{\tacorevd}[1]{\textcolor{purple}{#1}}
\definecolor{pink-hot}{rgb}{0.98, 0.40, 0.78}
\newcommand{\tacoreve}[1]{\textcolor{pink-hot}{#1}}
\newcommand{\tacorevcommon}[1]{\textcolor{blue}{#1}}
\fi

\newif\ifsubmission
%\submissionfalse
\submissiontrue

\ifsubmission

\newcommand{\new}[1]{#1}
\newcommand{\newnew}[1]{#1}
\newcommand{\omi}[1]{#1}
\newcommand{\omu}[1]{#1}
%\newcommandx{\change}[2][1=]{\todo[linecolor=blue,backgroundcolor=blue!25,bordercolor=blue,#1,size=\scriptsize]{#2}}
\newcommand{\revdel}[1]{}

%ATB
\definecolor{ups-truck}{rgb}{0.53, 0.28, 0.21}
\newcommand{\atb}[1]{\textcolor{black}{#1}}
\newcommand{\atbc}[1]{\textcolor{black}{\textbf{[@atb: #1]}}}

%JGL
\definecolor{dgreen}{rgb}{0.00, 0.75, 0.00}
\newcommand{\jgl}[1]{[\textit{{\color{black}JGL: #1}}]}
\newcommand{\juan}[1]{{\color{black}#1}}

%Behzad
\definecolor{dblue}{rgb}{0.00, 0.00, 1.00}
\newcommand{\behzadC}[1]{[\textit{{\color{black}Behzad: #1}}]}
\newcommand{\behzadT}[1]{{\color{black}#1}}

%\newcommand{\om}[1]{\textcolor{blue}{#1}}

\else
%ATB
\definecolor{ups-truck}{rgb}{0.53, 0.28, 0.21}
\newcommand{\atb}[1]{\textcolor{ups-truck}{#1}}
\newcommand{\atbc}[1]{\textcolor{ups-truck}{\textbf{[@atb: #1]}}}

%JGL
\definecolor{dgreen}{rgb}{0.00, 0.75, 0.00}
\newcommand{\jgl}[1]{[\textit{{\color{dgreen}JGL: #1}}]}
\newcommand{\juan}[1]{{\color{dgreen}#1}}

%Behzad
\definecolor{dblue}{rgb}{0.00, 0.00, 1.00}
\newcommand{\behzadC}[1]{[\textit{{\color{dblue}Behzad: #1}}]}
\newcommand{\behzadT}[1]{{\color{dblue}#1}}

\fi

\newcommand{\affilETH}[0]{\textsuperscript{\S}}
\newcommand{\affilETU}[0]{\textsuperscript{$\dagger$}}
\newcommand{\affilBSC}[0]{\textsuperscript{*}}

% Ensure letter paper
\pdfpagewidth=8.5in
\pdfpageheight=11in

\pagenumbering{arabic}

\definecolor{lightblue}{rgb}{0.980, 0.956, 0.623}
\newcommand{\MA}{}% Error, if \MA is already defined.
\DeclareRobustCommand{\MA}[1]{{\sethlcolor{lightblue}\hl{#1}}}

\let\oldmarginnote\marginnote
% renew the \marginpar command to draw 
% a node; it has a default setting which 
% can be overwritten
\renewcommand*{\marginfont}{\tiny}
\renewcommand{\marginnote}[2][rectangle,draw,fill=blue!40,rounded corners]{%
        \oldmarginnote{%
        \tikz \node at (0,0) [#1]{#2};}%
        }
        
\definecolor{lightyellow}{rgb}{0.980, 0.956, 0.623}



\newcommand{\boxbegin} {
	\begin{tcolorbox}[enhanced, frame hidden, colback=gray!50, breakable]
}

\newcommand{\boxend} {
	\end{tcolorbox}
}

\newcommand{\yboxbegin} {
	\begin{tcolorbox}[breakable, enhanced, frame hidden, colback=yellow!50]
}

\newcommand{\yboxend} {
	\end{tcolorbox}
}


\mdfdefinestyle{graybox}{
    splittopskip=0,%
    splitbottomskip=0,%
    frametitleaboveskip=0,
    frametitlebelowskip=0,
    skipabove=0,%
    skipbelow=0,%
    leftmargin=0,%
    rightmargin=0,%
    innertopmargin=0mm,%
    innerbottommargin=0mm,%
    roundcorner=1mm,%
    backgroundcolor=lightblue,
    hidealllines=true}
    
\mdfdefinestyle{graybox2}{
    splittopskip=0,%
    splitbottomskip=0,%
    frametitleaboveskip=0,
    frametitlebelowskip=0,
    skipabove=0,%
    skipbelow=0,%
    leftmargin=0,%
    rightmargin=0,%
    innertopmargin=2mm,%
    innerbottommargin=2mm,%
    roundcorner=2mm,%
    backgroundcolor=lightblue,
    hidealllines=true}

\newcommand{\bboxbegin}{
    \begin{mdframed}[style=graybox]
}

\newcommand{\bboxend}{
    \end{mdframed}
}

\newcommand{\yyboxbegin}{
    \begin{mdframed}[style=graybox2]
}

\newcommand{\yyboxend}{
    \end{mdframed}
}

%%%%%%%%%%%%%%%%%%%%%%%%%%%%%%%%%%%%
%\setstretch{1.11}
\title{PiDRAM: A Holistic End-to-end FPGA-based Framework\\for \underline{P}rocessing-\underline{i}n-\underline{DRAM}}
\author{
{Ataberk Olgun\affilETH}\qquad%
{Juan G\'omez Luna\affilETH}\qquad
{Konstantinos Kanellopoulos\affilETH}\qquad
{Behzad Salami\affilETH}\qquad\\
{Hasan Hassan\affilETH}\qquad%
{O\=guz Ergin\affilETU}\qquad%
{Onur Mutlu\affilETH}\qquad\vspace{-3mm}\\\\
% {\affilCMU Carnegie Mellon University \qquad \affilETH ETH Z{\"u}rich}%
{\vspace{-3mm}\affilETH \emph{ETH Z{\"u}rich}} \qquad \affilETU \emph{TOBB University of Economics and Technology}% \qquad \affilGSC \emph{Galician Supercomputing Center} %
}
%%
%% The "author" command and its associated commands are used to define
%% the authors and their affiliations.
%% Of note is the shared affiliation of the first two authors, and the
%% "authornote" and "authornotemark" commands
%% used to denote shared contribution to the research.

%% By default, the full list of authors will be used in the page
%% headers. Often, this list is too long, and will overlap
%% other information printed in the page headers. This command allows
%% the author to define a more concise list
%% of authors' names for this purpose.

% \setstretch{0.99}

\begin{document}

\maketitle

\thispagestyle{plain}
\pagestyle{plain}

\begin{abstract}
\begin{abstract}
The quality of a summarization evaluation metric is quantified by calculating the correlation between its scores and human annotations across a large number of summaries. Currently, it is unclear how precise these correlation estimates are, nor whether differences between two metrics' correlations reflect a true difference or if it is due to mere chance. In this work, we address these two problems by proposing methods for calculating confidence intervals and running hypothesis tests for correlations using two resampling methods, bootstrapping and permutation. After evaluating which of the proposed methods is most appropriate for summarization through two simulation experiments, we analyze the results of applying these methods to several different automatic evaluation metrics across three sets of human annotations. We find that the confidence intervals are rather wide, demonstrating high uncertainty in the reliability of automatic metrics. Further, although many metrics fail to show statistical improvements over ROUGE, two recent works, QA\-Eval and BERTScore, do in some evaluation settings.\footnote{
Our code is available at \url{https://github.com/CogComp/stat-analysis-experiments}.
}
\end{abstract}
\end{abstract}

%%%%%% -- PAPER CONTENT STARTS-- %%%%%%%%

\section{Introduction}
3D human pose estimation has ubiquitous applications in sport analysis, human-computer interaction, and fitness and dance teaching. While there has been remarkable progress in 3D pose estimation from a monocular image or video~\cite{hmrKanazawa17, Moon_2020_ECCV_I2L-MeshNet, kolotouros2019spin, kocabas2019vibe, xiang2019monocular}, inevitable challenges such as the depth ambiguity and the self-occlusion are still unsolved. 



\begin{figure}
     \centering
     \begin{subfigure}[h]{0.23\textwidth}
         \centering
         \includegraphics[width=\textwidth]{figures/cover/image-comp.jpg}
         \caption*{Input image}
     \end{subfigure}
     \begin{subfigure}[h]{0.23\textwidth}
         \centering
         \includegraphics[width=\textwidth]{figures/cover/smplify-comp.jpg}
         \caption*{SMPLify-X~\cite{SMPL-X:2019}}
     \end{subfigure}
     \vspace*{0.2cm}
     \begin{subfigure}[h]{0.45\textwidth}
         \centering
         \includegraphics[width=0.98\linewidth, trim=25 50 25 50]{figures/cover/scene_cover_green-comp.png}
         \caption*{3D visualization of our (left) and SMPLify-X (right) results}
     \end{subfigure}
     \vspace*{-0.2cm}
     \caption{While the state-of-the-art single-view 3D pose estimator~\cite{SMPL-X:2019} yields a small reprojection error, the recovered 3D poses may be erroneous due to the depth ambiguity. We make use of the mirror in the image to resolve the ambiguity and reconstruct more accurate human pose as well as the mirror geometry.}
     \vspace*{-0.5cm}
    \label{fig:demo1}
\end{figure}



In many scenes like dancing rooms and gyms, people are often in front of a mirror. In this case, we are able to see the person and his/her mirror image simultaneously. The mirror image actually provides an additional virtual view of the person, which can resolve the single-view depth ambiguity if the mirror is properly placed. Moreover, unseen part of the person can also be observed from the mirror image, so that the occlusion problem can be alleviated. 


In this paper, we investigate the feasibility of leveraging such mirror images to improve the accuracy of 3D human pose estimation. We develop an optimization-based framework with mirror symmetry constraints that are applicable without knowing the mirror geometry and camera parameters. We also provide a method to utilize the properties of vanishing points to recover the mirror normal along with the camera parameters, so that an additional mirror normal constraint can be imposed to further improve the human pose estimation accuracy. The effectiveness of our framework is validated on a new dataset for this new task with 3D pose ground-truth provided by a multi-view camera system. 


An important application of the proposed approach is to generate pseudo ground-truth annotations to train existing 3D pose estimators. To this end, we collect a large-scale set of Internet images that contain people and mirrors and generate 3D pose annotations with the proposed optimization method. The dataset is named Mirrored-Human.  
Compared with existing 3D human pose datasets~\cite{h36m_pami,mono-3dhp2017,vonMarcard2018} that are captured with very few subjects and background scenes, Mirrored-Human has a significantly larger diversity in human poses, appearances and backgrounds, as shown in Fig.~\ref{fig:dataset}. The experiments show that, by combining Mirrored-Human with existing datasets as training data, both accuracy and generalizability of existing 3D pose estimation methods can be significantly improved for both single-person and multi-person cases.   

In summary, we make the following contributions:
\begin{itemize}
    \item We introduce a new task of reconstructing  human pose from a single image in which we can see the person and the person's mirror image. 
    \item We develop a novel optimization-based framework with mirror symmetry constraints to solve this new task, as well as a method to recover mirror geometry from a single image.
    \item We collect a large-scale dataset named Mirrored-Human from the Internet, provide our reconstructed 3D poses as pseudo ground-truth, and show that training on this new dataset can improve the performance of existing 3D human pose estimators. 
\end{itemize}








\section{Background}
%\outline{DRAM Background}

{We provide the relevant background on DRAM organization, DRAM operation and commodity DRAM based PuM techniques. We refer the reader to prior works for more comprehensive background about DRAM organization and operation~\cite{salp,lee.hpca13,donghyuk-ddma,chang.sigmetrics17,ghose2018vampire,patel2017reaper,luo2020clr,ghose2019demystifying,kevinchang-thesis,yoongu-thesis,lee.thesis16,olgun2021quactrng}.}

\subsection{DRAM Background}
\label{sec:background-dram}
DRAM-based main memory is organized hierarchically. \new{Fig.~\ref{fig:dram-bank-timing-diagram} (top) depicts this organization.} A processor is connected to one or \newnew{more} {memory channels \new{(DDRx in the figure)~\boldone{}}}. Each channel has its own command, address, and data buses. Multiple {memory modules} can be plugged into a single channel. Each module contains several {DRAM chips}\new{~\boldtwo{}. Each chip} contains multiple {\new{DRAM} banks} \new{that can be accessed} independently\new{~\boldthree{}}.\revdel{A set of DDRx standards cluster multiple {banks} in {bank groups}~\cite{jedecDDR4,gddr5}.}
\newnew{D}ata transfers between DRAM memory modules and processors occur at {cache block} granularity. The cache block size is typically 64 bytes in \atb{current} systems.


\begin{figure}[h]
     \centering
     
     \begin{subfigure}[h]{.50\textwidth}
         \centering
         \includegraphics[width=\textwidth]{figures/dram-bank.pdf}
     \end{subfigure}
     \hfill
     \begin{subfigure}[h]{.45\textwidth}
         \centering
         \includegraphics[width=\textwidth]{figures/timing-diagram.pdf}
     \end{subfigure}
    
    \caption{DRAM organization (top). Timing diagram of \new{DRAM} commands (bottom).}
    
    \label{fig:dram-bank-timing-diagram}
\end{figure}

\new{Inside a DRAM bank, DRAM cells are laid out \newnew{as} a two dimensional array of wordlines \new{(i.e., DRAM rows)} and bitlines \newnew{(i.e., DRAM columns)~\boldfour{}. W}ordlines are depicted in blue and bitlines are depicted in red in Fig.~\ref{fig:dram-bank-timing-diagram}. Wordline drivers drive the wordlines and sense amplifers read the values on the bitlines.} \newnew{A DRAM cell is connected to a bitline via an access transistor\new{~\boldfive{}}}. When enabled, an access transistor allows charge to flow between a DRAM cell and the cell's bitline.

\noindent
\new{\textbf{DRAM Operation.}} When all DRAM rows in a {bank} are closed, DRAM bitlines are precharged to \newnew{a reference voltage level of} {$\frac{V_{DD}}{2}$}. The memory controller sends an activate ($ACT$) command \new{to the DRAM module} to drive a DRAM wordline (i.e., enable a DRAM row). Enabling a DRAM row starts the {charge sharing} process. \newnew{Each DRAM cell connected to the DRAM row starts sharing its charge with its bitline}. This \omi{causes} \newnew{the bitline voltage} to deviate from {$\frac{V_{DD}}{2}$} {(i.e., the charge in the cell perturbs the bitline voltage)}. The \new{sense amplifier} sense\newnew{s} the deviation in the bitline and amplif\newnew{ies} the voltage of the bitline either to {$V_{DD}$} or to $0$. \newnew{As such}, \newnew{an ACT command} copies one DRAM row to the \new{sense amplifiers} (i.e., row buffer). The memory controller can send READ\newnew{/}WRITE commands to transfer data from/to {the sense amplifier array}. {Once the memory controller needs to access another DRAM row, t}he memory controller can close the {enabled} DRAM row by sending a precharge (PRE) command on the command bus. The PRE command first disconnects DRAM cells from their bitlines by disabling the enabled wordline and then precharges the bitlines to {$\frac{V_{DD}}{2}$}.

%\outline{Timing Parameters}
\noindent
\new{\textbf{DRAM Timing Parameters.}} DRAM datasheets specify a set of timing parameters that define the minimum time window between valid combinations of DRAM commands~\cite{lee.hpca15, kevinchang-thesis, chang.sigmetrics16, kim2018solar}. The memory controller must wait for tRCD, tRAS, and tRP nanoseconds between successive ACT $\rightarrow$ RD, ACT $\rightarrow$ ACT, and PRE $\rightarrow$ ACT commands, respectively {(Figure~\ref{fig:dram-bank-timing-diagram}, bottom)}. Prior works show that these timing parameters can be violated (e.g., successive ACT $\rightarrow$ RD commands may be issued with a shorter time window than tRCD) to improve DRAM access latency~\cite{lee.hpca15, chang.sigmetrics16, kevinchang-thesis, lee.sigmetrics17, kim2018solar}, implement physical unclonable functions~\cite{kim.hpca18,talukder2019exploiting,orosa2021codic}, \newnew{generate true random numbers~\cite{olgun2021quactrng,olgun2021quactrngieee,kim.hpca19},} copy data~\cite{seshadri2013rowclone,gao2020computedram}, and perform bitwise AND/OR operations~\cite{seshadri.micro17,seshadri.thesis16,seshadri.arxiv16,seshadri.bookchapter17.arxiv,gao2020computedram} in commodity DRAM devices.

\noindent
\textbf{\new{{DRAM Internal Address Mapping.}}}
\Copy{R2/1C}{{DRAM manufacturers use DRAM-internal address mapping schemes~\cite{salp,cojocar2020susceptible,patel2022case} to translate from logical (e.g., row, bank, column) DRAM addresses \newnew{that are used by the memory controller} to physical DRAM addresses \newnew{that are internal to the DRAM chip} (e.g., the \newnew{physical} position of a DRAM row within the chip). These schemes allow (i) post-manufacturing row repair techniques to map erroneous DRAM rows to redundant DRAM rows and (ii) DRAM manufacturers to organize DRAM internals in a cost-efficient \newnew{and reliable} way~\cite{khan.dsn16,vandegoor2002address}. DRAM-internal address mapping schemes can be substantially different across different DRAM chips~\cite{barenghi2018software,cojocar2020susceptible,horiguchi1997redundancy,itoh2013vlsi,keeth2001dram,khan.dsn16,khan.micro17,kim-isca2014,lee.sigmetrics17,liu.isca13,orosaYaglikci2021deeper,saroiu2022price,patel2020bit,patel2022case}. Thus, \newnew{consecutive} logical DRAM row addresses might not point to physical DRAM rows in the same subarray.}}
%\outline{in-DRAM Computation: RowClone, D-RaNGe, AMBIT, ...}

%\todo{Juan: Good for the readers to get an idea on how PIM techniques work: AMBIT, RowClone... Recent effort on doing computation in DRAM.}

\subsection{PuM Techniques}

\label{sec:background_pudram}
Prior work proposes a variety of in-DRAM computation mechanisms (i.e., PuM techniques) that (i) have great potential to improve system performance and energy efficiency~\cite{chang.hpca16,seshadri.micro17, hajinazarsimdram,seshadri2013rowclone,seshadri2020indram,seshadri.bookchapter17.arxiv,seshadri.bookchapter17,seshadri.arxiv16,Seshadri:2015:ANDOR,angizi2019graphide, ferreira2021pluto} %and 
\juan{or} (ii) can provide low-cost security primitives~\cite{talukder2019prelatpuf,talukder2019exploiting,kim.hpca19,kim.hpca18,olgun2021quactrngieee,orosa2021codic}. A subset of these in-DRAM computation mechanisms are demonstrated on real DRAM chips~\cite{gao2020computedram,kim.hpca19, kim.hpca18, talukder2019exploiting,olgun2021quactrngieee,orosa2021codic}. {We describe the \newnew{major relevant} prior works \omi{briefly}}: 

%\textbf{RowClone~\cite{seshadri2013rowclone}} proposes minor modifications to DRAM arrays to enable bulk data copy \atb{and initialization} inside DRAM. \atb{RowClone activates two DRAM rows successively without precharging the bitlines. By the time the second row is activated, the bitlines are loaded with the data in the first row. Bulk-copy and initialization intensive workloads (e.g., fork, memcached~\cite{memcached}) can significantly benefit from the increase in copy and initialization throughput provided by RowClone.}

%\outline{Explain ComputeDRAM}

%\todo{This should come later (after we discuss AMBIT/RowClone). Or we can talk about it in the later sections.}
%\jgl{I think this section is good, after having introduced RowClone/Ambit. We do not need many details about these works here, since we can give more background in the specific use case.}

\noindent
\newnew{\textbf{RowClone~\cite{seshadri2013rowclone}} \omi{is} a low-cost DRAM architecture that can perform bulk data movement operations (e.g., copy, initialization) inside DRAM chips \omi{at high performance and low energy}.}

\noindent
\newnew{\textbf{Ambit~\cite{Seshadri:2015:ANDOR, seshadri.arxiv16, seshadri.micro17, seshadri.bookchapter17, seshadri2020indram}} \omi{is} a new DRAM substrate that can perform \omi{(i)} bitwise majority \omi{(and thus bitwise AND/OR)} \omi{operations} across three DRAM rows by simultaneously activating three DRAM rows \omi{and (ii) bitwise NOT} operation\omi{s} \omi{on a DRAM row} using 2-transistor 1-capacitor DRAM cells~\cite{kang2009one,lu2015improving}.}

\noindent
\textbf{ComputeDRAM~\cite{gao2020computedram}} demonstrates in-DRAM copy {(previously proposed by RowClone~\cite{seshadri2013rowclone})} and bitwise AND/OR operations {(previously proposed by Ambit~\cite{seshadri.micro17})} on real DDR3 chips. ComputeDRAM performs in-DRAM operations by issuing carefully-engineered, valid sequences of DRAM commands {with violated tRAS and tRP timing parameters (i.e., by not obeying manufacturer-recommended timing parameters defined in DRAM \newnew{chip} specifications~\cite{micron2016ddr4})}. By issuing command sequences {with violated timing parameters}, \omi{ComputeDRAM \omi{activates}} two or three DRAM rows in a DRAM bank \newnew{in quick succession} \omi{(i.e., performs two or three row activation\omi{s})}. \newnew{ComputeDRAM leverages (i) two row activation\omi{s} to transfer data between two DRAM rows and (ii) three row activation\omi{s} to perform the majority function in real unmodified DRAM chips.} 
%This allows multiple DRAM cells on a column to contribute to the charge sharing process. 
\revdel{ComputeDRAM leverages multiple row activation (i) to transfer data between two DRAM cells, and (ii) to perform {the} majority function across three rows {in real unmodified DRAM chips}.}

\noindent
%\outline{Explain D-RaNGe}
%\todo{This also should come later (but in background section).}
\textbf{D-RaNGe~\cite{kim.hpca19}} is a {state-of-the-art} high-throughput DRAM-based true random number generat{ion technique}. D-RaNGe leverages the randomness in DRAM activation (tRCD) failures as its entropy source. D-RaNGe extracts random bits from DRAM cells that fail with $50\%$ probability when accessed with a reduced \newnew{(i.e., violated)} tRCD. 
%The authors assess the quality of random bitstreams generated by D-RaNGe using various statistical tests and show that it generates high-quality true random numbers. 
D-RaNGe demonstrates {high-quality true random number generation} on a vast number of real DRAM chips across multiple generations.

\noindent
\newnew{\textbf{QUAC-TRNG~\cite{olgun2021quactrngieee}} demonstrates that four DRAM rows can be activated in a quick succession using an ACT-PRE-ACT command sequence \omi{(called QUAC)} with violated tRAS and tRP timing parameters in real DDR4 DRAM chips. QUAC-TRNG uses QUAC to generate true random numbers at high throughput and low latency.}

%\outline{RISC-V Rocket-Chip SoC generator}
%\subsection{RISC-V Rocket-Chip SoC Generator}
%Rocket-Chip is an open-source System-on-Chip (SoC) generator in Chisel3~\cite{chisel} hardware construction language, built by the RISC-V community~\cite{asanovic2016rocket}. Rocket-Chip is used to generate configurable RISC-V system designs.
%\jgl{This subsection seems to be coming from nowhere: "We use Rocket-Chip to implement a RISC-V system in our \X prototype."}

%\footnote{NIST Statistical Test Suite (STS) is a collection of statistical tests that are widely used in assessing the quality of random number generators. If a random number generator passes all NIST STS tests it is very likely that the random number generator outputs uncorrelated, random bitstreams.}

\section{Motivation}
%Modern computing systems are bottlenecked by data movement in performance and energy efficiency. The data movement bottleneck has continued to worsen in the recent years as the improvements in DRAM-based main memory performance has not caught up to the improvements in processor performance. Recent developments in process technology~\cite{X} and the worsening data movement bottleneck motivated researchers to work on mechanisms that aim to alleviate the data movement bottleneck in computing systems. This effort brought attention to the processing-in-memory (PIM) paradigm, which tries to mitigate the data movement bottleneck by moving computation physically closer to memory.

%Processing-using-Memory (PuM) mechanisms enable computation inside \atb{the} memory. These mechanisms provide data copy~\cite{seshadri2013rowclone}, bitwise AND/OR/NOT~\cite{seshadri.micro17}, \juan{arithmetic operations~\cite{hajinazarsimdram,deng.dac2018}}, random number generation~\cite{kim.hpca19,talukder2019exploiting}, and physical unclonable functions~\cite{kim.hpca18,talukder2019exploiting}
%\jgl{DrAcc does not do matrix multiplication. You can say "ternary weight neural network inference" or simply "neural network inference". Or easier, just remove and move the citation next to SIMDRAM.} 
%by exploiting memory array structures and memory cell characteristics. Recent works demonstrate that some of these mechanisms can be implemented in commodity DRAM devices to \atb{significantly} improve system performance and reduce system energy consumption~\cite{gao2020computedram} \juan{or implement security primitives~\cite{kim.hpca18, kim.hpca19}}. We refer to these mechanisms as DRAM-based PuM techniques. Such techniques can reduce the data movement overhead in a broad range of computing systems from low-power edge devices to high-performance servers as DRAM-based main memory is ubiquitous in these systems.
%\jgl{PIM does not need much motivation at this point. But it is good to mention here the amazing speedups and energy savings that it can provide. You can take some numbers from RowClone and Ambit papers.
%Then, motivate why PiDRAM is necessary. This means enumerating the challenges for system integration of PIM and explaining why solutions to these challenges cannot be easily explored in conventional computers, simulators, or tools like SoftMC.}

%\textcolor{red}{Why is it important to study in-DRAM based mechanisms?} 
%It is important to study end-to-end implementations of DRAM-based PuM techniques as they are applicable to a vast amount of computing systems and have great potential to improve system performance and energy efficiency by moving computation to DRAM-based main memory and mitigating data movement overheads. Such end-to-end implementations of PuM techniques require modifications to all elements across the hardware/software stack as discussed in many prior work~\cite{X,Y,Z}. However, the scope and implications of such modifications are not studied end-to-end in a real system, hence the challenges in integrating PuM techniques to real systems are left unexplored, and the end-to-end benefits that PuM techniques can provide in a real system is not clear.

%\outline{Why we cannot use other platforms to study DRAM-PuMs}

\atb{Implementing DRAM-based PuM techniques and integrating them into a real system requires modifications across the hardware and software stack. End-to-end implementations of PuM techniques require proper tools that (i) are flexible, to enable rapid development of PuM techniques and (ii) support real DRAM devices, to correctly observe the effects of reduced DRAM timing operations that are fundamental \atb{to} enabling {commodity DRAM based} PuM in {real unmodified} DRAM devices. Existing {general-purpose} computers, specialized DRAM testing platforms (e.g., \newnew{those aforementioned, Section~\ref{sec:introduction}}) cannot be used to study end-to-end implementations of {commodity DRAM based} PuM techniques.}

\atb{First, implementing new DDRx command sequences that perform PuM operations requires modifications to the memory controller. {Existing general purpose} computers do not support customizations to the memory controller {to dynamically modify manufacturer-recommended DRAM timing parameters}~\cite{lee.hpca15,kim2018solar,chang.sigmetrics16,kim.hpca19,hassan2017softmc}. This hinders the possibility of studying end-to-end implementations of PuM techniques on such platforms. Second, PuM techniques impose data mapping and allocation requirements (Section~\ref{sec:rowclone}) that are not satisfied by current memory management and allocation mechanisms (e.g., malloc~\cite{malloc}). Current OS memory management schemes must be augmented to satisfy these requirements. Existing specialized DRAM testing platforms (e.g., SoftMC~\cite{hassan2017softmc}) do not have system support \newnew{to enable this}. By design, these platforms are not built for system integration. Hence, it is difficult to evaluate system-level mechanisms that enable PuM techniques on DRAM testing platforms. Third, system simulators (i) do \emph{not} model DRAM operation \newnew{that violates} manufacturer-recommended {timing parameters}, (ii) do \emph{not} have a way of interfacing with {real DRAM chips} that {embody undisclosed and unique characteristics that have implications on how PuM techniques are integrated into real systems (e.g., proprietary and chip-specific DRAM internal address mapping~\cite{cojocar2020susceptible,salp,patel2022case})} that influence PuM operations{, and (iii) \emph{cannot} support studies on the reliability of PuM techniques since system simulators do \emph{not} model environmental conditions \newnew{and process variation}.}}
\atb{We summarize the limitations of the relevant experimental platforms \newnew{later in} Table~\ref{table:tools}.}
%\jgl{I think it would be good to refer to Table 3 for other experimental platforms that lack the necessary features.}

\atb{Our \textbf{goal} is to develop a flexible \juan{end-to-end} framework that enables rapid \newnew{system} integration of {commodity DRAM based} PuM techniques and facilitates studies on end-to-end \newnew{full-system} implementations of PuM techniques using real DRAM devices. To this end, we develop \X.}


%\todo{Fix this paragraph:}
%\atb{Current PuDRAM techniques require DRAM command sequences that violate manufacturer-recommended command timings. }
%\textcolor{red}{To give me an idea:} Reason why we don't want to do this with simulators: in-DRAM computation mechanisms fundamentally require operating DRAM devices below manufacturer thresholds. It is really hard to characterize and precisely model DRAM behavior under manufacturer thresholds, at varying environmental conditions (e.g. temperature, voltage), and even if there is such a model it would be costly to integrate into a simulator and would further impact simulator performance in a bad way.
%Reason why we don't want to do this with DRAM testing platforms: They are not designed to be connected to a system as a memory controller or anything else. They are incompatible, not designed for that purpose.

\section{PiDRAM}
%\jgl{I think an overview section would be good. Then, next section can go into implementation details.}
%\jgl{Having an overview figure soon can also help us organize the section/s.} 
%\vspace{-1mm}

\sloppy
\revdel{\atb{We design the \X framework to} {solve system integration challenges and analyze trade-offs of end-to-end implementations of {commodity DRAM based} PuM techniques}
facilitate end-to-end implementations of {commodity DRAM based} PuM techniques.}
\new{Implementing commodity DRAM based PuM techniques end-to-end requires developing new hardware (HW) and software (SW) components or augmenting existing components with new functionality (e.g., memory allocation for RowClone requires a new memory allocation routine in the OS, Section~\ref{sec:rowclone_alignment}).}
To ease the process of modifying various components across the hardware and software stack to implement new PuM techniques, \X provides key HW and SW components. Figure~\ref{fig:pidram-overview} presents an overview of the HW and SW components of the \X framework. \Copy{R4/8}{{Later in Section~\ref{sec:execution-overview}, we describe the general workflow for executing a PuM operation on PiDRAM.}}
\begin{figure*}[!h]
  \centering
  \includegraphics[width=0.9\textwidth]{figures/02_overview.pdf}
  \caption{{PiDRAM overview. {Modified h}ardware {(in green)} and software {(in blue)} components. \revd{Unmodified components are in gray.} {A pumolib function executes load and store instructions in the CPU to perform PuM operations (in red).} We use yellow to highlight the key hardware structures that are controlled by the user to perform PuM operations.}}% \jgl{Rocker Core -> RISC-V CPU Core}}% \jgl{Make those numbers a little bit smaller. They are ugly.}}
  \label{fig:pidram-overview}
\end{figure*}
\label{sec:pidram}
\begin{table*}[!b]
  \centering
  \caption{{Pumolib functions}}
  \label{table:pumolib}
  \scriptsize
  \begin{tabular}{@{} lm{15em}m{46em} @{}}
  \toprule
  {\textbf{Function}} &  {\textbf{Arguments}} &  {\textbf{Description}}\\        
  \midrule
  \textbf{set\_timings} & RowClone\_T1, RowClone\_T2, tRCD & Updates CRF registers with the timing parameters used in RowClone (\emph{T1} and \emph{T2}) and D-RaNGe (\emph{tRCD}) operations.\\
  \textbf{rng\_configure} & period, address, bit\_offsets & Updates CRF registers, configuring the random number generator to to access the DRAM cache block at \emph{address} every \emph{period} cycles and collect the bits at \emph{bit\_offsets} from the cache block.\\
  \textbf{copy\_row} & source\_address, destination\_address & Performs a RowClone-Copy operation in DRAM from the \emph{source\_address} to the \emph{destination\_address}.\\
  \textbf{activation\_failure} & address & Induces an activation failure in a DRAM location pointed by the \emph{address}.\\
  \textbf{buf\_\omi{size}} & - & Returns the \newnew{number of random words in} the random number buffer.\\
  \textbf{rand\_dram} & - & Returns 32 bits \newnew{(\omi{i.e.,} random words)} from the random number buffer.\\
  \midrule
  \end{tabular}
\end{table*}
%We identify \atb{four} common components that are shared across end-to-end implementations of DRAM-based PuM techniques. The two HW components: \textbf{(i) an easy-to-extend memory controller} enables rapid implementation of \juan{specific} DDRx command sequences \juan{necessary to control PuM operations, % beyond
%e.g., sequences that violate} manufacturer-recommended \juan{timing} thresholds to perform PuM operations, \textbf{(ii) an ISA-transparent PuM controller} maintains compatibility between various processor microarchitectures, and provides a way of controlling PuM operations without making invasive modifications to the underlying processor microarchitecture.} 
%\textbf{(iii) a general controller $\longleftrightarrow$ application interface \todo{protocol}} that enables system designers to directly communicate PuM execution parameters to the memory controller, 
%\atb{The two SW components: \textbf{(i) an extensible software library} contains customizable functions which are used by applications to perform PuM operations, \textbf{(ii) a custom supervisor software} provides the necessary OS primitives (e.g., virtual memory) to explore system-level solutions that are required to implement PuM techniques end-to-end. 


%We design \atb{the} \X framework to facilitate end-to-end implementations of \atb{PuM} techniques. Implementing support for PuM techniques requires modifications across the HW/SW stack. \X comprises \atb{\emph{the necessary} HW/SW components} and interfaces to facilitate implementing new PuM techniques. Figure~\ref{fig:pidram-overview} presents an overview of HW/SW components of the \X framework. \X provides (i) ISA-transparent control (i.e., using conventional STORE/LOAD operations) of PuM techniques, \atb{to provide a framework compatible with microprocessors that implement different ISAs, and to enable designers to implement new PuM techniques without interfering with the microarchitecture employed in the system}, (ii) a general \emph{application} $\longleftrightarrow{}$ \emph{controller} interface that \atb{enables applications to express} PuM techniques in hardware, (iii) a new, custom and easy-to-extend memory controller that \atb{facilitates implementation of new} PuM \atb{techniques}, (iv) an extensible software library \atb{that uses the \emph{application} $\longleftrightarrow{}$ \emph{controller} interface to} to control PuM operations, \atb{which enables system designers to quickly implement the software support for PuM techniques} and (v) a \atb{custom} supervisor software that supports the \atb{necessary} OS primitives (e.g., virtual memory) to enable end-to-end exploration of PuM techniques. 
% PiDRAM builds on \atb{top of} the open-source rocket-chip SoC generator, which has been used numerous times to generate silicon-proven hardware~\cite{X}. 

%We demonstrate a prototype of PiDRAM on the Xilinx ZC706 FPGA board. \atb{We explain each component \X comprises in the remainder of this section.}


\subsection{Hardware Components}
\label{sec:hardware-components}

%\todo{Why do we need these structures? What does each component offer, be general/abstract? Motivate why we need these. Bring examples (e.g., we will use this for RowClone as we describe in Section~\ref{X})}

{PiDRAM comprises two key hardware components. Both of these components are designed with the goal to provide a flexible and easy to use framework for evaluating PuM techniques.}

%\X comprises two key hardware components that facilitate the implementation of new \atb{PuM} \tacorevcommon{operations}. \atb{First, to enable ISA-transparent control of PuM techniques, we implement the PuM operations controller (POC). \revd{The CPU} access\revd{es} (using memory LOAD/STORE instructions) the memory-mapped registers in POC to execute in-DRAM operations. Second, to facilitate the implementation of new DDRx command sequences on \X, we implement a modular and flexible custom memory controller.}

\textbf{\ding{182} PuM Operations Controller \newnew{(POC)}.} {POC decodes and executes PiDRAM instructions (e.g., RowClone-Copy~\cite{seshadri2013rowclone} that are used by the programmer to perform PuM operations. POC communicates with the rest of the system over two well{-}defined interfaces. First, it communicates with the CPU over a memory-mapped interface{, where the CPU can send data to or receive data from POC using memory store and load instructions}. The CPU accesses the memory-mapped registers (\emph{instruction}, \emph{data}, and \emph{flag} registers) in POC to execute in-DRAM operations. This improves the portability of the framework and facilitates porting the framework to systems that employ different instruction set architectures. Second, {POC} communicates with the memory controller to perform PuM operations in the DRAM chip over a simple hardware interface. {To do so,} POC (i) requests the memory controller to perform a PuM operation, (ii) waits until the memory controller performs the operation, and (iii) receives the result of the PuM operation from the memory controller. The CPU can read the result of the operation by executing load instructions that target the \emph{data} register in POC.}

%The functions in PuM operations library (pumolib) perform PuM operations by executing LOAD/STORE instructions that target the \atb{instruction, data, and flag} registers.
%\jgl{Last sentence seems disconnected from the explanation of POC.}

%\textbf{PuM Operations \todo{protocol}.} \atb{The PuM operations interface (POI) defines the communication protocol between the POC and the memory controller.} POI implements a general interface that enables application developers to express the information required to execute PuM operations to the memory controller. POI abstracts the hardware communication protocol between the POC and the custom memory controller from the application developer. \atb{By setting the \emph{start} bit in the \emph{flag} register, the application developer initiates a PuM operation. The memory controller responds by setting the \emph{ack} bit in the \emph{flag} register upon receiving the PuM \emph{instruction} and starts performing the corresponding operation. When an operation (e.g., RowClone-Copy) completes, the memory controller sets the \emph{finish} bit in the flag register.}
%POI implements a modified VALID-IO~\cite{X} protocol \atb{on top of} the 128-bit instruction and the 512-bit data signals. POC initiates a PuM operation by setting the \emph{valid} signal \atb{on the POI}. If the memory controller is \emph{ready} to perform the in-DRAM operation, it acknowledges IDOC's request by setting the \emph{ack} signal. The memory controller similarly sets the \emph{valid} signal when it has valid data on the data bus. We implement IDOI to abstract the Chisel3 layer from \atb{PiDRAM designers}. To implement a new in-DRAM operation, it is sufficient for PiDRAM \atb{designers} to modify the pidlib and a \atb{small set of Verilog} modules in the custom memory controller.

\atb{\textbf{\ding{183} Custom Memory Controller.} {PiDRAM's memory controller provides an easy-to-extend basis for {commodity DRAM based} PuM techniques that require issuing DRAM commands with violated timing parameters~\cite{gao2020computedram,kim.hpca19, kim.hpca18, talukder2019exploiting,olgun2021quactrngieee}. 
The memory controller is designed modularly and requires {easy{-}to{-}make} modifications to its scheduler to implement new PuM techniques.} For instance, our modular design enables supporting RowClone operations (Section~\ref{sec:rowclone}) in just 60 lines of Verilog code on top of the baseline custom memory controller's scheduler that implements conventional DRAM operations (e.g., read, write).}

\Copy{R1/6}{\atb{The custom memory controller employs three key sub-modules to facilitate the implementation of new PuM techniques. %: 
(i) \juan{The \emph{Periodic Operations \revf{Module}}} %, which 
periodically \new{issues} DDR3 refresh~\cite{micron2018ddr3} and interface maintenance commands~\cite{softmc.github}. %, 
(ii) \juan{A} simple \emph{DDR3 Command Scheduler} %that 
supports {conventional DRAM operations (e.g., activate, precharge, read, and write)}. \juan{This} scheduler applies an open-bank policy (i.e., DRAM banks are left open following a DRAM row activation) to exploit temporal locality in memory accesses to the DRAM module. {LOAD/STORE memory requests are simply handled by the command scheduler in a latency-optimized way. \new{Thus,} new modules \new{that are implemented to provide new PuM functionality (e.g., a state machine that controls the execution of a new PuM operation)} in the custom memory controller do not compromise the performance of LOAD/STORE memory requests.}}} {(iii) \Copy{R3/6}{\juan{The \emph{Configuration Register File}} (CRF) comprises 16 user-programmable registers that store \newnew{the violated} timing parameters used for DDRx sequences that trigger PuM operations \newnew{(e.g., activation latency used in generating true random numbers using D-RaNGe~\cite{kim.hpca19}, \newnew{see Section~\ref{sec:drange}})} and miscellaneous parameters for PuM implementations (e.g., true random number generation period for D-RaNGe, \newnew{see Section~\ref{sec:drange}}). {In our implementation, CRF stores only the timing parameters used for performing PuM operations (e.g., RowClone and D-RaNGe). We do not store every standard DDRx timing parameter (i.e., non-violated, \omi{which are used exactly as} defined as in DRAM chip specifications) in the CRF. Instead these timings are embedded in the \newnew{command} scheduler.}}}

\begin{figure*}[!t]
  \centering
  \includegraphics[width=0.9\textwidth]{figures/03_flow.pdf}
  \caption{{Workflow for a PiDRAM RowClone-Copy operation}}% \jgl{Rocker Core -> RISC-V CPU Core}}% \jgl{Make those numbers a little bit smaller. They are ugly.}}
  \label{fig:flow}
  
\end{figure*}
\subsection{Software Components}
\label{sec:software-components}
{PiDRAM comprises two key software components that complement {and control} PiDRAM's hardware components {to} provid{e} a flexible and easy to use {end-to-end} PuM framework.} 

%\atb{\X comprises two major software components that are essential to facilitate the implementation and evaluation of PuM techniques \atb{end-to-end}. First, the PuM operations library (pumolib) encapsulates functions to execute PuM operations using POC. For example, a RowClone (Section~\ref{sec:rowclone}) operation requires executing a sequence of LOAD/STORE operations targeting POC registers. Pumolib's RowClone function controls the execution of RowClone operations in \X. Second, to enable end-to-end implementations of PuM techniques, the custom supervisor software provides the necessary OS primitives (e.g., virtual memory, system calls).}

\textbf{\ding{184} PuM Operations Library (pumolib).} {The extensible library (\emph{PuM} \emph{o}perations \emph{lib}rary) allows system designers to implement software support for PuM techniques. Pumolib contains customizable functions that interface with POC to perform PuM operations in real unmodified DRAM chips. The customizable functions \newnew{hide} the hardware implementation details of PuM techniques implemented in \X{} \newnew{from software developers (that use pimolib)}.} For example, {although we expose PuM techniques to software via memory LOAD/STORE operations (POC is exposed as a memory-mapped module, Section~\ref{sec:hardware-components}), PuM techniques can also be exposed via specialized instructions provided by ISA extensions.} Pumolib \newnew{hides} such implementation details from the user of the library and \atb{contributes to the modular design of the} framework. 

%\atb{\juan{The} supervisor software (i.e., system software)} uses this library to perform PuM operations.


We implement a general protocol that defines how {programmers} express the information required to execute PuM operations to the PuM operations controller (POC). {A typical function in {pumolib} performs a PuM operation in four steps: \Copy{R3/3}{It (i) writes a PiDRAM instruction to {the} POC's \emph{instruction} register, (ii) sets the \emph{Start} {flag} in POC's \emph{flag} register, (iii) waits for {the} POC to set the \emph{Ack} {flag} in POC's \emph{flag} register, and (iv) reads the result of the PuM operation from POC's \emph{data} register {(e.g., the true random number after performing a{n in-DRAM true random number generation} operation, Section~\ref{sec:drange})}. {We list the currently implemented pumolib functions in Table~\ref{table:pumolib}.}}}



%\atb{By setting the \emph{start} bit in the \emph{flag} register, the \revd{system designer} initiates a PuM operation. The memory controller responds by setting the \emph{ack} bit in the \emph{flag} register upon receiving the PuM \emph{instruction} and starts performing the corresponding operation. \Copy{R3/3}{When an operation (e.g., RowClone-Copy) completes, the memory controller sets the \emph{finish} bit in the flag register. Pumolib contains template functions that initiate PuM operations in the memory controller by communicating with the POC over the described protocol. \changev{\ref{q:r3q3}}\tacorevc{We list these functions in Table~\ref{table:pumolib}.}}}




\textbf{\ding{185} Custom Supervisor Software.}\revdel{End-to-end implementation of PuM techniques requires modifications across the hardware and software stack.} \X provides a custom supervisor software that \newnew{implements} the necessary OS primitives (i.e., virtual memory \juan{management}, memory \juan{allocation and alignment) \newnew{for end-to-end implementation of PuM techniques}.} %management.
{This facilitates developing end-to-end integration of PuM techniques {in the system} as these techniques require modifications across the software stack. For example, integrating RowClone end-to-end {in the full system} requires a new memory allocation mechanism (Section~\ref{sec:rowclone_alignment}) that can satisfy the memory allocation constraints of RowClone~\cite{seshadri2013rowclone}. \new{Thus, we implement the necessary functions and data structures in the custom supervisor software to implement an allocation mechanism that satisfies RowClone's constraints. This allows \X{} to be extended easily to implement support for new PuM techniques that share similar memory allocation constraints (\newnew{e.g., Ambit~\cite{seshadri.micro17}, SIMDRAM~\cite{hajinazarsimdram}, and QUAC-TRNG~\cite{olgun2021quactrngieee}, as shown in} Table~\ref{table:use-cases}).}}

%PK implements a variety of system calls and supports virtual memory. We implement system support for in-DRAM computation primitives on PK. We augment the memory management module of the PK \atb{to enable RowClone and D-RaNGe} and we implement the PiDRAM library (pidlib) as \atb{a} part of PK (\todo{Section~\ref{X}}).

\subsection{Execution of a PuM Operation}
\label{sec:execution-overview}

{We describe the general workflow for a PiDRAM operation (e.g., RowClone-Copy~\cite{seshadri2013rowclone}, random number generation using D-RaNGe~\cite{kim.hpca19}) in Figure~\ref{fig:flow} over an example \texttt{{copy\_row()}} function that is called by the user to {perform a RowClone-Copy operation} in DRAM.}


{{T}he user makes a system call to the custom supervisor software \scalebox{1.1}{{\ding{172}}} that in turn calls the {\texttt{copy\_row(source, destination)}} function in the {pumolib} \scalebox{1.1}{{\ding{173}}}. The function executes {two} store instructions in the RISC-V core \scalebox{1.1}{\ding{174}}. {The first store} instruction {update{s}} the \emph{instruction} register with the {copy\_row} instruction (i.e., the instruction that performs a {RowClone-Copy} operation in DRAM) \scalebox{1.1}{{\ding{175}}} and {the second store instruction} {set{s} the Start flag in the flag register to logic-1 \scalebox{1.1}{\ding{176}} in POC.} {When the Start flag is set,} POC instructs the PiDRAM memory controller to perform a {RowClone-Copy} operation using violated timing parameters \scalebox{1.1}{\ding{177}}. {{T}he {POC waits until the memory controller starts executing the {operation, after which it}} sets the Start flag to logic-0 and the Ack {flag} to logic-1 \scalebox{1.1}{\ding{178}}{, indicating that it started the execution of the PuM operation}.} {T}he PiDRAM memory controller performs the {RowClone-Copy} operation by issuing a set of DRAM commands with violated timing parameters \scalebox{1.1}{{\ding{179}}}. {{When the last DRAM command is issued, the memory controller} sets the Finish flag (denoted as Fin. in Figure~\ref{fig:flow}) in the flag register to logic-1 \scalebox{1.1}{{\ding{180}}}, indicating the end of execution for the last PuM operation that the memory controller acknowledged.} {The copy function periodically checks {either} the Ack {or the} Finish flag in the flag register {(depending on a user-supplied argument)} by executing load instructions that target the flag register \scalebox{1.1}{\ding{181}}. {When the periodically checked flag is set, the copy function returns.} This way, the copy function optionally blocks until the start {(i.e., the Ack flag is set)} or the end {(i.e., the Finish flag is set)} of the execution of the PuM operation (in this example, RowClone-Copy).\footnote{{The data register is not used in \newnew{a} RowClone-Copy~\cite{seshadri2013rowclone} operation because the result of the RowClone-Copy operation is stored {\emph{in memory}} (i.e., the source {memory row} is copied to the destination {memory row}). The data register is used in \newnew{a} D-RaNGe~\cite{kim.hpca19} operation{, as described in Section~\ref{sec:drange}}. {When used, t}he command scheduler store{s} the random numbers generated by {the} D-RaNGe operation in the data register. To read the generated random number, we implement a pumolib function {called} \texttt{rand\_dram()} that executes load instructions in the {RISC-V} core to retrieve the random number from the data register in POC.}}}}


%\atb{Figure~\ref{fig:pidram-overview} describes the general workflow for executing a PuM operation using the hardware and software components of \X. The application typically executes pumolib functions via system calls to the supervisor software (\boldone). Pumolib implements functions to perform PuM operations (\boldtwo). These functions STORE data to \emph{instruction} and \emph{flag} registers of the PuM operations controller (\boldthree). Setting the \emph{start} bit in the flag register initiates a PuM operation in the scheduler (\boldfour). The scheduler accesses DRAM using timing parameters defined in the configuration register file (CRF). Upon the start and at the end of PuM operations, the scheduler sets the \emph{ack} and the \emph{finish} bits in the flag register, respectively (\boldfive).}
%\jgl{This paragraph seems part of 4.2, but it is not. It deserves a subsection.}

\begin{table*}[b]
    \centering
    \scriptsize
    \caption{\omi{Various} \newnew{known} PuM techniques that can be studied using \X. PuM techniques we implement \juan{in this work} are highlighted in bold.}
    \hspace{1em}
    \begin{tabular}{m{12em}m{10em}m{45em}}
    \toprule
    \textbf{PuM Technique} & \textbf{Description} & \textbf{Integration Challenges} \\
    \midrule
    {\textbf{ComputeDRAM-based~\textbf{\cite{gao2020computedram}}}} \textbf{RowClone~\cite{seshadri2013rowclone}} & Bulk data-copy \juan{and initialization} within DRAM & (i) \emph{memory allocation \juan{and alignment} mechanisms} that map source \& destination operands of a copy operation into same DRAM subarray; (ii) \juan{\emph{memory coherence}, i.e.}, source \revdel{\& destination }operand must be up-to-date in DRAM.\\
    \midrule
    \textbf{D-RaNGe}~\cite{kim.hpca19} & True random number generation using DRAM & (i) periodic generation of true random numbers; (ii) \emph{memory scheduling policies} that minimize the interference caused by random number requests. \\
    \midrule
    {ComputeDRAM-based~\cite{gao2020computedram}} Ambit~\cite{seshadri.micro17} & Bitwise operations in DRAM & (i) \emph{memory allocation \juan{and alignment} mechanisms} that map operands of a bitwise operation into same DRAM subarray; (ii) \juan{\emph{memory coherence}, i.e.}, operands of the bitwise operations must be up-to-date in DRAM. \\
    \midrule
    SIMDRAM~\cite{hajinazarsimdram} & \juan{Arithmetic operations in DRAM} & (i) \emph{memory allocation \juan{and alignment} mechanisms} that map operands of an arithmetic operation into same DRAM subarray; (ii) \juan{\emph{memory coherence}, i.e.}, operands of the arithmetic operations must be up-to-date in DRAM; (iii) \juan{\emph{bit transposition}, i.e., operand bits must be laid out vertically in a single DRAM bitline}. \\
    \midrule
    DL-PUF~\cite{kim.hpca18} & Physical unclonable functions in DRAM & \emph{memory scheduling policies} that minimize the interference caused by generating PUF responses. \\
    \midrule
    \reva{QUAC-TRNG~\cite{olgun2021quactrng} \newnew{and Talukder+~\cite{talukder2019exploiting}}} & \reva{True random number generation using DRAM} & \reva{(i) periodic generation of true random numbers; (ii) \emph{memory scheduling policies} that minimize the interference caused by random number requests; (iii) efficient integration of the SHA-256 cryptographic hash function.} \\
    \bottomrule
    \end{tabular}
    
    \label{table:use-cases}
    
\end{table*}
\subsection{Use Cases}
\label{sec:use-cases}
\atb{\X is primarily designed to study end-to-end implementations of {commodity DRAM based} PuM techniques~\cite{olgun2021quactrng,gao2020computedram,kim.hpca18,kim.hpca19,talukder2019exploiting} on real systems. {Beyond commodity DRAM based PuM techniques}, many prior works propose minor modifications to DRAM arrays to enable various arithmetic~\cite{hajinazarsimdram,deng.dac2018,ferreira2021pluto,angizi2019graphide} and bitwise operations~\cite{seshadri.micro17,seshadri2020indram,seshadri.bookchapter17.arxiv,Seshadri:2015:ANDOR,angizi2019graphide} and security primitives~\cite{orosa2021codic}.}
%\jgl{Not matrix multiplication}. 
{These PuM techniques share common memory allocation and coherenc\newnew{e} requirements (Section~\ref{sec:rowclone_alignment}) that must be satisfied to enable their end-to-end integration {into a real system.} \X{} facilitates the implementation of PuM techniques and enables rapid exploration of such integration challenges on a real DRAM-based system.} \newnew{Table ~\ref{table:use-cases} describes some of the \juan{PuM} case studies \X can enable.}




{\new{Other than providing an easy-to-use basis for end-to-end implementations of commodity DRAM based PuM techniques,} \X{} can be \new{easily} extended with a programmable microprocessor placed near the memory controller to study system integration challenges of Processing-near-Memory (PnM) techniques (e.g., efficient pointer chasing~\cite{impica, hashemi.isca16,cont-runahead}, \new{general-purpose compute~\cite{upmem2018}, machine learning~\cite{kwon2021fimdram, ke2021near, kim2021aquabolt, lee2022isscc, niu2022isscc}, \newnew{databases~\cite{lee2022improving,boroumand2019conda,boroumand2016pim}}, \omi{graph processing~\cite{besta2021sisa}}}).\revdel{ \atb{\X} can be deployed on appropriate FPGA boards as a testbed for new memory devices with compute capability~\cite{upmem2018,kwon2021fimdram}.}}


\subsection{{PiDRAM's HW \& SW Components: Summary}}
\label{sec:component-summary}

\Copy{R3/2}{
{We identify and build two hardware components (PuM Operations Controller and Custom Memory Controller) and two software components (PuM Operations Library, Custom Supervisor Software) as key components that \new{are} commonly required by \new{end-to-end PuM implementaions}. We reuse these key components to implement two different PuM mechanisms (RowClone in Section~\ref{sec:rowclone} and D-RaNGe in Section~\ref{sec:drange}) in PiDRAM. The key components can be reused in the same way {to} implement other PuM mechanisms (e.g., the ones in Table~\ref{table:use-cases}). However, reusing a component does not mean that the component can {simply} be instantiated in a system and the system will be able to perform PuM operations immediately.} 

{We acknowledge that these components require modifications to implement new PuM techniques in PiDRAM and possibly to integrate PiDRAM into other systems. In fact, we quantify the degree of these modifications in our RowClone and D-RaNGe case studies. We show that the key components form a useful and easy-to-extend basis for PuM techniques with our Verilog and C code complexity analyses for both use cases (Sections~\ref{sec:rowclone-experimental-methodology} and~\ref{sec:drange-evaluation}).}
}
\subsection{\X Prototype}
\label{sec:prototype}

\Copy{R3/7B}{\atb{We develop a prototype of the \X framework on an FPGA-based platform. We use the Xilinx ZC706 FPGA board~\cite{zc706} to interface with real DDR3 modules. {Xilinx provides a DDR3 PHY IP~\cite{virtex7mig} that exposes a low-level ``DFI'' interface~\cite{dfi} to the DDR3 module on the board. We use this interface to issue DRAM commmands to the DDR3 module.} We \newnew{use} the existing RISC-V based SoC generator, \omi{Rocket Chip}~\cite{asanovic2016rocket}, to {generate the RISC-V hardware system}. Our custom supervisor software extends the RISC-V \newnew{Proxy Kernel~\cite{riscv-pk}} to support the necessary OS primitives on \X's prototype.} Figure~\ref{fig:prototype} shows our prototype.}

\begin{figure}[!h]
    \centering
    \includegraphics[width=1.0\linewidth]{figures/04_prototype.pdf}
    \caption{PiDRAM's FPGA prototype}
    \label{fig:prototype}
\end{figure}


\noindent
\Copy{R1/1B}{{{\textbf{\new{Simulation Infrastructure.}} To \newnew{aid} the users \newnew{in testing}} the correctness of any modifications \newnew{they make} on top of PiDRAM, we provide the developers with a Verilog simulation {environment} that injects regular READ/WRITE commands and custom commands (e.g., update {the Configurable Register File (CRF)}, perform RowClone-Copy, generate random numbers) to the memory controller. When used in conjunction with the Micron DDR3 Verilog model provided by Xilinx~\cite{virtex7mig}, the simulation \newnew{environment} can help the developers \newnew{to} easily understand if something unexpected is happening in their implementation (e.g., {if} timing {parameters} are violated).}}

\noindent
\omi{\textbf{Open Source Repository.} We make PiDRAM freely available to the research community as open source software at \url{https://github.com/CMU-SAFARI/PiDRAM}. Our repository includes the full PiDRAM prototype that has RowClone (Section~\ref{sec:rowclone}) and D-RaNGe (Section~\ref{sec:drange}) implemented end-to-end on the RISC-V system.}

\section{Case Study \#1: End-to-end RowClone}
%\jgl{For each use case, let's first give enough background on the technique. Then, enumerate the specific challenges/issues we deal with. Then, present explain the solution to each challenge. Finally, a subsection with the evaluation (I would include the experimental methodology as the first subsection of the evaluation).}

%\outline{Re-introduce RowClone briefly.}

\label{sec:rowclone}

\revdel{RowClone~\cite{seshadri2013rowclone} proposes \atb{minor changes to DRAM {chips} to copy data in DRAM to mitigate data movement overheads}. 
%RowClone provides up to \todo{X}\% system performance improvement on average \atb{across} real workloads. 
ComputeDRAM~\cite{gao2020computedram} demonstrates in-DRAM copy operations on contemporary, off-the-shelf DDR3 chips. \atb{Their results show that c}urrent DRAM devices can \atb{reliably} perform copy operations {at different temperatures and supply voltage levels using a set of violated tRAS and tRP timing parameters} in DRAM row granularity. \Copy{R3/1}{{None of the relevant prior works~\cite{seshadri.micro17,seshadri2013rowclone,wang2020figaro,gao2020computedram} provide a clear description {or a real system demonstration (like we do)} of a working memory allocation mechanism that can be integrated into a real operating system to expose RowClone capability to the programmer.}}}

We implement support for ComputeDRAM-\new{based} (i.e., using carefully-engineered sequences of valid DRAM commands {with violated timing parameters}) \new{RowClone (in-DRAM copy/initialization)} operations on PiDRAM to conduct a detailed study \newnew{of}\revdel{ (i) the system performance benefits that RowClone can provide and (ii)} the challenges associated with implementing RowClone end-to-end on a real system. \new{\Copy{R3/1}{{None of the relevant prior works~\cite{seshadri.micro17,seshadri2013rowclone,wang2020figaro,gao2020computedram,seshadri2020indram,seshadri.bookchapter17.arxiv,seshadri.thesis16,hajinazarsimdram} provide a clear description {or a real system demonstration} of a working memory allocation mechanism that can be \new{implemented in} a real operating system to expose RowClone capability to the programmer.}}} \revdel{\new{Using our real system prototype, we study the performance benefits that RowClone can provide in detail.}}

\subsection{Implementation Challenges}
%\outline{Talk about the problems briefly:} 

\noindent
\textbf{{Data Mapping.}}
\label{sec:rowclone_alignment}
{RowClone has data mapping and alignment requirements that {cannot be} satisfied by current memory allocation mechanisms (e.g., malloc~\cite{malloc}). We identify four major issues that complicate the process of implementing support for RowClone in real systems. First, \newnew{the} source and destination operands \new{(i.e., \omi{page (4 KiB)-sized} arrays)} of the copy operation must reside in the same DRAM subarray. We refer to this as the \emph{mapping} problem. Second, the source and destination operands must be aligned to DRAM rows. We refer to this as the \emph{alignment} problem. Third, the size of the copied data must be a multiple of the DRAM row size. The size constraint defines the granularity at which we can perform bulk-copy operations using RowClone. We refer to this as the \emph{granularity} problem.} Fourth, \new{RowClone must operate on up-to-date data that reside\newnew{s} in main memory.} Modern systems employ caches to exploit locality in memory accesses and reduce memory latency. \new{Thus,} cache blocks \new{(typically 64 B)} of either the source or the destination operand\newnew{s} of the RowClone operation {may} have \newnew{cache block} copies present in the cache hierarchy. Before performing RowClone, {the} cached copies \newnew{of pieces of both source and destination operands} must be invalidated and written back to main memory\revdel{ if necessary}. We refer to this as the %\emph{coherency} 
{\emph{memory coherence}} problem.


\new{We explain the data mapping and alignment requirements of RowClone \newnew{using} Figure~\ref{fig:rowclone_alignment}.} \reve{\new{The figure} depicts a \new{simplified version of a} DRAM {chip} with two banks and two subarrays. The {operand} Source 1 cannot be copied to the {operand} Target 1 as \new{the operands} do not satisfy the \emph{granularity} constraint (\boldone). Performing such a copy operation would overwrite the \newnew{remaining (i.e., non-Target 1)} data in \new{Target 1's DRAM row} with \newnew{the remaining (i.e., non-Source 1)} data in \new{Source 1's DRAM row}. Source 2 cannot be copied to Target 2 as Target 2 is not \emph{aligned} to its DRAM row (\boldtwo). Source 3 cannot be copied to Target 3, as these {operands} are not \emph{mapped} to the same DRAM subarray (\boldthree). \new{In contrast}, Source 4 can be copied to Target 4 using in-DRAM copy \new{because these operands} are (i) \emph{mapped} to the same DRAM subarray, (ii) aligned to their DRAM rows and (iii) occupy their rows completely (i.e., the {operands} have sizes equal to DRAM row size) (\boldfour).}
%\jgl{Add the circled numbers that the figure has.}
\iffalse
\begin{figure*}[!h]
     \centering
     \begin{subfigure}[b]{0.55\textwidth}
          \centering
          \includegraphics[width=\textwidth]{figures/rowclone_alignment.pdf}
          \caption{A DRAM {chip} with two banks and two subarrays. Only \omi{one} operation \omi{(i.e., operation~\newnew{\boldfour{}})} can succeed as its operands satisfy \omi{all of} \emph{mapping}, \emph{alignment} and \emph{granularity} constraints.}
          \label{fig:rowclone_alignment}
     \end{subfigure}
     \hfill
     \begin{subfigure}[b]{0.40\textwidth}
          \centering
          \includegraphics[width=\textwidth]{figures/memory-allocation-mechanism.pdf}
          \caption{Overview of our memory allocation mechanism\revdel{ Source and destination operands (Arrays A and B) \newnew{that are} comprised of page-sized memory blocks are placed in DRAM subarrays.}}
          \label{fig:memory-allocation-mechanism}
     \end{subfigure}
     \caption{\newnew{RowClone memory allocation requirements (left), memory allocation mechanism overview (right)}}
     \vspace{-2mm}
\end{figure*}
\fi
\begin{figure}[h]
  \centering
  \includegraphics[width=\linewidth]{figures/rowclone_alignment.pdf}
  \caption{A DRAM {chip} with two banks and two subarrays. Only \omi{one} operation \omi{(i.e., operation~\newnew{\boldfour{}})} can succeed as its operands satisfy \omi{all of} \emph{mapping}, \emph{alignment} and \emph{granularity} constraints.}
  \label{fig:rowclone_alignment}
\end{figure}

\subsection{Memory Allocation Mechanism}
\label{sec:rowclone_mechanism}

\new{C}omputing systems employ various layers of address mappings that obfuscate the DRAM row-bank-column address mapping from the programmer~\cite{helm2020Reliable,cojocar2020susceptible}, \new{which makes allocating source and target operands as depicted in Figure~\ref{fig:rowclone_alignment}-(\boldfour) difficult}. \new{DRAM manufacturers employ DRAM\newnew{-}internal address mapping schemes (Section~\ref{sec:background-dram}) that translate from logical (e.g., \newnew{memory-controller-visible} DRAM row, bank, column) \omi{addresses} to physical DRAM addresses.} {\new{G}eneral-purpose} processors use complex functions to map physical addresses to DDRx addresses \new{(e.g., DRAM banks, rows and columns)}~\cite{hillenbrand2017Physical}. \new{The} operating system (OS) maps virtual addresses to physical addresses to provide isolation between multiple processes\revdel{ and %an 
\juan{the} illusion of larger-than-available physical memory}. \new{Only these virtual addresses are exposed to the programmer}. Without control over the virtual address $\rightarrow{}$ DRAM address mapping, the programmer \emph{cannot} \new{easily} place data in a way %to satisfy 
\juan{that satisfies} the mapping and alignment requirements of \newnew{an} in-DRAM copy operation. 

%Either many layers of address mapping must be exposed to the programmer, or the OS should expose a memory allocation interface that satisfies the requirements of in-DRAM copy operations. 

%\todo{Emphasize that we can allocate data even if there is a complex DDRX mapping, we just need to find the DRAM rows that are in the same subarray}

\new{W}e implement a new memory allocation mechanism that {can perform} memory allocation for \new{RowClone (in-DRAM copy/initialization)} operations. \atb{\new{This} mechanism enables page-granularity \new{RowClone} operations (i.e., a virtual page can be copied to another virtual page using RowClone) \emph{without} introducing any changes to the programming model.} \new{The mechanism} places the operands of RowClone operations \new{in} the same DRAM subarray while maximizing the bank-level parallelism in regular DRAM accesses (reads \& writes) to these operands \new{(such that the commonly\newnew{-}performed streaming accesses to these operands benefit from bank-level parallelism in DRAM)}. {\atb{Figure~\ref{fig:memory-allocation-mechanism} depicts an overview of our memory allocation mechanism.}}

%\iffalse
\begin{figure}[!ht]
  \centering
  \includegraphics[width=\linewidth]{figures/memory-allocation-mechanism.pdf}
  \caption{Overview of our memory allocation mechanism\revdel{ Source and destination operands (Arrays A and B) \newnew{that are} comprised of page-sized memory blocks are placed in DRAM subarrays.}}
  \label{fig:memory-allocation-mechanism}
\end{figure}
%\fi

\new{At a high level, o}ur \new{memory allocation} mechanism \new{(i)} splits the source and destination operands into page-sized virtually-addressed memory blocks, \new{(ii)} allocate\new{s} two physical pages in different DRAM rows in the same DRAM subarray,\revdel{ These physical pages are aligned to the same page-size boundary (i.e., satisfying the \emph{alignment} requirement).} \new{(iii) assigns} these physical pages to virtual pages that correspond to the source and destination memory blocks at the same index such that the source block can be copied to the destination block using RowClone. We repeat this process until we exhaust the page-sized memory blocks. \new{As the mechanism processes subsequent page-sized memory blocks of the \newnew{two} operands, it allocates} physical pages from a different DRAM bank \new{to maximize bank-level parallelism in streaming accesses to these operands}.

To overcome the \emph{mapping}, \emph{alignment}, and \emph{granularity} problems, we implement our memory management mechanism in the custom supervisor software of \X.
We expose \new{the allocation mechanism} \newnew{using} the \texttt{alloc\_align(N, ID)} system call. The system call returns a pointer to a contiguous array of \emph{N} bytes in the virtual address space {(i.e., one operand)}. \Copy{R5/5}{Multiple calls with the same \emph{ID} to \texttt{alloc\_align(N, ID)} place the allocated arrays in the same subarray in DRAM, such that they can be copied from one to another using RowClone. {If \emph{N} is too large such that it exceeds the size of available physical memory, \texttt{alloc\_align} fails and causes an exception.} \revdel{Figure~\ref{fig:alloc-align-walkthrough}\revdel{shows the components of our mechanism and} presents the workflow of \texttt{alloc\_align()}.}} \revd{Our implementation of\revdel{\texttt{alloc\_align}} \new{RowClone} requires application developers to directly use \texttt{alloc\_align} to allocate data instead of \texttt{malloc} and similar function calls.}

%We characterize DRAM rows for their RowClone success rates and initialize the SAMT with DRAM row address pairs that have 100\% RowClone success rate. \revd{We conduct a reliability study by repeatedly performing RowClone operations using randomly initialized source and destination DRAM rows, and validate the correctness of RowClone operations by comparing the data on the destination row with the data on the source row. We observe no errors after performing 1000 RowClone operations using all possible (source, destination) row pairs in one DRAM subarray.}
%\jgl{Say here that Figure 5 shows the workflow and components.}

\iffalse
\begin{figure}[!ht]
  \centering
  \includegraphics[width=.40\textwidth]{figures/SAMT.pdf}
  \caption{Organization of the subarray mapping table. Each physical address in a tuple in the list addresses one half of a DRAM row}
  \label{fig:SAMT}
\end{figure}
\fi

{\new{The \omi{custom supervisor software}} maintains three key structures \new{to make \texttt{alloc\_align()} work}: (i) Subarray Mapping Table (SAMT), (ii) Allocation ID Table (AIT), and (iii) Initializer Rows Table (IRT).}

\noindent
\textbf{{1) Subarray Mapping Table (SAMT).}} We use the \textbf{S}ubarray \textbf{Ma}pping \textbf{T}able (SAMT) to maintain a list of physical page addresses that point to DRAM rows that are in the same DRAM subarray. {\new{\texttt{alloc\_align()}} queries SAMT to find physical addresses that map to rows in one subarray.} 
%Figure~\ref{fig:SAMT} depicts the organization of the SAMT. 

\new{SAMT} contains the physical pages that point to DRAM rows in each \new{subarray}. \new{SAMT is} indexed using subarray identifiers (SA IDs) in the range \emph{[0, number of subarrays\revdel{ in a DRAM bank})}. \new{\new{SAMT}} contains an entry for every subarray\revdel{ in the DRAM \new{bank}}. \new{An} entry consists of two elements: (i) the number of free physical address tuples and (ii) a list of physical address tuples. Each tuple in the list contains two physical addresses that \newnew{respectively} point to the first and second hal\newnew{ves} of the same DRAM row. The list of tuples \newnew{contains} all the physical addresses that point to DRAM rows in the DRAM subarray indexed by the SAMT\revdel{ \new{sub-table}} entry. We allocate free physical pages \new{listed in an} entry and assign them to the virtual pages (i.e., memory blocks) that make up the {row-copy operands (i.e., arrays)} allocated by \texttt{alloc\_align()}. We slightly modify our \new{high-level memory} allocation mechanism to allow for two memory blocks \new{(4 KiB virtually-addressed pages)} of an array to be placed in the same DRAM row, as the page size in our system is 4 KiB, and the size of a DRAM row is 8 KiB. \atb{We call two memory blocks in {the same operand} that are placed in the same DRAM row \emph{sibling memory blocks} \omi{(also called sibling pages)}. The parameter \emph{N} \new{of} the \texttt{alloc\_align()} call defines this relationship: We designate memory blocks that are precisely {\emph{N/2}} bytes apart as \emph{sibling memory blocks}.}


\noindent
\textbf{{Finding \new{the} DRAM Rows in a Subarray.}} \Copy{R2/1A}{{Finding the DRAM row addresses that belong \newnew{to} the same subarray is not straightforward due to DRAM-internal mapping schemes employed by DRAM manufacturers (Section~\ref{sec:rowclone_alignment}). It is extremely difficult to learn which DRAM address (i.e., bank-row-column) is actually mapped to a physical location (e.g., a subarray) in the DRAM device, as these mappings are not exposed through publicly accessible datasheets or standard definitions~\cite{jedecDDR4,micron2016ddr4,patel2022case}. We make the key observation that the entire mapping scheme need \emph{not} be available to successfully perform RowClone operations.}}

\Copy{R2/1B}{We observe that for a set of \emph{\{source, destination\}} DRAM row address pairs, RowClone operations \atb{repeatedly} succeed with a 100\% probability. We hypothesize that these pairs of DRAM row addresses are mapped to the same DRAM subarray. {We identify these row address pairs by conducting a \emph{\newnew{RowClone success rate}} experiment where we repeatedly perform RowClone operations between every \emph{source, destination} row address pair in a DRAM bank. Our experiment works in three steps{:} we (i) initialize both the source and the destination row with random data, (ii) perform a RowClone operation from the source to the destination row, and (iii) compare the data in the destination row with the source row. \omi{RowClone success rate is calculated as the number of bits that differ between the source and destination rows' data divided by the number of bits stored in a row (8 KiB in our prototype).} If there is no difference between the source and the destination rows' data \omi{(i.e., the RowClone success rate for the source and the destination row is 100\%)}, we {infer that} the RowClone operation {was} successful. We repeat the experiment for 1000 iterations for each row address pair and if every iteration is successful, we store the address pair in the SAMT{, indicating that the row address pair is mapped to different rows in the same DRAM subarray}.}} \Copy{R4/9}{{\newnew{The same RowClone success rate experiment} could be conducted \revdel{to identify DRAM row address pairs {that are mapped to the same DRAM subarray (i.e., data can be copied from one row address to the other row address in the pair} using RowClone operations) }in \newnew{other} systems that are based on PiDRAM or in a PiDRAM prototype that uses a different DRAM {module}. Since \newnew{the RowClone success rate experiment} is a one-time process, its overheads (e.g., time taken to iterate over all DRAM rows using our experiment) \omi{are} amortized over the lifetime of such a system.}}

\noindent
\textbf{{2) Allocation ID Table (AIT).}}
To {keep track of different operands} that are allocated by \texttt{alloc\_align} using the same \emph{ID} \newnew{(used to place different arrays in the same subarray)}, we use the \textbf{A}llocation \textbf{I}D \textbf{T}able (AIT).\revdel{ AIT is logically partitioned into sub-tables for each DRAM bank.} AIT entries are indexed by \emph{allocation ID}s (the parameter \emph{ID} \new{of} the \emph{alloc\_align} call). Each AIT entry stores a pointer to an SAMT entry. The SAMT entry pointed by the AIT entry contains the set of physical addresses that were allocated using the same \emph{allocation ID}. AIT entries are used by the \texttt{alloc\_align} function to find which DRAM subarray can be used to allocate DRAM rows from, such that the newly allocated array can be copied to other arrays allocated using the same \emph{ID}.



\noindent
\textbf{{3) Initializer Rows Table (IRT).}}
To find which row in a DRAM subarray can be used as the source operand in zero-initialization (RowClone-Initialize) operations, we maintain the \textbf{I}nitializer \textbf{R}ows \textbf{T}able (IRT). The IRT is indexed using physical page numbers. RowCopy-Initialize operations query the IRT to obtain the physical address of the DRAM row that is initialized with zeros and \new{that belong} to the same subarray as the destination operand (i.e., the DRAM row to be initialized with zeros). 


\atb{Figure~\ref{fig:alloc-align-walkthrough} describes how \texttt{alloc\_align()} works over an end-to-end example. \newnew{Using the RowClone success rate experiment (\omi{described above}), the custom supervisor software (\omi{CSS for short})} find\newnew{s} the DRAM rows that are in the same subarray (\boldone) and initialize\newnew{s} the \newnew{Subarray Mapping Table (SAMT)}. The programmer allocates two 128 KiB arrays, A and B, via \texttt{alloc\_align()} using the same \emph{allocation id} (\textbf{0}), with the intent to copy from A to B (\boldtwo). \newnew{\omi{CSS}} allocate\newnew{s} contiguous ranges of virtual addresses to A and B, \newnew{and} then split\newnew{s the virtual address ranges} into page-sized memory blocks (\boldthree). \newnew{\omi{CSS}} assigns consecutive memory blocks to consecutive DRAM banks and access\new{es} the \newnew{Allocation ID Table (AIT)} with the \emph{allocation id} (\boldfour) for each memory block. \newnew{By accessing the AIT, \omi{CSS} retrieves} the \emph{subarray id} that points to a SAMT entry. The SAMT entry corresponds to the subarray that contains the arrays that are allocated using the \emph{allocation id} (\boldfive). \newnew{\omi{CSS}} access\newnew{es} the SAMT entry to retrieve two physical addresses that point to the same DRAM row. \newnew{\omi{CSS}} map\newnew{s} a memory block and its \emph{sibling \omi{memory block}} \new{(i.e., the memory block that is N/2 bytes away from this memory block, where N is the \emph{size} argument of the \texttt{alloc\_align()} call)} to these two physical addresses, such that they are mapped to the first and the second halves of {the same} DRAM row (\boldsix). \reva{Once allocated, these physical addresses are pinned to main memory and cannot be swapped out to storage.} Finally, \newnew{\omi{CSS}} update\newnew{s} the page table with the physical addresses to map the memory blocks to the same DRAM row (\boldseven).}

\begin{figure*}[!ht]
  \centering
  \includegraphics[width=.8\textwidth]{figures/alloc-align-rcc.pdf}
  \caption{{\texttt{Alloc\_align() and RowClone-Copy (\texttt{rcc}, see Section~\ref{sec:rccrci}) workflow.}}} %\jgl{This figure may be better at the bottom of page 7}
  \label{fig:alloc-align-walkthrough}
  
\end{figure*}

\subsection{Maintaining \juan{Memory Coherence}}

\label{sec:coherency}
{Since memory instructions update the cached copies of data \newnew{(Section~\ref{sec:rowclone_alignment})},} a naive implementation of RowClone can potentially operate on stale data because cached copies of RowClone operands can be modified \new{by CPU store instructions}. 
%To maintain coherency, we flush dirty cache blocks of the operands of the RowClone operations.
\juan{\new{Thus,} we need to ensure memory coherence to prevent RowClone from operating on stale data.}

\new{W}e implement a new \juan{custom RISC-V} instruction, \juan{called \emph{CLFLUSH},} to flush dirty cache blocks to DRAM %, as 
(RISC-V does not implement any cache management operations~\cite{riscv-spec}) \juan{so as to ensure RowClone operates on up-to-date data}. %We implement a custom RISC-V cache flush instruction, CLFLUSH, in our infrastructure. We use CLFLUSH to flush 
\juan{\newnew{A} CLFLUSH \newnew{instruction} flushes \newnew{(invalidates)}} \newnew{a} {physically addressed} dirty \newnew{(clean)} cache block. \atb{CLFLUSH or other cache management operations with similar semantics are supported in X86~\cite{x86-manual} and ARM architectures~\cite{arm-cmos}. \new{Thus, the CLFLUSH instruction (that we implement) provides a minimally invasive solution (i.e., it requires no changes to the specification of commercial ISAs) to the memory coherence problem.}}

\revdel{ \new{This way, the effects of CLFLUSH on the performance of end-to-end RowClone (Section~\ref{sec:rowclone-experimental-methodology}) \revdel{likely}represents \new{a case} of a minimally invasive solution (i.e., it requires no changes to the specification of commercial ISAs).}} %Our solution to 
%\juan{With CLFLUSH, we address the \emph{memory coherence} problem %is 
%with a minimally invasive solution} (i.e., it requires no changes to the microarchitecture and the specification in commercial ISAs)}. 
%and likely represents the solution that would be implemented in a commercial system that implements RowClone.} 
%\jgl{I commented out the second part of the sentence because I find it unnecessary and risky.}
We modify the non-blocking data cache and the \omi{R}ocket core modules (defined in \textit{NBDCache.scala} and \textit{rocket.scala} in \omi{Rocket Chip}~\cite{asanovic2016rocket}, respectively) to implement CLFLUSH. We modify the RISC-V GNU compiler toolchain~\cite{riscv-gnu-toolchain} to expose CLFLUSH as an instruction to C/C++ applications. %We use the CLFLUSH instruction to flush dirty cache blocks that map to the source and destination rows in RowClone operations.
\juan{Before executing a RowClone \newnew{Copy or Initialization} operation \newnew{(see Section~\ref{sec:rccrci})}, \newnew{the \omi{custom supervisor software}} \new{flushes} \atb{(invalidate\new{s})} the cache blocks \new{of} the source \atb{(destination)} row of the RowClone operation \atb{using CLFLUSH}.}

\subsection{RowClone-Copy and RowClone-Initialize}
\label{sec:rccrci}
\new{W}e support the RowClone-Copy and RowClone-Initialize operations in our custom supervisor software via two functions: (i) \newnew{RowClone-Copy,} \texttt{rcc(void *dest, void *src, int size)} and (ii) \newnew{RowClone-Initialize,} \texttt{rci(void* dest, int size)}. \atb{\texttt{rcc} copies \emph{size} \newnew{number of} contiguous bytes in the virtual address space starting from the \emph{src} memory address to the \emph{dest} memory address. \texttt{rci} initializes \emph{size} \newnew{number of} contiguous bytes in the virtual address space starting from the \emph{dest} memory address.} We expose \texttt{rcc} and \texttt{rci} to user-level programs \newnew{using} system calls \newnew{defined in the \omi{custom supervisor software}}. 
% \jgl{Explain syntax and semantics of these APIs.}

\juan{\texttt{rcc}} \new{(i)} splits the source and destination operands into page-aligned, page-sized blocks, \new{(ii)} traverses the page table \atb{(Figure~\ref{fig:alloc-align-walkthrough}-\boldeight)} to find the physical address of each block (i.e., the address of a DRAM row), \new{(iii)} flushes all cache blocks corresponding to the source \newnew{operand} and \atb{invalidates all cache blocks corresponding to the} destination \newnew{operand}, \new{and (iv)}
%\jgl{Do we need to flush destination row or only source row?}, 
performs a RowClone operation from the source row to the destination row using pumolib{'s \texttt{copy\_row()} function}.

\juan{\texttt{rci}} \new{(i)} splits the destination operand into page-aligned, page-sized blocks, \new{(ii)} traverses the page table to find the physical address of \newnew{the destination operand}\new{, (iii)} queries the \newnew{Initializer Rows Table (IRT, see Section~\ref{sec:rowclone_mechanism})} to obtain the physical address of the initializer row \atb{(i.e., source operand)}, \new{(iv)} \atb{invalidates} the cache blocks corresponding to the \newnew{destination operand}, and \new{(v)} performs a RowClone operation from the initializer row to the destination row using using pumolib{'s \texttt{copy\_row()} function}. 
%\jgl{There is something confusing about CLFLUSH in combination with rcc. In 5.3, it seems that we expose CLFLUSH to the program. But here it seems that rcc calls CLFLUSH, and not the program directly. Please clarify.}


\subsection{Evaluation}

We evaluate our solutions for the challenges in implementing RowClone end-to-end on a real system using \X. We modify the custom memory controller to implement DRAM command sequences \atb{($ACT\rightarrow{}PRE\rightarrow{}ACT$)} to trigger RowClone operations. \atb{We set the $tRAS$ and $tRP$ parameters to 10 $ns$ {(below the manufacturer-recommended \SI{37.5}{\nano\second} for tRAS and \SI{13.5}{\nano\second} for tRP~\cite{micron2018ddr3})}.} We modify our custom supervisor software to implement our memory allocation mechanism and add support for RowClone-Copy \new{(\texttt{rcc})} and RowClone-Initialize \new{(\texttt{rci})} operations.

%\jgl{I don't know if "Evaluation" is the right place for this subsection. To me, this is related to the implementation.}

\subsubsection{Experimental Methodology}
\label{sec:rowclone-experimental-methodology}
\revdel{We use \X to implement RowClone end-to-end \juan{and evaluate our solutions to RowClone's system integration challenges, as explained in Section~\ref{sec:rowclone_alignment}}.} 
Table~\ref{table:system-configuration} describes the configuration of the components in our system. \atb{We use the pipelined and in-order \newnew{R}ocket core with 16 K\newnew{i}B L1 data cache and 4\newnew{-}entry TLB as the main processor of our system. We use the 1 GiB DDR3 module available on the ZC706 board as the main memory where we conduct PuM operations.} 

%\jgl{Describe the system briefly here. Place the table after this paragraph}. 

\iffalse
\begin{table}[!ht]
   
\caption{PiDRAM system configuration (left). Physical address to DRAM address mapping in \X (right). \newnew{B}yte offset is used to address the byte in the DRAM burst.}
    \centering
\begin{minipage}[h]{0.48\linewidth}\centering
\scriptsize

  \begin{tabular}{@{} l @{}}
  \toprule
  \textbf{CPU:} 50~MHz; in-order Rocket core \cite{asanovic2016rocket}; \textbf{TLB} 4 entries DTLB; LRU policy\\        
  \midrule
  \textbf{L1 Data Cache:} 16~KiB, 4-way; 
  64~B line; random replacement policy\\
  \midrule
  \textbf{DRAM Memory:} 1~GiB DDR3; 800MT/s; single rank; 8~KiB row size\\
  \midrule
  \label{table:system-configuration}
  \end{tabular}
  %}
  
\end{minipage}\hfill%
\begin{minipage}[h]{0.48\linewidth}\centering
\includegraphics[width=1.0\textwidth]{figures/rbc-mapping.pdf}
  \label{fig:DDR-address-mapping}
\end{minipage}

  \vspace{-8mm}
\end{table}
\fi

\begin{table}[!ht]
  \centering
  \caption{PiDRAM system configuration}
  
  \scriptsize
  %\tiny
  %\resizebox{\columnwidth}{!}
  
  \label{table:system-configuration}
\end{table}

\atb{Implementing RowClone require\newnew{s} an additional 198 lines of Verilog code over \X's existing Verilog design. We add 43 and 522 lines of C code to pumolib and to our custom supervisor software, respectively, to implement RowClone in the software components.}

%\jgl{I think this paragraph goes better after Table 2.}

Table~\ref{fig:DDR-address-mapping} describes the mapping scheme we use in our custom memory controller to translate from physical to DRAM row-bank-column addresses. We map physical addresses to DRAM columns, banks, and rows from lower-order bits to higher-order bits to exploit the bank-level parallelism in memory accesses to consecutive physical pages. We note that our memory management mechanism is compatible with other physical address $\rightarrow{}$ DRAM address mappings~\cite{hillenbrand2017Physical}. \Copy{R2/2}{{For example, for a mapping {scheme} where \omu{page offset bits (physical address (PA) [11:0]) include all or a subset of the bank address bits}\revdel{\omi{DRAM bank and DRAM row addresses, in that order, are mapped from the most significant to the least significant bits in physical addresses (i.e., physical address (PA) [29:27] maps to DRAM bank address and PA[26:13] maps to DRAM row address)}},\revdel{ bank address and the column address \newnew{in the address scheme depicted in Table~\ref{table:system-configuration}} {are} swapped\revdel{(Table~\ref{fig:DDR-address-mapping}},} a single RowClone operand \omu{(i.e., a 4 KiB page)} would be split across multiple DRAM banks. This \new{only} coarsens the granularity of RowClone operations as the \newnew{{sibling \omi{pages}}} that must be copied in unison, to satisfy the granularity constraint, increases.}} We expect that for \new{other complex or unknown} physical address $\rightarrow{}$ DRAM address mapping scheme\new{s}, the characterization of the DRAM device for RowClone success rate would take longer. In the worst case, DRAM row addresses that belong to the same DRAM subarray can be found by testing all combinations of physical addresses \newnew{for their RowClone success rate}. 
%\jgl{First, explain the mapping and why you use this one. Second, explain (if possible) that other mappings are possible.}. 

\begin{figure}[!ht]
  \centering
  \includegraphics[width=0.40\textwidth]{figures/rbc-mapping.pdf}
  \caption{Physical address to DRAM address mapping in \X. \newnew{B}yte offset is used to address the byte in the DRAM burst.}
  \label{fig:DDR-address-mapping}
\end{figure}
We evaluate \texttt{rcc} and \texttt{rci} operations under two configurations to understand the copy/initialization throughput improvements provided by \texttt{rcc} and \texttt{rci} \newnew{over traditional CPU-copy operations performed by the Rocket core}, and to understand the overheads introduced by end-to-end support for {commodity DRAM based} PuM operations. We test two configurations: (i) \emph{Bare-Metal}, to find the maximum RowClone throughput our implementation provides solely using pumolib, 
%(ii) \emph{E2E}, to understand the performance benefits our end-to-end implementation of RowClone can bring, 
and (ii) \emph{No Flush}, to understand the benefits our end-to-end implementation \new{(i.e., with system support)} of RowClone can provide in copy/initialization throughput when data in DRAM is up-to-date \omi{(i.e., when no coherence operations are needed)}.

%\jgl{Explain why different configurations and high level view of what is different}: 
% (i) \jgl{Start giving the name of the configuration. Then, explain} 
\noindent
\textbf{Bare-Metal.} We assume that RowClone operations always target data that is allocated correctly in DRAM (i.e., \newnew{there is} no overhead introduced by address translation, IRT accesses, and CLFLUSH operations). We directly issue RowClone operations via pumolib using physical addresses. \newnew{Traditional CPU-copy operations (executed on the Rocket core) also use physical addresses}.

%\textbf{E2E.} We assume that the programmer uses the \texttt{alloc\_align} function to allocate the operands of RowClone operations. We issue RowClone operations using the \texttt{rcc} and \texttt{rci} system calls.
\noindent
\textbf{No Flush.} We assume that the programmer uses the \texttt{alloc\_align} function to allocate the operands of RowClone operations. We use %modified 
{a} version of \texttt{rcc} and \texttt{rci} system calls that do not use CLFLUSH to flush cache blocks of source and destination operands of RowClone operations. We run the \emph{No Flush} configuration on our custom supervisor software. \newnew{Both \texttt{rcc} and \texttt{rci}, and traditional CPU-copy operations use virtual addresses}.

%we assume that data allocation and address translation incurs no performance overhead (i.e., we directly issue RowClone operations via pumolib using physical addresses), depicted as \textbf{Bare-Metal}, (ii) we only assume that data in DRAM is always up-to-date (i.e., we do not flush dirty cache blocks, but we allocate data for RowClone operations and translate from virtual to physical addresses), depicted as \textbf{No Flush} and (iii) we allocate data for RowClone operations and flush cache blocks to address the challenges described in Section~\ref{X}, depicted as \textbf{E2E}. We run the \textbf{Bare-Metal} configuration bare-metal on rocket (i.e., no supervisor software, no virtual memory). We run the other two configurations on our custom supervisor software. \jgl{I would have a paragraph for each configuration, with the name of the configuration with bold font in the beginning of the paragraph.}

\revcommon{\subsubsection{Workloads}}
\label{sec:methodology-workloads}
For the two configurations, we run a microbenchmark that consists of two programs, \emph{\newnew{copy}} and \emph{\newnew{init}}, on \new{our prototype}. Both programs take the argument $N$, where \emph{copy} copies an $N$-byte array to another $N$-byte array and \emph{init} initializes an $N$\newnew{-}byte array \newnew{to all} zeros. Both programs \atb{have two versions: (i) CPU-copy, which} copies/initializes data using memory loads and stores, (ii) RowClone, which uses RowClone operations to perform copy/initialization. \atb{\new{All programs use} \texttt{alloc\_align} \new{to allocate data}.}
%\jgl{Better say that each of the microbenchmarks has two versions: (1) a CPU version, and (2) a RowClone version}.
\atb{The performance results we present in this section are the average of a \omi{1000} runs.}
%\jgl{Bad wording. "The performance results we present in this section are the average of a thousand runs."}, 
\atb{To maintain the same initial system state for both CPU-copy and RowClone, \revdel{(i.e., initially the data is up-to-date in DRAM), }we flush all cache blocks \newnew{before} each \newnew{one} \new{of the \omi{1000}} runs.}
%\jgl{This should be a separate sentence. Say why you flush.}. 
We run each program for array sizes (\emph{N}) that are powers of two and {$8~KiB$} $< N < 8~MiB$,  
%\jgl{Say these are the size of the arrays. Say you test powers of 2 sizes between these two.} 
and find the average copy/initialization throughput \newnew{across all 1000 runs} (by measuring the \# of elapsed CPU cycles to execute copy/initialization operations) for CPU-copy, RowClone-Copy (\texttt{rcc}), and RowClone-Initialize (\texttt{rci}).\footnote{\Copy{R1/2}{{We tested RowClone operations using \texttt{alloc\_align()} with up to 8 MiB of allocation size {since} we observed diminishing returns on performance improvement provided by RowClone operations \newnew{on} {larger} array sizes.}}}


\atb{We analyze the overheads of CLFLUSH operations on copy/initialization throughput that \texttt{rcc} and \texttt{rci} can provide. We measure the execution time of CLFLUSH operations \new{in} our \new{prototype} to find how many CPU cycles it takes to flush a (i) dirty and (ii) clean cache block on average \newnew{across 1000 measurements}. We simulate various scenarios \new{(described in Figure~\ref{fig:system-flush-overhead})} where we assume a certain \newnew{fraction} of the operands of RowClone operations are cached and dirty.}

%\atb{Finally, we conduct a \emph{projection} study to approximate the copy throughput improvement RowClone can provide when enabled end-to-end in a contemporary system. We use a recent \textbf{AMD Zen2~\cite{X}} processor with a DDR4 module operating at \textbf{3200 MT/s} as main memory. We run the \emph{copy} microbenchmark for 8 MiB arrays to find the copy throughput the contemporary system can achieve. We run a microbenchmark to find the latency cost of flushing cache blocks in the contemporary system using the \texttt{clflush\_opt}\todo{<-check} instruction~\cite{X}. We tightly schedule the DDR4 commands that need to be issued to perform RowClone in DRAM to find the latency of performing one RowClone operation (i.e., copying 8 KBs). We pessimistically assume that the contemporary processor spends as many cycles as rocket does when running the \texttt{rcc} function.}

\subsubsection{Bare-Metal RowClone}

Figure~\ref{fig:bare-metal-speedup} \juan{shows} the throughput improvement \omi{provided by} \texttt{rcc} {and \texttt{rci}} for \emph{copy} {and \emph{initialize}} \omi{over CPU-copy and CPU-initialization}  
%and the relative increase in execution time \jgl{I'd better talk about "throughput improvement" and "latency increase".} of rcc operations between consecutive
for increasing array sizes. 

\begin{figure}[h] %[!t]
  \centering
  \includegraphics[width=\linewidth]{figures/bare-metal-speedup.pdf}
  \caption{\omi{RowClone-Copy} and \omi{RowClone-Initialize} over traditional CPU-copy and -initialization for the Bare-Metal configuration%\todo{Fix y axis title, shrink height.}
  %\jgl{I changed the position of this figure. It better goes after the paragraph where first referenced.}
  %\jgl{There is no triangle mark for 8KiB. It should be "1".}
  }
  \label{fig:bare-metal-speedup}
\end{figure}
\iffalse
\begin{figure*}[!h]
     \centering
     \begin{subfigure}[b]{0.47\textwidth}
          \centering
          \includegraphics[width=\textwidth]{figures/bare-metal-speedup.pdf}
          \caption{Throughput improvement provided by \newnew{\omi{RowClone-Copy} and \omi{RowClone-Initialize} over traditional CPU-copy and -initialization for the Bare-Metal configuration}%\todo{Fix y axis title, shrink height.}
          %\jgl{I changed the position of this figure. It better goes after the paragraph where first referenced.}
          %\jgl{There is no triangle mark for 8KiB. It should be "1".}
          }
          \label{fig:bare-metal-speedup}
     \end{subfigure}
     \hfill
     \begin{subfigure}[b]{0.47\textwidth}
         \centering
         \includegraphics[width=\textwidth]{figures/system-copy-init-speedup1.pdf}
         \caption{Throughput improvement provided by \newnew{\omi{RowClone-Copy} and \omi{RowClone-Initialize} over traditional CPU-copy and -initialization for the NoFlush configuration}}
         \label{fig:system-copy-speedup1}
     \end{subfigure}
     \caption{\newnew{\omi{RowClone-Copy} and \omi{RowClone-Initialize} throughput improvement for the Bare-Metal (left) and the NoFlush (right) configurations}}
     \vspace{-2mm}
\end{figure*}
\fi
We make two major observations. \newnew{First, we observe that \texttt{rcc} and \texttt{rci} provide significant throughput improvement over traditional CPU-copy and \omi{CPU}-initialization. The throughput improvement \omi{provided by \texttt{rcc}} ranges from 317.5$\times{}$ (for 8 KiB arrays) to 364.8$\times{}$ (for 8 MiB arrays). \omi{The throughput improvement provided by \texttt{rci} ranges} from 172.4$\times{}$ to 182.4$\times{}$.} \newnew{Second,} the throughput improvement provided by \texttt{rcc} and \texttt{rci} increases as the array size increases. This increase saturates when the array size \newnew{reaches} 1 MiB. 
\omi{The load/store instructions used by CPU-copy and CPU-initialization access the operands in a streaming manner. The eviction of dirty cache blocks (i.e., the destination operands of copy and initialization operations) interfere with other memory requests on the memory bus.\footnote{\omi{Because the data cache in our prototype employs random replacement policy, as the array size increases, the fraction of cache evictions among all memory requests also increase\omi{s}, causing \omu{larger} interference on the memory bus (i.e., more memory requests to satisfy all cache evictions). The interference saturates at 1 MiB array size.}} We attribute the observed saturation at 1 MiB array size to the interference on the memory bus.}

%We attribute this to the interference \omi{o}n the memory bus \newnew{in CPU-copy and -initialization} caused by the cache blocks that are evicted \newnew{(CPU-copy and -initialization operations make the cache blocks of the destination operands dirty)} from the cache as the \newnew{copy and init microbenchmarks} access the arrays in a streaming manner\changev{Because our cache is 16 KiB 4-way, and employs a random replacement policy, as the copied/inited array size increases, the fraction of cache evictions among all memory requests also increase. If this is not too much detail or the text is not satisfactory, I can integrate this explanation.}\changev{I removed some text that I deemed redundant/not important}.
\revdel{Second, the \emph{latency} of a RowClone-Copy operation to copy 8 KiB in DRAM (using pumolib) is only 58 CPU cycles.
%\jgl{Can you justify where these 74 come from? Precise breakdown is not needed though} 
%to perform one RowClone-Copy operation to copy an 8 KiB array. 
\atb{This time is spent on (i) running the pumolib function that executes memory requests (e.g., \omi{\emph{store}} instructions
%\jgl{SD? Do you mean STORE?}
) to store data in the instruction and flag registers in the POC to ask the memory controller to perform a RowClone operation, (ii) waiting for the memory controller to respond with an acknowledgment}. The latency of executing RowClone-Copy increases \emph{linearly} with the array size. {We make similar observations for RowClone-Initialize. RowClone-Initialize can provide nearly the half of the throughput improvement provided RowClone-Copy, 182.4$\times{}$ over the CPU-Initialization baseline.  Because the CPU needs to execute only half as many instructions compared to copy operations (one load and one store) in initialization operations (one store), it can perform an initialization operation approximately two times faster than a copy operation for the same array size.}}

%\atb{A constant factor constituting the execution time of rcc operations (e.g., function call overhead) \jgl{I do not understand this sentence}.} \atb{We attribute the increase in performance improvement provided by rcc across array sizes to the sub-linear increase in relative execution time of rcc operations.}\jgl{Can you explain why there is a sub-linear increase and there is no increase after 1MiB?}



\subsubsection{\juan{No Flush} RowClone}

We analyze the overhead in copy/initialization throughput introduced by system support (Section~\ref{sec:rowclone_mechanism})\revdel{that we implement to enable RowClone end-to-end}.
%, and CLFLUSH \jgl{Good instruction name? Isn't this x86?} instructions that we use to maintain coherency. 
\reva{Figure~\ref{fig:system-copy-speedup1}} shows the throughput improvement of copy and initialization provided in the \emph{No Flush} configuration by \texttt{rcc} and \texttt{rci} operations.


\begin{figure}[h] %[!ht]
  \centering
  \includegraphics[width=\linewidth]{figures/system-copy-init-speedup1.pdf}
  \caption{Throughput improvement provided by \newnew{\omi{RowClone-Copy} and \omi{RowClone-Initialize} over traditional CPU-copy and -initialization for the NoFlush configuration}.}
    %\jgl{Here it is "speedup", but it is "performance improvement" in Fig. 7. Be consistent. "Throughput improvement" is better, I think.}
  \label{fig:system-copy-speedup1}
\end{figure}

\revdel{\newnew{Figure~\ref{fig:system-copy-speedup2-abs} shows the execution time of \texttt{rcc} and \texttt{rci} operations in CPU cycles.} Figure~\ref{fig:system-copy-speedup2} shows the proportional increase in \texttt{rcc} and \texttt{rci}'s execution time between consecutive array sizes \newnew{(i.e., the y-axis value for an array size S (between 16 KiB and 8 MiB) shows the execution time of \texttt{rcc} or \texttt{rci} normalized to the execution time of \texttt{rcc} or \texttt{rci} for an array size of S/2)}. For example, the \newnew{circles} at 16 KiB \omi{(2 MiB)} shows the execution time of \texttt{rcc} \newnew{\texttt{and rci}} for 16 KiB \omi{(2 MiB)} arrays divided by the execution time of \texttt{rcc} \newnew{\texttt{and rci}} for 8 KiB \omi{(1 MiB)} arrays. 
%, and (ii) the \emph{E2E} configuration, where we flush all cache lines that map to source and destination DRAM rows prior to rcc operations. 
%\juan{Figure~\ref{fig:system-copy-ini-speedup} shows...}


\begin{figure}[ht]
     \centering
     \vspace{-2mm}
         \begin{subfigure}[b]{0.47\textwidth}
         \centering
         \includegraphics[width=\textwidth]{figures/perf-absolute.pdf}
         \caption{{The execution time of \texttt{rcc} and \texttt{rci}}}
         \label{fig:system-copy-speedup2-abs}
     \end{subfigure}
     \hfill
     \begin{subfigure}[b]{0.47\textwidth}
         \centering
         \includegraphics[width=\textwidth]{figures/system-copy-init-speedup2.pdf}
         \caption{The increase in \texttt{rcc} and \texttt{rci}'s execution time}
         \label{fig:system-copy-speedup2}
     \end{subfigure}

     \vspace{-3mm}
     \caption{\texttt{rcc} and \texttt{rci} execution time (left) and increase in the execution time as array size increases (right)}
     \vspace{-5mm}
\end{figure}
}

\iffalse
\begin{figure}[h] %[!ht]
  \centering
  \includegraphics[width=.46\textwidth]{figures/system-copy-init-speedup2.pdf}
  \caption{\reva{The proportional increase in \texttt{rcc} and \texttt{rci}'s execution time for consecutive array sizes}%\jgl{Say throughput are bars, and increase in exec. time are lines. Otherwise, the reader can only guess that due to the proximity of each legend to the corresponding axis.}}
  .}
    %\jgl{Here it is "speedup", but it is "performance improvement" in Fig. 7. Be consistent. "Throughput improvement" is better, I think.}
  \label{fig:system-copy-speedup2}
\end{figure}
\fi

We make two major observations: 
First, \juan{\texttt{rcc}} improves the copy throughput by 58.3$\times{}$ for 8 KiB and by 118.5$\times{}$ for 8 MiB arrays, whereas \texttt{rci} improves initialization throughput by 31.4$\times{}$ for 8 KiB and by 88.7$\times{}$ for 8 MiB arrays.
%\jgl{Bad writing. Also, the numbers are swapped.}. 
\atb{Second, we observe that the throughput improvement provided by \texttt{rcc} and \texttt{rci} improves \emph{non-linearly} as the array size increases. The \newnew{\omu{execution time (in Rocket core clock cycles)} of} \texttt{rcc} and \texttt{rci} operations \omu{(not shown in Figure~\ref{fig:system-copy-speedup1})} \emph{does not increase linearly} with the array size. \omu{For example, the execution time of \texttt{rcc} is 397 and 584 cycles at 8 KiB and 16 KiB array sizes, respectively, resulting in a $1.47\times{}$ increase in execution time between 8 KiB and 16 KiB array sizes. However, the execution time of \texttt{rcc} is 92,656 and 187,335 cycles at 4 MiB and 8 MiB array sizes, respectively, resulting in a $2.02\times{}$ increase in execution time between 4 MiB and 8 MiB array sizes. We make similar observations on the execution time of \texttt{rci}.} For every RowClone operation, \texttt{rcc} and \texttt{rci} walk the page table to find the physical addresses corresponding to the source (\texttt{rcc}) and the destination (\texttt{rcc} and \texttt{rci}) operands. We attribute the non-linear increase in \texttt{rcc} and \texttt{rci}'s execution time to (i) the locality exploited by \newnew{the R}ocket core in accesses to the page table and (ii) the \revdel{proportionally }diminishing constant cost in \newnew{the} execution time \newnew{of both \texttt{rcc} and \texttt{rci}} due to common instructions executed to perform a system call.}


%Second, we observe that the speedups provided by rcc do not remain stable as array size nears 8 MiB (e.g., 175.5$\times{}$ at 4 MiB vs. 170.5$\times{}$ at 8 MiB for \emph{No Flush}), as opposed to our observation in the \emph{Bare-Metal} configuration. We attribute this to the additional pressure on the L1 cache and the TLB brought by rcc's accesses to the page table. We hypothesize that the memory access patterns of the page-table walk operation are more detrimental to performance at specific array sizes (e.g., 2, 8 MiB)  \jgl{These guesses sound quite random to me. If there is no intuition behind, better to not try to explain these fluctuations.}. Third, we observe that the \emph{No Flush} configuration performs 7.3$\times{}$ (8.4$\times{}$,4.3$\times{}$) better than the \emph{E2E} configuration on average (maximum,minimum) across all array sizes \jgl{Can you explain why such performance degradation in terms of bandwidth usage during the flush operation?}.

%\changev{\ref{q:r1q3}}
\subsubsection{CLFLUSH Overhead}
\label{sec:clflush-overhead}
%\todo[fancyline,linecolor=blue,backgroundcolor=blue!25,bordercolor=blue,size=\scriptsize]{\ref{q:r1q3}}
\Copy{R1/3}{We find that our implementation of CLFLUSH takes 45 \new{Rocket core} \omi{clock} cycles to flush a dirty cache block and 6 \new{Rocket core} cycles to \newnew{invalidate} a clean cache block. {We estimate the throughput improvement of \newnew{\texttt{rcc} and \texttt{rci}} including the CLFLUSH overhead. We assume that all cache blocks of the source \newnew{and destination} operand\newnew{s are} cached, and that a \newnew{fraction} of the \newnew{all cached cache blocks} is dirty (quantified on the x-axis). We do not include the overhead of accessing the data \omi{(e.g., by using \omi{\emph{load}} instructions)} \emph{after} \omi{the data} \newnew{gets copied in DRAM}.} Figure~\ref{fig:system-flush-overhead} shows the {estimated} improvement in \newnew{copy and initialization} throughput \newnew{that} \newnew{\texttt{rcc} and \texttt{rci}} \newnew{provide} for 8 MiB arrays.}

\begin{figure}[!h] %[!ht]
  \centering
  \includegraphics[width=.47\textwidth]{figures/system-flush-overhead.pdf}
  \caption{Throughput improvement provided by \texttt{rcc} and \texttt{rci} \revb{with CLFLUSH} over \omi{R}ocket's CPU-copy.}
  %\vspace{-4mm}
  \label{fig:system-flush-overhead}
\end{figure}

We make three major observations. \newnew{First, even with inefficient cache flush operations, \texttt{rcc} and \texttt{rci} provide 3.2$\times{}$ and 3.9$\times{}$ higher throughput over the CPU-copy and \omi{CPU}-initialization operations, assuming 50\% of the cache blocks of the 8 MiB source operand \omi{are} dirty, respectively.}\revdel{ \newnew{Second,} the copy and initialization throughput provided by \texttt{rcc} and \texttt{rci} are 9.4$\times{}$ and 6.1$\times{}$ smaller when we invalidate clean cache blocks of the source and destination operands of the copy/initialization operations, compared to the \emph{No Flush} configuration, where we assume all data is up-to-date in DRAM. The throughput improvement provided by \texttt{rcc} and \texttt{rci} is 12.6$\times{}$ and 14.6$\times{}$ for 8 MiB arrays over the CPU-copy/initialization baselines.} \newnew{Second,} as the \newnew{fraction} of dirty cache blocks \newnew{increase\omi{s}}, the throughput improvement \newnew{provided by both \texttt{rcc} and \texttt{rci}} \newnew{decreases (down to}\revdel{. The improvement in throughput provided by \texttt{rcc} and \texttt{rci} goes down to} 1.9$\times{}$ for \texttt{rcc} and 2.3$\times{}$ \newnew{for \texttt{rci} \omi{for 100\% dirty cache block \omu{fraction}})}. Third, we observe that \texttt{rci} can provide better throughput improvement compared to \texttt{rcc} when we \omi{include the CLFLUSH overhead. This is because} \texttt{rci} flushes cache blocks of one operand (destination), whereas \texttt{rcc} flushes cache blocks of both operands (source and destination). 

\Copy{R1/4}{{We do not study the distribution of dirty cache block \omi{fractions} in real application\omi{s} as \omi{that} is not the goal of our CLFLUSH overhead analysis. However, if a large dirty {cache block} \omi{fraction} causes severe overhead in a real application, the system designer \newnew{or the user of the system} would likely decide not to offload the operation to PuM (i.e., performing \newnew{\texttt{rcc}} operations instead of CPU-Copy). PiDRAM's prototype can be useful for studies on different PuM system integration aspects, including such offloading decisions.}}

\omi{We observe that the CLFLUSH operations are inefficient in supporting coherence for RowClone operations. Even so, we see that RowClone-Copy and RowClone-Initialization provides throughput improvements ranging from 1.9$\times{}$ to 14.6$\times{}$. We expect the throughput improvement benefits to increase as coherence between the CPU caches and PIM accelerators become more efficient with new techniques~\cite{boroumand2019conda,boroumand2016pim,seshadri2014dirty}.}

\subsubsection{{Real Workload Study}}
\label{sec:real-workload-study}
{The benefit of \texttt{rcc} and \texttt{rci} on a full application depends on what fraction of execution time is spent on bulk data copy and initialization. We demonstrate the benefit of \texttt{rcc} and \texttt{rci} on \emph{forkbench}~\cite{seshadri2013rowclone} and \emph{compile}~\cite{seshadri2013rowclone} workloads with varying fractions of time spent on bulk data copy and initialization, to show that our infrastructure can enable end-to-end execution and estimation of benefits \omi{on real workloads}.\footnote{A full workload study \omi{(i.e., with system calls to a full operating system such as Linux)} of \emph{forkbench} and \emph{compile} is out of the scope of this paper. Our infrastructure currently cannot execute all \omi{possible} workloads due to \omu{the} limited library and system call functionality provided by the \omi{RISC-V Proxy Kernel~\cite{riscv-pk}}.} We especially study \emph{forkbench} in detail to demonstrate how the benefits vary with the time spent on data copying in the baseline for this workload.}

{\emph{Forkbench} first allocates N memory pages and copies data to these pages from a buffer in the process's memory and then accesses 32K random cache blocks within the newly allocated pages to emulate a workload that frequently spawns new processes. We evaluate \emph{forkbench} under varying bulk data copy sizes where we sweep N from 8 to {2048} in increasing powers of two. 

\emph{Compile} first zero-allocates (\texttt{calloc} or \texttt{rci}) two pages (\omi{8 KiBs}) and then executes a number of arithmetic and memory instructions to operate on the zero-allocated data. We carefully develop the \emph{compile} microbenchmark to maintain a realistic ratio between the number of arithmetic and memory instructions executed and zero-allocation function calls made, which we obtain by profiling \emph{gcc}~\cite{perfLinux}. We use the \emph{No-Flush} configuration of our RowClone implementation \omi{for both \emph{forkbench} and \emph{compile}}.}

{Figure~\ref{fig:fork-all} plots the speedup provided by \texttt{rcc} over \omi{the} CPU-copy (bars, left y-axis) baseline, and the proportion of time spent on \texttt{memcpy} functions by the CPU-copy baseline (blue curve, right y-axis), for various configurations of \emph{forkbench} on the x-axis.}

\begin{figure}[!h]
    \centering
    \includegraphics[width=1.0\linewidth]{figures/fork-all.pdf}
    \caption{\emph{Forkbench} speedup (bars, left y-axis) and time spent on \texttt{memcpy} by the CPU baseline (curve, right y-axis)}
    \label{fig:fork-all}
\end{figure}

\noindent
{\textbf{Forkbench.}} {We observe that RowClone-Copy can significantly improve the performance of \emph{forkbench} by up to {42.9}\%. \omi{RowClone-Copy's} performance improvement increases as the number of pages copied increase. This is because the copy operations accelerated by \texttt{rcc} contribute a larger amount to the total execution time of the workload. The \texttt{memcpy} function calls take {86\%} of the CPU-copy baseline's time during \emph{forkbench} execution for N = {2048}.}

\noindent
{\textbf{Compile.}} {RowClone-Initialize \omi{improves} the performance of \emph{compile} by 9\%. Only an estimated 17\% of the execution time of \emph{compile} is used for zero-allocation by the CPU-initialization baseline, \texttt{rci} reduces the overhead of zero-allocation by \omi{(i)} performing \omi{in-DRAM} bulk-initialization and \omi{(ii)} executing a smaller number of instructions.}

\noindent
{\textbf{Libquantum.} To demonstrate that \X can run real workloads, we run a SPEC2006~\cite{spec2006} workload (libquantum). We modify the \texttt{calloc} (allocates and zero initializes memory) function call to allocate data using \texttt{alloc\_align}, and initialize data using \texttt{rci} for allocations that are larger than 8 KiBs.}

% Libquantum test takes 295041863 instructions
% Initializing 512 KiB is going to take 512 K instructions
% libqt spends .2 of its instructions on initialization, therefore the low perf. improvement.
\revcommon{Using \texttt{rci} to bulk initialize data in libquantum improves end-to-end application performance by 1.3\% (compared to the baseline that uses CPU-Initialization). This improvement is brought by \texttt{rci}, which initializes a total amount of 512 KiBs of memory\footnote{\revcommon{\omi{In libquantum,} there are 16 calls to \texttt{calloc} that exceed the 8 KiB allocation size. \omi{W}e only bulk initialize data using \texttt{rci} for these 16 calls.}} using RowClone operations.} {We note that the proportion of store instructions executed by libquantum to initialize arrays in the CPU-initialization baseline is only 0.2\% of all dynamic instructions in the libquantum workload {which amounts to an estimated 2.3\% of the total runtime of libquantum.} \omi{T}hus, the \omi{1.3\% end-to-end} performance improvement provided by \texttt{rci}, which can \omi{ideally} speed up only 2.3\% of \omu{the} total runtime, is reasonable, and we expect it to increase with the initialization intensity of workloads.}

\noindent
\omi{\textbf{Summary.}}
\revd{We conclude from our {evaluation} that end-to-end implementations of RowClone (i) can be \omi{efficiently} supported in real systems by employing memory allocation mechanisms that satisfy the memory \emph{alignment}, \emph{mapping}, \emph{granularity} requirements (Section~\ref{sec:rowclone_alignment}) of RowClone operations, (ii) can greatly improve copy/initialization throughput in real systems, and (iii) require cache coherenc\omi{e} mechanisms (e.g., PIM-optimized coherenc\omi{e} management~\cite{boroumand2019conda,boroumand2016pim,seshadri2014dirty}) that can flush dirty cache blocks of RowClone operands efficiently to achieve optimal copy/initialization throughput improvement.} {\X{} can be used \omi{to} estimat\omi{e} end-to-end \omi{workload} execution benefits provided by RowClone operations.} \revcommon{Our experiment\omi{s} using libquantum\omi{, forkbench, and compile} show that (i) \X can run real workloads, (ii) our end-to-end implementation of RowClone operates correctly, and (iii) RowClone can improve the performance of real workloads \omi{in a real system, even when inefficient CLFLUSH operations are used to maintain memory coherence}.}

\iffalse
\subsubsection{Projection Study\todo{better title}}

We calculate the copy throughput of RowClone operations to be no less than 542.5 GiB/s by tightly scheduling the DRAM commands required to perform consecutive RowClone operations. We find that on average, it takes (i) 9.5 cycles to flush a dirty cache block and (ii) 6.8 cycles to flush (invalidate) a clean cache block using \texttt{clflush\_opt}\todo{<-check}. We find that RowClone cannot provide an improvement in copy throughput even when all cache blocks of source and destination operands are clean. 
\fi

\iffalse
\subsection{Interface}
To enable RowClone end-to-end we expose it to the user via the memcpy C library function.

\subsubsection{Hardware Interface}

\subsubsection{Software Library}

\subsection{Data Mapping}

We offer two mechanisms to manage the alignment problem: (i) a passive management mechanism and (ii) an active management mechanism that builds on top of the passive mechanism. To summarize, the passive management mechanism refers to a scheme where the run-time system (e.g. OS) does not spend additional effort on aligning user data. The user explicitly asks the run-time system to align a number of arrays in DRAM. In the active mechanism the OS spends effort on aligning data structures such that they can be copied using RowClone. 

\subsubsection{Passive Management}

We assume that the OS provides an interface for the user to align data in the DRAM address space. This interface can be a function call cpyalign where arr is an array of pointers to the data structures that the user wants to align, and n is the number of bytes that are aligned. For example, the user calls cpyalign(16384, A, B) to align the first 16KBs of data from two arrays in memory such that they can be copied to each other using RowClone.

We discuss the internals of the copy align function in this paragraph. To implement copyalign the OS must have control over the two address translation layers: (i) virtual to physical and (ii) physical to DRAM device. We assume that the OS already has \textbf{full} control over the virtual to physical address translation layer. Contemporary CPUs implement a non-intuitive address translation layer in the memory controller~\cite{X,Y,Z} that translates from physical addresses to DRAM channel, rank, bank, row and column addresses. The OS must have some knowledge over this mapping to implement the cpyalign() function. Specifically, the OS must know which physical addresses correspond to which DRAM subarrays in the DRAM device. The OS must group physical addresses that map to the rows in a subarray together. The OS stores these groups in a table that is indexed by subarray ids. Each element in this table (named subphy table) is a set of physical addresses that point to DRAM rows in the same subarray (the indexed subarray). The OS needs to have \textbf{partial} knowledge (as described) over the physical to DRAM address translation layer to support cpyalign(). 

We describe how cpyalign() works over an example in this paragraph. The user calls cpyalign(16384, A, B). The OS splits the arrays into 8KB large chunks. Since there are two arrays, there are two arrays of chunks of size 2 after the split. The OS then traverses these two arrays and for the chunks that are at the same offset it maps two physical addresses from a single entry of the subphy table to the virtual addresses of these chunks. The OS repeats this for all chunks.

The vendors can provide the OS with the subphy table contents, i.e. the physical addresses that map to the same subarrays. In the worst case, the OS can reverse engineer these mappings using RowClone once over a lifetime and store these mappings in a persistent storage device. Finally, existing works show that contemporary memory controller mappings can be decoded~\cite{X, Y}. This is not required since the OS does not need this mapping information at such fine granularity to implement cpyalign().

\subsubsection{Active Management}

The active management mechanism is built on top of the passive management mechanism as it requires the OS to provide the same functionality (the cpyalign() function). However, in this mechanism the user does not explicitly call the cpyalign() function. Depending on program behavior, the OS dynamically aligns data structures in DRAM to optimize for bulk-copy performance.

We discuss how the active management can be implemented in this paragraph.

\subsubsection{Effects of Cache Flushing}

We analyze the overhead of cache management operations that are required for correct RowClone operation as described in Section~\ref{sec:coherency}. We run a microbenchmark that uses RowClone-Copy to copy arrays with sizes ranging from 8KBs to 256MBs. Figure~\ref{fig:flush_vs_noflush} depicts the performance loss when we maintain coherency by using CLFLUSH instructions for differently sized source and destination operands for the RowClone operation. We study the effects of maintaining coherency on performance under two scenarios: (i) where we execute a CLFLUSH operation for each cache block in both source and destination operands, (ii) where we execute as many CLFLUSH operations required to evict all cache blocks in CPU caches. For the second scenario we assume 8MB large CPU caches. We compare the performance of these two scenarios against an ideal baseline where coherency is maintained with no cost, i.e. no cache flush operations are needed to maintain coherency. We observe that the cost of maintaining coherency goes up to 20X for an array size of \textcolor{red}{Y}. The performance compared to the baseline continuously worsens for the first scenario. This is due to ... For the second scenario, ... 

We conclude that in order to efficiently support RowClone operations, a system needs to employ more sophisticated coherency mechanisms~\cite{X,Y,Z} that reduce cache flush operations as much as possible. We leave the evaluation of such sophisticated mechanisms to future work.

\begin{figure*}[!th]
  \centering
  \includegraphics[width=1.0\textwidth]{figures/PLACEHOLDER_flush_vs_noflush.pdf}
  \caption{\textcolor{red}{PLACEHOLDER} Relative performance of RowClone when we maintain coherency via CLFLUSH instructions compared to the ideal case where coherency is not a problem (higher is better).}
  \label{fig:flush_vs_noflush}
\end{figure*}
\fi


%Implementing support for RowClone on PiDRAM requires modifications across the whole HW/SW stack. We modify our custom memory controller to implement the carefully-engineered valid DRAM command sequence~\cite{computedram} that is required to perform in-DRAM copy operations. This in-DRAM copy support is exposed to the programmer via pidlib over system calls. To overcome the mapping, alignment and granularity problems, we implement our memory management mechanism (Section~\ref{sec:memory-management}) in RISC-V PK. We expose our mechanism over the \emph{alloc\_align} system call, which enables programmers to allocate data in a way that satisfies in-DRAM copy mapping and alignment requirements. 


%\textbf{Initializer Rows Table (IRT).} The IRT holds physical addresses of the DRAM rows that are initialized with zeros. The IRT is indexed with physical page numbers. RowCopy-Initialize operations obtain the physical address of the DRAM row that is initialized with zeros which belongs in the same subarray as the operand of the initialization operation from the IRT. 

%\textbf{Allocation ID Table (AIT). }

\iffalse
We expose RowClone-Copy (RCC) and RowClone-Initialize (RCI) over two system calls: (i) \emph{RCC(src*,dest*,n)}


Figure~\ref{fig:memory-management-mechanism} shows the organization of RowClone-enabler structures and depicts how the memory management mechanism works over an end-to-end example for a 32 KB copy (i.e., we allocate two 32 KB arrays and copy one to another) operation.


\outline{How we implement RowClone on PiDRAM at a high level.}

We implement \emph{rcc} and \emph{rci} system calls which expose the RowClone-Copy and the RowClone-Initialize primitives to the programmer respectively. Table~\ref{table:pidlib-syscalls} shows the semantics of the three functions.

Figure~\ref{fig:rcc-e2e} depicts the end-to-end execution of a RowClone-Copy operation. First, the program calls the \emph{alloc\_align} function to allocate two 32 KB large arrays (A and B) that can be copied to each other using RowClone. Second, the program modifies array A by performing regular memory store operations. Third, the program calls the \emph{rcc} function to copy 32 KBs starting from A to B. Fourth, \emph{rcc} traverses the page table to find physical addresses that map to the virtual pages of A and B. Fifth, \emph{rcc} flushes all cache blocks that maps to the DRAM rows indicated by these physical addresses. Finally, \emph{rcc} performs the in-DRAM copy operations via the IDO controller.

Currently, RISC-V does not implement any cache maintenance operations. RISC-V memory model specification~\cite{X} states that the system designer is responsible for maintaining coherency. To maintain coherency, we implement a custom RISC-V cache flush instruction, CLFLUSH, in our infrastructure. CLFLUSH flushes a \textit{physically addressed} cache block. We modify the pipeline in the non blocking data cache and the \omi{R}ocket core modules (defined in \textit{NBDCache.scala} and \textit{rocket.scala} in \omi{R}ocket \omi{C}hip~\cite{X} respectively) to implement CLFLUSH, and we modify the RISC-V GNU compiler toolchain~\cite{X} to expose CLFLUSH as an instruction to C/C++ applications. We use the CLFLUSH instruction to flush dirty cache blocks that map to source rows and invalidate cache blocks that map to destination rows in RowClone operations.


\subsubsection{RowClone-Copy and RowClone-Initialize}
\todo{Pseudo-code for the algorithm}
\fi

\iffalse
Figure~\ref{fig:system-init-speedup} depicts the improvement in execution time for the same configurations (\emph{No Flush} and \emph{E2E}) for rci. 

\begin{figure}[h] %[!ht]
  \centering
  \includegraphics[width=.49\textwidth]{figures/system-init-speedup.pdf}
  \caption{}
  \label{fig:system-init-speedup}
\end{figure}


We make similar observations \juan{to those for \texttt{rcc} (Figure~\ref{fig:system-copy-speedup})}. 
First, rcc improves the execution time of initialization operations by 31.0$\times{}$ \textendash{} 11.5$\times{}$ for 8 KiB and 94.1$\times{}$ \textendash{} 15.5$\times{}$ for 8 MiB array sizes for E2E \textendash{} No Flush configurations \jgl{Swap}. 
Second, we observe a similar performance improvement behavior provided by rci and rcc routines, as array size increases. This observation is attributed to the additional pressure on caches brought by rci's accesses to the page table, and to the IRT (Section~\ref{X}). 
Third, rci under \emph{No Flush} configuration improves initialization performance by 5.2$\times{}$ (6.1$\times{}$,2.7$\times{}$) compared to the \emph{E2E} configuration on average (maximum,minimum) across all array sizes. 
Finally, we observe that the relative performance improvement provided by rci is smaller to rcc due to two reasons: (i) initialization operations require executing a single instruction (e.g., store) on the CPU to initialize data word by word, copy operations require two (one load, one store), resulting in rci's baseline to perform better than rcc's baseline (ii) rci routine has to access the IRT to find the DRAM row that is used as the source operand for the RowClone operation.
\jgl{The observations for both figures are very similar. You can eventually merge them if space needed.}
\fi


%First, the overhead due to the system calls (page table traversal, accessing SAMT) decreases with increasing array sizes. This overhead is the smallest as 18\% at 1MB array size. We find that this overhead is correlated with \todo{cache hit rates}, as the \emph{rci} and \emph{rcc} system calls access many RowClone-enabler structures along with the page table that can benefit from caching. Second, the overhead introduced by CLFLUSHs increase as array size increases, achieving a maximum overhead in execution time of 8.84$\times{}$ at 8 MB array size. \todo{similar explanation as the one for the behavior in Figure 1}

%\todo{Where are the bottlenecks? How long does it take to perform copy operation, how long it takes to flush the data. More insight into how these components affect the performance. Performance assuming everything in (i) cache, (ii) DRAM. }

\iffalse
\begin{figure}[!ht]
  \centering
  \includegraphics[width=.49\textwidth]{figures/system-flush-overhead.pdf}
  \caption{\todo{Change vert. axis label} Degradation in execution time introduced by system support and CLFLUSH instructions for RowClone. Baseline is in-DRAM copy. Y-axis shows the degradation in performance (increase in execution time) proportional to the baseline, X-axis shows the array sizes ranging from 8 KBs to 8 MBs.}
  \label{fig:system-xxxxxxxx}
\end{figure}
\fi

\iffalse
Figure~\ref{fig:perf-improvement}-c shows the improvement in execution time when using in-DRAM copy to initialize arrays over CPU-copy baseline with size ranging from 8 KBs to 8 MBs. We make two observations: (i) CPU-copy performs better than \emph{rci} for DRAM row-size (8 KB) initialization operations. We find that at 8KB initialization size, system call's contribution to the overall overhead of \emph{rci} is proportionally larger than that of it's contribution at larger array sizes. \todo{We observe that at 8KBs, accessing the page table and pidlib structures take more time because memory accesses to these structures miss in the D\$ (i.e. the structures are not warmed-up).} (ii) \emph{rci} improves execution time by 26.6\%. This improvement saturates as the array size increases.

\iffalse
\begin{figure}[!ht]
  \centering
  \includegraphics[width=0.49\textwidth]{figures/overall-performance.pdf}
  \caption{}
  \label{fig:overall-performance}
\end{figure}
\fi

\outline{Describe results.
\begin{itemize}
    \item RowClone overheads breakdown: syscalls, address translation, flush.
    \item Mapping \& alignment overheads: subarray characterization, copy align overhead.
    \item Other results: hardware area, memory overhead of sw components, verilog + chisel lines of code.
\end{itemize}}

\outline{Discuss the challenges in implementing RowClone.}

Figure~\ref{fig:rci-perf-breakdown} depicts the breakdown of execution time for the three key components of a RowClone operation: (i) system call, (ii) flush and (iii) in-DRAM copy, for array sizes ranging from 8 KB to 8 MB. We observe that CLFLUSH contributes most to the execution time at all array sizes, taking up to 80.8\% of execution time for an 8 MB in-DRAM copy operation.
% Stale from here on

\begin{figure}[!ht]
  \centering
  \includegraphics[width=0.49\textwidth]{figures/rci-perf-breakdown.pdf}
  \caption{Breakdown of execution time for key components of RowClone operations.}
  \label{fig:rci-perf-breakdown}
\end{figure}
\fi



\section{Case Study \#2: End-to-end D-RaNGe}


\label{sec:drange}

%\outline{reintroduce drange}
{Prior work on DRAM-based random number generation techniques~\cite{olgun2021quactrng,kim.hpca19,talukder2019exploiting} do not integrate and evaluate their techniques end-to-end in a real system.}
\revdel{D-RaNGe~\cite{kim.hpca19} (Section~\ref{sec:background_pudram}) {is a state-of-the-art} DRAM-based true random number generat{ion technique} that leverages the randomness in DRAM activation latency ($tRCD$) failures.} \new{We evaluate one DRAM-based true random number generation technique, D-RaNGe~\cite{kim.hpca19}, end-to-end using PiDRAM.} We implement support for D-RaNGe in PiDRAM by enabling \omi{access to DRAM with} \new{reduced activation latency} (i.e., \omi{$tRCD$ set to values lower than} manufacturer recommendations).

\subsection{D-RaNGe Implementation}

We implement a simple version of D-RaNGe in PiDRAM. PiDRAM's D-RaNGe \omi{controller} collects true random numbers from four DRAM cells in the same DRAM cache block %\footnote{D-RaNGe observes that there can be as many as four TRNG cells in a DRAM cache block} 
\omi{inside} one DRAM bank. 
%These random numbers are put into a 1 KiB buffer in the custom memory controller.
%Application developers can modify the rate at which true random numbers are generated via pumolib. This enables calibrating PiDRAM's TRNG according to workloads' random number requirements, i.e., if the workload requires a high-throughput stream of random numbers, the programmer can increase the rate at the expanse of increased interference with other memory requests.
%\todo{insert figure.}
We implement the \omi{D-RaNGe controller} within the \revf{Periodic Operations Module \new{(Section~\ref{sec:hardware-components})}}. The \omi{D-RaNGe controller} (i) periodically accesses a DRAM cache block with reduced tRCD, (ii) reads four of the TRNG DRAM cells in the cache block, (iii) stores the four bits read from the TRNG cells in a 1 KiB \new{random number buffer}. We reserve multiple configuration registers in the configuration register file (CRF) to configure (i) the \omi{TRNG period (in nanoseconds) used by} the \omi{D-RaNGe controller} to periodically generate random numbers \omi{by accessing DRAM with reduced activation latency} while the buffer is not full (the D-RaNGe controller accesses DRAM every TRNG period), (ii) the timing parameter ($tRCD$) used \omi{to} induc\omi{e} activation latency failures and (iii) the \new{physical} location \new{(DRAM bank, row, column addresses, and bit offset within the DRAM column)} of the TRNG cells \omi{to read}. We implement two pumolib functions: (i) \texttt{buf\_\omi{size}()}, which returns the number of random words (4-bytes) available in the buffer, and (ii) \texttt{rand\_dram()}, which returns one random word that is read from the buffer. {The two functions \new{first} execute PiDRAM instructions in the POC that \new{update} the data register either with (i) the number of random words available \new{(when buf\_s\omi{ize}() is called)} or (ii) a random word read from the random number buffer \new{(when rand\_dram() is called)}. The two functions \new{then} access the data register using LOAD instructions to retrieve \omi{either} the size of the random number buffer or a random number.} The application developer reads true random numbers using these two functions in pumolib.

\noindent
\textbf{\new{Random Cell Characterization.}} D-RaNGe requires the system designer to characterize the DRAM module for activation latency failures to find DRAM cells that fail with a 50\% probability \omi{(i.e., randomly)} when accessed with reduced $tRCD$. \revd{Following the methodology presented in~\cite{kim.hpca19}, the system designer can characterize a DRAM device or use an automated procedure to find cells that fail with a 50\% probability.} \omi{In PiDRAM,} we implement reduced latency access to DRAM by (i) extending the scheduler of the custom memory controller and (ii) adding a pumolib function \texttt{\omi{activation\_failure}(\omi{address})} which induces an activation failure on the DRAM cache block pointed by the \texttt{\omi{address}} parameter.

\subsection{Evaluation and Results}
\label{sec:drange-evaluation}
\textbf{Experimental Methodology.} We run a microbenchmark to understand the effect of the TRNG period on true random number \omi{generation} throughput observed by a program running on the \omi{R}ocket core. The microbenchmark consists of a loop that (i) checks the availability of random numbers using \texttt{buf\_\omi{size}()} and (ii) reads a random number from the buffer using \texttt{rand\_dram()}. \newnew{We execute the microbenchmark until we read one million bytes of random numbers.} 

\begin{figure}[!ht]
  \centering
  \vspace{-5mm}
  \includegraphics[width=0.40\textwidth]{figures/rng-throughput.pdf}
  \vspace{-3mm}
  \caption{TRNG throughput observed by our microbenchmark for TRNG periods ranging from 220 $ns$ to 1000 $ns$}
  \label{fig:trng-throughput}
\end{figure}

\noindent
\textbf{Results.} \omi{The D-RaNGe controller} can perform reduced-latency accesses frequently, every 220 $ns$. Figure~\ref{fig:trng-throughput} depicts the TRNG throughput observed by the microbenchmark for TRNG periods in the range [220 $ns$, 1000 $ns$] with increments of 10 $ns$. We observe that the TRNG throughput decreases from 8.30 Mb/s at 220 $ns$ TRNG period to 1.90 Mb/s at 1000 $ns$ TRNG period. D-RaNGe~\cite{kim.hpca19} reports 25.2 Mb/s TRNG throughput using a single DRAM bank when there are \new{four} random cells in a cache block. PiDRAM's \omi{D-RaNGe controller} can be optimized to generate random numbers more frequently to match D-RaNGe's observed maximum throughput.\footnote{\omi{D-RaNGe has a smaller true random number generation (TRNG) latency (i.e., takes a smaller amount of time to generate a 4-bit random number) than PiDRAM. PiDRAM has a larger TRNG latency due to (i) discrepancies in the data path (i.e., on-chip interconnect) in D-RaNGe's simulated system and PiDRAM's prototype and (ii) \omu{the TRNG period of the D-RaNGe controller (D-RaNGe controller performs a reduced $tRCD$ access only as frequently as one every \SI{220}{\nano\second}).} \omu{The D-RaNGe controller} can be optimized further to reduce the TRNG period \omu{by} down to the DRAM row cycle time ($tRC$ standard timing parameter, typically ~\SI{45}{\nano\second}~\cite{micron2018ddr3}).}} We leave such optimizations to PiDRAM's \omi{D-RaNGe controller} for future work.

%We make two observations.  We observe that the TRNG throughput starts deteriorating when the TRNG period hits 2400 $ns$. \todo{double check results and finalize observations.}

Including the modifications to the custom memory controller and pumolib, implementing D-RaNGe and reduced-latency DRAM access \new{requires} an additional \textbf{190} lines of Verilog and \textbf{74} lines of C code over \X's existing codebase. We conclude that our D-RaNGe implementation (i) provides a basis for \X developers to study end-to-end implementations of \omi{DRAM-based true random number generators}, (ii) shows that \X's hardware and software components facilitate the implementation of new {commodity DRAM based} PuM techniques\omi{, specifically those that are related to security}. Our reduced-latency DRAM access implementation provides a basis for other PuM techniques \omi{for security purposes}, such as \omi{the DRAM-latency physical unclonable functions (}DL-PUF~\cite{kim.hpca18}) \omi{and QUAC-TRNG~\cite{olgun2021quactrngieee} (Section~\ref{sec:use-cases})}. We leave further exploration on end-to-end implementations of D-RaNGe\omi{, DL-PUF, and QUAC-TRNG, as well as end-to-end analyses of \omu{the} security benefits they provide using PiDRAM} for future work.

%Algorithm~\ref{alg:trng_microbenchmark} describes our microbenchmark. Line 2 sets the TRNG period of PiDRAM's TRNG controller. Line 4 loops until the random number buffer contains a valid random number. Line 4 reads a 16-bit true random number from the buffer. Line 3 loops for 512K times such that the microbenchmark reads 1 MBs of true random numbers from the buffer.

\iffalse
\begin{algorithm}[tbh]\footnotesize
        \SetAlgoNlRelativeSize{1.0}
        %\SetAlgoNoLine
        \DontPrintSemicolon
        %\SetAlCapHSkip{0pt}
        \caption{TRNG Microbenchmark}
        \label{alg:trng_microbenchmark}
        for ($period = 100ns$ ; $period$ <= $10000ns$ ; $period$ += $100ns$)\par 
        ~~~~configure\_trng($period$) \par
        ~~~~for ($i = 0$ ; $i$ < 512K ; $i$++)\par
        ~~~~~~~~while ($random\_buffer\_size == 0$); \par
        ~~~~~~~~$read\_random\_number()$; \par
    \end{algorithm}
\fi


%\section{Limitations of \X}
%\section{Limitations}
The large widths of the CIs in \S\ref{sec:ci_experiments} and the lack of some statistically significant differences between metrics in \S\ref{sec:hypo_experiments} are directly tied to the size of the datasets that were used in our analyses.
However, to the best of our knowledge, the datasets we used are some of the largest available with annotations of summary quality.
Therefore, the results presented here are our best efforts at accurately measuring the metrics' performances with the data available.
If we had access to larger datasets with more summaries labeled across more systems, we suspect that the scores of the human annotators and automatic metrics would stabilize to the point where the CI widths would narrow and it would be easier to find significant differences between metrics.

Although it is desirable to have larger datasets, collecting them is difficult because obtaining human annotations of summary quality is expensive and prone to noise.
Some studies report having difficulty obtaining high-quality judgments from crowdworkers \citep{GillickLi10,FKMSR21}, whereas others have been successful using the crowdsourced Lightweight Pyramid Score \citep{SGGRPBAD19}, which was used in \citet{BGALN20}.

Then, it is unclear how well our experiments' conclusions will generalize to other datasets with different properties, such as documents coming from different domains or different length summaries.
The experiments in \citet{BGALN20} show that metric performance depends on which dataset you use to evaluate, whether it be TAC or CNN/DM, which is supported by our results.
However, our experiments also show variability in performance within the same dataset when using different quality annotations (see the differences in results between \citet{FKMSR21} and \cite{BGALN20}).
Clearly, more research needs to be done to understand how much of these changes in performance is due to differences in the properties of the input documents and summaries versus how the summaries were annotated.

\section{{Extending PiDRAM}}
\label{sec:extending-pidram}
{We briefly describe the modifications required to extend PiDRAM (i) with new DRAM commands and DRAM timing parameters\new{, (ii) with new case studies,} and (i\new{i}i) to support new FPGA boards.}

\noindent
\textbf{{New DRAM Commands and Timing Parameters.}}
\Copy{R1/1}{{Implementing new DRAM commands or modifying DRAM timing parameters require modifications to PiDRAM's memory controller. This is straightforward as PiDRAM's memory controller's Verilog design \omi{is modular and uses well-defined interfaces: It is composed} of multiple modules that perform separate tasks\revdel{and communicate with each other via well-defined interfaces}. For example, the memory request scheduler comprises \new{two} main components: (1) \emph{command timer}, \new{and} (2) \emph{command scheduler}\new{.}\revdel{, and (3) custom command scheduler} To serve {LOAD and STORE memory} requests, the command scheduler maintains state (e.g., which row is active) for every bank. The command scheduler selects the next DRAM command to satisfy the {LOAD or STORE memory} request and queries the command timer with the selected DRAM command. The command timer checks for all possible standard DRAM timing constraints and outputs a valid bit if the selected command can be issued in that FPGA clock cycle. To extend the memory controller with a new standard DRAM command (e.g., to implement a newer standard like DDR4 \omi{or DDR5}), a PiDRAM developer simply needs to \omi{(i)} add a new timing constraint by replicating the logic in the command timer and \omi{(ii)} extend the command scheduler to correctly maintain the bank state.}}

\noindent
{\textbf{New Case Studies.}}
\Copy{R5/2}{{Implementing new techniques (e.g., those that are listed in Table~\ref{table:use-cases}) to perform new case studies requires modifications to PiDRAM's hardware and software components. We describe the required modifications over an example {ComputeDRAM-based in-DRAM bitwise operations} case study.}

{To implement ComputeDRAM-based in-DRAM bitwise operations, the developers need to (i) extend the \emph{\new{custom command} scheduler} in PiDRAM's memory controller with a new state machine that schedules new DRAM command sequences (ACT-PRE-ACT) with an appropriate set of violated timing parameters (our ComputeDRAM-based in-DRAM copy implementation provides a \omi{solid} basis for this), (ii) expose the functionality to the processor by implementing new PiDRAM instructions in the PuM controller (e.g., by replicating \omi{and customizing} the existing logic for decoding and executing RowClone operations), (iii) and make modifications to the software library to expose the new instruction to the programmer (e.g., by replicating the copy\_row function's behavior, described in Table~\ref{table:pumolib}).}}

\noindent
\textbf{{Porting to New FPGA Boards.}}
\Copy{R3/7A}{{Developing new PiDRAM prototypes on different FPGA boards could require modifications to design constraints (e.g., top level input/outputs to physical FPGA pins) and the DDRx PHY IP depending on the FPGA board. Modifying design constraints is a straightforward task involving looking up the FPGA manufacturer datasheets and modifying design constraint files~\cite{designconstraints}. Manufacturers may provide different DDRx PHY IPs for different FPGAs. Fortunately, these IPs typically expose \omi{similar} (based on the DFI standard~\cite{dfi}) interface\omi{s} to user hardware (in our case, to PiDRAM's memory controller). Thus, \onur{other} PiDRAM prototypes on different FPGA boards can be developed with \omi{small yet careful} modifications to the ZC706 prototype design we provide.}}



\section{Related Work}
\label{sec:related-work}

To our knowledge, this is the first work to develop a flexible, open-source framework that enables integration and evaluation of \new{commodity DRAM based} \omi{processing-using-memory} (PuM) techniques on real DRAM chips by providing the necessary hardware and software components. %We design a memory allocation mechanism that satisfies the memory management requirements (Section~\ref{sec:rowclone}) of RowClone operations. 
We demonstrate the first end-to-end implementation of RowClone \omi{and D-RaNGe using real DRAM chips}. We compare the features of \X with other state-of-the-art \omi{prototyping and evaluation} platforms in Table~\ref{table:tools} and discuss them below. \omi{The four features we use for comparison are:} 

\begin{enumerate}
    \item \textbf{Interface with real DRAM chips:} The platform allows running experiments using real DRAM chips.
    \item \textbf{Flexible memory controller (MC) for PuM:} The platform provides a flexible memory controller that can easily be extended to perform (e.g., as in PiDRAM) or emulate (e.g., as in PiMulator~\cite{mosanu2022pimulator}) new PuM operations.
    \item \textbf{System software support:} The platform provides support for running system software such as operating systems or supervisor software (e.g., RISC-V PK~\cite{riscv-pk}).
    \item \textbf{Open-source:} The platform is available as open source software.
\end{enumerate}

\begin{table*}[!t]
\centering
\caption{Compari\omi{son of} \X with related state-of-the-art \omi{prototyping and evaluation} platforms}
\label{table:tools}
\resizebox{1.0\textwidth}{!}{\begin{tabular}{|l||c|c|c|c|}

\hline
           \textbf{Platforms} & \begin{tabular}[c]{@{}l@{}}\textbf{Interface with \omi{real} DRAM \omi{chips}}\end{tabular} & \begin{tabular}[c]{@{}l@{}}\textbf{Flexible MC for PuM}\end{tabular} & \begin{tabular}[c]{@{}l@{}}
           \textbf{System software support}\end{tabular}  & \begin{tabular}[c]{@{}l@{}}\textbf{Open-source}\end{tabular}  \\ \hline \hline
\textbf{Silent-PIM} \cite{kim2021SilentPIM} & \xmark & \xmark & \checkmark  & \xmark\\ \hline
\textbf{SoftMC}  \cite{hassan2017softmc}   & \checkmark (DDR3) &  \xmark  &  \xmark &  \checkmark\\ \hline
\textbf{ComputeDRAM}   \cite{gao2020computedram}    & \checkmark (DDR3) & \xmark & \xmark   & \xmark \\ \hline
\textbf{MEG}    \cite{zhang2020MEG}    & \checkmark (HBM) &  \xmark  & \checkmark  &   \checkmark  \\ \hline
{\textbf{PiMulator}}    \cite{mosanu2022pimulator}    & \xmark &  \checkmark  & \xmark  &   \checkmark  \\ \hline
\textbf{\omi{Commercial platforms (e.g.,} ZYNQ~\cite{zynq})} & \checkmark (DDR3/4) &  \xmark    & \checkmark   &  \xmark \\ \hline
\textbf{Simulators}  \cite{gem5-gpu,GEM5,ramulator,ramulator-pim,zhang2022pim,forlin2022sim2pim,yu2021multipim,xu2019pimsim} & \xmark &  \checkmark  &  \checkmark \omi{(potentially)}   & \checkmark  \\ \hline
\hline
\textbf{\X  (this work)}   & \checkmark (DDR3) &  \checkmark   &  \checkmark &  \checkmark  \\ \hline
\end{tabular}}

\end{table*}
%Several hardware platforms and system simulators provide different sets of facilities in using DRAM as the main memory. However, none of the existing platforms support PuM mechanisms. We compare the features of \X with the state-of-the-art platforms in Table \ref{table:tools} and discuss them below:  

\atb{\textbf{Silent-PIM~\cite{kim2021SilentPIM}.} Silent-PIM proposes a new DRAM \new{design} \revdel{(similar to FIMDRAM~\cite{kwon2021fimdram}) }that incorporates processing units capable of vector arithmetic computation. \revdel{The authors use the standard DRAM interface~\cite{jedecDDR4} to communicate with the DRAM device. }Silent-PIM's goal is to evaluate PIM techniques on a \emph{new, PIM-capable} DRAM device using standard DRAM commands \omi{(e.g., as defined in DDR4~\cite{jedecDDR4})}\omi{; it does not \omu{provide} an evaluation platform or prototype}. \atb{\new{In contrast}, \X is designed for researchers to rapidly integrate and evaluate PuM techniques that use \emph{real DRAM devices}. \X provides key hardware and software components that facilitate end-to-end implementations of PuM techniques.}}

\revb{\textbf{SoftMC~\cite{hassan2017softmc, softmc.github}} SoftMC is an FPGA-based DRAM testing infrastructure. SoftMC \revb{can issue} arbitrary sequences of DDR3 commands to real DRAM modules. {SoftMC is widely used in prior work that \omi{studies} the performance, reliability and security of real DRAM chips~\cite{frigo2020trr, hassan2021uncovering, kim2020revisiting, khan.micro17, chang.hpca16, talukder2020towards, farmani2021RHAT, kim.hpca19, talukder2019exploiting, lee.sigmetrics17, ghose2018vampire, orosaYaglikci2021deeper,talukder2019prelatpuf}}. \Copy{R5/2B}{SoftMC is built to test DRAM modules, \emph{not} to study end-to-end implementations of PuM techniques. \new{Thus,} SoftMC (i) does \emph{not} \omi{support application execution on a real system}, and (ii) \revb{\emph{cannot} use} DRAM modules as \revb{{main memory}}. {While SoftMC is useful in studies that perform exhaustive search on all possible sequences of DRAM commands to potentially uncover undocumented DRAM behavior (e.g., ComputeDRAM~\cite{gao2020computedram}, QUAC-TRNG~\cite{olgun2021quactrng}),} \X is developed to study end-to-end implementations of PuM techniques. \X provides an FPGA-based prototype that comprises a RISC-V system and supports using DRAM modules both for storing data \omi{(i.e., as main memory)} and \omi{performing PuM} computation.}}

\textbf{\atb{ComputeDRAM~\cite{gao2020computedram}.}} ComputeDRAM partially demonstrates \omi{that} two DRAM-based state-of-the-art PuM techniques\new{,} RowClone~\cite{seshadri2013rowclone} and Ambit~\cite{seshadri.micro17}\new{, \omi{are already possible} on real \omi{off-the-shelf} DDR3 chips}. ComputeDRAM uses SoftMC to demonstrate in-DRAM copy and bitwise \omi{AND/OR} operations on real DDR3 chips. ComputeDRAM's goal is \emph{not} to develop a framework to facilitate end-to-end implementations of PuM techniques. Therefore, it does \emph{not} provide (i) a flexible memory controller for PuM \omi{or}, (ii) support for system software. \X provides the necessary software and hardware components to facilitate end-to-end implementations of PuM techniques.

\textbf{MEG~\cite{zhang2020MEG}.} MEG is an open-source system emulation platform for enabling FPGA-based operation interfacing with High-Bandwidth Memory (HBM). MEG aims to efficiently retrieve data from HBM and perform the computation in the host processor implemented as a soft core on the FPGA. Unlike \X, MEG does \emph{not} implement a flexible memory controller that is capable of performing PuM operations. We demonstrate the flexibility of \X by implementing two state-of-the-art PuM techniques~\cite{seshadri2013rowclone,kim.hpca19}. \omi{We believe MEG and PiDRAM can be combined to get the functionality and prototyping power of both works.}

{\textbf{PiMulator~\cite{mosanu2022pimulator}.} PiMulator is an open-source PiM emulation platform. PiMulator implements a main memory and a PiM model using SystemVerilog, allowing FPGA emulation of PiM architectures. PiMulator enables easy emulation of new PiM techniques. However, it does \emph{not} allow end-to-end execution of workloads that use PiM techniques and it does not provide the user \omi{with} full control over the DRAM interface.}

\textbf{Commercial Platforms (e.g., ZYNQ \cite{zynq}).} Some commercial platforms implement CPU-FPGA heterogeneous computing systems. A memory controller and necessary hardware-software modules are provided to access DRAM as the main memory in such systems. However, in such systems, (i) there is \emph{no} support for PuM mechanisms\new{, and} (ii) the entire hardware-software stack is closed-source. \X can be integrated into these systems, using the closed-source computing system as the main processor. Our prototype utilizes an open-source system-on-chip (\omi{Rocket Chip}~\cite{asanovic2016rocket}) as the main processor, which enables developers to study architectural \omi{and microarchitectur\omu{al}} aspects of PuM techniques (e.g., \omi{data allocation and} coherence mechanisms). Such studies cannot be conducted \omi{using} closed-source computing systems.

\textbf{Simulators.} Many prior works propose full-system \omu{(e.g.,~\cite{GEM5,gem5-gpu})}, trace-based \omu{(e.g.,~\cite{ramulator-pim,ramulator,zhang2022pim,yu2021multipim,xu2019pimsim,scarab})}, and instrumentation-based \omu{(e.g.,~\cite{scarab,forlin2022sim2pim,xu2019pimsim})} simulators that can be used to evaluate PuM techniques. Although useful, these simulators do not model DRAM behavior and cannot integrate proprietary device characteristics (e.g., DRAM internal address mapping) into their simulations, \atb{without conducting a rigorous characterization study. Moreover, the effects of environmental conditions (e.g., temperature, voltage) on DRAM chips are unlikely to be modeled on \revcommon{accurate, full-system} simulators as it would require excessive computation, \revcommon{which} would \revcommon{negatively impact the} \revcommon{already poor} performance \revcommon{(200K instructions per second)} of \revcommon{full system simulators} ~\cite{zsim}}. \atb{\omi{In contrast}, \X interfaces with real DRAM devices \revcommon{and its prototype \omi{achieves} a 50 MHz clock speed (\omi{and} can be improved \omi{further}) which lets \X execute > 10M instructions per second (assuming < 5 cycles per instruction)}. \Copy{R3/5}{\X can be used to study end-to-end implementations of PuM techniques and explore solutions that take into account \revcommon{the effects related to the environmental conditions of real DRAM devices. {Future versions of \X{} could be easily extended (e.g., with real hardware that allows controlling DRAM temperature and voltage~\cite{pl068p,maxwellFT200}) to experiment with different DRAM temperature and voltage levels to better understand the effects of these environmental conditions on the reliability of PuM operations.} Using \X, experiments that require executing real workloads can take an order of magnitude \omi{shorter} wall clock time compared to using full-system simulators.}}}

{\textbf{Other Related Work.} \omi{Prior works (see Section~\ref{sec:background_pudram}) (i) propose or (ii) demonstrate using real DRAM chips, several DRAM-based PuM techniques that can perform computation~\cite{Seshadri:2015:ANDOR, seshadri.arxiv16, seshadri.micro17, seshadri.bookchapter17, seshadri2020indram,hajinazarsimdram,chang.hpca16,ferreira2021pluto,angizi2019graphide}, move data~\cite{seshadri2013rowclone,wang2020figaro}, or implement security primitives~\cite{olgun2021quactrngieee,kim.hpca18,kim.hpca19,talukder2019exploiting,talukder2019prelatpuf,orosa2021codic} in memory.}\revdel{{Seshadri et al.~\cite{Seshadri:2015:ANDOR, seshadri.arxiv16, seshadri.micro17, seshadri.bookchapter17, seshadri2020indram}} propose using triple row activation in DRAM to perform bitwise majority \omi{(and thus AND/OR) and NOT} operations across the three activated rows. ComputeDRAM~\cite{gao2020computedram} shows that a subset of real DDR3 chips can perform triple row activation when they receive an ACT-PRE-ACT command sequence with violated timing parameters.} SIMDRAM~\cite{hajinazarsimdram} develops a framework that provides a programming interface to perform in-DRAM computation using the majority operation. \omi{DR-STRANGE~\cite{bostanci2022drstrange} proposes an end-to-end system design for DRAM-based true random number generators.} None of these works provide an end-to-end in-DRAM computation framework that is integrated into a real system \omu{using real DRAM chips}.}

{We conclude that} existing platforms cannot substitute PiDRAM in studying \new{commodity DRAM based} PuM techniques end-to-end.

%\revcommon{PiDRAM's prototype (i) fully integrates our framework into a RISC-V system, (ii) provides a custom supervisor that supports the necessary OS primitives,  and (iii) enables rapid implementation of PuM techniques using real DRAM chips. Therefore, it enables researchers to conduct end-to-end, full-system studies of PuM techniques. Existing platforms cannot substitute PiDRAM in studying PuM techniques end-to-end.}



\section{Conclusion}
% 
% Clean
% 
Our work introduces a new approach 
for obtaining generalization bounds
that do not directly depend on the 
underlying complexity of the model class. 
For linear models, we provably obtain a bound 
in terms of the fit on randomly labeled data added during training. 
Our findings raise a number of questions to be explored next. 
While our empirical findings and theoretical results with 0-1 loss 
hold absent further assumptions
and shed light on why the bound 
may apply for more general models,
we hope to extend our proof 
that overfitting (in terms classification error)
to the finite sample of mislabeled data
occurs with SGD training on broader classes 
of models and loss functions. 
We hope to build on some early results
\citep{nakkiran2019sgd, hu2020surprising} 
which provide evidence that deep models
behave like linear models 
in the early phases of training.  
We also wish to extend our framework
to the interpolation regime.
Since many important aspects of neural network learning 
take place within early epochs
\citep{achille2017critical,frankle2020early},
including gradient dynamics converging 
to very small subspace~\citep{gur2018gradient},
we might imagine operationalizing our bounds
in the interpolation regime
by discarding the randomly labeled data
after initial stages of training. 
 



% 
% Working draft
% 

% % In this paper, \dots % Add more for conclusion
% % Our work takes a step towards obtaining generalization bounds 
% Our work introduces a new approach 
% for obtaining generalization bounds
% that do not directly depend on the 
% underlying complexity of the model class. 
% For linear models, we provably obtain a bound 
% in terms of the fit on randomly labeled data added during training. 
% Our findings raise a number of questions to be explored next. 
% While our empirical findings and theoretical results with 0-1 loss 
% hold absent further assumptions
% and shed light on why the bound 
% may apply for more general models,
% we hope to extend our proof 
% that overfitting (in terms classification error)
% to the finite sample of mislabeled data
% occurs with SGD training on broader classes 
% of models and loss functions. 
% % shed some light on why this might hold with general models, 
% % we believe theoretically extending our bounds to general settings 
% % with SGD training is an interesting next question. 
% We hope to build on some early results~\citep{nakkiran2019sgd, hu2020surprising} 
% which provide evidence that deep models
% behave like linear models 
% in the early phases of training.  
% % 
% % Subsequently, it may also be interesting 
% % to extend our results in the interpolation regime. 
% We also wish to connect our framework
% to the interpolation regime.
% % Several recent studies show 
% % that significant and consequential changes 
% % occur during the early stage of training
% % \citep{achille2017critical,frankle2020early},
% % including gradient dynamics converging 
% % to very small subspace~\citep{gur2018gradient}.
% Since many important aspects of neural network learning 
% take place within early epochs
% \citep{achille2017critical,frankle2020early},
% including gradient dynamics converging 
% to very small subspace~\citep{gur2018gradient},
% we might imagine operationalizing our bounds
% in the interpolation regime
% % it can be interesting to operationalize our bounds 
% % in the interpolation regime, 
% % perhaps 
% by discarding the randomly labeled data
% after initial stages of training. 
 

\section*{Acknowledgements}
This research was partially supported by ACCESS – AI Chip Center for Emerging Smart Systems, sponsored by InnoHK funding, Hong Kong SAR.

%%%%%%%%% -- BIB STYLE AND FILE -- %%%%%%%%
\balance
\begin{spacing}{0.75}
\begin{footnotesize}
\bibliographystyle{IEEEtranS}
\bibliography{PiDRAM}
%%%%%%%%%%%%%%%%%%%%%%%%%%%%%%%%%%%%
\end{footnotesize}
\end{spacing}
\newpage

\end{document}
