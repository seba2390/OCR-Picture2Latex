%\outline{DRAM Background}

{We provide the relevant background on DRAM organization, DRAM operation and commodity DRAM based PuM techniques. We refer the reader to prior works for more comprehensive background about DRAM organization and operation~\cite{salp,lee.hpca13,donghyuk-ddma,chang.sigmetrics17,ghose2018vampire,patel2017reaper,luo2020clr,ghose2019demystifying,kevinchang-thesis,yoongu-thesis,lee.thesis16,olgun2021quactrng}.}

\subsection{DRAM Background}
\label{sec:background-dram}
DRAM-based main memory is organized hierarchically. \new{Fig.~\ref{fig:dram-bank-timing-diagram} (top) depicts this organization.} A processor is connected to one or \newnew{more} {memory channels \new{(DDRx in the figure)~\boldone{}}}. Each channel has its own command, address, and data buses. Multiple {memory modules} can be plugged into a single channel. Each module contains several {DRAM chips}\new{~\boldtwo{}. Each chip} contains multiple {\new{DRAM} banks} \new{that can be accessed} independently\new{~\boldthree{}}.\revdel{A set of DDRx standards cluster multiple {banks} in {bank groups}~\cite{jedecDDR4,gddr5}.}
\newnew{D}ata transfers between DRAM memory modules and processors occur at {cache block} granularity. The cache block size is typically 64 bytes in \atb{current} systems.


\begin{figure}[h]
     \centering
     
     \begin{subfigure}[h]{.50\textwidth}
         \centering
         \includegraphics[width=\textwidth]{figures/dram-bank.pdf}
     \end{subfigure}
     \hfill
     \begin{subfigure}[h]{.45\textwidth}
         \centering
         \includegraphics[width=\textwidth]{figures/timing-diagram.pdf}
     \end{subfigure}
    
    \caption{DRAM organization (top). Timing diagram of \new{DRAM} commands (bottom).}
    
    \label{fig:dram-bank-timing-diagram}
\end{figure}

\new{Inside a DRAM bank, DRAM cells are laid out \newnew{as} a two dimensional array of wordlines \new{(i.e., DRAM rows)} and bitlines \newnew{(i.e., DRAM columns)~\boldfour{}. W}ordlines are depicted in blue and bitlines are depicted in red in Fig.~\ref{fig:dram-bank-timing-diagram}. Wordline drivers drive the wordlines and sense amplifers read the values on the bitlines.} \newnew{A DRAM cell is connected to a bitline via an access transistor\new{~\boldfive{}}}. When enabled, an access transistor allows charge to flow between a DRAM cell and the cell's bitline.

\noindent
\new{\textbf{DRAM Operation.}} When all DRAM rows in a {bank} are closed, DRAM bitlines are precharged to \newnew{a reference voltage level of} {$\frac{V_{DD}}{2}$}. The memory controller sends an activate ($ACT$) command \new{to the DRAM module} to drive a DRAM wordline (i.e., enable a DRAM row). Enabling a DRAM row starts the {charge sharing} process. \newnew{Each DRAM cell connected to the DRAM row starts sharing its charge with its bitline}. This \omi{causes} \newnew{the bitline voltage} to deviate from {$\frac{V_{DD}}{2}$} {(i.e., the charge in the cell perturbs the bitline voltage)}. The \new{sense amplifier} sense\newnew{s} the deviation in the bitline and amplif\newnew{ies} the voltage of the bitline either to {$V_{DD}$} or to $0$. \newnew{As such}, \newnew{an ACT command} copies one DRAM row to the \new{sense amplifiers} (i.e., row buffer). The memory controller can send READ\newnew{/}WRITE commands to transfer data from/to {the sense amplifier array}. {Once the memory controller needs to access another DRAM row, t}he memory controller can close the {enabled} DRAM row by sending a precharge (PRE) command on the command bus. The PRE command first disconnects DRAM cells from their bitlines by disabling the enabled wordline and then precharges the bitlines to {$\frac{V_{DD}}{2}$}.

%\outline{Timing Parameters}
\noindent
\new{\textbf{DRAM Timing Parameters.}} DRAM datasheets specify a set of timing parameters that define the minimum time window between valid combinations of DRAM commands~\cite{lee.hpca15, kevinchang-thesis, chang.sigmetrics16, kim2018solar}. The memory controller must wait for tRCD, tRAS, and tRP nanoseconds between successive ACT $\rightarrow$ RD, ACT $\rightarrow$ ACT, and PRE $\rightarrow$ ACT commands, respectively {(Figure~\ref{fig:dram-bank-timing-diagram}, bottom)}. Prior works show that these timing parameters can be violated (e.g., successive ACT $\rightarrow$ RD commands may be issued with a shorter time window than tRCD) to improve DRAM access latency~\cite{lee.hpca15, chang.sigmetrics16, kevinchang-thesis, lee.sigmetrics17, kim2018solar}, implement physical unclonable functions~\cite{kim.hpca18,talukder2019exploiting,orosa2021codic}, \newnew{generate true random numbers~\cite{olgun2021quactrng,olgun2021quactrngieee,kim.hpca19},} copy data~\cite{seshadri2013rowclone,gao2020computedram}, and perform bitwise AND/OR operations~\cite{seshadri.micro17,seshadri.thesis16,seshadri.arxiv16,seshadri.bookchapter17.arxiv,gao2020computedram} in commodity DRAM devices.

\noindent
\textbf{\new{{DRAM Internal Address Mapping.}}}
\Copy{R2/1C}{{DRAM manufacturers use DRAM-internal address mapping schemes~\cite{salp,cojocar2020susceptible,patel2022case} to translate from logical (e.g., row, bank, column) DRAM addresses \newnew{that are used by the memory controller} to physical DRAM addresses \newnew{that are internal to the DRAM chip} (e.g., the \newnew{physical} position of a DRAM row within the chip). These schemes allow (i) post-manufacturing row repair techniques to map erroneous DRAM rows to redundant DRAM rows and (ii) DRAM manufacturers to organize DRAM internals in a cost-efficient \newnew{and reliable} way~\cite{khan.dsn16,vandegoor2002address}. DRAM-internal address mapping schemes can be substantially different across different DRAM chips~\cite{barenghi2018software,cojocar2020susceptible,horiguchi1997redundancy,itoh2013vlsi,keeth2001dram,khan.dsn16,khan.micro17,kim-isca2014,lee.sigmetrics17,liu.isca13,orosaYaglikci2021deeper,saroiu2022price,patel2020bit,patel2022case}. Thus, \newnew{consecutive} logical DRAM row addresses might not point to physical DRAM rows in the same subarray.}}
%\outline{in-DRAM Computation: RowClone, D-RaNGe, AMBIT, ...}

%\todo{Juan: Good for the readers to get an idea on how PIM techniques work: AMBIT, RowClone... Recent effort on doing computation in DRAM.}

\subsection{PuM Techniques}

\label{sec:background_pudram}
Prior work proposes a variety of in-DRAM computation mechanisms (i.e., PuM techniques) that (i) have great potential to improve system performance and energy efficiency~\cite{chang.hpca16,seshadri.micro17, hajinazarsimdram,seshadri2013rowclone,seshadri2020indram,seshadri.bookchapter17.arxiv,seshadri.bookchapter17,seshadri.arxiv16,Seshadri:2015:ANDOR,angizi2019graphide, ferreira2021pluto} %and 
\juan{or} (ii) can provide low-cost security primitives~\cite{talukder2019prelatpuf,talukder2019exploiting,kim.hpca19,kim.hpca18,olgun2021quactrngieee,orosa2021codic}. A subset of these in-DRAM computation mechanisms are demonstrated on real DRAM chips~\cite{gao2020computedram,kim.hpca19, kim.hpca18, talukder2019exploiting,olgun2021quactrngieee,orosa2021codic}. {We describe the \newnew{major relevant} prior works \omi{briefly}}: 

%\textbf{RowClone~\cite{seshadri2013rowclone}} proposes minor modifications to DRAM arrays to enable bulk data copy \atb{and initialization} inside DRAM. \atb{RowClone activates two DRAM rows successively without precharging the bitlines. By the time the second row is activated, the bitlines are loaded with the data in the first row. Bulk-copy and initialization intensive workloads (e.g., fork, memcached~\cite{memcached}) can significantly benefit from the increase in copy and initialization throughput provided by RowClone.}

%\outline{Explain ComputeDRAM}

%\todo{This should come later (after we discuss AMBIT/RowClone). Or we can talk about it in the later sections.}
%\jgl{I think this section is good, after having introduced RowClone/Ambit. We do not need many details about these works here, since we can give more background in the specific use case.}

\noindent
\newnew{\textbf{RowClone~\cite{seshadri2013rowclone}} \omi{is} a low-cost DRAM architecture that can perform bulk data movement operations (e.g., copy, initialization) inside DRAM chips \omi{at high performance and low energy}.}

\noindent
\newnew{\textbf{Ambit~\cite{Seshadri:2015:ANDOR, seshadri.arxiv16, seshadri.micro17, seshadri.bookchapter17, seshadri2020indram}} \omi{is} a new DRAM substrate that can perform \omi{(i)} bitwise majority \omi{(and thus bitwise AND/OR)} \omi{operations} across three DRAM rows by simultaneously activating three DRAM rows \omi{and (ii) bitwise NOT} operation\omi{s} \omi{on a DRAM row} using 2-transistor 1-capacitor DRAM cells~\cite{kang2009one,lu2015improving}.}

\noindent
\textbf{ComputeDRAM~\cite{gao2020computedram}} demonstrates in-DRAM copy {(previously proposed by RowClone~\cite{seshadri2013rowclone})} and bitwise AND/OR operations {(previously proposed by Ambit~\cite{seshadri.micro17})} on real DDR3 chips. ComputeDRAM performs in-DRAM operations by issuing carefully-engineered, valid sequences of DRAM commands {with violated tRAS and tRP timing parameters (i.e., by not obeying manufacturer-recommended timing parameters defined in DRAM \newnew{chip} specifications~\cite{micron2016ddr4})}. By issuing command sequences {with violated timing parameters}, \omi{ComputeDRAM \omi{activates}} two or three DRAM rows in a DRAM bank \newnew{in quick succession} \omi{(i.e., performs two or three row activation\omi{s})}. \newnew{ComputeDRAM leverages (i) two row activation\omi{s} to transfer data between two DRAM rows and (ii) three row activation\omi{s} to perform the majority function in real unmodified DRAM chips.} 
%This allows multiple DRAM cells on a column to contribute to the charge sharing process. 
\revdel{ComputeDRAM leverages multiple row activation (i) to transfer data between two DRAM cells, and (ii) to perform {the} majority function across three rows {in real unmodified DRAM chips}.}

\noindent
%\outline{Explain D-RaNGe}
%\todo{This also should come later (but in background section).}
\textbf{D-RaNGe~\cite{kim.hpca19}} is a {state-of-the-art} high-throughput DRAM-based true random number generat{ion technique}. D-RaNGe leverages the randomness in DRAM activation (tRCD) failures as its entropy source. D-RaNGe extracts random bits from DRAM cells that fail with $50\%$ probability when accessed with a reduced \newnew{(i.e., violated)} tRCD. 
%The authors assess the quality of random bitstreams generated by D-RaNGe using various statistical tests and show that it generates high-quality true random numbers. 
D-RaNGe demonstrates {high-quality true random number generation} on a vast number of real DRAM chips across multiple generations.

\noindent
\newnew{\textbf{QUAC-TRNG~\cite{olgun2021quactrngieee}} demonstrates that four DRAM rows can be activated in a quick succession using an ACT-PRE-ACT command sequence \omi{(called QUAC)} with violated tRAS and tRP timing parameters in real DDR4 DRAM chips. QUAC-TRNG uses QUAC to generate true random numbers at high throughput and low latency.}

%\outline{RISC-V Rocket-Chip SoC generator}
%\subsection{RISC-V Rocket-Chip SoC Generator}
%Rocket-Chip is an open-source System-on-Chip (SoC) generator in Chisel3~\cite{chisel} hardware construction language, built by the RISC-V community~\cite{asanovic2016rocket}. Rocket-Chip is used to generate configurable RISC-V system designs.
%\jgl{This subsection seems to be coming from nowhere: "We use Rocket-Chip to implement a RISC-V system in our \X prototype."}

%\footnote{NIST Statistical Test Suite (STS) is a collection of statistical tests that are widely used in assessing the quality of random number generators. If a random number generator passes all NIST STS tests it is very likely that the random number generator outputs uncorrelated, random bitstreams.}