\label{sec:related-work}

To our knowledge, this is the first work to develop a flexible, open-source framework that enables integration and evaluation of \new{commodity DRAM based} \omi{processing-using-memory} (PuM) techniques on real DRAM chips by providing the necessary hardware and software components. %We design a memory allocation mechanism that satisfies the memory management requirements (Section~\ref{sec:rowclone}) of RowClone operations. 
We demonstrate the first end-to-end implementation of RowClone \omi{and D-RaNGe using real DRAM chips}. We compare the features of \X with other state-of-the-art \omi{prototyping and evaluation} platforms in Table~\ref{table:tools} and discuss them below. \omi{The four features we use for comparison are:} 

\begin{enumerate}
    \item \textbf{Interface with real DRAM chips:} The platform allows running experiments using real DRAM chips.
    \item \textbf{Flexible memory controller (MC) for PuM:} The platform provides a flexible memory controller that can easily be extended to perform (e.g., as in PiDRAM) or emulate (e.g., as in PiMulator~\cite{mosanu2022pimulator}) new PuM operations.
    \item \textbf{System software support:} The platform provides support for running system software such as operating systems or supervisor software (e.g., RISC-V PK~\cite{riscv-pk}).
    \item \textbf{Open-source:} The platform is available as open source software.
\end{enumerate}

\begin{table*}[!t]
\centering
\caption{Compari\omi{son of} \X with related state-of-the-art \omi{prototyping and evaluation} platforms}
\label{table:tools}
\resizebox{1.0\textwidth}{!}{\begin{tabular}{|l||c|c|c|c|}

\hline
           \textbf{Platforms} & \begin{tabular}[c]{@{}l@{}}\textbf{Interface with \omi{real} DRAM \omi{chips}}\end{tabular} & \begin{tabular}[c]{@{}l@{}}\textbf{Flexible MC for PuM}\end{tabular} & \begin{tabular}[c]{@{}l@{}}
           \textbf{System software support}\end{tabular}  & \begin{tabular}[c]{@{}l@{}}\textbf{Open-source}\end{tabular}  \\ \hline \hline
\textbf{Silent-PIM} \cite{kim2021SilentPIM} & \xmark & \xmark & \checkmark  & \xmark\\ \hline
\textbf{SoftMC}  \cite{hassan2017softmc}   & \checkmark (DDR3) &  \xmark  &  \xmark &  \checkmark\\ \hline
\textbf{ComputeDRAM}   \cite{gao2020computedram}    & \checkmark (DDR3) & \xmark & \xmark   & \xmark \\ \hline
\textbf{MEG}    \cite{zhang2020MEG}    & \checkmark (HBM) &  \xmark  & \checkmark  &   \checkmark  \\ \hline
{\textbf{PiMulator}}    \cite{mosanu2022pimulator}    & \xmark &  \checkmark  & \xmark  &   \checkmark  \\ \hline
\textbf{\omi{Commercial platforms (e.g.,} ZYNQ~\cite{zynq})} & \checkmark (DDR3/4) &  \xmark    & \checkmark   &  \xmark \\ \hline
\textbf{Simulators}  \cite{gem5-gpu,GEM5,ramulator,ramulator-pim,zhang2022pim,forlin2022sim2pim,yu2021multipim,xu2019pimsim} & \xmark &  \checkmark  &  \checkmark \omi{(potentially)}   & \checkmark  \\ \hline
\hline
\textbf{\X  (this work)}   & \checkmark (DDR3) &  \checkmark   &  \checkmark &  \checkmark  \\ \hline
\end{tabular}}

\end{table*}
%Several hardware platforms and system simulators provide different sets of facilities in using DRAM as the main memory. However, none of the existing platforms support PuM mechanisms. We compare the features of \X with the state-of-the-art platforms in Table \ref{table:tools} and discuss them below:  

\atb{\textbf{Silent-PIM~\cite{kim2021SilentPIM}.} Silent-PIM proposes a new DRAM \new{design} \revdel{(similar to FIMDRAM~\cite{kwon2021fimdram}) }that incorporates processing units capable of vector arithmetic computation. \revdel{The authors use the standard DRAM interface~\cite{jedecDDR4} to communicate with the DRAM device. }Silent-PIM's goal is to evaluate PIM techniques on a \emph{new, PIM-capable} DRAM device using standard DRAM commands \omi{(e.g., as defined in DDR4~\cite{jedecDDR4})}\omi{; it does not \omu{provide} an evaluation platform or prototype}. \atb{\new{In contrast}, \X is designed for researchers to rapidly integrate and evaluate PuM techniques that use \emph{real DRAM devices}. \X provides key hardware and software components that facilitate end-to-end implementations of PuM techniques.}}

\revb{\textbf{SoftMC~\cite{hassan2017softmc, softmc.github}} SoftMC is an FPGA-based DRAM testing infrastructure. SoftMC \revb{can issue} arbitrary sequences of DDR3 commands to real DRAM modules. {SoftMC is widely used in prior work that \omi{studies} the performance, reliability and security of real DRAM chips~\cite{frigo2020trr, hassan2021uncovering, kim2020revisiting, khan.micro17, chang.hpca16, talukder2020towards, farmani2021RHAT, kim.hpca19, talukder2019exploiting, lee.sigmetrics17, ghose2018vampire, orosaYaglikci2021deeper,talukder2019prelatpuf}}. \Copy{R5/2B}{SoftMC is built to test DRAM modules, \emph{not} to study end-to-end implementations of PuM techniques. \new{Thus,} SoftMC (i) does \emph{not} \omi{support application execution on a real system}, and (ii) \revb{\emph{cannot} use} DRAM modules as \revb{{main memory}}. {While SoftMC is useful in studies that perform exhaustive search on all possible sequences of DRAM commands to potentially uncover undocumented DRAM behavior (e.g., ComputeDRAM~\cite{gao2020computedram}, QUAC-TRNG~\cite{olgun2021quactrng}),} \X is developed to study end-to-end implementations of PuM techniques. \X provides an FPGA-based prototype that comprises a RISC-V system and supports using DRAM modules both for storing data \omi{(i.e., as main memory)} and \omi{performing PuM} computation.}}

\textbf{\atb{ComputeDRAM~\cite{gao2020computedram}.}} ComputeDRAM partially demonstrates \omi{that} two DRAM-based state-of-the-art PuM techniques\new{,} RowClone~\cite{seshadri2013rowclone} and Ambit~\cite{seshadri.micro17}\new{, \omi{are already possible} on real \omi{off-the-shelf} DDR3 chips}. ComputeDRAM uses SoftMC to demonstrate in-DRAM copy and bitwise \omi{AND/OR} operations on real DDR3 chips. ComputeDRAM's goal is \emph{not} to develop a framework to facilitate end-to-end implementations of PuM techniques. Therefore, it does \emph{not} provide (i) a flexible memory controller for PuM \omi{or}, (ii) support for system software. \X provides the necessary software and hardware components to facilitate end-to-end implementations of PuM techniques.

\textbf{MEG~\cite{zhang2020MEG}.} MEG is an open-source system emulation platform for enabling FPGA-based operation interfacing with High-Bandwidth Memory (HBM). MEG aims to efficiently retrieve data from HBM and perform the computation in the host processor implemented as a soft core on the FPGA. Unlike \X, MEG does \emph{not} implement a flexible memory controller that is capable of performing PuM operations. We demonstrate the flexibility of \X by implementing two state-of-the-art PuM techniques~\cite{seshadri2013rowclone,kim.hpca19}. \omi{We believe MEG and PiDRAM can be combined to get the functionality and prototyping power of both works.}

{\textbf{PiMulator~\cite{mosanu2022pimulator}.} PiMulator is an open-source PiM emulation platform. PiMulator implements a main memory and a PiM model using SystemVerilog, allowing FPGA emulation of PiM architectures. PiMulator enables easy emulation of new PiM techniques. However, it does \emph{not} allow end-to-end execution of workloads that use PiM techniques and it does not provide the user \omi{with} full control over the DRAM interface.}

\textbf{Commercial Platforms (e.g., ZYNQ \cite{zynq}).} Some commercial platforms implement CPU-FPGA heterogeneous computing systems. A memory controller and necessary hardware-software modules are provided to access DRAM as the main memory in such systems. However, in such systems, (i) there is \emph{no} support for PuM mechanisms\new{, and} (ii) the entire hardware-software stack is closed-source. \X can be integrated into these systems, using the closed-source computing system as the main processor. Our prototype utilizes an open-source system-on-chip (\omi{Rocket Chip}~\cite{asanovic2016rocket}) as the main processor, which enables developers to study architectural \omi{and microarchitectur\omu{al}} aspects of PuM techniques (e.g., \omi{data allocation and} coherence mechanisms). Such studies cannot be conducted \omi{using} closed-source computing systems.

\textbf{Simulators.} Many prior works propose full-system \omu{(e.g.,~\cite{GEM5,gem5-gpu})}, trace-based \omu{(e.g.,~\cite{ramulator-pim,ramulator,zhang2022pim,yu2021multipim,xu2019pimsim,scarab})}, and instrumentation-based \omu{(e.g.,~\cite{scarab,forlin2022sim2pim,xu2019pimsim})} simulators that can be used to evaluate PuM techniques. Although useful, these simulators do not model DRAM behavior and cannot integrate proprietary device characteristics (e.g., DRAM internal address mapping) into their simulations, \atb{without conducting a rigorous characterization study. Moreover, the effects of environmental conditions (e.g., temperature, voltage) on DRAM chips are unlikely to be modeled on \revcommon{accurate, full-system} simulators as it would require excessive computation, \revcommon{which} would \revcommon{negatively impact the} \revcommon{already poor} performance \revcommon{(200K instructions per second)} of \revcommon{full system simulators} ~\cite{zsim}}. \atb{\omi{In contrast}, \X interfaces with real DRAM devices \revcommon{and its prototype \omi{achieves} a 50 MHz clock speed (\omi{and} can be improved \omi{further}) which lets \X execute > 10M instructions per second (assuming < 5 cycles per instruction)}. \Copy{R3/5}{\X can be used to study end-to-end implementations of PuM techniques and explore solutions that take into account \revcommon{the effects related to the environmental conditions of real DRAM devices. {Future versions of \X{} could be easily extended (e.g., with real hardware that allows controlling DRAM temperature and voltage~\cite{pl068p,maxwellFT200}) to experiment with different DRAM temperature and voltage levels to better understand the effects of these environmental conditions on the reliability of PuM operations.} Using \X, experiments that require executing real workloads can take an order of magnitude \omi{shorter} wall clock time compared to using full-system simulators.}}}

{\textbf{Other Related Work.} \omi{Prior works (see Section~\ref{sec:background_pudram}) (i) propose or (ii) demonstrate using real DRAM chips, several DRAM-based PuM techniques that can perform computation~\cite{Seshadri:2015:ANDOR, seshadri.arxiv16, seshadri.micro17, seshadri.bookchapter17, seshadri2020indram,hajinazarsimdram,chang.hpca16,ferreira2021pluto,angizi2019graphide}, move data~\cite{seshadri2013rowclone,wang2020figaro}, or implement security primitives~\cite{olgun2021quactrngieee,kim.hpca18,kim.hpca19,talukder2019exploiting,talukder2019prelatpuf,orosa2021codic} in memory.}\revdel{{Seshadri et al.~\cite{Seshadri:2015:ANDOR, seshadri.arxiv16, seshadri.micro17, seshadri.bookchapter17, seshadri2020indram}} propose using triple row activation in DRAM to perform bitwise majority \omi{(and thus AND/OR) and NOT} operations across the three activated rows. ComputeDRAM~\cite{gao2020computedram} shows that a subset of real DDR3 chips can perform triple row activation when they receive an ACT-PRE-ACT command sequence with violated timing parameters.} SIMDRAM~\cite{hajinazarsimdram} develops a framework that provides a programming interface to perform in-DRAM computation using the majority operation. \omi{DR-STRANGE~\cite{bostanci2022drstrange} proposes an end-to-end system design for DRAM-based true random number generators.} None of these works provide an end-to-end in-DRAM computation framework that is integrated into a real system \omu{using real DRAM chips}.}

{We conclude that} existing platforms cannot substitute PiDRAM in studying \new{commodity DRAM based} PuM techniques end-to-end.

%\revcommon{PiDRAM's prototype (i) fully integrates our framework into a RISC-V system, (ii) provides a custom supervisor that supports the necessary OS primitives,  and (iii) enables rapid implementation of PuM techniques using real DRAM chips. Therefore, it enables researchers to conduct end-to-end, full-system studies of PuM techniques. Existing platforms cannot substitute PiDRAM in studying PuM techniques end-to-end.}

