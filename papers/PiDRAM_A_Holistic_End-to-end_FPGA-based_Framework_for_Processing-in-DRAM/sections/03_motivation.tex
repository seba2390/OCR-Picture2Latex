%Modern computing systems are bottlenecked by data movement in performance and energy efficiency. The data movement bottleneck has continued to worsen in the recent years as the improvements in DRAM-based main memory performance has not caught up to the improvements in processor performance. Recent developments in process technology~\cite{X} and the worsening data movement bottleneck motivated researchers to work on mechanisms that aim to alleviate the data movement bottleneck in computing systems. This effort brought attention to the processing-in-memory (PIM) paradigm, which tries to mitigate the data movement bottleneck by moving computation physically closer to memory.

%Processing-using-Memory (PuM) mechanisms enable computation inside \atb{the} memory. These mechanisms provide data copy~\cite{seshadri2013rowclone}, bitwise AND/OR/NOT~\cite{seshadri.micro17}, \juan{arithmetic operations~\cite{hajinazarsimdram,deng.dac2018}}, random number generation~\cite{kim.hpca19,talukder2019exploiting}, and physical unclonable functions~\cite{kim.hpca18,talukder2019exploiting}
%\jgl{DrAcc does not do matrix multiplication. You can say "ternary weight neural network inference" or simply "neural network inference". Or easier, just remove and move the citation next to SIMDRAM.} 
%by exploiting memory array structures and memory cell characteristics. Recent works demonstrate that some of these mechanisms can be implemented in commodity DRAM devices to \atb{significantly} improve system performance and reduce system energy consumption~\cite{gao2020computedram} \juan{or implement security primitives~\cite{kim.hpca18, kim.hpca19}}. We refer to these mechanisms as DRAM-based PuM techniques. Such techniques can reduce the data movement overhead in a broad range of computing systems from low-power edge devices to high-performance servers as DRAM-based main memory is ubiquitous in these systems.
%\jgl{PIM does not need much motivation at this point. But it is good to mention here the amazing speedups and energy savings that it can provide. You can take some numbers from RowClone and Ambit papers.
%Then, motivate why PiDRAM is necessary. This means enumerating the challenges for system integration of PIM and explaining why solutions to these challenges cannot be easily explored in conventional computers, simulators, or tools like SoftMC.}

%\textcolor{red}{Why is it important to study in-DRAM based mechanisms?} 
%It is important to study end-to-end implementations of DRAM-based PuM techniques as they are applicable to a vast amount of computing systems and have great potential to improve system performance and energy efficiency by moving computation to DRAM-based main memory and mitigating data movement overheads. Such end-to-end implementations of PuM techniques require modifications to all elements across the hardware/software stack as discussed in many prior work~\cite{X,Y,Z}. However, the scope and implications of such modifications are not studied end-to-end in a real system, hence the challenges in integrating PuM techniques to real systems are left unexplored, and the end-to-end benefits that PuM techniques can provide in a real system is not clear.

%\outline{Why we cannot use other platforms to study DRAM-PuMs}

\atb{Implementing DRAM-based PuM techniques and integrating them into a real system requires modifications across the hardware and software stack. End-to-end implementations of PuM techniques require proper tools that (i) are flexible, to enable rapid development of PuM techniques and (ii) support real DRAM devices, to correctly observe the effects of reduced DRAM timing operations that are fundamental \atb{to} enabling {commodity DRAM based} PuM in {real unmodified} DRAM devices. Existing {general-purpose} computers, specialized DRAM testing platforms (e.g., \newnew{those aforementioned, Section~\ref{sec:introduction}}) cannot be used to study end-to-end implementations of {commodity DRAM based} PuM techniques.}

\atb{First, implementing new DDRx command sequences that perform PuM operations requires modifications to the memory controller. {Existing general purpose} computers do not support customizations to the memory controller {to dynamically modify manufacturer-recommended DRAM timing parameters}~\cite{lee.hpca15,kim2018solar,chang.sigmetrics16,kim.hpca19,hassan2017softmc}. This hinders the possibility of studying end-to-end implementations of PuM techniques on such platforms. Second, PuM techniques impose data mapping and allocation requirements (Section~\ref{sec:rowclone}) that are not satisfied by current memory management and allocation mechanisms (e.g., malloc~\cite{malloc}). Current OS memory management schemes must be augmented to satisfy these requirements. Existing specialized DRAM testing platforms (e.g., SoftMC~\cite{hassan2017softmc}) do not have system support \newnew{to enable this}. By design, these platforms are not built for system integration. Hence, it is difficult to evaluate system-level mechanisms that enable PuM techniques on DRAM testing platforms. Third, system simulators (i) do \emph{not} model DRAM operation \newnew{that violates} manufacturer-recommended {timing parameters}, (ii) do \emph{not} have a way of interfacing with {real DRAM chips} that {embody undisclosed and unique characteristics that have implications on how PuM techniques are integrated into real systems (e.g., proprietary and chip-specific DRAM internal address mapping~\cite{cojocar2020susceptible,salp,patel2022case})} that influence PuM operations{, and (iii) \emph{cannot} support studies on the reliability of PuM techniques since system simulators do \emph{not} model environmental conditions \newnew{and process variation}.}}
\atb{We summarize the limitations of the relevant experimental platforms \newnew{later in} Table~\ref{table:tools}.}
%\jgl{I think it would be good to refer to Table 3 for other experimental platforms that lack the necessary features.}

\atb{Our \textbf{goal} is to develop a flexible \juan{end-to-end} framework that enables rapid \newnew{system} integration of {commodity DRAM based} PuM techniques and facilitates studies on end-to-end \newnew{full-system} implementations of PuM techniques using real DRAM devices. To this end, we develop \X.}


%\todo{Fix this paragraph:}
%\atb{Current PuDRAM techniques require DRAM command sequences that violate manufacturer-recommended command timings. }
%\textcolor{red}{To give me an idea:} Reason why we don't want to do this with simulators: in-DRAM computation mechanisms fundamentally require operating DRAM devices below manufacturer thresholds. It is really hard to characterize and precisely model DRAM behavior under manufacturer thresholds, at varying environmental conditions (e.g. temperature, voltage), and even if there is such a model it would be costly to integrate into a simulator and would further impact simulator performance in a bad way.
%Reason why we don't want to do this with DRAM testing platforms: They are not designed to be connected to a system as a memory controller or anything else. They are incompatible, not designed for that purpose.