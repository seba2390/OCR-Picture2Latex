We develop \X, a flexible and open-source \omi{prototyping} framework for integrating and evaluating end-to-end {commodity DRAM based} \omi{processing-using-memory} (PuM) techniques.
\X comprises the necessary \omi{hardware and software} structures to facilitate end-to-end implementation of PuM techniques.
We build an FPGA-based prototype of \X along with an open-source RISC-V system and enable computation on real DRAM chips.
Using \X, we implement and evaluate RowClone \omi{(in-DRAM data copy and initialization)} and D-RaNGe \omi{(in-DRAM true random number generation)} end-to-end \omi{in the entire real system}. 
Our results show that RowClone significantly improve\omi{s} data copy \omi{and initialization} throughput \omi{in a real system on real workloads}, and efficient cache coherence mechanisms are \omu{needed} 
to maximize \omi{RowClone's potential benefits}. Our implementation of D-RaNGe requir\omi{es small} additions to \omi{\X{}'s codebase and} \omi{provides true random numbers at high throughput and with low latency}.
We conclude that \omi{unlike existing prototyping and evaluation platforms,} \X enables (i) easy integration of \omi{existing and new} PuM techniques \omi{end-to-end in a real system} and (ii) novel studies on end-to-end implementations of PuM techniques \omi{using real DRAM chips}. \new{PiDRAM is \omi{freely} available} as an open-source tool \omu{for} researchers and \omi{designers} \omi{in both academia and industry to experiment with and build on.}
