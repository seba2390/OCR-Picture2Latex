% last updated in April 2002 by Antje Endemann
% Based on CVPR 07 and LNCS, with modifications by DAF, AZ and elle, 2008 and AA, 2010, and CC, 2011; TT, 2014; AAS, 2016
\documentclass[runningheads]{llncs}
\pdfoutput=1
\usepackage{graphicx}
\usepackage{amsmath,amssymb} % define this before the line numbering.
\usepackage{color}
\usepackage[width=122mm,left=12mm,paperwidth=146mm,height=193mm,top=12mm,paperheight=217mm]{geometry}

\usepackage{url}
\usepackage[table]{xcolor}
\usepackage{bbm}
\usepackage{booktabs}
\usepackage[T1]{fontenc}
\usepackage{fix-cm}
\usepackage{array}
\usepackage{epsfig}
%\usepackage{mathabx}
\usepackage{dsfont}
\usepackage{multirow}
\usepackage{hyperref}

\usepackage{times}
\usepackage{helvet}
\usepackage{courier}
\usepackage{graphicx}
\usepackage{bm}
\usepackage{color}
\usepackage{epstopdf}
\usepackage{caption}
\usepackage{subcaption}
\usepackage{enumitem}
\usepackage{calc}
\usepackage{multirow}
\usepackage{xspace}
\usepackage{booktabs}
\usepackage{mathrsfs}
\usepackage{array}
\usepackage{gensymb}

%\cvprfinalcopy % *** Uncomment this line for the final submission
\newcommand{\figref}[1]{Fig\onedot~\ref{#1}}
\newcommand{\equref}[1]{Eq\onedot~\eqref{#1}}
\newcommand{\secref}[1]{Sec\onedot~\ref{#1}}
\newcommand{\tabref}[1]{Tab\onedot~\ref{#1}}
\newcommand{\thmref}[1]{Theorem~\ref{#1}}
\newcommand{\prgref}[1]{Program~\ref{#1}}
\newcommand{\algref}[1]{Alg\onedot~\ref{#1}}
\newcommand{\clmref}[1]{Claim~\ref{#1}}
\newcommand{\lemref}[1]{Lemma~\ref{#1}}
\newcommand{\ptyref}[1]{Property\onedot~\ref{#1}}

\newcommand{\ve}[1]{{\mathbf #1}} % for displaying a vector or matrix
\newcommand{\hua}[1]{{\mathcal #1}}
\newcommand{\scr}[1]{{\mathcal #1}}
\newcommand{\spa}[1]{{\mathbb #1}}
\newcommand{\by}[2]{\ensuremath{#1 \! \times \! #2}}
\makeatletter
\newcommand{\printfnsymbol}[1]{%
  \textsuperscript{\@fnsymbol{#1}}%
}
\makeatother

%\makeatletter
\DeclareRobustCommand\onedot{\futurelet\@let@token\@onedot}
\def\onedot{\ifx\@let@token.\else.\null\fi\xspace}
\def\eg{\emph{e.g.}} 
\def\Eg{\emph{E.g}\onedot}
\def\any{\forall}
\def\ie{\emph{i.e.}} 
\def\Ie{\emph{I.e}\onedot}
\def\cf{\emph{cf}\onedot} 
\def\Cf{\emph{Cf}\onedot}
\def\etc{\emph{etc}\onedot} 
\def\vs{\emph{vs}\onedot}
\def\wrt{w.r.t\onedot} 
\def\dof{d.o.f\onedot}
\def\etal{\emph{et al.}}


\begin{document}
% \renewcommand\thelinenumber{\color[rgb]{0.2,0.5,0.8}\normalfont\sffamily\scriptsize\arabic{linenumber}\color[rgb]{0,0,0}}
% \renewcommand\makeLineNumber {\hss\thelinenumber\ \hspace{6mm} \rlap{\hskip\textwidth\ \hspace{6.5mm}\thelinenumber}}
% \linenumbers
\pagestyle{headings}
\mainmatter

\title{Depth Estimation via Affinity Learned with Convolutional Spatial Propagation Network} 
% Replace with your title

\titlerunning{CSPN}
% Replace with a meaningful short version of your title

\authorrunning{X. Cheng, P. Wang and R. Yang}
% Replace with shorter version of the author list. If there are more authors than fits a line, please use A. Author et al.

\author{Xinjing Cheng\thanks{equal contribution}, Peng Wang\printfnsymbol{1} and Ruigang Yang}

%Please write out author names in full in the paper, i.e. full given and family names. 
%If any authors have names that can be parsed into FirstName LastName in multiple ways, please include the correct parsing, in a comment to the volume editors:
%\index{Lastnames, Firstnames}
%(Do not uncomment it, because you may introduce extra index items if you do that, we will use scripts for introducing index entries...)


\institute{Baidu Research, Baidu Inc.\\
	\email{ \{chengxinjing,wangpeng54,yangruigang\}@baidu.com}
}

\maketitle

\begin{abstract}
Depth estimation from a single image is a fundamental problem in computer vision. In this paper, we propose a simple yet effective convolutional spatial propagation network (CSPN) to learn the affinity matrix for depth prediction. 
Specifically, we adopt an efficient linear propagation model, where the propagation is performed with a manner of recurrent convolutional operation, and the affinity among neighboring pixels is learned through a deep convolutional neural network (CNN). 
We apply the designed CSPN to two depth estimation tasks given a single image:  (1) Refine the depth output from existing state-of-the-art (SOTA)  methods;  (2) Convert sparse depth samples to a dense depth map by embedding the depth samples within the propagation procedure. The second task is inspired by the availability of LiDAR that provides sparse but accurate depth measurements. We experimented the proposed CSPN over the popular NYU v2~\cite{silberman2012indoor} and KITTI~\cite{geiger2012we} datasets, where we show that our proposed approach improves  not only quality (e.g., 30\% more reduction in depth error), but also speed (e.g., 2 to 5$\times$ faster) of depth maps than previous SOTA methods. %The codes of CSPN are available at: \url{https://github.com/XinJCheng/CSPN}.
\keywords{Depth estimation, Convolutional spatial propagation}
\end{abstract}

Reinforcement learning has achieved great success in areas such as Game-playing \citep{silver2018general,vinyals2019grandmaster}, robotics \cite{kober2013reinforcement}, large language models \citep{ouyang2022training}, etc.
However, due to safety concerns or physical limitations, in some real-world reinforcement learning problems, we must consider additional constraints that may influence the optimal policy and the learning process \citep{garcia2015comprehensive}.
% For example, a robotic arm must not take actions that may cause harm to itself or the environments.
A standard framework to handle such cases is the constrained Markov Decision Process (CMDP) \citep{altman1999constrained}.
Within the CMDP framework, the agent has to maximize
the expected cumulative reward while
obeying a finite number of constraints, which are usually in the form of expected cumulative cost criteria.

However, we are sometimes concerned with the problem with a continuum of constraints.
For example,
the constraints we meet might be time-evolving or subject to uncertain parameters, which
cannot be formulated as an ordinary CMDP
(see Examples \ref{Example_Time_Evolving} and  \ref{Example_Uncertain}).
In this paper we would study a generalized CMDP  
to address the above problem.  Because the constraints are not only infinite-number but also lie
in a continuous set,
the generalization is not trivial. Fortunately, we find that we can borrow the idea behind semi-infinite programming (SIP) \citep{remez1934determination, hettich1993semi} to deal with the semi-infinite constraints.
Accordingly, we propose \emph{semi-infinitely constrained Markov decision processes} (SICMDPs)
as a novel complement to the ordinary CMDP framework.
%More specifically,  an SICMDP model %, we consider 
%contains a continuum of constraints whereas an ordinary CMDP contains a finite number of constraints. 

%This generalization is natural but not trivial. However, we can brows the idea  
%The idea is quite natural and can be backtracked
%to the practice of extending linear programming to linear semi-infinite programming (LSIP) %\cite{remez1934determination, GobernaLSIO1998}.
%In addition, 
%As a complementary approach to the ordinary CMDP framework, 
%SICMDP can be used to model these problems  which cannot be described by a finite number of constraints
%that are not covered by .
%For example,
%the restrictions we consider can be time-evolving or subject to uncertain parameters
%, thus
%cannot be described by a finite number of constraints but a continuum of constraints 
%(see Examples \ref{Example_Time_Evolving} and  \ref{Example_Uncertain}).

We also present two reinforcement learning algorithms to solve SICMDPs called SI-CRL and SI-CPO, respectively.
SI-CRL is a model-based reinforcement learning algorithm designed for tabular cases, and SI-CPO is a policy optimization algorithm for non-tabular cases.
% and analyze its performance both theoretically and empirically.
The main challenge is that we need to deal with a continuum of constraints, thus reinforcement learning algorithms for ordinary CMDPs do not work anymore.
In SI-CRL, we tackle this difficulty by first transforming the reinforcement learning problem to an equivalent LSIP problem, which can then be solved using methods in the LSIP literature like the dual exchange methods \citep{Hu1990,reemtsen1998numerical}.
In SI-CPO, we resort to the idea of cooperative stochastic approximation developed in \cite{lan2020algorithms, wei2020comirror}.
As far as we know, we are the first to introduce tools from semi-infinitely programming (SIP) into the reinforcement learning community for solving constrained reinforcement learning problems.

% To the best of our knowledge, we are the first to apply tools from semi-infinitely programming (SIP) to solve reinforcement learning problems.
Furthermore, we give theoretical analysis for both SI-CRL and SI-CPO.
We decompose the error of SI-CRL into two parts: the statistical error from approximating the true SICMDP with an offline dataset and the optimization error due to the fact that the solution of the LSIP problem obtained by the dual exchange method is inexact.
On the optimization side, we show that the iteration complexity of SI-CRL is $O\left(\left\{\mathrm{diam}(Y)L\sqrt{|\gS|^2|\gA|m}/\left[(1-\gamma)\epsilon\right]\right\}^m\right)$.
On the statistical side, we show that the sample complexity of SI-CRL is $\widetilde O\left(\frac{|S|^2|A|^2}{\epsilon^2(1-\gamma)^3}\right)$ if the offline dataset is generated by a generative model, and $\widetilde O\left(\frac{|S||A|}{\nu_{\min} \epsilon^2(1-\gamma)^3}\right)$ if the dataset is generated by a probability measure $\nu$ as considered in \cite{chen2019information}.
Here $\widetilde O$ means that all logarithm terms are discarded.
For SI-CPO, things become a little more complicated because other than the statistical error and the optimization error, we also need to consider the function approximation error, which comes from imperfect policy parametrizations.
It is shown if the function approximation error can be controlled to $O(\epsilon)$ order, the iteration complexity of SI-CPO is $\widetilde{O}\left(\frac{1}{\epsilon^2(1-\gamma)^6}\right)$ and the sample complexity of SI-CPO is $\widetilde{O}(\frac{1}{\epsilon^4(1-\gamma)^{10}})$.
Here our iteration complexity bound is equivalent to a typical $\widetilde O(1/\sqrt{T})$ global convergence rate.

We perform a set of numerical experiments to illustrate the SICMDP model and validate our proposed algorithms.
Specifically, we examine two numerical examples, namely the discharge of sewage and ship route planning.
Through the discharge of sewage example, we show the advantage of the SICMDP framework over the CMDP baseline obtained by naive discretization in modeling realistic sequential decision-making problems.
Moreover, we demonstrate the effectiveness of the SI-CRL and SI-CPO algorithms in such tabular environments. 
In the ship route planning example, we illustrate the benefits of the SICMDP framework and the ability of the SI-CPO algorithm to address complex continuous control tasks involving continuous state spaces with modern deep reinforcement learning techniques.

% In summary, our contributions are listed as follows.
% First, we present the SICMDP model, which can be viewed as a generalization of the ordinary CMDP model.
% Second, we propose an algorithm to perform reinforcement learning for SICMDPs, which is called SI-CRL, and we believe that we are the first to apply tools from SIP
% to solve reinforcement learning problems.
% Third, we give a theoretical analysis of SI-CRL and identify both its sample complexity and iteration complexity.
% In addition, we perform numerical experiments to illustrate the SICMDP model and validate the SI-CRL algorithm.
% \{This paragraph can be removed!!! \}





\textbf{Related work}:
% Object detection related datasets/algo in non-medical domain
% Locally labeled CXR dataset
A few CXR datasets have localized abnormality annotations \cite{shih2019augmenting,filice2020crowdsourcing,jaeger2014two} that are curated manually. These are high quality gold standard ground truth datasets but tend to be smaller in scale (< 30,000 images) and have a narrow coverage, with typically only 1-2 labels. In addition, since most labeling efforts only have abnormality semantics attached, no direct relationships with the affected anatomical locations are available. 

%MEHDI: repeated concepts from above. I am removing the following: 

%The lack of anatomic semantics in the annotation is a limitation for complex multi-modal clinical reasoning work, e.g., differential diagnosis, since clinicians often integrate information along anatomical lines, and for downstream report generation tasks, which often requires describing not only the abnormality but also correctly communicate the location of the abnormalities (and medical devices) to the receiving clinicians. 

Two recent CXR datasets have labels for anatomies described in the reports. In \cite{datta2020dataset}, a small manually annotated dataset (2000 reports) included 10 abnormalities that are individually associated with 29 unique spatial locations (anatomies) at the report level. Another CXR dataset has automatically extracted abnormality and anatomy labels as disconnected concepts that are only correlated at the study level from  160,000 reports using a supervised NLP algorithm \cite{bustos2020padchest}. This was trained on a smaller set of manually annotated data. Neither datasets contain localized annotations for the associated CXR images, nor any comparison relation annotations between sequential exams, both of which are available in the Chest ImaGenome dataset. In Table \ref{tab:related}, we present a comparison of our Chest ImagGenome dataset with other datasets available in the literature.

% Table -- Kashyap

% MEdical imaging datasets to go here: Discussed that we will only focus on cxr datasets that are available for this paper. 
% \caption{\color{red} Kashyap, feel free to continue with the table. We should remove the questionmarks and add a line for our dataset (since all others are not graph). For longer text, using abbreviations and explaining them in the caption often works better. If fill in the values is not possible, it is better to remove the table altogether.}


\begin{table}[t!]
\caption{Summary of existing chest X-ray datasets}
\resizebox{\textwidth}{!}{%
\begin{tabular}{@{}lllllllll@{}}
\toprule
\textbf{Dataset} & \textbf{Annotation Level} & \textbf{Annotation Method} & \textbf{Num Labels} & \textbf{Anatomy Labeled} & \textbf{Graph} & \textbf{Dataset Size} & \textbf{Temporal Labels} & \textbf{Reports} \\ \midrule
SIIM-ACR Pneumothorax Segmentation \cite{filice2020crowdsourcing} & Segmentation & Manual + augmented & 1 & No & No & 12,047 & No & No \\
RSNA Pneumonia Detection Challenge   \cite{shih2019augmenting} & Bounding Boxes & Manual & 1 & No & No & 30,000 & No & No \\
Indiana University Chest X-ray collection \cite{demner2016preparing} & Global & Automated & 10 & No & No & 3,813 & No & Yes \\
NIH CXR dataset \cite{wang2017chestx} & Global & Automated & 14 & No & No & 112,120 & No & No \\
PLCO \cite{team2000prostate} & Global & Automated & 24 & Yes & No & 236,000 & Yes & No \\
Stanford CheXpert \cite{irvin2019chexpert} & Global & Automated & 14 & No & No & 224,316 & No & No \\
MIMIC-CXR \cite{johnson2019mimic} & Global & Automated & 14 & No & No & 377,110 & No & Yes \\
Dutta \cite{datta2020dataset} & Global & Manual & 10 & Yes & Yes & 2,000 & No & Yes \\
PadChest \cite{bustos2020padchest} & Global & Manual + automated & 297 & Yes & No & 160,868 & No & Yes \\
Montgomery County Chest X-ray   \cite{jaeger2014two} & Segmentation & Manual & 1 & Yes & No & 138 & No & No \\
Shenzen Hospital Chest X-ray   \cite{jaeger2014two} & Segmentation & Manual & 1 & Yes & No & 662 & No & No \\  \hline \hline
\textbf{Chest ImaGenome} & Bounding Boxes & Automated & 131 & Yes & Yes & 242,072 & Yes & Yes \\
\bottomrule
\end{tabular}%
}
\label{tab:related}
\vspace{-0.4cm}
\end{table}
% removed (Derived from MIMIC-CXR \cite{johnson2019mimic}) % makes table really small

\section{Approach}
\begin{figure}[t]
\centering
\resizebox{0.48\textwidth}{!}{ 
  \includegraphics[width=\textwidth]{figures/workflow.PNG}
}
  \caption{Workflow of \system}
  \label{fig:workflow}
\end{figure}

Figure ~\ref{fig:workflow} shows the overall workflow of \system. The triggers for using \system are usually alert(s) from automated anomaly detection, or sometimes an SRE engineer's suspicion. There are three major steps: constructing the service  dependency graph, constructing the event causality graph,  and root cause ranking. The outputs are the root causes ranked by the likelihood. To support fast human investigation experience, we build an interactive UI as shown in  Figure~\ref{fig:UI}: the service dependency, events with causal links and additional details such as raw metrics or the developer contact (of a code deployment event) are presented to the user for next steps. As an  offline part of human investigation, we label/collect a data set, perform validation, and summarize the knowledge for further improvement on all incidents on a daily basis. %as validations and heterogeneous graph learning (HGL)~\cite{qiao2020heterogeneous} to synthesize the knowledge from existing cases in order to further improve the system.

\subsection{Constructing Service Dependency Graph}
\label{sec:appgraph}

The construction of the service dependency graph starts with the initial alerted or suspicious service(s), denoted as $I$. For example, in Figure ~\ref{fig:ex1_dep}, $I=\{\textit{Checkout}\}$. $I$ can contain multiple services based on the range of the trigger alerts or suspicions. We maintain domain service lists where domain-level alerts can be triggered because there is no clear service-level indication.

At the back end, \system maintains a global service dependency graph $G_{global}$ via distributed tracing and log analysis. The directed edge from nodes $A$ to $B$ (two services or system components) in the dependency graph indicates a service invocation or other forms of dependency. In Figure~\ref{fig:ex1_dep}, the black arrows indicate such edges. Bi-directional edges and cycles between the services can be possible and exist. In this work, the global dependency graph is updated daily.%by extracting from one day's total site traffic.

The service dependency (sub)graph $G$ is constructed using $G_{global}$ and $I$. An extended service list $L$ is first constructed by traversing each service in $I$ over $G_{global}$ for a radius range $r$. Each service $u \in L$ can be traversed by at least one service $v \in I$ within $r$ steps: $L=\{u|\exists v\in I, \ dist(u,v)\le r\ or\ dist(v,u)\le r\}$. Then, the service dependency subgraph $G$ is constructed by the nodes in $L$ and the edges between them in $G_{global}$. In our current implementation, $r$ is set to $2$, since this dependency graph may be dynamically extended in the next steps based on events' detail for longer issue chains or additional dependencies.

\subsection{Constructing Event Causality Graph}
\label{sec:causality}

In the second step, \system collects all supported events for each service in $G$ and constructs the causal links between events. 

\subsubsection{Collecting Events}

Table~\ref{tab:events} presents some example event types and detection techniques for \system's production implementation. For detection techniques, ``De Facto'' indicates that the event can be directly collected via a specific API or storage. %The detection can be done passively at the back end continuously then store anomaly events in different databases; or done actively by pulling data and run detection on the fly to save compute resources. 
The detection either runs passively in the back end to reduce delay and improve accuracy, or runs actively for only the services within the dependency graph range to save resources. %For example, low-level error signals or logs are detected actively since they are too many to scale. 

There are three major categories of events: performance metrics, status logs, and developer activities:
\begin{itemize}
    \item \emph{Performance metrics} represent an anomaly of monitored time series metrics. For example, high CPU usage indicates that the service is causing high CPU usage on a certain machine. In this category, most events are continuously and passively detected and stored. %For high CPU usage, threshold indicates the event is created when CPU usage is higher than certain predefined value. TPS spike indicates a spike in transaction per second, since TPS is a moving average value, we use some statistical model learned from historical data to detect such events.
    \item \emph{Status logs} are caused by abnormal system status, such as spike of HTTP error code metrics while accessing other services' endpoints. Different types of error metrics are important and supported in \system, including third-party APIs. For example, Bad Host indicates abnormal patterns on some machines running the service, and can be detected by a  clustering-based ML approach.%Markdown indicates that the whole service is down. 
    \item \emph{Developer activities} are the events generated when a certain activity of developers is triggered, such as code deployment and config change.
\end{itemize}

\begin{table}[t]
\centering
\caption{List of example event types used in \system}
\resizebox{0.4\textwidth}{!}{ 
\begin{tabular}{|c|c|c|}
\hline
Type                                & Event Type                  & Detection Technique  \\ \hline
\multirow{6}{*}{Performance Metrics} & High GC (Overhead)      & Rule-based        \\ \cline{2-3} 
                                    & High CPU Usage          & Rule-based        \\ \cline{2-3} 
%                                    & Out of Memory           & Rule-based        \\ \cline{2-3} 
%                                    & LB Connection Stacking  & Statistical Model \\ \cline{2-3} 
                                    & Latency Spike           & Statistical Model \\ \cline{2-3} 
                                    & TPS Spike               & Statistical Model \\ \cline{2-3} 
                                    & Database Anomaly        & ML Model          \\ \cline{2-3} 
                                    & Business Metric Anomaly & ML Model          \\ \hline
\multirow{4}{*}{Status Logs}        & WebAPI Error            & Statistical Model \\ \cline{2-3} 
                                    & Internal Error          & Statistical Model \\ \cline{2-3} 
                                    & ServiceClient Error     & Statistical Model \\ \cline{2-3} 
                                    & Bad Host                & ML Model          \\ \hline %\cline{2-3} 
%                                    & Hystrix Circuit Break   & De Facto          \\ \hline
\multirow{3}{*}{Developer Activities} & Code Deployment         & De Facto          \\ \cline{2-3} 
                                    & Configuration Change    & De Facto          \\ \cline{2-3} 
                                    & Execute URL             & De Facto          \\ \hline
\end{tabular}
}
\label{tab:events}
\end{table}

In Groot, there are more than a dozen event types such as \emph{Latency Spike} as listed in the column 2 of Table~\ref{tab:events}. 
Each event type is characterized by three aspects: $Name$ indicates the name of this event type; $Lookback Period$ %\footnote{In Figure~\ref{fig:ex2_n1}, there are two periods, 1 day indicates the look-back range if the model has already finished deployment, 4 days indicates the range if the model deployment is still ongoing(incremental deployment).} 
indicates the time range to look back (from the time when the use of \system is triggered) for collecting events of this event type;  $PropertyType$ indicates the types of the properties that an event of this event type should hold. 
$PropertType$  is characterized by a vector of pairs, each of which indicates the string type for a property's name and the primitive type for the property's value such as string, integer, and float. 
Formally, an event type is defined as a tuple: 
$ET = <Name, Lookback Period, PropertyType>$ 
where 
$PropertyType = <(string, \textit{type}_1), ..., (string, \textit{type}_{n})>$ ($n$ is the number of properties that an event of this event type holds). 
%

Each event of a certain event type $ET$ is characterized by four aspects:
$\textit{Service}$ indicates the service name that the event belongs to; $\textit{Type}$ indicates $ET$'s $\textit{Name}$;  $\textit{StartTime}$ indicates the time when the event happens; $\textit{Properties}$ indicates the properties that the event  holds.
Formally, an event is defined as a tuple: 
$e = <Service, Type, StartTime, Properties>$ 
where $Properties$ is an instantiation of $ET$'s  $PropertyType$. 


%and each event is defined as $e = \{<\textit{Property}_i, \textit{value}_i>\}$. Each event type serves as a template for the event instantiation. such as a string, an integer, a float or a set of primitive types while $\textit{value}$ is limited to primitive data types. 
%
%Each event is defined as a sequence of property-value pairs where the set size is $n$.

For example, in Figure~\ref{fig:example1}, the generated event for \emph{Latency Spike in DataCenter-A} in \emph{Service-C} would be $<``\textit{Service-C}'', ``\textit{Latency\ Spike}'', \textit{2021/08/01-12:36:04}, <(``\textit{DataCenter}'',``\textit{DC-1}''),  ...>>$. %So for each service in $G$, we detect/collect and filter the events within specified time range of the alert.

\subsubsection{Constructing Causal Link}

After collecting all events on all services in $G$, in this step, causal links between these events are constructed for RCA ranking. The causal links (red arrows) in Figure~\ref{fig:ex1_cas} are such examples. A causal link represents that the source event can possibly be caused by the target event. SRE knowledge is engineered into rules and used to create causal links between the pairs of events. %As shown in Figure~\ref{fig:example2}, there are two categories of rules: basic rules and conditional rules. 

A rule for constructing a causal link is defined as a tuple:  $Rule = <Target\mbox{-}Type,  Source\mbox{-}Events, Target\mbox{-}Events, Direction,\\ Target\mbox{-}Service,  Condition>$  ($Condition$ can be optionally specified). $Target\mbox{-}Type$ indicates the type of the rule, being either $Static$ or $Dynamic$ (explained further later). $Source\mbox{-}Events$ indicates the type of the causal link's source event ($Source\mbox{-}Events$ are listed in the names of the rules shown in Figures~\ref{fig:ex2_n1},~\ref{fig:ex2_n2} and~\ref{fig:dynamic_example}).   $Target\mbox{-}Events$ indicates the type of the causal link's target event. $Direction$ indicates the direction of the casual link between the target event and source event. $Target\mbox{-}Service$ indicates the service that the target event should belong to. Note that $Target\mbox{-}Service$ in $Static$ rules can be  $Self$, which indicates that the target event would be within the same service as the source event, or $Outgoing$/$Incoming$, which indicates that the target event would belong to the downstream/upstream services of the service that the source event belongs to in $G$.

\begin{figure}[t]
\centering
\includegraphics[width=0.56\columnwidth]{figures/example3.png}
\caption{Example of dynamic rule}
\label{fig:dynamic_example}
\end{figure}

There are two categories of special rules. The first category is \emph{dynamic} rules (i.e., rules whose $Target\mbox{-}Type$  is set to $Dynamic$) to support dynamic dependencies. Here $Target\mbox{-}Service$ does not indicate any of the three possible options listed earlier but indicates the name of the target service that \system would need to create. For example, live DB dependencies are not available due to different tech stacks and high volume. In Figure~\ref{fig:dynamic_example}, a DB issue (DB Markdown) is shown in \emph{Service-A}. Based on the listed \emph{dynamic} rule, \system creates a new ``service'' \emph{DB-1} in $G$, a new event ``Issues'' that belongs to \emph{DB-1}, and a causal link between the two events.  In practice, the SRE teams use dynamic rules to cover a lot of third-party services and database issues since the live dependencies are not easy to maintain.  %However through the internal error messages and dynamic rules, \system is still able to handle these dependencies. %we can still support external inferences. 

The second category of special rules is \emph{conditional} rules. \emph{Conditional} rules are used when some prerequisite conditions should be satisfied before a certain causal link is created. In these rules, $Condition$ is specified with a boolean predicate. As shown in Figure~\ref{fig:ex2_n2}, the SRE teams believe \emph{Latency Spike} events from different services are related only when both events happen within the same data center. Based on this observation, \system would first evaluate the predicate in $Condition$ and build only the causal link when the predicate is true. A conditional rule overwrites the basic rule on the same source-target event pair.

When constructing causal links, \system first applies the \emph{dynamic} rules so that dynamic dependencies and events are first created at once. Then for every event in the initial services (denoted as $I$), if the rule conditions are satisfied, one or many causal links are created from this event to other events from the same or upstream/downstream services. When a causal link is created, the step is repeated recursively for the target event (as a new origin) to create new causal links. After no new causal links are created, the construction of the event causality graph is finished.

% When \system constructs the causal links, \system first processes all dynamic rules as they may create new event nodes in the graph. %\system enumerates the dynamic rules on each existing event node and also on the newly added nodes (There could also be rules applicable to the newly added nodes) until no new event nodes can be created. 


%Each rule is defined as a predicate containing both events' property-value pair. If the predicate evaluates to be true between two events, then we would add the edge in the causality graph. For example, in Figure~\ref{fig:example1}, the rule used to establish the edge between \emph{GC overhead in RNO} and \emph{Latency increase in LVS, RNO, SLC} would be like this: Suppose we are now determining whether there should be a link from event $u$ to event $v$, then this rule would be $u.\text{pool} = v.\text{pool}\ and\ u.\text{type} = ``\text{High GC Overhead}"\ and\ v.\text{type} = ``\text{Latency increase}"\ and\ u.\text{center} \cap v.\text{center} \ne \emptyset$ which holds true for these two events. Each causality link is also associated with a weight which represents the likelihood of causality - we set all initial values as $1.0$. Overtime these value are updated by the statistical analysis result of the collected data set.


\subsection{Root Cause Ranking}
Finally, \system ranks and recommends the most probable root causes from the event causality graph. Similar to how search engines infer the importance of pages by page links, we customize the PageRank \cite{manning2010introduction} algorithm to calculate the root cause ranking; the customized algorithm is named as GrootRank. The input is the event causality graph from the previous step. Each edge is associated with a weighted score for weighted propagation. The default value is set as $1$, and is set lower for alerts with high false-positive rates. 

Based on the observation that dangling nodes are more likely to be the root cause, we customize the personalization vector as $P_n = f_n $ or $P_d = 1$, where $P_d$ is the personalization score for dangling nodes, and $P_n$ is for the remaining nodes; and $f_n$ is a value smaller than 1 to enhance the propagation between dangling nodes. In our work, the parameter setting is $f_n = 0.5$, $\alpha = 0.85$, $max_{iter} = 100$ (which are parameters for the PageRank algorithm). Figure \ref{fig:person} illustrates an example. The grey circles are the events collected from three services and one database. The grey arrows are the dependency links and the red ones are the causal links with the weight of $1$. Both of the PageRank and GrootRank algorithms detect $event 5$ (DB issue) as the root cause, which is expected and correct. However, the PageRank algorithm ranks $event 4$ higher than $event 3$. But $event 3$ of $\textit{Service-C}$ is more likely to be the second most possible root cause (besides $event 5$), because the scores on dangling nodes are propagated to all others equally in each iteration. We can see that $event 3$ is correctly ranked as second using the GrootRank algorithm.

The second step of GrootRank is to break the tied results from the previous step. The tied results are due to the fact that the event graph can contain multiple disconnected sub-graphs with the same shape. We design two techniques to untie the ranking: 
\begin{figure}[t]
\centering
  \includegraphics[width=0.8\columnwidth]{figures/personalvector.png}
  \caption{Example of personalization vector customization}
  \label{fig:person}
\end{figure}

\begin{figure}[t]
\centering
  \includegraphics[width=0.8\columnwidth]{figures/accessdistance.png}
  \caption{Example of using access distance to untie the ranking results}
  \label{fig:untie}
\end{figure}
\begin{enumerate}
\item For each joint event, the access distance (sum) is calculated from the initial anomaly service(s) to the service where the event belongs to. If any ``access'' is not reachable, the distance is set as $d_m+1$ where $d_m$ is the maximum possible distance. The one with shorter access distance (sum) would be ranked higher and vice versa. Figure \ref{fig:untie} presents an example, where \emph{Service-A} and \emph{Service-B} are both initial anomaly services. Since \system suspects that $event 2$ is caused by either $event 3$ or $event 1$ with the same weight. The scores of $event 3$ and $event 1$ are tied. Then, $event 3$ has a score of $1$ (i.e., $0+1$) and $event 1$ has a score of 2 (i.e., $0+2$), since it is not reachable by \emph{Service-B}). Therefore, $event 3$ is ranked first and logical. 
\item For the remaining joint results with the same access distances, \system continues to untie by using the historical root cause frequency of the event types under the same trigger conditions (e.g., checkout domain alerts). This frequency information is generated from the manually labeled dataset. A more frequently occurred root cause type is ranked higher.% than the less frequent ones.
\end{enumerate}


\subsection{Rule Customization Management}

While \system users create or update the rules,  there could be overlaps, inconsistencies, or even conflicts being introduced such as the example in Figure~\ref{fig:ex2_n2}. \system uses two graphs to manage the rule relationships and avoid conflicts for users. One graph is to represent the link rules between events in the same service (\emph{Same-Graph}) while the other is to represent links between different services (\emph{Diff-Graph}). The nodes in these two graphs are the event types defined in Section~\ref{sec:causality}. There are three statuses between each (directional) pair of event types: (1) no rule, (2) only basic rule, and (3) conditional rule (since it overwrites the basic rule). In \emph{Same-Graph}, \system does not allow self-loop as it does not build links between an event and itself.
% but it is possible that we build links between different services with the same event type.

When rule change happens, existing rules are enumerated to build edges in \emph{Same-Graph} and \emph{Diff-Graph} based on $Target\mbox{-}Events$ and $Target\mbox{-}Service$. Based on the users' operation of 
% \begin{itemize}
%     \item 
    (1) ``remove a rule'',  \system removes the corresponding edge on the graphs;
    % \item 
    (2) ``add/update a rule'',  \system checks whether there are existing edges between the given event types, and then warns the users for possible overwrites. 
    % The users can also combine the conditional rules.   % while users are adding basic rules between event types if there are existing conditional rules between them.
    If there are no conflicts, \system just adds/updates edges between the event types.
    % \item Add conditional rules. We would first alert the possible overwrite. Then if users are about to add new conditional rules on the top of existing conditional rules, we would ask the users to combine these two conditions to add a new one. We then build or change all corresponding edges to status 3.
% \end{itemize} 

After all changes, \system extracts the rules from the graphs by converting each edge to a single rule. These rules are automatically implemented, and then tested against our labeled data set. The \system users need to review the changes with validation reports before the changes go online.

% Note that currently we don't check the consistencies between dynamic rules as we cannot process the dynamic event types, but this could be solved in the future by using nodes with symbolic values to represent such event types. 
\newcommand{\twomoons}{{\tt Twomoons}}
\newcommand{\gauss}{{\tt Gauss}}
\newcommand{\sculpture}{{\tt Sculpture}}
\newcommand{\baseline}{{\tt Baseline}}
\newcommand{\MM}{{\tt MsgPassing}}
\newcommand{\blackboard}{{\tt Blackboard}}
\newcommand{\ncut}{\text{ncut}}
\newcommand{\chensays}[2][]{\textcolor{blue} {\textsc{Jiecao #1:} \emph{#2}}}

\section{Experiments}
In this section we present experimental results for  graph clustering in the message passing and blackboard models. We will compare the following three algorithms. (1) \baseline: each site sends all the data to the coordinator directly; (2) \MM: our algorithm in the message passing model (Section~\ref{sec:gcmessage}); (3) 
\blackboard: our algorithm in  the blackboard model (Section~\ref{sec:bb}).


%Since both of our algorithms are crucially based on the use of spectral scarification, our main focus in the experiments is to investigate to what extend the quality of the spectral clustering algorithms will be affected by using spectral sparsification, the saving of communication costs by using spectral sparsificaion, ...
%
%
%The goal of this experiment is not to demonstrate the effectiveness of the spectral clustering algorithm. We mainly want to investigate the following, 
%\begin{itemize}
%\item to what extend the quality of clustered results will be affected by using spectral sparsification.
%\item saving of communication costs by using spectral sparsifier.
%\item the affect of constants in algorithms of the message passing/blackboard model.
%\end{itemize}
%
%
%\subsection{The Setup}
%\paragraph{Reference Algorithms}
%We compare different algorithms in our experiment.

%Note that we can also run \MM~ in the blackboard model.

Besides giving the visualized results of these algorithms on various datasets, we also measure the qualities of the results via the {\em normalized cut}, defined as 
\[
\ncut(A_1, \ldots, A_{k}) = \frac{1}{2}\sum_{i\in[k]}\frac{w(A_i, V\backslash A_i)}{\vol(A_i)},
\]
 which is a standard objective function to be minimized for spectral clustering algorithms. 
%We will compare the communication costs of these algorithms in different settings.

%We also compare the total communication costs of different algorithms/models. As the unit does not matter in our case, we normalize all communication costs by the cost of \baseline.  Whenever possible, we will visualize the clustered results.

We implemented the algorithms using multiple languages, including Matlab, Python and C++. Our experiments were conducted on an IBM NeXtScale nx360 M4 server, which is equipped with 2 Intel Xeon E5-2652 v2 8-core processors, 32GB RAM and 250GB local storage.


\subsection{Datasets.}
We test the algorithms in the following real and synthetic datasets, which is visualized in \figref{visualization}.


\begin{figure}[h]
     \centering
     \subfigure[\twomoons]{\includegraphics[width=0.23\textwidth]{twomoons-14000-original.png}\label{fig:twomoons}}
     ~~
     \subfigure[\gauss]{\includegraphics[width=0.23\textwidth]{gauss-10000-original.png}\label{fig:gauss}}
     ~~
     \subfigure[\sculpture]{\includegraphics[width=0.13\textwidth,height=0.16\textwidth]{sculpture-11680-original.jpg}\label{fig:sculpture}}
     \caption{Visualization of the datasets for our experiments.}
     \label{fig:visualization}
\end{figure}



\vspace{-1mm}
\begin{itemize}
\item \twomoons : this dataset contains $n=14,000$ coordinates in $\mathbb{R}^2$. We consider each point to be a vertex. For any two vertices $u, v$, we add an edge with weight $w(u,v) = \exp\{-\|u-v\|_2^2/\sigma^2\}$ with $\sigma = 0.1$ when one vertex is among the $7000$-nearest points of the other.  This construction results in a graph with about $110,000,000$ edges.

\item  \gauss : this dataset contains $n = 10,000$ points in $\mathbb{R}^2$. There are $4$ clusters in this dataset, each generated using a Gaussian distribution. We construct a complete graph as the similarity graph.  For any two vertices $u, v$, we define the weight $w(u,v) = \exp\{-\|u-v\|_2^2/\sigma^2\}$ with $\sigma = 1$. The resulting graph has about $100,000,000$ edges.

\item \sculpture : a photo of \textit{The Greek Slave}~\footnote{Available in e.g., \url{http://artgallery.yale.edu/collections/objects/14794}}. We use an $80\times 150$ version of this photo where each pixel is viewed as a vertex. To construct a similarity graph, we map each pixel to a point in $\mathbb{R}^5$, i.e., $(x, y, r, g, b)$, where the latter three coordinates are the RGB values. For any two vertices $u, v$, we  put an edge between $u, v$ with weight $w(u,v) = \exp\{-\|u-v\|_2^2/\sigma^2\}$ with $\sigma = 0.5$ if one of $u, v$ is among the $5000$-nearest points of the other. This results in a graph with about $70,000,000$ edges.
\end{itemize}
\vspace{-1mm}
In the distributed model edges are randomly partitioned across $s$ sites. 

%\vspace{-1.5mm}



\subsection{Results on clustering quality}
%{\em Quality.} \
\begin{figure*}[ht]
     \centering
     \subfigure[\baseline]{\includegraphics[width=0.2\textwidth]{twomoons-14000-original-clustered.png}\label{fig:twomoons-clustered-original}}
     \subfigure[\MM]{\includegraphics[width=0.2\textwidth]{twomoons-14000-sparsify-clustered-15.png}\label{fig:twomoons-clustered-sparsify}}
     \subfigure[\blackboard]{\includegraphics[width=0.2\textwidth]{twomoons-14000-chain-clustered.png}\label{fig:twomoons-clustered-chain}}
     \caption*{\twomoons, $k = 2$;}

\subfigure[\baseline]{\includegraphics[width=0.2\textwidth]{gauss-10000-original-clustered.png}\label{fig:gauss-clustered-original}}
     \subfigure[\MM]{\includegraphics[width=0.2\textwidth]{gauss-10000-sparsify-clustered-15.png}\label{fig:gauss-clustered-sparsify}}
     \subfigure[\blackboard]{\includegraphics[width=0.2\textwidth]{gauss-10000-chain-clustered.png}\label{fig:gauss-clustered-chain}}
     \caption*{\gauss, $k = 4$}


     \subfigure[\baseline]{\includegraphics[width=0.2\textwidth,height=0.2\textwidth]{sculpture-11680-original-clustered.png}\label{fig:sculpture-clustered-original}}  
     \subfigure[\MM]{\includegraphics[width=0.2\textwidth,height=0.2\textwidth]{sculpture-11680-sparsify-clustered-15.png}\label{fig:sculpture-clustered-sparsify}}
     \subfigure[\blackboard]{\includegraphics[width=0.2\textwidth,height=0.2\textwidth]{sculpture-11680-chain-clustered.png}\label{fig:sculpture-clustered-chain}}
     \caption*{\sculpture, $k = 3$. }


     
     \caption{Visualization of the results on \twomoons, \gauss\ and \sculpture. In the message passing model each site samples $5 n$ edges; in the blackboard model all sites jointly sample $10n$ edges (in \twomoons~ and \gauss) or $20n$ edges (in \sculpture) and the chain has length $18$. $s = 15$.}
     \label{fig:quality-1}
\end{figure*}

We visualize the clustered results for 
the \twomoons, \gauss\ and \sculpture\ in Figure~\ref{fig:quality-1}.
% and visualize the clustered results for \gauss\ and \sculpture in Figure~\ref{fig:quality-2}.
It can be seen that \baseline, \MM\ and \blackboard\ give results of very similar qualities.  For simplicity, here we only present the visualization for $s=15$. Similar results were observed when we varied the values of $s$.  
%\he{To Qin: Do you plan to have two titles (Results \& Quality)?}


% \begin{figure*}[h]
%      \centering
% \subfigure[\baseline]{\includegraphics[width=0.3\textwidth]{gauss-10000-original-clustered.png}\label{fig:gauss-clustered-original}}
%      \subfigure[\MM]{\includegraphics[width=0.3\textwidth]{gauss-10000-sparsify-clustered-15.png}\label{fig:gauss-clustered-sparsify}}
%      \subfigure[\blackboard]{\includegraphics[width=0.3\textwidth]{gauss-10000-chain-clustered.png}\label{fig:gauss-clustered-chain}}
%      \caption*{\gauss, $k = 4$}


%      \subfigure[\baseline]{\includegraphics[width=0.2\textwidth]{sculpture-11680-original-clustered.png}\label{fig:sculpture-clustered-original}}  
%      \subfigure[\MM]{\includegraphics[width=0.2\textwidth]{sculpture-11680-sparsify-clustered-15.png}\label{fig:sculpture-clustered-sparsify}}
%      \subfigure[\blackboard]{\includegraphics[width=0.2\textwidth]{sculpture-11680-chain-clustered.png}\label{fig:sculpture-clustered-chain}}
%      \caption*{\sculpture, $k = 3$. }

%      \caption{Visualization of results on \gauss\ and \sculpture; in the message passing model each site samples $5 n$ edges; in the blackboard model all sites jointly sample $10n$ (in \gauss) or $20n$ (in \sculpture) edges and the chain has length $18$.}
%      \label{fig:quality-2}
% \end{figure*}


We also compare the normalized cut (ncut) values of the clustering results of different algorithms.  The results are presented in Figure \ref{fig:quality}. In all datasets, the ncut values of different algorithms are very close. The ncut value of \MM\ slightly decreases when we increase the value of $s$, while the ncut value of \blackboard\ is independent of $s$.
%We comment that in general, it is difficult to compare \MM\ and \blackboard\ directly because they are affected by different parameters.


\begin{figure*}[!ht]
  \centering
  \subfigure[\twomoons]{\includegraphics[width=0.33\textwidth]{twomoons-14000-ncut.png}\label{fig:twomoons-quality}}\hspace*{-1.1em}
  \subfigure[\gauss]{\includegraphics[width=0.31\textwidth]{gauss-10000-ncut.png}\label{fig:gauss-quality}}\hspace*{-1.1em}
  \subfigure[\sculpture]{\includegraphics[width=0.31\textwidth]{sculpture-11680-ncut.png}\label{fig:sculpture-quality}}\hspace*{-1.1em}
  \subfigure{\includegraphics[width=0.14\textwidth]{legend.png}}
     \caption{Comparisons on normalized cuts. In the message passing model, each site samples $5n$ edges; in each round of the algorithm in the blackboard model, all sites jointly sample $10n$ edges (in \twomoons~and \gauss) or $20n$ edges (in \sculpture) edges and the chain has length $18$.}
     \label{fig:quality}
\end{figure*}

%\textcolor{red}{To Jiecao: Can you put the color lines indicating baseline, message passing, and blackboard within one row in Pic 2? Withthis we can save some space.}

%\vspace{-1.5mm}

\subsection{Results on communication costs} 
\begin{figure*}[!ht]
     \centering
     \subfigure[\twomoons]{\includegraphics[width=0.3\textwidth]{twomoons-14000-communication.png}\label{fig:twomoons-communication}}
     \subfigure[\gauss]{\includegraphics[width=0.3\textwidth]{gauss-10000-communication.png}\label{fig:gauss-communication}}
     \subfigure[\sculpture]{\includegraphics[width=0.3\textwidth]{sculpture-11680-communication.png}\label{fig:sculpture-communication}}


     \subfigure[\twomoons]{\includegraphics[width=0.32\textwidth]{twomoons-14000-communication-2.png}\label{fig:twomoons-communication-2}}
     \subfigure[\gauss]{\includegraphics[width=0.32\textwidth]{gauss-10000-communication-2.png}\label{fig:gauss-communication-2}}
     \subfigure[\sculpture]{\includegraphics[width=0.32\textwidth]{sculpture-11680-communication-2.png}\label{fig:sculpture-communication-2}}
     \caption{Comparisons on communication costs. In the message passing model, each site samples $5n$ edges; in each round of the algorithm in the blackboard model, all sites jointly sample $10n$ (in \twomoons~and \gauss) or $20n$ (in \sculpture) edges and the chain has length $18$. }
     \label{fig:communication}
\end{figure*}

We compare the communication costs of different algorithms in Figure \ref{fig:communication}. We observe that while achieving similar clustering qualities as \baseline, both \MM\ and \blackboard\ are significantly more communication-efficient (by one or two orders of magnitudes in our experiments). We also notice that the value of $s$ does not affect the communication cost of \blackboard, while the communication cost of \MM\ grows almost linearly with $s$; when $s$ is large, \MM\ uses significantly more communication than \blackboard. These confirm our theory.  %In Figure~\ref{fig:mm-const} and Figure~\ref{fig:blackboard-const}   in Appendix~\ref{sec:parameters} we present how the performance of \MM\ and \blackboard\ are affected by their parameters.

%
%
%\vspace{-1.5mm}
%\paragraph{Summary.}  From our experimental results we conclude that \MM\ and \blackboard\ achieve similar clustering quality as the native algorithm \baseline, while significantly reduce the communication cost.  When the number of sites is large, \blackboard\ is more communication efficient than \MM, as predicted by our theory.



\subsection{Parameters in \MM\ and \blackboard}
\label{sec:parameters}

Figure \ref{fig:mm-const} shows in \MM how the value of ncut is affected by the number of sites and the number of edges sampled in each site. 
Here, each site samples $cn$ edges. 
When $c=3$ and $s=1$, the ncut value diverges in all datasets. This is because with such a small $c$, the algorithm does not generate a valid sparsifier. In general, increasing $c$ or $s$ will slightly decrease the ncut value. But once they are above some thresholds, the ncut values of \MM\ and \baseline\ become very close.

Figure \ref{fig:blackboard-const} shows in \blackboard  how the ncut value is affected by the number of iterations and the number of edges sampled. When the number of iterations is set to be $5$, ncut values diverge in all datasets. This is because we cannot expect to generate a valid sparsifier by using such few iterations. It can be seen from \ref{fig:bb-gauss-constant} that for a fixed $c$, performing more iterations will help to reduce ncut values. From the same figure, one can also conclude that for fixed iterations, increasing $c$ also helps to reduce the ncut values.



\begin{figure*}[h!t]
     \centering
     \subfigure[\twomoons]{\includegraphics[width=0.3\textwidth]{twomoons-c.png}\label{fig:mm-twomoons-constant}}
     \subfigure[\gauss~dataset]{\includegraphics[width=0.3\textwidth]{gauss-c.png}\label{fig:mm-gauss-constant}}
     \subfigure[\sculpture]{\includegraphics[width=0.3\textwidth]{sculpture-c.png}\label{fig:mm-sculpture-constant}}
     \caption{The pictures above show the $\ncut$ values with respect to the values of $c$ and $s$ for the \MM\ algorithm. Here  
 each site samples $c n$ edges.}
     \label{fig:mm-const}
\end{figure*}


\begin{figure*}[h!t]
     \centering
     \subfigure[\twomoons]{\includegraphics[width=0.3\textwidth]{twomoons-iter.png}\label{fig:bb-twomoons-constant}}
     \subfigure[\gauss]{\includegraphics[width=0.3\textwidth]{gauss-iter.png}\label{fig:bb-gauss-constant}}
     \subfigure[\sculpture]{\includegraphics[width=0.3\textwidth]{sculpture-iter.png}\label{fig:bb-sculpture-constant}}
     \caption{The pictures above show how the $\ncut$ values are affected by the number of iterations and the value of $c$ for the \blackboard\ algorithm. Here 
all sites jointly sample $c n$ edges. }
     \label{fig:blackboard-const}
\end{figure*}







\section{Discussion and Conclusions}



Our method based on stabilizing forward and backward pass, resulted in improved accuracy over the baseline and it was able to predict optimal dampening, sharpness and tail-fatness before training. 
Our findings are coherent with the line of research that has established that stabilizing gradients and representations at initialization results in better performance \cite{glorot2010understanding, orthogonal_initialization, he2015delving, roberts2022principles, defazio2022scaling, bengio1994learning, hochreiter1997long, hochreiter2001gradient, arjovsky2016unitary, pascanu2013difficulty}. Moreover it gives an initial reply to the question raised by
\cite{surrogate2019, zenke2021remarkable}, which asked  for a theoretical justification of initialization and SG choice for Spiking Neural Networks. With a similar intention, \cite{rossbroich2022fluctuation} proposed an approach that guarantees sparsity of activity at initialization to pick the weights distribution at initialization, resulting in improved accuracy. Our method differs from theirs in that it starts from a principle of stability to derive constraints, instead of a principle of sparsity. It differs also in that we use it to define the SG shape at initialization, not only the weights distribution, and we can show mathematically how weights initialization is intertwined to the SG shape choice. Our results suggest that a tedious hyper-parameter grid-search can be often avoided by making use of sound and established principles of learning.

One of the conditions was designed to hit the most sensitive part of an SG, its center, which resulted in a low sparsity requirement at initialization. This is very uncommon in the Neuromorphic literature, since sparsity brings large energy gains \cite{henderson2020towards,blouw2019benchmarking, 9395703,taulsnn, rossbroich2022fluctuation}.
However, the energy gains of SNNs also come from their binary activity. A matrix-vector multiplication, with a $\mathbb{R}^{m\times n}$ matrix, has an energy cost of $mnE_{MAC}$ for a real vector, and of $mn\rho E_{AC}$ for a binary vector, where $\rho$ is the Bernouilli probability of the binary vector, and in our case the neuron firing rate, and $E_{AC}, E_{MAC}$ are the energies of an accumulate and a multiply-accumulate operation \cite{yin2021accurate, hunger2005floating}. Since MAC are more costly than AC, 31 times on a $45$nm complementary metal–oxide–semiconductor \cite{yin2021accurate, horowitz20141}, we have energy savings with any $\rho$, e.g., when all neurons fire ($\rho=1$) and when they fire half of the time steps ($\rho=1/2$). This gain does not depend on the simulation speed, since it compares a spiking and an analogue computation, at the same computation speed.
Typically requiring more sparsity through a sparsity encouraging loss term, leads to a measurable decrease in performance \cite{zenke2021remarkable, rossbroich2022fluctuation}. However we observed that it is actually possible to achieve higher performance with higher sparsity, by starting with a strong firing rate at initialization, since their synergy acts as a regularization mechanism. This was possible also because the sparsity encouraging loss term was introduced gradually, and because its contribution was kept comparable to the task loss towards the end of training.

We observed that the more complex the task is and the more complex the network to train is, the more drastic is the difference in performance of different SG shapes. It is known that learning is possible with a wide variety of SG shapes \cite{zenke2021remarkable} and the community has not yet settled for one shape or one method to reliably choose which SG to use in each case \cite{surrogate2019}. We showed how to apply a well known stability principle to the forward and backward pass of the simplest Spiking Neural Network, the LIF, as a starting point, but we think that the principles of good Neuromorphic initialization can be further elaborated, in order to tackle more complex tasks and networks.




\clearpage

\bibliographystyle{splncs}
\bibliography{egbib}
\end{document}
