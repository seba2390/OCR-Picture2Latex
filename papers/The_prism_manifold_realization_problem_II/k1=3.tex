\subsection{{\boldmath $k_1=3$}}\label{k1=3}

In this subsection we focus on the changemaker C-type lattices with 
\begin{equation}\label{x0k1=3}
x_0=e_0+e_3+e_{k_2} - e_{k_3}.
\end{equation}  
Recall that the changemaker starts with $(1,2,2,3)$.

\begin{lemma}\label{lem:k1=3 intervals}
		The intervals $[v_3]$ and $[v_1]$ share their right endpoint and $\epsilon_3 = \epsilon_1$. Moreover, $[v_2]$ abuts the right endpoint of $[v_1]$ and $[v_3]$.
\end{lemma}
\begin{proof}
Since $|v_3|=3$ and $\braket{v_1}{v_3}=2$, from Lemma~\ref{intervalproduct}, it must be the case that $\epsilon_1=\epsilon_3$ and $\delta([v_1], [v_3])=1$. The first statement of the lemma is now immediate because $v_3$ is orthogonal to $x_0$. Since $\braket{v_2}{v_1}\ne0$ and $\braket{v_2}{x_0}=0$, $[v_2]$ abuts the right endpoint of $[v_1]$.
\end{proof}


\begin{cor}\label{cor:k1=3 intervals}
		Suppose that there exists a vector $v_j$ such that $j>3$, $j\neq k_3$, and $\braket{v_j}{v_1} = 2$. Then $j=4$, and that $v_4 = e_0 + e_3-e_4$. 
\end{cor}

\begin{proof}
Suppose that $j$ is such an index. Therefore, $0\in \supp^+(v_j)$ and $1\not \in \supp^+(v_j)$. (This, in particular, implies that $|v_j|\ge3$). We claim that $\braket{v_j}{x_0}\neq 0$. Otherwise, assume $\braket{v_j}{x_0}= 0$. Since $\braket{v_j}{v_1}=2$, $x_{z_j}\in[v_1]$. Using Lemma~\ref{lem:k1=3 intervals} and Corollary~\ref{Cor:ZjDistinct}, $[v_1]$ contains at least $3$ high weight vertices $x_1,x_{z_j},x_{z_3}$, and $\delta([v_j],[v_1])=2$. Since $|v_1|=5$, we have $|x_{z_j}|=3$, so by Lemma~\ref{intervalproduct} we have $|\braket{v_j}{v_1}|=1$, a contradiction.
This justifies the claim, and therefore, $\braket{v_j}{x_0}=2$. Since $|v_j|\ge 3$ and $\delta([v_1], [v_j])\le 3$, to get $\braket{v_j}{v_1}=2$, we must have $\epsilon_1=\epsilon_j$. Thus, $\delta([v_j], [v_1])=1$ and $|v_j|=3$. That is, $v_j=e_0+e_{j-1}-e_{j}$. We now argue that $j=4$. Suppose for contradiction that $j>4$. Thus $\braket{v_j}{v_3}=1$. Using Lemma~\ref{lem:k1=3 intervals}, we get that the interval $[v_1]$ equals the union of $[v_j]$ and $[v_3]$. Since $|v_j|=|v_3|=3$, we get that $|v_1|=4$, which is a contradiction. 
\end{proof}

\begin{lemma}\label{lem:v1Orth}
Let $v_j$ be a vector such that $j>3$, $j\neq k_3$. Then $\braket{v_j}{v_1}\in\{0,2\}$. As a result, $\min\supp(v_j)\ge2$ unless $j=4$ and $v_4 = e_0 + e_3-e_4$.
\end{lemma}
\begin{proof}
Assume that $\braket{v_j}{v_1}\notin\{0,2\}$, then $\supp(v_j)\cap\{0,1\}=\{1\}$ or $\{0,1\}$. By Lemma~\ref{gappy3}, $2\in \supp(v_j)$. 
If $0\in\supp(v_j)$, since $|\braket{v_j}{v_3}|\le1$ by Corollary~\ref{unbreakablepairing}, we have $3\in\supp(v_j)$. Thus $|v_j|\ge5$. Since $\braket{x_0}{v_j}=2$,  $x_{z_j}=x_1$. By Corollary~\ref{Cor:ZjDistinct} and Lemma~\ref{lem:k1=3 intervals}, $x_{z_j}\ne x_{z_3}$. So \[5=|v_1|\ge|x_{z_j}|+|x_{z_3}|-2\ge5+1,\]
a contradiction.

We have shown that $0\notin\supp(v_j)$. If $3\notin\supp(v_j)$, then $j>4$ and $|v_j|\ge4$. As $\braket{v_j}{v_3}=1$, using Corollary~\ref{Cor:ZjDistinct}, $[v_j]$ and $[v_3]$ are consecutive. By Lemma~\ref{lem:k1=3 intervals} and the fact that  $\braket{v_j}{v_2}=0$ we conclude that $[v_j]\subset[v_1]$. Since $\braket{v_j}{x_0}=0$, $[v_1]$ contains at least three high weight vertices: $x_1, x_{z_j}, x_{z_3}$. This is impossible as $|v_1|=5$ and $|v_j|\ge4$.

Now we have $\supp(v_j)\cap\{0,1,2,3\}=\{1,2,3\}$, so $\braket{v_j}{v_3}=0$. By  Lemma~\ref{lem:j-1}, $|v_j|\ge5$ unless $j=4$. By Lemma~\ref{lem:k1=3 intervals} and the fact that  $\braket{v_j}{v_1}\ne0$ we conclude that $[v_j]\subset[v_1]$. So $[v_1]$ contains at least two high weight vertices: $x_{z_j}, x_{z_3}$. It follows that $|v_j|\le4$. So $j=4$ and $|v_4| = e_1+e_2 + e_3-e_4$. Since $|v_4|=4$, $[v_1]$ contains exactly two high weight vertices, so $x_1$ must be $x_{z_4}$. So $\braket{v_4}{x_0}\ne0$, which is not possible. This shows that $\braket{v_j}{v_1}\in\{0,2\}$.

If $\min\supp(v_j)<2$, then $\braket{v_j}{v_1}\ne0$. We must have $\braket{v_j}{v_1}=2$, so $j=4$ and $v_4 = e_0 + e_3-e_4$ by Corollary~\ref{cor:k1=3 intervals}.
\end{proof}

\begin{lemma}\label{lem:NoSingle}
Let $v_j$ be a vector such that $j>4$, $j\neq k_3$. Then $\supp(v_j)\cap\{0,1,2,3\}\ne\{2\}$ or $\{3\}$.
\end{lemma}
\begin{proof}
Assume that $\supp(v_j)\cap\{0,1,2,3\}$ contains only one element which is $2$ or $3$.
Then $|v_j|\ge3$, $\braket{v_{j}}{v_3}\ne0$ while $\braket{v_{j}}{v_1}=0$. By Lemma~\ref{lem:k1=3 intervals}, $[v_{j}]$ abuts the left endpoint of $[v_3]$, so $[v_j]\subset[v_1]$. Since $|v_j|\ge3$ and $\delta([v_j],[v_1])\le3$, using Lemma~\ref{intervalproduct}, we get that $\braket{v_{j}}{v_1}\ne 0$ unless $|v_j|=\delta([v_j],[v_1])=3$. However, if $\delta([v_j],[v_1])=3$, $[v_1]$ is contained in the union of $[v_j],[v_3]$ and $\{x_0\}$. Since $|v_j|=|v_3|=3$, we have $|v_1|=4$, a contradiction.
\end{proof}

\begin{lemma}\label{lem:k1=3 sigma3}
		$\sigma_4 \in \{3,4,5\}$. Furthermore, if $\sigma_4 = 3$ then $[v_4]$ abuts the left endpoint of $[v_3]$. If $\sigma_4 = 4$ then $[v_4]$ and $[v_1]$ share their left endpoint.% If $\sigma_4 = 5$ then $[v_1]\dagger[v_2]\dagger[v_4]$. 
\end{lemma}

\begin{proof}
If  $\min\supp(v_4)<2$,  using Lemma~\ref{lem:v1Orth},  $\sigma_4=4$. By Lemma~\ref{lem:j-1}, if $\min\supp(v_4)\ge2$, $v_4=e_2+e_3-e_4$ or $e_3-e_4$. So $\sigma_4=5$ or $3$.

When $\sigma_4=3$, $[v_4]$ abuts $[v_3]$ and $\braket{v_4}{v_1}=0$. By Lemma~\ref{lem:k1=3 intervals}, $[v_4]$ abuts the left endpoint of $[v_3]$.
When $\sigma_4 = 4$, $\braket{v_4}{v_1}=2=\braket{v_4}{x_0}$. So $\delta([v_4],[v_1])=1$ by Lemma~\ref{intervalproduct}. Thus  $[v_4]$ and $[v_1]$ share their left endpoint by Lemma~\ref{lem:k1=3 intervals}.
\end{proof}



\begin{prop}\label{prop1:k1=3}
If $\sigma_4=3$, the initial segment $(\sigma_0, \cdots, \sigma_{k_3})$ of $\sigma$ is $(1,2,2,3,3,7)$. 
\end{prop}
\begin{proof}
Suppose that $\sigma_4=3$ (see Lemma~\ref{lem:k1=3 sigma3}). This implies that $k_2=4$ (Lemma~\ref{Lem:X0Odd}). Using Equation~\eqref{x0k1=3}, we get that $\sigma_{k_3}=7$. If $k_3\not = 5$, using Lemma~\ref{Lem:X0Odd} and  Lemma~\ref{lem:v1Orth}, we must have $S_5\supset\{3,4\}$. By Lemma~\ref{lem:NoSingle}, we have $2\in S_5$, so $\braket{v_5}{x_0}=2$ and $v_5\sim v_2$. By Lemma~\ref{lem:k1=3 intervals}, $[v_1]$ is contained in the union of $x_0,[v_5],[v_2]$. So $|v_1|=|v_5|=4$, which is not possible.
\end{proof}

\begin{prop}\label{prop2:k1=3}
If $\sigma_4\neq 3$, the initial segment $(\sigma_0, \cdots, \sigma_{k_3})$ of $\sigma$ is $(1,2,2,3,4^{[s]},4s+5, 4s+9)$, $s\ge0$. 
\end{prop}

\begin{proof}
Suppose that $\sigma_4\neq 3$ (see Lemma~\ref{lem:k1=3 sigma3}). Furthermore, let $s\ge 0$ satisfy that $\sigma_i=4$ for any $4\le i < s+4$, and that $\sigma_{s+4}>4$. We have $k_2\ge s+4$ by Lemma~\ref{Lem:X0Odd}.
 Set $j=\min \supp(v_{s+4})<s+3$. Then $j\ge2$ by Lemma~\ref{lem:v1Orth}.  Also, $j \ne 3$ by Lemma~\ref{lem:NoSingle}.
If $4<j<s+3$, we will get a claw $(v_j, v_{j-1}, v_{s+4}, v_{j+1})$, and if $j=4$, the claw will be on $v_4, x_0, v_{s+4}, v_5$. This proves that $j=2$. By Lemma~\ref{lem:NoSingle}, $3\in \supp(v_{s+4})$. 

We will show that $\sigma_{s+4}=4s+5$. If $s=0$, $v_4=e_2+e_3-e_4$, and we are done. If $s>0$, since $2,3\in\supp(v_{s+4})$,
$|v_{s+4}|\ge 4$. Also, $v_{s+4}$ must be orthogonal to $v_4$, as otherwise, using Lemmas~\ref{lem:k1=3 sigma3}~and~\ref{lem:k1=3 intervals}, all the three intervals $[v_4], [v_{s+4}]$, and $[v_3]$ will be subsets of $[v_1]$, which implies that $|v_1|\ge 6$, a contradiction. That is, $4\in \supp(v_{s+4})$. Using Lemma~\ref{gappy3}, $v_{s+4}$ is just right and $\sigma_{s+4}=4s+5$. 

Using Lemma~\ref{Lem:X0Odd}, we see that $k_2=s+4$. By Equation~\eqref{x0k1=3}, we have $\sigma_{k_3}=4s+9$. Note that $k_2\in \supp(v_{k_2+1})$. Since the unbreakable vector $v_4$ has nonzero inner product with $x_0$, using Corollary~\ref{x0pairing}, we get that $k_3=k_2+1$.  
\end{proof}

\begin{prop}\label{1k1=3}
If $(\sigma_0, \cdots, \sigma_{k_3})=(1,2,2,3,3,7)$, then $n+1=k_3$ (i.e. $v_{k_3}$ is the last standard basis vector).
\end{prop}
\begin{proof}
We claim that the index $k_3+1$ (that is, $6$) does not exist. Using Lemmas~\ref{Lem:X0Odd}, \ref{lem:j-1}, \ref{gappy3}, and~\ref{lem:v1Orth}, $S'_6=\{4, 5\}$. Then $\braket{v_6}{v_4}\not = 0$, and also $v_6$ is orthogonal to $x_0$. Using Lemmas~\ref{lem:k1=3 sigma3}~and~\ref{lem:k1=3 intervals}, we must have $[v_6]\subset [v_1]$ which implies that $\braket{v_6}{v_1}\ne 0$ since $|v_6|\ge 3$. This contradicts Lemma~\ref{lem:v1Orth}.
\end{proof}

\begin{figure}
	\begin{align*}
	\xymatrix@R1.5em@C1.5em{
		& x_0^{(4)} \ar@{=}[d]^{+} \\
		v_2^{(2)} \ar@{-}[r] \ar@{-}[dr] & v_1^{(5)} \ar@{=}[d]^{+} \\
		& v_{3}^{(3)}\ar@{-}[d]\\
                & v_4^{(2)}\\
	}
	&&
	\xymatrix@R1.5em@C1.5em{
		& x_0^{(4)} \ar@{=}[d]^{+} \ar@{=}[dl]^{+} & \\
		v_4^{(3)}\ar@{=}[r]^{+}\ar@{-}[d] & v_1^{(5)} \ar@{=}[r]^{+} \ar@{-}[d] & v_3^{(3)} \ar@{-}[dl] \\
		v_5^{(2)}\ar@{-}[d] & v_{2}^{(2)} \ar@{-}[d] &  \\
		\vdots & v_{s+4}^{(s+3)} & \\
		v_{s+3}^{(2)}\ar@{-}[u] 
	}
	\end{align*}
	\caption{Pairing graphs when $(\sigma_0, \cdots, \sigma_{k_3})$ is $(1,2,2,3,3,7)$ (left) or $(1,2,2,3,4^{[s]},4s+5, 4s+9)$, $s>0$ (right).}
	\label{fig:case1 1223 G(S)}
\end{figure}

\begin{prop}\label{2k1=3}
		If $(\sigma_0, \cdots, \sigma_{k_3})=(1,2,2,3,4^{[s]},4s+5, 4s+9)$, $s\ge0$, then $v_{s+6}=e_{s+4}+e_{s+5}-e_{s+6}$ if it exists, and $|v_i|=2$ for $i>s+6$. In this case, $\sigma = (1,2,2,3,4^{[s]},4s+5, 4s+9, (8s+14)^{[t]})$, $t\ge0$.
\end{prop}
	
\begin{proof}
Suppose that $\ell>k_3=s+5$ is an index such that $S_{\ell}\ne\emptyset$. We will prove that $\ell=s+6$ and $S_{\ell}=\{s+4,s+5\}$. This, together with Lemma~\ref{lem:AllNorm2}, will imply our desired result.

By Lemmas~\ref{Lem:X0Odd},~\ref{lem:v1Orth}, and Corollary~\ref{x0pairing},  $S'_{\ell}$ is one of $\emptyset,\{3,4\},\{3,5\}$ and $\{4,5\}$ if $s=0$, and
one of $\emptyset$, $\{ 3, s+5\}$, and $\{s+4,s+5\}$ if $s>0$. Let $j=\min\supp(v_{\ell})$, then $j\ge2$ by Lemma~\ref{lem:v1Orth}.  Also, $j \ne 3$ by Lemma~\ref{lem:NoSingle}.

If $s=0$ and $S'_{\ell}=\{3,4\}$, we have $\braket{v_{\ell}}{x_0}=2$ and $|v_{\ell}|\ge4$, so $x_1\in[v_{\ell}]$. Using Lemma~\ref{intervalproduct}, we get $\braket{v_{\ell}}{v_1}\ne0$, a contradiction.

If $S'_{\ell}=\{ 3, s+5\}$, to avoid $\braket{v_{\ell}}{v_{s+4}}>1$, $2\notin\supp(v_{\ell})$. Thus, $j=3$, which is impossible.

Having proved $S'_{\ell}=\emptyset$ or $\{s+4, s+5\}$, we claim that $S_{\ell}=S_{\ell}'$. First, $j\ne2$ by Lemma~\ref{lem:NoSingle}. So our claim holds when $s=0$. When $s>0$,
if $4\le j<s+3$, we have a claw $(v_j,v_{j-1},v_{j+1},v_{\ell})$. If $j=s+3$, $\braket{v_{\ell}}{v_{s+3}}\neq 0$. By Lemma~\ref{lem:k1=3 sigma3}, $[v_4]$ and $[v_1]$ share their left endpoint. Since $|v_5|=\cdots=|v_{s+3}|=2$ and $v_4\sim v_5 \sim \cdots \sim v_{s+3}$, we have $[v_{\ell}]\subset[v_1]$ by Lemma~\ref{lem:k1=3 intervals}. Thus $\braket{v_{\ell}}{v_1}\ne 0$ by Lemma~\ref{intervalproduct}, a contradiction. So our claim is proved.

Now by Lemma~\ref{lem:j-1}, $s+5\in S_{s+6}$. So $S_{s+6}=\{s+4,s+5\}$ by the results in the previous two paragraphs. If there was $\ell>s+6$ satisfying $S_{\ell}=\{s+4,s+5\}$, we would have a heavy triple $(v_{s+4},v_{s+6},v_{\ell})$. Thus  $S_{\ell}=\emptyset$ whenever $\ell>s+6$.
\end{proof}


%We start by claiming that $\sigma_{s+6}=8s+14$, or equivalently, $v_{s+6}=e_{s+4}+e_{s+5}-e_{s+6}$. By combining the fact that $\braket{v_4}{x_0}=2$ with Lemma~\ref{Lem:X0Odd} and Corollary~\ref{x0pairing}, we see that $S'_{s+6}$ is either $\{0, k_3\}$, $\{k_1, k_3\}$, or $\{k_2, k_3\}$. If $S'_{s+6}=\{0, k_3\}$, by Corollary~\ref{cor:k1=3 intervals} and Lemma~\ref{gappy3}, we get $1,2\in S_{s+6}$. This implies that $\braket{v_{s+6}}{v_3}=2$, which contradicts Corollary~\ref{unbreakablepairing}. Suppose that $S'_{s+6}=\{ k_1, k_3\}$. To avoid $\braket{v_{s+6}}{v_{s+4}}>1$, neither $2$ nor any of $4, \cdots, s+3$ is in $\supp(v_{s+6})$. Thus, $\braket{v_{s+6}}{v_{s+4}}\neq 0$ and $\braket{v_{s+6}}{v_3}\neq 0$, and there will be a heavy triple $(v_3, v_{s+4}, v_{s+6})$: see Figure~\ref{fig:case1 1223 G(S)}. Therefore, $S'_{s+6}=\{k_2, k_3\}$. We claim that $S_{s+6}=S'_{s+6}$. First note that $v_{s+6}$ must be orthogonal to $v_1$. If not, then $1\in S_{s+6}$, and so, $2\in S_{s+6}$ (Lemma~\ref{gappy3}); in particular, $|v_{s+6}|\ge 5$. Since $\delta([v_1], [v_{s+6}])\le 3$, we get that $|\braket{v_1}{v_{s+6}}|\ge 2$, a contradiction (since it must be $\braket{v_1}{v_{s+6}}=-1$). If $2 \in S_{s+6}$ and $1\not \in S_{s+6}$, then $\braket{v_{s+6}}{v_2}\neq 0$ and $\braket{v_{s+6}}{v_3}\neq 0$, and there will be a heavy triple $(v_3, v_{s+6}, v_{s+4})$: again, see Figure~\ref{fig:case1 1223 G(S)}. The claim follows if $\sigma_4=5$. If $\sigma_4=4$, however, using Lemma~\ref{gappy3} and Corollary~\ref{unbreakablepairing}, we get that $S_{s+6}$ is one of the following cases: $\{ k_2-2, k_2-1, k_2, k_3\}$, $\{k_2-1, k_2, k_3\}$, or $\{k_2, k_3\}$. If $S_{s+6}=\{ k_2-2, k_2-1, k_2, k_3\}$, there will be a claw $(v_{s+2}, v_{s+3}, v_{s+6}, v_{s+1})$, unless $k_2-2=4$. But then $\braket{v_{s+4}}{v_4}\not = 0$, which implies that $\braket{v_{s+4}}{v_1}\not = 0$ (Lemma~\ref{lem:k1=3 sigma3}), a contradiction. Suppose $S_{s+6}=\{ k_2-1, k_2, k_3\}$. Since $\braket{v_{s+6}}{v_{s+3}}\neq 0$, $[v_4]\prec [v_1]$, $|v_5|=\cdots=|v_{s+3}|=2$, $v_4\sim v_5 \sim \cdots \sim v_{s+3}$ in the intersection graph (see Figure~\ref{fig:case1 1223 G(S)}), and $|v_{s+6}|=4$ in this case, we get that $\braket{v_{s+4}}{v_1}\not = 0$, a contradiction. That is, $S_{s+6}=\{ k_2, k_3\}$, and the claim follows. Suppose that there exists $\ell>s+6$ such that $|v_i|=2$ for any $s+6<i<\ell$. By Corollary~\ref{x0pairing}, $S'_{\ell}$ is either $\emptyset$, $\{ 0, k_3\}$, $\{ k_1, k_3\}$, or $\{ k_2, k_3\}$. As in the index $s+6$, $S'_{\ell}\neq \{0, k_3\}$ or $\{k_1, k_3\}$. If $S'_{\ell}=\{ k_2, k_3\}$, then to avoid $\braket{v_{\ell}}{v_{s+6}}=2$ (that violates Corollary~\ref{unbreakablepairing}), we must have $s+6\in \supp(v_{\ell})$. Using Lemma~\ref{gappy3}, $s+7, \cdots, \ell-2\in \supp(v_{\ell})$. This will give a heavy triple $(v_{s+4}, v_{s+6}, v_{\ell})$ if $v_{\ell} \sim v_{s+4}$. If $v_{\ell}\not \sim v_{s+4}$, however, it must be the case that either $2\in \supp(v_\ell)$ or $s+3\in \supp(v_\ell)$. We get a contradiction in each case by using similar arguments to those used for the index $s+6$. We have shown that $S'_{\ell}=\emptyset$. Set $j=\text{min }\supp(v_{\ell})$. If $s+6<j<\ell -1$, there will be a claw on $v_j, v_{j-1}, v_{\ell}, v_{j+1}$, and if $j=s+6$, the claw will be on $v_{s+6}, v_{s+4}, v_{\ell}, v_{s+7}$. So $j=\ell-1$, and by induction we see that $|v_i|=2$ for all $i>s+6$. 
