\section{Prism manifolds realizable by surgery on knots in $S^3$}\label{realizable}
Table~\ref{table:Types} gives a list of all prism manifolds $P(p,q)$, with $q>p$, that can possibly be realized by surgery on knots in $S^3$. In~\cite[Table~2]{Prism2016}, a list of realizable prism manifolds $P(p,q)$ with $q>0$ is provided. It is straightforward to verify that the manifolds in Table~\ref{table:Types} coincide with those of \cite[Table~2]{Prism2016} with $q>p$. That is, Table~\ref{table:Types} is a complete list of prism manifolds $P(p,q)$, with $q>p$, arising from surgery on knots in $S^3$. 

\subsection{Prism manifolds corresponding to more than one changemaker vector}

As we pointed out in Section~\ref{pq}, some of the prism manifolds in Table~\ref{table:Types} correspond to distinct changemaker vectors. In this subsection, we address this by providing distinct knots corresponding to such prism manifolds. Our strategy is as follows: let $\sigma$ be a changemaker vector whose orthogonal complement is isomorphic to $C(p,q)$ for some $p$ and $q$. Let $\sigma$ correspond to a knot $K$ in $S^3$ on which surgery results in $P(p,q)$. Using Lemma~\ref{lem:AlexanderComputation}, we compute the Alexander polynomial $\Delta_K(T)$. Then we exhibit a P/SF knot $K_\sigma$ that admits a surgery to $P(p,q)$. By directly computing $\Delta_{K_\sigma}(T)$ we show that the two Alexander polynomials coincide. That is, $K_\sigma$ matches with $\sigma$. See \cite[Section~13.2]{Prism2016}. The parameters beneath the P/SF knots in Table~\ref{Overlap} are explained in~\cite{Prism2016}.    

\subsection{Proof of the main results}

\begin{proof}[Proof of Theorem~\ref{thm:lattice}]
If $C(p,q)$ is isomorphic to a changemaker lattice $L$, then it belongs to one of the families enumerated in Sections~\ref{sec:Case2},~\ref{sec:k1k21},~and~\ref{sec:Case1}. Following Section~\ref{pq}, we can find a pair $(p',q')$ such that $L$ is isomorphic to $C(p', q')$, and $P(p',q')\in \mathcal P^+_{q>p}$. Now, Proposition~\ref{pp} finishes the proof.
\end{proof}

\begin{proof}[Proof of Theorem~\ref{thm:Realization}]
Suppose $P(p,q)\cong S^3_{4q}(K)$, it follows from Theorem~\ref{changemakerlatticeembedding} and Theorem~\ref{thm:lattice} that $P(p,q)$ belongs to one of the six families in Table~\ref{table:Types} and $P(p,q)\cong S^3_{4q}(K_0)$ for some Berge--Kang knot $K_0$. To get the result about $\widehat{HFK}$, we note that $K$ and $K_0$ correspond to the same changemaker vector. Using Lemma~\ref{lem:AlexanderComputation}, we know that $\Delta_K=\Delta_{K_0}$, so $\widehat{HFK}(K)\cong \widehat{HFK}(K_0)$ by \cite[Theorem~1.2]{OSzLens}.
\end{proof}

\begin{table}\centering
\caption{Prism manifolds $P(p,q)$ corresponding to more than one changemaker}
\resizebox{\textwidth}{!} {%
\ra{1.3}


   \begin{tabular}{@{}lllll@{}} \toprule
	
	Prism manifold	& Type & Changemaker & P/SF knot  & Braid word \\ \midrule
		& 
		{\small \begin{tabular}{l}{\bf 4}\end{tabular}} & {\small $(1,2,3,3,7,8^{[s]})$} &\begin{tabular}{l} 
		   {\small {\bf KIST IV}, $s>0$}  \\
		   {\small $(2,-3,-1,0,s+2)$}\\ \\
                    {\small {\bf KIST I}, $s=0$}  \\
		   {\small $(1,3,4,-2,-3)$}
		          \end{tabular} & {\small $(\sigma_7\cdots \sigma_{1})^{8s+23}(\sigma_{13} \cdots \sigma_1)^{-8}$}  \\  {\small \begin{tabular}{l}$P(8s+13,16s+18)$\end{tabular}} &&&& \\ 
	&	 {\small \begin{tabular}{l}{\bf 3A}, $s>0$\\ \\ {\bf 3B}, $s=0$\end{tabular}} & {\small $(1,1,3,5,6,8^{[s]})$} & \begin{tabular}{l}
	
		    {\small {\bf OPT II}} \\
		    {\small $(2,3,0,1,s+1)$}
		    \end{tabular} & \begin{tabular}{l}
		  {\small $(\sigma_7 \cdots \sigma_1)^{8s+11}(\sigma_1 \cdots \sigma_7)^{-2}$}
		    
		    \end{tabular}  \\        
&&&& \\  
		         
&{\small \begin{tabular}{l}{\bf 5}, $s=3$\end{tabular}} & {\small $(1,1,1,3,4,6,10,10)$} & \begin{tabular}{l}{\small {\bf KIST IV}}\\{\small $(2,1,1,-3,2)$}\end{tabular} & {\small $(\sigma_1 \cdots \sigma_{25})^{10}\sigma_{3}\sigma_2\sigma_1$}\\
\bottomrule 

         & {\small \begin{tabular}{l}{\bf 5}\end{tabular}}& {\small $(1,1,1,2,3,6,6)$}
	&\begin{tabular}{l} {\small {\bf KIST IV}} \\ {\small $(2,1,1,-3,1)$} \end{tabular}& {\small $(29,3)$--cable of $T(5,2)$} \\ {\small \begin{tabular}{l}$P(5,22)$\end{tabular}}&&&& \\
& {\small \begin{tabular}{l}{\bf 1A}\end{tabular}} & {\small $(1,1,2,2,2,5,7)$} 
		&\begin{tabular}{l}{\small {\bf TKM II}}\\{\small $(1,2,-1,2,2)$}\end{tabular} &{\small $(\sigma_1 \cdots \sigma_{11})^{7}\sigma_1^2$}\\ \bottomrule

 & {\small \begin{tabular}{l}{\bf 3B}\end{tabular}}& {\small $(1,1,3,5,6,6,6)$}
	&\begin{tabular}{l}{\small {\bf OPT III}}\\{\small $(2,3,0,1,2)$}\end{tabular} &{\small $(\sigma_1 \cdots \sigma_{22})^{6}\sigma_2\sigma_3\sigma_4\sigma_1\sigma_2\sigma_3$} \\ {\small \begin{tabular}{l}$P(25,36)$\end{tabular}}&&&& \\
& {\small \begin{tabular}{l}{\bf 5}\end{tabular}} & {\small $(1,1,1,3,4,4,10)$} 
	&\begin{tabular}{l}{\small {\bf KIST IV}}\\{\small $(2,1,1,-1,3)$}\end{tabular} &{\small $(\sigma_1 \cdots \sigma_{13})^{10}\sigma_1\sigma_2\sigma_3$}\\ \bottomrule

 & {\small \begin{tabular}{l}{\bf 3A}\end{tabular}}& {\small $(1,1,2,5,7,10,12,12)$}
	&\begin{tabular}{l}{\small {\bf OPT II}}\\{\small $(2,5,0,1,3)$}\end{tabular} &{\small $(\sigma_1 \cdots \sigma_{40})^{12}\left(\sigma_{1}\cdots\sigma_{11}\right)^{-2}$}\\ {\small \begin{tabular}{l}$P(43,117)$\end{tabular}}&&&& \\
& {\small \begin{tabular}{l}{\bf 4}\end{tabular}} & {\small $(1,1,2,3,5,6,14,14)$} 
		& \begin{tabular}{l}{\small {\bf KIST IV}}\\{\small $(2,-3,1,-3,1)$}\end{tabular} &{\small $(\sigma_1 \cdots \sigma_{33})^{14}\left(\sigma_{7}\cdots\sigma_1\right)^{-1}$}\\ \bottomrule
\end{tabular}%
}
\label{Overlap}

\end{table}

\clearpage

\begin{table}\centering
\caption{C--type changemakers and the corresponding prism manifolds, Part I}
\resizebox{\textwidth}{!} {%
\ra{.9}


   \begin{tabular}{@{}lll@{}} \toprule




\multicolumn{1}{l}{{\small Prop.}} &
\multicolumn{1}{l}{{\small Changemaker vector}} & \multicolumn{1}{l}{{\small Vertex basis (with $x_0$ omitted) $\{ x_1,\cdots,x_n \}$}}  \\ 

\midrule

 \\
{\small \ref{example} }	&	{\small \begin{tabular}{l}$(1,1, 2^{[s]},2s-1,2s+1)$\\ $s\ge2$\end{tabular}}&
		  $\{-v_2, \cdots, -v_{s+1}, v_{[3,s+2]}, v_1\}$  \\ \\

\bottomrule \\

&		  	{\small \begin{tabular}{l}$(1,1,2^{[s]},2s+1,2s+3, 4s+4, 8s+10)$\\$s\ge 1$\end{tabular}}
	     & $\{-v_2,\cdots, -v_{s+1},-v_{s+5}, v_{s+4}, v_{s+2}, v_1\}$   \\ \\
	   

{\small \ref{1k2>2}}	& {\small \begin{tabular}{l}$(1,1,2^{[s]},2s+1,2s+3, 4s+6, 8s+10)$\\$s\ge 1$\end{tabular}}
			 & $\{-v_2,\cdots, -v_{s+1},-v_{s+4}, v_{s+5}, v_{s+2}, v_1\}$   \\ \\
	
	    
	& 
   	{\small \begin{tabular}{l}$(1,1,2, 3, 5, 8^{[s]},8s+6, (8s+14)^{[t]})$\\$s\ge 1$\end{tabular}}
	     & $\{-v_2, v_{s+5}, v_1, -v_3-v_1,-v_5,\cdots,-v_{s+4},-v_{s+6},\cdots,-v_{s+t+5}\}$   \\ \\



& 
   	{\small \begin{tabular}{l}$(1,1,2, 3, 5, 6, 14^{[t]})$\end{tabular}}
	     & $\{-v_2, v_1+v_5, -v_1, -v_3, -v_6, \cdots, -v_{t+5}\}$   \\ \\

%& 
%   	{\small \begin{tabular}{l}$(1,1,2, 3, 5, 8^{[s]},8s+6)$\\$s\ge 1$\end{tabular}}
%	     & $\{-v_2, v_{s+5}, v_1, v_3,v_5,\cdots,v_{s+4}\}$   \\ \\

\bottomrule
\\

	 &  {\small \begin{tabular}{l}$(1,1,2^{[s]}, 2s+3,2s+5, (4s+6)^{[t]})$\\$s, t\ge 1$\end{tabular}}
	     & $\{-v_2, \cdots, -v_{s+1}, v_{[1,s+1]}+v_{[s+4,s+t+3]}-v_{s+2}, -v_{s+t+3}, \cdots, -v_{s+4}, -v_1\}$   \\ \\

{\small \ref{2k2>2}} &  {\small \begin{tabular}{l}$(1,1,2^{[s]}, 2s+3,2s+5)$\\$s\ge 1$\end{tabular}}
	     & $\{-v_2, \cdots, -v_{s+1}, v_{[1,s+1]}-v_{s+2}, -v_1\}$   \\ \\

	    
&	   	{\small \begin{tabular}{l}$(1,1,2^{[s]}, 2s+3, 2s+5, 4s+6, (4s+8)^{[t]})$\\$s,t\ge 1$\end{tabular}}
	     &  $\{-v_2,\dots,-v_{s+1},-v_{s+5},\dots,-v_{s+t+4},v_{[1,s+1]}+v_{[s+4,s+t+4]}-v_{s+2},-v_{s+4},-v_1\}$
		     \\ \\

%&	   	{\small \begin{tabular}{l}$(1,1,2^{[s]}, 2s+3, 2s+5, 4s+6)$\\$s\ge 1$\end{tabular}}
%	     &  $\{-v_2,\dots,-v_{s+1}, v_{[1,s+1]}+v_{s+4}-v_{s+2},-v_{s+4},-v_1\}$
%		     \\ \\

\bottomrule \\
	
	  
&{\small \begin{tabular}{l}$(1,1,3,5,6^{[t]})$\\$t\ge 1$\end{tabular}}
	     & $\{v_1+v_{[4,t+3]}-v_2, -v_{t+3},\dots, -v_4,-v_1\}$ 
		 \\ \\
	
{\small \ref{lem:k1=1,k2=2,v2tight}} &{\small \begin{tabular}{l}$(1,1,3,5)$\end{tabular}}
	     & $\{-v_2, v_1\}$ 
		 \\ \\
	   
	    &	{\small \begin{tabular}{l}$(1,1,3,5,6,8^{[t+1]})$\end{tabular}}
	     & $\{-v_5,\cdots,-v_{t+5}, v_1+v_{[4,t+5]}-v_2,-v_4, -v_1\}$ 
		    \\ \\ 

\bottomrule	    
 \\

	
	    &	{\small \begin{tabular}{l}$(1,1,1,3,4, 4^{[t]},4t+6,(4t+10)^{[s]})$\end{tabular}} 
	    	&  $\{-v_{t+5}, -v_1,-v_2,-v_4,\cdots,-v_{t+4}, -v_{t+6},\cdots,-v_{t+s+5}\}$  \\ \\ 

{\small \ref{prop:k1=1,k2=2,v2justright1}} 	& {\small \begin{tabular}{l}$(1,1,1,3,4,10)$ \end{tabular}} 
	    & $\{-v_5,v_4,v_2,v_1\}$ \\ \\

& {\small \begin{tabular}{l}$(1,1,1,3,6,10)$\end{tabular}} 
	    & $\{-v_4,v_5,v_2,v_1\}$
	    \\ \\
\bottomrule
\\
 {\small \ref{prop:k1=1,k2=2,v2justright2}}  &	{\small \begin{tabular}{l}$(1,1,1,2,3,6^{[t]})$\\
  $t\ge1$
   \end{tabular}}
	    	 & $\{-v_3,-v_1,-v_2,-v_5,\cdots,-v_{t+4} \}$ 
	   \\  \\
\bottomrule
	    \\
	{\small \ref{1k2=2}}  & {\small \begin{tabular}{l} $(1,2,3,4,5,9)$\end{tabular}} 
	   	 & {\small $\{-v_3,v_{[3,4]}-v_1,-v_4,v_2\}$} \\ \\

\bottomrule
\\
	   	
	   
{\small \ref{lem:k1=2,2t+3}} & {\small \begin{tabular}{l}$(1,2,3,3,7,8^{[s]},(8s+10)^{[t]})$\\ $s\ge 1$\end{tabular}}
	   	 & $\{v_{[5,s+4]}-v_1,-v_{s+4},\cdots,-v_5,v_2,v_3,v_{s+5},\cdots,v_{s+t+4} \}$  \\ \\

& {\small \begin{tabular}{l}$(1,2,3,3,7,10^{[t]})$\end{tabular}}
	   	 & $\{-v_1,v_2,v_3,v_{5},\cdots,v_{t+4} \}$  \\ \\

\bottomrule
\\

{\small \ref{lem:k1>1,2t+3}} & {\small \begin{tabular}{l}$(1,2,3,4^{[s]},4s+3,4s+7,(8s+10)^{[t]})$\\$s\ge 1$\end{tabular}}
	   	 & $\{-v_3,\cdots,-v_{s+2},v_{[3,s+2]}-v_1,v_2,v_{s+3}, v_{s+5}, \cdots,v_{s+t+4} \}$  \\ \\

\bottomrule 
\\
	
{\small \ref{1k1=3}}	  &  \begin{tabular}{l}$(1,2,2,3,3,7)$
\end{tabular}  & $\{v_{[3,4]}-v_1, -v_4,-v_3,-v_2\}$ \\  \\
\bottomrule 
\\
	   	 
	
	   
 {\small \ref{2k1=3}}& 	{\small \begin{tabular}{l}$(1,2,2,3,4^{[s]},4s+5,4s+9,(8s+14)^{[t]})$\\$s\ge 1$\end{tabular}} 
& $\{-v_4,\cdots,-v_{s+3},v_{[3,s+3]}-v_1, -v_3,-v_2,-v_{s+4},-v_{s+6},\cdots, -v_{s+t+5}\}$  \\ \\
	
& 	{\small \begin{tabular}{l}$(1,2,2,3,5,9,14^{[t]})$\end{tabular}} 
& $\{v_{3}-v_1, -v_3,-v_2,-v_{4},-v_{6},\cdots, -v_{t+5}\}$  \\ \\   	
	 	   
\bottomrule
\end{tabular}%
}


\end{table}

\clearpage
%%%%%%%%%%%%%%
%%%%%%%%%%%%%%
%%%%%%%%%%%%%%
%%%%%%%%%%%%%%
%%%%%%%%%%%%%%

\begin{table}
\addtocounter{table}{-1}
\caption{C--type changemakers and the corresponding prism manifolds, Part II}
{\small
\centering
\begin{adjustbox}{max width=\textwidth}
\ra{1.4}


   \begin{tabular}{@{}llll@{}} \toprule




\multicolumn{1}{l}{Prop.} &
\multicolumn{1}{l}{Vertex norms $\{a_1,\dots,a_n\}$} & \multicolumn{1}{l}{Prism manifold parameters} &\multicolumn{1}{l}{$\mathcal P^+_{q>p}$ type}  \\
\midrule
	 
		
	\ref{example}	&	$\{3,2^{[s-1]},s+1,2\}$ &
		
		\begin{tabular}{l}$p=2s-1$\\ $q=2s^2+s+1$	 \end{tabular}
			 & {\bf 1A} \\  \bottomrule
	   
			
		& $\{3,2^{[s-1]},5,3,s+2,2\}$ & 
	    \begin{tabular}{l}$p=22s+25$\\$q=22s^2+53s+32$ \end{tabular}
        & {\bf 1B} \\ \cline{2-4}
	    
       
			

	\ref{1k2>2}   & $\{3,2^{[s-1]},4,4,s+2,2\}$ & 
	\begin{tabular}{l}$p=22s+27$	\\ $q=22s^2+57s+37$	 \end{tabular}
			 & {\bf 1B} \\  \cline{2-4}  

	 &    $\{3,s+3,2,3,3,2^{[s-1]}, 3, 2^{[t-1]}\}$ & 
	    \begin{tabular}{l}$r=2s+3$\\$p=2r^2(t+1)-4r+1$\\$q=(2r+1)^2(t+1)-8r-6$ \end{tabular} & {\bf 4}\\ \cline{2-4}

&    $\{3,3,2,3,4,2^{[t-1]}\}$ & 
	    \begin{tabular}{l}$r=3$\\$p=18t+7$ \\ $q=49t+19$\end{tabular} & {\bf 4}\\ 

%&    $\{3,s+3,2,3,3,2^{[s-1]}\}$ & 
%	    \begin{tabular}{l}$r=2s+3$\\$p=2r^2-4r+1$ \end{tabular} & {\bf 4}\\ \cline{2-4}
	    
 \bottomrule 

	    & $\{3,2^{[s-1]}, 4, 2^{[t-1]},s+3,2\}$ &
	    	\begin{tabular}{l} $r=2t+1$\\$p=2r(s+1)+r+4$\\$q=\frac{1}{2}(2rs+3(r+1))(2s+3)$ \end{tabular}
	     &  {\bf 3B} \\ \cline{2-4}

 \ref{2k2>2}& $\{3,2^{[s-1]}, s+5, 2\}$ &
	    	\begin{tabular}{l} $r=1$\\$p=2s+7$\\$q=(s+3)(2s+3)$ \end{tabular}
	     &  {\bf 3B} \\ \cline{2-4}

 & $\{3,2^{[s-1]},3,2^{[t-1]},3,s+3,2\}$ &
	    
	    \begin{tabular}{l} $r=2t+3$\\$p=2r(s+2)+1$\\$q=(s+2)(2r(s+2)-3)$ \end{tabular}
	     &  {\bf 3A} \\ 

% & $\{3,2^{[s-1]},4,s+3,2\}$ &
%	    
%	    \begin{tabular}{l} $r=3$\\$p=6(s+2)+1$ \end{tabular}
%	     &  {\bf 3A} \\ 

 \bottomrule 
	 
	    & $\{5, 2^{[t-1]},3,2\}$ &
	   \begin{tabular}{l}
$r=2t+1$\\
$p=6t+7$\\
$q=9t+9$
	   \end{tabular}  & 
	    {\bf 3B} \\ \cline{2-4}

 \ref{lem:k1=1,k2=2,v2tight} & $\{6,2\}$ &
	   \begin{tabular}{l}
$r=1$\\
$p=7$\\
$q=9$
	   \end{tabular}  & 
	    {\bf 3B} \\ \cline{2-4}
	  
	    &   $\{4, 2^{[t]},3,3,2\}$ &
	    
	 \begin{tabular}{l}  $r=2t+5$\\$p=8t+21$\\$q=16t+34$ 	 \end{tabular}
	    	 &   {\bf 3A}
	   \\ 

\bottomrule

 
  

%	\begin{tabular}{l}
%	\\
%	\\
%	\end{tabular}
%	&   $\{4, 2^{[t-1]},3,3,2\}$ &
%	    
%	 \begin{tabular}{l}   	 \end{tabular}
%	    	 &   
%	   \\ \cdashline{2-4}
	   
	   
	    	
	
   
        

	
	   	
	
\end{tabular}%
\end{adjustbox}
}

\end{table}	

\clearpage

\begin{table}
\addtocounter{table}{-1}
\caption{C--type changemakers and the corresponding prism manifolds, Part III}\label{BigSummary}

{\small
\centering
\begin{adjustbox}{max width=\textwidth}
\ra{1.4}


   \begin{tabular}{@{}llll@{}} \toprule




\multicolumn{1}{l}{Prop.} &
\multicolumn{1}{l}{Vertex norms $\{a_1,\dots,a_n\}$} & \multicolumn{1}{l}{Prism manifold parameters} &\multicolumn{1}{l}{$\mathcal P^+_{q>p}$ type}  \\ 
\midrule 
     	
	   	
	
	
&  	   $\{t+4,2,2,3,2^{[t]},3,2^{[s-1]}\}$&
	   
	   \begin{tabular}{l}	
	   $r=2t+5$\\$p=(r^2-2r-1)(s+1)-2r+5$\\$q=r^2(s+1)-2r+1$
	   \end{tabular}
     & {\bf 5} \\ \cline{2-4}

\ref{prop:k1=1,k2=2,v2justright1} &   	$\{6,3,2,2\}$ &
	   
	  \begin{tabular}{l} 	
	 $p=25$\\$q=32$
	  \end{tabular} 	 & {\bf 1B} \\  \cline{2-4}

&   	$\{5,4,2,2\}$ &
	   \begin{tabular}{l}
	   	$p=27$\\$q=37$              
       \end{tabular}
       & {\bf 1B} \\ 

\bottomrule
\ref{prop:k1=1,k2=2,v2justright2} &    $\{3,2,2,4,2^{[t-1]}\}$ &
	    \begin{tabular}{l}
	   $r=3$\\
           $p=2t+1$\\
           $q=9t+4$
\end{tabular}& {\bf 5}
\\
\bottomrule
\ref{1k2=2} &   $\{3,3,3,3\}$ &
	 
	\begin{tabular}{l}
	  $p=13$\\
          $q=34$
	\end{tabular}   	
	   	 & {\bf Sporadic}  \\ 
\bottomrule	

	   	
 
	\ref{lem:k1=2,2t+3}   	&   	$\{4, 2^{[s-1]},3,3,2,s+3,2^{[t-1]}\}$ &
	 \begin{tabular}{l}
	 $r=-3-2s$\\$p=2r^2t-4r+1$\\$q=t(2r+1)^2-8r-6$
	 \end{tabular}
	 & {\bf 4} \\ \cline{2-4}

&   	$\{5,3,2,3,2^{[t-1]}\}$ &
	 \begin{tabular}{l}
	 $r=-3$\\
         $p=18t+13$\\
         $q=25t+18$
	 \end{tabular}
	 & {\bf 4} \\ 
\bottomrule


	   
\ref{lem:k1>1,2t+3}	&   	$\{3,2^{[s-1]},4,3,s+2,3,2^{[t-1]}\}$ &
	  \begin{tabular}{l} 
	  $r=-5-4s$\\
          $p=(-4r-2)t-2r+3$\\
          $q=r^2t+\frac{1}{2}(r^2-2r+1)$
	  \end{tabular}
	  & {\bf 2} \\ 

\bottomrule


	   	
\ref{1k1=3}	 &  	$\{4,2,3,2\}$ &\begin{tabular}{l}
	   $p=11$\\
            $q=19$     
	 \end{tabular}
	 & {\bf Sporadic} \\
   \bottomrule
         
\ref{2k1=3}	&   $\{3, 2^{[s-1]},3,3,2,s+3,3,2^{[t-1]}\}$ &
	  \begin{tabular}{l}
	     $r=7+4s$\\
             $p=(4r+2)t+2r+5$\\
             $q=r^2t+\frac{1}{2}(r^2+2r-1)$
	  \end{tabular} 
	   	
	   	 &  {\bf 2} \\ \cline{2-4}

&   $\{4, 3, 2,3,3,2^{[t-1]}\}$ &
	  \begin{tabular}{l}
	     $r=7$\\
             $p=30t+19$\\
             $q=49t+31$
	  \end{tabular} 
	   	
	   	 &  {\bf 2} \\ 
		
\bottomrule
\end{tabular}%

\end{adjustbox}
}
\caption*{\small In this table, $v_{[a,b]}$ means $v_a+v_{a+1}+\cdots+v_{b}$ for $a<b$. All vertex bases are presented in the form $\{x_1,\cdots,x_n \}$. The parameters $s,t\ge0$ unless otherwise stated.
A superscript $^{[-1]}$ at an element in the sequence of vertex norms means that the sequence is truncated at this element and the element preceding it. For example, the sequence $\{3,2^{[s-1]},4,3,s+2,3,2^{[t-1]}\}$ becomes $\{3,2^{[s-1]},4,3,s+2\}$ when $t=0$.}
\end{table}	
\clearpage