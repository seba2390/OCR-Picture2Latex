\subsection{{\boldmath $k_1=2$}}\label{k1=2}

This subsection is devoted to classifying the changemaker C-type lattices with 
\begin{equation}\label{x0k1=2}
x_0=e_0+e_2+e_{k_2} - e_{k_3}.
\end{equation} 
Recall that the changemaker starts with $(1,2,3)$. We have
\begin{equation}\label{eq:v2parings}
\braket{v_1}{v_2}=1,\quad \braket{v_2}{x_0}=0.
\end{equation}

\begin{lemma}\label{lem:k1>1 intervals}
		The intervals $[v_2]$ and $[v_1]$ are consecutive with $\epsilon_2 = -\epsilon_1$. 
\end{lemma}

\begin{proof}
Using (\ref{eq:v2parings}) and Lemma~\ref{intervalproduct}, either $[v_2]\pitchfork [v_1]$, $|[v_2]\cap[v_1]|=|[v_2]|=3$, $\delta([v_2], [v_1])=2$, and $\epsilon_2=\epsilon_1$, or $[v_2]\dagger [v_1]$, and $\epsilon_2 = -\epsilon_1$. In the former case, $v_2-v_1$ is the sum of two distant intervals, and so is reducible. However, we have $v_2=e_0+e_1-e_2$, and so $v_2-v_1$ is irreducible by Lemma~\ref{lem:SumIrr}~(2).
\end{proof}

\begin{lemma}\label{lem:NoSingle1}
There does not exist an index $j>3$, $j\ne k_3$, such that $\supp(v_j)\cap\{0,1,2\}=\{1\}$.
\end{lemma}
\begin{proof}
Otherwise, we will have $\braket{v_{j}}{v_1}=-\braket{v_{j}}{v_2}=-1$. We also have $\braket{v_{j}}{x_0}=0$ by Lemma~\ref{Lem:X0Odd}. By Lemma~\ref{lem:k1>1 intervals}, $[v_{j}]$ and $[v_1]$ share their right endpoint, so $\delta([v_j],[v_1])=1$. By  Lemma~\ref{intervalproduct}, $|\braket{v_1}{v_j}|=|v_{j}|-1>1$, a contradiction. 
\end{proof}



\begin{lemma}\label{lem:k1=2 sigma3}
		$\sigma_3 \in \{3,4\}$. Furthermore, if $\sigma_3 = 4$ then $[v_3]$ and $[v_1]$ share their left endpoint, and that $\epsilon_3 = \epsilon_1$.  
\end{lemma}

\begin{proof}
All the possibilities for $\sigma_3$ lie in $\{ 3,4,5,6\}$. If $\sigma_3=5$, we get that $v_3=e_1+e_2-e_3$. So $\braket{v_3}{v_1}=-1$ and $v_3$ is orthogonal to $v_2$. By Lemma~\ref{Lem:X0Odd}, $k_2=3$ and $\braket{v_3}{x_0}=0$.
Using Lemma~\ref{lem:k1>1 intervals}, we know that $[v_2]$ abuts $[v_1]$, and therefore, there will be a claw on $v_1, x_0, v_3, v_2$, unless $[v_3]\pitchfork [v_1]$, $|[v_1]\cap [v_3]|=|v_3|=3$, and $\epsilon_3=-\epsilon_1$. Thus $v_1+v_3$ is the sum of two distant intervals and so is reducible. However, $v_3+v_1$ is irreducible by Lemma~\ref{lem:SumIrr}, a contradiction. If $\sigma_3=6$, we see that $v_3=e_0+e_1+e_2-e_3$ (and, in particular, $|v_3|=4$). This implies that $\braket{v_3}{x_0}=2$ and $\braket{v_1}{v_3}=1$. The latter will only be possible if both $\delta([v_1], [v_3])=3$ and $\epsilon_1 = \epsilon_3$, a contradiction to Lemma~\ref{lem:delta}. 

If $\sigma_3=4$, we have $v_3=e_0+e_2-e_3$. Using Lemma~\ref{intervalproduct}, the second statement of the lemma is immediate because $\braket{v_3}{v_1}=\braket{v_3}{x_0}=\braket{v_1}{x_0}=2$ and $|v_3|=3$.
\end{proof}

\begin{lemma}\label{lem:0No2}
If $0\in\supp(v_j)$ and $2\notin\supp(v_j)$ for some $j>3$ and $j\ne k_3$, then $[v_j],[v_1]$ share their right endpoint, and $v_j=e_0+e_{j-1}-e_j$. Moreover, there exists at most one such $j$.
\end{lemma}
\begin{proof}
We have $1\not\in \supp(v_j)$, otherwise $\braket{v_2}{v_j}=2$, a contradiction to Corollary~\ref{unbreakablepairing}. So $\braket{v_1}{v_j}=2$. Since $\braket{v_{j}}{v_2}=1$, $[v_j]$ and $[v_2]$ are consecutive by Corollary~\ref{Cor:ZjDistinct}. It follows from Lemma~\ref{lem:k1>1 intervals} that $[v_{j}]$ and $[v_1]$ share their right endpoint, and so $\delta([v_{j}], [v_1])=1$. Then, to get $\braket{v_1}{v_{j}}=2$, we must have $|v_{j}|=3$ and $v_{j}=e_0+e_{j-1}-e_{j}$. Lastly, there exists at most one such $j$ by Corollary~\ref{Cor:ZjDistinct}.
\end{proof}


\begin{prop}\label{prop6.1}
If $\sigma_3=3$, the initial segment $(\sigma_0, \cdots, \sigma_{k_3})$ of $\sigma$ is $(1,2,3,3,7)$.
\end{prop}

\begin{proof}
Suppose that $\sigma_3=3$ (see Lemma~\ref{lem:k1=2 sigma3}). This implies that $k_2=3$ (Lemma~\ref{Lem:X0Odd}). Using Equation~\eqref{x0k1=2}, we see that $\sigma_{k_3}=7$. We claim that $k_3=k_2+1=4$. If $k_3\neq 4$, by Lemma~\ref{Lem:X0Odd}, $\sigma_4\in \{ 4, 6\}$. Suppose $\sigma_4=4$, or equivalently, $v_4=e_0+e_3-e_4$. This gives us that $\braket{v_4}{x_0}=2$ and $\braket{v_4}{v_2}=1$. By Lemma~\ref{lem:k1>1 intervals}, the interval $[v_1]$ will be a subset of $[v_4]\cup\{x_0\}$, which implies that $|v_1|= 3$, a contradiction. Suppose $\sigma_4=6$, or equivalently, $v_4=e_2+e_3-e_4$. Then there will be a claw $(v_2, v_1, v_4, v_3)$. This justifies the claim, that is, $\sigma_4=7$ and $k_3=4$. 
\end{proof}

\begin{prop}\label{lem:k1>1,2t+1}
If $\sigma_3=4$, the initial segment $(\sigma_0, \cdots, \sigma_{k_3})$ of $\sigma$ is either $(1,2,3,4,5,9)$ or $(1,2,3,4^{[s]}, 4s+3,4s+7)$, $s\ge1$.
\end{prop}

\begin{proof}
All the possibilities for $\sigma_4$ lie in $\{ 4,5,6,7,8,9,10\}$. We first argue that $\sigma_4\not\in\{6, 8, 9, 10\}$. Suppose $\sigma_4=6$, then $v_4=e_1+e_3-e_4$, contradicting Lemma~\ref{lem:NoSingle1}. If $\sigma_4=10$, then $v_4$ will have nonzero inner product with $v_2$ and $v_3$.  Using Lemmas~\ref{lem:k1=2 sigma3}~and~\ref{lem:k1>1 intervals}, the interval $[v_1]$ equals the union of $[v_3]$ and $[v_4]$, that is, $|v_1|=6$, a contradiction. If $\sigma_4=8$, then both the unbreakable vectors $v_3$ and $v_4$ will have nonzero inner product with $x_0$, contradicting Corollary~\ref{x0pairing}. When $\sigma_4=9$, $v_4=e_1+e_2+e_3-e_4$. Notice that $\braket{v_4}{v_1}=-1$ while $v_4$ is orthogonal to $x_0$. The latter gives us that $\delta([v_4], [v_1])\le 2$. Therefore, given that $|v_4|=4$, we must have $[v_4]$ and $[v_2]$ share their left endpoint by Lemma~\ref{lem:k1>1 intervals}, a contradiction to Corollary~\ref{Cor:ZjDistinct}. Therefore $\sigma_4\in \{4, 5, 7\}$.

Suppose that $\sigma_4=5$, that is, $v_4=e_0+e_3-e_4$. Using Lemma~\ref{Lem:X0Odd}, $k_2=4$, and so $\sigma_{k_3}=9$ by Equation~\eqref{x0k1=2}. Since $\braket{v_3}{x_0}=2$, $\braket{v_5}{x_0}=0$ by Corollary~\ref{x0pairing}, unless $k_3=5$. Since $4\in \supp(v_5)$, we get that $k_3=5$.

Let $s\ge 1$ be the integer satisfying that $\sigma_3=\cdots =\sigma_{s+2}=4$, and that $\sigma_{s+3}>4$. By Lemma~\ref{Lem:X0Odd}, $k_2\ge s+3$. Set $j=\text{min }\supp(v_{s+3})<s+2$. If $3<j<s+2$, there will be a claw $(v_j, v_{j-1}, v_{s+3}, v_{j+1})$, and if $j=3$, the claw will be $(v_3, x_0, v_{s+3}, v_4)$. If $j=1$, then $2\in\supp(v_{s+3})$ by Lemma~\ref{lem:NoSingle1}. Thus $|v_{s+3}|\ge 4$. Since $\braket{v_{s+3}}{x_0}=0$, $\delta([v_{s+3}], [v_1])\le 2$. Then 
\[
|\braket{v_{s+3}}{v_1}|\ge4-2\ge 2,
\] 
a contradiction. Lastly, suppose $j=0$. By Corollary~\ref{x0pairing}, $\braket{v_{s+3}}{x_0}=0$, it must be the case that $2\not \in\supp(v_{s+3})$. By Lemma~\ref{lem:0No2}, $v_{s+3}=e_0+e_{s+2}-e_{s+3}$. If $s=1$ and $\sigma_4=5$, this case was discussed in the previous paragraph. However, if $s>1$, then $\braket{v_{s+3}}{v_3}=1$, and so $[v_1]$ will be the union of $[v_3]$ and $[v_{s+3}]$ by Lemma~\ref{lem:k1=2 sigma3} and Lemma~\ref{lem:0No2}. Since $|v_3|=3$, to get $|v_1|=5$, it must be that $|v_{s+3}|=4$, a contradiction. So we are left with the case $j=2$. 

Note that $\braket{v_{s+3}}{v_2}=-1$, and $v_{s+3}$ is orthogonal to $v_1$ and $x_0$, so $[v_{s+3}]$ is distant from $[v_1]$ by Lemma~\ref{lem:k1>1 intervals}. Using Lemma~\ref{lem:k1=2 sigma3}, we get that $v_{s+3}$ is orthogonal to $v_{3}$, and so $3\in \supp(v_{s+3})$. By Lemma~\ref{gappy3}, we get that $4,\cdots, s+1 \in \supp(v_{s+3})$. That is, $\sigma_{s+3}=4s+3$, and that $k_2=s+3$. Using Equation~\eqref{x0k1=2}, we get that $\sigma_{k_3}=4s+7$. With the same argument as in the case $\sigma_4=5$, we get that $k_3=k_2+1=s+4$. This recovers the case $\sigma_4=7$ when $s=1$. 
\end{proof}

\begin{prop}\label{1k2=2}
If $(\sigma_0, \cdots, \sigma_{k_3})=(1,2,3,4,5,9)$, then $n+1=k_3$ (i.e. $v_{k_3}$ is the last standard basis vector).
\end{prop}

\begin{proof}
We claim that the index $6$ does not exist. Suppose for contradiction that it exists. Since $5\in S'_6$, then $S'_6$ must be one of $\{4, 5\}$, $\{2, 5\}$, or $\{0, 5\}$ (Lemma~\ref{Lem:X0Odd} and Corollary~\ref{x0pairing}). 

By Lemma~\ref{lem:0No2}, the intervals $[v_4]$ and $[v_1]$ share their right endpoint, and $S'_6\ne\{ 0,5\}$. 

Suppose that $S'_6=\{ 4, 5\}$ or $\{ 2, 5\}$, then $\braket{v_6}{x_0}=0$. We have that one of $\braket{v_6}{v_4}$ and $\braket{v_6}{v_3}$ is zero and the other one is nonzero, depending on whether or not $3\in S_6$. 
By Lemma~\ref{lem:k1>1 intervals} and Corollary~\ref{Cor:ZjDistinct}, $[v_6]$ and $[v_1]$ are not consecutive. 
Using Lemma~\ref{lem:k1=2 sigma3} and the fact that $[v_4]$ and $[v_1]$ share their right endpoint, we conclude that $[v_6]\subset[v_1]$ and $\delta([v_6], [v_1])\le 2$. Since $|v_6|\ge 3$, we must have $\braket{v_6}{v_1}\not = 0$. That is, $1\in \supp(v_6)$, and so $|v_6|\ge 4$. Using Lemmas~\ref{lem:k1>1 intervals}, \ref{lem:k1=2 sigma3} and Corollary~\ref{Cor:ZjDistinct}, $[v_1]$ will have all the high weight vertices of $[v_3]$, $[v_6]$, and $[v_4]$, and so, $|v_1|\ge 6$, a contradiction. 
 This proves the claim.
\end{proof}


\begin{prop}\label{lem:k1=2,2t+3}
When $(\sigma_0, \cdots, \sigma_{k_3})=(1,2,3,3,7)$, there exists $s\ge0$, such that $v_{s+5}=e_{3}+\cdots +e_{s+4}-e_{s+5}$, $v_5 = e_0 + e_4-e_5$ if $s>0$, and $|v_j|=2$ for $5<j<s+5$ and $j>s+5$. In this case, $\sigma =  (1,2,3,3,7,8^{[s]}, 8s+10^{[t]})$ ($s,t\geq 0$).
\end{prop}	
\begin{proof}
First suppose that $\sigma_5\neq 10$.
Since $k_3=4\in S'_5$, the set $S'_5$ is either $\{0,4\}$, $\{3,4\}$, or $\{0,2,3,4\}$ (Lemmas~\ref{Lem:X0Odd}~and~\ref{gappy3}). 
If $S'_5=\{3,4\}$, as $\sigma_5\neq 10$, we must have $1\in S_5$, a contradiction to Lemma~\ref{lem:NoSingle1}. If $S'_5=\{0,2,3,4\}$, then $\braket{v_1}{v_5}> 0$. Since $|v_5|\ge 5$ and $\delta([v_1], [v_5])\le 3$, we have $\epsilon_1=\epsilon_5$. Since $\braket{v_1}{v_5}\le2$, and that $|v_5|\ge 5$, we must have $\delta([v_5],[v_1])=3$. Since $\epsilon_1=\epsilon_5$, by Lemma~\ref{lem:delta}, $\braket{v_5}{x_0}=-\braket{v_1}{x_0}=\pm2$, which is not true. Therefore, $S'_5=\{ 0, 4\}$ and $v_5=e_0-e_4+e_5$ by Lemma~\ref{lem:0No2}.


%\noindent{\bf Claim 1.} For any $i>5$, if $|v_i|\ge3$, then at least one of $\braket{v_i}{v_1}$ and $\braket{v_i}{v_2}$ is zero.

%Assume for contradiction that both $\braket{v_i}{v_1}$ and $\braket{v_i}{v_2}$ are nonzero. By Corollary~\ref{Cor:ZjDistinct}, $[v_i]$ and $[v_2]$ are consecutive. By Lemma~\ref{lem:k1>1 intervals},  either $[v_1]$ and $[v_i]$ share their right endpoint or they are distant. The latter case cannot happen since $\braket{v_i}{v_1}\ne0$. However, we already proved that
%$[v_5]$ and $[v_1]$ share their right endpoint. So $[v_1]$ and $[v_i]$ cannot share their right endpoint by Corollary~\ref{Cor:ZjDistinct}. This finishes the proof of this claim.

We claim that if $S_j\ne\emptyset$ for some $j>5$, then $S_j=\{3,4\}$. Assume that $S_j\ne\emptyset$.
By Lemmas~\ref{Lem:X0Odd} and~\ref{gappy3}, $S_j'$ is one of $\emptyset$, $\{0,3\}$, $\{ 0,4\}$, $\{2, 3\}$, $\{ 3,4\}$, and $\{ 0,2, 3,4\}$. If $S_j'=\emptyset$, then $S_j=\{1\}$, contradicting Lemma~\ref{lem:NoSingle1}. Since $S'_5=\{ 0, 4\}$, $S_j'\ne\{0,3\}$ or $\{0,4\}$ by Lemma~\ref{lem:0No2}.
If $S'_j=\{ 2,3\}$, then $\braket{v_j}{x_0}=2$. Since $\delta([v_j], [v_1])\le 3, |v_j|\ge 4$, we have $\braket{v_j}{v_1}\ne0$. Since
$0\notin \supp(v_j)$, we must have $1\in \supp(v_j)$, and so $|v_j|\ge 5$. Using Lemma~\ref{intervalproduct}, we get $|\braket{v_j}{v_1}|>1$, contradicting the fact that $\braket{v_1}{v_j}=-1$. If $S'_{j} = \{0,2,3,4\}$, then we have $\braket{v_{j}}{x_0}=2$ and $|v_{j}|\ge 6$. Thus $x_1=x_{z_j}$ is contained in $[v_1]$. However, $|v_1|=5<6=|v_j|$, a contradiction. So $S'_j=\{3,4\}$. Using Lemma~\ref{lem:NoSingle1}, we conclude that $1\notin S_j$. So $S_j=\{3,4\}$. 

If $S_j=\emptyset$ for all $j>5$, it follows from Lemma~\ref{lem:AllNorm2} that $|v_j|=2$ whenever $j>5$. Now assume that $S_j\ne\emptyset$ for some $j>5$.
Let $s+5$ be the smallest such $j$. Then $S_{s+5}=\{3,4\}$ by the earlier discussion. We also know that $|v_i|=2$ for any $5<i<s+5$ by Lemma~\ref{lem:AllNorm2}. If $5\not \in \supp(v_{s+5})$, then $\braket{v_{s+5}}{v_5}\neq 0$ and $\braket{v_{s+5}}{v_3}\neq 0$, and so there will be a cycle $(v_{s+5}, v_3, v_2, v_5)$ of length bigger than $3$: see Figure~\ref{fig:case1 123 G(S)}. Thus $5 \in \supp(v_{s+5})$, and as a result $6, \cdots, s+4\in \supp(v_{s+5})$ by Lemma~\ref{gappy3}. Therefore, $\sigma_{s+5}=8s+10$.

Note that, $S_j=\emptyset$ when $j>s+5$. Otherwise, by the earlier discussion, $S_j=\{3,4\}$, and
we would have a heavy triple $(v_{j}, v_{s+5}, v_2)$. 
Given $j>s+5$, let $\ell=\min\supp(v_j)\ge5$. Note that
\[
v_5\sim v_6\sim \cdots \sim v_{s+4},
\]
$[v_5]$ and $[v_1]$ share their right endpoint, $\braket{v_i}{v_1}=0$ and $|v_i|=2$ when $5<i<s+5$, so $[v_{i}]\subset [v_1]$ when $5\le i<s+5$. If $\ell\le s+4$, then $\braket{v_j}{v_{\ell}}\ne0$.
Thus $[v_{j}]\cap [v_1]\neq \emptyset$. Note also that $\delta([v_{j}], [v_1])\le 2$ since $v_{j}$ is orthogonal to $x_0$. Since $|v_{j}|\ge 3$, we get that $|\braket{v_{j}}{v_1}|>0$, a contradiction. Thus we have proved that $\min\supp(v_j)\ge s+5$ when $j>s+5$. It follows from Lemma~\ref{lem:AllNorm2} that $|v_j|=2$ when $j>s+5$.

%Suppose that there exists $s>0$ such that $|v_i|=2$ for any $5<i<s+5$, and that $|v_{s+5}|>2$. Note that $\braket{v_i}{v_2}=0$ for all $5<i<s+5$ and, 
%\[
%v_5\sim v_6\sim \cdots \sim v_{s+4},
%\]
%and so $[v_i]\prec [v_1]$. Note also that $S'_{s+5}=\{ 0,4\}$, or $\{ 3,4\}$, or $\{ 0,2, 3,4\}$. If $S'_{s+5}=\{ 0,4\}$, then $\braket{v_{s+5}}{v_1}\neq 0$ and $\braket{v_{s+5}}{v_2}\neq 0$. Therefore, $[v_{s+5}]$ shares its right endpoint with  that of $[v_1]$. That is, $[v_{s+5}]$ and $[v_5]$ share their right endpoint, a contradiction. If $S'_{s+5} = \{0,2,3,4\}$, then we have $\braket{v_{s+5}}{x_0}=2$ and $|v_{s+5}|\ge 6$. Since $\delta([v_{s+5}], [v_1])\le 3$, then $|\braket{v_{s+5}}{v_1}|>2$, a contradiction. That is, $S'_{s+5}=\{ 3,4 \}$. The same argument used to rule out the possibility $S'_{s+5}=\{ 0,4 \}$ shows that $1\not \in S'_{s+5}$. If $5\not \in \supp(v_{s+5})$, then $\braket{v_{s+5}}{v_5}\neq 0$ and $\braket{v_{s+5}}{v_3}\neq 0$, and so there will be a cycle $(v_{s+5}, v_3, v_2, v_5)$ of length bigger than $3$: see Figure~\ref{fig:case1 123 G(S)}. Thus $5 \in \supp(v_{s+5})$, and as a result $6, \cdots, s+3\in \supp(v_{s+5})$ by Lemma~\ref{gappy3}. Therefore, $\sigma_{s+5}=8s+10$. Now suppose $t\ge 0$ be such that $|v_i|=2$ for any $s+5<i<s+t+6$, and suppose also that the index $s+t+6$ exists. We claim that $|v_{s+t+6}|=2$. The same argument used for the index $s+5$, shows that $S'_{s+t+6}\neq \{0,4\}$ or $\{ 0,2,3,4\}$, and that $1\not \in \supp(v_{s+t+6})$. Using Lemma~\ref{Lem:X0Odd}, the only possibilities for $S'_{s+t+6}$ consist of $\emptyset$ and $\{3,4\}$. If $S'_{s+t+6}=\{3,4\}$, then $\braket{v_{s+5}}{v_{s+t+6}}\neq 0$ and $\braket{v_{s+t+6}}{v_3}\neq 0$, and so there will be a heavy triple $(v_{s+t+5}, v_{s+5}, v_2)$, a contradiction. We have shown that $S'_{s+t+6}=S_{s+t+6}=\emptyset$. Set $j=\text{min }\supp(v_{s+t+6})$. We have proved $j>4$. We also observe that $j\ge s+3$, as otherwise $\braket{v_{s+t+6}}{v_{s+5}}>1$. If $j=s+3$, then $s+4\in \supp(v_{s+t+6})$ by Lemma~\ref{gappy3}. Also, to avoid $\braket{v_{s+t+6}}{v_{s+5}}=2$, we get $s+5\in \supp(v_{s+t+6})$. So $s+6, \cdots, s+t+4\in \supp(v_{s+t+6})$. We have $\braket{v_{s+t+6}}{v_{s+5}}\neq 0$ and $\braket{v_{s+3}}{v_{s+t+6}}\neq 0$, and there will be a cycle $(v_{s+5}, v_3, v_2, v_5, \cdots, v_{s+3}, v_{s+t+6})$ of length bigger than $3$: see Figure~\ref{fig:case1 123 G(S)}. Suppose $j=s+4$. Then $\braket{v_{s+t+6}}{v_{s+4}}\neq 0$. Note that, by our earlier discussion, we have $[v_{s+t+6}]\cap [v_1]\neq \emptyset$. Note also that $\delta([v_{s+t+6}], [v_1])\le 2$ since $v_{s+t+6}$ is orthogonal to $x_0$. Since $|v_{s+t+6}|\ge 3$, we get that $|\braket{v_{s+t+6}}{v_1}|>0$, a contradiction. There will be a claw if $s+4<j<s+t+5$: again, see Figure~\ref{fig:case1 123 G(S)}. So $j=s+t+5$, and by induction $|v_i|=2$ for any $i>s+5$. 

Finally suppose that $\sigma_5=10$. Assume that there exists $\ell>5$ such that %for any $5<i<\ell$, $\sigma_i=10$. 
$S_{\ell}\ne\emptyset$. By Lemmas~\ref{Lem:X0Odd}~and~\ref{gappy3}, $S'_\ell$ is one of $\emptyset$, $\{0,3\}$, $\{0,4\}$, $\{2, 3\}$, $\{ 3, 4\}$, and $\{0,2,3,4\}$. 
If $S'_\ell=\emptyset$, then $S_{\ell}=\{1\}$, contradicting Lemma~\ref{lem:NoSingle1}. By Lemma~\ref{lem:0No2}, $S'_\ell\ne\{0,3\}$ or $\{0,4\}$. Suppose $S'_\ell = \{2,3\}$. If $1\not \in \supp(v_\ell)$, there will be a claw $(v_2, v_1, v_\ell, v_3)$. If $1\in \supp(v_\ell)$, then $|\braket{v_\ell}{v_1}|=1$. By Lemma~\ref{lem:k1>1 intervals}, $[v_{\ell}]$ and $[v_1]$ are not consecutive. Since $\delta([v_\ell], [v_1])\le 3$ and $|v_\ell|\ge 5$, we get $|\braket{v_\ell}{v_1}|\ge2$, a contradiction. 
If $S'_\ell=\{3,4\}$, there will be a heavy triple $(v_5,v_\ell, v_2)$. If $S'_\ell=\{0,2,3,4\}$, then $|v_\ell|\ge 6$ and $\braket{v_{\ell}}{x_0}=2$, so $x_{z_{\ell}}=x_1$. Thus $|[v_1]|\ge |[v_\ell]|\ge6$, a contradiction. So we proved that $S_\ell=\emptyset$ whenever $\ell>5$. It follows from Lemma~\ref{lem:AllNorm2} that $|v_j|=2$ when $j>5$.
\end{proof}

\begin{figure}
	\begin{align*}
	\xymatrix@R1.5em@C1.5em{
		v_{3}^{(3)}\ar@{=}[r]^{+}\ar@{=}[dr]^{+} & x_0^{(4)} \ar@{=}[d]^{+} \\
		v_4^{(3)} \ar@{=}[r]^{+} \ar@{-}[dr]^{+} & v_1^{(5)} \ar@{-}[d]^{+} \\
		& v_{2}^{(3)} \\
	}
	&&
	\xymatrix@R1.5em@C1.5em{
		v_3^{(3)} \ar@{-}[d] \ar@{=}[r]^{+} \ar@{=}[dr]^{+} & x_0^{(4)} \ar@{=}[d]^{+} \\
		v_{4}^{(2)}\ar@{-}[d] & v_1^{(5)} \ar@{-}[d]^{+} \\
		\vdots & v_{2}^{(3)} \ar@{-}[d] \\
		v_{s+2}^{(2)}\ar@{-}[u] & v_{s+3}^{(s+2)}  
%\\
%		& %v_{s+5}^{(3)}
		%& v_{s+6}^{(2)} \ar@{-}[d] \\
		%& \vdots \\
		%& v_{s+t+4}^{(2)} \ar@{-}[u]
	}
	&&
	\xymatrix@R1.5em@C1.5em{
		& x_0^{(4)} \ar@{=}[d]^{+} \\
		%v_5^{(3)} \ar@{=}[r]^{+} \ar@{-}[dr]^{+} 
                     & v_1^{(5)} \ar@{-}[d]^{+} \\
		 & v_{2}^{(3)} \ar@{-}[d] \\
		& v_{3}^{(2)} \\
		%v_{s+4}^{(2)}\ar@{-}[u] & v_{s+5}^{(s+3)} 
		%& v_{s+6}^{(2)} \ar@{-}[d] 
		%& \vdots \\
		%& v_{s+t+4}^{(2)}\ar@{-}[u]
	}
	\end{align*}
	\caption{Pairing graphs when $(\sigma_0, \cdots, \sigma_{k_3})$ is $(1,2,3,4,5,9)$ (left), $(1,2,3,4^{[s]}, 4s+3, 4s+7)$ (center), or $(1,2,3,3,7)$ (right).}
	\label{fig:case1 123 G(S)}
\end{figure}

\begin{prop}\label{lem:k1>1,2t+3}
		If $(\sigma_0, \cdots \sigma_{k_3})=(1,2,3,4^{[s]}, 4s+3,4s+7)$, $s>0$, then $v_{s+5}=e_{s+3}+e_{s+4}-e_{s+5}$, and $\abs{v_j} = 2$ for $j>s+5$. In this case, $\sigma =  (1,2,3,4^{[s]}, 4s+3,4s+7, (8s+10)^{[t]})$ ($s>0,t \geq 0$). 
\end{prop}	

\begin{proof}
Suppose that $\ell>s+4$ is an index such that $S_{\ell}\ne\emptyset$. We will prove that $\ell=s+5$ and $v_{s+5}=e_{s+3}+e_{s+4}-e_{s+5}$. Our conclusion then follows from Lemma~\ref{lem:AllNorm2}.

\noindent{\bf Step 1}. $S'_{\ell}$ must be either $\emptyset$ or $\{ s+3, s+4\}$.

Using Lemma~\ref{gappy3} and Corollary~\ref{x0pairing}, $S'_\ell$ is either $\emptyset$, $\{ 0, s+4\}$, $\{ 2, s+4\}$, or $\{ s+3, s+4\}$. Suppose $S'_{\ell}=\{ 0, s+4\}$, by Lemma~\ref{lem:0No2}, $v_{\ell}=e_0+e_{s+4}-e_{s+5}$, $[v_{\ell}]$ and $[v_1]$ share their right endpoint. As $\braket{v_{\ell}}{v_3}\neq 0$, $[v_1]$ equals the union of $[v_3]$ and $[v_{\ell}]$ by Lemma~\ref{lem:k1=2 sigma3}, i.e. $|v_1|=4$, a contradiction. Suppose $S'_{\ell}=\{ 2, s+4\}$. If $1\not \in S_{\ell}$, as $s+3\not\in S_{\ell}$, $\braket{v_{ell}}{v_{s+3}}\ne0$, there will be a heavy triple $(v_{\ell}, v_{s+3}, v_2)$. If $1\in S_{\ell}$ (and consequently, $|v_{\ell}|\ge 4$), then there will be a claw $(v_1, x_0, v_{\ell}, v_2)$, unless $[v_{\ell}]\pitchfork [v_1]$. If $[v_{\ell}]\pitchfork [v_1]$, however, we get $\delta([v_{\ell}],[v_1])=2$, and so $|\braket{v_{\ell}}{v_1}|\ge 2$, a contradiction to the fact that $\braket{v_{\ell}}{v_1}=-1$.

\noindent{\bf Step 2}. If $S'_{\ell}=\emptyset$ or $\{ s+3, s+4\}$, then $S_{\ell}=S'_{\ell}$. In particular, $S_{s+5}=\{ s+3, s+4\}$.

Suppose that $S'_{\ell}=\emptyset$ or $\{ s+3, s+4\}$. Let $i=\min\supp(v_{\ell})$.  By Lemma~\ref{lem:NoSingle1}, $i\ne1$. That is, $\braket{v_\ell}{v_1}=0$. Also, note that $v_\ell$ is orthogonal to $x_0$, and so $\delta([v_\ell], [v_1])\le 2$. If $3\le i\le s+2$, since $v_3\sim v_4\sim  \cdots\sim v_{i}\sim v_{\ell}$, using Lemmas~\ref{lem:k1=2 sigma3}~and~\ref{lem:k1>1 intervals}, $x_{z_{\ell}}\in [v_1]$. Therefore, $\braket{v_1}{v_\ell}\not = 0$, a contradiction. So $i\ge s+3$ and hence $S_{\ell}=S'_{\ell}$. Clearly, $S_{s+5}=\{ s+3, s+4\}$ by Lemma~\ref{lem:j-1}.
%Now $[v_1]$ contains at least three high weight vertices $x_{z_3},x_{z_{\ell}},x_{z_2}$. Since $|v_1|=5$, we must have $|v_{\ell}|=|x_{z_{\ell}}|=3$. So $v_{\ell}=e_i+e_{\ell-1}-e_{\ell}$. Thus we have a contradiction when $S'_{\ell}=\{ s+3, s+4\}$, and our conclusion holds in this case. In particular, since $s+4\in S_{s+5}$, $S_{s+5}=\{ s+3, s+4\}$,
%by Step~1 and the $S'_{\ell}=\{ s+3, s+4\}$ case of Step~2.

%When $S'_{\ell}=\emptyset$, using Lemma~\ref{gappy3}, $i$ must be $s+2$. Then $\braket{v_\ell}{v_{s+3}}\neq 0$. Since %$v_{s+5}=e_{s+3}+e_{s+4}-%e_{s+5}$, $\braket{v_{s+5}}{v_{s+3}}\neq 0$. We also have $\braket{v_{2}}{v_{s+3}}\neq 0$. By %Corollary~\ref{Cor:ZjDistinct}, the $3$ intervals $[v_\ell]$, $[v_{s+5}]$ and $[v_2]$ are all consecutive to $[v_{s+3}]$, which is impossible by %Corollary~\ref{Cor:ZjDistinct}.

\noindent{\bf Step 3}. If $S_{\ell}=\{ s+3, s+4\}$, then $\ell=s+5$. 

Assume that $\ell>s+5$ and $S_{\ell}=\{ s+3, s+4\}$, then we have a heavy triple $(v_{s+3},v_{s+5},v_{\ell})$.
\end{proof}

%Since $v_3\sim v_4\sim  \cdots\sim v_{s+2}$ and $\braket{v_2}{v_i}=0$ for all $3\le i \le s+2$, using Lemmas~\ref{lem:k1=2 sigma3}~and~\ref{lem:k1>1 intervals}, $[v_i]\prec [v_1]$. Since $s+4\in \supp(v_{s+5})$, then $S'_{s+5}$ is one of $\{ 0, s+4\}$, $\{ 2, s+4\}$, or $\{ s+3, s+4\}$ (Lemma~\ref{Lem:X0Odd} and Corollary~\ref{x0pairing}). Suppose $S'_{s+5}=\{ 0, s+4\}$. Then we must have $1\not \in S_{s+5}$ (Corollary~\ref{unbreakablepairing}). So $\braket{v_{s+5}}{v_1}=2$ and $\braket{v_{s+5}}{v_2}\neq 0$, and therefore, $[v_{s+5}]$ and $[v_1]$ share their right endpoint by Lemma~\ref{lem:k1>1 intervals}. That is, $\delta([v_1], [v_{s+5}])=1$. To get $\braket{v_{s+5}}{v_1}=2$, it must be that $|v_{s+5}|=3$, and so $3\not \in \supp(v_{s+5})$. But the latter implies that $\braket{v_{s+5}}{v_3}\neq 0$, and so $[v_1]$ equals the union of $[v_3]$ and $[v_{s+5}]$ by Lemma~\ref{lem:k1=2 sigma3}, i.e. $|v_1|=4$, a contradiction. Suppose $S'_{s+5}=\{ 2, s+4\}$. If $1\not \in S_{s+5}$, there will be a heavy triple $(v_{s+5}, v_{s+3}, v_2)$: see Figure~\ref{fig:case1 123 G(S)}. If $1\in S_{s+5}$ (and consequently, $|v_{s+5}|\ge 4$), then there will be a claw $(v_1, x_0, v_{s+5}, v_2)$, unless $[v_{s+5}]\pitchfork [v_1]$. If $[v_{s+5}]\pitchfork [v_1]$, however, we get $\delta([v_{s+5}],[v_1])=2$, and so $|\braket{v_{s+5}}{v_1}|\ge 2$, a contradiction (since it must be $\braket{v_{s+5}}{v_1}=-1$). Having shown $S'_{s+5}=\{s+3, s+4\}$, we now argue that $S_{s+5}=S'_{s+5}$. If $1 \in S_{s+5}$, then $\braket{v_{s+5}}{v_1}=-1$ and $\braket{v_{s+5}}{v_2}\neq 0$. So, using Lemma~\ref{lem:k1>1 intervals}, we see that $[v_{s+5}]$ shares its right endpoint with that of $[v_1]$, and so $\delta([v_{s+5}], [v_1])=1$. Since $|v_{s+5}|\ge 4$, we get that $|\braket{v_{s+5}}{v_1}|\ge 3$, a contradiction. If $v_{s+5}$ has nonzero inner product with any of the $v_i$ for $i\in \{ 3, \cdots, s+2\}$, then, by our earlier discussion, $[v_{s+5}]\cap[v_1]\not =\emptyset$. Since $v_{s+5}$ is orthogonal to $x_0$, we must have $\delta([v_{s+5}], [v_1])\le 2$. Since $|v_{s+5}|\ge 3$ in this case, however, we get that $\braket{v_{s+5}}{v_1}\neq 0$, a contradiction. That is, $S_{s+5}=S'_{s+5}$, and therefore, $\sigma_{s+5}=8s+10$. Now suppose that there exists $\ell > s+5$ such that $|v_i|=2$ for any $s+5<i<\ell$. We claim that $\text{min }\supp(v_\ell)=\ell-1$. Using Corollary~\ref{x0pairing}, $S'_\ell$ is either $\emptyset$, $\{ 0, s+4\}$, $\{ 2, s+4\}$, or $\{ s+3, s+4\}$. If $S'_\ell=\{ 0, s+4\}$ or $\{ 2, s+4\}$, the same argument used for the index $s+5$ will give us a contradiction in each case. If $S'_\ell=\{s+3, s+4\}$, then $\braket{v_{\ell}}{v_{s+5}}\neq 0$. Either $v_{\ell}\sim v_{s+3}$ and there will be a heavy triple $(v_{\ell}, v_{s+3}, v_{s+5})$, or $v_{\ell}\not \sim v_{s+3}$, and so $s+2\in \supp(v_{\ell})$ and $3, \cdots, s+1\not \in \supp(v_\ell)$. So $\braket{v_{\ell}}{v_{s+2}}\neq 0$, and therefore, $[v_{\ell}]\cap[v_1]\neq \emptyset$ since $[v_{s+2}]\prec[v_1]$. Since $v_\ell$ is orthogonal to $x_0$, we have $\delta([v_\ell], [v_1])\le 2$. We have $|v_\ell|\ge 4$, so $|\braket{v_\ell}{v_1}|\ge 2$, a contradiction. That is, $S'_{\ell}=\emptyset$. Set $j=\text{min }\supp(v_\ell)$. If $s+5<j<\ell-1$, there will be a claw $(v_j, v_{j-1}, v_\ell, v_{j+1})$, and if $j=s+5$, the claw will be on $v_{s+5}, v_{s+3}, v_j, v_{s+6}$. If $j=s+2$, then $\braket{v_\ell}{v_{s+3}}\neq 0$. Also, for some $s+5\le i \le \ell-1$, $\braket{v_\ell}{v_i}\neq 0$: there will be a heavy triple $(v_\ell, v_{s+5}, v_{s+3})$. If $3\le j<s+2$, using Lemma~\ref{gappy3}, then $\braket{v_\ell}{v_{s+3}}>1$. If $j=1$, then $\braket{v_\ell}{v_1}\neq 0$ and $\braket{v_\ell}{v_2}\neq 0$, and therefore, $[v_\ell]$ and $[v_1]$ share their right endpoint, and so $\delta([v_{\ell}], [v_1])=1$. Again, since $|v_\ell|\ge 3$, this implies that $\braket{v_\ell}{v_1}>1$, a contradiction. This justifies the claim that $j=\ell-1$. By induction we see that $|v_i|=2$ for all $i>s+5$.
