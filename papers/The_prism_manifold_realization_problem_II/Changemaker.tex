\section{Changemaker Lattices}\label{sec:ChangemakerLattices}

%%%%%%%%%%%%%
%%%%%%%%%%%%%%%
%%%%%%%%%%%%%%%%%
%%%%%%%%%%%%%%%%
%%%%%%%%%%%%%%%%%%%

A lattice is called a {\it changemaker lattice} if it is isomorphic to the orthogonal complement of a changemaker vector. Whenever $P(p, q)$, with $q>p$, comes from positive integer surgery on a knot, $C(p,q)$ is isomorphic to a changemaker lattice $(\sigma)^\perp\subset \Z^{n+2}$. 
In this section, we will assemble some basic structural results about C-type lattices that are isomorphic to changemaker lattices.

Write $(e_0,e_1, \dots, e_{n+1})$ for the orthonormal basis of $\Z^{n+2}$, and write $\sigma = \sum_i \sigma_i e_i$. 
Since $C(p,q)$ is indecomposable (Corollary~\ref{indecomposable}), $\sigma_0\ne0$, otherwise $(\sigma)^\perp$ would have a direct summand $\mathbb Z$. So $\sigma_0=1$.

We will need several results from \cite[Section~3]{greene:LSRP} about changermaker lattices:

\begin{definition}\label{stbasis}
The {\it standard basis} of $(\sigma)^\perp$ is the collection $S = \{v_1, \dots, v_{n}\}$, where
\begin{equation*}
    v_j = \left(2e_0 + \sum_{i = 1}^{j - 1} e_i\right) - e_j
\end{equation*}
whenever $\sigma_j = 1 + \sigma_0 + \cdots + \sigma_{j-1}$, and
\begin{equation*}
    v_j = \left(\sum_{i \in A} e_i\right) - e_j
\end{equation*}
whenever $\sigma_j = \sum_{i \in A} \sigma_i$, with $A \subset \{0, \dots, j-1\}$ chosen to maximize the quantity $\sum_{i \in A} 2^i$.  A vector $v_j \in S$ is called \emph{tight} in the first case, \emph{just right} in the second case as long as $i < j-1$ and $i \in A$ implies that $i+1\in A$, and \emph{gappy} if there is some index $i$ with $i \in A$, $i < j-1$, and $i+1 \not \in A$. Such an index, $i$, is a \emph{gappy~index} for $v_j$.
\end{definition}


The standard basis $S$ is in fact a basis of $C(p,q)$.

\begin{definition}
For $v \in \Z^{n+2}$, $\supp v = \{i | \braket{e_i}{v} \neq 0\}$ and $\supp^+ v = \{i | \braket{e_i}{v} > 0\}$.
\end{definition}

\begin{lemma}[Lemma~3.12~(3) in \cite{greene:LSRP}] \label{gappy3}
If $|v_{k+1}|=2$, then $k$ is not a gappy index for any $v_j$ with $j \in \{1, \cdots, n+1 \}$.
\end{lemma}

\begin{lemma}[Lemma~3.13 in \cite{greene:LSRP}] \label{lem:irred}
Each $v_j \in S$ is irreducible.
\end{lemma}

\begin{lemma}[Lemma~3.15 in \cite{greene:LSRP}]\label{lem:BrIsTight}
If $v_j \in S$ is breakable, then it is tight.
\end{lemma}

\begin{lemma}[Lemma~3.14~(2)~(3) in \cite{greene:LSRP}]\label{lem:SumIrr}
Suppose that $v_t \in S$ is tight.
\newline(1) If $v_j = e_t + e_{j-1} - e_j$, $j > t$, then $v_t + v_j$ is irreducible.
\newline(2) If $v_{t+1}=e_0+e_1+\cdots+e_t-e_{t+1}$, then $v_{t+1}- v_t$ is irreducible.
\end{lemma}


\begin{lemma}[Lemma~4.9 in \cite{Prism2016}]\label{lem:j-1}
For any $v_j\in S$, we have $j-1\in\supp v_j$.
\end{lemma}




%%%%%%%%%%%%%
%%%%%%%%%%%%%%%
%%%%%%%%%%%%%%%%%
%%%%%%%%%%%%%%%%
%%%%%%%%%%%%%%%%%%%

For the rest of this section, suppose $\sigma = (\sigma_0,\sigma_1,\dots,\sigma_{n+1}) \in \Z^{n+2}$ is a changemaker vector such that $(\sigma)^\perp$ is isomorphic to a C-type lattice $C(p,q)$ with $q > p$. Also, let $x_0,\dots,x_n$ be the vertex basis of $C(p,q)$, and let $S = (v_1,\dots,v_{n+1})$ be the standard basis of $(\sigma)^\perp$. Each $v_i$ is an irreducible element in a C-type lattice (Lemma~\ref{lem:irred}), so corresponds to some interval (Proposition~\ref{prop:IntervalsIrreducible}). By a slight abuse of notation, denote $[v_i]$ for the interval corresponding to $v_i$. Let $\epsilon_i\in\{\pm1\}$ satisfy $v_i = \epsilon_i [v_i]$. 

The C-type lattice $C(p,q)$ contains an element $x_0$ with $|x_0| = 4$, and any vector of norm $4$ in $\Z^{n+2}$ is of the form either $ \pm 2e_k$ or $\pm e_{k_0}\pm e_{k_1}\pm e_{k_2}\pm e_{k_3}$ for distinct indices $k_i$. Vectors of the first form cannot be in $(\sigma)^\perp$ since $\sigma_0 \neq 0$, so $x_0$ must be of the second form. In fact, we can say a little bit more about how $x_0$ can be written in terms of the $e_i$. We start by the following lemma.

%Note that the vectors $v_1, \cdots, v_{n+1}$ are clearly linearly independent. It is also straightforward to see that they span $C(p,q)$. Using Proposition~\ref{prop:IntervalsIrreducible}, it turns out that every element of $C(p,q)$ can be written as a linear combination of the $[v_i]$.{\color{red}Note}
\begin{lemma}\label{Lem:NoV2}
There is no element $v\in C(p,q)$ with $\braket{v}{x_0} \neq 0$ and $|v| = 2$. 
\end{lemma}
\begin{proof}
Since $C(p,q)$ is indecomposible, it contains no $x$ with $|x| = 1$ (such an $x$ would generate a $\Z$-summand of $C(p,q)$). Therefore, if $v \in C(p,q)$ with $|v| = 2$, it must be irreducible, so $v = \pm[I]$ for $[I]$ an interval. By Lemma~\ref{Lem:IntervalNorm}, $[I]$ contains only $x_0$ or elements of norm $2$. In particular, $[I]$ does not contain $x_1$, since $a_1 \ge 3$. This means that $[I]$ also cannot contain $x_0$, since then $[I] = x_0$ and $|v| = 4$. Therefore, $\braket{[I]}{x_0} = 0$, and so $\braket{v}{x_0} = 0$. 
\end{proof}

\begin{prop}\label{x0}
For some indices $k_1 < k_2 < k_3$, $x_0$ is equal to one of ${e_0 + e_{k_1} + e_{k_2} - e_{k_3}}$ or ${e_0 - e_{k_1} - e_{k_2} + e_{k_3}}$, possibly after a global sign change in the isomorphism between $(\sigma)^\perp$ and $C(p,q)$. 
\end{prop}
\begin{proof}
Since $|x_0| = 4$ and $x_0 \in (\sigma)^\perp$,
\begin{equation*}
x_0 = \delta_0 e_{k_0} + \delta_1 e_{k_1} + \delta_2 e_{k_2} + \delta_2 e_{k_3}
\end{equation*}
for indices $k_0 < k_1 < k_2 < k_3$ and signs $\delta_i$ such that $\sum_i \delta_i \sigma_i = 0$. By a global sign change, we might as well assume that $\delta_0 = 1$. If $k_0 > 0$, $\braket{x_0}{v_{k_0}} = -1$ is odd, violating Lemma~\ref{Lem:X0Odd}. So $k_0 = 0$.

We claim that if $\sigma_{k_i} = \sigma_{k_j}$, then $\delta_i = \delta_j$. Otherwise $v = \delta_i e_{k_i} + \delta_j e_{k_j}$ would be in $(\sigma)^\perp$ with $|v| = 2$ and $\braket{v}{x_0} = 2$, which contradicts Lemma~\ref{Lem:NoV2}. Therefore, if $\delta_1 = -1$ then $\sigma_1 > \sigma_0$, and so $\delta_0 \sigma_0 + \delta_1 \sigma_1 < 0$. Therefore, $\delta_2 \sigma_2 + \delta_3 \sigma_3 > 0$. Since $\sigma_2 \le \sigma_3$, this means that $\delta_3 = 1$, and then $\delta_2 = -1$ since $\sigma_1 < \sigma_0 + \sigma_2 + \sigma_3$. In the other case, if $\delta_1 = 1$ then $\delta_0 \sigma_0 + \delta_1 \sigma_1 > 0$, so $\delta_2 \sigma_2 + \delta_3 \sigma_3 < 0$ and $\delta_3 = -1$. If also $\delta_2 = -1$, then
\begin{equation*}
\sigma_0 + \sigma_1 = \sigma_2 + \sigma_3.
\end{equation*}
Since $\sigma_0 \le \sigma_1 \le \sigma_2 \le \sigma_3$, this can only happen if all of the $\sigma_i$ are equal, again contradicting the fact that if $\sigma_i = \sigma_j$ we must have $\delta_i = \delta_j$.
\end{proof}

\begin{cor}\label{v1}
The vector $v_1$ is equal to $2e_0 - e_1$ if $k_1 > 1$, and $e_0 - e_1$ otherwise. If ${x_0 = e_0 - e_{k_1} - e_{k_2} + e_{k_3}}$, the first of these occurs. 
\end{cor}
\begin{proof}
Note that $v_1$ is always either $e_0 - e_1$ or $2e_0 - e_1$. Using Lemma~\ref{Lem:X0Odd}, the first statement of the lemma follows. For the second statement, if $k_1 = 1$ and $v_1 = e_1 - e_0$, then if ${x_0 = e_0 - e_{k_1} - e_{k_2} + e_{k_3}}$ we have that $\braket{v_1}{x_0} = 2$ and $|v_1| = 2$, contradicting Lemma~\ref{Lem:NoV2}. 
\end{proof}

\begin{lemma}\label{lem:tightvector}
If $k_1 > 1$, $v_1$ is the only tight vector. If $k_1 = 1$, $v_{k_2}$ can be tight but there is no other tight vector.
\end{lemma}
\begin{proof}
We claim that if $v_t$ is tight, then either $t < k_1$ or $t= k_2$. Using Lemma~\ref{Lem:X0Odd}, we must have that either $k_2 \le t < k_3$ or $t < k_1$ as otherwise $v_t$ will have odd pairing with $x_0$. If $k_2<t < k_3$, then 
\[
\sigma_t = 1 + \sigma_0 + \sigma_1 + \cdots + \sigma_{t-1} \ge 1 + \sigma_0 + \sigma_{k_1} + \sigma_{k_2}.
\]
 However, by Proposition~\ref{x0}, the fact that $\braket{x_0}{\sigma} = 0$ implies that
\begin{equation*}
\sigma_{k_3} = \sigma_{k_2} + \sigma_{k_1} \pm \sigma_0 \le \sigma_{k_2} + \sigma_{k_1} + \sigma_0 < \sigma_t,
\end{equation*}
contradicting the fact that $t < k_3$. The claim follows. 

If $k_1 = 1$, it is only possible that $t = k_2$, so the second statement of the lemma follows. Suppose now that $k_1 > 1$. We have that $v_1 = 2e_0 - e_1$ by Corollary~\ref{v1}. So if $v_t$ is tight with $t > 1$, we get that $\braket{v_1}{v_t} = 3$ and $|v_t| > |v_1| = 5$. Also, since either $t < k_1$ or $t = k_2$, $\braket{v_t}{x_0} = \braket{v_1}{x_0} = 2$. Therefore, either $\epsilon_1 = -1$ and $[v_1]$ has left endpoint $1$, or $\epsilon_1 = 1$ and $[v_1]$ has left endpoint $0$, and the same holds for $\epsilon_t$ and $[v_t]$. By Lemma~\ref{intervalproduct}, 
\begin{equation*}
3 = \braket{v_1}{v_t} = \epsilon_1\epsilon_t(|[v_1 \cap v_t]| - \delta([v_1],[v_t]) ),
\end{equation*}
$|[v_1 \cap v_t]| \ge 2$ and $\delta([v_1],[v_t]) \le 3$, so if $\epsilon_1\neq \epsilon_t$, the right hand side of this equation is at most $1$. Therefore, $\epsilon_1 = \epsilon_t$, and the left endpoints of $[v_1]$ and $[v_t]$ are equal. Since $|v_t| > |v_1|$, the right endpoint of $[v_t]$ is to the right of the right endpoint of $[v_1]$. This means that $\delta([v_1],[v_t]) = 1$ and $v_1 \cap v_t =v_1$, so
\begin{equation*}
\braket{v_1}{v_t} = \epsilon_1\epsilon_t(|[v_1 \cap v_t]| - \delta([v_1],[v_t]) ) = |[v_1]| - 1 = 4 \neq 3.
\end{equation*}
Therefore, $v_1$ is the only tight vector.
\end{proof}

%For $j = k_3$, this holds unless ${x_0 = e_0 - e_{k_1} - e_{k_2} + e_{k_3}}$ and either $v_{k_3} = -e_{k_3} + e_{k_1} + e_i$ of $v_{k_3} = -e_{k_3} + e_{k_2} + e_i$ for some $i > 1$. (I'm not sure we'll ever need this second statement, but it could be helpful for the case $k_1 > 1$). 


\begin{lemma}\label{positivity}
For $j \neq k_3$, $\braket{v_j}{x_0} \ge 0$. 
\end{lemma}
\begin{proof}
Using~Proposition~\ref{x0}, either ${x_0 = e_0 + e_{k_1} + e_{k_2} - e_{k_3}}$ or $x_0= e_0 - e_{k_1} - e_{k_2} + e_{k_3}$. If $x_0 = e_0 + e_{k_1} + e_{k_2} - e_{k_3}$, it would only be possible to have $\braket{v_j}{x_0} < 0$ for $j = k_1$ or $j = k_2$. However, in these cases one has $\braket{v_j}{x_0} \ge -1$, and since $\braket{v_j}{x_0}$ is even, it follows that $\braket{v_j}{x_0} \ge 0$. If ${x_0 = e_0 - e_{k_1} - e_{k_2} + e_{k_3}}$, then $\braket{v_j}{x_0}$ is always at least $-3$, since $\braket{v_j}{e_0} \ge 0$. Therefore, since it is even, $\braket{v_j}{x_0} \ge -2$. Given that $j \ne k_3$, the only possible way to have $\braket{v_j}{x_0} = -2$ is that $k_1, k_2 \in \supp^+(v_j)$, and $0,k_3 \not \in \supp^+(v_j)$. Observe that this cannot happen since then $v_j + x_0$ is still of the form $-e_j + \sum_{i \in A'} e_i$ for some $A' \subset \{0,\dots,j-1\}$, but $A'$ is lexicographically after $\supp^+ v_j$, contradicting the maximality criterion in Definition~\ref{stbasis}.
\end{proof}

%The third case cannot happen, since in order to have $\braket{v_j}{\sigma} = 0$, $\sigma_{k_3} \ge \sigma_{0} + \sigma_{k_1} + \sigma_{k_2}$, but since $\braket{x_0}{\sigma} = 0$, $\sigma_{k_3} = \sigma_{k_1} + \sigma_{k_2} - \sigma_0$.
%
%Finally, in the second case, $\braket{v_j}{v_1}$ is either $0$ or $-1$ since $0 \not \in \supp v_j$, and $|v_j| \ge 3$ since $\braket{v_j}{x_0} \neq 0$ and $a_0 \ge 3$. Since $\braket{v_1}{x_0} = 2$, either $\epsilon_1 = -1$ and the left endpoint of $[v_1]$ is $1$, or $\epsilon_1$ is $1$ and the left endpoint of $[v_1]$ is $0$. Since $\braket{v_j}{x_0} = -2$, the reverse holds for $v_j$. Therefore, either $\epsilon_1 = \epsilon_j$ and $[v_1]$ and $[v_j]$ have different left endpoints, or $\epsilon_1 \neq \epsilon_j$ and $[v_1]$ and $[v_j]$ have the same endpoint. Also, since $v_j$ is unbreakable, $|[v_1 \cap I_j]| = |[v_j]| = |v_j|$.  In the formula
%\begin{equation*}
%\braket{v_1}{v_j} = \epsilon_1\epsilon_j\left(|[v_1\cap I_j]| - \delta([v_1],[v_j])\right),
%\end{equation*}
%then, either $\epsilon_1\epsilon_j$ is positive and $\delta([v_1],[v_j])$ is $2$ or $3$, or $\epsilon_1\epsilon_j$ is negative and $\delta([v_1],[v_j]) = 1$. This second case cannot occur, since then the pairing would be at most $-2$ since $|[v_1\cap I_j]| \ge 3$. Therefore, the first case occurs, so $\braket{v_1}{v_j} \ge 0$ with equality only when $\delta([v_1],[v_j]) = |v_j| = 3$. Since $j = k_3$, $|v_j| = 3$, exactly one of $k_1,k_2$ is in $\supp(v_j)$, and $\braket{v_j}{v_1} = 0$, the statement follows. 


%The proof of the next rather technical lemma follows closely the proof of~\cite[Lemma~4.17]{Prism2016}. We provide the proof for completeness.

\begin{lemma}\label{excludedintervals}
If $v_i$ and $v_j$ are two unbreakable standard basis vectors with $i,j \neq k_3$, then it cannot be the case that $[v_i]$ contains $x_0$ and $[v_j]$ contains $x_1$ but not $x_0$. In particular, $\delta([v_i],[v_j]) \le 2$.
\end{lemma}
\begin{proof}
Assume the contrary. Since $i,j \neq k_3$, and $k_3 = \max \supp(x_0)$, neither $v_i$ nor $v_j$ is equal to $\pm x_0$, and by Lemma~\ref{positivity}, $\braket{v_i}{x_0}$ and $\braket{v_j}{x_0}$ are both nonnegative. Therefore, $\braket{v_i}{x_0} = \braket{v_j}{x_0} = 2$. Since $x_0$ is contained in $[v_i]$, the left endpoint of $[v_i]$ is $0$ and $\epsilon_i = 1$. Similarly, $[v_j]$ has left endpoint $1$ and $\epsilon_j = -1$. Therefore, $\delta([v_i], [v_j])$ is either $2$ or $3$, and since $v_i$ and $v_j$ are unbreakable and $a_1\ge3$, $z_i=z_j=1$ and $|[v_i \cap v_j]| = |v_i| = |v_j|=a_1$. This means that
\begin{equation}\label{OwnRef}
\braket{v_i}{v_j} = \epsilon_i \epsilon_j \left(|[v_i \cap v_j]| - \delta([v_i],[v_j])\right) = -|v_i| + \delta([v_i],[v_j]) = -|v_j| + \delta([v_i],[v_j])
\end{equation}
Since $v_i$ and $v_j$ are standard basis vectors, $\braket{v_i}{v_j} \ge -1$. Since $|v_i| \ge 3$ and $\delta([v_i],[v_j])$ is either $2$ or $3$, $|v_i|$ is either $3$ or $4$. That is, using~Equation~\eqref{OwnRef}, $\braket{v_i}{v_j}$ is equal to $-1$ if $|v_i| = 4$ and either $0$ or $-1$ if $|v_i| = 3$. In particular, 
\begin{equation}\label{eq:ijNeg}
\braket{v_i}{v_j}\le 0.
\end{equation}

Using Proposition~\ref{x0}, suppose first that $x_0 = e_0 + e_{k_1} + e_{k_2} - e_{k_3}$. Then since $\braket{v_i}{x_0} = \braket{v_j}{x_0} = 2$ and $i,j \neq k_3$, each of $\supp^+(v_i)$ and $\supp^+(v_j)$ contain at least two of $0, k_1$, and $k_2$, and $i,j\notin\{k_1,k_2\}$. In particular, $\supp^+(v_i)$ and $\supp^+(v_j)$ intersect, so $\braket{v_i}{v_j} \ge 0$. Therefore, using Equation~\eqref{OwnRef} and the earlier discussion, we must have $|v_i| = |v_j| = 3$, so $\supp^+(v_i)$ and $\supp^+(v_j)$ in fact contain no elements outside of $\{0,k_1,k_2\}$. In particular, $\supp^+(v_i)$ does not contain $j$, and vice versa, $\supp^+(v_j)$ does not contain $i$. Therefore, we get that $\braket{v_i}{v_j} \ge 1$ which is a contradiction to (\ref{eq:ijNeg}). 

If now $x_0 = e_0 - e_{k_1} - e_{k_2} + e_{k_3}$, then since $\braket{v_i}{x_0} = 2$ and $i \neq k_3$, there are two cases: Case~1 is that
$\supp^+(v_i)$ contains $0$ and $k_3$ but not $k_1$ and $k_2$, and Case~2 is that $i = k_2$ or $k_1$, $\supp^+(v_i)$ contains $0$, and (if $i = k_2$), $\supp^+(v_i)$ does not contain $k_1$. The same holds for $v_j$. 
If one of $v_i$ and $v_j$ is in Case~1, then $\braket{v_i}{v_j} \ge 1$, a contradiction to (\ref{eq:ijNeg}). If both $v_i$ and $v_j$ are in Case~2, we may assume $i=k_1$ and $j=k_2$, and we still have $\braket{v_i}{v_j} \ge 1$, a contradiction.
\end{proof}

\begin{cor}\label{unbreakablepairing}
If $v_i$ and $v_j$ are two unbreakable standard basis vectors with $i \neq j$ and $i,j \neq k_3$, then $|\braket{v_i}{v_j}| \le 1$, with equality if only if $[v_i]$ abuts $[v_j]$. 
\end{cor}
\begin{proof}
If neither $[v_i]$ nor $[v_j]$ contains $x_0$, then both $v_i$ and $v_j$ are contained in a linear sublattice of $C(p,q)$ and this reduces to~\cite[Lemma~4.4]{greene:LSRP}. Similarly, if one of $[v_i]$ or $[v_j]$ contains $x_0$ and the other contains neither $x_0$ nor $x_1$, or if both $[v_i]$ and $[v_j]$ contain $x_0$, then reflecting both $v_i$ and $v_j$ about $x_0^{\perp}$ puts both of them in a linear sublattice of $C(p,q)$. Using Lemma~\ref{excludedintervals}, these are the only possibilities.
\end{proof}



\begin{cor}\label{Cor:ZjDistinct}
If $v_i$ and $v_j$ are unbreakable with $|v_i|,|v_j| \ge 3$, $i \neq j$ and $i,j \neq k_3$, then ${z_i} \neq {z_j}$, where ${z_i}$ and ${z_j}$ are defined in Definition~\ref{Def:Zj}.
\end{cor}
\begin{proof}
Suppose for contradiction $x_{z_i} = x_{z_j}$. By Lemma~\ref{excludedintervals}, $\delta([v_i],[v_j]) \le 2$. Therefore, using Lemmas~\ref{intervalproduct}~and~\ref{Lem:IntervalNorm},
\begin{equation}\label{eq:vivjGe1}
\braket{[v_i]}{[v_j]} = |[v_i \cap v_j]| - \delta([v_i],[v_j]) = |x_{z_i}| - \delta([v_i],[v_j]) \ge 3 - 2 = 1,
\end{equation}
By Corollary~\ref{unbreakablepairing}, $\braket{[v_i]}{[v_j]} =1$ and $[v_i]$ abuts $[v_j]$. We would then have $\delta=1$, so the equality in (\ref{eq:vivjGe1}) cannot be attained, a contradiction.
\end{proof}

\begin{cor}\label{x0pairing}
There is at most one $j \neq k_3$ for which $v_j$ is unbreakable and $\braket{v_j}{x_0}$ is nonzero. 
\end{cor}
\begin{proof}
Since $a_1 \ge 3$, if there exists an unbreakable standard basis element $v_j$ for which $\braket{v_j}{x_0} \neq 0$, $j \neq k_3$, then $x_{z_j} =x_1$. It follows from Corollary~\ref{Cor:ZjDistinct} that there exists at most one such $j$.
\end{proof}

Since the pairings of $v_{k_3}$ with other standard basis vectors are difficult to control, and since Corollary~\ref{x0pairing} gives good control on the pairings between $x_0$ and the other standard basis vectors, it will be easier in what follows if we replace $S$ with the modified basis 
\begin{equation}\label{Eq:S'}
S' = \left(S \setminus \{v_{k_3}\}\right) \cup \{x_0\}. 
\end{equation}
The set $S'$ is still a basis of $(\sigma)^\perp$ because $\braket{x_0}{e_{k_3}} = \pm 1$ but $\braket{x_0}{e_j} = 0$ for $j > k_3$, so if we write $x_0$ as a linear combination of elements of $S$, the coefficient of $v_{k_3}$ will be $\pm 1$.

Using Lemmas~\ref{unbreakablepairing} and \ref{x0pairing}, we can relate the pairings between elements of $S'$ very closely to the geometry of the intervals. It will be convenient to use two graphs associated to a C-type lattice. Recall that the pairing graph $\hat{G}(V)$ for a subset $V$ of a lattice $L$ has vertex set $V$ and an edge $(v_i,v_j)$ whenever $\braket{v_i}{v_j} \neq 0$ (Definition~\ref{defn:pairinggraph}).

%\begin{definition} %moved to section 2
%Given a lattice $L$ and a subset $V\subset L$, the {\it pairing graph} is $\wh{G}(V) = (V, E)$, where $e = (v_i, v_j) \in E$ if $\langle v_i, v_j \rangle \neq 0$.
%\end{definition}
 
\begin{definition}
If $T$ is a set of irreducible vectors in a C-type lattice $C(p,q)$, the {\it intersection graph} $G(T)$ has vertex set $T$, and an edge between $v$ and $w$ if the intervals corresponding to $v$ and $w$ abut. We write $v\sim w$ if $v$ and $w$ are connected in $G(T)$.
\end{definition}

\begin{lemma}\label{lem:PairingNonzero}
If the intervals corresponding to $v$ and $w$ abut, then $\braket vw\ne0$.
\end{lemma}
\begin{proof}
If one of $v,w$ is $x_0$, $\braket vw=\pm2\ne0$. If none of $v,w$ is $x_0$, then $\delta([v],[w])=1$, our conclusion follows from Lemma~\ref{intervalproduct}.
\end{proof}

The following is immediate from Corollary~\ref{unbreakablepairing} and Lemma~\ref{lem:PairingNonzero}:
\begin{prop}
For $T \subset S'$, $G(T)$ is obtained from $\hat{G}(T)$ by removing some edges incident to breakable vectors.
\end{prop}

In particular, if we write $\bar{S'}$ for the set of unbreakable elements of $S'$, ${G(\bar{S'}) = \hat{G}(\bar{S'})}$. The main use we have for this result is the following structural facts about the intersection graph. 

\begin{definition}
A {\it claw} in a graph $G$ is a quadruple $(v,w_1,w_2,w_3)$ of vertices such that $v$ neighbors all the $w_i$, but no two of the $w_i$ neighbor each other.
\end{definition}

\begin{lemma}[Lemma~4.8 of~\cite{greene:LSRP}]\label{claw}
The intersection graph $G(T)$ has no claws.
\end{lemma}

\begin{definition}
Given a set $T$ of unbreakable elements in a C-type lattice and $v_1,v_2,v_3 \in T$, $(v_1,v_2,v_3)$ is a {\it heavy triple} if $|v_i| \ge 3$ and $v_i\ne\pm x_0$ for each $i$, and if each pair among the $v_i$ is connected by a path in $G(T)$ disjoint from the third. 
\end{definition}

\begin{lemma}[Based on Lemma~4.10 of~\cite{greene:LSRP}]\label{heavytriple}
$G(\bar{S'})$ has no heavy triples.
\end{lemma}
\begin{proof}
If $v_i,v_j,$ and $v_k$ are unbreakable and have norm at least $3$, and none of them is $\pm x_0$, then by Corollary~\ref{Cor:ZjDistinct} we might as well assume ${z_i} < {z_j} < {z_k}$. Then any path from $v_i$ to $v_k$ in $G(\bar{S'})$ includes some $v_\ell \in \bar{S'}$ such that $[v_\ell]$ contains $x_{z_j}$, where $\bar{S'}$ is defined in~\eqref{Eq:S'}. But then $\ell = j$, so $(v_i,v_j,v_k)$ is not heavy.
\end{proof}

The proof of the following lemma is identical to~\cite[Lemma~3.8]{greene:LSRP}.

\begin{lemma}\label{Lem:CompleteSubgraph}
If the elements of $T$ are linearly independent, any cycle in $G(T)$ induces a complete subgraph.
\end{lemma}

\begin{cor}[Based on Lemma~4.11 of~\cite{greene:LSRP}]\label{cycles}
Any cycle in $G(\bar{S'})$ has length three.
\end{cor}
\begin{proof}
By Corollary~\ref{x0pairing}, any cycle in $G(\bar{S'})$ does not contain $x_0$.
Using Lemma~\ref{Lem:CompleteSubgraph}, the cycle will contain at most two vertices of norm $>2$ to avoid producing a heavy triple. (See Definition~\ref{Def:HighNorm}.) If it had two vertices of norm $2$, using Lemma~\ref{Lem:CompleteSubgraph}, they would have nonzero inner product, so must be of the form $v_i = e_{i-1} - e_i$ and $v_{i+1} = e_i - e_{i+1}$ for some $i$. But for any other $j$ ($j\ne i, i+1$), Lemma~\ref{gappy3} implies that $\supp(v_j) \cap \{i-1,i,i+1\}$ is one of $\emptyset$, $\{i+1\}$, $\{i,i+1\}$, or $\{i-1,i,i+1\}$. In none of these cases does $v_j$ have nonzero inner product with both $v_i$ and $v_{i+1}$, a criterion that must be fulfilled by Lemma~\ref{Lem:CompleteSubgraph}. That is, any cycle in $G(\bar{S'})$ must be of length three. 
\end{proof}

\begin{lemma}\label{lem:AllNorm2}
Let $m<N$ be two possitive integers satisfying $k_3\notin[m,N]$. Suppose that $v_m$ is unbreakable and it neighbors either $x_0$ or some unbreakable $v_j$ with $j<m$. Suppose that for any index $i$ satisfying $m< i\le N$, we have $\min\supp(v_i)\ge m$, and $v_i$ is unbreakable. Then $|v_i|=2$ for any $i$ satisfying $m< i\le N$.
\end{lemma}
\begin{proof}
When $i=m+1$, we clearly have $|v_i|=2$. Now assume $|v_i|=2$ for any $i$ satisfying $m< i< l\le N$, we want to prove $|v_l|=2$. Let $t=\min\supp(v_l)\ge m$, then $v_l$ is just right by Lemmas~\ref{gappy3} and \ref{lem:j-1}. If $m<t<l-1$, we would have a claw $(v_t,v_l,v_{t-1},v_{t+1})$. If $t=m$ and $v_m$ neighbors $x_0$, we would have a claw $(v_m,v_l,x_0,v_{m+1})$ by Corollary~\ref{x0pairing}. If $t=m$ and $v_m$ neighbors an unbreakable $v_j$ with $j<m$, we would have a claw
$(v_m,v_l,v_{j},v_{m+1})$. So $t=l-1$ and $|v_l|=2$.
\end{proof}

