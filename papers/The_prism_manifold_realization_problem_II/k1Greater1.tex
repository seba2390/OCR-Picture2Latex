\section{$k_1 > 1$}\label{sec:Case1}
In the present section we classify all the changemaker C-type lattices that have 
\[
x_0=e_0 \pm e_{k_1} \pm e_{k_2} \pm e_{k_3},
\]
where $k_1>1$. Using Lemma~\ref{Lem:X0Odd}, we know that
\begin{equation}\label{eq:v1}
v_1=2e_0 - e_1,
\end{equation}
and therefore, $\sigma_1 =2$ and $|v_1|=5$. We remind the reader that, by Lemma~\ref{lem:tightvector}, $v_1$ is the only tight vector in the C-type lattices that concern us in this section. We also note that 
\begin{equation}\label{eq:0inVk1}
0\in\supp(v_{k_1})
\end{equation}
 by Lemma~\ref{Lem:X0Odd}.
Compared to Sections~\ref{sec:Case2}~and~\ref{sec:k1k21}, it will take longer to determine the initial segment $(\sigma_0, \cdots, \sigma_{k_3})$ of $\sigma$. We start by specifying the positive integer $k_1$. 


\begin{lemma}\label{lem:k1>1,v2+t}
		The segment $(\sigma_0, \cdots \sigma_{k_1})$ is either $(1,2,3)$ or $(1,2,2,3)$. In particular, $k_1=2$ or $3$, and $\sigma_{k_1}=3$. 
\end{lemma}

\begin{proof}
Using Lemma~\ref{lem:tightvector}, we get that $v_2$ is either $e_0+e_1-e_2$ or $e_1-e_2$. In the former case, using Lemma~\ref{Lem:X0Odd}, we get that $k_1=2$, and so $\sigma_{k_1}=3$. 

Now suppose that $v_2=e_1-e_2$. More generally, suppose that there exists $t\ge 1$ such that $(\sigma_0, \sigma_1, \cdots, \sigma_{t+1})=(1,2,2^{[t]})$, and that $|v_{t+2}|>2$. We will show that $t=1$, $k_1=3$, and that $\sigma_{t+2}$ (or simply $\sigma_3$) is $3$. 

Set $j=\min\supp(v_{t+2})$. We argue that $j=0$. (Note that, by Lemma~\ref{gappy3}, none of $1,2, \cdots, t$ is a gappy index for $v_{t+2}$.) If $1<j<t+1$, there will be a claw on $v_j, v_{j-1}, v_{t+2}, v_{j+1}$. If $j=1$, then $\braket{v_{t+2}}{v_1}=-1$ and $v_{t+2}$ will be orthogonal to $v_2$. Then $k_1>t+2$ by \eqref{eq:0inVk1}.
There will be a claw on $v_1, x_0, v_{t+2}, v_2$, unless $[v_{t+2}]\pitchfork [v_1]$, $|[v_1\cap v_{t+2}]|=|v_{t+2}|=3$, and $\epsilon_{t+2}=-\epsilon_1$. Thus $v_1+v_{t+2}$ is the sum of two distant intervals and so is reducible. Since $|v_{t+2}|=3$, $v_{t+2}=e_1+e_{t+1}-e_{t+2}$, and so $v_1+v_{t+2}$ is irreducible by Lemma~\ref{lem:SumIrr}, a contradiction. That is, $j=0$, and that, 
\begin{equation}\label{vt+2}
v_{t+2}=e_0+e_{i}+e_{i+1} + \cdots + e_{t+1}-e_{t+2},
\end{equation}
with $i\ge 1$. 

Since $0\in \supp^+(v_{t+2})$, using Lemma~\ref{Lem:X0Odd}, we get that $k_1=t+2$. Furthermore, we claim that 
$x_0=e_0+e_{t+2}+e_{k_2}-e_{k_3}$. See Proposition~\ref{x0}. If $x_0=e_0-e_{t+2}-e_{k_2}+e_{k_3}$, then $\braket{v_{t+2}}{x_0}=2$. 
% Since $|v_2|=2$ and $\braket{v_1}{v_2}=-1$, it cannot be the case that $[v_2]\pitchfork [v_1]$ (as otherwise $\braket{v_1}{v_2}=0$). Moreover, since $v_2$ is orthogonal to $x_0$, it must be that either $[v_1]$ and $[v_2]$ share the right endpoint of $[v_1]$, or $[v_2]\dagger[v_1]$ and the left endpoint of $[v_2]$ is to the right of the right endpoint of $[v_1]$. In either of these cases, since $[v_{t+2}]$ is an interval of high norm, it cannot be both $[v_{t+2}]\pitchfork [v_1]$ and $\braket{v_2}{v_{t+2}}$. 
Observe that $\braket{v_{t+2}}{v_1}=1$ or $2$ depending on whether or not $i=1$ in~\eqref{vt+2}; in particular, $\braket{v_{t+2}}{v_1}>0$. Since $|v_{t+2}\cap v_1|=|v_{t+2}|\ge 3$ and $\delta([v_1],[v_{t+2}])\le 3$, using Lemma~\ref{intervalproduct}, it must be that $\epsilon_1 = \epsilon_{t+2}$. Since $\braket{v_1}{x_0}=\braket{v_{t+2}}{x_0}=2$, $[v_1]$ and $[v_{t+2}]$ share their left endpoint, and $\delta([v_{t+2}], [v_1])=1$. Moreover, we must have $|v_{t+2}|=3$ (as otherwise $\braket{v_{t+2}}{v_1}>2$). That is, $v_{t+2}=e_0+e_{t+1}-e_{t+2}$. We have $\braket{v_2}{v_1}=-1$ and $\braket{v_2}{x_0}=0$, so $[v_2]$ abuts the right end of $[v_1]$.
Since also $v_{t+2}\sim v_{t+1}$, $|v_i|=2$ for $i\in \{ 2, \cdots, t+1\}$,
\[
v_2\sim v_3\sim\cdots\sim v_{t+1},
\]
 the interval $[v_1]$ is a subset of the union of the $[v_j]$ for $j\in \{ 2, \cdots, t+2\}$, which in turn implies that $|v_1|=|v_{t+2}|=3$, a contradiction. This shows that 
\[
x_0=e_0+e_{t+2}+e_{k_2}-e_{k_3}.
\]

We now argue that $1\not \in \supp(v_{t+2})$. Suppose for contradiction that $1\in \supp(v_{t+2})$. Using (\ref{vt+2}), we get
 that $|v_{t+2}|\ge 4$, $\braket{v_1}{v_{t+2}}=1$ and $\braket{v_2}{v_{t+2}}=0$. To avoid a claw on $v_1, x_0, v_{t+2}, v_2$, we must have $[v_{t+2}]\pitchfork [v_1]$. This implies that $\delta([v_1], [v_{t+2}])=2$. Using Lemma~\ref{intervalproduct} and that $|v_{t+2}|\ge 4$, we see that $|\braket{v_1}{v_{t+2}}|\ge 2$, a contradiction. That is, in~\eqref{vt+2}, we must have $i>1$. 

We claim that $i=2$. If $2<i<t+1$, there will be a claw on $v_i, v_{i-1}, v_{t+2}, v_{i+1}$. If $i=t+1$ (and $i>2$), to avoid a claw on $v_1, x_0, v_{t+2}, v_2$, it must be that $[v_{t+2}]\pitchfork [v_1]$, and so $\delta([v_{t+2}], [v_1])=2$. To get $\braket{v_{t+2}}{v_1}=2$, however, it must be $|v_{t+2}|=4$ which contradicts $i=t+1$. Therefore, in~\eqref{vt+2}, we have $i=2$. In particular, $v_2\sim v_{t+2}$.

Finally, we argue that $t=1$. If $t>1$, we must have $v_1\sim v_{t+2}$ as otherwise we get a claw $(v_2, v_1, v_{t+2}, v_3)$. That is, $[v_{t+2}]$ abuts $[v_1]$. Therefore, to fulfill $\braket{v_{t+2}}{v_1}=2$, $[v_{t+2}]\prec[v_1]$, and that $|v_{t+2}|=3$, which contradicts $t>1$ and (\ref{vt+2}). So $t=1$ as desired.
\end{proof}

As part of the proof of Lemma~\ref{lem:k1>1,v2+t}, we showed that $x_0 = e_0+e_{k_1}+e_{k_2}-e_{k_3}$ when $k_1=3$. Indeed, this is the case also when $k_1=2$.

\begin{lemma}\label{lem:k1>1, chi=2}
		Let $k_1>1$. Then $x_0 = e_0+e_{k_1}+e_{k_2}-e_{k_3}$.
\end{lemma}

\begin{proof}
We only need to show this for $k_1=2$. Suppose for contradiction $x_0 = e_0-e_{2}-e_{k_2}+e_{k_3}$ (see Proposition~\ref{x0}). Note that $v_2=e_0+e_1-e_2$, and therefore, $\braket{v_2}{x_0}=2=\braket{v_1}{x_0}$, and $\braket{v_2}{v_1}=1$. Since $|v_2|=3$, using Lemma~\ref{intervalproduct}, we see that $\epsilon_1=\epsilon_2$ and $\delta([v_1],[v_2])=2$. Since $\braket{[v_2]}{x_0}=\braket{[v_1]}{x_0}=\pm2$, $[v_1],[v_2]$ share their left end point, so we cannot have $\delta([v_1],[v_2])=2$, a contradiction.
\end{proof}

Now we proceed to determine the changemaker vectors. As in Section~\ref{sec:Case2}, we use the notation of~\eqref{Eq:Sj}~and~\eqref{Eq:S'j}. Also, we use the basis $S'$, defined in~\eqref{Eq:S'}, where $v_{k_3}$ is replaced by $x_0$.

\subsection{{\boldmath $k_1=2$}}\label{k1=2}

This subsection is devoted to classifying the changemaker C-type lattices with 
\begin{equation}\label{x0k1=2}
x_0=e_0+e_2+e_{k_2} - e_{k_3}.
\end{equation} 
Recall that the changemaker starts with $(1,2,3)$. We have
\begin{equation}\label{eq:v2parings}
\braket{v_1}{v_2}=1,\quad \braket{v_2}{x_0}=0.
\end{equation}

\begin{lemma}\label{lem:k1>1 intervals}
		The intervals $[v_2]$ and $[v_1]$ are consecutive with $\epsilon_2 = -\epsilon_1$. 
\end{lemma}

\begin{proof}
Using (\ref{eq:v2parings}) and Lemma~\ref{intervalproduct}, either $[v_2]\pitchfork [v_1]$, $|[v_2]\cap[v_1]|=|[v_2]|=3$, $\delta([v_2], [v_1])=2$, and $\epsilon_2=\epsilon_1$, or $[v_2]\dagger [v_1]$, and $\epsilon_2 = -\epsilon_1$. In the former case, $v_2-v_1$ is the sum of two distant intervals, and so is reducible. However, we have $v_2=e_0+e_1-e_2$, and so $v_2-v_1$ is irreducible by Lemma~\ref{lem:SumIrr}~(2).
\end{proof}

\begin{lemma}\label{lem:NoSingle1}
There does not exist an index $j>3$, $j\ne k_3$, such that $\supp(v_j)\cap\{0,1,2\}=\{1\}$.
\end{lemma}
\begin{proof}
Otherwise, we will have $\braket{v_{j}}{v_1}=-\braket{v_{j}}{v_2}=-1$. We also have $\braket{v_{j}}{x_0}=0$ by Lemma~\ref{Lem:X0Odd}. By Lemma~\ref{lem:k1>1 intervals}, $[v_{j}]$ and $[v_1]$ share their right endpoint, so $\delta([v_j],[v_1])=1$. By  Lemma~\ref{intervalproduct}, $|\braket{v_1}{v_j}|=|v_{j}|-1>1$, a contradiction. 
\end{proof}



\begin{lemma}\label{lem:k1=2 sigma3}
		$\sigma_3 \in \{3,4\}$. Furthermore, if $\sigma_3 = 4$ then $[v_3]$ and $[v_1]$ share their left endpoint, and that $\epsilon_3 = \epsilon_1$.  
\end{lemma}

\begin{proof}
All the possibilities for $\sigma_3$ lie in $\{ 3,4,5,6\}$. If $\sigma_3=5$, we get that $v_3=e_1+e_2-e_3$. So $\braket{v_3}{v_1}=-1$ and $v_3$ is orthogonal to $v_2$. By Lemma~\ref{Lem:X0Odd}, $k_2=3$ and $\braket{v_3}{x_0}=0$.
Using Lemma~\ref{lem:k1>1 intervals}, we know that $[v_2]$ abuts $[v_1]$, and therefore, there will be a claw on $v_1, x_0, v_3, v_2$, unless $[v_3]\pitchfork [v_1]$, $|[v_1]\cap [v_3]|=|v_3|=3$, and $\epsilon_3=-\epsilon_1$. Thus $v_1+v_3$ is the sum of two distant intervals and so is reducible. However, $v_3+v_1$ is irreducible by Lemma~\ref{lem:SumIrr}, a contradiction. If $\sigma_3=6$, we see that $v_3=e_0+e_1+e_2-e_3$ (and, in particular, $|v_3|=4$). This implies that $\braket{v_3}{x_0}=2$ and $\braket{v_1}{v_3}=1$. The latter will only be possible if both $\delta([v_1], [v_3])=3$ and $\epsilon_1 = \epsilon_3$, a contradiction to Lemma~\ref{lem:delta}. 

If $\sigma_3=4$, we have $v_3=e_0+e_2-e_3$. Using Lemma~\ref{intervalproduct}, the second statement of the lemma is immediate because $\braket{v_3}{v_1}=\braket{v_3}{x_0}=\braket{v_1}{x_0}=2$ and $|v_3|=3$.
\end{proof}

\begin{lemma}\label{lem:0No2}
If $0\in\supp(v_j)$ and $2\notin\supp(v_j)$ for some $j>3$ and $j\ne k_3$, then $[v_j],[v_1]$ share their right endpoint, and $v_j=e_0+e_{j-1}-e_j$. Moreover, there exists at most one such $j$.
\end{lemma}
\begin{proof}
We have $1\not\in \supp(v_j)$, otherwise $\braket{v_2}{v_j}=2$, a contradiction to Corollary~\ref{unbreakablepairing}. So $\braket{v_1}{v_j}=2$. Since $\braket{v_{j}}{v_2}=1$, $[v_j]$ and $[v_2]$ are consecutive by Corollary~\ref{Cor:ZjDistinct}. It follows from Lemma~\ref{lem:k1>1 intervals} that $[v_{j}]$ and $[v_1]$ share their right endpoint, and so $\delta([v_{j}], [v_1])=1$. Then, to get $\braket{v_1}{v_{j}}=2$, we must have $|v_{j}|=3$ and $v_{j}=e_0+e_{j-1}-e_{j}$. Lastly, there exists at most one such $j$ by Corollary~\ref{Cor:ZjDistinct}.
\end{proof}


\begin{prop}\label{prop6.1}
If $\sigma_3=3$, the initial segment $(\sigma_0, \cdots, \sigma_{k_3})$ of $\sigma$ is $(1,2,3,3,7)$.
\end{prop}

\begin{proof}
Suppose that $\sigma_3=3$ (see Lemma~\ref{lem:k1=2 sigma3}). This implies that $k_2=3$ (Lemma~\ref{Lem:X0Odd}). Using Equation~\eqref{x0k1=2}, we see that $\sigma_{k_3}=7$. We claim that $k_3=k_2+1=4$. If $k_3\neq 4$, by Lemma~\ref{Lem:X0Odd}, $\sigma_4\in \{ 4, 6\}$. Suppose $\sigma_4=4$, or equivalently, $v_4=e_0+e_3-e_4$. This gives us that $\braket{v_4}{x_0}=2$ and $\braket{v_4}{v_2}=1$. By Lemma~\ref{lem:k1>1 intervals}, the interval $[v_1]$ will be a subset of $[v_4]\cup\{x_0\}$, which implies that $|v_1|= 3$, a contradiction. Suppose $\sigma_4=6$, or equivalently, $v_4=e_2+e_3-e_4$. Then there will be a claw $(v_2, v_1, v_4, v_3)$. This justifies the claim, that is, $\sigma_4=7$ and $k_3=4$. 
\end{proof}

\begin{prop}\label{lem:k1>1,2t+1}
If $\sigma_3=4$, the initial segment $(\sigma_0, \cdots, \sigma_{k_3})$ of $\sigma$ is either $(1,2,3,4,5,9)$ or $(1,2,3,4^{[s]}, 4s+3,4s+7)$, $s\ge1$.
\end{prop}

\begin{proof}
All the possibilities for $\sigma_4$ lie in $\{ 4,5,6,7,8,9,10\}$. We first argue that $\sigma_4\not\in\{6, 8, 9, 10\}$. Suppose $\sigma_4=6$, then $v_4=e_1+e_3-e_4$, contradicting Lemma~\ref{lem:NoSingle1}. If $\sigma_4=10$, then $v_4$ will have nonzero inner product with $v_2$ and $v_3$.  Using Lemmas~\ref{lem:k1=2 sigma3}~and~\ref{lem:k1>1 intervals}, the interval $[v_1]$ equals the union of $[v_3]$ and $[v_4]$, that is, $|v_1|=6$, a contradiction. If $\sigma_4=8$, then both the unbreakable vectors $v_3$ and $v_4$ will have nonzero inner product with $x_0$, contradicting Corollary~\ref{x0pairing}. When $\sigma_4=9$, $v_4=e_1+e_2+e_3-e_4$. Notice that $\braket{v_4}{v_1}=-1$ while $v_4$ is orthogonal to $x_0$. The latter gives us that $\delta([v_4], [v_1])\le 2$. Therefore, given that $|v_4|=4$, we must have $[v_4]$ and $[v_2]$ share their left endpoint by Lemma~\ref{lem:k1>1 intervals}, a contradiction to Corollary~\ref{Cor:ZjDistinct}. Therefore $\sigma_4\in \{4, 5, 7\}$.

Suppose that $\sigma_4=5$, that is, $v_4=e_0+e_3-e_4$. Using Lemma~\ref{Lem:X0Odd}, $k_2=4$, and so $\sigma_{k_3}=9$ by Equation~\eqref{x0k1=2}. Since $\braket{v_3}{x_0}=2$, $\braket{v_5}{x_0}=0$ by Corollary~\ref{x0pairing}, unless $k_3=5$. Since $4\in \supp(v_5)$, we get that $k_3=5$.

Let $s\ge 1$ be the integer satisfying that $\sigma_3=\cdots =\sigma_{s+2}=4$, and that $\sigma_{s+3}>4$. By Lemma~\ref{Lem:X0Odd}, $k_2\ge s+3$. Set $j=\text{min }\supp(v_{s+3})<s+2$. If $3<j<s+2$, there will be a claw $(v_j, v_{j-1}, v_{s+3}, v_{j+1})$, and if $j=3$, the claw will be $(v_3, x_0, v_{s+3}, v_4)$. If $j=1$, then $2\in\supp(v_{s+3})$ by Lemma~\ref{lem:NoSingle1}. Thus $|v_{s+3}|\ge 4$. Since $\braket{v_{s+3}}{x_0}=0$, $\delta([v_{s+3}], [v_1])\le 2$. Then 
\[
|\braket{v_{s+3}}{v_1}|\ge4-2\ge 2,
\] 
a contradiction. Lastly, suppose $j=0$. By Corollary~\ref{x0pairing}, $\braket{v_{s+3}}{x_0}=0$, it must be the case that $2\not \in\supp(v_{s+3})$. By Lemma~\ref{lem:0No2}, $v_{s+3}=e_0+e_{s+2}-e_{s+3}$. If $s=1$ and $\sigma_4=5$, this case was discussed in the previous paragraph. However, if $s>1$, then $\braket{v_{s+3}}{v_3}=1$, and so $[v_1]$ will be the union of $[v_3]$ and $[v_{s+3}]$ by Lemma~\ref{lem:k1=2 sigma3} and Lemma~\ref{lem:0No2}. Since $|v_3|=3$, to get $|v_1|=5$, it must be that $|v_{s+3}|=4$, a contradiction. So we are left with the case $j=2$. 

Note that $\braket{v_{s+3}}{v_2}=-1$, and $v_{s+3}$ is orthogonal to $v_1$ and $x_0$, so $[v_{s+3}]$ is distant from $[v_1]$ by Lemma~\ref{lem:k1>1 intervals}. Using Lemma~\ref{lem:k1=2 sigma3}, we get that $v_{s+3}$ is orthogonal to $v_{3}$, and so $3\in \supp(v_{s+3})$. By Lemma~\ref{gappy3}, we get that $4,\cdots, s+1 \in \supp(v_{s+3})$. That is, $\sigma_{s+3}=4s+3$, and that $k_2=s+3$. Using Equation~\eqref{x0k1=2}, we get that $\sigma_{k_3}=4s+7$. With the same argument as in the case $\sigma_4=5$, we get that $k_3=k_2+1=s+4$. This recovers the case $\sigma_4=7$ when $s=1$. 
\end{proof}

\begin{prop}\label{1k2=2}
If $(\sigma_0, \cdots, \sigma_{k_3})=(1,2,3,4,5,9)$, then $n+1=k_3$ (i.e. $v_{k_3}$ is the last standard basis vector).
\end{prop}

\begin{proof}
We claim that the index $6$ does not exist. Suppose for contradiction that it exists. Since $5\in S'_6$, then $S'_6$ must be one of $\{4, 5\}$, $\{2, 5\}$, or $\{0, 5\}$ (Lemma~\ref{Lem:X0Odd} and Corollary~\ref{x0pairing}). 

By Lemma~\ref{lem:0No2}, the intervals $[v_4]$ and $[v_1]$ share their right endpoint, and $S'_6\ne\{ 0,5\}$. 

Suppose that $S'_6=\{ 4, 5\}$ or $\{ 2, 5\}$, then $\braket{v_6}{x_0}=0$. We have that one of $\braket{v_6}{v_4}$ and $\braket{v_6}{v_3}$ is zero and the other one is nonzero, depending on whether or not $3\in S_6$. 
By Lemma~\ref{lem:k1>1 intervals} and Corollary~\ref{Cor:ZjDistinct}, $[v_6]$ and $[v_1]$ are not consecutive. 
Using Lemma~\ref{lem:k1=2 sigma3} and the fact that $[v_4]$ and $[v_1]$ share their right endpoint, we conclude that $[v_6]\subset[v_1]$ and $\delta([v_6], [v_1])\le 2$. Since $|v_6|\ge 3$, we must have $\braket{v_6}{v_1}\not = 0$. That is, $1\in \supp(v_6)$, and so $|v_6|\ge 4$. Using Lemmas~\ref{lem:k1>1 intervals}, \ref{lem:k1=2 sigma3} and Corollary~\ref{Cor:ZjDistinct}, $[v_1]$ will have all the high weight vertices of $[v_3]$, $[v_6]$, and $[v_4]$, and so, $|v_1|\ge 6$, a contradiction. 
 This proves the claim.
\end{proof}


\begin{prop}\label{lem:k1=2,2t+3}
When $(\sigma_0, \cdots, \sigma_{k_3})=(1,2,3,3,7)$, there exists $s\ge0$, such that $v_{s+5}=e_{3}+\cdots +e_{s+4}-e_{s+5}$, $v_5 = e_0 + e_4-e_5$ if $s>0$, and $|v_j|=2$ for $5<j<s+5$ and $j>s+5$. In this case, $\sigma =  (1,2,3,3,7,8^{[s]}, 8s+10^{[t]})$ ($s,t\geq 0$).
\end{prop}	
\begin{proof}
First suppose that $\sigma_5\neq 10$.
Since $k_3=4\in S'_5$, the set $S'_5$ is either $\{0,4\}$, $\{3,4\}$, or $\{0,2,3,4\}$ (Lemmas~\ref{Lem:X0Odd}~and~\ref{gappy3}). 
If $S'_5=\{3,4\}$, as $\sigma_5\neq 10$, we must have $1\in S_5$, a contradiction to Lemma~\ref{lem:NoSingle1}. If $S'_5=\{0,2,3,4\}$, then $\braket{v_1}{v_5}> 0$. Since $|v_5|\ge 5$ and $\delta([v_1], [v_5])\le 3$, we have $\epsilon_1=\epsilon_5$. Since $\braket{v_1}{v_5}\le2$, and that $|v_5|\ge 5$, we must have $\delta([v_5],[v_1])=3$. Since $\epsilon_1=\epsilon_5$, by Lemma~\ref{lem:delta}, $\braket{v_5}{x_0}=-\braket{v_1}{x_0}=\pm2$, which is not true. Therefore, $S'_5=\{ 0, 4\}$ and $v_5=e_0-e_4+e_5$ by Lemma~\ref{lem:0No2}.


%\noindent{\bf Claim 1.} For any $i>5$, if $|v_i|\ge3$, then at least one of $\braket{v_i}{v_1}$ and $\braket{v_i}{v_2}$ is zero.

%Assume for contradiction that both $\braket{v_i}{v_1}$ and $\braket{v_i}{v_2}$ are nonzero. By Corollary~\ref{Cor:ZjDistinct}, $[v_i]$ and $[v_2]$ are consecutive. By Lemma~\ref{lem:k1>1 intervals},  either $[v_1]$ and $[v_i]$ share their right endpoint or they are distant. The latter case cannot happen since $\braket{v_i}{v_1}\ne0$. However, we already proved that
%$[v_5]$ and $[v_1]$ share their right endpoint. So $[v_1]$ and $[v_i]$ cannot share their right endpoint by Corollary~\ref{Cor:ZjDistinct}. This finishes the proof of this claim.

We claim that if $S_j\ne\emptyset$ for some $j>5$, then $S_j=\{3,4\}$. Assume that $S_j\ne\emptyset$.
By Lemmas~\ref{Lem:X0Odd} and~\ref{gappy3}, $S_j'$ is one of $\emptyset$, $\{0,3\}$, $\{ 0,4\}$, $\{2, 3\}$, $\{ 3,4\}$, and $\{ 0,2, 3,4\}$. If $S_j'=\emptyset$, then $S_j=\{1\}$, contradicting Lemma~\ref{lem:NoSingle1}. Since $S'_5=\{ 0, 4\}$, $S_j'\ne\{0,3\}$ or $\{0,4\}$ by Lemma~\ref{lem:0No2}.
If $S'_j=\{ 2,3\}$, then $\braket{v_j}{x_0}=2$. Since $\delta([v_j], [v_1])\le 3, |v_j|\ge 4$, we have $\braket{v_j}{v_1}\ne0$. Since
$0\notin \supp(v_j)$, we must have $1\in \supp(v_j)$, and so $|v_j|\ge 5$. Using Lemma~\ref{intervalproduct}, we get $|\braket{v_j}{v_1}|>1$, contradicting the fact that $\braket{v_1}{v_j}=-1$. If $S'_{j} = \{0,2,3,4\}$, then we have $\braket{v_{j}}{x_0}=2$ and $|v_{j}|\ge 6$. Thus $x_1=x_{z_j}$ is contained in $[v_1]$. However, $|v_1|=5<6=|v_j|$, a contradiction. So $S'_j=\{3,4\}$. Using Lemma~\ref{lem:NoSingle1}, we conclude that $1\notin S_j$. So $S_j=\{3,4\}$. 

If $S_j=\emptyset$ for all $j>5$, it follows from Lemma~\ref{lem:AllNorm2} that $|v_j|=2$ whenever $j>5$. Now assume that $S_j\ne\emptyset$ for some $j>5$.
Let $s+5$ be the smallest such $j$. Then $S_{s+5}=\{3,4\}$ by the earlier discussion. We also know that $|v_i|=2$ for any $5<i<s+5$ by Lemma~\ref{lem:AllNorm2}. If $5\not \in \supp(v_{s+5})$, then $\braket{v_{s+5}}{v_5}\neq 0$ and $\braket{v_{s+5}}{v_3}\neq 0$, and so there will be a cycle $(v_{s+5}, v_3, v_2, v_5)$ of length bigger than $3$: see Figure~\ref{fig:case1 123 G(S)}. Thus $5 \in \supp(v_{s+5})$, and as a result $6, \cdots, s+4\in \supp(v_{s+5})$ by Lemma~\ref{gappy3}. Therefore, $\sigma_{s+5}=8s+10$.

Note that, $S_j=\emptyset$ when $j>s+5$. Otherwise, by the earlier discussion, $S_j=\{3,4\}$, and
we would have a heavy triple $(v_{j}, v_{s+5}, v_2)$. 
Given $j>s+5$, let $\ell=\min\supp(v_j)\ge5$. Note that
\[
v_5\sim v_6\sim \cdots \sim v_{s+4},
\]
$[v_5]$ and $[v_1]$ share their right endpoint, $\braket{v_i}{v_1}=0$ and $|v_i|=2$ when $5<i<s+5$, so $[v_{i}]\subset [v_1]$ when $5\le i<s+5$. If $\ell\le s+4$, then $\braket{v_j}{v_{\ell}}\ne0$.
Thus $[v_{j}]\cap [v_1]\neq \emptyset$. Note also that $\delta([v_{j}], [v_1])\le 2$ since $v_{j}$ is orthogonal to $x_0$. Since $|v_{j}|\ge 3$, we get that $|\braket{v_{j}}{v_1}|>0$, a contradiction. Thus we have proved that $\min\supp(v_j)\ge s+5$ when $j>s+5$. It follows from Lemma~\ref{lem:AllNorm2} that $|v_j|=2$ when $j>s+5$.

%Suppose that there exists $s>0$ such that $|v_i|=2$ for any $5<i<s+5$, and that $|v_{s+5}|>2$. Note that $\braket{v_i}{v_2}=0$ for all $5<i<s+5$ and, 
%\[
%v_5\sim v_6\sim \cdots \sim v_{s+4},
%\]
%and so $[v_i]\prec [v_1]$. Note also that $S'_{s+5}=\{ 0,4\}$, or $\{ 3,4\}$, or $\{ 0,2, 3,4\}$. If $S'_{s+5}=\{ 0,4\}$, then $\braket{v_{s+5}}{v_1}\neq 0$ and $\braket{v_{s+5}}{v_2}\neq 0$. Therefore, $[v_{s+5}]$ shares its right endpoint with  that of $[v_1]$. That is, $[v_{s+5}]$ and $[v_5]$ share their right endpoint, a contradiction. If $S'_{s+5} = \{0,2,3,4\}$, then we have $\braket{v_{s+5}}{x_0}=2$ and $|v_{s+5}|\ge 6$. Since $\delta([v_{s+5}], [v_1])\le 3$, then $|\braket{v_{s+5}}{v_1}|>2$, a contradiction. That is, $S'_{s+5}=\{ 3,4 \}$. The same argument used to rule out the possibility $S'_{s+5}=\{ 0,4 \}$ shows that $1\not \in S'_{s+5}$. If $5\not \in \supp(v_{s+5})$, then $\braket{v_{s+5}}{v_5}\neq 0$ and $\braket{v_{s+5}}{v_3}\neq 0$, and so there will be a cycle $(v_{s+5}, v_3, v_2, v_5)$ of length bigger than $3$: see Figure~\ref{fig:case1 123 G(S)}. Thus $5 \in \supp(v_{s+5})$, and as a result $6, \cdots, s+3\in \supp(v_{s+5})$ by Lemma~\ref{gappy3}. Therefore, $\sigma_{s+5}=8s+10$. Now suppose $t\ge 0$ be such that $|v_i|=2$ for any $s+5<i<s+t+6$, and suppose also that the index $s+t+6$ exists. We claim that $|v_{s+t+6}|=2$. The same argument used for the index $s+5$, shows that $S'_{s+t+6}\neq \{0,4\}$ or $\{ 0,2,3,4\}$, and that $1\not \in \supp(v_{s+t+6})$. Using Lemma~\ref{Lem:X0Odd}, the only possibilities for $S'_{s+t+6}$ consist of $\emptyset$ and $\{3,4\}$. If $S'_{s+t+6}=\{3,4\}$, then $\braket{v_{s+5}}{v_{s+t+6}}\neq 0$ and $\braket{v_{s+t+6}}{v_3}\neq 0$, and so there will be a heavy triple $(v_{s+t+5}, v_{s+5}, v_2)$, a contradiction. We have shown that $S'_{s+t+6}=S_{s+t+6}=\emptyset$. Set $j=\text{min }\supp(v_{s+t+6})$. We have proved $j>4$. We also observe that $j\ge s+3$, as otherwise $\braket{v_{s+t+6}}{v_{s+5}}>1$. If $j=s+3$, then $s+4\in \supp(v_{s+t+6})$ by Lemma~\ref{gappy3}. Also, to avoid $\braket{v_{s+t+6}}{v_{s+5}}=2$, we get $s+5\in \supp(v_{s+t+6})$. So $s+6, \cdots, s+t+4\in \supp(v_{s+t+6})$. We have $\braket{v_{s+t+6}}{v_{s+5}}\neq 0$ and $\braket{v_{s+3}}{v_{s+t+6}}\neq 0$, and there will be a cycle $(v_{s+5}, v_3, v_2, v_5, \cdots, v_{s+3}, v_{s+t+6})$ of length bigger than $3$: see Figure~\ref{fig:case1 123 G(S)}. Suppose $j=s+4$. Then $\braket{v_{s+t+6}}{v_{s+4}}\neq 0$. Note that, by our earlier discussion, we have $[v_{s+t+6}]\cap [v_1]\neq \emptyset$. Note also that $\delta([v_{s+t+6}], [v_1])\le 2$ since $v_{s+t+6}$ is orthogonal to $x_0$. Since $|v_{s+t+6}|\ge 3$, we get that $|\braket{v_{s+t+6}}{v_1}|>0$, a contradiction. There will be a claw if $s+4<j<s+t+5$: again, see Figure~\ref{fig:case1 123 G(S)}. So $j=s+t+5$, and by induction $|v_i|=2$ for any $i>s+5$. 

Finally suppose that $\sigma_5=10$. Assume that there exists $\ell>5$ such that %for any $5<i<\ell$, $\sigma_i=10$. 
$S_{\ell}\ne\emptyset$. By Lemmas~\ref{Lem:X0Odd}~and~\ref{gappy3}, $S'_\ell$ is one of $\emptyset$, $\{0,3\}$, $\{0,4\}$, $\{2, 3\}$, $\{ 3, 4\}$, and $\{0,2,3,4\}$. 
If $S'_\ell=\emptyset$, then $S_{\ell}=\{1\}$, contradicting Lemma~\ref{lem:NoSingle1}. By Lemma~\ref{lem:0No2}, $S'_\ell\ne\{0,3\}$ or $\{0,4\}$. Suppose $S'_\ell = \{2,3\}$. If $1\not \in \supp(v_\ell)$, there will be a claw $(v_2, v_1, v_\ell, v_3)$. If $1\in \supp(v_\ell)$, then $|\braket{v_\ell}{v_1}|=1$. By Lemma~\ref{lem:k1>1 intervals}, $[v_{\ell}]$ and $[v_1]$ are not consecutive. Since $\delta([v_\ell], [v_1])\le 3$ and $|v_\ell|\ge 5$, we get $|\braket{v_\ell}{v_1}|\ge2$, a contradiction. 
If $S'_\ell=\{3,4\}$, there will be a heavy triple $(v_5,v_\ell, v_2)$. If $S'_\ell=\{0,2,3,4\}$, then $|v_\ell|\ge 6$ and $\braket{v_{\ell}}{x_0}=2$, so $x_{z_{\ell}}=x_1$. Thus $|[v_1]|\ge |[v_\ell]|\ge6$, a contradiction. So we proved that $S_\ell=\emptyset$ whenever $\ell>5$. It follows from Lemma~\ref{lem:AllNorm2} that $|v_j|=2$ when $j>5$.
\end{proof}

\begin{figure}
	\begin{align*}
	\xymatrix@R1.5em@C1.5em{
		v_{3}^{(3)}\ar@{=}[r]^{+}\ar@{=}[dr]^{+} & x_0^{(4)} \ar@{=}[d]^{+} \\
		v_4^{(3)} \ar@{=}[r]^{+} \ar@{-}[dr]^{+} & v_1^{(5)} \ar@{-}[d]^{+} \\
		& v_{2}^{(3)} \\
	}
	&&
	\xymatrix@R1.5em@C1.5em{
		v_3^{(3)} \ar@{-}[d] \ar@{=}[r]^{+} \ar@{=}[dr]^{+} & x_0^{(4)} \ar@{=}[d]^{+} \\
		v_{4}^{(2)}\ar@{-}[d] & v_1^{(5)} \ar@{-}[d]^{+} \\
		\vdots & v_{2}^{(3)} \ar@{-}[d] \\
		v_{s+2}^{(2)}\ar@{-}[u] & v_{s+3}^{(s+2)}  
%\\
%		& %v_{s+5}^{(3)}
		%& v_{s+6}^{(2)} \ar@{-}[d] \\
		%& \vdots \\
		%& v_{s+t+4}^{(2)} \ar@{-}[u]
	}
	&&
	\xymatrix@R1.5em@C1.5em{
		& x_0^{(4)} \ar@{=}[d]^{+} \\
		%v_5^{(3)} \ar@{=}[r]^{+} \ar@{-}[dr]^{+} 
                     & v_1^{(5)} \ar@{-}[d]^{+} \\
		 & v_{2}^{(3)} \ar@{-}[d] \\
		& v_{3}^{(2)} \\
		%v_{s+4}^{(2)}\ar@{-}[u] & v_{s+5}^{(s+3)} 
		%& v_{s+6}^{(2)} \ar@{-}[d] 
		%& \vdots \\
		%& v_{s+t+4}^{(2)}\ar@{-}[u]
	}
	\end{align*}
	\caption{Pairing graphs when $(\sigma_0, \cdots, \sigma_{k_3})$ is $(1,2,3,4,5,9)$ (left), $(1,2,3,4^{[s]}, 4s+3, 4s+7)$ (center), or $(1,2,3,3,7)$ (right).}
	\label{fig:case1 123 G(S)}
\end{figure}

\begin{prop}\label{lem:k1>1,2t+3}
		If $(\sigma_0, \cdots \sigma_{k_3})=(1,2,3,4^{[s]}, 4s+3,4s+7)$, $s>0$, then $v_{s+5}=e_{s+3}+e_{s+4}-e_{s+5}$, and $\abs{v_j} = 2$ for $j>s+5$. In this case, $\sigma =  (1,2,3,4^{[s]}, 4s+3,4s+7, (8s+10)^{[t]})$ ($s>0,t \geq 0$). 
\end{prop}	

\begin{proof}
Suppose that $\ell>s+4$ is an index such that $S_{\ell}\ne\emptyset$. We will prove that $\ell=s+5$ and $v_{s+5}=e_{s+3}+e_{s+4}-e_{s+5}$. Our conclusion then follows from Lemma~\ref{lem:AllNorm2}.

\noindent{\bf Step 1}. $S'_{\ell}$ must be either $\emptyset$ or $\{ s+3, s+4\}$.

Using Lemma~\ref{gappy3} and Corollary~\ref{x0pairing}, $S'_\ell$ is either $\emptyset$, $\{ 0, s+4\}$, $\{ 2, s+4\}$, or $\{ s+3, s+4\}$. Suppose $S'_{\ell}=\{ 0, s+4\}$, by Lemma~\ref{lem:0No2}, $v_{\ell}=e_0+e_{s+4}-e_{s+5}$, $[v_{\ell}]$ and $[v_1]$ share their right endpoint. As $\braket{v_{\ell}}{v_3}\neq 0$, $[v_1]$ equals the union of $[v_3]$ and $[v_{\ell}]$ by Lemma~\ref{lem:k1=2 sigma3}, i.e. $|v_1|=4$, a contradiction. Suppose $S'_{\ell}=\{ 2, s+4\}$. If $1\not \in S_{\ell}$, as $s+3\not\in S_{\ell}$, $\braket{v_{ell}}{v_{s+3}}\ne0$, there will be a heavy triple $(v_{\ell}, v_{s+3}, v_2)$. If $1\in S_{\ell}$ (and consequently, $|v_{\ell}|\ge 4$), then there will be a claw $(v_1, x_0, v_{\ell}, v_2)$, unless $[v_{\ell}]\pitchfork [v_1]$. If $[v_{\ell}]\pitchfork [v_1]$, however, we get $\delta([v_{\ell}],[v_1])=2$, and so $|\braket{v_{\ell}}{v_1}|\ge 2$, a contradiction to the fact that $\braket{v_{\ell}}{v_1}=-1$.

\noindent{\bf Step 2}. If $S'_{\ell}=\emptyset$ or $\{ s+3, s+4\}$, then $S_{\ell}=S'_{\ell}$. In particular, $S_{s+5}=\{ s+3, s+4\}$.

Suppose that $S'_{\ell}=\emptyset$ or $\{ s+3, s+4\}$. Let $i=\min\supp(v_{\ell})$.  By Lemma~\ref{lem:NoSingle1}, $i\ne1$. That is, $\braket{v_\ell}{v_1}=0$. Also, note that $v_\ell$ is orthogonal to $x_0$, and so $\delta([v_\ell], [v_1])\le 2$. If $3\le i\le s+2$, since $v_3\sim v_4\sim  \cdots\sim v_{i}\sim v_{\ell}$, using Lemmas~\ref{lem:k1=2 sigma3}~and~\ref{lem:k1>1 intervals}, $x_{z_{\ell}}\in [v_1]$. Therefore, $\braket{v_1}{v_\ell}\not = 0$, a contradiction. So $i\ge s+3$ and hence $S_{\ell}=S'_{\ell}$. Clearly, $S_{s+5}=\{ s+3, s+4\}$ by Lemma~\ref{lem:j-1}.
%Now $[v_1]$ contains at least three high weight vertices $x_{z_3},x_{z_{\ell}},x_{z_2}$. Since $|v_1|=5$, we must have $|v_{\ell}|=|x_{z_{\ell}}|=3$. So $v_{\ell}=e_i+e_{\ell-1}-e_{\ell}$. Thus we have a contradiction when $S'_{\ell}=\{ s+3, s+4\}$, and our conclusion holds in this case. In particular, since $s+4\in S_{s+5}$, $S_{s+5}=\{ s+3, s+4\}$,
%by Step~1 and the $S'_{\ell}=\{ s+3, s+4\}$ case of Step~2.

%When $S'_{\ell}=\emptyset$, using Lemma~\ref{gappy3}, $i$ must be $s+2$. Then $\braket{v_\ell}{v_{s+3}}\neq 0$. Since %$v_{s+5}=e_{s+3}+e_{s+4}-%e_{s+5}$, $\braket{v_{s+5}}{v_{s+3}}\neq 0$. We also have $\braket{v_{2}}{v_{s+3}}\neq 0$. By %Corollary~\ref{Cor:ZjDistinct}, the $3$ intervals $[v_\ell]$, $[v_{s+5}]$ and $[v_2]$ are all consecutive to $[v_{s+3}]$, which is impossible by %Corollary~\ref{Cor:ZjDistinct}.

\noindent{\bf Step 3}. If $S_{\ell}=\{ s+3, s+4\}$, then $\ell=s+5$. 

Assume that $\ell>s+5$ and $S_{\ell}=\{ s+3, s+4\}$, then we have a heavy triple $(v_{s+3},v_{s+5},v_{\ell})$.
\end{proof}

%Since $v_3\sim v_4\sim  \cdots\sim v_{s+2}$ and $\braket{v_2}{v_i}=0$ for all $3\le i \le s+2$, using Lemmas~\ref{lem:k1=2 sigma3}~and~\ref{lem:k1>1 intervals}, $[v_i]\prec [v_1]$. Since $s+4\in \supp(v_{s+5})$, then $S'_{s+5}$ is one of $\{ 0, s+4\}$, $\{ 2, s+4\}$, or $\{ s+3, s+4\}$ (Lemma~\ref{Lem:X0Odd} and Corollary~\ref{x0pairing}). Suppose $S'_{s+5}=\{ 0, s+4\}$. Then we must have $1\not \in S_{s+5}$ (Corollary~\ref{unbreakablepairing}). So $\braket{v_{s+5}}{v_1}=2$ and $\braket{v_{s+5}}{v_2}\neq 0$, and therefore, $[v_{s+5}]$ and $[v_1]$ share their right endpoint by Lemma~\ref{lem:k1>1 intervals}. That is, $\delta([v_1], [v_{s+5}])=1$. To get $\braket{v_{s+5}}{v_1}=2$, it must be that $|v_{s+5}|=3$, and so $3\not \in \supp(v_{s+5})$. But the latter implies that $\braket{v_{s+5}}{v_3}\neq 0$, and so $[v_1]$ equals the union of $[v_3]$ and $[v_{s+5}]$ by Lemma~\ref{lem:k1=2 sigma3}, i.e. $|v_1|=4$, a contradiction. Suppose $S'_{s+5}=\{ 2, s+4\}$. If $1\not \in S_{s+5}$, there will be a heavy triple $(v_{s+5}, v_{s+3}, v_2)$: see Figure~\ref{fig:case1 123 G(S)}. If $1\in S_{s+5}$ (and consequently, $|v_{s+5}|\ge 4$), then there will be a claw $(v_1, x_0, v_{s+5}, v_2)$, unless $[v_{s+5}]\pitchfork [v_1]$. If $[v_{s+5}]\pitchfork [v_1]$, however, we get $\delta([v_{s+5}],[v_1])=2$, and so $|\braket{v_{s+5}}{v_1}|\ge 2$, a contradiction (since it must be $\braket{v_{s+5}}{v_1}=-1$). Having shown $S'_{s+5}=\{s+3, s+4\}$, we now argue that $S_{s+5}=S'_{s+5}$. If $1 \in S_{s+5}$, then $\braket{v_{s+5}}{v_1}=-1$ and $\braket{v_{s+5}}{v_2}\neq 0$. So, using Lemma~\ref{lem:k1>1 intervals}, we see that $[v_{s+5}]$ shares its right endpoint with that of $[v_1]$, and so $\delta([v_{s+5}], [v_1])=1$. Since $|v_{s+5}|\ge 4$, we get that $|\braket{v_{s+5}}{v_1}|\ge 3$, a contradiction. If $v_{s+5}$ has nonzero inner product with any of the $v_i$ for $i\in \{ 3, \cdots, s+2\}$, then, by our earlier discussion, $[v_{s+5}]\cap[v_1]\not =\emptyset$. Since $v_{s+5}$ is orthogonal to $x_0$, we must have $\delta([v_{s+5}], [v_1])\le 2$. Since $|v_{s+5}|\ge 3$ in this case, however, we get that $\braket{v_{s+5}}{v_1}\neq 0$, a contradiction. That is, $S_{s+5}=S'_{s+5}$, and therefore, $\sigma_{s+5}=8s+10$. Now suppose that there exists $\ell > s+5$ such that $|v_i|=2$ for any $s+5<i<\ell$. We claim that $\text{min }\supp(v_\ell)=\ell-1$. Using Corollary~\ref{x0pairing}, $S'_\ell$ is either $\emptyset$, $\{ 0, s+4\}$, $\{ 2, s+4\}$, or $\{ s+3, s+4\}$. If $S'_\ell=\{ 0, s+4\}$ or $\{ 2, s+4\}$, the same argument used for the index $s+5$ will give us a contradiction in each case. If $S'_\ell=\{s+3, s+4\}$, then $\braket{v_{\ell}}{v_{s+5}}\neq 0$. Either $v_{\ell}\sim v_{s+3}$ and there will be a heavy triple $(v_{\ell}, v_{s+3}, v_{s+5})$, or $v_{\ell}\not \sim v_{s+3}$, and so $s+2\in \supp(v_{\ell})$ and $3, \cdots, s+1\not \in \supp(v_\ell)$. So $\braket{v_{\ell}}{v_{s+2}}\neq 0$, and therefore, $[v_{\ell}]\cap[v_1]\neq \emptyset$ since $[v_{s+2}]\prec[v_1]$. Since $v_\ell$ is orthogonal to $x_0$, we have $\delta([v_\ell], [v_1])\le 2$. We have $|v_\ell|\ge 4$, so $|\braket{v_\ell}{v_1}|\ge 2$, a contradiction. That is, $S'_{\ell}=\emptyset$. Set $j=\text{min }\supp(v_\ell)$. If $s+5<j<\ell-1$, there will be a claw $(v_j, v_{j-1}, v_\ell, v_{j+1})$, and if $j=s+5$, the claw will be on $v_{s+5}, v_{s+3}, v_j, v_{s+6}$. If $j=s+2$, then $\braket{v_\ell}{v_{s+3}}\neq 0$. Also, for some $s+5\le i \le \ell-1$, $\braket{v_\ell}{v_i}\neq 0$: there will be a heavy triple $(v_\ell, v_{s+5}, v_{s+3})$. If $3\le j<s+2$, using Lemma~\ref{gappy3}, then $\braket{v_\ell}{v_{s+3}}>1$. If $j=1$, then $\braket{v_\ell}{v_1}\neq 0$ and $\braket{v_\ell}{v_2}\neq 0$, and therefore, $[v_\ell]$ and $[v_1]$ share their right endpoint, and so $\delta([v_{\ell}], [v_1])=1$. Again, since $|v_\ell|\ge 3$, this implies that $\braket{v_\ell}{v_1}>1$, a contradiction. This justifies the claim that $j=\ell-1$. By induction we see that $|v_i|=2$ for all $i>s+5$.

\subsection{{\boldmath $k_1=3$}}\label{k1=3}

In this subsection we focus on the changemaker C-type lattices with 
\begin{equation}\label{x0k1=3}
x_0=e_0+e_3+e_{k_2} - e_{k_3}.
\end{equation}  
Recall that the changemaker starts with $(1,2,2,3)$.

\begin{lemma}\label{lem:k1=3 intervals}
		The intervals $[v_3]$ and $[v_1]$ share their right endpoint and $\epsilon_3 = \epsilon_1$. Moreover, $[v_2]$ abuts the right endpoint of $[v_1]$ and $[v_3]$.
\end{lemma}
\begin{proof}
Since $|v_3|=3$ and $\braket{v_1}{v_3}=2$, from Lemma~\ref{intervalproduct}, it must be the case that $\epsilon_1=\epsilon_3$ and $\delta([v_1], [v_3])=1$. The first statement of the lemma is now immediate because $v_3$ is orthogonal to $x_0$. Since $\braket{v_2}{v_1}\ne0$ and $\braket{v_2}{x_0}=0$, $[v_2]$ abuts the right endpoint of $[v_1]$.
\end{proof}


\begin{cor}\label{cor:k1=3 intervals}
		Suppose that there exists a vector $v_j$ such that $j>3$, $j\neq k_3$, and $\braket{v_j}{v_1} = 2$. Then $j=4$, and that $v_4 = e_0 + e_3-e_4$. 
\end{cor}

\begin{proof}
Suppose that $j$ is such an index. Therefore, $0\in \supp^+(v_j)$ and $1\not \in \supp^+(v_j)$. (This, in particular, implies that $|v_j|\ge3$). We claim that $\braket{v_j}{x_0}\neq 0$. Otherwise, assume $\braket{v_j}{x_0}= 0$. Since $\braket{v_j}{v_1}=2$, $x_{z_j}\in[v_1]$. Using Lemma~\ref{lem:k1=3 intervals} and Corollary~\ref{Cor:ZjDistinct}, $[v_1]$ contains at least $3$ high weight vertices $x_1,x_{z_j},x_{z_3}$, and $\delta([v_j],[v_1])=2$. Since $|v_1|=5$, we have $|x_{z_j}|=3$, so by Lemma~\ref{intervalproduct} we have $|\braket{v_j}{v_1}|=1$, a contradiction.
This justifies the claim, and therefore, $\braket{v_j}{x_0}=2$. Since $|v_j|\ge 3$ and $\delta([v_1], [v_j])\le 3$, to get $\braket{v_j}{v_1}=2$, we must have $\epsilon_1=\epsilon_j$. Thus, $\delta([v_j], [v_1])=1$ and $|v_j|=3$. That is, $v_j=e_0+e_{j-1}-e_{j}$. We now argue that $j=4$. Suppose for contradiction that $j>4$. Thus $\braket{v_j}{v_3}=1$. Using Lemma~\ref{lem:k1=3 intervals}, we get that the interval $[v_1]$ equals the union of $[v_j]$ and $[v_3]$. Since $|v_j|=|v_3|=3$, we get that $|v_1|=4$, which is a contradiction. 
\end{proof}

\begin{lemma}\label{lem:v1Orth}
Let $v_j$ be a vector such that $j>3$, $j\neq k_3$. Then $\braket{v_j}{v_1}\in\{0,2\}$. As a result, $\min\supp(v_j)\ge2$ unless $j=4$ and $v_4 = e_0 + e_3-e_4$.
\end{lemma}
\begin{proof}
Assume that $\braket{v_j}{v_1}\notin\{0,2\}$, then $\supp(v_j)\cap\{0,1\}=\{1\}$ or $\{0,1\}$. By Lemma~\ref{gappy3}, $2\in \supp(v_j)$. 
If $0\in\supp(v_j)$, since $|\braket{v_j}{v_3}|\le1$ by Corollary~\ref{unbreakablepairing}, we have $3\in\supp(v_j)$. Thus $|v_j|\ge5$. Since $\braket{x_0}{v_j}=2$,  $x_{z_j}=x_1$. By Corollary~\ref{Cor:ZjDistinct} and Lemma~\ref{lem:k1=3 intervals}, $x_{z_j}\ne x_{z_3}$. So \[5=|v_1|\ge|x_{z_j}|+|x_{z_3}|-2\ge5+1,\]
a contradiction.

We have shown that $0\notin\supp(v_j)$. If $3\notin\supp(v_j)$, then $j>4$ and $|v_j|\ge4$. As $\braket{v_j}{v_3}=1$, using Corollary~\ref{Cor:ZjDistinct}, $[v_j]$ and $[v_3]$ are consecutive. By Lemma~\ref{lem:k1=3 intervals} and the fact that  $\braket{v_j}{v_2}=0$ we conclude that $[v_j]\subset[v_1]$. Since $\braket{v_j}{x_0}=0$, $[v_1]$ contains at least three high weight vertices: $x_1, x_{z_j}, x_{z_3}$. This is impossible as $|v_1|=5$ and $|v_j|\ge4$.

Now we have $\supp(v_j)\cap\{0,1,2,3\}=\{1,2,3\}$, so $\braket{v_j}{v_3}=0$. By  Lemma~\ref{lem:j-1}, $|v_j|\ge5$ unless $j=4$. By Lemma~\ref{lem:k1=3 intervals} and the fact that  $\braket{v_j}{v_1}\ne0$ we conclude that $[v_j]\subset[v_1]$. So $[v_1]$ contains at least two high weight vertices: $x_{z_j}, x_{z_3}$. It follows that $|v_j|\le4$. So $j=4$ and $|v_4| = e_1+e_2 + e_3-e_4$. Since $|v_4|=4$, $[v_1]$ contains exactly two high weight vertices, so $x_1$ must be $x_{z_4}$. So $\braket{v_4}{x_0}\ne0$, which is not possible. This shows that $\braket{v_j}{v_1}\in\{0,2\}$.

If $\min\supp(v_j)<2$, then $\braket{v_j}{v_1}\ne0$. We must have $\braket{v_j}{v_1}=2$, so $j=4$ and $v_4 = e_0 + e_3-e_4$ by Corollary~\ref{cor:k1=3 intervals}.
\end{proof}

\begin{lemma}\label{lem:NoSingle}
Let $v_j$ be a vector such that $j>4$, $j\neq k_3$. Then $\supp(v_j)\cap\{0,1,2,3\}\ne\{2\}$ or $\{3\}$.
\end{lemma}
\begin{proof}
Assume that $\supp(v_j)\cap\{0,1,2,3\}$ contains only one element which is $2$ or $3$.
Then $|v_j|\ge3$, $\braket{v_{j}}{v_3}\ne0$ while $\braket{v_{j}}{v_1}=0$. By Lemma~\ref{lem:k1=3 intervals}, $[v_{j}]$ abuts the left endpoint of $[v_3]$, so $[v_j]\subset[v_1]$. Since $|v_j|\ge3$ and $\delta([v_j],[v_1])\le3$, using Lemma~\ref{intervalproduct}, we get that $\braket{v_{j}}{v_1}\ne 0$ unless $|v_j|=\delta([v_j],[v_1])=3$. However, if $\delta([v_j],[v_1])=3$, $[v_1]$ is contained in the union of $[v_j],[v_3]$ and $\{x_0\}$. Since $|v_j|=|v_3|=3$, we have $|v_1|=4$, a contradiction.
\end{proof}

\begin{lemma}\label{lem:k1=3 sigma3}
		$\sigma_4 \in \{3,4,5\}$. Furthermore, if $\sigma_4 = 3$ then $[v_4]$ abuts the left endpoint of $[v_3]$. If $\sigma_4 = 4$ then $[v_4]$ and $[v_1]$ share their left endpoint.% If $\sigma_4 = 5$ then $[v_1]\dagger[v_2]\dagger[v_4]$. 
\end{lemma}

\begin{proof}
If  $\min\supp(v_4)<2$,  using Lemma~\ref{lem:v1Orth},  $\sigma_4=4$. By Lemma~\ref{lem:j-1}, if $\min\supp(v_4)\ge2$, $v_4=e_2+e_3-e_4$ or $e_3-e_4$. So $\sigma_4=5$ or $3$.

When $\sigma_4=3$, $[v_4]$ abuts $[v_3]$ and $\braket{v_4}{v_1}=0$. By Lemma~\ref{lem:k1=3 intervals}, $[v_4]$ abuts the left endpoint of $[v_3]$.
When $\sigma_4 = 4$, $\braket{v_4}{v_1}=2=\braket{v_4}{x_0}$. So $\delta([v_4],[v_1])=1$ by Lemma~\ref{intervalproduct}. Thus  $[v_4]$ and $[v_1]$ share their left endpoint by Lemma~\ref{lem:k1=3 intervals}.
\end{proof}



\begin{prop}\label{prop1:k1=3}
If $\sigma_4=3$, the initial segment $(\sigma_0, \cdots, \sigma_{k_3})$ of $\sigma$ is $(1,2,2,3,3,7)$. 
\end{prop}
\begin{proof}
Suppose that $\sigma_4=3$ (see Lemma~\ref{lem:k1=3 sigma3}). This implies that $k_2=4$ (Lemma~\ref{Lem:X0Odd}). Using Equation~\eqref{x0k1=3}, we get that $\sigma_{k_3}=7$. If $k_3\not = 5$, using Lemma~\ref{Lem:X0Odd} and  Lemma~\ref{lem:v1Orth}, we must have $S_5\supset\{3,4\}$. By Lemma~\ref{lem:NoSingle}, we have $2\in S_5$, so $\braket{v_5}{x_0}=2$ and $v_5\sim v_2$. By Lemma~\ref{lem:k1=3 intervals}, $[v_1]$ is contained in the union of $x_0,[v_5],[v_2]$. So $|v_1|=|v_5|=4$, which is not possible.
\end{proof}

\begin{prop}\label{prop2:k1=3}
If $\sigma_4\neq 3$, the initial segment $(\sigma_0, \cdots, \sigma_{k_3})$ of $\sigma$ is $(1,2,2,3,4^{[s]},4s+5, 4s+9)$, $s\ge0$. 
\end{prop}

\begin{proof}
Suppose that $\sigma_4\neq 3$ (see Lemma~\ref{lem:k1=3 sigma3}). Furthermore, let $s\ge 0$ satisfy that $\sigma_i=4$ for any $4\le i < s+4$, and that $\sigma_{s+4}>4$. We have $k_2\ge s+4$ by Lemma~\ref{Lem:X0Odd}.
 Set $j=\min \supp(v_{s+4})<s+3$. Then $j\ge2$ by Lemma~\ref{lem:v1Orth}.  Also, $j \ne 3$ by Lemma~\ref{lem:NoSingle}.
If $4<j<s+3$, we will get a claw $(v_j, v_{j-1}, v_{s+4}, v_{j+1})$, and if $j=4$, the claw will be on $v_4, x_0, v_{s+4}, v_5$. This proves that $j=2$. By Lemma~\ref{lem:NoSingle}, $3\in \supp(v_{s+4})$. 

We will show that $\sigma_{s+4}=4s+5$. If $s=0$, $v_4=e_2+e_3-e_4$, and we are done. If $s>0$, since $2,3\in\supp(v_{s+4})$,
$|v_{s+4}|\ge 4$. Also, $v_{s+4}$ must be orthogonal to $v_4$, as otherwise, using Lemmas~\ref{lem:k1=3 sigma3}~and~\ref{lem:k1=3 intervals}, all the three intervals $[v_4], [v_{s+4}]$, and $[v_3]$ will be subsets of $[v_1]$, which implies that $|v_1|\ge 6$, a contradiction. That is, $4\in \supp(v_{s+4})$. Using Lemma~\ref{gappy3}, $v_{s+4}$ is just right and $\sigma_{s+4}=4s+5$. 

Using Lemma~\ref{Lem:X0Odd}, we see that $k_2=s+4$. By Equation~\eqref{x0k1=3}, we have $\sigma_{k_3}=4s+9$. Note that $k_2\in \supp(v_{k_2+1})$. Since the unbreakable vector $v_4$ has nonzero inner product with $x_0$, using Corollary~\ref{x0pairing}, we get that $k_3=k_2+1$.  
\end{proof}

\begin{prop}\label{1k1=3}
If $(\sigma_0, \cdots, \sigma_{k_3})=(1,2,2,3,3,7)$, then $n+1=k_3$ (i.e. $v_{k_3}$ is the last standard basis vector).
\end{prop}
\begin{proof}
We claim that the index $k_3+1$ (that is, $6$) does not exist. Using Lemmas~\ref{Lem:X0Odd}, \ref{lem:j-1}, \ref{gappy3}, and~\ref{lem:v1Orth}, $S'_6=\{4, 5\}$. Then $\braket{v_6}{v_4}\not = 0$, and also $v_6$ is orthogonal to $x_0$. Using Lemmas~\ref{lem:k1=3 sigma3}~and~\ref{lem:k1=3 intervals}, we must have $[v_6]\subset [v_1]$ which implies that $\braket{v_6}{v_1}\ne 0$ since $|v_6|\ge 3$. This contradicts Lemma~\ref{lem:v1Orth}.
\end{proof}

\begin{figure}
	\begin{align*}
	\xymatrix@R1.5em@C1.5em{
		& x_0^{(4)} \ar@{=}[d]^{+} \\
		v_2^{(2)} \ar@{-}[r] \ar@{-}[dr] & v_1^{(5)} \ar@{=}[d]^{+} \\
		& v_{3}^{(3)}\ar@{-}[d]\\
                & v_4^{(2)}\\
	}
	&&
	\xymatrix@R1.5em@C1.5em{
		& x_0^{(4)} \ar@{=}[d]^{+} \ar@{=}[dl]^{+} & \\
		v_4^{(3)}\ar@{=}[r]^{+}\ar@{-}[d] & v_1^{(5)} \ar@{=}[r]^{+} \ar@{-}[d] & v_3^{(3)} \ar@{-}[dl] \\
		v_5^{(2)}\ar@{-}[d] & v_{2}^{(2)} \ar@{-}[d] &  \\
		\vdots & v_{s+4}^{(s+3)} & \\
		v_{s+3}^{(2)}\ar@{-}[u] 
	}
	\end{align*}
	\caption{Pairing graphs when $(\sigma_0, \cdots, \sigma_{k_3})$ is $(1,2,2,3,3,7)$ (left) or $(1,2,2,3,4^{[s]},4s+5, 4s+9)$, $s>0$ (right).}
	\label{fig:case1 1223 G(S)}
\end{figure}

\begin{prop}\label{2k1=3}
		If $(\sigma_0, \cdots, \sigma_{k_3})=(1,2,2,3,4^{[s]},4s+5, 4s+9)$, $s\ge0$, then $v_{s+6}=e_{s+4}+e_{s+5}-e_{s+6}$ if it exists, and $|v_i|=2$ for $i>s+6$. In this case, $\sigma = (1,2,2,3,4^{[s]},4s+5, 4s+9, (8s+14)^{[t]})$, $t\ge0$.
\end{prop}
	
\begin{proof}
Suppose that $\ell>k_3=s+5$ is an index such that $S_{\ell}\ne\emptyset$. We will prove that $\ell=s+6$ and $S_{\ell}=\{s+4,s+5\}$. This, together with Lemma~\ref{lem:AllNorm2}, will imply our desired result.

By Lemmas~\ref{Lem:X0Odd},~\ref{lem:v1Orth}, and Corollary~\ref{x0pairing},  $S'_{\ell}$ is one of $\emptyset,\{3,4\},\{3,5\}$ and $\{4,5\}$ if $s=0$, and
one of $\emptyset$, $\{ 3, s+5\}$, and $\{s+4,s+5\}$ if $s>0$. Let $j=\min\supp(v_{\ell})$, then $j\ge2$ by Lemma~\ref{lem:v1Orth}.  Also, $j \ne 3$ by Lemma~\ref{lem:NoSingle}.

If $s=0$ and $S'_{\ell}=\{3,4\}$, we have $\braket{v_{\ell}}{x_0}=2$ and $|v_{\ell}|\ge4$, so $x_1\in[v_{\ell}]$. Using Lemma~\ref{intervalproduct}, we get $\braket{v_{\ell}}{v_1}\ne0$, a contradiction.

If $S'_{\ell}=\{ 3, s+5\}$, to avoid $\braket{v_{\ell}}{v_{s+4}}>1$, $2\notin\supp(v_{\ell})$. Thus, $j=3$, which is impossible.

Having proved $S'_{\ell}=\emptyset$ or $\{s+4, s+5\}$, we claim that $S_{\ell}=S_{\ell}'$. First, $j\ne2$ by Lemma~\ref{lem:NoSingle}. So our claim holds when $s=0$. When $s>0$,
if $4\le j<s+3$, we have a claw $(v_j,v_{j-1},v_{j+1},v_{\ell})$. If $j=s+3$, $\braket{v_{\ell}}{v_{s+3}}\neq 0$. By Lemma~\ref{lem:k1=3 sigma3}, $[v_4]$ and $[v_1]$ share their left endpoint. Since $|v_5|=\cdots=|v_{s+3}|=2$ and $v_4\sim v_5 \sim \cdots \sim v_{s+3}$, we have $[v_{\ell}]\subset[v_1]$ by Lemma~\ref{lem:k1=3 intervals}. Thus $\braket{v_{\ell}}{v_1}\ne 0$ by Lemma~\ref{intervalproduct}, a contradiction. So our claim is proved.

Now by Lemma~\ref{lem:j-1}, $s+5\in S_{s+6}$. So $S_{s+6}=\{s+4,s+5\}$ by the results in the previous two paragraphs. If there was $\ell>s+6$ satisfying $S_{\ell}=\{s+4,s+5\}$, we would have a heavy triple $(v_{s+4},v_{s+6},v_{\ell})$. Thus  $S_{\ell}=\emptyset$ whenever $\ell>s+6$.
\end{proof}


%We start by claiming that $\sigma_{s+6}=8s+14$, or equivalently, $v_{s+6}=e_{s+4}+e_{s+5}-e_{s+6}$. By combining the fact that $\braket{v_4}{x_0}=2$ with Lemma~\ref{Lem:X0Odd} and Corollary~\ref{x0pairing}, we see that $S'_{s+6}$ is either $\{0, k_3\}$, $\{k_1, k_3\}$, or $\{k_2, k_3\}$. If $S'_{s+6}=\{0, k_3\}$, by Corollary~\ref{cor:k1=3 intervals} and Lemma~\ref{gappy3}, we get $1,2\in S_{s+6}$. This implies that $\braket{v_{s+6}}{v_3}=2$, which contradicts Corollary~\ref{unbreakablepairing}. Suppose that $S'_{s+6}=\{ k_1, k_3\}$. To avoid $\braket{v_{s+6}}{v_{s+4}}>1$, neither $2$ nor any of $4, \cdots, s+3$ is in $\supp(v_{s+6})$. Thus, $\braket{v_{s+6}}{v_{s+4}}\neq 0$ and $\braket{v_{s+6}}{v_3}\neq 0$, and there will be a heavy triple $(v_3, v_{s+4}, v_{s+6})$: see Figure~\ref{fig:case1 1223 G(S)}. Therefore, $S'_{s+6}=\{k_2, k_3\}$. We claim that $S_{s+6}=S'_{s+6}$. First note that $v_{s+6}$ must be orthogonal to $v_1$. If not, then $1\in S_{s+6}$, and so, $2\in S_{s+6}$ (Lemma~\ref{gappy3}); in particular, $|v_{s+6}|\ge 5$. Since $\delta([v_1], [v_{s+6}])\le 3$, we get that $|\braket{v_1}{v_{s+6}}|\ge 2$, a contradiction (since it must be $\braket{v_1}{v_{s+6}}=-1$). If $2 \in S_{s+6}$ and $1\not \in S_{s+6}$, then $\braket{v_{s+6}}{v_2}\neq 0$ and $\braket{v_{s+6}}{v_3}\neq 0$, and there will be a heavy triple $(v_3, v_{s+6}, v_{s+4})$: again, see Figure~\ref{fig:case1 1223 G(S)}. The claim follows if $\sigma_4=5$. If $\sigma_4=4$, however, using Lemma~\ref{gappy3} and Corollary~\ref{unbreakablepairing}, we get that $S_{s+6}$ is one of the following cases: $\{ k_2-2, k_2-1, k_2, k_3\}$, $\{k_2-1, k_2, k_3\}$, or $\{k_2, k_3\}$. If $S_{s+6}=\{ k_2-2, k_2-1, k_2, k_3\}$, there will be a claw $(v_{s+2}, v_{s+3}, v_{s+6}, v_{s+1})$, unless $k_2-2=4$. But then $\braket{v_{s+4}}{v_4}\not = 0$, which implies that $\braket{v_{s+4}}{v_1}\not = 0$ (Lemma~\ref{lem:k1=3 sigma3}), a contradiction. Suppose $S_{s+6}=\{ k_2-1, k_2, k_3\}$. Since $\braket{v_{s+6}}{v_{s+3}}\neq 0$, $[v_4]\prec [v_1]$, $|v_5|=\cdots=|v_{s+3}|=2$, $v_4\sim v_5 \sim \cdots \sim v_{s+3}$ in the intersection graph (see Figure~\ref{fig:case1 1223 G(S)}), and $|v_{s+6}|=4$ in this case, we get that $\braket{v_{s+4}}{v_1}\not = 0$, a contradiction. That is, $S_{s+6}=\{ k_2, k_3\}$, and the claim follows. Suppose that there exists $\ell>s+6$ such that $|v_i|=2$ for any $s+6<i<\ell$. By Corollary~\ref{x0pairing}, $S'_{\ell}$ is either $\emptyset$, $\{ 0, k_3\}$, $\{ k_1, k_3\}$, or $\{ k_2, k_3\}$. As in the index $s+6$, $S'_{\ell}\neq \{0, k_3\}$ or $\{k_1, k_3\}$. If $S'_{\ell}=\{ k_2, k_3\}$, then to avoid $\braket{v_{\ell}}{v_{s+6}}=2$ (that violates Corollary~\ref{unbreakablepairing}), we must have $s+6\in \supp(v_{\ell})$. Using Lemma~\ref{gappy3}, $s+7, \cdots, \ell-2\in \supp(v_{\ell})$. This will give a heavy triple $(v_{s+4}, v_{s+6}, v_{\ell})$ if $v_{\ell} \sim v_{s+4}$. If $v_{\ell}\not \sim v_{s+4}$, however, it must be the case that either $2\in \supp(v_\ell)$ or $s+3\in \supp(v_\ell)$. We get a contradiction in each case by using similar arguments to those used for the index $s+6$. We have shown that $S'_{\ell}=\emptyset$. Set $j=\text{min }\supp(v_{\ell})$. If $s+6<j<\ell -1$, there will be a claw on $v_j, v_{j-1}, v_{\ell}, v_{j+1}$, and if $j=s+6$, the claw will be on $v_{s+6}, v_{s+4}, v_{\ell}, v_{s+7}$. So $j=\ell-1$, and by induction we see that $|v_i|=2$ for all $i>s+6$. 



