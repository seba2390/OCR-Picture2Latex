\section{{\bf $k_1=1, k_2>2$}}\label{sec:Case2}

In this section we consider, in the notation of Proposition~\ref{x0}, the case where $k_1=1$ and $k_2>2$. Using Corollary~\ref{v1}, one has
\begin{equation}\label{Eq:Case2X0}
x_0=e_0 + e_1 + e_{k_2} - e_{k_3}, 
\end{equation}
where $k_2>2$. 
Also, we have that $v_1 = e_0 - e_1$. So 
\begin{equation}\label{Eq:Case2Sigma}
\sigma_0 = \sigma_1 = 1.
\end{equation} 
By Lemmas~\ref{lem:BrIsTight} and \ref{lem:tightvector}, the only possible breakable vector is $v_{k_2}$.
In what follows we classify all changemaker vectors whose orthogonal complements are isomorphic to C-type lattices with $x_0$ as given in~\eqref{Eq:Case2X0} and $k_2>2$. We start by determining the first $k_3+1$ components of such changemaker vectors.  

\begin{prop}\label{prop:Segk_1=1k_2>2}
If $k_1 = 1$ and $k_2 > 2$, the initial segment $(\sigma_0,\sigma_1,\cdots,\sigma_{k_3})$ of $\sigma$ is equal to $(1,1,2^{[s]}, \sigma_{k_2},\sigma_{k_2} + 2)$ for some $s > 0$.
\end{prop}
\begin{proof}
We start by observing that, using Lemma~\ref{Lem:X0Odd}, we must have $v_2 = e_0 + e_1 - e_2$. So $\sigma_2 = 2$. By Corollary~\ref{x0pairing}, $\min\supp(v_i)\ge2$ for all $2 < i < k_2$. It follows from Lemma~\ref{lem:AllNorm2} that $|v_i| = 2$ for all $2 < i < k_2$. So $\sigma_i = 2$ for $2 \le i < k_2$. Now, using \eqref{Eq:Case2X0}~and~\eqref{Eq:Case2Sigma} together with the fact that $\braket{\sigma}{x_0} = 0$, we get that $\sigma_{k_3} = \sigma_{k_2} + 2$. We claim that $k_3 = k_2 +1$. Suppose for contradiction that $k_3 \neq k_2 +1$. The component $\sigma_{k_2 + 1}$ must be between $\sigma_{k_2}$ and $\sigma_{k_2} + 2 = \sigma_{k_3}$. If $\sigma_{k_2 + 1}$ is equal to either $\sigma_{k_2}$ or $\sigma_{k_3}$, there will be an element $v \in (\sigma)^\perp$ with $\braket{v}{x_0} = 1$, contradicting Lemma~\ref{Lem:X0Odd}. If $\sigma_{k_2 + 1} = \sigma_{k_2} + 1$, then $v_{k_2 + 1} = e_1 + e_{k_2} - e_{k_2 + 1}$. But then $\braket{v_{k_2 + 1}}{x_0} = 2 \neq 0$, contradicting Corollary~\ref{x0pairing} since $\braket{v_2}{x_0} = 2$. This finishes the proof.
\end{proof}

\begin{cor}\label{cor:sigmak2}
In the situation of Proposition~\ref{prop:Segk_1=1k_2>2}, the component $\sigma_{k_2}$ of the changemaker vector is one of $2s-1$, $2s+1$, or $2s+3$. These correspond to $v_{k_2}$ being gappy, just right, or tight, respectively.
\end{cor}
\begin{proof}
If $v_{k_2}$ is tight, the third of these possibilities occurs. If not, using Corollary~\ref{x0pairing}, we get that $\braket{v_{k_2}}{x_0} = 0$. (Note that $\braket{v_2}{x_0} = 2$.) So $1 \in \supp^+(v_{k_2})$ and $0 \not \in \supp^+(v_{k_2})$. Since $|v_j| = 2$ for $2 < j < k_2$, Lemma~\ref{gappy3} implies that the only possible gappy index for $v_{k_2}$ is $1$, so
\begin{equation*}
v_{k_2} = e_1 + e_j + e_{j+1} + \cdots + e_{k_2 - 1} - e_{k_2},
\end{equation*}
for some $1 < j < k_2$. If $j > 3$, the pairing graph will have a cycle on $v_2, \cdots, v_j, v_{k_2}$ of length larger than $3$, contradicting Corollary~\ref{cycles}. In particular, if $1$ is indeed a gappy index for $v_{k_2}$, then $j = 3$, and $\sigma_{k_2} = 2s -1$. Otherwise one has $j = 2$, and therefore $\sigma_{k_2} = 2s+1$. 
\end{proof}

It turns out that the classification will highly depend on the type of the vector $v_{k_2}$: whether it is tight, just right, or gappy. For $j > k_3$, let 
\begin{equation}\label{Eq:Sj}
S_j = \supp(v_j) \cap \{0,1,\dots,k_3\},
\end{equation} 
and let 
\begin{equation}\label{Eq:S'j}
S_j' = \supp(v_j) \cap \{0,1,k_2,k_3\}.
\end{equation} 
Given that $\braket{v_2}{x_0} = 2$ and, using Corollary~\ref{x0pairing}, we must have $\braket{v_j}{x_0} = 0$, and that $S_j'$ is one of $\emptyset$, $\{1,k_3\}$, or $\{k_2,k_3\}$ by Lemma~\ref{gappy3}. Figure~\ref{pairinggraphs} depicts the paring graphs of the possible changemaker C-type lattices on their first $k_3$ vectors of the basis $S'$, defined in~\eqref{Eq:S'}, depending on the type of $v_{k_2}$. With a slight abuse of notation, we often use $v_{k_3}$ in place of $x_0$.

\begin{figure}[t]
\centering
\begin{align*}
\xymatrix@R1.5em@C1.5em{
v_1^{(2)} \ar@{-}[d] & x_0^{(4)} \ar@{=}[d]^+ \\
v_{k_2}^{(s+2)} & v_2^{(3)} \ar@{-}[d] \\
 & v_{3}^{(2)} \ar@{-}[d] \\
 & \vdots \ar@{-}[d] \\
 & v_{k_2 - 1}^{(2)}
}
&&
\xymatrix@R1.5em@C1.5em{
v_1^{(2)} \ar@{-}[d]_+ & x_0^{(4)} \ar@{=}[d]^+ \ar@{=}[dl]_+\\
v_{k_2}^{(s+6)} \ar@{=}[r]_+ & v_2^{(3)} \ar@{-}[d] \\
 & v_{3}^{(2)} \ar@{-}[d] \\
 & \vdots \ar@{-}[d] \\
 & v_{k_2 - 1}^{(2)}
}
&&
\xymatrix@R1.5em@C1.5em{
v_1^{(2)} \ar@{-}[d] & x_0^{(4)} \ar@{=}[d]^+ \\
v_{k_2}^{(s+1)} \ar@{-}[r]^+ \ar@{-}[dr] & v_2^{(3)} \ar@{-}[d] \\
 & v_{3}^{(2)} \ar@{-}[d] \\
 & \vdots \ar@{-}[d] \\
 & v_{k_2 - 1}^{(2)}
}
\end{align*}
\caption{Pairing graphs of the standard basis when $v_{k_2}$ is just right (left), tight (center), and gappy (right). Superscripts give the norm of the basis vector, the number of edges gives the absolute value of the inner product, and an edge is labelled with $+$ if the inner product is positive. }\label{pairinggraphs}
\end{figure}

With the notation of this section in place:
\begin{lemma}\label{Sjempty}
If $S_j' = \emptyset$, $S_j$ is either $\emptyset$ or $\{k_2 - 1\}$. In the second case, $v_{k_2}$ is not gappy. 
\end{lemma}
\begin{proof}
Set $i = \min  S_j$. Suppose for contradiction that $S_j$ is nonempty and $i < k_2 - 1$. If $i > 2$,  then there will be a claw on $v_i, v_{i+1}, v_{i-1}, v_j$. If $i = 2$ there will be a claw $(v_2, v_3, x_0, v_j)$. Therefore, $i = k_2 - 1$, and so the first statement follows. If $S_j = \{k_2 - 1\}$, then $\braket{v_j}{v_{k_2 - 1}} = -1$, $\braket{v_j}{v_{k_2}} = 1$, and $\braket{v_j}{v_i} = 0$ for all other $i \le k_3$, so if $v_{k_2}$ is gappy there is a claw $(v_{k_2}, v_1, v_2, v_j)$ (see Figure~\ref{pairinggraphs}).
\end{proof}

\begin{lemma}\label{Sjup}
If $S_j' = \{k_2,k_3\}$, $S_j$ is either $\{k_2,k_3\}$ or $\{k_2 - 1, k_2, k_3\}$. In either case, $v_{k_2}$ is not gappy. 
\end{lemma}
\begin{proof}
Again, set $i = \min  S_j$. If $i< k_2-1$, there will be a claw on either $v_i, v_{i+1}, v_{i-1}, v_j$ or $v_2, v_3, x_0, v_j$, depending on whether $i>2$ or $i=2$. So the first statement follows. Corresponding to the two possibilities for $S_j$, the vector $v_j$ will have nonzero inner product with either $v_{k_2}$ or $v_{k_2 - 1}$, but no other $v_i$ with $i \le k_3$. If $v_{k_2}$ is gappy, this creates a claw $(v_{k_2}, v_1, v_2, v_j)$ in the first case, and  a heavy triple $(v_2,v_{k_2},v_j)$ in the second: again, see Figure~\ref{pairinggraphs}.
\end{proof}

\begin{lemma}\label{Sjdown}
If $S_j' = \{1,k_3\}$, either $S_j$ is one of $\{1,2,3,\dots,k_2 - 1, k_3\}$ and $\{1,3,\dots,k_2 - 1, k_3\}$ and $v_{k_2}$ is tight, or $S_j = \{1,k_3\}$, $s = 1$, and $v_{k_2}$ is not gappy.
\end{lemma}
\begin{proof}
Using Lemma~\ref{gappy3}, none of $2,\dots,k_2-2$ can be a gappy index for $v_j$. Thus, we must have either $S_j = \{1,k_3\}$ or
$%\begin{equation*}
S_j = \{1,k,k+1,\dots,k_2 - 1, k_3\}%,
$%\end{equation*}
for some $1 < k < k_2$. 

In the first case, $v_j$ will have nonzero inner product with just $v_1$, $v_2$, and $v_{k_2}$. If $v_{k_2}$ is gappy, this creates a heavy triple $(v_2, v_{k_2}, v_j)$. If $v_{k_2}$ is just right or tight, this creates a claw $(v_2, v_j, x_0, v_3)$, unless $s = 1$: see Figure~\ref{pairinggraphs}.

In the second case, to avoid a cycle $(v_2,v_3,\dots,v_k,v_j)$ of length longer than $3$ (Corollary~\ref{cycles}) we must have $k$ equal to $2$ or $3$. Then $\braket{v_j}{v_{k_2}}$ is either $s$ or $s+1$, and unless $v_{k_2}$ is tight this must be at most $1$ (Corollary~\ref{unbreakablepairing}). Since $s \ge 1$, if $v_{k_2}$ is not tight, we must have $\braket{v_j}{v_{k_2}} = s = 1$. Note that in this case $k_3 = 4$, $k_2=3$, $v_{k_2} = e_1 + e_2 - e_3$, and $S_j = \{ 1,2,4\}$. Consequently, $\braket{v_j}{v_3} = 2$, again contradicting Corollary~\ref{unbreakablepairing}.
\end{proof}

\begin{prop}\label{example}
If $v_{k_2}$ is gappy, then $s\ge2$ and $n+1 = k_3$ (i.e. $v_{k_3}$ is the last standard basis vector). The corresponding changemaker vectors are 
\[(1,1,2^{[s]},2s-1,2s+1), s\ge2.\]
\end{prop}
\begin{proof}
By Corollary~\ref{cor:sigmak2}, $\sigma_{k_2}=2s-1\ge2$, so $s\ge2$.
By Lemmas~\ref{Sjdown},~\ref{Sjup},~and~\ref{Sjempty}, we get that $S_{j} = \emptyset$ for all $j > k_3$. If $v_{k_3 +1}$ existed it would have $k_3 \in S_{k_3 + 1}$. 
\end{proof}

\begin{prop}\label{1k2>2}
If $v_{k_2}$ is just right, then one of the following holds:
\begin{enumerate}
\item $v_{k_3 + 1} = e_{k_2} + e_{k_3} - e_{k_3 + 1}$, $v_{k_3 + 2} = e_{k_2 - 1} + e_{k_2} + e_{k_3} + e_{k_3 + 1} - e_{k_3 + 2}$, and $k_3 + 2 = n+1$.
\item $v_{k_3 + 1} = e_{k_2 - 1} + e_{k_2} + e_{k_3} - e_{k_3 + 1}$, $v_{k_3 + 2} = e_{k_2} + e_{k_3} + e_{k_3 + 1} - e_{k_3 + 2}$, and $k_3 + 2 = n+1$.
\item $s = 1$, so $k_2 = 3$. $v_5 = e_3 + e_4 - e_5$, $|v_i| = 2$ for $5 < i < \ell$, $v_\ell = e_1 + e_4 + e_5 + \cdots + e_{\ell-1} - e_\ell$, and either $v_{\ell+1} = e_{\ell-1} + e_\ell - e_{\ell+1}$ and $|v_i| = 2$ for $i > \ell+1$, or $\ell = n+1$.
\item $s = 1$, so $k_2 = 3$. $v_5 = e_1 + e_4 - e_5$, and either $v_6 = e_3 + e_4 + e_5 - e_6$ and $|v_i| = 2$ for $i > 6$ or $5 = n+1$. 
\end{enumerate}
The corresponding changemaker vectors are
\begin{enumerate}
\item $(1,1,2^{[s]},2s+1,2s+3,4s+4,8s+10)$, $s\ge1$.
\item $(1,1,2^{[s]},2s+1,2s+3,4s+6,8s+10)$, $s\ge1$.
\item $(1,1,2,3,5,8^{[s]},8s+6,8s+14^{[t]})$, $s,t\ge0$, (the parameter $s$ in this family is not the previous $s$.)
\end{enumerate}
\end{prop}
\begin{proof}
We divide the proof into two cases, based on whether or not there is some $\ell$ with $S_\ell = \{1,k_3\}$. If there is no such $\ell$, then by Lemmas~\ref{Sjdown},~\ref{Sjup},~and~\ref{Sjempty}, for any $j > k_3$, $S_j$ is either empty or one of the three possibilities: $\{k_2 - 1\}$, $\{k_2,k_3\}$, or $\{k_2 - 1, k_2, k_3\}$. If $S_j=\{k_2 - 1\}$, $\braket{v_j}{v_{k_2 - 1}}$ and $\braket{v_j}{v_{k_2}}$ are both nonzero, but $\braket{v_j}{v_i} = 0$ for all other $i \le k_3$. If $S_j=\{k_2,k_3\}$, $\braket{v_j}{v_{k_2}}$ is nonzero but $\braket{v_j}{v_i} = 0$ for all other $i \le k_3$, and if $S_j=\{k_2-1,k_2,k_3\}$ only $\braket{v_j}{v_{k_2 - 1}}$ is nonzero. In particular, no $v_j$ with $j \le k_3$ except for $v_{k_2}$ and $v_{k_2 - 1}$ can have nonzero pairing with $v_i$ for some $i > k_3$. Furthermore, for $j$ equal to either $k_2$ or $k_2 - 1$, we claim that there can be at most one $i > k_3$ with $\braket{v_j}{v_i}$ nonzero: if there were two, there would be either a claw if they did not neighbor each other, or a heavy triple if they did. See Figure~\ref{pairinggraphs}. (For instance, if $v_r$ and $v_t$, with $r, t >k_3$, both have nonzero pairing with $v_{k_2-1}$, and also if $v_r$ and $v_t$ pair with each other, then there will be a heavy triple $(v_r, v_t, v_2)$.) Since the pairing graph of a basis must be connected, there in fact must be some $j > k_3$ with $\braket{v_j}{v_{k_2}}$ nonzero, and some $j > k_3$ with $\braket{v_j}{v_{k_2-1}}$ nonzero. This has two implications. First that the vector $v_{k_3+1}$ exists, and either $S_{k_3 + 1}=\{k_2,k_3\}$ or $S_{k_3 + 1}=\{k_2 - 1, k_2, k_3\}$. Second, there is another index $j' > k_3+1$ with $S_{j'}$ equal to the other of these two possibilities of $S_{k_3+1}$.

It remains only to show that $j' = k_3 +2$, and that there is no further standard basis vector. Since $S_{k_3 + 1}\cap S_{j'}=\{k_2,k_3\}$, in order to keep $\braket{v_{k_3 + 1}}{v_{j'}} \le 1$ (Corollary~\ref{unbreakablepairing}), it must be the case that $k_3 + 1 \in \supp^+(v_{j'})$, and in this case $\braket{v_{k_3 + 1}}{v_{j'}} = 1$. Therefore, $v_{k_3 + 1}$ and $v_{j'}$ are adjacent in the intersection graph. If $j' > k_3 + 2$, then since $S_{k_3 + 2} = \emptyset$, we get that $|v_{k_3 + 2}| = 2$. Therefore, using Lemma~\ref{gappy3}, $k_3 + 1$ cannot be a gappy index for $v_{j'}$, so $k_3 + 2 \in \supp^+(v_{j'})$. This means that $\braket{v_{j'}}{v_{k_3 + 2}} = 0$, so there is a claw on either $v_{k_3 + 1}, v_{k_2}, v_{k_3 + 2}, v_{j'}$ or $v_{k_3 + 1}, v_{k_2-1}, v_{k_3 + 2}, v_{j'}$, depending on the possibilities for $S_{k_3+1}$. Therefore, $j' = k_3 + 2$.

Finally, if $v_{k_3 + 3}$ existed, it would have $S_{k_3 + 3} = \emptyset$, so would equal either $e_{k_3 + 1} + e_{k_3 + 2} - e_{k_3 + 3}$ or $e_{k_3 + 2} - e_{k_3 + 3}$. Therefore, $v_{k_3 + 3}$ would have nonzero inner product with either $v_{k_3 + 1}$ or $v_{k_3 + 2}$ but not both, hence we get a claw centered at either $v_{k_3 + 1}$ or $v_{k_3 + 2}$. 

If there is some $\ell$ with $S_\ell = \{1,k_3\}$, then $s = 1$ by Lemma~\ref{Sjdown}. In this case, $\braket{v_\ell}{v_1} = -1$, $\braket{v_\ell}{v_2} = 1$, $k_2=3$, and $\braket{v_\ell}{v_{k_2}} = 1$. If, for any $ i > k_3$ with $i \neq \ell$, we had $\braket{v_i}{v_2} \neq 0$, there would be either a claw $(v_2,x_0,v_i,v_\ell)$ or a heavy triple $(v_2,v_i,v_\ell)$ depending on whether or not $[v_i]$ and $[v_\ell]$ abut. Since we must have $\braket{v_i}{v_2}=0$ for all $i > k_3$ with $i \neq \ell$, the set $S_i$ cannot be $\{1,k_3\}$, $\{ k_{2}-1\}$ or $\{ k_2-1, k_2, k_3\}$, so by Lemmas~\ref{Sjempty} \ref{Sjup} and \ref{Sjdown}, 
\begin{equation}\label{eq:Si2Cases}
S_i=\emptyset\text{ or }\{k_2,k_3\}.
\end{equation}
Also,  we have 
\begin{equation}\label{eq:pairingvl0}
\braket{v_i}{v_\ell} = 0,\quad \text{ for any $i > k_3$ with $i \neq \ell$}.
\end{equation}
 Otherwise, either $S_i = \emptyset$ in which case there would be a claw $(v_\ell, v_1, v_2, v_i)$, or $S_i = \{k_2,k_3\}$ and there would be a heavy triple $(v_i, v_\ell, v_{k_2})$. 

Now, $k_3 \in S_{k_3 + 1}$ (Lemma~\ref{lem:j-1}), so $S'_{k_3 + 1}$ is either $\{1,k_3\}$ or $\{k_2,k_3\}$. 
It follows from Lemmas~\ref{Sjup} and \ref{Sjdown} and (\ref{eq:Si2Cases}) that $S_{k_3 + 1}=S'_{k_3 + 1}$. If $S_{k_3 + 1} = \{1,k_3\}$, from (\ref{eq:pairingvl0}) we get that $\braket{v_{k_3+2}}{v_{k_3+1}} = 0$ if $n+1\ge k_3+2$, and therefore by (\ref{eq:Si2Cases}), $S_{k_3+2} = \{k_2,k_3\}$. We claim that $S_i = \emptyset$ for $i > k_3 + 2$, and also $k_3 + 1 \not \in \supp(v_i)$. Note that from (\ref{eq:Si2Cases}) if $S_i \not = \emptyset$, one necessarily has $S_i = \{ k_2, k_3\}$. Also, to avoid pairing with $v_{k_3+1}$, it must be the case that $k_3+1 \in \supp^+(v_i)$, but this would imply $\supp^+(v_i)\cap\supp^+(v_{k_3+2})=\{k_2,k_3,k_3+1\}$ hence $\braket{v_i}{v_{k_3+2}}\ge2$, contradicting Corollary~\ref{unbreakablepairing}. So $S_i=\emptyset$, hence $k_3 + 1 \not \in \supp(v_i)$ by (\ref{eq:pairingvl0}). This justifies the claim. It follows from Lemma~\ref{lem:AllNorm2} that $|v_i| = 2$ for $i > k_3 + 2$. This is the last of the possibilities listed in the statement of the proposition.

Lastly, suppose that $S_{k_3 + 1} = \{k_2,k_3\}$ (note that $S_\ell = \{ 1, k_3\}$). When $i>k_3+1$ and $i\ne\ell$, $S_i\ne\{k_2,k_3\}$, otherwise we get a heavy triple $(v_i,v_{k_2},v_{k_3+1})$. So $S_i=\emptyset$  by (\ref{eq:Si2Cases}).
By Lemma~\ref{lem:AllNorm2}, $|v_i| = 2$ for $k_3 + 1 < i < \ell$. By (\ref{eq:pairingvl0}), $v_\ell$ is orthogonal to all of $v_{k_3 + 1}, \dots, v_{\ell - 1}$, so all of $k_3 + 1,\dots,\ell-1$ are members of $\supp v_\ell$, forcing $v_\ell$ to be of the listed form. If $n+1\ge l+1$, $v_{\ell+1}$ is also orthogonal to $v_\ell$, so $\supp v_{\ell+1}\cap\{k_3 + 1,\dots,\ell-1\}$ contains exactly one element, which must be $\ell-1$ by Lemma~\ref{gappy3}. It follows that $v_{\ell+1} = e_{\ell-1} + e_\ell - e_{\ell+1}$, as desired. If, for some $i > \ell+1$, $\braket{v_i}{v_{\ell-1}}$ is nonzero, then $\ell-1\in\supp(v_i)$, and $\ell\in\supp(v_i)$ by (\ref{eq:pairingvl0}), so $\braket{v_i}{v_{\ell+1}}\ne0$ and hence  $(v_{k_3+1}, v_{\ell+1}, v_i)$ is a heavy triple. Therefore, $v_i$ is orthogonal to both $v_{\ell-1}$ and $v_\ell$ for $i > \ell+1$, so by Lemma~\ref{gappy3} $\min \supp v_{i} \ge \ell+1$. Then  Lemma~\ref{lem:AllNorm2} implies that $|v_i| = 2$ for $i > \ell+1$, so we are in the third listed situation.
\end{proof}

\begin{lemma}\label{lem:tightSj}
If $v_{k_2}$ is tight, $S_j$ is one of $\emptyset$, $\{k_2 - 1\}$, or $\{1,2,3,\dots,k_2 - 1, k_3\}$ for each $j > k_3$. 
\end{lemma}
\begin{proof}
By Lemmas~\ref{Sjdown},~\ref{Sjup},~and~\ref{Sjempty}, it suffices to show that $S_j$ cannot be $\{k_2,k_3\}$, $\{k_2 - 1, k_2, k_3\}$, $\{1,3,\dots,k_2 - 1, k_3\}$, or $\{1,k_3\}$. In the first case, $\braket{v_j}{v_{k_2}} = -1$ and $\braket{v_j}{v_i} = 0$ for all other $i \le k_3$. In particular since $v_j$ is orthogonal to $v_1$ and $v_2$, $v_j$ cannot neighbor $v_{k_2}$ in the intersection graph without creating a claw. Therefore, $[v_j] \pitchfork [v_{k_2}]$, and so $\delta([v_j],[v_{k_2}]) = 2$. In order to have $\braket{v_j}{v_{k_2}} = -1$, then, we must have $|v_j| =|[v_j\cap v_{k_2}]|= 3$ and $\epsilon_j = -\epsilon_{k_2}$. Since $\epsilon_j = -\epsilon_{k_2}$ and $[v_j] \pitchfork [v_{k_2}]$, $v_j + v_{k_2}$ is the sum of two distant intervals, so is reducible. However, since $|v_j| = 3$, $j = k_3 + 1$ and $v_j = e_{k_2} + e_{k_3} - e_{k_3 + 1}$, and so $v_{k_2} + v_j$ is irreducible by Lemma~\ref{lem:SumIrr}.

In the second case, $\braket{v_j}{v_{k_2 - 1}} = -1$ and all other $\braket{v_j}{v_i}$ with $i \le k_3$ are zero. Since $\braket{v_2}{x_0}\ne0$, $[v_2]$ contains $x_1$, so $3=|v_2|=|x_1|$.
Since $|v_{k_2}| > 3$, $[v_{k_2}]$ contains high weight elements other than $x_1$. Since $[v_2]$ contains $x_1$ and $v_{k_2 - 1}$ is connected by a path of norm-two vectors to $v_2$, the unique high weight element $x_{z_j}$ of $[v_j]$ is contained in $[v_{k_2}]$. This implies that $\braket{v_j}{v_{k_2}}$ must be nonzero, a contradiction. 

In the last two cases, $v_j$ has nonzero inner product with both $v_1$ and $v_2$, so $[v_j]$ abuts both $[v_1]$ and $[v_2]$. Since $[v_1]$ and $[v_2]$ abut $[v_{k_2}]$ at opposite ends, $[v_{k_2}]$ must be contained in the union of $[v_1], [v_2]$, and $[v_j]$. However, $\braket{v_j}{v_{k_2}} \le s$, so $|v_j| \le s + \delta([v_{k_2}],[v_j]) \le s+2$. This means that there are only two high weight elements in $[v_{k_2}]$, with one being $x_1$ and the other having norm at most $s+2$, so by Lemma~\ref{Lem:IntervalNorm}, $|v_{k_2}|\le s+3$. This contradicts the fact that $|v_{k_2}| = s+6$. 
\end{proof}

\begin{prop}\label{2k2>2}
If $v_{k_2}$ is tight, $v_{k_3 + 1} = e_1 + e_2 + \cdots + e_{k_2 - 1} + e_{k_3} - e_{k_3 + 1}$, $v_{k_3 + 2}$ is either $e_{k_3 + 1} - e_{k_3+2}$ or $e_{k_2 - 1} +  e_{k_3 + 1} - e_{k_3+2}$, and $|v_j| = 2$ for all $j > k_3 + 2$. (None of the vectors past $v_{k_3}$ are necessary to make the lattice C-type --- $n+1$ could be $k_3$ or anything larger.)

The corresponding changemaker vectors are
\begin{enumerate}
\item $(1,1,2^{[s]},2s+3,2s+5,4s+6^{[t]})$, $s\ge1,t\ge0$.
\item $(1,1,2^{[s]},2s+3,2s+5,4s+6,4s+8^{[t]})$, $s\ge1,t\ge1$.
\end{enumerate}
\end{prop}
\begin{proof}
Since $k_3 \in \supp(v_{k_3 + 1})$, $S_{k_3 + 1}$ is necessarily equal to $\{1,2,3,\dots,k_2 - 1, k_3\}$ by Lemma~\ref{lem:tightSj}, and so $v_{k_3 + 1} = e_1 + e_2 + \cdots + e_{k_2 - 1} + e_{k_3} - e_{k_3 + 1}$. 
For any other $j$ with $S_j = S_{k_3 + 1}$, we get that $\braket{v_j}{v_{k_3+1}} \ge k_2-1\ge2$, contradicting Corollary~\ref{unbreakablepairing}. Therefore, for $j > k_3 + 1$, $S_j$ is either $\emptyset$ or $\{k_2 - 1\}$. Suppose for some $j > k_3 + 1$ we have $S_j = \{k_2 - 1\}$. Then $\braket{v_j}{v_{k_2}} = 1$ while $v_j$ is orthogonal to both $x_0$ and $v_1$. Since $\braket{v_{k_2}}{v_1}=1$ and $\braket{x_0}{v_1}=0$, $[v_1]$ abuts the right endpoint of $[v_{k_2}]$. Hence $x_{z_j} \in [v_{k_2}]$.
By Lemma~\ref{intervalproduct}, we get that $|v_j| = 3$,  and $\epsilon_j = \epsilon_{k_2}$. Since also $\braket{v_{k_3 +1}}{v_{k_2}} = s+1$, $\epsilon_{k_3 + 1} = \epsilon_{k_2} = \epsilon_j$, so $\braket{v_j}{v_{k_3 + 1}}$ is either $-1$ or $0$ depending on whether their intervals abut. However, since $|v_j| = 3$, $v_j = e_{k_2 - 1} + e_{j-1} - e_j$, so $\braket{v_j}{v_{k_3 + 1}}$ is $1$ if $j > k_3 + 2$ and $0$ if $j = k_3 + 2$. Therefore, $j = k_3 + 2$ and $S_i = \emptyset$ for $i > k_3 + 2$. 
For any $i > k_3 + 2$, if $\min\supp(v_i)=k_3+1$, $v_i\sim v_{k_3+1}$. Since $v_{k_3+1}\sim v_1$, $\braket{v_{k_3+1}}{v_{k_2}}\ne0$ and $[v_1]$ abuts the right endpoint of $[v_{k_2}]$, $x_{z_{k_3+1}}$ is the rightmost high weight vertex in $[v_{k_2}]$ and $[v_1]$ abuts the right endpoint of  $[v_{k_3+1}]$. As $\braket{v_i}{v_{k_2}}=0$, $[v_i]$ must abut the right endpoint of $[v_{k_3+1}]$. We then conclude that $[v_1]$ and $[v_i]$ abut, which is impossible.
 So $\min\supp(v_i)>k_3+1$ when $i > k_3 + 2$.
Using Lemma~\ref{lem:AllNorm2}, we conclude that $|v_i| = 2$ for $i > k_3 + 2$. 
\end{proof}

%Bringing all this together, the possible changemaker vectors in the case $k_1 = 1$, $k_2 > 2$ are as follows. Here the parameters $s\ge1$, $t\ge0$, unless otherwise stated.
%\begin{itemize}
%\item $(1,1,2^{[s]},2s-1,2s+1)$, $s\ge2$
%\item $(1,1,2^{[s]},2s+1,2s+3,4s+4,8s+10)$
%\item $(1,1,2^{[s]},2s+1,2s+3,4s+6,8s+10)$
%\item $(1,1,2,3,5,8^{[s]},8s+6,8s+14^{[t]})$, $s\ge0$
%\item $(1,1,2^{[s]},2s+3,2s+5,4s+6^{[t]})$
%\item $(1,1,2^{[s]},2s+3,2s+5,4s+6,4s+8^{[t]})$, $t\ge1$
%\end{itemize}
%In all of these except the fourth, $s \ge 1$; in the fourth changemaker $s \ge 0$. In the last three cases, $t \ge 0$ From this, we can compute the coefficients $a_i$ of the C-type lattice in each case. These are
%\begin{enumerate}
%\item $4=3-2^{[s]}-(s+2)-2$
%\item $4=3-2^{[s-1]}-4-4-(s+2)-2$
%\item $4=3-2^{[s-1]}-5-3-(s+2)-2$
%\item \begin{itemize} \item $4=3-s+3-2-3-3-2^{[s-1]}-3-2^{[t-1]}$ if $s,t > 0$ \item $4=3-3-2-3-4-2^{[t-1]}$ if $s = 0$ \item $4=3-s+3-2-3-3-2^{[s-1]}$ if $t = 0$ \item $4=3-3-2-3$ if $s = t = 0$ \end{itemize}
%\item \begin{itemize} \item $4=3-2^{[s-1]}-4-2^{[t-1]}-(s+3)-2$ if $t > 0$ \item $4=3-2^{[s-1]}-(s+5)-2$ if $t = 0$ \end{itemize}
%\item \begin{itemize} \item $4=3-2^{[s-1]}-3-2^{[t-1]}-3-(s+3)-2$ if $t > 0$  \item $4=3-2-2^{[s-1]}-4-(s+3)-2$ if $t = 0$ \end{itemize}
%\end{enumerate}
%From these coefficients, we can compute $p$ and $q$ from the continued fraction. These will be given by the $\mathcal{P}^+$-type from our paper last summer:
%\begin{enumerate}
%\item Type 1A, $p = 2s+1$
%\item Type 1B, $p = 22s+27$
%\item Type 1B, $p = 22s+25$
%\item Type 4, $r = 2s+3$, $p = 2r^2(t+1) - 4r + 1$
%\item Type 3B, $r = 2t+1$, $p = 2r(s+1) + r + 4$
%\item Type 3A, $r = 2t+3$, $p = 2r(s+2) + 1$
%\end{enumerate}
%Type 1B with $p$ equal to $25$ and $27$, Type 3B with $p = 3r + 4$, and Type 3A with $p = 4r + 1$ all occur among the changemakers with $k_1 = 1, k_2 = 1$ as degenerate cases of the ones listed here, so together the changemakers with $k_1 = 1$ cover all cases of Types 1 and 3 with $q > p$, and all cases of Type 4 with $r > 0$.


