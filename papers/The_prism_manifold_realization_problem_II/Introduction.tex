\section{Introduction}\label{sec:Introduction}

This paper is a continuation of \cite{Prism2016}, where the authors studied the Dehn surgery realization problem of prism manifolds. Recall that prism manifolds are spherical three--manifolds with dihedral type fundamental groups.
Alternatively, an oriented prism manifold $P(p,q)$ has Seifert invariants 
\[
(-1; (2,1), (2,1), (p,q)),
\]
where $q$ and $p>1$ are relatively prime integers. A surgery diagram of $P(p,q)$ is depicted in Figure~\ref{links}A. 
%With this notation in place, in tandem with our earlier discussion, we may narrow down the spherical manifold realization problem as follows.
%\begin{question}\label{question}
%Which prism manifolds $P(p,q)$ can be realized by integral surgery on a knot in $S^3$? 
%\end{question}  
When $q<0$, the realization problem for prism manifolds was solved in \cite{Prism2016}. More precisely, a complete list of $P(p,q)$, with $q < 0$, that can be obtained by positive Dehn surgery on knots in $S^3$ is tabulated in~\cite[Table~1]{Prism2016}. Indeed, every manifold in the table can be obtained by surgery on a {\it Berge--Kang knot}~\cite{BergeKang}.
 Our main result, Theorem~\ref{thm:Realization} below, provides the solution for those $P(p,q)$ with $q>p$: see Table~\ref{table:Types}.

\begin{theorem}\label{thm:Realization}
Given a pair of relatively prime integers $p>1$ and $q>p$, the prism manifold $P(p,q)$ can be obtained by $4q$--surgery on a knot $K\subset S^3$ if and only if $P(p,q)$ belongs to one of the six families in Table~\ref{table:Types}. Moreover, in this case, there exists a Berge--Kang knot $K_0$ such that $P(p,q)\cong S^3_{4q}(K_0)$, and that $K$ and $K_0$ have isomorphic knot Floer homology groups.
%and $\widehat{HFK}(K)\cong \widehat{HFK}(K_0)$, where $\widehat{HFK}$ is the knot Floer homology.
\end{theorem}

%leaving open the case $0<q<p$. We plan to address this case in a future paper.

The methodology used to obtain Table~\ref{table:Types} is similar to that of~\cite{greene:LSRP,Prism2016}.  When $q>p$, the prism manifold $P(p,q)$ bounds a negative definite four--manifold $X=X(p,q)$ with a Kirby diagram as in Figure~\ref{links}D: see Section~\ref{sec:Preliminaries}. Let $P(p,q)$ arise from surgery on a knot $K\subset S^3$. Let also $W_{4q}=W_{4q}(K)$ be the corresponding two--handle cobordism obtained by attaching a two--handle to the four--ball along the knot $K$ with framing $4q$. Form the four--manifold $Z:=X\cup_{P(p,q)}(-W_{4q})$. It follows that $Z$ is a smooth, closed, negative definite four--manifold with $b_2(Z)=n+2$ for some $n\ge 1$: see Figure~\ref{links}D. Now, the celebrated theorem of Donaldson (``Theorem~A") implies that the intersection pairing on $H_2(Z)$ is isomorphic to $-\mathbb Z^{n+2}$~\cite{Donaldson1983}, the Euclidean integer lattice with the negation of its usual dot product. This provides a necessary condition for $P(p,q)$ to be positive integer surgery on a knot; namely, the lattice $C(p,q)$, specified by the negative of the intersection pairing on $H_2(X)$, must embed as a codimension one sublattice of $\mathbb Z^{n+2}$. The key idea we use to sharpen this into a necessary and sufficient condition is the work of Greene~\cite{greene:LSRP}, which is built mainly on the use of the {\it correction terms} in Heegaard Floer homology in tandem with Donaldson's theorem. In order to state his theorem, we first require a combinatorial definition.  
\begin{definition}\label{defn:changemaker}
A vector $\sigma=(\sigma_0,\sigma_1,\dots,\sigma_{n+1})\in\mathbb Z^{n+2}$ that satisfies $0\le\sigma_0\le\sigma_1\le\cdots\le\sigma_{n+1}$ is a {\it changemaker vector} if for every $k$, with $0\le k\le\sigma_0+\sigma_1+\cdots+\sigma_{n+1}$, there exists a subset $S\subset\{0,1,\dots,n+1\}$
such that $k=\sum_{i\in S}\sigma_i$.
\end{definition}

Using Lemma~\ref{XSharp}, the following is immediate from~\cite[Theorem~3.3]{Greene2015}.

\begin{theorem}\label{changemakerlatticeembedding}
Suppose $P(p, q)$ with $q>p$ arises from positive integer surgery on a knot in $S^3$. The lattice $C(p, q)$ is isomorphic to the orthogonal complement $(\sigma)^\perp$ of some changemaker vector $\sigma \in \Z^{n+2}$.
\end{theorem}



By determining the pairs $(p,q)$ which pass the embedding restriction of Theorem~\ref{changemakerlatticeembedding}, we get the list of all prism manifolds $P(p,q)$ with $q>p$ that can possibly be realized by integer surgery on a knot in $S^3$: again, see Table~\ref{table:Types}. We still need to verify that every manifold in our list is indeed realized by a knot surgery. In fact, this is the case.
\begin{theorem}\label{thm:lattice}
Given a pair of relatively prime integers $p>1$ and $q>p$,
$C(p,q)\cong(\sigma)^{\perp}$ for a changemaker vector $\sigma\in\mathbb Z^{n+2}$ if and only if $P(p,q)$ belongs to one of the six families in Table~\ref{table:Types}. Moreover, in this case, there exist a knot $K\subset S^3$ with $S^3_{4q}(K) \cong P(p,q)$ and an isomorphism of lattices
\[\phi: (\mathbb Z^{n+2},I)\to (H_2(Z),-Q_Z),\]
such that $\phi(\sigma)$ is a generator of $H_2(-W_{4q})$. Here $I$ denotes the standard inner product on $\mathbb Z^{n+2}$ and $Q_Z$ is the intersection form of $Z= X(p,q) \cup (-W_{4q})$.
\end{theorem}

{\rmk Theorem~\ref{thm:lattice}, in particular, highlights that the families in Table~\ref{table:Types} are divided so that each changemaker vector corresponds to a unique family. However, a prism manifold $P(p, q)$ may belong to more than one family in Table~\ref{table:Types}. %For instance, $P(5,22)$ lies in both Families~1A~and~5. 
We will address the overlaps between the families of Table~\ref{table:Types} in Section~\ref{realizable}: see Table~\ref{Overlap}.




%The list in Table~\ref{table:Types} is obtained by lattice theory techniques. In~\cite[Table~2]{Prism2016}, a list of realizable prism manifolds $P(p,q)$ with 
%$q>0$ is provided. By comparing Table~\ref{table:Types} with those of~\cite[Table~2]{Prism2016} with $q>p$, we verify that our list is complete. Indeed, 


Table~2 in \cite{Prism2016} gives a conjecturally complete list of prism manifolds $P(p,q)$ with $q>0$ that can be obtained by performing surgery on a knot in $S^3$. Every manifold in \cite[Table~2]{Prism2016} is obtained by integral surgery on a {Berge--Kang knot} (see~\cite[Table~4]{Prism2016} and~\cite{BergeKang}). Theorem~\ref{thm:Realization} proves~\cite[Conjecture~1.6]{Prism2016} for the case $q>p$ since the manifolds in Table~\ref{table:Types} coincide with those in~\cite[Table~2]{Prism2016} with $q>p$. We leave open the realization problem for prism manifolds $P(p,q)$ with $0<q<p$. We plan to address this case in a future paper.

  
%%%%%%%%%%%%%%%
%%%%%%%%%%%%%%%
%%%%%%%%%%%%%%%
%%%%%%%%%%%%%%%
%%%%%%%%%%%%%%%
%%%%%%%%%%%%%%%
%%%%%%%%%%%%%%%
\subsection{Organization} Section~\ref{sec:Preliminaries} collects the topological background on prism manifolds, and also reviews the essentials needed to prove our main results. In Section~\ref{sec:CLattices}, we study C--type lattices $C(p,q)$ that are central in the present work. To prove Theorem~\ref{thm:lattice}, we begin with a study of the changemaker lattices (Section~\ref{sec:ChangemakerLattices}), i.e. lattices of the form $(\sigma)^{\perp}\subset \mathbb Z^{n+2}$ for some changemaker vector $\sigma\in \mathbb Z^{n+2}$. We then study when a changemaker lattice, with a {\it standard basis}, is isomorphic to a C--type lattice, with its distinguished {\it vertex basis}. The key to answering this combinatorial question is detecting the {\it irreducible elements} in either of the lattices. Indeed, the standard basis elements of a changemaker lattice are irreducible (Lemma~\ref{lem:irred}), as are the vertex basis elements of a C--type lattice. Furthermore, the classification of the irreducible elements of C--type lattices is given in Proposition~\ref{prop:IntervalsIrreducible}. We collect many structural results about these lattices in Sections~\ref{sec:CLattices}~and~\ref{sec:ChangemakerLattices}.

We classify the changemaker C--type lattices based on how $x_0$, the first element in the ordered basis of a C--type lattice, is written in terms of the standard orthonormal basis elements of $\mathbb Z^{n+2}$. Accordingly, Sections~\ref{sec:Case2},~\ref{sec:k1k21},~and~\ref{sec:Case1} will enumerate the possible changemaker vectors whose orthogonal complements are C--type lattices. Section~\ref{pq} tabulates the corresponding prism manifolds. 

Finally, in Section~\ref{realizable}, we address the overlaps between the families in Table~\ref{table:Types}. More precisely, we provide distinct knots corresponding to distinct changemakers that result in the same prism manifold. We then proceed with proving Theorems~\ref{thm:Realization} and~\ref{thm:lattice}.

 
%%%%%%%%%%%%%%%
%%%%%%%%%%%%%%%
%%%%%%%%%%%%%%%
%%%%%%%%%%%%%%%
%%%%%%%%%%%%%%%
%%%%%%%%%%%%%%%
%%%%%%%%%%%%%%%
\subsection*{Acknowledgements} The heart of this project was done during the Caltech's Summer Undergraduate Research Fellowships (SURF) program in the summer of 2017. 
Y.~N. was partially supported by NSF grant number DMS-1252992
and an Alfred P. Sloan Research Fellowship. F.~V. was partially supported by an AMS-Simons Travel Grant.
W.~B. would like to thank William H. and Helen Lang, as well as Samuel P. and Frances Krown, for their generous support through the SURF program. T.~O. would like to thank Joanna Wall Muir and Mr. James Muir for their generous support through the SURF program.

