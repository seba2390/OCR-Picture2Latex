\section{Preliminaries}\label{sec:Preliminaries}

For a pair of relatively prime integers $p>1$ and $q$, the prism manifold $P(p,q)$ is a Seifert fibered space with a surgery description depicted in Figure~\ref{links}A. It is shown in~\cite{Prism2016} that if $P(p,q)$ is obtained by surgery on a knot in $S^3$, $p$ must be odd. 

An equivalent surgery description for $P(p,q)$ is depicted in Figure~\ref{links}D. To get the coefficients $a_i$, write $\frac{2q-p}{q-p}$ in a Hirzebruch--Jung continued fraction
\begin{equation}\label{eq:ContFrac}
\displaystyle \frac{2q-p}{q-p} = a_1 - \frac{1}{a_2 - \displaystyle\frac{1}{\ddots - \displaystyle\frac{1}{a_n}}} = [a_1,a_2,\dots,a_n]^-.
\end{equation}
From this point on in the paper, we assume that $q>p$. As a result, we have $a_1\ge 3$ in Equation~\eqref{eq:ContFrac}. Moreover, each $a_i\ge2$.

\begin{definition}\label{def:CType}
The \emph{C-type lattice} $C(p,q)$ has a basis 
\begin{equation}\label{VertexBasis}
\{x_0,\dots,x_n\},
\end{equation}
and inner product given by
\begin{equation*}
\braket{x_i}{x_j} = \begin{cases}
4 & i = j = 0 \\
a_i & i = j > 0 \\
-2 & \{i,j\} =\{ 0,1\} \\
-1 & |i-j| = 1, i > 0, j > 0\\
0 &|i-j|>1,
\end{cases}
\end{equation*}
where the coefficients $a_i$, for $i\in \{1,\cdots,n\}$, are defined by the continued fraction~\eqref{eq:ContFrac}. We call~\eqref{VertexBasis} the \emph{vertex basis} of $C(p,q)$.
\end{definition}

\begin{figure}
\centering
\def\svgwidth{\textwidth}
\input{SurgeryPresentationsNew.pdf_tex}
\caption{Surgery presentations of $P(p,q)$. A and B correspond to the two equivalent choices of Seifert invariants $(-1;(2,1),(2,1),(p,q))$ and $(1; (2,1),(2,-1),(p,q-p))$. To go from B to C, blow down two $1$-framed unknots in sequence: first blow down the middle unknot, changing the framing on the upper left unknot to $1$, and then blow down the upper left unknot. Finally, to get to D, use slam-dunk moves to expand $\frac{2q-p}{q-p}$ in a continued fraction. The last link gives a negative-definite four--manifold if $q < 0$ or $q > p$.}
\label{links}
\end{figure}

\begin{figure}
\def\svgwidth{\textwidth}
\includegraphics[scale=.7]{CType.pdf}
\put(-272,-7.5){$4$}
\put(-207,-7.5){$a_1$}
\put(-142,-7.5){$a_2$}
\put(-75,-7.5){$\cdots$}
\put(-9,-7.5){$a_n$}
\caption{A C--type lattice $C(p,q)$ with $\frac{2q-p}{q-p} = [a_1,a_2,\cdots,a_n]^-$. Note that $a_1\ge 3$ when $q>p$.}
\label{CType}
\end{figure}
Let $X=X(p,q)$ be the four--manifold, bounded by $P(p,q)$, with a Kirby diagram as depicted in Figure~\ref{links}D. The inner product space $(H_2(X), -Q_X)$ equals $C(p,q)$, where $Q_X$ denotes the intersection pairing of $X$: see Figure~\ref{CType}. Note that $b_2(X)=n+1$, where $n$ is defined in~\eqref{eq:ContFrac}.
%\begin{lemma}\label{lem:QXdefinite}
%$X(p,q)$ is a negative definite four--manifold.
%\end{lemma}
%\begin{proof}
%We prove that $-Q_X$ is positive definite. Given a vector $v\in H_2(X)$, it is straightforward to check that $-Q_X(v,v)$ is an increasing function of the $a_i$ for $i = 1,\dots, n$. In particular it suffices to prove the claim when each $a_i$ satisfies $a_i = 2$ for $i>1$, and $a_1=3$. By induction on $b_2(X) = n+1$ with $n\ge 2$, we get that
%		\[\mathrm{det}(-Q_X) = 4(n+1).\]
%Since all principal minors are positive by the induction hypothesis, we see that $-Q_X$ is positive definite.
%\end{proof}
{\rmk When $q<0$ in Equation~\eqref{eq:ContFrac}, it follows that $a_1=2$ and $C(p,q)$ is indeed isomorphic to a D--type lattice~\cite[Definition~2.8]{Prism2016}. The prism manifold realization problem is solved in this case~\cite{Prism2016}.
}

\begin{figure}

\centering
\def\svgwidth{.8\textwidth}
\input{MontesinosWithSlides.pdf_tex}
\caption{A handle decomposition of a surface embedded in $S^3$. The boundary of this surface is an alternating Montesinos link whose branched double cover is $P(p,q)$, and the branched double cover of $B^4$ over this surface with its interior pushed into the interior of $B^4$ is $X(p,q)$. Sliding the 1--handles in this picture along the red arrows and then cancelling all but one of the 0--handles gives Figure~\ref{SlidMontesinos}. This surface depends on parameters $b_1,\dots,b_m$ where $m$ is either $2k+1$ or $2k$; if $m = 2k$ omit the band labelled $b_{2k+1}$.}\label{Montesinos}

\end{figure}
\subsection{The four--manifold $X(p,q)$ revisited}

In this subsection, we present a different construction of the four--manifold $X(p,q)$ as the branched double cover of $B^4$ over a particular surface: see Figure~\ref{Montesinos}. As a Seifert fibered rational homology sphere, the prism manifold $P(p,q)$ is the branched double cover of $S^3$ branched along a Montesinos link~\cite{Montesinos1973}: choose $b_1,\dots,b_n$ so that
\begin{equation}
\frac{p}{q-p} = b_1 + \frac{1}{b_2 + \displaystyle \frac{1}{ \ddots + \displaystyle \frac{1}{b_m}}} = [b_1,b_2,\dots,b_m]^+.
\end{equation}
Since $q > p$, $\frac{p}{q-p} > 0$ and we can choose the $b_i$ so that $b_1 \ge 0$ and $b_i > 0$ for $i > 1$. The boundary of the surface $\Sigma$ drawn in Figure~\ref{Montesinos} is an alternating Montesinos link $L$, and $\Sigma$ itself is the surface formed by the black regions in a checkerboard coloring of the alternating diagram. We point out that we are using the coloring convention as in Figure~\ref{convention}. The branched double cover of $S^3$ branched along $L$ is $P(p,q)$. Let $X_{\Sigma}$ be the branched double cover of $B^4$ over the surface $\Sigma$ with its interior pushed into the interior of $B^4$. With this notation in place:

\begin{figure}[t]
\includegraphics[scale=.6]{BlackWhite.pdf}
\caption{The coloring convention}
\label{convention}
\end{figure}

\begin{figure}

\centering
\def\svgwidth{.8\textwidth}
\input{SlidMontesinos.pdf_tex}
\caption{Another view of the surface shown in Figure~\ref{Montesinos}. From this picture a Kirby diagram representing the branched double cover of $B^4$ over this surface (shown in Figure~\ref{CtypeFromMontesinos}) can be read off using the methods of Figure 4 in \cite{AkbulutKirby1980}. As before, if $m$ is even omit the band labelled $b_{2k+1}$.}\label{SlidMontesinos}

\end{figure}

\begin{prop}\label{BDC}
$X(p,q)\cong X_\Sigma$.
\end{prop}

We first recall the following lemma that will be used in the proof of Proposition~\ref{BDC} and also in Section~\ref{pq}. 
\begin{lemma}[Lemma~9.5~(1)~and~(3) of~\cite{greene:LSRP}]\label{greene9.5}
For integers $r,s,t \ge 0$,
\begin{itemize}
    \item[1.] $[\dots,r,2^{[s]},t,\dots]^- = [\dots, r-1, -(s+1), t-1,\dots]^-$, and
    \item[2.] $[\dots, s, 2^{[t]}]^- = [\dots, s-1, -(t+1)]^-$,
\end{itemize}
where $2^{[a]}$ means that the entry $2$ appears $a$ times.
\end{lemma}

We now proceed to prove Proposition~\ref{BDC}. In order to obtain a Kirby diagram of branched double covers, we closely follow the treatment of~\cite{AkbulutKirby1980}; in particular, see \cite[Figure~4]{AkbulutKirby1980}.

\begin{proof}[Proof of Proposition~\ref{BDC}]
Figure~\ref{Montesinos} depicts a handle decomposition of the surface $\Sigma$ whose branched double cover is $X_\Sigma$. By sliding the 1--handles along the red arrows in Figure~\ref{Montesinos} and then canceling all but only one of the 0--handles, we obtain the surface in Figure~\ref{SlidMontesinos}: a disc with several bands attached. The odd-numbered $b_{2i+1}$ with $0<i<\frac{m-1}2$ contribute bands with $b_{2i+1} + 2$ half-twists, $b_1$ contributes a band with $b_1 + 3$ half-twists, and $b_m$ contributes a band with $b_m + 1$ half-twists when $m$ is odd. The even-numbered $b_{2i}$ contribute $b_{2i} - 1$ bands each, each with $2$ half-twists. Therefore, the coefficients $a_1,\dots, a_n$ of Figure~\ref{CtypeFromMontesinos} are
\begin{equation}\label{equivalent}
(a_1,\dots,a_n) = \begin{cases}
(b_1 + 3, 2^{[b_2 - 1]}, b_3 + 2, 2^{[b_4 - 1]},\dots,2^{[b_{m-1} - 1]},b_m + 1) & m\text{ odd}, \\
(b_1 + 3, 2^{[b_2 - 1]}, b_3 + 2, 2^{[b_4 - 1]},\dots,b_{m-1} + 2,2^{[b_{m} - 1]}) & m\text{ even}.
\end{cases}
\end{equation}
 Using Lemma~\ref{greene9.5}, 
\begin{eqnarray*}
[a_1,\dots,a_n]^-& =& [b_1 + 2, -b_2, b_3, -b_4, \dots, \pm b_m]^-\\ &=& [b_1 + 2, b_2,\dots,b_m]^+ \\ & =& \frac{p}{q-p} + 2 \\ & =& \frac{2q-p}{q-p}. 
\end{eqnarray*} 
That is, the $a_i$ in Equation~\eqref{equivalent} are the same as those of Equation~\eqref{eq:ContFrac}. The branched double cover of $B^4$ branched over the surface in Figure~\ref{SlidMontesinos} is depicted in Figure~\ref{CtypeFromMontesinos}; comparing it with Figure~\ref{links}D, the result follows.
\end{proof}

\subsection{Input from Heegaard Floer homology}

We assume familiarity with Floer homology and only review the essential input here for completeness. See, for instance,~\cite{Ozsvath2004, Ozsvath2004a}. In~\cite{OSzAbGr}, Ozsv\'ath and Szab\'o defined the correction term $d(Y, \mathfrak t)$ that associates a rational number to an oriented rational homology sphere $Y$ equipped with a Spin$^c$ structure $\mathfrak t$. %They showed that this invariant obeys the relation 
%\[
%d(-Y,\mathfrak t) = -d(Y, \mathfrak t),
%\]
%where $-Y$ is the manifold $Y$ with the opposite orientation. 
If $Y$ is boundary of a negative definite four--manifold $X$, then
\begin{equation}\label{eq:CorrBound}
 c_1(\mathfrak s)^2 + b_2(X)\le 4d(Y, \mathfrak t),
\end{equation}
for any $\mathfrak s \in \text{Spin}^c(X)$ that extends $\mathfrak t \in \text{Spin}^c(Y)$.

\begin{definition}
A smooth, compact, negative definite four--manifold $X$ is {\it sharp} if for every $\mathfrak t \in \text{Spin}^c(Y)$, there exists some $\mathfrak s\in \text{Spin}^c(X)$ extending $\mathfrak t$ such that the equality is realized in Equation (\ref{eq:CorrBound}).
\end{definition}



\begin{figure}

\centering
\def\svgwidth{.8\textwidth}
\input{CtypeFromMontesinos.pdf_tex}
\caption{A Kirby diagram representing the branched double cover of the surface in Figure~\ref{Montesinos}. This is the same as the diagram defining $X(p,q)$. The grey box is not part of the link, but is included only to show the relationship with Figure~\ref{SlidMontesinos}.}\label{CtypeFromMontesinos}

\end{figure}

Using Proposition~\ref{BDC}, the following is immediate from~\cite[Theorem~3.4]{OSzBrDoub}.

{\lemma \label{XSharp} $X(p,q)$ is a sharp four--manifold.
}


\subsection{Alexander polynomials of knots on which surgery yield $P(p,q)$ with $q>p$}
Using techniques that will be developed in the next sections in tandem with Theorem~\ref{changemakerlatticeembedding}, we will find the classification of all C-type lattices $C(p,q)$ that are isomorphic to $(\sigma)^{\perp}$ for some changemaker vector $\sigma$ in $\mathbb Z^{n+2}$. If the corresponding prism manifold $P(p,q)$ is indeed arising from surgery on a knot $K \subset S^3$, we are able to compute the Alexander polynomial of $K$ from the values of the components of $\sigma$: let $S$ be the closed surface obtained by capping off a Seifert surface for $K$ in $W_{4q}$. It is straightforward to check that the class $[S]$ generates $H_2(W_{4q})$. It follows from Theorem~\ref{changemakerlatticeembedding} that, under the embedding $H_2(X)\oplus H_2(-W_{4q})\hookrightarrow H_2(Z)$, the homology class $[S]$ gets mapped to a changemaker vector $\sigma$. Let $\{e_0, e_1, \cdots, e_{n+1}\}$ be the standard orthonormal basis  for $\mathbb Z^{n+2}$, and write
\[
\sigma = \sum_{i=0}^{n+1} \sigma_i e_i.
\] 
Also, define the \textit{characteristic covectors} of $\mathbb{Z}^{n+2}$ to be 
\[
\text{Char}(\mathbb Z^{n+2})=\left \{ \left.\sum_{i=0}^{n+1}\mathfrak c_i e_i \right| \mathfrak c_i\text{ odd for all } i\right \}.
\]
We remind the reader that, writing the Alexander polynomial of $K$ as 
\begin{equation}\label{eq:AlexanderPolynomial}
\Delta_K(T)= b_0 + \sum_{i>0}b_i(T^i+T^{-i}),
\end{equation}
the $k$-th {\it torsion coefficient} of $K$ is
\[
t_k(K)= \sum_{j\ge 1}jb_{k+j},
\]
where $k\ge 0$. The following lemma is immediate from~\cite[Lemma~2.5]{Greene2015}.
\begin{lemma}\label{lem:AlexanderComputation}
The torsion coefficients satisfy
\[
t_i(K)=
\left\{
\begin{array}{cl}
\displaystyle\min_{\mathfrak c}\frac{\mathfrak c^2-n-2}8, &\text{for each $i\in\{0,1,\dots,2q\}$,}\\
&\\
0,&\text{for $i>2q$.}
\end{array}
\right.
\]
where $\mathfrak c$ is subject to 
\[
\mathfrak c\in\mathrm{Char}(\mathbb Z^{n+2}),\quad\langle\mathfrak c,\sigma\rangle+4q\equiv2i\pmod{8q}.
\]
And for $i>0$, 
\[
b_i=t_{i-1}-2t_i+t_{i+1},\quad\text{for }i>0,
\]
and
\[b_0=1-2\sum_{i>0}b_i,\]
where the $b_i$ are as in~\eqref{eq:AlexanderPolynomial}.
\end{lemma}






