\section{Determining $p$ and $q$}\label{pq}
\begin{table}[htb]\centering
\ra{1.2}
\caption{$\mathcal{P}^{+}_{q>p}$, table of $P(p,q)$ that are realizable, $q>p$}\label{table:Types}
   \begin{tabular}{@{}lll@{}} \toprule

    Type & \begin{tabular}{l}$P(p,q)$\end{tabular} &\begin{tabular}{l}Range of parameters  \\
    ($p$ and $r$ are always odd, $p>1$)\end{tabular}\\
    \midrule
    \\
  {\bf 1A}   &\begin{tabular}{l}$P\left(p, \frac{1}{2}(p^2 + 3p + 4)\right)$\end{tabular} & \bigskip  \\ \\ 
  {\bf 1B} &\begin{tabular}{l}$P\left(p, \frac{1}{22}(p^2 + 3p + 4)\right)$\end{tabular} & \begin{tabular}{l}$p\equiv5$ or $3 \pmod{22}$ \\  $p\ne 3, 5$\end{tabular}\\\bigskip \\   
  {\bf 2} &\begin{tabular}{l}$P\left(p, \frac{1}{|4r+2|}(r^2p - 1)\right)$\end{tabular} &\begin{tabular}{l}$r \equiv -1\pmod4$\\
$p \equiv -2r+3\pmod{4r+2}$ \\ $r\ne -5, -1, 3$\end{tabular} \bigskip  \\ \\   
  {\bf 3A} &\begin{tabular}{l}
  $P\left(p, \frac{1}{2r}(p-1)(p-4)\right)$\end{tabular} &\begin{tabular}{l}$p\equiv 1\pmod{2r}$ \\ $p\ne 2r+1$\\ $r  \ge 5$ \end{tabular}\bigskip\\ \\  
  {\bf 3B} &\begin{tabular}{l}$P\left(p, \frac{1}{2r}(p-1)(p-4)\right)$\end{tabular} &\begin{tabular}{l}$p\equiv r+4\pmod{2r}$ \\ $p> r+4$ \\ $r \ge 1$ \end{tabular} \bigskip\\ \\  \bigskip
  {\bf 4} &\begin{tabular}{l}$P\left(p, \frac{1}{2r^2}\left((2r+1)^2p - 1\right)\right)$\end{tabular} &\begin{tabular}{l}$p \equiv -4r+1\pmod{2r^2}$ \\ $r\ne 1,-1$ \end{tabular}\\ \\ \bigskip
  {\bf 5} &\begin{tabular}{l}$P\left(p, \frac{1}{r^2 - 2r - 1}(r^2p - 1)\right)$\end{tabular} &\begin{tabular}{l}$r >  1$\\ $p \equiv -2r + 5\pmod{r^2 - 2r - 1}$ \end{tabular}\\ \\ \bigskip
  {\bf Sporadic} &\begin{tabular}{l}$P(11,19)$, $P(13, 34)$\end{tabular}&\\
  \bottomrule
\end{tabular}

\end{table}

In Sections~\ref{sec:Case2},~\ref{sec:k1k21},~and~\ref{sec:Case1}, we have classfied all the $(n+1)$--dimensional C-type lattices that are isomorphic to changemaker lattices. In the present section, we list all the corresponding prism manifolds $P(p,q)$. To do so, we start with the refined basis $S'=\{v_1, \cdots, v_{n+1}\}\setminus \{v_{k_3}\}\cup \{x_0\}$ as defined in~\eqref{Eq:S'}. The first step is changing the basis into the vertex basis $\{x_0, x_1, \cdots, x_{n}\}$. We then recover the $a_i$ from the norms of vertex basis elements. By using Equation~\eqref{eq:ContFrac}, we obtain $p$ and $q$. 

%\begin{lemma}[Lemma~9.5~(1)~(3) of~\cite{greene:LSRP}]\label{greene9.5}
%For integers $r,s,t \ge 2$,
%\begin{itemize}
%    \item[1.] $[\dots,r,2^{[s]},t,\dots]^- = [\dots, r-1, -(s+1), t-1,\dots]^-$, and
%    \item[2.] $[\dots, s, 2^{[t]}]^- = [\dots, s-1, -(t+1)]^-$.
%\end{itemize}
%\end{lemma}


\begin{ex}
We present an example that clarifies how $(p,q)$ is computed in Proposition~\ref{example}. The changemaker is
\[
(1,1,2^{[s]}, 2s-1,2s+1), s=n-2\ge 2.
\]
Let $S'$ denote the modified standard basis for the changemaker lattice $L= (\sigma)^{\perp}$. It is straightforward to check that 
\[
\{x_0\}\cup\{-v_2, \cdots, -v_{s+1}, v_3+\cdots+v_{s+2}, v_1\}
\]
forms the vertex basis $S^*$. Also, the vertex norms are 
\[
\{3, 2^{[s-1]},s+1, 2\}.
\]
Using Lemma~\ref{greene9.5} together with Equation~\eqref{eq:ContFrac}, we have
\[
\displaystyle \frac{2q-p}{q-p}=[3, 2^{[s-1]},s+1, 2]=\frac{4s^2+3}{2s^2-s+2}.
\]
In particular, $p=2s-1$ and $q=2s^2+s+1$. We see that $q=\frac{1}{2}(p^2+3p+4)$, $p\ge3$.
\end{ex}

Similar computations give prism manifolds $P(p,q)$, with $q>p$, so that each falls into one of the families in Table~\ref{table:Types}. We denote the set of such prism manifolds $\mathcal{P}^+_{q>p}$. Here we divide the families so that each changemaker vector corresponds to a unique family. In some cases there are prism manifolds that correspond to more than one family in Table~\ref{table:Types}. For instance, it is straightforward to check that $P(5,22)$ belongs to both Families~5~and~1A. 
The detailed correspondence between the changemaker vectors and $P(p,q)$ can be found in Table~\ref{BigSummary}. Note that the positive integer $p$ is always odd.
