\section{{\bf $k_1 = 1$, $k_2 = 2$}}\label{sec:k1k21}

In this section we consider the case where $k_1=1$ and $k_2=2$. Using Corollary~\ref{v1}, we get that
\begin{equation}\label{Eq:Case1X0}
x_0=e_0 + e_1 + e_{2} - e_{k_3}. 
\end{equation}
Also, we have that $v_1 = e_0 - e_1$. So 
\begin{equation}\label{Eq:Case1Sigma}
\sigma_0 = \sigma_1 = 1.
\end{equation} 
By Lemma~\ref{lem:tightvector}, the only possible tight vector is $v_2$.
In what follows we classify all the changemaker vectors whose orthogonal complements are isomorphic to C-type lattices with $x_0$ as given in~\eqref{Eq:Case1X0}. As in the previous section, we start by determining the first $k_3+1$ components of such changemaker vectors. It turns out that the initial segment of $\sigma$ depends on whether or not $v_2$ is tight. 

\begin{lemma}\label{prop1}
If $v_2$ is tight, the initial segment $(\sigma_0,\sigma_1,\cdots,\sigma_{k_3})$ of $\sigma$ is equal to $(1, 1, 3, 5)$.
\end{lemma}
\begin{proof}
By assumption, $v_2 = 2e_0 + e_1 - e_2$, so $\sigma_2 = 3$ and $|v_2| = 6$. This together with~\eqref{Eq:Case1X0}~and~\eqref{Eq:Case1Sigma}, yields $\sigma_{k_3} = 5$. We claim that $k_3 = k_2+1 = 3$. Suppose for contradiction that $k_3 \not = k_2 + 1$. Recall from Lemma~\ref{lem:tightvector} that $v_{k_2+1}$ cannot be tight. By combining this together with Lemma~\ref{Lem:X0Odd}, it can only be the case that $\sigma_{k_2+1}=4$ and $v_3 = e_1+e_2-e_3$. Note that $\braket{v_2}{x_0}=2$, $\braket{v_1}{x_0}=0$, and $\braket{v_1}{v_2}=1$. Therefore, $[v_1]$ abuts the right endpoint of $[v_2]$. Given that $[v_3]$ abuts both $x_0$ and $[v_1]$, it follows that the only high weight vertex of $[v_2]$ is that of $[v_3]$ (see Definition~\ref{Def:HighNorm} and Lemma~\ref{lem:TwoIndBr}). This implies that $|[v_2]|=|[v_3]|=3$ which is a contradiction. Hence $k_3 = 3$ and $v_3 = e_0+e_1+e_2 - e_3$. 
\end{proof}

\begin{lemma}\label{prop2}
If $v_2$ is not tight, the initial segment $(\sigma_0,\sigma_1,\cdots,\sigma_{k_3})$ of $\sigma$ is equal to either $(1, 1, 1, 3)$ or $(1, 1, 1, 2, 3)$.
\end{lemma}

\begin{proof}
When $v_2$ is not tight, using Lemma~\ref{Lem:X0Odd} together with the fact that $k_2 = 2$, we get that $v_2 = e_1 - e_2$, so $\sigma_2 = 1$. This together with~\eqref{Eq:Case1X0}~and~\eqref{Eq:Case1Sigma}, gives us that $\sigma_{k_3} = 3$. Either $k_3=3$ and we get the first possibility stated in the proposition, or $k_3>3$. In the latter case, using Lemmas~\ref{Lem:X0Odd}~and~\ref{lem:j-1}, we must have that $v_3=e_1+e_2-e_3$, so $\sigma_3=2$. We claim that, if $k_3>3$, then $k_3=4$. If $k_3\not = 4$, then we must have $v_4=e_3-e_4$. That will produce a claw on $(v_3, v_4, x_0, v_1)$. This gives the second stated possibility.  
\end{proof}


We use the notation of Equations~\eqref{Eq:Sj}~and~\eqref{Eq:S'j} in Section~\ref{sec:Case2}. Again, we use the basis $S'$, defined in~\eqref{Eq:S'}. Note that in this section, $v_{k_3} = x_0$. Moreover, if $k_3=3$, then $S_j=S'_j$.


%	\begin{figure}
%		\begin{align*}
%		\xymatrix@R1.5em@C1.5em{
%			& x_0^{(4)} \ar@{=}[d]^{+} \\
%			v_1^{(2)} \ar@{-}[r]^{+} \ar@{=}[dr]^{+} & v_2^{(6)} \ar@{=}[d]^{+} \\
%			& (v_{4}^*)^{(3)} \ar@{-}[d] \\
%			& v_{5}^{(2)} \ar@{-}[d] \\
%			& \vdots
%		}
%		&&
%		\xymatrix@R1.5em@C1.5em{
%			\left(v_4^*\right)^{(3)} \ar@{=}[r]^{+}\ar@{-}[d]^{+} & v_2^{(6)} \ar@3{-}[dr]^{+} \ar@{=}[r]^{+} & x_0^{(4)} \ar@{=}[d]^{+} \\
%			v_1^{(2)} \ar@{-}[ur]^{+} & & v_{5}^{(4)} \ar@{-}[d] \\
%			& & v_{6}^{(2)} \ar@{-}[d] \\
%			& & \vdots
%		}
%		\end{align*}
%		\caption{Pairing graphs when $v_{2}$ is tight}
%		\label{fig:case3 1135 G(S)}
%	\end{figure}


\begin{prop}\label{lem:k1=1,k2=2,v2tight}
	If $v_2$ is tight, then one of the following is true: 
	\begin{enumerate}
		\item $\abs{v_3} = 4$, $v_4 = e_1 + e_3 - e_4$, and $\abs{v_j} = 2$ for all $5\leq j \leq 4+t$, $t\geq 0$. 
		\item $\abs{v_3} = 4$, $v_4 = e_1 + e_3 - e_4$, $v_5 = e_0 + e_1 +e_4-e_5$, and $\abs{v_j} = 2$ for all $6\leq j \leq 5 + t$, $t \geq 0$. 
	\end{enumerate}
	The corresponding changemaker vectors are:
	\begin{enumerate}
		\item $(1,1,3,5,6^{[t]})$
		\item $(1,1,3,5,6,8^{[t+1]})$
	\end{enumerate}
\end{prop}
\begin{proof}
When $v_2$ is tight, using Lemma~\ref{prop1}, the initial segment $(\sigma_0, \cdots, \sigma_{k_3})$ of $\sigma$ is $(1,1,3,5)$. For any $j > 3$, $S_j$ will be one of $\emptyset,\{ 1,2\}$, $\{ 2, 3\}$, $\{ 1,3\}$, $\{0,1\}$, or $\{ 0,1,2,3\}$ by Lemma~\ref{Lem:X0Odd} and Lemma~\ref{gappy3}. We will first show that $\{ 1,2\}$, $\{ 2, 3\}$ and $\{ 0,1,2,3\}$  do not occur. If $S_j = \{1,2\}$ for some $j > 4$, then $\braket{v_j}{v_1} = -1$, $\braket{v_j}{x_0} = 2$, and $\braket{v_j}{v_2} = 0$. Since $[x_0]$ and $[v_1]$ abut $[v_2]$ on opposite ends, and $[v_j]$ abuts both $[x_0]$ and $[v_1]$, the interval $[v_2]$ is contained in the union of $[x_0]$, $[v_j]$, and $[v_1]$. Therefore, $|[v_2 \cap v_j]| =|v_2|= 6$, so $|\braket{v_j}{v_2}| = 6 - \delta([v_j],[v_2]) \ge 3$, a contradiction. If $S_j = \{2,3\}$, then $\braket{v_j}{v_2} = -1$ but $\braket{v_j}{v_1} = \braket{v_j}{x_0} = 0$. To avoid a claw $(v_2,v_1,x_0,v_j)$, then, we must have $[v_2] \pitchfork [v_j]$. Since $v_j$ is orthogonal to $x_0$, this means that $\delta([v_2],[v_j]) = 2$, so $|v_j| = |[v_j\cap v_2]| = 3$ and $\epsilon_2 \neq \epsilon_j$. Therefore, $v_j + v_2$ is reducible. Since $j-1 \in \supp^+(v_j)$, the only way to have $|v_j| = 3$ is to have $j = 4$, but then $v_j + v_2$ is irreducible by Lemma~\ref{lem:SumIrr}. If $S_j = \{0,1,2,3\}$, $\braket{v_j}{v_2} = 2$, $\braket{v_j}{x_0} = 2$, and $\braket{v_j}{v_1} = 0$. Also, $|[v_2 \cap v_j]| = |v_j| \ge 5$, so in order to have $\braket{v_j}{v_2} = 2$ we must have $\epsilon_j = \epsilon_2$ and $\delta([v_2],[v_j]) = 3$. By Lemma~\ref{lem:delta}, $\braket{v_j}{x_0}=-\braket{v_2}{x_0}=\pm2$, a contradiction. Therefore, for each $j > 3$, $S_j$ is one of $\emptyset$, $\{0,1\}$ and $\{1,3\}$. Furthermore, if $S_j = \{0,1\}$, then $\braket{v_j}{x_0} \neq 0$, so by Corollary~\ref{x0pairing} there is at most one $j$ with $S_j = \{0,1\}$.

If the index $4$ exists, $3 \in S_4$, so $S_4 = \{1,3\}$, $v_4 = e_1 + e_3 - e_4$, and $\sigma_4 = 6$. If, for some $j > 4$, $S_j = \{1,3\}$, then also $4 \in \supp^+(v_j)$ by Corollary~\ref{unbreakablepairing}. Therefore, $|v_j| \ge 4$ and $\braket{v_j}{v_4} = 1$, so $[v_4]$ abuts $[v_j]$. Since $v_j$ is orthogonal to $x_0$, $\delta([v_2],[v_j])\le 2$, so since $|v_j| \ge 4$ and $\braket{v_j}{v_2} = 1$ we must have $[v_2]\dagger [v_j]$. Therefore, using Corollary~\ref{Cor:ZjDistinct}, either $[v_2]$ and $[v_4]$ are distant or they share a common end, but in either case we cannot have $\braket{v_2}{v_4} = 1$. Therefore, there is at most one $j > 4$ with $S_j = \{0,1\}$, and for all other $i$ we have $S_i = \emptyset$. 
Suppose that for some $j$ we have $S_j = \{0,1\}$. It follows from Lemma~\ref{lem:AllNorm2} that $|v_i|=2$ when $4<i<j$.
By Lemma~\ref{gappy3}, $v_j = e_0 + e_1 + e_k + e_{k+1} + \cdots + e_{j-1} - e_j$ for some $4 \le k < j$, and to avoid a claw $(v_j,v_1,x_0,v_k)$ we must have $k = 4$. Therefore, $|v_j| = j-1 \ge 4$. Since $\braket{v_j}{v_2} = 3$, we must have $\epsilon_j = \epsilon_2$, and since $\braket{v_j}{x_0}=\braket{v_2}{x_0}=2$ this means that $\delta([v_2],[v_j]) = 1$. Therefore, $|v_j| = \braket{v_j}{v_2} + 1 = 4$, so $j = 5$. This means that $S_5$ is either $\emptyset$ or $\{0,1\}$, and $S_i = \emptyset$ for $i > 5$. 

If $S_5=\emptyset$, by  Lemma~\ref{lem:AllNorm2}, $|v_i|=2$ when $i \ge 5$. If $S_5=\{0,1\}$, we will show that $\min \supp v_i \ge5$ when $i>5$.

We first claim that $x_{z_4}\in[v_2]$. Otherwise, as $\braket{v_4}{v_2}=1$, we get $[v_2]\dagger[v_4]$ and $\epsilon_2=-\epsilon_4$. We also have $\braket{v_2}{v_1}=-\braket{v_4}{v_1}=1$. Thus we have either $[v_1]\prec[v_2]$ or $[v_1]\prec[v_4]$. If $[v_1]\prec[v_2]$, then $\epsilon_1=\epsilon_2$ and $\epsilon_1=\epsilon_4$, a contradiction to $\epsilon_2=-\epsilon_4$. Similarly, we can rule out $[v_1]\prec[v_4]$. This proves the claim.

Note that $\sigma_0=\sigma_1$ are the only two $1$'s in the coordinates of $\sigma$, so there does not exist any norm $2$ vector $y\in (\sigma)^{\perp}$ such that $\braket{y}{v_1}=-1$. Thus $[v_1]$ contains only one vertex which does not neighbor any norm $2$ vertex. 
Since $v_1\sim v_2$ and $\braket{v_1}{x_0}=0$, $[v_1]$ abuts the right end of $[v_2]$. 
As $x_{z_4}\in[v_2]$ and $v_4\sim v_1$, $x_{z_4}$ is the rightmost high weight vertex in $[v_2]$.
If $\min \supp v_i=4$ for some $i>5$, then $v_i\sim v_4$ and $|v_i|\ge3$. 
As $\braket{v_i}{v_2}=0$, $x_{z_i}$ is the leftmost high weight vertex to the right of $[v_2]$. So $[v_1]$ is the unique vertex between $x_{z_4}$ and $x_{z_i}$. We then see that $[v_1]$ and $[v_i]$ abut, which is not possible as $\braket{v_1}{v_i}=0$.
This proves that  $\min \supp v_i \ge5$ when $i>5$.
By  Lemma~\ref{lem:AllNorm2}, $|v_i|=2$ when $i > 5$.
\end{proof}

%We claim that if the index $4$ exists, then $v_4 = e_1 + e_3 - e_4$, so $\sigma_4 = 6$. Since $3\in S_4$ (Lemma~\ref{lem:j-1}), by Lemma~\ref{Lem:X0Odd} we have that $S_4$ is one of $\{ 1, 3 \}$, $\{ 2, 3\}$ or $\{0, 1, 2, 3\}$. If $S_4=\{ 2,3\}$, $\braket{v_4}{v_{2}} = -1$ and $v_4$ is orthogonal to $v_1$ and $x_0$, so $v_4$ cannot neighbor $v_{2}$ in the intersection graph without creating a claw. Therefore, $[v_4] \pitchfork [v_{2}]$ and $\delta([v_4], [v_2])=2$. In order to have $\braket{v_4}{v_{2}} = -1$, however, we must have $\epsilon_4 = -\epsilon_{2}$. Since $\epsilon_4 = -\epsilon_{2}$ and $[v_4] \pitchfork [v_{2}]$, $[v_4] + [v_{2}]$ is the sum of two distant intervals, that is, $[v_4] + [v_{2}]$ is reducible. However, since $|v_4| = 3$ and $v_4 = e_{2} + e_{3} - e_{4}$, we get that $v_{2} + v_4$ is irreducible by Lemma~\ref{lem:SumIrr}, a contradiction. If $S_4=\{ 0,1,2,3\}$, then $|v_4|=5$, $\braket{v_4}{x_0}=2$, and $\braket{v_2}{v_4}=2$, so $\delta([v_2], [v_4]) \in \{ 1,3\}$ (note that $|v_2|=6$). Note also that $|[v_4]\cap [v_2]|\le 5$. Indeed, since $[v_4]$ contains a unique element of high weight, $|[v_4]\cap [v_2]|\in \{2, 5\}$. Using Lemma~\ref{intervalproduct}, to get $\braket{v_2}{v_4}=2$ we must have $\epsilon_4=\epsilon_2$ and $\delta([v_2], [v_4])=3$. The latter two cannot happen simultaneously given that both $v_2$ and $v_4$ have positive inner product with $x_0$, a contradiction. Therefore, $S_4=\{ 1, 3\}$, $\sigma_4=6$, and $v_4 = e_1 + e_3 - e_4$. 

%It remains to be shown that, if the index $5$ exists, then $\sigma_5$ is either $6$ or $8$, and that $|v_i| = 2$ for $i > 5$. For the first statement, we need to show that $S_5$ must be either $\emptyset$ or $\{0,1\}$, or equivalently that $S_5$ cannot be any of $\{1,2\}$, $\{2,3\}$, $\{1,3\}$, or $\{0,1,2,3\}$. If $S_5=\{ 1,2\}$, then $\braket{x_0}{v_5}=2$ and $v_5$ is orthogonal to $v_2$. Since both $v_5$ and $v_2$ pair with $x_0$ we get that $\delta([v_2], [v_5])\in \{ 1, 3\}$. Therefore, to avoid pairing between $v_2$ and $v_5$ we must have $|[v_2]\cap[v_5]|\in \{ 1, 3\}$, which is a contradiction (since $|[v_2]\cap[v_5]|$ is either $|[v_5]|=4$ or $2$, depending on whether or not the high norm vertex of $[v_5]$ is in the intersection of the two intervals). If $S_5=\{ 1, 3\}$, then $v_5=e_1+e_3+e_4-e_5$. Notice that $\braket{v_5}{v_2}=1$ and $\braket{v_5}{x_0}=0$, so it must be the case that $[v_5]$ and $[v_2]$ are consecutive and the left endpoint of $[v_5]$ is to the right of the right endpoint of $[v_2]$ (if it was the case that $[v_2]$ and $[v_5]$ shared an end, for instance, then they would share the end that does not abut $x_0$, and therefore, $|\braket{v_5}{v_2}|=3$). Note that $\braket{v_5}{v_4}=1$. If $[v_4]\pitchfork [v_2]$, to have $\braket{v_4}{v_2}=1$, we must have $|[v_2]\cap [v_4]|=3$, but that will give us a contradiction since in this case $\braket{v_4}{v_5}=0$ or $3$, depending on whether $[v_5]\pitchfork[v_4]$, or $[v_4]$ and $[v_5]$ share an end, respectively. Since $v_5$ and $v_4$ are not orthogonal and both have high norms, it cannot be the case that $[v_4]$ and $[v_5]$ share an end, as otherwise $|\braket{v_5}{v_4}|>1$. This, in particular, implies that $[v_2]$ and $[v_4]$ are not consecutive. Also, if $[v_4]$ and $[v_2]$ shared and end we would then get $|\braket{v_2}{v_4}|=2$, a contradiction. If $S_5=\{ 2, 3\}$, then $\braket{v_5}{v_2}=-1$, but $v_5$ is orthogonal to every other vector $v_i$ with $i\le k_3$. To avoid a claw on $v_2, x_0, v_1, v_5$, it must be the case that $[v_5]\pitchfork [v_2]$ and $\delta([v_5], [v_2])=2$. Note that $|[v_5]\cap [v_2]|$ is either $2$ or $|v_5|=4$, which correspond to $|\braket{v_2}{v_5}|=0$ or $2$, respectively. If $S_5=\{ 0,1,2,3\}$, then $|v_5|=6$, and that $\braket{v_2}{v_5}=2$. Since $\braket{v_5}{x_0}=2$, we get that $|[v_5]\cap [v_2]|=6$. Moreover, $\delta([v_5],[v_2])\in \{1,3\}$, and in either situation it cannot be the case that $\braket{v_2}{v_5}=2$, a contradiction. Therefore, $S_5$ is either $\emptyset$ or $\{0,1\}$. Since always $4 \in \supp^+(v_5)$, $v_5$ is either $e_4 - e_5$ or $e_0 + e_1 + e_4 - e_5$, and $\sigma_5$ is either $6$ or $8$.

%Suppose now that for some $s>5$, we have $|v_i|=2$ for any $5<i<s$. Set $j=\text{min }\supp(v_s)$. We claim that $j=s-1$. The set $S_s$ is one of $\emptyset$, $\{ 1,2\}$, $\{ 2, 3\}$, $\{ 1,3\}$, $\{0,1\}$, or $\{ 0,1,2,3\}$. If $S_s=\emptyset$ and $j<s-1$, using Lemma~\ref{gappy3}, we will get a claw either on $v_j, v_{j+1}, v_s, v_{j-1}$ or on $v_4, v_5, v_s, v_1$ depending on whether $4<j<s-1$ or $j=4$. If $S_s=\{ 0,1,2,3\}$, then to avoid $\braket{v_s}{v_4}=2$ (which violates Corollary~\ref{unbreakablepairing}), we must have $4\in \supp^+(v_s)$. Also, using Lemma~\ref{gappy3} if $\sigma_5 = 6$ or Corollary~\ref{unbreakablepairing} if $\sigma_5 = 8$, we must have $5\in \supp^+(v_s)$. This will give a claw on $v_4, v_s, v_5, v_1$. Suppose $S_s=\{ 0,1 \}$. To avoid a claw on $v_4, v_5, v_s, v_1$, and using Lemma~\ref{gappy3}, we must have $5,\cdots, s-2\in \supp^+(v_s)$. Therefore, $|v_s|\ge 5$. Note that $\braket{v_2}{v_s}=3$ and $\braket{v_s}{x_0}=2$. Since both $v_2$ and $v_s$ have nonzero inner product with $x_0$, we get that $\delta([v_2], [v_s])\in \{ 1,3\}$, and also that $|[v_s]|\cap|[v_2]|$ is one of $2$ or $|[v_s]|$. For $\braket{v_2}{v_s}=3$ to be true, it must be the case that $\delta([v_2], [v_s])=3$ and $\epsilon_2=\epsilon_s$, which cannot happen simultaneously. If $S_s=\{ 1,3\}$, then to avoid $\braket{v_s}{v_4}=2$ (which violates Corollary~\ref{unbreakablepairing}) we must have $4, \cdots, s-2\in \supp^+(v_s)$ (Lemma~\ref{gappy3}). In particular, $|v_s|\ge 6$. Note that $\braket{v_s}{v_2}=1$ and $v_s$ is orthogonal to $x_0$, so the intervals $[v_2]$ and $[v_s]$ are consecutive, and the left endpoint of $v_s$ must be to the right of the right endpoint of $v_2$. (Using Lemma~\ref{intervalproduct}, there is no way to get $\braket{v_s}{v_2}=1$ if $[v_s]$ share the end with $[v_2]$ that does not abut $x_0$, or $[v_2]\pitchfork [v_s]$.) Note also that $\braket{v_s}{v_4}=1$. Since $|v_s|, |v_4|>2$, the intervals $[v_s], [v_4]$ are consecutive, either at the end of $[v_s]$ that abuts $[v_2]$, or at the other end. In the first case one has $|\braket{v_2}{v_4}|=2$, and in the second case $\braket{v_2}{v_4}=0$, a contradiction. If $S_s=\{ 2,3\}$, then $\braket{v_s}{v_2}=-1$ and $|v_s|\ge 4$. Note that $v_s$ is orthogonal to both $v_1$ and $x_0$. Therefore, in order to avoid a claw on $v_2, v_1, x_0, v_s$ it must be the case that $v_s$ and $v_2$ are not connected to each other in the intersection graph. That is, $[v_2] \pitchfork [v_s]$ and $\delta([v_2], [v_s])=2$, which in turn implies that either $\braket{v_2}{v_s}=0$ or $|\braket{v_2}{v_s}|>1$, depending on whether $|[v_2]\cap [v_s]|$ is equal to $2$ or $|[v_s]|$. If $S_s=\{ 1,2\}$, using Lemma~\ref{gappy3}, we get that
%\[
%e_s= e_1+e_2+e_j+\cdots+e_{s-1}-e_s,
%\] 
%for some $j>3$. Note that $v_s$ is orthogonal to $v_2$ and $\braket{v_s}{v_1}=-1$. To avoid a cycle of length bigger than $3$ on $v_1, v_s, v_4, \cdots, v_j$, we must have $j=4$. In particular, $v_s$ is orthogonal to $v_4$. But that will produce a claw $(v_1, v_2, v_s, v_4)$, a contradiction. Therefore, $j = s-1$, so $|v_s| = 2$.




\begin{prop}\label{prop:k1=1,k2=2,v2justright1}
	If $v_2$ is not tight and $(\sigma_0,\dots,\sigma_{k_3})\not = (1,1,1,2,3)$, then one of the following is true (if only the norm of a standard basis vector is given, it is just right):
	\begin{enumerate}
		\item $\abs{v_3} = 4$, $\abs{v_4} = 3$, $\abs{v_j} = 2$ for $5\leq j \leq 4+t$, $v_{5+t} = e_1 + e_2 + e_4 + e_5 + \dots + e_{4+t} - e_{5+t}$, $\abs{v_{6+t}} = 3$, and $|v_j| = 2$ for $j > 6+t$ ($t \ge 0$).
		\item $\abs{v_3} = 4$, $\abs{v_4} = 3$, and $\abs{v_5} = 6$.
		\item $\abs{v_3} = 4$, $\abs{v_4} = 5$, and $\abs{v_5} = 4$.
			\end{enumerate}
	with corresponding changemaker vectors:
	\begin{enumerate}
		\item $(1,1,1,3,4,4^{[t]},4t+6, (4t+10)^{[s]})$, $s,t\ge0$
		\item $(1,1,1,3,4,10)$	
		\item $(1,1,1,3,6,10)$
			\end{enumerate}
\end{prop}
\begin{proof}
If $v_2$ is not tight and $(\sigma_0,\dots,\sigma_{k_3})\not = (1,1,1,2,3)$, using Lemma~\ref{prop2}, it follows that $(\sigma_0, \cdots, \sigma_{k_3})$ is $(1,1,1,3)$. Note that, using Lemmas~\ref{Lem:X0Odd} and \ref{gappy3}, 
\begin{equation}\label{eq:Siwheni>4}
S_i=\emptyset, \{1,2\}, \{2,3\}, \text{ or } \{0,1,2,3\},\quad\text{when }i\ge4.
\end{equation}
Using Lemma~\ref{lem:j-1}, we get that $S_4$ is either $\{ 2,3\}$ or $\{ 0,1,2,3\}$, that is, $\sigma_4$ is either $4$ or $6$. 

When $\sigma_4=6$, $v_4=e_0 + e_1 + e_2 + e_3 - e_4$.  Since $\braket{v_4}{x_0}=2$, using Corollary~\ref{x0pairing} and (\ref{eq:Siwheni>4}), 
\begin{equation}\label{eq:NewSiwheni>4}
S_i=\emptyset \text{ or }\{2,3\} \quad\text{ when }i>4.
\end{equation}
Since the intersection graph must be connected, there will be some index $j$ for which $S_j=\{2, 3\}$. 
Additionally, using Corollary~\ref{unbreakablepairing}, we get that $4\in \supp^+ v_j$, as otherwise $\braket{v_j}{v_4}=2$. It turns out that there is only one such $j$. In fact, if there were two such indices $j_1,j_2$, then $\{2,3,4\}\subset S_{j_1}\cap S_{j_2}$, we would have $\braket{v_{j_1}}{v_{j_2}}\ge2$, a contradiction.
We claim that $j=5$. If $j\not = 5$, then $S_5 = \emptyset$ by (\ref{eq:NewSiwheni>4}). Therefore, $|v_5|=2$, so, by Lemma~\ref{gappy3}, $4$ cannot be a gappy index for $v_j$. This will give us a claw $(v_4, x_0, v_5, v_j)$. This justifies the claim; in particular, $\sigma_5=10$. If the index $6$ existed, by (\ref{eq:NewSiwheni>4}) we must have $S_6=\emptyset$. Thus, $v_6$ is either $e_4 + e_5 - e_6$ or $e_5 - e_6$. In the first case, there will be a claw $(v_4, v_5, v_6, x_0)$ and in the second case there will be a claw $(v_5, v_4, v_6, v_2)$. So the index $6$ does not exist, and we get the third possibility listed in the proposition.

Now, suppose that $\sigma_4=4$. If $\sigma_5\not = 4, 6$, by Lemma~\ref{lem:j-1} and (\ref{eq:Siwheni>4}), $S_5$ is either $\{ 0, 1, 2, 3\}$ or $\{ 2,3\}$. If $S_i=\{ 0, 1, 2, 3\}$ or $\{ 2,3\}$ for some $i>5$, we will get a heavy triple $(v_4, v_5, v_i)$.
So $S_i=\emptyset$ or $\{1,2\}$ when $i>5$.

If $S_5=\{ 2,3\}$, then $e_5=e_2+e_3+e_4-e_5$. Since the pairing graph is connected, there exists an index $i>5$ such that
$S_i=\{1,2\}$. Using the path $v_i\sim v_1\sim v_2$, we will get a heavy triple $(v_4, v_5, v_i)$. 

If $S_5=\{0,1,2,3\}$, $\sigma_5=10$. If the index $6$ does exist, using Corollary~\ref{x0pairing}, $S_6=\emptyset$. We will have a claw $(v_4,v_2,v_5,v_6)$ or $(v_5,x_0,v_4,v_6)$, depending on whether or not $4\in \supp^+(v_6)$. So we get the second possibility listed in the proposition.

If $\sigma_5=6$, since $\braket{v_5}{x_0}=2$, by Corollary~\ref{x0pairing} and (\ref{eq:Siwheni>4}) we have $S_i=\emptyset$ or $\{2,3\}$ when $i>5$.
Assume that there exists $i>5$ such that $S_i=\{2,3\}$. Since $\braket{v_i}{v_4}\le1$, $4\in\supp(v_i)$. Since $\braket{v_i}{v_5}\le1$, $5\in\supp(v_i)$. We would then have a heavy triple $(v_4,v_5,v_i)$. So $S_i=\emptyset$ whenever $i>5$.
If $|v_6|=2$, there will be a claw $(v_5,v_1,x_0,v_6)$. So $v_6=e_4+e_5-e_6$.
Since $\braket{v_5}{x_0}=2$, $x_{z_5}=x_1$. Since $v_5$ is connected to $v_4$ by a path of norm 2 vectors, $x_{z_4}$ is the leftmost high weight vertex to the right of $x_{z_5}$. Since $v_4\sim v_6$, by Corollary~\ref{Cor:ZjDistinct}, 
$\braket{v_i}{v_5}=\braket{v_i}{v_4}=0$, whenever $i>6$.
We  then conclude that $\min\supp(v_i)\ge6$ when $i>6$. Using Lemma~\ref{lem:AllNorm2}, we get $|v_i|=2$ when $i>6$. This gives us the case $t=0$ in the first possibility listed in the proposition.

If $\sigma_5=4$, 
since the pairing graph is connected, there must be a unique index $j>4$ for which $\braket{v_j}{x_0}=2$. Then $\sigma_j>4$, and $S_j$ is either $\{ 0,1,2,3\}$ or $\{ 1,2\}$ by (\ref{eq:Siwheni>4}). Let $t+5$ be the index such that $\sigma_{t+4}=4<\sigma_{t+5}$.

If $S_j=\{ 0,1,2,3\}$, then in order to avoid $\braket{v_j}{v_4}=2$  (which contradicts Corollary~\ref{unbreakablepairing}) we must have $4\in \supp^+(v_j)$. Moreover, using Lemma~\ref{gappy3}, neither of $4,5,\dots, t+3$ can be a gappy index for $v_j$. Hence we get a claw $(v_4, v_2, v_j, v_5)$ as $j>t+4\ge5$. That is, we must have $S_j=\{ 1,2 \}$. 

We claim that $j=t+5$. Suppose for contradiction that $j \ne t+5$. Then, using Corollary~\ref{x0pairing} and (\ref{eq:Siwheni>4}), $S_{t+5}$ is either $\emptyset$ or $\{ 2, 3\}$. If $S_{t+5} = \{ 2,3\}$, then there will be a heavy triple $(v_4, v_{t+5}, v_j)$, where the paths connecting the three high norm vertices are through $v_1$ and/or $v_2$. If $S_{t+5}=\emptyset$, set $i=\min\supp(v_{t+5})$. Using Lemma~\ref{gappy3}, none of $4, \cdots, t+3$ can be a gappy index for $v_{t+5}$. Then there will be a claw on either $(v_i, v_{i-1}, v_{t+5}, v_{i+1})$ or $(v_4, v_2, v_{t+5}, v_5)$, depending on whether $4<i<t+4$ or $i=4$. (Note that $i\not = t+4$ since $\sigma_{t+5}>4$.) This finishes the proof of the claim, that is, $j=t+5$ and $S_{t+5}=\{ 1,2\}$. 

To avoid a cycle $v_{t+5}\sim v_4\sim v_2\sim v_1\sim v_{t+5}$ of length bigger than $3$ (which violates Corollary~\ref{cycles}), we must have $4\in \supp^+(v_{t+5})$. Furthermore, using Lemmas~\ref{gappy3}~and~\ref{lem:j-1}, all the indices $5, \cdots, t+4 \in \supp(v_{t+5})$, so $\sigma_{t+5}=4t+6$. For $i>t+5$, using Corollary~\ref{x0pairing} and (\ref{eq:Siwheni>4}), the set $S_i$ is either $\emptyset$ or $\{2, 3\}$. If $S_i=\{2,3\}$, we will get a heavy triple $(v_i, v_4, v_{t+5})$. This proves that $S_i=\emptyset$ whenever $i>t+5$. 
Set $\ell=\min \supp(v_{i})$. If $\ell=t+5$, there will be a claw $(v_{t+5}, x_0, v_{i}, v_1)$. If $4<\ell<t+4$, there will be a claw $(v_{\ell}, v_{\ell-1}, v_{i}, v_{\ell+1})$, and if $\ell=4$ the claw will be on $v_4, v_2, v_{i}, v_5$. Therefore $\ell=t+4$ or $\ell\ge t+6$. In particular, $e_{t+6}=e_{t+4}+e_{t+5}-e_{t+6}$ and $\sigma_{t+6}= 4t+10$. 
When $i>t+6$, if $\ell=t+4$, we get a heavy triple $(v_i,v_4,v_{t+6})$. So $\ell\ge t+6$ when $i>t+6$. Now we can apply Lemma~\ref{lem:AllNorm2} to conclude that $|v_i|=2$ whenever $i>t+6$, and we will get the first possibility listed in the proposition.  
\end{proof}



%Suppose there is some $s>t+6$ such that for all $t+6<i<s$, $|v_i|=2$. Set $\ell = \min \supp(v_s)$. We claim that $\ell=s-1$. Using Corollary~\ref{x0pairing} and to avoid a heavy triple on $v_s, v_4, v_{t+5}$, we must have $S_{s}=\emptyset$. If $t+6<\ell<s-1$, noting that none of $t+6, \cdots, s-2$ can be a gappy index for $v_s$ (Lemma~\ref{gappy3}), it follows that there will be a claw on $v_{\ell}, v_{\ell - 1}, v_s, v_{\ell + 1}$. If $\ell = t+6$, there will be a claw on $v_{t+6}, v_{t+4}, v_s, v_{t+7}$. If $\ell=t+5$, there will be a claw on $v_{t+5}, x_0, v_s, v_1$. If $4<\ell<t+5$, there will be a claw $(v_{\ell}, v_{\ell -1}, v_s, v_{\ell +1})$, and if $\ell = 4$, there will be a claw on $v_4, v_2, v_s, v_5$. This will give us that $\ell=s-1$, and we will get the first possibility listed in the proposition.  

%
%
%		\begin{figure}
%		\begin{align*}
%		\xymatrix@R1.5em@C1.5em{
%			x_0^{(4)} \ar@{=}[d]^{+} \\
%			v_j^{(5+A)} \ar@{-}[d]^{+} \\
%			v_{4}^{(3)} \ar@{-}[d] \\
%			v_{2}^{(2)} \ar@{-}[d] \\
%			v_{1}^{(2)} \\
%			\vdots
%		}
%		&&
%		\xymatrix@R1.5em@C1.5em{
%			x_0^{(4)} \ar@{=}[d]^{+} \\
%			v_j^{(3+B)} \ar@{-}[d]^{+} \\
%			v_{1}^{(2)} \ar@{-}[d] \\
%			v_{2}^{(2)} \ar@{-}[d] \\
%			v_{3}^{(3)} \\
%			\vdots
%		}
%		&&
%		\xymatrix@R1.5em@C1.5em{
%			x_0^{(4)} \ar@{=}[d]^{+} \\
%			v_4^{(5)} \ar@{-}[d]^{+} \\
%			v_{j}^{(3+C)} \ar@{-}[d] \\
%			v_{2}^{(2)} \ar@{-}[d] \\
%			v_{1}^{(2)} \\
%			\vdots
%		}
%		\end{align*}
%		\caption{Pairing graphs in the case that $v_2$ is just right and $k_3 = 3$}
%		\label{fig:case3 1134 G(S)}
%	\end{figure}
%



%\begin{figure}
%	\begin{align*}
%	\xymatrix@R1.5em@C1.5em{
%		x_0^{(4)} \ar@{=}[d]^{+} \\
%		v_3^{(3)} \ar@{-}[d] \\
%		v_{1}^{(2)} \ar@{-}[d] \\
%		v_{2}^{(2)} 
%	}
%	\end{align*}
%	\caption{Pairing graph when $v_{2}$ is just right and $k_3 = 4$}
%	\label{fig:case3 11123 G(S)}
%\end{figure}

\begin{prop}\label{prop:k1=1,k2=2,v2justright2}
If $(\sigma_0,\dots,\sigma_{k_3}) = (1,1,1,2,3)$, $v_5 = e_2 + e_3 + e_4 - e_5$, and $|v_j| = 2$ for $j > 5$. In this case, $\sigma = (1,1,1,2,3,6^{[t]})$, $t\ge1$.
\end{prop}
\begin{proof}
Since $4\in S'_5$, $S'_5 = \{ 2, 4\}$ by Lemma~\ref{Lem:X0Odd} and Corollary~\ref{x0pairing}, so the set $S_5$ is equal to either $\{ 2, 4\}$ or $\{ 2, 3, 4\}$. If $S_5=\{ 2, 4\}$, then there will be a cycle of length $4$ on $(v_3, v_1, v_2, v_5)$. Therefore, $S_5 = \{ 2, 3, 4\}$, and so, $\sigma_5=6$. There is a path $v_3\sim v_1\sim v_2\sim v_5$. 
For any $i>5$, to avoid a heavy triple $(v_i,v_3,v_5)$, $v_i$ cannot neighbor $v_1$ or $v_2$. Combined with Lemmas~\ref{gappy3} and~\ref{Lem:X0Odd} and Corollary~\ref{x0pairing}, we must have $S'_i=\emptyset$. If $3\in S_i$, we would have a claw $(v_3,v_i,v_1,x_0)$. So $S_i=\emptyset$. By Lemma~\ref{lem:AllNorm2}, we have $|v_i|=2$ whenever $i>5$.

Now  $\sigma = (1,1,1,2,3,6^{[t]})$, $t\ge0$. If $t=0$, then $p=1$, (see Section~\ref{pq}.) So we must have $t\ge1$.
\end{proof}


%Suppose that there exists $s>5$ such that for all $5< i < s$, $|v_i|=2$. Set $j= \min\supp v_{s}$. We claim that $j=s-1$. Since $|v_i|=2$ for all $5<i<s$, none of $5,\cdots,s-2$ can be a gappy index for $v_s$. If $5<j<s-1$, there will be a claw on $(v_j, v_s, v_{j-1}, v_{j+1})$. If $j=5$, there will be a claw on $(v_5, v_2, v_s, v_6)$. If $j \le 4$, using Lemma~\ref{Lem:X0Odd} and Corollary~\ref{x0pairing}, we must have that either $j=2$ and $4\in \supp^+(v_s)$, or $j=3$ and $4\not \in \supp^+(v_s)$. In the first case, there will be either a heavy triple on $(v_s, v_5, v_3)$ or a claw on $(v_2, v_1, v_s, v_5)$, depending on whether or not $v_5$ and $v_s$ have nonzero inner product. In the second case ($j=3$ and $4\not \in \supp^+(v_s)$) there will be a claw on $(v_3, v_1, v_s, x_0)$. This gives us that $j=s-1$. That is, for all $s>5$, we have $|v_s|=2$, which gives us the last possibility listed in the proposition. 
