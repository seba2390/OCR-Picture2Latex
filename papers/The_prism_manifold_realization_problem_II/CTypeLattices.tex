\section{C-Type Lattices}\label{sec:CLattices}

This section assembles facts about C-type lattices that will be used in the classification. We mainly use the notation of~\cite{greene:LSRP, Prism2016}. Recall that we always assume $q > p$, so $a_1 \ge 3$: see Figure~\ref{CType}.

Let $L$ be a lattice. Given $v \in L$, let $|v| = \langle v, v \rangle$ be the norm of $v$. An element $\ell \in L$ is {\it reducible} if $\ell = x+y$ for some nonzero $x, y \in L$, with $\langle x, y \rangle \ge 0$, and {\it irreducible} otherwise. An element $\ell \in L$ is {\it breakable} if $\ell = x+y$ with $|x|, |y| \ge 3$ and $\langle x, y \rangle =-1$, and {\it unbreakable} otherwise.

%%%%%%%%%%%%%%%%
%%%%%%%%%%%%%%%
%%%%%%%%%%%%%%%
%%%%%%%%%%%%%%%
%%%%%%%%%%%%%%

%\begin{prop} \label{basesirred}
%For each $i\in\{0, ..., n\}$, $x_i$ is an irreducible element of $C(p,q)$, .
%\end{prop}


Among the irreducible elements of a lattice, intervals are the most convenient for us:
\begin{definition}
In a C-type lattice, if $I$ is any subset of $\{x_0,x_1,\dots,x_n\}$ then write $[I] = \sum_{x \in A} x$. An {\it interval} is an element of the form $[I]$ with $I = \{x_a,x_{a+1},\dots,x_b\}$ for $0 \le a \le b \le n$. We say that $a$ is the left endpoint of the interval, and $b$ is the right endpoint of the interval. Say that $[I]$ contains $x_i$ if $I$ does. 
\end{definition}

%\begin{prop}
%Intervals are irreducible.
%\end{prop}


%%%%%%%%%%%%%%%%
%%%%%%%%%%%%%%%
%%%%%%%%%%%%%%%
%%%%%%%%%%%%%%%
%%%%%%%%%%%%%%
Given the fact that $a_1 \ge 3$, the following is immediate from~\cite[Proposition~3.3]{greene:LSRP}.

\begin{prop}\label{prop:IntervalsIrreducible}
If $v \in C(p,q)$ is irreducible, $v = \epsilon[I]$ for some $\epsilon = \pm 1$ and $[I]$ an interval.
\end{prop}

\begin{definition}\label{defn:pairinggraph}
Given a lattice $L$ and a subset $V\subset L$, the {\it pairing graph} is $\hat{G}(V) = (V, E)$, where $e = (v_i, v_j) \in E$ if $\langle v_i, v_j \rangle \neq 0$.
\end{definition}

\begin{cor}\label{indecomposable}
The lattice $C(p,q)$ is indecomposable; that is, $C(p,q)$ is not the direct sum of two nontrivial lattices.
\end{cor}
\begin{proof}
Suppose that $C(p,q) \cong L_1 \oplus L_2$. Then each $x_i$, being irreducible, must be in either $L_1$ or $L_2$. However, any element of $L_1$ has zero pairing with any element of $L_2$. Since $\braket{x_i}{x_{i+1}}\ne0$, $\hat{G}(\{x_0, \dots, x_n\})$ is connected. This means that all of the $x_i$ are in the same part of the decomposition, and the other is trivial.
\end{proof}

In a C-type lattice, we have that $|\langle x_0, x_1\rangle |=2$. It turns out that the inner product of $x_0$ with any other element in the $C$-type lattice lives in $2\mathbb Z$. The following lemma is straightforward to prove.


\begin{lemma}\label{Lem:X0Odd}
For any $v \in C(p,q)$, $\braket{x_0}{v}$ is even. In particular, the reflection $r_{x_0}: v \mapsto v - 2\frac{\braket{x_0}{v}}{\braket{x_0}{x_0}}x_0$ about $x_0^{\perp}$ is an involution of $C(p,q)$.
\end{lemma}

\begin{definition}\label{Def:HighNorm}
A vertex $x_i$ has {\it high weight} if $i > 0$ and $|x_i| = a_i > 2$.
\end{definition}

\begin{prop}
An element $\epsilon[I] \in C(p,q)$ with $\epsilon \in \{\pm 1 \}$ is unbreakable if and only if $[I]$ contains at most one element of high weight.
\end{prop}
\begin{proof}
The conclusion is obvious when $I=\{x_0\}$. Now we assume $I\ne\{x_0\}$.
If $[I]$ does not contain $x_0$, this reduces to the analogous fact about linear lattices~\cite[Corollary~3.5 (4)]{greene:LSRP}. The reflection $r_{x_0}$ exchanges intervals with left endpoint $0$ and intervals with left endpoint $1$, which reduces the case of intervals containing $x_0$ to the case of intervals not containing $x_0$.
\end{proof}

\begin{definition}
Consider the graph $C$ on vertex set $\{x_0,\dots,x_n\}$ that has two edges between $x_0$ and $x_1$ and one edge between $x_i$ and $x_{i+1}$ for $0 < i < n$. Given two intervals $[I]$ and $[J]$, say that an edge of $C$ is {\it dangling} if one of its ends is in $I$, the other is in $J$, and at least one of the ends is not in $I \cap J$. Write $\delta([I],[J])$ for the number of dangling edges.
\end{definition}

\begin{lemma}\label{intervalproduct}
For two intervals $[I], [J]$, $\braket{[I]}{[J]} = |[I \cap J]| - \delta([I],[J])$. 
\end{lemma}
\begin{proof}
Suppose $I = \{x_a,\dots,x_b\}$ and $J = \{x_c,\dots,x_d\}$. Then we can express
\begin{equation*}
\braket{[I]}{[J]} = \sum_{i = a}^b \sum_{j = c}^d \braket{x_i}{x_j}
\end{equation*}
Terms in this sum with $|i - j| > 1$ vanish. The remaining terms either have $x_i$ and $x_j$ in $I \cap J$, so occur as terms in the expansion of $|[I \cap J]|$, or have at least one of $x_i$ or $x_j$ not in $I \cap J$, so contribute to $\delta([I],[J])$. 
\end{proof}

We frequently use the following lemma, which is stated without proof.

\begin{lemma}\label{Lem:IntervalNorm}
Let $I\ne\{x_0\}$ be an interval.
Then 
\[|[I]|=2+\sum_{x_i\in I\setminus\{x_0\}}(|x_i|-2).\]
\end{lemma}
Given the structure of a C-type lattice, the following is immediate.

\begin{lemma}\label{lem:delta}
For any intervals $I,J$, $\delta([I],[J])$ is $0,1,2,$ or $3$. If $\delta([I],[J]) = 3$, then $\braket{x_0}{[I]}=-\braket{x_0}{[J]}=\pm 2$. 
\end{lemma}

To more precisely describe the value $\delta([I],[J])$, it will be convenient to use some terminology from \cite{greene:LSRP}:

\begin{definition}
For two intervals $[I]$ and $[J]$ with left endpoints $i_0,j_0$ and right endpoints $i_1,j_1$, say that $[I]$ and $[J]$ are {\it distant} if either $i_1 + 1 < j_0$ or $j_1 + 1 < i_0$, that $[I]$ and $[J]$ {\it share a common end} if $i_0 = j_0$ or $i_1 = j_1$, and that $[I]$ and $[J]$ are {\it consecutive} if $i_1 + 1 = j_0$ or $j_1 + 1 = i_0$. Write $[I] \prec [J]$ if $I \subset J$ and $[I]$ and $[J]$ share a common end, and $[I] \dagger [J]$ if they are consecutive. If $[I]$ and $[J]$ are either consecutive or share a common end, say that they {\it abut}. If $I \cap J$ is nonempty and $[I]$ and $[J]$ do not share a common end, write $[I] \pitchfork [J]$. 
\end{definition}

\begin{rmk}
If $\braket{[I]}{x_0} = \braket{[J]}{x_0}$ or if either $\braket{[I]}{x_0}$ or $\braket{[J]}{x_0}$ is zero, then $\delta([I],[J])$ is $0$ if $[I]$ and $[J]$ are distant, $1$ if $[I]$ and $[J]$ abut, and $2$ if $[I] \pitchfork [J]$. If $\braket{[I]}{x_0} \neq \braket{[J]}{x_0}$ and both are nonzero, $\delta([I],[J])$ is $2$ if $[I]$ and $[J]$ abut, and $3$ if $[I] \pitchfork [J]$. In the latter case, $[I]$ and $[J]$ are never distant. 
\end{rmk}

We will also need to know which irreducible elements of $C(p,q)$ are breakable. In light of Proposition~\ref{prop:IntervalsIrreducible}, we only need to study that for intervals.
\begin{lemma}[Lemma~3.10 of \cite{Prism2016}]\label{lem:TwoIndBr}
    An interval $[A]$ is breakable if there are at least two high weight vertices.
\end{lemma}

\begin{definition}\label{Def:Zj} 
For an unbreakable interval $[I_j] \in C(p,q)$ with $|[I_j]|\ge 3$, let $x_{z_j}$ be the unique element with $|x_{z_j}|\ge 3$.
\end{definition}
We end this section by determining when two C-type lattices are isomorphic.

\begin{prop}\label{pp}
If $C(p, q) \cong C(p', q')$, then $p = p'$ and $q = q'$. 
\end{prop}
\begin{proof}
If $L$ is a lattice isomorphic to $C(p,q)$, then to recover $p$ and $q$ from $L$ it suffices to recover the ordered sequence of norms $(|x_1|,|x_2|,\cdots,|x_n|)$. To do this, we will first identify the elements of this sequence that are at least $3$, and then fill in the $2$'s. 

We claim that unless $(p,q)=(2,3)$, there is a unique (up to sign) unbreakable irreducible element $y$ such that $|y|=4$ and $\braket{y}{v}$ is even for all $v$ in $L$, and $y=\pm x_0$. Let $I\ne\{x_0\}$ be any interval representing an unbreakable irreducible element with norm 4. Suppose $I=\{x_a,x_{a+1},\dots,x_b\}$.
If $a>1$, then $\braket{[I]}{x_{a-1}}=-1$ is odd. If $b<n$, then $\braket{[I]}{x_{b+1}}=-1$ is odd. So we assume $a=0$ or $1$, and $b=n$. If $I$ contains at least two high weight vertices, then $I$ is breakable. So $x_1$ is the only high weight vertex, and $4=|[I]|=|x_1|$. If $n>1$, then $\braket{[I]}{x_{b}}=1$ is odd. So $n=1$, $|x_1|=4$. From (\ref{eq:ContFrac}) we get $(p,q)=(2,3)$.

From now on, we assume $(p,q)\ne(2,3)$.
 Let $R$ be the sublattice of $L$ generated by $x_0$ and all vectors of norm $2$. 
Since $L$ contains no vectors of norm $1$, any vector of norm $2$ in $L$ is irreducible. By Lemma~\ref{Lem:IntervalNorm}, then, $R$ is generated by $x_0$ and the $x_i$ with $|x_i| = 2$.

Now, let $V_0$ be the set of irreducible, unbreakable elements of $L\setminus\{\pm x_0\}$ with norm at least $3$, and let $V$ be the quotient of $V_0$ by the relation $v \sim u$ whenever either $v-u \in R$ or $v+u \in R$. Every element of $V_0$ corresponds to an interval containing a unique high-weight vertex, and $v \sim u$ if and only if these high-weight vertices are the same. Therefore, $V$ consists of precisely the equivalence classes of the $x_i$ with $|x_i| \ge 3$, $i>0$, and if $v \in V_0$ with $v \sim x_i$ we have $|v| = |x_i|$.

Finally, let $W$ be the set of indecomposible components of $R$, so each element of $W$ corresponds to either $x_0$ or a run of $2$'s in the sequence of norms $(|x_1|,|x_2|,\dots,|x_n|)$. Let $\mathcal{B}$ be the bipartite graph with  vertex set $V \cup W$, and an edge between $v \in V$ and $w \in W$ if there is a representative $\tilde{v} \in L$ of $v$ and an element $\tilde{w} \in W$ such that $\braket{\tilde{v}}{\tilde{w}} = -1$, or $w$ corresponds to $x_0$ and $\braket{\tilde{v}}{x_0} = -2$. Then $v$ and $w$ neighbor in $\mathcal B$ if and only if the element $x_i$ representing $v$ is adjacent to $x_0$ or the run of $2$'s corresponding to $w$, so $\mathcal{B}$ is in fact a path. Furthermore, there is a unique element $w_0 \in W$ that contains $x_0$, and $w_0$ must be one of the ends of the path $\mathcal{B}$. We can now recover $(|x_1|,|x_2|,\dots,|x_n|)$ as follows:  The vertex $w_0$ neighbors a unique element $v \in V$ in $\mathcal{B}$. The rest of the sequence is completed in the following way - as we travel down the path $\mathcal{B}$, when we encounter an element $w \in W$ we add $\text{rk } w$-many $2$'s to the sequence, and when we encounter an element $v \in V$ we add $|\tilde{v}|$ to the sequence for $\tilde{v}$ a representative of $v$. 
\end{proof}
