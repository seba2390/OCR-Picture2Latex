%% ****** Start of file template.aps ****** %
%%
%%
%%   This file is part of the APS files in the REVTeX 4 distribution.
%%   Version 4.0 of REVTeX, August 2001
%%
%%
%%   Copyright (c) 2001 The American Physical Society.
%%
%%   See the REVTeX 4 README file for restrictions and more information.
%%
%
% This is a template for producing manuscripts for use with REVTEX 4.0
% Copy this file to another name and then work on that file.
% That way, you always have this original template file to use.
%
% Group addresses by affiliation; use superscriptaddress for long
% author lists, or if there are many overlapping affiliations.
% For Phys. Rev. appearance, change preprint to twocolumn.
% Choose pra, prb, prc, prd, pre, prl, prstab, or rmp for journal
%  Add 'draft' option to mark overfull boxes with black boxes
%  Add 'showpacs' option to make PACS codes appear
%  Add 'showkeys' option to make keywords appear
%\documentclass[aps,prl,preprint,groupedaddress]{revtex4}
%\documentclass[aps,prl,twocolumn,superscriptaddress,showpacs,longbibliography]{revtex4}  % for PRX
%\documentclass[aps,prl,twocolumn,superscriptaddress,showpacs]{revtex4}  % for PRL
\documentclass[aps,prl,superscriptaddress]{revtex4}  % for PRL
%\documentclass[aps,prl,preprintnumbers,twocolumn,superscriptaddress,groupedaddress]{revtex4} 

%\documentclass[aps,prl,twocolumn,groupedaddress]{revtex4}

% You should use BibTeX and apsrev.bst for references
% Choosing a journal automatically selects the correct APS
% BibTeX style file (bst file), so only uncomment the line
% below if necessary.
%\bibliographystyle{apsrev}
\usepackage[dvipdfmx]{graphicx}
%\usepackage[dvipdfmx]{color}
\usepackage{epstopdf}
\usepackage{amsmath}
\renewcommand{\theequation}{S\arabic{equation}}
\usepackage{color}
\usepackage{xcolor}
\usepackage{subfigure}
\usepackage{bm}    
\usepackage{amssymb}  
\usepackage{amsmath}
\definecolor{orange}{RGB}{255,127,0}
\definecolor{blue2}{RGB}{33,114,173}
\usepackage{caption}
\renewcommand{\thefigure}{S\arabic{figure}}
%\renewcommand{\thesubfigure}{S\arabic{figure}}
\captionsetup[figure]{justification=raggedright}
\begin{document}
% Use the \preprint command to place your local institutional report
% number in the upper righthand corner of the title page in preprint mode.
% Multiple \preprint commands are allowed.
% Use the 'preprintnumbers' class option to override journal defaults
% to display numbers if necessary
%\preprint{}
%Title of paper
\title{Supplemental materials to the manuscript of A framework of crossover of scaling law : dynamical impact of viscoelastic surface}
\maketitle

\section{Intermediate asymptotics}

In this section, I briefly explain the concept of {\it intermediate asymptoitcs} which is formalized by Barenblatt\cite{Barenblatt2003,Barenblatt1996, Barenblatt2014,Barenblatt1972} by using a simple example. Intermediate asymptotics is an asymptotic representation of a function valid in a certain range of independent variables, which corresponds to a kind of the formalization of the idealization which accompanies with the construction of physical model. To understand this concept, the following problem of dimensional analysis might be helpful. Imagine that the circle is pictured on the surface of the sphere (See Fig.~\ref{fig:FS1a}). In this problem, the physical parameters that are involved are the surface area of circle $S$, radius of the circle $r$ and the radius of sphere $R$. Here we would like to know the scaling behavior between $S$ and $r$. Therefore we assume the functional relation as follows: $S = f(r, R)$. 

%
\begin{figure}[h!]
\includegraphics[width=8.6cm]{SFigure1a_HM.pdf}
\caption{(Color online) A circle of which radius is $r$ and surface area is $S$, described on a sphere of which radius is $R$.  \label{fig:FS1a}  }
\end{figure}
%
In this case, we attempt to obtain the exact scaling behavior by dimensional consideration. According to dimensional analysis, as the dimension of physical parameters $[S]=L^2$, $[r]=L$ and $[R]=L$, selecting $r$ as a governing parameter of independent dimension, we have the following dimensionless function,
%
\begin{equation}
\Pi = \Phi \left( \theta \right)
\label{eq:ES1}
\end{equation}
%
where $\Pi = \frac{S}{r^2}$ and $\theta = \frac{r}{R}$. Eq.~\ref{eq:ES1} suggests that we expect the following scaling relation, $S \sim r^2$, if $\Pi$ is constant. However, we easily find that this guess depends on the behavior of $\Phi$. 

By the geometrical consideration, in this case, we can calculate the exact form of $\Phi$ as follows,
%
\begin{equation}
\Phi\left(\theta\right) =2\pi \frac{1- {\rm cos} \theta}{\theta^2}.
\label{eq:ES2}
\end{equation}
%
To know the behavior of $\Phi$ in the case in which $\theta \rightarrow 0$, which corresponds to the increase of $R$ or the decrease of $r$, Taylor expansion is applied to Eq.~(\ref{eq:ES2}) then we have,
%
\begin{equation}
\Pi =\Phi\left(\theta\right) \simeq \pi -\frac{\pi}{12}\theta^2 \cdots +\underset{\theta \rightarrow 0}{\longrightarrow} \pi.
\label{eq:ES3}
\end{equation}
%
As Eq.~(\ref{eq:ES3}) shows, $\Phi$ converges to a finite limit $\pi$, then we have a following intermeidate asymptotics as $\Pi = \frac{S}{r^2}$,
%
\begin{equation}
S = \pi r^2~\left(0 < r \ll R\right)
\label{eq:ES4}
\end{equation}
%
as far as the asymptotic condition $\theta \ll 1$, corresponding to $0 < r \ll R$, is satisfied.

Note that the scaling law Eq.~(\ref{eq:ES4}) is valid in the scale range  $(0 < r \ll R)$, in which the circle is significantly smaller than the sphere. Therefore, Eq.~(\ref{eq:ES4}) is an asymptotic expression which is valid in the certain range of variable $r$. This scaling law formalized {\it locally} is an intermediate asymptotic in this problem. Barenblatt insisted that this framework is applicable to the construction of a physical model.

The important point of this concept is that every physical problem has dimension and can be applied to dimensional analysis to obtain dimensionless function $\Phi$. By considering the convergence of $\Phi$, some similarity parameters can be selected to have the idealized solution effectively and practically as the convergence of $\Phi$ can be verified by the experimental or simulational results even if the exact form of $\Phi$ is not obtained. This procedure gives rise to the strategy of Barenblatt as it is formalized in the recipe\cite{Recipe}.

The second important point is that this process, in which one screens the similarity parameters of $\Phi$ depending on their convergence, corresponds to an idealization of problems. More or less, all the physical models involve idealizations such as ignorance of friction force, ignorance of quantum or relativity effect. All these assumptions corresponds to the idealizing process of dimensionless function. For example, the ideal gas equation can be considered as an intermediate asymptotic valid in the range where the volume of molecules $b$ and the molecular interaction $a$ are negligible on van der Waals equation as follows,
\begin{equation}
p=\frac{nRT}{V-nb}-\frac{an^2}{V^2} \longrightarrow \frac{nRT}{V}~~\left(\frac{an^2}{V^2} \ll p \ll \frac{RT}{b} \right).
\label{eq:ES5}
\end{equation}
This idealizing scale range is satisfied as far as $\Pi_a = \frac{a n^2}{pV^2} \ll 1$ and $\Pi_b = \frac{pb}{RT} \ll 1$. The interested readers may refer to Ref.\cite{Oono2013} for further discussion related with phenomenology, renormalization and asymptotic analysis on physics.

This concept suggests that every physical theory is {\it locally} valid. This localization is quantitatively and qualitatively formalized by the intermediate asymptotics. In the present work, the author focuses on this point and considers the case of the transition of this {\it locality}.

\section{Complete similarity and incomplete similarity}

In this section, I briefly explain {\it complete similarity} and {\it incomplete similarity}, as well as the {\it self-similarity of the first kind} and the {\it self-similarity of the second kind} \cite{SecondKind}. They are the category in terms of the convergence of dimensionless function. Zeldovich noted that there exists the type of self-similarity\cite{Zeldovich1956}. As the previous section showed, the self-similar solution is obtained by dimensional analysis. Supposing that a certain physical function, 
%
\begin{equation}
y = f(t,x,z)
\label{eq:ES6}
\end{equation}
%
in which $y$, $t$, $x$ and $z$ are certain physical quantities which have physical dimensions. Selecting $t$ as a governing parameter with independent dimension, which is defined as physical parameters which cannot be represented as a product of the remaining parameters, then we apply dimensional analysis to have
%
\begin{equation}
\Pi = \Phi(\eta, \xi)
\label{eq:ES7}
\end{equation}
%
where $\Pi = y/t^{\alpha}$, $\eta = x/t^{\beta}$ and $\xi = z/t^{\gamma}$. $\alpha$, $\beta$ and $\gamma$ can be fully determined by the consideration of dimension of parameters through dimensional analysis.

If $\Phi$ converges to a finite limit as $\xi$ goes to zero or infinity, this case corresponds to {\it complete similarity} or {\it similarity of the first kind} in the similarity parameter $\xi$. In this case, $\xi$ can be excluded on the consideration and we have an intermediate asymptotics. Once $\eta$ and $\xi$ both satisfy the complete similarity then $\Phi \rightarrow {\rm const}$ as $\eta \gg 1$ and $\xi \gg 1$, then we have a following intermediate asymptotic, $y = {\rm const}~t^{\alpha}~(0 < t \ll x^{1/\beta},~0 < t \ll z^{1/\gamma})$. When the similarity parameters satisfies the condition of complete similarity, $\Pi = \Phi( \xi, \eta )$ is corresponds to a {\it self-similar solution of the first kind}. In the previous section, as Eq.~(\ref{eq:ES3}) shows, the dimensionless function converges to a finite limit, therefore the problem belongs to complete similarity and Eq.~(\ref{eq:ES1}) is a self-similar solution of the first kind.

On the other hand, in the case in which the complete similarity is not satisfied, namely $\Phi$ does not converge to a finite limit as $\eta$ goes to zero or infinity, but the convergence is recovered by constructing new similarity parameters as the power-law monomial using dimensionless variables, this case corresponds to {\it incomplete similarity} or {\it similarity of the second kind}. In this case, we may have the following self-similar solution, which is called {\it self-similar solution of the second kind},
%
\begin{equation}
\Psi = \Phi(Z)
\label{eq:ES8}
\end{equation}
%
where $\Psi = \Pi / \eta^{\zeta}$ and $Z = \xi / \eta^{\epsilon}$. 

The first important point is that the power exponent $\zeta$ and $\epsilon$ cannot be determined by dimensional analysis in the case of the second kind while it is possible in the case of the first kind. We may occasionally determine $\zeta$ and $\epsilon$ by the method for nonlinear eigenvalue problems\cite{NEP} or renormalization group theory\cite{Goldenfeld1992,Goldenfeld} though we may consider them as simply empirical numbers\cite{Barenblatt1981}. It was suggested that self-similarity of the second kind corresponds to {\it fractal}\cite{fractal, Mandelbrot1983}, which was elaborated by Mandelbrot. The fractal is a geometrical object which is lacking in a characteristic length. If the objects possess a certain characteristic length, the scale of the object is apparent by scale transformation. On the other hand, the scale of fractal objects is not apparent but self-similar as the scale transformation. Such a geometrical property corresponds to the divergence of dimensionless function in dimensional analysis. 
 
The second important point is that there exists a hierarchy of self-similarity. Note that we can find a parallelism between the first kind and the second kind. Dimensional analysis transforms $y=f(t,x,z)$ to $y/t^{\alpha} = \Phi(x/t^{\beta}, z/t^{\gamma})$ while $\Pi = \Phi(\eta, \xi)$ is transformed to  $\Pi / \eta^{\zeta} = \Phi( \xi / \eta^{\epsilon} )$ in case of the second kind. In the present study, I focused on this hierarchy though the first kind and the second kind refer to the property of the convergence of dimensionless function, not to the classes to which dimensionless parameters belong. Thus, I introduced a word, {\it class} to characterize the hierarchical structure.

By considering the convergence of the similarity parameters,  the self-similar structure of the problem are explored, and intermediate asymptotics is finally obtained. Depending on the type of similarity, the asymptotics is called {\it intermediate asymptotics of the first kind} or {\it intermediate asymptotics of the second kind}.

\section{The time evolution of deformation}

In this work, the model assumes the main contribution of deformation is due to $\frac{d \delta}{dt} = v_i$, which corresponds to the square deformation. This behavior is well supported by the observation of experiment. Fig.~S2-4 (a) shows the time evolution of deformation for different sizes of spheres, $R$ = 4.0 mm (Fig. S2), $R $ = 6.0 mm (Fig.~S3), $R$ = 8.0 mm (Fig.~S4). After the contact, the rate of deformation corresponds to the impact velocity and the rate of deformation is maintained for a while then it steeply decreases in the end. Fig.~S2-4 (b) shows the comparison of each impact for normalized deformation and time. The deformation is normalized by the maximum deformation $\delta_m$ and the time is normalized by $t_{{\rm max}}$ at which $\delta$ reaches $\delta_m$. One can find that all the attractors overlap almost completely, which signifies the attractors in different impact speeds are similar. This means that the lower the impact-velocity is, the longer the contact time is. This relation is well observed in Fig. ~\ref{fig:Fig2C}, which shows the linear scaling relation between $t_{{\rm max}}$ and $\delta_m /v_i$ in different sizes of spheres. The scaling relation between $t_{\rm max}$ and $v_i$ is strong relation valid in every data in different sizes of spheres whether they belong to the elastic regime or viscoelastic regime. This scaling relation makes the velocity-contact time dependence which generates the crossover of scaling law.
%
\begin{figure}[h!]
\subfigure[The time evolution of deformation for $R = 4.0 ~{\rm mm}$]{
 \includegraphics[width=7.5cm]{SFigure9_HM_8.pdf}
\label{fig:Fig2A}}
\subfigure[The time evolution of normalized deformation for $R = 4.0 ~{\rm mm}$ ]{
  \includegraphics[width=7.5cm]{SFigure12_HM_8.pdf}
\label{fig:Fig2B}}
 \caption{(Color online) (a) The time evolution of deformation $\delta$ and (b) the normalized deformation $\delta/ \delta_m$ for $R = 4.0 ~{\rm mm}$, $v_i = 316~{\rm mm/s}$, $Z = 2.38$ ($\textcolor{blue}{\bullet}$),  $v_i = 549~{\rm mm/s}$, $Z = 1.89$  ($\textcolor{orange}{\bullet}$), $v_i = 692~{\rm mm/s}$, $Z = 1.79$ ($\textcolor{green}{\bullet}$), $v_i =1021 ~{\rm mm/s}$, $Z = 1.46$ ($\textcolor{red}{\bullet}$), $v_i =1510~{\rm mm/s}$, $Z =1.25 $ ($\textcolor{violet}{\bullet}$), $v_i =2238~{\rm mm/s}$, $Z =1.25 $ ($\textcolor{brown}{\bullet}$) and $v_i = 3110~{\rm mm/s}$, $Z =0.98$ ($\textcolor{black}{\bullet}$). $\delta_m$ is maximum deformation and $t_{ {\rm max}}$ is the time at which $\delta$ reaches to $\delta_m$. The vertical dashed line indicates the moment of contact time and the horizontal dashed line indicates the configuration at $\delta = 0$. \label{fig:FS8} }
\end{figure}
%
%
\begin{figure}[h!]
\subfigure[The time evolution of deformation for $R = 6.0 ~{\rm mm}$]{
 \includegraphics[width=7.5cm]{SFigure9_HM_12.pdf}
\label{fig:Fig2A_12}}
\subfigure[The time evolution of normalized deformation for $R = 6.0 ~{\rm mm}$ ]{
  \includegraphics[width=7.5cm]{SFigure12_HM_12.pdf}
\label{fig:Fig2B_12}}
 \caption{(Color online) (a) The time evolution of deformation $\delta$ and (b) the normalized deformation $\delta/ \delta_m$ for $R = 6.0 ~{\rm mm}$, $v_i = 351~{\rm mm/s}$, $Z = 3.24$ ($\textcolor{blue}{\bullet}$),  $v_i = 519~{\rm mm/s}$, $Z = 2.58$  ($\textcolor{orange}{\bullet}$), $v_i = 697~{\rm mm/s}$, $Z = 2.30$ ($\textcolor{green}{\bullet}$), $v_i =1013 ~{\rm mm/s}$, $Z = 1.95$ ($\textcolor{red}{\bullet}$), $v_i =1514 ~{\rm mm/s}$, $Z =1.63 $ ($\textcolor{violet}{\bullet}$), $v_i =2240~{\rm mm/s}$, $Z =1.40$ ($\textcolor{brown}{\bullet}$) and $v_i = 3125~{\rm mm/s}$, $Z =1.20$ ($\textcolor{black}{\bullet}$). $\delta_m$ is maximum deformation and $t_{ {\rm max}}$ is the time at which $\delta$ reaches to $\delta_m$. The vertical dashed line indicates the moment of contact time and the horizontal dashed line indicates the configuration at $\delta = 0$. \label{fig:FS8} }
\end{figure}
%
%
\begin{figure}[h!]
\subfigure[The time evolution of deformation for $R = 8.0 ~{\rm mm}$]{
 \includegraphics[width=7.5cm]{SFigure9_HM.pdf}
\label{fig:Fig2A_16}}
\subfigure[The time evolution of normalized deformation for $R = 8.0 ~{\rm mm}$ ]{
  \includegraphics[width=7.5cm]{SFigure12_HM.pdf}
\label{fig:Fig2B_16}}
 \caption{(Color online) (a) The time evolution of deformation $\delta$ and (b) the normalized deformation $\delta/ \delta_m$ for $R = 8.0 ~{\rm mm}$, $v_i = 372~{\rm mm/s}$, $Z = 3.86$ ($\textcolor{blue}{\bullet}$),  $v_i = 524~{\rm mm/s}$, $Z = 3.34$  ($\textcolor{orange}{\bullet}$), $v_i = 637~{\rm mm/s}$, $Z = 3.07$ ($\textcolor{green}{\bullet}$), $v_i =959 ~{\rm mm/s}$, $Z = 2.55$ ($\textcolor{red}{\bullet}$), $v_i =1492 ~{\rm mm/s}$, $Z =2.04 $ ($\textcolor{violet}{\bullet}$), $v_i =2219~{\rm mm/s}$, $Z =1.68 $ ($\textcolor{brown}{\bullet}$) and $v_i = 3104~{\rm mm/s}$, $Z =1.37$ ($\textcolor{black}{\bullet}$). $\delta_m$ is maximum deformation and $t_{ {\rm max}}$ is the time at which $\delta$ reaches to $\delta_m$. The vertical dashed line indicates the moment of contact time and the horizontal dashed line indicates the configuration at $\delta = 0$. \label{fig:FS8} }
\end{figure}
%

%
\begin{figure}[h!]
\includegraphics[width=8.6cm]{SFigure10_HM.pdf}
\caption{(Color online) The dependence between $t_{ {\rm max}}$ and $\delta_m / v_i$ for each size of sphere.  \label{fig:Fig2C}  }
\end{figure}
%


\section{The convergence of Eq. (6).}

Eq. (6) is seemingly an indeterminate form as $Z \rightarrow 0$ though it converges to a finite limit as follows. Using L'H{\^{o}}pital's rule, then we have
%
\begin{equation}
\lim_{Z \to 0} \frac{2}{3} \frac{Z}{1-e^{-Z}} = \lim_{Z \to 0} \frac{2}{3} \frac{\left(Z \right)^{'}}{\left(1-e^{-Z}\right)^{'}} =  \frac{2}{3} .
\label{eq:ES8a}
\end{equation}
%


\section{The derivation of Eq. (7)}

According to Eq.~(6), in the self-similarity of the first class, we have
%
\begin{equation}
\frac{\Pi^3 \phi}{\kappa \eta} = \frac{2}{3}\frac{ \frac{\Pi}{\theta \eta^{1/2}} }{\left[1 - \exp \left( -\frac{\Pi}{\theta \eta^{1/2}} \right) \right]}.
\label{eq:ES9}
\end{equation}
%
By multiplying $\frac{\kappa \eta}{\Pi^3 \phi}$ and we have the following form from Eq.~(\ref{eq:ES9}) 
%
\begin{equation}
\frac{2}{3} = \Pi^2 \theta \frac{\phi }{\kappa} \frac{1}{\eta^{1/2}}\left[1 - \exp \left( -\frac{\Pi}{\theta \eta^{1/2}} \right) \right].
\label{eq:ES10}
\end{equation}
%
In order to see the actual behavior of $\Pi$, I applied the third term perturbation method. As it belongs to the problem of the singular perturbation\cite{Holmes}, therefore here we assume
%
\begin{equation}
\Pi = \frac{1}{\varepsilon^{\gamma} } \left( \Pi_0 + \varepsilon^{\alpha} \Pi_1 + \varepsilon^{2\alpha} \Pi_2 + \cdots \right)
\label{eq:ES11}
\end{equation}
%
where $\gamma$ and $\alpha$ are constant, $\varepsilon = 1/ \theta \eta^{1/2}$.

By applying the Taylor expansion on the exponential part and substituting Eq.~(\ref{eq:ES11}) into Eq.~(\ref{eq:ES10}), we have
%
\begin{eqnarray}
\theta^2 \frac{\phi }{\kappa} \varepsilon \Pi^2 \left\{ \varepsilon \Pi  - \frac{1}{2}\varepsilon^2 \Pi^2 + \frac{1}{6}\varepsilon^3 \Pi^3 \cdots \right\} =  \frac{2}{3}  \nonumber   \\ 
\Leftrightarrow \theta^2 \frac{\phi }{\kappa} \varepsilon^{1-2 \gamma} \left( \Pi_0^2  + 2 \varepsilon^{\alpha} \Pi_1 \Pi_0 +  2 \varepsilon^{2 \alpha} \Pi_2 \Pi_0 +  \varepsilon^{2 \alpha} \Pi_1^2 + \cdots  \right) \{ \varepsilon^{1-\gamma} \left(\Pi_0 + \varepsilon^{\alpha} \Pi_1 + \varepsilon^{2\alpha} \Pi_2 + \cdots \right) \nonumber  \\  
- \frac{1}{2}\varepsilon^{2-2 \gamma} \left(\Pi_0^2  + 2 \varepsilon^{\alpha} \Pi_1 \Pi_0 +  2 \varepsilon^{2 \alpha} \Pi_2 \Pi_0 +  \varepsilon^{2 \alpha} \Pi_1^2 + \cdots \right) + \frac{1}{6}\varepsilon^{3-3 \gamma} \left(\Pi_0^3 \cdots \right)\cdots \} = \frac{2}{3} 
\label{eq:ES12}
\end{eqnarray}
%
as $\varepsilon \rightarrow 0$.

To balance each terms, we find that $\gamma=2/3$ and $\alpha=1/3$ then we obtain, 
%
\begin{eqnarray}
\theta^2 \frac{\phi }{\kappa}\left( \Pi_0^2  + 2 \varepsilon^{1/3} \Pi_1 \Pi_0 +  2 \varepsilon^{2/3} \Pi_2 \Pi_0 + \varepsilon^{2/3} \Pi_1^2 + \cdots  \right) \{\Pi_0 + \varepsilon^{1/3} \Pi_1 + \varepsilon^{2/3} \Pi_2 + \cdots \nonumber  \\  
- \frac{1}{2}\varepsilon^{1/3} \left(\Pi_0^2  + 2 \varepsilon^{1/3} \Pi_1 \Pi_0 +  2 \varepsilon^{2/3} \Pi_2 \Pi_0 + \varepsilon^{2/3} \Pi_1^2 + \cdots \right) + \frac{1}{6}\varepsilon^{2/3} \left(\Pi_0^3 \cdots \right)\cdots \} = \frac{2}{3}.   \label{eq:ES13}
\end{eqnarray}
%
From this we have
%
\begin{eqnarray}
O\left(1 \right) \Leftrightarrow \theta^2 \frac{\phi }{\kappa} \Pi_0^3 &=& \frac{2}{3} \nonumber \\ 
\Pi_0 &=& \left( \frac{2}{3} \right)^{\frac{1}{3}} \frac{1 }{\theta^{2/3}} \left( \frac{\kappa }{\phi}\right)^{\frac{1}{3}} 
\label{eq:ES14}
\end{eqnarray}
%
\begin{eqnarray}
O\left(\varepsilon^{1/3} \right) \Leftrightarrow 3\Pi_0^2 \Pi_1 - \frac{1}{2}\Pi_0^4 &=& 0 \nonumber \\
\Pi_1 &=& \frac{1}{6} \Pi_0^2 = \frac{1}{ \theta^{4/3}} \left( \frac{\kappa }{\phi} \right)^{\frac{2}{3}} \left(\frac{1}{486} \right)^{\frac{1}{3}}
\label{eq:ES15}
\end{eqnarray}
%
\begin{eqnarray}
O\left(\varepsilon^{2/3} \right) \Leftrightarrow 3\Pi_0^2 \Pi_2 - 2 \Pi_0^3\Pi_1 +\frac{1}{6} \Pi_0^5 +3 \Pi_0\Pi_1^2 &=& 0 \nonumber \\
\Pi_2 &=& \frac{2}{3} \Pi_0 \Pi_1   -\frac{1}{18} \Pi_0^3   - \frac{\Pi_1^2}{\Pi_0}= \frac{1}{54 \theta^2}  \frac{\kappa }{\phi}
\label{eq:ES16}
\end{eqnarray}
%
Using results of Eq.~(\ref{eq:ES14}), Eq.~(\ref{eq:ES15}), Eq.~(\ref{eq:ES16}), $\gamma = 2/3$ and $\alpha = 1/3$ for Eq.~(\ref{eq:ES11}) then we have a following result,
%%
\begin{equation}
\Pi = \frac{ \kappa}{54 \phi \theta^2 } + \left(\frac{\kappa^2 }{486 \phi^2 \theta^{3} }\right)^{\frac{1}{3}} \eta^{\frac{1}{6}} + \left( \frac{2 \kappa}{3 \phi} \right)^{\frac{1}{3}}\eta^{\frac{1}{3}}
\label{eq:ES17}
\end{equation}
%%
which corresponds to the Eq.~(7).

\section{Self-similarity of the first class and the second class for the data plots using the surface coated with grease}

In this paper, to focus on the viscoelastic behavior of the bulk and to eliminate the adhesion effect on the surface of PDMS board, chalk powder was dusted on the surface. Coating the surface with grease is expected to be the same effect though the plots are comparatively scattered particularly for the larger spheres. This may be due to the surface tension and that the surface interaction is not completely eliminated.
%
\begin{figure*}[t]
\begin{center}
\includegraphics[width=15cm]{FigureS2_HM.pdf}
\caption{(Color online) The different hierarchical structures of self-similarity for the experiments in which the PDMS surface is coated with grease (WD-40). (a) - (f) Self-similarity of the first class: the power law relations $\Pi$ and $\eta$ for spheres of different sizes. The dashed lines indicate the slope of 1/6, the solid lines indicate the slope of 1/3 and the colored dot-dashed line indicates Eq.~(7) for each size of the spheres. (g) Self-similarity of the second class: the plots between $\Psi$ and $Z$. $R = {\rm 3.0~mm}$ ($\textcolor{blue}{\bullet}$), ${\rm 4.0~mm}$ ($\textcolor{green}{\blacktriangle}$), {\rm 5.0~mm} ($\times$), {\rm 6.0~mm} ($\textcolor{orange}{\blacklozenge}$), {\rm 7.0~mm} ($\textcolor{red}{\blacksquare}$)  and {\rm 8.0~mm} ($\textcolor[rgb]{0.7, 0.4, 0.9}{\blacktriangledown}$) where $\Pi=\delta_m/R$, $\eta=\rho v_i^2/E$, $\Psi=\frac{\Pi^3 \phi}{\kappa \eta} = \frac{ \delta_m^3 E \phi}{ R^{2} h \rho v_i^2} $ and $Z = \frac{ \Pi}{ \theta \eta^{1/2}} = \frac{ E \delta_m }{ \mu  v_i}$. The red line in Figure (g) is Eq.~(6). The dashed line roughly indicates the line separating the region.}
\label{fig:FSS2}
\end{center}
\end{figure*} 
% 


\section{The perturbation of the Kelvin-Voigt Viscoelastic Foundation model}

Here we show the behavior of dynamical impact for the case in which the Kelvin-Voigt model is applied for the each unit of foundations instead of the Maxwell model. In the Kelvin-Voigt model, the relation between the stress and deformation is described as $\sigma = E \epsilon + \mu \frac{d \epsilon}{dt} $. As it was assumed in the manuscript $\frac{d \delta}{dt} = v_i$, the exchange of energy for the maximum deformation is described as follows,
%%
\begin{equation}
\frac{2}{3} \pi R^3 \rho v_i^2   = \frac{ \pi E \phi R \delta_m^3}{3 h } +  \frac{ \pi \mu \phi R \delta_m^2}{ h }\frac{d \delta}{dt}.
\label{eq:ES18}
\end{equation}
%%
Introducing the dimensionless numbers in Eq. (4), we have the following dimensionless form from Eq. (\ref{eq:ES18}) 
%%
\begin{equation}
\frac{2}{3}  = \frac{1}{3} \frac{\Pi^3 \phi}{\kappa \eta} +  \frac{\Pi^2 \theta \phi}{\kappa \eta^{1/2}}. 
\label{eq:ES19}
\end{equation}
%%

In the same way as MVF model, I apply the perturbation of Eq.~(\ref{eq:ES11}) as follows,
%
\begin{equation}
\frac{1}{3}\theta^2 \frac{\phi }{\kappa} \varepsilon^{2-3\gamma}  \left( \Pi_0^3  + 3 \varepsilon^{2\alpha} \Pi_0^2 \Pi_1 + \cdots \right)    + \theta^2 \frac{\phi }{\kappa} \varepsilon^{1-2\gamma}  \left( \Pi_0^2  + 2 \varepsilon^{\alpha} \Pi_0 \Pi_1 + \cdots \right) = \frac{2}{3}
\label{eq:ES20}
\end{equation}
%
as $\varepsilon \rightarrow 0$.

Considering the balance, there is two possibility: $\gamma = \frac{2}{3}$ or  $\gamma=\frac{1}{2}$. However, $\gamma = \frac{2}{3}$ is impossible as $O\left(\varepsilon^{- \frac{1}{3}}\right)$ appears and it is not higher order of $O\left(1\right)$ while $\gamma = \frac{1}{2}$ is possible and it is well ordered. Therefore, we have $\gamma = \frac{1}{2}$, $\alpha = \frac{1}{2}$, then the solution of the perturbation is  as follows,
%%
\begin{equation}
\Pi = \sqrt{\frac{2\kappa}{3\phi \theta} } \eta^{\frac{1}{4}}  - \frac{\kappa}{9 \phi\theta^2}.
\label{eq:ES21}
\end{equation}
%%

It reveals 1/4 power-law on $\eta$. This solution is not consistent with our experimental observation and demonstrates that the crossover of scaling law does not occur. However, this solution is inconsistent with many points. Therefore, if the material obeys the Kelvin-Voigt model, it is possible that one cannot expect $\frac{d \delta}{dt} = v_i$. 

The self-similar solution by the variables of the second class will be as follows,
%%
\begin{equation}
\Psi =\frac{2}{3}\frac{Z}{1+Z}
\label{eq:ES22}
\end{equation}
%%
where $\Psi = \frac{\phi \Pi^3}{\kappa \eta}$ and $Z = \frac{\Pi}{\theta \eta^{1/2}}$. The elastic regime is recovered when $Z \rightarrow \infty$ though it cannot be achieved by decreasing velocity, $\eta \rightarrow 0$ as the perturbation result shows. It can be achieved when $\theta \rightarrow 0$. Therefore, the Kelvin-Voigt model cannot reveal crossover of scaling law.

\section{The perturbation of the Zener Viscoelastic Foundation model}

In this section, I show the result of the perturbation for the Zener Viscoelastic foundation model. The Zener model is a hybrid of the Maxwell model and the Kelvin-Voigt model. The unit of each foundation is described by 
%%
\begin{equation}
\frac{\mu}{E}\frac{d \sigma}{dt} = -\sigma + E_K \epsilon + \mu \left(  \frac{E_K}{E}+1\right) \frac{d \epsilon}{dt}
\label{eq:ES23}
\end{equation}
%%
where $E_K$ is the spring unit parallelly connected with its Maxwell element composed of a serial connection of a dashpot $\mu$ and a spring $E$.  Assuming $\frac{d \delta}{dt} = v_i$, the energy exchange will be as follows,
%%
\begin{equation}
\frac{2}{3} \pi R^3 \rho v_i^2   = \frac{ \pi E_K \phi R \delta_m^3}{3 h } +  \frac{ \pi \mu \phi R \delta_m^2}{ h }\frac{d \delta}{dt}\left[ 1-\exp \left(- \frac{E t_c}{\mu}\right) \right] .
\label{eq:ES24}
\end{equation}
%%
By applying dimensional analysis based on Eq. (4) and using $t_c = \frac{\delta_m}{v_i}$, dimensionless form will be 
%%
\begin{equation}
\frac{2}{3}  = \frac{ \phi \Pi^3 }{3 \nu \kappa \eta } + \Pi^2 \theta \frac{\phi }{\kappa} \frac{1}{\eta^{1/2}}\left[1 - \exp \left( -\frac{\Pi}{\theta \eta^{1/2}} \right) \right]
\label{eq:ES25}
\end{equation}
%%
where $\nu = \frac{E}{E_K}$.  Applying the perturbation of Eq.~(\ref{eq:ES11}) and Taylor expansion to the exponential part, the result will be as follows,
%
\begin{eqnarray}
\theta^2 \frac{\phi }{\kappa} \left\{  \frac{1}{3 \nu} \varepsilon^2 \Pi^3 +   \varepsilon \Pi^2\left( \varepsilon \Pi  - \frac{1}{2}\varepsilon^2 \Pi^2 + \frac{1}{6}\varepsilon^3 \Pi^3 \cdots  \right) \right\} =  \frac{2}{3}  \nonumber   \\ 
\Leftrightarrow \theta^2 \frac{\phi }{\kappa} [ \frac{1}{3\nu} \varepsilon^{2-3\gamma} \left\{ \Pi_0^3 +3\varepsilon^{\alpha}\Pi_0^2 \Pi_1 + \left( 3 \Pi_0^2 \Pi_2 + 3 \Pi_0 \Pi_1^2 \right) \varepsilon^{2 \alpha}   + \cdots  \right\} +   \nonumber  \\  
\varepsilon^{1-2 \gamma} \left( \Pi_0^2  + 2 \varepsilon^{\alpha} \Pi_1 \Pi_0 +  2 \varepsilon^{2 \alpha} \Pi_2 \Pi_0 +  \varepsilon^{2 \alpha} \Pi_1^2 + \cdots  \right) \{ \varepsilon^{1-\gamma} \left(\Pi_0 + \varepsilon^{\alpha} \Pi_1 + \varepsilon^{2\alpha} \Pi_2 + \cdots \right)   \nonumber  \\  
- \frac{1}{2}\varepsilon^{2-2 \gamma} \left(\Pi_0^2  + 2 \varepsilon^{\alpha} \Pi_1 \Pi_0 +  2 \varepsilon^{2 \alpha} \Pi_2 \Pi_0 +  \varepsilon^{2 \alpha} \Pi_1^2 + \cdots \right) + \frac{1}{6}\varepsilon^{3-3 \gamma} \left(\Pi_0^3 \cdots \right)\cdots \} ] = \frac{2}{3} 
\label{eq:ES26}
\end{eqnarray}
%
as $\varepsilon \rightarrow 0$.

To balance each terms, $\gamma = \frac{2}{3}$, $\alpha = \frac{1}{3}$. By considering the orders $O\left(1\right)$, $O\left(\varepsilon^{\frac{1}{3}} \right)$ and $O\left(\varepsilon^{\frac{2}{3}} \right)$ to determine $\Pi_0$, $\Pi_1$ and $\Pi_2$, we have the following perturbation result,
%%
\begin{equation}
\Pi = \left(3K^3-2K^2\right) \frac{ \kappa}{54 \phi \theta^2 } + K^{\frac{5}{3}}\left(\frac{\kappa^2 }{486 \phi^2 \theta^{3} }\right)^{\frac{1}{3}} \eta^{\frac{1}{6}} +K^{\frac{1}{3}} \left( \frac{2 \kappa}{3 \phi} \right)^{\frac{1}{3}}\eta^{\frac{1}{3}}
\label{eq:ES27}
\end{equation}
%%
where $K = \frac{3 \nu}{3\nu +1}$. 

As $E \gg E_K $ recovers the Maxwell model, it corresponds to $\nu \gg 1$ and $K \rightarrow 1$. In this case, we have Eq.~(\ref{eq:ES17}), which is the same result of the MVF model.

The self-similar solution is, using $\Psi = \frac{\phi \Pi^3}{\kappa \eta}$ and $Z = \frac{\Pi}{\theta \eta^{1/2}}$
%%
\begin{equation}
\Psi =\frac{2 \nu Z}{Z + 3 \nu \left[1- \exp \left( -Z \right)\right]}.
\label{eq:ES28}
\end{equation}
%%
Eq. (\ref{eq:ES28}) corresponds to Eq. (6) by $\nu \rightarrow \infty$, which is the limit of Maxwell. As Eq. (\ref{eq:ES27}) shows, the Zener Viscoelastic Foundation model reveals the crossover of scaling law depending on the impact-velocity. The only difference from MVF model is just the coefficient of  $K = \frac{3 \nu}{3\nu +1}$. However, there is no perturbation result for the limit of the Kelvin-Voigt, which is realized by  $E \ll E_K $, as it gives $K \rightarrow 0$ then the solution vanishes. This consideration suggests that the Maxwell element, the serial connection of spring and dashpot, is essential to realize the crossover of scaling law.

\begin{thebibliography}{2}
\expandafter\ifx\csname natexlab\endcsname\relax\def\natexlab#1{#1}\fi
\expandafter\ifx\csname bibnamefont\endcsname\relax
  \def\bibnamefont#1{#1}\fi
\expandafter\ifx\csname bibfnamefont\endcsname\relax
  \def\bibfnamefont#1{#1}\fi
\expandafter\ifx\csname citenamefont\endcsname\relax
  \def\citenamefont#1{#1}\fi
\expandafter\ifx\csname url\endcsname\relax
  \def\url#1{\texttt{#1}}\fi
\expandafter\ifx\csname urlprefix\endcsname\relax\def\urlprefix{URL }\fi
\providecommand{\bibinfo}[2]{#2}
\providecommand{\eprint}[2][]{\url{#2}}

\bibitem{Barenblatt2003}
G. I. Barenblatt, {\it Scaling} (Cambridge University Press, 2003) pp.60-65. 
\bibitem{Barenblatt1996}
G. I. Barenblatt, {\it Scaling, self-similarity, and intermediate asymptotics} (Cambdrige University Press, 1996) pp.86-94.
\bibitem{Barenblatt2014}
G. I. Barenblatt, {\it Flow, Deformation and Fracture} (Cambridge University Press, 2014).
\bibitem{Barenblatt1972}
G. I. Barenblatt and Ya. B. Zeldovich, Self-similar solutions as intermediate asymptotics, Ann. Rev. Fluid Mech. {\bf 4}, 285 (1972).
\bibitem{Recipe}
See Refs. \cite{Barenblatt2003} (pp. 91-93), \cite{Barenblatt1996} (pp.159-160).
\bibitem{Oono2013}
Y. Oono, {\it The Nonlinear World} (Springer, 2013) Ch.3.
\bibitem{SecondKind}
See Refs. \cite{Barenblatt2003} (pp.82-91), \cite{Barenblatt1996} (pp.151-159) and \cite{Barenblatt2014} (pp. 153-163). 
\bibitem{Zeldovich1956}
Ya. B. Zeldovich,  The motion of a gas under the action of short term pressure shock. Sov. Phys. Acoustics, {\bf 2}, 25 (1956).
\bibitem{NEP}
See Refs. \cite{Barenblatt2003} Ch.3 and \cite{Barenblatt1996} Ch.3, 4.
\bibitem{Goldenfeld1992}
N. Goldenfeld, {\it Lecture On Phase Transitions And The Renormalization Group} (Addison-Wesley Publishing Company, 1992) Ch.10.
\bibitem{Goldenfeld}
N. Goldenfeld, O. Martin and Y. Oono, Intermediate asymptotics and renormalization group theory, J. Sci. Comput. {\bf 4}, 355 (1989).
\bibitem{Barenblatt1981}
G. I. Barenblatt and L. R. Botvina, Incomplete similarity of fatigue in a linear range of crack growth, Fatigue Eng. Mater. Struct. {\bf 3}, 193 (1981).
\bibitem{fractal}
See Refs. \cite{Barenblatt1996} Ch. 12, \cite{Barenblatt2003} Ch. 7, \cite{Barenblatt2014} Ch. 9.
\bibitem{Mandelbrot1983}
B. B. Mandelbrot, {\it The fractal geometry of nature} (Macmillan, 1983).
\bibitem{Holmes}
M. H. Holmes, {\it Introduction to Perturbation Methods} (Springer 2nd ed., 2013) pp.22-27.

\end{thebibliography}
%\bibliography{dwrreview}

\end{document}
