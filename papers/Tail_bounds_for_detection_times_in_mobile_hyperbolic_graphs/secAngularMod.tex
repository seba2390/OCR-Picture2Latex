In this section we consider angular movement only.
As it will become evident, restricting the motion to angular
movement alone simplifies the analysis considerably, 
compared to the general case, and in fact also compared
to restricting movement radially.
The simpler setting considered in this section is ideal for
illustrating our general strategy for
deriving tail bounds on detection times.
It is also handy for introducing some notation and approximations we shall 
use throughout the rest of the paper.
In later sections we will follow the same general strategy
argument but encounter technically more challenging obstacles.
In contrast, the calculations involved in the study of
the angular movement only case can be mostly reduced to
ones involving known facts about one dimensional Brownian motion.


\begin{figure}
  \begin{center}
    \begin{tikzpicture}[x=1cm,y=1cm,scale=0.5,
     decoration={
       markings,
       mark=at position 1 with {\arrow[scale=1.5,black]{latex}};
     }
   ]
      \def\c{12}
      \def\barr{4}
      \def\rrho{6}
      \def\radp{6}
      \def\phip{150}
      \def\phi0{60}
      \node[inner sep=0] (O) at (\c,\c) {};
      \node[inner sep=0] (P) at (2,4) {};
      \node[inner sep=0] (Q) at (\c,0) {};
      
      %\draw[gray!30,fill=gray!30] (O) circle (\barr);

      \draw[fill=black] (O) circle (0.035);

      \node[above] at (O) {$O$};
      %\draw[fill=black] (O) ++(\phip-360:\radp) circle (0.035);
      \draw[fill=black] (O) ++(\phi0-360:\radp) circle (0.07) node[above] {$x_0$};

      %\draw[postaction={decorate}] (O) -- node[midway,above] {$\widehat{r}$} ++(20:\barr);
      \draw[fill=gray!20] (11.929cm,0.000cm) -- (11.909cm,0.480cm) -- (11.884cm,0.961cm) -- (11.849cm,1.441cm) -- (11.803cm,1.922cm) -- (11.740cm,2.404cm) -- (11.653cm,2.887cm) -- (11.531cm,3.373cm) -- (11.359cm,3.865cm) -- (11.115cm,4.371cm) -- (10.769cm,4.906cm) -- (10.281cm,5.504cm) -- (9.604cm,6.239cm) -- (8.717cm,7.267cm) -- (8.385cm,7.723cm) -- (8.218cm,7.979cm) -- (8.051cm,8.258cm) -- (7.888cm,8.562cm) -- (7.732cm,8.892cm) -- (7.586cm,9.251cm) -- (7.456cm,9.640cm) -- (7.347cm,10.062cm) -- (7.267cm,10.518cm) -- (7.222cm,11.006cm) -- (7.223cm,11.527cm) -- (7.244cm,11.798cm) -- (7.281cm,12.076cm) -- (7.334cm,12.360cm) -- (7.406cm,12.648cm) -- (7.497cm,12.941cm) -- (7.610cm,13.235cm) -- (7.747cm,13.530cm) -- (7.908cm,13.823cm) -- (8.095cm,14.113cm) -- (8.309cm,14.396cm) -- (8.552cm,14.669cm) -- (8.825cm,14.929cm) -- (9.127cm,15.172cm) -- (9.459cm,15.394cm) -- (9.820cm,15.590cm) -- (10.209cm,15.755cm) -- (10.625cm,15.884cm) -- (11.065cm,15.971cm) -- (11.525cm,16.012cm) -- (12.000cm,16.015cm) -- (12.475cm,16.012cm) -- (12.935cm,15.971cm) -- (13.375cm,15.884cm) -- (13.791cm,15.755cm) -- (14.180cm,15.590cm) -- (14.541cm,15.394cm) -- (14.873cm,15.172cm) -- (15.175cm,14.929cm) -- (15.448cm,14.669cm) -- (15.691cm,14.396cm) -- (15.905cm,14.113cm) -- (16.092cm,13.823cm) -- (16.253cm,13.530cm) -- (16.390cm,13.235cm) -- (16.503cm,12.941cm) -- (16.594cm,12.648cm) -- (16.666cm,12.360cm) -- (16.719cm,12.076cm) -- (16.756cm,11.798cm) -- (16.777cm,11.527cm) -- (16.778cm,11.006cm) -- (16.733cm,10.518cm) -- (16.653cm,10.062cm) -- (16.544cm,9.640cm) -- (16.414cm,9.251cm) -- (16.268cm,8.892cm) -- (16.112cm,8.562cm) -- (15.949cm,8.258cm) -- (15.782cm,7.979cm) -- (15.615cm,7.723cm) -- (15.283cm,7.267cm) -- (14.396cm,6.239cm) -- (13.719cm,5.504cm) -- (13.231cm,4.906cm) -- (12.885cm,4.371cm) -- (12.641cm,3.865cm) -- (12.469cm,3.373cm) -- (12.347cm,2.887cm) -- (12.260cm,2.404cm) -- (12.197cm,1.922cm) -- (12.151cm,1.441cm) -- (12.116cm,0.961cm) -- (12.091cm,0.480cm) -- (12.071cm,0.000cm) -- (12.060cm,0.000cm) -- (12.045cm,0.000cm) -- (12.030cm,0.000cm) -- (12.015cm,0.000cm) -- (12.000cm,0.000cm) -- (11.985cm,0.000cm) -- (11.970cm,0.000cm) -- (11.955cm,0.000cm) -- (11.940cm,0.000cm) -- cycle;

      \draw[fill=black] (O) ++(\phip:\radp) circle (0.07);
      \draw[postaction={decorate}] (O) -- node[midway,above] {$\ \ _{r_{\ss}=r_0}$} ++(\phip:\radp) node[left] {$x_{\ss}$};
      \draw[dash dot] (O) circle (\rrho);
            
      \draw[fill=gray!40] (11.941cm,0.000cm) -- (11.865cm,2.001cm) -- (11.707cm,4.005cm) -- (11.404cm,6.030cm) -- (10.937cm,8.144cm) -- (10.842cm,8.591cm) -- (10.798cm,8.820cm) -- (10.758cm,9.051cm) -- (10.725cm,9.285cm) -- (10.698cm,9.521cm) -- (10.681cm,9.760cm) -- (10.675cm,9.999cm) -- (10.681cm,10.239cm) -- (10.704cm,10.477cm) -- (10.744cm,10.711cm) -- (10.804cm,10.938cm) -- (10.885cm,11.154cm) -- (10.988cm,11.356cm) -- (11.113cm,11.538cm) -- (11.260cm,11.696cm) -- (11.426cm,11.825cm) -- (11.608cm,11.921cm) -- (11.801cm,11.980cm) -- (12.000cm,12.000cm) --(12.199cm,11.980cm) -- (12.392cm,11.921cm) -- (12.574cm,11.825cm) -- (12.740cm,11.696cm) -- (12.887cm,11.538cm) -- (13.012cm,11.356cm) -- (13.115cm,11.154cm) -- (13.196cm,10.938cm) -- (13.256cm,10.711cm) -- (13.296cm,10.477cm) -- (13.319cm,10.239cm) -- (13.325cm,9.999cm) -- (13.319cm,9.760cm) -- (13.302cm,9.521cm) -- (13.275cm,9.285cm) -- (13.242cm,9.051cm) -- (13.202cm,8.820cm) -- (13.158cm,8.591cm) -- (13.063cm,8.144cm) -- (12.596cm,6.030cm) -- (12.293cm,4.005cm) -- (12.135cm,2.001cm) -- (12.059cm,0.000cm) -- (12.059cm,0.000cm) -- (12.045cm,0.000cm) -- (12.030cm,0.000cm) -- (12.015cm,0.000cm) -- (12.000cm,0.000cm) -- (11.985cm,0.000cm) -- (11.970cm,0.000cm) -- (11.955cm,0.000cm) -- (11.941cm,0.000cm) -- cycle;
      \draw[dotted] (O) -- (Q);
      \draw[dashed] (O) circle (4.03);
      \draw[postaction={decorate}] (O) -- node[midway,above] {$_{\widehat{r}}$} ++(0:4.03);
      \draw[postaction={decorate}] (\c,\c-2) arc (-90:\phip-360:2);
      \node at (\c-2.2,\c-1) {$_{\theta_{\ss}}$};
      \draw[thick] (O) circle (\c);
      \draw[fill=black] (Q) circle (0.07);
      \draw[fill=black] (O) circle (0.07);
      \node[below] at (Q) {$Q$};      
      \node[above] at (O) {$O$}; 
    \end{tikzpicture}
  \end{center}
  \caption{The position $x_{\ss}=(r_{\ss},\theta_{\ss})$ of particle $P$ at time
    $\ss$ is shown. The lightly shaded area corresponds to
    $\dt(\kappa)\setminus B_{Q}(R)$ and the strongly shaded
    region represents $B_Q(R)$. The dashed line corresponds to the arc
    segment where a particle $P$ at initial radial distance $r_0$ is,
    conditional under not having detected the target~$Q$ up to time
    $\ss$. The depicted dash-dot arc segment has length $C_{r_0}$.
    Each of the two dash-dot arc lengths inside the lightly shaded 
    region have (hyperbolic) length $\kappa\sqrt{\ss}$ corresponding to an angle $\theta_0$ such that
    $\phi(r_0)\leq |\theta_0| \le \phi(r_0)+\kappa \phi^{(\ss)}$.
    The dashed circle is the boundary of the largest ball centered at the origin which is contained in $\dt(\kappa)$.}\label{fig:angular}
\end{figure}

Following the proof strategy described in Section~\ref{sec:strategy}, we first define $\dt(\kappa)$. 
Recall that, roughly speaking, $\dt(\kappa)$ is the set of starting points from which a particle has a "good" chance to detect the target by time $\ss$, where "good" depends on a parameter $\kappa > 1$ (independent of $n$).
%Let $\ell_r$ denote the length of the perimeter of the ball $B_O(r)$ normalized by $2\pi$, that is, $\ell_{r}=\sinh(\beta r)$. 
Let $\phi^{(\ss)}$ be roughly proportional to the angular distance travelled by a particle at radial coordinate $r_0$ up to time $\ss$, more precisely, let $\phi^{(\ss)}:=\sqrt{\ss}e^{-\beta r_0}$ (that is,~$\phi^{(\ss)}$ corresponds to the standard deviation of a standard Brownian motion during an interval of time of length $\ss$ normalized by a term which is proportional to the perimeter of $B_O(r_0)$ for~$r_0$ bounded away from $0$). We can then define the following region (the shaded region in Figure~\ref{fig:angular}): %\cmk{About putting $\sqrt{s}/e^{\beta r}$ I kind of recalled that it was not a good idea because somewhere we then needed to approximate $\ell_{r}$ and added some asymptotic terms. In fact, I think that we might actually consider replacing the $\sqrt{\ss}/e^{\beta R}$ later by $\sqrt{\ss}/\ell_R$. Lets see.}\dmc{I saw that you changed a few, and I agree with them. But then it should probably also changed in the proof of the Lemma~\ref{lem:angular-muDt}, and then also in the statement, and also in the statement of the main theorem of this section. I first need to check whether indeed we need this approximation.}
\[
\dt = \dt(\kappa) := \big\{ x_0\in B_O(R) : |\theta_0| \le \phi(r_0)+\kappa \phi^{(\ss)}\big\}.
\]

In words, $\dt(\kappa)$ is the collection of points $x_0=(r_0,\theta_0)\in B_O(R)$ which are either contained in $B_Q(R)$ or form an angle at the origin of at most $\kappa \phi^{(\ss)}$ with a point that belongs to the boundary of $B_Q(R)$ and has radial coordinate exactly $r_0$.
Since $x_0\in B_{Q}(R)$ if and only if~$|\theta_0|\leq\phi(r_0)$, it is clear that $B_Q(R)$
is contained in $\dt(\kappa)$. The value $\kappa$ is a parameter that tells us at which scaling up to time $\ss$ we consider our region $\dt(\kappa)$. 

%\cmk{Changed statement below to make presentation consistent with strategy of proof section.}
The main goal of this section is to prove the result stated next
which is an instance of Theorem~\ref{thm:intro-Dt} for the case of angular movement only.
Thus, Theorem~\ref{thm:angularMain} will immediately follow  (by the proof strategy discussed in Section~\ref{sec:strategy}) once we show that~$\mu(\dt(\kappa))$ is of the right order:
\begin{theorem}\label{thm:mainangular}
Denote by $\Phi$ the standard normal distribution.
If $\kappa>0$ and 
  $\kappa\sqrt{\ss}\leq\frac{\pi}{2}(1-o(1))e^{\beta R}$, then
\[
\inf_{x_0\in\dt(\kappa)}\P_{x_0}(T_{det}\leq \ss)=\Omega(\Phi(-\kappa))
\qquad\text{and}\qquad
\sup_{x_0\in\ndt(\kappa)}\P_{x_0}(T_{det}\leq \ss)  = O(\Phi(-\kappa)).
\]
Furthermore,
\[
\int_{\ndt\!(\kappa)} \PP_{x_0}(T_{det}\le\ss)d\mu(x_0) = O(\mu(\dt(\kappa))).
\]
\end{theorem}
%\begin{theorem}\label{thm:mainangular}
%Let $\ss:=\ss(n)$. 
%If $\sqrt{\ss}\leq\frac{\pi}{2}(1-o(1))e^{\beta R}$, then  
%\[
%\P(T_{det}\geq \ss) = 
%\begin{cases}
%\exp\Big({-}\Theta\Big(\mkor{\nu}{1}+n\Big(\frac{\sqrt{\ss}}{e^{\beta R}}\Big)^{1\wedge \frac{\alpha}{\beta}}\Big)\Big), & \text{if $\alpha\neq\beta$,} \\[8pt]
%\exp\Big({-}\Theta\Big(\mkor{\nu}{1}+n\Big(\frac{\sqrt{\ss}}{e^{\beta R}}\Big)^{1\wedge \frac{\alpha}{\beta}}\log\big(\frac{e^{\beta R}}{1+\sqrt{\ss}}\big)\Big)\Big), & \text{if $\alpha=\beta$.}
%\end{cases}
%\]
%Furthermore, denote by $\Phi$ the standard normal distribution.
%If $\kappa>0$ and  $\kappa\sqrt{\ss}\leq\frac{\pi}{2}(1-o(1))e^{\beta R}$, then
%\[
%\inf_{x_0\in\dt(\kappa)}\P_{x_0}(T_{det}\leq \ss)=\Omega(\Phi(-\kappa))
%\qquad\text{and}\qquad
%\sup_{x_0\in\ndt(\kappa)}\P_{x_0}(T_{det}\leq \ss)  = O(\Phi(-\kappa)).
%\]
%\end{theorem}
%\cml{I think this is no longer true} In order to prove the previous result we will in fact separately establish even more detailed lower and upper bounds as the ones stated above. Theorem~\ref{thm:mainangular} will follow immediately from combining the mentioned bounds.

Before proving the previous result, we make a few observations and introduce some definitions. 
First, note that the perimeter of $B_{O}(r)$ that is outside $B_Q(R)$ has length (see Figure~\ref{fig:angular}) 
\[
C_{r}:=2(\pi-\phi(r))\sinh(\beta r).
\]
%Since, $\phi(r)\leq\phi(0)=\frac{\pi}{2}$ (by 
%Parts~\eqref{itm:phi2} and~\eqref{itm:phi3} of Lemma~\ref{lem:phi} ),
%\[\label{eqn:arcLengthBound}
%\tfrac12\pi e^{\beta r}(1-e^{-2\beta r})\leq C_r\leq \pi e^{\beta r}.
%\]


Since $\phi(\cdot)$ is decreasing and continuous 
(by Part~\eqref{itm:phi2} of Lemma~\ref{lem:phi}), 
we obtain the following:
\begin{fact}\label{fct:angular-monot}
The mapping $r\mapsto \frac12 C_r=(\pi-\phi(r))\sinh(\beta r)$ is increasing
and continuous for arguments in $[0,R]$ and takes values in $[0,\frac12 C_R]$.
\end{fact}
In particular, for  $\kappa\sqrt{\ss}\in [0,\frac12C_R]$ there is a unique value 
$\widehat{r}$ such that
\begin{equation}\label{eqn:angular-defrhat}
  \kappa\sqrt{\ss} = \tfrac12 C_{\widehat{r}}
    = (\pi-\phi(\widehat{r}))\sinh(\beta\widehat{r}).
\end{equation}
One may think of $\widehat{r}$ as chosen so that up to time $\kappa^2\ss$ a point at distance $\widehat{r}$ from the origin has a reasonable chance to detect the target due to their angular movement. Using Fact~\ref{fct:angular-monot}, we immediately obtain the following:
\begin{fact}\label{fct:angular-inclus}
For $r\in [0,R]$, the following holds:
$r\geq\widehat{r}$ if and only if $\kappa\sqrt{\ss}\leq\frac12 C_r$.
In particular, $B_O(\widehat{r})$ is the largest ball centered at the origin
contained in $\dt(\kappa)$.
\end{fact}
(See Figure~\ref{fig:angular} for an illustration of $B_O(\widehat{r})$.)
\begin{fact}\label{fct:angular-aproxhat} 
If $\widehat{r}>1$, then $\widehat{r}=\frac{1}{\beta}\log(\kappa\sqrt{\ss})+\Theta(1)$.
Moreover, if $\widehat{r}\leq 1$, then $\kappa\sqrt{\ss}=O(1)$.
\end{fact}
%Observe that $\kappa\sqrt{\ss}=\Omega(1)$ if and only if $\widehat{r}=\Omega(1)$ \cmk{Pending: Justify} and in this case $\kappa\sqrt{\ss}=\Theta(e^{\beta\widehat{r}})$. 
%Note also that all constants hidden inside the asymptotic notation are independent of $c$ (the approximation stemming from 
%the right hand side of~\eqref{eqn:sDef}). 

%Also, note that when bounding integrals over $\ndt$, for a given $r$, we may in fact assume that $\kappa\sqrt{\ss}\in [0,\frac12 C_r]$: indeed, if $\kappa\sqrt{\ss}> \frac12 C_r$, by definition of $\dt$, all of the boundary of $B_O(r)$ would be inside $\dt$, and hence when integrating over $\ndt$ for a given value of $r$, such a large value of $\ss$ would yield the empty set.
%\cmk{I could only find one place where the fact stated here is used. If so, rather than leaving this paragraph here, I would put it where its claim is used.}

%\subsection*{Angular motion: Lower bound.}
%
We will need the following claim whose proof is simple although the calculations involved are a bit tedious, mostly consisting in 
computing integrals and case analysis. 
\begin{lemma}\label{lem:angular-muDt}
If $\kappa>0$ and $\kappa\sqrt{\ss}\leq \frac{\pi}{2}(1-o(1))e^{\beta R}$, then
\[
\mu(\dt(\kappa))=
\begin{cases}
\Theta\Big(\mkor{\nu}{1}+n\Big(\frac{\kappa\sqrt{\ss}}{e^{\beta R}}\Big)^{1\wedge \frac{\alpha}{\beta}}\Big),
& \text{if $\alpha\neq\beta$,} \\[8pt]
\Theta\Big(\mkor{\nu}{1}+n\frac{\kappa\sqrt{\ss}}{e^{\beta R}}\Big(1+\log\big(\frac{e^{\beta R}}{\kappa\sqrt{\ss}}\big)\Big)\Big),
& \text{if $\alpha=\beta$.}
\end{cases}
\]
%\dmc{again would prefer to put the direct value}
%For the case where $\alpha=\beta$, the last identity holds with an additional $\log(\nnu^2/(\kappa\sqrt{\ss})^{\frac{1}{\beta}})$ factor 
%multiplying the second term inside the asymptotic notation.
\end{lemma}
\begin{proof}
If $\widehat{r}> R-\Theta(1)$, by~\eqref{eqn:angular-defrhat} and Part~\eqref{itm:phi3} of Lemma~\ref{lem:phi}, we see that $\kappa\sqrt{\ss}=\Theta(e^{\beta R})$ which always gives an expression of order $\Theta(n)$ on the right-hand side of the equality in the lemma's statement. On the other hand, by Fact~\ref{fct:angular-inclus}, we know that $B_O(\widehat{r})\subseteq\dt(\kappa)\subseteq B_O(R)$, so
from Lemma~\ref{lem:muBall} we get that $\mu(\dt(\kappa))=\Theta(\mu(B_O(\widehat{r})))=\Theta(n)$, and hence 
the claim holds for said large values of $\widehat{r}$.

Assume henceforth that $\widehat{r} \leq R-\Theta(1)$ and define $A:=\dt(\kappa)\setminus (B_Q(R)\cup B_O(\widehat{r}))$.
Clearly, 
\begin{equation}\label{eqn:angular-muSum}
\mu(\dt(\kappa)) = \mu(B_Q(R)\cup B_O(\widehat{r}))+\mu(A).
\end{equation}
Now, observe that $B_Q(R)$ is completely contained in half of the disk $B_O(R)$ (see Figure~\ref{fig:angular}), so
$\mu(B_Q(R)\cup B_O(\widehat{r}))=\mu(B_Q(R))+\Theta(\mu(B_O(\widehat{r})))$, and thus by Lemma~\ref{lem:muBall} and Fact~\ref{fct:angular-aproxhat}, for~$\widehat{r}>1$, 
\begin{equation}\label{eqn:angular-muNotA}
\mu(B_Q(R)\cup B_O(\widehat{r})))=\Theta\Big(n\Big(e^{-\frac{R}{2}}+\Big(\frac{\kappa\sqrt{\ss}}{e^{\beta R}}\Big)^{\frac{\alpha}{\beta}}\Big)\Big)
=\Theta\Big(\mkor{\nu}{1}+n\Big(\frac{\kappa\sqrt{\ss}}{e^{\beta R}}\Big)^{\frac{\alpha}{\beta}}\Big).
%\qquad\text{if $\widehat{r}>1$.}
\end{equation}
Moreover, the identity also holds when $\widehat{r}\leq 1$, since $\mu(B_Q(R))=\Theta(\mkor{\nu}{1})$ and $\mu(B_O(\widehat{r}))=O(ne^{-\alpha R})=o(1)$ (by Lemma~\ref{lem:muBall}, definition of $R$ and the fact that $\alpha>\frac12$), and $n(\kappa\sqrt{\ss}/e^{\beta R})^{\frac{\alpha}{\beta}}
=O(ne^{-\alpha R})=o(1)$ (by Fact~\ref{fct:angular-aproxhat}, definition of $R$ and since $\alpha>\frac12)$.


On the other hand, by Fact~\ref{fct:angular-inclus} and our choice of $\dt(\kappa)$, we get 
%\begin{equation}\label{eqn:angular-muA}
%\mu(A)= \Theta(ne^{-\alpha R})\int_{\widehat{r}}^R %\frac{\kappa\sqrt{\ss}}{\ell_{r_0}}\sinh(\alpha r_0)dr_0
%  = \Theta(ne^{-\alpha R}\kappa\sqrt{\ss})\int_{\widehat{r}}^R\frac{\sinh(\alpha r_0)}{\sinh(\beta r_0)}dr_0.
%\end{equation}
\begin{equation}\label{eqn:angular-muA}
\mu(A)= \Theta(ne^{-\alpha R}\kappa\sqrt{\ss})\int_{\widehat{r}}^R e^{-\beta r_0}\sinh(\alpha r_0)dr_0
%  = \Theta(ne^{-\alpha R}\kappa\sqrt{\ss})\int_{\widehat{r}}^R(e^{(\alpha-\beta)r_0}{-}e^{-(\alpha+\beta)r_0})dr_0.
\end{equation}
The next claim together with~\eqref{eqn:angular-muSum},  \eqref{eqn:angular-muNotA} and~\eqref{eqn:angular-muA} yield the lemma:
\[
ne^{-\alpha R}\kappa\sqrt{\ss}\int_{\widehat{r}}^R e^{-\beta r_0}\sinh(\alpha r_0)dr_0 =
\begin{cases}
O\Big(n\Big(\frac{\kappa\sqrt{\ss}}{e^{\beta R}}\Big)^{1\wedge \frac{\alpha}{\beta}}\Big),
& \text{if $\alpha\neq\beta$,} \\[8pt]
\Theta\Big(n\frac{\kappa\sqrt{\ss}}{e^{\beta R}}\log\big(\frac{e^{\beta R}}{\kappa\sqrt{\ss}}\big)\Big),
& \text{if $\alpha=\beta$.}
\end{cases}
\]
To prove the claim, not that when $\alpha=\beta$, the last integral equals $\Theta(R-\widehat{r})$.
If $\widehat{r}\leq 1$, by Fact~\ref{fct:angular-aproxhat} we have that $\kappa\sqrt{\ss}=O(1)$,
  so by definition of $R$ we get that $R-\widehat{r}=\Theta(R)=\Theta(\log(e^{\beta R}/(\kappa\sqrt{\ss})))$.
If on the other hand $\widehat{r}>1$, again by Fact~\ref{fct:angular-aproxhat}, we have that $\widehat{r}=\frac{1}{\beta}\log(\kappa\sqrt{\ss})+\Theta(1)$,
  so analogously to the previous calculations we obtain
  $R-\widehat{r}=\Theta(\log(e^{\beta R}/(\kappa\sqrt{\ss})))$.
Plugging back  establishes the claim when $\alpha=\beta$.  

For $\alpha > \beta$, since $\widehat{r}\leq R-\Theta(1)$, the claim follows 
because the integral therein
equals $\Theta(e^{(\alpha-\beta)R})$.
%Plugging this back into~\eqref{eqn:angular-muA} we see that $\mu(A)$ is of the right order and thus the claimed lower bound also holds when $\alpha>\beta$.
%Because of~\eqref{eqn:angular-muSum}, to obtain a matching upper bound for~$\mu(\dt(\kappa))$ it suffices to appropriately bound from above the right-hand side of~\eqref{eqn:angular-muNotA}. To do so, just note that since $\kappa\sqrt{\ss}\leq\frac{\pi}{2}e^{\beta R}$ and $\alpha>\beta$, we have that $(\kappa\sqrt{\ss}e^{-\beta R})^{\frac{\alpha}{\beta}} =O(\kappa\sqrt{\ss}e^{-\beta R})$.

Finally, when $\alpha<\beta$, the integral in the claim is
  $\Theta(e^{-(\beta-\alpha)\widehat{r}})$.
If $\widehat{r}>1$,  by Fact~\ref{fct:angular-aproxhat},
$e^{-(\beta-\alpha)\widehat{r}}
= \Theta((\kappa\sqrt{\ss})^{-(1-\frac{\alpha}{\beta})})$.
%Plugging this back into~\eqref{eqn:angular-muA} we obtain the desired upper bound on $\mu(A)$.
If $\widehat{r}\leq 1$, then $e^{-(\beta-\alpha)\widehat{r}}=O(1)$ and, again by Fact~\ref{fct:angular-aproxhat}, also $\kappa\sqrt{\ss}=O(1)$.
So, no matter the value of $\widehat{r}$ 
the claim holds when $\alpha<\beta$ which concludes the proof of the claim for all cases.
%Assume first $\alpha=\beta$. In this case the last integral equals $R-\widehat{r}$.
%If $\widehat{r}\leq 1$, by Fact~\ref{fct:angular-aproxhat} we have that $\kappa\sqrt{\ss}=O(1)$,
%  so by definition of $R$ we get that $R-\widehat{r}=\Theta(\log(e^{\beta R}/(\kappa\sqrt{\ss})))$.
%If on the other hand $\widehat{r}>1$, again by Fact~\ref{fct:angular-aproxhat}, we have that $\widehat{r}=\frac{1}{\beta}\log(\kappa\sqrt{\ss})+\Theta(1)$,  so analogously to the previous calculations we obtain   $R-\widehat{r}=\Theta(\log(e^{\beta R}/(\kappa\sqrt{\ss})))$.
%Plugging this back into~\eqref{eqn:angular-muA}, and then using~\eqref{eqn:angular-muSum} and~\eqref{eqn:angular-muNotA} 
%we obtain the claimed value for $\mu(\dt(\kappa))$ when $\alpha=\beta$.  
%
%Assume now $\alpha > \beta$. Then, the last integral in~\eqref{eqn:angular-muA} equals
%\[
%\int_{1\vee\widehat{r}}^1\frac{\sinh(\alpha r_0)}{\sinh(\beta r_0)}dr_0+\Theta(1)\int_{1\vee\widehat{r}}^{R}e^{(\alpha-\beta)r_0}dr_0 
%= \Theta(e^{(\alpha-\beta)R})
%\]
%because the first summand is either $0$ or $O(1)$ and where for the last equality we have used that $\widehat{r}\leq R-\Theta(1)$.
%Plugging this back into~\eqref{eqn:angular-muA} we see that $\mu(A)$ is of the right order and thus the claimed lower bound also holds when $\alpha>\beta$.
%Because of~\eqref{eqn:angular-muSum}, to obtain a matching upper bound for $\mu(\dt(\kappa))$  it suffices to appropriately bound from above the right-hand side of~\eqref{eqn:angular-muNotA}. 
%To do so, just note that since $\kappa\sqrt{\ss}\leq\frac{\pi}{2}e^{\beta R}$ and $\alpha>\beta$, we have that
%$(\kappa\sqrt{\ss}e^{-\beta R})^{\frac{\alpha}{\beta}}=O(\kappa\sqrt{\ss}e^{-\beta R})$.
%
%Finally, for the case where $\alpha<\beta$, by~\eqref{eqn:angular-muNotA} we see that $\mu(B_Q(R)\cup B_{O}(\widehat{r}))$ is of the right order and thus the claimed lower bound holds in this case as well. 
%Because of~\eqref{eqn:angular-muSum}, to obtain a matching upper bound for $\mu(\dt(\kappa))$, we need to establish the corresponding upper bound for $\mu(A)$.
%To do so, let us first consider the case where $\widehat{r}>1$. 
%Then, by Fact~\ref{fct:angular-aproxhat}, the last integral in~\eqref{eqn:angular-muA} now equals
%\[
%\int_{1\vee\widehat{r}}^1\frac{\sinh(\alpha r_0)}{\sinh(\beta r_0)}dr_0
%+ 
%\Theta(1)\int_{\widehat{r}}^Re^{-(\beta-\alpha)r_0}dr_0
%= \Theta(e^{-(\beta-\alpha)\widehat{r}})
%= \Theta((\kappa\sqrt{\ss})^{-(1-\frac{\alpha}{\beta})}).
%\]
%Plugging this back into~\eqref{eqn:angular-muA} we obtain the desired upper bound on $\mu(A)$.
%Consider next the case where $\widehat{r}\leq 1$. 
%Then, $\kappa\sqrt{\ss}=O(1)$ (by Fact~\ref{fct:angular-aproxhat}).
%Moreover, since $\alpha<\beta$, the last integral in~\eqref{eqn:angular-muA} is $O(1)$, so we get that $\mu(A)=O(ne^{-\alpha R})=o(1)$ (by definition of $R$ and given that~$\alpha>\frac12$).
%This completes the analysis of the case $\alpha<\beta$ and the proof for all cases.
\end{proof}

\medskip
We now have all the required ingredients to prove this section's main result.

\begin{proof}{(of Theorem~\ref{thm:mainangular})}
Throughout the ensuing discussion we let $\dt:=\dt(\kappa)$.
We begin by showing the uniform upper and lower bounds on $\P_{x_0}(T_{det}\leq \ss)$. To do so observe first that if $x_0\in B_Q(R)\subseteq \dt$, clearly $\P_{x_0} (T_{det} \le \ss)=1$ for any $\ss\ge0$ and hence the uniform lower bound follows directly for said $x_0$. Assume henceforth that $x_0\not\in B_Q(R)$ and observe that since there is only angular movement, a particle initially located at $(r_0,\theta_0)$ detects $Q$ if and only if it reaches $(r_0,\phi(r_0))$ or $(r_0,-\phi(r_0))$. Now, recall that the angular movement's
law is that of a variance $1$ Brownian motion $B_{\II{\ss}}$ with $\II{\ss}:=\int_{0}^{\ss} \cosech^2(\beta r_\ss)d\ss = (\phi^{(\ss)})^{2}$, so 
\begin{equation}\label{eqn:angular-exitt0}
\P_{x_0}(T_{det}\leq \ss)=\P(H_{[-a,b]}\leq \II{\ss})
\end{equation}
where we have used (with a slight abuse of notation) $\P$ for the law of a standard Brownian motion, and where $H_{[-a,b]}$ is its exit time from the interval $[-a,b]$ where in this case $a:=\phi(r_0)-|\theta_0|$ and $b:=2\pi-\phi(r_0)-|\theta_0|$. This last probability is a well studied function of~$a,b$ and $\II{\ss}$ (see~\cite{Borodin2002}, formula 3.0.2),
%This was written in more detail before, you left it out on purpose?}
which can be bounded using the reflection principle and the fact that~$a\leq b$, giving 
\begin{equation}\label{eqn:angular-exitt}
\P(H_{[-a,b]}\leq \II{\ss})=\Theta\big(\P(B_{\II{\ss}}\leq -a)\big)=\Theta\big(\Phi\big((\phi(r_0)-|\theta_0|)/\phi^{(\ss)}\big)\big).
\end{equation}
From our assumption $x_0\not\in B_Q(R)$ we deduce that $|\theta_0|>\phi(r_0)$, and hence the argument within~$\Phi$ above is always negative. It follows that for $\theta_0>0$ the mapping $\theta_0\mapsto\Phi\big((\phi(r_0)-\theta_0)/\phi^{(\ss)}\big)$ is decreasing, and so both uniform bounds on $\P_{x_0}(T_{det}\leq\ss)$ follow from the definition of $\dt(\kappa)$.

%To obtain the bound for $\P(T_{det}\geq \ss)$ recall that by~\eqref{eqn:main}
%\[\P(T_{det}\geq \ss)=\exp\Big(-\int_{\dt}\P_{x_0}(T_{det}\leq\ss)d\mu(x_0)-\int_{\ndt}\P_{x_0}(T_{det}\leq\ss)d\mu(x_0)\Big)\]
%for any $\kappa > 0$. Since for any $x_0\in\dt$ we have $\P_{x_0}(T_{det}\leq \ss)=\Omega(\Phi(-\kappa))$ it will be enough to take $\kappa=1$ to obtain that $\P_{x_0}(T_{det}\leq \ss)=\Theta(1)$, which gives $\int_{\dt}\P_{x_0}(T_{det}\leq\ss)d\mu(x_0)=\Theta(\mu(\dt))$, and hence it follows from Lemma~\ref{lem:angular-muDt} that this first term is already of the same order as the one in the theorem. Thus,  in order to conclude the theorem it only remains to prove that $\int_{\ndt}\P_{x_0}(T_{det}\leq\ss)d\mu(x_0)=O(\mu(\dt))$: 
We next establish the integral bound. Let $\dt:=\dt(\kappa)$.
%As observed in the discussion of our general strategy of proof (Section~\ref{sec:strategy}), given the uniform bounds just shown, to conclude it suffices to fix~$\kappa$, say $\kappa=1$, show that $\int_{\ndt}\P_{x_0}(T_{det}\leq\ss)d\mu(x_0)=O(\mu(\dt))$ and use Lemma~\ref{lem:angular-muDt}.
From~\eqref{eqn:angular-exitt0} and~\eqref{eqn:angular-exitt} we observe that
\begin{align*}
    \int_{\ndt}\P_{x_0}(T_{det}\leq\ss)d\mu(x_0)
    = \Theta(ne^{-\alpha R})\int_{\widehat{r}}^R\int_{\phi(r_0)+ \kappa\phi^{(\ss)}}^{\pi}\Phi\big((\phi(r_0)-\theta_0)/\phi^{(\ss)}\big)\sinh(\alpha r_0)d\theta_0dr_0.
\end{align*}
Applying the change of variables $y_0:=(\theta_0-\phi(r_0))/\phi^{(\ss)}$ and bounding $\pi$ by $\infty$ in the upper limit of the integral we obtain 
\begin{align*}
    \int_{\ndt}\P_{x_0}(T_{det}\leq\ss)d\mu(x_0)
    & = O(ne^{-\alpha R})\int_{\widehat{r}}^R\int_{\kappa}^{\infty}\Phi({-}y_0)\sinh(\alpha r_0)\phi^{(\ss)} dy_0dr_0 \\
    & =  O(ne^{-\alpha R}\sqrt{\ss})\int_{\widehat{r}}^R e^{-\beta r_0}\sinh(\alpha r_0)dr_0.
\end{align*}
The last expression is the same as one encountered in the proof of 
Lemma~\ref{lem:angular-muDt}. Substituting by the values obtained therein one gets a term which, by Lemma~\ref{lem:angular-muDt}, is $O(\mu(\dt(\kappa))$, thus completing the proof of the claimed integral upper bound.
\end{proof}